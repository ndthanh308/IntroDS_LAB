\documentclass[letterpaper, 10pt, conference]{ieeeconf}      % Use this line for a4 paper

\IEEEoverridecommandlockouts                              % This command is only needed if 
% you want to use the \thanks command

\overrideIEEEmargins                                      % Needed to meet printer requirements.

\pdfminorversion=4  
\usepackage{graphicx}
%\usepackage{epsfig}
\usepackage{algorithm}
\usepackage[noend]{algpseudocode}
\usepackage{amsmath}
\usepackage{nicematrix} % vertical text
%\usepackage{amssymb}
%\usepackage{lipsum}
%\usepackage{subfigure}
%\usepackage{booktabs}
%multi-row
\usepackage{multirow}  
\usepackage{comment}
\usepackage{float} % to put figure H



% for tikz stuff
\usepackage{tikz, pgfplots, fix-cm}
\usetikzlibrary{plotmarks}
\usepackage{graphicx,booktabs,array}



% markers
\usepackage{amssymb}% http://ctan.org/pkg/amssymb  
\usepackage{pifont}% http://ctan.org/pkg/pifont
\newcommand{\cmark}{\ding{51}}%
\newcommand{\xmark}{\ding{55}}%+


%% Key definitions for text elements. USE THEM
\def\secref#1{Sec.~\ref{#1}}
\def\figref#1{Fig.~\ref{#1}}
\def\tabref#1{Tab.~\ref{#1}}
\def\eqref#1{Eq.~(\ref{#1})}
\def\algref#1{Alg.~\ref{#1}}

\newcommand{\photo}[1]{% Figure removed}
\newcommand{\RNum}[1]{(\uppercase\expandafter{\romannumeral #1\relax})}

\title{\LARGE \bf BonnBot-I: A Precise Weed Management and Crop Monitoring Platform}
%\title{\LARGE \bf BonnBot-I: An Autonomous Weed Management Platform with Multiple Tools??}

\author{Alireza Ahmadi, Michael Halstead, and Chris McCool% <-this % stops a space
	\thanks{All authors are with the University of Bonn, Bonn 53115 Germany. 
			{\tt\small \{alireza.ahmadi, michael.halstead, cmccool\}@uni-bonn.de}}%
%	\thanks{This work has partly been supported by ...
%		%the EC under the grant number H2020-ICT-644227-Flourish. 
%		%the EC under the grant number H2020-ICT-645403-RobDREAM.
%		%the DFG under the grant number FOR~1505: Mapping on Demand.
%	}%
%	%\thanks{*This work was not supported by any organization}% <-this % stops a space
%	\thanks{$^{1}$Michael Halstead is with the Faculty of Electrical Engineering, Mathematics and Computer Science,
%		Bonn University, Bonn 53115, Germany
%		{\tt\small michael.halstead@uni-bonn.de}}%
%	\thanks{$^{2}$Chris McCool is with the Department of Electrical Engineering, Bonn University, Bonn 53115, Germany
%		{\tt\small cmccool@uni-bonn.de}} %
}

         
%% Other useful macros
\newcommand\todo[1]{\textbf{[TODO: #1}]}
\newcommand\etal{\emph{et al.}}
\newcommand\bbot{BonnBot-I}


%% Some math definition
\def\argmax{\mathop{\rm argmax}}
\def\argmin{\mathop{\rm argmin}}
\newcommand{\bigO}[1]{$\mathcal{O}(#1)$}

\newcommand\TODO[1]{\textbf{\textcolor{red}{#1}}}
\newcommand\UPDATED[1]{\textbf{\textcolor{blue}{#1}}}

\begin{document}

\maketitle
\thispagestyle{empty}
\pagestyle{empty}

%%%%%%%%%%%%%%%%%%%%%%%%%%%%%%%%%%%%%%%%%%%%%%%%%%%%%%%%%%%%%%%%%%%%%%%%%%%%%%%%
\begin{abstract}

Cultivation and weeding are two of the primary tasks performed by farmers today.
A recent challenge for weeding is the desire to reduce herbicide and pesticide treatments while maintaining crop quality and quantity.
In this paper we introduce \bbot\ a precise weed management platform which can also performs field monitoring.
Driven by crop monitoring approaches which can accurately locate and classify plants (weed and crop) we further improve their performance by fusing the platform available GNSS and wheel odometry.
This improves tracking accuracy of our crop monitoring approach from a normalized average error of $8.3\%$ to $3.5\%$, evaluated on a new publicly available corn dataset.
We also present a novel arrangement of weeding tools mounted on linear actuators evaluated in simulated environments.
We replicate weed distributions from a real field, using the results from our monitoring approach, and show the validity of our work-space division techniques which require significantly less movement (a $50\%$ reduction) to achieve similar results.
Overall, \bbot\ is a significant step forward in precise weed management with a novel method of selectively spraying and controlling weeds in an arable field.

\begin{comment}
In this paper we introduce \bbot\ a precise weed management platform that can locate, classify, and selectively execute plant-level interventions in arable fields.
%In this paper we introduce \bbot\ a precise weed management platform that can locate, classify, and selectively execute plant-level interventions in arable fields.
%The four linear axes are installed on the rear of robot and can perform precise interventions using less agro-chemicals.
We show that by fusing the platform available GNSS and wheel odometry our crop monitoring approach is able to improve its tracking accuracy from a normalized average error of $8.3\%$ to $3.5\%$, evaluated on a new publicly available corn dataset.
We present a novel arrangement of weeding tools mounted on linear actuators.
To accurately use this information for precise weed management we outline three planning systems for the linear axes and evaluate them in two simulated environments.
In these environments we are able to replicate a real field, using the results from our monitoring approach, and show the validity of our work-space division techniques which require significantly less movement ($10m$ compared to $5m$) to achieve similar results.
Overall, \bbot\ is a significant step forward in precise weed management with a novel method of selectively spraying and controlling weeds in an arable field.
\end{comment}


% \TODO{
% Cultivation and protecting plants can be named as the two most important tasks for the farmers. 
% One of the challenges in recent years is to protect environment by controlling and reducing the amount of herbicide and pesticide used for weed control.
% Smart farming, is an enabling technology for precision farming.
% In this paper we introduce \bbot\ a precision weed management platform that can detect, categorize and execute proper plant-level interventions in arable fields.
% Using a novel weeding tool design \bbot\ is able to conduct selective weed treatment encouraging less agro-chemicals to be used for crop protection in fields.
% by incorporating sensor fusion we could improve crop monitoring in a order of magnitude.
% We demonstrate that using our intervention model, with only four independent intervention head can manage $?\%$ percent of weeds in real crop row fields.
% Furthermore ,we introduce our newly papered Corn dataset, representing new challenges of crop type which has not been investigated before for weeding applications.
% We have used our novel weeding simulation framework specifically developed for developing and evaluating different weeding strategies and approaches.}

\textit{Keywords} — Robotics and Automation in Agriculture and Forestry; Agricultural Automation; Field Robotics.

\end{abstract}
%%%%%%%%%%%%%%%%%%%%%%%%%%%%%%%%%%%%%%%%%%%%%%%%%%%%%%%%%%%%%%%%%%%%%%%%%%%%%%%%

\section{Introduction}
\label{sec:indroduction}
\section{Introduction}
Current quantum hardware is unable to carry out universal quantum computations due to the buildup of errors that occur during the computation. 
The magnitude of the individual error is currently above the value that the Threshold Theorem requires in order to kick-start quantum error correction and fault-tolerant quantum computation~\cite[Section 10.6]{nielsen_chuang_2010}. 
Although the experimentally achieved fidelity rates are promising and the error bounds are inching closer to the required threshold, we will have to work for the foreseeable future with quantum hardware with errors that build-up during the computation.  This implies that we can only do a limited number of steps before the output of the computation has become completely uncorrelated with the intended one.

For fault-tolerant quantum computing, we repeat four steps: 
1) We apply a number of single and two-qubit quantum gates, in parallel whenever possible; 
2) We perform a syndrome measurement on a subset of the qubits; 
3) We perform fast classical computations to determine which errors have occurred and how to correct them; 
and, 4) We apply correction terms based on the classical computations.
We then repeat these four steps with a next sequence of gates. 
These four steps are essential to fault-tolerant quantum computing. 


The starting point of this work is to use the four steps outlined above, not to carry out error correction and fault-tolerant computation, but to enhance short, constant-depth, {\em uncorrected} quantum circuits that perform single qubit gates and {\em nearest-neighbor} two qubit gates. 
Since in the long run we will have to implement error-correction and fault-tolerant computation anyhow, and this is done by such a four-step process, why not make other use of this architecture? Moreover, on some of the quantum hardware platforms, these operations are already in place.
Embracing this idea we naturally arrive at the question: what is the computational power of \textit{low-depth} quantum-classical circuits organized as in the four steps outlined above? 
We thus investigate circuits that execute a small, ideally constant, number of stages, where at each stage we may apply, in parallel, single qubit gates and {\em nearest-neighbor} two qubit gates, followed by measurements, followed by low-depth classical computations of which the outcome can control quantum gates in later stages. 
It is not clear, at first, whether such circuits, especially with constant depth, can do anything remotely useful. 
But we will see that this is indeed the case: many quantum computations can be done by such circuits in constant depth. 
By parallelizing quantum computations in this way, we improve the overall computational capabilities of these circuits, as we do not incur errors on qubits that are idle, simply because qubits are not idle for a very long time. 
Furthermore, reducing the depth of quantum circuits, at the cost of increasing width, allows the circuit to be run faster even if errors occur.

The first usage of such a four-step layout, not to do error correction, but to perform computations, can be found in the paradigm of measurement-based quantum computing~\cite{gottesman1999demonstrating,raussendorf2001one,jozsa2006introduction,clark2007generalised}: 
A universal form of quantum computing where a quantum state is prepared and operations are performed by measuring qubits in different bases, depending on previous measurements and intermediate measurements.

\citeauthor{PhamSvore2013} were the first to formalize the four-step protocol for performing computations~\cite{PhamSvore2013}. They included specific hardware topologies by considering two-dimensional graphs for imposing constraints on qubit interactions. In their model, they develop circuits for particularly useful multi-qubit gates, including specifying costs in the width, number of qubits, depth, number of concurrent time steps, size, and total number of non-Identity operations.
As a result, they find an algorithm that factors integers in polylogarithmic depth.
\citeauthor{Browne:2011} showed that the main tool in the work by \citeauthor{PhamSvore2013}, the fan-out gate, can also be replaced by additional log-depth classical computations in the measurement-based quantum computing setting~\cite{Browne:2011}.

More recently, \citeauthor{Cirac:2021} introduced a scheme to implement unitary operations involving quantum circuits combined with Local Operations and Classical Communication ($\mathsf{LOCC}$) channels: $\mathsf{LOCC}$-assisted quantum circuits~\cite{Cirac:2021}. Similarly to the four-step scheme we just described, they allow for a short depth circuit to be run on the qubits, followed by one round of $\mathsf{LOCC}$, in which ancilla qubits are measured and local unitaries are applied based on the measurement outcomes. They show that in this model any 1D transitionally invariant matrix-product state (MPS) with fixed bond dimension is in the same phase of matter as the trivial state. Similar ideas can be found in~\cite{TVV_NonAbelianTopologicalOrder_2022, tantivasadakarn2021long}.

In this work, we introduce a new model, called \textit{Local Alternating Quantum-Classical Computations} ($\LAQCC$). In this model we alternate between running quantum circuits (constrained by locality), ending in the measurement of a subset of qubits, and fast classical computations based on the measurement results. The outcome of the classical computations are then used to control future quantum circuits. We allow for flexibility in this model, by giving different constraints to the power of both the quantum circuits and the classical circuits as well as the number of alternations between them. 
Most attention will be given to $\LAQCC$ containing quantum circuits of constant depth, classical circuits of logarithmic depth and at most a constant number of alternations between them. 
Any circuit constructed in this model is considered to be of constant depth. 
We restrict ourselves to logarithmic depth classical computations, as this is the first natural and non-trivial extension beyond constant-depth classical computations. 
Constant-depth classical computations do however also have an equivalent constant-depth quantum implementation.

The definition of $\LAQCC$ sharpens the original definition of \citeauthor{PhamSvore2013} by adding constraints to the intermediate classical computations. This allows us to bound the power of $\LAQCC$ from above. 

The main result of \citeauthor{Cirac:2021}, that 1D translational invariant MPS with fixed bond dimension can be prepared by $\mathsf{LOCC}$-assisted circuits, relies on local symmetries of the MPS. These symmetries allow them to prepare local states (on a constant number of qubits) and glue them together by doing one round of the appropriate entangling measurement and corrections, after which they run a round of local unitaries to get the desired result. This general scheme for preparing states that exhibit an MPS description with the appropriate local symmetries requires only geometrically local unitaries and one round of measurement and corrections an therefore is accessible in $\LAQCC$. Studying different local symmetries, known as Symmetry Protected Topological (SPT) phases of matter, to find measurement-based constant depth circuits for states is a broad ongoing field of research~\cite{TVV_NonAbelianTopologicalOrder_2022, tantivasadakarn2021long, smith2023deterministic}. 
All these schemes have a $\LAQCC$ implementation.

%$\LAQCC$-circuits also exist for general schemes of preparing local states, based on the local tensors, and gluing them together using one round of entangled measurement and corrections, based on the local symmetry. 
%The main result of \citeauthor{Cirac:2021}, that 1D translational invariant MPS with fixed bond dimension can be prepared by $\mathsf{LOCC}$-assisted circuits, relies heavily on local symmetries of the MPS and as a result also has an equivalent $\LAQCC$ implementation. 
%The corrections applied after the measurement round are local unitaries depending on the local symmetries of the MPS. 

 

%This general scheme of preparing local states, based on the local tensors, and gluing it together by doing one round of entangled measurement and corrections, based on the local symmetry, is accessible in $\LAQCC$.
Note however that \citeauthor{Cirac:2021} also suggest a circuit for the $W$-state.
This circuit uses sequentially and dependent measurement-based corrections of the ancilla qubits. 
These dependent measurements translate to sequential alternations between the quantum and classical circuits and therefore increase the total depth to linear depth, exceeding the constant-depth constraints imposed by $\LAQCC$-circuits. 

We study the power of the $\LAQCC$ model with respect to state preparation, showing that even with only constant quantum-depth and logarithmic classical depth it remains possible to prepare states with long-range entanglement.
Another surprising result is that it is unlikely that $\LAQCC$ circuits are classically simulatable. We show that any instantaneous quantum polynomial-time (IQP) circuit~\cite{Bremner2010,Shepherd2009} has an $\LAQCC$ implementation.
Classical simulation of IQP circuits implies the collapse of the polynomial hierarchy to the third level, which is not believed to be true~\cite{Bremner2017}. Therefore, we expect that $\LAQCC$ circuits are unlikely to be classically simulatable. We bound the power of $\LAQCC$ by showing that it is contained in $\QNC^1$, the class of polynomial-size, log-depth circuits.

Next, we also study the power that intermediate classical calculations can add to quantum computations, by considering a new model that alternates between polynomially many polynomial-depth quantum circuits and unbounded classical computations
We study this model by doing a complexity theoretical analysis, where we draw inspiration from the notions of complexity given by \citeauthor{RosenthalYuen:2022}, \citeauthor{MetgerYuen:2023}, and \citeauthor{Aaronson:2004}.
All three complexity notions are based on the notion of state preparation, instead of more traditional definition of complexity such as the decidability of a computational problem. 
The first two consider classes based on sequences of quantum states preparable by a polynomial-sized quantum circuit, where the circuits are uniformly generated by a computational class, for instance, the class $\mathsf{PSPACE}$, which results in the complexity class $\mathsf{StatePSPACE}$~\cite{RosenthalYuen:2022,MetgerYuen:2023}.
The third notion considers a relative complexity, where the complexity is measured between two given states, and is measured by the number of gates, from a given gate-set, required to transform one state in another state~\cite{Aaronson:2004}. 
For our definition of state preparation complexity, we drop the uniformity constraint from~\cite{RosenthalYuen:2022,MetgerYuen:2023} and define a class as $\mathsf{StateX}$, which refers to states preparable by circuits of type $\mathsf{X}$. 
As an example, if $\mathsf{X} = \QNC^0$, this results in the class $\mathsf{StateQNC^0}$, which is the set of states preparable from the $\ket{0}^n$ state by poly-size constant-depth circuits. 
This notion is similar to the relative complexity from~\cite{Aaronson:2004}, where one state is the  $\ket{0}^n$ state and instead of counting the number of gates we consider the set of states preparable by a fixed number of gates. Using this notion of complexity we show that any state preparable by an $\LAQCC^*$ circuit is also preparable by a $\mathsf{PostQPoly}$ circuit, the class of circuits of polynomial depth with an additional post-selection gate. 

All Clifford circuits have a constant-depth $\LAQCC$ implementation, implying that any stabilizer state can be implemented by a constant-depth $\LAQCC$ circuit, see Section~\ref{sec:clifford_circuits} for a proof of this statement. 
Efficient circuits for stabilizer states have been known already through measurement-based quantum computing. Therefore this paper focuses on the preparation of non-stabilizer states, and as a surprising result we find novel constant-depth protocols for four very natural classes of non-stabilizer states.
Despite the extensive research into these four classes of non-stabilizer states and the many applications of them, no efficient constant- or low-depth state preparation protocols are known yet. We specifically consider these four classes as they are all often used as initial states in other algorithms.

The first state is a uniform superposition over an arbitrary number of states. 
This state finds applications in many quantum algorithms, as they often start with a uniform superposition over multiple states. 
This superposition is often achieved by applying Hadamard gates to every qubit due to its simplicity to prepare. 
Yet, the analysis of many algorithms, such as Shor's algorithm~\cite{Shor:1997}, would benefit from a different initial superposition. 
The circuit to prepare the uniform superposition over an arbitrary number of states uses an exact version of Grover search as a subroutine, that turns a probabilistic circuit, with a known constant probability of success, into a deterministic circuit. 
We use the circuit for preparing a uniform superposition over an arbitrary number of states as a subroutine in the next two quantum state preparation protocols. 

The second state is the $W$-state, the uniform superposition over all computational basis states of Hamming-weight~$1$, a natural long-ranged entangled state that displays a fundamentally nonequivalent type of entanglement from the Greenberger–Horne–Zeilinger state~\cite{WState:2000}, for which $\LAQCC$-type constant-depth circuits were previously known~\cite{PhamSvore2013, Cirac:2021}. 
The $W$-state is often used as benchmark for new quantum hardware~\cite{Haffner2005,Neeley2010,GarciaPerez:2021}. 
A novel way to prepare the $W$-state therefore gives a new way to benchmark different quantum devices with each other. 
A circuit for preparing the $W$-state was given in~\cite{Cirac:2021}, but this implementation requires sequentially alternating measurements followed by local unitaries, which in the $\LAQCC$ model is not considered to be of constant depth. 
We improve this protocol by giving an $\LAQCC$ implementation of the $W$-state, based on a compress-uncompress method that links the one-hot and binary encoding of integers.

The third state considered is the Dicke state, a generalization of the $W$-state, a superposition over all computational basis states with Hamming-weight $k$~\cite{Dicke:1954}. 
Dicke states have relevance in various practical settings.
For instance, for quantum game theory~\cite{zdemir2007}, quantum storage~\cite{Bacon_Compress:2006,Plesch:2010}, quantum error correction~\cite{ouyang2014permutation}, quantum metrology~\cite{toth2012multipartite}, and quantum networking~\cite{prevedel2009experimental}. 
Dicke states have been used as a starting state for variational optimization algorithms, most notably Quantum Alternating Operator Ansatz (QAOA)~\cite{Hadfield2019}, to find solutions to problems such as Maximum k-vertex Cover~\cite{Brandhofer2022,cook2020quantum}.
The ground states of physical Hamiltonians describing one-dimensional chains tend to show a resemblance to Dicke states such as states resulting from the Bethe ansatz, making them an ideal starting state when investigating the ground state behavior of these Hamiltonians~\cite{TDL_BetheAnsatzDerivation:2010,B_ExcitedStateQuantumPhaseTransitions:2013,DickeTransitions:2021}. 
For instance, the algorithm by \citeauthor{van2021preparing}, who give an algorithm to prepare the Bethe ansatz eigenstates of the spin-1/2 XXZ spin chain, starts by first preparing a Dicke state~\cite{van2021preparing}. 
A Dicke-state preparation protocol based on the compress-uncompress methodology used in the $W$-state furthermore finds applications in entanglement distillation, where the entanglement of a large state is concentrated on only a few qubits. 
Efficient deterministic circuits for preparing Dicke states have been proposed by \citeauthor{bartschi2019deterministic}~\cite{bartschi2019deterministic, bartschi2022deterministic_short_depth}. 
They provide a quantum circuit of depth $\mathO(k \log(\frac{n}{k}))$, allowing arbitrary connectivity, to prepare a Dicke state, which they conjecture to be optimal when $k$ is constant. 
In this work, we provide a constant-depth $\LAQCC$ circuit below their conjectured bound already for constant $k$. 
However, this does not directly disprove their conjecture, as we allow for intermediate measurements and classical computations. 
More significantly, we even construct constant-depth $\LAQCC$ circuits for $k = \mathO(\sqrt{n})$ greatly improving their bound.
This construction extends the compress-uncompress method for the $W$-state combined with additional subroutines. 

We continue with a log-depth state preparation protocol for the Dicke-state for arbitrary $k$. 
This protocol implements an efficient transformation between the factoradic number representation and the combinatorial number representation of a positive integer. 
The combinatorial number representation relates directly to the Dicke state. 
The provided efficient transformation between number representation systems might be of independent interest. 

We conclude by modifying our protocol for preparing a Dicke-state to a protocol that prepares quantum many-body scar states in constant-depth. 
These states have low entanglement and longer coherence times than states with similar energy density.
These characteristics make many-body scar states interesting to analyze and relevant within physics.
Many-body scar states appear for instance in the AKLT model~\cite{AKLT:1987,MRBAR:2018,MRB:2018} and different spin models~\cite{SI:2019,MOBFR:2020}.
Known methods for preparing these states have polynomial-depth~\cite{Gustafson:2023}, whereas our circuit has constant depth. 

% We conclude by studying the power that intermediate classical calculations can add to quantum computations. 
% In this study, we define a new model that relaxes constant-depth quantum circuits to polynomial depth quantum circuits, log-depth classical calculations to unbounded classical computations and a constant number of alternations to a polynomial number of alternations. 
% We call this model $\LAQCC^*$. 
% We study this model by doing a complexity theoretical analysis, where we draw inspiration from the notions of complexity given by \citeauthor{RosenthalYuen:2022}, \citeauthor{MetgerYuen:2023}, and \citeauthor{Aaronson:2004}.
% All three complexity notions are based on the notion of state preparation, instead of more traditional definition of complexity such as the decidability of a computational problem. 
% The first two consider classes based on sequences of quantum states preparable by a polynomial-sized quantum circuit, where the circuits are uniformly generated by a computational class, for instance, the class $\mathsf{PSPACE}$, which results in the complexity class $\mathsf{StatePSPACE}$~\cite{RosenthalYuen:2022,MetgerYuen:2023}.
% The third notion considers a relative complexity, where the complexity is measured between two given states, and is measured by the number of gates, from a given gate-set, required to transform one state in another state~\cite{Aaronson:2004}. 
% For our definition of state preparation complexity, we drop the uniformity constraint from~\cite{RosenthalYuen:2022,MetgerYuen:2023} and define a class as $\mathsf{StateX}$, which refers to states preparable by circuits of type $\mathsf{X}$. 
% As an example, if $\mathsf{X} = \QNC^0$, this results in the class $\mathsf{StateQNC^0}$, which is the set of states preparable from the $\ket{0}^n$ state by poly-size constant-depth circuits. 
% This notion is similar to the relative complexity from~\cite{Aaronson:2004}, where one state is the  $\ket{0}^n$ state and instead of counting the number of gates we consider the set of states preparable by a fixed number of gates. Using this notion of complexity we show that any state preparable by an $\LAQCC^*$ circuit is also preparable by a $\mathsf{PostQPoly}$ circuit, the class of circuits of polynomial depth with an additional post-selection gate. 

\paragraph{Summary of results}
\begin{itemize}
    \item We give a new definition of a computational model that captures the power of the four step process: applying a constant number of layers of one- and two-qubit gates; performing a syndrome measurement; perform a fast classical computation determining corrections; apply corrections. We call this model \emph{Local Alternating Quantum Classical Computations}, or $\LAQCC$ for short. In this model we bound the allowed quantum operations, intermediate classical calculations, and number of rounds separately. In Section~\ref{sec:LAQCC_model} we define this model and give a list of operations based on results from literature contained in this computational model. In some of these operations we explicitly use that we allow for multiple, but at most constant, rounds  of corrections.
    \item  We show show that there exist $\LAQCC$ circuits that can not be weakly simulated in Section~\ref{sec:IQP_in_LAQCC}. We further show that for every $\LAQCC$ circuit there exists a $\QNC^1$ circuit simulating it perfectly, in Section~\ref{sec:LAQCC_in_QNC1}.
    \item We introduce a new type computational complexity for preparing states and show that the extension of $\LAQCC$ where we allow a polynomial number of rounds and unbounded classical computation, is contained in $\mathsf{PostQPoly}$, the class of polynomial circuits with post-selection, in Section~\ref{sec:Complexity results}.
    \item We show a protocol to prepare the uniform superposition state of size $q$ in $\LAQCC$ using $\mathO(\ceil{\log_2(q)}^2)$ qubits in Section~\ref{sec:superposition_modulo_q}. 
    \item We show a protocol to prepare the $W_n$ state in $\LAQCC$ using $\mathO(n\log(n))$ qubits in Section~\ref{sec:W_state_in_LAQCC}.
    \item We show two ways of preparing the Dicke-$(n,k)$ state. The first method is in $\LAQCC$, works up to $k = \mathO(\sqrt{n})$, uses $\mathO(n^2\log(n))$ qubits, and is found in Section~\ref{sec:dicke:small_k}. The second method is in $\LAQCC\text{-}\mathsf{LOG}$ (an extension of $\LAQCC$ allowing for logarithmic number of alterations instead of constant), works for any $k$, uses $\mathO(\text{poly}(n))$ qubits, and is found in Section~\ref{sec:Dicke_in_LAQCC_LOG}. 
    \item We extend on our $\LAQCC$ method of generating Dicke-$(n,k)$ states for $k = \mathO(\sqrt{n})$ and show a protocol to generate many-body scar states for a particular Hamiltonian in $\LAQCC$ (Section~\ref{sec:many_body_scar}). 
\end{itemize}
Summarized in a table, we provide the following state generation protocols:
\begin{table}[htb]
\centering
\begin{tabular}{l|l|l|l}
\textbf{State description} & \textbf{Width} & \textbf{Depth} & \textbf{Implementation}\\
\hline 
Uniform superposition mod $q$: $\frac{1}{\sqrt{q}} \sum_{i = 0}^{q-1}\ket{i}$ & $\mathO(\ceil{\log^2 q})$ & $\mathO(1)$ & Section~\ref{sec:superposition_modulo_q}\\

$W$-state: $\frac{1}{\sqrt{n}}\sum_{i = 0}^{n-1}\ket{e_i}$ & $\mathO(n \log n)$ & $\mathO(1)$ & Section~\ref{sec:W_state_in_LAQCC}\\

Dicke-$(n,k)$, $k = \mathO(\sqrt{n})$: $\binom{n}{k}^{-1/2}\sum_{x \in \{0,1\}^n: |x| = k} \ket{x}$ &  $\mathO(n^2\log n)$ & $\mathO(1)$ 
&Section~\ref{sec:dicke:small_k}\\

Dicke-$(n,k)$: $\binom{n}{k}^{-1/2}\sum_{x \in \{0,1\}^n: |x| = k} \ket{x}$ & $\mathO(\text{poly}(n))$ & $\mathO(\log n)$ &Section~\ref{sec:Dicke_in_LAQCC_LOG}\\

QMBS: $\ket{S_k} = \frac{1}{k! \sqrt{\mathcal N(n,k)}}(Q^\dagger)^k \ket{\Omega}$ &  $\mathO(n^2\log n)$ & $\mathO(1)$  &  Section~\ref{sec:many_body_scar}
\end{tabular}
\caption{Summary of state preparation protocols given in this paper.}
\label{tab:sate_prep}
\end{table}
In the entry for the quantum many-body scar state $Q$ denotes the raising operator and $\mathcal N(n,k)=\binom{n-k-1}{k}$. 
Section~\ref{sec:many_body_scar} will provide more details on the variables and the implementation. 

\paragraph{Organization of the paper}
\noindent We first introduce relevant preliminaries in Section~\ref{sec:preliminaries}. 
In Section~\ref{sec:LAQCC_model} we formally define the class of Local Alternating Quantum-Classical Computations ($\LAQCC$). We also show that any Clifford circuit can be implemented in constant depth $\LAQCC$ (a result based on a result from measurement-based quantum computing~\cite{jozsa2006introduction}). 
This result allows us to give many useful multi-qubit gates and routines in Section~\ref{sec:gates_created_in_LAQCC}. 
Beyond that we show that constant depth $\LAQCC$ circuits are contained in $\QNC^1$ and that any $\mathsf{IQP}$ circuit has an $\LAQCC$ implementation.
We conclude this section with an analysis of a more powerful instantiation of $\LAQCC$ and show an inclusion with respect to the class $\mathsf{PostQPoly}$, which is the class of circuits of polynomial depth with one additional post-selection gate. 
In Section~\ref{sec:state_prep_in_LAQCC} we give $\LAQCC$ circuit implementations for preparing the uniform superposition over an arbitrary number of states, the $W$-state and the Dicke state up to $k = \mathO(\sqrt{n})$. We furthermore give a log-depth circuit implementation for preparing the Dicke state for any $k$. We conclude by showing a $\LAQCC$ circuit for generating many body scar states of a particular type of Hamiltonian.



\section{Related Work}
\label{sec:relatedworks}
%-------------------------------------------------------------------------------
\section{Related Works} \label{works}
%-------------------------------------------------------------------------------
\parab{Internet and Datacenter multicast.} Multicast has been widely applied in large-scale Internet applications, such as Internet broadcast \cite{iptv}, video conferencing \cite{chen2011celerity}, and multiplayer games \cite{cho2009enabling}, \etc Prior works for the Internet \cite{chiang2018online, huang2016multicast,diab2020oktopus,ren2018optimal, diab2022yeti} mostly focus on the multicast routing, \ie, to find promising multicast paths, inside ISPs. For instance, Yeti~\cite{diab2022yeti} supports multicast routing with traffic engineering and service chaining requirements for large-scale ISPs. Yeti creates labels representing forwarding information for multicast graphs and processes these labels to forward packets to targeted paths. Although there are a bunch of prior works on the Internet, most of them merely provide best-effort delivery, which only works for applications without reliability requirement. 

There are some works~\cite{widmer2001extending, rizzo2000pgmcc} aim to provide reliability for datacenter applications upon approaches with best-effort delivery. However, existing reliable multicast solutions mainly adopt a TCP-like software stack and cannot meet the demand for high-speed communication in datacenters. In contrast, \sys  leverages the advanced RDMA stacks to process multicast traffic, providing high-speed reliable communication.

%And there are few works for in-fabric multicast in datacenters with technical details~\cite{sharp}. \todo{software datacenter multicast.} 
%MTRSA \cite{huang2016multicast} is a multi-tree routing algorithm attempting to optimize multicast routing paths inside ISP with consideration of node and link capacity. \cite{ren2018optimal} aims to find the optimal path to forward traffic while preserving the ordered access of a sequence of network services. These services are usually deployed as virtual functions on core routers in the ISP network. 

\parab{Multicast scalability.}
Datacenter applications impose a demand for high scalability. As the traditional IP multicast~\cite{crowcroft1988multicast}, along with its native group management, IGMP and tree construction protocol, PIM~\cite{estrin1998protocol}, are poor in scalability, many works~\cite{shahbaz2019elmo, diab2022orca, li2013scaling} attempt to address the scalability issue, \ie, supporting as much as possible multicast groups. For example, Elmo~\cite{shahbaz2019elmo} encodes the routing link of a multicast tree into rules formatted as packet header. Thus Elmo switch only needs to maintain rule parsing logic, reducing the total switch-maintained states. Orca~\cite{diab2022orca} utilizes the large memory space of the server, making servers assist in forwarding packets, reducing the switch's burden on maintaining states. 

These works that address the scalability issue are orthogonal with the \sys design. Our goal in this work is to provide a general multicast protocol with prominent RDMA features and reliability guarantee rather than compressing the switch-maintained states. As mentioned before, \sys can support at least 1K multicast groups using 0.92MB space, which is acceptable for a majority of multicast applications in datacenters. \sys can support even more multicast groups when getting extended further upon these works.

%\parab{Group communication.}
%The group communication in the datacenter is not limited to one-to-many and many-to-many. There are various patterns \cite{wan2020rat, rashidi2021enabling}, such as all-gather, reduce-scatter, and all-reduce. \sys can be extended to support these group communication patterns, and we leave this as our future work.


\section{BonnBot-I}
\label{sec:platformHardware}

\bbot\ is a platform capable of both field monitoring and weed management developed at the University of Bonn.
The base platform is a Thorvald system~\cite{grimstad2017thorvald} which is a lightweight four-wheel-drive (4WD) and four-wheel-steering (4WS) system.
With considerable modifications, we adapted this platform to an arable farming and phenotyping robot suitable for operation in the real phenotyping fields.

These modifications were carried out to ensure they met European farming standards for the distance between rows.
Based on these specifications the width from wheel-centre-to-wheel-centre was set to $1.5m$.
To ensure we could monitor the different growth cycles of our primary crops (sugar beet, wheat, and corn) the vertical clearance is set to $0.57m$.
The length of BonnBot-I, $1.39m$, was selected to ensure there was adequate space to install replicating weeding tools.
These dimensions also act to increase the stability of the platform on the uneven terrain witnessed in arable farmland.
An overview of the \bbot\ platform is provided in~\figref{fig:fields}.

% \TODO{CM to ALL: I would remove the paragraph below.
% Key to deploying \bbot\ is the sensor array enabling navigation, data capture, crop monitoring and weed management at the same time.
% Hence, \bbot\ houses sensors of multiple interests: localization, navigation and crop monitoring encouraging operations in a vast variety of crop-fields with different growth-stages, number of crop-rows. 
% Additionally, benefiting our novel weeding tool implement, we are able to conduct selective and precision plant-level interventions in arable fields.
% For more platform details see~\ref{fig:fields}, on the left is a detailed schematic of the robot and on the right is a 3D generated model. These images detail the construction level specifics of \bbot\.
% In the following we elaborate more on sensor setup and weeding implement.}


% \textcolor{blue}{
% The placement of the GPS antennas was chosen to be consistent with a second purely phenotyping platform. 
% The purely phenotyping platform was chosen to have a vertical clearance of $1.6m$ and to ensure consistency between the two platforms the antennas of \bbot\ were mounted at the same position.}
% \TODO{CM to AA: I suggest we mainly talk about the dimensions of the weeding platform. I've commented the previous text and then talked about there being another robot for phenotyping that constrained where the GPS antennas had to be mounted.}
% \textcolor{blue}{
% the phenotyping version of this robot had to be able to sense a variety of crops at different stages of growth.
% This led to the choice of a vertical clearance of $1.6m$.
% The length of the robot was then set to ensure room for weeding tools and was set to $1.4m$, this also reduced issues for instability.}
% \TODO{The above dimensions need to have the real numbers for vertical clearance and area for the weeding tools. Workshop?}
% This led to the overall dimensions of the robot of $1.62 \times 1.8 \times 2.175m$ (width, length, height), as shown in Figure~\ref{}.
% With these parameters, this lightweight robot could be mounted on a regular truck provided that the GPS antennas were rotated, see Figure~\ref{}.
% \TODO{We need the details of the truck with dimensions, etc. Workshop?}

\subsection{Sensor Configuration}
\label{subsec:sensorConfiguration}


We equipped BonnBot-I with a range of sensors to perform both in-field weed management and crop monitoring.
There are two sets of sensors, the first set is for ``localization and navigation'' and the second set is for ``robotic vision''.
%This infrastructure enables the platform the achieve state-of-the-art capabilities in arable fields to perform critical tasks like: autonomous navigation, localization, and environment perception encouraging selective and plant-level real-time interventions.
%The sensors range from intertial measurement units (IMUs) and GNSS sensors for accurate localization through to 2D and 3D vision sensors such as RGB-D cameras and lidar; we will expand on the specific components here.
%Below we briefly these describe these components.

\subsubsection{Localization}
\label{subsubsec:navigation}

The localization of BonnBot-I is performed with a compact Inertial Navigation System (INS), Ellipse2-D SBG Systems~\cite{sbg} which includes an IMU and a dual-antenna receiver, multi-band GNSS receivers fixed at the front and back of the robot at height of $1.85m$ above the ground.
%Benefiting an on-board high-frequency Extended-Kalman filter fusion of IMU and GPS data providing a horizontal and vertical position accuracy of $4cm$ and $3cm$, respectively. 
Using an on-board high-frequency extended-Kalman filter fusion of IMU and GPS data provides us with a horizontal and vertical position accuracy of $4cm$ and $3cm$ respectively. 
Furthermore, the heading of the platform can be determined with an accuracy of $0.1^{\circ}$ and $0.3^{\circ}$ in roll-pitch and yaw directions respectively.
%
% To provide a homogeneous view of the environment we added an Ouster-OS1 lidar to the front. %at the front of the robot. 
% The OS-1 lidar is a 64 beam lidar covering $360$ and $45$ degree field of view in the horizontal and vertical directions, with a maximum range of $120m$.
%degrees horizontal and vertical field of views with maximum range of $120m$ meters.
% This provides long-range sparse depth data to to enable 3D mapping, navigation, and for safety.

%Even-though, in \cite{ahmadi2021towards} we showed how BonnBot-I is able to automatically traverse multi-crop-row field without any human intervention and independent of any global localization service using only two symmetrically fixed Intel RealSense-D435i in front and back of the robot denoted by $C_{f}$ and $C_b$ in \figref{fig:robot}.

% Additionally, to enable automated row-crop field traversing we use two RGB-D Intel RealSense-D435i cameras, one on front and one on the back of the platform with a fixed tilt angle.
% In~\cite{ahmadi2021towards} we showed that this sensor configuration could reliably navigate a field regardless of the number of crop-rows under the robot or the crop type; this was achieved without using any global localization service. 

%we showed that this sensor configuration could be used for automatic navigation in arable fields with diverse crop types as well as multiple crop-rows achieving reliable performance without using any global localization service.

\subsubsection{Robotic Vision}
\label{subsubsec:detectionInference}
BonnBot-I is equipped a nadir-view cameras (Intel Real-sense D455) on the front of the platform used for monitoring purposes.
% \RNum{1} \textit{JAI Camera:}
% The AD-130GE is a Rolling shutter CCD multi-spectral GigE Vision compliant camera. 
% This device employs 2 CCD sensors, one for Bayer color and the other for NIR monochrome utilizing prism optics so that the AD-130GE can inspect the objects by visible color sensor and Near IR sensor with the same angle of view.
% \RNum{2} \textit{Real-sense D455 Camera:}
The Real-sense D455 is a global shutter camera which provides RGB and registered depth images at $15 Hz$.
On \bbot\ it is fixed at a height of $0.78m$ providing a view-able area of $1.4m\times0.78m$ covering the gap between the two front wheels. %, with a $15 Hz$ frame-rate. 
This is the sensor used to perform field monitoring.
% This RGB-D camera is the primary sensor to perform field monitoring.
%used to perform field monitoring
%This sensor obtains depth information from two NIR cameras. 
%It has a NIR projector that is used to project points onto the scene so that depth can be estimated on visually texture-less surfaces such as walls.

% As this is not a problem with the setting for an agricultural robot to avoid interference with other sensors (e.g. NIR channel of the JAI camera) we disable the projector and consider this to be an RGB+D+NIR sensor.

\subsection{Weeding Implements}
\label{subsec:weedingImplements}

% Here I want to only describe the mechanic and electronics of the weeding tool 

%The necessity of this problem is getting bolder as we find out more about the harms that broadcast weeding reinforcements are causing to our eco-system.
Achieving flexible, and repeatable weeding implements that can deploy a variety of end-effectors is a key objective for BonnBot-I.
This enables the system to change tools given the current soil and weed populations.
The proposed design utilizes independently controlled high-resolution Igus linear actuators fixed at height of $0.72m$ above the ground and creating a working space of $1.3m\times0.36m$; the current design uses 4 such linear actuators.
%The proposed design utilizes four independently controlled high-resolution igus ZLW-1040S linear actuators fixed on the BonnBot-I at height of $0.6m$ above the ground and creating a working space of $1.3m\times0.36m$ as shown in \figref{fig:robotScheme}.
Each linear axis has a length of $1.3m$ and is controlled by an Igus dryve D1 motor control system via Modbus connection with a maximum resolution of $0.01cm$ and is capable of performing translations with maximum velocity and acceleration of $5m/s$ and $10m/s^2$, respectively.
All linear axes are equally-spaced and currently carry spot-spray nozzles, however, the system design permits many kinds of end-effectors, such as mechanical hoeing, providing flexibility. % for possible weeding tools like mechanical spot hoeing.
%all linear axes carry spray valves so can be used independently to engage with weeds.

To control the linear actuators and spray valves we use a ROS operated Raspberry-Pi 3B and to ensure minimal action delay the nozzles are accessed via high speed N-channel MOSFET-transistors.
Hence, ultimate operation time for each spray head in our system adds up to $10\sim12$ms including valves On-Off time. 
The spray system consists of a reservoir tank capable of carrying $5L$s of compressed liquid with a maximum $16bar$ pressure, as well as a compact $8bar$ portable compress which is fixed on the robot.
As the droplet size from the spray nozzles depends on the liquid pressure we use individually adjustable valves.
This allows us to control the spray footprints on the ground individually for each nozzle between $0.02m$ to $0.13m$; for this paper we assume a constant spray footprint of $0.05m$.



\begin{comment}
To control Linear actuator and nozzle valves we use a Raspberry-Pi 3-B single-board computer which all its outputs are interfaced via PiXtend v1.3 I/O board. 
Pixtend is a PLC (Programmable Logic Controller) compliant logic board with a wide range of isolated digital and analog inputs and outputs.
It provides standard serial interfaces RS232, RS485, Ethernet, CAN and etc, and is shown in ~\figref{fig:weeding_controler}.
We use High speed N-Channel MOSFET-Transistor (IRFZ44N 55V, 41A) of Pixtend board to control the spray nozzle valves, this ensures minimize action delay in electronics. 
% % Figure environment removed 
\subsubsection{Spray Infrastructure}
all linear axes carry spray implements so can be used independently to engage with weeds.
We use ASCO™ solenoid L172V03 spray valve with On-Off time equal to $\sim10ms$, which are lowered to a height of $0.22m$ using an aluminum level arm. 
The spray system consist of a reservoir tank capable of carrying $4L$ compressed liquid with maximum $80bar$ pressure.
And each nozzle is being fed with compressed herbicide via isolated pipes and control valve. 
This way we can control output pressure of each spray separately.
Also as the droplet size of spray nozzles depend on the liquid pressure we can control the spray footprints on the ground and droplet sizes individually for each nozzle.
\end{comment}


% \subsubsection{Hoeing Implement}
% On the two other linear axes we have fixed a high-speed pneumatic cylinders with length of $12cm$ used for turning soil and taking out the weeds from soil which in agriculture it is called hoeing.
% Hoeing aims to sever the top growth from the roots, just below the soil surface, then leaves the weed in the sun to wither to dry-out.
% While, this approach does not guarantee successful treatment to all types of weeds spatially deep-rooted or perennial weeds~\cite{larkcom2013grow}.
% But, by throwing the plant out of the soil we ensure to increase the success rate.
% Furthermore, by considering different types of hoeing axles 
% Long-handled hoes are easier on the back, whereas a short-handled ‘onion hoe’ is better for closely planted areas, where you don’t want to damage nearby plants. 
% We propose the following shape which ...
% The pneumatic cylinders control the positions of a lever-arm carrying a hoeing head used turning soil.
% As~\figref{fig:heingTool} shows the mechanism is fixed with a vertically fixed Aluminium lever-arm  

% % Figure environment removed







% \section{BonnBot-I Software Architecture}
% \label{sec:platformSoftware}



%% Figure environment removed



\subsection{In-Field Intervention Pipeline}
\label{subsec:interventionPipeline}
All sensors on \bbot\ are directly connected to a high performance fanless embedded computer (DS-1202) powered by 7th Generation Intel® Core™ processor running the robot operating system (ROS). 
This computer is equipped with an Nvidia® Quadro P2200 featuring a Pascal GPU with 1280 CUDA cores, 5 GB GDDR5X on-board memory which is used for parallel processing and inference of DNNs.
Additionally, a dedicated Intel® NUC PC runs the controller for the platforms motion providing wheel odometry data and the status of batteries and electronic infrastructure on the platform. % Thorvald. 

% \bbot\ incorporates a vision-based monitoring approach to perceive field as it moves in rows of crop. 
The software architecture of \bbot\ contains four different nodes enable selective in-motion intervention.
\RNum{1} Field Monitoring: runs Mask-RCNN for instance-based semantic segmentation and intra-camera tracking which estimates necessary phenotypic information about the plants, more details are in~\secref{subsec:fieldMonitoring}.
% This node provides track-lets of crops and weeds along with their estimated phenotypic information.
% To enhance accuracy of track-lets (crop/weeds which are detected) tacking, directly effecting interventions performances. 
\RNum{2} In-field localization: improves the localization accuracy of the robot using an EKF to fuse GPS and wheel odometry data~\cite{wei2011intelligent}.
%\RNum{3} Intervention Controller: Is responsible for managing the targets within work-space of weeding tools. This includes planning paths for the intervention heads (sprays) using our path planning scheme elaborated on in~\secref{subsec:selectivePreciseIntervenstion}.
\RNum{3} Intervention Controller: manages the targets within the work-space of the weeding tools and includes planning paths for the intervention heads (sprays) described in~\secref{subsec:selectivePreciseIntervenstion}.
\RNum{4} Weeding Implement: provides low-level control of the weeding tool (e.g. actuation) by taking commands from the intervention controller.
%\RNum{4} Weeding Implement: the low-level controller of the weeding tool which acts on the targets by taking commands from the intervention controller.
% 
\subsection{Weeding Simulation Framework}
\label{subsec:simulation}
% 
% Figure environment removed

Conducting experiments in real fields is potentially time consuming and costly.
An accurate and reliable simulation environment that mimics real world situations is an invaluable tool which avoids this issue.
%To resolve this issue an accurate and reliable simulation environment that mimics real world situations is an invaluable tool.
% Hence, a proper and reliable simulation environment which can mimic the real world situations is priceless. 
We use two different types of simulation environments for development and evaluation of our methods on \bbot\, a ROS based one-to-one scale simulation and a native python simulator especially developed for weeding application and intervention motoring purposes. 
A demonstration of the ROS-based simulation model is shown in~\figref{fig:simulationModel}-(left), where all sensors and actuators are active. 
% and supporting field generator plugin which generate fields with row-crop field with different plant sizes and crops row shapes.
Then a field generator allows us to create row-crop fields with different plant sizes and crop-row shapes, simulating a real field.
To evaluate the weeding algorithms, we developed a native Python based framework capable which simulates the robot kinematics and generates synthetic crop-rows with varying weed distributions.
%To properly evaluate the weeding algorithms we developed a native Python based framework capable of simulating robot kinematics and generating synthetic crop/weed distribution of crop-rows with different weed densities.
This framework uses Open3D and Pyglet python libraries for rendering graphics, an simplified example view of the planning scenario is shown in~\figref{fig:simulationModel}-(right).
We used this environment to implement and evaluate different weeding strategies which we elaborate on in~\secref{subsec:selectivePreciseIntervenstion}.

 

\section{Field Monitoring} 
\label{subsec:fieldMonitoring}

%The main use case of BonnBot-I is to manage weeds in arable fields.
%To perform this task it is equipped with a novel weeding tool described in \secref{subsec:weedingImplements}.
%The actual treatment controlled based on an advanced DNN driven monitoring approach whcih is explained in \secref{subsec:cropWeedMonitoring}. 
%The monitoring approach estimates critical phenotypic information like crop/weed classification and segmentation, stem location and pixel-wise area of plants (or BBCH) of crops located underneath the robot in real-time while robot is moving through crop-row.

%The scarcity of proper agricultural datasets suited for precision weed management still is a challenge in agriculture.
%Hence we gathered and annotated two datasets which are being used for our plant monitoring system which is more elaborated in the following.




%\subsection{Dataset}
%\label{subsec:dataset}
%\section{Dataset Description}
\label{sec:dataset}


In this section, we describe our dataset. 
\dataset contains \num{1209} unique roots.
A root refers to the first element in a package name. 
For example, in the package name "com.example.mypackage", "com" is the root.

We also collected data on the number of fields in the package names of the apps in our dataset. 
A field refers to a dot-separated element in a package name.
For example, in the package name "com.example.mypackage", there are three fields: "com", "example", and "mypackage".
Results are visible in Table~\ref{table:num_of_fields}.
There are 733 package names with only one field, \num{14368} package names with two fields, etc.

\begin{table}
    \centering
    \caption{Number of package names per field in \dataset}
    \begin{adjustbox}{width=.7\columnwidth,center}
        \begin{tabular}{lr|lr}
            \hline
            Fields & Count & Fields & Count \\ \hline
            with 1 field & \num{733} & with 5 fields & \num{100}\\ 
            with 2 fields & \num{14368} & with 6 fields & \num{6}\\ 
            with 3 fields & \num{4231} & with 7 fields & \num{3}\\ 
            with 4 fields & \num{720} & with 8 fields & \num{1}\\
            \hline \hline
            \multicolumn{2}{l|}{Total} & \multicolumn{2}{r}{\libsAfterRefinement}
        \end{tabular}
    \end{adjustbox}
    \label{table:num_of_fields}
\end{table}



There are significantly fewer package names with four or more fields. 
The number of package names with one field is relatively low compared to the others.
This suggests that many package names in the dataset follow a standard naming convention with a domain name followed by one or more subpackages.
The presence of package names with four or more fields may indicate the use of more complex or specialized naming conventions\footnote{Examples of libraries are:  
\href{https://mvnrepository.com/artifact/riddley/riddley}{riddley}, 
\href{https://mvnrepository.com/artifact/jakarta.annotation}{jakarta.annotation}, 
\href{https://mvnrepository.com/artifact/com.vogle.sbpayment}{com.vogle.sbpayment}, 
\href{https://mvnrepository.com/artifact/pl.robakowski.jersey.bootstrap}{pl.robakowski.jersey.bootstrap}, 
\href{https://mvnrepository.com/search?q=de.tudresden.inf.lat.jsexp}{de.tudresden.inf.lat.jsexp}, 
\href{https://mvnrepository.com/artifact/de.hs_rm.cs.vs.tools.vocabularygenerator}{de.hs\_rm.cs.vs.tools.vocabularygenerator}, 
\href{https://mvnrepository.com/artifact/eu.adlogix.com.google.api.ads.dfp}{eu.adlogix.com.google.api.ads.dfp}, 
\href{https://mvnrepository.com/artifact/us.gov.dot.faa.ang.c55/huggs}{us.gov.dot.faa.ang.c55.gradle.huggs}.
}.


Table~\ref{table:top_ten} presents the top 10 most frequent roots and the top 10 most frequent fields found. 
In the first two columns, we can see that the root "com" is by far the most frequent, with more than \num{10000} occurrences. 
The second most frequent root is "net", with \num{1265} occurrences. 
In the second two columns, which represents the most frequent fields, including the roots, we can see that the field "com" is still the most frequent. 
The second two columns do not differ much from the first two columns, except for the two fields "gradle" and "android" that now appear.
This could indicate that Android libraries are prevalent in the dataset.
It is confirmed in the last two columns, which represent the most frequent fields without the roots. 
After "gradle" and "android", the third most frequent field is "sdk", with 138 occurrences.
We see a shift in the most prevalent fields. 
Instead of roots, we now see fields such as "sdk", "maven", "plugin(s)", "api", "tools", and "common".
This may be indicative of the types of libraries.
Overall, the results suggest that most package names are from the "com" domain and that Android libraries are well represented.


\begin{table}
    \centering
    \caption{Top 10 roots and fields present in \dataset}
    \begin{adjustbox}{width=.9\columnwidth,center}
        \begin{tabular}{c|c|c|c|c|c}
            \hline
            \multicolumn{2}{c|}{\textbf{Top 10 roots}} & \multicolumn{2}{c|}{\textbf{Top 10 most used fields}} & \multicolumn{2}{c}{\textbf{Top 10 most used fields w/o roots}}\\ \hline
            \textbf{Root} & \textbf{Count} & \textbf{Field} & \textbf{Count} & \textbf{Field} & \textbf{Count} \\ \hline \hline
            com & \num{10520} & com & \num{10551} & gradle & \num{250} \\ \hline
            net & \num{1265} & net & \num{1273} & android & \num{239} \\ \hline
            de & \num{917} & de & \num{918} & sdk & \num{138} \\ \hline
            cn & \num{663} & cn & \num{663} & plugin & \num{120} \\ \hline
            dev & \num{458} & dev & \num{465} & maven & \num{108} \\ \hline
            me & \num{411} & me & \num{413} & plugins & \num{93} \\ \hline
            eu & \num{223} & gradle & \num{254} & api & \num{60} \\ \hline
            ru & \num{202} & android & \num{240} & oss & \num{57} \\ \hline
            fr & \num{188} & eu & \num{224} & tools & \num{54} \\ \hline
            ch & \num{181} & ru & \num{203} & common & \num{39} \\ \hline
        \end{tabular}
    \end{adjustbox}
    \label{table:top_ten}
\end{table}

%\subsection{Crop/Weed Instance Monitoring and Tracking}
%\label{subsec:cropWeedMonitoring}
%

\TODO{To perform field monitoring we make use of Mask-RCNN in conjunction with re-projection as described in our prior work by Halstead et al.~\cite{halstead2021crop}. 
Mask-RCNN provides instance-based semantic segmentation which includes the class information (e.g. weed species) and viewable surface area from the RGB-D camera.
The strength of the approach lies in being able to estimate how far the camera has moved, for instance Halstead~et~al. used wheel odometry, to re-project information between frames and conduct better object tracking as well as counting.
Despite the success of the proposed approach with wheel odometry, the extra localization information (e.g. GPS and IMU) available on \bbot\ should further enhance this technique.}

\TODO{In this paper we enhance demonstrate that the extra sensors on \bbot enhance the localization information which also enhances field monitoring.
In particular, we fuse the available odomtery and GPS information with an Extended Kalman Filter(EKF)~\cite{wei2011intelligent}.
This algorithm recursively estimates the state of a non-linear system in an optimal way.
Furthermore, adding local source of motion estimation can considerably reduce the risk of outage due to lack of proper satellite observations~\cite{ahmadi2021towards}.}

\TODO{The classic EKF method consists of two steps: the prediction and the correction. 
In prediction step the algorithm tries to predict future state of the system based on motion model, while the correction step improves predicted state's accuracy using real measurements from sensors. 
Using this principle, the state of the system can be determined recursively in real time.
In our system, the wheel odometry is used as control data~$u_t=(d_t^{t+1}, w_t)$ in the prediction step and position and orientation of the embedded EKF solution for the SBG system provides corrections.
As the robot velocity is always controlled with a differential controller model within crop-rows we express its kinematics model in x-y plane as:}
% 
\begin{equation}
    \begin{array}{lr}
        x_{t+1} = x_t + d_t^{t+1} \cdot cos(\theta_{t+1}), \\
        y_{t+1} = y_t + d_t^{t+1} \cdot sin(\theta_{t+1}). \\
        \theta_{t+1} = \theta_t + w_t
    \end{array}
    \label{eq:ekf_motionModel}
\end{equation}
% 
where $d_t^{t+1}$ is circular arc traveled within $t$ to $t+1$ and $w_t^{t+1}$ is the rotation angle around z-axis.
So, let the vector $\Vec{\mathbf{X}}_t = [x_t, y_t, z_t, \phi_t, \psi_t, \theta_t]^T$ be the state vector of the \bbot\ at time $t$.
where $x_t, y_t, z_t$ denote the current position in \textit{ENU} coordinate frame, and $\phi_t, \psi_t, \theta_t$ specify roll, pitch and yaw angles of the robot in world coordinate frame $\mathcal{F}_w$.

\TODO{In our prior work~\cite{halstead2021crop} some of the most challenging plants were grasses and so we propose to evaluate improvements to field monitoring (via improved localization with more sensors) on a grass crop. 
In particular, our evaluations for field monitoring in Section~\ref{sec:exp} are conducted on corn fields which have similar weed species to our prior work but the crop (the dominant plant) has a thinner leaf structure which makes it more difficult to track.}

%The trained model is used to evaluate the tracking performance of the BonnBon-I platform using varying types of odometry information.
%Specifically to improve the accuracy of image-based tracking results we tend to use Extended Kalman Filter(EKF) to fuse odomtery and GPS information which is explained in the following.
%
%However, wheel odometry is a coarse
%methodand tracking is counting is achieving by performing tracking which uses the re-projection of masks.
%This re-projection can benefit from the 

%then performedas our based method for instance-based semantic segmentation with conjunction with reprojection mathod explained in~\secref{subsec:intraCameraTracking}. 

%Similar to~\cite{halstead2018fruit, halstead2021crop}, we augment a Mask-RCNN network to incorporate super-class and sub-class classification heads. 
%The super-class represents generalized object recognition as a binary background versus plant classifier, while the sub-class outputs species level classifications of the crop and weed types.
%The overall benefit of this approach is the ability to accurately segment objects (super-class) while providing fine-grained classification with minimal extra overhead.

\begin{comment}
The performance of this approach is outlined in~\cite{halstead2021crop}, where its benefit of being crop and environment agnostic was shown for both arable farmland and a glasshouse setting.
The strength of this approach is further outlined in~\cite{halstead2020} where the parallel network structure was compared to a standard $N$-class network.
The parallel structure was able to more accurately detect the presence of objects in the scene while still supplying relevant species level information.

This approach is used on the two datasets mentioned in~\secref{subsec:dataset}.
For SB20 we train a model using the same parameters outlined in~\cite{halstead2021crop} and store the fine-grained and bounding box location, sub-class label, and size information.
These characteristics are used for the intervention experiments.
%
% % Figure environment removed 
%
We once again show the flexibility of this approach by applying it to a novel crop - corn.
While the corn fields have similar weed species to the sugar beet dataset of~\cite{halstead2021crop} the thinner leaf structure of the early growth corn makes it a complex crop to classify and segment.
The trained model is used to evaluate the tracking performance of the BonnBon-I platform using varying types of odometry information.
Specifically to improve the accuracy of image-based tracking results we tend to use Extended Kalman Filter(EKF) to fuse odomtery and GPS information which is explained in the following.
\end{comment}

\begin{comment}
In~\cite{halstead2021crop} we showed that by augmenting Mask-RCNN architecture with super-class and sub-class heads we can avoid miss-classifications of less represented sub-classes. 
Specially for agriculture we can consider that their is a primary class (plant) with sub-classes (fine-grained classification) which in most of the cases the distribution of species appearance are not uniform.
The benefit of this approach is demonstrated in~\cite{halstead2021crop} for both glasshouses, and arable farmland and outlined the benefit of having crop, environment, and platform invariant approaches.
Figure~\ref{fig:subcls} outlines the basic sub-class structure inserted into the Mask-RCNN network.
This super-class (plant) and sub-class (species) alignment ensures that the super-class generalises to what a ``plant'' is while the sub-class provides the species level classification required for weeding~\cite{halstead2018fruit, halstead2020}.
While these less represented classes may be missed entirely in an $N$-class method the generalised super-class approach is still able to identify them as a ``plant''.


% Mask-RCNN is designed to be an $N$-class classifier with an bounding box regression and instance based segmentation within the bounded region.
% The strength of this is that multiple classes can be classified, regressed, and segmented within the same network.
% However, this approach is not always optimal.
% For agriculture we can consider that their is a primary class (plant) with sub-classes (fine-grained classification).

% This super-class (plant) and sub-class (species) alignment ensures that the super-class generalises to what a ``plant'' is while the sub-class provides the species level classification required for weeding.
% A key benefit of this technique~\cite{halstead2018fruit, halstead2020, halstead2021crop} is the ability to still recognise less represented classes.
% While these less represented classes may be missed entirely in an $N$-class method the generalised super-class approach is still able to identify them as a ``plant''.

% As a species level classifier~\cite{halstead2021crop} showed the benefit of this approach for both glasshouses, and arable farmland and outlined the benefit of having crop, environment, and platform invariant approaches.
% Figure~\ref{fig:subcls} outlines the basic sub-class structure inserted into the Mask-RCNN network.
% Generally, this network has a classification head, and regression head, and a mask head.
% In our approach we insert a sub-class classifier in the same network location as the classifier and regressor, which utilise the same embedding layer as these two.



% BonnBot-I is a arable farmland monitoring platform that works in row-crop fields.
% This allows it to capture data of any crop type that in these fields. 
Previously we have used sugar beet data captured on this platform in~\cite{halstead2021crop}, to show the flexibility of this approach we will utilise corn data captured on the same platform.

In this case the plants as a whole represent the super-class and the species represent the sub-class.

\begin{table}[!h]
    \vspace{-2mm}
	\centering
	\caption{\TODO{Dataset Sub Categories ... - MAH I don't think we need this table, we have already given the corn annotations previously.}}
	\begin{tabular}{l cc cc cc cc c}
	\toprule
	%&& \multicolumn{2}{c}{BUP} \\\hline
	  & \textbf{SB} & \textbf{CN} & \textbf{Bi} & \textbf{An} & \textbf{Chy} & \textbf{Pe} & \textbf{Th} & \textbf{Ch} & \textbf{Un} \\\hline
	 \midrule
    SB20 & 768 & - & 241 & 19 & 64 & 620 & 775 & 232 & 206 \\ 
    CN20 & - &  &  &  &  &  &  &  & \\
    \bottomrule
    \end{tabular}
    % \vspace{-4mm}
	\label{tab:dataset_cats}
\end{table}
\end{comment}
% Mask-RCNN with species level weed classification.

% \cite{halstead2018fruit, halstead2020, halstead2021crop}

% \begin{itemize}
%     \item Standard mask and faster pipeline. Talk about N Classes.
%     \item Refer to \cite{halstead2018fruit} and \cite{halstead2020} for an overview of why a super-class sub-class relationship does a better job when object counts are limited compared to an N-super-class classifier.
%     \item Describe the sub-class layer, maybe with a NEW figure.
%     \item refer to \cite{halstead2021crop} to show it working as a species-level classifier.
%     \item Talk about how the platform enables this stuff.
%     \item Corn seg
% \end{itemize}

% % Figure environment removed



\begin{comment}
\subsubsection{Precise in Field localization}
\label{sec:fieldLocalization}
The accuracy of in-field localization directly influences the accuracy of monitoring system and interventions. 
Hence to maximize weeding operations resolution a millimeter-level accurate position determination is crucial.
To increase the accuracy of robot localization from couple of centimeters (provided by SBG system) to level of a few millimeters we fuse SBG measurement with robot's wheel odometry using Extended Kalman Filter (EKF)~\cite{wei2011intelligent} which is an algorithm to recursively estimate the state of a non-linear system in an optimal way.
Furthermore, adding local source of motion estimation can considerably reduce the risk of outage due to lack of proper satellite observations~\cite{ahmadi2021towards}.

The classic EKF method consist of two steps:the prediction and the correction. 
In prediction step the algorithm tries to predict future state of the system based on motion model, while the correction step improves predicted state's accuracy using real measurements from sensors. 
Using this principle, the state of the system can be determined recursively in real time.
In our system, the wheel odometry is used as control data~ $u_t=(d_t^{t+1}, w_t)$ in prediction step and position and orientation of embedded EKF solution of SBG system provides corrections.
As the robot velocity is always controlled with a differential controller model within crop-rows we express its kinematics model in x-y plane as: 
% 
\begin{equation}
    \begin{array}{lr}
        x_{t+1} = x_t + d_t^{t+1} \cdot cos(\theta_{t+1}), \\
        y_{t+1} = y_t + d_t^{t+1} \cdot sin(\theta_{t+1}). \\
        \theta_{t+1} = \theta_t + w_t
    \end{array}
    \label{eq:ekf_motionModel}
\end{equation}
% 
where $d_t^{t+1}$ is circular arc traveled within $t$ to $t+1$ and $w_t^{t+1}$ is the rotation angle around z-axis.
So, let the vector $\Vec{\mathbf{X}}_t = [x_t, y_t, z_t, \phi_t, \psi_t, \theta_t]^T$ be the state vector of the \bbot\ at time $t$.
where $x_t, y_t, z_t$ denote the current position in \textit{ENU} coordinate frame, and $\phi_t, \psi_t, \theta_t$ specify roll, pitch and yaw angles of the robot in world coordinate frame $\mathcal{F}_w$.



% In the correction step, using SBG module output and wheel Odomtery readings we determine travelled distance based on odometry, GPS and O which are contained in measurement vector as $\mathbf{d}_{xy}^{o/s}$ in row. 
% This leads to the following measurement vector as time $t$:
% \begin{equation}
%   \label{eq:ekf_measurementVec}
%   \Vec{\mathbf{z}}_t = [x^s, y^s, z^s, \mathbf{d}_{xy}^{o/s}, a_z^i, \phi^i, \psi^i, \theta^i, \boldsymbol{\omega}_x^{i}, \boldsymbol{\omega}_y^{i}, \boldsymbol{\omega}_z^{i}]
% \end{equation}
% where, SBG ($s$) ensure the absolute positioning is obtained, 
% as , the IMU ($i$) provides absolute roll, pitch and yaw angles 
% where, $a_z^t$ represent the acceleration in z-direction and $\boldsymbol{\omega}_x^{i/v}, \boldsymbol{\omega}_y^{i/v}, \boldsymbol{\omega}_z^{i/v}$ indicate the angular velocities in x,y and z directions estimated from VO and IMU units.

% EKF explanation + VO version !! 
\begin{comment}
\subsubsection{Extended Kalman Filter (EKF)}: 
To fuse measurement of different systems we use the Extended Kalman filter (EKF)~\cite{??} which is an algorithm to recursively estimate the state of a non-linear system in an optimal way.
This classic method consist of two steps:the prediction and the correction. 
In prediction step the algorithm tries to predict future state of the system based on the system model, while the correction step improves this prediction using real measurements or additional information. 
By applying this principle, the state of the system can be determined recursively in real time.
Let $\Vec{\mathbf{X}}_t = [x_t, y_t, \phi_t, \psi_t, \theta_t]^T$ denote the state vector of the \bbot\ at time $t$.
where $x_t, y_t, z_t$ denote the current position in \textit{NEU} coordinate frame, and $\phi_t, \psi_t, \theta_t$ specifies the roll, pitch and yaw angles of the vehicle in world coordinate frame $\mathcal{F}_w$.
As the robot is always controlled with differential model within crop-rows we express its kinematic model in x-y plane as: 
\begin{equation}
    \begin{array}{lr}
        x_{t+1} = x_t + d_t^{t+1} \cdot cos(\theta_{t+1}), \\
        y_{t+1} = y_t + d_t^{t+1} \cdot sin(\theta_{t+1}). \\
        \theta_{t+1} = \theta_t + w_t^{t+1}
    \end{array}
    \label{eq:ekf_motionModel}
\end{equation}
Where $d_t^{t+1}$ is xy-distance traveled within $t$ to $t+1$ and $w_t^{t+1}$ is the rotation angle around z-axis.

In the correction step, using SBG module embedded EKF output and Visual-Odomtery we determine travelled distance based on odometry, GPS and VO which are contained in measurement vector as $\mathbf{d}_{xy}^{o/g/v}$ in row. 
This leads to the following measurement vector as time $t$:
\begin{equation}
  \label{eq:ekf_measurementVec}
  \Vec{\mathbf{z}}_t = [x^g, y^g, \mathbf{d}_{xy}^{o/g/v}, a_z^i, \phi^i, \psi^i, \theta^i, \boldsymbol{\omega}_x^{i/v}, \boldsymbol{\omega}_y^{i/v}, \boldsymbol{\omega}_z^{i/v}]
\end{equation}

as GPS ($g$) we ensure the absolute positioning is obtained, the IMU ($i$) provides absolute roll, pitch and yaw angles 
where, $a_z^t$ represent the acceleration in z-direction and $\boldsymbol{\omega}_x^{i/v}, \boldsymbol{\omega}_y^{i/v}, \boldsymbol{\omega}_z^{i/v}$ indicate the angular velocities in x,y and z directions estimated from VO and IMU units. 
\end{comment}

%  VO part asnd tracking part
\begin{comment}
\subsubsection{Intra-Camera Target Tracking}
Tracking-via-segmentation aims to exploit known properties of the agricultural scene as a robotic platform traverses a row.
As the scene remains relatively static between captures (from the camera on the platform) it can be assumed that the plants are both temporally and spatially static.
Assuming these properties we are able to conclude that an object captured at $t$ will be close in image location at $t+1$.
The platform movement and the frames-per-second of the camera allow us to make this assumption as we are not moving too rapidly that the scence changes dramatically.
Halstead et al.~\cite{halstead2021crop} provides a detailed outline of this approach for the original intersection-over-union based approach.

Early tracking-via-segmentation~\cite{halstead2018fruit} relies heavily on the  assumption of spatial and temporal consistancy. 
If the captured images are too far apart objects can not be matched which duplicates tracklets.
However, in~\cite{smitt2020pathobot} we showed that even with larger movements between captures can be aggregated together through reprojection using the wheel odometry of the platform and the depth information.
One problem with these early approaches was the matching criterion: IoU.
In~\cite{halstead2021crop} it was shown that for small discrepancies in image captures, even with reprojection, this criterion struggles to match objects, particularly small objects.

To allow for this~\cite{halstead2021crop} outlined a new matching criterion and compared it directly to the IoU.
This dynamic radius was able to match objects, even small objects, due to the underlying technique of using the Euclidean distance compare the center of mass between two objects, if this value is below a threshold, based on a radius value, the objects are aggregated into a single tracklet.
There is a single issue with this approach in that it can match 360$^o$ around the center of mass meaning incorrect matches do occur.
\TODO{total plant number od corn dataset is: row 7 & 545, row 9 & 802, row 11 & 713}
% \begin{table}
%     \centering
%     \caption{\TODO{Ground truth of the corn tracking data - MAH Put this in the text no reason for a table.}}
%     \label{tab:gttrack}
%     \begin{tabular}{|l|c|}
%         \hline
%         row id & total plants \\
%         \hline\hline
%         row 7 & 545 \\
%         row 9 & 802 \\
%         row 11 & 713 \\
%         \hline
%     \end{tabular}
% \end{table}

% \cite{smitt2020pathobot, halstead2021crop}
% \begin{itemize}
%     \item Describe tracking via segmentation with a figure.
%     \item Talk about \cite{smitt2020pathobot} as the initial trials to get this working.
%     \item Introduce \cite{halstead2021crop} and the reasons for DR over IoU and the benefits of it.
%     \item Talk about how the platform enables this type of thing.
%     \item Corn tracking?
% \end{itemize}
\begin{itemize}
    \item explain the necessity of VO and improvement of the tracking
    \item Describe VO approach briefly!
    \item explain relevance of VO output to prev. section (intra-camera tracking).
    \item explain ekf where VO gets fused with GPS output to take positioning accuracy from couple of centimeters to millimeter accuracy! (must be proved ... maybe be with simple map or orthomosaic!)
\end{itemize}
% Using Depth information of D455 sensor we are able to accurately localize and geo-reference each crops/weed on the ground, ??? increasing the intervention accuracy.
% Accessing consistent input modalities with such methods can reveal helpful representations of data and aid in estimating phenotypic information at without any extra costs for the system. 
% A like the depth modality which can be used for estimating growth-stage of crops and geometric based data association between consequent frames.
Visual Odometry~\cite{ahmadi2021registration} 
\end{comment}
\end{comment}


% \bbot\ performs not only weed management but also field monitoring.
One of the key benefits of \bbot\ as a platform is that it can monitor the state of the field while traversing it, this supplements its key function as a weeding platform.
We demonstrate the potential of \bbot\ for field monitoring by illustrating how the extra localization sensors can enhance the existing tracking algorithms in our prior work~\cite{halstead2021crop}.
This approach used Mask-RCNN to provide instance-based segmentation, species-level information (e.g. crop and weed species) and the viewable surface area. % from the RGB-D camera.
This approach included a tracking-via-segmentation technique that outlined the benefit of of spatial matching operator, coined dynamic radius (DR), over a pixel-wise version, referred to as intersection over union (IoU).
This technique also exploited re-projection between frames using wheel odometry and camera parameters.
This enabled more accurate tracking of objects in the scene, however, wheel odometry is often prone to errors.
Using the extra GPS sensors available on \bbot\ has the potential to increase performance by re-projecting more accurately between frames. %, especially over large frame skips.
% The strength of the approach lied in being able to estimate how far the camera had moved, for instance Halstead~et~al. used wheel odometry, to re-project information between frames and conduct better object tracking as well as counting.
% Despite the success of the proposed approach with wheel odometry, the extra localization information (e.g. GPS and IMU) available on \bbot\ should further enhance such an approach.

In this paper we demonstrate that the extra localization sensors on \bbot\ can be used to enhance the performance of field monitoring.
In particular, we fuse the available odometery and GPS information with an EKF~\cite{wei2011intelligent}.
This algorithm recursively estimates the state of a non-linear system in an optimal way.
Furthermore, adding local source of motion estimation can considerably reduce the risk of outage due to lack of proper satellite observations~\cite{ahmadi2021towards}.

\begin{comment}
The classic EKF method consists of two steps: the prediction and the correction. 
In prediction step the algorithm tries to predict future state of the system based on motion model, while the correction step improves predicted state's accuracy using real measurements from sensors. 
Using this principle, the state of the system can be determined recursively in real time.
In our system, the wheel odometry is used as control data~$u_t=(d_t^{t+1}, w_t)$ in the prediction step and position and orientation of the embedded EKF solution for the SBG system provides corrections.
As the robot velocity is always controlled with a differential controller model within crop-rows we express its kinematics model in x-y plane as:
% 
\begin{equation}
    \begin{array}{lr}
        x_{t+1} = x_t + d_t^{t+1} \cdot cos(\theta_{t+1}), \\
        y_{t+1} = y_t + d_t^{t+1} \cdot sin(\theta_{t+1}). \\
        \theta_{t+1} = \theta_t + w_t
    \end{array}
    \label{eq:ekf_motionModel}
\end{equation}
% 
where $d_t^{t+1}$ is circular arc traveled within $t$ to $t+1$ and $w_t^{t+1}$ is the rotation angle around z-axis.
So, let the vector $\Vec{\mathbf{X}}_t = [x_t, y_t, z_t, \phi_t, \psi_t, \theta_t]^T$ be the state vector of the \bbot\ at time $t$.
where $x_t, y_t, z_t$ denote the current position in \textit{ENU} coordinate frame, and $\phi_t, \psi_t, \theta_t$ specify roll, pitch and yaw angles of the robot in world coordinate frame $\mathcal{F}_w$.
\end{comment}

In our prior work~\cite{halstead2021crop} we concentrated on sweet pepper in a horticultural setting and sugar beet in arable farmland.
In both cases the objects we aim to detect are somewhat robust to external influences such as weather conditions.
However, some of the weed species witnessed in the arable farmland were grasses and their accurate localization proved difficult.
In this paper we further outline the ability for our approach to be crop agnostic by performing monitoring on a novel grass crop dataset consisting of corn.
Corn has a long leaf structure which makes it susceptible to weather conditions resulting in a difficult crop to localize, due to this, in~\secref{subsec:cropWeedMonitoring} we outline our performance using both the pixel-wise and spatial matching criteria from~\cite{halstead2021crop}.
% \TODO{MAH this needs some work, also missing reference:
% In our prior work~\cite{halstead2021crop}, some of the most challenging plants were grasses and so we propose to evaluate improvements to field monitoring (via improved localization with more sensors) on a grass crop. 
% In particular, our evaluations for field monitoring in Section~\ref{subsec:cropWeedMonitoring} are conducted on corn fields which is a grass crop that has a thin and long leaf structure which makes it more difficult to track.
% }

This corn data set (CN20) was acquired using \bbot\ from a phenotyping field at campus Klein-Altendorf (CKA) of the University of Bonn.
The data was captured using an Intel RealSense D435i sensor with a nadir view of the ground in front of the robot and resolution of $1280\times720$ with a frame-rate of $15Hz$. 
The non-overlapping training, validation, and evaluation data includes RGB-D frames for $170$, $43$ and $70$ images respectively.
This data comes from six different rows providing unique crop and weed distributions due to the non-homogeneous growth stage of the weeds.
% The data comes with $283$ RGB and depth frames covering six rows of crop providing a large distribution of crops sizes due to non-homogeneous growth stage and several weed types.
In total there are nine different categories of weeds containing a total of $2566$ and $1261$ instances of crop and weeds respectively. 
The data is annotated to include instance based pixel-wise segmentation, bounding boxes and stem locations of each instance in Coco format~\cite{lin2014microsoft}.
% To train out DNN-based monitoring system we divided the images of dataset in to sets of training, validation and evaluation of size $170$, $43$ and $70$, respectively.
% $149$, $39$ and $44$,
\figref{fig:datasetSample} shows an example annotated image of CN20 dataset.

% Figure environment removed 


\section{Selective Precise In-Field Intervention} 
\label{subsec:selectivePreciseIntervenstion}


\bbot\ is equipped with a novel weeding tool design enabling high precision plant-level field interventions. 
It consists of a set of replicated linear actuators and is controlled via the intervention controller unit consisting of several components which are elaborated in the following sections.
%The weeding implement is controlled via the intervention controller unit consisting of several components which are elaborated in the following sections.
We briefly explain the conceptual design of the weeding tool, its requirements and operation assumptions.
Then, we introduce our method for managing targets in the work-space of the weeding tools.
Finally, we elaborate on path planning strategies used for controlling intervention heads in action.

% To properly deploy proposed weeding implement one important challenges is how to efficiently plan paths for intervention heads, such that we minimize the number of not managed weeds in a weeding scenario.

% We define the weeding scenario as below:
% \RNum{1} weeds get detected, classified an tracked within the view-able area of down facing camera $C_{detect}$ via monitoring system explained in~\secref{subsec:cropWeedMonitoring}.
% \RNum{2} the track-lets of each detected instance is being passed to a proper weeding actuator weeding-tool manager node.
% \RNum{3} each intervention path planner plans an optimal path and action time relative to meta information of each track-let like: weed type, size and category as it is explained in ??? in more details. 
% \RNum{4} goal target commands gets published through ROS, sending nozzle on actuator $i$ to requested target $j$.
% \RNum{5} valve of nozzles get activated based on heads pose relative to target weed (stem or bbox?)


\subsection{Plant-level Treatment In Field}
\label{subsec:plantLevelTreatment}

%We assume the robot moves along a crop-row with constant speed $\gamma$ so the kinematics of weeds at time $t$ w.r.t the weeding implement on the robot could be visualized as~\figref{fig:kinematicsWeeding}.
%Consequently the intervention is time-critical and must respect the spatial ordering of the weeds.
%Similar to~\cite{bawden2017robot} we assume weeds are uniformly distributed in the field with density $\lambda$ weeds/$m^2$. 
%Hence, using a Poisson process we can explain the distance between the weeding implement and individual weeds by accounting for the arrival rate of $\eta =\lambda \times \Pi$.
%We use the motion along the $x$-axis of the robot frame $\mathcal{F}_R$, to explain the weeds interval distance ($\delta_x$).

We assume the robot moves along a crop-row with constant speed $\gamma$.
Consequently, intervention is time-critical and must respect the spatial ordering of the weeds.
There is a constant gap ($\Gamma$) between the tools and the area sensed by the camera ($C_{detect}$).
Similar to~\cite{bawden2017robot} we assume weeds are uniformly distributed in the field with density $\lambda$ weeds/$m^2$. 
Hence, using a Poisson process we can explain the distance between the weeding implement and individual weeds by accounting for the arrival rate of $\eta =\lambda \times \Pi$.
We use the motion along the $x$-axis of the robot frame $\mathcal{F}_R$ to explain the weeds interval distance ($\delta_x$), visualized in ~\figref{fig:kinematicsWeeding}.
This can be shown using the following probability density function,
% 
\begin{equation}
    \label{eq:weedsIntervaldistance}
    f(\delta_x) = \lambda \Pi e^{-\lambda \Pi \delta_x},
\end{equation}
% 
also the location of weeds on the $y$ axis can be represented via a uniformly distributed random variable $y$ as,
% 
\begin{equation}
    \label{eq:yPDF}
	f(y) =\left\{\begin{array}{cc}
	    \frac{1}{\Pi} & for  \ \ 0 \leq y \leq \Pi \\
	     0 & otherwise
	\end{array}\right..
\end{equation}
% 
To engage the $i$-th intervention head with the $j$-th weed it has to traverse,
% 
\begin{equation}
    \label{eq:yDistance}
    \delta^{ij}_{y} = | \textit{h}_{i} - \textit{n}_{j} |, 
    \text{ where }  0 \leq \delta^{ij}_{y} \leq \Pi ,
\end{equation}
%
where $\textit{h}_{i}$ is the current position of $i$-th intervention head and the $\textit{n}_{j}$ denotes to the position of the $j$-th weed.
Therefore, the probability of visiting the $j$-th weed with $i$-th intervention head can be calculated with,
% 
\begin{equation}
    \label{eq:weedingProb}
    \textit{P}_{ij} = \textit{P}\left( \dfrac{\gamma}{\vartheta}  <   \dfrac{\delta_{x}}{\delta_{y}} \right),
\end{equation}
%
where $\vartheta$ denotes to the maximum velocity of linear axes.
We assume all targets are detected by the time they reach the bottom edge of the camera's viewable area.
% \TODO{MAH confusing: We suppose targets get detected by the time passing the center of viewable area of camera $C_{detect}$.}
%As~\figref{fig:kinematicsWeeding} depicts there is a gap with a length of $\Gamma$ between the work-space of the weeding implement and the bottom edge of the viewable area of the front camera $C_{detect}$. % in front of robot.
%Finally, this distance provides extra time for the planning system, this is defined by $\tau_e=\Gamma/\gamma$ and depends on the linear velocity of robot and length of region~$\Gamma$.
%% \TODO{MAH confusing: Finally, this provides extra time for detection and planning which we refer to it by $\tau_e=\Gamma/\gamma$ that depends on the linear velocity of robot and length of region~$\Gamma$.}
% Let $\kappa = \tau_v + \tau_i$ represent the intervention head engagement time for our spot spray systems to be constant, 
% where $\tau_v$ shows the constant operation time of the nozzle valve (On-Off time) and $\tau_i$ denotes the engagement time with weed $i$ in milliseconds.
%
% Figure environment removed






\subsection{Target-Space Management}
\label{subsec:targetSpaceManagement}

Planning the motion of each intervention head must be done prior to targets entering the weeding tool work-space. % of weeding tool. 
In the proposed workflow, the intervention controller node receives the detected targets at time $t + \tau_d$ where $\tau_d$ is the time required for detection in the monitoring node.
The monitoring node provides plant specific information like: plant category, pixel-wise segmentation, estimated area, and the bounding box.
Furthermore, we estimate plant centers based on provided bounding box in the scene.  % around the detection.
%Then, the aim would be to find the best motion plans for all interventions heads to minimize the targets which are not being visited (sprayed).
%This information is then used in the target-space management step to assign targets between intervention heads.
This information is then used in the target-space management step to assign targets to the intervention heads.
Based on this the next step finds the best motion plan for each intervention head by maximizing the number of targets that are visited (sprayed).
%Based on this the next step finds the best motion plan for each intervention head that maximizes the number of targets that are visited (sprayed).

Let $\mathcal{H}$ denote the number of independent controllable intervention heads and $\mathcal{N}$ be the number of targets that appear under the robot. %to be the number of targets appeared in the underneath the robot. %, the main goal is to visit all the targets with at least one of the intervention heads as they pass the work-space of weeding tool without the need for stopping the robot.
%By considering the fact that robot is moving forward and, we refrain from moving backwards, the intervention is time-critical and must be in spacial order.
We use a uni-directional constrained node-graph to model the targets-space.

To obtain the global spatial order of targets in a segment we use the $\delta_x$ of each weed (see \figref{fig:kinematicsWeeding}).
In~\figref{fig:nodeGraph}(a), each node (circle) shows a weed along with the connecting path between nodes $j$ to $k$ represented with a uni-directional link (arrow) $l_{jk}$.
The link $l_{jk}$ exists if, node $j$ geometrically is located after node $k$ in the 3D world frame  $\mathcal{F}_w$ in the direction of motion.
Furthermore, the link $l_{jk}$ is associated with an inter-weed cost $\varrho_{jk}$ based on the distance of nodes $j$ and $k$ and a property denoting motion probability of $\textit{P}_{jk}$ based on~\eqref{eq:weedingProb}.
We calculate inter-weed costs using the top-right of the cost-matrix~$\mathcal{G}_{\mathcal{N} \times \mathcal{N}}$ (to respect the weeds spatial order).


% \begin{equation}
%     \label{eq:costMatrix} 
%     \varrho_{jk} = (n_j - n_k)^2
% \end{equation}
% 
There are $\mathcal{H}$ independent interventions heads and so multiple plans which can lead to the same number of targets being visited (sprayed).
%As the weeding tool contains $M$ independently controllable interventions heads, it provides this flexibility to plan a variety of engagement routes by considering different method of target assignments.
%Considering sets of weeds in any given work-space motivates a multi-query approach  to the problem.
To solve this problem, we consider the weeds as a sets of targets detected in one location, this motivates us to assign intervention targets to the $\mathcal{H}$ heads as either distance-based or work-space division-based assignments.
% To solve this problem, we consider weeds to be presented as a set which motivates us to consider assigning targets to the $\mathcal{H}$ intervention heads as either a distance-based assignment or a work-space division-based assignment approach.
%Prior to plan engagement routes, we use a high level planner to distribute targets between different intervention heads while, making sure that all the targets will be at least visited once. 
%we suggest to represent the work-space in two different ways:

\begin{enumerate}
    \item \textbf{\textit{Distance-based Target Assignment (D)}:}
        In this approach, target $j$ gets assigned to the laterally closest intervention head along the sliding direction ($y$-axis). 
        This means, selected intervention head $i$ has the least motion required to reach the weed $j$.
        The lateral distance between heads and weeds are defined based on 2D euclidean distance between projection of intervention head's position on ground plane and weed positions on same plane w.r.t the $\mathcal{F}_w$ frame.
        
    \item \textbf{\textit{Static Work-space Division-based Target Assignment (SD)}:}
        In this method, we divide the work-space of weeding tool to $\mathcal{H}$ sub-sections of width $\Pi/\mathcal{H}$ meters.
        Hence, each intervention head is only responsible for engaging with weeds laying within it's sub-work-space as shown in~\figref{fig:nodeGraph}(b)-top.
        
    \item \textbf{\textit{Dynamic Work-space Division-based Target Assignment (DD)}:}
        % \TODO{MAH let's talk about this, it still confuses me reading this and then looking at the figure.}
        In this model, for each new set of detected weeds we first determine the minimum region of intervention defined by $y_{min}$ and $y_{max}$ (see Fig.~\ref{fig:nodeGraph}~(b)). 
        The minimum region of intervention is then divided into $\mathcal{H}$ equal sub-regions.
        This process assists in optimizing the planning for weed engagement by potentially reducing the area any one tool has to cover.
        %In this model, for each new segment we define an intervention necessary region by finding the spared of weeds positions along $y$-axis underneath the robot. 
        %Then we divide this region into $\mathcal{H}$ equal sub-regions for performing weeding engagements as shown in~\figref{fig:nodeGraph}(b)-bottom.
        %This technique ensures more proper load balancing over interventions heads in case of facing portions of the fields with non-uniform weed distribution.
\end{enumerate}
% 
%We note that the above approaches have been derived to demonstrate the potential of our proposed system and we believe future work can derive more advanced approaches to provide even better coverage.

% Figure environment removed



\subsection{Intervention Heads Route Planning}
\label{subsec:routePlanning}

\begin{comment}
At this point we propose that a subset of targets at time $t + \tau_d + \tau_m$ are assigned to intervention head $i$, where $\tau_m$ is the time required for segment management.
The problem which must be addressed is how to plan $\mathcal{H}$ independent efficient routes using the the prior knowledge of an intervention head's position, robot linear speed as well as the limits of speed and acceleration of linear axes.
%The problem which must be addressed is: with the prior knowledge of an intervention head's position, robot linear speed and limits of speed and acceleration of linear axes, plan $\mathcal{H}$ independent efficient routes to guide intervention heads.
We refer to the time required for planning as $\tau_p$.
The planned routes must guide intervention heads through all their assigned targets while minimizing the chance of missing any target.
\end{comment}

%At this point we propose that a subset of targets at time $t + \tau_d + \tau_m$ are assigned to intervention head $i$, where $\tau_m$ is the time required for segment management.
Here we address how to plan $\mathcal{H}$ independent efficient routes.
The planned routes must guide intervention heads through all their assigned targets while minimizing the chance of missing any target.
This has to take into account the the prior knowledge of an intervention head's position, robot linear speed as well as the limits of speed and acceleration of linear axes.
%The problem which must be addressed is: with the prior knowledge of an intervention head's position, robot linear speed and limits of speed and acceleration of linear axes, plan $\mathcal{H}$ independent efficient routes to guide intervention heads.
%We refer to the time required for planning as $\tau_p$.

The planning approach generates~$m$ potential trajectories~$\Vec{\mathbf{T}}=[\Vec{T}_0,\dots,\Vec{T}_m]$ for each intervention head.
Each trajectory $\Vec{T}_i$ is an ordered list of length $q$ consisting of weed positions which can be visited.
%included in the trajectory with length $q$. }
%\TODO{CM to AA: You wrote $\Vec{T}_m$ but do you mean $\Vec{T}_{\mathcal{H}}$, No this is actually $\Vec{T}_m$ where $m$ is the number of predicted possible trajectories for each intervention head}.
%To obtain possible trajectories, we use the following approaches:
To obtain $\Vec{\mathbf{T}}$ we use the following approaches:
% 
\begin{enumerate}  
    \item \textbf{\textit{Brute-Force}}:
    In this case we compute all possible routes by finding the permutation of all nodes in the graph (without considering the direction of links).
    Then the routes with the lowest cost and maximum success rate will be selected from all predicted routes. % could be drawn out of the queue of all predicted routes.
    \item \textbf{\textit{Open Loop Traveling Salesman Planning}}:
    This approach, termed OTSP, is a variant of the classic travelling salesman problem where the agent must visit all nodes of a graph once without making a loop back (Hamilton loop) to the start node~\cite{chieng2014performance}.
    %The OTSP is a variant of classic TSP where the agent must visit all nodes of a graph once without making a loop back (Hamilton loop) to the start node~\cite{chieng2014performance}.
    %The problem which here is being solved is also 
    To solve this we use an approach similar to $n$OTSP where the agent only needs to visit $n$ nodes in the graph, however, in our problem setting we aim to maximize the number of visited nodes while considering other important criteria like cost and success rate.
    We use our constrained uni-directional node-graph representation as a base for solving $n$OTSP using dynamic programming.
\end{enumerate}

The optimal trajectory for each intervention head is obtained by considering two criteria: number of nodes successfully visited and the total movement of the predicted trajectory.
In every trajectory, we calculate the number of nodes that satisfy~\eqref{eq:weedingProb} to determine if a node can be successfully visited.
This gives us an updated set $\Vec{\mathbf{T}'}$ which only consists of nodes in the trajectories which are feasible.
From this updated set $\Vec{\mathbf{T}'}$ we then calculate the movement cost-matrix~$\mathcal{G}$,
% 
\begin{equation}
    \label{eq:trajectoryCost} 
    \mathcal{G}(\Vec{T}'_i) = \sum_{j=0}^{q-1} (n_j - n_{j+1})^2.
\end{equation}
%
After this process, the trajectory from $\Vec{\mathbf{T}'}$ with the maximum number of successfully vised nodes is passed to the intervention controller.
In the case multiple trajectories successfully visit the same number of nodes, the trajectory which also minimizes the movement cost will be passed to the intervention controller.









\section{Experimental Evaluations}
\label{sec:exp}
\section{Experiments}
% \haizhou{Follow the same way of introduction as we did in Section2.}
% \noindent In this section, we will introduce datasets and experimental setups that we used. Then we evaluate our method, other self-supervised methods, and supervised methods under different distribution shifts (\ie, concept shifts and covariate shifts) under common settings (\ie, transductive, inductive settings). It has to note that we focus on node-level tasks (\eg, node classification) in this work. As for graph-level tasks, we leave it as our future work and some simple experiments can be found in Appendix~\ref{app:graph_classification}. 
In this section, we first introduce the experimental setup including datasets, training, and evaluation protocol in Section~\ref{sec:dataset}~and~\ref{sec:unsupervised}. 
% Next, we present our experimental setup and conduct extensive experiments to evaluate our method in Section~\ref{sec:unsupervised}. 
We then perform an ablation study to demonstrate the effectiveness of each proposed component in Section~\ref{sec:ablation}. 
Additionally, we analyze the impact of important hyper-parameters in Section~\ref{sec:sensitivity}. 
Subsequently, we integrate our method with various encoding models, showcasing the model-agnostic nature of our recipe in Section~\ref{sec:other_models}. 
Finally, we provide some qualitative results such as feature visualization in Section~\ref{sec:vis}.
It is important to note that we focus on node-level tasks (\eg, node classification) in this work. As for graph-level tasks, we leave it as our future work, while some simple experiments are also provided in Appendix~\ref{app:graph_classification}.

\subsection{Datasets}\label{sec:dataset}
There exist some benchmarks for evaluating graph out-of-distribution generalization~\cite{good,ji2022drugood,gds}. 
Among them, GOOD~\cite{good} is the most representative and comprehensive benchmark that curates more diverse graph datasets with diverse tasks, including single/multi-task graph classification, graph regression, and node classification involving more distribution shifts (\ie, concept shifts and covariate shifts). Hence in this work, we follow the evaluation protocol proposed in \cite{good}. Furthermore, we validate the effectiveness of our method in the datasets (\ie, Amazon-Photo, Elliptic) that are used in EERM~\cite{eerm}. The statistics and detailed introduction to these datasets can be found in Table~\ref{tab:dataset} and Appendix~\ref{app:datasets}.

\begin{table*}[htp]
\caption{The descriptions of datasets. ``Domain-Level'' means splitting by graphs, ``Time-Aware'' denotes splitting according to chronological order.``Word'' and ``Degree'' represent splitting according to word diversity and node degree respectively. ``Language'' means splitting by user language, suggesting the prediction should not be impacted by the language the user use. ``University'' denotes splitting according to the domain university, implying that the prediction of webpages should be based on word contents and link connections rather than university features. ``Color'' means that nodes are split according to node differences in covariate shift and color-label correlations in concept shift.}
\label{tab:dataset}
\centering
\begin{tabular}{cccccccc}
\toprule
Datasets     & Network Type        & \#Nodes & \#Edges & \#Attributes &\#Classes& Train/Val/Test Split     & Metric   \\
% Cora         & Artificial Transformation & 2,703   &         &              &         &                      & Accuracy \\
Amazon-Photo\footnotemark
             & Co-purchasing network      & 7,650   & 119,081   & 755          & 10      & Domain-Level         & Accuracy \\
Elliptic\footnotemark  
             & Bitcoin transactions       & 203,769 & 234,355   & 165          & 2       & Time-Aware           & F1-Score \\
GOOD-Cora    & Scientific publications    & 19,793  & 126,842   & 8,710         & 70      & Word/Degree          & Accuracy \\
% GOOD-Arxiv   & arXiv papers               & 169,343 & 2,315,598 & 128          & 40      & Time/Degree          & Accuracy \\
GOOD-Twitch  & Gamer network              & 34,120  & 892,346   & 128          & 2       & Language             & ROC-AUC  \\
GOOD-CBAS    & A BA-house graph           & 700     & 3,962     & 4             & 4       & Color                & Accuracy \\
GOOD-WebKB   & Webpage network            & 617     & 1,138     & 1,703         & 5       & University           & Accuracy \\
\bottomrule
\end{tabular}
\end{table*}
\footnotetext[5]{This dataset is adopted from~\cite{yang2016revisiting}. \cite{eerm} constructs ten graphs with different environment id’s for each graph.} 
\footnotetext[6]{The original is available on \hyperlink{https://www.kaggle.com/ellipticco/elliptic-data-set}{https://www.kaggle.com/ellipticco/elliptic-data-set}}

\subsection{Unsupervised Representation Learning}\label{sec:unsupervised}
\subsubsection{Transductive Setting}~\label{sec:trans}
% \noindent\textbf{Baselines.}\quad We conduct experiments with 12 baselines which consist of three categories: supervised methods and self-supervised generative methods, self-supervised contrastive methods. Specifically, we compare with three supervised baselines: empirical risk minimization~(ERM)~\cite{erm}, invariant risk minimization (IRM)~\cite{irm}, and a recent proposed graph OOD method dubbed EERM~\cite{eerm}. We also compare various unsupervised node-level representation learning methods: three self-supervised generative methods including GAE~\cite{gae}, VGAE~\cite{gae}, GraphMAE~\cite{gmae} and seven self-supervised contrastive methods: DGI~\cite{dgi}, MVGRL~\cite{mvgrl}, GRACE~\cite{grace}, RoSA~\cite{rosa}, BGRL~\cite{bgrl}, COSTA~\cite{costa}, SwAV~\cite{swav}. The descriptions of these methods can be found in Appendix~\ref{app:baselines}.
In this subsection, we focus on validating our proposed algorithm under the transductive setting, where the test nodes will participate in message passing~\cite{gilmer2017neural} during training following~\cite{good}. 

\noindent\textbf{Baselines.} We conduct experiments with 12 baselines from three categories: (i)~supervised methods, including empirical risk minimization~(\textbf{ERM})~\cite{erm}, invariant risk minimization (\textbf{IRM})~\cite{irm}, and a recent proposed graph OOD method \textbf{EERM}~\cite{eerm}; (ii)~self-supervised generative methods including Graph Autoencoder (\textbf{GAE})~\cite{gae}, Variational Graph Autoencoder (\textbf{VGAE})~\cite{gae}, Self-Supervised Masked Graph Autoencoders (\textbf{GraphMAE})~\cite{gmae}; (iii)~self-supervised contrastive methods including Deep Graph Infomax (\textbf{DGI})~\cite{dgi}, Contrastive Multi-View Representation Learning on Graphs (\textbf{MVGRL})~\cite{mvgrl}, Deep Graph Contrastive Representation Learning (\textbf{GRACE})~\cite{grace}, A Robust Self-Aligned Framework for Node-Node Graph Contrastive Learning (\textbf{RoSA})~\cite{rosa}, Bootstrapped Representation Learning on Graphs (\textbf{BGRL})~\cite{bgrl}, Covariance-Preserving Feature Augmentation for Graph Contrastive Learning (\textbf{COSTA})~\cite{costa}, Unsupervised Learning of Visual Features by Contrasting Cluster Assignments (\textbf{SwAV})~\cite{swav}. The detailed descriptions of these baselines can be found in Appendix~\ref{app:baselines}.

\noindent\textbf{Experimental setup.} We use the same graph encoder across different datasets for a fair comparison following~\cite{good}. We use grid search to find other hyper-parameters (\eg, learning rate, epochs) for different methods. For all experiments, we select the best checkpoints for ID and OOD tests according to results on ID and OOD validation sets following~\cite{good}, respectively. Experimental details and hyper-parameter selections are provided in Appendix~\ref{app:hyper}. For evaluating unsupervised methods, a linear classifier will be built on the frozen trained encoder after finishing pre-training. The reported results are the mean performance with standard deviation after 10 runs following~\cite{good}.

\noindent\textbf{Analysis.}\quad Based on the experimental results listed in Table~\ref{tab:trans_concept} and \ref{tab:trans_covariate}, we can draw the following conclusions: firstly, we find strong self-supervised methods (\eg, GRACE, BGRL, COSTA) are more robust to distribution shifts (concept shift in Table~\ref{tab:trans_concept} and covariate shift in Table~\ref{tab:trans_covariate}) compared to supervised methods. For instance, on GOOD-CBAS and GOOD-WebKB datasets, GRACE surpasses the best supervised method by large margins (over 6\% absolute improvement). Interestingly, we find the methods designed for OOD generalization (\ie, IRM) and graph OOD generalization (\ie, EERM) do not attain superior performance than the standard ERM on most of the datasets. For example, EERM shows superior OOD performance compared to ERM in only one experiment, and IRM outperforms ERM in four out of ten experiments across the conducted evaluations. This phenomenon is also observed in \cite{good,ahuja2020empirical,rosenfeld2021risks}, showcasing the challenge of achieving invariant prediction in non-Euclidean graph settings. 

Furthermore, our method surpasses other SOTA self-supervised methods on the OOD test set of all datasets by a considerable margin while achieving comparable performance in the in-distribution test set. For instance, on small datasets such as GOOD-CBAS and GOOD-WebKB, our method outperforms GRACE\footnote{MARIO is built up on GRACE according to our recipe. So, we make a comparison with GRACE here.} by over 2\% absolute accuracy on the OOD test set. On larger datasets such as GOOD-Cora and GOOD-Twitch, our method still outperforms other methods which shows its superiority. For instance, under covariate shift, MARIO surpasses other methods by over 7\% absolute accuracy on the GOOD-Twitch OOD test set. These statistics prove the effectiveness of our design.


\begin{table*}[htp]
\caption{Experimental results of all methods under concept shift. The bold font means the top-1 performance and the underline represents the second performance across the unsupervised methods. 'ID' represents in-distribution test performance and 'OOD' means out-of-distribution test performance. (OOM: out-of-memory on a GPU with 24GB memory)}
\label{tab:trans_concept}
\centering
\scalebox{0.95}{
\begin{tabular}{l|cc|cc|cc|cc|cc}
\toprule
\toprule
\multirow{3}{*}{concept shift} & \multicolumn{4}{c|}{GOOD-Cora}                   & \multicolumn{2}{c|}{GOOD-CBAS} & \multicolumn{2}{c|}{GOOD-Twitch} & \multicolumn{2}{c}{GOOD-WebKB} \\
                           & \multicolumn{2}{c}{word} & \multicolumn{2}{c|}{degree}& \multicolumn{2}{c|}{color}    & \multicolumn{2}{c|}{language}   & \multicolumn{2}{c}{university} \\
                           & ID         & OOD         & ID          & OOD          & ID            & OOD           & ID             & OOD            & ID            & OOD            \\
\midrule
ERM                        & 66.38±0.45 & 64.44±0.18  & 68.60±0.40  & 60.76±0.34   & 89.79±1.39    & 83.43±1.19    & 80.80±1.00     & 56.92±0.92     & 62.67±1.53    & 26.33±1.09     \\
IRM                        & 66.42±0.41 & 64.29±0.31  & 68.57±0.35  & 61.45±0.24   & 89.64±1.21    & 82.29±1.14    & 78.87±1.04     & 59.30±1.79     & 62.67±1.10    & 26.88±1.42     \\
EERM                       & 65.10±0.44 & 62.45±0.19  & 66.95±0.44  & 56.58±0.25   & 79.07±2.12    & 64.50±1.01    & OOM            & OOM            & 62.50±2.01    & 28.07±3.23      \\
\midrule
% Random-Init                & 37.53±1.74 & 32.12±1.24  & 37.82±1.71  & 27.74±1.14   &               &               &                &                & 60.33±2.21    & 27.07±1.70     \\
GAE                        & 60.65±0.89 & 58.00±0.55  & 62.59±1.11  & 53.44±0.80   & 75.28±1.36    & 68.07±2.05    & 81.25±0.81     & 51.51±1.05     & 62.17±3.34    & 25.78±1.85     \\
VGAE                       & 63.19±0.53 & 60.35±0.47  & 61.65±0.66  & 54.28±0.28   & 76.50±0.50    & 59.07±0.56    & 80.46±0.53     & 55.56±4.53     & 62.50±2.38    & 24.40±2.57     \\
GraphMAE                   & \underline{66.44±0.46} & \underline{64.87±0.30}  & 67.95±0.46  & 59.41±0.39   & 89.14±0.89    & 82.93±0.93    & 80.05±0.64     & 59.38±1.49     & 61.83±3.37    & 29.27±2.15     \\
DGI                        & 63.33±0.56 & 60.71±0.49  & 65.93±1.02  & 55.83±0.53   & 91.22±1.47    & 85.00±1.66    & 80.05±0.87     & 59.16±1.88     & 61.83±2.83    & 28.63±1.92      \\
MVGRL                      & OOM        & OOM         & OOM         & OOM          & 88.57±1.15    & 76.50±1.17    & OOM            & OOM            & 62.00±3.79    & 28.26±4.20     \\
GRACE                      & 65.61±0.61 & 63.92±0.44  & \textbf{68.59±0.35}  & 60.15±0.45   & 92.00±1.39    & 88.64±0.67    & \textbf{83.43±0.63}     & \underline{60.45±1.46}     & 64.00±3.43    & \underline{34.86±3.43}  \\
RoSA                       & 64.06±0.67 & 62.44±0.39  & 67.07±0.65  & 57.68±0.44   & 90.78±2.27    & 85.93±2.14    & 82.39±0.42     & 57.45±2.16     & 64.17±4.10    & 32.20±2.15     \\
BGRL                       & 65.18±0.43 & 63.43±0.45  & 66.83±0.80  & 59.63±0.38   & 92.36±1.16    & 87.14±1.60    & 82.52±0.60     & 55.48±1.48     & 63.67±2.33    & 31.47±3.43     \\
COSTA                      & 65.05±0.80 & 62.37±0.45  & 66.76±0.87  & 55.73±0.36   & \underline{93.50±2.62}    & \underline{89.29±3.11}    & 83.15±0.30 & 55.03±3.22     & 61.66±2.58    & 32.39±2.13 \\
% ArCL                       &            &             & 67.64±0.57  & 59.71±0.44   &               &               &                &                & 65.00±3.94    & 35.41±1.97 \\      
SwAV                       & 62.22±0.53 & 59.79±0.53  & 64.65±0.94  & 55.06±0.39   & 89.00±0.79    & 81.72±0.66    & \underline{83.32±0.15}     & 59.69±1.97     & \underline{65.17±3.76}    & 29.36±2.01    \\
\midrule
MARIO                       & \textbf{67.11±0.46} & \textbf{65.28±0.34}  & \underline{68.46±0.40}  & \textbf{61.30±0.28}   & \textbf{94.36±1.21}    & \textbf{91.28±1.10}    & 82.31±0.54     & \textbf{63.33±1.72}     & \textbf{65.67±2.81}    & \textbf{37.15±2.37}     \\
\bottomrule
\end{tabular}}
\end{table*}

\begin{table*}[htp]
\caption{Experimental results of all methods under covariate shift. The bold font means the top-1 performance and the underline represents the second performance across the unsupervised methods. 'ID' represents in-distribution test performance and 'OOD' means out-of-distribution test performance. (OOM: out-of-memory on a GPU with 24GB memory)}
\label{tab:trans_covariate}
\centering
\scalebox{0.95}{
\begin{tabular}{l|cc|cc|cc|cc|cc}
\toprule
\toprule
\multirow{3}{*}{covariate shift} & \multicolumn{4}{c|}{GOOD-Cora}                                   & \multicolumn{2}{c|}{GOOD-CBAS} & \multicolumn{2}{c|}{GOOD-Twitch} & \multicolumn{2}{c}{GOOD-WebKB} \\
                           & \multicolumn{2}{c}{word} & \multicolumn{2}{c|}{degree}& \multicolumn{2}{c|}{color}    & \multicolumn{2}{c|}{language}   & \multicolumn{2}{c}{university} \\
                           & ID         & OOD         & ID          & OOD          & ID            & OOD           & ID             & OOD            & ID            & OOD            \\
\midrule
ERM                        & 70.50±0.41 & 64.69±0.33  & 72.46±0.49  & 55.53±0.50   & 92.00±3.08    & 77.57±1.29    & 70.98±0.41     & 49.35±5.09     & 39.34±1.79    & 14.52±3.14   \\
IRM                        & 70.48±0.26 & 64.53±0.57  & 71.98±0.34  & 53.72±0.46   & 90.86±2.41    & 78.86±1.67    & 69.81±0.95     & 49.11±2.82     & 38.52±3.30    & 13.97±2.80     \\
EERM                       & OOM        & OOM         & OOM         & OOM          & 65.00±2.57    & 57.43±3.60    & OOM            & OOM            & 46.07±4.55    & 27.40±7.65     \\
\midrule
GAE                        & 56.63±0.79 & 48.93±0.93  & 66.30±0.88  & 34.01±0.87   & 73.00±2.16    & 60.86±3.01    & 67.24±1.23     & 47.65±2.49     & 45.08±6.32    & 28.02±6.29    \\
VGAE                       & 62.02±0.66 & 54.12±0.86  & 69.41±0.57  & 44.20±1.29   & 62.29±2.04    & 63.29±1.11    & 66.99±1.43     & \underline{50.48±4.58}     & 48.85±4.68    & 20.87±6.69     \\
GraphMAE                   & 68.14±0.43 & 64.00±0.33  & \textbf{73.36±0.56}  & 53.75±0.55   & 67.28±3.03    & 67.28±1.49    & 68.84±1.20     & 48.02±2.79     & 48.03±4.34    & 30.00±8.09     \\
DGI                        & 60.85±0.75 & 57.03±0.67  & 68.97±0.41  & 41.75±0.88   & 69.57±4.09    & 59.71±3.43    & 68.43±1.05     & 44.83±1.61     & 48.52±5.04    & 21.11±7.50     \\
MVGRL                      & OOM        & OOM         & OOM         & OOM          & 65.00±1.94    & 64.15±0.77    & OOM            & OOM           & \textbf{54.10±5.39}    & 16.59±6.51     \\
GRACE                      & \underline{68.77±0.33} & \underline{64.21±0.41}  & 72.69±0.34  & \underline{56.10±0.63}   & \underline{93.57±1.83}    & \underline{89.29±3.40}    & \underline{71.12±0.87} & 46.21±1.54 & 49.67±5.82    & 28.10±4.68    \\
RoSA                       & 68.19±0.56 & 62.48±0.61  & 71.04±0.62  & 52.72±0.79   & 84.71±4.14    &79.14±3.51     & 70.58±0.36     & 45.83±1.72     & 52.30±4.24    & \underline{34.24±7.92}     \\
BGRL                       & 67.23±0.43 & 61.33±0.36  & 72.11±0.39  & 49.15±0.73   & 89.00±2.56    & 79.86±3.29    & \textbf{71.43±0.53}     & 43.86±0.94     & 51.80±5.55    & 30.32±7.61    \\
COSTA                      & 65.28±0.60 & 60.33±0.53  & 70.65±0.62  & 54.03±0.28   & 92.29±1.59    & 82.71±2.74    & 69.29±1.37     & 49.07±2.13     & 50.49±3.01    & 29.84±4.75   \\
SwAV                       & 63.29±1.01 & 56.98±0.94  & 70.27±0.73  & 43.00±0.52   & 89.57±1.12    & 81.43±1.69    & 69.19±0.93     & 49.37±2.96     & 49.84±4.82    & 30.55±6.72   \\
\midrule
MARIO                       & \textbf{69.99±0.54} & \textbf{65.06±0.34}  & \underline{72.73±0.43}  & \textbf{57.73±0.45}  & \textbf{94.57±2.46}    & \textbf{91.00±2.48}     & 68.31±0.78 & \textbf{57.37±1.37}     & \underline{53.94±3.23}    & \textbf{35.24±4.98}   \\
\bottomrule
\end{tabular}}

\end{table*}

\subsubsection{Inductive Setting}
In this subsection, we conduct experiments under the inductive settings, where the test nodes are kept unseen during training. This setting is more suitable for domain generalization.
% But we think it is more convincing that conduct experiments under inductive settings which means test nodes are unseen during training. This setting is more appropriate for domain generalization.

\noindent\textbf{Baselines:} For GOOD-WebKB and GOOD-CBAS datasets, we adopt ERM, IRM, GraphMAE, and GRACE as our baselines. And for Amazon-Photo and Elliptic datasets, we select ERM, EERM, and GRACE as our baselines.

\noindent\textbf{Experimental setup:} For GOOD-WebKB and GOOD-CBAS datasets, we use the same model configuration in Section~\ref{sec:trans}.
% Besides, we add experiments on Amazon-Photo dataset~\cite{yang2016revisiting} and Elliptic~\cite{elliptic} dataset in this subsection. 
For Amazon-Photo dataset~\cite{yang2016revisiting} and Elliptic~\cite{elliptic} dataset, they consist of many snapshots (training data and testing data use different snapshots) which are naturally inductive. For Amazon-Photo dataset, we use 2-layer GCN~\cite{gcn} as the encoder and for elliptic dataset, we use 5-layer GraphSAGE~\cite{sage} as encoder following~\cite{eerm}.

% Figure environment removed

\noindent\textbf{Analysis:}
According to Figure~\ref{fig:amazon},\ref{fig:elliptic},\ref{fig:ind_con},\ref{fig:ind_cov}, we can draw following conclusions:
firstly, based on Figure~\ref{fig:amazon}, it is evident that our method outperforms other representative supervised and self-supervised methods on all test graphs (T1$\sim$T8). This superiority is reflected in the larger median value of our method compared to others. For instance, MARIO achieves over a 3\% absolute improvement compared to ERM in terms of the mean value of eight median values. Additionally, our method demonstrates higher stability across different random initializations, as indicated by the closer proximity of the first and third quartile values to the median value~(\eg, the difference of first and third quartile values of ERM, EERM, GRACE and MARIO are 4.2, 3.3, 6.7 and 1.0 on T8 respectively which indicates MARIO is much more stable than other methods). Furthermore, our method exhibits consistent performance across different graphs (\eg, The standard deviation of median values on T1$\sim$T8 for ERM, EERM, GRACE, and MARIO are 0.4, 1.1, 1.2, and 0.3, respectively.), indicating its robustness to environmental variations and its ability to extract invariant features: $g(G^e) \approx g(G^{e'})$ for all $e, e' \in \mathcal{E}^\text{train}$. In summary, our method showcases enhanced OOD generalization capabilities.
% $g(G^e)g(G^e^\prime)$ where $any e, e^\prime in \mathcal{E}^{train}$

Secondly, from the results presented in Figure~\ref{fig:elliptic}, we can observe that our method averagely harvests 10.9\% absolute improvement over GRACE and 12.5\% absolute improvement over EERM in terms of F1 scores on Elliptic dataset. This demonstrates the effectiveness of our method in handling distribution shifts and improving performance compared to existing approaches. It is worth noting that GRACE's performance worsens over time, indicating its inability to handle distribution shifts effectively. In contrast, our method consistently achieves better F1 scores, except for T9, which is caused by the dark market shutdown occurred after T7~\cite{elliptic}. The emergence of such an event introduces significant variations in data distributions, which subsequently results in performance degradation for all methods. Indeed, this event serves as an unpredictable external factor that introduces significant challenges for models trained on limited training data. The results indicate that the performance heavily depends on available training data. Nonetheless, our approach outperforms other methods even in such an extreme case. This highlights the effectiveness of our method in addressing distribution shifts and improving generalization performance.

Finally, based on the observations from Figure~\ref{fig:ind_con} and Figure~\ref{fig:ind_cov} MARIO demonstrates the best performances on both ID and OOD test sets for GOOD-WebKB and GOOD-CBAS datasets, under both concept shift and covariate shift. Notably, MARIO outperforms other methods by more than 3\% and 10\% absolute improvement on GOOD-WebKB and GOOD-CBAS, respectively, under covariate shift. We can draw similar conclusions as discussed in Section~\ref{sec:trans}. Even under the inductive setting, our method continues to demonstrate excellent OOD generalization capabilities and achieves comparable or even improved in-distribution test performance. These statistical results further validate the effectiveness of our method in handling distribution shifts and enhancing generalization performance.

Overall, the observations we have made provide strong evidence of the great capacity of our method for handling distribution shifts, validating its effectiveness and potential for real-world applications.



% Figure environment removed

% Figure environment removed


% Figure environment removed


\subsection{Ablation Studies}\label{sec:ablation}
\noindent Table~\ref{tab:aba} provides a detailed analysis of the effect of each component according to our proposed recipe for improving OOD generalization in graph contrastive learning. Let's examine the different variants of our method and their impact on performance.
Specifically, MARIO~(w/o ad) represents MARIO without  adversarial augmentation. MARIO~(w/o cmi) denotes we only maximize the mutual information between positive pairs without considering conditional mutual information. MARIO~(w/o cmi, ad) means a vanilla graph contrastive method that is similar to GRACE. 

From Table~\ref{tab:aba}, we can find MARIO~(w/o cmi) lags far behind MARIO on OOD test set which demonstrates appropriately minimizing the redundant information (\ie, conditional mutual information) is essential to improve OOD generalization of GCL methods. And adversarial augmentation can also boost OOD generalization because it can approximately serve as a supermum operator to learn more invariant features  discussed in Section~\ref{sec:aug}. Based on the analysis of these variants, it is evident that the proposed improvements on data augmentation and contrastive loss in the recipe are both effective in enhancing graph OOD generalization. Each component contributes to the overall performance improvement, and their combination leads to a stronger self-supervised graph learner in terms of graph OOD generalization. 

In short, the findings from Table~\ref{tab:aba} support the rationale behind your proposed recipe and provide empirical evidence of the effectiveness of each proposed component. By incorporating these enhancements, our method achieves superior performance in handling distribution shifts and improving graph OOD generalization in graph contrastive learning.
\begin{table*}[htp]
\caption{Ablation studies for MARIO by masking each component.}
\label{tab:aba}
\centering
\scalebox{0.9}{
\begin{tabular}{l|cc|cc|cc|cc|cc}
\toprule
\toprule
\multirow{3}{*}{concept shift} & \multicolumn{4}{c|}{GOOD-Cora}                       & \multicolumn{2}{c|}{GOOD-CBAS} & \multicolumn{2}{c|}{GOOD-Twitch} & \multicolumn{2}{c}{GOOD-WebKB} \\
                           & \multicolumn{2}{c}{word} & \multicolumn{2}{c|}{degree}& \multicolumn{2}{c|}{color}    & \multicolumn{2}{c|}{language}   & \multicolumn{2}{c}{university} \\
                           & ID         & OOD         & ID          & OOD          & ID            & OOD           & ID             & OOD            & ID            & OOD            \\
\midrule
MARIO                      & \textbf{67.11±0.46} & \textbf{65.28±0.34}  & \textbf{68.46±0.40}  & \textbf{61.30±0.28}      & \textbf{94.36±1.21}  & \textbf{91.28±1.10}    & 82.31±0.54     & \textbf{63.33±1.72}     & \textbf{65.67±2.81}    & \textbf{37.15±2.37}     \\
MARIO(w/o ad)              & 66.23±0.53 & 64.02±0.18  & 67.88±0.38  & 60.46±0.29   & 93.21±1.25    & 90.29±0.91    & 82.42±0.73     & 60.50±1.02     & 64.83±2.83    & 36.51±3.25    \\
MARIO(w/o cmi)             & 65.32±0.60 & 63.51±0.32  & 68.14±0.32  & 61.19±0.34   & 94.15±1.23    & 90.57±1.96    & \textbf{82.51±0.56}     & 61.41±2.63     & 64.50±4.35    & 35.78±2.53     \\
MARIO(w/o cmi, ad)         & 64.67±0.55 & 63.11±0.32  & 67.95±0.65  & 60.01±0.57   & 93.36±1.66    & 89.64±1.73    & 81.90±0.75     & 60.12±1.60     & 64.17±3.67    & 34.13±2.38     \\
\bottomrule
\end{tabular}}
\end{table*}
% & 65.32±0.60 & 63.51±0.32 exchange 64.67±0.55 & 63.11±0.32
% 68.14±0.32       id ood test: 60.95±0.43       ood ood test: 61.19±0.34


\subsection{Sensitivity Analysis}\label{sec:sensitivity}
\noindent In this subsection, we will analyze some important hyper-parameters of our method. We conduct sensitivity analysis on GOOD-WebKB dataset with concept shift, we chose two sensitive hyper-parameters (\ie, the coefficient $\gamma$ of condition mutual information in Equation~\ref{equ:cmi} and the number of prototypes $|C|$ in Equation~\ref{equ:pq}). The coefficient of CMI range in $[0.001, 0.01, 0.1, 0.5, 1]$ and the number of prototypes $|C|$ ranges in $[10, 50, 100, 200, 300]$. From Figure~\ref{fig:sensitivity}, we can observe that $\gamma$ reaches 0.1 and $|C|$ reaches 100 or 200 can achieve the best OOD test accuracy. Both higher and lower values of $\gamma$ result in suboptimal performance. This finding aligns with previous research such as DIB~\cite{dib}, indicating that an appropriate compression level is crucial for achieving optimal performance. Extremely high or low compression values are not ideal. 

Regarding the number of prototypes $|C|$, based on the results shown in Figure~\ref{fig:sensitivity}, it is found that setting $|C|=100$ leads to the best performance in terms of OOD test accuracy. This choice provides a moderate number of pseudo labels, which is beneficial for the learning process. 

Based on the sensitivity analysis, we determined that setting $\gamma=0.1$ and $|C|=100$ on most datasets. These hyperparameter values strike a balance between compression level and the number of prototypes, resulting in improved graph OOD generalization.
% Figure environment removed


\subsection{Integrated with Other Models}\label{sec:other_models}
% Figure environment removed

\begin{table}[htp]
\caption{Results of different learning approaches with different encoding models (\ie, GCN, GraphSAGE, GAT).}
\label{tab:others}
\centering
\scalebox{0.9}{
\begin{tabular}{cc|cc|cc}
\toprule
\toprule
\multirow{3}{*}{Model}& \multirow{3}{*}{Method} & \multicolumn{2}{c|}{GOOD-CBAS} & \multicolumn{2}{c}{GOOD-WebKB} \\
                & & \multicolumn{2}{c|}{color}    & \multicolumn{2}{c}{university} \\
                &   & ID          & OOD         & ID          & OOD            \\
\midrule
\multirow{3}{*}{GCN} 
&ERM               & 89.79±1.39 & 83.43±1.19  &  62.67±1.53 & 26.33±1.09         \\
&GRACE             & 92.00±1.39 & 88.64±0.67  &  64.00±3.43 & 34.86±3.43        \\
&MARIO             & 94.36±1.21 & 91.28±1.10  &  65.67±2.81 & 37.15±2.37        \\ \bottomrule
\multirow{3}{*}{SAGE} 
&ERM               & 95.07±1.51 & 75.14±1.19  & 73.67±2.08  & 46.33±3.42       \\
&GRACE             & 95.29±1.11 & 74.43±2.36  & 70.50±5.06  & 49.54±3.83        \\
&MARIO             & 96.00±1.07 & 76.29±3.01  & 71.00±3.82  & 51.74±4.63        \\ \bottomrule
\multirow{3}{*}{GAT} 
&ERM               & 78.64±3.63 & 72.93±2.64  & 61.33±3.71  & 28.99±2.63        \\
&GRACE             & 84.57±1.79 & 78.36±1.60  & 59.50±2.36  & 35.78±3.26        \\
&MARIO             & 84.93±1.95 & 80.43±1.89  & 62.17±4.78  & 38.17±3.10        \\
\bottomrule
\end{tabular}}
\end{table}



\noindent In the subsection, we demonstrate the model-agnostic nature of the recipe by integrating it with various graph neural network (GNN) models, including GCN, GraphSAGE, and GAT.

From Table~\ref{tab:others}, it can be observed that regardless of the specific GNN model used as the encoder, our method consistently achieves the best performance on the OOD test set. This indicates the effectiveness and robustness of our method across different GNN models.
By achieving superior performance across different GNN models, MARIO demonstrates its versatility and ability to improve the OOD generalization of various graph neural models. This highlights the broad applicability and effectiveness of our recipe in enhancing the performance of different GNN encoders.

Furthermore, we integrate our recipe with other GCL methods in Appendix~\ref{app:other_methods}. The results demonstrate our recipe can boost the OOD generalization ability of various GCL methods which means our recipe can serve as a plug-in for many current classical GCL methods.

% Figure environment removed

\subsection{Visualization}\label{sec:vis}
\subsubsection{Metric Score Curves}
We present metric score curves for ERM and MARIO, including training, ID validation, ID testing, OOD validation, and OOD testing accuracy, in Figure~\ref{fig:curve2}. Notably, MARIO demonstrates superior convergence with approximately 10\% absolute improvement on the OOD test set compared to ERM. Furthermore, MARIO effectively narrows the performance gap between in-distribution and out-of-distribution performance, showcasing its efficacy in enhancing OOD generalization for graph data. More metric score curves can be found in Appendix~\ref{app:curves}.


\subsubsection{Feature Visualization}
In order to assess the quality of learned embeddings, we adopt t-SNE~\cite{tsne} to visualize the node embedding on GOOD-Cora dataset (concept shift in word domain) using random-init of GCN, EERM, GRACE, and MARIO, where different classes have different colors in Figure~\ref{fig:vis}. For clarity, we select eight classes with the largest number of nodes to enhance the informativeness and interpretability of the visualization. We can observe that the 2D projection of node embeddings learned by MARIO has a better separation of clusters, which indicates the model can help learn representative features for downstream tasks. It has to note that we depict both ID nodes and OOD nodes in the same figure. 

Besides, we also separately visualize ID nodes and OOD nodes in the different figures in the Appendix~\ref{app:feature}. And we can find MARIO performs a clearer separation of clusters whether on ID nodes or OOD nodes compared to other methods.




%%%%%%%%%%%%%%%%%%%%%%%%%%%%%%%%%%%%%%%%%%%%%%%%%%%%%%%%%%%%%%%%%%%%%%%%%%%%%%%
\section{Conclusion}
\label{sec:conc}

In this paper, we introduced \bbot\ a crop monitoring and weeding platform specifically designed for European phenotyping fields.
We showed how \bbot\ can greatly improving crop monitoring, reducing the NAE from $8.3\%$ to $3.5\%$, by combining the extra localization sensors it has available.
Furthermore, we present a novel arrangement of weeding tools mounted on linear actuators along with three associated target planning approaches.
These target planning approaches were evaluated in simulated environments which are able to mirror real-world weed distributions by using the results from our field monitoring approach.
Experiments on these simulated real-world fields demonstrated the validity of our proposed work-space division techniques and showed that they led to significantly less movement ($10m$ compared to $5m$) when compared to distance-based target assignment.
Finally, we found that while the real-world data was generally uniform in future any planning systems should be evaluated on real-world weed distributions.
%allow for areas in the field that contain more complicated weed densities.
%Finally, we found that while the real-world data was generally uniform, in future any planning systems should allow for areas in the field that contain more complicated weed densities.

\begin{comment}
In this paper, we introduced \bbot\ a crop monitoring and weeding platform specifically designed for European phenotyping fields.
\bbot\ can operate on a wide range of crop fields and in different growth-stages and conduct precision plant-level weed management using its novel multi-head weeding tool.
We showed how sensor fusion can greatly improving crop monitoring from an NAE of $8.3\%$ to $3.5\%$.
We also present a novel arrangement of weeding tools mounted on linear actuators along with associated planning approaches.
Three approaches for target management were proposed and then evaluated in simulated environments which replicated real-world weed distributions by using the results from our field monitoring approach.
Experiments on these simulated real-world fields demonstrated the validity of our proposed work-space division techniques and showed that they led to significantly less movement ($10m$ compared to $5m$) when compared to distance-based target assignment.
%Overall, \bbot\ is a significant step forward in precise weed management with a novel method of selectively spraying and controlling weeds in an arable field.
%Also, we introduced a weeding simulation framework capable of replicating real field plant distributions which is used for developing and evaluating our weeding strategies.
%We discussed three techniques for target management which outlines path planning on real data showing their validity in balancing the load of intervention between axes of weeding tool.
%An important conclusion from our results on, real-world weed distributions, is that any future work should not assume that weed distributions are uniform.
Finally, we found that while the real-world data was generally uniform, in future any planning systems should allow for areas in the field that contain more aggressive weed densities.
\end{comment}

% We have explored approaches to 
% This allows us to successfully ...
% We implemented and evaluated our approach on different datasets
% and provided comparisons to other existing techniques and supported
% all claims made in this paper. The experiments suggest that ...

%%%%%%%%%%%%%%%%%%%%%%%%%%%%%%%%%%%%%%%%%%%%%%%%%%%%%%%%%%%%%%%%%%%%%%%%%%%%%%%
\section*{Acknowledgements}

This work was funded by the Deutsche Forschungsgemeinschaft (DFG, German Research Foundation) under Germany’s Excellence Strategy - EXC 2070 – 390732324.

%%%%%%%%%%%%%%%%%%%%%%%%%%%%%%%%%%%%%%%%%%%%%%%%%%%%%%%%%%%%%%%%%%%%%%%%%%%%%%%
\bibliography{references}
\bibliographystyle{IEEEtran}
%%%%%%%%%%%%%%%%%%%%%%%%%%%%%%%%%%%%%%%%%%%%%%%%%%%%%%%%%%%%%%%%%%%%%%%%%%%%%%%

\end{document}