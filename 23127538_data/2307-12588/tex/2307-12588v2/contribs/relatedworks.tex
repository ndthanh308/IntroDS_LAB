% Introduce weed management
% - could mention bio diversity

Robotic platform based weed management techniques have rapidly evolved in the last decade with the aim to treat each weed as precisely as possible. %in an effort to achieve performance similar to a human in the field.
To achieve this, plant-level intervention needs to operate in different fields with varying crops, weed species and weed distributions.
To enable this, ``smart farming techniques'' aim to incorporate automated navigation~\cite{ahmadi2021towards}, crop monitoring~\cite{halstead2021crop}, and weeding~\cite{pretto2020building}. %selective image-based perception and intervention techniques~\cite{mccool2018efficacy}.
%
% One of the key elements to achieve precise weeding is species level classification and localization, leading to strategic decision making.
%
%One of the key elements to achieve human level weeding is species level classification and localization, leading to strategic decision making.
% Species-level weeding has the benefit of enabling strategic monitoring and intervention. 
%
One of the key elements to achieve precise weeding is plant-level treatment, where the treatment of each plant is dictated by its species, size and its impact upon not only the crop but also the environment~\cite{blaix2018quantification}.
%
% This can enable strategic weeding paradigms such as plant-level treatment, where the treatment of each plant is dictated by its species, size and its impact upon not only the crop but also the environment, that have the potential to increase bio-diversity as well as improve crop and soil health~\cite{blaix2018quantification}.
% Additional benefits include the potential to reduce pesticide use through controlled and precise weeding which can in turn have a positive impact on pollinators~\cite{raven2021agricultural}.
%, and result in better crop quality and yield. % can also increase the presence of pollinators~\cite{raven2021agricultural} resulting in better crop quality and yield. 
%
% Crop monitoring is an important component of any intervention technique weeding harvesting
%
% To achieve this goal advanced crop monitoring techniques that provide species-level classification and fine-grained localization are required.
Clearly, these approaches rely on the underlying perception or agricultural monitoring approaches which have gained significant research attention in recent years, including in glasshouses~\cite{smitt2021pathobot, smitt2022explicitly}, orchards~\cite{tian2019apple}, and fields (for weed intervention)~\cite{peruzzi2017machines}.
%However, for weeding applications many approaches still rely on coarse spraying or hoeing~\cite{Bawden17_1}; in this approach we propose a precise spraying platform.
% While a number of tools exist for weeding~\cite{peruzzi2017machines} in this paper we propose a precise spraying platform.

To achieve precise species level intervention recent methods have relied on deep learning due to its accuracy and generalizability.
Tian et~al.~\cite{tian2019apple} compared both Faster-RCNN~\cite{ren2015faster} and Yolo~\cite{redmon2018yolov3} and found Yolo to be superior in both performance and speed, while,~\cite{wan2020faster} was able increase both the performance and speed of Faster-RCNN.
Expanding on~\cite{halstead2020},~\cite{halstead2021crop} showed the viability of crop agnostic monitoring by exploiting an augmented Mask-RCNN framework. % for species level classification and pixel-wise segmentation.
To improve the accuracy of monitoring techniques in arable farmland~\cite{ahmadi2021virtual}, was able to utilize the temporal information captured by the platform.
To selectively weed~\cite{pretto2020building} built, based on the work in~\cite{lottes2018fully}, a fully convolutional network that was able to accurately classify weeds in the field.
% Chang et~al.~\cite{Chang21_mechanical_tools} exploited the Yolo network for a weeding application that was able to intervene based on the location of the bounding box.
% These approaches have increased system performance and enabled platforms that are able to accurately detect and treat weeds in the field.


\begin{comment}
\subsection{Monitoring Techniques}
% In general crop monitoring
% Traditional ML

Early robotic platforms designed for weeding combined traditional computer vision and traditional machine learning techniques to achieve accurate weeding intervention.
%Early research focused on combining computer vision (CV) with traditional machine learning (ML) to achieve accurate weeding intervention.
% Bawden et~al.~\cite{Bawden17_1} exploited various colour spaces along with local binary patterns to classify weeds in the field.
Using only a pattern recognition based approach~\cite{bakker2006autonomous} was able to create an early robotic platform for organic farming.
For weeding in carrot fields~\cite{utstumo2018robotic} created a three wheeled platform that used the HSV color space along with support vector machines to classify and spray weeds.
This was considered to be a drop-on-demand weeding platform that had the potential to increase herbicide strength through accurate detection of weeds.
%Leveraging CV and ML~\cite{utstumo2018robotic} was able to create a drop-on-demand weeding platform, potentially increasing herbicide strength through more accurate detection.
% This approach was able to increase herbicide strength as it was more precise while also protecting the soil health.
% However, a downside of traditional computer vision techniques is that, in general, these approaches are crop and environment specific with little chance to generalize.
A downside of traditional approaches is that they are usually designed for a single purpose and lack generalizability.
% \TODO{CM to ALL: the above sentence is a strong statement do we have a reference for it?}
%In general, these approaches are crop and environment specific with little chance to generalize. 
%designed and used for a single crop or in a single environment.
% \todo{probably need a couple more non-dl references.}

% Deep Learning

Recently, deep learning-based approaches have been shown to be both more accurate and more generalizable, and only bound by the available data.
% The generalizability of these techniques allows them to be deployed for multiple different crop types.
% These approaches are generally only bound by the available data.
% For crop monitoring based approaches Faster-RCNN~\cite{ren2015faster}, Mask-RCNN~\cite{he2017mask}, and Yolo~\cite{redmon2018yolov3} are common approaches to classifying and localising objects.
Tian et~al.~\cite{tian2019apple} compared both Faster-RCNN~\cite{ren2015faster} and Yolo~\cite{redmon2018yolov3} and found Yolo to be superior in both performance and speed, while,~\cite{wan2020faster} was able increase both the performance and speed of Faster-RCNN.
% By augmenting Faster-RCNN~\cite{halstead2018fruit} and Mask-RCNN~\cite{halstead2020} these techniques were able to achieve species level classification while maintaining model generalizability in unbalanced data.
Expanding on~\cite{halstead2020},~\cite{halstead2021crop} showed the viability of crop agnostic monitoring for applications in arable farmland exploiting an augmented Mask-RCNN framework for species level classification and pixel-wise segmentation.
% In their approach they showed that species-level classification, pixel-wise segmentation, and area estimation could be achieved, enabling more informed decision making by the farmer.
To improve the accuracy of monitoring techniques in arable farmland~\cite{ahmadi2021virtual}, was able to utilize the temporal information captured by the platform.
% In their work, they improved semantic segmentation accuracy by temporally augmenting a UNet~\cite{ronneberger2015u} structure.
To selectively weed~\cite{pretto2020building} built, based on the work in~\cite{lottes2018fully}, an architecture based on fully convolutional networks (FCN) that was able to accurately classify weeds in the field.
Chang et~al.~\cite{Chang21_mechanical_tools} exploited the Yolo network for a weeding application that was able to intervene based on the location of the bounding box.
% This approach was able to achieve considerable results in the early to mid parts of the day.
%To selectively weed~\cite{pretto2020building} built an architecture based on fully convolutional networks (FCN)~\cite{lottes2018fully} that was able to accurately classify weeds in the field.
These approaches have increased system performance and enabled platforms that are able to accurately detect and treat weeds in the field.

\end{comment}

% \subsection{Robotic Platforms}

% Similarly to techniques, in recent years the number of available platforms for weed management with specific weeding tools has increased.
In order to perform intervention platforms are required, in recent years the number of available robotic platforms for weed management has increased.
% Each of these platforms are designed with specific weeding tools, and operate in unique ways.
% Slaughter~\etal~\cite{Slaughter08_weeding_review} provides an early detailed review of different tools, While not specifically research into a platform.
An early, low cost, platform was developed by~\cite{Ruiz14_intracrop_weeding} which contained a mechanical weeding tool for intra-row intervention that required a human in the loop.
Bawden~\etal~\cite{Bawden17_1} proposed an automated platform that utilised a row of weeding hoes and spray nozzles to improve on broadcast applications and perform multi-modal weeding (physical and chemical).
Also combining mechanical tools (two ranks of stampers) and sprayers~\cite{pretto2020building} is able to selectively weed based on the overall size of the plant.
To design a spraying platform~\cite{zhou2021design} developed a technique that was able to specifically target regions or weeds. 
In an indoor experiment they were able to reduce the amount of chemicals used, compared to a uniform sprayer, by $46.8\%$.
Chang~\etal~\cite{Chang21_mechanical_tools} built a platform to evaluate two different mechanical weeding tools which was tested on a purpose built field with a single crop-row of 20m.
Their approach used Yolo to both locate weeds of interest in the field but also control the movement of the platform.
% \TODO{CM to ALL: for the above sentence, which fields currently it's too vague.}
% \todo{it wasn't actually tested in a field , all experiments was indoor on some green stuff like leaves}
% While, this approach was able to extract the full weed (including roots) the design of their platform meant it could only operate in certain fields.

A consistent trend with the above approaches is that they aim to improve on broadcast or uniform weeding by introducing equally spaced tools.
In doing so they ensure coverage but have the significant downside of having to replicate the tool across the entire width of the robot.
Yet, the optimal tool is dependent upon the field and plants as shown by the variety of tools: physical~\cite{chang2021mechanical}, chemical~\cite{wu2020robotic}, electrocuting~\cite{ascard200710} or laser-based~\cite{xiong2017development} implements.
As such, we conclude that to provide truly precise intervention a weeding platform needs the ability to mount multiple tools.
This issue exacerbates the problem of having replicating equally spaced tools as two or more sets of tools have to be replicated.
An alternative is to have movable tools on a robot, but this is only possible with an appropriate planning algorithm.
%and this can be most easily achieved by having movable tools on a robot.
%A key element to this is planning to move the weeding tools to perform weed management.

%This is further challenging when multi-modal weeding is considered as two or more sets of tools have to be replicated.
%Our approach expands on this limitation by designing sprayers that operate on linear axis allowing for full range precise spraying, while conforming to standard European planting distances.
%In many of these approaches they aim to improve on broadcast or uniform weeding by introducing equally spaced tools. 
%Our approach expands on this limitation by designing sprayers that operate on linear axis allowing for full range precise spraying, while conforming to standard European planting distances.

% In our paper we improve on the limitations of thses
% Still broadcast or selective broadcast lots sof draging and not precise.

% To achieve weeding intervention goals, robotic platforms are required.
% \cite{Ruiz14_intracrop_weeding} developed a low-cost an intra-row hoe mechanism tool on a mobile platform with a human partner in the loop to provide visual crop detection and controlling hoeing axis.

% \cite{Bawden17_1}proposed precise weed management using mechanical implements which are applied when a weed is detected. 
% Despite this approach being more precise than broadcast application (where the tools are applied to every portion of the field) these tools still had a width of 0.11m and were engaged in the soil for 0.2m.
% We aim to reduce the size of the tool and length of engagement time to enable precise weed management.

% In~\cite{Chang21_mechanical_tools} Chang~\etal\ presented two mechanical weeding tools deployed on a self-built mobile robot platform.
% Their mechanical weeding (hoeing) mechanics was able to bring out the roots of weeds.
% While due to limited work-space and length of operation was only deploy-able on specific crops and growth stages.

% Pretto~\etal~\cite{pretto2020building} propose an approach The weeds are treated mechanically with two ranks of stampers or chemically with one rank of sprayers. 
% They make the decision on which tool is used on which weed in our experiments is only based on size criteria: large weeds are sprayed while small weeds are stamped.

% Below reference doesn't look great for a publication
%A state-of-the-art intelligent commercial~\cite{Garford} introduced an intra-row cultivator based on a machine vision sensor and a rotating disc hoe for intra-row weed control in four vegetable crops in California.

% In \cite{zhou2021design} authors focused on the design and preliminary evaluation of a target spray platform. All components were effectively connected and evaluated especially regarding the response time and target spray accuracy.
% Using an indoor experimental setup shows that the developed system can reduce $46.8\%$ usage of chemicals compared to the uniform spray method.

% In~\cite{Chang21_mechanical_tools} Chang~\etal\ presented two mechanical weeding tools deployed on a self-built mobile robot platform.
% Their mechanical weeding (hoeing) mechanics was able to bring out the roots of weeds.
% While due to limited work-space and length of operation was only deploy-able on specific crops and growth stages.




%\subsection{Weeding planning}
%\TODO{MAH to AA you need to write this section with a couple of references at least. I currently don't know enough about the field to work out what to write from this.}

% The above prior work generally assumes that tools can be densely replicated.
%Yet, the optimal tool is dependent upon the field and plants as shown by the variety of tools considered from physical~\cite{chang2021mechanical}, chemical~\cite{wu2020robotic}, electrocuting~\cite{ascard200710} and laser-based~\cite{xiong2017development} implements.
%As such, we conclude that to provide truly precise intervention a weeding platform needs the ability to mount multiple tools and this can be most easily achieved by having movable tools on a robot.
%A key element to this is planning to move the weeding tools to perform weed management.

A frequently overlooked aspect for weeding is planning field based intervention.
% A frequently overlooked aspect for weeding is to plan the intervention in the field.
If a robot carries multiple movable tools then planning their deployment is essential, yet limited work has explored this aspect.
%We argue that detecting weeds alone is not sufficient for an automated weeding approach as in majority of weeding scenarios, multiple weeds must be treated with a single or multiple weeding tools on a platform which needs a planning scheme.
Lee~\etal~\cite{lee2014fast} is one of the few works in this area.
They presented a multi-query approach for efficiently planning paths using a single UR5 robot manipulator to enable precision weeding.
%They did this for a single UR5 robot manipulator to enable precision weeding.
This was achieved by maintaining a database of pre-computed paths that was constructed offline and the optimal path was chosen and adapted online.
%When multiple weeds exist in the same frame a technique to plan which weeds to hit and which tool to use is required...
% In a similar context, Xion~\etal~\cite{xiong2017development} propose to visible segment weeds into equally spaced regions along x-axis of viewable are of camera, which then their laser-based intervention head can engage with targets sequentially.
Xion~\etal~\cite{xiong2017development} proposed a laser-based intervention scheme that segmented the visible weeds into equally spaced regions along the x-axis, this then allowed them to engage targets sequentially.

% \todo{MAH to Alireza: What we need 4-5 papers with platform type (in built or thorvald etc.), steering type, computation, weeding tools. Not saying we will use all of these but that's the info I need to write up.}


In this work, we aim to alleviate some of the limitations of the previous approaches by introducing a set of movable and replicated tools on a single weeding platform.
%we introduce a weeding platform with four linear axis able to selectively spray weeds in a field.
Furthermore, we present a robot that is able to navigate down European standard crop planting patterns which automatically monitors the crop with species level classification and localization.
This provides both automated field monitoring and weed management capabilities.



\begin{comment}
Weed management techniques have evolved for decades towards selective plant-level interventions which try to provide treatment for every crop in the field regardless of their distribution and type.
Such approaches are being compared to the human performance of manual weeding tasks.
Smart farming technology utilizes a variety of techniques for enabling autonomous localization and navigation in open fields\cite{ahmadi2021towards} to selective image-based perception and intervention methods~\cite{mccool2018efficacy}.

To perform precise field monitoring and autonomous weed intervention a platform requires species-level plant classification and localization.
Agricultural monitoring approaches have gained significant research attention in recent years, including in glasshouses~\cite{smitt2021pathobot}, orchards~\cite{}, and fields~\cite{halstead2021crop}.
% The high diversity of farms means that in general monitoring approaches have been designed for a single crop or location.
For arable farmland, significant research into monitoring has been explored, with a specific focus on weed management~\cite{bakker2010systematic, peruzzi2017machines}.
Early techniques aimed to use computer vision and traditional machine learning for plants and vegetables such as~\cite{} or carrot~\cite{utstumo2018robotic}.
Furthermore, Bawden~\etal~\cite{Bawden17_1} showed local binary pattern features(LBP) is able to accurately classify a range of weed species.

Recently, deep learning-based approaches have been shown to be more generalizable in that they can be deployed for multiple different crop types.
Similarly, leveraging computer vision and machine learning~\cite{utstumo2018robotic} was able to reduce herbicide usage through a drop-on-demand weeding platform.
This approach has the potential to increase soil health with specific spraying or use more potent herbicides as only a small area was targeted.
To improve the accuracy of monitoring techniques in arable farmland~\cite{ahmadi2021virtual}, was able to utilize the temporal information captured on the platform.
In their work, they were able to improve the semantic segmentation accuracy of a temporally augmented UNet~\cite{ronneberger2015u} structure.
Generally, these approaches are designed for a single environment (arable farmland or glasshouse), crop, or built for a specific platform.
To alleviate these constraints~\cite{halstead2021crop} outlined a crop, environment, and platform agnostic monitoring approach.
Their technique employed a modified Mask-RCNN architecture that allows generalized and fine-grained classification in the same network structure.
This information species-level classification along with fine-grained localization and the addition of the size of the plant enables more informed decisions by the farmer.

For plant level monitoring in farmland, a more strategic process is required that includes fine-grained classification along with localization to empower weed management including intervention.
\cite{blaix2018quantification} outlines the benefit of strategic monitoring approach that enables more dynamic intervention, creating improved crop and soil health.
Similarly, a selective monitoring and intervention paradigm (selective weeding) can improve bio-diversity~\cite{blaix2018quantification, adeux2019mitigating} while increasing the presence of pollinators due to reduced herbicide use~\cite{raven2021agricultural}.

Additionally, for precise in field intervention using active weeding tools, accurate location of the stem of the plants must be estimated where, midtiby~\etal in~\cite{midtiby2012estimating} and Haug~\etal~\cite{haug2014plant} focused on exploiting the Plant Stem Emerging Point (PSEP) using hand-crafted features and sliding window technique, where a  used 
Where recently, Kraemer~\etal~\cite{kraemer2017plants} and  Lottes~\etal~\cite{lottes2018joint} used Convolutional Neural Networks (CNNs) to exploit a likelihood map of the stem emerging points.

A range of weed management robotic platforms has been presented often with alternative approaches to managing weeds.
Often the prior work has presented a single solution to the problem but they often lack generality.
One of the few examples that have a more general approach is AgBot~II~\cite{} which specifically explored multiple tools in a single system~\cite{}.
However, selective spraying still relies on herbicides application, which is not allowed for organic farming. 
A state-of-the-art intelligent commercial~\cite{Garford} introduced an intra-row cultivator based on a machine vision sensor and a rotating disc hoe for intra-row weed control in four vegetable crops in California.
% The physical weeding implements use a direct contact point with soil like rotary, sowing or hoeing mechanism~\cite{Garford}.
For automated weeding, which includes monitoring, AgBot~II~\cite{Bawden17_1} could selectively control the weeding medium (chemical or mechanical) 
In~\cite{Chang21_mechanical_tools} Chang~\etal\ presented two mechanical weeding tools whose deployment was driven by deep learning (Yolo-v3).
\cite{Ruiz14_intracrop_weeding} developing an intra-row mechanical weeding tool with a human in the loop?
Pretto~\etal~\cite{pretto2020building} propose an approach The weeds are treated mechanically with two ranks of stampers or chemically with one rank of sprayers. 
They make the decision on which tool is used on which weed in our experiments is only based on size criteria: large weeds are sprayed while small weeds are stamped.
\cite{Slaughter08_weeding_review} a review of different tools from 2008.

In \cite{zhou2021design} authors focused on the design and preliminary evaluation of a target spray platform. All components were effectively connected and evaluated, especially regarding the response time and target spray accuracy.
Using an indoor experimental setup shows that the developed system can reduce $46.8\%$ usage of chemicals compared to the uniform spray method.


% no repeated weeding tools!
The algorithm is based on a multi-query approach,  inspired by industrial bin picking,  where a database of high-quality paths is computed offline and paths are then selected and adapted online~\cite{lee2014fast}.
they consider the case where there is a single plant (such as capsicum or beetroot) that acts as an obstacle.  
The plant is represented by a sphere of known radius resting on the ground plane centered be-neath the robot base.
% TSP
%DTW



\end{comment}

% Figure environment removed









% \subsection{Similar agricultural platforms}
% Technologies based agricultural solution include a variety of techniques for enabling autonomous navigation in fields to selective image-based perception interventions leading the overall solution being more sustainable.
% Utilizing autonomous agricultural robots can improve productivity and enable targeted field interventions~\cite{perez2014co}, \cite{mccool2018efficacy}.
% Considering different scenarios in which an agricultural robot gets deployed include a variety of cultivars, crop-row structures, seeding patterns, etc~\cite{thuilot2002automatic}, \cite{utstumo2018robotic}.
% To facilitate infield interventions~\cite{underwood2015icra} and crop monitoring tasks, the primary need of the robotics systems is to be guided through each and every crop-row in the field \cite{aastrand2005mec}, \cite{billingsley1997cea}.
% B. Thuilot Automatic guidance of a
% farm tractor along curved paths, using a unique CP-DGPS”\cite{thuilot2002automatic}
% \cite{ahmadi2021towards}
% In addition, it was reported that a selective weeding strategy could decrease the energy input even more by 10 to 90~\cite{griepentrog2006autonomous}
% A state-of-the-art intelligent commercial (Garford, 2014) intra-row cultivator based on a machine vision sensor and a rotating disc hoe for intra-rowweed control in four vegetable crops in California
% Considering different scenarios in which an agricultural robot gets deployed include a variety of plant, crop-row structures, cultivation patterns, etc. A robust and reliable method needs to minimize the tuning and supervision required for execution in the fields. 
% In 2017, Bawden et al.~\cite{Bawden17_1} proposed a plant-specific weed management platform called AgBot~II.
% It consisted of replicated tools, both mechanical and spray, that was deployed with a vision using which utilized traditional rather than deep-learnt features.

% \subsection{Field Monitoring approaches}
% To perform precise weeding in a field from a robotic platform accurate plant classification and localization is required.
% Monitoring approaches such as this have gained significant research attention in recent years, including in glasshouses~\cite{smitt2021pathobot}, orchards~\cite{}, and fields~\cite{halstead2021crop}.
% %Early approaches towards crop monitoring concentrated on species specific techniques using computer vision and traditional machine learning~\cite{}, while recently more dynamic approaches have employed deep learning approaches~\cite{}.
% The high diversity of farms means that in general monitoring approaches have been designed for a single crop or location.
% Early techniques aimed to use computer vision and traditional machine learning for objects such as grapes~\cite{Nuske_2011_6891} or almonds~\cite{hung2013orchard}.
% Utilising conditional random fields and sparse auto-encoders~\cite{Hung:2013aa} and ~\cite{mccool2016visual} were able to accurately classify and localize almonds and sweet pepper respectively.

% Recently, deep learning based approaches have been shown to be more generalizable in that they can be deployed for multiple different crop types.
% Similarly, for cropping environments, \cite{koirala2019deep} and~\cite{tian2019apple} compared two seminal deep learning networks for mango and apple detection, Faster-RCNN~\cite{ren2015faster} and Yolo~\cite{Redmon_yolo_16}, they showed that Yolo was capable of both real-time performance and high accuracy.
% Recently,~\cite{halstead2020} showed the benefits of leveraging multi-task learning via Mask-RCNN~\cite{he2017mask} for accurate detection in-the-wild.
% They were able to detect the same crop type (sweet pepper) in vastly different cropping locations (glasshouse versus field) with high accuracy.

% For arable farmland significant research into monitoring has been explored, with a specific focus on weed management~\cite{bakker2010systematic, peruzzi2017machines}.
% For plant level monitoring in farmland a more strategic process is required that includes fine-grained classification along with localization to empower weed management including intervention.
% \cite{blaix2018quantification} outlines the benefit of strategic monitoring approach that enables more dynamic intervention, creating improved crop and soil health.
% Similarly, a selective monitoring and intervention paradigm (selective weeding) can improve bio-diversity~\cite{blaix2018quantification, adeux2019mitigating} while increasing the presence of pollinators due to reduced herbicide use~\cite{raven2021agricultural}.

% For automated weeding, which includes monitoring, AgBot~II~\cite{Bawden17_1} could selectively control the weeding medium (chemical or mechanical) based on the type of weed classified with a traditional vision based detection routine.
% Similarly, leveraging computer vision and machine learning~\cite{utstumo2018robotic} was able to reduce herbicide usage through a drop-on-demand weeding platform.
% This approach has the potential to increase soil health with specific spraying or use more potent herbicide as only a small area was targeted.
% To improve the accuracy of monitoring techniques in arable farmland~\cite{ahmadi2021virtual}, using BonnBot-I, was able to utilise the temporal information captured on the platform.
% In their work they were able to improve semantic segmentation accuracy of a temporally augmented UNet~\cite{ronneberger2015u} structure.
% Generally, these approaches are designed for a single environment (arable farmland or glasshouse) and built for a specific platform.
% Halstead et al.~\cite{halstead2021Frontiers} recently outlined the ability to create crop, environment, and platform agnostic approaches for monitoring.
% This technique employed a modified Mask-RCNN approach that allows generalized and fine-grained classification in the same network structure.
% This approach can, along with species-level classification, is able to supply size information to an operator enabling informed decisions by the farmer.

% % lpant stem location stem location estimation
%  \cite{lottes2018joint} stem location estimation 
% Haug~\etal~\cite{haug2014plant} proposed a sliding window based approach for detecting crop and weed stem locations in Multi-spectral images.
% In~\cite{midtiby2012estimating} Midtiby~\etal focused on exploiting the Plant Stem Emerging Point (PSEP) using hand-crafted features, where a Kraemer~\etal~\cite{kraemer2017plants} used Convolutional Neural Networks (CNNs) to exploit a likelihood map of the stem emerging points.

% \subsection{Weeding approaches}
% A range of robotic weed management approaches have been proposed.
% These systems often present an alternative approach to weed management.
% Often the prior work has presented a single solution to the problem but they often lack generality.
% One of the few examples that has a more general approach is AgBot~II~\cite{} which specifically explored multiple tools in a single system~\cite{}.
% However, selective spraying still relies on herbicides application, which is not allowed for organic farming. 
% In~\cite{Chang21_mechanical_tools} Chang~\etal\ presented two mechanical weeding tools whose deployment was driven by deep learning (Yolo-v3).
% \cite{Ruiz14_intracrop_weeding} developing an intra-row mechanical weeding tool with a human in the loop?
% Furthermore, such equipment's are expensive and vulnerable to outage.
% Pretto~\etal~\cite{pretto2020building} propose an approach The  weeds  are treated mechanically with two ranks of stampers or chemically with one rank of sprayers. 
% They make the  decision  on  which tool is used on which weed in our experiments is only based on a size criteria: large weeds are sprayed while small weeds are stamped.
% \cite{Slaughter08_weeding_review} a review of different tools from 2008.