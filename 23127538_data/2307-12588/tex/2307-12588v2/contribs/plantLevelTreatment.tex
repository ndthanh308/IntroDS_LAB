
%We assume the robot moves along a crop-row with constant speed $\gamma$ so the kinematics of weeds at time $t$ w.r.t the weeding implement on the robot could be visualized as~\figref{fig:kinematicsWeeding}.
%Consequently the intervention is time-critical and must respect the spatial ordering of the weeds.
%Similar to~\cite{bawden2017robot} we assume weeds are uniformly distributed in the field with density $\lambda$ weeds/$m^2$. 
%Hence, using a Poisson process we can explain the distance between the weeding implement and individual weeds by accounting for the arrival rate of $\eta =\lambda \times \Pi$.
%We use the motion along the $x$-axis of the robot frame $\mathcal{F}_R$, to explain the weeds interval distance ($\delta_x$).

We assume the robot moves along a crop-row with constant speed $\gamma$.
Consequently, intervention is time-critical and must respect the spatial ordering of the weeds.
There is a constant gap ($\Gamma$) between the tools and the area sensed by the camera ($C_{detect}$).
Similar to~\cite{bawden2017robot} we assume weeds are uniformly distributed in the field with density $\lambda$ weeds/$m^2$. 
Hence, using a Poisson process we can explain the distance between the weeding implement and individual weeds by accounting for the arrival rate of $\eta =\lambda \times \Pi$.
We use the motion along the $x$-axis of the robot frame $\mathcal{F}_R$ to explain the weeds interval distance ($\delta_x$), visualized in ~\figref{fig:kinematicsWeeding}.
This can be shown using the following probability density function,
% 
\begin{equation}
    \label{eq:weedsIntervaldistance}
    f(\delta_x) = \lambda \Pi e^{-\lambda \Pi \delta_x},
\end{equation}
% 
also the location of weeds on the $y$ axis can be represented via a uniformly distributed random variable $y$ as,
% 
\begin{equation}
    \label{eq:yPDF}
	f(y) =\left\{\begin{array}{cc}
	    \frac{1}{\Pi} & for  \ \ 0 \leq y \leq \Pi \\
	     0 & otherwise
	\end{array}\right..
\end{equation}
% 
To engage the $i$-th intervention head with the $j$-th weed it has to traverse,
% 
\begin{equation}
    \label{eq:yDistance}
    \delta^{ij}_{y} = | \textit{h}_{i} - \textit{n}_{j} |, 
    \text{ where }  0 \leq \delta^{ij}_{y} \leq \Pi ,
\end{equation}
%
where $\textit{h}_{i}$ is the current position of $i$-th intervention head and the $\textit{n}_{j}$ denotes to the position of the $j$-th weed.
Therefore, the probability of visiting the $j$-th weed with $i$-th intervention head can be calculated with,
% 
\begin{equation}
    \label{eq:weedingProb}
    \textit{P}_{ij} = \textit{P}\left( \dfrac{\gamma}{\vartheta}  <   \dfrac{\delta_{x}}{\delta_{y}} \right),
\end{equation}
%
where $\vartheta$ denotes to the maximum velocity of linear axes.
We assume all targets are detected by the time they reach the bottom edge of the camera's viewable area.
% \TODO{MAH confusing: We suppose targets get detected by the time passing the center of viewable area of camera $C_{detect}$.}
%As~\figref{fig:kinematicsWeeding} depicts there is a gap with a length of $\Gamma$ between the work-space of the weeding implement and the bottom edge of the viewable area of the front camera $C_{detect}$. % in front of robot.
%Finally, this distance provides extra time for the planning system, this is defined by $\tau_e=\Gamma/\gamma$ and depends on the linear velocity of robot and length of region~$\Gamma$.
%% \TODO{MAH confusing: Finally, this provides extra time for detection and planning which we refer to it by $\tau_e=\Gamma/\gamma$ that depends on the linear velocity of robot and length of region~$\Gamma$.}
% Let $\kappa = \tau_v + \tau_i$ represent the intervention head engagement time for our spot spray systems to be constant, 
% where $\tau_v$ shows the constant operation time of the nozzle valve (On-Off time) and $\tau_i$ denotes the engagement time with weed $i$ in milliseconds.
%
% Figure environment removed


