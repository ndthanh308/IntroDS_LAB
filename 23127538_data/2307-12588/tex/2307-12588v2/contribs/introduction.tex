The preference for consuming more natural and organic foods has rapidly increased in recent years~\cite{blasco2002ae}.
This has forced the agricultural industry to use fewer agri-chemicals when dealing with weeds while maintaining the quality and quantity of the crop.
Weed intervention is an important aspect of arable farming due to its competition with crops for nutrients in the soil~\cite{slaughter2008autonomous} which potentially reduces yield.
Currently, to alleviate this impact, the majority of farmers use uniform treatments for weed control for instance by treating the entire field with herbicide irrespective of weed presence. 
This approach has led to an increasing number of herbicide-resistant weed species~\cite{heap22_website} as well as negatively impacting the environment by increasing soil erosion and water contamination~\cite{mia2020sustainable}.

Robotic weed intervention has the potential to revolutionize weeding paradigms through plant-level weed management.
For instance, by only treating a plant if it is present and using the most appropriate action for the particular plant species~\cite{Bawden17_1}.
To achieve plant and species specific treatments, robots are driven by advanced perception systems that can also provide rich crop monitoring information~\cite{halstead2021crop}.
Yet, robotic design has primarily explored how to design the platform to perform weeding operation.

Several robotic weed control platforms have been introduced offering active and passive interventions in field.
A variety of weeding implements have been investigated including physical~\cite{chang2021mechanical}, chemical~\cite{wu2020robotic}, electrocuting~\cite{ascard200710}, laser-based~\cite{xiong2017development}.
Given the variety of tools, it is clear that there is no one best solution and robotic solutions should be able to cater to a variety of tools.
The multi-modal approach of Bawden et al.~\cite{Bawden17_1} provided a clear step in this direction, however, a downside of their approach was the need to densely replicate each tool as they were mounted statically.
We propose to overcome this limitation by considering replicated movable tools.
 
% Figure environment removed

In this paper we introduce \bbot, a robot that performs both field monitoring and precision weed management.
It enhances the capabilities of a recently published field monitoring technique~\cite{halstead2021crop} by using multiple localization sensors (GPS and odometry).
The plant counting performance is improved, reducing the normalized absolute error by more than half from $8.3\%$ down to $3.5\%$.
In developing \bbot\ we also propose a novel arrangement of weeding tools to enable precise weed management by replicating linear systems.
An advantage of this approach is that we can deploy both fewer and smaller (more precise) tools, rather than densely replicated tools~\cite{Bawden17_1}, while still allowing for good coverage.
We demonstrate that it is feasible to have a system with just 4 replicated tools of size $0.05m$ on linear systems to cover a width of $1.3m$ which would normally require at least $26$ non-overlapping tools.
This leads to the following novel contributions:
\begin{enumerate}
    \item We introduce BonnBot-I a fully autonomous precision weeding platform fully compatible with ROS.
    \item Propose a new concept for weeding tools that enables flexible high-precision weed management.
    \item Improve crop monitoring performance by exploiting the extra sensors available with \bbot. 
    \item Release a new dataset consisting of corn as the crop, CN20. 
    This is a challenging dataset for crop monitoring approaches as it is a grass crop. 
    \item Introduce a framework for testing different weeding intervention strategies using a simulation environment.
\end{enumerate}