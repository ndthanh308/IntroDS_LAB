
Planning the motion of each intervention head must be done prior to targets entering the weeding tool work-space. % of weeding tool. 
In the proposed workflow, the intervention controller node receives the detected targets at time $t + \tau_d$ where $\tau_d$ is the time required for detection in the monitoring node.
The monitoring node provides plant specific information like: plant category, pixel-wise segmentation, estimated area, and the bounding box.
Furthermore, we estimate plant centers based on provided bounding box in the scene.  % around the detection.
%Then, the aim would be to find the best motion plans for all interventions heads to minimize the targets which are not being visited (sprayed).
%This information is then used in the target-space management step to assign targets between intervention heads.
This information is then used in the target-space management step to assign targets to the intervention heads.
Based on this the next step finds the best motion plan for each intervention head by maximizing the number of targets that are visited (sprayed).
%Based on this the next step finds the best motion plan for each intervention head that maximizes the number of targets that are visited (sprayed).

Let $\mathcal{H}$ denote the number of independent controllable intervention heads and $\mathcal{N}$ be the number of targets that appear under the robot. %to be the number of targets appeared in the underneath the robot. %, the main goal is to visit all the targets with at least one of the intervention heads as they pass the work-space of weeding tool without the need for stopping the robot.
%By considering the fact that robot is moving forward and, we refrain from moving backwards, the intervention is time-critical and must be in spacial order.
We use a uni-directional constrained node-graph to model the targets-space.

To obtain the global spatial order of targets in a segment we use the $\delta_x$ of each weed (see \figref{fig:kinematicsWeeding}).
In~\figref{fig:nodeGraph}(a), each node (circle) shows a weed along with the connecting path between nodes $j$ to $k$ represented with a uni-directional link (arrow) $l_{jk}$.
The link $l_{jk}$ exists if, node $j$ geometrically is located after node $k$ in the 3D world frame  $\mathcal{F}_w$ in the direction of motion.
Furthermore, the link $l_{jk}$ is associated with an inter-weed cost $\varrho_{jk}$ based on the distance of nodes $j$ and $k$ and a property denoting motion probability of $\textit{P}_{jk}$ based on~\eqref{eq:weedingProb}.
We calculate inter-weed costs using the top-right of the cost-matrix~$\mathcal{G}_{\mathcal{N} \times \mathcal{N}}$ (to respect the weeds spatial order).


% \begin{equation}
%     \label{eq:costMatrix} 
%     \varrho_{jk} = (n_j - n_k)^2
% \end{equation}
% 
There are $\mathcal{H}$ independent interventions heads and so multiple plans which can lead to the same number of targets being visited (sprayed).
%As the weeding tool contains $M$ independently controllable interventions heads, it provides this flexibility to plan a variety of engagement routes by considering different method of target assignments.
%Considering sets of weeds in any given work-space motivates a multi-query approach  to the problem.
To solve this problem, we consider the weeds as a sets of targets detected in one location, this motivates us to assign intervention targets to the $\mathcal{H}$ heads as either distance-based or work-space division-based assignments.
% To solve this problem, we consider weeds to be presented as a set which motivates us to consider assigning targets to the $\mathcal{H}$ intervention heads as either a distance-based assignment or a work-space division-based assignment approach.
%Prior to plan engagement routes, we use a high level planner to distribute targets between different intervention heads while, making sure that all the targets will be at least visited once. 
%we suggest to represent the work-space in two different ways:

\begin{enumerate}
    \item \textbf{\textit{Distance-based Target Assignment (D)}:}
        In this approach, target $j$ gets assigned to the laterally closest intervention head along the sliding direction ($y$-axis). 
        This means, selected intervention head $i$ has the least motion required to reach the weed $j$.
        The lateral distance between heads and weeds are defined based on 2D euclidean distance between projection of intervention head's position on ground plane and weed positions on same plane w.r.t the $\mathcal{F}_w$ frame.
        
    \item \textbf{\textit{Static Work-space Division-based Target Assignment (SD)}:}
        In this method, we divide the work-space of weeding tool to $\mathcal{H}$ sub-sections of width $\Pi/\mathcal{H}$ meters.
        Hence, each intervention head is only responsible for engaging with weeds laying within it's sub-work-space as shown in~\figref{fig:nodeGraph}(b)-top.
        
    \item \textbf{\textit{Dynamic Work-space Division-based Target Assignment (DD)}:}
        % \TODO{MAH let's talk about this, it still confuses me reading this and then looking at the figure.}
        In this model, for each new set of detected weeds we first determine the minimum region of intervention defined by $y_{min}$ and $y_{max}$ (see Fig.~\ref{fig:nodeGraph}~(b)). 
        The minimum region of intervention is then divided into $\mathcal{H}$ equal sub-regions.
        This process assists in optimizing the planning for weed engagement by potentially reducing the area any one tool has to cover.
        %In this model, for each new segment we define an intervention necessary region by finding the spared of weeds positions along $y$-axis underneath the robot. 
        %Then we divide this region into $\mathcal{H}$ equal sub-regions for performing weeding engagements as shown in~\figref{fig:nodeGraph}(b)-bottom.
        %This technique ensures more proper load balancing over interventions heads in case of facing portions of the fields with non-uniform weed distribution.
\end{enumerate}
% 
%We note that the above approaches have been derived to demonstrate the potential of our proposed system and we believe future work can derive more advanced approaches to provide even better coverage.

% Figure environment removed