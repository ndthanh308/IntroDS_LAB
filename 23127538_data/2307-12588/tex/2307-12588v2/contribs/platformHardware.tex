
\bbot\ is a platform capable of both field monitoring and weed management developed at the University of Bonn.
The base platform is a Thorvald system~\cite{grimstad2017thorvald} which is a lightweight four-wheel-drive (4WD) and four-wheel-steering (4WS) system.
With considerable modifications, we adapted this platform to an arable farming and phenotyping robot suitable for operation in the real phenotyping fields.

These modifications were carried out to ensure they met European farming standards for the distance between rows.
Based on these specifications the width from wheel-centre-to-wheel-centre was set to $1.5m$.
To ensure we could monitor the different growth cycles of our primary crops (sugar beet, wheat, and corn) the vertical clearance is set to $0.57m$.
The length of BonnBot-I, $1.39m$, was selected to ensure there was adequate space to install replicating weeding tools.
These dimensions also act to increase the stability of the platform on the uneven terrain witnessed in arable farmland.
An overview of the \bbot\ platform is provided in~\figref{fig:fields}.

% \TODO{CM to ALL: I would remove the paragraph below.
% Key to deploying \bbot\ is the sensor array enabling navigation, data capture, crop monitoring and weed management at the same time.
% Hence, \bbot\ houses sensors of multiple interests: localization, navigation and crop monitoring encouraging operations in a vast variety of crop-fields with different growth-stages, number of crop-rows. 
% Additionally, benefiting our novel weeding tool implement, we are able to conduct selective and precision plant-level interventions in arable fields.
% For more platform details see~\ref{fig:fields}, on the left is a detailed schematic of the robot and on the right is a 3D generated model. These images detail the construction level specifics of \bbot\.
% In the following we elaborate more on sensor setup and weeding implement.}


% \textcolor{blue}{
% The placement of the GPS antennas was chosen to be consistent with a second purely phenotyping platform. 
% The purely phenotyping platform was chosen to have a vertical clearance of $1.6m$ and to ensure consistency between the two platforms the antennas of \bbot\ were mounted at the same position.}
% \TODO{CM to AA: I suggest we mainly talk about the dimensions of the weeding platform. I've commented the previous text and then talked about there being another robot for phenotyping that constrained where the GPS antennas had to be mounted.}
% \textcolor{blue}{
% the phenotyping version of this robot had to be able to sense a variety of crops at different stages of growth.
% This led to the choice of a vertical clearance of $1.6m$.
% The length of the robot was then set to ensure room for weeding tools and was set to $1.4m$, this also reduced issues for instability.}
% \TODO{The above dimensions need to have the real numbers for vertical clearance and area for the weeding tools. Workshop?}
% This led to the overall dimensions of the robot of $1.62 \times 1.8 \times 2.175m$ (width, length, height), as shown in Figure~\ref{}.
% With these parameters, this lightweight robot could be mounted on a regular truck provided that the GPS antennas were rotated, see Figure~\ref{}.
% \TODO{We need the details of the truck with dimensions, etc. Workshop?}

\subsection{Sensor Configuration}
\label{subsec:sensorConfiguration}


We equipped BonnBot-I with a range of sensors to perform both in-field weed management and crop monitoring.
There are two sets of sensors, the first set is for ``localization and navigation'' and the second set is for ``robotic vision''.
%This infrastructure enables the platform the achieve state-of-the-art capabilities in arable fields to perform critical tasks like: autonomous navigation, localization, and environment perception encouraging selective and plant-level real-time interventions.
%The sensors range from intertial measurement units (IMUs) and GNSS sensors for accurate localization through to 2D and 3D vision sensors such as RGB-D cameras and lidar; we will expand on the specific components here.
%Below we briefly these describe these components.

\subsubsection{Localization}
\label{subsubsec:navigation}

The localization of BonnBot-I is performed with a compact Inertial Navigation System (INS), Ellipse2-D SBG Systems~\cite{sbg} which includes an IMU and a dual-antenna receiver, multi-band GNSS receivers fixed at the front and back of the robot at height of $1.85m$ above the ground.
%Benefiting an on-board high-frequency Extended-Kalman filter fusion of IMU and GPS data providing a horizontal and vertical position accuracy of $4cm$ and $3cm$, respectively. 
Using an on-board high-frequency extended-Kalman filter fusion of IMU and GPS data provides us with a horizontal and vertical position accuracy of $4cm$ and $3cm$ respectively. 
Furthermore, the heading of the platform can be determined with an accuracy of $0.1^{\circ}$ and $0.3^{\circ}$ in roll-pitch and yaw directions respectively.
%
% To provide a homogeneous view of the environment we added an Ouster-OS1 lidar to the front. %at the front of the robot. 
% The OS-1 lidar is a 64 beam lidar covering $360$ and $45$ degree field of view in the horizontal and vertical directions, with a maximum range of $120m$.
%degrees horizontal and vertical field of views with maximum range of $120m$ meters.
% This provides long-range sparse depth data to to enable 3D mapping, navigation, and for safety.

%Even-though, in \cite{ahmadi2021towards} we showed how BonnBot-I is able to automatically traverse multi-crop-row field without any human intervention and independent of any global localization service using only two symmetrically fixed Intel RealSense-D435i in front and back of the robot denoted by $C_{f}$ and $C_b$ in \figref{fig:robot}.

% Additionally, to enable automated row-crop field traversing we use two RGB-D Intel RealSense-D435i cameras, one on front and one on the back of the platform with a fixed tilt angle.
% In~\cite{ahmadi2021towards} we showed that this sensor configuration could reliably navigate a field regardless of the number of crop-rows under the robot or the crop type; this was achieved without using any global localization service. 

%we showed that this sensor configuration could be used for automatic navigation in arable fields with diverse crop types as well as multiple crop-rows achieving reliable performance without using any global localization service.

\subsubsection{Robotic Vision}
\label{subsubsec:detectionInference}
BonnBot-I is equipped a nadir-view cameras (Intel Real-sense D455) on the front of the platform used for monitoring purposes.
% \RNum{1} \textit{JAI Camera:}
% The AD-130GE is a Rolling shutter CCD multi-spectral GigE Vision compliant camera. 
% This device employs 2 CCD sensors, one for Bayer color and the other for NIR monochrome utilizing prism optics so that the AD-130GE can inspect the objects by visible color sensor and Near IR sensor with the same angle of view.
% \RNum{2} \textit{Real-sense D455 Camera:}
The Real-sense D455 is a global shutter camera which provides RGB and registered depth images at $15 Hz$.
On \bbot\ it is fixed at a height of $0.78m$ providing a view-able area of $1.4m\times0.78m$ covering the gap between the two front wheels. %, with a $15 Hz$ frame-rate. 
This is the sensor used to perform field monitoring.
% This RGB-D camera is the primary sensor to perform field monitoring.
%used to perform field monitoring
%This sensor obtains depth information from two NIR cameras. 
%It has a NIR projector that is used to project points onto the scene so that depth can be estimated on visually texture-less surfaces such as walls.

% As this is not a problem with the setting for an agricultural robot to avoid interference with other sensors (e.g. NIR channel of the JAI camera) we disable the projector and consider this to be an RGB+D+NIR sensor.

\subsection{Weeding Implements}
\label{subsec:weedingImplements}

% Here I want to only describe the mechanic and electronics of the weeding tool 

%The necessity of this problem is getting bolder as we find out more about the harms that broadcast weeding reinforcements are causing to our eco-system.
Achieving flexible, and repeatable weeding implements that can deploy a variety of end-effectors is a key objective for BonnBot-I.
This enables the system to change tools given the current soil and weed populations.
The proposed design utilizes independently controlled high-resolution Igus linear actuators fixed at height of $0.72m$ above the ground and creating a working space of $1.3m\times0.36m$; the current design uses 4 such linear actuators.
%The proposed design utilizes four independently controlled high-resolution igus ZLW-1040S linear actuators fixed on the BonnBot-I at height of $0.6m$ above the ground and creating a working space of $1.3m\times0.36m$ as shown in \figref{fig:robotScheme}.
Each linear axis has a length of $1.3m$ and is controlled by an Igus dryve D1 motor control system via Modbus connection with a maximum resolution of $0.01cm$ and is capable of performing translations with maximum velocity and acceleration of $5m/s$ and $10m/s^2$, respectively.
All linear axes are equally-spaced and currently carry spot-spray nozzles, however, the system design permits many kinds of end-effectors, such as mechanical hoeing, providing flexibility. % for possible weeding tools like mechanical spot hoeing.
%all linear axes carry spray valves so can be used independently to engage with weeds.

To control the linear actuators and spray valves we use a ROS operated Raspberry-Pi 3B and to ensure minimal action delay the nozzles are accessed via high speed N-channel MOSFET-transistors.
Hence, ultimate operation time for each spray head in our system adds up to $10\sim12$ms including valves On-Off time. 
The spray system consists of a reservoir tank capable of carrying $5L$s of compressed liquid with a maximum $16bar$ pressure, as well as a compact $8bar$ portable compress which is fixed on the robot.
As the droplet size from the spray nozzles depends on the liquid pressure we use individually adjustable valves.
This allows us to control the spray footprints on the ground individually for each nozzle between $0.02m$ to $0.13m$; for this paper we assume a constant spray footprint of $0.05m$.



\begin{comment}
To control Linear actuator and nozzle valves we use a Raspberry-Pi 3-B single-board computer which all its outputs are interfaced via PiXtend v1.3 I/O board. 
Pixtend is a PLC (Programmable Logic Controller) compliant logic board with a wide range of isolated digital and analog inputs and outputs.
It provides standard serial interfaces RS232, RS485, Ethernet, CAN and etc, and is shown in ~\figref{fig:weeding_controler}.
We use High speed N-Channel MOSFET-Transistor (IRFZ44N 55V, 41A) of Pixtend board to control the spray nozzle valves, this ensures minimize action delay in electronics. 
% % Figure environment removed 
\subsubsection{Spray Infrastructure}
all linear axes carry spray implements so can be used independently to engage with weeds.
We use ASCO™ solenoid L172V03 spray valve with On-Off time equal to $\sim10ms$, which are lowered to a height of $0.22m$ using an aluminum level arm. 
The spray system consist of a reservoir tank capable of carrying $4L$ compressed liquid with maximum $80bar$ pressure.
And each nozzle is being fed with compressed herbicide via isolated pipes and control valve. 
This way we can control output pressure of each spray separately.
Also as the droplet size of spray nozzles depend on the liquid pressure we can control the spray footprints on the ground and droplet sizes individually for each nozzle.
\end{comment}


% \subsubsection{Hoeing Implement}
% On the two other linear axes we have fixed a high-speed pneumatic cylinders with length of $12cm$ used for turning soil and taking out the weeds from soil which in agriculture it is called hoeing.
% Hoeing aims to sever the top growth from the roots, just below the soil surface, then leaves the weed in the sun to wither to dry-out.
% While, this approach does not guarantee successful treatment to all types of weeds spatially deep-rooted or perennial weeds~\cite{larkcom2013grow}.
% But, by throwing the plant out of the soil we ensure to increase the success rate.
% Furthermore, by considering different types of hoeing axles 
% Long-handled hoes are easier on the back, whereas a short-handled ‘onion hoe’ is better for closely planted areas, where you don’t want to damage nearby plants. 
% We propose the following shape which ...
% The pneumatic cylinders control the positions of a lever-arm carrying a hoeing head used turning soil.
% As~\figref{fig:heingTool} shows the mechanism is fixed with a vertically fixed Aluminium lever-arm  

% % Figure environment removed





