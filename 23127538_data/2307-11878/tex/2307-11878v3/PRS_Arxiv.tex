\documentclass[12pt]{article}
\usepackage[a4paper,  left=1.3in, right=1.3in, top=2in, bottom=2in]{geometry}
\usepackage{amsmath}
\usepackage{graphicx,psfrag,epsf}
\usepackage{enumerate}
%\usepackage{natbib}
\usepackage{url} % not crucial - just used below for the URL 
\usepackage{booktabs}

\usepackage{ragged2e}
%\usepackage[utf8]{inputenc}
\usepackage{caption}
%\usepackage{amsmath}
\usepackage{amssymb}
%\usepackage{setspace}
%\usepackage{color}
\usepackage{graphics}
\usepackage{epstopdf}
\usepackage{lipsum}
%\usepackage{mwe}
\usepackage{bm}
%\usepackage[margin=1in]{geometry}
%\usepackage[]{mdframed}
\usepackage{bbm}
%\usepackage{comment}
\usepackage{placeins}
%\usepackage{soul}

%\usepackage{float}

\usepackage[english]{babel}
\usepackage{csquotes}

%\usepackage[style=authoryear-ibid,backend=biber]{biblatex}
%\addbibresource{PSIBiblio.bib}
 
\usepackage{arydshln}
\usepackage[usestackEOL]{stackengine} 

\usepackage{amsthm}
\theoremstyle{plain}
\newtheorem{definition}{Definition}
\newtheorem{Hypothesis}{Hypothesis}

%\pdfminorversion=4
% NOTE: To produce blinded version, replace "0" with "1" below.
\newcommand{\blind}{0}
\captionsetup{justification=Centering,
              skip=0.5\baselineskip}
% DON'T change margins - should be 1 inch all around.
\addtolength{\oddsidemargin}{-.5in}%
\addtolength{\evensidemargin}{-.5in}%
\addtolength{\textwidth}{1in}%
\addtolength{\textheight}{1.3in}%
\addtolength{\topmargin}{-.8in}%

\usepackage{setspace}
\usepackage{natbib}

\theoremstyle{plain} % Bold title, italic text (default style)
\newtheorem{theorem}{Theorem}
\newtheorem{proposition}[theorem]{Proposition}
\bibliographystyle{apalike}
\doublespacing

\begin{document}

%\bibliographystyle{natbib}

%\def\spacingset#1{\renewcommand{\baselinestretch}%
%{#1}\small\normalsize} \spacingset{1.45}

%%%%%%%%%%%%%%%%%%%%%%%%%%%%%%%%%%%%%%%%%%%%%%%%%%%%%%%%%%%%%%%%%%%%%%%%%%%%%%

  \title{\bf The Population Resemblance Statistic: A Chi-Square Measure of Fit for Banking}
  \author{CJ Potgieter\thanks{Department of Mathematics, Texas Christian University}, C van Zyl\thanks{Afrimat Limited; Absa Bank Limited; Centre for Business Mathematics and Informatics, North-West University}, WD Schutte\thanks{Absa Bank Limited; Centre for Business Mathematics and Informatics, North-West University}, F Lombard\thanks{Posthumous, Department of Statistics, University of Johannesburg }}
  \date{}
  \maketitle
  
\abstract{The Population Stability Index (PSI) is a widely used measure in credit risk modeling and monitoring within the banking industry. Its purpose is to monitor for changes in the population underlying a model, such as a scorecard, to ensure that the current population closely resembles the one used during model development. If substantial differences between populations are detected, model reconstruction may be necessary. Despite its widespread use, the origins and properties of the PSI are not well documented. Previous literature has suggested using arbitrary constants as a rule-of-thumb to assess resemblance (or ``stability''), regardless of sample size. However, this approach too often calls for model reconstruction in small sample sizes while not detecting the need often enough in large sample sizes.


This paper introduces an alternative discrepancy measure, the Population Resemblance statistic (PRS), based on the Pearson chi-square statistic. Properties of the PRS follow from the non-central chi-square distribution. Specifically, the PRS allows for critical values that are configured according to sample size and the number of risk categories. Implementation relies on the specification of a set of parameters, enabling practitioners to calibrate the procedure with their risk tolerance and sensitivity to population shifts. The PRS is demonstrated to be universally competent in a simulation study and with real-world examples.
}

{\textbf{Keywords:} credit model risk; discrete goodness-of-fit; non-central chi-square; population stability index (PSI); model validation and monitoring; Kullback-Leibler divergence.}


\maketitle


\section{Introduction}
\label{sec_intro}

Testing the stability of a population used for model development is common practice in model risk management. In credit risk modeling, the Population Stability Index (PSI) is the most widely used measure to monitor the evolution of the population underlying a model, through assessing the degree of discrepancy, conversely similarity, between two discrete probability distributions (see \citet[pp. 155 ff.]{thomas2002credit} and \citet[pp. 368 ff.]{siddiqi2017intelligent}). Small deviations in the population can result in inaccurate or unreliable model predictions. For example, consider modeling the Probability of Default (PD), or risk score, based on a given population of borrowers; if the latter changes substantively, the reliability of the PD model becomes questionable. In this case, the current population should \textit{resemble} the one used during model development as it is a requirement of the prudential authorities when the model is used in the calculation of regulatory capital (see \citet{ecb2024internalmodels, ecb2019validation}, \citet{federalreserve2023sr1107a1}, \citet{resbank2022credit}; and \citet{pruitt2010applied} for an application of the PSI in $SAS^\text{\textregistered}$). The same holds true for modeling Exposure at Default (EAD) or Loss Given Default (LGD) in credit risk. Other areas of application include insurance, healthcare, engineering, and marketing (see \citet{huang}, \citet{li}, \citet{sahu}, \citet{dong2022prediction}, \citet{wu2010enterprise}, \citet{mcadams2022risk}, \citet{chou2022expert}, \citet{karakoulas2004empirical} and \citet{brockett1995information}).

Despite its widespread use, the origins and properties of the PSI are not widely understood. The PSI is based on the Kullback-Leibler divergence, measuring difference between two probability distributions \citep[eq. (2.6)]{kullback1951information}. The earliest reference to {the PSI} measure can be found in \citet{lewis1994introduction}, who also coined the term ``Population Stability Index'' and {popularized use of} {the so-called} \textit{Lewis constants} as thresholds (see \citet[p.~155 ff.]{thomas2002credit}, \citet[p.~368 ff.]{siddiqi2017intelligent}). {\citet[p.~106]{lewis1994introduction} describes the PSI without formulating a hypothesis in the statistical sense but notes,} ``If a user finds the distribution of scores \textit{close together}, [they] can be confident that the population has not changed.'' In his example, a PSI below 0.10 indicates that the current population \textit{resembles} the original and no action is required, a value between 0.10 and 0.25 suggests that some investigation should be undertaken, and a value above 0.25 signals a substantial change in the incoming population that may necessitate model reconstruction. In our paper, the word \textit{resembles} is used in the sense of difference by no more than a specified small deviation -- see Definition \ref{deltaresemblance} in Section \ref{sec: measuring resemblance}.

The arbitrary nature of the Lewis constants has been acknowledged by authors and practitioners alike (\citet{yurdakul2020statistical},  \citet{dupisanievisagie2020} and \citet{moodys2021email}). These thresholds pose significant limitations in portfolios with a small number of borrowers, where the shift of a single borrower results in a distortion of the PSI beyond its thresholds, thereby unnecessarily prompting model reconstruction. See \citet[p.~71]{nedbankp3}, \citet[p.~51]{stbankp3} and \citet[p.~232]{frp3} for the prevalence of portfolios with less than (e.g.) 100 borrowers. Conversely, in large portfolios (e.g. $>1$ million borrowers in a retail portfolio), shifts of large volumes of borrowers may remain undetected. These scenarios highlight the PSI’s limitations as a universal discrepancy measure and emphasize the need for practitioners to interpret the results with caution.


\subsection{Research aims and objectives}

This paper introduces the Population Resemblance Statistic (PRS) as a novel and easy-to-use alternative to the PSI for population resemblance monitoring. Based on the Pearson chi-square statistic, the PRS leverages the non-central chi-square distribution to derive {critical} values adjusted for sample size and number of risk categories. This approach allows for a more nuanced detection of population shifts, particularly in sample settings where the PSI {is known to produce} unreliable results. Furthermore, this paper lists some additional measures from the literature to provide greater context for population monitoring. The key objectives of this research are as follows:
\begin{itemize}
    \item To demonstrate the limitations of the PSI in its current form, {including its sensitivity to minor deviations and challenges in various portfolio sizes}.
    \item {To introduce the concept of resemblance as a convenient way of constraining population shift, which is easily communicable to practitioners and that can be utilized to derive business outcomes.}
    \item To develop and formalize the PRS as an alternative population monitoring measure {that addresses the identified limitations of the PSI,} {through the setting of well-founded critical values for action}.
    \item {To formulate the PRS in a setting of a {composite null hypothesis where a point null hypothesis would be too strict, giving rise to use of the least favorable non-central chi-square distribution for decision making}.}
    \item To evaluate the performance of the PRS through simulation studies and real-world applications, {highlighting its competency in both small and large sample sizes}.
    \item To provide a practical framework for implementing the PRS in credit risk modeling and other areas where population monitoring is a concern.
\end{itemize}

\subsection{Paper structure}

The remainder of this paper is structured as follows: Section \ref{sec: measuring resemblance} formalizes the problem, introducing the PRS as an alternative discrepancy measure with a brief discussion of related methods. Section \ref{sec: PRS properties} outlines the statistical properties of the PRS, while Section \ref{sec:framework} details the derivation of sample-size dependent decision boundaries (critical values). Section \ref{sec:implementation} contains a practical guide to implementing the PRS, also giving a flowchart for its application. Section \ref{sec:PRS simulation} presents a comprehensive simulation study, and Section \ref{Application} applies the PRS to real-world data, comparing it with the widely used PSI and other measures. Finally, Section \ref{sec:Conclusion} summarizes key contributions and suggests directions for future research.

\section{Measuring population resemblance}
\label{sec: measuring resemblance}

\subsection{Statistical representation}

Consider an independent and identically distributed (i.i.d.) sample of ordinal scores, $X_1, X_2, \ldots, X_n$ from a discrete population with cumulative distribution function (cdf) $F$ defined on the set of integers $\{1,2,\ldots,B\}$. These ordinal scores often arise by discretizing continuous or numerical values into predefined categories, with each score representing membership in one of $B\geq 2$ disjoint categories, {reflecting, for example, level of risk}. 

Let $n_i$ denote the count of scores in category $i$, formally expressed as $n_i = \sum_{j=1}^{n} \mathbb{I}(X_j = i)$, where $\mathbb{I}(A)$ is the indicator function such that $\mathbb{I}(A)=1$ when $A$ is true and $\mathbb{I}(A)=0$ otherwise. The total number of observed scores, $n$, is given by the sum of all category counts, $n = \sum_{i=1}^{B} n_i$. 

The true probability of a score falling into category $i$ is $p_{i} = P(X_j = i) = F(i)-F(i-1)$ for $i=1,\ldots,B$.  Let $\bm{p}=(p_1,p_2,\ldots,p_B)^\top$. The observed category proportions, serving as unbiased estimators of the true probabilities, are denoted by $\widehat{\bm{p}} = (\hat{p}_1, \hat{p}_2, \ldots, \hat{p}_B)^\top$ with each $\hat{p}_i = n_i/n$ for $i = 1, 2, \ldots, B$. 

The aim is to determine whether the current population $\bm{p}$ resembles the reference population $\bm{p}_0 = (p_{01}, p_{02}, \ldots, p_{0B})^\top$, which was used to construct the model. The specific nature of this model is not germane. Rather, the question is whether $\bm{p}$ has “substantively” shifted from $\bm{p}_0$ or still “resembles” it. While the current population $\bm{p}$ is unknown in practice, $\widehat{\bm{p}}$ serves as an unbiased estimator. {We assume that the reference population probabilities satisfy $p_{0j}> 0$ for all $j=1,\ldots,B$, ensuring that each category in the reference population has a non-zero probability of occurrence. Additionally, the estimated probabilities $\hat{p}_j$, derived from the observed data, are non-negative ($\hat{p}_j \geq  0$) for all $j=1,...,B$.} Furthermore, the random variable $n\widehat{\bm{p}}$ follows a multinomial distribution with expectation $\mathrm{E}[n\widehat{\bm{p}}] = n\bm{p}$ and variance-covariance matrix $\mathrm{Var}[n\widehat{\bm{p}}] = n\big[\mathrm{diag}(\bm{p})-\bm{p}\bm{p}^\top\big].$ Here, $\mathrm{diag}(\bm{p})$ is a diagonal matrix with the elements of $\bm{p}$ on the diagonal.

\subsection{Existing measures of population resemblance}

Several measures have been proposed for quantifying population shift in credit risk modeling. The most widely used is the Population Stability Index (PSI), introduced by \citet{lewis1994introduction}. Defined as
\begin{equation}
    \mathrm{PSI} = \sum_{j=1}^{B} (\hat{p}_{j} - p_{0j})(\log{\hat{p}_j} - \log{p_{0j}}) \mathbb{I}(\hat{p}_{j} > 0), \label{eq:PSI}
\end{equation}
the PSI serves as a consistent estimator of the symmetric Kullback-Leibler divergence
\begin{equation}
\label{eqn:J}
J := J(\bm{p}, \bm{p}_0) = \sum_{j=1}^{B} ({p}_j - {p}_{0j})(\log{{p}_j} - \log{{p}_{0j}}).
\end{equation}
This $ J $, first introduced by \citet{jeffreys1948theory}, is a symmetrized version that addresses the inherent asymmetry of the original ``expected per observation information,'' here equivalent to 
$
I(\bm{p}, \bm{p}_0) = \sum_{j=1}^{B} p_j (\log{{p}_j} - \log{{p}_{0j}})$, see \citet{pichler2020entropy}.

Other measures in risk modeling, {inheriting their properties from the Kullback-Leibler diverge,} include the Information Value \citep[p.~184]{siddiqi2017intelligent} and the Characteristic Stability Index \citep[p.~369]{siddiqi2017intelligent}. Like the PSI, these measures rely on arbitrary thresholds that disregard sample size and statistical properties.

The chi-square divergence,
$$
\chi^2 := \chi^2(\bm{p}, \bm{p}_0) = \sum_{j=1}^{B} \frac{(p_j - p_{0j})^2}{p_{0j}},
$$
offers a statistically principled alternative for measuring population differences. Notably, both $J$ and $\chi^2$ belong to the family of $f$-divergences \citep{renyi1961measures,csiszar1967information}, with $\chi^2$ providing a local approximation to $J$ (and other $f$-divergences) when populations are similar \citep{csiszar2004information}. This property, combined with its well-understood statistical properties, makes $\chi^2$ an attractive foundation for population monitoring.

Building on these advantages, we propose the Population Resemblance Statistic (PRS), defined as 
\begin{equation}
    \mathrm{PRS} = \sum_{j=1}^{B} \frac{(\hat{p}_{j} - p_{0j})^2}{p_{0j}}, \label{eq:PRS}
\end{equation}
{offering a measure grounded in well-established statistical theory, thereby also addressing key limitations of the PSI.} Specifically, we will develop methods to incorporate sample-size-dependent {critical values}, using the limiting non-central $\chi^2$ properties of the PRS to provide a principled framework for evaluating population shifts.

Other measures include the Kolmogorov-Smirnov (KS) statistic \citep{dagostino1986goodness}, $\mathrm{KS}=\max_{j=1,...,B} \big|\hat{F}(j)-F_0(j)\big|$, where $\hat{F}(j) = \sum_{i=1}^{j} \hat{p}_i$ is the empirical cdf and $F_0(j)=\sum_{i=1}^{j}p_{0i}$ is the cdf of the model construction population. However, the KS statistic has limited utility for discrete distributions, as its (asymptotic) distribution -- unlike in the continuous setting -- depends on the underlying $\bm{p}_0$ \citep{conover1972kolmogorov}, making critical values less straightforward to determine. {The \texttt{R} package \texttt{dgof} offers an implementation of the KS test tailored for the discrete settings \citep{RDGOF}.} The more recent statistic proposed by \citet{dupisanievisagie2020}, $\mathrm{DPV}=\max_{j=1,\ldots,B^\ast}|\hat{p}_j - p_{0j}| / p_{0j}$, relies on an arbitrary selection of $B^\ast < B$ and calibration through Monte Carlo methods. Its use has been explored only in settings with $n \geq 10,000$. For further reading on the discrete goodness-of-fit problem in general, see \citet{agresti2012categorical}.

\subsection{Population resemblance framework}
\label{sec:PR Framework}

To formalize the assessment of population shift, we introduce the concept of $\delta$-resemblance, which quantifies acceptable deviations between probability distributions and provide a structured framework for analyzing the behavior of the PRS.

\begin{definition}\label{deltaresemblance}
Let $\delta>0$. For the probability vector $\bm{p}_0$, define the region  
$$
\mathcal{P}(\delta|\bm{p}_0) = \Bigg\{\tilde{\bm{p}} = (\tilde{p}_1,\ldots,\tilde{p}_B) : \max_{j=1,\ldots,B} |p_{0j} - \tilde{p}_j| \leq \delta,\ \sum_{j=1}^{B}\tilde{p}_j = 1 \Bigg\}. 
$$
The probability vector $\bm{p}$ is said to be $\delta$-resemblant of $\bm{p}_0$ whenever $\bm{p}\in \mathcal{P}(\delta|\bm{p}_0)$.
\end{definition}

Intuitively, $\mathcal{P}(\delta|\bm{p}_0)$ defines the set of all valid probability vectors where no category probability $p_j$ deviates from its reference value, $p_{0j}$, by more than $\delta$. This tolerance-based approach ensures that deviations remain manageable and within acceptable limits, making it an ideal framework for practical applications.

The methodology in this paper relies on the concept of $\delta$-resemblance, as well as additional technical conditions to ensure accurate implementation. Specifically, we assume that the reference probabilities are strictly positive ($p_{0j} > 0$ for all $j = 1, \ldots, B$) to ensure that every category in the reference distribution is represented. Furthermore, we impose the constraint $0 < \delta \leq \min\limits_{j=1, \ldots, B} p_{0j}$, which guarantees that the parameter $\delta$ does not exceed the smallest reference probabilities. This condition effectively limits the maximum allowable shift for any single category to its own probability mass. Finally, $\delta$ is a pre-specified parameter reflecting the organization's risk tolerance (i.e., the acceptable deviation from the current model), chosen to ensure consistency in decision-making rather than being estimated from the data. However, it is designed to scale with sample size, adapting to variability across different data sets.

{The formulation of Definition \ref{deltaresemblance} leverages the Chebyshev distance to quantify the maximum deviation in any single category’s probability. The Chebyshev distance is defined as the largest absolute difference between corresponding elements of two vectors. This distance is particularly advantageous in risk-sensitive applications due to its direct interpretability. By bounding the maximum allowable deviation in any category, the Chebyshev distance provides a precise and actionable metric for monitoring changes in probability distributions.}

{The Chebyshev distance aligns with the operational need to control critical deviations, as it isolates the largest shift in the data. This characteristic is especially important in credit risk monitoring, where changes in high-risk categories may have outsized implications. Unlike alternatives such as the Euclidean distance, which averages deviations across categories, the Chebyshev distance focuses on the worst-case deviation, offering sharper insights into localized shifts.}

In a credit risk management application, consider the portfolio-level probability of default (PD), computed as
$
PD(\bm{p}) = \sum_{i=1}^B n_i p_i$,
where $n_i$ is the number of borrowers in risk grade $i$, and $p_i$ represents the default rate for that grade. When two populations, $\bm{p}$ and $\bm{p}’$, are $\delta$-resemblant, the difference in their portfolio-level PD is bounded by
$
|PD(\bm{p}) - PD(\bm{p}’)| \leq \delta \sum_{i=1}^B n_i.
$
This result demonstrates that the $\delta$-resemblance framework combines a rigorous statistical foundation with practical relevance, enabling risk managers to quantify the population shift’s impact on portfolio performance, assess acceptable levels of deviation aligned with business risk tolerance, and decide when model recalibration is warranted.

{We also note that a two-sample formulation of the problem, i.e. treating $\bm{p}_0$ as obtained through random sampling from the model construction population, may have some appeal. In practice, however, this approach faces several challenges including dependence between the model construction and current sample, which may stem from at least temporal evolution and overlapping data. To address the issue of non-independence between the samples, we have thus resorted to a one-sample problem formulation, treating the model construction probabilities $\bm{p}_0$ as fixed and known. Even in situations as described where the model construction probabilities were established using sampling tools, the one-sample approach can still be employed, providing a conditional inference solution.} 

The statistical properties of the PSI and the proposed PRS under this framework are explored in Section \ref{sec: PRS properties}, encompassing both the case of population equality, $\bm{p} = \bm{p}_0$, and the more flexible scenario of $\delta$-resemblance, $\bm{p}\in\mathcal{P}(\delta|\bm{p}_0)$.

\section{Statistical properties of Population Resemblance}
\label{sec: PRS properties}

\subsection{Small sample behavior}
\label{sec:small_sample}

{In this subsection, we investigate the finite sample properties of the PSI and PRS, examining both scenarios where the current population $\bm{p}$ matches and departs from the reference population $\bm{p}_0$ used in model construction. Critically, $\bm{p}_0$ is assumed to be fixed and known throughout the paper.} Define the scaled statistics $T_n = n\times \mathrm{PSI}$ and $Q_n = n\times \mathrm{PRS}$ where $\mathrm{PSI}$ and $\mathrm{PRS}$ are given in \eqref{eq:PSI} and \eqref{eq:PRS}, respectively. Both of these are known to follow a (central) $\chi^2$ distribution with $B-1$ degrees of freedom under the assumption of no population shift, $\bm{p}=\bm{p}_0$, as the sample size increases. However, in smaller samples, their behavior can deviate significantly from asymptotic expectations, making it important to understand these differences for practical applications.

Despite the well-established asymptotic distribution of $T_n$ \citep[Chapter~6]{kullback1978information}, practitioners often rely on the thresholds of \citet{lewis1994introduction}, which prescribe $\mathrm{PSI} < 0.1$ to indicate acceptable population similarity and $\mathrm{PSI} \geq 0.25$ as a trigger for model reconstruction. These thresholds do not account for critical factors such as sample size or the number of categories, which can substantially affect the behavior and interpretation of the PSI in finite-sample settings.

{To illustrate the limitations of these fixed thresholds, we conducted a simulation study examining the probability of mandating model reconstruction under two scenarios:
\begin{enumerate}
\item No population shift: $\bm{p} = \bm{p}_0$
\item Moderate shift: $\bm{p}$ differs from $\bm{p}_0$ such that $J(\bm{p}, \bm{p}_0) = 0.1$, where $\bm{p}$ was determined by minimizing the Euclidean distance between $\bm{p}$ and $\bm{p}_0$ subject to a Lagrange multiplier constraint and the simplex constraint.
\end{enumerate}
Both scenarios assume equal model construction probabilities $\bm{p}_0 = (1/B, \ldots, 1/B)$. The probabilities $\hat{\bm{p}}$ used for calculating the $\mathrm{PSI}$ were obtained by scaling multinomial counts simulated with $n$ ranging from 50 to 500. Table \ref{TABLE1} presents the estimated probabilities of mandating reconstruction, $\widehat{P}(\mathrm{PSI} \geq 0.25)$, where $\widehat{P}$ represents the estimated probability based on $K = 10^6$ simulated datasets.}

\begin{table}[htbp]
    \centering
    \caption{Estimated probabilities of mandating \\reconstruction under fixed PSI thresholds.}
    \label{TABLE1}
    \footnotesize
    \begin{tabular}{p{1cm}p{2cm}p{2cm}p{2cm}p{2cm}}
        \toprule
         & \multicolumn{2}{c}{$J = 0$} & \multicolumn{2}{c}{$J = 0.1$} \\
        \cmidrule(lr){2-3} \cmidrule(lr){4-5}
        $n$& $B = 5$ & $B = 10$ & $B = 5$ & $B = 10$ \\
        \cmidrule(lr){1-5} 
        50  & 0.0226 & 0.2356 & 0.2542 & 0.5459 \\
        100 & 0.0001 & 0.0086 & 0.0872 & 0.2508 \\
        200 & 0.0000 & 0.0000 & 0.0131 & 0.0434 \\
        500 & 0.0000 & 0.0000 & 0.0001 & 0.0003 \\
 \bottomrule      
    \end{tabular}
\end{table}

The simulation results reveal two critical issues with fixed PSI thresholds: For small samples ($n = 50$), reconstruction is mandated too frequently, even under no shift. For large samples ($n = 500$), reconstruction is rarely triggered, even with moderate shifts.

To better understand these behaviors, we examine the mean and variance stability of the statistics. For $T_n$, define the stability ratios,
$$
\Lambda^{(1)}_{T_n} = \frac{\mathrm{E}[T_n]}{B-1} \qquad \text{and} \qquad \Lambda^{(2)}_{T_n} = \frac{\mathrm{Var}[T_n]}{2(B-1)}.
$$
The corresponding quantities for $Q_n$ are defined similarly. These ratios measure convergence to asymptotic moments, with values near 1 indicating stability. Using $K = 10^6$ Monte Carlo realizations for sample sizes from 20 to 1,000, we estimate these ratios for $B=5$ categories. Figures \ref{fig:mean_stab_ratios} and \ref{fig:var_stab_ratios} display the results.
% Figure environment removed



The PRS ($Q_n$) exhibits mean stability even at small sample sizes and variance stability beyond $n = 50$. In contrast, the PSI ($T_n$) shows substantial instability, with empirical moments deviating from asymptotic values by up to 8\% (mean) and 30\% (variance) in small samples. While close empirical and asymptotic moments do not guarantee convergence, large discrepancies signal lack of asymptotic convergence by the specific sample size.

\subsection{Non-central chi-square limiting distribution}
\label{sec:limiting}

Having established the superior small-sample properties of the PRS under no population shift in comparison to the PSI, we now examine its behavior under population shift, where the current population $\bm{p}$ may differ from $\bm{p}_0$ by up to a specified tolerance $\delta$. Recall that $Q_n = n\times \mathrm{PRS}$ denotes the sample-size normalized PRS statistic. 

Now, for $j = 1,\ldots,B$, define category-specific deviations $\delta_j = p_j - p_{0j}$. We assume these deviations are small, specifically of the order $n^{-1/2}$; that is, $p_j-p_{0j}=O(n^{-1/2})$. Formally, this implies the existence of finite constants $\xi_j$ such that $\lim_{n\rightarrow\infty} n^{1/2}\delta_j = \xi_j.$ In this framework, the true current probabilities are written as $p_j = p_{0j} + \delta_j$. 

To evaluate the asymptotic distribution of $Q_n$ under these deviations, define $$Z_j = \frac{\sqrt{n}(\hat{p}_j-p_j)}{\sqrt{p_j}}, \ j=1,\ldots,B.$$ Rewriting $Q_n$ in terms of the $Z_j$ yields $$Q_n = \sum_{j=1}^{B}\left\{\sqrt{\frac{p_j}{p_{0j}}}\,Z_j + \frac{\sqrt{n}(p_j-p_{0j})}{\sqrt{p_{0j}}}\,\right\}^2.$$
This formulation makes explicit how deviations from the reference distribution $\bm{p}_0$ influence the large-sample behavior of the statistic. Under multinomial sampling, the $Z_j$ are asymptotically standard normal. Combined with the assumption that $p_j-p_{0j}=O(n^{-1/2})$, we have $\mathrm{Var}(\sqrt{p_j/p_{0j}}\,Z_j) = 1+O(n^{-1/2})$. Consequently, $Q_n$ can be shown to converge to a non-central chi-square distribution with $B-1$ degrees of freedom and a finite non-centrality parameter. The explicit form of this non-centrality parameter in our context is given in Proposition 1; further details of this convergence can be found in \citet{cressie1984multinomial}.

\begin{proposition}
Under the assumption that $p_j = p_{0j} + \delta_j = O(n^{-1/2})$, $Q_n$ converges in distribution as $n \to \infty$ to a non-central chi-square distribution with $B-1$ degrees of freedom,
\[
Q_n \overset{d}{\longrightarrow} \chi^2_{B-1}(\lambda),
\]
where the non-centrality parameter $\lambda$ is given by
\[
\lambda = \sum_{j=1}^B \frac{n\delta_j^2}{p_{0j}} = \sum_{j=1}^B \frac{n(p_j - p_{0j})^2}{p_{0j}}.
\]
\end{proposition}

Since the true values of $p_j$ are typically unknown, the non-centrality parameter $\lambda$ cannot be calculated. Additionally, care is needed when interpreting $\lambda$, as the formula suggests a dependence on the sample size $n$. However, under the assumption that $p_j - p_{0j}$ decreases at a rate of $n^{-1/2}$, this dependence is offset, ensuring that $\lambda$ remains well-defined in the asymptotic framework.

To address the challenge of $\lambda$ being unknown, we adopt a conservative approach by framing the problem in terms of the maximal non-centrality parameter under $\delta$-resemblance. This aligns with the concept of least favorable distributions, where test statistics are evaluated under the worst-case scenario within the null hypothesis, see \citet{reinhardt1961use}. This approach allows us to construct robust decision-making critical values using the supremum of $\lambda$, defined as  
\[
\lambda_{\mathrm{sup}} = \sup_{\bm{p} \in \mathcal{P}(\delta | \bm{p}_0)} \sum_{j=1}^B \frac{n\,(p_j - p_{0j})^2}{p_{0j}}.
\]

A crucial property of the non-central chi-square distribution underpins this framework. Specifically, the distribution satisfies a stochastic ordering property: for $\lambda \leq \lambda_{\mathrm{sup}}$ with fixed degrees of freedom $B-1$,
\[
\chi^2_{B-1}(\lambda_{\mathrm{sup}}) \preceq_{\mathrm{st}} \chi^2_{B-1}(\lambda),
\]
where $\preceq_{\mathrm{st}}$ denotes stochastic dominance. This means that for random variables $X \sim \chi^2_{B-1}(\lambda_{\mathrm{sup}})$ and $Y \sim \chi^2_{B-1}(\lambda)$, the cumulative distribution functions satisfy $F_Y(x) \geq F_X(x) \text{ for all } x \in \mathbb{R}.$ This property ensures that the maximal non-centrality parameter $\lambda_{\mathrm{sup}}$ serves as a conservative basis for decision-making, enabling robust inference despite uncertainty about the true values of $\bm{p}$.

In the next proposition, we establish how the maximal non-centrality parameter $ \lambda_{\mathrm{sup}} $ depends on the baseline probabilities $ \bm{p}_0 $ and the partition size $ B $, providing a precise characterization of $ \lambda_{\mathrm{sup}} $ in terms of these parameters.

\begin{proposition}
Under a constraint of $\bm{p}\in \mathcal{P}(\delta|\bm{p}_0)$, the maximal non-centrality parameter is
\[
\lambda_{\mathrm{sup}} =
\begin{cases}
    n\delta^2 \displaystyle\sum_{j=1}^B p_{0j}^{-1}, & \text{if } B \text{ is even,} \\
    n\delta^2 \left( \displaystyle\sum_{j=1}^B p_{0j}^{-1} - p_{\ast}^{-1} \right), & \text{if } B \text{ is odd,}
\end{cases}
\]
where $p_{\ast} = \max\limits_{j=1,\ldots,B} p_{0j}$.
\end{proposition}

\begin{proof}
To derive the maximal non-centrality parameter $ \lambda_{\mathrm{sup}} $, we consider the supremum of $ \lambda $ under the constraint of $ \delta $-resemblance, $ \bm{p} \in \mathcal{P}(\delta \mid \bm{p}_0) $. The convexity of $ \lambda $ as a function of $ \bm{p} $ ensures that its maximum is attained at an extreme point of the feasible set $ \mathcal{P}(\delta \mid \bm{p}_0) $. Translating $ \mathcal{P}(\delta \mid \bm{p}_0) $ by $ \bm{p}_0 $, this set becomes the intersection of a $ \delta $-scaled $ \ell_\infty $-ball (hypercube) and a co-dimension 1 subspace orthogonal to the main diagonal. Each extreme point of this set has coordinates $ p_j \in \{ p_{0j} - \delta,\, p_{0j},\, p_{0j} + \delta \} $, with at most one $ p_j $ remaining at $ p_{0j} $, and with the number of $ +\delta $ and $ -\delta $ deviations being equal. Two distinct cases need to be considered.

\textit{Case 1}: For $ B $ even, no coordinate remains unperturbed, meaning all components $ p_j $ take values $ p_{0j} \pm \delta $. Consequently, all extreme points yield the same value of $ \lambda $, and we find
\[
\lambda_{\mathrm{sup}} = n\delta^2 \sum_{j=1}^B p_{0j}^{-1}.
\]

\textit{Case 2}: For $ B $ odd, symmetry requires that one coordinate $ p_j $ remains unperturbed ($ p_j = p_{0j} $). To maximize $ \lambda $, this zero increment is assigned to the index $ j $ corresponding to the largest $ p_{0j} $, as this minimizes the term $ 1 / p_{0j} $. Substituting this condition, the supremum becomes
\[
\lambda_{\mathrm{sup}} = n\delta^2 \left( \sum_{j=1}^B p_{0j}^{-1} - p_{\ast}^{-1} \right),
\]
where $ p_{\ast} = \max\limits_{j=1,\ldots,B} p_{0j} $.

Thus, the explicit form of $ \lambda_{\mathrm{sup}} $ is established for both even and odd $ B $, completing the proof.
\end{proof}

\subsection{Decision-making framework for population monitoring}
\label{sec:framework}

To implement the PRS for model monitoring, we propose a three-region decision-making framework that aligns with the expectations among risk management practitioners, as reflected by the Lewis constants for the PSI. This framework can, in principle, be extended to accommodate more than three regions, allowing for finer granularity in monitoring. It is parameterized by the risk tolerance  $\delta > 0$, representing the acceptable level of deviation from the model construction probabilities, and a multiplier  $M > 1$ such that $M\delta$ defines the boundary for full discrepancy (i.e., unacceptable level of deviation).

{In addition, two decision sensitivity parameters, $\alpha_1$ and $\alpha_2$, control how readily the procedure transitions between the three decision regions. Specifically, $\alpha_1$ controls the sensitivity to model reconstruction, reflecting the likelihood of classifying a population as fully discrepant when it is still $\delta$-resemblant. Similarly, $\alpha_2$ controls the sensitivity to continued model use, indicating the likelihood of maintaining the current model when the population is at the boundary of $M\delta$-resemblance. Both serve as operational decision parameters, enabling practitioners to calibrate the framework based on their desired balance between stickiness\footnote{The term ``stickiness'' is used here to denote the model’s resistance to change, which might otherwise be described as ``stability.'' However, we avoid ``stability'' to prevent confusion with population stability as used throughout this paper, where it has a distinct technical meaning.} (the tendency to favor model continuity) and responsiveness (the ability to quickly detect and react to population shifts). These parameters determine the decision regions for the PRS framework as follows:
\begin{definition}
The decision regions for the PRS framework are defined as follows, with $0\leq \alpha_1, \alpha_2 \leq 1$ corresponding to the level of risk aversion for erroneous decisions,
\begin{align*}
\mathcal{R}_1 &= \left\{\mathrm{PRS} \leq \tau_1 \right\} 
&&\text{(Continue using model, ``acceptable'')}\\
\mathcal{R}_2 &= \left\{\tau_1 < \mathrm{PRS} \leq\tau_2\right\} 
&&\text{(Enhanced monitoring, ``partially discrepant'')}\\
\mathcal{R}_3 &= \left\{\mathrm{PRS} > \tau_2 \right\}  && \text{(Reconstruct model, ``fully discrepant'')}
\end{align*}
where the region boundaries, or ``critical values'', are $$\tau_1:=\frac{F_{B-1}^{-1}(\alpha_2,\,M^2\lambda_{\mathrm{sup}})}{n} \ \ \ \text{and} \ \ \ \tau_2:=\frac{F_{B-1}^{-1}(1-\alpha_1,\,\lambda_{\mathrm{sup}})}{n}.$$ Herein, $F_{B-1}^{-1}(\alpha,\,\lambda)$ denotes the inverse cdf, i.e., the $100\alpha$-th percentile, of the non-central $\chi^2$-distribution with degrees of freedom $B-1$ and non-centrality parameter $\lambda$. 
\end{definition}

These decision regions are schematically illustrated in Figure \ref{fig:three_regions} that depicts the PRS density curves derived from the non-central $\chi^2$ distribution. Tail regions corresponding to $\alpha_1$ and $\alpha_2$ are highlighted.

% Figure environment removed

Practitioners should carefully select parameters $\alpha_1 $ and $ \alpha_2$, as excessively large values may cause the intermediate monitoring region $\mathcal{R}_2$ to shrink or disappear entirely if the decision boundaries overlap. Our findings suggest that values in the range $[0.01,0.2]$ generally yield stable results, though a trade-off may be necessary if {greater sensitivity} is required in one direction or the other. Furthermore, setting excessively large $ M $ can distort the framework by shifting the {critical values} too far apart, undermining the reliability of monitoring decisions.

As noted by a reviewer, the decision framework can be interpreted in a hypothesis-testing framework, where $H_{01} : \bm{p} \in \mathcal{P}(\delta\, |\, \bm{p}_0)$ and $H_{02} : \bm{p} \in \mathcal{P}(M\delta\, |\, \bm{p}_0)$ assert that the population is $\delta$-resemblant or $M\delta$-resemblant, respectively. Since $M>1$, the set of $\delta$-resemblant distributions is strictly contained within the set of $M\delta$-resemblant distributions, making $H_{01}$ nested within $H_{02}$. This structure enables a tiered decision-making process: rejecting $H_{01}$ but not $H_{02}$ places the model in the partially discrepant monitoring region $\mathcal{R}_2$, signaling a deviation beyond $\delta$ that does not yet exceed $M\delta$. This allows for increased scrutiny without immediate model reconstruction. In contrast, rejecting both hypotheses indicates a more substantial shift, necessitating full model reconstruction. The approach of using two null hypotheses for the same parameter(s) is uncommon but not new; for example, \citet{schuirmann1987comparison} applies a related framework in bioequivalence testing.

Within this framework, $\alpha_1$ serves as a Type I error rate under $H_{01}$, determining the likelihood of rejecting both $H_{01}$ and $H_{02}$ when the population is (only) $\delta$-resemblant. This corresponds to incorrectly concluding that the population is fully discrepant, leading to model reconstruction. As such, $\alpha_1$ defines the decision boundary for crossing from $\mathcal{R}_2$ to $\mathcal{R}_3$ and governs the corresponding upper critical value $\tau_2$ in Definition 2. 

Conversely, $1-\alpha_2$ is the likelihood of exiting $\mathcal{R}_1$ when the population is at the boundary of $M\delta$-resemblance. While this resembles a detection of power, it is still defined within the null hypothesis framework of $H_{02}$ rather than, as conventional power, against an alternative hypothesis. Instead, $\alpha_2$ controls the likelihood of maintaining the current model despite a potential shift of magnitude $M\delta$. Thus, it defines the decision boundary for crossing from $\mathcal{R}_1$ to $\mathcal{R}_2$, i.e. the lower critical value $\tau_1$ in Definition 2. 

This structure, informed by the stochastic dominance of the non-central chi-square distribution, upholds a conservative decision-making process while allowing practitioners to fine-tune decision sensitivity and maintain clear boundaries for monitoring and intervention.

\subsection{Implementation guide}
\label{sec:implementation}

To implement the PRS framework in practice, a principled approach is essential, particularly in parameter selection. This section offers practical guidance for risk managers and analysts to ensure effective implementation across risk portfolios of varying sizes. 

As informed by the asymptotic theory, the risk tolerance parameter $\delta$ depends inherently on the sample size $n$ and cannot be chosen arbitrarily.  This parameter, $\delta$, which represents the maximum acceptable category-wise deviation, should be chosen inversely proportional to the sample size. Since the quantity $\{p_{0j}(1-p_{0j})/n\}^{1/2}$ corresponds to the standard error of a sample proportion, it provides a natural benchmark for determining meaningful deviations. We therefore propose defining $\delta$ as
\begin{equation}
\delta = c\ \times \min_{j=1,\ldots,B} \left\{ \frac{p_{0j}(1-p_{0j})}{n} \right\}^{1/2},
\label{eqn:rec_delta}
\end{equation}
where $c>0$ is a relative scaling factor that adjusts the acceptable magnitude of shift in relation to this standard-error-like quantity. Analogous to the number of standard deviations from the model construction probabilities, $c$ provides a clear interpretation of how much deviation is considered acceptable. This formulation ensures that $\delta$ scales appropriately with sample size and category-specific uncertainty, serving as a measure for determining when the original model is no longer acceptable for use.

For example, in a credit scoring model with probabilities $p_{0j}$ distributed across customer risk categories, $\delta$ ensures that no single category experiences significant shift -- such as an increase in high-risk customers -- large enough to compromise the confidence that the model can be expected to operate as intended. The PRS is calibrated to detect when one or more categories exhibit such levels of shift. By linking $\delta$ to sampling variability, this framework prevents substantive deviations across categories while maintaining a principled connection to sample size and category-specific uncertainty. 

While the sections that follow restrict performance demonstration to equi-probable $\bm{p}_0$ for simplicity, the PRS procedure applies to arbitrary reference distributions. Note, however, that while the PRS statistic is asymptotically distribution-free, the decision boundaries determined according to \eqref{eqn:rec_delta} depend on $\bm{p}_0$. The recommendation in \eqref{eqn:rec_delta} is one proposed solution to appropriately adjust for sample size. Nevertheless, universal choices remain possible -- one could always calculate $\delta$ using an equi-probable reference model, setting $\bm{p}_0 = (1/B, \ldots, 1/B)$ for the purpose of determining $\delta$, while remembering to use the true $\bm{p}_0$ when calculating the PRS statistic, if a predefined benchmark is preferred.

The multiplier $M>1$ must satisfy $M\delta \leq \min\limits_{j=1,\ldots,B} p_{0j}$ to ensure mathematical validity. The choice of $M$ depends on the context: smaller values are suitable for critical models requiring swift intervention, while larger values allow for more flexibility in settings where greater shifts can be tolerated before corrective action is needed. Setting excessively large $ M $ (relative to $c$) can distort the framework by shifting the {critical values} too far apart, undermining the reliability of monitoring decisions.

Practically, the PRS is calibrated so that shifts statistically equivalent to $n\delta$ cases per risk category signal a transition from acceptable to partial discrepancy. Similarly, shifts statistically equivalent to $nM\delta$ cases per risk category mark the shift from partial to full discrepancy. While $n\delta$ and 
$nM\delta$ provide easy reference points, actual deviations vary across categories -- some shifts less, others more -- necessitating a statistical procedure to assess overall resemblance. 

{The sensitivity parameters $\alpha_1$ and $\alpha_2$ should be chosen based on the organization's desired balance between stickiness and responsiveness, ensuring consistent control over decision boundaries across monitoring regions -- lower values prioritize model continuity while higher values lead to increased sensitivity to population shifts. Additionally, while $M$ and $c$ can be tailored to each portfolio being monitored, a more robust approach -- one that mitigates the risk of cherry-picking results -- is to fix these parameters across all portfolios, as $\delta$ already accounts for sample size considerations. This approach simplifies implementation while maintaining a systematic and objective monitoring framework.}

The PRS framework can be efficiently implemented using standard statistical software. Key computational steps include calculating $\delta$, determining $\lambda_{\mathrm{sup}}$ based on $B$, $n$, and $\delta$, and computing the decision-making {critical values} using non-central $\chi^2$ quantiles. The PRS is then compared to these {critical values} to assess model performance. {The diagram in Figure \ref{fig:flow_diag} illustrates the implementation flow of the PRS.}
\vspace{0.5cm}

\begin{center}		% Figure removed 
		\captionof{figure}{Flowchart depicting the implementation of the PRS method.}
        \label{fig:flow_diag}
\end{center}	


\section{PRS method competency: A simulation study}  
\label{sec:PRS simulation}

To evaluate the performance of the PRS procedure, we conducted a comprehensive simulation study using a Monte Carlo approach. This study assesses the ability of the PRS method to distinguish between acceptable and discrepant deviations with respect to an equi-probable reference distribution under various sample sizes $ n $, numbers of categories $ B $, and specified degrees of deviation, denoted $ \delta_v $. 

For each scenario, the reference probability distribution was defined as $ \bm{p}_0 = (1/B, \dots, 1/B) $, representing $ B $ equally likely categories. A deviation of magnitude $ \delta_v $ was introduced to define perturbed probability distributions $ \bm{p}_v $, representing the \textit{current model}, given by  
\[
p_{v,j} = 
\begin{cases} 
1/B - \delta_v, & j \leq B/2, \\
1/B + \delta_v, & j \geq B/2 + 1,
\end{cases}
\]
where, when $ B $ is odd, the central category remains unchanged at $ 1/B $.
Under this setting, the PRS tolerance level defined in \eqref{eqn:rec_delta} becomes $\delta = c B^{-1} \sqrt{(B-1)/n}$ with specified scaling constant $c$. The boundary for unacceptable deviation is set at $ M\delta $ for specified $ M > 1 $.  Using the decision-making framework of Definition 2, we establish the PRS {critical values} $\tau_1$ and $\tau_2$, delineating the decision regions $\mathcal{R}_1$, $\mathcal{R}_2$ and $\mathcal{R}_3$. 

We evaluated the PRS classification probabilities across 30 values of $ \delta_v $, ranging from no deviation, $\delta_v = 0$, to an extreme deviation, $ \delta_v = (3M+2) \delta $. This range captures shifts from within the acceptable range, where the population remains resemblant to the model, to larger deviations that are fully discrepant, indicating a substantial shift and necessitating model reconstruction.

For each configuration of $ (n, B, \delta_v) $, we simulated $ K = 10^5 $ independent multinomial samples,  $\bm{X}_k \sim \text{Multinomial}(n, \bm{p}_v), \quad k = 1, \dots, K.$ For each sample, we estimated the empirical probability distribution $ \hat{\bm{p}}_k $ and computed the PRS statistic using $ \bm{p}_0 $ as the reference. The proportion of simulations falling into each of the three PRS decision regions was recorded. 

Simulations were conducted across a range of values for $ (n,B) $, varying $ 0.5 \leq c \leq 1 $, $ 1.2 < M < 2 $, and $ 0.01 \leq \alpha_1, \alpha_2 \leq 0.2 $. For illustrative purposes, results are presented for $(n,B) = (50,5)$ under two scenarios, firstly $(c,M)=(0.7,2)$ and $(\alpha_1,\alpha_2)=(0.1,0.05)$ -- see Figure \ref{fig:class_probs_n50_v1} -- and, secondly, $(c,M)=(1,1.6)$ and $(\alpha_1,\alpha_2)=(0.1,0.2)$ -- see Figure \ref{fig:class_probs_n50_v2}. This is to illustrate the versatility of the method across parameter choices. Finally, for $(n,B)=(10,000,20)$ with, $ (c,M)=(0.7,2)$ and $(\alpha_1,\alpha_2)=(0.05,0.1)$, we show the results in Figure \ref{fig:class_probs_n10000}. The empirical classification probabilities were plotted against $ \delta_v $ to illustrate classification behavior across the full range of model deviations.
\clearpage
% Figure environment removed

% Figure environment removed


% Figure environment removed

The results demonstrate the effectiveness of the PRS method in identifying deviations that warrant model reconstruction. When $ \delta_v = \delta $, the probability of PRS falling above the upper {critical value} $ \tau_2 $, $P(\mathcal{R}_3)$, closely matches the expected value of $ \alpha_1 = 0.05 $. Similarly, under a deviation of size $ \delta_v = M\delta $, the probability of falling below $ \tau_1 $, $P(\mathcal{R}_1)$, remains near $ \alpha_2 = 0.1 $. These findings confirm the validity of the PRS procedure and its effectiveness in classifying multinomial samples under structured deviations from a reference model. 

\section{Banking applications and performance analysis}
\label{Application}


This section presents an empirical validation of the PRS based on a set of anonymized credit risk models using data from a large South African financial service provider. The dataset, spanning retail and corporate portfolios, was  selected to evaluate the PRS methodology across operationally relevant scenarios. Portfolio sizes have been rounded as part of the anonymization effort. The examples encompass multiple portfolio configurations varying in sample size ($n$) and number of risk categories ($B$) where using equi-probable risk categories, ${p_{0i}}=1/B, \ i=1,...,B$, was deemed appropriate. Results are reported in Tables \ref{tab: n=50 example} through \ref{tab:n=10000 B=20 example}; the commonly used red-amber-green ($\mathbbm{rag}$) status as used in banks was assigned ($\mathbbm{r}$ indicating membership to $\mathcal{R}_3$, $\mathbbm{a}$ to $\mathcal{R}_2$, and $\mathbbm{g}$ to $\mathcal{R}_1$). The designs considered are $(n,B)\in\{ (50,5),\, (500,10),\, (2,000,10),\, (10,000,20)\}$.

We follow the procedure of Figure \ref{fig:flow_diag} to implement the PRS, using the recommended $\delta$ from (\ref{eqn:rec_delta}) and choosing sensitivity parameters $\alpha_1=0.1$ and $\alpha_2=0.05$. These choices reflect risk aversion for an erroneous decision: $\alpha_1$ represents a $10\%$ probability of an $\mathcal{R}_3$ (fully discrepant) classification under $\delta$-resemblance, while $\alpha_2$ represents a $5\%$ probability of an $\mathcal{R}_1$ (acceptable) classification under $M\delta$-resemblance.

For these examples, we set $c=0.7$ and $M=2$, reflecting an institutionally acceptable level of risk tolerance. Conceptually, shifts statistically equivalent to $n\delta$ cases across all risk categories mark the transition from acceptable to partial discrepancy, while shifts statistically equivalent to $nM\delta$ cases across all risk categories mark the transition to full discrepancy. For $(n,B) = (50,5)$ and $(n,B) = (10,000,20)$, these values are $1.98 \approx 2$ and $15.26 \approx 15$ cases per category for the transition to partial discrepancy, and $3.96 \approx 4$ and $30.51 \approx 31$ cases per category for full discrepancy.


The critical values $\tau_1$ and $\tau_2$ delineating the regions $\mathcal{R}_1\ (\mathbbm{g}reen),\ \mathcal{R}_2\ (\mathbbm{a}mber)$ and $\mathcal{R}_3\ (\mathbbm{r}ed)$ are tabulated in Table \ref{tab:Critical values_PRS} along with the corresponding values of $\delta$. In all cases, $\delta$ (and $M\delta$) is smaller than $1/B$ ($= \min\limits_{j=1,\ldots,B} p_{0j}$), which is practically sensible.

We compare the outcomes of using the PRS with the widely used PSI, employing both the Lewis constants and the critical values proposed by \citet{yurdakul2020statistical}, denoted by ``L'' and ``YN'', respectively. Of course, a direct comparison of the PRS with the PSI (using YN critical values) is not entirely fair because the latter is governed by the specification of two type I errors under a null hypothesis of exact equality. Moreover, the PSI(YN) is known to have weak power at small sample sizes (Table 4 in \citet{yurdakul2020statistical}), a limitation of PSI(YN) and perhaps explainable by the instability of the normalized PSI in small samples observed in Figures \ref{fig:mean_stab_ratios} and \ref{fig:var_stab_ratios}. The YN-normed critical values are $\tau=2n^{-1}F^{-1}_{B-1}(1-\alpha)$ (in the notation of \citet{yurdakul2020statistical} and assuming $m=n$), where $F_{\nu}^{-1}$ denotes the chi-square quantile function with $\nu$ degrees of freedom. We choose upper and lower significance levels of $1\%$ and $10\%$, respectively (if $p<1\%$ then $\mathbbm{r}$, if $p>10\%$ then $\mathbbm{g}$, otherwise $\mathbbm{a}$).

A further comparison of the PRS is made with the discrete Kolmogorov-Smirnov (KS) test, although also not entirely fair given a null hypothesis of exact equality. Given the discrete nature of the KS statistic, particularly in small samples, it is not feasible to match critical values at specified significance levels and we rather report the p-value, $p(\mathrm{KS})$ from the \texttt{dgof}\footnote{The syntax used is \texttt{ks.test(X, ecdf(1:K), exact=F, simulate.p.value=T, B=10000)} where \texttt{X} is a numeric vector containing the repeated category number based on $\hat{p}$ and $B$ therein is the number of simulations and \texttt{K} the number of categories.} package in \texttt{R}, version 1.5.1 (2024/10/09), based on \citet{RDGOF}. We also used the upper and lower significance levels of $1\%$ and $10\%$, respectively. {Keep in mind that the significance levels $\alpha$ used for PSI(YN) and KS do not hold the same interpretation as $\alpha_1$ and $\alpha_2$ for the PRS.}


\begin{table}[h!]
\footnotesize
  \centering
  \caption{ Critical values for the PRS, specifying $c=0.7$ and $M=2$, and $\alpha_1=10\%$ and $\alpha_2=5\%$, for $\delta$ from (\ref{eqn:rec_delta}).}
 \begin{tabular}{p{2cm}p{1cm}p{1cm}p{1cm}}
    %\hline 
\toprule
\multicolumn{1}{c}{$(n,B)$} & \multicolumn{1}{c}{$\delta$} & \multicolumn{1}{c}{$\tau_1$} & \multicolumn{1}{c}{$\tau_2$}  \\
    %\hline
\cmidrule(lr){1-4}
$(50,5)$&0.056569&0.07441&0.25722\\
$(500,10)$&0.013416&0.03063&0.04890\\
$(2\ 000,10)$&0.006708&0.00766&0.01222\\
$(10\ 000,20)$&0.002179&0.00394&0.00439\\
    %\hline
\bottomrule
\end{tabular}
  \label{tab:Critical values_PRS}
\end{table}

\begin{table}[htbp]
\footnotesize
  \centering
  \caption{Population resemblance comparison using the current population of size $n=50$ and observed category sample sizes $n_i$, $i=1,...,B$ for $B=5$. For all categories, $n_{0i}=10$.}
     \begin{tabular}{p{0.5cm}p{3cm}p{2cm}p{2cm}p{2cm}}
    %\hline
\toprule
     & \multicolumn{1}{c}{${\mathbf{n_i}}$} & \multicolumn{1}{c}{PSI (L,YN)} & \multicolumn{1}{c}{PRS} & \multicolumn{1}{c}{p(KS)} \\
    %\hline
    \cmidrule(lr){1-5}
    $t_1$ & (6, 9, 10, 11, 14)& 0.072 ($\mathbbm{g,g}$) & 0.068 ($\mathbbm{g}$)& 0.401 ($\mathbbm{g}$)\\
    $t_2$ & (4, 10, 11, 11, 14)& 0.141  ($\mathbbm{a,g}$) & 0.108 ($\mathbbm{a}$)& 0.232 ($\mathbbm{g}$)\\
    $t_3$ & (7, 8, 8, 10, 17)& 0.114 ($\mathbbm{a,g}$) &  0.132 ($\mathbbm{a}$)& 0.125 ($\mathbbm{g}$)\\
    $t_4$ & (3, 8, 12, 13, 14)& 0.227  ($\mathbbm{a,g}$) & 0.164 ($\mathbbm{a}$)& 0.027 ($\mathbbm{a}$)\\
    $t_5$ & (2, 9, 12, 13, 14)& 0.310  ($\mathbbm{r,g}$) & 0.188 ($\mathbbm{a}$)& 0.028 ($\mathbbm{a}$)\\
    $t_6$ & (2, 5, 13, 14, 16)& 0.426  ($\mathbbm{r,a}$) & 0.300 ($\mathbbm{r}$)& $<0.001$ ($\mathbbm{r}$)\\
 
    %\hline
 \bottomrule
    \end{tabular}
  

  \label{tab: n=50 example}
\end{table}

\begin{table}[htbp!]
\footnotesize
  \centering
  \caption{Population resemblance comparison using the current population of size $n=500$ and observed category sample sizes $n_i$, $i=1,...,B$, $B=10$ and $n_{0i}=50$.}
     \begin{tabular}{p{0.5cm}p{6cm}p{2cm}p{2cm}p{2cm}}
    %\hline
\toprule
     & \multicolumn{1}{c}{${\mathbf{n_i}}$} & \multicolumn{1}{c}{PSI (L,YN)} & \multicolumn{1}{c}{PRS} & \multicolumn{1}{c}{p(KS)} \\
    %\hline
    \cmidrule(lr){1-5}

    $t_1$ & \Centerstack{(35, 40, 45, 45, 47, 50, 55, 58, 60, 65)} & 0.032 ($\mathbbm{g,g}$) & 0.032 ($\mathbbm{a}$)& 0.002 ($\mathbbm{r}$)\\
    $t_2$ & \Centerstack{(40, 45, 45, 45, 47, 48, 55, 55, 60, 60)} & 0.017 ($\mathbbm{g,g}$) & 0.018 ($\mathbbm{g}$)& 0.024 ($\mathbbm{a}$)\\
    $t_3$ & \Centerstack{(35, 36, 42, 43, 44, 44, 60, 60, 61, 75)} & 0.060 ($\mathbbm{g,a}$) & 0.062 ($\mathbbm{r}$)& $<0.001$ ($\mathbbm{r}$)\\
    $t_4$ & \Centerstack{(20, 35, 35, 40, 40, 62, 65, 65, 65, 73)} & 0.131 ($\mathbbm{a,r}$)& 0.116 ($\mathbbm{r}$)& $<0.001$ ($\mathbbm{r}$)\\
    %\hline
    %\hline
\bottomrule
    \end{tabular}
  \label{tab:n=500,2000 example}
\end{table}

\begin{table}[htbp!]
\footnotesize
  \centering
  \caption{Population resemblance comparison using the current population of size $n=2,000$ and observed category sample sizes $n_i$, $i=1,...,B$, $B=10$ and $n_{0i}=200$.}
     \begin{tabular}{p{0.2cm}p{7cm}p{2cm}p{2cm}p{2cm}}
    %\hline
\toprule
     & \multicolumn{1}{c}{${\mathbf{n_i}}$} & \multicolumn{1}{c}{PSI (L,YN)} & \multicolumn{1}{c}{PRS} & \multicolumn{1}{c}{p(KS)} \\
    %\hline
    \cmidrule(lr){1-5}

    $t_1$ & \Centerstack{(160, 170, 180, 180, 190, 200, 210, 220, 240, 250)} & 0.020 ($\mathbbm{g,a}$) & 0.020 ($\mathbbm{r}$)& $<0.001$ ($\mathbbm{r}$)\\
    $t_2$ & \Centerstack{(180, 180, 184, 190, 194, 200, 200, 210, 222, 240)} & 0.008 ($\mathbbm{g,g}$) & 0.008 ($\mathbbm{a}$)& 0.004 ($\mathbbm{r}$)\\
    $t_3$ & \Centerstack{(180, 180, 190, 194, 200, 200, 204, 210, 220, 222)} & 0.005 ($\mathbbm{g,g}$) & 0.005 ($\mathbbm{g}$)& 0.035 ($\mathbbm{a}$)\\
    $t_4$ & \Centerstack{(160, 170, 170, 178, 180, 210, 210, 220, 242, 260)} & 0.025 ($\mathbbm{g,r}$) & 0.026 ($\mathbbm{r}$)& $<0.001$ ($\mathbbm{r}$)\\

    %\hline
\bottomrule
    \end{tabular}
  \label{tab:n=2000 example}
\end{table}


\begin{table}[h!]
\footnotesize
  \centering
  \caption{Population resemblance comparison using the current population of size $n=10,000$ and observed category sample sizes $n_i$, $i=1,...,B$, $B=20$ and $n_{0i}=500$.}
     \begin{tabular}{p{0.2cm}p{8cm}p{2cm}p{2cm}}
    %\hline
\toprule
     & \multicolumn{1}{c}{${\mathbf{n_i}}$} & \multicolumn{1}{c}{PSI (L,YN)} & \multicolumn{1}{c}{PRS}  \\
    %\hline
    \cmidrule(lr){1-4}

    $t_1$ & 
    \Centerstack{(425, 455, 480, 480, 480, 480, 485, 491, 495, 495,\\500, 502, 502, 502, 502, 520, 540, 546, 550, 570)} & 0.0042 ($\mathbbm{g,g}$) & 0.0042 ($\mathbbm{a}$)\\ \hdashline
    $t_2$ & \Centerstack{(150, 170, 400, 400, 450, 450, 460, 460, 525, 525,\\545, 545, 550, 550, 600, 620, 650, 650, 650, 650)} & 0.1060 ($\mathbbm{a,r}$) & 0.0769 ($\mathbbm{r}$)\\ \hdashline
    $t_3$ & \Centerstack{(445, 455, 480, 480, 485, 485, 490, 495, 500, 500,\\501, 502, 502, 510, 510, 520, 520, 530, 540, 550)} & 0.0025 ($\mathbbm{g,g}$) & 0.0025 ($\mathbbm{g}$)\\ \hdashline
    $t_4$ & \Centerstack{(425, 425, 440, 440, 445, 445, 460, 460, 475, 475,\\490, 490, 525, 525, 555, 555, 585, 585, 600, 600)} & 0.0139 ($\mathbbm{g,r}$) & 0.0142 ($\mathbbm{r}$)\\ \hdashline
    $t_5$ & \Centerstack{(390, 390, 450, 450, 450, 450, 460, 460, 475, 475,\\525, 525, 545, 545, 550, 550, 555, 555, 600, 600)} & 0.0153 ($\mathbbm{g,r}$) & 0.0150 ($\mathbbm{r}$)\\ \hdashline
    $t_6$ & \Centerstack{(440, 465, 465, 475, 475, 480, 480, 485, 485, 488,\\490, 490, 510, 510, 520, 520, 550, 550, 550, 575)} & 0.0045 ($\mathbbm{g,g}$) & 0.0045 ($\mathbbm{r}$)\\
       \cmidrule(lr){1-4}
        \multicolumn{4}{l}{Note: In all of $t_1,\dots,t_6$, $p(KS)<0.001$ ($\mathbbm{r}$).}\\
\bottomrule
    \end{tabular}  \label{tab:n=10000 B=20 example}
\end{table}


When comparing the PSI(L) with the PRS, the results are commensurate with the conclusions from Table \ref{TABLE1}: the PRS procedure less frequently indicates ``\textit{full discrepancy}'' in small samples ($n=50$), while doing so more frequently in larger samples ($n\geq 500$). The same conclusion holds when comparing the PSI(L) with the PSI(YN). Recall that the PSI(L) has an inflated {probability of indicating a shift when none has occurred} in small samples and, conversely, a too-close-to-zero {probability of indicating \textit{``full discrepancy''}} in large samples. Notably, in the cases where $n$ exceeds $500$, PSI(L) seems insensitive to \textit{discrepancy} (partial and full). Observe in all cases considered here, that the PRS indicates \textit{discrepancy} ($\mathbbm{r}$, or $\mathbbm{a}$) more frequently than the PSI(YN). 

If we considered an alternative setup for the methodology, say $c=0.9$, $M=1.5$ and $\alpha_1=0.2$, $\alpha_2=0.1$, the only differently assigned statuses (now $\mathbbm{g}$ i.s.o. $\mathbbm{a}$) would occur at $t_1$ of $(500,10)$, $t_2$ of $(n,B)=(2,000,10)$; as well as a now $\mathbbm{a}$ status instead of $\mathbbm{r}$ at $t_6$ of $(10,000,20)$. Here, the ``partial discrepancy'' ($\mathbbm{a}$) region is smaller, as evident from the choices of $M$ and larger values of $\alpha_1$ and $\alpha_2$. Nonetheless, there is a large degree of overlap between the two sets of results. 

A comparison with the discrete KS remains. As with the PSI(YN), the KS is based on a null hypothesis of exact equality, while the PRS includes a risk tolerance through $\delta$. The KS is also not distribution-free with respect to $\bm{p}_0$, a further drawback. As a first departure, observe Table \ref{tab: n=50 example}: the PRS is more likely to indicate \textit{discrepancy} ($\mathbbm{r}$, or $\mathbbm{a}$) than the KS. Further to this, it is well known that the power of the KS is small at small sample sizes (not unlike the PSI using YN critical values) possibly explaining the more frequent $\mathbbm{g}$ status where $n=50$, compared to the larger sample sizes. In all other cases (i.e. Tables \ref{tab:n=500,2000 example} and \ref{tab:n=10000 B=20 example}), the KS signals full discrepancy ($\mathbbm{r}$) almost always. This is a direct result of the substantial power of the KS at larger sample sizes, and by its design, that any shift away from the null will swiftly be detected. Perhaps, detecting {even the slightest} shifts so often might not be practically ideal to a risk practitioner. Utilizing the PRS in these cases allows for a range of detection capabilities (see case $t_1$ and $t_3$ of $(n,B)=(10,000,20)$ with an $\mathbbm{a}$ and $\mathbbm{g}$ status, respectively).

We conclude from this comparative real-world study that among the measures considered here, {the PRS is universally competent at a range of sample sizes, including small sample sizes. A clear advantage of the PRS over the PSI and KS is the inclusion of the concept of $\delta$-resemblance and the tuning parameters $c$ and $M$ allowing the practitioner to calibrate the procedure to align with their risk tolerance. The PRS clearly indicates \textit{discrepancy} sufficiently in smaller samples and often enough in larger samples.} Unlike the KS, the PRS is sensitive to ranges of shifts over multiple risk categories and has easily obtainable critical values that are unique in small sample sizes. 


\section{Conclusion}
\label{sec:Conclusion}

Monitoring for changes in the population underlying a developed model is a common practice, especially in credit risk modeling. Over the years, several measures -- most notably the PSI -- have been proposed. However, limitations in these methods have spurred research into alternative approaches for assessing population resemblance. Our contribution, the PRS, utilizing the Pearson chi-square statistic and non-central chi-square distribution to address these gaps. 

The main advantageous features are that the PRS accommodates sample-size dependent critical values and its explicit specification of risk tolerance. The PRS is statistically well-founded, performing reliably across a range of sample sizes, including in small samples. Unlike the discrete Kolmogorov-Smirnov test, the PRS is asymptotically distribution-free with respect to $\bm{p}_0$. Further, the PRS critical values are designed to account for acceptable levels of population shift, aligning with a composite null hypothesis framework. The risk tolerance $\delta$ explicitly incorporates the assumption limited population shift, distinguishing the PRS from the KS test and the PSI of Lewis and \citet{yurdakul2020statistical}, which do not accommodate this structured flexibility. The tolerance parameter is incorporated through the concept of $\delta$-resemblance, a convenient way to communication population shift to practitioners in using the concept to derive business outcomes. 

These competency characteristics of the PRS were demonstrated through Monte Carlo simulations and real-world applications. In the applications, the suitability of the PRS was showcased, measuring the resemblance between populations given both small sample sizes (often encountered in low default portfolios) and in larger samples frequently encountered in retail portfolios of a bank. We have clearly shown that the PRS indicates (partial and/or full) discrepancy sufficiently in smaller samples and often enough in larger samples. 

Future research could explore relaxing the assumption of fixed reference probabilities $\bm{p}_0$ by adopting a two-sample framework where $\bm{p}_0$ arises through random sampling. Another practical direction would be redefining the tolerance parameter $\delta$ to account for varying risk category costs, leading to a multivariate formulation that reflects their relative importance.

\section{Compliance with ethical standards}

All authors certify that they have no affiliations with or involvement in any organization or entity with any financial interest or non-financial interest in the subject matter or materials discussed in this manuscript. No funds, grants, or other support was received.


%\printbibliography
\bibliography{PSIBiblio.bib}
%\bibliography{PSIBiblio}
\end{document}
