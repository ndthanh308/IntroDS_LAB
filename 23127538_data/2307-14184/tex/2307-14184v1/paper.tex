%\documentclass[lineno]{jfm}
\documentclass[reprint,superscriptaddress,showpacs,preprintnumbers,amsmath,amssymb,onecolumn,notitlepage]{revtex4-1}
%\usepackage[font=small,labelfont=bf, justification=justified, format=plain]{caption}
%\usepackage{multicol}
%\usepackage{booktabs}
\usepackage[T1]{fontenc}
\usepackage[polish]{babel}
\usepackage[utf8]{inputenc}
\usepackage{epsfig}
\usepackage{color}
%\usepackage{braket}
\usepackage{graphicx}
\usepackage{subfigure}
\usepackage{amssymb,amsfonts,amsmath}
\usepackage{tikz}
\newcommand{\RB}[1]{\textcolor{red}{#1}}
\newcommand{\PP}[1]{\textcolor{magenta}{#1}}
\usepackage{graphicx}
\usepackage{bm}
%\usepackage{epstopdf,epsfig}
%\usepackage{newtxtext}
%\usepackage{newtxmath}
%\usepackage[square,sort,comma,numbers]{natbib}
%\usepackage[style=authoryear]{biblatex}
\usepackage{hyperref}
%\hypersetup{
%    colorlinks = true,
%    urlcolor   = blue,
%    citecolor  = black,
%}
%\newtheorem{lemma}{Lemma}
%\newtheorem{corollary}{Corollary}
%\newcommand{\RomanNumeralCaps}[1]
%\linenumbers


% {\MakeUppercase{\romannumeral #1}}

\begin{document}

\title{Lagrangian statistics of dense emulsions}

\author{Ivan Girotto}
\affiliation{The Abdus Salam, International Centre for Theoretical Physics, Trieste, Italy}
\affiliation{Department of Applied Physics, Eindhoven University of Technology, Eindhoven, The Netherlands}
\author{Andrea Scagliarini}
\affiliation{Institute for Applied Mathematics ``Mauro Picone'' (IAC), Italian National Research Council (CNR), Rome, Italy}
\author{Roberto Benzi}
\affiliation{Department of Physics and INFN, University of Tor Vergata, Rome, Italy}
\author{Federico Toschi}
\affiliation{Department of Applied Physics, Eindhoven University of Technology, Eindhoven, The Netherlands}
\affiliation{Institute for Applied Mathematics ``Mauro Picone'' (IAC), Italian National Research Council (CNR), Rome, Italy}


\begin{abstract}

\noindent The dynamics of dense
stabilized emulsions presents a rich phenomenology including
chaotic emulsification, non-Newtonian rheology and ageing dynamics at rest.
Macroscopic rheology results from the complex droplet microdynamics and, in turn,
droplet dynamics is influenced by macroscopic flows via the competing action of hydrodynamic and interfacial stresses,
giving rise to a complex tangle of elastoplastic effects, diffusion, breakups and coalescence events.
This tight multiscale coupling, together with the daunting challenge
of experimentally investigating droplets  under flow, hindered the
understanding of dense emulsions dynamics.
We present results from 3d numerical simulations of dense
stabilised emulsions, resolving the shape and dynamics of individual droplets,
along with the macroscopic flows. We investigate droplet dispersion statistics, measuring
probability density functions (PDF) of droplet displacements and velocities,
changing the concentration, in the stirred and 
ageing regimes. We provide the first measurements ever, in concentrated emulsions, of the relative droplet-droplet
separations PDF and of the droplet acceleration
PDF, which becomes strongly non-Gaussian as the volume
fraction is increased above the jamming point. Cooperative effects, arising when droplets are
in contact, are argued to be responsible of the anomalous superdiffusive behaviour of
the mean square displacement and of the pair separation at long times, in both the stirred and
in the ageing regimes. This superdiffusive behaviour is reflected in a non-Gaussian pair separation
PDF, whose analytical form is investigated, in the ageing regime, by means of theoretical
arguments.
This work paves the way to developing a connection between Lagrangian
dynamics and rheology in dense stabilized emulsions.  
 
\end{abstract}

\maketitle

\section{Introduction}\label{sec:intro}
Understanding the dynamics of dense suspensions of soft, athermal particles such as emulsions, foams or gels is crucial for
many natural and industrial processes~\citep{lar99,clementsfood,coussot}.
A key question concerns the connection between mechanisms occurring at the 
microstructure level with the macroscopic flow and rheological properties in these
systems~\citep{pit13,dollet14,bonn17,dijksman19}.  
For instance, irreversible topological rearrangements, corresponding to local yielding 
events, are known to be directly related to the inhomogeneous fluidisation of 
soft glassy materials~\citep{Goyon2008,BocquetPRL2009,Bouzid2015,Dollet2015,Fei2020}.
A clear comprehension of the relevant processes and time-scales characterizing the microdynamics relies on
tracking single material mesoconstituents
(droplets, bubbles, etc)~\citep{SquiresMason,Durian1991,MasonPRL1997,Durian1995,CipellettiFD2003,RuzickaPRL2004,CerbinoPRL2008}.
Highly packed emulsions/foams are typically characterized in simple flows (oscillatory Couette, Poiseuille, etc), 
or even at rest, in the ageing regime~\citep{MasonPRL1995,CipellettiFD2003,Ramos2001,Cipelletti2005,Li2019,GiavazziJPCM2021}.
Lagrangian studies of dispersions in complex, high
Reynolds number flows, on the other hand, are widely represented in the literature,
but in extremely diluted conditions~\citep{Toschi2009,Brandt2022}. The investigation of the microdynamics of
concentrated systems in complex flows, of relevance, e.g., for emulsification processes~\citep{Vankova1,Vankova2,Vankova3},
is, in fact, a formidable task due to the need to cope, at the same time, with the interface dynamics two-way-coupled
to the hydrodynamics and with the droplet/bubble tracking.
This is what we address here, namely the statistical Lagrangian
dynamics of droplets in dense emulsions subjected to chaotic flows. We remark that this is the first investigation of this kind.
We employ a mesoscopic numerical method recently developed
to simulate the hydrodynamics of immiscible fluid mixtures, stabilized against full phase separation.
In a previous contribution, we showed how, by means of a suitable combination of chaotic stirring and injection
of the disperse phase, it is possible to prepare a three dimensional dense emulsion, that was then rheologically characterized, evidencing its
yield stress character~\citep{girotto2022}.
In the present paper, we discuss and employ a tracking algorithm for the trajectories of individual droplets
to investigate the droplet dynamics, in both semi-diluted and highly concentrated conditions,
under stirring and during ageing.
We study the statistics of droplet velocities and accelerations, focusing on the detection of
non-Gaussian signatures and how they are related to the nature of
droplet-droplet interactions. 
We discuss single and pair droplet dispersion,
showing that at high volume fractions a superdiffusive behaviour is observed in both
the stirred and ageing regimes. For pair dispersion in the ageing regime we propose theoretical models that show good
agreement with the measurements. Let us underline that measurements of the droplet acceleration PDFs and of the droplet pair
dispersion in densely packed emulsions have not been addressed before. 
Remarkably, our results suggest that both non-Gaussian statistics and superdiffusion emerge as soon
as the volume fraction exceeds a value comparable with that of random close packing of spheres, to be considered a proxy of the jamming
point. Therefore, this phenomenology is likely to be ascribable to cooperative effects resulting from the complex elastoplastic dynamics of
the emulsion.
The paper is organized as follows. In section \ref{sec:methods} we present the numerical method and we provide an extensive introduction to the tracking algorithm.
The main results are reported in section \ref{sec:results}, organized in subsections relative to the stirred and ageing regimes.
Conclusions and perspectives are drawn in section \ref{sec:conclusions}.

\section{Methods}\label{sec:methods}

\subsection{Multicomponent emulsion modeling}\label{sec:numerical_models}
Our numerical model is based on a three-dimensional (3D) two-component
lattice Boltzmann method~\citep{BENZI1992145}
in the Shan-Chen formulation~\citep{PhysRevE.47.1815, PhysRevE.49.2941}.
The lattice Boltzmann equation for the discrete probability distribution functions, $f_{\sigma l}$, reads
(the time step is set to unity, $\Delta = 1$)
\begin{equation}
    f_{\sigma l}(\mathbf{x}+\mathbf{e}_l, t + 1) - f_{\sigma l}(\mathbf{x}, t) = 
    -\frac{1}{\tau_{\sigma}}\left(f_{\sigma l}(\mathbf{x}, t) - f_{\sigma l}^{\text{eq}}(\mathbf{x}, t) \right) + S^{\text{(tot)}}_{\sigma l}(\mathbf{x}, t)
\end{equation}
where the index $l$ runs over the discrete set of nineteen 3D lattice velocities ($D3Q19$ model) $\{\mathbf{e}_l\}$
($l=0,1,\dots,18$),
and $\sigma$ labels each of the two immiscible fluids,
conventionally indicated as $O$ and $W$ (for, e.g., 'oil' and 'water').
The equilibrium distribution function is given by 
the usual polynomial expansion of the Maxwell-Boltzmann distribution, valid in the limit of small fluid velocity, namely:
\begin{equation}
  f_{\sigma l}^{\text{eq}}(\mathbf{x}, t) = \rho_{\sigma}\omega_l\left(1 + \frac{\mathbf{e}_l\cdot\mathbf{u}}{c^2_s} + 
    \frac{(\mathbf{e}_l\cdot\mathbf{u})^2}{2c_s^4} - \frac{\mathbf{u}\cdot\mathbf{u}}{2c^2_s} \right)
\end{equation}
with $\omega_l$ being the usual set of suitably chosen weights so to maximise the algebraic
degree of precision in the computation of the hydrodynamic fields  
and $c_s=1/\sqrt{3}$ a characteristic molecular velocity (a constant of the model).
The hydrodynamical fields (densities and total momentum) can be computed out of the lattice probability density functions $f_{\sigma l}$ as
$\rho_{\sigma} = \sum_l f_{\sigma l}$ and
$\rho \mathbf{u} = \sum_{\sigma l} f_{\sigma l}\mathbf{e}_l$
(where $\rho = \sum_{\sigma} \rho_{\sigma}$ is the total fluid density).
The source term $S^{\text{(tot)}}_{\sigma l} = \omega_l \mathbf{e}_l \cdot \mathbf{F}^{\text{(tot)}}_{\sigma}/c_s^2$ stems from all the forces (internal and external) acting in the system,
$\mathbf{F}_{\sigma} = \mathbf{F}_{\sigma} + \mathbf{F}_{\sigma}^{\text{(ext)}}$. In particular, $\mathbf{F}_{\sigma}$, incorporates the two kinds of lattice interaction forces,
$\mathbf{F}_{\sigma} = \mathbf{F}_{\sigma}^{\text{(r)}} + \mathbf{F}_{\sigma}^{\text{(f)}}$ and $\mathbf{F}_{\sigma}^{\text{(r)}}$ is the standard Shan-Chen inter-species repulsion
of amplitude, which is responsible of phase separation, and reads $G_{\text{OW}}>0$:
\begin{equation}\label{eq:LBint1}
    \mathbf{F}_{\sigma}^{(\text{r})}(\mathbf{x}, t) = 
    -G_{\text{OW}}\rho_{\sigma}(\mathbf{x}, t)\sum_{l,\sigma\neq\bar{\sigma}}\omega_l \rho_{\bar{\sigma}}(\mathbf{x}+\mathbf{e}_l, t)\mathbf{e}_l.
\end{equation}
The second term, $\mathbf{F}_{\sigma}^{\text{(f)}}$, consists of a short range intra-species attraction, involving
only nearest-neighbouring sites ($\mathcal{I}_1$), and a long range self repulsion,
extended up to next-to-nearest-neighbours ($\mathcal{I}_2$)~\citep{doi:10.1063/1.3216105}, namely:
\begin{align}\label{eq:LBint2}
\begin{split}
  \mathbf{F}_{\sigma}^{\text{(f)}}(\mathbf{r}, t) = &-G_{\sigma\sigma,1}\rho_{\sigma}(\mathbf{x}, t)\sum_{l\in \mathcal{I}_1}\omega_l
  \rho_{\sigma}(\mathbf{x}+\mathbf{e}_l, t)\mathbf{e}_l  \\
  &-G_{\sigma\sigma,2}\rho_{\sigma}(\mathbf{x}, t)
  \sum_{l\in \mathcal{I}_1 \bigcup \mathcal{I}_2}p_l\rho_{\sigma}(\mathbf{x}+\mathbf{e}_l, t)\mathbf{e}_l,
\end{split}
\end{align}
where $G_{\text{OO},1}, G_{\text{WW},1} < 0$ and $G_{\text{OO},2}, G_{\text{WW},2} > 0$ and $p_l$ are the weights of
the $D3Q125$ model.
This type of repulsive interaction $G_{\sigma\sigma,2}$
represents a mesoscopic phenomenological modelling of surfactants and provides a mechanism of stabilisation against coalescence of close-to-contact droplets
(the superscript 'f' stands in fact for 'frustration'),
promoting the emergence of a positive disjoining pressure, $\Pi>0$, within the liquid film separating the approaching
interfaces~\citep{doi:10.1063/1.3216105, Benzi2010, C4SM00348A}.\\
The large-scale forcing needed to generate the chaotic stirring that mixes the two fluids enters into the model through $\mathbf{F}_{\sigma}^{\text{(ext)}}$,
which takes the following form:
\begin{equation}
  F^{\text{(ext)}}_{\sigma i}(\mathbf{x}, t) = A \rho_{\sigma}\sum_{j \neq i}\left[\sin(k_j r_j+\Phi_{k}^{(j)}(t))\right],
\label{eq:turbo}
\end{equation}
where $i,j=1,2,3$, $A$ is the forcing amplitude, $k_j$ are the wavevector components, and the sum is limited to $k^{2}=k_{1}^{2}+k_{2}^{2}+k_{3}^{2}\leq2$. 
The phases $\Phi_{k}^{(j)}$ are evolved in time according to independent Ornstein-Uhlenbeck processes with the same relaxation times $T=L/u_{\text{rms}}$,
where $L$ is the cubic box edge and $u_{\text{rms}}$ is a typical large-scale
velocity~\citep{Biferale_2011,doi:10.1063/1.4719144}.

\subsection{Droplet tracking}\label{sec:tracking}
We present now the tracking algorithm and discuss its implementation. The algorithm combines (i) the process of identification
of all droplets constituting the emulsion at two different and consecutive time steps (hereafter called {\it labeling}),
with (ii) a stage describing the kinematics of each droplet (the actual {\it tracking}).\\
In the labeling step, individual droplets, defined as connected clusters of lattice points such that the local density
exceeds a prescribed threshold (equal to $\rho_O^{(\text{max})}-\rho_O^{(\text{min})}$),
are identified by means of the Hoshen-Kopleman algorithm~\citep{PhysRevB.14.3438,FRIJTERS201592}.
This approach echoes what is known in the image processing jargon as
Connected Component Labeling~\citep{HE201725}; similar techniques have been recently applied to multiphase fluid
dynamics in Volume of Fluid simulations~\citep{Herrmann2010,HENDRICKSON2020104373}.\\  
The tracking is based on the computation of the probability to obtain volume transfer among droplets in space and time
as described in~\citep{10.1016/j.jcp.2022.111560,GAO2021103523,CHAN2021110156}.
This is performed as follows. 
Let us suppose that in the domain, at time $t_1$, there are $N_1$ droplets and at a later dump
$t_2 = t_1 + \Delta t$ there are $N_2$ droplets.
%We set the density field to $0$ value on all lattice points where there are no droplets, and
%to a value ($\geq{1}$) on all lattice sites inside a droplet.
We define a set of droplet indicator fields $\rho_k(x,t)$ (with $k$ running over the all droplets),
which are equal to $k$ if, at time $t$, $x$ is inside the $k$-droplet, 
and are equal to $0$ elsewhere. 
In the following, the ``state'' representing the droplet will be denoted in the ket notation
(reminiscent of quantum mechanics states)
as $|k,t\rangle$ (the state is assumed to be normalized by square root of the droplet volume, $\sqrt{V_k}$),
such that the transition probability for a droplet $k_1$ at a time $t_1$ to end up in a droplet
$k_2$ at a time $t_2$ is given by the bra-ket expression:
\begin{equation}\label{eq:tracking}
  P_{k_1\rightarrow k_2}=\langle k_2,t_2|k_1,t_1\rangle  = \frac{1}{V} \int \rho_{k_2}(\mathbf{x},t_2)\rho_{k_1}(\mathbf{x},t_1)d\mathbf{x}, \quad V=\sqrt{V_{k_1} V_{k_2}}
\end{equation}
This transition probability is equal to $1$ in the case droplets $k_1$ and $k_2$ perfectly overlap and it is zero if they do not overlap at all.
A high $P$ value gives us, therefore, the confidence in having re-identified the same droplet at two different time steps.
What happens if a droplet is not deforming and just translating with uniform velocity $\mathbf{v}$?
We expect that the probability will decrease due to an imperfect overlap between the droplet $k$ at time $t$ and the same droplet $k$, displaced
by $\mathbf{v}\Delta t$ at time $t+\Delta t$, where $\mathbf{v}$ is the average velocity of all grid points included into a droplet. Therefore, we expect that the maximal correlation will occur for:
\begin{equation}
\langle k,t+\Delta t|k,t\rangle={1\over V}\int \rho_k(\mathbf{x}+\mathbf{v}\Delta t,t)\rho_k(\mathbf{x},t+\Delta{t})d\mathbf{x}
\label{eq:tracking_2}
\end{equation}
Of course the amount of this effect is proportional to $\Delta t$. In order to reduce the effect of the translation
of the droplet $k$ at a given $\Delta t$
we implement a Kalman filter, evaluating the overlap against the predicted at the same
$t+\Delta t$ shifting the initial position at time $t$ forwards by $\mathbf{v}\Delta t$.
For all the data shown, the tracking is implemented with $\Delta t = 100$ lattice Boltzmann time steps.

\section{Results}\label{sec:results} 

% Figure environment removed
In this section we provide results aimed at characterizing the dynamics of individual droplets
(e.g., their velocities and accelerations, as well as absolute and relative dispersion), at changing the volume
fraction of the dispersed phase from $\phi=38\%$ to $\phi=77\%$. All simulations were performed on a cubic grid of side
$L=512$ lattice points, the kinematic viscosity was
$\nu=1/6$ for both components, (in lattice Boltzmann units; hereafter dimensional quantities will be all expressed
in such units) and the total density $\rho_f = 1.36$ (giving a dynamics viscosity $\eta = \rho_f \nu \approx 0.23$).
With reference to Fig.~\ref{fig:events_n_drops}, where we plot the temporal variation
of the number density of droplets $N_D/L^3$,
we give first a cursory description of the {/it emulsification} process, indicated as phase $(I)$ in the figure (further details can be found in~\citet{girotto2022}).
All simulations are run for a total of $9 \cdot 10^6$ time steps. 
Starting from an initial condition with $\phi=30\%$, where the two components are fully separated by a flat interface, 
the emulsion is created applying the large-scale stirring force, Eq.~(\ref{eq:turbo}), with magnitude $A=4.85\cdot 10^{-7}$,
while injecting the dispersed phase until the desired value is reached.
The duration of the injection phase, $t_{\text{inj}}$, depends, then, on the target volume fraction
(see Table~\ref{table:vol_frac_interface}). 
The forcing is applied up to $t_F=3 \cdot 10^6$. The evolution of the system is then monitored for further $6 \cdot 10^6$ time steps.
%(not shown in Fig.~\ref{fig:events_n_drops} since the number of droplets is essentially constant).
The tracking algorithm is activated at $t \geq t_0 = 1.25 \cdot 10^6$; for what we call, hereafter, {\it stirred regime} (phase (II) in Fig.~\ref{fig:events_n_drops})
we collect statistics over the interval $t \in [t_0,t_F]$ (which is statistically stationary, as it can be appreciated from the figure of $N_D(t)$, but also
looking at other observable, such as the mean square velocity), whereas, for the {\it ageing regime} (phase (III) in Fig.~\ref{fig:events_n_drops}),
we consider data in the interval $t \in [t_A^{(i)}, t_A^{(f)}]$, with $t_A^{(i)}= 7 \cdot 10^6$ and $t_A^{(f)}= 9 \cdot 10^6$,
(the intermediate relaxation phase, $t \in [t_F, t_A^{(i)}]$, is not shown in
Fig.~\ref{fig:events_n_drops}, for the sake of clarity of visualisation).
% Figure environment removed
In Fig.~\ref{fig:interface_volume_fraction_final} we show snapshots of the morphology of the emulsions at $t=t_F$, for different volume fractions.
Semi-diluted emulsions present a high number of small spherical droplets, whereas densely packed emulsions
are constituted of a smaller number of non-spherical larger droplets.
\begin{table*} 
  \centering
  \begin{tabular}{|c|c|c|c|c|c|c|c|c|c|c|c|c|c|}
 $\phi$ & $t_{\text{inj}}$ & $ \overline{N_D}^{(s)}$ & $U^{(s)}_{\text{rms}} $ & $a^{(s)}_{\text{rms}}$ & $T_L^{(s)}$ & $ \langle d \rangle $ & $d_{\text{rms}}$ & $t_G$ & $U^{(a)}_{\text{rms}} $ & $a^{(a)}_{\text{rms}}$  & $T_L^{(a)}$ \\\hline
    \toprule
 $\phi_1 = 38\%$ & $1.5 \cdot 10^5$ & $3300$ & $2.8 \cdot 10^{-2}$ & $4.6 \cdot 10^{-4}$ & $1.8 \cdot 10^4$ & $30$  & $4$ & $180$ & $2.0 \cdot 10^{-5}$ & $1.2 \cdot 10^{-7}$ & $2.6 \cdot 10^7$ \\
$\phi_2 = 49\%$ & $3 \cdot 10^5$ & $3100$ & $2.5 \cdot 10^{-2}$ & $3.6 \cdot 10^{-4}$ & $2.0 \cdot 10^4$ &  $34$  & $5$ & $97$ & $1.7 \cdot 10^{-5}$ & $7.7 \cdot 10^{-8}$ & $3.0 \cdot 10^7$ \\
$\phi_3 = 64\%$ & $4.6 \cdot 10^5$ & $2400$ & $2.0 \cdot 10^{-2}$ & $2.6 \cdot 10^{-4}$ & $2.6 \cdot 10^4$ & $40$  & $6$ & $80$ & $5.1 \cdot 10^{-5}$ & $2.6 \cdot 10^{-7}$ & $10^7$ \\
$\phi_4 = 70\%$ & $5.2 \cdot 10^5$ & $1600$ & $1.8 \cdot 10^{-2}$ & $2.2 \cdot 10^{-4}$ & $2.8 \cdot 10^4$ & $48$  & $8$ & $1.1$ & $6.0 \cdot 10^{-5}$ & $4.2 \cdot 10^{-7}$ & $8.5 \cdot 10^6$\\
$\phi_5 = 77\%$ & $5.8 \cdot 10^5$ &  $800$ & $1.7 \cdot 10^{-2}$ & $2.1 \cdot 10^{-4}$ & $3.0 \cdot 10^4$ & $60$  & $17$ & $0.1$ & $-$ & $-$ & $-$\\
$\phi_6 = 77\%$ & $5.8 \cdot 10^5$ &  $970$ & $-$ & $-$ & $-$ & $58$  & $12$ & $-$ & $7.0 \cdot 10^{-5}$ &  $6.7 \cdot 10^{-7}$ & $7.3 \cdot 10^6$\\
  \end{tabular}
  \caption{Relevant averaged quantities from the simulations at the different volume fractions $\phi$. The overline, $\overline{()}$,
    indicates time average, the brackets, $\langle () \rangle$, indicate an average over time and over the ensemble of droplets.
    The superscripts $(s,a)$ indicate that the averages have been taken over the statistically steady stirring regime ($t \in [t_0, t_F]$)
    and over the ageing regime ($t\in [7\cdot 10^6, 9\cdot 10^6]$), respectively.
    The various columns contain: $t_{\text{inj}}$, injection endtime; 
    $ \overline{N_D}^{(s)}$, mean droplet number (stirring);  $U^{(s)}_{\text{rms}}$, root mean square droplet velocity (stirring);
    $a^{(s)}_{\text{rms}}$, root mean square droplet acceleration (stirring); $T_L^{(s)}=L/U^{(s)}_{\text{rms}}$,
    large eddy turnover time (stirring); $ \langle d \rangle $, mean droplet diameter;
    $d_{\text{rms}}$, standard deviation of the droplet diameter; $t_G$, mean (dimensionless) droplet life time; $U^{(a)}_{\text{rms}} $, root mean square droplet velocity (ageing);
    $a^{(a)}_{\text{rms}}$, root mean square droplet acceleration (ageing); $T_L^{(a)}=L/U^{(a)}_{\text{rms}}$,
    large eddy turnover time (ageing).
    Numerical values of the simulations parameters are:
    kinematic viscosity, $\nu=1/6$; total fluid density, $\rho_f = 1.36$; surface tension, $\Gamma=0.0238$;
    forcing amplitude, $A=4.85\cdot 10^{-7}$ (except for the run $\phi_6$ for which $A=4.05 \cdot 10^{-7}$); simulation
    box side, $L=512$.}
  \label{table:vol_frac_interface}
\end{table*}
We report in Table~\ref{table:vol_frac_interface} the mean and root mean square (rms)
values of some relevant observables (averaged in space and in time over the stirred phase, $t\in [t_0, t_F]$, and over the ageing phase,
$t\in [7 \cdot 10^6, 9 \cdot 10^6]$, respectively),
for the various volume fractions considered. 
One can immediately notice that, at increasing the volume fraction, accelerations
and velocities rms decrease, implying a higher effective viscosity, while at the same time,
the trend of the root mean square droplet diameter, $d_{\text{rms}}$, shows an increase in polydispersity.\\
%It can be noticed as the initial interface fragmentation displays a rapid increase in the number of droplets, and therefore expected to be breakup dominated.
%On the other hand, the phase of injection needed to achieve the high-concentration is characterized by a rapid reduction of the number of droplets, so expected to be dominated
%by coalescences events.   
% Figure environment removed
Fig.~\ref{fig:events_vol_frac} shows the rates of breakup ($\overline{\beta}$) and coalescence ($\overline{\kappa}$),
i.e. the number of events per unit time, averaged over the stirred regime
as a function of the volume fraction $\phi$ (in the inset we report $\beta(t)$
and $\kappa(t)$ as a function of time for $\phi = 70\%, 77\%$). In the steady state the system is in a dynamical
equilibrium with $N_D(t)$ essentially constant (see Fig.~\ref{fig:events_n_drops}), therefore breakup and coalescence
rates approximately balance each other, $\beta(t) \approx \kappa(t)$; moreover,
both mean rates are extremely low ($\sim 10^{-3}$) for
$\phi < \phi_c = 64\%$, i.e. below jamming,
and increase steeply with $\phi$ for $\phi>\phi_c$. Interestingly $\phi_c$ is in the expected range of
volume fractions for
random close packing of spheres in 3D~\citep{Torquato2010}.
The growth of $\overline{\beta}$ and $\overline{\kappa}$ above $\phi_c$ is particularly steep, suggesting a divergent behaviour
as the volume fraction approaches a ``critical'' value which can be arguably identified with the occurrence of the catastrophic
phase inversion, $\phi_{\text{cpi}}$. In particular, the divergence turns out to have a power-law character (as highlighted
in the inset where we plot the cologarithm of $\overline{\beta}$ and $\overline{\kappa}$ vs $\log(\phi_{\text{cpi}}-\phi)$):
the solid line depicts the fitting function $g(\phi)=\frac{C}{(\phi_{\text{cpi}}-\phi)^q}$, with fitted values
$\phi_{\text{cpi}}=90.5\%$ and $q=4.5$.\\
The number of breakup and coalescence events depends, of course, on the intensity of the hydrodynamic stresses involved.
To see how these quantities
can give a flavour of the stability of the emulsion to the applied forcing, let us introduce an average  
{\it droplet life time}, $t_G = \frac{\overline{N_D}U_{\text{rms}}}{\overline{\beta}L}$,
adimensionalized by the large eddy turnover time; from Table~\ref{table:vol_frac_interface}, we see that, below
jamming ($\phi \leq \phi_c$) $t_G \gg 1$, i.e. the droplets, on average, tend to preserve their integrity along the whole
simulation, whereas in the densely packed systems ($\phi > \phi_c$) $t_G$ abruptly drops (down to $t_G \approx 0.1$ for
the largest volume fraction, $\phi=77\%$).

\subsection{Velocity and acceleration statistics: stirred regime}

% Figure environment removed
In Fig.~\ref{fig:pdf_vel} we report the PDFs of the droplet velocities for volume fractions
$\phi=38\%$, $\phi=64\%$ and $\phi=77\%$. In both cases the PDFs show a bell shape, but in a range of intermediate
values of velocities $|v|$ they develop regions
with non-monotonic curvature, which are more
pronounced in the concentrated case. We ascribe such peculiar shape to the droplet-droplet interactions
(collisions and/or elastoplatic deformations). The characteristic velocity $v_c$ can be then estimated from
the balance of the elastic force and the Stokesian drag for a spherical droplet of diameter $d$, $F_S = 3\pi \eta d v_c$.
The elastic force acting on droplets squeezed against each other is due to the disjoining pressure $\Pi$ stabilising 
the inter-droplet thin film; at mechanical equilibrium the disjoining pressure equals the capillary pressure at the
curved droplet interface~\citep{DeryaguinChuraev}, therefore the force can be estimated as
the Laplace pressure times the cross sectional area,
$F_{\text{el}} \sim \frac{2\Gamma}{d}\pi \left(\frac{d}{2}\right)^2$, where $\Gamma$ is the surface tension. Letting
$F_{\text{el}} \sim F_S$ gives $v_c \sim \frac{\Gamma}{6\eta} = 0.0175$; from Fig.~\ref{fig:pdf_vel} we see
that, indeed, the inflections are located around $v\sim \pm v_c$.
To test our conjecture further, we have run a simulation setting the competing
lattice interactions responsible for the emergence of the disjoining pressure to zero
 ($G_{\text{OO},1}=G_{\text{WW},1}=G_{\text{OO},2}=G_{\text{WW},2}=0$ in Eq.~\ref{eq:LBint2}),
thus, effectively, we enforce $\Pi=0$. The resulting velocity PDF
is reported in Fig.~\ref{fig:pdf_vel} where we observe that, in fact, the inflectional regions disappear.\\
% Figure environment removed
The PDF of droplet accelerations, reported in Fig.~\ref{fig:pdf_acc}, is
Gaussian in the (semi)diluted emulsion ($\phi = 38\%$)  but, as the volume fraction is increased above 
$\phi_c = 64\%$ the PDFs tend to develop  {\it fat} (non-Gaussian) tails. 
A working fitting function is a stretched exponential of the type:
\begin{equation}
  P(\tilde{a})=C \exp\left(-\frac{\tilde{a}^{2}}{\left(1+\left(\frac{\beta |\tilde{a}|}{\sigma}\right)^{\gamma}\right)\sigma^{2}}\right),
  \label{eq:exp_dist}
\end{equation}
where $\tilde{a}=a/a_{\text{rms}}$.
The non-Gaussianity here, unlike turbulence, cannot be grounded on the complexity of the velocity field and of the
associated multifractal distribution of the turbulent energy dissipation~\citep{biferalePRL2004}.
We are not facing, in fact, a fully developed turbulent flow and, moreover, the non-Gaussian signatures become evident at increasing
the volume fraction above the jamming point $\phi_c$, where the effective viscosity is higher (and, hence, the effective Reynolds number is lower).
The origin of the non-Gaussianity is to be sought in the complex elastoplastic dynamics of the system, which is driven
by long-range correlated irreversible stress relaxation events.
Remarkably, in concentrated emulsions it has been shown that the spatial distribution of stress drops
in the system displays a multifractal character~\citep{KumarSciRep} which might be responsible
of the acceleration statistical properties at high volume fraction, in a formal analogy with the phenomenology of turbulence.
The curves corresponding to Eq.~(\ref{eq:exp_dist}) are reported in Fig.~\ref{fig:pdf_acc},
for various values of the parameter $\gamma$, which gauges the deviations from the Gaussian form,
and fixed $\beta=0.35$ (obtained from best fitting) and
$\sigma=1$, the standard deviation of the Gaussian limit, such that for $\gamma = 0$
the PDF reduces to the normal distribution $\propto e^{-x^2/2}$.
This is, indeed, the case for the lowest volume fraction, $\phi = 38\%$; the non-Gaussianity parameter $\gamma$ then
 increases monotonically with $\phi$ up to $\gamma = 1.6$ for $\phi = 77\%$.
%This is consistent with what found with the phenomenon of Rayleigh-B\'enard convection in dense emulsions
%at low Rayleigh number, where it was shown that, in spite of the system not being turbulent, the statistics
%of heat flux fluctuations was strongly non-Gaussian~\citep{pelusi2021}.

\subsection{Dispersion: stirred regime}
We focus now on the spatial dispersion of both single droplets as well as of droplet pairs.
The single droplet (or absolute) dispersion is defined in terms of 
the statistics of displacements, $\Delta \mathbf{X} = \mathbf{X}(t_0 + T) - \mathbf{X}(t_0)$,
where $\mathbf{X}(t)$ is the position of the droplet centre of mass at time $t$ ($t_0=1.25 \cdot 10^6$, let us recall, is the starting time of tracking).
% Figure environment removed
In Fig.~\ref{fig:absolute_dispersion} we show the mean square displacement (MSD),
$\langle \Delta X^2 \rangle$ for volume fractions $\phi=38 \%, 64 \%$ and $77 \%$.
% Figure environment removed
Values are reported in logarithmic scale, with the time normalised (hereafter) by a characteristic large scale time defined as $t_L = \sqrt{L/A} \approx 32500$ (which is independent of the volume fraction), where $A$ is the
amplitude of the applied forcing.
In the diluted case, the MSD (Fig.~\ref{fig:absolute_dispersion}) shows a crossover at around $T \sim 5t_d$ between an initial ballistic
motion, $\langle \Delta X^2 \rangle \sim T^2$,
and a diffusive behaviour at later times, $\langle \Delta X^2 \rangle \sim T$.
This is consistent with the typical Lagrangian dynamics of particles advected by
chaotic and turbulent flows~\citep{falkovich2001}; for intermediate times, though, we observe a transitional region, in correspondence,
approximately, of the crossover, where the curve presents an inflection point with locally super-ballistic slope. This is
an interestingly non-trivial behaviour that certainly deserves further investigation.
At increasing the volume fraction the long time growth becomes steeper, suggesting that a superdiffusive,
regime $T^{\alpha}$, with $\alpha = 1.15$, may occur.\\
A deeper insight on the small scale dynamics can be grasped by looking at the pair dispersion,
namely the statistics of separations
\begin{equation}\label{eq:R} 
\mathbf{R}_{ij}(t) = \mathbf{X}_i(t) - \mathbf{X}_j(t),
\end{equation}
at time $t$ for all pairs of droplets $i,j$ that are nearest neighbours (i.e. such that their corresponding cells
in a Voronoi tessellation of the centre of masses distribution are in contact) at $t=t_0$.
The observable in Eq.~(\ref{eq:R}) is, in fact, insensitive to contamination from mean homogeneous large scale flows,
if present.
In Fig.~ \ref{fig:rel_dispersion_new} we report the mean square value
$R^2(t) \equiv \langle |\mathbf{R}_{ij}|^2 \rangle_{\{ij\}}$ (where the average is over the initially neighbouring pairs)
as a function of time, for $\phi = 38\%, 64 \%$ and $77\%$.
%As the time goes by, the PDF becomes more and more spread and the maximum shift
%to increasingly larger values.
Analogously to the MSD, $R^2(t)$ grows in time and, after an initial ballistic transient,
it follows a $t^{\alpha}$ law, which is diffusive ($\alpha=1$) for concentrations below $\phi_c$ and
superdiffusive ($\alpha = 1.2$) for $\phi>\phi_c$.

\subsection{Velocity and acceleration statistics: ageing regime}

% Figure environment removed
When the large scale forcing is switched off, in diluted conditions (below the close packing volume fraction),
the system relaxes
via a long transient where the kinetic energy decays to zero.
Instead, at high volume fraction (in the jammed phase), the emulsion is never completely at rest,
due to diffusion and droplets elasticity favouring the occurrence of plastic events,
local topological rearrangement of few droplets (i.e. during the ``ageing'' of the material).
Therefore, we consider here only the latter situation and focus on the case $\phi = 77\%$; hereafter, we present
data obtained with forcing amplitude $A=4.05 \cdot 10^{-7}$ (see the $\phi_6$ row in Table~\ref{table:vol_frac_interface}),
which yielded a larger number of droplets ($\sim 10^3$) in the steady state, such to improve the statistics.
The PDF of the droplet velocities is reported in Fig.~\ref{fig:pdfs_vel+acc}. Since there is no mean flow, the
PDF is an even function of its argument. We show, therefore, the distribution of the absolute values in logarithmic scale,
in order to highlight the power-law behaviour $v^{-3}$.
Interestingly, the PDF of acceleration also develops a power law tail $P(a) \sim a^{-3}$
(the PDFs for velocity and acceleration do, in fact, overlap, upon rescaling by the respective
standard deviations, see Fig.~\ref{fig:pdfs_vel+acc}), reflecting
the fact that, when stirring is switched off, the high effective viscosity overdamps the dynamics,
thus enslaving the acceleration to the velocity (by Stokesian drag), $a \sim u/\tau_s$
(assuming Stokes time equal for all droplets, which is reasonable given the very low spread of size
distribution in the ageing regime~\citep{girotto2022}).\\

\subsection{Dispersion: ageing regime} 

In the ageing regime at the largest volume fraction, $\phi=77\%$, the
  MSD goes as $\langle \Delta X^2 \rangle \sim T^2$ for short times, signalling a ballistic regime, followed by
  a super-diffusive regime $\langle \Delta X^2 \rangle  \sim T^{3/2}$ (see inset of Fig.~\ref{fig:pdfs+MSD}).
% Figure environment removed
The short time ballistic regime is consistent with a theoretical prediction based on
the superposition of randomly distributed elastic dipoles
(following structural micro-collapses)~\citep{BouchaudEPJE2001} and with results from experiments with colloidal
gels~\citep{CipellettiPRL2000} and foams~\citep{GiavazziJPCM2021}.
The scaling $\langle \Delta X^2 \rangle  \sim T^{3/2}$ is, instead, slightly steeper
than the experimentally measured $\sim T^{1.2}$~\citep{GiavazziJPCM2021}.\\
The ballistic regime $\langle \Delta X^2 \rangle  \sim T^2$ entails a power law tail of the
PDF of separations for short times, $P(\Delta X) \sim \Delta X^{-3}$, corresponding to the self part of the van Hove distribution~\citep{Hansen},
as reported in Fig.~\ref{fig:pdfs+MSD}~\citep{CipellettiFD2003,GiavazziJPCM2021}.
This observation finds correspondence, as one could expect, in the PDF of the droplet velocities,
shown in Fig.~\ref{fig:pdfs_vel+acc}.\\
The study of pair dispersion, reported in Fig.~\ref{fig:relative_andrea},
evidences, for the mean square pair separation (in the inset),
a ballistic regime, $\langle R^2 \rangle \sim t^2$, followed by a superdiffusive behaviour,
$\langle R^2 \rangle \sim t^{4/3}$.
The persistence of ballistic motion is expected to match the decorrelation of
trajectories following a plastic events, therefore the crossover time, $t_c$, can be approximately estimated
as the time taken by a droplet to travel over the typical size of a rearrangement, $\xi \approx 2 d $ (which is
an intrinsic scale for correlation lengths in soft glassy materials~\citep{Goyon2008,Dollet2015}); since
the characteristic velocity is $v_c \sim \frac{\Gamma}{6\eta}$ (see Fig.~\ref{fig:pdf_vel} and discussion thereof),
we get $t_c \sim \frac{12 d \eta}{\Gamma} \approx 7 \cdot 10^3$,
indicated in the inset of Fig.~\ref{fig:relative_andrea} with a dashed line.\\
%% Figure environment removed
By analogy with the Richardson's description of
turbulent diffusion~\citep{richardson1926,falkovich2001,boffetta2002}, we propose a phenomenological approach to derive the full pair separation PDF in the superdiffusive regime.
We assume that the such PDF evolves according
to a generalized diffusion equation, with a scale-dependent effective diffusivity, $D_{\text{eff}}$ which,
dimensionally, should be proportional to $\frac{d \langle R^2 \rangle}{dt}$. Since $\langle R^2 \rangle \sim t^{4/3}$ 
(and, consequently, $t \sim R^{3/2}$), we have
\begin{equation}
  D_{\text{eff}} \propto \frac{d \langle R^2 \rangle}{dt} \sim t^{1/3} \quad \Rightarrow \quad D_{\text{eff}} \propto R^{1/2}
  \equiv c R^{1/2}.
\end{equation}
The diffusion equation, thus, reads
\begin{equation}\label{eq:diffPDF}
\partial_t P(R,t) = \partial_{R} \left(cR^{1/2} \partial_{R} P(R,t)\right),
\end{equation}
that admits as solution (with the condition of unit area at all times) the following non-Gaussian distribution 
\begin{equation}\label{eq:pdftheo}
P(R,t) = \mathcal{A}\frac{2R^2}{27c^2 t^2}\exp\left(-\frac{4 R^{3/2}}{9ct}\right).
\end{equation}
The PDFs of pair separations measured at two instants of time in the superdiffusive regime, $t_1 \approx 150 t_c$ and
$t_2  \approx 300 t_c$, are shown in Fig.~\ref{fig:relative_andrea} together with the prediction of Eq.~(\ref{eq:pdftheo}),
with fitting parameter $c = 0.25 t_c^{-1}$, plotted as solid lines. The agreement obtained between
theory and numerics is quite remarkable.

\section{Conclusions}\label{sec:conclusions}
We presented results on the statistics of droplet velocities, accelerations and of droplet absolute
and relative dispersion in stabilized emulsions at various volume fractions, from semi-diluted to highly concentrated systems.
We employed a recently developed method for {\it in silico} emulsification of binary immiscible liquid mixtured with
high volume fractions of the dispersed phase, equipped with a novel tracking algorithm which allowed us to study
the emulsion physics at the droplet-resolved scale from a Lagrangian viewpoint, across various concentrations, from the semi-dilute to
jammed regimes.
Our results highlighted how the elastic properties and the plastic microdynamics of densely packed ensembles of droplets,
in close contact, 
are responsible of the non-Gaussian character of the droplet acceleration and, more moderately, of the velocity statistics.
We further investigated the single droplet diffusion in terms of both the mean square displacement and the self-part of the van Hove
distribution functions, finding that, while in the semi-dilute stirred case a ballistic-to-diffusive crossover is observed, in the
highly concentrated case a super-diffusive behaviour seems to emerge. Super-diffusion characterizes also the ageing regime, where
agreement is found with previous theoretical and experimental results.
Further investigations will focus on the dispersion properties on larger systems and for longer observation times, as well as
on the relation of the droplet Lagrangian properties with the stress distribution across the system.
In perspective, we foresee to extend the reach of the present work to extreme conditions of volume fractions and forcing amplitudes,
whereby the emulsion tends to loose stability and to undergo a catastrophic phase inversion. In this limit, too, the Lagrangian approach
is of invaluable utility.
Overall, our approach suggested a bridge between classical tools for Lagrangian high Reynolds number flows and 
complex fluid rheology, which paves the way to the inspection of unexplored aspects of the physics of soft materials.

\section*{Acknowledgements}
We are thankul to Chao Sun and Lei Yi for useful discussions and to Prasad Perlekar for a fruitful interaction at the
initial stage of the work. Numerical simulations were performed thanks to granted PRACE projects (ID: 2018184340 \& 2019204899)
along with CINECA and BSC for access to their HPC systems. This work was partially sponsored by NWO domain
Science for the use of supercomputer facilities.


%%%%%%%%%%%%%%% OLD MODEL %%%%%%%%%%%%%%%%%%%%%%%

%In order to explain theoretically these observations, let us first notice that:
%\begin{equation}\label{eq:Rdot}
%  \dot{R}(t) = \frac{d}{dt}\langle R^2 \rangle^{1/2} = \frac{1}{R}\langle \mathbf{R} \cdot \dot{\mathbf{R}} \rangle =
%  \frac{1}{R} \langle \mathbf{R} \cdot \Delta_R \mathbf{U} \rangle,
%\end{equation}
%in fact,
%$\dot{\mathbf{R}} = \dot{\mathbf{X}}_i - \dot{\mathbf{X}}_j = \mathbf{U}(\mathbf{X}_i(t),t)-\mathbf{U}(\mathbf{X}_j(t),t) \equiv \Delta_R \mathbf{U}$; in a mean field spirit, one can
%approximate
%$\langle \langle \mathbf{R} \cdot \Delta_R \mathbf{U} \rangle \approx \langle R^2 \rangle^{1/2} \langle \hat{R} \cdot \Delta_R \mathbf{U} \rangle$~\citep{BecJFM2010}, such that
%Eq.~(\ref{eq:Rdot}) becomes:
%\begin{equation}\label{eq:Rdot}
%\dot{R} \approx \langle \Delta_R \mathbf{U} \cdot \hat{R} \rangle \equiv \Delta_{\|}U(R),
%\end{equation}
%where the last term on the right hand side is known, in the language of the statistical theory of turbulence, as
%first order longitudinal structure function~\citep{frisch1995turbulence}; in other words,
%we have to determine the difference in displacement field over a distance $R$.
%Following~\citet{BouchaudEPJE2001}, we assume that the displacement field at a given point in space
%is the result of the superposition of
%contributions stemming from {\it randomly}
%distributed sources of elastic stress generated by local topological rearrangements
%(structural ``micro-collapses'' or ``plastic events'').
%The elementary elastic strain field at a position $\mathbf{x}$ due to a source localized at $\mathbf{x}^{\prime}$
%is $\mathbf{\Gamma}(\mathbf{x},\mathbf{x}^{\prime},\hat{n}) \propto \frac{(\mathbf{x} - \mathbf{x}^{\prime})\cdot \hat{n}}{|\mathbf{x} - \mathbf{x}^{\prime}|^3}\hat{n}$,
%where $\hat{n}$ is the orientation of the force dipole representing the plastic
%event~\citep{BouchaudEPJE2001,PicardEPJE2004}. 
%Therefore the displacement field at $\mathbf{X}_i$ generated by a distribution
%of plastic events with density $P(\mathbf{x},\hat{n}) \approx \rho(\mathbf{x}) \mu(\hat{n})$, assumed factorized,
%reads
%$\mathbf{U}(\mathbf{X}_i) \sim \int \int \rho(\mathbf{x}^{\prime})\phi(\hat{n})\mathbf{\Gamma}(\mathbf{X}_i,\mathbf{x}^{\prime},\hat{n})d\mathbf{x}^{\prime}d\hat{n}$,
%which yields (further assuming a uniform distribution of dipole orientation,
%$\mu(\hat{n}) \approx \mu_0$):
%\begin{equation}\label{eq:Ufield}
%  \mathbf{U}(\mathbf{X}_i) \sim \int_{\mathcal{B}_{\varepsilon}(\mathbf{X}_i)}
%  \frac{\rho(\mathbf{x}^{\prime}(\mathbf{X}_i - \mathbf{x}^{\prime})}{|\mathbf{X}_i - \mathbf{x}^{\prime}|^2}d\mathbf{x}^{\prime}.
%\end{equation}
%The integration domain, $\mathcal{B}_{\Lambda}(\mathbf{X}_i)$, is taken to be the ball of radius
%$\Lambda$ centered at $\mathbf{X}_i$, to account for the finite cutoff of the stress propagator in
%actual systems.
%A homologous equation holds for $\mathbf{U}(\mathbf{X}_j)$. Dimensionally, then, the difference
%$\Delta_{\|}U(R)$ can be estimated as
%\begin{equation}\label{eq:udiff}
%  \Delta_{\|}U(R) \sim \frac{\langle |\delta_{\Lambda+R} \rho| \rangle}{\Lambda^2}|\delta_R\mathcal{B}_{\Lambda}|,
%\end{equation}
%where $\delta_{\Lambda + R}\rho$ is the amplitude of the statistical density fluctuations on
%a scale $\Lambda + R$, averaged over $\delta_R\mathcal{B}_{\Lambda}$,
%the non-overlapping part of the influence domain, i.e.
%$\delta_R\mathcal{B}_{\Lambda} = (\mathcal{B}_{\Lambda}(\mathbf{X}_i) \cup  \mathcal{B}_{\Lambda}(\mathbf{X}_j))- (\mathcal{B}_{\Lambda}(\mathbf{X}_i) \cap
%\mathcal{B}_{\Lambda}(\mathbf{X}_j))$,
%whose volume is $|\delta_R\mathcal{B}_{\Lambda}| \propto \Lambda^2 R$.
%The density, in fact, can be considered strictly constant only in a ``thermodynamic'' sense,
%i.e. over an arbitrarily ``large'' system.
%On finite scales, fluctuations need to be taken into account.
%In particular, if we indicate the number of events in a domain of linear size $\varepsilon$ as $N_{\varepsilon}$, these can be written as 
%\begin{equation}
%N_{\varepsilon} = \overline{N} + \delta_{\varepsilon} N = \overline{N} \pm \sqrt{\overline{N}},
%\end{equation}
%where a Poissonian distribution of events was assumed~\citep{DuriEPL2006}. $\overline{N}$ stands for the 
%mean value as expected for a perfectly constant density $\rho_0$, i.e. $\overline{N}=\rho_0 \varepsilon^3$.
%Correspondingly, for the density we have:
%\begin{equation}\label{eq:rhoeps}
%\rho_{\varepsilon} \equiv \varepsilon^{-3} N_{\varepsilon} =  \rho_0 \pm \rho_0^{1/2}\varepsilon^{-3/2}.
%\end{equation}
%The velocity difference (\ref{eq:udiff}) then reads
%\begin{equation}\label{eq:drU}
%\Delta_{\|} U(R) \sim \frac{\rho_0^{1/2} (\Lambda+R)^{-3/2}}{\Lambda^2}\Lambda^2 R \sim (\Lambda + R)^{-3/2} R.
%\end{equation}
%Eq.~(\ref{eq:Rdot}), thus, results in an ODE that can be easily solved and, for $R \gg \Lambda$, gives
%\begin{equation}\label{eq:Rvst}
% \dot{R} \propto R^{-1/2} \quad \Rightarrow \quad   R^2(t) \sim t^{4/3}.
%\end{equation}  


%%%%%%%%%%%%%%%%%%%%%%%%%%%%%%%%%%%%%%%%%%%

%\bibliographystyle{jfm}
\documentclass[twocolumn,hyperpdf,amsmath,amssymb,aps,prd,10pt,superscriptaddress,nofootinbib,noeprint,preprintnumbers,floatfix]{revtex4-2}

%% \usepackage[utf8]{inputenc} % allow utf-8 input
%\usepackage[T1]{fontenc}    % use 8-bit T1 fonts
%\usepackage{hyperref}       % hyperlinks
\usepackage{url}            % simple URL typesetting
\usepackage{booktabs}       % professional-quality tables
\usepackage{multirow}    
\usepackage{amsfonts}       % blackboard math symbols
\usepackage{nicefrac}       % compact symbols for 1/2, etc.
\usepackage{microtype}      % microtypogrhy
% \usepackage{natbib}
\usepackage{enumerate}
%\usepackage{enumitem}
\usepackage{hhline}
\usepackage{makecell}
\usepackage{pifont}

% use Times
%\usepackage{times}
% For figures
\usepackage{graphicx} % more modern
%\usepackage{epsfig} % less modern
%\usepackage{subfigure}
\usepackage{caption}
\usepackage{subcaption}
% For citations
\usepackage{amsmath}
\usepackage{amsthm}
\usepackage{amssymb}
\usepackage{tikz}
\usepackage{xcolor}
\usetikzlibrary{arrows}

\allowdisplaybreaks

%for fonts
\usepackage{mathrsfs}

% For algorithms
\usepackage{algorithm}
\usepackage{algorithmic}
% \usepackage{algpseudocode}
% \usepackage[noend]{algpseudocode}
\usepackage{hyperref}
\usepackage{bm}
%\usepackage{todonotes}

%For theorems
\allowdisplaybreaks

%for convinience
\newcommand{\RR}{\mathbb{R}}
\newcommand{\vct}{\boldsymbol }
%\newcommand{\mat}{\mathbf}
\newcommand{\rnd}{\mathsf}
\newcommand{\ud}{\mathrm d}
\newcommand{\nml}{\mathcal{N}}
\newcommand{\loss}{\mathcal{L}}
\newcommand{\hinge}{\mathcal{R}}
\newcommand{\kl}{\mathrm{KL}}
\newcommand{\cov}{\mathrm{cov}}
\newcommand{\dir}{\mathrm{Dir}}
\newcommand{\mult}{\mathrm{Mult}}
\newcommand{\err}{\mathrm{err}}
\newcommand{\sgn}{\mathrm{sgn}}
%\renewcommand{\span}{\mathrm{span}}
% \newcommand{\argmin}{\mathrm{argmin}}
% \newcommand{\argmax}{\mathrm{argmax}}
\newcommand{\poly}{\mathrm{poly}}
% \newcommand{\rank}{\mathrm{rank}}
% \newcommand{\conv}{\mathrm{conv}}
%\newcommand{\E}{\mathbb{E}}
% \newcommand{\diag}{\mat{diag}}
\newcommand{\acc}{\mathrm{acc}}

\newcommand{\labs}{\left\vert}
\newcommand{\rabs}{\right\vert}
\newcommand{\lnorm}{\left\Vert}
\newcommand{\rnorm}{\right\Vert}

\newcommand{\aff}{\mathrm{aff}}
% \newcommand{\range}{\mathrm{Range}}
\newcommand{\Sgn}{\mathrm{sign}}

\newcommand{\hit}{\mathrm{hit}}
\newcommand{\cross}{\mathrm{cross}}
\newcommand{\Left}{\mathrm{left}}
\newcommand{\Right}{\mathrm{right}}
\newcommand{\Mid}{\mathrm{mid}}
\newcommand{\bern}{\mathrm{Bernoulli}}
\newcommand{\ols}{\mathrm{ols}}
\newcommand{\tr}{\operatorname{tr}}
\newcommand{\opt}{\mathrm{opt}}
%\newcommand{\ridge}{\mathrm{ridge}}
\newcommand{\unif}{\mathrm{Unif}}
\newcommand{\Image}{\mathrm{im}}
\newcommand{\Kernel}{\mathrm{ker}}
\newcommand{\supp}{\mathrm{supp}}
\newcommand{\pred}{\mathrm{pred}}
\newcommand{\distequal}{\stackrel{\mathbf{P}}{=}}
%\newcommand{\gege}{\textcircled{1}}
\newcommand{\gege}{{A(\vect{w},\vect{w}_*)}}
\newcommand{\gele}{{A(\vect{w},-\vect{w}_*)}}
\newcommand{\lele}{{A(-\vect{w},-\vect{w}_*)}}
\newcommand{\lege}{{A(-\vect{w},\vect{w}_*)}}
\newcommand{\firstlayer}{\mathbf{W}}
\newcommand{\firstlayerWN}{v}
\newcommand{\secondlayer}{a}
\newcommand{\inputvar}{\vect{x}}
\newcommand{\anglemat}{\mathbf{\Phi}}
\newcommand{\holder}{H\"{o}lder }
\newcommand{\real}{\mathbb{R}}
\newcommand{\approxerr}{\delta}

\def\R{\mathbb{R}}
\def\Z{\mathbb{Z}}
\def\cA{\mathcal{A}}
\def\cB{\mathcal{B}}
\def\cD{\mathcal{D}}
\def\cE{\mathcal{E}}
\def\cF{\mathcal{F}}
\def\cG{\mathcal{G}}
\def\cH{\mathcal{H}}
\def\cS{\mathcal{S}}
\def\cI{\mathcal{I}}
\def\cL{\mathcal{L}}
\def\cM{\mathcal{M}}
\def\cN{\mathcal{N}}
\def\cP{\mathcal{P}}
\def\cS{\mathcal{S}}
\def\cT{\mathcal{T}}
\def\cV{\mathcal{V}}
\def\cW{\mathcal{W}}
\def\cZ{\mathcal{Z}}
\def\SS{\mathbb{S}}
\def\NN{\mathbb{N}}
\def\bP{\mathbf{P}}
\def\TV{\mathrm{TV}}
\def\MSE{\mathrm{MSE}}

\def\vw{\mathbf{w}}
\def\va{\mathbf{a}}
\def\vZ{\mathbf{Z}}

\newcommand{\mat}[1]{#1}
\newcommand{\vect}[1]{#1}
\newcommand{\norm}[1]{\left\|#1\right\|}
\newcommand{\normop}[1]{\left\|#1\right\|_{\mathrm{op}}}
\newcommand{\simplex}{\triangle}
\newcommand{\abs}[1]{\left|#1\right|}
\newcommand{\expect}{\mathbb{E}}
\newcommand{\prob}{\mathbb{P}}
\newcommand{\proj}{\gP}
% \newcommand{\prox}[2]{\textbf{Prox}_{#1}\left\{#2\right\}}
\newcommand{\event}[1]{\mathscr{#1}}
\newcommand{\set}[1]{#1}
\newcommand{\diff}{\text{d}}
\newcommand{\difference}{\triangle}
\newcommand{\inputdist}{\mathcal{Z}}
\newcommand{\indict}{\mathbb{I}}
\newcommand{\rotmat}{\mathbf{R}}
\newcommand{\normalize}[1]{\overline{#1}}
\newcommand{\vectorize}[1]{\text{vec}\left(#1\right)}
\newcommand{\vclass}{\mathcal{G}}
\newcommand{\pclass}{\Pi}
\newcommand{\qclass}{\mathcal{Q}}
\newcommand{\rclass}{\mathcal{R}}
\newcommand{\classComplexity}[2]{N_{class}(#1,#2)}
\newcommand{\cclass}{\mathcal{F}}
\newcommand{\gclass}{\mathcal{G}}
\newcommand{\pthres}{p_{thres}}
\newcommand{\ethres}{\epsilon_{thres}}
\newcommand{\eclass}{\epsilon_{class}}
\newcommand{\states}{\mathcal{S}}
\newcommand{\trans}{P}
\newcommand{\lowprobstate}{\psi}
\newcommand{\actions}{\mathcal{A}}
\newcommand{\contexts}{\mathcal{X}}
\newcommand{\edges}{\mathcal{E}}
\newcommand{\variance}{\text{Var}}
\newcommand{\params}{\vect{w}}

\newcommand{\relu}[1]{\sigma\left(#1\right)}
\newcommand{\reluder}[1]{\sigma'\left(#1\right)}
\newcommand{\act}[1]{\sigma\left(#1\right)}

\newtheorem{thm}{Theorem}
% \newtheorem{thm}{Theorem}
\newtheorem{lem}{Lemma}
% Thm -> corollary 
\newtheorem{cor}{Corollary}
\newtheorem{prop}{Proposition}
\newtheorem{asmp}{Assumption}
\newtheorem{defn}{Definition}
\newtheorem{oracle}{Oracle}
\newtheorem{fact}{Fact}
\newtheorem{conj}{Conjecture}
\newtheorem{rem}{Remark}
\newtheorem{example}{Example}
\newtheorem{condition}{Condition}
\newtheorem{exercise}{Exercise}
\newtheorem{mess}{Message}
\newtheorem{claim}{Claim}
\newtheorem{ec}{Empirical Conclusion}






\usepackage[capitalize,noabbrev]{cleveref}
% \usepackage{cleveref}
\crefname{thm}{Theorem}{Theorems}
\crefname{lem}{Lemma}{Lemmas}
\crefname{cor}{Corollary}{Corollaries}
\crefname{prop}{Proposition}{Propositions}
\crefname{asmp}{Assumption}{Assumptions}
\crefname{defn}{Definition}{Definitions}
\crefname{oracle}{Oracle}{Oracles}
\crefname{fact}{Fact}{Facts}
\crefname{conj}{Conjecture}{Conjectures}
\crefname{rem}{Remark}{Remarks}
\crefname{claim}{Claim}{Claims}
\crefname{ec}{Empirical Observation}{Empirical Observations}


\renewcommand{\algorithmicrequire}{\textbf{Input:}}
\renewcommand{\algorithmicensure}{\textbf{Output:}}


\definecolor{red}{rgb}{1, 0, 0}
\newcommand{\RED}[1]{{\color{red} #1}}

\definecolor{green}{rgb}{0, 1, 0}
\definecolor{darkgreen}{rgb}{0.0, 0.2, 0.13}
\definecolor{darkseagreen}{rgb}{0.56, 0.74, 0.56}
\definecolor{officegreen}{rgb}{0.0, 0.5, 0.0}


\newcommand{\GREEN}[1]{{\color{green} #1}}

\definecolor{blue}{rgb}{0, 0, 1}
\newcommand{\BLUE}[1]{{\color{blue} #1}}

\definecolor{orange}{rgb}{1, 0.4, 0.0}
\newcommand{\ORANGE}[1]{{\color{orange} #1}}


\usepackage{graphicx, color}
\usepackage[dvipsnames]{xcolor}

%% text %%
\usepackage[letterspace=-10]{microtype} 

%% math, tables %%
\usepackage{bm, amsmath, amsfonts, amssymb,xfrac}
\usepackage{multirow, tabularx, dcolumn}
\usepackage{mathtools, leftidx, braket, slashed, cancel, bigdelim}
\usepackage{blkarray}
\usepackage[figures]{rotating}

\usepackage{tikz}
\usetikzlibrary[plotmarks]
\usepackage{anyfontsize}

%% referencing %%
\usepackage[utf8]{inputenc} 
\usepackage{hyperref}
\pdfstringdefDisableCommands{ \renewcommand{\bm}[1]{#1} }

%% colors %%
\definecolor{jlab_red}{RGB}{192,39,45}
\definecolor{jlab_orange}{RGB}{249,102,0}
\definecolor{jlab_blue}{RGB}{47,122,121}
\definecolor{jlab_green}{RGB}{65,125,10}
\definecolor{jlab_gray}{gray}{0.6}
\definecolor{magenta}{rgb}{0.5, 0, 0.5}

\newcommand\bef{% Figure environment removed}
\newcommand\beq{\begin{equation}}
\newcommand\eeq[1]{\label{#1}\end{equation}}
\newcommand\beqa{\begin{eqnarray}}
\newcommand\eeqa[1]{\label{#1}\end{eqnarray}}
\newcommand\bet{\begin{table}}
\newcommand\eet[1]{\label{tb:#1}\end{table}}
\newcommand\fgn[1]{Figure \ref{fg:#1}} 
\newcommand\eqn[1]{Eq.\ (\ref{#1})}
\newcommand\tbn[1]{Table \ref{tb:#1}} 
\newcommand\pmn[1]{\textcolor{red}{#1}} 
\newcommand\pmnc[1]{\textcolor{red}{\it Comment: #1}}
\newcommand\nma[1]{\textcolor{blue}{#1}} 
\newcommand\nmac[1]{\textcolor{blue}{\it NM Comment: #1}}
\newcommand\arad[1]{\textcolor{green}{#1}} 
\newcommand\aradc[1]{\textcolor{green}{\it AR Comment: #1}}

%% editing macros %%
\newcommand{\cm}{\ensuremath{\mathsf{cm}}}


%% pdf hypertext links
\hypersetup{%
pdftitle = {title},
pdfsubject = {},
pdfkeywords = {},
%pdfauthor = {Hadron Spectrum Collaboration},
colorlinks = {true},
filecolor = {black},
linkcolor = {jlab_blue},
menucolor = {black},
citecolor = {jlab_green},
urlcolor = {jlab_green},
}{}

\begin{document}

%%%%%%%%%%%%%%%%%%%%%%%%%%%%%%%%%%%%%%%%%%%%%%%%%%%%%%%%%%%%%%%%%%%%%%
%\preprint{JLAB-THY-22-xxxx}
%
\title{Bound isoscalar axial-vector $bc\bar u\bar d$ tetraquark $T_{bc}$ in QCD}
%
\author{M. Padmanath}
\email{padmanath@imsc.res.in}
\affiliation{The Institute of Mathematical Sciences, a CI of Homi Bhabha National Institute, Chennai, 600113, India}
%
\author{Archana Radhakrishnan}
\email{archana.radhakrishnan@tifr.res.in}
\affiliation{Department of Theoretical Physics, Tata Institute of Fundamental Research, \\ Homi Bhabha Road, Mumbai 400005, India }

%
\author{Nilmani Mathur}
\email{nilmani@theory.tifr.res.in}
\affiliation{Department of Theoretical Physics, Tata Institute of Fundamental Research, \\ Homi Bhabha Road, Mumbai 400005, India }
%
% \collaboration{for the Hadron Spectrum Collaboration}
%

\preprint{IMSc/23/05, TIFR/TH/23-14}

\date{\today}
\begin{abstract}

The Fast Reciprocal Square Root Algorithm is a well-established approximation technique consisting of two stages: first, a coarse approximation is obtained by manipulating the bit pattern of the floating point argument using integer instructions, and second, the coarse result is refined through one or more steps, traditionally using Newtonian iteration but alternatively using improved expressions with carefully chosen numerical constants found by other authors. The algorithm was widely used before microprocessors carried built-in hardware support for computing reciprocal square roots. At the time of writing, however, there is in general no hardware acceleration for computing other fixed fractional powers. This paper generalises the algorithm to cater to all rational powers, and to support any polynomial degree(s) in the refinement step(s), and under the assumption of unlimited floating point precision provides a procedure which automatically constructs provably optimal constants in all of these cases. It is also shown that, under certain assumptions, the use of monic refinement polynomials yields results which are much better placed with respect to the cost/accuracy tradeoff than those obtained using general polynomials. Further extensions are also analysed, and several new best approximations are given.

\end{abstract}

\maketitle
%%%%%%%%%%%%%%%%%%%%%%%%%%%%%%%%%%%%%%%%%%%%%%%%%%%%%%%%%%%%%%%%%%%%%%%%%%%%%%%%%

% Figure environment removed

\section{Introduction}
Automatic 3D reconstruction of clothed humans using image inputs has gained increasing significance due to its potential applications in a wide array of AR/VR scenarios. High-fidelity reconstructions typically depend on sophisticated capture systems, which are developed with dense camera arrays~\cite{collet2015high,joo2015panoptic,joo2018total}, programmable light-stages~\cite{Vlasic2009, guo2019relightables}, and depth sensors~\cite{newcombe2011kinectfusion,DoubleFusion,BodyFusion,dou2016fusion4d,newcombe2015dynamicfusion}. However, stringent capture environments equipped with complex hardware pose significant challenges for consumer-level applications.


In this context, considerable research effort has been dedicated to developing methods that allow for more flexible capture configurations, such as utilizing a few RGB inputs. Among these works, learning implicit functions \cite{iccv2020PIFu, saito2020pifuhd, hong2021stereopifu} has proven effective in achieving highly detailed reconstructions by integrating the advancements of deep neural networks. These methods employ large multi-layer perceptrons (MLPs) to predict the occupancy probability or truncated signed distance function (TSDF) value of every queried 3D point based on its associated local feature, which is extracted from images. They can recover a continuous surface at arbitrary resolutions without topology restrictions.


However, in typical MLP-based implicit networks, the occupancy or TSDF value at each location is solved independently with planar image features, rendering them less capable of addressing challenging cases such as occlusions. Consequently, these methods suffer from generalization and robustness issues, particularly when tackling strong occlusions caused by large motion or multiple interacting humans. 
Some follow-up studies  \cite{zheng2021deepmulticap,zheng2021pamir,huang2020arch} utilize an extra geometric model, SMPL~\cite{Loper2015}, to improve robustness by introducing strong shape priors. 
Their success typically relies on the assumption of geometrical similarity \cite{huang2020arch} between the shape prior and target reconstruction, making them intractable for handling complex cases with loose clothes and sensitive to errors in SMPL model fitting.



%\ping{this paragraph sounds like `TSDF is better than MLP/SMPL, and we use TSDF to solve the problem'. But in Sec 3, we are telling a different story, saying `MLP needs a 3D convolutional encoder'. We need to make these two sections consistent.}\sicong{I think in this paragraph we claim that the TSDF}


%We opt for Trucated Signed Distance Funtion (TSDF) volumetric representations as they are naturally suitable for convolution operations, which have shown remarkable performance for learning hierarchical features on 2D visual perception tasks \cite{SunXLW19}. 
%Meanwhile, TSDF also describes the gradual geometry change around shape surface, which is not reflected by occupancy volume. 

We instead revisit the 3D volumetric representation and resort to 3D convolutional neural networks (CNNs) for feature learning, due to their impressive performance in feature learning and the ability to incorporate spatial context. However, volumetric methods and 3D convolution involve discretization, which might raise concerns regarding whether a discretized volume can preserve subtle geometric details as continuous representations learned in implicit functions. We investigate the relationship between volume resolution and quantization error on synthetic data by converting target mesh objects to TSDF volumes, as shown in Figure~\ref{fig:quantization_error}. We observe that the quantization errors are significantly reduced by increasing volume resolution and become nearly negligible when reaching a relatively high resolution (e.g., 512 or higher). In other words, achieving fine-detailed reconstruction is not supposed to be restricted by the use of volume representations as long as a proper volume resolution is utilized. Therefore, we present a method with high-resolution feature volumes, e.g., 256 and 512, while traditional volumetric methods \cite{varol18_bodynet,gilbert2018volumetric} are often limited to much lower resolutions, such as 32 or 128.



On the other hand, an increase in volume resolution may lead to a cubic growth of memory overhead \cite{8100085}. Reducing memory costs while guaranteeing the granularity of volumetric representations is necessary for pursuing high-quality reconstruction. Thus, we adopt a coarse-to-fine approach and cull away irrelevant voxels to build a sparse high-resolution feature volume. At the coarse level, the network computes an initial TSDF by applying a U-Net with sparse 3D CNN \cite{3DSemanticSegmentationWithSubmanifoldSparseConvNet} on the sparse feature volume, which is carved by a visual hull. Through our experiments, it turns out that more than 95\% of the volume grids are discarded by the visual hull culling, making the sparse 3D CNN efficient. At the fine level, the network focuses on a narrow band near the zero-level set of the initial TSDF and discretizes the narrow band with smaller voxels. By employing this narrow-band culling, we further shrink the sampling space, resulting in a relatively small range of grid numbers (usually 300K--500K in our experiments) even with a high volume resolution of 512. The remaining voxels in the narrow band are associated with features that fuse high-frequency information from the computed normal maps upon the low-frequency shape from the coarse level to compute the TSDF at high resolution. The final mesh is then extracted from the TSDF using the Marching-Cube algorithm ~\cite{Lorensen87marchingcubes}.
% Different from the u-net sturcture to preserve global topology context, we then apply a shallow 3dcnn to compute the final TSDF $D_{final}$ which contain more local geometry detail.




% \ping{this paragraph can be expanded. It is an important contribution and often ignored by other works. stress on the novel idea of regressing blending weights instead of colors}

In addition to geometry, high-quality mesh texture is also a crucial factor contributing to visual appearance. Directly computing a color field in 3D space, as in \cite{iccv2020PIFu}, struggles to capture high-frequency texture details, while the neural radiance field (NeRF) \cite{yu2020pixelnerf} or the DoubleField~\cite{shao2022doublefield} require expensive per-instance optimization and are often unstable for sparse input images. In contrast, we adopt an image-based rendering approach to compute a texture atlas map, which is efficient and widely supported in existing computer graphics tools. 
Specifically, we compute a blending weight at each 3D point on the mesh surface to determine its color as a weighted average of the colors at its image projections. The blending weights can be computed at a relatively coarse resolution, e.g., 512 volume resolution in our case, and leave texture details to the high-resolution images, such as 1K or 2K. Unlike previous methods that generate blurry texturing results under sparse input, our method generalizes well on both synthetic and real data with just a few input views. 
Figure~\ref{fig:teaser} shows two examples reconstructed by our method. Despite the challenging garment, pose, and occlusion, our method recovers faithful shape, normal, and texture on the right.

%with a wide variety of poses and clothing styles, and it is also adaptive to handle input image with arbitrary resolutions.
%\sicong{For this concern we claim that when the resolution of dicretized volume meets certain threshold (which is 256 in our experiment), the quantization error can be neglected.} 



In summary, the main contributions of this paper are as follows:
\begin{itemize}
\vspace{-0.1in}
  \item 
  We revisit the 3D volumetric representation and demonstrate that it can support clothed human reconstruction with equal or even better performance compared to implicit representation. 
  \item 
  We develop a memory and computation-efficient method for high-resolution volumetric reconstruction using sophisticated sparse 3D CNN, coarse-to-fine estimation, and voxel culling by visual hull and narrow bands. 
  \item 
  We introduce a novel method to compute a texture atlas map, which captures rich appearance details from high-resolution input images.
  \item 
  We achieve impressive results on standard benchmark datasets Twindom and MultiHuman, significantly reducing the point-2-surface (P2S) precision to approximately 0.2cm from just six input views, with more than $50\%$ error reduction compared to the state-of-the-art methods, including DoubleField~\cite{shao2022doublefield} and PIFuHD~\cite{saito2020pifuhd}.
\end{itemize}
\section{Ensembles and fermion actions}\label{sec:lattice}

We use the same computational setup as in several of our previous publications \cite{Junnarkar:2019equ,
Junnarkar:2018twb,Basak:2014kma,Padmanath:2017lng,Basak:2012py,Basak:2013oya,Mathur:2016hsm,
Mathur:2018epb,Mathur:2018rwu,Junnarkar:2022yak,Mathur:2022ovu}, which we briefly summarize below for completeness. Four 
$N_f=2+1+1$ lattice QCD ensembles generated by the MILC collaboration are used in this study \cite{MILC:2012znn}, where
the dynamical quark flavors were simulated using Highly Improved Staggered Quark (HISQ) action on gauge fields 
that respect one-loop, tadpole-improved Symanzik gauge action with tuned coefficients through 
$\mathcal{O}(\alpha_sa^2, n_f\alpha_sa^2)$ \cite{Follana:2006rc}. The charm and strange quark masses are 
tuned to their respective physical values, whereas the dynamical light quarks are chosen such 
that $m_s/m_l\sim 5$. We list the relevant details of various lattice QCD ensembles used in \tbn{lattice}.

\bet[tbh]
  \begin{center}
	  \begin{tabular}{p{1.5cm}p{1.5cm}p{1.5cm}>{\hfill\arraybackslash}p{1.5cm}}
      \hline
Label & Symbol & $a~[fm]$     & $N_s^3\times N_t$ \\ \hline
$S_1$ & \pmb{\textcolor{red}{\tikz{\pgfsetplotmarksize{0.8ex}\pgfuseplotmark{diamond}}}} & 0.1207(11)   & $24^3\times64$ \\
$S_2$ & \pmb{\textcolor{magenta}{\tikz{\pgfsetplotmarksize{0.8ex}\pgfuseplotmark{pentagon}}}} & 0.0888(8)    & $32^3\times96$ \\
$S_3$ & \pmb{\textcolor{blue}{\tikz{\pgfsetplotmarksize{0.7ex}\pgfuseplotmark{o}}}} & 0.0582(4)    & $48^3\times144$ \\
$L_1$ & \pmb{\textcolor{OliveGreen}{\pgfsetplotmarksize{0.7ex}\tikz{\pgfuseplotmark{square}}}} & 0.1189(9)    & $40^3\times64$ \\   \hline
  \end{tabular}
  \end{center}
\caption{Relevant details of the lattice QCD ensembles used. The lattice spacing estimates 
are measured using the $r_1$ parameter \cite{MILC:2012znn}. $L$ in $L_1$ refers to large spatial volume, 
and $S$ in $S_1,~S_2$, and $S_3$ refer to small spatial volume. }
\eet{lattice}

The valence quark fields for the light, strange and charm flavors are realized using an overlap 
fermion action that is $\mathcal{O}(am)$ improved. To this end, we utilize the numerical 
implementation of the overlap action following Refs. \cite{Chen:2003im,xQCD:2010pnl}. Following 
the Fermilab prescription \cite{El-Khadra:1996wdx}, the bare charm quark mass on each ensemble was tuned 
using the kinetic mass of spin averaged $1S$ charmonia $\{a\overline M_{kin}^{\bar cc} = 0.75 aM_{kin}(J/\psi) + 0.25 aM_{kin}(\eta_c)\}$
determined for the respective ensembles. Further details on the tuning of charm quark mass, 
the tuned bare quark mass, and resulting discretization effects are discussed in Refs. \cite{Basak:2012py,Basak:2013oya}.
The bare strange quark mass is set by equating the lattice estimate for the fictitious pseudoscalar $\bar ss$ 
meson mass to 688.5 MeV \cite{Chakraborty:2014aca}. Additionally, we perform the quark propagator 
measurements in the valence sector using overlap fermion action for three other quark masses in 
all the ensembles corresponding to pseudoscalar masses of approximately 0.5, 0.6 and 1.0 GeV. 

We employ a nonrelativistic QCD (NRQCD) Hamiltonian \cite{Lepage:1992tx} for the bottom quark. 
We tuned the bottom quark mass using the Fermilab prescription \cite{El-Khadra:1996wdx}, by equating 
the lattice extracted kinetic mass of the spin averaged 1S bottomonia $\{\overline M_{kin}^{\bar bb} = 0.75 M_{kin}(\Upsilon) + 0.25 M_{kin}(\eta_b)\}$
to its experimental value, where the kinetic mass is evaluated from the dispersion relation 
$aM_{kin}^2 = ((ap)^2 - (a\Delta E)^2)/2a\Delta E$. The details of NRQCD Hamiltonian, the improvement 
coefficients, and bottom quark mass tuning on our setup are discussed in Ref. \cite{Mathur:2016hsm}.

\bef[tbh]
% Figure removed
\caption{A landscape plot of the pseudoscalar masses corresponding to the quark mass that we have utilized 
in this work for different lattice ensembles used. The horizontal gray bands indicate a representative 
$M_{ps}$ estimate to guide the eye for a similar pseudoscalar meson mass across all four ensembles.} 
\eef{mpiVslat}

In this work, we assume isospin symmetry ($m_u = m_d$), and then for the channel that study here, 
involves three quark masses: the bottom ($b$), the charm ($c$), and the light ($u/d$) quarks.
For the light quark mass, we investigate five 
different cases: three unphysical quark masses discussed above [referred in terms of the 
corresponding approximate pseudoscalar meson masses $M_{ps}\sim$0.5, 0.6, and 1.0 GeV], the 
strange quark mass [$M_{ps}\sim$0.7 GeV] and the charm quark mass [$M_{ps}\sim$3.0 GeV]. In 
\fgn{mpiVslat}, we present the landscape of the five light quark masses studied in terms of the 
corresponding $M_{ps}$ versus the ensembles used. Using this setup, we evaluate the finite-volume spectrum in the 
isoscalar axialvector channel with $bc\bar u\bar d$ flavor for all these five quark masses on all 
four ensembles, next investigate the scattering of $D$ and $B^*$ mesons in all five scenarios and then 
extract the $m_{u/d}$ (otherwise $M_{ps}$) dependence of the scattering parameters. We utilize a wall-smearing procedure for all 
our quark propagator measurements (see Refs. \cite{Mathur:2018epb,Junnarkar:2018twb,Mathur:2022ovu} 
for details), and our primary focus on the finite-volume spectrum is on the ground state in each case. 






\section{Measurements and interpolators}\label{sec:2ptIO}

Lattice determination of finite-volume spectrum follows through an evaluation or measurement of Euclidean 
two-point correlation functions $\mathcal{C}_{ij}(t)$, of interpolating operators 
$\mathcal{O}_i(\mathbf{x},t)$ with desired quantum numbers, given by
\beq
\mathcal{C}_{ij}(t) = \sum_{\mathbf{x}}\left<\mathcal{O}_i(\mathbf{x},t)\mathcal{O}_j^{\dagger}(\mathbf{0},0)\right> = \sum_n \frac{Z_i^nZ_j^{n\dagger}}{2E^n} e^{-E^nt}.
\eeq{c2pt}
Here the second equality suggests that $\mathcal{C}_{ij}(t)$ can be expressed as a sum of exponentials 
following a spectral decomposition. $Z_i^n = \bra{0}\mathcal{O}_i\ket{n}$ is the operator-state overlap that 
quantifies the efficacy of the interpolator $\mathcal{O}_i$ in determining the time evolution of the state $n$. 
The utilization of wall smearing for the quark sources effectively kills all the high-momentum modes
at the source, whereas a zero momentum projection at the sink time slice ($\sum_{\mathbf{x}}$), as shown
in \eqn{c2pt}, efficiently projects the correlation function to the rest frame.  

Our main focus is on the ground state in the $T_{1}^+$ irreducible representation (irrep) in the rest frame, 
which is the only relevant rest frame finite-volume irrep for studying states in the infinite-volume continuum 
with quantum numbers ($J^P = 1^+$). To this end, we use a similar set of operators in the $T_{1}^+$ irrep as 
was utilized in Ref. \cite{Francis:2018jyb} and we briefly discuss them below for completeness. Assuming isospin symmetry, 
the relevant low-lying two-meson thresholds in the order of increasing energy are $E_{DB^*} = M_{B^*}+M_{D}$, 
$E_{BD^*}=M_B+M_{D^*}$, and $E_{D^*B^*}=M_B^*{}+M_{D^*}$. Hence, we consider the following low-lying two-meson 
interpolators 
\beqa
\mathcal{O}_1(x) &=& [\bar u(x) \gamma_i b(x)][\bar d(x) \gamma_5 c(x)]  \nonumber \\&& - [\bar d(x) \gamma_i b(x)][\bar u(x) \gamma_5 c(x)] \nonumber \\
\mathcal{O}_2(x) &=& [\bar u(x) \gamma_5 b(x)][\bar d(x) \gamma_i c(x)]  \nonumber \\&& - [\bar d(x) \gamma_5 b(x)][\bar u(x) \gamma_i c(x)] \nonumber \\
\mathcal{O}'(x) &=& \epsilon_{ijk} [\bar u(x) \gamma_i b(x)][\bar d(x) \gamma_j c(x)] \nonumber \\&& - [\bar d(x) \gamma_i b(x)][\bar u(x) \gamma_j c(x)].
\eeqa{mmops}
We utilize $\mathcal{O}_1(x)$ and $\mathcal{O}_2(x)$ in the computation of correlation functions. 
$\mathcal{O}'(x)$ has its associated two-meson threshold sufficiently higher up in the energy 
spectrum compared to the other two thresholds and it was found to have no effects in the low-lying 
energy spectrum. Hence we disregard this operator
from the rest of our analysis. Note that the lowest three particle threshold $DB\pi$ is above $E_{BD^*}$
for all the considered heavier-than-physical light quark masses. At $m_{u/d}^{phys}$, the $BD\pi$ threshold is 
immediately below $E_{BD^*}$, yet it remains sufficiently above $E_{DB^*}$ 
to have any significant effects on the ground states that we extract. We also compute two-point 
correlation functions for $B$, $B^*$, $D$, and $D^*$ mesons, using standard local quark 
bilinear interpolators ($\overline Q~\Gamma~q$) with spin structures $\Gamma\sim\gamma_5$ and 
$\gamma_i$ for pseudoscalar and vector quantum numbers, respectively. 

Phenomenologically, doubly bottom tetraquark in the axialvector channel is expected to be deeply 
bound. Such a state is expected to be quite compact owing to its doubly heavy flavor content 
and deeply bound nature \cite{Francis:2016hui,Czarnecki:2017vco}. Consequently, a local 
diquark-antidiquark interpolator is naturally interesting. Such an operator has already 
been utilized in all lattice QCD studies of the doubly bottom as well as bottom charm tetraquarks 
in the past \cite{Bicudo:2015kna,Francis:2016hui,Bicudo:2017szl,Junnarkar:2018twb,Leskovec:2019ioa,
Francis:2018jyb,Hudspith:2020tdf,Meinel:2022lzo,Hudspith:2023loy} and we follow the same strategy. 
Along with operators in \eqn{mmops}, we employ a local diquark-antidiquark interpolator 
\beq
\mathcal{O}_3(x) = (\bar u(x)^T \Gamma_5 \bar d(x) - \bar d(x)^T \Gamma_5 \bar u(x))( b(x) \Gamma_i c(x)),
\eeq{dadops}
where $\Gamma_k = C\gamma_k$ with $C=i\gamma_y\gamma_t$ being the charge conjugation matrix and 
the diquarks (antidiquarks) in the color antitriplet (triplet) representations. 

Our final basis is composed of the above-mentioned three interpolators $\{\mathcal{O}_1(x), \mathcal{O}_2(x), \mathcal{O}_3(x)\}$, 
which is diverse enough to reliably determine the ground state in the energy spectra that we are interested in.. Using this basis we determine 
the correlation matrices, with elements evaluated as prescribed in \eqn{c2pt}. Then the correlation matrices $\mathcal{C}$
are analyzed following a variational approach \cite{Michael:1985ne} to determine the energy estimates for 
low-lying levels in the spectrum. In this procedure, we look for the solutions of the generalized eigenvalue 
problem (GEVP) given by 
\beq
\mathcal{C}(t)v^n(t) = \lambda^n(t) \mathcal{C}(t_0)v^n(t),
\eeq{gevp}
where $t_0$ is a reference timeslice at which the eigenvalues $\lambda^n$s are identically unity. 
\bef[h]
% Figure removed
\caption{Effective energy plot for the eigenvalue correlation function $\lambda^0(t)$ (square) and for the product 
of single-meson correlators (circle) representing the noninteracting two-meson correlation function 
($\mathcal{C}_{D}(t)\mathcal{C}_{B^*}(t)$). The data correspond to $M_{ps} \sim 700$ MeV in the finest ensemble. 
The bands shown are the energy fit estimates for the final chosen time intervals.}
\eef{effmass}
The eigensolutions in the large time limit represent the lowest $N$ eigenstates $E^n$, for which the time 
evolution is dictated by the eigenvalues as $\lim_{t\to\infty}\lambda^n(t) \sim A_ne^{-E^nt}$. 
The corresponding eigenvectors are represented by $v^n(t)$, which are related to the operator-state-overlaps as
\beq
Z_i^{n}=\bra{0}\mathcal{O}_i \ket{n} = \sqrt{2E^n}(V^{-1})_i^n e^{E^{n}(t_0)/2},
\eeq{overlaps}
where $V$ is a matrix built out of $v^n(t)$. $v^n(t)$ is expected to be time independent in the limit, 
where the signal in $\mathcal{C}$ is saturated by the lowest $N$ eigenstates of the system.  

Conventionally the signal in the two point correlator data $C(t)$ is first assessed based on the 
large time plateauing in effective energies defined as $aE_{eff} = [ln(C(t)/C(t+\delta t))]/\delta t$. In 
\fgn{effmass}, we present the effective energies as a function of time for the eigenvalue correlation 
function (squares) and the noninteracting two-meson ($\mathcal{C}_{D}(t)\mathcal{C}_{B^*}(t)$) 
correlation function (circles). These effective energies can be seen to saturate around timeslices 
24 to 28 in the example shown. The results presented correspond to the lowest eigenvalue correlator 
$\lambda^0(t)$ at the strange quark mass ($M_{ps}\sim0.7$ GeV) in the finest ensemble we study. 
Evidently, there is a negative shift in the energies in $\lambda^0(t)$ with respect to the 
noninteracting energies at all times, except at very large times where the signal-to-noise ratio 
degrades substantially. 

Extraction of the energy spectra proceeds via fitting the eigenvalue correlators, $\lambda_{n}(t)$, 
with the expected asymptotic exponential behaviour. Alternatively, one can fit the asymptotic time 
estimates for the ratio of correlators given by 
\beq
R^n(t)=\frac{\lambda^n(t)}{\mathcal{C}_{m_1}(t) \mathcal{C}_{m_2}(t)}, 
\eeq{ratio}
to a single exponential form ($Ae^{-\Delta E^nt}$), where $\Delta E^n$ is expected to saturate to 
$E^n-M_{m_1}-M_{m_2}$ at large times. Here, $\mathcal{C}_{m_i}$ is the correlation function for 
the meson $m_i$, and $M_{m_i}$ is its mass. Being a ratio, $R^n(t)$ is empirically known to efficiently 
mitigate correlated noise between the product of two meson correlators and the interacting correlator 
for the two-meson system \cite{Green:2021qol}. Note that the automatic cancellation of the additive 
mass renormalization, inherent to NRQCD formulation, is an added advantage in using \eqn{ratio} for 
the fits. In \fgn{fitcompare}, we present a representative plot showing the $t_{min}$ dependence of 
the $\Delta E^n$ fit estimates determined from the fits to $\lambda^n(t)$ and $R^n(t)$, respectively, 
where $t_{min}$ is the lower boundary of the time interval used for these fits for a fixed upper boundary 
timeslice for the time interval. The energy differences are evaluated from $\lambda^n(t)$ using the relation 
$\Delta E^n = E^n-M_{m_1}-M_{m_2}$, where $M_{m_1}$ and $M_{m_2}$ are mass estimates for individual 
mesons determined from separate fits to $\mathcal{C}_{m_1}(t)$ and $\mathcal{C}_{m_2}(t)$, respectively. 
The estimates from different procedures can be seen to agree asymptotically in time, based on which 
optimal $t_{min}$ values are chosen. Our final results are based on fitting the ratio correlators 
defined in \eqn{ratio}.

    
\bef[h]
% Figure removed
\caption{$t_{min}$ dependence of the $\Delta E^0$ fit estimates determined from the fits to $\lambda^0$
and $R^0(t)$ for the case $M_{ps} \sim 700$ MeV in the finest ensemble. Here the superscript 0 refers 
to the ground state. }
\eef{fitcompare}



%%%%%%%%%%%%%%%%%%%%%%%%%%%%%%%%%%%%%%%%%%%%%%%%%%%%%%%%%%%%%%
\section{Energy spectra in finite-volume}\label{fvresults}
%%%%%%%%%%%%%%%%%%%%%%%%%%%%%%%%%%%%%%%%%%%%%%%%%%%%%%%%%%%%%%
In this section, we present our results that we obtain from the finite-volume correlators. 
After presenting the energy spectrum extracted using variational techniques, we discuss 
the operator-state-overlaps and the operator basis dependence. In the final subsection, 
we describe our strategy for rebuilding the ground state energies that are corrected for 
the additive NRQCD offset and for using them in further amplitude fits. 

\subsection{Details of energy spectra}
% Figure environment removed
In \fgn{spectrum}, we present the finite-volume energy spectra of the isoscalar 
axialvector $bc{\bar{u}}{\bar{d}}$ channel that we extract on the four ensembles 
listed in \tbn{lattice}, at the five different $m_{u/d}$ values corresponding to 
$M_{ps}\sim$ 0.5, 0.6, 0.7, 1.0, and 3.0 GeV. The energy spectrum is shown in 
lattice units. Note that these levels are shown with unaccounted additive 
renormalization measures related to the NRQCD-based dynamics of the heavy bottom quarks. 
The noninteracting two-meson energy levels corresponding to $DB^*$ and $BD^*$ thresholds 
are indicated as dotted horizontal line segments for each lattice and each $M_{ps}$. 
A clear trend for negative energy shifts can be observed in all the cases, indicating 
a possible attractive interaction between the scattering particles involved \cite{scalarbc}.
The $B^*D^*$ threshold in each case is also shown in the figure by dashed lines. 

From the energy spectra in the lattices $L_1$, $S_2$ and $S_3$, it can be observed 
that a consistent pattern emerges with respect to the two-meson thresholds.
The relative positioning of the ground state energy with the elastic threshold in the 
$S_1$ ensemble is also consistent with the other three ensembles. This is an encouraging 
feature in the finite-volume spectrum, as our main interest is on reliable extraction of the 
ground state energies. It is this ground state energy from each ensemble that we 
later on employ to constrain the $DB^*$ scattering amplitude.  

The excited states in the $S_1$ ensemble for $M_{ps}$, other than at the charm point, 
indicate enhanced negative shifts compared to that on the other ensembles. This could be 
related to a combination of effects arising from various less attractive features of the 
$S_1$ lattice, which includes the coarsest lattice spacing, small spatial volume and 
possible insufficient statistics for the study at lighter $M_{ps}$. To this end, we 
perform two additional checks. First, we make an associated study of the $S_1$ and the 
$L_1$ ensembles at the level of variational analysis and fitting procedures to determine 
the low-lying spectra with emphasis on the ground and the first excited states. We discuss 
this part of the investigation in Appendix \ref{app:S1L1}. Secondly, we perform amplitude 
fits with and without results from the $S_1$ ensemble to verify the robustness in our estimates 
for the scattering length. We discuss this in detail in Section \ref{Ampfits}.

\subsection{Operator-state overlaps}\label{sec:OSO}
\bef[hbt!]
% Figure removed
\caption{Normalized operator-state overlaps $\tilde{Z}_i^n$ for a state indicated by $n={0, 1, 2}$ 
and an operator represented by $\mathcal{O}_i$, where $i={1, 2, 3}$ on the $L_1$ ensemble. 
The errors in the normalized overlap factors are smaller than the symbols and hence are 
suppressed. The five horizontal panes stand for the five different $M_{ps}$ values we 
investigate. The two vertical lines in each horizontal pane separate $\tilde{Z}_i^n$ for 
different operators $\mathcal{O}_i$. }
\eef{Zratiosl40}
Now we investigate the operator-state overlaps $Z_i^n$, as in \eqn{overlaps}, to evaluate 
the efficacy of the interpolators in determining the low-lying spectra. To this end, 
we define normalized operator-state overlaps $\tilde{Z}_i^n$ such that its largest value 
for any given operator $\mathcal{O}_i$ across all the states $\{n\}$ is unity \cite{Dudek:2009qf,Padmanath:2013zfa}.
$\tilde{Z}_i^n$ quantifies the relative relevance of any given operator across all the 
states. In \fgn{Zratiosl40}, we present $\tilde{Z}_i^n$ at all $M_{ps}$ values we have used
on the $L_1$ ensemble. Each square marker corresponds to the $\tilde{Z}_i^n$ for a given operator 
$\mathcal{O}_i$ on to a given state $n$. Each horizontal pane stands for an $M_{ps}$ indicated on the 
right-hand side, whereas the vertical lines in each horizontal pane part $\tilde{Z}_i^n$ 
for different operators indicated on the top pane. The $x$-axis ticks refer to the three low-lying 
states we have extracted. $\mathcal{O}_1$, the two-meson operator related to $DB^*$ threshold, 
can be seen to have the largest overlap with the ground state and has significantly small 
overlaps with the excited states. $\mathcal{O}_2$, the two-meson operator related to $BD^*$ 
threshold, has the largest overlap with the first excited state and a very small overlap with 
the ground state. $\mathcal{O}_2$ also have nonnegligible overlap factors with the second 
excited state indicating $BD^*$-type two-meson Fock component, which decreases with increasing 
$M_{ps}$. On the other hand, $\mathcal{O}_3$, the diquark-antidiquark type operator, have 
substantial overlap factors with all states at the two lightest $M_{ps}$ values, whereas with 
an increased $M_{ps}$ its largest overlap is with the second excited state. Note that 
$\mathcal{O}_3$ is Fierz related to two-meson interpolators \cite{Padmanath:2015era}, and 
the large $\tilde{Z}_3^n$ values of $\mathcal{O}_3$ for all $n$ could be related to this 
underlying connection between two-meson and diquark-antidiquark operators. 


A summary from the above observations on overlap factors is as follows. $\mathcal{O}_1$
predominantly determines the ground state, whereas it has a significantly small coupling with 
the excited states. Similar patterns of overlap factors are also observed for other ensembles, 
all of which indicate that $\mathcal{O}_1$ predominantly determines the ground state. 
The two excited states have strong two-meson and diquark-antidiquark Fock components 
in the two lightest $M_{ps}$ values. The two-meson Fock components in the second excited state 
and the diquark-antidiquark Fock components in the first excited state decreases with 
increasing $M_{ps}$. This is consistent with the phenomenological expectation, which suggests 
that the binding energy in doubly heavy tetraquarks increases with increasing 
relative heaviness for the heavy quarks with respect to the light quarks \cite{Francis:2016hui,
Czarnecki:2017vco,Junnarkar:2018twb}. A deeply bound state could be significantly compact 
and hence could have large Fock components of a compact object such as that of a 
diquark-antidiquark. In other words, the relevance of compact diquark-antiquark operators for 
the low-lying spectrum increases with decreasing light quark mass, as is evident from \fgn{Zratiosl40}. 

\subsection{Operator basis dependence}
\bef[tbh!]
% Figure removed
\caption{Operator basis dependence of the low energy spectra of the $L_1$ ensemble and 
$M_{ps}\sim$700 MeV for all possible operator basis that can be built out of the three operators 
discussed in Section \ref{sec:2ptIO}. The basis is presented in digital notation ($x$-axis tick labels) 
where the operators are arranged in the order $\{\mathcal{O}_1, \mathcal{O}_2, \mathcal{O}_3\}$.
The horizontal lines refer to the $DB^*$, $BD^*$, and $D^*B^*$ thresholds. The bands indicate the 
bounds of the ground and first excited state energy estimates from the full three-operator basis. }
\eef{basisdep}
Next, we look into the basis dependence of the finite-volume energy spectra presented in 
\fgn{spectrum}. In \fgn{basisdep}, we show this basis dependence as determined for $M_{ps}\sim$ 
700 MeV in the $L_1$ ensemble, for various operator basis build out of $\mathcal{O}_1$, 
$\mathcal{O}_2$, and $\mathcal{O}_3$ operators as defined in \eqn{mmops} and \eqn{dadops}. The digital 
indexing on the $x$-axis tick labels refers to various operator basis in the order 
$\{\mathcal{O}_1, \mathcal{O}_2, \mathcal{O}_3\}$, with an overline on the third index 
as a visual aid within the plot to highlight the diquark-antidiqaurk interpolator. 1 (0) 
indicates an operator is included in (excluded from) the basis. The horizontal 
lines refer to the $DB^*$, $BD^*$ and $B^*D^*$ thresholds. The gray horizontal bands 
refer to the two lowest levels in the full basis indicated by $11\overline{1}$. A level 
below the threshold appears only when $\mathcal{O}_1$ is present in the basis. The first 
excited state in the full basis $11\overline{1}$ is faithfully reproduced in those bases 
where $\mathcal{O}_2$ is included. $\mathcal{O}_3$ alone does not precisely determine 
any level in the energy spectrum using full basis. Similar observations are also made 
on other ensembles. 

In summary, the ground state in the full basis $11\overline{1}$ is reliably determined 
with $\mathcal{O}_1$ and is unaffected by the inclusion of $\mathcal{O}_2$ and 
$\mathcal{O}_3$ operators. The excited states have nonnegligible overlap factors with 
$\mathcal{O}_2$ and $\mathcal{O}_3$ operators. Given our setup with only a few energy 
levels, any assumption more complicated than a simple elastic $DB^*$ assumption for 
the amplitude fits is beyond the scope of this work. Such an assumption is justified within the 
isosymmetric limit as the lowest inelastic threshold ($BD^*$ at unphysically heavy 
$m_{u/d}$ or $BD\pi$ for $m_{u/d}^{phys}$) is significantly high. In light of all 
these observations above, we limit ourselves to using only ground states determined 
from all ensembles at various $M_{ps}$ values to constrain the elastic $S$-wave
$DB^*$ scattering amplitude.

\subsection{The ground state energies}

\begin{figure}[h]
% Figure removed
\caption{The ground state energies in units of the elastic threshold ($DB^*$) on all 
ensembles (see \tbn{lattice} for color-symbol conventions) for all $M_{ps}$ values
(different vertical panes).}
\eef{gsspectrum}

In this subsection, we discuss how we obtain ground state energies after adjusting the 
additive correction that is inherent to an NRQCD calculation. The set of numbers that 
we extract from our variational analysis and eigenvalue correlator fitting procedures 
are the single meson masses $M_{B}$, $M_{B^*}$, $M_{D}$, and $M_{D^*}$ and the energy 
splittings $\Delta E_n = E_n - M_{m_1} - M_{m_2}$ (see \eqn{ratio}). %for the interacting system from the reference 
%two meson level $m_1m_2$ determined from the ratio of correlators defined in \eqn{ratio}. 
First, we account for the NRQCD corrections in single meson masses (involving a bottom quark) as 
\beq
\tilde{M}_{B^{(*)}} = M_{B^{(*)}} - 0.5\overline M^{\bar bb}_{lat} + 0.5 \overline M^{\bar bb}_{phys},
\eeq{msnNRcor}
where $\overline M^{\bar bb}_{lat}(\overline M^{\bar bb}_{phys})$ refers to the spin averaged mass 
of the $1S$ bottomonium measured on the lattice (experiments). We follow this procedure as we 
have tuned the bottom quark mass through the spin average bottomonia at each ensemble.

For the interacting energy spectrum, the NRQCD offset is automatically canceled in the energy splittings
$\Delta E^n$. One can then build the energy estimates $\tilde{E}^n$ of interacting spectrum by adding 
the noninteracting level energy ($M_{m_1} + M_{m_2}$) with the energy splittings $\Delta E^n$ as,
\beq 
\tilde{E}^n = \Delta E^n + M_{m_1} + M_{m_2}.
\eeq{intNRcor}
If either $m_1$ or $m_2$ is a bottom meson, we use the corresponding corrected $\tilde{M}_{m_i}$\footnote{From 
the next section, for brevity we suppress the $\tilde{}$ notation indicating corrected masses and energies.} 
determined using \eqn{msnNRcor}, instead of $M_{m_i}$. 

In \fgn{gsspectrum}, we present the corrected ground state energy estimates, 
in units of the energy of elastic threshold $E_{DB^*}$, at various $M_{ps}$ and 
for all the ensembles we have employed. The spectrum clearly shows a trend of decreasing 
energy spitting, hence decreasing interaction strength, with increasing $M_{ps}$. 
Another feature worth noting here is that the lattice spacing dependence of the 
ground state energies on similar volume ensembles ($S_1$, $S_2$, $S_3$) for the 
non-charm $M_{\pi}$ are opposite to that at the charm point. We will revisit this 
point when we discuss extraction of $DB^*$ scattering amplitude using these energy levels. 



%%%%%%%%%%%%%%%%%%%%%%%%%%%%%%%%%%%%%%%%%%%%%%%%%%%%%%%%%%%%%%
\section{$\mathbf{DB^*}$ scattering amplitude}\label{Ampfits}
%%%%%%%%%%%%%%%%%%%%%%%%%%%%%%%%%%%%%%%%%%%%%%%%%%%%%%%%%%%%%%

\subsection{Strategy}

The finite-volume energy splittings determined in the previous section are related to 
the infinite-volume scattering physics via L\"uscher's finite-volume prescription 
\cite{Luscher:1990ux} and its generalizations, e.g. \cite{Briceno:2014oea}. Assuming 
these energy splittings are purely described by an elastic scattering in the $DB^*$ 
system, we utilize them to constrain the associated $S$-wave scattering amplitude. Here 
we consider only the ground states in all ensembles for all quark mass scenarios, as 
the excited states are found to be affected by the inelastic $BD^*$ channel. 

It is interesting that even the excited states are also found to have statistically 
significant shifts with respect to the inelastic $BD^*$ threshold (see \fgn{spectrum}), 
possibly indicating nontrivial interactions between $B$ and $D^*$ mesons. If the $DB^*$ 
and $BD^*$ channels were totally decoupled, such shifts point to equivalent interactions 
in both channels \cite{bdsbc}. However, independent elastic analysis for the excited 
states is not well justified. On the other hand, the inclusion of excited states in our 
analysis demands an inelastic treatment involving more parameters than the available 
degrees of freedom in the amplitude fits, which is beyond the scope of this work. We 
also assume only negligible effects from higher partial waves or any off-shell pion 
exchange interactions that can induce coupling between $DB^*$ and $BD^*$ channels 
\cite{Du:2023hlu}, for the same reason. 

\subsection{Amplitude fits and continuum extrapolations}
For the scattering of a $D$ and a $B^*$ meson in the $S$-wave leading to total angular 
momentum and parity $J^P=1^+$, the scattering phase shifts $\delta_{l=0}(k)$ are related to 
the finite-volume energy spectrum through \cite{Luscher:1990ux}:
\beq
kcot[\delta_0(k)] = \frac{2Z_{00}[1;(\frac{kL}{2\pi})^2)]}{L\sqrt{\pi}},
\eeq{luscher}
where $k$ is the momentum of either mesons in the center of momentum frame corresponding 
to the center of momentum energy $E_{cm}=\sqrt{s}$. $k$ and $E_{cm}$ are related to each other through 
\beq
4sk^2 = (s-(M_{D}+M_{B^*})^2)(s-(M_{D}-M_{B^*})^2).
\eeq{k2cm}
A sub-threshold pole singularity in the $S$-wave scattering amplitude $t = ({\mathrm{cot}}\delta_0 - i)^{-1}$ 
occurs when $k{\mathrm{cot}}\delta_0 = \pm\sqrt{-k^2}$ 
\bet[hb]
  \begin{center}
          \begin{tabular}{p{2.0cm}p{2.0cm}p{2.0cm}>{\hfill\arraybackslash}p{2.cm}}
      \hline
      \hline
$M_{ps}$ [GeV] & $\chi^2/d.o.f$ & $A^{[0]}/E_{DB^*}$ & $A^{[1]}/E_{DB^*}$ \\\hline
\multirow{2}{*}{0.5} & 2.1/2 & $-0.05(1)$ & $~0.17(_{-11}^{+13})$ \\\cline{2-4} 
                     & 1.3/1 & $-0.05(1)$ & $~0.13(_{-12}^{+13})$ \\ \hline
\multirow{2}{*}{0.6} & 0.5/2 & $-0.044(_{-8}^{+9})$ & $~0.10(_{-9}^{+9})$ \\ \cline{2-4} 
                     & 0.3/1 & $-0.043(_{-8}^{+9})$ & $~0.09(_{-10}^{+9})$ \\ \hline
\multirow{2}{*}{0.7} & 3.0/2 & $-0.042(_{-6}^{+8})$ & $~0.09(_{-7}^{+6})$ \\ \cline{2-4} 
                     & 1.5/1 & $-0.040(_{-6}^{+8})$ & $~0.06(_{-8}^{+6})$ \\ \hline
\multirow{2}{*}{1.0} & 2.9/2 & $-0.043(4)$ & $~0.11(_{-5}^{+5})$ \\ \cline{2-4} 
                     & 0.4/1 & $-0.041(4)$ & $~0.14(_{-4}^{+5})$ \\ \hline
\multirow{2}{*}{3.0} & 3.6/2 & $~0.006(_{-5}^{+6})$ & $-0.20(_{-5}^{+4})$ \\ \cline{2-4} 
                     & 1.9/1 & $~0.010(_{-5}^{+6})$ & $-0.25(_{-5}^{+4})$ \\ \hline
      \hline
  \end{tabular}
  \end{center}
\caption{Results from amplitude fits for different light quark mass scenarios indicated 
in terms of $M_{ps}$ in the first column. For each $M_{ps}$, two independent fits are 
performed with (top row) and without (bottom row) the level from $S_1$ ensemble. All fits 
are performed with the parameterization in \eqn{linparam}, where the optimized parameter 
values in the table are presented in units of the $DB^*$ threshold, $E_{DB^*}$. }
\eet{Ampfits1}
for scattering in $S$-wave. We follow the procedure outlined in Appendix B of Ref. 
\cite{Padmanath:2022cvl} in constraining the amplitude, such that the parametrization of 
$k{\mathrm{cot}}\delta_0$ is tuned to satisfy Eq. (\ref{luscher}). The parametrized 
$k{\mathrm{cot}}\delta_0$ is then investigated for poles of $t$ in the complex energy plane.  

\begin{figure}[h]
% Figure removed
\caption{$k{\mathrm{cot}}\delta_0$, in units of the elastic threshold $E_{DB^*}$, versus $a$ 
(lattice spacing) for all $M_{\pi}$ values. We follow the marker/color coding in \tbn{lattice} 
for the data points referring to the simulated data. The colored/gray bands indicate the fit 
results to the continuum extrapolation fit form in \eqn{linparam} with/without the data from 
$S_1$ ensemble.} 
\eef{alatdep}
Since we use only the ground states for amplitude fits, we limit ourselves to a scattering 
amplitude parametrization that is completely described by scattering length $a_0$ in an effective 
range expansion near the threshold. Additionally, we also consider a lattice spacing dependence 
on the parametrization of $k{\mathrm{cot}}\delta_0$. We find that a linear functional form
given by 
\beq
k{\mathrm{cot}}\delta_0 = A^{[0]} + aA^{[1]}
\eeq{linparam}
provides acceptable fits to the scattering amplitudes. Such an $a$ dependence was also found to 
be necessary in our previous investigations using NRQCD framework as well \cite{Mathur:2022ovu}, 
and is consistent with the leading $a$ dependence of observables
involving an NRQCD evolution. In this form, $A^{[0]}=-1/a_0$, where $a_0$ is the scattering 
length in the continuum limit. 
We list the results from different amplitude fits in \tbn{Ampfits1}. In \fgn{alatdep}, we present 
the quality of these fits by comparing the fit results with the data points. The colored/gray 
bands indicate the fit results including/excluding results from the $S_1$ ensemble to the respective 
fits. It can be clearly seen that fit results are less affected by inputs from $S_1$ ensemble, 
which is obvious given the large uncertainties associated with them, in contrast to inputs from 
$L_1$, $S_2$, and $S_3$. In \fgn{pcotdelta_summary}, we present $k{\mathrm{cot}}\delta_0$ versus 
$k^2$ based on the ground state energies presented in \fgn{gsspectrum} following \eqn{luscher}. 
The colored/gray bands indicate continuum extrapolated results including/excluding results from 
the $S_1$ ensemble to the respective fits. Clearly, there are no statistically significant effects
from the inclusion/exclusion of the energy levels from the $S_1$ ensemble observed.
\begin{figure}[h]
% Figure removed
\caption{$k{\mathrm{cot}}\delta_0$ versus $k^2$ for all $M_{\pi}$ values studied in units of 
the elastic threshold $E_{DB^*}$. The data points refer to the simulated data and follow 
the color coding in \tbn{lattice}. The dashed orange (cyan) curve indicates the constraint 
for the existence of a sub-threshold pole in the scattering amplitude. The horizontal bands
are the continuum extrapolated estimates of $k{\mathrm{cot}}\delta_0$ for the respective
$M_{\pi}$ (see \fgn{alatdep}). }
\eef{pcotdelta_summary}

Our main aim is to reliably determine $A^{[0]}=-1/a_0$, the sign of which determines the fate 
of the near threshold pole, if there exists one. A negative (positive) value of $A^{[0]}$($a_0$) 
indicates that the interaction potential is strong enough to form a real bound state\cite{Landau:1991wop}. 
It can be seen from \tbn{Ampfits1} and \fgn{alatdep} that for the non-charm light quark masses, 
$A^{[0]}$, the continuum extrapolated value for $k{\mathrm{cot}}\delta_0$ is negative, which 
indicates a possibly strong attractive interaction sufficient enough to host a real bound state. 
Whereas at the charm point, despite the unambiguous negative energy shifts in the finite-volume 
ground state energies with respect to the elastic threshold, the attraction is weak to host any 
real bound state as suggested by the positive value of $k{\mathrm{cot}}\delta_0$ in the continuum 
limit. This observation goes in line with the phenomenological expectation for doubly heavy four 
quark ($QQ'l_1l_2$) systems with $m_{l_1}=m_{l_2}$ that the binding increases with increased 
relative heaviness of the heavy quarks with respect to its light quark content
\cite{Francis:2016hui,Czarnecki:2017vco,Junnarkar:2017sey}. 

Another interesting observation is related to the lattice spacing dependence of $k{\mathrm{cot}}\delta_0$
values. At the charm point, $A^{[1]}$ (see \eqn{linparam}) acquires a different signature in contrast 
to that for the light quark masses. This suggests that for a doubly heavy four quark ($QQ'l_1l_2$) system 
with $(m_{l_1} = m_{l_2}, ~m_{Q},m_{Q'}>>m_{l})$, the cut off effects weaken the finite-volume energy 
splitting of the ground state with the elastic threshold. On the other hand, at the charm point (where 
$m_{Q},m_{Q'}\sim m_{l}$) such effects enhance this energy splitting in the $QQ'l_1l_2$ system determined 
in a finite-volume. Relatively large errors at the noncharm $M_{ps}$ values partially obscure these effects, 
if any exist, while at the charm point such effects are clearly reflected. 

\subsection{Light quark mass dependence}
Following the individual amplitude fits to different light quark mass cases, now we investigate the light 
quark mass ($m_{u/d}$) or $M_{ps}$ dependence of the parameters $A^{[0]}$ and $A^{[1]}$. Due to leading order 
$M_{ps}^2$ terms in the chiral expansion, we assume the $M_{ps}$ dependence of hadron masses for 
light $m_{u/d}$ values ($m_q\lesssim\Lambda_{QCD}$) to be linear in $M_{ps}^2$. Whereas towards the heavy 
$m_{u/d}$ regime ($m_q>>\Lambda_{QCD}$) heavy hadron masses are expected to be proportional to the quark mass, 
hence to $M_{ps}$ \cite{Neubert:1993mb}. With these assumptions, we work with three following fit forms that 
could be useful. 
\beqa
	f_l(M_{ps}) &=& \alpha_c + \alpha_l M_{ps}, \nonumber \\
	f_s(M_{ps}) &=& \beta_c + \beta_s M_{ps}^2, \mbox{~~~and} \nonumber \\
	f_q(M_{ps}) &=& \theta_c + \theta_l M_{ps} + \theta_s M_{ps}^2.
\eeqa{mqdep}
Fits to determine the $M_{ps}$ dependence were made by minimizing a single cost function 
defined combinedly for $A^{[0]}$ and $A^{[1]}$ as %\cite{FullLuscher}
\beq
	\chi^2 =\sum_{\substack{x, y \\ \in \{A^{[j]}_{i}\}}}\left(f_x-f_{px}(M_{ps})\right)\tilde{\mathcal{C}}^{-1}_{xy}\left(f_y-f_{py}(M_{ps})\right),
\eeq{chi2mqdep}
where the summation runs over all fitted parameters $\{A^{[j]}_{i}\}$ with $j\in\{0, 1\}$ and $i$ 
referring to the five different light quark masses studied. In \tbn{Ampfits1}, we list the fit results 
for $f_{x,y}$. $\tilde{\mathcal{C}}_{ij}$ is the associated data covariance determined following Ref. \cite{Prelovsek:2020eiw}. 
$f_{pn}(M_{ps})$ are the fit forms incorporating the $m_{u/d}$ dependence in parameters $\{A^{[j]}_{i}\}$.  
In \fgn{a0a1_separate}, we show the fit results for $A^{[0]}=-1/a_0$ to the fit forms in \eqn{mqdep}. The large 
circles represent the $A^{[0]}$ values at different $M_{ps}$, the bands represent the fit results 
with different fit forms in \eqn{mqdep}, and the two stars represent $A^{[0]}$ at the physical 
$M_{ps}$ (equivalently the physical scattering length $a_0^{phys}$) and the critical $M_{ps}$ at which 
$A^{[0]}$ changes its sign (positive to negative), in other words, the system becomes unbound. It is 
indeed desired to have more points in the intermediate mass regime between the charm and the strange 
\begin{figure}[h]
% Figure removed
\caption{Continuum extrapolated $k{\mathrm{cot}}\delta_0$ or $A^{[0]}=-1/a_0$ estimates of the $DB^*$ system 
as a function of $M_{ps}^2$ in units of $E_{DB^*}$. The band indicates fit results to the simulated results. 
The legend carries info on the fit forms presented (see also \eqn{mqdep}) and the quality of fits. The dotted 
vertical line close to the $y$-axis indicates the physical $M_{ps}$. The two star symbols represent the 
amplitude at the physical $M_{ps}$ and the critical $M_{ps}$ at which the system becomes unbound.}
\eef{a0a1_separate}
quark masses to further constrain the dependence. Yet, our fits in this work demonstrate near independence in 
the fit forms as can be observed from the consistency between the error bands from different fit 
forms. 

\begin{figure}[h]
% Figure removed
\caption{The landscape of the continuum scattering length $A^{[0]}$ versus $A^{[1]}$ (see \eqn{linparam}) 
for all $M_{ps}$ values (indicated in the legend) studied. The central values are represented by black 
edged circles with color fillings, whereas the scattered points are the bootstrap samples. The band 
represents the correlated $M_{ps}$ dependence of the fitted parameters.} 
\eef{a0a1_combined}
Next we look at the correlated pion mass dependence in the parameters $A^{[0]}$ and $A^{[1]}$ (see 
\eqn{chi2mqdep} for the definition of the cost function) presented in \fgn{a0a1_combined}. The black 
bordered symbols are the central values of parameters determined for each $M_{ps}$, whereas scattered 
small circles indicate the bootstrap sample distribution in the $[A^{[0]},~A^{[1]}]$ landscape. The bands 
in the figure represent the uncertainty in the parameters, with the inner band quantifying the statistical 
errors, while the outer band also incorporates the systematic uncertainty arising from different fit forms 
added in quadrature symmetrically. A negative correlation can clearly be observed between the parameters 
across different quark masses studied, which is accounted in the fits through the data covariance matrix 
entering the cost function. This correlation can also be observed within the distribution of the bootstrap 
samplings at all quark masses. This observation clearly demonstrates the need for a careful treatment of 
cutoff errors, particularly in heavy hadron systems with interesting near threshold features, such as this. 

In the chiral regime ($m_{u/d}\lesssim\Lambda_{QCD}$), leading $m_{u/d}$ dependence in hadronic observables 
is assumed to go as linear in $M_{ps}^2$. Based on the fit form $f_s(M_{ps})$, we find that the scattering length 
of the $DB^*$ system at the physical light quark mass ($m_{u/d}^{phys}$) to be
\beq
a_0^{phys} = 0.57(^{+4}_{-5})(17) \mbox{~fm}.
\eeq{scatlen}
The asymmetric errors indicate the statistical uncertainties, whereas the second parenthesis quotes 
the systematic uncertainties with the most dominant contribution arising from the chiral extrapolation 
fit forms. We elaborate on various systematic uncertainties towards end of this section. The positive 
value of the scattering length is an unambiguous evidence for the ability/strength of the hadron-hadron 
interaction potential to host a real bound state (when $k~{\mathrm{cot}}\delta_0 = -\sqrt{-k^2}$). 
The observed scattering length at physical light quark mass suggests the presence of a real $bc\bar u\bar d$ 
tetraquark bound state $T_{bc}$ with binding energy 
\beq
\delta m_{T_{bc}} = -43(^{+6}_{-7})(^{+14}_{-24}) \mbox{~MeV},
\eeq{betbc}
with respect to $E_{DB^*}$. The systematic effects on the $a_0^{phys}$ and $\delta m_{T_{bc}}$ estimates 
of ignoring the charm point in the fits to the $m_{u/d}$ dependence are found to be very small, 
compared to the number quoted for systematic uncertainties in \eqn{scatlen}. 

Towards the heavy quark regime ($m_{u/d}>>\Lambda_{QCD}$), the heavy hadron masses can have 
leading linear dependence in $M_{ps}$ as $M_{ps}\propto m_{u/d}\sim m_{Q}$ \cite{Neubert:1993mb}. 
Following the fit form $f_l(M_{ps})$, which is linear in quark mass, the critical light quark mass 
$m_{u/d}^*$ at which the scattering length diverges, then changes its signature such that the 
interaction potential is not able host a real bound state, corresponds to the critical pseudoscalar 
meson mass given by 
\beq
M^{*}_{ps} = 2.73(21)(14) \mbox{~GeV}.
\eeq{unitary}
This corresponds to the star symbol at the zero crossing in the $x$-axis ($A^{[0]}=0$) in \fgn{a0a1_separate}. 
Once again the first parenthesis indicates the statistical errors and the second one quantifies various 
systematic uncertainties added in quadrature. 


Now we briefly comment on other possible sources of systematic uncertainties in this calculation. Our lattice setup, 
discussed in Section \ref{sec:lattice}, together with the bare bottom and charm quark mass tuning procedure 
has been demonstrated to reproduce the $1S$ hyperfine splittings in quarkonia with uncertainties less than 6 MeV
\cite{Mathur:2022ovu,Mathur:2016hsm}. Additionally, our strategy of evaluating the energy differences 
and working with mass ratios has also been shown to significantly mitigate the systematic uncertainties related 
heavy quark masses \cite{Mathur:2018epb,Mathur:2022ovu}. Our fitting procedure discussed in Section \ref{sec:2ptIO} 
involves careful and conservative determination of statistical errors, and uncertainties related to the 
excited-state-contamination and fit-window errors. The amplitude determination and followed extrapolations
are performed with results from varying the fit-windows to evaluate the uncertainties propagated to our final 
results. The uncertainties related to the fit forms used in chiral extrapolations are observed to be dominant,  
and the number in the second parenthesis in Eqs. \ref{scatlen}, \ref{betbc}, and \ref{unitary} are the total 
systematic uncertainties added in quadrature. Uncertainty related to scale setting are also found to be negligibly 
small in comparison to the statistical uncertainties \cite{Mathur:2018epb,Mathur:2022ovu}. 


%!TEX root = ../Schur indices and line operators.tex


\section{Discussion}












\begin{table*}[t]
\caption{Summary of the top-performing teams in each track of the RoboDepth Challenge.}
\centering\scalebox{1}{
\begin{tabular}{c|p{5cm}|p{5cm}}
\toprule
\textbf{Rank} & \textbf{\#1: Robust Self-Supervised MDE} & \textbf{\#2: Robust Supervised MDE}
\\\midrule\midrule
\multirow{13}{*}{\textcolor{robo_blue}{\textbf{1st Place}}} & \textbf{Team Name} & \textbf{Team Name}
\\
& \textcolor{robo_blue}{OpenSpaceAI} & \textcolor{robo_blue}{USTCxNetEaseFuxi}
\\
\cmidrule{2-3}
& \textbf{Team Members} & \textbf{Team Members}
\\
& Ruijie Zhu$^1$, Ziyang Song$^1$, Li Liu$^1$, Tianzhu Zhang$^{1,2}$ & Jun Yu$^1$, Mohan Jing$^1$, Pengwei Li$^1$, Xiaohua Qi$^1$, Cheng Jin$^2$, Yingfeng Chen$^2$, Jie Hou$^2$
\\
\cmidrule{2-3}
& \textbf{Affiliations} & \textbf{Affiliations}
\\
& $^1$University of Science and Technology of China, $^2$Deep Space Exploration Lab & $^1$University of Science and Technology of China, $^2$NetEase Fuxi
% \\
% \cmidrule{2-3}
% & \textbf{Approach} & \textbf{Approach}
% \\
% & IRUDepth with MPViT as depth encoder and PoseNet for camera poses and depth maps with AugMix& <...>
\\\cmidrule{2-3}
& \textbf{Contact} $\textrm{\Letter}$ & \textbf{Contact} $\textrm{\Letter}$
\\
& \texttt{ruijiezhu@mail.ustc.edu.cn} & \texttt{USTC\_IAT\_United@163.com}
\\\midrule\midrule
\multirow{17}{*}{\textcolor{robo_red}{\textbf{2nd Place}}} & \textbf{Team Name} & \textbf{Team Name}
\\
& \textcolor{robo_red}{USTC-IAT-United} & \textcolor{robo_red}{OpenSpaceAI}
\\
\cmidrule{2-3}
& \textbf{Team Members} & \textbf{Team Members}
\\
& Jun Yu$^1$, Xiaohua Qi$^1$, Jie Zhang$^2$, Mohan Jing$^1$, Pengwei Li$^1$, Zhen Kan$^1$, Qiang Ling$^1$, Liang Peng$^3$, Minglei Li$^3$, Di Xu$^3$, Changpeng Yang$^3$ & Li Liu$^1$, Ruijie Zhu$^1$, Ziyang Song$^1$, Tianzhu Zhang$^{1,2}$
\\
\cmidrule{2-3}
& \textbf{Affiliations} & \textbf{Affiliations}
\\
& $^1$University of Science and Technology of China, $^2$Central South University, $^3$Huawei Cloud Computing Technology Co., Ltd & $^1$University of Science and Technology of China, $^2$Deep Space Exploration Lab
\\
\cmidrule{2-3}
& \textbf{Contact} $\textrm{\Letter}$ & \textbf{Contact} $\textrm{\Letter}$
\\
& \texttt{USTC\_IAT\_United@163.com} & \texttt{liu\_li@mail.ustc.edu.cn}
\\\midrule\midrule
\multirow{11}{*}{\textcolor{robo_green}{\textbf{3rd Place}}} & \textbf{Team Name} & \textbf{Team Name}
\\
& \textcolor{robo_green}{YYQ} & \textcolor{robo_green}{GANCV}
\\
\cmidrule{2-3}
& \textbf{Team Members} & \textbf{Team Members}
\\
& Yuanqi Yao$^1$, Gang Wu$^1$, Jian Kuai$^1$, Xianming Liu$^1$, Junjun Jiang$^1$ & Jiamian Huang$^1$, Baojun Li$^1$
\\
\cmidrule{2-3}
& \textbf{Affiliations} & \textbf{Affiliations}
\\
& $^1$Harbin Institute of Technology & $^1$Individual Researcher
\\
\cmidrule{2-3}
& \textbf{Contact} $\textrm{\Letter}$ & \textbf{Contact} $\textrm{\Letter}$
\\
& \texttt{yuanqiyao@stu.hit.edu.cn} & \texttt{huang176368745@gmail.com}
\\\bottomrule
\end{tabular}
}
\label{tab:summary}
\end{table*}
%%%%%%%%%%%%%%%%%%%%%%%%%%%%%%%%%%%%%%%%%%%%%%%%%%%%%%%%%%%%%%%%%%%%%%%%%%%%%%%%
%% ACKNOWLEDGMENTS
\begin{acknowledgments}
%
This work is supported by the Department of Atomic Energy, Government of India, under Project Identification Number RTI 4002. We are thankful to the MILC collaboration and in particular to S. Gottlieb for providing us with the HISQ lattice ensembles. We thank Sara Collins for a careful reading of the manuscript. We thank the authors of Ref. \cite{Morningstar:2017spu} for making the {\it TwoHadronsInBox} package utilized in this work. We also thank Gunnar Bali, Parikshit Junnarkar and Sayantan Sharma for discussions. Computations were carried out on the Cray-XC30 of ILGTI, TIFR. Amplitude analyses were performed on Nandadevi computing cluster at IMSc Chennai. N. M. would also like to thank A. Salve and K. Ghadiali for computational support.
\end{acknowledgments}



\begin{comment}
\section{System Architecture}
\label{appendix:architecture}
\system has a novel modularized system architecture with three key components: 
\emph{StreamManager}, 
\emph{TxnManager} and \emph{TxnScheduler}. 
These components are instantiated in each thread locally.
The execution outline of \system is presented in Algorithm~\ref{alg:algo}.
Transactional stream processing is continuous and potentially never ends (Line 1$\sim$8).
The dependency resolution and execution of state transactions are separated into two non-overlapping phases by punctuations~\cite{Tucker:2003:EPS:776752.776780} (Line 2 and 5), which guarantees that no subsequent input event will have a smaller timestamp. 
Effectively, a batch of state transactions is collected during the first phase, and processed during the second phase.

In the first phase (i.e., stream processing phase), 
the \emph{StreamManager} conducts preprocessing for every input event ($e$). Similar to some prior works~\cite{tstream}, state transactions may be issued but not immediately processed during preprocessing (Line 3).
The \emph{pre\_processing} and \emph{post\_processing} functions are exposed as APIs to users.
The \emph{TxnManager} handles dependency resolution (Line 4) among state transactions and insert decomposed operations to construct a \tpg. We discuss the detailed two-phase \tpg construction process in Section~\ref{subsec:construction}.

In the second phase  (i.e., transaction processing phase), 
the \emph{TxnManager} is first involved again to refine (Line 6) the constructed \tpg with further dependency resolution.
The \emph{TxnScheduler} 
schedules operations for concurrent execution based on the constructed \tpg according to the three dimensions of scheduling decisions (Line 7). 
In particular, a scheduling decision model $M$ is instantiated based on the constructed \tpg (Line 14).
\textbf{\circled{1}} Guided by $M$, execution threads adopt an exploration strategy (Section~\ref{subsec:explore}) to explore the constructed \tpg for operations available to be scheduled constrained by dependencies. 
\textbf{\circled{2}} 
During exploration, one or multiple operations may be treated as the 
% basic 
unit of scheduling (Section~\ref{subsec:granularity}). 
Subsequently, \textbf{\circled{3}} every thread executes operation(s) in the unit of scheduling with various abort handling mechanisms (Section~\ref{subsec:abort_handling}).
Only when state transactions are processed (i.e., committed or aborted) can the associated input events be postprocessed (Line 8) by the \emph{StreamManager} based on transaction processing results.
\end{comment}

\begin{comment}
\begin{algorithm}
\footnotesize
    \KwData{$e$ \tcp{Input event}}
    \KwData{$txn_{ts}$ \tcp{State transaction}}
    \KwData{$G$ \tcp{The currently constructed TPG}}
    \While{!finish processing of input streams}{
        \eIf(\tcp*[h]{Phase 1}){\text{$e$ is not a $punctuation$}}{
                $txn_{ts}$ $\gets$ PRE\_Processing($e$)\;
                \textbf{TPG\_Construction}($G$, $txn_{ts}$)\; 
          }(\tcp*[h]{Phase 2}){
                \textbf{TPG\_Refinement}($G$)\; 
                \textbf{TXN\_Scheduling}($G$)\; 
                POST\_Processing()\;
          }
    }
    
    \SetKwFunction{FMain}{TPG\_Construction}
    \SetKwProg{Fn}{Function}{:}{}
    \Fn{\FMain{$G$, $txn_{ts}$}}{
        $O_{1..k}$ $\gets$ \textbf{Partition} $txn_{ts}$\;
        \ForEach{\text{operation $O_{i}$ $\in$ $O_{1..k}$}}{
            \textbf{Identify} its \ld\;
            $G$ $\gets$ $G$ + $O_{i}$ \;
        }
    }
    \SetKwFunction{FMain}{TPG\_Refinement}
    \SetKwProg{Fn}{Function}{:}{}
    \Fn{\FMain{$G$}}{
        \ForEach{\text{vertex $e_{i}$ $\in$ $G$}}{
            \textbf{Identify} its \td, \pd\;
        }
    }
    
    \SetKwFunction{FMain}{TXN\_Scheduling}
    \SetKwProg{Fn}{Function}{:}{}
    \Fn{\FMain{$G$}}{
        $M$ $\gets$ Instantiated with $G$;\tcp{A decision model}
        \While{!finish scheduling of $G$
        }{
          \textbf{\circled{2}} $Scheduling Unit$ $\gets$ \textbf{\circled{1}} \emph{Explore}($G$, $M$)\; 
            \textbf{\circled{3}} \emph{Execute with Abort Handling} ($Scheduling Unit$)\; 
        }
    }
  \caption{Execution Outline of \system}
  \label{alg:algo}
\end{algorithm}
\end{comment}



%%%%%%%%%%%%%%%%%%%%%%%%%%%%%%%%%%%%%%%%%%%%%%%%%%%%%%%%%%%%%%%%%%%%%%%%%%%%%%%%%
%% BIBILOGRAPHY
%\bibliographystyle{apsrev4-1}
%\bibliography{bib}
\bibliography{paper}

%%%%%%%%%%%%%%%%%%%%%%%%%%%%%%%%%%%%%%%%%%%%%%%%%%%%%%%%%%%%%%%%%%%%%%%%%%%%%%%%%


\end{document}


%\bibliography{paper}

\end{document}
