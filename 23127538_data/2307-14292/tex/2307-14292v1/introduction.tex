%!TEX root = main.tex

%%%%%%%%%%%%%%%%%%%%%%%%%%%%%%%%%%%%%%%%%%%%%%%%%%%
\section{Introduction}
%%%%%%%%%%%%%%%%%%%%%%%%%%%%%%%%%%%%%%%%%%%%%%%%%%%

Checking whether a distributed system is in a legal global state with respect to some boolean predicate occurs in several domains  of distributed computing, including the following.
\begin{itemize}
\item Fault-tolerance: the \revision{occurrence} of faults may turn the system into an illegal state that needs to be detected for allowing the system to return to a legal state.
\item The use of subroutines as black boxes: some of these subroutines may contain bugs, and produce incorrect outputs that need to be checked before use in the protocol calling the subroutines. 
\item Algorithm design for specific classes of systems: an algorithm dedicated to some specific class of networks (e.g., algorithms for trees, or for planar networks) may cause deadlocks or live-locks whenever running on a network outside the class. The membership to the class needs to be checked before running the algorithm. 
 \end{itemize}
In all three  cases above, the checking procedure may be impossible to implement without significant communication overhead. A typical example is bipartiteness, whether it be applied to the network itself, or to an overlay network produced by some subroutine. 

\subsection{Proof-Labeling Schemes}

\emph{Proof-labeling scheme} (PLS)~\cite{KormanKP10} is a popular mechanisms enabling to certify correctness w.r.t.~predicates involving some global property, like bipartiteness. A PLS involves a \emph{prover} and a \emph{verifier}. The prover has access to the global state of the network (including its structure), and has unlimited computational power. It assigns \emph{certificates} to the nodes. The verifier is a distributed algorithm running at each node, performing in a single round, which consists for each node to send its certificate to its neighbors. Upon reception of the certificates of its neighbors, every node performs some local computation and outputs \textit{accept} or \textit{reject}. To be correct, a PLS for a predicate~$\Pi$ must satisfy:
 \[
 \begin{array}{c}
 \mbox{the global state of the network satisfies $\Pi$}\\
 \Updownarrow \\
 \mbox{the prover can assign certificates such that the verifier accepts at all nodes.} 
 \end{array}
 \]
 For instance, for bipartiteness, the prover assigns a color~0 or~1 to the nodes, and each node verifies that its color is 0 or~1, and is different from the color of each of its neighbors. If the network is bipartite then the prover can properly 2-color the nodes such that they all accept, and if the network is not bipartite then, for every 2-coloring of the nodes, some of them reject as this coloring cannot be proper. 
 
 The PLS certification mechanism has several desirable features. First, if the certificates are small then the verification is performed efficiently, in a single round consisting merely of an exchange of a small message between every pair of adjacent nodes. As a consequence, verification can be performed regularly and frequently without causing significant communication overhead. Second, if the network state does not satisfy the predicate, then at least one node rejects. Such a node can raise an alarm or launch a recovery procedure for allowing the system to return to a correct state,  or can stop a program running in an environment for which it was not designed. Third, the prover is an abstraction, for the certificates can be computed offline, either by the nodes themselves in a distributed manner, or by the system provider in a centralized manner. For instance, a protocol constructing an overlay network that is supposed to be bipartite, may properly 2-color the overlay for certifying its bipartiteness.  It follows from their features that PLSs are versatile  certification mechanisms that are also quite efficient whenever the certificates for legal instances are small. 
 
%----------------------------------------------------------------------
%\subsection{Meta-Theorems for PLS}
 %----------------------------------------------------------------------

Many PLSs have been designed for certifying specific predicates on labeled graphs, including cycle-freeness~\cite{KormanKP10}, minimum-weight spanning tree (MST)~\cite{KormanK07}, planarity~\cite{FeuilloleyFMRRT21}, bounded genus~\cite{EsperetL22}, $H$-minor-freeness for small~$H$~\cite{BousquetFP21}, etc. In 2021, a breakthrough has been obtained, as a ``meta-theorem'' stating that a large set of properties have compact PLSs in a large class of networks (see~\cite{FeuilloleyBP22}). Namely, for every  $\MSO_2$ property\footnote{Monadic second-order logic (MSO) is the fragment of second-order logic where the second-order quantification is limited to quantification over sets. $\MSO_1$ refers to MSO on graphs with quantification over sets of vertices, whereas $\MSO_2$ refers to MSO on graphs with quantification over sets of vertices and sets of edges. }~$\Pi$, there exists a PLS for~$\Pi$ with $O(\log n)$-bit certificates for all graphs of bounded \emph{tree-depth}, where the tree-depth of a graph intuitively measures how far it is from being a star. This result has been extended to the larger class of  graphs with bounded \emph{tree-width}  (see~\cite{FraigniaudMRT22}), using certificates on $O(\log^2 n)$ bits, where the tree-width of a graph intuitively measures how far it is from being a tree. Although the class of all graphs with bounded tree-width includes many common graph families such as trees, series-parallel graphs, outerplanar graphs, etc., it does not contain families of \emph{dense} graphs.
 In this paper, we focus on the families of graphs with bounded \emph{clique-width}, which include families of dense graphs. 
 
 \subsection{Clique-Width}
 
Intuitively, the definition of clique-width is based on a ``programming language'' for constructing graphs, using only the following four instructions (see \cite{CourcelleMR00} for more details): 

 \begin{itemize}
 \setlength\itemsep{0em}
 \item Creation of a new vertex $v$ with some color $i$, denoted by $\mathsf{color}(v,i)$; 
 \item Disjoint union of two colored graphs $G$ and $H$, denoted by $G \parallel H$;
\item Joining by an edge every vertex colored $i$ to every vertex colored $j\neq i$, denoted by $i \Join j$;
\item Recolor $i$ into color $j$, denoted by $\mathsf{recolor}(i,j)$.
 \end{itemize}
 
For instance, the $n$-node clique can be constructed by creating a first node with color blue, and then repeating $n-1$ times the following: (1)~the creation of a new node, with color red, (2)~joining red and blue, and (3)~recoloring red into blue. Therefore, cliques can be constructed by using two colors only.  Similarly, trees can be constructed with three colors only. This can be proved by induction. The induction statement is that, for every tree~$T$, every vertex~$r$ of~$T$, and every two colors $c_1,c_2\in\{\mbox{blue, red, green}\}$, $T$~can be constructed with colors blue, red, and green such that $r$ is eventually colored~$c_1$, and every other vertex is colored~$c_2$. The statement is trivial for the single-node tree. Let $T$ be a tree with at least two nodes, let $r$ be one of its vertices, and let $c_1,c_2$ be two colors. Given an arbitrary neighbor~$s$ of~$r$, removing the edge $\{r,s\}$ results in two trees~$T_r$ and~$T_s$. By induction, construct~$T_r$ and $T_s$ separately so that~$r$ (resp.,~$s$) is eventually colored~$c_1$ (resp.,~$c_2$) and all the other nodes of $T_r$ and $T_s$ are colored $c_3\notin\{c_1,c_2\}$. Then form the graph $T_r \parallel T_s$, and, in this graph, join colors $c_1$ and $c_2$, and recolor $c_3$ into $c_2$. 

The clique-width of a graph~$G$, denoted by $\cw(G)$, is the smallest~$k\geq 0$ such that $G$ can be constructed by using $k$ colors. For instance, $\cw(K_n)\leq 2$ for every $n\geq 1$, and,  for every tree~$T$, $\cw(T)\leq 3$. A family of graphs has bounded clique-width if there exists $k\geq 0$ such that, for every graph~$G$ in the family, $\cw(G)\leq k$. Any graph family with bounded tree-depth or bounded tree-width has bounded clique-width~\cite{CorneilR05,NesetrilM12}. However, there are important graph families with unbounded tree-width (and therefore unbounded tree-depth) that have bounded clique-width. Typical examples \revision{(see \cite{CourcelleO00})} are cliques (i.e., complete graphs), $P_4$-free graphs (i.e., graphs excluding a \revision{path on four vertices} as an induced subgraph, a.k.a., cographs), and distance hereditary graphs (the distances in any connected induced subgraph are the same as they are in the original graph).

Many  NP-hard optimization problems can be solved efficiently by dynamic programming in the family of graphs with bounded clique-width. In fact, every $\MSO_1$ property on graphs has a linear-time algorithm for graphs of bounded clique-width~\cite{CourcelleMR00}. In this paper we show a similar form of ``meta-theorem'', regarding the size of certificates of PLS for monadic second-order properties of graphs with bounded clique-width. 

%----------------------------------------------------------------------
\subsection{Our Results}
 %----------------------------------------------------------------------
 
Our main result is the following. Recall that a labeled graph is a pair $(G,\ell)$, where $G$ is a graph, and $\ell:V(G)\to\{0,1\}^\star$ is a function assigning a label to every node in~$G$. 
 
\begin{theorem}\label{theo:main}
Let $k$ be a non-negative integer, and let  $\Pi$ be an $\MSO_1$ property on node-labeled graphs with constant-size labels.
There exists a PLS certifying $\Pi$ for labeled graphs with clique-width at most~$k$, using $O(\log^2 n)$-bit certificates on $n$-node graphs. 
\end{theorem}

The same way several NP-hard problems become solvable in polynomial time in graphs of bounded clique-width, Theorem~\ref{theo:main} implies that several predicates for which every PLS has  certificates of polynomial size in arbitrary graphs have a PLS with certificates of polylogarithmic  size on graphs with bounded clique-width. This is for instance the case of non-3-colorability (which is a $\MSO_1$ predicate), for which every PLS has certificates of size $\tilde{\Omega}(n^2)$ bits in arbitrary graphs~\cite{GoosS16}. Theorem~\ref{theo:main} implies that non-3-colorability has a PLS with certificates on $O(\log^2 n)$ bits in graphs with bounded clique-width, and therefore in \revision{graphs of bounded tree-width}, cographs, distance-hereditary graphs, etc. \revision{This of course is extended to non-k-colorability, as well as other problems definable in $\MSO_1$ such as detecting whether the input graph does not contain a fixed subgraph \(H\) as a subgraph, induced subgraph, minor, etc.}


 In fact, Theorem~\ref{theo:main}  can be extended to properties including certifying solutions to maximization or minimization problems whose admissible solutions are defined by $\MSO_1$ properties. \revision{For instance maximum independent set, minimum vertex cover, minimum dominating set, etc}.



In the proof of Theorem \ref{theo:main}, we provide a  PLS that constructs a particular decomposition using at most \(k\cdot 2^{k-1}\) colors
(the clique-width of the decomposition). It is through that decomposition that the PLS certifies that the input graph satisfies \(\Pi\). 

An  application of Theorem~\ref{theo:main} is the certification of certain families of graphs. That is, given a graph family $\mathcal{F}$, designing a PLS for certifying the  membership to $\mathcal{F}$. Interestingly, there are some graph classes $\mathcal{F}$ that are expressible in \(\MSO_1\) and, at the same time, have clique-width at most~\(k\). Theorem~\ref{theo:main} provides  a PLS for certifying the membership to $\mathcal{F}$ in such cases. Indeed, the PLS first tries to build a decomposition of clique-width at most 
$k \cdot 2^{k-1} $.
If there is no such decomposition, then the input graph does not belong to 
$\mathcal{F}$. 
Otherwise, the PLS uses the decomposition to check the \(\MSO_1\) property that defines  $\mathcal{F}$.

\begin{corollary}
Let $k$ be a non-negative integer, and let  $\mathcal{F}$ be graph family expressible in $\MSO_1$ such that all graphs of the family have clique-width at most~$k$. Membership to $\mathcal{F}$ can be certified with a PLS using  $O(\log^2 n)$-bit certificates in $n$-node graphs. 
\end{corollary}

For instance, for every $k\geq 0$, the class of graphs with tree-width at most~$k$  can be certified with a PLS using  $O(\log^2 n)$-bit certificates. Indeed, ``tree-width at most~$k$'' is expressible in $\MSO_1$, and the class of graphs with tree-width at most~$k$  forms a family with clique-width at most~$3\cdot 2^{k-1}+1$~\cite{CorneilR05}.
Another interesting application is the certification of $P_4$-free graphs. Indeed, ``excluding $P_4$ as induced subgraph'' is expressible in $\MSO_1$, and $P_4$-free graphs form a family with clique-width at most~2~\cite{CourcelleO00}. It follows that $P_4$-free graphs can be certified with a PLS using  $O(\log^2 n)$-bit certificates. This is in contrast to the class of $C_4$-free graphs (\revision{i.e. graphs not containing a cycle on four vertices,} whether it be as induced subgraph or merely subgraph), which requires certificates on~$\tilde{\Omega}(\sqrt{n})$ bits~\cite{DruckerKO13}. In fact, in the case of cographs, the techniques in the proof of Theorem~\ref{theo:main} can be adapted so that to save one log-factor, as stated below. 

\begin{theorem}\label{theo:cographs}
The class of (induced) $P_4$-free graphs can be certified with a PLS using  $O(\log n)$-bit certificates  in $n$-node graphs. 
\end{theorem}
 
Note that there is a good reason for the huge gap in terms of certificate-size between $P_4$-free graphs and $C_4$-free graphs. The point is that, for any graph pattern~$H$, the class of $H$-free graphs has bounded clique-width if and only if $H$ is an induced subgraph of $P_4$ \cite{dabrowski2016clique}. Therefore, $C_4$-free graphs (as well as triangle-free graphs) do not have bounded clique-width, as opposed to $P_4$-free graphs (and $P_3$-free graphs, which are merely cliques). 


%----------------------------------------------------------------------
\subsection{Related Work}
 %----------------------------------------------------------------------

Proof-Labeling Schemes (PLSs) have been introduced and thoroughly studied in~\cite{KormanKP10}. Variants have been been considered in~\cite{GoosS16} and~\cite{FraigniaudKP13}, which slightly differ from PLSs: the former allows each node to transfer no only its certificates, but also its state, and the latter restricts the power of the oracle, which is bounded to produce certificates independent of the IDs assigned to the nodes. All these forms of distributed certifications have been extended in various directions, including tradeoffs between the size of the certificates and the number of rounds of the verification protocol~\cite{FeuilloleyFHPP21}, 
PLSs with computationally restricted provers~\cite{EmekGK22}, randomized PLSs~\cite{FraigniaudPP19}, quantum PLSs~\cite{FraigniaudGNP21}, PLSs rejecting at more nodes whenever the global state is ``far'' from being correct~\cite{FeuilloleyF22}, PLSs using global certificates in addition to the local ones~\cite{FeuilloleyH18}, and several hierarchies of certification mechanisms, including games between a prover and a disprover~\cite{BalliuDFO18,FeuilloleyFH21}, interactive protocols~\cite{CrescenziFP19,KolOS18,NaorPY20}, and even recently zero-knowledge distributed certification~\cite{BickKO22},  and distributed quantum interactive protocols~\cite{GallMN22}. 

All the aforementioned distributed certification mechanisms have been used for certifying a wide variety of global system states, including MST~\cite{KormanK07}, routing tables~\cite{BalliuF19}, and a plethora of (approximated) solutions to optimization problems~\cite{Censor-HillelPP17,EmekG20}. A vast literature has also been dedicated to certifying membership to graph classes, including cycle-freeness~\cite{KormanKP10}, planarity~\cite{FeuilloleyFMRRT21}, bounded genus~\cite{EsperetL22}, absence of symmetry~\cite{GoosS16}, $H$-minor-freeness for small~$H$~\cite{BousquetFP21}, etc. In 2021, a breakthrough has been obtained, as a ``meta-theorem'' stating that, for every  $\MSO_2$ property~$\Pi$, there exists a PLS for~$\Pi$ with $O(\log n)$-bit certificates for all graphs of bounded \emph{tree-depth}~\cite{FeuilloleyBP22}. This result has been extended to the larger class of  graphs with bounded \emph{tree-width}, using certificates on $O(\log^2 n)$ bits~\cite{FraigniaudMRT22}. To our knowledge, this is the largest class of graphs, and the largest class of boolean predicates on graphs for which it is known that PLSs with polylogarithmic certificates exist. 

The class of $H$-free graphs (i.e., the absence of~$H$ as a subgraph), for a given fixed graph~$H$, has attracted lot of attention in the distributed setting, mostly in the \textsf{CONGEST} model. Two main approaches have been considered. One, called distributed property testing, aims at deciding between the case where the input graph is $H$-free, and the case where the input graph is ``far'' from being $H$-free (see, e.g.,~\cite{BrakerskiP11,Censor-HillelFS19,EvenFFGLMMOORT17,FraigniaudRST16}). In this setting, the objective is to design (randomized) algorithms performing in a constant number of rounds. Such algorithms have been designed for small graphs~$H$, but it is not known whether there is a distributed algorithm for testing $K_5$-freeness in a constant number of rounds.  The other approach aims at designing algorithms deciding $H$-freeness performing in a small number of rounds. For instance, it is known that deciding $C_4$-freeness can be done in $\tilde{O}(\sqrt{n})$ rounds, and this is optimal~\cite{DruckerKO13}. The $\tilde{\Omega}(\sqrt{n})$-round lower bounds for $C_4$-freeness also holds for deciding $C_{2k}$-freeness, for every $k\geq 4$. Nevertheless, the best known algorithm performs in essentially $\tilde{O}(n^{1-\Theta(1/k^2)})$ rounds~\cite{eden2022sublinear}, even if faster algorithms exists for $k=2,3,4,5$, running in $\tilde{O}(n^{1-\Theta(1/k)})$ rounds~\cite{censorhillel_et_al:LIPIcs:2020:13111,drucker2014power}. Deciding $P_k$-freeness (as subgraph) can be done efficiently for all $k\geq 0$~\cite{FraigniaudO19}. However, this is not the case of deciding the absence of an \emph{induced}~$P_k$, and no efficient algorithms are known apart for the trivial cases $k=1,2,3$. The first non-trivial case is deciding cographs, i.e.,   $P_4$-freeness (as induced subgraph). 

The terminology \emph{meta-theorem} is used in logic to refer to a statement about a formal system proven in a language used to describe another language. In the study of graph algorithms, Courcelle's theorem~\cite{Courcelle90} is often referred to as  a meta-theorem. It says that every graph property definable in the monadic second-order logic $\MSO_2$ of graphs can be decided in linear time on graphs of bounded treewidth. This theorem was extended to clique-width, but for a smaller set of graph properties. Specifically, every graph property definable in the monadic second-order logic $\MSO_1$ of graphs  can be decided in linear-time  on graphs of bounded clique-width~\cite{CourcelleMR00}. Note that the classes of languages in $\MSO_1$ and $\MSO_2$ include languages that are NP-hard to decide (e.g., 3-colorability and Hamiltonicity, respectively). 
\revision{We remind that \(\MSO_2\) is as an extension of  \(\MSO_1\) which also allows quantification on sets of edges -- see Footnote~1 for a short description, or~\cite{CourcelleE12} for full details. Some graph properties, e.g., Hamiltonicity, are expressible in   \(\MSO_2\) but not in  \(\MSO_1\), nevertheless  \(\MSO_1\)  captures a large set of properties, including many classical NP-hard problems as explained above. Eventually, we emphasize again that, when comparing the two most famous meta-theorems, (1) \emph{$\MSO_2$ properties are decidable in linear time on bounded treewidth graphs} vs. (2) \emph{$\MSO_1$ properties are decidable in linear time on bounded clique-width graphs}, the former concerns a larger class of properties, but the latter concerns larger classes of graphs. }













