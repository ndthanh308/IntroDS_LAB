%\documentclass[a4paper,UKenglish,cleveref,autoref, thm-restate]{lipics-v2021}
\documentclass[a4paper,UKenglish]{lipics-v2021}
\usepackage{amssymb}
\usepackage{color}
\usepackage{nicefrac}
\usepackage{graphicx}
\usepackage{amsthm}
\usepackage{amsmath}

%%%%%%%%%%%%%%%%%%%%%%%%%%%%%%%%%%%%%%%%%%%%
%%%%%%%% Included by Pedro to compile \pierre and \citetodo commands %%%%%
\usepackage[colorinlistoftodos]{todonotes}
\usepackage{xargs}  
%\newcommandx{\citetodo}[2][1=]{\cite{??}\todo[linecolor=red,backgroundcolor=red!25,bordercolor=red,#1]{Missing Reference #2}}
\newcommandx{\pierre}[2][1=]{\todo[linecolor=blue,backgroundcolor=blue!25,bordercolor=blue,#1]{\tiny Pierre: #2}}
\newcommand{\ioan}[2][1=]{\todo[linecolor=green,backgroundcolor=green!25,bordercolor=blue,#1]{\tiny Ioan: #2}}
\newcommandx{\pedro}[2][1=]{\todo[linecolor=green,backgroundcolor=green!25,bordercolor=blue,#1]{\tiny Pedro: #2}}
%\newcommandx{\pedro}[2][1=]{\todo[linecolor=blue,backgroundcolor=yellow!25,bordercolor=blue,#1]{\tiny Pedro: #2}}
%%%%%%%%%%%%%%%%%%%%%%%%%%%%%%%%%%%%%%%%%%%%
\newcommandx{\revision}[1]{{\color{black}{#1}}}

\nolinenumbers %uncomment to disable line numbering

\hideLIPIcs 
%\usepackage[utf8]{inputenc}
%\usepackage[T1]{fontenc}

%\newtheorem{theorem}{Theorem}
%\newtheorem{proposition}{Proposition}
%\newtheorem{corollary}{Corollary}
%\newtheorem{lemma}{Lemma}
%\newtheorem{fact}{Fact}
%\newtheorem{claim}{Claim}
%\newtheorem{question}{Question}
%\newtheorem{definition}{Definition}

\author{Pierre Fraigniaud}{IRIF, Universit\'e Paris Cit\'e and CNRS, France.}{pierre.fraigniaud@irif.fr}{}{Additional support for ANR projects QuData and DUCAT.}
\author{Fr\'ed\'eric Mazoit}{LaBRI, Universit\'e de Bordeaux, France}{frederic.mazoit@labri.fr}{}{}
\author{Pedro Montealegre}{Facultad de Ingenier\'ia y Ciencias, Universidad Adolfo Ib\'a\~nez, Santiago, Chile.}{p.montealegre@uai.cl}{}{This work was supported by Centro de Modelamiento Matem\'atico (CMM), FB210005, BASAL funds for centers of excellence from ANID-Chile, and ANID-FONDECYT 1230599. }
\author{Ivan Rapaport}{DIM-CMM (UMI 2807 CNRS), Universidad de Chile, Chile.}{rapaport@dim.uchile.cl}{}{This work was supported by Centro de Modelamiento Matem\'atico (CMM), FB210005, BASAL funds for centers of excellence from ANID-Chile, and  ANID-FONDECYT 1220142. }
\author{Ioan Todinca}{LIFO, Universit\'e d'Orl\'eans and INSA Centre-Val de Loire, France.}{ioan.todinca@univ-orleans.fr}{}{}

\authorrunning{P. Fraigniaud et al.} %TODO mandatory. First: Use abbreviated first/middle names. Second (only in severe cases): Use first author plus 'et al.'

\Copyright{Pierre Fraigniaud, Fr\'ed\'eric Mazoit, Pedro Montealegre, Ivan Rapaport and Ioan Todinca} %TODO mandatory, please use full first names. LIPIcs license is "CC-BY";  http://creativecommons.org/licenses/by/3.0/

\ccsdesc[500]{Theory of computation~Distributed algorithms}
%\ccsdesc[100]{\textcolor{red}{Replace ccsdesc macro with valid one}} %TODO mandatory: Please choose ACM 2012 classifications from https://dl.acm.org/ccs/ccs_flat.cfm 

\keywords{CONGEST, Proof Labelling Schemes, clique-width, MSO} %TODO mandatory; please add comma-separated list of keywords


%%
%% The "title" command has an optional parameter,
%% allowing the author to define a "short title" to be used in page headers.
\title{Distributed Certification for Classes of Dense Graphs}


\newcommand{\MSO}{\mathrm{MSO}}
\newcommand{\id}{\mathsf{id}}
\newcommand{\cw}{\mathsf{cw}}
\newcommand{\nlcw}{\mathsf{nlcw}}
\newcommand{\NLC}{\mathrm{NLC}}
\newcommand{\tw}{\mathsf{tw}}
\newcommand{\cC}{\mathcal{C}}
\newcommand{\recolor}{\mathsf{recolor}}

\bibliographystyle{plainurl}% the mandatory bibstyle

%%
%% end of the preamble, start of the body of the document source.
\begin{document}


%%
%% This command processes the author and affiliation and title
%% information and builds the first part of the formatted document.
\maketitle


\begin{abstract}

The Fast Reciprocal Square Root Algorithm is a well-established approximation technique consisting of two stages: first, a coarse approximation is obtained by manipulating the bit pattern of the floating point argument using integer instructions, and second, the coarse result is refined through one or more steps, traditionally using Newtonian iteration but alternatively using improved expressions with carefully chosen numerical constants found by other authors. The algorithm was widely used before microprocessors carried built-in hardware support for computing reciprocal square roots. At the time of writing, however, there is in general no hardware acceleration for computing other fixed fractional powers. This paper generalises the algorithm to cater to all rational powers, and to support any polynomial degree(s) in the refinement step(s), and under the assumption of unlimited floating point precision provides a procedure which automatically constructs provably optimal constants in all of these cases. It is also shown that, under certain assumptions, the use of monic refinement polynomials yields results which are much better placed with respect to the cost/accuracy tradeoff than those obtained using general polynomials. Further extensions are also analysed, and several new best approximations are given.

\end{abstract}


\section{Introduction}
Current quantum hardware is unable to carry out universal quantum computations due to the buildup of errors that occur during the computation. 
The magnitude of the individual error is currently above the value that the Threshold Theorem requires in order to kick-start quantum error correction and fault-tolerant quantum computation~\cite[Section 10.6]{nielsen_chuang_2010}. 
Although the experimentally achieved fidelity rates are promising and the error bounds are inching closer to the required threshold, we will have to work for the foreseeable future with quantum hardware with errors that build-up during the computation.  This implies that we can only do a limited number of steps before the output of the computation has become completely uncorrelated with the intended one.

For fault-tolerant quantum computing, we repeat four steps: 
1) We apply a number of single and two-qubit quantum gates, in parallel whenever possible; 
2) We perform a syndrome measurement on a subset of the qubits; 
3) We perform fast classical computations to determine which errors have occurred and how to correct them; 
and, 4) We apply correction terms based on the classical computations.
We then repeat these four steps with a next sequence of gates. 
These four steps are essential to fault-tolerant quantum computing. 


The starting point of this work is to use the four steps outlined above, not to carry out error correction and fault-tolerant computation, but to enhance short, constant-depth, {\em uncorrected} quantum circuits that perform single qubit gates and {\em nearest-neighbor} two qubit gates. 
Since in the long run we will have to implement error-correction and fault-tolerant computation anyhow, and this is done by such a four-step process, why not make other use of this architecture? Moreover, on some of the quantum hardware platforms, these operations are already in place.
Embracing this idea we naturally arrive at the question: what is the computational power of \textit{low-depth} quantum-classical circuits organized as in the four steps outlined above? 
We thus investigate circuits that execute a small, ideally constant, number of stages, where at each stage we may apply, in parallel, single qubit gates and {\em nearest-neighbor} two qubit gates, followed by measurements, followed by low-depth classical computations of which the outcome can control quantum gates in later stages. 
It is not clear, at first, whether such circuits, especially with constant depth, can do anything remotely useful. 
But we will see that this is indeed the case: many quantum computations can be done by such circuits in constant depth. 
By parallelizing quantum computations in this way, we improve the overall computational capabilities of these circuits, as we do not incur errors on qubits that are idle, simply because qubits are not idle for a very long time. 
Furthermore, reducing the depth of quantum circuits, at the cost of increasing width, allows the circuit to be run faster even if errors occur.

The first usage of such a four-step layout, not to do error correction, but to perform computations, can be found in the paradigm of measurement-based quantum computing~\cite{gottesman1999demonstrating,raussendorf2001one,jozsa2006introduction,clark2007generalised}: 
A universal form of quantum computing where a quantum state is prepared and operations are performed by measuring qubits in different bases, depending on previous measurements and intermediate measurements.

\citeauthor{PhamSvore2013} were the first to formalize the four-step protocol for performing computations~\cite{PhamSvore2013}. They included specific hardware topologies by considering two-dimensional graphs for imposing constraints on qubit interactions. In their model, they develop circuits for particularly useful multi-qubit gates, including specifying costs in the width, number of qubits, depth, number of concurrent time steps, size, and total number of non-Identity operations.
As a result, they find an algorithm that factors integers in polylogarithmic depth.
\citeauthor{Browne:2011} showed that the main tool in the work by \citeauthor{PhamSvore2013}, the fan-out gate, can also be replaced by additional log-depth classical computations in the measurement-based quantum computing setting~\cite{Browne:2011}.

More recently, \citeauthor{Cirac:2021} introduced a scheme to implement unitary operations involving quantum circuits combined with Local Operations and Classical Communication ($\mathsf{LOCC}$) channels: $\mathsf{LOCC}$-assisted quantum circuits~\cite{Cirac:2021}. Similarly to the four-step scheme we just described, they allow for a short depth circuit to be run on the qubits, followed by one round of $\mathsf{LOCC}$, in which ancilla qubits are measured and local unitaries are applied based on the measurement outcomes. They show that in this model any 1D transitionally invariant matrix-product state (MPS) with fixed bond dimension is in the same phase of matter as the trivial state. Similar ideas can be found in~\cite{TVV_NonAbelianTopologicalOrder_2022, tantivasadakarn2021long}.

In this work, we introduce a new model, called \textit{Local Alternating Quantum-Classical Computations} ($\LAQCC$). In this model we alternate between running quantum circuits (constrained by locality), ending in the measurement of a subset of qubits, and fast classical computations based on the measurement results. The outcome of the classical computations are then used to control future quantum circuits. We allow for flexibility in this model, by giving different constraints to the power of both the quantum circuits and the classical circuits as well as the number of alternations between them. 
Most attention will be given to $\LAQCC$ containing quantum circuits of constant depth, classical circuits of logarithmic depth and at most a constant number of alternations between them. 
Any circuit constructed in this model is considered to be of constant depth. 
We restrict ourselves to logarithmic depth classical computations, as this is the first natural and non-trivial extension beyond constant-depth classical computations. 
Constant-depth classical computations do however also have an equivalent constant-depth quantum implementation.

The definition of $\LAQCC$ sharpens the original definition of \citeauthor{PhamSvore2013} by adding constraints to the intermediate classical computations. This allows us to bound the power of $\LAQCC$ from above. 

The main result of \citeauthor{Cirac:2021}, that 1D translational invariant MPS with fixed bond dimension can be prepared by $\mathsf{LOCC}$-assisted circuits, relies on local symmetries of the MPS. These symmetries allow them to prepare local states (on a constant number of qubits) and glue them together by doing one round of the appropriate entangling measurement and corrections, after which they run a round of local unitaries to get the desired result. This general scheme for preparing states that exhibit an MPS description with the appropriate local symmetries requires only geometrically local unitaries and one round of measurement and corrections an therefore is accessible in $\LAQCC$. Studying different local symmetries, known as Symmetry Protected Topological (SPT) phases of matter, to find measurement-based constant depth circuits for states is a broad ongoing field of research~\cite{TVV_NonAbelianTopologicalOrder_2022, tantivasadakarn2021long, smith2023deterministic}. 
All these schemes have a $\LAQCC$ implementation.

%$\LAQCC$-circuits also exist for general schemes of preparing local states, based on the local tensors, and gluing them together using one round of entangled measurement and corrections, based on the local symmetry. 
%The main result of \citeauthor{Cirac:2021}, that 1D translational invariant MPS with fixed bond dimension can be prepared by $\mathsf{LOCC}$-assisted circuits, relies heavily on local symmetries of the MPS and as a result also has an equivalent $\LAQCC$ implementation. 
%The corrections applied after the measurement round are local unitaries depending on the local symmetries of the MPS. 

 

%This general scheme of preparing local states, based on the local tensors, and gluing it together by doing one round of entangled measurement and corrections, based on the local symmetry, is accessible in $\LAQCC$.
Note however that \citeauthor{Cirac:2021} also suggest a circuit for the $W$-state.
This circuit uses sequentially and dependent measurement-based corrections of the ancilla qubits. 
These dependent measurements translate to sequential alternations between the quantum and classical circuits and therefore increase the total depth to linear depth, exceeding the constant-depth constraints imposed by $\LAQCC$-circuits. 

We study the power of the $\LAQCC$ model with respect to state preparation, showing that even with only constant quantum-depth and logarithmic classical depth it remains possible to prepare states with long-range entanglement.
Another surprising result is that it is unlikely that $\LAQCC$ circuits are classically simulatable. We show that any instantaneous quantum polynomial-time (IQP) circuit~\cite{Bremner2010,Shepherd2009} has an $\LAQCC$ implementation.
Classical simulation of IQP circuits implies the collapse of the polynomial hierarchy to the third level, which is not believed to be true~\cite{Bremner2017}. Therefore, we expect that $\LAQCC$ circuits are unlikely to be classically simulatable. We bound the power of $\LAQCC$ by showing that it is contained in $\QNC^1$, the class of polynomial-size, log-depth circuits.

Next, we also study the power that intermediate classical calculations can add to quantum computations, by considering a new model that alternates between polynomially many polynomial-depth quantum circuits and unbounded classical computations
We study this model by doing a complexity theoretical analysis, where we draw inspiration from the notions of complexity given by \citeauthor{RosenthalYuen:2022}, \citeauthor{MetgerYuen:2023}, and \citeauthor{Aaronson:2004}.
All three complexity notions are based on the notion of state preparation, instead of more traditional definition of complexity such as the decidability of a computational problem. 
The first two consider classes based on sequences of quantum states preparable by a polynomial-sized quantum circuit, where the circuits are uniformly generated by a computational class, for instance, the class $\mathsf{PSPACE}$, which results in the complexity class $\mathsf{StatePSPACE}$~\cite{RosenthalYuen:2022,MetgerYuen:2023}.
The third notion considers a relative complexity, where the complexity is measured between two given states, and is measured by the number of gates, from a given gate-set, required to transform one state in another state~\cite{Aaronson:2004}. 
For our definition of state preparation complexity, we drop the uniformity constraint from~\cite{RosenthalYuen:2022,MetgerYuen:2023} and define a class as $\mathsf{StateX}$, which refers to states preparable by circuits of type $\mathsf{X}$. 
As an example, if $\mathsf{X} = \QNC^0$, this results in the class $\mathsf{StateQNC^0}$, which is the set of states preparable from the $\ket{0}^n$ state by poly-size constant-depth circuits. 
This notion is similar to the relative complexity from~\cite{Aaronson:2004}, where one state is the  $\ket{0}^n$ state and instead of counting the number of gates we consider the set of states preparable by a fixed number of gates. Using this notion of complexity we show that any state preparable by an $\LAQCC^*$ circuit is also preparable by a $\mathsf{PostQPoly}$ circuit, the class of circuits of polynomial depth with an additional post-selection gate. 

All Clifford circuits have a constant-depth $\LAQCC$ implementation, implying that any stabilizer state can be implemented by a constant-depth $\LAQCC$ circuit, see Section~\ref{sec:clifford_circuits} for a proof of this statement. 
Efficient circuits for stabilizer states have been known already through measurement-based quantum computing. Therefore this paper focuses on the preparation of non-stabilizer states, and as a surprising result we find novel constant-depth protocols for four very natural classes of non-stabilizer states.
Despite the extensive research into these four classes of non-stabilizer states and the many applications of them, no efficient constant- or low-depth state preparation protocols are known yet. We specifically consider these four classes as they are all often used as initial states in other algorithms.

The first state is a uniform superposition over an arbitrary number of states. 
This state finds applications in many quantum algorithms, as they often start with a uniform superposition over multiple states. 
This superposition is often achieved by applying Hadamard gates to every qubit due to its simplicity to prepare. 
Yet, the analysis of many algorithms, such as Shor's algorithm~\cite{Shor:1997}, would benefit from a different initial superposition. 
The circuit to prepare the uniform superposition over an arbitrary number of states uses an exact version of Grover search as a subroutine, that turns a probabilistic circuit, with a known constant probability of success, into a deterministic circuit. 
We use the circuit for preparing a uniform superposition over an arbitrary number of states as a subroutine in the next two quantum state preparation protocols. 

The second state is the $W$-state, the uniform superposition over all computational basis states of Hamming-weight~$1$, a natural long-ranged entangled state that displays a fundamentally nonequivalent type of entanglement from the Greenberger–Horne–Zeilinger state~\cite{WState:2000}, for which $\LAQCC$-type constant-depth circuits were previously known~\cite{PhamSvore2013, Cirac:2021}. 
The $W$-state is often used as benchmark for new quantum hardware~\cite{Haffner2005,Neeley2010,GarciaPerez:2021}. 
A novel way to prepare the $W$-state therefore gives a new way to benchmark different quantum devices with each other. 
A circuit for preparing the $W$-state was given in~\cite{Cirac:2021}, but this implementation requires sequentially alternating measurements followed by local unitaries, which in the $\LAQCC$ model is not considered to be of constant depth. 
We improve this protocol by giving an $\LAQCC$ implementation of the $W$-state, based on a compress-uncompress method that links the one-hot and binary encoding of integers.

The third state considered is the Dicke state, a generalization of the $W$-state, a superposition over all computational basis states with Hamming-weight $k$~\cite{Dicke:1954}. 
Dicke states have relevance in various practical settings.
For instance, for quantum game theory~\cite{zdemir2007}, quantum storage~\cite{Bacon_Compress:2006,Plesch:2010}, quantum error correction~\cite{ouyang2014permutation}, quantum metrology~\cite{toth2012multipartite}, and quantum networking~\cite{prevedel2009experimental}. 
Dicke states have been used as a starting state for variational optimization algorithms, most notably Quantum Alternating Operator Ansatz (QAOA)~\cite{Hadfield2019}, to find solutions to problems such as Maximum k-vertex Cover~\cite{Brandhofer2022,cook2020quantum}.
The ground states of physical Hamiltonians describing one-dimensional chains tend to show a resemblance to Dicke states such as states resulting from the Bethe ansatz, making them an ideal starting state when investigating the ground state behavior of these Hamiltonians~\cite{TDL_BetheAnsatzDerivation:2010,B_ExcitedStateQuantumPhaseTransitions:2013,DickeTransitions:2021}. 
For instance, the algorithm by \citeauthor{van2021preparing}, who give an algorithm to prepare the Bethe ansatz eigenstates of the spin-1/2 XXZ spin chain, starts by first preparing a Dicke state~\cite{van2021preparing}. 
A Dicke-state preparation protocol based on the compress-uncompress methodology used in the $W$-state furthermore finds applications in entanglement distillation, where the entanglement of a large state is concentrated on only a few qubits. 
Efficient deterministic circuits for preparing Dicke states have been proposed by \citeauthor{bartschi2019deterministic}~\cite{bartschi2019deterministic, bartschi2022deterministic_short_depth}. 
They provide a quantum circuit of depth $\mathO(k \log(\frac{n}{k}))$, allowing arbitrary connectivity, to prepare a Dicke state, which they conjecture to be optimal when $k$ is constant. 
In this work, we provide a constant-depth $\LAQCC$ circuit below their conjectured bound already for constant $k$. 
However, this does not directly disprove their conjecture, as we allow for intermediate measurements and classical computations. 
More significantly, we even construct constant-depth $\LAQCC$ circuits for $k = \mathO(\sqrt{n})$ greatly improving their bound.
This construction extends the compress-uncompress method for the $W$-state combined with additional subroutines. 

We continue with a log-depth state preparation protocol for the Dicke-state for arbitrary $k$. 
This protocol implements an efficient transformation between the factoradic number representation and the combinatorial number representation of a positive integer. 
The combinatorial number representation relates directly to the Dicke state. 
The provided efficient transformation between number representation systems might be of independent interest. 

We conclude by modifying our protocol for preparing a Dicke-state to a protocol that prepares quantum many-body scar states in constant-depth. 
These states have low entanglement and longer coherence times than states with similar energy density.
These characteristics make many-body scar states interesting to analyze and relevant within physics.
Many-body scar states appear for instance in the AKLT model~\cite{AKLT:1987,MRBAR:2018,MRB:2018} and different spin models~\cite{SI:2019,MOBFR:2020}.
Known methods for preparing these states have polynomial-depth~\cite{Gustafson:2023}, whereas our circuit has constant depth. 

% We conclude by studying the power that intermediate classical calculations can add to quantum computations. 
% In this study, we define a new model that relaxes constant-depth quantum circuits to polynomial depth quantum circuits, log-depth classical calculations to unbounded classical computations and a constant number of alternations to a polynomial number of alternations. 
% We call this model $\LAQCC^*$. 
% We study this model by doing a complexity theoretical analysis, where we draw inspiration from the notions of complexity given by \citeauthor{RosenthalYuen:2022}, \citeauthor{MetgerYuen:2023}, and \citeauthor{Aaronson:2004}.
% All three complexity notions are based on the notion of state preparation, instead of more traditional definition of complexity such as the decidability of a computational problem. 
% The first two consider classes based on sequences of quantum states preparable by a polynomial-sized quantum circuit, where the circuits are uniformly generated by a computational class, for instance, the class $\mathsf{PSPACE}$, which results in the complexity class $\mathsf{StatePSPACE}$~\cite{RosenthalYuen:2022,MetgerYuen:2023}.
% The third notion considers a relative complexity, where the complexity is measured between two given states, and is measured by the number of gates, from a given gate-set, required to transform one state in another state~\cite{Aaronson:2004}. 
% For our definition of state preparation complexity, we drop the uniformity constraint from~\cite{RosenthalYuen:2022,MetgerYuen:2023} and define a class as $\mathsf{StateX}$, which refers to states preparable by circuits of type $\mathsf{X}$. 
% As an example, if $\mathsf{X} = \QNC^0$, this results in the class $\mathsf{StateQNC^0}$, which is the set of states preparable from the $\ket{0}^n$ state by poly-size constant-depth circuits. 
% This notion is similar to the relative complexity from~\cite{Aaronson:2004}, where one state is the  $\ket{0}^n$ state and instead of counting the number of gates we consider the set of states preparable by a fixed number of gates. Using this notion of complexity we show that any state preparable by an $\LAQCC^*$ circuit is also preparable by a $\mathsf{PostQPoly}$ circuit, the class of circuits of polynomial depth with an additional post-selection gate. 

\paragraph{Summary of results}
\begin{itemize}
    \item We give a new definition of a computational model that captures the power of the four step process: applying a constant number of layers of one- and two-qubit gates; performing a syndrome measurement; perform a fast classical computation determining corrections; apply corrections. We call this model \emph{Local Alternating Quantum Classical Computations}, or $\LAQCC$ for short. In this model we bound the allowed quantum operations, intermediate classical calculations, and number of rounds separately. In Section~\ref{sec:LAQCC_model} we define this model and give a list of operations based on results from literature contained in this computational model. In some of these operations we explicitly use that we allow for multiple, but at most constant, rounds  of corrections.
    \item  We show show that there exist $\LAQCC$ circuits that can not be weakly simulated in Section~\ref{sec:IQP_in_LAQCC}. We further show that for every $\LAQCC$ circuit there exists a $\QNC^1$ circuit simulating it perfectly, in Section~\ref{sec:LAQCC_in_QNC1}.
    \item We introduce a new type computational complexity for preparing states and show that the extension of $\LAQCC$ where we allow a polynomial number of rounds and unbounded classical computation, is contained in $\mathsf{PostQPoly}$, the class of polynomial circuits with post-selection, in Section~\ref{sec:Complexity results}.
    \item We show a protocol to prepare the uniform superposition state of size $q$ in $\LAQCC$ using $\mathO(\ceil{\log_2(q)}^2)$ qubits in Section~\ref{sec:superposition_modulo_q}. 
    \item We show a protocol to prepare the $W_n$ state in $\LAQCC$ using $\mathO(n\log(n))$ qubits in Section~\ref{sec:W_state_in_LAQCC}.
    \item We show two ways of preparing the Dicke-$(n,k)$ state. The first method is in $\LAQCC$, works up to $k = \mathO(\sqrt{n})$, uses $\mathO(n^2\log(n))$ qubits, and is found in Section~\ref{sec:dicke:small_k}. The second method is in $\LAQCC\text{-}\mathsf{LOG}$ (an extension of $\LAQCC$ allowing for logarithmic number of alterations instead of constant), works for any $k$, uses $\mathO(\text{poly}(n))$ qubits, and is found in Section~\ref{sec:Dicke_in_LAQCC_LOG}. 
    \item We extend on our $\LAQCC$ method of generating Dicke-$(n,k)$ states for $k = \mathO(\sqrt{n})$ and show a protocol to generate many-body scar states for a particular Hamiltonian in $\LAQCC$ (Section~\ref{sec:many_body_scar}). 
\end{itemize}
Summarized in a table, we provide the following state generation protocols:
\begin{table}[htb]
\centering
\begin{tabular}{l|l|l|l}
\textbf{State description} & \textbf{Width} & \textbf{Depth} & \textbf{Implementation}\\
\hline 
Uniform superposition mod $q$: $\frac{1}{\sqrt{q}} \sum_{i = 0}^{q-1}\ket{i}$ & $\mathO(\ceil{\log^2 q})$ & $\mathO(1)$ & Section~\ref{sec:superposition_modulo_q}\\

$W$-state: $\frac{1}{\sqrt{n}}\sum_{i = 0}^{n-1}\ket{e_i}$ & $\mathO(n \log n)$ & $\mathO(1)$ & Section~\ref{sec:W_state_in_LAQCC}\\

Dicke-$(n,k)$, $k = \mathO(\sqrt{n})$: $\binom{n}{k}^{-1/2}\sum_{x \in \{0,1\}^n: |x| = k} \ket{x}$ &  $\mathO(n^2\log n)$ & $\mathO(1)$ 
&Section~\ref{sec:dicke:small_k}\\

Dicke-$(n,k)$: $\binom{n}{k}^{-1/2}\sum_{x \in \{0,1\}^n: |x| = k} \ket{x}$ & $\mathO(\text{poly}(n))$ & $\mathO(\log n)$ &Section~\ref{sec:Dicke_in_LAQCC_LOG}\\

QMBS: $\ket{S_k} = \frac{1}{k! \sqrt{\mathcal N(n,k)}}(Q^\dagger)^k \ket{\Omega}$ &  $\mathO(n^2\log n)$ & $\mathO(1)$  &  Section~\ref{sec:many_body_scar}
\end{tabular}
\caption{Summary of state preparation protocols given in this paper.}
\label{tab:sate_prep}
\end{table}
In the entry for the quantum many-body scar state $Q$ denotes the raising operator and $\mathcal N(n,k)=\binom{n-k-1}{k}$. 
Section~\ref{sec:many_body_scar} will provide more details on the variables and the implementation. 

\paragraph{Organization of the paper}
\noindent We first introduce relevant preliminaries in Section~\ref{sec:preliminaries}. 
In Section~\ref{sec:LAQCC_model} we formally define the class of Local Alternating Quantum-Classical Computations ($\LAQCC$). We also show that any Clifford circuit can be implemented in constant depth $\LAQCC$ (a result based on a result from measurement-based quantum computing~\cite{jozsa2006introduction}). 
This result allows us to give many useful multi-qubit gates and routines in Section~\ref{sec:gates_created_in_LAQCC}. 
Beyond that we show that constant depth $\LAQCC$ circuits are contained in $\QNC^1$ and that any $\mathsf{IQP}$ circuit has an $\LAQCC$ implementation.
We conclude this section with an analysis of a more powerful instantiation of $\LAQCC$ and show an inclusion with respect to the class $\mathsf{PostQPoly}$, which is the class of circuits of polynomial depth with one additional post-selection gate. 
In Section~\ref{sec:state_prep_in_LAQCC} we give $\LAQCC$ circuit implementations for preparing the uniform superposition over an arbitrary number of states, the $W$-state and the Dicke state up to $k = \mathO(\sqrt{n})$. We furthermore give a log-depth circuit implementation for preparing the Dicke state for any $k$. We conclude by showing a $\LAQCC$ circuit for generating many body scar states of a particular type of Hamiltonian.



% !TEX program = pdflatex
% !TEX root = main.tex


\section{The Model}

We represent a series of interactions between $N$ individuals as a sequence of weighted directed networks with adjacency matrix $A^t$ for $t=0,1,2,\ldots,T$. For each $t$, its entry $A_{ij}^t$ is the outcome of interactions $i \rightarrow j$ suggesting that $i$ is ranked above $j$. This allows both cardinal and ordinal inputs. For instance, in team sports, $A_{ij}^t$ could be the number of points by which team $i$ beat team $j$, or we could simply set $A_{ij}^t=1$ to indicate that $i$ won and $j$ lost. We can include the case where individuals interact multiple times at time $t$ by summing the corresponding entries.

We assume that the values of $A_{ij}^t$ are influenced by a vector of real-valued ranks $\v{s}^t=(s_{1}^t,\dots, s_{N}^t)$, where $s_i^t$ is $i$'s skill, strength or prestige at time $t$.
To model these interactions, we follow SpringRank's approach of imagining the network as a physical system~\cite{de2018physical}. Specifically, each node $i$ is embedded in $\mathbb{R}$ at position $s_i^t$, and each directed edge $i \rightarrow j$ becomes an oriented spring with a non-zero resting length and displacement $s_i^t-s_j^t$. Since we are free to rescale latent space and the energy scale, we set the spring constant and resting length to $1$. The spring corresponding to an edge $i \rightarrow j$ at time $t$ then has energy
\be\label{eqn:staticH}
H_{ij}(s_i^t,s_j^t)=\f{1}{2} \bup{s_i^t-s_j^t-1}^{2} \, .
\ee
If there were no other effects, the total energy of the system at time $t$ would then be 
\be\label{eqn:totalstaticH}
H^t(\v{s}^t) = \sum_{i,j=1}^{N} A_{ij}^t \,H_{ij}(s_i^t,s_j^t) \, .
\ee
If we determined $\v{s}^t$ by minimizing $H^t$ for each $t$ separately, we would simply be applying the static SpringRank model separately to each ``snapshot'' of the network. This would ignore all previous (and future) interactions, and ignore the hypothesis that ranks change smoothly from one time-step to the next.

% Figure environment removed

To model this smoothness, we also assume a dependence between ranks at successive time-steps. Specifically, we extend the Hamiltonian~\eqref{eqn:totalstaticH} with an extra term that models the \emph{self-interaction} between past and current ranks,
\begin{equation}\label{eqn:selfH}
\Hself^t(\v{s}^t,\v{s}^{t-1}) 
= \frac{\kself}{2} \sum_{i=1}^N (s_i^t-s_i^{t-1})^2 \, .
\end{equation}
This can be seen as a set of additional ``self-springs'' that connect the rank of each individual with its own previous rank. The spring constant $\kself$ parametrizes how smoothly we want the ranks to change from one step to the next. In inference terms, $\kself$ is a hyperparameter which we tune using cross-validation.

Summing over all time-steps $0 < t \le T$ and adding this to the pairwise interactions at each time-step then gives a total energy

\begin{align}\label{eqn:fullH}
\Htotal(\{\v{s}^t\}) = \sum_{t=0}^T H^t(\v{s}^t) + \sum_{t=1}^T \Hself^t(\v{s}^t,\v{s}^{t-1}) \, .
\end{align}
We call this the dynamical SpringRank Hamiltonian. The optimal ranks $\v{s}^0,\v{s}^1,\ldots,\v{s}^T$ are those that minimize it.


There are two ways to minimize $\Htotal$. One is to proceed in an online way, moving forward in time. In this approach, we use the static SpringRank model Eq.~\eqref{eqn:totalstaticH} to find the initial ranks $\v{s}^0$ by minimizing $H^0(\v{s}^0)$. As in Ref.~\cite{de2018physical}, the energy is unchanged if we add a constant to all the ranks; we can break this translational symmetry by setting the mean initial rank $(1/N) \sum_{i=1}^N v_i^0$ to zero.
Then, at each subsequent time-step $t \ge 1$, we update the ranks by taking into account both the new pairwise interactions and the self-springs connecting the ranks with their previous values. Namely, given $\v{s}^{t-1}$ and $A^t$, we find the ranks $\v{s}^t$ that minimize $H^t(\v{s}^t) + \Hself^t(\v{s}^t,\v{s}^{t-1})$.

Since this is a convex function of $\v{s}^t$, we can find its minimum by setting its gradient to zero, or equivalently by balancing all the forces $v_i^t$. This yields a system of linear equations:
\begin{align}\label{eqn:fullsolution}
\rup{ D^{out,t}+D^{in,t}- \bup{A^t + (A^t)^\dagger}+\kself\id} \,\v{s}^t
&=\rup{D^{out,t}-D^{in,t}}\v{1} \nonumber \\& +\kself\, \v{s}^{t-1} \, . 
\end{align}

Here 
$D^{out,t}$ and $D^{in,t}$ are diagonal matrices whose entries are the weighted out- and in-degrees $D^{out,t}_{ii}=\sum_{j}A^t_{ij}$ and $D^{in,t}_{ii}=\sum_{j}A^t_{ji}$; 
$\dagger$ denotes the transpose; 
$\id$ is the identity matrix; 
and $\v{1}$ is the all-ones vector.

The matrix on the left side of~\Cref{eqn:fullsolution} is invertible if $\kself > 0$. In particular, its eigenvector $\v{1}$ has eigenvalue $N \kself$. Thus for each $A^t$ and each $\v{s}^{t-1}$, Eq.~\eqref{eqn:fullsolution} has a unique solution $\v{s}^t$. Overall, Eq.~\eqref{eqn:fullsolution} is similar to the regularized version of SpringRank~\cite{de2018physical} with regularization parameter $\alpha= \kself$. However, unlike the static model, there is a term on the right-hand side containing the previous ranks $\v{s}^{t-1}$, creating a Markovian dependence between successive time-steps. We refer to this model as \dsrfull\ (\dsr).

Importantly the online DSR approach does not actually minimize $\Htotal$, instead solving a sequence of minimization problems, one for each time step. To minimize $\Htotal$ instead, we set $\nabla \Htotal(\v{s}^t) = 0$, solving for the minimizers $\v{s}^t$ over all $N(T+1)$ ranks simultaneously, yielding the following system of equations (SI \Cref{sec:h_total_derive}):

\begin{align}\label{eqn:h_total}
\rup{ D^{out,t}+D^{in,t} - \bup{A^t+(A^t)^\dagger} + 2\kself\id}\,\v{s}^t 
&=\rup{D^{out,t}-D^{in,t}}\v{1} \nonumber\\ 
& +\kself \,\bup{\v{s}^{t-1} + \v{s}^{t+1}} \, . 
\end{align}
This differs from \Cref{eqn:fullH} in that the right-hand side now includes both past and future ranks (which doubles the contribution of $\kself$ on the left). We remove the terms $\v{s}^{t-1}$ and $\v{s}^{t+1}$ for $t=0$ and $t=T$ respectively. This entire system has translational symmetry, since the energy Eq.~\eqref{eqn:fullH} remains the same if we add the same constant to all ranks at all times, but we can again break this symmetry by setting the mean rank to zero.

Additionally, in contrast to \Cref{eqn:fullsolution}, the ranks at $t$ now depend on both $t-1$ and $t+1$, which themselves depend on ranks at adjacent time-steps, so that ranks are affected by interactions in both the past and the future. In computer science, methods like this where the entire history is provided to the algorithm are called \emph{offline}, to distinguish them from \emph{online} approaches that update their results in real time as data becomes available. Thus we refer to this model as \nmdsrfull\ (\nmdsr).  

The cost of solving \Cref{eqn:fullsolution} for a single time-step is the same as static SpringRank with only one additional parameter to be tuned using cross-validation, and there are $T$ such $N$-dimensional equations to be solved successively. On the other hand, \Cref{eqn:h_total} requires solving a single  system of dimension $NT$, whose operator consists of $T$ blocks, each of dimension $N\times N$. While these two approaches feature numbers of non-zero entries that are fundamentally determined by the number of total edges across all time steps, the cost of solving \dsr vs \nmdsr will depend on the particular choice of linear solver~\cite{peng2021solving}.

Philosophically, Eqns.~\eqref{eqn:fullsolution} and~\eqref{eqn:h_total} are trying to do two different things. If we are given all the data $A^0,A^1,\ldots,A^T$ and we want to infer retrospectively how each individual's rank changed over time, it makes sense to include both past and future interactions as in~\eqref{eqn:h_total} so that $s_i^t$ is affected by $i$'s entire history. 

In contrast, \eqref{eqn:fullsolution} can be viewed as modeling each individual's perceived rank at the time, based only on the interactions that have occurred so far.

In principle, one could envisage other ways to formally incorporate an explicit dependence on  $\v{s}^{t-1}$ into the model, and we provide one example in SI \Cref{sec:sidynl}. However, we found that the approaches presented in this Section provide a natural interpretation, result in good prediction performance on both real and synthetic datasets (see \Cref{sec:results}) and are computationally scalable. 

We close this section with two possible extensions to these models. First, in some settings we might have timestamps $t$ that are not successive integers $0,1,\ldots,T$. In this case, if the time interval between two successive times is $\Delta t$, one could scale the spring constant of the self-springs between time-steps as $\kself/\Delta t$. This corresponds to the fact that if we have $\Delta$ identical springs in series, each of which is stretched by $(s^t-s^{t-1})/\Delta$, their total energy is $(1/2)(\kself/\Delta)(s^t-s^{t-1})^2$. The same expression applies if the timestamps are real-valued so that $\Delta$ is not an integer.

Second, if we believe that not just the ranks themselves but their rates of change behave smoothly over time, one could add a momentum term to the Hamiltonian which is quadratic in the discrete second derivative of the ranks. Since
\begin{gather*}
\left( (s^{t+1}-s^t) - (s^t-s^{t-1}) \right)^2
= \left( s^{t+1} - 2 s^t + s^{t-1} \right)^2 \\
= 2 (s^t-s^{t-1})^2 + 2 (s^{t+1}-s^t)^2 - (s^{t+1} - s^{t-1})^2 \, ,
\end{gather*}
this is equivalent to adding a repulsive force, i.e., a spring with negative spring constant, between ranks two time-steps apart. Note that the system nevertheless remains convex: this momentum term is positive semidefinite, so adding it to~\eqref{eqn:fullH} keeps the coupling matrix positive definite except for translational symmetry. Of course, these terms are second-order in time. In the online approach, one would have to determine $\v{s}^0$ from the static model, $\v{s}^1$ from the first-order model~\eqref{eqn:fullsolution}, and then use the model including this momentum term for $\v{s}^t$ for $t \ge 2$. We have not pursued this here, but it may make sense for certain datasets.


\subsection{Moving-window SpringRank}\label{subsec:mwsr}

Before we test the various versions of \dsrfull\ defined above, we consider a simpler model as a baseline. 
The simplest way to extend SpringRank to a dynamical context is to apply the static model to the interactions in a series of ``windows,'' where in each window we sum the interactions over a series of consecutive time-steps. For instance, we can compute $\v{s}^t$ for each $t$ by applying the static model to a window of width $\tau$, i.e., replacing $A^t$ with $\sum_{t'=t}^{t+\tau-1} A^{t'}$. Since these windows overlap, the resulting estimates $\v{s}^t$ will be smooth to some extent, even without imposing an explicit dependence between $\v{s}^t$ and $\v{s}^{t-1}$. We use this method, which we call \mwsrfull\ (\mwsr), as a baseline to compare with the dynamical models presented above.

Roughly speaking, a larger $\tau$ is like a larger self-spring constant $\kself$, since it induces more overlap between windows and thus a stronger correlation between the inferred ranks. However, like a decaying-history approach, \mwsr\ assumes a particular kernel for the importance of past time-steps: namely, that all $t'$ in the window are equally important. In contrast, \dsrfull\ infers the importance of past time-steps by coupling $\v{s}^t$ with $\v{s}^{t-1}$.

However, both models have a free parameter that needs to be tuned, i.e., $\kself$ and $\tau$. A shorter window $\tau$ or smaller spring constant $\kself$ allows the ranks to respond quickly to new interactions, while a longer window or larger spring constant more tightly couples nearby estimates. This trade-off suggests the existence of an optimal window length $\tau_{\opt}$. We tune $\tau$ using a cross-validation procedure as explained in SI \Cref{sisec:tuning}.


\subsection{Generative Model and Synthetic Data}
\label{sec:genmod}

Analogous to a model presented in~\cite{de2018physical}, we propose a probabilistic generative model for dynamic data. It takes as input the ranks $\v{s}^t$ and generates a sequence of weighted directed networks with adjacency matrix $A^t$ at time $t$. One can also imagine models that generate the ranks, for instance with a random walk with Gaussian steps whose log-probability is the self-spring Hamiltonian~\eqref{eqn:selfH}, but we treat $\v{s}^t$ as an input since we want the user of this model to have control over how the ground-truth ranks vary with time.  For instance, in our experiments below we generate synthetic data where the ranks vary sinusoidally.

The generative model has two real-valued parameters: a signal-to-noise ratio or inverse temperature $\beta$, and an overall density of edges $c$. Given the ranks $\v{s}^t$, it generates weighted, directed edges between each pair of nodes $i,j$ independently, as follows. The probability $P_{ij}^t(\beta)$ of $i$ ``beating'' $j$ at time $t$, giving a directed edge $i \to j$, is a logistic function as in~\cite{de2018physical} or the Bradley-Terry-Luce model~\cite{bradley1952,luce1959}:
\bea
\nonumber P_{ij}^t(\beta)=\frac{1}{1+\e^{-2\beta(s_i^t-s_j^t)}} \, .
\eea
The number of such edges, which gives the integer weight $A_{ij}^t$, is then drawn from a Poisson distribution whose mean $\lambda_{ij}^t$ is $cP^t_{ij}\,(\beta)$: 
\be
\label{generative_poiss}
A^t_{ij} \sim \Poi\left(\lambda_{ij}^t=\frac{c}{1+\e^{-2\beta(s_i^t-s_j^t)}}\right).
\ee
Since $P_{ij}^t(\beta) + P_{ji}^t(\beta)=1$, for any pair $i,j$ the total number of interactions $A_{ij}^t + A_{ji}^t$ is Poisson-distributed with mean $c$. The rank differences $s_i^t-s_j^t$ are used only to choose the directions of these edges. This  is equivalent to a model where we define a random multigraph where the number of edges between $i$ and $j$ is $\Poi(c)$, and then we choose the direction of each edge independently according to $P_{ij}^t$.

This is different from the generative model proposed in the static case in~\cite{de2018physical}. In that model the probability that $i$ and $j$ interact depends on $s_i-s_j$ so that nodes are more likely to interact if their ranks are fairly close. This is consistent with SpringRank's assumption that if $i$ beats $j$ then $j$ is below $i$, but not too far below it (since the springs have resting length $1$). This assumption makes sense for some datasets but not for others. By generating synthetic data without this dependence, our intent is to pose a greater challenge to SpringRank by modeling (for example) round-robin tournaments where every team plays each other.

\subsection{Model Evaluation}
\label{sec:testing}

Assessing a ranking model on real datasets is not straightforward since we do not know the true values of the underlying ranks. Nevertheless, we may measure the extent to which inferred ranks are accurate in the sense that they can predict the outcome of new observations. 

There are several performance metrics that can be used for prediction evaluation. From coarse-grained measures capable of predicting the likely winner to more fine-grained measures that also estimate odds, we consider four main metrics in our experiments, detailed in \Cref{sisec:evaluation}. We measure prediction performance using a cross-validation protocol where datasets are divided into training and test sets. The training set is used for hyperparameter tuning and parameter estimation while performance is evaluated on the test set. In order to preserve the chronological ordering of the data, the test set contains future observations, i.e., observations that chronologically follow those used in training. Hyperparameters for each method are tuned using grid-search in order to maximize the performance metrics as described in SI \Cref{sisec:tuning}.





%%% Local Variables:
%%% mode: latex
%%% TeX-master: "main"
%%% End:


%!TEX root = main.tex

%%%%%%%%%%%%%%%%%%%%%%%%%%%%%%%%%%%%%%%%%%%
\section{Overview of our Techniques}
\label{se:hl}
%%%%%%%%%%%%%%%%%%%%%%%%%%%%%%%%%%%%%%%%%%%

The objective of this section is to provide the reader with a general idea of our proof-labeling scheme. Our construction bears some similarities with the approach used in~\cite{FraigniaudMRT22} for the certification of \(\MSO_2\) properties on graphs of bounded tree-width, with certificates of size \(O(\log^2 n)\) bits. However, extending this approach to a proof-labeling scheme for graphs with bounded clique-width requires to overcome several significant obstacles. We therefore start by summarizing the main tools used for 
the certification of \(\MSO_2\) properties on graphs of bounded tree-width (see Section~\ref{subsec:recalltw}), and then proceed with the description of the new tools required for extending the result to graphs of bounded clique-width, to the cost of reducing the class of certified properties from  \(\MSO_2\) to \(\MSO_1\) (see Sections~\ref{subsec:cw-et-nlcw}-\ref{subsec:summarylabelsize}). 

%-----------------------------------------------------------------------------------------------------------
\subsection{Certifying $\MSO_2$ Properties in Graphs of Bounded Tree-Width}
\label{subsec:recalltw}
%-----------------------------------------------------------------------------------------------------------

Recall that a tree-decomposition of a graph \(G\) is a tree \(T\) where each node \(x\) of \(T\), also called \emph{bag}, is a subset of \(V(G)\), satisfying the following three conditions: 
(1) for every vertex \(v\in V(G)\) there is a bag \(x \in V(T)\) that contains~\(v\), 
(2) for every edge \(\{u,v\} \in E(G)\), there is a bag \(x\) containing both its endpoints, and 
(3) for each vertex \(v \in V(G)\), the set of bags that contains \(v\) forms a (connected) subtree of \(T\). 
Let  \(\Pi\) be an \(\MSO_2\) property, and let $T$ be a tree-decomposition of the  graph~$G$. The proof-labeling scheme aims at providing each vertex with sufficient information for certifying the correctness of~\(T\), as well as the fact that \(G\) satisfies~\(\Pi\). To do so, the certificate of each vertex is divided into two parts, one called \emph{main messages}, and the other called \emph{auxiliary messages}.

\subparagraph{Main messages.} 

 The main message of a node \(v\) is a sequence \(\textrm{seq}_v\) representing a path of bags in \(T\) that connects a leaf with the root, such that \(v\) is contained in at least one bag of \(\textrm{seq}_v\).  For each bag \(x \in \textrm{seq}_v\), the main message includes, roughly: the set of vertices contained in \(x\), the identifier of a vertex \(\ell_x\) in \(x\), called the \emph{leader} of \(x\), and a data structure \(c_x\) used to verify the \(\MSO_2\) property $\Pi$ on~\(G\). The leader \(\ell_x\) of \(x\) is chosen arbitrarily among the vertices of \(x\) that are adjacent to a vertex \(u\) belonging to the parent bag \(p(x)\) of \(x\) in \(T\). The vertex \(u\) is said to be \emph{responsible} for \(x\) in \(p(x)\). Let us assume the following consistency condition: for every bag \(x\) of \(T\),  every vertex in \(x\) received the same information about all the bags from \(x\) to the root of~\(T\). Under the promise that the consistency condition holds, it is possible to show that the vertices can collectively verify that \(T\) is indeed a tree-decomposition of~$G$, and that \(G\) satisfies~\(\Pi\).  

\subparagraph{Auxiliary messages.} 

The role of the auxiliary messages is precisely to check the above consistency condition.  For each bag \(x\), let  \(\tau_x\) be a Steiner tree \revision{(i.e. a minimal tree connecting a set of vertices denoted \emph{terminals})}  in $G$ rooted at the leader~\(\ell_x\), with all the nodes of \(x\) as terminals. Every vertex in \(\tau_x\) receives an auxiliary message containing the certification of~$\tau_x$ \revision{(each vertex of \(\tau_x\) receives the identifier of a root, of its parent and the distance to the root)}, and a copy of the information about \(x\) given to the nodes in the bag~$x$, through their main messages. By using the auxiliary messages, the leader $\ell_x$ can verify whether the subgraph $G[x]$ of \(G\) induced by the union of the bags in \(T[x]\) satisfies~$\Pi$, where \revision{\(T[x]\) the subtree of \(T\) containing \(x\) and all its descendants}. Specifically, this verification is performed by simulating the dynamic programming algorithm in Courcelle's Theorem~\cite{Courcelle90} as in the version of Boire, Parker and Tovey \cite{BoPaTo92}. This uses a \revision{constant-size} data structure \(c_x\) stored in the auxiliary messages that ``encodes'' the predicate \(\Pi(G[x])\). Its correctness can be verified by a composition of the values~\(c_y\) for each child \(y\) of \(x\) in~$T$. The tree \(\tau_x\) is actually used to transfer the information about \(c_y\) from the node $\ell_y$ in \(x\) responsible for~\(y\), to the leader \(\ell_x\).  

\subparagraph{Certificate size.} 

If \(T\) is of depth \(d\), then the main messages are of size \(O(d \log n)\) bits.  Crucially, for every graph \(G\), there is a tree-decomposition \(T\) satisfying that, for every bag \(x\), there is a Steiner tree \(\tau_x\) completely contained in \(G[x]\). Such a decomposition is called \emph{coherent} in~\cite{FraigniaudMRT22} (Lemma 3). It follows that every node participates in a Steiner tree with at most \(d\) bags, which implies that the auxiliary messages can be encoded in \(O(d \log n)\) bits. Thanks to a construction by Bodlaender~\cite{bodlaender1989nc}, it is possible to choose a coherent tree-decomposition with depth \(d = O(\log n)\), up to increasing the sizes of the bags by a constant factor only. It follows that the certificates are of size $O(\log^2n)$ bits. 
 
\bigbreak 

Our construction also follows the general structure described above. However, each element of this construction has to be adapted in a highly non-trivial way. Indeed, the grammar of clique-with, and the related structure of NLC decomposition, differ in several significant ways from the grammar of tree-width. The rest of the section is dedicated to providing the reader with a rough idea of how this can be done. 

%----------------------------------------------------------------------
\subsection{Clique-Width and NLC-Width}
\label{subsec:cw-et-nlcw}
 %----------------------------------------------------------------------
 
 First, instead of working with clique-width, it is actually more convenient  to work with the NLC-width, where NLC stands for \emph{node-label controlled}. Every graph of clique-width at most~$k$ has NLC-width at most~$k$, and every graph of NLC-width at most~$k$ has clique-width at most~$2k$~\cite{Johansson98}. As clique-width, NLC-width can be viewed as the following grammar for constructing graphs, bearing similarities with the grammar for clique-width:
 %
 \begin{itemize}
 \setlength\itemsep{0em}
 \item Creation of a new vertex $v$ with color $i\in\mathbb{N}$, denoted by $\mathsf{newVertex}_i$;
\item Given a set $S$ of ordered pairs of colors, and an ordered pair $(G,H)$ of vertex-disjoint colored graphs, create a new graph as the union of $G$ and $H$, then join by an edge every vertex colored~$i$ of~$G$ to every vertex colored~$j$ of~$H$, for all $(i,j)\in S$; this operation is denoted by $G \Join_S H$;
\item Recolor the graph, denoted by $\mathsf{recolor}_R$ where $R:\mathbb{N}\to\mathbb{N}$ is any function.

\end{itemize}
%
If $k\geq 1$ colors are used, a recoloring function~$R$ is a function $R:[k]\to [k]$. When $R$ is used,  for every $i\in [k]$, vertices with color~$i$ are recolored~$R(i)\in [k]$ (all colors are treated simultaneously, in parallel). Note that the recoloring operation in the definition of clique-width is limited to functions~$R$ that preserve all colors but one. Note also that, for $S=\varnothing$, the operation  $G \Join_S H$ is merely the same as  $G \parallel H$ for clique-width. We therefore use $G \Join_\varnothing H$ or $G \parallel H$ indistinctly. The NLC-width of a graph~$G$ is the smallest number of colors such that $G$ can be constructed using the  operations above. It is denoted by $\nlcw(G)$. For instance, the $n$-node clique can be constructed by creating a first node $v_1$ with color~1, and then repeating, for all $i=1,\dots,n-1$, (1)~the creation of a new node~$v_{i+1}$, with color~1 as well, and (2)~applying $v_{i+1} \Join_{\{(1,1)\}}K_i$ to get the clique~$K_{i+1}$ on $i+1$ vertices. Therefore, cliques can be constructed by using one color only, i.e., $\nlcw(K_n)=1$ for every $n\geq 1$. 

\subparagraph{NLC-decomposition.} For every $k\geq 1$,  the construction of a graph~$G$ with $\nlcw(G)\leq k$ can be described by a binary tree~$T$, whose leaves are the (colored) vertices of~$G$. In~$T$, every internal node~$x$ has an identified  left child~$x'$ and an identified right child~$x''$, and is labeled by~$\parallel$ or~$\Join_S$ for some non-empty set $S\subseteq [k]\times [k]$. This label indicates the operation performed on the (left) graph $G'$ with vertex-set equal to the leaves of the subtree~$T_{x'}$ of~$T$ rooted at~$x'$, and the (right) graph $G''$ with vertex-set equal to the leaves of the subtree~$T_{x''}$ of~$T$ rooted at~$x''$. That is, node $x$ corresponds to the operation $G_{x'}\parallel G_{x''}$ or $G_{x'}\Join_S G_{x''}$, depending on the label of~$x$.  In addition to its label ($\parallel$~or~$\Join_S$ for some~$S\neq\varnothing$), a node may possibly also include a recoloring function~$R:[k]\to [k]$, which indicates a recoloring to be performed \emph{after} the join operation, see Figure~\ref{fig:NLCdec} for an example.

% Figure environment removed

%-----------------------------------------------------------------------------------------------------------
\subsection{From Tree-Width to NLC-Width: The Main Messages}
%-----------------------------------------------------------------------------------------------------------

Let \(\Pi\) be an \(\MSO_1\) property, and let $T$ be an NLC-decomposition tree of a graph \(G\) with \(\cw(G)\leq k\). That is, we can choose the tree \(T\) as one using at most \(k\) colors. In the following, to avoid confusion, we call \emph{vertices} the elements of the vertex set of~\(G\), and \emph{nodes} the elements of the vertex-set of the decomposition tree~\(T\). The structure of our certificates differ from the one in~\cite{FraigniaudMRT22}, and now we decompose the certificate assigned to each node~$v$ into three parts: \emph{main messages}, \emph{auxiliary messages}, and \emph{service messages}. This subsection focuses on the main messages. 

Our main messages have, to some extent, a structure similar to the main messages used in~\cite{FraigniaudMRT22}  for the tree-width.  In particular, vertex \(v\) receives a sequence \(\textsf{path}(v)\), listing all the nodes, i.e., the whole set of operations, in the path from the root of  \(T\)  to the leaf of $T$ where \(v\) was created. For each node \(x\) in $\textsf{path}(v)$, the main message also includes the vertex identifier of a leader for~$x$, called \emph{exit vertex of \(x\)}, and denoted by \(\textsf{exit}(x)\). The main message also includes a data structure \(h(x)\) that encodes the truth value of the \(\MSO_1\) property on \(G[x]\). However, unlike the case of tree-width, where the nodes of the tree-decomposition are sets of vertices (i.e., bags) of bounded size, the contents of a non-leaf node in an NLC-decomposition tree \(T\) does not necessarily include information about the vertices created in \(T[x]\). For that reason, our proof-labeling scheme includes additional information in the main message of \(v\) in order to verify the correctness of the given decomposition. It may actually be worth providing a concrete example to explain the need for additional information. 

\subparagraph{Example.} 

Let us pick an arbitrary edge \(\{u,v\} \in E(G)\), and denote by \((x_1(u), \dots, x_{t_1}(u))\) and \((x_1(v), \dots, x_{t_2}(v))\)  the sequences \(\textsf{path}(u)\) and \(\textsf{path}(v)\), respectively, where \(x_1(u) = x_1(v)\) is the root of \(T\), and \(x_{t_1}(u)\) and \(x_{t_2}(v)\) are the nodes where \(u\) and \(v\) are respectively created.  Let \(x_1, \dots, x_{t_3}\) be the longest common prefix of these two sequences, i.e., the information contained in their main messages coincide on the first  \(t_3\) elements,  but \(x_{t_3+1}(u) \neq x_{t_3+1}(v)\). In the tree~$T$, \(x_{t_3+1}(u)\) and \(x_{t_3+1}(v)\) are two children of \(x_{t_3}\). The sequence of operations described in \(x_{t_1}(u), \dots, x_{t_3+1}(u)\) defines the color \(c(u)\) that \(u\) has in \(x_{t_3+1}(u)\). Similarly, the color \(c(v)\) of \(v\) in \(x_{t_3+1}(v)\) is defined by \(x_{t_2}(v), \dots, x_{t_3+1}(v)\). In order to create the edge \(\{u,v\}\), the operations described in \(x_{t_3}\) must specify a \(\Join\) operation between vertices with color~\(c(u)\) and vertices with color~\(c(v)\). However, the join operations described in an NLC-decomposition tree make a clear distinction between the left and right children of a  node. Therefore, in our example, for checking that the edge \(\{u,v\}\) is indeed correctly created in the given decomposition tree, the vertices \(u\) and \(v\) must be able to distinguish which of the two children \(x_{t_3+1}(u)\) and \(x_{t_3+1}(v)\) of \(x_{t_3}\) is the left child, and which one is the right child. 

\bigbreak

For a node \(x\) different from the root, let us denote by \(p(x)\) the parent of \(x\) in \(T\). The main message of~\(v\) includes 
a sequence \(\textsf{links}(v)\) that specifies, for each node \(x\) in \(\textsf{path}(v)\) different from the root, whether $x$ is the left or right child of \(p(x)\). For instance, in the example of Figure~\ref{fig:NLCdec}, we have  \(\textsf{links}(c) = (1,0)\), indicating that, to reach the leaf creating vertex $c$ from the root, one must follow the right child~(1), and then the left child~(0). S imilarly, \(\textsf{links}(d)=(1,1,0)\).  The sequences \(\textsf{links}\) are
also used to determine the longest common prefixes of the main messages, when the  same operations are repeated between two children of a same node (consider for instance the case where the same operation is performed at all the nodes of the decomposition tree). Back to our example above, let us suppose that the sequences \(\textsf{links}(u)\) and \(\textsf{links}(v)\) specify that \(x_{t_3+1}(u)\) is the left neighbor of \(x_{t_3}\), and \(x_{t_3+1}(v)\) is the right neighbor of \(x_{t_3}\). Using this information, \(u\) and \(v\) can  infer that it is an operation \(\Join_S\), with \((c(u), c(v)) \in S\) that is specified in the description of \(x_{t_3}\). 
With the given information, each vertex can thus check that all its incident edges are indeed created at some node of the decomposition tree~$T$. 

It remains to check that the decomposition does not define non-existent edges. To do so, the main message of every vertex \(v\) also includes, for each node \(x\) in \(\textsf{path}(v)\), and for each \(i \in [k]\),  the integers \(\textsf{color}_i(x)\) representing the number of vertices of \(G[x]\) that are colored \(i\) in the root of \(T[x]\). (Recall that the subgraph $G[x]$ is the subgraph of $G$ induced by the vertices created in the subtree \(T[x]\) of~$T$). Returning to our example, vertex \(v\) checks that it has exactly \(\textsf{color}_{c(u)}(x_{t_3+1}(u))\) neighbors with the same longest common prefix as \(u\) colored \(c(u)\) in the left children of \(x_{t_3}\). Also, vertex \(v\) checks, for each \(i\in [k]\), that the number of vertices colored~\(i\) in node \(x_{t_3}\) corresponds to the sum of the number of vertices colored \(j\) in \(x_{t_3+1}(u)\) and \(x_{t_3+1}(v)\), for each color \(j\) that is recolored \(i\) by the recoloring operation defined in \(x_{t_3}\). So, let us assume that the following consistency condition (analogous to the one for the certification of tree-decompositions) holds: 

\begin{description}
\item[C1:] For every pair of vertices \(u,v \in V(G)\), and for every  node \(x\) in both \(\textsf{path}(u)\) and \(\textsf{path}(v)\),  \(u\)~and \(v\) receive the same information about all nodes in the path from \(x\) to the root of \(T\) in their main messages, and 
\item[C2:]  If \(x\) is the root of \(T\), then the data structure \(h(x)\) describes an accepting instance (i.e., \(G\) satisfies \(\Pi\)). 
\end{description}

Assuming that the consistency condition is satisfied, it is not difficult to show that the vertices can collectively check that the given certificates indeed represent an NLC-decomposition tree, and that \(G\) satisfies \(\Pi\).  The difficulty is however in checking that the consistency condition holds. This is the role of the auxiliary and service messages, described next. 

%-----------------------------------------------------------------------------------------------------------
\subsection{Checking Consistency: Auxiliary, and Service Messages }
%-----------------------------------------------------------------------------------------------------------

We use auxiliary and service messages for allowing our proof-labeling scheme to check the first condition~\textbf{C1} of the consistency condition defined at the end of the previous subsection. 

Auxiliary messages can easily be defined for every node \(x\) of \(T\) satisfying that  \(G[x]\) is connected. In that case, the auxiliary messages of all the vertices \(v\) in \(T[x]\) contain the certificates for certifying a spanning tree \(\tau_x\) of \(G[x]\) rooted at the exit vertex of \(x\). Each vertex \(v\) can verify that the longest common prefix common to $v$ and its parent in \(\tau_x\) contains all the nodes from the root up to \(x\), and that the information given in the main messages coincide for all such nodes. Observe that every vertex \(v\) may potentially contain one auxiliary message for each node in  \(\textsf{path}(v)\). 

 The case where \(G[x]\) is not connected is fairly more complicated, and we need to introduce another type of decomposition. 

\subparagraph{NLC+ decompositions trees.}

 Observe that \(G\) itself is connected. Therefore, there must exist an ancestor \(z\) of \(x\) for which  \(G[z]\) is connected. We could provide the vertices in $G[z]$ with a spanning tree of \(G[z]\) for checking the consistency in \(T[x]\). However, the vertices in \(G[z]\) do not necessarily contain \(x\) in the prefixes of their node sequences, so we would have to put a copy of the main message associated to \(x\) on every node participating in the spanning tree. Since an NLC-decomposition tree does not allow to provide a bound on the distance between \(z\) and \(x\) in the tree, we have no control on how many copies of main messages a vertex should handle.
 
  Therefore, to cope with the case where \(G[x]\) is disconnected, we define a specific type of NLC decompositions trees, called NLC+ decompositions trees. The NLC+ decomposition trees are similar to NLC-decomposition trees, up to two important differences. 
\begin{itemize}
\item First, we allow the nodes corresponding to a \(\parallel\) operation to have arbitrary large arity, and thus NLC+ decomposition trees are not binary trees, as opposed to NLC-decomposition trees. 
\item Second, if a node \(x\) induces a disconnected subgraph \(G[x]\), then its parent node \(p(x)\) must satisfy that \(G[p(x)]\) is connected. Observe that  \(p(x)\) must then correspond to a \(\Join\) operation, and thus $p(x)$ has only two children: $x$ and another child, denoted by $y$. 
\end{itemize}

\subparagraph{Service trees.}

 A \emph{service tree} \(S_x\) for a node \(x\) such that $G[x]$ is disconnected is a Steiner tree in \(G[p(x)]\) rooted at the exit vertex of \(x\), and with all the vertices of \(G[x]\) as terminals. Each vertex of  \(S_x\) (i.e., all vertices in \(G[x]\), plus some vertices in \(G[y]\) is given a \emph{service message}, which contains the certificate for the tree \(S_x\), as well as a copy of the information about \(x\) given in the main messages of the vertices in \(G[x]\). Each vertex in \(S_x\) can then check that it shares the same information about \(x\) than its parent. The properties of the NLC+ decomposition guarantee that a vertex \(v\) participates to at most two service trees, for each node \(x\) in the sequence \(\textsf{path}(v)\). Indeed, vertex \(v\) necessarily participates in \(S_x\) when \(x\) is of type \(\parallel\),  and may also participate in \(S_y\) whenever  the sibling $y$ of \(x\) is of type \(\parallel\). There are significantly more subtle details concerning service trees, but they will be described in Section~\ref{se:cwd}.

It remains to check the second condition~\textbf{C2} of the consistency condition defined at the end of the previous subsection, which consists in verifying the correctness of \(h(x)\), for every node \(x\) of~\(T\). This is explained next. 

%-----------------------------------------------------------------------------------------------------------
\subsection{Dealing with $\MSO_1$ Predicates}
%-----------------------------------------------------------------------------------------------------------

In their seminal work, Courcelle, Makowsky and Rotics~\cite{CourcelleMR00} proved that every $\MSO_1$ predicate $\Pi$ on vertex-labeled graphs can be decided in linear time on graphs of bounded clique-width, and hence on graphs of bounded NLC-width, whenever a decomposition tree is part of the input. The running time of the algorithm is~$O(n)$, i.e., linear in the number~$n$ of vertices of the input graph, with constants hidden in the big-O notation that depend on the clique-width bound, on the number of labels, and on the $\MSO_1$ formula encoding the predicate~$\Pi$.  Note that this result does not hold for $\MSO_2$ predicates, which is why our proof-labeling scheme applies to $\MSO_1$ predicates only. We discuss the possible extension to $\MSO_2$ properties in the conclusion (see Section~\ref{sec:conclusion}).

For our purpose it is convenient to see the linear-time decision algorithm  as a dynamic programming algorithm over the NLC-decomposition tree of the input graph. We formalize this dynamic programming approach following the vocabulary and notations used by Borie, Parker and Tovey~\cite{BoPaTo92}. Note that the latter  provided an alternative proof of Courcelle's theorem, but for graphs of bounded tree-width, i.e., specific to a graph grammar defining tree-width.  To design our proof-labeling scheme, we adapt their approach to a graph grammar defining  NLC-width.

\subparagraph{Homomorphism Classes.}

For a fixed property $\Pi$ and a fixed parameter $k$, there is a finite set $\mathcal{C}$ of \emph{homomorphism classes} (whose size depends only on $\Pi$ and $k$) such that we can associate to each graph  $G$ of clique-width at most \(k\) its class $h(G) \in \mathcal{C}$ \revision{(for more details see Proposition \ref{pr:reg})}. Whenever $G$ is obtained from two graphs $G_1$ and $G_2$ by a $\Join_S$ operation potentially followed by a recoloring operation~$R$, the class $h(G)$ only depends on $h(G_1)$, $h(G_2)$, $S \subseteq [k] \times [k]$, and $R:[k] \to [k]$. This property also holds whenever $\Join_S$ is replaced by $\parallel$. Moreover, we also extend the notion to arbitrary arity so that it holds for the NLC+ decomposition trees. Importantly, Courcelle's theorem~\cite{CourcelleMR00} provides a "compiler" allowing to compute $h(G)$ whenever $G$ is formed by a single vertex of color $j \in [k]$, and to compute $h(G)$ from $h(G_1)$, $h(G_2)$, $S$ and $R$ whenever $G = R(G_1 \Join_S G_2)$. 

\subparagraph{Checking Condition \textbf{C2}.}

In our proof-labeling scheme, for each node $x$ of the NLC+ decomposition tree, we specify \(h(x)\) as the class $h(G[x])$. Following the same principles as before, the consistency of these classes can be checked by simulating a bottom-up parsing of the decomposition tree, in a way very similar to what we described before for checking the consistency of $\textsf{color}(x)$, but replacing the mere additions by updates of the homomorphism classes as described above. 

This completes the rough description of our proof-labeling scheme. 

%-----------------------------------------------------------------------------------------------------------
\subsection{Certificate Size}
\label{subsec:summarylabelsize}
%-----------------------------------------------------------------------------------------------------------

For each vertex \(v\), the main, auxiliary, and service messages of \(v\) can be encoded using \(O(\log n)\) bits for each node \(x\) in \(\textsf{path}(v)\), for the following reasons. 
\begin{itemize}
\item The main message associated to a node \(x\) contains the following information. First, the list of operations described in the node, which can be encoded in \(O(k^2)\) bits.  Second, the corresponding index of \(\textsf{links}\), which is just one bit representing whether \(x\) is the left or right children of its parent. Third, the homomorphism class \(h(x)\) that can be encoded in \(f(k)\) bits for some function~$f$ depending on the $\MSO_1$ property under consideration --- see the remark further in the text for a discussion about~$f$. Finally, it includes the node identifier of the exit vertex of \(x\), and the integers \(\textsf{color}_i\) for each \(i\in [k]\).  All these latter items can be encoded on  \(O(\log n)\) bits. 

\item The auxiliary message associated to node \(x\) (whenever \(G[x]\) is connected) corresponds to the certification of a spanning tree of \(G[x]\), which can be encoded in \(O(\log n)\) bits (see~\cite{KormanKP10}). 

\item For the service messages, note that vertex \(v\) participates in at most two service trees associated to \(x\): the one of \(x\) (whenever \(G[x]\) is disconnected), plus the one of the sibling \(y\) of \(x\) (when \(G[y]\) is disconnected). Again, each of these trees can be certified using \(O(\log n)\) bits.
\end{itemize}

Therefore, the total size of the certificates is \(O(d\cdot \log n)\) bits, where \(d\) is the depth of the NLC+ decomposition tree~\(T\). Our final certificate size depends then on how much we can bound the depth~$d$ of \(T\). Courcelle and Kant\'e \cite{courcelle2007graph} show that there always exists an NLC decomposition tree of logarithmic depth, but it comes with a price: the width of the small depth decomposition can be exponentially larger than the width of the original decomposition. Specifically, Courcelle and Kant\'e have shown that every \(n\)-node graph of NLC-width \(k\) admits an NLC-decomposition of width \(k\cdot 2^{k+1}\) such that the corresponding decomposition tree \(T\) has depth \(\mathcal{O}(\log n)\). Fortunately, our construction of NLC+ decomposition trees does not increase the depth of a given NLC-decomposition tree. In other words, we can use the result of Courcelle and Kant\'e to also show that  NLC+ decomposition trees have logarithmic depth. Overall, we conclude that the certificate size is \(O(\log^2 n)\) bits. 

\subparagraph{Remark.}

Our asymptotic bound on the size of the certificates hides a large dependency on  the clique-width \(k\) of the input graph. For certifying the NLC+ decomposition only, the constant hidden in the big-O notation is single-exponential in $k$, given that the width of the NLC+ decomposition tree with logarithmic depth grows to \(k\cdot 2^{k+1}\). However, for certifying an \(\MSO_1\) property, the dependency on \(k\) can be much larger, as it depends on the number of homomorphism classes.  It is known that, for  $\MSO_1$ properties, the number of homomorphism classes is at most a tower of exponentials in $k$, where the height of the tower depends on the number of quantifiers in the $\MSO_1$ formula. Moreover, this non-elementary dependency on $k$ can not be improved significantly~\cite{FrickGrohe04}. This exponential or even super-exponential dependency on the clique-width~$k$ is however inherent to the theory of algorithms for graphs of bounded clique-width. The same type of phenomenon
 occurs when dealing with graphs of 
 %bounded tree-depth~\cite{FeuilloleyBP22} or 
 bounded tree-width (see~\cite{FrickGrohe04}), and the proof-labeling scheme in~\cite{FraigniaudMRT22} is actually subject to the same type of dependencies in the bound~$k$. On the other hand, the certificate size of our proof-labeling scheme grows only polylogarithmically with the size of the graphs. 



%!TEX root = main.tex

\section{Certifying Cographs}\label{se:cographs}

In this section we describe a PLS for the recognition of cographs, using $O(\log n)$-bit certificates in $n$-node graphs. That is, this section is entirely dedicated to  proving Theorem~\ref{theo:cographs}. Our scheme uses two known technical lemmas. The first lemma (Theorem~4 in~\cite{courcelle2007graph}) states that every graph of NLC-width \(k\) admits an NLC-decomposition of \emph{logarithmic} depth, and width still bounded by a function of~$k$. 

\begin{lemma}[Courcelle and Kant\'e \cite{courcelle2007graph}]\label{lem:logdepth}
Every \(n\)-node graph of NLC-width \(k\) admits an NLC-decomposition of width \(k\cdot 2^{k+1}\) such that the corresponding decomposition tree \(T\) has depth \(\mathcal{O}(\log n)\).
\end{lemma} 

The second technical lemma used to establish Theorem~\ref{theo:cographs} states that every cograph has a spanning tree with very small diameter. 
% \(\tau\) of depth \(2\). We are going to use this lemma to gather the certificates of all vertices into a single one, which corresponds to the root of \(\tau\).

\begin{lemma}[Montealegre, Ram\'{i}rez-Romero, and Rapaport \cite{montealegre2021compact}]\label{lem:depth2tree}
Every cograph has a rooted spanning tree of depth~2 in which every node at depth~1 in the tree has at most one child.
\end{lemma}

Our proof  of Theorem~\ref{theo:cographs} is structured as follows. First, we describe the certificates assigned by the prover at each node. Next, we describe the verification algorithm, and we prove that the scheme satisfies soundness and completeness. Finally, we establish the desired upper bound on the size of the certificates.

\subsection{Certificate Assignment }  

Let $G=(V,E)$ be the considered graph, and let \(u\in V\).  The certificate $c(u)$ is divided in two parts, respectively called \emph{main message} and \emph{auxiliary message}.

\subparagraph{Main messages.} 

Lemma~\ref{lem:logdepth} states that there exists an NLC-decomposition of cographs, with width~4, and such that the corresponding decomposition tree \(T_{dec}\) has depth \(\mathcal{O}(\log n)\). These main messages are used to encode such an NLC-decomposition tree \(T_{dec}\).  At every node~$u$, the main message contains the following data:
\begin{itemize}
\item The identifier $\id(u)$, and an integer \(\textsf{deg}(u)\) representing the degree of \(u\) in $G$.

\item A sequence \(\textsf{path}(u) = (x_1(u), \dots, x_{d}(u))\) of values, representing a path in \(T_{dec}\) from the root of $T_{dec}$ to the leaf of $T_{dec}$ where \(u\) is created thanks to $\mathsf{newVertex}$. Here \(d = d(u)\) represents the length of \(\textsf{path}(u)\). For each \(i \in \{1, \dots, d\}\), the value \(x_i(u)\) is a list of all the operations (type of join, potential recoloring, etc.) performed at the $i$th node of the path, starting from the root. For simplicity, we also refer to  \(x_i(u)\)  as this $i$th node. 

\item A sequence \(\textsf{links}(u) = (\ell_1(u), \dots, \ell_{d}(u)) \in \{0,1\}^{d}\), representing the sequence of edges that are followed to reach the $i$th node \(x_i(u)\) of the path $\textsf{path}(u)$ from the root $x_1(u)$. More precisely, \(\ell_1(u) = 0\) and for each \( i \in \{2, \dots, d\}\),  \(\ell_i(u)= 0\) if \(x_{i}(u)\) is the left child of \(x_{i-1}(u)\), and \(\ell_i(u) = 1\) otherwise. 
\end{itemize}

Note that since the prover provides \(u\) with the whole list of  \(\mathcal{O}(\log n)\) operations from the node of $T_{dec}$ where $u$ is created  to the root of~$T_{dec}$, there is not enough space for assigning a unique identifier to each nodes of the tree~$T_{dec}$, as this would results in consuming $O(\log^2n)$ bits in $\textsf{path}(u)$. Instead, a node of the decomposition will be uniquely identified by the sequence \(\textsf{links}(u)\) and by the content of the values stored in \(\textsf{path}(u)\). We shall show that this is sufficient. 


\subparagraph{Auxiliary messages.} 

Lemma~\ref{lem:depth2tree} states that there is a rooted spanning tree $T_{span}$ of $G$, with depth~2, and in which every node at depth~1 in the tree has at most one child. The auxiliary messages are used to gather all the main messages on a single node \(r \in V\). Each node receives the information required to certify a depth-2 spanning tree $T_{span}$ rooted at $r$. In addition, every node  at depth~1 in $T_{span}$ receives the main messages of its child in~$T_{span}$. Formally, the auxiliary message provided by the prover to a node \(u\) contains the following data:

\begin{itemize}
\item The identifier $\rho(u)=\id(r)$ of the root $r$ of $T_{span}$.
\item An integer \(\textsf{depth}(u) \in \{0,1,2\}\) representing the depth of \(u\) in $T_{span}$.
\item If \(\textsf{depth}(u) = 2\), a node identifier \(\textsf{parent}(u)\) representing the parent of \(u\) in $T_{span}$.
\item If \(\textsf{depth}(u) = 1\), a variable \(\textsf{child}(u)\), either representing the node identifier of the child of \(u\) in $T_{span}$, or \(\bot\) if \(u\) has no children in $T_{span}$.
 \item If \(\textsf{depth}(u) = 1\), and \(\textsf{child}(u) \neq \bot\), then \(u\) receives a variable \(\textsf{M}(u)\) representing  the main message of \(\textsf{child}(u)\).
\end{itemize}

\subsection{Verification Scheme} 

We now describe the verification algorithm performed by every vertex \(u\) of the actual graph. First,  \(u\) verifies that \(\textsf{deg}(u)\) corresponds to its degree, and that all the values stored in \(\textsf{path}(u)= (x_1(u), \dots, x_{d}(u))\) effectively correspond to a list of operations of an NLC-decomposition of width~4. In particular, $u$ checks that  \(x_{d}(u)\) contains the unique operation  \(\textsf{NewVertex}_i\) for some \(i\in \{1,\dots,4\}\). Concretely, after having shared its certificate with its neighbors,   \(u\) checks the following conditions, for each \(v \in N(u)\):

\begin{enumerate}
\item \label{item:pierre-1} \(\textsf{links}(u)\) and \(\textsf{links}(v) \) have a common prefix. More precisely, \(u\) checks that there exists an index \(i \in \{1, \dots, d(u)\}\)  such that \(\ell_j(u) = \ell_j(v)\) for every \(j \leq i\). Over all such indices  we denote by \(i^*\) the maximum one.

\item \label{item:pierre-2} \(\textsf{path}(u)\) and \(\textsf{path}(\revision{v})\) share the same the first \(i^*\) coordinates. More precisely, \(u\) verifies that \(x_j(u) = x_j(v)\) for every \(j \leq i^*\).

\item \label{item:pierre-3} Let \(\mathsf{currentcolor}(u,i^*)\) be the color of \(u\)  resulting from the \(\mathsf{NewVertex}\) operation specified in \(x_{d(u)}\), and all the \(\mathsf{recolor}\) operations in all the nodes in \(\textsf{path}(u)\) up to \(x_{i^*}(u)\), but not including the \(\mathsf{recolor}\) operations in \(x_{i^*}\)). The value \(\mathsf{currentcolor}(v,i^*)\) is defined the same for node~$v$. The following holds:
\begin{itemize}
\item  If \(\ell_{i^*}(u) = 0\) then \(u\) checks that  the join operation $\Join_S$ in  \(x_{i^*}\) satisfies  
\[(\mathsf{currentcolor}(u,i^*), \mathsf{currentcolor}(v,i^*)) \in S.\]
 \item If \(\ell_{i^*}(u) \neq 0\) then \(u\) checks that  the join operation $\Join_S$ in  \(x_{i^*}\) satisfies  
 \[(\mathsf{currentcolor}(v,i^*),\mathsf{currentcolor}(u,i^*)) \in S.\]
 \end{itemize}
\item \label{item:agree-root} Nodes $u$ and $v$ agree on the root of $T_{span}$, i.e., \(\rho(u) = \rho(v)\).  

% For simplicity, in the following we denote \(\rho = \rho(u)\). 

 \item \label{item:there-exists-parent} If \(\textsf{depth}(u)=2\) then \(u\) checks that there exists \(v \in N(u)\) such that \({\textsf{parent}(u) = v}\). 
 
\item \label{item:neighbor-root} If \(\textsf{depth}(u)=1\), then \(u\) checks that 
\begin{itemize}
\item \(\rho(u) \in \{\id(v):v\in N(u)\}\);
\item If  \(\textsf{child}(u)  \neq \bot\) then (1)~\(\textsf{child}(u) \in N(u)\),  (2)~\(\textsf{depth}(\textsf{child}(u)) = 2\), and (3)~every \(v \in N(u)\smallsetminus \{\textsf{child}(u) \} \) with  \(\textsf{depth}(v) = 2\) satisfies that \(\textsf{parent}(v) \neq u\);
\item If  \(\textsf{child}(u)  \neq \bot\) then \(\textsf{M}(u)\) equals the main message of \(\textsf{child}(u)\).
\end{itemize}

\item  \label{item:number-9-number-9} If \(\id(u) = \rho(u)\) then, \(u\) checks that the following holds\footnote{If all previous conditions are satisfied, then the root \(r=u\) obtains from its neighbors all the main messages of the nodes in~\(G\)}:
\begin{itemize}
\item The information in \(\{\textsf{path}(v): v\in V\}\) and \(\{\textsf{links}(v):v\in V\}\) is consistent, that is, for every \(v_1, v_2 \in V\), and for  every \(i \in \mathbb{N}\),
\[ \Big(\forall j \leq i, \; \ell_j(v_1) = \ell_j(v_2) \Big) \Longrightarrow \Big (\forall j \leq i,  \; x_j(v_1) = x_j(v_2)\Big).\]
Observe that if this condition is satisfied, then necessarily \(\{\textsf{path}(v): v\in V\}\) and \(\{\textsf{links}(v):v\in V\}\) describe a unique NLC-decomposition tree. Let us denote this tree by~\(T(u)\), and let \(G^*\) be the graph corresponding to the realization of the NLC-decomposition tree given by \(T(u)\). Note that \(u\) can obtain all the vertices and edges of \(G^*\) from \(T(u)\). 
\item Node \(u\) checks that \(G^*\) is a cograph.
\item Finally \(u\) checks that, for every node \(v \in V\),  the number of neighbors of \(v\) in \(G^*\)  equals \(\textsf{deg}(v)\). 
\end{itemize}
\end{enumerate}

\subsection{Completeness and Soundness} 

The completeness of our scheme directly follows from the fact that, thanks to Lemma~\ref{lem:logdepth}, every cograph admits a NLC-decomposition of width \(4\), and,  by Lemma \ref{lem:depth2tree}, one can define the main and auxiliary messages in a way that every node accepts.

For the soundness, let us assume that all nodes of a graph $G$ accept in the verification protocol. From condition~\ref{item:agree-root}, we have that all nodes agree on the same root $r$ of $T_{span}$. From the first item of condition~\ref{item:neighbor-root}, all nodes of depth~1 in $T_{span}$ are adjacent to~$r$. From condition~\ref{item:there-exists-parent}, and from the second item of condition~\ref{item:neighbor-root}, we have that every node at depth \(2\) is adjacent to a node of depth~1. Finally, the third item of condition~\ref{item:neighbor-root} guarantees that the root~$r$ receives all the main messages of the nodes \(G\). Then, by the first item of condition~\ref{item:number-9-number-9}, we have that $r$ can recover all the vertices and all the edges of \(G^*\), and by the second item of condition~\ref{item:number-9-number-9}, we have that \(G^*\) is a cograph.  From conditions \ref{item:pierre-1}-\ref{item:pierre-3}, we get that every edge \(\{u,v\} \in E(G)\) is necessarily included in \(G^*\), meaning that \(G\) is a spanning subgraph of \(G^*\). Finally, the third item of condition~\ref{item:number-9-number-9} guarantees that the number of edges in \(G^*\) equals the number of edges in \(G\), and thus \(G =G^*\). It follows that \(G\) is indeed a cograph.

\subsection{Certificate Size} 

By Lemma~\ref{lem:logdepth}, every cograph admits an NLC-decomposition tree of depth \(\mathcal{O}(\log n)\). Therefore, in the certificate assigned to node~$u$, we have \(d(u) = \mathcal{O}(\log n)\). Each value $x_i(u)$ can be encoded using $O(1)$ bits as  every operation involves a constant number of colors. It follows that  \(\textsf{path}(u)\) can be encoded on $O(\log n)$ bits. The variable \(\textsf{links}(u)\) can be encoded with \( \mathcal{O}(\log n)\) bits as well, by construction. Every node identifier,  and every node degree can be encoded with \( \mathcal{O}(\log n)\) bits. Therefore, all the main messages can be encoded with \( \mathcal{O}(\log n)\) bits. This also implies that the auxiliary messages can be encoded with \( \mathcal{O}(\log n)\) bits. We conclude that, in total, our PLS uses certificates on  \( \mathcal{O}(\log n)\) bits, which completes the proof of Theorem~\ref{theo:cographs}. 




%:
%!TEX root = main.tex

\section{$\MSO_1$ Properties on Labeled Graphs of Bounded Clique-Width}\label{se:cwd}

This section is dedicated to the proof of Theorem~\ref{theo:main}. To avoid overloading the notation, a labeled graph is simply denoted by~$G=(V,E,\ell)$ where $\ell$ is the vertex-labeling function. We will often simply speak of "graphs" instead of "vertex-labeled graphs". 
Before describing the certificates and the verification protocol, let us first establish some preliminary technical results. 

\subsection{Regularity of $\MSO_1$ predicates}\label{sse:reg}

In their seminal work~\cite{CourcelleMR00}, Courcelle, Makowsky and Rotics proved that any $\MSO_1$ predicate $\Pi$ can be decided in linear time on graphs of bounded clique-width, and hence on graphs of bounded NLC-width, whenever a decomposition tree is part of the input. The running time of the algorithm is linear in the number~$n$ of vertices of the input graph, but the constant hidden in the big-O notation depends on~$k$, on the number of labels and on the $\MSO_1$ formula encoding the predicate $\Pi$. The algorithm in~\cite{CourcelleMR00} is described using tools from automata theory. For our purpose it is more convenient to see it as a dynamic programming algorithm over the decomposition tree of the input graph. Let us formalize this dynamic programming approach following the vocabulary and notations that Borie, Parker and Tovey~\cite{BoPaTo92}. Note that~\cite{BoPaTo92} is alternative proof of Courcelle's theorem on bounded treewidth graphs, specific to a graph grammar defining treewidth; here we simply adapt the definitions to NLC-width and the NLC grammar.

Let $\NLC_k$ denote the class of labeled graphs of graphs of $\NLC$-width at most $k$ and let $\Pi$ be a graph property, assigning to each graph $G$ a boolean value $\Pi(G)$. Intuitively, two graphs $G_1,G_2  \in \NLC_k$ can be considered as equivalent w.r.t. $\Pi$ if, whenever a graph $G'_1 \in \NLC_k$ is obtained by a sequence of NLC operations performed on $G_1$, then the graph $G'_2$ obtained from the same sequence of operations but performed on $G_2$ has the same behaviour w.r.t. property $\Pi$ as $G'_1$. Informally again, a property $\Pi$ is said to be \emph{NLC-regular} if the number of such equivalence classes, that will be called \emph{homomorphism classes} as in~\cite{BoPaTo92}, is upper bounded by a constant: the number of such classes does not depend on the size of the graphs, but only on parameter $k$ and the property $\Pi$ itself. As we shall see, if $\Pi$ is an $\MSO_1$-definable boolean predicate, then it is also NLC-regular, and this is the crux for deciding $\MSO_1$ properties for graphs of bounded NLC-width.

\begin{definition}[NLC-regular property]\label{de:reg}
A graph property $\Pi$ is called \emph{NLC-regular} if, for any value $k$, we can associate a finite set $\cC$ of \emph{homomorphism classes} and a \emph{homomorphism function} $h$, assigning to each graph $G \in \NLC_k$ a class $h(G) \in \cC$ such that:
\begin{enumerate}
\item If $h(G_1) = h(G_2)$ then $\Pi(G_1) = \Pi(G_2)$.
\item For each operation $\Join_S$ there exists a function $\odot_{\Join_S}: \cC \times \cC \rightarrow \cC$ such that, for any two graphs $G_1$ and $G_2$ in $\NLC_k$,
 $$h(\Join_S(G_1,G_2)) = \odot_{\Join_S}(h(G_1),h(G_2)),$$
and for each  operation $\recolor_R$ of there is a function $\odot_{\recolor_R}: \cC \rightarrow \cC$ such that, for any graph $G \in \NLC_k$,
$$h(\recolor_R(G)) = \odot_{\recolor_R}(h(G)).$$
\end{enumerate}
\end{definition}

Observe that NLC-regularity of property $\Pi$ not only implies that the set of homomorphism classes does not depend on the size of the graph, but also that, given an NLC-decomposition of some graph $G \in NLC_k$, the class $h(G)$ can be computed by dynamic programming from the leaves to the root. At each node $x$ of the decomposition tree $T$, the class of $G[x]$, the subgraph of $G$ corresponding to the subtree $T[x]$ rooted at $x$ only depends on the classes at the children nodes, and the operations at node $x$. The first condition of Definition~\ref{de:reg} partitions $\cC$ into a set of accepting classes, i.e., classes $c \in \cC$ such that $h(G) = c \rightarrow \Pi(G)$, and rejecting classes, corresponding to graphs that do not satisfy the property. %An example illustrating Definition~\ref{de:reg}  is provided in Appendix~\ref{example:de:reg}. 


We illustrate Definition~\ref{de:reg} on the predicate $\Pi$ corresponding to non-3-colorability, in order to prove that this predicate is NLC-regular, and how this regularity allows to decide the non-3-colorability on graphs in $\NLC_k$. 
It is  convenient to view a proper 3-coloring of a graph $G$ as a partition of its vertex $V$ set into three independent sets $(X_1, X_2, X_3)$. For such a partition $(X_1, X_2, X_3)$ of a graph $G \in \NLC_k$, $i \in [k]$, we encode this partition as a triple $(b_1,b_2,b_3)$ of boolean vectors of length $k$. Vector $b_j$ encodes the intersection of set $X_j$ with the $k$ possible colours of the NLC decomposition of $G$. That is, $b_j[i] = 1$ if set $X_j$ contains some vertex coloured $i \in [k]$, otherwise $b_j[i] = 0$. Eventually, the homomorphism class $h(G)$ of graph $G$ is the set of all triples $(b_1,b_2,b_3) \in \{0,1\}^k \times  \{0,1\}^k \times \{0,1\}^k$, such that there is some partition of the vertices $G$ into three independent sets $(X_1, X_2, X_3)$ and $(b_1,b_2,b_3)$ is the encoding of this partition as described above. In particular, observe that $G$ is non-3-colourable iff $h(G) = \emptyset$, so $\emptyset$ is the only accepting class.

It is a matter of exercise to see that the number of homomorphism classes is upper bounded by a function on $k$, and even to understand how to construct the class of graphs with a single vertex, and how functions $\odot_{\Join_S}$ and $\odot_{\recolor_R}$ of Definition~\ref{de:reg} can be obtained for all $\Join_S$ and $\recolor_R$ operations. For the sake of completeness, we give the construction in full details.

If $G$ consists of a single vertex $x$ coloured $i$, then the possible partitions of $G$ into three independent sets are $(\{x\},\emptyset,\emptyset)$,  $(\emptyset,\{x\},\emptyset)$ and $(\emptyset,\emptyset,\{x\})$. Thus $h(G)$ is formed by three triples: $(u_{i,k},0^k,0^k)$, $(0^k,u_{i,k},0^k)$, and $(0^k,0^k,u_{i,k})$, where $0^k$ denotes the boolean vector of $k$ zeros, and $u_{i,k}$ is formed of $k-1$ zeros, and a one at position $i$.

Consider now two graphs $G_1,G_2 \in \NLC_k$ and let $G = G_1 \Join_S G_2$ for some $S \in [k] \times [k]$. We describe function $\odot_{\Join_S}$, constructing $h(G)$ from $h(G_1)$ and $h(G_2)$. Note that for each 3-partition $(X_1,X_2,X_3)$ of $G$ into three independent sets, the intersection of $X_1,X_2,X_3$ with the vertex set of $G_1$ (resp. $G_2$) induces a 3-partition $(Y_1,Y_2,Y_3)$ (resp. $(Z_1,Z_2,Z_3)$) into independent sets. Conversely, given partitions  $(Y_1,Y_2,Y_3)$ of $G_1$ and $(Z_1,Z_2,Z_3)$ of $G_2$ into independent sets, $(X_1 = Y_1 \cup Z_1, X_2 = Y_2 \cup Z_2, X_3 = Y_3 \cup Z_3)$ forms a partition into independent sets of $G$ unless operation $\Join_S$ creates an edge between $Y_1$ and $Z_1$, or $Y_2$ and $Z_2$, or $Y_3$ and $Z_3$. Therefore, the homomorphism class of $h(G)$ can be constructed as follows. For each triple of boolean vectors $(y_1,y_2,y_3) \in h(G_1)$ and $(z_1,z_2,z_3) \in h(G_2)$ (corresponding to partitions $(Y_1,Y_2,Y_3)$ of $G_1$ and $(Z_1,Z_2,Z_3)$ of $G_2$ respectively), we add to $h(G)$ the triple $(y_1 \lor z_1, y_2 \lor z_2, y_3 \lor z_3)$ (where $x \lor y$ denotes the bit-wise OR operation), unless there is some pair $(p,q) \in  S$ such that $y_1[p] \land z_1[q]$ or  $y_2[p] \land z_2[q]$ or  $y_3[p] \land z_2[q]$ is true. The latter condition verifies that operation $\Join_S$ does not create an edge in $X_1$ or in $X_2$ or in $X_3$ in graph $G$. 

Eventually, let $G' = \recolor_R(G)$ for some $G \in \NLC_k$, and $R:[k] \to [k]$. We describe function $\odot_{recolor_R}$. Note that $(X_1,X_2,X_3)$ is a partition of $G'$ into independent sets iff it is also a partition of $G$ into independent sets. Therefore we only need to describe how the recoloring $R$ changes the encoding of $(X_1,X_2,X_3)$ from $G$ to $G'$. For each $(b_1,b_2,b_3) \in h(G)$, we add $(\recolor_R(b_1),\recolor_R(b_2), \recolor_R(b_3))$ to $h(G')$, where $b' = \recolor_R(b)$ is defined as follows on boolean vector $b \in \{0,1\}^k$: for each $q \in [k]$ we set $b'[q] = 1$ if and only if there is some $p \in [k]$ such that $R(p) = q$ and $b[p]=1$. In full words, $b'[q]$ is set to true iff the set $X$ encoded by $b$ contains some vertex colored $p$ in $G$, recolored $q$ in $G'$.

This does not only prove that property $\Pi(G)$: ``$G$ is non-3-colourable'' is NLC-regular, but provides all ingredients to decide the property for graphs $G \in \NLC_k$, when a decomposition tree is part of the input. Indeed we need to compute, at each node $x$ of the decomposition tree, the homomorphism class of $G[x]$, the induced subgraph corresponding to thee $T[x]$. At the leaves, graph $G[x]$ is formed by a unique vertex, and as described above the homomorphism class is defined by the colour of that vertex. Then, for each node of the tree, the homomorphism class can be updated from the classes of its children using functions $\odot_{\Join_S}$ and $\odot_{\recolor_R}$. Finally, at the root, we accept if and only if the homomorphism class of the whole graph is the empty set. The running time is linear in the number of nodes of the tree, so it is $O(n)$~--- of course it also depends (exponentially) in parameter $k$.\\

% \subsubsection{Regularity of  $\MSO_1$ predicates}

The result of Courcelle, Makowsky and Rotics~\cite{CourcelleMR00}, although expressed in terms of automata theory (see also Theorem 4.2 in~\cite{GanianH10} for an alternative proof), can be restated as follows:

\begin{proposition}\label{pr:reg}
Any graph property $\Pi$ expressible by an $\MSO_1$ predicate is NLC-regular. Moreover, given the corresponding $\MSO_1$ formula and parameter $k$, one can explicitely compute the set of homomorphism classes $\cC$, as well as   functions $\odot_{\Join_S}$ and $\odot_{\recolor_R}$ for any $S \in [k] \times [k]$ and any $R : [k] \to [k]$, and the homomorphism class of the graph formed by a unique vertex coloured $i$, for any $i \in [k]$.
\end{proposition}

Therefore all $\MSO_1$ properties can be decided in $O(n)$ time on graphs of bounded NLC-width, if a decomposition tree is part of the input; again, the big-Oh notation hides a dependency in $k$ and the $\MSO_1$ formula. We also use these ingredients for our proof labeling scheme.

\subsection{General description $\NLC_+$-Width}

Before getting into the full details of our PLS, let us give a general description of the certification algorithm for an $\MSO_1$ property $\Pi$ on graphs of $\NLC$-width at most~$k$. As in the case of cographs (cf. Section~\ref{se:cographs}), given a graph $G \in \NLC_k$, we use an NLC-decomposition tree~$T_{dec}$ of depth $\mathcal{O}(\log n)$ and of width $k' \leq k\cdot 2^{k+1}$, provided by Lemma~\ref{lem:logdepth}. For any vertex $u$ of $G$, recall that $\textsf{path}(u) = (x_1(u), \dots, x_d(u))$ denotes the path in $T_{dec}$ from the root $x_1(u)$ to the leaf $x_d(u)$ corresponding to the node where vertex $u$ is created (again, we abuse notation by identifying the nodes of the tree with the values describing the operations performed in those nodes). 

As in Section~\ref{se:cographs}, the so-called main message of $u$ contains its identifier~$\id(u)$, the sequence $\textsf{path}(u)$, as well as the sequence $\textsf{links}(u)  = (\ell_1(u), \dots, \ell_{d}(u))$, where, for \(i \geq 2\), $\ell_i(u)\in\{0,1\}$ indicates whether $x_{i}(u)$ is the left or right child of $x_{i-1}(u)$ in~$T_{dec}$, when \(x_{i-1}(u)\) is of type \(\Join\) . For certifying a predicate $\Pi$, let $h_i(u)$ denote the homomorphism class at node $x_i(u)$ w.r.t. ${\Pi}$ restricted to graphs of $\NLC$-width at most \(k'\). That is, for $i=1,\dots,d$, 
\[
h_i(u)=h(G[x_i(u)]).
\]
The sequence 
\begin{equation}\label{eq:sequence-h}
\textsf{h}(u) = (h_1(u), \dots, h_d(u))
\end{equation}
is added to the main message of $u$. Note that, since $d \in \mathcal{O}(\log n)$, each sequence is of logarithmic length. Moreover, for every $i\in\{1,\dots,d\}$,  $(x_i(u), \ell_i(u), h_i(u))$ can be encoded on $O(1)$ bits. Therefore the total size of the main message is $\mathcal{O}(\log n)$. 

Unlike the case of cographs, the diameter of a graph of bounded NLC-width is not necessarily bounded by a constant (i.e., Lemma~\ref{lem:depth2tree} does not hold in general for such graphs). It follows that the main messages cannot be gathered in a single vertex as in cographs. To overcome this difficulty, the prover places additional information in  the main message of $u$. First, %instead of providing only the degree of $u$ in $G$, the prover provides the sequence of degrees 
%\[
%\textsf{deg}(u) = (\deg_1(u), \dots, \deg_d(u))
%\]
%where, for $i=1,\dots,d$, $\deg_i(u)$ denotes the degree of $u$ in $G[x_i(u)]$. 
for each color $j \in [k']$ and \(i \in [d(u)]\) let $\textsf{color}_i^j(u)$ denote the number of vertices colored \(j\) in $G[x_i(u)]$ after the recoloring operations performed at node~$x_i(u)$. The sequence 
\[
\textsf{color}^j(u) = (\textsf{color}^j_1(u), \dots, \textsf{color}^j_d(u))
\]
is also added to the main message of $u$, for all $j\in [k']$. 

Finally, we need to guarantee that, for each node \(x\), every vertex that belongs to \(G[x]\) receives the same information about the operations performed in \(x\), as well as the number of vertices colored with each one of the colors.  This verification is especially hairy for nodes \(x\) where \(G[x]\) is disconnected, where this consistency verification is done with the help of vertices outside \(G[x]\).  In order to cope with this issue, let us slightly modify the notion of NLC-decomposition tree by allowing nodes~$\parallel$ (i.e., nodes of $T_{dec}$ at which a disjoint union operation is performed) to be of arbitrarily large arity (in the original decomposition $T_{dec}$, the arity of an internal node is~2). Moreover all disconnected graphs $G[x]$ will correspond to parallel nodes $x$, and for the parent $y$ of such a node, $G[y]$ has to be connected. Such a decomposition tree will be called an $\NLC_+$-decomposition tree. 

%More precisely, additional information must be added to the certificates for ensuring consistency of the corresponding main messages. 


%The information given in the main messages would be sufficient for verifying~$\Pi$ if, for each node $x$ of the decomposition tree $T_{dec}$, the graph corresponding to $T[x]$ were connected. However, such a connectivity property does not necessarily hold, and it is required to handle the situation where there are nodes $x$ for which $T[x]$ is disconnected. The difficulty comes from the fact that the vertices of $G$ belonging to different connected components do not communicate in the verification part of the protocol\pierre{what do you mean?}. 
\begin{definition}\label{def:nlc+}
Let $k\geq 1$. A rooted tree $T$ is an $\NLC_+$-decomposition tree of width \(k\) if the following conditions holds:  
 \begin{enumerate}
 \setlength\itemsep{0em}
 \item Every leaf of $T$ is labeled $\mathsf{newVertex}_i$, for some \(i \in [k]\);
 \item Every internal node of \(T\) is labelled \(\Join\) or \(\parallel\);
\item Every node labeled  \(\Join\) has exactly \(2\) children, and such a node is associated to a set $S\in [k]\times[k]$ and to a function $\mathsf{recolor}_R$ where $R:[k] \to[k]$;
\item Every node labeled \(\parallel\) has at least 2 children, and, for every node labeled \(\parallel\) distinct from the root, its parent is labeled \(\Join\);
\item \label{item:connectivityNLC+} Every graph defined by the subtree rooted in a \(\Join\) node is connected.
\end{enumerate}
\end{definition}

In Definition~\ref{def:nlc+}, the nodes labeled  \(\Join\)  represent the join and recoloring operations performed as in \(\NLC\)-decomposition trees, i.e., first join, and then recolor.  Similarly, the nodes \(\parallel\)  represent the disjoint union of the graphs defined by their children. However, a crucial difference compare to $\NLC$-decomposition is that instead of systematically involving two vertex-disjoint graphs, the $\NLC_+$-decomposition allows an arbitrary number of vertex-disjoint graphs. Another crucial difference with $\NLC$-decomposition is Condition~\ref{item:connectivityNLC+}, which imposes connectivity of the subgraph hanging at every  \(\Join\) node. 

To cope with the notion of $\NLC_+$-decomposition, we merely extend the functions $\odot$ given in Definition~\ref{de:reg} by introducing the operator \(\odot_{\parallel}\) with arbitrary arity. Specifically, let \(G\) be a graph obtained from the disjoint union of a set of \(\NLC_+\) graphs $\{G_1, \dots, G_p\}$, in any order, with \(p\geq 2\). We define 
\begin{align*}
h(G)  =  \odot_{\parallel }(h(G_1), \dots, h(G_p)) = \odot_{\parallel }\Bigg(h(G_1), \odot_{\parallel }\bigg(h(G_2), \odot_{\parallel }\Big(\dots, \odot_{\parallel }\big(h(G_{p-1}), h(G_p)\big)\Big)\bigg)\Bigg). 
\end{align*}
We say that a graph has \(\NLC_+\) width \(k\) if it can be constructed according to an \(\NLC_+\) decomposition tree of width \(k\). For any node \(x\) of an \(\NLC_+\) decomposition tree \(T\), we also define \(T[x]\) and \(G[x]\) in the same way as for NLC-decompositions trees. We now show that allowing large arity and imposing connectivity does not ruin the good properties of $\NLC$-decomposition.

\begin{lemma}\label{lem:ldc} 
For every $k\geq 1$, all \(n\)-node connected graphs of $\NLC$-width \(k\) have \(\NLC_+\)-decomposition trees of width at most \(k\cdot 2^{k+1}\), and depth \(\mathcal{O}(\log n)\).
\end{lemma} 


The proof of the  lemma is based on the following statement.

\begin{claim}\label{claim:pourlem:ldc} 
Let $G=(V,E)$ be a connected graph of $\NLC$-width $w\geq 1$. Let $T$ be an $\NLC$-decomposition tree of $G$ of width~$w$ and depth~$d$. Then $G$ admits an \(\NLC_+\)-decomposition tree $T_+$  of width~$w$ and depth at most~$2d$. Moreover, the color of each vertex of $G$ at the root of $T_+$ is the same as the color of this vertex at the root of~$T$.
\end{claim}

\begin{proof}
The proof is by induction on $d$. 
The claim is straightforward for $d=1$ as  $T$ is also an \(\NLC_+\)-decomposition tree in this case. Let $d>1$, and let us assume by induction hypothesis that the claim holds for all connected graphs having an NLC-decomposition tree of depth smaller than $d$. Let $x$ be a node of tree $T$ such that $G[x]$ (corresponding to the decomposition subtree $T[x]$) is connected, but, for at least one of the children $z_1,z_2$ of $x$, $G[z_i]$ is disconnected. Note that if no such $x$ exists, then $T$ is an \(\NLC_+\)-decomposition tree, the connectivity condition being satisfied at each node. We choose $x$ closest to the root, in the sense that no other node from $x$ to the root has this property. (Of course, there might be several such nodes $x$, none of them being ancestor of the other in the tree.)
For each $i \in \{1,2\}$, let $D_i^1, D_i^2,\dots, D_i^{p_i}$ be the connected components of $G[z_i]$. Observe that each $D_i^j$, $1\leq j \leq p_i$ has an $\NLC$-decomposition tree $T_i^j$ of depth at most the depth of $T[z_i]$, obtained by trimming from $T[z_i]$ all leaves that do not correspond to vertices of $D_i^j$. By induction hypothesis, graph $D_i^j$ has an \(\NLC_+\)-decomposition tree $T_+(i,j)$ of width $w$ and of depth at most twice the depth of $T[z_i]$.

We obtain the tree $T_+$ as follows. For node $x$, if $G[z_i]$ has a unique component $D_i^1$, we replace the subtree $T[z_1]$ by $T_+(i,1)$. If $G[z_i]$ has several components $D_i^1, D_i^2,\dots, D_i^{p_i}$, then we replace $T[z_i]$ by a subtree of root $\parallel$, with $p_i$ children, the $j$th child being the root of $T_+(i,j)$. Observe that in this way, the label of $x$ remains unchanged, and $T_+[x]$ is an  \(\NLC_+\)-decomposition tree of $G[x]$. By performing in parallel these operations on all such nodes $x$, we obtain an  \(\NLC_+\)-decomposition tree of $G$. The width of the decomposition has not changed. Moreover the depth has been increased by a factor~2 at most. Indeed, for each node $x$, we replaced $T[z_i]$ by subtrees of depth at most twice the original depth, and the node $x$ itself might have caused the addition of a new layer of children labelled $\parallel$. Therefore, the depth of $T_+[x]$ is at most $2 \cdot \max\{\mbox{depth}(T[z_1]),\mbox{depth}(T[z_2])\}$+2, hence at most twice the depth of $T[x]$. Altogether, the depth of $T_+$ is at most twice the depth of $T$. This concludes the induction step, and the proof of  Claim~\ref{claim:pourlem:ldc}. 
\end{proof}

\begin{proof}[Proof of Lemma~\ref{lem:ldc}.]
Thanks to Lemma~\ref{lem:logdepth}, every $n$-node connected graph $G$ of NLC-width $k$ admits an NLC-decomposition tree of width $k \cdot 2^{k+1}$ and depth $O(\log n)$. By Claim~\ref{claim:pourlem:ldc}, we obtain an \(\NLC_+\)-decomposition tree of the same width $k \cdot 2^{k+1}$, and still of logarithmic depth. 
\end{proof}


Let \(T\) be an $\NLC_+$ -decomposition. Recall that, for every node \(x\) of \(T\), we denote by \(h(x)\) the homomorphism class of \(G[x]\) w.r.t. property \({\Pi}\). Moreover, we set 
\begin{equation}\label{eq:sequence-color}
\textsf{color}(x)=(\textsf{color}^1(x), \dots, \textsf{color}^{k'}(x))
\end{equation}
where, for every $j\in \{1, \dots, k'\}$, $\textsf{color}^j(x)$ is the number of vertices colored \(j\) at node~$x$ after the recoloring operations. Finally, we denote by \(\textsf{exit}(x)\) the identifier of some vertex that belongs to \(G[x]\), called the \textit{exit vertex} of \(G[x]\). When \(x\) is of type \(\parallel\) then \(\textsf{exit}(x)\) is an arbitrary vertex in \(G[x]\).  When \(x\) is of type \(\Join\), we have by Definition~\ref{def:nlc+} that \(G[x]\) is connected. Let \(z^0\) and \(z^1\) be the left and right children of \(x\), respectively. Then, we choose \(\textsf{exit}(x)\) as an arbitrary node belonging to \(G[z^0]\) that is adjacent to some node in \(G[z^1]\).

%We have now all the ingredients necessary to describe our proof-labeling scheme. 

\subsection{Certificate Assignment }  

As for cographs, the certificates are divided in several parts, called \emph{main messages} and \emph{auxiliary messages}. We add a third part, called \emph{service messages}. Let us fix some \(\NLC_+\)-decomposition of width \(k\) and depth \(\mathcal{O}(\log n)\) of the input graph \(G\). 

\subparagraph{Main messages.}  
%
These messages are used to check the local correctness of the decomposition tree. The main message of node \(u \in V\) contains the following information. 

\begin{itemize}

\item The sequence \(\textsf{path}(u) = (x_1(u), \dots, x_{d}(u))\) as defined in Section~\ref{se:cographs} for certifying cographs.

\item The sequence \(\textsf{links}(u) = (\ell_1(u), \dots, \ell_{d}(u)) \in \{0,1,\bot\}^d\), representing the sequence of edges that are to be followed to reach \(x_d(u)\) from \(x_1(u)\), similarly to Section~\ref{se:cographs}, but taking into account the presence of nodes  \(\parallel\) with large arity. More precisely, \(\ell_1(u) = \bot\), and for every $i\geq 2$,
\begin{itemize}
\item if \(x_{i-1}(u)\) is of type \(\parallel\) then \(\ell_i(u) = \bot\);
\item if \(x_{i-1}(u)\) is of type \(\Join\) then \(\ell_i(u)= 0 \) whenever \(x_{i}(u)\) is the left children of \(x_{i}(u)\), and  \(\ell_i(u)= 1\) otherwise.  
\end{itemize}

\item The sequence $\textsf{h}(u) = (h_1(u), \dots, h_d(u))$ defined in Eq.~\eqref{eq:sequence-h}.

\item The sequence $\textsf{color}(u) = (\textsf{color}_1(u), \dots, \textsf{color}_d(u))$, such that, for each \(i \in \{1, \dots, d\}\), \(\textsf{color}_i(u)\) is the sequence $\textsf{color}(x_i(u))=(\textsf{color}^1_i(u), \dots,\textsf{color}^{k'}_i(u))$ defined in Eq.~\eqref{eq:sequence-color}. 


\item The sequence \(\textsf{exit}(u) = (\textsf{exit}_1(u), \dots, \textsf{exit}_d(u))\), where, for each \(i \in \{1, \dots, d\}\), \(\textsf{exit}_i(u)\) is the identifier of node \(\textsf{exit}(x_i(u))\). 
\end{itemize}

\noindent In the following, for every \(i \in [d]\), we denote
 $
 \textsf{main}(u)= (\textsf{main}_1(u), \dots, \textsf{main}_d(u))
 $
 where, for each \(i \in [d]\),
\[ 
\textsf{main}_i(u)  = \Big( x_i(u),\; \ell_i(u),\; h_i(u),\; \textsf{color}_i(u),\; \textsf{exit}_i(u) \Big).
\]

\subparagraph{Auxiliary messages.} 
%
These messages are used to certify the connectivity of the subtrees rooted at nodes \(x\) of type \(\Join\). Let \(z^0\) and \(z^1\) be the children of \(x\) in \(T\). An \emph{auxiliary tree associated to \(x\)}, denoted by~\(A(x)\),  is a spanning tree of \(G[x]\). In the auxiliary messages, we certify the existence of \(A(x)\) using the standard certification for trees~\cite{KormanKP10}. That is, we give each node of the tree the identifier of the root, the identifier of its parent, and its distance to the root.  We also use the auxiliary messages to certify that each vertex created in \(G[z^0]\) (respectively \(G[z^1]\)) received the same information about \(z^0\). Formally, every vertex~\(u\) receives the sequence 
\[
\textsf{aux}(u) = (\textsf{aux}_1(u), \dots, \textsf{aux}_{d}(u))
\]
where, for each \(i \in \{1, \dots, d\}\), \(\textsf{aux}_i(u)=\bot\) whenever \(x_i(u)\) is of type $\parallel$, and, otherwise, 
\[
\textsf{aux}_i(u)= (\textsf{root}_i(u), \textsf{parent}_i(u), \textsf{distance}_i(u), \textsf{childrenMain}_i(u))
\] 
where 
\begin{itemize}
 \setlength\itemsep{0em}
\item \( \textsf{root}_i(u)\) is the identifier of the node \(\textsf{exit}(x_i(u))\);
\item  \( \textsf{parent}_i(u)\) is the identifier of the parent of \(u\)  in \(A(x_i(u))\), where  \(\textsf{parent}_i = \bot\) if $u = \textsf{root}_i(u)$;
\item \( \textsf{distance}_i(u)\) is the distance  between \(u\) and \(\textsf{exit}(x_i(u))\)  in \(A(x_i(u))\);
%\item \(\textsf{terminals}_i(u)\) is a pair of vertex identifiers representing \(\textsf{exit}(z_0), \textsf{exit}(z_1)\). 
%\item \(\textsf{terminalClasses}_i(u)\) is a pair of  classes for \(\equiv_{\Pi}\) representing the pair \((h(z_0), h(z_1))\).
%\item  $\textsf{terminalColors}_i(u)$ represents the pair \((\textsf{color}(z_0), \textsf{color}(z_1))\)
\item \(\textsf{childrenMain}^j_i(u)=(z^j, h(z^j), \textsf{color}(z^j), \textsf{exit}(z^j))\) for \(j \in \{0,1\}\). %\ioan{Est-ce qu'on ne devrait pas garder aussi l'indice $i$ dans $z^j$?}
%\item \(\textsf{boolPair}_i(u)\) is a pair of Booleans \((\textsf{bool}_i^0(u), \textsf{bool}_i^1(u))\)  where \(\textsf{bool}_i^0(u)\) (respectively \(\textsf{bool}_i^1(u)\)) is true and only if \(\textsf{ex}_i^0(u)\) (respectively \(\textsf{ex}_i^1(u)\)) appears in the subtree of \(A(x_i(u))\) rooted at \(u\).
\end{itemize}

\subparagraph{Service messages.}  These messages are used to check consistency in the subgraphs induced by the subtrees rooted at nodes of type \(\parallel\). In other words, they are used to handle the case of nodes in the tree~$T$ constructing non-connected subgraphs. Before explaining these messages, let us define some additional data structures. 

Let \(x\) be a node of type \(\parallel\), and let $y$ denote the parent of $x$ in $T$. Again by Lemma~\ref{lem:ldc}, for each child $z_i$, $1 \leq i \leq p$, of $x$,  the graph $G[z_i]$ corresponds to a connected component of the graph $G[x]$ (which is disconnected by construction). A \emph{service tree associated to} \(x\), denoted by \(S(x)\), 
is a Steiner tree in \(G[y]\) with terminals \(\{\textsf{exit}(z_1), \dots, \textsf{exit}(z_p)\}\), and root \(\textsf{exit}(x)\). The service messages are used to certify the existence of \(S(x)\) in a  way similar to the auxiliary tree.  We also use the service messages  to certify the consistency between \(h(z_1), \dots, h(z_p)\) and \(h(x)\).  This latter certification is slightly more complicated than for auxiliary trees because a parallel node may have an arbitrarily large number of children, and one cannot store  the identifiers, the classes and the colors of all the terminals if one want to keep the certificate size small.


Let \(u\) be a vertex in \(S(x)\). The service message of \(u\) contains the root of \(S(x)\), the parent of \(u\), and the depth of \(u\) in \(S(x)\). Let us denote by \(S(x,u)\) the subtree of \(S(x)\) rooted at~\(u\).  Furthermore, let us call \(\textsf{inCharge}_x(u)\) the set of the indices of the terminals of \(S(x)\) contained in \(S(x,u)\). Formally,
\[
\textsf{inCharge}_x(u) = \{i \in \{1, \dots, p\} \mid \textsf{exit}(z_i) \in S(x,u) \} 
\]
Now let us denote by \( G[ \textsf{inCharge}_x(u) ] \) the subgraph of \(G\) induced by the disjoint union of all graphs \(G[z_i]\), \(i \in \textsf{inCharge}_x(u)\). The service message of \(u\) includes the homomorphism class \(h(G[ \textsf{inCharge}_x(u) ])\). The correctness of this homomorphism class will be verified using only  the homomorphism classes of the children of \(u\) in \(S(x)\).  Observe that \(S(x, \textsf{exit}(x)) = S(x)\), and \[h(G[ \textsf{inCharge}_x(\textsf{exit}(x))] = h(x).\]

Observe that node  \(u\) is necessarily contained in \(G[y]\), but not necessarily in \(G[x]\). Therefore, it is possible that \(x\) does not appear in the main message of \(u\). Therefore, for each \(j \in \{0,1\}\), vertex \(u\) also receives the sequence
\[
\textsf{service}^j(u) = (\textsf{service}^j_1(u), \dots, \textsf{service}^j_{d}(u)), 
\]
where, for each \(i \in [d]\), the value of \(\textsf{service}^j_i(u)\) represents the part of the certification of the service tree \(S(x)\), where \(x\) is the left child (respectively the right child) of \(x_{i-1}(u)\). 
Specifically, if \(x\) is not of type \(\parallel\), or if \(u\) does not participate in \(S(x)\), then \(\textsf{service}^j_i(u)=\bot\). Otherwise, 
\[
\textsf{service}^j_i(u) = (\textsf{root}^j_i(u), \textsf{parent}^j_i(u), \textsf{distance}^j_i(u), \textsf{class}^j_i(u), \textsf{colorCharge}^j(u))
\]
where
\begin{itemize}
 \setlength\itemsep{0em}
\item \( \textsf{root}^j_i(u)\) is the identifier of the root of \(S(x)\); 
\item  \( \textsf{parent}^j_i(u)\) is the identifier of the parent of \(u\)  in \(S(x)\), with \(\textsf{parent}^j_i (u)= \bot\) if \(u = \textsf{root}^j_i(u)\),;
\item \( \textsf{distance}^j_i(u)\) is the distance between \(u\) and the root \(\textsf{exit}(x)\)  in \(S(x)\);
\item \(\textsf{class}^j_i(u)\) is the homomorphism class of \(G[\textsf{inCharge}_{x}(u)]\);
\item  $\textsf{colorCharge}_i^j(u)=(\textsf{colorCharge}_i^{j,1}(u), \dots, \textsf{colorCharge}_i^{j,k'}(u))$ where $\textsf{colorCharge}_i^{j,s}(u)$ is the number of vertices colored $s$ in $G[\textsf{inCharge}_{x}(u)]$, for every $s \in [k']$.
\end{itemize}


\subsection{Verification Scheme} 
%
The main role of the verification procedure is to check that  the \(\NLC_+\)-decomposition tree is correct, that is (1)~it corresponds to the graph $G$, and (2)~at each node $x$ of the decomposition tree, the homomorphism class for property $\Pi$ provided by the prover corresponds to $h(G[x])$.

Each vertex \(u \in V\) checks that all its messages in its certificate are correctly formatted.  Then, the verification is split in three parts: (1)~verification of the main messages, (2)~verification of the auxiliary messages, and (3)~verification of the service messages. 
 
\subparagraph{Verification of main messages. }

Each vertex \(u \in V\) verifies the following conditions:

\begin{enumerate}

\item \label{verif:1} \(x_{d}(u)\) is of type $\mathsf{newVertex}_s$ for some \(s \in [k']\);
\item  \label{verif:2} $\textsf{exit}_d(u)=\id(u)$;
\item  \label{verif:3} \(\textsf{color}^j_d(u) = 0\) for every \(j\in [k']\setminus\{s\}\), and \(\textsf{color}^s_d(u) = 1\);
 \item  \label{verif:4} \(h_d(u)\) is the homomorphism class for \({\Pi}\) of a graph equal to a single node labelled $\ell(u)$, colored $s$;
 \item  \label{verif:5} \(x_1(u)\) is of type~\(\Join\), and for each \(i\in \{2, \dots, d(u)\}\), if \(x_i\) is of type \(\parallel\) then \(x_{i-1}(u)\) is of type \(\Join\);
 \item \label{verif:5.1} \(h_1(u)\) is an \emph{accepting} homomorphism class for \({\Pi}\), i.e., graphs of \(\NLC_{+}\)-width \(k'\) having this class satisfy~\(\Pi\).
\end{enumerate}
Let us define \(\mathsf{currentcolor}(u,i)\) in the same way that we did in the verification of cographs. Moreover, for a vertex \(v\in V\), let us denote \(i^*=\textsf{index}(u,v)\) as the maximum index \(i\in [d(u)]\) such that \(\ell_j(u) = \ell_j(v)\) for every \(j \leq i\). Then, \(u\) computes \(\textsf{index}(u,v)\) for each of its neighbors \(v\in N(v)\), and checks:


\begin{enumerate}
\setcounter{enumi}{6}
\item \label{verif:6} \(\textsf{main}_i(u) = \textsf{main}_i(v)\) for every \(i \leq i^*\).
%All the sequences given in the main messages to \(u\) have a common prefix with the ones given to \(v\). Formally, \(u\) verifies that for every \(i \leq \textsf{index}(u,v)\), \(x_i(u) = x_i(v)\),  \(\textsf{children}_i(u) = \textsf{children}_i(v)\), \(h_i(u) = h_i(v)\), \(\textsf{deg}_i(u) = \textsf{deg}_i(v)\), \(\textsf{exit}_i(u) = \textsf{exit}_i(v)\) and also for every \(j\in [k']\),  \(\textsf{color}^j_i(u) = \textsf{color}^j_i(v)\). 

  
\item \label{verif:7} \(x_{i^*}(u)\) is of type \(\Join\), and 
\begin{itemize}
\item   if \(\ell_{i^*+1}(u) = 0\) then \(u\) checks that \(x_{i^*}\) contains the join operation $\Join_S$ with \[(\mathsf{currentcolor}(u,i^*), \mathsf{currentcolor}(v,i^*)) \in S\]
\item   if \(\ell_{i^*+1}(u) = 1\) then \(u\) checks that  \(x_{i^*}\) contains the join operation $\Join_S$ with \[(\mathsf{currentcolor}(v,i^*),\mathsf{currentcolor}(u,i^*)) \in S.\]
\end{itemize}

\item \label{verif:11} For each \(j \in [k']\), vertex \(u\) checks that it has exactly \(\textsf{color}^j_{i^*+1}(v)\) neighbors \(w\) such that \(\textsf{index}(u,w) = i^*\), \(\ell_{i^*+1}(w) \neq \ell_{i^*+1}(u)\), and \(\mathsf{currentcolor}(w,i^*+1) = j\);

%\item \label{verif:8} Vertex $u$ checks that its degree $\textsf{deg}_{i^*}(u)$ at node $x_{i^*}$ corresponds to its degree $deg_{i^*+1}(u)$, plus the cardinality of the set \(\{v \in N(u) \mid \textsf{index}(u,v) = i^*\}\).  Vertex \(u\) also verifies that \(\textsf{deg}_1(u)\) corresponds to its degree in \(G\).

\item \label{verif:9} Vertex $u$ then considers the homomorphism classes $h_{i^*}(u)$, $h_{i^*+1}(u)$ and $h_{i^*+1}(v)$. (Recall that these encode the homomorphism classes of relation ${\Pi}$ for graphs $G[x_{i^*}(u)]$, $G[x_{i^*+1}(u)]$ and $G[x_{i^*+1}(v)]$, respectively.) 
\begin{itemize}
\item If \(\ell_{i*+1} = 0\) then $x_{i^*+1}(u)$ must be the left child of $x_{i^*}(u)$, and vertex  $u$ checks that \[h_{i^*}(u) = \odot_{\mathsf{recolor}_R} (\odot_{\Join_S}(h_{i^*+1}(u), h_{i^*+1}(v))).\] 
\item If  \(\ell_{i*+1} = 1\) then $x_{i^*+1}(u)$ must be the right child of $x_{i^*}(u)$, and vertex  $u$ checks  that \[h_{i^*}(u) = \odot_{\mathsf{recolor}_R} (\odot_{\Join_S}(h_{i^*+1}(v)),h_{i^*+1}(u)).\] 
\end{itemize}


\item \label{verif:10}  For each \(j\in [k']\), \[\textsf{color}^j_{i^*}(u) = \sum_{s \in R^{-1}(j)}  \left(\textsf{color}^{s}_{i^*+1}(u) + \textsf{color}^{s}_{i^*+1}(v)\right).\]


\item  \label{verif:12} Finally, for each \(i \in [d(u)]\) such that \(x_{i}(u)\) is of type \(\Join\), if \(\textsf{exit}_i(u)=\id(u)\), then \(u\) checks that \(\ell_{i+1}(u) = 0\), and that there exists \(v\in N(u)\) such that \(\textsf{index}(u,v) = i\) and \(\ell_{i+1}(v) = 1\).
\end{enumerate}

\subparagraph{Verification of auxiliary messages.}  

For every \(i \in [d(u)]\) vertex \(u\) checks that \(\textsf{aux}_i \neq \bot \) if and only if \(x_i(u)\) is of type \(\Join\). Let us suppose now that \(\textsf{aux}_i \neq \bot\). Then \(u\) checks the following conditions

\begin{enumerate}
\setcounter{enumi}{12}
\item \label{verif:13} \(\textsf{root}_i(u) = \textsf{exit}_i(u)\);
\item  \label{verif:14} if \( \textsf{parent}_i(u) = \perp\) then \(\textsf{root}_i(u)=\id(u)\);
 \item  \label{verif:15} if \(v = \textsf{parent}_i(u) \neq \bot\) then \(v \in N(u)\), \(\textsf{index}(u,v) \geq i\), \(\textsf{aux}_i(v) \neq \perp\), \(\textsf{root}_i(v) = \textsf{root}_i(u)\), \(\textsf{childrenMain}^0_i(u) = \textsf{childrenMain}^0_i(v) \),  \(\textsf{childrenMain}^1_i(u) = \textsf{childrenMain}^1_i(v) \), and \[\textsf{distance}_i(v) = \textsf{distance}_i(u) -1;\]
\item  \label{verif:16} if \(j = \ell_{i+1}(u) \) then \(\textsf{childrenMain}^j_i(u) = \textsf{main}_{i+1}(u)\).  


%For each \(j \in \{0,1\}\), if \(u\) is the identifier of \(\textsf{ex}_i^j(u)\), then \(\textsf{bool}_i^j(u) = \texttt{true}\). If \(u \neq \textsf{ex}_i^j(u)\), we define \(\textsf{children}_i(u) = \{w \in V: \textsf{parent}_i(w) = u\}\).  If \(\textsf{children}_i(u) = \emptyset\) then \( \textsf{bool}_i^j = \texttt{false}\). If \(\textsf{children}_i(u) \neq \emptyset\) then,
 %\[\textsf{bool}(u)_i^j=\bigvee_{w \in \textsf{children}_i(u)} \textsf{bool}_i^j(w) \].
\end{enumerate}


\subparagraph{Verification of service messages.}   

For every \(i \in [d(u)]\) and \(j \in \{0,1\}\) for which \(\textsf{service}^j_i(u)\neq \perp\), \(u\) checks the following conditions:


\begin{enumerate}
\setcounter{enumi}{16}
\item \label{verif:17}  If \(\ell_i(u) = j\) then \(x_i(u)\) is of type \(\parallel\).

\item  \label{verif:18} If \(v = \textsf{parent}^j_i(u) \neq \perp\) then \(v \in N(u)\), \(\textsf{index}(u,v) \geq i-1\), \(\textsf{service}^j_i(v) \neq \perp\), \(\textsf{root}^j_i(v) = \textsf{root}^j_i(u)\) and \(\textsf{distance}^j_i(v) = \textsf{distance}_i^j(u) -1\).
\item  \label{verif:19} If \( \textsf{parent}^j_i(u) = \perp\) then \(\textsf{root}^j_i(u)=\id(u)\).
%\item \label{verif:14} If \(j = \ell_i(u)\), then 
%\(\displaystyle{\textsf{index}(u,v) \leq \left\{ \begin{array}{cl} i  & \text{if } x_{i+1}(u) \text{ is of type} \parallel, \text{ and}\\  i+1 & \textsf{if } x_{i+1}(u) \text{ is of type} \Join. \end{array}\right.}\)\\
%If \(j \neq \ell_i(u)\), then 
%\(\displaystyle{\textsf{index}(u,v) \leq \left\{ \begin{array}{cl} i  & \text{if } x_{i+1}(v) \text{ is of type} \parallel, \text{ and}\\  i+1 & \textsf{if } x_{i+1}(v) \text{ is of type} \Join. \end{array}\right.}\)

\end{enumerate}
If \(u\) has no neighbors \(w \in V\) such that \(\textsf{parent}^j_i(w) = u\), then \(u\) deduces that it is a leaf of the tree \(S(x_i^j)\). In that case, \(u\) checks the following conditions:
\begin{enumerate}
\setcounter{enumi}{19}

\item \label{verif:20}  \(u\) is a terminal, so \(\textsf{exit}_{i+1}(u)=\id(u)\);%\pierre{It's a claim, or something to be checked?}
\item \label{verif:21} \(\textsf{class}_i^j(u) = h_{i+1}(u)\) and \(\textsf{colorCharge}_i^j(u) = \textsf{color}_{i+1}(u)\).%, and \(\textsf{numCharge}_i^j(u) = 1\).

\end{enumerate}
If $u$ is not a leaf of \(S(x_i^j)\), then \(u\) computes the set \(\textsf{children}_i^j(u)\) of nodes  \(w\in V\) such that \({\textsf{parent}_i^j(w)=u}\). Then \(u\) checks the following conditions:

\begin{enumerate}
\setcounter{enumi}{21}
%\item \label{verif:22} Let \(\delta = 1\) if \(u = \textsf{exit}_{i+1}(u)\) and \(\ell_i(u) = j\). Otherwise, \(\delta = 0\). Then \(u\) checks that: \[\displaystyle{\textsf{numCharge}_i^j(u)  = \sum_{w \in \textsf{children}_i^j(u)} \textsf{numCharge}_i^j(w) + \delta.}\]
\item \label{verif:23} For each \(s\in [k']\), let \(\gamma = \textsf{color}_{i+1}^s(u)\) if \(u = \textsf{exit}_{i+1}(u)\) and \(\ell_i(u) = j\), and let  \(\gamma = 0\) otherwise.  Then \(u\) checks that: 

\[\textsf{colorCharge}^{j,s}(u)  = \sum_{w \in \textsf{children}_i^j(u)} \textsf{colorCharge}^{j,s}(w) + \gamma\]



\item \label{verif:24} If \(\textsf{children}_i^j(u)\) contains a single vertex \(w\), then \(u\) checks that \(\textsf{class}_i^j(u) = \textsf{class}_i^j(w)\).  
\item \label{verif:25} If \(\textsf{children}_i^j(u)\) contains two or more vertices, then \(u\) defines an arbitrary order of the vertices in \(\textsf{children}_i^j(u)\), namely \(w_1, \dots, w_p\). Then, \(u\) defines a sequence of homomorphism classes \(c_1, \dots, c_{p}\) where, \(c_1 = \textsf{class}_i^j(w_1)\), and for each \(i \in \{2, \dots, p\}\):
\[c_i = \odot_{\parallel} (c_{i-1},  \textsf{class}_i^j(w_{i})).\]
where \(\odot_{\parallel}\) corresponds to the function \(\odot_{\Join_{\emptyset}}\). Finally checks that \(c_p = \textsf{class}_i^j(u) \).
\end{enumerate}
 If \(u = \textsf{exit}_{i}(u)\), then \(u\) checks the following additional conditions for each \(j \in \{0,1\}\):
\begin{enumerate}
\setcounter{enumi}{24}
\item \label{verif:26} \(\textsf{parent}_i^j(u) = \bot\), and  \(\textsf{distance}_i^j(u) = 0\);
\item \label{verif:27} \(\textsf{class}_i^j(u) = h_{i}(u)\), and \(\textsf{colorCharge}_i^j(u) = \textsf{color}_{i}(u)\).%, and \(\textsf{numCharge}_i^j(u) = \textsf{children}_{i+1}(u)\).
\end{enumerate}

%%%%%%%%%%%%%%%%%%%%%%%%%%%%%%%%%%%%%%%%%%%


%
\subsection{Completeness and Soundness} 
%
The completeness follows directly from the existence of an $\NLC_+$-decomposition $T$ of $G$ as described in Lemma~\ref{lem:ldc} and from the construction of the main, auxiliary and service messages, based on this decomposition. 

For soundness, assume that the verification protocol accepts at every vertex.  Let us define the set of all main messages as \(\textsf{mainset} = \{\textsf{main}(u) : u \in V\}\), and let \(d^* = \max_{u\in V} d(u)\). For every \(M \in \textsf{mainset}\) we denote \(d(M) = d(u)\) where \(u\) is such that \(\textsf{main}(u) = M\). Now, we define  \(\textsf{Prefix}(M,i)\) as the set of vertices \(v\in V\) that have the same first \(i\) values in their main messages. Formally,  
\[
\textsf{Prefix}(M,i) = \{v \in V: M_j = \textsf{main}_j(v),  \textrm{ for every } 1\leq  j \leq  i\}.
\] 
By condition~\ref{verif:2} in the verification protool, \(\textsf{exit}_{d}(u)=\id(u)\). Thus \(\textsf{main}(u) \neq \textsf{main}(v)\) for every pair of vertices \(u \neq v\). Therefore, \(\textsf{Prefix}(\textsf{main}(u),d(u)) = \{u\}\). Also, by condition~\ref{verif:6} we have that \(\textsf{Prefix}(M,1) = V\). Given \(M \in \textsf{mainset}\),  and \(i \in [d(M)]\), let 
\[
M_i = (x_i(M), \ell_i(M), h_i(M), \textsf{color}_i(M), \textsf{exit}_i(M))
\] 
be the list \(\textsf{main}_i(v)\) for a vertex \(v\) such that \(\textsf{main}(v) = M\).


\begin{lemma}\label{lem:completeness} 
For every \(M \in \textsf{mainset}\), and every \(i \in [d(M)]\), there exists a \(\NLC_+\)-decomposition tree \(T[M,i]\) of \(G[M,i] = G[\textsf{Prefix}(M,i)]\)  such that, for all \(u \in \textsf{Prefix}(M,i)\), the following holds. 
\begin{itemize}
\item For every $j$ with \(i \leq j \leq d(u)\), \(x_j(u)\) contains the operations in the \(j\)-th node in path from the root of \(T[M,i]\) to the node where \(u\) is created;
\item \(h_i(u) = h(G[M,i])\);
\item  For every \(s\in [k']\), \(\textsf{color}^s_i(u) \) is the number of vertices colored \(s\) in the root of \(T[M,i])\).
\end{itemize}
\end{lemma}

\begin{proof}
The proof is by induction on \(i\), in decreasing order for each \(M\). Let us fix \(M \in \textsf{mainset}\). The base case is \(i = d(M)\), and \(\textsf{Prefix}(M,d(M))\) is a single vertex \(u\) satisfying \(\textsf{main}(u) = M\).  In this case the lemma is true by conditions~\ref{verif:1}-\ref{verif:4}. For the inductive case, let us suppose that there exists \(t >1 \)  such that the lemma is true for every \(M \in \textsf{mainset}\), and \(i \in \{t, \dots, d(M)\}\), and let us show that the lemma holds an arbitrary  pair \(M \in \textsf{mainset}\) and \(i = t-1\). 

\medskip

\noindent -- Let us suppose first that \(x_{i}(M)\) is of type \(\Join\). Consider the set
\[
\beta_i(M) =  \{(\textsf{root}_i(u), \textsf{parent}_i(u),  \textsf{distance}_i(u)) \mid u \in \textsf{Prefix}(M,i)\}.
\] 
From conditions~\ref{verif:13}-\ref{verif:15}, we have that \(\beta_i(M)\) certifies a spanning tree of \(\textsf{Prefix}(M,i)\) rooted at \(\textsf{exit}_i(M)\). In particular, we have that \(\textsf{Prefix}(M,i)\) induces a connected subgraph of \(G\). Let us call \(u^0\) the vertex with identifier \(\textsf{exit}_i(M)\). By condition~\ref{verif:12}, we have \(\ell_{i+1}(u^0) = 0\), and there exists a vertex \(u^1 \in N(u)\) such that \(\textsf{index}(u,v) = i\) and \(\ell_{i+1}(u^1) = 1\). Let us call \(M^0= \textsf{main}(u^0)\) and \(M^1 = \textsf{main}(v^1)\). Our candidate for \(T[M,i]\) is the tree defined by a root \(x_i(M)\) with two children. The left children induces the subtree \(T[M^0, i+1]\) while the right children induces the subtree \(T[M^1,{i+1}]\).

Observe first that \(\textsf{Prefix}(M, i)\) can be partitioned in \(\textsf{Prefix}(M^0, i+1)\) and  \(\textsf{Prefix}(M^1, i+1)\). Indeed, let \(w\) be a vertex in \(\textsf{Prefix}(M, i)\), and let \(j =\ell_{i+1}(w) \). By condition~\ref{verif:16},   we have that \(\textsf{childrenMain}^j_i(w)=\textsf{main}_{i+1}(w)\). By condition~\ref{verif:15}, and the fact that \(\textsf{Prefix}(M, i)\) is connected, we have that \(\textsf{main}_{i+1}(w) = \textsf{main}_{i+1}(w')\) for every \(w' \in \textsf{Prefix}(M, i)\) for which \(\ell_{i+1}(w') = j\). In particular \(\textsf{main}_{i+1}(w) = \textsf{main}_{i+1}(u^j)\), and therefore \(w\)  belongs to \(\textsf{Prefix}(M^j, i+1)\).


By the induction hypothesis, for each \(j\in \{0,1\}\), \(T[M^j, i+1]\) is an \(\NLC_+\) decomposition tree of \(G[M^j, i+1]\). Moreover, \(h_{i+1}(M^j) = h(G[M^j, i+1])\), and, for each \(s\in [k']\), we have that \(\textsf{color}^s_{i+1}(M^j)\) is the number of vertices colored \(s\) in the root of \(T[M^j, i+1]\).
Let \(S\) be the set of join operations defined in \(x_i(M)\), and let \(u \in \textsf{Prefix}(M^0,i+1)\) and \(v \in \textsf{Prefix}(M^1,i+1)\) be such that \(\{u,v\} \in E(G[\textsf{Prefix}(M,i)])\). By condition~\ref{verif:7}, we have that 
\[
(\textsf{currentcolor}(u,i), \textsf{currentcolor}(v,i)) \in S.
\] 
Moreover, by condition~\ref{verif:11}, vertex \(u\) has exactly \(\textsf{color}_{i+1}^s(M^1)\) neighbors  \(v \in \textsf{Prefix}(M^1,i+1)\) such that \(\textsf{currentcolor}(v,i) = s\). This implies that  the join operations defined in \(x_i(M)\) create exactly the set of edges \(\{u,v\} \in E(G[M,i])\) such that \(\textsf{index}(u,v) = i\), and no other edges. It follows that \(T[M,i]\) is an \(\NLC_+\)-decomposition tree of \(G[M,i]\). Finally, by condition~\ref{verif:9} applied at \(u^0\), we have that \(h_i(M) = h(G[M,i])\), and, by condition~\ref{verif:10} applied at \(u^0\), we have that, for every \(s\in [k']\), \(\textsf{color}_i^s(M)\) is the number of vertices colored \(s\) in the root of \(T[M,i]\).

\medskip 

\noindent -- Let us suppose now  that \(x_{i}(M)\) is of type \(\parallel\), and let us define the following subset of \textsf{mainset}:
\[\textsf{mainPrefix}(M,i) = \{ M' \in \textsf{mainset} \mid \textsf{Prefix}(M,i) = \textsf{Prefix}(M',i) \}\]
Observe that from the induction hypothesis, we have that, for each \(M' \in \textsf{mainPrefix}(M,i)\), there is an \(\NLC_{+}\)-decomposition tree  \(T[M',i+1]\) of \(G[M',i+1]\), satisfying the conditions of the lemma.
Our candidate for \(T[M,i]\) is a tree rooted in  \(x_{i}(M)\), where the children of \(x_{i}(M)\) induce the set of trees 
\[
\{ T[M', i+1]: M' \in \textsf{mainPrefix}(M,i) \}.
\]
Indeed,  every vertex \(u\) in \(\textsf{Prefix}(M,i)\) satisfies that \(x_i(u) = x_i(M)\). Then, by condition~\ref{verif:7},  \(u\) cannot have a neighbor \(v\) such that \(\textsf{index}(u,v) = i\). Therefore \(T[M,i]\) does define an \(\NLC_{+}\) decomposition of  \(G[M,i]\). Let \(j = \ell_{i}(M)\), and let us consider the set
\[
\alpha_i^j(M) =  \{(\textsf{root}^j_i(u), \textsf{parent}^j_i(u),  \textsf{distance}^j_i(u)) \mid u \in \textsf{Prefix}(M,i-1) \; \text{and} \; \textsf{service}_i^j(u) \neq \bot\}.
\] 
Let \(u\) be a vertex such that \(\textsf{main}(u) \in \textsf{mainPrefix}(M,i)\). From condition~\ref{verif:5},  \(x_{i+1}(u)\) is of type \(\Join\). Hence, \(\textsf{Prefix}(\textsf{main}(u), i+1)\) induces a connected set of vertices in \(G\). Therefore, by conditions~\ref{verif:17}-\ref{verif:20} and~\ref{verif:26},  \(\alpha_i^j(M)\) certifies a Steiner tree in \(G\) rooted at \(\textsf{exit}_{i}(M)\), with set of terminals
\[
\textsf{terminals}_i(M) =  \{u \in V \mid  \exists  M' \in \textsf{mainPrefix}(M,i), \id(u)= \textsf{exit}_{i+1}(M')\}. 
\]  
From conditions~\ref{verif:21} and \ref{verif:24}-\ref{verif:27},   \(h_i(M)\)  is the homomorphism class of \({\Pi}\) obtained from the disjoint union function  \({\odot}_{\parallel}\) over the set  
\[
\{h_{i+1}(M'): M' \in \textsf{mainPrefix}(M,i) \}.
\] 
By the induction hypothesis, it follows that \(h_i(M)\) is the homomorphism class of \(G[M',i+1]\).  From conditions~\ref{verif:21}, \ref{verif:23},  and~\ref{verif:27}, we have that, for each \(s \in [k']\),
\[
\textsf{color}^s_{i}(M) = \sum_{u~ \in ~\textsf{terminals}_i(M)} \textsf{color}^{s}_{i+1}(u).
\]
Again, by the induction hypothesis, \(\textsf{color}_i^s(M)\)  is the number of vertices colored \(s\) in the root of~\(T[M,i]\).
\end{proof}

Finally, observe that by condition \(\ref{verif:6}\), for every $M$ and $M' \in \textsf{mainset}$, we have \(M_1 = M'_1\). Then, \(\textsf{Prefix}(M,1) = V\). Applying Lemma~\ref{lem:completeness} to an arbitrary \(M \in \textsf{mainset}\) and for \(i = 1\), we deduce that  \(T[M,1]\) is an \(\NLC_+\)-decomposition tree of \(G\). It follows that \(G\) is a graph of \(\NLC_+\)-width~\(k'\). Moreover, also by Lemma~\ref{lem:completeness}, we have \(h_1(M) = h(G)\).   Finally, by condition~\ref{verif:5.1} we conclude that \(G\) satisfies property \(\Pi\). \hfill\qed


%
\subsection{Certificate Size} 
%
For the  certificate size, let \(u\) an arbitrary vertex, let \(i \in d(u)\), and let \(j \in \{0,1\}\). Observe that each item of \(\textsf{main}_i(u)\), \(\textsf{aux}_i(u)\), and \(\textsf{service}^j_i(u)\) can be encoded with \(\mathcal{O}(\log n)\) bits.  By Lemma~\ref{lem:ldc}, we have that \(d(u)= \mathcal{O}(\log n)\). Therefore, the certificate of \(u\) is of size \(\mathcal{O}(\log^2 n)\). 







\section{Context Optimization}
\label{appendix:optimization}


\begin{table}%
\centering
\begin{subtable}{0.75\textwidth}
\resizebox{\columnwidth}{!}{
\begin{tabular}{ c|c|c|c|c }

 \multicolumn{1}{c}{} & \multicolumn{2}{c}{\underline{Unknown Drug}} & \multicolumn{2}{c}{\underline{Unknown Cell Line}} \\ 
 \multicolumn{1}{c}{Strategy} & \multicolumn{1}{c}{ROC-AUC} & PR-AUC & ROC-AUC & \multicolumn{1}{c}{PR-AUC} \\
\thickhline
 Typical Unknown-First & 79.2 & 63.8 & 85.2 & 74.9 \\
 \hline
 Best Unknown-First & 80.8 & 66.4 & 85.6 & 75.7 \\
 \hline
 Error Reduction & 75.4 & 59.0 & 84.9 & 74.5 \\
 \hline
 Genetic Algorithm & \textbf{81.5} & \textbf{66.9} & \textbf{86.1} & \textbf{76.5} \\

\end{tabular}}
\end{subtable}
\caption{Test-set performance of different context optimization methods applied to the Unknown-First strategy. Note that the same model parameters are used in all cases and only the input context is changed. Considering Unknown-First as the distribution being sampled from, the genetic algorithm solution has a z-score of 4.02 indicating $p < 0.0001$.}
\label{tab:full_optimization}
\end{table} 

\subsection{Genetic Algorithm}
In the case of the genetic algorithm, each context bank synergy tuple $x \in \mathfrak{D}^c$ is considered as a gene which can be selected by the algorithm. Given $p$ ``unknown'' drugs or cell lines, each has $n$ slots for context examples in its prompt, which makes for $np$ total genes. We also enforce that each $x$ contains the relevant unknown drug $d^h$ or cell line $c^h$. We disallow each context example from being selected multiple times; the reasons for this is two-fold. First, in early experiments we found that if we use the same example for the entire context (e.g. 20 repeats of $x$), then the model performs poorly. This is likely because the model is not trained on duplicate input, so it is trying to make meaningless connections between the same $x$. Second, repeating $x$ in the context provides no new information to the model. Although we enforce this constraint, in practice without it the model will likely do the same thing on its own. 

For the genetic algorithm in context optimization, we use a population of 8 for 50 epochs. We use steady-state parent selection with 4 parents, single-point cross-over, 10\% gene mutation, and elitism. Each example in the context bank is considered a gene and we disallow repeated genes. This results in 351 evaluations on the validation set. 

\subsection{Error Reduction for Context Optimization} \label{appendix:error_reduction}
Using the Unknown-First strategy, we sample a context for some heldout tuple in the validation set. We then calculate the absolute error $\epsilon$ for the heldout tuple. For each context example $x^c$ in the heldout tuple's input context $P^n$, we store $\epsilon$ and the relevant heldout entity, $h$. After some number (for fairness we use the same number of times as the genetic algorithm evaluates ROC-AUC on the validation set--351) of epochs on the validation set, we calculate a mean error $\hat{\epsilon}_{h}(x^c)$ for that context example $x^c$. Finally, for each heldout drug or cell $h$, we select the $n$ context examples $x^c_i$ with the lowest $\hat{\epsilon}_{h}(x^c_i)$. 

As shown in Table \ref{tab:full_optimization}, this strategy produces poor performance. This indicates that simply selecting all of the most individually informative context examples is not useful. Rather, there is a more complex, non-linear interaction between examples which is informative to the model. This is intuitive, because the interaction between cellular pathways in complex and still not well understood. The ability for the context to be optimized by a genetic algorithm but not error reduction indicates that data collection strategies which emphasize diversity may be important to consider for constructing new drug synergy datasets. 


\section{Conclusion and Future Work}
In this work, I design corruption-robust algorithms for the Lipschitz contextual search problem. I present the \emph{agnostic checking} technique and demonstrate its effectiveness in designing corruption-robust algorithms. There are several open problems for future research. First, in the algorithm I propose for pricing loss, the schedule for agnostic checks is fixed upfront. Can the learner design an adaptive checking schedule for the pricing loss? Second, this work assumes the learner has knowledge of the Lipschitz constant $L$. Can the learner design efficient no-regret algorithms without knowledge of $L$? 

%\newpage

%%
%% The next two lines define the bibliography style to be used, and
%% the bibliography file.
%\bibstyle{plainurl}
\bibliography{biblio-PLS-CW-MSO}

%%
%% If your work has an appendix, this is the place to put it.




\end{document}
\endinput
%%
%% End of file `sample-acmsmall-conf.tex'.
