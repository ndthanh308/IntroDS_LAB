\documentclass[a4paper,10pt]{amsart}

\usepackage[latin1]{inputenc}
\usepackage[T1]{fontenc}
\usepackage[english]{babel}

\usepackage{amsmath,amssymb,amsthm,enumerate,xspace}

\usepackage{mathrsfs}
\usepackage{amscd}
\usepackage{amsfonts}
\usepackage{amsmath}
\usepackage{amssymb}
\usepackage{amstext}
\usepackage{amsthm}
\usepackage{amsbsy}

%\usepackage[notcite, notref]{showkeys} %for displaying labels, citations, references...

\usepackage{xspace}
\usepackage[all]{xy}
\usepackage{graphicx}
\usepackage{url}
\usepackage{latexsym}

\usepackage{hyperref}

\usepackage{cleveref}
\usepackage{tikz-cd}

%================================= 
\newtheorem{theo}{Theorem}[section]
\newtheorem{lem}[theo]{Lemma}
\newtheorem{prop}[theo]{Proposition}
\newtheorem{cor}[theo]{Corollary}
\newtheorem{conj}[theo]{Conjecture}
\newtheorem{que}[theo]{Question}

%===== for intro =====
\newtheorem{thmintro}{Theorem}
\renewcommand{\thethmintro}{\Alph{thmintro}}
\newtheorem{conjintro}[thmintro]{Conjecture}
\newtheorem{corintro}[thmintro]{Corollary}
\newtheorem{propintro}[thmintro]{Proposition}



\newtheorem{innercustomgeneric}{\customgenericname}
\providecommand{\customgenericname}{}
\newcommand{\newcustomtheorem}[2]{%
  \newenvironment{#1}[1]
  {%
   \renewcommand\customgenericname{#2}%
   \renewcommand\theinnercustomgeneric{##1}%
   \innercustomgeneric
  }
  {\endinnercustomgeneric}
}

\newcustomtheorem{customthm}{Theorem}
\newcustomtheorem{customprop}{Proposition}
\newcustomtheorem{customlemma}{Lemma}
\newcustomtheorem{customcorollary}{Corollary}


 \theoremstyle{definition}
\newtheorem{dfn}[theo]{Definition}

\newtheorem{ex}[theo]{Example}
\newtheorem{nota}[theo]{Notation}
\newtheorem{fact}[theo]{Fact}
\newtheorem{rem}[theo]{Remark}
%=================================
\newcommand{\N}{\ensuremath{\mathbb{N}}}  
\newcommand{\Z}{\ensuremath{\mathbb{Z}}}
\newcommand{\Q}{\ensuremath{\mathbb{Q}}}
\newcommand{\R}{\ensuremath{\mathbb{R}}}
\newcommand{\C}{\ensuremath{\mathbb{C}}}
\newcommand{\T}{\ensuremath{\mathbb{T}}}
\newcommand{\G}{\ensuremath{\mathbb{G}}}
 
%structures
\newcommand{\M}{\ensuremath{\mathcal{M}}}
\newcommand{\Rs}{\ensuremath{\mathcal{R}}}
\newcommand{\Ns}{\ensuremath{\mathcal{N}}}
\newcommand{\Ks}{\ensuremath{\mathcal{K}}}

%Lie algebras
\newcommand{\g}{\ensuremath{\mathfrak{g}}}
\newcommand{\h}{\ensuremath{\mathfrak{h}}}

\newcommand{\sll}{\ensuremath{\mathfrak{sl}}}

\newcommand{\lb}{\ensuremath{\mathfrak{b}}}
\newcommand{\n}{\ensuremath{\mathfrak{n}}}

\newcommand{\Rpl}{\ensuremath{M_{\oplus}}}
\newcommand{\Rx}{\ensuremath{M_{\otimes}}}

\newcommand{\vs}{\vspace{0.2cm}}

\newcommand{\mtf}{\ensuremath{\mathcal{N}}}
\newcommand{\nil}{\ensuremath{\mathcal{N}_\text{nil}}}

\newcommand{\ur}{\ensuremath{\mathcal{U}}}
\newcommand{\Atf}{\ensuremath{\mathcal{A}}}

\newcommand{\se}{\subseteq}
\newcommand{\inv}{^{-1}}
\newcommand{\ro}{\varrho}
\newcommand{\fhi}{\varphi}
\newcommand{\teta}{\vartheta}
\newcommand{\epsi}{\epsilon}
\newcommand{\lra}{\longrightarrow}
\newcommand{\Id}{\mathrm{Id}}
\newcommand{\wt}{\widetilde}
\newcommand{\wh}{\widehat}
\newcommand{\ul}{\underline}
\newcommand{\ol}{\overline}

\newcommand{\df}{\rangle_\mathrm{def}}
\newcommand{\la}{\langle}
\newcommand{\ra}{\rangle}
%================================= 
\DeclareMathOperator{\Aut}{Aut} 
\DeclareMathOperator{\Inn}{Inn}  
\DeclareMathOperator{\Comm}{Comm}
\DeclareMathOperator{\Ad}{Ad}
\DeclareMathOperator{\ad}{ad}
\DeclareMathOperator{\End}{End}
\DeclareMathOperator{\im}{Im}

\DeclareMathOperator{\GL}{GL} 
\DeclareMathOperator{\SL}{SL} 
\DeclareMathOperator{\SO}{SO} 
\DeclareMathOperator{\Or}{O}
\DeclareMathOperator{\PSL}{PSL}
\DeclareMathOperator{\UT}{UT}
\DeclareMathOperator{\card}{card}

\DeclareMathOperator{\tr}{tr}

\DeclareMathOperator{\Lie}{Lie}
\DeclareMathOperator{\id}{id}    

\DeclareMathOperator{\dom}{dom}
\DeclareMathOperator{\Tor}{Tor}

\DeclareMathOperator{\rk}{rk_\mathrm{def}}
\DeclareMathOperator{\w}{w_\mathrm{def}}
%\DeclareMathOperator{\ab}{G_\mathrm{def}^{ab}}

\newcommand{\ab}{G^{ab}}
 
\renewcommand{\leq}{\leqslant}
\renewcommand{\geq}{\geqslant}
%================================= 
%================================= 
%REFERENCES
\newcommand{\book}[2]{{\scshape#1}, {\bf #2}}
\newcommand{\pre}[2]{{\scshape#1}, #2}
\newcommand{\publ}[6]
{{\scshape#1}, #2, {\itshape #3}, {\bf #4} (#5), pp.~#6.}
%================================= 
%================================= 
\begin{document}

\author{Annalisa Conversano} 
 

 
\title[Definable rank, o-minimal groups, and Wiegold's problem]{Definable rank, o-minimal groups,  \\ and Wiegold's problem}

 

\address{Massey University Auckland, New Zealand} 
%
\email{a.conversano@massey.ac.nz}

 
\noindent
\date{\today} 
 
\maketitle

%\vspace{-.5cm}
\begin{abstract} 
 We show that every definable group $G$ in an o-minimal structure 
is definably finitely generated. That is, $G$ 
contains a finite subset that is not included in any proper definable subgroup. This provides another proof, and a generalization to o-minimal groups, of algebraic groups (over an algebraically closed field of characteristic $0$) containing a Zariski-dense finitely generated subgroup.  

  



When the group is definably connected, its dimension provides an upper bound for the definable rank. The upper bound is often strict. For instance, this is always the case for $0$-groups and, more generally, for solvable groups with torsion.

 
 
We further prove that every perfect definable group is normally monogenic, generalizing the finite group case. This yields a positive answer to Wiegold's problem in the o-minimal setting.


 
\end{abstract}


 

\thispagestyle{empty}
 

\bigskip
\section{Introduction and preliminaries}

Every algebraic group over an algebraically closed field $\mathcal{K} = (K, +, \cdot)$ is definable in $\mathcal{K}$. In characteristic $0$, the field $\mathcal{K}$ is definable in any maximal real closed field $\Rs = (R, <, +, \cdot)$. It follows that every algebraic group $G$ over $K$ is definable in the o-minimal structure $\Rs$. On the other hand, a considerable volume of work shows that definable groups in arbitrary o-minimal structures are very closely related to algebraic groups in general. For the compact case, see for instance \cite{BB12, Berarducci-Mamino, BMO10, Edmundo-Otero, HPP1}. For the non-compact case, see \cite{BBO19, me-nilpotent, JC, PPSI, PPSII, PPSIII}.

In this paper we present yet again another similarity, by showing that every definable group in an arbitrary o-minimal structure is definably finitely generated, generalizing the well-known fact that algebraic groups over an algebraically closed field of characteristic $0$ contain a finitely generated Zariski-dense subgroup \cite[Proposition 1]{Tret}.
 


\medskip
The rank of a group is the smallest cardinality of a generating set.
By analogy, we call the \emph{definable rank} of a definable group $G$, the smallest cardinality of a set $X$ such that $G = \la X \df$, where $\la X \df$ denotes the smallest definable subgroup of $G$ containing $X$. We will say that $X$ is a \emph{definably generating set} for $G$. 

In o-minimal structures, $\la X \df$ always exists because the descending chain condition for definable groups holds \cite[Proposition 2.12]{Pillay - groups}.

If $G$ has finite definable rank $n \in \N$, we say that $G$ is \emph{definably finitely generated}, and write $\rk(G) = n$. Following \cite{Strzebonski}, when  $\rk(G) = 1$ we say that $G$ is \emph{monogenic}.  That is, $G$
is monogenic when $G = \la g \df$, for some $g \in G$. 

\medskip
For a definable group, a definably generating set plays the same role as a generating set of an abstract group. For instance, every definable homomorphism is uniquely determined by the value on a definably generating set. In Section 2 we show that every definable group has finite definable rank:


\begin{thmintro}\label{theo:def-fg}
Every definable group is definably finitely generated. If $G$ is definably connected, $\rk(G) \leq \dim G$.
\end{thmintro}

 


%\medskip
In the proof of \Cref{theo:def-fg}, o-minimality plays a fundamental role. For instance, in the real field $\M = (\R, <, +, \cdot)$ the additive group $G = (\R, +)$ is monogenic, as for any $a \neq 0$, $\langle a \df = G$. However, if we add to \M\ a predicate for the integers,  $G$ is not definably finitely generated in 
$\Ns = (\R, <, +, \cdot, \Z)$, as for each finite set $X \subset \R$, the subgroup generated by $X$ is definable in \Ns. This begs the question of which model-theoretic properties on a structure ensure finite definable generation.



 





\medskip
When $G$ is definably connected and not $1$-dimensional nor torsion-free, $\dim G$ is often (always?) a strict upper bound for $\rk(G)$. In particular, we prove this is   the case for groups of small dimension, $0$-groups and, more generally, for solvable groups:

\begin{thmintro}\label{theo:small}
Let $G$ be a definably connected group with $\dim G = n \in \{2, 3\}$. Assume $G$ is not torsion-free. Then either $G$ is monogenic or $\rk(G) = 2$. In particular, 
$\rk(G) < \dim G$ and $\rk(G) = 2$ whenever $G$ is not abelian.
\end{thmintro}

\begin{thmintro}\label{prop:0-mon}
Every $0$-group is monogenic.
\end{thmintro}

\begin{thmintro}\label{theo:solv}
Let $G$ be a solvable definable connected group and $\mtf(G)$ its maximal normal torsion-free definable subgroup. Then
\[
\rk(G) \leq \dim(\mtf(G)).
\]
\end{thmintro}



\Cref{prop:0-mon} provides us with monogenic groups of arbitrary dimension. At the other end of the spectrum, there are definable groups, like $\SL_2(\Rs)$ or $\SO_3(\Rs)$, where for each non-torsion element $g \in G$, $\la g \df$ is $1$-dimensional. See \Cref{lem:SL2} and \Cref{lem:SO3} for details. 

\medskip
Monogenic groups are abelian.  In section 2 we find conditions for an abelian definable group to be monogenic:

\begin{thmintro} \label{theo:ab-conn}
Let $G$ be an abelian definable group. Then $G$ is monogenic if and only if $G^0$
and $G/G^0$ are monogenic.
\end{thmintro}

\begin{corintro} \label{prop:ab}
Let $G$ be an abelian definably connected group. If $\mtf(G)$ is monogenic, $G$ is monogenic. 
\end{corintro}

 

The converse of \Cref{prop:ab} does not hold. In \Cref{ex:subgr} we present a monogenic definably connected group where $\mtf(G)$ has arbitrary definable rank. Moreover, $G^0$ in \Cref{theo:ab-conn} cannot be replaced by an arbitrary definable subgroup. For instance, $\R$ and $\Z/2\Z \times \R$ are monogenic, but $\Z/2\Z \times \R^2$ is not. 

 





\medskip
A group $G$ is said to be \emph{normally generated} by $x \in G$, if $x$ is not contained in any proper normal subgroup of $G$. In general, the smallest normal subgroup of $G$ containing $x$ is called the \emph{normal closure} of $x$ and denoted by $\la x \ra^G$. When $G$ is finitely generated, $G$ is the normal closure of a single element if and only if $G$ is homomorph of a knot group \cite[Theorem 1]{knot}.

Every perfect finite group is normally generated by a single element (see for instance \cite[4.2]{LW}). For infinite groups, it is a well-known open problem since the 1970s attributed to J. Wiegold (see \cite[FP14]{problems02} and \cite[5.52]{notebook}): 

\begin{que}[Wiegold] 
Suppose $G$ is a finitely generated perfect group. Is $G$ normally generated by a single element? 
\end{que}

If $G$ is not assumed to be finitely generated, infinite direct sums of perfect groups provide easy counterexamples. In a first-order setting, we want to restrict ourselves to definable subgroups.
 

 \begin{dfn}
 We say that a definable group $G$ is \emph{normally monogenic} when there is some $x \in G$ that is not contained in any proper normal definable subgroup of $G$. We call such element $x$ a \emph{normal generator} of $G$ and write $\la x \df^G = G$.
 \end{dfn}

Although it is commonly believed that finitely generated perfect counterexamples do exist, a positive answer to Wiegold's question has been given in \cite{EM13} for the compactly generated locally compact groups that do not admit infinite discrete quotients. The compactly generated assumption is the necessary replacement of finite generation. In this paper we give a positive solution in the o-minimal setting, with no further assumption, in the light of \Cref{theo:def-fg}:

\begin{thmintro}\label{theo:Wiegold}
Every perfect definable group is normally monogenic.
\end{thmintro}

The proof of \Cref{theo:Wiegold} is constructive and a normal generator is found explicitely. There are two main structural ingredients: 1) the decomposition of centerless semisimple definable groups into a direct product of definably simple groups \cite[Theo 4.1]{PPSI}, each containing a definable torus \cite[Theo 5.1]{PPSIII} (for the connected case), and 2) the existence of a smallest definable subgroup of finite index \cite[Prop 2.12]{Pillay - groups}, (for the disconnected case). 

\medskip
Throughout the paper groups are definable in an arbitrary o-minimal structure \M. By \emph{definable}, we always mean ``definable with parameters in \M''.

 




 


 


 

 



 

%=====================================
%=====================================
%=====================================
\section{The definable rank}

 

In abstract groups, the rank of a subgroup can be bigger than the rank of the group. 
In fact, there are finitely generated free groups containing subgroups that are not finitely generated. Similarly, there are definable groups containing definable subgroups with bigger definable rank. An example is given below.

\begin{ex}\label{ex:subgr}
 Let $S$ be the semialgebraic group in \cite[5.3]{Strzebonski}. That is, $S = \R \times [1, e[$ with the operation defined by
 \[
 (x, u) \ast (y, v) = 
 \begin{cases}
 (x + y, uv)  & \mbox{ if $uv < e$} \\
 (x+y+1, uv/e) & \mbox{ otherwise}
 \end{cases}
 \]

Fix $n \in \N$, $n > 1$, and set $G = S^n$. That is, $G$ is the direct product of $n$ copies of $S$. The subgroup $H = (\R \times \{1\})^n$ is definably isomorphic to $(\R^n, +)$, so $\rk(H) = n$. However, $G$ is monogenic. For instance, any element of the form $(0, u)^n$ is a definable generator, where $u$ is any non-torsion element in $[1, e[$ with multiplication $\mod e$. In fact, $G$ is a $0$-group and  in the next section  we will show that $0$-groups are always monogenic.
 \end{ex}

In view of \Cref{theo:def-fg}, \Cref{ex:subgr} represents the worse possible scenario for a definable subgroup. Definable quotients are better behaved, instead:



\begin{lem}\label{lem:rk-dec}
Let $H$ be a normal definable subgroup of a definable group $G$. Then
\[
\rk(G/H) \leq \rk(G) \leq \rk(H) + \rk(G/H). 
\]
\end{lem}

\begin{proof}
Let $X = \{g_i : i \in I\} \subset G$ be a set of minimal cardinality of definable generators and
$\ol X = \{\ol g_i : i \in I\} \subset G/H$ the set of their images by the canonical projection $G \to G/H$. If $\la \ol X \df$ is a proper subgroup of $G/H$, its pre-image is a proper definable subgroup of $G$ containing $X$, contradiction. Therefore  $\rk(G/H) \leq \rk(G)$.

Let $s \colon G/H \to G$ be a section of the canonical projection $G \to G/H$.  Suppose $Y \subset G/H$ is a subset of minimal cardinality such that $\la Y \df = G/H$ and set $Y' = \{s(y) \in G : y \in Y\}$. Then $G = \la H \cup Y \df$ and 

\[
\rk(G) \leq \rk(H) + \rk(G/H),
\]

\noindent \medskip
as claimed.
\end{proof}

Every definable group  contains a smallest definable subgroup of finite index $G^0$   \cite[Prop 2.12]{Pillay - groups}. $G^0$ is normal and coincide with the definably connected component of the identity. From \Cref{lem:rk-dec} we have:


\begin{cor}\label{cor:disc}
Let $G$ be a definable group. Then $\rk(G) \leq \rk(G^0) + |G/G^0|$. 
\end{cor}

In particular, $G$ has finite rank if and only if $G^0$ does. For this reason, in proving  \Cref{theo:def-fg} we can restrict ourselves to definably connected groups. In this and the following sections we will make occasionally use of the o-minimal Euler characteristic $E(G)$. We refer to Section 2 of  \cite{JC} for some background. 
As observed in the introduction of \cite{PPSI}:

\begin{fact} \label{fact:tricotomy}
If $G$ is definably connected, then either $E(G) = \pm 1$ (iff $G$ is torsion-free) or $E(G) = 0$.
\end{fact}
 
 
Yves de Cornulier pointed out to us that every algebraic group over an algebraically closed field of characteristic $0$ contains a Zariski-dense 
finitely generated subgroup \cite[Prop 1]{Tret}. Because algebraic groups over an algebraically closed field coincide with the definable groups in it, which in turn are definable in any maximal real closed field, the following provides another (closely related and simplified) proof of this fact, and a generalization to groups definable in o-minimal structures.  
 

\begin{proof}[\textbf{Proof of \Cref{theo:def-fg}}]
 Let $G$ be a definable group. We want to show there is a finite $X \subset G$
such that $\la X \df = G$. We can assume $G$ is definably connected by \Cref{cor:disc}. 

Suppose $\dim G = n$. We will show that we can take $X$ with $|X| \leq n$, by induction on $n$.

If $n = 1$, $G$ has no infinite proper definable subgroup. It follows that $\la g \df = G$ for any non-torsion element $g \in G$.

Suppose $n > 1$. If $G$ is monogenic, there is nothing to prove. Assume $G$ is not monogenic, and let 
$H < G$ be a proper definably connected subgroup of largest possible dimension $k$.   Note that $k < n$, because $G$ is definably connected. We claim that $k \geq 1$.

Since $G$ is not monogenic, for any $g \in G$, $\la g \df^0$ is a proper definably connected subgroup. If $k = 0$, every element of $G$ has finite order. Therefore, $E(G) = 0$ by \Cref{fact:tricotomy}, and $G$ contains an infinite $0$-group by \cite[Lem 2.9]{Strzebonski}. This is a contradiction with $G$ being periodic. Hence $k \geq 1$, as claimed.
 
By induction hypothesis, $\rk(H) \leq k$.

If $H$ is normal in $G$, $\rk(G/H) \leq n-k$ by induction hypothesis again. Moreover,  $\rk(G) \leq \rk(H) + \rk(G/H) = n$ by \Cref{lem:rk-dec}, as wanted.

Assume $H$ is not normal in $G$. Then there is $g \in G$ such that $H \neq gHg\inv$.

Set $X = Y \cup \{g\}$, where $Y$ is a set of minimal cardinality such that $\la Y \df = H$. Define $K = \la X \df$.

$K$ contains $H$ and $gHg\inv$, which are both definably connected. If $\dim K = k$, $K^0 = H = gHg\inv$, contradiction. Thus $\dim K > k$. In particular, $\dim K^0 > k$. By maximality of $\dim H$, $K = K^0 = G$.
 
Therefore, $\rk(G) \leq |X| = |Y| + 1 \leq k+1 \leq n$, as wanted.
\end{proof}

 

\begin{que}\label{que:fg}
In which other structures/theories, besides o-minimal ones, are all definable groups   definably finitely generated? 
\end{que}


%==========================
%==========================
 


\medskip
Monogenic definable groups are always abelian.
More generally, if the elements in $X \subset G$ commute pairwise, $\la X \df$ is abelian. We investigate below which abelian definable groups are monogenic. 

\medskip
We start with definably connected groups. As definably connected $1$-dimensional
 groups have no proper infinite definable subgroups, they are monogenic:

\begin{fact} \label{fact:1dim}
If $G$ is a 1-dimensional definably connected group, $\rk(G) = 1$.
\end{fact}

 

 

Strzebonski showed in \cite[Cor 5.14]{Strzebonski} that semialgebraic $0$-groups 
are monogenic. We generalize below his proof to any $0$-group.
  
 

\begin{proof}[\textbf{Proof of \Cref{prop:0-mon}}]
Suppose first $G$ is definably compact. Then $G$ and each of its definably connected subgroups is a definable torus and, in particular, a $0$-group. We claim that $G$ has countably many definable subgroups. 

Indeed, $G$ has finitely many $k$-torsion elements for each $k \in \N$, so its torsion subgroup is countable. Moreover, every $0$-group is the definable subgroup generated by its torsion elements \cite[Prop 2.20]{me-nilpotent}. It follows that $G$
contains at most countably many $0$-subgroups.

If $H$ is a definable subgroup that is not definably connected, it is an extension of a finite group by $H^0$. As $G/H^0$ has countably many finite subgroups, and there are at most countably many possibilities for $H^0$, $G$ has countably many definable subgroups, as claimed.

Since $G$ is definably connected, each of its proper definable subgroup has smaller dimension and by \cite[Lem 5.12]{Strzebonski}, no definable group can be covered by countably many definable subgroups of smaller dimension. Therefore, $G$ must be monogenic.

Suppose now $G$ is not definably compact, and set $N = \mtf(G)$. The quotient
$G/N$ is a definable torus, so monogenic by the above. Let $g \in G$ such that $\ol g \in G/N$ is a definable generator and set $K = \la g \df$. As $G = NK$,  
\[
E(G/K) = E(N/(N \cap K)) = \pm 1.
\]

\medskip \noindent
However, $G$ is a $0$-group, so it must be $G = K$.
 \end{proof}


 
 
 

One may wonder if $0$-groups are the only monogenic abelian definably connected groups that are not $1$-dimensional. The answer is negative. For instance, if $S$ is Strebonski's group from \Cref{ex:subgr},  $G = \R \times S$ is definably generated by $(1, x)$, where $x$ is any definable generator of $S$. In this case, $G$ is not a $0$-group, as $E(G/S) = E(\R) = -1$. In general, we prove below that whenever the maximal torsion-free definable subgroup $\mtf(G)$ of a definably connected abelian group $G$ is monogenic, then $G$ is monogenic as well. It will be a consequence of more general results about solvable groups. 

We first show the definable rank of a solvable definably connected group $G$ is bounded by the dimension of its maximal normal definable torsion-free subgroup $\mtf(G)$:

 

\begin{proof}[\textbf{Proof of \Cref{theo:solv}}]
 

Set $N = \mtf(G)$. Suppose $\rk(N) = k$ and $\dim N = d$. As torsion-free definable groups are definably connected, $k \leq d$ by \Cref{theo:def-fg}.

Fix a definably generating set $X = \{x_1, \dots, x_k\} \subset N$. We will prove our claim by induction on $n = \dim G$.  

If $G$ is torsion-free, $G = N$ and there is nothing to prove. Otherwise, $G = AN$ by \cite[Prop 3.1]{me2}, where $A$ is a $0$-Sylow subgroup of $G$. By \Cref{prop:0-mon}, $A$ is monogenic. Say, $A = \la a \df$. We will produce a definably generating $k$-subset of $G$ including $a$ and $k-1$ elements from $N$.
  
Set $Y = \{a, x_2, \dots, x_k\}$ and $H = \la Y \df$. The subgroup 
$\la x_2, \dots x_k \df < N$ is torsion-free, so definably connected. Thus $\la x_2, \dots x_k \df \subset H^0$. Similarly, $A = \la a \df \subset H^0$. It follows that
$H = H^0$, $\la H, x_1 \df = G$ and $\rk(G) \leq \rk(H) + 1$.

If $H = G$, $\rk(H) = \rk(G) \leq k$ and our claim is proved.

Otherwise, $\dim H < \dim G$ because $G$ is definably connected.
By induction hypothesis, $\rk(H) \leq \dim \mtf(H)$.


Moreover, $\mtf(H) \subseteq N$, because $G/N$ is definably compact. As $N$ is definably connected, $\dim \mtf(H) < \dim N$. Therefore $\rk(H) < \dim N$ and 

\[
\rk(G) \leq \rk(H) + 1 \leq \dim N.
\]
\end{proof}
 
\begin{cor}
Let $G$ be a definably connected solvable group. If $G$ is not torsion-free, 
\[
\rk(G) < \dim G.
\]
\end{cor}
 
 

\medskip
When $G$ has a unique $0$-Sylow subgroup, \Cref{theo:solv} can be improved by replacing the dimension of $\mtf(G)$ with its definable rank:

\begin{prop} \label{prop:solv-unique}
Let $G$ be a solvable definable connected group. If $G$ has a unique (possibly trivial) $0$-Sylow subgroup, then 
\[
\rk(G) \leq \rk(\mtf(G)).
\]
\end{prop}

\begin{proof}
Set $N = \mtf(G)$ and suppose $\rk(N) = k$. Fix a definably generating set $X = \{x_1, \dots, x_k\} \subset N$. If $G$ is torsion-free, $G = N$ and there is nothing to prove. Otherwise, $G = AN$ by \cite[Prop 3.1]{me2}, where $A$ is a $0$-Sylow subgroup of $G$. By \Cref{prop:0-mon}, $A$ is monogenic. Say, $A = \la a \df$. Set $Y = \{ax_1, \dots, ax_k \}$
and $K = \la Y \df$. We will show that $K = G$.

%By assumption $A$ is the unique $0$-Sylow subgroup of $G$, so it is normal. 

Let $p_1 \colon G \to G/N$ be the canonical projection, set $p_1(a) = \bar{a}$ and $H_1 = \la \bar{a} \df$. The pre-image of $H_1$ in $G$ is a definable subgroup containing $N$ and $a$ generating $A$. Hence it cannot be a proper definable subgroup. Since $p_1(ax_i) = p_1(a) = \bar{a}$, the restriction of $p_1$ to $K$ is a surjective map, and $G = NK$. Even if $K$ is not a normal subgroup of $G$, there is a definable bijection between definable sets $G/K$ and $N/(N \cap K)$. Hence $|E(G/K)|=1$ and every $0$-Sylow subgroup of $K$ is a $0$-Sylow subgroup of $G$. Because $G$ has a unique $0$-Sylow subgroup by assumption,  $A \subseteq K$.
In particular, $a \in K$. It follows that $X \subset K$ too, as $x_i = a\inv (ax_i) \in K$,
and $K = G$, as wanted.
\end{proof}

\begin{cor}\label{cor:ab}
Let $G$ be a definably connected abelian group. If $\mtf(G)$ is monogenic, $G$
is monogenic. 
\end{cor}

\begin{proof}
Every abelian group has a unique $0$-Sylow subgroup and \Cref{prop:solv-unique} applies.
\end{proof}


 




 \begin{rem}
The converse of  \Cref{cor:ab} does not hold. The semialgebraic group $G$ in \Cref{ex:subgr} is definably connected and monogenic, but $\mtf(G) = H$ is not. 
\end{rem}
 
We do not know if there are monogenic torsion-free definable groups that are not $1$-dimensional. More generally:

\begin{que}
Is there a torsion-free definable group $G$ such that 
\[
\rk(G) < \dim G?
\]
\end{que}

%=====================================
%=====================================
%=====================================


\medskip
Monogenic groups do not need to be definably connected. For instance:
 

 
\begin{ex} \label{ex:mono-disc} 
Let $G = \Z/2\Z \times \R$. For each $x \in \R$, $x \neq 0$, we can check that 
\[
G = \la (1, x) \df.
\]
 
 \noindent
Indeed, set $K = \la (1, x) \df$. Since $(1, x)^2 = (0, 2x) \in K$, $\la (0, 2x) \df = \{0\} \times \R \subset K$. In particular, $(0, x) \in K$ and $(1, x)(0, x)\inv = (1, 0) \in K$. 
\end{ex}

In fact, we can show that an abelian definable group $G$ is monogenic if and only if 
$G^0$ and $G/G^0$ are monogenic. We will use the following:

\begin{fact}\cite[Theo 1.8]{JC} \label{fact:disconnected}
For any definable group $G$ there are finite subgroups $F$ such that $G = FG^0$.
\end{fact}

\begin{proof}[\textbf{Proof of \Cref{theo:ab-conn}}]
Suppose $G = \la g \df$ is monogenic. The quotient $G/G^0$ is monogenic by \Cref{lem:rk-dec}. In particular, any definable generator of $G$ projects to a definable generator of $G/G^0$. 

By \Cref{fact:disconnected}, there is a finite subgroup $F$ such that $G = FG^0$. Let $x \in F$, $y \in G^0$ such that $g = xy$,
and set $K = \la y \df$. We claim that $K = G^0$.

The image of $g$ in $G/F$ is $\ol y$ and $G/F = \la \ol y \df$, otherwise the pre-image of $\la \ol y \df$ would be a proper definable subgroup of $G$ containing $g$, contradiction. It follows that $G = FK$ and $K$ contains $G^0$. By minimality, $K = G^0$.

Conversely, suppose $G/G^0 = \la \ol x \df$ and $G^0 = \la y \df$. Let $g = xy$, where $x \in F$ is an element in the pre-image of $\ol x$, and set $K = \la g \df$.
We claim that $G = K$.

The image of $g$ in $G/F$ through the canonical projection is equal to image of $y$, so it is a definable generator, and $G = FK$. Hence $\dim K = \dim G$ and $G^0 \subseteq K$.

On the other hand, $g$ is mapped to the definable generator $\ol x$ through the canonical projection $G \to G/G^0$, so $G = KG^0$. Therefore, $G = K$.  
\end{proof}

 

 

 
%===========================
 
 
\section{Groups of small dimension}
 
 
Definably connected groups of small dimension are completely characterized.

\begin{fact}\cite[Theo 2.6]{NPR} \label{fact:2dim}
Let $G$ be a definably connected $2$-dimensional group. Then
$G$ is solvable. Moreover, either $G$ is abelian, or $G$ is centerless and definably
isomorphic to a semidirect product of $\Rs_a$ and $\Rs_m$, where $\Rs = (R, < +, \cdot)$ is a definable real closed field, $\Rs_a = (R, +)$ and $\Rs_m = (R_{>0}, \cdot)$.
\end{fact}

\begin{fact}\cite[Prop 3.1 \& 3.5]{NPR} \label{fact:3dim}
Let $G$ be a definably connected $3$-dimensional group. If $G$ is not solvable, then it is semisimple and $G/Z(G)$ is definably isomorphic to either $\PSL_2(\Rs)$ or $\SO_3(\Rs)$, for some definable real closed field $\Rs$. 
\end{fact}

 

For definably connected groups of small definable rank the situation is much more complicated. For instance, there are semisimple definable groups of definable rank $2$, like $\SL_2(\Rs)$ or $\SO_3(\Rs)$. 


 


 
 


\begin{lem}\label{lem:SL2}
Let $\Rs$ be a real closed field and $G = \SL_2(\Rs)$. Then $\rk(G) = 2$. Moreover, for every non-torsion element $g \in G$, $\dim \la g \df = 1$.
\end{lem}

\begin{proof}
 $G$ is a $3$-dimensional definably connected group. Let $\tr g$ denote the trace of $g \in G$. The element $g$ is
\emph{elliptic} if $|\tr g | < 2$, \emph{parabolic} if $|\tr g | = 2$,
\emph{hyperbolic} if $|\tr g| > 2$. In $\SL_2(\R)$ an element is conjugate to a rotation if and only if it is elliptic. It follows that $\la g \df < G$ is definably compact if and only if $g$ is elliptic. Moreover, the Lie algebra 
$\sll_2(\R)$ has no $2$-dimensional abelian subalgebras, so every $2$-dimensional definable subgroup of $G$ is torsion-free and non-abelian (see \cite{PPSI} for the Lie algebra theory of definable groups). In particular, $\la g \df$ is $1$-dimensional torsion-free if and only if $g$ is either parabolic or hyperbolic.
Therefore, given $X = \{a, b\}$, whenever $a$ is non-torsion elleptic and $b$ is either parabolic or hyperbolic, $\dim \la X \df > 2$ and  $\la X \df =  G$. 
\end{proof} 

\begin{lem}\label{lem:SO3}
Let $\Rs$ be a real closed field and $G = \SO_3(\Rs)$. Then $\rk(G) = 2$. Moreover, for every non-torsion element $g \in G$, $\dim \la g \df = 1$.
\end{lem}

\begin{proof}
$G$ is a $3$-dimensional definably connected definably compact group. Let $g \in G$
be a non-torsion element. $\la g \df \neq G$, as $G$ is not abelian. So $\dim \la g \df < 3$. Moreover, $\dim \la g \df \neq 2$, as the maximal definable torus of $G$ is $1$-dimensional. It follows that $\dim \la g \df = 1$ and $T = \la g \df^0$ is a maximal definable torus of $G$. 

 Let $h \in G$ be any element that is not in the normalizer of $T$.  Set $H = \la g, h \df$. We claim that $H = G$. $H$ contains the $1$-dimensional definably connected subgroups $T$ and $hTh\inv$. If $\dim H = 1$, $H^0 = T = hTh\inv$, contradiction.
If $\dim H = 2$, then $H^0$ is abelian, contradiction again. Therefore $\dim H = 3$, and $G = H$, because $G$ is definably connected.
 \end{proof}

 

\begin{proof}[\textbf{Proof of \Cref{theo:small}}]
Suppose $G$ is definably compact. If $G$ is abelian, $G$ is a definable torus and monogenic by \Cref{prop:0-mon}. If $G$ is not abelian, $\dim G \neq 2$, because no $2$-dimensional non-abelian definable group is definably compact,  by \Cref{fact:2dim}. Therefore, $\dim G = 3$ and by \Cref{fact:3dim}, $G/Z(G)$ is definably isomorphic to $\SO_3(\Rs)$, for some definable real closed field $\Rs$. 

 By \Cref{lem:SO3}, $\rk(G) = 2$. Any two elements $a, b \in G$ in the pre-image of a definably generating set for 
$\SO_3(\Rs)$ is a definably generating set, because $G$ is definably connected and $Z(G)$ is finite.

 
Suppose $G$ is not definably compact. Assume $G$ is solvable. Then $\mtf(G)$ is an infinite definable subgroup of $G$.  

As $G$ is not torsion-free, $G = \mtf(G)A$, where $A$ is any $0$-Sylow subgroup of $G$ and $\dim G = 3$ by \Cref{fact:2dim} again. If $\dim \mtf(G) = 1$, $\mtf(G)$ is central by \cite{JC}. It follows that $G$ is abelian and monogenic by \Cref{prop:ab}. If $\dim \mtf(G) = 2$, then $\rk(G) = 2$ by \Cref{theo:solv}.  

Assume $G$ is not solvable. Then $\dim G = 3$ and by \Cref{fact:3dim}, $G/Z(G)$ is definably isomorphic to $\PSL_2(\Rs)$, for some definable real closed field $\Rs$.  Any two elements $a, b \in G$ in the pre-image of a definably generating set for 
$\PSL_2(\Rs)$ from \Cref{lem:SL2} is a definably generating set, because $G$ is definably connected and $Z(G)$ is finite.
\end{proof}
   
 
\begin{que}
Is there a definably connected group $G$ that is not torsion-free and such that  
\[
\rk(G) = \dim G?
\]
\end{que}



 

  

%=====================================
%=====================================
%=====================================

 

 


%\medskip
\section{On Wiegold's problem}

 

 
 
 
In this section we present a solution to Wiegold's problem in the o-minimal setting. 

 

\begin{proof}[\textbf{Proof of \Cref{theo:Wiegold}}]
We first consider the case where $G$ is definably connected. Since $G$ is perfect, $G$ is not solvable. 

\medskip
Assume $G$ is semisimple. By \cite[Theo 4.1]{PPSI}, the quotient $\ol G = G/Z(G)$ is a direct product $H_1 \times \dots \times H_k$
of definably simple groups $H_i$. We can see that $\ol G$ is normally monogenic by induction on $k$. It suffices to consider the case $k = 2$. 

Let $x_1 \in H_1$ be a $2$-torsion element and $x_2 \in H_2$ be a $3$-torsion element.
We claim that $(x_1, x_2) \in H_1 \times H_2$ is a normal generator. Set $N = \la (x_1, x_2) \df^{\ol G}$.  

As $(x_1, x_2)^2 = (e, x_2^2) \in N$ and $x_2^2 \neq e$, $N \cap (\{e\} \times H_2)$ is a normal non-trivial definable subgroup of the definably simple $H_2$. Hence $\{e\} \times H_2 \subset N$. Similarly, $(x_1, x_2)^3 = (x_1, e) \in N$ and $N \cap (H_1 \times \{e\}) = H_1 \times \{e\} \subset N$. Therefore, $N = \ol G$.

Note that we could have deduced $(x_1, e) \in N$ by $(x_1, x_2), (e, x_2) \in N$, regardless of the fact that $x_2$ is a $3$-torsion element. However, when $k > 2$, we need the stronger assumption for the $x_i's$ to be $p_{i+1}$-torsion elements to ensure that $N = \ol G$. We know that each $H_i$ contains torsion elements of each prime order because definably simple groups are elementarily equivalent to centerless simple Lie groups \cite[Theo 5.1]{PPSIII}, which always contain a $1$-dimensional torus.

Let now $g \in G$ be any element in the pre-image of a normal generator of $\ol G$ and set $N = \la g \df ^G$. As $Z(G)N = G$ and $Z(G)$ is finite, $\dim N = \dim G$.
However, $G$ is definably connected, so $G = N$ and $G$ is normally monogenic.

\medskip
Assume $G$ is not semisimple. Then $G$ contains a maximal solvable definably connected subgroup $R$, its solvable radical, such that $G/R$ is semisimple. By the semisimple case, $G/R$ is normally monogenic. Let $g \in G$ be any element in the pre-image of a normal generator of $G/R$.


 
 

Suppose $H$ is a normal definable subgroup containing $g$. Since $RH = G$, the quotient $G/H = R/(R \cap H)$ is solvable. However, $G$ is perfect, so it must be $G = H$. Therefore, $G = \la g \df^G$.

\bigskip
Let's now consider the case where $G$ is not definably connected. 

\medskip
\textbf{Claim I.} There is  $x \in G$ such that $\langle x \df^G = H$ satisfies 
$G = HG^0$ and $H$ is of minimal dimension.

 \begin{proof}[Proof of Claim I]
We know that $G/G^0$ is a perfect finite group, so normally monogenic. Let $\ol x \in G/G^0$ be a normal generator. For any $x \in G$ in the pre-image of $\ol x$, every normal definable subgroup $H$ containing $x$ maps onto $G/G^0$ and $G = HG^0$. We can take $x$ such that $H$ of minimal dimension.  
\end{proof}


 
\medskip
The quotient $G/H = (HG^0)/H = G^0/(G^0 \cap H)$ is a perfect definably connected group, so normally monogenic by the connected case above. Let $\ol y \in G/H$ be a normal generator and $y \in G$ an element in the pre-image of $\ol y$. Since $G^0$ maps onto $G/H$, we can take $y \in G^0$. 
%Set $K = \langle y \df^G$. Then $HK = G$.


Set $g = xy$ and $K = \langle g \df^G$. Recall that $x$ is any element in the pre-image of a normal generator $\ol x \in G/G^0$. We will show that $g$ is a normal generator. That is, $K = G$.

 
Since $y \in G^0$, the image of $g$ in $G/G^0$ is $\ol x$ that normally generates. Therefore $K$ maps onto $G/G^0$ and $KG^0 = G$. 

On the other hand, the image of $g$ in $G/H$ is $\ol y$ that normally generates too.
Therefore $K$ maps onto $G/H$ and $KH = G$. Hence
\[
G/H = (KH)/H = K/(K \cap H) = G^0/(G^0 \cap H).
\]

\bigskip
\textbf{Claim II.} $(K \cap H)K^0 = K$.

\begin{proof}[Proof of Claim II]
Set $A = K \cap H$ and $B = AK^0$. We need to show that $B = K$. The quotient 
\[
K/B = \dfrac{K/K^0}{B/K^0} 
\]

\medskip  
is a finite group, because $K/K^0$ is finite. On the other hand,
\[
K/B = \dfrac{K/A}{B/A}
\]

\medskip
is definably connected, as it is a quotient of the group $K/A$ that is definably connected, because isomorphic to $G^0/(G^0 \cap H)$.
Therefore $B = K$, as claimed. 
\end{proof}


\bigskip
\textbf{Claim III.}  There is a finite subgroup $F$ of $K \cap H$ such that $F G^0 = G$. 

\begin{proof}[Proof of Claim III]
By \Cref{fact:disconnected}, there is a finite subgroup $F$ of $K \cap H$ such that $K \cap H = F(K \cap H)^0$. Since $(K \cap H)^0 \subseteq K^0$, it follows that $K = (K \cap H)K^0 = FK^0$. Moreover, $KG^0 = G$ and $K^0 \subseteq G^0$, hence $FG^0 = G$.
 \end{proof}


\noindent
Because $G/G^0$ is normally monogenic, there is some $a \in F$ such that $\langle a \df^G \cdot G^0 = G$. Set $N = \langle a \df^G $. Since $H$ is of minimal dimension,

\[
\dim H \leq \dim N \leq \dim (K \cap H),
\]

\medskip
because $a \in K \cap H$, which is a normal definable subgroup of $G$, as $K$ and $H$ are.
It follows that $\dim H = \dim (K \cap H)$ and $H^0 \subseteq K$. Since $G = KH$, it must be $\dim G = \dim K$. Therefore, $G^0 = K^0$. On the other hand, 
$G = KG^0$ so $G = K$, as wanted.
\end{proof}

\begin{que}\label{que:normally-mon}
In which other structures/theories, besides o-minimal ones, is every definable perfect (definably finitely generated) group normally monogenic? 
\end{que}
 

%=================================
%=================================
 
 
\medskip \noindent
 \textbf{Acknowledgements.} Thanks to Yves de Cornulier for useful references and remarks about algebraic groups.  Thanks to Nicolas Monod for introducing me to Wiegold's problem. 
 
 
 
 
 

 


 



\vspace{.2cm}

\begin{thebibliography}{99}


 
 

\bibitem{BB12} E. Baro and A. Berarducci, Topology of definable abelian groups in o-minimal structures, Bulletin of the London Mathematical Society 44 (2012), no. 3, 473--479. \\

 

\bibitem{BBO19} E. Baro, A. Berarducci and M. Otero, Cartan subgroups and regular points of o-minimal groups,  Journal of the London Mathematical Society, 100 (2019), 361--382.\\

\bibitem{problems02} G. Baumslag, A. G. Myasnikov, and V. Shpilrain. Open problems in combinatorial group theory. Second
edition. In Combinatorial and geometric group theory (New York, 2000/Hoboken, NJ, 2001), volume 296
of Contemp. Math., Amer. Math. Soc., Providence, RI (2002), 1--38. \\


\bibitem{Berarducci-Mamino} A. Berarducci and M. Mamino,  On the homotopy type of definable groups in an o-minimal structure, Journal of the London Mathematical Society, 83 (2011), 563--586.\\

\bibitem{BMO10} A. Berarducci, M. Mamino and M. Otero. Higher homotopy of groups definable in o-minimal structures, Israel Journal of Mathematics, 180(1), 143--161, 2010. \\
%
 

 
\bibitem{me2} A. Conversano, Maximal compact subgroups in the o-minimal setting, Journal of Mathematical Logic, vol. 13 (2013), 1--15. \\
%
% 
%
\bibitem{me-nilpotent} A. Conversano, Nilpotent groups, o-minimal Euler characteristic and linear algebraic groups, Journal of Algebra, 587 (2021), 295--309.  \\
%
% 

\bibitem{JC} A. Conversano,  A Jordan-Chevalley decomposition beyond algebraic groups, arXiv:2203.02637. \\  
%
 
\bibitem{Edmundo-Otero}  M. Edmundo and M. Otero, Definably compact abelian groups, Journal of Mathematical Logic,  4 (2004), 163--180. \\

 \bibitem{EM13} A. Eisenmann and N. Monod, Normal generation of locally compact groups, Bulletin of the London Mathematical Society 45 No. 4 (2013), 734--738. \\ 

 

\bibitem{knot} F. Gonzalez-Acuna, Homomorphs of knot groups, Annals of Mathematics, 102 (1975), 373--377.\\

\bibitem{HPP1} E. Hrushovski, Y. Peterzil and A. Pillay, Groups, measures, and the NIP, Journal of the American Mathematical Society, 21 (2008), 563--596. \\

\bibitem{notebook}
Evgeny~I. Khukhro and Viktor~Danilovich Mazurov (eds.), The Kourovka
  notebook, Sobolev Institute of Mathematics. Russian Academy of Sciences.
  Siberian Branch, Novosibirsk, 2022, Unsolved problems in group theory,
  Twentieth edition, June 2022 update. \\

\bibitem{LW} J. C. Lennox and J. Wiegold, Generators and killers for direct and free products, Arch. Math. , 34(4):296--300, 1980. \\
 

\bibitem{NPR}  A. Nesin,  A. Pillay, and V. Razenj,   Groups of dimensions two and three over o-minimal structures, Annals of Pure and Applied Logic, 53(3) (1991), 279--296. \\
% 
\bibitem{PPSI} Y. Peterzil, A. Pillay and S. Starchenko, Definably simple groups in o-minimal structures, Transactions of the American Mathematical Society, 352 (2000), 4397--4419.\\

\bibitem{PPSII} Y. Peterzil, A. Pillay, and S. Starchenko, Simple algebraic and semialgebraic groups over real closed fields, Transactions of the American Mathematical Society, 352 (2000), 4421-4450.\\
 
\bibitem{PPSIII} Y. Peterzil, A. Pillay, and S. Starchenko, Linear groups definable in o-minimal stuctures, 
Journal of Algebra, 247 (2002), 1--23.\\
 
\bibitem{Pillay - groups}  A. Pillay, On groups and fields definable in o-minimal structures, Journal of Pure and Applied Algebra, 53 (1988), 239-255. \\

 
\bibitem{Strzebonski} A. Strzebonski, Euler characteristic in
semialgebric and other o-minimal groups, Journal of Pure and Applied Algebra, 86 (1994), 173--201.\\

\bibitem{Tret} C. Tretkoff and M. Tretkoff, Solution of the inverse problem in differential Galois theory in the classical case, American Journal of Mathematics 101
(1979), 1327--1332.  

 \end{thebibliography}

\end{document}