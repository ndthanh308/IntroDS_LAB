% \documentclass[a4paper, 10pt]{article}

% % PREAMBLE

% % Improved formatting of tables
% \usepackage{booktabs}

% % Package helping in the formatting of e.g. CO2
% \usepackage{chemformula}

% % Enable hyperlinks; colour them all blue
% \usepackage[colorlinks=true, linkcolor=blue, citecolor=blue, urlcolor=blue]{hyperref}

% % Help formatting figure and subfigure captions correctly
% \usepackage{caption}
% \usepackage{subcaption}

% % Package helping to format intra-document references to figures, tables etc. 
% \usepackage[capitalise, noabbrev]{cleveref}

% % Package helping formatting the title and author section of the paper
% \usepackage{authblk}

% % Language, hyphenation and locale
% \usepackage[UKenglish]{babel}

% % Enable including grahpics
% \usepackage{graphicx}

% % Skip line on new paragraph (instead of indent)
% \usepackage{parskip}

% % Page border layout
% \usepackage[left=1.5cm, right=1.5cm, top=1.5cm, bottom=2cm]{geometry}

% % Set up font
% \usepackage{amsmath}
% \usepackage[notextcomp]{stix2}

% % Section title formatting
% \usepackage[small]{titlesec}

% % Ensure that figures are put in their own section
% \usepackage[section]{placeins}

\pretocmd{\subsection}{\FloatBarrier}{}{}


% Rename figures to be of format S<figure number>
\renewcommand{\thefigure}{S\arabic{figure}}
\renewcommand{\thetable}{S\arabic{table}}


% \begin{document}

\appendix

\section{Additional methodological details}

\subsection{Energy system optimisation models and dual variables}

Many energy system optimisation models (such as PyPSA-Eur) are formulated as a linear program, which means they have a linear objective and linear constraints:
\begin{align*}
    \min_{x \in \mathbb{R}^N} c^T \cdot x &\text{ s.t. } Ax \leq b, x \geq 0 \\
    A \in \mathbb{R}^{M \times N}, b \in \mathbb{R}^M, c \in \mathbb{R}^N, &\text{ for } M, N \in \mathbb{N}.
\end{align*}
This formulation gives rise to a dual problem
\begin{align*}
    \max_{y \in \mathbb{R}^M} b^T \cdot y &\text{ s.t. } A^Ty \geq c, y \geq 0 \\
    A \in \mathbb{R}^{M \times N}, b \in \mathbb{R}^M, c \in \mathbb{R}^N, &\text{ for } M, N \in \mathbb{N}.
\end{align*}
We are interested in the dual variables that stem from the nodal energy balance constraints for every time step $t$ and node $n$; equation (12) in Brown et al.\ \cite{brown-horsch-ea-2018}.
These constraints ensure that supply meets a given inelastic electricity demand at each hour and node, and following Brown et al.\ \cite{brown-horsch-ea-2018} we denote their respective dual variables $\lambda_{n,t}$ (also known as marginal or shadow prices).
By definition, $\lambda_{n,t}$ is the rate of change of the objective function, here total system costs, with respect to demand at node $n$ and time $t$.
More usefully, $\lambda_{n,t}$ (given in EUR / MWh) can be interpreted as the marginal electricity price at each node and time step.
Letting $d_{n, t}$ be electricity demand, $d_{n,t} \cdot \lambda_{n,t}$ is the cost of satisfying electricity load at node $n$ and hour $t$.
It follows that $\sum_{n, t} d_{n,t} \cdot \lambda_{n,t}$ is the total cost of electricity over the entire modelling horizon.

It should be noted that the marginal prices $\lambda_{n,t}$ typically do not follow the same profile as real electricity market prices; this is due to the inclusion of capacity expansion in our model.
This leads $\lambda_{n,t}$ to not only be driven by marginal operating costs of power plants, as in free electricity markets, but mainly by conditions triggering investments.
Thus, the shadow prices $\lambda_{n,t}$ typically stay very low most of the type, and increase drastically during periods necessitating additional investment in generation, storage and transmission capacity.
Nonetheless, $\sum_{n,t} d_{n,t} \lambda_{n,t} / \sum_{n,t} d_{n,t}$ gives a good indication of the system-average electricity price resulting from the model.

In a simple greenfield capacity expansion model, with no included existing infrastructure, the total cost of electricity $\sum_{n,t} d_{n,t} \lambda_{n,t}$ (plus the shadow cost of emissions in case of a global emission constraint) is equal to the objective value of the optimisation problem; this following from strong duality for linear programs.
Since our model includes existing transmission, hydropower, nuclear and biomass generation infrastructure whose costs are not included in the objective function, the objective value is lower than the total electricity cost.
Still, $\sum_{n,t} d_{n,t}$ is a good indicator for total system cost.


\subsection{Transmission congestion and value of stored energy}
For each transmission line $l$, the electricity flow $f_{l,t}$ over that line at time $t$ is subject to the constraints $f_{l,t} \geq -F_l$ and $f_{l,t} \leq F_l$ where $F_l$ is the capacity of the line in MW and the sign determines the direction of the flow.
The dual variables $\mu^{\text{lower}}_{l,t}$ and $\mu^{\text{upper}}_{l,t}$ to these constraints are called the shadow prices of congestion.
The capacity-weighted sum $\sum_l (\mu^{\text{lower}}_{l,t} + \mu^{\text{upper}}_{l,t}) F_l$ is the congestion rent of the network, and equal to the surplus gained by the transmission grid at time $t$ \cite{biggar-hesamzadeh-2014}.
This way we can judge whether certain periods are determining in the transmission expansion decisions.

Similarly, constraints preserving the state of charge from one hour to the next give rise to dual variables which can be interpreted as the marginal value of stored energy, with each storage unit discharging if and only if its value of stored energy is below the marginal price of electricity at the network node it is connected to \cite{crampes-trochet-2019, williams-green-2022}.
It should be noted that these considerations can be a useful indicator for locating crucial regions.


\subsection{Selection of system-defining events}

Recall that an event starting at $t_0$ and lasting for $T$ hours is considered system-defining if
\begin{equation}
    \sum_{n} \sum_{t=t_0}^{t_0 + T - 1} d_{n,t} \cdot \lambda_{n,t} \geq C
\end{equation}
for $C = 100$ bn EUR and $T \leq 336$ (the number of hours in two weeks).

A priori, many overlapping time periods of the same or different lengths can attain the above thresholds.
For example, if the period $[t_0, t_1]$ is system-defining and strictly shorter than two weeks, then $[t_0, t_1 + 1]$ is also system-defining.
For the purposes of this study, we select a disjoint subset of all system-defining events.
In particular, we build up the subset iteratively by going through system-defining events from shorter to longer events (and in decreasing order of total electricity cost for events of the same length), and only adding each event to the selected subset if it does not overlap with previously selected events.
This corresponds to imposing a partial order on all system-defining events by defining $e_1 < e_2$ if and only if $e_1$ and $e_2$ overlap and $e_1$ is shorter than $e_2$ or, if of the same length, is more expensive; our selected subset consists of the minimal elements of the resulting partially ordered set.

As a final step, we extend the selected events on either side as long as this does not decrease event-average hourly electricity cost.
Thus, for the left side of each event, we extend from $[t_0, t_1]$ to $[t_0 - 1, t_1]$ as long as
\begin{align}
    \frac{1}{t_1 - t_0} \sum_{n} \sum_{t=t_0}^{t_1} d_{n,t} \cdot \lambda_{n,t} \leq \frac{1}{t_1 - (t_0 - 1)} \sum_{n} \sum_{t=t_0 - 1}^{t_1} d_{n,t} \cdot \lambda_{n,t}.
\end{align}
The right side of the events is extended similarly.


\subsection{Validation using load shedding}

To compute load shedding profiles to compare to shadow prices, we fix system designs $D_j$, each obtained by a capacity expansion based on a weather year $y_j$, $j \in \{1980/81, \dots, 2019/20\}$ (preserving winters from July -- June), and optimise the dispatch of $D_j$ year-by-year with all weather years $y_i, i \in \{1980/81, \dots, 2019/20\}$.
The forty initial optimisations lead to different electricity networks with large discrepancies in total system costs (as in Grochowicz et al.\ \cite{grochowicz-vangreevenbroek-ea-2023}) and are often inadequate for weather conditions that are not represented in the inputs.
Keeping the capacities of $D_j$ fixed, we add an artificial generator at each node $n$ which can supply electricity at very high variable (and no capital) costs if demand cannot be met any other way.
The power supplied by this artificial generator, $g^j_{n,t}$ can be interpreted as load shedding and quantifies the extent and times during which the system fails to meet demand.

For each weather year $y_i$, we compute the average load shedding $\bar{\ell_t}$ across all 40 system designs $D_j$ (although $D_i$ cannot have any load shedding for $y_i$ by the model formulation), thus obtaining values for each time step between July 1980 and June 2020:
\begin{align}
    \bar{\ell_t} = \frac{1}{40} \sum_j \sum_n g^j_{n,t},
\end{align}
where $g^j_{n,t}$ is the load shedding at node $n$ when the system design $D_j$ is operated at time $t$.

One advantage of using load shedding over electricity shadow prices is that latter may suffer from ``overshadowing'' effects.
Since shadow prices indicate events triggering investment, one event might overshadow another in the same weather year if one is slightly more severe than the other but similar otherwise, thus triggering investments (leading to high shadow prices) that render the second event benign.
We see limited evidence of this in Fig. S16 (comparing electricity shadow prices and load shedding), but shadow prices and load shedding match well for the most severe events (Figs. S12--15).



\section{Additional figures}

\subsection{Inter-annual variability of investment decisions}
\label{sec:investment-variation}

% Figure environment removed

\subsection{Duration and cost of system-defining events}
\label{sec:event-duration-cost}

% Figure environment removed

\newpage

\subsection{System-defining events across different years}
\label{sec:events-years}

% Figure environment removed

% Figure environment removed



% Figure environment removed

% Figure environment removed

% Figure environment removed

% Figure environment removed




\subsection{Key metrics for the system-defining events}
\label{sec:key-metrics}
% Figure environment removed

% Figure environment removed

\begin{table}
\begin{tabular}{p{2.5cm}p{2.5cm}p{1cm}p{1cm}p{1cm}p{2cm}p{1.5cm}p{1.5cm}p{1.5cm}}
    \toprule
    Start & End & Wind anom. [GW] & Solar anom. [GW] & Load anom. [GW] & Transmission [EUR/MW] & Hydrogen [EUR/MWh] & Battery [EUR/MWh] & Hydro [EUR/MWh] \\
    \midrule
    1981-02-13 04:00 & 1981-02-21 08:00 & -93.9 & 14.8 & 34.4 & 22.3 & 587.2 & 1102.1 & 18.2 \\
    1982-02-16 12:00 & 1982-02-25 10:00 & -81.4 & -3.5 & 26.5 & 31.3 & 359.5 & 983.0 & 15.6 \\
    1982-11-27 12:00 & 1982-12-03 23:00 & -87.1 & -5.1 & 21.9 & 36.5 & 390.1 & 1329.8 & 18.1 \\
    1985-01-07 16:00 & 1985-01-12 17:00 & -138.4 & 25.2 & 78.7 & 48.5 & 610.0 & 1582.3 & 19.8 \\
    1985-11-19 16:00 & 1985-11-30 15:00 & -100.2 & -3.1 & 48.7 & 23.8 & 553.5 & 799.5 & 6.1 \\
    1987-01-19 16:00 & 1987-01-24 01:00 & -137.6 & 9.6 & 52.2 & 49.7 & 707.4 & 1885.4 & 38.6 \\
    1987-11-26 15:00 & 1987-12-09 13:00 & -54.1 & -4.4 & 20.7 & 24.3 & 297.3 & 692.5 & 12.2 \\
    1989-12-31 06:00 & 1990-01-05 21:00 & -109.7 & 18.9 & 8.9 & 52.7 & 545.9 & 1561.2 & 20.4 \\
    1990-12-14 04:00 & 1990-12-19 22:00 & -132.3 & 11.3 & 28.7 & 48.9 & 702.5 & 1493.6 & 6.1 \\
    1991-01-27 16:00 & 1991-02-05 08:00 & -121.3 & 18.6 & 36.4 & 23.6 & 597.3 & 1020.7 & 20.5 \\
    1992-12-21 02:00 & 1992-12-30 10:00 & -76.5 & 7.3 & -2.0 & 34.6 & 625.2 & 1003.7 & 28.3 \\
    1993-11-21 16:00 & 1993-11-30 06:00 & -37.7 & 8.0 & 50.8 & 46.5 & 297.9 & 992.7 & 14.0 \\
    1994-12-15 16:00 & 1994-12-25 03:00 & -25.9 & 1.7 & 10.4 & 65.9 & 218.1 & 822.4 & 15.2 \\
    1995-12-16 20:00 & 1995-12-21 21:00 & -121.5 & -8.6 & 17.8 & 62.5 & 391.6 & 1527.4 & 24.1 \\
    1997-01-05 16:00 & 1997-01-09 10:00 & -122.8 & -11.4 & 45.3 & 64.1 & 982.9 & 2121.7 & 19.1 \\
    1997-11-24 04:00 & 1997-12-05 23:00 & -56.5 & -8.7 & 20.8 & 45.3 & 444.8 & 689.2 & 15.1 \\
    1998-11-18 13:00 & 1998-11-25 23:00 & -56.7 & 20.9 & 54.5 & 46.0 & 752.4 & 1083.4 & 16.2 \\
    2000-01-17 06:00 & 2000-01-27 09:00 & 10.3 & 10.5 & 23.5 & 37.4 & 122.6 & 706.4 & 5.0 \\
    2001-01-16 15:00 & 2001-01-19 20:00 & -124.3 & 7.8 & 38.5 & 76.0 & 1333.8 & 2702.0 & 45.8 \\
    2003-02-06 17:00 & 2003-02-15 07:00 & -91.1 & 9.5 & 18.8 & 24.5 & 304.5 & 991.8 & 3.9 \\
    2004-11-28 16:00 & 2004-12-03 09:00 & -90.4 & -18.8 & 20.9 & 60.1 & 868.4 & 1785.0 & 34.1 \\
    2004-12-08 16:00 & 2004-12-15 20:00 & -71.0 & 26.0 & 14.9 & 71.2 & 683.1 & 1343.6 & 28.1 \\
    2006-01-26 15:00 & 2006-02-06 06:00 & -112.6 & 8.2 & 18.0 & 26.2 & 188.2 & 740.4 & 33.5 \\
    2006-12-17 15:00 & 2006-12-26 09:00 & -64.7 & 4.0 & 1.6 & 42.3 & 527.5 & 1065.7 & 18.8 \\
    2007-12-17 16:00 & 2007-12-24 08:00 & -75.1 & 15.6 & 24.7 & 82.4 & 622.4 & 1383.0 & 29.5 \\
    2009-01-02 15:00 & 2009-01-10 08:00 & -82.9 & 0.8 & 34.5 & 59.9 & 588.2 & 1217.3 & 20.7 \\
    2010-01-14 06:00 & 2010-01-26 21:00 & -82.5 & -3.6 & 14.1 & 31.0 & 379.0 & 680.3 & 12.8 \\
    2013-01-08 14:00 & 2013-01-16 23:00 & -102.3 & -6.3 & 16.3 & 50.0 & 562.6 & 905.4 & 19.5 \\
    2015-01-19 06:00 & 2015-01-22 09:00 & -108.7 & -11.1 & 30.6 & 89.2 & 1080.6 & 2386.3 & 42.9 \\
    2016-01-18 16:00 & 2016-01-20 17:00 & -131.9 & 15.0 & 55.9 & 89.6 & 916.2 & 3591.0 & 12.1 \\
    2017-01-16 01:00 & 2017-01-25 10:00 & -76.2 & 18.7 & 26.4 & 57.6 & 394.7 & 822.4 & 22.3 \\
    2019-01-20 15:00 & 2019-01-24 21:00 & -85.5 & 0.6 & 37.1 & 100.7 & 970.2 & 1929.0 & 38.4 \\
    \bottomrule
\end{tabular}


\caption{Key metrics for all identified system-defining events. The anomalies (to the mean for 1980--2020) for wind power production, solar production, and load are hourly averages in GW, and the values for transmission and the different storage technologies are hourly averages for shadow prices of congestion (in EUR/MW) and value of stored energy (in EUR/MWh).}
\end{table}

\subsection{Examples of a system-defining event}
% Figure environment removed

% Figure environment removed

% Figure environment removed

\subsection{Load shedding provides an alternative method to shadow prices}

% Figure environment removed

% Figure environment removed

% Figure environment removed

% Figure environment removed




% \bibliographystyle{naturemag}
% \bibliography{references.bib}


% \end{document}
