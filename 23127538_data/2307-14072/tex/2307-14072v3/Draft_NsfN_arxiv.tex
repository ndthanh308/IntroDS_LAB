\documentclass[reprint,aps,english,twocolumn,notitlepage]{revtex4-2}
\usepackage[utf8]{inputenc}
\usepackage[T1]{fontenc}
\usepackage{babel}
\usepackage{epsfig,epsf,psfrag}
\usepackage{pslatex}
\usepackage{epic,eepic}
\usepackage{float}
\usepackage[colorlinks=true,allcolors=blue]{hyperref}
\usepackage{chngcntr}
\usepackage{color}
\usepackage{graphicx,wrapfig,setspace}
\usepackage[caption=false]{subfig} % loads caption3
\usepackage{ragged2e} % for the \justifying macro
\DeclareCaptionJustification{justified}{\justifying}
\usepackage{pstricks}
\usepackage{mathtools}
\usepackage{amssymb}
\usepackage{amsmath}
\usepackage{txfonts}
\usepackage{centernot}
\usepackage{multirow}
\usepackage[normalem]{ulem}
\usepackage{natbib}
\usepackage[edges]{forest}
\usetikzlibrary{arrows.meta}
% \usepackage{forest}
\usepackage{tikz-qtree}
%\usepackage{floatrow}
\usepackage{fancyhdr,hyperref}
\pagestyle{fancy}
\parindent0em


\begin{document}
\title{Spin-flip scattering engendered negative $\Delta_T$ noise}
\author{Tusaradri Mohapatra}
\author{Colin Benjamin}\email{colin.nano@gmail.com}
 \affiliation{School of Physical Sciences, National Institute of Science Education \& Research, Jatni-752050, India, \\ Homi Bhabha National Institute, Training School Complex, Anushaktinagar, Mumbai 400094, India}

\begin{abstract}
$\Delta_T$ noise generated due to a temperature gradient in the absence of charge current has recently attracted a lot of interest. In this paper, for the first time, we derive spin-polarised charge $\Delta_T$ noise and spin $\Delta_T$ noise along with its shot noise-like and thermal noise-like contributions. Introducing a spin flipper at the interface of a bilayer metallic junction with a temperature gradient, we examine the impact of spin-flip scattering via $\Delta_T$ noise auto-correlation. We ensure that the net charge or spin current transported is always zero. We find that charge $\Delta_T$ noise is negative. In contrast, spin $\Delta_T$ noise is positive. Spin-flip scattering exhibits the intriguing effect of a change in sign in charge $\Delta_T$ noise, which can help probe spin-polarised transport. Both charge and spin $\Delta_T$ noise depend on the sign and characteristics of the inherent spin correlation, i.e., same-spin correlation or opposite-spin correlation. The change in the sign of the charge $\Delta_T$ noise is induced by the opposite-spin correlation contribution to $\Delta_T$ thermal noise, while $\Delta_T$ shot noise is always positive.
\end{abstract}

\maketitle

\section{Introduction}
Quantum noise~\cite{qnoise} has been used to probe many facets of electron transport, like current-current correlations\cite{II}, fractional charges\cite{fraction}, wave-particle duality, and also serves as an entanglement detector. There are two distinct contributions to electronic noise in mesoscopic devices: thermal noise and shot noise, which have differing origins~\cite{noise,noises}. At equilibrium, the temporal current fluctuations due to the thermal agitation of electrons is referred to as thermal noise, also known as Johnson-Nyquist noise~\cite{thermalnoise}. Thermal noise vanishes at zero temperature and is related to the conductance of the system. The non-equilibrium temporal current fluctuations lead to shot noise which exists even at zero temperature. Shot noise arises due to the discreteness of the particles and can be used as an entanglement detector~\cite{NSN}, and to probe wave-particle duality~\cite{qnoise, shotnoise}. Shot noise is crucial in providing more detailed information on quantum transport compared to the conductance~\cite{thermalnoise}. 

$\Delta_T$ noise arises due to a temperature gradient in non-equilibrium systems~\cite{dTtheory,generalbound,atomicscaleexpt}. This intriguing noise, in a non-equilibrium situation, measured in recent experiments has generated much interest from both theoretical~\cite{dTtheory,popoff,generalbound,qint} and experimental perspectives~\cite{atomicscaleexpt,qcircuitexpt,tjunctionexpt}. It has been experimentally observed across nanoscale conductors~\cite{atomicscaleexpt},  in elementary quantum circuits~\cite{qcircuitexpt}, and metallic tunnel junctions~\cite{tjunctionexpt}. General bounds for charge $\Delta_T$ noise have been found in Ref.~\cite{generalbound}; which show that the shot noise-like contribution to charge $\Delta_T$ noise, i.e., $\Delta_{Tsh}$ is less than or equal to thermal noise contribution to $\Delta_T$ noise, i.e., $\Delta_{Tth}$ or $ \Delta_{Tsh}/ \Delta_{Tth} \leq 1$. This general bound holds for both charge $\Delta_T$ noise (charge current-current auto-correlations in the absence of charge current) and spin $\Delta_T$ noise (spin current-current auto-correlations in the absence of spin current) ~\cite{generalbound,generalbound1}.  {Sometimes in experimental setups, unintentional temperature differences can lead to abrupt noise increases, which might be mistaken for noise arising from interactions between the charge carriers or from other subtle effects, see \cite{atomicscaleexpt}. This can be effectively examined and understood using $\Delta_T$ noise, as it is a versatile probe without any particular design limitation \cite{qcircuitexpt}. Further, as it is not limited to a specific temperature range, $\Delta_T$ noise can be applied to conductors of different sizes down to the atomic scale \cite{atomicscaleexpt, qcircuitexpt}.}

Recent works have proposed that the sign of charge $\Delta_T$ shot noise contribution can reveal quantum statistics \cite{qstat}, being positive for fermions and either positive or negative for bosons. Sign of the charge $\Delta_T$ noise is generically determined by the nature of the dominant tunneling process along with quantum statistics as discussed in Refs. \cite{qstat, qint}, which we also see in this paper. Further, we see that charge and spin $\Delta_T$ shot noise contribution is always positive for electrons regardless of spin-flip scattering. However, apart from statistics, the sign of charge (or spin) $\Delta_T$ noise is determined by the sign of the dominating charge (or spin) $\Delta_T$ thermal noise contribution. In contrast to $\Delta_T$ noise, for quantum noise, the sign of equilibrium quantum thermal noise remains independent of quantum statistics~\cite{noise}, while quantum shot noise cross-correlation can probe quantum statistics~\cite{noise, shotnoise}.

In this paper, we study charge and spin $\Delta_T$ noise in the presence of spin-flip scattering. Spin-flip scattering occurs when an incident electron interacts with an interfacial spin flipper. Spin current is of great importance in spintronics, which motivates us to study both spin and charge $\Delta_T$ noise. We look at the effect of spin-flip scattering at vanishing charge current via charge $\Delta_T$ noise and at vanishing spin current via spin $\Delta_T$ noise along with their respective $\Delta_T$ shot noise-like contribution and $\Delta_T$ thermal noise-like contributions. Apart from a single work in a spin-biased setup \cite{generalbound1}, in which transmission is spin-independent, there is a complete absence of published works on spin $\Delta_T$ noise with spin-polarised transmission. We in this work are the first to calculate spin $\Delta_T$ noise for a junction with spin-polarised transmission, which includes spin $\Delta_T$ thermal and shot noise contributions along with the same-spin and opposite-spin correlations. 

Further, for the first time, we provide a more general approach to the study of charge and spin $\Delta_T$ noise, with finite temperature gradient for reservoirs at comparable temperatures. We also calculate charge $\Delta_T$ noise in the absence of any spin-flip scattering, wherein it is always positive. Previously, there have been a few works on calculating $\Delta_T$ noise. One such work focused on a configuration with one hot and one cold reservoir in \cite{generalbound} with two generic conductors. Another work considers reservoirs with comparable temperatures, akin to our setup, in~\cite{popoff} with two normal metals separated by a quantum dot. In all these works, spin-dependent transport is absent, and only charge $\Delta_T$ noise is calculated. One should note that no one has calculated spin polarised charge and spin $\Delta_T$ noise yet.

The main findings of this work are: 

[1] We see that both charge and spin $\Delta_T$ noise reveal information about the sign and nature of the dominant spin correlation, either opposite-spin or same-spin correlation. 

[2] Negative charge $\Delta_T$ noise arises due to the dominant thermal noise-like contribution, which is negative. As far as we know, negative charge $\Delta_T$ noise or negative charge $\Delta_T$ thermal noise contribution for spin-polarised transport has not been seen in any system. However, negative charge $\Delta_T$ thermal noise-like contribution has been predicted in Ref.~\cite{popoff}.  

[3] Due to the finite temperature gradient, opposite-spin charge thermal noise correlation turns negative, thus, charge $\Delta_{T}$ noise also turns negative. This is in contrast to the total charge quantum noise auto-correlation, which is always positive. 

[4] Moreover, we observe charge and spin $\Delta_T$ shot noise to be always positive as also predicted in Ref. \cite{qstat} for fermions, because of the contributions from both same-spin and opposite-spin shot noise correlation.

% We also find that the contribution from opposite-spin correlation to charge and spin $\Delta_{T}$ shot noise is always negative \cite{spincor}, due to interaction among electrons of opposite spins, depending on the finite temperature gradient.}

{[5] Interestingly charge (spin) $\Delta_{T}$ thermal noise is either greater than or equal to charge (spin) $\Delta_{T}$ shot noise in our bilayer metallic junction with spin-flip scattering, establishing a general bound in case of charge (spin) $\Delta_{T}$ noise, while in case of quantum noise, there is no such bound on quantum shot and thermal noise \cite{noise,noise1}.}
% Notably, we also see that in our bilayer metallic junction with spin-flip scattering, both charge $\Delta_T$ noise and spin $\Delta_T$ noise obey this general bound, i.e., the magnitude of charge (spin) shot noise-like contribution is always less than charge (spin) thermal noise-like contribution to charge (spin) $\Delta_T$ noise.

The layout of the paper is as follows: in section~\ref{theory}, we first discuss our chosen setup metal($N_1$)/spin-flipper(SF)/metal($N_2$) briefly, then in subsection~\ref{spinflip}, we explain the phenomena of spin-flip scattering by introducing a spin-flipper at the metal/metal interface. Subsequently, we discuss finite temperature charge and spin quantum noise followed by spin-polarised $\Delta_T$ noise. In section~\ref{results}, we discuss charge (spin) $\Delta_T$ noise and the ratio of charge (spin) $\Delta_T$ shot noise to charge (spin) $\Delta_T$ thermal noise contribution and investigate the impact of spin-flip scattering in a bilayer metallic junction. Next, in section~\ref{analysis}, we analyze via a table the results for the charge as well as spin $\Delta_T$ noise. We also discuss same-spin and opposite-spin correlation contributions to charge and spin $\Delta_T$ noise, followed by the experimental realization and the conclusions to our work in section~\ref{conclusion}. The theoretical calculations of the current, and thermovoltage needed to calculate $\Delta_T$ noise are given in Appendix \ref{App_I}, followed by the calculation of spin polarised quantum noise and both charge and spin $\Delta_T$ noise in Appendix \ref{App_Qn} and Appendix \ref{App_DT} respectively. 


\section{Theory}
\label{theory}

% Figure environment removed

In Fig.~\ref{fig:NsfN}, a 1D bilayer metallic junction consisting of a spin magnetic impurity at the interface $x=0$ is shown. When a spin-up or spin-down electron from left normal metal $N_1$ is incident on the interfacial spin-flipper ($x=0$), there can be mutual spin-flip, and the electron may be reflected to normal metal $N_1$ or transmitted to normal metal $N_2$ as a spin-down or spin-up electron. The Hamiltonian of our bilayer metallic junction with a spin-flipper~\cite{spinflipper} at the interface is,
\begin{equation}
H= \frac{p^2}{2 m^*}- H_{sf},
\label{Eq:H}
\end{equation}
where $H_{sf}=J_0 \delta(x) \Vec{s} \cdot \Vec{S}$ with effective mass $m^*$ and $J_0$ being the relative strength of exchange interaction between electron spin $\Vec{s}$ and magnetic impurity spin $\Vec{S}$. To compare with the case of no spin-flip scattering, we replace the spin-flipper with a delta potential $\delta(x)$ at the interface $x=0$.  The exchange interaction in $H_{sf}$ is given as 
\begin{eqnarray}
\Vec{s} \cdot \Vec{S}= s_z \cdot S_z+ \frac{1}{2} (s^- S^+ + s^+ S^-),
\end{eqnarray}
where $s^{\pm}=s_x \pm i s_y$ and $S^{\pm} = S_x \pm i S_y$ are raising and lowering operators for electron's spin and spin-flipper's spin. $s_x,s_y, s_z$, and $S_x, S_y, S_z$ are $x,y,z$ components of the electron's spin operator and spin-flipper's spin operator, respectively.

\subsection{Spin-flip scattering}
\label{spinflip}

The wave functions in $N_1$ and $N_2$ region, for spin-up electron incident from $N_1$ can be written as (see, Fig.~\ref{fig:NsfN}),
\begin{align}
\psi_{N_1}(x) &= \left( \begin{array}{c}
1\\
0\\
\end{array} \right) (e^{i k x } + r^{\uparrow \uparrow} e^{-i k x} ) \phi^s_m +
r^{\uparrow \downarrow} \left( \begin{array}{c}
0\\
1\\
\end{array} \right) e^{-i k x} \phi^s_{m+1} ,~ \text{for} \hspace{0.08cm} x < 0, \nonumber \\
\psi_{N_2}(x) &= t^{\uparrow \uparrow} \left( \begin{array}{c}
1\\
0
\end{array} \right) e^{i k x} \phi^s_{m} +
t^{\uparrow \downarrow} \left( \begin{array}{c}
0\\
1
\end{array} \right) e^{-i k x} \phi^s_{m+1} ,~\text{ for} \hspace{0.09cm} x > 0,
\label{eqn:wf}
\end{align}

where $\phi^s_m$ is the eigenfunction of $S_z$ such that $S_z \phi^s_m = m \phi^s_m$, with $m$ being spin magnetic moment and $k$ being the wave vector of electron, i.e., $k = \sqrt{\frac{2 m^*}{\hbar^2} ( E_F + E )} = k_F \sqrt{( 1 + \frac{E}{E_F} )}$ with energy of incident electron $E>0$, and $k_F=\sqrt{\frac{2 m^* E_F}{\hbar^2}}$ is Fermi wave vector with $E_F$ being the Fermi energy. Reflection amplitudes for an incident spin-up electron to be reflected as spin-up electron is denoted as $r^{\uparrow \uparrow}$ and to be reflected as spin-down electron is denoted as $r^{\uparrow \downarrow}$. $t^{\uparrow \uparrow}$ is the transmission amplitude for an incident spin-up electron to be transmitted as spin-up, and $t^{\uparrow \downarrow}$ is the transmission amplitude for an incident spin-up electron to be transmitted as spin-down.

The electron spin and spin-flipper spin operators $\Vec{s}$ and $\Vec{S}$ operating on the spin-up electron spinor~\cite{spinflipper} and the spin-flipper eigen function gives,
\begin{eqnarray}
\Vec{s}. \Vec{S} \left( \begin{array}{c}
1\\
0\\
\end{array} \right) \phi^s_m = \frac{m}{2} \left( \begin{array}{c}
1\\
0\\
\end{array} \right) \phi^s_{m} + \frac{\tau}{2} \left( \begin{array}{c}
0\\
1\\
\end{array} \right) \phi^s_{m+1}.
\label{sfu}
\end{eqnarray}
Similarly, $\Vec{s}. \Vec{S}$ acting on the spin-down electron spinor and the spin-flipper eigen function gives,
\begin{eqnarray}
\Vec{s}. \Vec{S} \left( \begin{array}{c}
0\\
1\\
\end{array} \right) \phi^s_{m} = -\frac{m}{2} \left( \begin{array}{c}
0\\
1\\
\end{array} \right) \phi^s_{m} + \frac{\tau_1}{2} \left( \begin{array}{c}
1\\
0\\
\end{array} \right) \phi^s_{m-1},
\label{sfd}
\end{eqnarray}
where $\tau=\sqrt[]{(S-m)(S+m+1)}$, $\tau_1=\sqrt[]{(S+m)(S-m+1)}$ are spin-flip probabilities for up and down spin electron incident at left normal metal, with $S$ being the spin-flipper's spin.

% Figure environment removed
When the electron's elastic scattering time is much greater than the spin-flipper's relaxation time $\tau_e \gg \tau_{sf}$, the spin-flipper will flip back before encountering the next incident electron, see Fig.~\ref{fig:avgm}. Therefore, spin magnetic moment $m$ for a specific spin-flipper's spin $S$ is not fixed, and to calculate any transport quantity, the average over all possible $m$ values is taken. We consider the cases where a spin-up incident electron interacts in either spin-configuration 1 ($S=m$) or spin-configuration 2 ($S \neq m$). Likewise, for spin-down incident electron, the spin-flipper interacts via spin-configuration 3 ($S \neq -m$) or spin-configuration 4 ($S=-m$), as shown in Fig. \ref{fig:avgm}.

The beauty of the spin-flip scattering approach as pioneered in Ref. \cite{spinflipper} is that a two-body problem (electron scattering from a static spin flipper) can be addressed effectively via a single particle Schrodinger Hamiltonian. The spin-flipper's behavior is not governed by time-dependent dynamics nor any many-body approach. One can mistake the spin-flipper in our N$_1$/SF/N$_2$ junction to be like a Kondo impurity. However, it should be noted that the Kondo effect arises when the impurity's localized magnetic moment interacts with the conduction electrons, leading to the formation of a Kondo spin-singlet state in a time-dependent many-body picture \cite{kondo}. Unlike the Kondo effect, in our work spin-flipper impurity \cite{spinflip,spinflipper} is analyzed in single particle non-time dependent picture. In our study, therefore, there is no Kondo-like behaviour.


The boundary conditions at the interface $x=0$ are (see, Fig~\ref{fig:NsfN}),
\begin{eqnarray}
\left. \psi_{N1}(x) \right|_{x=0} &=& \left. \psi_{N2}(x) \right|_{x=0} , \nonumber \\
\left. \frac{d \psi_{N2}}{dx} \right|_{x=0} - \left. \frac{d \psi_{N1}}{dx} \right|_{x=0} &=& \left. \frac{-2 m^* J_0 \Vec{s} \cdot \Vec{S}}{\hbar^2} \psi_{N1} \right|_{x=0}.
\label{BC}
\end{eqnarray}

Substituting Eq.~(\ref{eqn:wf}) in Eq.~(\ref{BC}) and using Eqs.~(\ref{sfu}) and~(\ref{sfd}), we get:
\begin{eqnarray}
&& r^{\uparrow \uparrow}- t^{\uparrow \uparrow} =-1, ~~~~~~ r^{\uparrow \downarrow }- t^{\uparrow \downarrow } =0, \nonumber \\
&& (1 - i J^{\prime} m)r^{\uparrow \uparrow} - i J^{\prime} \tau r^{\uparrow \downarrow} + t^{\uparrow \uparrow} = 1 + i J^{\prime} m, \nonumber \\
&& -i J^{\prime} \tau r^{\uparrow \uparrow} + (1+ i J^{\prime} (m+1)) r^{\uparrow \downarrow } + t^{\uparrow \downarrow} = i J^{\prime} \tau,
\label{sup}
\end{eqnarray}
where $J^{\prime}= J/\sqrt{( 1 + \frac{E}{E_F} )}$, and the parameter denoting the exchange interaction strength $J$ is dimensionless, i.e., $J=\frac{m^* J_0}{\hbar^2 k_F}$.  

Solving Eq.~(\ref{sup}), we get normal reflection amplitude for no-flip ($r^{\uparrow \uparrow} = s^{\uparrow \uparrow}_{11} $), normal reflection amplitude for spin-flip ($r^{\uparrow \downarrow} = s^{\uparrow \downarrow}_{11}$), transmission amplitude for no-flip ($t^{\uparrow \uparrow} = s^{\uparrow \uparrow}_{12}$) and transmission amplitude for spin-flip ($t^{\uparrow \downarrow} = s^{\uparrow \downarrow}_{12}$) scattering. Reflection probabilities for spin-up incident electron for no-flip and spin-flip are $\mathcal{R}^{\uparrow \uparrow}=|r^{\uparrow \uparrow}|^2$ and $\mathcal{R}^{\uparrow \downarrow}=|r^{\uparrow \downarrow}|^2$. Transmission probabilities for spin-up incident electron for no-flip and spin-flip are $\mathcal{T}^{\uparrow \uparrow}=|t^{\uparrow \uparrow}|^2$ and $\mathcal{T}_{\uparrow \downarrow}=|t^{\uparrow \downarrow}|^2$. The scattering probabilities for a spin-up incident electron can be simplified as follows:
\begin{eqnarray}
\mathcal{R}^{\uparrow \uparrow} = |s^{\uparrow \uparrow}_{11}|^2 &=& \frac{ \tau^2 J^{\prime 4} ( \tau^2 + 2 m (1 + m)) + J^{\prime 2} m^2 (4 + J^{\prime 2} (1 + m)^2)}{Dn}, \nonumber \\
~~ \mathcal{R}^{\uparrow \downarrow} = |s^{\uparrow \downarrow}_{11}|^2 &=& \frac{4 \tau^2 J^{\prime 2} }{Dn}, ~~~\mathcal{T}^{\uparrow \uparrow} = |s^{\uparrow \uparrow}_{12}|^2 = \frac{ 4 ( 4 + J^{\prime 2} (1 + m)^2 ) }{Dn}, \nonumber \\
~~\mathcal{T}^{\uparrow \downarrow} = |s^{\uparrow \downarrow}_{12}|^2 &=& \frac{4 \tau^2 J^{\prime 2}}{Dn},
% s^{\uparrow \uparrow}_{11} &=& \frac{J^{\prime} \left( \tau^2 J^{\prime} + m \left(J^{\prime}- 2 i + J^{\prime} m \right) \right)}{Dn},~~
% s^{\uparrow \downarrow}_{11}= \frac{2 i \tau J^{\prime} }{Dn}, \nonumber \\
% s^{\uparrow \uparrow}_{12} &=& \frac{ 2 i \left( J^{\prime} + J^{\prime} m - 2 i \right)}{Dn}, ~~
% s^{\uparrow \downarrow}_{12}= \frac{2 i \tau J^{\prime}}{Dn}, ~~ Dn=4 + J^{\prime} \left( 2 i + J^{\prime} \left( \tau^2 + m + m^2 \right) \right),
\label{Prob_up}
\end{eqnarray}
where $Dn= 4 J^{\prime 2} + (4 + J^{\prime 2} (\tau^2 + m + m^2))^2$. 

Similarly, reflection probabilities for spin-down incident electron for no-flip and spin-flip are $\mathcal{R}^{\downarrow \downarrow}=|r^{\downarrow \downarrow}|^2$ and $\mathcal{R}^{\downarrow \uparrow}=|r^{\downarrow \uparrow}|^2$. Transmission probabilities for spin-down incident electron for no-flip and with spin-flip are $\mathcal{T}^{\downarrow \downarrow}=|t^{\downarrow \downarrow}|^2$ and $\mathcal{T}^{ \downarrow \uparrow}=|t^{\downarrow \uparrow}|^2$. The scattering probabilities for a spin-down incident electron can be simplified as follows:
\begin{eqnarray}
\mathcal{R}^{\downarrow \downarrow} = |s^{\downarrow \downarrow}_{11}|^2 &=& \frac{ \tau^2_1 J^{\prime 4} ( \tau^2_1 + 2 (-1 + m) m) + J^{\prime 2} (4 + J^{\prime 2} (-1 + m)^2) m^2}{Dn1}, \nonumber \\
\mathcal{R}^{\downarrow \uparrow} = |s^{ \downarrow \uparrow}_{11}|^2 &=& \frac{4 \tau^2_1 J^{\prime 2} }{Dn1}, ~~
\mathcal{T}^{\downarrow \downarrow} = |s^{\downarrow \downarrow}_{12}|^2 = \frac{4 (4 + J^{\prime 2} (-1 + m)^2)}{Dn1}, \nonumber \\
\mathcal{T}^{\downarrow \uparrow} = |s^{\downarrow \uparrow}_{12}|^2 &=& \frac{4 \tau^2_1 J^{\prime 2}}{Dn1},
% s^{\downarrow \downarrow}_{11}| &=& \frac{ J^{\prime} \left( J^{\prime} \tau^2_1 + m \left(- J^{\prime} + 2 i + J^{\prime} m \right) \right)}{Dn1},~~
% s^{ \downarrow \uparrow}_{11}| = \frac{2 i \tau_1 J^{\prime} }{Dn1}, \nonumber \\
% s^{\downarrow \downarrow}_{12} &=& \frac{ 2 i \left( J^{\prime} - J^{\prime} m - 2 i \right) }{Dn1}, ~s^{\downarrow \uparrow}_{12}= \frac{2 i \tau_1 J^{\prime} }{Dn1}, ~ Dn1= 4 + J^{\prime} \left( 2 i + J^{\prime} \tau^2_1 - J^{\prime} m + J^{\prime} m^2 \right)
\label{Prob_dn}
\end{eqnarray}
where $Dn1= 4 J^{\prime 2} + (4 + J^{\prime 2} (\tau_1^2 + (-1 + m) m))^2$. In case of NIN, without any spin-slip scattering ($s^{ \downarrow \uparrow}_{11}=s^{\uparrow \downarrow}_{11}=0$), reflection and transmission probabilities are expressed as $\mathcal{R}^N=|s^N_{11}|^2=  J^{\prime 2} / \left( 1 - J^{\prime 2} \right)$ and $\mathcal{T}^N=|s^N_{12}|^2 = 1 / \left(1 - J^{\prime 2} \right)$.

In the following subsection, we calculate the charge and spin current, the quantum charge noise, and quantum spin noise due to spin-flip scattering in N$_1$/SF/N$_2$ junction as shown in Fig~\ref{fig:NsfN}.


\subsection{Charge and spin current}
\label{current}


For the setup as given in Fig.~\ref{fig:NsfN}, the average spin polarised current can be written as~\cite{datta,noise},
\begin{eqnarray}
\langle I^{\sigma}_{\alpha} \rangle &=& \frac{e}{h} \int^{\infty}_{-\infty} \sum_{\substack{\gamma, \gamma^{\prime} \in \{1,2\}; \\ \rho, \rho^{\prime} \in \{\uparrow, \downarrow\}}} \langle a^{\rho \dagger}_{\gamma} a^{\rho^{\prime}}_{\gamma^{\prime}} \rangle A^{\rho \rho^{\prime}}_{\gamma^{\prime} \gamma}(\alpha,\sigma,E) dE ,
\end{eqnarray}

where $A^{\rho \rho^{\prime}}_{\gamma^{\prime} \gamma}(\alpha,\sigma,E) = \delta_{\gamma \alpha} \delta_{\gamma^{\prime} \alpha} \delta_{\sigma \rho} \delta_{\sigma \rho^{\prime}} - s^{\sigma \rho \dagger}_{\alpha \gamma} s^{\sigma \rho^{\prime}}_{\alpha \gamma^{\prime}} $, with $\alpha, \gamma^{\prime}$ and $\gamma$ indices labelling normal metals $1$ and $2$. The indices $\sigma$, $\rho^{\prime}$, and $\rho$ label the spin of an electron, i.e., spin-up electron ($\uparrow$) or spin-down electron ($\downarrow$). $a^{\rho \dagger}_{\gamma}$ ($a^{\rho^{\prime}}_{\gamma^{\prime}}$) are electron creation (annihilation) operators in normal metal $\gamma (\gamma^{\prime})$ with spin $\rho (\rho^{\prime})$. $s^{\sigma \rho}_{\alpha \gamma}$ represents scattering amplitude for electron incident from normal metal $\alpha$ with spin $\sigma$ and to be scattered into normal metal $\gamma$ with spin $\rho$. The quantum statistical average of electron creation and annihilation operator for fermions using Wick's theorem can be simplified as $\langle a^{\rho \dagger}_{\gamma} a^{\rho^{\prime}}_{\gamma^{\prime}} \rangle = \delta_{\gamma \gamma^{\prime}} \delta_{\rho \rho^{\prime}} f^{\rho}_{\gamma}$, see Ref~\cite{noise}. In this manuscript, the Fermi function is independent of spin. Thus, $f^{\rho}_{\gamma} =f_{\gamma} = \left[ 1+ e^{\frac{E - V_{\gamma}}{k_B T_{\gamma}}} \right]^{-1}$, $k_B$ is the Boltzmann constant, $T_{\gamma}$ is temperature, and $V_{\gamma}$ is applied bias in normal metal $\gamma$. $I^{\uparrow}_{1}$ is the spin-up electronic current while $I^{\downarrow}_{1}$ is the spin-down electronic current. The average charge current in normal metal $N_1$ is $I^{ch}_1 = \langle I^{\uparrow}_1\rangle + \langle I^{\downarrow}_1 \rangle$, given in Appendix \ref{App_I}.
\begin{widetext}
Before proceeding to calculate current-current correlations, we need to calculate the average charge current in $N_1$, i.e.,
\begin{eqnarray}
&&~~ I^{ch}_{1} = \frac{e}{h} \sum_{\sigma \in \{\uparrow, \downarrow\}} \int^{\infty}_{-\infty}\sum_{\substack{\gamma, \gamma^{\prime} \in \{1,2\}; \\ \rho, \rho^{\prime} \in \{\uparrow, \downarrow\}}} \langle a^{\rho \dagger}_{\gamma} a^{\rho^{\prime}}_{\gamma^{\prime}} \rangle A^{\rho \rho^{\prime}}_{\gamma^{\prime} \gamma}(1,\sigma,E) dE
= \frac{e}{h} \sum_{\sigma \in \{\uparrow, \downarrow\}} \sum_{\substack{\gamma, \gamma^{\prime} \in \{1,2\}; \\ \rho, \rho^{\prime} \in \{\uparrow, \downarrow\}}} \int^{\infty}_{-\infty}dE (\delta_{\gamma 1} \delta_{\gamma^{\prime} 1} \delta_{\sigma \rho} \delta_{\sigma \rho^{\prime}} - s^{\sigma \rho \dagger}_{1 \gamma} s^{\sigma \rho^{\prime}}_{1 \gamma^{\prime}}) . (\delta_{\gamma \gamma^{\prime}} \delta_{\rho \rho^{\prime}} f_{\gamma}) \nonumber \\
&& ~~~~~ = \frac{e}{h} \int^{\infty}_{-\infty}\biggl\{ \left(1 - |s^{\uparrow \uparrow}_{11}|^2 - |s^{\uparrow \downarrow}_{11}|^2 + 1 - |s^{\downarrow \uparrow}_{11}|^2 - |s^{\downarrow \downarrow}_{11}|^2 \right) f_{1} - \left( |s^{\uparrow \uparrow}_{12}|^2 + |s^{\uparrow \downarrow}_{12}|^2 + |s^{\downarrow \uparrow}_{12}|^2 + |s^{\downarrow \downarrow}_{12}|^2 \right) f_{2} \biggr\} dE \nonumber \\
% && ~~~~ = \frac{e}{h} \int^{\infty}_{-\infty}\biggl\{ \left( 1- \mathcal{R}^{\uparrow \uparrow} - \mathcal{R}^{\uparrow \downarrow} + 1 - \mathcal{R}^{\downarrow \uparrow} - \mathcal{R}^{\downarrow \downarrow} \right) f_1 - \left( \mathcal{T}^{\uparrow \uparrow} + \mathcal{T}^{\uparrow \downarrow} + \mathcal{T}^{\downarrow \uparrow} + \mathcal{T}^{\downarrow \downarrow}\right) f_2 \biggr\}dE \nonumber \\
&& ~~~~~ = \frac{e}{h} \int^{\infty}_{-\infty}\left( \mathcal{T}^{\uparrow \uparrow} + \mathcal{T}^{\uparrow \downarrow} + \mathcal{T}^{\downarrow \uparrow} + \mathcal{T}^{\downarrow \downarrow} \right) (f_1 -f_2) dE = \frac{e}{h} \int^{\infty}_{-\infty}F^{ch}_I (f_1(E) -f_2(E)) dE,
\label{Ich}
\end{eqnarray}
where $F^{ch}_{I} =$ $\mathcal{T}^{\uparrow \uparrow} +$ $\mathcal{T}^{\uparrow \downarrow}$ $+ \mathcal{T}^{\downarrow \uparrow}$ $+ \mathcal{T}^{\downarrow \downarrow} $, where  $\mathcal{T}^{\sigma \sigma^{\prime}}=|s^{\sigma \sigma^{\prime}}_{12}|^2$, with $\sigma,\sigma^{\prime}= \{\uparrow,\downarrow\}$. $F^{ch}_{I}$, and $\mathcal{T}^{\sigma \sigma'}$ being functions $J$, $S$, $m$ and since spin-flip scattering is symmetric, one sees $s^{\sigma \rho}_{\alpha \gamma} = s^{\rho \sigma}_{\gamma \alpha}$.
\end{widetext}
By probability conservation for a spin-up electron incident from left normal metal, $\mathcal{R}^{\uparrow \uparrow} + \mathcal{R}^{\uparrow \downarrow} +\mathcal{T}^{\uparrow \uparrow} +\mathcal{T}^{\uparrow \downarrow} =1$ and similarly for a spin-down electron incident from left normal metal, $\mathcal{R}^{\downarrow \downarrow}+ \mathcal{R}^{\downarrow \uparrow}+\mathcal{T}^{\downarrow \downarrow} +\mathcal{T}^{\downarrow \uparrow} =1$.

The average spin current $I^{sp}_1 = \langle I^{\uparrow}_1 \rangle - \langle I^{\downarrow}_1 \rangle$ is given as~\cite{spinnoise},
\begin{eqnarray}
I^{sp}_1 &=& \frac{e}{h} \int^{\infty}_{-\infty}\left( \mathcal{T}^{\uparrow \uparrow} - \mathcal{T}^{\uparrow \downarrow} - \mathcal{T}^{\downarrow \uparrow} + \mathcal{T}^{\downarrow \downarrow} \right) (f_1 -f_2 ) dE \nonumber \\
&=& \frac{e}{h} \int^{\infty}_{-\infty}F^{sp}_I (f_1(E) -f_2 (E) ) dE,
\label{Isp}
\end{eqnarray}

where $F^{sp}_I  =$ $\mathcal{T}^{\uparrow \uparrow}$ $- \mathcal{T}^{\uparrow \downarrow}$ $- \mathcal{T}^{\downarrow \uparrow} $ $+ \mathcal{T}^{\downarrow \downarrow}$, with $F^{sp}_{I}$ being functions of $J$, $S$, and $m$. For a NIN junction without spin-flip scattering, $I^N= \frac{2e}{h} \int^{\infty}_{-\infty}F^{N}_I (f_1(E) -f_2(E)) dE$, and $F^N_I =\mathcal{T}^N=|s^N_{12}|^2 $ and $|s^N_{11}|^2+|s^N_{12}|^2=1$. 


\subsection{Quantum noise}
\label{qnoise}


Quantum noise auto(cross) correlation defines the correlation between the current in normal metal $N_1$ and the current in normal metal $N_1$ ($N_2$). Charge noise correlations between the charge current in normal metals $\alpha$ and $\beta$ at time $t$ and $t^{\prime}$ is defined as $S_{\alpha \beta}(t-t^{\prime}) \equiv \langle \Delta I_{\alpha}(t) \Delta I_{\beta}(t^{\prime}) + \Delta I_{\beta}(t^{\prime}) \Delta I_{\alpha}(t) \rangle $, where $\Delta I_{j}(t)= I_{j}(t) - \langle I_{j}(t) \rangle$, $j \in \alpha, \beta$, see Ref.~\cite{datta}. Spin polarised noise correlation between spin current in normal metals $\alpha$ and $\beta$ with spin $\sigma$ and $\sigma^{\prime}$ at finite $t$ and $t^{\prime}$ is defined as $S^{\sigma \sigma^{\prime}}_{\alpha \beta}(t-t^{\prime}) \equiv \langle \Delta I^{\sigma}_{\alpha}(t) \Delta I^{\sigma^{\prime}}_{\beta}(t^{\prime}) + \Delta I^{\sigma}_{\beta}(t^{\prime}) \Delta I^{\sigma^{\prime}}_{\alpha}(t) \rangle $, where $\Delta I^{\sigma}_{j}(t)= I^{\sigma}_{j}(t) - \langle I^{\sigma}_{j}(t) \rangle$, $j \in \alpha, \beta$. Fourier transforming the spin-polarised noise correlation, we get the spin-polarised noise power~\cite{spinnoise} which can be written in terms of frequency $\omega$ and $\omega^{\prime}$ as $ 2 \pi \delta(\omega + \omega^{\prime})S^{\sigma {\sigma}^{\prime}}_{\alpha \beta}(\omega) \equiv \langle \Delta I^{\sigma}_{\alpha}(\omega) \Delta I^{\sigma^{\prime}}_{\beta}(\omega^{\prime}) + \Delta I^{\sigma}_{\beta}(\omega^{\prime}) \Delta I^{\sigma^{\prime}}_{\alpha}(\omega) \rangle $.
Thus, the charge noise power for charge current $I^{ch}_{\alpha} = \langle I^{\uparrow}_{\alpha} \rangle + \langle I^{\downarrow}_{\alpha} \rangle$ can be written as, $S^{ch}_{\alpha \beta} = S^{ \uparrow \uparrow}_{\alpha \beta} + S^{ \uparrow \downarrow}_{\alpha \beta} + S^{ \downarrow \uparrow}_{\alpha \beta} + S^{ \downarrow \downarrow}_{\alpha \beta}$, while the spin noise power for spin current $I^{sp}_{\alpha} = \langle I^{\uparrow}_{\alpha} \rangle - \langle I^{\downarrow}_{\alpha} \rangle$ is $S^{sp}_{\alpha \beta} = S^{ \uparrow \uparrow}_{\alpha \beta} - S^{ \uparrow \downarrow}_{\alpha \beta} - S^{ \downarrow \uparrow}_{\alpha \beta} + S^{ \downarrow \downarrow}_{\alpha \beta}$. 

The general expression for zero frequency spin-polarised noise power correlation can be derived as (see, Eq. (\ref{B_Snoise}) in Appendix \ref{App_Qn}),

\begin{widetext}

\begin{eqnarray}
S^{\sigma \sigma^{\prime}}_{\alpha \beta} &=& \frac{e^2}{2 \pi \hbar} \int^{\infty}_{-\infty} \sum_{\substack{\gamma, \gamma^{\prime} \in \{1,2\}; \\ \rho, \rho^{\prime} \in \{\uparrow, \downarrow\} }} \biggl[ \left\{ \langle a^{\sigma \dagger}_{\alpha} a^{\sigma^{\prime}}_{\beta} a^{\rho \dagger}_{\gamma} a^{\rho^{\prime}}_{\gamma^{\prime}} \rangle - \langle a^{\sigma \dagger}_{\alpha} a^{\sigma^{\prime}}_{\beta} \rangle \langle a^{\rho \dagger}_{\gamma} a^{\rho^{\prime}}_{\gamma^{\prime}} \rangle \right\} A^{\sigma^{\prime} \sigma}_{\alpha \beta}(\alpha,\sigma,E) A^{\rho^{\prime} \rho}_{\gamma^{\prime} \gamma}(\beta,\sigma^{\prime},E) + \left\{ \langle a^{\sigma^{\prime} \dagger}_{\beta} a^{\sigma}_{\alpha} a^{\rho^{\prime} \dagger}_{\gamma^{\prime}} a^{\rho}_{\gamma} \rangle - \langle a^{\sigma^{\prime} \dagger}_{\beta} a^{\sigma}_{\alpha} \rangle \langle a^{\rho^{\prime} \dagger}_{\gamma^{\prime}} a^{\rho}_{\gamma} \rangle \right\} \nonumber \\
&& \times A^{\sigma \sigma^{\prime}}_{\beta \alpha}(\beta,\sigma^{\prime},E) A^{\rho \rho^{\prime}}_{\gamma \gamma^{\prime}}(\alpha,\sigma,E) \biggr] dE ~ = ~ \frac{e^2}{2 \pi \hbar} \int^{\infty}_{-\infty} \sum_{\substack{\gamma, \gamma^{\prime} \in \{1,2\}; \\ \rho, \rho^{\prime} \in \{\uparrow, \downarrow\}}}  A^{\rho \rho^{\prime}}_{\gamma \gamma^{\prime}}(\alpha,\sigma,E) A^{\rho^{\prime} \rho}_{\gamma^{\prime} \gamma}(\beta,\sigma^{\prime},E) \left\{ f_{\gamma} (1- f_{\gamma^{\prime}}) + f_{\gamma^{\prime}} (1- f_{\gamma}) \right\} ~ dE \nonumber \\
&=& \dfrac{e^2}{2 \pi \hbar} \int^{\infty}_{-\infty} F^{\sigma \sigma^{\prime}}_{\alpha \beta th} \bigl\{ f_{1} (1- f_{1}) + f_{2} (1- f_{2}) \bigr\}  dE + \frac{e^2}{2 \pi \hbar} \int^{\infty}_{-\infty} F^{\sigma \sigma^{\prime}}_{\alpha \beta sh} ( f_{1} - f_{2})^2 dE = S^{\sigma \sigma^{\prime}}_{\alpha \beta th} + S^{\sigma \sigma^{\prime}}_{\alpha \beta sh}, 
\label{eqn:noise}
\end{eqnarray}
with $F^{\sigma \sigma^{\prime}}_{\alpha \beta th}$ and $F^{\sigma \sigma^{\prime}}_{\alpha \beta sh}$ being the sum of the scattering probabilities for shot noise and thermal noise (see, Appendix \ref{App_Qn}), and are given for $\alpha=\beta=1$, below
\begin{eqnarray}
    F^{\sigma \sigma^{\prime}}_{11th} &=&  \left\{ \delta_{\sigma \sigma^{\prime}} - | s^{\sigma \sigma^{\prime}}_{11} |^2  \right\}, ~~\text{and}~~ F^{\sigma \sigma^{\prime}}_{11sh} = \delta_{\sigma \sigma^{\prime}} \left\{ | s^{\sigma \sigma^{\prime}}_{12} |^2 ( 1 - |s^{\sigma \sigma^{\prime}}_{12} |^2 ) + | s^{\sigma \bar{\sigma}}_{12} |^2 ( 1 - |s^{\sigma \bar{\sigma}}_{12} |^2 ) \right\} + ( 1 - \delta_{\sigma \sigma^{\prime}} ) \left\{ 2 | s^{\sigma \sigma}_{12} |^2 |s^{\sigma \sigma^{\prime}}_{12}|^2 \right\}, \nonumber 
\end{eqnarray}
where spin $\{\sigma, \sigma^{\prime} \} = \downarrow$ (or $\uparrow$), then $\bar{\sigma} = \uparrow$ (or $\downarrow$).
\end{widetext}
In Eq.~(\ref{eqn:noise}), the quantum statistical expectation value of four annihilation and creation operators for fermions can be simplified as $\langle a^{\sigma \dagger}_{\alpha} a^{\sigma^{\prime}}_{\beta} a^{\rho \dagger}_{\gamma} a^{\rho^{\prime}}_{\gamma^{\prime}} \rangle - \langle a^{\sigma \dagger}_{\alpha} a^{\sigma^{\prime}}_{\beta} \rangle \langle a^{\rho \dagger}_{\gamma} a^{\rho^{\prime}}_{\gamma^{\prime}} \rangle = \delta_{\alpha \gamma^{\prime}} \delta_{\beta \gamma} \delta_{\sigma \rho^{\prime}} \delta_{\sigma^{\prime} \rho} f_{\beta}(1-f_{\alpha}) = f_{\gamma}(1-f_{\gamma^{\prime}})$, see Ref.~\cite{noise}. $S^{\sigma \sigma^{\prime}}_{\alpha \beta th}$ denotes thermal noise, and $S^{\sigma \sigma^{\prime}}_{\alpha \beta sh}$ is the shot noise contribution to spin-polarised quantum noise $S^{\sigma \sigma^{\prime}}_{\alpha \beta}$. $F^{\sigma \sigma^{\prime}}_{\alpha \beta sh}$ and $F^{\sigma \sigma^{\prime}}_{\alpha \beta th}$, are sum of the scattering probabilities for shot noise ($S^{\sigma \sigma^{\prime}}_{\alpha \beta sh}$) and thermal noise ($S^{\sigma \sigma^{\prime}}_{\alpha \beta th}$) contributions, and depend on scattering amplitudes, and their detailed expressions are given in Eqs. (\ref{B_Ssh}) and (\ref{B_Sth}) of Appendix \ref{App_Qn}. We focus on the spin-polarised charge and spin noise auto-correlation ($\alpha=\beta=1$), which is significant for spin-polarised $\Delta_T$ noise in our study, i.e., $S^{ch(sp)}_{11}$. The two contributions to charge (spin) quantum noise are identified as thermal noise $S^{ch(sp)}_{11th}$ and shot noise $S^{ch(sp)}_{11sh}$, i.e., $S^{ch(sp)}_{11}= S^{ch(sp)}_{11sh} + S^{ch(sp)}_{11th}$. 


For a NIN junction, quantum noise auto-correlation ($\alpha=\beta=1$) in the absence of spin-flip scattering  ($S^{\uparrow \downarrow}_{11}=S^{\downarrow \uparrow}_{11}=0$), can be written as $S^{N}_{11} = 2e^2/h \int^{\infty}_{-\infty}\left[ F^{N}_{11th} \bigl\{ f_{1} (1- f_{1}) + f_{2} (1- f_{2}) \bigr\} + F^{N}_{11sh} ( f_{1} - f_{2})^2 \right] dE = S^{N}_{11th} + S^{N}_{11sh}$, wherein the scattering factors in terms  transmission and reflection probabilities are  $F^{N}_{11th} = \mathcal{T}^{N}$ and $F^{N}_{11sh}=\mathcal{T}^{N} ~ \mathcal{R}^{N}$. In the next subsection, we focus on the $\Delta_T$ noise.


\subsection{$\Delta_T$ noise}
\label{dtnoise}

The noise generated via a temperature gradient in the absence of current is known as $\Delta_T$ noise~\cite{generalbound}. Spin polarised $\Delta_T$ noise is of two types: charge $\Delta_T$ noise ($\Delta^{ch}_T$) for the case of vanishing charge current and spin $\Delta_T$ noise ($\Delta^{sp}_T$) for the case of vanishing spin current. Charge $\Delta^{ch}_T$ or spin $\Delta^{sp}_T$ noise can be calculated from charge noise ($S^{ch}_{11}$) or spin noise ($S^{sp}_{11}$) for vanishing charge or spin current by expanding in a power series of $\frac{\Delta T}{2 \Bar{T}}$, with temperature difference ($\Delta T=T_1-T_2$) and average temperature $\Bar{T}= (T_1+ T_2)/2$. The expansion in $\frac{\Delta T}{2 \Bar{T}}$ is a useful tool for investigating the impact of temperature gradient on $\Delta_T$ noise, as also discussed in Refs.~\cite{popoff,dTtheory}.

The two contributions to charge $\Delta^{ch}_T$ noise are the $\Delta_T$ thermal noise contribution ($\Delta^{ch}_{Tth}$) and $\Delta_T$ shot noise contribution ($\Delta^{ch}_{Tsh}$). Thus, for charge $\Delta^{ch}_{T} = \Delta^{ch}_{Tth} + \Delta^{ch}_{Tsh}$. Similarly, for spin: $\Delta^{sp}_{T} = \Delta^{sp}_{Tth} + \Delta^{sp}_{Tsh}$, where $\Delta^{sp}_{Tth}$ is thermal noise contribution to $\Delta^{sp}_{T}$ and $\Delta^{sp}_{Tsh}$ is shot noise contribution to $\Delta^{sp}_{T}$. {Charge and spin $\Delta_T$ noise can be further dissected into their opposite-spin and same-spin correlations. Specifically, charge $\Delta_T$ noise is represented as $\Delta^{ch}_{T} = \Delta^{ch,sa}_{T} + \Delta^{ch,op}_{T}$. The opposite-spin correlation is given by $\Delta^{ch,op}_{T} = \Delta^{ch,\uparrow \downarrow}_{T} + \Delta^{ch,\downarrow \uparrow}_{T}$, and the same-spin correlation is expressed as $\Delta^{ch,sa}_{T} = \Delta^{ch,\uparrow \uparrow}_{T} + \Delta^{ch,\downarrow \downarrow}_{T}$. Their formulation enables a more detailed examination of charge and spin $\Delta_T$ noise by breaking them down into their opposite-spin and same-spin correlations. The following general formula offers a more comprehensive approach, allowing the expression of charge and spin $\Delta_T$ noise, $\Delta_T$ shot noise, and $\Delta_T$ thermal noise in terms of opposite-spin and same-spin correlations as, 
\begin{eqnarray}
    \Delta^{\beta}_{\gamma} && = \Delta^{\beta,sa}_{\gamma} + sgn(\beta)~ \Delta^{\beta,op}_{\gamma}, \nonumber \\
    \text{with}~\Delta^{\beta,sa}_{\gamma} && = \sum_{\sigma \in \{ \uparrow, \downarrow \}} \Delta^{\beta,\sigma \sigma}_{\gamma},~~\text{and}~ \Delta^{\beta,op}_{\gamma} = \sum_{\substack{\sigma, \sigma^{\prime} \in \{ \uparrow, \downarrow \}; \\  \sigma \neq \sigma^{\prime} }} \Delta^{\beta,\sigma \sigma^{\prime}}_{\gamma}, ~~~
    \label{eqn:Deltasfsame}
\end{eqnarray}
where $\beta=ch(sp)$ for charge (spin), with  $sgn(ch)=1$, $sgn(sp)=-1$, and $\gamma = T, Tsh, Tth$ represents total $\Delta_T$ noise, $\Delta_{Tsh}$ noise, or $\Delta_{Tth}$ noise. Spin-polarised $\Delta^{\beta,\sigma \sigma^{\prime}}_{\gamma}$ noise is calculated from spin-polarised quantum noise $S^{\sigma \sigma^{\prime}}_{\gamma}$  (see, Eq.~(\ref{eqn:noise})) at vanishing charge or spin current, with $\sigma,\sigma^{\prime}=\{ \uparrow,\downarrow \}$.} For a NIN junction, there is no spin $\Delta_T$ noise and $\Delta^{NIN}_{T} = \Delta^{NIN}_{Tsh}+\Delta^{NIN}_{Tth}$. The following subsection gives the derivation of $\Delta_T$ noise for vanishing current. Vanishing charge (spin) current can be achieved by taking zero bias voltage and imposing zero charge (spin) current condition~\cite{generalbound1}, i.e., $I^{ch}_1 (I^{sp}_1) = 0$.


\subsection{Spin-polarised $\Delta_T$ noise}
\label{comp res}

% Figure environment removed

In Fig.~\ref{NSfN_comp}, we represent the case of reservoirs with comparable temperatures at zero bias to study spin-polarised $\Delta_T$ noise in N$_1$/SF/N$_2$ junction. Here, we consider the temperature difference $\Delta T = T_1-T_2$ to be much smaller than average temperature $\bar{T}= (T_1+T_2)/2$, where the temperature in normal metal $N_1$ as $T_1$, and in normal metal $N_2$ as $T_2$, and zero applied bias ($\Delta V=V_1-V_2=0$), i.e., applied voltage bias $V_1=V_2=V$, as shown in Fig.~\ref{NSfN_comp}. 

Thermovoltage is the applied bias voltage required to tune the average charge (spin) current to zero, i.e., $I^{ch(sp)}_1 =0$ ~\cite{generalbound}. To calculate $\Delta^{\eta}_T$ ($\eta = ch, sp, NIN$) noise, we expand the $\Delta^{\eta}_T$ noise in a power series of $\left( \frac{\Delta T}{2 \bar{T}} \right)$. In this paper, for our chosen reservoirs with comparable temperatures setup, we have taken temperature, with $\frac{\Delta T}{2 \Bar{T}} \approx 0.14$ (as we consider comparable temperatures, i.e., $T_1=4$K and $T_2=3$K). For zero applied bias ($\Delta V= V_1-V_2=0$), Fermi distribution function in $N_1$ is $f_{1}(E-V)= \left[ 1 + e^{\frac{E-V}{k_B T_1}} \right]^{-1}$ and in $N_2$ is $f_{2}(E-V) = \left[ 1 + e^{\frac{E-V}{k_B T_2}} \right]^{-1}$. 

% For setup 2, i.e., finite bias case ($V_1=0$,$V_2\neq 0$), Fermi functions can be written as $f_{1}(E)= \left[ 1 + e^{\frac{E}{k_B T_1}} \right]^{-1}$ and $f_{2}(E-V_2) = \left[ 1 + e^{\frac{E-V_2}{k_B T_2}} \right]^{-1}$.

The examination of small temperature gradients on spin-polarised transport is important, especially in the low-temperature range \cite{lowtemp}, as these can arise unintentionally in any experiment \cite{atomicscaleexpt}.

Shot noise-like contribution to the $\Delta^{\eta}_T$ noise, i.e., $\Delta^{\eta}_{Tsh}$ is calculated from quantum shot noise $S^{\eta}_{11sh}$ (see, Eq.~(\ref{eqn:noise})) numerically, by replacing the thermovoltage (to ensure the vanishing current condition) in the voltage bias $V$, for temperature gradient $\frac{\Delta T}{2 \Bar{T}}\ll 1$, spin polarised $\Delta^{\eta}_{Tsh}$ noise is calculated as,
\begin{eqnarray}
\Delta^{\eta}_{Tsh} && = \frac{2 e^2}{h} \int^{\infty}_{-\infty} F^{\eta}_{11sh} \biggl\{ 4 \left( k_B \Bar{T} \frac{\partial f(E-V)}{\partial k_B T} \right)^2 \left( \frac{\Delta T}{2 \Bar{T}} \right)^2 \nonumber \\
&& + \frac{4}{3} \left(k_B \Bar{T} \right)^4 \left( \frac{\partial f(E-V)}{\partial k_B T}  \frac{\partial^3 f(E-V)}{\partial (k_B T)^3} \right) \left( \frac{\Delta T}{2 \Bar{T}} \right)^4 \biggr\} dE 
\label{Dsh_setup1}
\end{eqnarray}
where $\eta=ch (sp)$ denotes charge (spin) in a N$_1$/SF/N$_2$ junction, with $\eta=NIN$ denoting a NIN junction without any spin-flip scattering. The detailed derivation of Eq. (\ref{Dsh_setup1}) is given in Appendix \ref{App_DT}. $F^{\eta}_{11sh}$ is function of reflection and transmission amplitudes, with $F^{ch}_{11sh} = F^{\uparrow \uparrow}_{11sh}+ F^{\uparrow \downarrow}_{11sh}+ F^{\downarrow \uparrow}_{11sh}+ F^{\downarrow \downarrow}_{11sh} $, and $F^{sp}_{11sh}= F^{\uparrow \uparrow}_{11sh} - F^{\uparrow \downarrow}_{11sh} - F^{\downarrow \uparrow}_{11sh}$ + $F^{\downarrow \downarrow}_{11sh}$. Expressions for $F^{\sigma \sigma^{\prime}}_{11sh}$ are given below:
\begin{eqnarray}
F^{\uparrow \uparrow}_{11sh} &=& \mathcal{T}^{\uparrow \uparrow} (1-\mathcal{T}^{\uparrow \uparrow}) + \mathcal{T}^{\uparrow \downarrow} (1-\mathcal{T}^{\uparrow \downarrow}),  \nonumber \\
F^{\downarrow \downarrow}_{11sh} &=& \mathcal{T}^{\downarrow \downarrow} (1 - \mathcal{T}^{\downarrow \downarrow}) + \mathcal{T}^{\downarrow \uparrow} ( 1 - \mathcal{T}^{\downarrow \uparrow} ), \nonumber \\
F^{\uparrow \downarrow}_{11sh} &=& 2 \mathcal{T}^{\uparrow \downarrow} \mathcal{T}^{\uparrow \uparrow},~~~ F^{\downarrow \uparrow}_{11sh} = 2 \mathcal{T}^{\downarrow \uparrow} \mathcal{T}^{\downarrow \downarrow}, 
% F^{\uparrow \uparrow}_{11sh} &=& |s^{\uparrow \uparrow}_{12}|^2 (1-|s^{\uparrow \uparrow}_{12}|^2) + |s^{\uparrow \downarrow}_{12}|^2 (1-|s^{\uparrow \downarrow}_{12}|^2),  \nonumber \\
% F^{\downarrow \downarrow}_{11sh} &=& |s^{\downarrow \downarrow}_{12}|^2 (1-|s^{\downarrow \downarrow}_{12}|^2) + |s^{\downarrow \uparrow}_{12}|^2 (1-|s^{\downarrow \uparrow}_{12}|^2), \nonumber \\
% F^{\uparrow \downarrow}_{11sh} &=& 2 |s^{\uparrow \downarrow}_{12}|^2 |s^{\uparrow \uparrow}_{12}|^2,~~~ F^{\downarrow \uparrow}_{11sh} = 2 |s^{\downarrow \uparrow}_{11}|^2 |s^{\downarrow \downarrow}_{12}|^2, 
\label{Fsh}
\end{eqnarray}
and their detailed derivation is given in Appendix \ref{App_Qn}. Thermal noise-like contribution to the $\Delta^{\eta}_T$ noise, i.e., $\Delta^{\eta}_{Tth}$ is calculated numerically from quantum noise $S^{\eta}_{11th}$ (see, Eq. (\ref{eqn:noise})). For vanishing current condition, replacing the thermovoltage in the voltage, for $\frac{\Delta T}{2 \Bar{T}}\ll 1$, $\Delta^{\eta}_{Tth}$ noise can be calculated as, 
\begin{eqnarray}           
\Delta^{\eta}_{Tth} && = - \frac{2 e^2}{h}  \int^{\infty}_{-\infty}  F^{\eta}_{11th} \Biggl[ 2 k_B \Bar{T} \frac{\partial f(E-V)}{\partial E} + (k_B \Bar{T})^2 \biggl\{ k_B \Bar{T} \nonumber \\
&& \frac{\partial}{\partial E} \left( \frac{\partial^2 f(E-V)}{ \partial (k_B T)^2}  \right) + 2 \frac{\partial}{\partial E} \left( \frac{\partial f(E-V)}{ \partial k_B T}  \right) \biggr\} \left( \frac{\Delta T}{2 \Bar{T}} \right)^2  \Biggr] dE
\label{Dth_setup1}
\end{eqnarray}
again see Appendix \ref{App_DT}, for derivation of $\Delta^{\eta}_{Tth}$. $F^{\eta}_{11th}$ is a function of reflection and transmission amplitudes, with $F^{ch}_{11th}=F^{\uparrow \uparrow}_{11th} + F^{\uparrow \downarrow}_{11th} + F^{\downarrow \uparrow}_{11th}$ + $F^{\downarrow \downarrow}_{11th}$, and, $F^{sp}_{11th} = F^{\uparrow \uparrow}_{11th} - F^{\uparrow \downarrow}_{11th} - F^{\downarrow \uparrow}_{11th}$ + $F^{\downarrow \downarrow}_{11th}$. Expressions for $F^{\sigma \sigma^{\prime}}_{11th}$ are given as,
\begin{eqnarray}
F^{\uparrow \uparrow}_{11th} &=& ( 1 - \mathcal{R}^{\uparrow \uparrow}), ~~~ F^{\uparrow \downarrow}_{11th} = (- \mathcal{R}^{\uparrow \downarrow}), \nonumber \\
F^{\downarrow \uparrow}_{11th} &=& (- \mathcal{R}^{\downarrow \uparrow}), ~~~
F^{\downarrow \downarrow}_{11th} = (1 - \mathcal{R}^{\downarrow \downarrow} ),
% F^{\uparrow \uparrow}_{11th} &=& ( 1 - |s^{\uparrow \uparrow}), ~~~ F^{\uparrow \downarrow}_{11th} = (- |s^{\uparrow \downarrow}_{11}|^2), \nonumber \\
% F^{\downarrow \uparrow}_{11th} &=& (- |s^{\downarrow \uparrow}_{11}|^2), ~~~
% F^{\downarrow \downarrow}_{11th} = (1- |s^{\downarrow \downarrow}_{11}|^2 ),
\label{Fth}
\end{eqnarray}
and the detailed derivation of Eq. (\ref{Fth}) is given in Appendix \ref{App_Qn}.   

\section{Results and Discussion}
\label{results}

In this section, we discuss $\Delta^{\eta}_T$ noise ($\eta=ch, sp, NIN$) along with the ratio of shot noise-like contribution to thermal noise-like contribution with the spin-flipper's spin $S=1/2$.

We evaluate the thermovoltage for each spin-configuration shown in Fig.~\ref{fig:avgm} and calculate the respective $\Delta_T$ noise, and plot $\Delta^{\eta}_T$ noise, $\eta=ch, sp, NIN$ versus exchange interaction strength $J$, for a fixed value of the spin-flipper's spin $S=1/2$ and we take the sum over all possible spin magnetic moment: $m \in \{ \frac{-1}{2}, \frac{1}{2} \}$. Additionally, for a deeper understanding of $\Delta_T$ noise, we discuss $\Delta_{T}$ shot noise and $\Delta_{T}$ thermal noise contribution to charge and spin $\Delta_T$ noise.


In Fig.~\ref{fig:Setup 1}, we plot $\Delta^{\eta}_T$, the ratio $|\Delta^{\eta}_{Tsh}/\Delta^{\eta}_{Tth}|$ and $\Delta^{\eta}_{T}$ noise vs. $J$ with $\frac{\Delta T}{2 \Bar{T}} \simeq 0.14$, where $\eta=ch, sp$ and $NIN$. In the absence of spin-flip scattering, the $\Delta^{NIN}_{T}$ noise remains positive regardless of change in $J$, see Fig.~\ref{fig:Setup 1}(a). Similarly, $\Delta^{sp}_{T}$ noise is positive in the entire range of $J$. Conversely, the $\Delta^{ch}_{T}$ noise changes sign and becomes negative for high exchange interaction strength. The magnitude of the $\Delta^{NIN}_{Tsh}$ noise is always smaller than that of the $\Delta^{NIN}_{Tth}$ noise, consistent with the general bound, as described in~\cite{generalbound}. Furthermore, even in the presence of spin-flip scattering, the charge and spin noise adhere to the general bound ($|\Delta^{ch(sp)}_{Tsh}/\Delta^{ch(sp)}_{Tth}| \leq 1$), unlike quantum noise which does not obey such bound. $\Delta^{\eta}_T$ noise is symmetric with respect to the reversal in sign of exchange interaction strength $J$. This is because spin-flip and no-flip reflection and transmission probabilities are symmetric with respect to reversal in sign of $J$, see, Eqs. (\ref{Prob_up}) and (\ref{Prob_dn}). 

Unlike quantum noise, the magnitude of charge or spin $\Delta_T$ thermal noise is either greater than or equal to charge or spin $\Delta_T$ shot noise. The interaction among electrons of opposite spins, depending on the finite temperature gradient, results in negative $\Delta^{ch}_{T}$ noise, a phenomenon absent in a NIN junction where positive $\Delta^{NIN}_{T}$ noise persists irrespective of $J$. 

\begin{widetext}

% Figure environment removed

\end{widetext}

Previous works on NIN junction, show that charge $\Delta_T$ noise is always positive~\cite{popoff,generalbound,dTtheory}. However, in this paper for a N$_1$/SF/N$_2$ junction with spin-flip scattering, the charge $\Delta^{ch}_{T}$ noise can be negative due to the exchange interaction. On the other hand, $\Delta^{sp}_T$ noise is always positive. $\Delta^{\eta}_{T}$ noise shows similar behavior as the $\Delta^{\eta}_{Tth}$ thermal noise (Fig. \ref{fig:Setup 1}(c)), which is the dominant contribution to $\Delta^{\eta}_{T}$ noise. As predicted in Refs. \cite{qstat,qint}, both charge as well as spin $\Delta_{T}$ shot noise are positive throughout the range of $J$ (see, inset in Fig. \ref{fig:Setup 1} (c)). Charge $\Delta^{ch}_{Tth}$ noise turns negative at large $J$, while spin $\Delta^{sp}_{Tth}$ noise is always positive irrespective of any change in $J$ (see, Fig. \ref{fig:Setup 1}(c)). In a NIN junction, both $\Delta^{NIN}_{Tth}$ and $\Delta^{NIN}_{Tth}$ noise are positive.


\subsection{Leading order contributions to $\Delta_{T} $ noise}

Finally, we look at the leading order contributions to $\Delta^{\eta}_{T} $ noise and in turn the $\Delta^{\eta}_{Tsh}$ and $\Delta^{\eta}_{Tth}$ noise, where $\eta \in \{ ch,sp,NIN \}$. $\Delta_T$ noise can be a tool to probe the temperature difference of a junction. 

{The $\Delta^{\eta}_{Tsh}$ noise at leading order is from $(\Delta T)^2$ term, see Eq. (\ref{Dsh_setup1}). Likewise, for $\Delta^{\eta}_{Tth}$ noise, the leading order contribution from the temperature gradient is quadratic, i.e., $(\Delta T)^2$ (see, Eq. (\ref{Dth_setup1})). However, it is to be noted that there is a term independent of temperature difference ($\Delta T$) in charge and spin $\Delta^{ch(sp)}_{Tth}$ noise, see Eq. (\ref{Dth_setup1}). Consequently, for charge and spin $\Delta^{ch(sp)}_T$ noise, the leading order contribution also arises from $(\Delta T)^2$, stemming from both $\Delta^{ch(sp)}_{Tsh}$ and $\Delta^{ch(sp)}_{Tth}$ noise, as $\Delta^{ch(sp)}_{Tsh(th)}$ noise shows a quadratic dependence on $(\Delta T)^2$ in N$_1$/SF/N$_2$ junction. Again, because of the term independent of temperature difference ($\Delta T$) in charge and spin $\Delta^{ch(sp)}_{Tth}$ noise, there is consequently a term independent of $\Delta T$ in total charge and spin $\Delta^{ch(sp)}_T$ noise. Similarly, leading order contribution to total $\Delta^{NIN}_{T}$ noise is from $(\Delta T)^2$ in a NIN junction without any spin-flip scattering. There is a term independent of $\Delta T$ in $\Delta^{NIN}_{Tth}$ noise, thus, leads to a term independent of $\Delta T$ in total $\Delta^{NIN}_{T}$ noise, see Eq. (\ref{Dth_setup1}). These dependencies are also observed in previous work~\cite{popoff}. This serves as a consistency check for our work.}

\begin{table}[h!]
\caption{Leading order contributions to $\Delta^{\eta}_{T}$ noise  along with $\Delta^{\eta}_{Tsh}$ and $\Delta^{\eta}_{Tth}$ noise. $^*$ There is a term independent of temperature difference ($\Delta T$) in $\Delta^{\eta}_{Tth}$ noise, see Eq. (\ref{Dth_setup1}).}
\begin{tabular}{|c|c|c|c|}
\hline
& $\Delta_{Tsh}$  & $\Delta^*_{Tth}$  \\ \hline
Charge  & $ \propto (\Delta T)^2$ & $ \propto (\Delta T)^2$  \\ \hline
Spin  & $ \propto (\Delta T)^2$  & $ \propto (\Delta T)^2$  \\ \hline
NIN  & $ \propto  (\Delta T)^2$  & $ (\Delta T)^2$   \\ \hline
\end{tabular}
\label{Table 3}
\end{table}

{In Table~\ref{Table 3}, we present a summary of the leading order contributions arising from temperature gradient ($\Delta T$) to $\Delta^{\eta}_{T}$ noise along with $\Delta^{\eta}_{Tsh}$ and $\Delta^{\eta}_{Tth}$ noise where $\eta=\{ch,~sp\}$ for charge and spin in N$_1$/SF/N$_2$ junction and $\eta=NIN$ for NIN junction.
By measuring the total $\Delta^{\eta}_T$ noise as a function of temperature, one can discern the temperature-dependent behavior of $\Delta^{\eta}_{Tth}$ noise, which is the dominant contribution~\cite{popoff}. }


\section{Analysis}
\label{analysis}

In Table~\ref{Table_DT}, we have summarized the characteristics of charge $\Delta^{ch}_T$ noise and spin $\Delta^{sp}_T$ noise and their respective same-spin and opposite-spin correlation contribution along with the magnitude of $|\Delta^{ch(sp)}_{Tsh}/\Delta^{ch(sp)}_{Tth}|$. This general bound ($\Delta_{Tsh}/\Delta_{Tth}$) defines a fundamental difference between various transport via charge, spin and heat $\Delta_T$ noise at vanishing charge, spin, and heat current, respectively. This distinction is based on whether they adhere to the general bound or not. For instance, heat $\Delta_T$ noise does not adhere to this boundary \cite{generalbound}. We also include the characteristics of $\Delta^{NIN}_{T}$, and magnitude of $|\Delta^{NIN}_{Tsh}/\Delta^{NIN}_{Tth}|$ for a NIN junction.  The same-spin and opposite-spin correlation contributions are not applicable in a NIN junction without any spin-flip scattering. 

\begin{table}[h!]
\hspace{-0.5cm}
\caption{Summary of the sign of $\Delta^{\eta}_T$, sign of same-spin and opposite-spin correlations $\Delta^{\eta,sa(op)}_T$, and magnitude of $\mid \frac{\Delta^{\eta}_{Tsh}}{\Delta^{\eta}_{Tth}} \mid$ at higher $J$ with spin-flipper's spin $S=1/2$, where $\eta=\{ch,~sp\}$ for charge and spin in N$_1$/SF/N$_2$ junction and $\eta=NIN$ for NIN junction. $\Delta^{NIN,sa(op)}_T$ in a NIN junction are not applicable in the absence of spin-flip scattering which is denoted as NA.}
\label{Table_DT}
\scalebox{0.98}{\begin{tabular}{|c|c|c|c|c|} 
\hline
~ & $\Delta^{\eta}_T$ &  $\mid \frac{\Delta^{\eta}_{Tsh}}{\Delta^{\eta}_{Tth}} \mid$  & $\Delta^{\eta,sa}_{T}$ &  $\Delta^{\eta,op}_T$ \\ 
\hline
Charge & Negative &  \begin{tabular}[c]{@{}c@{}} $ \leq 1 $ (around $|J| \rightarrow 5, \simeq 1$) \\ (see Fig. \ref{fig:Setup 1} (b)) \end{tabular} & Positive & Negative \\ 
\hline
Spin & Positive &  \begin{tabular}[c]{@{}c@{}} $ \ll 1$ \\ (see Fig. \ref{fig:Setup 1} (b)) \end{tabular} & Positive & Negative  \\ 
\hline
NIN & Positive &  \begin{tabular}[c]{@{}c@{}} $ \ll 1$ \\ (see Fig. \ref{fig:Setup 1} (b)) \end{tabular} & NA & NA  \\
\hline
\end{tabular}}
\end{table}

For the first time, we find negative $\Delta^{ch}_T$ noise due to spin-flip scattering. This change of sign in $\Delta^{ch}_T$ noise can be a tool to study temperature gradient in spin-polarised transmission. To our knowledge, spin-polarised $\Delta^{sp}_{T}$, $\Delta^{sp}_{Tth}$ and $\Delta^{sp}_{Tsh}$ noise due to spin-flip scattering, has never been calculated before. However, $\Delta^{sp}_{Tsh}/\Delta^{sp}_{Tth}$ ratio with temperature bias and spin bias (i.e., spin-dependent chemical potential, with spin-independent transmission) in the absence of any spin-flip scattering has been calculated, see Ref.~\cite{generalbound1}, which obeys the general bound, i.e., $\Delta^{sp}_{Tsh}/\Delta^{sp}_{Tth} \leq 1$. When we introduce spin-flipper at the interface of a bilayer metallic junction (see Fig.~\ref{fig:NsfN}), we find that spin-polarised charge and spin $\Delta_T$ noise obey the general bound in the presence of spin-flip scattering. 

Below, we compare our results for the N$_1$/SF/N$_2$ case with the NIN case when spin-flip scattering is absent. $\Delta^{NIN}_{T}$ noise is always positive as it is in absence of any spin-flip scattering. In an earlier work, for a system of two normal metals separated by a quantum dot with temperature bias and zero voltage bias case similar to our setup, it has been shown that $\Delta^{NIN}_T$ noise is always positive, see Ref.~\cite{popoff}.  

It has been seen that the sign of $\Delta^{NIN}_{Tsh}$ can probe quantum statistics, i.e., positive for fermions and either positive or negative for bosons ~\cite{qstat}. $\Delta^{NIN}_{Tsh}$ in this work is always positive. Our results are similar to Ref.~\cite{popoff} wherein the temperature gradient and voltage bias are applied to a bilayer metallic junction separated by a quantum dot. 


We see negative $\Delta^{ch}_{T}$ noise for the first time due to the interplay between spin-flip scattering and the temperature gradient. Spin-flip scattering generates negative $\Delta^{ch}_T$ noise with change in exchange interaction strength ($J$), which indicates the dominant contributions from negative $\Delta^{ch}_{Tth}$ noise, as $\Delta^{ch}_{Tsh}$ noise is always positive. However, spin $\Delta_T$ noise is always positive. In Ref.~\cite{popoff}, negative $\Delta^{ch}_{Tth}$ thermal noise is predicted to occur for a specially designed transmission probability function. The setup used in Ref.~\cite{popoff} to predict this is with a temperature bias: $\frac{\Delta T}{2 \Bar{T}}\ll 1$, and with $\Delta T = T_1 -T_2$ and $\Bar{T} = (T_1 + T_2 )/2$ and for zero bias voltage: $\Delta V =0$. This setup is similar to the voltage and temperature gradients applied in our setup without any spin-flipper. To understand the origin of negative $\Delta_T$ noise, we first look at the opposite-spin and same-spin correlations (see, Eq. (\ref{eqn:Deltasfsame}) and  Eqs. (\ref{DTsh_ch}-\ref{DTth_sp}) in Appendix \ref{App_DT}).

\begin{widetext}

    % Figure environment removed 

\end{widetext}

Contributions from same-spin correlation $\Delta^{ch,sa}_{Tth}$ are positive, while opposite-spin correlation $\Delta^{ch,op}_{Tth}$ are negative, resulting in a negative $\Delta^{ch}_{Tth}$ noise (see, Fig. \ref{fig:DTch_saopp}), while contributions from both same-spin as well as opposite-spin correlations to $\Delta^{ch}_{Tsh}$ noise are positive. Thus, $\Delta^{ch}_{Tsh}$ noise is positive.  The change in the sign of $\Delta^{ch}_T$ noise is induced by the negative opposite-spin correlation at high exchange interaction strength ($J$). Consequently, it can serve as a valuable tool for discerning dominant spin correlation. This strong exchange interaction, in particular, has a notable impact on the opposite-spin correlation compared to same-spin correlation (see, Fig. \ref{fig:DTch_saopp}).


\begin{widetext}

    % Figure environment removed 

\end{widetext}


The dominant spin same-spin ($\Delta^{sp,sa}_{Tth}$) correlation is positive, while opposite-spin ($\Delta^{sp,op}_{Tth}$) correlation is negative, resulting in a positive $\Delta^{sp}_{Tth}$ noise, see Fig. \ref{fig:DTsp_saopp}, contrary to $\Delta^{ch}_{Tth}$ noise. Similar to $\Delta^{ch}_{Tsh}$ noise, same-spin ($\Delta^{sp,sa}_{Tsh}$) and opposite-spin ($\Delta^{sp,op}_{Tsh}$) correlation contributions to $\Delta^{sp}_{Tsh}$ noise are also positive, resulting in $\Delta^{sp}_{Tsh}$ noise being positive.  

\subsection{$J \rightarrow $ Zero}

As the exchange interaction approaches zero ($J \rightarrow 0$), no-flip transmission probabilities $\mathcal{T}^{\uparrow \uparrow}$ and $\mathcal{T}^{\downarrow \downarrow} \rightarrow 1$ , while spin-flip transmission probabilities $\mathcal{T}^{\uparrow \downarrow}$ and $\mathcal{T}^{\downarrow \uparrow} \rightarrow 0$. Hence, the same-spin correlation $\Delta^{ch(sp),sa}_{Tsh}$, which depends on the scattering terms $\mathcal{T}^{\uparrow \uparrow} (1-\mathcal{T}^{\uparrow \uparrow}) + \mathcal{T}^{\uparrow \downarrow} (1-\mathcal{T}^{\uparrow \downarrow})$ and $\mathcal{T}^{\downarrow \downarrow} ( 1 - \mathcal{T}^{\downarrow \downarrow}) + \mathcal{T}^{\downarrow \uparrow} (1-\mathcal{T}^{\downarrow \uparrow})$, vanishes as $J \rightarrow 0$. Similarly, the opposite-spin correlation $\Delta^{ch(sp),op}_{Tsh}$, depends on the scattering terms $\mathcal{T}^{\uparrow \uparrow} \mathcal{T}^{\uparrow \downarrow}$ and $\mathcal{T}^{\downarrow \uparrow} \mathcal{T}^{\downarrow \downarrow}$, also vanishes as $J \rightarrow 0$, see Eq. (\ref{B_Ssh}) in Appendix \ref{App_Qn}. Consequently, as $J \rightarrow 0$, both the same-spin and opposite correlations contributions lead to vanishing charge and spin $\Delta_{Tsh}$ noise, see Figs. \ref{fig:DTch_saopp}(a)-\ref{fig:DTsp_saopp}(a).


For charge and spin $\Delta_T$ thermal noise, the same-spin correlation $\Delta^{ch(sp),sa}_{Tth}$ which depends on the scattering terms:
$( 1 - \mathcal{R}^{\uparrow \uparrow} = \mathcal{R}^{\uparrow \downarrow} + \mathcal{T}^{\uparrow \uparrow} + \mathcal{T}^{\uparrow \downarrow})$ and $( 1 - \mathcal{R}^{\downarrow \downarrow} = \mathcal{R}^{\downarrow \uparrow} + \mathcal{T}^{\downarrow \downarrow} + \mathcal{T}^{\downarrow \uparrow})$  remains finite and exhibits a peak at $J \rightarrow 0$. This peak arises from the contributions of the transmission probabilities without spin-flip, as $\mathcal{T}_{\uparrow \uparrow}$, and $\mathcal{T}_{\downarrow \downarrow}$ $\rightarrow 1$. Conversely, the opposite-spin correlation $\Delta^{ch(sp),op}_{Tth}$ which depends on spin-flip reflection probabilities $\mathcal{R}^{\uparrow \downarrow}$ and $\mathcal{R}^{\downarrow \uparrow}$, vanishes at $J \rightarrow 0$ as $\mathcal{R}_{\uparrow \downarrow}$ and $\mathcal{R}_{\downarrow \uparrow} \rightarrow 0$.

Consequently, as $J$ approaches 0, with finite transmission probability of electrons without spin-flip, only contributions from same-spin correlation matter for both charge and spin $\Delta_T$ noise (see Figs. \ref{fig:DTch_saopp}-\ref{fig:DTsp_saopp}). 


\subsection{$J \rightarrow $ large}

As the exchange interaction strength $J$ increases, both no-flip ($\mathcal{T}^{\uparrow \uparrow}$ and $\mathcal{T}^{\downarrow \downarrow}$) and spin-flip transmission probabilities ($\mathcal{T}^{\uparrow \downarrow}$ and $\mathcal{T}^{\downarrow \uparrow}$) decrease. Thus, the same-spin correlation $\Delta^{ch(sp),sa}_{Tsh}$, which depends on the scattering terms ($\mathcal{T}^{\uparrow \uparrow} (1- \mathcal{T}^{\uparrow \uparrow})$ and $\mathcal{T}^{\uparrow \downarrow} (1- \mathcal{T}^{\uparrow \downarrow})$), decreases for large values of the exchange interaction strength $J$ and vanishes as $J \rightarrow \infty$. Similarly, the opposite-spin correlation $\Delta^{ch(sp),op}_{Tsh}$, which depends on the scattering terms ($\mathcal{T}^{\uparrow \uparrow} \mathcal{T}^{\uparrow \downarrow}$ and $\mathcal{T}^{\downarrow \downarrow} \mathcal{T}^{\downarrow \uparrow}$), decreases for large values of $J$ and vanishes as $J \rightarrow \infty$. Thus, the $\Delta^{ch(sp)}_{Tsh}$ noise vanishes at $J \rightarrow \infty$ due to both no-flip and spin-flip transmission probabilities.

For charge and spin $\Delta_T$ thermal noise, the same-spin correlation $\Delta^{ch(sp),sa}_{Tth}$, which depends on the scattering terms $(1 - \mathcal{R}^{\uparrow \uparrow})$ and $(1 - \mathcal{R}^{\downarrow \downarrow})$, decreases for large $J$ and vanishes as $J \rightarrow \infty$. This is because the no-flip reflection probabilities $\mathcal{R}^{\uparrow \uparrow}$ and $\mathcal{R}^{\downarrow \downarrow} \rightarrow 1$ as $J \rightarrow \infty$. However, spin-flip reflection probabilities $\mathcal{R}^{\uparrow \downarrow}$ and $\mathcal{R}^{\downarrow \uparrow} \rightarrow 0$ at $J \rightarrow \infty$. Thus, the opposite-spin correlation $\Delta^{ch(sp),op}_{Tth}$, which depends on $(- \mathcal{R}^{\uparrow \downarrow})$ and $(- \mathcal{R}^{\downarrow \uparrow})$ vanishes for the limit $J \rightarrow \infty$.

The charge $\Delta^{\sigma \sigma^{\prime}, ch}_{\alpha \beta}$ noise is derived from the $ S^{\sigma \sigma^{\prime}}_{\alpha \beta}$ quantum noise at vanishing charge current, while the spin $\Delta^{\sigma \sigma^{\prime}, sp}_{\alpha \beta}$ noise is derived from the $S^{\sigma \sigma^{\prime}}_{\alpha \beta}$ quantum noise at vanishing spin current (see, Appendix \ref{App_DT}).  These different conditions for vanishing charge/spin current, lead to different charge or spin thermovoltage, which in turn depend on the temperature gradient and the scattering factors $F^{ch}_I=\mathcal{T}^{\uparrow \uparrow} + \mathcal{T}^{\uparrow \downarrow}+\mathcal{T}^{\downarrow \uparrow} + \mathcal{T}^{\downarrow \downarrow} $ and $F^{sp}_I= \mathcal{T}^{\uparrow \uparrow} - \mathcal{T}^{\uparrow \downarrow} - \mathcal{T}^{\downarrow \uparrow} + \mathcal{T}^{\downarrow \downarrow}$ (see Appendix \ref{App_I}). Subsequently, leading to differing spin correlation contributions to charge and spin $\Delta_T$ noise.

For high values of $J$, the opposite-spin correlation $\Delta^{ch,op}_{Tth}$ dominates the same-spin correlation $\Delta^{ch,sa}_{Tth}$ due to the interplay between spin-flip reflection probabilities and temperature gradient. This leads to negative $\Delta^{ch}_{Tth}$ noise. However, the same-spin correlation $\Delta^{sp,sa}_{Tth}$ dominates the opposite-spin correlation $\Delta^{sp,op}_{Tth}$. Subtracting $\Delta^{sp,op}_{Tth}$ from $\Delta^{sp,sa}_{Tth}$ also leads to positive $\Delta^{sp}_{Tth}$ noise. 

Our most important result, i.e., negative $\Delta^{ch}_{T}$ noise in comparison to the positive $\Delta^{NIN}_{T}$ noise underlines the impact of spin-flip scattering in a bilayer metallic junction with a finite temperature gradient. 


\subsection{Experimental realization}
\label{expt}

Recently, the temperature gradient generated $\Delta_T$ quantum noise has been measured for zero charge current along with thermal noise-like contribution and shot noise-like contributions~\cite{atomicscaleexpt,qcircuitexpt,tjunctionexpt}. Charge $\Delta_T$ shot noise contribution has been calculated by subtracting the charge $\Delta_T$ thermal noise contribution from total charge $\Delta_T$ noise. The negative charge and spin $\Delta_T$ noise in a N$_1$/SF/N$_2$ junction can be easily measured experimentally and presents an accessible means to probe spin-flip scattering. In Ref.~\cite{atomicscaleexpt}, total $\Delta_T$ noise is measured in an atomic scale molecular junction using the break junction technique by creating a nanoscale gap or junction with the use of a piezoelectric material, with a temperature gradient similar to our case (see, Fig.~\ref{NSfN_comp}). $\Delta_T$ noise measurement is an adaptable probe to study transport in a system without any specific design limitations~\cite{atomicscaleexpt}. This flexibility in measurement allows for the experimental exploration of our proposed spin-polarised device without any significant challenges, even at low temperatures \cite{lowtemp}.
% For temperature-generated noise, charge and spin $\Delta_T$ noise calculated for a finite temperature gradient at zero voltage bias in a N$_1$/SF/N$_2$ junction, with a magnetic impurity doped at the interface, which acts as a spin-flipper~\cite{spinflipper}. 

We opt for materials where electron transport occurs in the quantum ballistic regime. This makes two-dimensional electron gases (2DEGs) particularly suitable for our purpose. The spin flipper, functioning as a magnetic impurity with a delta potential, can be analyzed in a manner analogous to an Anderson impurity but different from a Kondo impurity~\cite{spinflip, spinflipper}. One can utilize a quantum dot that incorporates spin-paired electrons alongside an extra unpaired electron, so as to play the role of a magnetic impurity or spin-flipper \cite{Cdot}.
 

\section{Conclusion}
\label{conclusion}

We have calculated charge and spin $\Delta_T$ noise with respective contributions from $\Delta_T$ thermal and $\Delta_T$ shot noise for vanishing charge current and spin currents, respectively. Spin-polarised charge and spin $\Delta_T$ noise are enhanced compared to $\Delta^{NIN}_T$ noise. Charge $\Delta_T$ noise can turn negative due to spin-flip scattering depending on the exchange interaction strength ($J$).

One can explain negative $\Delta^{ch}_T$ noise by looking at shot noise $\Delta^{ch}_{T}$ thermal noise contribution and their sign. $\Delta^{ch}_{Tth}$ noise, which is the dominant contribution is negative, thus leads to negative $\Delta^{ch}_{T}$ noise. In case of $\Delta^{sp}_{T}$ noise, the dominant contribution from $\Delta^{sp}_{Tth}$ noise is positive which gives positive $\Delta^{sp}_{T}$ noise. However, $\Delta^{ch(sp)}_{Tsh}$ noise is always positive as predicted for fermions \cite{qstat}. 

Further, one can explain that negative $\Delta^{ch}_{T}$ noise by looking at the contribution from same-spin and opposite-spin correlation and their sign. The opposite-spin correlation contribution ($\Delta^{ch,op}_{T}$) is negative and thus leads to negative $\Delta^{ch}_{T}$ noise. However, the same-spin correlation contribution ($\Delta^{sp,sa}_{T}$), which is the dominant spin correlation is positive, and leads to positive $\Delta^{sp}_{T}$ noise. Both same-spin and opposite-spin correlation contributions to $\Delta^{ch(sp)}_{Tsh}$ are positive. 

% However, $\Delta^{ch(sp)}_T$ shot noise, is always positive as predicted for fermionic statistics \cite{qstat} due to dominant same-spin correlation.
% , while opposite-spin correlation is negative \cite{spincor}. 

We confirm the general bound for $\Delta_T$ noise as also seen in ~\cite{generalbound, generalbound1}, i.e., shot noise-like contribution is less than or equal to thermal noise-like contribution ($| \Delta^{NIN}_{Tsh}/ \Delta^{NIN}_{Tth} | \leq 1$). Charge (spin) $\Delta_{T}$ thermal noise is either greater than or equal to charge (spin) $\Delta_{T}$ shot noise, establishing a general bound irrespective of the temperature gradient or voltage bias, unlike quantum thermal noise, which is less than quantum shot noise in the low-temperature limit \cite{noise,noises,noise1}. 


Temperature gradient along with the thermovoltage (which ensures a vanishing current condition), plays a pivotal role in shaping the distinctive behavior of the thermal noise contribution to $\Delta_T$ noise in contrast to quantum thermal noise, which shows similar behaviour as the differential conductance. This intriguing interplay between spin-flip scattering and temperature gradient has the potential to induce a change in sign of $\Delta^{ch}_T$ noise, which can be readily experimentally measured, and also one can predict its origin whether it is due to same-spin correlation or opposite-spin correlation. This encourages us to delve deeper and investigate spin-polarised $\Delta_T$ shot noise in the presence of a superconductor, as it might reveal signatures of bunching or anti-bunching effect \cite{NIS}.


\acknowledgments This work was supported by the grant "Josephson junctions with strained Dirac materials and their application in
quantum information processing" from the Science \& Engineering Research Board (SERB), New Delhi, Government of India, under Grant No. $CRG/2019/006258$.

\section*{Appendix}

The Appendix consists of four sections. First, in Appendix \ref{App_I}, we calculate the charge and spin thermovoltage after imposing the vanishing charge and spin current condition, respectively. Next, we derive charge and spin quantum noise in a N$_1$/SF/N$_2$ junction in Appendix \ref{App_Qn}. In Appendix \ref{App_DT}, we derive $\Delta_T$ shot noise-like contribution and $\Delta_T$ thermal noise-like contribution.


\appendix
\label{Appendix}

We consider finite temperature gradient along with zero voltage to examine the impact of temperature gradient on spin-polarised charge and spin $\Delta_T$ noise.

\begin{widetext}
    
\section[Average charge and spin current]{Average charge and spin current with finite temperature gradient}
\label{App_I}

The average charge current in normal metal $N_1$ is given as,
\begin{eqnarray}
I^{ch}_{1} &=& \sum_{\sigma \in \{\uparrow, \downarrow\}} \langle  I^{\sigma}_{1} \rangle = \frac{e}{h} \int^{\infty}_{-\infty} \sum_{\sigma \in \{\uparrow, \downarrow\}} \sum_{\substack{\gamma, \gamma^{\prime} \in \{1,2\}; \\ \rho, \rho^{\prime} \in \{\uparrow, \downarrow\}}} \langle a^{\rho \dagger}_{\gamma}  a^{\rho^{\prime}}_{\gamma^{\prime}} \rangle A^{\rho \rho^{\prime}}_{\gamma^{\prime} \gamma}(1,\sigma,E) dE  = \frac{e}{h}  \sum_{\sigma \in \{\uparrow, \downarrow\}} \sum_{\substack{\gamma, \gamma^{\prime} \in \{1,2\}; \\ \rho, \rho^{\prime} \in \{\uparrow, \downarrow\}}} \int^{\infty}_{-\infty} dE (\delta_{\gamma 1} \delta_{\gamma^{\prime} 1} \delta_{\sigma \rho} \delta_{\sigma \rho^{\prime}} - s^{\sigma \rho \dagger}_{1 \gamma} s^{\sigma \rho^{\prime}}_{1 \gamma^{\prime}}) (\delta_{\gamma \gamma^{\prime}} \delta_{\rho \rho^{\prime}} f_{\gamma})  \nonumber \\
&=& \frac{e}{h} \int^{\infty}_{-\infty} \biggl\{ \left(1 - s^{\uparrow \uparrow \dagger}_{11} s^{\uparrow \uparrow}_{11} - s^{\uparrow \downarrow \dagger}_{11} s^{\uparrow \downarrow}_{11} +1 - s^{\downarrow \uparrow \dagger}_{11} s^{\downarrow \uparrow}_{11} - s^{\downarrow \downarrow \dagger}_{11} s^{\downarrow \downarrow}_{11} \right) f_{1}  - \left( s^{\uparrow \uparrow \dagger}_{12} s^{\uparrow \uparrow}_{12} + s^{\uparrow \downarrow \dagger}_{12} s^{\uparrow \downarrow}_{12} + s^{\downarrow \uparrow \dagger}_{12} s^{\downarrow \uparrow}_{12} + s^{\downarrow \downarrow \dagger}_{12} s^{\downarrow \downarrow}_{12} \right) f_{2} \biggr\}dE \nonumber \\
&=& \frac{e}{h} \int^{\infty}_{-\infty} \biggl\{ \left(1 - \mathcal{R}^{\uparrow \uparrow}   - \mathcal{R}^{\uparrow \downarrow}  + 1 - \mathcal{R}^{\downarrow \uparrow }  -\mathcal{R}^{\downarrow \downarrow}   \right) f_{1} - \left( \mathcal{T}^{\uparrow \uparrow}  + \mathcal{T}^{\uparrow \downarrow} + \mathcal{T}^{\downarrow \uparrow}   + \mathcal{T}^{\downarrow \downarrow}   \right) f_{2} \biggr\}dE \nonumber \\
&=& \frac{e}{h} \int^{\infty}_{-\infty} \biggl\{ \left( \mathcal{T}^{\uparrow \uparrow}  + \mathcal{T}^{\uparrow \downarrow}   + \mathcal{T}^{\downarrow \uparrow}   + \mathcal{T}^{\downarrow \downarrow}   \right) (f_1 -f_2 ) \biggr\} dE = \frac{e}{h} \int^{\infty}_{-\infty} \biggl\{ F^{ch}_{I}   (f_1 -f_2 ) \biggr\}dE,
\label{A_Ich}
\end{eqnarray}
where $F^{ch}_{I}  =\left( \mathcal{T}^{\uparrow \uparrow}  + \mathcal{T}^{\uparrow \downarrow}  + \mathcal{T}^{\downarrow \uparrow}   + \mathcal{T}^{\downarrow \downarrow}  \right)$, $\mathcal{T}^{\uparrow \uparrow}  = |s^{\uparrow \uparrow}_{12}|^2=|t^{\uparrow \uparrow}|^2$, $\mathcal{T}^{\downarrow \uparrow}  = |s^{\downarrow \uparrow}_{12}|^2=|t^{\downarrow \uparrow}|^2$, $\mathcal{T}^{\uparrow \downarrow}  = |s^{\uparrow \downarrow}_{12}|^2=|t^{\uparrow \downarrow}|^2$, $\mathcal{T}^{\downarrow \downarrow}  = |s^{\downarrow \downarrow}_{12}|^2=|t^{\downarrow \downarrow}|^2$, $\mathcal{R}^{\uparrow \uparrow}  = |s^{\uparrow \uparrow}_{11}|^2=|r^{\uparrow \uparrow}|^2$, $\mathcal{R}^{\downarrow \uparrow}  = |s^{\downarrow \uparrow}_{11}|^2=|r^{\downarrow \uparrow}|^2$, $\mathcal{R}^{\uparrow \downarrow}  = |s^{\uparrow \downarrow}_{11}|^2 =|r^{\uparrow \downarrow}|^2$ and $\mathcal{R}^{\downarrow \downarrow}  = |s^{\downarrow \downarrow}_{11}|^2 =|r^{\downarrow \downarrow}|^2$. $f_{\gamma}(E) = \left[ 1+ e^{\frac{E-V_{\gamma}}{k_B T_{\gamma}}} \right]^{-1}$ is Fermi function in normal metal $\gamma=1,2$, $k_B$ is Boltzmann constant, $T_{\gamma}$ is temperature and $V_{\gamma}$ is applied bias in normal metal $\gamma$, and $E$ is energy of incident electron measure from Fermi energy $E_F$.  

The average spin current in normal metal $N_1$, i.e., $\langle  I^{\uparrow}_{1} \rangle - \langle  I^{\downarrow}_{1} \rangle$ is then,
\begin{eqnarray}
    I^{sp}_{1} &=&  \frac{e}{h} \int^{\infty}_{-\infty} \left( \mathcal{T}^{\uparrow \uparrow}   - \mathcal{T}^{\uparrow \downarrow}   - \mathcal{T}^{\downarrow \uparrow}  + \mathcal{T}^{\downarrow \downarrow}   \right) (f_1 -f_2 ) dE = \frac{e}{h} \int^{\infty}_{-\infty} F^{sp}_{I}   (f_1 -f_2 )dE,
    \label{A_Isp}
\end{eqnarray}
where $F^{sp}_{I}   =\left( \mathcal{T}^{\uparrow \uparrow}    - \mathcal{T}^{\uparrow \downarrow}    - \mathcal{T}^{\downarrow \uparrow}    + \mathcal{T}^{\downarrow \downarrow}   \right)$. For NIN junction, average current is $I^N_1 = \frac{e}{h} \int^{\infty}_{-\infty}  \mathcal{T}^{N}  (f_1 -f_2 ) dE = \frac{e}{h} \int^{\infty}_{-\infty} F^{N}_{I}   (f_1 -f_2 ) dE$, without spin-flip scattering ($|s^{\downarrow \uparrow}_{12}|^2=|s^{\uparrow \downarrow}_{12}|^2=0$), where $F^{N}_{I}  = \mathcal{T}^N$.


\subsection{Average charge and spin current}
\label{A1}

We start by calculating the thermovoltage for our setup at zero bias $V_1=V_2=V$. This thermovoltage is denoted as $V$, and $T_{\gamma}$ represents the temperature of the reservoir $\gamma$. We assume that $T_1\neq T_2\neq 0$, with $\frac{\Delta T}{2 \Bar{T}}\ll 1$ ($\Delta T = T_1-T_2$, and $\Bar{T} = (T_1+T_2)/2$). The Fermi functions, are defined as $f_{1}= \left[ 1 + e^{\frac{E-V}{k_B T_1}} \right]^{-1}$ and $f_{2}= \left[ 1 + e^{\frac{E-V}{k_B T_2}} \right]^{-1}$, where $T_1 = \Bar{T} + \Delta T/2$ and $T_2 = \Bar{T} - \Delta T/2$. We can express $f_1(E-V,k_B T_1)$ as $f_1(E-V, k_B \Bar{T} + k_B \Delta T/2)$ and $f_2(E-V,k_B T_2)$ as $f_2(E-V, k_B \Bar{T} - k_B \Delta T/2)$.

To investigate the thermovoltage due to temperature gradient, we expand the Fermi functions in powers of $\frac{\Delta T}{2\Bar{T}}$. The Fermi functions in the average current $I^{ch}_1$, as given in Eq.~(\ref{A_Ich}), can be derived using this expansion
  
\begin{eqnarray}
f_{1} (E-V, k_B T_1) - f_2(E-V, k_B T_2) &=&  2 k_B \Bar{T} \frac{\partial f(E-V)}{\partial k_B T} \left( \frac{\Delta T}{2 \Bar{T}} \right) +  O\left( \frac{\Delta T}{2 \Bar{T}} \right)^3,
\label{f_s1}
\end{eqnarray}
where $f(E-V)=1/(1+e^{\frac{E-V}{k_B \Bar{T}}})$ and $\left. \frac{\partial f(E-V)}{\partial k_B T} = \frac{\partial f(E-V)}{\partial k_B T} \right|_{T=\Bar{T}}$. We adopt the approximation that $\frac{\Delta T }{ 2 \Bar{T}} \ll 1$, for reservoirs at comparable temperatures and disregard higher-order terms $ O\left( \frac{\Delta T}{2 \Bar{T}} \right)^3$. Scattering term $F^{\eta}_I$ are function of $E$. Now average charge current in N$_1$ is,
\begin{eqnarray}
 I^{ch}_{1} &=& \frac{e}{h} \int^{\infty}_{-\infty} F^{ch}_I\left( E \right) \biggl( f_1 - f_2 \biggr) dE  = \frac{e}{h} \int^{\infty}_{-\infty} F^{ch}_I \left( E \right) 2 k_B \Bar{T} \frac{\partial f(E-V)}{\partial k_B T} \left( \frac{\Delta T}{2 \Bar{T}} \right) dE,
\end{eqnarray}
where $F^{ch}_I \left( E \right)= \mathcal{T}^{\uparrow \uparrow} \left( E \right) + \mathcal{T}^{\uparrow \downarrow}\left( E \right) + \mathcal{T}^{\downarrow \uparrow} \left( E \right) + \mathcal{T}^{\downarrow \downarrow}\left( E \right)$. 

Equating $I^{ch}_{1} = 0$, we calculate the thermovoltage numerically in Mathematica. The thermovoltage with temperature gradient and zero voltage, for our setup has never been calculated. The thermovoltage for each spin configuration differs, as depicted in Fig.~\ref{fig:avgm}. The thermovoltage for vanishing charge current in spin-configuration 1 is denoted as $V^{s1}_{ch1}$, with $F^{ch}_I(E)=\mathcal{T}^{\uparrow \uparrow}(E)$. On the other hand, for spin-configuration 2, the thermovoltage is denoted as $V^{s1}_{ch2}$, and $F^{ch}_I(E)=\mathcal{T}^{\uparrow \uparrow}(E)+\mathcal{T}^{\uparrow \downarrow}(E)$. For spin-configuration 3, the thermovoltage is denoted as $V^{s1}_{ch3}$, and $F^{ch}_I(E)=\mathcal{T}^{\downarrow \downarrow}(E)+\mathcal{T}^{\downarrow \uparrow}(E)$. Finally, in spin-configuration 4, where $F^{ch}_I(E)=\mathcal{T}^{\downarrow \downarrow}(E)$, the thermovoltage is denoted as $V^{s1}_{ch4}$. However, energy-independent transmission implies zero charge or spin current irrespective of voltage and temperature bias.

Now, the spin current in N$_1$, in the limit $\frac{\Delta T }{ 2 \Bar{T}} \ll 1$, 

\begin{eqnarray}
I^{sp}_{1} &=& \frac{e}{h} \int^{\infty}_{-\infty} F^{sp}_I \left( E \right) \biggl( f_1 - f_2 \biggr) dE  = \frac{e}{h} \int^{\infty}_{-\infty} F^{sp}_I \left( E \right) 2 k_B \Bar{T} \frac{\partial f(E-V)}{\partial k_B T} \left( \frac{\Delta T}{2 \Bar{T}} \right) dE,
\end{eqnarray}

where $F^{sp}_I \left( E \right)= \mathcal{T}^{\uparrow \uparrow} \left( E \right) - \mathcal{T}^{\uparrow \downarrow}\left( E \right) - \mathcal{T}^{\downarrow \uparrow} \left( E \right) + \mathcal{T}^{\downarrow \downarrow}\left( E \right)$. For vanishing spin current, equate $I^{sp}_{1}=0$ to get the thermovoltage numerically in Mathematica. 

In spin configurations 1 and 4, with no-flip scattering, the thermovoltage for vanishing spin-current is equivalent to the thermovoltage for vanishing charge current. The thermovoltage for vanishing spin current in spin-configuration 1 in case of reservoirs with comparable temperatures is denoted as $V^{s1}_{sp1}$, with $F^{sp}_I(E)=\mathcal{T}^{\uparrow \uparrow}(E)$. For spin-configuration 2, the thermovoltage is denoted as $V^{s1}_{sp2}$, and $F^{sp}_I(E)=\mathcal{T}^{\uparrow \uparrow}(E)-\mathcal{T}^{\uparrow \downarrow}(E)$ with an up-spin electron incident. In spin-configuration 3, the thermovoltage is denoted as $V^{s1}_{sp3}$, and $F^{sp}_I(E)=\mathcal{T}^{\downarrow \downarrow}(E) - \mathcal{T}^{\downarrow \uparrow}(E)$. Finally, in spin-configuration 4, with $F^{sp}_I(E)=\mathcal{T}^{\downarrow \downarrow}(E)$, the thermovoltage is denoted as $V^{s1}_{sp4}$.

Without spin-flip scattering, for a NIN junction, charge current is derived as, 

\begin{eqnarray}
I^{NIN}_{1} &=& \frac{e}{h} \int^{\infty}_{-\infty} F^{N}_I \left( E \right) \biggl( f_1 - f_2 \biggr) dE  = \frac{e}{h} \int^{\infty}_{-\infty} F^{N}_I \left( E \right) 2 k_B \Bar{T} \frac{\partial f(E-V)}{\partial k_B T} \left( \frac{\Delta T}{2 \Bar{T}} \right) dE,
\end{eqnarray}
For vanishing spin current, equate $I^{sp}_{1}=0$ to get the thermovoltage numerically in Mathematica. In the case of NIN junction, only two possible spin configurations exist: spin-configuration 1 and spin-configuration 2, see Fig~\ref{fig:avgm}. The transmission amplitudes for both spin configurations are identical, i.e., $F^{N}_I(E)= \mathcal{T}^N(E)$ and the thermovoltage is denoted as $V^{s1}_{N}$. 

\section{Charge and spin quantum noise}
\label{App_Qn}

The noise auto-correlation~\cite{noise, thermalnoise} between contacts $\alpha$ and $\beta$ is defined $S_{\alpha \beta} (t-t^{\prime}) = \frac{1}{2 \pi} \langle \Delta I_{\alpha}(t) \Delta I_{\beta}(t^{\prime}) + \Delta I_{\beta}(t^{\prime}) \Delta I_{\alpha}(t) \rangle$. Similarly, spin-polarised noise~\cite{spinnoise} is defined as,
$S^{\sigma \sigma^{\prime}}_{\alpha \beta} (t-t^{\prime}) = \frac{1}{2 \pi} \langle \Delta I^{\sigma}_{\alpha}(t) \Delta I^{\sigma^{\prime}}_{\beta}(t^{\prime}) + \Delta I^{\sigma^{\prime}}_{\beta}(t^{\prime}) \Delta I^{\sigma}_{\alpha}(t)\rangle$, with $\Delta I^{\sigma}_{\alpha} = I^{\sigma}_{\alpha} - \langle I^{\sigma}_{\alpha} \rangle$, where $I^{\sigma}_{\alpha}$ is current with spin $\sigma =\{ \uparrow, \downarrow \}$ in lead $\alpha$. The charge shot noise for the charge current, $I^{ch}_{\alpha} = I^{\uparrow}_{\alpha} + I^{\downarrow}_{\alpha}$ is $S^{ch}_{\alpha \beta} = S^{\uparrow \uparrow}_{\alpha \beta} + S^{\uparrow \downarrow}_{\alpha \beta} + S^{\downarrow \uparrow}_{\alpha \beta} + S^{\downarrow \downarrow}_{\alpha \beta}$. The spin-shot noise for the spin current $I^{sp}_{\alpha} = I^{\uparrow}_{\alpha} - I^{\downarrow}_{\alpha}$ is $S^{sp}_{\alpha \beta} = S^{\uparrow \uparrow}_{\alpha \beta} - S^{\uparrow \downarrow}_{\alpha \beta} - S^{\downarrow \uparrow}_{\alpha \beta} + S^{\downarrow \downarrow}_{\alpha \beta}$. Noise at zero frequency, $S^{\sigma \sigma^{\prime}}_{\alpha \beta}$ is derived as,
\begin{eqnarray}
S^{\sigma \sigma^{\prime}}_{\alpha \beta} &=& \frac{e^2}{h} \int^{\infty}_{-\infty}dE \sum_{\substack{\gamma, \gamma^{\prime} \in \{1,2\}; \\ \rho, \rho^{\prime} \in \{\uparrow, \downarrow\} }} \left[ \left\{ \langle a^{\sigma \dagger}_{\alpha}  a^{\sigma^{\prime}}_{\beta} a^{\rho \dagger}_{\gamma}  a^{\rho^{\prime}}_{\gamma^{\prime}} \rangle - \langle a^{\sigma \dagger}_{\alpha}  a^{\sigma^{\prime}}_{\beta} \rangle \langle a^{\rho \dagger}_{\gamma}  a^{\rho^{\prime}}_{\gamma^{\prime}} \rangle \right\} A^{\sigma^{\prime} \sigma}_{\alpha \beta}(\alpha,\sigma) A^{\rho^{\prime} \rho}_{\gamma^{\prime} \gamma}(\beta,\sigma^{\prime}) \right. \nonumber \\
&& \left. + \left\{ \langle a^{\sigma^{\prime} \dagger}_{\beta} a^{\sigma}_{\alpha} a^{\rho^{\prime} \dagger}_{\gamma^{\prime}}  a^{\rho}_{\gamma} \rangle - \langle  a^{\sigma^{\prime} \dagger}_{\beta} a^{\sigma}_{\alpha}  \rangle \langle  a^{\rho^{\prime} \dagger}_{\gamma^{\prime}}  a^{\rho}_{\gamma} \rangle \right\} A^{\sigma \sigma^{\prime}}_{\beta \alpha}(\beta,\sigma^{\prime}) A^{\rho \rho^{\prime}}_{\gamma \gamma^{\prime}}(\alpha,\sigma) \right],
\end{eqnarray}
where $A^{\rho \rho^{\prime}}_{\gamma^{\prime} \gamma}(\alpha,\sigma) = \delta_{\gamma \alpha} \delta_{\gamma^{\prime} \alpha} \delta_{\sigma \rho} \delta_{\sigma \rho^{\prime}} - s^{\sigma \rho \dagger}_{\alpha \gamma} s^{\sigma \rho^{\prime}}_{\alpha \gamma^{\prime}} $. $\langle a^{\sigma \dagger}_{\alpha}  a^{\sigma^{\prime}}_{\beta} \rangle = \delta_{\alpha \beta} \delta_{\sigma \sigma^{\prime}} f^{\sigma}_{\alpha}$, where we have considered the Fermi function to be independent of spin, hence $f^{\sigma}_{\alpha}=f_{\alpha}$, and $\langle a^{\sigma \dagger}_{\alpha}  a^{\sigma^{\prime}}_{\beta} a^{\rho \dagger}_{\gamma}  a^{\rho^{\prime}}_{\gamma^{\prime}} \rangle - \langle a^{\sigma \dagger}_{\alpha}  a^{\sigma^{\prime}}_{\beta} \rangle \langle a^{\rho \dagger}_{\gamma}  a^{\rho^{\prime}}_{\gamma^{\prime}} \rangle = \delta_{\alpha \gamma^{\prime}} \delta_{\beta \gamma} \delta_{\sigma \rho^{\prime}} \delta_{\sigma^{\prime} \rho} f_{\beta}(1-f_{\alpha}) = f_{\gamma}(1-f_{\gamma^{\prime}})$. Thus,
\begin{eqnarray}
S^{\sigma \sigma^{\prime}}_{\alpha \beta} &=& \frac{e^2}{h} \int^{\infty}_{-\infty}dE \sum_{\substack{\gamma, \gamma^{\prime} \in \{1,2\}; \\ \rho, \rho^{\prime} \in \{\uparrow, \downarrow\} }} \left[  A^{\rho \rho^{\prime}}_{\gamma \gamma^{\prime}}(\alpha,\sigma) A^{\rho^{\prime} \rho}_{\gamma^{\prime} \gamma}(\beta,\sigma^{\prime}) \left\{ f_{\gamma} (1- f_{\gamma^{\prime}}) + f_{\gamma^{\prime}} (1- f_{\gamma}) \right\} \right] \nonumber \\
&=& \frac{e^2}{h} \int^{\infty}_{-\infty}dE \sum_{\substack{\gamma, \gamma^{\prime} \in \{1,2\}; \\ \rho, \rho^{\prime} \in \{\uparrow, \downarrow\} }} \left[ \left( \delta_{\gamma \alpha} \delta_{\gamma^{\prime} \alpha} \delta_{\sigma \rho} \delta_{\sigma \rho^{\prime}} - s^{\sigma \rho \dagger}_{\alpha \gamma} s^{\sigma \rho^{\prime}}_{\alpha \gamma^{\prime}} \right) \left( \delta_{\gamma^{\prime} \beta} \delta_{\gamma \beta} \delta_{\sigma^{\prime} \rho^{\prime}} \delta_{\sigma^{\prime} \rho} - s^{\sigma^{\prime} \rho^{\prime} \dagger}_{\beta \gamma^{\prime}} s^{\sigma^{\prime} \rho}_{\beta \gamma} \right)  \left\{ f_{\gamma} (1- f_{\gamma^{\prime}}) + f_{\gamma^{\prime}} (1- f_{\gamma}) \right\} \right] .~~~
\end{eqnarray}
Noise auto-correlation $S^{\sigma \sigma^{\prime}}_{11}$ is derived as,
\begin{eqnarray}
S^{\sigma \sigma^{\prime}}_{11} &=& \frac{e^2}{h} \int^{\infty}_{-\infty}dE \sum_{\substack{\gamma, \gamma^{\prime} \in \{1,2\}; \\ \rho, \rho^{\prime} \in \{\uparrow, \downarrow\} }} \left[ \left( \delta_{\gamma 1} \delta_{\gamma^{\prime} 1} \delta_{\sigma \rho} \delta_{\sigma \rho^{\prime}} - s^{\sigma \rho \dagger}_{1 \gamma} s^{\sigma \rho^{\prime}}_{1 \gamma^{\prime}} \right) \left( \delta_{\gamma^{\prime} 1} \delta_{\gamma 1} \delta_{\sigma^{\prime} \rho^{\prime}} \delta_{\sigma^{\prime} \rho} - s^{\sigma^{\prime} \rho^{\prime} \dagger}_{1 \gamma^{\prime}} s^{\sigma^{\prime} \rho}_{1 \gamma} \right)  \left\{ f_{\gamma} (1- f_{\gamma^{\prime}}) + f_{\gamma^{\prime}} (1- f_{\gamma}) \right\} \right] \nonumber \\
&=& \frac{e^2}{h} \int^{\infty}_{-\infty} dE \sum_{\rho, \rho^{\prime} \in \{\uparrow, \downarrow\} } \left[ \left\{ \left( \delta_{11} \delta_{1 1} \delta_{\sigma \rho} \delta_{\sigma \rho^{\prime}} - s^{\sigma \rho \dagger}_{1 1} s^{\sigma \rho^{\prime}}_{1 1} \right) \left( \delta_{1 1} \delta_{11} \delta_{\sigma^{\prime} \rho^{\prime}} \delta_{\sigma^{\prime} \rho} - s^{\sigma^{\prime} \rho^{\prime} \dagger}_{1 1} s^{\sigma^{\prime} \rho}_{1 1} \right) f_{1} (1- f_{1}) \right\} +\left\{ \left( \delta_{21} \delta_{2 1} \delta_{\sigma \rho} \delta_{\sigma \rho^{\prime}} - s^{\sigma \rho \dagger}_{1 2} s^{\sigma \rho^{\prime}}_{1 2} \right) \right. \right. \nonumber \\
&& \left. \left.  \left( \delta_{2 1} \delta_{21} \delta_{\sigma^{\prime} \rho^{\prime}} \delta_{\sigma^{\prime} \rho} - s^{\sigma^{\prime} \rho^{\prime} \dagger}_{1 2} s^{\sigma^{\prime} \rho}_{1 2} \right) f_{2} (1- f_{2}) \right\} +\left\{ \left( \delta_{11} \delta_{2 1} \delta_{\sigma \rho} \delta_{\sigma \rho^{\prime}} - s^{\sigma \rho \dagger}_{11} s^{\sigma \rho^{\prime}}_{12} \right) \left( \delta_{2 1} \delta_{11} \delta_{\sigma^{\prime} \rho^{\prime}} \delta_{\sigma^{\prime} \rho} - s^{\sigma^{\prime} \rho^{\prime} \dagger}_{1 2} s^{\sigma^{\prime} \rho}_{1 1} \right) \right. \right. \nonumber \\ 
&& \left. \left. + \left( \delta_{21} \delta_{11} \delta_{\sigma \rho} \delta_{\sigma \rho^{\prime}} - s^{\sigma \rho \dagger}_{12} s^{\sigma \rho^{\prime}}_{11} \right) \left( \delta_{11} \delta_{21} \delta_{\sigma^{\prime} \rho^{\prime}} \delta_{\sigma^{\prime} \rho} - s^{\sigma^{\prime} \rho^{\prime} \dagger}_{11} s^{\sigma^{\prime} \rho}_{12} \right) \right\}  \left( f_1 (1-f_{2}) + f_2 (1- f_{1} \right) \right],
\label{B_Snoise}
\end{eqnarray}
where terms with Fermi function coefficients $f_1(1-f_1)$ and $f_2(1-f_2)$ correspond to thermal noise, which is the equilibrium noise contribution to quantum noise. Subtracting thermal noise from total quantum noise will give the shot noise. However, the terms with $(f_1-f_2)^2$ correspond to the non-equilibrium shot noise that vanishes at equilibrium. The spin contribution to the quantum noise, $S^{\sigma \sigma^{\prime}}_{11}$ given in Eq.~(\ref{B_Snoise}) can be decomposed into thermal noise ($S^{\sigma \sigma^{\prime}}_{11th}$) and shot noise ($S^{\sigma \sigma^{\prime}}_{11sh}$). Spin-polarised thermal noise contribution is derived from Eq.~(\ref{B_Snoise}) as $\frac{2e^2}{h} \int^{\infty}_{-\infty} \left( f_{1} (1- f_{1}) + f_{2} (1- f_{2}) \right) \left\{ \delta_{\sigma \sigma^{\prime}} - | s^{\sigma \sigma^{\prime}}_{11} |^2  \right\} dE$ which can be further written elaborately as,
\begin{eqnarray}
S^{\uparrow \uparrow}_{11th} &=& \frac{2e^2}{h} \int^{\infty}_{-\infty} dE \left[ \left\{ 1 - \mathcal{R}^{\uparrow \uparrow} \right\} \left( f_{1} (1- f_{1}) + f_{2} (1- f_{2}) \right) \right] = \frac{2e^2}{h} \int^{\infty}_{-\infty}  F^{\uparrow \uparrow}_{11th} \left( f_{1} (1- f_{1}) + f_{2} (1- f_{2}) \right) dE, \nonumber \\
S^{\uparrow \downarrow}_{11th} &=& \frac{2e^2}{h} \int^{\infty}_{-\infty} dE \left[ \left\{ - \mathcal{R}^{\uparrow \downarrow} \right\} \left( f_{1} (1- f_{1}) + f_{2} (1- f_{2}) \right) \right] = \frac{2e^2}{h} \int^{\infty}_{-\infty}  F^{\uparrow \downarrow}_{11th} \left( f_{1} (1- f_{1}) + f_{2} (1- f_{2}) \right) dE, \nonumber \\
S^{\downarrow \uparrow}_{11th} &=& \frac{2e^2}{h} \int^{\infty}_{-\infty} dE \left[ \left\{  - \mathcal{R}^{\downarrow \uparrow} \right\} \left( f_{1} (1- f_{1}) + f_{2} (1- f_{2}) \right) \right] = \frac{2e^2}{h} \int^{\infty}_{-\infty}  F^{\downarrow \uparrow}_{11th} \left( f_{1} (1- f_{1}) + f_{2} (1- f_{2}) \right) dE, \nonumber \\
S^{\downarrow \downarrow}_{11th} &=& \frac{2e^2}{h} \int^{\infty}_{-\infty} dE \left[ \left\{ 1-  \mathcal{R}^{\downarrow \downarrow}  \right\} \left( f_{1} (1- f_{2}) + f_{2} (1- f_{1}) \right) \right] = \frac{2e^2}{h} \int^{\infty}_{-\infty}  F^{\downarrow \downarrow}_{11th} \left( f_{1} (1- f_{1}) + f_{2} (1- f_{2}) \right) dE,
\label{B_Sth}
\end{eqnarray}
where reflection probabilities in terms of scattering amplitudes are defined as $\mathcal{R}^{\uparrow \uparrow} = |s^{\uparrow \uparrow}_{11}|^2$, $\mathcal{R}^{\uparrow \downarrow} =|s^{\uparrow \downarrow}_{11}|^2$, $\mathcal{R}^{\downarrow \downarrow} =|s^{\downarrow \downarrow}_{11}|^2$, $\mathcal{R}^{\downarrow \uparrow} =|s^{\downarrow \uparrow}_{11}|^2$ (see, Eqs. (\ref{Prob_up}) and (\ref{Prob_dn})). \\
The spin-polarised shot noise auto-correlation $S^{\sigma \sigma^{\prime}}_{11sh}$ is calculated by subtracting the spin-polarised thermal noise $S^{\sigma \sigma^{\prime}}_{11th}$ from the spin-polarised quantum noise $S^{\sigma \sigma^{\prime}}_{11}$. Shot noise auto-correlation for the different spin-dependent contributions can be derived from Eq.~(\ref{B_Snoise}) as 
\begin{eqnarray}
S^{\uparrow \uparrow}_{11sh} &=& \frac{2e^2}{h} \int^{\infty}_{-\infty} dE  \left\{ \mathcal{T}^{\uparrow \uparrow} (1 - \mathcal{T}^{\uparrow \uparrow} ) + \mathcal{T}^{\uparrow \downarrow} ( 1 - \mathcal{T}^{\uparrow \downarrow} ) \right\} \left(f_{1}- f_{2} \right)^2 = \frac{2e^2}{h} \int^{\infty}_{-\infty}  F^{\uparrow \uparrow}_{11sh} \left(f_{1}- f_{2} \right)^2 dE, \nonumber \\
S^{\uparrow \downarrow}_{11sh} &=& \frac{2e^2}{h} \int^{\infty}_{-\infty} dE \left\{ 2 ~ \mathcal{T}^{\uparrow \downarrow} ~\mathcal{T}^{\uparrow \uparrow} ~ \right\} \left(f_{1}- f_{2} \right)^2 = \frac{2e^2}{h} \int^{\infty}_{-\infty}  F^{ \uparrow \downarrow}_{11sh} \left(f_{1}- f_{2} \right)^2 dE, \nonumber \\
S^{\downarrow \uparrow}_{11sh} &=& \frac{2e^2}{h} \int^{\infty}_{-\infty} dE  \left\{ 2 ~ \mathcal{T}^{\downarrow \uparrow} ~ \mathcal{T}^{\downarrow \downarrow} ~ \right\} \left(f_{1}- f_{2} \right)^2 = \frac{2e^2}{h} \int^{\infty}_{-\infty}  F^{\downarrow \uparrow}_{11sh} \left(f_{1}- f_{2} \right)^2 dE, \nonumber \\
S^{\downarrow \downarrow}_{11sh} &=& \frac{2e^2}{h} \int^{\infty}_{-\infty} dE  \left\{  \mathcal{T}^{\downarrow \downarrow} (1- \mathcal{T}^{\downarrow \downarrow} ) + \mathcal{T}^{\downarrow \uparrow} (1 - \mathcal{T}^{\downarrow \uparrow} ) \right\} \left(f_{1}- f_{2} \right)^2  = \frac{2e^2}{h} \int^{\infty}_{-\infty}  F^{\downarrow \downarrow}_{11sh} \left(f_{1}- f_{2} \right)^2 dE.
\label{B_Ssh}
\end{eqnarray}
where transmission probabilities in terms of scattering amplitudes are defined as $\mathcal{T}^{\uparrow \uparrow} = |s^{\uparrow \uparrow}_{12}|^2$, $\mathcal{T}^{\uparrow \downarrow} =|s^{\uparrow \downarrow}_{12}|^2$, $\mathcal{T}^{\downarrow \downarrow} =|s^{\downarrow \downarrow}_{12}|^2$, $\mathcal{T}^{\downarrow \uparrow} =|s^{\downarrow \uparrow}_{12}|^2$ (see, Eqs. (\ref{Prob_up}) and (\ref{Prob_dn})). \\
For NIN junction, total quantum noise without any spin-flip scattering is $S^{N}_{11}=S^{N}_{11sh}+S^{N}_{11th}$, where the shot noise-like contribution ($S^{N}_{11sh}$) and thermal noise-like contribution ($S^{N}_{11th}$) are given as,

\begin{eqnarray}
 \text{Thus,}~~ S^{N}_{11sh} &=& \frac{2e^2}{h} \int^{\infty}_{-\infty} dE \left[ \left\{  \mathcal{T}^{N} ~ \mathcal{R}^{N} \right\} \left(f_{1}- f_{2} \right)^2 \right] = \frac{2e^2}{h} \int^{\infty}_{-\infty}  F^{N}_{11sh} \left(f_{1}- f_{2} \right)^2 dE, \nonumber \\
  \text{and}~~S^{N}_{11th} &=& \frac{2e^2}{h} \int^{\infty}_{-\infty} dE \left[ \left\{ \mathcal{T}^{N} \right\} \left( f_{1} (1- f_{1}) + f_{2} (1- f_{2}) \right) \right] =\frac{2e^2}{h} \int^{\infty}_{-\infty}  F^{N}_{11th} \left( f_{1} (1- f_{1}) + f_{2} (1- f_{2}) \right) dE,
  \label{B_NN}
\end{eqnarray}

where $F^{N}_{11sh}=\mathcal{T}^N ~ \mathcal{R}^N =|s^{N}_{12}|^2 ~ |s^{N}_{11}|^2$ and $F^{N}_{11th}= \mathcal{T}^N=|s^{N}_{12}|^2$. In earlier studies, $\Delta_T$ auto-correlations have been studied in Refs. ~\cite{popoff,generalbound,generalbound1} without spin-flip scattering. Here, we focus on $\Delta_T$ noise auto-correlations to study the spin transport for the first time. 

\section[Charge and spin $\Delta_T$ noise]{Charge and spin $\Delta_T$ noise with shot noise-like ($\Delta^{ch}_{Tsh}$, $\Delta^{sp}_{Tsh}$) and thermal noise-like ($\Delta^{ch}_{Tth}$, $\Delta^{sp}_{Tth}$) contributions}
\label{App_DT}

Next, we calculate the charge $\Delta_T$ noise and spin $\Delta_T$ noise in case of reservoirs with comparable temperatures, with $\frac{\Delta T}{2 \Bar{T}}\ll 1$ ($\Delta T = T_1-T_2$ and $\Bar{T} = (T_1+T_2)/2$) and at zero-bias $\Delta V =0$, $V_1=V_2=V$. Fermi functions are $f_1(E-V)=1/(1+e^{\frac{E-V}{k_B T_1}})$ and $f_2(E-V)=1/(1+e^{\frac{E-V}{k_B T_2}})$, as $T_1 = \Bar{T} + \Delta T/2$, $T_2 = \Bar{T} - \Delta T/2$, $f_1(E-V,k_B T_1) = f_1(E-V, k_B \Bar{T}+ k_B \Delta T/2)$, and $f_2(E-V,k_B T_2) = f_2(E-V,k_B \Bar{T} - k_B \Delta T/2)$.

We can decompose charge and spin  $\Delta_T$ noise into shot noise-like and thermal noise-like contributions, i.e., $\Delta^{ch(sp)}_T=\Delta^{ch(sp)}_{Tsh} + \Delta^{ch}_{Tth}$. 

Charge and spin $\Delta_T$ noise, $\Delta_T$ shot noise, and $\Delta_T$ thermal noise in terms of opposite-spin and same-spin correlations can be written as, 
\begin{eqnarray}
    \Delta^{\beta}_{\gamma} && = \Delta^{\beta,sa}_{\gamma} + sgn(\beta)~ \Delta^{\beta,op}_{\gamma}, \nonumber \\
    \text{with}~\Delta^{\beta,sa}_{\gamma} &=& \sum_{\sigma \in \{ \uparrow, \downarrow \}} \Delta^{\beta,\sigma \sigma}_{\gamma},~\text{and}~ \Delta^{\beta,op}_{\gamma} = \sum_{\substack{\sigma, \sigma^{\prime} \in \{ \uparrow, \downarrow \}; \\  \sigma \neq \sigma^{\prime} }} \Delta^{\beta,\sigma \sigma^{\prime}}_{\gamma}, 
    \label{App:Deltasfsame}
\end{eqnarray}
where $\beta=ch(sp)$ for charge (spin), with  $sgn(ch)=1$, $sgn(sp)=-1$, and $\gamma = T, Tsh, Tth$ represents total $\Delta_T$ noise, $\Delta_{Tsh}$ noise, or $\Delta_{Tth}$ noise. Spin-polarised $\Delta^{\beta,\sigma \sigma^{\prime}}_{\gamma}$ noise is calculated from spin-polarised quantum noise $S^{\sigma \sigma^{\prime}}_{\gamma}$  (see Eq.~(\ref{B_Snoise})) at vanishing charge or spin current, with $\sigma,\sigma^{\prime}=\{ \uparrow,\downarrow \}$.


Spin polarised charge and spin $\Delta^{\beta, \sigma \sigma^{\prime}}_{Tsh}$ shot noise-like contribution to $\Delta^{\beta, \sigma \sigma^{\prime}}_T$ noise can be calculated from quantum shot noise given in Eq.~(\ref{B_Ssh}), at vanishing charge and spin current. We expand in powers of $\frac{\Delta T}{2\Bar{T}}$, which is a useful tool for investigating the impact of temperature gradient on $\Delta_T$ noise. The Fermi functions in $\Delta_T$ shot noise-like contribution are derived as- 
\begin{eqnarray}
\label{C1_f}
(f_{1} (E-V,k_B T_1) - f_2(E-V,k_B T_2))^2 &=& 4 \left( \frac{\partial f(E-V)}{\partial k_B T} \right)^2 \left( \frac{k_B \Delta T}{2} \right)^2 + \left( \frac{2}{3!} \right)^2 \left( \frac{\partial^3 f(E-V)}{\partial (k_B T)^3} \right)^2  \left( \frac{k_B \Delta T}{2} \right)^6 + 2  \frac{4}{3!} \frac{\partial f(E-V)}{\partial k_B T}  \frac{\partial^3 f(E-V)}{\partial (k_B T)^3} \left( \frac{k_B \Delta T}{2} \right)^4   \nonumber \\
&=&  4 \left( k_B \Bar{T} \frac{\partial f(E-V)}{\partial k_B T} \right)^2 \left( \frac{\Delta T}{2 \Bar{T}} \right)^2 + \frac{4}{3} \left(k_B \Bar{T} \right)^4 \frac{\partial f(E-V)}{\partial k_B T}  \frac{\partial^3 f(E-V)}{\partial (k_B T)^3} \left( \frac{\Delta T}{2 \Bar{T}} \right)^4 +  O\left( \frac{\Delta T}{2 \Bar{T}} \right)^6,
\end{eqnarray}
where, $f(E-V)=1/(1+e^{\frac{E-V}{k_B \Bar{T}}})$ and $\left. \frac{\partial f(E-V)}{\partial k_B T} = \frac{\partial f(E-V)}{\partial k_B T} \right|_{T=\Bar{T}}$. This Fermi function expansion for $\Delta_{Tsh}$ noise in power series of $\left( \frac{\Delta T}{2 \Bar{T}} \right)$ aligns with Fermi function expansion of $\Delta_{Tsh}$ noise as derived in Ref. \cite{dTtheory} where they consider temperature gradient and voltage bias similar to our case.
We adopt the approximation that $\frac{\Delta T}{2 \Bar{T}}\ll 1$, and to simplify the calculation and ensure that the results remain unaffected, we include terms up to $\left( \frac{\Delta T}{2 \Bar{T}} \right)^4$, while disregarding higher-order terms $ O\left( \frac{\Delta T}{2 \Bar{T}} \right)^6$. Spin polarised charge $\Delta^{ch, \sigma \sigma^{\prime}}_T$ shot noise-like contribution can be then derived as
\begin{eqnarray}
\Delta^{ch, \sigma \sigma^{\prime}}_{Tsh} &=& \frac{2 e^2}{h} \int^{\infty}_{-\infty} F^{\sigma \sigma^{\prime}}_{11sh}\left( E \right) (f_{1} (E-V,k_B T_1) - f_2(E-V,k_B T_2))^2  dE \nonumber \\
&=& \frac{2 e^2}{h} \int^{\infty}_{-\infty}  F^{\sigma \sigma^{\prime}}_{11sh}\left( E \right) \biggl\{ 4 \left( k_B \Bar{T} \frac{\partial f(E-V)}{\partial k_B T} \right)^2 \left( \frac{\Delta T}{2 \Bar{T}} \right)^2 + \frac{4}{3} \left(k_B \Bar{T} \right)^4 \left( \frac{\partial f(E-V)}{\partial k_B T}  \frac{\partial^3 f(E-V)}{\partial (k_B T)^3} \right) \left( \frac{\Delta T}{2 \Bar{T}} \right)^4 \biggr\} dE,
\label{DTsh_ch}
\end{eqnarray}
where $\sigma, \sigma^{\prime} \in \{ \uparrow, \downarrow \}$, and the scattering term $F^{\sigma \sigma^{\prime}}_{11sh}$ is given in Eq. (\ref{B_Ssh}).

We calculate the $\Delta^{ch}_{Tsh}$ by separately calculating its no-flip ($\Delta^{ch, \uparrow \uparrow}_{Tsh}$ and $\Delta^{ch, \downarrow \downarrow}_{Tsh}$) and spin-flip contributions ($\Delta^{ch, \uparrow \downarrow}_{Tsh}$ and $\Delta^{ch, \downarrow \uparrow}_{Tsh}$) in each spin-configuration (1-4), see Fig. \ref{fig:avgm}. We replace the bias ($V$) in Eq. (\ref{DTsh_ch}) with the charge thermovoltage value ($V^{s1}_{ch1},V^{s1}_{ch2},V^{s1}_{ch3},V^{s1}_{ch4}$) for respective spin-configurations. By summing over the contributions from all spin configurations, we obtain the same-spin and opposite-spin contributions to the shot noise-like $\Delta^{ch}_{Tsh}$. Finally, we sum the same-spin and opposite-spin contributions to obtain the total $\Delta^{ch}_{Tsh}$ (see, Eq. (\ref{App:Deltasfsame})).

Similarly, spin polarised charge and spin $\Delta^{\beta, \sigma \sigma^{\prime}}_{Tth}$ thermal noise-like contribution to $\Delta^{\beta, \sigma \sigma^{\prime}}_T$ noise can be calculated from quantum thermal noise in Eq.~(\ref{B_Sth}) at vanishing charge and spin current by applying the temperature gradient and zero bias. Fermi functions in the $\Delta_T$ thermal noise-like expansion in power series of $\frac{\Delta T}{2\Bar{T}}$, with the approximation $\frac{\Delta T }{2 \Bar{T}} \ll 1$ is
\begin{eqnarray}
f_{1} (E-V,k_B T_1) &-& f_1(E-V,k_B T_1)^2 + f_{2} (E-V,k_B T_2) - f_2(E-V,k_B T_2)^2 = - k_B T_1 \frac{\partial f_1(E-V,k_B T_1)}{\partial E} - k_B T_2 \frac{\partial f_2(E-V,k_B T_2)}{\partial E} \nonumber \\
&=& - 2 k_B \Bar{T} \frac{\partial f(E-V)}{\partial E} - \frac{\partial}{\partial E} \left(\frac{ \partial^2 k_B T f(E-V)}{\partial (k_B T)^2} \right)_{T=\Bar{T}}  \left( \frac{ k_B \Delta T}{2} \right)^2 -  \frac{1}{3} \frac{\partial}{\partial E} \left( \frac{\partial^3 f(E-V)}{\partial (k_B T)^3}  + \frac{k_B T}{4}\frac{\partial^4 f(E-V)}{\partial (k_B T)^4} \right)_{T=\Bar{T}} \left( \frac{k_B \Delta T}{2} \right)^4  \nonumber\\
&=& - \Biggl[ 2 k_B \Bar{T} \frac{\partial f(E-V)}{\partial E} + (k_B \Bar{T})^2 \biggl\{ 2 \frac{\partial}{\partial E} \left( \frac{\partial f(E-V)}{ \partial k_B T}  \right) + k_B \Bar{T} \frac{\partial}{\partial E} \left( \frac{\partial^2 f(E-V)}{ \partial (k_B T)^2}  \right) \biggr\} \left( \frac{\Delta T}{2 \Bar{T}} \right)^2  +  \frac{( k_B\Bar{T})^4}{3} \Biggl\{ \frac{\partial}{\partial E} \left( \frac{\partial^3 f(E-V)}{ \partial (k_B T)^3}  \right) \nonumber \\
&&~ + \frac{k_B \Bar{T}}{4} \frac{\partial}{\partial E} \left( \frac{\partial^4 f(E-V)}{ \partial (k_B T)^4}  \right) \Biggr\} \left( \frac{\Delta T}{2 \Bar{T}} \right)^4 \Biggr] +  O\left( \frac{\Delta T}{2 \Bar{T}} \right)^6.
\end{eqnarray}
Charge $\Delta^{ch, \sigma \sigma^{\prime}}_T$ thermal noise-like contribution can be then derived as,
\begin{eqnarray}
\Delta^{ch, \sigma \sigma^{\prime}}_{Tth} &=& - \frac{2 e^2}{h}\int^{\infty}_{-\infty}  F^{\sigma \sigma^{\prime}}_{11th}\left( E \right) \Biggl[ 2 k_B \Bar{T} \frac{\partial f(E-V)}{\partial E} + (k_B \Bar{T})^2 \biggl\{ 2 \frac{\partial}{\partial E} \left( \frac{\partial f(E-V)}{ \partial k_B T}  \right) + k_B \Bar{T} \frac{\partial}{\partial E} \left( \frac{\partial^2 f(E-V)}{ \partial (k_B T)^2}  \right) \biggr\} \left( \frac{\Delta T}{2 \Bar{T}} \right)^2  + \frac{(k_B \Bar{T})^4}{3}  \nonumber \\
&&~  \Biggl\{ \frac{\partial}{\partial E} \left( \frac{\partial^3 f(E-V)}{ \partial (k_B T)^3}  \right) + \frac{k_B \Bar{T}}{4} \frac{\partial}{\partial E} \left( \frac{\partial^4 f(E-V)}{ \partial (k_B T)^4}  \right) \Biggr\} \left( \frac{\Delta T}{2 \Bar{T}} \right)^4 \Biggr] dE,
\label{DTth_ch}
\end{eqnarray}
with $\sigma, \sigma^{\prime} \in \{ \uparrow, \downarrow \}$, and the scattering term $F^{\sigma \sigma^{\prime}}_{11th}$ is given in Eq. (\ref{B_Sth}).

We calculate $\Delta^{ch}_{Tth}$ in a way similar to shot noise-like by separately calculating its no-flip ($\Delta^{ch,\uparrow \uparrow}_{Tth}$ and $\Delta^{ch,\downarrow \downarrow}_{Tth}$) and spin-flip contributions ($\Delta^{ch,\uparrow \downarrow}_{Tth}$ and $\Delta^{ch,\downarrow \uparrow}_{Tth}$) in each spin-configuration (1-4), see, Fig. \ref{fig:avgm}. For each spin-configuration, we replace the value of $V$ in Eq. (\ref{DTth_ch}) with the corresponding charge thermovoltage ($V^{s1}_{ch1},V^{s1}_{ch2},V^{s1}_{ch3},V^{s1}_{ch4}$) to obtain the no-flip and spin-flip contributions. By summing over the contributions from all spin configurations, we obtain the same-spin and opposite-spin contributions to the shot noise-like $\Delta^{ch}_{Tth}$. Finally, we add the same-spin and opposite-spin contributions to obtain the total $\Delta^{ch}_{Tth}$, see Eq. (\ref{App:Deltasfsame}).


Similarly, spin polarised spin $\Delta^{sp, \sigma \sigma^{\prime}}_T$ shot noise-like contribution can be then derived as
\begin{eqnarray}
\Delta^{sp, \sigma \sigma^{\prime}}_{Tsh} &=& \frac{2 e^2}{h} \int^{\infty}_{-\infty}  F^{\sigma \sigma^{\prime}}_{11sh}\left( E \right) \biggl\{ 4 \left( k_B \Bar{T} \frac{\partial f(E-V)}{\partial k_B T} \right)^2 \left( \frac{\Delta T}{2 \Bar{T}} \right)^2 + \frac{4}{3} \left(k_B \Bar{T} \right)^4 \left( \frac{\partial f(E-V)}{\partial k_B T}  \frac{\partial^3 f(E-V)}{\partial (k_B T)^3} \right) \left( \frac{\Delta T}{2 \Bar{T}} \right)^4 \biggr\} dE,
\label{DTsh_sp}
\end{eqnarray}
where $\sigma, \sigma^{\prime} \in \{ \uparrow, \downarrow \}$, and the scattering term $F^{\sigma \sigma^{\prime}}_{11sh}$ is given in Eq. (\ref{B_Ssh}).


Spin $\Delta^{sp, \sigma \sigma^{\prime}}_T$ thermal noise-like contribution can be then derived as,
\begin{eqnarray}
\Delta^{sp, \sigma \sigma^{\prime}}_{Tth} &=& - \frac{2 e^2}{h} \int^{\infty}_{-\infty}  F^{\sigma \sigma^{\prime}}_{11th}\left( E \right) \Biggl[ 2 k_B \Bar{T} \frac{\partial f(E-V)}{\partial E} + (k_B \Bar{T})^2 \biggl\{ 2 \frac{\partial}{\partial E} \left( \frac{\partial f(E-V)}{ \partial k_B T}  \right) + k_B \Bar{T} \frac{\partial}{\partial E} \left( \frac{\partial^2 f(E-V)}{ \partial (k_B T)^2}  \right) \biggr\} \left( \frac{\Delta T}{2 \Bar{T}} \right)^2  + \frac{(k_B \Bar{T})^4}{3}  \nonumber \\
&&~  \Biggl\{ \frac{\partial}{\partial E} \left( \frac{\partial^3 f(E-V)}{ \partial (k_B T)^3}  \right) + \frac{k_B \Bar{T}}{4} \frac{\partial}{\partial E} \left( \frac{\partial^4 f(E-V)}{ \partial (k_B T)^4}  \right) \Biggr\} \left( \frac{\Delta T}{2 \Bar{T}} \right)^4 \Biggr] dE,
\label{DTth_sp}
\end{eqnarray}
with $\sigma, \sigma^{\prime} \in \{ \uparrow, \downarrow \}$, and the scattering term $F^{\sigma \sigma^{\prime}}_{11th}$ is given in Eq. (\ref{B_Sth}).


In order to calculate spin $\Delta^{sp}_{Tsh}$ $(\Delta^{sp}_{Tth})$, we first calculate separately its no-flip ($\Delta^{sp,\uparrow \uparrow}_{Tsh} (\Delta^{sp,\uparrow \uparrow}_{Tth})$ and $\Delta^{sp,\downarrow \downarrow}_{Tsh} (\Delta^{sp,\downarrow \downarrow}_{Tth})$) and spin-flip ($\Delta^{sp,\uparrow \downarrow}_{Tsh} (\Delta^{sp,\uparrow \downarrow}_{Tth})$ and $\Delta^{sp,\downarrow \uparrow}_{Tsh}(\Delta^{sp,\downarrow \uparrow}_{Tth})$) contributions in each spin-configuration, see, Fig. \ref{fig:avgm}. We accomplish this by replacing the value of $V$ in Eqs. (\ref{DTsh_sp}) and (\ref{DTth_sp}) with the corresponding spin thermovoltage ($V^{s1}_{sp1},V^{s1}_{sp2},V^{s1}_{sp3},V^{s1}_{sp4}$) for vanishing spin-current in each configuration. This procedure to calculate same-spin and opposite-spin correlations are analogous to that used to calculate charge $\Delta^{ch}_{Tsh}$ (or $\Delta^{ch}_{Tth}$). Total shot noise-like $\Delta^{sp}_{Tsh}$ (or thermal noise-like $\Delta^{sp}_{Tth}$) can be calculated by subtracting the opposite-spin correlation contribution $\Delta^{sp,op}_{Tsh}$ ($\Delta^{sp,op}_{Tth}$) from same-spin correlation contribution $\Delta^{sp,sa}_{Tth}$ ($\Delta^{sp,sa}_{Tth}$), as given in Eq. (\ref{App:Deltasfsame}). 


In a NIN junction, $\Delta^{NIN}_T$ shot noise-like contribution can be then derived as
\begin{eqnarray}
\Delta^{NIN}_{Tsh} &=& \frac{2 e^2}{h} \int^{\infty}_{-\infty}  F^{N}_{11sh}\left( E \right) \biggl\{ 4 \left( k_B \Bar{T} \frac{\partial f(E-V)}{\partial k_B T} \right)^2 \left( \frac{\Delta T}{2 \Bar{T}} \right)^2 + \frac{4}{3} \left(k_B \Bar{T} \right)^4 \left( \frac{\partial f(E-V)}{\partial k_B T}  \frac{\partial^3 f(E-V)}{\partial (k_B T)^3} \right) \left( \frac{\Delta T}{2 \Bar{T}} \right)^4 \biggr\} dE,
\label{DTsh_NIN}
\end{eqnarray}
where the scattering term $F^{N}_{11sh}= \mathcal{R}^N ~ \mathcal{T}^N= |s^{N}_{11}|^2 ~ |s^{N}_{12}|^2$ is given in Eq. (\ref{B_NN}).

$\Delta^{NIN}_T$ thermal noise in a NIN junction can be then derived as,
\begin{eqnarray}
\Delta^{NIN}_{Tth} &=& - \frac{2 e^2}{h} \int^{\infty}_{-\infty} F^{N}_{11th} \left( E \right) \Biggl[ 2 k_B \Bar{T} \frac{\partial f(E-V)}{\partial E} + (k_B \Bar{T})^2 \biggl\{ 2 \frac{\partial}{\partial E} \left( \frac{\partial f(E-V)}{ \partial k_B T}  \right) + k_B \Bar{T} \frac{\partial}{\partial E} \left( \frac{\partial^2 f(E-V)}{ \partial (k_B T)^2}  \right) \biggr\} \left( \frac{\Delta T}{2 \Bar{T}} \right)^2  + \frac{(k_B \Bar{T})^4}{3}  \nonumber \\
&&~  \Biggl\{ \frac{\partial}{\partial E} \left( \frac{\partial^3 f(E-V)}{ \partial (k_B T)^3}  \right) + \frac{k_B \Bar{T}}{4} \frac{\partial}{\partial E} \left( \frac{\partial^4 f(E-V)}{ \partial (k_B T)^4}  \right) \Biggr\} \left( \frac{\Delta T}{2 \Bar{T}} \right)^4 \Biggr] dE,
\label{DTth_NIN}
\end{eqnarray}
with the scattering term $F^{N}_{11th} =\mathcal{T}^N= |s^{N}_{12}|^2$ is given in Eq. (\ref{B_NN}).


Same-spin and opposite-spin correlations are not applicable in a NIN junction without any spin-flip scattering. In the absence of spin-flip scattering in the NIN junction, the contributions from spin configurations 2 and 3 vanish. On the other hand, spin-configurations 1 and 4 produce identical results with thermovoltage $V^{s1}_{N}$, and their sum is referred to as the total $\Delta^{NIN}_{T}$ noise.

 

\end{widetext}


\begin{thebibliography}{}
\bibitem{qnoise}
C. Beenakker and C. Sch\"{o}nenberger, Quantum Shot Noise, Phys. Today 56, 5, 37 (2003).

\bibitem{II}
J. Wei and V. Chandrasekhar, Positive noise cross-correlation in hybrid superconducting and normal-metal three-terminal devices, Nature Physics 6, 494 (2010); M. Hashisaka, T. Ota, M. Yamagishi, T. Fujisawa, and K. Muraki, Cross-correlation measurement of quantum shot noise using homemade transimpedance amplifiers, Rev Sci Instrum 85, 054704 (2014); I. Neder, et. al., Interference between two indistinguishable electrons from independent sources, Nature 448, 333 (2007).

\bibitem{fraction}
M. Hashisaka, T. Ota, K. Muraki, and T. Fujisawa, Shot-Noise Evidence of Fractional Quasiparticle Creation in a Local Fractional Quantum Hall State Phys. Rev. Lett. PRL 114, 056802 (2015); R. de-Picciotto, et. al., Direct observation of a fractional charge, Nature 389, 162 (1997).

\bibitem{noise}
Ya M. Blanter, and M. B\"{u}ttiker, Shot Noise in Mesoscopic Conductors, Phys. Reports 336, 1 (2000).

\bibitem{noise1}
M. B\"{u}ttiker, Scattering theory of current and intensity noise correlations in conductors and waveguides, Phys. Rev. B 46, 12485 (1992).

\bibitem{noises}
T. Martin, Nano phys.: Coherence and Transport, Les Houches Session LXXXI, edited by H. Bouchiat et. al., 283 (Elsevier 2005).

\bibitem{thermalnoise}
L. P. Kouwenhoven, G. Sch\"{o}n, and L.L. Sohn, Introduction to mesoscopic electron transport, NATO ASI Series E, Kluwer Academic Publishing, Dordrecht, 225, 345 (1997).

\bibitem{NSN}
R. Melin, C. Benjamin and T. Martin, Positive cross-correlations of noise in superconducting hybrid structures: Roles of interfaces and interactions, Phys. Rev. B 77, 094512 (2008).

\bibitem{shotnoise}
M. Henny, et. al., The Fermionic Hanbury Brown and Twiss Experiment, Science 284, 296 (1999).

\bibitem{qcircuitexpt}
E. Sivre, et. al., Electronic heat flow and thermal shot noise in quantum circuits, Nat. Comm. 10, 5638 (2019).

\bibitem{tjunctionexpt}
S. Larocque, et. al., Shot Noise of a Temperature-Biased Tunnel Junction, Phys. Rev. Lett. 125, 106801 (2020).

\bibitem{dTtheory}
J. Rech, et. al., Negative Delta-T Noise in the Fractional Quantum Hall Effect, Phys. Rev. Lett. 125, 086801 (2020); M. Hasegawa and K. Saito, Delta-T noise in the Kondo regime, Phys. Rev. B 103, 045409 (2021).

\bibitem{generalbound}
J. Eriksson, et. al., General Bounds on Electronic Shot Noise in the Absence of Currents, Phys. Rev. Lett. 127, 136801 (2021).

\bibitem{atomicscaleexpt}
O. S. Limbroso, et. al., Electronic noise due to temperature differences in atomic-scale junctions, Nature 562, 240 (2018).

\bibitem{popoff}
A. Popoff, et. al., Scattering theory of non-equilibrium noise and delta T current fluctuations through a quantum dot, J. Phys.: Condens. Matter 34, 185301 (2022).

\bibitem{generalbound1}
L. Tesser, et. al., Charge, spin, and heat shot noises in the absence of average currents: Conditions on bounds at zero and finite frequencies, Phys. Rev. B 107, 075409 (2023).

\bibitem{qstat}
G. Zhang, I. V. Gornyi and C. Spaanslatt, Delta-T noise for weak tunneling in one-dimensional systems: Interactions versus quantum statistics, Phys. Rev. B 105, 195423 (2022).

\bibitem{qint}
G. Rebora, et al., Delta-T noise for fractional quantum Hall states at different filling factor, Phys. Rev. Res. 4, 043191 (2022).

% \bibitem{spincor}
% Y. He, D. Hou, and R. Han, Spin-current shot noise in mesoscopic conductors, J. Appl. Phys. 101, 023710 (2007).

\bibitem{spinflipper}
O. L. T. de Menezes and J. S. Helman, Spin flip enhancement at resonant transmission, American Journal of Phys. 53, 1100 (1985); S. Pal and C. Benjamin, Spin-flip scattering engendered quantum spin torque in a Josephson junction, Proc. Roy. Soc. A 475, 20180775 (2019).

\bibitem{datta}
M. P. Anantram and S. Datta, Current fluctuations in mesoscopic systems with Andreev scattering, Phys. Rev. B 53, 16390 (1996).

\bibitem{spinnoise}
O. Sauret and D. Feinberg, Spin-Current Shot Noise as a Probe of Interactions in Mesoscopic Systems, Phys. Rev. Lett. 92, 106601 (2004).

% \bibitem{suppl}
% See Supplemental Material at " " for a detailed derivation of thermovoltage and $\Delta_T$.

\bibitem{lowtemp}
Yu Wing Wa, Low temperature spintronics: probing charge and spin states with two-dimensional
electron gas, M.Sc. Thesis, University of British Columbia (2009).

% \bibitem{github} Mathematica notebook for the $\Delta_{T}$ noise calculation and plots  can be found in \href{https://github.com/TUSARADRI/Delta-T-noise-in-NsfN-junction}{$\Delta_{T}$ noise code}.

\bibitem{spinflip}
A Dargys, Boundary conditions and transmission reflection of electron spin in a quantum well, Semicond. Sci. Technol. 27, 045009 (2012).

\bibitem{kondo}
A. V. Balatsky, I. Vekhter, and Jian-Xin Zhu, Impurity-induced states in conventional and unconventional superconductors, Rev. Mod. Phys. 78, 373 (2006).

\bibitem{kondo1}
L. Kouwenhoven and L. Glazman, Revival of the Kondo effect, Phys. World 14, 33 (2001).

\bibitem{Cdot}
S. J. Chorley, et. al., Transport Spectroscopy of an Impurity Spin in a Carbon Nanotube Double Quantum Dot, Phys. Rev. Lett. 106, 206801 (2011).

\bibitem{NIS}
T. Mohapatra and C. Benjamin, Andreev reflection mediated $\Delta_T$ noise, Manuscript under preparation.


\end{thebibliography}


\end{document}
