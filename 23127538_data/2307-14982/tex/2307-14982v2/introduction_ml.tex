Multilayer networks have become a recent focal point of research in the field of statistical network analysis  
\citep[e.g.,][]{LeChLy20, CaGo20, arroyo_multi, KrKoMa20, Mac21, chen2022, sosa2022, huang2022},
arising in applications where a common set of elements in a population
interact through multiple modes or relationships with other elements in the population.
A prototypical example in the literature might be the Lazega law firm network \citep{Lazega2001}, 
in which attorneys are linked through various forms of interaction,
such as advice seeking, friendship, collaboration, etc.,
each of which would form a distinct layer in the multilayer network  \citep{KrKoMa20}.  
In essence, a multilayer network is a composite structure, where each layer captures a specific type of interaction or relationship between the same set of elements.


Edges in one layer of the multilayer network may depend on edges in other layers, 
creating what is known as cross-layer dependence. 
Understanding the drivers of edge formation in multilayer networks requires 
learning the dependence structures across these layers. 
A key challenge lies in the fact that the cross-layer dependence can be highly varied and complex,
and the development of statistical models with theoretical guarantees for network data with dependent edges 
is challenging.  
Current methodological frameworks for multilayer networks can be broadly categorized into two main groups:
\ben
\item Statistical models equipped with theoretical guarantees often rely on latent variable constructions \cite[e.g.,][]{Mac21, arroyo_multi, huang2022}. These models typically assume conditional independence of edges given the latent variables, following standard practices within the field.
\item Statistical models that do not provide formal theoretical guarantees \citep[e.g.,][]{CaGo20, KrKoMa20}. Instead, these methods extend existing approaches by explicitly allowing for edge dependencies, thereby relaxing the conditional independence assumptions present in the first class of models.
\een
In this work, 
we address a critical gap in the literature by introducing a separable multilayer network modeling framework for multilayer networks.
Our approach not only accommodates dependent edges but also provides theoretical guarantees for both estimation and inference without relying on any latent variables.
Specifically, we extend single-layer network models to the multilayer setting, with a central focus on identifying and understanding cross-layer dependence structures.
A key advantage of our proposed framework is that we are able to distinguish the network formation process  
from the layer formation process.
This allows us to create a wide range of novel multilayer network models derived from established single-layer network models, such as
exponential-family random graph models, stochastic block models, and latent space models.
By employing Markov random field specifications, we develop adaptable and comprehensive models to capture cross-layer dependencies in multilayer networks. 
As a result, 
our framework jointly models both network structures and cross-layer dependence, thus enabling any single-layer network model to be extended to the multilayer setting. 
Our main contributions in this work include: 
\ben%[topsep=.1em,itemsep=.1em]
\item Introducing a novel framework for modeling 
cross-layer dependence in multilayer networks 
that synchronizes with current network models in the literature.
\item Deriving non-asymptotic theoretical guarantees in scenarios where the number of parameters tends to infinity, 
which establishes bounds on the:  
\ben
\item Statistical error of maximum likelihood estimators.  
\item Error of the multivariate normal approximation of estimators.
\een
\item Elaborating a model selection algorithm which controls the false discovery rate. 
\een 


