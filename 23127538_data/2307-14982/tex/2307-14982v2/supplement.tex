

\pagebreak

\appendix

\makeatletter

\setcounter{page}{1}

\setcounter{section}{0}

\setcounter{com}{0}



\begin{center}
\Large\bf\textsc\bf
Supplement:\\
Learning cross-layer dependence structure in multilayer networks \s\s

\normalsize
{\normalfont\textsc{By Jiaheng Li and Jonathan R. Stewart}} \s 
\\
{\normalfont\em Department of Statistics, Florida State University} \s\s
\end{center}

\s\s


\noindent
{Appendix \ref{sec:prop_proof}: Proof of Proposition \ref{prop:inference}}\dotfill\pageref{sec:prop_proof}\s\\ 
{Appendix \ref{sec:lem1_proof}: Proof of Lemma \ref{lem:min-eig}}\dotfill\pageref{sec:lem1_proof}\s\\
{Appendix \ref{sec:concentration}: Concentration inequalities for multilayer networks}\dotfill\pageref{sec:concentration}\s\\
{Appendix \ref{sec:pf_thm1}: Proof of Theorem \ref{thm1} and Corollary \ref{corollary}}\dotfill\pageref{sec:pf_thm1}\s\\
{Appendix \ref{sec:pf_minimax}: Proof of Theorem \ref{thm:minimax} and Corollary \ref{cor:minimax}}\dotfill\pageref{sec:pf_thm1}\s\\
{Appendix \ref{sec:pf_prop2}: Proposition \ref{prop:suff_norm} and proof}\dotfill\pageref{sec:pf_prop2}\s\\
{Appendix \ref{sec:pf_thm2}: Proof of Theorem \ref{thm2}}\dotfill\pageref{sec:pf_thm2}\s\\
{Appendix \ref{sec:add_sim}: Additional simulation results}\dotfill\pageref{sec:add_sim}\s\\ 


\section{Proof of Proposition \ref{prop:inference}} 
\label{sec:prop_proof} 

We prove Proposition \ref{prop:inference} from Section \ref{sec2}. 
\pproof \ref{prop:inference}.
For the first and second results,
define the set
\beno
\mA_{+}
&\coloneqq& \left\{ (\bx, \by) \in \mbX \times \mbY \;:\; h(\bx, \by) = 1 \right\},
\ee
and the vector-valued map $\bm{\varphi} : \mbX \mapsto \mbY$ by defining its components to be
\beno
\varphi_{i,j}(\bx)
\= \one(\norm{\bx_{i,j}}_1 > 0),
&& \{i,j\} \subset \mN, 
\ee
populating the vector $\bm{\varphi}(\bx)$ in the lexicographical ordering of the dyad indices $\{i,j\} \subset \mN$.  
By the definition of $h : \mbX \times \mbY \mapsto \{0, 1\}$ and $\bm{\varphi} : \mbX \mapsto \mbY$,
$\bm{\varphi}(\bx)  = \by$ for each pair $(\bx, \by) \in \mA_{+}$.  
Furthermore, 
the element $\by$ is unique for a given $\bx \in \mbX$,
because if there would exists some $\by^\prime \in \mbY$ 
such that $\by \neq \by^\prime$ 
with the property that 
$\{(\bx, \by), (\bx, \by^\prime)\} \subseteq \mA_{+}$, 
then there would exist a  pair $\{i,j\} \subset \mN$
such that $y_{i,j} = 1 - y^\prime_{i,j}$,
implying
$\one(\norm{\bx_{i,j}}_1 > 0) \neq y^\prime_{i,j}$,
in which case $h(\bx, \by^\prime) = 0$,
contradicting the assumption that $\{(\bx, \by)\} \in \mA_{+}$. 
By \eqref{general model},
the functions $f$ and $g$ are assumed to be strictly positive in their respective domains.
Hence, 
$(\mbX \times \mbY) \setminus \mA_{+}$ is the largest null set of $\mbX \times \mbY$,
i.e.,
$\sepmodel(\mA) = 0$ if and only if $\mA \subseteq (\mbX \times \mbY) \setminus \mA_{+}$.
Thus, 
the first and second results are established. 


For the third result,
note that $g$ is assumed to be strictly positive on its domain $\mbY$. 
Hence, 
$g(\by) = \mbP_{\nat}(\bY = \by) > 0$ for all $\by \in \mbY$
and  
$\sepmodel(\bX = \bx \,|\, \bY = \by)$
is therefore well-defined.
By definition,
\beno
\sepmodel(\bX = \bx \,|\, \bY = \by)
\= \dfrac{\sepmodel(\bX = \bx, \, \bY = \by)}{\sepmodel(\bY = \by)},
\ee
where $\sepmodel(\bY = \by)$ is the marginal probability of event $\bY = \by$ and is assumed to be equal to $g(\by)$.
The model form for $\sepmodel$ given in \eqref{general model} implies 
\beno
\dfrac{\sepmodel(\bX = \bx, \, \bY = \by)}{\sepmodel(\bY = \by)}
\= \dfrac{f(\bx, \nat) \, g(\by) \, \psi(\nat, \by)}{g(\by)} 
\= \exp(\log f(\bx, \nat) + \log \psi(\nat, \by)),
\ee
under the assumption that $h(\bx, \by) = 1$.
Hence,
\beno
\sepmodel(\bX = \bx, \bY = \by)
\= \sepmodel(\bX = \bX \;|\; \bY = \by) \; \sepmodel(\bY = \by)
\ee
so that
\beno
\log \, \sepmodel(\bX = \bx, \bY = \by)
\= \log \, \sepmodel(\bX = \bX \;|\; \bY = \by) + \log \, g(\by),
\ee
as $g(\by)$ is the marginal probability mass function of $\bY$,
i.e.,
$\sepmodel(\bY = \by) = g(\by)$. 
Lemma \ref{lem:s_hetero} establishes that $\sepmodel(\bX = \bX \;|\; \bY = \by)$ belongs to a minimal exponential family,
completing the proof of the third and last result of the proposition.  

\qed 

\s\s


\section{Proof of Lemma \ref{lem:min-eig}}
\label{sec:lem1_proof}

We prove Lemma \ref{lem:min-eig} from Section \ref{sec2}.
Using \eqref{general_model},  
\beno
-\mbE \, \nabla_{\nat}^2 \, \ell(\nat; \bX, \bY)
\= \dsum_{\by \in \mbY} \, \dsum_{\bx \in \mbX} \, - \nabla_{\nat}^2 \, \ell(\nat; \bx, \by) \, 
\sepmodel(\bX = \bx \,|\, \bY = \by) \, g(\by) \s \\ 
\= \dsum_{\by \in \mbY} \, g(\by)  \dsum_{\bx \in \mbX} \, - \nabla_{\nat}^2 \, \ell(\nat; \bx, \by) \, 
\sepmodel(\bX = \bx \,|\, \bY = \by) \s \\ 
\= \dsum_{\by \in \mbY} \, g(\by) \, \dsum_{\{i,j\} \subset \mN \,:\, y_{i,j} = 1} \, \mcI(\nat) \s \\ 
\= \mcI(\nat) \, \dsum_{\by \in \mbY} \, g(\by)  \, \norm{\by}_1 \s\\
\= \mcI(\nat) \, \mbE \norm{\bY}_1.  
\ee
The above follows by 
exploiting the conditional independence of vectors $\bx_{i,j}$ ($\{i,j\} \subset \mN$) 
given $\bY = \by$ under \eqref{general_model}, 
which implies 
\beno
\ell(\nat; \bx, \by)
\= \dsum_{\{i,j\} \subset \mN} \log \sepmodel(\bX_{i,j} = \bx_{i,j} \,|\, \bY = \by), 
\ee
and from the fact that the conditional probability distribution of $\bX_{i,j}$ 
given $\bY$ 
is a degenerate point mass at $\bm{0}$ 
when $Y_{i,j} = 0$ so that $- \nabla_{\nat}^2 \, \ell(\nat; \bx, \by)$ is a sum of $\norm{\by}_1$ 
matrices each equal to $\mcI(\nat)$,
i.e.,
given $\by \in \mbY$,
we have 
\beno
&& \dsum_{\bx \in \mbX} - \nabla_{\nat}^2 \, \ell(\nat; \bx, \by) \, \sepmodel(\bX = \bx \,|\, \bY = \by) \s\\ 
\= \dsum_{\{i,j\} \subset \mN} \, \mbE\left[- \nabla_{\nat}^2 \, L_{i,j}(\nat, \bX_{i,j}, \bY) \,|\, \bY = \by \right]  
\;\;= \dsum_{\{i,j\} \subset \mN \,:\, y_{i,j} = 1} \, \mcI(\nat). 
\ee
The fact that $\mcI(\nat)$ is constant for all pairs $\{i,j\} \subset \mN$ satisfying $Y_{i,j} = 1$ 
follows from the form of \eqref{general_model},
which assumes each vector $\bX_{i,j}$ ($\{i,j\} \subset \mN$)
is conditionally independent and identically distributed, 
conditional on $\bY$.  
Hence, 
\beno
\mbE \left[ -\nabla_{\nat}^2 \, \ell(\nat; \bX, \bY)  \right]
\= \mcI(\nat) \, \mbE \, \norm{\bY}_1,
\ee 
which in turn implies 
\beno
\lambda_{\min}(-\mbE \nabla_{\nat}^2 \,\ell(\nat; \bX, \bY)) = \lambda_{\min}(\mcI(\nat)) \, \mbE \, \norm{\bY}_1\s\\
\lambda_{\max}(-\mbE \nabla_{\nat}^2 \,\ell(\nat; \bX, \bY)) = \lambda_{\max}(\mcI(\nat)) \, \mbE \, \norm{\bY}_1.
\ee

\qed 

\s


\section{Concentration inequalities for multilayer networks} 
\label{sec:concentration}


We establish the concentration inequality of gradients of log-likelihood functions 
of multilayer networks in Lemma \ref{lem:concentration_likelihood}.
Recall the definition $[D_{g}]^{+} \coloneqq \max\{0, \, D_{g}\}$, 
where  
\beno 
D_{g}
&\coloneqq& \dsum_{\{i,j\} \prec \{v,w\} \subset \mN} \, \cov(Y_{i,j}, \, Y_{v,w}), 
\ee 
with $\{i,j\} \prec \{v,w\}$ implying the sum is taken with respect to the lexicographical ordering 
of pairs of nodes, 
and where $g : \mbY \mapsto (0, 1)$ is the marginal probability mass function of $\bY$.  

\s\s


\begin{lemma}
\label{lem:concentration_likelihood}
Consider a multilayer network model following the form of equation \eqref{general_model} and is 
defined on a set of $N \geq 3$ nodes and $K \geq 1$ layers. 
Define 
$\gradL \coloneqq - \nabla_{\nat} \, \ell(\nat; \bx, \by)$,  
where $\ell(\nat; \bx, \by)$ is the log-likelihood function. 
Then,
for all $t > 0$ and $\nat \in \mbR^p$, the probability
\beno
\mbP\left(\norm{\GradL - \mbE \, \GradL}_2 \geq t \right) 
\ee
is bounded above by
\beno
 \exp\left( -\dfrac{t^2}{36 \, \widetilde{\lambda}_{\max}^{\star} \, (\mbE \, \norm{\bY}_1 + [D_{g}]^{+}) \,  + 2 \, \sqrt{p} \, t} + \log \, p \right) + \dfrac{1}{\mbE \norm{\bY}_1}.
\ee
\end{lemma}

\llproof \ref{lem:concentration_likelihood}. 
By Proposition \ref{prop:inference},
\beno 
\ell(\nat;\bx,\by)
\= \log \, \mbP_{\nat}(\bX = \bx \,|\, \bY = \by)
+ \log \, g(\by).  
\ee
Thus, 
\be
\label{eq:887}
-\gradL
\= \nabla_{\nat} \, \log \, \mbP_{\nat}(\bX = \bx \,|\, \bY = \by)
 + \nabla_{\nat} \, \log \, g(\by) \s \\
\= s(\bx) - \mbE_{\nat} \, s(\bX),  
\ee
as $g(\by) = \mbP_{\nat}(\bY = \by)$ is assumed to not be a function of $\nat$. 
The last equation in \eqref{eq:887} follows from 
Lemma \ref{lem:s_hetero},
which showed that 
$\mbP_{\nat}(\bX = \bx \,|\, \bY = \by)$ is a minimal exponential family
with the natural parameter vector $\nat \in \mbR^p$ and the sufficient statistic vector $\bs(\bx)$ defined in Lemma \ref{lem:s_hetero},
inserting the familiar form of the score equation of an exponential family with respect to the natural parameter vector 
\citep[e.g., Proposition 3.10, p. 32,][]{Su19}.
Thus, 
\beno
- (\GradL   - \mbE \, \GradL ) 
\= \bs(\bX) - \mbE_{\nat} \, \bs(\bX) 
- \mbE \left[ \bs(\bX) - \mbE_{\nat} \, \bs(\bX)\right] \s \\
\= \bs(\bX) - \mbE \, \bs(\bX). 
\ee
Let $t > 0$ and $\nat \in \mbR^p$ be arbitrary and fixed 
and define  
$\mD_{2}(\nat,t)$ to be the event that 
$\norm{\GradL - \mbE \, \GradL}_{2}  \geq t$, i.e., 
\beno
\mD_{2}(\nat,t) \= \left\{ \bx \in \mbX \,:\, \norm{\bs(\bx)-\mbE  \, \bs(\bX)}_{2} \geq t \right\}.
\ee
Let $\epsilon > 0$ and define  
$\mE(\epsilon)$ to be the event that $|\norm{\bY}_1 - \mbE \norm{\bY}_1| \le \epsilon$,
i.e.,
\beno
\mE(\epsilon)
\= \left\{ \by \in \mbY \,:\, \left|\norm{\by}_1 - \mbE \norm{\bY}_1\right| \leq \epsilon \right\}. 
\ee 
We assume that $\epsilon > 0$ is chosen so that $\mE(\epsilon)$ is not empty, 
which implies $\mbP(\mE(\epsilon)) > 0$ 
as $g(\by)$ is assumed to be strictly positive on $\mbY$.  
By the law of total probability, 
\be
\label{divide and conquer}
\mbP\left(\mD_{2}(\nat,t) \right) 
&=& \mbP\left( \mD_{2}(\nat, t) \,|\, \mE(\epsilon) \right)\,
\mbP\left(\mE(\epsilon) \right) + 
\mbP\left( \mD_{2}(\nat,t) \,|\, \mE(\epsilon)^c \right) \, \mbP\left( \mE(\epsilon)^c \right) \s \\
&\leq& \mbP\left( \mD_{2}(\nat,t) \,|\, \mE(\epsilon) \right) + \mbP\left( \mE(\epsilon)^c \right).
\ee
Note that we have not necessarily guaranteed that $\mbP\left( \mE(\epsilon)^c \right) > 0$.  
However, 
if $\mbP\left( \mE(\epsilon)^c \right) = 0$ the non-conditional form of the law of total probability would yield the bound   
\beno
\mbP\left(\mD_{2}(\nat,t) \right)
&\leq&  \mbP\left( \mD_{2}(\nat,t) \,|\, \mE(\epsilon) \right),
\ee
which is strictly sharper than the bound we give in \eqref{divide and conquer}. 
We will use a divide-and-conquer strategy to bound each probability in \eqref{divide and conquer} in turn. 

To bound the first term in \eqref{divide and conquer}, let $\mU \coloneqq \{\bu \in \mbR^p \, : \, \norm{\bu}_2 \leq 1\}$ be a closed unit ball in $\mbR^p$. Define an $\epsilon$-net $\mV_\epsilon$ of $\mU \subset \mbR^p$. By Corollary 4.2.13 of \citep{Vershynin18}, there exists an $\epsilon$-net $\mV_\epsilon \subset \mU$ such that its cardinality satisfies $\log\, |\mV_\epsilon| \leq p \, \log \, (2\,\epsilon^{-1} + 1)$. Taking $\epsilon = 1/2$, for each $\bu \in \mU$, there exists a $\bv \in \mV_{1/2}$ such that $\norm{\bu - \bv}_2 \leq 1/2$, and by the Cauchy-Schwarz inequality, 
\be
\label{inner product of score eq}
\langle \, \bu \, , \, \nabla_{\nat} \, \ell(\nat;\bx,\by) \, \rangle \= \langle \, \bv \, ,\, \nabla_{\nat} \, \ell(\nat;\bx,\by) \,\rangle \, + \, \langle \, \bu - \bv \, ,\, \nabla_{\nat} \, \ell(\nat;\bx,\by)\,\rangle \s \\
& \leq & \langle \, \bu \, , \, \nabla_{\nat} \, \ell(\nat;\bx,\by) \, \rangle \, + \, \norm{\bu - \bv}_2 \, \norm{\nabla_{\nat} \, \ell(\nat;\bx,\by)}_2 \s \\
& \leq & \langle \, \bu \, , \, \nabla_{\nat} \, \ell(\nat;\bx,\by) \, \rangle \, + \, \dfrac{1}{2} \, \norm{\nabla_{\nat} \, \ell(\nat;\bx,\by) }_2.
\ee
If $\norm{\nabla_{\nat} \, \ell(\nat;\bx,\by) }_2 \neq 0$, we can choose 
\beno
u_i \= \dfrac{\nabla_{\nat} \, \ell(\nat;\bx,\by)_i}{\norm{\nabla_{\nat} \, \ell(\nat;\bx,\by) }_2},
\ee
so that $\norm{\bu}_2 \leq 1$ and $\bu \in \mU$. Next, re-write 
\beno
\langle \, \bu \, , \, \nabla_{\nat} \, \ell(\nat;\bx,\by) \, \rangle \= \dfrac{1}{\norm{\langle \, \bu \, , \, \nabla_{\nat} \, \ell(\nat;\bx,\by) \, \rangle }_2} \, \dsum_{i=1}^p \, (\nabla_{\nat} \, \ell(\nat;\bx,\by)_i)^2 \s \\
\= \norm{\nabla_{\nat} \, \ell(\nat;\bx,\by)}_2,
\ee
and together with \eqref{inner product of score eq}, we have
\be
\label{score ineq}
\norm{\nabla_{\nat} \, \ell(\nat;\bx,\by)}_2 \, \leq \, 2 \, \max\limits_{\bv \in \mV_{1/2}} \, \langle \, \bv \, , \, \nabla_{\nat} \, \ell(\nat;\bx,\by) \, \rangle. 
\ee
If $\norm{\nabla_{\nat} \, \ell(\nat;\bx,\by) }_2 = 0$, the inequality \eqref{score ineq} holds trivially. As a result of \eqref{score ineq}, for any $t > 0$,
\beno
\mbP\left(\mD_{2}(\nat,t) \,|\, \mE(\epsilon) \right) & \leq & \mbP \left(  2 \, \max\limits_{\bv \in \mV_{1/2}}  \, \langle \, \bv \, , \, \nabla_{\nat} \, \ell(\nat;\bx,\by) \, \rangle \, \ge \, t \right) \s \\
& \leq & \dsum_{\bv \in \mV_{1/2}} \, \mbP \left( \,\langle \, \bv \, , \, \nabla_{\nat} \, \ell(\nat;\bx,\by) \, \rangle \, \ge \,\dfrac{t}{2} \,  \right) \s \\
& \leq & \exp \, \left( p\, \log \, 5\right) \max\limits_{\bv \in \mV_{1/2}} \, \mbP \left( \,\langle \, \bv \, , \, \nabla_{\nat} \, \ell(\nat;\bx,\by) \, \rangle \, \ge \,\dfrac{t}{2} \,  \right). 
\ee
The last inequality is true because $\log\, |\mV_{1/2}| \leq p \, \log \, 5$. Note that
\be
\label{sum_over_dim}
\langle \, \bv \, , \, \nabla_{\nat} \, \ell(\nat;\bx,\by) \, \rangle  \= \dsum_{l=1}^p \, v_l \, [\nabla_{\nat} \, \ell(\nat;\bx,\by)]_l \s \\
\= \dsum_{l=1}^p \, v_l \, [s_l(\bx) - \mbE \, s_l(\bX)].
\ee
The form of \eqref{general_model} implies, 
through factorization principles,
that the dyad-based vectors $\bX_{i,j}$ ($\{i,j\} \subset \mN$) are conditionally independent given $\bY$ 
\citep[e.g.,][p. 11--13]{graphical_model_handbook}. 
Hence,
using Lemma \ref{lem:s_hetero},  
the components of the sufficient statistic vector decompose into the sum  
\beno
s_l(\bX) 
\= \dsum_{\{i,j\} \subset \mN} \, s_{l,i,j}(\bX_{i,j}),
&& l \in \{1, \ldots, p\},
\ee
so that the components of $\bs(\bX)$ are sums of bounded 
conditionally independent random variables given $\bY$.
As a result, equation \eqref{sum_over_dim} can be further decomposed into sums of independent random variables:
\beno
\langle \, \bv \, , \, \nabla_{\nat} \, \ell(\nat;\bx,\by) \, \rangle  \= \dsum_{\{i,j\} \subset \mN} \, \dsum_{l=1}^p \, v_l \, [s_{l,i,j}(\bx_{i,j}) \, - \, \mbE \, s_{l,i,j}(\bX_{i,j})].
\ee
Using the forms for $ s_{l,i,j}(\bX_{i,j})$ given in Lemma \ref{lem:s_hetero},
we have $0 \le s_{l,i,j}(\bX_{i,j}) \le Y_{i,j}$ $\mbP$-almost surely,
because $s_{l,i,j}(\bX_{i,j}) \in \{0, 1\}$ and $s_{l,i,j}(\bX_{i,j}) = 0$ if $Y_{i,j} = 0$ $\mbP$-almost surely. Then for each $\{i,j\} \subset \mN$, we have
\beno
\mbE \, \dsum_{l=1}^p \, v_l \, [s_{l,i,j}(\bx_{i,j}) \, - \, \mbE \, s_{l,i,j}(\bX_{i,j})] \= 0,
\ee
and by the Cauchy-Schwarz inequality, we obtain
\beno
\left|\,\dsum_{l=1}^p \, v_l \, [s_{l,i,j}(\bx_{i,j}) \, - \, \mbE \, s_{l,i,j}(\bX_{i,j})] \, \right| &\leq & \scalebox{0.97}{$\norm{\bv}_2 \, \sqrt{p} \, \norm{\bs_{l,i,j}(\bx_{i,j}) - \mbE\,\bs_{l,i,j}(\bX_{i,j})}_{\infty}$} \s \\
& \leq & \dfrac{3}{2} \, \sqrt{p}.
\ee
The last inequality follows from
\beno
\norm{\bv}_2 &\leq & \norm{\bu}_2 \, + \, \norm{\bu - \bv}_2 &\leq &1 \, + \, \dfrac{1}{2} &\leq & \dfrac{3}{2}.
\ee
The inequality is true because the construction of the $\epsilon$-net $\mV_{1/2} \subset \mU$ with $\epsilon=1/2$ ensures that such a $\bu \in \mU$ exists. We next bound the variance by 
\beno
\var \, \dsum_{\{i,j\}\subset \mN} \, \dsum_{l=1}^p \, v_l \, [s_{l,i,j}(\bx_{i,j}) \, - \, \mbE \, s_{l,i,j}(\bX_{i,j})] \s \\
= \dsum_{\{i,j\}\subset \mN} \, \dsum_{m=1}^p \, \dsum_{n=1}^p \, \cov \, (v_m \, s_{m,i,j} (\bX_{i,j}) \, ,\, v_n \, s_{n,i,j(\bX_{i,j})}) \s \\ 
= \dsum_{\{i,j\}\subset \mN}\, \dsum_{m=1}^p \, \dsum_{n=1}^p \, v_m \, v_n \, \cov(s_{m,i,j} (\bX_{i,j})\, s_{n,i,j} (\bX_{i,j})) \s \\
= \langle \, \bv \, , \, \norm{\by}_1 \, \mcI(\truth) \, \bv \,  \rangle \s \\
\leq  \norm{\bv}_2^2 \, \norm{\by}_1 \, \widetilde{\lambda}_{\max}^{\star} \s \\
 \leq  \dfrac{9}{4} \, \norm{\by}_1 \, \widetilde{\lambda}_{\max}^{\star},
\ee
where $\widetilde{\lambda}_{\max}^{\star}$ is the largest eigenvalue of the Fisher information of individual activated dyad defined in Lemma \ref{lem:min-eig} evaluated at the data-generating parameter $\truth$.
We then apply the one-sided Bernstein's inequality to obtain the upper bound for the conditional probability of $\mD_{2}(\nat,t)$ as follows: \cite[e.g.,][Theorem 2.8.4]{Vershynin18}
\be
\label{eq:hoef_lik}
\scalebox{0.9}{$\mbP\left( \mD_{2}(\nat,t) \given \bY = \by \right)$}
&\leq& \scalebox{0.9}{$\exp \, \left( p\, \log \, 5\right) \max\limits_{\bv \in \mV_{1/2}} \, \mbP \left( \,\langle \, \bv \, , \, \nabla_{\nat} \, \ell(\nat;\bx,\by) \, \rangle \, \ge \,\dfrac{t}{2} \,  \right)$} \s \\
& \leq & \scalebox{0.9}{$\exp\left(\dfrac{- \dfrac{(t/2)^2}{2}}{\dfrac{9}{4} \, \norm{\by}_1 \,\widetilde{\lambda}_{\max}^{\star} \, + \, \dfrac{1}{3}\, \dfrac{3}{2} \, \sqrt{p} \, \dfrac{t}{2} } \, + \, p \, \log \, 5 \right)$} \s \\
\= \scalebox{0.9}{$\exp\left(\dfrac{- t^2}{18 \, \norm{\by}_1 \,\widetilde{\lambda}_{\max}^{\star} \, + \, 2\, \sqrt{p} \, t } \, + \, p \, \log \, 5 \right)$}.
\ee
Using the law of total probability, 
we bound $\mbP\left( \mD_{2}(\nat, t) \given \mE(\epsilon) \right)$ as follows: 
\be
\label{divide_conquer}
\scalebox{0.9}{$\mbP\left( \mD_{2}(\nat, t) \,|\, \mE(\epsilon) \right)$}
&=& \scalebox{0.9}{$\dsum_{\by \in \mbY} \, \mbP\left( \mD_{2}(\nat,t) \cap [\bY = \by] \,|\, \mE(\epsilon) \right)$}   \s \\
\= \scalebox{0.9}{$\dsum_{\by \in \mE(\epsilon)} \, \mbP\left( \mD_{2}(\nat,t) \cap [\bY = \by] \,|\, \mE(\epsilon) \right)$}   \s \\
\= \scalebox{0.9}{$\dsum_{\by \in \mE(\epsilon)} \, \mbP\left( \mD_{2}(\nat,t) \,|\, [\bY = \by] \cap \mE(\epsilon)\right) \, 
\mbP(\bY = \by \,|\, \mE(\epsilon))$} \s \\ 
\= \scalebox{0.9}{$\dsum_{\by \in \mE(\epsilon)} \, \mbP(\mD_{2}(\nat,t) \,|\, \bY = \by) \, 
\dfrac{\mbP(\bY = \by)}{\mbP(\mE(\epsilon))}$},
\ee
noting that $[\bY = \by] \cap \mE(\epsilon) = [\bY = \by]$ whenever $\by \in \mE(\epsilon)$
and in the case when $\by \not\in \mE(\epsilon)$,
the intersection is empty,
implying 
\beno
\mbP(\bY = \by \,|\, \mE(\epsilon))
\= \dfrac{\mbP([\bY = \by] \cap \mE(\epsilon))}{\mbP(\mE(\epsilon))}
\= \begin{cases} 
\dfrac{\mbP(\bY = \by)}{\mbP(\mE(\epsilon))} & \by \in \mE(\epsilon) \\ 
0 & \by \not\in \mE(\epsilon)
\end{cases}. 
\ee
We now bound \eqref{divide_conquer} using the bound in \eqref{eq:hoef_lik}:  
\beno
& \dsum_{\by \in \mE(\epsilon)} \, \mbP(\mD_{2}(\nat,t) \,|\, \bY = \by) \,
\dfrac{\mbP(\bY = \by)}{\mbP(\mE(\epsilon))}  
\s \\ 
& \leq \,  \dsum_{\by \in \mE(\epsilon)} \, \exp\left(\dfrac{- t^2}{18 \, \norm{\by}_1 \,\widetilde{\lambda}_{\max}^{\star} \, + \, 2\, \sqrt{p} \, t } \, + \, p \, \log \, 5 \right) \, 
\dfrac{\mbP(\bY = \by)}{\mbP(\mE(\epsilon))} \s\\
&\leq  \, \exp\left(\dfrac{- t^2}{18 \, (\mbE\,\norm{\bY}_1 \, + \, \epsilon )\,\widetilde{\lambda}_{\max}^{\star} \, + \, 2\, \sqrt{p} \, t }  \, + \, p \, \log \, 5\right)  \, 
\dsum_{\by \in \mE(\epsilon)} \, \dfrac{\mbP(\bY = \by)}{\mbP(\mE(\epsilon))} \s\\ 
& = \exp\left(\dfrac{- t^2}{18 \, (\mbE\,\norm{\bY}_1 \, + \, \epsilon )\,\widetilde{\lambda}_{\max}^{\star} \, + \, 2\, \sqrt{p} \, t }  \, + \, p \, \log \, 5\right),
\ee
showing 
\beno
\mbP\left( \mD_{2}(\nat, t) \,|\, \mE(\epsilon) \right)
&\leq& \exp\left(\dfrac{- t^2}{18 \, (\mbE\,\norm{\bY}_1 \, + \, \epsilon )\,\widetilde{\lambda}_{\max}^{\star} \, + \, 2\, \sqrt{p} \, t }  \, + \, p \, \log \, 5\right). 
\ee
The replacement of $\norm{\by}_1$ by $\mbE\norm{\bY}_1 + \epsilon$ follows because 
$\norm{\by}_1 \le \mbE\norm{\bY}_1 + \epsilon$ for $\by \in \mE(\epsilon)$, 
resulting in the upper bound above.  
We bound the second term in the inequality \eqref{divide and conquer} 
using Chebyshev's inequality:
\beno
\mbP(\mE(\epsilon)^c) 
\= \mbP(\left| \norm{\bY}_1 - \mbE \, \norm{\bY}_1 \right| > \epsilon) \s \\
&\leq& \mbP(\left| \norm{\bY}_1 - \mbE \, \norm{\bY}_1 \right| \geq \epsilon) \s \\
&\leq&\dfrac{\var(\norm{\bY}_1)}{\epsilon^2}.
\ee
We bound the variance $\var(\norm{\bY}_1)$ as follows:
\beno
\var(\norm{\bY}_1)
\= \dsum_{\{i,j\} \subset \mN} \, \var \, Y_{i,j} 
+ 2 \, \dsum_{\{i,j\} \prec \{v,w\} \subset \mN} \, \cov(Y_{i,j}, \, Y_{v,w})\s\\
&\leq& \mbE \, \norm{\bY}_1 + 2 \, \dsum_{\{i,j\} \prec \{v,w\} \subset \mN} \, \cov(Y_{i,j}, \, Y_{v,w}),  
\ee
noting $Y_{i,j} \in \{0,1\}$ so that 
$\var \, Y_{i,j} = \mbP(Y_{i,j} = 1) \, \mbP(Y_{i,j} = 0) \leq \mbE \, Y_{i,j}$. 
Hence, 
\be
\label{ineq:var}
\mbP(\mE(\epsilon)^c)  
&\leq& \dfrac{\mbE \, \norm{\bY}_1 + 2 \, \sum_{\{i,j\} \prec \{v,w\} \subset \mN} \, \cov(Y_{i,j}, \, Y_{v,w})}{\epsilon^2}  \s \\
\= \dfrac{\mbE \, \norm{\bY}_1 + 2 \, \left[ D_{g} \right]^{+}}{\epsilon^2}.  
\ee 
Taking $\epsilon = \mbE \, \norm{\bY}_1 + 2 \, \left[ D_{g} \right]^{+} > 0$
shows that 
$\mbP(\mE(\epsilon)^c) \leq (\mbE \, \norm{\bY}_1)^{-1}$ and 
\beno
\mbP\left( \mD_{2}(\nat, t) \,|\, \mE(\epsilon) \right)
&\leq& \exp\left(\dfrac{-t^2}{36\,\widetilde{\lambda}_{\max}^{\star} \, (\mbE \, \norm{\bY}_1 + [D_{g}]^{+}) \, + \, 2\, \sqrt{p} \, t } \, + \, p \, \log \, 5 \right).
\ee
Combining all results shows that 
\beno
\mbP\left(\norm{\GradL - \mbE \, \GradL}_{2} \geq t \right)
\ee
is bounded above by 
\beno
\exp\left(\dfrac{-t^2}{36\,\widetilde{\lambda}_{\max}^{\star} \, (\mbE \, \norm{\bY}_1 + [D_{g}]^{+}) \, + \, 2\, \sqrt{p} \, t } \, + \, p \, \log \, 5 \right) \, + \, \dfrac{1}{\mbE \norm{\bY}_1}. 
\ee
As a final matter, 
note that this choice of $\epsilon > 0$ ensures $\mE(\epsilon)$ contains all $\by \in \mbY$ 
with $\norm{\by}_1 \in [0, \, 2 (\mbE \, \norm{\bY} + [D_g]^{+})]$
as the empty graph is an element of $\mbY$ with $0$ edges. 

\qed

\s 

\hide{
We next prove a related result for gradients of log-pseudolikelihood functions 
of multilayer networks in Lemma \ref{lem:concentration_pl}.  
The proof of Lemma \ref{lem:concentration_pl} essentially follows the same proof of Lemma \ref{lem:concentration_likelihood},
and as a result we do not repeat key arguments, instead opting to only outline the changes in the proof.  
}



%\begin{lemma}
\label{lem:concentration_pl}
Consider multilayer networks satisfying \eqref{general model}
which are 
defined on a set of $N \geq 3$ nodes and $K \geq 1$ layers. 
Define
$\gradPL \coloneqq - \nabla_{\nat} \, \pl(\nat; \bx, \by)$,
where $\pl(\nat; \bx, \by)$ is the log-pseudolikelihood function. 
Then,
for all $t > 0$ and $\nat \in \mbR^p$,
\beno
\mbP\left(\norm{\GradPL - \mbE \, \GradPL}_{\infty} \geq t \right)
&\leq&
2 \, \exp\left( -\dfrac{t^2}{K^2 \, (\mbE \, \norm{\bY}_1 + [D_{g}]^{+})} + \log p \right) + \dfrac{1}{\mbE \norm{\bY}_1}.  
\ee
\end{lemma}


%\llproof \ref{lem:concentration_pl}. 
%From \eqref{eq:log-pseudo},
the log-pseudolikelihood function is given by
\beno
\pl(\nat; \bx, \by) = \dsum_{\{i,j\} \subseteq \mN} \, \dsum_{k=1}^{K} \, 
\log \, \sepmodel(X_{i,j}^{(k)} = x_{i,j}^{(k)} \,|\, \bX_{i,j}^{(-k)} = \bx_{i,j}^{(-k)}, \bY = \by).
\ee
By Lemma \ref{lem:exp_pseudo}, 
$\sepmodel(X_{i,j}^{(k)} = x_{i,j}^{(k)} \,|\, \bX_{i,j}^{(-k)} = \bx_{i,j}^{(-k)}, \bY = \by)$ 
is an exponential family with sufficient statistic vector $s : \mbX \mapsto \mbR^p$ 
defined in Lemma \ref{lem:s_hetero} and natural parameter vector $\nat \in \mbR^p$.
Hence,  
\beno 
\nabla_{\nat} \, \pl(\nat;\bx,\by)
\=  \dsum_{\{i,j\} \subseteq \mN} \, \dsum_{k=1}^{K}  
\nabla_{\nat} \log \, \sepmodel(X_{i,j}^{(k)} = x_{i,j}^{(k)} \,|\, \bX_{i,j}^{(-k)} = \bx_{i,j}^{(-k)}, \bY = \by) \s \\
\= \dsum_{\{i,j\} \subseteq \mN} \, \dsum_{k=1}^{K} \, 
\left[ \, \bs(\bx) - \mbE_{\nat}\left[ \bs(\bX) \, | \, \bX_{i,j}^{-(k)}  = \bx_{i,j}^{(-k)}, \bY = \by \right]  \, \right],
\ee
where $\bX_{i,j}^{-(k)}$ denotes the ($K$-$1$)-dimensional 
vector of edge variables of dyad $\{i,j\}$ in $\bX_{i,j}$ which excludes the single edge variable $X_{i,j}^{(k)}$,
and by inserting the familiar form of the score equation of an exponential family 
with respect to the natural parameter vector 
\citep[e.g., Proposition 3.10, p. 32,][]{Su19}. 
Note that $\sepmodel(X_{i,j}^{(k)} = x_{i,j}^{(k)} \,|\, \bX_{i,j}^{(-k)} = \bx_{i,j}^{(-k)}, \bY = \by)$
may not belong to a minimal exponential family. 
This presents no issues as we do not require the conditional probability distributions of individual edge variables
belong to a minimal exponential family. 
Under the assumption that $(\bX, \bY)$ follow \eqref{general model}, 
the vectors $\bX_{i,j}$ ($\{i,j\} \subset \mN$) are conditionally independent given $\bY$ 
(as discussed in the proof of Lemma \ref{lem:concentration_likelihood}). 
Therefore,
the $l^{th}$ component $s_l(\bX)$ decomposes into the sum of conditionally independent Bernoulli random variables:
\beno
s_l(\bX)
\= \dsum_{\{i,j\} \subset \mN} \, s_{l,i,j}(\bX_{i,j}),
&& l \in \{1, \ldots, p\},
\ee
so that the components of $\bs(\bX)$ are sums of bounded conditionally independent random variables given $\bY$. 
Thus,
\beno
\nabla_{\nat} \, \pl(\nat;\bx,\by)
\= \dsum_{\{i,j\} \subseteq \mN} \, \dsum_{k=1}^{K} \,
\left( s_{l,i,j}(\bX_{i,j}) - E_{l,i,j}(\nat, \bx_{i,j}, \by) \right),
\ee
where
\beno
E_{l,i,j}(\nat, \bx_{i,j}, \by)
&\coloneqq& 
\mbE_{\nat}\left[s_{l,i,j}(\bX_{i,j}) \,|\, \bX_{i,j}^{(-k)} = \bx_{i,j}^{(-k)}, \bY = \by \right].
\ee
Using the form of $s_l(\bX)$ and $s_{l,i,j}(\bX_{i,j})$ outlined in Lemma \ref{lem:s_hetero},
we have $0 \le s_{l,i,j}(\bX_{i,j}) \le Y_{i,j}$ $\mbP$-almost surely,
because $s_{l,i,j}(\bX_{i,j}) \in \{0, 1\}$ and $s_{l,i,j}(\bX_{i,j}) = 0$ if $Y_{i,j} = 0$ $\mbP$-almost surely.
This also implies 
$0 \leq E_{l,i,j}(\nat, \bx_{i,j}, \by) \leq Y_{i,j}$ $\mbP$-almost surely. 
Taken together, 
\beno
0 
\,\leq\, \left| \dsum_{k=1}^{K} \,
\left( s_{l,i,j}(\bX_{i,j}) - E_{l,i,j}(\nat, \bx_{i,j}, \by) \right) \right| 
\,\leq\, \dsum_{k=1}^{K} \left| s_{l,i,j}(\bX_{i,j}) - E_{l,i,j}(\nat, \bx_{i,j}, \by) \right|  
\,\leq\, K \, Y_{i,j}, 
\ee
$\mbP$-almost surely. 
From here,
the remainder of the proof follows the proof of Lemma \ref{lem:concentration_likelihood},
with the sole exception using 
the bound $K \, Y_{i,j}$ in the application of Hoeffding's inequality. 
Reiterating the proof of Lemma \ref{lem:concentration_likelihood} with this change will yield  
\beno
\mbP\left(\norm{\GradPL - \mbE \, \GradPL}_{\infty} \geq t \right)
&\leq&
2 \, \exp\left( -\dfrac{t^2}{K^2 \, (\mbE\, \norm{\bY}_1 + [D_{g}]^{+})} + \log p \right) + \dfrac{1}{\mbE \norm{\bY}_1}.  
\ee

\qed 







\s\s
\subsection{Auxiliary results}
\begin{lemma}
\label{lem:s_hetero}
Consider multilayer networks satisfying \eqref{general model} 
with maximum interaction term $H \leq K$
and 
defined on a set of $N \geq 3$ nodes and $K \geq 1$ layers.  
Then the following hold: 
\ben
\item The conditional probability mass function of $\bX$ given $\bY$ is an exponential family:  
\beno
\sepmodel(\bX = \bx \mid \bY = \by)
&\propto& h(\bx, \, \by) \; \exp\left( \langle \nat, \, \bs(\bx) \rangle \right),
\ee
where 
\beno
h(\bx, \, \by)
\= \dprod_{\{i,j\} \subset \mN} \, \one(\norm{\bx_{i,j}}_1 > 0)^{y_{i,j}} \; 
\one(\norm{\bx_{i,j}}_1 = 0)^{1 - y_{i,j}},
\ee
sufficient statistic vector $s : \mbX \mapsto \mbR^p$ and natural parameter vector $\nat \in \mbR^p$. 

\item For each $l \in \{1, \ldots, p\}$,
there exists $h \in \{1, \ldots, H\}$ and $\{k_1, \ldots, k_h\} \subseteq \{1, \ldots, K\}$ such that 
the $l^{\text{th}}$ component of the sufficient statistic vector $s(\bx)$ can be written as 
\be
\label{eq:suff}
s_l(\bx)
\= \dsum_{\{i,j\} \subset \mN} \, s_{l,i,j}(\bx)
\= \dprod_{r=1}^{h} \, x_{i,j}^{(k_r)}. 
\ee
\item The exponential family outlined above is both minimal, full, and regular. 
\een
\end{lemma}

\s 

\llproof \ref{lem:s_hetero}.
First, 
the form of the conditional probability distribution of $\bX$ given $\bY$
derived in Proposition \ref{prop:inference} is given by 
\be
\label{eq:1999}
\mbP_{\nat}(\bX = \bx \,|\, \bY = \by)
\= \exp\left( \log \, f(\bx, \nat) + \log \, \psi(\nat, \by) \right), 
\ee
provided $h(\bx, \by) = 1$.
The form of \eqref{general model} 
suggests that \eqref{eq:1999} will be a minimal exponential family in canonical form
due to the form of the Markov random field specification for $f(\nat, \bx)$
and the definition of $\psi(\nat, \by)$. 
From the form of $f(\bx, \nat)$ in \eqref{general model},
\beno
\log f(\bx, \nat)
\,=\, \dsum_{\{i,j\} \subset \mN} \,
\left(\dsum_{k=1}^K \theta_{k} x_{i,j}^{(k)} 
+  \dsum_{\substack{k < l}}^{K}  \theta_{k,l} x_{i,j}^{(k)} x_{i,j}^{(l)} + \ldots  
+ \dsum_{k_1 <\ldots< k_H}^{K} \theta_{k_1,k_2,\ldots,k_H}  x_{i,j}^{(k_1)} \cdots x_{i,j}^{(k_H)} \right),
\ee
where $H \le K$ is the highest order of cross-layer interactions included in the model.
We write $\theta_{k_1,k_2,\ldots,k_h}$ to reference the $h$-order interaction parameter
for the interaction term among layers $\{k_1, \ldots, k_h\} \subseteq \{1, \ldots, K\}$.
As specified, 
$\psi(\nat, \by)$ is the normalizing constant for the exponential family. 
As such, 
the natural parameter space of the exponential family is $\mbR^p$ 
as the support of $\mbX$ is finite, 
which implies $\psi(\nat, \by) < \infty$ for all $\nat \in \mbR^p$ and $\by \in \mbY$.
We establish minimality by noting that the components of the parameter vector $\nat$ 
satisfy no linear or affine constraints.  
Attached to each parameter $\theta_{k_1,\ldots,k_h}$ 
($\{k_1,\ldots, k_h\} \subset \{1, \ldots K\}$, 
$h \in \{1, \ldots, H\}$) 
is the sufficient statistic 
\beno
s_{k_1, \ldots, k_h}(\bx) 
\= \dsum_{\{i,j\} \subset \mN} \, x_{i,j}^{(k_1)} \,\cdots\, x_{i,j}^{(k_h)}.
\ee
Each statistic $s_{k_1, \ldots, k_h}$ is a function of distinct, non-degenerate random variables,
provided $\norm{\by}_1 > 0$,
and so none of the statistics $s_{k_1, \ldots, k_h}$ satisfy any linear or affine constraints. 
Hence, 
\eqref{general model}
specifies a minimal and full exponential family with natural parameter space $\mbR^p$
of dimension $p = \sum_{h=1}^{H} \, \binom{K}{h}$
and sufficient statistic vector $s(\bx)$ 
with components 
$s_{k_1, \ldots, k_h}(\bx)$ 
($\{k_1, \ldots, k_h\} \subseteq \{1, \ldots, K\}, h = 1, \ldots, H$). 
Regularity follows trivially 
\citep[e.g., Proposition 3.7, pp. 28,][]{Su19}. 
The form of \eqref{eq:suff} outlines this for a linear indexing of the components of the sufficient statistic vector. 

\qed

\s\s

\begin{lemma}
\label{lem:exp_pseudo}
Consider multilayer networks satisfying \eqref{general model} 
with maximum interaction term $H \leq K$ 
and defined on a set of $N \geq 3$ nodes and $K \geq 1$ layers. 
Then the conditional probability mass function of $X_{i,j}^{(k)}$ given 
$\bY = \by$ and $\bX_{i,j}^{(-k)} = \bx_{i,j}^{(-k)}$ is an exponential family 
\beno
\sepmodel(X_{i,j}^{(k)} = x_{i,j}^{(k)} \,|\, \bX_{i,j}^{(-k)} = \bx_{i,j}^{(-k)}, \bY = \by)
&\propto& h(\bx, \, \by) \; \exp\left( \langle \nat, \, \bs(\bx) \rangle \right),
\ee
with sufficient statistic vector $\bs : \mbX \mapsto \mbR^p$ defined in Lemma \ref{lem:s_hetero}, 
natural parameter vector $\nat \in \mbR^p$,  
and 
\beno
h(\bx, \, \by)
\= \dprod_{\{i,j\} \subset \mN} \, \one(\norm{\bx_{i,j}}_1 > 0)^{y_{i,j}} \; 
\one(\norm{\bx_{i,j}}_1 = 0)^{1 - y_{i,j}}.  
\ee
\end{lemma}

\s 

\llproof \ref{lem:exp_pseudo}. First,  
note that the form of \eqref{general model} and Proposition \ref{prop:inference} suggests that 
\be
 \sepmodel(X_{i,j}^{(k)} = x_{i,j}^{(k)} \,|\, \bX_{i,j}^{(-k)} = \bx_{i,j}^{(-k)}, \bY = \by) \s \\
 \quad\quad\quad = \dfrac{h(x_{i,j}^{(k)}, \, \bx_{i,j}^{(-k)}, \, \by) \; \exp\left( \langle \nat, \, \bs(x_{i,j}^{(k)}, \, \bx_{i,j}^{(-k)}) \rangle \right)}{\dsum_{x_{i,j}^{(k)} \in \{0,1\} }\, h(x_{i,j}^{(k)}, \, \bx_{i,j}^{(-k)}, \, \by) \; \exp\left( \langle \nat, \, \bs(x_{i,j}^{(k)}, \, \bx_{i,j}^{(-k)}) \rangle \right)}
\ee
is an exponential family in canonical form
using the Markov random field specification for $f(\nat, \bx)$ 
and the form of the conditional probability distribution of $X_{i,j}^{(k)}$ given $\bY$ 
and $\bX_{i,j}^{(-k)}$.  
However,
this exponential family may not be full rank due to possible 0 values of components of the given ($K$-$1$)-dimensional vector $\bx_{i,j}^{(-k)}$ and thus may not be minimal. 

\qed





\section{Proof of Theorem \ref{thm1}}
\label{sec:pf_thm1}
We prove Theorem \ref{thm1} from Section \ref{sec3}.


\ttproof \ref{thm1}. 
We first prove the theorem for maximum likelihood estimators,
and then discuss extensions and changes necessary to prove the result for maximum pseudolikelihood estimators. 
By Proposition \ref{prop:inference}, 
observing $\bX = \bx$ implies we observe $\bY = \by$,
as for each given $\bx \in \mbX$,
$\bY = \by$ ($\mbP$-a.s.) for one and only one $\by \in \mbY$ given by 
\beno
y_{i,j} 
\= \one\left( \norm{\bx_{i,j}}_1 \,>\, 0 \right),
&& \{i,j\} \subset \mN. 
\ee 
Denote the gradient of $-\ell(\nat; \bx, \by)$ by 
\beno
\gradL 
&\coloneqq& -\nabla_{\nat} \, \ell(\nat; \bx, \by)
\ee
and the expected Hessian matrix of the negative log-likelihood by 
\beno
\bH(\nat) 
&\coloneqq& -\mbE \, \nabla_{\nat}^2 \, \ell(\nat; \bX, \bY). 
\ee 
Theorem 6.3.4 of \citet{OrRh20} states that if 
\beno 
\label{eq:ex_cond} 
(\nat - \truth)^{\top} \, \gradL
&\geq& 0 
\ee
for all $\nat \in \partial \, \mB_2(\truth, \epsilon)$, 
where $\partial \, \mB_2(\truth, \epsilon)$ is the boundary of the set 
\beno
\mB_2(\truth, \epsilon) 
\= \{ \nat \in \mbR^p \,:\, \norm{\nat - \truth}_2 < \epsilon\},
\ee 
then $\gradL$ 
has a root in $\mB_2(\truth, \epsilon) \cup \partial \mB_2(\truth, \epsilon)$,
i.e.,
$\mle$ exists and satisfies $\norm{\mle - \truth}_2 \leq \epsilon$.
Note that a root of $\gradL$ is also a root of $-\gradL$;
in what follows, 
we consider finding a maximizer of $\ell(\nat;\bx,\by)$ by finding a minimizer of $-\ell(\nat;\bx,\by)$. 
The classification of roots as maximizers/minimizers 
is justified from the fact that that $\ell(\nat; \bx, \by)$ is concave in $\nat$,
a fact which follows from Proposition \ref{prop:inference},
as $g(\by)$ is constant in $\nat$ and $\log \, \mbP_{\nat}(\bX = \bx \,|\, \bY = \by)$ 
is the log-likelihood of a minimal, full, and regular exponential family with natural parameter vector $\nat$
and thus is strictly concave in $\nat$ \citep[Proposition 3.10, p. 32,][]{Su19}. 
By the multivariate mean-value theorem \citep[][Theorem 5]{FuMa91}, 
\beno
(\nat - \truth)^{\top}  \mbE \, \GradL 
\= (\nat - \truth)^{\top}  \mbE \, \gamma_{\truth}(\bX, \bY) 
+ (\nat - \truth)^{\top}  \bH(\dot\nat)  (\nat - \truth) \s \\ 
\=(\nat - \truth)^{\top}  \bH(\dot\nat)  (\nat - \truth), 
\ee
where $\dot\nat = t \, \nat + (1 - t) \, \truth$ (some $t \in [0, 1]$) 
and by invoking Lemma 2 of \citet{StSc21},
which shows that both the expected log-likelihood and log-pseudolikelihood 
of a minimal exponential family is uniquely maximized at the data-generating parameter vector $\truth$, 
implying $\mbE \, \gamma_{\truth}(\bX, \bY) = 0$. 
Let $\epsilon \in (0, \epsilon^\star)$
and arbitrarily take $\nat \in \partial \mB_2(\truth, \epsilon)$. 
Then 
\beno
(\nat - \truth)^{\top} \bH(\dot\nat) (\nat - \truth)
= \dfrac{(\nat - \truth)^{\top}  \bH(\dot\nat)  (\nat - \truth)}{(\nat - \truth)^{\top} (\nat- \truth)}  
\norm{\nat - \truth}_2^2
\,\geq\, \epsilon^2 \, \lambda_{\min}(\bH(\dot\nat)),  
\ee
since $\norm{\nat - \truth}_2 = \epsilon$ as $\nat \in \partial \mB_2(\truth, \epsilon)$ 
and because the Rayleigh quotient of a matrix is bounded below by the smallest eigenvalue of that matrix so that
\beno
\dfrac{(\nat - \truth)^{\top}  \bH(\dot\nat)  (\nat - \truth)}{(\nat - \truth)^{\top} (\nat- \truth)}
\;\geq\; \lambda_{\min}(\bH(\dot\nat))
\;\geq\; \inf\limits_{\nat \in \mB_2(\truth, \epsilon^\star)} \, \lambda_{\min}(\bH(\nat)),
\ee
where $\lambda_{\min}(\bH(\dot\nat))$ is the smallest eigenvalue of $\bH(\dot\nat)$,
noting that 
\beno
\norm{\dot\nat - \truth}_2 
\= \norm{t \, \nat + (1 - t) \, \truth - \truth}_2
\= t \, \norm{\nat - \truth}_2 
&\leq& \norm{\nat - \truth}_2
&\leq& \epsilon^\star,  
\ee
since $t \in [0, 1]$. 
Lemma \ref{lem:min-eig} showed that 
\beno
\lambda_{\min}(\bH(\nat)) 
\= \lambda_{\min}(\mcI(\nat)) \; \mbE \, \norm{\bY}_1, 
\ee
which in turn implies 
\beno
\inf\limits_{\nat \in \mB_2(\truth, \epsilon^\star)} \, \lambda_{\min}(\bH(\nat))
\= \xi_{\epsilon^\star} \; \mbE \, \norm{\bY}_1,
\ee
where 
\beno
\xi_{\epsilon^\star}
&\coloneqq& \inf\limits_{\nat \in \mB_2(\truth, \epsilon^\star)} \, \lambda_{\min}(\mcI(\nat)),
\ee
with $\mcI(\nat)$ defined in Lemma \ref{lem:min-eig}. 
Hence,
for $\nat \in \partial \mB_2(\truth, \epsilon)$ ($\epsilon \in (0, \epsilon^\star)$),  
\beno
(\nat - \truth)^{\top}  \mbE \, \GradL
&\geq& \epsilon^2 \; \xi_{\epsilon^\star} \; \mbE \, \norm{\bY}_1. 
\ee
We next turn to showing  
\beno
\mbP\left( \inf\limits_{\nat \in \mB_2(\truth,\epsilon)} \, 
(\nat - \truth)^{\top} \, \GradL \,\geq\, 0 \right) 
&\geq& 1 - 2 \, (\mbE \norm{\bY}_1)^{-1},
\ee
by showing that the event  
\beno
\sup\limits_{\nat \in \mB_2(\nat^\star, \epsilon)} \, 
|(\nat - \truth)^{\top} \, ( \mbE \, \GradL - \GradL)|
\;<\; \epsilon^2 \; \Lam
\ee
occurs with probability at least $1 - 2 \,  (\mbE \norm{\bY}_1)^{-1}$, 
in turn implying that the event   
$\norm{\mle - \truth}_2 \leq \epsilon$ 
happens with probability at least $1 - 2 \,  (\mbE \norm{\bY}_1)^{-1}$. 
Applying the Cauchy-Schwarz inequality and utilizing standard vector norm inequalities,
\beno
|(\nat - \truth)^{\top} \, ( \mbE \, \GradL - \GradL)| 
&\leq& \norm{\nat - \truth}_2 \, \norm{\GradL  - \mbE \, \GradL}_2 \s \\ 
&\leq&  \epsilon \; \sqrt{p} \; \norm{\GradL - \mbE \, \GradL}_{\infty}, 
\ee
noting $\nat \in \partial \mB_2(\truth, \epsilon)$.
It suffices to demonstrate,
for all $\nat \in \partial \mB_2(\truth, \epsilon)$, 
that  
\beno
\mbP\left(\norm{\GradL - \mbE \, \GradL}_{\infty} \,<\, \epsilon \, p^{-\frac{1}{2}} \, \Lam \right)
&\geq& 1 - 2 \, (\mbE \norm{\bY}_1)^{-1}. 
\ee
For ease of presentation,
we define $\mD_{N,\epsilon,p}$ to be the event 
\beno
\norm{\GradL - \mbE \, \GradL}_{\infty} 
&\geq& \epsilon \, p^{-\frac{1}{2}} \, \Lam. 
\ee
Applying Lemma \ref{lem:concentration_likelihood}, 
\beno
\mbP\left(\mD_{N,\epsilon,p} \right)
\;\leq\;
2 \, \exp\left( -\dfrac{(\epsilon \, \xi_{\epsilon^\star} \; \mbE \, \norm{\bY}_1)^2}
{p \, (\mbE \, \norm{\bY}_1 + [D_{g}]^{+})} + \log \, p \right) + \dfrac{1}{\mbE \norm{\bY}_1}, 
\ee
where 
$[D_{g}]^{+} \coloneqq \max\{0, \, D_{g}\}$,  
recalling 
\beno
D_{g}
&\coloneqq& \dsum_{\{i,j\} \prec \{v,w\} \subset \mN} \, \cov(Y_{i,j}, \, Y_{v,w}),
\ee
where $\{i,j\} \prec \{v,w\}$ implies the sum is taken with respect to the lexicographical ordering
of pairs of nodes.
Under the assumption that $\mbE \norm{\bY}_1 \geq 1$,
\beno
2 \, \exp\left( -\dfrac{\epsilon^2\, \xi_{\epsilon^\star}^2 \, (\mbE \norm{\bY}_1)^2}
{p \, (\mbE \, \norm{\bY}_1 + [D_g]^{+})} + \log \, p \right)
&\leq& 2 \, \exp\left( -\dfrac{\epsilon^2\, \xi_{\epsilon^\star}^2 \, \mbE \norm{\bY}_1}{p \, (1 + [D_{g}]^{+})} + \log \, p \right).
\ee
Take 
\beno
\epsilon 
\= \sqrt{\dfrac{3 \, p \, \log N}{\mbE \norm{\bY}_1}} \; \dfrac{\sqrt{1 + [D_{g}]^{+}}}{\xi_{\epsilon^\star}}.  
\ee
If 
\beno
\lim\limits_{N \to \infty} \, 
\sqrt{\dfrac{3 \, p \, \log N}{\mbE \norm{\bY}_1}} \; \dfrac{\sqrt{1 + [D_{g}]^{+}}}{\xi_{\epsilon^\star}}
\= 0,
\ee
then for $N$ sufficiently large,
we will have $\epsilon < \epsilon^\star$,
which ensures $\epsilon^\star$ may be chosen independent of $N$ and $p$.  
While $\epsilon^\star$ can be chosen independent of $N$ and $p$, 
note that $p$ is expected to be a function of $N$ and thus $\xi_{\epsilon^\star}$ 
will not (in general) be independent of $N$,
possibly holding implications for how fast $p$ may grow with $N$ for certain $\truth$ and $\epsilon^\star$.  
This choice of $\epsilon$ in turn implies 
\beno
2 \, \exp\left( -\dfrac{\epsilon^2\, \xi_{\epsilon^\star}^2 \, \mbE \norm{\bY}_1}{p \, (1 + [D_{g}]^{+})} + \log \, p \right)
\= 2 \, \exp\left(- 3 \log N + \log p \right)
&\leq& 2 \, N^{-2},
\ee
under the assumption that $p \leq N$,
which ensures $-3 \log N + \log p \leq -2 \, \log N$.  
Note that $\mbE \norm{\bY}_1 \leq \binom{N}{2} \leq N^{2}$.
We have thus shown,
for all $\nat \in  \partial \mB_2(\truth, \epsilon)$, 
that 
\beno
\mbP\left(\norm{\GradL - \mbE \, \GradL}_{\infty} \,\leq\, \epsilon \, p^{-\frac{1}{2}} \, \Lam \right)
&\geq& 1 - 3 \, (\mbE \, \norm{\bY}_1)^{-1}, 
\ee
under the above conditions. 
As a result, 
there exists $N_0 \geq 3$ such that,
for all $N \geq N_0$  
and with probability at least $1 - 3 \, (\mbE \, \norm{\bY}_1)^{-1}$, 
the set $\Mle$ is non-empty and the unique element of the set $\mle \in \Mle$ 
satisfies (uniqueness following from minimality, as discussed in Section \ref{sec3}) 
\beno
\norm{\mle - \truth}_2 &\leq& 
\sqrt{\dfrac{3 \, p \, \log N}{\mbE \norm{\bY}_1}} \; \dfrac{\sqrt{1 + [D_{g}]^{+}}}{\xi_{\epsilon^\star}}.  
\ee
The above proof can be extended to maximum pseudolikelihood estimators by substituting 
the relevant quantities (e.g., $\widetilde\xi_{\epsilon^\star}$ for $\xi_{\epsilon^\star}$, etc.). 
The one change of note is that instead of applying the concentration inequality in Lemma \ref{lem:concentration_likelihood},
we apply the concentration inequality in Lemma \ref{lem:concentration_pl},
which includes an additional factor of $K^2$. 
Following these steps and repeating the above proof will show that
there exists $N_0 \geq 3$ such that,
for all $N \geq N_0$ and 
with probability at least $1 - 3 \, (\mbE \, \norm{\bY}_1)^{-1}$, 
the set $\Mple$ is non-empty and each $\mple \in \Mple$
satisfies 
\beno
\norm{\mple - \truth}_2 &\leq&
\sqrt{\dfrac{3 \, p \, K^2 \, \log N}{\mbE \norm{\bY}_1}} \; \dfrac{\sqrt{1 + [D_{g}]^{+}}}{\widetilde\xi_{\epsilon^\star}}, 
\ee
where
\beno
\widetilde\xi_{\epsilon^\star}
&\coloneqq& \inf\limits_{\nat \in \mB_2(\truth, \epsilon^\star)} \, \lambda_{\min}(\widetilde\mcI(\nat)).
\ee
\qed 



\s\s
\section{Proposition \ref{prop:suff_norm} and proof} 
\label{sec:pf_prop2}

In order to establish a bound on the error of the multivariate normal approximation 
for estimators of data-generating parameters, 
we first establish an error bound on the multivariate normal approximation 
of a standardization of the 
sufficient statistic vector $\bs(\bX)$ of the exponential family distribution of $\bX$ given $\bY$,
derived in Lemma \ref{lem:s_hetero},
in Proposition \ref{prop:suff_norm} 
using a Lyapunov type bound presented in \citet{Raic19}.
Proposition \ref{prop:suff_norm} provides the basis for our normality proof for estimators
which we present in Theorem \ref{thm2}.
\begin{proposition}
\label{prop:suff_norm}
Consider a separable multilayer network model following the form of equation \eqref{general_model} and is 
defined on a set of $N \geq 3$ nodes and $K \geq 1$ layers.
Denote by $\bs(\bX) \in \mbR^p$  the sufficient statistic vector of the exponential family $\mbP(\bX = \bx \,|\, \bY = \by)$
as defined in Lemma \ref{lem:s_hetero}. 
Let $\mbE^{\bY}$ be the random conditional expectation operator 
for the distribution  of $\bX$ conditional on $\bY$, and define
\beno
\bS_{\mN} 
&\coloneqq& 
(I(\truth) \, \norm{\bY}_1)^{-1/2} \, (\bs(\bX) - \mbE^{\bY} \bs(\bX)) \s \\ 
\= \dsum_{\{i,j\} \subset \mN} \, (I(\truth) \, \norm{\bY}_1)^{-1/2} \, (\bs_{i,j}(\bX) - \mbE^{\bY} \bs_{i,j}(\bX)). 
\ee
For any measurable convex set $\mA \subset \mbR^p$, 
\beno
\label{ineq normal error homo}
\left| \, \mbP(\bS_\mN\in \mA)-\Phi(\bZ \in \mA)\,  \right| &\leq&
 \dfrac{83}{(\widetilde{\lambda}_{\min}^{\epsilon})^{3/2}} \, 
\sqrt{\dfrac{p^{7/2}}{\mbE \, \norm{\bY}_1}}
+  \dfrac{4}{\mbE \, \norm{\bY}_1} + \dfrac{8 \, \left[D_{g}\right]^{+}}{\left(\mbE \, \norm{\bY}_1 \right)^2}, 
\ee
where $\Phi$ is the standard multivariate normal measure and $\bZ \sim \text{MvtNorm}(\bm{0}_p, \bI_p)$,
where $\bm{0}_p$ is the $p$-dimensional vector of zeros 
and $\bI_p$ is the $p\times p$ identity matrix. 
\end{proposition}



Before we prove Proposition \ref{prop:suff_norm}, we introduce a Lyapunov type bound in Lemma \ref{lem:raic} provided by Theorem 1 of Raic \citep{Raic19}. 

\begin{lemma}
\label{lem:raic}
Consider a sequence of $n \geq 1$ independent random vectors $\bW_i \in \mbR^p$. 
Assume that $\mbE \, \bW_i = \bm{0}_p$ and $\sum_{i=1}^{n} \, \var \, \bW_i = \bI_p$  
where $\bm{0}_p$ is the $p$-dimensional vector of zeros and $\bI_p$ is the $p \times p$ identity matrix.
Define 
\beno
\bS_n 
\= \dsum_{i=1}^{n} \, \bW_i
\ee
and let $\bZ$ be the standard multivariate normal variable, i.e., $\bZ \sim \text{MvtNorm}(\bm{0}_p, \bI_p)$.  
Then,
for all measurable convex sets $\mA \subset \mbR^p$,  
\beno
\left| \mbP(\bS_n \in \mA) - \Phi(\bZ \in \mA) \right|
&\leq& (42 \, p^{1/4} + 16) \, \dsum_{i=1}^{n} \, \mbE \, \norm{\bW_i}_2^3,
\ee 
where $\Phi$ is the standard multivariate normal measure.
\end{lemma}

\s\s

We now turn to proving Proposition \ref{prop:suff_norm}. 

\s

\pproof \ref{prop:suff_norm}. 
By Proposition \ref{prop:inference} and Lemma \ref{lem:s_hetero},   
the conditional distribution of the multilayer network $\bX$ given $\bY$ 
follows an exponential family with sufficient statistic vector that can be decomposed 
into the sum of conditionally independent dyad-based statistics: 
\beno
\bs(\bX) 
\= \dsum_{\{i,j\} \subset \mN} \, \bs_{i,j}(\bX),
\ee
with the precise formula for $\bs_{i,j}(\bX)$ given in Lemma \ref{lem:s_hetero}. 
Define 
\beno
\bS_{\mN} 
&\coloneqq& 
(I(\truth) \, \norm{\bY}_1)^{-1/2} \, (\bs(\bX) - \mbE^{\bY} \bs(\bX)) \s \\ 
\= \dsum_{\{i,j\} \subset \mN} \, (I(\truth) \, \norm{\bY}_1)^{-1/2} \, (\bs_{i,j}(\bX) - \mbE^{\bY} \bs_{i,j}(\bX)), 
\ee
where $I(\truth)$ is the Fisher information matrix of a single dyad $X_{i,j}$ for $\{i,j\} \subset \mN$ 
satisfying 
$Y_{i,j} = 1$ (i.e., the subset of activated dyads) evaluated at $\truth$ 
per Lemma \ref{lem:min-eig} 
and where 
$\mbE^{\bY}$ is the random conditional expectation operator with respect to the distribution of $\bX$ conditional on $\bY$. 
For $\epsilon > 0$ satisfying $\epsilon < \mbE \, \norm{\bY}_1$, define the event $\mE(\epsilon)$ by
\beno
\mE(\epsilon) & \coloneqq &  \left\{\, \by \in \mbY \,:\, 
\norm{\by}_1 \geq \mbE \norm{\bY}_1 - \epsilon \right\}. 
\ee
In words, 
$\mE(\epsilon)$ is the subset of configurations of the single-layer network $\bY$ 
which have number of edges equal to at least the  expected number of activated dyads 
$\mbE \, \norm{\bY}_1$ minus $\epsilon > 0$. 
The restrictions placed on $\epsilon$ ensure that $\mbE \, \norm{\bY}_1 - \epsilon > 0$ 
which implies that $\mE(\epsilon)$ will not contain the empty graph which has no edges
and that $\mE(\epsilon)$ will contain the complete graph with $\binom{N}{2}$ edges 
as $\mbE \, \norm{\bY}_1 < \binom{N}{2}$ (strict inequality following from the fact that 
$g(\by)$, the marginal probability mass function of $\bY$, 
is assumed to be strictly positive on $\mbY$).
Hence, 
$\mbP(\mE(\epsilon)) > 0$ and $\mbP(\mE(\epsilon)^c) > 0$.   
Let $\mA \subset \mbR^p$ be a measurable convex set.  
By the law of total probability and the triangle inequality, 
we have
\be
\label{tri_ineq}
\left|\, \mbP(\bS_\mN \in \mA) - \Phi(\bZ \in \mA) \, \right| 
&\leq&
\left|\mbP(\bS_n \in \mA \, | \, \mE(\epsilon))  - \Phi(\bZ \in \mA)\right| \, \mbP(\mE(\epsilon)) \s \\
 && +  \; \left|\mbP(\bS_n \in \mA \, | \, \mE^c(\epsilon))  - \Phi(\bZ \in \mA)\right| \, \mbP(\mE^c(\epsilon)) \s \\ 
 &\leq & \sup\limits_{\by \in \mE(\epsilon)} \,\left| \, \mbP(\bS_{\mN} \in \mA \, | \, \bY = \by) - \Phi(\bZ \in \mA)\,  \right| \; + \; \mbP(\mE^c(\epsilon)),
\ee
noting $\left|\mbP(\bS_n \in \mA \, | \, \mE^c(\epsilon))  - \Phi(\bZ \in \mA)\right| \leq 1$ 
and $\mbP(\mE(\epsilon)) \leq 1$. 
Taking
\beno
\bW_{i,j} & = & (I(\truth) \, \norm{\bY}_1)^{-1/2} \, (\bs_{i,j}(\bX) - \mbE^{\bY} \, \bs_{i,j}(\bX)),
\ee
we have
\beno
\mbE \, [\bW_{i,j} \, | \, \bY = \by ] \= 0, 
\ee
a result of the tower property of conditional expectation, 
and 
\beno 
\var \, \left[\sum_{\{i,j\} \subset \mN} \, \bW_{i,j}  \, | \, \bY = \by \right] \= \bI_p,  
\ee 
which follows from Lemma \ref{lem:min-eig} which establishes that 
$\var[s_{i,j}(\bX) \,|\, \bY = \by] = I(\truth)$ when $Y_{i,j} = 1$, 
recalling the form of the Fisher information matrix of exponential families to be the 
covariance matrix of the sufficient statistic vector 
\citep[e.g., Proposition 3.10, pp. 32,][]{Su19}, 
and due to the fact that 
$\var[s_{i,j}(\bX) \,|\, \bY = \by] = \bm{0}_{p,p}$ when $Y_{i,j} = 0$. 
Applying Lemma \ref{lem:raic} to the first term of the summation of \eqref{tri_ineq}, for any measurable convex set 
$\mA \subset \mbR^p$,
\beno
\left| \mbP(\bS_{\mN} \in \mA \, | \, \bY = \by) - \Phi(\bZ \in \mA) \right|
&\leq& (42 \, p^{1/4} + 16) \, \dsum_{\{i,j\} \subset \mN} \, \mbE \, \left[\norm{\bW_{i,j}}_2^3 \, | \, \bY =\by \right].
\ee
Given $\bY = \by$, using standard matrix and vector norm inequalities,
%and given $\bY = \by$, using the standard matrix and vector norm inequalities, 
\beno
\norm{\bW_{i,j}}_2
\= \norm{(I(\truth) \, \norm{\by}_1)^{-1/2} \, (\bs_{i,j}(\bX) - \mbE \, \bs_{i,j}(\bX))}_2 \s \\ 
& \leq & \norm{\by}_1^{-1/2} \,\mnorm{I(\truth)^{-1/2}}_2 \, \norm{\bs_{i,j}(\bX) - \mbE \, \bs_{i,j}(\bX)}_2 \s \\ 
& \leq & (\norm{\by}_1 \, \xi_{\epsilon^\star})^{-1/2} \, p^{1/2} \, y_{i,j}, 
\ee
where $\mnorm{\cdot}_2$ denotes the spectral norm of a $p \times p$ matrix and 
\beno
\xi_{\epsilon^\star}
&\coloneqq& \inf\limits_{\nat \in \mB_2(\truth, \epsilon^\star)} \, \lambda_{\min}(I(\nat)),  
\ee
for a given and fixed $\epsilon^\star > 0$, 
as defined in Section \ref{sec3}.
From proofs of Lemma \ref{lem:concentration_likelihood}
and \ref{lem:concentration_pl}, 
\beno
0 &\leq& s_{l,i,j}(\bx) &\leq& 1,
&& \mbox{all } l = 1, \ldots, p, \, \{i,j\} \subset \mN, 
\ee
$\mbP$-almost surely. 
Hence, 
\beno
\mbP(\norm{\bs_{i,j}(\bX) - \mbE^{\bY} \bs_{i,j}(\bX)}_\infty \leq y_{i,j} \,|\, \bY = \by) 
\= 1,
\ee 
implying (conditional on $\bY = \by$) 
\beno
\norm{\bs_{i,j}(\bX) - \mbE^{\bY} \, \bs_{i,j}(\bX)}_2 & \leq & (p \, y_{i,j})^{1/2} 
\= p^{1/2} \, y_{i,j},
\ee
$\mbP$-almost surely. 
As a result,
\beno
\mbE \, \left[\norm{\bW_{i,j}}_2^3 \, | \, \bY =\by \right]
&\leq &  (\norm{\by}_1  \, \xi_{\epsilon^\star})^{-3/2} \, p^{3/2} \, y_{i,j},
\ee
noting that $y_{i,j}^3 = y_{i,j} \in \{0, 1\}$. 
Using the fact that $42 \, p^{1/4} + 16 \leq 58 \, p^{1/4}$ ($p \geq 1$), we have
\beno
(42 \, p^{1/4} + 16) \, \dsum_{\{i,j\} \subset \mN} \,\mbE \, \left[\norm{\bW_{i,j}}_2^3 \, | \, \bY =\by \right]
&\leq& 58 \,  p^{7/4} \, \dsum_{\{i,j\} \subset \mN} \, y_{i,j} \, (\norm{\by}_1 \, \xi_{\epsilon^\star})^{-3/2} \s \\
& = & 58 \, p^{7/4} \,\norm{\by}_1^{-1/2} \,\xi_{\epsilon_\star}^{-3/2} \s \\
& \leq & 58 \, p^{7/4} \,(\mbE \, \norm{\bY}_1 - \epsilon)^{-1/2} \,\xi_{\epsilon_\star}^{-3/2},
\ee
as the conditioning event
$\mE(\epsilon)$ and choice of $\epsilon$
ensure that $\norm{\by}_1 \geq \mbE \norm{\bY}_1 - \epsilon > 0$.
We bound the second term in \eqref{tri_ineq} by Chebyshev's inequality using equation \eqref{ineq:var} in the proof of Lemma \ref{lem:concentration_likelihood}:
\beno
\mbP(\mE^c(\epsilon))  
&\leq& \dfrac{\mbE \, \norm{\bY}_1 + 2 \, \left[ D_{g} \right]^{+}}{\epsilon^2}.  
\ee 
Taking $\epsilon = 2^{-1} \, \mbE \, \norm{\bY}_1 > 0$,
we have  
\beno
\mbP(\mE^c(\epsilon))  
&\leq&  \dfrac{4}{\mbE \, \norm{\bY}_1} + \dfrac{8 \, \left[D_{g}\right]^{+}}{\left(\mbE \, \norm{\bY}_1 \right)^2}.   
\ee
Combining terms, 
we obtain the bound
\beno 
\left| \mbP(\bS_\mN \in \mA) - \Phi(\bZ \in \mA) \right|
&\leq& \dfrac{83}{\xi_{\epsilon^\star}^{3/2}} \, 
\sqrt{\dfrac{p^{7/2}}{\mbE \, \norm{\bY}_1}}
+  \dfrac{4}{\mbE \, \norm{\bY}_1} + \dfrac{8 \, \left[D_{g}\right]^{+}}{\left(\mbE \, \norm{\bY}_1 \right)^2}.
%(\mbE\,\norm{\bY}_1 - \epsilon)^{-1/2}  \,\xi_{\epsilon_\star}^{-3/2} & + & \dfrac{1}{(\mbE \, \norm{\bY}_1)^{1/2}}.
\ee
\qed
\s \s


Note that the asymptotic multivariate normality can be established provided 
\beno
\lim\limits_{N \to \infty} \, \left[
\dfrac{83}{\xi_{\epsilon^\star}^{3/2}} \,
\sqrt{\dfrac{p^{7/2}}{\mbE \, \norm{\bY}_1}}
+  \dfrac{4}{\mbE \, \norm{\bY}_1} + \dfrac{8 \, \left[D_{g}\right]^{+}}{\left(\mbE \, \norm{\bY}_1 \right)^2} 
\right]
\= 0,  
\ee
resulting in following the asymptotic convergence in distribution:  
%We assume $\mbE \, \norm{\bY}_1  > \epsilon$, and when $\mbE \, \norm{\bY}_1$ scales with the number of observations and $\left[ D_{g} \right]^{+} = o(\mbE \, \norm{\bY}_1)$, the vector of sufficient statistics $\bs(\bX)$ of the multilayer network $\bX$ given $\bY$ is asymptotically multivariate normal, and
\beno
\bS_\mN &\overset{D}{\longrightarrow}& \bZ &\sim& \text{MvtNorm}\left(\textbf{0}, \, \bI_p \right). 
\ee

\s\s
\section{Proof of Theorem \ref{thm2}}
\label{sec:pf_thm2}

In order to demonstrate the feasibility of the normal approximation for maximum likelihood estimators $\mle$  of $\truth$,
we first start with a standard Taylor expansion of the negative score equation:
\be
\label{eq:expansion} 
-\nabla_{\nat} \, \ell(\mle; \bx, \by)
\= -\nabla_{\nat} \,  \ell(\truth; \bx, \by) - \nabla_{\nat}^2 \, \ell(\truth; \bx, \by) \, (\mle - \truth) - \bR,
\ee
where $\bR \in \mbR^p$ is the vector of remainders given in the Lagrange form. 
Denoting by $R_i$, $(\mle - \truth)_i$, and $(\nabla_{\nat} \, \ell(\nat;\bx,\by))_i$ the $i^{\text{th}}$ component of $\bR$, $(\mle - \truth)$,
and the score function 
$\nabla_{\nat} \, \ell(\nat;\bx,\by)$,
respectively.
The remainder term $R_i$ ($i = 1, \ldots, p$) is given by 
\beno
R_i 
\=  \dsum_{j=1}^p \, \dfrac{1}{2} \, 
\dfrac{\partial^2\,(\nabla_{\nat} \,\ell(\dot\nat_i;\bx,\by))_i}{\partial \, \theta_j^2} 
(\mle - \truth)_j^2 \s \\
&& + \; \dsum_{1 \le j < k \le p} \, \dfrac{\partial^2 \, 
(\nabla_{\nat} \ell(\dot\nat_i; \bx,\by))_i}{\partial\, \theta_j \, \partial\, \theta_k} \, 
(\mle - \truth)_j \, (\mle - \truth)_k,
\ee
where $\dot\nat_i = t_i \, \mle + (1 - t_i) \, \truth$ (for some $t_i \in [0, 1]$).  
By Proposition \ref{prop:inference}, 
\beno
\ell(\nat; \bx, \by) 
\= \log \, \mbP_{\nat}(\bX = \bx \,|\, \bY = \by) + \log \, g(\by). 
\ee
By Lemma \ref{lem:s_hetero}, 
the probability mass function $\mbP_{\nat}(\bX = \bx \,|\, \bY = \by)$ 
belongs to a minimal exponential family with the sufficient statistic vector $\bs(\bx)$ 
given by equation \eqref{eq:suff} in Lemma \ref{lem:s_hetero}.  
We then have,
\beno
& - \nabla_{\nat} \, \ell(\nat;\bx, \by) 
\= -(\bs(\bx) - \mbE_{\nat}^{\by} \, \bs(\bX)) \s \\ 
& - \nabla_{\nat}^2 \, \ell(\nat;\bx,\by)
\= \var_{\nat}^{\by} \,  \bs(\bX)
\quad = \quad \mcI(\truth) \, \norm{\by}_1,
\ee
where $\mbE_{\nat}^{\by}$ and $\var_{\nat}^{\by}$ are the 
conditional expectation and variance operators,
respectively, 
of the conditional distribution of $\bX$ given $\bY = \by$,
and by using standard formulas for exponential families 
\citep[e.g., Proposition 3.8, pp. 29,][]{Su19}
and the results of Lemma \ref{lem:min-eig}.  
Note $\nabla_{\nat} \,  \ell(\mle; \bx,\by) = 0$, 
as the maximum likelihood estimator $\mle$ solves the score equation by definition. 
We re-arrange \eqref{eq:expansion} and multiply both sides by $(\mcI(\truth) \, \norm{\bY}_1)^{-1/2}$ to obtain
\be
\label{eq:bridge}
&& (\mcI(\truth) \, \norm{\bY}_1)^{1/2} \, (\mle - \truth) - \; (\mcI(\truth) \, \norm{\bY}_1)^{-1/2} \, \bR\s \\ 
\= (\mcI(\truth) \, \norm{\bY}_1)^{-1/2} \, (s(\bX) - \mbE^{\bY} \, s(\bX)). 
\ee
Define $\Delta \coloneqq  (\mcI(\truth) \, \norm{\bY}_1)^{-1/2} \, \bR$.
Let $\mA \subset \mbR^p$ be any measurable convex subset of $\mbR^p$ 
and $\bZ \sim \text{MvtNorm}(\bm{0}_p, \, \bI_p)$.
We are interested in bounding the quantity  
\beno
\left|\mbP((\mcI(\truth) \, \norm{\bY}_1)^{1/2} \, (\mle - \truth) - \Delta \in \mA) - \Phi(\bZ \in \mA)\right|.
\ee 
Then from \eqref{eq:bridge}, 
\beno
\mbP\left((\mcI(\truth) \, \norm{\bY}_1)^{1/2} \, (\mle - \truth) - \Delta \in \mA\right) \s \\
 = \quad \mbP\big((\mcI(\truth) \, \norm{\bY}_1)^{-1/2} \, (s(\bX) - \mbE^{\bY} \, s(\bX)) \in \mA\big).  
\ee
Applying Proposition \ref{prop:suff_norm},
for all measurable convex sets $\mA \subseteq \mbR^p$, 
\beno
&& \left| \mbP\left((\mcI(\truth) \, \norm{\bY}_1)^{-1/2} \, (s(\bX) - \mbE^{\bY} \, s(\bX)) \in \mA\right) 
- \Phi(\bZ \in \mA) \right|
\ee
is bounded above by
\beno
\dfrac{83}{(\widetilde{\lambda}_{\min}^{\epsilon})^{3/2}} \,
\sqrt{\dfrac{p^{7/2}}{\mbE \, \norm{\bY}_1}}
+  \dfrac{4}{\mbE \, \norm{\bY}_1} + \dfrac{8 \, \left[D_{g}\right]^{+}}{\left(\mbE \, \norm{\bY}_1 \right)^2}. 
\ee
Hence, 
\beno
\left|\mbP((\mcI(\truth) \, \norm{\bY}_1)^{1/2} \, (\mle - \truth) - \Delta \in \mA) - \Phi(\bZ \in \mA)\right| 
\ee
is bounded above by 
\beno
\dfrac{83}{(\widetilde{\lambda}_{\min}^{\epsilon})^{3/2}} \,
\sqrt{\dfrac{p^{7/2}}{\mbE \, \norm{\bY}_1}}
+  \dfrac{4}{\mbE \, \norm{\bY}_1} + \dfrac{8 \, \left[D_{g}\right]^{+}}{\left(\mbE \, \norm{\bY}_1 \right)^2}. \s 
\ee
We lastly handle the term $\Delta$
by showing that $\norm{\Delta}_2$ is small with high probability. 
We first use standard vector/matrix norm inequalities to bound 
\beno
\norm{\Delta}_2
\= \norm{(\mcI(\truth) \, \norm{\bY}_1)^{-1/2} \, \bR}_2 
&\leq& \dfrac{\mnorm{\mcI(\truth)^{-1/2}}_2}{\sqrt{\norm{\bY}_1}} \; \norm{\bR}_2
&\leq& \dfrac{\norm{\bR}_2}{\sqrt{\widetilde{\lambda}_{\min}^{\epsilon}  \norm{\bY}_1}},  
\ee
noting that the spectral norm $\mnorm{\mcI(\truth)^{-1/2}}_2$ 
is equal to the largest eigenvalue of $\mcI(\truth)^{-1/2}$ which will be the 
reciprocal of the smallest eigenvalue of $\mcI(\truth)^{1/2}$,
which is bounded below by $\sqrt{\widetilde{\lambda}_{\min}^{\epsilon}}$. 
Using a standard result from the Taylor theorem for functions with multiple variables,
if for each $i = 1, \ldots, p$,
there exists constants $M_i > 0$ such that
\beno
\sup\limits_{\nat \in \mbR^p \,:\, \norm{\nat - \truth}_1 \leq \norm{\mle - \truth}_1} \,
\left| \, \dfrac{\partial^2 (\nabla_{\nat} \, \ell(\nat;\bx,\by))_i}
{\partial\, \theta_j \, \partial\, \theta_k} \, \right|
&\leq& M_i,
&& 1 \leq j \leq k \leq p, 
\ee
then the Lagrange remainder is bounded above by
\beno
R_i & \le & \dfrac{M_i}{2} \, \norm{\mle - \truth}_1^2
\ee
on the set $\{\nat \in \mbR^p : \norm{\nat - \truth}_1 \leq \norm{\mle - \truth}_2\}$. 
By Lemma \ref{lem:Taylor},
conditional on $\bY = \by$,
we have,
for all $i = 1, \ldots, p$, 
the bound $M_i \le 2 \, \norm{\by}_1$. 
Hence,
\be
\label{eq:Rbound1}
\norm{\Delta}_2
&\leq& \scalebox{0.9}{$\dfrac{1}{\sqrt{\widetilde{\lambda}_{\min}^{\epsilon} \, \norm{\by}_1}} \; 
\sqrt{\dsum_{i=1}^{p} \, R_i^2}  
\quad\leq\quad \dfrac{1}{\sqrt{\widetilde{\lambda}_{\min}^{\epsilon} \, \norm{\by}_1}} \, 
\sqrt{\dsum_{i=1}^{p} \, \norm{\by}_1^2 \, \norm{\mle - \truth}_1^4}$} \s\s\\
&\leq& \scalebox{0.9}{$\dfrac{1}{\sqrt{\widetilde{\lambda}_{\min}^{\epsilon} \, \norm{\by}_1}} \,
\sqrt{p \, \norm{\by}_1^2 \, \norm{\mle - \truth}_1^4}  
\quad\leq\quad \dfrac{\sqrt{p} \, \norm{\by}_1 \, \norm{\mle - \truth}_1^2}{\sqrt{\widetilde{\lambda}_{\min}^{\epsilon} \, \norm{\by}_1}}$} \s\s\\
&\leq& \dfrac{\sqrt{p} \, \sqrt{\norm{\by}_1} \, p \, \norm{\mle - \truth}_2^2}{\sqrt{\widetilde{\lambda}_{\min}^{\epsilon}}} 
\quad\leq\quad \dfrac{p^{3/2}\, \sqrt{\norm{\by}_1} \, \norm{\mle - \truth}_2^2}{\sqrt{\widetilde{\lambda}_{\min}^{\epsilon}}}. 
\ee
By Chebyshev's inequality---as in the proof of Lemma \ref{lem:concentration_likelihood}---we can establish that
\be
\label{eq:event1}
\mbP\left(\left|\norm{\bY}_1 - \mbE \, \norm{\bY}_1\right| > \dfrac{1}{2} \, \mbE \, \norm{\bY}_1\right)
&\leq& \dfrac{4}{\mbE \, \norm{\bY}_1} + \dfrac{8 \, [D_{g}]^{+}}{(\mbE \, \norm{\bY}_1)^2}.
\ee
Under Assumptions \ref{assump1}, \ref{assump2} and \ref{assump3}, Theorem \ref{thm1} established that there exist constants $C>0$ and $N_0 \geq 3$ such that,
for all $N \geq N_0$, 
the event 
\be
\label{eq:event2}
\norm{\mle - \truth}_2 &\leq&
C \; \dfrac{\sqrt{\widetilde{\lambda}_{\max}^{\star}} }{\widetilde{\lambda}_{\min}^{\epsilon}} \; \sqrt{\dfrac{p}{\mbE \norm{\bY}_1}}
\ee
occurs with probability at least $1 - \exp\,(-2\,p) \, - \, (\mbE \, \norm{\bY}_1)^{-1}$.
Define $\mE_1$ to be the event
\beno 
|\norm{\bY}_1 - \mbE \, \norm{\bY}_1| 
&\leq& \dfrac{1}{2} \, \mbE \, \norm{\bY}_1
\ee
and $\mE_2$ to be the event in \eqref{eq:event2},
and define $\mcR$ to be the corresponding values of $\Delta$ in
the event $(\bX, \bY) \in \mE_1 \cap \mE_2$,
under which we have the bound
\be
\label{eq:Rbound2}
\norm{\Delta}_2
&\leq& \dfrac{p^{3/2}\, \sqrt{\norm{\by}_1}}{\sqrt{\widetilde{\lambda}_{\min}^{\epsilon}}} \, 
C^2 \; \dfrac{\widetilde{\lambda}_{\max}^{\star} }{(\widetilde{\lambda}_{\min}^{\epsilon})^2} \; \dfrac{p}{\mbE \norm{\bY}_1} \s\s\\
&\leq& \dfrac{C^2 \, p^{5/2} \, \sqrt{2 \, \mbE \, \norm{\bY}_1}}{\mbE \norm{\bY}_1} \, 
\dfrac{\widetilde{\lambda}_{\max}^{\star} }{(\widetilde{\lambda}_{\min}^{\epsilon})^{5/2}} \s \\
\= \dfrac{\sqrt{2} \, C^2 \, p^{5/2}}{\sqrt{\mbE \norm{\bY}_1}} \, 
\dfrac{\widetilde{\lambda}_{\max}^{\star} }{(\widetilde{\lambda}_{\min}^{\epsilon})^{5/2}}.
\ee
The first inequality in \eqref{eq:Rbound2} is obtained by combining the bounds in \eqref{eq:Rbound1} and \eqref{eq:event2}. The second inequality in \eqref{eq:Rbound2} is using the fact that 
\beno
\norm{\by}_1 
&\leq&  \mbE \, \norm{\bY}_1 + \dfrac{1}{2} \, \mbE \, \norm{\bY}_1 
&\leq& 2 \, \mbE \, \norm{\bY}_1
\ee 
in the event $\by \in \mE_1$.  
Moreover,
a union bound shows that
\beno
\mbP(\Delta \not\in \mcR) &\leq&
\mbP(\mE_1^c) \, + \, \mbP(\mE_2^c)
\s \\
&\leq & \exp\,(-2\,p) \, + \, \dfrac{5}{\mbE \, \norm{\bY}_1} + \dfrac{8 \, [D_{g}]^{+}}{(\mbE \, \norm{\bY}_1)^2} \s \\
&\leq& \exp\,(-2\,p) \, + \, \dfrac{5+ 8 \, C_0}{\mbE \, \norm{\bY}_1}, 
\ee
where the constant $C_0$ and the last inequality follow from Assumption \ref{assump1}.
Hence, 
\beno
\label{eq:R_bound}
\mbP\left(\norm{\Delta}_2 \leq 
\dfrac{\sqrt{2} \, C^2 \, p^{5/2}}{\sqrt{\mbE \norm{\bY}_1}} \, 
\dfrac{\widetilde{\lambda}_{\max}^{\star} }{(\widetilde{\lambda}_{\min}^{\epsilon})^{5/2}} \right)
&\geq& 1 - \exp\,(-2\,p) \, - \, \dfrac{5+ 8 \, C_0}{\mbE \, \norm{\bY}_1}. 
\ee
Taken together, 
we have shown under the assumptions of Theorem \ref{thm1} that there exists $N_0 \geq 3$ such that,
for all $N \geq N_0$,
the error of the multivariate normal approximation 
\beno
\left|\mbP((\mcI(\truth) \, \norm{\bY}_1)^{1/2} \, (\mle - \truth) - \Delta \in \mA) - \Phi(\bZ \in \mA)\right|
\ee
is bounded above by
\beno 
\dfrac{83}{(\widetilde{\lambda}_{\min}^{\epsilon})^{3/2}} \,
\sqrt{\dfrac{p^{7/2}}{\mbE \, \norm{\bY}_1}}
+  \dfrac{4}{\mbE \, \norm{\bY}_1} + \dfrac{8 \, \left[D_{g}\right]^{+}}{\left(\mbE \, \norm{\bY}_1 \right)^2}
\ee
where $\Delta$ satisfies 
\beno
\mbP\left(\norm{\Delta}_2 \leq 
\dfrac{\sqrt{2} \, C^2 \, p^{5/2}}{\sqrt{\mbE \norm{\bY}_1}} \, 
\dfrac{\widetilde{\lambda}_{\max}^{\star} }{(\widetilde{\lambda}_{\min}^{\epsilon})^{5/2}} \right)
&\geq& 1 - \exp\,(-2\,p) \, - \, \dfrac{5+ 8 \, C_0}{\mbE \, \norm{\bY}_1}.
\ee

\qed







\subsection{Auxiliary results for proof of Theorem \ref{thm2}} 
\begin{lemma}
\label{lem:Taylor}
Consider a separable multilayer network model following the form of equation \eqref{general_model} and is 
defined on a set of $N \geq 3$ 
and $K \geq 1$ layers and denote by $\ell(\nat; \bx,\by)$ the log-likelihood function.
Then,
for each  $i = 1, \ldots, p$ and $\epsilon > 0$,
\beno
\sup\limits_{\nat \in \mbR^p \,:\, \norm{\nat - \truth}_2 \leq \epsilon} \quad  
\left| \dfrac{\partial^2 \, (\nabla_{\nat} \, \ell(\nat;\bx,\by))_i}{\partial\, \theta_j \, \partial\, \theta_k} \right|
&\leq& 2 \, \norm{\by}_1,
\ee 
where $(\nabla_{\nat} \, \ell(\nat;\bx,\by))_i$ is the $i^{th}$ component of the score function 
$\nabla_{\nat} \, \ell(\nat;\bx,\by)$. 
\end{lemma}

\s 

\llproof \ref{lem:Taylor}. 
By Proposition \ref{prop:inference}, 
given the observation $\bx$ of $\bX$ (i.e., observation of the event $\bX = \bx$), 
$\bY$ is predictable with unique value $\by \in \mbY$ 
given by the formula in Proposition \ref{prop:inference},
and $(\bx, \by)$ is network concordant. 
Further, 
by Proposition \ref{prop:inference}  
\beno
\ell(\nat; \bx, \by)
\= \log \, \mbP_{\nat}(\bX = \bx \mid \bY = \by) 
+ \log \, g(\by), 
\ee
where $\log \, \mbP_{\nat}(\bX = \bx | \bY = \by)$ 
is the log-likelihood of a minimal, full, and regular 
exponential family. 
Thus, 
the second order derivative of 
$\ell(\nat; \bx, \by)$ with respect to the $i^{\text{th}}$ and $j^{\text{th}}$ 
components of $\nat$ correspond to the variance (in the case $i = j$) 
or covariance (in the case of $i \neq j$) 
of corresponding sufficient statistic(s) of the exponential family \citep[e.g., Proposition 3.8, p. 29,][]{Su19},
with sufficient statistics given in Lemma \ref{lem:s_hetero}.  
For $\{i, j\} \subseteq \{1, \ldots,p\}$, 
\beno
\dfrac{\partial \, (\nabla_{\nat} \, \ell(\nat; \bx, \by))_i}{\partial\, \theta_j} 
\= \dfrac{\partial^2 \, \ell(\nat;\bx,\by)}{\partial\, \theta_i \; \partial\, \theta_j}
\= \cov_{\nat}(s_i(\bX),s_j(\bX) \,|\, \bY = \by),
\ee
and when $i = j \in \{1, \ldots, p\}$,
\beno
\dfrac{\partial \, (\nabla_{\nat} \, \ell(\nat;\bx,\by))_i}{\partial\, \theta_i} 
\= \dfrac{\partial^2 \, \ell(\nat; \bx, \by)}{\partial \, \theta_i^2}
\= \var_{\nat}(s_i(\bX) \,|\, \bY = \by).
\ee
As a result, for $\{i,j\} \subseteq \{1,\ldots,p\}$ and $k \in \{1, \ldots, p\}$,
\beno
\left| \, \dfrac{\partial^2 \, (\nabla_{\nat} \, \ell(\nat;\bx,\by))_i}{\partial\, \theta_j \, \partial\, \theta_k} \, \right| 
\=
\left| \, \dfrac{\partial \,\cov_{\nat}(s_i(\bX),s_j(\bX) \,|\, \bY = \by)}{\partial \, \theta_k} \, \right|,
\ee
and when $i = j \in \{1, \ldots, p\}$ and $k \in \{1, \ldots, p\}$,
\beno
\left| \, \dfrac{\partial^2 \, (\nabla_{\nat} \, \ell(\nat;\bx,\by))_i}{\partial\, \theta_i \, \partial\, \theta_k} \, \right| 
\=
\left| \, \dfrac{\partial \,\var_{\nat}(s_i(\bX) \,|\, \bY = \by)}{\partial \, \theta_k} \, \right|.
\ee
By Lemma \ref{lem:s_hetero} equation \eqref{eq:suff}, 
conditional on $\bY = \by$, 
each sufficient statistic $s_i(\bX)$ ($i \in \{1, \ldots, p\}$) 
can be decomposed into the sum of conditionally independent statistics of each dyad 
$\bX_{v,w}$, for $\{ v,w \} \subseteq \mN$. 
We can then write 
\beno
\scalebox{0.95}{$\cov_{\nat}(s_i(\bX),s_j(\bX) \,|\, \bY = \by)$} 
\=  \scalebox{0.95}{$\dsum_{\{v,w\} \subset \mN} \, \cov_{\nat}(s_{i,v,w}(\bX_{v,w}), \, s_{j,v,w}(\bX_{v,w}) \,|\, \bY = \by)$}, 
\ee
noting that by conditional independence 
$\cov_{\nat}(s_{i,v,w}(\bX_{v,w}),s_{j,r,t}(\bX_{r,t}) \,|\, \bY = \by) = 0$
whenever $\{r,t\} \neq \{v,w\}$, 
and when $i = j$, we can write
\beno
\var_{\nat}(s_i(\bX) \,|\, \bY = \by) 
\= \dsum_{\{v,w\} \subset \mN} \,  \var_{\nat}(s_{i,v,w}(\bX_{v,w}) \,|\, \bY = \by),
\ee
again appealing to the conditional independence given $\bY$ of the random variables 
$s_{i,v,w}(\bX_{v,w})$ ($\{v,w\} \subset \mN$). 
As a result, for $k \in \{1, \ldots, p\}$, it suffices to show that, 
\beno
\left|\,\dfrac{\partial \,\cov_{\nat}(s_{i,v,w}(\bX_{v,w}), \, s_{j,v,w}(\bX_{v,w}) \,|\, \bY = \by )}{\partial \, \theta_k} \,\right|
& \leq & 2,
\ee
and 
\beno
\left| \,\dfrac{\partial \,\var_{\nat}(s_{i,v,w}(\bX_{v,w}) \,|\, \bY = \by)}{\partial \, \theta_k} \, \right|
& \leq & 1.
\ee
Recall that the sufficient statistic $s_{i,v,w}(\bX)$ ($i = 1, \ldots ,p$) 
is defined in Lemma \ref{lem:s_hetero} by
\beno
s_{i,v,w}(\bX_{v,w}) \= \dprod_{t=1}^{h} \, X_{v,w}^{(k_{t})},  
&& \{v,w\} \subset \mN, 
\ee
for some $h \in \{1, \ldots, H\}$ 
and $\{k_1, \ldots, k_h\} \subseteq \{1, \ldots, K\}$. 
Define the set $S_{i,v,w}$ of components of the sufficient statistic vector $\bs_{v,w}(\bX)$ for $\{v,w\} \subset \mN$ and $i = 1,\ldots,p$ by
\beno
S_{i,v,w} \; \coloneqq \; \left\{ \dprod_{t=1}^{h^\prime} \, X_{v,w}^{(l_{t})} \; : \; \{l_1, \ldots, l_{h^\prime} \}  \subset \{k_1,\ldots,k_h\}, \; h^\prime < h \right\},
\ee
where $h \in \{1, \ldots, H\}$ and $\{k_1, \ldots, k_h\} \subseteq \{1, \ldots, K\}$.
The set $S_{i,v,w}$ is the set of components of the sufficient statistic vector $\bs_{v,w}(\bX)$ of dyad $\{v,w\} \subset \mN$ that have a value of 1 when $s_{i,v,w}(\bX) = 1$.
For the ease of notation, 
let $I_{S_{i,v,w}}$ be the set of dimension indices whose corresponding components of the sufficient statistic vector $\bs_{v,w}(\bX)$ belong to the set $S_{i,v,w}$:
\beno
I_{S_{i,v,w}} \; \coloneqq \; \{ j \in \{1,\ldots,p\} \; : \; s_{j,v,w}(\bX) \in S_{i,v,w} \}.
\ee
Define the conditional expectation of $s_{i,v,w}(\bX)$ given $\bY = \by$ for any $i \in \{1,\ldots,p\}$ and $\{v,w\} \subset \mN$ by
\beno
P_{i,v,w}(\nat ; \bX, \by) \; \coloneqq  \; \mbP_{\nat} \, (s_{i,v,w}(\bX) = 1 \, | \, \bY = \by). 
\ee
Denote by $L_i$ the set of layer indices $\{k_1, \ldots, k_h\} \subseteq \{1, \ldots, K\}$ that define the $i^{th}$ component $s_{i,v,w}(\bX_{v,w})$ of the sufficient statistic vector $\bs_{v,w}(\bX)$ 
for any $\{v,w\} \subset \mN$, $j \in \{1,\ldots,p\}$, and some $h \in \{1, \ldots, H\}$.
We then define
\beno
\bX_{v,w }^{(L_i)} & \coloneqq & \left\{X_{v,w}^{(k_1)}, \ldots, X_{v,w}^{(k_h)} \right\}, & \; &
\bX_{v,w }^{(-L_i)} \; \coloneqq \; \bX_{v,w} \setminus \bX_{v,w }^{(L_i)}, 
\ee
and the corresponding sample space
\beno
\mbX_{v,w }^{(L_i)} & \coloneqq & \{0,1\}^h, & \; &
\mbX_{v,w }^{(-L_i)} \; \coloneqq \; \{0,1\}^{H-h}, 
\ee
for some $h \in \{1, \ldots, H\}$.
Then we can write
\beno
P_{i,v,w}(\nat ; \bX, \by) 
\= \mbP_{\nat}\left(\dprod_{l \in L_i} \, X_{v,w}^{(l)} = 1 \, | \, \bY = \by \right) \s \\
\= \dfrac{\dsum_{\mbX_{v,w }^{(-L_i)}} \, \exp\,\left(\sum_{j \in I_{S_{i,v,w}}}\,\theta_j + \sum_{j \in I_{S_{i,v,w}}^c}\, \theta_j\, s_{j,v,w}(\bx)\right)}{\dsum_{\mbX_{v,w}} \, \exp\left( \sum_{j = 1}^p \, \theta_j \, s_{j,v,w}(\bx) \right)}.
\ee
Let 
\beno
Z(\nat) & \coloneqq & \dsum_{\mbX_{v,w}} \, \exp\left( \sum_{j = 1}^p \, \theta_j \, s_{j,v,w}(\bx) \right),
\ee
and take the derivative of $P_{i,v,w}(\nat; \bx,\by)$ with respect to $\theta_k$ for $k = 1,\ldots,p$. We have
\be
\label{cov_ineq}
\dfrac{\partial\, P_{i,v,w}(\nat ; \bX, \by) }{\partial\,\theta_k} \s \\
 \leq \quad
 \scalebox{0.9}{$\dfrac{\dsum_{\mbX_{v,w }^{(-L_i)}} \, \exp\,\left(\sum_{j \in I_{S_{i,v,w}}}\,\theta_j + \sum_{j \in I_{S_{i,v,w}}^c}\, \theta_j\, s_{j,v,w}(\bx)\right) \, \left(Z(\nat) - \dfrac{\partial \, Z(\nat) }{ \partial \, \theta_k}\right)}{Z(\nat)^2}$} \s \\
 =  \scalebox{0.7}{$\dfrac{\dsum_{\mbX_{v,w }^{(-L_i)}} \, \exp\,\left(\sum_{j \in I_{S_{i,v,w}}}\,\theta_j + \sum_{j \in I_{S_{i,v,w}}^c}\, \theta_j\, s_{j,v,w}(\bx)\right) \, \left(\dsum_{\mbX_{v,w}} \, \exp\left( \sum_{j = 1}^p \, \theta_j \, s_{j,v,w}(\bx)\right) \, (1-s_{k,v,w}(\bx))\right)}{Z(\nat)^2}$} \s \\
 \leq \quad 
 \scalebox{0.9}{$\dfrac{\dsum_{\mbX_{v,w }^{(-L_i)}} \, \exp\,\left(\sum_{j \in I_{S_{i,v,w}}}\,\theta_j + \sum_{j \in I_{S_{i,v,w}}^c}\, \theta_j\, s_{j,v,w}(\bx)\right) \, Z(\nat)}{Z(\nat)^2}$} \s \\
 \leq \quad
1.
\ee
The first inequality is obtained because $s_{k,v,w} (\bx) \leq 1$, and the last inequality is due to the fact that
\beno
\dsum_{\mbX_{v,w }^{(-L_i)}} \, \exp\,\left(\sum_{j \in I_{S_{i,v,w}}}\,\theta_j + \sum_{j \in I_{S_{i,v,w}}^c}\, \theta_j\, s_{j,v,w}(\bx)\right) 
\ee
is bounded above by 
\beno
\dsum_{\mbX_{v,w}} \, \exp\left( \sum_{j = 1}^p \, \theta_j \, s_{j,v,w}(\bx) \right).
\ee
Now we turn to show the derivative of the conditional variance and covariance of the sufficient statistics of each dyad are bounded. Given $\bY = \by$, for all $\{i\} \subset \{ 1,\ldots, p\}$, $s_{i,v,w} (\bX)$ are conditionally independent across $\{v,w\} \subseteq \mN$.
Then we have 
\beno
& \cov_{\nat}(s_{i,v,w}(\bX_{v,w}), \, s_{j,v,w}(\bX_{v,w}) \,|\, \bY = \by) \s \\
 = & \scalebox{0.95}{$\mbE \,[ s_{i,v,w}(\bX) \, s_{j,v,w}(\bX) \,|\, \bY = \by] - \mbE\, [s_{i,v,w}(\bX)\,|\, \bY = \by] \, \mbE \, [s_{j,v,w}(\bX)\,|\, \bY = \by]$} \s \\
 = & \mbP_{\nat} \, (s_{i,v,w}(\bX)  = 1, s_{j,v,w}(\bX) =1 \,|\, \bY = \by) - P_{i,v,w} (\nat;\bX,\by) \, P_{j,v,w} (\nat;\bX,\by) \s \\
 = & \mbP_{\nat} \left( \dprod_{l \in L_i \cup L_j} \bX_{v,w}^{(l)} = 1  \, | \, \bY = \by \right) - P_{i,v,w} (\nat;\bX,\by) \, P_{j,v,w} (\nat;\bX,\by).
\ee
Using the inequality derived in \eqref{cov_ineq} and suppressing the notation of $\{v,w\}$ and $(\bX, \by)$ in $P_{i,v,w} (\nat;\bX,\by)$, the derivative of the covariance with respect to $\theta_k$, $k = 1,\ldots, p$ is given by
\beno
& \left| \, \dfrac{\partial \,\cov_{\nat}(s_{i,v,w}(\bX_{v,w}), \, s_{j,v,w}(\bX_{v,w}) \,|\, \bY = \by)}{\partial \, \theta_k} \, \right| \s \\
= & \dfrac{\partial \, \mbP_{\nat} \left(\dprod_{l \in L_i \cup L_j} \bX_{v,w}^{(l)} = 1  \, | \, \bY = \by \right)}{\partial \, \theta_k} - \dfrac{\partial \, P_i(\nat)}{\partial \, \theta_k} \, P_j(\nat) - P_i(\nat) \, \dfrac{\partial \, P_j(\nat)}{\partial \, \theta_k}  \s \\
 \leq & 2.
\ee
Using the same inequality and notation in \eqref{cov_ineq}, the derivative of the variance of a Bernoulli random variable $s_{i,v,w} (\bX)$ is given by
\beno
\left| \,\dfrac{\partial \,\var_{\nat}(s_{i,v,w}(\bX_{v,w}) \,|\, \bY = \by)}{\partial \, \theta_k} \, \right| 
\= \left|\,\left(1 - 2 \, P_i(\nat) \right) \, \dfrac{\partial \, P_i(\nat)}{\partial \, \theta_k} \,\right|
& \leq & 
1.
\ee
Finally, for $\{i,j\} \subseteq \{1,\ldots,p\}$ and $k \in \{1, \ldots, p\}$, we obtain
\beno
\left| \, \dfrac{\partial^2 \, (\nabla_{\nat} \, \ell(\nat;\bx,\by))_i}{\partial\, \theta_j \, \partial\, \theta_k} \, \right| 
\=
\left| \, \dfrac{\partial \,\cov_{\nat}(s_i(\bX),s_j(\bX) \,|\, \bY = \by)}{\partial \, \theta_k} \, \right| \s \\
& \leq & \scalebox{0.95}{$\dsum_{\{v,w\} \subset \mN} \, \left| \dfrac{\partial \, \cov_{\nat}(s_{i,v,w}(\bX_{v,w}), \, s_{j,v,w}(\bX_{v,w}) \,|\, \bY = \by)}{\partial \, \theta_k} \, \right|$} \s \\
& \leq & 2 \, \norm{\by}_1
\ee
due to the fact that $\cov_{\nat}(s_{i,v,w}(\bX_{v,w}), \, s_{j,v,w}(\bX_{v,w}) \,|\, \bY = \by) = 0$ when $Y_{v,w} = 0$ for $\{v,w\} \subset \mN$.
Similarly, $\var_{\nat}(s_{i,v,w}(\bX_{i,v,w}) \,|\, \bY = \by) = 0$ when $Y_{v,w} = 0$ for $\{v,w\} \subset \mN$, 
and when $i = j \in \{1, \ldots, p\}$ and $k \in \{1, \ldots, p\}$, we have
\beno
\left| \, \dfrac{\partial^2 \, (\nabla_{\nat} \, \ell(\nat;\bx,\by))_i}{\partial\, \theta_i \, \partial\, \theta_k} \, \right| 
\=
\left| \, \dfrac{\partial \,\var_{\nat}(s_i(\bX) \,|\, \bY = \by)}{\partial \, \theta_k} \, \right| \s \\
& \leq & \norm{\by}_1.
\ee

\qed



\section{Additional simulation results} 
\label{sec:add_sim} 

Additional simulation results that enhance those contained in Section \ref{sec:sim} are provided in this section.


\subsection{Normal approximation with different basis networks}
\label{subsec:norm_sim}

The multivariate normality of $\mple$ is tested by Zhou-Shao's multivariate normal test \citep{Zhou13}, and the p-values are provided in tabel \ref{ZS-test}. Q-Q plots of $\mple$ estimated from 6 different model-generating parameters with a dense Bernoulli basis network, a sparse Bernoulli basis network, a stochastic block model (SBM) generated basis network, and a latent space model (LSM) generated basis network are shown in Figure \ref{qqplot_dense}, \ref{qqplot_sparse}, \ref{qqplot_SBM} and \ref{qqplot_LSM}, respectively.

\begin{table}[t]
\begin{center}
\caption{\label{ZS-test} P-values of the Zhou-Shao's test for multivariate normality of $\mple$ for 6 model-generating parameters ($\truth_1$, $\truth_2$, $\truth_3$, $\truth_4$, $\truth_5$, $\truth_6$) estimated from 250 network samples at size 1000 on four basis network structures. All p-values are larger than .1. \s} 
\begin{tabular}{ l  r r r  r r  r  } 
\hline
 Basis network model  & $\truth_1$ & $\truth_2$  & $\truth_3$ & $\truth_4$ & $\truth_5$ & $\truth_6$ \\ 
\hline
  Dense Bernoulli & .138 & .473 & .053 & .699 & .587 & .983  \\

 Sparse Bernoulli & .554 & .132 & .232 & .634 & .904 & .373  \\

 SBM & .650 & .891 & .982 & .975 & .871 & .674 \\

 LSM  & .859 & .831 & .500 & .227 & .613 & .409  \\
\hline
\end{tabular}
\end{center}
\end{table}

\vspace{.25in}

\begin{center}
% Figure removed 
\captionof{figure}{Q-Q plots and p-values of six components of $\mple$ estimated from 250 multilayer network samples at size 1000 on the dense Bernoulli basis network for 6 model-generating parameters on each row.}\label{qqplot_dense}
\end{center}

\begin{center}
% Figure removed 
\captionof{figure}{Q-Q plots and p-values of six components of $\mple$ estimated from 250 multilayer network samples at size 1000 on the sparse Bernoulli basis network for 6 model-generating parameters on each row.}\label{qqplot_sparse}
\end{center}

\begin{center}
% Figure removed
\captionof{figure}{Q-Q plots and p-values of six components of $\mple$ estimated from 250 multilayer network samples at size 1000 on the SBM generated basis network for 6 model-generating parameters on each row.}\label{qqplot_SBM}
\end{center}


\begin{center}
% Figure removed
\captionof{figure}{Q-Q plots and p-values of six components of $\mple$ estimated from 250 multilayer network samples at size 1000 on the LSM generated basis network for 6 model-generating parameters on each row.}\label{qqplot_LSM}
\end{center}



\subsection{Additional results on the false discovery rate}
\label{more fdr}
The false discovery rate (FDR) of the multiple testing correction procedures of
Bonferroni, Benjamini-Hochberg, Hochberg, and Holm to detect non-zero components of $\truth$ at a family-wise significance level of $\alpha = 0.05$ with a sparse Bernoulli basis network, an SBM generated basis network and an LSM generated basis network are provided in Table \ref{fdr_sparse}, \ref{fdr_SBM} and \ref{fdr_LSM}, respectively (recall that components $\theta_{1,3}^\star$ and $\theta_{3}^\star$ of $\truth$ are set to 0). The receiver operating characteristic (ROC) curves for $\mple$ of 6 selected model-generating parameters on four basis network structures are provided in each of the subplot of Figure \ref{ROC}.


\begin{table}[t]
\begin{center}
\caption{\label{fdr_sparse} False discovery rates of four procedures for detecting non-zero effects of six model-generating parameters ($\truth_1$, $\truth_2$, $\truth_3$, $\truth_4$, $\truth_5$, $\truth_6$) estimated from 250 multilayer network samples at size 1000 on the sparse Bernoulli basis network. All FDRs are smaller than 0.05.} 
\begin{tabular}{l  r r r  r r  r } 
\hline
  Procedure & $\truth_1$ & $\truth_2$  & $\truth_3$ & $\truth_4$ & $\truth_5$ & $\truth_6$ \\ 
\hline
  Bonferroni    & .002  & .003 & .003 & .003 & .003 & .011  \\

  Benjamini-Hochberg   & .020 & .011 & .022 & .022 & .014 & .017  \\

 Hochberg's &    .009 & .008 & .012 & .010 & .010 & .014 \\

 Holm's  &   .007 & .008 & .011 & .009 & .006 &  .014 \\
\hline
\end{tabular}
\end{center}
\end{table}

\begin{table}%[b]
\begin{center}
\caption{\label{fdr_SBM} False discovery rates of four procedures for detecting non-zero effects of six model-generating parameters ($\truth_1$, $\truth_2$, $\truth_3$, $\truth_4$, $\truth_5$, $\truth_6$) estimated from 250 multilayer network samples at size 1000 on the SBM generated basis network. All FDRs are smaller than 0.05.} 
\begin{tabular}{l  r r r  r r  r } 
\hline
  Procedure & $\truth_1$ & $\truth_2$  & $\truth_3$ & $\truth_4$ & $\truth_5$ & $\truth_6$ \\ 
\hline
  Bonferroni    & .002  & .002 & .003 & .001 & .001 & .004  \\

  Benjamini-Hochberg   & .022 & .013 & .014 & .015 & .015 & .018  \\

 Hochberg's &    .009 & .014 & .01 & .008 & .011 & .014 \\

 Holm's  &   .009 & .013 & .005 & .009 & .009 &  .011 \\
\hline
\end{tabular}
\end{center}
\end{table}

\begin{table}%[b]
\begin{center}
\caption{\label{fdr_LSM} False discovery rates of four procedures for detecting non-zero effects of six model-generating parameters ($\truth_1$, $\truth_2$, $\truth_3$, $\truth_4$, $\truth_5$, $\truth_6$) estimated from 250 multilayer network samples at size 1000 on the LSM generated basis network. All FDRs are smaller than 0.05.} 
\begin{tabular}{l  r r r  r r  r } 
\hline
  Procedure & $\truth_1$ & $\truth_2$  & $\truth_3$ & $\truth_4$ & $\truth_5$ & $\truth_6$ \\ 
\hline
  Bonferroni    & .004  & .006 & .000 & .005 & .003 & .004  \\

  Benjamini-Hochberg   & .016 & .013 & .011 & .015 & .016 & .017  \\

 Hochberg's &    .009 & .014 & .009 & .011 & .010 & .011 \\

 Holm's  &   .008 & .014 & .009 & .011 & .007 &  .010 \\
\hline
\end{tabular}
\end{center}
\end{table}

% Figure environment removed

\pagebreak

\input{new-submission.bbl}



 




