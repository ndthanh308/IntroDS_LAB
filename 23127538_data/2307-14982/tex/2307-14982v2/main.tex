\documentclass{imsart}
%% Packages
\RequirePackage{amsthm,amsmath,amsfonts,amssymb}
\RequirePackage[numbers]{natbib}
%\RequirePackage[authoryear]{natbib}%% uncomment this for author-year citations
\RequirePackage[colorlinks,citecolor=blue,urlcolor=blue]{hyperref}
\RequirePackage{graphicx}

%\arxiv{2010.00000}
\startlocaldefs
%%%%%%%%%%%%%%%%%%%%%%%%%%%%%%%%%%%%%%%%%%%%%%
%%                                          %%
%% Uncomment next line to change            %%
%% the type of equation numbering           %%
%%                                          %%
%%%%%%%%%%%%%%%%%%%%%%%%%%%%%%%%%%%%%%%%%%%%%%
%\numberwithin{equation}{section}
%%%%%%%%%%%%%%%%%%%%%%%%%%%%%%%%%%%%%%%%%%%%%%
%%                                          %%
%% For Axiom, Claim, Corollary, Hypothesis, %%
%% Lemma, Theorem, Proposition              %%
%% use \theoremstyle{plain}                 %%
%%                                          %%
%%%%%%%%%%%%%%%%%%%%%%%%%%%%%%%%%%%%%%%%%%%%%%
\theoremstyle{plain}
\newtheorem{axiom}{Axiom}
\newtheorem{claim}[axiom]{Claim}
\newtheorem{theorem}{Theorem}[]
\newtheorem{lemma}[]{Lemma}
\newtheorem{corollary}{Corollary}
%%%%%%%%%%%%%%%%%%%%%%%%%%%%%%%%%%%%%%%%%%%%%%
%%                                          %%
%% For Assumption, Definition, Example,     %%
%% Notation, Property, Remark, Fact         %%
%% use \theoremstyle{definition}            %%
%%                                          %%
%%%%%%%%%%%%%%%%%%%%%%%%%%%%%%%%%%%%%%%%%%%%%%
\theoremstyle{definition}
\newtheorem{definition}[theorem]{Definition}
\newtheorem{assumption}{Assumption}
\newtheorem*{example}{Example}
\newtheorem*{fact}{Fact}

%%%%%%%%%%%%%%%%%%%%%%%%%%%%%%%%%%%%%%%%%%%%%%
%%                                          %%
%% For Case use \theoremstyle{remark}       %%
%%                                          %%
%%%%%%%%%%%%%%%%%%%%%%%%%%%%%%%%%%%%%%%%%%%%%%
\theoremstyle{remark}
\newtheorem{case}{Case}
%%%%%%%%%%%%%%%%%%%%%%%%%%%%%%%%%%%%%%%%%%%%%%
%% Please put your definitions here:        %%
%%%%%%%%%%%%%%%%%%%%%%%%%%%%%%%%%%%%%%%%%%%%%%
\usepackage{appendix}
%\usepackage[subtle]{savetrees}
%\usepackage[margin=2cm]{geometry}
\usepackage{tikz,amsmath, amssymb,bm,color, amsthm,amsfonts}
\usetikzlibrary{positioning, calc,chains,fit,shapes}
%\usetikzlibrary{circuits.logic.US,circuits.logic.IEC,fit}
\usepackage{enumerate}
\usepackage{comment}
\usepackage{tikz}
\usepackage{graphics}
%\usepackage[cm]{fullpage}
\usepackage{longtable}
\usepackage{mdframed}
\usepackage{caption}
\usepackage{subcaption}
\usepackage{slashbox}
\usepackage{url}
\usepackage{framed}
\usepackage{array}
\usepackage{tabu}
\usepackage{lscape}
\usepackage{multirow}
\usepackage{ulem}
\usepackage{multicol}
\usepackage{placeins}
\usepackage{cite}
\usepackage{enumitem}
\usepackage{mathtools}
%\usepackage[numbers]{natbib}
%\usepackage{mathtools}
%\usepackage{authblk}

\mdfsetup{skipabove=2pt,skipbelow=2pt}
%\setlenght {\marginparwidth }{2cm}
%\usepackage{todonotes}

%\usepackage{floatrow}
%\usepackage{adjustbox}
%\setlength{\extrarowheight}{.05ex}
%\renewcommand\thesubfigure{\roman{subfigure}}


%\newtheorem{theorem}{Theorem}[section]
%\newtheorem{lemma}[theorem]{Lemma}
%\newtheorem{observation}[theorem]{Observation}
%\newtheorem{corollary}[theorem]{Corollary}
%\newtheorem{proposition}[theorem]{Proposition}
%\newtheorem{definition}[theorem]{Definition}
\newtheorem{construction}{Construction}
%\newtheorem{conjecture}{Conjecture}
%\newtheorem{remark}[theorem]{Remark}

\newcommand{\pname}[1]{\textnormal{\textsc{#1}}}
\newcommand{\cclass}[1]{\textnormal{\textsf{#1}}}
\newcommand{\nog}{nine} % no of members in the gang!
\newcommand{\nogd}{nineteen} % no of members in the gang - for deletion/completion
\newcommand{\nogl}{eighteen} % no of members in the larger gang - for editing
\newcommand{\nogld}{thirty eight} % no of members in the larger gang - for deletion/completion
\newcommand{\diffnog}{ten} %
%\newcommand{\dominatedby}{dominated by} %
%\newcommand{\dominatingset}{dominating set} %
%\newcommand{\dominates}{dominates} %
\newcommand{\simulates}{simulates} %
\newcommand{\baseset}{base} %
\newcommand{\issimulatedby}{is simulated by} %

\newcommand{\StarSAT}{\pname{8-SAT$_{\geq 6}$}}
\newcommand{\FSAT}{\pname{4-SAT$_{\geq 2}$}}
\newcommand{\FISAT}{\pname{5-SAT$_{\geq 3}$}}
\newcommand{\SIXSAT}{\pname{6-SAT$_{\geq 4}$}}
\newcommand{\ESAT}{\pname{8-SAT$_{\geq 6}$}}
\newcommand{\KSAT}{\pname{$k$-SAT$_{\geq {k-2}}$}}
\newcommand{\KSATO}{\pname{$k$-SAT}}
\newcommand{\ESATO}{\pname{8-SAT}}
\newcommand{\FSATO}{\pname{4-SAT}}
\newcommand{\FISATO}{\pname{5-SAT}}
\newcommand{\TSAT}{\pname{3-SAT}}
\newcommand{\HED}{\pname{${H}$-free Edge Deletion}}
\newcommand{\AEE}{\pname{${A}$-free Edge Editing}}
\newcommand{\AED}{\pname{${A}$-free Edge Deletion}}
\newcommand{\TSED}{\pname{$t$-star-free Edge Deletion}}
\newcommand{\ATSED}{\pname{Annotated $t$-star-free Edge Deletion}}
\newcommand{\AFSED}{\pname{Annotated $4$-star-free Edge Deletion}}
\newcommand{\FSED}{\pname{$4$-star-free Edge Deletion}}
\newcommand{\FVSED}{\pname{$5$-star-free Edge Deletion}}
\newcommand{\HEE}{\pname{${H}$-free Edge Editing}}
\newcommand{\HEC}{\pname{${H}$-free Edge Completion}}
\newcommand{\HDEE}{\pname{${H'}$-free Edge Editing}}
\newcommand{\HDDEE}{\pname{${H''}$-free Edge Editing}}
\newcommand{\HDED}{\pname{${H'}$-free Edge Deletion}}
\newcommand{\HDEC}{\pname{${H'}$-free Edge Completion}}
\newcommand{\HBEE}{\pname{${\overline{H}}$-free Edge Editing}}
\newcommand{\HBED}{\pname{${\overline{H}}$-free Edge Deletion}}
\newcommand{\HBEC}{\pname{${\overline{H}}$-free Edge Completion}}
\newcommand{\HOEDCE}{\pname{${H_1}$-free Edge Deletion(Completion/Editing)}}
\newcommand{\HEDCE}{\pname{${H}$-free Edge Deletion(Completion/Editing)}}
\newcommand{\HEEDC}{\pname{${H}$-free Edge Editing(Deletion/Completion)}}
\newcommand{\HDEEDC}{\pname{${H'}$-free Edge Editing(Deletion/Completion)}}
\newcommand{\BFED}{\pname{Bow-free Edge Deletion}}
\newcommand{\ABFED}{\pname{Annotated Bow-free Edge Deletion}}
\newcommand{\DTIS}{\pname{Distance-3 Independent Set}}
\newcommand{\SVC}{\pname{Strong Vertex Cover}}
\newcommand{\CLIQUE}{\pname{Clique}}
\newcommand{\IS}{\pname{Independent Set}}
\newcommand{\PFS}{\pname{Propagational-$f$ Satisfiability}}
\newcommand{\RHED}{\pname{Restricted ${H}$-free Edge Deletion}}
\newcommand{\RHEC}{\pname{Restricted ${H}$-free Edge Completion}}
\newcommand{\RHDED}{\pname{Restricted ${H'}$-free Edge Deletion}}
\newcommand{\RHDEC}{\pname{Restricted ${H'}$-free Edge Completion}}
\newcommand{\RHEE}{\pname{Restricted ${H}$-free Edge Editing}}
\newcommand{\PH}{$\cclass{NP} \subseteq \cclass{coNP/poly}$}
\newcommand{\NOPH}{$\cclass{NP} \not\subseteq \cclass{coNP/poly}$}
\newcommand{\LG}{\mathcal{W}}
\newcommand{\LGD}{\mathcal{W}'}
\newcommand{\LGDD}{\mathcal{W}''}


%\let\oldvee\vee
\renewcommand\vee{\boxtimes}

\newcommand\addvmargin[1]{
  \node[fit=(current bounding box),inner ysep=#1,inner xsep=0]{};
}
\setlength{\fboxrule}{0pt}

\newcommand{\defstage}[2]{% PGD Version
  \hfill\\\smallskip\noindent%
  \begin{tabularx}{\textwidth}{|l X|}%
    \hline%
    \multicolumn{2}{|l|}{\textbf{#1}}\\%
    &#2\\\hline%
  \end{tabularx}%
%  \smallskip%
}
\setlength\extrarowheight{15pt}

\newcounter{rowcntr}[table]
\renewcommand{\therowcntr}{\thetable.\arabic{rowcntr}}

% A new columntype to apply automatic stepping
\newcolumntype{N}{>{\refstepcounter{rowcntr}\therowcntr}c}

% Reset the rowcntr counter at each new tabular
\AtBeginEnvironment{longtabu}{\setcounter{rowcntr}{0}}

\newcounter{rowcntra}[table]
\renewcommand{\therowcntra}{\arabic{rowcntra}}

% A new columntype to apply automatic stepping
\newcolumntype{M}{>{\refstepcounter{rowcntra}\therowcntra}c}

% Reset the rowcntr counter at each new tabular
\AtBeginEnvironment{tabular}{\setcounter{rowcntra}{0}}

\newcommand{\NPC}{NP-Complete}


\newcommand{\highlight}[1]{\textcolor{blue}{#1}}
\newcommand{\dhanya}[1]{\textcolor{blue}{dhanya: #1}}


%\newcommand{\XCD1}[1]{\pname{$\chi_{cd}$\ensuremath{(#1)}}}
\newcommand{\XCD}{\pname{$\chi_{cd}$}}
\newcommand{\SC}{\pname{$\omega_{s}$}}

\newcommand{\CDC}{\textsc{CD-coloring}}
\newcommand{\SCP}{\textsc{Separated-Cluster}}
\newcommand{\TD}{\textsc{Total Domination}}
\newcommand{\ISP}{\textsc{Independent Set}}
\newcommand{\CC}{\textsc{Clique Cover}}
\newcommand{\TETHS}{Further, the problem cannot be solved in time \ensuremath{2^{o(|V(G)|)}}, unless the ETH fails}
%\usetikzlibrary{positioning,chains,shapes,calc}
\usetikzlibrary{fit}
\thispagestyle{empty}
\usetikzlibrary{
  graphs,
  graphs.standard
}
\endlocaldefs



% \usepackage{amsmath}
% \usepackage{amssymb}
% \usepackage{enumerate}
% \usepackage{natbib}
% \usepackage{url} % not crucial - just used below for the URL 
% \usepackage{amsfonts}
% \usepackage{tikz}
% \usepackage{graphicx}
% \usepackage{mathtools}
% \usepackage{enumitem}
% \usepackage{bigints}
% \usepackage{mathrsfs}
% \usepackage{fancybox}
% \usepackage{pdfpages}

% \usetikzlibrary{arrows}
% \usetikzlibrary{positioning}
% \usepackage[colorlinks,citecolor=blue,urlcolor=blue]{hyperref}
% \usepackage{appendix}
% \usepackage{etoc}
% \usepackage[english]{babel}
% \newtheorem{theorem}{Theorem}
% \newtheorem{corollary}{Corollary}
% \newtheorem{lemma}{Lemma}
% \newtheorem{assumption}{Assumption}

% \usepackage{titlesec}

% \titleformat*{\section}{\large\bfseries}

% \usepackage[pagewise]{lineno}
% %\linenumbers

% %\usepackage[subtle]{savetrees}
%\usepackage[margin=2cm]{geometry}
\usepackage{tikz,amsmath, amssymb,bm,color, amsthm,amsfonts}
\usetikzlibrary{positioning, calc,chains,fit,shapes}
%\usetikzlibrary{circuits.logic.US,circuits.logic.IEC,fit}
\usepackage{enumerate}
\usepackage{comment}
\usepackage{tikz}
\usepackage{graphics}
%\usepackage[cm]{fullpage}
\usepackage{longtable}
\usepackage{mdframed}
\usepackage{caption}
\usepackage{subcaption}
\usepackage{slashbox}
\usepackage{url}
\usepackage{framed}
\usepackage{array}
\usepackage{tabu}
\usepackage{lscape}
\usepackage{multirow}
\usepackage{ulem}
\usepackage{multicol}
\usepackage{placeins}
\usepackage{cite}
\usepackage{enumitem}
\usepackage{mathtools}
%\usepackage[numbers]{natbib}
%\usepackage{mathtools}
%\usepackage{authblk}

\mdfsetup{skipabove=2pt,skipbelow=2pt}
%\setlenght {\marginparwidth }{2cm}
%\usepackage{todonotes}

%\usepackage{floatrow}
%\usepackage{adjustbox}
%\setlength{\extrarowheight}{.05ex}
%\renewcommand\thesubfigure{\roman{subfigure}}


%\newtheorem{theorem}{Theorem}[section]
%\newtheorem{lemma}[theorem]{Lemma}
%\newtheorem{observation}[theorem]{Observation}
%\newtheorem{corollary}[theorem]{Corollary}
%\newtheorem{proposition}[theorem]{Proposition}
%\newtheorem{definition}[theorem]{Definition}
\newtheorem{construction}{Construction}
%\newtheorem{conjecture}{Conjecture}
%\newtheorem{remark}[theorem]{Remark}

\newcommand{\pname}[1]{\textnormal{\textsc{#1}}}
\newcommand{\cclass}[1]{\textnormal{\textsf{#1}}}
\newcommand{\nog}{nine} % no of members in the gang!
\newcommand{\nogd}{nineteen} % no of members in the gang - for deletion/completion
\newcommand{\nogl}{eighteen} % no of members in the larger gang - for editing
\newcommand{\nogld}{thirty eight} % no of members in the larger gang - for deletion/completion
\newcommand{\diffnog}{ten} %
%\newcommand{\dominatedby}{dominated by} %
%\newcommand{\dominatingset}{dominating set} %
%\newcommand{\dominates}{dominates} %
\newcommand{\simulates}{simulates} %
\newcommand{\baseset}{base} %
\newcommand{\issimulatedby}{is simulated by} %

\newcommand{\StarSAT}{\pname{8-SAT$_{\geq 6}$}}
\newcommand{\FSAT}{\pname{4-SAT$_{\geq 2}$}}
\newcommand{\FISAT}{\pname{5-SAT$_{\geq 3}$}}
\newcommand{\SIXSAT}{\pname{6-SAT$_{\geq 4}$}}
\newcommand{\ESAT}{\pname{8-SAT$_{\geq 6}$}}
\newcommand{\KSAT}{\pname{$k$-SAT$_{\geq {k-2}}$}}
\newcommand{\KSATO}{\pname{$k$-SAT}}
\newcommand{\ESATO}{\pname{8-SAT}}
\newcommand{\FSATO}{\pname{4-SAT}}
\newcommand{\FISATO}{\pname{5-SAT}}
\newcommand{\TSAT}{\pname{3-SAT}}
\newcommand{\HED}{\pname{${H}$-free Edge Deletion}}
\newcommand{\AEE}{\pname{${A}$-free Edge Editing}}
\newcommand{\AED}{\pname{${A}$-free Edge Deletion}}
\newcommand{\TSED}{\pname{$t$-star-free Edge Deletion}}
\newcommand{\ATSED}{\pname{Annotated $t$-star-free Edge Deletion}}
\newcommand{\AFSED}{\pname{Annotated $4$-star-free Edge Deletion}}
\newcommand{\FSED}{\pname{$4$-star-free Edge Deletion}}
\newcommand{\FVSED}{\pname{$5$-star-free Edge Deletion}}
\newcommand{\HEE}{\pname{${H}$-free Edge Editing}}
\newcommand{\HEC}{\pname{${H}$-free Edge Completion}}
\newcommand{\HDEE}{\pname{${H'}$-free Edge Editing}}
\newcommand{\HDDEE}{\pname{${H''}$-free Edge Editing}}
\newcommand{\HDED}{\pname{${H'}$-free Edge Deletion}}
\newcommand{\HDEC}{\pname{${H'}$-free Edge Completion}}
\newcommand{\HBEE}{\pname{${\overline{H}}$-free Edge Editing}}
\newcommand{\HBED}{\pname{${\overline{H}}$-free Edge Deletion}}
\newcommand{\HBEC}{\pname{${\overline{H}}$-free Edge Completion}}
\newcommand{\HOEDCE}{\pname{${H_1}$-free Edge Deletion(Completion/Editing)}}
\newcommand{\HEDCE}{\pname{${H}$-free Edge Deletion(Completion/Editing)}}
\newcommand{\HEEDC}{\pname{${H}$-free Edge Editing(Deletion/Completion)}}
\newcommand{\HDEEDC}{\pname{${H'}$-free Edge Editing(Deletion/Completion)}}
\newcommand{\BFED}{\pname{Bow-free Edge Deletion}}
\newcommand{\ABFED}{\pname{Annotated Bow-free Edge Deletion}}
\newcommand{\DTIS}{\pname{Distance-3 Independent Set}}
\newcommand{\SVC}{\pname{Strong Vertex Cover}}
\newcommand{\CLIQUE}{\pname{Clique}}
\newcommand{\IS}{\pname{Independent Set}}
\newcommand{\PFS}{\pname{Propagational-$f$ Satisfiability}}
\newcommand{\RHED}{\pname{Restricted ${H}$-free Edge Deletion}}
\newcommand{\RHEC}{\pname{Restricted ${H}$-free Edge Completion}}
\newcommand{\RHDED}{\pname{Restricted ${H'}$-free Edge Deletion}}
\newcommand{\RHDEC}{\pname{Restricted ${H'}$-free Edge Completion}}
\newcommand{\RHEE}{\pname{Restricted ${H}$-free Edge Editing}}
\newcommand{\PH}{$\cclass{NP} \subseteq \cclass{coNP/poly}$}
\newcommand{\NOPH}{$\cclass{NP} \not\subseteq \cclass{coNP/poly}$}
\newcommand{\LG}{\mathcal{W}}
\newcommand{\LGD}{\mathcal{W}'}
\newcommand{\LGDD}{\mathcal{W}''}


%\let\oldvee\vee
\renewcommand\vee{\boxtimes}

\newcommand\addvmargin[1]{
  \node[fit=(current bounding box),inner ysep=#1,inner xsep=0]{};
}
\setlength{\fboxrule}{0pt}

\newcommand{\defstage}[2]{% PGD Version
  \hfill\\\smallskip\noindent%
  \begin{tabularx}{\textwidth}{|l X|}%
    \hline%
    \multicolumn{2}{|l|}{\textbf{#1}}\\%
    &#2\\\hline%
  \end{tabularx}%
%  \smallskip%
}
\setlength\extrarowheight{15pt}

\newcounter{rowcntr}[table]
\renewcommand{\therowcntr}{\thetable.\arabic{rowcntr}}

% A new columntype to apply automatic stepping
\newcolumntype{N}{>{\refstepcounter{rowcntr}\therowcntr}c}

% Reset the rowcntr counter at each new tabular
\AtBeginEnvironment{longtabu}{\setcounter{rowcntr}{0}}

\newcounter{rowcntra}[table]
\renewcommand{\therowcntra}{\arabic{rowcntra}}

% A new columntype to apply automatic stepping
\newcolumntype{M}{>{\refstepcounter{rowcntra}\therowcntra}c}

% Reset the rowcntr counter at each new tabular
\AtBeginEnvironment{tabular}{\setcounter{rowcntra}{0}}

\newcommand{\NPC}{NP-Complete}


\newcommand{\highlight}[1]{\textcolor{blue}{#1}}
\newcommand{\dhanya}[1]{\textcolor{blue}{dhanya: #1}}


%\newcommand{\XCD1}[1]{\pname{$\chi_{cd}$\ensuremath{(#1)}}}
\newcommand{\XCD}{\pname{$\chi_{cd}$}}
\newcommand{\SC}{\pname{$\omega_{s}$}}

\newcommand{\CDC}{\textsc{CD-coloring}}
\newcommand{\SCP}{\textsc{Separated-Cluster}}
\newcommand{\TD}{\textsc{Total Domination}}
\newcommand{\ISP}{\textsc{Independent Set}}
\newcommand{\CC}{\textsc{Clique Cover}}
\newcommand{\TETHS}{Further, the problem cannot be solved in time \ensuremath{2^{o(|V(G)|)}}, unless the ETH fails}
%\usetikzlibrary{positioning,chains,shapes,calc}
\usetikzlibrary{fit}
\thispagestyle{empty}
\usetikzlibrary{
  graphs,
  graphs.standard
}
% \newcommand{\qed}{$\hfill\blacksquare$}

% %\pdfminorversion=4
% % NOTE: To produce blinded version, replace "0" with "1" below.
% \newcommand{\blind}{1}

% % DON'T change margins - should be 1 inch all around.
% \addtolength{\oddsidemargin}{-.5in}%
% \addtolength{\evensidemargin}{-1in}%
% \addtolength{\textwidth}{1in}%
% \addtolength{\textheight}{1.7in}%
% \addtolength{\topmargin}{-1in}%


% \pdfminorversion=4


\begin{document}
\begin{frontmatter}
\title{Learning cross-layer dependence structure in multilayer networks}
%\title{A sample article title with some additional note\thanksref{t1}}
\runtitle{Learning cross-layer dependence structure in multilayer networks}
%\thankstext{T1}{A sample additional note to the title.}

\begin{aug}
%%%%%%%%%%%%%%%%%%%%%%%%%%%%%%%%%%%%%%%%%%%%%%%
%% Only one address is permitted per author. %%
%% Only division, organization and e-mail is %%
%% included in the address.                  %%
%% Additional information can be included in %%
%% the Acknowledgments section if necessary. %%
%% ORCID can be inserted by command:         %%
%% \orcid{0000-0000-0000-0000}               %%
%%%%%%%%%%%%%%%%%%%%%%%%%%%%%%%%%%%%%%%%%%%%%%%
\author[A]{\fnms{Jiaheng}~\snm{Li}\ead[label=e1]{jl20gx@fsu.edu}},
\author[A]{\fnms{Jonathan}~\snm{Stewart}\ead[label=e2]{jrstewart@fsu.edu}}

%%%%%%%%%%%%%%%%%%%%%%%%%%%%%%%%%%%%%%%%%%%%%%
%% Addresses                                %%
%%%%%%%%%%%%%%%%%%%%%%%%%%%%%%%%%%%%%%%%%%%%%%
\address[A]{Department of Statistics,
Florida State University\printead[presep={,\ }]{e1,e2}}
\runauthor{J. Li et al.}
\end{aug}
%%%%%%%%%%%%%%%%%%%%%%%%%%%%%%%%%%%%%%%%%%%%%%%%%%%%%%%%%%%%%%%%%%%%%%%%%%%%%%


% \if1\blind
% {
%   \title{\bf Learning cross-layer dependence structure in multilayer networks} 
%   \author{
%     Jiaheng Li\\ 
%     Department of Statistics, Florida State University \\\\ 
%     and \\\\
%     Jonathan R. Stewart\footnote{Corresponding author: Department of Statistics, Florida State University,
% 117 N Woodward Ave,
% Tallahassee, FL 32306-4330,
% E-mail:\ jrstewart@fsu.edu.} \\ Department of Statistics, Florida State University}
% %\thanks{Supported by the Test Resource Management Center (TRMC) within the Office of the Secretary of Defense (OSD), contract \#FA807518D0002.}
%   \maketitle
% } \fi

% \if0\blind
% {
%   \bigskip
%   \bigskip
%   \bigskip
%   \begin{center}
%     {\LARGE\bf Learning cross-layer dependence structure in multilayer networks}
% \end{center}
%   \medskip
% } \fi
\begin{abstract}
We propose a novel class of separable multilayer network models to capture cross-layer dependencies in multilayer networks,
enabling the analysis of how interactions in one or more layers may influence interactions in other layers.
Our approach separates the network formation process from the layer formation process,
and is able to extend existing single-layer network models to multilayer network models
that accommodate cross-layer dependence.
We establish non-asymptotic and minimax-optimal error bounds for maximum likelihood estimators
and demonstrate the convergence rate
in scenarios of increasing parameter dimension.
Additionally, we establish non-asymptotic error bounds for multivariate normal approximations and propose a model selection method that controls the false discovery rate. 
Simulation studies and an application to the Lazega lawyers network show that our framework and method perform well in realistic settings.
\end{abstract}

\begin{keyword}
\kwd{Multilayer networks}
\kwd{statistical network analysis}
\kwd{social network analysis}
\kwd{network data}
\kwd{Markov random fields}
\kwd{graphical models}
\end{keyword}

%\spacingset{1.8} % DON'T change the spacing!
%\spacingset{1.25}
\end{frontmatter}


\section{Introduction} 

Multilayer networks have become a recent focal point of research in the field of statistical network analysis  
\citep[e.g.,][]{LeChLy20, CaGo20, arroyo_multi, KrKoMa20, chen2022, sosa2022, huang2022},
arising in applications where a common set of elements of a population of interest 
have multiple modes of interaction with or relation to other elements in the population.
A prototypical example in the literature might be the Lazega law firm network \citep{Lazega2001}, 
in which attorneys within a law firm have multiple modes of linkage,
which include advice seeking, friendship, collaboration, etc.,
each of which would form individual layers of the multilayer network  \citep{KrKoMa20}.  
A multilayer network is therefore a composite of multiple individual networks,
each defined by a distinct mode of interaction or relation.  


Often, 
edges in one layer may depend on edges in another layer, 
giving rise to what we call cross-layer dependence. 
Understanding drivers of edge formation in multilayer networks requires 
learning  dependence structures of the layers of multilayer networks. 
A key challenge lies in the fact that the cross-layer dependence can be varied and complex. 
In this work, 
we present a novel modeling framework for multilayer networks which provides a flexible platform for extending 
single-layer network models to multilayer networks,
with the primary goal of learning cross-layer dependence structures of multilayer networks. 
A key advantage of our framework is that we are able to account for and separate out the network formation process  
from the layer formation process,
enabling us to create a wide-range of novel classes of multilayer network models 
by extending popular classes of network models 
(e.g., exponential-family random graph models, stochastic block models, latent space models),
and employing Markov random field specifications to develop flexible and comprehensive 
models of cross-layer dependence in multilayer networks. 
As a result, 
we are able to jointly model both network structure and cross-layer dependence through what we refer to as a network separable 
framework for modeling multilayer networks. 


Our main contributions in this work include: 
\ben
\item Introducing a novel framework for modeling 
cross-layer dependence in multilayer networks 
that synchronizes with current network models in the literature.
\item Deriving non-asymptotic theoretical guarantees in scenarios where the number of parameters tends to infinity, 
which establishes bounds on the:  
\ben
\item Statistical error of both maximum likelihood and pseudolikelihood estimators.  
\item Error of the multivariate normal approximation of estimators.
\een
\item Elaborate a model selection algorithm which controls the false discovery rate. 
\een 




The rest of the paper is organized as follows. 
Section \ref{sec2} introduces our modeling framework and includes illustrative examples.
The consistency and the minimax optimal results are contained in Section \ref{sec3}.
The multivariate normal approximation theory is presented in Section \ref{sec:normal}. 
The results of simulation studies are provided in Section \ref{sec:sim}, 
together with different testing procedures for model selection
which control the false discovery rate.
An application of our developed framework and methodology is given in Section \ref{sec:app},
concluding with a discussion presented in Section \ref{sec:disc}.
%The R package we developed for the simulation analysis in Section \ref{sec:sim} and the application analysis of Lazega's corporate law partnership data in Section \ref{sec:app} is available on GitHub: 
The code and data to reproduce the simulations and analyses can be found in our package online.\footnote{
\url{https://github.com/jiaheng-li/mlyrnetwork}}

\section{Modeling cross-layer dependence in multilayer networks} 
\label{sec2}

A multilayer network can be represented as 
a sequence of $1 \leq K < \infty$ random graphs $\bX^{(1)}, \ldots, \bX^{(K)}$
each defined on a common set of $N \geq 3$ nodes, 
which we take without loss to be the set $\mN = \{1,\ldots,N\}$. 
We call the graphs $\bX^{(1)}, \ldots, \bX^{(K)}$ the {\it layers} of the network,
and represent the multilayer network as the quantity $\bX = (\bX^{(1)}, \ldots, \bX^{(K)})$. 

Connections between pairs of nodes $\{i,j\} \subset \mN$ in each layer $k \in \{1, \ldots, K\}$
are modeled by random variables  
\beno
X_{i,j}^{(k)}
\= \begin{cases}
1 & \mbox{nodes } i \mbox{ and } j \mbox{ are connected in layer } k \\
0 & \mbox{otherwise}
\end{cases}. 
\ee
We refer to all connections of a pair of nodes $\{i,j\} \subset \mN$ across the $K$ layers 
as a {\it  dyad} which we denote by  
$\bX_{i,j} = (X_{i,j}^{(1)}, \ldots, X_{i,j}^{(K)})\in \{0,1\}^{K}$. 
A multilayer network can be represented by a collection of dyads as $\bX = (\bX_{i,j})_{\{i,j\} \subset \mN}$ alternatively.

For notational ease, 
we will consider undirected multilayer networks, 
which imply that the network layers $\bX^{(1)}, \ldots, \bX^{(K)}$ are undirected random graphs; 
extensions to directed multilayer networks or mixed multilayer networks with both directed and undirected 
layers will typically be straightforward, 
involving only notational adaptations in subscripts in most cases. 
We adopt the usual conventions for undirected networks, 
i.e., 
we assume that $X_{i,j}^{(k)} = X_{j,i}^{(k)}$ (all $\{i,j\} \subset \mN$, $1 \leq k \leq K$) 
and $X_{i,i}^{(k)} = 0$ (all $i \in \mN$, $1 \leq k \leq K$). 
The sample space of each layer $\bX^{(k)}$ is therefore 
the product space $\mbX^{(k)} \coloneqq \{0, 1\}^{\binom{N}{2}}$ ($k = 1, \ldots, K$), 
and the sample space $\mbX$ of $\bX$ is the product space of the sample spaces of the individual layers, 
i.e.,
$\mbX \coloneqq \mbX^{(1)} \times \cdots \times \mbX^{(K)}$. 
The sample space of dyad $\{i,j\} \subset \mN$ is the product space $\mbX_{i,j} \coloneqq \{0,1\}^{K}$. 

%A natural starting point for constructing models of cross-layer dependence is to use Markov random field specifications. 
A challenge in the statistical modeling of network data lies in the fact that 
networks have many distinguishing properties, 
including:  
\ben
\item {\bf Sparsity.} Many real-world networks are sparse, 
in the sense that the expected number of edges in the network grows at a rate slower than $\binom{N}{2}$. 
The phenomena of network sparsity manifests in a variety of different applications, 
usually due to constraints, 
such as time or financial constraints, 
which can limit the number of connections any node can maintain at a given point in time
\citep[][]{KrHaMo11,butts:jms:2018}.
\item {\bf Node heterogeneity.} Different actors in a social network will have different properties,
called node covariates,
which can lead to different propensities to form edges. 
A key example is assortative and disassortative mixing patterns in networks 
\citep{McSmCo01,KrHaRaHo07},
as well as differences in structural patterns in the network \citep{Albert02, LiXu12}.  
\item {\bf Edge dependence.} In addition to node-based effects that give rise to 
heterogeneity 
in propensities for nodes to form edges, 
scientific and statistical evidence suggests edges are dependent in many applications 
\citep{HpLs72,Fo80,block2015reciprocity},
and modeling a single system of multiple binary random variables  without replication is a challenging statistical problem
inherent to many statistical network analysis applications. 
\een
Each of the above gives rise to distinct challenges for modeling network data and performing statistical inference
in statistical network analysis applications,  
and it is not straightforward to construct
models that due justice to each of these and more.  
To address these challenges, 
a plethora of statistical models have been proposed to model network data, 
which for single-layer networks have included  
exponential-families of random graph models 
\citep[e.g.,][]{ergm.book,ScKrBu17},
stochastic block models
\citep[e.g.,][]{HoLaLe83},
latent metric space models 
\citep[e.g.,][]{HpRaHm01},
random dot product graphs \citep[e.g.,][]{Athreya2018}, 
exchangeable random graph models \citep[e.g.,][]{CaFo17,CrDe16}, 
and more. 
In this work, 
we build upon the many classes of single-layer network data models by introducing a separable multilayer network modeling framework. This framework enables existing single-layer network models to be extended to the multilayer setting and simultaneously enables learning cross-layer dependence and interactions across different layers in the multilayer network.

\subsection{Separable multilayer network models}  
\label{sec:2.1}
Multilayer networks are subject to the same forces and phenomena as single layer networks, as multiple modes of relation or interaction do not remove constraints or properties of nodes which are fundamental to network data applications. 
The same set of nodes is defined across all layers in a multilayer network, and because all layers share the same set of nodes, the dyadic connections among these nodes fundamentally define the network formation process. By specifying a single-layer network as the foundational structure reference, we can separate the network formation process from the layer formation process. In doing so, the single-layer network serves as the baseline for establishing dyadic relationships that represent the relational structure across all layers of the multilayer network. 
As a result, the network formation process determines which dyads have the potential to form connections, i.e., which pair of nodes may exhibit at least one edge in any of the layers. 
In contrast,
The layer formation process dictates the particular layers in which these  connections appear. 
To learn the effects of cross-layer dependence in multilayer networks, we propose the class of separable multilayer network models, 
which extend the broad literature on single-layer network models into the multilayer realm.
These models can incorporate an arbitrary single-layer network structure as the foundational baseline and ensures that the underlying single-layer network can be recovered from observations of the multilayer network. We illustrate this approach and its advantages through our proposed modeling framework.

We introduce a network separable model for multilayer networks by specifying probability distributions 
on a double of networks $(\bX, \bY)$, 
where $\bY$ will represent the network formation process,
which we will call the {\it basis network},  
and $\bX$ will represent the realized multilayer network. 
We assume that $\bY \in \mbY \coloneqq \{0, 1\}^{\binom{N}{2}}$ 
is an undirected, single-layer network defined on the set of nodes $\mN$ 
where 
\beno
Y_{i,j}
\= \begin{cases}
1 & \mbox{nodes } i \mbox{ and } j \mbox{ are connected in the basis network} \s \\ 
0 & \mbox{otherwise} 
\end{cases},
\ee 
for each $\{i,j\} \subset \mN$, 
making the usual conventions for undirected networks mentioned previously.
We consider semi-parametric families of probability distributions 
$\{\sepmodel : \nat \in \mbR^p\}$ for $(\bX, \bY)$ 
which are absolutely continuous with respect to a $\sigma$-finite measure $\nu$ defined on $\mP(\mbX \times \mbY)$,
where $\mP(\mbX \times \mbY)$ is the power set of $\mbX \times \mbY$. 
Typically,
$\nu$ will be the counting measure, 
however sparsity inducing reference measures are also admissible and have found application in network data applications
in order to model sparse networks \citep{butts:jms:2018,StSc21}. 
We say the family $\sepfam \coloneqq \{\sepmodel : \nat \in \mbR^p\}$ is {\it network separable} if 
each $\sepmodel \in \sepfam$ admits the form: 
\be
\label{general model}
\sepmodel(\{(\bx, \by)\}) 
\= f(\bx,\nat) \; g(\by) \; h(\bx, \by) \; \psi(\nat, \by),
&&  (\bx, \by) \in \mbX \times \mbY, 
\ee
where 
\bi
\item $h : \mbX \times \mbY \mapsto \{0, 1\}$ is given by 
\beno
h(\bx, \by) 
\= \dprod_{\{i,j\} \subset \mN} \, 
\one(\norm{\bx_{i,j}}_1 > 0)^{y_{i,j}} \; \one(\norm{\bx_{i,j}}_1 = 0)^{1 - y_{i,j}}, 
\ee 
where $\bx_{i,j} = (x_{i,j}^{(1)}, \ldots, x_{i,j}^{(K)}) \in \mbX_{i,j}$ ($\{i,j\} \subset \mN$). \s   
\item $f : \mbX \times \mbR^p \mapsto (0, 1)$ is given by 
\beno
f(\bx, \nat)
&= \, \dprod_{\{i,j\} \subset \mN} \,
\exp&\left(\dsum_{k=1}^K\,\theta_{k}\,x_{i,j}^{(k)} +  \dsum_{\substack{k < l}}^{K}\, \theta_{k,l} \, x_{i,j}^{(k)} \, x_{i,j}^{(l)} + \ldots \right.\\[20pt]
&& \left.+ \dsum_{k_1 < \,\ldots\,< k_H}^{K}\, \theta_{k_1,k_2,\ldots,k_H} \, x_{i,j}^{(k_1)} \cdots\, x_{i,j}^{(k_H)} \right),
\ee
where $H \le K$ is the highest order of cross-layer interactions included in the model.
We write $\theta_{k_1,k_2,\ldots,k_h}$ to reference the $h$-order interaction parameter 
for the interaction term among layers $\{k_1, \ldots, k_h\} \subseteq \{1, \ldots, K\}$. \s  
\item $\psi : \bTheta \times \mbY \mapsto (0, \infty)$ is defined by 
\beno 
\psi(\nat, \by) 
\= \left[ \, \dsum_{\bx \in \mbX} \, f(\bx, \nat) \, h(\bx, \by) \right]^{-1}, 
\ee
and functions to ensure summation to one so that the specification in \eqref{general model}
will be a valid probability mass function for $(\bX, \bY)$. 
\item $g : \mbY \mapsto (0, 1)$ is the marginal probability mass function of $\bY$ 
and is assumed to be strictly positive on $\mbY$. 
\ei

The notation $\sepmodel(\{(\bx, \by)\})$ is well-defined for each $(\bx, \by) \in \mbX \times \mbY$, 
as $\sepmodel$ is a probability measure defined on $\mP(\mbX \times \mbY)$. 
In an abuse of notation, 
we will frequently write probability expressions 
$\sepmodel(\bX = \bx, \bY = \by)$ for the joint probability of $\{(\bx, \by)\}$, 
and $\sepmodel(\bX = \bx \,|\, \bY = \by)$ for the conditional probability of the event $\bX = \bx$ 
conditional on the event $\bY = \by$. 
We denote the data-generating parameter vector by $\truth \in \mbR^p$,
and the corresponding probability measure and expectation operator  
by $\mbP \equiv \mbP_{\truth}$ and $\mbE \equiv \mbE_{\truth}$, respectively.

The terminology {\it network separable} is motivated by the fact that the specification in \eqref{general model} 
separates the network formation process $\bY$,
specified by $g(\by)$, 
from the layer formation process, 
specified by $f(\bx, \nat)$. 
The two are joined by the function $h(\bx, \by)$,
which ensures $\norm{\bx_{i,j}}_1 = 0$ 
whenever $Y_{i,j} = 0$
and $\norm{\bx_{i,j}}_1 > 0$ whenever $Y_{i,j} = 1$,
and by $\psi(\nat, \by)$ which ensures the resulting product of functions will be a valid probability mass function. 
The latter has less of a direct role in modeling the cross-layer dependence and interaction between $\bX$ and $\bY$,
essentially fulfilling the role of a normalizing constant for the conditional probability distribution of $\bX$ given $\bY$,
as derived in Proposition \ref{prop:inference}. 
We call dyads $\{i, j\} \subset \mN$ 
with $Y_{i,j} = 1$ {\it activated dyads},
as we allow edges between nodes $i \in \mN$ and $j \in \mN$ in $\bX$ if and only if $\{i,j\}$ 
is an activated dyad.
Such specifications have the advantage of being able to specify the network formation process separately 
from the process that populates the layers of activated dyads,
thus modeling the cross-layer dependence conditional on the network $\bY$. 
A pair $(\bx, \by) \in \mbX \times \mbY$ that satisfies $h(\bx, \by) =1$ is said to be {\it network concordant}.

To illustrate the flexibility and generality of \eqref{general model},
observe that $g(\by)$ is allowed to  be any probability mass function for a single layer network $\bY$ 
(e.g., exponential-family random graph model, stochastic block model, latent space model),
provided $g(\by) > 0$ for all $\by \in \mbY$. 
We therefore view our framework as semi-parametric as $g(\by)$ need not assume a specific parametric form.  
Moreover, 
our framework can be viewed as non-parametric 
within the family of network separable multilayer networks 
when the maximal possible order interaction terms are included in \eqref{general model},
a point on which we further elaborate later. 
An important feature of our framework lies in the fact that 
the choice of the probability distribution for the network formation process does not directly 
influence inference for the cross-layer dependence structure,
i.e.,
the choice of $g(\by)$ does not directly influence inference for $\truth$. 
Proposition \ref{prop:inference} demonstrates this point in the case of likelihood-based inference.  

\begin{proposition}
\label{prop:inference}
Let $\{\sepmodel : \nat \in \mbR^p\}$ satisfy \eqref{general model}. 
Then the following hold:   
\ben
\item 
For each $\bx \in \mbX$, 
$\bY = \by$ ($\sepmodel$-a.s.) for one and only one $\by \in \mbY$. \s  
\item $\bY$ is predictable via $\bX$, 
i.e.,
for each $\bx \in \mbX$,  
$\mbP_{\nat}(\bY = \by \,|\, \bX = \bx) = 1$ where %$\by$ is given by 
\beno
y_{i,j} \= \one(\norm{\bx_{i,j}}_1 \,>\, 0), 
&& \{i,j\} \subset \mN. 
\ee  
\item For all $(\bx, \by) \in \mbX \times \mbY$ with $h(\bx, \by) = 1$, 
\beno
\log \, \sepmodel(\bX = \bx, \bY = \by)
\;=\; \log \, \sepmodel(\bX = \bx \mid \bY = \by) + \log \, g(\by),
\ee
where $\sepmodel(\bX = \bx \,|\, \bY = \by)$ belongs to a minimal exponential family 
with natural parameter vector $\nat \in \mbR^p$ and is given by 
\beno
\sepmodel(\bX = \bx \mid \bY = \by)
\;=\; \exp(\log f(\bx, \nat) + \log \psi(\nat, \by)).  
\ee
\een 
\end{proposition}


%We prove Proposition \ref{prop:inference} in Appendix \ref{sec:prop_proof}. 
Proposition \ref{prop:inference} establishes a few key facts for the inference of cross-layer dependence structures 
in multilayer networks. 
First, 
we are able to observe $\bY$ through $\bX$,
as given any observation $\bx \in \mbX$ of the multilayer network $\bX$,
$\sepmodel(\bY = \by \,|\, \bX = \bx) = 1$ for one, and only one, $\by \in \mbY$. 
In other words, 
through the observation of $\bx$,
we can infer with probability $1$ the corresponding $\by$
due to the form of \eqref{general_model}.  
The significance of this result is that we do not need to treat the basis network $\bY$ as a latent network, 
which would require additional statistical and computational methodology to handle the latent missing network data.  
Second, 
we see that the inference for $\truth$ is unaffected by the choice of $g(\by)$; 
although, the statistical guarantees for estimators of $\truth$ will be indirectly influenced by the choice of $g(\by)$,
a point which we discuss in later sections.
Moreover, 
the above choice for $f(\bx, \nat)$
and the functional form of $\sepmodel(\bX = \bx \,|\, \bY = \by)$ derived in Proposition \ref{prop:inference}
establishes that $\log \, \sepmodel(\bX = \bx \,|\, \bY = \by)$
corresponds to the log-likelihood of a minimal exponential family, 
accessing a broad literature of statistical methodology and theory \citep[e.g.,][]{Su19}.
We note that other specifications for $f(\bx, \nat)$ are possible, 
but that Markov random field specifications provide a powerful class of models for dependent data 
\citep[e.g.,][]{WaJo08},
and in the case of the saturated model with maximal interaction term $H = K$, it
completely specifies all possible probabilities of outcomes $\bx_{i,j} \in \{0, 1\}^K$,
presenting a non-parametric model class for multilayer networks. 





\subsection{Example of a multilayer network with pairwise interactions} 
We illustrate cross-layer dependence among layers in our modeling framework
by considering a separable multilayer network model using the Markov random field specification 
for $f(\bx, \nat)$ given in the previous section and maximal interaction term $H = 2$:
%i.e.,
%we consider a Markov random field specification which includes all single-layer effects and all pairwise interaction effects
%between layers.  
%We can write this model down as 
\be
\label{pairwise}
f(\bx, \nat)
\= \dprod_{\{i,j\} \subset \mN} \, 
\exp\left( \, \dsum_{k=1}^{K}\,\theta_{k} \,  x_{i,j}^{(k)} +  
\dsum_{k<l}^{K} \, \theta_{k,l} \,  x_{i,j}^{(k)} \, x_{i,j}^{(l)} \right).
\ee
The dimension of the parameter vector $\nat$ is $\dim(\nat) = K + \binom{K}{2}$,
with $K$ parameters governing the single-layer effects for the $K$ layers and $\binom{K}{2}$ combinations 
of layers to form the pairwise interactions for the cross-layer dependence effects.  

Define the ($K$-$1$)-dimensional vector 
$X_{i,j}^{(-k)} \coloneqq (X_{i,j}^{(l)} : l \in \{1, \ldots, K\} \setminus \{k\})$ 
to be the 
vector of edge variables in $\bX_{i,j}$ which excludes the edge variable $X_{i,j}^{(k)}$,
i.e.,
excluding the edge variable between nodes $i$ and $j$ in layer $k$.  
The conditional log-odds of edge $X_{i,j}^{(k)}$ takes the form:    
\beno
\scalebox{0.95}{$\log \, \dfrac{\mbP(X_{i,j}^{(k)} = 1 \,|\, \bX_{i,j}^{(-k)} = \bx_{i,j}^{(-k)}, Y_{i,j} = 1)}
{\mbP(X_{i,j}^{(k)} = 0 \,|\, \bX_{i,j}^{(-k)} = \bx_{i,j}^{(-k)}, Y_{i,j} = 1)}$} 
= \begin{cases}
\theta_{k} +  \dsum_{l \neq k}^{K}\, \theta_{k,l} \, x_{i,j}^{(l)}, & \norm{\bx_{i,j}^{(-k)}}_1 > 0 \\ 
+\infty, & \norm{\bx_{i,j}^{(-k)}}_1 = 0  
\end{cases}.
\ee
A primary advantage and motivation
of using a parametric Markov random field specification for $f(\bx, \nat)$ lies in the interpretability of the model. 
An effective approach to analyzing and understanding marginal network effects in such specifications is to study 
conditional log-odds of edges under different conditioning statements \citep[e.g.,][]{StScBoMo19}.
By the form of $h(\bx, \by)$,
when $Y_{i,j} = 1$,
we require $\norm{\bx_{i,j}}_1 > 0$,
meaning nodes $i$ and $j$ must have at least one connection in $\bX$. 
This is seen through the log-odds formula above,
where the log-odds of edge $X_{i,j}^{(k)}$ is equal to $+\infty$ when $\norm{\bx_{i,j}^{(-k)}}_1 = 0$. 
In contrast, 
when $\norm{\bx_{i,j}^{(-k)}}_1 > 0$,
the constraint $\norm{\bx_{i,j}}_1 > 0$ is already satisfied,
and the log-odds of edge $X_{i,j}^{(k)}$ depends on the layer specific parameter $\theta_k$, 
as well as the pairwise interaction effects where edges present in other layers $l \in \{1, \ldots, K\} \setminus \{k\}$ 
can influence the likelihood of the edge $X_{i,j}^{(k)}$ depending on the signs and magnitudes of the pairwise interaction 
parameters $\theta_{k,l}$ ($\{k,l\} \subseteq \{1, \ldots K\})$. 



\section{Estimation of cross-layer dependence structure} 
\label{sec3}

Maximum likelihood estimation for network data with dependent edges faces significant computational challenges,
as the normalizing constants for such models are often computationally intractable,
which makes direct maximization of likelihood functions infeasible in general cases.
For network separable multilayer networks satisfying  \eqref{general model}, 
Proposition \ref{prop:inference} establishes that the log-likelihood function takes the form 
\be
\label{loglikelihood}
\ell(\nat; \bx, \by) 
&\coloneqq& \log \,\sepmodel(\bX = \bx, \bY = \by) 
\= \log \, \mbP_{\nat}(\bX = \bx \,|\, \bY = \by) + \log \, g(\by).
\ee
Given an observation $\bx \in \mbX$ of the multilayer network $\bX$,  
and therefore an observation $\by \in \mbY$ of $\bY$ by Proposition \ref{prop:inference},
we denote the set of maximum likelihood estimators by 
\beno
\Mle &\coloneqq& \left\{ \nat \in \mbR^p 
\,:\, \ell(\nat; \bx, \by) = \sup\limits_{\nat^{\prime} \in \mbR^p} \, \ell(\nat^{\prime}; \bx, \by) \right\},
\ee 
and reference individual elements of the set by $\mle \in \Mle$.
As Proposition \ref{prop:inference} establishes 
$\log \, \mbP_{\nat}(\bX = \bx \,|\, \bY = \by)$ to be a minimal, 
and by construction regular, 
exponential family, 
$|\Mle| \in \{0, 1\}$, 
i.e., 
when the maximum likelihood estimator exists, 
the set $\Mle$ will contain a unique element when non-empty
\citep[Proposition 3.11, pp. 32--33,][]{Su19}.


Two predominant methods of approximating $\truth$ when the likelihood function is computationally intractable 
have emerged in the literature. 
Monte Carlo maximum likelihood estimation (MCMLE) \citep{GeTh92},
which constructs a simulation-based approximation to the likelihood function 
in order to approximate the maximum likelihood estimator,
is an established method for approximating maximum likelihood estimators 
in the statistical network analysis literature \citep{HuHa06}.
While able to provide accurate estimates of maximum likelihood estimators for complex models 
\citep[e.g.,][]{StScBoMo19,ScKrBu17}, 
a drawback of MCMLE,
and other simulation-based estimation methodology, 
is the computational burden which can scale with both the complexity of the model and the size of the network \citep{BaBrSl11}.  
In settings where the computation of the MCMLE is impractical, 
a computationally efficient alternative is provided via the maximum pseudolikelihood estimator (MPLE) \citep{Bj74},
whose application to social network analysis and to statistical network analysis dates back to \citet{StIk90}.  
As Proposition \ref{prop:inference} establishes that $\bY$ is observable through $\bX$,
\beno
\mbP(Y_{i,j} = y_{i,j} \,|\, \bX = \bx, \bY_{-\{i,j\}} = \by_{-\{i,j\}})
\= 1,
\ee
when $y_{i,j} = \one(\norm{\bx_{i,j}}_1 \,>\, 0)$ and $\bY_{-\{i,j\}}$ is defined to be the 
($\binom{N}{2}$-$1$)-dimensional vector of edge variables in $\bY$ which excludes $Y_{i,j}$. 
As a result, 
if $(\bx, \by)$ is network concordant,
then  
\beno
\log \, \mbP(Y_{i,j} = y_{i,j} \,|\, \bX = \bx, \bY_{-\{i,j\}} = \by_{-\{i,j\}}) \= 0, 
&& \mbox{for all } \{i,j\} \subset \mN.
\ee 
The log-pseudolikelihood of \eqref{general model} can then be written down as 
\be
\label{eq:log-pseudo}
\pl(\nat; \bx, \by) 
\,\coloneqq \dsum_{\{i,j\} \subset \mN}  \dsum_{k=1}^K \, \log  
\sepmodel(X_{i,j}^{(k)} = x_{i,j}^{(k)} \, |\, \bX_{i,j}^{(-k)} = \bx_{i,j}^{(-k)}, \bY = \by), 
\ee
provided $(\bx, \by)$ is network concordant
and by exploiting the conditional independence properties implied by \eqref{general model}. 
We denote the set of maximum pseudolikelihood estimators of the data-generating parameter vector $\truth$ by 
\beno
\Mple &\coloneqq& \left\{ \nat \in \mbR^p \,: \, \pl(\nat; \bx, \by) = \sup\limits_{\nat^{\prime} \in \mbR^p} \, \pl(\nat^{\prime}; \bx, \by) \right\}.
\ee
Individual elements are referenced by $\mple \in \Mple$. 
Uniqueness of maximum pseudolikelihood estimators for exponential families is more complicated than 
for maximum likelihood estimators. 
However, 
our theoretical results establish that all elements $\mple \in \Mple$ will all be within the same Euclidean distance to $\truth$.
The assumption that $(\bx, \by)$ is network concordant comes at no cost 
since $\bY$ is predictable through $\bX$,
as discussed above,
meaning that given an observation $\bx$ of $\bX$,
we can find the unique network concordant pair $(\bx, \by)$ with probability one.  
The advantage of \eqref{eq:log-pseudo} is that 
the conditional probabilities 
$\sepmodel(X_{i,j}^{(k)} = x_{i,j}^{(k)} \, |\, \bX_{i,j}^{(-k)} = \bx_{i,j}^{(-k)}, \bY = \by)$ 
of edges in the multilayer network 
are often computationally tractable 
since the conditional distribution is a Bernoulli distribution when $Y_{i,j} = 1$,
and is a degenerate point mass at $0$ when $Y_{i,j} = 0$.

\hide{
Pseudolikelihood-based estimators have the following computational advantages:
\ben
\item Algorithms are generally deterministic and do not require simulation-based approximation schemes,
which aids in reproducibility of results; 
\item Algorithms are generally more scalable, 
relative to alternatives such as MCMLE and other simulation-based approximations, 
and are able to be parallelized to take advantage of larger  multicore computing infrastructures which are becoming increasingly common.  
\een
}

In this work, 
we consider both maximum likelihood estimators and maximum pseudolikelihood estimators. 
As seen from the forms of $\ell(\nat; \bx, \by)$ and $\pl(\nat; \bx, \by)$ given above, 
the gradients and Hessians of the log-likelihood and log-pseudolikelihood equations 
do not directly depend on $g(\by)$, 
echoed by the results in Proposition \ref{prop:inference}.
However, 
as mentioned in the previous section, 
theoretical guarantees for estimators of $\truth$ will be indirectly influenced by the choice of $g(\by)$,
a point supported by the following lemma. 




\begin{lemma}
\label{lem:min-eig} 
Consider a family $\{\sepmodel : \nat \in \mbR^p\}$ of separable multilayer network models satisfying \eqref{general_model} 
and an observation $\bx \in \mbX$ of $\bX$.
Let $(\bx, \by)$ be the concordant pair where $\by$ is given by Proposition \ref{prop:inference}. 
Define,
for each pair of nodes $\{i,j\} \subset \mN$,
\beno
L_{i,j}(\nat, \bx_{i,j}, \by)
&\coloneqq& \log \, \sepmodel(\bX_{i,j} = \bx_{i,j} \,|\, \bY = \by).
\ee
Then there exists a $p \times p$ matrix 
$\mcI(\nat)$ such that 
%$\mcI_{i,j}(\nat)$ and  $\widetilde\mcI_{i,j}(\nat)$
\beno
\mbE\left[ - \nabla_{\nat}^2 L_{i,j}(\nat, \bX_{i,j}, \bY) \,|\, \bY = \by \right] 
\=
\begin{cases}
    \mcI(\nat) & Y_{i,j} = 1 \\
    \bm{0}_{p,p} & Y_{i,j} = 0,
\end{cases}
\ee
for all $\{i,j\} \subset \mN$,  
where $\bm{0}_{p,p}$ is the $p\times p$ matrix with all $0$ entries, 
and 
\beno 
&& \lambda_{\min}(-\mbE \, \nabla_{\nat}^2 \, \ell(\nat; \bX, \bY)) 
\= \lambda_{\min}(\mcI(\nat)) \, \mbE \, \norm{\bY}_1 \s \\
&& \lambda_{\max}(-\mbE \, \nabla_{\nat}^2 \, \ell(\nat; \bX, \bY)) 
\= \lambda_{\max}(\mcI(\nat)) \, \mbE \, \norm{\bY}_1,  
\ee
where $\lambda_{\min}(\bA)$ and $\lambda_{\max}(\bA)$ are the smallest and the largest eigenvalue of matrix $\bA \in \mbR^{p \times p}$, respectively.
\end{lemma}



\s 

In classical settings with independent and identically distributed observations, 
the expected negative Hessian of the log-likelihood function 
is the Fisher information matrix and 
is expected to scale with the number of observations.  
In such cases, 
standard matrix theory indicates that the smallest eigenvalue of this expected negative Hessian matrix 
will scale with the sample size,
provided the smallest eigenvalue of the Fisher information matrix is bounded from below.
Lemma \ref{lem:min-eig} extends this notion by establishing similar scaling behavior concerning the expected number of activated dyads $\mbE \, \norm{\bY}_1$,
proxying as an effective sample size. 
Analogously,  $\mcI(\nat)$ can be seen as the Fisher information of the population distribution governing individual activated dyads in $\bY$, mirroring the role of Fisher information for population distributions in classical independent and identically distributed scenarios.  

Before we present our theoretical guarantees for maximum likelihood estimators in Theorem \ref{thm1}, we define some notations and outline some regularity assumptions for our theorem to follow.  
As we will show in Theorem \ref{thm1}, the choice of $g(\by)$ influences the estimation error through the expected 
number of edges in $\bY$ and through the covariances of edge variables in $\bY$. 
Define
\beno
D_{g}
&\coloneqq& \dsum_{\{i,j\} \prec \{v,w\} \subset \mN} \, \cov(Y_{i,j}, \, Y_{v,w}),
\ee
where $\{i,j\} \prec \{v,w\}$ implies the sum is taken with respect to the lexicographical ordering of pairs of nodes. 
Define $[D_{g}]^{+} \coloneqq \max\{0, \, D_{g}\}$ to be the positive part of $D_{g}$.
Let $\epsilon > 0$ be fixed independent of $N$ and $p$, and denote the $\epsilon$-ball of the data-generating parameter $\truth$ by $\mB_2(\truth, \epsilon) = \{\nat \in \mbR^p : \norm{\truth - \nat}_2 \leq \epsilon\}$. Define 
\beno
\widetilde{\lambda}_{\min}^{\epsilon}
\;\coloneqq\; \inf\limits_{\nat \in \mB_2(\truth, \epsilon)} \, \lambda_{\min}(\mcI(\nat)) 
&&\mbox{and}&&
\widetilde{\lambda}_{\max}^{\star}
\;\coloneqq\; \lambda_{\max}(\mcI(\truth)),  
\ee
where $\lambda_{\min}(\bA)$ and $\lambda_{\max}(\bA)$ are the smallest and the largest eigenvalue of matrix $\bA \in \mbR^{p \times p}$, respectively. 
\begin{assumption}
\label{assump1}
 Assume there exists a $C_0 > 0$ such that $\mbE\,\norm{\bY}_1 \ge 1$ and 
\beno
\dfrac{[D_{g}]^{+}}{\mbE\, \norm{\bY}_1} &\leq& C_0,
\ee
for all network sizes $N$. 
\end{assumption}

\begin{assumption}
\label{assump2}
Assume the parameter dimension $p$ satisfies 
\beno
p 
&\leq& \sqrt{\widetilde{\lambda}_{\max}^{\star} \, \mbE\, \norm{\bY}_1},
\ee
for all network sizes $N$. 
\end{assumption}

\begin{assumption}
\label{assump3}
Assume that 
$\widetilde{\lambda}_{\max}^{\star}$ and $\widetilde{\lambda}_{\min}^{\epsilon}$ satisfy,
as a function of the network size $N$,  
 \beno
    \dfrac{\sqrt{\widetilde{\lambda}_{\max}^{\star}} }{\widetilde{\lambda}_{\min}^{\epsilon}} \= o \;\left(\sqrt{\dfrac{\mbE \norm{\bY}_1}{p}} \right).
 \ee
\end{assumption}

Assumptions \ref{assump1}–\ref{assump3} provide a foundation for Theorem \ref{thm1} to establish the consistency result of the maximum likelihood estimator in large network settings. Assumption \ref{assump1} imposes a lower bound on the expected number of activated dyads relative to the covariance as the network size \(N\) grows. Assumption \ref{assump2} restricts the growth rate of \(p\) in relation to the network size and the largest eigenvalue of the Fisher information $\mcI(\truth)$. Finally, Assumption \ref{assump3} sets a constraint on the ratio between $\sqrt{\widetilde{\lambda}_{\max}^{\star}}$ and $\widetilde{\lambda}_{\min}^{\epsilon}$, balancing eigenvalue magnitudes in a way that preserves estimator consistency under increasing network size. 




\begin{theorem} 
\label{thm1}
Consider a multilayer network model satisfying \eqref{general model} 
defined on a set of $N \geq 3$ nodes and $K \geq 1$ layers 
and assume that $\mbE \, \norm{\bY}_1 \geq 1$ and $p \leq N$.  
Then there exists $N_0 \geq 3$ such that, 
for all $N \geq N_0$,
the following hold with probability at least $1 - 3 \, (\mbE \, \norm{\bY}_1)^{-1}$: 
\ben 
\item (MLE) The set $\Mle$ is non-empty and the unique element $\mle \in \Mle$ satisfies  
\beno
\norm{\mle - \truth}_2 &\leq&
\sqrt{\dfrac{3 \, p \, \log N}{\mbE \norm{\bY}_1}} \; \dfrac{\sqrt{1 + [D_{g}]^{+}}}{\xi_{\epsilon^\star}}, 
\ee
provided the right-hand side tends to $0$ as $N \to \infty$. 
\item (MPLE) The set $\Mple$ is non-empty and each $\mple \in \Mple$ satisfies  
\beno 
\norm{\mple - \truth}_2 &\leq&
\sqrt{\dfrac{3 \, p \, K^2 \, \log N}{\mbE \norm{\bY}_1}} \; \dfrac{\sqrt{1 + [D_{g}]^{+}}}{\widetilde\xi_{\epsilon^\star}},
\ee
provided the right-hand side tends to $0$ as $N \to \infty$. 
\een
\end{theorem}

The results of Theorem \ref{thm1} establish a few key facts concerning statistical estimation 
of the parameter vector $\truth$. 
First, 
we can view the quantity $\xi_{\epsilon^\star} \, \sqrt{\mbE \, \norm{\bY}_1} \,/\, \sqrt{1 + [D_{g}]^{+}}$ 
as the effective sample size in order to compare our results to 
classical settings with independent and identically distributed data. 
The effective sample size, 
together with the dimension of the model $p$, 
helps to determine the rate of convergence (with respect to the Euclidean distance) 
of maximum likelihood and pseudolikelihood estimators. 
As previously mentioned, 
the quantities $\mbE \norm{\bY}_1$ and $[D_{g}]^{+}$ 
are determined by properties of $g(\by)$,
the marginal probability mass function of $\bY$. 
While specification of $g(\by)$ does not directly influence estimation algorithms, 
the statistical guarantees of estimators will depend on $g(\by)$ producing enough activated dyads and not possessing overly strong 
dependence among edges in the single network $\bY$.
The requirement that the right-hand side of the bounds in Theorem \ref{thm1} tend to $0$ as $N \to \infty$ 
ensures that all regularity assumptions remain valid. 
Namely, 
key to our approach lies in the ability to control minimum eigenvalues of matrices 
$\mcI(\nat)$ and $\widetilde\mcI(\nat)$ 
in a neighborhood of the data-generating parameter vector $\truth$. 
The condition that the bounds tend to $0$ ensures that it is sufficient to control the smallest eigenvalue
in a bounded set,
i.e.,
we may let $\epsilon^\star$ be fixed independent of $N$, 
and moreover, 
to ensure consistency in the sense that $\norm{\mle - \truth}_2 \to 0$ 
and $\norm{\mple - \truth}_2 \to 0$ (as $N \to \infty$) with probability approaching $1$.  

 
 




\section{Error of the normal approximation and model selection} 
\label{sec:normal}
In this section, 
we establish the asymptotic multivariate normality of the maximum likelihood estimator (MLE) for the data-generating parameter vector $\truth$ as its dimension grows.
Specifically,
we derive a non-asymptotic bound on the quality of the multivariate normal approximation and exhibit scaling conditions on both the model dimension $p$ 
and the expected number of activated dyads $\mbE \, \norm{\bY}_1$---under which the approximation error vanishes as the network size tends to infinity.  
Based on this result, 
we present a model selection method using multiple hypothesis testing procedures that control the false discovery rate. 
The main result is presented in Theorem \ref{thm2},
the proof of which 
is based on a Taylor expansion of the log-likelihood function  
and through the application of a Lyapunov type bound presented in \citet{Raic19}. 


In the following, 
$\bZ$ will denote a standard multivariate normal random vector,
i.e.,
with mean vector equal to the zero vector and covariance matrix equal to the identity matrix
(each of appropriate dimension), 
and $\Phi$ will denote the corresponding probability measure. 
 

\begin{theorem}
\label{thm2}
Under the assumptions of Theorem \ref{thm1}, 
there exists $N_0 \geq 3$ such that,
for all $N \geq N_0$ and any measurable convex set $\mA \subseteq \mbR^p$, 
the error of the multivariate normal approximation
\beno
\left|\mbP((I(\truth) \, \norm{\bY}_1)^{1/2} \, (\mle - \truth) - \tilde\bR \in \mA) - \Phi(\bZ \in \mA)\right|
\ee
is bounded above by
\beno
\dfrac{83}{\xi_{\epsilon^\star}^{3/2}} \,
\sqrt{\dfrac{p^{7/2}}{\mbE \, \norm{\bY}_1}}
+  \dfrac{4}{\mbE \, \norm{\bY}_1} + \dfrac{8 \, \left[D_{g}\right]^{+}}{\left(\mbE \, \norm{\bY}_1 \right)^2}
\ee
where $\tilde\bR$ satisfies
\beno
\mbP\left(\norm{\tilde\bR}_2 \leq 
\dfrac{3 \, \sqrt{2} \, (1 + [D_{g}]^{+})}{\xi_{\epsilon^\star}^2} \;
\dfrac{p^{5/2} \, \log N}{\sqrt{\mbE \, \norm{\bY}_1}} \right)
%\dfrac{3 \, \sqrt{2} \, (1 + [D_{g}]^{+})}{\xi_{\epsilon^\star}^{5/2}} \,
%\dfrac{p^3\, \log N}{(\mbE \, \norm{\bY}_1)^{3/2}} \right)
&\geq& 1 - \dfrac{7}{\mbE \, \norm{\bY}_1} - \dfrac{8 \, [D_{g}]^{+}}{(\mbE \, \norm{\bY}_1)^2}.
\ee
\end{theorem}

\s


Theorem \ref{thm2} serves as a foundation for establishing the asymptotic normality of maximum likelihood estimators $\mle$
and maximum pseudolikelihood estimators $\mple$, 
noting Theorem \ref{thm1} established conditions under which both $\mle$ and $\mple$ are consistent estimators of 
$\truth$ (with respect to the Euclidean distance metric),
assumptions which are met by Theorem \ref{thm2}.   
If 
\beno
\lim\limits_{N \to \infty} \; 
\left[ \dfrac{83}{\xi_{\epsilon^\star}^{3/2}} \,
\sqrt{\dfrac{p^{7/2}}{\mbE \, \norm{\bY}_1}}
+  \dfrac{4}{\mbE \, \norm{\bY}_1} + \dfrac{8 \, \left[D_{g}\right]^{+}}{\left(\mbE \, \norm{\bY}_1 \right)^2} \right] 
\= 0,  
%\;\,\to\,\; 0
\ee
Theorem \ref{thm2} implies 
$(I(\truth) \, \norm{\bY}_1)^{1/2} \, (\mle - \truth) - \tilde\bR$ will converge
in distribution
to a standard multivariate normal random vector,
as error bound on the multivariate normal approximation will vanish in this case.  
The term $\tilde\bR$ can be viewed as an error term,
resulting from the fact that the normal approximation in Theorem \ref{thm2} is obtained via a multivariate Taylor approximation 
in order to bridge the distributional gap between key statistics which admit forms amenable to existing 
theorems for the normal approximation 
and the parameter vectors of interest,
thus introducing an additional source of error in the normal approximation. 
The same theory may be exported to the case of maximum pseudolikelihood estimators 
by exploiting the consistency of both $\mle$ and $\mple$ (with respect to the Euclidean distance metric)
implied via Theorem \ref{thm1} as the triangle inequality implies  
$\norm{\mle - \mple}_2 
\leq \norm{\mle - \truth}_2 + \norm{\mple - \truth}_2$.  

While involved, 
the above condition for asymptotic multivariate normality essentially places restrictions 
on the dependence induced through the single-layer network $\bY$ 
measured by $[D_{g}]^{+}$,
as well as the smallest eigenvalue of the dyad-based information matrix $\mcI(\nat)$ 
in a neighborhood of the data-generating parameter vector $\truth$ as measured by $\xi_{\epsilon^\star}$, 
and the dimension of the model $p$. 
As a result, 
if the information matrix $\mcI(\nat)$ is nearly singular at $\truth$, 
in which case $\xi_{\epsilon^\star}$ will be small,
the error of the normal approximation will be uniformly larger (all else equal).
Likewise, 
if the edge dependence in $\bY$ is large as measured by $[D_{g}]^{+}$,
we may not have sufficient activated dyads to ensure the error bound is small,
as $\norm{\bY}_1$ may not be tightly concentrated around $\mbE \, \norm{\bY}_1$. 
The dependence of the error approximation on the dimension of the random vector is a known challenge 
in establishing multivariate normality \citep[see, e.g.,][]{Raic19}.  
All quantities which are not explicit constants can increase or decrease with $N$,
with the rates of these increases or decreases having implications for the rate of convergence in distribution. 
Theorem \ref{thm2} demonstrates 
that the allowable scaling for most of quantities is with respect to the expected number of activated dyads 
$\mbE \, \norm{\bY}_1$.

We further examine Theorem \ref{thm2} through an example where  
$\bY$ is a Bernoulli random graph model,
which assumes edge variables are independent Bernoulli random variables with probability $\pi \in (0, 1)$. 
Under this model, 
$[D_{g}]^{+} = 0$ owing to the independence of edge variables
and $\mbE \norm{\bY}_1 = \pi \, \binom{N}{2}$. 
Under this scenario, 
we can show that 
\beno
\left|\mbP((I(\truth) \, \norm{\bY}_1)^{1/2} (\mle - \truth) - \tilde\bR \in \mA) - \Phi(\bZ \in \mA)\right|
&\leq& \dfrac{166}{\sqrt{\pi \, \xi_{\epsilon^\star}^{3}}} \; \dfrac{p^{1.75}}{N} + \dfrac{16}{\pi  N^2}, 
\ee
with the additional bound   
\beno
\mbP\left( \norm{\tilde\bR}_2 \,\leq\, \dfrac{6 \sqrt{2}}{\xi_{\epsilon^\star}^2} \, \dfrac{p^{2.5} \, \log N}{\pi \, N} \right) 
&\geq& 1 - \dfrac{28}{\pi \, N^2}. 
\ee 
If $\xi_{\epsilon^\star}$ and $\pi$ are both bounded away from $0$,
then the error of the normal approximation will convergence to $0$ 
provided $(p^{2.5} \log N) \,/\, N \to 0$ as $N \to \infty$,
which is sufficient to ensure $\norm{\tilde\bR}_2$ converges in probability to $0$. 
Under the fully saturated model specification for \eqref{general model} ($H = K$), 
the Binomial theorem shows that $p = 2^{K} - 1 \leq 2^K$. 
Hence, 
the dimension restriction on $p$ in turn implies a restriction on the allowable rate of growth of the number of layers $K$ with $N$,
where a sufficient condition for $(p^{2.5} \log N) \,/\, N \to 0$ 
is for $K \leq .5 \, \log N$. 
In other words, 
the number of layers $K$ can grow at most logarithmically with $N$ in the fully saturated model. 
In cases when the number of interaction terms included in the cross-layer dependence probability model 
is fixed, 
$K$ may admit a sublinear scaling  with $N$. 






%We prove Theorem \ref{thm2} in Appendix \ref{sec:pf_thm2}.

\subsection{Model selection via univariate testing with FDR control} 

Provided with the consistency results and the multivariate normal approximation of $\mle$
through Theorems \ref{thm1} and \ref{thm2},
we outline a procedure for model selection that controls the false discovery rate.
Hotelling's $T$-squared statistic can be used to conduct a global test for 
$H_0: \truth = \bmu$ versus $H_1:\truth \neq \bmu$, where $\bmu \in \mbR^p$ is the value of $\nat$ we want to test \citep[Chapter 5,][]{multivariate_test}.
We will mostly be interested in the case when $\bmu = \bm{0}_p$,
i.e.,
the zero vector of dimension $p$. 

If the global test is rejected, 
or if the global test is not of interest,  
we can perform model selection by leveraging the multivariate normal approximation 
to obtain univariate normal approximation results for the components of $\mle$
and proceed to test each component: $ H_{i,0} : \theta^\star_i=\mu_i$ 
versus $H_{i,1}: \theta^\star_i\neq\mu_i$,
for $i = 1, \ldots p$ and $\mu_i \in \mbR$. 
%\begin{equation*}
%\begin{array}{lllllllllllllllll}
%    H_{i,0} : \theta^\star_i=\mu_i & \text{versus} & H_{i,1}: \theta^\star_i\neq\mu_i,
%&&& \mbox{for } i = 1, \ldots, p,
%& \mbox{with } \mu_i \in \mbR.  
%\end{array}
%\end{equation*}
In general, 
$\mu_i = 0$ will allow us to test whether the estimated effect $\widehat\theta_i$ is present in the model
(i.e., whether $\theta^\star_i \neq 0$).  
One challenge in this approach lies in the fact that the model selection procedure 
is sensitive to multiple testing error. 
We propose to control the multiple testing error by appropriate multiple testing adjustments
by elaborating a model selection algorithm which will control the false discovery rate
in order to accurately learn the cross-layer dependence effects present in the multilayer network,
and in effect learning the cross-layer dependence structure of the multilayer network.   
We provide simulation examples of four different univariate testing procedures including Bonferroni, Benjamini-Hochberg, 
Hochberg, 
and Holm procedures in Section \ref{sec:sim_norm}. 
In simulation studies, 
all four univariate testing procedures exhibit strong statistical power 
for detecting non-zero parameters while controlling the false discovery rate at a preset family-wise significance level. 
As Theorem \ref{thm1} establishes the consistency of both $\mle$ and $\mple$,
the above procedure remains justifiable for performing model selection with maximum pseudolikelihood estimators as well,  
as it is straightforward to prove a corollary to Theorem \ref{thm1} 
which establishes that $\norm{\mle - \mple}_2$ converges in probability to $0$ under the assumptions of Theorem \ref{thm1},
further obtaining convergence in distribution. 








\section{Simulation studies}
\label{sec:sim}
Directly simulating maximum likelihood estimators for network data with dependent edges is challenging because the normalizing constants are often computationally intractable. Computing the normalizing constant requires enumerating all $2^{\binom{N}{2}}$ possible edge combinations for each layer to maximize the true likelihood function. Additionally, dependencies among network dyads prevent factorization of the likelihood, which further complicates direct maximization. As a result, direct maximization of likelihood functions is generally infeasible in these cases.
Two predominant methods of approximating the maximum likelihood estimator $\truth$ when the likelihood function is computationally intractable 
have emerged in the literature. 
Monte Carlo maximum likelihood estimation (MCMLE) \citep{GeTh92},
which constructs a simulation-based approximation to the likelihood function 
in order to approximate the maximum likelihood estimator,
is an established method for approximating maximum likelihood estimators 
in the statistical network analysis literature \citep{HuHa06}.
While able to provide accurate estimates of maximum likelihood estimators for complex models 
\citep[e.g.,][]{StScBoMo19,ScKrBu17}, 
a drawback of MCMLE,
and other simulation-based estimation methodology, 
is the computational burden which can scale with both the complexity of the model and the size of the network \citep{BaBrSl11}.  
In settings where the computation of the MCMLE is impractical, 
a computationally efficient alternative is provided via the maximum pseudolikelihood estimator (MPLE) \citep{Bj74},
whose application to social network analysis and to statistical network analysis dates back to \citet{StIk90}. 
Pseudolikelihood-based estimators have the following computational advantages:
\ben
\item Algorithms are generally deterministic and do not require simulation-based approximation schemes,
which aids in reproducibility of results; 
\item Algorithms are generally more scalable, 
relative to alternatives such as MCMLE and other simulation-based approximations, 
and are able to be parallelized to take advantage of larger  multicore computing infrastructures which are becoming increasingly common.  
\een


\hide{
Proposition \ref{prop:inference} establishes that $\bY$ is observable through $\bX$,
i.e., 
\beno
\mbP(Y_{i,j} = y_{i,j} \,|\, \bX = \bx, \bY_{-\{i,j\}} = \by_{-\{i,j\}})
\= 1,
\ee
when $y_{i,j} = \one(\norm{\bx_{i,j}}_1 \,>\, 0)$ and $\bY_{-\{i,j\}}$ is defined to be the 
($\binom{N}{2}$-$1$)-dimensional vector of edge variables in $\bY$ which excludes $Y_{i,j}$. 
As a result, 
if $(\bx, \by)$ is network concordant,
then  
\beno
\log \, \mbP(Y_{i,j} = y_{i,j} \,|\, \bX = \bx, \bY_{-\{i,j\}} = \by_{-\{i,j\}}) \= 0, 
&& \mbox{for all } \{i,j\} \subset \mN.
\ee 

The log-pseudolikelihood of \eqref{general model} can then be written down as 
\be
\label{eq:log-pseudo}
\pl(\nat; \bx, \by) 
\,\coloneqq \dsum_{\{i,j\} \subset \mN}  \dsum_{k=1}^K \, \log  
\sepmodel(X_{i,j}^{(k)} = x_{i,j}^{(k)} \, |\, \bX_{i,j}^{(-k)} = \bx_{i,j}^{(-k)}, \bY = \by), 
\ee
provided $(\bx, \by)$ is network concordant
and by exploiting the conditional independence properties implied by \eqref{general model}. 
We denote the set of maximum pseudolikelihood estimators of the data-generating parameter vector $\truth$ by 
\beno
\Mple &\coloneqq& \left\{ \nat \in \mbR^p \,: \, \pl(\nat; \bx, \by) = \sup\limits_{\nat^{\prime} \in \mbR^p} \, \pl(\nat^{\prime}; \bx, \by) \right\}.
\ee
Individual elements are referenced by $\mple \in \Mple$.
When it exists, 
the maximum pseudolikelihood estimator may or may not be unique.
However,
our theoretical results establish that all elements $\mple \in \Mple$ will all be within the same Euclidean distance to $\truth$.
The assumption that $(\bx, \by)$ is network concordant comes at no cost,
since $\bY$ is predictable through $\bX$,
as discussed above. 
%meaning that given an observation $\bx$ of $\bX$,
%we can find the unique network concordant pair $(\bx, \by)$ with probability one.  
The advantage of \eqref{eq:log-pseudo} is that 
the conditional probabilities 
$\sepmodel(X_{i,j}^{(k)} = x_{i,j}^{(k)} \, |\, \bX_{i,j}^{(-k)} = \bx_{i,j}^{(-k)}, \bY = \by)$ 
of edges in the multilayer network 
are often computationally tractable 
since the conditional distribution is a Bernoulli distribution when $Y_{i,j} = 1$,
and is a degenerate point mass at $0$ when $Y_{i,j} = 0$.

}

In this simulation, 
we consider the maximum pseudolikelihood estimator, denoted by $\mple$.
We conduct simulation studies to investigate the performance of the maximum pseudolikelihood estimator $\mple$ (MPLE),
supplementing the theoretical results established in Sections \ref{sec3} and \ref{sec:normal} for maximum likelihood estimators. 
As will be discussed later in Section \ref{sec:app}, we successfully reproduced the sufficient statistics using the MPLE in the application, suggesting that the MPLE for the multilayer network model solves a score equation similar to that of the MLE. This indicates that the MPLE serves as a close approximation and can be a good proxy for the MLE.
In section \ref{sec:sim_con}, 
we demonstrate the consistency results of Theorem \ref{thm1} 
in settings of different data-generating parameters and increasing model dimensions.
We conduct simulation studies of the multivariate normal approximation established by Theorem \ref{thm2} 
in Section \ref{sec:sim_norm}.  
Lastly, 
we discuss several testing procedures for selecting non-zero effects 
while controlling the false discovery rate (FDR) at a given family-wise significance level $\alpha$. 
 
In all simulation studies, 
we sample concordant multilayer networks $(\bX,\bY)$ from \eqref{general_model} 
with the maximum order of corss-layer interaction $H=2$:
\be
\label{eq:sim_model}
f(\bx, \nat) = \dprod_{\{i,j\} \subset \mN} \, \exp\left( \dsum_{k=1}^K\theta_{k}\,x_{i,j}^{(k)} + \dsum_{\substack{k < l}}^{K}\, \theta_{k,l} \, x_{i,j}^{(k)} \, x_{i,j}^{(l)} \right).
\ee
Unless otherwise specified, the basis network $\bY$ is generated from the Bernoulli random graph model.

% Figure environment removed

% Figure environment removed


\subsection{Consistency} 
\label{sec:sim_con}
The consistency is demonstrated through the decay of the relative $\ell_2$-errors between $\mple$ and the data-generating parameter $\truth$ as the expected number of activated dyads $\mbE\norm{\bY}_1$ increases. 
We generated $M = 250$ multilayer networks with $N=300$ nodes, using $M$ different data-generating parameters. We created these networks for each of ten evenly spaced numbers of activated dyads increasing from $3000$ to $30000$, and for four different numbers of layers increasing from $K=3$ to $6$. The model dimension increases from 6 to 21 as $K$ increases from 3 to 6.
For each number of activated dyads, 
number of layers $K$, 
and replicate, 
we sample a multilayer network $\bX$ from \eqref{general_model} using the specification in \eqref{eq:sim_model}
with the data-generating parameter vector $\truth$ populated by randomly selecting each component from the uniform distribution on $(-1,1)$.
We make the exception that components $\theta_{3}^\star$ and $\theta_{1,3}^\star$ are set to $0$.
In each replicate, 
we compute the maximum pseudolikelihood estimator. 
The results of this simulation study are given in Figure \ref{M_theta_error}, 
which shows the decay of the relative $\ell_2$-errors between $\mple$ and $\truth$ as the number of activated dyads increases in networks with different number of layers. 
The broad selection of data-generating parameter values on networks with increasing number of layers verifies that Theorem \ref{thm1} holds in many practical settings with increasing model dimensions. 





\hide{
The decay rate of the relative $\ell_2$-error as the network size increases can be estimated by the slope of the OLS fitted line in the $\log$-$\log$ plot as shown in Figure \ref{Figure1}(b). The slope of the OLS line is $-1$, indicating a decay rate of order $1/N$, which follows the result in Theorem \ref{thm1}. We report the MPLE $\mple$ averaged from $M = 500$ samples with the selected data-generating parameter $\truth$ at network size 1000 in table \ref{t1}.

\begin{table}[t]
\begin{center}
\caption{\label{t1}Values of the data-generating parameter $\truth$ and mean of MPLE $\mple$ from 500 replications at network size 1000. Standard errors are in the parenthesis.}
\begin{tabular}{| c | c | c | c | c | c | c |} 
\hline
  & $\theta_{1}$&$\theta_{2}$&$\theta_{3}$& $\theta_{1,2}$ & $\theta_{1,3}$ & $\theta_{2,3}$ \\ 
\hline
$\truth$ & $-3$ & $-2$ & $-1$ & $.5$ & $0$ & $0$  \\
\hline
$\mple$ & $-2.999 \, (.02)$ & $-1.998 \, (.02)$ & $-.999 \, (.02)$ & $.499 \, (.02)$ & $-.001 \, (.02)$ & $-.002 \, (.02)$\\
\hline
\end{tabular}
\end{center}
\end{table}

}

% % Figure environment removed



\subsection{Multivariate normality and model selection} 
\label{sec:sim_norm}


\begin{table}[t]
\begin{center}
\caption{\label{fdr} False discovery rates of four procedures for detecting non-zero effects of 6 data-generating parameters 
($\truth_1$, $\truth_2$, $\truth_3$, $\truth_{4}$, $\truth_{5}$, $\truth_{6}$) estimated from 250 multilayer network samples at size 1000 on the dense Bernoulli basis network. All FDRs are smaller than $.05$.} 
\begin{tabular}{ l  r  r  r  r  r  r  } 
\hline
  Procedure & $\truth_1$ & $\truth_2$  & $\truth_3$ & $\truth_4$ & $\truth_5$ & $\truth_6$ \\ 
\hline
  Bonferroni    & .004  & .002 & .001 & .002 & .001 & .005  \\

  Benjamini-Hochberg   & .014 & .014 & .014 & .011 & .017 & .020  \\

 Hochberg &    .012 & .008 & .009 & .008 & .011 & .016 \\

 Holm  &   .010 & .008 & .006 & .008 & .007 &  .013 \\
\hline
\end{tabular}
\end{center}
\end{table}



As stated in Section \ref{sec:normal} and Theorem \ref{thm2}, 
the distribution of the maximum likelihood estimator $\mle$ 
converges in distribution to a multivariate normal distribution asymptotically. 
In order to study the quality of the normal approximation---especially for univariate testing 
which would be used for the false discovery rate control and model selection---we 
randomly select $6$ of the $250$ data-generating parameter vectors $\truth$
used to study the consistency results of Theorem \ref{thm1} 
in the simulation study conducted in Section \ref{sec:sim_con}.
We then generate $250$ replicates of multilayer network samples by each of these $6$ parameter vectors, using specification \eqref{eq:sim_model} on four basis network structures with the number of layers $K=3$:
the Bernoulli random graph model (dense and sparse), 
the stochastic block model, 
and the latent space model. 

The multivariate normality of $\mple$ passed Zhou-Shao's multivariate normal test \citep{Zhou13}, 
with $p$-values provided in the Appendix \ref{subsec:norm_sim} in the supplement to this paper.  
We visualize the marginal normality of individual component in $\mple$ with a dense Bernoulli basis network
in Figure \ref{qqplot_denseBer},
through Q-Q plots of the simulated maximum pseudolikelihood estimators. 
Univariate tests for normality failed to reject the null hypothesis that
each component of $\mple$ is marginally normal at a significance level of $.05$. 
Additional results studying the multivariate normality of $\mple$ on different basis network structures 
are provided in Appendix \ref{subsec:norm_sim} in the supplement to this paper.
 

 

\hide{
We first demonstrate through Q-Q plots in Figure \ref{Figure3} that each component of the MPLE $\mple$ follows a marginal normal distribution. 
% We constructed 95\% confidence ellipses for bivariate components of $\mple$ in figure \ref{Figure4}. The two axes in each sub-plot of figure \ref{Figure4} correspond to two different components of $\mple$. The red ellipse is the 95\% confidence ellipse generated by the mean and covariance matrix of the MPLEs in the plot under the bivariate normal assumption. All 15 bivariate confidence ellipses cover most of estimates points, suggesting probable bivariate normal distributions of $\mple$. 


\begin{table}%[t]
\begin{center}
\caption{\label{ZS-test} $p$-values of the Zhou-Shao's test for multivariate normality of $\mple$ for 6 data-generating parameters ($\truth_1$, $\truth_2$, $\truth_3$, $\truth_4$, $\truth_5$, $\truth_6$) estimated from 250 network samples at size 1000 on four basis network structures. All $p$-values are larger than .05. \s} 
\begin{tabular}{| c | c | c | c | c | c | c | } 
\hline
 Basis network model  & $\truth_1$ & $\truth_2$  & $\truth_3$ & $\truth_4$ & $\truth_5$ & $\truth_6$ \\ 
\hline
  Dense Bernoulli & .138 & .473 & .053 & .699 & .587 & .983  \\
\hline
 Sparse Bernoulli & .554 & .132 & .232 & .634 & .904 & .373  \\
\hline
 SBM & .65 & .891 & .982 & .975 & .871 & .674 \\
\hline
 LSM  & .859 & .831 & .5 & .227 & .613 & .409  \\
\hline
\end{tabular}
\end{center}
\end{table}
}
\hide{
We additionally performed marginal tests for normality for each component. In each case, the marginal test failed to reject the null hypotheses that $\mple_i$ is marginal normal at the same significance level, for each component $i = 1,\ldots, p$, the result of which is consistent with the $Q$-$Q$ plots. 
}


\hide{
% Figure environment removed


\begin{table}[t]
\begin{center}
\caption{\label{t3} Empirical FDR and power at significance level $.05$ for multiple testing of MPLE $\mple$ 
estimated from $500$ networks of size $1000$.  } 
\begin{tabular}{| c | c | c | c |} 
\hline
 Procedure  & Empirical FDR & FDR 95\% one-sided CI  & Empirical power \\ 
\hline
  Bonfferoni & .138 & 0.473 & .038 & .699 & .587 & .983  \\
\hline
  Benjamini-Hochberg's &  .018 & (0, .134) & 1  \\
\hline
 Hochberg's  & .016 & (0, .127) & 1 \\
\hline
 Holm's  & .013 &  (0, .114) & 1  \\
\hline
\end{tabular}
\end{center}
\end{table}
}

We then implement the multiple testing correction procedures of
Bonferroni, Benjamini-Hochberg, Hochberg, and Holm, for the 6 selected data-generating parameter vectors $\truth$ with 250 replicates to detect components that are significantly different from $0$ 
while controlling the false discovery rate (FDR) at a family-wise significance level of $\alpha = .05$---recall 
that $\theta_{1,3}^\star$ and $\theta_{3}^\star$ of $\truth$ are set to $0$ 
in each simulation replicate.  
We estimate the FDR of the four procedures by averaging the false discovery proportions from 250 replicates of each of the 6 randomly selected data-generating parameters $\truth$.
We provide the estimated FDRs for $\truth$ on a dense Bernoulli basis network in Table \ref{fdr}. 
In addition, we show the receiver operating characteristic (ROC) curves for $\mple$ estimating the 6 selected data-generating parameters in each of the subplot of Figure \ref{ROC}, on four basis network structures in Appendix \ref{more fdr} in the supplement to the paper. 
Simulation results suggest that the false discovery rate is controlled below the preset threshold $\alpha$. Different data-generating parameter values affect the trade-off between the sensitivity and the specificity of the model selection. In general, multilayer networks with a larger effective sample size lead to a larger area under the ROC curve which offers a tool to choose appropriate correction procedures and thresholds for model selection in different scenarios. 
Additional results on the false discovery rate with different basis network structures are provided in Appendix \ref{more fdr} in the supplement to the paper. 



\section{Application} 
\label{sec:app}

We present a case study using a dataset on corporate law partnership among a Northeastern US corporate law firm in New England collected by \citet{Lazega2001}. The dataset collected information about three types of cooperation among 71 lawyers in the corporate law firm, resulting in three networks including the strong-coworker network, the advice network, and the friendship network. Since the cooperation relationship collected are not symmetric, we only consider a connection to be present when both sides acknowledged their cooperation. We treat these three types of networks as a three-layer multilayer network embedded among the 71 lawyers. A summary of this multilayer network is provided in Table \ref{t_datasum}. We apply the separable multilayer network model in \eqref{general_model} with the specification in \eqref{eq:sim_model} to Lazega's lawyer network, i.e., $\nat = (\theta_{1},\,\theta_{2},\,\theta_{3},\,\theta_{1,2},\,\theta_{1,3}, \,\theta_{2,3})$. The maximum pseudolikelihood estimator $\mple$ is computed from the observed network,
the results of which are provided in Table \ref{t_Lazeg_mple}.
 % Figure environment removed


\begin{table}[t]
\begin{center}
\caption{\label{t_datasum} Summary of Lazega's corporate law partnership data with 71 lawyers (nodes).  }
\begin{tabular}{ l  r  r }
\hline
 & Average Node Degree & Number of Edges \\
\hline
 Co-Worker Layer  & 11 & 378  \\

Advice Layer & 5 & 175  \\

Friendship Layer & 5 & 176  \\
\hline
\end{tabular}
\end{center}
\end{table}


\begin{table}[t]
\begin{center}
\caption{\label{t_Lazeg_mple}MPLEs (and standard errors) of the separable multilayer network model for the Lazega's lawyer network.}
\resizebox{\columnwidth}{!}{
\begin{tabular}{ c  c  c  c  c  c } 
\hline
$\widetilde\theta_{1}$ &$\widetilde\theta_{2}$ &$\widetilde\theta_{3}$ & $\widetilde\theta_{1,2}$ & $\widetilde\theta_{1,3}$ & $\widetilde\theta_{2,3}$ \\ 
\hline
$-1.450 \; (.263)$ & $-3.334 \; (.244)$ & $-2.695\; (.256)$ & $1.801 \; (.244)$ & $0.218 \; (.247)$ & $2.458 \; (.231)$ \\

Coworker (C) & Advice (A) & Friendship (F) & C $\times$ A & C $\times$ F & A $\times$ F \\
\hline
\end{tabular}
}
\end{center}
\end{table}


As shown in Table \ref{t_Lazeg_mple}, 
the maximum pseudolikelihood estimates 
$\widetilde\theta_{1}, \, \widetilde\theta_{2}$, and $\widetilde\theta_{3}$ correspond to the estimated single-layer effects of the coworker layer, the advice layer, and the friendship layer, respectively,
whereas $\theta_{1,2}, \, \theta_{1,3}$, and $\theta_{2,3}$ correspond to the layer interaction effects.  
We can calculate the conditional log-odds of each edge being present in the multilayer network given the rest of the network. For example,  if lawyer $i$ and lawyer $j$ are observed to be coworkers and are friends at the same time, the odds of these two lawyers to have an advice relationship is given by
\beno
\dfrac{\mbP(X_{i,j}^{(A)} = 1 \,|\, \bX_{i,j}^{(C)} = 1, \, \bX_{i,j}^{(F)} = 1)}
{\mbP(X_{i,j}^{(A)} = 0 \,|\, \bX_{i,j}^{(C)} = 1, \, \bX_{i,j}^{(F)} = 1)}
\quad = \quad \exp\left(
\widetilde\theta_{2} +  \widetilde\theta_{1,2} \, x_{i,j}^{(C)} + \widetilde\theta_{2,3} \, x_{i,j}^{(F)}\right) \s \\
 = \quad \exp\left(
-3.334 +  1.801 \, x_{i,j}^{(C)} + 2.458 \, x_{i,j}^{(F)}\right) 
 \quad = \quad  2.522,
\ee
providing interpretation of the interaction and influence among the different layers.

% Figure environment removed

Next, 
we use the MPLE to reproduce multilayer networks of the same size and compare the sufficient statistics 
of the simulated networks and the Lazega's lawyer network.
We recover the basis network according to Proposition \ref{prop:inference}, i.e.,
a dyad is activated if and only if at least one of its layers has a present edge in the Lazega's lawyer multilayer network.
We then populate layers of all activated dyads according to equation \eqref{eq:sim_model} by the MPLEs obtained in Table \ref{t_Lazeg_mple}.
Comparisons of the sufficient statistics between the observed Lazega's lawyer network and the simulated networks with 10 replications are provided in 
Figure \ref{Figure_Lazega}. 
Such comparisons serve two key purposes. 
First, 
such comparisons are an established method of diagnosing model fit in the statistical network analysis literature 
\citep{HuGoHa08}, 
and second, 
provide a check on the approximate solution to the score equation.
Note that MPLEs are not guaranteed to reproduce (on average) observed values of sufficient statistics in exponential families---in contrast to MLEs.
The relative $\ell_2$-error of the sufficient statistics between the observed and the average of the 10 simulated networks is $0.09$, suggesting a successful re-construction of the observed network statistics.

\hide{
The R package we developed for the simulation analysis in Section \ref{sec:sim} and the application analysis of Lazega's corporate law partnership data in Section \ref{sec:app} is available on GitHub: https://github.com/jiaheng-li/cross-layer-dependence-mlyrnetwork-simulation.
}




\section{Discussion} 
\label{sec:disc} 

In this work, 
we introduced a flexible class of statistical models for multilayer networks. 
Key to our approach lies in the integrative nature by which we establish our framework, 
extending arbitrary strictly positive probability distributions for single-layer networks 
to multilayer-network models through a novel separable framework with Markov random field specifications. 
We established the foundations for statistical inference through consistency and multivariate normality results,
the results of which have been demonstrated in simulation studies and in an application.  
The key assumption to our approach lies in the network separability assumption, 
which necessitates network dyads be conditionally independent given the basis network. 
This assumption may or may not be valid in practice,
which would necessitate the development of generalizations of the framework we established in this work 
through the relaxation of the conditional independence assumption.  
Such relaxations would result in more complex dependence structures,
requiring 
new and careful theoretical treatment in order to establish similar statistical foundations of models 
to the ones we have developed here,  
representing potential avenues for future research. 

\section*{Acknowledgements} 

Jonathan R. Stewart was supported by 
NSF award SES-2345043 and 
the Department of Defense Test Resource Management Center under contracts FA8075-18-D-0002 and FA8075-21-F-0074.


\bibliographystyle{agsm}
\bibliography{base} 




\newpage

\begin{appendices}


\newpage
\appendix

\section{Proof of Lemma \ref{lemma, equivalence of two def of MDDO}}
\begin{proof}
For any ${\bs{\beta}}\in\mc H$, according to the definition of $G_{\bs s}$ (see Definition $\ref{def: MDDO}$), one has
\begin{align*}
\langle G_{\bs s},{\bs{\beta}}\rangle&=\int_{[0,1]} G_{\bs s}(t){\bs{\beta}}(t)~\mathrm{d}t=\int_{[0,1]}\mathrm{cov}\hspace{-0.9mm}\left(\bs{X}(t),\mathrm{e}^{\mi\langle \bs s,\Y\rangle}\right){\bs{\beta}}(t)~\mathrm{d}t\\
&=\int_{[0,1]}\mathrm{cov}\hspace{-0.9mm}\left(\bs{X}(t){\bs{\beta}}(t),\mathrm{e}^{\mi \langle \bs s,\Y\rangle}\right)~\mathrm{d}t.
\end{align*}
By Fubini theorem, under Assumption $\ref{as:joint distribution assumption}$, one can exchange the order of integration and covariance above and get that
\begin{align*}
 \langle G_{\bs s},{\bs{\beta}}\rangle&=\int_{[0,1]}\mathrm{cov}\hspace{-0.9mm}\left(\bs{X}(t){\bs{\beta}}(t),\mathrm{e}^{\mi \langle \bs s,\Y\rangle}\right)~\mathrm{d}t\\ &=\mathrm{cov}\hspace{-0.9mm}\left(\int_{[0,1]}\bs{X}(t){\bs{\beta}}(t)~\mathrm{d}t,\mathrm{e}^{\mi \langle \bs s,\Y\rangle}\right)=\mathrm{cov}\hspace{-0.9mm}\left(\langle \bs{X},{\bs{\beta}}\rangle,\mathrm{e}^{\mi \langle \bs s ,\Y\rangle}\right).
\end{align*}
Thus for any $\bs\alpha(t),{\bs{\beta}}(t)\in\mc H$, one can get
\begin{align*}
\big\langle \big(G_{\bs s}\otimes \overline{G}_{\bs s}\big)\bs\alpha,{\bs{\beta}}\big\rangle=\langle G_{\bs s},\bs\alpha\rangle\langle \overline{G}_{\bs s},{\bs{\beta}}\rangle=\mathrm{cov}\hspace{-0.9mm}\left(\langle \bs{X},\bs\alpha\rangle,\mathrm{e}^{\mi \langle \bs s,\Y\rangle}\right)\hspace{-0.9mm}\mathrm{cov}\hspace{-0.9mm}\left(\langle \bs{X},{\bs{\beta}}\rangle,\mathrm{e}^{-\mi\langle \bs s,\Y\rangle}\right)\\
=\mb{E}\hspace{-0.9mm}\left(\langle \bs{X},\bs\alpha\rangle\mathrm{e}^{\mi \langle \bs s,\Y\rangle}\right)\mb{E}\hspace{-0.8mm}\left(\langle \bs{X},{\bs{\beta}}\rangle\mathrm{e}^{-\mi \langle \bs s,\Y\rangle}\right)=\mb{E}\Big(\langle \bs{X},\bs\alpha\rangle\langle \bs{X}',{\bs{\beta}}\rangle\mathrm{e}^{\mi \langle \bs s,\Y-\Y'\rangle}\Big).
\end{align*}
Considering that $\mb{E}\big(\langle \bs{X},\alpha\rangle\langle \bs{X}',{\bs{\beta}}\rangle\big)=0$, one has
\begin{align*}
\big\langle \big(G_{\bs s}\otimes \overline{G}_{\bs s}\big)\bs\alpha,{\bs{\beta}}\big\rangle
=- \mb{E}\Big(\langle \bs{X},\bs\alpha\rangle\langle \bs{X}',{\bs{\beta}}\rangle\big(1-\mr{e}^{\mi \langle \bs s,\Y-\Y'\rangle}\big)\Big)&\\
=- \mb{E}\Big(\langle \bs{X},\bs\alpha\rangle\langle \bs{X}',{\bs{\beta}}\rangle\big[1-\cos\big(\langle \bs s,\Y-\Y'\rangle\big)\big]\Big)&\\
+\mi\mb{E}\Big(\langle \bs{X},\bs\alpha\rangle\langle \bs{X}',{\bs{\beta}}\rangle\big[\sin\big(\langle\bs s,\Y-\Y'\rangle\big)\big]\Big)&.
\end{align*}
It is easy to check that
\[\int_{\mb R^q}\frac{\sin \big(\langle\bs s,\Y-\Y'\rangle)\big)}{\|\bs s\|^{1+q}}~\mr{d}\bs s=\lim_{\varepsilon\to0^+}\int_{\bs s\in\mb{R}^q:\varepsilon\leqslant\|\bs s\|\leqslant \varepsilon^{-1}}\frac{\sin \big(\langle \bs s,\Y-\Y'\rangle\big)}{\|\bs s\|^{1+q}}~\mr{d}\bs s=0,\]
because the integrand is an odd function. By Lemma 1 in \cite{szekely2007measuring},  one can also get
\[\int_{\R^q}\frac{1-\cos\big(\langle \bs s,\Y-\Y'\rangle\big)}{\|\bs s\|^{1+q}}~\mr{d}\bs s=c_q\|\Y-\Y'\|.
\]
Combining above results with Definition $\ref{def: MDDO}$, one can obtain that 
\begin{align}\label{proof: lemma MDDO}
\langle\mathrm{MDDO}(\bs{X}|Y)\bs\alpha,{\bs{\beta}}\rangle=- \mb{E}\Big(\langle \bs{X},\bs\alpha\rangle\langle \bs{X}',{\bs{\beta}}\rangle\|\Y-\Y'\|\Big) .
\end{align}
Then by the arbitrariness of $\bs\alpha,{\bs{\beta}}\in\mc H$, the proof is completed. 
\end{proof}

\section{Proof of Theorem \ref{theorem, MDDO and conditional mean independence}}



According to \eqref{proof: lemma MDDO}, one can get the following useful lemma.
\begin{lemma}\label{lemma, MDDO and FMDD}
Under Assumption $\ref{as:joint distribution assumption}$, for all ${\bs{\beta}}\in\mathcal H$, $\|{\bs{\beta}}\|=1$, we have
\begin{align*}
\langle \mathrm{MDDO}(\boldsymbol{X}|\Y)({\bs{\beta}}),{\bs{\beta}}\rangle &=- \mathbb E\Big[ \langle \boldsymbol{X},{\bs{\beta}}\rangle \langle \boldsymbol{X}',{\bs{\beta}}\rangle \|\Y-\Y'\|\Big]\\
&=- \mathbb E\Big[\big\langle\langle \boldsymbol{X},{\bs{\beta}}\rangle{\bs{\beta}},\langle \boldsymbol{X}',{\bs{\beta}}\rangle{\bs{\beta}}\big\rangle\|\Y-\Y'\|\Big].
\end{align*}
\end{lemma}
This conclusion links MDDO with functional martingale
difference divergence  (FMDD, \citealt{lee2020testing}). 
Next we give the following two lemmas to finish the proof of Theorem $\ref{theorem, MDDO and conditional mean independence}$.
\begin{lemma}\label{lem: Txx=0tuiTx=0}If $T$ is a positive semi-definite operator on a Hilbert space $\wt{\mathcal{H}}$, then for all $x\in\wt{\mathcal{H}}$, one has $\langle Tx,x\rangle=0\Longleftrightarrow Tx=0$.
\end{lemma}
\begin{proof}
`$\Longleftarrow$': It is obvious.

`$\Longrightarrow$': It is easy to check that $f(a,b)=\langle Ta,b\rangle$ $(a,b\in\wt{\mc H})$ is a 
positive semi-definite Hermitian form. Thus, for any $y\in\wt{\mathcal{H}}$, one can use Cauchy inequality to get
\[|\langle Tx,y\rangle|^2\leqslant\langle Tx,x\rangle\langle Ty,y\rangle=0\Longrightarrow \langle Tx,y\rangle=0.\]
By the arbitrariness of $y\in\wt{\mc H}$, one has $Tx=0$.
\end{proof}

Our proof of Theorem $\ref{theorem, MDDO and conditional mean independence}$ is mainly inspired by the following property of
FMDD in \cite{lee2020testing}.
\begin{lemma}[Proposition 1 of \cite{lee2020testing}]\label{lem:prop1inlee}
If $\E[\|\X\|+\|\Y\|]<\infty$ and $\E[\|\bs X\|\|\Y\|]<\infty$, then we have
\[\E[\langle \X,\X'\rangle\|\Y-\Y'\|]=0\Longleftrightarrow \E[\X|\Y]=0\quad\text{almost surely},\]
where $(\X',\Y')$ is an i.i.d. copy of $(\X,\Y)$.
\end{lemma}
\paragraph{Proof of Theorem $\ref{theorem, MDDO and conditional mean independence}$}
\begin{proof}
Clearly, (ii) is a direct consequence of Lemma $\ref{lemma, equivalence of two def of MDDO}$ and the following lemma.

\begin{lemma}[Lemma 15 in \citealt{chen2023optimality}]\label{lem:cov TX}
If $T$ is an operator defined on $\mc H_1\to\mc H_2$ where $\mc H_i,i=1,2$ is a Hilbert space. $\bs X\in\mc H_1$ is a random element satisfying $\mb E[\bs X]=0$ . Then we have $\mr{var}(T\bs X)=T\mr{var}(\bs X)T^*$.
\end{lemma}

Now we start  to prove (i).
 First, one has
\begin{align*}\mathrm{MDDO}(\boldsymbol{X}|\Y)=0 &\Longleftrightarrow \mathrm{MDDO}(\boldsymbol{X}|\Y)({\bs{\beta}})=0,\quad\forall{\bs{\beta}}\in\mb{S}_{\mathcal H};\\
\mathbb E[\boldsymbol{X}|\Y]=0~~\text{a.s.}&\Longleftrightarrow\langle\mb E[\boldsymbol{X}|\Y],{\bs{\beta}}\rangle{\bs{\beta}}=0~~\text{a.s.} \quad\forall{\bs{\beta}}\in\mb{S}_{\mathcal H},
\end{align*}
where $\mb{S}_\mc{H}=\{{\bs{\beta}}\in\mc H:\|{\bs{\beta}}\|=1\}$. Second, from Lemma $\ref{lem: Txx=0tuiTx=0}$, one knows that
\begin{align*}
\mathrm{MDDO}(\boldsymbol{X}|\Y)({\bs{\beta}})=0&\Longleftrightarrow\langle\mathrm{MDDO}(\boldsymbol{X}|\Y)({\bs{\beta}}),{\bs{\beta}}\rangle=0.
\end{align*}
 Then under Assumption $\ref{as:joint distribution assumption}$, by Lemma $\ref{lemma, MDDO and FMDD}$ and $\ref{lem:prop1inlee}$, one has
\begin{align*}
&\langle\mathrm{MDDO}(\boldsymbol{X}|\Y)({\bs{\beta}}),{\bs{\beta}}\rangle=0\Longleftrightarrow\mathbb E[\big\langle\langle \boldsymbol{X},{\bs{\beta}}\rangle{\bs{\beta}},\langle \boldsymbol{X}',{\bs{\beta}}\rangle{\bs{\beta}}\rangle\|\Y-\Y'\|]=0\\
&\qquad\qquad\qquad\qquad\qquad\Longleftrightarrow\mathbb E[\langle \bs X,{\bs{\beta}}\rangle{\bs{\beta}}|\Y]=\langle \mb E[\boldsymbol{X}|\Y],{\bs{\beta}}\rangle{\bs{\beta}}=0~~\text{a.s.}
\end{align*}
This finishes the proof of Theorem $\ref{theorem, MDDO and conditional mean independence}$.
\end{proof}

% {\color{blue}\paragraph{Proof of Lemma \ref{lem:cov TX} (Repeated)}
% \begin{proof}
% For any $\u_1,\u_2\in\mc H_2$, we have
% \begin{align*}
% &\left\langle  T\mr{var}(\vX)T^*\u_1,\u_2  \right\rangle=\left\langle  T\mb E[\vX\otimes\vX]T^*\u_1,\u_2  \right\rangle
% =\left\langle  \mb E[\vX\otimes\vX]T^*\u_1,T^*\u_2  \right\rangle    
% \end{align*}
% since $\mb E[\vX]=0$. By the definition of convariance operator and expectation, we have 
% \begin{align*}
% \left\langle  \mb E[\vX\otimes\vX]T^*\u_1,T^*\u_2  \right\rangle=&\left\langle  \mb E[\left\langle\vX,  T^*\u_1 \right\rangle       \vX            ],T^*\u_2  \right\rangle
% =\mb E[  \left\langle\vX,  T^*\u_1 \right\rangle      \left\langle \vX            ,T^*\u_2  \right\rangle].
% \end{align*}
% Similarly, we have
% \begin{align*}
%  \left\langle  \mr{var}(T\vX)\u_1,\u_2  \right\rangle=\left\langle  \mb E[T\vX\otimes T\vX]\u_1,\u_2  \right\rangle=\mb E[  \left\langle T\vX,  \u_1 \right\rangle      \left\langle T\vX            ,\u_2  \right\rangle].\\    
% \end{align*}
% Then the proof is completed by noticing the following
% \begin{align*}
% \mb E[  \left\langle T\vX,  \u_1 \right\rangle      \left\langle T\vX            ,\u_2  \right\rangle]=\mb E[  \left\langle\vX,  T^*\u_1 \right\rangle      \left\langle \vX            ,T^*\u_2  \right\rangle].
% \end{align*}
% \end{proof}}



\section{Proof of Lemma \ref{lemma: SE=GammaS}}
Recall the following fact in FSIR.
\begin{lemma}\label{lemma, direct result of linearity condition}~\\
Under Assumption $\ref{as:Linearity condition and Coverage condition}~ \boldsymbol{\mathrm{{i)}}}$, we have $\mathcal S_{\mathbb E(\boldsymbol{X}|\Y)}\subseteq \Gamma \mc S_{\Y|\bs X}\subseteq \mc H$.
\end{lemma}
It is a trivial generalization of    \cite[Theorem 2.1]{ferre2003functional} from univariate response to multivariate response.
\paragraph{Proof of Lemma $\ref{lemma: SE=GammaS}$}
\begin{proof}
First, we prove that $\mathcal{S}_{\mathbb{E}(\bs X|\Y)}^\perp\subseteq \mathrm{Im}\{\mathrm{var(\mb{E}(\bs X|\Y))}\}^\perp$. For any ${\bs{\beta}}\in\mathcal{S}_{\mathbb{E}(\bs X|\Y)}^\perp$, one has $\langle{\bs{\beta}},\mb{E}(\bs X|\Y)\rangle=0$ a.s. Then for any $\bs\alpha\in\mathcal{H}$, one can get
\begin{align*}
\langle{\bs{\beta}},\mathrm{var}(\mb{E}(\bs X|\Y))\bs\alpha\rangle&=\langle{\bs{\beta}},\mb E\lmi\mb{E}(\bs X|\Y)\otimes \mb{E}(\bs X|\Y)\rmi\bs\alpha\rangle\\
&=\mb E\big(\langle\mb{E}(\bs X|\Y),\bs\alpha\rangle\langle{\bs{\beta}},\mb{E}(\bs X|\Y)\rangle\big)=0,
\end{align*}
which means that ${\bs{\beta}}\in\mathrm{Im}\{\mathrm{var}(\mb{E}(\bs X|\Y))\}^\perp$. Moreover, one has
\begin{align*}\mathcal{S}_{\mathbb{E}(\bs X|\Y)}^\perp\subseteq \mathrm{Im}\{\mathrm{var}(\mb{E}(\bs X|\Y))\}^\perp
%&\Rightarrow\left(\mathcal{S}_{\mathbb{E}(\bs X|Y)}^\perp\right)^\perp\supseteq \left(\mathrm{Im}\{\mathrm{var(\mb{E}(\bs X|Y))}\}^\perp\right)^\perp\\&
\Longrightarrow\overline{\mathcal{S}_{\mathbb{E}(\bs X|\Y)}}\supseteq\overline{\mathrm{Im}}\{\mathrm{var}(\mb{E}(\bs X|\Y))\}.
\end{align*}
Thus, $\overline{\mathrm{Im}}\{\mathrm{var}(\mb{E}(\bs X|\Y))\}\subseteq\overline{\mathcal{S}_{\mathbb{E}(\bs X|\Y)}}\subseteq\overline{\Gamma\mathcal{S}_{\Y|\bs X}}$ by Lemma $\ref{lemma, direct result of linearity condition}$. According to Assumption $\ref{as:Linearity condition and Coverage condition}$ \textbf{ii)}, one can get
\[\mathrm{dim}\left(\overline{\mathrm{Im}}\{\mathrm{var}(\mb{E}(\bs X|\Y))\}\right)=\mathrm{dim}\left(\overline{\mathcal{S}_{\mathbb{E}(\bs X|\Y)}}\right)=\mathrm{dim}(\overline{\Gamma\mathcal{S}_{\Y|\bs X}})=d.\]
One can complete the proof since finite dimension subspaces are closed.
\end{proof}

\section{Proof of Theorem \ref{theorem, MDDO and IRS}}
\begin{proof}
For convenience, we abbreviate $\mathrm{MDDO}(\boldsymbol{X}|\Y)$ to ${M}$. According to Theorem $\ref{theorem, MDDO and conditional mean independence}$ and Lemma $\ref{lem: Txx=0tuiTx=0}$, one can get
\begin{align*}{\bs{\beta}}\in\mathcal S_{\mb E(\boldsymbol{X}|\Y)}^\perp&\Longleftrightarrow\langle {\bs{\beta}},\mathbb E(\boldsymbol{X}|\Y)\rangle=0~~\text{a.s.}\Longleftrightarrow\mathbb E(\langle {\bs{\beta}},\boldsymbol{X}\rangle|\Y)=0~~\text{a.s.}\\
&\Longleftrightarrow\mathrm{MDDO}(\langle {\bs{\beta}},\boldsymbol{X}\rangle|\Y)=0\Longleftrightarrow\langle {M}{\bs{\beta}},{\bs{\beta}}\rangle=0\\
&\Longleftrightarrow{M}{\bs{\beta}}=0\Longleftrightarrow {\bs{\beta}}\in\mathrm{null}(M)=\overline{\mathrm{Im}}(M)^\perp,
\end{align*}
which means that $\mathcal S_{\mb E(\boldsymbol{X}|\Y)}^\perp=\overline{\mathrm{Im}}(M)^\perp$ and $\overline{\mathcal S_{\mb E(\boldsymbol{X}|\Y)}}=\overline{\mathrm{Im}}(M)$.
One can complete the proof since finite dimension subspaces are closed.
\end{proof}
\section{Proof of Lemma \ref{lemma, way of estimate truncate central subspace}}
Before proving Lemma $\ref{lemma, way of estimate truncate central subspace}$, we give the following lemma.
\begin{lemma}\label{lem: colPBP equal colPB operator}
Assume that $P$ is a bounded linear operator from a Hilbert space $\wt{\mc H}$ to itself and $B$ is a positive semi-definite operator from $\wt{\mc H}$ to itself. 
Then we have $\overline{\mathrm{Im}}(PBP^*)=\overline{\mathrm{Im}}(PB)$.
\end{lemma}
\begin{proof}
It suffices to show that $\mnull(BP^*)=\mnull(PBP^*)$. First, since $B$ is positive semi-definite, one has $\langle x,PBP^*x\rangle = \langle P^*x,BP^*x \rangle\geqslant 0~(\forall x\in\wt{\mc H})$. Thus $PBP^*$ is a positive semi-definite operator on $\wt{\H}$.
For any $y\in\wt{\H}$, we have 
\begin{align*}
PBP^*y=0\overset{(a)}{\Longleftrightarrow}\langle y,PBP^*y\rangle = \langle P^*y,BP^*y \rangle=0\overset{(b)}{\Longleftrightarrow} BP^*y=0. 
\end{align*}
where $(a)$ and $(b)$ come from Lemma $\ref{lem: Txx=0tuiTx=0}$.
Thus $\mnull(PBP^*)=\mnull(BP^*)$.
\end{proof}

\paragraph{Proof of Lemma $\ref{lemma, way of estimate truncate central subspace}$}
\begin{proof}
For convenience, we abbreviate $\mathrm{MDDO}(\boldsymbol{X}|\Y)$ and $\mathrm{MDDO}(\boldsymbol{X}^{(m)}|\Y)$ to ${M}$ and $M_m$ respectively. 

By Corollary $\ref{corollary, MDDO and central subspace}$, one can get $\Gamma\mathcal{S}_{\Y|\boldsymbol{X}}=\mathrm{Im}(M)$. Thus,
\begin{align}\label{eq: corollary, MDDO and central subspace}
\Pi_m\Gamma\mathcal{S}_{\Y|\boldsymbol{X}}=\Pi_m\mathrm{Im}(M)=\mathrm{Im}(\Pi_m M).
\end{align}
It is easy to check that
\begin{align}
\Gamma_m&:=\mathrm{var}(\bs X^{(m)})=\Pi_m\Gamma\Pi_m=\Pi_m\Gamma=\Gamma\Pi_m=\sum\limits_{i=1}^m\lambda_i\phi_i\otimes\phi_i.\label{eq: Gamma m def}
\end{align}
On the one hand, by the definition of $\mathcal{S}^{(m)}_{{\Y|\boldsymbol{X}}}$ and $\Gamma_m$ (see \eqref{def: truncated central subspace} and \eqref{eq: Gamma m def}), one can get
\begin{align}\label{eq:Pim Gamma S}
\Pi_m\Gamma\mathcal{S}_{\Y|\boldsymbol{X}}&=\Pi_m\Gamma\Pi_m\mathcal{S}_{\Y|\boldsymbol{X}}=(\Pi_m\Gamma)(\Pi_m\mathcal{S}_{\Y|\boldsymbol{X}})=\Gamma_m\mathcal{S}^{(m)}_{{\Y|\boldsymbol{X}}}.
\end{align}
On the other hand, one has $\overline{\mathrm{Im}}(\Pi_m M)=\overline{\mathrm{Im}}(\Pi_m M\Pi_m)$ by Lemma $\ref{lem: colPBP equal colPB operator}$. Since $\Pi_m M$ and $\Pi_m M\Pi_m$ are both of finite rank, one can further get
\begin{align*}
\mathrm{Im}(\Pi_m M)&=\overline{\mathrm{Im}}(\Pi_m M)=\overline{\mathrm{Im}}(\Pi_m M\Pi_m)=\mathrm{Im}(\Pi_mM\Pi_m).
\end{align*}
Then according to Theorem $\ref{theorem, MDDO and conditional mean independence}$(ii), one has
\begin{align}
\mathrm{Im}(\Pi_m M)=\mathrm{Im}(\Pi_mM\Pi_m)=\mathrm{Im}(M_m).\label{eq:Pim span M}
\end{align}
Combining \eqref{eq:Pim Gamma S}, \eqref{eq:Pim span M} with \eqref{eq: corollary, MDDO and central subspace}, one has $\Gamma_m\mathcal{S}^{(m)}_{{\Y|\boldsymbol{X}}}=\mathrm{Im}\{M_m\}$.
Finally, one can get $ \Gamma_m^\dagger\mathrm{Im}\{M_m\}=\Gamma_m^\dagger\Gamma_m\mathcal{S}^{(m)}_{{\Y|\boldsymbol{X}}}=\Pi_m\mathcal{S}^{(m)}_{{\Y|\boldsymbol{X}}}=\mathcal{S}^{(m)}_{{\Y|\boldsymbol{X}}}$.
\end{proof}

\section{Wely Inequality for a Self-adjoint and Compact Operator}\label{ap:Wely inequality for self-adjoint and compact operators}
First, we show the following three results in standard functional analysis textbook.
\begin{lemma}[Spectral theorem]\label{thm: Spectral theorem}Let $\wt{\mathcal{H}}$ be a Hilbert space and $A:\wt{\mc{H}}\to\wt{\mc{H}}$ be a compact, self-adjoint operator. There is an at most countable orthonormal basis $\{\wt e_j\}_{j\in J}$ ($J=\{1,\cdots,n\}$ or $\mathbb{Z}_{\geqslant1}$) of $\wt{\mathcal{H}}$ and eigenvalues $\{\wt\lambda_j\}_{j\in J}$ with $|\wt\lambda_1|\geqslant|\wt\lambda_2|\geqslant\cdots\geqslant0$ converging to zero, such that
\begin{align*}
x=\sum_{j\in J}\langle x,\wt e_j\rangle \wt e_j;\qquad Ax=\sum_{j\in J}\wt\lambda_j\langle x,\wt e_j\rangle \wt e_j,\qquad x \in\wt{\mathcal{H}}.
\end{align*}
\end{lemma}

\begin{lemma}[Rayleigh's principle]\label{lem:Rayleigh operator}Let $A$ be a compact, self-adjoint operator. If $\{\wt e_j\}_{j\in J}$ and $\{\wt\lambda_j\}_{j\in J}$ are eigenvectors and eigenvalues define in Lemma $\ref{thm: Spectral theorem}$ respectively. Then
\[|\wt\lambda_1|=\mathop{\sup\limits_{\|u\|=1}}|\langle Au,u\rangle|;\qquad|\wt\lambda_n|=\mathop{\sup\limits_{\|u\|=1}}_{u\in\{\wt e_1,\cdots,\wt e_{n-1}\}^\perp}|\langle Au,u\rangle|~(n\geqslant 2).\]
\end{lemma}
\begin{lemma}[Minimax theorem]\label{lem:minimax operator}
Assume that $A$ is a positive semi-definite and compact operator with its eigenvalues $\{\wt\lambda_i\}$ ordered as $\wt\lambda_1\geqslant\dots\geqslant \wt\lambda_n\geqslant\dots\geqslant 0$, then
$$
\wt\lambda_n=\inf_{E_{n-1}}\sup_{x\in E_{n-1}^\perp,\|x\|=1}\langle Ax,x\rangle
$$
where $E_{n-1}$ with dimension $n-1$ is a closed linear subspace of $\wt{\mc H}$.
\end{lemma}
Then we give the Wely inequality for a self-adjoint and compact operator.
\begin{proposition}\label{prop: wely operator}
Let $M=N+R$ where $M$, $N$ and $R$ are three self-adjoint and compact operators defined on a Hilbert space $\wt{\mc H}$. Also, $M$ and $N$ are positive semi-definite with their respective eigenvalues $\{\mu_i\},\{\nu_i\}$ ordered as follows
\begin{align*}
M:\mu_1\geqslant\dots\geqslant \mu_n\geqslant\dots\geqslant 0;\qquad
N:\nu_1\geqslant\dots\geqslant \nu_n\geqslant\dots\geqslant 0,
\end{align*}
while $R$'s eigenvalues are $\{\rho_i\}$ ordered as follows:
\[R:|\rho_1|\geqslant\dots\geqslant |\rho_n|\geqslant\dots\geqslant 0.\]
Then the following inequalities hold: $|\mu_k-\nu_k|\leqslant|\rho_1|=\|R\| $, $k\geqslant1$.
\end{proposition}
\begin{proof}
From Lemma $\ref{lem:minimax operator}$, we have:
\[\mu_n=\inf_{E_{n-1}}\sup_{x\in E_{n-1}^\perp,\|x\|=1}\langle Mx,x\rangle;\qquad\nu_n=\inf_{E_{n-1}}\sup_{x\in E_{n-1}^\perp,\|x\|=1}\langle Nx,x\rangle,\]
where $E_{n-1}$ with dimension $n-1$ is a closed linear subspace of $\wt{\mc H}$.
By Lemma $\ref{lem:Rayleigh operator}$, we have:
$$
\sup_{\|u\|=1}|\langle Ru,u\rangle|=|\rho_1|.
$$
Since $\langle Mu,u\rangle=\langle Nu,u\rangle+\langle Ru,u\rangle$, for any $\|u\|=1$, we have:
$$
\langle Nu,u\rangle-|\rho_1|\leqslant\langle Mu,u\rangle \leqslant \langle Nu,u\rangle+|\rho_1|.
$$
Then for any given $n-1$ dimensional closed linear subspace of $\wt{\mc H}$, we conclude
\begin{equation}\label{eq:max ineq}
\sup_{u\in E_{n-1}^\perp,\|u\|=1}\langle Nu,u\rangle-|\rho_1|\leqslant\sup_{u\in E_{n-1}^\perp,\|u\|=1}\langle Mu,u\rangle\leqslant \sup_{u\in E_{n-1}^\perp,\|u\|=1}\langle Nu,u\rangle+|\rho_1|.
\end{equation}
Take the infimum with respective to $E_{n-1}$ in \eqref{eq:max ineq}, we have
\[\nu_n-|\rho_1|\leqslant\mu_n\leqslant \nu_n+|\rho_1|\]
by Lemma $\ref{lem:minimax operator}$.
\end{proof}
The next result is a direct corollary of Proposition $\ref{prop: wely operator}$.
\begin{corollary}\label{coro:wely ineq operator}
Let $M$ and $N$ be two self-adjoint, positive semi-definite and compact operators defined on a Hilbert space $\wt{\mc H}$ with their respective eigenvalues $\{\mu_i\},\{\nu_i\}$ ordered as follows
\begin{align*}
M:\mu_1\geqslant\dots\geqslant \mu_n\geqslant\dots\geqslant 0\quad\text{and}\quad
N:\nu_1\geqslant\dots\geqslant \nu_n\geqslant\dots\geqslant 0.
\end{align*}
Then the following inequalities hold: $|\mu_k-\nu_k|\leqslant\|M-N\| $, $ k\geqslant1$.
\end{corollary}




\section{Proof of Proposition \ref{prop:bound hatMmd Mm}}
Before proving Proposition $\ref{prop:bound hatMmd Mm}$, we give the following conclusion, whose proof is deferred to the end of this section.
\begin{proposition}\label{proposition, concentration of MDDO}
Under Assumptions $\ref{as:joint distribution assumption}$ and $\ref{assumption: sub-Gaussian}$, for all $\gamma\in(0,1/2)$, there exist positive constants $D_0=D_0(\gamma,\sigma_0,\sigma_1)$, $D_1=D_1(\sigma_1)$, $D_2=D_2(\sigma_0,\sigma_1)$ and $n_0=n_0(\gamma,\sigma_0,\sigma_1)$ such that for all $n\geqslant n_0$ and
\[C\in \l D_0n^{\frac{2\gamma}{5}}-\ln\l D_1m^2n \r,D_2 n^{\frac{1}{5}}-\ln\l D_1m^2n \r \rmi,\]
we have
\begin{equation*}
\mathbb{P}\l\left\|\wh M_m- M_m\right\| <\l \frac{C+\ln( D_1m^2n)}{D_2}\r^{\frac52}\frac{12m}{\sqrt n}\r\geqslant 1-\exp(- C).
\end{equation*}
\end{proposition}
\paragraph{Proof of Proposition $\ref{prop:bound hatMmd Mm}$}
\begin{proof}




Using Corollary $\ref{coro:wely ineq operator}$, one can get
$
\lambda_i\l\wh M_m\r\leqslant \lno\wh M_m-M_m\rno +\lambda_i\l M_m\r
$. 
Since $\rank(M_m)=d$, one can get $\lambda_i(M_m)=0,~i\geqslant d+1$. Thus by Proposition $\ref{proposition, concentration of MDDO}$, one has
\begin{align}\label{eq:lambdai hat Mm upper bound}
\mathbb{P}\l\lambda_{d+1}(\wh M_m)<\l \frac{C+\ln\l D_1m^2n\r}{D_2}\r^{\frac52}\frac{12m}{\sqrt n}\r\geqslant 1-\exp(- C)\qquad(i\geq d+1). 
\end{align}
Notice that 
\begin{align*}\lno\wh M_m^d- M_m\rno &\leqslant\lno M_m-\wh M_m\rno +\lno\wh M_m-\wh M_m^d\rno ;\\
\lno\wh M_m-\wh M_m^d\rno &=\left\|\sum_{i=d+1}^\infty\wh\mu_i\wh\gamma_i\otimes \wh\gamma_i\right\| =\widehat{\lambda}_{d+1}=\lambda_{d+1}(\widehat{M}_m)
\end{align*}
by \eqref{wh M_m spectral decomposition}.
Then combing Proposition $\ref{proposition, concentration of MDDO}$ with \eqref{eq:lambdai hat Mm upper bound} can complete the proof.
\end{proof}


\paragraph{Proof of Proposition \ref{proposition, concentration of MDDO}}
\begin{proof}
Note that $\boldsymbol{X}^{(m)}=\sum\limits_{j=1}^m\langle \boldsymbol{X},\phi_j\rangle\phi_j$, then a simple calculation leads to
\begin{align*}
M_m&=-\sum_{i,j=1}^m\mathbb E\big[\langle \boldsymbol{X},\phi_i\rangle\langle \boldsymbol{X}',\phi_{j}\rangle\|\Y-\Y'\|\big]\phi_i\otimes\phi_j;\\
\widehat{M}_m&=-\sum_{i,j=1}^m\frac1{n^2}\sum_{k,\ell=1}^n\langle \boldsymbol{X}_k,\phi_i\rangle\langle \boldsymbol{X}_\ell,\phi_j\rangle\|\Y_k-\Y_\ell\|\phi_i\otimes\phi_j.
\end{align*}

For a operator $\Gamma'$ that can be expanded as $\Gamma':=\sum\limits_{i,j=1}^\infty a_{ij}\phi_i\otimes\phi_{j}$, let us define its maximal norm as $\|\Gamma'\|_{\mathrm{max}}=\sup\limits_{i,j}|a_{ij}|$.



\begin{lemma}\cite[Theorem 1]{mai2021slicing}\label{lemma, concentration of MDDOnm}
Under Assumptions $\ref{as:joint distribution assumption}$ and $\ref{assumption: sub-Gaussian}$, for all
$\gamma\in(0,1/2)$, there exist positive
constants $C_0=C_0(\gamma,\sigma_0,\sigma_1)$, $C_1=C_1(\sigma_1)$, $C_2 = C_2(\sigma_0;\sigma_1)$ and $n_0 = n_0(\gamma,\sigma_0,\sigma_1)$
such that for all $n\geqslant n_0$ and $\varepsilon\in(C_0 n^{-(1/2-\gamma)},1]$, we have
\begin{equation*}
\mathbb{P}\l\lno \widehat{M}_m-M_m\rno_{\max}>12\varepsilon\r\leqslant C_1 m^2n\exp\l- C_2\l\varepsilon^2 n\r^{1/5}\r.
\end{equation*}
\end{lemma}
\noindent Since $\lno\widehat{M}_m-M_m\rno \leqslant m\lno\widehat{M}_m-M_m\rno_{\mathrm{max}}$, one has
\begin{equation*}
\mathbb{P}\l\lno\widehat{M}_m-M_m\rno >12m\varepsilon\r\leqslant C_1 m^2n\exp\l-C_2\l\varepsilon^2 n\r^{1/5}\r.
\end{equation*}
Let $C=C_2\l\ve^2n\r^{1/5}-\ln\l C_1m^2n\r$ satisfying 
\begin{align*}
C\in\l C_2C_0^{2/5}n^{2\gamma/5}-\ln\l C_1m^2n\r,C_2n^{1/5}-\ln\l C_1m^2n\r\rmi,
\end{align*}
then one has
\begin{equation*}
\mathbb{P}\l\lno\widehat{M}_m-M_m\rno \leqslant\l \frac{C+\ln\l C_1m^2n\r}{C_2}\r^{\frac52}\frac{12m}{\sqrt{n}}\r>1- \exp(- C).
\end{equation*}
Then in order to complete the proof, one only need to choose $D_0$, $D_1$ and $D_2$ to be $C_2C_0^{2/5}$, $C_1$ and $C_2$ respectively. 
\end{proof}





\section{Properties of Sub-Gaussian Random Vectors}
We first review the definition of sub-Gaussian random variables.
\begin{definition}[Sub-Gaussian random variable and its upper-exponentially bounded constant]\label{def:sub gaussian variable}
A random variable $X$ is called a sub-Gaussian random variable if $X$ satisfies one of the following equivalent properties:
\begin{itemize}
 \item[1).] Tails. $\P(|X|>t)\leqslant \exp(1-t^{2}/K^{2}_{1})$ for any $t>0$;
 \item[2).] Moments. $\E[|X|^{p}]^{1/p}\leqslant K_{2}\sqrt{p}$ for any $p\geqslant 1$;
 \item[3).]Super-exponential moment: $\E[\exp(X^{2}/K^{2}_{3})]\leqslant \mr{e}$.

\noindent Moreover, if $\E[X]=0$, then the properties $1)-3)$ are also equivalent to the following one:
\item[4).] Moment generating function: $\E[\exp(tX)]\leqslant \exp(t^{2}K^{2}_{4})$ for all $t\in\R$.
\end{itemize}
Here $K_1$, $K_2$, $K_3$ and $K_4$ are four constants.
$K$ is called an upper-exponentially bounded constant of $X$ if 
$K\geqslant \max\{K_{1},K_{2},K_{3},K_{4}\}$.
\end{definition}
\begin{definition}[Sub-Gaussian random vector and its upper-exponentially bounded constant]\label{def,sub-Gaussian random vector,upper-exponentially bounded constant}
 ${X}\in\R^m$ is called a sub-Gaussian random vector if for all $x\in\R^m$, one-dimensional marginal $\langle{X},x\rangle$ is sub-Gaussian random variable. $K$ is called an upper-exponentially bounded constant of $X$ if $K$ satisfies:
 \begin{align*}
K\geqslant \sup_{x\in\mb{S}^{m-1}}K(\langle X,x\rangle) 
 \end{align*}
 where $K(\langle X,x\rangle)$ denotes an upper-exponentially bounded constant of $\langle X,x\rangle$.
Moreover, $K$ is called a uniform (about $m$) upper-exponentially bounded constant of $X$ if $K$ satisfies:
 \begin{align*}
K\geqslant \sup_m\sup_{x\in\mb{S}^{m-1}}K\l \langle X,x\rangle\r.
 \end{align*}
Furthermore, $X$ is called a uniform (about $m$) sun-Gaussian random vector.
 \end{definition}
The following is an application of sub-Gaussian random vectors.
\begin{lemma}[\citealt{vershynin2010introduction}]\label{lem:esgrm}
 Let $\M=[\bs m_1~\cdots~\bs m_n]$ be an $m\times n$ matrix ($n>m$) whose columns $\m_{i}$ are 
 independent centered sub-Gaussian random vectors with 
 covariance matrix $\mathbf{I}_{m}$. Let $\sigma^{+}_{\min}(\M)$ and $\sigma_{\max}(\M)$ be the infimum and supremum of positive singular values of $\M$ respectively. Then, for any $t>0$, with probability at least $1-2\exp(- C^{\prime}t^{2})$, we have
 \begin{equation*}
 \sqrt{n}-C_0\sqrt{m}-t\leqslant \sigma^{+}_{\min}(\M)\leqslant \sigma_{\max}(\M)\leqslant \sqrt{n}+C_0\sqrt{m}+t
 \end{equation*}
 where $C'$ and $C_0$ are two positive constants depending only on $K(\bs m_1)$:
 the upper-exponentially bounded constant of $\bs m_1$.
\end{lemma}
\noindent Let $t=\sqrt m$, then one can easily get
\begin{align}\label{equation, min max eval}
\begin{split}
\lambda_{\max}\left(\frac1n \M\M^\top\right)\leqslant \left(1+\frac{(C_0+1)\sqrt m}{\sqrt n}\right)^2;\\
\lambda_{\min}^+\left(\frac1n \M\M^\top\right)\geqslant \left(1-\frac{(C_0+1)\sqrt m}{\sqrt n}\right)^2, 
\end{split}
\end{align}
with probability at least $1-2\exp(- C'm)$ where $\lambda^{+}_{\min}(\cdot)$ and $\lambda_{\max}(\cdot)$ stands for the infimum and supremum of the positive spectrum respectively.



\begin{lemma}\label{lemma, estiamtion error of inverse sample cov}
Assume that $\x_1,\x_2,...,\x_n$ are $n$ i.i.d. samples from an $m$-dimensional centered sub-Gaussian vector with an invertible covariance matrix $\Sigma$. Let $\wh\Sigma:=\frac1n\sum_i \x_i\x_i^\top$.
Then there exists a positive constant $n_1'=n_1'(K(\bs m_1),c_1)$ ($c_1$ is defined in \eqref{eq: m n relationship}), such that when $n\geqslant n_1'$, we have
\begin{align*}
\lno\wh{\Sigma}-\Sigma\rno\hspace{-1.5mm}&\leqslant (C_0+2)^2\lambda_{\max}(\Sigma)\sqrt{\frac mn}~~\text{and}~~ \lno\wh{\Sigma}^{-1}-\Sigma^{-1}\rno\hspace{-1.5mm}\leqslant \frac{4(C_0+2)^2}{\lambda_{\min}(\Sigma)}\sqrt{\frac mn},
 \end{align*}
 with probability at least $1-2\exp(- C'm)$, where $C_0$ is defined in Lemma $\ref{lem:esgrm}$.
\end{lemma}
\begin{proof}
Let $\x_i=\Sigma^{\frac12}\m_i$ and $\bs{M}=[\bs m_1~\cdots~\bs m_n]$ where $\m_i$ is a centered sub-Gaussian random vector with covariance $\mathbf I_m$. Then one has 
\begin{align*}
\lno\wh\Sigma-\Sigma\rno&\leqslant\lno\Sigma^{\frac12}\rno\cdot\left\|\frac1n \M\M^\top-\mathbf I\right\|\cdot\lno\Sigma^{\frac12}\rno\\
&= \lambda_{\max}(\Sigma)\cdot\left[\lambda_{\max}\left(\frac1n \M\M^\top\right)-1\right]
\end{align*}
and 
\begin{align*}
\lno\wh{\Sigma}^{- 1}-\Sigma^{- 1}\rno
&\leqslant \lno\Sigma^{-\frac12}\rno\cdot\left\|\frac1n \M\M^\top-\mathbf I\right\|\cdot\lno\l\frac1n \M\M^\top\r^{-1}\rno\cdot\lno\Sigma^{-\frac12}\rno\\
&=\frac{1}{\lambda_{\min}(\Sigma)}\left[\lambda_{\max}\left(\frac1n \M\M^\top\right)-1\right]\cdot\lambda_{\min}\left(\frac1n \M\M^\top\right)^{-1}.
\end{align*}
By \eqref{equation, min max eval}, it is easy to check that
\begin{align*}&\lambda_{\max}\left(\frac1n \M\M^\top\right)-1\leqslant\left(1+\frac{(C_0+1)\sqrt m}{\sqrt n}\right)^2-1\leqslant\frac{(C_0+2)^2\sqrt m}{\sqrt n};\\
&\lambda_{\min}\left(\frac1n \M\M^\top\right)\geqslant \left(1-\frac{(C_0+1)\sqrt m}{\sqrt n}\right)^2\geqslant \frac14~\text{for}~n\geqslant [2(C_0+1)]^{\frac2{1-c_1}},
\end{align*}
with probability at least $1-2\exp(- C'm)$. Thus the proof is completed by choosing $n_1'(C_0,c_1):=[2(C_0+1)]^{\frac{2}{1-c_1}}$. 
\end{proof}

\section{Proof of Proposition \ref{prop:concentration Gammam dag Mmd}}\label{ap:concentration inequality}
We first give the following lemma whose proof is deferred to the end of this section.
\begin{lemma}\label{lem:PimTPimtoT}If $T$ is of finite rank, then we have $\lim\limits_{m\to \infty}\|\Pi_m T\Pi_m-T\| =0$.
\end{lemma}
A direct corollary of this lemma is as follows.
\begin{corollary}\label{lemma, M go to Mm}
%For any $\varepsilon>0$, one has $\|M-M_m\| <\varepsilon$ when $m$ is sufficiently large.
Under Assumptions $\ref{as:joint distribution assumption}$ and $\ref{as:Linearity condition and Coverage condition}$, we have $\lim\limits_{m\to\infty}\|M-M_m\| =0$.
\end{corollary}
\noindent We denote by $m_M(\varepsilon)$ the minimal integer $m_M$ satisfying $\|M-M_m\| \leqslant \varepsilon$ for all $m\geqslant m_M$.

Proposition $\ref{prop:concentration Gammam dag Mmd}$ is a direct corollary of the following Proposition.
\begin{proposition}
\label{prop:bound of finite estimate}
 Suppose that Assumptions $\ref{as:joint distribution assumption}$ to $\ref{assumption: rate-type condition}$ hold, then $\forall \gamma\in(0,1/2)$, there exist positive constants
 \begin{align*}
 n_1=n_1(\gamma,\sigma_0,\sigma_1,\bs K,m_M(1),c_1),\quad D_3=D_3(\|M\| ,\wt C,\bs K) 
 \end{align*}
and $C'=C'(\bs K)$
, such that when $n\geqslant n_1$, we have
\begin{equation*}
\begin{aligned}
\mb P\l \lno\widehat\Gamma_m^\dagger \widehat M_m^d-\Gamma_m^\dagger M_m\rno  \leqslant \left[\frac{C+\ln(D_1m^2n)}{D_2}\right]^{\frac52}\frac{24m^{\alpha_1+1}}{\wt C\sqrt n}+D_3\frac{m^{(2\alpha_1+1)/2}}{n^{1/2}} \r&\\
\geqslant 1-\exp(- C)-2\exp(- C'm).&
\end{aligned}
\end{equation*}
Here $D_1,D_2$ and $C$ are defined in Proposition $\ref{prop:bound hatMmd Mm}$ and $\bs K$ is the uniform upper-exponentially bounded constant of $(\sqrt{\lambda_1}w_1,\dots,\sqrt{\lambda_m}w_m)$. 
\end{proposition}
\begin{proof}
By triangle inequality, one has
\begin{align*}
&\lno\widehat{\Gamma}_m^\dagger \widehat M_m^d-\Gamma_m^\dagger M_m\rno 
=\lno\widehat\Gamma_m^\dagger \widehat M_m^d-\wh\Gamma_m^\dagger M_m+\wh\Gamma_m^\dagger M_m-\Gamma_m^\dagger M_m\rno 
\\&\qquad\leqslant\lno\Gamma_m^\dagger\rno \cdot \lno\widehat M_m^d-M_m\rno +\lno\widehat\Gamma_m^\dagger-\Gamma_m^\dagger\rno \cdot \lno M_m\rno .
\end{align*}
Thus one can bound $\lno\Gamma_m^{\dag}M_m-\widehat\Gamma_m^{\dag}\widehat M_m^d\rno $ by bound $\lno\Gamma_m^\dagger\rno $, $\lno\widehat\Gamma_m^\dagger-\Gamma_m^\dagger\rno $, $\lno\widehat M_m^d-M_m\rno $ and $\lno M_m\rno $ respectively.
\begin{itemize}
 \item\textbf{Bound of $\lno\Gamma_m^\dagger\rno $}: By Assumption $\ref{assumption: rate-type condition}$, one has 
\begin{align}\label{eq:bound Gammam dagger}
\lambda_j\geqslant \wt C j^{-\alpha_1}\Rightarrow\lno\Gamma_m^\dagger\rno =\lambda_m^{-1}\leqslant \wt{C}^{-1} m^{\alpha_1}. 
\end{align} 
 \item\textbf{Bound of $\lno\widehat\Gamma_m^\dagger-\Gamma_m^\dagger\rno $}:
 Let us define $\mc H_m:=\mathrm{span}\{\phi_1,\dots,\phi_m\}$ where $\{\phi_i\}$ is introduced in Equation $\eqref{eq:X expansion}$. It is easy to check that
$\lno\widehat\Gamma_m^\dagger-\Gamma_m^\dagger\rno =\lno(\widehat\Gamma_m^\dagger-\Gamma_m^\dagger)|_{\mc H_m}\rno $ since $\l\widehat\Gamma_m^\dagger-\Gamma_m^\dagger\r{\bs{\beta}}=0$ for any ${\bs{\beta}}\in\mc{H}_m^\perp$. 
Because $\l\widehat\Gamma_m^\dagger-\Gamma_m^\dagger\r|_{\mc H_m}$ can be represented by matrix $\widehat{\Sigma}^{-1}-\Sigma^{-1}$ defined in Lemma $\ref{lemma, estiamtion error of inverse sample cov}$ under orthonormal basis $\{\phi_i\}_{i=1}^m$, one can get $\lno\widehat\Gamma_m^\dagger-\Gamma_m^\dagger\rno =\|\widehat{\Sigma}^{-1}-\Sigma^{-1}\|$.
Similarly, one can also get $\lno\Gamma_m^\dagger\rno =\lno\Sigma^{-1}\rno=\lambda_{\min}^{-1}(\Sigma)$. Thus, by Lemma $\ref{lemma, estiamtion error of inverse sample cov}$ one has
\[\mb P\l\lno\widehat\Gamma_m^\dagger-\Gamma_m^\dagger\rno \leqslant {4(C_0+2)^2}\lno\Gamma^{\dag}_m\rno \sqrt{\frac mn}\r\geqslant 1-2\exp(- C'm)\]
for sufficiently large $n\geqslant n_1'(\bs K,c_1)$
. Combing with $\lno\Gamma_m^\dagger\rno \hspace{-1mm}\leqslant \wt{C}^{-1} m^{\alpha_1}$, one can get
\begin{equation}\label{eq: distance hat gamma m dagger hat gamma m dagger}
\mb P\l\lno\widehat\Gamma_m^\dagger-\Gamma_m^\dagger\rno \leqslant \frac{4(C_0+2)^2m^{(2\alpha_1+1)/2}}{\wt Cn^{1/2}}\r\geqslant 1-2\exp(- C'm)
\end{equation}
for sufficiently large $n\geqslant n_1'(\bs K,c_1)$.
 \item\textbf{Bound of $\lno\widehat M_m^d-M_m\rno $}:
 See Proposition $\ref{prop:bound hatMmd Mm}$.
 \item \textbf{Bound of $\lno M_m\rno $}: By Corollary $\ref{lemma, M go to Mm}$, $\|M-M_m\| \leqslant 1$ for sufficiently large $m\geqslant m_M(1)$. Then by triangle inequality, one can get
\[\|M_m\| -\|M\| \leqslant \|M-M_m\| \leqslant 1.\]
Hence,
\begin{align}\label{eq:Mm leq M C}
\|M_m\| \leqslant \|M\| +1.
\end{align}
\end{itemize}
Combing \eqref{eq:bound Gammam dagger}, \eqref{eq: distance hat gamma m dagger hat gamma m dagger}, Proposition $\ref{prop:bound hatMmd Mm}$ with \eqref{eq:Mm leq M C}, one can choose $D_3$ and $n_1$ to be $\frac{4(C_0+2)^2(\|M\| +1)}{\wt C}$ and $\max\{n_0,n_1'(\bs K,c_1),m_M(1)^{1/c_1}\}$ respectively to
complete the proof where $n_0$ is defined in Proposition $\ref{prop:bound hatMmd Mm}$.
\end{proof}

\paragraph{Proof of Lemma \ref{lem:PimTPimtoT}}
\begin{proof}By the triangle inequality and compatibility of operator norm, one has
\begin{align*}
\|\Pi_m T\Pi_m-T\| &\leqslant\|\Pi_mT\Pi_m-\Pi_mT\| +\|\Pi_mT-T\| \\
&\leqslant\|(\Pi_m-I)T^*\| +\|(\Pi_m-I)T\| 
\end{align*}
where $I=\sum\limits_{i=1}^\infty\phi_i\otimes\phi_i$ for $\{\phi_i\}_{i\in\mb{Z}_{\geqslant 1}}$ defined in \eqref{eq:X expansion} being an orthonormal basis of $\mc H$. 
% Since the adjoint of $M(\Pi_m-I)$ is $(\Pi_m-I)M$, we have
% \begin{align*}&\|M(\Pi_m-I)\| +\|(\Pi_m-I)M\| \\
% =&
% \end{align*}

Since $T$ is of finite rank, let us assume that $\{e_i\}_{i=1}^k$ is an orthonormal basis of $\mathrm{Im}(T)$ where $k=\mr{rank}(T)$. For any ${\bs{\beta}}\in\mathcal{H}$ such that $\|{\bs{\beta}}\|=1$, one has $\|T{\bs{\beta}}\|\leqslant\|T\| \|{\bs{\beta}}\|=\|T\| $, so one can assume that $T{\bs{\beta}}\in\mathrm{Im}(T)$ admits the following expansion under basis $\{e_i\}_{i=1}^k$:
\[T{\bs{\beta}}=\sum_{i=1}^k b_ie_i,\quad \sum_{i=1}^k b^2_i\leqslant\|T\| ^2<\infty.\]
Thus
\[\|(I-\Pi_m)T{\bs{\beta}}\|=\left\|\sum_{i=1}^k(I-\Pi_m) b_ie_i\right\|\leqslant\sum_{i=1}^k |b_i|\cdot\|(I-\Pi_m) e_i\|.\]
Clearly, $\|(\Pi_m-I)\alpha\|~(\forall\alpha\in\H)$ tends to $0$ as $m\to\infty$ since 
\[(I-\Pi_m)\alpha=\left(\sum_{i={m+1}}^\infty\phi_i\otimes\phi_i\right)\left(\sum\limits_{i=1}^\infty c_i\phi_i\right)=\sum_{i=m+1}^\infty c_i\phi_i\xrightarrow{m\to\infty} 0\]
where we have assumed that $\alpha=\sum\limits_{i=1}^\infty c_i\phi_i$ .

Thus $\forall\varepsilon>0$, there exists some $N_i>0$ such that $\forall m> N_i$ one has $\|(\Pi_m-I)e_i\|<\varepsilon$, $(\forall i=1,...,k)$. Let $N=\max\{N_1,\cdots,N_k\}$, then $\forall m>N$ one has
\[\|(I-\Pi_m)T{\bs{\beta}}\|\leqslant\sum_{i=1}^k |b_i|\cdot\|(I-\Pi_m) e_i\|\leqslant\sum_{i=1}^k |b_i|\varepsilon\leqslant k\varepsilon\|T\| ,\]
which means that $\forall m>N$, one has
\begin{align*}
\|(\Pi_m-I)T\| &=\sup_{\|{\bs{\beta}}\|=1}\|(\Pi_m-I)T{\bs{\beta}}\|\leqslant k\varepsilon\|T\| . 
\end{align*}
Thus $\lim\limits_{m\to\infty}\|(\Pi_m-I)T\| =0$. 

Similarly, one can also get $\lim\limits_{m\to\infty}\|(\Pi_m-I)T^*\| =0$. Then the proof of Lemma $\ref{lem:PimTPimtoT}$ is completed.
\end{proof}
% \section{Sin Theta Theorem}\label{ap:Sin Theta theorem}
% \subsection{Sin Theta Theorem for Self-adjoint Operators}
% \begin{lemma}[Proposition 2.3 in \cite{seelmann2014notes}]\label{lemma, sin theta of infinite dimension operator}
% Let $B$ be a self-adjoint operator on a separable Hilbert space $\widetilde{\mathcal{H}}$, and let ${V}\in\mathcal{L}(\widetilde{\mathcal{H}})$ be another self-adjoint operator where $\mathcal{L}\l\widetilde{\mc H}\r$ stands for the space of bounded linear operators from a Hilbert space $\widetilde{\mc H}$ to $\widetilde{\mc H}$.
% Write \[\mathrm{spec}( B)=\sigma\cup\Sigma\quad\text{and}\quad \mathrm{spec}( B+ V)=\omega\cup\Omega
% \]
% with $\sigma\cap\Sigma=\varnothing=\omega\cap\Omega$, and suppose that there is $\widehat d>0$ such that
% \[\mathrm{dist}(\sigma,\Omega)\geqslant \widehat d\quad\text{and}\quad\mathrm{dist}(\Sigma,\omega)\geqslant \wh d\]
% where $\mathrm dist(\sigma,\Sigma):=\min\{|a-b|:a\in\sigma,b\in\Omega\}$.
% Then, the operator angle $\Theta=\Theta(P_{ B}(\sigma),P_{ B+ V}(\omega))$ satisfies the bound
% \[\|\sin\Theta\|:=\|P_{{B}}(\sigma)-P_{{B}+{V}}(\omega)\| \leqslant\frac\pi2\frac{\| V\| }{\wh d}\]
% where $P_{ B}(\sigma)$ denotes the spectral projection for $ B$ associated with $\sigma$, i.e., 
% \[P_{B}(\sigma):=\frac{1}{2\pi\mathrm{i}}\oint_{\gamma}\frac{\mathrm{d}z}{z-B},\]
% where $\gamma$ is a contour on $\mathbb{C}$ that encloses $\sigma$ but no other elements of $\mathrm{spec}( B)$.
% \end{lemma}
% \begin{remark}
% We note that, 
% if further $ B$ is compact, 
% the spectral projection coincide with projection operator onto the closure of the space spanned by the eigenfunctions associated with the eigenvalues in $\sigma$.

% If $B$ is compact, by the spectral decomposition theorem one has
% \[B=\sum_{i=1}^\infty\mu_ie_i\otimes e_i\quad\text{and}\quad(z- B)^{-1}=\sum_{i=1}^\infty(z-\mu_i)^{-1}e_i\otimes e_i,\]
% where $\mr{spec}(B):=\{\mu_i\}_{i=1}^\infty$ satisfies $|\mu_i|\xrightarrow{i\to\infty} 0$.
% Then $\forall v\in \mathcal{H}$,
% \begin{align*}P_{B}(\sigma)v&=\frac{1}{2\pi\mathrm{i}}\oint_{\gamma}({z-B})^{-1}v~{\mathrm{d}z}=\frac{1}{2\pi\mathrm{i}}\oint_{\gamma}\sum_{i=1}^\infty(z-\mu_i)^{- 1}\langle e_i,v\rangle e_i~{\mathrm{d}z}\\
% &=\sum_{i=1}^\infty\left[\left(\frac{1}{2\pi\mathrm{i}}\oint_{\gamma}(z-\mu_i)^{-1}~{\mathrm{d}z}\right)\langle e_i,v\rangle e_i\right]=\sum_{i\in\{i:\mu_i\in\sigma\}}\langle e_i,v\rangle e_i.
% \end{align*}
% Especially, if $\sigma=\mr{spec}(B)\backslash\{0\}$, then $P_{B}(\sigma)$ is the projection operator onto the $\overline{\mathrm{Im}}(B)$.
% \end{remark}
% Splitting eigenvalues into nonzero part and zero part yields the following useful corollary.
% \begin{corollary}\label{cor: sin theta self adjoint}
% Let $B$ and $B'$ be two positive semi-definite {and compact} operators with finite rank on a separable Hilbert space $\widetilde{\mathcal{H}}$. Let $\lambda_{\min}^+( B)$ and $\lambda_{\min}^+(B')$ be the infimum of the positive eigenvalues of ${B}$ and ${B}'$ respectively. Then we have
% \[\left\|P_{ B}-P_{ B'}\right\| \leqslant\frac\pi2\frac{\| B- B'\| }{\min\{\lambda_{\min}^+( B),\lambda_{\min}^+( B')\}}.\]
% \end{corollary}
% \subsection{Sin Theta Theorem for General Operators}
% When ${B}$ and ${V}$ in Lemma $\ref{lemma, sin theta of infinite dimension operator}$ are not self-adjoint, we use the symmetrization trick, which mainly depends on the following Lemma.
% \begin{lemma}\label{lem:projection equality}
% $P_A=P_{AA^*}$ for any bounded linear operator $A$ from a Hilbert space $\wt\H$ to $\wt\H$. Especially, $P_A=P_{AA^{\top}}$ for any matrix $A$.
% \end{lemma}
% \begin{proof}This lemma is a direct corollary of Lemma $\ref{lem: colPBP equal colPB operator}$.
% \end{proof}

% Then we have the following Sin Theta theorem for general operator.
% \begin{lemma}\label{lemma, sin theta of nonadjoint operator}
% Let $ B,B'\in\mathcal{L}(\widetilde{\mathcal{H}})$ be two compact operators (not necessarily self-adjoint) with finite rank.
% Then we have
% \begin{align*}
% \left\|P_{ B}-P_{ B'}\right\| &\leqslant\frac\pi2\frac{\| B B^*- B'B'^*\| }{\min\lb\sigma_{\min}^+( B)^2,\sigma_{\min}^+(B')^2\rb}\\
% &\leqslant \frac\pi2\frac{\| B- B'\| ^2+2\| B- B'\| \| B'\| }{\min\lb\sigma_{\min}^+( B)^2,\sigma_{\min}^+( B')^2\rb}.
% \end{align*}
% \end{lemma}
% \begin{proof}By Lemma $\ref{lem:projection equality}$, one can get $\left\|P_{ B}-P_{ B'}\right\| =\left\|P_{ B B^*}-P_{ B' B'^*}\right\| $.
% Since $ BB^*, B'B'^*$ are both self-adjoint and compact, by Lemma $\ref{cor: sin theta self adjoint}$, one has
% \begin{align*}
% \left\|P_{ B B^*}-P_{ B' B'^*}\right\| \leqslant \frac{\pi}{2}\frac{\| B B^*- B' B'^*\| }{\min\lb\lambda_{\min}^+\l B B^*\r,\lambda_{\min}^+\l B' B'^*\r\rb}.
% \end{align*}
% Then the proof is completed in view of the following inequality:
% % of $\| B B^*- B' B'^*\| $:
% \begin{align}
% \lno B B^*- B' B'^*\rno &= \|( B- B')( B- B')^*\hspace{-0.5mm}+\hspace{-0.5mm}( B-B')(B')^*\hspace{-0.5mm}+\hspace{-0.5mm} B'( B- B')^*\| \nonumber\\
% &\leqslant \| B- B'\| ^2+2\| B- B'\| \| B'\| . \label{eq:sy ineq}
% \end{align}
% \end{proof}


\section{Sin Theta Theorem}\label{ap:Sin Theta theorem}
\subsection{Sin Theta Theorem for Self-adjoint Operators}
\begin{lemma}[Proposition 2.3 in \cite{seelmann2014notes}]\label{lemma, sin theta of infinite dimension operator}
Let $B$ be a self-adjoint operator on a separable Hilbert space $\widetilde{\mathcal{H}}$, and let ${V}\in\mathcal{L}(\widetilde{\mathcal{H}})$ be another self-adjoint operator where $\mathcal{L}\left(\widetilde{\mc H}\right)$ stands for the space of bounded linear operators from a Hilbert space $\widetilde{\mc H}$ to $\widetilde{\mc H}$.
Write the spectra of $B$ and $B+V$ as \[\mathrm{spec}( B)=\sigma\cup\Sigma\quad\text{and}\quad \mathrm{spec}( B+ V)=\omega\cup\Omega
\]
with $\sigma\cap\Sigma=\varnothing=\omega\cap\Omega$, and suppose that there is $\widehat d>0$ such that
\[\mathrm{dist}(\sigma,\Omega)\geqslant \widehat d\quad\text{and}\quad\mathrm{dist}(\Sigma,\omega)\geqslant \wh d\]
where $\mathrm dist(\sigma,\Sigma):=\min\{|a-b|:a\in\sigma,b\in\Omega\}$.
Then it holds that
\[\|P_{{B}}(\sigma)-P_{{B}+{V}}(\omega)\| \leqslant\frac\pi2\frac{\| V\| }{\wh d}\]
where $P_{ B}(\sigma)$ denotes the spectral projection for $ B$ associated with $\sigma$, i.e., 
\[P_{B}(\sigma):=\frac{1}{2\pi\mathrm{i}}\oint_{\gamma}\frac{\mathrm{d}z}{z-B},\]
where $\gamma$ is a contour on $\mathbb{C}$ that encloses $\sigma$ but no other elements of $\mathrm{spec}( B)$.
\end{lemma}
\begin{remark}
We note that, 
if further $ B$ is compact, 
the spectral projection coincide with projection operator onto the closure of the space spanned by the eigenfunctions associated with the eigenvalues in $\sigma$. 
% For more details, see, e.g., Remark 1 in \cite{chen2023optimality}.

Specifically, if $B$ is compact, by the spectral decomposition theorem one has
\[B=\sum_{i=1}^\infty\mu_ie_i\otimes e_i\quad\text{and}\quad(z- B)^{-1}=\sum_{i=1}^\infty(z-\mu_i)^{-1}e_i\otimes e_i,\]
where $\mr{spec}(B):=\{\mu_i\}_{i=1}^\infty$ satisfies $|\mu_i|\xrightarrow{i\to\infty} 0$.
Then $\forall v\in \mathcal{H}$, it holds that
\begin{align*}P_{B}(\sigma)v&=\frac{1}{2\pi\mathrm{i}}\oint_{\gamma}({z-B})^{-1}v~{\mathrm{d}z}=\frac{1}{2\pi\mathrm{i}}\oint_{\gamma}\sum_{i=1}^\infty(z-\mu_i)^{- 1}\langle e_i,v\rangle e_i~{\mathrm{d}z}\\
&=\sum_{i=1}^\infty\left[\left(\frac{1}{2\pi\mathrm{i}}\oint_{\gamma}(z-\mu_i)^{-1}~{\mathrm{d}z}\right)\langle e_i,v\rangle e_i\right]=\sum_{i\in\{i:\mu_i\in\sigma\}}\langle e_i,v\rangle e_i.
\end{align*}
In particular, if $\sigma=\mr{spec}(B)\backslash\{0\}$, then $P_{B}(\sigma)$ is the projection operator onto the $\overline{\mathrm{Im}}(B)$.
\end{remark}

Splitting eigenvalues into nonzero part and zero part yields the following useful corollary.
\begin{corollary}\label{cor: sin theta self adjoint}
Let $B$ and $B'$ be two positive semi-definite {and compact} operators with finite rank on a separable Hilbert space $\widetilde{\mathcal{H}}$. Let $\lambda_{\min}^+( B)$ and $\lambda_{\min}^+(B')$ be the infimum of the positive eigenvalues of ${B}$ and ${B}'$ respectively. Then we have
\[\left\|P_{ B}-P_{ B'}\right\| \leqslant\frac\pi2\frac{\| B- B'\| }{\min\{\lambda_{\min}^+( B),\lambda_{\min}^+( B')\}}.\]
\end{corollary}
\subsection{Sin Theta Theorem for General Operators}
When ${B}$ and ${V}$ in Lemma $\ref{lemma, sin theta of infinite dimension operator}$ are not self-adjoint, we use the symmetrization trick, which mainly depends on the following Lemma.
\begin{lemma}\label{lem:projection equality}
$P_A=P_{AA^*}$ for any bounded linear operator $A$ from a Hilbert space $\wt\H$ to $\wt\H$. Especially, $P_A=P_{AA^{\top}}$ for any matrix $A$.
\end{lemma}
\begin{proof}First we show that the null space of  $A^*$ is the same as the null space of $AA^*$.
On the one hand, 
\[x\in\mathrm{null}(A^*)\Longrightarrow
A^*x=0\Longrightarrow AA^*x=0\Longrightarrow x\in\mathrm{null}(AA^*); 
\]
One the other hand,
\begin{align*}x\in\mathrm{null}(AA^*)&\Longrightarrow
AA^*x=0\Longrightarrow \langle x,AA^*x\rangle=\langle A^*x,A^*x\rangle=\|A^*x\|^2=0\\
&\Longrightarrow A^*x=0\Longrightarrow x\in\mathrm{null}(A^*).
\end{align*}
Hence, we have $\mathrm{null}(A^*)=\mathrm{null}(AA^*)$. Take the orthogonal complement of the both sides of this equality, we can get
\[\mathrm{null}(A^*)^{\perp}=\mathrm{null}(AA^*)^{\perp}\Longrightarrow {\mathrm{Im}(A)}={\mathrm{Im}(AA^*)}.\]
\end{proof}
Then we have the following Sin Theta theorem for general operator.
\begin{lemma}\label{lemma, sin theta of nonadjoint operator}
Let $ B,B'\in\mathcal{L}(\widetilde{\mathcal{H}})$ be two compact operators (not necessarily self-adjoint) with finite rank.
Then we have
\begin{align*}
\left\|P_{ B}-P_{ B'}\right\| &\leqslant\frac\pi2\frac{\| B B^*- B'B'^*\| }{\min\left\{\sigma_{\min}^+( B)^2,\sigma_{\min}^+(B')^2\right\}}\\
&\leqslant \frac\pi2\frac{\| B- B'\| ^2+2\| B- B'\| \| B'\| }{\min\left\{\sigma_{\min}^+( B)^2,\sigma_{\min}^+( B')^2\right\}}.
\end{align*}
\end{lemma}
\begin{proof}By Lemma $\ref{lem:projection equality}$, one can get $\left\|P_{ B}-P_{ B'}\right\| =\left\|P_{ B B^*}-P_{ B' B'^*}\right\| $.
Since $ BB^*, B'B'^*$ are both self-adjoint and compact, by Lemma $\ref{cor: sin theta self adjoint}$, one has
\begin{align*}
\left\|P_{ B B^*}-P_{ B' B'^*}\right\| \leqslant \frac{\pi}{2}\frac{\| B B^*- B' B'^*\| }{\min\left\{\lambda_{\min}^+\left( B B^*\right),\lambda_{\min}^+\left( B' B'^*\right)\right\}}.
\end{align*}
Then the proof is completed in view of the following inequality:
% of $\| B B^*- B' B'^*\| $:
\begin{align}
\left\| B B^*- B' B'^*\right\| &= \|( B- B')( B- B')^*\hspace{-0.5mm}+\hspace{-0.5mm}( B-B')(B')^*\hspace{-0.5mm}+\hspace{-0.5mm} B'( B- B')^*\| \nonumber\\
&\leqslant \| B- B'\| ^2+2\| B- B'\| \| B'\| . \label{eq:sy ineq}
\end{align}
\end{proof}


\section{Proof of Theorem \ref{theorem, total convergence rate}}
Thanks to the triangle inequality, one can bound the subspace estimation error by bounding the error term (i): $\mathbf{ Loss}_1:=\left\|P_{\mc S_{\Y|\X}^{(m)}}-P_{ \widehat {\mc S}_{\Y|\X}^{(m)}}\right\| $ and error term (ii): $\mathbf{ Loss}_2:= \left\|P_{\mathcal S_{\Y|\boldsymbol{X}}}-P_{\mathcal S_{\Y|\boldsymbol{X}}^{(m)}}\right\| $ respectively.
\subsection{Upper bound of error term (i)}
We first give the following lemmas, whose proofs are all deferred to the end of this section.
\begin{lemma}\label{lem:Gammam dagger Mm uniformly bounded}
% Under Assumptions $\ref{as:joint distribution assumption}$ and $\ref{as:Linearity condition and Coverage condition}$,
% $\{\|\Gamma_m^\dagger M_m\| \}_{m=1}^\infty$ is uniformly (about $m$) bounded by $\|\Gamma^{-1}M\| $.
Under Assumptions $\ref{as:joint distribution assumption}$ and $\ref{as:Linearity condition and Coverage condition}$, it holds that $\|\Gamma_m^\dagger M_m\| \leq \|\Gamma^{-1}M\| (\forall m).$
% \begin{align*}
% \|\Gamma_m^\dagger M_m\| \leq \|\Gamma^{-1}M\| \quad\forall m.
% \end{align*}
% $\{\|\Gamma_m^\dagger M_m\| \}_{m=1}^\infty$ is uniformly (about $m$) bounded by $\|\Gamma^{-1}M\| $.
\end{lemma}
\begin{lemma}\label{lem: Gamma inverse M to Gammam dagger Mm}Under Assumptions $\ref{as:joint distribution assumption}$ and $\ref{as:Linearity condition and Coverage condition}$, we have \[\lim\limits_{m\to\infty}\lno\Gamma^{-1}M-\Gamma_m^\dagger M_m\rno =0.\]
\end{lemma}
\noindent We denote by $m_T(\varepsilon)$ the minimal integer $m_T$ satisfying $\lno\Gamma^{- 1}M-\Gamma_m^\dagger M_m\rno \hspace{-1mm}\leqslant \varepsilon$ for all $m\geqslant m_T$ and define an event 
$$\ttE:=\lb \left\|\widehat\Gamma_m^\dagger \widehat M_m^d-\Gamma_m^\dagger M_m\right\|  \leqslant\hspace{-0.5mm}\left(\tfrac{D_0+1}{D_2}\right)^{\frac52}\tfrac{24}{\wt C}n^{c_1(\alpha_1+1)+\gamma-\frac{1}{2}}+D_3n^{\frac{c_1(2\alpha_1+1)-1}{2}}\rb.$$
Then by taking $C$ to be $(D_0+1)n^{\frac{2\gamma}{5}}-\ln\l D_1m^2n \r$ in  Proposition \ref{prop:bound of finite estimate}, one has: for $n\geqslant \l\frac{D_0+1}{D_2}\r^{\frac{5}{1-2\gamma}}$,
$$\P(\ttE)\geq 1-D_1m^2n\exp\left[-(D_0+1)n^{\frac{2\gamma}{5}}\right] -2\exp(- C'm).$$
\begin{lemma}\label{lem:lower bound sigma min total}
Introducing $
\bigtriangleup :=\max\lb \frac{\sigma_d(\Gamma^{-1} M)}{2},\frac{\sigma_d(\Gamma^{-1} M)^2}{4\|\Gamma^{-1}M\| } \rb$.
Suppose that Assumptions $\ref{as:joint distribution assumption}$ to $\ref{assumption: rate-type condition}$ hold, $c_1(2\alpha_1+1)-1<0$ and $2(c_1(\alpha_1+1)+\gamma)-1<0$. Then there exists a positive constant
\begin{align*}
n_2'=n_2'\l\sigma_d(\Gamma^{-1}M),\|\Gamma^{-1}M\| , \gamma,\sigma_0,\sigma_1,\bs K,m_M(1),c_1,m_T\l \tfrac{\bigtriangleup}{2}\r,\wt C,\alpha_1\r
\end{align*}
such that when $n\geqslant n_2'$, we have
\begin{align}
\sigma_{\min}^+(\Gamma_m^{\dagger} M_m)^2\geqslant \tfrac{\sigma_d(\Gamma^{-1}M)^2}{2} \label{eq: lower bound of sigma min}. 
\end{align}
Furthermore, Conditioning on $\ttE$, we have
\begin{align}\label{eq: lower bound of sigma min hat}
&\sigma_{\min}^+(\wh\Gamma_m^{\dagger}\wh M_m^d)^2\geqslant \tfrac{\sigma_d(\Gamma^{-1}M)^2}{2}.
\end{align}
\end{lemma}
The following proposition is an upper bound of error term (i):
\begin{proposition}\label{proposition, estimation error}
Positive constants $D_1$, $D_2$ and $C'$  as in Proposition $\ref{prop:bound of finite estimate}$,
suppose that Assumptions $\ref{as:joint distribution assumption}$ to $\ref{assumption: rate-type condition}$ hold, then $\forall \gamma\in(0,1/2)$, if $c_1$ satisfies $2c_1(\alpha_1+1)+2\gamma-1<0$ and $c_1(2\alpha_1+1)-1<0$, there exists a positive constant $C_1:=C_1\l \|\Gamma^{-1}M\| ,\sigma_d(\Gamma^{-1}M) ,\wt C,\gamma,\sigma_0,\sigma_1\r$ such that
\begin{align*}
\P\l
\lno P_{\mc{S}_{\Y|\X}^{(m)}}-P_{ \widehat{\mc{S}}_{\Y|\X}^{(m)}}\rno \leqslant C_1\frac{m^{\alpha_1+1}}{n^{1/2-\gamma}}\r\geqslant1-2\exp(- C'm)&\\
- D_1m^2n\exp\l -(D_0+1)n^{\frac{2\gamma}{5}} \r&,
\end{align*}
when 
\begin{align*}
n\geqslant\max\Bigg\{ n_1,\l\tfrac{D_0+1}{D_2}\r^{\frac{5}{1-2\gamma}},\left[\tfrac{\|\Gamma^{-1}M\|  \wt C}{48}\l\tfrac{D_2}{D_0+1}\r^{\frac52}\right]^{\frac{2}{2(c_1(\alpha_1+1)+\gamma)-1}}&,\\
\l \tfrac{\|\Gamma^{-1}M\| }{2D_3}\r^{\frac{2}{c_1(2\alpha_1+1)-1}},n_2',\left[ \tfrac{D_3\wt C}{24}\l \tfrac{D_2}{D_0+1} \r^{\frac52} \right]^{\frac2{2\gamma+c_1}}&\Bigg\}
\end{align*}
where $n_2'$ is defined in Lemma $\ref{lem:lower bound sigma min total}$.
\end{proposition}
\begin{proof}
By Lemma $\ref{lemma, way of estimate truncate central subspace}$, $\eqref{def: estimator central subspace}$ and Lemma $\ref{lemma, sin theta of nonadjoint operator}$, one has
\begin{align}
&\left\|P_{\mc S_{\Y|\vX}^{(m)}}-P_{\wh{\mc{S}}_{\Y|\vX}^{(m)}}\right\| =\left\|P_{\Gamma_m^{\dagger}M_m}-P_{\wh\Gamma_m^{\dagger}\wh M_m^d}\right\| \nonumber\\
&\qquad\leqslant\frac{\pi}{2}\frac{\lno\widehat\Gamma_m^\dagger \widehat M_m^d-\Gamma_m^\dagger M_m\rno ^2+\lno\widehat\Gamma_m^\dagger \widehat M_m^d-\Gamma_m^\dagger M_m\rno \lno\Gamma_m^\dagger M_m\rno }{\min\lb\sigma_{\min}^+\l\wh\Gamma_m^\dagger \wh M_m^d\r^2,\sigma_{\min}^+\l\Gamma_m^\dagger M_m\r^2\rb}\label{eq: PS minus P hat S norm}.
% &\leqslant C_5\|\widehat\Gamma_m^\dagger \widehat M_m^d-\Gamma_m^\dagger M_m\|\\
% &=\widetilde O_{\mathbb{P}}\l\frac{m^{\alpha_1+1}}{n^{1/2}}\r,
\end{align}
% with probability at least $1-\exp(- C)-2\exp(- C'm)$.
Because of $c_1(2\alpha_1+1)-1<0$ and $2(c_1(\alpha_1+1)+\gamma)-1<0$, it is easy to check that when
\[n\geqslant\max\lb\left[\tfrac{\|\Gamma^{-1}M\|  \wt C}{48}\l\tfrac{D_2}{D_0+1}\r^{\frac52}\right]^{\frac{2}{2(c_1(\alpha_1+1)+\gamma)-1}},\l \tfrac{\|\Gamma^{-1}M\| }{2D_3}\r^{\frac{2}{c_1(2\alpha_1+1)-1}}\rb,\]
both $\l\tfrac{D_0+1}{D_2}\r^{\frac52}\tfrac{24}{\wt C}n^{c_1(\alpha_1+1)+\gamma-\frac{1}{2}}$ and $D_3n^{\frac{c_1(2\alpha_1+1)-1}{2}}$ are less than or equal to $\frac{\|\Gamma^{-1}M\| }{2}$. Thus, on the event $\ttE$,
\begin{align}\label{eq: high prob upper bound is Gamma minus 1 M}
\lno\widehat\Gamma_m^\dagger \widehat M_m^d-\Gamma_m^\dagger M_m\rno \leqslant \lno\Gamma^{-1}M\rno .
\end{align}
By Lemma $\ref{lem:Gammam dagger Mm uniformly bounded}$, inserting \eqref{eq: high prob upper bound is Gamma minus 1 M} into \eqref{eq: PS minus P hat S norm} leads to
$$
\lno P_{\mc{S}_{\Y|\X}^{(m)}}-P_{ \widehat{\mc{S}}_{\Y|\X}^{(m)}}\rno
\leqslant \frac{\pi\lno\widehat\Gamma_m^\dagger \widehat M_m^d-\Gamma_m^\dagger M_m\rno \lno\Gamma^{-1}M\rno }{\min\lb\sigma_{\min}^+\l\wh\Gamma_m^\dagger \wh M_m^d\r^2,\sigma_{\min}^+\l\Gamma_m^\dagger M_m\r^2\rb},
$$
on the event $\ttE$.
Furthermore, when $n\geqslant \left[ \frac{D_3\wt C}{24}\l \frac{D_2}{D_0+1} \r^{\frac52} \right]^{\frac2{2\gamma+c_1}}$ and $n\geq n_2'$, one can get
$\l \tfrac{D_0+1}{D_2}\r^{\frac52}\tfrac{24m^{\alpha_1+1}}{\wt C n^{1/2-\gamma}}$ is greater than or equal to $D_3\tfrac{m^{(2\alpha_1+1)/2}}{n^{1/2}}$
and then on the event $\ttE$,
\begin{align*}
\lno P_{\mc{S}_{\Y|\X}^{(m)}}-P_{ \widehat{\mc{S}}_{\Y|\X}^{(m)}}\rno \leqslant \tfrac{96\pi\|\Gamma^{-1}M\| }{\sigma_d(\Gamma^{-1}M)^2}\l \tfrac{D_0+1}{D_2}\r^{\frac52}\tfrac{m^{\alpha_1+1}}{\wt C n^{1/2-\gamma}}.
\end{align*}
 by
Lemma $\ref{lem:lower bound sigma min total}$.
Then choosing $C_1=\tfrac{96\pi\|\Gamma^{-1}M\| }{\wt C\sigma_d(\Gamma^{-1}M)^2}\l \tfrac{D_0+1}{D_2}\r^{\frac52}$ can complete the proof.
\end{proof}



\paragraph{Proof of Lemma \ref{lem:Gammam dagger Mm uniformly bounded}}
\begin{proof}
First, it is easy to check that:
\begin{align}
\Gamma^\dag_m=\Pi_m\Gamma^{-1}\Pi_m=\Pi_m\Gamma^{-1}=\Gamma^{-1}\Pi_m=\sum\limits_{i=1}^m\lambda_i^{-1}\phi_i\otimes\phi_i.\label{eq: Gamma m dag def}
\end{align}
According to \eqref{eq: Gamma m dag def} and $M_m=\Pi_mM\Pi_m$, it is easy to check that $\Gamma_m^\dagger M_m=\Pi_m \Gamma^{- 1}M\Pi_m$. Then by the compatibility of operator norm, one can get
\begin{align*}
\lno\Gamma_m^\dagger M_m\rno =\lno\Pi_m \Gamma^{-1}M\Pi_m\rno \leqslant \lno\Pi_m\rno  \lno\Gamma^{-1}M\rno \lno\Pi_m\rno =\lno\Gamma^{-1}M\rno .
\end{align*}
Note that $\Gamma^{-1}M$ is bounded since $\Gamma^{-1}M$ is of finite rank by Corollary $\ref{corollary, MDDO and central subspace}$. Thus the proof is completed. 
\end{proof}


\paragraph{Proof of Lemma \ref{lem: Gamma inverse M to Gammam dagger Mm}}
\begin{proof}
It is easy to check that
$\Gamma_m^\dagger M_m=\Pi_m\Gamma^{-1}M\Pi_m$ and $\Gamma^{-1}M$ is of finite rank by Corollary $\ref{corollary, MDDO and central subspace}$.
Thus the proof is completed by Lemma $\ref{lem:PimTPimtoT}$.
\end{proof}
\paragraph{Proof of Lemma \ref{lem:lower bound sigma min total}}
\begin{proof}
We first prove \eqref{eq: lower bound of sigma min}.
By Corollary $\ref{corollary, MDDO and central subspace}$ and Lemma $\ref{lem:projection equality}$, one has $\rank(\Gamma^{- 1}M)=\rank\l\Gamma^{- 1}M(\Gamma^{- 1}M)^*\r=d$. Thus
\begin{align*}
\sigma_{\min}^+(\Gamma^{-1}M)^2=\lambda_{\min}^+\l\Gamma^{-1}M(\Gamma^{-1}M)^*\r=\lambda_d\l \Gamma^{-1}M(\Gamma^{-1}M)^*\r. 
\end{align*}
 It is easy to see $\rank(\Gamma_m^\dagger M_m)=\rank\l \Gamma_m^\dagger M_m(\Gamma_m^\dagger M_m)^*\r\leqslant d$ by $\Gamma_m^\dagger M_m=\Pi_m \Gamma^{-1} M \Pi_m$ and Lemma $\ref{lem:projection equality}$, thus one can assume that 
 \begin{align*}
\sigma_{\min}^+(\Gamma^\dagger_m M_m)^2=\lambda_{\min}^+\l\Gamma_m^\dagger M_m(\Gamma_m^\dagger M_m)^*\r=\lambda_j\l \Gamma_m^\dagger M_m(\Gamma_m^\dagger M_m)^*\r
\end{align*}
for some $j\leqslant d$.
By Corollary $\ref{coro:wely ineq operator}$, $\eqref{eq:sy ineq}$ and
% (Notice that $M_m$ and $M$ are both compact and self-adjoint)
Lemma $\ref{lem: Gamma inverse M to Gammam dagger Mm}$
%and Lemma \ref{lem:Gammam dagger Mm uniformly bounded}
, one has
\begin{align*}
&\left|\sigma_{\min}^+(\Gamma^\dagger_m M_m)^2\hspace{-0.5mm}-\hspace{-0.5mm}\sigma_j(\Gamma^{-1} M)^2\right|\hspace{-0.5mm}=\hspace{-0.5mm}\left|\lambda_{j}\hspace{-1mm}\l\Gamma^\dagger_m M_m(\Gamma^\dagger_m M_m)^{*}\hspace{-0.5mm}\r\hspace{-0.5mm}-\hspace{-0.5mm}\lambda_j\hspace{-1mm}\l \Gamma^{-1} M(\Gamma^{-1} M)^*\hspace{-0.5mm}\r\right|\\
&\qquad\leqslant
\|\Gamma^{-1} M(\Gamma^{-1} M)^*- \Gamma_m^\dagger M_m(\Gamma_m^\dagger M_m)^*\| \\
&\qquad\leqslant \|\Gamma^{-1} M- \Gamma_m^\dagger M_m\| ^2+
\|\Gamma^{-1} M- \Gamma_m^\dagger M_m\| \cdot\|\Gamma^{-1} M\| \xrightarrow{m\to\infty} 0. 
% &\leqslant\|\Gamma^{-1} M- \Gamma_m^\dagger M_m\|\cdot3\|\Gamma^{-1} M\|
\end{align*}
Thus for 
$
n\geqslant m_T(\bigtriangleup)^{\frac1{c_1}}=m_T\l\max\lb\frac{\sigma_d(\Gamma^{-1} M)}{2},\frac{\sigma_d(\Gamma^{-1} M)^2}{4\|\Gamma^{-1}M\| }\rb\r^{\frac1{c_1}}, 
$
one has $\|\Gamma^{-1} M- \Gamma_m^\dagger M_m\| ^2$ and $\|\Gamma^{-1} M- \Gamma_m^\dagger M_m\| \cdot\|\Gamma^{-1} M\| $ are both less than or equal to $\frac{1}{4}\sigma_d(\Gamma^{-1} M)^2$. Hence one can get
$\left|\sigma_{\min}^+(\Gamma^\dagger_m M_m)^2-\sigma_j(\Gamma^{-1} M)^2\right|\leqslant\frac{1}{2}\sigma_d(\Gamma^{-1} M)^2$
% \begin{align*}\label{eq:sigma min Mm}
% \left|\sigma_{\min}^+(\Gamma^\dagger_m M_m)^2-\sigma_j(\Gamma^{-1} M)^2\right|\leqslant\frac{1}{2}\sigma_d(\Gamma^{-1} M)^2
% \|
% \lambda_j\l \Gamma_m^\dagger M_m\l\Gamma_m^\dagger M_m\r^*\r\geqslant \lambda_j\l \Gamma^{-1} M\l\Gamma^{-1} M\r^*\r-\frac{\lambda_d\l \Gamma^{-1} M\l\Gamma^{-1} M\r^*\r}{2}
% \geqslant\frac{\lambda_d\l \Gamma^{-1} M\l\Gamma^{-1} M\r^*\r}{2}. 
% \end{align*}
and
\begin{equation}
\sigma_{\min}^+(\Gamma^\dagger M_m)^2\geqslant \sigma_j(\Gamma^{-1} M)^2-\frac{1}{2}\sigma_d(\Gamma^{-1} M)^2\geqslant\frac{1}{2}\sigma_d(\Gamma^{-1} M)^2
\end{equation}
for sufficiently large $n$. This completes the proof of \eqref{eq: lower bound of sigma min}.





Next we prove $\eqref{eq: lower bound of sigma min hat}$. Combining Proposition $\ref{prop:bound of finite estimate}$ with Lemma $\ref{lem: Gamma inverse M to Gammam dagger Mm}$ leads to that on the event $\ttE$, 
$$
\lno\wh \Gamma_m^\dag\wh M^d_m- \Gamma^{-1}M\rno \leqslant\ve+\l\tfrac{D_0+1}{D_2}\r^{\frac52}\tfrac{24}{\wt C}n^{c_1(\alpha_1+1)+\gamma-\frac{1}{2}}+D_3n^{\frac{c_1(2\alpha_1+1)-1}{2}}
$$
for  $n\geqslant \max\{n_1,m_T( \ve)^{1/c_1}\}$.
Assuming that $c_1(2\alpha_1+1)-1<0$ and $2(c_1(\alpha_1+1)+\gamma)-1<0$, it is easy to check that when $$n\geqslant\max\lb\left[\frac{\bigtriangleup \wt C}{96}\l\frac{D_2}{D_0+1}\r^{\frac52}\right]^{\frac{2}{2(c_1(\alpha_1+1)+\gamma)-1}},\l \frac{\bigtriangleup}{4D_3}\r^{\frac{2}{c_1(2\alpha_1+1)-1}}\rb$$, both $\l\tfrac{D_0+1}{D_2}\r^{\frac52}\tfrac{24}{\wt C}n^{c_1(\alpha_1+1)+\gamma-\frac{1}{2}}$ and $D_3n^{\frac{c_1(2\alpha_1+1)-1}{2}}$ are less than or equal to $\frac{\bigtriangleup}{4}$. Letting $\varepsilon=\frac12\bigtriangleup$, one can get on the event $\ttE$,
when
\begin{align*}
n&\geqslant n_2'=n_2'\hspace{-0.5mm}\l\hspace{-0.5mm}\sigma_d(\Gamma^{-1}M),\|\Gamma^{-1}M\| , \gamma,\sigma_0,\sigma_1,\bs K,m_M(1),c_1,m_T\l \tfrac{\bigtriangleup}{2}\r,\wt C,\alpha_1\hspace{-0.5mm}\r\\
&:=\max\bigg\{ n_1,m_T\l \tfrac{\bigtriangleup}{2}\r^{1/c_1}, \left[\tfrac{\bigtriangleup \wt C}{96}\l\tfrac{D_2}{D_0+1}\r^{\frac52}\right]^{\frac{2}{2(c_1(\alpha_1+1)+\gamma)-1}},\l \tfrac{\bigtriangleup}{4D_3}\r^{\frac{2}{c_1(2\alpha_1+1)-1}}\bigg\},
\end{align*}
one has $\lno\wh \Gamma_m^\dag\wh M^d_m- \Gamma^{-1}M\rno \leqslant\bigtriangleup$ and further
$\sigma_{\min}^+(\wh\Gamma^\dagger \wh M^d_m)^2\hspace{-1mm}\geqslant\hspace{-1mm} \tfrac{\sigma_d(\Gamma^{-1} M)^2}{2}$ by the same argument as the proof of \eqref{eq: lower bound of sigma min}.
 This completes the proof of \eqref{eq: lower bound of sigma min hat}.
Considering that $m_T(\bigtriangleup)\leqslant m_{T}\l\frac\bigtriangleup2\r$, one can also get $\eqref{eq: lower bound of sigma min}$ when $n\geqslant n_2'$. Thus the proof is completed.
\end{proof}
\subsection{Upper bound of error term (ii)}\label{ap, subs, truncation error}
\begin{proposition}\label{proposition, truncation error}
Under Assumption $\ref{assumption: rate-type condition}$, there exists a positive constant $C_2:=C_2\l d,\wt C,\lambda_d(\mc{B}),\alpha_2\r$ where $\mc{B}:=\sum\limits_{i=1}^d {\bs{\beta}}_i\otimes{\bs{\beta}}_i$ for ${\bs{\beta}}_i$ defined in \eqref{def: central subspace}, such that when $n\geqslant \l \frac{\lambda_d({\mc{B}})}{4d\wt C^2}\sqrt{\frac{2\alpha_2-1}{\zeta(2\alpha_2)}}\r^{\frac{2}{c_1(1-2\alpha_2)}}$, we have
\begin{equation}\label{equation, truncation error}
 \left\|P_{\mathcal S_{\Y|\boldsymbol{X}}}-P_{\mathcal S_{\Y|\boldsymbol{X}}^{(m)}}\right\| \leqslant C_2m^{-\frac{2\alpha_2-1}{2}},
\end{equation}
where $\zeta(\cdot)$ is Riemann $\zeta$ function.
% \begin{equation}
% \|P_{\mathcal S_{Y|\boldsymbol{X}}}-P_{\mathcal S_{Y|\boldsymbol{X}}^{(m)}}\|\leqslant O_{\mathbb{P}}(dn^{-(\alpha_2-1)/(2\alpha_1+\alpha_2)}) 
% \end{equation}
\end{proposition}
\begin{proof}
Let ${\mc{B}^{(m)}}:=\sum\limits_{i=1}^d {\bs{\beta}}_i^{(m)}\otimes{\bs{\beta}}_i^{(m)}$ for ${\bs{\beta}}_i^{(m)}$ defined in \eqref{def: truncated central subspace}.
Combing with Equation $\eqref{def: central subspace}$, it is easy to check that $\left\|P_{\mathcal S_{\Y|\boldsymbol{X}}}-P_{\mathcal S_{\Y|\boldsymbol{X}}^{(m)}}\right\| =\|P_{\mc{B}}-P_{\mc{B}^{(m)}}\| $. By Corollary $\ref{cor: sin theta self adjoint}$, we have
\begin{align}\label{eq:sin theta for B Bm}
\|P_{\mc{B}}-P_{\mc{B}^{(m)}}\| \leqslant \frac{\pi}{2}\frac{\|{\mc{B}}-{\mc{B}^{(m)}}\| }{\min\{\lambda_{\min}^+({\mc{B}}),\lambda_{\min}^+({\mc{B}^{(m)}})\}}.
\end{align}

Note that ${\mc{B}}-{\mc{B}^{(m)}}$ is self-adjoint, then
\begin{align*}
&\lno{\mc{B}}-{\mc{B}^{(m)}}\rno =\sup_{{\bs{\beta}}\in\mathbb{S}_{ \mathcal H}}|\langle ({\mc{B}}-{\mc{B}^{(m)}})({\bs{\beta}}),{\bs{\beta}}\rangle|=\sup_{{\bs{\beta}}\in\mathbb{S}_{\mathcal H}}|\langle {\mc{B}}{\bs{\beta}},{\bs{\beta}}\rangle-\langle {\mc{B}^{(m)}}{\bs{\beta}},{\bs{\beta}}\rangle|\\
&~~=\sup_{{\bs{\beta}}\in\mathbb{S}_{\mathcal H}}\hspace{-0.9mm}\left|\sum_{i=1}^d\hspace{-0.9mm}\left[\langle{\bs{\beta}}_i,{\bs{\beta}}\rangle^2-\langle{\bs{\beta}}_i^{(m)},{\bs{\beta}}\rangle^2\right]\right|=\sup_{{\bs{\beta}}\in\mathbb{S}_{\mathcal H}}\hspace{-0.9mm}\left| \sum_{i=1}^d\langle{\bs{\beta}}_i-{\bs{\beta}}_i^{(m)},{\bs{\beta}}\rangle\langle{\bs{\beta}}_i+{\bs{\beta}}_i^{(m)},{\bs{\beta}}\rangle\right|\\
&~~\leqslant\sup_{{\bs{\beta}}\in\mathbb{S}_{\mathcal H}}\sum_{i=1}^d\left| \langle{\bs{\beta}}_i-{\bs{\beta}}_i^{(m)},{\bs{\beta}}\rangle\langle{\bs{\beta}}_i+{\bs{\beta}}_i^{(m)},{\bs{\beta}}\rangle\right|
\leqslant\sum_{i=1}^d\left\|{\bs{\beta}}_i-{\bs{\beta}}_i^{(m)}\right\|\left\|{\bs{\beta}}_i+{\bs{\beta}}_i^{(m)}\right\|,
\end{align*}
where the first inequality comes from the triangle inequality, and the 
second inequality comes from the Cauchy-Schwarz inequality and $\|{\bs{\beta}}\|=1$. 
 Then one has ${\bs{\beta}}_i=\sum\limits_{j=1}^\infty b_{ij}\phi_j$ and 
\[{\bs{\beta}}^{(m)}_i=\Pi_m{\bs{\beta}}_i=\sum_{j'=1}^m\phi_{j'}\otimes\phi_{j'}\sum_{j=1}^\infty b_{ij}\phi_j=\sum_{j'=1}^m\sum_{j=1}^\infty\langle\phi_{j'},\phi_j\rangle b_{ij}\phi_{j'}=\sum_{j=1}^mb_{ij}\phi_j.\]
According to Assumption $\ref{assumption: rate-type condition}$, one can get
\begin{align*}
\left\|{\bs{\beta}}_i-{\bs{\beta}}_i^{(m)}\right\|&=\left\|\sum_{j=m+1}^\infty b_{ij}\phi_j\right\|=\sqrt{\sum_{j=m+1}^\infty b_{ij}^2}\leqslant \wt C\sqrt{\sum_{j=m+1}^\infty j^{-2\alpha_2}};\\
\left\|{\bs{\beta}}_i+{\bs{\beta}}_i^{(m)}\right\|&\leqslant\|{\bs{\beta}}_i\|+\lno{\bs{\beta}}_i^{(m)}\rno\leqslant2\|{\bs{\beta}}_i\|=2\sqrt{\sum_{j=1}^\infty b_{ij}^2}\leqslant 2\wt C\sqrt{\sum_{j=1}^\infty j^{- 2\alpha_2}}.
\end{align*}
Because $\alpha_2>1/2$, one has
\[\sum\limits_{j=m+1}^\infty \frac{1}{j^{2\alpha_2}}\leqslant \frac{1}{2\alpha_2-1}\frac{1}{m^{2\alpha_2-1}};\qquad \sum_{j=1}^\infty \frac 1{j^{2\alpha_2}}=\zeta(2\alpha_2)\text{ is convergent},\]
where $\zeta(\cdot)$ is Riemann $\zeta$ function. Thus, one can get
\begin{equation}\label{eq: upper bound of operator norm of A minus B}
\lno{\mc{B}}-{\mc{B}^{(m)}}\rno \leqslant 2d\wt C^2\sqrt{\frac{\zeta(2\alpha_2)}{2\alpha_2-1}}m^{-\frac{2\alpha_2-1}{2}}.
\end{equation}

Furthermore, 
{since $\mr{rank}(\mc{B})=d$, one can get that $\lambda_{\min}^+(\mc{B})=\lambda_{d}(\mc{B})$. It is easy to see $\rank(\mc{B}^{(m)})\leqslant d$ by $\mc{B}^{(m)}=\Pi_m \mc{B} \Pi_m$, thus one can assume that $\lambda_{\min}^+(\mc{B}^{(m)})=\lambda_j( \mc{B}^{(m)})$ for some $j\leqslant d$.
By Corollary $\ref{coro:wely ineq operator}$
% (Notice that $M_m$ and $M$ are both compact and self-adjoint)
and \eqref{eq: upper bound of operator norm of A minus B}, one has:
$$
|\lambda_j( \mc{B}^{(m)})-\lambda_j\l \mc{B}\r|\leqslant\lno \mc{B}-\mc{B}^{(m)}\rno \leqslant 2d\wt C^2\sqrt{\frac{\zeta(2\alpha_2)}{2\alpha_2-1}}m^{-\frac{2\alpha_2-1}{2}}.
$$
Thus for sufficiently large {$n\geqslant \l \frac{\lambda_d({\mc{B}})}{4d\wt C^2}\sqrt{\frac{2\alpha_2-1}{\zeta(2\alpha_2)}} \r^{\frac{2}{c_1(1-2\alpha_2)}}$}, one has
\begin{align}
&\lambda_j\l \mc{B}^{(m)}\r\geqslant \lambda_j\l \mc{B}\r-\frac{\lambda_d\l \mc{B}\r}{2}
\geqslant\frac{\lambda_d\l \mc{B}\r}{2}\nonumber\\
&\qquad\Longrightarrow \min\{\lambda_{\min}^+({\mc{B}}),\lambda_{\min}^+({\mc{B}^{(m)}})\}\geqslant \frac{\lambda_d({\mc{B}})}{2}. \label{eq:lower bound lambda min plus B Bm}
\end{align}}
Inserting \eqref{eq: upper bound of operator norm of A minus B} and \eqref{eq:lower bound lambda min plus B Bm} into \eqref{eq:sin theta for B Bm} leads to
\begin{align*}
\left\|P_{\mathcal S_{\Y|\boldsymbol{X}}}-P_{\mathcal S_{\Y|\boldsymbol{X}}^{(m)}}\right\| \leqslant \frac{2\pi d\wt C^2}{\lambda_{d}(\mc{B})}\sqrt{\frac{\zeta(2\alpha_2)}{2\alpha_2-1}}m^{-\frac{2\alpha_2-1}{2}}.
\end{align*}
Then choosing $C_2:=\frac{2\pi d\wt C^2}{\lambda_d({\mc{B}})}\sqrt{\frac{\zeta(2\alpha_2)}{2\alpha_2-1}}$ can complete the proof.
\end{proof}



\subsection{Proof of Theorem \ref{theorem, total convergence rate}}
\begin{proof}
Note that
\begin{equation}
\begin{aligned}
\left\|P_{\mc{S}_{\Y|\X}}-P_{\widehat{\mc{S}}_{\Y|\X}^{(m)}}\right\| 
&\leqslant \left\|P_{\mc{S}_{\Y|\X}}-P_{\mc{S}_{\Y|\X}^{(m)}}\right\| +\left\|P_{\mc{S}_{\Y|\X}^{(m)}}-P_{ \widehat{\mc{S}}_{\Y|\X}^{(m)}}\right\| .\\
\end{aligned}
\end{equation}
Next we select $m$ to be $n^{\frac{1-2\gamma}{2\alpha_1+2\alpha_2+1}}$, i.e.,  $c_1:=\frac{1-2\gamma}{2\alpha_1+2\alpha_2+1}$. And it is easy to check that $c_1$ satisfies $2c_1(\alpha_1+1)+2\gamma-1=-\frac{(1-2\gamma)(2\alpha_2-1)}{2\alpha_1+2\alpha_2+1}<0$ and $c_1(2\alpha_1+1)-1=-\frac{2[\gamma(2\alpha_1+1)+\alpha_2]}{2\alpha_1+2\alpha_2+1}<0$.
Then combining Proposition $\ref{proposition, estimation error}$ with Proposition $\ref{proposition, truncation error}$ leads to
\begin{align*}
\P\left[\left\|P_{\mc S_{\Y|\X}}-P_{\widehat{\mc{S}}_{\Y|\X}^{(m)}}\right\| \leqslant\hspace{-0.5mm} (C_1+C_2)n^{-\frac{(2\alpha_2-1)(1-2\gamma)}{2(2\alpha_1+2\alpha_2+1)}}\right]\hspace{-1mm}\geqslant\hspace{-1mm} 1-2\exp\hspace{-0.5mm}\l\hspace{-1mm}- C'n^{\frac{1-2\gamma}{2\alpha_1+2\alpha_2+1}}\r&\\
-\exp\left[\ln\l D_1n^{\frac{2\alpha_1+2\alpha_2+3-4\gamma}{2\alpha_1+2\alpha_2+1}} \r-(D_0+1)n^{\frac{2\gamma}{5}}\right]&
\end{align*}
when $n\geqslant n_3'$, where
\begin{align*}
n_3'=\max\Bigg\{n_1,n_2',\left[\tfrac{\|\Gamma^{-1}M\|  \wt C}{48}\l\tfrac{D_2}{D_0+1}\r^{\frac52}\right]^{\frac{2}{2(c_1(\alpha_1+1)+\gamma)-1}}\hspace{-0.9mm},\l \tfrac{\|\Gamma^{-1}M\| }{2D_3}\r^{\frac{2}{c_1(2\alpha_1+1)-1}}\hspace{-0.9mm},\\
\l\tfrac{D_0+1}{D_2}\r^{\frac{5}{1-2\gamma}},\left[ \tfrac{D_3\wt C}{24}\l \tfrac{D_2}{D_0+1} \r^{\frac52} \right]^{\frac2{2\gamma+c_1}},\l\tfrac{\lambda_d(\mc{B})}{4d\wt C^2}\sqrt{\tfrac{{2\alpha_2-1}}{\zeta(2\alpha_2)}} \r^{\frac{2}{c_1(1-2\alpha_2)}}\Bigg\}
\end{align*}

It is easy to check that as long as $\frac{2\gamma}{5}<\frac{1-2\gamma}{2\alpha_1+2\alpha_2+1}\Longrightarrow\gamma<\frac{5}{4(\alpha_1+\alpha_2+3)}$, 
there exists a constant $n_3''=n_3''\l \gamma,\alpha_1,\alpha_2,D_0,D_1,C'\r$ such that when $n\geqslant n_3'$ further, we have 
\begin{align*}
\P\l\left\|P_{\mc S_{\Y|\X}}-P_{\widehat{\mc{S}}_{\Y|\X}^{(m)}}\right\| \leqslant (C_1+C_2)n^{-\frac{(2\alpha_2-1)(1-2\gamma)}{2(2\alpha_1+2\alpha_2+1)}} \r
\geqslant1-2\exp\l-\tfrac{D_0+1}{2}n^{\frac{2\gamma}{5}} \r.
\end{align*}
Thus one can choose $n_3=\max\{n_3',n_3''\}$ to get the following conclusion.
\begin{proposition}
Under Assumptions $\ref{as:joint distribution assumption}$ to $\ref{assumption: rate-type condition}$, for any $\gamma\in\l0,\tfrac{5}{4(\alpha_1+\alpha_2+3)}\r$, choosing 
$m=n^{\frac{1-2\gamma}{2\alpha_1+2\alpha_2+1}}$ (i.e.,  $c_1=\frac{1-2\gamma}{2\alpha_1+2\alpha_2+1}$) yields a positive constant
\begin{align*}
D_4:=D_4\l \|\Gamma^{-1}M\| ,\sigma_d(\Gamma^{-1}M) ,\gamma,\sigma_0,\sigma_1,d,\wt C,\lambda_d\l\sum\limits_{i=1}^d {\bs{\beta}}_i\otimes{\bs{\beta}}_i\r,\alpha_2\r 
\end{align*}
such that when $n$ is sufficiently large, we have:
\begin{align*}
\P\l\left\|P_{\mc{S}_{\Y|\X}}-P_{\widehat{\mc{S}}_{\Y|\X}^{(m)}}\right\| \leqslant D_4n^{-\frac{(2\alpha_2-1)(1-2\gamma)}{2(2\alpha_1+2\alpha_2+1)}} \r
\geqslant1-2\exp\l -\tfrac{D_0+1}{2}n^{\frac{2\gamma}{5}} \r,
\end{align*}
where $D_0$ and $D_1$ are defined in Proposition $\ref{prop:bound hatMmd Mm}$.
\end{proposition}
\noindent
% Theorem $\ref{theorem, total convergence rate}$ is a direct corollary of above proposition.
Define 
$$\mathtt F:=\left\{\left\|P_{\mc{S}_{\Y|\X}}-P_{\widehat{\mc{S}}_{\Y|\X}^{(m)}}\right\| \leqslant D_4n^{-\frac{(2\alpha_2-1)(1-2\gamma)}{2(2\alpha_1+2\alpha_2+1)}}\right\}.$$
Then 
\begin{align*}
 \mb E\left[\left\|P_{\mc{S}_{\Y|\X}}-P_{\widehat{ \mc{S}}_{\Y|\X}^{(m)}}\right\|^2\right] =&
  \mb E\left[\left\|P_{\mc{S}_{\Y|\X}}-P_{\widehat{ \mc{S}}_{\Y|\X}^{(m)}}\right\|^21_{\mathtt{F}}\right] +
   \mb E\left[\left\|P_{\mc{S}_{\Y|\X}}-P_{\widehat{ \mc{S}}_{\Y|\X}^{(m)}}\right\|^21_{\mathtt{F}^c}\right]\\ 
 \leqslant &
 D_4^2n^{-\frac{(2\alpha_2-1)(1-2\gamma)}{2\alpha_1+2\alpha_2+1}}+4\mb P\left( \mathtt F^c\right)\\
 \lesssim&n^{-\frac{(2\alpha_2-1)(1-2\gamma)}{2\alpha_1+2\alpha_2+1}}+\exp\l -\tfrac{D_0+1}{2}n^{\frac{2\gamma}{5}} \r\\
\lesssim&n^{-\frac{(2\alpha_2-1)(1-2\gamma)}{2\alpha_1+2\alpha_2+1}}.
\end{align*}
This completes the proof of  Theorem \ref{theorem, total convergence rate}.
\end{proof}







\section{Additional Simulation Results of Section \ref{sec:Synthetic}}
This section contains the additional  simulation results  of Sections \ref{sec:Synthetic}  when $\varepsilon\sim N(0,1)$.



We show the average $\mc D(\bs B;\bs{\wh B})$ with different $m$ or $\rho$ for three methods under $\mc M_1$ to $\mc M_3$ in Figure \ref{fig:error 3models,noise1},
where we mark minimal error in each model with red `$\times$'. The shaded areas represent the standard error associated with these estimates and all of them are less than  $0.01$. For FSFIR, the  minimal errors for $\mc M_1-\mc M_3$ are  $0.08,0.02,0.01$ respectively.
For TFSIR, the  minimal errors are  $0.08,0.02,0.01$ and for regularized FSIR,  the  minimal errors are $0.13,0.06,0.01$.  

% Figure environment removed


Figure \ref{fig:error 3models,noise1} shows that FSFIR attains the best performance among  all models. 
Moreover, FSFIR is easier to practice as it does not need a slice number $H$ in advance. 





\end{appendices}

\label{last.page}



\end{document}
