We present a case study using a dataset on corporate law partnership among a Northeastern US corporate law firm in New England collected by \citet{Lazega2001}. The dataset collected information about three types of cooperation among 71 lawyers in the corporate law firm, resulting in three networks including the strong-coworker network, the advice network, and the friendship network. Since the cooperation relationship collected are not symmetric, we only consider a connection to be present when both sides acknowledged their cooperation. We treat these three types of networks as a three-layer multilayer network embedded among the 71 lawyers. A summary of this multilayer network is provided in Table \ref{t_datasum}. We apply the separable multilayer network model in \eqref{general_model} with the specification in \eqref{eq:sim_model} to Lazega's lawyer network, i.e., $\nat = (\theta_{1},\,\theta_{2},\,\theta_{3},\,\theta_{1,2},\,\theta_{1,3}, \,\theta_{2,3})$. The maximum pseudolikelihood estimator $\mple$ is computed from the observed network,
the results of which are provided in Table \ref{t_Lazeg_mple}.
 % Figure environment removed


\begin{table}[t]
\begin{center}
\caption{\label{t_datasum} Summary of Lazega's corporate law partnership data with 71 lawyers (nodes).  }
\begin{tabular}{ l  r  r }
\hline
 & Average Node Degree & Number of Edges \\
\hline
 Co-Worker Layer  & 11 & 378  \\

Advice Layer & 5 & 175  \\

Friendship Layer & 5 & 176  \\
\hline
\end{tabular}
\end{center}
\end{table}


\begin{table}[t]
\begin{center}
\caption{\label{t_Lazeg_mple}MPLEs (and standard errors) of the separable multilayer network model for the Lazega's lawyer network.}
\resizebox{\columnwidth}{!}{
\begin{tabular}{ c  c  c  c  c  c } 
\hline
$\widetilde\theta_{1}$ &$\widetilde\theta_{2}$ &$\widetilde\theta_{3}$ & $\widetilde\theta_{1,2}$ & $\widetilde\theta_{1,3}$ & $\widetilde\theta_{2,3}$ \\ 
\hline
$-1.450 \; (.263)$ & $-3.334 \; (.244)$ & $-2.695\; (.256)$ & $1.801 \; (.244)$ & $0.218 \; (.247)$ & $2.458 \; (.231)$ \\

Coworker (C) & Advice (A) & Friendship (F) & C $\times$ A & C $\times$ F & A $\times$ F \\
\hline
\end{tabular}
}
\end{center}
\end{table}


As shown in Table \ref{t_Lazeg_mple}, 
the maximum pseudolikelihood estimates 
$\widetilde\theta_{1}, \, \widetilde\theta_{2}$, and $\widetilde\theta_{3}$ correspond to the estimated single-layer effects of the coworker layer, the advice layer, and the friendship layer, respectively,
whereas $\theta_{1,2}, \, \theta_{1,3}$, and $\theta_{2,3}$ correspond to the layer interaction effects.  
We can calculate the conditional log-odds of each edge being present in the multilayer network given the rest of the network. For example,  if lawyer $i$ and lawyer $j$ are observed to be coworkers and are friends at the same time, the odds of these two lawyers to have an advice relationship is given by
\beno
\dfrac{\mbP(X_{i,j}^{(A)} = 1 \,|\, \bX_{i,j}^{(C)} = 1, \, \bX_{i,j}^{(F)} = 1)}
{\mbP(X_{i,j}^{(A)} = 0 \,|\, \bX_{i,j}^{(C)} = 1, \, \bX_{i,j}^{(F)} = 1)}
\quad = \quad \exp\left(
\widetilde\theta_{2} +  \widetilde\theta_{1,2} \, x_{i,j}^{(C)} + \widetilde\theta_{2,3} \, x_{i,j}^{(F)}\right) \s \\
 = \quad \exp\left(
-3.334 +  1.801 \, x_{i,j}^{(C)} + 2.458 \, x_{i,j}^{(F)}\right) 
 \quad = \quad  2.522,
\ee
providing interpretation of the interaction and influence among the different layers.

% Figure environment removed

Next, 
we use the MPLE to reproduce multilayer networks of the same size and compare the sufficient statistics 
of the simulated networks and the Lazega's lawyer network.
We recover the basis network according to Proposition \ref{prop:inference}, i.e.,
a dyad is activated if and only if at least one of its layers has a present edge in the Lazega's lawyer multilayer network.
We then populate layers of all activated dyads according to equation \eqref{eq:sim_model} by the MPLEs obtained in Table \ref{t_Lazeg_mple}.
Comparisons of the sufficient statistics between the observed Lazega's lawyer network and the simulated networks with 10 replications are provided in 
Figure \ref{Figure_Lazega}. 
Such comparisons serve two key purposes. 
First, 
such comparisons are an established method of diagnosing model fit in the statistical network analysis literature 
\citep{HuGoHa08}, 
and second, 
provide a check on the approximate solution to the score equation.
Note that MPLEs are not guaranteed to reproduce (on average) observed values of sufficient statistics in exponential families---in contrast to MLEs.
The relative $\ell_2$-error of the sufficient statistics between the observed and the average of the 10 simulated networks is $0.09$, suggesting a successful re-construction of the observed network statistics.

\hide{
The R package we developed for the simulation analysis in Section \ref{sec:sim} and the application analysis of Lazega's corporate law partnership data in Section \ref{sec:app} is available on GitHub: https://github.com/jiaheng-li/cross-layer-dependence-mlyrnetwork-simulation.
}


