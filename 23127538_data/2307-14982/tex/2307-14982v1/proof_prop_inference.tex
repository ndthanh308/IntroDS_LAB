\pproof \ref{prop:inference}.
For the first and second results,
define the set
\beno
\mA_{+}
&\coloneqq& \left\{ (\bx, \by) \in \mbX \times \mbY \;:\; h(\bx, \by) = 1 \right\},
\ee
and the vector-valued map $\bm{\varphi} : \mbX \mapsto \mbY$ by defining its components to be
\beno
\varphi_{i,j}(\bx)
\= \one(\norm{\bx_{i,j}}_1 > 0),
&& \{i,j\} \subset \mN, 
\ee
populating the vector $\bm{\varphi}(\bx)$ in the lexicographical ordering of the dyad indices $\{i,j\} \subset \mN$.  
By the definition of $h : \mbX \times \mbY \mapsto \{0, 1\}$ and $\bm{\varphi} : \mbX \mapsto \mbY$,
$\bm{\varphi}(\bx)  = \by$ for each pair $(\bx, \by) \in \mA_{+}$.  
Furthermore, 
the element $\by$ is unique for a given $\bx \in \mbX$,
because if there would exists some $\by^\prime \in \mbY$ 
such that $\by \neq \by^\prime$ 
with the property that 
$\{(\bx, \by), (\bx, \by^\prime)\} \subseteq \mA_{+}$, 
then there would exist a  pair $\{i,j\} \subset \mN$
such that $y_{i,j} = 1 - y^\prime_{i,j}$,
implying
$\one(\norm{\bx_{i,j}}_1 > 0) \neq y^\prime_{i,j}$,
in which case $h(\bx, \by^\prime) = 0$,
contradicting the assumption that $\{(\bx, \by)\} \in \mA_{+}$. 
By \eqref{general model},
the functions $f$ and $g$ are assumed to be strictly positive in their respective domains.
Hence, 
$(\mbX \times \mbY) \setminus \mA_{+}$ is the largest null set of $\mbX \times \mbY$,
i.e.,
$\sepmodel(\mA) = 0$ if and only if $\mA \subseteq (\mbX \times \mbY) \setminus \mA_{+}$.
Thus, 
the first and second results are established. 


For the third result,
note that $g$ is assumed to be strictly positive on its domain $\mbY$. 
Hence, 
$g(\by) = \mbP_{\nat}(\bY = \by) > 0$ for all $\by \in \mbY$
and  
$\sepmodel(\bX = \bx \,|\, \bY = \by)$
is therefore well-defined.
By definition,
\beno
\sepmodel(\bX = \bx \,|\, \bY = \by)
\= \dfrac{\sepmodel(\bX = \bx, \, \bY = \by)}{\sepmodel(\bY = \by)},
\ee
where $\sepmodel(\bY = \by)$ is the marginal probability of event $\bY = \by$ and is assumed to be equal to $g(\by)$.
The model form for $\sepmodel$ given in \eqref{general model} implies 
\beno
\dfrac{\sepmodel(\bX = \bx, \, \bY = \by)}{\sepmodel(\bY = \by)}
\= \dfrac{f(\bx, \nat) \, g(\by) \, \psi(\nat, \by)}{g(\by)} 
\= \exp(\log f(\bx, \nat) + \log \psi(\nat, \by)),
\ee
under the assumption that $h(\bx, \by) = 1$.
Hence,
\beno
\sepmodel(\bX = \bx, \bY = \by)
\= \sepmodel(\bX = \bX \;|\; \bY = \by) \; \sepmodel(\bY = \by)
\ee
so that
\beno
\log \, \sepmodel(\bX = \bx, \bY = \by)
\= \log \, \sepmodel(\bX = \bX \;|\; \bY = \by) + \log \, g(\by),
\ee
as $g(\by)$ is the marginal probability mass function of $\bY$,
i.e.,
$\sepmodel(\bY = \by) = g(\by)$. 
Lemma \ref{lem:s_hetero} establishes that $\sepmodel(\bX = \bX \;|\; \bY = \by)$ belongs to a minimal exponential family,
completing the proof of the third and last result of the proposition.  

\qed 
