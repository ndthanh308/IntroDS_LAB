\hide{
% Figure environment removed
}

\begin{table}
\begin{center}
\caption{\label{t_datasum} Summary of Lazega's corporate law partnership data with 71 lawyers (nodes).  }
\begin{tabular}{| c | c | c |}
\hline
 & Average Node Degree & Number of Edges \\
\hline
 Co-Worker Layer  & 11 & 378  \\
\hline
Advice Layer & 5 & 175  \\
\hline
Friendship Layer & 5 & 176  \\
\hline
\end{tabular}
\end{center}
\end{table}
We present a case study using a dataset on corporate law partnership among a Northeastern US corporate law firm in New England collected by \citet{Lazega2001}. The dataset collected information about three types of cooperation among 71 lawyers in the corporate law firm, resulting in three networks including the strong-coworker network, the advice network, and the friendship network. Since the cooperation relationship collected are not symmetric, we only consider a connection to be present when both sides acknowledged their cooperation. We treat these three types of networks as a three-layer multilayer network embedded among the 71 lawyers. A summary of this multilayer network is provided in Table \ref{t_datasum}. We apply model \eqref{general model} with up to 2-layer interaction terms to the Lazega dataset, i.e., $\nat = (\theta_{1},\,\theta_{2},\,\theta_{3},\,\theta_{1,2},\,\theta_{1,3}, \,\theta_{2,3})$. The maximum pseudolikelihood estimator $\mple$ is obtained for $\nat$ and the results are provided in Table \ref{t_Lazeg_mple}.

\begin{table}
\begin{center}
\caption{\label{t_Lazeg_mple}MPLEs for parameters of the Lazega network.}
\begin{tabular}{| c | c | c | c | c | c || c |} 
\hline
$\widetilde\theta_{1}$ &$\widetilde\theta_{2}$ &$\widetilde\theta_{3}$ & $\widetilde\theta_{1,2}$ & $\widetilde\theta_{1,3}$ & $\widetilde\theta_{2,3}$ & $\mbP(Y_{i,j} = 1)  $ \\ 
\hline
$-1.450$ & $-3.334$ & $-2.695$ & $1.801$ & $0.218$ & $2.458$ & 0.208 \\
\hline
Coworker (C) & Advice (A) & Friendship (F) & C $\times$ A & C $\times$ F & A $\times$ F & Basis network $\bY$\\
\hline
\end{tabular}
\end{center}
\end{table}


As shown in Table \ref{t_Lazeg_mple} of the MPLE of the Lazega network data, $\theta_{1},\, \theta_{2}$, and $\theta_{3}$ correspond to single-layer effects of the strong-coworker network, the advice network, and the friendship network, respectively,
whereas $\theta_{1,2}, \, \theta_{1,3}$, and $\theta_{2,3}$ correspond to the layer interaction effects.  
We can calculate the conditional log-odds of each edge being present in the multilayer network given the rest of the network. For example, if lawyer $i$ and lawyer $j$ are observed to have an advice relationship and are friends at the same time, the odds of these two lawyers to have a strong-coworker relationship is given by
\beno
\dfrac{\mbP(X_{i,j}^{(C)} = 1 \,|\, \bX_{i,j}^{(A)} = 1, \, \bX_{i,j}^{(F)} = 1)}
{\mbP(X_{i,j}^{(C)} = 0 \,|\, \bX_{i,j}^{(A)} = 1, \, \bX_{i,j}^{(F)} = 1)} 
\= \exp\left(
\theta_{1} +  \theta_{1,2} \, x_{i,j}^{(A)} + \theta_{1,3} \, x_{i,j}^{(F)}\right) = 1.767,
\ee
providing interpretation of the interaction and influence  among the different layers.

% Figure environment removed

Next, 
we use the estimated MPLE $\mple$ to simulate networks of the same size and calculate the sufficient statistics 
of the simulated networks.   
Comparisons of the sufficient statistics between the observed Lazega network and the simulated networks are provided in 
Figure \ref{Figure_Lazega}. 
Such comparisons serve two key purposes. 
First, 
such comparisons are an established method of diagnosing model fit in the statistical network analysis literature 
\citep{HuGoHa08}, 
and second, 
provide a check on the approximate solution to the score equation.
Note that MPLEs are not guaranteed to reproduce (on average) observed values of sufficient statistics in exponential families---in contrast to MLEs.
The relative $\ell_2$-error of the sufficient statistics between the observed and the average of 10 simulated networks is $0.013$, suggesting a successful re-construction of the observed network.



