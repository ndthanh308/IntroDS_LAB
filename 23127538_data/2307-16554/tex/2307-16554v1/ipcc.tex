\documentclass[preprint,12pt,3p]{elsarticle}
\journal{Economic Analysis and Policy}

\usepackage{matlab-prettifier}
\usepackage{lineno}
%\linenumbers
\usepackage{graphicx}
\usepackage{amsmath}
\usepackage{xfrac}
%\usepackage{subfigure}
\usepackage{parskip}
%\usepackage{epsfig}
\usepackage[backref=page]{hyperref}
\usepackage{adjustbox}
\usepackage{setspace}
\hypersetup{ 
backref=true, % link backreferences (i.e. citations to their list)
bookmarks=true, % show bookmarks bar?
unicode=false, % non-Latin characters in Acrobats bookmarks
pdftoolbar=true, % show Acrobats toolbar?
pdfmenubar=true, % show Acrobats menu?
pdffitwindow=false, % window fit to page when opened
pdfstartview={FitH}, % fits the width of the page to the window
pdftitle={}, % title
pdfauthor={}, % author
pdfsubject={}, % subject of the document
pdfkeywords={}{}, % list of keywords
pdfnewwindow=true, % links in new window
colorlinks=true, % false: boxed links; true: colored links
linkcolor=BlueViolet, % colour of internal links
citecolor=maroon, % colour of links to bibliography
urlcolor=blue,
}
\usepackage{comment}
\usepackage{subfig}
\usepackage{caption}
%\usepackage{subcaption}
\usepackage{longtable}
\usepackage{amssymb}
\usepackage{gensymb}
\usepackage{siunitx}

\usepackage{natbib}
\bibliographystyle{plainnat}
\biboptions{comma,round}
\setcitestyle{authoryear}

\begin{document}
%\begin{spacing}{1.5}
\begin{frontmatter}

\title{The fiscal implications of stringent climate policy}

%\begin{comment}
\author[label1,label2,label3,label4,label5,label6,label7]{Richard S.J. Tol\corref{cor1}\fnref{label8}}
\address[label1]{Department of Economics, University of Sussex, Falmer, UK}
\address[label2]{Institute for Environmental Studies, Vrije Universiteit, Amsterdam, The Netherlands}
\address[label3]{Department of Spatial Economics, Vrije Universiteit, Amsterdam, The Netherlands}
\address[label4]{Tinbergen Institute, Amsterdam, The Netherlands}
\address[label5]{CESifo, Munich, Germany}
\address[label6]{Payne Institute for Public Policy, Colorado School of Mines, Golden, CO, USA}
\address[label7]{College of Business, Abu Dhabi University, UAE}

\cortext[cor1]{Jubilee Building, BN1 9SL, UK}
\fntext[labe8]{This paper was presented at 7th Shanghai-Edinburgh-London Green Finance Conference, the 6th Ethical Finance and Sustainability Conference, and the Commodity and Energy Markets Association Annual Conference 2023.}

\ead{r.tol@sussex.ac.uk}
\ead[url]{http://www.ae-info.org/ae/Member/Tol\_Richard}
%\end{comment}

\begin{abstract}
Stringent climate policy compatible with the targets of the 2015 Paris Agreement would pose a substantial fiscal challenge. Reducing carbon dioxide emissions by 95\% or more by 2050 would raise 7\% (1-17\%) of GDP in carbon tax revenue, half of current, global tax revenue. Revenues are relatively larger in poorer regions. Subsidies for carbon dioxide sequestration would amount to 6.6\% (0.3-7.1\%) of GDP. These numbers are conservative as they were estimated using models that assume first-best climate policy implementation and ignore the costs of raising revenue. The fiscal challenge rapidly shrinks if emission targets are relaxed. 
\textit{Keywords}: climate policy\\
\medskip\textit{JEL codes}: H20, Q54
\end{abstract}

\end{frontmatter}

\newpage

\section{Introduction}
Much has been written about how to reduce greenhouse gas emissions and how much that would cost \citep[see][for a review of recent studies]{Riahi2022IPCC} but there is little about the implications for the public finances. This is an odd omission. Rapid emission reduction requires a major overhaul of the energy sector and energy-intensive activities \citep{IEA2021}. The energy transition will not just affect energy but everything it touches, including tax revenue and government spending. \citet{IEA2022}, for instance, reports that investment in the energy sector needs to double between 2020 and 2030, from 2\% to 4\% of GDP. This paper uses results from commonly-used integrated assessment models to study the impact of stringent climate policy on tax revenue and public expenditure, revealing the potential size of the carbon industry in the process.

The climate economics literature has focused on how best to reduce emissions \citep{Dubash2022IPCC} and what that would cost \citep{Riahi2022IPCC}. Much attention has been paid to the technical feasibility of rapid emission reduction \citep{Clarke2009} and to the required transition of energy, agriculture, and transport. The accompanying changes in the public sector have been largely ignored, with one exception, namely how to best use the revenues from a carbon tax (or permit auction). Using such revenue to reduce other taxes, which hold back economic growth or job creation, could result in a double dividend \citep{Goulder1995} or, if the income distribution improves too, a triple dividend \citep{vanHeerden2006}; \citet{Distefano2023} explore a quadruple dividend, adding public debt to the mix.

The multiple-dividend literature is focused on the structure of tax revenue, but ignores its size. Indeed, for analytical clarity, these papers \emph{assume} budget-neutrality. \citet{Belfiori2023} show that revamped consumption, energy, and income taxes can be a first-best policy, correcting the climate externality without an explicit Pigou tax. However, \citet{Tol2012CCL} argues that stringent climate policy may well require an overall increase in tax revenue and so lead to an expansion of the state. 

\citet{Tol2012CCL} defines the \emph{Leviathan tax} as that carbon tax whose revenue could replace the revenue of all other taxes combined.\footnote{Note that no assumptions are made on the desirable level of total tax revenue.} Figure \ref{fig:leviathan} shows the Leviathan tax for 2019. It is calculated as the greenhouse gas emission intensity of the economy\textemdash emissions over output\textemdash times the tax revenue as a share of GDP. Data are available from the World Bank for 145 countries. Figure \ref{fig:leviathan} ranks these countries by their Leviathan tax, and plots this against their share in global emissions. The Central African Republic has the lowest Leviathan tax: A carbon tax of \$8/tCO\textsubscript{2eq} would be budget-neutral if all other taxes are abolished. Sweden has the highest Leviathan tax: \$3,263/tCO\textsubscript{2eq}. The global average is \$242/tCO\textsubscript{2eq}

The Sixth Assessment Report of Working Group III of the Intergovernmental Panel on Climate Change \citep{Riahi2022IPCC} reports that, according to the median model, a carbon tax of around \$100/tCO\textsubscript{2eq} is needed in 2030 to have a good chance of meeting the 2\celsius target of the 2015 Paris Agreement. India's Leviathan tax (for 2019) is \$95/tCO\textsubscript{2eq}, China's \$96/tCO\textsubscript{2eq}, and Indonesia's \$102/tCO\textsubscript{2eq}. Stringent climate policy is therefore not just a technical and economic challenge, but a fiscal challenge too.

Fiscal problems would arise long before the Leviathan tax is reached. \citet{Besley2013} show that fiscal capacity has grown slowly and that the structure of tax revenues has developed gradually. Rapid, massive change in tax collection is unprecedented and would be difficult, or so the historical record suggests. Climate policy would require two tax revolutions. First, taxes should shift to carbon from everything else to drive emissions to zero\textemdash and then taxes would have to shift back to maintain tax revenue.

\citet{Dowlatabadi2000} was perhaps the first to warn about possible tax revolts \citep{Burg2004, Keen2021} in the context of climate policy. One example is the 2018 protests by \textit{les gilets jeunes} in France in response to a modest carbon tax on transport fuels \citep{STOLL2021}. The carbon taxes needed to meet the Paris targets are not modest\textemdash and they will need to apply in countries that are not as used to high taxes as France is.

Throughout the paper, I write about climate policy as if a carbon tax were the sole policy instrument. The reason for this is that the models I rely on make this assumption. Although the optimal climate policy is a carbon tax, a uniform carbon tax, and nothing but a carbon tax \citep{Tol2023bk}, the bulk of past and present climate policies rely on other instruments. There is no reason to assume future climate policy will be any different.

Some of the insights carry over. Cap-and-trade with auctioned permits behaves much like a carbon tax, the key difference being that permit prices fluctuate and taxes do not. The revenue of permit auctions can be used to reduce taxes.

If permits are grandparented instead of auctioned, climate policy is like a carbon tax (at the margin) plus lump-sum capital subsidies for the recipients of free permits. These capital subsidies pose no burden on the fiscal budget as the government costlessly creates the permits before giving them away. In this case, taxes cannot be reduced. Instead, the public sector expands.

Subsidies, another popular policy instrument, are negative taxes. Other taxes would need to go up substantially if subsidies are used to reduce emissions at the required scale.

Any technical standard has an equivalent tax \citep{Baumol1971}. If standards are the policy instrument of choice\textemdash as they often are\textemdash the tax burden calculated below is a measure of the changes needed in the economy. Fiscal implications would be indirect.

More troublesome than the assumption of a carbon tax is the assumption, again taken from the models I rely on, that climate policy will be cost-effective.\footnote{This paper shies away from a discussion of optimal climate policy targets, which are treated extensively elsewhere \citep{Nordhaus1992, Tol1999kyoto, Tol2012EP, Tol2023NCC}.} Current climate policy most definitely is not \citep[e.g.,][]{Grimm2022}. However, this strengthens the argument below. If \emph{cost-effective} policy implies unrealistically large fiscal shocks, then \emph{sub-optimal} policy (with the same emissions target) implies even larger shocks. Admittedly, without a carbon tax, those shocks may not be to the public finances; they will be to the economy instead.

The paper proceeds as follows. Section \ref{sc:methods} discusses the materials and methods used. Section \ref{sc:results} presents the results. Section \ref{sc:conclude} concludes.

\section{Materials and methods}
\label{sc:methods}
The \href{https://data.ece.iiasa.ac.at/ar6/}{IPCC AR6 scenario database} contains projections of GDP, greenhouse gas emissions, carbon dioxide sequestration, and emission taxes for a range of \textit{ex-ante} models and a range of scenarios with and without emission reduction targets. The database contains a host of variables on the structure of energy demand and supply, agriculture, land use, and so on. I here only use GDP, gross carbon dioxide emissions, gross carbon uptake, and carbon taxes. For most models, results are reported for 10-year intervals until 2100.

While generally well-structured, the database, unfortunately, does not match baseline and policy scenarios; this was added, manually, based on scenario names. Missing rows were replaced by missing observations. This then leads to the percentage reduction of GDP and emissions from baseline.

Total carbon tax revenue (subsidy) follows from multiplying gross carbon dioxide emissions (sequestration) with carbon taxes. 

As highlighted by \citet{Riahi2022IPCC}, the models in the database show a wide range of results. This is not a surprise, as the models have different structures and use different assumptions on economic growth, on relative prices, on technological change, on income, price and substitution elasticities, and on reserves, resources and potentials. Some models are computable general equilibrium models, others energy system models, and yet others are growth, econometric or new Keynesian models. All models have some foresight, many perfect foresight. The only commonality is that all models have been used to study \emph{future} climate policy.

Note that I do not correct the IPCC database for reporting bias \citep{Tavoni2010}. This omission likely leads to an underestimate of the true cost of climate policy.

I follow \citet{Tol2014en} and compare these \textit{ex-ante} models to the data, but where \citet{Tol2014en} relied on a fairly basic statistical analysis, I here use five advanced econometric studies of the efficacy of carbon pricing \citep{Rafaty2020, Kohlscheen2021, Sen2018, Metcalf2020, Best2020}. These \textit{ex-post} studies use different estimators and different samples, but they all study the effect of \emph{past} climate policy on past emissions. The efficacy of a carbon tax is here defined as the percentage emission reduction per dollar per tonne of carbon dioxide carbon tax. This measure is reported by, or easily derived from the five econometric studies. It is also readily calculated from the data in the IPCC AR6 scenario database.

I use Bayesian statistics to assess the credibility of the different models. I use a non-informative prior. The results of the econometric models are the likelihood. Combined, this gives the posterior estimate of the tax efficacy. Alternatively, I shrunk the five estimates to a single, combined one \citep{Goldberger1964}. In a second step, as a prior, I assumed that each IPCC model is equally likely. The posterior likelihood of the tax efficacy implies a probability that an \textit{ex-ante} model is able to reproduce observed climate policy as measured by the \textit{ex-post} models. 

While the methods are well-established, this is their first application to the fiscal implications of stringent climate policy.

\section{Results}
\label{sc:results}

\subsection{Model skill}
Before discussing the key results, I need to establish which model is most credible. This is because the range of model range is so large. Some models find that climate policy is too cheap to meter, others that it would lead to economic ruin.

Table \ref{tab:taxefficacy} shows the efficacy of a carbon price for the 24 models in the \href{https://data.ece.iiasa.ac.at/ar6/}{IPCC AR6 scenario database} for which this information was available. Tax efficacy is the percentage CO\textsubscript{2} emission reduction (from baseline) in 2030 divided by the carbon tax or permit price in the same year. (Recall that the models assume foresight.) Efficacy differs by \emph{three} orders of magnitude from 0.0042\%/\$ for \textsc{ices} to 4.8\%/\$ for \textsc{coffee}.

At the bottom of \ref{tab:taxefficacy}, five econometric estimates of the same metric are shown \citep{Rafaty2020, Kohlscheen2021, Sen2018, Metcalf2020, Best2020}. Three of these studies agree that a carbon price of \$1/tCO\textsubscript{2} would cut emissions by some 0.1\%, higher than 2 of the 24 IPCC models and lower than 21. The other two econometric studies find that carbon pricing is more effective. The minimum and maximum differ by one order of magnitude.

The posterior mean, weighted average, or shrunk estimate is a reduction of 0.13\% per dollar per metric tonne of carbon dioxide. This implies, assuming linearity, that a carbon tax of \$792/tCO\textsubscript{2} would fully decarbonize the world economy.

The short-run Leviathan tax is discussed in the introduction. It assumes that the imposed carbon tax does not affect emissions. Figure \ref{fig:leviathan} also shows the long-run Leviation tax, using the central estimate of 0.13\% emission reduction per dollar carbon tax. The Leviathan tax increases, but not sufficiently so that the IPCC's \$100/tCO\textsubscript{2} carbon tax looks materially less problematic. 

Only the \textsc{imaclim} model \citep{Crassous2006, Sassi2010, Waisman2012, Bibas2015, Mejean2019} is close to the majority of the empirical evidence.\footnote{I have criticized this model for having so many distortions that it is hard to interpret the results. That said, the economy is full of distortions.} Indeed, 95.5\% of the posterior probability mass goes to \textsc{imaclim}. The posterior probability of \textsc{gemini} is 0.5\%. The probabilities of the remaining models are very small.

\subsection{The impact of stringent climate policy}
Table \ref{tab:taxes} shows the main result. Twelve models in the IPCC AR6 database report scenarios that cut global carbon dioxide emissions by 95\% or more in 2050. Table \ref{tab:taxes} shows the carbon price and the value of carbon capture and emissions, all averaged across the scenarios for each of the models. The carbon price is either the explicit carbon tax, the price of tradable permits, or the shadow price of the emissions constraint. The value of emissions is the total revenue of either a carbon tax or the auction of carbon permits. The value of carbon capture is either the total expenditure on carbon removal subsidies or the sum total spent on carbon offsets. Both values are given as a share of GDP.

The results vary widely. The most optimistic model is again the \textsc{coffee} model. As in Table \ref{tab:taxefficacy}, this model finds that a minimal carbon tax would completely decarbonize the economy. Revenues and expenditures are therefore small too. At the other extreme, \textsc{dne21} has a carbon tax revenue of 3 times GDP, and on top spends 2 times GDP on carbon removal. One would hope this is a reporting error rather than a genuine result of what would be a mistaken model.

Discarding the two outliers, carbon tax revenue ranges from 1 to 17\% of GDP. This range is wide. A tax reform that brings in 1\% of GDP by 2050 is feasible. Tax reforms at this scale happen regularly \citep{OWID2016}. The high end of the range is more difficult. The global average tax revenue was 14\% of GDP in 2019.\footnote{See \href{https://data.worldbank.org/indicator/GC.TAX.TOTL.GD.ZS?view=chart}{World Bank}.} An expansion of the public sector by 3\% in 30 years is doable. Reducing if not abolishing all other taxes would, of course, be an election winner\textemdash although taxes are rarely abolished \citep{Seelkopf2021}. However, as emissions approach zero, the tax base would get narrower and narrower and the carbon tax higher and higher, so that the fiscal system becomes increasingly distortionary. As emissions go to zero, so does carbon tax revenue\textemdash other taxes will have to be reintroduced, a politically more challenging prospect.\footnote{In \textsc{GCAM}, emissions fall to zero before 2050. Its fiscal transition is even faster.} \textsc{imaclim}, the most credible model, has total carbon tax revenues at 7\% of GDP in 2050, replacing ``only'' half of all other taxes (if government budget neutrality is assumed).

Total carbon removal subsidies, or payments for offsets, range from 0.3\% (\textsc{aim}) to 7\% (\textsc{grape}) of GDP. The model that compares best to the data, \textsc{imaclim}, is at the high end of this range. A subsidy that is a few tenths of a percent of GDP is no problem. Climate change has been a key concern of many people around the world for decades \citep{Leiserowitz2006, Lee2015, Rettig2023}\textemdash the vocal protests of a small minority notwithstanding. Spending a small fraction of income on solving the climate problem should not be a problem. However, expenditure is much larger at the high end of the range, roughly equal to expenditures on health care. Public spending on health care is like motherhood and apple pie\textemdash we all rely on doctors and nurses to heal ourselves and our loved ones, and we all have friends and family who work in medicine and who deserve a decent salary. Carbon capture is very different. It solves a distant and abstract problem, rather than one that is close and obvious like ill-health. If climate policy is successful, there is not much of a problem to solve anymore, making it harder to continue to justify spending large sums of money. In order to keep costs down, carbon capture will be done where land is cheap\textemdash that is, where few people live\textemdash and heavily mechanized. Paying 7\% of your income in taxes to keep grandma alive and your nurse friend in work is one thing. Paying 7\% to a multinational company to suck carbon dioxide out of the air in a faraway country is something else.

\subsection{Regional results}
The above results are for the world as a whole. The models in the IPCC database also report regional results. I restrict the attention to \textsc{imaclim} and one particular scenario which reduces emissions by 94\% in 2050. The carbon tax is \$300/tCO\textsubscript{2} in 2030, rising to \$1,298/tCO\textsubscript{2} in 2040 and \$2,253/tCO\textsubscript{2} in 2050. Figure \ref{fig:regions} shows carbon tax revenue and sequestration subsidy, as a percentage of GDP, for 2030, 2040, and 2050.

Global carbon tax revenue is 4\% of GDP in 2050, a reasonable number, but 11\% in 2030 and 19\% in 2040\textemdash underlining yet again the fiscal challenge posed by stringent climate policy.

The results in Figure \ref{fig:regions} are ordered by per capita income in 2010. Carbon tax revenue is below the global average in the three richest regions, but above the global average in the seven poorest regions\textemdash with the exception of almost completely decarbonized India in 2050. The carbon tax revenue is very high in the carbon-intensive economies of the Middle East and the former Soviet Union.

The bottom panel of Figure \ref{fig:regions} shows the sequestration subsidies. The world total is 0.04\% of GDP in 2030, rising to 3.8\% in 2040 and 15\% in 2050. As with tax revenues, the numbers are lower for the three rich regions and higher for the seven poor regions. Note, however, that it may well be that there will substantial transfers between regions. This is less likely with direct subsidies, more likely with tradable permits and offsets.

That said, Figure \ref{fig:regions} highlights the scale of the activity. The sequestration sector would occupy almost 15\% of the world economy, over 35\% of the economy in the former Soviet Union.

\subsection{Results for more lenient climate policy}
The above results are for very stringent climate policy. Cutting carbon dioxide emissions by 95\% or more by 2050 is highly ambitious. The major fiscal implications highlighted above rapidly disappear for less stringent climate policy. This is because the fiscal implications are the product of carbon price and emissions. Take the subsidies for carbon dioxide removal first. A more lenient target would mean a lower volume at a lower price. The carbon tax revenue would fall too: Emissions would be higher but the carbon price lower; the former is linear, the latter exponential.

Figure \ref{fig:imaclim} illustrates this for the \textsc{imaclim} model for 2050. The top left panel plots the carbon price against emission reduction from baseline. The carbon price inches up until emissions are halved and then starts rising very quickly. However, emissions, shown in the bottom left panel, continue to fall steadily. Sequestration, in the bottom right panel, similarly shows no profound non-linearity. The top right panel shows the drop in GDP, which accelerates around a 50\% emission reduction. This accentuates carbon tax revenue and carbon sequestration expenditures relative to GDP.

\section{Discussion and conclusion}
\label{sc:conclude}
Stringent climate policy would pose a substantial fiscal challenge. The global revenue of the carbon tax needed to meet the targets of the 2015 Paris Agreement would be larger than the revenue of all other taxes combined, while a very large subsidy would need to be paid to remove carbon dioxide from the atmosphere. Tax revenues are larger still in poor parts of the world. Climate policy by other means than taxes and subsidies would shift, perhaps hide, probably exacerbate the fiscal burden.

The fiscal challenge rapidly shrinks as the emission reduction target becomes less stringent. The policy implication is thus to adopt a more lenient climate policy\textemdash or rather, as the gap between nominal targets and actual climate policy had widened \citep{UNEP2022}, to adopt more realistic rhetoric.

The implications for research are more profound. Model results show a very large range for the costs of future climate policy. This is partly inevitable. The future is inherently uncertain. However, the skill of \textit{ex-ante} models can be tested against over 30 years of experience with actual climate policy. This is here done with a single variable, tax efficacy, but these models generate many more variables, most of which are directly observed.

Even before testing their skills, two of the models in the IPCC database report patent nonsense. Either the database or the models need to be vetted better. The problems do not stop there. Many of the integrated assessment models used by the IPCC do not have a rich representation of the fiscal system\textemdash and none report this. Environmental regulation and general taxation interact \citep{Sandmo1975}. Ignoring prior tax distortions leads to unnecessarily expensive climate policy \citep{Barrage2019}. A tax is more distortionary as it rises and its base narrows\textemdash exactly what happens as emissions approach zero. Ignoring the excess burden in the climate policy endgame seems to be a crucial omission in integrated assessment models.

Unlike tax distortions, tax revolts are unpredictable\textemdash but the probability of tax revolts varies systematically with observable variables \citep{Dowlatabadi2000}. Dynamic stochastic general equilibrium models are now regularly used to study climate policy \citep{Cai2019, Bremer2021}. It strikes me that tax revolts are a key stochastic element. Tax revolts may be more likely if the costs of climate policy are distributed in a way that is seen to be unfair \citep{Chepeliev2021, Landis2021, Vandyck2021, Bohringer2022, Wu2022} and if assets are stranded and firms go bankrupt \citep{Davis2010, Tong2019, Ploeg2020jeem, Semieniuk2022, Flora2023}.

All this complicates climate policy and makes it more expensive. Adding the analytically convenient but unrealistic assumption of first-best policy implementation, it appears that policy-makers are ill-advised by the IPCC and its choice of models. More importantly, current emission reduction targets may need to be relaxed.

\bibliography{master}

\begin{table}[p]
    \centering
     \caption{Carbon tax efficacy according to 24 \textit{ex-ante} models and 5 \textit{ex-post} policy evaluations.}
    \label{tab:taxefficacy}
    \begin{tabular}{l r r r r} \hline
model & \# & mean & st.err. & prob. \\ \hline
\textsc{coffee}	&	63	&	4.883\%	&	0.584\%	& 0.000\\
\textsc{aim}	&	123	&	1.103\%	&	0.352\%	& 0.000\\
\textsc{image}	&	81	&	0.802\%	&	0.128\%	& 0.000\\
\textsc{remind}	&	286	&	0.705\%	&	0.045\%	& 0.000\\
\textsc{witch}	&	142	&	0.646\%	&	0.028\%	& 0.000\\
\textsc{gcam}	&	47	&	0.612\%	&	0.082\%	& 0.000\\
\textsc{gem-e3}	&	49	&	0.604\%	&	0.025\%	& 0.000\\
\textsc{message}	&	258	&	0.566\%	&	0.042\%	& 0.000\\
\textsc{poles}	&	134	&	0.544\%	&	0.046\%	& 0.000\\
\textsc{farm}	&	12	&	0.529\%	&	0.060\%	& 0.000\\
\textsc{prometheus}	&	6	&	0.442\%	&	0.065\%	& 0.000\\
\textsc{eppa}	&	4	&	0.373\%	&	0.040\%	& 0.000\\
\textsc{bet}	&	14	&	0.367\%	&	0.054\%	& 0.000\\
\textsc{grape}	&	17	&	0.331\%	&	0.042\%	& 0.000\\
\textsc{en-roads}	&	2	&	0.318\%	&	0.007\%	& 0.000\\
\textsc{dne21}	&	34	&	0.317\%	&	0.030\%	& 0.000\\
\textsc{muse}	&	6	&	0.238\%	&	0.099\%	& 0.000\\
\textsc{c3iam}	&	4	&	0.228\%	&	0.008\%	& 0.000\\
\textsc{tiam-ucl}	&	5	&	0.225\%	&	0.051\%	& 0.000\\
\textsc{tiam-ecn}	&	58	&	0.202\%	&	0.028\%	& 0.000\\
\textsc{gemini}	&	5	&	0.165\%	&	0.051\%	& 0.005\\
\textsc{imaclim}	&	51	&	0.121\%	&	0.021\%	& 0.995\\
\textsc{env-linkages}	&	13	&	0.005\%	&	0.006\%	& 0.000\\
\textsc{ices}	&	6	&	0.004\%	&	0.001\%	& 0.000\\ \hline
Average	&	24	&	0.597\%	&	0.194\%	& \\
Weighted average	&	24	&	0.009\%	&	0.001\%	& \\ \hline
\citet{Rafaty2020}	&		&	0.110\%	&	1.779\%	& \\
\citet{Metcalf2020}	&		&	0.125\%	&	0.013\%	& \\
\citet{Kohlscheen2021}	&		&	0.130\%	&	0.030\%	& \\
\citet{Sen2018}	&		&	0.730\%	&	0.640\%	& \\
\citet{Best2021}	&		&	2.960\%	&	0.987\%	& \\ \hline
Weighted average	&		&	0.126\%	&	0.012\%	& \\ \hline
    \end{tabular}
\caption*{\scriptsize For the selected 24 IPCC models, the table shows the number of emission reduction scenarios and the mean and its standard error of the carbon tax efficacy, that is, the carbon dioxide emission reduction in 2030 divided by the carbon tax levied in 2030. The posterior probability that the model agrees with the empirical studies in the bottom rows is in the right-most column. The table also shows the average across models and the average weighted by the inverse of the squared standard error. For the 5 empirical studies, mean and standard error of the estimated carbon efficacy are shown, as well as the weighted average across studies.}
\end{table}

\begin{table}[p]
    \centering
        \caption{Value of carbon capture and emissions as share of GDP in 2050.}
    \label{tab:taxes}
    \begin{tabular}{l r r r} \hline
model	& tax &	sequestration	&	emissions	\\
 & \$/tCO\textsubscript{2} & \%GDP & \%GDP \\ \hline
\textsc{coffee}	& 3 &	-0.07	&	0.20	\\
\textsc{aim}	& 119 &	-0.29	&	1.73	\\
\textsc{gem-e3}	& 385 &	-0.30	&	1.07	\\
\textsc{gcam}	& 1720 &	-0.31	&	-4.21	\\
\textsc{remind}	& 537 &	-1.76	&	2.66 \\
\textsc{image}	& 586 &	-2.26	&	3.35 \\
\textsc{message} & 823	&	-2.26	&	4.83	\\
\textsc{witch}	& 1204 &	-3.08	&	5.84	\\
\textsc{poles}	& 4601 &	-4.09	&	17.08	\\
\textsc{imaclim}	& 913 &	-6.56	&	7.41 \\
\textsc{grape}	& 1196 &	-7.09	&	20.58	\\
\textsc{dne21}	& 977 &	-230.08	&	301.22	\\ \hline
    \end{tabular}
\caption*{\scriptsize For the selected 10 IPCC models, the table shows the gross carbon tax revenue and the total subsidy for carbon dioxide sequestration for 2050, both as a share of Gross Domestic Product. The carbon tax is shown too.}
\end{table}

% Figure environment removed

% Figure environment removed

% Figure environment removed

\end{document}