\section{Lower Bounds}\label{sec:lower}

%All the approximation guarantees presented in the previous section may appear weak at first.
Every algorithm presented in Section \ref{sec:upper} is value-agnostic. It might seem reasonable to assume, therefore, that there are more powerful approximation schemes that exploit the numerical values in $H$ in some way. Remarkably, we show in this section that this is \emph{not} the case, and value-agnostic algorithms are strong enough to give us nearly optimal approximation guarantees. Specifically, we show inapproximability results matching the upper bounds (up to $\polylog$ factors) in Section \ref{sec:upper} for almost every class of graphs considered. Our lower bounds will use reductions from the {\UTP} problem on carefully constructed gadgets.

\begin{definition}[\TP]
Given a multiset of $3m$ positive integers $A = \{a_1, \dots, a_{3m}\} \subseteq \mathbb{N}_{> 0}$ and a positive integer $T \in \mathbb{N}_{> 0}$ such that $\sum_{j \in [3m]} a_j = mT$, {\TP} asks whether the multiset $A$ can be partitioned into $m$ triplets $(S_1, S_2, \dots, S_m)$ such that the sum of each triplet is equal to $T$.
\end{definition}

The \TP problem is NP-complete even when all the inputs are given in unary and each of the items in $A$ are strictly between $\ffrac{T}{4}$ and $\ffrac{T}{2}$ \citep{garey1979computers}. We refer to this variant as {\UTP}. Note that but {\UTP} is just a reformulation of \textsc{Bin Packing}: there are $3m$ integers that sum to $mT$, and we wish to fit these integers into $m$ bins each of capacity $T$. The condition of three integers in each bin does not need to be stated separately, as it is implied by the constraint that each integer is strictly between $T/4$ and $T/2$.

We present our lower bounds in increasing order of proof complexity, starting with trees and moving to general graphs, planar graphs, and bounded-degree graphs. We give the complete proof only in the first case, relegating the remaining proofs to the appendix and providing proof sketches here instead.

\subsection{Trees and Planar Graphs}\label{sec:treelowerbound}
Recall that we presented two approximation guarantees for trees, $O(n)$ (Proposition \ref{prop:trivialgeneral}) and $O(\Delta \log n)$ (Corollary \ref{cor:cutwidth-upperbounds}). Both of these results are $\tilde{O}(n)$ in the worst case. 

\begin{theorem}\label{thm:trees-approx-lower-bound}
For any constant $\varepsilon > 0$, there is no efficient $O(n^{1-\varepsilon})$ approximation algorithm for {\GHA} on depth-$2$ trees unless P = NP.
\end{theorem}
\begin{proof}
% Figure environment removed

We give a reduction from \UTP. For some constant $\varepsilon > 0$, assume there is an efficient $O(n^{1- \varepsilon})$ approximation algorithm $\mathsf{ALG}_{\cal G}$ where $\cal G$ corresponds to the class of depth-$2$ trees. In other words, there is a constant $\gamma$ such that for all instances with $G \in \mathcal{G}$, $\mathsf{ALG}_{\cal G}$ outputs an allocation with total envy within a multiplicative factor of $\gamma n^{1- \varepsilon}$ to the optimal envy.

Given an instance of \UTP, we construct an instance $(G, H)$ of {\GHA} as follows: The graph $G$ is a rooted depth-$2$ tree where the root $r$ of the tree has $3m$ children $\{x_1, \dots, x_{3m}\}$. Each of these nodes $x_i$ has $C a_i - 1$ children (see Figure \ref{fig:trees-reduction-tree}). $C$ is a positive integer whose exact value we shall decide later.

The total number of nodes in $G$ is $1 + \sum_{i \in [3m]} Ca_i = 1 + CmT$, and so we must specify $1 + CmT$ house values in $H$. We define $H$ with $CT$ values of $i$ for $i \in [m]$, together with a single value of $0$. Note that this construction can be done in polynomial time as long as $C$ is polynomially large, since the input to the $3$-partition instance is given in unary. 

We show that, for an appropriate choice of $C$, the minimum total envy output by $\mathsf{ALG}_{\cal G}$ for the instance $(G, H)$ is at least $\left \lceil (12\gamma m^3 T + 1)^{\ffrac{1}{\varepsilon}} \right \rceil$ if and only if there exists {\em no} valid partition for the original \textsc{3-Partition} instance, i.e., the original instance was a NO instance. 


$(\Rightarrow)$ Assume there is a valid $3$-partition $(S_1, \ldots, S_m)$ for the original instance. We denote this $3$-partition using a mapping $\mu : A \to \{S_1, \ldots, S_m\}$ that maps each number in the multiset $A$ to one of the triplets $S_j$. We construct an allocation for the graph $G$ as follows: for any item $i$, if $\mu(i) = S_j$, we allocate houses with value $j$ to $x_i$ and all of its children. We finally allocate the house with value $0$ to the root. Note that this is a valid allocation: for any $j \in [m]$, we allocate exactly $\sum_{i \in [3m]: \mu(i) = S_j} Ca_i = CT$ houses of value $j$. 

We can easily upper bound the envy of this allocation; this upper bound also serves as an upper bound for the minimum total envy for the instance $(G, H)$. There is no envy between any $x_i$ and any of its children. So the only edges with potential envy are the ones incident on the root. There are $3m$ such edges, each incurring envy at most $m$. This gives us an upper bound of $3m^2$ on the total envy. This implies, when there is a valid $3$-partition, the approximation algorithm $\mathsf{ALG}_{\cal G}$ will output an allocation with envy at most $3m^2\gamma(1 + CmT)^{1-\varepsilon}$.

$(\Leftarrow)$ Assume there is no valid $3$-partition in the original instance. We will show that any allocation has a total envy of at least $C$. We do this by examining the houses allocated to the depth-1 nodes $\{x_1, \dots, x_{3m}\}$. If any $x_i$ is allocated a value of $0$ in some allocation $\pi$, then the allocation $\pi$ has a total envy of at least $C$ since any $x_i$ has at least $C -1$ children and $1$ parent receiving a value of at least $1$ each. 

So now assume no $x_i$ is allocated a value of $0$ in $\pi$. For notational convenience, assume without loss of generality that $x_1, \dots, x_{\ell_1}$ are allocated a value $1$, $x_{\ell_1 +1}, \dots, x_{\ell_1 + \ell_2}$ are allocated a value $2$ and so on. If for all $j \in [m]$, $\sum_{h \in [\ell_j]} a_{\ell_0 + \ell_1 + \dots + \ell_{j-1} + h} = T$ (with $\ell_0 = 0$), then we violate our assumption that there is no valid $3$-partition in the original instance. Therefore there exists one $j \in [m]$ such that $\sum_{h \in [\ell_j]}a_{\ell_0 + \ell_1 + \dots + \ell_{j-1} +h} > T$. Assume again for notational convenience that $j = 1$. Since all values are integers, we can restate the inequality above as $\sum_{h \in [\ell_1]}a_{h} \ge T + 1$. This implies $\sum_{h \in [\ell_1]} Ca_{h} \ge CT + C$. 

Coming back to the allocation $\pi$, we have that $\sum_{h \in [\ell_1]} Ca_{h} \ge CT + C$ implies that there are at least $C$ nodes out of all the children of $\{x_{1}, x_2, \dots, x_{\ell_1}\}$ which are not allocated a value of $1$. The envy that each of these $C$ nodes will have towards their parents is at least $1$. This implies that the total envy of allocation $\pi$ is at least $C$. 

We set $C = \left \lceil (12\gamma m^3 T + 1)^{\ffrac{1}{\varepsilon}} \right \rceil$ to complete the reduction. When there is no valid $3$-partition, the total minimum envy (and therefore, the envy output by $\mathsf{ALG}_{\cal G}$) is at least $C =\left \lceil (12\gamma m^3 T + 1)^{\ffrac{1}{\varepsilon}} \right \rceil$. However, when there is a valid $3$-partition, the envy output by $\mathsf{ALG}_{\cal G}$ is strictly upper bounded by:
\begin{align*}
    3m^2\gamma(CmT+1)^{1-\varepsilon} \le 6m^3 \gamma TC^{1-\varepsilon} \le 6m^3 \gamma T \left ( \left \lceil (12\gamma m^3 T + 1)^{\ffrac{1}{\varepsilon}} \right \rceil \right )^{1 - \varepsilon} < \left \lceil (12\gamma m^3 T + 1)^{\ffrac{1}{\varepsilon}} \right \rceil.
\end{align*}
This completes the proof.
\end{proof}


\subsection{General and Bounded-Degree Graphs}

In this section, we generalize the arguments from Section \ref{sec:treelowerbound} to other classes of graphs. We relegate all technical proofs from this section to the appendix, instead providing intuitive proof sketches.

Our first result matches the $O(n^2)$ worst case upper bound for general connected graphs (\Cref{prop:trivialgeneral} and \Cref{cor:cutwidth-upperbounds}). See Appendix \ref{apx:lower} for the full proof.

\begin{restatable}{theorem}{generalgraphs}\label{thm:general-approx-lower-bound}
For any constant $\varepsilon > 0$, there is no efficient $O(n^{2-\varepsilon})$ approximation algorithm for {\GHA} on connected graphs unless P = NP.
\end{restatable}

% Figure environment removed

\begin{proof}[Proof sketch]
    Figure \ref{fig:general-reduction-graph} shows the graph we use for this reduction. The construction is similar to that in Theorem \ref{thm:trees-approx-lower-bound}, except that we use cliques of size $Ca_i$ attached to a common root $r$, instead of a vertices $x_i$ with $Ca_i - 1$ dangling leaves. The idea is to use the high density of the clique to show that the envy is $\Omega(C^2)$ when there is no valid $3$-partition. The house values $H$ are defined as in Theorem \ref{thm:trees-approx-lower-bound}. A similar argument works here too. A valid $3$-partition can be ``packed'' into the clusters of values, which would attain only envy from the edges incident to the root $r$. Conversely, if there is no valid $3$-partition, again using a packing argument, it can be shown that there are several high envy edges, giving rise to $\Omega(C^2)$ envy.
\end{proof}





%\subsection{Bounded Degree Planar Graphs}

So far in our two lower bounds (Theorems \ref{thm:trees-approx-lower-bound} and \ref{thm:general-approx-lower-bound}), we were able to use simple counting techniques, because counting edges with non-zero envy in stars and cliques is straightforward. Our next results will require much more careful analysis.

We will start with bounded-degree planar graphs. Our reduction uses grid graphs instead of stars and cliques, and so we will need a technical lemma (whose proof is in the appendix) to help us with estimating the edges with nonzero envy.

\begin{restatable}{lemma}{gridlemma}\label{lem:grid-graph-property}
    Let $G = Grid(r, c)$ be a grid graph with $r$ rows and $c$ columns such that $r \le c$. Let $A \subseteq V$ be any set of nodes in this graph such that $|A| \le \ffrac{rc}{2}$. Then, $|\delta_G(A)| \geq \min\{\sqrt{|A|}, \ffrac{r}{2}\}$.
\end{restatable}

Armed with Lemma \ref{lem:grid-graph-property}, we can state our lower bound on bounded-degree planar graphs. See Appendix \ref{apx:lower} for the full proof.

\begin{restatable}{theorem}{bdp}\label{thm:bounded-degree-planar-approx-lower-bound}
For any constant $\varepsilon > 0$, there is no efficient $O(n^{0.5-\varepsilon})$ approximation algorithm for {\GHA} on bounded-degree planar graphs unless P = NP.
\end{restatable}

% Figure environment removed

\begin{proof}[Proof sketch]
    Figure \ref{fig:bdp-reduction-graph} shows the graph we use for this reduction. Again, the construction is similar to before. The graph $G$ has $3m$ grids, each with $C$ rows and $Ca_i$ columns, attached by a single edge to a leaf of a ``small'' binary tree $T_r$. Note that this is a bounded-degree planar graph. The values $H$ are defined similarly to before, with a cluster of $C^2T$ values at each positive integer in $[3m]$, and $|T_r|$ values at $0$.

    If there is a $3$-partition, the argument is exactly similar to those earlier in this section, with the only envy coming from the edges of the tree, incurring a total envy of $3m^2$.

    Conversely, suppose there is no $3$-partition. Now, using Lemma \ref{lem:grid-graph-property}, we can show that a grid which does not have a majority of its vertices from the same cluster must incur at least $C/4$ envy. Otherwise, assuming each grid has a majority of its vertices in the same cluster, a packing argument shows that some grid must have a cut going across two different values, and this incurs at least $C/2$ envy, once again by Lemma \ref{lem:grid-graph-property}.
\end{proof}

Note that Theorem \ref{thm:bounded-degree-planar-approx-lower-bound} matches the $O(\sqrt{n})$ upper bound from \Cref{cor:cutwidth-upperbounds}.

Our final lower bound applies to arbitrary bounded-degree graphs and matches the $O(n)$ upper bound from Proposition \ref{prop:trivialgeneral} and Corollary \ref{cor:cutwidth-upperbounds}. In this reduction, we use the recent polynomial-time algorithm to compute bipartite Ramanujan multi-graphs for any even number of vertices $m$ and any degree $d \ge 3$ \citep{cohen2016ramanujan}. At a high level, we replace the grid graphs from Theorem \ref{thm:bounded-degree-planar-approx-lower-bound} with these Ramanujan graphs and use the expansion properties of Ramanujan graphs to prove a lemma similar to (and stronger than) \Cref{lem:grid-graph-property}.

One key element of the construction we will use is that we will take a gadget consisting of a $3$-regular Ramanujan bipartite multi-graph of a given even size (at least $6$), and remove any of its repeated edges until it is a simple graph. The crucial observation is that even though these new gadget graph is neither regular nor Ramanujan, it still has ``enough'' expansion for our purposes. We will need the following well-known result for this, stated without proof.

\begin{lemma}[Cheeger's Inequality]\label{lem:cheegers-inequality}
Let $G'$ be a $d$-regular Ramanujan (multi-)graph defined on a set of $V$ nodes. The following holds:
\begin{align*}
    \min_{S \subseteq V: 0 < |S| \le |V|/2} \frac{\delta_{G'}(S)}{|S|} \ge \frac12(d - 2\sqrt{d-1}),
\end{align*}
where $\delta_{G'}(S)$ denotes the number of edges in the $(S, V \setminus S)$ cut in the graph $G'$.
\end{lemma}

The following theorem characterizes the inapproximability on bounded-degree graphs. See Appendix \ref{apx:lower} for the full proof.


\begin{restatable}{theorem}{boundeddeg}\label{thm:bounded-degree-approx-lower-bound}
For any constant $\varepsilon > 0$, there is no efficient $O(n^{1-\varepsilon})$ approximation algorithm for {\GHA} on bounded-degree graphs unless P = NP.
\end{restatable}

% Figure environment removed

\begin{proof}[Proof sketch]
    Figure \ref{fig:bounded-degree-reduction-graph} shows the graph we use for this reduction. The graph $G$ is constructed as follows. For each $a_i$ in the given {\UTP} instance, construct in polynomial time a $3$-regular Ramanujan bipartite multi-graph of size $Ca_i$ (using a construction by \cite{cohen2016ramanujan}). Remove any repeated edges to convert them into simple graphs. The resulting graphs can be shown to have sufficient amount of expansion properties, using Cheeger's inequality (Lemma \ref{lem:cheegers-inequality}). We can now attach these graphs to the leaves of a sufficiently small binary tree. This is a bounded-degree graph.

    If there is a $3$-partition, we can exhibit a small-envy allocation in the same way as in most of the other proofs in this section.

    % Conversely, if there is no valid $3$-partition, we can once again show that if some gadget has no majority value, then it spans many different values, and by its expansion properties, it has many edges in a cut across these different values, incurring $\Omega(C)$ envy. Otherwise, if each gadget has a majority value, we can use the packing argument once more to show that some gadget has to have a cut going across different values, incurring $\Omega(C)$ envy once again.
    Conversely, if there is no valid $3$-partition, we can show that some of these gadgets are going to be allocated multiple different values. We then use the expansion properties of these gadgets to show that the number of high envy edges within each of these gadgets is $\Omega(C)$.
\end{proof}




