\section{Conclusions}\label{sec:conclusions}

We explored the approximability of {\GHA}, presenting tight approximation algorithms for several classes of connected graphs, to our knowledge the first such results in the area. In particular, we gave polynomial-time algorithms exploiting graph structures to approximate the optimal envy on general graphs, trees, planar graphs, bounded-degree graphs, and bounded-degree planar graphs; for each of these classes, we also gave a matching lower bound. Our algorithms
% upper bounds 
% are straightforward algorithms that run in polynomial time, and are in fact
are value-agnostic, i.e., they take into account only the input graph and the ordering among the house values but not the values themselves. We also refuted a recent conjecture about the structural properties of optimal allocations on complete binary trees, but gave a value-agnostic algorithm to show a $3$-approximation on all such instances.

The following open question is an important one in further understanding the boundary between {\GHA} and {\MLA}: \cite{canon} showed that {\GHA} is NP-hard even for disjoint unions of paths, whereas \cite{mlatrees} showed {\MLA} is exactly solvable in polynomial time on forests. The following question asks us whether connectivity buys us anything in {\GHA}.
\begin{open}
    What is the complexity of {\GHA} on bounded-degree trees?
\end{open}

Another intriguing special case of this problem is determining whether there is a polynomial-time algorithm for complete binary trees. We know by the results in Section \ref{sec:completebintrees} that such an algorithm cannot be value-agnostic, but there seems to be no obvious way of leveraging the values, on even such a structured class of graphs.
\begin{conjecture}
    {\GHA} is polynomial-time solvable on complete binary trees.
\end{conjecture}