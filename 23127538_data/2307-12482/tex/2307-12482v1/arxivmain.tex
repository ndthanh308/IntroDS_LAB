\documentclass{article}
\usepackage{graphicx} % Required for inserting images

\usepackage{tikz,fullpage}
\usepackage{amsmath}
\usepackage{mathtools}
\usepackage{natbib}
\usepackage{caption}
\captionsetup[subfigure]{justification=centering}
% \usepackage{amsmath, amssymb, amsthm, mathrsfs, tikz, footmisc}
% \usepackage{multirow}
% \usepackage{algorithm2e}
% \usepackage{hyperref}
% \usetikzlibrary{shapes.geometric, arrows}
% \usepackage[sort]{cite}
\usetikzlibrary{arrows.meta,
                chains,
                decorations.pathreplacing,calligraphy,
                positioning,
                quotes,
                shapes.geometric
                }
\DeclareMathOperator*{\argmax}{arg\,max}
\DeclareMathOperator*{\argmin}{arg\,min}

\input{arxiv00preamble}

\newcommand{\HH}[1]{{\color{magenta}{HH: }{#1} \color{red}}}

\title{Tight Approximations for  Graphical House Allocation}
\author{%
  % Author\thanks{Use footnote for providing further information
  %   about author (webpage, alternative address)---\emph{not} for acknowledging
  %   funding agencies.} \\
  Hadi Hosseini\\
  Penn State University\\
  \texttt{hadi@psu.edu} \\
  \and
  Andrew McGregor \\
  UMass Amherst\\
  \texttt{mcgregor@cs.umass.edu} \\
  \and
  Rik Sengupta \\
  UMass Amherst\\
  \texttt{rsengupta@cs.umass.edu} \\
  \and
  Rohit Vaish \\
  IIT Delhi\\
  \texttt{rvaish@iitd.ac.in} \\
  \and
  Vignesh Viswanathan \\
  UMass Amherst\\
  \texttt{vviswanathan@umass.edu} \\
}
\date{}



\begin{document}

\maketitle

\begin{abstract}
%The \GHA{} problem is concerned with assigning $n$ houses to $n$ agents that are placed on the vertices of an undirected graph. The goal is to minimize the total amount of envy---measured as the absolute difference between an agent's value for its own house and the value it attributes to the neighboring agents' houses.
%
%This problem, alternatively, can be viewed as the generalization of the classical {\MLA} problem that aims at assigning $n$ nonnegative real numbers to the vertices of a graph so as to minimize the edgewise sum of absolute differences of the endpoints. \HH{The last sentence may need some paraphrasing as "endpoints" are not quite clear.} Nonetheless, \cite{canon} recently showed that the \GHA{} problem is significantly more challenging than the \MLA{} problem as the latter admits fast algorithms while the former is inapproximable even on simple classes of graphs (e.g. \HH{perhaps we give an example or two}).

The {\GHA} problem asks: how can $n$ houses (each with a fixed non-negative value) be assigned to the vertices of an undirected graph $G$, so as to minimize the ``aggregate local envy'', i.e., the sum of absolute differences along the edges of $G$? This problem generalizes the classical {\MLA} problem, as well as the well-known \emph{House Allocation Problem} from Economics, the latter of which has notable practical applications in organ exchanges. Recent work has studied the computational aspects of {\GHA} and observed that the problem is NP-hard and inapproximable even on particularly simple classes of graphs, such as vertex disjoint unions of paths. However, the dependence of any approximations on the structural properties of the underlying graph had not been studied.

%In this work, we focus on identifying a variety of graph structures and parameters that admit approximate algorithms. In particular, we present algorithms to approximate the optimal envy and analyze their guarantees on
%\begin{inparaenum}[(a)]
%    \item general graphs,
%    \item general trees, 
%    \item planar graphs,
%    \item bounded-degree graphs, and 
%    \item bounded-degree planar graphs.
%\end{inparaenum}
%In each case, we prove \emph{matching} lower bounds, showing that no significant improvements can be attained unless P = NP. Moreover, we present general approximation ratios as a function of structural parameters of the underlying graphs---e.g. cutwidth and treewidth---which subsequently enables us to achieve better approximations for several natural subclasses of graphs while matching the previous tight upper bounds.
%
%Finally, we investigate the special case of bounded-degree trees and refute a conjecture posed by \cite{canon} about the structural properties of exact optimal allocations on binary trees, through a counterexample on a depth-$3$ complete binary tree.
%Nevertheless, we present a linear-time algorithm that attains a 2.7-approximation on complete binary trees. 

% Our results primarily rely on exploiting graph structures to reason about optimal arrangements. In particular, several of our hardness reductions are quite technical; in one case, we make use of the expansion properties of Ramanujan graphs. 
%To our knowledge, the use of these graphs and these techniques in the context of resource allocation problems is novel.



%%%%%% OLD Abstract %%%%%%%

% The {\GHA} problem was introduced by \cite{canon} as a generalization of the \emph{House Allocation Problem}, and a cardinal version of the notion of local envy-freeness defined by \cite{beynier2018localenvy}. In its canonical form, {\GHA} asks: given $n$ agents placed on the vertices of an undirected graph $G$, and $n$ houses with arbitrary nonnegative values, how do we allocate each house to an agent so as to minimize the total envy on the graph $G$? Here, an agent envies a neighboring agent in $G$ if the latter has a higher-valued house, and the amount of envy is the absolute difference in those two values.

% Note that the {\GHA} problem can be viewed as a purely combinatorial problem: given an $n$-vertex graph $G$ and $n$ nonnegative real numbers, how do we assign the numbers to the vertices of $G$ so as to minimize the edgewise sum of absolute differences of the endpoints? While this is a natural generalization of the classical {\MLA} problem, it was shown by \cite{canon} that {\GHA} is significantly harder, as it is NP-hard and inapproximable even on particularly simple classes of graphs, which have fast algorithms and nice structural guarantees for {\MLA}.

In this work, we give a nearly complete characterization of the approximability of {\GHA}. We present algorithms to approximate the optimal envy on
%:
%\begin{inparaenum}[(a)]
%     \item general graphs,
%     \item trees, 
%     \item planar graphs,
%     \item bounded-degree graphs, and 
%     \item bounded-degree planar graphs.
%\end{inparaenum}
general graphs, trees, planar graphs, bounded-degree graphs, and bounded-degree planar graphs.
For each of these graph classes, we then prove \emph{matching} lower bounds, showing that in each case, no significant improvement can be attained unless P = NP. We also present general approximation ratios as a function of structural parameters of the underlying graph, such as treewidth; these match the aforementioned tight upper bounds in general, and are significantly better approximations for many natural subclasses of graphs. Finally, we investigate the special case of bounded-degree trees in some detail. We first refute a conjecture by \cite{canon} about the structural properties of exact optimal allocations on binary trees by means of a counterexample on a depth-$3$ complete binary tree. This refutation, together with our hardness results on trees, might suggest that approximating the optimal envy even on complete binary trees is infeasible. Nevertheless, we present a linear-time algorithm that attains a 3-approximation on complete binary trees. 
% %We show that a natural class of algorithms (``value-agnostic'' procedures) cannot attain an approximation ratio better than 1.66 on complete binary trees.
We leave open the intriguing question of whether {\GHA} is NP-hard on binary trees.

Some of the technical highlights of our work are the use of expansion properties of Ramanujan graphs in the context of a classical resource allocation problem, and approximating optimal cuts in binary trees by analyzing the behavior of consecutive runs in bitstrings.
% Most of our results rely on exploiting graph structures to reason about optimal arrangements. In particular, several of our hardness reductions are quite technical; in one case, we make use of the expansion properties of Ramanujan graphs. To our knowledge, the use of these graphs and these techniques in the context of resource allocation problems is novel.
\end{abstract}

\section{Introduction}

%\hadi{1. make WLOG and wlog consistent throughout. \\
%2. In some places (e.g. Section 5) we use cuts on valuation intervals. But I can't see it explicitly defined. It'd be nice to include it in preliminaries. \\
%3. we use `good', `houses', and `items' even in the intro. I suggest using \textit{items} and \textit{houses} only throughout.}

In the EconCS community, the \emph{House Allocation Problem} has been a topic of significant interest for some time \citep{shapley1974cores, svensson1999strategy, beynier2018localenvy, gsvfairhouse, kmsfairhouse}. In its canonical form, the problem involves a set of $n$ agents, a set of $n$ items (or ``houses''), and possibly different valuation functions from the agents to the items. In general, given this framework, the problem asks for an ``optimally fair'' allocation of the houses to the agents. For instance, we might wish to minimize the total envy, or maximize the number of envy-free pairs of agents. In this context, as it is common in the literature of fairness, an agent $i$ envies an agent $j$ in a particular allocation if according to agent $i$'s valuation function, the item received by agent $j$ is worth more than the item received by agent $i$; the amount of envy is the difference in these two values. The canonical problem has been studied in a variety of contexts, and is well-known as an algorithmically difficult problem to solve, for most reasonable fairness objectives.

\cite{canon} introduced a variant of the house allocation problem called {\GHA}. In this setting, there are $n$ agents, but now they are placed on the vertices of an undirected $n$-vertex graph $G = (V, E)$. There are still $n$ items with arbitrary values, but the agents are \emph{identical} in how they value these $n$ items (i.e., they all agree on the value of each house). {\GHA} now asks: how do we allocate each house to an agent so as to minimize the total envy along the edges of $G$?

We remark here that the setting where the agents are on the vertices of a graph and only the envy along the graph edges is considered was studied before as well by \cite{beynier2018localenvy}, who considered ordinal preferences in such a setting, and were interested in maximizing the number of envy-free edges in the underlying graph. Note that {\GHA} is the variant where we have a utilitarian objective, of minimizing the total envy.

Observe that {\GHA} can be thought of as a purely combinatorial problem: we are given an $n$-vertex graph $G = (V, E)$ and a multiset $H = \{h_1, \ldots, h_n\} \subseteq \R_{\geq 0}$. We wish to find the bijective function $\pi : V \to H$ that minimizes
$\sum_{(x, y) \in E}|\pi(x) - \pi(y)|$.

If the set of values were $H = \{1, \ldots, n\}$, then {\GHA} would be identical to the well-known {\MLA} problem. This was observed by \cite{canon}, who went on to show some remarkable differences between the two problems. For instance, while all hardness results carry over from {\MLA} to {\GHA}, the latter is actually a significantly harder problem even on very simple graphs. In particular, in {\MLA}, we can assume without loss of generality that the underlying graph is connected; this is because an optimal solution is given by taking each connected component separately, and optimally assigning a contiguous subset of values to it. We lose this guarantee in {\GHA}, even for  small disconnected graphs with just two connected components. As a typical example of the differences between the two problems, observe that if the underlying graph is a disjoint union of paths, then solving {\MLA} optimally takes linear time, but even on these simple instances, {\GHA} is NP-complete \citep{canon}.

We do note, however, that all the hardness constructions by \cite{canon} used the disconnectedness of the underlying graphs crucially, in finding reductions from bin packing instances. Their results also show that for very simple classes of disconnected graphs, {\GHA} is inapproximable to any finite factor. However, these proof techniques do not carry over to \emph{connected} graphs, and so it was not known whether any of these reductions would go through for connected graphs. For instance, a well known result by \cite{mlatrees} states that {\MLA} is solvable in polynomial time on trees, but the complexity of this problem for {\GHA} was open.

%Note that it is relatively straightforward to show that an $\alpha$ approximation algorithm for the \MLA~problem yields an $\alpha \cdot \phi$ approximation for the $\GHA$ problem where $\phi=\max_{1\leq i\leq n-1} (h_{i+1}-h_i)/\min_{1\leq i\leq n-1} (h_{i+1}-h_i)$.


%\andrew{Do we also want to mention that min-bisection is a special case of graphical house allocation but that general graphical house allocation is strictly harder, e.g., min bisection is poly-time on trees but we show graphical house allocatiton is NP hard on trees.}

\subsection{Our Contributions}

We present a nearly complete characterization of the approximability of {\GHA} on various large and natural classes of connected graphs, summarized in Table \ref{table:summary1}. In particular, for any {\GHA} instance on the following graph classes, we first show a polynomial-time algorithm on an $n$-vertex graph $G$ (with maximum degree $\Delta$) in that class for obtaining the stated multiplicative approximation to the optimal envy, and then go on to demonstrate a matching lower bound that shows that any polynomial improvement on the approximation ratio is impossible on that graph class unless P = NP (In what follows, $\tilde{O}$ hides $\polylog(n)$ factors):
%Specifically, we show the following results for the multiplicative approximation to the optimal envy for an instance of {\GHA} defined on a graph $G$ with $n$ nodes and maximum degree $\Delta$:
\begin{itemize}
    \item If $G$ is any connected graph, we have a trivial upper bound of $O(n^2)$, attained by \emph{any} allocation whatsoever (Proposition \ref{prop:trivialgeneral}). In Theorem \ref{thm:general-approx-lower-bound}, we show that we cannot have an $O(n^{2 - \epsilon})$-approximation for any $\epsilon > 0$. Using known approximations to the treewidth of $G$, in Corollary \ref{cor:cutwidth-upperbounds} we show a polynomial-time algorithm that achieves an $\tilde{O}(\tw(G)\cdot\Delta)$-approximation. %Note that assuming connectivity is essential for this upper bound since, as noted above, previous work establishes that even determining whether the optimum envy is non-zero is NP hard on graphs node-disjoint paths.
    
    \item If $G$ is a tree, Proposition \ref{prop:trivialgeneral} again gives us a trivial upper bound of $O(n)$, attained by \emph{any} allocation. In Theorem \ref{thm:trees-approx-lower-bound}, we show that we cannot have an $O(n^{1 - \epsilon})$-approximation for any $\epsilon > 0$. This is in stark contrast to {\MLA}, where there are sub-quadratic algorithms for \emph{exact} solutions on trees \citep{mlatrees}. Again, Corollary \ref{cor:cutwidth-upperbounds} gives us a polynomial-time algorithm that achieves an $O(\Delta
    \log n)$-approximation. We explicitly show a simple divide-and-conquer procedure (Algorithm \ref{alg:treelogn}) that gives the same $O(\Delta\log n)$-approximation in $O(n\log n)$ time.

    \item If $G$ is a planar graph, Corollary \ref{cor:cutwidth-upperbounds} gives us a polynomial-time algorithm to achieve an $\tilde{O}(\sqrt{n\Delta})$-approximation to the optimal envy. In the worst case, $\Delta = \Theta(n)$, so this is a worst-case approximation of $\tilde{O}(n)$. Once again, Theorem \ref{thm:trees-approx-lower-bound} shows that we cannot have an $O(n^{1 - \epsilon})$-approximation for any $\epsilon > 0$.

    \item If $G$ is a bounded-degree graph, Corollary \ref{cor:cutwidth-upperbounds} gives us a polynomial-time algorithm to achieve an $\tilde{O}(\tw(G))$-approximation to the optimal envy. Again, this is a worst-case approximation of $\tilde{O}(n)$. Using Theorem \ref{thm:bounded-degree-approx-lower-bound}, we show that we cannot have an $O(n^{1 - \epsilon})$-approximation for any $\epsilon > 0$. While all our lower bounds rely on 3-partition constructions, this is our %hardest
    most involved technical result, and it uses expansion properties of Ramanujan graphs to construct our gadgets.

    \item If $G$ is a bounded-degree planar graph, Corollary \ref{cor:cutwidth-upperbounds} gives us a polynomial-time algorithm to achieve an $\tilde{O}(\sqrt{n})$-approximation to the optimal envy. We match this upper bound using Theorem \ref{thm:bounded-degree-planar-approx-lower-bound}, which shows that we cannot have an $O(n^{0.5 - \epsilon})$-approximation for any $\epsilon > 0$.
\end{itemize}

\begin{table}
    \centering
    \def\arraystretch{1.3}
    \begin{tabular}{ |l|l|l|  }
 \hline
 \multicolumn{3}{|c|}{Approximations for {\GHA}} \\
 \hline
 \bf Graph Class & \bf Upper Bound & \bf Lower Bound \\
 \hline
\multirow{2}{15em}{Connected graphs}  & $O(n^2)$ (Prop.~\ref{prop:trivialgeneral}) & \multirow{2}{10em}{\centering $\omega(n^{2 - \varepsilon})$ (Thm.~\ref{thm:general-approx-lower-bound})} \\
& $O(\tw(G)\cdot\Delta\log^{2.5} n)$ (Cor.~\ref{cor:cutwidth-upperbounds}(\ref{cor:cwgeneral})) & \\
\hline 
\multirow{2}{15em}{Trees}  & $O(n)$ (Prop.~\ref{prop:trivialgeneral}) & \multirow{2}{10em}{\centering $\omega(n^{1 - \varepsilon})$ (Thm.~\ref{thm:trees-approx-lower-bound})} \\
& $O(\Delta\log n)$ (Alg.~\ref{alg:treelogn}, Cor.~\ref{cor:cutwidth-upperbounds}(\ref{cor:cwtrees})) & \\
\hline
Planar graphs  & $O(\sqrt{n\Delta}\log^{1.5}n)$ (Cor.~\ref{cor:cutwidth-upperbounds}(\ref{cor:cwplanar})) & $\omega(n^{1 - \varepsilon})$ (Thm.~\ref{thm:trees-approx-lower-bound}) \\
\hline
Bounded-degree graphs  & $O(\tw(G)\cdot\log^{2.5}n)$ (Cor.~\ref{cor:cutwidth-upperbounds}(\ref{cor:cwgeneral})) & $\omega(n^{1 - \varepsilon})$ (Thm.~\ref{thm:bounded-degree-approx-lower-bound}) \\
\hline
Bounded-degree planar graphs &  $O(\sqrt{n}\log^{1.5}n)$ (Cor.~\ref{cor:cutwidth-upperbounds}(\ref{cor:cwplanar})) & $\omega(n^{0.5 - \varepsilon})$ (Thm.~\ref{thm:bounded-degree-planar-approx-lower-bound}) \\
\hline
Bounded-degree trees & $O(\log n)$ (Thm.~\ref{thm:treelogn}, Cor.~\ref{cor:cutwidth-upperbounds}(\ref{cor:cwtrees})) & \textcolor{red}{open} \\
 \hline
\end{tabular}
\caption{Summary of our results. Here, $\Delta$ is the maximum degree of the graph in question, and the lower bounds assume P $\neq$ NP. Note that in all cases, the lower bound corresponds to the worst-case upper bound up to polylogarithmic factors, showing that nontrivial improvements to these upper bounds are impossible unless P = NP. All our upper bound algorithms run in polynomial time.%\rohit{Is it possible to add the results of \cite{canon} in this table (for ease of comparison and quick reference)?}
}
\label{table:summary1}
\end{table}

Note that assuming connectivity in the results above is necessary. As mentioned earlier, \cite{canon} showed that disconnected graphs cannot have the optimal envy approximated to any finite factor. To the best of our knowledge, we give the first known results for connected graphs.


The only remaining large class of graphs in Table \ref{table:summary1} are bounded-degree trees, exemplified by binary trees. We investigate these trees in some detail. Of course, Corollary \ref{cor:cutwidth-upperbounds} and Algorithm \ref{alg:treelogn} give us an $O(\log n)$-approximation to the optimal envy on bounded-degree trees, but we do not know whether we can get rid of the logarithmic factor at all. In fact, we show that the special class of binary trees trees is not ``well-behaved'', by refuting a conjecture by \cite{canon} about the structural properties of exact optimal allocations on binary trees by means of a counterexample (Section \ref{sec:boundeddegreetrees}) on a depth-$3$ complete binary tree. The hardness results in Theorems \ref{thm:trees-approx-lower-bound}, \ref{thm:bounded-degree-approx-lower-bound}, and \ref{thm:bounded-degree-planar-approx-lower-bound} might have suggested that even complete binary trees cannot have $o(\log n)$-approximations in general. We refute this, however, by showing that just the in-order traversal on a complete binary tree achieves a 3-approximation to the optimal envy (Theorem \ref{thm:inorder}). We further show that a natural class of algorithms (``value-agnostic'' procedures) cannot attain an approximation ratio better than 1.67 on complete binary trees. However, we do not resolve the question of whether attaining an exact solution on complete binary trees is NP-hard or not, and leave this for future research.

We also consider a generalized version of our problem on trees wherein the tree is \emph{edge-weighted} (and the envy along each edge has to be multiplied by the weight of that edge). For these graphs, we can adapt Algorithm \ref{alg:treelogn} easily, giving us an $O(\Delta\log n)$-approximation to the optimal envy on such trees in nearly linear time. Here $\Delta$ is the maximum \emph{weighted} degree, the sum of weights of all the edges incident on a vertex.

These results on special cases are summarized in Table \ref{table:summary2}.

Our paper is organized as follows. In Section \ref{sec:prelims}, we set up preliminaries and notation. In Sections \ref{sec:upper} and \ref{sec:lower}, we present our upper and lower bounds respectively from Table \ref{table:summary1}. In Section \ref{sec:completebintrees}, we discuss binary trees. We finish with concluding remarks and open directions in Section \ref{sec:conclusions}.

%\hadi{this paragraph can be removed to save space since most of it already exists under contributions.} Our paper is organized as follows. In Section \ref{sec:prelims}, we set up the preliminaries and notation for {\GHA}, including defining the classes of algorithms we will be studying. In Section \ref{sec:upper}, we start with a divide-and-conquer algorithm for trees, and later go on to use the parameter cutwidth to generalize this result for other classes of graphs, giving us all our upper bounds from Table \ref{table:summary1}. In Section \ref{sec:lower}, we give our lower bounds, showing reductions from bin-packing for several classes of graphs to give us all our lower bounds from Table \ref{table:summary1}. In Section \ref{sec:completebintrees}, we delve into complete binary trees in some detail, refuting the conjecture by \cite{canon} and then going on to show a constant approximation on complete binary trees. We finish with some concluding remarks and open directions in Section \ref{sec:conclusions}.



\begin{table}
    \centering
    \def\arraystretch{1.3}
    \begin{tabular}{ |l||l|l|l|  }
 \hline
  \multicolumn{4}{|c|}{Approximations for Some Special Classes} \\
 \hline
\bf  Graph Class & \bf Upper Bound & \bf Lower Bound & \bf 
Runtime \\
 \hline
Edge-weighted trees & $O(\Delta_w\log n)$  (Cor.~\ref{cor:weightedtrees}) & $\omega(n^{1 - \varepsilon})$ (Thm.~\ref{thm:trees-approx-lower-bound})
& $O(n\log n)$ \\
\hline
Complete binary trees & $3$ (Thm.~\ref{thm:inorder}) & $\geq 1$ (exact solution \textcolor{red}{open}) & $O(n)$ \\
\hline
\end{tabular}
\caption{Summary of our results for two specific graph classes. Here, $\Delta_w$ is the \emph{weighted} maximum degree, defined in Section \ref{sec:boundeddegreetrees}. Moreover, the upper and lower bounds match in the worst case. Note that we know the lower bound for complete binary trees has to be at least $1.67$ for value-agnostic algorithms.}
\label{table:summary2}
\end{table}



\subsection{Other Related Work} 
Our work is very close to the large body of results on the computability of {\MLA}. While finding optimal linear arrangements is intractable in general \citep{mlabinaryhard}, there have been several papers presenting approximation algorithms for the problem \citep{richarao2005mla,feige2007mla, even200mla}, with the best known approximation ratio being $O(\sqrt{\log n}\log{\log n})$ \citep{feige2007mla}. Note that it is relatively straightforward to show that an $\alpha$-approximation algorithm for the \MLA~problem yields an $\alpha\phi$ approximation for the $\GHA$ problem where $\phi=\max_{1\leq i\leq n-1} (h_{i+1}-h_i)/\min_{1\leq i\leq n-1} (h_{i+1}-h_i)$.

Our problem also generalizes the classical problem of \textsc{Minimum Bisection}, which asks how to partition a graph $G$ into two almost equally-sized components with the smallest number of edges going across the cut. This problem is NP-complete \citep{mlahard} and it is also known to be inapproximable by an additive factor of $n^{2-\epsilon}$ \citep{bj92}. These lower bounds carry over to the \GHA{} problem as well, although the latter is strictly harder. For instance, \textsc{Minimum Bisection} is known to be solvable exactly in polynomial time for forests, but {\GHA} is NP-hard on forests~\citep{canon}.

The canonical house allocation problem has also been well-studied in the literature. Recall that, in the canonical house allocation problem, agents are allowed to disagree on the values of the houses. In this setting, the existence and computational complexity of envy-free allocations on graphs have been reasonably well-studied \citep{beynier2018localenvy,eiben2020parameterized,bredereck2022envy}, with the problem, unsurprisingly, being computationally intractable in most settings. %Indeed, this is unsurprising given that the canonical house allocation problem is a generalization of the \GHA{} problem.
There have also been a few papers studying the complexity of minimizing various notions of envy when the underlying graph is {\em complete} \citep{gsvfairhouse, kamiyama2021envy, aigner2022envy,MMS23complexity}. 




\section{Model and Preliminaries}\label{sec:prelims}

\subsection{Preliminaries from House Allocation}

We have a set of $n$ {\em agents} $V = [n]$ placed on the vertices of an undirected graph $G = (V, E)$. 
There are $n$ {\em houses}, each with a nonnegative \emph{value}, that need to be allocated to the agents. We represent the houses simply by the multiset of values $H = \{h_1, \ldots, h_n\}$, and assume without loss of generality that $h_1 \leq \ldots \leq h_n$. We will interchangeably talk about the house with value $h_i$ and the real number $h_i$. The pair $(G, H)$ defines an \emph{instance} of {\GHA}. %\hadi{we are sometimes referring to \textit{the} \GHA{} \textit{problem} and sometimes \GHA{} alone. we should be  consistent throughout the paper.}

An {\em allocation} $\pi: V \rightarrow H$ is a bijective mapping from agents (or nodes) to house values. Given an allocation $\pi$ and an edge $(i, j) \in E$, we define the {\em envy} along the edge $(i, j)$ as $|\pi(i) - \pi(j)|$. Our goal in {\GHA} is to compute an allocation $\pi^\ast$ that {\em minimizes} the total envy %\hadi{sometimes we say aggregate, let's be consistent} 
along all the edges of $G$:
\begin{align*}
    \Envy(\pi, G) := \sum_{(i, j) \in E} |\pi(i) - \pi(j)|.
\end{align*}

% The following definition, given by \cite{canon}, gives us a geometric way to visualize any allocation on an instance $(G, H)$ of {\GHA}.
We adopt the following definition from  \cite{canon} that provides a geometric representation to visualize any allocation on an instance $(G, H)$ of {\GHA}.

\begin{definition}[Valuation Interval]\label{def:valn_interval}
For an instance $(G, H)$ of {\GHA}, define the \emph{valuation interval} as the closed interval $\left[h_1, h_n\right] \subset \R_{\geq 0}$. For any allocation $\pi$, the envy along the edge $(i, j) \in E$ is exactly the length of the interval $[\pi(i), \pi(j)]$ (assuming $\pi(i) \leq \pi(j)$). We sometimes call the intervals $[h_i, h_{i+1}]$ for $1 \leq i \leq n - 1$ the \emph{smallest subintervals} of the valuation interval.
\end{definition}


 An optimal allocation $\pi^\ast$ would minimize the sum of the lengths of the intervals corresponding to each of its edges.
%
An allocation $\pi$ is \emph{$\alpha$-approximate} if $\Envy(\pi, G) \le \alpha\cdot\Envy(\pi^\ast, G)$. 




%In most cases of the {\GHA} problem, computing an optimal $\pi^\ast$ is intractable. Therefore, we focus on approximations. 
% An allocation $\pi$ is \emph{$\alpha$-approximate} if $\Envy(\pi, G) \le \alpha\cdot\Envy(\pi^\ast, G)$. 

Fix any arbitrary class $\cal G$ of graphs (we allow $\cal G$ to be a singleton class as well). We say an algorithm $\mathsf{ALG}_\mathcal{G}$ is \emph{defined} on $\cal G$ if $\mathsf{ALG}_\mathcal{G}$ is well-specified and outputs a valid allocation on every instance $(G, H)$ of {\GHA} with $G \in \mathcal{G}$. Such an algorithm $\mathsf{ALG}_\mathcal{G}$ is an \emph{$\alpha$-approximation} if for all instances $(G, H)$ of {\GHA} with $G \in \mathcal{G}$, $\mathsf{ALG}_{\mathcal{G}}$ always outputs an allocation that is $\alpha$-approximate. A $1$-approximation is an exact algorithm.

We are now ready to formulate the following definition.

\begin{definition}[Value-Agnostic Algorithms]
\label{defn:valueagnostic}
An algorithm $\mathsf{ALG}_\mathcal{G}$ defined on a graph class $\mathcal{G}$ is \emph{value-agnostic} if on every input $(G, H)$ with $G \in \mathcal{G}$, $\mathsf{ALG}_\mathcal{G}$ returns the same allocation on all instances where the \emph{ordering} of house values is the same (in other words, the algorithm only requires the ordinal ranking without requiring the numerical values).
If the graph class $\cal G$ admits a value-agnostic $\alpha$-approximation algorithm, we say $\cal G$ is \emph{$\alpha$-value-agnostic}. Otherwise, it is \emph{$\alpha$-value-sensitive}.
\end{definition}

How can we re-frame existing results on {\GHA} in the light of Definition \ref{defn:valueagnostic}? \cite{canon} show that, unless P = NP, there is no $1$-approximation algorithm $\mathsf{ALG}_\mathcal{G}$ when $\cal G$ is the set of vertex-disjoint unions of paths, cycles, or stars. In contrast, they show that value-agnostic \emph{exact} algorithms exist when $\cal G$ is the set of paths, cycles, or stars, and therefore these classes are all $1$-value-agnostic.
% ; see Figure \ref{fig:value_agnostic_ex}~(a).

Of course, value-agnostic $\alpha$-approximations are extremely powerful algorithms, as they can exploit the graph structure \emph{independent} of the values in the {\GHA} instance. As we would expect, value-agnostic $1$-approximations do not always exist, even on very simple singleton graph classes and even if we allow for exponential amount of time. For instance, consider the graph consisting of the disjoint union of $K_2$ and $K_3$. Figure \ref{fig:value_agnostic_ex} shows that this graph does not admit an $\alpha$-value-agnostic algorithm for any finite $\alpha$.
%\hadi{Conisder moving this example to right after Definition 2.1 when we introduce "valuation intervals". Then we can refer again to the same example when discussing value-agnostic approximations.}
%\rik{Will fix once figures are split and corrected.}

% Figure environment removed

Although all our examples so far use the disconnectedness of the graphs to illustrate value-sensitivity, we will see in Section \ref{sec:completebintrees} that there are value-sensitive connected graphs as well.




\subsection{Preliminaries from Structural Graph Theory}

We will use a few standard concepts from structural graph theory, most notably that of \emph{treewidth}. %Let us formally state the definition below.

\begin{definition}[Treewidth]\label{def:treewidth}
    For any graph $G = (V, E)$, a \emph{tree decomposition} $\mathfrak{T} = (T; \{X_i\}_{i = 1}^t)$ of $G$ is a tree $T$ whose nodes are subsets $X_1, \ldots, X_t \subseteq V$, satisfying the following three properties:
    \begin{enumerate}
        \item $\bigcup_{i = 1}^t X_i = V$;
        \item If $v \in V$ is in $X_i$ and $X_j$, then $v$ is in every $X_k$ in the unique path in $T$ between $X_i$ and $X_j$;
        \item For every edge $(u, v) \in E$, there is some $X_i$ containing both $u$ and $v$.
    \end{enumerate}
    The \emph{width} of the tree decomposition $\mathfrak{T}$ is
    % \begin{equation*}
    $
        \mathsf{width}(\mathfrak{T}) := \max_{i}\left ( |X_i| - 1\right ).
    $
    % \end{equation*}
    The \emph{treewidth} of a graph $G$ is defined as $\tw(G) := \min_{\mathfrak{T}}\mathsf{width}(\mathfrak{T})$.
    % \begin{equation*}
    %     \tw(G) := \min_{\mathfrak{T}}\mathsf{width}(\mathfrak{T}).
    % \end{equation*}
\end{definition}

Note that a connected graph $G$ with two or more vertices satisfies $\tw(G) = 1$ if and only if $G$ is a tree. Graphs with treewidth at most $2$ are exactly the series-parallel graphs, and graphs with treewidth $n - 1$ are exactly the complete graphs.

In Definition \ref{def:treewidth}, if we constrain the tree $T$ to be a path, then the resulting notion of width is called the \emph{pathwidth} of $G$, denoted $\pw(G)$. A tree with two or more vertices satisfies $\pw(G) = 1$ if and only if $G$ is a caterpillar graph.
Treewidth and pathwidth are deep and fundamental parameters for graphs, used widely in structural graph theory. Many problems on graphs become provably easier when a (near-optimal) tree decomposition or path decomposition is part of the input. Note that the two parameters are related, as
\begin{equation*}
    \tw(G) \leq \pw(G) \leq O(\tw(G)\cdot\log n),
\end{equation*}
for any $n$-vertex graph $G$.

In Section \ref{sec:cutwidth}, we will visit another parameter, called the $\cw$, that is related to both of these parameters, and can be directly applied to {\GHA} for finding good approximations.

%\hadi{Should we move cutwidth also to this section? Especially because we mention it immediately in the next section?}

For any graph $G = (V, E)$, and any subset $S \subseteq V$, we denote by $\delta_G(S)$ the set of edges going across the cut $(S, V - S)$ in $G$. A number of our bounds will rely on estimating $|\delta_G(S)|$ for various subsets $S$. For $1 \leq k \leq n-1$, we also define $\delta_G(k) := \min_{|S| = n}|\delta_G(S)|$ as the size of the smallest cut in $G$ with $k$ vertices on one side. Of course, $\delta(k) = \delta(n - k)$ for all $k$.
\input{arxiv03upperbounds}
\section{Lower Bounds}\label{sec:lower}

%All the approximation guarantees presented in the previous section may appear weak at first.
Every algorithm presented in Section \ref{sec:upper} is value-agnostic. It might seem reasonable to assume, therefore, that there are more powerful approximation schemes that exploit the numerical values in $H$ in some way. Remarkably, we show in this section that this is \emph{not} the case, and value-agnostic algorithms are strong enough to give us nearly optimal approximation guarantees. Specifically, we show inapproximability results matching the upper bounds (up to $\polylog$ factors) in Section \ref{sec:upper} for almost every class of graphs considered. Our lower bounds will use reductions from the {\UTP} problem on carefully constructed gadgets.

\begin{definition}[\TP]
Given a multiset of $3m$ positive integers $A = \{a_1, \dots, a_{3m}\} \subseteq \mathbb{N}_{> 0}$ and a positive integer $T \in \mathbb{N}_{> 0}$ such that $\sum_{j \in [3m]} a_j = mT$, {\TP} asks whether the multiset $A$ can be partitioned into $m$ triplets $(S_1, S_2, \dots, S_m)$ such that the sum of each triplet is equal to $T$.
\end{definition}

The \TP problem is NP-complete even when all the inputs are given in unary and each of the items in $A$ are strictly between $\ffrac{T}{4}$ and $\ffrac{T}{2}$ \citep{garey1979computers}. We refer to this variant as {\UTP}. Note that but {\UTP} is just a reformulation of \textsc{Bin Packing}: there are $3m$ integers that sum to $mT$, and we wish to fit these integers into $m$ bins each of capacity $T$. The condition of three integers in each bin does not need to be stated separately, as it is implied by the constraint that each integer is strictly between $T/4$ and $T/2$.

We present our lower bounds in increasing order of proof complexity, starting with trees and moving to general graphs, planar graphs, and bounded-degree graphs. We give the complete proof only in the first case, relegating the remaining proofs to the appendix and providing proof sketches here instead.

\subsection{Trees and Planar Graphs}\label{sec:treelowerbound}
Recall that we presented two approximation guarantees for trees, $O(n)$ (Proposition \ref{prop:trivialgeneral}) and $O(\Delta \log n)$ (Corollary \ref{cor:cutwidth-upperbounds}). Both of these results are $\tilde{O}(n)$ in the worst case. 

\begin{theorem}\label{thm:trees-approx-lower-bound}
For any constant $\varepsilon > 0$, there is no efficient $O(n^{1-\varepsilon})$ approximation algorithm for {\GHA} on depth-$2$ trees unless P = NP.
\end{theorem}
\begin{proof}
% Figure environment removed

We give a reduction from \UTP. For some constant $\varepsilon > 0$, assume there is an efficient $O(n^{1- \varepsilon})$ approximation algorithm $\mathsf{ALG}_{\cal G}$ where $\cal G$ corresponds to the class of depth-$2$ trees. In other words, there is a constant $\gamma$ such that for all instances with $G \in \mathcal{G}$, $\mathsf{ALG}_{\cal G}$ outputs an allocation with total envy within a multiplicative factor of $\gamma n^{1- \varepsilon}$ to the optimal envy.

Given an instance of \UTP, we construct an instance $(G, H)$ of {\GHA} as follows: The graph $G$ is a rooted depth-$2$ tree where the root $r$ of the tree has $3m$ children $\{x_1, \dots, x_{3m}\}$. Each of these nodes $x_i$ has $C a_i - 1$ children (see Figure \ref{fig:trees-reduction-tree}). $C$ is a positive integer whose exact value we shall decide later.

The total number of nodes in $G$ is $1 + \sum_{i \in [3m]} Ca_i = 1 + CmT$, and so we must specify $1 + CmT$ house values in $H$. We define $H$ with $CT$ values of $i$ for $i \in [m]$, together with a single value of $0$. Note that this construction can be done in polynomial time as long as $C$ is polynomially large, since the input to the $3$-partition instance is given in unary. 

We show that, for an appropriate choice of $C$, the minimum total envy output by $\mathsf{ALG}_{\cal G}$ for the instance $(G, H)$ is at least $\left \lceil (12\gamma m^3 T + 1)^{\ffrac{1}{\varepsilon}} \right \rceil$ if and only if there exists {\em no} valid partition for the original \textsc{3-Partition} instance, i.e., the original instance was a NO instance. 


$(\Rightarrow)$ Assume there is a valid $3$-partition $(S_1, \ldots, S_m)$ for the original instance. We denote this $3$-partition using a mapping $\mu : A \to \{S_1, \ldots, S_m\}$ that maps each number in the multiset $A$ to one of the triplets $S_j$. We construct an allocation for the graph $G$ as follows: for any item $i$, if $\mu(i) = S_j$, we allocate houses with value $j$ to $x_i$ and all of its children. We finally allocate the house with value $0$ to the root. Note that this is a valid allocation: for any $j \in [m]$, we allocate exactly $\sum_{i \in [3m]: \mu(i) = S_j} Ca_i = CT$ houses of value $j$. 

We can easily upper bound the envy of this allocation; this upper bound also serves as an upper bound for the minimum total envy for the instance $(G, H)$. There is no envy between any $x_i$ and any of its children. So the only edges with potential envy are the ones incident on the root. There are $3m$ such edges, each incurring envy at most $m$. This gives us an upper bound of $3m^2$ on the total envy. This implies, when there is a valid $3$-partition, the approximation algorithm $\mathsf{ALG}_{\cal G}$ will output an allocation with envy at most $3m^2\gamma(1 + CmT)^{1-\varepsilon}$.

$(\Leftarrow)$ Assume there is no valid $3$-partition in the original instance. We will show that any allocation has a total envy of at least $C$. We do this by examining the houses allocated to the depth-1 nodes $\{x_1, \dots, x_{3m}\}$. If any $x_i$ is allocated a value of $0$ in some allocation $\pi$, then the allocation $\pi$ has a total envy of at least $C$ since any $x_i$ has at least $C -1$ children and $1$ parent receiving a value of at least $1$ each. 

So now assume no $x_i$ is allocated a value of $0$ in $\pi$. For notational convenience, assume without loss of generality that $x_1, \dots, x_{\ell_1}$ are allocated a value $1$, $x_{\ell_1 +1}, \dots, x_{\ell_1 + \ell_2}$ are allocated a value $2$ and so on. If for all $j \in [m]$, $\sum_{h \in [\ell_j]} a_{\ell_0 + \ell_1 + \dots + \ell_{j-1} + h} = T$ (with $\ell_0 = 0$), then we violate our assumption that there is no valid $3$-partition in the original instance. Therefore there exists one $j \in [m]$ such that $\sum_{h \in [\ell_j]}a_{\ell_0 + \ell_1 + \dots + \ell_{j-1} +h} > T$. Assume again for notational convenience that $j = 1$. Since all values are integers, we can restate the inequality above as $\sum_{h \in [\ell_1]}a_{h} \ge T + 1$. This implies $\sum_{h \in [\ell_1]} Ca_{h} \ge CT + C$. 

Coming back to the allocation $\pi$, we have that $\sum_{h \in [\ell_1]} Ca_{h} \ge CT + C$ implies that there are at least $C$ nodes out of all the children of $\{x_{1}, x_2, \dots, x_{\ell_1}\}$ which are not allocated a value of $1$. The envy that each of these $C$ nodes will have towards their parents is at least $1$. This implies that the total envy of allocation $\pi$ is at least $C$. 

We set $C = \left \lceil (12\gamma m^3 T + 1)^{\ffrac{1}{\varepsilon}} \right \rceil$ to complete the reduction. When there is no valid $3$-partition, the total minimum envy (and therefore, the envy output by $\mathsf{ALG}_{\cal G}$) is at least $C =\left \lceil (12\gamma m^3 T + 1)^{\ffrac{1}{\varepsilon}} \right \rceil$. However, when there is a valid $3$-partition, the envy output by $\mathsf{ALG}_{\cal G}$ is strictly upper bounded by:
\begin{align*}
    3m^2\gamma(CmT+1)^{1-\varepsilon} \le 6m^3 \gamma TC^{1-\varepsilon} \le 6m^3 \gamma T \left ( \left \lceil (12\gamma m^3 T + 1)^{\ffrac{1}{\varepsilon}} \right \rceil \right )^{1 - \varepsilon} < \left \lceil (12\gamma m^3 T + 1)^{\ffrac{1}{\varepsilon}} \right \rceil.
\end{align*}
This completes the proof.
\end{proof}


\subsection{General and Bounded-Degree Graphs}

In this section, we generalize the arguments from Section \ref{sec:treelowerbound} to other classes of graphs. We relegate all technical proofs from this section to the appendix, instead providing intuitive proof sketches.

Our first result matches the $O(n^2)$ worst case upper bound for general connected graphs (\Cref{prop:trivialgeneral} and \Cref{cor:cutwidth-upperbounds}). See Appendix \ref{apx:lower} for the full proof.

\begin{restatable}{theorem}{generalgraphs}\label{thm:general-approx-lower-bound}
For any constant $\varepsilon > 0$, there is no efficient $O(n^{2-\varepsilon})$ approximation algorithm for {\GHA} on connected graphs unless P = NP.
\end{restatable}

% Figure environment removed

\begin{proof}[Proof sketch]
    Figure \ref{fig:general-reduction-graph} shows the graph we use for this reduction. The construction is similar to that in Theorem \ref{thm:trees-approx-lower-bound}, except that we use cliques of size $Ca_i$ attached to a common root $r$, instead of a vertices $x_i$ with $Ca_i - 1$ dangling leaves. The idea is to use the high density of the clique to show that the envy is $\Omega(C^2)$ when there is no valid $3$-partition. The house values $H$ are defined as in Theorem \ref{thm:trees-approx-lower-bound}. A similar argument works here too. A valid $3$-partition can be ``packed'' into the clusters of values, which would attain only envy from the edges incident to the root $r$. Conversely, if there is no valid $3$-partition, again using a packing argument, it can be shown that there are several high envy edges, giving rise to $\Omega(C^2)$ envy.
\end{proof}





%\subsection{Bounded Degree Planar Graphs}

So far in our two lower bounds (Theorems \ref{thm:trees-approx-lower-bound} and \ref{thm:general-approx-lower-bound}), we were able to use simple counting techniques, because counting edges with non-zero envy in stars and cliques is straightforward. Our next results will require much more careful analysis.

We will start with bounded-degree planar graphs. Our reduction uses grid graphs instead of stars and cliques, and so we will need a technical lemma (whose proof is in the appendix) to help us with estimating the edges with nonzero envy.

\begin{restatable}{lemma}{gridlemma}\label{lem:grid-graph-property}
    Let $G = Grid(r, c)$ be a grid graph with $r$ rows and $c$ columns such that $r \le c$. Let $A \subseteq V$ be any set of nodes in this graph such that $|A| \le \ffrac{rc}{2}$. Then, $|\delta_G(A)| \geq \min\{\sqrt{|A|}, \ffrac{r}{2}\}$.
\end{restatable}

Armed with Lemma \ref{lem:grid-graph-property}, we can state our lower bound on bounded-degree planar graphs. See Appendix \ref{apx:lower} for the full proof.

\begin{restatable}{theorem}{bdp}\label{thm:bounded-degree-planar-approx-lower-bound}
For any constant $\varepsilon > 0$, there is no efficient $O(n^{0.5-\varepsilon})$ approximation algorithm for {\GHA} on bounded-degree planar graphs unless P = NP.
\end{restatable}

% Figure environment removed

\begin{proof}[Proof sketch]
    Figure \ref{fig:bdp-reduction-graph} shows the graph we use for this reduction. Again, the construction is similar to before. The graph $G$ has $3m$ grids, each with $C$ rows and $Ca_i$ columns, attached by a single edge to a leaf of a ``small'' binary tree $T_r$. Note that this is a bounded-degree planar graph. The values $H$ are defined similarly to before, with a cluster of $C^2T$ values at each positive integer in $[3m]$, and $|T_r|$ values at $0$.

    If there is a $3$-partition, the argument is exactly similar to those earlier in this section, with the only envy coming from the edges of the tree, incurring a total envy of $3m^2$.

    Conversely, suppose there is no $3$-partition. Now, using Lemma \ref{lem:grid-graph-property}, we can show that a grid which does not have a majority of its vertices from the same cluster must incur at least $C/4$ envy. Otherwise, assuming each grid has a majority of its vertices in the same cluster, a packing argument shows that some grid must have a cut going across two different values, and this incurs at least $C/2$ envy, once again by Lemma \ref{lem:grid-graph-property}.
\end{proof}

Note that Theorem \ref{thm:bounded-degree-planar-approx-lower-bound} matches the $O(\sqrt{n})$ upper bound from \Cref{cor:cutwidth-upperbounds}.

Our final lower bound applies to arbitrary bounded-degree graphs and matches the $O(n)$ upper bound from Proposition \ref{prop:trivialgeneral} and Corollary \ref{cor:cutwidth-upperbounds}. In this reduction, we use the recent polynomial-time algorithm to compute bipartite Ramanujan multi-graphs for any even number of vertices $m$ and any degree $d \ge 3$ \citep{cohen2016ramanujan}. At a high level, we replace the grid graphs from Theorem \ref{thm:bounded-degree-planar-approx-lower-bound} with these Ramanujan graphs and use the expansion properties of Ramanujan graphs to prove a lemma similar to (and stronger than) \Cref{lem:grid-graph-property}.

One key element of the construction we will use is that we will take a gadget consisting of a $3$-regular Ramanujan bipartite multi-graph of a given even size (at least $6$), and remove any of its repeated edges until it is a simple graph. The crucial observation is that even though these new gadget graph is neither regular nor Ramanujan, it still has ``enough'' expansion for our purposes. We will need the following well-known result for this, stated without proof.

\begin{lemma}[Cheeger's Inequality]\label{lem:cheegers-inequality}
Let $G'$ be a $d$-regular Ramanujan (multi-)graph defined on a set of $V$ nodes. The following holds:
\begin{align*}
    \min_{S \subseteq V: 0 < |S| \le |V|/2} \frac{\delta_{G'}(S)}{|S|} \ge \frac12(d - 2\sqrt{d-1}),
\end{align*}
where $\delta_{G'}(S)$ denotes the number of edges in the $(S, V \setminus S)$ cut in the graph $G'$.
\end{lemma}

The following theorem characterizes the inapproximability on bounded-degree graphs. See Appendix \ref{apx:lower} for the full proof.


\begin{restatable}{theorem}{boundeddeg}\label{thm:bounded-degree-approx-lower-bound}
For any constant $\varepsilon > 0$, there is no efficient $O(n^{1-\varepsilon})$ approximation algorithm for {\GHA} on bounded-degree graphs unless P = NP.
\end{restatable}

% Figure environment removed

\begin{proof}[Proof sketch]
    Figure \ref{fig:bounded-degree-reduction-graph} shows the graph we use for this reduction. The graph $G$ is constructed as follows. For each $a_i$ in the given {\UTP} instance, construct in polynomial time a $3$-regular Ramanujan bipartite multi-graph of size $Ca_i$ (using a construction by \cite{cohen2016ramanujan}). Remove any repeated edges to convert them into simple graphs. The resulting graphs can be shown to have sufficient amount of expansion properties, using Cheeger's inequality (Lemma \ref{lem:cheegers-inequality}). We can now attach these graphs to the leaves of a sufficiently small binary tree. This is a bounded-degree graph.

    If there is a $3$-partition, we can exhibit a small-envy allocation in the same way as in most of the other proofs in this section.

    % Conversely, if there is no valid $3$-partition, we can once again show that if some gadget has no majority value, then it spans many different values, and by its expansion properties, it has many edges in a cut across these different values, incurring $\Omega(C)$ envy. Otherwise, if each gadget has a majority value, we can use the packing argument once more to show that some gadget has to have a cut going across different values, incurring $\Omega(C)$ envy once again.
    Conversely, if there is no valid $3$-partition, we can show that some of these gadgets are going to be allocated multiple different values. We then use the expansion properties of these gadgets to show that the number of high envy edges within each of these gadgets is $\Omega(C)$.
\end{proof}





\input{arxiv05completebinarytrees}
\section{Conclusions}\label{sec:conclusions}

We explored the approximability of {\GHA}, presenting tight approximation algorithms for several classes of connected graphs, to our knowledge the first such results in the area. In particular, we gave polynomial-time algorithms exploiting graph structures to approximate the optimal envy on general graphs, trees, planar graphs, bounded-degree graphs, and bounded-degree planar graphs; for each of these classes, we also gave a matching lower bound. Our algorithms
% upper bounds 
% are straightforward algorithms that run in polynomial time, and are in fact
are value-agnostic, i.e., they take into account only the input graph and the ordering among the house values but not the values themselves. We also refuted a recent conjecture about the structural properties of optimal allocations on complete binary trees, but gave a value-agnostic algorithm to show a $3$-approximation on all such instances.

The following open question is an important one in further understanding the boundary between {\GHA} and {\MLA}: \cite{canon} showed that {\GHA} is NP-hard even for disjoint unions of paths, whereas \cite{mlatrees} showed {\MLA} is exactly solvable in polynomial time on forests. The following question asks us whether connectivity buys us anything in {\GHA}.
\begin{open}
    What is the complexity of {\GHA} on bounded-degree trees?
\end{open}

Another intriguing special case of this problem is determining whether there is a polynomial-time algorithm for complete binary trees. We know by the results in Section \ref{sec:completebintrees} that such an algorithm cannot be value-agnostic, but there seems to be no obvious way of leveraging the values, on even such a structured class of graphs.
\begin{conjecture}
    {\GHA} is polynomial-time solvable on complete binary trees.
\end{conjecture}

\section*{Acknowledgments}

We wish to thank Justin Payan for many helpful discussions during the early stages of this work. We also thank Paul Seymour for some discussions related to the material in Section \ref{sec:completebintrees}. Rohit Vaish acknowledges support from SERB grant no. CRG/2022/002621 and DST INSPIRE grant no. DST/INSPIRE/04/2020/000107. Andrew McGregor and Rik Sengupta acknowledge support from NSF grant CCF-1934846. 
Hadi Hosseini acknowledges support from NSF IIS grants \#2144413 and \#2107173.


\bibliographystyle{plainnat}
\bibliography{abb,arxivreferences}

\appendix

\section{Proofs from Section \ref{sec:lower}}\label{apx:lower}

\generalgraphs*
\begin{proof}
We present a similar reduction from the \UTP Problem. The main difference in construction from Theorem \ref{thm:trees-approx-lower-bound} is that $x_i$ and its children are replaced with a clique of size $Ca_i$. We use the high density of the clique to show that the envy is much higher $(\Omega(C^2))$ in the absence of a valid $3$-partition.  

For some constant $\varepsilon > 0$, assume there is an efficient $O(n^{2- \varepsilon})$ approximation algorithm $\mathsf{ALG}_{\cal G}$ where $\cal G$ is the class of all connected graphs. Assume there is some constant $\gamma$ such that for all instances on connected graphs, $\mathsf{ALG}_{\cal G}$ outputs an allocation with total envy within a multiplicative factor of $\gamma n^{2- \varepsilon}$ to the optimal envy.

Given an instance of \UTP, we construct an instance of {\GHA} as follows: for each value $a_i$ in the multiset $A$, we create a clique of size $Ca_i$. We then connect these $3m$ cliques to a node $r$ (as described in Figure \ref{fig:general-reduction-graph}). Once again, $C$ is a positive integer whose exact value we shall choose later.
The total number of nodes in this graph is $CmT+1$, same as in the other proofs before in this section. The house values are defined similarly as well: for each $j \in [m]$, there are $CT$ houses valued at $j$ and there is one house valued at $0$. 


We show that, with the appropriate $C$, the total envy output by $\mathsf{ALG}_{\cal G}$ for the constructed {\GHA} instance is greater than or equal to $\left \lceil (96\gamma m^4 T^2 + 1)^{\ffrac{1}{\varepsilon}} \right \rceil^2$ if and only if there exists {\em no} valid solution for the original \textsc{3-Partition} instance, i.e.,~the original instance was a NO instance. 


$(\Leftarrow)$ Assume there is a valid $3$-partition for the original instance. This case follows the exact same way as Theorem \ref{thm:trees-approx-lower-bound}. The minimum envy is upper bounded by $3m^2$ and the envy output by $\mathsf{ALG}_{\cal G}$ is upper bounded by $3m^2\gamma(CmT+1)^{2-\varepsilon}$.

$(\Rightarrow)$ Assume there is no valid $3$-partition in the original instance. We will show that any allocation must have an envy of at least $\ffrac{C^2}{4}$. 

If, for some allocation $\pi$, there exists a clique where no value $j \in [m] \cup \{0\}$ is allocated to more than half of its nodes, then this lower bound trivially holds. Since the clique has a size of at least $C$, each node in the clique envies at least $C/2$ neighbors by at least $1$. 

Assume that for all cliques, there is some value allocated to at least half the nodes in the clique; we refer to this value as a {\em majority value} of the clique. Let clique $i$ correspond to the clique $K_{Ca_i}$. For notational convenience, assume without loss of generality that cliques $\{1, \dots, \ell_1\}$ have majority value $1$, $\{{\ell_1 +1}, \dots, {\ell_1 + \ell_2}\}$ have majority value $2$ and so on. Using analysis similar to \Cref{thm:trees-approx-lower-bound}, there exists at least one $j \in [m]$ such that $\sum_{h \in [\ell_j]}a_{\ell_1 + \dots + \ell_{j-1} +h} > T$. Assume again for notational convenience that $j = 1$. Since all these values are integers, we can restate the equation above as $\sum_{h \in [\ell_1]}a_{h} \ge T+ 1$. This implies $\sum_{h \in [\ell_1]} Ca_{h} \ge CT + C$. 

Coming back to our allocation $\pi$, $\sum_{h \in [\ell_1]} Ca_{h} \ge CT + C$ implies that there are at least $C$ nodes in cliques $\{1,2, \dots, \ell_1\}$ which are not allocated a value of $1$. Since $1$ is a majority value in each of these cliques, the envy that each of the $C$ nodes in $S$ will have towards the nodes with value $1$ is at least $C/2$. This implies that the total envy of allocation $\pi$ is at least $C^2/2$. 

We set $C = 2\left \lceil (96\gamma m^4 T^2 + 1)^{\ffrac{1}{\varepsilon}} \right \rceil$ to complete the reduction. When there is no valid $3$-partition, the minimum total envy (and therefore, the envy output by $\mathsf{ALG}_{\cal G}$) is at least $\ffrac{C^2}{4} = \left \lceil (96\gamma m^4 T^2 + 1)^{\ffrac{1}{\varepsilon}} \right \rceil^2$. However, when there is a valid $3$-partition, the envy output by $\mathsf{ALG}_{\cal G}$ is strictly upper bounded by:
\begin{align*}
    3m^2\gamma(CmT+1)^{2-\varepsilon} \le 6m^4 \gamma T^2C^{2-\varepsilon} \le 24m^4 \gamma T^2 \left ( \left \lceil (96\gamma m^4 T^2 + 1)^{\ffrac{1}{\varepsilon}} \right \rceil \right )^{2 - \varepsilon} < \left \lceil (96\gamma m^4 T^2 + 1)^{\ffrac{1}{\varepsilon}} \right \rceil^2.
\end{align*}
This concludes the proof.
\end{proof}


\gridlemma*
\begin{proof}
If $A$ consists of at least one node from each row, then since $|A| \le \ffrac{rc}{2}$, there will be at least $r/2$ rows with a node in $V \setminus A$. Therefore, there will be at least $\ffrac{r}{2}$ edges in the cut. Similarly, if $A$ consists of at least one node from each column, there will be $\ffrac{c}{2} \ge \ffrac{r}{2}$ edges in the cut. 

Otherwise, there is some row and some column containing only nodes in $V \setminus A$.
Note that there must either be at least $\sqrt{|A|}$ rows with a node in $A$ or at least $\sqrt{|A|}$ columns with a node in $A$. Assume WLOG there are at least $\sqrt{|A|}$ rows with a node in $A$. Each of these rows intersects the column that only has nodes from $V\setminus A$, and so each of the $\sqrt{|A|}$ rows must contain an edge between $A$ and $V\setminus A$.
\end{proof}


\bdp*
\begin{proof}
We present a similar reduction from the \UTP Problem. 
For some constant $\varepsilon > 0$, assume there is an efficient $O(n^{0.5 - \varepsilon})$ approximation algorithm $\mathsf{ALG}_{\cal G}$ where $\cal G$ corresponds to the class of all planar graphs with max degree at most $5$. In other words, for all instances on these graphs, $\mathsf{ALG}_{\cal G}$ outputs an allocation with total envy within a multiplicative factor of $\gamma n^{0.5- \varepsilon}$ to the optimal envy, where $\gamma$ is some fixed constant.

Given an instance of \UTP, we construct an instance of {\GHA} as follows: the graph $G$ has $3m$ grids, each grid $i \in [3m]$ has $C$ rows and $C a_i$ columns for some $C$ we will define later. Each of the grids has a single edge to a unique leaf of a binary tree $T_r$ (see Figure \ref{fig:bdp-reduction-graph}). 

The binary tree $T_r$ is constructed such that it has at most $12m$ nodes. It is easy to see that the smallest balanced binary tree with at least $3m$ leaves satisfies this constraint. This construction is trivially a planar graph with degree at most $5$. There are $C^2mT + |T_r|$ nodes in this graph. The house values are defined roughly the same way as before: for each $j \in [m]$, there are $C^2T$ houses valued at $j$ and there are $|T_r|$ houses valued at $0$. As long as $C$ and $|T_r|$ are polynomial, this is a polynomial-time reduction. 


We show that, with an appropriate choice of $C$, the minimum total envy output by $\mathsf{ALG}_{\cal G}$ for the house allocation instance is greater than or equal to $\left \lceil (288\gamma m^3 T + 1)^{\ffrac{1}{(2\varepsilon)}} \right \rceil$ if and only if there exists {\em no} valid partition for the original \textsc{3-Partition} instance i.e. the original instance was a NO instance. 


$(\Leftarrow)$ Assume there is a valid $3$-partition for the original instance. This case follows the exact same way as Theorem \ref{thm:trees-approx-lower-bound}. The minimum envy is upper bounded by $3m^2$ and the envy output by $\mathsf{ALG}_{\cal G}$ is upper bounded by $3m^2\gamma(C^2mT+|T_r|)^{0.5-\varepsilon} \le 3m^2\gamma(C^2mT+12m)^{0.5-\varepsilon}$.

$(\Rightarrow)$ Assume there is no valid $3$-partition in the original instance. We will show that any allocation must have an envy of at least $\ffrac{C}{4}$. 

Similar to Theorem \ref{thm:general-approx-lower-bound}, we refer to a value $j \in [m] \cup \{0\}$ as a majority value of a grid $\Grid(C, Ca_i)$ if this value has been allocated to at least $\ffrac{C^2a_i}{2}$ nodes in the grid. 

For some allocation $\pi$, assume there exists a grid, $\Grid(C, Ca_1)$ with no majority value. Then this grid has nodes with at least $3$ different values. Let $B_j$ be the set of nodes with value $j$ in this grid under allocation $\pi$. Note that the number of edges from $B_j$ to nodes with a different value is at least $\min\{\sqrt{|B_j|}, \ffrac{C}{2}\}$ (Lemma \ref{lem:grid-graph-property}). Therefore the total number of edges between nodes with different value is lower bounded by 
% \begin{align*}
%     \frac{\sum_{j=0}^k \min\{\sqrt{|B_j|}, \ffrac{C}{2}\}}{2} &\ge
%     \frac{\min \{\sum_{j=0}^k \sqrt{|B_j|}, \ffrac{C}{2}\}}{2} \\
%     &\ge \frac{\min\{ \sqrt{\sum_{j=0}^k |B_j|}, \ffrac{C}{2}\}}{2} \\
%     &\ge \ffrac{C}{4}.
% \end{align*}
\[    \frac{\sum_{j=0}^k \min\{\sqrt{|B_j|}, \ffrac{C}{2}\}}{2} \ge
    \frac{\min \{\sum_{j=0}^k \sqrt{|B_j|}, \ffrac{C}{2}\}}{2} 
    \ge \frac{\min\{ \sqrt{\sum_{j=0}^k |B_j|}, \ffrac{C}{2}\}}{2} 
    \ge \ffrac{C}{4}.
\]
Note that we divide by $2$ since each edge gets counted at most twice. Since each of these edges has an envy of $1$, we are done.

From here on, assume that each grid has a majority value. Note that a grid can only have one majority value.
Let grid $i$ correspond to $\Grid(C,Ca_i)$. For notational convenience, assume without loss of generality that grids $\{1, \dots, \ell_1\}$ have majority value $1$, $\{{\ell_1 +1}, \dots, {\ell_1 + \ell_2}\}$ have majority value $2$ and so on. Using analysis similar to \Cref{thm:trees-approx-lower-bound}, there exists one $j \in [m]$ such that $\sum_{h \in [\ell_j]}a_{\ell_1 + \dots + \ell_{j-1} +h} > T$. Assume again for notational convenience that $j = 1$. Since all these values are integers, we can restate it as $\sum_{h \in [\ell_1]}a_{h} \ge T + 1$. This implies $\sum_{h \in [\ell_1]} C^2a_{h} \ge C^2T + C^2$. 

Coming to our allocation $\pi$, $\sum_{h \in [\ell_1]} C^2a_{h} \ge C^2T + C^2$ implies that there are at least $C^2$ nodes in grids $\{1,2, \dots, \ell_1\}$ which are not allocated a value of $1$. Let $A_i$ correspond to the set of nodes in grid $i$ allocated a value other than $1$. Note that $\sum_{i \in [\ell_1]} |A_i| \ge C^2$. We can lower bound the number of edges from $A_i$ to nodes with value $1$ using Lemma \ref{lem:grid-graph-property} as follows:
% \begin{align*}
%     \sum_{i \in [\ell_1]} \min\{\sqrt{|A_i|}, \ffrac{C}{2}\} &\ge
%     \min\{\sum_{i \in [\ell_1]} \sqrt{|A_i|}, \ffrac{C}{2}\} \\
%     &\ge \min\{ \sqrt{\sum_{i \in [\ell_1]}|A_i|}, \ffrac{C}{2}\} \\
%     &\ge \ffrac{C}{2}.
% \end{align*}
\[
    \sum_{i \in [\ell_1]} \min\{\sqrt{|A_i|}, \ffrac{C}{2}\} \ge
    \min\{\sum_{i \in [\ell_1]} \sqrt{|A_i|}, \ffrac{C}{2}\} 
    \ge \min\{ \sqrt{\sum_{i \in [\ell_1]}|A_i|}, \ffrac{C}{2}\} 
    \ge \ffrac{C}{2} \ .
\] 
Each of these edges have envy at least $1$.

We set $C = 4\left \lceil (288\gamma m^3 T + 1)^{\ffrac{1}{(2\varepsilon)}} \right \rceil$ to complete the reduction. When there is no valid $3$-partition, the total minimum envy (and therefore, the envy output by $\mathsf{ALG}_{\cal G}$) is at least $\ffrac{C}{4} = \left \lceil (288\gamma m^3 T + 1)^{\ffrac{1}{(2\varepsilon)}} \right \rceil^2$. However, when there is a valid $3$-partition, the envy output by $\mathsf{ALG}_{\cal G}$ is strictly upper bounded by:
\begin{align*}
    3m^2\gamma(C^2mT+12m)^{0.5-\varepsilon} \le 36m^3 \gamma TC^{1-2\varepsilon} <& 144m^3 \gamma T \left ( \left \lceil (288\gamma m^3 T + 1)^{\ffrac{1}{(2\varepsilon)}} \right \rceil \right )^{1 - 2\varepsilon} \\ <& \left \lceil (288\gamma m^3 T + 1)^{\ffrac{1}{(2\varepsilon)}} \right \rceil.
\end{align*}
\end{proof}


\boundeddeg*
\begin{proof}
We, again, present a reduction from the \UTP Problem. 
For some constant $\varepsilon > 0$, assume there is an efficient $O(n^{1 - \varepsilon})$ approximation algorithm $\mathsf{ALG}_{\cal G}$ where $\cal G$ is the set of all connected graphs with degree at most $4$. For all instances, $\mathsf{ALG}_{\cal G}$ outputs an allocation with total envy within a multiplicative factor of $\gamma n^{1- \varepsilon}$ to the optimal envy for some constant $\gamma$.

Given an instance of \UTP, we construct $3m$ disjoint graphs as follows: for each $a_i$ in the multi-set $A$, we construct a 3-regular Ramanujan bipartite multi-graph of size $Ca_i$. This can be done in polynomial time and is well-defined when $Ca_i$ is an even integer greater than or equal to $6$ \citep{cohen2016ramanujan}. We will ensure this by setting $C$ appropriately. Note that these $3m$ multi-graphs still may have multiple edges connecting the same $2$ nodes. To convert them into simple graphs, we simply remove any repeated edges. For each $a_i \in A$, we refer to this graph using $R'(3, Ca_i)$ or  $R'$-graph $i$. These graphs are neither Ramanujan graphs nor are they $3$-regular. However, as we will crucially show, they still have the expansion properties we require. This result will use the Cheeger's inequality from \citet[Section 9.2]{alon04probabilistic} applied to Ramanujan graphs \citep{lubotzky1988ramanujan}. While the result in \citet[Section 9.2]{alon04probabilistic} is for simple graphs, the exact same proof can be extended to multi-graphs; so we present it without proof.

\begin{lemma}[Cheeger's Inequality]\label{lem:cheegers-inequality-apdx}
Let $G'$ be a $d$-regular Ramanujan (multi-)graph defined on a set of $V$ nodes. The following holds:
\begin{align*}
    \min_{S \subseteq V: 0 < |S| \le |V|/2} \frac{\delta_{G'}(S)}{|S|} \ge \frac12(d - 2\sqrt{d-1}),
\end{align*}
where $\delta_{G'}(S)$ denotes the number of edges in the $(S, V \setminus S)$ cut in the graph $G'$.
\end{lemma}

\begin{lemma}\label{lem:ramanujan-graph-property}
For any $a_i \in A$, let $G' = R'(3, Ca_i)$ be a connected graph on a set of nodes $V$ with $R'(3, Ca_i)$ defined as above. Let $B$ be a set of nodes in $G'$ such that $|B| \le \ffrac{Ca_i}{2}$. Then, the cut $(B, V \setminus B)$ has at least $\ffrac{|B|}{100}$ edges; that is $\delta_{G'}(B) \ge \ffrac{|B|}{100}$. 
\end{lemma}
\begin{proof}
Consider the original Ramanujan multi-graph $R(3, Ca_i)$ which creates $R'(3, Ca_i)$ after removing repeated edges. In $R(3, Ca_i)$, the cut $(B, V \setminus B)$ has at least $\ffrac{3|B|}{100}$ edges. This follows from the Cheeger's inequality (Lemma \ref{lem:cheegers-inequality-apdx}). 

In $R'(3, Ca_i)$, the cut size drops by a factor of at most $3$ since we only remove repeated edges and there can be at most $3$ edges between any two nodes in the graph. This gives us the required result.
\end{proof}

We construct an instance of the graphical house allocation problem by placing these $3m$ $R'$-graphs at unique leaves of a binary tree $T_r$ (see Figure \ref{fig:bounded-degree-reduction-graph}). Like Theorem \ref{thm:bounded-degree-planar-approx-lower-bound}, the binary tree $T_r$ is constructed such that it has at least $3m$ leaves and at most $12m$ nodes. It is easy to see that the smallest balanced binary tree with at least $3m$ leaves satisfies this constraint. It also easy to see that this graph has maximum a degree of $4$.

There are $CmT + |T_r|$ nodes in this graph.
The valuations of the houses are defined the same way as before: for each $j \in [m]$, there are $CT$ houses valued at $j$ and there are $|T_r|$ houses valued at $0$. $C$ again will be decided later. Along with ensuring $C$ is polynomial, we will also ensure that $C$ is even and greater than or equal to $6$. 


We show that the minimum total envy output by $\mathsf{ALG}_{\cal G}$ for the house allocation instance is greater than or equal to $\left \lceil (14400\gamma m^3 T + 1)^{\ffrac{1}{\varepsilon}} \right \rceil$ if and only if there exists {\em no} valid partition for the original \textsc{3-Partition} instance i.e. the original instance was a NO instance. 


$(\Leftarrow)$ Assume there is a valid $3$-partition for the original instance. This case follows the exact same way as Theorem \ref{thm:trees-approx-lower-bound}. The minimum envy is upper bounded by $3m^2$ and the envy output by $\mathsf{ALG}_{\cal G}$ is upper bounded by $3m^2\gamma(CmT+|T_r|)^{1-\varepsilon} \le 3m^2\gamma(CmT+12m)^{1-\varepsilon}$.

$(\Rightarrow)$ Assume there is no valid $3$-partition in the original instance. We will show that any allocation must have an envy of at least $\ffrac{C}{200}$. 

Similar to Theorem \ref{thm:general-approx-lower-bound}, we refer to a value $j \in [m] \cup \{0\}$ as a majority value of a graph $R'(3, Ca_i)$ if this value has been allocated to at least $\ffrac{Ca_i}{2}$ nodes in the grid. 

For some allocation $\pi$, assume there exists a graph $R'(3, Ca_1)$ with no majority value. Let $B_j$ be the set of nodes with value $j$ in this graph under allocation $\pi$. Note that the number of edges from $B_j$ to nodes with a different value is at least $\ffrac{|B_j|}{100}$ (Lemma \ref{lem:ramanujan-graph-property}). Therefore the total number of edges between nodes with different value is lower bounded by $\ffrac{C}{200}$.
We divide by $2$ since each edge gets counted at most twice. Since each of these edges has an envy of $1$, we are done.

From here on, assume that each $R'$-graph has a majority value. 
Let graph $i$ correspond to $R'(3,Ca_i)$. For notational convenience, assume without loss of generality that graphs $\{1, \dots, \ell_1\}$ have majority value $1$, $\{{\ell_1 +1}, \dots, {\ell_1 + \ell_2}\}$ have majority value $2$ and so on. Using analysis similar to \Cref{thm:trees-approx-lower-bound}, there exists one $j \in [m]$ such that $\sum_{h \in [\ell_j]}a_{\ell_1 + \dots + \ell_{j-1} +h} > T$. Assume again for notational convenience that $j = 1$. Since all these values are integers, we can restate it as $\sum_{h \in [\ell_1]}a_{h} \ge T + 1$. This implies $\sum_{h \in [\ell_1]} Ca_{h} \ge CT + C$. 

Coming to our allocation $\pi$, $\sum_{h \in [\ell_1]} Ca_{h} \ge CT + C$ implies that there are at least $C$ nodes in graphs $\{1,2, \dots, \ell_1\}$ which are not allocated a value of $1$. Let $A_i$ correspond to the set of nodes in grid $i$ allocated a value other than $1$. Note that $\sum_{i \in [\ell_1]} |A_i| \ge C$. For each $i$, the number of edges between nodes in $A_i$ and nodes with value $1$, is at least $\ffrac{|A_i|}{100}$ (Lemma \ref{lem:ramanujan-graph-property}). This gives us a lower bound of $\ffrac{C}{100}$ edges with envy at least $1$.


We set $C = 200\left \lceil (14400\gamma m^3 T + 1)^{\ffrac{1}{\varepsilon}} \right \rceil$ to complete the reduction; this setting crucially ensures that each $Ca_i$ is even and at least $6$. When there is no valid $3$-partition, the total minimum envy (and therefore, the envy output by $\mathsf{ALG}_{\cal G}$) is at least $\ffrac{C}{200} = \left \lceil (14400\gamma m^3 T + 1)^{\ffrac{1}{\varepsilon}} \right \rceil$. However, when there is a valid $3$-partition, the envy output by $\mathsf{ALG}_{\cal G}$ is strictly upper bounded by:
\begin{align*}
    3m^2\gamma(CmT+12m)^{1-\varepsilon} \le 36m^3 \gamma TC^{1-\varepsilon} < 7200m^3 \gamma T \left ( \left \lceil (14400\gamma m^3 T + 1)^{\ffrac{1}{\varepsilon}} \right \rceil \right )^{1 - \varepsilon} < \left \lceil (14400\gamma m^3 T + 1)^{\ffrac{1}{\varepsilon}} \right \rceil.
\end{align*}
This concludes the proof.
\end{proof}

\end{document}