\appendix

\section{Proofs from Section \ref{sec:lower}}\label{apx:lower}

\generalgraphs*
\begin{proof}
We present a similar reduction from the \UTP Problem. The main difference in construction from Theorem \ref{thm:trees-approx-lower-bound} is that $x_i$ and its children are replaced with a clique of size $Ca_i$. We use the high density of the clique to show that the envy is much higher $(\Omega(C^2))$ in the absence of a valid $3$-partition.  

For some constant $\varepsilon > 0$, assume there is an efficient $O(n^{2- \varepsilon})$ approximation algorithm $\mathsf{ALG}_{\cal G}$ where $\cal G$ is the class of all connected graphs. Assume there is some constant $\gamma$ such that for all instances on connected graphs, $\mathsf{ALG}_{\cal G}$ outputs an allocation with total envy within a multiplicative factor of $\gamma n^{2- \varepsilon}$ to the optimal envy.

Given an instance of \UTP, we construct an instance of {\GHA} as follows: for each value $a_i$ in the multiset $A$, we create a clique of size $Ca_i$. We then connect these $3m$ cliques to a node $r$ (as described in Figure \ref{fig:general-reduction-graph}). Once again, $C$ is a positive integer whose exact value we shall choose later.
The total number of nodes in this graph is $CmT+1$, same as in the other proofs before in this section. The house values are defined similarly as well: for each $j \in [m]$, there are $CT$ houses valued at $j$ and there is one house valued at $0$. 


We show that, with the appropriate $C$, the total envy output by $\mathsf{ALG}_{\cal G}$ for the constructed {\GHA} instance is greater than or equal to $\left \lceil (96\gamma m^4 T^2 + 1)^{\ffrac{1}{\varepsilon}} \right \rceil^2$ if and only if there exists {\em no} valid solution for the original \textsc{3-Partition} instance, i.e.,~the original instance was a NO instance. 


$(\Leftarrow)$ Assume there is a valid $3$-partition for the original instance. This case follows the exact same way as Theorem \ref{thm:trees-approx-lower-bound}. The minimum envy is upper bounded by $3m^2$ and the envy output by $\mathsf{ALG}_{\cal G}$ is upper bounded by $3m^2\gamma(CmT+1)^{2-\varepsilon}$.

$(\Rightarrow)$ Assume there is no valid $3$-partition in the original instance. We will show that any allocation must have an envy of at least $\ffrac{C^2}{4}$. 

If, for some allocation $\pi$, there exists a clique where no value $j \in [m] \cup \{0\}$ is allocated to more than half of its nodes, then this lower bound trivially holds. Since the clique has a size of at least $C$, each node in the clique envies at least $C/2$ neighbors by at least $1$. 

Assume that for all cliques, there is some value allocated to at least half the nodes in the clique; we refer to this value as a {\em majority value} of the clique. Let clique $i$ correspond to the clique $K_{Ca_i}$. For notational convenience, assume without loss of generality that cliques $\{1, \dots, \ell_1\}$ have majority value $1$, $\{{\ell_1 +1}, \dots, {\ell_1 + \ell_2}\}$ have majority value $2$ and so on. Using analysis similar to \Cref{thm:trees-approx-lower-bound}, there exists at least one $j \in [m]$ such that $\sum_{h \in [\ell_j]}a_{\ell_1 + \dots + \ell_{j-1} +h} > T$. Assume again for notational convenience that $j = 1$. Since all these values are integers, we can restate the equation above as $\sum_{h \in [\ell_1]}a_{h} \ge T+ 1$. This implies $\sum_{h \in [\ell_1]} Ca_{h} \ge CT + C$. 

Coming back to our allocation $\pi$, $\sum_{h \in [\ell_1]} Ca_{h} \ge CT + C$ implies that there are at least $C$ nodes in cliques $\{1,2, \dots, \ell_1\}$ which are not allocated a value of $1$. Since $1$ is a majority value in each of these cliques, the envy that each of the $C$ nodes in $S$ will have towards the nodes with value $1$ is at least $C/2$. This implies that the total envy of allocation $\pi$ is at least $C^2/2$. 

We set $C = 2\left \lceil (96\gamma m^4 T^2 + 1)^{\ffrac{1}{\varepsilon}} \right \rceil$ to complete the reduction. When there is no valid $3$-partition, the minimum total envy (and therefore, the envy output by $\mathsf{ALG}_{\cal G}$) is at least $\ffrac{C^2}{4} = \left \lceil (96\gamma m^4 T^2 + 1)^{\ffrac{1}{\varepsilon}} \right \rceil^2$. However, when there is a valid $3$-partition, the envy output by $\mathsf{ALG}_{\cal G}$ is strictly upper bounded by:
\begin{align*}
    3m^2\gamma(CmT+1)^{2-\varepsilon} \le 6m^4 \gamma T^2C^{2-\varepsilon} \le 24m^4 \gamma T^2 \left ( \left \lceil (96\gamma m^4 T^2 + 1)^{\ffrac{1}{\varepsilon}} \right \rceil \right )^{2 - \varepsilon} < \left \lceil (96\gamma m^4 T^2 + 1)^{\ffrac{1}{\varepsilon}} \right \rceil^2.
\end{align*}
This concludes the proof.
\end{proof}


\gridlemma*
\begin{proof}
If $A$ consists of at least one node from each row, then since $|A| \le \ffrac{rc}{2}$, there will be at least $r/2$ rows with a node in $V \setminus A$. Therefore, there will be at least $\ffrac{r}{2}$ edges in the cut. Similarly, if $A$ consists of at least one node from each column, there will be $\ffrac{c}{2} \ge \ffrac{r}{2}$ edges in the cut. 

Otherwise, there is some row and some column containing only nodes in $V \setminus A$.
Note that there must either be at least $\sqrt{|A|}$ rows with a node in $A$ or at least $\sqrt{|A|}$ columns with a node in $A$. Assume WLOG there are at least $\sqrt{|A|}$ rows with a node in $A$. Each of these rows intersects the column that only has nodes from $V\setminus A$, and so each of the $\sqrt{|A|}$ rows must contain an edge between $A$ and $V\setminus A$.
\end{proof}


\bdp*
\begin{proof}
We present a similar reduction from the \UTP Problem. 
For some constant $\varepsilon > 0$, assume there is an efficient $O(n^{0.5 - \varepsilon})$ approximation algorithm $\mathsf{ALG}_{\cal G}$ where $\cal G$ corresponds to the class of all planar graphs with max degree at most $5$. In other words, for all instances on these graphs, $\mathsf{ALG}_{\cal G}$ outputs an allocation with total envy within a multiplicative factor of $\gamma n^{0.5- \varepsilon}$ to the optimal envy, where $\gamma$ is some fixed constant.

Given an instance of \UTP, we construct an instance of {\GHA} as follows: the graph $G$ has $3m$ grids, each grid $i \in [3m]$ has $C$ rows and $C a_i$ columns for some $C$ we will define later. Each of the grids has a single edge to a unique leaf of a binary tree $T_r$ (see Figure \ref{fig:bdp-reduction-graph}). 

The binary tree $T_r$ is constructed such that it has at most $12m$ nodes. It is easy to see that the smallest balanced binary tree with at least $3m$ leaves satisfies this constraint. This construction is trivially a planar graph with degree at most $5$. There are $C^2mT + |T_r|$ nodes in this graph. The house values are defined roughly the same way as before: for each $j \in [m]$, there are $C^2T$ houses valued at $j$ and there are $|T_r|$ houses valued at $0$. As long as $C$ and $|T_r|$ are polynomial, this is a polynomial-time reduction. 


We show that, with an appropriate choice of $C$, the minimum total envy output by $\mathsf{ALG}_{\cal G}$ for the house allocation instance is greater than or equal to $\left \lceil (288\gamma m^3 T + 1)^{\ffrac{1}{(2\varepsilon)}} \right \rceil$ if and only if there exists {\em no} valid partition for the original \textsc{3-Partition} instance i.e. the original instance was a NO instance. 


$(\Leftarrow)$ Assume there is a valid $3$-partition for the original instance. This case follows the exact same way as Theorem \ref{thm:trees-approx-lower-bound}. The minimum envy is upper bounded by $3m^2$ and the envy output by $\mathsf{ALG}_{\cal G}$ is upper bounded by $3m^2\gamma(C^2mT+|T_r|)^{0.5-\varepsilon} \le 3m^2\gamma(C^2mT+12m)^{0.5-\varepsilon}$.

$(\Rightarrow)$ Assume there is no valid $3$-partition in the original instance. We will show that any allocation must have an envy of at least $\ffrac{C}{4}$. 

Similar to Theorem \ref{thm:general-approx-lower-bound}, we refer to a value $j \in [m] \cup \{0\}$ as a majority value of a grid $\Grid(C, Ca_i)$ if this value has been allocated to at least $\ffrac{C^2a_i}{2}$ nodes in the grid. 

For some allocation $\pi$, assume there exists a grid, $\Grid(C, Ca_1)$ with no majority value. Then this grid has nodes with at least $3$ different values. Let $B_j$ be the set of nodes with value $j$ in this grid under allocation $\pi$. Note that the number of edges from $B_j$ to nodes with a different value is at least $\min\{\sqrt{|B_j|}, \ffrac{C}{2}\}$ (Lemma \ref{lem:grid-graph-property}). Therefore the total number of edges between nodes with different value is lower bounded by 
% \begin{align*}
%     \frac{\sum_{j=0}^k \min\{\sqrt{|B_j|}, \ffrac{C}{2}\}}{2} &\ge
%     \frac{\min \{\sum_{j=0}^k \sqrt{|B_j|}, \ffrac{C}{2}\}}{2} \\
%     &\ge \frac{\min\{ \sqrt{\sum_{j=0}^k |B_j|}, \ffrac{C}{2}\}}{2} \\
%     &\ge \ffrac{C}{4}.
% \end{align*}
\[    \frac{\sum_{j=0}^k \min\{\sqrt{|B_j|}, \ffrac{C}{2}\}}{2} \ge
    \frac{\min \{\sum_{j=0}^k \sqrt{|B_j|}, \ffrac{C}{2}\}}{2} 
    \ge \frac{\min\{ \sqrt{\sum_{j=0}^k |B_j|}, \ffrac{C}{2}\}}{2} 
    \ge \ffrac{C}{4}.
\]
Note that we divide by $2$ since each edge gets counted at most twice. Since each of these edges has an envy of $1$, we are done.

From here on, assume that each grid has a majority value. Note that a grid can only have one majority value.
Let grid $i$ correspond to $\Grid(C,Ca_i)$. For notational convenience, assume without loss of generality that grids $\{1, \dots, \ell_1\}$ have majority value $1$, $\{{\ell_1 +1}, \dots, {\ell_1 + \ell_2}\}$ have majority value $2$ and so on. Using analysis similar to \Cref{thm:trees-approx-lower-bound}, there exists one $j \in [m]$ such that $\sum_{h \in [\ell_j]}a_{\ell_1 + \dots + \ell_{j-1} +h} > T$. Assume again for notational convenience that $j = 1$. Since all these values are integers, we can restate it as $\sum_{h \in [\ell_1]}a_{h} \ge T + 1$. This implies $\sum_{h \in [\ell_1]} C^2a_{h} \ge C^2T + C^2$. 

Coming to our allocation $\pi$, $\sum_{h \in [\ell_1]} C^2a_{h} \ge C^2T + C^2$ implies that there are at least $C^2$ nodes in grids $\{1,2, \dots, \ell_1\}$ which are not allocated a value of $1$. Let $A_i$ correspond to the set of nodes in grid $i$ allocated a value other than $1$. Note that $\sum_{i \in [\ell_1]} |A_i| \ge C^2$. We can lower bound the number of edges from $A_i$ to nodes with value $1$ using Lemma \ref{lem:grid-graph-property} as follows:
% \begin{align*}
%     \sum_{i \in [\ell_1]} \min\{\sqrt{|A_i|}, \ffrac{C}{2}\} &\ge
%     \min\{\sum_{i \in [\ell_1]} \sqrt{|A_i|}, \ffrac{C}{2}\} \\
%     &\ge \min\{ \sqrt{\sum_{i \in [\ell_1]}|A_i|}, \ffrac{C}{2}\} \\
%     &\ge \ffrac{C}{2}.
% \end{align*}
\[
    \sum_{i \in [\ell_1]} \min\{\sqrt{|A_i|}, \ffrac{C}{2}\} \ge
    \min\{\sum_{i \in [\ell_1]} \sqrt{|A_i|}, \ffrac{C}{2}\} 
    \ge \min\{ \sqrt{\sum_{i \in [\ell_1]}|A_i|}, \ffrac{C}{2}\} 
    \ge \ffrac{C}{2} \ .
\] 
Each of these edges have envy at least $1$.

We set $C = 4\left \lceil (288\gamma m^3 T + 1)^{\ffrac{1}{(2\varepsilon)}} \right \rceil$ to complete the reduction. When there is no valid $3$-partition, the total minimum envy (and therefore, the envy output by $\mathsf{ALG}_{\cal G}$) is at least $\ffrac{C}{4} = \left \lceil (288\gamma m^3 T + 1)^{\ffrac{1}{(2\varepsilon)}} \right \rceil^2$. However, when there is a valid $3$-partition, the envy output by $\mathsf{ALG}_{\cal G}$ is strictly upper bounded by:
\begin{align*}
    3m^2\gamma(C^2mT+12m)^{0.5-\varepsilon} \le 36m^3 \gamma TC^{1-2\varepsilon} <& 144m^3 \gamma T \left ( \left \lceil (288\gamma m^3 T + 1)^{\ffrac{1}{(2\varepsilon)}} \right \rceil \right )^{1 - 2\varepsilon} \\ <& \left \lceil (288\gamma m^3 T + 1)^{\ffrac{1}{(2\varepsilon)}} \right \rceil.
\end{align*}
\end{proof}


\boundeddeg*
\begin{proof}
We, again, present a reduction from the \UTP Problem. 
For some constant $\varepsilon > 0$, assume there is an efficient $O(n^{1 - \varepsilon})$ approximation algorithm $\mathsf{ALG}_{\cal G}$ where $\cal G$ is the set of all connected graphs with degree at most $4$. For all instances, $\mathsf{ALG}_{\cal G}$ outputs an allocation with total envy within a multiplicative factor of $\gamma n^{1- \varepsilon}$ to the optimal envy for some constant $\gamma$.

Given an instance of \UTP, we construct $3m$ disjoint graphs as follows: for each $a_i$ in the multi-set $A$, we construct a 3-regular Ramanujan bipartite multi-graph of size $Ca_i$. This can be done in polynomial time and is well-defined when $Ca_i$ is an even integer greater than or equal to $6$ \citep{cohen2016ramanujan}. We will ensure this by setting $C$ appropriately. Note that these $3m$ multi-graphs still may have multiple edges connecting the same $2$ nodes. To convert them into simple graphs, we simply remove any repeated edges. For each $a_i \in A$, we refer to this graph using $R'(3, Ca_i)$ or  $R'$-graph $i$. These graphs are neither Ramanujan graphs nor are they $3$-regular. However, as we will crucially show, they still have the expansion properties we require. This result will use the Cheeger's inequality from \citet[Section 9.2]{alon04probabilistic} applied to Ramanujan graphs \citep{lubotzky1988ramanujan}. While the result in \citet[Section 9.2]{alon04probabilistic} is for simple graphs, the exact same proof can be extended to multi-graphs; so we present it without proof.

\begin{lemma}[Cheeger's Inequality]\label{lem:cheegers-inequality-apdx}
Let $G'$ be a $d$-regular Ramanujan (multi-)graph defined on a set of $V$ nodes. The following holds:
\begin{align*}
    \min_{S \subseteq V: 0 < |S| \le |V|/2} \frac{\delta_{G'}(S)}{|S|} \ge \frac12(d - 2\sqrt{d-1}),
\end{align*}
where $\delta_{G'}(S)$ denotes the number of edges in the $(S, V \setminus S)$ cut in the graph $G'$.
\end{lemma}

\begin{lemma}\label{lem:ramanujan-graph-property}
For any $a_i \in A$, let $G' = R'(3, Ca_i)$ be a connected graph on a set of nodes $V$ with $R'(3, Ca_i)$ defined as above. Let $B$ be a set of nodes in $G'$ such that $|B| \le \ffrac{Ca_i}{2}$. Then, the cut $(B, V \setminus B)$ has at least $\ffrac{|B|}{100}$ edges; that is $\delta_{G'}(B) \ge \ffrac{|B|}{100}$. 
\end{lemma}
\begin{proof}
Consider the original Ramanujan multi-graph $R(3, Ca_i)$ which creates $R'(3, Ca_i)$ after removing repeated edges. In $R(3, Ca_i)$, the cut $(B, V \setminus B)$ has at least $\ffrac{3|B|}{100}$ edges. This follows from the Cheeger's inequality (Lemma \ref{lem:cheegers-inequality-apdx}). 

In $R'(3, Ca_i)$, the cut size drops by a factor of at most $3$ since we only remove repeated edges and there can be at most $3$ edges between any two nodes in the graph. This gives us the required result.
\end{proof}

We construct an instance of the graphical house allocation problem by placing these $3m$ $R'$-graphs at unique leaves of a binary tree $T_r$ (see Figure \ref{fig:bounded-degree-reduction-graph}). Like Theorem \ref{thm:bounded-degree-planar-approx-lower-bound}, the binary tree $T_r$ is constructed such that it has at least $3m$ leaves and at most $12m$ nodes. It is easy to see that the smallest balanced binary tree with at least $3m$ leaves satisfies this constraint. It also easy to see that this graph has maximum a degree of $4$.

There are $CmT + |T_r|$ nodes in this graph.
The valuations of the houses are defined the same way as before: for each $j \in [m]$, there are $CT$ houses valued at $j$ and there are $|T_r|$ houses valued at $0$. $C$ again will be decided later. Along with ensuring $C$ is polynomial, we will also ensure that $C$ is even and greater than or equal to $6$. 


We show that the minimum total envy output by $\mathsf{ALG}_{\cal G}$ for the house allocation instance is greater than or equal to $\left \lceil (14400\gamma m^3 T + 1)^{\ffrac{1}{\varepsilon}} \right \rceil$ if and only if there exists {\em no} valid partition for the original \textsc{3-Partition} instance i.e. the original instance was a NO instance. 


$(\Leftarrow)$ Assume there is a valid $3$-partition for the original instance. This case follows the exact same way as Theorem \ref{thm:trees-approx-lower-bound}. The minimum envy is upper bounded by $3m^2$ and the envy output by $\mathsf{ALG}_{\cal G}$ is upper bounded by $3m^2\gamma(CmT+|T_r|)^{1-\varepsilon} \le 3m^2\gamma(CmT+12m)^{1-\varepsilon}$.

$(\Rightarrow)$ Assume there is no valid $3$-partition in the original instance. We will show that any allocation must have an envy of at least $\ffrac{C}{200}$. 

Similar to Theorem \ref{thm:general-approx-lower-bound}, we refer to a value $j \in [m] \cup \{0\}$ as a majority value of a graph $R'(3, Ca_i)$ if this value has been allocated to at least $\ffrac{Ca_i}{2}$ nodes in the grid. 

For some allocation $\pi$, assume there exists a graph $R'(3, Ca_1)$ with no majority value. Let $B_j$ be the set of nodes with value $j$ in this graph under allocation $\pi$. Note that the number of edges from $B_j$ to nodes with a different value is at least $\ffrac{|B_j|}{100}$ (Lemma \ref{lem:ramanujan-graph-property}). Therefore the total number of edges between nodes with different value is lower bounded by $\ffrac{C}{200}$.
We divide by $2$ since each edge gets counted at most twice. Since each of these edges has an envy of $1$, we are done.

From here on, assume that each $R'$-graph has a majority value. 
Let graph $i$ correspond to $R'(3,Ca_i)$. For notational convenience, assume without loss of generality that graphs $\{1, \dots, \ell_1\}$ have majority value $1$, $\{{\ell_1 +1}, \dots, {\ell_1 + \ell_2}\}$ have majority value $2$ and so on. Using analysis similar to \Cref{thm:trees-approx-lower-bound}, there exists one $j \in [m]$ such that $\sum_{h \in [\ell_j]}a_{\ell_1 + \dots + \ell_{j-1} +h} > T$. Assume again for notational convenience that $j = 1$. Since all these values are integers, we can restate it as $\sum_{h \in [\ell_1]}a_{h} \ge T + 1$. This implies $\sum_{h \in [\ell_1]} Ca_{h} \ge CT + C$. 

Coming to our allocation $\pi$, $\sum_{h \in [\ell_1]} Ca_{h} \ge CT + C$ implies that there are at least $C$ nodes in graphs $\{1,2, \dots, \ell_1\}$ which are not allocated a value of $1$. Let $A_i$ correspond to the set of nodes in grid $i$ allocated a value other than $1$. Note that $\sum_{i \in [\ell_1]} |A_i| \ge C$. For each $i$, the number of edges between nodes in $A_i$ and nodes with value $1$, is at least $\ffrac{|A_i|}{100}$ (Lemma \ref{lem:ramanujan-graph-property}). This gives us a lower bound of $\ffrac{C}{100}$ edges with envy at least $1$.


We set $C = 200\left \lceil (14400\gamma m^3 T + 1)^{\ffrac{1}{\varepsilon}} \right \rceil$ to complete the reduction; this setting crucially ensures that each $Ca_i$ is even and at least $6$. When there is no valid $3$-partition, the total minimum envy (and therefore, the envy output by $\mathsf{ALG}_{\cal G}$) is at least $\ffrac{C}{200} = \left \lceil (14400\gamma m^3 T + 1)^{\ffrac{1}{\varepsilon}} \right \rceil$. However, when there is a valid $3$-partition, the envy output by $\mathsf{ALG}_{\cal G}$ is strictly upper bounded by:
\begin{align*}
    3m^2\gamma(CmT+12m)^{1-\varepsilon} \le 36m^3 \gamma TC^{1-\varepsilon} < 7200m^3 \gamma T \left ( \left \lceil (14400\gamma m^3 T + 1)^{\ffrac{1}{\varepsilon}} \right \rceil \right )^{1 - \varepsilon} < \left \lceil (14400\gamma m^3 T + 1)^{\ffrac{1}{\varepsilon}} \right \rceil.
\end{align*}
This concludes the proof.
\end{proof}