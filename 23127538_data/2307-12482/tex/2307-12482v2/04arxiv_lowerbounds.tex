\section{Lower Bounds}\label{sec:lower}

%All the approximation guarantees presented in the previous section may appear weak at first.
Every algorithm presented in Section \ref{sec:upper} is value-agnostic. It might seem reasonable to assume, therefore, that there are more powerful approximation schemes that exploit the numerical values in $H$ in some way. Indeed, our results on random graphs suggest that, for most graphs, we can do significantly better. Remarkably, we show in this section that this is \emph{not} the case, and our value-agnostic algorithms are strong enough to give us nearly optimal approximation guarantees. Specifically, we show inapproximability results matching our upper bounds (up to $\polylog$ factors) for every class of graphs considered. Our lower bounds will use reductions from the {\UTP} problem.

\begin{definition}[\TP]
Given a multiset of $3m$ naturals $A = \{a_1, \dots, a_{3m}\} \subseteq \mathbb{N}_{> 0}$ and a natural $T \in \mathbb{N}_{> 0}$ such that $\sum_{j \in [3m]} a_j = mT$, {\TP} asks whether $A$ can be partitioned into $m$ triplets $(S_1, S_2, \dots, S_m)$ such that the sum of each triplet is equal to $T$.
\end{definition}

The \TP problem is NP-complete even when all the inputs are given in unary and each item in $A$ is strictly between $\ffrac{T}{4}$ and $\ffrac{T}{2}$ \citep{garey1979computers}. We refer to this variant as {\UTP}. Note that {\UTP} is just a reformulation of \textsc{Bin Packing}: there are $3m$ integers that sum to $mT$, and we wish to fit these integers into $m$ bins each of capacity $T$. The condition of three integers in each bin is redundant, as it is implied by the constraint that each integer is strictly between $T/4$ and $T/2$.

% We present our lower bounds in increasing order of proof complexity, starting with trees and moving to general graphs, planar graphs, bounded-degree graphs, and finally, bounded-degree trees.
Some of our results and proofs in this section (specifically Theorems \ref{thm:trees-approx-lower-bound} and \ref{thm:bounded-degree-planar-approx-lower-bound}) are very similar to results about the inapproximability of the {\em balanced graph partition} problem \citep{feldmann2012gridpartition, feldmann2015treepartition}. The rest of our proofs use novel gadgets and techniques.
% We relegate all proofs in this section to the appendix but provide basic sketches and ideas wherever possible.

\subsection{Trees and Planar Graphs}\label{sec:treelowerbound}
Recall that we presented two approximation guarantees for trees, $O(n)$ (Proposition \ref{prop:trivialgeneral}) and $O(\Delta \log n)$ (Corollary \ref{cor:cutwidth-upperbounds}). Both of these results are $\tilde{O}(n)$ in the worst case. 

\begin{restatable}{theorem}{thmtreesapproxlowerbound}\label{thm:trees-approx-lower-bound}
For any constant $\varepsilon > 0$, there is no efficient $O(n^{1-\varepsilon})$ approximation algorithm for {\GHA} on depth-$2$ trees unless P = NP.
\end{restatable}
% \begin{proof}[Proof Sketch]
% Figure environment removed

% Given a \UTP instance, we construct a graph according to Figure \ref{fig:trees-reduction-tree} where $C$ is some positive integer we will decide later. The multiset of house values consists of $CT$ houses with value $j$ for each $j \in [m]$, and one house with value $0$.

% If there is a valid $3$-partition, we can construct an allocation with envy at most $3m^2$. If there is no $3$-partition, any allocation must have envy at least $C$. We can now set $C$ appropriately.
% \end{proof}

\begin{proof}
We give a reduction from \UTP. For some constant $\varepsilon > 0$, assume there is an efficient $O(n^{1- \varepsilon})$ approximation algorithm $\mathsf{ALG}_{\cal G}$ where $\cal G$ corresponds to the class of depth-$2$ trees. In other words, there is a constant $\gamma$ such that for all instances $(G, H)$ with $G \in \mathcal{G}$, $\mathsf{ALG}_{\cal G}$ outputs an allocation with total envy within a multiplicative factor of $\gamma n^{1- \varepsilon}$ to the optimal envy.

Given an instance of \UTP, we construct an instance $(G, H)$ of {\GHA} as follows: The graph $G$ is a rooted depth-$2$ tree where the root $r$ of the tree has $3m$ children $\{x_1, \dots, x_{3m}\}$. Each of these nodes $x_i$ has $C a_i - 1$ children (see Figure \ref{fig:trees-reduction-tree}). Here, $C$ is a positive integer whose exact value we shall determine later.

The total number of nodes in $G$ is $1 + \sum_{i \in [3m]} Ca_i = 1 + CmT$, and so we must specify $1 + CmT$ house values in $H$. We define $H$ with $CT$ values of $j$ for $j \in [m]$, together with a single value of $0$. Note that this construction can be done in polynomial time as long as $C$ is polynomially large, since the input to the $3$-partition instance is given in unary. 

We show that, for an appropriate choice of $C$, the minimum total envy output by $\mathsf{ALG}_{\cal G}$ for the instance $(G, H)$ is at least $\left \lceil (12\gamma m^3 T + 1)^{\ffrac{1}{\varepsilon}} \right \rceil$ if and only if there exists {\em no} valid partition for the original \textsc{3-Partition} instance, i.e., the original instance was a NO instance. 


$(\Rightarrow)$ Assume there is a valid $3$-partition $(S_1, \ldots, S_m)$ for the original instance. We denote this $3$-partition using a mapping $\mu : A \to \{S_1, \ldots, S_m\}$ that maps each number in the multiset $A$ to one of the triplets $S_j$. We construct an allocation for the graph $G$ as follows: for any item $i$, if $\mu(i) = S_j$, we allocate houses with value $j$ to $x_i$ and all of its children. We finally allocate the house with value $0$ to the root. Note that this is a valid allocation: for any $j \in [m]$, we allocate exactly $\sum_{i \in [3m]: \mu(i) = S_j} Ca_i = CT$ houses of value $j$. 

We can easily upper bound the envy of this allocation; this upper bound also serves as an upper bound for the minimum total envy for the instance $(G, H)$. There is no envy between any $x_i$ and any of its children. So the only edges with potential envy are the ones incident on the root. There are $3m$ such edges, each incurring envy at most $m$. This gives us an upper bound of $3m^2$ on the total envy. This implies, when there is a valid $3$-partition, the approximation algorithm $\mathsf{ALG}_{\cal G}$ will output an allocation with envy at most $3m^2\gamma(1 + CmT)^{1-\varepsilon}$.

$(\Leftarrow)$ Assume there is no valid $3$-partition in the original instance. We will show that any allocation has a total envy of at least $C$. We do this by examining the houses allocated to the depth-1 nodes $\{x_1, \dots, x_{3m}\}$. If any $x_i$ is allocated a value of $0$ in some allocation $\pi$, then the allocation $\pi$ has a total envy of at least $C$ since any $x_i$ has at least $C -1$ children and $1$ parent receiving a value of at least $1$ each. 

So now assume no $x_i$ is allocated a value of $0$ in $\pi$. For notational convenience, assume WLOG that $x_1, \dots, x_{\ell_1}$ are allocated a value $1$, $x_{\ell_1 +1}, \dots, x_{\ell_1 + \ell_2}$ are allocated a value $2$ and so on. If for all $j \in [m]$, $\sum_{h \in [\ell_j]} a_{\ell_0 + \ell_1 + \dots + \ell_{j-1} + h} = T$ (with $\ell_0 = 0$), then we violate our assumption that there is no valid $3$-partition in the original instance. Therefore there exists one $j \in [m]$ such that $\sum_{h \in [\ell_j]}a_{\ell_0 + \ell_1 + \dots + \ell_{j-1} +h} > T$. Assume again for notational convenience that $j = 1$. Since all values are integers, we can restate the inequality above as $\sum_{h \in [\ell_1]}a_{h} \ge T + 1$. This implies $\sum_{h \in [\ell_1]} Ca_{h} \ge CT + C$. 

Coming back to the allocation $\pi$, we have that $\sum_{h \in [\ell_1]} Ca_{h} \ge CT + C$ implies that there are at least $C$ nodes out of all the children of $\{x_{1}, x_2, \dots, x_{\ell_1}\}$ which are not allocated a value of $1$. The envy that each of these $C$ nodes will have towards their parents is at least $1$. This implies that the total envy of allocation $\pi$ is at least $C$. 

We set $C = \left \lceil (12\gamma m^3 T + 1)^{\ffrac{1}{\varepsilon}} \right \rceil$ to complete the reduction. When there is no valid $3$-partition, the total minimum envy (and therefore, the envy output by $\mathsf{ALG}_{\cal G}$) is at least $C =\left \lceil (12\gamma m^3 T + 1)^{\ffrac{1}{\varepsilon}} \right \rceil$. However, when there is a valid $3$-partition, the envy output by $\mathsf{ALG}_{\cal G}$ is strictly upper bounded by:
\begin{align*}
    3m^2\gamma(CmT+1)^{1-\varepsilon} \le 6m^3 \gamma TC^{1-\varepsilon} \le 6m^3 \gamma T \left ( \left \lceil (12\gamma m^3 T + 1)^{\ffrac{1}{\varepsilon}} \right \rceil \right )^{1 - \varepsilon} < \left \lceil (12\gamma m^3 T + 1)^{\ffrac{1}{\varepsilon}} \right \rceil.
\end{align*}
This completes the proof.
\end{proof}



\subsection{General and Bounded-Degree Graphs}
In this section, we generalize the arguments from Section \ref{sec:treelowerbound} to other classes of graphs. 
The main technique is similar to that of Theorem \ref{thm:trees-approx-lower-bound}, so we just present ideas for the graph construction in each of these proofs, with the details in Appendix \ref{apdx:lower}.

We first match the $O(n^2)$ upper bound for connected graphs (\Cref{prop:trivialgeneral} and \Cref{cor:cutwidth-upperbounds}).

\begin{restatable}{theorem}{generalgraphs}\label{thm:general-approx-lower-bound}
For any constant $\varepsilon > 0$, there is no efficient $O(n^{2-\varepsilon})$ approximation algorithm for {\GHA} on connected graphs unless P = NP.
\end{restatable}
\begin{proof}[Proof Sketch]
We replace the $Ca_i$-sized stars in Figure \ref{fig:trees-reduction-tree} with $Ca_i$-sized cliques. The rest of the proof is similar to Theorem \ref{thm:trees-approx-lower-bound}.
\end{proof}



% \begin{proof}[Proof sketch]
%     Figure \ref{fig:general-reduction-graph} shows the graph we use for this reduction. The construction is similar to that in Theorem \ref{thm:trees-approx-lower-bound}, except that we use cliques of size $Ca_i$ attached to a common root $r$, instead of a vertices $x_i$ with $Ca_i - 1$ dangling leaves. The idea is to use the high density of the clique to show that the envy is $\Omega(C^2)$ when there is no valid $3$-partition. The house values $H$ are defined as in Theorem \ref{thm:trees-approx-lower-bound}. A similar argument works here too. A valid $3$-partition can be ``packed'' into the clusters of values, which would attain only envy from the edges incident to the root $r$. Conversely, if there is no valid $3$-partition, again using a packing argument, it can be shown that there are several high envy edges, giving rise to $\Omega(C^2)$ envy.
% \end{proof}





%\subsection{Bounded Degree Planar Graphs}

So far in our two lower bounds (Theorems \ref{thm:trees-approx-lower-bound} and \ref{thm:general-approx-lower-bound}), we were able to use simple counting techniques, because counting edges with non-zero envy in stars and cliques is straightforward. Our next results will require much more careful analysis.

We will start with bounded-degree planar graphs. Our reduction uses grid graphs instead of stars and cliques, and so we will need a technical lemma to help us with estimating the number of edges with nonzero envy.

\begin{restatable}{lemma}{gridlemma}\label{lem:grid-graph-property}
    Let $G = Grid(r, c)$ be a grid graph with $r$ rows and $c$ columns such that $r \le c$. Let $A \subseteq V$ be any set of nodes in this graph such that $|A| \le \ffrac{rc}{2}$. Then, $\delta_G(A) \geq \min\{\sqrt{|A|}, \ffrac{r}{2}\}$.
\end{restatable}
\begin{proof}
If $A$ consists of at least one node from each row, then since $|A| \le \ffrac{rc}{2}$, there will be at least $r/2$ rows with a node in $V \setminus A$. Therefore, there will be at least $\ffrac{r}{2}$ edges in the cut. Similarly, if $A$ consists of at least one node from each column, there will be $\ffrac{c}{2} \ge \ffrac{r}{2}$ edges in the cut. 

Otherwise, there is some row and some column containing only nodes in $V \setminus A$.
Note that there must either be at least $\sqrt{|A|}$ rows with a node in $A$ or at least $\sqrt{|A|}$ columns with a node in $A$. Assume WLOG there are at least $\sqrt{|A|}$ rows with a node in $A$. Each of these rows intersects the column that only has nodes from $V\setminus A$, and so each of the $\sqrt{|A|}$ rows must contain an edge between $A$ and $V\setminus A$.
\end{proof}



Armed with Lemma \ref{lem:grid-graph-property}, we can now present our lower bound on bounded-degree planar graphs.

\begin{restatable}{theorem}{bdp}\label{thm:bounded-degree-planar-approx-lower-bound}
For any constant $\varepsilon > 0$, no efficient $O(n^{0.5-\varepsilon})$ approximation algorithm exists for {\GHA} on bounded-degree planar graphs unless P = NP.
\end{restatable}
\begin{proof}[Proof Sketch]
We replace the stars of size $Ca_i$ in Figure \ref{fig:trees-reduction-tree} with grid graphs containing $C$ rows and $Ca_i$ columns. The rest of the proof flows similarly to Theorem \ref{thm:trees-approx-lower-bound}. Lemma \ref{lem:grid-graph-property} helps in estimating the envy blow-up if there is no $3$-partition.
\end{proof}



% \begin{proof}[Proof sketch]
%     Figure \ref{fig:bdp-reduction-graph} shows the graph we use for this reduction. Again, the construction is similar to before. The graph $G$ has $3m$ grids, each with $C$ rows and $Ca_i$ columns, attached by a single edge to a leaf of a ``small'' binary tree $T_r$. Note that this is a bounded-degree planar graph. The values $H$ are defined similarly to before, with a cluster of $C^2T$ values at each positive integer in $[3m]$, and $|T_r|$ values at $0$.

%     If there is a $3$-partition, the argument is exactly similar to those earlier in this section, with the only envy coming from the edges of the tree, incurring a total envy of $3m^2$.

%     Conversely, suppose there is no $3$-partition. Now, using Lemma \ref{lem:grid-graph-property}, we can show that a grid which does not have a majority of its vertices from the same cluster must incur at least $C/4$ envy. Otherwise, assuming each grid has a majority of its vertices in the same cluster, a packing argument shows that some grid must have a cut going across two different values, and this incurs at least $C/2$ envy, once again by Lemma \ref{lem:grid-graph-property}.
% \end{proof}

Note that Theorem \ref{thm:bounded-degree-planar-approx-lower-bound} matches the $O(\sqrt{n})$ upper bound from \Cref{cor:cutwidth-upperbounds}.

Our next lower bound applies to arbitrary bounded-degree graphs and matches the $O(n)$ upper bound from Proposition \ref{prop:trivialgeneral} and Corollary \ref{cor:cutwidth-upperbounds}. In this reduction, we use the recent polynomial-time algorithm \citep{cohen2016ramanujan} to compute bipartite Ramanujan multigraphs for any even number $m$ of vertices, and any degree $d \ge 3$. At a high level, we replace the star gadgets from the proof of Theorem \ref{thm:trees-approx-lower-bound} with these Ramanujan graphs and use the expansion properties of Ramanujan graphs to prove a lemma similar to (and stronger than) \Cref{lem:grid-graph-property}. 
% The analysis is quite involved and relegated to the appendix.

% One key element of the construction we will use is that we will take a gadget consisting of a $3$-regular Ramanujan bipartite multi-graph of a given even size (at least $6$), and remove any of its repeated edges until it is a simple graph. The crucial observation is that even though these new gadget graph is neither regular nor Ramanujan, it still has ``enough'' expansion for our purposes. We will need the following well-known result for this, stated without proof.

% \begin{lemma}[Cheeger's Inequality]\label{lem:cheegers-inequality}
% Let $G'$ be a $d$-regular Ramanujan (multi-)graph defined on a set of $V$ nodes. The following holds:
% \begin{align*}
%     \min_{S \subseteq V: 0 < |S| \le |V|/2} \frac{\delta_{G'}(S)}{|S|} \ge \frac12(d - 2\sqrt{d-1}),
% \end{align*}
% where $\delta_{G'}(S)$ denotes the number of edges in the $(S, V \setminus S)$ cut in the graph $G'$.
% \end{lemma}

% The following theorem characterizes the inapproximability on bounded-degree graphs. See Appendix \ref{apx:lower} for the full proof.


\begin{restatable}{theorem}{boundeddeg}\label{thm:bounded-degree-approx-lower-bound}
For any constant $\varepsilon > 0$, there is no efficient $O(n^{1-\varepsilon})$ approximation algorithm for {\GHA} on bounded-degree graphs unless P = NP.
\end{restatable}

% Figure environment removed

\begin{proof}[Proof Sketch]
    Figure \ref{fig:bounded-degree-reduction-graph} shows the graph we use for this reduction. The graph $G$ is constructed as follows. For each $a_i$ in the given {\UTP} instance, construct in polynomial time a $3$-regular Ramanujan bipartite multi-graph of size $Ca_i$ (using a result of \cite{cohen2016ramanujan}). Remove any repeated edges to convert them into simple graphs. The resulting graphs can be shown to have sufficient expansion properties, using Cheeger's inequality (Lemma \ref{lem:cheegers-inequality-apdx}). We can now attach these graphs to the leaves of a sufficiently small binary tree. This is a bounded-degree graph.

    If there is a $3$-partition, we can exhibit a small-envy allocation in the same way as in the other proofs in this section.
    % Conversely, if there is no valid $3$-partition, we can once again show that if some gadget has no majority value, then it spans many different values, and by its expansion properties, it has many edges in a cut across these different values, incurring $\Omega(C)$ envy. Otherwise, if each gadget has a majority value, we can use the packing argument once more to show that some gadget has to have a cut going across different values, incurring $\Omega(C)$ envy once again.
    Conversely, if there is no valid $3$-partition, we can show that some of these gadgets are going to be allocated multiple different values. We then use the expansion properties of these gadgets to show that the number of high envy edges within each of these gadgets is $\Omega(C)$.
\end{proof}


% \begin{proof}[Proof sketch]
%     Figure \ref{fig:bounded-degree-reduction-graph} shows the graph we use for this reduction. The graph $G$ is constructed as follows. For each $a_i$ in the given {\UTP} instance, construct in polynomial time a $3$-regular Ramanujan bipartite multi-graph of size $Ca_i$ (using a construction by \cite{cohen2016ramanujan}). Remove any repeated edges to convert them into simple graphs. The resulting graphs can be shown to have sufficient amount of expansion properties, using Cheeger's inequality (Lemma \ref{lem:cheegers-inequality}). We can now attach these graphs to the leaves of a sufficiently small binary tree. This is a bounded-degree graph.

%     If there is a $3$-partition, we can exhibit a small-envy allocation in the same way as in most of the other proofs in this section.

%     % Conversely, if there is no valid $3$-partition, we can once again show that if some gadget has no majority value, then it spans many different values, and by its expansion properties, it has many edges in a cut across these different values, incurring $\Omega(C)$ envy. Otherwise, if each gadget has a majority value, we can use the packing argument once more to show that some gadget has to have a cut going across different values, incurring $\Omega(C)$ envy once again.
%     Conversely, if there is no valid $3$-partition, we can show that some of these gadgets are going to be allocated multiple different values. We then use the expansion properties of these gadgets to show that the number of high envy edges within each of these gadgets is $\Omega(C)$.
% \end{proof}

\subsection{Bounded-Degree Trees}
Our final lower bound shows that {\GHA} is NP-hard even when the underlying graph is a bounded degree tree. We still use \UTP in our reduction but this proof is significantly different from the previous ones. Our reduction will use a gadget we call the {\em flower}.\footnote{To the best of our knowledge, our specific flower graph is novel but it is possible (likely even) that the term ``flower'' has appeared before in the graph theory literature.}

\begin{definition}
The flower $F(n, k)$ is a rooted tree with $n$ nodes and maximum degree $k+1$, defined recursively as follows: for any $k \geq 1$, $F(1, k)$ is simply an isolated vertex which is the root node. For $n > 1$, $F(n, k)$ consists of a root node connected to the root nodes of $d$ other flowers $F(n_1, k), \dots, F(n_d, k)$ such that 
\begin{enumerate}[(a)]
    \item $\sum_{i = 1}^d n_i = n-1$, 
    \item if $n-1 \ge k$, then $d = k$ if $n$ and $k$ have different parities, and $d = k-1$ otherwise,
    \item each $n_i$ is odd, 
    \item for any $i, j \in [d]$, $|n_i - n_j| \le 2$.
\end{enumerate}
To ensure consistency with floral terminology, we refer to the root node of the flower $F(n, k)$ as its {\em pistil} and the (recursively smaller) flowers $F(n_1, k), \dots, F(n_d, k)$ as its {\em petals}.
\end{definition}

Before we use flowers, we show that they are well-defined and efficiently constructible. 

\begin{restatable}{lemma}{lemflowercomputation}\label{lem:flower-computation}
For any $n \ge 1$ and $k \ge 3$, the flower $F(n, k)$ exists and can be constructed in $\text{poly}(n, k)$ time. 
\end{restatable}
\begin{proof}
To prove existence, we only need to show that the numbers $n_1, \dots, n_d$ are guaranteed to exist. If $n -1 < k$, this is trivial: each $n_i = 1$.

Otherwise, if $n$ is even, then $d$ is required to be an odd value via condition (b). What we need to do is write $n-1$ as the sum of $d$ odd numbers. We set each $n_i$ as the greatest odd number which is at most $\lfloor \frac{n-1}{d} \rfloor$. Once we do this, $n-1 - \sum_{i = 1}^d n_i$ is guaranteed to be a non-negative even number which is strictly less than $2d$. So, we increment some of the $n_i$'s by $2$ till property (a) is satisfied. This construction satisfies the other two properties as well. 
% We can use a similar construction for the case where $n$ is odd. 

The above argument not only shows that the numbers $n_1, \dots, n_d$ are guaranteed to exist but also presents a way to compute them in polynomial time. Since each $n_i$ is strictly less than $n$, we can use this subroutine to compute $F(n, k)$ recursively in polynomial time. More formally, let $T(n, k)$ be the complexity of constructing the flower $F(n, k)$. We have

\begin{align*}
    T(n, k) &= \left (\sum_{i = 1}^d T(n_i, k) \right ) + O(d) \\
    &\le k\cdot T((n-1)/(k-1) + 2, k) + O(k) \\
    &\le k\cdot T((n/k + 2, k) + O(k).
\end{align*}
If $n/k < 2$, then $T(n/k + 2, k)$ can be trivially constructed in $O(1)$ time since $n/k + 2 -1 \le 3 \le k$ and the graph is a constant-sized star. If $n/k \ge 2$, we can simplify the above expression as follows:
\begin{align*}
    T(n, k) \le k\cdot T(2n/k, k) + O(k),
\end{align*}
which simplifies to $T(n, k) = O(n^\frac{\ln 3}{\ln{3} - \ln2})$.
%To solve this we go back our undergraduate algorithms class to recall the Master theorem and simplify the above expression even further to $T(n, k) = O(n^\frac{\ln k}{\ln{k} - \ln2})$. This value decreases as $k$ increases. Since $k \ge 3$, we can upper bound it as $T(n, k) = O(n^\frac{\ln 3}{\ln{3} - \ln2})$
\end{proof}


The reason we build flowers is because they satisfy the two following useful properties.
\begin{restatable}{lemma}{lemflowerproperties}\label{lem:flower-properties}
Let $F(n, k)$ be a flower on the set of vertices $N$, and suppose $n \ge 10k$, and $n$ and $k$ have different parities. Then, $F(n, k)$ satisfies the following properties:
\begin{enumerate}[(i)]
    \item For any $A \subseteq N$ such that $|A|$ is even and $A$ does not contain the pistil, $\delta(A) \ge 2$.
    \item Each petal of $F(n, k)$ has size in the interval $\left [  \frac{4n}{5k}, \frac{6n}{5k} \right]$.
\end{enumerate}
\end{restatable}
\begin{proof}
(i) follows from property (c) in the definition of a flower. 
(ii) follows from the fact that there are $k$ petals (property (b)) and each petal has size in the interval $\left [ \frac{n}{k}-2, \frac{n}{k} + 2\right ]$ (property (d)).
\end{proof}




These simple properties are all we need to show the hardness of {\GHA} on bounded-degree trees.
\begin{restatable}{theorem}{thmboundeddegreetreesnp}\label{thm:bounded-degree-trees-np-complete}
{\GHA} is NP-hard on bounded-degree trees.   
\end{restatable}

% Flowers
% Figure environment removed

% \begin{proof}[Proof Sketch]
% Given a \UTP instance, we construct a graph according to Figure \ref{fig:boundedtrees-reduction-graph}; the shaded circles correspond to pistils and the white triangular blocks correspond to petals. The house values are defined as follows: we have $4m+1$ unique values such that the gaps between these values are exponentially decreasing. That is, the gap between the least and the second least value is significantly larger than the gap between the second least and the third least value and so on. For each unique value, there are $T$ houses with that value in the multiset $H$, with the exception of the largest value which has enough houses (with that value) to ensure the total size of the multiset $H$ is equal to the number of nodes in the graph.

% We can show that in any optimal allocation, the first $3m$ clusters must be allocated to flowers of the form $F(T, 99)$. The next $m$ clusters must be allocated in a way that creates a $3$-partition to minimize envy. That is, each of these values must be allocated to three flowers of the form $F(a_i, 99)$ such that the total size of these three flowers sums up to $T$. If it is not possible to do this, the minimum envy of the allocation is strictly higher. This allows us to separate instances with a valid \TP.
% \end{proof}

\begin{proof}
We present a reduction to {\UTP}. We assume the input $3$-partition instance is scaled up such that each $a_i$ is even and at least $1000$. This comes with no loss of generality --- we can multiply each $a_i$ and $T$ by a $1000$ without changing the output of the instance. We also assume $T$ is even. If $T$ is odd, the instance is trivially a NO instance.

We construct the graph of the {\GHA} instance as follows (see Figure \ref{fig:boundedtrees-reduction-graph}). We start with a path of $3m$ flowers $F(T, 99)$, connected by their pistils. We number these flowers $1$ to $3m$ from left to right. To each flower $i$, we connect a flower $F(a_i, 99)$ and to this flower, we connect two flowers $F(10T, 999)$. Again, connections between flowers are made via an edge between the pistils of the flowers. We refer to the flowers of the form $F(a_i, 99)$ as {\em small} flowers, flowers of the form $F(T, 99)$ as {\em medium} flowers and flowers of the form $F(10T, 999)$ as {\em large} flowers. We define small, medium and large pistils similarly. Note that large flowers have $999$ petals, while small and medium flowers have $99$ petals (since each $a_i$ and $T$ can be assumed to be even). This graph can be constructed in polynomial time using Lemma \ref{lem:flower-computation} and the fact that the inputs are given in unary.



Now we describe the house values. We need to describe $64mT$ house values since this is the number of nodes in the graph. For this, it is easier to first define the following function/series, $s(j) = (|E|+1)^{2j}$ for any $j \in \mathbb{N}$ where $|E|$ is the number of edges in our constructed graph. Note that equivalently, we can write $s(j) = (64mT)^{2j}$. We define the multiset of house values as the following:
\begin{align*}
    T &\text{ houses with value } 0 \\
    T &\text{ houses with value } s(4m) \\
    T &\text{ houses with value } s(4m) + s(4m-1) \\
    \dots \\
    T &\text{ houses with value } \sum_{j = 2}^{4m} s(j) \\
    60mT &\text{ houses with value } \sum_{j = 1}^{4m} s(j)
\end{align*}
The crucial property these values satisfy is that the intervals between two values are exponentially decreasing in size, so much so that minimum envy allocation must lexicographically minimize the number of edges passing through these gaps. That is, to minimize envy, we need to first minimize the number of lines passing through the $(0, s(4m))$ interval, subject to that minimize the number of lines passing through the $(s(4m), s(4m) + s(4m-1))$ interval, and so on. For ease of readability, we no longer refer to the exact value of the houses but simply refer to them as {\em clusters} of values. More specifically, the $T$ instances of $0$ are the \emph{first} cluster of values, the $T$ instances of $s(4m)$ are the \emph{second} cluster, and so on. There are $4m+1$ clusters, each of which have size $T$ except for the highest value (the $(4m+1)$-th cluster) which has size $60mT$. The largest house value is at most $(4m + 1)\cdot s(4m)$, which requires $O(\log(4m + 1) + (4m)\log(|E| + 1))$ bits to write, and there are still only polynomially many such values, and so this multiset of house values can be written in polynomial time and space.

Given this constructed instance, let us study the optimal allocation $\pi^{*}$. Note that the optimal allocation must first minimize the number of envy lines between the first two clusters. Therefore, the placement of the first cluster must be done in such a way so as to minimize the cut size between the first cluster and the rest of the graph. We will show that, to minimize envy, the first $3m$ clusters must be allocated to the $3m$ medium flowers, and the next $m$ clusters (from the $(3m+1)$-th to the $(4m)$-th cluster) must each be allocated to exactly three small flowers with sizes $a_i, a_j$ and $a_k$ such that $a_i + a_j + a_k = T$. This is possible if and only if the original instance is a YES instance, and the total envy of the corresponding house allocation is given by
\begin{equation*}
    \envy_{\text{YES}} = \sum_{j = 1}^{3m - 1}(j + 1)\cdot s(4m+1-j) + (3m)\cdot s(m+1)  + \sum_{j = 1}^m(3m + 3j)\cdot s(m+1-j) .
\end{equation*}
Any other allocation that does not follow this structure has a strictly greater envy. Specifically, this means that when the original instance is a NO instance, the optimal allocation has a greater total envy than $\envy_{\text{YES}}$ defined above.

To do this, we will need some bounds on petal sizes. The size of each petal in the medium flowers is at most $\frac{6T}{99 \times 5} \le \frac{T}{80}$ (Lemma \ref{lem:flower-properties}). Similarly the size of each petal in the large flowers is $\le \frac{T}{80}$ and the size of each petal in a small flower is $\frac{6a_i}{99 \times 5} \le \frac{6T}{2 \times 99 \times 5} \le \frac{T}{160}$. In particular, all petals throughout the graph have size at most $T/80$, a fact that we shall use several times.

Let us now consider where the first cluster is allocated in any optimal allocation. Note that in an allocation where the first cluster completely fills up one of the medium flowers at either end of the path (either leftmost or rightmost), the cut size across the first interval is $2$. Any optimal allocation therefore needs to ensure that this cut size is at most $2$. This observation will enable us to rule out many other possibilities.

If the elements of the first cluster are not allocated to any pistil, then the cut size is at least $80$ since each petal in the graph has size at most $\frac{T}{80}$; so the first cluster must be allocated to at least $80$ petals, each of which must add at least one edge to the cut. So in an optimal allocation, at least one pistil must get an element from the first cluster.

Suppose a large pistil is given a value in the first cluster. Then, since each petal of the large flower has size at least $T/125$, there must be at least $999-125 \ge 800$ petals of the large flower that remain unfilled by the cluster of values. Each of these unfilled petals adds at least one edge between the first cluster to another cluster, and these edges are all disjoint, and so the cut size must be at least $800$. Any such allocation, therefore, is suboptimal.

Suppose a small pistil is given a value in the first cluster. This pistil has two neighboring large pistils. We know from the arguments before that these neighbors cannot be in the first cluster, and so these two outgoing edges are across the cut. But then, the entire graph without these two large flowers is still connected, so at least one more edge needs to go across the first cut, and therefore, this allocation is also suboptimal.

It follows that any optimal allocation must allocate only medium pistils to the first cluster. If multiple such pistils are allocated, say from medium flowers $F_1$ and $F_2$, note that the cluster cannot contain either flower in its entirety, so there is at least one edge from each of them across the cut; but there also must be at least one other edge, from a path that goes from any of these two pistils to any node in the graph placed in a different cluster (such a node must exist, just by counting). This is therefore suboptimal. Similarly, if the first cluster contains exactly one pistil from a medium flower $F_1$, and no other pistil, but it does not contain all of $F_1$, it is easy to see that the cut will have size at least $3$, and so will be suboptimal.

%If on the other hand, the first cluster contains completely fills up one of the $T$ sized flowers, the cut size is $2$, if the $T$ sized flower picked is one of the extreme ones (leftmost or rightmost). This is optimal. Any other allocation which incompletely fills a $T$-sized flower, or fills any other $T$-sized flower is strictly worse. An allocation that exclusively fills up small flowers \rik{I thought the casework at this point was for pistils of flowers, no? The remaining case is to argue that a small pistil cannot be in the first cluster.} are also worse since each $a_i$ is in the interval $\left [ \frac{T}{4}, \frac{T}{2} \right ]$ and so multiple such $a_i$'s must be filled up and each $a_i$ has an edge to two pistils of flowers of size $10T$ --- who we have already shown is a bad idea to allocate to.

So, assume from here on out that the first cluster is allocated entirely to the leftmost (or the first) medium flower under $\pi^*$. Our goal is to show that the first $3m$ clusters must be allocated to the $3m$ medium flowers from left to right. We do this by induction.

Assume the first $k$ clusters are allocated to the first $k$ medium flowers (from the left). Note that the cut size or the total number of lines of envy between the first $k$ clusters and the rest of the graph is $k+1$. Out of all the ways of allocating the rest of the items, we wish to find the one that minimizes the number of lines of envy going through the next interval, between the $(k + 1)$-th cluster and the $(k + 2)$-th one. We will show that in order to achieve this, the $(k+1)$-th cluster must be allocated to the $(k+1)$-th medium flower. If we allocate the $(k+1)$-th cluster to the $(k+1)$-th medium flower, the cut size between the first $k+1$ clusters and the rest of the graph increases by $1$ with respect to the first $k$ clusters i.e. it increases from $k+1$ to $k+2$. The increase in cut size is just an easier way to account for the number of edges between the first $k+1$ clusters and the rest of the graph. We show that all other allocations of the $(k+1)$-th cluster are strictly suboptimal i.e. all other allocations increase the cut size by at least $2$.

If the $(k+1)$-th cluster is not allocated to any pistil, then the cut size increases by at least $80$. If the $(k+1)$-th cluster is allocated to a large pistil, then the cut size increases by at least $800$. Both of these statements can be proved using similar arguments to the first cluster. 

For every small pistil that gets a value from the $(k+1)$-th cluster (regardless of whether it is attached to a medium flower that has already been assigned a cluster or not), the cut size increases by at least $1$ --- this is because each of these pistils has $2$ edges to large pistils. So even if the edge connecting the small pistil to the medium pistil is removed from the cut, at least two new edges are added. This argument rules out allocating the $(k+1)$-th cluster to multiple small flowers. If, on the other hand, no medium pistil is allocated and only one small pistil is allocated, at least $T/2$ values from the cluster must be allocated to petals of flowers whose pistil remains unallocated; this comes from the fact that each small flower has size upper bounded by $T/2$, and so the remaining $T/2$ values must be allocated to petals. Since every petal of any kind of flower has size at most $T/80$, we need at least $40$ different petals to be represented among the remaining values, and each of them has to add a distinct edge across the cut, and so the cut size must increase by at least $40$.

The only cases we have left are ones which involve at least one medium pistil. If multiple of these medium pistils are allocated, then since each of their petals has size at least $T/125$, there can be at most $125$ such petals represented in the cluster, and so we must have at least $198 - 125 \ge 50$ unfinished petals, each of which adds a distinct edge across the cut, increasing the cut size by at least $1$ each.

So exactly one medium pistil must be allocated to this cluster. We could still have a combination of one medium pistil and at least one small pistil, but this increases the cut size by at least $2$ since each of the small pistils increase the cut size by at least $1$ and the medium flower must be unfinished which adds at least another edge to the cut. 

Finally, we have the case where exactly one medium pistil and no other pistil is allocated. It is easy to see, using arguments similar to before, that the strictly optimal place to put this medium pistil is the pistil corresponding to the $(k+1)$-th medium flower and the strictly optimal allocation is to fill up the cluster entirely with the $(k+1)$-th medium flower.

We can conclude that the first $3m$ clusters must be allocated entirely to the medium flowers in order from left to right (or right to left). 
Note that we have not said anything yet to diffentiate instances with a valid $3$-partition. We will do that next.

The $(3m +1)$-th cluster still must be allocated to some pistil, but cannot be allocated to a large pistil, by the arguments from before. Note that all the medium flowers have already been allocated values from the first $3m$ clusters. Therefore, the $(3m + 1)$-th cluster must contain at least one pistil, and all pistils in it must be small. If the $(3m+1)$-th cluster fills up exactly $3$ different small flowers, then the cut size increases by exactly $3$ relative to the first $3m$ clusters. This is because $6$ edges from the small pistils to the large pistils are added, but the $3$ edges from the medium pistils to the small pistils are taken away. We will show that this is the best option, and any other allocation is strictly worse. To show this, we must consider the following four possible cases.

\noindent\textbf{Case 1: The $(3m+1)$-th cluster is allocated to exactly $1$ pistil.} This must be the pistil of a small flower, which we know has at most $T/2$ vertices by assumption. Therefore, at least $T/2$ values in the cluster must be allocated to petals of other flowers. Each of these petals adds at least one edge to the cut since their corresponding pistils have not been allocated values in the first $3m+1$ clusters, and these edges are all distinct. We know each petal in the graph has a size of at most $T/80$, so at least $40$ petals must be represented in this cluster, so the cut size increases by at least $40$, which is strictly suboptimal.

\noindent\textbf{Case 2: The $(3m+1)$-th cluster is allocated to exactly $2$ pistils.} Both these pistils must correspond to small flowers (say $F_1$ and $F_2$). Both these flowers have a combined size of strictly less than $T$, so at least one other flower outside these two is allocated a value from the $(3m+1)$-th cluster. If either $F_1$ or $F_2$ are not completely filled, then the cut size increases by at least $4$ --- two extra edges come from the pistils of $F_1$ and $F_2$ to the large pistils, the third is from at least one of $F_1$ or $F_2$ being unfinished, and the fourth is from the third flower which is allocated a value (note that it cannot be completely contained in this cluster, because its pistil is in a different cluster).

If both $F_1$ and $F_2$ are completely filled, then the argument is a little more subtle. Recall that each small pistil adds a $1$ to the cut size from the edges to the large pistils. We need to show that at least $2$ other edges are added to the cut from the values allocated outside $F_1$ and $F_2$. The number of values from the $(3m+1)$-th cluster allocated outside of $F_1$ and $F_2$ is {\em even}, since both $F_1$ and $F_2$ have even size by assumption. (Also, this number of values is nonzero, as $F_1$ and $F_2$ have a combined size of less than $T$ by assumption.) Therefore, if these values are allocated to two different flowers, at least $2$ more edges are added to the cut (since those two flowers have their pistils in other clusters) and we are done. Otherwise, if they are allocated to the same flower, the cut size still increases by at least $2$ because of Lemma \ref{lem:flower-properties} and we are done.

\noindent\textbf{Case 3: The $(3m+1)$-th cluster is allocated to exactly $3$ pistils but at least one of the flowers is incomplete.}
Here the cut size increase is trivially at least $4$ --- an increase of $3$ from the pistils having edges to the large pistils, and a fourth from the fact that at least one of the flowers is incomplete.

\noindent\textbf{Case 4: The $(3m+1)$-th cluster is allocated to $4$ or more pistils.}
It is a straightforward argument to show that the number of edges added across the cut is $4$ or more in this case.


Similarly, for the $(3m+2)$-th to the $(4m)$-th cluster, the best possible option is to allocate each cluster in a way that fills up $3$ small flowers. If any of these clusters fail to do so, the cut size is strictly higher. Given allocations of the first $4m$ clusters, there is only one possible allocation for the $(4m+1)$-th cluster; that is, the allocation where values from the $(4m+1)$-th cluster to all nodes which have not been allocated a value from the first $4m$ clusters.

It is easy to see that it is possible for all the $(3m+1)$-th to $(4m)$-th clusters to fill up three small flowers each if and only if there is a valid $3$-partition in the original instance. More specifically, if each of the $(3m+1)$-th to $(4m)$-th clusters are allocated in a way that attains a lower bound on the number of edges between the $(3m+k)$-th and the $(3m+k+1)$-th interval, this allocation defines a valid $3$-partition. This is because each of the $(3m+1)$-th to $(4m)$-th cluster must be allocated to exactly three small flowers and must fill them up entirely. It is also easy to see that given a valid $3$-partition, we can construct an allocation that attains the ideal total envy.

To compute the threshold, we proceed as follows. For $1 \leq i \leq 4m$, let $\ell_i$ be the length of the valuation interval between the $i$-th cluster and the $(i + 1)$-th cluster, which is precisely $s(4m + 1 - i)$. If the original {\UTP} instance is a YES instance, then by our arguments above, we can incur an envy of at most
\begin{equation*}
    \envy_{\text{YES}} := \sum_{j = 1}^{3m - 1}(j + 1)\cdot \ell_j + (3m)\cdot \ell_{3m} + \sum_{j = 1}^m(3m + 3j)\cdot\ell_{3m + j}.
\end{equation*}
Here, the interval $\ell_{3m}$ has increased the cut size by $0$, because it represents the rightmost medium flower being placed in the $3m$-th cluster. Note that the expression above can be computed in polynomial time, since each $\ell_j = s(4m + 1 - i)$ is representable in $O((4m + 1 - i)\log(|E|))$ bits, and all arithmetic operations can be done in polynomial space and time in this number of bits. There are only polynomially many operations to do, so we can compute the threshold $\envy_{\text{YES}}$ in polynomial time.

From all our analysis above, if the original {\UTP} instance is a YES instance, we can attain an envy of at most $\envy_{\text{YES}}$ on our constructed instance. On the other hand, if the original {\UTP} instance is a NO instance, we know any allocation is lexicographically worse than this expression. By our choice of the function $s(\cdot)$, this envy must be strictly more than $\envy_{\text{YES}}$. It follows that our {\GHA} instance has an allocation with envy at most $\envy_{\text{YES}}$ if and only if the {\UTP} instance is a YES instance. Since the construction is done entirely in polynomial time, this finishes the proof.
%any other allocation that does not have this structure has strictly worse total envy. Therefore, any instance with no valid $3$ partition has strictly worse envy. Since we know exactly how much the cut size increases for each of the first $4m$ clusters, we know exactly how many lines pass through each interval in the minimum envy allocation when there is a $3$ partition and when there is no $3$ partition. We can easily compute a threshold using these values and separate the instances with a valid $3$-partition. 
\end{proof}






