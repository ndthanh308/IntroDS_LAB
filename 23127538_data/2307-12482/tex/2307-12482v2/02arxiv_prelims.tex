\section{Model and Preliminaries}\label{sec:prelims}

%\subsection{Preliminaries from House Allocation}

We have a set of $n$ {\em agents} $V = [n]$ placed on the vertices of an undirected graph $G = (V, E)$. 
There are $n$ {\em houses}, each with a nonnegative \emph{value}, that need to be allocated to the agents. We represent the houses simply by the multiset of values $H = \{h_1, \ldots, h_n\}$, and assume WLOG that $h_1 \leq \ldots \leq h_n$. We will interchangeably talk about the house with value $h_i$ and the real number $h_i$. The pair $(G, H)$ defines an \emph{instance} of {\GHA}. %\hadi{we are sometimes referring to \textit{the} \GHA{} \textit{problem} and sometimes \GHA{} alone. we should be  consistent throughout the paper.}

An {\em allocation} $\pi: V \rightarrow H$ is a bijective mapping from agents (or nodes) to house values. Given an allocation $\pi$ and an edge $(i, j) \in E$, we define the {\em envy} along the edge $(i, j)$ as $|\pi(i) - \pi(j)|$. Our goal in {\GHA} is to compute an allocation $\pi^\ast$ that {\em minimizes} the total envy %\hadi{sometimes we say aggregate, let's be consistent} 
along all the edges of $G$:
\begin{align*}
    \Envy(\pi, G) := \sum_{(i, j) \in E} |\pi(i) - \pi(j)|.
\end{align*}

% The following definition, given by \cite{canon}, gives us a geometric way to visualize any allocation on an instance $(G, H)$ of {\GHA}.
We adopt the following definition from  \citet{canon} that provides a geometric representation to visualize allocations.

\begin{definition}[Valuation Interval]\label{def:valn_interval}
For an instance $(G, H)$ of {\GHA}, define the \emph{valuation interval} as the closed interval $\left[h_1, h_n\right] \subset \R_{\geq 0}$. For any allocation $\pi$, the envy along the edge $(i, j) \in E$ is exactly the length of the interval $[\pi(i), \pi(j)]$ (assuming $\pi(i) \leq \pi(j)$). We sometimes call the intervals $[h_i, h_{i+1}]$ for $1 \leq i \leq n - 1$ the \emph{smallest subintervals} of the valuation interval.
\end{definition}


 An optimal allocation $\pi^\ast$ would minimize the sum of the lengths of the intervals corresponding to each of its edges.
%
An allocation $\pi$ is \emph{$\alpha$-approximate} if $\Envy(\pi, G) \le \alpha\cdot\Envy(\pi^\ast, G)$. 




%In most cases of the {\GHA} problem, computing an optimal $\pi^\ast$ is intractable. Therefore, we focus on approximations. 
% An allocation $\pi$ is \emph{$\alpha$-approximate} if $\Envy(\pi, G) \le \alpha\cdot\Envy(\pi^\ast, G)$. 

Fix any arbitrary class $\cal G$ of graphs (we allow $\cal G$ to be a singleton set). We say an algorithm $\mathsf{ALG}_\mathcal{G}$ is \emph{defined} on $\cal G$ if $\mathsf{ALG}_\mathcal{G}$ is well-specified and outputs a valid allocation on every instance $(G, H)$ of {\GHA} with $G \in \mathcal{G}$. Such an algorithm $\mathsf{ALG}_\mathcal{G}$ is an \emph{$\alpha$-approximation} if for all instances $(G, H)$ of {\GHA} with $G \in \mathcal{G}$, $\mathsf{ALG}_{\mathcal{G}}$ always outputs an $\alpha$-approximate allocation. A $1$-approximation is an exact algorithm.

%We are now ready to formulate the following definition.

\begin{definition}[Value-Agnostic Algorithms]
\label{defn:valueagnostic}
An algorithm $\mathsf{ALG}_\mathcal{G}$ defined on a graph class $\mathcal{G}$ is \emph{value-agnostic} if on every input $(G, H)$ with $G \in \mathcal{G}$, $\mathsf{ALG}_\mathcal{G}$ returns the same allocation on all instances where the \emph{ordering} of house values is the same (in other words, the algorithm only requires the ordinal ranking and not the numerical values).
If the graph class $\cal G$ admits a value-agnostic $\alpha$-approximation algorithm, we say $\cal G$ is \emph{$\alpha$-value-agnostic}. Otherwise, it is \emph{$\alpha$-value-sensitive}.
\end{definition}

How can we re-frame existing results on {\GHA} in the light of Definition \ref{defn:valueagnostic}? \citet{canon} show that, unless P = NP, there is no $1$-approximation algorithm $\mathsf{ALG}_\mathcal{G}$ when $\cal G$ is the set of vertex-disjoint unions of paths, cycles, or stars. In contrast, they show that value-agnostic \emph{exact} algorithms exist when $\cal G$ is the set of paths, cycles, or stars.

Of course, value-agnostic $\alpha$-approximations are extremely powerful algorithms, as they can exploit the graph structure \emph{independently} of the values in the {\GHA} instance. As we would expect, value-agnostic $1$-approximations do not always exist, even on very simple graph classes and even if we allow for unlimited time. For instance, consider the graph consisting of the disjoint union of $K_2$ and $K_3$. Figure \ref{fig:value_agnostic_ex} shows that this graph does not admit an $\alpha$-value-agnostic algorithm for any finite $\alpha$.

% Figure environment removed

Although all our examples so far use the disconnectedness of the graphs to illustrate value-sensitivity, we will see in Section \ref{sec:completebintrees} that there are value-sensitive connected graphs as well.




%\subsection{Preliminaries from Structural Graph Theory}

For any graph $G = (V, E)$, and $S \subseteq V$, we denote by $\delta_G(S)$ the number of edges going across the cut $(S, V - S)$ in $G$. We will often estimate $\delta_G(S)$ for various subsets $S$. For $k \leq n-1$, we define $\delta_G(k) := \min_{|S| = k}\delta_G(S)$ as the size of the smallest cut in $G$ with $k$ vertices on one side. Of course, $\delta_G(k) = \delta_G(n - k)$ for all $k$. A $(k, n-k)$-cut in $G$ will be any cut $(S, V-S)$ with $|S| = k$.

We will use a few concepts from structural graph theory, most notably that of \emph{cutwidth}.

\begin{definition}\label{def:cutwidth}
    For a graph $G = (V, E)$ on $n$ vertices, let $\sigma = (v_1, \ldots, v_n)$ be any ordering of $V$. The \emph{width} of $\sigma$ is defined as
    \begin{equation*}
\mathsf{width}(\sigma, G):=        \max_{1 \leq \ell \leq n-1} \delta_G(\{v_1, \ldots, v_\ell\}).
    \end{equation*}
    The \emph{cutwidth} of $G$ is  the minimum width over all orderings of $G$, i.e.,
    \begin{equation*}
        \cw(G) := \min_{\sigma \in S_n}\mathsf{width}(\sigma, G)~.
    \end{equation*}
\end{definition}

The ordering $\sigma$ is often called a \emph{layout}. An \emph{optimal} layout is an ordering that achieves the cutwidth of $G$. The cutwidth is closely related to other standard notions of width used in structural graph theory. In particular, we have the following chain of inequalities (see \citet{korach1993cutwidth}):
\begin{align}
    &\tw(G)  \leq \pw(G) \leq \cw(G) \notag\\ &\qquad \leq O(\Delta\cdot\pw(G)) \leq O(\Delta\cdot\tw(G)\cdot\log n) 
    \label{eq:twpwcw}
\end{align}

Finding the exact cutwidth of $G$ in general is a difficult algorithmic problem. It can be computed exactly for trees (along with an optimal ordering) in time $O(n\log n)$ \citep{yannakakis1985treecutwidth}. However, even for planar graphs, the problem is NP-complete \citep{monien1988cutwidth}. 

If $G$ is sufficiently dense, there is a polynomial-time approximation scheme for the cutwidth \citep{frieze1996cutwidthdense}. In general, there is an efficient $O(\log^{1.5} n)$-approximation of the cutwidth known \citep{leighton1999cutwidthapprox}, which also returns a layout achieving this ratio. We will use this process as a subroutine several times in Section \ref{sec:upper}, for our upper bounds.