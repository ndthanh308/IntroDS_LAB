\appendix

% \section{Proofs from Section \ref{sec:upper}}\label{apdx:upper}

% \thmtreelogn*
% \begin{proof}
% We will show that Algorithm \ref{alg:treelogn} provides the desired guarantee. The algorithm starts by locating the center of gravity $v$ of the given tree $T$ (which is guaranteed to exist by Fact \ref{fact:folklore}). Then it assigns the largest (i.e., rightmost) value to node $v$, and recursively constructs the assignment for each of the disjoint subtrees in $T - v$.
%       %Note that, by Fact \ref{fact:folklore}, line 4 in Algorithm \ref{alg:treelogn} can always be done, and therefore it is well-specified.
      
%       It is easiest to visualize the allocation resulting from Algorithm \ref{alg:treelogn} as in Figure \ref{fig:treelogn}. All recursive calls in line 9 occur in a single ``level'' of the figure, and all subsequent recursive calls from the subtrees $T_1, \ldots, T_k$ can also be packed into a single level, as the edges in $T_i$ and the edges in $T_j$ do not overlap, for any $i \neq j$. The crucial point is that the envy incurred strictly within disjoint subtrees $T_i$ and $T_j$ cannot involve the same smallest subintervals of the original instance.

%       Let us analyze the total envy in the final allocation that is output by Algorithm \ref{alg:treelogn}. There are at most $\Delta$ edges adjacent to $v$, and each of them incur their envy in level $1$ of Figure \ref{fig:treelogn}. Each edge gets an envy of at most $(h_n - h_1)$, and therefore, the total envy on these edges is at most $\Delta\cdot(h_n - h_1)$. The total envy along the edges adjacent to $v_1, \ldots, v_k$ (except the ones accounted for in the levels above) are at most $\Delta\cdot(h_{n_1} - h_1), \ldots, \Delta\cdot(h_{n_1+\ldots+n_k} - h_{n_1+\ldots+n_{k-1}+1})$. Since the subintervals are all disjoint, this level accounts for an envy of at most $\Delta\cdot(h_n - h_1)$ as well. We can continue this argument through the lower levels.

%       How many levels are there? Because each vertex picked at each recursive call is a center of gravity of the next subtree, the size of each subtree is at most half the size of the subtree at its parent level. The number of levels, therefore, is at most $\log n$. This gives us a total envy of $\Delta\cdot(h_n - h_1)\cdot\log n$.

%       Note that the optimum envy has to be at least $h_n - h_1$ for any connected graph. This gives us an approximation ratio of $\Delta\log n$.

%       The bound on the running time also arises from Fact \ref{fact:folklore}, which ensures that line 4 can be done in time $O(n)$. For each subtree $T_i$ in level $1$, we can find a center of gravity in time $O(n_i)$, so the total amount of work done to find the centers of gravity at this level is $O(n_1) + \ldots + O(n_k) = O(n)$. Since this is summed over $\log n$ recursive levels, the total running time is $O(n\log n)$.
% \end{proof}


% % Figure environment removed

% \thmcwgeneric*
% \begin{proof}
% We construct the allocation $\pi$ as follows: for each agent $i \in V$, if $\sigma(i) = j$, we set $\pi(i) = h_j$; that is, we give $i$ the $j$-th least-valued house. 
% Since $G$ is connected, the total envy of any allocation is at least $(h_n - h_1)$. In the allocation $\pi$, the number of edges of $G$ spanning any smallest subinterval of the valuation interval is at most $\mathsf{width}(\sigma, G)$ by definition, and therefore the envy from $\pi$ is at most $\sum_{i = 1}^{n-1}\mathsf{width}(\sigma, G)\cdot(h_{i+1} - h_i) = \mathsf{width}(\sigma, G)\cdot(h_n - h_1)$. Hence, if $\sigma$ is an $\beta$-approximation for the cutwidth, then $\pi$ is an $\beta \cdot \cw(G)$-approximation to the optimal envy of $G$, as claimed.
% \end{proof}


% \randomlemma*
% \begin{proof}
% The expected size of the cut $\delta_G(S)$ is $E[\delta_G(S)]=|S|(n-|S|)/2$. By applying the Chernoff bound, we obtain:
% \[\Pr[|\delta_G(S)-E[\delta_G(S)]|\geq \epsilon E[\delta_G(S)]] \leq  2 \exp( -\epsilon^2 E[\delta_G(S)]/3) \ . \]
% Hence, by the union bound the probability there exists a set $S$ of size $k \le n/2$ such that $|\delta_G(S)-E[\delta_G(S)]|\geq \epsilon E[\delta_G(S)]$ is at most
% \begin{align*}    
% 2\exp( - \epsilon^2 k(n-k)/6) \binom{n}{k}
% & \leq 2 \exp( - \epsilon^2 kn/12 + k \ln n) \\
% %& \leq & 2 \exp( - \epsilon^2 kn/12 + k \ln n) \\ 
% & \leq 2 \exp( - \epsilon^2 kn/24),
% \end{align*}
%   assuming $\epsilon \geq \sqrt{24 \ln (n)/n}$. 
%  % \vignesh{Should this be $\epsilon = O(\sqrt{\ln (n)/n})$?}
% % Hence, the probability there exists two subsets of nodes of the same size where the number of edges leaving the subsets differs by more than a factor $(1+\epsilon)/(1-\epsilon) = 1+O(\epsilon)$
% The lemma then follows by taking the union bound over $k$ and noting that \[\sum_{k=1}^{n/2} 2 \exp( - \epsilon^2 kn/24)=\exp(-\Omega(\epsilon^2 n)) \ .\qedhere \]
% \end{proof}










\section{Proofs from Section \ref{sec:lower}}\label{apdx:lower}

% \thmtreesapproxlowerbound*
% \begin{proof}
% We give a reduction from \UTP. For some constant $\varepsilon > 0$, assume there is an efficient $O(n^{1- \varepsilon})$ approximation algorithm $\mathsf{ALG}_{\cal G}$ where $\cal G$ corresponds to the class of depth-$2$ trees. In other words, there is a constant $\gamma$ such that for all instances $(G, H)$ with $G \in \mathcal{G}$, $\mathsf{ALG}_{\cal G}$ outputs an allocation with total envy within a multiplicative factor of $\gamma n^{1- \varepsilon}$ to the optimal envy.

% Given an instance of \UTP, we construct an instance $(G, H)$ of {\GHA} as follows: The graph $G$ is a rooted depth-$2$ tree where the root $r$ of the tree has $3m$ children $\{x_1, \dots, x_{3m}\}$. Each of these nodes $x_i$ has $C a_i - 1$ children (see Figure \ref{fig:trees-reduction-tree}). Here, $C$ is a positive integer whose exact value we shall determine later.

% The total number of nodes in $G$ is $1 + \sum_{i \in [3m]} Ca_i = 1 + CmT$, and so we must specify $1 + CmT$ house values in $H$. We define $H$ with $CT$ values of $j$ for $j \in [m]$, together with a single value of $0$. Note that this construction can be done in polynomial time as long as $C$ is polynomially large, since the input to the $3$-partition instance is given in unary. 

% We show that, for an appropriate choice of $C$, the minimum total envy output by $\mathsf{ALG}_{\cal G}$ for the instance $(G, H)$ is at least $\left \lceil (12\gamma m^3 T + 1)^{\ffrac{1}{\varepsilon}} \right \rceil$ if and only if there exists {\em no} valid partition for the original \textsc{3-Partition} instance, i.e., the original instance was a NO instance. 


% $(\Rightarrow)$ Assume there is a valid $3$-partition $(S_1, \ldots, S_m)$ for the original instance. We denote this $3$-partition using a mapping $\mu : A \to \{S_1, \ldots, S_m\}$ that maps each number in the multiset $A$ to one of the triplets $S_j$. We construct an allocation for the graph $G$ as follows: for any item $i$, if $\mu(i) = S_j$, we allocate houses with value $j$ to $x_i$ and all of its children. We finally allocate the house with value $0$ to the root. Note that this is a valid allocation: for any $j \in [m]$, we allocate exactly $\sum_{i \in [3m]: \mu(i) = S_j} Ca_i = CT$ houses of value $j$. 

% We can easily upper bound the envy of this allocation; this upper bound also serves as an upper bound for the minimum total envy for the instance $(G, H)$. There is no envy between any $x_i$ and any of its children. So the only edges with potential envy are the ones incident on the root. There are $3m$ such edges, each incurring envy at most $m$. This gives us an upper bound of $3m^2$ on the total envy. This implies, when there is a valid $3$-partition, the approximation algorithm $\mathsf{ALG}_{\cal G}$ will output an allocation with envy at most $3m^2\gamma(1 + CmT)^{1-\varepsilon}$.

% $(\Leftarrow)$ Assume there is no valid $3$-partition in the original instance. We will show that any allocation has a total envy of at least $C$. We do this by examining the houses allocated to the depth-1 nodes $\{x_1, \dots, x_{3m}\}$. If any $x_i$ is allocated a value of $0$ in some allocation $\pi$, then the allocation $\pi$ has a total envy of at least $C$ since any $x_i$ has at least $C -1$ children and $1$ parent receiving a value of at least $1$ each. 

% So now assume no $x_i$ is allocated a value of $0$ in $\pi$. For notational convenience, assume WLOG that $x_1, \dots, x_{\ell_1}$ are allocated a value $1$, $x_{\ell_1 +1}, \dots, x_{\ell_1 + \ell_2}$ are allocated a value $2$ and so on. If for all $j \in [m]$, $\sum_{h \in [\ell_j]} a_{\ell_0 + \ell_1 + \dots + \ell_{j-1} + h} = T$ (with $\ell_0 = 0$), then we violate our assumption that there is no valid $3$-partition in the original instance. Therefore there exists one $j \in [m]$ such that $\sum_{h \in [\ell_j]}a_{\ell_0 + \ell_1 + \dots + \ell_{j-1} +h} > T$. Assume again for notational convenience that $j = 1$. Since all values are integers, we can restate the inequality above as $\sum_{h \in [\ell_1]}a_{h} \ge T + 1$. This implies $\sum_{h \in [\ell_1]} Ca_{h} \ge CT + C$. 

% Coming back to the allocation $\pi$, we have that $\sum_{h \in [\ell_1]} Ca_{h} \ge CT + C$ implies that there are at least $C$ nodes out of all the children of $\{x_{1}, x_2, \dots, x_{\ell_1}\}$ which are not allocated a value of $1$. The envy that each of these $C$ nodes will have towards their parents is at least $1$. This implies that the total envy of allocation $\pi$ is at least $C$. 

% We set $C = \left \lceil (12\gamma m^3 T + 1)^{\ffrac{1}{\varepsilon}} \right \rceil$ to complete the reduction. When there is no valid $3$-partition, the total minimum envy (and therefore, the envy output by $\mathsf{ALG}_{\cal G}$) is at least $C =\left \lceil (12\gamma m^3 T + 1)^{\ffrac{1}{\varepsilon}} \right \rceil$. However, when there is a valid $3$-partition, the envy output by $\mathsf{ALG}_{\cal G}$ is strictly upper bounded by:
% \begin{align*}
%     3m^2\gamma(CmT+1)^{1-\varepsilon} \le 6m^3 \gamma TC^{1-\varepsilon} \le 6m^3 \gamma T \left ( \left \lceil (12\gamma m^3 T + 1)^{\ffrac{1}{\varepsilon}} \right \rceil \right )^{1 - \varepsilon} < \left \lceil (12\gamma m^3 T + 1)^{\ffrac{1}{\varepsilon}} \right \rceil.
% \end{align*}
% This completes the proof.
% \end{proof}



\generalgraphs*
\begin{proof}
% Figure environment removed
We present a similar reduction from the \UTP Problem. The main difference in construction from Theorem \ref{thm:trees-approx-lower-bound} is that $x_i$ and its children are replaced with a clique of size $Ca_i$. We use the high density of the clique to show that the envy is much higher $(\Omega(C^2))$ in the absence of a valid $3$-partition.  

For some constant $\varepsilon > 0$, assume there is an efficient $O(n^{2- \varepsilon})$ approximation algorithm $\mathsf{ALG}_{\cal G}$ where $\cal G$ is the class of all connected graphs. Assume there is some constant $\gamma$ such that for all instances on connected graphs, $\mathsf{ALG}_{\cal G}$ outputs an allocation with total envy within a multiplicative factor of $\gamma n^{2- \varepsilon}$ to the optimal envy.

Given an instance of \UTP, we construct an instance of {\GHA} as follows: for each value $a_i$ in the multiset $A$, we create a clique of size $Ca_i$. We then connect these $3m$ cliques to a node $r$ (as described in Figure \ref{fig:general-reduction-graph}). Once again, $C$ is a positive integer whose exact value we shall choose later.
The total number of nodes in this graph is $CmT+1$, similar to before. The house values are defined similarly as well: for each $j \in [m]$, there are $CT$ houses valued at $j$ and there is one house valued at $0$. 


We show that, with the appropriate $C$, the total envy output by $\mathsf{ALG}_{\cal G}$ for the constructed {\GHA} instance is greater than or equal to $\left \lceil (96\gamma m^4 T^2 + 1)^{\ffrac{1}{\varepsilon}} \right \rceil^2$ if and only if there exists {\em no} valid solution for the original \textsc{3-Partition} instance, i.e.,~the original instance was a NO instance. 


$(\Leftarrow)$ Assume there is a valid $3$-partition for the original instance. This case follows the exact same way as Theorem \ref{thm:trees-approx-lower-bound}. The minimum envy is upper bounded by $3m^2$ and the envy output by $\mathsf{ALG}_{\cal G}$ is upper bounded by $3m^2\gamma(CmT+1)^{2-\varepsilon}$.

$(\Rightarrow)$ Assume there is no valid $3$-partition in the original instance. We will show that any allocation must have an envy of at least $\ffrac{C^2}{4}$. 

If, for some allocation $\pi$, there exists a clique where no value $j \in [m] \cup \{0\}$ is allocated to more than half of its nodes, then this lower bound trivially holds. Since the clique has a size of at least $C$, each node in the clique envies at least $C/2$ neighbors by at least $1$. 

Assume that for all cliques, there is some value allocated to at least half the nodes in the clique; we refer to this value as a {\em majority value} of the clique. Let clique $i$ correspond to the clique $K_{Ca_i}$. For notational convenience, assume WLOG that cliques $\{1, \dots, \ell_1\}$ have majority value $1$, $\{{\ell_1 +1}, \dots, {\ell_1 + \ell_2}\}$ have majority value $2$ and so on. Using analysis similar to \Cref{thm:trees-approx-lower-bound}, there exists at least one $j \in [m]$ such that $\sum_{h \in [\ell_j]}a_{\ell_1 + \dots + \ell_{j-1} +h} > T$. Assume again for notational convenience that $j = 1$. Since all these values are integers, we can restate the equation above as $\sum_{h \in [\ell_1]}a_{h} \ge T+ 1$. This implies $\sum_{h \in [\ell_1]} Ca_{h} \ge CT + C$. 

Coming back to our allocation $\pi$, $\sum_{h \in [\ell_1]} Ca_{h} \ge CT + C$ implies that there are at least $C$ nodes in cliques $\{1,2, \dots, \ell_1\}$ which are not allocated a value of $1$. Since $1$ is a majority value in each of these cliques, the envy that each of the $C$ nodes in $S$ will have towards the nodes with value $1$ is at least $C/2$. This implies that the total envy of allocation $\pi$ is at least $C^2/2$. 

We set $C = 2\left \lceil (96\gamma m^4 T^2 + 1)^{\ffrac{1}{\varepsilon}} \right \rceil$ to complete the reduction. When there is no valid $3$-partition, the minimum total envy (and therefore, the envy output by $\mathsf{ALG}_{\cal G}$) is at least $\ffrac{C^2}{4} = \left \lceil (96\gamma m^4 T^2 + 1)^{\ffrac{1}{\varepsilon}} \right \rceil^2$. However, when there is a valid $3$-partition, the envy output by $\mathsf{ALG}_{\cal G}$ is strictly upper bounded by:
\begin{align*}
    3m^2\gamma(CmT+1)^{2-\varepsilon} \le 6m^4 \gamma T^2C^{2-\varepsilon} \le 24m^4 \gamma T^2 \left ( \left \lceil (96\gamma m^4 T^2 + 1)^{\ffrac{1}{\varepsilon}} \right \rceil \right )^{2 - \varepsilon} < \left \lceil (96\gamma m^4 T^2 + 1)^{\ffrac{1}{\varepsilon}} \right \rceil^2.
\end{align*}
This concludes the proof.
\end{proof}


% \gridlemma*
% \begin{proof}
% If $A$ consists of at least one node from each row, then since $|A| \le \ffrac{rc}{2}$, there will be at least $r/2$ rows with a node in $V \setminus A$. Therefore, there will be at least $\ffrac{r}{2}$ edges in the cut. Similarly, if $A$ consists of at least one node from each column, there will be $\ffrac{c}{2} \ge \ffrac{r}{2}$ edges in the cut. 

% Otherwise, there is some row and some column containing only nodes in $V \setminus A$.
% Note that there must either be at least $\sqrt{|A|}$ rows with a node in $A$ or at least $\sqrt{|A|}$ columns with a node in $A$. Assume WLOG there are at least $\sqrt{|A|}$ rows with a node in $A$. Each of these rows intersects the column that only has nodes from $V\setminus A$, and so each of the $\sqrt{|A|}$ rows must contain an edge between $A$ and $V\setminus A$.
% \end{proof}


\bdp*
\begin{proof}
% Figure environment removed
We present a similar reduction from the \UTP Problem. 
For some constant $\varepsilon > 0$, assume there is an efficient $O(n^{0.5 - \varepsilon})$ approximation algorithm $\mathsf{ALG}_{\cal G}$ where $\cal G$ corresponds to the class of all planar graphs with max degree at most $5$. In other words, for all instances on these graphs, $\mathsf{ALG}_{\cal G}$ outputs an allocation with total envy within a multiplicative factor of $\gamma n^{0.5- \varepsilon}$ to the optimal envy, where $\gamma$ is some fixed constant.

Given an instance of \UTP, we construct an instance of {\GHA} as follows: the graph $G$ has $3m$ grids, each grid $i \in [3m]$ has $C$ rows and $C a_i$ columns for some $C$ we will define later. Each of the grids has a single edge to a unique leaf of a binary tree $B_r$ (see Figure \ref{fig:bdp-reduction-graph}). 

The binary tree $B_r$ is constructed such that it has at most $12m$ nodes. It is easy to see that the smallest balanced binary tree with at least $3m$ leaves satisfies this constraint. This construction is trivially a planar graph with degree at most $5$. There are $C^2mT + |B_r|$ nodes in this graph. The house values are defined roughly the same way as before: for each $j \in [m]$, there are $C^2T$ houses valued at $j$ and there are $|B_r|$ houses valued at $0$. As long as $C$ and $|B_r|$ are polynomial, this is a polynomial-time reduction. 


We show that, with an appropriate choice of $C$, the minimum total envy output by $\mathsf{ALG}_{\cal G}$ for the house allocation instance is greater than or equal to $\left \lceil (288\gamma m^3 T + 1)^{\ffrac{1}{(2\varepsilon)}} \right \rceil$ if and only if there exists {\em no} valid partition for the original \textsc{3-Partition} instance i.e. the original instance was a NO instance. 


$(\Leftarrow)$ Assume there is a valid $3$-partition for the original instance. This case follows the exact same way as Theorem \ref{thm:trees-approx-lower-bound}. The minimum envy is upper bounded by $3m^2$ and the envy output by $\mathsf{ALG}_{\cal G}$ is upper bounded by $3m^2\gamma(C^2mT+|B_r|)^{0.5-\varepsilon} \le 3m^2\gamma(C^2mT+12m)^{0.5-\varepsilon}$.

$(\Rightarrow)$ Assume there is no valid $3$-partition in the original instance. We will show that any allocation must have an envy of at least $\ffrac{C}{4}$. 

Similar to Theorem \ref{thm:general-approx-lower-bound}, we refer to a value $j \in [m] \cup \{0\}$ as a majority value of a grid $\Grid(C, Ca_i)$ if this value has been allocated to at least $\ffrac{C^2a_i}{2}$ nodes in the grid. 

For some allocation $\pi$, assume there exists a grid, $\Grid(C, Ca_1)$ with no majority value. Then this grid has nodes with at least $3$ different values. Let $Q_j$ be the set of nodes with value $j$ in this grid under allocation $\pi$. Note that the number of edges from $Q_j$ to nodes with a different value is at least $\min\{\sqrt{|Q_j|}, \ffrac{C}{2}\}$ (Lemma \ref{lem:grid-graph-property}). Therefore the total number of edges between nodes with different value is lower bounded by 
% \begin{align*}
%     \frac{\sum_{j=0}^k \min\{\sqrt{|B_j|}, \ffrac{C}{2}\}}{2} &\ge
%     \frac{\min \{\sum_{j=0}^k \sqrt{|B_j|}, \ffrac{C}{2}\}}{2} \\
%     &\ge \frac{\min\{ \sqrt{\sum_{j=0}^k |B_j|}, \ffrac{C}{2}\}}{2} \\
%     &\ge \ffrac{C}{4}.
% \end{align*}
\[    \frac{\sum_{j=0}^k \min\{\sqrt{|Q_j|}, \ffrac{C}{2}\}}{2} \ge
    \frac{\min \{\sum_{j=0}^k \sqrt{|Q_j|}, \ffrac{C}{2}\}}{2} 
    \ge \frac{\min\{ \sqrt{\sum_{j=0}^k |Q_j|}, \ffrac{C}{2}\}}{2} 
    \ge \ffrac{C}{4}.
\]
Note that we divide by $2$ since each edge gets counted at most twice. Since each of these edges has an envy of $1$, we are done.

From here on, assume that each grid has a majority value. Note that a grid can only have one majority value.
Let grid $i$ correspond to $\Grid(C,Ca_i)$. For notational convenience, assume WLOG that grids $\{1, \dots, \ell_1\}$ have majority value $1$, $\{{\ell_1 +1}, \dots, {\ell_1 + \ell_2}\}$ have majority value $2$ and so on. Using analysis similar to \Cref{thm:trees-approx-lower-bound}, there exists one $j \in [m]$ such that $\sum_{h \in [\ell_j]}a_{\ell_1 + \dots + \ell_{j-1} +h} > T$. Assume again for notational convenience that $j = 1$. Since all these values are integers, we can restate it as $\sum_{h \in [\ell_1]}a_{h} \ge T + 1$. This implies $\sum_{h \in [\ell_1]} C^2a_{h} \ge C^2T + C^2$. 

Coming to our allocation $\pi$, $\sum_{h \in [\ell_1]} C^2a_{h} \ge C^2T + C^2$ implies that there are at least $C^2$ nodes in grids $\{1,2, \dots, \ell_1\}$ which are not allocated a value of $1$. Let $A_i$ correspond to the set of nodes in grid $i$ allocated a value other than $1$. Note that $\sum_{i \in [\ell_1]} |A_i| \ge C^2$. We can lower bound the number of edges from $A_i$ to nodes with value $1$ using Lemma \ref{lem:grid-graph-property} as follows:
% \begin{align*}
%     \sum_{i \in [\ell_1]} \min\{\sqrt{|A_i|}, \ffrac{C}{2}\} &\ge
%     \min\{\sum_{i \in [\ell_1]} \sqrt{|A_i|}, \ffrac{C}{2}\} \\
%     &\ge \min\{ \sqrt{\sum_{i \in [\ell_1]}|A_i|}, \ffrac{C}{2}\} \\
%     &\ge \ffrac{C}{2}.
% \end{align*}
\[
    \sum_{i \in [\ell_1]} \min\{\sqrt{|A_i|}, \ffrac{C}{2}\} \ge
    \min\{\sum_{i \in [\ell_1]} \sqrt{|A_i|}, \ffrac{C}{2}\} 
    \ge \min\{ \sqrt{\sum_{i \in [\ell_1]}|A_i|}, \ffrac{C}{2}\} 
    \ge \ffrac{C}{2} \ .
\] 
Each of these edges have envy at least $1$.

We set $C = 4\left \lceil (288\gamma m^3 T + 1)^{\ffrac{1}{(2\varepsilon)}} \right \rceil$ to complete the reduction. When there is no valid $3$-partition, the total minimum envy (and therefore, the envy output by $\mathsf{ALG}_{\cal G}$) is at least $\ffrac{C}{4} = \left \lceil (288\gamma m^3 T + 1)^{\ffrac{1}{(2\varepsilon)}} \right \rceil^2$. However, when there is a valid $3$-partition, the envy output by $\mathsf{ALG}_{\cal G}$ is strictly upper bounded by:
\begin{align*}
    3m^2\gamma(C^2mT+12m)^{0.5-\varepsilon} \le 36m^3 \gamma TC^{1-2\varepsilon} <& 144m^3 \gamma T \left ( \left \lceil (288\gamma m^3 T + 1)^{\ffrac{1}{(2\varepsilon)}} \right \rceil \right )^{1 - 2\varepsilon} \\ <& \left \lceil (288\gamma m^3 T + 1)^{\ffrac{1}{(2\varepsilon)}} \right \rceil.\qedhere
\end{align*}
\end{proof}


\boundeddeg*
\begin{proof}
% % Figure environment removed

We, again, present a reduction from the \UTP Problem. 
For some constant $\varepsilon > 0$, assume there is an efficient $O(n^{1 - \varepsilon})$ approximation algorithm $\mathsf{ALG}_{\cal G}$ where $\cal G$ is the set of all connected graphs with degree at most $4$. For all instances, $\mathsf{ALG}_{\cal G}$ outputs an allocation with total envy within a multiplicative factor of $\gamma n^{1- \varepsilon}$ to the optimal envy for some constant $\gamma$.

Given an instance of \UTP, we construct $3m$ disjoint graphs as follows: for each $a_i$ in the multiset $A$, we construct a 3-regular Ramanujan bipartite multigraph of size $Ca_i$. This can be done in polynomial time and is well-defined when $Ca_i$ is an even integer greater than or equal to $6$ \citep{cohen2016ramanujan}. We will ensure this by setting $C$ appropriately. Note that these $3m$ multigraphs still may have multiple edges between the same pair of nodes. To convert them into simple graphs, we simply remove any repeated edges. For each $a_i \in A$, we refer to this graph as the $R'$-graph $i$ (or simply $R'(3, Ca_i)$). These graphs are neither Ramanujan graphs, nor are they $3$-regular. However, as we will crucially show, they still have the expansion properties we require. This result will use the Cheeger's inequality from \citet[Section 9.2]{alon04probabilistic} applied to Ramanujan graphs \citep{lubotzky1988ramanujan}. While the result in \citet[Section 9.2]{alon04probabilistic} is for simple graphs, the exact same proof can be extended to multigraphs; so we present it without proof.

\begin{lemma}[Cheeger's Inequality]\label{lem:cheegers-inequality-apdx}
Let $G'$ be a $d$-regular Ramanujan graph (or multigraph) defined on a set of $V$ nodes. Then,
\begin{align*}
    \min_{S \subseteq V: 0 < |S| \le |V|/2} \frac{\delta_{G'}(S)}{|S|} \ge \frac12(d - 2\sqrt{d-1}).
\end{align*}
\end{lemma}

\begin{lemma}\label{lem:ramanujan-graph-property}
For any $a_i \in A$, let $G' = R'(3, Ca_i)$ be the connected graph defined as above, on a set of nodes $V$, say. Let $B \subseteq V$ be such that $|B| \le \ffrac{Ca_i}{2}$. Then, $\delta_{G'}(B) \ge \ffrac{|B|}{100}$. 
\end{lemma}
\begin{proof}
In the original Ramanujan multigraph $R(3, Ca_i)$ (which creates $R'(3, Ca_i)$ after removing repeated edges), the cut $(B, V \setminus B)$ has at least $\ffrac{3|B|}{100}$ edges, using Cheeger's inequality (Lemma \ref{lem:cheegers-inequality-apdx}). In $R'(3, Ca_i)$, the cut size drops by a factor of at most $3$ since we only remove repeated edges, and there can be at most $3$ edges between any two nodes in the graph.
\end{proof}

We construct an instance of {\GHA} by placing these $3m$ $R'$-graphs at unique leaves of a binary tree $B_r$ (see Figure \ref{fig:bounded-degree-reduction-graph}). Like Theorem \ref{thm:bounded-degree-planar-approx-lower-bound}, the binary tree $B_r$ is constructed such that it has at least $3m$ leaves and at most $12m$ nodes. It is easy to see that the smallest balanced binary tree with at least $3m$ leaves satisfies this constraint. Note that the constructed graph has maximum degree $4$.

There are $CmT + |B_r|$ nodes in this graph.
The valuations of the houses are defined the same way as before: for each $j \in [m]$, there are $CT$ houses valued at $j$ and there are $|B_r|$ houses valued at $0$. $C$ again will be decided later. Along with ensuring $C$ is polynomial, we will also ensure that $C$ is even and greater than or equal to $6$. 


We show that the minimum total envy output by $\mathsf{ALG}_{\cal G}$ for the house allocation instance is greater than or equal to $\left \lceil (14400\gamma m^3 T + 1)^{\ffrac{1}{\varepsilon}} \right \rceil$ if and only if there exists {\em no} valid partition for the original \textsc{3-Partition} instance i.e. the original instance was a NO instance. 


$(\Leftarrow)$ Assume there is a valid $3$-partition for the original instance. This case follows the exact same way as Theorem \ref{thm:trees-approx-lower-bound}. The minimum envy is upper bounded by $3m^2$ and the envy output by $\mathsf{ALG}_{\cal G}$ is upper bounded by $3m^2\gamma(CmT+|B_r|)^{1-\varepsilon} \le 3m^2\gamma(CmT+12m)^{1-\varepsilon}$.

$(\Rightarrow)$ Assume there is no valid $3$-partition in the original instance. We will show that any allocation must have an envy of at least $\ffrac{C}{200}$. 

Similar to Theorem \ref{thm:general-approx-lower-bound}, we refer to a value $j \in [m] \cup \{0\}$ as a majority value of a graph $R'(3, Ca_i)$ if this value has been allocated to at least $\ffrac{Ca_i}{2}$ nodes in the grid. 

For some allocation $\pi$, assume there exists a graph $R'(3, Ca_1)$ with no majority value. Let $Q_j$ be the set of nodes with value $j$ in this graph under allocation $\pi$. Note that the number of edges from $Q_j$ to nodes with a different value is at least $\ffrac{|Q_j|}{100}$ (Lemma \ref{lem:ramanujan-graph-property}). Therefore the total number of edges between nodes with different value is lower bounded by $\ffrac{C}{200}$.
We divide by $2$ since each edge gets counted at most twice. Since each of these edges has an envy of $1$, we are done.

From here on, assume that each $R'$-graph has a majority value. 
Let graph $i$ correspond to $R'(3,Ca_i)$. For notational convenience, assume WLOG that graphs $\{1, \dots, \ell_1\}$ have majority value $1$, $\{{\ell_1 +1}, \dots, {\ell_1 + \ell_2}\}$ have majority value $2$ and so on. Using analysis similar to \Cref{thm:trees-approx-lower-bound}, there exists one $j \in [m]$ such that $\sum_{h \in [\ell_j]}a_{\ell_1 + \dots + \ell_{j-1} +h} > T$. Assume again for notational convenience that $j = 1$. Since all these values are integers, we can restate it as $\sum_{h \in [\ell_1]}a_{h} \ge T + 1$. This implies $\sum_{h \in [\ell_1]} Ca_{h} \ge CT + C$. 

Coming to our allocation $\pi$, $\sum_{h \in [\ell_1]} Ca_{h} \ge CT + C$ implies that there are at least $C$ nodes in graphs $\{1,2, \dots, \ell_1\}$ which are not allocated a value of $1$. Let $A_i$ correspond to the set of nodes in grid $i$ allocated a value other than $1$. Note that $\sum_{i \in [\ell_1]} |A_i| \ge C$. For each $i$, the number of edges between nodes in $A_i$ and nodes with value $1$, is at least $\ffrac{|A_i|}{100}$ (Lemma \ref{lem:ramanujan-graph-property}). This gives us a lower bound of $\ffrac{C}{100}$ edges with envy at least $1$.


We set $C = 200\left \lceil (14400\gamma m^3 T + 1)^{\ffrac{1}{\varepsilon}} \right \rceil$ to complete the reduction; this setting crucially ensures that each $Ca_i$ is even and at least $6$. When there is no valid $3$-partition, the total minimum envy (and therefore, the envy output by $\mathsf{ALG}_{\cal G}$) is at least $\ffrac{C}{200} = \left \lceil (14400\gamma m^3 T + 1)^{\ffrac{1}{\varepsilon}} \right \rceil$. However, when there is a valid $3$-partition, the envy output by $\mathsf{ALG}_{\cal G}$ is strictly upper bounded by:
\begin{align*}
    3m^2\gamma(CmT+12m)^{1-\varepsilon} \le 36m^3 \gamma TC^{1-\varepsilon} < 7200m^3 \gamma T \left ( \left \lceil (14400\gamma m^3 T + 1)^{\ffrac{1}{\varepsilon}} \right \rceil \right )^{1 - \varepsilon} < \left \lceil (14400\gamma m^3 T + 1)^{\ffrac{1}{\varepsilon}} \right \rceil.
\end{align*}
This concludes the proof.
\end{proof}


% \lemflowercomputation*
% \begin{proof}
% To prove existence, we only need to show that the numbers $n_1, \dots, n_d$ are guaranteed to exist. If $n -1 < k$, this is trivial: each $n_i = 1$.

% Otherwise, if $n$ is even, then $d$ is required to be an odd value via condition (b). What we need to do is write $n-1$ as the sum of $d$ odd numbers. We set each $n_i$ as the greatest odd number which is at most $\lfloor \frac{n-1}{d} \rfloor$. Once we do this, $n-1 - \sum_{i = 1}^d n_i$ is guaranteed to be a non-negative even number which is strictly less than $2d$. So, we increment some of the $n_i$'s by $2$ till property (a) is satisfied. This construction satisfies the other two properties as well. 
% % We can use a similar construction for the case where $n$ is odd. 

% The above argument not only shows that the numbers $n_1, \dots, n_d$ are guaranteed to exist but also presents a way to compute them in polynomial time. Since each $n_i$ is strictly less than $n$, we can use this subroutine to compute $F(n, k)$ recursively in polynomial time. More formally, let $T(n, k)$ be the complexity of constructing the flower $F(n, k)$. We have

% \begin{align*}
%     T(n, k) &= \left (\sum_{i = 1}^d T(n_i, k) \right ) + O(d) \\
%     &\le k\cdot T((n-1)/(k-1) + 2, k) + O(k) \\
%     &\le k\cdot T((n/k + 2, k) + O(k).
% \end{align*}
% If $n/k < 2$, then $T(n/k + 2, k)$ can be trivially constructed in $O(1)$ time since $n/k + 2 -1 \le 3 \le k$ and the graph is a constant-sized star. If $n/k \ge 2$, we can simplify the above expression as follows:
% \begin{align*}
%     T(n, k) \le k\cdot T(2n/k, k) + O(k),
% \end{align*}
% which simplifies to $T(n, k) = O(n^\frac{\ln 3}{\ln{3} - \ln2})$.
% %To solve this we go back our undergraduate algorithms class to recall the Master theorem and simplify the above expression even further to $T(n, k) = O(n^\frac{\ln k}{\ln{k} - \ln2})$. This value decreases as $k$ increases. Since $k \ge 3$, we can upper bound it as $T(n, k) = O(n^\frac{\ln 3}{\ln{3} - \ln2})$
% \end{proof}

%\lemflowerproperties*


% \thmboundeddegreetreesnp*
% \begin{proof}
% We present a reduction to {\UTP}. We assume the input $3$-partition instance is scaled up such that each $a_i$ is even and at least $1000$. This comes with no loss of generality --- we can multiply each $a_i$ and $T$ by a $1000$ without changing the output of the instance. We also assume $T$ is even. If $T$ is odd, the instance is trivially a NO instance.

% We construct the graph of the {\GHA} instance as follows (see Figure \ref{fig:boundedtrees-reduction-graph}). We start with a path of $3m$ flowers $F(T, 99)$, connected by their pistils. We number these flowers $1$ to $3m$ from left to right. To each flower $i$, we connect a flower $F(a_i, 99)$ and to this flower, we connect two flowers $F(10T, 999)$. Again, connections between flowers are made via an edge between the pistils of the flowers. We refer to the flowers of the form $F(a_i, 99)$ as {\em small} flowers, flowers of the form $F(T, 99)$ as {\em medium} flowers and flowers of the form $F(10T, 999)$ as {\em large} flowers. We define small, medium and large pistils similarly. Note that large flowers have $999$ petals, while small and medium flowers have $99$ petals (since each $a_i$ and $T$ can be assumed to be even). This graph can be constructed in polynomial time using Lemma \ref{lem:flower-computation} and the fact that the inputs are given in unary.



% Now we describe the house values. We need to describe $64mT$ house values since this is the number of nodes in the graph. For this, it is easier to first define the following function/series, $s(j) = (|E|+1)^{2j}$ for any $j \in \mathbb{N}$ where $|E|$ is the number of edges in our constructed graph. Note that equivalently, we can write $s(j) = (64mT)^{2j}$. We define the multiset of house values as the following:
% \begin{align*}
%     T &\text{ houses with value } 0 \\
%     T &\text{ houses with value } s(4m) \\
%     T &\text{ houses with value } s(4m) + s(4m-1) \\
%     \dots \\
%     T &\text{ houses with value } \sum_{j = 2}^{4m} s(j) \\
%     60mT &\text{ houses with value } \sum_{j = 1}^{4m} s(j)
% \end{align*}
% The crucial property these values satisfy is that the intervals between two values are exponentially decreasing in size, so much so that minimum envy allocation must lexicographically minimize the number of edges passing through these gaps. That is, to minimize envy, we need to first minimize the number of lines passing through the $(0, s(4m))$ interval, subject to that minimize the number of lines passing through the $(s(4m), s(4m) + s(4m-1))$ interval, and so on. For ease of readability, we no longer refer to the exact value of the houses but simply refer to them as {\em clusters} of values. More specifically, the $T$ instances of $0$ are the \emph{first} cluster of values, the $T$ instances of $s(4m)$ are the \emph{second} cluster, and so on. There are $4m+1$ clusters, each of which have size $T$ except for the highest value (the $(4m+1)$-th cluster) which has size $60mT$. The largest house value is at most $(4m + 1)\cdot s(4m)$, which requires $O(\log(4m + 1) + (4m)\log(|E| + 1))$ bits to write, and there are still only polynomially many such values, and so this multiset of house values can be written in polynomial time and space.

% Given this constructed instance, let us study the optimal allocation $\pi^{*}$. Note that the optimal allocation must first minimize the number of envy lines between the first two clusters. Therefore, the placement of the first cluster must be done in such a way so as to minimize the cut size between the first cluster and the rest of the graph. We will show that, to minimize envy, the first $3m$ clusters must be allocated to the $3m$ medium flowers, and the next $m$ clusters (from the $(3m+1)$-th to the $(4m)$-th cluster) must each be allocated to exactly three small flowers with sizes $a_i, a_j$ and $a_k$ such that $a_i + a_j + a_k = T$. This is possible if and only if the original instance is a YES instance, and the total envy of the corresponding house allocation is given by
% \begin{equation*}
%     \envy_{\text{YES}} = \sum_{j = 1}^{3m - 1}(j + 1)\cdot s(4m+1-j) + (3m)\cdot s(m+1)  + \sum_{j = 1}^m(3m + 3j)\cdot s(m+1-j) .
% \end{equation*}
% Any other allocation that does not follow this structure has a strictly greater envy. Specifically, this means that when the original instance is a NO instance, the optimal allocation has a greater total envy than $\envy_{\text{YES}}$ defined above.

% To do this, we will need some bounds on petal sizes. The size of each petal in the medium flowers is at most $\frac{6T}{99 \times 5} \le \frac{T}{80}$ (Lemma \ref{lem:flower-properties}). Similarly the size of each petal in the large flowers is $\le \frac{T}{80}$ and the size of each petal in a small flower is $\frac{6a_i}{99 \times 5} \le \frac{6T}{2 \times 99 \times 5} \le \frac{T}{160}$. In particular, all petals throughout the graph have size at most $T/80$, a fact that we shall use several times.

% Let us now consider where the first cluster is allocated in any optimal allocation. Note that in an allocation where the first cluster completely fills up one of the medium flowers at either end of the path (either leftmost or rightmost), the cut size across the first interval is $2$. Any optimal allocation therefore needs to ensure that this cut size is at most $2$. This observation will enable us to rule out many other possibilities.

% If the elements of the first cluster are not allocated to any pistil, then the cut size is at least $80$ since each petal in the graph has size at most $\frac{T}{80}$; so the first cluster must be allocated to at least $80$ petals, each of which must add at least one edge to the cut. So in an optimal allocation, at least one pistil must get an element from the first cluster.

% Suppose a large pistil is given a value in the first cluster. Then, since each petal of the large flower has size at least $T/125$, there must be at least $999-125 \ge 800$ petals of the large flower that remain unfilled by the cluster of values. Each of these unfilled petals adds at least one edge between the first cluster to another cluster, and these edges are all disjoint, and so the cut size must be at least $800$. Any such allocation, therefore, is suboptimal.

% Suppose a small pistil is given a value in the first cluster. This pistil has two neighboring large pistils. We know from the arguments before that these neighbors cannot be in the first cluster, and so these two outgoing edges are across the cut. But then, the entire graph without these two large flowers is still connected, so at least one more edge needs to go across the first cut, and therefore, this allocation is also suboptimal.

% It follows that any optimal allocation must allocate only medium pistils to the first cluster. If multiple such pistils are allocated, say from medium flowers $F_1$ and $F_2$, note that the cluster cannot contain either flower in its entirety, so there is at least one edge from each of them across the cut; but there also must be at least one other edge, from a path that goes from any of these two pistils to any node in the graph placed in a different cluster (such a node must exist, just by counting). This is therefore suboptimal. Similarly, if the first cluster contains exactly one pistil from a medium flower $F_1$, and no other pistil, but it does not contain all of $F_1$, it is easy to see that the cut will have size at least $3$, and so will be suboptimal.

% %If on the other hand, the first cluster contains completely fills up one of the $T$ sized flowers, the cut size is $2$, if the $T$ sized flower picked is one of the extreme ones (leftmost or rightmost). This is optimal. Any other allocation which incompletely fills a $T$-sized flower, or fills any other $T$-sized flower is strictly worse. An allocation that exclusively fills up small flowers \rik{I thought the casework at this point was for pistils of flowers, no? The remaining case is to argue that a small pistil cannot be in the first cluster.} are also worse since each $a_i$ is in the interval $\left [ \frac{T}{4}, \frac{T}{2} \right ]$ and so multiple such $a_i$'s must be filled up and each $a_i$ has an edge to two pistils of flowers of size $10T$ --- who we have already shown is a bad idea to allocate to.

% So, assume from here on out that the first cluster is allocated entirely to the leftmost (or the first) medium flower under $\pi^*$. Our goal is to show that the first $3m$ clusters must be allocated to the $3m$ medium flowers from left to right. We do this by induction.

% Assume the first $k$ clusters are allocated to the first $k$ medium flowers (from the left). Note that the cut size or the total number of lines of envy between the first $k$ clusters and the rest of the graph is $k+1$. Out of all the ways of allocating the rest of the items, we wish to find the one that minimizes the number of lines of envy going through the next interval, between the $(k + 1)$-th cluster and the $(k + 2)$-th one. We will show that in order to achieve this, the $(k+1)$-th cluster must be allocated to the $(k+1)$-th medium flower. If we allocate the $(k+1)$-th cluster to the $(k+1)$-th medium flower, the cut size between the first $k+1$ clusters and the rest of the graph increases by $1$ with respect to the first $k$ clusters i.e. it increases from $k+1$ to $k+2$. The increase in cut size is just an easier way to account for the number of edges between the first $k+1$ clusters and the rest of the graph. We show that all other allocations of the $(k+1)$-th cluster are strictly suboptimal i.e. all other allocations increase the cut size by at least $2$.

% If the $(k+1)$-th cluster is not allocated to any pistil, then the cut size increases by at least $80$. If the $(k+1)$-th cluster is allocated to a large pistil, then the cut size increases by at least $800$. Both of these statements can be proved using similar arguments to the first cluster. 

% For every small pistil that gets a value from the $(k+1)$-th cluster (regardless of whether it is attached to a medium flower that has already been assigned a cluster or not), the cut size increases by at least $1$ --- this is because each of these pistils has $2$ edges to large pistils. So even if the edge connecting the small pistil to the medium pistil is removed from the cut, at least two new edges are added. This argument rules out allocating the $(k+1)$-th cluster to multiple small flowers. If, on the other hand, no medium pistil is allocated and only one small pistil is allocated, at least $T/2$ values from the cluster must be allocated to petals of flowers whose pistil remains unallocated; this comes from the fact that each small flower has size upper bounded by $T/2$, and so the remaining $T/2$ values must be allocated to petals. Since every petal of any kind of flower has size at most $T/80$, we need at least $40$ different petals to be represented among the remaining values, and each of them has to add a distinct edge across the cut, and so the cut size must increase by at least $40$.

% The only cases we have left are ones which involve at least one medium pistil. If multiple of these medium pistils are allocated, then since each of their petals has size at least $T/125$, there can be at most $125$ such petals represented in the cluster, and so we must have at least $198 - 125 \ge 50$ unfinished petals, each of which adds a distinct edge across the cut, increasing the cut size by at least $1$ each.

% So exactly one medium pistil must be allocated to this cluster. We could still have a combination of one medium pistil and at least one small pistil, but this increases the cut size by at least $2$ since each of the small pistils increase the cut size by at least $1$ and the medium flower must be unfinished which adds at least another edge to the cut. 

% Finally, we have the case where exactly one medium pistil and no other pistil is allocated. It is easy to see, using arguments similar to before, that the strictly optimal place to put this medium pistil is the pistil corresponding to the $(k+1)$-th medium flower and the strictly optimal allocation is to fill up the cluster entirely with the $(k+1)$-th medium flower.

% We can conclude that the first $3m$ clusters must be allocated entirely to the medium flowers in order from left to right (or right to left). 
% Note that we have not said anything yet to diffentiate instances with a valid $3$-partition. We will do that next.

% The $(3m +1)$-th cluster still must be allocated to some pistil, but cannot be allocated to a large pistil, by the arguments from before. Note that all the medium flowers have already been allocated values from the first $3m$ clusters. Therefore, the $(3m + 1)$-th cluster must contain at least one pistil, and all pistils in it must be small. If the $(3m+1)$-th cluster fills up exactly $3$ different small flowers, then the cut size increases by exactly $3$ relative to the first $3m$ clusters. This is because $6$ edges from the small pistils to the large pistils are added, but the $3$ edges from the medium pistils to the small pistils are taken away. We will show that this is the best option, and any other allocation is strictly worse. To show this, we must consider the following four possible cases.

% \noindent\textbf{Case 1: The $(3m+1)$-th cluster is allocated to exactly $1$ pistil.} This must be the pistil of a small flower, which we know has at most $T/2$ vertices by assumption. Therefore, at least $T/2$ values in the cluster must be allocated to petals of other flowers. Each of these petals adds at least one edge to the cut since their corresponding pistils have not been allocated values in the first $3m+1$ clusters, and these edges are all distinct. We know each petal in the graph has a size of at most $T/80$, so at least $40$ petals must be represented in this cluster, so the cut size increases by at least $40$, which is strictly suboptimal.

% \noindent\textbf{Case 2: The $(3m+1)$-th cluster is allocated to exactly $2$ pistils.} Both these pistils must correspond to small flowers (say $F_1$ and $F_2$). Both these flowers have a combined size of strictly less than $T$, so at least one other flower outside these two is allocated a value from the $(3m+1)$-th cluster. If either $F_1$ or $F_2$ are not completely filled, then the cut size increases by at least $4$ --- two extra edges come from the pistils of $F_1$ and $F_2$ to the large pistils, the third is from at least one of $F_1$ or $F_2$ being unfinished, and the fourth is from the third flower which is allocated a value (note that it cannot be completely contained in this cluster, because its pistil is in a different cluster).

% If both $F_1$ and $F_2$ are completely filled, then the argument is a little more subtle. Recall that each small pistil adds a $1$ to the cut size from the edges to the large pistils. We need to show that at least $2$ other edges are added to the cut from the values allocated outside $F_1$ and $F_2$. The number of values from the $(3m+1)$-th cluster allocated outside of $F_1$ and $F_2$ is {\em even}, since both $F_1$ and $F_2$ have even size by assumption. (Also, this number of values is nonzero, as $F_1$ and $F_2$ have a combined size of less than $T$ by assumption.) Therefore, if these values are allocated to two different flowers, at least $2$ more edges are added to the cut (since those two flowers have their pistils in other clusters) and we are done. Otherwise, if they are allocated to the same flower, the cut size still increases by at least $2$ because of Lemma \ref{lem:flower-properties} and we are done.

% \noindent\textbf{Case 3: The $(3m+1)$-th cluster is allocated to exactly $3$ pistils but at least one of the flowers is incomplete.}
% Here the cut size increase is trivially at least $4$ --- an increase of $3$ from the pistils having edges to the large pistils, and a fourth from the fact that at least one of the flowers is incomplete.

% \noindent\textbf{Case 4: The $(3m+1)$-th cluster is allocated to $4$ or more pistils.}
% It is a straightforward argument to show that the number of edges added across the cut is $4$ or more in this case.


% Similarly, for the $(3m+2)$-th to the $(4m)$-th cluster, the best possible option is to allocate each cluster in a way that fills up $3$ small flowers. If any of these clusters fail to do so, the cut size is strictly higher. Given allocations of the first $4m$ clusters, there is only one possible allocation for the $(4m+1)$-th cluster; that is, the allocation where values from the $(4m+1)$-th cluster to all nodes which have not been allocated a value from the first $4m$ clusters.

% It is easy to see that it is possible for all the $(3m+1)$-th to $(4m)$-th clusters to fill up three small flowers each if and only if there is a valid $3$-partition in the original instance. More specifically, if each of the $(3m+1)$-th to $(4m)$-th clusters are allocated in a way that attains a lower bound on the number of edges between the $(3m+k)$-th and the $(3m+k+1)$-th interval, this allocation defines a valid $3$-partition. This is because each of the $(3m+1)$-th to $(4m)$-th cluster must be allocated to exactly three small flowers and must fill them up entirely. It is also easy to see that given a valid $3$-partition, we can construct an allocation that attains the ideal total envy.

% To compute the threshold, we proceed as follows. For $1 \leq i \leq 4m$, let $\ell_i$ be the length of the valuation interval between the $i$-th cluster and the $(i + 1)$-th cluster, which is precisely $s(4m + 1 - i)$. If the original {\UTP} instance is a YES instance, then by our arguments above, we can incur an envy of at most
% \begin{equation*}
%     \envy_{\text{YES}} := \sum_{j = 1}^{3m - 1}(j + 1)\cdot \ell_j + (3m)\cdot \ell_{3m} + \sum_{j = 1}^m(3m + 3j)\cdot\ell_{3m + j}.
% \end{equation*}
% Here, the interval $\ell_{3m}$ has increased the cut size by $0$, because it represents the rightmost medium flower being placed in the $3m$-th cluster. Note that the expression above can be computed in polynomial time, since each $\ell_j = s(4m + 1 - i)$ is representable in $O((4m + 1 - i)\log(|E|))$ bits, and all arithmetic operations can be done in polynomial space and time in this number of bits. There are only polynomially many operations to do, so we can compute the threshold $\envy_{\text{YES}}$ in polynomial time.

% From all our analysis above, if the original {\UTP} instance is a YES instance, we can attain an envy of at most $\envy_{\text{YES}}$ on our constructed instance. On the other hand, if the original {\UTP} instance is a NO instance, we know any allocation is lexicographically worse than this expression. By our choice of the function $s(\cdot)$, this envy must be strictly more than $\envy_{\text{YES}}$. It follows that our {\GHA} instance has an allocation with envy at most $\envy_{\text{YES}}$ if and only if the {\UTP} instance is a YES instance. Since the construction is done entirely in polynomial time, this finishes the proof.
% %any other allocation that does not have this structure has strictly worse total envy. Therefore, any instance with no valid $3$ partition has strictly worse envy. Since we know exactly how much the cut size increases for each of the first $4m$ clusters, we know exactly how many lines pass through each interval in the minimum envy allocation when there is a $3$ partition and when there is no $3$ partition. We can easily compute a threshold using these values and separate the instances with a valid $3$-partition. 
% \end{proof}







% \section{Proofs from Section \ref{sec:completebintrees}}\label{apdx:binary}

% \elegancecuts*

% \begin{proof}
%     Consider \emph{any} $(i, n - i)$-cut in $B_k$, say $(S, V \setminus S)$, with $|S| = i$, and suppose there are $m'$ edges going across the cut. We will construct a repunit representation of $i$ with at most $m' + 1$ terms. Suppose the tree $B_k$ is rooted at the node $r$, and direct each edge from parent node to child node. Initialize a sequence $\vec{v}$ to be empty, and take any edge $e$ going across the cut $(S, V \setminus S)$. Observe that $e$ is either directed from $S$ to $V \setminus S$ or the other way round. This edge must have a complete binary subtree $B_{k_1}$ on one side in $B_k$: specifically, the subtree rooted at the child of $e$. If the edge $e$ is directed from $V \setminus S$ to $S$, then append the term $k_1 + 1$ to $\vec{v}$ (this corresponds to adding a repunit), and if $e$ is directed from $S$ to $V \setminus S$, append the term $-(k_1 + 1)$ (corresponding to subtracting a repunit). Once this is done for all edges $e$ going across the cut, we end up with a finite sequence $\vec{v}$. It is now easy to check that either this sequence $\vec{v}$, or the sequence $\vec{v}$ appended with the term $k + 1$, is a valid repunit representation of $i$. It follows that if the cut had been the minimum one, we would have a repunit representation of $i$ with at most $\delta_{B_k}(i) + 1$ terms. Therefore, $\mathsf{elegance}(i) \leq \delta_{B_k}(i) + 1$.

%     Conversely, consider any optimal valid repunit representation of $i$ (which, therefore, has $\mathsf{elegance}(i)$ terms). Note that we can assume WLOG that all the terms are distinct in absolute value. This is because adding and subtracting the same term gives us a suboptimal representation, whereas $(2^{k_1} - 1) + (2^{k_1} - 1) = 2^{k_1 + 1} - 1 - 1$, so we can replace two additive repunits of the same length by two other unequal-length repunits without changing the result. We claim that we can assume WLOG that the largest repunit in this representation is at most $2^k - 1$. Otherwise, if it is $2^{k + 1} - 1$ or larger, then note that the next most significant repunit needs to be $-2^k + 1$, as otherwise the distinctness assumption gives us $i \geq (2^{k+1} - 1) - \sum_{j = 1}^{k-1}(2^j - 1) \geq 2^k$, contradiction. We can replace these two terms using $(2^{k+1} - 1) - (2^k - 1) = (2^k - 1) + 1$, which would replace these two terms by two other repunits without changing the value. This would be a valid representation, with all distinct terms unless the original representation also had a $+1$ term in it. We could then replace the $+1 + 1$ by $+3 - 1$, which would again be valid, unless the original had a $+3$ term in it. We could then replace the $+1 + 1 + 3$ by $+7 - 3 + 1$, which would again be valid, unless the original had a $+7$ term in it. We can keep going this way. What is the largest additive term in the original that we can run into in this way? Note that if we get to an additive term of $2^{k-1} - 1$, then we would have $i \geq (2^{k+1} - 1) - (2^k - 1) + (2^{k-1} - 1) - \sum_{j = 1}^{k-2}(2^j - 1) \geq 2^k$, contradiction. So the largest (additive) term can only be $2^{k-2} - 1$, and we would then terminate.
    
%     So now we have an optimal repunit representation of $i$ with all distinct terms in absolute value, with the largest term being at most $2^k - 1$. We now claim that this repunit representation gives rise to an $(i, n - i)$ cut in $B_k$. We can just take complete binary subtrees of the sizes determined by the terms of the repunit representation, and include or exclude them on one side of the cut (according to the sign of the relevant term). The edges going across the cut will exactly be the edges to the roots of these subtrees, and the number of these edges will be exactly the number of terms in the original representation. This is \emph{some} $(i, n - i)$-cut of $B_k$, and so the smallest one has at most as many edges going across it as this one. It follows that $\delta_{B_k}(i) \leq \mathsf{elegance}(i)$.
% \end{proof}


% \binarylowerbound*
% \begin{proof}
%     Take a large enough binary tree $B_k$ with $n \gg 100$ vertices, and consider the numbers 89 and 94. Note that $\mathsf{elegance}(89) = 3$, as $89 = 127 - 31 - 7$, and $\mathsf{elegance}(94) = 2$, as $94 = 63 + 31$. Furthermore, these are unique minimum-length repunit representations. We claim that no layout $\sigma = (v_1, \ldots, v_n)$ would attain $\delta_{B_k}(\{v_1, \ldots, v_{89}\}) = 3$ and $\delta_{B_k}(\{v_1, \ldots, v_{94}\}) = 2$ simultaneously. Indeed, if a layout $\sigma$ satisfies the first condition, then the root of a subtree with $127$ must receive one of the lowest $89$ values. However, if $\sigma$ also satisfies the second condition, then the $94$ lowest values fill up exactly two complete binary subtrees of size $63$ and $31$, and so the root of any subtree of size $127$ could not have have any of these values. This is a contradiction, and therefore, any value-agnostic algorithm for this complete binary tree needs to choose at most one of these two options. However, now consider two instances, $(B_k, H_1)$ and $(B_k, H_2)$, where $H_1$ consists of $89$ values of $0$ and $n - 89$ values of $1$, whereas $H_2$ consists of $94$ values of $0$ and $n - 94$ values of $1$. The optimal envy on $(B_k, H_1)$ is $\mathsf{elegance}(89) = 3$, whereas the optimal envy on $(B_k, H_2)$ is $\mathsf{elegance}(94) = 2$. A value-agnostic algorithm will yield a sub-optimal result on at least one of these two instances. If it is wrong on $(B_k, H_1)$, it has to have at least $5$ edges spanning the only nontrivial smallest subinterval (since it needs an odd number crossing the cut $\delta_{B_k}(\{v_1, \ldots, v_{89}\})$, as any repunit representation of $89$ needs an odd number of terms, by parity) and it will be off by a factor of at least $5/3 \approx 1.67$ on this instance. If it is wrong on $(B_k, H_2)$, it has to have at least $4$ edges crossing the only nontrivial smallest subinterval (since it needs an even number crossing the cut $\delta_{B_k}(\{v_1, \ldots, v_{94}\})$, again by parity) and it will be off by a factor of at least $4/2 = 2$ on this instance. Therefore, the approximation ratio has to be at least $1.67$.
% \end{proof}



% \inorder*
% \begin{proof}
%     Suppose we allocate the houses in sorted order to the vertices of $B_k$ in the standard in-order traversal. For any $i \leq 2^k - 1$, consider the number of edges of $B_k$ spanning the subinterval $(h_i, h_{i+1})$ of the valuation interval. It can be shown that under the in-order traversal, the number of edges spanning this interval is exactly $\mathsf{runs}(i)$, the number of runs of contiguous $0$s or $1$s in the binary representation of $i$, by a simple argument\footnote{ For instance, the two quantities follow the same recurrence relation: $f(2^k + i) = f(2^k - i + 1) + 1$ for $k \geq 0$ and $0 < i \leq 2^k$, with the same base cases.}.
    
%     Our main claim will be to show that for all $i$, $\mathsf{runs}(i) \leq 3\cdot\mathsf{elegance}(i) - 2$. Let $\mathsf{elegance}(i) = r$, and consider an optimal repunit representation $(a_1, \ldots, a_r)$ of $i$. WLOG suppose $|a_1| \geq \ldots \geq |a_r|$. Then $\mathsf{sgn}(a_1) = 1$. We will start with the binary representation $11\ldots 1$ of $2^{a_1} - 1$, which contains a single run of exactly $a_1$ $1$s. We will then add or subtract all the other terms $a_2, \ldots, a_r$, performing all our operations in binary. We will carefully keep track of how each of these operations can affect the number of runs.

%     Consider an arbitrary binary integer, with $t$ runs, and consider adding a repunit to it. Adding such a repunit can be thought of as adding a single power of $2$ (which is a binary integer of the form $10\ldots 0$), and then subtracting a single $1$. When we add the power of $2$, starting from the right, the $0$s do not change the number of runs, until we get to the leading $1$. Observe that adding or subtracting a single $1$ can increase the number of runs by at most $1$. Therefore, at the leading $1$, we can add a new run by a mismatched bit between the $0$ and the $1$, and can also add a new run by adding the $1$ itself. Therefore, adding a power of $2$ can increase the number of runs by at most $2$. After that, subtracting the $1$ adds at most another run, as observed. Therefore, adding a repunit adds at most three runs to the original binary integer. By a symmetric argument, subtracting a repunit (which is equivalent to subtracting a power of $2$, and then adding a $1$) can also increase the number of runs by at most $3$.

%     Since we started with $2^{a_1} - 1$, which contained a single run, and then added or subtracted $r - 1$ other repunits, the total number of runs in the final integer is at most $1 + 3(r - 1)$. This immediately implies that $\mathsf{runs}(i) \leq 3\cdot\mathsf{elegance}(i) - 2$.

%     Coming back to $B_k$, we have just shown that for $i \leq 2^k - 1$, the number of edges in the in-order traversal spanning the subinterval $(h_i, h_{i+1})$ is at most $3\cdot\mathsf{elegance}(i) - 2$, which by Proposition \ref{prop:elegance} is at most $3\cdot\delta_{B_k}(i) + 1$. We now consider a couple of cases.

%     We note that, for $i \leq 2^k - 1$, we have $\delta_{B_k}(i) = 1$ if and only if $i = 2^{k_1} - 1$ for some $k_1 \leq k$. This follows just by observing that every edge in a complete binary tree has a complete binary subtree on one side, and the other side cannot have size less than $2^k$. But in that case, $\mathsf{runs}(i) = \delta_{B_k}(i)$, and so on these subintervals, the in-order traversal only subtends a single edge (and is therefore optimal).

%     On the other hand, for $i \leq 2^k - 1$, if $\delta_{B_k}(i) \geq 2$, then $1 \leq (\ffrac{1}{2})\cdot\delta_{B_k}(i)$.

%     Therefore, the number of edges over any subinterval $(h_i, h_{i+1})$ in the range $i \leq 2^k - 1$ is at most $3\cdot\delta_{B_k}(i) + (\ffrac{1}{2})\cdot\delta_{B_k}(i) = 3.5\cdot\delta_{B_k}(i)$. By symmetry, this is true over all smallest subintervals of the valuation interval. Therefore, the number of edges passing over every smallest subinterval is at most $3.5$ times the minimum possible number of edges passing over that subinterval, and this yields the desired result.
% \end{proof}

% \begin{remark}\label{rem:optimizedinorderproof}
%     The proof of Theorem \ref{thm:inorder} can be optimized slightly for a more nuanced analysis. As observed in the proof, adding or subtracting repunits can be thought of as adding or subtracting powers of $2$, followed by subtracting or adding off $1$s. Consider performing all operations with the powers of $2$ first, and then finally adding or subtracting the number obtained by the $\pm 1$s. Each power-of-$2$ operation increases the number of runs by at most $2$, and the final additional number is at most $r - 1$, which has $1 + \lfloor\log(r - 1)\rfloor$ bits. Therefore, the total number of runs is actually $2 + 2(r - 1) + \lfloor\log(r - 1)\rfloor$. Note that this number is at most $(7/3)r$ for all $r \geq 18$, and in fact, the ratio of this number to $r$ gets arbitrarily close to $2$ as $r$ gets larger.
% \end{remark}