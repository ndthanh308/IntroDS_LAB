% \documentclass[sigconf]{aamas} 
\documentclass{article} 
\usepackage{graphicx} % Required for inserting images

%%% AAMAS-2024 copyright block (do not change!)

% \setcopyright{ifaamas}
% \acmConference[AAMAS '24]{Proc.\@ of the 23rd International Conference
% on Autonomous Agents and Multiagent Systems (AAMAS 2024)}{May 6 -- 10, 2024}
% {Auckland, New Zealand}{N.~Alechina, V.~Dignum, M.~Dastani, J.S.~Sichman (eds.)}
% \copyrightyear{2024}
% \acmYear{2024}
% \acmDOI{}
% \acmPrice{}
% \acmISBN{}


\usepackage{tikz}
\usepackage{amsmath}
\usepackage{mathtools}
\usepackage{natbib}
\usepackage{caption}
\captionsetup[subfigure]{justification=centering}
% \usepackage{amsmath, amssymb, amsthm, mathrsfs, tikz, footmisc}
% \usepackage{multirow}
% \usepackage{algorithm2e}
% \usepackage{hyperref}
% \usetikzlibrary{shapes.geometric, arrows}
% \usepackage[sort]{cite}
\usetikzlibrary{arrows.meta,
                chains,
                decorations.pathreplacing,calligraphy,
                positioning,
                quotes,
                shapes.geometric
                }
\DeclareMathOperator*{\argmax}{arg\,max}
\DeclareMathOperator*{\argmin}{arg\,min}



% \usepackage{subcaption}

\usetikzlibrary{matrix,decorations.pathreplacing}

\usepackage{xcolor}
% If you use BibTeX in apalike style, activate the following line:
% \bibliographystyle{apalike}
\usepackage{easybmat}
\usepackage{multirow,bigdelim}
\usepackage{amsmath}
\usepackage{amsthm}
\usepackage{paralist}
\let\Bbbk\relax
\usepackage{amssymb}
\usepackage{algorithm}
\usepackage[justification = centering]{subcaption}
\usepackage{color}
% \usepackage[english]{babel}
\usepackage{graphicx}
\usepackage{wrapfig,epsfig}
\usepackage{epstopdf}
\usepackage{url}
\usepackage{graphicx}
\usepackage{color}
\usepackage{epstopdf}
\usepackage[noend]{algpseudocode}
%\usepackage{algorithmicx}
% \usepackage{scrextend}
\usepackage[T1]{fontenc}
\usepackage{bbm}
\usepackage{comment}
\usepackage{xspace}
\usepackage{thmtools}
%%% print refs in table of contents
\let\C\relax
\usepackage{tikz}
\usetikzlibrary{calc}
\usepackage[backref=page]{hyperref}  %%% arxiv don't allow this.
% \renewcommand*{\backref}[1]{}
% \renewcommand*{\backrefalt}[4]{%
%     \ifcase #1 (Not cited.)%
%     \or        (Cited on page~#2)%
%     \else      (Cited on pages~#2)%
%     \fi}
\usepackage{cleveref}
\hypersetup{colorlinks=true,citecolor=blue,linkcolor=blue} 
\usetikzlibrary{arrows}
\usepackage[lmargin=1in,rmargin=1in,tmargin=0.8in,bmargin=0.8in]{geometry}
% \usepackage[margin=1in]{geometry}
% \linespread{1}
% %\newcommand{\QED}{\hfill$\qed$}
% % \graphicspath{{./figs/}}
% \newcommand{\indep}{\rotatebox[origin=c]{90}{$\models$}}
% \newcommand*{\centernot}{%
%   \mathpalette\@centernot
% }

% \usepackage{enumitem}
% \setitemize{leftmargin=*}

\newenvironment{CompactItemize}{
\begin{list}{$\bullet$}{%
\setlength{\leftmargin}{12pt}
\setlength{\itemindent}{5pt}
\setlength{\topsep}{1pt}
\setlength{\itemsep}{-2pt}
}}
{\end{list}}

\newenvironment{CompactEnumerate}{
\begin{list}{\roman{enumi}.}{%
\usecounter{enumi}
\setlength{\leftmargin}{12pt}
\setlength{\itemindent}{5pt}
\setlength{\topsep}{1pt}
\setlength{\itemsep}{-2pt}
}}
{\end{list}}

% \def\@centernot#1#2{%
%   \mathrel{%
%     \rlap{%
%       \settowidth\dimen@{$\m@th#1{#2}$}%
%       \kern.5\dimen@
%       \settowidth\dimen@{$\m@th#1=$}%
%       \kern-.5\dimen@
%       $\m@th#1\not$%
%     }%
%     {#2}%
%   }%
% }



% \author{
% }



\newtheorem{theorem}{Theorem}[section]
\newtheorem{lemma}[theorem]{Lemma}
\newtheorem{definition}[theorem]{Definition}
\newtheorem{notation}[theorem]{Notation}
\newtheorem{proposition}[theorem]{Proposition}
\newtheorem{corollary}[theorem]{Corollary}
\newtheorem{conjecture}[theorem]{Conjecture}
\newtheorem{assumption}[theorem]{Assumption}
\newtheorem{observation}[theorem]{Observation}
\newtheorem{fact}[theorem]{Fact}
\newtheorem{remark}[theorem]{Remark}
\newtheorem{claim}[theorem]{Claim}
\newtheorem{example}[theorem]{Example}
\newtheorem{problem}[theorem]{Problem}
\newtheorem{open}[theorem]{Open Problem}
\newtheorem{question}[theorem]{Question}
\newtheorem{hypothesis}[theorem]{Hypothesis}


\newcommand{\pparagraph}[1]{\noindent \textbf{#1}}
% \newcommand{\tail}[2]{#1_{\overline{[#2]}}}
\newcommand{\abs}[1]{|#1|}
\newcommand{\tabs}[1]{\left|#1\right|}
%\newcommand{\norm}[1]{\|#1\|}
%\newcommand{\inner}[1]{\langle#1\rangle}
\newcommand{\ffrac}[2]{#1/#2}
\newcommand{\envy}{\mathsf{envy}}
\newcommand{\efxenvy}{\textit{strong-envy}}
\newcommand{\wh}{\widehat}
\newcommand{\wt}{\widetilde}
\newcommand{\ov}{\overline}
\newcommand{\eps}{\epsilon}
\newcommand{\N}{\mathcal{N}}
\newcommand{\R}{\mathbb{R}}
\newcommand{\volume}{\mathrm{volume}}
\newcommand{\RHS}{\mathrm{RHS}}
\newcommand{\LHS}{\mathrm{LHS}}
\newcommand{\Grid}{\mathrm{Grid}}
\renewcommand{\i}{\mathbf{i}}
\newcommand{\norm}[1]{\left\lVert#1\right\rVert}
\renewcommand{\varepsilon}{\epsilon}
\renewcommand{\tilde}{\wt}
\renewcommand{\hat}{\wh}
\newcommand{\floor}[1]{\left\lfloor #1 \right\rfloor}
\newcommand{\ceil}[1]{\left\lceil #1 \right\rceil}
\DeclareMathOperator*{\E}{{\bf {E}}}
%\DeclareMathOperator*{\Pr}{{\bf{Pr}}}
\DeclareMathOperator*{\var}{\mathrm{Var}}
\DeclareMathOperator*{\Z}{\mathbb{Z}}
%\DeclareMathOperator*{\R}{\mathbb{R}}
%\DeclareMathOperator*{\argmin}{argmin}
\DeclareMathOperator*{\C}{\mathbb{C}}
\DeclareMathOperator*{\median}{median}
\DeclareMathOperator*{\mean}{mean}
\DeclareMathOperator{\OPT}{OPT}
\DeclareMathOperator{\supp}{supp}
\DeclareMathOperator{\poly}{poly}
\DeclareMathOperator{\Tr}{Tr}
\DeclareMathOperator{\nnz}{nnz}
\DeclareMathOperator{\loc}{loc}
\DeclareMathOperator{\repeats}{repeat}
\DeclareMathOperator{\heavy}{heavy}
\DeclareMathOperator{\emp}{emp}
\DeclareMathOperator{\est}{est}
\DeclareMathOperator{\sparsity}{sparsity}
\DeclareMathOperator{\rank}{rank}
\DeclareMathOperator{\tucker}{tucker}
\DeclareMathOperator{\train}{train}
\DeclareMathOperator{\Diag}{diag}
\DeclareMathOperator{\dist}{dist}
\DeclareMathOperator{\rect}{rect}
\DeclareMathOperator{\sinc}{sinc}
\DeclareMathOperator{\Gram}{Gram}
\DeclareMathOperator{\Gaussian}{Gaussian}
\DeclareMathOperator{\dis}{dis}
\DeclareMathOperator{\Sym}{Sym}
\DeclareMathOperator{\Comb}{Comb}
\DeclareMathOperator{\signal}{signal}
% \DeclareMathOperator{\sign}{sign}
\DeclareMathOperator{\EMD}{EMD}
\DeclareMathOperator{\EEMD}{EEMD}
\DeclareMathOperator{\constraints}{constraints}
\DeclareMathOperator{\degree}{degree}
\DeclareMathOperator{\variables}{variables}
\DeclareMathOperator{\TV}{TV}
\DeclareMathOperator{\KL}{KL}
\DeclareMathOperator{\cost}{cost}
\DeclareMathOperator{\vect}{vec}
\DeclareMathOperator{\tr}{tr}
\DeclareMathOperator{\RAM}{RAM}
\DeclareMathOperator{\ati}{ati}
\DeclareMathOperator{\sym}{mon}
\DeclareMathOperator{\mon}{mon}
\DeclareMathOperator{\relu}{ReLU}
\DeclareMathOperator{\reg}{reg}
\DeclareMathOperator{\group}{group}
\newcommand{\Lat}{\mathcal{L}}
\newcommand{\SAT}{{\sf 3SAT}~}
\newcommand{\ESAT}{{\sf E3SAT}~}
\newcommand{\ESATB}{{\sf E3SAT(B)}~}
\newcommand{\UTP}{{\sc Unary 3-Partition}\xspace}
\newcommand{\TP}{{\sc 3-Partition}\xspace}
\newcommand{\MAX}{{\sf MAX}}
\newcommand{\BBB}{{\mathcal B}}
\newcommand{\AAA}{{\mathcal A}}
\newcommand{\DDD}{{\mathcal D}}
\newcommand{\MAXCUT}{{\sf MAX-CUT}~}
\newcommand{\CNF}{{\sf CNF}~}
\newcommand{\nCNF}{{\sf 3-CNF}~}
\newcommand{\kSUM}{{\sf k-SUM}~}
\newcommand{\kClique}{{\sf k-Clique}~}
\newcommand{\ETH}{{\sf ETH}~}
\newcommand{\NP}{{\bf{NP}}}
\newcommand{\RP}{{\bf{RP}}}
\newcommand{\EFXtwo}{{\textsf{EFX2} }}
\renewcommand{\cal}[1]{\mathcal{#1}}
\newcommand{\GHA}{\textsc{Graphical House Allocation}}
\newcommand{\MLA}{\textsc{Minimum Linear Arrangement}}

\newcommand{\cw}{\mathsf{cutwidth}}
\newcommand{\pw}{\mathsf{pathwidth}}
\newcommand{\tw}{\mathsf{treewidth}}

\newcommand{\polylog}{\text{polylog}}


\definecolor{mygreen}{RGB}{80,180,0}
\definecolor{b2}{RGB}{51,153,255}
\definecolor{mycy2}{RGB}{255,51,255}


\newcommand{\COMMENT}[1]{\color{\blue} #1}
\newcommand{\Envy}{\mathsf{Envy}}


\colorlet{thechosenone}{red}
\colorlet{thechosentwo}{blue}
\makeatletter
\newcommand*{\RN}[1]{\expandafter\@slowromancap\romannumeral #1@}
\makeatother
\usepackage{lineno}
\def\linenumberfont{\normalfont\small}

%\input{libtheorems.tex}

\newif\ifcomments
\commentstrue
\ifcomments
\newcommand{\rik}[1]{{\textcolor{red}{Rik: { #1}}}}
\else
\newcommand{\rik}[1]{}
\fi
\newif\ifcomments
\commentstrue
\ifcomments
\newcommand{\justin}[1]{{\textcolor{blue}{Justin: { #1}}}}
\else
\newcommand{\justin}[1]{}
\fi
\newif\ifcomments
\commentstrue
\ifcomments
\newcommand{\vignesh}[1]{{\textcolor{orange}{Vignesh: { #1}}}}
\else
\newcommand{\vignesh}[1]{}
\fi
\newif\ifcomments
\commentstrue
\ifcomments
\newcommand{\rohit}[1]{{\textcolor{magenta}{Rohit: { #1}}}}
\else
\newcommand{\rohit}[1]{}
\fi
\newif\ifcomments
\commentstrue
\ifcomments
\newcommand{\hadi}[1]{{\textcolor{purple}{Hadi: { #1}}}}
\else
\newcommand{\hadi}[1]{}
\fi
\newif\ifcomments
\commentstrue
\ifcomments
\newcommand{\andrew}[1]{{\textcolor{teal}{Andrew: { #1}}}}
\else
\newcommand{\andrew}[1]{}
\fi





\title{Tight Approximations for  Graphical House Allocation}
\author{%
  % Author\thanks{Use footnote for providing further information
  %   about author (webpage, alternative address)---\emph{not} for acknowledging
  %   funding agencies.} \\
  Hadi Hosseini\\
  Penn State University\\
  \texttt{hadi@psu.edu} \\
  \and
  Andrew McGregor \\
  UMass Amherst\\
  \texttt{mcgregor@cs.umass.edu} \\
  \and
  Rik Sengupta \\
  UMass Amherst\\
  \texttt{rsengupta@cs.umass.edu} \\
  \and
  Rohit Vaish \\
  IIT Delhi\\
  \texttt{rvaish@iitd.ac.in} \\
  \and
  Vignesh Viswanathan \\
  UMass Amherst\\
  \texttt{vviswanathan@umass.edu} \\
}
\date{}





\begin{document}

\maketitle

\begin{abstract}
The {\GHA} problem asks: how can $n$ houses (each with a fixed non-negative value) be assigned to the vertices of an undirected graph $G$, so as to minimize the ``aggregate local envy'', i.e., the sum of absolute differences along the edges of $G$? This problem generalizes the classical {\MLA} problem, as well as the well-known \emph{House Allocation Problem} from Economics, the latter of which has notable practical applications in organ exchanges. Recent work has studied the computational aspects of {\GHA} and observed that the problem is NP-hard and inapproximable even on particularly simple classes of graphs, such as vertex disjoint unions of paths. However, the dependence of any approximations on the structural properties of the underlying graph had not been studied.

In this work, we give a complete characterization of the approximability of {\GHA}. We present algorithms to approximate the optimal envy on general graphs, trees, planar graphs, bounded-degree graphs, bounded-degree planar graphs, and bounded-degree trees.
For each of these graph classes, we then prove \emph{matching} lower bounds, showing that in each case, no significant improvement can be attained unless P = NP. We also present general approximation ratios as a function of structural parameters of the underlying graph, such as treewidth; these match the aforementioned tight upper bounds in general, and are significantly better approximations for many natural subclasses of graphs. Finally, we present constant factor approximation schemes for the special classes of complete binary trees and random graphs. 
% present a simple $3$-approximation algorithm to compute a minimum envy allocation on complete binary trees. We complement this result by showing that a natural class of algorithms (``value-agnostic'' procedures) cannot attain an approximation ratio better than 1.66 on complete binary trees.
% Finally, we investigate the special case of bounded-degree trees in some detail. We first refute a conjecture by \cite{canon} about the structural properties of exact optimal allocations on binary trees by means of a counterexample on a depth-$3$ complete binary tree. This refutation, together with our hardness results on trees, might suggest that approximating the optimal envy even on complete binary trees is infeasible. Nevertheless, we present a linear-time algorithm that attains a 3-approximation on complete binary trees. 
% %We show that a natural class of algorithms (``value-agnostic'' procedures) cannot attain an approximation ratio better than 1.66 on complete binary trees.
% We leave open the intriguing question of whether {\GHA} is NP-hard on binary trees.

Some of the technical highlights of our work are the use of expansion properties of Ramanujan graphs in the context of a classical resource allocation problem, and approximating optimal cuts in binary trees by analyzing the behavior of consecutive runs in bitstrings.
% Most of our results rely on exploiting graph structures to reason about optimal arrangements. In particular, several of our hardness reductions are quite technical; in one case, we make use of the expansion properties of Ramanujan graphs. To our knowledge, the use of these graphs and these techniques in the context of resource allocation problems is novel.
\end{abstract}

\section{Introduction}

%\hadi{1. make WLOG and wlog consistent throughout. \\
%2. In some places (e.g. Section 5) we use cuts on valuation intervals. But I can't see it explicitly defined. It'd be nice to include it in preliminaries. \\
%3. we use `good', `houses', and `items' even in the intro. I suggest using \textit{items} and \textit{houses} only throughout.}

In the EconCS community, the \emph{House Allocation Problem} has been a topic of significant interest for some time \citep{shapley1974cores, svensson1999strategy, beynier2018localenvy, gsvfairhouse, kmsfairhouse}. In its canonical form, the problem involves a set of $n$ agents, a set of $n$ items (``houses''), and possibly different valuation functions for each agent. In general, given this framework, the problem asks for an ``optimally fair'' allocation of the houses to the agents. For instance, we might wish to minimize the total envy, or maximize the number of envy-free pairs of agents. In this context, as it is common in the fairness literature, an agent $i$ envies an agent $j$ in a particular allocation if according to agent $i$'s valuation function, the item received by agent $j$ is worth more than the item received by agent $i$; the amount of envy is the difference in these two values. The canonical problem has been studied in a variety of contexts, and is well-known as an algorithmically difficult problem to solve, for most reasonable fairness objectives.

\citet{canon} introduced a variant of the house allocation problem called {\GHA}. In this setting, there are $n$ agents, but now they are placed on the vertices of an undirected $n$-vertex graph $G = (V, E)$. There are still $n$ items with arbitrary values, but the agents are \emph{identical} in how they value these $n$ items (i.e., they all agree on the value of each house). {\GHA} now asks: how do we allocate each house to an agent so as to minimize the total envy along the edges of $G$?

We remark here that the setting where the agents are on a graph and only the envy along the graph edges is considered was studied before as well by \citet{beynier2018localenvy}, who considered ordinal preferences in such a setting, and were interested in maximizing the number of envy-free edges in the underlying graph. %Note that {\GHA} is the variant where we have a utilitarian objective, of minimizing the total envy.

Observe that {\GHA} is a purely combinatorial problem: we are given an $n$-vertex graph $G = (V, E)$ and a multiset $H = \{h_1, \ldots, h_n\} \subseteq \R_{\geq 0}$. We wish to find the bijective function $\pi : V \to H$ that minimizes
$\sum_{(x, y) \in E}|\pi(x) - \pi(y)|$.

If the set of values were $H = \{1, \ldots, n\}$, then {\GHA} would be identical to the well-known {\MLA} problem. This was observed by \citet{canon}, who went on to show some remarkable differences between the two problems. For instance, while all hardness results carry over from {\MLA} to {\GHA}, the latter is actually a significantly harder problem even on very simple graphs. In particular, in {\MLA}, we can assume without loss of generality that the underlying graph is connected; this is because an optimal solution is given by taking each connected component separately, and optimally assigning a contiguous subset of values to it. We lose this guarantee in {\GHA}, even for  small graphs with just two connected components. As a typical example of the differences between the two problems, observe that if the underlying graph is a disjoint union of paths, then solving {\MLA} optimally takes linear time, but even on these simple instances, {\GHA} is NP-complete \citep{canon}.

We do note, however, that all the hardness constructions by \citet{canon} used the disconnectedness of the underlying graphs crucially, in finding reductions from bin packing instances. Their results also show that for very simple classes of disconnected graphs, {\GHA} is inapproximable to any finite factor. However, these proof techniques do not carry over to \emph{connected} graphs, and so it was not known whether any of these reductions would go through for connected graphs. For instance, a well known result by \citet{mlatrees} states that {\MLA} is solvable in polynomial time on trees; the complexity of this problem for {\GHA} was open.

%Note that it is relatively straightforward to show that an $\alpha$ approximation algorithm for the \MLA~problem yields an $\alpha \cdot \phi$ approximation for the $\GHA$ problem where $\phi=\max_{1\leq i\leq n-1} (h_{i+1}-h_i)/\min_{1\leq i\leq n-1} (h_{i+1}-h_i)$.


%\andrew{Do we also want to mention that min-bisection is a special case of graphical house allocation but that general graphical house allocation is strictly harder, e.g., min bisection is poly-time on trees but we show graphical house allocatiton is NP hard on trees.}

\subsection{Our Contributions}

We present a complete characterization of the approximability of {\GHA} on various classes of connected graphs, summarized in Table \ref{table:summary1}. In particular, for any instance on the following graph classes, we show a polynomial-time\footnote{ In all our results, $\tilde{O}$ hides $\polylog(n)$ factors} algorithm on an $n$-vertex graph $G$ (with maximum degree $\Delta$) in that class for obtaining the stated multiplicative approximation to the optimal envy, and then demonstrate a matching lower bound that shows that any polynomial improvement on the approximation ratio is impossible on that graph class unless P = NP:
%Specifically, we show the following results for the multiplicative approximation to the optimal envy for an instance of {\GHA} defined on a graph $G$ with $n$ nodes and maximum degree $\Delta$:
\begin{itemize}
    \item If $G$ is any connected graph, \emph{any} allocation attains the trivial upper bound of $O(n^2)$ (Proposition \ref{prop:trivialgeneral}). In Theorem \ref{thm:general-approx-lower-bound}, we show that we cannot have an $O(n^{2 - \epsilon})$-approximation for any $\epsilon > 0$. We also give a polynomial-time $\tilde{O}(\tw(G)\cdot\Delta)$-approximation algorithm (Corollary \ref{cor:cutwidth-upperbounds}).
    
    \item If $G$ is a tree, \emph{any} allocation attains the trivial upper bound of $O(n)$ (Proposition \ref{prop:trivialgeneral}). In Theorem \ref{thm:trees-approx-lower-bound}, we show that we cannot have an $O(n^{1 - \epsilon})$-approximation for any $\epsilon > 0$. This is in stark contrast to {\MLA}, where there are sub-quadratic algorithms for \emph{exact} solutions on trees \citep{mlatrees}. We also explicitly show a simple divide-and-conquer procedure (Algorithm \ref{alg:treelogn}) that gives the same $O(\Delta\log n)$-approximation in $O(n\log n)$ time.

    \item If $G$ is planar, Corollary \ref{cor:cutwidth-upperbounds} gives us a polynomial-time algorithm to achieve an $\tilde{O}(\sqrt{n\Delta})$-approximation to the optimal envy. In the worst case, $\Delta = \Theta(n)$, so this is a worst-case approximation of $\tilde{O}(n)$. Once again, Theorem \ref{thm:trees-approx-lower-bound} shows that we cannot have an $O(n^{1 - \epsilon})$-approximation for any $\epsilon > 0$.

    \item If $G$ is a bounded-degree graph, Corollary \ref{cor:cutwidth-upperbounds} gives us a polynomial-time algorithm to achieve an $\tilde{O}(\tw(G))$-approximation to the optimal envy. Again, this is a worst-case approximation of $\tilde{O}(n)$. Using Theorem \ref{thm:bounded-degree-approx-lower-bound}, we show that we cannot have an $O(n^{1 - \epsilon})$-approximation for any $\epsilon > 0$. This is our %hardest
    most involved technical result, and it uses expansion properties of Ramanujan graphs.

    \item If $G$ is a bounded-degree planar graph, Corollary \ref{cor:cutwidth-upperbounds} gives us a polynomial-time $\tilde{O}(\sqrt{n})$-approximation algorithm. We match this by showing that we cannot have an $O(n^{0.5 - \epsilon})$-approximation for any $\epsilon > 0$ (Theorem \ref{thm:bounded-degree-planar-approx-lower-bound}).

    \item If $G$ is a bounded-degree tree, both Algorithm \ref{alg:treelogn} and Corollary \ref{cor:cutwidth-upperbounds} give us a polynomial-time algorithm that outputs an $\tilde{O}(1)$-approximation to the optimal envy. We show that finding the exact optimal envy is NP-hard (Theorem \ref{thm:bounded-degree-trees-np-complete}).
\end{itemize}

\begin{table*}
    \centering
    \def\arraystretch{1.3}
    \begin{tabular}{ |l|l|c|  }
 \hline
 \multicolumn{3}{|c|}{Approximations for {\GHA}} \\
 \hline
 \bf Graph Class & \bf Upper Bound & \bf Lower Bound \\
 \hline
\multirow{2}{15em}{Connected graphs}  & $O(n^2)$ (Prop.~\ref{prop:trivialgeneral}) & \multirow{2}{10em}{\centering $\omega(n^{2 - \varepsilon})$ (Thm.~\ref{thm:general-approx-lower-bound})} \\
& $O(\tw(G)\cdot\Delta\log^{2.5} n)$ (Cor.~\ref{cor:cutwidth-upperbounds}(\ref{cor:cwgeneral})) & \\
\hline 
\multirow{2}{15em}{Trees}  & $O(n)$ (Prop.~\ref{prop:trivialgeneral}) & \multirow{2}{10em}{\centering $\omega(n^{1 - \varepsilon})$ (Thm.~\ref{thm:trees-approx-lower-bound})} \\
& $O(\Delta\log n)$ (Alg.~\ref{alg:treelogn}, Cor.~\ref{cor:cutwidth-upperbounds}(\ref{cor:cwtrees})) & \\
\hline
Planar graphs  & $O(\sqrt{n\Delta}\log^{1.5}n)$ (Cor.~\ref{cor:cutwidth-upperbounds}(\ref{cor:cwplanar})) & $\omega(n^{1 - \varepsilon})$ (Thm.~\ref{thm:trees-approx-lower-bound}) \\
\hline
Bounded-degree graphs  & $O(\tw(G)\cdot\log^{2.5}n)$ (Cor.~\ref{cor:cutwidth-upperbounds}(\ref{cor:cwgeneral})) & $\omega(n^{1 - \varepsilon})$ (Thm.~\ref{thm:bounded-degree-approx-lower-bound}) \\
\hline
Bounded-degree planar graphs &  $O(\sqrt{n}\log^{1.5}n)$ (Cor.~\ref{cor:cutwidth-upperbounds}(\ref{cor:cwplanar})) & $\omega(n^{0.5 - \varepsilon})$ (Thm.~\ref{thm:bounded-degree-planar-approx-lower-bound}) \\
\hline
Bounded-degree trees & $O(\log n)$ (Thm.~\ref{thm:treelogn}, Cor.~\ref{cor:cutwidth-upperbounds}(\ref{cor:cwtrees})) & $> 1$ (NP-hard, Thm.~\ref{thm:bounded-degree-trees-np-complete}) \\
 \hline \hline
 Random graphs & $1+ O(\sqrt{\ln (n)/n})$ (Thm.~\ref{thm:random})) \emph{w.h.p.} & -- \\
 \hline
 Complete binary trees & $3.5$ (Thm.~\ref{thm:inorder})) & \textcolor{red}{open} (Conj.~\ref{conj:completebintrees}) \\
 \hline
\end{tabular}
\caption{Summary of our results. Here, $\Delta$ is the maximum degree of the graph in question, and the lower bounds assume P $\neq$ NP. Note that in all cases, the upper and lower bounds match up to polylogarithmic factors, showing that nontrivial improvements to these upper bounds are impossible unless P = NP. All our upper bounds are polynomial time.}
\label{table:summary1}
\end{table*}

Note that assuming connectivity in the results above is necessary, since \citet{canon} showed that disconnected graphs cannot have the optimal envy approximated to any finite factor. We give the first known results for connected graphs.

We also show that for random graphs, any allocation is a $(1 + o(1))$-approximation with high probability (Theorem \ref{thm:random}).

Finally, we investigate complete binary trees in further detail. We first show that the class of binary trees is not ``well-behaved'', by refuting a conjecture by \citet{canon} about the structural properties of exact optimal allocations on binary trees by means of a counterexample (Section \ref{sec:boundeddegreetrees}). The hardness results in Theorems \ref{thm:trees-approx-lower-bound}, \ref{thm:bounded-degree-approx-lower-bound}, and \ref{thm:bounded-degree-trees-np-complete} might have suggested that complete binary trees cannot have $o(\log n)$-approximations in general. We show, however, that just the in-order traversal on a complete binary tree achieves a $3.5$-approximation to the optimal envy (Theorem \ref{thm:inorder}). We also show that this approximation ratio cannot be improved beyond 1.67 by a natural class (``value-agnostic'') of algorithms. %However, we do not resolve the question of whether attaining an exact solution on complete binary trees is NP-hard or not, and leave this for future research.

% Finally, we consider a generalized version of our problem on trees wherein the tree is \emph{edge-weighted} (and the envy along each edge has to be multiplied by the weight of that edge). For these graphs, we can adapt Algorithm \ref{alg:treelogn} easily, giving us an $O(\Delta\log n)$-approximation to the optimal envy on such trees in nearly linear time. Here $\Delta$ is the maximum \emph{weighted} degree, the sum of weights of all the edges incident on a vertex.

% These results on special cases are summarized in Table \ref{table:summary2}.

Our paper is organized as follows. In Section \ref{sec:prelims}, we set up preliminaries. In Sections \ref{sec:upper} and \ref{sec:lower}, we present our upper and lower bounds respectively from Table \ref{table:summary1}. In Section \ref{sec:completebintrees}, we discuss binary trees. We finish with concluding remarks and open directions in Section \ref{sec:conclusions}.
% Due to space constraints, we present brief proof sketches of our main results; detailed proofs can be found in the appendix.

%\hadi{this paragraph can be removed to save space since most of it already exists under contributions.} Our paper is organized as follows. In Section \ref{sec:prelims}, we set up the preliminaries and notation for {\GHA}, including defining the classes of algorithms we will be studying. In Section \ref{sec:upper}, we start with a divide-and-conquer algorithm for trees, and later go on to use the parameter cutwidth to generalize this result for other classes of graphs, giving us all our upper bounds from Table \ref{table:summary1}. In Section \ref{sec:lower}, we give our lower bounds, showing reductions from bin-packing for several classes of graphs to give us all our lower bounds from Table \ref{table:summary1}. In Section \ref{sec:completebintrees}, we delve into complete binary trees in some detail, refuting the conjecture by \cite{canon} and then going on to show a constant approximation on complete binary trees. We finish with some concluding remarks and open directions in Section \ref{sec:conclusions}.



% \begin{table*}
%     \centering
%     \def\arraystretch{1.3}
%     \begin{tabular}{ |l||l|l|l|  }
%  \hline
%   \multicolumn{4}{|c|}{Approximations for Some Special Classes} \\
%  \hline
% \bf  Graph Class & \bf Upper Bound & \bf Lower Bound & \bf 
% Runtime \\
%  \hline
% Edge-weighted trees & $O(\Delta_w\log n)$  (Cor.~\ref{cor:weightedtrees}) & $\omega(n^{1 - \varepsilon})$ (Thm.~\ref{thm:trees-approx-lower-bound})
% & $O(n\log n)$ \\
% \hline
% Complete binary trees & $3$ (Thm.~\ref{thm:inorder}) & $\geq 1$ (exact solution \textcolor{red}{open}) & $O(n)$ \\
% \hline
% \end{tabular}
% \caption{Summary of our results for two specific graph classes. Here, $\Delta_w$ is the \emph{weighted} maximum degree, defined in Section \ref{sec:boundeddegreetrees}. Moreover, the upper and lower bounds match in the worst case. Note that we know the lower bound for complete binary trees has to be at least $1.67$ for value-agnostic algorithms.}
% \label{table:summary2}
% \end{table*}



\subsection{Other Related Work} 
Our work is very close to the large body of results on the computability of {\MLA}. While finding optimal linear arrangements is intractable in general \citep{mlabinaryhard}, there have been several papers presenting approximation algorithms for the problem \citep{richarao2005mla,feige2007mla, even200mla}, with the best known approximation ratio being $O(\sqrt{\log n}\log{\log n})$ \citep{feige2007mla}. Note that it is relatively straightforward to show that an $\alpha$-approximation algorithm for the \MLA~problem yields an $\alpha\phi$ approximation for the $\GHA$ problem where $\phi=\max_{1\leq i\leq n-1} (h_{i+1}-h_i)/\min_{1\leq i\leq n-1} (h_{i+1}-h_i)$.

Our problem also generalizes the classical problem of \textsc{Minimum Bisection}, which asks how to partition a graph $G$ into two almost equally-sized components with the smallest number of edges going across the cut. This problem is NP-complete \citep{mlahard} and it is also known to be inapproximable by an additive factor of $n^{2-\epsilon}$ \citep{bj92}. These lower bounds carry over to the \GHA{} problem as well, although the latter is strictly harder. For instance, \textsc{Minimum Bisection} is known to be solvable exactly in polynomial time for forests, but {\GHA} is NP-hard~\citep{canon}.

The canonical house allocation problem has also been well-studied in the literature. Recall that, in the canonical house allocation problem, agents are allowed to disagree on the values of the houses. In this setting, the existence and computational complexity of envy-free allocations on graphs have been reasonably well-studied \citep{beynier2018localenvy,eiben2020parameterized,bredereck2022envy}, with the problem, unsurprisingly, being computationally intractable in most settings. %Indeed, this is unsurprising given that the canonical house allocation problem is a generalization of the \GHA{} problem.
There have also been a few papers studying the complexity of minimizing various notions of envy when the underlying graph is {\em complete} \citep{gsvfairhouse, kamiyama2021envy, aigner2022envy,MMS23complexity}. 




\section{Model and Preliminaries}\label{sec:prelims}

%\subsection{Preliminaries from House Allocation}

We have a set of $n$ {\em agents} $V = [n]$ placed on the vertices of an undirected graph $G = (V, E)$. 
There are $n$ {\em houses}, each with a nonnegative \emph{value}, that need to be allocated to the agents. We represent the houses simply by the multiset of values $H = \{h_1, \ldots, h_n\}$, and assume WLOG that $h_1 \leq \ldots \leq h_n$. We will interchangeably talk about the house with value $h_i$ and the real number $h_i$. The pair $(G, H)$ defines an \emph{instance} of {\GHA}. %\hadi{we are sometimes referring to \textit{the} \GHA{} \textit{problem} and sometimes \GHA{} alone. we should be  consistent throughout the paper.}

An {\em allocation} $\pi: V \rightarrow H$ is a bijective mapping from agents (or nodes) to house values. Given an allocation $\pi$ and an edge $(i, j) \in E$, we define the {\em envy} along the edge $(i, j)$ as $|\pi(i) - \pi(j)|$. Our goal in {\GHA} is to compute an allocation $\pi^\ast$ that {\em minimizes} the total envy %\hadi{sometimes we say aggregate, let's be consistent} 
along all the edges of $G$:
\begin{align*}
    \Envy(\pi, G) := \sum_{(i, j) \in E} |\pi(i) - \pi(j)|.
\end{align*}

% The following definition, given by \cite{canon}, gives us a geometric way to visualize any allocation on an instance $(G, H)$ of {\GHA}.
We adopt the following definition from  \citet{canon} that provides a geometric representation to visualize allocations.

\begin{definition}[Valuation Interval]\label{def:valn_interval}
For an instance $(G, H)$ of {\GHA}, define the \emph{valuation interval} as the closed interval $\left[h_1, h_n\right] \subset \R_{\geq 0}$. For any allocation $\pi$, the envy along the edge $(i, j) \in E$ is exactly the length of the interval $[\pi(i), \pi(j)]$ (assuming $\pi(i) \leq \pi(j)$). We sometimes call the intervals $[h_i, h_{i+1}]$ for $1 \leq i \leq n - 1$ the \emph{smallest subintervals} of the valuation interval.
\end{definition}


 An optimal allocation $\pi^\ast$ would minimize the sum of the lengths of the intervals corresponding to each of its edges.
%
An allocation $\pi$ is \emph{$\alpha$-approximate} if $\Envy(\pi, G) \le \alpha\cdot\Envy(\pi^\ast, G)$. 




%In most cases of the {\GHA} problem, computing an optimal $\pi^\ast$ is intractable. Therefore, we focus on approximations. 
% An allocation $\pi$ is \emph{$\alpha$-approximate} if $\Envy(\pi, G) \le \alpha\cdot\Envy(\pi^\ast, G)$. 

Fix any arbitrary class $\cal G$ of graphs (we allow $\cal G$ to be a singleton set). We say an algorithm $\mathsf{ALG}_\mathcal{G}$ is \emph{defined} on $\cal G$ if $\mathsf{ALG}_\mathcal{G}$ is well-specified and outputs a valid allocation on every instance $(G, H)$ of {\GHA} with $G \in \mathcal{G}$. Such an algorithm $\mathsf{ALG}_\mathcal{G}$ is an \emph{$\alpha$-approximation} if for all instances $(G, H)$ of {\GHA} with $G \in \mathcal{G}$, $\mathsf{ALG}_{\mathcal{G}}$ always outputs an $\alpha$-approximate allocation. A $1$-approximation is an exact algorithm.

%We are now ready to formulate the following definition.

\begin{definition}[Value-Agnostic Algorithms]
\label{defn:valueagnostic}
An algorithm $\mathsf{ALG}_\mathcal{G}$ defined on a graph class $\mathcal{G}$ is \emph{value-agnostic} if on every input $(G, H)$ with $G \in \mathcal{G}$, $\mathsf{ALG}_\mathcal{G}$ returns the same allocation on all instances where the \emph{ordering} of house values is the same (in other words, the algorithm only requires the ordinal ranking and not the numerical values).
If the graph class $\cal G$ admits a value-agnostic $\alpha$-approximation algorithm, we say $\cal G$ is \emph{$\alpha$-value-agnostic}. Otherwise, it is \emph{$\alpha$-value-sensitive}.
\end{definition}

How can we re-frame existing results on {\GHA} in the light of Definition \ref{defn:valueagnostic}? \citet{canon} show that, unless P = NP, there is no $1$-approximation algorithm $\mathsf{ALG}_\mathcal{G}$ when $\cal G$ is the set of vertex-disjoint unions of paths, cycles, or stars. In contrast, they show that value-agnostic \emph{exact} algorithms exist when $\cal G$ is the set of paths, cycles, or stars.

Of course, value-agnostic $\alpha$-approximations are extremely powerful algorithms, as they can exploit the graph structure \emph{independently} of the values in the {\GHA} instance. As we would expect, value-agnostic $1$-approximations do not always exist, even on very simple graph classes and even if we allow for unlimited time. For instance, consider the graph consisting of the disjoint union of $K_2$ and $K_3$. Figure \ref{fig:value_agnostic_ex} shows that this graph does not admit an $\alpha$-value-agnostic algorithm for any finite $\alpha$.

% Figure environment removed

Although all our examples so far use the disconnectedness of the graphs to illustrate value-sensitivity, we will see in Section \ref{sec:completebintrees} that there are value-sensitive connected graphs as well.




%\subsection{Preliminaries from Structural Graph Theory}

For any graph $G = (V, E)$, and $S \subseteq V$, we denote by $\delta_G(S)$ the number of edges going across the cut $(S, V - S)$ in $G$. We will often estimate $\delta_G(S)$ for various subsets $S$. For $k \leq n-1$, we define $\delta_G(k) := \min_{|S| = k}\delta_G(S)$ as the size of the smallest cut in $G$ with $k$ vertices on one side. Of course, $\delta_G(k) = \delta_G(n - k)$ for all $k$. A $(k, n-k)$-cut in $G$ will be any cut $(S, V-S)$ with $|S| = k$.

We will use a few concepts from structural graph theory, most notably that of \emph{cutwidth}.

\begin{definition}\label{def:cutwidth}
    For a graph $G = (V, E)$ on $n$ vertices, let $\sigma = (v_1, \ldots, v_n)$ be any ordering of $V$. The \emph{width} of $\sigma$ is defined as
    \begin{equation*}
\mathsf{width}(\sigma, G):=        \max_{1 \leq \ell \leq n-1} \delta_G(\{v_1, \ldots, v_\ell\}).
    \end{equation*}
    The \emph{cutwidth} of $G$ is  the minimum width over all orderings of $G$, i.e.,
    \begin{equation*}
        \cw(G) := \min_{\sigma \in S_n}\mathsf{width}(\sigma, G)~.
    \end{equation*}
\end{definition}

The ordering $\sigma$ is often called a \emph{layout}. An \emph{optimal} layout is an ordering that achieves the cutwidth of $G$. The cutwidth is closely related to other standard notions of width used in structural graph theory. In particular, we have the following chain of inequalities (see \citet{korach1993cutwidth}):
\begin{align}
    &\tw(G)  \leq \pw(G) \leq \cw(G) \notag\\ &\qquad \leq O(\Delta\cdot\pw(G)) \leq O(\Delta\cdot\tw(G)\cdot\log n) 
    \label{eq:twpwcw}
\end{align}

Finding the exact cutwidth of $G$ in general is a difficult algorithmic problem. It can be computed exactly for trees (along with an optimal ordering) in time $O(n\log n)$ \citep{yannakakis1985treecutwidth}. However, even for planar graphs, the problem is NP-complete \citep{monien1988cutwidth}. 

If $G$ is sufficiently dense, there is a polynomial-time approximation scheme for the cutwidth \citep{frieze1996cutwidthdense}. In general, there is an efficient $O(\log^{1.5} n)$-approximation of the cutwidth known \citep{leighton1999cutwidthapprox}, which also returns a layout achieving this ratio. We will use this process as a subroutine several times in Section \ref{sec:upper}, for our upper bounds.
\section{Upper Bounds}\label{sec:upper}
%\hadi{I cut down and rewrote this section.. the original one is commented out.}\rik{I like the changes}

The hardness of achieving optimal envy even on simple classes of graphs (e.g.,~disjoint unions of paths) \citep{canon} immediately gives rise to the question of whether we can approximate optimal solutions. As stated before, we need to assume connectivity in general.

We start by making a trivial observation (Proposition \ref{prop:trivialgeneral}): \textit{any} allocation of values to a connected graph is an $O(n^2)$-approximation to the optimal envy, and in fact an $O(n)$-approximation when the graph is a tree. This is due to the fact that every smallest subinterval of the valuation interval is covered by at most $|E|$ edges, but connectivity requires that it be covered by at least one edge. 

\begin{proposition}\label{prop:trivialgeneral}
    For \emph{any} instance of {\GHA} on a connected graph $G = (V, E)$, \emph{any} allocation is an $|E|$-approximation to the optimal value.
\end{proposition}

In what follows, we first discuss how to improve this bound for bounded-degree trees and then generalize this result to graphs based on a structural parameter called the \emph{cutwidth}. Finally, we showcase how our bounds can be significantly improved for the special class of random (Erd\H{o}s-Renyi) graphs.


\begin{comment}

Several results in \citep{canon} show that the house allocation problem is a very hard one to solve in general, if we care about an exact solution. Finding the exact optimal envy is NP-hard even on particularly easy classes of graphs, such as disjoint unions of paths, cycles, or stars. 
% In fact, even if we contend ourselves with finding approximately optimal values, \citep{bj92} tells us that the problem in general is hard to approximate to within a factor of $n^{2 - \varepsilon}$ \vignesh{This seems to be an additive factor}. 
This immediately implies that we need to consider restricted classes of graphs to have any possibility of approaching interesting approximability results. In this section, we present the first few results for such classes of graphs.

At the outset, we make the following important trivial observation, stated as a proposition for the sake of convenience.

% \begin{proposition}\label{prop:trivialgeneral}
%     For \emph{any} instance of {\GHA} on a connected graph $G = (V, E)$, \emph{any} allocation whatsoever is an $n^2$-approximation to the optimal value.
% \end{proposition}

\begin{proposition}\label{prop:trivialgeneral}
    For \emph{any} instance of {\GHA} on a connected graph $G = (V, E)$, \emph{any} allocation whatsoever is an $|E|$-approximation to the optimal value.
\end{proposition}


The result follows simply by observing that in any allocation, every subinterval of the valuation interval is covered by at most $|E|$ edges, but connectivity requires that it be covered by at least one edge. Note that Proposition \ref{prop:trivialgeneral} immediately implies that every allocation is an $O(n^2)$-approximation to the optimal envy, and in fact an $O(n)$-approximation when the graph is a tree.
% The second result is exactly similar, but $|E| = n - 1$. We omit the details of both proofs, as they are trivial.

In what follows, we first present a result on bounded-degree trees, and then generalize this to a class of results on graphs based on a structural parameter called the \emph{cutwidth}.

\end{comment}

% \vignesh{Should we label the following section as warm up maybe?}
\subsection{Trees}\label{sec:boundeddegreetrees}

In this section, we will present a recursive polynomial-time $O(\Delta\log n)$-approximation algorithm for any instance of {\GHA} where the underlying graph is any tree with maximum degree $\Delta$. Thus, for any tree with maximum degree $\Delta = o(n/\log n)$, our algorithm provides a better approximation than \Cref{prop:trivialgeneral}.

We will use the following folklore fact\footnote{ For a proof of this fact, see, for instance, \citet{mlatrees}, who attributes this as a folklore result to \citet{seidvasser}, who claims the fact is well-known, but proves it anyway.} without a proof.

\begin{fact}[Folklore]\label{fact:folklore}
    Every $n$-vertex tree $T$ has a \emph{center of gravity}: i.e., a vertex $v$ such that all connected components of $T - v$ have at most $n/2$ vertices. This vertex $v$ can be found in $O(n)$ time.
\end{fact}

\begin{algorithm}[t]
    \caption{Recursive Algorithm $\mathsf{TrickleDown}(T, H)$}
    \hspace{\algorithmicindent} 
    \textbf{Input:} {A {\GHA} instance on a tree $T$ and a set of values $H = \{h_1, \ldots, h_n\}$.} \\
    % \hspace*{-\algorithmicindent} 
    \textbf{Output:} {An $O(\Delta\log n)$-approximate allocation.}
    \begin{algorithmic}[1]
        \If{$|T| = 1$}
        \State \textbf{Allocate} the only house to the only vertex.
        \Comment{Base case}
        \Else
        \State Find a center of gravity $v$ of $T$.
        \State Let $T - v = T_1 + \ldots + T_k$, with $|T_i| = n_i$. 
        \Comment{$k \leq \Delta$, $n_i \leq n/2$.}
        \State Partition $H$ into the following contiguous sets: %\hadi{we may say instead: Partition $H$ such that $H_1 = \{h_1, \ldots, h_{n_1}\}$, ... }
        % \begin{align*}
        %     H_1 &= \{h_1, \ldots, h_{n_1}\}, \\
        %     H_2 &= \{h_{n_1+1}, \ldots, h_{n_1 + n_2}\}, \\
        %     \vdots \\
        %     H_k &= \{h_{n_1 + \ldots + n_{k-1} + 1}, \ldots, h_{n_1 + \ldots + n_k}\}.
        % \end{align*}
        \begin{align*}
            H_1 &= \{h_1, \ldots, h_{n_1}\}, \\
            H_2 &= \{h_{n_1+1}, \ldots, h_{n_1 + n_2}\} \\
            &\vdots \\
            H_k &= \{h_{n_1 + \ldots + n_{k-1} + 1}, \ldots, h_{n_1 + \ldots + n_k}\}. 
        \end{align*}
        \State \textbf{Allocate} $h_n$ to vertex $v$.
        \For{$i \in \{1, \ldots, k\}$}
            \State Recursively call $\mathsf{TrickleDown}(T_i, H_i)$.
        \EndFor
        \EndIf
    \State \Return the resulting allocation.
    \end{algorithmic}
    \label{alg:treelogn}
\end{algorithm}


%\hadi{we need a filler here to show how Fact 3.2 is related to our result and explain a bit of our algorithm (high level). Something like...}

We will use Fact \ref{fact:folklore} in developing a recursive algorithm (Algorithm~\ref{alg:treelogn}) that obtains an $O(\Delta\log n)$-approximation on trees. In each call, the algorithm first finds a center of gravity of the tree and subsequently uses this vertex to identify disjoint subtrees and solve the subproblems recursively on disjoint subintervals of the valuation interval. 


% The following theorem serves as the analysis for Algorithm \ref{alg:treelogn}.



\begin{restatable}{theorem}{thmtreelogn}\label{thm:treelogn}
There is an $O(n\log n)$-time algorithm that, given any instance on a tree with maximum degree $\Delta$, returns an allocation whose envy is at most $\Delta\log n$ times the optimal envy.
%Algorithm \ref{alg:treelogn} runs in time $\tilde{O}(n)$, and returns an $O(\Delta\log n)$-approximation to the optimal envy.
\end{restatable}
\begin{proof}

% Figure environment removed

We will show that Algorithm \ref{alg:treelogn} provides the desired guarantee. The algorithm starts by locating the center of gravity $v$ of the given tree $T$ (which is guaranteed to exist by Fact \ref{fact:folklore}). Then it assigns the largest (i.e., rightmost) value to node $v$, and recursively constructs the assignment for each of the disjoint subtrees in $T - v$.
      %Note that, by Fact \ref{fact:folklore}, line 4 in Algorithm \ref{alg:treelogn} can always be done, and therefore it is well-specified.
      
      It is easiest to visualize the allocation resulting from Algorithm \ref{alg:treelogn} as in Figure \ref{fig:treelogn}. All recursive calls in line 9 occur in a single ``level'' of the figure, and all subsequent recursive calls from the subtrees $T_1, \ldots, T_k$ can also be packed into a single level, as the edges in $T_i$ and the edges in $T_j$ do not overlap, for any $i \neq j$. The crucial point is that the envy incurred strictly within disjoint subtrees $T_i$ and $T_j$ cannot involve the same smallest subintervals of the original instance.

      Let us analyze the total envy in the final allocation that is output by Algorithm \ref{alg:treelogn}. There are at most $\Delta$ edges adjacent to $v$, and each of them incur their envy in level $1$ of Figure \ref{fig:treelogn}. Each edge gets an envy of at most $(h_n - h_1)$, and therefore, the total envy on these edges is at most $\Delta\cdot(h_n - h_1)$. The total envy along the edges adjacent to $v_1, \ldots, v_k$ (except the ones accounted for in the levels above) are at most $\Delta\cdot(h_{n_1} - h_1), \ldots, \Delta\cdot(h_{n_1+\ldots+n_k} - h_{n_1+\ldots+n_{k-1}+1})$. Since the subintervals are all disjoint, this level accounts for an envy of at most $\Delta\cdot(h_n - h_1)$ as well. We can continue this argument through the lower levels.

      How many levels are there? Because each vertex picked at each recursive call is a center of gravity of the next subtree, the size of each subtree is at most half the size of the subtree at its parent level. The number of levels, therefore, is at most $\log n$. This gives us a total envy of $\Delta\cdot(h_n - h_1)\cdot\log n$.

      Note that the optimum envy has to be at least $h_n - h_1$ for any connected graph. This gives us an approximation ratio of $\Delta\log n$.

      The bound on the running time also arises from Fact \ref{fact:folklore}, which ensures that line 4 can be done in time $O(n)$. For each subtree $T_i$ in level $1$, we can find a center of gravity in time $O(n_i)$, so the total amount of work done to find the centers of gravity at this level is $O(n_1) + \ldots + O(n_k) = O(n)$. Since this is summed over $\log n$ recursive levels, the total running time is $O(n\log n)$.
\end{proof}

We remark that with a slightly more careful analysis,\footnote{  Technically this involves tweaking the algorithm such that the center of gravity is assigned slightly differently in line 7, and the partition of $H$ is consistent with this.} we can improve the approximation ratio to $(1/2)\cdot(1 + \Delta +\Delta\log n)$. In particular, for any instance on a binary tree, the optimal envy can be $(2\log n)$-approximated in $O(n\log n)$ time. % The following observation readily follows from \Cref{thm:treelogn}. \andrew{personally, i'd skip this corollary. but if we keep it, note that $\Delta=3$ for a binary tree so while a $2\log n$ approximation is possible, you'd only get $3\log n$ from the above theorem.}

%\begin{corollary}\label{cor:bintrees}
%    For any instance of {\GHA} on a binary tree, the optimal envy can be $(2\log n)$-approximated $O(n\log n)$ time. 
%\end{corollary}

\if 0
We remark here that Algorithm \ref{alg:treelogn} and Theorem \ref{thm:treelogn} apply to an even more general problem, the \emph{weighted} version of {\GHA}.
This version in general asks: given an edge-weighted graph $G = (V, E, w)$, where $w : E \to \R_{\geq 0}$ is a weight function on the edges, and a set of $n$ values $H$, how do we find an allocation $\pi : V \to H$ minimizing the total \emph{weighted} envy, i.e., $\sum_{(i, j) \in E}|\pi(i) - \pi(j)|\cdot w(i, j)$? Of course, this problem is at least as hard as {\GHA}. There is some known work for weighted variants of {\MLA} \citep{richarao2005mla}.

Clearly, Algorithm \ref{alg:treelogn} is well-defined on weighted instances $(T, H)$, where $T = (V, E, w)$ is an edge-weighted tree. WLOG assume each weight is at least $1$ (otherwise, scale all the weights appropriately, at the cost of introducing this scaling factor in the subsequent result). Define the maximum degree of the weighted tree $T$ as $\Delta_w(T) := \max_{v \in V}\sum_{u \in \text{Nbd}_T(v)}w(u, v)$, i.e., the maximum weight of edges leaving any vertex in $T$. Of course, if all weights were $1$, this is the same as the maximum degree.
%
Thus, we have the following corollary for weighted variant of the \GHA{} on trees.
% The following corollary to Theorem \ref{thm:treelogn} shows us that we can approximate the optimal envy on this instance.

\begin{corollary}\label{cor:weightedtrees}
    There is an $O(n\log n)$ time algorithm that returns an $O(\Delta_w(T)\cdot\log n)$-approximation to the optimal envy on $(T, H)$ for \emph{weighted} trees $T$.
\end{corollary}

\begin{proof}
    Because we ignore the weights, the proof of termination and running time remain unchanged.  To check the approximation, note that the total envy in level $1$ is incurred by the edges incident on the vertex $v$, which has a total maximum weight of $\Delta_w(T)$, and so the total envy accounted for in this level is at most $\Delta_w(T)\cdot(h_n - h_1)$. Because each set of weighted edges in the subsequent levels remain disjoint, and span disjoint subintervals of $H$, the rest of the argument carries through. The optimum envy is at least $h_n - h_1$ under the assumption that each weight is at least $1$, and this gives us the result.
\end{proof}

\fi



\subsection{Cutwidth}\label{sec:cutwidth}

In this section, we generalize the result from Section \ref{sec:boundeddegreetrees} using the structural graph theoretic property of cutwidth (Definition \ref{def:cutwidth}). This will enable us to have a black-box process to obtain envy approximations parameterized by the cutwidth. All of these algorithms will be value-agnostic.

\begin{restatable}{theorem}{thmcwgeneric}\label{thm:cwgeneric}
Let $(G, H)$ be a {\GHA} instance defined on a connected graph $G$. Given a layout $\sigma$ that $\beta$-approximates $\cw(G)$, we can efficiently construct an allocation $\pi$ that is a $(\beta \cdot \cw(G))$-approximation to the optimal envy.
\end{restatable}
\begin{proof}
We construct the allocation $\pi$ as follows: for each agent $i \in V$, if $\sigma(i) = j$, we set $\pi(i) = h_j$; that is, we give $i$ the $j$-th least-valued house. 
Since $G$ is connected, the total envy of any allocation is at least $(h_n - h_1)$. In the allocation $\pi$, the number of edges of $G$ spanning any smallest subinterval of the valuation interval is at most $\mathsf{width}(\sigma, G)$ by definition, and therefore the envy from $\pi$ is at most $\sum_{i = 1}^{n-1}\mathsf{width}(\sigma, G)\cdot(h_{i+1} - h_i) = \mathsf{width}(\sigma, G)\cdot(h_n - h_1)$. Hence, if $\sigma$ is an $\beta$-approximation for the cutwidth, then $\pi$ is an $\beta \cdot \cw(G)$-approximation to the optimal envy of $G$, as claimed.
\end{proof}
% \begin{proof}[Proof Sketch]
%     For each agent $i \in V$, if $\sigma(i) = j$, we give $i$ the $j$-th least-valued house. The number of edges spanning any smallest subinterval of the valuation interval is $\mathsf{width}(\sigma, G) \leq \beta\cdot\cw(G)$. So our total envy is at most $\beta\cdot\cw(G)\cdot(h_n - h_1)$, but the minimum envy is at least $(h_n - h_1)$.
% \end{proof}


The next corollary follows from Theorem \ref{thm:cwgeneric} and Equation \ref{eq:twpwcw} when combined with existing bounds on the cutwidth, treewidth, or pathwidth \citep{korach1993cutwidth, leighton1999cutwidthapprox, djidjev2006cutwidth} of certain graph families along with the best known approximation results of these quantities \citep{yannakakis1985treecutwidth,leighton1999cutwidthapprox}.

\begin{corollary}\label{cor:cutwidth-upperbounds}
There exist polynomial-time value-agnostic approximation algorithms for the following classes:
\begin{enumerate}[(i)]
    \item An $O(\Delta \log n)$-approximation algorithm on trees,\label{cor:cwtrees}
    \item An $O(\sqrt{n \Delta} \log^{1.5} n)$-approximation algorithm on planar graphs,\label{cor:cwplanar}
    \item An $ O(\tw(G) \cdot \Delta \log^{2.5} n)$-approximation algorithm on general connected graphs.\label{cor:cwgeneral}
\end{enumerate}
%\begin{proof}
%We first compute the exact or approximate cutwidth layout and then use Theorem \ref{thm:cwgeneric} to construct allocations from the computed cutwidth layout. To analyze the approximation guarantee for a class of graphs, it suffices to upper bound the cutwidth for the specific class of graphs. 
%
%The first result uses the fact that there is an efficient algorithm to compute the exact cutwidth for trees. Combining this with the fact that the cutwidth of a tree is $O(\Delta \log n)$ \citep{korach1993cutwidth} using Theorem \ref{thm:cwgeneric}, we get the required result.
%
%The second and third result uses the $O(\log^{1.5}n)$ approximation algorithm for cutwidth \citep{leighton1999cutwidthapprox}. The second result combines this approximation algorithm with the fact that for any planar graph $G$, $\cw(G) \le O(\sqrt{n\Delta})$ \citep{djidjev2006cutwidth}. The third result directly plugs in \eqref{eq:twpwcw} to get the required result.
%\end{proof}
\end{corollary}

Note that for each class of graphs listed above $\Delta$ can be $O(n)$ in the worst case, and for general connected graphs, $\tw(G)$ can be $O(n)$ in the worst case as well. So, in the worst case, the first and third results are asymptotically worse than the trivial bound given by \Cref{prop:trivialgeneral}. However, for many natural subclasses of these graphs, such as bounded-degree graphs and bounded-degree trees, \Cref{cor:cutwidth-upperbounds} yields strictly better approximation guarantees.


\subsection{Random Graphs}\label{subsec:random-graphs}
We next consider random graphs, specifically Erd\H{o}s-Renyi graphs, where $G\sim {\mathcal G}_{n,1/2}$ denotes a random graph on $n$ nodes where every edge is present with probability $1/2$ and all edges are independent. We show that \GHA{} on such graphs can be approximated up to a factor $1+o(1)$ regardless of the valuation interval. The central observation is that for any subset of nodes $S$, $\delta_G(S)$ is tightly concentrated around $|S|(n-|S|)/2$. 
% The proof is relegated to Appendix \ref{apdx:upper}.
\begin{restatable}{lemma}{randomlemma}\label{lem:random}
    For $G\sim {\mathcal G}_{n,1/2}$, \[\Pr \left [ \forall S\subseteq V, (1-\epsilon) \leq \frac{\delta_G(S)}{|S|(n-|S|)/2} \leq (1+\epsilon) \right ] \geq 1- \exp(-\Omega(\epsilon^2 n)) \ , \]
for any $\epsilon \geq \sqrt{24 \ln (n)/n}$.

% \vignesh{I think you mean \[\Pr \left [ \forall A\subseteq V~,~ (1-\epsilon) \leq \frac{\delta_G(A)}{|A|(n-|A|)/2} \leq (1+\epsilon) \right ] \ge 1 - \exp(-\Omega(\epsilon^2 n)) \ , \]
% for any $\epsilon=O(\sqrt{\ln (n)/n})$.}
\end{restatable}
\begin{proof}
The expected size of the cut $\delta_G(S)$ is $E[\delta_G(S)]=|S|(n-|S|)/2$. By applying the Chernoff bound, we obtain:
\[\Pr[|\delta_G(S)-E[\delta_G(S)]|\geq \epsilon E[\delta_G(S)]] \leq  2 \exp( -\epsilon^2 E[\delta_G(S)]/3) \ . \]
Hence, by the union bound the probability there exists a set $S$ of size $k \le n/2$ such that $|\delta_G(S)-E[\delta_G(S)]|\geq \epsilon E[\delta_G(S)]$ is at most
\begin{align*}    
2\exp( - \epsilon^2 k(n-k)/6) \binom{n}{k}
& \leq 2 \exp( - \epsilon^2 kn/12 + k \ln n) \\
%& \leq & 2 \exp( - \epsilon^2 kn/12 + k \ln n) \\ 
& \leq 2 \exp( - \epsilon^2 kn/24),
\end{align*}
  assuming $\epsilon \geq \sqrt{24 \ln (n)/n}$. 
 % \vignesh{Should this be $\epsilon = O(\sqrt{\ln (n)/n})$?}
% Hence, the probability there exists two subsets of nodes of the same size where the number of edges leaving the subsets differs by more than a factor $(1+\epsilon)/(1-\epsilon) = 1+O(\epsilon)$
The lemma then follows by taking the union bound over $k$ and noting that \[\sum_{k=1}^{n/2} 2 \exp( - \epsilon^2 kn/24)=\exp(-\Omega(\epsilon^2 n)) \ .\qedhere \]
\end{proof}

Lemma \ref{lem:random}, with $\epsilon=\sqrt{24 \ln(n)/n},$ implies that with high probability, the cost of the optimum solution is at least \[\sum_{i=1}^{n-1} (h_{i+1}-h_{i}) \delta_G(i)\geq \sum_{i=1}^{n-1} (h_{i+1}-h_{i})(1-\epsilon) i(n-i)/2, \]
whereas the cost of an arbitrary allocation is at most  \[\sum_{i=1}^{n-1} (h_{i+1}-h_{i})(1+\epsilon) i(n-i)/2  . \]
Therefore an arbitrary allocation is a $(1+\epsilon)/(1-\epsilon)= 1+ O(\sqrt{\ln (n)/n})$-approximation.

\begin{theorem}\label{thm:random}
For  $G \sim \cal G_{n, 1/2}$, any allocation is a $1+ O(\sqrt{\ln (n)/n})$ approximation with probability at least $1 - 1/\poly(n)$.
\end{theorem}


\section{Lower Bounds}\label{sec:lower}

%All the approximation guarantees presented in the previous section may appear weak at first.
Every algorithm presented in Section \ref{sec:upper} is value-agnostic. It might seem reasonable to assume, therefore, that there are more powerful approximation schemes that exploit the numerical values in $H$ in some way. Indeed, our results on random graphs suggest that, for most graphs, we can do significantly better. Remarkably, we show in this section that this is \emph{not} the case, and our value-agnostic algorithms are strong enough to give us nearly optimal approximation guarantees. Specifically, we show inapproximability results matching our upper bounds (up to $\polylog$ factors) for every class of graphs considered. Our lower bounds will use reductions from the {\UTP} problem.

\begin{definition}[\TP]
Given a multiset of $3m$ naturals $A = \{a_1, \dots, a_{3m}\} \subseteq \mathbb{N}_{> 0}$ and a natural $T \in \mathbb{N}_{> 0}$ such that $\sum_{j \in [3m]} a_j = mT$, {\TP} asks whether $A$ can be partitioned into $m$ triplets $(S_1, S_2, \dots, S_m)$ such that the sum of each triplet is equal to $T$.
\end{definition}

The \TP problem is NP-complete even when all the inputs are given in unary and each item in $A$ is strictly between $\ffrac{T}{4}$ and $\ffrac{T}{2}$ \citep{garey1979computers}. We refer to this variant as {\UTP}. Note that {\UTP} is just a reformulation of \textsc{Bin Packing}: there are $3m$ integers that sum to $mT$, and we wish to fit these integers into $m$ bins each of capacity $T$. The condition of three integers in each bin is redundant, as it is implied by the constraint that each integer is strictly between $T/4$ and $T/2$.

% We present our lower bounds in increasing order of proof complexity, starting with trees and moving to general graphs, planar graphs, bounded-degree graphs, and finally, bounded-degree trees.
Some of our results and proofs in this section (specifically Theorems \ref{thm:trees-approx-lower-bound} and \ref{thm:bounded-degree-planar-approx-lower-bound}) are very similar to results about the inapproximability of the {\em balanced graph partition} problem \citep{feldmann2012gridpartition, feldmann2015treepartition}. The rest of our proofs use novel gadgets and techniques.
% We relegate all proofs in this section to the appendix but provide basic sketches and ideas wherever possible.

\subsection{Trees and Planar Graphs}\label{sec:treelowerbound}
Recall that we presented two approximation guarantees for trees, $O(n)$ (Proposition \ref{prop:trivialgeneral}) and $O(\Delta \log n)$ (Corollary \ref{cor:cutwidth-upperbounds}). Both of these results are $\tilde{O}(n)$ in the worst case. 

\begin{restatable}{theorem}{thmtreesapproxlowerbound}\label{thm:trees-approx-lower-bound}
For any constant $\varepsilon > 0$, there is no efficient $O(n^{1-\varepsilon})$ approximation algorithm for {\GHA} on depth-$2$ trees unless P = NP.
\end{restatable}
% \begin{proof}[Proof Sketch]
% Figure environment removed

% Given a \UTP instance, we construct a graph according to Figure \ref{fig:trees-reduction-tree} where $C$ is some positive integer we will decide later. The multiset of house values consists of $CT$ houses with value $j$ for each $j \in [m]$, and one house with value $0$.

% If there is a valid $3$-partition, we can construct an allocation with envy at most $3m^2$. If there is no $3$-partition, any allocation must have envy at least $C$. We can now set $C$ appropriately.
% \end{proof}

\begin{proof}
We give a reduction from \UTP. For some constant $\varepsilon > 0$, assume there is an efficient $O(n^{1- \varepsilon})$ approximation algorithm $\mathsf{ALG}_{\cal G}$ where $\cal G$ corresponds to the class of depth-$2$ trees. In other words, there is a constant $\gamma$ such that for all instances $(G, H)$ with $G \in \mathcal{G}$, $\mathsf{ALG}_{\cal G}$ outputs an allocation with total envy within a multiplicative factor of $\gamma n^{1- \varepsilon}$ to the optimal envy.

Given an instance of \UTP, we construct an instance $(G, H)$ of {\GHA} as follows: The graph $G$ is a rooted depth-$2$ tree where the root $r$ of the tree has $3m$ children $\{x_1, \dots, x_{3m}\}$. Each of these nodes $x_i$ has $C a_i - 1$ children (see Figure \ref{fig:trees-reduction-tree}). Here, $C$ is a positive integer whose exact value we shall determine later.

The total number of nodes in $G$ is $1 + \sum_{i \in [3m]} Ca_i = 1 + CmT$, and so we must specify $1 + CmT$ house values in $H$. We define $H$ with $CT$ values of $j$ for $j \in [m]$, together with a single value of $0$. Note that this construction can be done in polynomial time as long as $C$ is polynomially large, since the input to the $3$-partition instance is given in unary. 

We show that, for an appropriate choice of $C$, the minimum total envy output by $\mathsf{ALG}_{\cal G}$ for the instance $(G, H)$ is at least $\left \lceil (12\gamma m^3 T + 1)^{\ffrac{1}{\varepsilon}} \right \rceil$ if and only if there exists {\em no} valid partition for the original \textsc{3-Partition} instance, i.e., the original instance was a NO instance. 


$(\Rightarrow)$ Assume there is a valid $3$-partition $(S_1, \ldots, S_m)$ for the original instance. We denote this $3$-partition using a mapping $\mu : A \to \{S_1, \ldots, S_m\}$ that maps each number in the multiset $A$ to one of the triplets $S_j$. We construct an allocation for the graph $G$ as follows: for any item $i$, if $\mu(i) = S_j$, we allocate houses with value $j$ to $x_i$ and all of its children. We finally allocate the house with value $0$ to the root. Note that this is a valid allocation: for any $j \in [m]$, we allocate exactly $\sum_{i \in [3m]: \mu(i) = S_j} Ca_i = CT$ houses of value $j$. 

We can easily upper bound the envy of this allocation; this upper bound also serves as an upper bound for the minimum total envy for the instance $(G, H)$. There is no envy between any $x_i$ and any of its children. So the only edges with potential envy are the ones incident on the root. There are $3m$ such edges, each incurring envy at most $m$. This gives us an upper bound of $3m^2$ on the total envy. This implies, when there is a valid $3$-partition, the approximation algorithm $\mathsf{ALG}_{\cal G}$ will output an allocation with envy at most $3m^2\gamma(1 + CmT)^{1-\varepsilon}$.

$(\Leftarrow)$ Assume there is no valid $3$-partition in the original instance. We will show that any allocation has a total envy of at least $C$. We do this by examining the houses allocated to the depth-1 nodes $\{x_1, \dots, x_{3m}\}$. If any $x_i$ is allocated a value of $0$ in some allocation $\pi$, then the allocation $\pi$ has a total envy of at least $C$ since any $x_i$ has at least $C -1$ children and $1$ parent receiving a value of at least $1$ each. 

So now assume no $x_i$ is allocated a value of $0$ in $\pi$. For notational convenience, assume WLOG that $x_1, \dots, x_{\ell_1}$ are allocated a value $1$, $x_{\ell_1 +1}, \dots, x_{\ell_1 + \ell_2}$ are allocated a value $2$ and so on. If for all $j \in [m]$, $\sum_{h \in [\ell_j]} a_{\ell_0 + \ell_1 + \dots + \ell_{j-1} + h} = T$ (with $\ell_0 = 0$), then we violate our assumption that there is no valid $3$-partition in the original instance. Therefore there exists one $j \in [m]$ such that $\sum_{h \in [\ell_j]}a_{\ell_0 + \ell_1 + \dots + \ell_{j-1} +h} > T$. Assume again for notational convenience that $j = 1$. Since all values are integers, we can restate the inequality above as $\sum_{h \in [\ell_1]}a_{h} \ge T + 1$. This implies $\sum_{h \in [\ell_1]} Ca_{h} \ge CT + C$. 

Coming back to the allocation $\pi$, we have that $\sum_{h \in [\ell_1]} Ca_{h} \ge CT + C$ implies that there are at least $C$ nodes out of all the children of $\{x_{1}, x_2, \dots, x_{\ell_1}\}$ which are not allocated a value of $1$. The envy that each of these $C$ nodes will have towards their parents is at least $1$. This implies that the total envy of allocation $\pi$ is at least $C$. 

We set $C = \left \lceil (12\gamma m^3 T + 1)^{\ffrac{1}{\varepsilon}} \right \rceil$ to complete the reduction. When there is no valid $3$-partition, the total minimum envy (and therefore, the envy output by $\mathsf{ALG}_{\cal G}$) is at least $C =\left \lceil (12\gamma m^3 T + 1)^{\ffrac{1}{\varepsilon}} \right \rceil$. However, when there is a valid $3$-partition, the envy output by $\mathsf{ALG}_{\cal G}$ is strictly upper bounded by:
\begin{align*}
    3m^2\gamma(CmT+1)^{1-\varepsilon} \le 6m^3 \gamma TC^{1-\varepsilon} \le 6m^3 \gamma T \left ( \left \lceil (12\gamma m^3 T + 1)^{\ffrac{1}{\varepsilon}} \right \rceil \right )^{1 - \varepsilon} < \left \lceil (12\gamma m^3 T + 1)^{\ffrac{1}{\varepsilon}} \right \rceil.
\end{align*}
This completes the proof.
\end{proof}



\subsection{General and Bounded-Degree Graphs}
In this section, we generalize the arguments from Section \ref{sec:treelowerbound} to other classes of graphs. 
The main technique is similar to that of Theorem \ref{thm:trees-approx-lower-bound}, so we just present ideas for the graph construction in each of these proofs, with the details in Appendix \ref{apdx:lower}.

We first match the $O(n^2)$ upper bound for connected graphs (\Cref{prop:trivialgeneral} and \Cref{cor:cutwidth-upperbounds}).

\begin{restatable}{theorem}{generalgraphs}\label{thm:general-approx-lower-bound}
For any constant $\varepsilon > 0$, there is no efficient $O(n^{2-\varepsilon})$ approximation algorithm for {\GHA} on connected graphs unless P = NP.
\end{restatable}
\begin{proof}[Proof Sketch]
We replace the $Ca_i$-sized stars in Figure \ref{fig:trees-reduction-tree} with $Ca_i$-sized cliques. The rest of the proof is similar to Theorem \ref{thm:trees-approx-lower-bound}.
\end{proof}



% \begin{proof}[Proof sketch]
%     Figure \ref{fig:general-reduction-graph} shows the graph we use for this reduction. The construction is similar to that in Theorem \ref{thm:trees-approx-lower-bound}, except that we use cliques of size $Ca_i$ attached to a common root $r$, instead of a vertices $x_i$ with $Ca_i - 1$ dangling leaves. The idea is to use the high density of the clique to show that the envy is $\Omega(C^2)$ when there is no valid $3$-partition. The house values $H$ are defined as in Theorem \ref{thm:trees-approx-lower-bound}. A similar argument works here too. A valid $3$-partition can be ``packed'' into the clusters of values, which would attain only envy from the edges incident to the root $r$. Conversely, if there is no valid $3$-partition, again using a packing argument, it can be shown that there are several high envy edges, giving rise to $\Omega(C^2)$ envy.
% \end{proof}





%\subsection{Bounded Degree Planar Graphs}

So far in our two lower bounds (Theorems \ref{thm:trees-approx-lower-bound} and \ref{thm:general-approx-lower-bound}), we were able to use simple counting techniques, because counting edges with non-zero envy in stars and cliques is straightforward. Our next results will require much more careful analysis.

We will start with bounded-degree planar graphs. Our reduction uses grid graphs instead of stars and cliques, and so we will need a technical lemma to help us with estimating the number of edges with nonzero envy.

\begin{restatable}{lemma}{gridlemma}\label{lem:grid-graph-property}
    Let $G = Grid(r, c)$ be a grid graph with $r$ rows and $c$ columns such that $r \le c$. Let $A \subseteq V$ be any set of nodes in this graph such that $|A| \le \ffrac{rc}{2}$. Then, $\delta_G(A) \geq \min\{\sqrt{|A|}, \ffrac{r}{2}\}$.
\end{restatable}
\begin{proof}
If $A$ consists of at least one node from each row, then since $|A| \le \ffrac{rc}{2}$, there will be at least $r/2$ rows with a node in $V \setminus A$. Therefore, there will be at least $\ffrac{r}{2}$ edges in the cut. Similarly, if $A$ consists of at least one node from each column, there will be $\ffrac{c}{2} \ge \ffrac{r}{2}$ edges in the cut. 

Otherwise, there is some row and some column containing only nodes in $V \setminus A$.
Note that there must either be at least $\sqrt{|A|}$ rows with a node in $A$ or at least $\sqrt{|A|}$ columns with a node in $A$. Assume WLOG there are at least $\sqrt{|A|}$ rows with a node in $A$. Each of these rows intersects the column that only has nodes from $V\setminus A$, and so each of the $\sqrt{|A|}$ rows must contain an edge between $A$ and $V\setminus A$.
\end{proof}



Armed with Lemma \ref{lem:grid-graph-property}, we can now present our lower bound on bounded-degree planar graphs.

\begin{restatable}{theorem}{bdp}\label{thm:bounded-degree-planar-approx-lower-bound}
For any constant $\varepsilon > 0$, no efficient $O(n^{0.5-\varepsilon})$ approximation algorithm exists for {\GHA} on bounded-degree planar graphs unless P = NP.
\end{restatable}
\begin{proof}[Proof Sketch]
We replace the stars of size $Ca_i$ in Figure \ref{fig:trees-reduction-tree} with grid graphs containing $C$ rows and $Ca_i$ columns. The rest of the proof flows similarly to Theorem \ref{thm:trees-approx-lower-bound}. Lemma \ref{lem:grid-graph-property} helps in estimating the envy blow-up if there is no $3$-partition.
\end{proof}



% \begin{proof}[Proof sketch]
%     Figure \ref{fig:bdp-reduction-graph} shows the graph we use for this reduction. Again, the construction is similar to before. The graph $G$ has $3m$ grids, each with $C$ rows and $Ca_i$ columns, attached by a single edge to a leaf of a ``small'' binary tree $T_r$. Note that this is a bounded-degree planar graph. The values $H$ are defined similarly to before, with a cluster of $C^2T$ values at each positive integer in $[3m]$, and $|T_r|$ values at $0$.

%     If there is a $3$-partition, the argument is exactly similar to those earlier in this section, with the only envy coming from the edges of the tree, incurring a total envy of $3m^2$.

%     Conversely, suppose there is no $3$-partition. Now, using Lemma \ref{lem:grid-graph-property}, we can show that a grid which does not have a majority of its vertices from the same cluster must incur at least $C/4$ envy. Otherwise, assuming each grid has a majority of its vertices in the same cluster, a packing argument shows that some grid must have a cut going across two different values, and this incurs at least $C/2$ envy, once again by Lemma \ref{lem:grid-graph-property}.
% \end{proof}

Note that Theorem \ref{thm:bounded-degree-planar-approx-lower-bound} matches the $O(\sqrt{n})$ upper bound from \Cref{cor:cutwidth-upperbounds}.

Our next lower bound applies to arbitrary bounded-degree graphs and matches the $O(n)$ upper bound from Proposition \ref{prop:trivialgeneral} and Corollary \ref{cor:cutwidth-upperbounds}. In this reduction, we use the recent polynomial-time algorithm \citep{cohen2016ramanujan} to compute bipartite Ramanujan multigraphs for any even number $m$ of vertices, and any degree $d \ge 3$. At a high level, we replace the star gadgets from the proof of Theorem \ref{thm:trees-approx-lower-bound} with these Ramanujan graphs and use the expansion properties of Ramanujan graphs to prove a lemma similar to (and stronger than) \Cref{lem:grid-graph-property}. 
% The analysis is quite involved and relegated to the appendix.

% One key element of the construction we will use is that we will take a gadget consisting of a $3$-regular Ramanujan bipartite multi-graph of a given even size (at least $6$), and remove any of its repeated edges until it is a simple graph. The crucial observation is that even though these new gadget graph is neither regular nor Ramanujan, it still has ``enough'' expansion for our purposes. We will need the following well-known result for this, stated without proof.

% \begin{lemma}[Cheeger's Inequality]\label{lem:cheegers-inequality}
% Let $G'$ be a $d$-regular Ramanujan (multi-)graph defined on a set of $V$ nodes. The following holds:
% \begin{align*}
%     \min_{S \subseteq V: 0 < |S| \le |V|/2} \frac{\delta_{G'}(S)}{|S|} \ge \frac12(d - 2\sqrt{d-1}),
% \end{align*}
% where $\delta_{G'}(S)$ denotes the number of edges in the $(S, V \setminus S)$ cut in the graph $G'$.
% \end{lemma}

% The following theorem characterizes the inapproximability on bounded-degree graphs. See Appendix \ref{apx:lower} for the full proof.


\begin{restatable}{theorem}{boundeddeg}\label{thm:bounded-degree-approx-lower-bound}
For any constant $\varepsilon > 0$, there is no efficient $O(n^{1-\varepsilon})$ approximation algorithm for {\GHA} on bounded-degree graphs unless P = NP.
\end{restatable}

% Figure environment removed

\begin{proof}[Proof Sketch]
    Figure \ref{fig:bounded-degree-reduction-graph} shows the graph we use for this reduction. The graph $G$ is constructed as follows. For each $a_i$ in the given {\UTP} instance, construct in polynomial time a $3$-regular Ramanujan bipartite multi-graph of size $Ca_i$ (using a result of \cite{cohen2016ramanujan}). Remove any repeated edges to convert them into simple graphs. The resulting graphs can be shown to have sufficient expansion properties, using Cheeger's inequality (Lemma \ref{lem:cheegers-inequality-apdx}). We can now attach these graphs to the leaves of a sufficiently small binary tree. This is a bounded-degree graph.

    If there is a $3$-partition, we can exhibit a small-envy allocation in the same way as in the other proofs in this section.
    % Conversely, if there is no valid $3$-partition, we can once again show that if some gadget has no majority value, then it spans many different values, and by its expansion properties, it has many edges in a cut across these different values, incurring $\Omega(C)$ envy. Otherwise, if each gadget has a majority value, we can use the packing argument once more to show that some gadget has to have a cut going across different values, incurring $\Omega(C)$ envy once again.
    Conversely, if there is no valid $3$-partition, we can show that some of these gadgets are going to be allocated multiple different values. We then use the expansion properties of these gadgets to show that the number of high envy edges within each of these gadgets is $\Omega(C)$.
\end{proof}


% \begin{proof}[Proof sketch]
%     Figure \ref{fig:bounded-degree-reduction-graph} shows the graph we use for this reduction. The graph $G$ is constructed as follows. For each $a_i$ in the given {\UTP} instance, construct in polynomial time a $3$-regular Ramanujan bipartite multi-graph of size $Ca_i$ (using a construction by \cite{cohen2016ramanujan}). Remove any repeated edges to convert them into simple graphs. The resulting graphs can be shown to have sufficient amount of expansion properties, using Cheeger's inequality (Lemma \ref{lem:cheegers-inequality}). We can now attach these graphs to the leaves of a sufficiently small binary tree. This is a bounded-degree graph.

%     If there is a $3$-partition, we can exhibit a small-envy allocation in the same way as in most of the other proofs in this section.

%     % Conversely, if there is no valid $3$-partition, we can once again show that if some gadget has no majority value, then it spans many different values, and by its expansion properties, it has many edges in a cut across these different values, incurring $\Omega(C)$ envy. Otherwise, if each gadget has a majority value, we can use the packing argument once more to show that some gadget has to have a cut going across different values, incurring $\Omega(C)$ envy once again.
%     Conversely, if there is no valid $3$-partition, we can show that some of these gadgets are going to be allocated multiple different values. We then use the expansion properties of these gadgets to show that the number of high envy edges within each of these gadgets is $\Omega(C)$.
% \end{proof}

\subsection{Bounded-Degree Trees}
Our final lower bound shows that {\GHA} is NP-hard even when the underlying graph is a bounded degree tree. We still use \UTP in our reduction but this proof is significantly different from the previous ones. Our reduction will use a gadget we call the {\em flower}.\footnote{To the best of our knowledge, our specific flower graph is novel but it is possible (likely even) that the term ``flower'' has appeared before in the graph theory literature.}

\begin{definition}
The flower $F(n, k)$ is a rooted tree with $n$ nodes and maximum degree $k+1$, defined recursively as follows: for any $k \geq 1$, $F(1, k)$ is simply an isolated vertex which is the root node. For $n > 1$, $F(n, k)$ consists of a root node connected to the root nodes of $d$ other flowers $F(n_1, k), \dots, F(n_d, k)$ such that 
\begin{enumerate}[(a)]
    \item $\sum_{i = 1}^d n_i = n-1$, 
    \item if $n-1 \ge k$, then $d = k$ if $n$ and $k$ have different parities, and $d = k-1$ otherwise,
    \item each $n_i$ is odd, 
    \item for any $i, j \in [d]$, $|n_i - n_j| \le 2$.
\end{enumerate}
To ensure consistency with floral terminology, we refer to the root node of the flower $F(n, k)$ as its {\em pistil} and the (recursively smaller) flowers $F(n_1, k), \dots, F(n_d, k)$ as its {\em petals}.
\end{definition}

Before we use flowers, we show that they are well-defined and efficiently constructible. 

\begin{restatable}{lemma}{lemflowercomputation}\label{lem:flower-computation}
For any $n \ge 1$ and $k \ge 3$, the flower $F(n, k)$ exists and can be constructed in $\text{poly}(n, k)$ time. 
\end{restatable}
\begin{proof}
To prove existence, we only need to show that the numbers $n_1, \dots, n_d$ are guaranteed to exist. If $n -1 < k$, this is trivial: each $n_i = 1$.

Otherwise, if $n$ is even, then $d$ is required to be an odd value via condition (b). What we need to do is write $n-1$ as the sum of $d$ odd numbers. We set each $n_i$ as the greatest odd number which is at most $\lfloor \frac{n-1}{d} \rfloor$. Once we do this, $n-1 - \sum_{i = 1}^d n_i$ is guaranteed to be a non-negative even number which is strictly less than $2d$. So, we increment some of the $n_i$'s by $2$ till property (a) is satisfied. This construction satisfies the other two properties as well. 
% We can use a similar construction for the case where $n$ is odd. 

The above argument not only shows that the numbers $n_1, \dots, n_d$ are guaranteed to exist but also presents a way to compute them in polynomial time. Since each $n_i$ is strictly less than $n$, we can use this subroutine to compute $F(n, k)$ recursively in polynomial time. More formally, let $T(n, k)$ be the complexity of constructing the flower $F(n, k)$. We have

\begin{align*}
    T(n, k) &= \left (\sum_{i = 1}^d T(n_i, k) \right ) + O(d) \\
    &\le k\cdot T((n-1)/(k-1) + 2, k) + O(k) \\
    &\le k\cdot T((n/k + 2, k) + O(k).
\end{align*}
If $n/k < 2$, then $T(n/k + 2, k)$ can be trivially constructed in $O(1)$ time since $n/k + 2 -1 \le 3 \le k$ and the graph is a constant-sized star. If $n/k \ge 2$, we can simplify the above expression as follows:
\begin{align*}
    T(n, k) \le k\cdot T(2n/k, k) + O(k),
\end{align*}
which simplifies to $T(n, k) = O(n^\frac{\ln 3}{\ln{3} - \ln2})$.
%To solve this we go back our undergraduate algorithms class to recall the Master theorem and simplify the above expression even further to $T(n, k) = O(n^\frac{\ln k}{\ln{k} - \ln2})$. This value decreases as $k$ increases. Since $k \ge 3$, we can upper bound it as $T(n, k) = O(n^\frac{\ln 3}{\ln{3} - \ln2})$
\end{proof}


The reason we build flowers is because they satisfy the two following useful properties.
\begin{restatable}{lemma}{lemflowerproperties}\label{lem:flower-properties}
Let $F(n, k)$ be a flower on the set of vertices $N$, and suppose $n \ge 10k$, and $n$ and $k$ have different parities. Then, $F(n, k)$ satisfies the following properties:
\begin{enumerate}[(i)]
    \item For any $A \subseteq N$ such that $|A|$ is even and $A$ does not contain the pistil, $\delta(A) \ge 2$.
    \item Each petal of $F(n, k)$ has size in the interval $\left [  \frac{4n}{5k}, \frac{6n}{5k} \right]$.
\end{enumerate}
\end{restatable}
\begin{proof}
(i) follows from property (c) in the definition of a flower. 
(ii) follows from the fact that there are $k$ petals (property (b)) and each petal has size in the interval $\left [ \frac{n}{k}-2, \frac{n}{k} + 2\right ]$ (property (d)).
\end{proof}




These simple properties are all we need to show the hardness of {\GHA} on bounded-degree trees.
\begin{restatable}{theorem}{thmboundeddegreetreesnp}\label{thm:bounded-degree-trees-np-complete}
{\GHA} is NP-hard on bounded-degree trees.   
\end{restatable}

% Flowers
% Figure environment removed

% \begin{proof}[Proof Sketch]
% Given a \UTP instance, we construct a graph according to Figure \ref{fig:boundedtrees-reduction-graph}; the shaded circles correspond to pistils and the white triangular blocks correspond to petals. The house values are defined as follows: we have $4m+1$ unique values such that the gaps between these values are exponentially decreasing. That is, the gap between the least and the second least value is significantly larger than the gap between the second least and the third least value and so on. For each unique value, there are $T$ houses with that value in the multiset $H$, with the exception of the largest value which has enough houses (with that value) to ensure the total size of the multiset $H$ is equal to the number of nodes in the graph.

% We can show that in any optimal allocation, the first $3m$ clusters must be allocated to flowers of the form $F(T, 99)$. The next $m$ clusters must be allocated in a way that creates a $3$-partition to minimize envy. That is, each of these values must be allocated to three flowers of the form $F(a_i, 99)$ such that the total size of these three flowers sums up to $T$. If it is not possible to do this, the minimum envy of the allocation is strictly higher. This allows us to separate instances with a valid \TP.
% \end{proof}

\begin{proof}
We present a reduction to {\UTP}. We assume the input $3$-partition instance is scaled up such that each $a_i$ is even and at least $1000$. This comes with no loss of generality --- we can multiply each $a_i$ and $T$ by a $1000$ without changing the output of the instance. We also assume $T$ is even. If $T$ is odd, the instance is trivially a NO instance.

We construct the graph of the {\GHA} instance as follows (see Figure \ref{fig:boundedtrees-reduction-graph}). We start with a path of $3m$ flowers $F(T, 99)$, connected by their pistils. We number these flowers $1$ to $3m$ from left to right. To each flower $i$, we connect a flower $F(a_i, 99)$ and to this flower, we connect two flowers $F(10T, 999)$. Again, connections between flowers are made via an edge between the pistils of the flowers. We refer to the flowers of the form $F(a_i, 99)$ as {\em small} flowers, flowers of the form $F(T, 99)$ as {\em medium} flowers and flowers of the form $F(10T, 999)$ as {\em large} flowers. We define small, medium and large pistils similarly. Note that large flowers have $999$ petals, while small and medium flowers have $99$ petals (since each $a_i$ and $T$ can be assumed to be even). This graph can be constructed in polynomial time using Lemma \ref{lem:flower-computation} and the fact that the inputs are given in unary.



Now we describe the house values. We need to describe $64mT$ house values since this is the number of nodes in the graph. For this, it is easier to first define the following function/series, $s(j) = (|E|+1)^{2j}$ for any $j \in \mathbb{N}$ where $|E|$ is the number of edges in our constructed graph. Note that equivalently, we can write $s(j) = (64mT)^{2j}$. We define the multiset of house values as the following:
\begin{align*}
    T &\text{ houses with value } 0 \\
    T &\text{ houses with value } s(4m) \\
    T &\text{ houses with value } s(4m) + s(4m-1) \\
    \dots \\
    T &\text{ houses with value } \sum_{j = 2}^{4m} s(j) \\
    60mT &\text{ houses with value } \sum_{j = 1}^{4m} s(j)
\end{align*}
The crucial property these values satisfy is that the intervals between two values are exponentially decreasing in size, so much so that minimum envy allocation must lexicographically minimize the number of edges passing through these gaps. That is, to minimize envy, we need to first minimize the number of lines passing through the $(0, s(4m))$ interval, subject to that minimize the number of lines passing through the $(s(4m), s(4m) + s(4m-1))$ interval, and so on. For ease of readability, we no longer refer to the exact value of the houses but simply refer to them as {\em clusters} of values. More specifically, the $T$ instances of $0$ are the \emph{first} cluster of values, the $T$ instances of $s(4m)$ are the \emph{second} cluster, and so on. There are $4m+1$ clusters, each of which have size $T$ except for the highest value (the $(4m+1)$-th cluster) which has size $60mT$. The largest house value is at most $(4m + 1)\cdot s(4m)$, which requires $O(\log(4m + 1) + (4m)\log(|E| + 1))$ bits to write, and there are still only polynomially many such values, and so this multiset of house values can be written in polynomial time and space.

Given this constructed instance, let us study the optimal allocation $\pi^{*}$. Note that the optimal allocation must first minimize the number of envy lines between the first two clusters. Therefore, the placement of the first cluster must be done in such a way so as to minimize the cut size between the first cluster and the rest of the graph. We will show that, to minimize envy, the first $3m$ clusters must be allocated to the $3m$ medium flowers, and the next $m$ clusters (from the $(3m+1)$-th to the $(4m)$-th cluster) must each be allocated to exactly three small flowers with sizes $a_i, a_j$ and $a_k$ such that $a_i + a_j + a_k = T$. This is possible if and only if the original instance is a YES instance, and the total envy of the corresponding house allocation is given by
\begin{equation*}
    \envy_{\text{YES}} = \sum_{j = 1}^{3m - 1}(j + 1)\cdot s(4m+1-j) + (3m)\cdot s(m+1)  + \sum_{j = 1}^m(3m + 3j)\cdot s(m+1-j) .
\end{equation*}
Any other allocation that does not follow this structure has a strictly greater envy. Specifically, this means that when the original instance is a NO instance, the optimal allocation has a greater total envy than $\envy_{\text{YES}}$ defined above.

To do this, we will need some bounds on petal sizes. The size of each petal in the medium flowers is at most $\frac{6T}{99 \times 5} \le \frac{T}{80}$ (Lemma \ref{lem:flower-properties}). Similarly the size of each petal in the large flowers is $\le \frac{T}{80}$ and the size of each petal in a small flower is $\frac{6a_i}{99 \times 5} \le \frac{6T}{2 \times 99 \times 5} \le \frac{T}{160}$. In particular, all petals throughout the graph have size at most $T/80$, a fact that we shall use several times.

Let us now consider where the first cluster is allocated in any optimal allocation. Note that in an allocation where the first cluster completely fills up one of the medium flowers at either end of the path (either leftmost or rightmost), the cut size across the first interval is $2$. Any optimal allocation therefore needs to ensure that this cut size is at most $2$. This observation will enable us to rule out many other possibilities.

If the elements of the first cluster are not allocated to any pistil, then the cut size is at least $80$ since each petal in the graph has size at most $\frac{T}{80}$; so the first cluster must be allocated to at least $80$ petals, each of which must add at least one edge to the cut. So in an optimal allocation, at least one pistil must get an element from the first cluster.

Suppose a large pistil is given a value in the first cluster. Then, since each petal of the large flower has size at least $T/125$, there must be at least $999-125 \ge 800$ petals of the large flower that remain unfilled by the cluster of values. Each of these unfilled petals adds at least one edge between the first cluster to another cluster, and these edges are all disjoint, and so the cut size must be at least $800$. Any such allocation, therefore, is suboptimal.

Suppose a small pistil is given a value in the first cluster. This pistil has two neighboring large pistils. We know from the arguments before that these neighbors cannot be in the first cluster, and so these two outgoing edges are across the cut. But then, the entire graph without these two large flowers is still connected, so at least one more edge needs to go across the first cut, and therefore, this allocation is also suboptimal.

It follows that any optimal allocation must allocate only medium pistils to the first cluster. If multiple such pistils are allocated, say from medium flowers $F_1$ and $F_2$, note that the cluster cannot contain either flower in its entirety, so there is at least one edge from each of them across the cut; but there also must be at least one other edge, from a path that goes from any of these two pistils to any node in the graph placed in a different cluster (such a node must exist, just by counting). This is therefore suboptimal. Similarly, if the first cluster contains exactly one pistil from a medium flower $F_1$, and no other pistil, but it does not contain all of $F_1$, it is easy to see that the cut will have size at least $3$, and so will be suboptimal.

%If on the other hand, the first cluster contains completely fills up one of the $T$ sized flowers, the cut size is $2$, if the $T$ sized flower picked is one of the extreme ones (leftmost or rightmost). This is optimal. Any other allocation which incompletely fills a $T$-sized flower, or fills any other $T$-sized flower is strictly worse. An allocation that exclusively fills up small flowers \rik{I thought the casework at this point was for pistils of flowers, no? The remaining case is to argue that a small pistil cannot be in the first cluster.} are also worse since each $a_i$ is in the interval $\left [ \frac{T}{4}, \frac{T}{2} \right ]$ and so multiple such $a_i$'s must be filled up and each $a_i$ has an edge to two pistils of flowers of size $10T$ --- who we have already shown is a bad idea to allocate to.

So, assume from here on out that the first cluster is allocated entirely to the leftmost (or the first) medium flower under $\pi^*$. Our goal is to show that the first $3m$ clusters must be allocated to the $3m$ medium flowers from left to right. We do this by induction.

Assume the first $k$ clusters are allocated to the first $k$ medium flowers (from the left). Note that the cut size or the total number of lines of envy between the first $k$ clusters and the rest of the graph is $k+1$. Out of all the ways of allocating the rest of the items, we wish to find the one that minimizes the number of lines of envy going through the next interval, between the $(k + 1)$-th cluster and the $(k + 2)$-th one. We will show that in order to achieve this, the $(k+1)$-th cluster must be allocated to the $(k+1)$-th medium flower. If we allocate the $(k+1)$-th cluster to the $(k+1)$-th medium flower, the cut size between the first $k+1$ clusters and the rest of the graph increases by $1$ with respect to the first $k$ clusters i.e. it increases from $k+1$ to $k+2$. The increase in cut size is just an easier way to account for the number of edges between the first $k+1$ clusters and the rest of the graph. We show that all other allocations of the $(k+1)$-th cluster are strictly suboptimal i.e. all other allocations increase the cut size by at least $2$.

If the $(k+1)$-th cluster is not allocated to any pistil, then the cut size increases by at least $80$. If the $(k+1)$-th cluster is allocated to a large pistil, then the cut size increases by at least $800$. Both of these statements can be proved using similar arguments to the first cluster. 

For every small pistil that gets a value from the $(k+1)$-th cluster (regardless of whether it is attached to a medium flower that has already been assigned a cluster or not), the cut size increases by at least $1$ --- this is because each of these pistils has $2$ edges to large pistils. So even if the edge connecting the small pistil to the medium pistil is removed from the cut, at least two new edges are added. This argument rules out allocating the $(k+1)$-th cluster to multiple small flowers. If, on the other hand, no medium pistil is allocated and only one small pistil is allocated, at least $T/2$ values from the cluster must be allocated to petals of flowers whose pistil remains unallocated; this comes from the fact that each small flower has size upper bounded by $T/2$, and so the remaining $T/2$ values must be allocated to petals. Since every petal of any kind of flower has size at most $T/80$, we need at least $40$ different petals to be represented among the remaining values, and each of them has to add a distinct edge across the cut, and so the cut size must increase by at least $40$.

The only cases we have left are ones which involve at least one medium pistil. If multiple of these medium pistils are allocated, then since each of their petals has size at least $T/125$, there can be at most $125$ such petals represented in the cluster, and so we must have at least $198 - 125 \ge 50$ unfinished petals, each of which adds a distinct edge across the cut, increasing the cut size by at least $1$ each.

So exactly one medium pistil must be allocated to this cluster. We could still have a combination of one medium pistil and at least one small pistil, but this increases the cut size by at least $2$ since each of the small pistils increase the cut size by at least $1$ and the medium flower must be unfinished which adds at least another edge to the cut. 

Finally, we have the case where exactly one medium pistil and no other pistil is allocated. It is easy to see, using arguments similar to before, that the strictly optimal place to put this medium pistil is the pistil corresponding to the $(k+1)$-th medium flower and the strictly optimal allocation is to fill up the cluster entirely with the $(k+1)$-th medium flower.

We can conclude that the first $3m$ clusters must be allocated entirely to the medium flowers in order from left to right (or right to left). 
Note that we have not said anything yet to diffentiate instances with a valid $3$-partition. We will do that next.

The $(3m +1)$-th cluster still must be allocated to some pistil, but cannot be allocated to a large pistil, by the arguments from before. Note that all the medium flowers have already been allocated values from the first $3m$ clusters. Therefore, the $(3m + 1)$-th cluster must contain at least one pistil, and all pistils in it must be small. If the $(3m+1)$-th cluster fills up exactly $3$ different small flowers, then the cut size increases by exactly $3$ relative to the first $3m$ clusters. This is because $6$ edges from the small pistils to the large pistils are added, but the $3$ edges from the medium pistils to the small pistils are taken away. We will show that this is the best option, and any other allocation is strictly worse. To show this, we must consider the following four possible cases.

\noindent\textbf{Case 1: The $(3m+1)$-th cluster is allocated to exactly $1$ pistil.} This must be the pistil of a small flower, which we know has at most $T/2$ vertices by assumption. Therefore, at least $T/2$ values in the cluster must be allocated to petals of other flowers. Each of these petals adds at least one edge to the cut since their corresponding pistils have not been allocated values in the first $3m+1$ clusters, and these edges are all distinct. We know each petal in the graph has a size of at most $T/80$, so at least $40$ petals must be represented in this cluster, so the cut size increases by at least $40$, which is strictly suboptimal.

\noindent\textbf{Case 2: The $(3m+1)$-th cluster is allocated to exactly $2$ pistils.} Both these pistils must correspond to small flowers (say $F_1$ and $F_2$). Both these flowers have a combined size of strictly less than $T$, so at least one other flower outside these two is allocated a value from the $(3m+1)$-th cluster. If either $F_1$ or $F_2$ are not completely filled, then the cut size increases by at least $4$ --- two extra edges come from the pistils of $F_1$ and $F_2$ to the large pistils, the third is from at least one of $F_1$ or $F_2$ being unfinished, and the fourth is from the third flower which is allocated a value (note that it cannot be completely contained in this cluster, because its pistil is in a different cluster).

If both $F_1$ and $F_2$ are completely filled, then the argument is a little more subtle. Recall that each small pistil adds a $1$ to the cut size from the edges to the large pistils. We need to show that at least $2$ other edges are added to the cut from the values allocated outside $F_1$ and $F_2$. The number of values from the $(3m+1)$-th cluster allocated outside of $F_1$ and $F_2$ is {\em even}, since both $F_1$ and $F_2$ have even size by assumption. (Also, this number of values is nonzero, as $F_1$ and $F_2$ have a combined size of less than $T$ by assumption.) Therefore, if these values are allocated to two different flowers, at least $2$ more edges are added to the cut (since those two flowers have their pistils in other clusters) and we are done. Otherwise, if they are allocated to the same flower, the cut size still increases by at least $2$ because of Lemma \ref{lem:flower-properties} and we are done.

\noindent\textbf{Case 3: The $(3m+1)$-th cluster is allocated to exactly $3$ pistils but at least one of the flowers is incomplete.}
Here the cut size increase is trivially at least $4$ --- an increase of $3$ from the pistils having edges to the large pistils, and a fourth from the fact that at least one of the flowers is incomplete.

\noindent\textbf{Case 4: The $(3m+1)$-th cluster is allocated to $4$ or more pistils.}
It is a straightforward argument to show that the number of edges added across the cut is $4$ or more in this case.


Similarly, for the $(3m+2)$-th to the $(4m)$-th cluster, the best possible option is to allocate each cluster in a way that fills up $3$ small flowers. If any of these clusters fail to do so, the cut size is strictly higher. Given allocations of the first $4m$ clusters, there is only one possible allocation for the $(4m+1)$-th cluster; that is, the allocation where values from the $(4m+1)$-th cluster to all nodes which have not been allocated a value from the first $4m$ clusters.

It is easy to see that it is possible for all the $(3m+1)$-th to $(4m)$-th clusters to fill up three small flowers each if and only if there is a valid $3$-partition in the original instance. More specifically, if each of the $(3m+1)$-th to $(4m)$-th clusters are allocated in a way that attains a lower bound on the number of edges between the $(3m+k)$-th and the $(3m+k+1)$-th interval, this allocation defines a valid $3$-partition. This is because each of the $(3m+1)$-th to $(4m)$-th cluster must be allocated to exactly three small flowers and must fill them up entirely. It is also easy to see that given a valid $3$-partition, we can construct an allocation that attains the ideal total envy.

To compute the threshold, we proceed as follows. For $1 \leq i \leq 4m$, let $\ell_i$ be the length of the valuation interval between the $i$-th cluster and the $(i + 1)$-th cluster, which is precisely $s(4m + 1 - i)$. If the original {\UTP} instance is a YES instance, then by our arguments above, we can incur an envy of at most
\begin{equation*}
    \envy_{\text{YES}} := \sum_{j = 1}^{3m - 1}(j + 1)\cdot \ell_j + (3m)\cdot \ell_{3m} + \sum_{j = 1}^m(3m + 3j)\cdot\ell_{3m + j}.
\end{equation*}
Here, the interval $\ell_{3m}$ has increased the cut size by $0$, because it represents the rightmost medium flower being placed in the $3m$-th cluster. Note that the expression above can be computed in polynomial time, since each $\ell_j = s(4m + 1 - i)$ is representable in $O((4m + 1 - i)\log(|E|))$ bits, and all arithmetic operations can be done in polynomial space and time in this number of bits. There are only polynomially many operations to do, so we can compute the threshold $\envy_{\text{YES}}$ in polynomial time.

From all our analysis above, if the original {\UTP} instance is a YES instance, we can attain an envy of at most $\envy_{\text{YES}}$ on our constructed instance. On the other hand, if the original {\UTP} instance is a NO instance, we know any allocation is lexicographically worse than this expression. By our choice of the function $s(\cdot)$, this envy must be strictly more than $\envy_{\text{YES}}$. It follows that our {\GHA} instance has an allocation with envy at most $\envy_{\text{YES}}$ if and only if the {\UTP} instance is a YES instance. Since the construction is done entirely in polynomial time, this finishes the proof.
%any other allocation that does not have this structure has strictly worse total envy. Therefore, any instance with no valid $3$ partition has strictly worse envy. Since we know exactly how much the cut size increases for each of the first $4m$ clusters, we know exactly how many lines pass through each interval in the minimum envy allocation when there is a $3$ partition and when there is no $3$ partition. We can easily compute a threshold using these values and separate the instances with a valid $3$-partition. 
\end{proof}







\section{The Curious Case of Complete Binary Trees}\label{sec:completebintrees}

%\hadi{I think what we study here are technically called ``Perfect Binary Trees'' as we assumed all leaves are at the same level. The standard definition of Complete Binary Tree allows for the last level to be incomplete.}

In this section, we investigate {\GHA} on instances where the underlying graph is a complete binary tree $B_k$. Recall that such a tree has depth $k$, and $2^{k+1} - 1$ vertices in total, of which $2^k$ are leaves. All leaves, furthermore, are at the same depth.

In \citet[Theorem 4.11]{canon}, it was shown that for any binary tree (complete or otherwise), at least one optimal allocation satisfies the \emph{local median property}: the value at every internal node is the median among the values given to that node and its two children. The same authors surmised that, for any binary tree, at least one optimal allocation satisfies the stronger \emph{global median property}: for every internal node $v$, either its left subtree gets strictly lower-valued houses and its right subtree gets strictly higher-valued houses, or the other way round. Note that if true, this would lead to a straightforward recursive polynomial-time algorithm that would compute an optimal allocation on (nearly) balanced binary trees. %Note that a complete binary tree has only one global median allocation (up to reordering the subtrees at any of the internal nodes), and so the conjecture would imply a nearly linear-time value-agnostic algorithm for complete binary trees.
%\andrew{Would the running time really be $O(2^d)$? It seems that if the conjecture was true, you should just sort the elements in $H$ and place them on the leaves appropriately. But sorting is $O(n\log n)$} (see chat)

We now give a refutation of this conjecture. We illustrate an instance on a complete binary tree of depth $3$, in which no optimal allocation satisfies the global median property. This is a quite surprising result that shows that the general problem on complete binary trees may be much harder than expected.

\begin{example}\label{ex:globalrefutation}
    Consider the instance $(B_3, H)$, where
    \begin{equation*}
        H = \{0,0,0,0,0,0,0,1,1,1, 2,3,3,3,3\}.
    \end{equation*}
    \begin{comment}
    % Figure environment removed

% % Figure environment removed
\end{comment}
    See Figure \ref{fig:global-min-arg}. The top shows the only allocation satisfying the global median property (up to re-ordering). The total non-negligible envy incurred by this assignment comes out of the thick red edges of the $B_3$, which incur a total envy of $6$. However, the bottom shows an allocation with an envy of $5$ (incurred by the thick red edges), showing that the global median is strictly sub-optimal.
    % % Figure environment removed
\end{example}



    % Figure environment removed

% \begin{table*}[th]
%     \centering
%     \begin{tabular}{ |c|c|c|c|c|c|c|c|c|c|c|c|c|c|c|c|c|c|c|c|c| } 
%  \hline
%  $m$ & 1 & 2 & 3 & 4 & 5 & 6 & 7 & 8 & 9 & 10 & 11 & 12 & 13 & 14 & 15 & 16 & 17 & 18 & 19 & 20 \\
%  \hline
%  $\mathsf{elegance}(m)$ & 1 & 2 & 1 & 2 & 3 & 2 & 1 & 2 & 3 & 2 & 3 & 2 & 3 & 2 & 1 & 2 & 3 & 2 & 3 & 4 \\
%  \hline
% \end{tabular}
%     \caption{List of $\mathsf{elegance}(m)$ for $1 \leq m \leq 20$.}
%     \label{tab:repunit}
% \end{table*}


Fix an arbitrary instance of {\GHA} on the complete binary tree $B_k$ on $n = 2^{k+1} - 1$ vertices, and consider the valuation interval. There are $n$ values on the interval. Of particular interest to us is the size of the \emph{smallest} $(i, n - i)$-cut, i.e., $\delta_{B_k}(i)$. Since $\delta_{B_k}(i) = \delta_{B_k}(n - i)$, we can WLOG take $i \leq \lceil n/2\rceil$. We now need a definition.

\begin{definition}[Repunit Representation and Elegance]\label{def:repunitrepresentation}
    For any $m \geq 1$, let a \emph{repunit representation of $m$} be any finite sequence $(a_1, \ldots, a_r) \in \mathbb{Z}^r$ satisfying
    \begin{equation*}
        m = \sum_{i =1}^r {\mathsf{sgn}(a_i)} \cdot (2^{|a_i|} - 1)
    \end{equation*}
    where $\mathsf{sgn}(a_i)$ is $1$ (resp.~$-1$) if $a_i \geq 0$ (resp.~$a_i < 0$). Note that every $m \geq 1$ has a repunit representation (e.g.,~the length-$m$ sequence of all ones). We define $\mathsf{elegance}(m)$ as the smallest $r$ for which $m$ has a repunit representation $(a_1, \ldots, a_r)$ of length $r$.
\end{definition}

The intuition behind Definition \ref{def:repunitrepresentation} is to capture the most ``efficient'' way to write $m$ in binary as the sum or difference of binary repunits, i.e., numbers of the form $11\ldots 1$. For instance, $\mathsf{elegance}(10) = 2$, because $10 = (2^3 - 1) + (2^2 - 1)$, and there is no shorter repunit representation. Similarly, $\mathsf{elegance}(12) = 2$, as $12 = (2^4 - 1) - (2^2 - 1)$. Note that $12$ cannot be written as the \emph{sum} of two repunits. Table \ref{tab:repunit} summarizes the elegance of all numbers up to $20$.

\begin{table*}[h!]
    \centering
    \begin{tabular}{ |c|c|c|c|c|c|c|c|c|c|c|c|c|c|c|c|c|c|c|c|c| } 
 \hline
 $m$ & 1 & 2 & 3 & 4 & 5 & 6 & 7 & 8 & 9 & 10 & 11 & 12 & 13 & 14 & 15 & 16 & 17 & 18 & 19 & 20 \\
 \hline
 $\mathsf{elegance}(m)$ & 1 & 2 & 1 & 2 & 3 & 2 & 1 & 2 & 3 & 2 & 3 & 2 & 3 & 2 & 1 & 2 & 3 & 2 & 3 & 4 \\
 \hline
\end{tabular}
    \caption{List of $\mathsf{elegance}(m)$ for $1 \leq m \leq 20$.}
    \label{tab:repunit}
\end{table*}

% \begin{center}


The following proposition relates elegance to the size of the smallest $(i,n-i)$-cut in a complete binary tree, namely $\delta_{B_k}(i)$.


\begin{restatable}{proposition}{elegancecuts}\label{prop:elegance}
    Let $B_k$ be the complete binary tree on $n = 2^{k + 1} - 1$ vertices. Then for $i \leq 2^k - 1$, $\mathsf{elegance}(i) - 1 \leq \delta_{B_k}(i) \leq \mathsf{elegance}(i)$.
\end{restatable}
\begin{proof}
    Consider \emph{any} $(i, n - i)$-cut in $B_k$, say $(S, V \setminus S)$, with $|S| = i$, and suppose there are $m'$ edges going across the cut. We will construct a repunit representation of $i$ with at most $m' + 1$ terms. Suppose the tree $B_k$ is rooted at the node $r$, and direct each edge from parent node to child node. Initialize a sequence $\vec{v}$ to be empty, and take any edge $e$ going across the cut $(S, V \setminus S)$. Observe that $e$ is either directed from $S$ to $V \setminus S$ or the other way round. This edge must have a complete binary subtree $B_{k_1}$ on one side in $B_k$: specifically, the subtree rooted at the child of $e$. If the edge $e$ is directed from $V \setminus S$ to $S$, then append the term $k_1 + 1$ to $\vec{v}$ (this corresponds to adding a repunit), and if $e$ is directed from $S$ to $V \setminus S$, append the term $-(k_1 + 1)$ (corresponding to subtracting a repunit). Once this is done for all edges $e$ going across the cut, we end up with a finite sequence $\vec{v}$. It is now easy to check that either this sequence $\vec{v}$, or the sequence $\vec{v}$ appended with the term $k + 1$, is a valid repunit representation of $i$. It follows that if the cut had been the minimum one, we would have a repunit representation of $i$ with at most $\delta_{B_k}(i) + 1$ terms. Therefore, $\mathsf{elegance}(i) \leq \delta_{B_k}(i) + 1$.

    Conversely, consider any optimal valid repunit representation of $i$ (which, therefore, has $\mathsf{elegance}(i)$ terms). Note that we can assume WLOG that all the terms are distinct in absolute value. This is because adding and subtracting the same term gives us a suboptimal representation, whereas $(2^{k_1} - 1) + (2^{k_1} - 1) = 2^{k_1 + 1} - 1 - 1$, so we can replace two additive repunits of the same length by two other unequal-length repunits without changing the result. We claim that we can assume WLOG that the largest repunit in this representation is at most $2^k - 1$. Otherwise, if it is $2^{k + 1} - 1$ or larger, then note that the next most significant repunit needs to be $-2^k + 1$, as otherwise the distinctness assumption gives us $i \geq (2^{k+1} - 1) - \sum_{j = 1}^{k-1}(2^j - 1) \geq 2^k$, contradiction. We can replace these two terms using $(2^{k+1} - 1) - (2^k - 1) = (2^k - 1) + 1$, which would replace these two terms by two other repunits without changing the value. This would be a valid representation, with all distinct terms unless the original representation also had a $+1$ term in it. We could then replace the $+1 + 1$ by $+3 - 1$, which would again be valid, unless the original had a $+3$ term in it. We could then replace the $+1 + 1 + 3$ by $+7 - 3 + 1$, which would again be valid, unless the original had a $+7$ term in it. We can keep going this way. What is the largest additive term in the original that we can run into in this way? Note that if we get to an additive term of $2^{k-1} - 1$, then we would have $i \geq (2^{k+1} - 1) - (2^k - 1) + (2^{k-1} - 1) - \sum_{j = 1}^{k-2}(2^j - 1) \geq 2^k$, contradiction. So the largest (additive) term can only be $2^{k-2} - 1$, and we would then terminate.
    
    So now we have an optimal repunit representation of $i$ with all distinct terms in absolute value, with the largest term being at most $2^k - 1$. We now claim that this repunit representation gives rise to an $(i, n - i)$ cut in $B_k$. We can just take complete binary subtrees of the sizes determined by the terms of the repunit representation, and include or exclude them on one side of the cut (according to the sign of the relevant term). The edges going across the cut will exactly be the edges to the roots of these subtrees, and the number of these edges will be exactly the number of terms in the original representation. This is \emph{some} $(i, n - i)$-cut of $B_k$, and so the smallest one has at most as many edges going across it as this one. It follows that $\delta_{B_k}(i) \leq \mathsf{elegance}(i)$.
\end{proof}
% \begin{proof}[Proof Sketch]
%     For each edge going across a cut in $B_k$, one of its endpoints is the root of a binary subtree, and it contributes a term in a repunit representation (possibly along with an extra additive term). Conversely, any repunit representation gives rise to a cut. Therefore, cuts correspond to repunit representations up to a single additive term. Minimizing both sides yields the result.
% \end{proof}

We note that if $i \ll n$, then in fact $\delta_{B_k}(i) = \mathsf{elegance}(i)$. Therefore, $\mathsf{elegance}(i)$ actually characterizes the size of the minimum $(i, n - i)$ cut in any sufficiently large binary tree.

Consider a value-agnostic algorithm for complete binary trees. Such an algorithm would need to assign the house values in any instance in some fixed order $(v_1, \ldots, v_n)$ to the vertices of $B_k$. %Certainly, if we could find some ordering where for each $1 \leq i \leq n - 1$, we had $\delta_{B_k}(\{v_1, \ldots, v_i\}) = \mathsf{elegance}(i)$, then by Proposition \ref{prop:elegance} we would have an optimal allocation.
The following proposition shows that doing this cannot simultaneously achieve the optimal cut on all smallest subintervals, and this leads to a lower bound on the approximability.

\begin{restatable}{proposition}{binarylowerbound}\label{prop:traversalimpossible}
    There is no value-agnostic algorithm for complete binary trees that attains an approximation better than $(5/3) \approx 1.67$.
\end{restatable}
\begin{proof}
    Take a large enough binary tree $B_k$ with $n \gg 100$ vertices, and consider the numbers 89 and 94. Note that $\mathsf{elegance}(89) = 3$, as $89 = 127 - 31 - 7$, and $\mathsf{elegance}(94) = 2$, as $94 = 63 + 31$. Furthermore, these are unique minimum-length repunit representations. We claim that no layout $\sigma = (v_1, \ldots, v_n)$ would attain $\delta_{B_k}(\{v_1, \ldots, v_{89}\}) = 3$ and $\delta_{B_k}(\{v_1, \ldots, v_{94}\}) = 2$ simultaneously. Indeed, if a layout $\sigma$ satisfies the first condition, then the root of a subtree with $127$ must receive one of the lowest $89$ values. However, if $\sigma$ also satisfies the second condition, then the $94$ lowest values fill up exactly two complete binary subtrees of size $63$ and $31$, and so the root of any subtree of size $127$ could not have have any of these values. This is a contradiction, and therefore, any value-agnostic algorithm for this complete binary tree needs to choose at most one of these two options. However, now consider two instances, $(B_k, H_1)$ and $(B_k, H_2)$, where $H_1$ consists of $89$ values of $0$ and $n - 89$ values of $1$, whereas $H_2$ consists of $94$ values of $0$ and $n - 94$ values of $1$. The optimal envy on $(B_k, H_1)$ is $\mathsf{elegance}(89) = 3$, whereas the optimal envy on $(B_k, H_2)$ is $\mathsf{elegance}(94) = 2$. A value-agnostic algorithm will yield a sub-optimal result on at least one of these two instances. If it is wrong on $(B_k, H_1)$, it has to have at least $5$ edges spanning the only nontrivial smallest subinterval (since it needs an odd number crossing the cut $\delta_{B_k}(\{v_1, \ldots, v_{89}\})$, as any repunit representation of $89$ needs an odd number of terms, by parity) and it will be off by a factor of at least $5/3 \approx 1.67$ on this instance. If it is wrong on $(B_k, H_2)$, it has to have at least $4$ edges crossing the only nontrivial smallest subinterval (since it needs an even number crossing the cut $\delta_{B_k}(\{v_1, \ldots, v_{94}\})$, again by parity) and it will be off by a factor of at least $4/2 = 2$ on this instance. Therefore, the approximation ratio has to be at least $1.67$.
\end{proof}
% \begin{proof}[Proof Sketch]
%     In a sufficiently large complete binary tree, consider the smallest cut separating 89 vertices and the smallest cut separating 94 vertices. Using $\mathsf{elegance}(89) = 3$, as $89 = 127 - 31 - 7$, and $\mathsf{elegance}(94) = 2$, as $94 = 63 + 31$, we can argue that no layout $\sigma = (v_1, \ldots, v_n)$ attains $\delta_{B_k}(\{v_1, \ldots, v_{89}\}) = 3$ and $\delta_{B_k}(\{v_1, \ldots, v_{94}\}) = 2$ simultaneously, and so a value-agnostic algorithm will yield a sub-optimal result on at least one of these two instances.
% \end{proof}


%The proof of Proposition \ref{prop:traversalimpossible} also shows that we cannot attain an approximation ratio better than $3/2$ for any value-agnostic algorithm on complete binary trees. We can tweak our examples slightly to observe that the bounds for $\delta(177) = \delta(127 + 63 - 15 + 3 - 1) = 5$ and the conditions for $183 = 127 + 63 - 7$ cannot be satisfied simultaneously by any layout $\sigma$. this result also shows that there is no value-agnostic algorithm for $B_k$ that can achieve an approximation ratio better than $5/3$ \rik{Check this with Vignesh.} \vignesh{You can use the same example you have above.}.
%
%
%Proposition \ref{prop:traversalimpossible} and the nonexistence of the global median property (as discussed in Example \ref{ex:globalrefutation}) suggest that it may not be reasonable to be optimistic about an algorithm with \emph{any} constant approximation ratio, even for complete binary trees. 
The counterexample in Proposition \ref{prop:traversalimpossible} and the failure of the global median property (Example \ref{ex:globalrefutation}) may seem to suggest that, even for complete binary trees, \emph{any} constant approximation ratio is unattainable. 
%
Remarkably, the following result shows that this is not the case: there is a \emph{value-agnostic} algorithm attaining a constant approximation on any complete binary tree. Indeed, ordering the vertices of $B_k$ in the standard in-order traversal and allocating the (sorted) values in that order yields a $3.5$-approximation.

\begin{restatable}{theorem}{inorder}\label{thm:inorder}
    Let $B_k$ be the complete binary tree on $n = 2^{k+1} - 1$ vertices. Then, on any house allocation instance on $B_k$, assigning the houses in increasing order to the vertices of $B_k$ in the standard in-order traversal gives us a total envy at most $3.5$ times the optimal value.
\end{restatable}
\begin{proof}
    Suppose we allocate the houses in sorted order to the vertices of $B_k$ in the standard in-order traversal. For any $i \leq 2^k - 1$, consider the number of edges of $B_k$ spanning the subinterval $(h_i, h_{i+1})$ of the valuation interval. It can be shown that under the in-order traversal, the number of edges spanning this interval is exactly $\mathsf{runs}(i)$, the number of runs of contiguous $0$s or $1$s in the binary representation of $i$, by a simple argument\footnote{ For instance, the two quantities follow the same recurrence relation: $f(2^k + i) = f(2^k - i + 1) + 1$ for $k \geq 0$ and $0 < i \leq 2^k$, with the same base cases.}.
    
    Our main claim will be to show that for all $i$, $\mathsf{runs}(i) \leq 3\cdot\mathsf{elegance}(i) - 2$. Let $\mathsf{elegance}(i) = r$, and consider an optimal repunit representation $(a_1, \ldots, a_r)$ of $i$. WLOG suppose $|a_1| \geq \ldots \geq |a_r|$. Then $\mathsf{sgn}(a_1) = 1$. We will start with the binary representation $11\ldots 1$ of $2^{a_1} - 1$, which contains a single run of exactly $a_1$ $1$s. We will then add or subtract all the other terms $a_2, \ldots, a_r$, performing all our operations in binary. We will carefully keep track of how each of these operations can affect the number of runs.

    Consider an arbitrary binary integer, with $t$ runs, and consider adding a repunit to it. Adding such a repunit can be thought of as adding a single power of $2$ (which is a binary integer of the form $10\ldots 0$), and then subtracting a single $1$. When we add the power of $2$, starting from the right, the $0$s do not change the number of runs, until we get to the leading $1$. Observe that adding or subtracting a single $1$ can increase the number of runs by at most $1$. Therefore, at the leading $1$, we can add a new run by a mismatched bit between the $0$ and the $1$, and can also add a new run by adding the $1$ itself. Therefore, adding a power of $2$ can increase the number of runs by at most $2$. After that, subtracting the $1$ adds at most another run, as observed. Therefore, adding a repunit adds at most three runs to the original binary integer. By a symmetric argument, subtracting a repunit (which is equivalent to subtracting a power of $2$, and then adding a $1$) can also increase the number of runs by at most $3$.

    Since we started with $2^{a_1} - 1$, which contained a single run, and then added or subtracted $r - 1$ other repunits, the total number of runs in the final integer is at most $1 + 3(r - 1)$. This immediately implies that $\mathsf{runs}(i) \leq 3\cdot\mathsf{elegance}(i) - 2$.

    Coming back to $B_k$, we have just shown that for $i \leq 2^k - 1$, the number of edges in the in-order traversal spanning the subinterval $(h_i, h_{i+1})$ is at most $3\cdot\mathsf{elegance}(i) - 2$, which by Proposition \ref{prop:elegance} is at most $3\cdot\delta_{B_k}(i) + 1$. We now consider a couple of cases.

    We note that, for $i \leq 2^k - 1$, we have $\delta_{B_k}(i) = 1$ if and only if $i = 2^{k_1} - 1$ for some $k_1 \leq k$. This follows just by observing that every edge in a complete binary tree has a complete binary subtree on one side, and the other side cannot have size less than $2^k$. But in that case, $\mathsf{runs}(i) = \delta_{B_k}(i)$, and so on these subintervals, the in-order traversal only subtends a single edge (and is therefore optimal).

    On the other hand, for $i \leq 2^k - 1$, if $\delta_{B_k}(i) \geq 2$, then $1 \leq (\ffrac{1}{2})\cdot\delta_{B_k}(i)$.

    Therefore, the number of edges over any subinterval $(h_i, h_{i+1})$ in the range $i \leq 2^k - 1$ is at most $3\cdot\delta_{B_k}(i) + (\ffrac{1}{2})\cdot\delta_{B_k}(i) = 3.5\cdot\delta_{B_k}(i)$. By symmetry, this is true over all smallest subintervals of the valuation interval. Therefore, the number of edges passing over every smallest subinterval is at most $3.5$ times the minimum possible number of edges passing over that subinterval, and this yields the desired result.
\end{proof}
% \begin{proof}[Proof Sketch]
%     For any $i \leq 2^k - 1$, consider the number of edges of $B_k$ spanning the subinterval $(h_i, h_{i+1})$ of the valuation interval. It can be shown that under the in-order traversal, the number of edges spanning this interval is exactly $\mathsf{runs}(i)$, the number of runs of contiguous $0$s or $1$s in the binary representation of $i$. A careful argument shows that adding a single repunit to any binary number cannot increase the number of runs by more than $3$. It follows that for all $i$, $\mathsf{runs}(i) \leq 3\cdot\mathsf{elegance}(i) - 2 \leq 3\cdot\delta_{B_k}(i) + 1$, by Proposition \ref{prop:elegance}. Some straightforward casework now yields the result.
% \end{proof}

% In fact, the proof of Theorem \ref{thm:inorder} can be optimized a little bit to improve the approximation ratio slightly. We refer interested readers to Remark \ref{rem:optimizedinorderproof} in Appendix \ref{apdx:binary}.

\begin{remark}\label{rem:optimizedinorderproof}
    The proof of Theorem \ref{thm:inorder} can be optimized slightly for a more nuanced analysis. As observed in the proof, adding or subtracting repunits can be thought of as adding or subtracting powers of $2$, followed by subtracting or adding off $1$s. Consider performing all operations with the powers of $2$ first, and then finally adding or subtracting the number obtained by the $\pm 1$s. Each power-of-$2$ operation increases the number of runs by at most $2$, and the final additional number is at most $r - 1$, which has $1 + \lfloor\log(r - 1)\rfloor$ bits. Therefore, the total number of runs is actually $2 + 2(r - 1) + \lfloor\log(r - 1)\rfloor$. Note that this number is at most $(7/3)r$ for all $r \geq 18$, and in fact, the ratio of this number to $r$ gets arbitrarily close to $2$ as $r$ gets larger.
\end{remark}


It is instructive to check why this technique does not hold for arbitrary binary trees. Proposition \ref{prop:elegance} does not hold in general for non-complete binary trees. A complete binary tree ensures that there is always a binary subtree of the size given by a repunit representation to include on one side of the cut, but we lose this guarantee for non-complete trees.

We leave it as an open problem to construct either value-agnostic deterministic algorithms that achieve an approximation ratio better than $3.5$, or to obtain any polynomial-time algorithm (which cannot be value-agnostic) to obtain any approximation ratio better than $1.67$ for complete binary trees. We believe there should be an exact algorithm for this very special class of graphs, and hope that this will instigate future research into this problem.
\section{Conclusions}\label{sec:conclusions}

We explored the approximability of {\GHA}, presenting tight approximation algorithms for several classes of connected graphs, to our knowledge the first such results in the area. In particular, we gave polynomial-time algorithms exploiting graph structures to approximate the optimal envy on general graphs, trees, planar graphs, bounded-degree graphs, bounded-degree planar graphs, and bounded-degree trees; for each of these classes, we also gave a matching lower bound. Our algorithms were value-agnostic, i.e., they took into account only the input graph and the ordering among the house values but not the values themselves. We showed that any allocation on a random graph is a $(1 + o(1))$-approximation, and also gave a value-agnostic algorithm to show a $3.5$-approximation on all instances on complete binary trees.

%The following open question is an important one in further understanding the boundary between {\GHA} and {\MLA}: \cite{canon} showed that the former is NP-hard even for disjoint unions of paths, whereas \cite{mlatrees} showed the latter is exactly solvable in polynomial time on forests. The following question asks whether connectivity buys us anything in {\GHA}.
%\begin{open}
%    What is the complexity of {\GHA} on bounded-degree trees?
%\end{open}

The main question we leave for future work is the complexity of \GHA{} on complete binary trees. We know by the results in Section \ref{sec:completebintrees} that no exact algorithm can be value agnostic, but there seems to be no obvious way of leveraging the values, on even such a structured class of graphs.  
\begin{conjecture}\label{conj:completebintrees}
    {\GHA} is polynomial-time solvable on complete binary trees.
\end{conjecture}

% Another interesting problem is showing the hardness of \GHA on arbitrary binary trees. Our bounded-degree trees hardness result

% \newpage

\section*{Acknowledgments}

We wish to thank Justin Payan for many helpful discussions during the early stages of this work. We also thank Paul Seymour for some discussions related to the material in Section \ref{sec:completebintrees}. Rohit Vaish acknowledges support from SERB grant no. CRG/2022/002621 and DST INSPIRE grant no. DST/INSPIRE/04/2020/000107. Andrew McGregor and Rik Sengupta acknowledge support from NSF grant CCF-1934846. 
Hadi Hosseini acknowledges support from NSF IIS grants \#2144413 and \#2107173.


\bibliographystyle{plainnat}
\bibliography{abb,references}

\newpage

\clearpage
\appendix

\section{Distillation Backbones}

\subsection{Datasets and Networks}
Following ~\cite{zhao2020dataset, zhao2021dataset, cazenavette2022dataset, deng2022remember}, we use a ConvNet with 3 blocks for CIFAR10 and CIFAR100~\cite{krizhevsky2009learning}, a ConvNet with 4 blocks for Tiny-ImageNet~\cite{le2015tiny}, and a ConvNet with 5 blocks for Nette (a subset of ImageNet)~\cite{Howard_Imagenette_2019}. Each block in the ConvNets contains a 3 $\times$ 3 convolutional layer with 128 channels, followed by instance normalization~\cite{ulyanov2016instance}, ReLU~\cite{nair2010rectified} and a 2 $\times$ 2 average pooling layer with stride 2. We apply Kornia ZCA~\cite{riba2020kornia} on CIFAR10 and CIFAR100 for distillation backbones~\cite{zhao2020dataset, zhao2021dataset, cazenavette2022dataset}. We pick the ConvNet in each distillation backbone because it gives the best distillation performance while keeping the distillation process under an acceptable time and computational budget.

\section{Additional Experiments}

\subsection{Details in Masked Temperature Scaling}
We sample from all the distilled data we have as the validation set to update the temperature parameter $T$ in our proposed Masked Temperature Scaling. Instead of sampling from all the shuffled data at once, we perform a per-class sampling such that there is no missing class or over-sampled class, which is especially important for distillation settings that aim for aggressive compression rates such as image-per-class $\leq$ 10. The traditional temperature scaling~\cite{guo2017calibration} separates all the data available into a training set and a validation set and uses the validation set only for updating $T$. This separated use of the distilled data is not applicable when image-per-class = 1. Moreover, a data split of 10\% can hurt training accuracy by as much as 1.68\% on the Nette subset of ImageNet, while our proposed during-training calibration method (MDT) only hurts accuracy by 0.24\%, as reported in Table~\ref{tab:ts_acc_table}. In addition, our proposed after-training method Masked Temperature Scaling keeps original training accuracy and achieves better calibration results than temperature scaling as reported in our main text.

\begin{table}[!t]
    \addtolength{\tabcolsep}{0pt}
    \renewcommand{\arraystretch}{0.9}
    \centering
    \caption{Accuracy (\%) drops by as much as 1.68\% when training with 90\% of distilled Nette (a subset of ImageNet). The rest 10\% is used in temperature scaling (TS). Our proposed after-training MTS (\colorbox[gray]{0.8}{shadow}) keeps the original accuracy. Our proposed during-training MDT (\colorbox[gray]{0.9}{shadow}) keeps a higher accuracy than that of dropping 10\% of training data for TS. We use MTT~\cite{cazenavette2022dataset} as the distillation backbone.}
    \label{tab:ts_acc_table}

    \scalebox{0.83}{

    \setlength{\aboverulesep}{0pt}
    \setlength{\belowrulesep}{0pt}
    \setlength{\extrarowheight}{0.8ex}

    \begin{tabular}{l|>{\columncolor[gray]{0.8}}c|c>{\columncolor[gray]{0.9}}c}
    \toprule
    \multicolumn{1}{c|}{Dataset}         & Full, MTS (Ours)        & TS (10\%)                 & MDT (Ours)                      \\ \hline
    \multicolumn{1}{l|}{CIFAR10}         & 70.48 $\pm$ 0.2          & 69.78 $\pm$ 0.5         & 69.98 $\pm$ 0.4        \\
    \multicolumn{1}{l|}{CIFAR100}        & 47.47 $\pm$ 0.2          & 47.10 $\pm$ 0.2      & 46.21 $\pm$ 0.4                 \\
    \multicolumn{1}{l|}{Tiny ImageNet}   & 27.76 $\pm$ 0.2          & 27.35 $\pm$ 0.2         & 27.62 $\pm$ 0.4        \\
    \multicolumn{1}{l|}{ImageNette}      & 63.04 $\pm$ 1.3          & 61.36 $\pm$ 1.6         & 62.80 $\pm$ 1.2                 \\ \bottomrule
    \end{tabular}
    }
    \vspace{-0.5cm}
\end{table}

% Figure environment removed

\subsection{More Results on SVD of Distilled Data and Full Data}

As we discussed in our main text, distilled data contain more concentrated information that easily gets grouped by algorithms such as SVD. We here illustrate the cumulative explained ratio of top singular values of data distilled by different backbones. We expect that concentrated information leads to a curve skewed to the top left and evenly distributed information leads to a smooth curve close to the diagonal. This will show how much each component corresponding to the singular values in $\Sigma$ contributes to the data reconstruction. As shown in Figure \ref{fig:supp_svd}, the cumulative explained ratio given by ours grows at the most steady rate, showing that our method produces more evenly distributed information in distilled data compared to the overly condensed information in other distillation backbones. As we concluded in our main text, this serves as a regularization to the distillation process such that it cannot discard too much information that is unrelated to the classification task but semantically meaningful for other tasks, leading to more calibratable networks trained on the resulting distilled data.

\subsection{More Results on DDNNs' Limited Encoding Capacity}

We provide more visualizations of projections of intermediate feature vectors obtained from DDNNs trained with different during-training calibration methods. The methods we use are mixup, focal loss, and label smoothing, in addition to the original training with cross-entropy loss. We can see in Figure~\ref{fig:supp_proj} that our proposed during-training calibration MDT alleviates the issue of concentrate features for all the traditional methods used, giving better encoding potentials of DDNNs for transfer learning tasks, which leads to more calibratable DDNNs.

\subsection{More Results on CIFAR100: ECE on different IPCs, max logits}
We show in Figure\ref{fig:supp_ipc_ece} that our MTS outperforms others in ECE on different IPCs. 
In the main paper, we mainly present IPC = 10 on Tiny-ImageNet \& Subsets with MTT, 10 on CIFAR100 with DC/DSA (released), and 50 on others. These DD settings have higher accuracy and would better represent real-world settings.
% Figure environment removed

We also provide visualization of maximum logits of DDNN on original MTT in Figure~\ref{fig:supp_max_logit_100}, in addition to the results on CIFAR10 in our main paper.
% Figure environment removed


\subsection{Performance Analysis of FDNNs}
\begin{table}[!t]
    \addtolength{\tabcolsep}{-2pt}
    \renewcommand{\arraystretch}{1}
    \centering
    \caption{ECE (\%) of different calibration methods on FDNNs. With a low masking ratio $r$, our results (\colorbox[gray]{0.9}{shadow}) are comparable to temperature scaling and most of the time beats other methods. As our method is specifically designed for DDNNs, in the case of FDNNs where traditional methods are suitable, we can simply convert our method to temperature scaling by setting $r$ to 0.}
    \label{tab:fd_ece_table}
    \scalebox{0.9}{

    \setlength{\aboverulesep}{0pt}
    \setlength{\belowrulesep}{0pt}
    \setlength{\extrarowheight}{.75ex}
    
    \begin{tabular}{l|c|cccc>{\columncolor[gray]{0.9}}cc}
    \toprule
    \multicolumn{1}{c|}{Dataset} & Raw & TS   & MX           & LS & FL      & MTS               \\ 
    \hline
    CIFAR10                           & 4.50       & 0.99 & 14.80 & 11.85 & 1.78 & 2.67                                \\
    CIFAR100                          & 13.05       & 1.41 & 10.69          &   7.17     &  3.49        & 1.84                                  \\
    Tiny ImageNet                      & 22.26       & 4.95          & 6.34          &   3.29      & 12.55         & 4.93                         \\
    ImageNette                    & 10.90       & 2.81 & 11.22          & 22.24 &   5.21       & 2.87                             \\ \bottomrule
    \end{tabular}}
    \vspace{-0.4cm}
\end{table}

We further test MTS on the more calibratable FDNNs. We calibrate networks trained on the full CIFAR10, CIFAR100, TinyImageNet, and Nette subset of ImageNet. We report the mean of 2 runs due to limited computational resources. As reported in Table~\ref{tab:fd_ece_table}, our method performs comparably with existing well-developed methods. In realistic settings with a large amount of training data, we can set the masking ratio $r$ to 0, which converts the MTS back to normal temperature scaling.

% Figure environment removed


\end{document}