\section{Conclusions}\label{sec:conclusions}

We explored the approximability of {\GHA}, presenting tight approximation algorithms for several classes of connected graphs, to our knowledge the first such results in the area. In particular, we gave polynomial-time algorithms exploiting graph structures to approximate the optimal envy on general graphs, trees, planar graphs, bounded-degree graphs, bounded-degree planar graphs, and bounded-degree trees; for each of these classes, we also gave a matching lower bound. Our algorithms were value-agnostic, i.e., they took into account only the input graph and the ordering among the house values but not the values themselves. We showed that any allocation on a random graph is a $(1 + o(1))$-approximation, and also gave a value-agnostic algorithm to show a $3.5$-approximation on all instances on complete binary trees.

%The following open question is an important one in further understanding the boundary between {\GHA} and {\MLA}: \cite{canon} showed that the former is NP-hard even for disjoint unions of paths, whereas \cite{mlatrees} showed the latter is exactly solvable in polynomial time on forests. The following question asks whether connectivity buys us anything in {\GHA}.
%\begin{open}
%    What is the complexity of {\GHA} on bounded-degree trees?
%\end{open}

The main question we leave for future work is the complexity of \GHA{} on complete binary trees. We know by the results in Section \ref{sec:completebintrees} that no exact algorithm can be value agnostic, but there seems to be no obvious way of leveraging the values, on even such a structured class of graphs.  
\begin{conjecture}\label{conj:completebintrees}
    {\GHA} is polynomial-time solvable on complete binary trees.
\end{conjecture}

% Another interesting problem is showing the hardness of \GHA on arbitrary binary trees. Our bounded-degree trees hardness result