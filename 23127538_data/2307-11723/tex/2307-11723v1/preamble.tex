%\usepackage[a4paper]{geometry}
\usepackage{amssymb,amsmath,amsthm}
\usepackage{mathtools}
\usepackage{mathrsfs}
\usepackage{graphicx}
\usepackage{xtab}
\usepackage{xcolor}
\usepackage{bbm}
\usepackage{enumitem}
\usepackage{url}
\usepackage{bbm}
\usepackage{tikz}
\usetikzlibrary{matrix,arrows}

% standard maths symbols
 \newcommand{\Z}{\ensuremath{\mathbb{Z}}}% integers
 \newcommand{\N}{\ensuremath{\mathbb{N}}}% natural numbers
 \newcommand{\R}{\ensuremath{\mathbb{R}}}% real numbers
 \newcommand{\C}{\ensuremath{\mathbb{C}}}% complex numbers

% MY COMMANDS
\setlist[enumerate,1]{label=(\roman*)}

\newcommand{\norm}[1]{\left\lVert#1\right\rVert}
%\newcommand{\dhnorm}[1]{\norm{#1}_{\mathcal{H}}}
\newcommand{\pnorm}[1]{\left|#1\right|}
\newcommand{\norminf}[1]{\left|#1\right|_\infty}
\newcommand{\eqd}{=_d}
\newcommand{\holsp}[1][\eta]{\mathcal{C}^{#1}}
\newcommand{\sepholsp}[1][\eta]{\mathcal{S}^{#1}}
\newcommand{\Ex}[2][]{\mathbb{E}_{#1\!\!}\left[#2\right]}

\newcommand{\indic}[1]{\mathbbm{1}\left\{#1\right\}}
\newcommand{\indicd}[1]{\mathbbm{1}_{#1}}
\newcommand{\tcr}[1]{\textcolor{red}{#1}}

\newcommand{\E}{\ensuremath{\mathbb{E}}}
\newcommand{\bbG}{\ensuremath{\mathbb{G}}}
\newcommand{\bbX}{\ensuremath{\mathbb{X}}}% integers
\newcommand{\bbW}{\ensuremath{\mathbb{W}}}% integers
\newcommand{\bbS}{\ensuremath{\mathbb{S}}}% integers
\newcommand{\cB}{\mathcal{B}}
\newcommand{\cC}{\mathcal{C}}
\newcommand{\cQ}{\mathcal{Q}}
\newcommand{\cF}{\mathcal{F}}
\newcommand{\cL}{\mathcal{L}}

\newcommand{\tf}{T}
\newcommand{\ms}{M}
\newcommand{\conv}{\rightarrow}
\newcommand{\lconv}{\rightarrow}
\newcommand{\eps}{\varepsilon}

\newcommand{\pconv}{\ensuremath{\overset{p}{\longrightarrow}}}
\newcommand{\wconv}{\ensuremath{\overset{w}{\longrightarrow}}}
\newcommand{\limn}{\ensuremath{\lim_{n\rightarrow \infty}}}
\newcommand{\Var}{\mathrm{Var}}
\newcommand{\leb}{\mathrm{Leb}}
% hack so that don't boldface vectors in multidim CLT section 
\newcommand{\bs}[1]{#1}
\newcommand{\Cov}[3][]{\mathrm{Cov}_{#1}\left(#2,#3\right)}
% avoiding writing accents
\newcommand{\holder}{H\"{o}lder}
\newcommand{\pene}{P{\`e}ne}
\newcommand{\ito}{It\^{o}}
\newcommand{\cadlag}{c{\`a}dl{\`a}g}

\newcommand{\map}[2]{\colon#1\rightarrow#2}
\newcommand{\Lip}{\mathrm{Lip}}
% Others 
%\newcommand{\st}{\ensuremath{:}}% such that
%\newcommand{\Tau}{\ensuremath{\mathcal{T}}}
%\newcommand{\Nat}{\mathbb{N}}

%\usepackage[numbers]{natbib}% round braces, sort multiple citations
%\usepackage{hyperref}% creates hypertext links in pdf files (natbib compatible)
%\usepackage{setspace}
%\usepackage{fancyhdr}
%\usepackage{mathrsfs}
%\usepackage{textcomp}
%\usepackage{color}
%\usepackage{graphicx}
%\usepackage{framed}
% \usepackage{algorithmic}% format pseudocode
%\usepackage[vlined,boxed,commentsnumbered,algochapter]{algorithm2e}
% \usepackage[chapter]{algorithm}% float wrapper for algorithms
%\usepackage{amsmath}% American Mathematical Society macros - essential!
%\usepackage{amssymb}% contains amsfonts
%\usepackage{amsthm}% allows more flexibility with theorems
\usepackage[nodayofweek]{datetime} % change the format of printed dates (no american style!) 
%\usepackage{ifdraft}% perform operations conditional on the draft option

% \usepackage{mathptmx}
% \usepackage{mathpazo}
%\usepackage{amscd}
% \usepackage{xy}
% \usepackage{diagxy}
%\usepackage{diagrams}

%\usepackage{marginnote}
%\usepackage{rotating}
%\usepackage{multirow}
% \usepackage{polski}
% \usepackage[T]{fontenc}
% \usepackage{tikzpicture}
% \usetikzlibrary{matrix,arrows}

% \usepackage[inline]{showlabels}
%\usepackage{booktabs}

% Date format

% Use the datetime package
\newdateformat{monthyear}{\monthname[\THEMONTH], \THEYEAR}% new date format
\DeclareMathOperator{\supp}{supp}
\DeclareMathOperator{\coni}{coni}

\newtheorem{prop}{Proposition}[section]
\newtheorem{lemma}[prop]{Lemma}
\newtheorem{conjecture}[prop]{Conjecture}
\newtheorem{theorem}[prop]{Theorem}
\newtheorem{hyp}[prop]{Hypothesis}
\newtheorem{ass}[prop]{Assumption}
\newtheorem{corol}[prop]{Corollary}
\newtheorem{claim}[prop]{Claim}
\newtheorem{notation}[prop]{Notation}
\newtheorem{defn}[prop]{Definition}
\newtheorem{exa}[prop]{Example}
\newtheorem{problem}[prop]{Problem}
% remove italics from remarks
\theoremstyle{remark}
\newtheorem{remark}[prop]{Remark}

% Stack Exchange custom overbar command
\makeatletter
\let\save@mathaccent\mathaccent
\newcommand*\if@single[3]{%
	\setbox0\hbox{${\mathaccent"0362{#1}}^H$}%
	\setbox2\hbox{${\mathaccent"0362{\kern0pt#1}}^H$}%
	\ifdim\ht0=\ht2 #3\else #2\fi
}
%The bar will be moved to the right by a half of \macc@kerna, which is computed by amsmath:
\newcommand*\rel@kern[1]{\kern#1\dimexpr\macc@kerna}
%If there's a superscript following the bar, then no negative kern may follow the bar;
%an additional {} makes sure that the superscript is high enough in this case:
\newcommand*\widebar[1]{\@ifnextchar^{{\wide@bar{#1}{0}}}{\wide@bar{#1}{1}}}
%Use a separate algorithm for single symbols:
\newcommand*\wide@bar[2]{\if@single{#1}{\wide@bar@{#1}{#2}{1}}{\wide@bar@{#1}{#2}{2}}}
\newcommand*\wide@bar@[3]{%
	\begingroup
	\def\mathaccent##1##2{%
		%Enable nesting of accents:
		\let\mathaccent\save@mathaccent
		%If there's more than a single symbol, use the first character instead (see below):
		\if#32 \let\macc@nucleus\first@char \fi
		%Determine the italic correction:
		\setbox\z@\hbox{$\macc@style{\macc@nucleus}_{}$}%
		\setbox\tw@\hbox{$\macc@style{\macc@nucleus}{}_{}$}%
		\dimen@\wd\tw@
		\advance\dimen@-\wd\z@
		%Now \dimen@ is the italic correction of the symbol.
		\divide\dimen@ 3
		\@tempdima\wd\tw@
		\advance\@tempdima-\scriptspace
		%Now \@tempdima is the width of the symbol.
		\divide\@tempdima 10
		\advance\dimen@-\@tempdima
		%Now \dimen@ = (italic correction / 3) - (Breite / 10)
		\ifdim\dimen@>\z@ \dimen@0pt\fi
		%The bar will be shortened in the case \dimen@<0 !
		\rel@kern{0.6}\kern-\dimen@
		\if#31
		\overline{\rel@kern{-0.6}\kern\dimen@\macc@nucleus\rel@kern{0.4}\kern\dimen@}%
		\advance\dimen@0.4\dimexpr\macc@kerna
		%Place the combined final kern (-\dimen@) if it is >0 or if a superscript follows:
		\let\final@kern#2%
		\ifdim\dimen@<\z@ \let\final@kern1\fi
		\if\final@kern1 \kern-\dimen@\fi
		\else
		\overline{\rel@kern{-0.6}\kern\dimen@#1}%
		\fi
	}%
	\macc@depth\@ne
	\let\math@bgroup\@empty \let\math@egroup\macc@set@skewchar
	\mathsurround\z@ \frozen@everymath{\mathgroup\macc@group\relax}%
	\macc@set@skewchar\relax
	\let\mathaccentV\macc@nested@a
	%The following initialises \macc@kerna and calls \mathaccent:
	\if#31
	\macc@nested@a\relax111{#1}%
	\else
	%If the argument consists of more than one symbol, and if the first token is
	%a letter, use that letter for the computations:
	\def\gobble@till@marker##1\endmarker{}%
	\futurelet\first@char\gobble@till@marker#1\endmarker
	\ifcat\noexpand\first@char A\else
	\def\first@char{}%
	\fi
	\macc@nested@a\relax111{\first@char}%
	\fi
	\endgroup
}
\makeatother
