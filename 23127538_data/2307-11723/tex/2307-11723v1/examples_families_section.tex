\section{Examples}\label{section:examples}
In this section we consider examples of families of dynamical systems for which we can verify the hypotheses of Theorem~\ref{thm:homog_from_fcb_family}. 
\subsection{Intermittent Baker's maps}\label{subsection:intermittent_baker}
Let $ I=[0,1],$ $M=I^2 $. Fix a family of intermittent Baker's maps $ T_n:M\to M $, $ n\in \N\cup\{\infty\} $, as in~\eqref{eq:intermittent_baker} with parameters $ \alpha_n\in(0,\frac{1}{2}) $ such that $ \lim_{n\to\infty}\alpha_n = \alpha_\infty \in (0,\frac{1}{2})$. Recall that $ T_n $ is a skew product map of the form 
\[ T_n(x,z)=(\bar T_n(x),h_n(x,z)),\quad h_n(x,z)=\begin{cases}
	g_{n,0}(z), & x<\frac{1}{2},\\
	g_{n,1}(z), & x\ge\frac{1}{2},
\end{cases} \]
where $ \bar T_n:I\to I $ is the Liverani-Saussol-Vaienti map with parameter $ \alpha_n $ and $ g_{n,0}$, $g_{n,1} $ are the inverse branches of $\bar T_n $. The projection $ \pi:M\to I $, $ \pi(x,z)=x $ defines a semiconjugacy between $ T_n $ and $ \bar T_n $. By~\cite{lsvmaps}, there is a unique $\bar T_n $-invariant ergodic probability measure $ \bar\mu_n $ which is absolutely continuous with respect to Lebesgue. Let $ \mu_n $ be the unique $ T_n $-invariant ergodic probability measure such that $ \pi_*\mu_n =\bar\mu_n $ (we construct this measure in Proposition~\ref{prop:lifted_meas_skew}).

\begin{lemma}
	$ T_n $, $ n\in \N\cup\{\infty\} $, is a uniform family of nonuniformly hyperbolic maps with $ O(k^{-1/\alpha}) $ tails, where $ \alpha=\sup_n \alpha_n  $.
\end{lemma} 
\begin{proof}
For each $ n $, we take $ \bar Y=[1/2,1]$ and let $ \bar{\phi}_n:\bar Y\to \Z^+ $ be the first return time to $\bar Y $, i.e.\ $\bar \phi_n(y)=\inf\{k\ge 1:\bar T_n^k(y)\in \bar Y \} $. Then by~\cite[Section 6]{young1999recurrence}, the first return map $ \bar F_n=\bar T_n^{\bar \phi_n}:\bar Y\to \bar Y $ is a Gibbs-Markov map and there exists a constant $ C>0 $ such that
$\bar\mu_n(\bar\phi_n>k)\le Ck^{-1/\alpha_n}$ for all $ k\ge 1$.
 Moreover, by~\cite[Example 5.1]{kkmaveraging} both the constant $ C $ and the constants that appear in the definition of Gibbs-Markov map can be chosen independently of $ n $. It follows that condition (ii) in Definition~\ref{def:nuh_family} is satisfied.
 
 Note that $ \{\bar \phi_n = 1\}=[3/4,1] $. By~\cite[Lemma 2.4]{lsvmaps}, $ \inf_n \inf_I d\bar \mu_n/d\leb>0 $. Hence $ \inf_n \bar\mu_n|_{\bar Y}(\bar \phi_n = 1)>0 $, and so condition (iii) in Definition~\ref{def:nuh_family} is satisfied.

Let $ Y=\bar Y\times I $ and let $ \phi_n:Y\to \Z^+ $ be the first return time to $ Y $.  Then $ \phi_n=\bar\phi_n\circ \pi|_Y $ and $ \pi|_Y $ defines a semiconjugacy between $ F_n = T_n^{\phi_n}:Y\to Y $ and $ \bar F_n $.

We now complete the proof of condition (i) in Definition~\ref{def:nuh_family} by verifying that~\eqref{eq:two_sided_GM_contraction} and~\eqref{eq:yt_intermediate_iterate_bd} hold with constants $ D_0, \theta $ that are uniform in $ n $. Denote $ \psi_{n,k}(x)=\#\{j=0,\dots,k-1:\bar T_n^j x\in \bar Y\} $. We claim that 
\begin{equation}\label{eq:intermittent_baker_yt_bound}
	d(T_n^k(x_1,z_1),T_n^k(x_2,z_2))\le 2(\tfrac{1}{2})^{s(x_1,x_2)-\psi_{n,k}(x_1)}+(\tfrac{1}{2})^{\psi_{n,k}(x_1)}d(z_1,z_2).
\end{equation}
for all $ k\ge 1 $, $ n\in \N\cup\{\infty\} $, $ (x_1,z_1), (x_2,z_2)\in Y $. It is straightforward to check that both~\eqref{eq:two_sided_GM_contraction} and \eqref{eq:yt_intermediate_iterate_bd} follow from this claim with $ D_0=2 $, $ \theta=\frac{1}{2} $.

Let $ x_1, z_1, z_2\in I $. Then for $ i=1,2 $,
\begin{equation}\label{eq:skew_product_iterates}
	T^k_n(x_1,z_i)=(\bar T_n^k x_1, g_{n,a(k-1)}\circ \dots \circ g_{n,a(0)}(z_i)),
\end{equation}
where $ a(j)=\indic{\bar T_n^j x_1\in \bar Y} $. Since $ \norminf{g'_{n,0}}\ge 1 $ and $ \norminf{g'_{n,1}}=\frac{1}{2} $, by the mean value theorem it follows that 
\begin{align}\label{eq:intermittent_baker_stable_bd}
	d(T_n^k(x_1,z_1),T_n^k(x_1,z_2))\le \prod_{j=0}^{k-1}\big|g'_{n,a(j)}\big|_\infty d(z_1,z_2)\le (\tfrac{1}{2})^{\psi_{n,k}(x_1)}d(z_1,z_2).
\end{align}
Let $ x_1, x_2 \in \bar{Y} $. Without loss of generality assume that $ s(x_1,x_2)>\psi_{n,k}(x_1) $, for otherwise~\eqref{eq:intermittent_baker_yt_bound} is satisfied trivially. Since $ \phi_n $ is the first return time to $ \bar Y $, it follows that for all $ 0\le j<k $ we have $ \bar T_n^j x_1\in \bar Y $ if and only if $ \bar T_n^j x_2 \in \bar Y $. Hence by~\eqref{eq:skew_product_iterates},
$ d(T_n^k(x_1,z_1),T_n^k(x_2,z_1))=d(\bar T_n^k x_1,\bar T_n^k x_2). $ Note that $ \bar F_n^{\psi_{n,k}+1}=\bar T_n^r $, where $ r = \sum_{\ell=0}^{\psi_{n,k}}\bar\phi_n \circ \bar F_n^\ell $. Since $ r(x_1)=r(x_2)>k $ and $ \bar T'_n\ge 1 $, it follows that 
\[ d(\bar T_n^k x_1,\bar T_n^k x_2)\le d(\bar T_n^{r(x_1)} x_1,\bar T_n^{r(x_1)} x_2)=d(\bar F_n^{\psi_{n,k}+1} x_1,\bar F_n^{\psi_{n,k}+1} x_2). \]
Now $\bar T_n'\ge 1 $ and $\bar T_n'=2 $ on $ \bar Y $ so $ d(\bar F_n y, \bar F_n y')\ge 2d(y,y') $ whenever $ y,y'\in \bar Y $ belong to the same partition element. By~\cite[Lemma 3.2]{alves_book}, it follows that $ d(\bar F_n^{\psi_{n,k}+1} x_1,\bar F_n^{\psi_{n,k}+1} x_2)\le (\frac{1}{2})^{s(x_1,x_2)-(\psi_{n,k}(x_1)+1)} $. The claim~\eqref{eq:intermittent_baker_yt_bound} follows by combining this with~\eqref{eq:intermittent_baker_stable_bd}. 
\end{proof}
By Theorem~\ref{thm:uniform_fcb}, it follows that the family $ T_n $ satisfies the Functional Correlation Bound uniformly with rate $ k^{-(1/\alpha - 1)} $. In the remainder of this subsection we verify that condition~\eqref{eq:conv_correlation_fns} is satisfied by verifying the conditions (A1), (A2) from Remark~\ref{rk:sufficient_conditions_conv}. It follows that our main result, Theorem~\ref{thm:homog_from_fcb_family}, applies to the family $ T_n $.

By~\cite{korepanovlinearresponse,baladitodd}, $ \bar\mu_n $ is \textit{strongly statistically stable}, i.e.\ $ d\bar \mu_n/d\leb\to d\bar\mu_\infty/d\leb $ in $ L^1(\leb) $.  The following proposition immediately implies that (A2) holds and will also be useful in the proof that $ \mu_n$ is statistically stable, as required by (A1).
\begin{prop}\label{prop:T_n_conv_prob}
	For all $ a>0 $, for all $ j\in \N $,
	\begin{equation}\label{eq:mu_bar_n_conv_in_prob}
		\lim_{n\to\infty}\bar\mu_n\Big(x:\sup_{z\in I}d(T_n^j(x,z),T_\infty^j(x,z))>a \Big)=0.
	\end{equation}
\end{prop}
\begin{proof}
	Let $ \eps>0 $. Choose $ K\subset I $ compact such that $ \frac{1}{2}\notin \bar T_\infty^i(K) $ for $ 0\le i\le j $ and $ \bar \mu_\infty(K)\ge 1-\eps $. Then $ T_n^j \to T_\infty^j $ uniformly on $ K\times I $ so for all $ a>0 $,
	\[ \limsup_{n\to\infty}\bar\mu_\infty\Big(x:\sup_{z\in I}d(T_n^j(x,z),T_\infty^j(x,z))>a \Big)\le \bar\mu_\infty(I\setminus K)<\eps. \]
	It follows that~\eqref{eq:mu_bar_n_conv_in_prob} holds with $ \bar\mu_\infty $ in place of $ \bar\mu_n $. The inequality~\eqref{eq:mu_bar_n_conv_in_prob} follows by strong statistical stability.
\end{proof}
Note that $ T_\infty $ is continuous on $ I^2\setminus (\{\frac{1}{2}\}\times I) $ so $ T_\infty $ is continuous $ \mu_\infty $-a.e. In the remainder of this subsection, we complete the verification of condition (A2) by showing that $ \mu_n $ is statistically stable. We closely follow the strategy that Alves \& Soufi~\cite{alvessoufi} used to prove statistical stability for the Poincar\'e maps of geometric Lorenz attractors.

First let us recall the standard procedure for constructing invariant measures for skew products with contracting fibres. Given a bounded, measurable function $ \phi:M\to \R $, define $ \phi^+:I\to \R $ by 
$ \phi^+(x)=\sup_{z\in I}\phi(x,z). $
\begin{prop}\label{prop:lifted_meas_skew}
	Let $ n\in \N\cup \{\infty\} $. There exists a unique probability measure $ \mu_n $ such that for any continuous function $ \phi:M\to \R $,
	\begin{equation}\label{eq:lifted_meas_def}
		\int_M \phi\, d\mu_n = \lim_{m\to \infty}\int_{I} (\phi\circ T_n^m)^{+}d\bar\mu_n.
	\end{equation}
	Moreover, the convergence is uniform in $ n $. Besides, $ \mu_n $ is the unique $ T_n $-invariant ergodic probability measure such that $ \pi_* \mu_n = \bar\mu_n $.
\end{prop}
\begin{proof}
We first show that the maps $ T_n $ uniformly contract fibres in the sense that 
\begin{equation}\label{eq:unif_contraction}
	\mathrm{diam}\, T_n^m\pi^{-1}(x) \to 0 \text { as }m\to \infty, \text{ uniformly in $ x $ and $ n $.}
\end{equation}
Fix $ x $ and $ n $. By~\eqref{eq:skew_product_iterates} and the fact that $ g_{n,0} $ and $ g_{n,1} $ are inverse branches of $ \bar T_n $,
\[ T_n^m\pi^{-1}(x)=\{\bar T_\infty^m(x)\}\times H(I) \]
where $ H:I\to I $ is an inverse branch of $ \bar T^m_n $. By~\cite[equation (5)]{leppanen2017functional}, there exists $ C>0 $ such that for all $ m,n\ge 1 $, for any inverse branch $ H $ of $ \bar T_n^m $ we have
$ \mathrm{diam}\,H(I)\le Cm^{-1/\sup_n\alpha_n}. $
This proves~\eqref{eq:unif_contraction}. The rest of the proof that the limit~\eqref{eq:lifted_meas_def} exists and the convergence is uniform in $ n $ proceeds exactly as in~\cite[Proposition 3.3]{alvessoufi} (with $ P_n $ and $ f_n $ changed to $ T_n $ and $ \bar T_n $). In~\cite[Corollary 6.4]{singularhyperbolicchaotic} it is shown that $ \mu_n $ indeed defines a $ T_n $-invariant probability measure and that the ergodicity of $ \mu_n $ follows from the ergodicity of $ \bar\mu_n $. By~\cite[Remark 2(b)]{melbourne_butterley}, $ \mu_n $ is the unique $ T_n $-invariant ergodic probability measure such that $ \pi_* \mu_n =\bar\mu_n $.
\end{proof}
\begin{prop}\label{prop:swapped_lims}
	For all $ m\ge 1 $,
	\[ \lim_{n\to \infty}\int (\phi \circ T_n^m)^{+}d\bar \mu_n = \int (\phi \circ T_\infty^m)^{+}d\bar \mu_\infty. \]
\end{prop}
\begin{proof}We proceed as in~\cite[Lemma 3.2]{alvessoufi}.
	
	Write 
	$ \int (\phi \circ T_n^m)^{+}d\bar \mu_n - \int (\phi \circ T_\infty^m)^{+}d\bar \mu_\infty=I_n+J_n $, 
	where 
	\begin{align*}
		I_n&= \int (\phi \circ T_n^m)^{+}-(\phi \circ T_\infty^m)^{+}\, d\bar \mu_n,\quad
		J_n= \int (\phi \circ T_\infty^m)^{+}d\bar \mu_n - \int (\phi \circ T_\infty^m)^{+}d\bar \mu_\infty.
	\end{align*}
Now for all $ x\in I $, 
\begin{equation}\label{eq:diff_phi_+_ineq}
	|(\phi \circ T_n^m)^{+}(x)-(\phi \circ T_\infty^m)^{+}(x)|\le \sup_{y\in I}|\phi \circ T_n^m(x,z)-\phi \circ T_\infty^m(x,z)|.
\end{equation}
Since $ M $ is compact, $ \phi $ is uniformly continuous on $ M $. Hence for any $ \eps>0 $ there exists $ \delta>0 $ such that $ |\phi(z)-\phi(z')|<\eps $ for all $ z,z'\in M $ with $ d(z,z')<\delta $. Let 
\[ S = \Big\{x\in I:\sup_{y\in I}d(T_n^m(x,y),T_\infty^m(x,y))\ge\delta\Big\}. \]
Then by~\eqref{eq:diff_phi_+_ineq},
\begin{align*}
	|I_n|&\le \int_{S} |(\phi \circ T_n^m)^{+}-(\phi \circ T_\infty^m)^{+}|\, d\bar \mu_n + \int_{I\setminus S} |(\phi \circ T_n^m)^{+}-(\phi \circ T_\infty^m)^{+}|\, d\bar \mu_n\\
	&\le 2\norminf{\phi}\bar\mu_n(S)+\eps.
\end{align*}
By Proposition~\ref{prop:T_n_conv_prob}, $ \bar \mu_n(S)\to 0 $ as $ n\to \infty $. Since $ \eps>0 $ is arbitrary, it follows that $ I_n\to 0 $.

Finally, note that
\begin{align*}
	|J_n|= \bigg|\int (\phi \circ T_\infty^m)^{+}\bigg(\frac{d\bar\mu_n}{d\leb} - \frac{d\bar\mu_\infty}{d\leb}\bigg) d\leb\bigg|\le \norminf{\phi}\bigg|\frac{d\bar\mu_n}{d\leb} - \frac{d\bar\mu_\infty}{d\leb}\bigg|_{L^1(\leb)}.
\end{align*}
Hence by strong statistical stability, $ J_n\to 0 $ as $ n\to \infty $.
\end{proof}
We can now complete the proof that $ \mu_n $ is statistically stable, i.e.\ $ \mu_n\wconv \mu_\infty $. 
	Let $ \phi:M\to \R $ be continuous. Then by Propositions~\ref{prop:lifted_meas_skew} and \ref{prop:swapped_lims},
	\[ \int_M \phi\, d\mu_\infty = \lim_{m\to \infty}\int_{I}(\phi\circ \bar T_\infty^m)^{+}d\bar\mu_\infty=\lim_{m\to \infty}\lim_{n\to\infty}\int_{I}(\phi\circ \bar T_n^m)^{+}d\bar\mu_n. \]
	Since $ \int_{I} (\phi\circ \bar T_n^m)d\bar\mu_n\to \int_M \phi\, d\mu_n$ as $ m\to\infty $ uniformly in $ n $, we can swap the limits as $ m\to \infty $ and $ n\to \infty $ in the above expression. Thus
	\[ \int_M \phi\,d\mu_\infty = \lim_{n\to \infty}\lim_{m\to\infty}\int_{I}(\phi\circ \bar T_n^m)^{+}d\bar\mu_n=\lim_{n\to \infty}\int_M \phi\, d\mu_n, \]
	as required.
\subsection{Externally forced dispersing billiards}\label{subsection:sinai_billiard_external}
A Sinai billiard table on the two-torus $ \mathbb{T}^2 $ is a set of the form $ Q=\mathbb{T}^2\setminus \cup_i B_i $ where $ \{B_i\} $ is a finite collection of open sets such that $\bar B_i \cap \bar B_j =\emptyset $ for $ i\ne j $. It is assumed that the sets $ B_i $ have $ C^3 $ boundaries with positive curvature. The billiard flow on $ Q\times S^1 $ is induced by the motion of a particle that moves in straight lines at unit speed on $ Q $ and collides elastically with the boundary $ \partial Q $. We say that the table has finite horizon if there exists a constant $ L>0 $ such that any line of length $ L $ in $ \mathbb{T}^2 $ intersects $ \partial Q $.

In~\cite{chernov_small_external_I,chernov_small_external_II} Chernov studied perturbations of the finite horizon Sinai billiard flow where a small stationary force $ F $ acts on the particle between its collisions with $ \partial Q $. We refer to~\cite[Section 2]{chernov_small_external_I} for the precise details of the model. In particular, it is assumed that the force preserves an additional integral of motion and that the phase space obtained by restricting to one of its level sets is a compact 3-dimensional manifold. 

Consider the flow obtained by restricting to one of these level sets. The assumptions then guarantee that the collision map $ T_F $ with the table can be parametrised on the same space $ M = \partial Q\times [-\pi/2,\pi/2] $ as the collision map of the unperturbed Sinai billiard flow. Let $ (F_n)_{n\in \N} $ be a sequence of admissible forces such that $ F_n\to F_{\infty}=0 $ in $ C^2 $ and define $ T_n=T_{F_n}:M\to M $. Let $ \mu_n $ denote the unique SRB measure for $ T_n $. 
\begin{prop}\label{prop:fcb_externally_forced}
	For all $ \gamma>1 $ the family $ T_n:M\to M $, $ n\in \N\cup\{\infty\} $, satisfies the Functional Correlation Bound uniformly with rate $ k^{-\gamma} $.
\end{prop}
\begin{remark}
	In principle, it should be possible to prove this proposition by verifying that the family $ T_n $ is a uniform family of nonuniformly hyperbolic maps and applying Theorem~\ref{thm:uniform_fcb}. Indeed, for each $ n $, the system $ T_n $ is modelled by a Young tower with exponential tails~\cite{chernov_small_external_I}. However, the construction of the base of the tower in~\cite{chernov_small_external_I} is quite intricate so it seems difficult to check condition (iii) in Definition~\ref{def:nuh_family}. 
\end{remark}
\begin{proof}
	In \cite{leppanen2017billiards}, Lepp\"{a}nen \& Stenlund considered the finite horizon Sinai billiard map and proved a functional correlation bound for separately dynamically H\"{o}lder functions. (Note that their definition of dynamical H\"{o}lder continuity differs from that in Section~\ref{subsection:explicit_fcb}.) Recall the definition of the past/future separation times $ s_{\pm} $ and the dynamically H\"{o}lder function classes $ \mathcal{H}_{\pm} $ from~\cite[Section 2]{leppanen2017billiards}.
	
	We first show that separately H\"{o}lder functions are separately dynamically H\"{o}lder with parameters independent of $ n $. By \cite[p.95]{chernov_small_external_II}, there exist constants $ C>0 $, $ \Lambda>1 $ independent of $ n $ such that $ d(x,y)\le C\Lambda^{-s_{+}(x,y)} $ whenever $ x,y\in M $ belong to the same local unstable manifold. Similarly, $ d(x,y)\le C\Lambda^{-s_{-}(x,y)} $ whenever $ x,y\in M $ belong to the same local stable manifold. Let $ v\in C^\eta(M) $. It follows that 
	\[ |v(x)-v(y)|\le [v]_{\eta}d(x,y)^{\eta}\le C^\eta[v]_{\eta}(\Lambda^{-\eta})^{s_{+}(x,y)} \]
	whenever $ x $ and $ y $ belong to the same local unstable manifold. Hence $ v\in \mathcal{H}_{+}(C^\eta[v]_{\eta},\Lambda^{-\eta}) $. Similarly, $ v\in \mathcal{H}_{-}(C^\eta[v]_\eta,\Lambda^{-\eta}) $. Let $ G:M^q\to\R $, $ q\ge 1 $ be separately $ \eta $-H\"{o}lder and set $ c=C^\eta \max_i [G]_{\eta,i} $, $ \vartheta =\Lambda^{-\eta} $. Then $ G(x_0,\dots,x_{i-1},\cdot,x_{i+1},\dots,x_{q-1})\in \mathcal{H}_{-}(c,\vartheta)\cap \mathcal{H}_{+}(c,\vartheta) $ for all $ x_0,\dots,x_{q-1}\in M $, $ 0\le i<q $.
	
	It remains to explain why the arguments used in the proof of~\cite[Theorem 2.4]{leppanen2017billiards} go through with system constants $ M_0, M_1 $ and $ \theta_0$,  $\theta_1$ uniform in $ n $. The result then follows by applying~\cite[Theorem 2.4]{leppanen2017billiards} with $ K=2 $, $ F=G $ and $ c,\vartheta $ as defined above.
	
	Note that~\cite[Lemma 4.1]{leppanen2017billiards} merely gives the usual decomposition of $ \mu $ into a standard family. Let $ \{(\xi_q,\nu_q):q\in \mathcal{Q}\} $ be as defined in that lemma. By~\cite[p.\ 96]{chernov_small_external_II}, $ \{(\xi_q,\nu_q):q\in\mathcal{Q}\} $ is a proper standard family. In particular, there exists a constant $ M_1>0 $ independent of $ n $ such that $ \lambda(\{q\in \mathcal{Q}:|\xi_q|\le \eps\})\le M_1\eps $ for all $ \eps>0 $. By~\cite[p.171]{chernov_markarian}, it follows that there exists a constant $ M'_1 $ independent of $ n $ such that $ \int_{\mathcal{Q}}|\xi_q|^{-1}d\lambda(q)\le M'_1 $, so~\cite[Lemma 4.3]{leppanen2017functional} goes through. Finally, it follows from the growth lemma (\cite[Proposition 5.3]{chernov_small_external_I}) and the equidistribution property (\cite[Proposition 2.2]{chernov_small_external_II}) that~\cite[Lemma 4.2]{leppanen2017billiards} goes through with constants $ a_0 $, $ M_0 $ and $ \theta_0 $ that are uniform in $ n $. The rest of the proof of~\cite[Theorem 2.4]{leppanen2017billiards} proceeds exactly as in~\cite{leppanen2017billiards}.
\end{proof}
We finish this subsection by showing that condition~\eqref{eq:conv_correlation_fns} is satisfied. It follows that Theorem~\ref{thm:homog_from_fcb_family} applies to the family $ T_n $. 
\begin{prop}\label{prop:conv_correlations_forced}
Condition~\eqref{eq:conv_correlation_fns} is satisfied.
\end{prop}
\begin{proof}
	For $ n\in\N\cup\{\infty\} $, Demers \& Zhang~\cite{demers_zhang_perturbations} considered the action of the transfer operator $ \mathcal{L}_n $ associated with $ T_n $ on certain spaces of distributions. In particular, if $ \nu $ is a finite signed measure, then $ \mathcal{L}_n \nu=(T_n)_{*}\nu $.
		
		 Fix $ \eta\in (0,1] $. The article~\cite{demers_zhang_perturbations} constructs Banach spaces $ (\mathcal{B},\norm{\cdot}_{\mathcal{B}}) $ and $ (\mathcal{B}_w,\norm{\cdot}_{\mathcal{B}_w}) $ with the following properties: 
		\begin{enumerate}
			\item There is a sequence of continuous embeddings $ \mathcal{B}\hookrightarrow \mathcal{B}_w \hookrightarrow (C^\eta(M))'$.
			\item For each $ n $, $ \mathcal{L}_n $ is a well-defined bounded linear operator on both $ \mathcal{B} $ and $ \mathcal{B}_w $. Moreover, $ \sup_n \norm{\mathcal L_n}_{\mathcal{B}_w}<\infty $.
			\item (\cite[Theorem 2.2]{demers_zhang_perturbations}) For each $ n $, we have $ \mu_n \in \mathcal{B} $ and $ \mu_n $ is the unique element of $ \mathcal{B} $ such that $ \mathcal{L}_n \mu_n = \mu_n $ and $ \mu_n(1)=1 $.
			\item (\cite[Theorem 2.11]{demers_zhang_perturbations}) $ \norm{\mathcal{L}_n - \mathcal{L}_\infty}_{\mathcal{B}\to \mathcal{B}_w}\to 0 $ as $ n\to\infty $. By~\cite[Theorem 2.1]{demers_zhang_perturbations}, it follows that $ \mu_n \to \mu_\infty $ in $ \mathcal{B}_w $.
			\item (\cite[Lemma 5.3]{demers_zhang_perturbations}) Let $ v\in C^\eta(M) $. Then $ v $ is a bounded multiplier on $ \mathcal{B} $ (that is, $ h\mapsto vh $ is a well-defined bounded operator on $ \mathcal{B} $). Moreover, $ v $ is a bounded multiplier on $ \mathcal{B}_w $.\footnote{Since the definitions of the weak norm and the strong stable norm are similar, this follows easily from the arguments used to bound the strong stable norm at the beginning of the proof of~\cite[Lemma~5.3]{demers_zhang_perturbations}. }
		\end{enumerate}
	Fix $ v\in C^\eta(M) $ and $ k\ge 0 $. For $ n\in\N\cup\{\infty\} $, define a signed probability measure $ \nu_n $ by $ \nu_n=v\mu_n $. Then $ \mathcal{L}_n^k \nu_n(w)=\nu_n(w\circ T_n^k)=\mu_n(vw\circ T_n^k) $ for all $ w\in C^\eta(M) $, so it sufficient to prove that $ \mathcal{L}_n^k \nu_n \to\mathcal{L}_\infty^k \nu_\infty $ in $ (C^\eta(M))' $. Now $ \mu_n\in\mathcal{B} $ so by properties (v) and (ii) we have $ \nu_n\in \mathcal{B} $ and $ \mathcal{L}_n^k \nu_n\in \mathcal{B} $. Hence by (i), it suffices to show that $ \mathcal{L}_n^k \nu_n\to \mathcal{L}_\infty^k \nu_\infty $ in $ \mathcal{B}_w $. 
	
	Write $\mathcal{L}_n^k \nu_n -\mathcal{L}_\infty^k \nu_\infty=I_n+J_n $, where $ I_n = \mathcal{L}^k_n (\nu_n - \nu_\infty) $ and 
\begin{align*}
	J_n = \mathcal{L}^k_n\nu_\infty-\mathcal{L}^k_\infty\nu_\infty= \sum_{j=0}^{k-1}\mathcal{L}_n^j(\mathcal{L}_n - \mathcal{L}_\infty)\mathcal{L}_\infty^{k-j-1} \nu_\infty.
\end{align*}
By properties (ii) and (v),
\begin{align*}
	\norm{I_n}_{\mathcal{B}_w}\le \norm{\mathcal{L}^k_n}_{\mathcal{B}_w}\norm{v\mu_n - v\mu_\infty}_{\mathcal{B}_w}\le C\norm{\mu_n - \mu_\infty}_{\mathcal{B}_w},
\end{align*}
where $ C $ depends only on $ v $. By (iv), it follows that $ \lim_{n\to \infty}\norm{I_n}_{\mathcal{B}_w}=0$. Now by property (ii), there exists a constant $ C'>0 $ such that
\begin{align*}
	\norm{J_n}_{\mathcal{B}_w}&\le \sum_{j=0}^{k-1} \norm{\mathcal{L}^j_n}_{\mathcal{B}_w}\norm{\mathcal{L}_n - \mathcal{L}_\infty}_{\mathcal{B}\to \mathcal{B}_w}\norm{\mathcal{L}_\infty^{k-j-1}}_{\mathcal{B}} \norm{\nu_\infty}_{\mathcal{B}}\\
	&\le C'\norm{\mathcal{L}_n - \mathcal{L}_\infty}_{\mathcal{B}\to \mathcal{B}_w}\!\norm{\nu_\infty}_{\mathcal{B}}.
\end{align*}
Hence by (iv), $ \lim_{n\to \infty}\norm{J_n}_{\mathcal{B}_w}=0 $, which completes the proof.
\end{proof}
\begin{remark}
	Note that we have not used any facts about the anisotropic Banach spaces in~\cite{demers_zhang_perturbations} apart from properties (i)-(v). These properties arise naturally in situations where statistical stability is proved by Keller-Liverani perturbation theory (see e.g.~\cite{gouezel_liverani_anisotropic,demers_liverani_pw_hyperbolic}). 
\end{remark}
%\subsection{Further examples}
%	Consider the family of H\'{e}non maps $ \tf_{a,b}\map{\R^2}{\R^2} $ given by $ \tf_{a,b}(x,y)=(1-ax^2+y,bx) $ where $ a\in (1,2), b>0 $. Benedicks \& Carleson \cite{benedickscarleson91} showed that there is a positive Lebesgue measure set $ \mathcal{BC} $ such that for $ (a,b)\in \mathcal{BC} $ the map $ \tf $ has a non-hyperbolic attractor. Benedicks \& Young~\cite{benedicksyoung93,henon_young_tower} showed that for $ (a,b)\in \mathcal{BC} $ there is a unique SRB measure supported on the attractor and $ T_{a,b} $ is modelled by a Young tower with exponential tails. For $ n\in \N\cup \{\infty\} $ let $ (a_n,b_n)\in \mathcal{BC} $ with $ \lim_{n\rightarrow \infty}(a_n,b_n)=(a_\infty,b_\infty) $ and define $ \tf_n=\tf_{a_n,b_n}. $ By \cite{acfhenonss}, the family $ \tf_n $ is statistically stable so (A1) holds. Since $ T^j_n \to T^j_\infty $ uniformly for all $ j\in \N $, (A2) holds. \tcr{Language on saying effectively checked uniform family but without exact definitions}
%\begin{itemize}
%	\item Give example of Henon maps from \cite{afcyoungtowerss}. See other document for notes on how to show that condition (iii) in Definition~\ref{def:nuh_family} holds
%	\item Give example of Lorenz maps. In \cite{larkin_random_lorenz} it is checked that such maps are modelled by Young towers with uniform constants, including condition (iii) in Definition~\ref{def:nuh_family}
%	
%\end{itemize}
%\tcr{Comment about how we can cover examples from~\cite{korepanov2022deterministic} i.e.\ Viana maps and unimodal maps}
