This section is dedicated to the proof of Theorem~\ref{thm:homog_from_fcb_family}. We proceed by applying~\cite[Theorem~2.17]{chevyrev2022deterministic}. In Subsection~\ref{subsection:iwip_preliminaries}, we recall some consequences of the Functional Correlation Bound from \cite{fleming2022}, including iterated moment bounds. In Subsection~\ref{subsection:iterated_wip}, we prove the iterated WIP, which is the other main hypothesis of~\cite[Theorem~2.17]{chevyrev2022deterministic}. Finally, in Subsection~\ref{subsection:pf_main_result}, we prove Theorem~\ref{thm:homog_from_fcb_family}.
\subsection{Preliminaries}\label{subsection:iwip_preliminaries}
Let $ T:M\to M $ be a map that satisfies the Functional Correlation Bound with rate $ k^{-\gamma} $, $ \gamma > 0 $.
We repeatedly use the following weak dependence lemma.

Let $ e,q\ge 1 $ be integers. For $ G=(G_1,\dots,G_e)\map{\ms^q}{\R^e}$ and $ 0\le i<q $ we define $ \sdsemi{G}{i}=\sum_{j=1}^e \sdsemi{G_j}{i}. $ We write $ G\in \sepdholsp{q}(\ms,\R^e) $ if $ \norminf{G}+\sum_{i=0}^{q-1}\sdsemi{G}{i}<\infty. $
Let $k\ge 1$ and consider $ k $ disjoint blocks of integers $\{\ell_i,\ell_i+1,\dots,u_i\}$, $1\le i\le k$ with $ \ell_i\le u_i<\ell_{i+1}. $ Consider $ \R^e $-valued random vectors $ X_i $ on $ (\ms,\mu) $ of the form
\begin{equation*}
	X_i(x)=\varPhi_i(\tf^{\ell_i}x,\dots,\tf^{u_i}x)
\end{equation*}
where $ \varPhi_i\in \sepdholsp{u_i-\ell_i+1}(\ms,\R^e) $, $ 0\le i< k. $

When the gaps $\ell_{i+1}-u_i$ between blocks are large, the random vectors $ X_1,\dots,X_k $ are weakly dependent. Let $\widehat{X}_1,\dots,\widehat{X}_k$ be independent random vectors with $\widehat{X}_i{\eqd X_i}$.
\begin{lemma}\label{lemma:fcb_weak_dep_multidim}
Let $R=\max_i \norminf{\varPhi_i}$. Then for all Lipschitz $ F\map{B(0,R)^k}{\R} $,
	\begin{multline*}
		\big|\Ex[\mu]{F(X_1,\dots,X_k)}-\Ex{F(\shat{X}_1,\dots,\shat{X}_k)}\big|\\
		\le C\sum_{r=1}^{k-1}(\ell_{r+1}-u_r)^{-\gamma}\biggl(\norminf{F}+\Lip(F)\sum_{i=1}^{k}\sum_{j=0}^{u_i-\ell_i}\sdsemi{\varPhi_i}{j}\biggr),
	\end{multline*}
	where $ C>0 $ is the constant given by \eqref{eq:fcb_bd}.
\end{lemma}
\begin{proof}
	This is proved for $ e=1 $ in \cite[Lemma~4.1]{fleming2022}. The proof of the general case follows the same lines.
\end{proof}
\begin{lemma}\label{lemma:moment_bds}
	Let $ \gamma > 1 $. Then there exists a constant $ C>0 $ such that for all $ k\ge 1 $, for any mean zero $ v,\, w\in \holsp(\ms) $, 
	\begin{enumerate}
		\item $ \pnorm{\sum_{0\le r<k}v\circ \tf^r}_{2\gamma}\le Ck^{1/2}\dhnorm{v}. $
		\item $ \pnorm{\sum_{0\le r<s<k}v\circ \tf^r\ w\circ \tf^s}_{\gamma}\le Ck\dhnorm{v}\dhnorm{w}. $
	\end{enumerate}
\end{lemma}
\begin{proof}
	By \cite[Theorem~2.4]{fleming2022}, a functional correlation bound for separately \textit{dynamically} \holder{} functions implies analogous moment bounds for dynamically \holder{} observables. It is easily checked that the same arguments apply with separately \holder{} functions in place of separately dynamically \holder{} ones.
\end{proof}
\subsection{The Iterated WIP}\label{subsection:iterated_wip}
Let $ \gamma>1 $. Throughout this subsection, $ \tf_n,\, n\ge 1 $ is a family of maps that satisfies the \fcb{} uniformly with rate $ k^{-\gamma} $.

\sloppy Fix $ d\ge 1. $ Let $ v_n\map{\ms}{\R^d} $, $ n\ge 1 $ be a family of observables with $ \sup_{n\ge1}\dhnorm{v_n}<\infty $ and $ \int v_n\, d\mu_n=0 $. For $ t\ge 0 $ define
\[ W_n(t)=n^{-1/2}\sum_{0\le r<[nt]}v_n\circ \tf_n^r, \qquad \bbW_n(t)=n^{-1}\sum_{0\le r<s<[nt]}v_n\circ \tf_n^r\otimes v_n\circ \tf_n^s. \]
Let $ v\map{\ms}{\R^d} $ and $ k,\,n\ge 1 $. Define $ S_v(k,n)=\sum_{0\le r<k}v\circ \tf_n^r $ and $ \bbS_v(k,n)=\sum_{0\le r<s<k}v\circ \tf_n^r \otimes v\circ \tf_n^s. $
\begin{prop}\label{prop:uniform_conv_var}
	For each $ n\ge 1 $, the limits 
	\[\Sigma_n=\lim_{k\rightarrow\infty}k^{-1}\Ex[\mu_n]{S_{v_n}(k,n)\otimes S_{v_n}(k,n)}, \quad E_n=\lim_{k\rightarrow\infty}k^{-1}\Ex[\mu_n]{\bbS_{v_n}(k,n)}  \] 
	exist and are given by 
	\begin{align}\label{eq:covar_drift_formulas}
		\begin{split}
		\Sigma_n &= \Ex[\mu_n]{v_n\otimes v_n}+\sum_{\ell\ge 1}\big(\Ex[\mu_n]{v_n\otimes v_n\circ \tf_n^\ell}+\Ex[\mu_n]{v_n\circ \tf_n^\ell\otimes v_n}\! \big),\\
		E_n &= \sum_{\ell\ge 1}\Ex[\mu_n]{v_n\otimes v_n\circ \tf_n^\ell}.
		\end{split}
	\end{align}
	Moreover, the convergence is uniform in $ n $.
\end{prop}
\begin{proof}
	We prove the existence of the limit $E_n $. The proof of the existence of the limit $ \Sigma_n $ is similar. Note that
	\begin{align*}
		\Ex[\mu_n]{\bbS_{v_n}(k,n)}=\sum_{\ell=1}^{k-1}\sum_{r=0}^{k-\ell-1}\Ex[\mu_n]{v_n\otimes v_n\circ \tf_n^\ell}=\sum_{\ell=1}^{k-1}(k-\ell)\Ex[\mu_n]{v_n\otimes v_n\circ \tf_n^\ell}.
	\end{align*}
	Let $ 1\le i,j\le d $ and $ \ell\ge 1 $. Define $ G\map{\ms^2}{\R} $ by $ G(x,y)=v_n^i(x)v_n^j(y) $. By the \fcb{},
	\begin{align*}
		|\Ex[\mu_n]{v_n^i v_n^j \circ \tf_n^\ell}|&=\bigg|\int G(x,\tf_n^\ell x)d\mu_n(x)\bigg|\\
		&\ll \ell^{-\gamma}\dhnorm{v_n^i}\dhnorm{v_n^j}+\bigg|\int v_n^i d\mu_n \int v_n^j d\mu_n\bigg|=\ell^{-\gamma}\dhnorm{v_n^i}\dhnorm{v_n^j}.
	\end{align*}
	It follows that for all $ n\ge 1, $
	\begin{flalign*}
		\quad &\bigg|\sum_{\ell\ge 1}\Ex[\mu_n]{v_n\otimes v_n\circ \tf_n^\ell}-k^{-1}\Ex[\mu_n]{\bbS_{v_n}(k,n)}\!\bigg| &\\
		&\qquad\qquad\le k^{-1}\sum_{\ell=1}^{k-1}\ell\big|\Ex[\mu_n]{v_n\otimes v_n\circ \tf_n^\ell}\!\big|+\sum_{\ell\ge k}\big|\Ex[\mu_n]{v_n\otimes v_n\circ \tf_n^\ell}\!\big|\\
		& 
		\qquad\qquad\ll k^{-1}\sum_{\ell=1}^{k-1}\ell^{1-\gamma}+\sum_{\ell\ge k}\ell^{-\gamma}\ll k^{-1}(1+k^{2-\gamma})+k^{1-\gamma}=o(1),
	\end{flalign*}
	as required.
\end{proof}
We are now ready to state the main result of this subsection:
\begin{theorem}[Iterated WIP]\label{thm:iterated_wip}
	Suppose that $ \lim_{n\rightarrow\infty}\Sigma_n=\Sigma $ and $ \lim_{n\rightarrow\infty}E_n=E $. Then $ (W_n,\bbW_n)\rightarrow (W,\bbW) $ in the sense of finite-dimensional distributions, where $ W $ is a Brownian motion with covariance $ \Sigma $ and $ \bbW(t)=\int_0^t W\otimes dW+Et $. This means that for all $ \ell \ge 1 $ and $ 0\le t_1,\dots,t_\ell\le 1 $,
	\[ ((W_n,\bbW_n)(t_1),\dots,(W_n,\bbW_n)(t_\ell))\rightarrow_{\mu_n}((W,\bbW)(t_1),\dots,(W,\bbW)(t_\ell)). \]
\end{theorem}
Our proof of the Iterated WIP (Theorem~\ref{thm:iterated_wip}) is inspired by the proof of the central limit theorem in \cite[Chap.\ 7]{chernov_markarian}, which is based on Bernstein's `big block-small block' technique. Let $ 0<b<a<1 $. We split $ \{0,\dots,n-1\} $ into alternating big blocks of length $ p=[n^a] $ and small blocks of length $ q=[n^b] $. Let $ k $ denote the number of big blocks, which is equal to the number of small blocks. Then $ k=[n/(p+q)]=O(n^{1-a}). $ The last remaining block is of length at most $ p+q $.

Let $ \mathcal{B}\subset \{0,\dots,n-1\} $ denote the set of terms contained in big blocks. Let $ t\in [0,1] $. Then $ W_n(t)=I_1(t)+I_2(t) $ and $ \bbW_n(t)=J_1(t)+J_2(t)+J_3(t) $, where
\begin{align}
	\begin{split}\label{eq:iterated_big_block_cont}
		I_1(t)&=\frac{1}{n^{1/2}}\sum_{0\le r<[nt]\colon r\in \mathcal{B}}v_n\circ\tf_n^r,\qquad I_2(t)=\frac{1}{n^{1/2}}\sum_{0\le r<[nt] \colon r\notin \mathcal{B}}v_n\circ\tf_n^r,  \\
		J_1(t)&=\frac{1}{n}\sum_{0\le r<s<[nt]\colon r,s\in \mathcal{B}}v_n\circ \tf_n^r \otimes v_n\circ \tf_n^s,\\
		J_2(t)&=\frac{1}{n}\sum_{0\le r<s<[nt]\colon r\notin \mathcal{B},s\in \mathcal{B}}v_n\circ \tf_n^r \otimes v_n\circ \tf_n^s, \\ J_3(t)&=\frac{1}{n}\sum_{0\le r<s<[nt]\colon s\notin \mathcal{B}}v_n\circ \tf_n^r \otimes v_n\circ \tf_n^s.
	\end{split}
\end{align}
\begin{remark}
	In \cite[Chap.\ 7]{chernov_markarian} the central limit theorem is proved under a hypothesis on decay of multiple correlations, where the Functional Correlation Bound~\eqref{eq:fcb_bd} is only assumed for functions $ G:M^q\to \R $ of the form $ G(x_0,\dots,x_{q-1})=\prod_{i=0}^{q-1}v_i(x_i) $. This hypothesis is strong enough to control the characteristic function of $ I_1(t) $. However, functions $ G$ which are not of the above form arise naturally when we consider the characteristic function of $ J_1(t) $.
\end{remark}
We first show that the terms $ I_2(t) $, $ J_2(t) $, $ J_3(t) $ that involve small blocks can be neglected.
\begin{lemma}\label{lemma:small_blocks_negligible}
	Suppose that $ a>\frac{b+1}{2} $. Let $ t\in [0,1] $. Then $I_2(t)\rightarrow_{\mu_n}0$, $ J_2(t)\rightarrow_{\mu_n}0 $ and $ J_3(t)\rightarrow_{\mu_n}0 $ as $ n\rightarrow\infty $.
\end{lemma}
\begin{proof}
	We show that $ \pnorm{J_3(t)}_{L^1(\mu_n)}\rightarrow 0$. By the same line of argument, $ \pnorm{I_2(t)}_{L^1(\mu_n)}\conv 0 $ and $ \pnorm{J_2(t)}_{L^1(\mu_n)}\rightarrow 0$.
	
	Write $ \{0,\dots,[nt]-1\}\setminus \mathcal{B}=\bigcup_{i=1}^{k+1} C_i $ where $ C_i$ denotes the intersection of $ \{0,\dots,[nt]-1\} $ with the $ i $th small block for $ 1\le i\le k $. Also, $ C_{k+1} $ denotes the intersection of $ \{0,\dots,[nt]-1\} $ with the last remaining block. Write $ C_i=\{\ell,\ell+1,\dots,u\} $. Then
	\begin{multline*}
		\sum_{0\le r<s<[nt]\colon s\in C_i}v_n\circ \tf_n^r\otimes v_n\circ \tf_n^s=\sum_{r=0}^{\ell-1}\sum_{s=\ell}^{u} v_n\circ \tf_n^r \otimes v_n\circ \tf_n^s\\
		+\sum_{\ell\le r<s\le u}v_n\circ \tf_n^r\otimes v_n\circ \tf_n^s.
	\end{multline*}
	Hence by Lemma~\ref{lemma:moment_bds},
	\begin{flalign}
		\pnorm{\sum_{0\le r<s<[nt]\colon s\in C_i}\!\! v_n\circ \tf_n^r\otimes v_n\circ \tf_n^s}_{L^1(\mu_n)}
		&\le \pnorm{\sum_{r=0}^{\ell-1} v_n\circ \tf_n^r}_{L^2(\mu_n)}\! \pnorm{\sum_{s=\ell}^{u} v_n\circ \tf_n^s}_{L^2(\mu_n)}\nonumber\\
		&\qquad\qquad \quad+\pnorm{\sum_{\ell\le r\le s\le u}v_n\circ \tf_n^r\otimes v_n\circ \tf_n^s}_{L^1(\mu_n)}\nonumber\\
		&\ll \ell^{1/2}\# C_i^{1/2}+\# C_i\ll \# C_i^{1/2}n^{1/2}\label{eq:individual_small_block_bd}.
	\end{flalign}
	Let $ 1\le i\le k $. Then $ \#C_i\le q=[n^b] $. Also, $ k=O(n^{1-a}) $ and $ \# C_{k+1}=O(n^a) $. Thus
	\begin{align*}
		\pnorm{J_3(t)}_{L^1(\mu_n)}&\le \frac{1}{n}\sum_{i=1}^{k+1} \pnorm{\sum_{0\le r<s<[nt]\colon s\in C_i}\!\!v_n\circ \tf_n^r\otimes v_n\circ \tf_n^s}_{L^1(\mu_n)}\\
		&\ll \frac{1}{n}(n^{1-a}n^{\frac{1}{2}(b+1)}+n^{\frac{1}{2}(a+1)})\ll n^{\frac{1}{2}(b+1)-a}+n^{\frac{1}{2}(a-1)}=o(1),
	\end{align*}
	as required.
\end{proof}
For $ 1\le i\le k $ let
\begin{align*}
	X_i&=n^{-1/2}\sum_{0\le r<p}v_n\circ\tf_n^{r+(i-1)(p+q)},\\
	\bbX_i&=n^{-1}\sum_{0\le r<s< p}(v_n\circ \tf_n^r \otimes v_n\circ \tf_n^s) \circ \tf_n^{(i-1)(p+q)}.
\end{align*}
For $ 0\le t\le 1 $ define 
\[ \widetilde W_n(t)=\sum_{1\le i\le[kt]} X_i,\quad \widetilde \bbW_n(t)=\sum_{1\le i<j\le [kt]}X_i\otimes X_j. \]
\begin{prop}\label{prop:W_n_tilde_W_n_approx}
	Suppose that $ a>\frac{b+1}{2} $. Let $ t\in[0,1] $. Then 
	\[ W_n(t)-\widetilde W_n(t)\rightarrow_{\mu_n} 0,\qquad \bbW_n(t)-\widetilde \bbW_n(t)-\sum_{i=1}^{[kt]}\bbX_i\rightarrow_{\mu_n} 0. \]
\end{prop}
\begin{proof}
	Recall the definitions of $ I_1 $ and $ J_1$ from \eqref{eq:iterated_big_block_cont}. Let $ t\in [0,1] $. 
	Since $ [kt](p+q) $ is the first term of the $([kt]+1)$th big block,
	\begin{equation*}
		I_1\left(\frac{[kt](p+q)}{n}\right)=\sum_{1\le i\le[kt]}X_i,\quad J_1\left(\frac{[kt](p+q)}{n}\right)=\sum_{1\le i\le [kt]}\bbX_i+\sum_{1\le i<j\le [kt]}X_i\otimes X_j.
	\end{equation*}
	Hence
	\[ W_n([kt]\tfrac{p+q}{n})=\sum_{1\le i\le[kt]}X_i+I_2([kt]\tfrac{p+q}{n})=\widetilde{W_n}(t)+I_2([kt]\tfrac{p+q}{n}) \]
	and similarly 
	\[  \bbW_n([kt]\tfrac{p+q}{n})=\stilde \bbW_n(t)+\sum_{1\le i\le[kt]}\bbX_i+J_2([kt]\tfrac{p+q}{n})+J_3([kt]\tfrac{p+q}{n}). \]
	By Lemma~\ref{lemma:small_blocks_negligible}, it follows that $ W_n([kt]\frac{p+q}{n})-\stilde W_n(t)\rightarrow_{\mu_n} 0 $ and $ \bbW_n([kt]\tfrac{p+q}{n})-\stilde \bbW_n(t)-\sum_{i=1}^{[kt]}\bbX_i\rightarrow_{\mu_n} 0. $ It remains to show that $ \bbW_n(t)-\bbW_n([kt]\frac{p+q}{n})\rightarrow_{\mu_n} 0$ and $ W_n(t)-W_n([kt]\frac{p+q}{n}) \rightarrow_{\mu_n} 0.$ Let $ 0\le t'\le t $. Let $ C=\{[nt'],\dots,[nt]-1\} $. By~\eqref{eq:individual_small_block_bd},
	\begin{align*}
		\pnorm{\bbW_n(t)-\bbW_n(t')}_{L^1(\mu_n)}&=\pnorm{n^{-1}\sum_{0\le r<s<[nt] \colon s\in C}v_n\circ \tf_n^r \otimes v_n\circ \tf_n^s}_{L^1(\mu_n)}\\
		&\ll n^{-1/2}\# C^{1/2}\ll n^{-1/2}([nt]-[nt'])^{1/2}\\
		&\ll (n^{-1}+t-t')^{1/2}.
	\end{align*}
	Now, $ [kt]\frac{p+q}{n}=\big[\big[\frac{n}{p+q}\big]t\big]\frac{p+q}{n}\rightarrow t $ as $ n\rightarrow \infty $ so $ \pnorm{\bbW_n(t)-\bbW_n([kt]\frac{p+q}{n})}_{L^1(\mu_n)}\rightarrow 0. $ By a similar argument, $ W_n(t)-W_n([kt]\frac{p+q}{n}) \rightarrow_{\mu_n} 0.$
\end{proof}
\begin{lemma}\label{lemma:lln_diagonal}
	Suppose that $ b>\gamma^{-1} $. Let $ t\in [0,1] $. Then $ \sum_{i=1}^{[kt]}\bbX_i\rightarrow_{\mu_n} tE$ as $ n\rightarrow \infty $.
\end{lemma}
\begin{proof}First note that $ [kt]/n\sim t/p $. Hence by Proposition~\ref{prop:uniform_conv_var} and the fact that $ \lim_{n\rightarrow\infty}E_n=E $,
	\begin{align*}
		\lim_{n\rightarrow\infty}\sum_{i=1}^{[kt]}\Ex[\mu_n]{\bbX_i}&=\lim_{n\rightarrow\infty}[kt]\Ex[\mu_n]{\bbX_1}=\lim_{n\rightarrow\infty}\Ex[\mu_n]{\frac{t}{p}\sum_{0\le r<s< p}v_n\circ \tf_n^r \otimes v_n\circ \tf_n^s}\\
		&=\lim_{n\rightarrow \infty}tp^{-1}\Ex[\mu_n]{\bbS_{v_n}(p,n)} = tE.
	\end{align*}
	It remains to show that
	\[ \lim_{n\rightarrow \infty}\Ex[\mu_n]{\Bigg|\sum_{i=1}^{[kt]}(\bbX_i-\Ex[\mu_n]{\bbX_i})\Bigg|}=0. \]
	Write $ \bbX_i(x)=\varPhi_i(\tf_n^{\ell_i}x,\dots,\tf_n^{u_i}x) $ where $ \ell_i=(i-1)(p+q)$, $u_i=\ell_i+p-1$ and
	\[ \varPhi_i(y_0,\dots,y_{u_i-\ell_i})=\frac{1}{n}\sum_{0\le r<s\le u_i-\ell_i}v_n(y_r)\otimes v_n(y_s). \]
	Let $ R=\max_i \norminf{\varPhi_i} $ and define $ F\map{B_{\R^{d\times d}}(0,R)^{[kt]}}{\R} $ by $ F(z_1,\dots,z_{[kt]})=|\sum_{i=1}^{[kt]}(z_i - \Ex[\mu_n]{\bbX_i})| $. 
	
	Let $ (\widehat\bbX_i) $ be independent copies of $ (\bbX_i) $. By Lemma~\ref{lemma:fcb_weak_dep_multidim},
	\[ \Ex[\mu_n]{\bigg|\sum_{i=1}^{[kt]}(\bbX_i-\Ex[\mu_n]{\bbX_i})\bigg|}\le A+\Ex{\bigg|\sum_{i=1}^{[kt]}(\widehat\bbX_i-\Ex[\mu_n]{\bbX_i})\bigg|} \]
	where 
	\[ |A|\le C\sum_{r=1}^{[kt]-1}(\ell_{r+1}-u_r)^{-\gamma}\biggl(\norminf{F}+\Lip(F)\sum_{i=1}^{[kt]}\sum_{j=0}^{u_i-\ell_i}\sdsemi{\varPhi_i}{j}\biggr). \]
	Note that $ \norminf{\varPhi_i}\le \frac{p^2}{n}\norminf{v_n}^2 $. By a similar calculation to the bound on $ \sdsemi{\Phi_i}{j} $ in the proof of \cite[Lemma 5.5]{fleming2022}, $ \sdsemi{\varPhi_i}{j}\le \frac{1}{n}(u_i-\ell_i)\dhnorm{v_n}^2=\frac{p-1}{n}\dhnorm{v_n}^2$. 
	
	Let $ z=(z_1,\dots,z_{[kt]}), z'=(z'_1,\dots,z'_{[kt]})\in (\R^{d\times d})^{[kt]} $. Then
	\[ |F(z)-F(z')|=\bigg|\sum_{i=1}^{[kt]}z_i-z'_i \bigg|\le \sum_{i=1}^{[kt]}|z_i-z'_i|=|z-z'| \]
	so $ \Lip(F)\le 1 $. Moreover, 
	\[ \norminf{F}\le \sum_{i=1}^{[kt]}(R+|\Ex[\mu_n]{\bbX_i}|)\le \sum_{i=1}^{[kt]}(R+\norminf{\varPhi_i})\le \frac{2kp^2}{n}\norminf{v_n}^2 . \]
	Now $ pk\le n $ and $ \ell_{r+1}-u_r =  q+1\ge n^{b} $ so
	\begin{align*}
		|A|&\le Ckq^{-\gamma}\left(\frac{2kp^2}{n}\norminf{v_n}^2+\frac{kp^2}{n}\dhnorm{v_n}^2\right)\ll \frac{k^2 p^2}{n} q^{-\gamma}\le  nq^{-\gamma}\\
		&\ll n^{1-b\gamma}=o(1).
	\end{align*}
	It remains to prove that $ \Ex{|\sum_{i=1}^{[kt]}(\widehat\bbX_i-\Ex[\mu_n]{\bbX_i})|}\rightarrow 0 $. Without loss of generality take $ \gamma\le 2. $ By von Bahr-Esseen's inequality,
	\[ \Ex{\bigg|\sum_{i=1}^{[kt]}(\widehat\bbX_i-\Ex[\mu_n]{\bbX_i})\bigg|}\le \pnorm{\sum_{i=1}^{[kt]}(\widehat{\bbX}_i-\Ex[\mu_n]{\bbX_i})}_\gamma
	\ll\left(\sum_{i=1}^{[kt]}\pnorm{\widehat{\bbX}_i-\Ex{\bbX_i}}_\gamma^\gamma\right)^{1/\gamma}. \]
	Now by Lemma~\ref{lemma:moment_bds}, 
	\begin{align*}
		\pnorm{\widehat{\bbX}_i-\Ex[\mu_n]{\bbX_i}}_\gamma&=\pnorm{\bbX_i-\Ex[\mu_n]{\bbX_i}}_\gamma\le 2\pnorm{\bbX_i}_\gamma\\
		&=\frac{2}{n}\pnorm{\sum_{0\le r<s< p}v_n\circ \tf_n^r \otimes v_n\circ \tf_n^s}_\gamma=O(p/n)
	\end{align*}
	so
	\[ \Ex{\bigg|\sum_{i=1}^{[kt]}(\widehat\bbX_i-\Ex[\mu_n]{\bbX_i})\bigg|}\ll ([kt](p/n)^{\gamma})^{1/\gamma}\ll k^{(1-\gamma)/\gamma}=o(1), \]
	as required.
\end{proof}
\begin{lemma}\label{lemma:iterated_clt_non_diagonal}
	Suppose that $ a+\gamma b>2$. Then
	\[ (\widetilde W_n,\widetilde \bbW_n)\rightarrow_{\mu_n} \left(W,\int W\otimes dW\right) \text { in }D([0,1],\R^d\times \R^{d\times d}). \]
\end{lemma}
\begin{proof}
	By the portmanteau theorem \cite[p.53]{bogachev_weak_conv}, it suffices to show that 
	\[ \lim_{n\rightarrow \infty}\Ex[\mu_n]{G(\widetilde W_n,\widetilde \bbW_n)}=\Ex{G\left(W,\int W\otimes dW\right)} \]
	for all Lipschitz, bounded $ G\map{D([0,1],\R^d\times \R^{d\times d})}{\R} $. 
	
	Note that
	\[ X_i(x)=\varPhi_i(\tf_n^{\ell_i}x,\dots,\tf_n^{u_i}x) \]
	where $ \ell_i=(i-1)(p+q)$, $u_i=\ell_i+p-1 $ and \[ \varPhi_i(y_0,\dots,y_{u_i-\ell_i})=n^{-1/2}\sum_{r=0}^{u_i-\ell_i}v_n(y_r). \]
	Let $ 0\le r\le u_i-\ell_i $. Then $ \sdsemi{\varPhi_i}{r}\le n^{-1/2}\dsemi{v_n}.$ Let $ R=\max_i \norminf{\varPhi_i}\le pn^{-1/2}\norminf{v_n}.$ Define $ \pi_k\map{B(0,R)^k}{D([0,1],\R^d\times \R^{d\times d})} $ by
	\[ \pi_k(x_1,\dots,x_k)(t)=\left(\sum_{i=1}^{[kt]}x_i,\sum_{1\le i<j\le {[kt]}}x_i\otimes x_j\right) \]
	for $ t\in [0,1] $. Then $ (\stilde W_n,\stilde \bbW_n)=\pi_k(X_1,\dots,X_k) $. 
	
	Now for all $ (x_1,\dots,x_k),(x'_1,\dots,x'_k)\in B(0,R)^k $,
	\[ \sup_{t\in[0,1]}\Bigg|\sum_{i=1}^{[kt]}x_i-\sum_{i=1}^{[kt]}x'_i\Bigg|\le \sum_{i=1}^k|x_i-x_i'| \]
	and 
	\begin{align*}
		\sup_{t\in[0,1]}\Bigg|\sum_{1\le i<j\le {[kt]}}(x_i\otimes x_j-x'_i\otimes x'_j)\Bigg|
		&\le\sum_{1\le i<j\le k}|x_i \otimes x_j-x'_i\otimes x'_j|\\
		&\le \sum_{1\le i<j\le k}|x_i\otimes (x_j-x'_j)|+|(x_i-x'_i)\otimes x'_j|\\
		&\le 2kR\sum_{j=1}^{k}|x_j-x'_j|.
	\end{align*}
	Thus $ \Lip(\pi_k)\le 1+2kR. $
	
	Let $ (\shat X_i) $ be independent copies of $ (X_i) $ and define 
	\[  (\shat W_n,\shat \bbW_n)(t)=\pi_k(\shat X_1,\dots,\shat X_k)(t)=\left(\sum_{1\le i\le[kt]}\shat X_i,\sum_{1\le i<j\le {[kt]}}\shat X_i\otimes \shat X_j\right).  \]
	By Lemma~\ref{lemma:fcb_weak_dep_multidim},
	$|\Ex[\mu_n]{G(\stilde W_n,\stilde \bbW_n)}-\Ex{G(\shat W_n,\shat \bbW_n)}|\le A,$
	where
	\begin{align*}
		A &= C\sum_{r=1}^{k-1}(\ell_{r+1}-u_r)^{-\gamma}\biggl(\norminf{G\circ \pi_k}+\Lip(G\circ \pi_k)\sum_{i=1}^{k}\sum_{j=0}^{u_i-\ell_i}\sdsemi{\varPhi_i}{j}\biggr)\\
		&\le C\sum_{r=1}^{k-1}(\ell_{r+1}-u_r)^{-\gamma}\biggl(\norminf{G}+(1+2kR)\Lip(G)\sum_{i=1}^{k}\sum_{j=0}^{u_i-\ell_i}\sdsemi{\varPhi_i}{j}\biggr)\\
		&\le Ck(q+1)^{-\gamma}(\norminf{G}+(1+2kpn^{-1/2}\norminf{v_n})\Lip(G) kp n^{-1/2}\dsemi{v_n})\\
		&\ll n^{1-a}n^{-b\gamma}n=n^{2-a-b\gamma}=o(1).
	\end{align*}
	
	It remains to show that 
	\[(\shat W_n,\shat\bbW_n)\rightarrow_{\mu_n} \left(W,\int W\otimes dW\right) \text { in }D([0,1],\R^d\times \R^{d\times d}).  \]
	Indeed, once we have proved this it follows that
	\begin{align*}
		\lim_{n\rightarrow \infty}\Ex[\mu_n]{G(\stilde W_n,\stilde\bbW_n)}=\lim_{n\rightarrow \infty}\Ex{G(\shat W_n,\shat\bbW_n)}=\Ex{G\left(W,\int W\otimes dW\right)},
	\end{align*}
	completing the proof of this lemma.
	
	Note that $ n/k\sim p $. Thus by Proposition~\ref{prop:uniform_conv_var} and the fact that $ \lim_{n\rightarrow\infty}\Sigma_n=\Sigma $, 
	\begin{align*}
		\lim_{n\rightarrow \infty}\sum_{i=1}^{[kt]}\Ex{\shat X_i \otimes \shat X_i}&=\lim_{n\rightarrow \infty}[kt]\Ex{\shat X_1\otimes \shat X_1}\\
		&=\lim_{n\rightarrow \infty}\frac{[kt]}{n}\Ex[\mu_n]{\sum_{r=0}^{p-1}v_n\circ \tf_n^r\otimes \sum_{r=0}^{p-1}v_n\circ \tf_n^r}\\
		&=\lim_{n\rightarrow \infty}\frac{t}{p}\Ex[\mu_n]{S_{v_n}(p,n)\otimes S_{v_n}(p,n)}=\Sigma t.
	\end{align*}
	Hence hypothesis (i) of Lemma~\ref{lemma:lindeberg_wip} is satisfied with $ \chi_{n,i}=\shat X_i $. Now by Lemma~\ref{lemma:moment_bds},
	\begin{align*}
		\sum_{i=1}^{k}\Ex{|\shat X_i|^{2\gamma}}&=kn^{-\gamma}\Ex[\mu_n]{\bigg|\sum_{r=0}^{p-1}v_n\circ\tf_n^r\bigg|^{2\gamma}}\ll kn^{-\gamma}p^\gamma \\
		&\le k(kp)^{-\gamma}p^\gamma=k^{1-\gamma}=o(1)
	\end{align*}
	so hypothesis~(ii) of Lemma~\ref{lemma:lindeberg_wip} is satisfied with $ p=\gamma $. This completes the proof.
\end{proof}
We are now ready to prove the iterated WIP (Theorem~\ref{thm:iterated_wip}).
\begin{proof}[Proof of Theorem~\ref{thm:iterated_wip}]
	Since $ \gamma>1 $, we can choose $ 0<b<a<1 $ such that $ b>\gamma^{-1} $, $ a>\frac{b+1}{2} $ and $ a+\gamma b>2 $. Then the conditions of Proposition~\ref{prop:W_n_tilde_W_n_approx} and Lemmas~\ref{lemma:lln_diagonal} and~\ref{lemma:iterated_clt_non_diagonal} are satisfied. Let $ 0\le t_1, t_2, \dots , t_\ell\le 1$, $ \ell\ge 1 $. Write
	\[ \big((W_n,\bbW_n)(t_1),(W_n,\bbW_n)(t_2),\dots,(W_n,\bbW_n)(t_\ell)\big)=K_1+K_2+K_3, \]
	where
	\begin{align*}
		K_1&=(A(t_1),A(t_2),\dots,A(t_\ell)),\\
		K_2&=\Biggl(\biggl(0,\sum_{i=1}^{[kt_1]}\bbX_i\biggr),\biggl(0,\sum_{i=1}^{[kt_2]}\bbX_i\biggr),\dots,\biggl(0,\sum_{i=1}^{[kt_{\ell}]}\bbX_i\biggr)\Biggr),\\
		K_3&=\big((\stilde W_n,\stilde\bbW_n)(t_1),(\stilde W_n,\stilde \bbW_n)(t_2),\dots,(\stilde W_n,\stilde \bbW_n)(t_\ell)\big).
	\end{align*}
	Here $ A(t)=(W_n(t)-\stilde W_n(t),\bbW_n(t)-\stilde\bbW_n(t)-\sum_{i=1}^{[kt]}\bbX_i). $
	
	By Proposition~\ref{prop:W_n_tilde_W_n_approx}, $ K_1\rightarrow_{\mu_n} 0 $ and by Lemma~\ref{lemma:lln_diagonal}, 
	\[ K_2\rightarrow_{\mu_n} ((0,t_1 E),(0,t_2 E),\dots,(0,t_\ell E)). \]
	Moreover, by Lemma~\ref{lemma:iterated_clt_non_diagonal},
	\[ K_3 \rightarrow_{\mu_n}\! \Big(\Big(W(t_1),\int_0^{t_1}W\otimes dW\Big),\Big(W(t_2),\int_0^{t_2}W\otimes dW\Big),\ldots,\Big(W(t_\ell),\int_0^{t_\ell}W\otimes dW\Big)\Big). \]
	Hence by Slutsky's theorem,
	\[ K_1+K_2+K_3 \rightarrow_{\mu_n}\big((W,\bbW)(t_1),(W,\bbW)(t_2),\dots,(W,\bbW)(t_\ell)\big), \]
	as required.
\end{proof}
\subsection{Proof of Theorem~\ref{thm:homog_from_fcb_family}}\label{subsection:pf_main_result}
We now have all the ingredients needed to prove Theorem~\ref{thm:homog_from_fcb_family}. 
\begin{proof}[Proof of Theorem~\ref{thm:homog_from_fcb_family}]
We proceed by applying~\cite[Theorem~2.17]{chevyrev2022deterministic}, so we need to check Assumptions 2.11 and 2.12 from~\cite{chevyrev2022deterministic}. 

By the arguments in the proof of~\cite[Proposition 3.9]{korepanov2022deterministic}, Assumptions 2.11 and 2.12(ii)(a) follow from~\eqref{eq:conv_correlation_fns} and Lemma~\ref{lemma:moment_bds}. Since $ \mu_n $ is $ T_n $-invariant for all $ n $, Assumption 2.12(i) also follows from Lemma~\ref{lemma:moment_bds} (cf.\ \cite[Remark 2.13]{chevyrev2022deterministic}). 

It remains to verify Assumption 2.12(ii)(b). Let $ v_n\in \holsp(M,\R^d) $, $ n\in \N\cup\{\infty\} $, with $ \Ex[\mu_n]{v_n}=0 $ and $\sup_{n\ge1}\dhnorm{v_n}<\infty$. We assume that $ \lim_{n\to \infty}\norminf{v_n-v_\infty}=0$. Define $ \Sigma_n $ and $ E_n $ as in~\eqref{eq:covar_drift_formulas}. It suffices to prove that $ \lim_{n\to\infty}\Sigma_n=\Sigma_\infty $ and $ \lim_{n\to\infty}E_n=E_\infty $. Assumption 2.12(ii)(b) then follows from Theorem~\ref{thm:iterated_wip}, with
\[ \mathfrak{B}_1(v,w)=\Ex[\mu_\infty]{vw},\quad \mathfrak{B}_2(v,w)=\sum_{\ell\ge 1}^\infty \Ex[\mu_\infty]{vw\circ T_\infty^\ell}.  \]
We show that $ \lim_{n\to\infty}E_n=E_\infty $; the proof that $ \lim_{n\to\infty}\Sigma_n = \Sigma_\infty $ is similar. By Proposition~\ref{prop:uniform_conv_var}, the series $ E_n = \sum_{\ell\ge 1}\Ex[\mu_n]{v_n\otimes v_n\circ \tf_n^\ell} $ is convergent uniformly in $ n $. Hence it suffices to show that $ \lim_{n\rightarrow \infty}\Ex[\mu_n]{v_n\otimes v_n\circ \tf_n^\ell}=\Ex[\mu_\infty]{v_\infty\otimes v_\infty\circ \tf_\infty^\ell} $ for each fixed $ \ell\ge 1 $. Write $ \Ex[\mu_n]{v_n\otimes v_n\circ \tf_n^\ell}-\Ex[\mu_\infty]{v_\infty\otimes v_\infty\circ \tf_\infty^\ell}=A(n)+B(n) $, where
\begin{align*}
	A(n)&=\Ex[\mu_n]{v_n\otimes v_n\circ \tf_n^\ell-v_\infty\otimes v_\infty\circ \tf_n^\ell},\\
	 \quad B(n)&=\Ex[\mu_n]{v_\infty\otimes v_\infty\circ \tf_n^\ell}-\Ex[\mu_\infty]{v_\infty\otimes v_\infty\circ \tf_\infty^\ell}.
\end{align*}
Now, 
\[ \norminf{v_n\otimes v_n\circ \tf_n^\ell-v_\infty\otimes v_\infty\circ \tf_n^\ell}\le \norminf{v_n-v_\infty}\norminf{v_n}+\norminf{v_\infty}\norminf{v_n-v_\infty} \]
so $ A(n)\to 0 $ as $ n\to\infty $. Finally, by~\eqref{eq:conv_correlation_fns}, we have $ B(n)\to 0 $.
\end{proof}
%
%By (A1), $ \mu_n\wconv \mu_\infty $ and by~\cite[Corollary 2.2.10]{bogachev_weak_conv}, it follows that $ \int f\, d\mu_n\to \int f\, d\mu_\infty $ whenever $ f $ is bounded, measurable and continuous $ \mu_\infty $-a.e. Fix $ j\in \N $ and let $ v,w:M\to \R $ be continuous and bounded. By (A1), $ T_\infty $ is continuous $ \mu_\infty $-a.e. Since $ \mu_\infty $ is $ T_\infty $-invariant, it follows that $ vw\circ T_\infty^j $ is continuous $ \mu_\infty $-a.e. Hence $ \int vw\circ T_\infty^j d\mu_n\to \int vw\circ T_\infty^j d\mu_\infty $. By combining this with (A2), it follows that condition (3.2) in~\cite{korepanov2022deterministic} is satisfied.
%
%As shown in~\cite[Sect.~3.3]{korepanov2022deterministic}, Assumptions 2.11 and 2.12(ii)(a) follow from~\cite[(3.2)]{korepanov2019explicit}. 