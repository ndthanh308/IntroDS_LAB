\section{A limit theorem for triangular arrays of random vectors}
In this appendix, we state and prove an iterated WIP for triangular arrays of random vectors. Our assumptions are similar to those of Lyapunov's classical central limit theorem.
\begin{lemma}\label{lemma:lindeberg_wip}
	Fix $d\ge 1$. Let $(\chi_{n,i})_{n\ge 1,1\le i\le k_n}$ be an array of mean zero $\R^d$-valued random vectors such that $(\chi_{n,i})_{1\le i\le k_n}$ are independent for each $n\ge 1$. Suppose that 
	\begin{enumerate}[label=(\roman*)]
		\item There exists a matrix $\Sigma\in \R^{d\times d}$ such that for all $ t\in [0,1], $
		\[\lim_{n\rightarrow \infty}\Ex{\sum_{i=1}^{[tk_n]}\chi_{n,i}\otimes \sum_{i=1}^{[tk_n]}\chi_{n,i}}=t\Sigma.\]
		\item There exists $p>1$ such that $\lim_n \sum_{i=1}^{k_n}\Ex{|\chi_{n,i}|^{2p}}=0.$
	\end{enumerate}
	Define \cadlag{} processes $\shat{W}_n\in D([0,1],\R^d) ,\shat{\bbW}_n\in D([0,1],\R^{d\times d})$ by
	\begin{equation*}
		\shat W_n(t)=\sum_{1\le i\le[tk_n]}\chi_{n,i}\quad \text{and}\quad\shat{\bbW}_n(t)=\sum_{1\le i<j\le  [tk_n]}\chi_{n,i}\otimes \chi_{n,j}.
	\end{equation*}
	Then $(\shat W_n,\shat\bbW_n)\wconv (W,\int W\otimes dW)$ in $D([0,1],\R^d \times \R^{d\times d})$ with sup-norm topology, where $W$ is a Brownian motion with covariance $\Sigma$.
\end{lemma}
\begin{proof}
	For $ n\ge 1 $, $ t\in [0,1] $ let $ \mathcal{F}_t^n $ be the $ \sigma $-algebra generated by $ \chi_{n,1},\ldots,\chi_{n,[tk_n]} $. Then $ (\mathcal{F}^n_t)_{t\in[0,1]} $ forms a filtration and $ \shat{W}_n $ is an $  (\mathcal{F}^n_t)_{t\in[0,1]} $-martingale.
	
	We first show that $ \shat W_n \wconv W $ by verifying the hypotheses of \cite[Theorem~2.1]{whittmartingalefclt}. 	Let $ \varepsilon>0 $. Note that
	\begin{align}\label{eq:lindeberg_condition}
		\sum_{i=1}^{k_n}\Ex{|\chi_{n,i}|^2\indic{|\chi_{n,i}|>\varepsilon}}\le \sum_{i=1}^{k_n}\frac{1}{\varepsilon^{2p-2}}\Ex{|\chi_{n,i}|^{2p}\indic{|\chi_{n,i}|>\varepsilon}}\longrightarrow 0
	\end{align}
	by Lemma~\ref{lemma:lindeberg_wip}(ii). Now for all $ n\ge 1 $, $ 1\le i\le k_n $ we have
	\begin{align*}
		|\chi_{n,i}|^2\le \varepsilon^2 +|\chi_{n,i}|^2\indic{|\chi_{n,i}|^2>\varepsilon}.
	\end{align*}
	Hence 
	\begin{align*}
		\bigg(\mathbb{E}\bigg[\max_{1\le i\le k_n} |\chi_{n,i}|\bigg]\bigg)^2\le\Ex{\max_{1\le i\le k_n} |\chi_{n,i}|^2}&\le \varepsilon^2 + \Ex{\max_{1\le i\le k_n} |\chi_{n,i}|^2\indic{|\chi_{n,i}|^2>\varepsilon}}\\
		&\le \varepsilon^2 + \Ex{\sum_{i=1}^{k_n}|\chi_{n,i}|^2\indic{|\chi_{n,i}|>\varepsilon}}.
	\end{align*}
	Hence by \eqref{eq:lindeberg_condition},
	$ \limsup_n \Ex{\max_{1\le i\le k_n} |\chi_{n,i}|}\le \varepsilon. $ Since $ \varepsilon>0 $ is arbitrary, it follows that $ \limsup_n \Ex{\max_{1\le i\le k_n} |\chi_{n,i}|}=0$.
	
	Fix $ t\in [0,1] $ and $ 1\le a,b\le d $. Let $ [\shat{W}^a_n,\shat W_n^b](t) $ denote the quadratic covariation of $ \shat W^a_n $ and $ \shat W^b_n $ at time $ t $. By condition (i) of Lemma~\ref{lemma:lindeberg_wip},
	\begin{align}\label{eq:limiting_ex_chi_dot_chi_sum}
		\lim_{n\rightarrow \infty}\Ex{[\shat{W}^a_n,\shat W_n^b](t)}=\lim_{n\rightarrow \infty}\mathbb{E}\Biggl[\sum_{i=1}^{[tk_n]}\chi^a_{n,i} \chi^b_{n,i}\Biggr]=t\Sigma_{ab}.
	\end{align}
	Let $ 1<p<2 $ be as in Lemma~\ref{lemma:lindeberg_wip}. By von Bahr-Esseen's inequality, 
	\begin{align*}
		\Ex{\Biggl|\sum_{i=1}^{[tk_n]}(\chi^a_{n,i}\, \chi^b_{n,i}-\Ex{\chi^a_{n,i}\, \chi^b_{n,i}}) \Biggr|^{p}}&\ll \sum_{i=1}^{[tk_n]}\Ex{\big|\chi^a_{n,i}\, \chi^b_{n,i}-\Ex{\chi^a_{n,i}\, \chi^b_{n,i}}\! \big|^{p}}.
	\end{align*}
	Now, 
	\begin{align*}
		\Ex{\big|\chi^a_{n,i}\, \chi^b_{n,i}-\Ex{\chi^a_{n,i}\, \chi^b_{n,i}}\! \big|^{p}}&\le 2^{p}\Ex{|\chi^a_{n,i}\, \chi^b_{n,i}|^{p}}\le 2^{p}\Ex{|\chi_{n,i}|^{2p}}.
	\end{align*}
	By condition (ii) of Lemma~\ref{lemma:lindeberg_wip}, it follows that $ \sum_{i=1}^{[tk_n]}(\chi^a_{n,i}\, \chi^b_{n,i}-\Ex{\chi^a_{n,i}\, \chi^b_{n,i}})\rightarrow 0 $ in $ L^p $. Hence by \eqref{eq:limiting_ex_chi_dot_chi_sum}, 
	\begin{align}\label{eq:quadratic_covar_conv_lp}
		[\shat{W}^a_n,\shat W_n^b](t)=\sum_{i=1}^{[tk_n]}\chi^a_{n,i}\, \chi^b_{n,i}\longrightarrow t\Sigma_{ab}\text{ in }L^p.
	\end{align}
	Hence we have verified the hypotheses of \cite[Theorem~2.1(i)]{whittmartingalefclt}, it follows that $ \shat W_n\wconv W $ in $ D([0,1],\R^d). $
	
	We complete the proof of this lemma by verifying the hypotheses of~\cite[Theorem 2.2]{kurtzprotter} (with $ \delta=\infty $ and $ A_n\equiv 0 $). By the continuous mapping theorem, $ (\shat W_n,\shat W_n)\wconv (W,W) $ in $ D([0,1],\R^d\times \R^d)$. By \eqref{eq:quadratic_covar_conv_lp}, $ \sup_n \Ex{[\shat W_n^a,\shat W_n^a](t)}<\infty $ for all $ t\in[0,1] $ and $ 1\le a\le d $, so condition C2.2(i) in~\cite[Theorem~2.2]{kurtzprotter} is satisfied. Hence, by \cite[Theorem~2.2]{kurtzprotter} it follows that $ (\shat W_n,\shat \bbW_n)\wconv {(W,\int W\otimes dW)} $ in $ D([0,1],\R^d\times \R^{d\times d}) $ with Skorohod topology. Since $ (W, \int W\otimes dW) $ has continuous sample paths, by \cite[Section 15]{billingsley_conv_prob} we also have weak convergence in the sup-norm topology.
\end{proof}