\section{Families of nonuniformly hyperbolic maps}\label{section:families_nuh_maps}
In this section, we consider uniform families of nonuniformly hyperbolic maps. Our main result, Theorem~\ref{thm:uniform_fcb}, is that such families satisfy the \fcb{} with uniform rate. In Subsection~\ref{subsection:nuh_map_def}, we recall the definition of a nonuniformly hyperbolic map. In Subsection~\ref{subsection:explicit_fcb}, we prove some estimates for a fixed nonuniformly hyperbolic map. In Subsection~\ref{subsection:uniform_families}, we describe our uniformity criteria for families of nonuniformly hyperbolic maps and prove Theorem~\ref{thm:uniform_fcb}.
\subsection{Nonuniformly hyperbolic maps}\label{subsection:nuh_map_def}
In this subsection, we recall the notion of a nonuniformly hyperbolic map in the sense of Young \cite{young1998statistical,young1999recurrence}. The definition we use is based on \cite{korepanov2019explicit} and \cite{bmtmaps}. In particular, we do not assume uniform contraction along stable manifolds.

\quad\textbf{Gibbs-Markov maps}\quad Let \((\bar Y,\bar\mu_Y)\) be a probability space and let $\bar F\map{\bar Y}{\bar Y}$ be ergodic and measure-preserving. Let $\alpha$ be an at most countable, measurable partition of $\bar Y$. We assume that there exist constants $D_0>0,\ \theta\in (0,1)$ such that for all elements $a\in \alpha$:
\begin{itemize}
	\item (Full-branch condition) The map $\bar F|_{a}\map{a}{\bar Y}$ is a measurable bijection.
	\item For all distinct $y,y'\in \bar Y$ the separation time 
	\begin{equation*}
			s(y,y')=\inf\{n\ge 0: \bar F^n y,\, \bar F^n y' \text{ lie in distinct elements of }\alpha\}<\infty.
		\end{equation*}
	\item \label{item:GM_bdd_distortion} Define $\zeta\map{a}{\R^+}$ by $\zeta=d\bar\mu_Y/(d\, (F|_a^{-1})_* \bar\mu_Y)$. We have $ |\log \zeta(y)-\log \zeta(y')|\le D_0\theta^{s(y,y')} $ for all $ y,y'\in a $.
\end{itemize}
Then we call $\bar{F}\map{\bar Y}{\bar Y}$ a full-branch Gibbs-Markov map.

\quad\textbf{Two-sided Gibbs-Markov maps}\quad Let $(Y,d)$ be a bounded metric space with Borel probability measure $\mu_Y$ and let $F\map{Y}{Y}$ be ergodic and measure-preserving. Let $\bar{F}\map{\bar{Y}}{\bar{Y}}$ be a full-branch Gibbs-Markov map with associated measure $\bar\mu_Y$.

We suppose that there exists a measure-preserving semi-conjugacy $\bar{\pi}\map{Y}{\bar{Y}}$, so $\bar{\pi}\circ F=\bar{F}\circ \bar{\pi}$ and $\bar{\pi}_{*}\mu_Y=\bar{\mu}_Y.$ The separation time $s(\cdot,\cdot)$ on $\bar Y$ lifts to a separation time on $Y$ given by $s(y,y')=s(\bar\pi y,\bar\pi y')$. Suppose that there exist constants $D_0>0$, $\theta\in(0,1)$ such that
\begin{equation}\label{eq:two_sided_GM_contraction}
	d(F^k y,F^k y')\le D_0(\theta^k+\theta^{s(y,y')-k}) \text{ for all }y,y'\in Y,k\ge 0.
\end{equation}
Then we call $F\map{Y}{Y}$ a \textit{two-sided Gibbs-Markov map.}

\quad\textbf{One-sided Young towers}\quad Let $\bar\phi\map{\bar Y}{\Z^+}$ be integrable and constant on partition elements of $\alpha$. We define the one-sided Young tower $\bar\Delta=\bar Y^{\bar\phi}$ and tower map $\bar f\map{\bar \Delta}{\bar \Delta}$ by  
\begin{equation}\label{eq:one-sided_yt_def}
	\bar\Delta=\{(\bar y,\ell)\in \bar Y\times \Z: 0\le \ell<\bar \phi(y)\},\ \bar{f}(\bar y,\ell)=\begin{cases}
			(\bar y,\ell+1),& \ell<\bar\phi(y)-1,\\
			(\bar{F}\bar{y},0),& \ell=\bar\phi(y)-1.
		\end{cases}
\end{equation}
We extend the separation time $s(\cdot,\cdot)$ to $\bar\Delta$ by defining
\[s((\bar y,\ell),(\bar y',\ell'))=\begin{cases}
		s(\bar{y},\bar{y}'),& \ell=\ell',\\
		0,& \ell\ne \ell'.
	\end{cases}\]
	Note that for $\theta\in (0,1)$ we can define a metric by $d_\theta(\bar{p},\bar{q})=\theta^{s(\bar{p},\bar{q})}$.
	
	Now, $\bar{\mu}_\Delta=(\bar{\mu}_Y \times \text{counting})/\int_{\bar{Y}}\bar{\phi} d\bar{\mu}_Y$ is an ergodic $\bar{f}$-invariant probability measure on $\bar{\Delta}$.
	\vspace{1em}
	
	\textbf{Two-sided Young towers}\quad Let $F\map{Y}{Y}$ be a two-sided Gibbs-Markov map and let $\phi\map{Y}{\Z^+}$ be an integrable function that is constant on $\bar{\pi}^{-1}a$ for each $a\in \alpha$. In particular, $\phi$ projects to a function $\bar{\phi}\map{\bar{Y}}{M}$ that is constant on partition elements of $\alpha$.
	
	Define the one-sided Young tower $\bar{\Delta}=\bar{Y}^{\bar{\phi}}$ as in \eqref{eq:one-sided_yt_def}. Using $\phi$ in place of $\bar{\phi}$ and $F\map{Y}{Y}$ in place of $\bar{F}\map{\bar{Y}}{\bar{Y}}$, we define the \textit{two-sided Young tower} $\Delta=Y^{\phi}$ and tower map $f\map{\Delta}{\Delta}$ in the same way. Likewise, we define an ergodic $ f $-invariant probability measure on $\Delta$ by $\mu_\Delta=(\mu_Y \times \text{counting})/\int_{Y}\phi\, d\mu_Y$. 
	
	We extend $\bar{\pi}\map{Y}{\bar{Y}}$ to a map $\bar{\pi}\map{\Delta}{\bar{\Delta}}$ by setting $\bar{\pi}(y,\ell)=(\bar{\pi}y,\ell)$ for all $(y,\ell)\in \Delta$. Note that $\bar{\pi}$ is a measure-preserving semi-conjugacy; $\bar{\pi}\circ f=\bar{f}\circ \bar{\pi}$ and $\bar{\pi}_{*}\mu_\Delta=\bar{\mu}_\Delta$. The separation time $s$ on $\bar{\Delta}$ lifts to $\Delta$ by defining $s(y,y)=s(\bar{\pi}y,\bar{\pi}y').$
	
	Let $ \tf\map{\ms}{\ms} $ be a measure-preserving transformation on a probability space $ (\ms,\mu) $. Suppose that there exists $Y\subset M$ measurable with $\mu(Y)>0$ such that:
	\begin{itemize}
		\item $F=T^{\phi}\map{Y}{Y}$ is a two-sided Gibbs-Markov map with respect to some probability measure $\mu_Y$.
		\item $\phi$ is constant on partition elements of $\bar{\pi}^{-1}\alpha$, so we can define Young towers $\Delta=Y^\phi$ and $\bar\Delta=\bar{Y}^{\bar{\phi}}$.
		\item There exist constants $ D_0>0 $ and $ \theta\in(0,1) $ such that for all $ y,y'\in Y $, $ 0\le \ell <\phi(y)$ we have
		\begin{equation}\label{eq:yt_intermediate_iterate_bd}
				d(\tf^\ell y,\tf^\ell y')\le D_0(d(y,y')+\theta^{s(y,y')})
			\end{equation}
		\item The map $\pi_M\map{\Delta}{M}$, $\pi_M(y,\ell)=T^\ell y$ is a measure-preserving semiconjugacy.
	\end{itemize}
	Then we call $\tf\map{\ms}{\ms}$ a \textit{nonuniformly hyperbolic map.}
	\begin{remark}
		Note that we have not assumed that $ \gcd\{\phi(y)\colon y\in Y\}=1 $. By assuming that $ \mu $ is mixing we are able to reduce to this case, as we briefly explain in the next subsection.
	\end{remark}
	\subsection{Explicit estimates for nonuniformly hyperbolic maps}\label{subsection:explicit_fcb}
	In this subsection, we consider a fixed nonuniformly hyperbolic map $ \tf\map{\ms}{\ms} $. We assume that $ \tf $ is mixing and that there exist constants $ \beta>1 $ and $ C_\phi>0 $ such that $ \mu_Y(\phi\ge k)\le C_\phi k^{-\beta} $ for all $ k\ge 1 $.
	\begin{defn}
		For $ v\map{\ms}{\R} $ define
		\[ [v]_{\mathcal{H}}=\sup_{y\ne y'}\sup_{0\le \ell<\phi(y)}\frac{|v(\tf^\ell y)-v(\tf^\ell y')|}{d(y,y')+\theta^{s(y,y')}}. \]
	\end{defn}
	\begin{prop}\label{prop:holder_implies_dyn_holder}
		Let $ v\map{\ms}{\R} $ be Lipschitz. Then $ [v]_{\mathcal{H}}\le D_0 \Lip(v) $.
	\end{prop}
	\begin{proof}
		Let $ y,y'\in Y $, $ 0\le \ell<\phi(y) $. By \eqref{eq:yt_intermediate_iterate_bd},
		\begin{align*}
				|v(\tf^\ell y)-v(\tf^\ell y')|&\le \Lip(v) d(\tf^\ell y,\tf^\ell y')\\
				&\le D_0 \Lip(v)(d(y,y')+\theta^{s(y,y')}).\qedhere
			\end{align*}
	\end{proof}
	Let $ q\ge 1 $. Given a function $ G\map{\ms^q}{\R} $ and $ 0\le i<q $ we denote
	\[ [G]_{\mathcal{H},i}=\sup_{x_0,\dots,x_{q-1}\in \ms}[G(x_0,\dots,x_{i-1},\cdot,x_{i+1},\dots,x_{q-1})] _{\mathcal{H}}. \]
	We call $ G $ separately dynamically \holder{} if $ \norminf{G}+\sum_{i=0}^{q-1}[G]_{\mathcal{H},i}<\infty $.
	
	We are now ready to state the main result of this subsection. Note that there exist $ \delta>0 $ and a finite set $ I\subset \mathbb{N} $ with $ \gcd \{I\}=\gcd\{\phi(y):y\in Y\} $ and $ {\mu_Y(\phi=k)}\ge \delta $ for all $ k\in I $.
	\begin{lemma}\label{lemma:fcb_explicit}There exists a constant $ C>0 $ depending continuously on $ D_0,\theta, \delta$, $\max\{I\}, \beta $ and $ C_\phi$ such that 
		for all $ 0\le p<q $, $ 0\le k_0\le \cdots\le k_{q-1} $,
		\begin{align}\label{eq:dyn_holder_fcb}
			&\bigg|\int_\ms  G(\tf^{k_0}x,\dots,\tf^{k_{q-1}}x)d\mu(x)\nonumber\\
			&\qquad\qquad\qquad -\int_{\ms^2} G(\tf^{k_0}x_0,\dots,\tf^{k_{p-1}}x_0,\tf^{k_p}x_1,\dots,\tf^{k_{q-1}} x_1)d\mu(x_0)d\mu(x_1)\bigg|	\nonumber\\
			&\qquad\le C(k_p-k_{p-1})^{-(\beta-1)}\biggl(\norminf{G}+\sum_{i=0}^{q-1} [G]_{\mathcal{H},i}\biggr)
			\end{align}
		for any separately dynamically \holder{} $ G\map{\ms^q}{\R} $.
	\end{lemma}
In \cite[Theorem~2.3]{fleming2022} we proved the estimate \eqref{eq:dyn_holder_fcb} without showing that $ C $ depends continuously on the constants mentioned above. We now briefly outline how the arguments in~\cite{fleming2022} can be modified to prove Lemma~\ref{lemma:fcb_explicit}.

In \cite[Sect.~3.2]{fleming2022} the proof of~\eqref{eq:dyn_holder_fcb} is reduced to the case where $ \gcd\{\phi(y):y\in Y\}=1 $. It is easy to see that the implicit constant in \cite[Sect.~3.2]{fleming2022} depends continuously on $ \beta $ and $ \gcd\{\phi(y):y\in Y\}\le \max\{I\} $. Similarly, it is straightforward to check that most of the constants that appear in~\cite[Sects.~3.3 and 3.4]{fleming2022} can be written explicitly and vary continuously with $ D_0,\theta, \beta $ and $ C_\phi $. The sole exception to this is \cite[Lemma~3.4]{fleming2022}\footnote{There is a typo in \cite[Lemma~3.4]{fleming2022} where $ \norminf{\indicd{\bar\Delta_0}L^n v-\int_{\bar\Delta} v\, d\bar\mu_\Delta} $ should be replaced by $ \norminf{\indicd{\bar\Delta_0}\big(L^n v-\int_{\bar\Delta} v\, d\bar\mu_\Delta\big)} $.}. (Indeed, the proof of that lemma uses operator renewal theory, so it is unclear how the bound obtained depends on the data associated with $ \tf $.)

	Let $ L $ denote the transfer operator corresponding to $ f:\bar\Delta\to \bar\Delta $. Let $ \bar{\Delta}_0=\{(y,\ell)\in\bar\Delta\colon\ell=0\} $ denote the base of $ \bar\Delta $. For any $ d_\theta $-Lipschitz $ v\map{\bar\Delta}{\R} $ write $ \norm{v}_\theta=\norminf{v}+\Lip(v) $. We now complete the proof of Lemma~\ref{lemma:fcb_explicit} by providing the following uniform version of \cite[Lemma~3.4]{fleming2022}:
	\begin{lemma}Suppose that $ \gcd\{\phi(y)\colon y\in Y\}=1 $. 
	Then there exists a constant $ C>0 $ depending continuously on $ D_0,\theta, \delta, \max\{I\}, \beta $ and $ C_\phi$ such that for all $ d_\theta $-Lipschitz $ v\map{\bar\Delta}{\R} $ and for any $ k\ge 1 $,
	\[ \norminf{\indicd{\bar\Delta_0}\biggl(L^k v-\int v\, d\bar\mu_\Delta\biggr)}\le C \norm{v}_\theta k^{-(\beta-1)}. \]
\end{lemma}
	\begin{proof}
	Recall that $ L $ is given pointwise by 
	\[(Lv)(x)=\sum_{\bar fz=x}g(z)v(z), \text{ where } g(y,\ell)=\begin{cases}
		\zeta(y),& \ell=\phi(y)-1,\\
		1,& \ell<\phi(y)-1
	\end{cases}.\]
	Hence $ (L^k v)(x)=\sum_{\bar f^k z=x}g_k(z)v(z) $, where $ g_k = \prod_{i=0}^{k-1}g \circ \bar f^i $. 
	
	Throughout this proof, $ C>0 $ denotes a constant depending continuously on $ D_0,\,\theta,\, \delta,\, \max\{I\},\, \beta $ and $ C_\phi$. We first show that for all $ d_\theta $-Lipschitz $ w\map{\bar\Delta}{\R} $ and all $ n\ge 0 $,
	\begin{equation}\label{eq:sup_transfer_op_base_bd}
		\sup_{\bar\Delta_0}|L^n w|\le C(\Lip(w)|\theta^{\psi_n}|_1+|w|_1),
	\end{equation}
	where $ \psi_n(x)= \#\{j=1,\dots,n\colon \bar f^j x\in \bar\Delta_0 \}$ denotes the number of returns to $ \bar\Delta_0 $ by time $ n $.
	
	Let $ x,x'\in \bar\Delta_0 $. Then we can pair preimages $ z,z' $ of $ x,x' $ so that $ \bar f^j z, \bar f^j z' $ lie in the same partition element of $ \bar\Delta $ for $ 0\le j<n $, so $ s(z,z')=\psi_n(z')+s(x,x') $. Write $ (L^n w)(x)-(L^n w)(x')=I_1+I_2 $, where
	\begin{align*}
		I_1=\sum_{\bar f^n z=x}g_n(z)(w(z)-w(z')), \quad I_2=\sum_{\bar f^n z'=x'}w(z')(g_n(z)-g_n(z')).
	\end{align*}
	Note that
	\begin{align*}%\label{eq:transfer_op_lipschitz_part}
		|I_1|\le \Lip(w)\sum_{\bar f^n z=x}g_n(z)\theta^{\psi_n(z')}d_\theta(x,x')=\Lip(w)(L^n \theta^{\psi_n})(x')d_\theta(x,x').
	\end{align*}
	By bounded distortion (see for example~\cite[Proposition~5.2]{korepanov2019explicit}),
	\begin{align*}%\label{eq:transfer_op_bdd_distortion_part}
		|I_2|\le C\sum_{\bar f^n z'=x'}|w(z')|g_n(z')d_\theta(x,x')=C(L^n |w|)(x')d_\theta(x,x').
	\end{align*}
	It follows that
	$ |(L^n w)(x)|\le |(L^n w)(x')|+ \Lip(w)(L^n \theta^{\psi_n})(x') +C(L^n |w|)(x'). $
	Hence integrating over $ x'\in \bar\Delta_0 $ gives
	\begin{align*}
		|(L^n w)(x)|&\le \bar\mu_\Delta(\bar\Delta_0)^{-1}\int_{\bar\Delta_0}(|L^n w|+\Lip(w)|L^n\theta^{\psi_n}|+CL^n|w|)d\bar\mu_\Delta\\
		&\le \bar\mu_\Delta(\bar\Delta_0)^{-1}\bigl((1+C)\pnorm{w}_1+\Lip(w)\pnorm{\theta^{\psi_n}}_1\bigr).
	\end{align*}
	The proof of \eqref{eq:sup_transfer_op_base_bd} follows by noting that 
	$ \bar\mu_\Delta(\bar\Delta_0)^{-1}=\int_{\bar Y} \phi\, d\bar\mu_Y\le C_\phi \sum_{k\ge 1}k^{-\beta} $.
%	
%	Next we show that
%	\begin{equation}\label{eq:transfer_op_lipschitz_bd}
%		\Lip(L^n w)\le C\norm{w}_\theta \text{ for all }n\ge 0.
%	\end{equation}
%	Let $ x,x'\in \bar\Delta $. If $ d_\theta(x,x')=1 $, then
%	\begin{align}\label{eq:transfer_op_trivial_bd}
%		|(L^n w)(x)-(L^n w)(x')|\le 2\norminf{w}=2\norminf{w}d_\theta(x,x').
%	\end{align}
%	Otherwise, we can pair preimages $ z,z' $ of $ x,x'$. Then by \eqref{eq:transfer_op_lipschitz_part} and \eqref{eq:transfer_op_bdd_distortion_part},
%	\begin{align}
%		|(L^n w)(x)-(L^n w)(x')|&\le|I_1|+|I_2|\le d_\theta(x,x')(\Lip(w)+C\norminf{w} )\label{eq:transfer_op_paired_bd}.
%	\end{align}
%	The bound \eqref{eq:transfer_op_lipschitz_bd} follows by combining \eqref{eq:transfer_op_trivial_bd} and \eqref{eq:transfer_op_paired_bd}.
	
	Finally, let $ v\map{\bar\Delta}{\R} $ be $ d_\theta $-Lipschitz and let $ k\ge 1 $. Without loss of generality take $ \int v \, d\bar\mu_\Delta=0 $ and set $ w=L^{k-[k/2]}v $. By \cite[Lemma~2.2]{dedecker_prieur_intermittent}, $ \Lip(w)\le C\norm{v}_\theta$. By \eqref{eq:sup_transfer_op_base_bd}, it follows that
	\begin{align*}
		\sup_{\bar\Delta_0}|L^k v|=\sup_{\bar\Delta_0}|L^{[k/2]} w|&\le C(\Lip(w)|\theta^{\psi_{[k/2]}}|_1+|w|_1)\\
		&\le C(\norm{v}_\theta |\theta^{\psi_{[k/2]}}|_1+|L^{k-[k/2]}v|_1 ).
	\end{align*}
	Now by~\cite[Theorem~2.7]{korepanov2019explicit}, $ \pnorm{L^{k-[k/2]}v}_1\le C(k-[k/2])^{-(\beta-1)}\norm{v}_\theta $. By \cite[Lemma~5.5]{korepanov2019explicit},
	\begin{align*}
		\pnorm{\theta^{\psi_k}}_1 \le \frac{2}{\int \phi d\bar\mu_Y}\sum_{j>k/3}\bar\mu_Y(\phi>j)+k \sum_{\ell= 0}^\infty\theta^{\ell+1}\bar\mu_Y\biggl(\sum_{j=0}^{\ell-1}\phi \circ \bar F^j \ge \tfrac{1}{3}k\biggr).
	\end{align*} 
Thus $ |\theta^{\psi_{[k/2]}}|_1\le C[k/2]^{-(\beta-1)} $. It follows that
	$ \sup_{\bar\Delta_0}|L^k v|\le Ck^{-(\beta-1)}\norm{v}_\theta $,
	as required.
\end{proof}
	\subsection{Uniform families of nonuniformly hyperbolic maps}\label{subsection:uniform_families}
Let $ \tf_n\map{\ms}{\ms},\ n\ge 1 $ be a family of mixing nonuniformly hyperbolic maps as defined in Subsection~\ref{subsection:nuh_map_def} with invariant measures $ \mu_n $.
\begin{defn}\label{def:nuh_family}
	Let $ \beta>1 $. We call $ \tf_n\map{\ms}{\ms} $ a \textit{uniform family} of nonuniformly hyperbolic maps with $ O(k^{-\beta}) $ tails if:
	\begin{enumerate}
		\item The constants $ D_0>0$, $\theta\in(0,1) $ can be chosen independent of $ n\ge 1 $.
		\item There exists $ C_\phi>0 $ such that the return time functions $ \phi_n\map{Y_n}{\Z^+} $ satisfy $ \mu_{Y_n}(\phi_n\ge k)\le C_\phi k^{-\beta} $ for all $ n,k\ge 1 $.
		\item There exist $ \delta>0 $ and $ K>0 $ such that for all $ n $ there exists $ I_n\subset [1,K] $ with $ \mu_{Y_n}(\phi_n=k)\ge \delta $ for $ k\in I_n $ and $ \gcd\{I_n\}=\gcd\{\phi_n(y)\colon y \in Y_n\} $.
	\end{enumerate}
\end{defn}
We are now ready to state the main result of this section:
\begin{theorem}\label{thm:uniform_fcb}
	Let $ \tf_n $ be a uniform family of nonuniformly hyperbolic maps with  $ O(k^{-\beta}) $ tails. Then the family $ \tf_n $ satisfies the \fcb{} uniformly with rate $ k^{-(\beta-1)} $.
\end{theorem}
\begin{proof}
	Without loss of generality take $ \eta=1 $. Indeed, let $ d_\eta(x,y)=d(x,y)^\eta $. If $ \tf_n $ is a uniform family on $ (\ms,d) $, then $ \tf_n $ is a uniform family on $ (\ms,d_\eta) $ with slightly different constants, and separately $ \eta $-\holder{} functions with respect to $ d $ are separately Lipschitz with respect to $ d_\eta $.
	
By Lemma~\ref{lemma:fcb_explicit}, there exists a constant $ C>0 $ such that $ T=T_n $ satisfies \eqref{eq:dyn_holder_fcb} for all $ n\ge 1 $. Let $ G:\ms^q\to \R$, $ q\ge 1 $ be separately Lipschitz. Then by Proposition~\ref{prop:holder_implies_dyn_holder}, we have $ [G]_{\mathcal{H},i}\le D_0 [G]_{1,i} $ for all $ 0\le i<q $. Hence, for all $ n\ge 1 $, $ T_n $ satisfies the \fcb{} with constant $ C\max\{D_0,1\} $.
\end{proof}