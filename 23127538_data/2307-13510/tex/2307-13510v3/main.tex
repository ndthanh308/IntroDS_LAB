\documentclass[journal]{IEEEtran}
\usepackage{amsmath,amsfonts}
\usepackage{algorithmic}
\usepackage{algorithm}
\usepackage{array}
\usepackage{textcomp}
\usepackage{stfloats}
\usepackage{url}
\usepackage{verbatim}
\usepackage{ulem}
\usepackage{graphicx}
\usepackage{cite}
\usepackage[caption=false,font=footnotesize,labelfont=rm,textfont=rm]{subfig}
% ======
\usepackage{color}
\usepackage{lipsum}
\usepackage{multirow}
\usepackage{booktabs}
\usepackage{stfloats}
\usepackage{amssymb}
\DeclareMathOperator{\st}{s.t.}
\usepackage{cleveref}

\usepackage{titlesec}
\titleformat{\paragraph}[runin]
  {\normalfont\normalsize\bfseries}{}{0pt}{}[.\hspace{0.5em}\hspace{0.5em}]
\titlespacing{\paragraph}{0pt}{0pt}{0pt}
% \setlength{\parskip}{0pt}

% ======
% updated with editorial comments 8/9/2021
\renewcommand{\uwave}{}
\begin{document}
\title{HeightFormer: Explicit Height Modeling \\
without Extra Data for Camera-only \\
3D Object Detection in Bird’s Eye View}


\author{
  
       Yiming~Wu$^\dag$, Ruixiang~Li$^\dag$, Zequn~Qin$^{*}$, Xinhai~Zhao, Xi~Li$^{*}$

\IEEEcompsocitemizethanks{\IEEEcompsocthanksitem
Yiming~~Wu is with Polytechnic Institute, Zhejiang University, Hangzhou 310015, China.
Ruixiang~Li, Zequn~Qin, and Xi~Li are with College of Computer Science and Technology, Zhejiang University, Hangzhou 310027, China.
Xinhai~Zhao is with Noah's Ark Lab, Huawei Technologies, Shanghai 201206, China.
E-mail: \{nolva,ruixli\}@zju.edu.cn, zequnqin@gmail.com, zhaoxinhai1@huawei.com, xilizju@zju.edu.cn.
\protect
}

\thanks{($\dag$: Equal contributions. Corresponding authors: Zequn~Qin and Xi~Li.)}}

\maketitle

% \markboth{IEEE TRANSACTIONS ON IMAGE PROCESSING}{HeightFormer: Explicit Height Modeling \\
% without Extra Data for Camera-only \\
% 3D Object Detection in Bird’s Eye View}

\justify
\begin{abstract}
We propose a novel \underline{e}dge guided \underline{g}enerative \underline{a}dversarial \underline{n}etwork with \underline{c}ontrastive learning (ECGAN) for the challenging semantic image synthesis task. Although considerable improvements have been achieved by the community in the recent period, the quality of synthesized images is far from satisfactory due to three largely unresolved challenges. 1) The semantic labels do not provide detailed structural information, making it challenging to synthesize local details and structures; 2) The widely adopted CNN operations such as convolution, down-sampling, and normalization usually cause spatial resolution loss and thus cannot fully preserve the original semantic information, leading to semantically inconsistent results (e.g., missing small objects); 3) Existing semantic image synthesis methods focus on modeling ``local'' semantic information from a single input semantic layout. However, they ignore ``global'' semantic information of multiple input semantic layouts, i.e., semantic cross-relations between pixels across different input layouts. To tackle 1), we propose to use the edge as an intermediate representation which is further adopted to guide image generation via a proposed attention guided edge transfer module. Edge information is produced by a convolutional generator and introduces detailed structure information. To tackle 2), we design an effective module to selectively highlight class-dependent feature maps according to the original semantic layout to preserve the semantic information. To tackle 3), inspired by current methods in contrastive learning, we propose a novel contrastive learning method, which aims to enforce pixel embeddings belonging to the same semantic class to generate more similar image content than those from different classes.  We further propose a novel multi-scale contrastive learning method that aims to push same-class features from different scales closer together being able to capture more semantic relations by explicitly exploring the structures of labeled pixels from multiple input semantic layouts from different scales. Experiments on three challenging datasets show that our methods achieve significantly better results than state-of-the-art approaches. The source code is available at
\url{https://github.com/Ha0Tang/ECGAN}.
\end{abstract}


\begin{IEEEkeywords}
3D object detection, BEV perception, Height modeling.
\end{IEEEkeywords}


\section{Introduction}

Vision-based Bird's Eye View (BEV) representation\cite{lu2021graph,xie2023x, yang2023bevformer, bartoccioni2023lara, lin2022sparse4d} is an emerging perception formulation for autonomous driving. It transforms and maps the information from the image space to a unified 3D BEV space, which can be used for various perception tasks like 3D object detection and BEV map segmentation. Moreover, the unified BEV space can directly fuse other modalities like LiDAR without any cost, which is of great scalability.

As shown in \cref{fig:first}, the essence of BEV representation is the 2D to 3D mapping, which is a one-to-many ill-posed problem because a point in the image space corresponds to infinite collinear 3D points along the camera ray. To resolve this problem, we need to add an extra condition to make the 2D to 3D mapping a one-to-one well-posed problem. For the added extra condition, there are two kinds of methods, which are LSS\cite{philion2020lift} and OFT\cite{roddick2018orthographic}. LSS proposes to predict latent depth as the extra condition, which is implicitly estimated by end-to-end training. OFT directly maps the 2D information to 3D in the one-to-many fashion, while a network in BEV space is needed to implicitly select the dense mapped information in the vertical or height direction, which is also realized by end-to-end training. Both methods use extra depth or height conditions to resolve the mapping problem, but the extra condition is implicitly trained and used. In this way, the correctness of the mapping is not guaranteed, which might affect the performance of BEV representation.

% Figure environment removed

Motivated by the above observations, we propose to explicitly add and model extra conditions to realize better 2D to 3D mapping.
Similarly, some works\cite{park2021pseudo,li2022bevdepth} propose to directly learn depth as the extra condition with depth pre-training or LiDAR information. Different from using depth, we explicitly model the height condition in the mapping for the following reasons. First, we prove that height in the BEV space is equivalent to depth in the image space for the 2D to 3D mapping problem. Both ways can provide equivalent conditions to resolve the problem of mapping. In this way, we can realize well-defined one-to-one mapping between 2D and 3D. Second, the height information in the BEV space can be retrieved from the BEV annotations without any other data modalities like LiDAR, while depth condition needs extra pre-training or LiDAR. In this work, we use the height information from the object's 3D bounding box, which can be directly accessed from the ground truth. Third, the modeling in height can fit arbitrary camera rigs and types. For example, on NuScenes\cite{nuscenes2019}, the focal length of the backward camera is different from other cameras, resulting in different depth estimation patterns. In other words, different depth estimation network is needed for different cameras. While for the height condition, no matter which kind of camera configuration is used, it is processed with the same pattern in the BEV space. In this way, the height condition is more robust and flexible.

In this work, we propose a network that explicitly models height in the BEV space, which fulfills the condition needed for 2D to 3D mapping, termed as HeightFormer. Moreover, based on the height modeling, self-recursive height predictors are proposed to introduce the uncertainty of heights and segmentation maps which are used in the BEV query mask mechanism to produce high-precision detection results. In summary, the main contributions of our work are summarized as follows:
% \begin{itemize}
    % \item
    1) We give theoretical proof of the equivalence between height-based methods in BEV and depth-based methods in images, which is the basis of our work. The proof also demonstrates the feasibility of detection in the BEV space generated with predicted heights.
    % \item 
    2) {{\color{blue}}We propose to explicitly model heights in the BEV space without extra LiDAR supervision. A self-recursive predictor is proposed to model heights and a corresponding segmentation-based query mask is designed to handle positions whose heights cannot be defined.
    % \item 
    3) Experiments on NuScenes\cite{nuscenes2019} show that the proposed HeightFormer achieves the SOTA performance. Extensive quantitative and qualitative results show that it is feasible and effective to model heights in the BEV space and construct the BEV representation with predicted heights. The generalization analysis also shows that the proposed method can be applied to different methods, as a plugin and as compensation for depth modeling. }
% \end{itemize} 


\section{Related work}

% start of Related work
\paragraph{BEV-based camera-only 3D object detection} Most recent works in camera-only 3D object detection operate in 3D space. And there are two main methods of mapping 2D features into 3D space, which are pseudo-lidar methods\cite{wang2019pseudo, you2019pseudo, ma2020rethinking} and voxel-based BEV methods\cite{roddick2018orthographic, lang2019pointpillars, yin2021center}. Due to the efficient representation of voxels, BEV methods are commonly used now. 
%An intuitive method is to use a geometry-based method: Inverse Perspective Mapping(IPM)\cite{mallot1991inverse}.% 
The key challenge of BEV-based methods is to construct the BEV space with multi-view images. OFT\cite{roddick2018orthographic} utilizes orthographic transformation to map monocular 2D features into BEV space. LSS\cite{philion2020lift} predicts depth distribution in image space and lifts 2D features to 3D space by outer product. Following LSS, CaDDN\cite{reading2021categorical} and BEVDet\cite{huang2021bevdet} utilize the depth distribution to construct the BEV representation. 

Due to the success of transformer\cite{vaswani2017attention,brown2020language,liu2021swin}, mainstream works deal with detection tasks with a transformer-like pipeline. DETR3D\cite{wang2022detr3d} follows DETR\cite{carion2020end} to generate 3D reference points from queries. To simplify the feature sampling process in DETR3D, PETR\cite{liu2022petr,liu2022petrv2} proposes 3D positional embedding and adds the temporal information to 3D PE to align different frames. 
BEVDet\cite{huang2021bevdet} uses LSS\cite{philion2020lift} method in the BEV encoder and proposes specific data augmentation and scale-NMS tricks to improve the performance. BEVFormer\cite{li2022bevformer} introduces spatial cross-attention and temporal self-attention to generate spatiotemporal grid features from history BEV information and multi-view scenes. 
To fit the nature of the ego car's perspective, PolarFormer\cite{jiang2022polarformer} advocates the exploitation of the polar coordinate system. SOLOFusion\cite{park2022time} designs an efficient but strong temporal multi-view 3D detector to leverage long-term temporal information. TBP-Former\cite{fang2023tbp} proposes a temporal BEV pyramid transformer for spatial-temporal synchronization and BEV states prediction.


\paragraph{Depth estimation} Depth estimation is essential for 3D object detection. Due to the high similarity between depth estimation and height estimation, we learn from the experiences in depth estimation. Early works focus on geometry-based methods for stereo images\cite{scharstein2002taxonomy, flynn2016deepstereo}. In the monocular situation, there are two mainstream ways to estimate depth, which are direct regression by a well-designed network\cite{eigen2014depth, fu2018deep, xu2018multi, ding2020learning} and geometry depth derived from the pinhole imaging model\cite{cai2020monocular}. 

For the first way, many works\cite{chen2020monopair, qin2022monoground, wang2022probabilistic} adopt uncertainty to get accurate depth. Following the practice of MonoPair\cite{chen2020monopair}, most works assume that the depth follows a Laplacian distribution, and they regress it with L1 loss. 
For example, MonoFlex\cite{zhang2021objects} designs an adaptive ensemble of estimators to predict the depth of paired diagonal key points. 
% MonoRUn\cite{chen2021monorun} uses a mixed KL loss to mitigate the issue that L1 loss is not differentiable at some points.
In multi-view task, BEVDepth\cite{li2022bevdepth} proposes that the final detection loss establishes an implicit depth supervision and utilizes an explicit depth supervision with LiDAR points to enable a trustworthy depth estimation.

For the second way, due to the geometric relationship between heights and depths, many works try to model heights to estimate depths. GUPNet\cite{lu2021geometry} models the height distribution and uses 3D height $h_{3d}$ and 2D height $h_{2d}$ to express depth, which is easier than regressing depth directly. MonoRCNN\cite{shi2021geometry} decomposes distances into physical heights and the reciprocals of projected visual heights.

% Height estimation in BEV is similar to depth estimation in images, so some designs in this work are inspired by depth estimation. As the first work to
% % solve the ill-posed 2D to 3D mapping problem by explicitly modeling heights in BEV, 
% explicitly model heights in BEV, 

% % end of related work
{{\color{blue}}
\paragraph{Height modeling} Compared to depth modeling, constructing BEV features via height modeling is adopted by fewer works. BEVFormer\cite{li2022bevformer} and PolarFormer\cite{chen2022polar} obtain BEV queries by sampling 3D points around fixed heights and assigning to them different attention weights, which is an implicit way of height modeling. BEVHeight\cite{yang2023bevheight} proposes to model object heights by predicting heights for image pixels and then lifts image features to BEV space via geometric transformation, which leverages the high camera height and few occlusions in roadside situations. 

Different from previous works, we explicitly estimate heights in BEV space and focus on car-side situations, in which cameras are mounted on the car roof and occlusion is common. This work will also prove the equivalence between the height-based solution and the depth-based solution for constructing the BEV representation.}
\section{Edge Guided GANs with Contrastive Learning}

\noindent \textbf{Framework Overview.}
Figure~\ref{fig:method} shows the overall structure of ECGAN for semantic image synthesis, which consists of a semantic and edge guided generator $G$ and a multi-modality discriminator $D$.
The generator $G$ consists of eight components:
(1) a parameter-sharing convolutional encoder $E$ is proposed to produce deep feature maps $F$; 
(2) an edge generator $G_e$ is adopted to generate edge maps $I'_e$ taking as input deep features from the encoder;
(3) an image generator $G_i$ is used  to produce intermediate images $I'$;
(4) an attention guided edge transfer module $G_t$ is designed to forward useful structure information from the edge generator to the image generator;
(5) the semantic preserving module $G_s$ is developed to selectively highlight class-dependent feature maps according to the input label for generating semantically consistent images $I''$;
(6) a label generator $G_l$ is employed to produce the label from $I''$;
(7) the similarity loss is proposed to calculate the intra-class and inter-class relationships.
(8) the contrastive learning module $G_c$ aims to 
model global semantic relations between training pixels, guiding pixel embeddings towards cross-image
category-discriminative representations that eventually improve the generation performance.

Meanwhile, to effectively train the network, we propose a multi-modality discriminator $D$ that distinguishes the outputs from both modalities, i.e., edge and image.

\subsection{Edge Guided Semantic Image Synthesis}
\noindent \textbf{Parameter-Sharing Encoder.}
The backbone encoder $E$ can employ any deep network architecture, e.g., the commonly used AlexNet \cite{krizhevsky2012imagenet},
VGG \cite{simonyan2015very}, and ResNet \cite{he2016deep}. 
We directly utilize the feature maps  from the last convolutional layer as deep feature representations, i.e., $F {=} E(S)$, where $E$ represents the encoder; $S{\in} \mathbb{R}^{N \times H \times W}$ is the input label, with $H$ and $W$ as width and height of the input semantic labels, and $N$ as the total number of semantic classes.
Optionally, one can always combine multiple intermediate feature maps to enhance the feature representation.
The encoder is shared by the edge generator and the image generator.
Then, the gradients from the two generators all contribute  to updating the parameters of the encoder.
This compact design can potentially enhance the deep representations as the encoder can simultaneously learn structure representations from the edge generation branch and appearance representations from the image generation branch.

% Figure environment removed

\noindent \textbf{Edge Guided Image Generation.}
As discussed, the lack of detailed structure or geometry guidance makes it extremely difficult for the generator to produce realistic local structures and details.
To overcome this limitation, we propose to adopt the edge as guidance.
A novel edge generator $G_e$ is designed to directly generate the edge maps from the input semantic labels.
This also facilitates the shared encoder to learn more local structures of the targeted images.
Meanwhile, the image generator $G_i$ aims to generate photo-realistic images from the input labels.
In this way, the encoder is boosted to learn the appearance information of the targeted images.

 

Previous works \cite{park2019semantic,liu2019learning,qi2018semi,chen2017photographic,wang2018high} directly use deep networks to generate the target image, which is challenging since the network needs to simultaneously learn appearance and structure information from the input labels.
In contrast, the proposed method  learns structure and appearance separately via the proposed edge generator and image generator. 
Moreover, the explicit guidance from the ground truth edge maps can also facilitate the training of the encoder.
The framework of both edge and image generators is illustrated in Figure~\ref{fig:edge_block}.
Given the feature maps from the last convolutional layer of the encoder, i.e., $F {\in} \mathbb{R}^{C \times H \times W}$, 
where $H$ and $W$ are the width and height of the features, and $C$ is the number of channels, the edge generator produces edge features and edge maps which are further utilized to guide the image generator to generate the intermediate image $I'$.
The edge generator $G_e$ contains $n$ convolution layers and correspondingly produces $n$ intermediate feature maps $F_e {=} \{ F_e^j\}_{j=1}^n$.
After that, another convolution layer with Tanh non-linear activation is utilized to generate the edge map $I'_e{\in} \mathbb{R}^{3 \times H \times W}$.
Meanwhile, the feature maps $F$ is also fed into the image generator $G_i$ to generate $n$ intermediate feature maps $F_i{=}\{F_i^j\}_{j=1}^n$.
Then another convolution operation with Tanh non-linear activation is adopted to produce the intermediate image $I'_i{\in} \mathbb{R}^{3 \times H \times W}$.
In addition, the intermediate edge feature maps $F_e$ and the edge map $I'_e$ are utilized to guide the generation of the image feature maps $F_i$ and the intermediate image $I'$ via the Attention Guided Edge Transfer as detailed below.

\noindent \textbf{Attention Guided Edge Transfer.}
We further propose a novel attention guided edge transfer module $G_t$ to explicitly employ the edge structure information to refine the intermediate image representations.
The architecture of the proposed transfer module $G_t$ is illustrated in Figure~\ref{fig:edge_block}.
To transfer useful structure information from edge feature maps $F_e {=} \{ F_e^j\}_{j=1}^n$ to the image feature maps $F_i{=}\{F_i^j\}_{j=1}^n$, the edge feature maps are firstly processed by a Sigmoid activation function to generate the corresponding attention maps $F_a {=}{\rm Sigmoid}(F_e) {=} \{ F_a^j\}_{j=1}^n$.
The attention aims to provide structural information (which cannot be provided by the input label map) within each semantic class.
Then, we multiply the generated attention maps with the corresponding image feature maps to obtain the refined maps, which incorporate local structures and details.
Finally, the edge refined features are element-wisely summed with the original image features to produce the final edge refined  features, which are further fed to the next convolution layer as $F_i^j {=}{\rm Sigmoid}(F_e^j) {\times} F_i^j {+} F_i^j  (j  {=}  1, \cdots, n)$.
In this way, the image feature maps also contain the local structure information provided by the edge feature maps.
Similarly, to directly employ the structure information from the generated edge map $I_e^{'}$ for image generation, we adopt the attention guided edge transfer module to refine the generated image directly with edge information as
\begin{equation}
\begin{aligned}
I' = {\rm Sigmoid}(I'_e) \times I'_i +  I'_i,
\end{aligned}\label{eqn:image}
\end{equation}
where $I'_a{=}{\rm Sigmoid}(I'_e)$ is the generated attention map. We also visualize the results in Figure~\ref{fig:diff2}.


% Figure environment removed


\subsection{Semantic Preserving Image Enhancement}

\noindent \textbf{Semantic Preserving Module}. Due to the spatial resolution loss caused by  convolution, normalization, and down-sampling layers, existing models \cite{wang2018high,park2019semantic,qi2018semi,chen2017photographic} cannot fully preserve the semantic information of the input labels as illustrated in Figure~\ref{fig:city_seg}.
For instance, the small ``pole'' is missing, and the large ``fence'' is incomplete.
To tackle this problem, we propose a novel semantic preserving module, which aims to select class-dependent feature maps and further enhance it through the guidance of the original semantic layout. 
An overview of the proposed semantic preserving module $G_s$ is shown in Figure \ref{fig:semantic_block}(left).
Specifically, the input of the module denoted as $\mathcal{F}$,
is the concatenation of the input label $S$, the generated intermediate edge map $I'_e$ and image $I'$, and the deep feature $F$ produced from the shared encoder $E$.
Then, we apply a convolution operation on $\mathcal{F}$ to produce a new feature map $\mathcal{F}_c$ with the number of channels equal to the number of semantic categories, where each channel corresponds to a specific semantic category (a similar conclusion can be found in \cite{fu2019dual}).
Next, we apply the averaging pooling operation on $\mathcal{F}_c$ to obtain the global information of each class, followed by a  Sigmoid activation function to derive scaling factors $\gamma'$ as in $\gamma' {=} {\rm Sigmoid} ({\rm AvgPool}(\mathcal{F}_c))$, where each value represents the importance of the corresponding class. 
Then, the scaling factor $\gamma'$ is adopted to reweight the feature map $\mathcal{F}_c$ and highlight corresponding class-dependent feature maps.
The reweighted feature map is further added with the original feature $\mathcal{F}_c$ to compensate for information loss due to multiplication, and 
produces $\mathcal{F}'_c {=} \mathcal{F}_c  {\times} \gamma' {+} \mathcal{F}_c$, where $\mathcal{F}'_c {\in} \mathbb{R}^{N \times H \times W}$.

After that, we perform another convolution operation on $\mathcal{F}'_c$ to obtain the feature map $\mathcal{F}' {\in} \mathbb{R}^{(C+N+3+3) \times H \times W}$ to enhance the representative capability of the feature. In addition, $\mathcal{F}'$ has the same size as the original input one $\mathcal{F}$, which makes the module flexible and can be plugged into other existing architectures without modifications of other parts to refine the output.
In Figure~\ref{fig:semantic_block}(right), we visualize three channels in~$\mathcal{F}'$ on Cityscapes, i.e., road, car, and vegetation.
We can easily observe that each channel learns well the class-level deep
representations.

Finally, the feature map $\mathcal{F}'$ is fed into a convolution layer followed by a Tanh non-linear activation layer to obtain the final result $I''$.
Our semantic preserving module enhances the representational power of the model by adaptively recalibrating semantic class-dependent feature maps, and shares similar spirits with style transfer \cite{huang2017arbitrary}, and SENet \cite{hu2018squeeze} and EncNet \cite{zhang2018context}. 
One intuitive example of the utility of the module is for the generation of small object classes: these classes are easily missed in the generation results due to spatial resolution loss, while our scaling factor can put an emphasis on small objects and help preserve them.

\noindent \textbf{Similarity Loss.} 
Preserving semantic information from isolated pixels is very challenging for deep networks. To explicitly enforce the network to capture the relationship between semantic categories, a new similarity loss is introduced. This loss forces the network to consider both intra-class and inter-class pixels for each pixel in the label. Specifically, a state-of-the-art pretrained model (i.e., SegFormer \cite{xie2021segformer}) is used to transfer the generated image $I''$ back to a label $S'' {\in} \mathbb{R}^{N \times H \times W}$, where $N$ is the total number of semantic classes, and $H$ and $W$ represent the width and height of the image, respectively. A conventional method uses the cross entropy loss between $S''$ and $S$ to address this problem. 
However, such a loss only considers the isolated pixel while ignoring the semantic correlation with other pixels.

To address this limitation, we construct a similarity map from $S{\in} \mathbb{R}^{N \times H \times W}$. 
Firstly, we reshape $S$ to $\hat{S}{\in} \mathbb{R}^{N {\times} M}$, where $M {=} H {×}W$. Next, we perform a matrix multiplication to obtain a similarity map $A{=}\hat{S}\hat{S}^\top {\in} \mathbb{R}^{M{\times}M}$. This similarity map encodes which pixels belong to the same category, meaning that if the j-\textit{th} pixel and the i-\textit{th} pixel belong to the same category, then the value of the j-\textit{th} row and the i-\textit{th} column in $A$ is 1; otherwise, it is 0.
Similarly, we can obtain a similarity map $A''$ from the label $S''$.
Finally, we calculate the binary cross entropy loss between the two similarity maps $\{a_m {\in}A, m{\in} [1, M^2]\}$ and $\{a''_m{\in}A'', m{\in}[1, M^2]\}$ as
\begin{equation}
	\begin{aligned}
\mathcal{L}_{sim}(S, S'') = - \frac{1}{M^2} \sum_{m=1}^{M^2} (a_m \log a''_m + (1-a_m)\log (1-a''_m)).	\end{aligned}\label{eq:similarityloss}
\end{equation}
This loss explicitly captures intra-class and inter-class semantic correlation, leading to better generation results.



% Figure environment removed


\subsection{Contrastive Learning for Semantic Image Synthesis}

\noindent\textbf{Pixel-Wise Contrastive Learning.}
Existing semantic image synthesis models use deep networks to map labeled pixels to a non-linear embedding space. However, these models often only take into account the ``local'' context of pixel samples within an individual input semantic layout, and fail to consider the ``global'' context of the entire dataset, which includes the semantic relationships between pixels across different input layouts. This oversight raises an important question: what should the ideal semantic image synthesis embedding space look like? Ideally, such a space should not only enable accurate categorization of individual pixel embeddings, but also exhibit a well-structured organization that promotes intra-class similarity and inter-class difference. 
That is, pixels from the same class should generate more similar image content than those from different classes in the embedding space.
Previous approaches to representation learning propose that incorporating the inherent structure of training data can enhance feature discriminativeness. 
Hence, we conjecture that despite the impressive performance of existing algorithms, there is potential to create a more well-structured pixel embedding space by integrating both the local and global context.

The objective of unsupervised representation learning is to train an encoder that maps each training semantic layout $S$ to a feature vector $v {=} B(S)$, where $B$ represents the backbone encoder network. The resulting vector $v$ should be an accurate representation of $S$. To accomplish this task, contrastive learning approaches use a training method that distinguishes a positive from multiple negatives, based on the similarity principle between samples. The InfoNCE \cite{van2018representation,gutmann2010noise} loss function, a popular choice for contrastive learning, can be expressed as
\begin{equation}
		\mathcal{L}_S =   -\log
		\frac {\exp(v  \cdot  v_+ /  \tau  )}{
			\exp (v  \cdot  v_+  / \tau) +  \sum _ {v_- \in N_ {S}} {\exp(v\cdot v_- / \tau) }},
\end{equation}
where $v_+$ represents an embedding of a positive for $S$, and $N_S$ includes embeddings of negatives. The symbol ``·'' refers to the inner (dot) product, and $\tau {>}0$ is a temperature hyper-parameter. It is worth noting that the embeddings used in the loss function are normalized using the $L_2$ method.

One limitation of this training objective design is that it only penalizes pixel-wise predictions independently, without considering the cross-relationship between pixels. 
To overcome this limitation, we take inspiration from \cite{wang2021exploring,khosla2020supervised} and propose a contrastive learning method that operates at the pixel level and is intended to regularize the embedding space while also investigating the global structures present in the training data (see Figure \ref{fig:method_contrastive}).
Specifically, our contrastive loss computation uses training semantic layout pixels as data samples. For a given pixel $i$ with its ground-truth semantic label $c$, the positive samples consist of other pixels that belong to the same class $c$, while the negative samples include pixels belonging to other classes $C{\setminus}{c}$. As a result, the proposed pixel-wise contrastive learning loss is defined as follows
\begin{equation}
		\mathcal{L}_i =   \frac {1}{|P_ {i}|}   \sum_{i_+ \in P_i}  -\log
		\frac {\exp(i  \cdot  i_+ /  \tau  )}{
			\exp (i  \cdot  i_+  / \tau) +  \sum _ {i_- \in N_ {i}} {\exp(i\cdot i_- / \tau) }}.
\label{eq:contrastive}
\end{equation}
For each pixel $i$, we use $P_i$ and $N_i$ to represent the pixel embedding collections of positive and negative samples, respectively. Importantly, the positive and negative samples and the anchor $i$ are not required to come from the same layout. The goal of this pixel-wise contrastive learning approach is to create an embedding space in which same-class pixels are pulled closer together, and different-class pixels are pushed further apart. The result of this process is that pixels with the same class generate image contents that are more similar, which can lead to superior generation performance.


\noindent \textbf{Multi-Scale Contrastive Learning.}
In this part, we extend the pixel-level loss function $\mathcal{L}_i$ in Eq. \eqref{eq:contrastive} to an
arbitrary scale loss function $\mathcal{L}_i^s$ for calculating the contrastive learning loss, where $s$ means the $s$-\textit{th} scale feature representation, and we have a total of $\mathcal{S}$ different scales. 
This strategy regularizes the feature space of different scales by pulling features of the same class closer and pulling features of different classes apart, leading to a more well-structured feature space.

The overview framework of the proposed multi-scale contrastive learning is shown in Figure \ref{fig:method_contrastive_multi}. 
First, the input layouts go through the backbone encoder network $B$ to obtain multi-scale representation.
Next, we use a weighted sum at different scales to constraint the multi-scale features
\begin{equation}
\begin{aligned}
& \mathcal{L}_{i}^{ms}  =  \sum_{s=1}^\mathcal{S} w_s \mathcal{L}_i^s = w_1 \mathcal{L}_i^1 + \cdots + w_s \mathcal{L}_i^s + \cdots + w_\mathcal{S} \mathcal{L}_i^\mathcal{S} = \\
&  w_1 \frac {1}{|P_{i}^1|}   \sum_{i_+^1 \in P_i^1}  {-}\log \frac {\exp(i^1  \cdot  i_+^1 /  \tau  )}{\exp (i^1  \cdot  i_+^1  / \tau) +  \sum _ {i_-^1 \in N_{i}^1} {\exp(i^1\cdot i_-^1 / \tau) }} \\
& + \cdots + \\
&  w_s \frac {1}{|P_{i}^s|}   \sum_{i_+^s \in P_i^s}  {-}\log \frac {\exp(i^s  \cdot  i_+^s /  \tau  )}{\exp (i^s  \cdot  i_+^s  / \tau) +  \sum _ {i_-^s \in N_{i}^s} {\exp(i^s\cdot i_-^s / \tau) }} \\
   & + \cdots + \\
& w_\mathcal{S} \frac {1}{|P_{i}^\mathcal{S}|}   \sum_{i_+^\mathcal{S} \in P_i^\mathcal{S}} {-}\log \frac {\exp(i^\mathcal{S}  \cdot  i_+^\mathcal{S} /  \tau  )}{
\exp (i^\mathcal{S}  \cdot  i_+^\mathcal{S}  / \tau) +  \sum _ {i_-^\mathcal{S} \in N_{i}^\mathcal{S}} {\exp(i^\mathcal{S}\cdot i_-^\mathcal{S} / \tau) }}.
\label{eq:contrastive_multi}
\end{aligned}
\end{equation}
To identify the semantic classes in each pixel of different scale feature maps, we use the original input layout downsampled to the spatial dimensions.
We select the feature pairs with the same semantic label and at the same scale as positive pairs. On the contrary, we choose the feature pairs with different semantic labels and within the same scale as negative pairs.
Specifically, for each pixel $i^s$, we use $P_i^s$ and $N_i^s$ to represent the pixel embedding collections of positive and negative samples at the $s$-\textit{th} scale feature representation, respectively. Noth that the positive and negative samples and the anchor $i^s$ are from different layouts but the same scale feature embedding space. The weights $[w_1, \cdots, w_s, \cdots, w_\mathcal{S}]$ control the contribution of each scale to the overall loss.
Note that the first scale loss $\mathcal{L}_i^1$ is the same as the pixel-wise contrastive learning $\mathcal{L}_i$ in Eq. \eqref{eq:contrastive}.

As shown in Figure \ref{fig:method_contrastive_multi}, we also need to push same-class features from different scales closer together and pull different-class features apart.
For instance, if we have two scales $s_p$ and $s_q$, we hope features of the same class to be close on scales $s_p$ and $s_q$ ($s_p{\neq}s_q$), and features of different classes to be far apart on both scales $s_p$ and $s_q$.
That is, local features should describe parts of objects/regions of their global structure of the object and vice versa.
Thus the cross-scale contrastive learning loss can be formulated as
\begin{equation}
\begin{aligned}
&\mathcal{L}_{i}^{cs} =  \sum_{s_p=1}^{s_p=S} \sum_{s_q=1}^{s_q=S} w_{s_p, s_q} \mathcal{L}_i^{s_p, s_q} = \\
& w_{1, 2} \mathcal{L}_i^{1, 2} + \cdots  + w_{1, s} \mathcal{L}_i^{1, s} + \cdots + w_{1, \mathcal{S}} \mathcal{L}_i^{1, \mathcal{S}} + \cdots + w_{s, \mathcal{S}} \mathcal{L}_i^{s, \mathcal{S}}.
\end{aligned}
\label{eq:contrastive_cross}
\end{equation}
We downsample the original input layout into layouts of different scales on the spatial dimension so that we can obtain the semantic labels at each scale. We select the feature pairs with the same semantic label but at different scales as positive samples. In contrast, we select feature pairs with different semantic labels and at different scales as negative samples.
By doing so, we can achieve a bidirectional local-global consistency for learning the encoder network.
The weights $[w_{1,2}, \cdots, w_{1,s}, \cdots, w_{1,\mathcal{S}}, \cdots, w_{s, \mathcal{S}}]$ control the contribution of each scale to the overall loss.

Eq. \eqref{eq:contrastive_multi} and \eqref{eq:contrastive_cross} can be added together to obtain our complete contrastive learning loss.

\noindent \textbf{Class-Specific Pixel Generation.}
To overcome the challenges posed by training data imbalance between different classes and size discrepancies between different semantic objects, we introduce a new approach that is specifically designed to generate small object classes and fine details. Our proposed method is a class-specific pixel generation approach that focuses on generating image content for each semantic class. Doing so can avoid the interference from large object classes during joint optimization, and each subgeneration branch can concentrate on a specific class generation, resulting in similar generation quality for different classes and yielding richer local image details.

An overview of the class-specific pixel generation method is provided in Figure~\ref{fig:method_contrastive}. 
After the proposed pixel-wise contrastive learning, we obtain a class-specific feature map for each pixel. 
Then, the feature map is fed into a decoder for the corresponding semantic class, which generates an output image $\hat{I}_{i}$. 
Since we have the proposed contrastive learning loss, we can use the parameter-shared decoder to generate all classes.
To better learn each class, we also utilize a pixel-wise $L_1$ reconstruction loss, which can be expressed as $\mathcal{L}_{L_1} {=} \sum_{i=1}^{N} \mathbb{E}_{I_i, \hat{I}_i} \lbrack \vert\vert I_i {-} \hat{I}_i \vert\vert_1 \rbrack.$
The final output $I_g$ from the pixel generation network can be obtained by performing an element-wise addition of all the class-specific outputs:
$I_g {=} I_{g_1} \oplus I_{g_2} \oplus \cdots \oplus I_{g_N}.$


\subsection{Model Training}
\noindent \textbf{Multi-Modality Discriminator.}
To facilitate the training of the proposed method for high-quality edge and image generation, a novel multi-modality discriminator is developed to simultaneously distinguish outputs from two modality spaces, i.e., edge and image. 
Since the edges and RGB images share the same structure, they can be learned using the multi-modality discriminator. In the preliminary experiment, we also tried to use two discriminators (i.e., an edge discriminator and an image discriminator), but no performance improvement was observed while increasing the model complexity. Thus, we use the proposed multi-modality discriminator.
The framework of the multi-modality discriminator is shown in Figure~\ref{fig:method}, which is capable of discriminating both real/fake images and edges. 
To discriminate real/fake edges, the discriminator loss considering the semantic label $S$ and the generated edge $I'_e$ (or the real edge $I_e$) is as
\begin{equation}
\begin{aligned}
\mathcal{L}_{\mathrm{CGAN}}(G_e, D) & = 
\mathbb{E}_{S, I_e} \left[ \log D(S, I_e) \right] \\
& +  \mathbb{E}_{S, I'_e} \left[\log (1 - D(S, I'_e)) \right],
\end{aligned}
\label{eqn:discriminator1}
\end{equation}
which guides the model to distinguish real edges from fake generated edges.
Further, to discriminate real/fake images, the discriminator loss regarding  semantic label $S$ and the generated images $I'$, $I''$ (or the real image $I$) is as Eq.~\eqref{eqn:discriminator2}, which guides the model to discriminate real/fake images,
\begin{equation}
	\begin{aligned}
\mathcal{L}_{\mathrm{CGAN}}(G_i, G_s, D)  & = (\lambda + 1) \mathbb{E}_{S, I} \left[ \log D(S, I) \right]  \\
& +  \mathbb{E}_{S, I'} \left[\log (1 - D(S, I')) \right] \\
& +  \lambda \mathbb{E}_{S, I''} \left[\log (1 - D(S, I'')) \right],
\end{aligned}
\label{eqn:discriminator2}
\end{equation}
where $\lambda$ controls the losses of the two generated images. 
The inclusion of $I'$ and $I''$ is a cascaded coarse-to-fine generation strategy \cite{tang2019multi}, i.e., $I'$ is the coarse result, while $I''$ is the refined result. 
The intuition is that $I''$ will be better generated based on $I'$, so we provide $I'$ to the discriminator to ensure that $I'$ is also realistic. 

% Figure environment removed



\noindent \textbf{Optimization Objective.}
Equipped with the multi-modality discriminator, we elaborate on the training objective for the proposed method as follows.
Five different losses, i.e., the multi-modality adversarial loss, the similarity loss, the contrastive learning loss, the discriminator feature matching loss $\mathcal{L}_{f}$, and the perceptual loss $\mathcal{L}_{p}$ are used to optimize the proposed ECGAN,
\begin{equation}
\begin{aligned}
\min_{G} \max_{D} \mathcal{L} & = \lambda_{c} \underbrace{(\mathcal{L}_{\mathrm{CGAN}}(G_e, D) 
+ \mathcal{L}_{\mathrm{CGAN}}(G_i, G_s, D))}_{\text{Multi-Modality Adversarial Loss}} \\
&  + \lambda_{s} \underbrace{\mathcal{L}_{sim}(S, S')  + \mathcal{L}_{sim}(S, S'')}_{\text{Similarity Loss}} \\
& +  \lambda_{l} \underbrace{\mathcal{L}_{i}^{ms} + \mathcal{L}_{i}^{cs} + \mathcal{L}_{L_1}}_{\text{Contrastive Learning Loss}}  \\
& + \lambda_{f}\underbrace{(\mathcal{L}_{f}(I_e, I'_e) {+} \mathcal{L}_{f}(I, I') {+} \lambda \mathcal{L}_{f}(I, I''))}_{\text{Discriminator  Feature Matching Loss}} \\
& + \lambda_{p} \underbrace{(\mathcal{L}_{p}(I_e, I'_e) {+} \mathcal{L}_{p}(I, I') {+} \lambda \mathcal{L}_{p}(I, I''))}_{\text{Perceptual Loss}},
\label{eq:loss} 
\end{aligned}
\end{equation}
where $\lambda_{c}$, $\lambda_{s}$, $\lambda_{l}$, $\lambda_{f}$, and $\lambda_{p}$ are the parameters of the corresponding loss that contributes to the total loss $\mathcal{L}$;
where $\mathcal{L}_{f}$ matches the discriminator intermediate features between the generated images/edges and the real images/edges; where $\mathcal{L}_{p}$ matches the VGG extracted features between the generated images/edges and the real images/edges.
By maximizing the discriminator loss, the generator is promoted to simultaneously  generate reasonable edge maps that can capture the local-aware structure information and generate realistic images semantically aligned with the input labels.




\subsection{Implementation Details}
\label{sm:Implementation}

For both the image generator $G_i$ and edge generator $G_e$, the kernel size and padding size of convolution layers are all $3 {\times} 3$ and 1 for preserving the feature map size.
We 	set $n{=}3$ for  generators $G_i$, $G_s$, and $G_t$.
The channel size of feature $F$ is set to $C{=}64$. 
For the semantic preserving module $G_s$, we adopt an adaptive average pooling operation.
Spectral normalization \cite{miyato2018spectral} is applied to all the layers in both the generator and discriminator.
Our method incorporates the use of the Canny edge detector \cite{canny1986computational} for the purpose of deriving edge maps essential to our training process. In the subsequent testing phase, our approach necessitates no supplemental data, maintaining the fairness of comparisons with other existing methods.

\begin{table*}[!t] \small
	\centering
	\caption{User study on Cityscapes, ADE20K, and COCO-Stuff. The numbers indicate the percentage of users who favor the results of the proposed ECGAN over the competing methods.}
%		\resizebox{1\linewidth}{!}{% 
	\begin{tabular}{lccc} \toprule
		AMT $\uparrow$                               & Cityscapes & ADE20K  &  COCO-Stuff \\ \midrule
		Our ECGAN vs. CRN~\cite{chen2017photographic}     & 88.8 {$\pm$ 3.4}      & 94.8 {$\pm$ 2.7} & 95.3 {$\pm$ 2.1}\\
		Our ECGAN vs. Pix2pixHD~\cite{wang2018high}        & 87.2 {$\pm$ 2.9}       & 93.6 {$\pm$ 3.1}  &  93.9 {$\pm$ 2.4} \\ 
		Our ECGAN vs. SIMS~\cite{qi2018semi}                      & 85.3 {$\pm$ 3.8}       & -     & - \\
		Our ECGAN vs. GauGAN~\cite{park2019semantic}    & 84.7 {$\pm$ 4.3}       & 88.4 {$\pm$ 3.7}  &  90.8 {$\pm$ 2.5} \\ 
		Our ECGAN vs. DAGAN~\cite{tang2020dual}             & 81.8 {$\pm$ 3.9}       & 86.2 {$\pm$ 3.6}  & -\\
		Our ECGAN vs. CC-FPSE~\cite{liu2019learning}         & 79.5 {$\pm$ 4.2}       & 85.1 {$\pm$ 3.9}  &  86.7 {$\pm$ 2.8} \\  
		Our ECGAN vs. LGGAN \cite{tang2020local}              & 78.4 {$\pm$ 4.7}       & 82.7 {$\pm$ 4.5} & - \\
		Our ECGAN vs. OASIS \cite{sushko2020you}             & 76.7 {$\pm$ 4.8}        & 80.6 {$\pm$ 4.5}  & 82.5 {$\pm$ 3.1}  \\	\bottomrule
	\end{tabular}
	\label{tab:atm1}
		\vspace{-0.2cm}
\end{table*}

\begin{table*}[!t] \small
	\centering
	\caption{User study on Cityscapes, ADE20K, and COCO-Stuff. The numbers indicate the percentage of users who favor the results of the proposed ECGAN++ over the proposed ECGAN.}
%		\resizebox{1\linewidth}{!}{% 
	\begin{tabular}{lccc} \toprule
		AMT $\uparrow$                               & Cityscapes & ADE20K  &  COCO-Stuff \\ \midrule
		Our ECGAN++ vs. Our ECGAN \cite{tang2023edge}           & 64.3 {$\pm$ 3.2}        & 67.5 {$\pm$ 3.8}  & 70.4 {$\pm$ 2.6}  \\	\bottomrule
	\end{tabular}
	\label{tab:atm2}
		\vspace{-0.2cm}
\end{table*}

\begin{table*}[!t] \small
	\centering
	\caption{Quantitative comparison of different methods on Cityscapes, ADE20K, and COCO-Stuff.
	}
	\resizebox{1\linewidth}{!}{% 
	\begin{tabular}{rlllllllll} \toprule
		\multirow{2}{*}{Method}  & \multicolumn{3}{c}{Cityscapes} & \multicolumn{3}{c}{ADE20K} & \multicolumn{3}{c}{COCO-Stuff} \\ \cmidrule(lr){2-4} \cmidrule(lr){5-7} \cmidrule(lr){8-10} 
		& mIoU $\uparrow$    & Acc $\uparrow$  & FID  $\downarrow$ & mIoU $\uparrow$    & Acc $\uparrow$  & FID  $\downarrow$  & mIoU $\uparrow$    & Acc $\uparrow$  & FID  $\downarrow$ \\ \midrule
		CRN~\cite{chen2017photographic}  & 52.4  & 77.1 & 104.7  & 22.4 & 68.8 & 73.3 & 23.7 & 40.4 & 70.4\\
		SIMS~\cite{qi2018semi}           & 47.2  & 75.5 & 49.7 & - & - & - & - & - & - \\
		Pix2pixHD~\cite{wang2018high}    & 58.3  & 81.4 & 95.0  & 20.3 & 69.2 & 81.8  & 14.6 & 45.8 & 111.5  \\ 
  		GauGAN~\cite{park2019semantic}   & 62.3  & 81.9 & 71.8  & 38.5 & 79.9 & 33.9 & 37.4 & 67.9 & 22.6\\
         DPGAN \cite{tang2021layout} & 65.2 & 82.6 & 53.0 & 39.2 & 80.4 & 31.7 & - & - & - \\
		DAGAN \cite{tang2020dual} & 66.1 & 82.6 & 60.3 &   40.5 &  81.6 & 31.9 & - & - & -\\
            SelectionGAN \cite{tang2019multi} & 83.8 & 82.4 & 65.2 & 40.1 & 81.2 & 33.1 & - & - & -\\
            SelectionGAN++ \cite{tang2022multi} & 64.5 & 82.7 & 63.4 & 41.7 & 81.5 & 32.2 & - & - & - \\
		LGGAN \cite{tang2020local} & 68.4 & 83.0 & 57.7 & 41.6 & 81.8 & 31.6 & - & - & -\\
            LGGAN++ \cite{tang2022local}  & 67.7 & 82.9 & 48.1 & 41.4 & 81.5 & 30.5 & - & - & - \\
		CC-FPSE~\cite{liu2019learning}  & 65.5 & 82.3 & 54.3 & 43.7 & 82.9 & 31.7 & 41.6 & 70.7 & 19.2 \\
            \hao{SCG \cite{wang2021image}} & \hao{66.9} & \hao{82.5} & \hao{49.5} & \hao{45.2} & \hao{83.8} & \hao{29.3} & \hao{42.0} & \hao{72.0} & \hao{18.1}\\
		OASIS \cite{sushko2020you} &   69.3 & - & 47.7 & 48.8 & - & 28.3 & 44.1 & - & 17.0  \\
		\hao{RESAIL \cite{shi2022retrieval}} & \hao{69.7} & \hao{83.2}
		& \hao{45.5} & \hao{49.3} & \hao{84.8} &\hao{30.2} & \hao{44.7} & \hao{73.1} & \hao{18.3} \\
		\hao{SAFM \cite{lv2022semantic}} &\hao{70.4} & \hao{83.1} &\hao{49.5} &\hao{50.1}&\textbf{\hao{86.6}}&\hao{32.8}& \hao{43.3} & \textbf{\hao{73.4}} & \hao{24.6} \\
            \hao{PITI \cite{wang2022pretraining}} & \hao{-} & \hao{-} & \hao{-} & \hao{-} & \hao{-} & \hao{-} & \hao{-} & \hao{-} & \hao{19.36}\\
            \hao{T2I-Adapter \cite{mou2023t2i}} & \hao{-} & \hao{-} & \hao{-} & \hao{-} & \hao{-} & \hao{-} & \hao{-} & \hao{-} & \hao{16.78}\\
            \hao{SDM \cite{wang2022semantic}} & \hao{-} & \hao{-} & \hao{\textbf{42.1}} & \hao{-} & \hao{-} & \hao{27.5} & \hao{-} & \hao{-} & \hao{15.9}\\
		ECGAN (Ours)                        & 72.2    & 83.1 & 44.5 & 50.6 & 83.1 & 25.8 & 46.3 & 70.5 & 15.7 \\
		ECGAN++ (Ours) & \textbf{73.3} (+1.1) & \textbf{83.9} (+0.8) & 42.2 (-2.3) & \textbf{52.7} (+2.1) & 85.9 (+2.8) & \textbf{24.7} (-1.1) & \textbf{47.9} (+1.6) & 72.3 (+1.8) & \textbf{14.9} (-0.8) \\
		\bottomrule
	\end{tabular}}
	\label{tab:sota}
	\vspace{-0.4cm}
\end{table*}

In our computation of the contrastive learning loss, we observe a direct correlation between the number of layouts used and the resultant performance, i.e., more layouts lead to enhanced performance. However, a plateau is reached when the count exceeds 8 layouts; additional layouts contribute only marginal improvements to performance, while significantly slowing down the overall training process. Thus, with the objective of striking a balance between performance efficiency and computational time, we elect to use 8 layouts as input for the calculation of contrastive learning loss.
We use features from four scales in Eq. \eqref{eq:contrastive_multi}, with feature map output strides of 1, 4, 8, and 16, to calculate the multi-scale contrastive learning loss. 
Meanwhile, we also downsample the input layout by 4, 8, and 16 times to obtain the label of the corresponding scale for calculating the multi-scale contrastive learning loss.
The weights $w_s$ in Eq. \eqref{eq:contrastive_multi} are set to  1, 0.7, 0.4, and 0.1 in a decreasing way for feature maps of strides 1, 4, 8, and 16, respectively.
Moreover, in order to balance the performance and efficiency, we adopt two cross-scale contrastive learning in Eq. \eqref{eq:contrastive_cross}, i.e., (s4, s8) and (s4, s16).
We set both weights in Eq. \eqref{eq:contrastive_cross} to 0.1.

Also, we follow the training procedures of GANs \cite{goodfellow2014generative} and alternatively train the generator $G$ and discriminator $D$, i.e., one gradient descent step on the discriminator and generator alternately. 
We use the Adam solver \cite{kingma2014adam} and set $\beta_1{=}0$, $\beta_2{=}0.999$.
$\lambda_{c}$,  $\lambda_{s}$, $\lambda_{l}$, $\lambda_{f}$, and $\lambda_{p}$ in Eq.~\eqref{eq:loss} is set to 1, 1, 1, 10, and 10, respectively.
All $\lambda$ in both Eq.~\eqref{eqn:discriminator2} and \eqref{eq:loss} are set to 2.
We conduct experiments on an NVIDIA DGX1 with 8 V100 GPUs. 

\section{Experiments}


\subsection{Experimental details}
\paragraph{Dataset} The proposed method is evaluated on the NuScenes\cite{nuscenes2019} detection dataset.
As a challenging dataset for autonomous driving, NuScenes provides data of 1k scenes, from a sensor suite that consists of 6 cameras, 1 LiDAR, 5 RADAR, 1 GPS, and 1 IMU. The {\it train} set contains 28,130 samples and the \uwave{{\it validation} set} contains 6,019 samples.

\paragraph{Metrics} The main metrics that reflect the performance are mAP and NDS. mAP is widely used in detection tasks. NDS stands for NuScenes Detection Score, which consists of six sub-metrics: mAP, mATE, mASE, mAOE, mAVE, and mAAE. NDS is computed as:
\begin{equation}
    \text{NDS}=\cfrac{1}{10}[5\text{mAP}+\sum_{\text{mTP}\in \mathbb{TP}}[1-\min(1, \text{mTP})],
    \label{eq:nds}
\end{equation}
where $\mathbb{TP}$ is the set of sub-metrics expect mAP. mATE, mASE, and mAOE measure the localization error, the scale error, and the orientation error respectively. mAVE measures velocity prediction error. mAAE measures the error of object classification. For NDS and mAP, higher is better. For other sub-metrics, the opposite is true.

\paragraph{Settings} As our model is built based on BEVFormer\cite{li2022bevformer}, we share most settings with BEVFormer. \uwave{For time fusion module, 4 history BEV queries are used in the {\it base} setting and 3 BEV queries are used in the {\it tiny} setting.} The BEV space is divided into 200$\times$200 grids in both {\it base} and {\it tiny} settings, with each grid standing for an area of 0.512 meters by 0.512 meters. We supervise the height predictor of the last encoder layer with the loss function stated in \cref{eq:loss0}. The extra segmentation map is supervised with binary focal loss\cite{lin2017focal}. A position-aware loss weight called BEV centerness\cite{xie2022m} which pays more attention to distant areas is also used to improve the height estimation accuracy of distant grids.
% It is calculated as 
% \begin{equation}
%     C(x,z) = 1 + \sqrt{\frac{x^2+z^2}{X^2+Z^2}}.
%     \label{eq:bevc}
% \end{equation}
% Here, $X$ and $Z$ stand for the range of BEV space, and they are both equal to 51.2 meters.


\subsection{Ablation study}


\paragraph{Performance upper bound} We first replace the height predictors with ground truth heights to evaluate whether accurate height estimation could improve the performance of detection, and we call this method HeightFormer-gt. For BEV girds containing objects, predicted heights are replaced with ground truth heights. For others, predicted heights are replaced with $y_{xz}=0.5$ and $h_{xz}=1.0$ (normalized), which degrades into the situation of BEVFormer. Results are shown in \cref{tab:expa0}.

\begin{table}[hbt]
    \centering
    \caption{The upper bound of performance. The anchor heights of HeightFormer-gt are uniformly distributed in the range of $[y_{gt} - h_{gt}/2, y_{gt} + h_{gt}/2 ]$. *: The experiments about Lift-Splat\cite{philion2020lift} is done by BEVDepth\cite{li2022bevdepth}.}
    \begin{tabular}{lccc}
        \toprule
        Model             & Condition & NDS$\uparrow$ & mAP$\uparrow$ \\
        \midrule
        Lift-Splat$^*$    & -         & 0.327         & 0.282         \\
        Lift-Splat-gt$^*$ & depth     & {\bf 0.515}         & {\bf 0.470}         \\
        \midrule
        BEVFormer         & -         & 0.517         & 0.416         \\
        HeightFormer-gt      & height    & {\bf 0.725}   & {\bf 0.789}   \\
        \bottomrule
    \end{tabular}
    \label{tab:expa0}
\end{table}


As shown in \cref{tab:expa0}, HeightFormer-gt exceeds BEVFormer by 0.208 in NDS and 0.374 in mAP, which can be considered as the performance upper bound. 
\uwave{However, in reality, height or depth estimation error could be amplified when considering the final detection error. As a result, it is not feasible to achieve this theoretical upper bound.}
In the meanwhile, Lift-Splat\cite{philion2020lift} in \cref{tab:expa0} is a method that constructs the BEV space with depth modeling and according to BEVDepth\cite{li2022bevdepth} introducing ground truth depth improves mAP of the Lift-Splat style detector by 0.188. The experiments in \cref{tab:expa0} show that there is room for improvement whether in depth modeling or height modeling. 


Besides, it should be noted that the two types of methods, LSS and BEVFormer, cannot be compared directly as they have different network architectures. \uwave{We list these results here to show the potential of height-based methods.}

\paragraph{Effectiveness of explicit height modeling} We present the vanilla HeightFormer which turns fixed reference points into adaptive reference points. The adaptive reference points are generated with predicted anchor heights. In this way, height information is explicitly learned by the height predictor. To show the effectiveness of explicit height modeling, it is compared with BEVFormer which encodes height information into attention weights of spatial cross-attention in an implicit way. The vanilla HeightFormer simply has a multilayer perceptron as the height predictor. Results are shown in \cref{tab:expa1}. \uwave{Explicit height modeling brings a gain of 0.5 percentage points in both NDS and mAP for the ``base'' model. For the ``tiny'' model, the improvements come to 1.8 and 1.1 percentage points.
% We also list the results of BEVHeight, which is also a height-based method. The proposed style of height modeling outperforms it on mAP by 0.8 percentage points.
}


% For the supervision of height prediction, we do not take the . One only supervises the height predictions of the last encoder layer, and the other supervises all encoder layers by averaging height prediction losses of them. The result is listed in \cref{tab:expa1}. Supervising the last layer works better here. A potential reason is that it is difficult for shallow layers to estimate heights, and thus these layers might not learn well. In all the following experiments, we will adopt the practice of only supervising the height predictor of the last encoder layer. %Besides NDS and mAP, we notice that there is an improvement of 4.5\% in mAVE, which measures the performance of velocity estimation. Theoretically, explicit height supervision of two successive frames indirectly provides velocity information, which helps the model learn velocity estimation.

\begin{table}[hbt]
    \centering
    \caption{Ablation study on height modeling. *: Vanilla HeightFormer which predicts heights in a standalone way. $\dag$: Official results, no history frames.}
    \begin{tabular}{lcccc}
        \toprule
        Model    & Config  & Height   & NDS$\uparrow$ & mAP$\uparrow$ \\
        \midrule
        BEVFormer & Base & implicit & 0.517         & 0.416         \\
        % \midrule
        HeightFormer* & Base & explicit & {\bf 0.522}         & {\bf 0.421}     \\
        % \,+self-recursive & explicit & {\bf 0.525}   & {\bf 0.422}   \\
        \midrule
        BEVFormer     & Tiny & implicit & 0.403         & 0.288         \\
        % BEVHeight$^\dag$     & ResNet50 & explicit & 0.342 & 0.291 \\
        HeightFormer* & Tiny & explicit & {\bf 0.421}   & {\bf 0.299}     \\
        \bottomrule
    \end{tabular}
    \label{tab:expa1}
\end{table}


\paragraph{Effectiveness of the network design}
Based on the vanilla HeightFormer which predicts heights standalone in each layer, height embeddings are added to formulate self-recursive predictors. For BEV grids that do not cover any object, the query mask is applied. Two types of masks are designed for comparison. The uncertainty-based mask filters BEV queries that have high uncertainties of the predicted heights. The segmentation-based mask filters BEV queries that are less likely to have objects. Results are shown in \cref{tab:expa3}.

The self-recursive way of predicting heights brings an improvement of 0.3 percentage points in NDS. Besides, we observe that there is a giant gap between the outputs of successive predictors when they are standalone, and refining the heights layer by layer in a self-recursive way mitigates this issue. 

The two types of masks both improve the NDS, while the segmentation-based mask brings a gain of 0.5  percentage points in mAP. The improvement mainly comes from filtering irrelevant features and introducing segmentation as an auxiliary task.
In the proposed method, there is no proper way to define the heights of BEV grids that do not cover any object. As a result, the predicted heights in these grids are not explainable and the sampled features are from irrelevant background areas. Applying a mask mitigates this issue.


\begin{table}[hbt]
    \centering
    \caption{Ablation study on the effectiveness of the network design. SR: a self-recursive way of predicting heights. Unc.M: uncertainty-based mask. Seg.M: segmentation-based mask.}
    \begin{tabular}{ccc|cc}
        \toprule
        SR         & Unc.M      & Seg.M      & NDS$\uparrow$ & mAP$\uparrow$ \\
        \midrule
                   &            &            & 0.522         & 0.421         \\
        \checkmark &            &            & 0.525         & 0.422         \\
        \checkmark & \checkmark &            & {\bf 0.528}   & 0.424         \\
        \checkmark &            & \checkmark & 0.527         & {\bf 0.427}   \\
        \bottomrule
    \end{tabular}
    \label{tab:expa3}
\end{table}


\paragraph{Robustness of height modeling} 
% Compared with depth modeling in the image view, modeling heights can fit arbitrary camera rigs and types. In the NuScenes dataset, there is a great manufacturing difference between the back camera and all other cameras. We introduce ``mAP-cam'' which evaluates the detection performance in a single camera. It can be calculated by filtering the ground truth and predicted bounding boxes whose centers are not seen by the camera. We choose BEVDepth\cite{li2022bevdepth} for comparison. BEVDepth takes camera parameters as the input of its network. Results are shown in \cref{tab:expa4}. HeightFormer-tiny shows better performance on the back camera compared to BEVDepth-r50, although HeightFormer-tiny has a lower overall mAP. This demonstrates the advantage of height modeling in fitting different cameras.
As described in Introduction, although both depth and height modeling can provide an extra condition for 2D to 3D mapping, height modeling has a unique advantage in that it can process any camera rig. This is because estimating depth in the image space might be influenced by the camera rigs, but estimating height is simple and robust since all information has been mapped to the unified BEV space. \uwave{To verify this property, we test our method with depth modeling specifically with the back camera, which has a different focal length from other cameras.} We use BEVDepth\cite{li2022bevdepth} as the depth modeling method. Results are shown in \cref{tab:expa4}. 


\begin{table}[htb]
    \centering
    \caption{Ablation study on the robustness of height modeling. Although the depth modeling method has higher overall performance (due to extra LIDAR data), our method still achieves higher performance on the back camera which has a different focal length and camera configuration. This shows the robustness of height modeling.}
    \begin{tabular}{c|cc|cc}
        \toprule
        \multirow{2}{*}{Extra condition}
                  & \multicolumn{2}{c|}{Overall} & \multicolumn{2}{c}{Back Cam}  \\
         & mAP & NDS & mAP & NDS  \\
        \midrule
        Depth modeling   & {\bf 0.330}  &  {\bf  0.436}   & 0.273  &   0.398             \\
        Height modeling & 0.299     & 0.421 & {\bf 0.279}   &  {\bf  0.413}  \\
        \bottomrule
    \end{tabular}
    \label{tab:expa4}
\end{table}


From \cref{tab:expa4} we can see that although BEVDepth has higher overall performance \uwave{(due to extra LiDAR data training and different model architectures)}, height modeling's performance on the back camera is higher, which highlighted a great performance gap between the back camera and other cameras for BEVDepth. This means height modeling is more robust to different camera rigs, even when the cameras' focal lengths are different.

\subsection{LiDAR supervision}
% Figure environment removed

 Ground truth heights are critical for the HeightFormer module. However, with the bounding boxes of objects, we can only obtain the heights of grids that have objects, which limits the performance of the proposed method. To show how much further improvement can be achieved with extra LiDAR supervision, we introduce LiDAR supervision in this part.

First, LiDAR points are projected into BEV grids with the height range meshed into 16 intervals. For each BEV grid, its height $y_{xz}$ is the height of the interval containing the most points. Its lower bound is the height of the lowest point, which formulates $y_{xz} - h_{xz} / 2$. Furthermore, heights from LiDAR points and heights from bounding boxes are fused to formulate the ground truth heights. For each BEV gird, the height from ground truth bounding boxes is preferred. The fused heights serving as the ground truth heights are shown in \cref{fig:fused}.

With the extra supervision of LiDAR information, the performance is improved by 0.3 percentage points. The ablation study is shown in \cref{tab:explidar}. 

\begin{table}[htb]
    \centering
    \caption{Ablation study on LiDAR supervision. The models here take ResNet50 as the backbone and the BEV space is meshed into 200 grids by 200 grids. }
    \begin{tabular}{lccc}
        \toprule
        Model     & Height           & NDS$\uparrow$ & mAP$\uparrow$ \\
        \midrule
        BEVFormer & implicit         & 0.403         & 0.288         \\
        % \midrule
        HeightFormer & explicit         & 0.421         & 0.299         \\
        HeightFormer & explicit + lidar & {\bf 0.428}         & {\bf 0.307}         \\
        \bottomrule
    \end{tabular}
    \label{tab:explidar}
\end{table}

We can note the extra LiDAR supervision brings little improvement to the performance of the proposed method compared with the improvement brought by explicit height modeling. \uwave{A potential reason is that LiDAR points are sparse and occluded positions are not covered. This might yield inconsistent supervision information. This is different from the situation of depth-modeling, in which most pixels can be paired with one or more LiDAR points. However, the advantage of our method is that, it is a cost-free approach for improving the performance and requires no extra data.}

\subsection{Generalization ability}
In this part, we will show that the proposed HeightFormer can also serve as a plugin to refine other types of BEV representations. The pipeline is shown in \cref{fig:refine}.

% Figure environment removed


Taking BEVDepth as an example, it lifts image features with predicted depth distribution into voxel features and splats voxel features into BEV features with voxel pooling. Here we make HeightFormer a plugin of BEVDepth and improve the performance of BEVDepth with the plugin. The results are shown in \cref{tab:expgen}.

\begin{table}[htb]
    \centering
    \caption{Performance of height-based BEV feature refinement. The baseline models both take ResNet50 as the backbone.}
    \begin{tabular}{lccc}
        \toprule
        Model     & Refinement & NDS$\uparrow$ & mAP$\uparrow$ \\
        \midrule
        BEVFormer &            & 0.403         & 0.288         \\
        BEVFormer & 3 layers & {\bf 0.421}         & {\bf 0.299}         \\
        \midrule
        BEVDepth  &            & 0.366         & {\bf 0.273}             \\
        % \midrule
        BEVDepth  & 1 layer & {\bf 0.369}         & {\bf 0.273}             \\
        \bottomrule
    \end{tabular}
    \label{tab:expgen}
\end{table}

We first train a BEVDepth model without BEV data augmentation for 20 epochs. Its best performance is 0.366 in NDS. We insert a single layer of the HeightFormer plugin into BEVDepth and train the new model with the same settings. The new model's best performance is 0.369 in NDS. The NDS is improved by 0.3 percentage points, which shows that the HeightFormer plugin effectively improves the BEV representations and thus improves the performance.

\subsection{Qualitative Results}


\paragraph{Height prediction error} The performance of height predictors is key to the process of gathering features. To show that heights could be estimated well, we plot the prediction error of $y$ in \cref{fig:hpe}. The data points in the figure stand for random-chosen BEV grids of all samples. For close BEV grids, 75\% of them have an error of less than 0.2m. For distant BEV grids, 75\% of them have an error of less than 0.5m. This level of error will keep most reference points falling on objects.


% Figure environment removed

\paragraph{Visualization of bounding boxes} In this part, we take a glance at the detected bounding boxes. \cref{fig:det} shows the detection results of one sample by HeightFormer-{\it tiny}. In this sample, most objects have been correctly detected, while a few distant or occluded objects are misclassified.  % However, the bounding boxes of some distant objects slightly deviate from their ground truth positions. We also notice that a distant object which has been occluded is not detected.
This is because distant objects occupy few pixels in images and occluded objects' features cannot be sampled.

% Figure environment removed





\subsection{Benchmark results}


\paragraph{NuScenes \uwave{{\it validation} set}}
We report the results on NuScenes \uwave{{\it validation} set} in \cref{tab:exp1} and compare the proposed HeightFormer with state-of-the-art camera-only methods.
Compared with BEVFormer, HeightFormer improves the NDS by 1.5 percentage points.
Compared with the SOTA BEVDepth which introduces LiDAR in training and takes data augmentation in both image space and BEV space, the proposed HeightFormer has one percentage point of improvement in mAP, which is the key metric of detection.

{{\color{blue}}
After a detailed analysis of all sub-metrics, we find that the improvement of mAP mainly comes from AP at small thresholds, for example, AP@1.0m. In the meanwhile, the improvement in the detection of rare classes is larger than that of common classes: the improvements of AP about ``bus", ``trailer" and ``motorcycle" are larger than that of ``car".
% bus, trailer, and motorcycle
A potential reason is that the proposed query mask filters out the background and thus reduces false positives. This is helpful for the detection of rare classes.}


\begin{table}[bth]
    \centering
    \caption{3D detection results on NuScenes \uwave{{\it validation} set}. The listed models mostly take R101-DCN as the backbone and require no extra LiDAR data except for BEVDepth. $\dag$: Trained with CBGS\cite{zhu2019class}. *: Take LiDAR as auxiliary information in the training phase.
    }
    \begin{tabular}{l|c|cc}
        \toprule
        Model                             & Backbone & NDS$\uparrow$ & mAP$\uparrow$ \\
        \midrule
        BEVDepth$^*$\cite{li2022bevdepth} & R101-DCN & 0.538         & 0.419         \\
        \midrule
        FCOS3D\cite{wang2021fcos3d}       & R101-DCN & 0.372         & 0.295         \\
        DETR3D\dag\cite{wang2022detr3d}   & R101-DCN & 0.434         & 0.349         \\
        PGD\cite{wang2022probabilistic}   & R101-DCN & 0.428         & 0.369         \\
        BEVDet\dag\cite{huang2021bevdet}  & Swin-T   & 0.472         & 0.393         \\
        PolarDETR-T\cite{chen2022polar}   & R101-DCN & 0.488         & 0.383         \\
        UVTR\cite{li2022unifying}         & R101-DCN & 0.483         & 0.379         \\
        PETR\dag\cite{liu2022petr}        & R101-DCN & 0.442         & 0.370         \\
        Ego3RT\cite{lu2022learning}       & R101-DCN & 0.450         & 0.375         \\
        BEVFormer\cite{li2022bevformer}   & R101-DCN & 0.517         & 0.416         \\
        Ours                         & R101-DCN & {\bf 0.532}   & {\bf 0.429}   \\
        \bottomrule
    \end{tabular}
    \label{tab:exp1}
\end{table}

\begin{table}[htb]
    \centering
    \caption{3D detection results on NuScenes {\it test} set. The listed models are all camera-only methods. The proposed HeightFormer is trained without tricks. *: Extra LiDAR supervision.}
    \begin{tabular}{l|c|c|c}
        \toprule
        Model                           & Backbone & NDS$\uparrow$ & mAP$\uparrow$ \\
        \midrule
        BEVDepth$^*$\cite{li2022bevdepth}   & V2-99    & 0.600         & 0.503 \\
        \midrule
        DD3D\cite{park2021pseudo}       & V2-99    & 0.477         & 0.418         \\
        BEVDet\cite{huang2021bevdet}    & V2-99    & 0.488         & 0.424         \\
        DETR3D\cite{wang2022detr3d}     & V2-99    & 0.479         & 0.412         \\
        UVTR\cite{li2022unifying}       & V2-99    & 0.551         & 0.472         \\
        PETR\cite{liu2022petr}          & V2-99    & 0.504         & 0.441         \\
        Ego3RT\cite{lu2022learning}     & V2-99    & 0.473         & 0.425         \\
        BEVFormer\cite{li2022bevformer} & V2-99    & 0.569         & 0.481         \\
        HeightFormer                       & V2-99    & {\bf 0.573}   & {\bf 0.481}   \\
        \bottomrule
    \end{tabular}
    \label{tab:exptest}
\end{table}


\paragraph{NuScenes {\it test} set}
We report the results on NuScenes {\it test} set in \cref{tab:exptest}. The listed models all take VoVNet (V2-99)\cite{lee2019energy} initialized from DD3D\cite{park2021pseudo} as the backbone. The proposed HeightFormer improves the NDS of BEVFormer by 0.4\%. \uwave{However, this improvement is not significant compared with the improvement on the {\it validation} set. A potential reason is that the commonly adopted VoVNet backbone is pre-trained on a depth estimation task, which does not bring much benefit to height modeling.}

\section{Conclusion and Limitation}
In this work, we analyze the 2D to 3D mapping problem in constructing the BEV space and give proof of the equivalence between depth estimation in the image space and height estimation in the BEV space. Based on the proof, {{\color{blue}}
we propose HeightFormer which explicitly models heights in BEV without extra LiDAR supervision for car-side situations.} According to our experiments, the proposed self-recursive height predictor can model heights accurately, and the segmentation-based query mask effectively improves the performance of detection. We also show that height modeling is more robust to different camera rigs compared to depth modeling.

{{\color{blue}}
However, there is a limitation in this work. The ground truth heights are acquired by projecting bounding boxes, meaning only a few queries have heights defined. Even if we introduce LiDAR supervision, most grids have no LiDAR points falling in them. As a result, the heights of queries at these grids are still learned implicitly. To mitigate this issue, we filter these queries to avoid introducing irrelevant features in the sampling procedure.}




{\appendix

\section*{Detailed derivation of $\delta_{d,max}$}

According to the analysis of depth prediction error in \cref{sec:equivalence}, if an object is correctly detected, its projected position $(x,z)$ in the BEV space should fall in the $\epsilon$-neighbourhood of its ground truth position $(x_{gt},z_{gt})$. That is to say:
    \begin{align}
        |x-x_{gt}|&+|z-z_{gt}| \le \epsilon, \label{eq:d1}\\
        \st 
        % \begin{aligned}
            \left[
                \begin{array}{c}x \\ y \\ z \end{array}
            \right] &=K^{-1}\left[
                \begin{array}{c}u_{gt}\cdot d \\ v_{gt} \cdot d \\ d \end{array}
            \right], \label{eq:d2}\\
            \left[
                \begin{array}{c}x_{gt} \\ y_{gt} \\ z_{gt} \end{array}
            \right]   &=K^{-1}\left[
                \begin{array}{c}u_{gt}\cdot d_{gt} \\ v_{gt} \cdot d_{gt} \\ d_{gt} \end{array}
            \right] \label{eq:d3} \\
        %     K &= \left[
        % \begin{array}{ccc}
        %     f_x & 0   & u_0 \\
        %     0   & f_y & v_0 \\
        %     0   & 0   & 1
        % \end{array}
        % \right],
        % \end{aligned}
        \label{eq:d4}
    \end{align}
where $(u_{gt}, v_{gt})$ is the position of the object in the image.
$d$ is the given depth at $(u_{gt}, v_{gt})$. The feature at $(u_{gt}, v_{gt})$ is gathered into the BEV grid at $(x,z)$. $K$ is the camera's intrinsic matrix which does the transformation between the image frame and the BEV frame as stated in \cref{eq:intrinsic}.$\footnote{Theoretically, there is a camera frame other than the BEV frame. Without losing generality, this manuscript does not distinguish between the camera frame and the BEV frame, because we assume that the extrinsic parameter matrix %which does transformation between the two frames 
is an identity matrix.}$
Expanding \cref{eq:d2}, \cref{eq:d3} and \cref{eq:d4}, and we get
\begin{align}
    x-x_{gt} &= \frac{1}{f_x}(u_{gt}-u_0)(d-d_{gt}),                      \\
    z-z_{gt} &= d-d_{gt}.
\end{align}
Substituting them into \cref{eq:d1}, and we can solve out the upper bound of $|d-d_{gt}|$ as follows:
\begin{equation}
    \delta_{d,max}=\epsilon \cdot \frac{f_x}{|u_{gt}-u_0|+f_x},
\end{equation}
which is described in the \cref{eq:deptherror}.


\section*{Detailed derivation of $\delta_{y,max}$}

\label{sec:intro}
According to the analysis of height prediction error in \cref{sec:equivalence}, if an object is correctly detected, the sampling locations should cover the ground truth position of the object's feature. This is concluded as:
\begin{equation}
    \begin{aligned}
        (u_{gt},&v_{gt})^T\in S_{\epsilon},                                  \\
        \st\; S_{\epsilon}\triangleq\Bigg\{ (u,v)^T&=\frac{1}{z}\left[
            \begin{array}{ccc}
                f_x & 0   & u_0 \\
                0   & f_y & v_0 \\
            \end{array}
        \right]\cdot         
        \left[ \begin{array}{c}x \\
                       y% y_{gt}\pm \delta_{y}(x,z) 
                       \\
                       z\end{array}\right] \\
        & \bigg|                \; |x-x_{gt}|+|z-z_{gt}| \le \epsilon \Bigg\}.
    \end{aligned}
    \label{eq:prob}
\end{equation}
The above problem is described in \cref{eq:hpe}. $(u_{gt},v_{gt})$ is the ground truth position of an object in the image frame, while $(x_{gt},y_{gt},z_{gt})$ is the ground truth position of the object in the BEV frame.  $S_\epsilon$ is the sampling location set corresponding to the $\epsilon$-neighbourhood of $(x_{gt},z_{gt})$. $y$ is the predicted height at $(x,z)$. Expanding \cref{eq:prob}, we get two equations and one inequality:
\begin{align}
    u_{gt}z=    & f_x x+u_0 z,                      \\
    v_{gt}z=    & f_y y+v_0 z,                      \\
    |x-x_{gt}|+ & |z-z_{gt}|\le\epsilon,\label{ieq}
\end{align}
where $y$ is an abbreviation for $y(x,z)$. According to the transformation between the image frame and the BEV frame:
\begin{align}
    u_{gt}z_{gt}=f_x x_{gt}+u_0 z_{gt}, \\
    v_{gt}z_{gt}=f_y y_{gt}+v_0 z_{gt},
\end{align} we can conclude the relationships among $x-x_{gt}$, $y-y_{gt}$, and $z-z_{gt}$. They follow:
\begin{align}
    (u_{gt}-u_0)(z-z_{gt})=f_x(x-x_{gt}),\label{e1} \\
    (v_{gt}-v_0)(z-z_{gt})=f_y(y-y_{gt}).\label{e2}
\end{align}
Substituting \cref{e1} and \cref{e2} into \cref{ieq}, we can get an inequality about $y$:
\begin{equation}
    \frac{f_y}{f_x}\cdot\frac{|u_{gt}-u_0|}{|v_{gt}-v_0|}|y-y_{gt}|+\frac{f_y}{|v_{gt}-v_0|}|y-y_{gt}| \le \epsilon.
    \label{e3}
\end{equation}
Substituting $y=y_{gt}\pm \delta_y$ into \cref{e3}, we can get the inequality about the height prediction error:
\begin{equation}
    \delta_y\le\epsilon\cdot\frac{|v_{gt}-v_0|}{f_y}\cdot \frac{f_x}{|u_{gt}-u_0|+f_x}.
\end{equation}
As a result, we get the upper bound of the height prediction error:
\begin{equation}
    \delta_{y,max}=\epsilon\cdot\frac{|v_{gt}-v_0|}{f_y}\cdot \frac{f_x}{|u_{gt}-u_0|+f_x},
\end{equation}
which is described in \cref{eq:heighterror}.

}

\section*{Comparison to BEVHeight}

In this appendix, we provide a detailed comparison between the proposed method and BEVHeight\cite{yang2023bevheight}, which also constructs BEV features via height modeling.

% Figure environment removed

\begin{enumerate}
    \item {\bf Modeling:} BEVHeight models s for each image pixel, and then projects image features into the BEV space. This is {\bf LSS-style}\cite{philion2020lift} modeling, and it can be turned into depth modeling by mapping height bins into depth bins. In contrast, HeightFormer adopts {\bf OFT-style}\cite{roddick2018orthographic} modeling. BEV features are gathered from image features by projecting reference points into images.
    
    \item {\bf Scenario:} BEVHeight is primarily designed for roadside 3D object detection tasks, whereas HeightFormer is tailored for vehicle-mounted scenarios. BEVHeight operates optimally with {cameras with high installation}, as highlighted in its paper. The proposed method has no such requirements as it models heights with BEV features which have historical information about objects.

    \item {\bf Height:} BEVHeight focuses on modeling {\bf surface heights} relative to the { flat ground}, while HeightFormer characterizes both {\bf the center and the range} of an object along the height axis.

    \item {\bf Implementation:} BEVHeight employs a strategy of dividing heights into {\bf discrete} bins and performing classifications for each image pixel. In contrast, HeightFormer utilizes Laplacian priors to model heights in {\bf contiguous} space and conducts height regression for each BEV grid.
\end{enumerate}


\begin{table}[h]
    \centering
    \caption{Comparative experiments with BEVHeight. $\dag$: Official results, single frame model. $*$: Explicit height modeling without other designs or tricks.}
    \begin{tabular}{ccc}
    \toprule
     Model    &   NDS & mAP  \\
     \midrule
      BEVFormer & 0.403 & 0.288  \\
      BEVHeight$^\dag$ & 0.342 & 0.291  \\
      HeightFormer$^*$  & 0.421 & 0.299 \\
      \bottomrule
    \end{tabular}
    \label{tab:vs}
\end{table}

Furthermore, we compare the performance. The \cref{tab:vs} shows the comparative experiments with BEVHeight. Because BEVHeight does not use history frames and its code does not contain training configurations on NuScenes, we compare mAP only. The proposed method outperformed BEVHeight by 0.8 percentage points.


% \section*{Acknowledgment}
% This work is supported by National Science Foundation for Distinguished Young Scholars under Grant 62225605, Zhejiang Provincial Natural Science Foundation of China under Grant LD24F020016, National Natural Science Foundation of China under Grant U20A20222, the Ningbo Science and Technology Innovation Project (No.2024Z294), Zhejiang Key Research and Development Program under Grant 2023C03196.

\bibliographystyle{IEEEtran}
\bibliography{IEEEabrv,main}

% \input{chaps/6profile.tex}
\end{document}


