\subsection{Qualitative Results}


\paragraph{Height prediction error} The performance of height predictors is key to the process of gathering features. To show that heights could be estimated well, we plot the prediction error of $y$ in \cref{fig:hpe}. The data points in the figure stand for random-chosen BEV grids of all samples. For close BEV grids, 75\% of them have an error of less than 0.2m. For distant BEV grids, 75\% of them have an error of less than 0.5m. This level of error will keep most reference points falling on objects.


% Figure environment removed

\paragraph{Visualization of bounding boxes} In this part, we take a glance at the detected bounding boxes. \cref{fig:det} shows the detection results of one sample by HeightFormer-{\it tiny}. In this sample, most objects have been correctly detected, while a few distant or occluded objects are misclassified.  % However, the bounding boxes of some distant objects slightly deviate from their ground truth positions. We also notice that a distant object which has been occluded is not detected.
This is because distant objects occupy few pixels in images and occluded objects' features cannot be sampled.

% Figure environment removed
