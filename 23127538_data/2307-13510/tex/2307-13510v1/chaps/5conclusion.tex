\section{Conclusion and Limitation}
In this work, we analyze the 2D to 3D mapping problem in constructing the BEV space and give proof of the equivalence between depth estimation in the image space and height estimation in the BEV space. Based on the proof, {{\color{blue}}
we propose HeightFormer which explicitly models heights in BEV without extra LiDAR supervision for car-side situations.} According to our experiments, the proposed self-recursive height predictor can model heights accurately, and the segmentation-based query mask effectively improves the performance of detection. We also show that height modeling is more robust to different camera rigs compared to depth modeling.

{{\color{blue}}
However, there is a limitation in this work. The ground truth heights are acquired by projecting bounding boxes, meaning only a few queries have heights defined. Even if we introduce LiDAR supervision, most grids have no LiDAR points falling in them. As a result, the heights of queries at these grids are still learned implicitly. To mitigate this issue, we filter these queries to avoid introducing irrelevant features in the sampling procedure.}


