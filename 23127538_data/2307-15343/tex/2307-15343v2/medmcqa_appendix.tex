\documentclass[11pt,a4paper]{article}
\usepackage[hyperref]{acl2021}
\usepackage{times}
\usepackage{latexsym}
\renewcommand{\UrlFont}{\ttfamily\small}
\usepackage{mathtools}
\usepackage{graphicx}
\usepackage{subcaption}
\usepackage[justification=centering]{caption}
\usepackage{color}
\usepackage{rotating}
\definecolor{mygreen}{gray}{0.8}

%% new commands 
\usepackage{booktabs}
\usepackage{makecell}
\usepackage{xcolor}
% \definecolor{mygreen}{RGB}{0,255,0}
\usepackage{tabularx}
\renewcommand{\theadfont}{\footnotesize\bfseries}

\usepackage[inline]{enumitem}
\newlist{tabenum}{enumerate}{1}
\setlist[tabenum]{label*=\alph*),
                %   font=\bfseries,
                  leftmargin=*,
                  nosep,
                  before=\begin{minipage}[t]{\hsize},
                  after=\end{minipage}}

\setlength\fboxsep{0pt}
\newcommand{\correctitem}{\item[\stepcounter{tabenumi}\colorbox{mygreen}{\thetabenumi}]}
%% new commands 


\usepackage{booktabs,colortbl}  
\definecolor{rowgray}{gray}{0.9}
\usepackage{amssymb}% http://ctan.org/pkg/amssymb
\usepackage{pifont}% http://ctan.org/pkg/pifont
\newcommand{\cmark}{\ding{51}}%
\newcommand{\xmark}{\ding{55}}%


% This is not strictly necessary, and may be commented out,
% but it will improve the layout of the manuscript,
% and will typically save some space.
\usepackage{microtype}

% \aclfinalcopy % Uncomment this line for the final submission
%\def\aclpaperid{***} %  Enter the acl Paper ID here

%\setlength\titlebox{5cm}
% You can expand the titlebox if you need extra space
% to show all the authors. Please do not make the titlebox
% smaller than 5cm (the original size); we will check this
% in the camera-ready version and ask you to change it back.

\newcommand\BibTeX{B\textsc{ib}\TeX}



\begin{document}

\appendix

\section{Appendices}\label{section-appendix}

\subsection{Error Analysis}\label{apd:error_anaysis}
The error analysis details on a sample set of mispredictions by the best baseline model (PubMedBERT) is given in this section. This could be used for further research to improve the models/methods on the dataset.

\begin{itemize}
    \item \textbf{Multi-hop reasoning}: It was observed that the model often mispredicted the questions related to the cause of an event (diagnosis) and the right course of action (treatment) in a given medical situation. Such questions typically require information on multiple symptoms, ailments, and treatments to select the most appropriate choice. This multiplicity of information is not likely to be present in one passage, possibly the reason for the mispredictions.
    \item \textbf{Incorrect context passages}: It is observed that inadequate contexts from the retriever are also major contributors to the mispredictions.
    \item It is found that the models mispredicted the questions requiring arithmetic reasoning. This is in line with the observations in \cite{Dua2019} on BERT-based models.
\end{itemize}



% \begin{sidewaysfigure}
% %   % Figure removed
%   % Figure removed
%   \caption{ \footnotesize}
%   \label{fig:topic_subj}
% \end{sidewaysfigure}

% Figure environment removed

\begin{table*}
\footnotesize
\caption{Example questions from the dataset for each reasoning types (identified based on the sampled data)}
\label{tab:reasoning_type_table}
\begin{tabularx}{\textwidth}{cX}
\toprule
 \thead{Reasoning Type} & \thead{Example} \\
\midrule
Select wrong ones & Select a side effect that is not caused by a Progestogen-only pill (POP)?   \\
        &    \begin{tabenum}
                  \item Venous thromboembolism {\cmark}
                  \item  Ovarian cysts
                  \item  Ectopic pregnancy 
                  \item  Increased risk of diabetes mellitus
                  \end{tabenum} \\
\midrule
Factual  & Choose the correct pin-code index of N2O? \\
        &    \begin{tabenum}
                  \item 2,5
                  \item  1,6
                  \item  3,5 {\cmark}
                  \item  2,6
                  \end{tabenum} \\
\midrule
Explanation & Choose the most appropriate explanation of seasonal trend : \\
        &    \begin{tabenum}
                  \item Some diseases occur in cyclic spread over short periods of time.
                  \item  Seasonal variation of disease occurrence may be related to environmental conditions. {\cmark}
                  \item  Non-infectious conditions never show periodic fluctuations. 
                  \item  Some disease occurs in cyclic changes over a long period of time.
                  \end{tabenum} \\
\midrule
MultiHop Reasoning  & Which is the most recommended drug for treating Porphyromonas and Provetella bacteria? \\
        &       \begin{tabenum}
                  \item Ciprofloxacin 
                  \item  Metronidazole
                  \item  Erythromycin 
                  \item  Tetracycline {\cmark}
                  \end{tabenum} \\
\midrule
Analogy  & Select the feature that is most similar  between cerebral infarct  abscess \\
        &    \begin{tabenum}
                  \item Liquefactive necrosis {\cmark}
                  \item  Heal by collagen formation
                  \item  always develop from emboli from other site 
                  \item  Coagulative necrosis
                  \end{tabenum} \\
\midrule
Teleology/purpose  & Purpose of the 8${^{th}}$ cranial nerve is linked to? \\
        &       \begin{tabenum}
                  \item touch
                  \item  taste
                  \item  balance {\cmark}
                  \item  smell
                  \end{tabenum} \\
\midrule
Comparison  & On comparing follicular Ca, papillary Ca of the thyroid is related to \\
        &    \begin{tabenum}
                  \item Radiation exposure {\cmark}
                  \item  Iodine deficiency
                  \item  Increased mortality 
                  \item  More male preponderance
                  \end{tabenum}\\
\midrule
Fill in the blanks  & The full-term neonate has an avg. Central aortic pressure of \noindent\rule{0.8cm}{0.4pt} mm Hg? \\
        &    \begin{tabenum}
                  \item 20/10
                  \item  75/50 {\cmark}
                  \item  60/40
                  \item  40/20
                  \end{tabenum}
 
 
 \\
\midrule
Natural language inference  & Statement 1:  Until the child is 17 years of age, permanent crowning is not recommended Statement 2: it is because the dentofacial structures’ development would not be finished by then \\
        &   \begin{tabenum}
                  \item Both the statements are false
                  \item  Second statement is false and the first is true
                  \item  Second statement is true and the first is false
                  \item  Both the statements are true {\cmark}
                  \end{tabenum} \\
\midrule
Mathematical & In a population of 100K , 100 people have been diagnosed with lung cancer. 80 patients out of the 100 were smoker and the total population has 200K smokers. Calculate PAR for this scenario ? \\
        &        \begin{tabenum}
                  \item 75  {\cmark}
                  \item  80
                  \item  90 
                  \item  60
                  \end{tabenum} \\
\bottomrule 
\end{tabularx}
\end{table*}
\bibliographystyle{acl_natbib}
\bibliography{anthology,acl2021}
\end{document}