\documentclass[a4wide,10pt]{article}%
%Submission to Optimization Methods and Software. Same as the test case of Troeltzsch \cite{bib:TW04}}
\usepackage{geometry}                % See geometry.pdf to learn the layout options. There are lots.
\usepackage{tikz} 
\usetikzlibrary{positioning}
%\usepackage[draft]{graphicx}
\usepackage{graphicx}
\usepackage{authblk}
\usepackage{amssymb}
%\usepackage{showkeys}
\usepackage{epstopdf}
\usepackage{subfigure}  % use for side-by-side figures
\usepackage[sans]{dsfont}
\usepackage[applemac]{inputenc}
\usepackage[english]{babel}
\usepackage{latexsym}
\usepackage{mathrsfs}
\usepackage{amscd}
\usepackage{graphicx}
\usepackage{color}
\usepackage{float}
%\usepackage{mathabx}
\frenchspacing
\usepackage{amsmath}
\usepackage{amsfonts}
\numberwithin{equation}{section}
\usepackage{enumerate}
\usepackage{amsthm}
\usepackage[numbers,sort]{natbib}
\usepackage[bookmarks=true,colorlinks=true,linkcolor={blue},urlcolor={blue}, citecolor={blue},pdfstartview={XYZ null null 1.22}]{hyperref}%

%\usepackage{showkeys}

%%% For margin notes
%\newcommand{\note}[1]{%
%  \marginpar[{\color{red}{\raggedleft\small\sffamily #1\\}}]{%
%  {\color{red}  {\raggedright\small\sffamily #1\\}}}}
%
%\usepackage{geometry}
\geometry{letterpaper}
\providecommand{\keywords}[1]{\textbf{{Key words.}} #1}

%    \graphicspath{{pics/}}

%%%%%%%%%%%%%%%%%%%%%%%%%%%%%%%%%%%%%%%%%%%%%%%%%%%%%%%%%%%%%
\def\BigO{ {\mathcal {O}}}
\def\bigM{ {\mathcal {M}}}
\def\bigF{ {\mathcal {F}}}
\def\B{\hat \beta}
\def\f{\hat f}
\def\g{\hat g}
\def\Q{\hat Q}
\def\real{\mathbb{R}}
\newcommand{\R}{\mathbb R}
\newcommand{\N}{\mathbb N}
\newcommand{\dt}{\Delta t}
%\renewcommand{\O}{\Omega}
\def\be#1\ee{\begin{equation}#1\end{equation}}
\newcommand{\fer}[1]{(\ref{#1})}



%\newtheorem{theorem}{Theorem}[section]
%\newtheorem{corollary}{Corollary}
%\newtheorem{lemma}[theorem]{Lemma}
\newtheorem{proposition}{Proposition}
\newtheorem{conjecture}{Conjecture}
\theoremstyle{definition}
\newtheorem{alg}{Algorithm}[section]
%\newtheorem{definition}[theorem]{Definition}
\newtheorem{remark}{Remark}
\newcommand{\ep}{\varepsilon}
\newcommand{\eps}[1]{{#1}_{\varepsilon}}
\newcommand{\xx}{{\bf x}}
\newcommand{\yy}{{\bf y}}

\def\RR{\mathbb R}
\def\e{\varepsilon}
\def\cm{c_{\textrm{max}}}
\def\ps{\p_{\infty}}
\def\Ur{\textcolor{black}{V_r}}
\def\Ua{\textcolor{black}{V_a}}
\def\p{\rho}

\def\N{\mathcal N}
\def\T{\mathcal T}
\def\L{\mathcal L}

%%%%%%%%%%%%%%%%%%%%%%%%%%%%%%%%%%%%%%%%%%
\def\be{\begin{equation}}
	\def\ee{\end{equation}}
\def\bea{\begin{eqnarray}}
	\def\eea{\end{eqnarray}}

%%%%%%%%%%%%%%%%%%%%%%%%%%%%%%%%%%
\newenvironment{equations}{\equation\aligned}{\endaligned\endequation}
%%%%%%%%%%%%%%%%%%%%%%%%%%%%%%%%%%%%%%%%%%%%%%%%%%%%%%

\title{Kinetic description of swarming dynamics with topological interaction and emergent leaders}

% Place all authors' names in [ ] shown as running head, Leave { } empty
% Please use `and' to connect the last two names if applicable
% Use FirstNameInitial.  MiddleNameInitial. LastName, or only last names of authors if there are too many authors
\author{Giacomo Albi\footnote{	Dipartimento di Informatica, Universit\`a di Verona, Verona, Italy, e-mail: giacomo.albi@univr.it}, and Federica Ferrarese\footnote{		Dipartimento di Matematica, Universit\`a di Trento, e-mail: federica.ferrarese@unitn.it}}

% It is required to enter 2010 MSC.
%\subjclass{Primary: 58F15, 58F17; Secondary: 53C35.}
% Please provide minimum  5 keywords.
% \keywords{Opinion dynamic, kinetic equations, scale-free networks, collective behavior, big data, Monte Carlo methods, finite-difference schemes.}


\begin{document}
	\date{}
	\maketitle
	
	%% Enter the first author's name and address:
	%\centerline{\scshape Giacomo Albi}
	%\medskip
	%{\footnotesize
	%% please put the address of the first author
	% \centerline{TU M\"unchen, Faculty of Mathematics}
	%   \centerline{Boltzmannstra\ss e 3, D-85748, Garching (M\"uünchen), Germany}
	%%   \centerline{ Springfield, MO 65801-2604, USA}
	%} % Do not forget to end the {\footnotesize by the sign }
	%
	%\medskip
	%
	%\centerline{\scshape Lorenzo Pareschi and Mattia Zanella}
	%\medskip
	%{\footnotesize
	% % please put the address of the second  and third author
	% \centerline{ University of
	%Ferrara, Department of Mathematics and Computer Science}
	%   \centerline{Via N. Machiavelli 35, 44121, Ferrara, Italy}
	%%   \centerline{Springfield, MO 65810, USA}
	%}
	%
	%\bigskip
	%
	%% The name of the associate editor will be entered by an editorial staff
	%% "Communicated by the associate editor name" is not needed for special issue.
	%% \centerline{(Communicated by the associate editor name)}
	%
	%
	%%The abstract of your paper
	\begin{abstract}

In this paper, we present a model describing the collective motion of birds. We explore the dynamic relationship between followers and leaders, wherein a select few agents, known as leaders, can initiate spontaneous changes in direction without being influenced by external factors like predators. Starting at the microscopic level, we develop a kinetic model that characterizes the behaviour of large crowds with transient leadership. One significant challenge lies in managing topological interactions, as identifying nearest neighbors in extensive systems can be computationally expensive. To address this, we propose a novel stochastic particle method to simulate the mesoscopic dynamics and reduce the computational cost of identifying closer agents from quadratic to logarithmic complexity using a $k$-nearest neighbours search algorithm with a binary tree.
Lastly, we conduct various numerical experiments for different scenarios to validate the algorithm's effectiveness and investigate collective dynamics in both two and three dimensions.	\end{abstract}
	{\bf Keywords:} mean-field models, kinetic equations, Monte-Carlo methods, topological interactions, transient leadership
	\\
	{\bf AMS classification}: 65C05, 65Y20, 82C80, 92B05
	
	\tableofcontents
	
	\section{Introduction}\label{sec:intro}
	 In the last decades, there has been a notable surge in interest regarding the study of mathematical models describing collective behaviour of animals such as bacterial swarm \cite{koch1998social}, self-organization in insects \cite{dussutour2006collective,bonabeau1997self},  bird flocking \cite{cucker2007emergent,parrish1999complexity,ballerini2008empirical,lukeman2010inferring}, and  fish schooling \cite{hemelrijk2008self,d2006self}.
	 This captivating area of investigation has garnered substantial interest, with researchers increasingly delving into the complexities of emergent behaviours exhibited by natural systems, but also spanned to a wider range of applications such as swarm of robots \cite{jadbabaie2003coordination,choi2019collisionless,king2023biologically}, as well as social sciences and economics, \cite{albi2016opinion,rainer2002opinion,toscani2006kinetic,dimarco2020wealth}, vehicular and pedestrian traffic \cite{cristiani2014multiscale,albi2019vehicular,bressan2014flows}.
	 
	 These large ensemble of models incorporate rules governing the behaviour of individual entities within the system. By integrating such mechanisms, these models effectively capture the impact of each entity on others, taking into account their relative positions and velocities.  In this manuscript, our focus centers around the dynamics governing animal swarms, building upon the recent model proposed in \cite{cristiani2021all}. This model introduces spontaneous changes of direction within the swarm, independent of external factors like predators, but rather influenced by a hierarchical interaction structure comprising two dynamic subpopulations labelled as leaders and followers. The key concept is that each bird possesses the potential to initiate turns and become a leader, consequently influencing its nearest neighbours who adopt follower status. This change of labels is characterized as a stochastic process, where each occurrence represents a random event. Our primary interest lies in exploring the phenomenon of \textit{transient leadership}, wherein agents can alter their labels over time, as examined for example in \cite{albi2019leader,loy2020non}, and also at different scales in \cite{li2015cucker, morandotti2020mean, bernardi2021leadership,cristiani2023kinmacr}.
	 Here, we examine a simplified version of the second-order stochastic differential equations presented in \cite{cristiani2021all}. Specifically, we remove delay effects and consider that each agent (bird) can interact with a maximum of $M$ nearest neighbours, consistent with observations in \cite{ballerini2008empirical}. Our objective is to study such dynamics at the mesoscopic scale, formally introducing a kinetic model of the swarms with topological-type interaction dynamics and deriving the associated mean-field limit. For alternative mean-field and kinetic models with topological interactions, we refer to \cite{blanchet2017kinetic, blanchet2016topological, haskovec2013flocking}, and for rigorous derivations, we refer specifically to \cite{degond2019propagation, benedetto2022mean}.
	 Another primary objective of this study is to efficiently perform numerical simulations of high-dimensional non-local dynamics. One of the main challenges arises from the presence of topological-type interactions, necessitating the implementation of ad-hoc methods to reduce the complexity of the nearest neighbour search process. To address this computational burden at the mesoscopic scale, we introduce a novel stochastic simulation algorithm for the simulation of kinetic models such as, \cite{pareschi2013interacting, albi2013binary,carrillo2019particle}, in particular introducing the label switching feature, and adopting $k-$nearest neighbours search strategy, following the approach proposed in \cite{friedman1977algorithm}. By implementing this method, we successfully reduce the computational complexity of the nearest neighbour search from quadratic to logarithmic scale, significantly enhancing the efficiency of numerical simulations.
 
	The paper is organized as follows. In Section \ref{sec:micro_model} we introduce the microscopic model describing which are the forces that act on followers and on leaders. In Section \ref{sec:kinetic_model}, we extend the study to the kinetic level, describing the evolution of the densities and how the change of labels occurs. In Section \ref{sec:numerical_methods}, we introduce the algorithms that can be used to simulate the binary interaction rules and the change of labels. In Section \ref{sec:validation} we perform two different validations experiments, testing both the accuracy and the efficiency of the numerical methods introduced. In particular, we show how it is possible to reduce the computational costs in dealing with non-locality. In Section \ref{sec:2D3Dexperiments} we simulate the dynamics at the microscopic and kinetic level for both the two and three dimensional cases. 
	
	\section{Swarming models with leaders-followers dynamics}\label{sec:micro_model}
	We consider a large system of  $N$ interacting agents represented by points moving in a $d$-dimensional space with an evolving hierarchy of interactions ruled by follower-leaders dynamics. 
	For every $i = 1,\ldots,N$, let $(x_i(t),v_i(t)) \in \RR^{2d}$ denote position and velocity of the $i$-th agent at time $t$, with $d=1,2,3$, and $\lambda_i(t)\in\Lambda \equiv {\left\{0,1\right\}}$ the space of labels indicating  at time $t$ the status of agent $i$ to be either {\em follower} ($F$) for  $\lambda_i(t)=0$, or {\em leader}  ($L$) for $\lambda_i(t)=1$.   Moreover we account for $N_{{src}}$ fixed target positions located at $x_k^{{src}}\in\RR^{d}$ for $k = 1,\ldots, N_{{src}} $, indicating positions of interested for the swarm such as nest, or foraging areas \cite{bernardi2018particle,bidari2019social}. 
	
	We assume the system of agents  evolving according to ODEs system,
	\begin{equation}\label{eq:dynamics} 
		\begin{aligned}
			\dot{x}_i &= v_i,\cr
			%        	\dot{v}_i &= \frac{1}{M} \sum\limits_{\left\{j~:~x_j\in \mathcal{B}_M(x_i;\xx)\right\}}\left[ A^{rep}(x_i,x_j) + \left(  A^{ali}(v_i,v_j) + A^{att}(x_i,x_j)\right)(1-\lambda_i(t))\right] \cr
			%			&\qquad \qquad\qquad+ \left[ A^{src}(x_i) + A^{ctr}(x_i) +S(v_i) \right]\lambda_i(t),  \qquad i=1,\ldots,N, \cr
			\dot{v}_i &= \frac{1}{M} \sum\limits_{\left\{j~:~x_j\in \mathcal{B}_M(x_i;\xx)\right\}}\left[ A^{rep}(x_i,x_j) + \left(  A^{ali}(v_i,v_j) + A^{att}(x_i,x_j)\right)(1-\lambda_i(t))\right] \cr
			&\qquad \qquad\qquad+ \left[ A^{src}(x_i) + A^{ctr}(x_i) +S(v_i) \right]\lambda_i(t),  \qquad i=1,\ldots,N, \cr
		\end{aligned}
	\end{equation}
	%where, the set $\mathcal{T}_M (x_i;\xx)$ is the topological index set of the $M$ closest agent to the agent $i$, i.e.
	%\[
	%\mathcal{T}_M (x_i;\xx) := \left\{j | x_j\in\mathcal{B}_M(x_i;\xx) \right\}
	%\]
	where we denoted by $\mathcal{B}_M(x_i;\xx)$ the ball centred at $x_i$, with $\xx=(x_1,\ldots,x_N)$, containing the $M$ nearest neighbors to $i$-agent,   assuming that in case of ambiguity, e.g. more than one agent is at the same distance from agent in position $x_i$, we select the first $M$ agents giving priority according to the indexing order. 
	Hence, the dynamics encodes different behaviours according to the value of the label $\lambda_i(t)$. 
	% with the lower index is included, and $\chi_{B_M(x;\xx,\yy)(y)}$ is the characteristic function of the topological ball $\mathcal{B}_M(x;\xx)$. Hence, 
	\begin{itemize}
		\item 
		For $\lambda_i(t) = 0$, we have follower-type interactions characterized by
		\begin{itemize}
			\item  repulsion force
			\begin{equation}\label{eq:repulsion} 
				A^{rep}(x,x') =-C_{rep} \frac{ x' - x}{\Vert  x' - x\Vert^2 },
			\end{equation}
			\item alignment force
			\begin{equation}\label{eq:aligment} 
				A^{ali}(v,v') =C_{ali} \left( v' - v\right),
			\end{equation}
			\item and attraction force  
			\begin{equation}\label{eq:atraction} 
				A_i^{att}(x,x') =C_{att} \left( x' - x\right),
			\end{equation}
		\end{itemize} 
		where $C_{rep}\geq0$, $C_{ali}\geq0$ and $C_{att}\geq0$ are non-negative constants.
		\item For $\lambda_i(t)=1$, we have leaders-type dynamics  characterized by a repulsion force defined as in equation \eqref{eq:repulsion} and by a self-propulsion friction term $S(\cdot)$ defined as 
		\begin{equation}\label{eq:relaxation}
			S(v) = C_v (s-\Vert v \Vert^2)v,
		\end{equation}
		where $s$ is a given characteristic speed and $C_v\geq0$. In presence of sources terms, leaders are driven by 
		\begin{equation}\label{eq:food}
			A^{src}(x) = C_{src} \sum_{k=1}^{N_{src}}\varphi_\epsilon(\Vert x_k^{src}-x\Vert;\overline r) \frac{x_k^{src}-x}{\Vert x_k^{src}-x\Vert },
		\end{equation}
		\textcolor{black}{where $C_{src}\geq0$, $x_k^{src}$ denotes the position of the attraction source (nest, or food) and $ \varphi_\epsilon(\cdot)$ is a sigmoid function of the following type
			\begin{equation}
				\varphi_\epsilon(r;\overline r) := 	\frac{1}{1+\exp\{(r-\overline r)/\epsilon\}},
			\end{equation}
		 with regularization parameter
			$\epsilon >0$, modelling a {perception area around the source} activating when the distance of the agent is below the threshold value $\overline r>0$. 
			Furthermore, leaders can  be forced to move toward the centre of mass $x_c$ according to the force
			\begin{equation}\label{eq:centre_of_mass}
				A^{ctr}(x) = C_{ctr} \Big(1-{\varphi_\epsilon} (\Vert x_c - x \Vert;\underline r)\Big) \frac{x_c - x}{\Vert x_c - x \Vert },
			\end{equation}
			where $C_{ctr}\geq0$, and when the distance with respect to the centre of mass is larger than  $\underline r$.
			%\begin{equation}
			%	{\varphi}(r) = (\frac{e^{[\bar{c}(r-\bar{r})]}}{1+\exp\{[\bar{c}(r-\bar{r})]\}}),
			%	\end{equation}
			%where $\bar{c}\geq0$ and $\bar{r}\geq0$. 
		}
	\end{itemize}
	\subsection{Stochastic process for leaders emergence}\label{sec:leaders_micro}
	Agents have the ability to switch between being leaders and followers, and vice versa. Such a change in status is treated as a stochastic process, where each occurrence represents a random event governed by an assigned probability distribution. Each event is associated with a transition rate, which quantifies the probability of its occurrence per unit time. Therefore, for $\boldsymbol{\lambda} = (\lambda_1,\ldots,\lambda_N)$, each label $\lambda_i(t)$ will follow a jump process in this manner
		\begin{itemize}
		\item if $\lambda_i(t) = 1$ then it switches to $0$ with rate $\pi_{L\to F}(t,x_i,v_i,\lambda_i;\xx,\boldsymbol{v},\boldsymbol{\lambda})$,
		\item if $\lambda_i(t) = 0$ then it switches to $1$ with rate $\pi_{F\to L}(t,x_i,v_i,\lambda_i;\xx,\boldsymbol{v},\boldsymbol{\lambda})$,
	\end{itemize}
	where the transition rates $\pi_{F\to L}(\cdot), \pi_{L\to F}(\cdot)$ in general are non-linear functions of the state of the system. In what follows we will consider different choices for the labels' switching rules, ranging from random, density dependent and aiming at organizing agents toward a common target. These choices will be detailed in Section \ref{sec:leaders_kinetic}.
	%%%%%%%%%%%%%%%%%%%%%%%%%%%%%%%%%%%%%%%%%%%%%
	%%%%%%%%%%%%%%%%%%%%%%%%%%%%%%%%%%%%%%%%%%
	%%%%%%%%%%%%%%%%%%%%%%%%%%%%%%%%%%%%%%%%%
	\section{Kinetic modelling of swarming dynamics}\label{sec:kinetic_model}
	In this section, we will provide a kinetic description of the swarming model with leader emergence and topological interaction, we refer to \cite{albi2016opinion,albi2016invisible,loy2020non} for related studies in the context of kinetic models. 
	
	
	Thus, we associate to each agent a position and velocity $(x,v)\in\R^d\times\R^d$ and a leadership-level $\lambda$, as a discrete binary variable in the label space $\Lambda = \{0,1\}$. We are interested in the evolution of the probability density function 
	\begin{equation}\label{eq:def_f}
		f=f(x,v,\lambda,t), \qquad f: \R^d\times\R^d\times\left\{0,1\right\} \times \R_+\rightarrow \R_+
	\end{equation}
	where $t\in\RR^+$ denotes as usual the time variable. For each time $t\ge 0$, $\lambda\in\{0,1\}$, we have  the following marginal density
	\begin{equation}\label{eq:integration_wc}
		\p(\lambda,t)=\int_{\R^{d}\times \R^d}f(x,v,\lambda,t)d(x,v), 
	\end{equation}
	which defines the quantity of agents with label $\lambda$ at time $t$. In the sequel, we will assume that the total number of agents is conserved, namely 
	\be\label{eq:mass}
	\p(1,t)+\p(0,t)=1.
	\ee
	Likewise, we define the marginal density  for agents in space and velocity
	\begin{equation}\label{eq:total_density}
		g(x,v,t)=\sum_{\lambda} f(x,v,\lambda,t),\qquad \lambda\in \{0,1\}.
	\end{equation}
	Next, we assume the density $f(x,v,\lambda,t)$ to be solution of a kinetic equation accounting for pairwise interactions among agents, and for labels transition. 
	
	\paragraph{Notational convention.}  To ease the writing, we will use an equivalent notation for functions  depending on $\lambda$,  where we introduce the indexing given by the discrete label space $\Lambda$, as follows
	\[
	F_\lambda(\cdot) := F(\cdot,\lambda).
	\]
	Then, for example, the density $f(x,v,t,\lambda)$ will be denoted by $f_\lambda(x,v,t)$ or the mass $\rho(\lambda,t)$ by $\rho_\lambda(t)$.
	\subsection{Povzner-Boltzmann-type model}\label{sec:kinetic_model_0}
	We assume that each agent modifies its velocity through a binary interaction occurring with an other agent within the topological ball $\mathit{B}_{r^*}(x,t)$, the ball centred in $x$ whose radius is defined, for a fixed $t \geq 0$, by the following variational problem 
	\begin{equation}\label{eq:radius}
		r^*(x,t) = \arg \min_{\alpha>0} \left\{\sum_{\lambda} \int_{\mathit{B}_\alpha (x,t)\times\RR^{d}} f_\lambda(x,v,t) dx dv \geq \rho^* \ \right\},
	\end{equation}
	where $\rho^*\in(0,1]$ is the target topological mass, namely the ratio $\rho^* = M/N$  associated to the microscopic model  \eqref{eq:dynamics}.
	
	Hence, we consider pairwise interactions among an agent with state $(x,v,\lambda)\in\RR^{2d}\times{\left\{0,1\right\}}$ and $(x_*,v_*,\lambda_*)\in\mathit{B}_{r^*}(x,t)\times\RR^d\times{\left\{0,1\right\}}$, where the post-interaction velocities are given by
	\begin{equation}\begin{cases}\label{eq:binary}
			v'   &=v   + \alpha \mathcal{F}_\lambda(x,x_*,v,v_*), \\
			v_*' &=v_*,%+ \alpha \mathcal{F}(x_*,x,v_*,v,\lambda_*),
	\end{cases}\end{equation}
	where $v,v_*\in\mathbb{R}^d$ denote the pre-interaction velocities and $v',v_*'$ the velocities after the exchange of information between the two agents. In  \eqref{eq:binary} we assume 
	\begin{equation}\label{eq:binary2}
		\begin{split}
			&\mathcal{F}_\lambda(x,x_*,v,v_*) =A^{rep}(x,x_*) +\bigl[ A^{ali}(v,v_*)+A^{att}(x,x_*)\bigl](1-\lambda) \\
			&\qquad \qquad\qquad \qquad \qquad \qquad + \bigl[ A^{src}(x) + A^{ctr}(x) + S(v)\bigr]\lambda. 
		\end{split}
	\end{equation}
	%% &\mathcal{A}^{food}(\cdot) = A^{food}(\cdot) M, \quad \mathcal{A}^{centre}(\cdot) = A^{centre}(\cdot) M,
	%
	%\end{equation}
	For $\lambda\in \{0,1\}$, the evolution in time of the density function $f_\lambda(x,v,t)$ is described by a integro-differential equation of the Povzner-Boltzmann type  \cite{povzner1962boltzmann,fornasier2011fluid}  as follows
	\begin{equation}\label{eq:boltz_lin}
		\partial_t f_\lambda(x,v,t) + v\cdot\nabla_x f_\lambda (x,v,t)-\T_\lambda[f](x,v,t)= Q_\lambda(f,f)(x,v,t),
	\end{equation}
	where  $\mathcal T_\lambda [f](\cdot)$ accounts for the evolution of the agents in the discrete label space and $Q_\lambda(\cdot,\cdot)$ is  the interaction operator defined as follows
	\begin{equation}	\label{eq:Bo}
		Q_\lambda(f,f)(x,v,t) =\eta \sum_{\lambda_*}\int_{\mathit{B}_{r^*(x,t)}\times\mathbb{R}^d} \left(\dfrac{1}{J_\lambda}f_{\lambda}(x,'v,t)f_{\lambda_*}(x_*,'v_*,t)-f_\lambda(x,v,t)f_{\lambda_*}(x_*,v_*,t)\right)d(x_*,v_*),
	\end{equation}
	where $('v,'v_*)$ are the pre-interaction velocities, and the term $J_\lambda$ denotes the Jacobian of the transformation $(v,v_*)\rightarrow (v',v_*')$ with  $ (v',v_*')$ the post-interaction velocities, and $\eta>0$ is a constant relaxation rate representing the interaction frequency.
	
	% the kernels $'B,B$ define the binary interaction .
	%We will consider constant interaction kernels of the following form 
	%\begin{equation}
	%	B_{(v,v_*)\rightarrow (v',v_*')}=\eta,
	%\end{equation}
	
	%where $\eta>0$ is a constant relaxation rate representing the interaction frequency. 
	
	%Here and in the rest of the Section, for notation simplicity, the explicit dependence from the time variable is omitted. 
	\subsection{Master equation for leaders transition }\label{sec:leaders_kinetic}
	In the previous section, we have introduced the transition operator $\mathcal T_\lambda [f](x,v,t)=\mathcal T[f](x,v,\lambda,t)$ characterizing the evolution of the agents in the discrete space of labels $\Lambda = \{0,1\}$ (leaders/followers). Such operator is defined as follows
	\begin{equation}\label{eq:master_0}
		\begin{split}
			\mathcal{T}_0[f](x,v,t) =&  \pi_{L\to F}f_1(x,v,t)-\pi_{F\to L} f_0(x,v,t), \\
			\mathcal{T}_1[f](x,v,t) =&   \pi_{F\to L}f_0(x,v,t)-\pi_{L\to F}f_1(x,v,t),
		\end{split}
	\end{equation}
	where $\pi_{F\to L}:= \pi_{F\to L}(x,v,t;f)$ and $\pi_{L\to F}:= \pi_{L\to F}(x,v,t;f)$ are certain transition rates.
	
	Thus the evolution of the transition process of labels can be described by the evolution equation for $\rho_\lambda (t)= \rho(\lambda,t)$, 
	\begin{equation}\label{eq:Ldef2}
		\dfrac{d}{dt}\p_\lambda(t) - \int_{\RR^{2d}} \T_\lambda[f](x,v,t)\,d(x,v)=0.
	\end{equation}
	From the definition of the transition operator $\T_\lambda[\cdot]$ and \eqref{eq:mass} it follows the  conservation of the mass,
	\begin{equation}
		\dfrac{d}{dt}\sum_{\lambda}\p_\lambda(t) =  \sum_{\lambda}\int_{\RR^{2d}} \T_\lambda[f](x,v,t)\,d(x,v) = 0.
		\label{eq:tnc}
	\end{equation}
	In the sequel we list possible choices of transition rates in \eqref{eq:master_0}.
	
	\paragraph{Constant rates.}
	Leaders emerge with rate $q_{FL}>0$ and return to the followers status with rate $q_{LF}>0$. Hence, the transition rates write as follows
	\begin{equation}\label{eq:rates_test_0}
		%	\begin{split}
		\pi_{L\to F}  = q_{LF},\qquad \pi_{F\to L} = q_{FL}. 
		%	\end{split} 
	\end{equation} 
	%Generally, assuming the transition rates indepedent on $(x,v)$, 
	%\begin{equation*}
	%	\pi_{L\to F}(\cdot) = \alpha(\lambda,t),\qquad \pi_{F\to L}(\cdot) = \beta(\lambda,t),
	%\end{equation*}
	Thus, we can rewrite equation \eqref{eq:Ldef2} as 
	\begin{equation}\label{eq:master_constant}
		\begin{aligned}
			\partial_{t} \rho_1(t) &= q_{FL} \rho_1(t) - q_{LF} \rho_0(t),\\
			\partial_{t} \rho_0(t) &= q_{LF}	\rho_1(t)-q_{FL} \rho_0(t),
		\end{aligned}
	\end{equation}
	and find the stationary solution of equation \eqref{eq:master_constant} that is 
	\begin{equation}\label{eq:stationary}
		\rho_1^\infty = \frac{q_{FL}}{q_{LF} + q_{FL}},\qquad 	\rho_0^\infty = \frac{q_{LF}}{q_{LF} +q_{FL}}.
	\end{equation}
	
	\paragraph{Density-dependent rates.} Leaders emerge with higher probability where the followers density is higher and the leaders one is lower and they return to the followers status with higher probability if the followers concentration around them is lower, similarly to \cite{albi2022mean}. The transition rates reads 
	\begin{equation}\label{eq:rates_test}
		\begin{split}
			\pi_{L\to F} = q_F~ (1-\mathcal{D}_F[f](x,t)),\qquad
			\pi_{F\to L} = q_L~ (1-\mathcal{D}_L[f](x,t)),
		\end{split}
	\end{equation} 
	where $q_F$, $q_L$ are constant parameters and the functions $\mathcal{D}_F[f](x,t)$ and $\mathcal{D}_L[f](x,t)$ represent the concentration of leaders and followers in position $x$ and are defined as 
	\begin{equation}\label{eq:concentration}
		\begin{split}
			&\mathcal{D}_F [f](x,t) = S_F (t) \int_{\RR^{2d}} e^{-\frac{\vert x - y\vert^2 }{\delta^2}} f_0(y,w) d(y,w),\\
			&\mathcal{D}_L [f](x,t) = S_L (t) \int_{\RR^{2d}} e^{-\frac{\vert x - y\vert^2}{\delta^2}} f_1(y,w) d(y,w),
		\end{split}
	\end{equation}
	with $S_F (t)$, $S_L (t)$ normalization constants to ensure that the above quantities are bounded by one and with $\delta>0$.
	\paragraph{Target-oriented rates.} Leaders emerge when their direction is oriented in the correct direction toward a target position, $\bar{x}$,  such as the nesting or foraging area. We consider the following rates
	\begin{equation}\label{eq:rates_opt}
		\begin{aligned}
			\pi_{F\to L} &
			= \begin{cases}
				0,\qquad  \text{if } \alpha(x,v,t;f) < \overline{\alpha},\\
				1, \qquad \text{if } \alpha(x,v,t;f) \geq \overline{\alpha},
			\end{cases}\qquad
			\pi_{L\to F}&=
			\begin{cases}
				0,\qquad  \text{if } \alpha(x,v,t;f) \geq \underline{\alpha},\\
				1, \qquad \text{if } \alpha(x,v,t;f) < \underline{\alpha},
			\end{cases}
		\end{aligned}
	\end{equation}
	with $\underline{\alpha},\overline{\alpha}\in[-1,1]$ and
		%\begin{equation}\label{eq:alpha}
		%	\alpha(x,v,t;f) = \frac{\langle \bar{x}-x, G(x,v,t;f)\rangle}{\langle G(x,v,t;f),G(x,v,t;f)\rangle},
		%\end{equation}
		\begin{equation}\label{eq:alpha}
			\alpha(x,v,t;f) = \cos\left(\angle \left(\bar{x}-x, 	\mathcal{G}[f](x,v,t)\right)\right),
		\end{equation}
	with $\angle(\cdot,\cdot)$ denoting the angle between two vectors.
		%\begin{equation}\label{eq:lambda_opt} 
		%	\lambda = \sigma \left( \frac{\left\langle  \sum_{n=1}^d (\bar{x}-x), (G_i)_n\right\rangle }{\sum_{n=1}^d (G^2_i)_n}\right),
		%\end{equation}
		%where $\sigma(\alpha) = 1$ if $\alpha \geq0.5$ and $\sigma(\alpha) = 0$ if $\alpha <0.5$.
	  The functional $	\mathcal{G}[f](\cdot)$ accounts for the  directional information of agents according to
		\begin{equation}\label{eq:G_test1}
			\mathcal{G}[f](x,v,t) =  S(v) - \mathcal{X}_c[f](x,t) - \mathcal{V}_c[f](x,v,t),
		\end{equation} 
		where $S(v)$ is the self-propulsion term, and the terms $\mathcal{X}_c[f](\cdot), \mathcal{V}_c[f](\cdot)$ account for the average influence induced by neighbours  as follows
		\begin{equation*}
			\begin{split}	
				\mathcal{X}_c[f] (x,t) &= \int_{\mathit{B}_{r^*(x,t)}\times\mathbb{R}^d} A^{att}(x,x_*) f_\lambda(x_*,v_*,t) dx_* dv_*,\\
				\mathcal{V}_c[f] (x,v,t) &= \int_{\mathit{B}_{r^*(x,t)}\times\mathbb{R}^d} A^{ali}(v,v_*) f_\lambda(x_*,v_*,t) dx_* dv_*,
			\end{split}
		\end{equation*}
		with $A^{ali}(\cdot,\cdot)$, $A^{att}(\cdot,\cdot)$ defined as in equation \eqref{eq:aligment}-\eqref{eq:atraction}.
	Note that in \eqref{eq:G_test1}, when the term $\mathcal{G}[f]$ is partially aligned with the target direction $\bar x - x$, i.e., $\alpha(x,v,t;f) \geq \overline{\alpha}$, agents switch to, or remain in, leader status, naturally steering their dynamics towards the target $\bar x$.
	Conversely, if $\alpha(x,v,t;f) \leq \underline{\alpha}$, the agent with position and velocity $(x,v)$ remains in, or is switched to, follower status. Figure \ref{fig:configurations} illustrates two possible configurations.	
	% Figure environment removed
	
	
	\begin{remark}[Multiple-label case and continuous limit]\label{remark_continuos}
		%	\paragraph{Different levels of leadership.}
		We observe that the previous formulation can be extended to include multiple levels of leadership, up to a continuous space of labels \cite{cristiani2023kinmacr, albi2019leader}.
		Hence, we consider $\lambda \in \Lambda = \{\lambda_1,\ldots,\lambda_{N_\ell}\}$ such that $\lambda_k = k \Delta \lambda$ with $\lambda_1 = 0$ and $\lambda_{{N}_\ell} =1$. 
		The transition operator $\mathcal{T}_k[\cdot]$ in the multiple-label case for $k=2,\ldots,{N}_\ell-1$ reads
			\begin{equation}\label{eq:lambda_continuous}
			\mathcal{T}_k[f](t)= \Big(	\mathcal{T}_{k+1}^+[f](t)-	\mathcal{T}_{k}^+[f](t)\Big)  -	\Big(	\mathcal{T}_{k}^-[f](t)-	\mathcal{T}_{k-1}^-[f](t)\Big),
		\end{equation}
		and for the boundary values $\lambda_1=0, \lambda_{{N}_\ell} = 1$ we have
		\begin{equation}\label{eq:lambda_continuos_bc}
			\begin{split}
				\mathcal{T}_1[f](t)  &=	\mathcal{T}_{2}^+[f](t)-	\mathcal{T}_{1}^+[f](t),\\
				\mathcal{T}_{{N}_\ell}[f](t)  &=  \mathcal{T}_{{N}_\ell}^-[f](t)-	\mathcal{T}_{{N}_\ell-1}^-[f](t).
			\end{split}
		\end{equation}
		In the above expressions
			\begin{equation*}
			\begin{split}
				\mathcal T_{k}^+[f](t) := {\pi_{\lambda_{k}\rightarrow\lambda_{k-1}}} f_k(t),\qquad	\mathcal T_{k}^-[f](t) = {\pi_{\lambda_{k}\rightarrow\lambda_{k+1}}} f_k(t),
			\end{split}
		\end{equation*}
		where we denoted
		by $\pi_{\lambda_m \rightarrow \lambda_n} := \pi_{\lambda_m \rightarrow \lambda_n}(x,v,t;f)$  the transition rates from the state $\lambda_m$ to the state $\lambda_n$, and we used the shorten notation for the density $f_k(t) := f(x,v,\lambda_k,t)$ and the transition operator $\mathcal{T}_k[f](t) := \mathcal{T}[f](x,v,\lambda_k,t)$. Hence, the evolution of the density in the label space is ruled by 
		\begin{equation}\label{eq:master_multiple}
		\frac{d}{dt} f_k(t) = \mathcal{T}_k[f](t), \quad k=1,\ldots,{N}_\ell.
		\end{equation}
%		where we neglected the dynamics on the phase space $(x,v)$.
		%Furthermore,  we introduce also the boundary  labels, $\lambda_0 = \lambda_1 -\Delta \lambda$ and $\lambda_{\bar{N}+1} = \lambda_{\bar{N}} + \Delta \lambda$, with transition rates such that $\pi_{\lambda_0 \rightarrow \lambda_1} = \pi_{\lambda_1 \rightarrow \lambda_0} = \pi_{\lambda_{\bar{N}}\rightarrow \lambda_{\bar{N} + 1}} = \pi_{\lambda_{\bar{N}+1}\rightarrow \lambda_{\bar{N}}} = 0$. 
		Furthermore, from \eqref{eq:lambda_continuous} we can retrieve a transition operator for a continuous label space, $\lambda\in[0,1]$ by scaling the time by $1/\Delta \lambda$ and by considering the limit for ${N}_\ell\to\infty$
%		 and scaling  the transition rates as
%		$\pi_{\lambda_{k}\rightarrow\lambda_{k-1}}\to {\pi_{\lambda_{k}\rightarrow\lambda_{k-1}}}/{\Delta \lambda}$.
%		Under these assumptions, the transition operator \eqref{eq:lambda_continuos} writes
%		\begin{equation}\label{eq:lambda_continuous_1}
%			\mathcal{T}_k[f](t)= \frac{	\mathcal{T}_{k+1}^+[f](t)-	\mathcal{T}_{k}^+[f](t)}{\Delta \lambda}  -	\frac{	\mathcal{T}_{k}^-[f](t)-	\mathcal{T}_{k-1}^-[f](t)}{\Delta \lambda},
%		\end{equation}
%		where we denoted 
%		\begin{equation*}
%			\begin{split}
%				\mathcal T_{k}^+[f](t) := {\pi_{\lambda_{k}\rightarrow\lambda_{k-1}}} f_k(x,v,t),\qquad	\mathcal T_{k}^-[f](t) = {\pi_{\lambda_{k}\rightarrow\lambda_{k+1}}} f_k(x,v,t).
%			\end{split}
%		\end{equation*}
		and $\Delta \lambda \to 0$ in equation \eqref{eq:master_multiple},
		\begin{equation}\label{eq:master_continuous_2}
			\partial_t f_\lambda( t)= 
			\partial_{\lambda}\Big[\mathcal{T}_\lambda^{+}[f](t) - \mathcal{T}_\lambda^{-}[f]( t)\Big],
		\end{equation}
		for any $\lambda \in [0,1]$ where 
		\begin{equation*}
\mathcal{T}_\lambda^{+}[f]( t) = \lim_{\Delta \lambda \to 0 }  \frac{	\mathcal{T}_{k+1}^+[f](t)-	\mathcal{T}_{k}^+[f](t)}{\Delta \lambda},\qquad 
\mathcal{T}_\lambda^{-}[f]( t) = \lim_{\Delta \lambda \to 0 } \frac{	\mathcal{T}_{k}^-[f](t)-	\mathcal{T}_{k-1}^-[f](t)}{\Delta \lambda}.
		\end{equation*}
		
		 Since at the boundary we have no inflow and outflow of mass, thanks to \eqref{eq:lambda_continuos_bc}, we have 
		\begin{equation}\label{eq:bc}
			\partial_{\lambda}\Big[\mathcal{T}_\lambda^{+}[f]( t) - \mathcal{T}_\lambda^{-}[f]( t)\Big] = 0, \qquad \lambda\in\{0,1\}.
		\end{equation}
		Finally, we can write the master equation for the density integrating \eqref{eq:master_continuous_2} as follows
		\begin{equation}\label{eq:master_continuos}
			\partial_t\rho(\lambda,t) =		\partial_{\lambda}\int_{\RR^{2d}}	 \Big[\mathcal{T}_\lambda^{+}[f](x,v,t) - \mathcal{T}_\lambda^{-}[f](x,v,t)\Big] d(x,v),
		\end{equation}
		%	for any $\lambda \in (0,1)$. Since at the boundary we have no inflow and outflow of mass, for $\lambda \in \{0,1\}$, equation \eqref{eq:lambda_continuous_2} reads 
		%	\begin{equation}\label{eq:bc}
		%		\partial_{\lambda}\Big[\mathcal{G}^{+}(x,v,\lambda;f) - \mathcal{G}^{-}(x,v,\lambda;f)\Big] = 0. 
		%	\end{equation}
		%	Indeed, for $\lambda = \lambda_1$ equation \eqref{eq:lambda_continuous_0} writes 
		%	\begin{equation}\label{eq:master_continuos_bis}
		%		\partial_{t}f(x,v,\lambda_1,t) = \frac{\pi_{\lambda_2 \rightarrow \lambda_1} f(x,v,\lambda_2,t) - \pi_{\lambda_1 \rightarrow \lambda_2} f(x,v,\lambda_1,t)}{\Delta \lambda}.
		%	\end{equation}
		%	Since we are assuming to have no flux at the boundary, the right hand side of equation \eqref{eq:master_continuos_bis} is equal to zeros. Finally, integrating with respect to  $x$ and $v$ we get equation \eqref{eq:bc}.
		%	A similar computation can be performed to prove the result for $\lambda = \lambda_{\bar{N}}$. \\
		where, for transition operators of type  $\mathcal{T}_\lambda^{\pm}[f](x,v,t) = \kappa^\pm(\lambda) f_\lambda(x,v,t)$,
		we retrieve the transport equation for the density $\rho_\lambda(t)$ in the label space in the following form
		\begin{equation}\label{eq:master_continuos_1}
			\partial_t\rho_\lambda(t)  = \partial_{\lambda} \Big[ \left( \kappa^+(\lambda) - \kappa^-(\lambda)\right)  \rho_\lambda(t) \Big].
		\end{equation}
	\end{remark}
	%\textcolor{red}{
	%\begin{remark}\label{remark_delay}
	%	In \cite{cristiani2021all} authors assume that leaders emerge with probability $q\in[0,1]$ and return to the followers status with probability one if the distance from their nearest neighbor overcome $\bar{R}$ space units or if they remain in the leaders status for more than $\tau$ time units. The transition rates read as follows
	%	\begin{equation}\label{eq:rates} 
	%		\begin{split}
	%			&\pi_{L\to F}(x,v,\lambda;f,t,t-\tau) = \chi_{\{R_f(x,t)>\bar{R}\}}~ \lambda(t) + \lambda(t-\tau),\\ 
	%			&\pi_{F\to L}(x,v,\lambda;f,t,t-\tau) = q.
	%		\end{split}
	%	\end{equation}
	%\end{remark}
	%}
	
	
	\subsection{Mean-field asymptotics} \label{sec:kinetic_model_1}
	In order to retrieve asymptotic behaviour of the Boltzmann-type equation \eqref{eq:boltz_lin}, we  resort on a mean-field approximation of the interaction dynamics. Thus we introduce a grazing collision limit for the interaction operator \eqref{eq:Bo}, following the approach in \cite{pareschi2013interacting,carrillo2010particle}. Thus, we rescale the interaction frequency $\eta$ and the interaction propensity $\alpha$  to maintain asymptotically the memory of the microscopic interactions, as follows
	\begin{equation}\label{eq:scaling}
		\alpha = \varepsilon, \qquad \eta = \frac{1}{\varepsilon},
	\end{equation}
for $\varepsilon>0$,
	which corresponds to the case where the interaction kernel concentrates on
	binary interactions producing very small changes in the agents velocity but at the same
	time the number of interactions becomes very large.
	For now on, for simplicity we remove the dependence on time $t$. 
	We introduce the test function $\psi(x,v)\in C^1_0(\RR^d\times\RR^d)$ and  we write the weak form of the scaled kinetic equation \eqref{eq:boltz_lin} %collision operator $Q(\cdot,\cdot)$ in weak form as follows
	\begin{equation}
		\begin{split}\label{eq:boltz_weak_0}
			&\int_{\RR^{2d}} \left( \partial_t  f_\lambda(x,v) + v\cdot \nabla_x  f_\lambda(x,v)\right) \psi(x,v)d(x,v)- \int_{\RR^{2d}} \T_\lambda[ f](x,v)\psi(x,v)d(x,v) = \\
			&\quad\frac{1}{\varepsilon}\sum_{\lambda_*} \int_{\RR^{2d}}\int_{\mathit{B}_{r_*(x,t)}\times \RR^{d} } \left(\psi(x,v')-\psi(x,v)\right)f_{\lambda_*}(x_*,v_*)  f_\lambda(x,v) d(x,v)d(x_*,v_*),
		\end{split}
	\end{equation}
	%\begin{equation}\begin{split}\label{eq:collisional_op}
	%		\int_{\RR^{2d}} & Q(f,f)(x,v,\lambda)\psi(x,v)d(x,v) =\eta \sum_{\lambda_*\in\{0,1\}} \int_{\RR^{2d}}\int_{\mathit{B}_{r_*(x,t)}\times \RR^{d} }\ \left(\psi(x,v')-\psi(x,v)\right) df_* df ,
	%\end{split}\end{equation}
	%The scaled equation \eqref{eq:boltz_weak_0} reads as
	%\begin{equation}\label{eq:boltz_weak_scaled} 
	%	\begin{split}
	%		\int_{\RR^{2d}}& \left( \partial_t f(x,v,\lambda) + v\cdot \nabla_x f(x,v,\lambda)\right) \psi(x,v)d(x,v)- \int_{\RR^{2d}} \T[ f](x,v,\lambda)\psi(x,v)d(x,v) = \\
	%		&\frac{1}{\varepsilon}\sum_{\lambda_*\in\{0,1\}} \int_{\RR^{2d}}\int_{\mathit{B}_{r_*(t,x)}\times \RR^{d} } \left(\psi(x,v')-\psi(x,v)\right)df_* df ,
	%	\end{split}
	%\end{equation}
	with scaled  interactions \eqref{eq:binary} as follows
	\begin{equation}\label{eq:binary_scaled}
		v'-v =  \varepsilon \mathcal{F}_\lambda(x,x_*,v,v_*).
	\end{equation}
	Since as $\varepsilon \to 0$, we have $v'\to v$ we can expand $\psi(x,v')$ in Taylor series centred in $(x,v)$ up to second order and rewrite the right hand side of equation \eqref{eq:boltz_weak_0} as
	\begin{equation}\label{eq:boltz_weak_1}
		\begin{split}
			\frac{1}{\varepsilon} &\sum_{\lambda_*}\int_{\RR^{2d}}\int_{\mathit{B}_{r_*(x,t)}\times \RR^{d} }  \left(\psi(x,v')-\psi(x,v)\right) df_{\lambda_*} df_{\lambda} =\\&\frac{1}{\varepsilon} \sum_{\lambda_* } \int_{\RR^{2d}} \int_{\mathit{B}_{r_*(x,t)}\times \RR^{d} } \nabla_v \psi (x,v) \cdot (v'-v) df_* df+ R(\varepsilon),
		\end{split}
	\end{equation} 
	where we used the shorten notation $df_{\lambda_*}=f(x_*,v_*,\lambda_*)d(x_*,v_*)$, $df_\lambda=f(x,v,\lambda)d(x,v)$, and where $R(\varepsilon)$ indicates the remainder which is given by 
	\begin{equation}\label{eq:remainder}
		R(\varepsilon) = \frac{1}{2\varepsilon}   \sum_{\lambda_* } \int_{\RR^{2d}} \int_{\mathit{B}_{r_*(x,t)}\times \RR^{d} } \left[ \sum_{i,j = 1}^d \partial_v^{(i,j)}\psi (x,\bar{v})(v'-v)_i(v'-v)_j \right]  df_{\lambda_*} df_{\lambda},
	\end{equation}
	with 
	\begin{equation*}
		\bar{v}= \gamma v + (1-\gamma)v',
	\end{equation*}
	for some $\gamma \in [0,1]$.
	Therefore, the scaled binary interaction term \eqref{eq:boltz_weak_1} reads 
	\begin{equation}\label{eq:boltz_weak_1_scaled}
		\begin{split}
			\sum_{\lambda_*} \int_{\RR^{2d}} \int_{\mathit{B}_{r_*(x,t)}\times \RR^{d} } \nabla_v \psi (x,v) \cdot \mathcal{F}_\lambda(x,x_*,v,v_*)  df_{\lambda_*} df_{\lambda} + R(\varepsilon).
		\end{split}
	\end{equation} 
	Integrating equation \eqref{eq:boltz_weak_1_scaled} by parts and taking the limit $\varepsilon \to 0$ we have
	\begin{equation}\label{eq:boltz_weak_2}
		\begin{split}
			& \sum_{\lambda_*} \int_{\RR^{2d}}\int_{\mathit{B}_{r_*(x,t)}\times \RR^{d} } \nabla_v 
			\psi (x,v) \cdot \mathcal{F}_\lambda(x,x_*,v,v_*)   df_{\lambda_*} df_{\lambda}=\\&-
			\sum_{\lambda_*}\left\langle  \nabla_v \cdot \Bigl[f_\lambda(x,v) \int_{\mathit{B}_{r_*(t,x)}\times \RR^{d} }   \mathcal{F}_\lambda(x,x_*,v,v_*)    df_{\lambda_*}\Bigr], \psi(x,v)\right\rangle, 
		\end{split}
	\end{equation}
	where  we denoted the inner scalar product
	\begin{equation}\label{eq:scalar_product}
		\left\langle h, \phi \right\rangle : = \int_{\RR^{2d}} h(x,v) \phi(x,v) d(x,v), 
	\end{equation}
	for any function $h(x,v)$, $\phi(x,v)$ for which the integral in \eqref{eq:scalar_product} is well defined. By similar arguments of \cite{albi2016invisible}, it can be shown rigorously that $R(\varepsilon) \to 0$, as $\varepsilon \to 0$. 
	Thus, we can rewrite equation \eqref{eq:boltz_weak_0} as follows
	\begin{equation}\begin{split}\label{eq:boltz_weak}
			&\left\langle \partial_{t} f_\lambda(x,v) + v \cdot \nabla_x f_\lambda(x,v) - \T_\lambda[ f](x,v),\psi(x,v)\right\rangle  = \\
			&	 -\sum_{\lambda_*}\left\langle  \nabla_v \cdot \Bigl[f_\lambda(x,v) \int_{\mathit{B}_{r_*(x,t)}\times \RR^{d} }   \mathcal{F}_\lambda(x,x_*,v,v_*)   df_*\Bigr], \psi(x,v)\right\rangle.
	\end{split}\end{equation}
	Finally, we retrieve the mean-field equation as the strong form of \eqref{eq:boltz_weak} 
	\begin{equation}
		\begin{aligned}\label{eq:boltz_strong}
			& \partial_{t} f_\lambda(x,v) + v \cdot \nabla_x f_\lambda(x,v) - \T_\lambda[f](x,v) =\\
			&\qquad\quad	 - \nabla_v \cdot \Bigl[f_\lambda(x,v) \int_{\mathit{B}_{r_*(x,t)}\times \RR^{d} }   \mathcal{F}_\lambda(x,x_*,v,v_*)   \sum_{\lambda_*} f_{\lambda_*}(x_*,v_*)d(x_*,v_*)\Bigr].
		\end{aligned}
	\end{equation}
	Summing over the values of $\lambda$ in equation \eqref{eq:boltz_strong} the transition operator vanishes as in \eqref{eq:tnc} and we obtain the mean-field model for the total density $g(x,v)$ as 
	\begin{equation}
		\begin{split}\label{eq:boltz_strong_g}
			&\partial_{t} g(x,v) + v \cdot \nabla_x g(x,v) = 
			- \nabla_v \cdot \Bigl[\sum_{\lambda} f_\lambda(x,v)\int_{\mathit{B}_{r_*(x,t)}\times \RR^{d} }  \mathcal{F}_\lambda(x,x_*,v,v_*)  g(x_*,v_*) d(x_*,v_*)\Bigr].
		\end{split}
	\end{equation}
	\begin{remark}
		Note that the continuous mean-field model \eqref{eq:boltz_strong} and the microscopic one \eqref{eq:dynamics} are equivalent
		when we consider the empirical distribution of the $N$-particles
		\begin{equation}\label{eq:empirical_dist}
			f^N(x,v,\lambda,t) = \frac{1}{N} \sum_{i=1}^N \delta(x-x_i(t))\delta(v-v_i(t))\delta(\lambda-\lambda_i(t)),
		\end{equation}
		where $\delta(\cdot)$ indicates the Dirac-delta function.
	\end{remark}
	
	%%%%%%%%%%%%%%%%%%%%%%%%%%%%%%%%%%%%%%%%%%
	%%%%%%%%%%%%%%%%%%%%%%%%%%%%%%%%%%%%%%%%%
	%%%%%%%%%%%%%%%%%%%%%%%%%%%%%%%%%%%%%%%%
	\section{Stochastic particle-based approximation}\label{sec:numerical_methods}
	We aim at solving the large system of agent \eqref{eq:dynamics} for $N\gg1$, solving  the mean-field model  \eqref{eq:boltz_strong}  by means of the scaled Boltzmann equation in the asymptotic regime \eqref{eq:scaling}.
	In particular, we aim at developing  asymptotic stochastic algorithms for the simulation of the swarming dynamics, such as in \cite{albi2013binary, pareschi2013interacting}. These approaches, based on Monte-Carlo algorithms are based of direct simulation Monte-Carlo methods (DSMCs) for kinetic equations \cite{nanbu1986theoretical,pareschi2001time}. We mention also Random Batch Methods (RBMs) which, similarly, have been devised for simulating large systems of interacting agents \cite{jin2020random}.
	
	\subsection{Asymptotic Nanbu-type algorithm}
	In order to solve the mean-field dynamics we consider the Boltzmann-type equation \eqref{eq:boltz_lin} in the scaling limit \eqref{eq:scaling}, and we split the dynamics evaluating in three different steps the free transport, the label evolution and the interaction process, as follows
	\eqref{eq:boltz_lin}
	\begin{align}
		\partial_t f_\lambda(x,v) & = -v\cdot \nabla_x f_\lambda(x,v) \label{eq:boltz_transport}\\
		\partial_t f_\lambda(x,v) & = \mathcal{T}_\lambda[f](x,v)\label{eq:boltz_label}\\
		\partial_t f_\lambda(x,v) & = Q^{\varepsilon}_\lambda(f_\lambda,f_\lambda)(x,v).\label{eq:boltz_collision}
	\end{align}
	%then solving the interaction step 
	%\begin{equation}\label{eq:boltz_collision}
	%	\partial_t f = Q(f,f),
	%\end{equation}
	%
	%	\begin{equation} \label{eq:boltz_transport}
	%		\partial_t f = -v\cdot \nabla_x f,
	%	\end{equation}
	%then solving the interaction step 
	%\begin{equation}\label{eq:boltz_collision}
	%	\partial_t f = Q(f,f),
	%\end{equation}
	In order to approximate the time evolution of the density $f_\lambda(x,v,t)$ we assume to sample $N_s$ particles $(x_i^0,v_i^0,\lambda_i^0)$ from the initial distribution. We consider a time interval $[0, T]$ discretized in $N_t$ intervals of size $\Delta t$.
	
	\paragraph{Transport step.}
	First, we focus on the transport step in equation \eqref{eq:boltz_transport} and we approximate the solution at time $t^{n+1}$ by 
	\begin{equation}\label{eq:transport}
		x_i^{n+1} = x_i^{n} + \Delta t v_i^n, \qquad i = 1,\ldots,N_s
	\end{equation}
	\paragraph{Labels switching.}
	Secondly, we simulate how the labels change denoting by $f_\lambda^{n}$ the approximation of $f_\lambda(x, v,n\Delta t)$, and writing the discrete version of the equation \eqref{eq:boltz_label}, for the transition operator \eqref{eq:master_0} as follows 
	\begin{equation}\label{eq:lambda_evolution}
		\begin{split}
			f_0^{n+1} = (1-\Delta t ~\pi_{F\to L}^n)~ f_0^n+ \Delta t ~\pi_{L\to F}^n ~f_1^n,\\
			f_1^{n+1} = (1-\Delta t ~\pi_{L\to F}^n)~ f_1^n + \Delta t ~\pi_{F\to L}^n~ f_0^n,
		\end{split}
	\end{equation}
	%where $\pi_{F\to L}(\cdot)$ and $\pi_{L\to F}(\cdot)$ are the transition rates. then we 
	The following Algorithm \ref{alg_lambda} describes how to simulate equation \eqref{eq:lambda_evolution} in a time interval $[0,T]$ divided into $N_t$ time steps. 
	
	\begin{alg}~[Labels switching]~ \label{alg_lambda}
		\begin{enumerate}
			\item[\texttt 1.] Given $N_s$ samples $(x_i^0,v_i^0,\lambda_i^0)$ from the initial distribution $f_\lambda^0$; 
			\item[\texttt 2.] \texttt{for} $n=0$ \texttt{to} $N_t$ 
			\begin{enumerate}
				\item \texttt{for} $i=1$ \texttt{to}  $N_s$
				\begin{enumerate}
					\item compute the following probabilities rates 
					\[
					p_{L} =\Delta t ~\pi_{F\to L}, \qquad 	p_{F}= \Delta t ~\pi_{L\to F}, 
					\]  
					\item \texttt{if} $\lambda_i^n = 0$,\\  with probability $p_{L}$ agent $i$ becomes a leader: $\lambda_i^{n+1} = 1$,
					\item \texttt{if} $\lambda_i^n = 1$,\\  with probability $p_{F}$ agent $i$ becomes a follower: $\lambda_i^{n+1} = 0$,
				\end{enumerate}
				\texttt{end for}
			\end{enumerate}
			\texttt{end for}
		\end{enumerate}
	\end{alg}
	\paragraph{Interaction step.} Finally, we consider the interaction step \eqref{eq:boltz_collision} decomposing the interaction operator \eqref{eq:Bo} in its gain and loss part,
	\[
	Q^{\varepsilon}_\lambda(f_\lambda,f_\lambda) = \frac{1}{\varepsilon} \left[Q^{\varepsilon,+}_\lambda(f_\lambda,f_\lambda)- \rho^* f_\lambda \right],
	\]
	where $\rho^*=M/N$ is the topological mass. Considering a forward discretization we obtain
	\begin{equation}\label{eq:collision}
		f_\lambda^{n+1} =\left( 1-\frac{\rho^*\Delta t}{\varepsilon} \right)  f_\lambda^n + \frac{\rho^*\Delta t}{\varepsilon}  \frac{Q^{\varepsilon,+}(f_\lambda^n,f_\lambda^n )}{\rho^*}.
	\end{equation} 
	Equation \eqref{eq:collision} can be interpreted as follows. With probability $1-\rho^*\Delta t/\varepsilon$ an individual in position $x$, velocity $v$ and label $\lambda$ will not interact with other individuals and, with probability $\rho^*\Delta t /\varepsilon$, it will interact with another individual according to
	\begin{equation}\label{eq:binary_interaction}
		v^{n+1}_i = v^{n}_i + \varepsilon \mathcal{F}_{\lambda_i^n}(x^n_i,x^n_j,v_i^n,v_j^n),
	\end{equation}
	for any $i=1,\ldots,N_s$, and where $(x^n_j,v_j^n)$ is selected randomly among the nearest neighbours belonging to the topological ball $\mathit{B}_{r^*}(x_i,t)$. We will assume $\rho^*\Delta t = \varepsilon$ to maximize the total number of interactions and ensure that at each time step all agents interact with another individual with probability one.
	
	
	Note that the sampling procedure of agents from the topological ball  $\mathit{B}_{r^*}(x_i,t)$ can have extremely high computational costs, especially when the sample size is large, since it requires the explicit computation of the distances between each agent $i$ and all the others agents. 
	In order to improve the computational efficiency of this step we propose a procedure based on two steps: $a)$ an approximation of the topological ball, $b)$ $k$--Nearest Neighbours ($k$--NN) search.
	
	{\em $a)$ Topological ball approximation.} 
	To avoid the expensive procedure of computing the topological ball over the whole sample, we consider a subsample of size $N_c$ of the $N_s$ selected particles such that $N_c<N_s$, and we define the approximation to radius of the topological ball as follows
	\begin{equation}\label{eq:topological_ball}
		\tilde r_*(x_i,t) = \arg\min_{r>0}\left\{\frac{1}{N_c} \sum_{k=1}^{N_c}\chi_{{B}_{r}(x_i)}(x_k)\geq \mathcal \rho^* \right\},
	\end{equation}    
	where $\rho^*$ is the target topological mass. 
	
	{\em $b)$ \textit{$k$--NN} search.}  We perform a \textit{ $k$--NN search} over a $k$-$d$ binary tree. First, we construct the binary tree on the subsample of size $N_c$ in such a way to partition the space and organize the points optimally dividing them according to their medians. We assume that every leaf-node contains at most $N_l$ points. Then, we use a $k$-NN algorithm to find the $\rho^*N_c$ nearest neighbours to a given agent $i$, using the tree structure.  
	We will show in the numerical experiments that this algorithm reduces the computational costs from the original quadratic to logarithmic. 
	We refer to \cite{friedman1977algorithm} for further details about this procedure. 
	
	
	
	Algorithm \ref{alg_binary} describes how to solve equation \eqref{eq:collision} in a time interval $[0,T]$ divided into $N_t$ time steps. 
	
	\begin{alg}~[Asymptotic Nanbu algorithm]\label{alg_binary}
		\begin{enumerate}
			\item[\texttt 1.] Give  $N_s$ samples $(x_i^0,v_i^0,\lambda_i^0)$ from the initial distribution $f_\lambda^0$; 
			\item[\texttt 2.] Set the value of the topological mass $\rho^*$ and of the subsample size $N_c$;
			\item[\texttt 3.] \texttt{for} $n=0$ \texttt{to} $N_t$ 
			\begin{enumerate}
				\item select a subsample of size $N_c$,
				\item construct a binary tree over the subsample, with leaf-nodes of size at most $N_l$
				\item \texttt{for} $i=1$ \texttt{to}  $N_s$
				\begin{enumerate}
					\item find the $\rho^*N_c$ nearest agents using a $k$--NN search algorithm on the tree,
					\item select randomly an index $j$ among the   nearest neighbors,
					\item compute the velocity change $v_i^{n+1}$ as in equation \eqref{eq:binary_interaction},
					\item Update the position $x_i$ according to \eqref{eq:transport}, with $\rho^*\Delta t = \varepsilon$.
				\end{enumerate}
				\texttt{end for}
			\end{enumerate}
			\texttt{end for}
		\end{enumerate}
	\end{alg}
	
	%\begin{remark}\label{rmk:cost}
	%	Note that the main advantage of Algorithm \ref{alg_binary} is that it allows us to construct the topological ball $\mathcal{B}_{r_M^*(t,x_i)} $ reducing the computational cost from the original $\mathcal{O}(N_s^2)$ to $\mathcal{O}(M N_s \log{N_s})$. Indeed, in general to find the nearest neighbors to a given agent we should first compute the distances between the selected agent and all the others, then sort the results and finally identify the index of the $M$ nearest agents. We will call this procedure \textit{exhaustive search}. In Algorithm \ref{alg_binary} instead we perform a $k$-nn search on a binary tree. The computational cost required to construct the binary tree over the subsample of size $N_c$ is $\mathcal{O}(N_s \log{N_s})$ and the one needed to search for each agent its $M$ nearest neighbor is $\mathcal{O}(N_s M \log{N_s})$. Hence the total cost is  $\mathcal{O}(N_s M \log{N_s})$. For further details we refer to \cite{friedman1977algorithm}.
	%\end{remark}
	%%%%%%%%%%%%%%%%%%%%%%%%%%%%%%%%%%%%%%%
	%%%%%%%%%%%%%%%%%%%%%%%%%%%%%%%%%%%%%%
	%%%%%%%%%%%%%%%%%%%%%%%%%%%%%%%%%%%
	%%%%%%%%%%%%%%%%%%%%%%%%%%%%%%%%%%%%%%%%%%%%%%%%%%%%%%%%%%%%
	%%%%%%%%%%%%%%%%%%%%%%%%%%%%%%%%%%%%%%%%%%%%%%%%%%%%%%%%%%%%%%
	%%%%%%%%%%%%%%%%%%%%%%%%%%%%%%%%%%%%%%%%%%%%%%%%%%%%%%%%%%%%%%
	%\section{Numerical experiments} \label{sec:numerical_experiments} 
	
	\subsection{Numerical validation}\label{sec:validation} 
	In this section we perform different numerical experiments to test both the accuracy and the efficiency of the Asymptotic Nanbu Algorithm \ref{alg_binary} with $k$--NN search. 
	\paragraph{Accuracy.}\label{sec:accuracy} 
	Consider a model in which $N_s$ agents with position $x_i$ and velocity $v_i$ interact with their nearest $M$ neighbours without changing their labels and their position. Assume agents are subjected just to alignment forces. Hence, their dynamics at the microscopic level is governed by the following ODE  for $i=1,\ldots,N_s$,
	\begin{equation}\label{eq:validation_model_ord1}
		\dot{v}_i =\frac{1}{\rho^*}\frac{1}{N_s} \sum_{j=1}^{N_s}  (v_j-v_i)  \chi_{\mathcal{B}_M(x_i;\xx)}(x_j),
	\end{equation}
	where $\rho^* = M/N_s$ is the target topological mass.
	At the kinetic level, suppose that agents modify their velocity according to binary interactions. Assume that at any time step an agent with position and velocity $(x,v)$ meets another agent with position and velocity $(x_*,v_*)\in \mathit{B}_{r^*_M}(t,x)$ where $r^*_M$ is defined as in \eqref{eq:topological_ball}. Its post-interaction velocity is given by
	\begin{equation}\label{eq:binary_1}
		v' = v + \varepsilon (v_*-v),
	\end{equation}
	where $\varepsilon >0$ is a small parameter. 
	Recall that the ball $B_{r^*_M}(t,x)$ by definition contains a certain percentage of mass, that we suppose to be $\rho^*$. If we denote by $f(x,v,t)$ the density of agents at time $t$ with position $x$ and velocity $v$, then the kinetic equation describing its evolution reads 
	\begin{equation}\label{eq:boltzmann_ali}
		\partial_t f(x,v,t) = -\nabla_v \cdot  \Big[ f(x,v,t) \int_{\mathit{B}_{r^*(t,x)}\times\mathbb{R}^d} (v_*-v) f(x_*,v_*,t) dx_* dv_* \Big].
	\end{equation}
	The microscopic model in \eqref{eq:validation_model_ord1} can be solved exactly and the evolution of the velocity is given by 
	\begin{equation}\label{eq:exact_sol}
		v_i(t) = v_i(0)e^{-t} +(1-e^{-t}) \bar{v}_i,\qquad \text {with } \qquad\bar{v}_i = \frac{1}{\rho^*}\frac{1}{N_s} \sum_{j=1}^{N_s}  v_j  \chi_{\mathcal{B}_M(x_i;\xx)}(x_j).
	\end{equation}
	We choose as initial distribution the sum of two 2d-Gaussian in the plane $(x,v)$ one with mean $(1,0.5)$ and standard deviation $(0.25,0.125)$ and the other with mean $(-1,-0.5)$ and standard deviation $(0.25,0.125)$.
	The dynamics at the kinetic level is simulated with Algorithm \ref{alg_binary}, where we compute the velocity change as in equation \eqref{eq:binary_1}.  We suppose $N_s=10^5$, and $\varepsilon = 10^{-5},\ldots,10^0$. We perform the computations assuming the subsample is made with the $p = 100N_c/N_s$ percent of the total mass, for a certain $p$. We run $S=100$ simulations and in Figure \ref{fig:validation_ord2_ali} in the first row, we plot the initial velocity distribution, in the second row, the mean and the standard deviation as a shaded area of the velocity distribution at time $T=3$ for $N_s=10^4$, $\rho^*=0.01, 0.35,0.75$ and $\varepsilon =  10^{-3}$ for different values of $p$.
	
	% Figure environment removed

	In Figure \ref{fig:error_ord2_ali} for different values of $\rho^*$, the $L_2$-norm of the error between the solution to the kinetic equation in \eqref{eq:boltzmann_ali}, simulated by means of the Asymptotic Nanbu algorithm \ref{alg_binary} (one simulation) for different values of $p$, and the exact solution in \eqref{eq:exact_sol}. Note that we observe a saturation effect for $\varepsilon \approx 10^{-2}$. 
	% Figure environment removed

	%%%%%%%%%%%%%%%%%%%%%%%%%%%%%%%%%%%%%%%%%%%%%%%
	%%%%%%%%%%%%%%%%%%%%%%%%%%%%%%%%%%%%%%%%%%%%%%%%
\paragraph{Computational costs.}\label{sec:costs} We now compare the computational costs of the exhaustive search and the $k$--NN search. 
	The computational cost of an exhaustive search is $\mathcal{O}(d N_s^2)$, where $d$ is the space dimension and $N_s$ the number of particles. Indeed, first one needs to compute the distances between each point and all the others, with a cost of $\mathcal{O}(d N_s^2)$, and then to sort them, with a cost of $\mathcal{O}(N_s^2  log(N_s))$. The cost of a $k$--NN search is logarithmic in time. First one needs to organize agents optimally with a $k$-$d$ tree. The cost of this operation is proportional to $N_s log(N_s)$. Then the idea is to perform a search over the tree to select which are the nearest agents. It can be shown (see \cite{friedman1977algorithm}) that the $k$--NN search algorithm examines the nodes in optimal order, that is in order of increasing dissimilarities, and that the number of nodes that should be examined is proportional to $((\rho^*N_s)^{1/d}+1)^d$. Hence, the total cost to construct a $k$-$d$ tree and to perform the search over it is
	\begin{equation}\label{eq:cost_knn}
		\mathcal{O}(max(((\rho^*N_s)^{1/d}+1)^d log(N_s), N_s log(N_s)).
	\end{equation}
	In Figure \ref{fig:cost} we see the comparison between the computational cost to perform one exhaustive and one $k$--NN search as $N_s$ varies for different values of $\rho^*$. We set $N_s = 5\times 10^{3},\ldots, 1.5\times 10^4$.  The $k$--NN computational cost increases as $\rho^*$ increases. In Figure \ref{fig:cost_1}  we see a comparison between the computational costs of a $k$--NN search as $N_s$ varies for different subsamples percentage size $p$. We see that the computational cost decreases proportionally to the subsample size.
	% Figure environment removed 
	% Figure environment removed 
	%%%%%%%%%%%%%%%%%%%%%%%%%%%%%%%%%%%%%%%%%%%%%%%%%%%%%%%%%%
	%%%%%%%%%%%%%%%%%%%%%%%%%%%%%%%%%%%%%%%%%%%%%%%%%%%%%%%%%%%%%
	%%%%%%%%%%%%%%%%%%%%%%%%%%%%%%%%%%%%%%%%%%%%%%%%%%%%%%%%%%%%%%%
	\section{Numerical experiments} \label{sec:2D3Dexperiments} 
	We present different numerical experiments simulating the two and three dimensional dynamics both at the microscopic and mesoscopic levels.  The dynamics at microscopic level is discretized by a forward Euler scheme with a time step $\Delta t = 0.01$, whereas the evolution of the
	kinetic dynamics is approximated by the Asymptotic Nanbu  algorithm described in \ref{alg_binary} with $\varepsilon = 0.01$. The time evolution of the labels is computed with Algorithm \ref{alg_lambda} at both the microscopic and mesoscopic levels. In the microscopic case we set $N=400$. In the mesoscopic case we choose a sample
	of $O(N_s)$ particles, with $N_s =5\times 10^5$, and a subsample of $O(N_c)$ particles, with $N_c=10^4$ that corresponds to a percentage $p=2\%$ of the total mass, for the approximation of the density. We assume the topological target mass to be $\rho^* = 0.01$. Table \ref{tab:all_parameters} reports the parameters of
	the model that remain unchanged in the various scenarios.
	\begin{table}[!h]
		\begin{center}
			\caption{Model parameters for the different scenarios.}\label{tab:all_parameters}
			\begin{tabular}{c|ccccccccc}
				&$C_{rep}$ & $C_{ali}$ & $C_{att}$ & $C_{v}$ &s&$\overline r$ & $\underline{r}$&$\epsilon$ \\
				\hline
				2D model & 100 & 12 &  0.7 &5&10&200&1&200\\
				\hline
				3D model & 100 & 12 &  0.7 &5&10&350&20&150\\
				\hline
			\end{tabular}			
		\end{center}
	\end{table}
	The other parameters will be specified later.
	%%%%%%%%%%%%%%%%%%%%%%%%%%%%%%%%%%%%%%%%%%%%%%%%%%
	%%%%%%%%%%%%%%%%%%%%%%%%%%%%%%%%%%%%%%%%%%%%%%%%%%%
	%%%%%%%%%%%%%%%%%%%%%%%%%%%%%%%%%%%%%%%%%%%%%%%%%%%
	%%%%%%%%%%%%%%%%%%%%%%%%%%%%%%%%%%%%%%%%%%%%%
	%%%%%%%%%%%%%%%%%%%%%%%%%%%%%%%%%%%%%%%%%%%%
	
	\subsection{Numerical test in two spatial dimensions}\label{sec:test2D} 
	We consider the swarming dynamics  evolving on the spatial space $(x,y)\in \R^2$ and velocity space $(v_x,v_y)\in \R^2$.
	\subsubsection{Test 2D with no food sources}\label{sec:2Dtest_nofood} 
	Suppose the model includes no food sources, i.e. $C_{src} = 0$, and no attraction to the centre of mass, i.e. $C_{ctr} = 0$. We simulate the dynamics up to time $T = 500$, and we report in Figure \ref{fig:2D_initial_configuration_nofood} the initial configuration for both the microscopic and mesoscopic dynamics.
	% Figure environment removed  

	At time $t=0$, agents are normally distributed with mean $\mu = 500$ and variance $\sigma^2 = 25^2$ and are in the followers status. Labels change according to the transition rates defined in \eqref{eq:rates_test_0} with $q_{FL}=2\times 10^{-4}$ and $q_{LF}= 4\times 10^{-3}$.
	\paragraph{Microscopic case.}
 In Figure \ref{fig:micro2D_nofood_dynamics} we report three snapshots of the dynamics at time $t=50$, $t=300$ and $t=500$, for the dynamics without leaders' emergence (top row) and with leaders' emergence (bottom row). We observe that, without leaders, agents align and form a compact swarm, whereas  with leaders' emergence we observe the formation of different groups. The splitting is not symmetric since leaders' emergence occurs randomly and this is reflected in the cluster formation.
	% Figure environment removed

	\paragraph{Mesoscopic case.}
	In Figure \ref{fig:meso2D_nofood_dynamics} we report three snapshots of the dynamics at time $t=50$, $t=300$, $t=500$. In the first row, the time evolution of the total density and in red the velocity vector field of the leaders. In the second row, the evolution of the leaders' density. The behavior is similar to the one of the microscopic case, where we observe the formation of various clusters, and the emergence of leaders uniformly over the swarm density.
	
	% Figure environment removed

	In Figure \ref{fig:meso2D_nofood_percentages} the agents percentages for the dynamics with leaders. 
	% Figure environment removed
	The videos of the simulations of this subsection are available at \href{https://drive.google.com/drive/folders/1VsO4ffzQvbMb5mvG3pHLoEpA-QyKlkbL?usp=share_link}{[VIDEO]}.
	%%%%%%%%%%%%%%%%%%%%%%%%%%%%%%%%%%%%%%%%%%%%%%%%%%%%%%%%%%%%%%%
	%%%%%%%%%%%%%%%%%%%%%%%%%%%%%%%%%%%%%%%%%%%%%%%%%%%%%%%%%%%%%%%

	\subsubsection{Test 2D: two food sources }\label{sec:2Dtwofood} 
	Assume the model includes two food sources located in $x_1^{src} = (300,500)$ and $x_2^{src} = (1000,500)$. Assume $C_{ctr} = C_{src}=0.75$. Run the dynamics until time $T=200$.  In Figure \ref{fig:meso2D_initial_configuration} the initial configuration for the microscopic and mesoscopic case. 
	We assume that initially the $87.5\%$ of agents is in the followers status. Among them the $75\%$ is normally distributed with mean $\mu = 550$ and variance $\sigma^2 = 10^2$ while the $12.5\%$ is  normally distributed with mean $\mu = 650$ and variance $\sigma^2 = 50^2$. The remaining $12.5\%$ is in the leaders status and it is normally distributed with mean $\mu = 800$ and variance $\sigma^2 = 10^2$. New leaders emerge with higher probability where the followers concentration is higher. Leaders return in the follower status with higher probability if the followers concentration around their position is lower.   Hence we consider density dependent transition rates defined in equation \eqref{eq:rates_test} with $q_L = 4\times 10^{-3}$ and $q_F = 3\times 10^{-3}$.
	% Figure environment removed 

	\paragraph{Microscopic case.}
	In Figure \ref{fig:micro2D_dynamics} three snapshots of the dynamics at time $t=50$, $t=100$, $t=200$. Agents that at time $t=0$ were in the leaders status change immediately  their labels since no followers are positioned around them. A large group is attracted by one of the two food sources while the remaining part moves subjected just to attraction, repulsion and alignment forces without being attracted by the other food source. Once this smaller group  moves far away from the main group, leaders start to be attracted to the centre of mass. In late time, all agents join and move toward one of the two food sources. 
	% Figure environment removed
	%In Figure \ref{fig:micro2D_twofood_percentages} the agents percentages for the dynamics with leaders. 
	%% Figure environment removed
	%%%%%%%%%%%%%%%%%%%%%%%%%%%%%%%%%%%%%%%%%%%%%%%%%%%%%%%%%%%%%%%%%%%%%%%%%%%%%%%%%%%%%%%%%%%%%%%%%%%%%%%%%%%%%%%%
	\paragraph{Mesoscopic case.}
	In Figure \ref{fig:meso2D_dynamics} three snapshots of the dynamics at time $t=50$, $t=100$, $t=200$. In the first row the time evolution of the total density and in red the velocity vector field of the leaders. In the second row the time evolution of the leaders' density. The behaviour is similar to the one observed in the microscopic case.
	%% In Figure \ref{fig:meso2D_percentage} the mass percentages in time.  On the left $\phi = 0$ and on the right $\phi = 0.5$.
	% Figure environment removed




	In Figure \ref{fig:percentages} the time evolution of the percentages of leaders and followers for the two dimensional spatial test with two food sources. The agents percentages have been computed both by counting the effective number of followers and leaders per time steps, and as stationary solution to the master equation \eqref{eq:master_constant}. 
	The densities reach the positive equilibrium defined in equation \eqref{eq:stationary}. Indeed, the transition rates defined in \eqref{eq:rates_test} 
	can be approximated by constant values. In particular, for any fixed time $t>0$, for any $\lambda \in \{0,1\}$ and for any $x\in \RR^{d}$ we have 
	\begin{equation}
		\pi_{L\to F}(x,\lambda;f,t) = \bar{\alpha}(t), \qquad 	\pi_{F\to L}(x,\lambda;f,t) = \bar{\beta}(t) 
	\end{equation}
	where 
	\begin{equation}
		\bar{\alpha}(t) = \mathbb{E}_x\left( 	\pi_{L\to F}(\cdot)\right) , \qquad \bar{\beta}(t) =  \mathbb{E}_x\left( 	\pi_{F\to L}(\cdot)\right) ,
	\end{equation}
	with $\mathbb{E}_x(\cdot)$ denoting the mean value with respect to $x$. \\
	Similar results can be obtained for the 2D model without food sources since the transition rates are constants values,  by definition. 
	% Figure environment removed
	The videos of the simulations of this subsection are available at \href{https://drive.google.com/drive/folders/1KZ2vOeMNzmx8MBMX-_0kulGGcjQr_shi?usp=share_link}{[VIDEO]}.
	%%%%%%%%%%%%%%%%%%%%%%%%%%%%
	%%%%%%%%%%%%%%%%%%%%%%%%%%%%%%%%%%%%%%%%%%%%%

	\subsubsection{Test 2D: one food source}\label{sec:test2D_optimal_control} 
	Assume the model includes one food source located in $x_1^{src} = (300,500)$. Run the simulation until time $T = 120$. 
	Suppose labels change aiming at organizing agents toward a common target, that in this case is supposed to be the food source $x_1^{src}$. In particular, assume $\lambda$ varies with rates \eqref{eq:rates_opt} with $\bar{\alpha} = 0.7$ and $\underline{\alpha} = 0.3$.
	In Figure \ref{fig:initial_configuration_opt} the initial configuration for both the microscopic and mesoscopic case.
	% Figure environment removed 

	\paragraph{Microscopic case.}
	In Figure \ref{fig:micro2D_dynamics_opt} three snapshots of the dynamics at time $t=5$, $t=20$ and $t=50$ with leaders and with $\mathcal{G}[f](\cdot)$ chosen as in \eqref{eq:G_test1}. Followers are driven by leaders and reach the target position. 
	% Figure environment removed
	
	\paragraph{Mesoscopic case.}
	In Figure \ref{fig:meso2D_dynamics_opt} three snapshots of the dynamics at time $t=5$, $t=20$, $t=50$ with transition rates depending on the orientation according to \eqref{eq:rates_opt}. In the first row, the evolution of the whole mass, and in red the velocity vector field of the leaders. In the second row, the evolution of the leaders mass. 
	% Figure environment removed




	In Figure \ref{fig:velocity} we report in the first row  the angle velocity distribution at time $t=100$, $t=120$, $t=180$ and in the second row the correspondent velocity vector field, outlining the milling behaviour around the target positions $\overline x=x_1^{src}$.
	% Figure environment removed
	The videos of the simulations of this subsection are available at \href{https://drive.google.com/drive/folders/1RS0LyB18zSoyKqVmy99NBwsIyRvR-t_m?usp=share_link}{[VIDEO]}.
	%%%%%%%%%%%%%%%%%%%%%%%%%%%%%%%%%%%%%%%%%%%%%
	%%%%%%%%%%%%%%%%%%%%%%%%%%%%%%%%%%%%%%%%%%%%%

	%%%%%%%%%%%%%%%%%%%%%%%%%%%%%%%%%%%%%%%%%%%%%%%%%%%%%%%%%%
	%%%%%%%%%%%%%%%%%%%%%%%%%%%%%%%%%%%%%%%%%%%%%%%%%%%%%%%%%%%%%%%%%%%%%%%%%%%%%%%%%%%%%%%%%%%%%%%%%%%%%%%%%%%%%%%%%%
	\subsection{Numerical test in 3D with two food sources} \label{sec:test3D} 
	We consider the three dimensional model in space and velocity, simulating the swarming dynamics up time $T=200$.
	Initially agents are normally distributed with mean $\mu = 500$ and
	variance $\sigma^2 = 25^2$ in both spatial and velocity dimension, and are all in the followers status.
	We report in Figure \ref{fig:micro3D_initial_configuration} the initial configuration for both the microscopic and mesoscopic case. For the mesoscopic case, we also depict on the $(x,y)$ plane the projection of the spatial density. 
	% Figure environment removed

	Assume the two food sources to be located in $x_1^{src} = (200,500,500)$ and $x_2^{src} = (800,500,500)$. Suppose that leaders emerge with density-dependent transition rates as defined in \eqref{eq:rates_test} and where we assume the constants to be $q_L = 4\times 10^{-3}$ and $q_F = 3\times 10^{-3}$.   
	\paragraph{Microscopic case.}
	In Figure \ref{fig:micro3D_dynamics} three snapshots of the dynamics at time $t=50$, $t=100$, $t=200$. First row: $C_{centre}= 0$, $C_{food} = 0.75$. Agents split in two groups moving toward the two food sources. Second row: $C_{ctr}= 4$, $C_{src} = 0.75$. At final time agents move toward one of the two food sources.
	% Figure environment removed
	%In Figure \ref{fig:micro3D_food_percentages} the agents percentages for the 3D tests with food sources.  
	%% Figure environment removed
	%%%%%%%%%%%%%%%%%%%%%%%%%%%%%%

	\paragraph{Mesoscopic case.}
	In Figure \ref{fig:meso3D_dynamics} we report  three snapshots of the dynamics at time $t=50$, $t=100$, $t=200$ and in red the velocity vector field. We add also the density distribution of the whole flock projected over the plane $(x,y)$ and in red the leaders velocity vector field. First row: $C_{crt}= 0$, $C_{src} = 0.75$. Second row: $C_{ctr}= 4$, $C_{src} = 0.75$. The behaviour is similar to the one in the microscopic case. 
	% Figure environment removed
	%In Figure \ref{fig:3D_food_percentages} the agents percentages for the 3D tests with food sources in both the microscopic and mesoscopic case.  
	%% Figure environment removed
	The videos of the simulations of this subsection are available at \href{https://drive.google.com/drive/folders/1RS0LyB18zSoyKqVmy99NBwsIyRvR-t_m?usp=share_link}{[VIDEO].}
	%	\subsubsection{Test 3D with one food sources }\label{sec:3Donefood} 
	%Assume the model include just one food source located in $x_1^{food} = (200,500,500)$.   Suppose labels change with the aim to organize agents toward a common rate, that is with rates defined as in \eqref{eq:rates_opt}. 
	%\paragraph{Microscopic case.}
	%In Figure \ref{fig:micro3D_dynamics_opt} three snapshots of the dynamics at time $t=50$, $t=150$, $t=300$. Agents reach the target and start moving around it. 
	%% Figure environment removed
	%%In Figure \ref{fig:micro3D_food_percentages} the agents percentages for the 3D tests with food sources.  
	%%% Figure environment removed
	%%%%%%%%%%%%%%%%%%%%%%%%%%%%%%%
	%\paragraph{Mesoscopic case.}
	%In Figure \ref{fig:meso3D_dynamics_opt} three snapshots of the dynamics at time $t=50$, $t=150$, $t=300$ and in white the velocity vector field. We add also the density distribution of the whole flock projected over the plane $(x,y)$.
	%% Figure environment removed
	%%%%%%%%%%%%%%%%%%%%%%%%%%%%%%%%%%%%%%%%%%%%%%%%%%%%%
	%%%%%%%%%%%%%%%%%%%%%%%%%%%%%%%%%%%%%%%%%%%%%%%%%%%
	%\subsection{Agents percentages}\label{sec:agents_percentages} 
	%Let us focus on the mesoscopic two dimensional model. 
	%In Figure \ref{fig:percentages} the time evolution of the percentages of leaders and followers for the two dimensional tests without and with food sources. The agents percentages have been computed both numerically, by counting the effective number of followers and leaders per time steps, and theoretically as stationary solution to the master equation \eqref{eq:master_constant}. 
	%% Figure environment removed
	%The densities reach the positive equilibrium defined in equation \eqref{eq:stationary}. Indeed, it can be proven that both the transition rates defined in \eqref{eq:rates_test_0} and \eqref{eq:rates_test} 
	%can be approximated by constant values. In particular, for any fixed time $t>0$, for any $\lambda \in \{0,1\}$ and for any $x\in \RR^{d}$ we have 
	%\begin{equation}
	%	\pi_{L\to F}(x,\lambda;f,t) = \bar{\alpha}(t), \qquad 	\pi_{F\to L}(x,\lambda;f,t) = \bar{\beta}(t) 
	%\end{equation}
	%where 
	%\begin{equation}
	%		\bar{\alpha}(t) = \mathbb{E}_x\left( 	\pi_{L\to F}(\cdot)\right) , \qquad \bar{\beta}(t) =  \mathbb{E}_x\left( 	\pi_{F\to L}(\cdot)\right) ,
	%\end{equation}
	%where $\mathbb{E}_x(\cdot)$ denotes the mean value with respect to $x$.  
	%Similar results can be obtained for the other tests. 
	\section{Conclusions}\label{sec:conclusion}
In this paper, we have studied collective behaviour of birds under a follower-leaders dynamics, starting from a simplified version of the model presented in \cite{cristiani2021all}.Through the emergence of leaders, we recover the ability to split the initial configuration and initiate directional changes without the need of external influences. We derived a kinetic model to effectively depict the motion of a large swarm with transient leadership and topological interactions, and subsequently we simulated the dynamics introducing a novel stochastic particle method.
A significant emphasis was placed on studying topological interactions. We tackled the issue of the numerical evaluation of nearest Neighbors reducing the computational costs of the search from quadratic to logarithmic by optimally organizing agents in a binary tree and performing a $k$-NN search. Moreover, we directed our attention to transient leadership, showcasing how labels can change over time, particularly for driving agents towards a common target. Various strategies for leaders' emergence were explored, including continuous leadership levels, as introduced in Remark \ref{remark_continuos}. Additionally, it would be intriguing to describe the original model from a kinetic viewpoint, reintroducing delay, which, as demonstrated in \cite{cristiani2021all}, appears to play a crucial role in achieving desired configurations.
Finally, several questions arise concerning the study of non-local terms in high dimensions. For instance, it could be beneficial to further enhance the numerical scheme implemented, focusing on other useful strategies for approximating topological interactions.
	\section*{Acknowledgments}
	GA and FF were partially supported by the MIUR-PRIN Project 2022 No. 2022N9BM3N	``Efficient numerical schemes and optimal control methods for time-dependent PDEs 
	\bibliographystyle{abbrv}
	\bibliography{biblio}
\end{document}
