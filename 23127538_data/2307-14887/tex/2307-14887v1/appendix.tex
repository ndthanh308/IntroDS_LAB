\clearpage
\appendix

\section{Proofs}\label{app:Results}
\begin{lemma}\label{le:VposHom}
For any $i\in\{1,\ldots,d\}$ and $k\in\{0,\ldots,T-1\}$ let $S^{i}\overset{(d)}{=}S^i_{t_{k+1}}$ and define
\begin{align*}
    V_k^{i}(y, \left( s^{i},s^{i-}\right)):=\E_{S^{i-}|s^{i-}}\left[\E_{S^{i}| s^{i}}\left[\lambda V_{k+1}\left(y+q^is^i\left(\left(\frac{S^i}{s^i}\right)-1\right), \left(S^{i},S^{i-}\right)\right)\right]\right].
\end{align*}
Then we have that for every $\lambda>0$, $V_k^{i}\left(\lambda y, \left(\lambda s^{i},s^{i-}\right)\right)=\lambda V_k^{i}\left( y, \left( s^{i},s^{i-}\right)\right)$. 
\end{lemma}
\begin{proof}
%By backwards induction. First, $V_T\left(\lambda y, \left(\lambda s^{i},s^{i-}\right)\right)=|\lambda y|=\lambda |y| =\lambda V_T\left( y, \left( s^{i},s^{i-}\right)\right)$. Then, via the inductive step and a change of measure we get
First,
\begin{align*}
   % V_k(\lambda y, \left(\lambda s^{i},s^{i-}\right))=\max_{i, q^i\in\{\pm 1\}} 
   V_{T-1}^i\left(\lambda y, \left(\lambda s^{i},s^{i-}\right)\right)&=\E_{S^{i-}|s^{i-}}\left[\E_{S^{i}| \lambda s^{i}}\left[V_{T}\left(y\lambda+q^is^i\lambda\left(\left(\frac{S^i}{s^i\lambda}\right)-1\right), \left(S^{i}\frac{\lambda}{\lambda},S^{i-}\right)\right)\right]\right]\\
   =&\E_{S^{i-}|s^{i-}}\left[\E_{S^{i}| \lambda s^{i}}\left[\left|y\lambda+q^is^i\lambda\left(\left(\frac{S^i}{s^i\lambda}\right)-1\right)\right|\right]\right]\\
   =&\lambda \E_{S^{i-}|s^{i-}}\left[\E_{S^{i}| \lambda s^{i}}\left[\left|y+q^is^i\left(\left(\frac{S^i}{s^i\lambda}\right)-1\right)\right|\right]\right]\\
   =&\lambda \E_{S^{i-}|s^{i-}}\left[\E_{S^{i}|  s^{i}}\left[\left|y+q^is^i\left(\left(\frac{S^i}{s^i}\right)-1\right)\right|\right]\right]\\
   =&\lambda V_{T-1}^i\left( y, \left( s^{i},s^{i-}\right)\right),
\end{align*}
where the penultimate equality follows from a change of measure. Analogously, the same can be shown for $k\in\{0,\ldots,T-2\}$.
\end{proof}


\begin{lemma}\label{le:Vrepresentation}
For any fixed $s\in\R_+^d$,
%$i\in\{1,\ldots,d\}$
and $k\in\{1,\ldots,N\}$, define the function $\psi:\R\to\R_+$, 
$\psi(y):=V_k(y,s)$. %, S^{i-})$. 
There exist a probability measure $\mu$ (that depends on $s$) on $\R_+$ s.t.
\begin{equation}
    \psi(y)=\int_{\R_+}\max(|y|,z)\,d\mu(z).
\end{equation}
\end{lemma}
\begin{proof}
We first show by induction over k that uniformly in $s$
\begin{equation*}
    \lim_{x\to\infty}V_k(x,s)-x =0, \quad \forall k\in\{1,\ldots,N\}.
\end{equation*}

For $k=0$, we have $\psi(x)-x=V_0(x,s)-x=|x|-x$, which goes to zero for $x\to\infty$ uniformly in $s$. Assume now that for some $k\in\Np, \lim_{x\to\infty}V_k(x,s)-x =0$ uniformly in $s$. Then
\begin{align*}
    V_{k+1}(x,s)-x&=\max_{q\in\Diamond}\Exsk\left[V_k\left(\underbrace{x\left(\frac{S^i_{t_{k+1}}}{s^i}\right)+q^is^i\left(1-\left(\frac{S^i_{t_{k+1}}}{s^i}\right)\right)}_{=:\tilde{x}}, S_{t_{k+1}}\right)-x\right]\\
    &=\max_{q\in\Diamond}\Exsk\left[V_k\left(\tilde{x}, S_{t_{k+1}}\right)-\tilde{x}+\sum_{j=1}^dq^js^j\left(\frac{S^j_{t_{k+1}}}{s^j}-1\right)\right].
\end{align*}
This implies that
\begin{align*}
    \lim_{x\to\infty} V_{k+1}(x,s)-x&=\max_{q\in\Diamond}\Exsk\left[\underbrace{\left(\lim_{x\to\infty}V_k\left(\tilde{x}, S_{t_{k+1}}\right)-\tilde{x}\right)}_{=0}+\sum_{j=1}^dq^js^j\left(\frac{S^j_{t_{k+1}}}{s^j}-1\right)\right]\\
    &=\max_{q\in\Diamond}\Exsk\left[\sum_{j=1}^dq^js^j\left(\frac{S^j_{t_{k+1}}}{s^j}-1\right)\right]\\
    &=\max_{q\in\Diamond}\sum_{j=1}^dq^js^j\Exsk\left[\frac{S^j_{t_{k+1}}}{s^j}-1\right]=0.
\end{align*}

Analogously to \citep[Lemma 6.3.]{delbaen2002passport}, one can further show that
\begin{enumerate}
    \item $\psi$ is convex,
    \item $\psi(-x)=\psi(x)$,
    \item $\psi(x)\ge |x|$ and $\lim_{x\to\infty}\psi(x)/x=1$.
\end{enumerate}
The result then follows from the proof of \cite[Lemma 6.4.]{delbaen2002passport}.

\end{proof}

\begin{remark}\label{re:varphiRepresentations}
Note that with the notation of \Cref{def:pmInvestment}, we have (dropping the time index in the notation of the asset)
\begin{align*}
 \varphi^i_+(z)&:=\Esicxsk\left[\max\left\{\bigg|S^i\left(\frac{|x|-s^i}{s^i}\right)+s^i\bigg|,z\right\}\right],\\
        \varphi^i_{-}(z)&:=\Esicxsk\left[\max\left\{\bigg|S^i\left(\frac{|x|+s^{i}}{s^i}\right)-s^i\bigg|,z\right\}\right].
    %  \varphi^i_+(z)&:=\Esicxsk\left[\max\left\{\bigg|S^i-(s^i-|x|)\bigg|,z\right\}\right],\\
    %       \varphi^i_-(z)&:=\Esicxsk\left[\max\left\{\bigg|S^i-(s^i+|x|)\bigg|,z\right\}\right],\\
\end{align*}
for $S^i\sim\logN\left(\log(s^i)-\frac{(\sigma^i)^2\Delta t_{k+1}}{2}, \sigi\sqrt{\Delta t_{k+1}}\right)$, with $\Delta t_{k+1}:=t_{k+1}-t_{k}$. Further, let $\kappa:=\frac{|x|+s^{i}}{s^i}, \tilde{\kappa}:=\frac{|x|-s^{i}}{s^i}$. Then we get
 \begin{align*}
     \varphi^i_-(z)&=\Esicxsk\left[z+\left(S^i\kappa-s^i-z\right)_++\left(-S^i\kappa+s^i-z\right)_+\right],\\
     &=z+\Esicxsk\left[\left(S^i\kappa-(s^i+z)\right)_+\right]\\
     &\hspace{3mm}+\Esicxsk\left[\left(S^i\kappa-(s^i-z)\right)_+\right]-\Esicxsk\left[S^i\kappa\right]+s^i-z\\
     &=\Esicxsk\left[\left(S^i\kappa-(s^i+z)\right)_+\right]+\Esicxsk\left[\left(S^i\kappa-(s^i-z)\right)_+\right]-|x|.
\end{align*}
Analogously,
\begin{align*}
      \varphi^i_+(z)&=\Esicxsk\left[\left(S^i\tilde{\kappa}+(s^i+z)\right)_+\right]+\Esicxsk\left[\left(S^i\tilde{\kappa}+(s^i-z)\right)_+\right]-|x|.
\end{align*}
\end{remark}


\begin{lemma}\label{le:negSignMax}
With the notation of \Cref{def:pmInvestment}, we have
\begin{equation*}
    \max\left\{\int_{\R_+}\varphi^i_+(z)\,d\muim(z),\int_{\R_+}\varphi^i_-(z)\,d\muim(z)\right\}=\int_{\R_+}\varphi^i_-(z)\,d\muim(z),\quad i=1\ldots, d,
\end{equation*}
provided that $x\neq 0$.
\end{lemma}
\begin{proof}
Let $i\in\{1,\ldots,d\}$ and define $f:\R_+\to\R, f(z):=\varphi^i_-(z)-\varphi^i_+(z)$ for $i=1,\ldots,d$. We will show that
$f(z)\ge0$ for all $z\in\R_+$.
By \cite[Lemma 6.5.]{delbaen2002passport} this follows, if \begin{enumerate}
    \item $\lim_{z\to\infty}f(z)=0$,
    \item for some $z$ big enough, $f(z)>0$,
    \item $f(0)\ge0$,
    \item there is at most one $z_0>0$ such that $f'(z)=0$.
\end{enumerate}
By \Cref{re:varphiRepresentations}, \begin{align*}
f(z)&=\Esicxsk\left[\left(S^i\kappa-(s^i+z)\right)_+\right]+\Esicxsk\left[\left(S^i\kappa-(s^i-z)\right)_+\right]\\
&-\Esicxsk\left[\left(S^i\tilde{\kappa}+(s^i+z)\right)_+\right]-\Esicxsk\left[\left(S^i\tilde{\kappa}+(s^i-z)\right)_+\right]
% f(z)&=\Esicxsk\left[\left(S^i-(s^i+|x|+z)\right)_+\right]+\Esicxsk\left[\left(S^i-(s^i+|x|-z)\right)_+\right]\\
% &-\Esicxsk\left[\left(S^i-(s^i-|x|+z)\right)_+\right]-\Esicxsk\left[\left(S^i-(s^i-|x|-z)\right)_+\right]
.\end{align*}
\begin{enumerate}
    \item For large enough $z$ we thus have
    
    \begin{align*}
        \lim_{z\to\infty}f(z)&= \lim_{z\to\infty}\Esicxsk\left[\left(S^i\kappa-(s^i-z)\right)_+-\left(S^i\tilde{\kappa}+(s^i+z)\right)_+\right]\\
        &=  \lim_{z\to\infty}\Esicxsk\left[\left(S^i\kappa-(s^i-z)\right)-\left(S^i\tilde{\kappa}+(s^i+z)\right)\right]=0\\
    \end{align*}
 and thus $\lim_{z\to\infty}f(z)=0$.
 \item By continuity of $f$, there exists $z$ such that $f(z)>0$, due to 3.
    \item \begin{align*}
        f(0)&=\Esicxsk\left[\bigg|S^i\left(\frac{|x|+s^{i}}{s^i}\right)-s^i\bigg|\right]-\Esicxsk\left[\bigg|S^i\left(\frac{|x|-s^i}{s^i}\right)+s^i\bigg|\right]\\
        &=\Esicxsk\left[\bigg|S^i-(s^i+|x|)\bigg|\right]-\Esicxsk\left[\bigg|S^i-(s^i-|x|)\bigg|\right]
    \end{align*} and thus $f(0)>0$ by \Cref{le:medianIneq}.
    \item For any $z>0$
    \begin{align*}
        \frac{\partial}{\partial z}f(z) %&= \Exsk\left[1_{|S^i-(s^i+|x|)|<z}-1_{|S^i-(s^i-|x|)|<z}\right]\\
    %  &=-\P\left[(s^i+|x|)-z<S^i\le(s^i+|x|)+z\right]+\P\left[(s^i-|x|)-z<S^i<(s^i-|x|)+z\right]\neq 0
    &=\P\left[S^i\kappa>s^i+z\right]+\P\left[S^i\kappa>s^i-z\right]-\P\left[S^i\tilde{\kappa}>-s^i-z\right]-\P\left[S^i\tilde{\kappa}>-s^i+z\right]\\
    &=-\P\left[s^i-z<S^i\kappa\le s^i+z\right]+\P\left[s^i-z<S^i\tilde{\kappa}\le s^i+z\right]\neq 0
    \end{align*}
since %$\logN$ is non-symmetric and 
$\kappa\neq\tilde{\kappa}$.
\end{enumerate}

Thus $f(z)\ge0$ for every $z\ge0$ and hence
\begin{align*}
    \int_{\R_+}\varphi^i_+(z)\,d\muim(z)<\int_{\R_+}\varphi^i_-(z)\,d\muim(z).
\end{align*}
\end{proof}

\begin{lemma}\label{le:medianIneq}
Let $S\sim\logN(\mu,\sigma)$ with median $m=e^\mu$, and $c_1,c_2\in\R$ s.t. $c_1>\max\{m,c_2\}$. Then
\begin{equation}
    |m-c_1|>|m-c_2|\implies\E[|S-c_1|]>\E[|S-c_2|].
\end{equation}
\end{lemma}
\begin{proof}
    Note that $c_1\ge m$. First, we re-write
    \begin{align*}
        \E[|S-c_1|]=\E[|S-m+\overbrace{m-c_1}^{=-|m-c_1|}|]=& \E[\left(|S-m|+|m-c_1|\right)\1_{\{S\le m\}}]\\
        &+\E[-\left(|S-m|-|m-c_1|\right)\1_{\{ m<S\le c_1\}}]\\
         &+\E[\left(|S-m|-|m-c_1|\right)\1_{\{c_1\le S\}}]\\
         =& \E[|S-m|]-2\E[|S-m|\1_{\{ m<S\le c_1\}}]\\
         &+|m-c_1|\underbrace{\left(\P[S\le c_1]-\P[S\ge c_1]\right)}_{=2\P[m\le S\le c_1]}.
    \end{align*}
    Denoting by $f$ and $F$ the density and cdf of $S$ respectively, we thus get
    \begin{align*}
         \E[|S-c_1|]&=\E[|S-m|]+2\left(\int_m^{c_1}\left[(m-s)-(m-c_1)\right]f(s)\,ds\right)\\
         &=\E[|S-m|]+2\left((c_1-s)F(s)\bigg|^{c_1}_m+  \int_m^{c_1}F(s)\,ds\right) \\
         &=\E[|S-m|]+2\left(0.5(c_1-m)+ \int_m^{c_1}F(s)\,ds\right).
         \end{align*}
    Furthermore we distinguish two cases:
    \begin{enumerate}
        \item $m<c_2<c_1$: Analogously to above we get
        \begin{align*}
             \E[|S-c_2|]&=\E[|S-m|]+2\left(0.5(c_2-m)+ \int_m^{c_2}F(s)\,ds\right)\\
             &<\E[|S-m|]+2\left(0.5(c_1-m)+ \int_m^{c_1}F(s)\,ds\right)\\
             &=\E[|S-c_1|],
        \end{align*}
        since $c_1>c_2$ and $F(s)>0$ for all $s$.
        \item $c_2<m<c_1$: We first re-write
        \begin{align*}
             \E[|S-c_2|]&=\E[\left(|S-m|+|m-c_2|\right)\1_{\{ m\le S\}}]\\
        &+\E[\left(-|S-m|+|m-c_2|\right)\1_{\{ c_2<S\le m\}}]\\
         &+\E[-\left(-|S-m|+|m-c_2|\right)\1_{\{ S\le c_2\}}]\\
         &=\E[|S-m|]-2\E[|S-m|\1_{\{ c_2<S\le m\}}]\\
         &+|m-c_2|\underbrace{\left(\P[c_2\le S]-\P[S\le c_2]\right)}_{=2\P[c_2\le S\le m]}.
        \end{align*}
        Furthermore,
         \begin{align*}
         \E[|S-c_2|]&=\E[|S-m|]+2\left(\int_{c_2}^{m}\left[(s-m)+(m-c_2)\right]f(s)\,ds\right)\\
         &=\E[|S-m|]+2\left((s-c_2)F(s)\bigg|_{c_2}^m-  \int_{c_2}^{m}F(s)\,ds\right) \\
         &=\E[|S-m|]+2\left(0.5(m-c_2)- \int^m_{c_2}F(s)\,ds\right)\\
         &<=\E[|S-m|]+2\left(0.5(m-c_1)\right)\\
         &<\E[|S-c_1|],
         \end{align*}
         since by assumption $|m-c_1|>|m-c_2|$.
    \end{enumerate}
\end{proof}

\begin{lemma}\label{le:varphi_CP_condition}
    Let $i,j\in\{1,\ldots,d\}$, %$|x|,s^i\neq s^j, \sigi>\sigma^j$ 
    and assume the notations as in the proof of \Cref{le:whichAssetIsMax}, and $\varphi_-$ as in \Cref{def:pmInvestment}. It holds that\begin{align*}
    %  \int_{\R_+}DC^{i^*}(z)\,d\mu(z)=\max_{i=1\ldots,d} \int_{\R_+}DC^i(z)\,d\mu(z)
    \varphi^j_-(z)-\varphi^i_-(z)\ge 0, \forall z\ge 0,
\end{align*}
if and only if 
\begin{align*}
CP^j(s^j/\kappa^j)\kappa^j>CP^i(s^i/\kappa^i)
\kappa^i,   % i^*&=\argmax_i s^i\Phi(d_1^i)-\left(|x|+s^i\right)\Phi(d_1^i-\sigi\sqrt{\Delta t_{k+1}})\\
    % d_1^i&=
    % \frac{1}{\sigma^i\sqrt{\Delta t_{k+1}}}\ln\left(\frac{s^i}{s^i+|x|}\right)+\frac{1}{2}\sigma^i\sqrt{\Delta t_{k+1}}.
\end{align*}
where $CP^i(K)$ denotes the call price with maturity $\Delta t_n$ on asset $S^i$ with strike $K$.
\end{lemma}

\begin{proof}
    Let $S^i$ respectively $ S^j$ as in \Cref{re:varphiRepresentations} and
\begin{equation*}
    f(z):=  \varphi^j_-(z)-\varphi^i_-(z).
\end{equation*}


Note that for $z$ large enough, 
\begin{align*}
    f(z)&= |x|+ \Esicxsk[j]\left[\left(S^j\kappa^j-(s^j-z)\right)_+\right]-|x|-\Esicxsk[i]\left[\left(S^i{\kappa^i}-(s^i-z)\right)_+\right]\\
        &=  \Esicxsk[j]\left[S^j\kappa^j-(s^j-z)\right]-\Esicxsk[i]\left[S^i{\kappa^i}-(s^i-z)\right]\\
         &= |x|+z-|x|-z=0.\\
%   \varphi^i_-(z)&=\Esicxsk\left[\left(S^i-(s^i+|x|+z)\right)_+\right]+\Esicxsk\left[S^i-(s^i+|x|-z)\right]\\
%     &=\Esicxsk\left[\left(S^i-(s^i+|x|+z)\right)_+\right]+|x|+z, \quad \forall i,
\end{align*}
% and thus for such a $z>\max_i(s^i+|x|)$
% \begin{align*}
%     f(z)&=\Esicxsk\left[\left(S^i-(s^i+|x|+z)\right)_+\right]\\
%     &\hspace{5mm}-\E_{S^{j}\mid x,s,k}\left[\left(S^j-(s^j+|x|+z)\right)_+\right].
% \end{align*}
Moreover, by the Black Scholes pricing formula for every $i=1\ldots, d$ and $z\ge0$ we have
\begin{align}
    \Esicxsk\left[\left(S^i\kappa^i-(s^i+z)\right)_+\right]&=\kappa^i\left(s^i \Phi(d_1^i)-\frac{s^i+z}{\kappa^i}\Phi(d_2^i)\right), \\
    &=(s^i+|x|) \Phi(d_1^i)-(s^i+z)\Phi(d_2^i), \\
    d_1^i&=\frac{\log((s^i+|x|)/(s^i+z))+\frac{1}{2}(\sigma^i)^2\Delta t_{k+1}}{\sigi\sqrt{\Delta t_{k+1}}},\notag\\
    d_2^i &=d_1^i-\sigi\sqrt{\Delta t_{k+1}}.\notag\\
\end{align}
Furthermore, for $z=0$, 
\begin{align*}
     \varphi_-^i(z)&=|x|+2\kappa^i\Esicxsk\left[\left(S^i-\frac{s^i}{\kappa^i}\right)_+\right].
\end{align*}
%is twice the call price of asset $S^i$ with strike $s^i+|x|$.

Thus, the following hold.
\begin{enumerate}
\item$\lim_{z\to\infty}f(z)=0.$
    %  \begin{align*}
    %     \lim_{z\to\infty}f(z)&=\lim_{z\to\infty}  (2|x|+z)\left(1-F^+\right)=\bigg\{\begin{matrix}+\infty,& F^+<1\\
    %     -\infty,& F^+\ge1\end{matrix}.
    % \end{align*}
    %If $s^i\neq s^j$, then $F^+<1$ for small $\Delta t_n$ and hence $ \lim_{z\to\infty}f(z)>0.$
\item
Since $f$ is continuous, 3. implies that there exists some $z_0$ such that $f(z_0)\ge 0$.
\item \begin{align*}
    f(0)&=|x|+2\kappa^j\left(\Esicxsk[j]\left[\left(S^j-\frac{s^j}{\kappa^j}\right)_+\right]\right)-|x|-2\kappa^i\left(\Esicxsk\left[\left(S^i-\frac{s^i}{\kappa^i}\right)_+\right]\right)\\
    &=2\left((s^j+|x|) \Phi(d_1^j)-s^j\Phi(d_2^j)-(s^i+|x|) \Phi(d_1^i)+s^i\Phi(d_2^i)\right)\big|_{z=0}
\end{align*}
% By monotonicity of the cdf $\Phi$, $f(0)> 0$ if and only if

We have $f(0)> 0$ if and only if
\begin{align*}
%s^i\Phi(d_1^i)-\left(|x|+s^i\right)\Phi(d_1^i-\sigi\sqrt{\Delta t_{k+1}})>s^j\Phi(d_1^j)-\left(|x|+s^j\right)\Phi(d_1^j-\sigma^j\sqrt{\Delta t_{k+1}}).
&\left((s^j+|x|) \Phi(d_1^j)-s^j\Phi(d_2^j)-(s^i+|x|) \Phi(d_1^i)+s^i\Phi(d_2^i)\right)\big|_{z=0}>0.
\end{align*}
\item For any $z>0$
    \begin{align*}
        \frac{\partial}{\partial z}f(z)
     &=\P\left[S^j\kappa^j>s^j+z\right]+\P\left[S^j\kappa^j>s^j-z\right]-\left(\P\left[S^i{\kappa^i}>s^i+z\right]+\P\left[S^i{\kappa^i}>s^i-z\right]\right)\\
     &=-\P\left[s^j-z<S^j\kappa^j\le s^j+z\right]+\P\left[s^i-z< S^i{\kappa^i}\le s^i+z\right]\\
     %&\bigg\{\begin{matrix}>(1-F^+)-1+F^+=0,& F^+<1\\
     %<(1-F^+)-1+F^+=0,& F^+\ge1\end{matrix}
     %P\left[S^j<s^j+|x|+z\right]+\P\left[S^j<s^j+|x|-z\right]\\
     %&-P\left[S^i<s^i+|x|+z\right]-\P\left[S^i<s^i+|x|-z\right]\\
          %&=\left(P\left[S^j-s^j<|x|+z\right]-P\left[S^i-s^i<|x|+z\right]\right)\\
          %&+\left(\P\left[S^j-s^j<|x|-z\right]-\P\left[S^i-s^i<|x|-z\right]\right),\\
    \end{align*}
%i.e., we are summing differences of two centered $\logN$ cdfs. Since the lognormal is not symmetric around its mean, $\frac{\partial}{\partial z}f(z)$ can only be zero if $|x|=z=0$.
Therefore, $\frac{\partial}{\partial z}f(z)$ can only be zero if and only if $$\P\left[s^j-z<S^j\kappa^j\le s^j+z\right]=\P\left[s^i-z< S^i{\kappa^i}\le s^i+z\right].$$
Thus, unless $s^i=s^j$, and $\sigi=\sigma^j$, for any $z>0$ $\frac{\partial}{\partial z}f(z)\neq 0$. 
\end{enumerate}

Thus, by \cite[Lemma 6.5.]{delbaen2002passport}, it follows from 1.-4. that $f(z)\ge0$ for all $z\ge0$ if and only if
\begin{align*}
%s^i\Phi(d_1^i)-\left(|x|+s^i\right)\Phi(d_1^i-\sigi\sqrt{\Delta t_{k+1}})>s^j\Phi(d_1^j)-\left(|x|+s^j\right)\Phi(d_1^j-\sigma^j\sqrt{\Delta t_{k+1}}).
&\left((s^j+|x|) \Phi(d_1^j)-s^j\Phi(d_2^j)-(s^i+|x|) \Phi(d_1^i)+s^i\Phi(d_2^i)\right)\big|_{z=0}>0.
\end{align*}
\end{proof}


\begin{lemma}\label{le:varphi_CP_condition_Sufficient}
    Let $i,j\in\{1,\ldots,d\}$,% $|x|,s^i\neq s^j, \sigi>\sigma^j$ 
    and assume the notation as in the proof of \Cref{le:whichAssetIsMax}.
    %, and $\varphi_-$ as in \Cref{eq:Defvarphii}.
    It holds that\begin{align*}
    %  \int_{\R_+}DC^{i^*}(z)\,d\mu(z)=\max_{i=1\ldots,d} \int_{\R_+}DC^i(z)\,d\mu(z)
    \eqref{eq:CPcondition}\implies \int_{\Rp}\varphi^j_-(z)-\varphi^i_-(z)\,dz\ge 0.
\end{align*}

\end{lemma}

\begin{proof}
Recall for all $i$ the definition
  \begin{align*}
             \varphi^i_-(z):=&\Exsk\left[\max\{|\kappai{S^i_{T-1}}-s^i|,z\}\,\sum_{l=1}^d \frac{2\Phi'(d_1^l)}{(z+\Sl_{T-1})\sigl\sqrt{\Delta T}}M_l(z,\Sl)\right]\\
             &+2\Exsk\left[\left|\kappai{S^i_{T-1}}-s^i\right|\sum_{l=1}^d\left(2\Phi(\sigl\sqrt{\Delta T}/2)-1\right)\1_{\{\Sl_{T-1}\CPlOO\text{ is max}\}}\right].
         \end{align*}
    Let $S^i$ respectively $ S^j$ as in \Cref{re:varphiRepresentations} and
\begin{equation*}
    f(z):=  \varphi^j_-(z)-\varphi^i_-(z).
\end{equation*}

         For further derivations, we note that
         \begin{itemize}
             \item[a)]  $\varphi^i_-(\cdot)$ are continuous, and thus also $f$ is continuous. 
             \item[b)]for all $i$, \begin{align}\label{eq:varphii0}
             \varphi^i_-(0)=&\Exsk\left[\left|\kappai{S^i_{T-1}}-s^i\right|\sum_{l=1}^d\left(4\Phi\left(\frac{\sigl\sqrt{\Delta T}}{2}\right)-2+\frac{2\Phi'\left(\frac{\sigl\sqrt{\Delta T}}{2}\right)}{\Sl_{T-1}\sigl\sqrt{\Delta T}}\right)\1_{\{\Sl_{T-1}\CPlOO\text{ is max}\}}\right].
         \end{align}
                \item[c)] For all fixed $\Delta T>0$ $\varphi^9_-(0)$ is finite:
                \begin{align*}
                      \varphi^i_-(0)&\le\Exsk\left[\left|\kappai{S^i_{T-1}}-s^i\right|\right]\max_{l=1,\ldots d}\left(4\Phi\left(\frac{\sigl\sqrt{\Delta T}}{2}\right)-2\right)\\
                      &+\sum_{l=1}^d \underbrace{\Exsk\left[\left|\kappai{S^i_{T-1}}-s^i\right|\frac{2\Phi'\left(\frac{\sigl\sqrt{\Delta T}}{2}\right)}{\Sl_{T-1}\sigl\sqrt{\Delta T}}\right]}_{=:A^{i,l}}\\
                      &<\infty.
                \end{align*}
                \begin{small}Note that term $A^{i,l}$ can be further re-written. For $i\neq l$, due to independence of the assets, we get
        \begin{align*}
            A^{i,l}&=\Exsk\left[\left|\kappai{S^i_{T-1}}-s^i\right|\right]\frac{2\Phi'\left(\frac{\sigl\sqrt{\Delta T}}{2}\right)}{\sigl\sqrt{\Delta T}}\frac{1}{\ssl}\exp((\sigl)^2\Delta T)\\
            &=\left(2CP^i_{\kappai}(1)\ssi-|x|\right)\frac{2\Phi'\left(\frac{\sigl\sqrt{\Delta T}}{2}\right)}{\sigl\sqrt{\Delta T}}\frac{1}{\ssl}\exp((\sigl)^2\Delta T).
        \end{align*}
        Furthermore,
                \begin{align*}
            A^{i,i}&=\Exsk\left[\left|\kappai{\Si_{T-1}}-\ssi\right|\frac{1}{\Si_{T-1}}\right]\frac{2\Phi'\left(\frac{\sigi\sqrt{\Delta T}}{2}\right)}{\sigi\sqrt{\Delta T}}\\
            &=\Exsk\left[\left|\frac{\ssi}{\Si_{T-1}}-\kappai\right|\right]\frac{2\Phi'\left(\frac{\sigi\sqrt{\Delta T}}{2}\right)}{\sigi\sqrt{\Delta T}}\\
            &=\Exsk\left[\left|\Si_{T-1}-\kappai\exp((\sigi)^2\Delta T)\right|\right]\frac{2\Phi'\left(\frac{\sigi\sqrt{\Delta T}}{2}\right)}{\sigi\sqrt{\Delta T}}\frac{1}{\ssi}\exp((\sigi)^2\Delta T)\\
            &={\left(2CP^i_{1}\left(\kappai\exp^{(\sigi)^2\Delta T}\right)\ssi-\ssi+(|x|-\ssi)\exp^{(\sigi)^2\Delta T}\right)}
            %_{\overset{\Delta T\approx 0}{\approx}
            %\left(2CP^l_{1}\left(\kappal\right)\ssl-2\ssl+|x|%\right)=2\ssl\left(CP^l_1(\kappal)+\kappal-2\right)-|x|}
            \frac{2\Phi'\left(\frac{\sigi\sqrt{\Delta T}}{2}\right)}{\sigi\sqrt{\Delta T}}\frac{1}{\ssi}\exp((\sigi)^2\Delta T).
        \end{align*}
        \end{small}
        %           \item[d)]  The indicator functions of equation \eqref{eq:varphii0} split the $d$-dimensional positive hypercube into $2^{d}$ regions, multiplied by different factors\footnote{The factor $\left(4\Phi(\sigl\sqrt{\Delta T}/2)-2+\frac{2\Phi'(\sigl\sqrt{\Delta T}/2)}{\Sl_{T-1}\sigl\sqrt{\Delta T}}\right)$ in equation \eqref{eq:varphii0} has a nice interpretation: it consists of 4 times the Delta and 2 times the Vega of of an at-the-money call option with volatility $\sigl$ and initial value $\Sl_{T-1}$.}. For $\Delta T$ small enough however, all log-normal distributions will concentrate around the assets' current values, and the relevant weight of the expectation will only be put on one of these regions. In other words, there is an $\epsilon>0$ such that for every $\gamma>0$, 
        %  there is $\Delta T$ small enough such that $\Q\left[S_{T-1}\in U^c_\epsilon(s)\right]<\gamma$. (Here, $U^c_\epsilon(s):=\{y\in\R^d_+: ||s-y||>\epsilon\}$.) Moreover, for all $S_{T-1}\in U_\epsilon(s)$,% there is some $l$ and all $i=1,\ldots,d$,
        %  the condition ${\{\Sl_{T-1}\CPlOO\text{ is max}\}}$ becomes equivalent to the condition 
        %  \begin{align*}
        %  {\ssl_{T-1}\CPlOO\text{ is max}}&\iff \ssl_{T-1}\left(\Phi\left(\frac{\sigl\sqrt{\Delta T}}{2}\right)-\Phi\left(\frac{-\sigl\sqrt{\Delta T}}{2}\right)\right)\text{ is max}\\
        %  &\iff \ssl_{T-1}\left(2\Phi\left(\frac{\sigl\sqrt{\Delta T}}{2}\right)-1\right)\text{ is max}.\end{align*}
        %  Thus, for all $S_{T-1}\in U_\epsilon(s)$, there is some $l$ s.t. for all $i=1,\ldots,d$,
        %  \begin{align*}
        %        \varphi^i_-(0)&=\Exsk\left[\left|\kappai{S^i_{T-1}}-s^i\right|\right]\underbrace{\left(4\Phi(\sigl\sqrt{\Delta T}/2)-2\right)}_{> 0}\\
        %        &+\underbrace{\Exsk\left[\left|\kappai{S^i_{T-1}}-s^i\right|\frac{2\Phi'\left(\frac{\sigl\sqrt{\Delta T}}{2}\right)}{\Sl_{T-1}\sigl\sqrt{\Delta T}}\right]}_{=:A^i}\\
        %        % &\lessapprox\Exsk\left[\left|\kappai{S^i_{T-1}}-s^i\right|\right]\left(4\Phi(\sigl\sqrt{\Delta T}/2)+2\right)\\
        %        %   &+\underbrace{\Exsk\left[\left|\kappai{S^i_{T-1}}-s^i\right|\right]\left(\frac{2\Phi'\left(\frac{\sigl\sqrt{\Delta T}}{2}\right)}{(\ssl_{T-1}\pm\epsilon/2)\sigl\sqrt{\Delta T}}\right)}_{\approx 0}.\\
        %        %  % \varphi^i_-(0)&>\Exsk\left[\left|\kappai{S^i_{T-1}}-s^i\right|\right]\underbrace{\left(4\Phi(\sigl\sqrt{\Delta T}/2)+2+\frac{2\Phi'(d_1^l)}{(\ssl_{T-1}+\epsilon/2)\sigl\sqrt{\Delta T}}\right)}_{=:F^-}-\gamma.
        %  \end{align*}
        % %For small $\Delta T$, $A^i$ becomes the dominant summand.
        % We further investigate term $A^i$. For $i\neq l$, due to independence of the assets, we get
        % \begin{align*}
        %     A^i&=\Exsk\left[\left|\kappai{S^i_{T-1}}-s^i\right|\right]\frac{2\Phi'\left(\frac{\sigl\sqrt{\Delta T}}{2}\right)}{\sigl\sqrt{\Delta T}}\frac{1}{\ssl}\exp((\sigl)^2\Delta T)\\
        %     &=\left(2CP^i_{\kappai}(1)\ssi-|x|\right)\frac{2\Phi'\left(\frac{\sigl\sqrt{\Delta T}}{2}\right)}{\sigl\sqrt{\Delta T}}\frac{1}{\ssl}\exp((\sigl)^2\Delta T).
        % \end{align*}
        % Furthermore,
        %         \begin{align*}
        %     A^l&=\Exsk\left[\left|\kappal{\Sl_{T-1}}-\ssl\right|\frac{1}{\Sl_{T-1}}\right]\frac{2\Phi'\left(\frac{\sigl\sqrt{\Delta T}}{2}\right)}{\sigl\sqrt{\Delta T}}\\
        %     &=\Exsk\left[\left|\frac{\ssl}{\Sl_{T-1}}-\kappal\right|\right]\frac{2\Phi'\left(\frac{\sigl\sqrt{\Delta T}}{2}\right)}{\sigl\sqrt{\Delta T}}\\
        %     &=\Exsk\left[\left|\Sl_{T-1}-\kappal\exp((\sigl)^2\Delta T)\right|\right]\frac{2\Phi'\left(\frac{\sigl\sqrt{\Delta T}}{2}\right)}{\sigl\sqrt{\Delta T}}\frac{1}{\ssl}\exp((\sigl)^2\Delta T)\\
        %     &={\left(2CP^l_{1}\left(\kappal\exp^{(\sigl)^2\Delta T}\right)\ssl-\ssl+(|x|-\ssl)\exp^{(\sigl)^2\Delta T}\right)}
        %     %_{\overset{\Delta T\approx 0}{\approx}
        %     %\left(2CP^l_{1}\left(\kappal\right)\ssl-2\ssl+|x|%\right)=2\ssl\left(CP^l_1(\kappal)+\kappal-2\right)-|x|}
        %     \frac{2\Phi'\left(\frac{\sigl\sqrt{\Delta T}}{2}\right)}{\sigl\sqrt{\Delta T}}\frac{1}{\ssl}\exp((\sigl)^2\Delta T).
        % \end{align*}
         \end{itemize}
Note furthermore that for $z$ large enough, 
 \begin{align*}
             \varphi^i_-(z)=&\Exsk\left[\left(\left(\kappai{S^i_{T-1}}-(s^i-z)\right)_++0-\Si_{T-1}\kappai+\ssi\right)\,\sum_{l=1}^d \frac{2\Phi'(d_1^l)}{(z+\Sl_{T-1})\sigl\sqrt{\Delta T}}M_l(z,\Sl_{T-1})\right]\\
             &+2\Exsk\left[\left|\kappai{S^i_{T-1}}-s^i\right|\sum_{l=1}^d\left(2\Phi(\sigl\sqrt{\Delta T}/2)-1\right)\1_{\{\Sl_{T-1}\CPlOO\text{ is max}\}}\right]\\
            &=\Exsk\underbrace{\left[z\,\sum_{l=1}^d \frac{2\overbrace{\Phi'(d_1^l)}^{\overset{z\to\infty} {\to}0}}{(z+\Sl_{T-1})\sigl\sqrt{\Delta T}}\overbrace{M_l(z,\Sl_{T-1})}^{\le 1}\right]}_{\overset{z\to\infty}{\to}0}\\
             &+2\Exsk\left[\left|\kappai{S^i_{T-1}}-s^i\right|\sum_{l=1}^d\left(2\Phi(\sigl\sqrt{\Delta T}/2)-1\right)\1_{\{\Sl_{T-1}\CPlOO\text{ is max}\}}\right],
         \end{align*}
%          and with small $\Delta T$,
% \begin{align*}
%     \lim_{z\to \infty}f(z)&= \left(-2|x|+ 4\Esicxsk[j]\left[\left(S^j_{T-1}\kappa^j-s^j\right)_+\right]
%     +2|x|-4\Esicxsk[i]\left[\left(S^i_{T-1}{\kappa^i}-s^i\right)_+\right]\right)\\
%     &\cdot\left(2\Phi\left(\frac{\sigl\sqrt{\Delta T}}{2}\right)-1\right)\\
%         &=  \left(CP^j_{\kappa^j}(1)s^j-CP^i_{\kappai}(1)\ssi\right)4\underbrace{\left(2\Phi\left(\frac{\sigl\sqrt{\Delta T}}{2}\right)-1\right)}_{>0}.
% %   \varphi^i_-(z)&=\Esicxsk\left[\left(S^i-(s^i+|x|+z)\right)_+\right]+\Esicxsk\left[S^i-(s^i+|x|-z)\right]\\
% %     &=\Esicxsk\left[\left(S^i-(s^i+|x|+z)\right)_+\right]+|x|+z, \quad \forall i,
% \end{align*}
and thus
\begin{align*}
    \lim_{z\to \infty}\varphi^i_-(z)&< \Exsk\left[\left|\kappai{S^i_{T-1}}-s^i\right|\right]
    \cdot\underbrace{\max_{l=1,\ldots d}2\left(2\Phi\left(\frac{\sigl\sqrt{\Delta T}}{2}\right)-1\right)}_{=:F_+},\\
        \lim_{z\to \infty}\varphi^i_-(z)&> \Exsk\left[\left|\kappai{S^i_{T-1}}-s^i\right|\right]
    \cdot\underbrace{\min_{l=1,\ldots d}2\left(2\Phi\left(\frac{\sigl\sqrt{\Delta T}}{2}\right)-1\right)}_{=:F_-}.
%   \varphi^i_-(z)&=\Esicxsk\left[\left(S^i-(s^i+|x|+z)\right)_+\right]+\Esicxsk\left[S^i-(s^i+|x|-z)\right]\\
%     &=\Esicxsk\left[\left(S^i-(s^i+|x|+z)\right)_+\right]+|x|+z, \quad \forall i,
\end{align*}
With this we get the necessary condition
\begin{align}\label{eq:Nej}
     \lim_{z\to \infty}\varphi^j_-(z)>  \lim_{z\to \infty}\varphi^i_-(z) &\Rightarrow \Exsk\left[\left|\kappa^j{S^j_{T-1}}-s^j\right|\right]>\frac{F_-}{F_+}\Exsk\left[\left|\kappai{S^i_{T-1}}-s^i\right|\right], \tag{Ne\textsuperscript{j}}
\end{align}
and the sufficient condition
\begin{align}\label{eq:Suj}
     \lim_{z\to \infty}\varphi^j_-(z)>  \lim_{z\to \infty}\varphi^i_-(z)&\Leftarrow \Exsk\left[\left|\kappa^j{S^j_{T-1}}-s^j\right|\right]>\frac{F_+}{F_-}\Exsk\left[\left|\kappai{S^i_{T-1}}-s^i\right|\right] , \tag{Su\textsuperscript{j}}
\end{align}
for $ \lim_{z\to \infty}\varphi^j_-(z)-\varphi^i_-(z)>0$.
Since the negation of conditions \eqref{eq:Suj} and \eqref{eq:Nej} correspond to the necessary condition (Ne\textsuperscript{i}) respectively the sufficient condition (Su\textsuperscript{i}) for $\lim_{z\to \infty}\varphi^j_-(z)<  \lim_{z\to \infty}\varphi^i_-(z) $, we get \eqref{eq:Suj} $\iff$\eqref{eq:Nej}. Moreover,
\begin{align*}
\eqref{eq:Suj} \implies \Exsk\left[\left|\kappa^j{S^j_{T-1}}-s^j\right|\right]>\Exsk\left[\left|\kappai{S^i_{T-1}}-s^i\right|\right]\implies \eqref{eq:Nej}.
\end{align*}
Thus (with reformulations from \Cref{re:varphiRepresentations} for $z=0$) we get that $ \lim_{z\to \infty}\varphi^j_-(z)-\varphi^i_-(z)>0$ if and only if
\begin{align*}
     \Exsk\left[\left|\kappa^j{S^j_{T-1}}-s^j\right|\right]-\Exsk\left[\left|\kappai{S^i_{T-1}}-s^i\right|\right]&>0\\
     \iff  \Exsk\left[\left(\kappa^j{S^j_{T-1}}-s^j\right)_+\right]-\Exsk\left[\left(\kappai{S^i_{T-1}}-s^i\right)_+\right]&>0.
\end{align*}
Therefore, \eqref{eq:CPcondition} is equivalent to $\lim_{z\to\infty}f(z)>0$. By items b) and c), $f(0)=\varphi_-^j(0)-\varphi_-^i(0)=C<\infty$ is equal to some finite constant $C\in\R$, and by continuity of $f$ we get that $\int_{\Rp}f(z)\,dz>0$.
% and thus for such a $z>\max_i(s^i+|x|)$
% \begin{align*}
%     f(z)&=\Esicxsk\left[\left(S^i-(s^i+|x|+z)\right)_+\right]\\
%     &\hspace{5mm}-\E_{S^{j}\mid x,s,k}\left[\left(S^j_{T-1}-(s^j+|x|+z)\right)_+\right].
% \end{align*}

\end{proof}


\begin{lemma}\label{le:CPconditionsEquivalences}
    Let $i,j\in\{1,\ldots,d\}$, and $\kappa$ as in \Cref{re:varphiRepresentations}.%, and $|x|,s^i\neq s^j, \sigi>\sigma^j$ as in the proof of \Cref{le:whichAssetIsMax}. 
    The following are equivalent
    \begin{enumerate}
    \item \begin{align*}
CP^j(s^j/\kappa^j)\kappa^j>CP^i(s^i/\kappa^i)
\kappa^i, 
\end{align*}
    \item \begin{align*}
s^jCP^j_{\kappa^j}(1)>\ssi CP^i_{\kappai}(1)
, 
\end{align*}
%     \item \begin{align*}
% s^jCP^j_1(1)>\ssi CP^i_1(1)
% , 
% \end{align*}
\end{enumerate}
 where $ CP^i_{s}(k)$ denotes the call price of an asset with starting value $s$, volatility $\sigi$, and strike price $k$ for maturity $\Delta T$.
\end{lemma}
\begin{proof}
    To see the equivalence, note that \begin{align*}
    \kappa^i CP^i(s^i/\kappa^i)=\kappai\E\left[\left(\Si-\frac{\ssi}{\kappai}\right)_+\right]=\ssi\E\left[\left(\Si\frac{\kappai}{\ssi}-1\right)_+\right]=\ssi\E\left[\left(\tilde{\Si}-1\right)_+\right]=\CPikapO\ssi,
\end{align*}
where the penultimate equality follows from a change of measure.
% For the second equivalence, we define for all $i$ $\phi^i(x):=\ssi CP^i_{\kappai}(1)=\ssi CP^i_{\frac{|x|+\ssi}{\ssi}}(1)$. Then, 
% \begin{align*}
% s^jCP^j_{\kappa^j}(1)>\ssi CP^i_{\kappai}(1)
% \iff f(x):=\phi^j-\phi^i (x)> 0 \quad \forall x\in\R.
% \end{align*}
% Since $f$ is continuous
\end{proof}


% \begin{lemma}\label{le:phiepsilonCondition}
% Let $i,j\in\{1,\ldots,d\}$, $F^+$ and $\epsilon$ as in \eqref{eq:defF+}, and $\varphi_-$ as in \Cref{def:pmInvestment}. Furthermore, let $F$ as in \eqref{eq:defF}, $\kappa$ as in \Cref{re:varphiRepresentations}, and $|x|,s^i\neq s^j, \sigi>\sigma^j$ as in the proof of \Cref{le:whichAssetIsMax}. If \begin{align*}
% CP^j(s^j/\kappa^j)\kappa^j>CP^i(s^i/\kappa^i)
% \kappa^i,   % i^*&=\argmax_i s^i\Phi(d_1^i)-\left(|x|+s^i\right)\Phi(d_1^i-\sigi\sqrt{\Delta t_{k+1}})\\
%     % d_1^i&=
%     % \frac{1}{\sigma^i\sqrt{\Delta t_{k+1}}}\ln\left(\frac{s^i}{s^i+|x|}\right)+\frac{1}{2}\sigma^i\sqrt{\Delta t_{k+1}}.
% \end{align*}
% then it holds that\begin{align*}
%     %  \int_{\R_+}DC^{i^*}(z)\,d\mu(z)=\max_{i=1\ldots,d} \int_{\R_+}DC^i(z)\,d\mu(z)
%      \varphi^{+\epsilon}(z)\ge 0, \forall z\ge 0,
% \end{align*}
% where
% \begin{align*}
%    \varphi^{+\epsilon}(z)&:=\varphi^j_-(z)-\varphi^i_-(z) F^+, z\ge 0\\
% \end{align*}
% %if for $F$ as in \eqref{eq:defF}, $\kappa$ as in \Cref{re:varphiRepresentations}, and $|x|,s^i\neq s^j, \sigi>\sigma^j$ as in the proof of \Cref{le:whichAssetIsMax}
% and where $CP^i(K)$ denotes the call price with maturity $\Delta t_n$ on asset $S^i$ with strike $K$.
% \end{lemma}
% \begin{proof} Let $S^i$ respectively $ S^j$ as in \Cref{re:varphiRepresentations} and
% \begin{equation*}
%     f(z):=  \varphi^j_-(z)-\varphi^i_-(z).
% \end{equation*}
% Since $F^+<1$, $f(z)\ge0$ for all $z\ge 0$ implies $ \varphi^{+\epsilon}(z)\ge 0$ for all $z\ge 0$.
% % Note that for $z$ large enough, 
% % \begin{align*}
% %     f(z)&= |x|+ \Esicxsk[j]\left[\left(S^j\kappa^j-(s^j-z)\right)_+\right]-|x|-\Esicxsk[i]\left[\left(S^i{\kappa^i}-(s^i-z)\right)_+\right]\\
% %         &=  \Esicxsk[j]\left[S^j\kappa^j-(s^j-z)\right]-\Esicxsk[i]\left[S^i{\kappa^i}-(s^i-z)\right]\\
% %          &= |x|+z-|x|-z=0.\\
% % %   \varphi^i_-(z)&=\Esicxsk\left[\left(S^i-(s^i+|x|+z)\right)_+\right]+\Esicxsk\left[S^i-(s^i+|x|-z)\right]\\
% % %     &=\Esicxsk\left[\left(S^i-(s^i+|x|+z)\right)_+\right]+|x|+z, \quad \forall i,
% % \end{align*}
% % % and thus for such a $z>\max_i(s^i+|x|)$
% % % \begin{align*}
% % %     f(z)&=\Esicxsk\left[\left(S^i-(s^i+|x|+z)\right)_+\right]\\
% % %     &\hspace{5mm}-\E_{S^{j}\mid x,s,k}\left[\left(S^j-(s^j+|x|+z)\right)_+\right].
% % % \end{align*}
% % Moreover, by the Black Scholes pricing formula for every $i=1\ldots, d$ and $z\ge0$ we have
% % \begin{align}
% %     \Esicxsk\left[\left(S^i\kappa^i-(s^i+z)\right)_+\right]&=\kappa^i\left(s^i \Phi(d_1^i)-\frac{s^i+z}{\kappa^i}\Phi(d_2^i)\right), \\
% %     &=(s^i+|x|) \Phi(d_1^i)-(s^i+z)\Phi(d_2^i), \\
% %     d_1^i&=\frac{\log((s^i+|x|)/(s^i+z))+\frac{1}{2}(\sigma^i)^2\Delta t_{k+1}}{\sigi\sqrt{\Delta t_{k+1}}},\notag\\
% %     d_2^i &=d_1^i-\sigi\sqrt{\Delta t_{k+1}}.\notag\\
% % \end{align}
% % Furthermore, for $z=0$, 
% % \begin{align*}
% %      \varphi_-^i(z)&=|x|+2\kappa^i\Esicxsk\left[\left(S^i-\frac{s^i}{\kappa^i}\right)_+\right].
% % \end{align*}
% % %is twice the call price of asset $S^i$ with strike $s^i+|x|$.

% % Thus, the following hold.
% % \begin{enumerate}
% % \item$\lim_{z\to\infty}f(z)=0.$
% %     %  \begin{align*}
% %     %     \lim_{z\to\infty}f(z)&=\lim_{z\to\infty}  (2|x|+z)\left(1-F^+\right)=\bigg\{\begin{matrix}+\infty,& F^+<1\\
% %     %     -\infty,& F^+\ge1\end{matrix}.
% %     % \end{align*}
% %     %If $s^i\neq s^j$, then $F^+<1$ for small $\Delta t_n$ and hence $ \lim_{z\to\infty}f(z)>0.$
% % \item
% % Since $f$ is continuous, 3. implies that there exists some $z_0$ such that $f(z_0)\ge 0$.
% % \item \begin{align*}
% %     f(0)&=|x|+2\kappa^j\left(\Esicxsk[j]\left[\left(S^j-\frac{s^j}{\kappa^j}\right)_+\right]\right)-|x|-2\kappa^i\left(\Esicxsk\left[\left(S^i-\frac{s^i}{\kappa^i}\right)_+\right]\right)\\
% %     &=2\left((s^j+|x|) \Phi(d_1^j)-s^j\Phi(d_2^j)-(s^i+|x|) \Phi(d_1^i)+s^i\Phi(d_2^i)\right)\big|_{z=0}
% % \end{align*}
% % % By monotonicity of the cdf $\Phi$, $f(0)> 0$ if and only if

% % We have $f(0)> 0$ if and only if
% % \begin{align*}
% % %s^i\Phi(d_1^i)-\left(|x|+s^i\right)\Phi(d_1^i-\sigi\sqrt{\Delta t_{k+1}})>s^j\Phi(d_1^j)-\left(|x|+s^j\right)\Phi(d_1^j-\sigma^j\sqrt{\Delta t_{k+1}}).
% % &\left((s^j+|x|) \Phi(d_1^j)-s^j\Phi(d_2^j)-(s^i+|x|) \Phi(d_1^i)+s^i\Phi(d_2^i)\right)\big|_{z=0}>0.
% % \end{align*}
% % \item For any $z>0$
% %     \begin{align*}
% %         \frac{\partial}{\partial z}f(z)
% %      &=\P\left[S^j\kappa^j>s^j+z\right]+\P\left[S^j\kappa^j>s^j-z\right]-\left(\P\left[S^i{\kappa^i}>s^i+z\right]+\P\left[S^i{\kappa^i}>s^i-z\right]\right)\\
% %      &=-\P\left[s^j-z<S^j\kappa^j\le s^j+z\right]+\P\left[s^i-z< S^i{\kappa^i}\le s^i+z\right]\\
% %      %&\bigg\{\begin{matrix}>(1-F^+)-1+F^+=0,& F^+<1\\
% %      %<(1-F^+)-1+F^+=0,& F^+\ge1\end{matrix}
% %      %P\left[S^j<s^j+|x|+z\right]+\P\left[S^j<s^j+|x|-z\right]\\
% %      %&-P\left[S^i<s^i+|x|+z\right]-\P\left[S^i<s^i+|x|-z\right]\\
% %           %&=\left(P\left[S^j-s^j<|x|+z\right]-P\left[S^i-s^i<|x|+z\right]\right)\\
% %           %&+\left(\P\left[S^j-s^j<|x|-z\right]-\P\left[S^i-s^i<|x|-z\right]\right),\\
% %     \end{align*}
% % %i.e., we are summing differences of two centered $\logN$ cdfs. Since the lognormal is not symmetric around its mean, $\frac{\partial}{\partial z}f(z)$ can only be zero if $|x|=z=0$.
% % Therefore, $\frac{\partial}{\partial z}f(z)$ can only be zero if and only if $$\P\left[s^j-z<S^j\kappa^j\le s^j+z\right]=\P\left[s^i-z< S^i{\kappa^i}\le s^i+z\right].$$
% % Thus, unless $s^i=s^j$, and $\sigi=\sigma^j$, for any $z>0$ $\frac{\partial}{\partial z}f(z)\neq 0$. 
% % \end{enumerate}

% % By \cite[Lemma 6.5.]{delbaen2002passport}, it follows from 1.-4. that $f(z)\ge0$ for all $z\ge0$. 

% Thus, by \Cref{le:varphi_CP_condition}, if $i, j$ are such that
% \begin{align*}
% CP^j(s^j/\kappa^j)\kappa^j>CP^i(s^i/\kappa^i)
% \kappa^i,
% \end{align*}
% then the above implies that
% \begin{align*}
%    \varphi^{+\epsilon}\ge0,  \forall z\ge 0.
% \end{align*}
% \end{proof}


\begin{lemma}\label{le:whichAssetIsMax}
Let $(x,s)$ and $k$ be fixed. Then investing in the $j$\textsuperscript{th} asset is preferred over investing in the $i$\textsuperscript{th} asset, i.e.,
\begin{align*}
     V^{j}_k(x,s)>V^{i}_k(x,s),
\end{align*}
if
\begin{align}\label{eq:CPcondition}
    CP^j(s^j/\kappa^j)\kappa^j>CP^i(s^i/\kappa^i)\kappa^i,
\end{align}
where for each $i$, $\kappa^i=\frac{|x|+s^i}{s^i}$ and
\begin{align*}
CP^j(s^i/\kappa^i)\kappa^i&=(s^i+|x|) \Phi(d_1^i)-s^i\Phi(d_2^i), \\
    d_1^i&=\frac{\log(1+|x|/s^i)+\frac{1}{2}(\sigma^i)^2\Delta t_{k+1}}{\sigi\sqrt{\Delta t_{k+1}}},\\
    d_2^i &=d_1^i-\sigi\sqrt{\Delta t_{k+1}}.\\
    % CP^j(|x|+s^j)&=s^j\Phi(d_1^j)-\left(|x|+s^j\right)\Phi(d_1^j-\sigma^j\sqrt{\Delta t_{n}})\notag\\
    % d_1^j&=
    % \frac{1}{\sigma^j\sqrt{\Delta t_{n}}}\ln\left(\frac{s^j}{s^j+|x|}\right)+\frac{1}{2}\sigma^j\sqrt{\Delta t_n}.
\end{align*}
\end{lemma}
\begin{proof}
    We proceed backwards in time. 
    \begin{itemize}
    \item[$\mathbf{ V_T}$] First, for $x\in\Rp, s\in\R^d_+$, $V_T(x,s)=|x|=\int_{\Rp}\max(|x|,z) \delta_0(dz)$, and thus the measure $\mu$ from \Cref{le:Vrepresentation} representing $V_T$ is equal to a Dirac delta at $0$. In particular, it is independent of $s$.
        \item[$\mathbf{ q_T^*}$]  We thus get that 
    \begin{align*}
        V_{T-1}(x,s)&= \max_{i=1\ldots,d}\,V_{T-1}^i(x,s)\\
        &= \max_{i=1\ldots,d}\,\Esimxsk\left[\int_{\R_+}\varphi^i_-(z)\,\mu(dz)\right]\\
        &=\max_{i=1\ldots,d}\,\int_{\R_+}\varphi^i_-(z)\,\mu(dz),
    \end{align*}
    where the last equality follows since $\varphi^i$ are independent of $S^{i-}$. By \Cref{le:varphi_CP_condition},
    \begin{align*}
      \varphi^j_-(z)-\varphi^i_-(z)\ge 0, \forall z\ge 0,
\end{align*}
if and only if 
\begin{align*}
CP^j(s^j/\kappa^j)\kappa^j>CP^i(s^i/\kappa^i)
\kappa^i.   
\end{align*}
Therefore, condition \eqref{eq:CPcondition} is sufficient for $ V^{j}_k(x,s)>V^{i}_k(x,s)$.\footnote{
In fact also conversely,
\begin{align*}
    V^{j}_k(x,s)>V^{i}_k(x,s) \iff \int_{\R_+}\varphi^j_-(z)\,\mu(dz)>\int_{\R_+}\varphi^i_-(z)\,\mu(dz),
\end{align*}
%which implies that 
%$\varphi^j_-(z)-\varphi^i_-(z)\ge 0$ for $\mu$-a.e. $z\ge 0$. 
Since $\mu=\delta_0$, it further implies that $\varphi^j_-(0)-\varphi^i_-(0)\ge 0$, which by the proof of \Cref{le:varphi_CP_condition} is equivalent to equation \eqref{eq:CPcondition}.} Thus, $q_{t_N}^*(x,s)$ is given by \eqref{eq:optimalStrat}.

\item[$\mathbf{ V_{T-1}}$] We first define the indicator \begin{align*}
M_i(x,s):=\left\{\begin{matrix}
    1,& \kappa^i CP^i(s^i/\kappa^i) \text{ is max, }\\
    0,& \text{ else.}
\end{matrix}\right.
\end{align*}
Note that by \Cref{le:CPconditionsEquivalences}, equivalently
%\begin{align*}
%     \kappa^i CP^i(s^i/\kappa^i)=\kappai\E\left[\left(\Si-\frac{\ssi}{\kappai}\right)_+\right]=\ssi\E\left[\left(\Si\frac{\kappai}{\ssi}-1\right)_+\right]=\ssi\E\left[\left(\tilde{\Si}-1\right)_+\right]=\CPikapO\ssi,
% \end{align*}
%  where the penultimate equality follows from a change of measure and $ CP^i_{s}(k)$ denotes the call price of an asset with starting value $s$, volatility $\sigi$, and strike price $k$ for maturity $\Delta T$. Thus, equivalently,
 \begin{align*}
M_i(x,s):=\left\{\begin{matrix}
    1,& \CPikapO\ssi \text{ is max, }\\
    0,& \text{ else.}
\end{matrix}\right.
\end{align*}
 By the previous step, inserting $q_T^*$ we get that
\begin{align*}
        V_{T-1}(x,s)&= \sum_{i=1}^d\E_{x,s,T-1}\left[\left|x\left(\frac{S^i_{T}}{s^i}\right)-\sign(x)s^i\left(1-\left(\frac{S^i_{T}}{s^i}\right)\right)\right|\right]M_i(x,s)\\
        &= \sum_{i=1}^d\E_{x,s,T-1}\left[\left|\kappa^i{S^i_{T}}-s^i\right|\right]M_i(x,s)\\
         &= \sum_{i=1}^d\si\E_{x,\kappa^i,T-1}\left[\left|\tilde{S^i_{T}}-1\right|\right]M_i(x,s)\\
          &= \sum_{i=1}^d2\si \left(CP^i_{\kappa^i}(1)-|x|\right)M_i(x,s),
    \end{align*}
    where the penultimate equality again follows from a change of measure.
    Thus we obtain the first derivative w.r.t. $x$
    \begin{align*}
        \partial_xV_{T-1}(x,s)&=\sum_{i=1}^d(2\Phi(d_1^i)-1)\sign(x)M_i(x,s),\\
        \end{align*}
        where $\Phi$ is the standard Gaussian cdf, and $d_1^i$ as in the formulation of \Cref{thm:optimalStrat}.
    Moreover, the second (distributional) derivative is given as
     \begin{align*}
        \partial^2_xV_{T-1}(x,s)&=\sum_{i=1}^d\left(\frac{2\Phi'(d_1^i)}{(|x|+\ssi)\sigi\sqrt{\Delta T}}\sign(x)+2\delta_0(x)(2\Phi(d_1^i)-1)\right)M_i(x,s)=:\mu(x).\\
        \end{align*}
        Therefore, the measure $\mu$ from \Cref{le:Vrepresentation} representing $V_{T-1}$ depends on the values of $s$.
        With this we get the representation
        \begin{align*}
            V_{T-1}(x,s) &=\int_{\Rp}\max\{|x|,z\}\,\mu(dz)\\
            &=\sum_{i=1}^d \bigg(\int_{\Rp}\max\{|x|,z\}\frac{2\Phi'(d_1^i)}{(|x|+\ssi)\sigi\sqrt{\Delta T}}\Mi\,dz\\
            &+2|x|\left(2\Phi(\sigi\sqrt{\Delta T}/2)-1\right)\Mi\bigg)
        \end{align*}

         \item[$\mathbf{ q_{T-1}^*}$] We now want to find out which asset we should invest in at time point $k=T-2$, i.e., we want to find $q_{T-1}^{*}$ that solves
         \begin{align*}
             \max_{i=1,\ldots,d}\Exsk\left[V_{T-1}\left(x\left(\frac{S^i_{T-1}}{s^i}\right)+q_{T-1}^is^i\left(1-\left(\frac{S^i_{T-1}}{s^i}\right)\right),\left(s^{i},S_{T-1}^{i-}\right)\right)\right].
         \end{align*}
         With the representation of $V_{T-1}$ from the previous step, this yields the objective
          \begin{align*}
             &\max_{i=1,\ldots,d}\Exsk\left[ \int_{\Rp}\max\{|x\left(\frac{S^i_{T-1}}{s^i}\right)+q_{T-1}^is^i\left(1-\left(\frac{S^i_{T-1}}{s^i}\right)\right)|,z\}\,\mu(dz)\right]\\
             =&\max_{i=1,\ldots,d}\Exsk\left[ \int_{\Rp}\max\{|\kappai{S^i_{T-1}}-s^i|,z\}\,\mu(dz)\right]\\
             =&\max_{i=1,\ldots,d}\Exsk\left[ \int_{\Rp}\max\{|\kappai{S^i_{T-1}}-s^i|,z\}\,\sum_{l=1}^d \frac{2\Phi'(d_1^l)}{(z+\Sl_{T-1})\sigl\sqrt{\Delta T}}M_l(z,\Sl)\,dz\right.\\
             &\left.+2\left|\kappai{S^i_{T-1}}-s^i\right|\sum_{l=1}^d\left(2\Phi(\sigl\sqrt{\Delta T}/2)-1\right)\1_{\{\Sl_{T-1}\CPlOO\text{ is max}\}}\right].
         \end{align*}
         We apply Fubini and define
         \begin{align}\label{eq:Defvarphii}
             \varphi^i_-(z):=&\Exsk\left[\max\{|\kappai{S^i_{T-1}}-s^i|,z\}\,\sum_{l=1}^d \frac{2\Phi'(d_1^l)}{(z+\Sl_{T-1})\sigl\sqrt{\Delta T}}M_l(z,\Sl)\right]\\
             &+2\Exsk\left[\left|\kappai{S^i_{T-1}}-s^i\right|\sum_{l=1}^d\left(2\Phi(\sigl\sqrt{\Delta T}/2)-1\right)\1_{\{\Sl_{T-1}\CPlOO\text{ is max}\}}\right].
         \end{align}
         With this our objective now is to show that asset $j$ solves
                  \begin{align*}
             \max_{i=1,\ldots,d}\int_{\Rp}\varphi^i_-(z)\,dz,
         \end{align*}
         if it fulfils condition \Cref{eq:CPcondition} for all $i\neq j$. 
        By \Cref{le:varphi_CP_condition_Sufficient} this holds true. %for $\Delta T$ small enough.
         \item[$\mathbf{ V_t, q_t^*}$] Analogous steps yield the result for $t=T-2,\ldots,0$.

         
        %  For further derivations, we note that
        %  \begin{itemize}
        %      \item[a)]  $\varphi^i_-(\cdot)$ are continuous, and that
        %      \item[b)] \begin{align}\label{eq:varphii0}
        %      \varphi^i_-(0)=&\Exsk\left[\left|\kappai{S^i_{T-1}}-s^i\right|\sum_{l=1}^d\left(4\Phi\left(\frac{\sigl\sqrt{\Delta T}}{2}\right)-2+\frac{2\Phi'\left(\frac{\sigl\sqrt{\Delta T}}{2}\right)}{\Sl_{T-1}\sigl\sqrt{\Delta T}}\right)\1_{\{\Sl_{T-1}\CPlOO\text{ is max}\}}\right].
        %  \end{align}
        %  \item[c)]  The indicator functions of equation \eqref{eq:varphii0} split the $d$-dimensional positive hypercube into $2^{d}$ regions, multiplied by different factors\footnote{The factor $\left(4\Phi(\sigl\sqrt{\Delta T}/2)-2+\frac{2\Phi'(\sigl\sqrt{\Delta T}/2)}{\Sl_{T-1}\sigl\sqrt{\Delta T}}\right)$ in equation \eqref{eq:varphii0} has a nice interpretation: it consists of 4 times the Delta and 2 times the Vega of of an at-the-money call option with volatility $\sigl$ and initial value $\Sl$.}. For $\Delta T$ small enough however, all log-normal distributions will concentrate around the assets' current values, and the relevant weight of the expectation will only be put on one of these regions. In other words, there is an $\epsilon>0$ such that for every $\gamma>0$, 
        %  there is $\Delta T$ small enough such that $\Q\left[S_{T-1}\in U^c_\epsilon(s)\right]<\gamma$. Moreover, for all $S_{T-1}\in U_\epsilon(s)$,% there is some $l$ and all $i=1,\ldots,d$,
        %  the condition ${\{\Sl_{T-1}\CPlOO\text{ is max}\}}$ becomes equivalent to the condition 
        %  \begin{align*}
        %  {\ssl_{T-1}\CPlOO\text{ is max}}&\iff \ssl_{T-1}\left(\Phi\left(\frac{\sigl\sqrt{\Delta T}}{2}\right)-\Phi\left(\frac{-\sigl\sqrt{\Delta T}}{2}\right)\right)\text{ is max}\\
        %  &\iff \ssl_{T-1}\left(2\Phi\left(\frac{\sigl\sqrt{\Delta T}}{2}\right)-1\right)\text{ is max}.\end{align*}
        %  Thus, for all $S_{T-1}\in U_\epsilon(s)$, there is some $l$ s.t. for all $i=1,\ldots,d$,
        %  \begin{align*}
        %        \varphi^i_-(0)&=\Exsk\left[\left|\kappai{S^i_{T-1}}-s^i\right|\right]\underbrace{\left(4\Phi(\sigl\sqrt{\Delta T}/2)-2\right)}_{\searrow\, 0}\\
        %        &+\underbrace{\Exsk\left[\left|\kappai{S^i_{T-1}}-s^i\right|\frac{2\Phi'\left(\frac{\sigl\sqrt{\Delta T}}{2}\right)}{\Sl_{T-1}\sigl\sqrt{\Delta T}}\right]}_{=:A^i}\\
        %        % &\lessapprox\Exsk\left[\left|\kappai{S^i_{T-1}}-s^i\right|\right]\left(4\Phi(\sigl\sqrt{\Delta T}/2)+2\right)\\
        %        %   &+\underbrace{\Exsk\left[\left|\kappai{S^i_{T-1}}-s^i\right|\right]\left(\frac{2\Phi'\left(\frac{\sigl\sqrt{\Delta T}}{2}\right)}{(\ssl_{T-1}\pm\epsilon/2)\sigl\sqrt{\Delta T}}\right)}_{\approx 0}.\\
        %        %  % \varphi^i_-(0)&>\Exsk\left[\left|\kappai{S^i_{T-1}}-s^i\right|\right]\underbrace{\left(4\Phi(\sigl\sqrt{\Delta T}/2)+2+\frac{2\Phi'(d_1^l)}{(\ssl_{T-1}+\epsilon/2)\sigl\sqrt{\Delta T}}\right)}_{=:F^-}-\gamma.
        %  \end{align*}
        % For small $\Delta T$, $A^i$ becomes the dominant summand.
        % We further investigate term $A^i$. For $i\neq l$, due to independence of the assets, we get
        % \begin{align*}
        %     A^i&=\Exsk\left[\left|\kappai{S^i_{T-1}}-s^i\right|\right]\frac{2\Phi'\left(\frac{\sigl\sqrt{\Delta T}}{2}\right)}{\sigl\sqrt{\Delta T}}\frac{1}{\ssl}\exp((\sigl)^2\Delta T)\\
        %     &=\left(2CP^i_{\kappai}(1)\ssi-|x|\right)\frac{2\Phi'\left(\frac{\sigl\sqrt{\Delta T}}{2}\right)}{\sigl\sqrt{\Delta T}}\frac{1}{\ssl}\exp((\sigl)^2\Delta T).
        % \end{align*}
        % Furthermore,
        %         \begin{align*}
        %     A^l&=\Exsk\left[\left|\kappal{\Sl_{T-1}}-\ssl\right|\frac{1}{\Sl_{T-1}}\right]\frac{2\Phi'\left(\frac{\sigl\sqrt{\Delta T}}{2}\right)}{\sigl\sqrt{\Delta T}}\\
        %     &=\Exsk\left[\left|\frac{\ssl}{\Sl_{T-1}}-\kappal\right|\right]\frac{2\Phi'\left(\frac{\sigl\sqrt{\Delta T}}{2}\right)}{\sigl\sqrt{\Delta T}}\\
        %     &=\Exsk\left[\left|\Sl_{T-1}-\kappal\exp((\sigl)^2\Delta T)\right|\right]\frac{2\Phi'\left(\frac{\sigl\sqrt{\Delta T}}{2}\right)}{\sigl\sqrt{\Delta T}}\frac{1}{\ssl}\exp((\sigl)^2\Delta T)\\
        %     &=\underbrace{\left(2CP^l_{1}\left(\kappal\exp^{(\sigl)^2\Delta T}\right)\ssl-\ssl+(|x|-\ssl)\exp^{(\sigl)^2\Delta T}\right)}_{\overset{\Delta T\approx 0}{\approx}
        %     \left(2CP^l_{1}\left(\kappal\right)\ssl-2\ssl+|x|\right)=2\ssl\left(CP^l_1(\kappal)+\kappal-2\right)-|x|}\frac{2\Phi'\left(\frac{\sigl\sqrt{\Delta T}}{2}\right)}{\sigl\sqrt{\Delta T}}\frac{1}{\ssl}\exp((\sigl)^2\Delta T).
        % \end{align*}
        %  \end{itemize}

        
         % Since for all $l, x$ and $s$, $CP^l_{\kappal}(1)>CP^l_1(\kappal)+\kappal-2$, $$A^l<\left(2\ssl CP^l_{\kappal}(1)-|x|\right)\frac{2\Phi'\left(\frac{\sigl\sqrt{\Delta T}}{2}\right)}{\sigl\sqrt{\Delta T}}\frac{1}{\ssl}\exp((\sigl)^2\Delta T),$$
         % and therefore $CP^i_{\kappai}(1)>CP^l_{\kappal}(1)\implies A^i>A^l$ for all $i\neq l$. Furthermore, $CP^i_{\kappai}(1)>CP^l_{\kappal}(1)\implies \varphi^i_-(0)>\varphi^l_-(0)$. Thus, for $\Delta T$ small enough, condition \eqref{eq:CPcondition} is sufficient for $\varphi^i_-(0)>\varphi^l_-(0)$.
         
         % For each fixed $z$, the indicator functions split the $d$-dimensional positive hypercube into $2^{2d}$ regions. As 

        % Thus for small $\Delta T$, \begin{align}\label{eq:varphi0equivCPconds}
        %     \varphi^i_-(0)>\varphi^j_-(0)&\iff CP^i_{\kappai}(1)>CP^j_{\kappa^j}(1), \qquad i,j\neq l, \text{ and }\\
        %     \varphi^i_-(0)>\varphi^l_-(0)&\iff CP^i_{\kappai}(1)>CP^l_{1}(\kappal)+\kappal-2.\notag
        % \end{align}
          
        %  We now show that

        %  \begin{align*}
        %     \eqref{eq:CPcondition} \implies j = \argmax_{i=1,\ldots,d}\int_{\Rp}\varphi^i_-(z)\,dz.
        %  \end{align*}
        %  \begin{itemize}
        %  \item[``$\Rightarrow$''] First, if asset $j$ is to be preferred over asset $i$, i.e.,
        %  % . By \eqref{eq:varphi0equivCPconds},  \Cref{eq:CPcondition} is equivalent to $\varphi^i_-(0)<\varphi^j_-(0)$. 
        %  %     \item[``$\Rightarrow$''] Second, we show that condition \Cref{eq:CPcondition} is necessary for $j$ to be the optimal asset. If  
        %  \begin{align*}
        %      \int_{\Rp}\varphi^i_-(z)\,dz<\int_{\Rp}\varphi^j_-(z)\,dz,
        %  \end{align*}
        %  then this implies that $\varphi^i_-(z)<\varphi^j_-(z)$ for Lebesgue-almost every $z\ge 0$. Since $ \varphi^i_-(\cdot)$ are continuous for every $i$, this implies that also $\varphi^i_-(0)<\varphi^j_-(0)$. With equations \eqref{eq:varphi0equivCPconds}, condition\eqref{eq:CPcondition} follows.
        %  \item[``$\Leftarrow$''] If on the other hand asset $i$ is to be preferred over asset $j$, analogous arguments give that 
        %  \begin{align*}
        %      & CP^i_{\kappai}(1)>CP^j_{\kappa^j}(1), \qquad \text{ if } i,j\neq l,\\
        %     & CP^i_{\kappai}(1)>CP^l_{1}(\kappal)+\kappal-2 \qquad\text{ if } j= l, \text{ and }\\
        %      & CP^i_{\kappa^i}(1)+\kappai-2>CP^j_{\kappa^j}(1), \qquad \text{ if } i=l.
        %  \end{align*}
        %  Thus, equation \eqref{eq:CPcondition} (which is the negation of above equations), implies the converse, namely that $\varphi^i_-(0)<\varphi^j_-(0)$
        %  \end{itemize}
    \end{itemize}
   

\end{proof}
% \begin{proof}
% \begin{itemize}
% \item Consider two assets with indices $i$ and $j$ and let w.l.o.g. $\sigi>\sigma^{j}$ (the case $\sigma^j>\sigi$ follows analogously).
%     \item  We first reformulate
% \begin{align*}
%     V^{i}_k(x,s)&=\Esimxsk\left[\int_{\R_+}\varphi^i_-(z)\,d\muim(z)\right]\\
%     &=\int \int_{\Rp}\varphi^i_-(z)\,d\muim(z)f_j(y)f_{ij-}(s_{ij-})\,dyds_{ij-}\\
%     &=\int \int_{\Rp}\varphi^i_-(z)\frac{f_j(y)}{f_i(y)}\,d\mu^{j-}(z)f_i(y)f_{ij-}(s_{ij-})\,dyds_{ij-}\\
% \end{align*}
% where $f_{ij-}(\cdot)=\prod_{l=1, l\neq i\neq j}^df_{l}(\cdot)$. 
% For $\Delta t_n$ small enough, all log-normal distributions will concentrate around the assets' current values.

% Thus for every $\epsilon >0$ there is $\delta_\epsilon>0$ s.t.
% \begin{align*}
%     \left|\frac{f_j(y)}{f_i(y)}-\frac{f_j(s^i)}{f_i(s^i)}\right|<\epsilon
% \end{align*}
% for $y\in U_{\delta_\epsilon}(s^i)$, and for every $\gamma>0$ there is $\Delta t_n$ small enough s.t.
% \begin{align*}
%    \int_{U_{\delta_\epsilon}^c(s^i)}f_i(y)\,dy<\gamma, \forall i\in\{1,\ldots,d\}.
% \end{align*}

% Moreover, 
% \begin{align}
%     \frac{f_j(s^i)}{f_i(s^i)}
%     &=\frac{\sigi}{\sigma^j}\exp\left(-\frac{(\ln(s^i/s^j)-(\sigma^j)^2/2\Delta t_n)^2}{2(\sigma^j)^2\Delta t_n}+\frac{((\sigi)^2/2\Delta t_n)^2}{2(\sigi)^2\Delta t_n}\right)\notag\\
%     &=\sqrt{\frac{s^i}{s^j}}\frac{\sigi}{\sigma^j}\exp\left(-\frac{\ln(s^i/s^j)^2}{2(\sigma^j)^2\Delta t_n}+\frac{\Delta t_n}{2}((\sigi)^2-(\sigma^j)^2)\right)\notag\\
%     &= \sqrt{\frac{s^i}{s^j}}\frac{\sigi}{\sigma^j} \tilde{\epsilon}(\Delta t_n)=:F(\Delta t_n),\label{eq:defF}
% \end{align}
% where $\tilde{\epsilon}(\Delta t_n)$ goes to zero with  $\Delta t_n$ (if $s^i\neq s^j$). 
% This yields
% \begin{align}\label{eq:defF+}
%     \frac{f_j(y)}{f_i(y)}&< F(\Delta t_n)+\epsilon=:F^+,\\
%     \frac{f_j(y)}{f_i(y)}&>F(\Delta t_n)-\epsilon=:F^-\label{eq:defF-},
% \end{align}
% for $y\in U_{\delta_\epsilon}(s^i)$.
% %(with $\tilde\tilde{\epsilon}(\Delta t_n)$ small enough since  $\frac{f_j(y)}{f_i(y)}>0$)
% \item Note that in case $s^i\neq s^j$, $F^+<1$ and $F^-<1$ for small $\Delta t_n$. Moreover, if $s^i\neq s^j$, then
% $F(\Delta t_n)=\frac{\sigi}{\sigma^j}\exp\left(\frac{\Delta t_n}{2}((\sigi)^2-(\sigma^j)^2)\right)>1$, and thus for $\epsilon\approx 0$, we have $F^+>1$ and $F^->1$ for small $\Delta t_n$.

% \item With the above, we get
% \begin{align*}
%     V^{i}_k(x,s)&<\int \int_{\Rp}\varphi^i_-(z)F^+\,d\mujm(z)f_i(y)f_{ij-}(s_{ij-})\,dyds_{ij-}\\
%     &=F^+\int\int_{\Rp}\varphi^i_-(z)\,d\mujm(z)f_i(y)f_{ij-}(s_{ij-})\,dyds_{ij-}+\gamma C.\\
% \end{align*}
% For $\Delta t_n$ small enough the last term is negligible.
% We disintegrate the former and rewrite the above as
% \begin{align*}
% V^{i}_k(x,s)&<F^+\int_{\R_+}\int \varphi^i_-(z)\,d(y,s_{ij-})|z\,d\mu(z),\\
% &=F^+\int_{\R_+}\varphi^i_-(z)\underbrace{\int \,d(y,s_{ij-})|z}_{=1}\,d\mu(z),
% \end{align*}
% for probability measures $(y,s_{ij-})|z$ and $\mu$.
% Analogously,
% \begin{align*}
%     V^{i}_k(x,s)&>
% F^-\int_{\R_+}\varphi^i_-(z)\,d\mu(z),
% \end{align*}
% \item Second, we rewrite (again disintegrating)
% \begin{align*}
%     V^{j}_k(x,s)
%     &=\int \int_{\Rp}\varphi^j_-(z)\,d\mu^{j-}(z)f_i(y)f_{ij-}(s_{ij-})\,dyds_{ij-}\\
%     &=\int_{\R_+}\int\varphi^j_-(z) \,d(y,s_{ij-})|z\,d\mu(z)\\
%     &=\int_{\R_+}\varphi^j_-(z)\,d\mu(z).
% \end{align*}
% \item 
% Let for every $z\ge 0$
% \begin{align*}
%    \varphi^{+\epsilon}(z)&:=\varphi^j_-(z)-\varphi^i_-(z) F^+,\\
%       \varphi^{-\epsilon}(z)&:=\varphi^i_-(z) F^--\varphi^j_-(z).\\
% \end{align*}
% Then
% \begin{align}\label{eq:sufficient}
%    \varphi^{+\epsilon}(z)\ge0\,\forall z\ge 0&\implies V^{j}_k(x,s)> V^{i}_k(x,s),\\
%        V^{j}_k(x,s)> V^{i}_k(x,s)&\implies\varphi^{-\epsilon}(z)\le0 \,\text{ for } \mu-\text{a.e. } z\ge 0.\label{eq:necessary}
% \end{align}

% \item In case $s^i\neq s^j$ we get the following sufficient and necessary conditions for the $j$\textsuperscript{th}
% asset to be preferred over the $i$\textsuperscript{th} asset:
% \begin{itemize}
%     \item By \Cref{le:phiepsilonCondition}, $\varphi^{+\epsilon}(z)>0$ for all $z>0$, if 

% % \begin{align*}
% %     %s^j\Phi(d_1^j)-\left(|x|+s^j\right)\Phi(d_1^j-\sigma^j\sqrt{\Delta t_{n}})>s^i\Phi(d_1^i)-\left(|x|+s^i\right)\Phi(d_1^i-\sigma^i\sqrt{\Delta t_{n}}).
% %     CP^j(s^j/\kappa^j)\kappa^j+|x|\frac{1-F}{2}-CP^i(s^i/\kappa^i)
% % \kappa^i>\epsilon \frac{|x|}{2}.
% % \end{align*}
% % For $\epsilon\to0$ and thus $\Delta t_n\to 0$, we then obtain the sufficient condition 
% \begin{align}\label{eq:OptimalityCondition}
%     %s^j\Phi(d_1^j)-\left(|x|+s^j\right)\Phi(d_1^j-\sigma^j\sqrt{\Delta t_{n}})>s^i\Phi(d_1^i)-\left(|x|+s^i\right)\Phi(d_1^i-\sigma^i\sqrt{\Delta t_{n}}).
%     CP^j(s^j/\kappa^j)\kappa^j>CP^i(s^i/\kappa^i)
% \kappa^i.
% \end{align}
% \item Moreover, $\varphi^{-\epsilon}(z)<0$ for $\mu$-almost every $z>0$ implies by continuity that $\varphi^{-\epsilon}(0)<0$ which is equivalent to (since $F^-<1$)
% \begin{align*}
% 0&>\varphi^i_-(0) -\varphi^j_-(0)F^-\\
% \implies0&>\varphi^i_-(0) -\varphi^j_-(0)\\
% \iff 0&> 2CP^i(s^i/\kappa^i)\kappa^i-2CP^j(s^j/\kappa^j)\kappa^j.
%     %  |x|(1-F^-)+2CP^j(s^j/\kappa^j)\kappa^j-2CP^i(s^i/\kappa^i)\kappa^i&>0\\
%     %  |x|\frac{1-F}{2}+CP^j(s^j/\kappa^j)\kappa^j-CP^i(s^i/\kappa^i)\kappa^i&>-\frac{\epsilon}{2}|x|.
% \end{align*}
% %For $\epsilon\to0$ and thus $\Delta t_n\to 0$, 
% Thus we obtain the same condition \eqref{eq:OptimalityCondition} as before.


% \end{itemize}

% \item In case $s^i= s^j$ we have that $F>1$ and we obtain the following sufficient and necessary conditions for the $i$\textsuperscript{th}
% asset to be preferred over the $j$\textsuperscript{th} asset:
% \begin{itemize}
%     \item By eq. \eqref{eq:necessary}, if $\varphi^{-\epsilon}>0$ for $z\in B\subset\Rp$, with $\mu(B)>0$, then $ V^{i}_k(x,s)> V^{j}_k(x,s)$. Analogously to \Cref{le:phiepsilonCondition}, one can show that $\varphi^{-\epsilon}(z)>0$ for all $z>0$ (and thus also for $z\in B$), if $\varphi^i_-(z) -\varphi^j_-(z)>0$ for all $z>0$,
%     % \begin{align*}
%     %     |x|(1-F^-)+2CP^j(s^j/\kappa^j)\kappa^j-2CP^i(s^i/\kappa^i)\kappa^i&<0\\
%     %      \iff|x|\frac{1-F}{2}+CP^j(s^j/\kappa^j)\kappa^j-CP^i(s^i/\kappa^i)\kappa^i&<-\frac{\epsilon}{2}|x|.
%     % \end{align*}
%     which is then implied by
% %         \begin{align*}
% % CP^j(s^j/\kappa^j)\kappa^j<CP^i(s^i/\kappa^i)\kappa^i,
% %     \end{align*}
% %     and for $\epsilon\to0$ and thus $\Delta t_n\to 0$, we obtain the condition
%             \begin{align}\label{eq:ConditionForSameAssetValues}
% CP^j(s^j/\kappa^j)\kappa^j<CP^i(s^i/\kappa^i)\kappa^i.
%     \end{align}
%     \item 
%     %By eq. \eqref{eq:sufficient}, if $ V^{i}_k(x,s)> V^{j}_k(x,s)$ then $\varphi^{+\epsilon}(z)<0$ for some $z\ge0$.
%     In our current setting, where $\sigi>\sigma^j$ and $s^i=s^j$ (which implies $\kappa^i=\kappa^j$), the condition \eqref{eq:ConditionForSameAssetValues} is generally fulfilled.
%     %$\varphi^{+\epsilon}<0$ for all $z>0$ is fulfilled for any $\epsilon$.
%     %Rewriting $\varphi^{+\epsilon}<0$ for all $z>0$ and letting $\Delta t_n$ go to zero we again obtain condition \eqref{eq:ConditionForSameAssetValues}.
% \end{itemize}
% Thus, in case $s^i=s^j$, 
% \begin{align*}
%  V^{i}_k(x,s)> V^{j}_k(x,s)&\iff CP^j(s^j/\kappa^j)\kappa^j<CP^i(s^i/\kappa^i)\kappa^i,
%     \end{align*}
% or equivalently
% \begin{align*}
%  V^{j}_k(x,s)> V^{i}_k(x,s)&\iff CP^j(s^j/\kappa^j)\kappa^j>CP^i(s^i/\kappa^i)\kappa^i.
%     \end{align*}



% \item Finally, consider the case $\sigi=\sigma^j$. First, if $s^i=s^j$, then $V_k^i(x,s)=\int_{\Rp}\varphi^i_-(z)\,d\mu(z)$ and we directly get via
% \begin{align*}
%    \varphi^{+\epsilon}(z)&:=\varphi^j_-(z)-\varphi^i_-(z),\\
%       \varphi^{-\epsilon}(z)&:=\varphi^i_-(z) -\varphi^j_-(z),\\
% \end{align*}
% that
% \begin{align*}
%  V^{j}_k(x,s)> V^{i}_k(x,s)&\iff CP^j(s^j/\kappa^j)\kappa^j>CP^i(s^i/\kappa^i)\kappa^i.
%     \end{align*}
%     Second, in the setting of $s^i<s^j$, we have that
%     \begin{align*}
%  CP^j(s^j/\kappa^j)\kappa^j>CP^i(s^i/\kappa^i)\kappa^i,
%     \end{align*}
%    and hence this is a necessary condition for $V^{j}_k(x,s)> V^{i}_k(x,s)$. In order to show that this condition is also sufficient, we bound by above
%    \begin{align*}
%     V^{i}_k(x,s)&<
% (F^++\gamma f_j(s^j))\int\int_{\Rp}\varphi^i_-(z)\,d\mujm(z)f_i(y)f_{ij-}(s_{ij-})\,dyds_{ij-}
% \end{align*}
% with
% \begin{align*}
%     F^+:=\sqrt{\frac{s^i}{s^j}}.
% \end{align*}
% Since $F^+\leq 1$,
%             \begin{align}
% 0&<CP^j(s^j/\kappa^j)\kappa^j-CP^i(s^i/\kappa^i)\kappa^i\\
% \implies0&<CP^j(s^j/\kappa^j)\kappa^j-F^+CP^i(s^i/\kappa^i)\kappa^i
%     \end{align}
% By \Cref{le:phiepsilonCondition} and eq. \eqref{eq:sufficient}, this implies $ V^{j}_k(x,s)> V^{i}_k(x,s)$. 

% \end{itemize}
% Thus, investing in the asset with maximal call price is a necessary and sufficient condition for maximizing the value $V_k$. 
% This concludes the proof.
% \end{proof}
\newpage
\section{Figures}\label{app:sec:figures}
In this section, we give further figures referenced in the experiments of \Cref{sec:experiments}.
\subsection{2D Market with Independent Assets}
As in \Cref{subsubsec:2Duncorrelated}, we show price surfaces estimated by the trained NN strategies in \Cref{fig:2DvaluesAsym} in an asymmetric BS market with $d=2$ uncorrelated, risky assets with squared volatilities $(\sigma^1)^2=0.04, (\sigma^2)^2=0.03$. We show MC estimates of {$e^{-rT}\mathbb{E}{[(X^{\pith}_T)^{+}\mid X_0^{\pith}=0, S_0=s]}$} for both PG and A2C strategies $\pith$ for a grid of initial asset values $s=(s^1,s^2)\in[0,3]^2$. The price corresponding to the trained critic $\Vph_0$ from \Cref{alg:A2C} is shown in \Cref{subfig:2DvaluesCor0A2CAsym}. (For each initial value $s$, we scale $\Vph(s,0)$ as in \Cref{le:absEquiv} to obtain an estimate $\Vph(s,0)/2$ of $V(s,0)$.) 

% Figure environment removed
\clearpage
\subsection{2D Market with Negatively Correlated Assets}
 Analogously to the experiment in \Cref{subsubsec:2Dcorrelated}, we sample a test set of 1000 asset paths and compute the evolutions of portfolio values (PV) $X^{\pith}$ over time until maturity $T=32$, when actions are taken based on the trained network strategies $\pith$. We show these PV evolutions for markets with negative correlations $\rho\in\{-0.1,-0.5,-0.9\}$ in \Cref{fig:2DstratsalongPVnegcor}. In each of the sub-figures of \Cref{fig:2DstratsalongPVnegcor}, the color again signifies the probability that the respective trained network strategy $\pith$ assigns to taking each respective action $q\in\Diamond$ listed in the sub-titles. 
As in the markets with positive correlation of \Cref{subsubsec:2Dcorrelated}, all trained strategies $\pith$ choose to go short for negative portfolio values and long otherwise. Likewise, we observe the same trend that for both network strategies the trading action in asset $S^2$ increases with increasing (negative)correlation in the market. There still is a clear preference for the more volatile asset $S^1$ over $S^2$ (as in the uncorrelated setting shown in \Cref{fig:2DstratsalongPVAsym}), however we see in \Cref{fig:2DstratsalongPVnegcor} that as the correlation in the market increases (i.e., as $\rho$ decreases towards $-1$), the probability of investing in asset $S^2$ increases. 
As in the asymmetric setting with positive correlations, this effect is only slightly visible for $\pith$ trained with the A2C algorithm of \Cref{sec:A2C} (right subplots of \Cref{fig:2DstratsalongPVnegcor}). For the network strategy $\pith$ trained with the PG algorithm of \Cref{sec:PGA} (left subplots of \Cref{fig:2DstratsalongPVnegcor}), the tendency however is less strong as in the positively correlated market. For $\rho \in\{-0.5, -0.9\}$, the network strategy still tells to invest in both assets $S^1$ and $S^2$. 

Furthermore, we also show distributions of terminal values $X_T^{\pith}$ and terminal payoffs $|X_T^{\pith}|$ over 100 000 test paths achieved by different strategies in \Cref{fig:2DPVDistnegcor} for markets with negative correlations. Analogously to the previous experiments of \Cref{subsec:1Dexperiments,subsubsec:2Duncorrelated,subsubsec:2Dcorrelated}, we consider distributions resulting from the following strategies $\pith$: first, the trained PG and A2C strategies obtained from \Cref{alg:policyGrad,alg:A2C} (PG and A2C in \Cref{fig:2DPVDistnegcor}), second, a constant buy-and-hold strategy, i.e., $\pith_t(s,x)(q)=1$ for strategy $q\in\Diamond$ fixed for all $t\in\R_+$, $(s,x)\in\mathcal{X}$, and a strategy that randomly chooses from actions $q\in\Diamond$ in each step, i.e., $\pith_t(s,x)(q)=0.25$ for $q\in\Diamond$ (Const and Rand in \Cref{fig:2DPVDistnegcor}), and third, the strategy of \Cref{thm:optimalStrat} (Opt in \Cref{fig:2DPVDistnegcor}). Note once more that we keep referring to the latter strategy as ``optimal strategy'', where optimal now means that it would be optimal in the same BS market without correlation.
Across correlations, we observe that analogously to the outcome in positively correlated markets of \Cref{fig:2DPVDistposcor}, the trained network strategies $\pith$ and the ``optimal'' strategy of \Cref{thm:optimalStrat} outperform the random and constant strategies, since the estimates of expected terminal payoffs (i.e., the means in the right sub-plots of \Cref{fig:2DPVDistnegcor}) corresponding to the former are significantly larger than the ones resulting from trading with the latter strategies.
In particular, we observe in \Cref{fig:2DPVDistnegcor} that both the PG and the A2C algorithms yield strategies that perform comparably to the strategy of \Cref{thm:optimalStrat} that was optimal only in the market with $\rho=0$.

% Figure environment removed

% Figure environment removed