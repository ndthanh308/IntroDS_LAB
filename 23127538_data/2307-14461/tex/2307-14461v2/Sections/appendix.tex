\newpage
\section*{Appendix}

\subsection*{Proof of Proposition 10}

    Proving~\autoref{prop: Homotopy posets are functorial} requires to build a hefty amount of theory, which is why we reserve the Appendix for this.

    \begin{definition}[Past extension] \label{dfn:past_extension}
    Let $\cat{A}$ be a category.
    A \emph{past extension of $\cat{A}$} is a functor $\imath\colon \cat{A} \incl \cat{B}$ with the following property: there exists a functor $\indic{\cat{A}}\colon \cat{B} \to \wkarr$ such that
    \begin{equation} \label{eq:past_extension}
    \begin{tikzcd}[sep=scriptsize]
	\cat{A} && \Term \\
	\\
	\cat{B} && \wkarr
	\arrow["{!}", from=1-1, to=1-3]
	\arrow[hook, "\imath", from=1-1, to=3-1]
	\arrow["\indic{\cat{A}}", from=3-1, to=3-3]
	\arrow[hook, "{1}", from=1-3, to=3-3]
	\arrow["\lrcorner"{anchor=center, pos=0.125}, draw=none, from=1-1, to=3-3]
    \end{tikzcd}
    \end{equation}
    is a pullback in $\catcat$.
    \end{definition}
    %
    %
    \begin{remark} \label{rmk:past extension collage}
    The following is an equivalent characterisation of past extensions: there exist a category $\cat{\bar{A}}$ and a profunctor $\fun{H}\colon \opp{\cat{\bar{A}}} \times \cat{A} \to \catset$ such that
    \begin{enumerate}
        \item $\cat{B}$ is isomorphic to the \emph{collage}, also known as \emph{cograph}, of $\fun{H}$, and
        \item $\imath$ is, up to isomorphism, the inclusion of $\cat{A}$ into the collage.
    \end{enumerate}
    A technical name for a functor satisfying the condition on $\imath$ is \emph{codiscrete coopfibration}; it is one leg of a two-sided codiscrete cofibration of categories.

    The idea is that $\imath$ embeds $\cat{A}$ into a larger category, whose objects outside of the image of $\cat{A}$ only have morphisms pointing \emph{towards} $\cat{A}$, hence are ``in the past'' of $\cat{A}$ if we interpret the direction of morphisms as a time direction.
    Notice that the fact that (\ref{eq:past_extension}) is a pullback implies that $\imath$ is injective on objects and morphisms, using their representation as functors from $\Term$ and $\wkarr$, respectively.
    
    The following picture illustrates the bipartition of $\cat{B}$ induced by $\indic{A}$, with the fibre $\cat{\bar{A}}$ of 0 ``in the past'' of the fibre $\cat{A}$ of 1:
    \[\begin{tikzcd}[sep=scriptsize]
	{\cat{B}} & {\blue{\cat{\bar{A}}}} & {\blue{\bullet}} &&& {\red{\bullet}} & {\red{\cat{A}}} \\
	& {\blue{\bullet}} &&&&& {\red{\bullet}} \\
	&& {\blue{\bullet}} &&& {\red{\bullet}} \\
	\wkarr && {\blue{0}} &&& {\red{1}}
	\arrow[color={rgb,255:red,92;green,92;blue,214}, curve={height=-6pt}, from=2-2, to=1-3]
	\arrow[color={rgb,255:red,92;green,92;blue,214}, curve={height=6pt}, from=3-3, to=1-3]
	\arrow[color={rgb,255:red,92;green,92;blue,214}, curve={height=-6pt}, from=2-2, to=3-3]
	\arrow[curve={height=-6pt}, from=1-3, to=1-6]
	\arrow[color={rgb,255:red,214;green,92;blue,92}, curve={height=-6pt}, from=1-6, to=2-7]
	\arrow["{\indic{A}}", from=1-1, to=4-1]
	\arrow[color={rgb,255:red,214;green,92;blue,92}, curve={height=-6pt}, from=1-6, to=3-6]
	\arrow[curve={height=-6pt}, from=3-3, to=3-6]
	\arrow[curve={height=-18pt}, from=1-3, to=3-6]
	\arrow["a", from=4-3, to=4-6]
    \end{tikzcd}\]
    \end{remark}
    %
    %
    \begin{definition}[Category of past extensions] \label{dfn:category_of_past_extensions}
    Let $\cat{A}$ be a category.
    The \emph{category of past extensions of $\cat{A}$} is the large category $\pastext{\cat{A}}$ whose
    \begin{itemize}
        \item objects are past extensions $\imath\colon A \incl B$, and
        \item a morphism from $(\imath\colon A \incl B)$ to $(j\colon A \incl B')$ is a factorisation of $j$ through $\imath$, that is, a functor $\fun{K}\colon \cat{B} \to \cat{B'}$ such that $j = \imath\Cp \fun{K}$.
    \end{itemize}
    \end{definition}
    %
    %
    \begin{proposition}[The indexed category of past extensions of functors]
    Let $\cat{A}$ and $\cat{C}$ be categories.
    Then there exists a functor
    \begin{equation*}
        \extfun{\cat{A}}{\cat{C}}\colon \opp{\pastext{\cat{A}}} \times \cat{C}^\cat{A} \to \catcat
    \end{equation*}
    whose object part is defined as follows: 
    given a past extension $\imath\colon \cat{A} \incl \cat{B}$ and a functor $\fun{F}\colon \cat{A} \to \cat{C}$, the category $\extfun{\cat{A}}{\cat{C}}(\imath, \fun{F})$ is the subcategory of $\cat{C}^\cat{B}$ whose
    \begin{itemize}
        \item objects are (strict) extensions of $\fun{F}$ along $\imath$, that is, functors $\fun{\tilde{F}}\colon \cat{B} \to \cat{C}$ such that
        \[\begin{tikzcd}[sep=scriptsize]
	   {\cat{A}} && {\cat{C}} \\
	   \\
	   {\cat{B}}
	   \arrow["\imath", hook, from=1-1, to=3-1]
	   \arrow["{\fun{F}}", from=1-1, to=1-3]
	   \arrow["{\fun{\tilde{F}}}"', from=3-1, to=1-3]
    \end{tikzcd}\]
    strictly commutes, and
    \item morphisms from $\fun{\tilde{F}_1}$ to $\fun{\tilde{F}_2}$ are natural transformations $\tau\colon \fun{\tilde{F}_1} \Rightarrow \fun{\tilde{F}_2}$ that restrict along $\imath$ to the identity natural transformation on $\fun{F}$.
    \end{itemize}
    \end{proposition}
    %
    \begin{proof}
        Given a morphism $\fun{K}\colon (\imath\colon \cat{A} \incl \cat{B}) \to (j\colon \cat{A} \incl \cat{B'})$ in $\pastext{\cat{A}}$,
\begin{equation*}
    \fun{K}^* \eqdef \extfun{\cat{A}}{\cat{C}}(\fun{K},\fun{F})\colon \extfun{\cat{A}}{\cat{C}}(j, \fun{F}) \to \extfun{\cat{A}}{\cat{C}}(\imath, \fun{F})
\end{equation*}
is the functor that acts by precomposition, sending
\begin{itemize}
    \item $\fun{\tilde{F}}\colon \cat{B'} \to \cat{C}$ to $\fun{K}\Cp\fun{\tilde{F}}\colon \cat{B} \to \cat{C}$, and
    \item $\tau\colon \fun{\tilde{F}_1} \Rightarrow \fun{\tilde{F}_2}$ to $\fun{K}\Cp\tau\colon \fun{K}\Cp\fun{\tilde{F}_1} \Rightarrow \fun{K}\Cp\fun{\tilde{F}_2}$.
\end{itemize}
This is well-defined as
\begin{equation*}
    \imath\Cp\fun{K}\Cp\fun{\tilde{F}} = j\Cp\fun{\tilde{F}} = \fun{F}, \quad \quad
    \imath\Cp\fun{K}\Cp\tau = j\Cp\tau = \idd{\fun{F}}.
\end{equation*}
Moreover, it is straightforward to check that
\begin{equation*}
    (\idd{\imath})^* = \idd{\extfun{\cat{A}}{\cat{C}}(\imath, \fun{F})}, \quad \quad 
    (\fun{K}\Cp\fun{L})^* = \fun{L}^*\Cp \fun{K}^*
\end{equation*}
for any composable pair $\fun{K}, \fun{L}$ of morphisms in $\pastext{\cat{A}}$.

Given a natural transformation $\alpha\colon \fun{F} \Rightarrow \fun{G}$ between functors $\fun{F}, \fun{G}\colon \cat{A} \to \cat{C}$, the functor
\begin{equation*}
    \alpha_* \eqdef \extfun{\cat{A}}{\cat{C}}(\imath, \alpha)\colon \extfun{\cat{A}}{\cat{C}}(\imath, \fun{F}) \to \extfun{\cat{A}}{\cat{C}}(\imath, \fun{G})
\end{equation*}
is defined as follows.
Given an object $\fun{\tilde{F}}\colon \cat{B} \to \cat{C}$ of $\extfun{\cat{A}}{\cat{C}}(\imath, \fun{F})$, the functor $\alpha_*\fun{\tilde{F}}\colon \cat{B} \to \cat{C}$ is defined, on each morphism $f\colon x \to y$ in $\cat{B}$, by
\begin{equation*}
    \alpha_*\fun{\tilde{F}}(f) \eqdef
    \begin{cases}
        \fun{G}(f')
            & \text{if $\indic{A}(f) = 1$ and $f = \imath(f')$}, \\
        \fun{\tilde{F}}(f)\Cp\alpha_{y'}
            & \text{if $\indic{A}(f) = a$ and $y = \imath(y')$}, \\
        \fun{\tilde{F}}(f)
            & \text{if $\indic{A}(f) = 0$}.
    \end{cases}
\end{equation*}
By construction $\imath\Cp \alpha_*\fun{\tilde{F}} = \fun{G}$.
The following picture illustrates the definition.
\[\begin{tikzcd}[sep=scriptsize]
	&&&&& {\magenta{\fun{G}y}} & {\magenta{\fun{G}\cat{A}}} \\
	&&&&&& {\magenta{\bullet}} \\
	{\blue{\fun{\tilde{F}}\cat{\bar{A}} = \alpha_*\fun{\tilde{F}}\cat{\bar{A}}}} & {\blue{\fun{\tilde{F}}x}} &&& {\red{\fun{F}y}} & {\magenta{\bullet}} \\
	{\blue{\bullet}} &&&&& {\red{\bullet}} \\
	& {\blue{\bullet}} &&& {\red{\bullet}} & {\red{\fun{F}\cat{A}}}
	\arrow[color={rgb,255:red,92;green,92;blue,214}, curve={height=-6pt}, from=4-1, to=3-2]
	\arrow[color={rgb,255:red,92;green,92;blue,214}, curve={height=6pt}, from=5-2, to=3-2]
	\arrow[color={rgb,255:red,92;green,92;blue,214}, curve={height=-6pt}, from=4-1, to=5-2]
	\arrow["{\fun{\tilde{F}}f}"', curve={height=-6pt}, from=3-2, to=3-5]
	\arrow[color={rgb,255:red,214;green,92;blue,92}, curve={height=-6pt}, from=3-5, to=4-6]
	\arrow[color={rgb,255:red,214;green,92;blue,92}, curve={height=-6pt}, from=3-5, to=5-5]
	\arrow[curve={height=-6pt}, from=5-2, to=5-5]
	\arrow[curve={height=-18pt}, from=3-2, to=5-5]
	\arrow[color={rgb,255:red,214;green,92;blue,214}, curve={height=-6pt}, from=1-6, to=2-7]
	\arrow[color={rgb,255:red,214;green,92;blue,214}, curve={height=-6pt}, from=1-6, to=3-6]
	\arrow["{\alpha_y}"', color={rgb,255:red,36;green,143;blue,36}, from=3-5, to=1-6]
	\arrow[color={rgb,255:red,36;green,143;blue,36}, from=4-6, to=2-7]
	\arrow[color={rgb,255:red,36;green,143;blue,36}, from=5-5, to=3-6]
	\arrow["{\alpha_*\fun{\tilde{F}}f}", curve={height=-12pt}, dashed, from=3-2, to=1-6]
	\arrow[curve={height=-18pt}, dashed, from=3-2, to=3-6]
	\arrow[curve={height=-12pt}, dashed, from=5-2, to=3-6]
\end{tikzcd}\]
Let us show that $\alpha_*\fun{\tilde{F}}$ is well-defined as a functor.
\begin{enumerate}
    \item Given an identity $\idd{x}$ in $\cat{B}$, necessarily $\indic{A}(\idd{x}) = 0$, in which case 
    \[ \alpha_*\fun{\tilde{F}}(\idd{x}) = \fun{\tilde{F}}(\idd{x}) = \idd{\fun{\tilde{F}}(x)}, \] 
    or $\indic{A}(\idd{x}) = 1$, in which case
    \[ \alpha_*\fun{\tilde{F}}(\idd{x}) = \fun{G}(\idd{x'}) = \idd{\fun{G}(x')}, \]
    where $x'$ is the unique lift of $x$ to $\cat{A}$.
    Thus $\alpha_*\fun{\tilde{F}}$ preserves identities.

    \item Given a composable pair $f\colon x \to y$, $g\colon y \to z$, we have the following cases.
    \begin{itemize}
        \item If $\indic{A}(f) = \indic{A}(g) = 1$, then $\indic{A}(f\Cp g) = 1$, and
        \[
            \alpha_*\fun{\tilde{F}}(f)\Cp \alpha_*\fun{\tilde{F}}(g) = \fun{G}(f')\Cp\fun{G}(g') = \fun{G}(f'\Cp g') = \alpha_*\fun{\tilde{F}}(f\Cp g),
        \]
        where $f', g'$ are the unique lifts of $f, g$ to $\cat{A}$.
        \item If $\indic{A}(f) = \indic{A}(g) = 0$, then $\indic{A}(f\Cp g) = 0$, and
        \[
            \alpha_*\fun{\tilde{F}}(f)\Cp  \alpha_*\fun{\tilde{F}}(g) = \fun{\tilde{F}}(f)\Cp \fun{\tilde{F}}(g) = \fun{\tilde{F}}(f\Cp g) = \alpha_*\fun{\tilde{F}}(f\Cp g).
        \]
        \item If $\indic{A}(f) = 0$ and $\indic{A}(g) = a$, then $\indic{A}(f\Cp g) = a$, and
        \[
            \alpha_*\fun{\tilde{F}}(f)\Cp  \alpha_*\fun{\tilde{F}}(g) = \fun{\tilde{F}}(f)\Cp \fun{\tilde{F}}(g)\Cp \alpha_{z'} = \fun{\tilde{F}}(f\Cp g)\Cp \alpha_{z'} = \alpha_*\fun{\tilde{F}}(f\Cp g),
        \]
        where $z'$ is the unique lift of $z$ to $\cat{A}$.
        \item If $\indic{A}(f) = a$ and $\indic{A}(g) = 1$, then $\indic{A}(f\Cp g) = a$, and
        \[
            \alpha_*\fun{\tilde{F}}(f)\Cp  \alpha_*\fun{\tilde{F}}(g) = \fun{\tilde{F}}(f)\Cp \alpha_{y'}\Cp \fun{G}(g') = \fun{\tilde{F}}(f)\Cp \fun{F}(g')\Cp \alpha_{z'},
        \]
        where $g'\colon y' \to z'$ is the unique lift of $g$ to $\cat{A}$, and we used naturality of $\alpha$.
        
        Since $\fun{F}(g') = \tilde{\fun{F}}(\imath(g')) = \tilde{\fun{F}}(g)$, this is equal to 
        \[ \fun{\tilde{F}}(f)\Cp \fun{\tilde{F}}(g)\Cp \alpha_{z'} = \alpha_*\fun{\tilde{F}}(f\Cp g). \]
    \end{itemize}
    No other cases are possible.
\end{enumerate}
This proves that $\alpha_*\fun{\tilde{F}}$ is well-defined.

Given a morphism $\tau\colon \fun{\tilde{F}_1} \Rightarrow \fun{\tilde{F}_2}$ of $\extfun{\cat{A}}{\cat{C}}(\imath, \fun{F})$, the natural transformation $\alpha_*\tau\colon \alpha_*\fun{\tilde{F}_1} \Rightarrow \alpha_*\fun{\tilde{F}_2}$ is defined, on each object $x$ in $\cat{B}$, by
\begin{equation*}
    (\alpha_*\tau)_x \eqdef
    \begin{cases}
        \idd{\fun{G}(x')} & \text{if $\indic{A}(x) = 1$ and $x = \imath(x')$}, \\
        \tau_x & \text{if $\indic{A}(x) = 0$}.
    \end{cases}
\end{equation*}
To show that this is well-defined as a natural transformation, consider a morphism $f\colon x \to y$ in $\cat{B}$.
\begin{itemize}
    \item If $\indic{A}(f) = 1$ and $f'\colon x' \to y'$ is the unique lift of $f$ to $\cat{A}$, then
    \[
        \alpha_*\fun{\tilde{F}_1}(f)\Cp  (\alpha_*\tau)_y =
        \fun{G}(f')\Cp  \idd{\fun{G}(y')} = \idd{\fun{G}(x')}\Cp  \fun{G}(f') = (\alpha_*\tau)_x\Cp  \alpha_*\fun{\tilde{F}_2}(f).
    \]
    \item If $\indic{A}(f) = a$ and $y'$ is the unique lift of $y$ to $\cat{A}$, then
    \[
        \alpha_*\fun{\tilde{F}_1}(f)\Cp  (\alpha_*\tau)_y =
        \fun{\tilde{F}_1}(f)\Cp  \alpha_{y'}\Cp  \idd{\fun{G}(y')} = 
        \fun{\tilde{F}_1}(f)\Cp  \tau_{y}\Cp  \alpha_{y'}
    \]
    since $\tau_y = \tau_{\imath(y')} = \idd{\fun{F}(y')}$.
    By naturality of $\tau$, this is equal to
    \[
        \tau_x\Cp  \fun{\tilde{F}_2}(f)\Cp  \alpha_{y'} =
        (\alpha_*\tau)_x\Cp  \alpha_*\fun{\tilde{F}_2}(f).
    \]
    \item If $\indic{A}(f) = 0$, then
    \[
        \alpha_*\fun{\tilde{F}_1}(f)\Cp  (\alpha_*\tau)_y =
        \fun{\tilde{F}_1}(f)\Cp  \tau_{y} = \tau_x\Cp  \fun{\tilde{F}_2}(f) = (\alpha_*\tau)_x\Cp  \alpha_*\fun{\tilde{F}_2}(f).
    \]
\end{itemize}
This concludes the definition of $\alpha_*$.
It is straightforward to check that
\begin{equation*}
    (\idd{\fun{F}})_* = \idd{\extfun{\cat{A}}{\cat{C}}(\imath, \fun{F})}, \quad \quad 
    (\alpha\Cp \beta)_* = \alpha_* \Cp  \beta_*
\end{equation*}
for all pairs of natural transformations $\alpha, \beta$ composable as morphisms in $\cat{C}^\cat{A}$.
Finally, one can verify that, for all morphisms $\fun{K}\colon \imath \to j$ in $\pastext{\cat{A}}$ and $\alpha\colon \fun{F} \to \fun{G}$ in $\cat{C}^\cat{A}$, the diagram of functors
\[
\begin{tikzcd}[sep=scriptsize]
	\extfun{\cat{A}}{\cat{C}}(j, \fun{F}) && \extfun{\cat{A}}{\cat{C}}(\imath, \fun{F}) \\
	\\
	\extfun{\cat{A}}{\cat{C}}(j, \fun{G}) && \extfun{\cat{A}}{\cat{C}}(\imath, \fun{G})
	\arrow["\fun{K}^*", from=1-1, to=1-3]
	\arrow["\alpha_*", from=1-1, to=3-1]
	\arrow["\fun{K}^*", from=3-1, to=3-3]
	\arrow["\alpha_*", from=1-3, to=3-3]
\end{tikzcd}
\]
commutes in $\catcat$.
Thus we can define $\extfun{\cat{A}}{\cat{C}}(\fun{K}, \alpha)$ as either path in the commutative diagram, and conclude that $\extfun{\cat{A}}{\cat{C}}$ is well-defined as a functor.
    \end{proof}

    \begin{proposition}[Covariance of the $\extfun{\cat{A}}{\cat{C}}$]
    The assignment $\cat{C} \mapsto \extfun{\cat{A}}{\cat{C}}$ extends to a functor
    \begin{equation*}
        \extfun{\cat{A}}{}\colon \catcat \to \laxslice{\lcatcat}{\catcat}.
    \end{equation*}
    \end{proposition}
    \begin{proof}
    Given a functor $\fun{P}\colon \cat{C} \to \cat{D}$, post-composition with $\fun{P}$ defines a functor $\fun{P}_*\colon \cat{C}^\cat{A} \to \cat{D}^\cat{A}$.
    Then there is a natural transformation
    \begin{equation} \label{eq:naturality_extfun}
    \begin{tikzcd}
	{\opp{\pastext{\cat{A}}} \times \cat{C}^\cat{A}} &&& \catcat \\
	\\
	{\opp{\pastext{\cat{A}}} \times \cat{D}^\cat{A}}
	\arrow["{\idd{} \times \fun{P}_*}"', from=1-1, to=3-1]
	\arrow["{\extfun{\cat{A}}{\cat{D}}}"', from=3-1, to=1-4]
	\arrow[""{name=0, anchor=center, inner sep=0}, "{\extfun{\cat{A}}{\cat{C}}}", from=1-1, to=1-4]
	\arrow["{\extfun{\cat{A}}{\fun{P}}}"', shorten <=17pt, shorten >=26pt, Rightarrow, from=0, to=3-1]
    \end{tikzcd}
    \end{equation}
    defined as follows: given a past extension $\imath\colon \cat{A} \incl \cat{B}$ and a functor $\fun{F}\colon \cat{A} \to \cat{C}$, the functor
    \begin{equation*}
        \extfun{\cat{A}}{\fun{P}}(\imath, \fun{F})\colon \extfun{\cat{A}}{\cat{C}}(\imath, \fun{F}) \to \extfun{\cat{A}}{\cat{D}}(\imath, \fun{F}\Cp \fun{P})
    \end{equation*}
    acts both on objects and on morphisms by post-composition with $\fun{P}$.
    It is straightforward to check that the assignment $\fun{P} \mapsto \extfun{\cat{A}}{\fun{P}}$ respects identities and composition in $\catcat$.
    \end{proof}

    \begin{remark}[General functoriality pattern] \label{rmk: functoriality pattern}
    A fixed morphism $\fun{K}$ in $\pastext{\cat{A}}$ is classified by a functor $\wkarr \to \pastext{\cat{A}}$.
    Evaluating $\extfun{\cat{A}}{\cat{C}}$ at $\fun{K}$ thus determines a functor
    \begin{equation*}
        \extfun{\cat{A}}{\cat{C}}(\fun{K}, -)\colon \wkarr \times \cat{C}^\cat{A} \to \catcat,
    \end{equation*}
    which we can curry to obtain a functor
    \begin{equation} \label{eq:general_pattern_cat}
        \Lambda.\extfun{\cat{A}}{\cat{C}}(\fun{K}, -)\colon \cat{C}^\cat{A} \to \catcat^\wkarr.
    \end{equation}
    Given a functor $\fun{P}\colon \cat{C} \to \cat{D}$, we can also ``curry the natural transformation'' in (\ref{eq:naturality_extfun}) to obtain a diagram
    \begin{equation} \label{eq:general_functor_covariance}
    \begin{tikzcd}[sep=large]
	{\cat{C}^\cat{A}} &&& \catcat^\wkarr \\
	\\
	{\cat{D}^\cat{A}}
	\arrow["{\fun{P}_*}"', from=1-1, to=3-1]
	\arrow["{\Lambda.\extfun{\cat{A}}{\cat{D}}(\fun{K},-)}"', from=3-1, to=1-4]
	\arrow[""{name=0, anchor=center, inner sep=0}, "{\Lambda.\extfun{\cat{A}}{\cat{C}}(\fun{K},-)}", from=1-1, to=1-4]
	\arrow["{\Lambda.\extfun{\cat{A}}{\fun{P}}(\fun{K},-)}"{description}, shorten <=17pt, shorten >=26pt, Rightarrow, from=0, to=3-1]
    \end{tikzcd}
    \end{equation}
    which is part of a functor $\catcat \to \laxslice{\lcatcat}{\catcat^\wkarr}$.

    Post-composing with the functor $\catcat^\wkarr \to \pointed{\catpos}$ from (\ref{eq: quotient and posref}) we obtain a covariant family of functors $\cat{C}^\cat{A} \to \pointed{\catpos}$.

    We will show that, for suitable choices of $\cat{A}$ and $\fun{K}$, the image of these functors is included in the subcategory of $\pointed{\catpos}$ on the zeroth and first homotopy posets of $\cat{C}$ or categories associated with $\cat{C}$, exhibiting various kinds of functorial dependence of homotopy posets.
    \end{remark}
    
    \begingroup
    \def\theproposition{\ref{prop: Homotopy posets are functorial}}
    \begin{proposition}[Functoriality of the homotopy posets]
        Let $\cat{C}$ be a category, $i \in \{0, 1\}$.
        Then:
        \begin{enumerate}
            \item the assignment $x \mapsto \dhom{i}{\cat{C}}{x}$ extends to a functor
                $\dhom{i}{\cat{C}}{-}\colon \cat{C} \to \pointed{\catpos}$;
            \item a functor $\fun{F}\colon \cat{C} \to \cat{D}$ induces a natural transformation 
                $\pi_i(\fun{F})\colon \dhom{i}{\cat{C}}{-} \Rightarrow \dhom{i}{\cat{D}}{\fun{F}-}.$
        \end{enumerate}
        Given another functor $\fun{G}\colon \cat{D} \to \cat{E}$,  this assignment satisfies
        \begin{equation*}
            \pi_i(\fun{F}\Cp \fun{G}) = \pi_i(\fun{F}) \Cp \pi_i(\fun{G}), \quad \quad \pi_i(\idd{C}) = \idd{\dhom{i}{\cat{C}}{-}}.
        \end{equation*}
    \end{proposition}
    \addtocounter{proposition}{-1}
    \endgroup
    \begin{proof}
    We will derive the results for both $i \in \{0, 1\}$ from the general functoriality pattern of \autoref{rmk: functoriality pattern}.
    
    First we consider the case $i = 0$.
    Let $\Term$ be the terminal category.
    The inclusion $\fun{K_0}$ of the endpoints of the walking arrow induces a morphism in $\pastext{\Term}$, depicted as follows:
    \begin{equation*}
    \begin{tikzcd}[sep=small]
	{\blue{\bullet}} & {} & {\red{\bullet}} &&& {\blue{\bullet}} & {} & {\red{\bullet}}
	\arrow[from=1-6, to=1-8]
	\arrow[color={rgb,255:red,102;green,102;blue,102}, curve={height=-24pt}, shorten <=15pt, shorten >=15pt, dashed, hook, from=1-2, to=1-7]
    \end{tikzcd} \quad \quad
    \begin{tikzcd}[sep=scriptsize]
	& \Term \\
	\\
	{\Term+\Term} && \wkarr
	\arrow["{\fun{K_0}}"', hook, from=3-1, to=3-3]
	\arrow["{\imath_1}"', hook, from=1-2, to=3-1]
	\arrow["1", hook, from=1-2, to=3-3]
    \end{tikzcd}\end{equation*}
    We claim that, up to isomorphism of categories,
    \begin{equation*}
        \Lambda.\extfun{\Term}{\cat{C}}(\fun{K_0}, -)\colon \cat{C}^1 \to \catcat^\wkarr
    \end{equation*}
    sends an object $x$ of $\cat{C}^\Term$ --- which is, equivalently, an object of $\cat{C}$ --- to the slice projection functor
    \begin{equation*}
        \mathrm{dom}\colon \slice{\cat{C}}{x} \to \cat{C}.
    \end{equation*}
    The domain of $\Lambda.\extfun{\Term}{\cat{C}}(\fun{K_0}, x)$ is the category $\extfun{\Term}{\cat{C}}(1, x)$ whose
    \begin{itemize}
        \item objects are functors $f\colon \wkarr \to \cat{C}$ such that 
    \[\begin{tikzcd}[sep=scriptsize]
	   {\Term} && {\cat{C}} \\
	   \\
	   {\wkarr}
	   \arrow["1", hook, from=1-1, to=3-1]
	   \arrow["x", from=1-1, to=1-3]
	   \arrow["f"', from=3-1, to=1-3]
    \end{tikzcd}\]
        commutes, which are in bijection with morphisms $f$ of $\cat{C}$ whose codomain is $x$, and
        \item morphisms from $f$ to $g$ are natural transformations $h\colon f \Rightarrow g$ --- which are in bijection with commutative squares
        \[\begin{tikzcd}[sep=scriptsize]
	   y && z \\
	   \\
	   x && x
	   \arrow["f"', from=1-1, to=3-1]
	   \arrow["{h_1}"', from=3-1, to=3-3]
	   \arrow["{h_0}", from=1-1, to=1-3]
	   \arrow["g", from=1-3, to=3-3]
        \end{tikzcd}\]
        in $\cat{C}$ --- that restrict to the identity along $1\colon \Term \incl \wkarr$, that is, are such that $h_1 = \idd{x}$.
        These are in bijection with factorisations of $f$ through $g$.
    \end{itemize}
    This establishes an isomorphism between $\extfun{\Term}{\cat{C}}(1, x)$ and $\slice{\cat{C}}{x}$.
    %
    The codomain of $\Lambda.\extfun{\Term}{\cat{C}}(\fun{K_0}, x)$ is the category $\extfun{\Term}{\cat{C}}(\imath_1, x)$ whose
    \begin{itemize}
        \item objects are functors $(y_0, y_1)\colon \Term + \Term \to \cat{C}$ such that 
    \[\begin{tikzcd}[sep=scriptsize]
	   {\Term} && {\cat{C}} \\
	   \\
	   {\Term + \Term}
	   \arrow["{\imath_1}", hook, from=1-1, to=3-1]
	   \arrow["x", from=1-1, to=1-3]
	   \arrow["{(y_0, y_1)}"', from=3-1, to=1-3]
    \end{tikzcd}\]
        commutes, which are in bijection with pairs of objects $(y_0, y_1)$ of $\cat{C}$ such that $y_1 = x$, which are in bijection with objects of $\cat{C}$, and
        \item morphisms from $(y, x)$ to $(z, x)$ are in bijection with pairs of morphisms
    \[\begin{tikzcd}[sep=scriptsize]
	y && z \\
	x && x
	\arrow["{h_1}"', from=2-1, to=2-3]
	\arrow["{h_0}", from=1-1, to=1-3]
    \end{tikzcd}\]
        in $\cat{C}$ that restrict to the identity along $\imath_1$, that is, are such that $h_1 = \idd{x}$.
        These are in bijection with morphisms $y \to z$.
    \end{itemize}
    This establishes an isomorphism between $\extfun{\Term}{\cat{C}}(\imath_1, x)$ and $\cat{C}$.
    The functor $\extfun{\Term}{\cat{C}}(\fun{K_0}, x)$ acts by restriction of $f\colon \wkarr \to \cat{C}$ along $\fun{K_0}\colon \Term+\Term \incl \wkarr$; through the isomorphisms, this acts by mapping $f\colon y \to x$ to its domain $y$.
    This is, by inspection, the same as the action of $\mathrm{dom}$.

    We define
    \begin{equation*}
        \dhom{0}{\cat{C}}{-}\colon \cat{C} \to \pointed{\catpos}
    \end{equation*}
    to be the post-composition of $\Lambda.\extfun{\Term}{\cat{C}}(\fun{K_0}, -)$ with the functor of \autoref{eq: quotient and posref}.
    It follows from our argument that, up to isomorphism, this sends $x$ to the homotopy poset $\dhom{0}{\cat{C}}{x}$.
    The covariance in $\cat{C}$ then follows as an instance of \autoref{eq:general_functor_covariance}: given a functor $\fun{F}\colon \cat{C} \to \cat{D}$, we whisker the natural transformation $\Lambda.\extfun{\Term}{\fun{F}}(\fun{K_0}, -)$ with the functor of (\ref{eq: quotient and posref}) to obtain $\pi_0(\fun{F})\colon \dhom{0}{\cat{C}}{-} \Rightarrow \dhom{0}{\cat{D}}{\fun{F}-}$.

    Now, let us focus on the first homotopy poset.
    The functor $\fun{K_1}$ identifying two parallel arrows also induces a morphism in $\pastext{\Term}$, depicted as follows:
    \begin{equation*}
    \begin{tikzcd}[sep=small]
	{\blue{\bullet}} & {} & {\red{\bullet}} &&& {\blue{\bullet}} & {} & {\red{\bullet}}
	\arrow[curve={height=-6pt}, from=1-1, to=1-3]
	\arrow[curve={height=6pt}, from=1-1, to=1-3]
	\arrow[from=1-6, to=1-8]
	\arrow[color={rgb,255:red,102;green,102;blue,102}, curve={height=-24pt}, shorten <=15pt, shorten >=15pt, dashed, two heads, from=1-2, to=1-7]
    \end{tikzcd}
    \quad \quad
    \begin{tikzcd}[sep=scriptsize]
	& \Term \\
	\\
	  \mathrm{Par} && \wkarr
	\arrow["{\fun{K_1}}"', hook, from=3-1, to=3-3]
	\arrow["c"', hook, from=1-2, to=3-1]
	\arrow["1", hook, from=1-2, to=3-3]
    \end{tikzcd}
    \end{equation*}
    Here, $\mathrm{Par}$ denotes the ``walking parallel pair of arrows''.
    We claim that, up to isomorphism of categories,
    \begin{equation*}
        \Lambda.\extfun{\Term}{\cat{C}}(\fun{K_1}, -)\colon \cat{C} \to \catcat^\wkarr
    \end{equation*}
    sends an object $x$ of $\cat{C}$ to the slice projection functor
    \begin{equation*}
        \mathrm{dom}\colon \slice{\pararr{\cat{C}}{x}}{(\idd{x}, \idd{x})} \to \pararr{\cat{C}}{x}.
    \end{equation*}
    We have already established that the domain of $\Lambda.\extfun{\Term}{\cat{C}}(\fun{K_1}, x)$, which is the category $\extfun{\Term}{\cat{C}}(1, x)$, is isomorphic to $\slice{\cat{C}}{x}$, which can be shown to be isomorphic to $\slice{\pararr{\cat{C}}{x}}{(\idd{x}, \idd{x})}$ using Proposition \ref{prop: subterminal as weak terminal parallel arrow}.

    The codomain of $\Lambda.\extfun{\Term}{\cat{C}}(\fun{K_1}, x)$ is the category $\extfun{\Term}{\cat{C}}(c, x)$ whose
    \begin{itemize}
        \item objects are functors $(f_0, f_1)\colon \mathrm{Par} \to \cat{C}$ such that 
    \[\begin{tikzcd}[sep=scriptsize]
	{\Term} && {\cat{C}} \\
	\\
	{{\mathrm{Par}}}
	\arrow["c", hook, from=1-1, to=3-1]
	\arrow["x", from=1-1, to=1-3]
	\arrow["{(f_0, f_1)}"', from=3-1, to=1-3]
    \end{tikzcd}\]
    commutes, which are in bijection with pairs of morphisms $(f_0, f_1)$ of $\cat{C}$ whose codomain is $x$, and
    \item morphisms from the pair $(f_0, f_1)$ to $(g_0, g_1)$ are natural transformations $h\colon (f_0, f_1) \Rightarrow (g_0, g_1)$ that restrict to the identity along $c$, which are in bijection with morphisms $h$ such that $f_0 = h;g_0$ and $f_1 = h;g_1$.
    \end{itemize}
    This establishes an isomorphism between $\extfun{\Term}{\cat{C}}(c, x)$ and $\pararr{\cat{C}}{x}$.

    The functor $\extfun{\Term}{\cat{C}}(\fun{K_1}, x)$ acts by precomposing $f\colon \wkarr \to \cat{C}$ with $\fun{K_1}\colon \mathrm{Par} \to \wkarr$, which through the isomorphisms sends a pair $(f, f)$ with its unique morphism to $(\idd{x}, \idd{x})$ to the pair $(f, f)$ on its own.
    This is, by inspection, the same as the action of $\mathrm{dom}$.

    We define
    \begin{equation*}
        \dhom{1}{\cat{C}}{-}\colon \cat{C} \to \pointed{\catpos}
    \end{equation*}
    to be the post-composition of $\Lambda.\extfun{\Term}{\cat{C}}(\fun{K_1}, -)$ with the functor of \autoref{eq: quotient and posref}.
    It follows from our argument that, up to isomorphism, this sends $x$ to the homotopy poset $\dhom{1}{\cat{C}}{x}$.
    Again, we obtain covariance in $\cat{C}$ by whiskering instances of \autoref{eq:general_functor_covariance}.
    This completes the proof.
    \end{proof}
