\documentclass[%
 aip,
% jmp,
% bmf,
% sd,
% rsi,
 amsmath,amssymb,
preprint,%
%  reprint,%
%author-year,%
% author-numerical,%
% Conference Proceedings
]{revtex4-1}

\pdfoutput=1
% \setcitestyle{super}
\usepackage[usenames,dvipsnames]{xcolor}
\usepackage{hyperref}
\usepackage{soul}
% \usepackage{refcheck}
\usepackage{tablefootnote}
\usepackage{mathtools}
\usepackage{graphicx}
\usepackage{amssymb}
\usepackage[english]{babel}
\usepackage{amsmath}
\usepackage{epstopdf}
\usepackage{diagbox}
\usepackage{bbm}
\usepackage{dsfont}
\usepackage{bbold}
\usepackage{color,soul}
\usepackage{dcolumn}
\usepackage{soul}
\usepackage{latexsym}
\usepackage{bm}
\DeclareMathOperator\erf{erf}
\usepackage{upgreek}
\raggedbottom
\usepackage{ulem}
\usepackage{listings}
\usepackage{cases}
\usepackage{enumerate}

\renewcommand{\eqref}[1]{\ref{#1}}
\renewcommand{\thetable}{S\arabic{table}}
\renewcommand{\theequation}{S\arabic{equation}}
\renewcommand{\thefigure}{S\arabic{figure}}
\renewcommand{\thepage}{S\arabic{page}}


\definecolor{lg}{gray}{0.93}

\DeclareRobustCommand*{\citen}[1]{%
  \begingroup
    \romannumeral-`\x % remove space at the beginning of \setcitestyle
    \setcitestyle{numbers}%
    \cite{#1}%
  \endgroup
}

\newcommand{\coloredeq}[1]{\begin{empheq}[box=\colorbox{pink!10!white}]{align}#1\end{empheq}}
\bibliographystyle{apsrev4-1}

\begin{document}

\title{Supplementary Information: The First Direct Detection of Kirkwood Transitions in Concentrated Aqueous Electrolytes using Small Angle X-ray Scattering}


\author{Mohammadhasan  Dinpajooh}
\affiliation
{Physical and Computational Sciences Directorate, Pacific Northwest National Laboratory, Richland WA}
\author{Elisa  Biasin}
\affiliation
{Physical and Computational Sciences Directorate, Pacific Northwest National Laboratory, Richland WA}
\author{Christopher J. Mundy}
\affiliation
{Physical and Computational Sciences Directorate, Pacific Northwest National Laboratory, Richland WA}
%\phone{(509) 375-2404}
\affiliation
{Department of Chemical Engineering, University of Washington, Seattle WA}
\author{Gregory K. Schenter}
\affiliation
{Physical and Computational Sciences Directorate, Pacific Northwest National Laboratory, Richland WA}
\author{Emily T. Nienhuis}
\affiliation
{Physical and Computational Sciences Directorate, Pacific Northwest National Laboratory, Richland WA}
\author{Sebastian T. Mergelsberg}
\affiliation
{Physical and Computational Sciences Directorate, Pacific Northwest National Laboratory, Richland WA}
\author{Chris J. Benmore}
\affiliation
{Advanced Photon Source, Argonne National Laboratory, Chicago IL}
\author{John Fulton}
\affiliation
{Physical and Computational Sciences Directorate, Pacific Northwest National Laboratory, Richland WA}
\author{Shawn M. Kathmann}
\affiliation
{Physical and Computational Sciences Directorate, Pacific Northwest National Laboratory, Richland WA}


\date{\today}

\maketitle

\section{Experimental Details}

The SAXS measurements were performed using three different beamlines, 6-ID-D, 12-ID-B and 9-ID-D at the Advanced Photon Source (Argonne National Laboratory, USA) in order to cover a very broad range of Q values ($0.005$ to $3.0$ \AA$^{-1}$) with the best possible S/N using different optical configurations and energies.
Beamline 6-ID-D provided a flux of 5$\times$1011 photons per second at an energy of  100 keV with beamsize of  500 (H)  $\times$ 500 (W) $\mu$m. A Varex CT4343 a-Si area detector was used to cover a range from $0.2 < Q < 3.0$ \AA$^{-1}$.
Beamline 12-ID-B provided a flux of 1012 photons per second at an energy of 13 keV with a beamsize of 200 (H)  $\times$  40 (W) $\mu$m. A 2 M-pixel Pilatus was used to cover a range from $0.03 < Q < 0.9$  \AA$^{-1}$.
Beamline 9-ID-D provided a flux of 1012 photons per second at an energy of 21 keV with a beamsize of 200 (H)  $\times$ 100 (W) $\mu$m beam. This USAXS beamline uses a Bonse-Hart configuration that was set to cover a range from $0.005 < Q < 0.025$  \AA$^{-1}$.\cite{Ilavsky2018}

 Samples were sealed in borosilicate capillaries having nominal diameters of $1$ or $1.5$  mm OD and wall thicknesses of approximately $10$ $\mu$m. The measurements were performed at room temperature. For the SrCl$_2$ and ErCl$_3$ samples, that were run at beamline 12-ID-B, the capillaries were further hand-selected to produce sets of capillaries having diameter variability of  less than $\pm 50$ $\mu$m in order to provide the best possible uniformity for comparisons of the total scattered intensity and for cases requiring background subtraction of pure water spectra. The ErCl$_3$ samples were prepare at pH 1.5 using an HCl stock solution in order to avoid hydrolysis. Scattering patterns were radially integrated using the FIT2D package\cite{Hammersley2016} while other corrections were applied using custom software. The background removal and data treatment follows methods previously described.\cite{Skinner2013,Skinner2016}

A non-linear least square fitting method was used to fit the Teubner-Strey functional form to the experimental spectra using the parameters of $Q_o$, $a_o$, and $\delta$, representing the peak position, the peak width and frequency phase shifts, respectively, as defined within the Teubner-Strey function. A linear background corrections ($m Q + b$) was included in the fitting of the I(Q) signals.


% Figure environment removed


% Figure environment removed

\clearpage


% COMMENT The plan would be to include a few plots in SI of the April and October data with a figure insert of the USAXS data.  This will provide information about the functional form of the salt scattering at lower concentrations and the absence (USAXS) of significant structure at very long distances. (for Q = 0.005 \AA$^{-1}$)

% Another reference for DIT2D:  Hammersley, A. P. FIT2D: An Introduction and Overview; ESRF Internal Report ESRF97HA02T, 1997. 


\section{Computational Details}

\subsection{Molecular dynamics simulations}

The molecular dynamics (MD) simulations were performed using the LAMMPS software program\cite{Plimpton1995} at $298$ K and a pressure of $1$ atm.  
The periodic boundary conditions were applied in 3-dimensions and the MD simulations were performed in the isothermal-isobaric ensemble (NPT) using the Nos$\rm{\acute{e}}$-Hoover thermostat and barostat.
The standard velocity-Verlet time integrator was used with a time step of $2$ fs and the SHAKE procedure was used to conserve the intramolecular constraints.
The Lennard Jones (LJ) and real-space part of the Coulombic interactions were truncated at $10$ \AA\ with additional switching/shifting functions that ramped the energy smoothly to zero between $10$ and $14$ \AA.
The conducting metal (tinfoil) boundary conditions were used to treat the electrostatic interactions and the particle-particle particle-mesh solver was used.
In the initial setup, at least $8000$ water molecules were initially placed in cubic lattices and the ions were then deposited between water molecules using the LAMMPS fix deposit command by avoiding the overlaps between the ions and water molecules considering the periodic boundary conditions.
The initial setups were then equilibrated for about $2-5$ ns and the production MD simulations were performed for $20$ ns, where $50000$ configurations were used to obtain the structural properties such as the radial distribution functions.

The SPC/E water model was used and the force fields for the ions used in this work are presented in Table \ref{pot_param}.
Following the potential energy model developed by Ribeiro for EuCl$_3$,\cite{Ribeiro2006} the current available potential parameters for EuCl$_3$ and ErCl$_3$ \cite{Merz2014} were used to scale the Eu$^{+3}$ LJ  Ribeiro parameters in order to get the Er$^{+3}$ LJ parameters for this work. The SD model\cite{SD-1995} was used for Cl$^{-}$ LJ parameters when used in ErCl$_3$ solutions. 
For SrCl$_2$ and CaCl$_2$, the Kirkwood-Buff force fields (KBFFs) were used.\cite{KBF-NaCl,KBF-M2}

\begin{table}[tbh]
\centering
\caption{Force field (FF) parameters used for the ions and water molecules in the MD and HNC calculations. The Cl$^-$ FF parameters of the Smith-Dang (SD) model was used for the anions in ErCl$_3$ and NaCl electrolyte solutions. The Kirkwood-Buff FF (KBFF) parameters were used for CaCl$_2$ and SrCl$_2$ electrolyte solutions. The Lorentz-Berthelot rule was used for the non-bonded LJ potentials.}
\begin{tabular}{ccccc}
    FF & Site & $\epsilon$(kcal mol$^{-1}$) & $\sigma$(\AA) & $q$(e)  \\
     \hline
 SPC/E   & O & $0.1554$ & $3.166$ & $-0.8476$ \\
 SPC/E   & H & $0$ & $0$ & $+0.4238$ \\
 this work   & Er$^{3+}$ & $2.7127$  & $1.696$  & $+3$   \\
 SD    & Cl$^{-}$ & $0.1000$  & $4.40$ &   $-1$   \\
 SD    & Na$^{+}$ & $0.1300$  & $2.35$ &   $+1$   \\
 KBFF   & Sr$^{2+}$ & $0.1195$  & $3.10$  & $+2$   \\
 KBFF   & Ca$^{2+}$ & $0.1123$  & $2.90$ &   $+2$   \\
 KBFF   & Cl$^{-}$ & $0.1123$  & $4.40$ &  $-1$   \\
     \hline
\end{tabular}
\label{pot_param}
\end{table}

\clearpage
\subsection{Ornstein-Zernike Solver for Electrolyte Solutions}

The ion-ion correlations for the electrolyte solutions, in this work, were obtained by solving the Ornstein-Zernike (OZ) for a mixture of two ions and treating the water as a continuum with the dielectric constant of $80$ ($\epsilon_w = 80$). This, in turn, can significantly simplify the OZ equations for electrolyte solutions to get the ion-ion correlations, but requires accurate effective ion-ion interactions. The effective interaction potentials between the ions (mean-field interactions) were obtained directly from the MD simulations by calculating
the potential of mean forces (PMFs) between the ions in the dilute limit directly  of two ions in the explicit SPC/E water molecules (see Table \ref{pot_param}).
However, the long-range parts of the effective interactions cannot be extracted from MD simulations due to the limitations/uncertainties in the MD simulations. Therefore, it is assumed that the MD effective interactions can be split to the short-range, $u^{\mathrm{SR}}_{ij}(r)$, and long-range electrostatics contributions, $u^{\mathrm{LR}}_{ij}(r) $, as 
\begin{equation}
W^{\mathrm{MD}}_{ij}(r) = u^{\mathrm{SR}}_{ij}(r) + u^{\mathrm{LR}}_{ij}(r) 
\end{equation}

Interestingly, such a decomposition is very useful for solving the OZ equation numerically because once the short-range effective interactions between the ions are determined from the MD simulations, one can reasonably approximate the effective long-range electrostatics interactions between ions and improve the convergence.\cite{Ng-1974,Warren2013} In this work, we assumed that the short-range effective interactions between the ions can be extracted from MD simulations by simply subtracting the $z_i z_j/(\epsilon_w r)$ terms corresponding to ions with charges of $z_i$ and $z_j$    
from the full effective interactions between the ions from MD simulations (MD PMFs, $W^{\mathrm{MD}}_{ij}$):
\begin{equation}
u^{\mathrm{SR}}_{ij}(r) = W^{\mathrm{MD}}_{ij}(r) - \frac{z_i z_j}{\epsilon_w r}
\label{sr-u}
\end{equation}
where $W^{\mathrm{MD}}_{ij}(r)$ shows the asymptotic behaviors of the long-range electrostatics interactions; therefore, the raw MD PMFs obtained either from the thermodynamics integration or WHAM methods were shifted vertically to match $z_i z_j/(\epsilon_w r)$ at a reasonable $r$.  
It would be an interesting future project to extract $W^{\mathrm{MD}}_{ij}(r)$ from other methods. 


The OZ equations were then solved using the hyper-netted chain (HNC) closure.
We used a modified version of SunlightHNC code (PNNL-SunlightHNC),\cite{Warren2013}
where the inputs to the PNNL-SunlightHNC code were the short-range effective interactions between ions (see Eq. \ref{sr-u}), ion charges, and ion concentrations.
The long-range electrostatic interactions in the PNNL-SunlightHNC code were treated as

\begin{equation}
u^{\mathrm{LR}}_{ij}(r) = k_{\mathrm{B}}T z_i z_j l_\mathrm{B} \frac{\erf \left[ r/(2\sigma) \right]}{r}
\end{equation}
where $l_\mathrm{B}$ is the Bjerrum length setting the magnitude of the long-range electrostatics interaction at a given temperature $T$ ($l_\mathrm{B} \approx 7$ \AA\ for water at room temperature), $ k_{\mathrm{B}}$ is the Bolzmann constant, $\sigma$ is the size (width) of the Gaussian charges. 
A $\sigma$ value of $1$ \AA\ was used in the calculations; therefore, at distances above 4 \AA, $u^{\mathrm{LR}}_{ij}(r) \approx  z_i z_j/(\epsilon r)$, which justifies the use of Eq. \ref{sr-u} to extract the short-range mean-field ion-ion potentials for the HNC code.
We also checked that consistent results were obtained using a $\sigma$ value of $0.2$ \AA. 


The assumptions we made, in the above procedures, were verified by directly comparing the results from  the OZ solver using the HNC closure (hereafter HNC results) and the MD results. For instance, Fig. \ref{compare-md-hnc} shows excellent agreements between MD and HNC results for the ErEr radial distribution functions at various concentrations. 




% Figure environment removed


\section{Definitions of static structure factors}

Consider a solution consisting of $n_i$ atoms for each species with a total number of $n$ atoms and an atomic number density of $\rho=n/V$.

\begin{equation}
S_{ij}(Q) = c_i \delta_{ij} + c_i c_j \rho \hat{h}_{ij}(Q)
\label{Sij}
\end{equation}
where $c_i=n_i/n$ is the atomic fraction of species $i$.

\begin{equation}
S_{NN}(Q) = \frac{1}{N} \langle \rho^N_{\mathbf{Q}} \rho^N_{\mathbf{-Q}} \rangle = \sum_i \sum_j S_{ij}(Q)
\label{SNN}
\end{equation}

\begin{equation}
S_{zz}(Q) = \frac{1}{N} \langle \rho^z_{\mathbf{Q}} \rho^z_{\mathbf{-Q}} \rangle = \sum_i \sum_j z_i z_j S_{ij}(Q)
\label{Szz}
\end{equation}
where $z_i$ is the charge for species $i$.

Making use of $S_{zz}(Q)$, one can define
\begin{equation}
\epsilon(Q) = \frac{1}{1-\frac{k^2_{\rm D} S_{zz}(Q)}{\zeta Q^2} }
\label{chi}
\end{equation}

\begin{equation}
\chi_\mathrm{L}(Q) = 1 - \frac{1}{\epsilon(Q)}
\label{chi}
\end{equation}

with $ k_{\rm D}^2 = 4 \pi \rho l_{\mathrm{B}} \zeta$ and $\zeta = \sum c_i z_i^2 $, $l_{\mathrm{B}} = e^2/(\epsilon k_{\mathrm{B}} T)$.


The measured X-ray intensity $I(Q)$ is related to the structure factor $S_{X}(Q)$
by
\begin{equation}
I(Q)\approx S_{X}(Q)-1=\rho \sum_{i,j}c_{i}c_{j}f_{ij}(Q)\hat{h}_{ij}(Q)
\end{equation}
where
\begin{equation}
f_{ij}(Q)=\frac{f_{i}(Q)f_{j}(Q)}{\left[{\displaystyle \sum_{i=1}^{N}}c_{i}f_{i}(Q)\right]^{2}}
\end{equation}
where $f_{i}(Q)$ is the X-ray form
factor for species $i$, and $N$ is the total number of atomic species in the system.

One may approximate the measured X-ray intensity by assuming that the form factors are constant and equal to the atomic number as 

\begin{equation}
S_{ZZ}(Q)= 
1 + \rho f \sum_{i,j} Z_{i}Z_{j}c_{i}c_{j}\hat{h}_{ij}(Q)
\label{SZZ_X}
\end{equation}
where
\begin{equation}
f=\frac{1}{\left[ \sum_i Z_i c_i  \right]^{2}}
\end{equation}
with $Z_i$ is the the atomic number for species $i$.



\section{HNC Root Structures for the ErCl$_3$ Electrolytes: $\chi(Q)$}

Following the discussions in the main text, we use the following functions to extract the roots structures from $\chi(Q)$:

\begin{equation}
 \chi_\mathrm{L}(Q) = \frac{\kappa_x^2}{Q^2 + \kappa_x^2 - Q^2 \tilde{\Sigma}(Q)}
 \label{gen_chi}
\end{equation}

For the auxiliary field model, we used 
\begin{equation}
\tilde{\Sigma}^\mathrm{Aux}(Q)= \frac{A^2}{1+g(Q)} 
 \label{aux_f}
\end{equation}
with $\displaystyle g(Q)=\sum_{i=1}^{4} b_i Q^{2i}$ and $\kappa_x = \kappa_s$, where the coefficients $A$, $b_i$ and $\kappa_s$ are determined during the analyses.


For the Pad$\acute{\mathrm{e}}$ function, we chose 
\begin{equation}
 \tilde{\Sigma}^\mathrm{Pad\acute{e}}(Q)= \frac{- (a + bQ^{2})}{1+cQ^{2}+dQ^{4}} 
 \label{pade_f}
\end{equation}
where $\kappa_x=\kappa_\mathrm{D}$ and $a$, $b$, $c$, and $d$ coefficients are determined during the analyses.

The root structures are extracted by first fitting $\chi(Q)$ in the low $Q$ region (up to about $2.5$ \AA$^{-1})$ to functions of type Eq. \ref{gen_chi} for each functional. The root structures are then obtained by finding the roots of the aforementioned functions making use of Muller's method in {\it mpmath} Python library, which is recommended for complex roots. In short, it starts with three initial assumptions of the root, and then constructing a parabola through these three points, and taking the intersection of the x-axis with the parabola to be the next approximation. This process continues until a root with the desired level of accuracy is found.
We use the initial guesses from the total correlation functions of $h_{++}(r)$ or $h_{+-}(r)$.

We also obtain the root structures for $\chi(Q)$ using the Teubner-Stey (TS) function presented below noting that the auxiliary field model and the Pad$\acute{\mathrm{e}}$ functions are generally able to capture the $\chi(Q)$ behavior in a wider range of $Q$-space.
\begin{equation}
 \chi^{\mathrm{TS}}_\mathrm{L}(Q)= \frac{4\pi A\left[\left(a_0^{2}-Q_0^{2}+Q^{2}\right)\cos[\delta]+2a_0 Q_o\sin[\delta]\right]}{\left[a_o^{2}+Q_o^{2}\right]^{2}+2\left[a_o^{2}-Q_o^{2}\right]Q^{2}+Q^{4}} + m Q + b
 \label{TS_f}
\end{equation}




% Figure environment removed


As can be seen in Figs. \ref{ercl3_root} and \ref{ercl3_root_busy}, for ErCl$_3$ all the root structures from the aforementioned functionals agree reasonably well. However, the auxiliary functional and Pad\'e functional can give the correct root structures over a larger region of $Q$-space than the ones from the TS functional.  

% Figure environment removed



\clearpage
\section{Complex Root Structure for Various Hierarchies}

As discussed above and in the main text, we observe that the root structures for ErCl$_3$ significantly deviate from the traditional Kirkwood transition (KT). For SrCl$_2$ and CaCl$_2$, we observe slight deviations from the traditional KT while for NaCl the traditional picture almost holds. Below, we present the root structures from various hierarchies showing that they all show the root structure behaviors consistent with the traditional KT.  

% Figure environment removed


\clearpage



\clearpage
\section{Related Fitting Procedures}

In the main text and the Supplementary Information, several signals are fitted to extract the $a_0$ and $Q_0$ values. In this Section, the fitting procedure is discussed in more details.

The X-ray signals, $I(Q)$, are fitted to TS distributions with linear background according to
\begin{equation}
%\begin{split}
I(Q) = 
\frac{4\pi A_0\left[\left(a_0^{2}-Q_o^{2}+Q^{2}\right)\cos[\delta]+2a_o Q_0\sin[\delta]\right]}{\left[a_o^{2}+Q_o^{2}\right]^{2}+2\left[a_o^{2}-Q_o^{2}\right]Q^{2}+Q^{4}} + m Q + b
%\end{split}
\end{equation}
where $m$ and $n$ are the coefficients of the linear background and it is assumed that X-ray signal can be described by only one phase, $\delta$.

In $r$-space, the total correlation functions, $h(r)$, for various species are fitted to the damped oscillator functions of
\begin{equation}
h(r) = A_0 e^{-a_or}\cos(Q_o r-\delta)/r
\end{equation}


In HNC or MD calculations, one can get the response electrostatic potential as described in the main text. 
One may also extract the $a_0$ and $Q_0$ from the response electrostatic potentials according to
\begin{equation}
%\begin{split}
\phi_{j}^{\mathrm{resp}}(r)=-\frac{z_{j}e}{\varepsilon r}e^{-a_0r}
\left[\frac{\left(a_0^{2}-Q_0^{2}\right)\sin[Q_0r-\delta_{s}]+2a_0Q_0\cos[Q_0r-\delta_{s}]}{\left(a_0^{2}-Q_0^{2}\right)\sin[\delta_{s}]-2a_0Q_0\cos[\delta_{s}]}\right]
-\frac{z_{j}e}{\varepsilon r}
\label{eqn_resp}
%\end{split}
\end{equation}
where $\delta_{s}=(\delta_{n}+\delta_{m})/2$ and $n$ and $m=$ 1 and 2 for cations and 2 and 3 for anions, respectively.

In addition, one can calculate the ionic atmosphere around a given ion $i$ in HNC or MD simulations via

\begin{equation}
S_0(r) = 4 \pi \int_0^r \rho_i(r') r'^2 dr' 
\label{atm_ion}
\end{equation}
where $\rho_j(r')$ is the charge density around the ion. 

Similarly, one may extract the $a_0$ and $Q_0$ values from $S_0(r)$ according to

\begin{equation}
%\begin{split}
S_0(r)=-z_{j}e[a\,+e^{-a_0r}\left(b\,\cos[Q_0r-\delta_s]+c\,\sin[Q_0r-\delta_s]\right)]
\label{eqn_S0}
%\end{split}
\end{equation}
where
\begin{equation}
\begin{split}
&a=\frac{\left(a_0^{2}-Q_0^{2}\right)\sin[\delta_{s}]-2a_0Q_0\cos[\delta_{s}]}{\left(a_0^{2}-Q_0^{2}\right)\sin[\delta_{s}]+2a_0Q_0\cos[\delta_{s}]}\\
&b=\frac{2a_0Q_0+Q_0(a_0^{2}+Q_0^{2})r}{\left(a_0^{2}-Q_0^{2}\right)\sin[\delta_{s}]+2a_0Q_0\cos[\delta_{s}]}\\
&c=\frac{a_0^2-Q_0^2+a_0(a_0^{2}+Q_0^{2})r}{\left(a_0^{2}-Q_0^{2}\right)\sin[\delta_{s}]+2a_0Q_0\cos[\delta_{s}]}.
\end{split}
\end{equation}

Table \ref{fit_param1} presents the fitting parameters obtained from the experiment, MD, and HNC calculations using the above procedures for ErCl$_3$ at 3m. As can be seen, the $a_0$ and $Q_0$ values obtained from $I(Q)$ and $h_{++}(r)$ routes are in excellent agreement. 
Less agreements are observed when one uses $\phi_+^{\mathrm{resp}}(r)$ and $S_o(r)$ to extract $a_0$
 and $Q_0$ values; however, considering the uncertainties in the fitting procedures, they are still in good agreement. 
 Figures \ref{fig:Sample_phi} and \ref{fig:sample_KBI} show how well the fitting functions match the MD or HNC data.
 
\begin{table}[tbh]
\centering
\caption{Fitting parameters for the analytic functions provided above for $I(Q)$, $h_{++}(r)$, $\phi_+^{\mathrm{resp}}(r)$, and $S_o(r)$ from the experimental signals of ErCl$_3$ as well as MD and HNC calculations at $c=3\,m$. Linear background corrections were included in the estimations of $I(Q)$ and $\phi_+^{\mathrm{resp}}(r)$. To report $a_o$ and $Q_o$ in the main text, only $I(Q)$ and $h_{++}(r)$ functions were used.}
\vspace{0.3in}

\begin{tabular}{|c|c|cccccc|}
    \hline 
Method  &  Function & \multicolumn{6}{c|}{Fitting Parameters}\\
\hline 
&  $I(Q)$  & $a_o$[\AA$^{-1}$] & $Q_o$[\AA$^{-1}$] & $\delta$[rad] & $A$[\AA$^{-2}$] & $m$[\AA] & $b$  \\
     \hline
Expt &   & $0.162$ & $0.758$ & $0.95$ & $1.34$ & $14.20$ & $27.49$   \\
MD &   & $0.149$ & $0.805$ & $0.98$ & $-1.34$ & $9.40$ & $-46.58$   \\
\hline
&  $h_{++}(r)$  & $a_o$[\AA$^{-1}$] & $Q_o$[\AA$^{-1}$] & $\delta_{++}$[rad] & $A_0$[\AA] &  &  \\
     \hline
MD &   & $0.150$ & $0.824$ & $1.10$ & $17.13$ &  & \\
HNC  &   & $0.151$ & $0.782$ & $0.56$ & $24.48$ &  & \\
\hline
&  $\phi_+^{\mathrm{resp}}(r)$  & $a_o$[\AA$^{-1}$] & $Q_o$[\AA$^{-1}$] & $\delta_n$[rad] & $\delta_m$[rad] & $M$[V/\AA] &  $B$[V]  \\
     \hline
MD &    & $0.222$ & $0.767$ & $0.62$ & $2.09$ & $-1.0\times10^{-3}$ & $-8.5\times10^{-3}$ \\
HNC &    & $0.197$ & $0.757$ & $0.29$ & $2.84$ & $3.8\times10^{-6}$ & $-5.1\times10^{-5}$   \\
\hline 
&  $S_0(r)$   & $a_o$[\AA$^{-1}$] & $Q_o$[\AA$^{-1}$] & $\delta_s$[rad] &  & &  \\
\hline
MD &   & $0.190$ & $0.757$ & $1.58$ &  &  &  \\
HNC &   & $0.164$ & $0.738$ & $1.57$ &  &  & \\
     \hline
\end{tabular}
\label{fit_param1}
\end{table}

% Figure environment removed


% Figure environment removed

\clearpage
\section{Screening Lenghts and Periodicity}

\subsection{Ion-Ion vs. Water-Ion \& Water-Water Correlations}

Figure \ref{ion-water} compares the ion-ion correlation behaviors with ion-water and water-water correlations in ErCl$_3$ electrolyte solutions at relatively low and high concentrations. The bottom panel of Fig. \ref{ion-water} shows that at a high ion concentration of $3$ m, the length scales associated with periodicity are almost consistent for ion-ion, ion-water, and water-water correlation. Nevertheless, the top panel of Fig. \ref{ion-water} at a lower ion concentration of $0.5$ m, such length scales for ion-ion correlations tend to be about twice as much as the ion-water and water-water correlations. 

% Figure environment removed

\subsection{SrCl$_2$ \& CaCl$_2$}

Following the discussions in the main text, the $\lambda$ and $d$ values for SrCl$_2$ and CaCl$_2$ from various approaches are presented in Figs. \ref{fig:SrCl2_aoQo} and \ref{cacl2}.

% Figure environment removed


% Figure environment removed




\clearpage
\section{On the Convergence of MD simulations}

Figure \ref{erer-ln_rhr_md-hnc} shows the convergence of $r|h(r)|$ for ErEr from MD simulations at concentrations equal or higher than $1$ m. As can be seen, as the concentration increases, the noise in the tails of $r|h(r)|$ decrease.

% Figure environment removed


Figure \ref{compare-md-kbi} shows the convergence of $S_0(r)$ (Eq. \ref{atm_ion}) from MD simulations for various concentrations of ErCl$_3$. As can be seen, the $S_0(r)$ at various distances oscillate around $-3$ e, but due to finite size effects in MD simulations there are deviations. 

% Figure environment removed



\clearpage



%merlin.mbs apsrev4-1.bst 2010-07-25 4.21a (PWD, AO, DPC) hacked
%Control: key (0)
%Control: author (72) initials jnrlst
%Control: editor formatted (1) identically to author
%Control: production of article title (-1) disabled
%Control: page (0) single
%Control: year (1) truncated
%Control: production of eprint (0) enabled
\begin{thebibliography}{20}%
\makeatletter
\providecommand \@ifxundefined [1]{%
 \@ifx{#1\undefined}
}%
\providecommand \@ifnum [1]{%
 \ifnum #1\expandafter \@firstoftwo
 \else \expandafter \@secondoftwo
 \fi
}%
\providecommand \@ifx [1]{%
 \ifx #1\expandafter \@firstoftwo
 \else \expandafter \@secondoftwo
 \fi
}%
\providecommand \natexlab [1]{#1}%
\providecommand \enquote  [1]{``#1''}%
\providecommand \bibnamefont  [1]{#1}%
\providecommand \bibfnamefont [1]{#1}%
\providecommand \citenamefont [1]{#1}%
\providecommand \href@noop [0]{\@secondoftwo}%
\providecommand \href [0]{\begingroup \@sanitize@url \@href}%
\providecommand \@href[1]{\@@startlink{#1}\@@href}%
\providecommand \@@href[1]{\endgroup#1\@@endlink}%
\providecommand \@sanitize@url [0]{\catcode `\\12\catcode `\$12\catcode
  `\&12\catcode `\#12\catcode `\^12\catcode `\_12\catcode `\%12\relax}%
\providecommand \@@startlink[1]{}%
\providecommand \@@endlink[0]{}%
\providecommand \url  [0]{\begingroup\@sanitize@url \@url }%
\providecommand \@url [1]{\endgroup\@href {#1}{\urlprefix }}%
\providecommand \urlprefix  [0]{URL }%
\providecommand \Eprint [0]{\href }%
\providecommand \doibase [0]{http://dx.doi.org/}%
\providecommand \selectlanguage [0]{\@gobble}%
\providecommand \bibinfo  [0]{\@secondoftwo}%
\providecommand \bibfield  [0]{\@secondoftwo}%
\providecommand \translation [1]{[#1]}%
\providecommand \BibitemOpen [0]{}%
\providecommand \bibitemStop [0]{}%
\providecommand \bibitemNoStop [0]{.\EOS\space}%
\providecommand \EOS [0]{\spacefactor3000\relax}%
\providecommand \BibitemShut  [1]{\csname bibitem#1\endcsname}%
\let\auto@bib@innerbib\@empty
%</preamble>
\bibitem [{\citenamefont {Ilavsky}\ \emph {et~al.}(2018)\citenamefont
  {Ilavsky}, \citenamefont {Zhang}, \citenamefont {Andrews}, \citenamefont
  {Kuzmenko}, \citenamefont {Jemian}, \citenamefont {Levine},\ and\
  \citenamefont {Allen}}]{Ilavsky2018}%
  \BibitemOpen
  \bibfield  {author} {\bibinfo {author} {\bibfnamefont {J.}~\bibnamefont
  {Ilavsky}}, \bibinfo {author} {\bibfnamefont {F.}~\bibnamefont {Zhang}},
  \bibinfo {author} {\bibfnamefont {R.~N.}\ \bibnamefont {Andrews}}, \bibinfo
  {author} {\bibfnamefont {I.}~\bibnamefont {Kuzmenko}}, \bibinfo {author}
  {\bibfnamefont {P.~R.}\ \bibnamefont {Jemian}}, \bibinfo {author}
  {\bibfnamefont {L.~E.}\ \bibnamefont {Levine}}, \ and\ \bibinfo {author}
  {\bibfnamefont {A.~J.}\ \bibnamefont {Allen}},\ }\href {\doibase
  10.1107/S160057671800643X} {\bibfield  {journal} {\bibinfo  {journal} {J.
  Appl. Crystallogr.}\ }\textbf {\bibinfo {volume} {51}},\ \bibinfo {pages}
  {867} (\bibinfo {year} {2018})}\BibitemShut {NoStop}%
\bibitem [{\citenamefont {Hammersley}(2016)}]{Hammersley2016}%
  \BibitemOpen
  \bibfield  {author} {\bibinfo {author} {\bibfnamefont {A.~P.}\ \bibnamefont
  {Hammersley}},\ }\href {\doibase 10.1107/S1600576716000455} {\bibfield
  {journal} {\bibinfo  {journal} {J. Appl. Crystallogr.}\ }\textbf {\bibinfo
  {volume} {49}},\ \bibinfo {pages} {646} (\bibinfo {year} {2016})}\BibitemShut
  {NoStop}%
\bibitem [{\citenamefont {Skinner}\ \emph {et~al.}(2013)\citenamefont
  {Skinner}, \citenamefont {Huang}, \citenamefont {Schlesinger}, \citenamefont
  {Pettersson}, \citenamefont {Nilsson},\ and\ \citenamefont
  {Benmore}}]{Skinner2013}%
  \BibitemOpen
  \bibfield  {author} {\bibinfo {author} {\bibfnamefont {L.~B.}\ \bibnamefont
  {Skinner}}, \bibinfo {author} {\bibfnamefont {C.}~\bibnamefont {Huang}},
  \bibinfo {author} {\bibfnamefont {D.}~\bibnamefont {Schlesinger}}, \bibinfo
  {author} {\bibfnamefont {L.~G.~M.}\ \bibnamefont {Pettersson}}, \bibinfo
  {author} {\bibfnamefont {A.}~\bibnamefont {Nilsson}}, \ and\ \bibinfo
  {author} {\bibfnamefont {C.~J.}\ \bibnamefont {Benmore}},\ }\href {\doibase
  10.1063/1.4790861} {\bibfield  {journal} {\bibinfo  {journal} {J. Chem.
  Phys.}\ }\textbf {\bibinfo {volume} {138}},\ \bibinfo {pages} {074506}
  (\bibinfo {year} {2013})}\BibitemShut {NoStop}%
\bibitem [{\citenamefont {Skinner}\ \emph {et~al.}(2016)\citenamefont
  {Skinner}, \citenamefont {Galib}, \citenamefont {Fulton}, \citenamefont
  {Mundy}, \citenamefont {Parise}, \citenamefont {Pham}, \citenamefont
  {Schenter},\ and\ \citenamefont {Benmore}}]{Skinner2016}%
  \BibitemOpen
  \bibfield  {author} {\bibinfo {author} {\bibfnamefont {L.~B.}\ \bibnamefont
  {Skinner}}, \bibinfo {author} {\bibfnamefont {M.}~\bibnamefont {Galib}},
  \bibinfo {author} {\bibfnamefont {J.~L.}\ \bibnamefont {Fulton}}, \bibinfo
  {author} {\bibfnamefont {C.~J.}\ \bibnamefont {Mundy}}, \bibinfo {author}
  {\bibfnamefont {J.~B.}\ \bibnamefont {Parise}}, \bibinfo {author}
  {\bibfnamefont {V.-T.}\ \bibnamefont {Pham}}, \bibinfo {author}
  {\bibfnamefont {G.~K.}\ \bibnamefont {Schenter}}, \ and\ \bibinfo {author}
  {\bibfnamefont {C.~J.}\ \bibnamefont {Benmore}},\ }\href {\doibase
  10.1063/1.4944935} {\bibfield  {journal} {\bibinfo  {journal} {J. Chem.
  Phys.}\ }\textbf {\bibinfo {volume} {144}},\ \bibinfo {pages} {134504}
  (\bibinfo {year} {2016})}\BibitemShut {NoStop}%
\bibitem [{\citenamefont {Plimpton}(1995)}]{Plimpton1995}%
  \BibitemOpen
  \bibfield  {author} {\bibinfo {author} {\bibfnamefont {S.}~\bibnamefont
  {Plimpton}},\ }\href {\doibase 10.1006/jcph.1995.1039} {\bibfield  {journal}
  {\bibinfo  {journal} {J. Comput. Phys.}\ }\textbf {\bibinfo {volume} {117}},\
  \bibinfo {pages} {1 } (\bibinfo {year} {1995})}\BibitemShut {NoStop}%
\bibitem [{\citenamefont {Ribeiro}(2006)}]{Ribeiro2006}%
  \BibitemOpen
  \bibfield  {author} {\bibinfo {author} {\bibfnamefont {M.~C.~C.}\
  \bibnamefont {Ribeiro}},\ }\href {\doibase 10.1103/PhysRevB.73.014201}
  {\bibfield  {journal} {\bibinfo  {journal} {Phys. Rev. B}\ }\textbf {\bibinfo
  {volume} {73}},\ \bibinfo {pages} {014201} (\bibinfo {year}
  {2006})}\BibitemShut {NoStop}%
\bibitem [{\citenamefont {Li}\ \emph {et~al.}(2015)\citenamefont {Li},
  \citenamefont {Song},\ and\ \citenamefont {Merz}}]{Merz2014}%
  \BibitemOpen
  \bibfield  {author} {\bibinfo {author} {\bibfnamefont {P.}~\bibnamefont
  {Li}}, \bibinfo {author} {\bibfnamefont {L.~F.}\ \bibnamefont {Song}}, \ and\
  \bibinfo {author} {\bibfnamefont {K.~M.~J.}\ \bibnamefont {Merz}},\ }\href
  {\doibase 10.1021/jp505875v} {\bibfield  {journal} {\bibinfo  {journal} {J.
  Phys. Chem. B}\ }\textbf {\bibinfo {volume} {119}},\ \bibinfo {pages} {883}
  (\bibinfo {year} {2015})}\BibitemShut {NoStop}%
\bibitem [{\citenamefont {Dang}\ and\ \citenamefont {Smith}(1995)}]{SD-1995}%
  \BibitemOpen
  \bibfield  {author} {\bibinfo {author} {\bibfnamefont {L.~X.}\ \bibnamefont
  {Dang}}\ and\ \bibinfo {author} {\bibfnamefont {D.~E.}\ \bibnamefont
  {Smith}},\ }\href {\doibase 10.1063/1.468572} {\bibfield  {journal} {\bibinfo
   {journal} {J. Chem. Phys.}\ }\textbf {\bibinfo {volume} {102}},\ \bibinfo
  {pages} {3483} (\bibinfo {year} {1995})}\BibitemShut {NoStop}%
\bibitem [{\citenamefont {Weerasinghe}\ and\ \citenamefont
  {Smith}(2003)}]{KBF-NaCl}%
  \BibitemOpen
  \bibfield  {author} {\bibinfo {author} {\bibfnamefont {S.}~\bibnamefont
  {Weerasinghe}}\ and\ \bibinfo {author} {\bibfnamefont {P.~E.}\ \bibnamefont
  {Smith}},\ }\href {\doibase 10.1063/1.1622372} {\bibfield  {journal}
  {\bibinfo  {journal} {J. Chem. Phys.}\ }\textbf {\bibinfo {volume} {119}},\
  \bibinfo {pages} {11342} (\bibinfo {year} {2003})}\BibitemShut {NoStop}%
\bibitem [{\citenamefont {Naleem}\ \emph {et~al.}(2018)\citenamefont {Naleem},
  \citenamefont {Bentenitis},\ and\ \citenamefont {Smith}}]{KBF-M2}%
  \BibitemOpen
  \bibfield  {author} {\bibinfo {author} {\bibfnamefont {N.}~\bibnamefont
  {Naleem}}, \bibinfo {author} {\bibfnamefont {N.}~\bibnamefont {Bentenitis}},
  \ and\ \bibinfo {author} {\bibfnamefont {P.~E.}\ \bibnamefont {Smith}},\
  }\href {\doibase 10.1063/1.5019454} {\bibfield  {journal} {\bibinfo
  {journal} {J. Chem. Phys.}\ }\textbf {\bibinfo {volume} {148}},\ \bibinfo
  {pages} {222828} (\bibinfo {year} {2018})}\BibitemShut {NoStop}%
\bibitem [{\citenamefont {Ng}(2003)}]{Ng-1974}%
  \BibitemOpen
  \bibfield  {author} {\bibinfo {author} {\bibfnamefont {K.}~\bibnamefont
  {Ng}},\ }\href {\doibase 10.1063/1.1682399} {\bibfield  {journal} {\bibinfo
  {journal} {J. Chem. Phys.}\ }\textbf {\bibinfo {volume} {61}},\ \bibinfo
  {pages} {2680} (\bibinfo {year} {2003})}\BibitemShut {NoStop}%
\bibitem [{\citenamefont {Warren}\ \emph {et~al.}(2013)\citenamefont {Warren},
  \citenamefont {Vlasov}, \citenamefont {Anton},\ and\ \citenamefont
  {Masters}}]{Warren2013}%
  \BibitemOpen
  \bibfield  {author} {\bibinfo {author} {\bibfnamefont {P.~B.}\ \bibnamefont
  {Warren}}, \bibinfo {author} {\bibfnamefont {A.}~\bibnamefont {Vlasov}},
  \bibinfo {author} {\bibfnamefont {L.}~\bibnamefont {Anton}}, \ and\ \bibinfo
  {author} {\bibfnamefont {A.~J.}\ \bibnamefont {Masters}},\ }\href {\doibase
  10.1063/1.4807057} {\bibfield  {journal} {\bibinfo  {journal} {J. Chem.
  Phys.}\ }\textbf {\bibinfo {volume} {138}},\ \bibinfo {pages} {204907}
  (\bibinfo {year} {2013})}\BibitemShut {NoStop}%
\bibitem [{\citenamefont {Kirkwood}(1936)}]{Kirkwood1936}%
  \BibitemOpen
  \bibfield  {author} {\bibinfo {author} {\bibfnamefont {J.~G.}\ \bibnamefont
  {Kirkwood}},\ }\href@noop {} {\bibfield  {journal} {\bibinfo  {journal}
  {Chem. Rev.}\ }\textbf {\bibinfo {volume} {19}},\ \bibinfo {pages} {275}
  (\bibinfo {year} {1936})}\BibitemShut {NoStop}%
\bibitem [{\citenamefont {Kirkwood}\ and\ \citenamefont
  {Poirier}(1954)}]{Kirkwood-1954}%
  \BibitemOpen
  \bibfield  {author} {\bibinfo {author} {\bibfnamefont {J.~G.}\ \bibnamefont
  {Kirkwood}}\ and\ \bibinfo {author} {\bibfnamefont {J.~C.}\ \bibnamefont
  {Poirier}},\ }\href {\doibase 10.1021/j150518a004} {\bibfield  {journal}
  {\bibinfo  {journal} {J. Phys. Chem.}\ }\textbf {\bibinfo {volume} {58}},\
  \bibinfo {pages} {591} (\bibinfo {year} {1954})}\BibitemShut {NoStop}%
\bibitem [{\citenamefont {Singer}(1975)}]{Singer1975}%
  \BibitemOpen
  \bibfield  {author} {\bibinfo {author} {\bibfnamefont {K.}~\bibnamefont
  {Singer}},\ }in\ \href@noop {} {\emph {\bibinfo {booktitle} {{S}tatistical
  {M}echanics: {V}olume 2}}}\ (\bibinfo  {publisher} {The Royal Society of
  Chemistry},\ \bibinfo {year} {1975})\ Chap.\ \bibinfo {chapter} {3.
  Equilibrium Theory of Electrolyte Solutions by C. W. Outhwaite}, pp.\
  \bibinfo {pages} {188--255}\BibitemShut {NoStop}%
\bibitem [{\citenamefont {Lee}\ and\ \citenamefont
  {Fisher}(1997)}]{LeeFisher1997}%
  \BibitemOpen
  \bibfield  {author} {\bibinfo {author} {\bibfnamefont {B.~P.}\ \bibnamefont
  {Lee}}\ and\ \bibinfo {author} {\bibfnamefont {M.~E.}\ \bibnamefont
  {Fisher}},\ }\href@noop {} {\bibfield  {journal} {\bibinfo  {journal}
  {Europhys. Lett.}\ }\textbf {\bibinfo {volume} {39}},\ \bibinfo {pages} {611}
  (\bibinfo {year} {1997})}\BibitemShut {NoStop}%
\bibitem [{\citenamefont {Kjellander}(1995)}]{Kjellander1995}%
  \BibitemOpen
  \bibfield  {author} {\bibinfo {author} {\bibfnamefont {R.}~\bibnamefont
  {Kjellander}},\ }\href@noop {} {\bibfield  {journal} {\bibinfo  {journal} {J.
  Phys. Chem.}\ }\textbf {\bibinfo {volume} {99}},\ \bibinfo {pages} {10392}
  (\bibinfo {year} {1995})}\BibitemShut {NoStop}%
\bibitem [{\citenamefont {Kjellander}(2019)}]{Kjellander2019}%
  \BibitemOpen
  \bibfield  {author} {\bibinfo {author} {\bibfnamefont {R.}~\bibnamefont
  {Kjellander}},\ }\href@noop {} {\bibfield  {journal} {\bibinfo  {journal}
  {Soft Matter}\ }\textbf {\bibinfo {volume} {15}},\ \bibinfo {pages} {5866}
  (\bibinfo {year} {2019})}\BibitemShut {NoStop}%
\bibitem [{\citenamefont {Adar}\ \emph {et~al.}(2019)\citenamefont {Adar},
  \citenamefont {Safran}, \citenamefont {Diamant},\ and\ \citenamefont
  {Andelman}}]{Adar2019}%
  \BibitemOpen
  \bibfield  {author} {\bibinfo {author} {\bibfnamefont {R.~M.}\ \bibnamefont
  {Adar}}, \bibinfo {author} {\bibfnamefont {S.~A.}\ \bibnamefont {Safran}},
  \bibinfo {author} {\bibfnamefont {H.}~\bibnamefont {Diamant}}, \ and\
  \bibinfo {author} {\bibfnamefont {D.}~\bibnamefont {Andelman}},\ }\href@noop
  {} {\bibfield  {journal} {\bibinfo  {journal} {Phys. Rev. E}\ }\textbf
  {\bibinfo {volume} {100}},\ \bibinfo {pages} {042615} (\bibinfo {year}
  {2019})}\BibitemShut {NoStop}%
\bibitem [{\citenamefont {Outhwaite}(1974)}]{Outhwaite-1974}%
  \BibitemOpen
  \bibfield  {author} {\bibinfo {author} {\bibfnamefont {C.}~\bibnamefont
  {Outhwaite}},\ }\href {\doibase 10.1080/00268977400101651} {\bibfield
  {journal} {\bibinfo  {journal} {Mol. Phys.}\ }\textbf {\bibinfo {volume}
  {28}},\ \bibinfo {pages} {217} (\bibinfo {year} {1974})}\BibitemShut
  {NoStop}%
\end{thebibliography}%

\end{document}
