\documentclass[aps,prl,twocolumn,showpacs,amsmath,amssymb]{revtex4-2}
\usepackage{amsmath}
\usepackage{graphicx}
\usepackage{subfigure}
\usepackage{epstopdf}
\usepackage{color}
\usepackage{multirow}
\usepackage{setspace}
\usepackage{overpic}
\usepackage{amssymb}
\usepackage[bookmarksnumbered, pdfstartview=FitH,colorlinks,urlcolor=blue, citecolor=blue,linkcolor=blue] {hyperref}
\usepackage{lineno}
\usepackage{bm}
\usepackage{rotating}
\usepackage[utf8]{inputenc}

\hyphenpenalty=5000 \tolerance=100

\setlength{\oddsidemargin}{-0.5cm}
\setlength{\oddsidemargin}{-0.5cm} \addtolength{\topmargin}{10mm}
\hoffset -0.2 in

\renewcommand\figurename{\rm Fig.}
\renewcommand\tablename{\rm Table}
\renewcommand{\thetable}{\arabic{table}}

\let\oldequation\equation
\let\oldendequation\endequation

\renewenvironment{equation}
  {\linenomathNonumbers\oldequation}
  {\oldendequation\endlinenomath}

\oddsidemargin  -0.2cm
\evensidemargin -0.2cm

\begin{document}

\title{{\bf \boldmath Observation of $D^+_s\to \eta^\prime \mu^+\nu_\mu$ and Measurements of $D^+_s\to \eta^{(\prime)}\mu^+\nu_\mu$ Decay Dynamics}}


\author{
M.~Ablikim$^{1}$, M.~N.~Achasov$^{13,b}$, P.~Adlarson$^{73}$, R.~Aliberti$^{34}$, A.~Amoroso$^{72A,72C}$, M.~R.~An$^{38}$, Q.~An$^{69,56}$, Y.~Bai$^{55}$, O.~Bakina$^{35}$, I.~Balossino$^{29A}$, Y.~Ban$^{45,g}$, V.~Batozskaya$^{1,43}$, K.~Begzsuren$^{31}$, N.~Berger$^{34}$, M.~Berlowski$^{43}$, M.~Bertani$^{28A}$, D.~Bettoni$^{29A}$, F.~Bianchi$^{72A,72C}$, E.~Bianco$^{72A,72C}$, J.~Bloms$^{66}$, A.~Bortone$^{72A,72C}$, I.~Boyko$^{35}$, R.~A.~Briere$^{5}$, A.~Brueggemann$^{66}$, H.~Cai$^{74}$, X.~Cai$^{1,56}$, A.~Calcaterra$^{28A}$, G.~F.~Cao$^{1,61}$, N.~Cao$^{1,61}$, S.~A.~Cetin$^{60A}$, J.~F.~Chang$^{1,56}$, T.~T.~Chang$^{75}$, W.~L.~Chang$^{1,61}$, G.~R.~Che$^{42}$, G.~Chelkov$^{35,a}$, C.~Chen$^{42}$, Chao~Chen$^{53}$, G.~Chen$^{1}$, H.~S.~Chen$^{1,61}$, M.~L.~Chen$^{1,56,61}$, S.~J.~Chen$^{41}$, S.~M.~Chen$^{59}$, T.~Chen$^{1,61}$, X.~R.~Chen$^{30,61}$, X.~T.~Chen$^{1,61}$, Y.~B.~Chen$^{1,56}$, Y.~Q.~Chen$^{33}$, Z.~J.~Chen$^{25,h}$, W.~S.~Cheng$^{72C}$, S.~K.~Choi$^{10A}$, X.~Chu$^{42}$, G.~Cibinetto$^{29A}$, S.~C.~Coen$^{4}$, F.~Cossio$^{72C}$, J.~J.~Cui$^{48}$, H.~L.~Dai$^{1,56}$, J.~P.~Dai$^{77}$, A.~Dbeyssi$^{19}$, R.~ E.~de Boer$^{4}$, D.~Dedovich$^{35}$, Z.~Y.~Deng$^{1}$, A.~Denig$^{34}$, I.~Denysenko$^{35}$, M.~Destefanis$^{72A,72C}$, F.~De~Mori$^{72A,72C}$, B.~Ding$^{64,1}$, X.~X.~Ding$^{45,g}$, Y.~Ding$^{33}$, Y.~Ding$^{39}$, J.~Dong$^{1,56}$, L.~Y.~Dong$^{1,61}$, M.~Y.~Dong$^{1,56,61}$, X.~Dong$^{74}$, S.~X.~Du$^{79}$, Z.~H.~Duan$^{41}$, P.~Egorov$^{35,a}$, Y.~L.~Fan$^{74}$, J.~Fang$^{1,56}$, S.~S.~Fang$^{1,61}$, W.~X.~Fang$^{1}$, Y.~Fang$^{1}$, R.~Farinelli$^{29A}$, L.~Fava$^{72B,72C}$, F.~Feldbauer$^{4}$, G.~Felici$^{28A}$, C.~Q.~Feng$^{69,56}$, J.~H.~Feng$^{57}$, K~Fischer$^{67}$, M.~Fritsch$^{4}$, C.~Fritzsch$^{66}$, C.~D.~Fu$^{1}$, Y.~W.~Fu$^{1}$, H.~Gao$^{61}$, Y.~N.~Gao$^{45,g}$, Yang~Gao$^{69,56}$, S.~Garbolino$^{72C}$, I.~Garzia$^{29A,29B}$, P.~T.~Ge$^{74}$, Z.~W.~Ge$^{41}$, C.~Geng$^{57}$, E.~M.~Gersabeck$^{65}$, A~Gilman$^{67}$, K.~Goetzen$^{14}$, L.~Gong$^{39}$, W.~X.~Gong$^{1,56}$, W.~Gradl$^{34}$, S.~Gramigna$^{29A,29B}$, M.~Greco$^{72A,72C}$, M.~H.~Gu$^{1,56}$, Y.~T.~Gu$^{16}$, C.~Y~Guan$^{1,61}$, Z.~L.~Guan$^{22}$, A.~Q.~Guo$^{30,61}$, L.~B.~Guo$^{40}$, R.~P.~Guo$^{47}$, Y.~P.~Guo$^{12,f}$, A.~Guskov$^{35,a}$, X.~T.~H.$^{1,61}$, W.~Y.~Han$^{38}$, X.~Q.~Hao$^{20}$, F.~A.~Harris$^{63}$, K.~K.~He$^{53}$, K.~L.~He$^{1,61}$, F.~H.~Heinsius$^{4}$, C.~H.~Heinz$^{34}$, Y.~K.~Heng$^{1,56,61}$, C.~Herold$^{58}$, T.~Holtmann$^{4}$, P.~C.~Hong$^{12,f}$, G.~Y.~Hou$^{1,61}$, Y.~R.~Hou$^{61}$, Z.~L.~Hou$^{1}$, H.~M.~Hu$^{1,61}$, J.~F.~Hu$^{54,i}$, T.~Hu$^{1,56,61}$, Y.~Hu$^{1}$, G.~S.~Huang$^{69,56}$, K.~X.~Huang$^{57}$, L.~Q.~Huang$^{30,61}$, X.~T.~Huang$^{48}$, Y.~P.~Huang$^{1}$, T.~Hussain$^{71}$, N~H\"usken$^{27,34}$, W.~Imoehl$^{27}$, M.~Irshad$^{69,56}$, J.~Jackson$^{27}$, S.~Jaeger$^{4}$, S.~Janchiv$^{31}$, J.~H.~Jeong$^{10A}$, Q.~Ji$^{1}$, Q.~P.~Ji$^{20}$, X.~B.~Ji$^{1,61}$, X.~L.~Ji$^{1,56}$, Y.~Y.~Ji$^{48}$, Z.~K.~Jia$^{69,56}$, P.~C.~Jiang$^{45,g}$, S.~S.~Jiang$^{38}$, T.~J.~Jiang$^{17}$, X.~S.~Jiang$^{1,56,61}$, Y.~Jiang$^{61}$, J.~B.~Jiao$^{48}$, Z.~Jiao$^{23}$, S.~Jin$^{41}$, Y.~Jin$^{64}$, M.~Q.~Jing$^{1,61}$, T.~Johansson$^{73}$, X.~K.$^{1}$, S.~Kabana$^{32}$, N.~Kalantar-Nayestanaki$^{62}$, X.~L.~Kang$^{9}$, X.~S.~Kang$^{39}$, R.~Kappert$^{62}$, M.~Kavatsyuk$^{62}$, B.~C.~Ke$^{79}$, A.~Khoukaz$^{66}$, R.~Kiuchi$^{1}$, R.~Kliemt$^{14}$, L.~Koch$^{36}$, O.~B.~Kolcu$^{60A}$, B.~Kopf$^{4}$, M.~Kuessner$^{4}$, A.~Kupsc$^{43,73}$, W.~K\"uhn$^{36}$, J.~J.~Lane$^{65}$, J.~S.~Lange$^{36}$, P. ~Larin$^{19}$, A.~Lavania$^{26}$, L.~Lavezzi$^{72A,72C}$, T.~T.~Lei$^{69,k}$, Z.~H.~Lei$^{69,56}$, H.~Leithoff$^{34}$, M.~Lellmann$^{34}$, T.~Lenz$^{34}$, C.~Li$^{46}$, C.~Li$^{42}$, C.~H.~Li$^{38}$, Cheng~Li$^{69,56}$, D.~M.~Li$^{79}$, F.~Li$^{1,56}$, G.~Li$^{1}$, H.~Li$^{69,56}$, H.~B.~Li$^{1,61}$, H.~J.~Li$^{20}$, H.~N.~Li$^{54,i}$, Hui~Li$^{42}$, J.~R.~Li$^{59}$, J.~S.~Li$^{57}$, J.~W.~Li$^{48}$, Ke~Li$^{1}$, L.~J~Li$^{1,61}$, L.~K.~Li$^{1}$, Lei~Li$^{3}$, M.~H.~Li$^{42}$, P.~R.~Li$^{37,j,k}$, S.~X.~Li$^{12}$, T. ~Li$^{48}$, W.~D.~Li$^{1,61}$, W.~G.~Li$^{1}$, X.~H.~Li$^{69,56}$, X.~L.~Li$^{48}$, Xiaoyu~Li$^{1,61}$, Y.~G.~Li$^{45,g}$, Z.~J.~Li$^{57}$, Z.~X.~Li$^{16}$, Z.~Y.~Li$^{57}$, C.~Liang$^{41}$, H.~Liang$^{1,61}$, H.~Liang$^{69,56}$, H.~Liang$^{33}$, Y.~F.~Liang$^{52}$, Y.~T.~Liang$^{30,61}$, G.~R.~Liao$^{15}$, L.~Z.~Liao$^{48}$, J.~Libby$^{26}$, A. ~Limphirat$^{58}$, D.~X.~Lin$^{30,61}$, T.~Lin$^{1}$, B.~J.~Liu$^{1}$, B.~X.~Liu$^{74}$, C.~Liu$^{33}$, C.~X.~Liu$^{1}$, D.~~Liu$^{19,69}$, F.~H.~Liu$^{51}$, Fang~Liu$^{1}$, Feng~Liu$^{6}$, G.~M.~Liu$^{54,i}$, H.~Liu$^{37,j,k}$, H.~B.~Liu$^{16}$, H.~M.~Liu$^{1,61}$, Huanhuan~Liu$^{1}$, Huihui~Liu$^{21}$, J.~B.~Liu$^{69,56}$, J.~L.~Liu$^{70}$, J.~Y.~Liu$^{1,61}$, K.~Liu$^{1}$, K.~Y.~Liu$^{39}$, Ke~Liu$^{22}$, L.~Liu$^{69,56}$, L.~C.~Liu$^{42}$, Lu~Liu$^{42}$, M.~H.~Liu$^{12,f}$, P.~L.~Liu$^{1}$, Q.~Liu$^{61}$, S.~B.~Liu$^{69,56}$, T.~Liu$^{12,f}$, W.~K.~Liu$^{42}$, W.~M.~Liu$^{69,56}$, X.~Liu$^{37,j,k}$, Y.~Liu$^{37,j,k}$, Y.~B.~Liu$^{42}$, Z.~A.~Liu$^{1,56,61}$, Z.~Q.~Liu$^{48}$, X.~C.~Lou$^{1,56,61}$, F.~X.~Lu$^{57}$, H.~J.~Lu$^{23}$, J.~G.~Lu$^{1,56}$, X.~L.~Lu$^{1}$, Y.~Lu$^{7}$, Y.~P.~Lu$^{1,56}$, Z.~H.~Lu$^{1,61}$, C.~L.~Luo$^{40}$, M.~X.~Luo$^{78}$, T.~Luo$^{12,f}$, X.~L.~Luo$^{1,56}$, X.~R.~Lyu$^{61}$, Y.~F.~Lyu$^{42}$, F.~C.~Ma$^{39}$, H.~L.~Ma$^{1}$, J.~L.~Ma$^{1,61}$, L.~L.~Ma$^{48}$, M.~M.~Ma$^{1,61}$, Q.~M.~Ma$^{1}$, R.~Q.~Ma$^{1,61}$, R.~T.~Ma$^{61}$, X.~Y.~Ma$^{1,56}$, Y.~Ma$^{45,g}$, F.~E.~Maas$^{19}$, M.~Maggiora$^{72A,72C}$, S.~Maldaner$^{4}$, S.~Malde$^{67}$, A.~Mangoni$^{28B}$, Y.~J.~Mao$^{45,g}$, Z.~P.~Mao$^{1}$, S.~Marcello$^{72A,72C}$, Z.~X.~Meng$^{64}$, J.~G.~Messchendorp$^{14,62}$, G.~Mezzadri$^{29A}$, H.~Miao$^{1,61}$, T.~J.~Min$^{41}$, R.~E.~Mitchell$^{27}$, X.~H.~Mo$^{1,56,61}$, N.~Yu.~Muchnoi$^{13,b}$, Y.~Nefedov$^{35}$, F.~Nerling$^{19,d}$, I.~B.~Nikolaev$^{13,b}$, Z.~Ning$^{1,56}$, S.~Nisar$^{11,l}$, Y.~Niu $^{48}$, S.~L.~Olsen$^{61}$, Q.~Ouyang$^{1,56,61}$, S.~Pacetti$^{28B,28C}$, X.~Pan$^{53}$, Y.~Pan$^{55}$, A.~~Pathak$^{33}$, Y.~P.~Pei$^{69,56}$, M.~Pelizaeus$^{4}$, H.~P.~Peng$^{69,56}$, K.~Peters$^{14,d}$, J.~L.~Ping$^{40}$, R.~G.~Ping$^{1,61}$, S.~Plura$^{34}$, S.~Pogodin$^{35}$, V.~Prasad$^{32}$, F.~Z.~Qi$^{1}$, H.~Qi$^{69,56}$, H.~R.~Qi$^{59}$, M.~Qi$^{41}$, T.~Y.~Qi$^{12,f}$, S.~Qian$^{1,56}$, W.~B.~Qian$^{61}$, C.~F.~Qiao$^{61}$, J.~J.~Qin$^{70}$, L.~Q.~Qin$^{15}$, X.~P.~Qin$^{12,f}$, X.~S.~Qin$^{48}$, Z.~H.~Qin$^{1,56}$, J.~F.~Qiu$^{1}$, S.~Q.~Qu$^{59}$, C.~F.~Redmer$^{34}$, K.~J.~Ren$^{38}$, A.~Rivetti$^{72C}$, V.~Rodin$^{62}$, M.~Rolo$^{72C}$, G.~Rong$^{1,61}$, Ch.~Rosner$^{19}$, S.~N.~Ruan$^{42}$, N.~Salone$^{43}$, A.~Sarantsev$^{35,c}$, Y.~Schelhaas$^{34}$, K.~Schoenning$^{73}$, M.~Scodeggio$^{29A,29B}$, K.~Y.~Shan$^{12,f}$, W.~Shan$^{24}$, X.~Y.~Shan$^{69,56}$, J.~F.~Shangguan$^{53}$, L.~G.~Shao$^{1,61}$, M.~Shao$^{69,56}$, C.~P.~Shen$^{12,f}$, H.~F.~Shen$^{1,61}$, W.~H.~Shen$^{61}$, X.~Y.~Shen$^{1,61}$, B.~A.~Shi$^{61}$, H.~C.~Shi$^{69,56}$, J.~L.~Shi$^{12}$, J.~Y.~Shi$^{1}$, Q.~Q.~Shi$^{53}$, R.~S.~Shi$^{1,61}$, X.~Shi$^{1,56}$, J.~J.~Song$^{20}$, T.~Z.~Song$^{57}$, W.~M.~Song$^{33,1}$, Y. ~J.~Song$^{12}$, Y.~X.~Song$^{45,g}$, S.~Sosio$^{72A,72C}$, S.~Spataro$^{72A,72C}$, F.~Stieler$^{34}$, Y.~J.~Su$^{61}$, G.~B.~Sun$^{74}$, G.~X.~Sun$^{1}$, H.~Sun$^{61}$, H.~K.~Sun$^{1}$, J.~F.~Sun$^{20}$, K.~Sun$^{59}$, L.~Sun$^{74}$, S.~S.~Sun$^{1,61}$, T.~Sun$^{1,61}$, W.~Y.~Sun$^{33}$, Y.~Sun$^{9}$, Y.~J.~Sun$^{69,56}$, Y.~Z.~Sun$^{1}$, Z.~T.~Sun$^{48}$, Y.~X.~Tan$^{69,56}$, C.~J.~Tang$^{52}$, G.~Y.~Tang$^{1}$, J.~Tang$^{57}$, Y.~A.~Tang$^{74}$, L.~Y~Tao$^{70}$, Q.~T.~Tao$^{25,h}$, M.~Tat$^{67}$, J.~X.~Teng$^{69,56}$, V.~Thoren$^{73}$, W.~H.~Tian$^{50}$, W.~H.~Tian$^{57}$, Y.~Tian$^{30,61}$, Z.~F.~Tian$^{74}$, I.~Uman$^{60B}$, B.~Wang$^{1}$, B.~L.~Wang$^{61}$, Bo~Wang$^{69,56}$, C.~W.~Wang$^{41}$, D.~Y.~Wang$^{45,g}$, F.~Wang$^{70}$, H.~J.~Wang$^{37,j,k}$, H.~P.~Wang$^{1,61}$, K.~Wang$^{1,56}$, L.~L.~Wang$^{1}$, M.~Wang$^{48}$, Meng~Wang$^{1,61}$, S.~Wang$^{37,j,k}$, S.~Wang$^{12,f}$, T. ~Wang$^{12,f}$, T.~J.~Wang$^{42}$, W.~Wang$^{57}$, W. ~Wang$^{70}$, W.~H.~Wang$^{74}$, W.~P.~Wang$^{69,56}$, X.~Wang$^{45,g}$, X.~F.~Wang$^{37,j,k}$, X.~J.~Wang$^{38}$, X.~L.~Wang$^{12,f}$, Y.~Wang$^{59}$, Y.~D.~Wang$^{44}$, Y.~F.~Wang$^{1,56,61}$, Y.~H.~Wang$^{46}$, Y.~N.~Wang$^{44}$, Y.~Q.~Wang$^{1}$, Yaqian~Wang$^{18,1}$, Yi~Wang$^{59}$, Z.~Wang$^{1,56}$, Z.~L. ~Wang$^{70}$, Z.~Y.~Wang$^{1,61}$, Ziyi~Wang$^{61}$, D.~Wei$^{68}$, D.~H.~Wei$^{15}$, F.~Weidner$^{66}$, S.~P.~Wen$^{1}$, C.~W.~Wenzel$^{4}$, U.~Wiedner$^{4}$, G.~Wilkinson$^{67}$, M.~Wolke$^{73}$, L.~Wollenberg$^{4}$, C.~Wu$^{38}$, J.~F.~Wu$^{1,61}$, L.~H.~Wu$^{1}$, L.~J.~Wu$^{1,61}$, X.~Wu$^{12,f}$, X.~H.~Wu$^{33}$, Y.~Wu$^{69}$, Y.~J~Wu$^{30}$, Z.~Wu$^{1,56}$, L.~Xia$^{69,56}$, X.~M.~Xian$^{38}$, T.~Xiang$^{45,g}$, D.~Xiao$^{37,j,k}$, G.~Y.~Xiao$^{41}$, H.~Xiao$^{12,f}$, S.~Y.~Xiao$^{1}$, Y. ~L.~Xiao$^{12,f}$, Z.~J.~Xiao$^{40}$, C.~Xie$^{41}$, X.~H.~Xie$^{45,g}$, Y.~Xie$^{48}$, Y.~G.~Xie$^{1,56}$, Y.~H.~Xie$^{6}$, Z.~P.~Xie$^{69,56}$, T.~Y.~Xing$^{1,61}$, C.~F.~Xu$^{1,61}$, C.~J.~Xu$^{57}$, G.~F.~Xu$^{1}$, H.~Y.~Xu$^{64}$, Q.~J.~Xu$^{17}$, W.~L.~Xu$^{64}$, X.~P.~Xu$^{53}$, Y.~C.~Xu$^{76}$, Z.~P.~Xu$^{41}$, F.~Yan$^{12,f}$, L.~Yan$^{12,f}$, W.~B.~Yan$^{69,56}$, W.~C.~Yan$^{79}$, X.~Q~Yan$^{1}$, H.~J.~Yang$^{49,e}$, H.~L.~Yang$^{33}$, H.~X.~Yang$^{1}$, Tao~Yang$^{1}$, Y.~Yang$^{12,f}$, Y.~F.~Yang$^{42}$, Y.~X.~Yang$^{1,61}$, Yifan~Yang$^{1,61}$, Z.~W.~Yang$^{37,j,k}$, M.~Ye$^{1,56}$, M.~H.~Ye$^{8}$, J.~H.~Yin$^{1}$, Z.~Y.~You$^{57}$, B.~X.~Yu$^{1,56,61}$, C.~X.~Yu$^{42}$, G.~Yu$^{1,61}$, T.~Yu$^{70}$, X.~D.~Yu$^{45,g}$, C.~Z.~Yuan$^{1,61}$, L.~Yuan$^{2}$, S.~C.~Yuan$^{1}$, X.~Q.~Yuan$^{1}$, Y.~Yuan$^{1,61}$, Z.~Y.~Yuan$^{57}$, C.~X.~Yue$^{38}$, A.~A.~Zafar$^{71}$, F.~R.~Zeng$^{48}$, X.~Zeng$^{12,f}$, Y.~Zeng$^{25,h}$, Y.~J.~Zeng$^{1,61}$, X.~Y.~Zhai$^{33}$, Y.~H.~Zhan$^{57}$, A.~Q.~Zhang$^{1,61}$, B.~L.~Zhang$^{1,61}$, B.~X.~Zhang$^{1}$, D.~H.~Zhang$^{42}$, G.~Y.~Zhang$^{20}$, H.~Zhang$^{69}$, H.~H.~Zhang$^{57}$, H.~H.~Zhang$^{33}$, H.~Q.~Zhang$^{1,56,61}$, H.~Y.~Zhang$^{1,56}$, J.~J.~Zhang$^{50}$, J.~L.~Zhang$^{75}$, J.~Q.~Zhang$^{40}$, J.~W.~Zhang$^{1,56,61}$, J.~X.~Zhang$^{37,j,k}$, J.~Y.~Zhang$^{1}$, J.~Z.~Zhang$^{1,61}$, Jiawei~Zhang$^{1,61}$, L.~M.~Zhang$^{59}$, L.~Q.~Zhang$^{57}$, Lei~Zhang$^{41}$, P.~Zhang$^{1}$, Q.~Y.~~Zhang$^{38,79}$, Shuihan~Zhang$^{1,61}$, Shulei~Zhang$^{25,h}$, X.~D.~Zhang$^{44}$, X.~M.~Zhang$^{1}$, X.~Y.~Zhang$^{53}$, X.~Y.~Zhang$^{48}$, Y.~Zhang$^{67}$, Y. ~T.~Zhang$^{79}$, Y.~H.~Zhang$^{1,56}$, Yan~Zhang$^{69,56}$, Yao~Zhang$^{1}$, Z.~H.~Zhang$^{1}$, Z.~L.~Zhang$^{33}$, Z.~Y.~Zhang$^{74}$, Z.~Y.~Zhang$^{42}$, G.~Zhao$^{1}$, J.~Zhao$^{38}$, J.~Y.~Zhao$^{1,61}$, J.~Z.~Zhao$^{1,56}$, Lei~Zhao$^{69,56}$, Ling~Zhao$^{1}$, M.~G.~Zhao$^{42}$, S.~J.~Zhao$^{79}$, Y.~B.~Zhao$^{1,56}$, Y.~X.~Zhao$^{30,61}$, Z.~G.~Zhao$^{69,56}$, A.~Zhemchugov$^{35,a}$, B.~Zheng$^{70}$, J.~P.~Zheng$^{1,56}$, W.~J.~Zheng$^{1,61}$, Y.~H.~Zheng$^{61}$, B.~Zhong$^{40}$, X.~Zhong$^{57}$, H. ~Zhou$^{48}$, L.~P.~Zhou$^{1,61}$, X.~Zhou$^{74}$, X.~K.~Zhou$^{6}$, X.~R.~Zhou$^{69,56}$, X.~Y.~Zhou$^{38}$, Y.~Z.~Zhou$^{12,f}$, J.~Zhu$^{42}$, K.~Zhu$^{1}$, K.~J.~Zhu$^{1,56,61}$, L.~Zhu$^{33}$, L.~X.~Zhu$^{61}$, S.~H.~Zhu$^{68}$, S.~Q.~Zhu$^{41}$, T.~J.~Zhu$^{12,f}$, W.~J.~Zhu$^{12,f}$, Y.~C.~Zhu$^{69,56}$, Z.~A.~Zhu$^{1,61}$, J.~H.~Zou$^{1}$, J.~Zu$^{69,56}$
\\
\vspace{0.2cm}
(BESIII Collaboration)\\
\vspace{0.2cm} {\it
$^{1}$ Institute of High Energy Physics, Beijing 100049, People's Republic of China\\
$^{2}$ Beihang University, Beijing 100191, People's Republic of China\\
$^{3}$ Beijing Institute of Petrochemical Technology, Beijing 102617, People's Republic of China\\
$^{4}$ Bochum  Ruhr-University, D-44780 Bochum, Germany\\
$^{5}$ Carnegie Mellon University, Pittsburgh, Pennsylvania 15213, USA\\
$^{6}$ Central China Normal University, Wuhan 430079, People's Republic of China\\
$^{7}$ Central South University, Changsha 410083, People's Republic of China\\
$^{8}$ China Center of Advanced Science and Technology, Beijing 100190, People's Republic of China\\
$^{9}$ China University of Geosciences, Wuhan 430074, People's Republic of China\\
$^{10}$ Chung-Ang University, Seoul, 06974, Republic of Korea\\
$^{11}$ COMSATS University Islamabad, Lahore Campus, Defence Road, Off Raiwind Road, 54000 Lahore, Pakistan\\
$^{12}$ Fudan University, Shanghai 200433, People's Republic of China\\
$^{13}$ G.I. Budker Institute of Nuclear Physics SB RAS (BINP), Novosibirsk 630090, Russia\\
$^{14}$ GSI Helmholtzcentre for Heavy Ion Research GmbH, D-64291 Darmstadt, Germany\\
$^{15}$ Guangxi Normal University, Guilin 541004, People's Republic of China\\
$^{16}$ Guangxi University, Nanning 530004, People's Republic of China\\
$^{17}$ Hangzhou Normal University, Hangzhou 310036, People's Republic of China\\
$^{18}$ Hebei University, Baoding 071002, People's Republic of China\\
$^{19}$ Helmholtz Institute Mainz, Staudinger Weg 18, D-55099 Mainz, Germany\\
$^{20}$ Henan Normal University, Xinxiang 453007, People's Republic of China\\
$^{21}$ Henan University of Science and Technology, Luoyang 471003, People's Republic of China\\
$^{22}$ Henan University of Technology, Zhengzhou 450001, People's Republic of China\\
$^{23}$ Huangshan College, Huangshan  245000, People's Republic of China\\
$^{24}$ Hunan Normal University, Changsha 410081, People's Republic of China\\
$^{25}$ Hunan University, Changsha 410082, People's Republic of China\\
$^{26}$ Indian Institute of Technology Madras, Chennai 600036, India\\
$^{27}$ Indiana University, Bloomington, Indiana 47405, USA\\
$^{28}$ INFN Laboratori Nazionali di Frascati , (A)INFN Laboratori Nazionali di Frascati, I-00044, Frascati, Italy; (B)INFN Sezione di  Perugia, I-06100, Perugia, Italy; (C)University of Perugia, I-06100, Perugia, Italy\\
$^{29}$ INFN Sezione di Ferrara, (A)INFN Sezione di Ferrara, I-44122, Ferrara, Italy; (B)University of Ferrara,  I-44122, Ferrara, Italy\\
$^{30}$ Institute of Modern Physics, Lanzhou 730000, People's Republic of China\\
$^{31}$ Institute of Physics and Technology, Peace Avenue 54B, Ulaanbaatar 13330, Mongolia\\
$^{32}$ Instituto de Alta Investigaci\'on, Universidad de Tarapac\'a, Casilla 7D, Arica, Chile\\
$^{33}$ Jilin University, Changchun 130012, People's Republic of China\\
$^{34}$ Johannes Gutenberg University of Mainz, Johann-Joachim-Becher-Weg 45, D-55099 Mainz, Germany\\
$^{35}$ Joint Institute for Nuclear Research, 141980 Dubna, Moscow region, Russia\\
$^{36}$ Justus-Liebig-Universitaet Giessen, II. Physikalisches Institut, Heinrich-Buff-Ring 16, D-35392 Giessen, Germany\\
$^{37}$ Lanzhou University, Lanzhou 730000, People's Republic of China\\
$^{38}$ Liaoning Normal University, Dalian 116029, People's Republic of China\\
$^{39}$ Liaoning University, Shenyang 110036, People's Republic of China\\
$^{40}$ Nanjing Normal University, Nanjing 210023, People's Republic of China\\
$^{41}$ Nanjing University, Nanjing 210093, People's Republic of China\\
$^{42}$ Nankai University, Tianjin 300071, People's Republic of China\\
$^{43}$ National Centre for Nuclear Research, Warsaw 02-093, Poland\\
$^{44}$ North China Electric Power University, Beijing 102206, People's Republic of China\\
$^{45}$ Peking University, Beijing 100871, People's Republic of China\\
$^{46}$ Qufu Normal University, Qufu 273165, People's Republic of China\\
$^{47}$ Shandong Normal University, Jinan 250014, People's Republic of China\\
$^{48}$ Shandong University, Jinan 250100, People's Republic of China\\
$^{49}$ Shanghai Jiao Tong University, Shanghai 200240,  People's Republic of China\\
$^{50}$ Shanxi Normal University, Linfen 041004, People's Republic of China\\
$^{51}$ Shanxi University, Taiyuan 030006, People's Republic of China\\
$^{52}$ Sichuan University, Chengdu 610064, People's Republic of China\\
$^{53}$ Soochow University, Suzhou 215006, People's Republic of China\\
$^{54}$ South China Normal University, Guangzhou 510006, People's Republic of China\\
$^{55}$ Southeast University, Nanjing 211100, People's Republic of China\\
$^{56}$ State Key Laboratory of Particle Detection and Electronics, Beijing 100049, Hefei 230026, People's Republic of China\\
$^{57}$ Sun Yat-Sen University, Guangzhou 510275, People's Republic of China\\
$^{58}$ Suranaree University of Technology, University Avenue 111, Nakhon Ratchasima 30000, Thailand\\
$^{59}$ Tsinghua University, Beijing 100084, People's Republic of China\\
$^{60}$ Turkish Accelerator Center Particle Factory Group, (A)Istinye University, 34010, Istanbul, Turkey; (B)Near East University, Nicosia, North Cyprus, 99138, Mersin 10, Turkey\\
$^{61}$ University of Chinese Academy of Sciences, Beijing 100049, People's Republic of China\\
$^{62}$ University of Groningen, NL-9747 AA Groningen, The Netherlands\\
$^{63}$ University of Hawaii, Honolulu, Hawaii 96822, USA\\
$^{64}$ University of Jinan, Jinan 250022, People's Republic of China\\
$^{65}$ University of Manchester, Oxford Road, Manchester, M13 9PL, United Kingdom\\
$^{66}$ University of Muenster, Wilhelm-Klemm-Strasse 9, 48149 Muenster, Germany\\
$^{67}$ University of Oxford, Keble Road, Oxford OX13RH, United Kingdom\\
$^{68}$ University of Science and Technology Liaoning, Anshan 114051, People's Republic of China\\
$^{69}$ University of Science and Technology of China, Hefei 230026, People's Republic of China\\
$^{70}$ University of South China, Hengyang 421001, People's Republic of China\\
$^{71}$ University of the Punjab, Lahore-54590, Pakistan\\
$^{72}$ University of Turin and INFN, (A)University of Turin, I-10125, Turin, Italy; (B)University of Eastern Piedmont, I-15121, Alessandria, Italy; (C)INFN, I-10125, Turin, Italy\\
$^{73}$ Uppsala University, Box 516, SE-75120 Uppsala, Sweden\\
$^{74}$ Wuhan University, Wuhan 430072, People's Republic of China\\
$^{75}$ Xinyang Normal University, Xinyang 464000, People's Republic of China\\
$^{76}$ Yantai University, Yantai 264005, People's Republic of China\\
$^{77}$ Yunnan University, Kunming 650500, People's Republic of China\\
$^{78}$ Zhejiang University, Hangzhou 310027, People's Republic of China\\
$^{79}$ Zhengzhou University, Zhengzhou 450001, People's Republic of China\\
\vspace{0.2cm}
$^{a}$ Also at the Moscow Institute of Physics and Technology, Moscow 141700, Russia\\
$^{b}$ Also at the Novosibirsk State University, Novosibirsk, 630090, Russia\\
$^{c}$ Also at the NRC "Kurchatov Institute", PNPI, 188300, Gatchina, Russia\\
$^{d}$ Also at Goethe University Frankfurt, 60323 Frankfurt am Main, Germany\\
$^{e}$ Also at Key Laboratory for Particle Physics, Astrophysics and Cosmology, Ministry of Education; Shanghai Key Laboratory for Particle Physics and Cosmology; Institute of Nuclear and Particle Physics, Shanghai 200240, People's Republic of China\\
$^{f}$ Also at Key Laboratory of Nuclear Physics and Ion-beam Application (MOE) and Institute of Modern Physics, Fudan University, Shanghai 200443, People's Republic of China\\
$^{g}$ Also at State Key Laboratory of Nuclear Physics and Technology, Peking University, Beijing 100871, People's Republic of China\\
$^{h}$ Also at School of Physics and Electronics, Hunan University, Changsha 410082, China\\
$^{i}$ Also at Guangdong Provincial Key Laboratory of Nuclear Science, Institute of Quantum Matter, South China Normal University, Guangzhou 510006, China\\
$^{j}$ Also at Frontiers Science Center for Rare Isotopes, Lanzhou University, Lanzhou 730000, People's Republic of China\\
$^{k}$ Also at Lanzhou Center for Theoretical Physics, Lanzhou University, Lanzhou 730000, People's Republic of China\\
$^{l}$ Also at the Department of Mathematical Sciences, IBA, Karachi , Pakistan\\
}
}



%\linenumbers


\begin{abstract}
By analyzing 7.33 fb$^{-1}$ of $e^+e^-$ annihilation data
collected at center-of-mass energies between 4.128 and 4.226 GeV with the BESIII detector,
we report the observation of the semileptonic decay $D^+_s\to
\eta^\prime \mu^+\nu_\mu$ and studies of the $D_s^+ \to
\eta\mu^+\nu_\mu$ and $D_s^+ \to \eta^\prime\mu^+\nu_\mu$ decay dynamics for the first time.
The branching fractions of $D_s^+ \to \eta\mu^+\nu_\mu$ and $D_s^+ \to \eta^\prime\mu^+\nu_\mu$ are determined to be
$(2.215\pm0.051_{\rm stat}\pm0.052_{\rm syst})\%$ and
$(0.801\pm0.055_{\rm stat}\pm0.031_{\rm syst})\%$, respectively,
with precision improved by factors of 5.9 and 6.5 compared to the previous best measurements.
Combined with the results for the decays $D_s^+ \to \eta e^+\nu_e$ and $D_s^+ \to \eta^\prime e^+\nu_e$,
 the ratios of the decay widths are examined, both inclusively and in several $\mu^+\nu_\mu$ four-momentum transfer ranges. No evidence for lepton flavor universality violation is found within the current statistics.
The products of the hadronic form factors $f_+^{\eta^{(\prime)}}(0)$ and the $c\to s$ Cabibbo-Kobayashi-Maskawa matrix element $|V_{cs}|$ are determined. The results based on the two-parameter series expansion are $f^{\eta}_+(0)|V_{cs}| = 0.451\pm0.010_{\rm stat}\pm0.008_{\rm syst}$ and $f^{\eta^{\prime}}_+(0)|V_{cs}| = 0.506\pm0.037_{\rm stat}\pm0.011_{\rm syst}$, which constrain present models on $f_{+}^{\eta^{(\prime)}}(0)$. The $\eta-\eta^\prime$ mixing angle in the quark flavor basis is determined to be
$\phi_{\rm P} =(40.2\pm2.1_{\rm stat}\pm0.7_{\rm syst})^\circ$.
\end{abstract}

\maketitle

\oddsidemargin  -0.2cm
\evensidemargin -0.2cm

The couplings between the three families of leptons and the gauge bosons are equal in the standard model (SM). This property is known as lepton flavor universality (LFU). In recent years, however, hints of tensions between experimental measurements and the SM predictions were reported in the semileptonic (SL) $B$ decays~\cite{HFLAV}, the anomalous magnetic moment of the muon~\cite{Muong-2:2006rrc,Muong-2:2021ojo} and the Cabibbo angle anomaly~\cite{Coutinho:2019aiy,Crivellin:2020lzu}. 
For example, the measured branching fraction (BF) ratios
${\mathcal R}_{D^{(*)}}^{\tau/\ell}={\mathcal B}_{B\to \bar D^{(*)}\tau^+\nu_\tau}/{\mathcal B}_{B\to \bar D^{(*)}\ell^+\nu_\ell}$~($\ell=\mu$, $e$)~\cite{babar_1,babar_2,lhcb_1,belle2015,belle2016,Belle:2019rba,LHCbreport}
 deviate from the SM predictions by $3.3\sigma$~\cite{HFLAV}. 
Although these tensions have been explained by various theoretical models~\cite{BhupalDev:2020zcy,Nomura:2021oeu,Bordone:2016gaq,Altmannshofer:2017yso,Crivellin:2017zlb,Becirevic:2016yqi,BFajfer2012,Fajfer2012,Celis2013,Crivellin2015,Crivellin2016,Bauer2016}, no definite conclusion is established yet. Precision tests of LFU in various SL decays of heavy mesons can provide deeper insight into these anomalies. 
Possible LFU in SL $D^+_s$ decays is not yet well tested, due to poor knowledge of semimuonic $D^+_s$ decays. There may indeed be observable LFU violation effects in the SL decays mediated via $c\to s\ell^+\nu_\ell$~\cite{Fajfer2015}. In the SM, the ratio ${\mathcal R}^{\eta^{(\prime)}}_{\mu/e}={\mathcal B}_{D_s^+\to\eta^{(\prime)}\mu^+\nu_\mu}/{\mathcal B}_{D_s^+\to\eta^{(\prime)} e^+\nu_e}$ is predicted to be 0.95-0.99~\cite{Hu:2021zmy,Ivanov:2019nqd,Cheng:2017pcq}. 
Precision measurements of $D^+_s\to \eta^{(\prime)}\mu^+\nu_\mu$ are important to test $\mu$-$e$ LFU with the SL decays $D^+_s\to \eta^{(\prime)}\ell^+\nu_\ell$, which is expected to be the most competitive mode in the $D^+_s$ sector. 

In the SM, the decay rates of $D^+_s\to \eta^{(\prime)}\mu^+\nu_\mu$
can be expressed as functions of the $c\to s$ Cabibbo-Kobayashi-Maskawa (CKM) matrix element $|V_{cs}|$ and
the hadronic form factors~(FFs) $f_+^{\eta^{(\prime)}}(q^2)$, which describe the weak and strong interaction effects, respectively. Here, $q$ is the four-momentum transfer to the $\mu^+\nu_\mu$ system. Measurements of $D^+_s\to\eta^{(\prime)}\ell^+\nu_\ell$
decay dynamics therefore provide an opportunity to determine $|V_{cs}|$ and $f_+^{\eta^{(\prime)}}(0)$~\cite{Riggio:2017zwh,Zhang:2018jtm,Fang:2014sqa}.
The $f^{\eta^{(\prime)}}_+(0)$ are calculated with 
lattice quantum-chromodynamics~(QCD)~\cite{Bali:2014pva}, QCD light-cone sum rules~\cite{Hu:2021zmy,Offen:2013nma,Duplancic:2015zna,Azizi:2010zj}, traditional and covariant light-front quark models~\cite{Verma:2011yw,Cheng:2017pcq}, constituent quark model~\cite{Melikhov:2000yu}, covariant confined quark model~\cite{Soni:2018adu,Ivanov:2019nqd}, QCD sum rules~\cite{Colangelo:2001cv}, with results in a wide range $0.432-0.78~(0.404-0.78)$. Determinations of $|V_{cs}|$ and $f^{\eta^{(\prime)}}_+(0)$ are crucial to test CKM matrix unitarity and validate FF calculations.  References~\cite{Koponen:2012di,Koponen:2013tua,FermilabLattice:2022gku,HeavyFlavorAveragingGroup:2022wzx} state
that the FFs of the SL decays mediated via $c\to s(d)\ell^+\nu_\ell$ are insensitive to spectator quarks. Specifically, the FFs of $D^{0(+)}\to \bar K\ell^+\nu_\ell$ and $D_s^+\to\eta\ell^+\nu_\ell$ are expected to agree within 3\%~\cite{Koponen:2012di}.  Comparisons of the measured FFs of $D^+_s\to \eta\mu^+\nu_\mu$ and $D^{0(+)}\to \bar K\ell^+\nu_\ell$ offer important tests on this expectation.  Once this prediction is validated, it will be helpful to constrain the QCD calculations of the FFs of the SL $B$ decays, which are crucial inputs for precise determinations of the CKM matrix elements~\cite{Koponen:2012di,Koponen:2013tua,Brambilla:2014jmp,Bailey:2012rr}.
In addition, the measured BFs allow extraction of the singlet-octet $\eta$-$\eta^{\prime}$-gluon mixing angle~\cite{Christ:2010dd,Dudek:2011tt}, $\phi_{\rm P}$,  via
$\cot^4\phi_{\rm P}=\frac{\Gamma_{D^+_s\to\eta^\prime \mu^+\nu_\mu}/\Gamma_{D^+_s\to\eta
    \mu^+\nu_\mu}}{\Gamma_{D^+\to\eta^\prime \mu^+\nu_\mu}/\Gamma_{D^+\to\eta \mu^+\nu_\mu}}$, in which a possible gluon component cancels~\cite{DiDonato:2011kr}. A precise value of $\phi_{P}$ is useful for the understanding of non-perturbative QCD confinement. 

Previously, BESIII reported the BFs of $D^+_s\to\eta^{(\prime)}\mu^+\nu_\mu$~\cite{Ablikim:2018} with large uncertainties of $20~(51)\%$ using 0.482 fb$^{-1}$ of $e^+e^-$ collision data taken at a center-of-mass energy $E_{\rm CM}=4.009$ GeV. In this Letter, we report significantly improved measurements of the BFs of $D^+_s\to \eta^{(\prime)}\mu^+\nu_\mu$ and investigate for the first time their decay dynamics,
by analyzing 7.33~fb$^{-1}$ of $e^+e^-$ collision data taken at $E_{\rm CM}$ between 4.128 and 4.226 GeV with the BESIII
detector.  Furthermore, we test $\mu$-$e$ LFU in $D^+_s\to \eta^{(\prime)}\ell^+\nu_\ell$ decays 
in the full kinematic range and several $q^2$ intervals. Charge-conjugate modes are implied throughout this Letter. 

A description of the design and performance of the BESIII detector can be found in
Ref.~\cite{Ablikim2010345}. 
About 83\% of the data analyzed in this Letter profits from an upgrade of the end cap time-of-flight system
with multi-gap resistive plate chambers with a time resolution of
60\,ps~\cite{Lxin,Gyingxiao}.
 Monte Carlo (MC) simulated events are generated with a
{\sc{geant4}}-based~\cite{Agostinelli:2002hh} simulation software, which includes
the geometric description~\cite{Huang:2022wuo} and a simulation of the response of the detector. An inclusive MC sample with an equivalent luminosity of 40 times that of the data is produced at $E_{\rm CM}$ between 4.128 and 4.226~GeV. It
includes open charm processes, initial state radiation (ISR) production of charmonium [$\psi(3770)$, $\psi(3686)$ and $J/\psi$], $q\bar q\,(q=u,\,d,\,s$) continuum processes, along with Bhabha
scattering, $\mu^+\mu^-$, $\tau^+\tau^-$ and $\gamma\gamma$ events.  The open charm processes are
generated using {\sc{conexc}}~\cite{Ping:2013jka}. The effects of ISR and final state radiation are included. 
 Signal MC samples of the SL decays $D^+_s\to \eta^{(\prime)} \mu^+\nu_\mu$ are simulated with the two-parameter series expansion
model~\cite{Chikilev:1999zn}, with parameters obtained in this work. 
 The input cross section of $e^+e^-\to D^\pm_sD^{*\mp}_s$ is taken from Ref.~\cite{crosssection}.
In the MC generation, known particle decays are generated by {\sc{evtgen}}~\cite{ref:evtgen} with the BFs taken from the Particle Data Group~\cite{PDG2022}, and other modes are generated using {\sc{lundcharm}} \cite{ref:lundcharm}.

In $e^+e^-$ collisions at $E_{\rm CM}$ between 4.128 and 4.226~GeV, the $D_s^+$ mesons are produced
predominantly via $e^+e^-\to D^\pm_sD_s^{*\mp}$.
Candidates in which one $D_s^-$ is fully reconstructed in one of the fourteen hadronic decay modes,
$D^-_s\to K^+K^-\pi^-$, $K^+K^-\pi^-\pi^0$, $K^0_SK^-$,
$K^0_SK^-\pi^0$,
$K^0_SK^0_S\pi^-$,
$K^0_SK^+\pi^-\pi^-$,
$K^0_SK^-\pi^+\pi^-$,
$\pi^+\pi^-\pi^-$,
$\eta_{\gamma\gamma}\pi^-$, $\eta_{\pi^0\pi^+\pi^-}\pi^-$,
$\eta^\prime_{\eta_{\gamma\gamma}\pi^+\pi^-}\pi^-$, $\eta^\prime_{\gamma\rho^0}\pi^-$,
$\eta_{\gamma\gamma}\rho^-$, and
$\eta_{\pi^+\pi^-\pi^0}\rho^-$,
 are called the single-tag (ST) $D^-_s$. 
 Those in which the ST $D^-_s$, the transition $\gamma(\pi^0)$ of the $D_s^{*\mp}$ decay and the signal $D^+_s$ decay of interest
are simultaneously reconstructed are called double-tag~(DT) events.
Based on these, we determine the BF of the signal decay by
\begin{equation}
{\mathcal B}_{\rm sig} = N_{\rm DT}/(N^{\rm tot}_{\rm ST}\cdot \epsilon_{\gamma(\pi^0)\rm sig}),
\end{equation}
where $N^{\rm tot}_{\rm ST}=\sum_k N^k_{\rm ST}$ and $N_{\rm DT}$ are the ST and DT yields in data,
and $\epsilon_{\gamma(\pi^0)\rm sig}$ is the effective signal efficiency of selecting $\gamma(\pi^0)\eta^{(\prime)}\mu^+\nu_\mu$
in the presence of ST $D^-_s$. The $\epsilon_{\gamma(\pi^0)\rm sig}$ is the averaged efficiency of $\epsilon_{\gamma\rm sig}$ and $\epsilon_{\pi^0\rm sig}$, and estimated by
$\sum_k\frac{N^k_{\rm ST}}{N^{\rm tot}_{\rm ST}}\frac{\epsilon^k_{\rm DT}}{\epsilon^k_{\rm ST}}$,
where $\epsilon^k_{\rm ST}$ and $\epsilon^k_{\rm DT}$ are the ST and DT efficiencies for the $k$-th tag mode, respectively. 

The selection criteria for all ST candidates are the same as Ref.~\cite{BAM595}, and a detailed description can be found in this reference.
For each tag mode, the ST yield is extracted from a fit to the corresponding invariant mass ($M_{\rm tag}$) spectrum of the ST candidates. The details of the fit results can be found in Ref.~\cite{BAM595}.
Summing over all tag modes, we obtain the total ST yield $N^{\rm tot}_{\rm ST}=(817.0\pm3.4_{\rm stat})\times 10^3$.

We select candidates for the transition $\gamma\,(\pi^0)$ from $D^{*+}_s$ 
decay among the unused particles recoiling against the ST $D_s^-$, 
by using the kinematic variable 
$\Delta E\equiv E_{D^-_s} + E_{\gamma(\pi^0)} + \sqrt{|\vec{p}_{\gamma(\pi^0)}+\vec{p}_{D^-_s}|^2c^2 + m^2_{D_s^+}c^4} - E_{\rm CM}$,
where $E_{\gamma, \pi^0, D_s^-}$ and $\vec{p}_{\gamma, \pi^0, D_s^-}$ are
the energy and momentum of $\gamma$, $\pi^0$, or $D_s^-$.
All the $\gamma$ or $\pi^0$ candidates unused in the tag selection are looped over.
If multiple candidates survive the selection, the one giving the minimum $|\Delta E|$ is kept for further analysis.  No further requirement on $|\Delta E|$ is applied.

In the presence of ST $D_s^-$ and transition $\gamma\,(\pi^0)$,
signal decay candidates are selected with the remaining unused tracks.
Particle identification (PID) for muons combines the information of the specific ionization energy loss in the multilayer drift chamber,
the flight time in the time-of-flight system, and the energy deposited in the electromagnetic
calorimeter~(EMC). Likelihoods under various particle hypotheses (${\mathcal L}_i$, $i=e$, $\mu$, and $K$) are calculated.
Charged tracks satisfying ${\mathcal L}_\mu>0.001$, ${\mathcal L}_\mu>{\mathcal L}_e$, ${\mathcal L}_\mu>{\mathcal L}_K$,
and $E_{\rm EMC} \in (0.10, 0.28)~\rm GeV$ are assigned as muon candidates,
where $E_{\rm EMC}$ is the energy deposited in the EMC of muon candidates.
After also reconstructing the signal-side $\eta^{(\prime)}$ with the same criteria used for the ST daughters, the DT candidates are vetoed if they contain any additional charged tracks ($N_{\rm extra}^{\rm char}$)  or $\pi^0$ reconstructed by two unused photons ($N_{\rm extra}^{\pi^0}$). Furthermore, the energy of any unused shower ($E_{\rm \gamma~extra }^{\rm max}$) in an event is required to be less  than 0.2\,GeV.
The energy and momentum of the missing neutrino of the signal SL decay are derived
as $E_\nu \equiv E_{\rm CM} - \Sigma_i E_i$ and $\vec{p}_\nu \equiv -\Sigma_i \vec{p}_{i}$, respectively, where
$E_i$ and $\vec{p}_i$ are the energy and momentum of the particle $i$, with $i$ running over
the ST $D^-_s$, the transition $\gamma\,(\pi^0)$, and the $\eta^{(\prime)}$ and $\mu^+$ of the signal side.
To further reject the peaking backgrounds from $D_s^+\to\eta^{(\prime)} \pi^+$,
the invariant masses of $\eta^{(\prime)} \mu^+$, $M_{\eta^{(\prime)} \mu^+}$, are required to be less than 1.8~GeV/$c^2$.
Moreover, the peaking backgrounds from $D_s^+\to\eta^{(\prime)} \pi^+\pi^0$ are rejected by requiring the invariant masses of $\eta^{(\prime)} \nu_\mu$, $M_{\eta^{(\prime)} \nu_\mu}$,  to be greater than 0.97(1.27)~GeV/$c^2$.


The yield of  signal events is determined by a fit to the distribution of the kinematic variable
$ {\rm MM}^2 \equiv E_\nu^2/c^4 - |\vec{p}_\nu|^2/c^2$.
To improve the $\rm MM^2$ resolution, the candidate tracks, along with the missing neutrino, are subjected to a 3-constraint kinematic fit requiring energy and momentum conservation, constraining the invariant mass of each  $D_s^\pm$ meson to the known $D_s^\pm$ mass, and constraining the invariant
mass of the $D_s^-\gamma(\pi^0)$ or $D_s^+\gamma(\pi^0)$ combination to the known $D_s^{*\pm}$ mass. The combination with the lower $\chi^2$ is kept. 
The $\chi^2$ for $D_s^+\to\eta^\prime_{\gamma\rho^0}\mu^+\nu_\mu$ is required to be less than 30 to further suppress the non-$D_s^\pm D_s^{*\mp}$ backgrounds.

After imposing all above selection criteria, the resulting MM$^2$ distributions of the accepted candidates for
the various signal decay modes are exhibited in Fig.~\ref{fig:signal_yields_fromdata}.
For $D_s^+\to\eta^{(\prime)} \mu^+\nu_\mu$, the signal yield is extracted from a simultaneous unbinned maximum likelihood fit
to the MM$^2$ spectra for the two $\eta$ or $\eta^\prime$ reconstruction modes,
with BFs constrained to be the same in the fit.
The signal, peaking background of $D_s^+\to\eta^{(\prime)}\pi^+(\pi^0)$, and other background shapes are modeled by the individual simulated shapes taken from the inclusive MC sample.  The signal and peaking background shapes are convolved with a Gaussian resolution function to account for the differences between data and simulation.  The yields of the peaking backgrounds are fixed to the expectation from simulation, and the other yields are left free. 
The obtained signal efficiencies, signal yields, and resultant BFs are shown in Table~\ref{table:br}. The signal efficiencies have been corrected for small data-MC differences (overall factor $f^{\rm cor}=0.992 \sim 1.018$) due to $\eta, \pi^0$ reconstruction, $\mu^+$ PID, $E_{\gamma,~\rm extra}^{\rm max}, N_{\rm extra}^{\rm char}\, N_{\rm extra}^{\pi^0}$,  $M_{\eta^{(\prime)}\mu^+}, M_{\eta^{(\prime)}\nu_\mu}$, and $\chi^2$ requirements.

% Figure environment removed


   \begin{table}[hbtp]
   \centering
    \caption{\small Signal efficiencies ($\epsilon_{\rm \gamma(\pi^0){\rm sig}}$), signal yields ($N_{\rm DT}$), and obtained BFs ($\mathcal{B}_{\rm sig}$) for various semimuonic decays.  Efficiencies are averaged by $\epsilon_{\rm \gamma{\rm sig}}$ and $\epsilon_{\rm \pi^0{\rm sig}}$, and  include the BFs of the $\eta^{(\prime)}$ and $D_s^{*\mp}$ sub-decays.
    Numbers in the first and second parentheses are the most significant digits of the statistical and systematic uncertainties, respectively.  \label{table:br}}
           \begin{tabular}[t]{c|cc|cc}\hline\hline
Decay&\multicolumn{2}{c|}{$\eta\mu^+\nu_\mu$}&\multicolumn{2}{c}{$\eta^\prime\mu^+\nu_\mu$}\\
\hline
$\eta^{(\prime)}$ decay&$\gamma\gamma$& $\pi^0\pi^+\pi^-$&$\eta\pi^+\pi^-$&$\gamma\rho^0$\\
$\epsilon_{\rm \gamma(\pi^0){\rm sig}}$~(\%)& 14.05(02)&2.89(01)&2.27(01)&3.64(01)\\
$N_{\rm DT}$& \multicolumn{2}{c|}{3098(71)}& \multicolumn{2}{c}{387(27)} \\
$\mathcal{B}_{\rm sig}$~(\%)&\multicolumn{2}{c|}{2.215(51)(52)}&\multicolumn{2}{c}{0.801(55)(31)}\\
\hline\hline
        \end{tabular}
\end{table}

The systematic uncertainties in the BF measurements are listed in Table 1 of Ref.~\cite{Supplement} and are discussed below.

The uncertainty in the ST $D^{-}_s$ yields is studied 
by examining the change of the ST $D^{-}_s$ yields by varying the matched angle for signal shape and the order of Chebychev function for background shape.
The uncertainties in the tracking or PID efficiencies of $\pi^\pm$ and $\mu^+$ are studied with the
control samples of $e^+e^-\to K^+K^-\pi^+\pi^-$ and $e^+e^-\to \gamma \mu^+\mu^+$, respectively. 
The uncertainty of $\pi^0$ or $\eta$ reconstruction is assigned 
by studying the control sample of $e^+e^-\to K^+K^-\pi^+\pi^-\pi^0$. 
The uncertainty in the transition $\gamma(\pi^0)$ reconstruction is studied with the control sample of  $J/\psi\to \pi^0\pi^+\pi^-$~\cite{Ablikim:2011kv}.
The  uncertainty from the selection of the transition $\gamma(\pi^0)$ from $D_s^{*+}$ with the smallest $|\Delta E|$ method is estimated by using the control samples of $D_s^+\to K^+K^-\pi^+$ and $D_s^+\to\eta\pi^+\pi^0$.


The uncertainties due to the signal model are estimated 
by comparing the DT efficiencies by varying the input hadronic FFs measured in this Letter by $\pm 1\sigma$. 
The uncertainties due to the $M_{\eta^{(\prime)}\mu^+}$ and $M_{\eta^{(\prime)}\nu_\mu}$ requirements are estimated by using the DT events of
$D_s^+\to\eta^{(\prime)}e^+\nu_e$. 
The uncertainties from the $\eta^{(\prime)}$ reconstruction are estimated 
by analyzing the control sample of $J/\psi\to \phi\eta^{(\prime)}$.
The systematic uncertainties in the ${\rm MM}^2$ fit are studied 
by repeating the fits with different signal and background shapes.
The uncertainty in the peaking background yield is propagated 
by varying its size by $\pm1 \sigma$ of the corresponding BF~\cite{PDG2022}.

The uncertainties due to different multiplicities of tag environments~\cite{Ablikim:2018jun} are assigned by studying of data-MC efficiency differences.  
The uncertainty due to the finite MC statistics, which is dominated by that of the DT efficiency, is considered as a systematic uncertainty.

The uncertainties of the $E_{\rm \gamma~extra }^{\rm max}$, $N_{\rm extra}^{\pi^0}$, and
 $N_{\rm extra}^{\rm char}$ requirements are analyzed with  the
 DT events of $D_s^+\to\eta^{(\prime)}\pi^+(\pi^0)$ and $D_s^+\to\eta^{(\prime)}e^+\nu_e$.
The uncertainties of the $\chi^2$ requirements for $D_s^+\to\eta^{\prime}_{\gamma\rho^0}\mu^+\nu_\mu$ are studied with 
the DT events of $D_s^+\to\eta^{\prime}_{\gamma\rho^0}\pi^+$ and $D_s^+\to\eta^{\prime}_{\gamma\rho^0}e^+\nu_e$.
The uncertainties due to the quoted BFs of $\eta^{(\prime)}$ and $D_s^{*+}$ decays are cited from Ref.~\cite{PDG2022}.

The systematic uncertainties on the ST $D^{-}_s$ yields, $\mu^+$ and $\pi^\pm$ tracking and PID, photon selection, $\pi^0$ or $\eta$ reconstruction, peaking backgrounds, selection of the transition $\gamma(\pi^0)$, signal model, and $M_{\eta^{(\prime)}\mu^+}\&M_{\eta^{(\prime)}\nu_\mu}$ are correlated uncertainties between the two $\eta^{(\prime)}$ decay modes. The remaining uncertainties are uncorrelated. 
The total correlated systematic uncertainties are 2.2\% and 3.1\% for $D_s^+\to\eta\mu^+\nu_\mu$ and $D_s^+\to\eta^\prime\mu^+\nu_\mu$,  respectively; the total uncorrelated systematic uncertainties are 0.9\%, 2.1\%, 2.8\%, and 3.2\% for $D_s^+\to\eta_{\gamma\gamma}\mu^+\nu_\mu$, $D_s^+\to\eta_{\pi^0\pi^+\pi^-}\mu^+\nu_\mu$, $D_s^+\to\eta^\prime_{\eta\pi^+\pi^-}\mu^+\nu_\mu$, and $D_s^+\to\eta^\prime_{\gamma\rho^0}\mu^+\nu_\mu$, respectively.
The combined systematic uncertainties are 2.4\% for $D_s^+\to\eta\mu^+\nu_\mu$ and 3.8\% for $D_s^+\to\eta^{\prime}\mu^+\nu_\mu$, taking into account  correlated and uncorrelated systematic uncertainties with the method described in Ref.~\cite{Schmelling:1994pz}. 

The decay dynamics of $D^+_s\to\eta^{(\prime)} \mu^+\nu_\mu$ are investigated by dividing
individual candidate events into $m=8(3)$ intervals of $q^2$. Using the measured and theoretically expected partial decay rates in the $i$-th $q^2$ interval,
$\Delta\Gamma^i_{\rm msr}$ and $\Delta\Gamma^i_{\rm th}$,
we reconstruct a $\chi^2_{\rm FF}$ as
\begin{equation}
\small
    \chi^2_{\rm FF} = \underset{i,j=1}{\sum^{m}}(\Delta \Gamma^i_{\rm msr} - \Delta \Gamma^i_{\rm th})
    C_{ij}^{-1}
    (\Delta \Gamma^j_{\rm msr} - \Delta \Gamma^j_{\rm th}),
    \label{eq:chi_square}
\end{equation}
where $C_{ij}$ is the covariance matrix including correlations of $\Delta\Gamma^i_{\rm msr}$ among $q^2$ intervals.
The $\Delta\Gamma^i_{\rm th}$ relate to the hadronic FF via~\cite{Faustov:2019mqr}

\begin{widetext}
\begin{equation}
\begin{array}{l}
        \displaystyle \frac{d\Gamma}{dq^2} =
   \frac{G_{F}^{2}|V_{cs}|^{2}}{24\pi^{3}}\frac{(q^{2}-m^{2}_{\mu})^2|p_{\eta^{(\prime)}}|}{q^{4}m^{2}_{D_s^+}}
\left [(1+\frac{m^{2}_{\mu}}{2q^{2}})m^{2}_{D_s^+}|p_{\eta^{(\prime)}}|^2|f^{\eta^{(\prime)}}_{+}(q^{2})|^{2}   
+\frac{3m^{2}_{\mu}}{8q^{2}}(m^{2}_{D_s^+}-m^{2}_{\eta^{(\prime)}})^{2}|f^{\eta^{(\prime)}}_{0}(q^{2})|^{2}\right ],
\end{array}
\end{equation}
\end{widetext}
where  $G_F$ is the Fermi coupling constant~\cite{PDG2022}, $|p_{\eta^{(\prime)}}|$ is the 3-momentum of $\eta^{(\prime)}$ in the $D_s^+$ rest frame, $m_{\mu(\eta^{(\prime)})}$ is the $\mu^+(\eta^{(\prime)})$ mass. The parameters of hadronic FFs are extracted by minimizing $\chi^2_{\rm FF}$.

The hadronic FFs $f_+^{\eta^{(\prime)}}(q^2)$ are parameterized with one of three models.
The first (Model I) is a two-parameter series expansion~\cite{Becher:2005bg}, the second (Model II) is a modified pole parametrization~\cite{Becirevic:1999kt}, and the third (Model III) is a simple pole parametrization~\cite{Becirevic:1999kt}.
We fix the pole mass at the known $D^{*+}_s$ mass~\cite{PDG2022}. The same formulas are applied for $f_0^{\eta^{(\prime)}}(q^2)$ after replacing the pole mass with $m_{D_s^{*0}}$.  


The $\Delta\Gamma^i_{\rm msr}$ are determined by
$\Delta \Gamma^i_{\rm msr} \equiv \int_i\frac{d\Gamma}{dq^2}dq^2 = \frac{N_{\rm prd}^i}{\tau_{D_s^+}
  \cdot N^{\rm tot}_{\rm ST}}$, where
$\tau_{D_s^+}$ is the $D_s^+$ meson lifetime~\cite{PDG2022,Aaij:2017vqj}
and $N^i_{\rm prd}= \sum^{m}_{j}(\epsilon^{-1})_{ij} N_{\rm DT}^{j}$ is the corresponding produced signal yield.
 The observed signal yield ($N^j_{\rm DT}$) is obtained from a similar fit of the corresponding MM$^2$ distribution. The signal efficiency matrix ($\epsilon_{ij}$) is determined via
$\epsilon_{ij} = \sum_k\left[(1/N_{\rm ST}^{\rm tot})\cdot (N^{ij}_{\rm DT}/N^j_{\rm gen})_k
\cdot (N^k_{\rm ST}/\epsilon^k_{\rm ST})\cdot f^{\rm cor}\right]$,
where
$N^j_{\rm gen}$ is the total signal yield produced in the $j$-th $q^2$ interval,
$N^{ij}_{\rm rec}$ is the number of events generated in the $j\text{-}$th $q^{2}$ interval and reconstructed in the $i$-th $q^{2}$ interval,
and $k$ sums over all tag modes.
Details of $q^2$ divisions, the weighted signal efficiency matrices, $N^i_{\rm DT}$, $N^i_{\rm prd}$, and $\Delta\Gamma^i_{\rm msr}$ of various $q^2$ intervals  for $D^+_s\to \eta \mu^+\nu_\mu$ and $D^+_s\to \eta^\prime \mu^+\nu_\mu$ are shown in Tables 2-5 of Ref.~\cite{Supplement}, respectively.

The statistical and systematic covariance matrices are constructed as
$C_{ij}^{\rm stat} = (\frac{1}{\tau_{D_s^+}N_{\rm ST}^{\rm
    tot}})^2\sum_{\alpha}\epsilon_{i\alpha}^{-1}\epsilon_{j\alpha}^{-1}[\sigma(N^{\alpha}_{\rm
  DT})]^2$ and
$C_{ij}^{\rm syst} = \delta(\Delta \Gamma^i_{\rm msr})\delta(\Delta \Gamma^j_{\rm msr})$,
respectively,
where $\sigma(N^\alpha_{\rm DT})$ and $\delta(\Delta\Gamma^i_{\rm msr})$
are the statistical and systematic uncertainties in the $\alpha$-th and $i$-th $q^2$ interval.
The $C_{ij}^{\rm syst}$ is obtained by summing the covariance matrices
for all systematic uncertainties, where
a systematic uncertainty on $\tau_{D^+_s}$, 0.8\%~\cite{PDG2022,Aaij:2017vqj},
is included with those from the BF measurements.
The obtained $C_{ij}^{\rm stat}$ and $C_{ij}^{\rm syst}$ for various signal decays are shown in
Tables 6-7 of Ref.~\cite{Supplement}.
The final $C_{ij}$ is obtained by $C_{ij}=C_{ij}^{\rm stat}+C_{ij}^{\rm syst}$.

For each signal decay, we perform a simultaneous fit on the differential decay rates measured by the two $\eta^{(\prime)}$ sub-decays,
where the two modes are constrained to have same parameters for the hadronic FF.
Figure~\ref{fig:combine_FF} shows the results of the fits to the differential decay rates and the extracted hadronic FFs.
The systematic uncertainty of $\tau_{D^+_s}$ is correlated and the classification of other systematic uncertainties are assumed the same as the BF measurements.
The obtained parameters of hadronic FFs are summarized in Table~\ref{tab:FF_combine}.

Combining the BFs measured in this work with our measurements
${\mathcal B}_{D^+_s\to \eta e^+\nu_e}=(2.251\pm0.039_{\rm stat}\pm0.051_{\rm syst})\%$ and
${\mathcal B}_{D^+_s\to \eta^\prime e^+\nu_e}=(0.810\pm0.038_{\rm stat}\pm0.024_{\rm syst})\%$~\cite{BAM595},
we obtain
${\mathcal R}^{\eta}_{\mu/e}=0.984\pm0.028_{\rm stat}\pm0.016_{\rm syst}$ and
${\mathcal R}^{\eta^\prime}_{\mu/e}=0.989\pm0.082_{\rm stat}\pm0.034_{\rm syst}$,
which are consistent with the SM predictions~\cite{Hu:2021zmy,Ivanov:2019nqd,Cheng:2017pcq}.
In addition, we examine the ${\mathcal R}^{\eta}_{\mu/e}$ and ${\mathcal R}^{\eta^\prime}_{\mu/e}$
in various $q^2$ intervals after considering the correlated uncertainties, with results shown in the third row of Fig.~\ref{fig:combine_FF}; these are also consistent with the SM predictions.


% Figure environment removed

\begin{table}
\centering
\caption{\small Fitted parameters of hadronic FFs.
Quantities in the first (second) parentheses are the least two significant digits of statistical (systematic) uncertainties. NDF is the number of degrees of freedom.\label{tab:FF_combine}}
\begin{tabular}{lcccc}\hline\hline
Decay & Model & $f_+^{\eta^{(\prime)}}(0)|V_{cs}|$ &  $M_{\rm pole}$/$\alpha$/$r_1$ &$\chi^2_{\rm FF}$/NDF\\ \hline
\multirow{3}{*}{$\eta \mu^+ \nu_\mu$}
                         &I&$0.451(10)(08)$  &$-2.91(58)(29)$ &5.2/14\\
                         &II &$0.450(10)(08)$  &$0.42(11)(05)$ &5.0/14\\
                         & III &$0.456(09)(07)$  &$1.86(05)(02)$ &4.6/14\\

                         \hline
\multirow{3}{*}{$\eta^\prime\mu^+\nu_\mu$}
                         & I &$0.506(37)(11)$   &$ -10.6(53)(13)$ &2.4/4  \\
                         &II &$0.513(32)(11)$   &$ 1.12(58)(15)$ &2.6/4  \\
                         &III&$0.519(30)(11)$   &$1.51(18)(06)$ &2.7/4  \\
                         \hline\hline
    \end{tabular}
\end{table}

In summary, we present for the first time the observation of $D^+_s\to \eta^\prime \mu^+\nu_\mu$ and the dynamic studies of $D^+_s\to \eta^{(\prime)}\mu^+\nu_\mu$. The BFs of $D^+_s\to \eta \mu^+\nu_\mu$ and $D^+_s\to \eta^\prime \mu^+\nu_\mu$ are measured with precision improved by  factors of 5.9 and 6.5 over the previous best measurements~\cite{PDG2022}, respectively.  Combining with the BESIII measurements of $D_s^+\to\eta^{(\prime)}e^+\nu_e$~\cite{BAM595}, we calculate the $\mathcal R_{\mu/e}^{\eta^{(\prime)}}$ ratios in separate $q^2$ intervals and in the full range.
No significant evidence of LFU violation is found with current statistics.
Combining with the BFs of $D^+ \to\eta \mu^+ \nu_\mu$~\cite{PDG2022} and
$D^+ \to\eta^\prime \mu^+ \nu_\mu$~\cite{Dptoetapmv}, we determine the $\eta-\eta^\prime$ mixing angle $\phi_{\rm P}=(40.2\pm2.1_{\rm stat}\pm0.7_{\rm syst})^\circ$.
By analyzing the $D^+_s\to \eta^{(\prime)} \mu^+\nu_\mu$ decay dynamics, 
we obtain the products $f_+^{\eta^{(\prime)}}(0)|V_{cs}|$.
Taking $|V_{cs}|$ from the SM global fit~\cite{PDG2022} as input,
we determine $f^{\eta}_+(0) = 0.463\pm0.010_{\rm stat}\pm0.008_{\rm syst}$ and $f^{\eta^{\prime}}_+(0)=0.520\pm0.038_{\rm stat}\pm0.011_{\rm syst}$ for the two-parameter series expansion. 
The results are consistent with the QCD light-cone and QCD sum rule calculations~\cite{Hu:2021zmy,Colangelo:2001cv,Azizi:2010zj,Offen:2013nma,Duplancic:2015zna}. They disfavor the lattice QCD, covariant light-cone, and covariant confined quark model calculations~\cite{Cheng:2017pcq,Ivanov:2019nqd,Bali:2014pva,Verma:2011yw,Melikhov:2000yu,Soni:2018adu} by more than $4\sigma$. 
Unlike the comparable FFs in the SL decays mediated via $c\to d e^+\nu_e$~\cite{BESIII:2018xre},  the $D^+_s\to \eta$ FFs measured in this work deviates with that of $f_+^{D\to K}(0)=0.7327\pm0.0049$ obtained via $D^0\to K^-\mu^+\nu_\mu$ ~\cite{BESIII:2018ccy} by more than $5\sigma$. This rules out the expectation for comparable FFs for $D^{0(+)}\to \bar K \mu^+\nu_\mu$ and $D^+_s\to \eta \mu^+\nu_\mu$~\cite{Koponen:2012di}.
Conversely, by taking the
$f^{\eta^{(\prime)}}_+(0)$ predicted by theory~\cite{Duplancic:2015zna} as inputs, we obtain $|V_{cs}|_\eta=0.911\pm0.020_{\rm stat}\pm0.016_{\rm syst}{^{+0.055}_{-0.053}}_{\rm theo}$ and $|V_{cs}|_{\eta^\prime}=0.907\pm0.067_{\rm stat}\pm0.019_{\rm syst}{^{+0.076}_{-0.073}}_{\rm theo}$, where the third uncertainty originates from the input FFs.


\begin{acknowledgments}

%\textbf{Acknowledgement}
The authors thank Prof. Haibing Fu, Prof. Shan Cheng, Prof. Xianwei Kang, and Prof. Khlopov for helpful dis- cussions.
The BESIII Collaboration thanks the staff of BEPCII and the IHEP computing center for their strong support. This work is supported in part by National Key R\&D Program of China under Contracts No. 2020YFA0406400, No. 2023YFA1606000, No. 2023YFA1606704, and No. 2020YFA0406300; National Natural Science Foundation of China (NSFC) under Contracts No. 12305089, No. 12375092, No. 11635010, No. 11735014, No. 11835012, No. 11935015, No. 11935016, No. 11935018, No. 11961141012, No. 12022510, No. 12025502, No. 12035009, No. 12035013, No. 12061131003, No. 12192260, No. 12192261, No. 12192262, No. 12192263, No. 12192264, No. 12192265; 
the Chinese Academy of Sciences (CAS) Large-Scale Scientific Facility Program; the CAS Center for Excellence in Particle Physics (CCEPP); Joint Large-Scale Scientific Facility Funds of the NSFC and CAS under Contract No. U1932102, U1832207; CAS Key Research Program of Frontier Sciences under Contracts Nos. QYZDJ-SSW-SLH003, QYZDJ-SSW-SLH040; 100 Talents Program of CAS; 
Jiangsu Funding Program for Excellent Postdoctoral Talent under Contracts No. 2023ZB833; 
Project funded by China Postdoctoral Science Foundation under Contracts No. 2023M732547;
The Institute of Nuclear and Particle Physics (INPAC) and Shanghai Key Laboratory for Particle Physics and Cosmology; ERC under Contract No. 758462; European Union's Horizon 2020 research and innovation programme under Marie Sklodowska-Curie grant agreement under Contract No. 894790; German Research Foundation DFG under Contracts Nos. 443159800, 455635585, Collaborative Research Center CRC 1044, FOR5327, GRK 2149; Istituto Nazionale di Fisica Nucleare, Italy; Ministry of Development of Turkey under Contract No. DPT2006K-120470; National Research Foundation of Korea under Contract No. NRF-2022R1A2C1092335; National Science and Technology fund; National Science Research and Innovation Fund (NSRF) via the Program Management Unit for Human Resources \& Institutional Development, Research and Innovation under Contract No. B16F640076; Polish National Science Centre under Contract No. 2019/35/O/ST2/02907; Suranaree University of Technology (SUT), Thailand Science Research and Innovation (TSRI), and National Science Research and Innovation Fund (NSRF) under Contract No. 160355; The Royal Society, UK under Contract No. DH160214; The Swedish Research Council; U. S. Department of Energy under Contract No. DE-FG02-05ER41374.

%\textbf{Other Fund Information}
%
%To be inserted with an additional sentence into papers that are relevant to the topic of special funding for specific topics. Authors can suggest which to Li Weiguo and/or the physics coordinator.
%        Example added sentence: This paper is also supported by the NSFC under Contract Nos. 10805053, 10979059, ....National Natural Science Foundation of China (NSFC), 10805053, PWANational Natural Science Foundation of China (NSFC), 10979059, Lund弦碎裂强子化模型及其通用强子产生器研究National Natural Science Foundation of China (NSFC), 10775075, National Natural Science Foundation of China (NSFC), 10979012, baryonsNational Natural Science Foundation of China (NSFC), 10979038, charmoniumNational Natural Science Foundation of China (NSFC), 10905034, psi(2S)->B BbarNational Natural Science Foundation of China (NSFC), 10975093, D 介子National Natural Science Foundation of China (NSFC), 10979033, psi(2S)->VPNational Natural Science Foundation of China (NSFC), 10979058, hcNational Natural Science Foundation of China (NSFC), 10975143, charmonium rare decays

\end{acknowledgments}





\begin{thebibliography}{99}


\bibitem{HFLAV}
Y.~S.~Amhis {\it et al.} (HFLAV Collaboration),
\href{https://journals.aps.org/prd/abstract/10.1103/PhysRevD.107.052008}{Phys. Rev. D {\bf 107}, 052008 (2023).}

\bibitem{Muong-2:2021ojo}
B.~Abi {\it et al.} (Muon g-2 Collaboration),
\href{https://journals.aps.org/prl/abstract/10.1103/PhysRevLett.126.141801}{Phys. Rev. Lett. {\bf 126}, 141801 (2021).}

\bibitem{Muong-2:2006rrc}
G.~W.~Bennett {\it et al.} (Muon g-2 Collaboration),
\href{https://journals.aps.org/prd/abstract/10.1103/PhysRevD.73.072003}{Phys. Rev. D {\bf 73}, 072003 (2006).}


\bibitem{Coutinho:2019aiy}
A.~M.~Coutinho, A.~Crivellin and C.~A.~Manzari,
\href{https://journals.aps.org/prl/abstract/10.1103/PhysRevLett.125.071802}{Phys. Rev. Lett. {\bf 125}, 071802 (2020).}

\bibitem{Crivellin:2020lzu}
A.~Crivellin and M.~Hoferichter,
\href{https://journals.aps.org/prl/abstract/10.1103/PhysRevLett.125.111801}{Phys. Rev. Lett. {\bf 125}, 111801 (2020).}



\bibitem{babar_1} J. P. Lees {\it et al.} (BaBar Collaboration),
\href{https://journals.aps.org/prl/abstract/10.1103/PhysRevLett.109.101802}{Phys. Rev. Lett. {\bf 109}, 101802 (2012).}

\bibitem{babar_2} J. P. Lees {\it et al.} (BaBar Collaboration),
\href{https://journals.aps.org/prd/abstract/10.1103/PhysRevD.88.072012}{Phys. Rev. D {\bf 88}, 072012 (2013).}

\bibitem{lhcb_1} R. Aaij {\it et al.} (LHCb Collaboration),
\href{https://journals.aps.org/prl/abstract/10.1103/PhysRevLett.115.111803}{Phys. Rev. Lett. {\bf 115}, 111803 (2015).}

\bibitem{belle2015} M. Huschle {\it et al.} (Belle Collaboration),
\href{https://journals.aps.org/prd/abstract/10.1103/PhysRevD.92.072014}
{Phys. Rev. D {\bf 92}, 072014 (2015).}

\bibitem{belle2016} Y. Sato {\it et al.} (Belle Collaboration),
\href{https://journals.aps.org/prd/abstract/10.1103/PhysRevD.94.072007}
{Phys. Rev. D {\bf 94}, 072007 (2016).}




\bibitem{Belle:2019rba}
G.~Caria {\it et al.}  (Belle Collaboration),
\href{https://journals.aps.org/prl/abstract/10.1103/PhysRevLett.124.161803}{Phys. Rev. Lett. {\bf 124}, 161803 (2020).}


\bibitem{LHCbreport}
R.~Aaij {\it et al.} (LHCb Collaboration), 
\href{https://arxiv.org/pdf/2302.02886.pdf}{arXiv:2302.02886} (submitted to Phys. Rev. Lett.). 





\bibitem{BhupalDev:2020zcy}
P.~S.~Bhupal Dev, R.~Mohanta, S.~Patra and S.~Sahoo,
\href{https://journals.aps.org/prd/abstract/10.1103/PhysRevD.102.095012}{Phys. Rev. D {\bf 102}, 095012 (2020).}


\bibitem{Nomura:2021oeu}
T.~Nomura and H.~Okada,
\href{https://journals.aps.org/prd/abstract/10.1103/PhysRevD.104.035042}{Phys. Rev. D {\bf 104}, 035042 (2021).}




\bibitem{Bordone:2016gaq}
M.~Bordone, G.~Isidori and A.~Pattori,
\href{https://link.springer.com/article/10.1140/epjc/s10052-016-4274-7}{Eur. Phys. J. C {\bf 76}, 440 (2016).}

\bibitem{Altmannshofer:2017yso}
W.~Altmannshofer, P.~Stangl and D.~M.~Straub,
\href{https://journals.aps.org/prd/abstract/10.1103/PhysRevD.96.055008}{Phys. Rev. D {\bf 96}, 055008 (2017).}

\bibitem{Crivellin:2017zlb}
A.~Crivellin, D.~M\"uller and T.~Ota,
\href{https://link.springer.com/article/10.1007/JHEP09(2017)040}{JHEP {\bf 09}, 040 (2017).}

\bibitem{Becirevic:2016yqi}
D.~Be\v{c}irevi\'c, S.~Fajfer, N.~Ko\v{s}nik and O.~Sumensari,
\href{https://journals.aps.org/prd/abstract/10.1103/PhysRevD.94.115021}{Phys. Rev. D {\bf 94}, 115021 (2016).}


\bibitem{BFajfer2012} S. Fajfer, J. F. Kamenik and I. Nisandzic,
\href{https://journals.aps.org/prd/abstract/10.1103/PhysRevD.85.094025}
{Phys. Rev. D. {\bf 85}, 094025 (2012).}

\bibitem{Fajfer2012} S. Fajfer {\it et al.},
\href{https://journals.aps.org/prl/abstract/10.1103/PhysRevLett.109.161801}
{Phys. Rev. Lett. {\bf 109}, 161801 (2012).}

\bibitem{Celis2013} A. Celis {\it et al.},
\href{https://link.springer.com/article/10.1007/JHEP01(2013)054}
{JHEP {\bf 1301}, 054 (2013).}

\bibitem{Crivellin2015} A. Crivellin, G. D'Ambrosio and J. Heeck,
\href{https://journals.aps.org/prl/abstract/10.1103/PhysRevLett.114.151801}
{Phys. Rev. Lett. {\bf 114}, 151801 (2015).}

\bibitem{Crivellin2016} A. Crivellin, J. Heeck and P. Stoffer,
\href{https://journals.aps.org/prl/abstract/10.1103/PhysRevLett.116.081801}
{Phys. Rev. Lett. {\bf 116}, 081801 (2016).}

\bibitem{Bauer2016} M. Bauer and M. Neubert,
\href{https://journals.aps.org/prl/abstract/10.1103/PhysRevLett.116.141802}
{Phys. Rev. Lett. {\bf 116}, 141802 (2016).}


\bibitem{Fajfer2015} S. Fajfer, I. Nisandzic and U. Rojec,
\href{https://journals.aps.org/prd/abstract/10.1103/PhysRevD.91.094009}
{Phys. Rev. D {\bf 91}, 094009 (2015).}


\bibitem{Hu:2021zmy}
D.~D.~Hu, H.~B.~Fu, T.~Zhong, L.~Zeng, W.~Cheng and X.~G.~Wu,
\href{https://link.springer.com/article/10.1140/epjc/s10052-021-09958-0}{Eur. Phys. J. C {\bf 82}, 12 (2022).}

\bibitem{Cheng:2017pcq}
H.~Y.~Cheng and X.~W.~Kang,
\href{https://link.springer.com/article/10.1140/epjc/s10052-017-5170-5}{Eur. Phys. J. C {\bf 77},  587 (2017),}
[erratum: \href{https://link.springer.com/article/10.1140/epjc/s10052-017-5423-3}{Eur. Phys. J. C {\bf 77}, 863 (2017).}]

\bibitem{Ivanov:2019nqd}
M.~A.~Ivanov, J.~G.~K\"orner, J.~N.~Pandya, P.~Santorelli, N.~R.~Soni and C.~T.~Tran,
\href{https://link.springer.com/article/10.1007/s11467-019-0908-1}{Front. Phys. (Beijing) {\bf 14}, 64401 (2019).}







\bibitem{Riggio:2017zwh}
 L.~Riggio, G.~Salerno and S.~Simula,
  \href{https://link.springer.com/article/10.1140/epjc/s10052-018-5943-5}{Eur. Phys. J. C {\bf 78}, 501 (2018).}

  \bibitem{Zhang:2018jtm}
J.~Zhang, C.~X.~Yue and C.~H.~Li,
  \href{https://link.springer.com/article/10.1140/epjc/s10052-018-6184-3}{Eur. Phys. J. C {\bf 78}, 695 (2018).}

\bibitem{Fang:2014sqa}
Y.~Fang, G.~Rong, H.~L.~Ma and J.~Y.~Zhao,
  \href{https://link.springer.com/article/10.1140/epjc/s10052-014-3226-3}{Eur.\ Phys.\ J.\ C {\bf 75}, 10 (2015).}



\bibitem{Bali:2014pva}
  G.~S.~Bali, S.~Collins, S.~D\"urr and I.~Kanamori,
  \href{https://journals.aps.org/prd/abstract/10.1103/PhysRevD.91.014503}
  {Phys.\ Rev.\ D {\bf 91}, 014503 (2015).}
  
  
  
\bibitem{Offen:2013nma}
  N.~Offen, F.~A.~Porkert and A.~Sch{\"a}fer,
  \href{https://journals.aps.org/prd/abstract/10.1103/PhysRevD.88.034023}
  {Phys.\ Rev.\ D {\bf 88}, 034023 (2013).}

\bibitem{Duplancic:2015zna}
  G.~Duplan\v{c}i\'{c} and B.~Melic,
  \href{https://link.springer.com/article/10.1007/JHEP11(2015)138}
  {JHEP {\bf 11}, 138 (2015).}
  
    \bibitem{Azizi:2010zj}
  K.~Azizi, R.~Khosravi and F.~Falahati,
  \href{https://iopscience.iop.org/article/10.1088/0954-3899/38/9/095001}
  {J.\ Phys.\ G {\bf 38}, 095001 (2011).}
  

    \bibitem{Verma:2011yw}
  R.~C.~Verma,
  \href{https://iopscience.iop.org/article/10.1088/0954-3899/39/2/025005}
  {J.\ Phys.\ G {\bf 39}, 025005 (2012).}
  
  
    \bibitem{Melikhov:2000yu}
  D.~Melikhov and B.~Stech,
  \href{https://journals.aps.org/prd/abstract/10.1103/PhysRevD.62.014006}
  {Phys.\ Rev.\ D {\bf 62}, 014006 (2000).}


\bibitem{Soni:2018adu}
 N.~R.~Soni, M.~A.~Ivanov, J.~G.~K\"orner, J.~N.~Pandya, P.~Santorelli and C.~T.~Tran,
  \href{https://journals.aps.org/prd/abstract/10.1103/PhysRevD.98.114031}
  {Phys.\ Rev.\ D {\bf 98}, 114031 (2018).}



  \bibitem{Colangelo:2001cv}
    P.~Colangelo and F.~De Fazio,
  \href{https://linkinghub.elsevier.com/retrieve/pii/S0370269301011121}
  {Phys.\ Lett.\ B {\bf 520}, 78 (2001).}


\bibitem{Koponen:2012di}
J.~Koponen {\it et al.}  (HPQCD Collaboration),
\href{https://arxiv.org/pdf/1208.6242.pdf}{arXiv:1208.6242 [hep-lat].}
\bibitem{Koponen:2013tua}
J.~Koponen {\it et al.}  (HPQCD Collaboration),
\href{https://arxiv.org/pdf/1305.1462.pdf}{arXiv:1305.1462 [hep-lat].}

\bibitem{FermilabLattice:2022gku}
A.~Bazavov {\it et al.}  (Fermilab Lattice and MILC Collaborations),
\href{https://journals.aps.org/prd/abstract/10.1103/PhysRevD.107.094516}{Phys. Rev. D {\bf 107}, 094516 (2023).}

\bibitem{HeavyFlavorAveragingGroup:2022wzx}
Y.~S.~Amhis  {\it et al.}  (Heavy Flavor Averaging Group Collaboration),
\href{https://journals.aps.org/prd/abstract/10.1103/PhysRevD.107.052008}{Phys. Rev. D {\bf 107}, 052008 (2023).}


\bibitem{Brambilla:2014jmp}
N.~Brambilla  {\it et al.} 
\href{https://link.springer.com/article/10.1140/epjc/s10052-014-2981-5}{Eur. Phys. J. C {\bf 74}, 2981 (2014).}

\bibitem{Bailey:2012rr}
J.~A.~Bailey  {\it et al.} 
\href{https://journals.aps.org/prd/abstract/10.1103/PhysRevD.85.114502}{Phys. Rev. D {\bf 85}, 114502 (2012),}

\bibitem{Christ:2010dd}
  N.~H.~Christ, C.~Dawson, T.~Izubuchi, C.~Jung, Q.~Liu, R.~D.~Mawhinney, C.~T.~Sachrajda, A.~Soni, R.~Zhou,
  \href{https://journals.aps.org/prl/abstract/10.1103/PhysRevLett.105.241601}
  {Phys.\ Rev.\ Lett.\  {\bf 105}, 241601 (2010).}

\bibitem{Dudek:2011tt}
  J.~J.~Dudek, R.~G.~Edwards, B.~Joo, M.~J.~Peardon, D.~G.~Richards and C.~E.~Thomas,
  \href{https://journals.aps.org/prd/abstract/10.1103/PhysRevD.83.111502}
  {Phys.\ Rev.\ D {\bf 83}, 111502 (2011).}
  
  \bibitem{DiDonato:2011kr}
C.~Di Donato, G.~Ricciardi and I.~I.~Bigi,
\href{https://journals.aps.org/prd/abstract/10.1103/PhysRevD.85.013016}{Phys.\ Rev.\ D {\bf 85}, 013016 (2012).}


\bibitem{Ablikim:2018}
  M.~Ablikim {\it et al.} (BESIII Collaboration),
  \href{https://journals.aps.org/prd/abstract/10.1103/PhysRevD.97.012006}
  {Phys.\ Rev.\ D {\bf 97}, 012006 (2018).}
  


\bibitem{Ablikim2010345} M. Ablikim {\it et al.} (BESIII Collaboration),
\href{https://linkinghub.elsevier.com/retrieve/pii/S0168900209023870}
{Nucl. Instrum. Methods. A {\bf 614}, 345 (2010).}

\bibitem{Lxin} X. Li {\it et al.},
\href{https://link.springer.com/article/10.1007/s41605-017-0014-2}
{Radiat. Detect. Technol. Methods {\bf 1}, 13 (2017).}

\bibitem{Gyingxiao} Y. X. Guo {\it et al.},
\href{https://link.springer.com/article/10.1007/s41605-017-0012-4}
{Radiat. Detect. Technol. Methods {\bf 1}, 15 (2017).}

\bibitem{Agostinelli:2002hh} S.~Agostinelli {\it et al.} ({\sc{geant4}} Collaboration),
\href{https://linkinghub.elsevier.com/retrieve/pii/S0168900203013688}
{Nucl.\ Instrum.\ Meth.\ A {\bf 506}, 250 (2003).}

\bibitem{Huang:2022wuo}
K.~X.~Huang, Z.~J.~Li, Z.~Qian, J.~Zhu, H.~Y.~Li, Y.~M.~Zhang, S.~S.~Sun and Z.~Y.~You,
\href{https://link.springer.com/article/10.1007/s41365-022-01133-8}{Nucl. Sci. Tech. {\bf 33}, 142 (2022).}


 \bibitem{Ping:2013jka} R.~G.~Ping,
 \href{https://iopscience.iop.org/article/10.1088/1674-1137/38/8/083001}
 {Chin. Phys. C {\bf 38}, 083001 (2014).}

\bibitem{Chikilev:1999zn}
  O.~G.~Tchikilev,
  \href{https://linkinghub.elsevier.com/retrieve/pii/S0370269399014112}
  {Phys.\ Lett.\ B {\bf 471}, 400 (2000),}
  Erratum: [\href{https://linkinghub.elsevier.com/retrieve/pii/S0370269300003245}{Phys.\ Lett.\ B {\bf 478}, 459 (2000)}].

\bibitem{crosssection}
M. Ablikim {\it et al.} (BESIII collaboration),
{``Measurement of the cross section for $e^+e^-\to D^\pm_s D^{*\mp}_{s}$ up to 4.7 GeV'', publication in preparation.}



\bibitem{ref:evtgen}  D.~J.~Lange,
\href{https://linkinghub.elsevier.com/retrieve/pii/S0168900201000894}
{Nucl. Instrum.\ Meth.\ A {\bf 462}, 152 (2001);}
  R. G. Ping,
  \href{https://iopscience.iop.org/article/10.1088/1674-1137/32/8/001}
  {Chin. Phys. C {\bf 32}, 599 (2008).}

  \bibitem{PDG2022} R.~L. Workman {\it et al.} (Particle Data Group), 
\href{https://pdglive.lbl.gov/Viewer.action}
{Prog. Theor. Exp. Phys. {\bf 2022}, 083C01 (2022).}



\bibitem{ref:lundcharm} J. C. Chen, G. S. Huang, X. R. Qi, D. H. Zhang and Y. S. Zhu,
\href{https://journals.aps.org/prd/abstract/10.1103/PhysRevD.62.034003}
{Phys. Rev. D {\bf 62}, 034003 (2000);}
R. L. Yang, R. G. Ping and H. Chen, 
\href{https://iopscience.iop.org/article/10.1088/0256-307X/31/6/061301}{Chin. Phys. Lett. {\bf 31}, 061301 (2014).}
	


 \bibitem{BAM595} M. Ablikim {\it et al.} (BESIII Collaboration), \href{https://arxiv.org/abs/2306.05194}{arXiv:2306.05194.}


   
 \bibitem{Supplement} See Supplement Material at [URL will be inserted by publisher] for the systematic uncertainties in the BF measurements, the measured partial widths in various reconstructed $q^2$ intervals of the signal side, the statistical and systematic covariance matrices for various signal decays.
  
  
 \bibitem{Ablikim:2011kv} M.~Ablikim {\it et al.} (BESIII Collaboration),
 \href{https://journals.aps.org/prd/abstract/10.1103/PhysRevD.83.112005}{Phys. Rev. D {\bf 83}, 112005 (2011).}
 
  \bibitem{Ablikim:2018jun}
  M.~Ablikim {\it et al.} (BESIII Collaboration),
  \href{https://journals.aps.org/prl/abstract/10.1103/PhysRevLett.122.071802}{Phys. Rev. Lett.  {\bf 122}, 071802 (2019).}

 
   \bibitem{Schmelling:1994pz}
 M. Schmelling, \href{https://iopscience.iop.org/article/10.1088/0031-8949/51/6/002}{Phys. Scripta {\bf 51}, 676 (1995)}.

 
\bibitem{Faustov:2019mqr}
R.~N.~Faustov, V.~O.~Galkin and X.~W.~Kang,
\href{https://journals.aps.org/prd/abstract/10.1103/PhysRevD.101.013004}{Phys. Rev. D {\bf 101},  013004 (2020).}
 

 \bibitem{Becher:2005bg}
  T.~Becher and R.~J.~Hill,
  \href{https://linkinghub.elsevier.com/retrieve/pii/S0370269305017235}
  {Phys. Lett. B {\bf 633}, 61 (2006).}


\bibitem{Becirevic:1999kt}
  D.~Becirevic and A.~B.~Kaidalov,
  \href{https://linkinghub.elsevier.com/retrieve/pii/S0370269300002902}
  {Phys.\ Lett.\ B {\bf 478}, 417 (2000).}

 \bibitem{Aaij:2017vqj}
  R.~Aaij {\it et al.} (LHCb Collaboration),
  \href{https://journals.aps.org/prl/abstract/10.1103/PhysRevLett.119.101801}
  {Phys.\ Rev.\ Lett.\  {\bf 119}, 101801 (2017).}


 
 
 \bibitem{Dptoetapmv} Under the assumption of LFU, the BF of $D^+\to\eta^\prime \mu^+\nu_\mu$ is assumed the same as the BF of $D^+\to\eta^\prime e^+\nu_e$~\cite{PDG2022}.
 
\bibitem{BESIII:2018xre}
M. Ablikim {\it et al.} (BESIII Collaboration),
\href{https://journals.aps.org/prl/abstract/10.1103/PhysRevLett.122.061801}{Phys. Rev. Lett. {\bf 122}, 061801 (2019).}

 
 \bibitem{BESIII:2018ccy}
M. Ablikim {\it et al.} (BESIII Collaboration),
\href{https://journals.aps.org/prl/abstract/10.1103/PhysRevLett.122.011804}{Phys. Rev. Lett. {\bf 122}, 011804 (2019).}







\end{thebibliography}
	
	
\clearpage
\appendix
\twocolumngrid
\setcounter{table}{0}
\setcounter{figure}{0}
	
\section*{Appendices}\label{Supplement}


Table~\ref{sys} summarizes the sources of the systematic uncertainties in the measurements of the branching fractions of $D_s^+\to\eta^{(\prime)}\mu^+\nu_\mu$. In this table, the contributions to the systematic uncertainties listed in the top part are treated as correlated, while those in the bottom part are treated as uncorrelated.


Tables~\ref{tab:effmatrixa} and~\ref{tab:effmatrixb} give the weighted efficiency matrices averaged over all fourteen ST modes for $D^+_s\to \eta \mu^+\nu_\mu$ and $D^+_s\to \eta^\prime \mu^+\nu_\mu$, respectively.

Tables~\ref{tab:decayratea} and ~\ref{tab:decayrateb} present the numbers of the  reconstructed events $N_{\rm DT}^i$ in data obtained from the $\rm MM^2$ fits, the numbers of the  produced events $N_{\rm prd}^i$, and the measured partial decay rate $\Delta\Gamma_{\rm msr}^i$  in various $q^2$ intervals for $D^+_s\to \eta \mu^+\nu_\mu$ and $D^+_s\to \eta^\prime \mu^+\nu_\mu$, respectively.

Table~\ref{tab:cova} summarizes the statistical covariance matrices for the measured partial decay rates in different $q^2$ intervals for $D^+_s\to \eta \mu^+\nu_\mu$ and $D^+_s\to \eta^\prime \mu^+\nu_\mu$. 

Table ~\ref{tab:covb} summarizes the systematic covariance matrices for the measured partial decay
rates in different $q^2$ intervals for $D^+_s\to \eta \mu^+\nu_\mu$ and $D^+_s\to \eta^\prime \mu^+\nu_\mu$. 




\begin{table*}[htp]
\centering
\caption{Relative systematic uncertainties (in \%) on the measurements of the branching fractions
of $D_s^+\to \eta \mu^+\nu_\mu$ and $D_s^+\to \eta^\prime \mu^+\nu_\mu$. The top and the bottom sections are correlated and uncorrelated, respectively. The uncertainty in the uncorrelated $\pi^\pm$ tracking is obtained as the square root of the quadratic difference of the total uncertainty in the $\pi^\pm$ tracking and the correlated portion. The last row of combined uncertainties  are total uncertainties of $D_s^+\to \eta \mu^+\nu_\mu$ and $D_s^+\to \eta^\prime \mu^+\nu_\mu$ after taking into account correlated and uncorrelated systematic uncertainties.
}
\begin{tabular}{lcc|cc}
  \hline
  \hline
  Source  & $\eta_{\gamma\gamma}\mu^+\nu_\mu$&$\eta_{\pi^0\pi^+\pi^-}\mu^+\nu_\mu$&$\eta^\prime_{\eta\pi^+\pi^-}\mu^+\nu_\mu$&$\eta^\prime_{\gamma\rho^0}\mu^+\nu_\mu$  \\
  \hline
ST $D^{-}_s$ yields                                      &0.5  &0.5  &0.5 &0.5    \\
  $\mu^+$ tracking                                             &0.3 &0.3 & 0.3&0.3        \\
  $\mu^+$ PID                                                  &0.3 &0.3 & 0.3&0.3        \\
     $\pi^\pm$ tracking                                           &--&1.2&0.6&0.6\\
  $\pi^\pm$ PID                                                &--&0.4&0.4&0.4\\

    $\pi^0$ or $\eta_{\gamma\gamma}$ reconstruction                                  &1.1  &1.1  &0.8 &--    \\
 Transition $\gamma(\pi^0)$ reconstruction      &1.0&1.0&1.0&1.0\\
 Smallest $|\Delta E|$                                            &1.0&1.0&1.0&1.0\\
  Signal model                                         &0.3&0.3&0.6 &0.6        \\
    $M_{\rm \eta^{(\prime)}\mu^+}$ and $M_{\rm \eta^{(\prime)}\nu_\mu}$ requirements                                 &0.7&0.7&2.4&2.4         \\
Peaking background &0.7&0.7&1.0&1.0\\

\hline
   $\pi^\pm$ tracking                                           &--&--&1.7&--\\
$\eta^{(\prime)}$ reconstruction &--&0.1&0.1&1.4\\
  $M_{\rm miss}^2$ fit                                                    & 0.2&0.7&1.2&1.2          \\

  Tag bias                                                     &0.5&0.2 &0.2&0.2         \\
    MC  statistics                                               &0.3     &0.3&0.3&0.3           \\
   $E_{\rm \gamma~extra }^{\rm max}$,  $N_{\rm extra}^{\rm char}$, and $N_{\rm extra}^{\pi^0}$ requirements &0.4&0.8&0.8&1.2      \\

  $\chi^2$ requirement&--&--&--&1.6\\

  Quoted branching fractions                                   &0.5&1.1&1.3&1.4\\
  \hline
  Total                                                        &2.3           &3.0  &4.2&    4.4    \\
  \hline
  Combined&\multicolumn{2}{c|}{2.4}&\multicolumn{2}{c}{3.8}\\

  \hline
  \hline
\end{tabular}
\label{sys}
\end{table*}



\begin{table*}[htbp]\centering
\caption{The efficiency matrices for $D_s^+\to\eta\mu^+\nu_\mu$ averaged over all fourteen ST modes, where $\varepsilon_{ij}$ represents the
efficiency of events produced in the $j$-th $q^2$ interval and reconstructed in the $i$-th $q^2$ interval.}
\label{tab:effmatrixa}
\begin{tabular}{c|cccccccc|cccccccc}\hline\hline
\multirow{2}{*}{$\varepsilon_{ij}~(\%)$}&\multicolumn{8}{c|}{$D_s^+\to\eta_{\gamma\gamma}\mu^+\nu_\mu$}&\multicolumn{8}{c}{$D_s^+\to\eta_{\pi^0\pi^+\pi^-}\mu^+\nu_\mu$}\\
&1&2&3&4&5&6&7&8&1&2&3&4&5&6&7&8\\
\hline
1&13.16&0.82&0.02&0.00&0.00&0.00&0.00&0.00&2.97&0.14&0.00&0.00&0.00&0.00&0.00&0.00\\
2&1.03&12.10&1.02&0.04&0.00&0.00&0.00&0.00&0.20&2.72&0.17&0.01&0.00&0.00&0.00&0.00\\
3&0.04&1.12&11.91&1.12&0.04&0.01&0.00&0.00&0.01&0.24&2.56&0.16&0.01&0.00&0.00&0.00\\
4&0.01&0.04&1.27&11.83&1.10&0.04&0.00&0.00&0.00&0.01&0.27&2.50&0.18&0.01&0.00&0.00\\
5&0.00&0.01&0.06&1.27&11.61&1.10&0.02&0.00&0.00&0.00&0.02&0.26&2.37&0.17&0.01&0.00\\
6&0.00&0.00&0.01&0.05&1.20&11.55&1.06&0.01&0.00&0.00&0.00&0.02&0.26&2.21&0.15&0.00\\
7&0.00&0.00&0.01&0.01&0.05&1.11&11.37&0.57&0.00&0.00&0.00&0.01&0.02&0.24&2.09&0.08\\
8&0.00&0.00&0.00&0.01&0.02&0.05&1.01&12.83&0.00&0.00&0.00&0.00&0.01&0.02&0.20&2.14\\
\hline\hline
\end{tabular}
\end{table*}


\begin{table*}[htbp]\centering
\caption{
The efficiency matrices for $D_s^+\to\eta^\prime\mu^+\nu_\mu$ averaged over all fourteen ST modes, where $\varepsilon_{ij}$ represents the
efficiency of events produced in the $j$-th $q^2$ interval and reconstructed in the $i$-th $q^2$ interval.
}
\label{tab:effmatrixb}
\begin{tabular}{c|ccc|ccc}\hline\hline
\multirow{2}{*}{$\varepsilon_{ij}~(\%)$}&\multicolumn{3}{c|}{$D_s^+\to\eta^\prime_{\eta\pi^+\pi^-}\mu^+\nu_\mu$}&\multicolumn{3}{c}{$D_s^+\to\eta^\prime_{\gamma\rho^0}\mu^+\nu_\mu$}\\
&1&2&3&1&2&3\\
\hline
1&2.10&0.10&0.00&3.27&0.11&0.05\\
2&0.09&2.12&0.13&0.10&3.51&0.17\\
3&0.00&0.07&2.34&0.04&0.11&3.92\\
\hline\hline
\end{tabular}
\end{table*}








\begin{table}[htp]\centering
\caption{The partial decay rates of $D_s^+\to\eta\mu^+\nu_\mu$ in various $q^{2}$ intervals of data, where the uncertainties of partial decay rates are statistical only. }
\label{tab:decayratea}
\scalebox{0.93}{
\begin{tabular}{cccccccccc}\hline\hline
\multicolumn{2}{c}{$i$}&1&2&3&4&5&6&7&8\\
\multicolumn{2}{c}{$q^2$ $(\mathrm{GeV}^{2}/c^{4})$}&($m_\mu^2,\,0.2$)&($0.2,\,0.4$)&($0.4,\,0.6$)&($0.6,\,0.8$)&($0.8,\,1.0$)&($1.0,\,1.2$)&($1.2,\,1.4$)&($1.4,\,q_{\rm max}^2$)\\
\hline
\multirow{3}{*}{$D_s^+\to\eta_{\gamma\gamma} \mu^+\nu_\mu$}&$N_{\mathrm{DT}}^i$&402(26)&456(28)&396(27)&353(24)&305(21)&239(19)&188(19)&234(21)\\
&$N_{\mathrm{prd}}^i$&2845(201)&3289(232)&2768(231)&2464(206)&2177(186)&1703(172)&1384(170)&1701(166)\\
&$\Delta\Gamma^i_{\rm msr}$ $(\mathrm{ns^{-1}})$&6.91(49)&7.99(56)&6.72(56)&5.98(50)&5.29(45)&4.13(42)&3.36(41)&4.13(40)\\
\hline
\multirow{3}{*}{$D_s^+\to\eta_{\pi^0\pi^+\pi^-} \mu^+\nu_\mu$}&$N_{\mathrm{DT}}^i$&97(13)&91(12)&83(12)&58(11)&70(10)&52(09)&35(08)&43(09)\\
&$N_{\mathrm{prd}}^i$&3114(431)&2933(446)&2825(471)&1797(447)&2583(446)&1937(419)&1356(370)&1846(413)\\
&$\Delta\Gamma^i_{\rm msr}$ $(\mathrm{ns^{-1}})$&7.56(105)&7.12(108)&6.86(114)&4.36(109)&6.27(108)&4.70(102)&3.29(090)&4.48(100)\\
\hline\hline
\end{tabular}
}
\end{table}

\begin{table}[htp]\centering
\caption{
The partial decay rates of $D_s^+\to\eta^\prime \mu^+\nu_\mu$ in various $q^{2}$ intervals. Numbers in the parentheses are the statistical uncertainties. }
\label{tab:decayrateb}

\scalebox{0.95}{
\begin{tabular}{ccccc}\hline\hline
\multicolumn{2}{c}{$i$}&1&2&3\\
\multicolumn{2}{c}{$q^2$ $(\mathrm{GeV}^{2}/c^{4})$}&($m_\mu^2,\,0.3$)&($0.3,\,0.6$)&($0.6,\,q_{\rm max}^2$)\\
\hline
\multirow{3}{*}{$D_s^+\to\eta^\prime_{\eta\pi^+\pi^-} \mu^+\nu_\mu$}&$N_{\mathrm{DT}}^i$&64(11)&52(09)&29(07)\\
&$N_{\mathrm{prd}}^i$&2956(504)&2287(424)&1155(318)\\
&$\Delta\Gamma^i_{\rm msr}$ $(\mathrm{ns^{-1}})$&7.18(123)&5.55(103)&2.80(077)\\
\hline
\multirow{3}{*}{$D_s^+\to\eta^\prime_{\gamma\rho^0} \mu^+\nu_\mu$}&$N_{\mathrm{DT}}^i$&98(13)&106(13)&40(10)\\
&$N_{\mathrm{prd}}^i$&2873(403)&2891(363)&914(264)\\
&$\Delta\Gamma^i_{\rm msr}$ $(\mathrm{ns^{-1}})$&6.98(98)&7.02(88)&2.22(64)\\
\hline\hline
\end{tabular}
}

\end{table}



\begin{table*}[htp]\centering
\caption{Statistical and systematic density matrices for the measured partial decay rates of $D_s^+\to\eta \mu^+\nu_\mu$ in different $q^2$ intervals.}
\label{tab:cova}
\scalebox{0.9}{
\begin{tabular}{ccccccccccccccccc}\hline\hline
\multicolumn{17}{c}{Statistical correlation matrix}\\
\multirow{2}{*}{$\rho_{ij}^{\rm stat}$}&\multicolumn{8}{c}{$D_s^+\to\eta_{\gamma\gamma}\mu^+\nu_\mu$}&\multicolumn{8}{c}{$D_s^+\to\eta_{\pi^0\pi^+\pi^-}\mu^+\nu_\mu$}\\
&1&2&3&4&5&6&7&8&1&2&3&4&5&6&7&8\\
\hline
1	&1.000	&-0.144	&0.015	&-0.002	&-0.000	&-0.000	&-0.000	&-0.000	&0.000	&0.000	&0.000	&0.000	&0.000	&0.000&0.000	&0.000	\\
2	&&1.000	&-0.177	&0.022	&-0.003	&0.000	&-0.000	&-0.000	&0.000	&0.000	&0.000	&0.000	&0.000	&0.000	&0.000&0.000	\\
3	&&&1.000	&-0.202	&0.023	&-0.003	&-0.000	&-0.000	&0.000	&0.000	&0.000	&0.000	&0.000	&0.000	&0.000&0.000	\\
4	&&&&1.000	&-0.203	&0.023	&-0.003	&-0.000	&0.000	&0.000	&0.000	&0.000	&0.000	&0.000	&0.000	&0.000\\
5	&&&&&1.000	&-0.198	&0.021	&-0.003	&0.000	&0.000	&0.000	&0.000	&0.000	&0.000	&0.000	&0.000	\\
6	&&&&&&1.000	&-0.187	&0.015	&0.000	&0.000	&0.000	&0.000	&0.000	&0.000	&0.000	&0.000	\\
7	&&&&&&&1.000	&-0.129	&0.000	&0.000	&0.000	&0.000	&0.000	&0.000	&0.000	&0.000	\\
8	&&&&&&&&1.000	&0.000	&0.000	&0.000	&0.000	&0.000	&0.000	&0.000	&0.000	\\
1	&&&&&&&&&1.000	&-0.120	&0.010	&-0.001	&-0.000	&0.000	&0.000	&-0.000	\\
2	&&&&&&&&&&1.000	&-0.153	&0.014	&-0.001	&-0.000	&0.000	&-0.002	\\
3	&&&&&&&&&&&1.000	&-0.172	&0.013	&-0.002	&-0.001	&-0.001	\\
4	&&&&&&&&&&&&1.000	&-0.179	&0.013	&-0.003	&-0.001	\\
5	&&&&&&&&&&&&&1.000	&-0.187	&0.014	&-0.002	\\
6	&&&&&&&&&&&&&&1.000	&-0.185	&0.008	\\
7	&&&&&&&&&&&&&&&1.000	&-0.124	\\
8	&&&&&&&&&&&&&&&&1.000	\\



\hline
\multicolumn{17}{c}{Systematic correlation matrix}\\
\multirow{2}{*}{$\rho_{ij}^{\rm syst}$}&\multicolumn{8}{c}{$D_s^+\to\eta_{\gamma\gamma}\mu^+\nu_\mu$}&\multicolumn{8}{c}{$D_s^+\to\eta_{\pi^0\pi^+\pi^-}e^+\nu_e$}\\
&1&2&3&4&5&6&7&8&1&2&3&4&5&6&7&8\\
\hline
1	&1.000	&0.836	&0.728	&0.798	&0.279	&0.480	&0.538	&0.471	&0.512	&0.632	&0.474	&0.495	&0.303	&0.204&0.005	&0.480	\\
2	&&1.000	&0.751	&0.799	&0.237	&0.466	&0.473	&0.477	&0.523	&0.660	&0.541	&0.571	&0.274	&0.179	&-0.03&0.446	\\
3	&&&1.000	&0.731	&0.483	&0.578	&0.575	&0.629	&0.338	&0.626	&0.691	&0.697	&0.426	&0.391	&0.239&0.439	\\
4	&&&&1.000	&0.355	&0.562	&0.580	&0.558	&0.428	&0.623	&0.546	&0.554	&0.363	&0.288	&0.113	&0.451\\
5	&&&&&1.000	&0.619	&0.582	&0.688	&-0.134	&0.252	&0.419	&0.352	&0.579	&0.622	&0.652	&0.342	\\
6	&&&&&&1.000	&0.508	&0.667	&0.090	&0.399	&0.446	&0.403	&0.510	&0.500	&0.450	&0.398	\\
7	&&&&&&&1.000	&0.541	&0.148	&0.415	&0.431	&0.399	&0.450	&0.427	&0.357	&0.382	\\
8	&&&&&&&&1.000	&0.073	&0.415	&0.521	&0.482	&0.527	&0.524	&0.476	&0.384	\\
1	&&&&&&&&&1.000	&0.583	&0.402	&0.327	&0.046	&0.018	&-0.052	&0.523	\\
2	&&&&&&&&&&1.000	&0.717	&0.724	&0.543	&0.450	&0.263	&0.670	\\
3	&&&&&&&&&&&1.000	&0.795	&0.617	&0.576	&0.470	&0.616	\\
4	&&&&&&&&&&&&1.000	&0.539	&0.510	&0.360	&0.545	\\
5	&&&&&&&&&&&&&1.000	&0.658	&0.646	&0.590	\\
6	&&&&&&&&&&&&&&1.000	&0.668	&0.525	\\
7	&&&&&&&&&&&&&&&1.000	&0.442	\\
8	&&&&&&&&&&&&&&&&1.000	\\


\hline\hline
\end{tabular}
}
\end{table*}



\begin{table}[htp]\centering
\caption{Statistical and systematic density matrices for the measured partial decay rates of  $D_s^+\to\eta^\prime \mu^+\nu_\mu$ in different
$q^2$ intervals.}
\label{tab:covb}
\begin{tabular}{ccccccc}\hline\hline
\multicolumn{7}{c}{Statistical correlation matrix}\\
\multirow{2}{*}{$\rho_{ij}^{\rm stat}$}&\multicolumn{3}{c}{$D_s^+\to\eta^\prime_{\eta\pi^+\pi^-}\mu^+\nu_\mu$}&\multicolumn{3}{c}{$D_s^+\to\eta^\prime_{\gamma\rho^0}\mu^+\nu_\mu$}\\
&1&2&3&1&2&3\\
\hline
1	&1.000	&-0.088	&0.003	&0.000	&0.000	&0.000	\\
2	&&1.000	&-0.085	&0.000	&0.000	&0.000	\\
3	&&&1.000	&0.000	&0.000	&0.000	\\
1	&&&&1.000	&-0.062	&-0.020	\\
2	&&&&&1.000	&-0.071	\\
3	&&&&&&1.000	\\


\hline
\multicolumn{7}{c}{Systematic correlation matrix}\\
\multirow{2}{*}{$\rho_{ij}^{\rm syst}$}&\multicolumn{3}{c}{$D_s^+\to\eta^\prime_{\eta\pi^+\pi^-}\mu^+\nu_\mu$}&\multicolumn{3}{c}{$D_s^+\to\eta^\prime_{\gamma\rho^0}\mu^+\nu_\mu$}\\
&1&2&3&1&2&3\\
\hline
1	&1.000	&0.812	&0.465	&0.512	&0.536	&0.367	\\
2	&&1.000	&0.593	&0.429	&0.567	&0.401	\\
3	&&&1.000	&0.503	&0.363	&0.515	\\
1	&&&&1.000	&0.553	&0.637	\\
2	&&&&&1.000	&0.620	\\
3	&&&&&&1.000	\\


\hline\hline
\end{tabular}
\end{table}





\end{document}


