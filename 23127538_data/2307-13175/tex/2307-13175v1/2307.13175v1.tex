\documentclass[11pt,a4paper,reqno]{amsart}
\usepackage{amsmath,amssymb,amsfonts,epsfig,mathrsfs,cite}
\usepackage[T1]{fontenc}
%\usepackage[utf8]{inputenc}
\usepackage{color}
\usepackage{array}
\usepackage{amsthm}
\usepackage{amstext}
\usepackage{graphicx}
\usepackage{setspace}



\makeatletter
\@namedef{subjclassname@2020}{%
  \textup{2020} Mathematics Subject Classification}
\makeatother

%\usepackage{mathrsfs}
%\usepackage{hyperref}
%\usepackage{esint}
%
\usepackage[margin=2.5cm]{geometry}
%\usepackage{bbm}
%\usepackage{xparse}
\usepackage{color}
%\usepackage{Mathtools}
%\usepackage{natbib}
\usepackage{enumitem}
%\usepackage{undertilde}
%\usepackage[square]{natbib}
%\setcitestyle{numbers}
\setstretch{1.2}

%new
\usepackage{amscd,psfrag}
\usepackage{yhmath}
\usepackage[mathscr]{eucal}
%\usepackage{xspace}
%%%

	\setcounter{section}{-1}

\usepackage{comment}

\allowdisplaybreaks[4]

\usepackage{slashed}

\makeatletter
%%%%%%%%%%%%%%%%%%%%%%%%%%%%%% LyX specific LaTeX commands.
\pdfpageheight\paperheight
\pdfpagewidth\paperwidth

\setlength{\parindent}{0pt}
\setlength{\parskip}{2.0pt}
%\setlength{\parskip}{6pt}
\usepackage{epstopdf}
%\linespread{1.5}
\usepackage{chngcntr}
\counterwithin{figure}{section}
\usepackage{mathrsfs}

%\usepackage{graphicx,amssymb}
\setlength{\parindent}{28pt}
\usepackage{indentfirst}	

\usepackage[normalem]{ulem}
\theoremstyle{plain}
%\theoremstyle{definition}
\numberwithin{equation}{section}

\newcommand{\im}{{\sqrt{-1}}}

\newtheorem{definition}{Definition}[section]
\newtheorem{theorem}[definition]{Theorem}
\newtheorem*{theorem*}{Theorem}

\newtheorem{assumption}[definition]{Assumption}
\newtheorem{remark}[definition]{Remark}
\newtheorem{fact}[definition]{Fact}
\newtheorem*{remark*}{Remark}
\newtheorem*{sideremark*}{Side Remark}
\newtheorem{preliminaries}{Preliminaries}[section]
\newtheorem*{claim*}{Claim}
\newtheorem*{q*}{Question}
\newtheorem{lemma}[definition]{Lemma}
\newtheorem{corollary}[definition]{Corollary}
\newtheorem*{corollary*}{Corollary}
\newtheorem{example}[definition]{Example}
\newtheorem{proposition}[definition]{Proposition}
\newtheorem{exercise}[definition]{Exercise}
\newtheorem{convention}[definition]{Convention}
\newtheorem{notation}[definition]{Notation}
\newtheorem{statement}[definition]{Statement}
\newtheorem{conjecture}[definition]{Conjecture}


\newcommand{\red}[1]{\textcolor{red}{#1}}

\newcommand{\blue}[1]{\textcolor{blue}{#1}}

\newcommand{\CC}{{\mathscr{C}}}
\newcommand{\R}{\mathbb{R}}
\newcommand{\vf}{\Gamma(TM)}
\newcommand{\na}{\nabla}
\newcommand{\emb}{\hookrightarrow}
\newcommand{\lp}{{L^p}}
\newcommand{\id}{\text{Id}}
\newcommand{\glnk}{\mathfrak{gl}(n+k;\R)}
\newcommand{\p}{\partial}
\newcommand{\loc}{{\rm loc}}
\newcommand{\weak}{\rightharpoonup}
\newcommand{\e}{\varepsilon}
\newcommand{\C}{\mathbb{C}}
\newcommand{\dd}{{\rm d}}
\newcommand{\g}{{\mathfrak{g}}}
\newcommand{\dvg}{{\,\dd {\rm vol}_g^X}}
\newcommand{\G}{\Gamma}
\newcommand{\linf}{{L^\infty}}
\newcommand{\Hom}{{\rm Hom}}
\newcommand{\two}{{\rm II}}
\newcommand{\M}{{\mathcal{M}}}
\newcommand{\bra}{\left\langle}
\newcommand{\ket}{\right\rangle}

\newcommand{\proj}{\mathbf{\Pi}^{\dd}}

\newcommand{\poly}{{\mathscr{P}}}

\newcommand{\mres}{\mathbin{\vrule height 1.6ex depth 0pt width
0.13ex\vrule height 0.13ex depth 0pt width 1.3ex}}

\newcommand{\bara}{{\overline{\alpha}}}

\newcommand{\barb}{{\overline{\beta}}}

\newcommand{\ppl}{{\sigma_{\bf ppl}}}

\newcommand{\bart}{{\overline{\theta}}}

\def\Xint#1{\mathchoice
{\XXint\displaystyle\textstyle{#1}}%
{\XXint\textstyle\scriptstyle{#1}}%
{\XXint\scriptstyle\scriptscriptstyle{#1}}%
{\XXint\scriptscriptstyle\scriptscriptstyle{#1}}%
\!\int}
\def\XXint#1#2#3{{\setbox0=\hbox{$#1{#2#3}{\int}$ }
\vcenter{\hbox{$#2#3$ }}\kern-.6\wd0}}
\def\ddashint{\Xint=}
\def\dashint{\Xint-}
        

\title{Wedge product theorem in  compensated compactness theory with critical exponents on Riemannian manifolds} 

%\author{Gui-Qiang G. Chen}
%\address{G.-Q. Chen: Mathematical Institute, University of Oxford, Oxford, OX2 6GG, UK}
%\email{\texttt{chengq@maths.ox.ac.uk}}

\author{Siran Li}

\address{Siran Li: School of Mathematical Sciences $\&$ CMA-Shanghai, Shanghai Jiao Tong University, No.~6 Science Buildings,
800 Dongchuan Road, Minhang District, Shanghai, China (200240)}

\email{\texttt{siran.li@sjtu.edu.cn}}

%\author{Marshall Slemrod}
%\address{M. Slemrod: Department of Mathematics, University of Wisconsin, Madison, WI 53706, USA}
%\email{\texttt{slemrod@math.wisc.edu}}

\keywords{Compensated compactness; weak continuity; differential form; div-curl lemma; critical exponents; isometric immersion.}

\subjclass[2020]{58C07; 53C42}
\date{\today}


\pagestyle{plain}
\begin{document}


\begin{abstract}
We formulate and prove compensated compactness theorems concerning the limiting behaviour of wedge products of weakly convergent differential forms on closed Riemannian manifolds. The case of critical regularity exponents is considered, which goes beyond the regularity regime entailed by H\"{o}lder's inequality. Implications on the weak continuity of $L^p$-extrinsic geometry of isometric immersions of Riemannian manifolds are discussed.
\end{abstract}
\maketitle


\section{Introduction}\label{sec: intro}



We establish compensated compactness theorems with critical exponents for wedge products of differential forms on closed manifolds. Our work generalises and is, in fact, primarily motivated by the works of Robbin--Rogers--Temple \cite{key1} and Briane--Casado-D\'{i}az--Murat \cite{bcm}. Throughout, by closed manifold we mean a compact Riemannian manifold with no boundary.


Compensated compactness has played an important role in nonlinear analysis, especially in nonlinear PDEs arising from fluid mechanics (\emph{cf}. DiPerna \cite{dip}, C. Dafermos \cite{dafermos}), nonlinear elasticity (\emph{cf}. Re\u{s}etnjak \cite{res}, Ball \cite{ball}, Murat \cite{mur1}, M\"{u}ller \cite{mul}), and geometric analysis (\emph{cf}. H\'{e}lein \cite{h}), as well as calculus of variations (\emph{cf}. Tartar \cite{tar1, tar2}, Ball--Currie--Olver \cite{bco},  Fonseca--Leoni--Mal\'{y} \cite{flm}). The above list of references is by no means exhaustive; we refer to the monograph by Evans \cite{evans} and the recent survey \cite{chen} by Chen for further details.


The \emph{div-curl lemma} introduced by Murat \cite{mur1, mur2, mur3} and Tartar \cite{tar1, tar2} marks a cornerstone of the compensated compactness theory. It maintains that a nonlinear functional of weakly convergent sequences will converge in coarse topologies when specific derivatives of these sequences are compact in natural function spaces. The nonlinear functional and differential constraints in consideration need to satisfy some conditions that are essentially algebraic in nature. In the simplest and original form, the div-curl lemma reads as follows:
\begin{lemma}\label{lem: div-curl}
Let $\{u^n\}$, $\{v^n\}$ be sequences of vectorfields in $L^2_\loc(\R^3,\R^3)$ such that $u^n \weak \bar{u}$ and $v^n \weak \bar{v}$  in $L^2_\loc(\R^3,\R^3)$. Assume  $\left\{{\rm div}(u^n)\right\}$ and $\left\{{\rm curl}(v^n)\right\}$ are precompact in $W^{-1,2}_\loc(\R^3,\R)$ and  $W^{-1,2}_\loc(\R^3,\R^3)$, respectively. Then $u^n \cdot v^n$ converge to $\bar{u} \cdot \bar{v}$ in the sense of distributions. 
\end{lemma}


This result has been proved, reproved, and generalised since its first appearance. Murat and Tartar's  arguments (\cite{mur1, mur2, mur3, tar1, tar2}) utilised Fourier transforms. Later, two harmonic analytic proofs were provided in the seminal paper by Coifman--Lions--Meyer--Semmes \cite{clms}, which furthermore established that $\left\{u^n \cdot v^n\right\}$ in Lemma~\ref{lem: div-curl} is bounded in Hardy space $\mathcal{H}^1_\loc$. From another perspective, Robbin--Rogers--Temple \cite{key1} first observed that, by writing 
\begin{align*}
u^n = \Delta \Delta^{-1}u^n = \left({\rm grad} \circ {\rm div} - {\rm curl} \circ {\rm curl} \right)\Delta^{-1}u^n,
\end{align*}
one may decompose $u^n$, and similarly for $v^n$, into a weakly convergent part and a strongly convergent part. The pairing of the two weakly convergent parts is shown to converge in the sense of distributions by employing the ellipticity of $\Delta$, the commutativity of $\Delta^{-1}$ with divergence and curl, as well as the first-order different constraints in the assumption\footnote{For simplicity of presentations, here we do not state carefully the boundary conditions.}. This idea has been exploited to extend the div-curl lemma to more general domains (\emph{e.g.}, Riemannian manifolds) and differential operators (\emph{e.g.}, exterior differential $\dd$ and codifferential $\dd^*$, or elliptic complexes in greater generality), based on which a functional analytic framework for div-curl lemma has been developed. See Kozono--Yanagisawa \cite{ky}, Chen--Li \cite{cl1, cl2}, Waurick \cite{wau}, and Pauly \cite{pau}. %We also refer to Tartar \cite{tar3} for some historical perspectives on the div-curl lemma.



In fact, Robbin--Rogers--Temple established a more general result \cite[Theorem~1.1]{key1} than Lemma~\ref{lem: div-curl}, formulated in terms of wedge products of weakly convergent differential forms. 

\begin{lemma}[(Multilinear) wedge product theorem; Theorem~1.1 in \cite{key1}]\label{lemma: RRT}
Let $\{\alpha_1^n, \ldots, \alpha_L^n\}_{n \in \mathbb{N}}$ be sequences of differential forms on $\Omega \subset \R^N$ of degrees $s_1, \ldots, s_L$, respectively; $\sum_{i=1}^L s_i \leq N$. Assume for each $i \in \{1,\ldots,L\}$ that
\begin{align*}
\alpha_i^n \weak \overline{\alpha_i} \text{ in } L^{p_i} \text{ with } \sum_{i=1}^L\frac{1}{p_i}=1
\end{align*}
and that
\begin{align*}
\left\{\dd \alpha_i^n\right\} \text{ lies in a compact subset of } W^{-1,p_i}_\loc.
\end{align*}
Then we have
\begin{align*}
\alpha_1^n \wedge \cdots \wedge \alpha_L^n \longrightarrow \overline{\alpha}_1 \wedge \cdots \wedge \overline{\alpha}_L \qquad \text{ in } \mathcal{D}'.
\end{align*}
\end{lemma}

In terms of applications, we find that such formulation is particularly convenient for weak continuity considerations of isometric immersions of (semi)-Riemannian manifolds. See \cite{cl1, cl2}, in which some generalisations of Lemma~\ref{lemma: RRT} are proved and exploited. This observation has been further extended to study weak continuity  of gauge equations by Chen--Giron \cite{cg}.


The theory of compensated compactness as well as the descendants/relatives of the div-curl lemma remains an important topic of research in nonlinear analysis today. Some recent endeavours in this field are devoted to  the development of nonhomogeneous, endpoint, and fractional versions of div-curl lemmas (\cite{daf, ls, bc, bcm, ms, brezis, grs}), detailed characterisations of the structures --- especially, algebraic structures --- of compensated compactness quantities (\cite{mur3, lmzw, zhou, io, mm, rin1}), and connections with broader contexts in geometric measure theory (\cite{sil}). 


\emph{Trente ans après} the introduction of the div-curl lemma by Murat--Tartar \cite{mur1, tar1}, Briane--Casado-D\'{i}az--Murat further proved in \cite{bcm} several endpoint div-curl-type theorems and applied them to study the $G$-convergence for unbounded monotone operators. The setting is as follows: Let $\{u^n\} \subset L^p$ and $\{v^n\}\subset L^q$ be two sequences of weakly convergent vectorfields on $\Omega \subset \R^N$ to limits $\bar{u}$ and $\bar{v}$, respectively, such that $|u^n - \bar{u}|^p$ and $|v^n - \bar{v}|^q$ both converge weakly-$\star$ as Radon measures. In addition, assume the strong convergence of ${\rm div} (u^n)$ and ${\rm curl} (v^n)$ in $W^{-1,q'}$ and $W^{-1,p'}$, respectively. (Notice the ``twist'' of indices $p$, $q$ here!) Then the dot products $u^n \cdot v^n$ converges to $\bar{u} \cdot \bar{v} + \varpi$ in the sense of distributions, where $\varpi$ is a defect Radon measure that can be characterised in detail. A crucial point here is that the range of indices $(p,q)$ is $\frac{1}{p} + \frac{1}{q} \leq 1+\frac{1}{N}$, which goes beyond the ``subcritical'' case of $(p,q)$ entailed by H\"{o}lder's inequality as in Lemma~\ref{lemma: RRT}. This motivates us to generalise the wedge product theorem to ``critical'' indices. 


Our main Theorem~\ref{thm: wedge} below extends Lemma~\ref{lemma: RRT}, the wedge product theorem of compensated compactness \emph{\`{a} la} Robbin--Rogers--Temple \cite{key1}, in the spirit of Briane--Casado-D\'{i}az--Murat \cite{bcm}. Here we focus on the bilinear case $L=2$. See Theorem~\ref{thm: multiple wedge prod} below for an extension for general $L$, namely the multilinear case. Meanwhile, Theorem~\ref{thm: wedge} also generalises \cite{bcm}, by way of extending the results therein from vectorfields on Euclidean domains to differential forms of arbitrary degree on closed manifolds.

Before stating the theorem, we first introduce some notations used throughout this work.
 
 \begin{notation}
For a closed (\emph{i.e.}, compact and boundaryless) smooth Riemannian manifold $X$, denote by $\dvg$ the Riemannian volume measure on $X$, and by $\M(X)$ the space of Radon measures on $X$. In view of the Riesz representation theorem, $\M(X)$ equipped with the total variation norm is isometrically isomorphic to $\left[\CC^0(X)\right]^*$ as a Banach space, topologised by the weak-$\star$ topology on $\left[\CC^0(X)\right]^*$. Write the limit of a sequence of Radon measures as $\M-\lim_{n \to \infty}$. 


We write $\dd$ ($\wedge$, resp.) for the exterior differential (wedge product, resp.) on the space of differential forms, and $\dd^*$ for the codifferential, namely that the formal $L^2$-adjoint of $\dd$ taken with respect to the Riemannian metric. We reserve the symbol $\delta$ for Dirac delta measures. Also denote $\Delta:=\dd\dd^*+\dd^*\dd$, which is the Laplace--Beltrami operator. It differs from the Hodge Laplacian by a sign; for example, $\Delta = - \sum_{j=1}^N \p_{jj}$ on $\R^N$.

  
For a regularity scale $\mathcal{R} = W^{k,p}, \CC^{0,\alpha}, \CC^\infty, \M, \mathcal{D}' \cdots$ (as usual, $\mathcal{D}'$ denotes distributions), we shall designate $\mathcal{R}\left(X, \bigwedge^\ell T^*X\right)$ for the space of differential $\ell$-forms of regularity $\mathcal{R}$ on $X$. Lengths/moduli and inner products on $\bigwedge^\bullet T^*X$, unless specified otherwise, are always taken with respect to the Riemannian metric on $X$. A differential form $\alpha \in \mathcal{R}\left(X, \bigwedge^\bullet T^*X\right)$ is said to be exact (coexact, resp.) if and only if $\dd\alpha=0$ ($\dd^*\alpha=0$, resp.) in the sense of distributions.

We write $A \lesssim_{c_1, \ldots, c_n} B$ to mean the inequality $A \leq CB$, where the constant $C$ depends only on the parameters $c_1, \ldots, c_n$. Einstein's summation convention is adopted throughout.

\end{notation}

Now we may state our generalised bilinear wedge product theorem theorem:
\begin{theorem}\label{thm: wedge}
Let $\left(X,g\right)$ be an $N$-dimensional closed Riemannian manifold; $N \geq 2$. Let $1<p,q<\infty$ be such that
\begin{equation}\label{p,q condition}
1 \leq \frac{1}{p} + \frac{1}{q} \leq 1+\frac{1}{N}.
\end{equation}
Consider sequences of differential forms $\left\{ \alpha^n \right\}\subset L^p\left(X, \bigwedge^{\ell_1} T^*X \right)$ and $\left\{ \beta^n \right\}\subset L^q\left(X, \bigwedge^{\ell_2} T^*X \right)$ satisfying the following conditions:
\begin{enumerate}
\item
$\alpha^n \weak \bara$ weakly in $L^p$;
\item
$\beta^n \weak \barb$ weakly in $L^q$;
\item
$\left| \alpha^n - \bara \right|^p\dvg \weak \mu$ weakly-$\star$ in $\M$;
\item
$\left| \beta^n - \barb \right|^q\dvg \weak \nu$ weakly-$\star$ in $\M$;
\item
$\dd\alpha^n \to \dd\bara$ strongly in $W^{-1,q'}$;
\item
$\dd \beta^n \to \dd\barb$ strongly in $W^{-1,p'}$. 
\end{enumerate}
Then we have the following convergence (modulo subsequences) in the sense of  distributions:
\begin{equation}\label{conclusion of wedge prod thm}
\alpha^n \wedge \beta^n \longrightarrow  \bara \wedge \barb + \sum_{k=1}^\infty \dd \left(v^k\delta_{x^k}\right) \qquad \text{in } \mathcal{D}',
\end{equation}
for some sequences of points $\left\{x^k\right\} \subset X$ and $(\ell_1+\ell_2-1)$-covectors $\left\{v^k\right\}$. Moreover,
\begin{align*}
\left|v^k\right| \lesssim_{p,N,X} \left[\mu\left(\left\{x^k\right\}\right)\right]^{\frac{1}{p}}\left[\nu\left(\left\{x^k\right\}\right)\right]^{\frac{1}{q}}.
\end{align*} 

If, in addition, $\frac{1}{p}+\frac{1}{q}<1+\frac{1}{N}$ in \eqref{p,q condition}, then all $v^k$ are zero.
 \end{theorem}
 
 
 We shall refer to $\sum_{k=1}^\infty \dd \left(v^k\delta_{x^k}\right)$ in \eqref{conclusion of wedge prod thm} as the \emph{concentration distribution} in the sequel. The indices $(p,q)$ satisfying \eqref{p,q condition} are said to be in the \emph{critical} range.
 
 
\begin{remark}
More precisely, \eqref{conclusion of wedge prod thm} means the following:
\begin{align*}
\bra\alpha^n \wedge \beta^n, \Xi\ket \longrightarrow  \bra \bara \wedge \barb, \Xi\ket + \sum_{k=1}^\infty \bra v^k, \dd^*\Xi \ket \delta_{x^k} \qquad \text{for any } \Xi \in \CC^\infty\left(X;\bigwedge^{\ell_1+\ell_2} T^*X\right).
\end{align*}
The pairings $\bra\bullet,\bullet\ket$, as before, are taken with respect to the metric $g$. In geometric measure theoretic terminologies,  the convergence in \eqref{conclusion of wedge prod thm} should be understood as weak convergence of $(\ell_1+\ell_2)$-dimensional currents on $(X,g)$. The same convention is adopted throughout this paper. 
\end{remark}
 

If $\frac{1}{p}+\frac{1}{q}< 1$ is assumed instead of \eqref{p,q condition}, Theorem~\ref{thm: wedge} is then reduced to the classical wedge product theorem of Robbin--Rogers--Temple
(Lemma~\ref{lemma: RRT} above with $L=2$). Thus, throughout this paper we  focus solely on the range of indices  in \eqref{p,q condition}. In this case, however, notice below:



\begin{remark}\label{rem: low reg wedge}
The definition of the  wedge products $\alpha^n \wedge \beta^n$ and $\bara \wedge \barb$ requires further  clarification: \emph{a priori} it is unclear if they are well defined distributions for $p^{-1}+q^{-1}>1$. We resort to Hodge decomposition for this purpose. See \S\ref{subsec: weak weak pairing} below for details.
\end{remark}


The proof of Theorem~\ref{thm: wedge} occupies \S\ref{sec: proof of wedge}. A crucial assumption in this theorem is that $1<p,q<\infty$, which shall be referred to as the non-endpoint case --- in contrast to the \emph{endpoint} case tackled in \S\ref{sec: endpoint critical case}, in which either $p=1$ or $q=1$. Our strategies of the proof follow  Briane--Casado-D\'{i}az--Murat \cite{bcm}, with the key tool for tackling critical exponents being P.-L. Lions' theory of concentrated compactness  \cite{lions}. 


Next, in \S\ref{sec: consequences of wedge prod thm}, we discuss several corollaries of the wedge product Theorem~\ref{thm: wedge}. It will be explained that  Theorem~\ref{thm: wedge} encompasses the classical div-curl lemma of Murat--Tartar \cite{mur1, mur2, tar1, tar2} (\emph{cf}. Coifman--Lions--Meyer--Semmes \cite{clms} too), as well as the generalisation by Briane--Casado-D\'{i}az--Murat \cite{bcm}. We also present a multilinear version of the wedge product theorem in Theorem~\ref{thm: multiple wedge prod}, which addresses the distributional convergence of $\alpha_1^n \wedge \cdots \wedge \alpha_L^n$ for sequences of differential forms $\{\alpha^n_i\}_{i \in \{1,2,\ldots,L\};\,n \in \mathbb{N}}$ of $L^{p_i}$-regularity, respectively, provided that $1\leq \sum_{i=1}^L \frac{1}{p_i} \leq 1+\frac{1}{\dim X}$. Consequences of the exactness  of the concentration measure will be discussed too. 


The following section \S\ref{sec: endpoint critical case} deals with the ``endpoint critical case'' of Theorem~\ref{thm: wedge}. By ``critical'' we mean $\frac{1}{p} + \frac{1}{q} = 1+\frac{1}{N}$, and by ``endpoint'' we further mean that either $p$ or $q$ equals $1$. The main result in this case, which generalises \cite[\S\S 3$\&$4]{bcm}, is as follows:

\begin{theorem}\label{thm: endpoint}
Let $\left(X^N,g\right)$ be an $N$-dimensional closed Riemannian manifold; $N \geq 2$. 
Consider two sequences of differential forms $\left\{ \alpha^n \right\}\subset \M\left(X, \bigwedge^{\ell_1} T^*X \right)$ and $\left\{ \beta^n \right\}\subset L^N\left(X, \bigwedge^{\ell_2} T^*X \right)$ satisfying the following conditions:
\begin{enumerate}
\item
$\alpha^n \weak \bara$ weakly-$\star$ in $\M$;
\item
$\beta^n \weak \barb$ weakly in $L^N$;
\item
$\left| \alpha^n - \bara \right| \weak \mu$ weakly-$\star$ in $\M$;
\item
$\left| \beta^n - \barb \right|^N  \weak \nu$ weakly-$\star$ in $\M$;
\item
$\dd\alpha^n \to \dd \bara$ strongly in $W^{-1,N'}$;
\item
$\dd\beta^n \to \dd \barb$ strongly in $L^N$.
\end{enumerate}
Then we have the following convergence (modulo subsequences) in the sense of  distributions:
\begin{equation}
\alpha^n \wedge \beta^n \longrightarrow  \bara \wedge \barb + \sum_{k=1}^\infty \dd \left(v^k\delta_{x^k}\right) \qquad \text{in } \mathcal{D}'%\left(X, \bigwedge^{\ell_1+\ell_2-1}T^*X\right),
\end{equation}
for some sequences of points $\left\{x^k\right\} \subset X$ and $(\ell_1+\ell_2-1)$-covectors $\left\{v^k\right\}$. Moreover,
\begin{equation*}
\left|v^k\right| \lesssim_{N,X}
\left[\mu\left(\left\{x^k\right\}\right)\right]\left[\nu\left(\left\{x^k\right\}\right)\right]^{\frac{1}{N}}.
\end{equation*}
\end{theorem}



\begin{remark}
In this work, any theorem concerning $L^1$ will always be formulated for $\M$, with  $L^1$-norm of integrable functions replaced by total variation of Radon measures. 
\end{remark}


\begin{remark}
For the endpoint critical wedge product Theorem~\ref{thm: endpoint}, one may tend to complain that Condition~(6) appears too strong. But even replacing this by ``$\dd\beta^n \weak \dd \barb$ weakly in $L^N$'' will lead to counterexamples. See \cite[Example~3.4]{bcm}.
\end{remark}


Our proof of Theorem~\ref{thm: endpoint}  follows the strategies in \cite[\S 3]{bcm}, for which more delicate analysis for endpoint Sobolev embeddings are needed. Key tools include the endpoint estimates for elliptic  systems \emph{\`{a} la} Bourgain--Brezis \cite{bb04} and Brezis--Van Schaftingen \cite{bv}. 



The two endpoint critical div-curl lemmas  by Briane--Casado-D\'{i}az--Murat --- \cite[Theorems~3.1 and 4.1]{bcm} ---  formulated and proved in rather different ways in \cite{bcm}, are generalised to the wedge product Theorem~\ref{thm: endpoint} in a unified manner. This is because in our formulation, as in Robbin--Rogers--Temple \cite{key1}, the asymmetry between divergence and curl is circumvented by the exploitation of the wedge product, which is instead (super)symmetric. 




In the final section \S\ref{sec: isom imm} of this paper, we apply the theoretical results established in previous sections to study the weak continuity of the compatibility equations for curvatures --- named after Gauss, Codazzi(--Mainardi), and Ricci --- of isometric immersions of Riemannian manifolds with extrinsic geometry of $L^p$-regularity. 


More precisely, consider a family of (approximate, resp.) isometric immersions of a fixed Riemannian surface $(\Sigma,g)$ with $W^{2,p}$-regularity into $\R^3$. Then the associated second fundamental forms $\{\two^\e\}$, which are $2$-tensorfields on $\Sigma$ with $L^p$-regularity, are (approximate, resp.) weak solutions to the Gauss--Codazzi equations. One hopes to find the smallest possible index $p$ which ensures that any $L^p$-weak limit of $\{\two^\e\}$ remains a weak solution to the Gauss--Codazzi equations. The weak continuity property has been used to prove the existence of isometric immersions with weak regularity of certain negatively curved surfaces. See S. Mardare \cite{mar}, Chen--Slemrod--Wang \cite{csw0}, and the subsequent works \cite{csw, chw1, chw2, liarma, cg}. The aforementioned investigations have potential  applications to nonlinear elasticity. See \cite{elas1, elas2, elas3} and the many references therein.

We apply the generalised wedge product Theorem~\ref{thm: wedge} to establish the following:
\begin{theorem}\label{thm: isom imm, 2D}
Let $\left(\Sigma,g\right)$ be an immersed Riemannian surface in $\R^3$. Consider a family $\left\{\two^\e\right\} \subset L^p_\loc\left(X; \left[TX \otimes TX\right]^*\right)$  of weak solutions to the Gauss--Codazzi equations which converges to $\overline{\two}$ in the weak-$L^p_\loc$-topology, where $p \in \left]\frac{4}{3},\infty\right[$. Suppose that the coexact parts (\emph{i.e.}, projection onto the $\dd^*$-image via Hodge decomposition) of $\left\{\two^\e\right\}$ are precompact in the $L^{p'}_\loc$-topology. Then $\overline{\two}$ is still a weak solution to the Gauss--Codazzi equations. 
 \end{theorem}

Here we are able to go beyond the critical exponent $p_{\bf CS}=2$ as in \cite{csw0, csw} (which makes quadratic functions on $\two^\e$ well defined via the Cauchy--Schwarz inequality; see Remark~\ref{rem: low reg wedge}). The weak continuity of Gauss--Codazzi equations is established in the regime $p>4/3$, with compactness in higher regularity classes (\emph{i.e.} $L^{p'}_\loc$) assumed only for some components of $\two^\e$. For $p>2$, Theorem~\ref{thm: isom imm, 2D} agrees with the weak compactness results established in \cite{csw, cl1, cl2}.

 In fact, we shall prove in \S\ref{sec: isom imm} a more general result (namely, Theorem~\ref{thm: isom imm}) than Theorem~\ref{thm: isom imm, 2D}, which applies to isometric immersions of arbitrary dimensions and codimensions. 



Two technical lemmas pertaining to concentration measures and endpoint elliptic regularity, which are  variants of well-known theorems   on Euclidean domains, are collected in Appendices.


To conclude the introduction, we comment  that it is expected that the entire programme of the present paper can be carried out for more general metric measure spaces, especially for RCD spaces. This is left for future investigations.





\begin{comment}

\red{Quadratic functions emcompass constant-rank conditions in Tartar's quadratic theorem.}


Let $\left(E, g^E\right)$ and $\left(F, g^F\right)$ be vector bundles over the same compact Riemannian manifold $\left(X^N, g\right)$. Consider for $m \geq 0$ the space of degree-$m$ differential operators:
\begin{align*}
{\bf Diff}^m(X;E,F) :=\left\{ \poly: \G(E) \to \G(F):\, \text{$\poly$ is a differential operator of order $m$} \right\}.
\end{align*}
The \emph{principal symbol} $\ppl$ of a differential operator $\poly$ can be defined intrinsically:
\begin{align*}
\ppl: {\bf Diff}^m(X;E,F) \longrightarrow \left\{\text{degree-$m$-homogeneous polynomials $T^*X \to {\rm Hom}\left(E; F^\C\right)$}\right\},
\end{align*}
where for each $x \in X$, $\xi \in T^*_xX$, and any $f \in \CC^\infty(X)$ with $\dd_xf=\xi$, 
\begin{align*}
\ppl\{\poly\}(x,\xi) := \lim_{T \to \infty} \frac{e^{-2\pi\im Tf} \circ \poly \circ e^{2\pi\im Tf}}{T^m} \in {\rm Hom}\left(E_x; F_x\right),
\end{align*}
and $F^\C$ is the complexification of $F$. It is easy to check that $\ppl\{\poly\}(x,\xi)$ is indeed degree-$m$-homogeneous, namely that
\begin{align*}
\ppl\{\poly\}(x, c\xi) = \left|c\right|^m\sigma_m(\poly)(x,\xi)\qquad\text{for all } x \in X,\, \xi \in T^*_xX,\,c\in\C.
\end{align*}


Admissible oscillations not precluded by the differential constraint associated with $\poly \in  {\bf Diff}^m(X;E,F)$ are captured by the set
\begin{align*}
\mathcal{V}_\poly := \left\{ \xi \otimes \lambda \in \G\left(T^*X \bigotimes E\right):\, \left[\ppl (\xi)\right](\lambda) = 0 \in F^\C \right\}.
\end{align*}
As is customary, we define
\begin{align}\label{cone, def}
\Lambda_\poly := \left\{ \lambda \in \G(E):\, \exists \xi \in \G(T^*X) \sim \{0\}\text{ such that } \xi \otimes \lambda \in \mathscr{V}_\poly  \right\}
\end{align}
to be the \emph{operator cone} of $\poly$. It should be noted that, unlike the classical case considered in the pioneering works \cite{tar1, tar2}  of Tartar (see also Rogers--Temple \cite{key2}, Kozono--Yanagisawa \cite{ky}, and many others), in which $X$ and $E$ are Euclidean spaces and $\poly$ is a first-order differential constraint, in the general setting the roles played by $\xi \in \G(T^*X)$ and $\lambda \in \G(E)$ are essentially different. The frequency variable $\xi$ signifies the direction of admissible oscillations, and $\lambda$ is a generic section of $E$.


Define the ``\emph{rank-one oscillators}'' with a small parameter $\e>0$:
\begin{align}\label{rank 1 osc}
&\Omega^\e: \G(TX) \longrightarrow \G(E),\nonumber\\
& \Omega^\e (V) := h\left(\frac{\bra V,\xi\ket}{\e}\right) \lambda\qquad \text{for fixed $h \in \CC^\infty\left(\R\slash\mathbb{Z}\right)$, $\xi \in \G(T^*X)$, and $\lambda \in \G(E)$.}
\end{align}
We write $h \in \CC^\infty\left(\R\slash\mathbb{Z}\right)$ to emphasise its periodicity. The bracket $\bra\bullet,\bullet\ket$ here denotes the duality pairing between $TX$ and $T^*X$. Now suppose that 
\begin{equation}\label{assumption on cone + homog}
\xi \otimes \lambda \in \mathscr{V}_\poly\quad \text{and} \quad \poly \text{ is degree-$m$-homogeneous.}\footnote{In the context of weak continuity, there is no loss of generality to assume the homogeneity of $\poly$. See \cite[p.410, Remark~2]{key2}.} 
\end{equation}


One may directly see that 



\medskip
\end{comment}

\section{Proof of the wedge product Theorem~\ref{thm: wedge}}\label{sec: proof of wedge}

This section is devoted to the proof of the generalised wedge product Theorem~\ref{thm: wedge}. 

For this purpose, we combine (with nontrivial modifications) ideas from Robbin--Rogers--Temple \cite{key1}  and Briane--Casado-D\'{i}az--Murat \cite{bcm}. We first exploit a nice ``substructure'' of $\alpha^n$ and $\beta^n$  based on the Hodge decomposition, which  explains why $\alpha^n\wedge\beta^n$ is well defined in the sense of distributions. The non-critical case $\frac{1}{p}+\frac{1}{q}<1+\frac{1}{N}$ follows from a direct  Sobolev embedding argument, while the critical case encompasses additional difficulties arising from  Sobolev embeddings with critical exponents, which will be treated via P.-L. Lions' theory of concentrated compactness \cite{lions}. 



\subsection{Hodge-type decomposition of differential forms}

We say that a pair of sequences $\left(\left\{x^n\right\},\left\{y^n\right\}\right)$ is a \emph{``weak-strong'' duality pairing} if for some $r \in ]1,\infty[$, one of $\{x^n\}$ or $\{y^n\}$ converges weakly in $L^r$ and the other converges strongly in $L^{r'}$. Hence, for a quadratic form $Q$, we have that $\{Q(x^n,y^n)\}$ converges weakly in $L^1$. Similarly we shall speak of ``weak-weak'' or ``strong-strong'' pairings, which are by now self-explanatory.




Proposition~\ref{prop: decomposition of diff forms in wedge prod} below is essentially \cite[ Lemma~4.1]{key1}, generalised from Euclidean domains to compact manifolds and with  $(p,q)$ relaxed from H\"{o}lder conjugate exponents to the critical indices satisfying \eqref{p,q condition}. Our proof is slightly different from the corresponding arguments in \cite{key1, bcm}.

\begin{proposition}\label{prop: decomposition of diff forms in wedge prod}
Let $\left(X^N,g\right)$, $p$, $q$, $\left\{\alpha^n\right\}$, and $\left\{\beta^n\right\}$ be as in Theorem~\ref{thm: wedge}. One can find  differential forms $\gamma_n \in W^{1,q'}\left(X, \bigwedge^{\ell_1 - 1} T^*X \right)$, $\xi^n \in L^p\left(X, \bigwedge^{\ell_1} T^*X \right)$, $\zeta_n \in W^{1,q}\left(X, \bigwedge^{\ell_2-1} T^*X \right)$, and $\eta^n \in L^{p'}\left(X, \bigwedge^{\ell_2} T^*X \right)$ for each $n = 1,2,3,\ldots$ such that 
\begin{enumerate}
\item
the following decompositions hold \emph{a.e.}:
\begin{equation*}
\begin{cases}
\alpha^n = \dd \gamma_n + \xi^n,\\
\beta^n = \dd \zeta_n + \eta^n;
\end{cases}
\end{equation*}
\item
the following co-exactness conditions holds in the sense of distributions: 
\begin{equation*}
\dd^* \xi^n = 0 \quad \text{and} \quad \dd^* \eta^n = 0\qquad \text{ in } \mathcal{D}';
\end{equation*}
\item
the following convergence results hold (modulo subsequences) in  indicated topologies:
\begin{eqnarray*}
&& \gamma_n \weak \gamma \qquad \text{ weakly in } W^{1,p}\left(X, \bigwedge^{\ell_1 - 1} T^*X \right),\\
&& \xi^n \longrightarrow \xi\qquad \text{ strongly in } L^{q'}\left(X, \bigwedge^{\ell_1} T^*X \right),\\
&&\zeta_n \weak \zeta \qquad \text{ weakly  in } W^{1,q}\left(X, \bigwedge^{\ell_2-1} T^*X \right),\\
&& \eta^n \longrightarrow \eta  \qquad \text{ strongly in } L^{p'}\left(X, \bigwedge^{\ell_2} T^*X \right).
\end{eqnarray*}
\end{enumerate}
\end{proposition}

Here and hereafter the following notational convention is adopted in accordance with \cite{bcm}.


\begin{notation}
For the terms obtained from decomposing given weakly convergent sequences $\left\{\alpha^n\right\}$ and $\left\{\beta^n\right\}$, those with superscripts (subscripts resp.) converge strongly (weakly resp.) with respect to natural topologies given in the context.
\end{notation}

\begin{proof}[Proof of Proposition~\ref{prop: decomposition of diff forms in wedge prod}]



\begin{comment}
\noindent
{\bf Step 1.} The decomposition for $\alpha^n$ and the relevant convergence results are easy to establish. 


Indeed, by classical Hodge decomposition we have
\begin{align}\label{hodge for alpha n}
\alpha^n = \dd \gamma^n + \dd^* k^n + h^n
\end{align}
for differential $(\ell_1 + 1)$-forms $\left\{k^n\right\}$ of $W^{1,p}$-regularity, and harmonic $\ell_1$-forms $\left\{h^n\right\}$ of $L^p$-regularity. Thanks to \cite[p.86, Theorem~2.4.7]{sch},  here one may further select $\gamma^n$ and $k^n$ to be minimal, in the sense that $\dd^*\gamma^n = 0$ and $\dd k^n=0$ weakly. Then 
\begin{align*}
\dd \alpha^n = \dd\dd^* k^n = \Delta k^n,
\end{align*}
where $\Delta = \dd^*\dd + \dd\dd^*$ is the Laplace--Beltrami operator on $(X,g)$. 


It is assumed (see Theorem~\ref{thm: wedge} (5)) that $\dd\alpha^n \to \dd\bara$ strongly in $W^{-1,q'}$. Thus 
\begin{align}\label{kn}
k^n \longrightarrow \Delta^{-1}{\dd\bara}\qquad \text{strongly  in } W^{1,q'}\left(X, \bigwedge^{\ell_1+1}T^*X\right). 
\end{align}
Here $\Delta^{-1}$ denotes the Green's operator for $\Delta$ defined on the de Rham cohomology class of $\dd\bara$, and \eqref{kn} holds by standard elliptic estimates. Note that as $1 \leq p \leq q'$ and $X$ is compact, \eqref{kn} holds for $W^{1,p}$ in place of $W^{1,q'}$.


The  projection from $L^p\left(X, \bigwedge^{\ell_1} T^*X \right)$ onto $L^p$-harmonic $\ell_1$-forms is ($L^p\to L^p$) bounded, so there exists an $L^p$-harmonic $\ell_1$-form $h$ such that
\begin{equation}\label{hn}
h^n \weak h \qquad \text{ weakly in } L^p\left(X, \bigwedge^{\ell_1} T^*X \right).
\end{equation}
Throughout, by \emph{harmonicity} of an $L^p$-form $\omega$ we mean that $\Delta \omega=0$ in the sense of distributions.



To conclude for $\alpha^n$, we set 
\begin{align*}
\xi^n:= \dd^*k^n+h^n\quad\text{and}\quad \xi := \dd^*\Delta^{-1}\dd\bara + h.
\end{align*}
Clearly $\dd^*\xi^n = 0$ in the distributional sense, for $\dd^*\circ\dd^*=0$ and $h^n$ is harmonic. The weak $L^p$-convergence $\xi^n \weak \xi$ follows from \eqref{kn} and \eqref{hn}. Finally, by the Hodge decomposition \eqref{hodge for alpha n} and the ensuing remarks for $\gamma^n$, we apply codifferential to \eqref{hodge for alpha n} and use $\dd^*\xi^n = 0$ to deduce that 
\begin{align*}
\dd^*\alpha^n = \dd^*\dd\gamma^n = \Delta\gamma^n.
\end{align*}

\end{comment}




 We first observe that condition~\eqref{p,q condition} on the range of $(p,q)$ implies
 \begin{align*}
 1 < p \leq q'\qquad \text{ and } \qquad 1 < q \leq p'.
\end{align*}
The results for $\alpha^n$ and $\beta^n$ are completely parallel, so we argue for $\alpha^n$ only. 


Indeed, by classical Hodge decomposition we have
\begin{align}\label{hodge for alpha n}
\alpha^n = \dd \gamma_n + \dd^* k^n + h^n
\end{align}
with differential $(\ell_1 + 1)$-forms $\left\{k^n\right\}$ of $W^{1,p}$-regularity, and harmonic $\ell_1$-forms $\left\{h^n\right\}$ of $L^p$-regularity. Thanks to \cite[p.86, Theorem~2.4.7]{sch},  here one may further select $\gamma_n$ and $k^n$ to be ``minimal'' in the sense that $\dd^*\gamma^n = 0$ and $\dd k^n=0$ weakly. Then 
\begin{align*}
\dd \alpha^n = \dd\dd^* k^n = \Delta k^n,
\end{align*}
where $\Delta = \dd^*\dd + \dd\dd^*$ is the Laplace--Beltrami operator on $(X,g)$. By the ellipticity of $\Delta$ and the assumption that $\left\{\dd\alpha^n\right\}$ converges weakly in $W^{-1,q'}$, we deduce that 
\begin{equation}\label{kn conv}
\text{$\{k^n\}$ converges strongly in $W^{1,q'}$.}
\end{equation}
 Notice that although $k^n$ may fail to be uniquely soluble from $\dd \alpha^n = \Delta k^n$, it is determined up to addition by harmonic forms, which is immaterial to \eqref{hodge for alpha n} as $\dd^*k^n$ is uniquely determined. 

On the other hand, the projection $\pi$ from $L^p\left(X, \bigwedge^{\ell_1} T^*X \right)$ onto the subspace $\mathcal{H}$ of harmonic $L^p$-forms [$\pi(\alpha^n)=h^n$] is compact, for the range of $\pi$ is finite-dimensional. This follows from the $\CC^\infty$-regularity of $L^p$-harmonic forms and the fact that  $\mathcal{H}$ here coincides with the $\ell_1^{\text{th}}$-cohomology group. In this regard, one may equip $\mathcal{H}$ with the $\CC^\infty$-(Fr\'{e}chet) topology to maintain the compactness of $\pi$. See \cite[\S 2.6]{sch} for rudiments of de Rham theory. 
 In particular, 
\begin{equation}\label{hn conv}
\text{$\left\{h^n\right\}$ converges strongly in $L^{q'}$.}
\end{equation}

Therefore, setting
\begin{align*}
\xi^n := \dd^*k^n+h^n,
\end{align*}
we have its strong convergence in $L^{q'}$ from \eqref{kn conv} and \eqref{hn conv}. Also, since $\dd^* \circ \dd^*=0$ and $h^n$ is harmonic, we find that $\dd^*\xi^n=0$ in the sense of distributions.


Finally, taking $\dd^*$ to \eqref{hodge for alpha n} and recalling $\dd^*\xi^n=0$ as well as the choice of $\gamma_n$, we have
\begin{align*}
\dd^*\alpha^n = \dd^*\dd\gamma_n=\Delta \gamma_n.
\end{align*}
As $\left\{\alpha^n\right\}$ is weakly convergent in $L^p$ by assumption, using the ellipticity (hence $L^p_w \to L^p_w$ boundedness) of $\Delta$, we deduce that $\left\{\gamma_n\right\}$ is weakly convergent in $W^{1,p}$. Again, $\gamma_n$ is only determined up to harmonic forms, but then $\dd \gamma_n$ is uniquely determined; so, in view of \eqref{hodge for alpha n}, without loss of generality we may fix the cohomology class of $\gamma_n$ once and for all.  \end{proof}



\subsection{Definition of the weak-weak pairing}\label{subsec: weak weak pairing}

Proposition~\ref{prop: decomposition of diff forms in wedge prod} allows us to define $\alpha^n \wedge \beta^n$ in the sense of distributions, utilising the algebraic structure of wedge product (bilinearity and superdistributivity under exterior differential). As commented in Remark~\ref{rem: low reg wedge}, without probing into the ``substructures'' of $\alpha^n$ and $\beta^n$, one has apparently no sufficient regularity  to define their wedge products as   distributions.


\subsubsection{Weak-weak paring is well defined in the sense of distributions} 
First, due to the  bilinearity of wedge product we (formally should) have
\begin{align}\label{formal}
\alpha^n \wedge \beta^n = \red{\dd \gamma_n \wedge \dd\zeta_n} + \xi^n \wedge \dd\zeta_n + \dd\gamma_n \wedge \eta^n + \xi^n \wedge \eta^n.
\end{align}
The last three terms on the right-hand side of \eqref{formal}, being weak-strong or strong-strong duality pairings, all converge weakly in $L^1\left(X,\bigwedge^{\ell_1 + \ell_2}T^*X\right)$. 


The most singular term is the weak-weak pairing $\dd \gamma_n \wedge \dd\zeta_n$. We ``differentiate by parts'' to find that it is well defined in the sense of distributions. Indeed, by $\dd\circ\dd = 0$ in $\mathcal{D}'(X)$ and the super-distributivity of $\dd$ over wedge product, we have
\begin{align}\label{bad}
\dd \gamma_n \wedge \dd\zeta_n = \dd\left(\gamma_n \wedge \dd \zeta_n\right),
\end{align}
provided that $\gamma_n$ and $\zeta_n$ are of class  $\CC^1$. But $\{\dd\zeta_n\}$ is contained in $L^q$, while $\{\gamma_n\}$ lies in $L^{p^*}$ by virtue of Sobolev embedding (as usual, $p^*=\frac{Np}{N-p}$ denotes the Sobolev conjugate exponent). By assumption we have that $\left[q^{-1} + (p^*)^{-1}\right]^{-1} = \left[p^{-1} +q^{-1}-N^{-1}\right]^{-1} \geq 1$, so $\gamma_n \wedge \dd \zeta_n$ is a well defined $L^1$-function \emph{for each fixed $n$}. Hence, one may now define $\dd \gamma_n \wedge \dd\zeta_n$ via \eqref{bad} as a differential $(\ell_1+\ell_2)$-form-valued) distribution.


Of course, the argument above does not ensure that $\gamma_n \wedge \dd \zeta_n$ (hence $\dd \gamma_n \wedge \dd\zeta_n$) tend to any limit as $n\to\infty$ in the sense of distributions, as the weak $L^1$-topology is not compactly embedded in $\mathcal{D}'$.


\subsubsection{ Independence of representation}
 We check that the definition of $\alpha^n \wedge \beta^n$ given by \eqref{formal} $\&$ \eqref{bad} is independent of the decomposition of $\alpha^n$ and $\beta^n$ as in  Proposition~\ref{prop: decomposition of diff forms in wedge prod}. 

\begin{lemma}
Let $\alpha \in L^{p}\left(X, \bigwedge^{\ell_1}T^*X\right)$ and $\beta \in L^{q}\left(X, \bigwedge^{\ell_2}T^*X\right)$ be differential forms on a closed Riemannian manifold $\left(X^N,g\right)$, where $p,q$ satisfy \eqref{p,q condition}. Assume the decompositions
\begin{equation*}
\begin{cases}
\alpha = \dd\gamma + \xi = \dd \gamma' + \xi',\\
\beta = \dd\zeta + \eta = \dd\zeta' + \eta',
\end{cases}
\end{equation*}
where $\gamma, \gamma' \in W^{1,q'}\left(X, \bigwedge^{\ell_1 - 1} T^*X \right)$, $\xi,\xi' \in L^p\left(X, \bigwedge^{\ell_1} T^*X \right)$, $\zeta, \zeta' \in W^{1,q}\left(X, \bigwedge^{\ell_2-1} T^*X \right)$, and $\eta, \eta' \in L^{p'}\left(X, \bigwedge^{\ell_2} T^*X \right)$. Also, assume that $\dd^*\xi=\dd^*\xi'=0$ and $\dd^*\eta=\dd^*\eta'=0$ in the sense of distributions. Then it holds, in the sense of distributions too, that
\begin{align*}
\dd\left(\gamma\wedge\dd\zeta\right) + \xi \wedge \dd\zeta + \dd\gamma \wedge \eta + \xi \wedge \eta = \dd\left(\gamma'\wedge\dd\zeta'\right) + \xi' \wedge \dd\zeta' + \dd\gamma' \wedge \eta' + \xi' \wedge \eta'.
\end{align*}
\end{lemma}



\begin{proof}
Our argument is an adaptation and generalisation of the arguments in \cite[p.480]{bcm}.  


First let us check that 
   \begin{align}\label{a'}
   \dd\left(\gamma\wedge\dd\zeta\right) + \xi \wedge \dd\zeta + \dd\gamma \wedge \eta + \xi \wedge \eta = \dd\left(\gamma'\wedge\dd\zeta\right) + \xi' \wedge \dd\zeta + \dd\gamma' \wedge \eta + \xi' \wedge \eta.
   \end{align}
Indeed, the difference between the members on the left and right hand sides equals
\begin{align*}
D := \dd\left[\left(\gamma-\gamma'\right)\wedge\dd\zeta\right] + \dd\left(\gamma-\gamma'\right) \wedge \eta + \left(\xi-\xi'\right)\wedge \beta.
\end{align*}
Noting $$\dd\left(\gamma-\gamma'\right) = \xi'-\xi$$ due to the decompositions of $\alpha$, we combine the last two terms to deduce that
\begin{align*}
D = \dd\left[\left(\gamma-\gamma'\right)\wedge\dd\zeta\right] +  \left(\xi-\xi'\right)\wedge\dd\zeta.
\end{align*}
Now, for the first term on the right-hand side, as $\gamma-\gamma' \in W^{1,q'}$ and $\zeta \in W^{1,q}$, we have $\left(\gamma-\gamma'\right)\wedge (\dd\dd\zeta)=0$ in the sense of distributions. So 
\begin{align*}
D = \dd\left(\gamma-\gamma'\right)\wedge\dd\zeta  +  \left(\xi-\xi'\right)\wedge\dd\zeta.
\end{align*}
Finally, using the fact that $\dd\left(\gamma-\gamma'\right)=\xi'-\xi$, we obtain
\begin{align*}
D=0.
\end{align*}

In the above, one may be skeptical about the pairing $\left(\xi-\xi'\right)\wedge\dd\zeta$, since $\xi-\xi' \in L^p$ and $\dd\zeta \in L^q$, while $\frac{1}{p}+\frac{1}{q}$ is only known to be greater than or equal to $1$. However, since $\dd\left(\gamma-\gamma'\right)=\xi'-\xi$, one has $\dd\left(\xi'-\xi\right)=0$. On the other hand, by assumption we have $\dd^*\left(\xi'-\xi\right)=0$. So $\xi'-\xi$ is harmonic on $X$, and hence is $\CC^\infty$ by elliptic regularity.



Next we check that 
\begin{align}\label{a''}
\dd\left(\gamma'\wedge\dd\zeta\right) + \xi' \wedge \dd\zeta + \dd\gamma' \wedge \eta + \xi' \wedge \eta = \dd\left(\gamma'\wedge\dd\zeta'\right) + \xi' \wedge \dd\zeta' + \dd\gamma' \wedge \eta' + \xi' \wedge \eta'.
\end{align}
Indeed, the difference between left- and right-hand sides equals
\begin{align*}
&\dd\left(\gamma'\wedge\dd\zeta\right) - \dd \left(\gamma'\wedge\dd\zeta'\right) + \dd\gamma' \wedge\left(\eta-\eta'\right)\\
&= \dd\left[\gamma'\wedge\dd\left(\zeta-\zeta'\right)\right] + \dd\gamma' \wedge\left(\eta-\eta'\right)\\
&= \dd\gamma' \wedge\left[\dd\left(\zeta-\zeta'\right) - \left(\eta-\eta'\right)\right]\\
&=0.
\end{align*}


The lemma now follows from \eqref{a'} $\&$ \eqref{a''}, both understood in the distributional sense.  
\end{proof}



\subsection{Proof for the subcritical case: $1<p,q<\infty$ and $1 \leq \frac{1}{p}+\frac{1}{q} < 1+\frac{1}{N}$}\label{subsec: subcrit}

In this case we have $p'<q^*$, the Sobolev conjugate of $q$. Thus the embedding $W^{1,q}\left(X, \bigwedge^{\ell_2-1} T^*X \right)\emb\emb L^{p'}\left(X, \bigwedge^{\ell_2-1} T^*X \right)$ is compact. So, $\gamma_n \wedge \dd\zeta_n$ in \eqref{bad} becomes a \emph{strong-strong} pairing of $L^p$- and $L^{p'}$-differential forms. The wedge product converges strongly in $L^1$ by H\"{o}lder's inequality, hence $\dd \gamma_n \wedge \dd\zeta_n$ (appropriately interpreted as in \S\ref{subsec: weak weak pairing}) converges strongly in $W^{-1,1}$. This together with the weak-$L^1$-convergence of the other three terms on the right-hand side of \eqref{formal} implies that $\alpha^n \wedge \beta^n$ converges in the sense of distributions. This proves \eqref{thm: wedge} in the ``sub-critical'' case. 




\subsection{Proof for the (non-endpoint) critical case: $1<p,q<\infty$ and $\frac{1}{p}+\frac{1}{q} = 1+\frac{1}{N}$}\label{subsec, where conc comp is used}

As in the above proof for the subcritical case, the only term needs to be treated is $\alpha^n \wedge \beta^n$ in \eqref{formal}. In the ``non-endpoint'' critical case (\emph{i.e.}, $p \neq 1 \neq q$), we have 
\begin{equation}\label{indices}
1 < p \leq q',\quad 1 < q \leq p',\quad 1<p<N,\quad 1<q<N,\quad p'=q^*,\quad \text{and}\quad q'=p^*. 
\end{equation}

Due to the assumptions (see Theorem~\ref{thm: wedge}, (3)) and Proposition~\ref{prop: decomposition of diff forms in wedge prod}, it holds that
\begin{equation}\label{d gamma convergence}
\left|\dd\left(\gamma_n-\gamma\right)\right|^p\dvg \weak \mu \qquad \text{weakly-$\star$ in $\M(X)$}. 
\end{equation}
Here we used that $\xi^n\to\xi$ strongly in $L^{q'}$, hence strongly in $L^p$, as $X$ is compact and $1<p \leq q'$ by \eqref{indices}. Meanwhile, we may choose $\gamma_n$ and $\gamma$ to be coexact; see \cite[p.86, Theorem~2.4.7]{sch} and the arguments in the proof of Proposition~\ref{prop: decomposition of diff forms in wedge prod}.   This observation, together with \eqref{d gamma convergence} and the classical Gaffney's inequality --- \emph{i.e.}, the $W^{1,q}$-norm of a differential form $\omega$ can be bounded by the $L^q$-norms of $\dd\omega$, $\dd^*\omega$, and $\omega$; see \cite{gaff1, gaff2, gaff3, gaff4} ---  shows that $\left\{\gamma_n\right\}$ is bounded in $W^{1,p}\left(X, \bigwedge^{\ell_2-1} T^*X \right)$. 

We are in the situation of the critical, noncompact Sobolev embedding $W^{1,p} \emb L^{p^*}=L^{q'}$; see \eqref{indices}. By the second concentration compactness lemma \emph{\`{a} la} P.-L. Lions \cite[p.158, Lemma~I.1]{lions}, we have countably many points $\{x^j\} \subset X$ and nonnegative constants $\{c_j\}$ such that
\begin{align}\label{conc comp}
&\left|\gamma
_n - \gamma\right|^{q'}\dvg \weak \lambda' := \sum_{j=1}^\infty c_j \delta_{x^j} \qquad \text{weakly-$\star$ in $\M(X)$},\nonumber\\
&\qquad\qquad \text{ with } \left(c_j\right)^{{p}\slash{q'}} \lesssim_{p,N,X} \mu\left(\left\{x^j\right\}\right) \quad\text{ for each $j$}. 
\end{align}


On the other hand, by assumption (Theorem~\ref{thm: wedge}, (4)) and Proposition~\ref{prop: decomposition of diff forms in wedge prod}, we have 
\begin{equation*}
\left|\dd\left(\zeta_n - \zeta\right)\right|^{q}\dvg\weak \nu \qquad \text{weakly-$\star$ in $\M(X)$}. 
\end{equation*}

An application of Lemma~\ref{lem: appendix} in the appendix yields a measure $$\eth \in \M\left(X; \bigwedge^{\ell_1+\ell_2 - 1} T^*X  \right)$$
such that
\begin{equation}\label{defect meas}
\gamma_n\wedge\dd\zeta_n \weak \gamma \wedge \dd\zeta + \eth \qquad \text{weakly-$\star$ in $\M(X)$},
\end{equation}
and that, for each Borel set $E \subset X$, the total variation of $\eth$ is estimated by 
\begin{equation*}
\left|\eth\right|(E) \leq \left[\lambda'(E)\right]^{\frac{1}{q'}} \left[\nu(E)\right]^{\frac{1}{q}}. 
\end{equation*}
By the structure of $\lambda'$ characterised in \eqref{conc comp} via concentration compactness, we have
\begin{align*}
\left|\eth\right|(E) &\leq  \left[\nu(E)\right]^{\frac{1}{q}} \left\{ \sum_{\{j:\,x^j \in E\}} c_j \right\}^{\frac{1}{q'}}\nonumber\\
&\lesssim_{p,N,X}\,\sum_{\{j:\,x_j \in E\}} \left[\mu\left(\left\{x^j\right\}\right)\right]^{\frac{1}{p}}\left[\nu\left(\left\{x^j\right\}\right)\right]^{\frac{1}{q}}. 
\end{align*}
The defect measure $\eth$ is thus atomic:
\begin{align}\label{defect meas char}
\eth = \sum_{j=1}^\infty v^j \delta_{x^j} \quad \text{with}\quad
\left|v^j\right| \lesssim_{p,N,X} \left[\mu\left(\left\{x^j\right\}\right)\right]^{\frac{1}{p}}\left[\nu\left(\left\{x^j\right\}\right)\right]^{\frac{1}{q}}\quad \text{ for each $j$}.
\end{align}


Note also that \eqref{defect meas char} together with the assumptions in Theorem~\ref{thm: wedge}, (1)--(4) (as well as the lower semicontinuity of Radon measures) implies that $$\sum_{j=1}^\infty \left|v^j\right|<\infty,$$ so $\eth$ is a well defined distribution.


We may now conclude \eqref{conclusion of wedge prod thm} and the ensuing estimates from \eqref{defect meas char} and \eqref{defect meas}. The proof of Theorem~\ref{thm: wedge} is now complete.  



\begin{remark}
In the above proof, all the lengths of vectorfields and differential forms on $X$ are taken with respect to the Riemannian metric. This is in agreement with \cite[Remark I.5]{lions}, for we are applying concentration compactness to $(X,g)$ as a metric measure space. This observation motivates the concluding paragraph in the Introduction \S\ref{sec: intro}. 
\end{remark}


\section{Consequences of the Wedge Product Theorem~\ref{thm: wedge}}\label{sec: consequences of wedge prod thm}

\subsection{A multilinear wedge product theorem} By an inductive argument, we obtain the following generalisation of the multiple wedge product theorem in Robbin--Rogers--Temple \cite[Theorem~1.1]{key1} (see Lemma~\ref{lemma: RRT} above). 


For its formulation we need one more notation: for an $L^p$-differential form $\omega \in L^p\left(X,\bigwedge^\bullet T^*X\right)$ with $1<p<\infty$, Hodge decomposition (see, \emph{e.g.}, \cite{sch}) yields that
\begin{align*}
\omega = \dd \omega_1 + \dd^*\omega_2 + \omega_h,
\end{align*}
where $\omega_h$ is harmonic and $\dd\omega_1$, $\dd^*\omega_2$, and $\omega_h$ are $L^p$-forms. Moreover, one has the uniqueness result: if it also holds that $\omega = \dd\omega_1' + \dd^*\omega_2 + \omega_h'$ with $\dd\omega_1'$, $\dd^*\omega_2'$, $\omega_h' \in L^p$ and $\omega_h'$ being harmonic, then $\dd\omega_1=\dd\omega_1'$, $\dd^*\omega_2 = \dd^*\omega_2'$, and $\omega_h=\omega_h'$. So there is a well-defined norm-1 projection 
\begin{align}\label{proj}
\proj:  L^p\left(X,\bigwedge^\bullet T^*X\right) \longrightarrow  L^p\left(X,\bigwedge^\bullet T^*X\right),\qquad \proj(\omega) := \dd\omega_1.
\end{align}
That is, $\proj$ is the projection of $L^p$-differential forms onto the exact part in Hodge decomposition.


%In other words, by Hodge decomposition  each $L^p$-form can be uniquely decomposed into the sum of $\dd$-, $\dd^*$-, and harmonic parts. Then $\proj$ is the projection onto the $\dd$-part. 
 
Our multiple wedge product theorem is stated below. We have an additional ``no-loss in norm'' condition, which we currently do not know whether it can be dropped.

\begin{theorem}\label{thm: multiple wedge prod}
Let $\left(X^N,g\right)$ be an $N$-dimensional closed Riemannian manifold; $N \geq 2$. Let $1<p_1, \ldots, p_L<\infty$ be such that $$1\leq \sum_{i=1}^L \frac{1}{p_i} \leq 1+\frac{1}{N}$$ where $L$ is a natural number, and assume that for each $i \in \{1,\ldots,L\}$, one has $$q_i := \frac{1}{\sum_{j \in \{1,\ldots,L\}\sim\{i\}} \frac{1}{p_j}}>1.$$ 


Consider $L$ sequences of differential forms $\left\{ \alpha_{1}^n \right\}, \ldots, \left\{ \alpha_{L}^n \right\}$, where $ \alpha_{i}^n \in L^{p_i}\left( X,\bigwedge^{\ell_i} T^*X \right)$ for each $n$ and $0 \leq \ell_i \leq n$, that satisfy the following conditions:
\begin{enumerate}
\item
$\alpha_i^n \weak \bara_i$ weakly in $L^{p_i}$ for each $i \in \{1,2,\ldots,L\}$;
\item
$\left|\alpha^n_i - \bara_i\right|^{p_i}\dvg \weak \mu_i$ weakly-$\star$ in $\M$ for each $i \in \{1,2,\ldots,L\}$;
\item
$\dd \alpha_i^n \to \dd \bara_i$ strongly in $W^{-1,q_i'}$ for each $i \in \{1,2,\ldots,L\}$; and
\item (``no-loss in norm'')
there exists $k \in \{1,2,\ldots,L\}$ such that 
\begin{align}\label{no loss}
&\left\|\alpha^n_1 \wedge \ldots \wedge \widehat{\alpha^n_k} \wedge \ldots \wedge \alpha^n_L - \proj\alpha^n_1 \wedge \ldots \wedge \widehat{\proj\alpha^n_k} \wedge \ldots \wedge \proj\alpha^n_L \right\|_{L^{p_k'}}\nonumber\\
&\quad \longrightarrow \left\|\bara_1 \wedge \ldots \wedge \widehat{\bara_k} \wedge \ldots \wedge \bara_L - \proj\bara_1 \wedge \ldots \wedge \widehat{\proj\bara_k} \wedge \ldots \wedge \proj\bara_L \right\|_{L^{p_k'}} \quad\text{as } n \to \infty,
\end{align}
wherein the projection operator $\proj$ has been defined in \eqref{proj}.
\end{enumerate}

Then the convergence result below holds after passing to subsequences:
\begin{align*}
\alpha^n_1 \wedge \ldots \wedge \alpha^n_L \longrightarrow \bara_1 \wedge \ldots\wedge \bara_L + \sum_{k=1}^\infty\dd\left(v^k \delta_{x^k}\right) \qquad \text{in } \mathcal{D}'%\left(X,\bigwedge^{\sum_{j=1}^L \ell_j} T^*X\right)
\end{align*}
for some sequences of points $\left\{x^k\right\} \subset X$ and $\left(\sum_{j=1}^L \ell_j - 1\right)$-covectors $\left\{v^k\right\}$. In addition, the weights $\left\{v^k\right\}$ have their moduli bounded as follows:
\begin{align*}
\left|v^k\right| \lesssim_{p_1,\ldots,p_L,N,X} \left\{ \prod_{j=1}^L \left[\mu^j\left(\left\{x^k\right\}\right)\right]^{\frac{1}{p_j}}\right\}.
\end{align*} 
\end{theorem}

\begin{notation}
Here and hereafter, %the moduli $|\bullet|$ of differential forms are taken with respect to the Riemannian metric $g$, and 
 circumflex in expressions of the form $\mathfrak{b_1} \wedge \ldots \wedge \widehat{\mathfrak{b}_k} \wedge \ldots \wedge \mathfrak{b}_L$ (and the like)  designates the omission of the $k^{\text{th}}$ entry $\mathfrak{b}_k$. 
\end{notation}

\begin{proof}[Proof of Theorem~\ref{thm: multiple wedge prod}]
First of all, by supercommutativity of the wedge product, there is no loss of generality to prove for only one $k$ in the ``no-loss in norm'' condition~(4). Throughout the proof we shall fix $k=L$.   The thesis follows immediately from Theorem~\ref{thm: wedge} by identifying $\left(\alpha^n_1 \wedge \ldots \wedge \alpha^n_{L-1}, \alpha^n_L, \bara_1 \wedge \ldots\wedge \bara_{L-1},\bara_L\right)$ with $\left( \beta^n,\alpha^n, \barb,\bara\right)$ and $(q_L, p_L)$ with $(q,p)$, \emph{provided that the hypotheses in Theorem~\ref{thm: wedge} are verified}.  The estimate on $v^k$ are proved by a simple induction argument, of which the case $L=2$ is covered by Theorem~\ref{thm: wedge}. 


 The verification for the hypotheses in Theorem~\ref{thm: wedge} is carried out below in  two steps.

\smallskip
\noindent
{\bf Step 1.} First, observe that $\left\{\alpha^n_1 \wedge \ldots \wedge \alpha^n_{L-1}\right\}$ is bounded in $L^{q_L}\left(X,\bigwedge^{\sum_{j=1}^{L-1} \ell_j}T^*X\right)$, as each $\left\{\alpha^n_i\right\}$ is bounded in $L^{p_i}$ by assumption. We are in the subcritical case in the sense of \S\ref{subsec: subcrit} (for only $(L-1)$ forms are taken into account here, so that $1\leq \sum_{i=1}^{L-1} \frac{1}{p_i} < 1+\frac{1}{N}$). Thus, a direct induction and Theorem~\ref{thm: wedge} yield that $\alpha^n_1 \wedge \ldots \wedge \alpha^n_{L-1} \to \bara_1 \wedge \ldots \wedge \bara_{L-1}$ in the distributional sense. This together with the $L^{q_L}$-boundedness established above implies that the convergence takes place in the weak $L^{q_L}$-topology. 

As a result, by a standard compactness argument, there exists a positive finite Radon measure $\mu$ on $X$ such that
\begin{align*}
\left|\alpha^n_1 \wedge \ldots \wedge \alpha^n_{L-1} - \bara_1 \wedge \ldots \wedge \bara_{L-1}\right|^{q_L}\dvg \weak \mu \qquad \text{ weakly-$\star$ in $\M(X)$}.
\end{align*}


\smallskip
\noindent
{\bf Step 2.} We also need to justify that $\left\{\dd\left(\alpha^n_1 \wedge \ldots \wedge \alpha^n_{L-1}\right)\right\}$ converges strongly in $W^{-1, p_L'}$. This relies on the cancellation phenomenon exploited in \S\ref{subsec: weak weak pairing}, and will be presented  in four substeps. 


\noindent
{\bf Step~2A.}  By Proposition~\ref{prop: decomposition of diff forms in wedge prod}, one can decompose each $\alpha^n_j$ for $n \in \mathbb{N}$ and $j \in \{1,\ldots, L-1\}$ into
\begin{align*}
\alpha^n_j = \dd\gamma_{n,j} + \xi^n_j,
\end{align*}
where
\begin{equation*}
\begin{cases}
\gamma_{n,j} \weak \gamma_j\qquad \text{ weakly in } W^{1,p_j},\\
\xi^n_j \longrightarrow \xi_i\qquad \text{ strongly in } L^{q_j'}. 
\end{cases} 
\end{equation*}
We need to show that 
\begin{align}\label{strong conv, to prove}
&\dd\left\{\alpha^n_1 \wedge \ldots \wedge \alpha^n_{{L-1}}\right\} = \dd\left\{\left(\dd\gamma_{n,1} + \xi^n_1\right) \wedge \ldots \wedge  \left(\dd\gamma_{n,L-1} + \xi^n_{L-1}\right)\right\} \nonumber\\
&\quad \longrightarrow \dd\left\{\left(\dd\gamma_{1} + \xi_1\right) \wedge \ldots \wedge  \left(\dd\gamma_{L-1} + \xi_{L-1}\right)\right\} = \dd\left\{\bara_1 \wedge \ldots \wedge \bara_{L-1}\right\} \text{ strongly in } W^{-1,p_L'}.
\end{align}

\begin{remark}
Of course, \eqref{strong conv, to prove} would follow immediately if one could prove that
\begin{align}\label{false}
&\alpha^n_1 \wedge \ldots \wedge \alpha^n_{{L-1}} = \left(\dd\gamma_{n,1} + \xi^n_1\right) \wedge \ldots \wedge  \left(\dd\gamma_{n,L-1} + \xi^n_{L-1}\right)\nonumber\\
&\quad \longrightarrow \left(\dd\gamma_{1} + \xi_1\right) \wedge \ldots \wedge  \left(\dd\gamma_{L-1} + \xi_{L-1}\right) = \bara_1 \wedge \ldots \wedge \bara_{L-1} \qquad\text{ strongly in } L^{p_L'}.
\end{align} 
This, unfortunately, is impossible --- the term $\dd\gamma_{n,1} \wedge \ldots \wedge \dd\gamma_{n,L-1}$ is an $(L-1)$-multilinear combination of weakly convergent sequences in $L^{p_j}$ for $j \in \{1,\ldots, L-1\}$, which by H\"{o}lder's inequality is only bounded in $L^{\left[\sum_{j=1}^{L-1} \frac{1}{p_j}\right]^{-1}} = L^{q_L}$. But as for \eqref{indices} we have $q_L \leq p_L'$, so the false claim~\eqref{false} is stronger than what we can actually obtain. 
\end{remark}


\noindent
{\bf Step~2B.}   Nevertheless, $\dd\gamma_{n,1} \wedge \ldots \wedge \dd\gamma_{n,L-1}$ is the \emph{only} troublesome term here --- we assert that
\begin{align}\label{assert}
&\text{all the other $\left(2^{L-1}-1\right)$ terms in $\left(\dd\gamma_{n,1} + \xi^n_1\right) \wedge \ldots \wedge  \left(\dd\gamma_{n,L-1} + \xi^n_{L-1}\right)$}\nonumber\\
&\text{converge strongly in $L^{p_L'}$.}
\end{align}


To verify the assertion, first we note that 
\begin{align*}
&\bigg\{\text{the sum of all such $\left(2^{L-1}-1\right)$ terms in $\left(\dd\gamma_{n,1} + \xi^n_1\right) \wedge \ldots \wedge  \left(\dd\gamma_{n,L-1} + \xi^n_{L-1}\right)$}\bigg\} \\
&\qquad \equiv  \left(\alpha^n_{1} \wedge \ldots \wedge \alpha^n_{L-1}\right) - \left(\proj\alpha^n_{1} \wedge \ldots \wedge \proj\alpha^n_{L-1}\right).
\end{align*}
This follows from the definition of $\proj$, since it holds for each $j \in \{1,\ldots,L-1\}$ that $$\proj \alpha^n_j = \dd\gamma^n_j.$$ 


In addition, if we can prove that $$\left\{\left(\alpha^n_{1} \wedge \ldots \wedge \alpha^n_{L-1}\right) - \left(\proj\alpha^n_{1} \wedge \ldots \wedge \proj\alpha^n_{L-1}\right)\right\}$$ converges \emph{weakly} in $L^{p_L'}$, then the ``no-loss in norm'' condition~\eqref{no loss} will promote it to \emph{strong} convergence in $L^{p_L'}$, by virtue of the Radon--Riesz theorem. See, \emph{e.g.}, \cite[p.17, Theorem~1.37]{afp}.  


For this purpose, notice that any of the $\left(2^{L-1}-1\right)$ terms in consideration takes the form $\mathfrak{a}_{n,1} \wedge \ldots \wedge \mathfrak{a}_{n,L-1}$, where
\begin{itemize}
\item
each $\mathfrak{a}_{n,j}$ is either $\dd\gamma_{n,j}$ or $\xi^n_j$; and
\item
for at least one $j \in \{1,\ldots, L-1\}$, $\mathfrak{a}_{n,j} = \xi^n_j$.
\end{itemize}
It suffices to prove the convergence of those terms with $\mathfrak{a}_{n,j} = \xi^n_j$ for \emph{only one} $j \in \{1,\ldots, L-1\}$, because at each place $\mathfrak{a}_{n,j}$ the choice $\xi_j^n$ has higher regularity than $\dd\gamma_{n,j}$. In fact, $\left\{\xi_j^n\right\}$ is strongly convergent in $L^{q_j'}$ and $\left\{\dd\gamma_{n,j}\right\}$ is weakly convergent in $L^{p_j}$, with the indices $p_j \leq q_j'$ (again argued as for \eqref{indices}). Thus, as the underlying space $\left(X,\dvg\right)$ is compact, the more $\xi_j^n$ we have in $\mathfrak{a}_{n,j}$, the better convergence results one may obtain. 

Summarising the above discussions, it now suffices show  that
\begin{align}\label{conv, true}
\left\{\dd \gamma_{n,1} \wedge \ldots \wedge \dd\gamma_{n, j-1} \wedge \xi^n_j \wedge \dd\gamma_{n,j+1} \wedge \ldots \wedge \dd \gamma_{n, L-1}\right\}\quad\text{converges weakly in } L^{p_L'}
\end{align}
for each $j \in \{1,\ldots, L-1\}$.


In the $(L-1)$-multilinear combination above, we have one strongly convergent term $\xi^n_j$. By H\"{o}lder's inequality, $\left\{\dd \gamma_{n,1} \wedge \ldots \wedge \dd\gamma_{n, j-1} \wedge \xi^n_j \wedge \dd\gamma_{n,j+1} \wedge \ldots \wedge \dd \gamma_{n, L-1}\right\}$ converges weakly in $L^{\mathcal{Q}}$, with the index
\begin{align*}
\mathcal{Q} &= \left[ \left(\frac{1}{p_1} + \ldots + \frac{1}{p_{j-1}} + \frac{1}{p_{j+1}} + \ldots +\frac{1}{p_{L-1}}\right) + \frac{1}{q_j'} \right]^{-1}\\
&= \left[\left(\frac{1}{q_j} - \frac{1}{p_L}  \right)+ \frac{1}{q_j'}\right]^{-1}\\
&= \left[1-\frac{1}{p_L}\right]^{-1} \\
&= p_L'.
\end{align*}


Thus, \eqref{conv, true} is proved, and hence follows the assertion~\eqref{assert}. 



\noindent
{\bf Step~2C.} Let us return to the  term $\dd\gamma_{n,1} \wedge \ldots \wedge \dd\gamma_{n,L-1}$. Recall from the fake claim~\eqref{false} and the ensuing paragraph that it cannot be proved to converge strongly in $L^{p_L'}$. However,  
\begin{equation}\label{dd}
\dd \left[\dd\gamma_{n,1} \wedge \ldots \wedge \dd\gamma_{n,L-1}\right] = 0\qquad \text{in the sense of distributions.}
\end{equation}
Once established, this identity together with \eqref{assert} above implies \eqref{strong conv, to prove}. 



 The key to the justification of  \eqref{dd} lies in its definition: the term $\left[\dd\gamma_{n,1} \wedge \ldots \wedge \dd\gamma_{n,L-1}\right]$ therein is well defined only when understood as follows:
\begin{align}\label{ddd}
\left[\dd\gamma_{n,1} \wedge \ldots \wedge \dd\gamma_{n,L-1}\right] &:= \dd \left\{\gamma_{n,1} \wedge \dd\gamma_{n,2} \wedge \ldots \wedge \dd\gamma_{n,L-1}\right\}\nonumber\\
&\qquad\qquad \text{in the sense of distributions},
\end{align}
Using \eqref{ddd} and that $\dd\circ\dd =0$ in the sense of distributions, we obtain \eqref{dd} immediately.

 

\noindent
{\bf Step~2D.} It remains to justify \eqref{ddd}. The crucial observation is as in Step~1 above: since  only $(L-1)$ forms are taken into account, we are in the subcritical regime; that is, $\sum_{i=1}^{L-1} \frac{1}{p_i} < 1+\frac{1}{N}$. In the sequel, let us show that $\left\{\gamma_{n,1} \wedge \dd\gamma_{n,2} \wedge \ldots \wedge \dd\gamma_{n,L-1}\right\}$ is weakly convergent in $L^1$.

 
For this purpose, note that $\left\{\dd \gamma_{n,j}\right\}_{n \in \mathbb{N}}$ is weakly convergent in $L^{p_j}$ for each $j \in \{2, \ldots, L-1\}$, while $\left\{\gamma_{n,1}\right\}$ is weakly convergent in $W^{1,p_j}$, hence strongly convergent in $L^r$ for any $r \in \left[1, p_j^\star = \frac{Np_j}{N-p_j}\right[$. One may take $r=q_1' - \e$ for arbitrary $\e>0$. In fact, for each $j \in \{1,2,\ldots,L\}$,
\begin{align*}
q_j' = \frac{1}{1-\frac{1}{q_j}}
= \frac{1}{1- \left(\sum_{i=1}^L \frac{1}{p_i}\right) + \frac{1}{p_j}} \leq  \frac{1}{\frac{1}{p_j} - \frac{1}{N}} = p_j^\star.
\end{align*} 

As a consequence, $\left\{\gamma_{n,1} \wedge \dd\gamma_{n,2} \wedge \ldots \wedge \dd\gamma_{n,L-1}\right\}$ converges weakly in the $(L-1)$-fold product of Lebegue spaces $\left[L^{p_2}\cdot \ldots \cdot L^{p_{L-1}} \cdot L^{q_1' - \e}\right]$. The indices thereof satisfy 
\begin{align*}
\frac{1}{\frac{1}{p_2} + \ldots + \frac{1}{p_{L-1}} + \frac{1}{q_1'-\e}} = \frac{1}{\frac{1}{q_1} - \frac{1}{p_L} + \frac{1}{q_1' - \e}}  = \frac{1}{1-\frac{1}{p_L}} - \kappa = p_{L}' -\kappa\quad \text{for } \kappa = \mathfrak{o}(\e),
\end{align*}
where $\kappa = \mathfrak{o}(\e)$ means that $\kappa$ can be made arbitrarily small by taking $\e >0$ sufficiently close to zero. But $p_L<\infty$, so by choosing $\e>0$ sufficiently small and applying the H\"{o}lder's inequality, we can require that $p_L'-\kappa \geq 1$. This choice warrants the  convergence of $\left\{\gamma_{n,1} \wedge \dd\gamma_{n,2} \wedge \ldots \wedge \dd\gamma_{n,L-1}\right\}$ in the weak $L^1$-topology. 

Thus we arrive at \eqref{ddd}. The proof is now complete.  \end{proof}


\begin{comment}
%observe that it is easy to establish the weak convergence of $\dd\left(\alpha^n_1 \wedge \ldots \wedge \alpha^n_{L-1}\right)$ in $W^{-1, p_L'}$ --- by H\"{o}lder's inequality and the boundedness of $\left\{\alpha_i^n\right\} \subset L^{p_i}$, we know that $\left\{\alpha^n_1 \wedge \ldots \wedge \alpha^n_{L-1}\right\}$ is bounded in $L^{q_L}$, where $q_L:=\left[\sum_{j=1}^{L-1} \frac{1}{p_j}\right]^{-1}$. But by assumption we have $q_L$


 For this purpose, note that for any testform $\psi \in W^{1,p_L}\left(X,\bigwedge^{N -1- \ell_1 - \ldots - \ell_{L-1}}T^*X\right)$ it holds that
\begin{align*}
\int_X \dd\left(\alpha^n_1 \wedge \ldots \wedge \alpha^n_{L-1}\right) \wedge \psi = (-1)^\sigma \int_X \alpha^n_1 \wedge \ldots \wedge \alpha^n_{L-1} \wedge \dd\psi,
\end{align*}
where $\sigma = \sum_{j=1}^{L-1} \ell_j$. This makes sense as 




To this end, we first recall that
\begin{align*}
\dd\left(\alpha^n_1 \wedge \ldots \wedge \alpha^n_{L-1}\right) = \sum_{j=1}^{L-1}\left(-1\right)^{j+1} \alpha^n_1 \wedge \ldots \wedge \alpha^n_{j-1} \wedge \dd\alpha^n_j \wedge \alpha^n_{j+1}\wedge\ldots\wedge\alpha^n_{L-1};
\end{align*}
so, establishing the strong $W^{-1, q_L'}$-convergence of $\alpha^n_1 \wedge \ldots \wedge \alpha^n_{j-1} \wedge \dd\alpha^n_j \wedge \alpha^n_{j+1}\wedge\ldots\wedge\alpha^n_{L-1}$ for each $j \in \{1,2,\ldots, L-1\}$ will suffice. Let us denote momentarily
\begin{align*}
\mathcal{Q}_j := {\left[p_1^{-1} +\ldots+ p_{j-1}^{-1} + p_{j+1}^{-1}+\ldots p_{L-1}^{-1}\right]^{-1}}.
\end{align*}
As $\frac{1}{\mathcal{Q}_j} + \frac{1}{p_j} = \frac{1}{q_L}<1$, we know that $\mathcal{Q}_j > p_j'$, the H\"{o}lder conjugate of $p_j$. In addition, the definition of $q_j$ and the condition $1\leq \sum_{i=1}^L \frac{1}{p_i} \leq 1+\frac{1}{N}$ yields that $p_j' \geq q_j^\star := \frac{N q_j}{N-q_j}$, the Sobolev conjugate of $q_j$. (Compare with \eqref{indices}.) So we have the compact Sobolev embedding $L^{\mathcal{Q}_j} \emb\emb W^{-1, q_j'}$. 

 $\left\{\dd\alpha_j^n\right\}$ converges strongly in $W^{-1, q_j'}$, so it suffices to prove the strong 


By an iterative application of Theorem~\ref{thm: wedge}, we see that
\begin{align*}
\alpha^n_1 \wedge \ldots \wedge \alpha^n_{j-1} \wedge \alpha^n_{j+1} \wedge\ldots\wedge\alpha^n_{L-1} \longrightarrow \bara_1 \wedge \ldots\wedge \bara_{j-1}\wedge \bara_{j+1}\wedge \ldots \wedge \alpha_{L-1}
\end{align*}
in the sense of distributions. Indeed, $1 \leq p_1^{-1} +\ldots+ p_{j-1}^{-1} + p_{j+1}^{-1}+\ldots p_{L-1}^{-1} <1+N^{-1}$, thus we are in the subcritical case so that there are no atoms in the distributional limit.  On the other hand, as each $\left\{\alpha^n_i\right\}$ is bounded (hence weakly convergent modulo subsequences) in $L^{p_i}$, the distributional convergence can be promoted to weak convergence in $L^{\left[p_1^{-1} +\ldots+ p_{j-1}^{-1} + p_{j+1}^{-1}+\ldots p_{L-1}^{-1}\right]^{-1}}$. 

\end{comment}


\subsection{The div-curl formulation} The prototype of the wedge product theorem of Robbin--Rogers--Temple \cite{key1}, hence of Theorem~\ref{thm: wedge} in this paper, is the classical \emph{div-curl lemma} of Murat and Tartar  (\cite{mur1, mur2, tar1, tar2}). An analogous div-curl-type result follows easily from Theorem~\ref{thm: wedge}.


\begin{corollary}\label{cor: div-curl}
Let $\left(X^N,g\right)$ be an $N$-dimensional closed Riemannian manifold; $N \geq 2$. Let $1<p,q<\infty$ be such that
\begin{equation*}
1 \leq \frac{1}{p} + \frac{1}{q} \leq 1+\frac{1}{N}.
\end{equation*}
Consider two sequences of differential $\ell$-forms $\left\{ \alpha^n \right\}\subset L^p\left(X, \bigwedge^{\ell} T^*X \right)$ and $\left\{ \theta^n \right\}\subset L^q\left(X, \bigwedge^{\ell} T^*X \right)$ satisfying the following conditions:
\begin{enumerate}
\item
$\alpha^n \weak \bara$ weakly in $L^p$;
\item
$\theta^n \weak \bart$ weakly in $L^q$;
\item
$\left| \alpha^n - \bara \right|^p\dvg \weak \mu$ weakly-$\star$ in $\M$;
\item
$\left| \theta^n - \bart \right|^q\dvg \weak \nu$ weakly-$\star$ in $\M$;
\item
$\dd\alpha^n \to \dd\bara$ strongly in $W^{-1,q'}$;
\item
$\dd^* \theta^n \to \dd^*\bart$ strongly in $W^{-1,p'}$. 
\end{enumerate}
Then we have the following convergence (modulo subsequences) in the sense of  distributions:
\begin{equation*}
\bra \alpha^n, \theta^n\ket_g \longrightarrow  \bra\bara, \bart\ket_g + \sum_{k=1}^\infty {\rm div}_g\left(r^k\delta_{x^k}\right) \qquad \text{in } \mathcal{D}'
\end{equation*}
for some sequences of points $\left\{x^k\right\} \subset X$ and vectors $\left\{r^k\right\}$ such that
\begin{align*}
\left|r^k\right| \lesssim_{p,N,X} \left[\mu\left(\left\{x^k\right\}\right)\right]^{\frac{1}{p}}\left[\nu\left(\left\{x^k\right\}\right)\right]^{\frac{1}{q}}.
\end{align*} 

Moreover, if $\frac{1}{p}+\frac{1}{q}<1+\frac{1}{N}$ in \eqref{p,q condition}, then all $r^k$ are zero.
 \end{corollary}


Recall that $\bra\alpha, \theta\ket_g := \star\left[\alpha \wedge \star \theta\right]$ for differential $\ell$-forms  $\alpha$ and $\theta$ on $X$.


When $\ell=1$ (with differential one-forms identified with vectorfields via the musical isomorphisms), Corollary~\ref{cor: div-curl} reduces to \cite[Theorem~2.3]{bcm} in Briane--Casado-D\'{i}az--Murat. The Hodge star $\star$,  the codifferential $\dd^* = (-1)^{N(\ell+1)+1}\star \dd\star$, the divergence ${\rm div}_g$, and the inner product $\bra \bullet, \bullet\ket_g$ are all taken with respect to the Riemannian metric $g$. Meanwhile, as commented beneath the statement of Theorem~\ref{thm: wedge}, when $\frac{1}{p} + \frac{1}{q} <1$, classical div-curl lemma or wedge product theorem applies to yield that $\bra\alpha^n,\theta^n\ket_g \to \bra\bara,\bart\ket_g$ in the sense of distributions. 


\begin{proof}[Proof of Corollary~\ref{cor: div-curl}]
Applying Theorem~\ref{thm: wedge} to $\alpha^n$ and $\beta^n := \star \theta^n \in L^q\left(X,\bigwedge^{N-\ell}T^*X\right)$, one obtains that
\begin{align*}
\alpha^n \wedge \star \beta^n	 \longrightarrow \bara \wedge \star\barb + \sum_{k=1}^\infty \dd  \left(v^k\delta_{x^k}\right) \qquad \text{ in } \mathcal{D}'\left(X, \bigwedge^{N}T^*X\right),
\end{align*}
where $v_k$ are $(N-1)$-covectors. We conclude the proof by taking the Hodge star on both sides and identifying $r^k$ with $\star v^k$ via the musical isomorphism $ \sharp: TX \cong T^*X$, using the fact that both Hodge star and $\sharp$ are Sobolev space isometries. \end{proof}


\subsection{Smooth cycles} We observe from Theorem~\ref{thm: wedge} that, even in the critical case $\frac{1}{p} + \frac{1}{q} = \frac{1}{N} + 1$, the concentration distribution is ``exact''. Thus we make the following observation (where $\langle\bullet,\bullet\rangle$ denotes the duality paring between currents and differential forms):
\begin{corollary}
Let $\left(X^N,g\right)$ be an $N$-dimensional closed Riemannian manifold; $N \geq 2$. Let $1<p,q<\infty$ be such that $\frac{1}{p} + \frac{1}{q} = \frac{1}{N} + 1$. Consider differential forms $\{\alpha^n\}$ and $\{\beta^n\}$ as in Theorem~\ref{thm: wedge}, where $\ell_1 + \ell_2 = N = \dim X$. Then for any smooth  N-current $T \in \mathcal{R}^N(X)$ which is a cycle (\emph{i.e.}, $\p T = 0$), we have $\bra T, \alpha^n \wedge \beta^n \ket \to \bra T, \bara \wedge \barb\ket$ as $n \to \infty$. 
\end{corollary}  

\begin{proof}
This follows from $\bra T, \dd\left(\sum_k v^k\delta_{x^k}\right)\ket = \bra \p T, \sum_k v^k\delta_{x^k}\ket = 0$, where $ \dd\left(\sum_k v^k\delta_{x^k}\right)$   is the concentration distribution as in Theorem~\ref{thm: wedge}. Alternatively, rectifiable currents  with $\CC^\infty$-regularity can be represented by smooth vectorfields, for which the cycle condition $\p T=0$ is equivalent to the divergence-free condition. So the corollary also follows from $\dd^*\circ \dd^* = 0.$  \end{proof}







\section{The ``endpoint critical'' cases $\{p,q\} = \{1,N\}$}\label{sec: endpoint critical case}

In this section we discuss the generalisation of \S\S 3 $\&$ 4 in \cite{bcm}, \emph{i.e.}, the cases $(p,q) = (1,N)$ or $(p,q) = (N,1)$, to differential forms on closed manifolds. These are the endpoint cases of the critical regime $\frac{1}{p}+ \frac{1}{q} = \frac{1}{N}+1$, for which we need stronger compactness assumptions for $\dd\alpha^n$ and $\dd\beta^n$ (namely, Assumptions~(5)+(6) in Theorem~\ref{thm: endpoint}) to ensure the distributional convergence of wedge products. 

For the convenience of readers, let us reproduce Theorem~\ref{thm: endpoint} below.

\begin{theorem*}
Let $\left(X^N,g\right)$ be an $N$-dimensional closed Riemannian manifold; $N \geq 2$. 
Consider two sequences of differential forms $\left\{ \alpha^n \right\}\subset \M\left(X, \bigwedge^{\ell_1} T^*X \right)$ and $\left\{ \beta^n \right\}\subset L^N\left(X, \bigwedge^{\ell_2} T^*X \right)$ satisfying the following conditions:
\begin{enumerate}
\item
$\alpha^n \weak \bara$ weakly-$\star$ in $\M$;
\item
$\beta^n \weak \barb$ weakly in $L^N$;
\item
$\left| \alpha^n - \bara \right| \weak \mu$ weakly-$\star$ in $\M$;
\item
$\left| \beta^n - \barb \right|^N  \weak \nu$ weakly-$\star$ in $\M$;
\item
$\dd\alpha^n \to \dd \bara$ strongly in $W^{-1,N'}$;
\item
$\dd\beta^n \to \dd \barb$ strongly in $L^N$.
\end{enumerate}
Then we have the following convergence (modulo subsequences) in the sense of  distributions:
\begin{equation}
\alpha^n \wedge \beta^n \longrightarrow  \bara \wedge \barb + \sum_{k=1}^\infty \dd \left(v^k\delta_{x^k}\right) \qquad \text{in } \mathcal{D}'%\left(X, \bigwedge^{\ell_1+\ell_2-1}T^*X\right),
\end{equation}
for some sequences of points $\left\{x^k\right\} \subset X$ and $(\ell_1+\ell_2-1)$-covectors $\left\{v^k\right\}$. Moreover,
\begin{equation*}
\left|v^k\right| \lesssim_{N,X}
\left[\mu\left(\left\{x^k\right\}\right)\right]\left[\nu\left(\left\{x^k\right\}\right)\right]^{\frac{1}{N}}.
\end{equation*}
\end{theorem*}


Recall that $\M\left(X, \bigwedge^\bullet T^*X\right)$ is understood as the space of $\bigwedge^\bullet T^*X$-valued finite Radon measures on $X$, where $\bigwedge^\bullet T^*X$ is a vector space. Thus the total variation measure (hence the total variation norm) of each of its elements is well defined. Also note that $N'=\frac{N}{N-1}$.  


Our proof essentially follows the strategies in \cite[\S 3]{bcm}, and the idea is similar to that of Theorem~\ref{thm: wedge}, except that more delicate analysis for endpoint Sobolev embeddings are required. To avoid redundancies, we shall only elaborate on the necessary modifications.


\begin{proof}[Proof of Theorem~\ref{thm: endpoint}] We divide our arguments into seven steps below.

\smallskip
\noindent
{\bf Step 1.} As for Theorem~\ref{thm: wedge}, consider the decomposition
\begin{equation*}
\begin{cases}
\alpha^n = \dd\gamma_n + \xi^n,\\
\beta^n = \dd\zeta^n +\eta^n.
\end{cases}
\end{equation*}
The proof for Proposition~\ref{prop: decomposition of diff forms in wedge prod} yields that
\begin{align*}
&\zeta^n \weak \zeta  \text{ weakly in } W^{1,N}\left(X,\bigwedge^{\ell_2-1}T^*X\right),\\
&\eta^n \longrightarrow \eta \text{ strongly in } W^{1,N}\left(X,\bigwedge^{\ell_2}T^*X\right),\\
& \barb = \dd\zeta + \eta.
\end{align*}


Estimates for $\gamma_n$ and $\xi^n$ pertains to the endpoint case of  elliptic estimates \emph{\`{a} la} Bourgain--Brezis \cite{bb04} and Brezis--Van Schaftingen \cite{bv}. Thanks to \cite{bb04,bv}, $\gamma_n$ and $\xi^n$ can be chosen to verify
\begin{align*}
&\gamma_n \longrightarrow \gamma  \text{ strongly in } W^{1,N'}\left(X,\bigwedge^{\ell_1-1}T^*X\right),\\
&\xi^n \weak \xi \text{ weakly in } W^{-1,N'}\left(X,\bigwedge^{\ell_1}T^*X\right) \text{ and weakly-$\star$ in  }\M \left(X,\bigwedge^{\ell_1}T^*X\right),\\
& \bara = \dd\gamma + \xi.
\end{align*}



\smallskip
\noindent
{\bf Step 2.} Now let us define
\begin{align*}
\alpha^n \wedge \beta^n &:= \dd \gamma_n \wedge \dd\zeta^n  + \xi^n \wedge \dd\zeta^n + \dd\gamma_n \wedge \eta^n + \xi^n \wedge \eta^n \nonumber\\
&= J_1^n + J_2^n + J_3^n + J_4^n \qquad \text{in the sense of distributions}.
\end{align*}
Indeed, $J^n_3$ is a strong-strong pairing of $L^{N'}$ and $W^{1,N}$, while $J^n_4$ is a weak-strong pairing of $W^{-1,N'}$ and $W^{1,N}$; hence, they converge in the sense of distributions. Also, for $J_1^n$ one observes that $\gamma_n \wedge \dd\zeta^n$ is a weak-strong pairing of $L^N$ and $L^{N'}$, so it converges in the weak $L^1$-topology. As a consequence, we may define $\dd\gamma_n \wedge \dd\zeta^n := \dd \left(\gamma_n \wedge \dd\zeta^n\right)$ in the distributional sense.


It remains to investigate the (well-definedness as a distribution and) convergence of the weak-weak pairing $J_2^n$. 


\smallskip
\noindent
{\bf Step 3.} To see that $$J_2^n:=
\xi^n \wedge \dd\zeta^n$$ is well defined as a differential form-valued distribution on $X$, one needs to prove that for any testform $\varrho \in \CC^\infty\left(X, \bigwedge^{N-\ell_1-\ell_2} T^*X\right)$, the following expression is well defined:
\begin{equation*}
I^n \equiv \int_X \xi^n \wedge \dd\zeta^n \wedge \varrho.
\end{equation*}
Clearly there is nothing to prove when $\ell_1 + \ell_2 > N$.
 

For this purpose, we integrate by parts  using the superdistributivity of $\dd$ over wedge product and the Stokes' theorem to define
\begin{align*}
I^n &= I_1^n + I_2^n\\
&:= \left[(-1)^{\ell_1+1}\int_X \dd\xi^n \wedge \zeta^n \wedge \varrho\right] + \left[(-1)^{\ell_2+1} \int_X\xi^n \wedge \zeta^n \wedge \dd\varrho\right].
\end{align*}

 For $I^n_1$, we make use of the decomposition $\alpha^n = \dd\gamma_n + \xi^n$ to infer that $\dd\alpha^n = \dd\xi^n$ in the sense of distributions. But it is assumed that $\dd\alpha^n \to \dd \bara$ strongly in $W^{-1,N'}$ (see assumption~(5)), so the term $\dd\xi^n \wedge \zeta^n$ appearing in the integrand of $I^n_1$ is a strong-weak pairing of $W^{-1,N'}$ and $W^{1,N}$, thus making $I^n_1$ well defined and converge as $n \to \infty$ to $(-1)^{\ell_1 + 1} \int_X \dd\xi \wedge \zeta \wedge \varrho$ as desired. 



It now remains to investigate the convergence of
\begin{equation*}
\tilde{I}^n_2:=\int_X\xi^n \wedge \zeta^n \wedge \dd\varrho,
\end{equation*}
which is nothing but $I_2^n$ with the immaterial sign neglected. Here $\varrho$ is an arbitrary testform in $\CC^\infty\left(X, \bigwedge^{N-\ell_1-\ell_2} T^*X\right)$, and $\xi^n \wedge \zeta^n$ is a  weak-weak pairing of $W^{-1,N'}$ and $W^{1,N}$.


\smallskip
\noindent
{\bf Step 4.} To make sense of $\xi^n \wedge \zeta^n$,  recall that $\xi^n$ is chosen to be coexact: by Hodge decomposition $\xi^n = \dd^* k^n + h^n$ where $h^n$ is harmonic. Then one may find from Lemma~\ref{lem: brezis-vs} that $$\sigma^n = \Delta^{-1}\xi^n \in W^{1,N'}\left(X;\bigwedge^{\ell_1}T^*X\right),$$ which is unique by fixing the cohomology class once and for all. Moreover, we have $\dd^*\sigma^n=0$. 

The above arguments lead to
\begin{align*}
\tilde{I}^n_2 &= \int_X \Delta \sigma^n \wedge \zeta^n \wedge \dd\varrho \\ 
&= \int_X \left(\dd\dd^*\sigma^n+\dd^*\dd \sigma^n\right) \wedge \zeta^n \wedge \dd\varrho\\
&= \int_X \dd\sigma^n \wedge \dd^*\left[\zeta^n\wedge \dd\varrho\right]. 
\end{align*}
In the right-most term, $\left\{\dd\sigma^n\right\}$ is weakly convergent in $L^{N'}$, and $\left\{\dd^*\left[\zeta^n\wedge \dd\varrho\right]\right\}$ is weakly convergent in $L^{N}$, after passing to subsequences if necessary. Denote the weak limits of $\left\{\sigma^n\right\}$ in $W^{1,N'}$ (existence follows from the quantitative statement in Lemma~\ref{lem: brezis-vs}) and $\left\{\zeta^n\right\}$ in $W^{1,N}$ by $\sigma$ and $\zeta$, respectively.   We are now in the situation of applying Lemma~\ref{lem: appendix} to obtain a defect measure $\varpi = \varpi[\varrho] \in \M\left(X;\bigwedge^N T^*X \right)$ that is linear in the testform $\varrho$:
\begin{align*}
\lim_{n \to \infty} \tilde{I}^n_2 = \int_X \dd\sigma \wedge \dd^*\left[\zeta\wedge\dd\varrho\right] +  \varpi (X).
\end{align*}
By Stokes' theorem and the fact that $\dd^*\sigma=0$ in the sense of distributions, we can rewrite this identity as follows:
\begin{align*}
\lim_{n \to \infty} \tilde{I}^n_2 &= \int_X \Delta \sigma \wedge \zeta \wedge \dd\varrho + \varpi (X) \\&= \int_X \xi \wedge \zeta \wedge \dd\varrho + \varpi (X). 
\end{align*}


It thus remains to investigate the defect measure(-valued $N$-form) $\varpi$. 



\smallskip
\noindent
{\bf Step 5.} Consider the limiting measures 
\begin{align*}
&\lambda := \M-\lim_{n\to\infty} \left|\dd^*\big[\zeta^n \wedge \dd\varrho\big] - \dd^*\big[\zeta \wedge \dd\varrho\big] \right|^N,\\
&\lambda' := \M-\lim_{n\to\infty} \left|\dd \left(\sigma^n-\sigma\right) \right|^{N'}.
\end{align*}
We {\bf claim} that the total variation measure of $\lambda$ is controlled by $\nu$, the measure in assumption~(4):
\begin{equation}\label{claim, lambda and nu}
|\lambda|(B) \lesssim _{(X,g), \|\varrho\|_{\CC^2}} \nu(B)\qquad \text{ for any Borel } B \subset X.
\end{equation}

\begin{proof}[Proof of Claim~\eqref{claim, lambda and nu}]
It suffices to prove for $B = {\bf B}_1(0) \subset \R^N$, the Euclidean ball, and $X$ is a compact subset of $\R^N$. To see this, let $\left\{\chi_1, \ldots, \chi_K\right\}$ be a $\CC^\infty$-partition of unity  on the compact manifold $(X,g)$  subordinate to an atlas of $\CC^\infty$-charts, each of which is diffeomorphic to ${\bf B}_1(0)$. Then, viewing $|\bullet|^N$ as positive measures on $X$, we have
\begin{align*}
\left|\dd^*\big[\zeta^n \wedge \dd\varrho\big] - \dd^*\big[\zeta \wedge \dd\varrho\big] \right|^N(B)&=\left|\dd^*\left[\left(\zeta^n - \zeta\right) \wedge \dd\varrho\left(\sum_{i=1}^K \chi_i\right)\right] \right|^N(B)\\
&\lesssim_{N,K} \sum_{i=1}^K \big| \dd^* \left[\chi_i(\zeta^n-\zeta) \wedge \dd\varrho\right] \big|^N(B)\\
&\lesssim_{K,(X,g)} \left\|\varrho\right\|_{\CC^2}^N\cdot\left\{\sum_{i=1}^K \left|\na \left(\chi_i\zeta^n-\chi_i\zeta\right)\right|^N(B)\right\},
\end{align*}
where the last inequality depends on the $\CC^1$-geometry of $(X,g)$. So it is enough to prove for $\left(\chi_i\zeta^n,\chi_i\zeta\right)$ in place of $\left(\zeta^n,\zeta\right)$. Also, as $\zeta^n-\zeta$ is coexact, we may also assume that $\chi_i\zeta^n-\chi_i\zeta$ is coexact, since 
\begin{align*}
\left|\dd^* \left[\chi_i \left(\zeta^n-\zeta\right)\right]\right|^N =  \left|\chi_i \dd^*  \left(\zeta^n-\zeta\right) \right|^N + R,
\end{align*}
where the remainder term $R$ involves no derivative in $\zeta^n-\zeta$, hence $|R| \lesssim_{\left\|\chi_i\right\|_{\CC^1}} |\zeta^n-\zeta|^N \to 0$ as $n \to \infty$ in the sense of Radon measures. 

A similar computation for commuting $\na$ with the diffeomorphism between the chart containing the support of $\chi_i$ and the Euclidean ball ${\bf B}_1(0)$ shows that \eqref{claim, lambda and nu} is invariant under such diffeomorphisms. Thus, we may assume without loss of generality that $B={\bf B}_1(0)$ and $\chi_i\zeta^n-\chi_i\zeta$ is coexact and compactly supported in $B$ for each fixed $i \in \{1, \ldots, K\}$. We relabel $\left(\chi_i\zeta^n,\chi_i\zeta\right)$ in the above as $\left(\zeta^n,\zeta\right)$ from now on.


It remains to argue, in view of the previous reduction, that 
\begin{align}\label{gaffney in RN}
\left|\na\left(\zeta^n-\zeta\right)\right|^N(B) \lesssim \left|\dd \left(\zeta^n-\zeta\right)\right|^N(B). 
\end{align}
Indeed, assuming this and recalling from Step~1 that $\dd \left(\zeta^n-\zeta\right) = \left(\beta^n-\barb\right) - \left(\eta^n-\eta\right)$, where $\eta^n \to \eta$ strongly in $W^{1,N}$ and $\left|\beta^n-\barb\right|^N\weak \nu$ in $\M$ by Assumption~(4), we conclude the {\bf claim}. 

To see \eqref{gaffney in RN}, we consider a 
standard mollifier $\mathcal{J}_\delta$ acting on $\left|\na\left(\zeta^n-\zeta\right)\right|^N$, which is possible since $\zeta^n-\zeta$ is compactly supported in $B$. Then, for $s^n:=\na\left(\zeta^n-\zeta\right)$ we have that
\begin{align*}
\Big|\left\|s^n\right\|_{L^N(B)} - \left\|s^n\star\mathcal{J}_\delta\right\|_{L^N(B)}\Big| &\leq \left\|s^n-s^n \star \mathcal{J}_\delta\right\|_{L^N(B)} \lesssim \mathfrak{o}(1) \quad \text{ as } \delta \to 0^+. 
\end{align*}
So one can further assume $\zeta^n, \zeta \in \CC^\infty$. In this case, since $\dd^*(\zeta^n-\zeta)=0$ and $\zeta^n-\zeta \in W^{1,N}_0\left(B;\bigwedge^{\ell_2-1} \R^N\right)$ (so that the usual Poincar\'{e}'s inequality applies), we deduce \eqref{gaffney in RN} directly from the Gaffney's inequality. See \cite{gaff1, gaff2, gaff3, gaff4} among other references.  

The proof of the {\bf claim} is now complete.   
\end{proof}






\smallskip
\noindent
{\bf Step 6.} As argued in \S\ref{subsec, where conc comp is used}, it  remains to prove the bound
\begin{equation}\label{LN'-->L1 bd}
\left\{ \int_X|\varphi|^{N'}\,\dd\lambda' \right\}^{\frac{1}{N'}}   \lesssim_{N,X}   \int_X|\varphi|\,\dd \mu \quad \text{for any } \varphi \in \CC^\infty(X).
\end{equation}
Once this is established, an application of the second concentration compactness lemma \emph{\`{a} la} P.-L. Lions \cite[p.158, Lemma~I.1]{lions} together with Lemma~\ref{lem: appendix} readily concludes the proof. See \eqref{conc comp} and the ensuing arguments, as well as \cite[p.486, from Equation~(50) to the end of the proof of Theorem~3.1]{bcm}.

To this end, observe that
\begin{align}\label{Mlim}
\M-\lim_{n\to\infty}\left| \Delta \left(\sigma^n-\sigma\right)\right| &= \M-\lim_{n\to\infty}\left|\xi^n-\xi\right| \nonumber\\
&= \M-\lim_{n\to\infty}\left|\left(\alpha^n + \dd\gamma_n\right) - \left(\alpha+\dd\gamma\right)\right| \nonumber\\
&= \M-\lim_{n\to\infty} \left|\alpha^n-\alpha\right| \nonumber\\
&= \mu.
\end{align}
The penultimate equality follows from  the strong convergence $\dd\gamma_n \to \dd\gamma$ in $L^{N'}$, and the final one from the definition of $\mu$. Also, as $\sigma^n$ is coexact, so is $\Delta \left(\sigma^n-\sigma\right)$. By \eqref{Mlim} and  $\lambda' := \M-\lim_{n\to\infty} \left|\dd \left(\sigma^n-\sigma\right) \right|^{N'}$, we see that \eqref{LN'-->L1 bd} follows immediately from the bound
\begin{align}\label{LN'-->L1 bd, eqvt}
\limsup_{n \to \infty} \left\| \varphi\, \dd\left(\sigma^n-\sigma\right) \right\|_{L^{N'}(X)} &\lesssim_{N,X} \limsup_{n \to \infty} \left\|\varphi\, \Delta \left(\sigma^n-\sigma\right)\right\|_{\M(X)}\quad \text{for any } \varphi \in \CC^\infty(X).
\end{align}
[In contrast to \eqref{LN'-->L1 bd}, the norms $\|\bullet\|_{L^{N'}(X)}$ and $\|\bullet\|_{\M(X)}$ are taken with respect to the metric $g$ as usual.] 


To prove \eqref{LN'-->L1 bd, eqvt}, notice that it is equivalent to 
\begin{align}\label{LN'-->L1 bd, eqvt2}
\limsup_{n \to \infty} \left\|\dd \left\{\varphi\left(\sigma^n-\sigma\right)\right\} \right\|_{L^{N'}(X)} &\lesssim_{N,X} \limsup_{n \to \infty} \left\| \Delta\left\{\varphi\left(\sigma^n-\sigma\right)\right\}\right\|_{\M(X)},
\end{align}
in view of the identities 
\begin{align*}
\dd \left[\varphi\left(\sigma^n-\sigma\right)\right] = \dd\varphi \wedge\left(\sigma^n-\sigma\right) +  \varphi\, \dd\left(\sigma^n-\sigma\right),
\end{align*}
where $\dd\varphi \wedge\left(\sigma^n-\sigma\right) \to 0$ in $L^{N'}$, and 
\begin{align*}
 \Delta\left\{\varphi\left(\sigma^n-\sigma\right)\right\} = \left(\Delta\varphi\right)\left(\sigma^n-\sigma\right) + 2g^{ij} \na_i\varphi\na_j\left(\sigma^n-\sigma\right) + \varphi \Delta \left(\sigma^n-\sigma\right),
\end{align*}
where $\left(\Delta\varphi\right)\left(\sigma^n-\sigma\right) \to 0$ in $L^{N'}$ and $2g^{ij} \na_i\varphi\na_j\left(\sigma^n-\sigma\right) \weak 0$ weakly-$\star$ in $\M$.




It remains to justify \eqref{LN'-->L1 bd, eqvt2}. For the moment, let us work with \emph{an additional assumption}: 
\begin{align}\label{additional assumption}
\text{$X$ has no nontrivial harmonic $\ell_1$-form, namely that $H_{\rm dR}^{\ell_1}(X)=\{0\}$}.
\end{align}
It then follows from the endpoint elliptic estimate in Brezis--Van Schaftingen  \cite{bv} (see also Lemma~\ref{lem: brezis-vs} in the appendix) that
\begin{align*}
\left\|\dd \left\{\varphi\left(\sigma^n-\sigma\right)\right\} \right\|_{L^{N'}(X)} &\lesssim_{N,X} \left\| \Delta\left\{\varphi\left(\sigma^n-\sigma\right)\right\}\right\|_{\M(X)} + \left\|\dd^*\Delta\left\{\varphi\left(\sigma^n-\sigma\right)\right\}\right\|_{W^{-2,N'}}.
\end{align*}
The last term on the right-hand side vanishes under $\limsup_{n \to \infty}$, because
\begin{align*}
\dd^*\Delta\left\{\varphi\left(\sigma^n-\sigma\right)\right\} &= \Delta\dd^*\left\{\varphi\left(\sigma^n-\sigma\right)\right\}\\
&= (-1)^{N(\ell_1+1)+1} \Delta\star \dd\left\{ \varphi \left[\star\left(\sigma^n-\sigma\right)\right]\right\} \\
&= (-1)^{N(\ell_1+1)+1}  \Delta \star\Big\{ \dd\varphi \wedge \left[\star\left(\sigma^n-\sigma\right)\right] \Big\} + \Delta \left[\varphi\, \dd^*\left(\sigma^n-\sigma\right)\right] \\
&=: A_1+A_2,
\end{align*}
thanks to the commutativity between $\dd^*$ and $\Delta$, the definition of codifferential $\dd^*$, and the superdistributivity of $\dd$ under wedge product. The term $A_2$ is zero in $W^{-2,N'}$ since $\sigma^n-\sigma \in W^{1,N'}$ is coexact, while $\lim_{n \to \infty}\|A_1\|_{W^{-2,N'}}=0$ since $\sigma^n - \sigma$ converges weakly to zero in $W^{1,N'}$, hence strongly in $L^{N'}$ by Rellich's lemma.



Thus, \eqref{LN'-->L1 bd, eqvt2} is established under the hypothesis \eqref{additional assumption}, from which \eqref{LN'-->L1 bd} follows. 


\smallskip
\noindent
{\bf Step~7.} Finally, let us remove the additional assumption \eqref{additional assumption}. This follows from a simple observation: the statement of Theorem~\ref{thm: endpoint} that we are proving is essentially local. More precisely, it suffices to prove the theorem in each local chart, and then globalise it via partition of unity. Thus, without loss of generality, one may assume that $X$ is $\ell_1$-connected (indeed, contractible). 



The proof of the theorem is now complete.   \end{proof}


 







\section{Application to isometric immersions}\label{sec: isom imm}








Our main result of this section is the following theorem, which generalises Theorem~\ref{thm: isom imm, 2D} in the Introduction \S\ref{sec: intro} to arbitrary dimensions and codimensions. 


\begin{theorem}\label{thm: isom imm}
Let $\left(X,g\right)$ be a Riemannian immersed submanifold of dimension $N \geq 2$ in $\R^{N+k}$. Consider a family $\left\{\two^\e\right\} \subset L^p_\loc\left(X; {\rm Hom}\left(TX \otimes TX, T^\perp X\right)\right)$ of weak solutions to the Gauss--Codazzi--Ricci equations which converges to $\overline{\two}$ in the weak-$L^p_\loc$-topology, where $p \in \left]\frac{2N}{N+1},\infty\right[$. Suppose that the coexact parts of $\left\{\two^\e\right\}$ are precompact in strong $L^{p'}_\loc$-topology. Then $\overline{\two}$ is still a weak solution to the Gauss--Codazzi--Ricci equations. 
 \end{theorem}
 
 As the main scope of the current paper is on wedge product theorems in compensated compactness theory, we refrain ourselves from giving a detailed exposition on  isometric immersions and/or the Gauss--Codazzi--Ricci equations. Let us only point out that the Gauss--Codazzi--Ricci equations are compatibility PDEs of curvatures for the existence of isometric immersions. See do Carmo \cite[Chapter~6]{doc} and Tenenblat \cite{ten} for the derivation of these equations, and the monograph \cite{hh} for histories and up-to-date developments of the isometric immersions problem.


For closed manifold $X$, 
\begin{equation}\label{coexact part, leray}
\text{coexact part of } \two^\e = \left({\bf Id} - \dd\Delta^{-1}\dd^*\right) \two^\e,
\end{equation}
where $\Delta^{-1}$ is defined with respect to any cohomology group. 


In passing we remark that, for any vectorfield $V \in \G(TX)$, the expression $\left({\bf Id} - \dd\Delta^{-1}\dd^*\right) V$ modulo musical isomorphisms between $TX$ and $T^*X$ is the well-known \emph{Leray projection} of $V$.








\begin{proof}[Proof of Theorem~\ref{thm: isom imm}]
Without loss of generality, we assume that $X$ is compact and hence drop the subscripts ``loc''.


Consider the structural equation
\begin{align}\label{structural eq}
\dd\Omega^\e + \Omega^\e \wedge \Omega^\e = 0. 
\end{align}
Here $\left\{\Omega^\e\right\}$ is bounded in $L^p$, as $$\Omega^\e = \begin{bmatrix}
\na & \two^\e\\
-\two^\e & \na^{\perp,\e}
\end{bmatrix}$$ via the Cartan formalism. One may refer to Clelland \cite{clelland}, Tenenblat \cite{ten}, as well as \cite{cl1, cl2} for details on the Cartan formalism applied to isometric immersions.

By Rellich's lemma, $\left\{\dd\Omega^\e\right\}$ is bounded in $W^{-1,p}$. In fact, we have a stronger mode of convergence --- consider the Hodge decomposition $\Omega^\e = \dd\Upsilon_\e + \Xi^\e$, where $\Xi^\e$ is the divergence-free part that is assumed to be precompact in $L^{p'}$. So $\left\{\dd\Omega^\e\right\} = \left\{\dd\Xi^\e\right\}$ is precompact in $W^{-1,p'}$. 

Now, from the generalised wedge product Theorem~\ref{thm: wedge}, we infer that $$\Omega^\e \wedge \Omega^\e \to \overline{\Omega} \wedge \overline{\Omega}\qquad\text{in the sense of distributions.}$$  Here, as in \S\ref{subsec: weak weak pairing}, the wedge product is defined as follows:
\begin{align*}
\Omega^\e \wedge \Omega^\e = \dd\left(\Upsilon_\e \wedge \dd\Upsilon_\e \right) + \Xi^\e\wedge\Omega^\e + \dd\Upsilon_\e \wedge \Xi^\e.
\end{align*}
and similarly for $\overline{\Omega} \wedge \overline{\Omega}$. The last two terms on the right-hand side are weak-strong pairings of $L^p$ and $L^{p'}$. The first term is also well defined in the sense of distributions ---  $\left\{\Upsilon_\e\right\}$ is bounded in $W^{1,p}$, hence is precompact in $L^{p^\star - \delta}$ for arbitrary $\delta>0$; recall that $p^\star = \frac{Np}{N-p}$ is the Sobolev conjugate of $p$. Thus $\left(\Upsilon_\e \wedge \dd\Upsilon_\e\right)$ is a weak-strong pairing provided that $\frac{1}{p} + \frac{1}{p^\star} < 1$. But this condition is equivalent to $p>\frac{2N}{N+1}$. Note also that we are always in the subcritical regime in the sense of \S\ref{subsec: subcrit}: this is because $\frac{2}{p} < 1+ \frac{1}{N}$ whenever $p>\frac{2N}{N+1}$. 

Therefore, we may pass to the distributional limits separately for $\dd\Omega^\e$ and $\Omega^\e\wedge\Omega^\e$ in \eqref{structural eq}. In this way, one obtains $\dd\overline{\Omega} + \overline{\Omega}\wedge \overline{\Omega} = 0$ in the sense of distributions.  \end{proof}

We actually proved that the wedge product sequence $\left\{\Omega^\e\wedge\Omega^\e\right\}$ converges in a stronger topology than $\mathcal{D}'$; that is, in the negative Sobolev space $W^{-1,p}$. 

\begin{remark}
The above theorem and its proof distinguishes for the Gauss--Codazzi--Ricci equations one critical exponent $p_{\bf crit} := \frac{2N}{N+1}$ other than the obvious exponent $p_{\bf CS}=2$. (The latter is critical due to the Cauchy--Schwarz inequality.) Observe that
\begin{align}\label{critical exponent}
p_{\bf crit} = \frac{2N}{N+1} < p \leq p_{\bf CS}=2 \leq p' < \frac{2N}{N-1} = p_{\bf crit}^\star \equiv p_{\bf crit}'.
\end{align}


\end{remark}


\bigskip
\appendix
\section{A lemma on weak-weak pairing}

We have the following variant of \cite[Lemma~2.11]{bcm} by Briane--Casado-D\'{i}az--Murat:
\begin{lemma}\label{lem: appendix}
Assume that $1<r<\infty$, and let $E$ be a locally compact Hausdorff topological space. Assume that $\{u_n\} \subset L^r_\loc(E;\C)$ and $\{u'_n\} \subset L^{r'}_\loc(E;\C)$ satisfy $|u_n-u|^r\weak \lambda$ and $|u'_n-u|^{r'} \weak \lambda'$, both weakly-$\star$ in $\M_\loc(E;\C)$. Then, for each quadratic polynomial $\mathbf{q}$, there exists a Radon measure $\varpi \in \M_\loc(E;\C)$ such that 
\begin{align*}
{\bf q}\left(u_n, u_n'\right) \weak {\bf q}\left(u,u'\right) + \varpi \qquad \text{weakly-$\star$ in $E$},
\end{align*}
with the total variation of $\varpi$ satisfying the bound
\begin{align*}
|\varpi|(B) \lesssim_{\bf q} \left[\lambda(B)\right]^{\frac{1}{r}}\left[\lambda'(B)\right]^{\frac{1}{r'}}\qquad \text{ for any $B \subset E$ Borel}.
\end{align*} 
 
\end{lemma}


\section{Endpoint elliptic estimates}

The following result is a variant of \cite[Theorem~3.1]{bv} by Brezis--Van Schaftingen and its generalisation to Radon measures on Euclidean balls in \cite[Proposition~B.1, Appendix B]{bcm}:

\begin{lemma}\label{lem: brezis-vs}
Let $\left(X^N,g\right)$ be a closed Riemannian manifold, and let $\xi \in L^1\left(X,\bigwedge^\ell T^*X\right)$ be a differential form of $L^1$-regularity. Then we can solve for $\sigma \in W^{1,N'}\left(X,\bigwedge^\ell T^*X\right)$ from $\Delta \sigma = \xi$ on $X$, where $\Delta = \dd^*\dd + \dd\dd^*$ is the Laplace--Beltrami operator. The solution is unique modulo harmonic $\ell$-forms, and one has
\begin{align}\label{estimate in Lemma B}
\inf\left\{ \left\|\dd\sigma + h\right\|_{L^{N'}}:\, h \text{ is a harmonic form} \right\} \lesssim_{(X,g)} \|\xi\|_{L^1} + \left\|\dd^*\xi\right\|_{W^{-2,N'}}.
\end{align}

The same result holds for $\xi \in \M\left(X,\bigwedge^\ell T^*X\right)$, with the right-hand side of \eqref{estimate in Lemma B} replaced by $\|\xi\|_{\M(X)}$, namely the total variation norm of $\xi$. Recall that $N'=\frac{N}{N-1}$.
\end{lemma}


Note that \cite[Theorem~3.1]{bv} is originally formulated as an $L^1$-estimate under the  Dirichlet boundary condition on Euclidean domains. By a routine partition of unity  argument one can generalise it to closed manifold $(X,g)$, but the uniqueness of solution can only be retained modulo harmonic $\ell$-forms, which are nontrivial in general on $X$. %This observation explains the necessity of Step~7 in the proof of Theorem~\ref{thm: endpoint}.



\bigskip
\noindent
{\bf Acknowledgement}.
The research of SL is supported by NSFC (National Natural Science Foundation of China) Project $\#$12201399, the SJTU-UCL joint seed fund WH610160507/067, and the Shanghai Frontier Research Institute for Modern Analysis. I am indebted to Professor Armin Schikorra, from whom I learned a lot about Hodge decomposition and compensated compactness over time, and I thank Professor Gui-Qiang Chen for his unwavering support. I also thank New York University-Shanghai for providing excellent working atmosphere during my adjunct professorship and visiting scholarship. 

\begin{thebibliography}{99}

\bibitem{afp}
Ambrosio, Luigi; Fusco, Nicola; Pallara, Diego. \textit{Functions of Bounded Variation and Free Discontinuity Problems}, Oxford Mathematical Monographs. Oxford University Press Inc., New York, 2000. ISBN 0-19-850245-1.

%\bibitem{acm}
%Amrouche, Chérif; Ciarlet, Philippe G.; and Mardare, Cristinel. On a lemma of Jacques-Louis Lions and its relation to other fundamental results. \textit{J. Math. Pures Appl. (9)} \textbf{104} (2015), no. 2, 207--226. 

%\bibitem{ael}
%Antonić, Nenad; Erceg, Marko; Lazar, Martin. Localisation principle for one-scale H-measures. \textit{J. Funct. Anal.} \textbf{272} (2017), no. 8, 3410--3454.

\bibitem{ball}
Ball, John M.
Convexity conditions and existence theorems in nonlinear elasticity. \textit{Arch. Rational Mech. Anal.}  \textbf{63} (1976/77), no. 4, 337--403.

\bibitem{bco}
Ball, John M.; Currie, John C.; Olver, Peter J. Null Lagrangians, weak continuity, and variational problems of arbitrary order. \textit{J. Functional Analysis} \textbf{41} (1981), no. 2, 135--174. 

\bibitem{bds}
Bandyopadhyay, Saugata; Dacorogna, Bernard; Sil, Swarnendu. Calculus of variations with differential forms. \textit{J. Eur. Math. Soc. (JEMS)} \textbf{17} (2015), no. 4, 1009--1039.


%\bibitem{bb98}
%Bellieud, Michel; Bouchitté, Guy.  Homogenization of elliptic problems in a fiber reinforced structure. Nonlocal effects. \textit{Ann. Scuola Norm. Sup. Pisa Cl. Sci. (4)} \textbf{26} (1998), no. 3, 407--436.

\bibitem{gaff2}
Borchers, Wolfgang; Sohr, Hermann. On the equations $rot v=g$ and $div u=f$ with zero boundary conditions. \textit{Hokkaido Math. J.} \textbf{19} (1990): 67--87.

\bibitem{bb04}
Bourgain, Jean; Brezis, Haïm. New estimates for the Laplacian, the div–curl, and related Hodge systems. \textit{C. R. Acad. Sci. Paris Ser. I} \textbf{338} (2004), no. 7, 539--543.

\bibitem{brezis}
Brezis, Haïm; Nguyen, Hoai-Minh. The Jacobian determinant revisited. \textit{Invent. Math.} \textbf{185} (2011), no. 1, 17--54.

\bibitem{bv}
Brezis, Haïm; Van Schaftingen, Jean. Boundary estimates for elliptic systems with $L^1$-data. \textit{Calc. Var. Partial Differential Equations} \textbf{30} (2007), no. 3, 369--388.

\bibitem{bc}
Briane, Marc; Casado-Díaz, Juan. A new div-curl result. Applications to the homogenization of elliptic systems and to the weak continuity of the Jacobian. \textit{J. Differential Equations} \textbf{260} (2016), no. 7, 5678--5725.

\bibitem{bcm}
Briane, Marc; Casado-Díaz, Juan; Murat, François. The div-curl lemma ``trente ans après'': an extension and an application to the G-convergence of unbounded monotone operators. \textit{J. Math. Pures Appl. (9)} \textbf{91} (2009), no. 5, 476--494.


\bibitem{chw1}
Cao, Wentao; Huang, Feimin; Wang, Dehua.  Isometric immersions of surfaces with two classes of metrics and negative Gauss curvature. \textit{Arch. Ration. Mech. Anal.} \textbf{218} (2015), no. 3, 1431--1457.


\bibitem{chw2}
Cao, Wentao; Huang, Feimin; Wang, Dehua.  Isometric immersion of surface with negative Gauss curvature and the Lax-Friedrichs scheme. \textit{SIAM J. Math. Anal.} \textbf{48} (2016), no. 3, 2227--2249.



\bibitem{chen}
Chen, Gui-Qiang G. Weak continuity and compactness for nonlinear partial differential equations. \textit{Chinese Ann. Math. Ser. B} \textbf{36} (2015), no. 5, 715--736. 

\bibitem{cl1}
Chen, Gui-Qiang G.; Li, Siran. Global weak rigidity of the Gauss--Codazzi--Ricci equations and isometric immersions of Riemannian manifolds with lower regularity. \textit{J. Geom. Anal. } \textbf{28} (2018), no. 3, 1957--2007.


\bibitem{cl2}
Chen, Gui-Qiang G.; Li, Siran. Weak continuity of the Cartan structural system and compensated compactness on semi-Riemannian manifolds with lower regularity. \textit{Arch. Ration. Mech. Anal.} \textbf{241} (2021), no. 2, 579--641.


\bibitem{csw0}
Chen, Gui-Qiang, G.; Slemrod, Marshall; Wang, Dehua. Weak continuity of the Gauss-Codazzi-Ricci system for isometric embedding. \textit{Comm. Math. Phys} \textbf{294} (2010), no. 2, 411–-437

\bibitem{csw}
Chen, Gui-Qiang, G.; Slemrod, Marshall; Wang, Dehua. Weak continuity of the Gauss-Codazzi-Ricci system for isometric embedding. \textit{Proc. Amer. Math. Soc.} \textbf{138} (2010), no. 5, 1843--1852.

\bibitem{cg}
Chen, Gui-Qiang G.; Giron, Tristan P. Weak continuity of curvature for connections in $L^p$. 
ArXiv preprint, math.AP. (2021), arXiv:2108.13529 




\bibitem{elas1}
Ciarlet, Philippe G.; Gratie, Liliana; Mardare, Cristinel. A new approach to the fundamental theorem of surface theory. \textit{Arch. Ration. Mech. Anal.} \textbf{188} (2008), no. 3, 457--473. 


\bibitem{clelland}
Clelland, Jeanne N. \textit{From Frenet to Cartan: The Method of Moving Frames}. Graduate Studies in Mathematics, vol. 178. American Mathematical Society, Providence (2017).

\bibitem{clms}
Coifman, Ronald R.; Lions, Pierre-Louis; Meyer, Yves; Semmes, Stephen. Compensated compactness and Hardy spaces. \textit{J. Math. Pures Appl. (9)} \textbf{72} (1993), no. 3, 247--286.

\bibitem{gaff3}
Csat\'{o}, Gyula; Dacorogna, Bernard; Sil, Swarnendu. On the best constant in Gaffney inequality. \textit{J. Funct. Anal.} \textbf{274} (2018), no. 2, 461--503.

\bibitem{dafermos}
Dafermos, Constantine M. \textit{Hyperbolic conservation laws in continuum physics.} Fourth edition. Grundlehren der mathematischen Wissenschaften [Fundamental Principles of Mathematical Sciences], 325. Springer-Verlag, Berlin, 2016. xxxviii+826 pp. 

\bibitem{daf}
Dafni, Galia.  Nonhomogeneous div-curl lemmas and local Hardy spaces. \textit{Adv. Differential Equations} \textbf{10} (2005), no. 5, 505--526.

\bibitem{dip}
DiPerna, Ronald J. Compensated compactness and general systems of conservation laws. \textit{Trans. Amer. Math. Soc.} \textbf{292} (1985), no. 2, 383--420. 


\bibitem{doc}
do Carmo, Manfredo Perdigão. \textit{Riemannian Geometry}. Translated from the second Portuguese edition by Francis Flaherty. Mathematics: Theory $\&$ Applications. Birkhäuser Boston, Inc., Boston, MA, 1992. 

\bibitem{evans}
Evans, Lawrence Craig. \textit{Weak Convergence Methods for Nonlinear Partial Differential Equations.} Seminario, Providence 1990.

%\bibitem{fnp}
%Feireisl, Eduard; Novotný, Antonín; Petzeltová, Hana. On the existence of globally defined weak solutions to the Navier--Stokes equations. \textit{J. Math. Fluid Mech.} \textbf{3} (2001), no. 4, 358--392.

\bibitem{flm}
Fonseca, Irene; Leoni, Giovanni; Malý, Jan. Weak continuity and lower semicontinuity results for determinants. 
\textit{Arch. Ration. Mech. Anal.} \textbf{178} (2005), no. 3, 411--448.

\bibitem{gaff1}
Gaffney; Matthew P. A special Stokes' theorem for complete manifolds. \textit{Ann. of Math.} \textbf{60} (1954), 140--145.

\bibitem{grs}
Guerra, André; Raiţă, Bogdan; Schrecker, Matthew R.~I. Compensated compactness: continuity in optimal weak topologies. \textit{J. Funct. Anal.} \textbf{283} (2022), no. 7, Paper No. 109596, 46 pp.


\bibitem{hh}
Han, Qing; Hong, Jia-Xing. \textit{Isometric embedding of Riemannian manifolds in Euclidean spaces}. Mathematical Surveys and Monographs, 130. American Mathematical Society, Providence, RI, 2006.


\bibitem{h}
Hélein, Frédéric. \textit{Harmonic Maps, Conservation Laws and Moving Frames}. Cambridge University Press, Cambridge, 2002.

\bibitem{io}
Iwaniec, Tadeusz; Onninen, Jani. $\mathcal{H}^1$-estimates of Jacobians by subdeterminants. \textit{Math. Ann.} \textbf{324} (2002), no. 2, 341--358.

\bibitem{ky}
Kozono, Hideo; Yanagisawa, Taku. Global compensated compactness theorem for general differential operators of first order. \textit{Arch. Ration. Mech. Anal.}  \textbf{207} (2013), no. 3, 879--905.

\bibitem{elas2}
Kupferman, Raz; Solomon, Jake P. A Riemannian approach to reduced plate, shell, and rod theories. \textit{J. Funct. Anal.} \textbf{266} (2014), no. 5, 2989--3039. 


\bibitem{elas3}
Kupferman, Raz; Maor, Cy; Shachar, Asaf. Reshetnyak rigidity for Riemannian manifolds. \textit{Arch. Ration. Mech. Anal.} \textbf{231} (2019), no. 1, 367--408.

\bibitem{ms}
Mazowiecka, Katarzyna; Schikorra, Armin. Fractional div-curl quantities and applications to nonlocal geometric equations. \textit{J. Funct. Anal.} \textbf{275} (2018), no. 1, 1--44.

\bibitem{mm}
Mišur, Marin; Mitrović, Darko. On a generalization of compensated compactness in the $L^p$-$L^q$ setting. \textit{J. Funct. Anal.} \textbf{268} (2015), no. 7, 1904--1927. 

\bibitem{ls}
Lanzani, Loredana; Stein, Elias M. A note on div-curl inequalities. \textit{Math. Res. Lett.} \textbf{12} (2005), no. 1, 57--61. 

\bibitem{lmzw}
Li, Chun; McIntosh, Alan G.~R.; Zhang, Kewei; Wu, ZhiJian.  Compensated compactness, paracommutators, and Hardy spaces. \textit{J. Funct. Anal.} \textbf{150} (1997), no. 2, 289--306.

\bibitem{liarma}
Li, Siran. On the existence of $C^{1,1}$-isometric immersions of several classes of negatively curved surfaces into $\R^3$. \textit{Arch. Ration. Mech. Anal.} \textbf{236} (2020), no. 1, 419--449.

\bibitem{gaff4}
Li, Siran. A new proof of Gaffney's inequality for differential forms on manifolds-with-boundary: the variational approach \emph{à la} Kozono-Yanagisawa. \textit{Acta Math. Sci. Ser. B (Engl. Ed.)} \textbf{42} (2022), no. 4, 1427--1452.

\bibitem{lions}
Lions, Pierre-Louis. The concentration-compactness principle in the calculus of variations. The limit case. I. \textit{Rev. Mat. Iberoamericana} \textbf{1} (1985), no. 1, 145--201. 

\bibitem{mar}
Mardare, Sorin. On Pfaff systems with $L^p$ coefficients and their applications in differential geometry, \textit{J. Math. Pures Appl.} (9), \textbf{84} (2005), 1659--1692.

\bibitem{mul}
Müller, Stefan. Weak continuity of determinants and nonlinear elasticity, \textit{C. R. Acad. Sci. Paris I} \textbf{307} (1988) 501--506.

\bibitem{mur1}
Murat, François.  Compacité par compensation. \textit{Ann. Scuola Norm. Sup. Pisa Cl. Sci. (4)} \textbf{5} (1978), no. 3, 489--507.

\bibitem{mur2}
Murat, François. 
Compacité par compensation. II.  \textit{
Proceedings of the International Meeting on Recent Methods in Nonlinear Analysis}
(Rome, 1978), pp. 245--256, Pitagora, Bologna, 1979.


\bibitem{mur3}
Murat, François.  Compacité par compensation: condition nécessaire et suffisante de continuité faible sous une hypothèse de rang constant, \textit{Ann. Scuola Norm. Sup. Pisa Cl. Sci.} (4) \textbf{8} (1981), no. 1, 69--102.

\bibitem{pau}
Pauly, Dirk. A global div-curl-lemma for mixed boundary conditions in weak Lipschitz domains and a corresponding generalized $A^*_0$-$A_1$-lemma in Hilbert spaces. \textit{Analysis (Berlin)} \textbf{39} (2019), no. 2, 33--58.

\bibitem{res}
Re\u{s}etnjak, Yuri. Weak convergence and completely additive vector functions on a set, \textit{Sibir. Math.} \textbf{9} (1968) 1039--1045.

\bibitem{rin1}
Rindler, Filip.  Directional oscillations, concentrations, and compensated compactness via microlocal compactness forms.  \textit{Arch. Ration. Mech. Anal.} \textbf{215} (2015), no. 1, 1--63.

\bibitem{key1}
Robbin, Joel W.; Rogers, Robert C.; Temple, Blake. On weak continuity and the Hodge decomposition. \textit{Trans. Amer. Math. Soc.} \textbf{303} (1987), no. 2, 609--618.

%\bibitem{key2}
%Rogers, Robert C.; Temple, Blake. A characterization of the weakly continuous polynomials in the method of compensated compactness. \textit{Trans. Amer. Math. Soc.} \textbf{310} (1988), no. 1, 405--417.

\bibitem{sch}
 Schwarz, Günter.  Hodge decomposition -- a method for solving boundary value problems. \textit{Lecture Notes in Mathematics}, 1607. Springer-Verlag, Berlin, 1995. viii+155 pp. 

\bibitem{sil}
Šilhavý, Miroslav.  Normal currents: structure, duality pairings and div-curl lemmas.  \textit{Milan J. Math.} \textbf{76} (2008), 275--306.

\bibitem{tar1}
Tartar, Luc. Compensated compactness and applications to partial differential equations. \textit{Nonlinear analysis and mechanics: Heriot-Watt Symposium, Vol. IV}, pp. 136--212, Res. Notes in Math., 39, Pitman, Boston, Mass.-London, 1979. 



\bibitem{tar2}

Tartar, Luc. The compensated compactness method applied to systems of conservation laws. \textit{Systems of nonlinear partial differential equations (Oxford, 1982)}, 263--285, NATO Adv. Sci. Inst. Ser. C: Math. Phys. Sci., 111, Reidel, Dordrecht, 1983.



%\bibitem{tar3}
%Tartar, Luc.  Compensated compactness with more geometry. \textit{Differential geometry and continuum mechanics}, pp. 3--26, Springer Proc. Math. Stat., 137, Springer, Cham, 2015.


\bibitem{ten}
Tenenblat, Keti. {On isometric immersions of Riemannian manifolds}, \textit{Bol. Soc. Brasil. Mat.} \textbf{2} (1971), 23--36.


\bibitem{wau}
Waurick, Marcus. A functional analytic perspective to the div-curl lemma. \textit{J. Operator Theory} \textbf{80} (2018), no. 1, 95--111.


\bibitem{zhou}
Zhou, Yi. An $L^p$ theorem for compensated compactness. 
\textit{Proc. Roy. Soc. Edinburgh Sect. A} \textbf{122} (1992), no. 1--2, 177--189.



\end{thebibliography}

\end{document}
