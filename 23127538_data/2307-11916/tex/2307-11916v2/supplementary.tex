
% %%%%%%%%%%%%%%%%% END OF PREAMBLE %%%%%%%%%%%%%%%%
% Use only LaTeX2e, calling the article.cls class and 12-point type.
\documentclass[12pt]{article}
% \documentclass[sn-nature]{sn-jnl}
\usepackage[comma, super, sort&compress]{natbib}

\usepackage{placeins}
% Tables
\usepackage[table,xcdraw]{xcolor}
\usepackage{amsfonts,amsmath,amssymb,amsthm}

% The preamble here sets up new/revised commands and
% \usepackage{scicite}
\usepackage{braket}

% Font package
\usepackage{times}
\usepackage{subcaption}

%Units
\usepackage{siunitx}
\DeclareSIUnit{\Torr}{Torr}
\DeclareSIUnit{\atomicpercent}{at.\%}
\DeclareSIUnit{\percent}{\%}
\DeclareSIUnit{\sccm}{sccm}
\DeclareSIUnit{\electronvolts}{eV}
\DeclareSIUnit{\ppm}{ppm}
\DeclareSIUnit{\ppb}{ppb}
\DeclareSIUnit{\gauss}{gauss}
\DeclareSIUnit{\rpm}{rpm} %radian per minute
\DeclareSIUnit{\electronvolt}{eV}
\DeclareSIUnit{\count}{ct}

\sisetup{detect-weight=true, detect-family=true}

%Chemical formulas using kroeger-vink notation
\usepackage{chemformula}
\setchemformula{kroeger-vink=true}

% Clickable ref to citations
\usepackage{hyperref}
\hypersetup{
    colorlinks=true,
    linkcolor=black,
    filecolor=black,      
    urlcolor=black,
    citecolor=black,
}
\urlstyle{same}

% % The following parameters provide a reasonable page setup.
\topmargin -0.5cm
\oddsidemargin 0.2cm
\textwidth 16cm 
\textheight 22cm
\footskip 1.0cm

\renewcommand{\thefigure}{S\arabic{figure}}
\setcounter{figure}{0}
\renewcommand{\thetable}{S\arabic{table}}
\setcounter{table}{0}

% TODO Package
\setlength {\marginparwidth }{2cm}
\usepackage[colorinlistoftodos]{todonotes}
\usepackage{float, epstopdf}

\begin{document} 
% Include your paper's title here
\title{Supplementary information: Microwave-based quantum control and coherence protection of tin-vacancy spin qubits in a strain-tuned diamond membrane heterostructure} 
% Make the title.
\author
{Xinghan Guo$^{1,\dagger}$, Alexander M. Stramma$^{2,\dagger}$, Zixi Li$^{1}$, William G. Roth$^{2}$, \\
Benchen Huang$^{3}$, Yu Jin$^{3}$, Ryan A. Parker$^{2}$, Jesús Arjona Martínez$^{2}$,\\
Noah Shofer$^{2}$, Cathryn P. Michaels$^{2}$, Carola P. Purser$^{2}$, Martin H. Appel$^{2}$,\\
Evgeny M. Alexeev$^{2,4}$,Tianle Liu$^{5}$,
Andrea C. Ferrari$^{4}$, David D. Awschalom$^{1,5,6}$, \\ 
Nazar Delegan$^{1,6}$, Benjamin Pingault$^{6,7}$, 
Giulia Galli$^{1,3,6}$, \\
F. Joseph Heremans$^{1,6}$, Mete Atatüre$^{2,\ast}$, Alexander A. High$^{1,6,\ast}$\\
\textit{\normalsize{$^{1}$Pritzker School of Molecular Engineering, University of Chicago,
Chicago, IL 60637, USA}}\\
\textit{\normalsize{$^{2}$Cavendish Laboratory, University of Cambridge, Cambridge CB3 0HE, United Kingdom}}\\
\textit{\normalsize{$^{3}$Department of Chemistry, University of Chicago, Chicago, IL 60637, USA}}\\
\textit{\normalsize{$^{4}$Cambridge Graphene Centre, University of Cambridge, }}\\
\textit{\normalsize{Cambridge CB3 0FA, United Kingdom }}\\
\textit{\normalsize{$^{5}$Department of Physics, University of Chicago, Chicago, IL 60637, USA}}\\
\textit{\normalsize{$^{6}$Center for Molecular Engineering and Materials Science Division, }}\\
\textit{\normalsize{Argonne National  Laboratory, Lemont, IL 60439, USA}}\\
\textit{\normalsize{$^{7}$ QuTech, Delft University of Technology, 2600 GA Delft, The Netherlands}} \\
\textit{\normalsize{$^{\dagger}$ These authors contributed equally to this work.}}\\
\textit{\normalsize{$^\ast$E-mail: ma424@cam.ac.uk}}\\
\textit{\normalsize{$^\ast$E-mail: ahigh@uchicago.edu}}
}

% Include the date command, but leave its argument blank.
\date{}
\maketitle 
%%%%%%%%%%%%%%%%% END OF PREAMBLE %%%%%%%%%%%%%%%%
\section{Tin vacancy center (\ch{SnV-}) in strained diamond membranes}
\subsection{Hamiltonian of the strained \ch{SnV-}}
\label{section:Hamiltonian}
The \ch{SnV-} center is a spin-$1/2$ system. In a mean-field orbital picture, the system has three electrons in four spin orbitals ($\{\ket{e_x \uparrow}, \ket{e_x \downarrow}, \ket{e_y \uparrow}, \ket{e_y \downarrow}\}$). Both its electronic ground and excited states are doubly-degenerate; the degeneracy may be lifted by applying strain and/or by spin-orbit interaction.  We write the spin Hamiltonian of the system in the minimum model of 4 electrons and 3 orbitals, for the ground ($g$) and excited ($u$) state $H_{g, u}$, as the sum of four terms: spin-orbit (SO) interaction ($\hat{H}_{\text{SO}}$); electron-phonon interaction due to the Jahn-Teller effect; strain field, and interaction with an external, static magnetic field $B$ (Zeeman effect, $\hat{H}_{Z}$). Following Ref~\cite{Hepp2014}, we write the term arising from Jahn-Teller distortions in the same form as that describing the strain interaction. Below we merge the two terms into one, that for simplicity we call $\hat{H}_{\text{strain}}$. Hence the Hamiltonian is written as:
\begin{equation}
    \hat{H}_{\text{sys}} = \hat{H}_{\text{SO}} + \hat{H}_{\text{strain}} + \hat{H}_{Z}. \label{eq:system_hamiltonian}
\end{equation}
In the following three subsections, we discuss each term of the Hamiltonian.

\subsubsection{Spin-orbit coupling}
The component of the orbital angular momentum operator $\hat{L}_x, \hat{L}_y$ vanish for the Hamiltonian expressed in the $\{\ket{e_x}, \ket{e_y}\}$ basis~\cite{Hepp2014} and only the following term is non-zero:
\begin{equation}
    \hat{L}_z = \left[\begin{matrix}
    0 & -i\\
    i & 0
    \end{matrix}\right],
\end{equation}
where we have set $\hbar$ to 1. Therefore, using the $\{\ket{e_x \uparrow}, \ket{e_x \downarrow}, \ket{e_y \uparrow}, \ket{e_y \downarrow}\}$ basis, the SO Hamiltonian can be represented as:
\begin{align}
    \hat{H}_{SO}=\lambda\hat{L}_z\hat{S}_z=\frac{\lambda}{2}
    \left[\begin{matrix}
    0 & -i\\
    i & 0
    \end{matrix}\right] \otimes
    \left[\begin{matrix}
    1 & 0\\
    0 & -1
    \end{matrix}\right]=
        \left[\begin{matrix}
    0 & 0 & -i\lambda/2 & 0\\
    0 & 0 & 0 & i\lambda/2\\
    i\lambda/2 & 0 & 0 & 0\\
    0 & -i\lambda/2 & 0 &0
    \end{matrix}\right].  
\end{align}

\subsubsection{Strain field}
The term of the Hamiltonian representing the presence of a strain field can be written as:
\begin{align}\label{strainhamiltonian}
    \hat{H}_{\text {strain}}=\left[\begin{array}{cc}
\varepsilon_{A_{1}}-\varepsilon_{E_{x}} & \varepsilon_{E_{y}} \\
\varepsilon_{E_{y}} & \varepsilon_{A_{1}}+\varepsilon_{E_{x}}
\end{array}\right] \otimes \mathbb{I}_{2}.
\end{align}
The elements $\{\varepsilon_{A_1}, \varepsilon_{E_x}, \varepsilon_{E_y}\}$ represent the energy response induced by strain belonging to the different irreducible representations $A_1, E_x, E_y$ of the $D_{3d}$ point group of the defect, and are expressed in the \ch{SnV-} center's local frame, where the $z$-axis corresponds to the high symmetry axis of the \ch{SnV} which is the quantization axis. For example,  $\varepsilon_{A_1} = \braket{\Psi|(H - H_0)|\Psi}$, where $H_0$ is the electronic Hamiltonian in the absence of strain and $H$ is the electronic Hamiltonian, which includes the strain field applied to the supercell by changing the lattice parameters. Here $\ket{\Psi}$ represents a Slater determinant expressed in the $\{\ket{e_x \uparrow}, \ket{e_x \downarrow}, \ket{e_y \uparrow}, \ket{e_y \downarrow}\}$ basis.

We can write each term of Eq. \ref{strainhamiltonian} as a linear combination of the components of the strain tensor ($\epsilon$):
\begin{equation}
\begin{split}
    \varepsilon_{A_{1}} & = t_{\perp} \left(\epsilon_{xx} + \epsilon_{yy} \right) + t_{\|} \epsilon_{zz},\\
    \varepsilon_{E_{x}} & = d\left(\epsilon_{xx}-\epsilon_{yy}\right) + f \epsilon_{zx},\\
    \varepsilon_{E_{y}} & = -2d \epsilon_{xy} + f \epsilon_{yz},
\end{split}
\end{equation}
where $\epsilon_{xx}, \epsilon_{yy}, \epsilon_{zz}$ represent the diagonal components of the strain tensor in the $x, y, z$ directions and $\epsilon_{xy}, \epsilon_{yz}, \epsilon_{zx}$ represent the shear strain components; $t_{\perp}, t_{\|}, d$, and $f$ are partial derivatives written as $\frac{\partial \varepsilon_{A_1}}{\partial (\epsilon_{xx} + \epsilon_{yy})}, \frac{\partial \varepsilon_{A_1}}{\partial \epsilon_{zz}}, \frac{\partial \varepsilon_{E_x}}{\partial (\epsilon_{xx} - \epsilon_{yy})}, \frac{\partial \varepsilon_{E_x}}{\partial \epsilon_{zx}}$, respectively. These four strain-susceptibility parameters completely describe the strain-response of the ground and excited electronic states. In the following,  we ignore the diagonal term $\epsilon_{A_{1}}$, which amounts to  a global emission wavelength shift. Hence, the strain Hamiltonian has the following form: 
\begin{align}
    \hat{H}_{\text {strain }}=\left[\begin{array}{cc}
-\varepsilon_{E_{x}} & \varepsilon_{E_{y}} \\
\varepsilon_{E_{y}} & \varepsilon_{E_{x}}
\end{array}\right] \otimes \mathbb{I}_{2} = \left[\begin{array}{cccc}
-\varepsilon_{E_{x}} & 0 & \varepsilon_{E_{y}} &0 \\
0 & -\varepsilon_{E_{x}} & 0 & \varepsilon_{E_{y}}\\
\varepsilon_{E_{y}} &0 & \varepsilon_{E_{x}} & 0\\
0 & \varepsilon_{E_{y}} & 0 & \varepsilon_{E_{x}}
\end{array}\right]. \label{eq:strain_hamiltonian}
\end{align}

\subsubsection{Zeeman effect} 
\label{subsection: Zeeman}
Due to the $D_{3d}$ symmetry of the defect, the orbital component $H_{Z,L}$ of the Hamiltonian  $H_{Z}$ only includes a term  $\hat{L}_z B_z$~\cite{Hepp2014}, with a pre-factor $q$~\cite{thiering2018ab}, called in the literature  {\it effective reduction factor}, accounting for: (i) electron-phonon interaction (so-called Ham term), and  (ii) the symmetry of the defect being lower than $O(3)$ (so-called Steven's factor). Note that both terms have different values for the ground and excited states and hence the $q$ parameter is different in the ground and excited states. 
%We will discuss the impact of the crystal strain to the value of $q$ in the section \textcolor{red}{Fill this later}. 
The $H_{Z}$ Hamiltonian is written as the sum of an orbital $H_{Z,L}$ and spin component $H_{Z,S}$
\begin{equation} \label{eq:Zeeman}
    \hat{H}_Z =\hat{H}_{Z,L}+\hat{H}_{Z,S}= q \mu_B\gamma_L \hat{L}_z B_z + g\mu_B \hat{\textbf{S}} \cdot \textbf{B} - 2\mu_B\delta_f \hat{S}_z B_z,
    %&= q \gamma_{L} \left[\begin{smallmatrix}
%0 & 0 & -i B_{z} & 0 \\
%0 & 0 & 0 & -i B_{z} \\
%i B_{z} & 0 & 0 & 0 \\
%0 & i B_{z} & 0 & 0
%\end{smallmatrix}\right] +\frac{\gamma_{S}}{2}\left[\begin{%smallmatrix}
%(1-2\delta_f)B_{z} & B_{x}-i B_{y} & 0 & 0 \\
%B_{x}+i B_{y} & -(1-2\delta_f)B_{z} & 0 & 0 \\
%0 & 0 & (1-2\delta_f)B_{z} & B_{x}-i B_{y} \\
%0 & 0 & B_{x}+i B_{y} & -(1-2\delta_f)B_{z}
%\end{smallmatrix}\right], \nonumber 
\end{equation}
where $\mu_B$ is the Bohr magneton and $B_x, B_y, B_z$ are the components of the external, static magnetic field along the crystal frame $x, y, z$ directions, respectively. The last term on the right hand side of Eq. \ref{eq:Zeeman} originates from correcting with a factor $\delta$ the electronic Landé $g$ factor to account for spin-phonon interaction mediated by spin-orbit coupling~\cite{thiering2018ab}. For all the experimental interpretations, we'll only use the Ham factor and set the Steven's factor as $1$ except section~\ref{section:optical splitting with B field} where we gave estimates on the possible values of Steven's factor.

\subsection{Strain susceptibility}
\label{section: strain susceptibility}
In the presence of a strain field, the degeneracy of the ground ($gs$) and excited ($es$) states is lifted and we call $\triangle_{\text{gs}(\text{es})}$ the energy difference between the two states split by the degeneracy. By diagonalizing the strain Hamiltonian defined in Eq.~\ref{eq:strain_hamiltonian}, we obtain:
\begin{equation}
    \triangle_{\text{gs}(\text{es})} = 2\sqrt{[d_{\text{gs}(\text{es})}(\epsilon_{xx} - \epsilon_{yy}) + f_{\text{gs}(\text{es})}\epsilon_{zx}]^2 + [-2d_{\text{gs}(\text{es})}\epsilon_{xy} + f_{\text{gs}(\text{es})}\epsilon_{yz}]^2},
    \label{eq:strain_splitting}
\end{equation}
where the strain-susceptibilities are computed from density functional theory (DFT) calculations. We performed DFT calculations employing both the PBE~\cite{perdew1996generalized} and SCAN~\cite{sun2015strongly} functionals, and a 511-atom supercell with a [0.5, 0.5] occupation number for the $\ket{e_x \downarrow}, \ket{e_y \downarrow}$ orbitals. We approximated the splittings by the energy difference of the corresponding Kohn-Sham (KS) orbitals. The strain susceptibilities $d_{\text{gs}}, d_{\text{es}}, f_{\text{gs}}, f_{\text{es}}$ can be obtained from Eq.~\ref{eq:strain_splitting} by varying the lattice parameters of the supercell to generate $(\epsilon_{xx} - \epsilon_{yy})$ and $\epsilon_{zx}$ strain, respectively. Our results are summarized in Table.~\ref{tab:strain_susceptibility}. Note the similarity of results obtained with the two different functionals.

\begin{table}[h!]
\centering
\begin{tabular}{c c c c c}
 \hline
 Functional & $d_{\text{gs}}$ & $d_{\text{es}}$ & $f_{\text{gs}}$ & $f_{\text{es}}$ \\
 \hline
 PBE & 0.787 & 0.956 & -0.562 & -2.555 \\ 
 SCAN & 0.834 & 0.921 & -0.563 & -2.592 \\
 \hline
\end{tabular}
\caption{Computed strain susceptibilities (see text) of the SnV$^-$ defect in diamond, in units of  PHz/strain, obtained with the PBE and SCAN functionals.}
\label{tab:strain_susceptibility}
\end{table}

\subsection{Strain magnitude simulation}
We use COMSOL to simulate the strain profile of the suspended area measured in experiments. Since the strain expression $\epsilon$ in section \ref{section:Hamiltonian} is defined in local \ch{SnV} frame while the simulation result $\Tilde{\epsilon}$ returns to the lab frame, a combination of rotation matrices are applied. From $\braket{100}$ to $\braket{110}$ to $\braket{111}$, the rotation operators are $\hat{R}_z(45^\circ)$ and $\hat{R}_y(54.7^\circ)$, respectively:
\begin{align}
    \epsilon = \hat{R}^{\dagger}_y(54.7^\circ) \hspace{1mm} \hat{R}^{\dagger}_z(45^\circ) \hspace{1mm} \Tilde{\epsilon} \hspace{1mm} \hat{R}_z(45^\circ) \hspace{1mm} \hat{R}_y(54.7^\circ)
\end{align}
Here $\hat{R}_y(\theta)$ and $\hat{R}_z(\theta)$ refer to: 
\begin{align}
\label{eq:rotation}
    \hat{R}_y(\theta) = \left[\begin{matrix}
    \text{cos}(\theta) & 0 & \text{sin}(\theta)\\
    0 & 1 & 0\\
    -\text{sin}(\theta) & 0 & \text{cos}(\theta)
    \end{matrix}\right], \hspace{10mm}
    \hat{R}_z(\theta) = \left[\begin{matrix}
    \text{cos}(\theta) & -\text{sin}(\theta) & 0\\
    \text{sin}(\theta) & \text{cos}(\theta) & 0\\
    0 & 0 & 1
    \end{matrix}\right]
\end{align}

In COMSOL simulation, we use the actual three dimensional (3D) geometry for the diamond membrane and the trench. The temperature-dependent thermal expansion ratio for diamond and fused silica are obtained from these references:\cite{hahn1972thermal, oikawa1999thermal, reeber1996thermal, mcskimin1972elastic}. The initial strain-free temperature is set to \SI{450}{\celsius} which is the HSQ healing temperature \cite{siew2000thermal}, while the final temperature is set to \SI{4}{\kelvin}. We note that thermal expansion ratios for both fused silica and diamond become negligible below \SI{30}{\kelvin}, thus the simulated strain profile is nearly constant within the temperature range of interest (\SIrange{1.7}{7}{\kelvin}) in this study. The simulated structure and the strain distributions of $E_{xx}$ and $E_{yy}$ are shown in Figure \ref{subfig:COMSOL strain}. Since the off-diagonal shear strain is 2-3 orders of magnitude smaller than the diagonal tensile strain, we use the following matrix to represent the simulated strain value:
\begin{align}
    \Tilde{\epsilon} = \left[\begin{matrix}
    1.3e^{-3} & 0 & 0\\
    0 & 6.8e^{-4} & 0\\
    0 & 0 & -2.5e^{-4}
    \end{matrix}\right], \hspace{10mm}
    \epsilon = \left[\begin{matrix}
    1.6e^{-4} & -1.8e^{-4} & 5.8e^{-4}\\
    -1.8e^{-4} & 9.9e^{-4} & -2.5e^{-4}\\
    5.8e^{-4} & -2.5e^{-4} & 5.8e^{-4}
\end{matrix}\right]
\end{align}
We note that although $\Tilde{\epsilon}$ only includes diagonal elements, the transformed strain tensor $\epsilon$ in SnV local frame contains non-negligible off-diagonal elements which could affect the properties of the SnV center through both $d$ and $f$ parameters. By comparing the simulated branch splitting value using equation \ref{eq:strain_splitting} and PBE results with the actual experimental values, we observed the actual strain to be 0.55 times the simulated value, as shown in Figure \ref{subfig:COMSOL strain} (d). This magnitude mismatch could come from either the approximation of energy splittings from KS orbitals being inaccurate, the mismatch of the thermal expansion ratios between COMSOL simulation and reality, or an even lower softening temperature of HSQ rather than the healing temperature \cite{siew2000thermal}. More comprehensive studies of the energy response to strain would require a higher level of method, e.g., embedding theory~\cite{sheng2022green}, which we left for future investigations. In the following calculations, we add this 0.55 pre-factor to the simulated strain tensor to best capture the system properties.

% Figure environment removed

\subsection{Strain magnitude discussion}
We qualitatively categorize the strain magnitude to different regimes via the ground state energy splitting $\Delta_{gs}$. In the spin-orbit regime, this energy splitting is nearly constant, while the splitting is linear with the external strain when in high-strain regime. Guided by that, we use $\Delta_{gs}=\SI{1200}{\giga\hertz}$ as the boundary between the spin-orbit regime and the intermediate regime, and $\Delta_{gs}=\SI{2600}{\giga\hertz}$ to identify intermediate and high strain regime. For our work, those values corresponding to strain magnitudes of \SI{0.055}{\percent} and \SI{0.143}{\percent}. Here to plot the optical transitions, the magnetic field is set to \SI{80}{\milli\tesla} along the quantization axis ($\braket{111}$ direction) and the strain profile is set to be the same as COMSOL simulated profile but with an additional scaling factor. The calculated relative energy difference of the four \{$A1, A2, B1, B2$\} transitions are plotted in Figure \ref{subfig:spin conserving}. Compared with the defined low strain ``spin-orbit'' regime and the high strain regime, our experimentally observed result sits in between, indicating a non-trivial intermediate region where neither spin-orbit coupling or strain shall we treated as perturbation terms. We note that unlike SiV centers \cite{meesala2018strain}, SnV obtains different quenching $q$ factors at ground (0.471) and excited states (0.125) \cite{thiering2018ab}, leading to a non-zero splitting between spin-conserving transitions. We also extrapolate a qubit frequency $\omega_s$ of \SI{2.1}{\giga\hertz} which is lower than the ODMR frequency reported in the main text (\SI{2.755}{\giga\hertz}). This mismatch could originate from the slight difference between the displayed and the real magnetic field due to hysteresis and the deviation of the effective reduction factors under strain from that in Ref~\cite{thiering2018ab}.

% Figure environment removed

% \subsection{Two operation regimes of the SnV} 
% Except for the very low strain or high strain regime, the orientation of the B field has to be carefully aligned to some specific directions to maximize the splitting between spin conserving transitions. As will be discussed in section \textcolor{red}{Fill this later}, There are two potential operation regimes, with B field either roughly parallel or perpendicular to the quantization axis. In this section we use analytical method to characterize the two scenarios. 

% Two easy version of the Hamiltonian with different configurations: B field aligned to the quantization axis or aligned to the equator of the axis.

% Note: this part is a bit off the main topic so it is commented out for now.

\section{MW control of the SnV}
\subsection{MW magnetic response}
At zero strain, the SnV spin qubit cannot be manipulated by microwave due to different orbits associated with the spin level. As stated in \cite{meesala2018strain}, the presence of the crystal strain introduces orbit superposition to SnV's spin qubit eigenstates, allowing for the coherent control of SnV via external microwave field with qubit frequency $\omega_s$. Here we use the electronic $g$ factor to characterize the ability of the MW field (AC B filed) to the spin state of the SnV, including both spin and orbit response of the external magnetic field:
\begin{align}
    g = \frac{2}{\mu_{B}}\bra{e1\downarrow}(\hat{H}_{Z,L}^{ac}+\hat{H}_{Z,S}^{ac})\ket{e2\uparrow}
\end{align}   
\begin{align}
    \hat{H}_{Z,L}^{ac}+\hat{H}_{Z,S}^{ac} = \left[\begin{matrix}
    B_{z}^{ac} & B_{x}^{ac}-i B_{y}^{ac} & -i q B_{z}^{ac} & 0 \\
    B_{x}^{ac}+i B_{y}^{ac} & -B_{z}^{ac} & 0 & -i q B_{z}^{ac} \\
    i q B_{z}^{ac} & 0 & B_{z}^{ac} & B_{x}^{ac}-i B_{y}^{ac} \\
    0 & i q B_{z}^{ac} & B_{x}^{ac}+i B_{y}^{ac} & -B_{z}^{ac}
    \end{matrix}\right]
\end{align}

Here the $\textbf{B}^{ac}$ is a vector with unitary length indicating the direction of the oscillating B field of the microwave. The $\bra{e1\downarrow}$ and $\ket{e2\uparrow}$ are the two spin states of the SnV under external, static B field. The Ham reduction factor $q$ of the ground state is set to 0.471 according to \cite{thiering2018ab}. First we investigate the effect of strain magnitude to the transverse and longitudinal $g$ factor. The result is shown in Figure \ref{subfig:MW g factor} (a), indicating a $g$ factor of 1.64. Here the static \textbf{B} field set to be \SI{80}{\milli\tesla} along the quantization axis, which is in line with our experimental setup. We then investigate the angular dependence of the transverse $g$ factor with different static field orientations using our experimentally observed strain profile. We note that the $g$ factor has a weak angular dependence, indicating a consistently efficient MW driving efficiency regardless of the static B field orientation, highlighting the robustness of the strained SnV centers.

% Figure environment removed

\subsection{Device info}
In this work we utilized on-chip coplanar waveguide (CPW) to deliver microwave signals to target SnV centers. Compared with wire-bonded metal striplines \cite{nv_microwave_wire}, lithography-defined CPW offers deterministic and reproducible microwave power and magnetic field strength at target location. We designed our waveguide to match the impedance ($50 \Omega$) of other electronics in the setup. Ignoring the local dielectric variation near the diamond membrane region, we designed the layout of the CPW based on the permittivity of the fused silica (3.7). The width of the center and the gap is set to \SI{60}{\micro\meter} and \SI{6}{\micro\meter}, respectively. To enhance the local field strength near the SnV region on the membrane, the center of the CPW is reduced to \SI{6}{\micro\meter}. The ground lines of the waveguide is designed to across the membrane to compensate for the trench design, offering a balanced microwave delivering mode. We used a two-port microwave transmission design, demonstrating the potential of driving centers in multiple on-chip devices in the future. The two-port design also allows the microwave signal to be transmitted and dissipated outside of the chamber, relaxing the requirements for reflected signal management (such as circulator). The microscope image of the whole CPW design is shown in Figure \ref{subfig:MW device} (a), with the transmission data of an identical device shown in Figure \ref{subfig:MW device} (b). We show that the transmission loss is low from dc to \SI{15}{\giga\hertz}, with the thermal loss ($P_{\text{in}} - P_{\text{out}} - P_{\text{reflected}}$) around the operation point (\SI{2.75}{\giga\hertz}) to be 1.5 dB. 

% Figure environment removed

\subsection{MW field simulation}
% Figure environment removed
We use COMSOL to simulate the magnetic field acting on nearby color centers. In the simulation, we set the microwave drive power to 24 dBm, the microwave drive frequency to \SI{2.75}{\giga\hertz}, and set the characteristic impedance of the coplanar waveguide to 50 Ohm. The simulated structure and the magnetic field distribution are presented in \autoref{fig:mwsimulation}. Based on the simulation results, we expect the effective B field applied to the color centers ranges from \SI{0.2}{\tesla} to \SI{0.4}{\tesla}, corresponding to a transverse B field of \SI{0.12}{\tesla} to \SI{0.23}{\tesla}.

\subsection{Heating effect of the system}
To investigate the power dependence of the SnV Rabi oscillation, we sweep the MW drive power and extract the Rabi frequency. We observe the expected $\sqrt{p}$-behaviour for low drive powers $p<24$ dBm, but a clear deviation for larger drive powers. All the power and voltages are referred to the estimated value on the device, extracted by a separate calibration measurements in transmission geometry. We note that no increase in cryostat temperature is observed during the pulsed Rabi measurement.

% Figure environment removed

The effect of heating on the emitter can be modeled as depicted in figure \ref{subfig:MW heat} (a) where we follow the approach taken in Ref \cite{lukin_heating}. The emitter is treated as a point-like object at a fixed distance $X$ from the microwave line. Compared to the SnV center the extent of the gold strip is well approximated as infinite such that we can model this as a 1D problem. Assuming the gold heats and cools instantaneously at the beginning and end of a microwave pulse, a solution to the 1D heat equation yields a temperature increase at the SnV center $\Delta T_{SnV} \propto erf\left(\frac{d}{\sqrt{4 \alpha t}}\right)$ where $\alpha$ is the thermal diffusivity in diamond. Figure \ref{subfig:MW heat} (b) shows that the temperature at the emitter at asymptotically approaches the temperature of the microwave line. When higher Rabi frequencies are used another interesting effect is observed for more complex pulse sequences. Figure \ref{subfig:MW heat} (c) shows the effect of a sequence of pulses with a constant interpulse spacing $\tau_0$. If $\tau_0$ is significantly smaller than the time per pulse, the heat cannot flow away fast enough such that a net heating effect is observed per applied pulse. This means at high Rabi frequencies the coherence time of the spin can depend on the time between pulses. \\

% Figure environment removed
\clearpage

\section{Additional optical properties of the strained SnV}
\subsection{Polarization of the SnV}
We probe the polarization of the strained SnVs by inserting a motor-mounted half-wave-plate and a linear polarizer in the detection path. 
The total intensity of the C-peak and the D-peak over polarisation angle are shown in Fig. \ref{subfig:CPeak}) and Fig. \ref{subfig:DPeak}), respectively. No magnetic field was applied.  $0^\circ$ in the graph indicates the magnet $x$-axis. The solid line is a fit of the expected polarisation, linear for the C-peak and circular for the D-peak, projected into the lab-frame according to the model in Ref. \cite{Hepp2014}. Both figures indicate a polarisation behaviour commensurate with bulk group IV color centers, showing that the polarisation is not changed when introducing strain. 

% Figure environment removed

\subsection{Optical lifetime}
We extract the optical lifetime of the SnV by driving the C-transition at zero magnetic field with a single EOM-sideband and turning it off abruptly. The fall time is limited to \SI{200}{\pico\second} by the EOM. The decay time of the single-exponential is \SI{4.933\pm0.190}{\nano\second} which is similar to the bulk value, as shown in Fig. \ref{subfig:Lifetime} \cite{trusheim2020transform}.
% Figure environment removed

\subsection{Power saturation}
We extract the initialization rate, optical cyclicity and saturation power by prior knowledge of the optical lifetime and by sweeping the laser power \cite{Debroux2021}. The initialisation rates are fitted by $\frac{1}{\eta}\frac{\Gamma}{2}\frac{p/p_\text{sat}}{1+p/p_\text{sat}}$ and we extract a saturation power of \SI{7.96}{\nano\watt} and an optical cyclicity of $\eta \approx 2018$. For the microwave spin control measurement we operate at a saturation parameter of $s=p/p_\text{sat}\approx 10$ for the initialization and readout pulses.

% Figure environment removed


\subsection{Long term stability of PLE}
We acquire PLE for more than 11 hours to test the long-time stability of the SnV (see Fig. \ref{subfig:LongtimePLE}). We observe a very good frequency stability and only modest spectral wandering. 
% Figure environment removed
\clearpage

We fit each acquired PLE trace and extract the common mode shift of the spin-conserving transition \cite{arjona_indistinguishability} (Fig. \ref{subfig:CommonMode}).The Gaussian distribution of the shot-to-shot center frequencies has a standard deviation of $\sigma=$\SI{23.8 \pm 0.1}{\mega\hertz}. Similarly, the distribution of extracted spin-conserving splittings (Fig. \ref{subfig:SplittingShift}) has a standard deviation of only $\sigma=$\SI{13.28 \pm 0.06}{\mega\hertz}.

% Figure environment removed


\section{Optical control of the SnV spin}
\subsection{Optical splitting with external B field} \label{section:optical splitting with B field}
We scan the magnetic field over the whole sphere at fixed magnitude. The path between the approximately equidistant points is numerically minimised. The hysteresis of the $B$-field is on the order of \SI{10}{\percent} as estimated from linear sweeps along a single magnet axis.

The splitting of the $A1, B2$ optical transitions with varying $B$ fields can be computed by diagonalizing the system Hamiltonian $H_{\text{sys}}$ of Eq.~\ref{eq:system_hamiltonian}, and the results are shown in Fig~\ref{subfig:optical_splitting_B_field} (b) and (c). When constructing the Hamiltonian, we considered the Steven's term $g_L$ in the reduction factor $q$ as a free parameter. The Steven's term, as discussed in subsection~\ref{subsection: Zeeman}, originates from the defect symmetry being lower than $O(3)$. Here we determined the range of $g_L$ by matching the experiments. We plot the difference of the splitting when the $B$ field is aligned with the defect quantization axis ($\theta_B = 0$), and aligned along the equator ($\theta_B = \pi/2$) with varying $g_L \in [0, 1]$ in Fig~\ref{subfig:optical_splitting_B_field} (c). The white region in the plot (values close to zero) corresponds to the two splittings being close in energy, matching the experimental observations. Therefore our calculations enabled the narrowing down of the the possible values of Steven's factor to $g_{L, \text{gs}} \in [0.5, 1.0]$ and $g_{L, \text{es}} \sim 2g_{L, \text{gs}} - 1$.

% Figure environment removed
\clearpage

%\begin{equation}
%    H_{\text{sys}} = \left[\begin{matrix}
%        -\epsilon_{E_x} + \frac{(\gamma_S - 2\delta_f)}{2}B_z & \frac{\gamma_S}{2}B_x & \epsilon_{E_y} -\frac{i\lambda}{2} - iq\frac{\gamma_S B_z}{2} & 0\\
%        \frac{\gamma_S}{2}B_x & -\epsilon_{E_x} - \frac{(\gamma_S - 2\delta_f)}{2}B_z & 0 & \epsilon_{E_y} + \frac{i\lambda}{2} - iq\frac{\gamma_S B_z}{2}\\
%        \epsilon_{E_y} + \frac{i\lambda}{2} + iq\frac{\gamma_S B_z}{2} & 0 & \epsilon_{E_x} + \frac{(\gamma_S - 2\delta_f)}{2}B_z & \frac{\gamma_S}{2}B_x\\
%        0 & \epsilon_{E_y} -\frac{i\lambda}{2} + iq\frac{\gamma_S B_z}{2} & \frac{\gamma_S}{2}B_x & \epsilon_{E_x} - \frac{(\gamma_S - 2\delta_f)}{2}B_z
%    \end{matrix}\right]. \nonumber
%\end{equation}


\subsection{Optical cyclicity of the SnV}
We coarsely align the $B$-field by matching it to the polarisation of the optical dipoles (see SI section 3.1) and obtain an optical cyclicity of $\eta \approx 2018$. The cyclicity has a single local maximum close to the pole of the emitter axis, such that we can increase it by sweeping two of the three magnet axes independently. We extract the cyclicity by measuring the decay rate of one of the spin-conserving transitions. The frequency of the sidebands driving the transitions is fixed, noting that the change in $B$-field magnitude corresponds to a change in spin-conserving splitting within one optical linewidth. One can see only a modest increase in cyclicity in Fig. \ref{subfig:Bx_scan} and \ref{subfig:By_scan}, such that we conclude that strain limits the maximum achievable cyclicity. Nevertheless, the error introduced by the finite cyclicity will be negligible in spin-photon entanglement protocols due to the relatively high value and enable single-shot readout with nanostructures or microcavities.

% Figure environment removed

The optical cyclicity, as a branching ratio between spin-conserving and spin-flipping transitions, depends on both system properties and the external optical field. Theoretically, we can only investigate the system property side. Here we investigate an alternative problem---spontaneous emission rate ratio between spin-conserving and spin-flipping transitions---optical cyclicity with absence of the external optical excitation.

According to Ref~\cite{Hepp2014}, we use optical dipole matrices to calculate the emission rate of the two transitions. The rate (probablity) can be expressed using Fermi's Golden Rule:
\begin{equation}
    P =2 \pi \rho |\bra{\psi_f}|e \cdot \hat{\textbf{r}} |\ket{\psi_i}|^2 = 2 \pi \rho |\bra{\psi_f}|\hat{\textbf{p}} |\ket{\psi_i}|^2
\end{equation}
Where $\rho$ is the density of states where we set to 1, the $\ket{\psi_f}$ and $\ket{\psi_i}$ are the final and initial state of the SnV which we assign to the excited state minimum $\ket{e_A \downarrow}$ and ground states $\ket{e_1 \downarrow}$ ($\ket{e_2 \uparrow}$) for spin conserving (flipping) transitions. The transition probablity (rate) $P$ is related to the optical dipole $\hat{\textbf{p}}$ which is defined as:

\begin{align}
    \hat{p_x} = e\left[\begin{matrix}
    1 & 0 & 0 & 0\\
    0 & 1 & 0 & 0\\
    0 & 0 & -1 & 0\\
    0 & 0 & 0 & -1
    \end{matrix}\right],
    \hat{p_y} = e\left[\begin{matrix}
    0 & 0 & -1 & 0\\
    0 & 0 & 0 & -1\\
    -1 & 0 & 0 & 0\\
    0 & -1 & 0 & 0
    \end{matrix}\right],
    \hat{p_z} = e\left[\begin{matrix}
    1 & 0 & 0 & 0\\
    0 & 1 & 0 & 0\\
    0 & 0 & 1 & 0\\
    0 & 0 & 0 & 1
\end{matrix}\right]
\end{align}
Using the above definition, we can calculate the spin flip ratio which is the inverse of spontaneous cyclicity $\frac{1}{\eta}=\frac{P_{\text{flipping}}}{P_{\text{conserving}}}$ with respect to the strain magnitude and the $B$ field polar angle $\theta$, as shown in Figure \ref{subfig:cyclicity}. The operation point of the MW-based (all-optical) control of the SnV spin qubit is highlighted in white (black) stars, showing a cyclicity of \SI{\geq2000}{} if $\theta<\SI{4}{\degree}$ and a cyclicity of \SI{\approx20}{} if $\theta>\SI{85}{\degree}$, in a rough agreement with the experimentally observed values. We note that the presence of the moderate-level strain will make the overall cyclicity lower than the strain-free case, but the achievable value is still compatible with single shot readout requirements if the signal count can be improved by device design or setup optimization. 

% Figure environment removed

\subsection{All-optical spin control and operation at perpendicular $B$-fields}
We implemented the all-optical control technique shown in Ref. \cite{Debroux2021} on Device 1 on a strained SnV with a ground state splitting of $\Delta_\text{GS}=$\SI{1384}{\giga\hertz}. We extracted an optical lifetime of \SI{7.04\pm0.10}{\nano\second} which is compatible for an SnV in proximity of a surface \cite{gorlitz2020}. We set the magnetic field to $|B| = 100 $ mT perpendicular to the emitter axis. From the saturation power measurement in Fig. \ref{subfig:SatCurvePerp} we extract a saturation power of $p_\text{sat} = $\SI{4.82\pm0.81}{\nano\watt} and a cyclicity of $\eta = $ \SI{5.78\pm0.36}. The low saturation power and low cyclicity indicate that efficient all-optical control should be possible in principle. 

% Figure environment removed

We implement an optical lambda system on the spin-conserving transition A1 and spin-flipping transition A2 and measure coherent population trapping (CPT) by driving both transitions simultaneously (Fig. \ref{subfig:CPT}). Fitting the data with the model in Ref. \cite{Debroux2021,fleischhauer2005electromagnetically}, we get an excited state decay rate of  $\Gamma/2 \pi = $\SI{26.52\pm0.91}{\mega\hertz}, only a factor of 1.17 larger than the transform-limited linewidth $\Gamma_0/(2 \pi)$ = \SI{22.60\pm0.05}{\mega\hertz}.

% Figure environment removed

For all-optical Raman control, we choose to operate at a single-photon detuning of $\Delta = 1.5$ GHz. At the lowest laser sideband powers of $p=455$ nW, we get an intrinsic ODMR linewidth of $\delta f =1/T_{2^*} = $ \SI{899\pm54}{\kilo\hertz} (Fig. \ref{subfig:ODMRAllOptical}). The qubit frequency of $f_\text{qubit} = 2.321 $ GHz, yields a $g$-factor of $g=0.83$, meaning that perpendicular fields can couple to the SnV efficiently due to strain.

% Figure environment removed

We sweep the Raman drive time $T$ at laser sideband powers of $p=1012$ nW and extract a Rabi frequency of $\Omega/2 \pi$ = \SI{450\pm47}{\kilo\hertz}(Fig. \ref{subfig:RabiAllOptical}) and a $\pi$-gate fidelity of $F_\pi = $ \SI{83\pm2}{\percent}, similar to Ref. \cite{Debroux2021}. 

% Figure environment removed

Ramsey measurements (Fig. \ref{subfig:RamseyAllOptical}) yield an inhomogeneous dephasing time of $T_{2*} = $ \SI{1.13\pm0.07}{\micro\second} and a Hahn-Echo measurements (Fig. \ref{subfig:HahnEchoAllOptical}) a dephasing time of $T_{2} = $ \SI{35.5\pm3.0}{\micro\second}.

% Figure environment removed

% Figure environment removed

We additionally measured the spin decay time $T_1$ at the perpendicular field orientation and found much shorter times on the order of 100 µs (Fig. \ref{subfig:SpinT1}).

% Figure environment removed

\FloatBarrier

\subsection{Spin $T_1$ analysis}
Phonon-induced depolarization of group IV centers' spin is the dominant source of decoherence. Therefore, improving the spin decay time $T_1$ is the central task to enhance the coherence of the system. As discussed in Ref. \cite{meesala2018strain}, the spin decay time $T_1$ in the group IV has two meanings, the orbital relaxation time $T_1^{\text{orbit}}$ between the energy levels in different orbital branches but with same spin projection, and the $T_1^{\text{spin}}$ between the qubit levels with frequency $\omega_s$. Ultimately, the $T_1^{\text{spin}}$ is the factor that directly relates to the coherence of the SnV, so we limit our $T_1$ discussion to $T_1^{\text{spin}}$ only. 

There are three phonon-induced $T_1^{\text{spin}}$ decay path \cite{meesala2018strain}, including direct single phonon, resonant two phonon (Orbach process) \cite{Orbach1961}, and off-resonant two phonon (Raman process). Similar to SiV, the SnV shows much slower single phonon and Raman spin decay, so we focus on the Orbach process and study its dependence with the $B$ field orientation ($\theta$) and the strain magnitude. Adapted from Ref. \cite{Orbach1961}, we write the decay rate $\gamma_{spin}^{2}$ as follows:
\begin{equation}
\gamma_{spin}^{2} \propto \frac{ \Delta_{gs}^3}{\exp{(h\Delta_{gs}/{k_BT})}-1} \frac{\left|\sum \bra{e_1\downarrow}H_\epsilon^{AC}\ket{e_j}\bra{e_j}H_\epsilon^{AC}\ket{e_2\uparrow}\right|^2}{ \sum\left|\bra{e_i}H_\epsilon^{ac}\ket{e_j}\right|^2}
\end{equation}
where $i$ represent the states of the lower orbital branch ($\ket{e_1\downarrow}$, $\ket{e_2\uparrow}$) and $j$ represent the states of the upper orbital branch ($\ket{e_3\downarrow}$, $\ket{e_4\uparrow}$). The $H_\epsilon^{ac}$ denotes an AC strain field which correlates to the phonon interaction in the crystal. We used balanced magnitude for the $H_{\epsilon_{E_x}}^{AC}$ and $H_{\epsilon_{E_y}}^{AC}$ by setting the $H_\epsilon^{ac}$ as follows:
\begin{align}
    H_\epsilon^{ac} = e\left[\begin{matrix}
    -1 & 0 & 1 & 0\\
    0 & -1 & 0 & 1\\
    1 & 0 & 1 & 0\\
    0 & 1 & 0 & 1
    \end{matrix}\right],
\end{align}

The relative decay rate at temperature \SI{4}{\kelvin} with the maximum normalized to 1 is shown in Figure \ref{subfig:spin T1}, with MW-based (small $\theta$) and all-optical (large $\theta$) operation points for spin control of the SnV highlighted in black (white) stars. We  observe a ratio of \SIrange{500}{1200}{} between the two $T_1^{\text{spin}}$, which is roughly inline with our experimental values measured at \SI{6}{\kelvin} (MW-based control $T_1^{\text{spin}}=$\SI{2.5}{\milli\second}, all-optical control $T_1^{\text{spin}}=$\SI{1.65}{\micro\second}). We note that as a pre-factor, the temperature would not change the decay rate ratio between the two configurations. This ratio reiterates the fact that the Orbach process is the dominant factor for the $T_1^{\text{spin}}$ decay.

% Figure environment removed

\section{MW control of the SnV spin}
\subsection{Randomized benchmarking}
The gates are chosen from the Clifford group and are \{$I$, $\pi_x$, $\pi_y$, $\pi_x/2$, $-\pi_x/2$, $\pi_y/2$, $-\pi_y/2$\}. We randomly choose ($N-1$)-gates and use the last gate to undo the sequence, followed by a z-basis measurement. The last gate is part of the Clifford group. 
We get the $\pi$-gate from Rabi measurements and adjust the time $t_\pi$ accordingly. The identity is implemented as wait-time for $t_\pi$, whereas $\pi/2$-gates have a duration of $t_{\pi/2}$. No buffer times are used which would make the qubit prone to dephasing errors, but the drive amplitude is reduced such that local heating effects is not a limiting factor. All randomized benchmarking measurements were taken at a Rabi frequency of $\Omega/(2\pi)=$ \SI{2.8}{\mega\hertz}. A total of 10 randomized sequences were applied each time to average out over different implementations. The fidelity $F$ is extracted by fitting the readout with $A*F^N+B$, from which we get the error per Clifford gate \cite{knill2008randomized}.

\subsection{Ramsey $T_{2*}$ at different qubit frequency}
% Figure environment removed
We change the applied magnetic field magnitude from 81.5 to 117 mT and measure the Ramsey dephasing time $T_2^*$. We extract $T_2^*$ = \SI{2.63\pm0.14}{\micro\second} at the qubit frequency of 3.694 GHz (see Fig. \ref{subfig:Ramsey117mT}), indicating that $g$-factor fluctuations as reported in Ref.\cite{Sukachev2017} are not limiting the observed $T_2^*$.

\subsection{Ramsey measurements with phase-readout}
% Figure environment removed

Fig. \ref{subfig:Ramsey} shows the measurement from which we extract $T_{2*}$ Ramsey and the modulation of the qubit frequency in Fig. 4 (a) in the main text. We fit for every time delay a modulation of the phase by $A*\cos{\phi}+B$, where $A$ is the visibility shown in the main text and $B$ is the mean value for all time delays averaged over all phases. We then extract the inhomogenous dephasing time $T_{2*}$ by fitting $A$ over delay time $\tau$ with an Gaussian envelope $\propto \exp{((-\tau/T_{2*})^2)}$. With this technique, we can distinguish with certainty a real modulation of the qubit frequency (loss in coherence and visibility)  versus a detuning error (no total loss of visibility, but no readout at certain delays and phases). The origin of the beating pattern needs further investigation. The MW-electronics were tested for any modulation. Likely candidates could be surface spins or substitutional nitrogen centres (P1-centres) with a large gyromagnetic ratio close to the one of a free electron ($g = 2$)  resulting in relatively large couplings even at large distances. 

\subsection{XY-sequences}
% Figure environment removed

Fig. \ref{subfig:XY}  shows the coherence over total time delay $\tau$ in the XY-family of dynamical-decoupling sequences. We observe an almost identical scaling of coherence time over the number of $\pi$-pulses (see main text Fig. 4 c). Because of the much smaller sensitivity of XY- pulse sequences with regards to rotation- or offset errors compared to CPMG sequences, we conclude that our control does not limit the coherence \cite{suterDD}.

\section{Experimental details}
\subsection{Device fabrication}
The diamond membrane substrate was generated via \ch{He+} implantation and annealing. \SI{450}{\nano\meter}-thick diamond overgrowth layer was introduced in a diamond chemical vapor deposition (CVD) chamber, followed by a \ch{^{120}Sn} implantation with \SI{2e8}{\per\centi\meter\squared} dose and \SI{40}{\nano\meter} target depth. Individual membranes were patterned via lithography and electrochemically etched to undercut the graphitized layer. The target membrane was then transferred to a HSQ-coated fused silica substrate with a \SI{5}{\micro\meter}-deep etched trench to the generate suspended area. The substrate was annealed at \SI{600}{\celsius} for \SI{8}{\hour} in argon atmosphere. Membrane was thinned down to \SI{160}{\nano\meter} via ICP RIE etching using \ch{Ar/Cl2}, \ch{O2/Cl2}, and \ch{O2} recipes. The microwave coplanar waveguide was lithographically defined, followed by \ch{Ti} and \ch{Au} deposition with thicknesses of \SI{10}{\nano\meter} and \SI{200}{\nano\meter}, respectively. Excess resist was lift-off using \SI{80}{\celsius} NMP solutions.  

\subsection{Measurement setup}
All the measurement data in this work were taken in Cambridge, UK. 
The device was studied in a closed-cycle cryostat (attoDRY 2100) with a base temperature of \SI{1.7}{\kelvin} at the device and in which the temperature can be tuned with a resistive heater located under the sample mount. Superconducting coils around the sample space allow the application of a vertical magnetic field from 0 to \SI{9}{\tesla} and a horizontal magnetic field from 0 to \SI{1}{\tesla}. Unless explicitly stated otherwise, all measurements were conducted at $T=$ \SI{1.7}{\kelvin}. 
The optical part of the set-up consists of a confocal microscope mounted on top of the cryostat and a microscope objective with numerical aperture 0.82 inside the cryostat. The device is moved with respect to the objective utilizing piezoelectric stages (ANPx101/LT and ANPz101/LT) on top of which the device is mounted. Resonant excitation around \SI{619}{\nano\meter} is performed by a second harmonic generation stage (ADVR RSH-T0619-P13FSAL0) consisting of a frequency doubler crystal pumped by a \SI{1238}{\nano\meter} diode laser (Sacher Lasertechnik Lynx TEC 150). The frequency is continuously stabilized through feedback from a wavemeter (High Finesse WS/7). The charge environment of the SnV- is reset with microsecond pulses at \SI{532}{\nano\meter} (Roithner LaserTechnik CW532-100). PL measurements were done with a Teledyne Princeton Instruments PyLoN:400BR eXcelon CCD and SpectraPro HRS-750-SS Spectrograph. Optical pulses are generated with an acousto-optic modulator (Gooch and Housego 3080-15 in the \SI{532}{\nano\meter} path and AA Opto Electronics MT350-A0,2-VIS) controlled by a delay generator (Stanford Research Instruments DG645). For resonant excitation measurements, a long-pass filter at \SI{630}{\nano\meter} (Semrock BLP01-633R-25) is used to separate the fluorescence from the phonon-sideband from the laser light. The fluorescence is then sent to a single photon counting module (PerkinElmer SPCM-AQRH-16-FC), which generates TTL pulses sent to a time-to-digital converter (Swabian Timetagger20) triggered by an arbitrary waveform generator (Tektronix AWG70002A). Photon counts during ``initialize" and ``readout" pulses are histogrammed in the time-tagger to measure the spin-population. Sidebands driving both resonantly transitions as well as off-resonant all-optical control were generated by an amplitude electro-optic modulator (Jenoptik AM635), and the amplitude, phase, and frequency of the sidebands are controlled by a 25 Gs$/$sec arbitrary waveform generator (Tektronix AWG70002A). The EOM is locked to its interferometric minimum with a lock-in amplifier and PID (Red Pitaya, STEMlab 125-14) and using a freely available Lock-in+PID application \cite{Luda2019} with a feedback loop on the signal generated by a photodetector (Thorlabs PDA100A2). 

Microwave pulses are generated with the second channel of the arbitrary waveform generator and amplified with a low-noise amplifier (Minicircuits ZX60-83LN12+) and a high-power amplifier (Minicircuits ZVE-3W-83+). Microwave signals inside the cryostat are delivered via the in-built pico-coax cables, self-soldered cables and a customised PCB. The signal is transmitted through a second line and terminated outside of the cryostat with 50 Ohms. 

% Figure environment removed

\begin{thebibliography}{10}

\bibitem{Hepp2014}
C.~Hepp, {\it et~al.\/}, {\it Physical Review Letters\/} {\bf 112} (2014).

\bibitem{thiering2018ab}
G.~Thiering, A.~Gali, {\it Physical Review X\/} {\bf 8}, 021063 (2018).

\bibitem{perdew1996generalized}
J.~P. Perdew, K.~Burke, M.~Ernzerhof, {\it Physical Review Letters\/} {\bf 77},
  3865 (1996).

\bibitem{sun2015strongly}
J.~Sun, A.~Ruzsinszky, J.~P. Perdew, {\it Physical Review Letters\/} {\bf 115},
  036402 (2015).

\bibitem{hahn1972thermal}
T.~Hahn, R.~Kirby, {\it AIP Conference Proceedings\/} (American Institute of
  Physics, 1972), vol.~3, pp. 13--24.

\bibitem{oikawa1999thermal}
N.~OIKAWA, A.~MAESONO, R.~TYE, {\it THERMAL CONDUCTIVITY\/} {\bf 24}, 405
  (1999).

\bibitem{reeber1996thermal}
R.~R. Reeber, K.~Wang, {\it Journal of Electronic Materials\/} {\bf 25}, 63
  (1996).

\bibitem{mcskimin1972elastic}
H.~McSkimin, P.~Andreatch~Jr, {\it Journal of Applied Physics\/} {\bf 43}, 2944
  (1972).

\bibitem{siew2000thermal}
Y.~Siew, {\it et~al.\/}, {\it Journal of The Electrochemical Society\/} {\bf
  147}, 335 (2000).

\bibitem{sheng2022green}
N.~Sheng, C.~Vorwerk, M.~Govoni, G.~Galli, {\it Journal of Chemical Theory and
  Computation\/} {\bf 18}, 3512 (2022).

\bibitem{meesala2018strain}
S.~Meesala, {\it et~al.\/}, {\it Physical Review B\/} {\bf 97}, 205444 (2018).

\bibitem{nv_microwave_wire}
N.~D. Lai, D.~Zheng, F.~Jelezko, F.~Treussart, J.-F. Roch, {\it Applied Physics
  Letters\/} {\bf 95} (2009).

\bibitem{lukin_heating}
C.~T. Nguyen, {\it et~al.\/}, {\it Phys. Rev. B\/} {\bf 100}, 165428 (2019).

\bibitem{trusheim2020transform}
M.~E. Trusheim, {\it et~al.\/}, {\it Physical Review Letters\/} {\bf 124},
  023602 (2020).

\bibitem{Debroux2021}
R.~Debroux, {\it et~al.\/}, {\it Physical Review X\/} {\bf 11} (2021).

\bibitem{arjona_indistinguishability}
J.~Arjona~Mart\'{\i}nez, {\it et~al.\/}, {\it Phys. Rev. Lett.\/} {\bf 129},
  173603 (2022).

\bibitem{gorlitz2020}
J.~G{\"o}rlitz, {\it et~al.\/}, {\it New Journal of Physics\/} {\bf 22}, 013048
  (2020).

\bibitem{fleischhauer2005electromagnetically}
M.~Fleischhauer, A.~Imamoglu, J.~P. Marangos, {\it Reviews of modern physics\/}
  {\bf 77}, 633 (2005).

\bibitem{Orbach1961}
R.~Orbach, {\it Proceedings of the Royal Society of London. Series A,
  Mathematical and Physical Sciences\/} {\bf 264}, 458 (1961).

\bibitem{knill2008randomized}
E.~Knill, {\it et~al.\/}, {\it Physical Review A\/} {\bf 77}, 012307 (2008).

\bibitem{Sukachev2017}
D.~D. Sukachev, {\it et~al.\/}, {\it Physical Review Letters\/} {\bf 119}
  (2017).

\bibitem{suterDD}
A.~M. Souza, G.~A. {\'A}lvarez, D.~Suter, {\it Philosophical Transactions of
  the Royal Society A: Mathematical, Physical and Engineering Sciences\/} {\bf
  370}, 4748 (2012).

\bibitem{Luda2019}
M.~A. Luda, M.~Drechsler, C.~T. Schmiegelow, J.~Codnia, {\it Review of
  Scientific Instruments\/} {\bf 90}, 023106 (2019).

\end{thebibliography}

\bibliographystyle{Science}
\end{document}
