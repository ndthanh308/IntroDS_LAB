\documentclass[twocolumn,hyperpdf,amsmath,amssymb,aps,prl,10pt,superscriptaddress,nofootinbib,noeprint,preprintnumbers,floatfix]{revtex4-2}

%% \usepackage[utf8]{inputenc} % allow utf-8 input
%\usepackage[T1]{fontenc}    % use 8-bit T1 fonts
%\usepackage{hyperref}       % hyperlinks
\usepackage{url}            % simple URL typesetting
\usepackage{booktabs}       % professional-quality tables
\usepackage{multirow}    
\usepackage{amsfonts}       % blackboard math symbols
\usepackage{nicefrac}       % compact symbols for 1/2, etc.
\usepackage{microtype}      % microtypogrhy
% \usepackage{natbib}
\usepackage{enumerate}
%\usepackage{enumitem}
\usepackage{hhline}
\usepackage{makecell}
\usepackage{pifont}

% use Times
%\usepackage{times}
% For figures
\usepackage{graphicx} % more modern
%\usepackage{epsfig} % less modern
%\usepackage{subfigure}
\usepackage{caption}
\usepackage{subcaption}
% For citations
\usepackage{amsmath}
\usepackage{amsthm}
\usepackage{amssymb}
\usepackage{tikz}
\usepackage{xcolor}
\usetikzlibrary{arrows}

\allowdisplaybreaks

%for fonts
\usepackage{mathrsfs}

% For algorithms
\usepackage{algorithm}
\usepackage{algorithmic}
% \usepackage{algpseudocode}
% \usepackage[noend]{algpseudocode}
\usepackage{hyperref}
\usepackage{bm}
%\usepackage{todonotes}

%For theorems
\allowdisplaybreaks

%for convinience
\newcommand{\RR}{\mathbb{R}}
\newcommand{\vct}{\boldsymbol }
%\newcommand{\mat}{\mathbf}
\newcommand{\rnd}{\mathsf}
\newcommand{\ud}{\mathrm d}
\newcommand{\nml}{\mathcal{N}}
\newcommand{\loss}{\mathcal{L}}
\newcommand{\hinge}{\mathcal{R}}
\newcommand{\kl}{\mathrm{KL}}
\newcommand{\cov}{\mathrm{cov}}
\newcommand{\dir}{\mathrm{Dir}}
\newcommand{\mult}{\mathrm{Mult}}
\newcommand{\err}{\mathrm{err}}
\newcommand{\sgn}{\mathrm{sgn}}
%\renewcommand{\span}{\mathrm{span}}
% \newcommand{\argmin}{\mathrm{argmin}}
% \newcommand{\argmax}{\mathrm{argmax}}
\newcommand{\poly}{\mathrm{poly}}
% \newcommand{\rank}{\mathrm{rank}}
% \newcommand{\conv}{\mathrm{conv}}
%\newcommand{\E}{\mathbb{E}}
% \newcommand{\diag}{\mat{diag}}
\newcommand{\acc}{\mathrm{acc}}

\newcommand{\labs}{\left\vert}
\newcommand{\rabs}{\right\vert}
\newcommand{\lnorm}{\left\Vert}
\newcommand{\rnorm}{\right\Vert}

\newcommand{\aff}{\mathrm{aff}}
% \newcommand{\range}{\mathrm{Range}}
\newcommand{\Sgn}{\mathrm{sign}}

\newcommand{\hit}{\mathrm{hit}}
\newcommand{\cross}{\mathrm{cross}}
\newcommand{\Left}{\mathrm{left}}
\newcommand{\Right}{\mathrm{right}}
\newcommand{\Mid}{\mathrm{mid}}
\newcommand{\bern}{\mathrm{Bernoulli}}
\newcommand{\ols}{\mathrm{ols}}
\newcommand{\tr}{\operatorname{tr}}
\newcommand{\opt}{\mathrm{opt}}
%\newcommand{\ridge}{\mathrm{ridge}}
\newcommand{\unif}{\mathrm{Unif}}
\newcommand{\Image}{\mathrm{im}}
\newcommand{\Kernel}{\mathrm{ker}}
\newcommand{\supp}{\mathrm{supp}}
\newcommand{\pred}{\mathrm{pred}}
\newcommand{\distequal}{\stackrel{\mathbf{P}}{=}}
%\newcommand{\gege}{\textcircled{1}}
\newcommand{\gege}{{A(\vect{w},\vect{w}_*)}}
\newcommand{\gele}{{A(\vect{w},-\vect{w}_*)}}
\newcommand{\lele}{{A(-\vect{w},-\vect{w}_*)}}
\newcommand{\lege}{{A(-\vect{w},\vect{w}_*)}}
\newcommand{\firstlayer}{\mathbf{W}}
\newcommand{\firstlayerWN}{v}
\newcommand{\secondlayer}{a}
\newcommand{\inputvar}{\vect{x}}
\newcommand{\anglemat}{\mathbf{\Phi}}
\newcommand{\holder}{H\"{o}lder }
\newcommand{\real}{\mathbb{R}}
\newcommand{\approxerr}{\delta}

\def\R{\mathbb{R}}
\def\Z{\mathbb{Z}}
\def\cA{\mathcal{A}}
\def\cB{\mathcal{B}}
\def\cD{\mathcal{D}}
\def\cE{\mathcal{E}}
\def\cF{\mathcal{F}}
\def\cG{\mathcal{G}}
\def\cH{\mathcal{H}}
\def\cS{\mathcal{S}}
\def\cI{\mathcal{I}}
\def\cL{\mathcal{L}}
\def\cM{\mathcal{M}}
\def\cN{\mathcal{N}}
\def\cP{\mathcal{P}}
\def\cS{\mathcal{S}}
\def\cT{\mathcal{T}}
\def\cV{\mathcal{V}}
\def\cW{\mathcal{W}}
\def\cZ{\mathcal{Z}}
\def\SS{\mathbb{S}}
\def\NN{\mathbb{N}}
\def\bP{\mathbf{P}}
\def\TV{\mathrm{TV}}
\def\MSE{\mathrm{MSE}}

\def\vw{\mathbf{w}}
\def\va{\mathbf{a}}
\def\vZ{\mathbf{Z}}

\newcommand{\mat}[1]{#1}
\newcommand{\vect}[1]{#1}
\newcommand{\norm}[1]{\left\|#1\right\|}
\newcommand{\normop}[1]{\left\|#1\right\|_{\mathrm{op}}}
\newcommand{\simplex}{\triangle}
\newcommand{\abs}[1]{\left|#1\right|}
\newcommand{\expect}{\mathbb{E}}
\newcommand{\prob}{\mathbb{P}}
\newcommand{\proj}{\gP}
% \newcommand{\prox}[2]{\textbf{Prox}_{#1}\left\{#2\right\}}
\newcommand{\event}[1]{\mathscr{#1}}
\newcommand{\set}[1]{#1}
\newcommand{\diff}{\text{d}}
\newcommand{\difference}{\triangle}
\newcommand{\inputdist}{\mathcal{Z}}
\newcommand{\indict}{\mathbb{I}}
\newcommand{\rotmat}{\mathbf{R}}
\newcommand{\normalize}[1]{\overline{#1}}
\newcommand{\vectorize}[1]{\text{vec}\left(#1\right)}
\newcommand{\vclass}{\mathcal{G}}
\newcommand{\pclass}{\Pi}
\newcommand{\qclass}{\mathcal{Q}}
\newcommand{\rclass}{\mathcal{R}}
\newcommand{\classComplexity}[2]{N_{class}(#1,#2)}
\newcommand{\cclass}{\mathcal{F}}
\newcommand{\gclass}{\mathcal{G}}
\newcommand{\pthres}{p_{thres}}
\newcommand{\ethres}{\epsilon_{thres}}
\newcommand{\eclass}{\epsilon_{class}}
\newcommand{\states}{\mathcal{S}}
\newcommand{\trans}{P}
\newcommand{\lowprobstate}{\psi}
\newcommand{\actions}{\mathcal{A}}
\newcommand{\contexts}{\mathcal{X}}
\newcommand{\edges}{\mathcal{E}}
\newcommand{\variance}{\text{Var}}
\newcommand{\params}{\vect{w}}

\newcommand{\relu}[1]{\sigma\left(#1\right)}
\newcommand{\reluder}[1]{\sigma'\left(#1\right)}
\newcommand{\act}[1]{\sigma\left(#1\right)}

\newtheorem{thm}{Theorem}
% \newtheorem{thm}{Theorem}
\newtheorem{lem}{Lemma}
% Thm -> corollary 
\newtheorem{cor}{Corollary}
\newtheorem{prop}{Proposition}
\newtheorem{asmp}{Assumption}
\newtheorem{defn}{Definition}
\newtheorem{oracle}{Oracle}
\newtheorem{fact}{Fact}
\newtheorem{conj}{Conjecture}
\newtheorem{rem}{Remark}
\newtheorem{example}{Example}
\newtheorem{condition}{Condition}
\newtheorem{exercise}{Exercise}
\newtheorem{mess}{Message}
\newtheorem{claim}{Claim}
\newtheorem{ec}{Empirical Conclusion}






\usepackage[capitalize,noabbrev]{cleveref}
% \usepackage{cleveref}
\crefname{thm}{Theorem}{Theorems}
\crefname{lem}{Lemma}{Lemmas}
\crefname{cor}{Corollary}{Corollaries}
\crefname{prop}{Proposition}{Propositions}
\crefname{asmp}{Assumption}{Assumptions}
\crefname{defn}{Definition}{Definitions}
\crefname{oracle}{Oracle}{Oracles}
\crefname{fact}{Fact}{Facts}
\crefname{conj}{Conjecture}{Conjectures}
\crefname{rem}{Remark}{Remarks}
\crefname{claim}{Claim}{Claims}
\crefname{ec}{Empirical Observation}{Empirical Observations}


\renewcommand{\algorithmicrequire}{\textbf{Input:}}
\renewcommand{\algorithmicensure}{\textbf{Output:}}


\definecolor{red}{rgb}{1, 0, 0}
\newcommand{\RED}[1]{{\color{red} #1}}

\definecolor{green}{rgb}{0, 1, 0}
\definecolor{darkgreen}{rgb}{0.0, 0.2, 0.13}
\definecolor{darkseagreen}{rgb}{0.56, 0.74, 0.56}
\definecolor{officegreen}{rgb}{0.0, 0.5, 0.0}


\newcommand{\GREEN}[1]{{\color{green} #1}}

\definecolor{blue}{rgb}{0, 0, 1}
\newcommand{\BLUE}[1]{{\color{blue} #1}}

\definecolor{orange}{rgb}{1, 0.4, 0.0}
\newcommand{\ORANGE}[1]{{\color{orange} #1}}


\usepackage{graphicx, color}
\usepackage[dvipsnames]{xcolor}

%% text %%
\usepackage[letterspace=-10]{microtype} 

%% math, tables %%
\usepackage{bm, amsmath, amsfonts, amssymb,xfrac}
\usepackage{multirow, tabularx, dcolumn}
\usepackage{mathtools, leftidx, braket, slashed, cancel, bigdelim}
\usepackage{blkarray}
\usepackage[figures]{rotating}

\usepackage{tikz}
\usepackage{soul}
\usetikzlibrary[plotmarks]
\usepackage{anyfontsize}

%% referencing %%
\usepackage[utf8]{inputenc} 
\usepackage{hyperref}
\pdfstringdefDisableCommands{ \renewcommand{\bm}[1]{#1} }

%% colors %%
\definecolor{jlab_red}{RGB}{192,39,45}
\definecolor{jlab_orange}{RGB}{249,102,0}
\definecolor{jlab_blue}{RGB}{47,122,121}
\definecolor{jlab_green}{RGB}{65,125,10}
\definecolor{jlab_gray}{gray}{0.6}
\definecolor{magenta}{rgb}{0.5, 0, 0.5}

\newcommand\bef{% Figure environment removed}
\newcommand\beq{\begin{equation}}
\newcommand\eeq[1]{\label{#1}\end{equation}}
\newcommand\beqa{\begin{eqnarray}}
\newcommand\eeqa[1]{\label{#1}\end{eqnarray}}
\newcommand\bet{\begin{table}}
\newcommand\eet[1]{\label{tb:#1}\end{table}}
\newcommand\fgn[1]{Figure \ref{fg:#1}} 
\newcommand\eqn[1]{Eq.\ (\ref{#1})}
\newcommand\tbn[1]{Table \ref{tb:#1}} 

%% editing macros %%
\newcommand{\cm}{\ensuremath{\mathsf{cm}}}


%% pdf hypertext links
\hypersetup{%
pdftitle = {title},
pdfsubject = {},
pdfkeywords = {},
colorlinks = {true},
filecolor = {black},
linkcolor = {jlab_blue},
menucolor = {black},
citecolor = {jlab_green},
urlcolor = {jlab_green},
}{}

\begin{document}

%%%%%%%%%%%%%%%%%%%%%%%%%%%%%%%%%%%%%%%%%%%%%%%%%%%%%%%%%%%%%%%%%%%%%%
%
\title{Bound isoscalar axial-vector $bc\bar u\bar d$ tetraquark $T_{bc}$ from lattice QCD using two-meson and diquark-antidiquark variational basis}
%
\author{M. Padmanath}
\email{padmanath@imsc.res.in}
\affiliation{The Institute of Mathematical Sciences, a CI of Homi Bhabha National Institute, Chennai, 600113, India}
%
\author{Archana Radhakrishnan}
\email{archana.radhakrishnan@tifr.res.in}
\affiliation{Department of Theoretical Physics, Tata Institute of Fundamental Research, \\ Homi Bhabha Road, Mumbai 400005, India }

%
\author{Nilmani Mathur}
\email{nilmani@theory.tifr.res.in}
\affiliation{Department of Theoretical Physics, Tata Institute of Fundamental Research, \\ Homi Bhabha Road, Mumbai 400005, India }
%

\preprint{IMSc/23/05, TIFR/TH/23-14}

\date{\today}
%\begin{abstract}

The Fast Reciprocal Square Root Algorithm is a well-established approximation technique consisting of two stages: first, a coarse approximation is obtained by manipulating the bit pattern of the floating point argument using integer instructions, and second, the coarse result is refined through one or more steps, traditionally using Newtonian iteration but alternatively using improved expressions with carefully chosen numerical constants found by other authors. The algorithm was widely used before microprocessors carried built-in hardware support for computing reciprocal square roots. At the time of writing, however, there is in general no hardware acceleration for computing other fixed fractional powers. This paper generalises the algorithm to cater to all rational powers, and to support any polynomial degree(s) in the refinement step(s), and under the assumption of unlimited floating point precision provides a procedure which automatically constructs provably optimal constants in all of these cases. It is also shown that, under certain assumptions, the use of monic refinement polynomials yields results which are much better placed with respect to the cost/accuracy tradeoff than those obtained using general polynomials. Further extensions are also analysed, and several new best approximations are given.

\end{abstract}

\begin{abstract}
We report a lattice QCD study of the heavy-light meson-meson interactions with an explicitly exotic flavor content 
$bc\bar u\bar d$, isospin $I\!=\!0$, and axialvector $J^P=1^+$ quantum numbers in search of possible tetraquark bound states. 
The calculation is performed at four values of lattice spacing, ranging $\sim$0.058 to $\sim$0.12 fm, and at five 
different values of valence light quark mass $m_{u/d}$, corresponding to pseudoscalar meson mass $M_{ps}$ of about 
0.5, 0.6, 0.7, 1.0, and 3.0 GeV. The energy eigenvalues in the finite-volume are determined through a variational 
procedure applied to correlation matrices built out of two-meson interpolating operators as well as  
diquark-antidiquark operators. The continuum limit estimates for $D\bar B^*$ elastic $S$-wave scattering amplitude are
extracted from the lowest finite-volume eigenenergies, corresponding to the ground states, using amplitude 
parametrizations supplemented by a lattice spacing dependence. Light quark mass $m_{u/d}$ dependence of the 
$D\bar B^*$ scattering length ($a_0$) suggests that at the physical pion mass $a_0^{phys} = +0.57(^{+4}_{-5})(17)$ fm, 
which clearly points to an attractive interaction between the $D$ and $\bar B^*$ mesons that is strong enough to 
host a real bound state $T_{bc}$, with a binding energy of $-43(_{-7}^{+6})(_{-24}^{+14})$ MeV with respect to 
the $D\bar B^*$ threshold. We also find that the strength of the binding decreases with increasing $m_{u/d}$ and 
the system becomes unbound at a critical light quark mass $m^{*}_{u/d}$ corresponding to $M^{*}_{ps} = 2.73(21)(19)$ GeV. 
\end{abstract}

\maketitle
%%%%%%%%%%%%%%%%%%%%%%%%%%%%%%%%%%%%%%%%%%%%%%%%%%%%%%%%%%%%%%%%%%%%%%%%%%%%%%%%%

%% Figure environment removed

\section{Introduction}
Automatic 3D reconstruction of clothed humans using image inputs has gained increasing significance due to its potential applications in a wide array of AR/VR scenarios. High-fidelity reconstructions typically depend on sophisticated capture systems, which are developed with dense camera arrays~\cite{collet2015high,joo2015panoptic,joo2018total}, programmable light-stages~\cite{Vlasic2009, guo2019relightables}, and depth sensors~\cite{newcombe2011kinectfusion,DoubleFusion,BodyFusion,dou2016fusion4d,newcombe2015dynamicfusion}. However, stringent capture environments equipped with complex hardware pose significant challenges for consumer-level applications.


In this context, considerable research effort has been dedicated to developing methods that allow for more flexible capture configurations, such as utilizing a few RGB inputs. Among these works, learning implicit functions \cite{iccv2020PIFu, saito2020pifuhd, hong2021stereopifu} has proven effective in achieving highly detailed reconstructions by integrating the advancements of deep neural networks. These methods employ large multi-layer perceptrons (MLPs) to predict the occupancy probability or truncated signed distance function (TSDF) value of every queried 3D point based on its associated local feature, which is extracted from images. They can recover a continuous surface at arbitrary resolutions without topology restrictions.


However, in typical MLP-based implicit networks, the occupancy or TSDF value at each location is solved independently with planar image features, rendering them less capable of addressing challenging cases such as occlusions. Consequently, these methods suffer from generalization and robustness issues, particularly when tackling strong occlusions caused by large motion or multiple interacting humans. 
Some follow-up studies  \cite{zheng2021deepmulticap,zheng2021pamir,huang2020arch} utilize an extra geometric model, SMPL~\cite{Loper2015}, to improve robustness by introducing strong shape priors. 
Their success typically relies on the assumption of geometrical similarity \cite{huang2020arch} between the shape prior and target reconstruction, making them intractable for handling complex cases with loose clothes and sensitive to errors in SMPL model fitting.



%\ping{this paragraph sounds like `TSDF is better than MLP/SMPL, and we use TSDF to solve the problem'. But in Sec 3, we are telling a different story, saying `MLP needs a 3D convolutional encoder'. We need to make these two sections consistent.}\sicong{I think in this paragraph we claim that the TSDF}


%We opt for Trucated Signed Distance Funtion (TSDF) volumetric representations as they are naturally suitable for convolution operations, which have shown remarkable performance for learning hierarchical features on 2D visual perception tasks \cite{SunXLW19}. 
%Meanwhile, TSDF also describes the gradual geometry change around shape surface, which is not reflected by occupancy volume. 

We instead revisit the 3D volumetric representation and resort to 3D convolutional neural networks (CNNs) for feature learning, due to their impressive performance in feature learning and the ability to incorporate spatial context. However, volumetric methods and 3D convolution involve discretization, which might raise concerns regarding whether a discretized volume can preserve subtle geometric details as continuous representations learned in implicit functions. We investigate the relationship between volume resolution and quantization error on synthetic data by converting target mesh objects to TSDF volumes, as shown in Figure~\ref{fig:quantization_error}. We observe that the quantization errors are significantly reduced by increasing volume resolution and become nearly negligible when reaching a relatively high resolution (e.g., 512 or higher). In other words, achieving fine-detailed reconstruction is not supposed to be restricted by the use of volume representations as long as a proper volume resolution is utilized. Therefore, we present a method with high-resolution feature volumes, e.g., 256 and 512, while traditional volumetric methods \cite{varol18_bodynet,gilbert2018volumetric} are often limited to much lower resolutions, such as 32 or 128.



On the other hand, an increase in volume resolution may lead to a cubic growth of memory overhead \cite{8100085}. Reducing memory costs while guaranteeing the granularity of volumetric representations is necessary for pursuing high-quality reconstruction. Thus, we adopt a coarse-to-fine approach and cull away irrelevant voxels to build a sparse high-resolution feature volume. At the coarse level, the network computes an initial TSDF by applying a U-Net with sparse 3D CNN \cite{3DSemanticSegmentationWithSubmanifoldSparseConvNet} on the sparse feature volume, which is carved by a visual hull. Through our experiments, it turns out that more than 95\% of the volume grids are discarded by the visual hull culling, making the sparse 3D CNN efficient. At the fine level, the network focuses on a narrow band near the zero-level set of the initial TSDF and discretizes the narrow band with smaller voxels. By employing this narrow-band culling, we further shrink the sampling space, resulting in a relatively small range of grid numbers (usually 300K--500K in our experiments) even with a high volume resolution of 512. The remaining voxels in the narrow band are associated with features that fuse high-frequency information from the computed normal maps upon the low-frequency shape from the coarse level to compute the TSDF at high resolution. The final mesh is then extracted from the TSDF using the Marching-Cube algorithm ~\cite{Lorensen87marchingcubes}.
% Different from the u-net sturcture to preserve global topology context, we then apply a shallow 3dcnn to compute the final TSDF $D_{final}$ which contain more local geometry detail.




% \ping{this paragraph can be expanded. It is an important contribution and often ignored by other works. stress on the novel idea of regressing blending weights instead of colors}

In addition to geometry, high-quality mesh texture is also a crucial factor contributing to visual appearance. Directly computing a color field in 3D space, as in \cite{iccv2020PIFu}, struggles to capture high-frequency texture details, while the neural radiance field (NeRF) \cite{yu2020pixelnerf} or the DoubleField~\cite{shao2022doublefield} require expensive per-instance optimization and are often unstable for sparse input images. In contrast, we adopt an image-based rendering approach to compute a texture atlas map, which is efficient and widely supported in existing computer graphics tools. 
Specifically, we compute a blending weight at each 3D point on the mesh surface to determine its color as a weighted average of the colors at its image projections. The blending weights can be computed at a relatively coarse resolution, e.g., 512 volume resolution in our case, and leave texture details to the high-resolution images, such as 1K or 2K. Unlike previous methods that generate blurry texturing results under sparse input, our method generalizes well on both synthetic and real data with just a few input views. 
Figure~\ref{fig:teaser} shows two examples reconstructed by our method. Despite the challenging garment, pose, and occlusion, our method recovers faithful shape, normal, and texture on the right.

%with a wide variety of poses and clothing styles, and it is also adaptive to handle input image with arbitrary resolutions.
%\sicong{For this concern we claim that when the resolution of dicretized volume meets certain threshold (which is 256 in our experiment), the quantization error can be neglected.} 



In summary, the main contributions of this paper are as follows:
\begin{itemize}
\vspace{-0.1in}
  \item 
  We revisit the 3D volumetric representation and demonstrate that it can support clothed human reconstruction with equal or even better performance compared to implicit representation. 
  \item 
  We develop a memory and computation-efficient method for high-resolution volumetric reconstruction using sophisticated sparse 3D CNN, coarse-to-fine estimation, and voxel culling by visual hull and narrow bands. 
  \item 
  We introduce a novel method to compute a texture atlas map, which captures rich appearance details from high-resolution input images.
  \item 
  We achieve impressive results on standard benchmark datasets Twindom and MultiHuman, significantly reducing the point-2-surface (P2S) precision to approximately 0.2cm from just six input views, with more than $50\%$ error reduction compared to the state-of-the-art methods, including DoubleField~\cite{shao2022doublefield} and PIFuHD~\cite{saito2020pifuhd}.
\end{itemize}
The discovery of a doubly charmed tetraquark\footnote{We follow the nomenclature that a ``tetraquark"
refers to any bound state or resonance with dominant four-quark Fock component, whether it is 
a compact four-quark object or a two-meson molecule or a mixture of both.}, $T_{cc}$, marks an 
important milestone \cite{LHCb:2021vvq} in spectroscopy of hadrons. Phenomenologically, doubly heavy 
tetraquarks in the heavy quark limit are long hypothesized to form deeply bound states \cite{Ader:1981db,
Ballot:1983iv,Zouzou:1986qh,Heller:1986bt,Carlson:1987hh,Manohar:1992nd,Janc:2004qn,Ebert:2007rn,
Navarra:2007yw,Eichten:2017ffp,Karliner:2017qjm} with binding energy $\mathcal{O}$(100 MeV) with respect 
to the elastic strong decay threshold. While doubly bottom tetraquarks are suitable candidates for such 
deeply bound states, as predicted by multiple lattice QCD calculations \cite{Bicudo:2015kna,Francis:2016hui,
Bicudo:2017szl,Junnarkar:2018twb,Leskovec:2019ioa,Hudspith:2023loy}, $T_{cc}$ is found to be $360$ keV 
below the lowest two-meson threshold ($D^0D^{*+}$). A handful of recent experimental 
developments involving multiple heavy quark production such as the recent discoveries of $\Xi_{cc}$
\cite{LHCb:2017iph}, $T_{cc}$ \cite{LHCb:2021vvq}, reports of tri-$J/\psi$ \cite{CMS:2021qsn}, associated 
$J/\psi\Upsilon$ \cite{LHCb:2023qgu}, and di-$\Upsilon$ \cite{CMS:2016liw} productions, and recent 
proposals of inclusive search strategies \cite{Gershon:2018gda,Qin:2021zqx} augment promising prospects 
for the doubly heavy hadron sector in the near future. In light of these advancements, a doubly heavy 
tetraquark with a bottom and a charm quark with a valence quark configuration $T_{bc} \equiv bc\bar u\bar d$ 
is going to be one of the most sought-after hadron in this decade \cite{Polyakov:2023had}. In this work, 
using lattice QCD calculations, we show a clear evidence of an attractive interaction between the $D$ and 
$\bar B^*$ mesons that is strong enough to host a real bound state $T_{bc}$. This finding will further boost 
the search for such bottom-charm tetraquarks.

The phenomenological picture on deeply bound doubly heavy tetraquarks is based on a compact heavy
diquark-light antidiquark interpretation \cite{Francis:2016hui,Czarnecki:2017vco}, whereas the shallow 
binding energy of $T_{cc}$ could possibly be a reflection of its dominant noncompact molecular nature 
\cite{Janc:2004qn,Agaev:2022ast}. Bottom-charm tetraquarks form an intermediate platform, where there 
could be complicated interplay between these pictures. A collective and refined knowledge of the low 
energy spectra in all these three doubly heavy systems ($T_{bb}$, $T_{bc}$ and $T_{cc}$) could culminate 
in a deeper understanding of strong interaction dynamics across a wide quark mass regime spanning from 
charm to the bottom quarks. The isoscalar bottom-charm tetraquarks with quantum numbers [$I(J^P) = 0(1^+)$] 
have been investigated previously both using lattice \cite{Francis:2018jyb,Hudspith:2020tdf,Meinel:2022lzo} 
and nonlattice methodologies \cite{Heller:1986bt,Carlson:1987hh,Janc:2004qn,Ebert:2007rn,Chen:2013aba,
Sakai:2017avl,Eichten:2017ffp,Karliner:2017qjm,Czarnecki:2017vco,Carames:2018tpe,Park:2018wjk,Deng:2018kly,
Yang:2019itm,Agaev:2019kkz,Lu:2020rog,Tan:2020ldi,Braaten:2020nwp}. The predictions from nonlattice 
approaches are quite scattered from being unbound to deeply bound, whereas the difference in conclusions 
from the three existing lattice QCD investigations\cite{Francis:2018jyb,Hudspith:2020tdf,Meinel:2022lzo} 
call for more detailed efforts in this regard. 

In this work, we perform a lattice QCD simulation of coupled $D\bar B^*$ and $\bar BD^*$ two-meson 
channels\footnote{We work in the isosymmetric limit with no QED effects and $m_u=m_d$. Hence we choose 
to call the degenerate ($D^+B^-,~D^0\bar{B}^0$) threshold as $D\bar B$, and equivalently for others like 
$D\bar B^*$, $\bar BD^*$ and $D^*\bar B^*$.} that are the relevant lowest two strong decay thresholds, 
in the order of increasing energies, $E_{D\bar B^*} = M_{\bar B^*}+M_{D}$ and $E_{\bar BD^*}=M_{\bar B}+M_{D^*}$, 
where $M_{h}$ is the mass of the hadron $h$. The extracted finite-volume ground state energies are utilized 
to constrain the continuum extrapolated elastic $D\bar B^*$ scattering amplitudes following the L\"uscher's 
finite-volume prescription \cite{Luscher:1990ux,Briceno:2014oea}. The light quark mass $m_{u/d}$ dependence 
of the extracted amplitudes suggests a binding energy of $-43(^{+6}_{-7})(^{+14}_{-24})$ MeV for the 
$bc\bar u\bar d$ tetraquark pole with respect to $E_{D\bar B^*}$ at the physical point $m_{u/d}^{phys}$. 


%\section{Ensembles and fermion actions}\label{sec:lattice}

We use the same computational setup as in several of our previous publications \cite{Junnarkar:2019equ,
Junnarkar:2018twb,Basak:2014kma,Padmanath:2017lng,Basak:2012py,Basak:2013oya,Mathur:2016hsm,
Mathur:2018epb,Mathur:2018rwu,Junnarkar:2022yak,Mathur:2022ovu}, which we briefly summarize below for completeness. Four 
$N_f=2+1+1$ lattice QCD ensembles generated by the MILC collaboration are used in this study \cite{MILC:2012znn}, where
the dynamical quark flavors were simulated using Highly Improved Staggered Quark (HISQ) action on gauge fields 
that respect one-loop, tadpole-improved Symanzik gauge action with tuned coefficients through 
$\mathcal{O}(\alpha_sa^2, n_f\alpha_sa^2)$ \cite{Follana:2006rc}. The charm and strange quark masses are 
tuned to their respective physical values, whereas the dynamical light quarks are chosen such 
that $m_s/m_l\sim 5$. We list the relevant details of various lattice QCD ensembles used in \tbn{lattice}.

\bet[tbh]
  \begin{center}
	  \begin{tabular}{p{1.5cm}p{1.5cm}p{1.5cm}>{\hfill\arraybackslash}p{1.5cm}}
      \hline
Label & Symbol & $a~[fm]$     & $N_s^3\times N_t$ \\ \hline
$S_1$ & \pmb{\textcolor{red}{\tikz{\pgfsetplotmarksize{0.8ex}\pgfuseplotmark{diamond}}}} & 0.1207(11)   & $24^3\times64$ \\
$S_2$ & \pmb{\textcolor{magenta}{\tikz{\pgfsetplotmarksize{0.8ex}\pgfuseplotmark{pentagon}}}} & 0.0888(8)    & $32^3\times96$ \\
$S_3$ & \pmb{\textcolor{blue}{\tikz{\pgfsetplotmarksize{0.7ex}\pgfuseplotmark{o}}}} & 0.0582(4)    & $48^3\times144$ \\
$L_1$ & \pmb{\textcolor{OliveGreen}{\pgfsetplotmarksize{0.7ex}\tikz{\pgfuseplotmark{square}}}} & 0.1189(9)    & $40^3\times64$ \\   \hline
  \end{tabular}
  \end{center}
\caption{Relevant details of the lattice QCD ensembles used. The lattice spacing estimates 
are measured using the $r_1$ parameter \cite{MILC:2012znn}. $L$ in $L_1$ refers to large spatial volume, 
and $S$ in $S_1,~S_2$, and $S_3$ refer to small spatial volume. }
\eet{lattice}

The valence quark fields for the light, strange and charm flavors are realized using an overlap 
fermion action that is $\mathcal{O}(am)$ improved. To this end, we utilize the numerical 
implementation of the overlap action following Refs. \cite{Chen:2003im,xQCD:2010pnl}. Following 
the Fermilab prescription \cite{El-Khadra:1996wdx}, the bare charm quark mass on each ensemble was tuned 
using the kinetic mass of spin averaged $1S$ charmonia $\{a\overline M_{kin}^{\bar cc} = 0.75 aM_{kin}(J/\psi) + 0.25 aM_{kin}(\eta_c)\}$
determined for the respective ensembles. Further details on the tuning of charm quark mass, 
the tuned bare quark mass, and resulting discretization effects are discussed in Refs. \cite{Basak:2012py,Basak:2013oya}.
The bare strange quark mass is set by equating the lattice estimate for the fictitious pseudoscalar $\bar ss$ 
meson mass to 688.5 MeV \cite{Chakraborty:2014aca}. Additionally, we perform the quark propagator 
measurements in the valence sector using overlap fermion action for three other quark masses in 
all the ensembles corresponding to pseudoscalar masses of approximately 0.5, 0.6 and 1.0 GeV. 

We employ a nonrelativistic QCD (NRQCD) Hamiltonian \cite{Lepage:1992tx} for the bottom quark. 
We tuned the bottom quark mass using the Fermilab prescription \cite{El-Khadra:1996wdx}, by equating 
the lattice extracted kinetic mass of the spin averaged 1S bottomonia $\{\overline M_{kin}^{\bar bb} = 0.75 M_{kin}(\Upsilon) + 0.25 M_{kin}(\eta_b)\}$
to its experimental value, where the kinetic mass is evaluated from the dispersion relation 
$aM_{kin}^2 = ((ap)^2 - (a\Delta E)^2)/2a\Delta E$. The details of NRQCD Hamiltonian, the improvement 
coefficients, and bottom quark mass tuning on our setup are discussed in Ref. \cite{Mathur:2016hsm}.

\bef[tbh]
% Figure removed
\caption{A landscape plot of the pseudoscalar masses corresponding to the quark mass that we have utilized 
in this work for different lattice ensembles used. The horizontal gray bands indicate a representative 
$M_{ps}$ estimate to guide the eye for a similar pseudoscalar meson mass across all four ensembles.} 
\eef{mpiVslat}

In this work, we assume isospin symmetry ($m_u = m_d$), and then for the channel that study here, 
involves three quark masses: the bottom ($b$), the charm ($c$), and the light ($u/d$) quarks.
For the light quark mass, we investigate five 
different cases: three unphysical quark masses discussed above [referred in terms of the 
corresponding approximate pseudoscalar meson masses $M_{ps}\sim$0.5, 0.6, and 1.0 GeV], the 
strange quark mass [$M_{ps}\sim$0.7 GeV] and the charm quark mass [$M_{ps}\sim$3.0 GeV]. In 
\fgn{mpiVslat}, we present the landscape of the five light quark masses studied in terms of the 
corresponding $M_{ps}$ versus the ensembles used. Using this setup, we evaluate the finite-volume spectrum in the 
isoscalar axialvector channel with $bc\bar u\bar d$ flavor for all these five quark masses on all 
four ensembles, next investigate the scattering of $D$ and $B^*$ mesons in all five scenarios and then 
extract the $m_{u/d}$ (otherwise $M_{ps}$) dependence of the scattering parameters. We utilize a wall-smearing procedure for all 
our quark propagator measurements (see Refs. \cite{Mathur:2018epb,Junnarkar:2018twb,Mathur:2022ovu} 
for details), and our primary focus on the finite-volume spectrum is on the ground state in each case. 





\bet[tbh]
  \begin{center}
	  \begin{tabular}{p{1.5cm}p{1.5cm}p{1.5cm}>{\hfill\arraybackslash}p{1.5cm}>{\hfill\arraybackslash}p{1.0cm}}
      \hline
   Label & Symbol & $a~[fm]$     & $N_s^3\times N_t$  & $M_{ps}^{sea}$ \\ \hline
$S_1$ & \pmb{\textcolor{red}{\tikz{\pgfsetplotmarksize{0.8ex}\pgfuseplotmark{diamond}}}} & 0.1207(11)   & $24^3\times64$ & 305 \\
$S_2$ & \pmb{\textcolor{magenta}{\tikz{\pgfsetplotmarksize{0.8ex}\pgfuseplotmark{pentagon}}}} & 0.0888(8)    & $32^3\times96$ & 312\\
$S_3$ & \pmb{\textcolor{blue}{\tikz{\pgfsetplotmarksize{0.7ex}\pgfuseplotmark{o}}}} & 0.0582(4)    & $48^3\times144$ & 319\\
$L_1$ & \pmb{\textcolor{OliveGreen}{\pgfsetplotmarksize{0.7ex}\tikz{\pgfuseplotmark{square}}}} & 0.1189(9)    & $40^3\times64$ & 217 \\   \hline
  \end{tabular}
  \end{center}
\caption{Relevant details of the lattice QCD ensembles used. The lattice spacing estimates 
are measured using the $r_1$ parameter \cite{MILC:2012znn}. $L_1$ refers to large spatial volume, 
and $S_1,~S_2$, and $S_3$ refer to small spatial volumes.}
\eet{lattice}

{\it Lattice setup}: We use four lattice QCD ensembles (see \tbn{lattice} for relevant details) with $N_f=2+1+1$ 
dynamical Highly Improved Staggered Quark (HISQ) fields generated by the MILC collaboration 
\cite{MILC:2012znn}. The charm and strange quark masses in the sea are tuned to their respective 
physical values, whereas the dynamical light quark masses correspond to sea pion masses as listed 
in \tbn{lattice}. We utilize a partially quenched setup on these configurations with valence quark 
fields up to the charm quark masses realized using an overlap fermion action as in 
Refs. \cite{Chen:2003im,xQCD:2010pnl}. We employ a nonrelativistic QCD (NRQCD) Hamiltonian 
\cite{Lepage:1992tx} for the bottom quark. Following the Fermilab prescription \cite{El-Khadra:1996wdx}, 
the bare charm \cite{Basak:2012py,Basak:2013oya} and bottom \cite{Mathur:2016hsm} quark masses on each 
ensemble are tuned using the kinetic mass of spin averaged $1S$ quarkonia 
$\{a\overline M_{kin}^{\bar QQ} = {3\over 4} aM_{kin}(V) + {1\over 4} aM_{kin}(PS)\}$ determined on 
the respective ensembles. The bare strange quark mass is set by equating the lattice estimate for 
the fictitious pseudoscalar $\bar ss$ meson mass to 688.5 MeV \cite{Chakraborty:2014aca}. 

For the valence $m_{u/d}$, we investigate five different cases: three unphysical quark masses 
[corresponding to approximate pseudoscalar meson masses $M_{ps}\sim$0.5, 0.6, and 1.0 GeV], 
the strange quark mass [$M_{ps}\sim$0.7 GeV] and the charm quark mass [$M_{ps}\sim$3.0 GeV]. 
We evaluate the finite-volume spectrum for all these five quark masses on all four ensembles, 
investigate the scattering of $D$ and $\bar B^*$ mesons in all five cases and then extract the 
$m_{u/d}$ (otherwise $M_{ps}$) dependence of the scattering parameters. 




%
\section{Measurements and interpolators}\label{sec:2ptIO}

Lattice determination of finite-volume spectrum follows through an evaluation or measurement of Euclidean 
two-point correlation functions $\mathcal{C}_{ij}(t)$, of interpolating operators 
$\mathcal{O}_i(\mathbf{x},t)$ with desired quantum numbers, given by
\beq
\mathcal{C}_{ij}(t) = \sum_{\mathbf{x}}\left<\mathcal{O}_i(\mathbf{x},t)\mathcal{O}_j^{\dagger}(\mathbf{0},0)\right> = \sum_n \frac{Z_i^nZ_j^{n\dagger}}{2E^n} e^{-E^nt}.
\eeq{c2pt}
Here the second equality suggests that $\mathcal{C}_{ij}(t)$ can be expressed as a sum of exponentials 
following a spectral decomposition. $Z_i^n = \bra{0}\mathcal{O}_i\ket{n}$ is the operator-state overlap that 
quantifies the efficacy of the interpolator $\mathcal{O}_i$ in determining the time evolution of the state $n$. 
The utilization of wall smearing for the quark sources effectively kills all the high-momentum modes
at the source, whereas a zero momentum projection at the sink time slice ($\sum_{\mathbf{x}}$), as shown
in \eqn{c2pt}, efficiently projects the correlation function to the rest frame.  

Our main focus is on the ground state in the $T_{1}^+$ irreducible representation (irrep) in the rest frame, 
which is the only relevant rest frame finite-volume irrep for studying states in the infinite-volume continuum 
with quantum numbers ($J^P = 1^+$). To this end, we use a similar set of operators in the $T_{1}^+$ irrep as 
was utilized in Ref. \cite{Francis:2018jyb} and we briefly discuss them below for completeness. Assuming isospin symmetry, 
the relevant low-lying two-meson thresholds in the order of increasing energy are $E_{DB^*} = M_{B^*}+M_{D}$, 
$E_{BD^*}=M_B+M_{D^*}$, and $E_{D^*B^*}=M_B^*{}+M_{D^*}$. Hence, we consider the following low-lying two-meson 
interpolators 
\beqa
\mathcal{O}_1(x) &=& [\bar u(x) \gamma_i b(x)][\bar d(x) \gamma_5 c(x)]  \nonumber \\&& - [\bar d(x) \gamma_i b(x)][\bar u(x) \gamma_5 c(x)] \nonumber \\
\mathcal{O}_2(x) &=& [\bar u(x) \gamma_5 b(x)][\bar d(x) \gamma_i c(x)]  \nonumber \\&& - [\bar d(x) \gamma_5 b(x)][\bar u(x) \gamma_i c(x)] \nonumber \\
\mathcal{O}'(x) &=& \epsilon_{ijk} [\bar u(x) \gamma_i b(x)][\bar d(x) \gamma_j c(x)] \nonumber \\&& - [\bar d(x) \gamma_i b(x)][\bar u(x) \gamma_j c(x)].
\eeqa{mmops}
We utilize $\mathcal{O}_1(x)$ and $\mathcal{O}_2(x)$ in the computation of correlation functions. 
$\mathcal{O}'(x)$ has its associated two-meson threshold sufficiently higher up in the energy 
spectrum compared to the other two thresholds and it was found to have no effects in the low-lying 
energy spectrum. Hence we disregard this operator
from the rest of our analysis. Note that the lowest three particle threshold $DB\pi$ is above $E_{BD^*}$
for all the considered heavier-than-physical light quark masses. At $m_{u/d}^{phys}$, the $BD\pi$ threshold is 
immediately below $E_{BD^*}$, yet it remains sufficiently above $E_{DB^*}$ 
to have any significant effects on the ground states that we extract. We also compute two-point 
correlation functions for $B$, $B^*$, $D$, and $D^*$ mesons, using standard local quark 
bilinear interpolators ($\overline Q~\Gamma~q$) with spin structures $\Gamma\sim\gamma_5$ and 
$\gamma_i$ for pseudoscalar and vector quantum numbers, respectively. 

Phenomenologically, doubly bottom tetraquark in the axialvector channel is expected to be deeply 
bound. Such a state is expected to be quite compact owing to its doubly heavy flavor content 
and deeply bound nature \cite{Francis:2016hui,Czarnecki:2017vco}. Consequently, a local 
diquark-antidiquark interpolator is naturally interesting. Such an operator has already 
been utilized in all lattice QCD studies of the doubly bottom as well as bottom charm tetraquarks 
in the past \cite{Bicudo:2015kna,Francis:2016hui,Bicudo:2017szl,Junnarkar:2018twb,Leskovec:2019ioa,
Francis:2018jyb,Hudspith:2020tdf,Meinel:2022lzo,Hudspith:2023loy} and we follow the same strategy. 
Along with operators in \eqn{mmops}, we employ a local diquark-antidiquark interpolator 
\beq
\mathcal{O}_3(x) = (\bar u(x)^T \Gamma_5 \bar d(x) - \bar d(x)^T \Gamma_5 \bar u(x))( b(x) \Gamma_i c(x)),
\eeq{dadops}
where $\Gamma_k = C\gamma_k$ with $C=i\gamma_y\gamma_t$ being the charge conjugation matrix and 
the diquarks (antidiquarks) in the color antitriplet (triplet) representations. 

Our final basis is composed of the above-mentioned three interpolators $\{\mathcal{O}_1(x), \mathcal{O}_2(x), \mathcal{O}_3(x)\}$, 
which is diverse enough to reliably determine the ground state in the energy spectra that we are interested in.. Using this basis we determine 
the correlation matrices, with elements evaluated as prescribed in \eqn{c2pt}. Then the correlation matrices $\mathcal{C}$
are analyzed following a variational approach \cite{Michael:1985ne} to determine the energy estimates for 
low-lying levels in the spectrum. In this procedure, we look for the solutions of the generalized eigenvalue 
problem (GEVP) given by 
\beq
\mathcal{C}(t)v^n(t) = \lambda^n(t) \mathcal{C}(t_0)v^n(t),
\eeq{gevp}
where $t_0$ is a reference timeslice at which the eigenvalues $\lambda^n$s are identically unity. 
\bef[h]
% Figure removed
\caption{Effective energy plot for the eigenvalue correlation function $\lambda^0(t)$ (square) and for the product 
of single-meson correlators (circle) representing the noninteracting two-meson correlation function 
($\mathcal{C}_{D}(t)\mathcal{C}_{B^*}(t)$). The data correspond to $M_{ps} \sim 700$ MeV in the finest ensemble. 
The bands shown are the energy fit estimates for the final chosen time intervals.}
\eef{effmass}
The eigensolutions in the large time limit represent the lowest $N$ eigenstates $E^n$, for which the time 
evolution is dictated by the eigenvalues as $\lim_{t\to\infty}\lambda^n(t) \sim A_ne^{-E^nt}$. 
The corresponding eigenvectors are represented by $v^n(t)$, which are related to the operator-state-overlaps as
\beq
Z_i^{n}=\bra{0}\mathcal{O}_i \ket{n} = \sqrt{2E^n}(V^{-1})_i^n e^{E^{n}(t_0)/2},
\eeq{overlaps}
where $V$ is a matrix built out of $v^n(t)$. $v^n(t)$ is expected to be time independent in the limit, 
where the signal in $\mathcal{C}$ is saturated by the lowest $N$ eigenstates of the system.  

Conventionally the signal in the two point correlator data $C(t)$ is first assessed based on the 
large time plateauing in effective energies defined as $aE_{eff} = [ln(C(t)/C(t+\delta t))]/\delta t$. In 
\fgn{effmass}, we present the effective energies as a function of time for the eigenvalue correlation 
function (squares) and the noninteracting two-meson ($\mathcal{C}_{D}(t)\mathcal{C}_{B^*}(t)$) 
correlation function (circles). These effective energies can be seen to saturate around timeslices 
24 to 28 in the example shown. The results presented correspond to the lowest eigenvalue correlator 
$\lambda^0(t)$ at the strange quark mass ($M_{ps}\sim0.7$ GeV) in the finest ensemble we study. 
Evidently, there is a negative shift in the energies in $\lambda^0(t)$ with respect to the 
noninteracting energies at all times, except at very large times where the signal-to-noise ratio 
degrades substantially. 

Extraction of the energy spectra proceeds via fitting the eigenvalue correlators, $\lambda_{n}(t)$, 
with the expected asymptotic exponential behaviour. Alternatively, one can fit the asymptotic time 
estimates for the ratio of correlators given by 
\beq
R^n(t)=\frac{\lambda^n(t)}{\mathcal{C}_{m_1}(t) \mathcal{C}_{m_2}(t)}, 
\eeq{ratio}
to a single exponential form ($Ae^{-\Delta E^nt}$), where $\Delta E^n$ is expected to saturate to 
$E^n-M_{m_1}-M_{m_2}$ at large times. Here, $\mathcal{C}_{m_i}$ is the correlation function for 
the meson $m_i$, and $M_{m_i}$ is its mass. Being a ratio, $R^n(t)$ is empirically known to efficiently 
mitigate correlated noise between the product of two meson correlators and the interacting correlator 
for the two-meson system \cite{Green:2021qol}. Note that the automatic cancellation of the additive 
mass renormalization, inherent to NRQCD formulation, is an added advantage in using \eqn{ratio} for 
the fits. In \fgn{fitcompare}, we present a representative plot showing the $t_{min}$ dependence of 
the $\Delta E^n$ fit estimates determined from the fits to $\lambda^n(t)$ and $R^n(t)$, respectively, 
where $t_{min}$ is the lower boundary of the time interval used for these fits for a fixed upper boundary 
timeslice for the time interval. The energy differences are evaluated from $\lambda^n(t)$ using the relation 
$\Delta E^n = E^n-M_{m_1}-M_{m_2}$, where $M_{m_1}$ and $M_{m_2}$ are mass estimates for individual 
mesons determined from separate fits to $\mathcal{C}_{m_1}(t)$ and $\mathcal{C}_{m_2}(t)$, respectively. 
The estimates from different procedures can be seen to agree asymptotically in time, based on which 
optimal $t_{min}$ values are chosen. Our final results are based on fitting the ratio correlators 
defined in \eqn{ratio}.

    
\bef[h]
% Figure removed
\caption{$t_{min}$ dependence of the $\Delta E^0$ fit estimates determined from the fits to $\lambda^0$
and $R^0(t)$ for the case $M_{ps} \sim 700$ MeV in the finest ensemble. Here the superscript 0 refers 
to the ground state. }
\eef{fitcompare}



{\it Interpolators and measurements}: The finite-volume spectrum is determined from Euclidean two-point 
correlation functions $\mathcal{C}_{ij}(t)$, between interpolating operators $\mathcal{O}_{i,j}(\mathbf{x},t)$ 
with desired quantum numbers, given by
\beq
\mathcal{C}_{ij}(t) = \sum_{\mathbf{x}}\left<\mathbb{O}_j^{\dagger}(0)\mathcal{O}_i(\mathbf{x},t)\right> \approx \sum_n \mathbb Z_j^{n\dagger}Z_i^n e^{-E^nt}.
\eeq{c2pt}
Here $E^n$ is the energy of the $n^{th}$ state and $Z_i^n = \bra{0}\mathcal{O}_i\ket{n}$ is the operator-state 
overlap between the sink operator $\mathcal{O}_i$ and state $n$. We use $\mathbb{O}$ and $\mathbb Z$ to 
represent the source operator and overlaps to distinguish them from that for the sink as we follow 
a wall-source to point-sink construction in our $\mathcal{C}_{ij}$ evaluations. This is well-established 
procedure in ground state energy determination, despite the non-Hermitian setup in \eqn{c2pt} (see Refs.
\cite{Francis:2016hui,Francis:2018jyb,Mathur:2018epb,Junnarkar:2018twb,Hudspith:2020tdf,Mathur:2022ovu} for details).
We use the following set of linearly independent, yet Fierz related \cite{Padmanath:2015era}, operators, 
\beqa
\mathcal{O}_1(x) &=& [\bar u(x) \gamma_i b(x)][\bar d(x) \gamma_5 c(x)]  \nonumber \\&& - [\bar d(x) \gamma_i b(x)][\bar u(x) \gamma_5 c(x)] \nonumber\\
\mathcal{O}_2(x) &=& [\bar u(x) \gamma_5 b(x)][\bar d(x) \gamma_i c(x)]  \nonumber \\&& - [\bar d(x) \gamma_5 b(x)][\bar u(x) \gamma_i c(x)] \label{eq:mmops} \\
\mathcal{O}_3(x) &=& (\bar u(x)^T \Gamma_5 \bar d(x) - \bar d(x)^T \Gamma_5 \bar u(x))( b(x) \Gamma_i c(x)). \nonumber
\eeqa{mmops1}
$\mathcal{O}_1$ and $\mathcal{O}_2$ are two-meson operators of the type $D\bar B^*$ and $\bar BD^*$, 
respectively. $\mathcal{O}_3$ is a diquark-antidiquark type operator.
Here $\Gamma_k = C\gamma_k$ with $C=i\gamma_y\gamma_t$ being the charge conjugation matrix and 
the diquarks (antidiquarks) in the color antitriplet (triplet) representations. Other high lying 
two-meson ($D^*\bar B^*$) and three-meson ($D\bar B\pi$) interpolators are ignored in this analysis as 
they are sufficiently high in energy to have any effects on the extracted ground states. Bilocal 
two-meson interpolators with nonzero internal meson momenta are also not considered, which would 
be an important step ahead \cite{Wagner:2022bff}. We also compute two-point correlation functions 
for $\bar B$, $\bar B^*$, $D$, and $D^*$ mesons, using standard local quark bilinear interpolators 
($\bar q ~\Gamma~Q$) with spin structures $\Gamma\sim\gamma_5$ and $\gamma_i$ for pseudoscalar and 
vector quantum numbers, respectively. 

{\it Analysis}: The correlation matrices $\mathcal{C}$ evaluated for the basis in Eq. (\ref{eq:mmops}) 
are analyzed following a variational procedure \cite{Michael:1985ne} by solving the generalized eigenvalue 
problem (GEVP), $\mathcal{C}(t)v^n(t) = \lambda^n(t) \mathcal{C}(t_0)v^n(t)$. The eigenvalues in 
the large time limit represent the time evolution the low lying eigenenergies $\mathcal{E}^n$ as  
$\lim_{t\to\infty}\lambda^n(t) \sim A_ne^{-\mathcal{E}^nt}$. The corresponding eigenvectors $v^n(t)$ are 
related to the operator-state-overlaps $Z_i^n$. 

Eigenenergy extraction proceeds via fitting the eigenvalue correlators, $\lambda_{n}(t)$, 
or the ratios $R^n(t)=\lambda^n(t)/\mathcal{C}_{m_1}(t) \mathcal{C}_{m_2}(t)$, with the expected 
asymptotic exponential behaviour. Here, $\mathcal{C}_{m_i}$ is the two-point correlation function for 
the meson $m_i$. $R^n(t)$ is empirically known to efficiently mitigate correlated noise between 
the product of two single hadron correlators and the interacting correlator for the two-hadron 
system \cite{Green:2021qol}. Note that the automatic cancellation of the additive quark mass offset, 
inherent to NRQCD formulation, is an added advantage in using $R^n(t)$ for the fits. The systematics 
associated with the chosen time interval for fitting are assessed by varying the lower boundary of 
the time interval, $t_{min}$, with a fixed upper boundary, $t_{max}$, chosen considering the noise level. 
In \fgn{fitcompare}, we present a representative plot showing this $t_{min}$ dependence of the energy 
splittings ($\Delta \mathcal{E}^n$) determined from the fits to $\lambda^n(t)$ and  $R^n(t)$, respectively. 
The energy differences are evaluated from $\lambda^n(t)$ using the relation $\Delta \mathcal{E}^n = \mathcal{E}^n-M_{m_1}-M_{m_2}$, 
whereas the fits to $R^n(t)$ directly yield the respective estimates. We choose the  optimal $t_{min}$ 
values where the two different procedures found to agree asymptotically in time. We also perform 
additional checks considering an alternative quark smearing with different smearing widths to affirm 
our energy estimates, see Appendix A. Our final results are based on fitting the ratio correlators $R^n(t)$.
    
\bef[h]
% Figure removed
\caption{$t_{min}$ dependence of the $\Delta \mathcal{E}^0$ fit estimates determined from the fits to $\lambda^0$
and $R^0(t)$ for the case $M_{ps} \sim 700$ MeV in the finest ensemble. Here the superscript 0 refers 
to the lowest eigenenergy. }
\eef{fitcompare}



%%%%%%%%%%%%%%%%%%%%%%%%%%%%%%%%%%%%%%%%%%%%%%%%%%%%%%%%%%%%%%%
\section{Energy spectra in finite-volume}\label{fvresults}
%%%%%%%%%%%%%%%%%%%%%%%%%%%%%%%%%%%%%%%%%%%%%%%%%%%%%%%%%%%%%%
In this section, we present our results that we obtain from the finite-volume correlators. 
After presenting the energy spectrum extracted using variational techniques, we discuss 
the operator-state-overlaps and the operator basis dependence. In the final subsection, 
we describe our strategy for rebuilding the ground state energies that are corrected for 
the additive NRQCD offset and for using them in further amplitude fits. 

\subsection{Details of energy spectra}
% Figure environment removed
In \fgn{spectrum}, we present the finite-volume energy spectra of the isoscalar 
axialvector $bc{\bar{u}}{\bar{d}}$ channel that we extract on the four ensembles 
listed in \tbn{lattice}, at the five different $m_{u/d}$ values corresponding to 
$M_{ps}\sim$ 0.5, 0.6, 0.7, 1.0, and 3.0 GeV. The energy spectrum is shown in 
lattice units. Note that these levels are shown with unaccounted additive 
renormalization measures related to the NRQCD-based dynamics of the heavy bottom quarks. 
The noninteracting two-meson energy levels corresponding to $DB^*$ and $BD^*$ thresholds 
are indicated as dotted horizontal line segments for each lattice and each $M_{ps}$. 
A clear trend for negative energy shifts can be observed in all the cases, indicating 
a possible attractive interaction between the scattering particles involved \cite{scalarbc}.
The $B^*D^*$ threshold in each case is also shown in the figure by dashed lines. 

From the energy spectra in the lattices $L_1$, $S_2$ and $S_3$, it can be observed 
that a consistent pattern emerges with respect to the two-meson thresholds.
The relative positioning of the ground state energy with the elastic threshold in the 
$S_1$ ensemble is also consistent with the other three ensembles. This is an encouraging 
feature in the finite-volume spectrum, as our main interest is on reliable extraction of the 
ground state energies. It is this ground state energy from each ensemble that we 
later on employ to constrain the $DB^*$ scattering amplitude.  

The excited states in the $S_1$ ensemble for $M_{ps}$, other than at the charm point, 
indicate enhanced negative shifts compared to that on the other ensembles. This could be 
related to a combination of effects arising from various less attractive features of the 
$S_1$ lattice, which includes the coarsest lattice spacing, small spatial volume and 
possible insufficient statistics for the study at lighter $M_{ps}$. To this end, we 
perform two additional checks. First, we make an associated study of the $S_1$ and the 
$L_1$ ensembles at the level of variational analysis and fitting procedures to determine 
the low-lying spectra with emphasis on the ground and the first excited states. We discuss 
this part of the investigation in Appendix \ref{app:S1L1}. Secondly, we perform amplitude 
fits with and without results from the $S_1$ ensemble to verify the robustness in our estimates 
for the scattering length. We discuss this in detail in Section \ref{Ampfits}.

\subsection{Operator-state overlaps}\label{sec:OSO}
\bef[hbt!]
% Figure removed
\caption{Normalized operator-state overlaps $\tilde{Z}_i^n$ for a state indicated by $n={0, 1, 2}$ 
and an operator represented by $\mathcal{O}_i$, where $i={1, 2, 3}$ on the $L_1$ ensemble. 
The errors in the normalized overlap factors are smaller than the symbols and hence are 
suppressed. The five horizontal panes stand for the five different $M_{ps}$ values we 
investigate. The two vertical lines in each horizontal pane separate $\tilde{Z}_i^n$ for 
different operators $\mathcal{O}_i$. }
\eef{Zratiosl40}
Now we investigate the operator-state overlaps $Z_i^n$, as in \eqn{overlaps}, to evaluate 
the efficacy of the interpolators in determining the low-lying spectra. To this end, 
we define normalized operator-state overlaps $\tilde{Z}_i^n$ such that its largest value 
for any given operator $\mathcal{O}_i$ across all the states $\{n\}$ is unity \cite{Dudek:2009qf,Padmanath:2013zfa}.
$\tilde{Z}_i^n$ quantifies the relative relevance of any given operator across all the 
states. In \fgn{Zratiosl40}, we present $\tilde{Z}_i^n$ at all $M_{ps}$ values we have used
on the $L_1$ ensemble. Each square marker corresponds to the $\tilde{Z}_i^n$ for a given operator 
$\mathcal{O}_i$ on to a given state $n$. Each horizontal pane stands for an $M_{ps}$ indicated on the 
right-hand side, whereas the vertical lines in each horizontal pane part $\tilde{Z}_i^n$ 
for different operators indicated on the top pane. The $x$-axis ticks refer to the three low-lying 
states we have extracted. $\mathcal{O}_1$, the two-meson operator related to $DB^*$ threshold, 
can be seen to have the largest overlap with the ground state and has significantly small 
overlaps with the excited states. $\mathcal{O}_2$, the two-meson operator related to $BD^*$ 
threshold, has the largest overlap with the first excited state and a very small overlap with 
the ground state. $\mathcal{O}_2$ also have nonnegligible overlap factors with the second 
excited state indicating $BD^*$-type two-meson Fock component, which decreases with increasing 
$M_{ps}$. On the other hand, $\mathcal{O}_3$, the diquark-antidiquark type operator, have 
substantial overlap factors with all states at the two lightest $M_{ps}$ values, whereas with 
an increased $M_{ps}$ its largest overlap is with the second excited state. Note that 
$\mathcal{O}_3$ is Fierz related to two-meson interpolators \cite{Padmanath:2015era}, and 
the large $\tilde{Z}_3^n$ values of $\mathcal{O}_3$ for all $n$ could be related to this 
underlying connection between two-meson and diquark-antidiquark operators. 


A summary from the above observations on overlap factors is as follows. $\mathcal{O}_1$
predominantly determines the ground state, whereas it has a significantly small coupling with 
the excited states. Similar patterns of overlap factors are also observed for other ensembles, 
all of which indicate that $\mathcal{O}_1$ predominantly determines the ground state. 
The two excited states have strong two-meson and diquark-antidiquark Fock components 
in the two lightest $M_{ps}$ values. The two-meson Fock components in the second excited state 
and the diquark-antidiquark Fock components in the first excited state decreases with 
increasing $M_{ps}$. This is consistent with the phenomenological expectation, which suggests 
that the binding energy in doubly heavy tetraquarks increases with increasing 
relative heaviness for the heavy quarks with respect to the light quarks \cite{Francis:2016hui,
Czarnecki:2017vco,Junnarkar:2018twb}. A deeply bound state could be significantly compact 
and hence could have large Fock components of a compact object such as that of a 
diquark-antidiquark. In other words, the relevance of compact diquark-antiquark operators for 
the low-lying spectrum increases with decreasing light quark mass, as is evident from \fgn{Zratiosl40}. 

\subsection{Operator basis dependence}
\bef[tbh!]
% Figure removed
\caption{Operator basis dependence of the low energy spectra of the $L_1$ ensemble and 
$M_{ps}\sim$700 MeV for all possible operator basis that can be built out of the three operators 
discussed in Section \ref{sec:2ptIO}. The basis is presented in digital notation ($x$-axis tick labels) 
where the operators are arranged in the order $\{\mathcal{O}_1, \mathcal{O}_2, \mathcal{O}_3\}$.
The horizontal lines refer to the $DB^*$, $BD^*$, and $D^*B^*$ thresholds. The bands indicate the 
bounds of the ground and first excited state energy estimates from the full three-operator basis. }
\eef{basisdep}
Next, we look into the basis dependence of the finite-volume energy spectra presented in 
\fgn{spectrum}. In \fgn{basisdep}, we show this basis dependence as determined for $M_{ps}\sim$ 
700 MeV in the $L_1$ ensemble, for various operator basis build out of $\mathcal{O}_1$, 
$\mathcal{O}_2$, and $\mathcal{O}_3$ operators as defined in \eqn{mmops} and \eqn{dadops}. The digital 
indexing on the $x$-axis tick labels refers to various operator basis in the order 
$\{\mathcal{O}_1, \mathcal{O}_2, \mathcal{O}_3\}$, with an overline on the third index 
as a visual aid within the plot to highlight the diquark-antidiqaurk interpolator. 1 (0) 
indicates an operator is included in (excluded from) the basis. The horizontal 
lines refer to the $DB^*$, $BD^*$ and $B^*D^*$ thresholds. The gray horizontal bands 
refer to the two lowest levels in the full basis indicated by $11\overline{1}$. A level 
below the threshold appears only when $\mathcal{O}_1$ is present in the basis. The first 
excited state in the full basis $11\overline{1}$ is faithfully reproduced in those bases 
where $\mathcal{O}_2$ is included. $\mathcal{O}_3$ alone does not precisely determine 
any level in the energy spectrum using full basis. Similar observations are also made 
on other ensembles. 

In summary, the ground state in the full basis $11\overline{1}$ is reliably determined 
with $\mathcal{O}_1$ and is unaffected by the inclusion of $\mathcal{O}_2$ and 
$\mathcal{O}_3$ operators. The excited states have nonnegligible overlap factors with 
$\mathcal{O}_2$ and $\mathcal{O}_3$ operators. Given our setup with only a few energy 
levels, any assumption more complicated than a simple elastic $DB^*$ assumption for 
the amplitude fits is beyond the scope of this work. Such an assumption is justified within the 
isosymmetric limit as the lowest inelastic threshold ($BD^*$ at unphysically heavy 
$m_{u/d}$ or $BD\pi$ for $m_{u/d}^{phys}$) is significantly high. In light of all 
these observations above, we limit ourselves to using only ground states determined 
from all ensembles at various $M_{ps}$ values to constrain the elastic $S$-wave
$DB^*$ scattering amplitude.

\subsection{The ground state energies}

\begin{figure}[h]
% Figure removed
\caption{The ground state energies in units of the elastic threshold ($DB^*$) on all 
ensembles (see \tbn{lattice} for color-symbol conventions) for all $M_{ps}$ values
(different vertical panes).}
\eef{gsspectrum}

In this subsection, we discuss how we obtain ground state energies after adjusting the 
additive correction that is inherent to an NRQCD calculation. The set of numbers that 
we extract from our variational analysis and eigenvalue correlator fitting procedures 
are the single meson masses $M_{B}$, $M_{B^*}$, $M_{D}$, and $M_{D^*}$ and the energy 
splittings $\Delta E_n = E_n - M_{m_1} - M_{m_2}$ (see \eqn{ratio}). %for the interacting system from the reference 
%two meson level $m_1m_2$ determined from the ratio of correlators defined in \eqn{ratio}. 
First, we account for the NRQCD corrections in single meson masses (involving a bottom quark) as 
\beq
\tilde{M}_{B^{(*)}} = M_{B^{(*)}} - 0.5\overline M^{\bar bb}_{lat} + 0.5 \overline M^{\bar bb}_{phys},
\eeq{msnNRcor}
where $\overline M^{\bar bb}_{lat}(\overline M^{\bar bb}_{phys})$ refers to the spin averaged mass 
of the $1S$ bottomonium measured on the lattice (experiments). We follow this procedure as we 
have tuned the bottom quark mass through the spin average bottomonia at each ensemble.

For the interacting energy spectrum, the NRQCD offset is automatically canceled in the energy splittings
$\Delta E^n$. One can then build the energy estimates $\tilde{E}^n$ of interacting spectrum by adding 
the noninteracting level energy ($M_{m_1} + M_{m_2}$) with the energy splittings $\Delta E^n$ as,
\beq 
\tilde{E}^n = \Delta E^n + M_{m_1} + M_{m_2}.
\eeq{intNRcor}
If either $m_1$ or $m_2$ is a bottom meson, we use the corresponding corrected $\tilde{M}_{m_i}$\footnote{From 
the next section, for brevity we suppress the $\tilde{}$ notation indicating corrected masses and energies.} 
determined using \eqn{msnNRcor}, instead of $M_{m_i}$. 

In \fgn{gsspectrum}, we present the corrected ground state energy estimates, 
in units of the energy of elastic threshold $E_{DB^*}$, at various $M_{ps}$ and 
for all the ensembles we have employed. The spectrum clearly shows a trend of decreasing 
energy spitting, hence decreasing interaction strength, with increasing $M_{ps}$. 
Another feature worth noting here is that the lattice spacing dependence of the 
ground state energies on similar volume ensembles ($S_1$, $S_2$, $S_3$) for the 
non-charm $M_{\pi}$ are opposite to that at the charm point. We will revisit this 
point when we discuss extraction of $DB^*$ scattering amplitude using these energy levels. 


\begin{figure}[thb!]
% Figure removed 
\caption{The GEVP eigenenergies in finite-volume for isoscalar axialvector $bc\bar u\bar d$ 
channel on the $L_1$ ensemble. Five panels show the results obtained at various pseudoscalar 
meson masses ($M_{ps}$) 0.5, 0.6, 0.7, 1.0, and 3.0, respectively.}
\eef{spectrum}
{\it Finite-volume eigenenergies}: In \fgn{spectrum}, we present the finite-volume GEVP eigenenergies, 
in lattice units, for the isoscalar axialvector $bc{\bar{u}}{\bar{d}}$ channel. The results shown 
are for the $L_1$ ensemble at the five different $m_{u/d}$ values corresponding to $M_{ps}\sim$ 0.5, 
0.6, 0.7, 1.0, and 3.0 GeV. Note that these estimates include the additive offsets related to the 
NRQCD-based bottom quark dynamics. The non-interacting two-meson energy levels corresponding to 
$D\bar B^*$ and $\bar BD^*$ thresholds are indicated as dotted horizontal line segments and those 
related to $\bar B^*D^*$ threshold by dashed lines for each $M_{ps}$. The lowest eigenenergy or 
the ground state energy is dominated by the $Z_1^0$ factor corresponding to $\mathcal{O}_1$, that 
is related to the $D\bar B^*$ threshold and is determined unambiguously by the operator $\mathcal{O}_1$, 
see Ref.~\cite{Suppl} for details. The most important observation is a clear trend for negative energy 
shifts in the ground state energies, which can be observed in all the cases, indicating a possible 
attractive interaction between the $D$ and $\bar B^*$ mesons \cite{scalarbc}. A similar pattern of 
low lying eigenenergies and ground state negative energyshifts are also observed in other ensembles, 
see details in Ref.~\cite{Suppl}. We expect that for our choice of interpolating operators and 
the accessible values of $t$, $\mathcal{E}^0$ will be an accurate estimate of $E^0$, whereas our 
setup is unable to accurately estimate excited-state energies. This means the excited eigenenergies 
presented in \fgn{spectrum} may not correspond to the higher lying elastic excitations of the $D\bar B^*$ 
channel. The location of lowest two non-interacting finite-volume levels related to the $D\bar B^*$ 
channel along with the ground state eigenenergies are presented in Appendix B. Hence we focus only on 
the ground state energies ($\mathcal{E}^0\sim E^0$) for the rest of the analysis. 


\begin{figure}[h]
% Figure removed
\caption{The ground state energies in units of $E_{D\bar B^*}$ on all ensembles (see \tbn{lattice} 
for color-symbol conventions) for all $M_{ps}$ values (different vertical panels).}
\eef{gsspectrum}

In \fgn{gsspectrum}, we present the ground state energy estimates, in units of $E_{D\bar B^*}$, 
at various $M_{ps}$ and for all the ensembles. These estimates are evaluated as 
$E^n = \Delta E^n + M_{D} + \tilde{M}_{\bar{B}^*}$, where $\Delta E^n$ is the estimate from fit 
to $R^n(t)$, $\tilde{M}_{\bar B^{*}} = M_{\bar B^{*}}- 0.5\overline M^{\bar bb}_{lat} $ 
$+ 0.5 \overline M^{\bar bb}_{phys}$ accounts for the NRQCD additive offset, and 
$\overline M^{\bar bb}_{lat}(\overline M^{\bar bb}_{phys})$ refers to the spin averaged mass of 
the $1S$ bottomonium measured on the lattice (experiments). The eigenenergies clearly show 
a trend of decreasing energy spitting, hence decreasing interaction strength, with increasing 
$M_{ps}$. Another interesting feature to note here is the nonzero lattice spacing ($a$) 
dependence of the ground state energies on similar volume ensembles ($S_1$, $S_2$, $S_3$), 
which we account for through an $a$ dependence in the parametrized amplitude as discussed below. 

In \fgn{gsspectrum}, we also indicate the branch point location of the left hand cut (lhc) arising 
out of an off-shell pion exchange process for different $M_{ps}$ by horizontal dashed lines.
Recent developments point to the importance of lhc effects on virtual subthreshold poles related 
to the $T_{cc}$ tetraquark \cite{Du:2023hlu}. Such effects on bound states are the subject of future 
studies where one could successfully solve the relevant three particle integral equations. This is 
beyond the scope of this work and we ignore such effects in our analysis.



%
%%%%%%%%%%%%%%%%%%%%%%%%%%%%%%%%%%%%%%%%%%%%%%%%%%%%%%%%%%%%%%
\section{$\mathbf{DB^*}$ scattering amplitude}\label{Ampfits}
%%%%%%%%%%%%%%%%%%%%%%%%%%%%%%%%%%%%%%%%%%%%%%%%%%%%%%%%%%%%%%

\subsection{Strategy}

The finite-volume energy splittings determined in the previous section are related to 
the infinite-volume scattering physics via L\"uscher's finite-volume prescription 
\cite{Luscher:1990ux} and its generalizations, e.g. \cite{Briceno:2014oea}. Assuming 
these energy splittings are purely described by an elastic scattering in the $DB^*$ 
system, we utilize them to constrain the associated $S$-wave scattering amplitude. Here 
we consider only the ground states in all ensembles for all quark mass scenarios, as 
the excited states are found to be affected by the inelastic $BD^*$ channel. 

It is interesting that even the excited states are also found to have statistically 
significant shifts with respect to the inelastic $BD^*$ threshold (see \fgn{spectrum}), 
possibly indicating nontrivial interactions between $B$ and $D^*$ mesons. If the $DB^*$ 
and $BD^*$ channels were totally decoupled, such shifts point to equivalent interactions 
in both channels \cite{bdsbc}. However, independent elastic analysis for the excited 
states is not well justified. On the other hand, the inclusion of excited states in our 
analysis demands an inelastic treatment involving more parameters than the available 
degrees of freedom in the amplitude fits, which is beyond the scope of this work. We 
also assume only negligible effects from higher partial waves or any off-shell pion 
exchange interactions that can induce coupling between $DB^*$ and $BD^*$ channels 
\cite{Du:2023hlu}, for the same reason. 

\subsection{Amplitude fits and continuum extrapolations}
For the scattering of a $D$ and a $B^*$ meson in the $S$-wave leading to total angular 
momentum and parity $J^P=1^+$, the scattering phase shifts $\delta_{l=0}(k)$ are related to 
the finite-volume energy spectrum through \cite{Luscher:1990ux}:
\beq
kcot[\delta_0(k)] = \frac{2Z_{00}[1;(\frac{kL}{2\pi})^2)]}{L\sqrt{\pi}},
\eeq{luscher}
where $k$ is the momentum of either mesons in the center of momentum frame corresponding 
to the center of momentum energy $E_{cm}=\sqrt{s}$. $k$ and $E_{cm}$ are related to each other through 
\beq
4sk^2 = (s-(M_{D}+M_{B^*})^2)(s-(M_{D}-M_{B^*})^2).
\eeq{k2cm}
A sub-threshold pole singularity in the $S$-wave scattering amplitude $t = ({\mathrm{cot}}\delta_0 - i)^{-1}$ 
occurs when $k{\mathrm{cot}}\delta_0 = \pm\sqrt{-k^2}$ 
\bet[hb]
  \begin{center}
          \begin{tabular}{p{2.0cm}p{2.0cm}p{2.0cm}>{\hfill\arraybackslash}p{2.cm}}
      \hline
      \hline
$M_{ps}$ [GeV] & $\chi^2/d.o.f$ & $A^{[0]}/E_{DB^*}$ & $A^{[1]}/E_{DB^*}$ \\\hline
\multirow{2}{*}{0.5} & 2.1/2 & $-0.05(1)$ & $~0.17(_{-11}^{+13})$ \\\cline{2-4} 
                     & 1.3/1 & $-0.05(1)$ & $~0.13(_{-12}^{+13})$ \\ \hline
\multirow{2}{*}{0.6} & 0.5/2 & $-0.044(_{-8}^{+9})$ & $~0.10(_{-9}^{+9})$ \\ \cline{2-4} 
                     & 0.3/1 & $-0.043(_{-8}^{+9})$ & $~0.09(_{-10}^{+9})$ \\ \hline
\multirow{2}{*}{0.7} & 3.0/2 & $-0.042(_{-6}^{+8})$ & $~0.09(_{-7}^{+6})$ \\ \cline{2-4} 
                     & 1.5/1 & $-0.040(_{-6}^{+8})$ & $~0.06(_{-8}^{+6})$ \\ \hline
\multirow{2}{*}{1.0} & 2.9/2 & $-0.043(4)$ & $~0.11(_{-5}^{+5})$ \\ \cline{2-4} 
                     & 0.4/1 & $-0.041(4)$ & $~0.14(_{-4}^{+5})$ \\ \hline
\multirow{2}{*}{3.0} & 3.6/2 & $~0.006(_{-5}^{+6})$ & $-0.20(_{-5}^{+4})$ \\ \cline{2-4} 
                     & 1.9/1 & $~0.010(_{-5}^{+6})$ & $-0.25(_{-5}^{+4})$ \\ \hline
      \hline
  \end{tabular}
  \end{center}
\caption{Results from amplitude fits for different light quark mass scenarios indicated 
in terms of $M_{ps}$ in the first column. For each $M_{ps}$, two independent fits are 
performed with (top row) and without (bottom row) the level from $S_1$ ensemble. All fits 
are performed with the parameterization in \eqn{linparam}, where the optimized parameter 
values in the table are presented in units of the $DB^*$ threshold, $E_{DB^*}$. }
\eet{Ampfits1}
for scattering in $S$-wave. We follow the procedure outlined in Appendix B of Ref. 
\cite{Padmanath:2022cvl} in constraining the amplitude, such that the parametrization of 
$k{\mathrm{cot}}\delta_0$ is tuned to satisfy Eq. (\ref{luscher}). The parametrized 
$k{\mathrm{cot}}\delta_0$ is then investigated for poles of $t$ in the complex energy plane.  

\begin{figure}[h]
% Figure removed
\caption{$k{\mathrm{cot}}\delta_0$, in units of the elastic threshold $E_{DB^*}$, versus $a$ 
(lattice spacing) for all $M_{\pi}$ values. We follow the marker/color coding in \tbn{lattice} 
for the data points referring to the simulated data. The colored/gray bands indicate the fit 
results to the continuum extrapolation fit form in \eqn{linparam} with/without the data from 
$S_1$ ensemble.} 
\eef{alatdep}
Since we use only the ground states for amplitude fits, we limit ourselves to a scattering 
amplitude parametrization that is completely described by scattering length $a_0$ in an effective 
range expansion near the threshold. Additionally, we also consider a lattice spacing dependence 
on the parametrization of $k{\mathrm{cot}}\delta_0$. We find that a linear functional form
given by 
\beq
k{\mathrm{cot}}\delta_0 = A^{[0]} + aA^{[1]}
\eeq{linparam}
provides acceptable fits to the scattering amplitudes. Such an $a$ dependence was also found to 
be necessary in our previous investigations using NRQCD framework as well \cite{Mathur:2022ovu}, 
and is consistent with the leading $a$ dependence of observables
involving an NRQCD evolution. In this form, $A^{[0]}=-1/a_0$, where $a_0$ is the scattering 
length in the continuum limit. 
We list the results from different amplitude fits in \tbn{Ampfits1}. In \fgn{alatdep}, we present 
the quality of these fits by comparing the fit results with the data points. The colored/gray 
bands indicate the fit results including/excluding results from the $S_1$ ensemble to the respective 
fits. It can be clearly seen that fit results are less affected by inputs from $S_1$ ensemble, 
which is obvious given the large uncertainties associated with them, in contrast to inputs from 
$L_1$, $S_2$, and $S_3$. In \fgn{pcotdelta_summary}, we present $k{\mathrm{cot}}\delta_0$ versus 
$k^2$ based on the ground state energies presented in \fgn{gsspectrum} following \eqn{luscher}. 
The colored/gray bands indicate continuum extrapolated results including/excluding results from 
the $S_1$ ensemble to the respective fits. Clearly, there are no statistically significant effects
from the inclusion/exclusion of the energy levels from the $S_1$ ensemble observed.
\begin{figure}[h]
% Figure removed
\caption{$k{\mathrm{cot}}\delta_0$ versus $k^2$ for all $M_{\pi}$ values studied in units of 
the elastic threshold $E_{DB^*}$. The data points refer to the simulated data and follow 
the color coding in \tbn{lattice}. The dashed orange (cyan) curve indicates the constraint 
for the existence of a sub-threshold pole in the scattering amplitude. The horizontal bands
are the continuum extrapolated estimates of $k{\mathrm{cot}}\delta_0$ for the respective
$M_{\pi}$ (see \fgn{alatdep}). }
\eef{pcotdelta_summary}

Our main aim is to reliably determine $A^{[0]}=-1/a_0$, the sign of which determines the fate 
of the near threshold pole, if there exists one. A negative (positive) value of $A^{[0]}$($a_0$) 
indicates that the interaction potential is strong enough to form a real bound state\cite{Landau:1991wop}. 
It can be seen from \tbn{Ampfits1} and \fgn{alatdep} that for the non-charm light quark masses, 
$A^{[0]}$, the continuum extrapolated value for $k{\mathrm{cot}}\delta_0$ is negative, which 
indicates a possibly strong attractive interaction sufficient enough to host a real bound state. 
Whereas at the charm point, despite the unambiguous negative energy shifts in the finite-volume 
ground state energies with respect to the elastic threshold, the attraction is weak to host any 
real bound state as suggested by the positive value of $k{\mathrm{cot}}\delta_0$ in the continuum 
limit. This observation goes in line with the phenomenological expectation for doubly heavy four 
quark ($QQ'l_1l_2$) systems with $m_{l_1}=m_{l_2}$ that the binding increases with increased 
relative heaviness of the heavy quarks with respect to its light quark content
\cite{Francis:2016hui,Czarnecki:2017vco,Junnarkar:2017sey}. 

Another interesting observation is related to the lattice spacing dependence of $k{\mathrm{cot}}\delta_0$
values. At the charm point, $A^{[1]}$ (see \eqn{linparam}) acquires a different signature in contrast 
to that for the light quark masses. This suggests that for a doubly heavy four quark ($QQ'l_1l_2$) system 
with $(m_{l_1} = m_{l_2}, ~m_{Q},m_{Q'}>>m_{l})$, the cut off effects weaken the finite-volume energy 
splitting of the ground state with the elastic threshold. On the other hand, at the charm point (where 
$m_{Q},m_{Q'}\sim m_{l}$) such effects enhance this energy splitting in the $QQ'l_1l_2$ system determined 
in a finite-volume. Relatively large errors at the noncharm $M_{ps}$ values partially obscure these effects, 
if any exist, while at the charm point such effects are clearly reflected. 

\subsection{Light quark mass dependence}
Following the individual amplitude fits to different light quark mass cases, now we investigate the light 
quark mass ($m_{u/d}$) or $M_{ps}$ dependence of the parameters $A^{[0]}$ and $A^{[1]}$. Due to leading order 
$M_{ps}^2$ terms in the chiral expansion, we assume the $M_{ps}$ dependence of hadron masses for 
light $m_{u/d}$ values ($m_q\lesssim\Lambda_{QCD}$) to be linear in $M_{ps}^2$. Whereas towards the heavy 
$m_{u/d}$ regime ($m_q>>\Lambda_{QCD}$) heavy hadron masses are expected to be proportional to the quark mass, 
hence to $M_{ps}$ \cite{Neubert:1993mb}. With these assumptions, we work with three following fit forms that 
could be useful. 
\beqa
	f_l(M_{ps}) &=& \alpha_c + \alpha_l M_{ps}, \nonumber \\
	f_s(M_{ps}) &=& \beta_c + \beta_s M_{ps}^2, \mbox{~~~and} \nonumber \\
	f_q(M_{ps}) &=& \theta_c + \theta_l M_{ps} + \theta_s M_{ps}^2.
\eeqa{mqdep}
Fits to determine the $M_{ps}$ dependence were made by minimizing a single cost function 
defined combinedly for $A^{[0]}$ and $A^{[1]}$ as %\cite{FullLuscher}
\beq
	\chi^2 =\sum_{\substack{x, y \\ \in \{A^{[j]}_{i}\}}}\left(f_x-f_{px}(M_{ps})\right)\tilde{\mathcal{C}}^{-1}_{xy}\left(f_y-f_{py}(M_{ps})\right),
\eeq{chi2mqdep}
where the summation runs over all fitted parameters $\{A^{[j]}_{i}\}$ with $j\in\{0, 1\}$ and $i$ 
referring to the five different light quark masses studied. In \tbn{Ampfits1}, we list the fit results 
for $f_{x,y}$. $\tilde{\mathcal{C}}_{ij}$ is the associated data covariance determined following Ref. \cite{Prelovsek:2020eiw}. 
$f_{pn}(M_{ps})$ are the fit forms incorporating the $m_{u/d}$ dependence in parameters $\{A^{[j]}_{i}\}$.  
In \fgn{a0a1_separate}, we show the fit results for $A^{[0]}=-1/a_0$ to the fit forms in \eqn{mqdep}. The large 
circles represent the $A^{[0]}$ values at different $M_{ps}$, the bands represent the fit results 
with different fit forms in \eqn{mqdep}, and the two stars represent $A^{[0]}$ at the physical 
$M_{ps}$ (equivalently the physical scattering length $a_0^{phys}$) and the critical $M_{ps}$ at which 
$A^{[0]}$ changes its sign (positive to negative), in other words, the system becomes unbound. It is 
indeed desired to have more points in the intermediate mass regime between the charm and the strange 
\begin{figure}[h]
% Figure removed
\caption{Continuum extrapolated $k{\mathrm{cot}}\delta_0$ or $A^{[0]}=-1/a_0$ estimates of the $DB^*$ system 
as a function of $M_{ps}^2$ in units of $E_{DB^*}$. The band indicates fit results to the simulated results. 
The legend carries info on the fit forms presented (see also \eqn{mqdep}) and the quality of fits. The dotted 
vertical line close to the $y$-axis indicates the physical $M_{ps}$. The two star symbols represent the 
amplitude at the physical $M_{ps}$ and the critical $M_{ps}$ at which the system becomes unbound.}
\eef{a0a1_separate}
quark masses to further constrain the dependence. Yet, our fits in this work demonstrate near independence in 
the fit forms as can be observed from the consistency between the error bands from different fit 
forms. 

\begin{figure}[h]
% Figure removed
\caption{The landscape of the continuum scattering length $A^{[0]}$ versus $A^{[1]}$ (see \eqn{linparam}) 
for all $M_{ps}$ values (indicated in the legend) studied. The central values are represented by black 
edged circles with color fillings, whereas the scattered points are the bootstrap samples. The band 
represents the correlated $M_{ps}$ dependence of the fitted parameters.} 
\eef{a0a1_combined}
Next we look at the correlated pion mass dependence in the parameters $A^{[0]}$ and $A^{[1]}$ (see 
\eqn{chi2mqdep} for the definition of the cost function) presented in \fgn{a0a1_combined}. The black 
bordered symbols are the central values of parameters determined for each $M_{ps}$, whereas scattered 
small circles indicate the bootstrap sample distribution in the $[A^{[0]},~A^{[1]}]$ landscape. The bands 
in the figure represent the uncertainty in the parameters, with the inner band quantifying the statistical 
errors, while the outer band also incorporates the systematic uncertainty arising from different fit forms 
added in quadrature symmetrically. A negative correlation can clearly be observed between the parameters 
across different quark masses studied, which is accounted in the fits through the data covariance matrix 
entering the cost function. This correlation can also be observed within the distribution of the bootstrap 
samplings at all quark masses. This observation clearly demonstrates the need for a careful treatment of 
cutoff errors, particularly in heavy hadron systems with interesting near threshold features, such as this. 

In the chiral regime ($m_{u/d}\lesssim\Lambda_{QCD}$), leading $m_{u/d}$ dependence in hadronic observables 
is assumed to go as linear in $M_{ps}^2$. Based on the fit form $f_s(M_{ps})$, we find that the scattering length 
of the $DB^*$ system at the physical light quark mass ($m_{u/d}^{phys}$) to be
\beq
a_0^{phys} = 0.57(^{+4}_{-5})(17) \mbox{~fm}.
\eeq{scatlen}
The asymmetric errors indicate the statistical uncertainties, whereas the second parenthesis quotes 
the systematic uncertainties with the most dominant contribution arising from the chiral extrapolation 
fit forms. We elaborate on various systematic uncertainties towards end of this section. The positive 
value of the scattering length is an unambiguous evidence for the ability/strength of the hadron-hadron 
interaction potential to host a real bound state (when $k~{\mathrm{cot}}\delta_0 = -\sqrt{-k^2}$). 
The observed scattering length at physical light quark mass suggests the presence of a real $bc\bar u\bar d$ 
tetraquark bound state $T_{bc}$ with binding energy 
\beq
\delta m_{T_{bc}} = -43(^{+6}_{-7})(^{+14}_{-24}) \mbox{~MeV},
\eeq{betbc}
with respect to $E_{DB^*}$. The systematic effects on the $a_0^{phys}$ and $\delta m_{T_{bc}}$ estimates 
of ignoring the charm point in the fits to the $m_{u/d}$ dependence are found to be very small, 
compared to the number quoted for systematic uncertainties in \eqn{scatlen}. 

Towards the heavy quark regime ($m_{u/d}>>\Lambda_{QCD}$), the heavy hadron masses can have 
leading linear dependence in $M_{ps}$ as $M_{ps}\propto m_{u/d}\sim m_{Q}$ \cite{Neubert:1993mb}. 
Following the fit form $f_l(M_{ps})$, which is linear in quark mass, the critical light quark mass 
$m_{u/d}^*$ at which the scattering length diverges, then changes its signature such that the 
interaction potential is not able host a real bound state, corresponds to the critical pseudoscalar 
meson mass given by 
\beq
M^{*}_{ps} = 2.73(21)(14) \mbox{~GeV}.
\eeq{unitary}
This corresponds to the star symbol at the zero crossing in the $x$-axis ($A^{[0]}=0$) in \fgn{a0a1_separate}. 
Once again the first parenthesis indicates the statistical errors and the second one quantifies various 
systematic uncertainties added in quadrature. 


Now we briefly comment on other possible sources of systematic uncertainties in this calculation. Our lattice setup, 
discussed in Section \ref{sec:lattice}, together with the bare bottom and charm quark mass tuning procedure 
has been demonstrated to reproduce the $1S$ hyperfine splittings in quarkonia with uncertainties less than 6 MeV
\cite{Mathur:2022ovu,Mathur:2016hsm}. Additionally, our strategy of evaluating the energy differences 
and working with mass ratios has also been shown to significantly mitigate the systematic uncertainties related 
heavy quark masses \cite{Mathur:2018epb,Mathur:2022ovu}. Our fitting procedure discussed in Section \ref{sec:2ptIO} 
involves careful and conservative determination of statistical errors, and uncertainties related to the 
excited-state-contamination and fit-window errors. The amplitude determination and followed extrapolations
are performed with results from varying the fit-windows to evaluate the uncertainties propagated to our final 
results. The uncertainties related to the fit forms used in chiral extrapolations are observed to be dominant,  
and the number in the second parenthesis in Eqs. \ref{scatlen}, \ref{betbc}, and \ref{unitary} are the total 
systematic uncertainties added in quadrature. Uncertainty related to scale setting are also found to be negligibly 
small in comparison to the statistical uncertainties \cite{Mathur:2018epb,Mathur:2022ovu}. 


{\it $D\bar B^*$ scattering amplitude}: Assuming these energy splittings in ground states are purely 
described by elastic scattering in the $D\bar B^*$ system, we utilize them to constrain the associated 
$S$-wave scattering amplitude following L\"uscher's finite-volume prescription \cite{Luscher:1990ux,
Briceno:2014oea}. For the low energy scattering of $D$ and $\bar B^*$ mesons, where other multi-particle 
thresholds are sufficiently high \cite{Draper:2021clv}, in the $S$-wave leading to the total angular 
momentum and parity $J^P=1^+$, the scattering phase shifts $\delta_{l=0}(k)$ are related to the 
finite-volume energy spectrum through $kcot[\delta_0(k)] = 2Z_{00}[1;(\frac{kL}{2\pi})^2)]/(L\sqrt{\pi})$. 
Here, $k$($E_{cm}=\sqrt{s}$) is the momentum (energy) in the center of momentum frame such that 
$4sk^2 = (s-(M_{D}+M_{\bar B^*})^2)(s-(M_{D}-M_{\bar B^*})^2)$. We follow the procedure outlined in 
Appendix B of Ref. \cite{Padmanath:2022cvl} to constrain the amplitude. A sub-threshold pole in 
the $S$-wave scattering amplitude $t = ({\mathrm{cot}}\delta_0 - i)^{-1}$ occurs when 
$k{\mathrm{cot}}\delta_0 = \pm\sqrt{-k^2}$ for scattering in $S$-wave. 

We parametrize the elastic $D\bar B^*$ scattering amplitude in terms of the scattering length $a_0$ 
in an effective range expansion near the threshold, supplemented by a lattice spacing dependence. 
This is required to incorporate the cut-off effects observed in the ground state energy estimates. 
We find that a linear functional form given by $k{\mathrm{cot}}\delta_0 = A^{[0]} + aA^{[1]}$, where 
$A^{[0]}=-1/a_0$, accommodate the $a$ dependence of the $k{\mathrm{cot}}\delta_0$ estimates. We 
present the fit results for $A^{[0]}=-1/a_0$ in \fgn{a0a1_separate} (circle symbols) as a function 
of $M_{ps}$ involved.  Alternative fitting choices with a leading quadratic dependence or using 
only data from non-charm $M_{ps}$ are consistent with results in \fgn{a0a1_separate}, see details 
in Supplemental material \cite{Suppl}. 

The sign of $A^{[0]}=-1/a_0$ determines the fate of the near-threshold pole, if there exists one. 
A negative (positive) value of $A^{[0]}$($a_0$) indicates that the interaction potential is strong 
enough to form a real bound state\cite{Landau:1991wop}. After considering all possible systematics, 
we find that for all the non-charm light quark masses, $A^{[0]}$ is negative, which indicates an 
attractive interaction strong enough to host a real bound state. On the contrary, at the charm point, 
despite the unambiguous negative energy shifts in the ground states, the attraction is weak to host 
any real bound state as suggested by the positive value of $k{\mathrm{cot}}\delta_0$ in the continuum 
limit. This observation goes in line with the phenomenological expectation for doubly heavy four 
quark ($QQ'l_1l_2$) systems with $m_{l_1}=m_{l_2}$ that the binding increases with increased relative 
heaviness of the heavy quarks with respect to its light quark content\cite{Francis:2016hui,Czarnecki:2017vco,Junnarkar:2017sey}. 

Now we investigate the light quark mass ($m_{u/d}$) or $M_{ps}$ dependence of the fitted parameters. 
To this end, we consider three different parametrizations: a linear dependence ($f_l(M_{ps}) = 
\alpha_c + \alpha_l M_{ps}$) to probe the heavy light quark mass case, a leading $M_{ps}^2$ 
dependence ($f_s(M_{ps}) = \beta_c + \beta_s M_{ps}^2$) to assess the chiral behaviour, and a 
quadratic dependence ($f_q(M_{ps}) = \theta_c + \theta_l M_{ps} + \theta_s M_{ps}^2$) to quantify 
the associated systematics. In \fgn{a0a1_separate}, we show the fit results for this $M_{ps}$ 
dependence in colored bands. The two stars represent $A^{[0]}$ at the physical $M_{ps}$ 
(equivalently the physical scattering length $a_0^{phys}$) and the critical $M_{ps}$ at which 
$A^{[0]}$ changes its sign or above which the system becomes unbound. It is indeed desired to have 
more points in the intermediate mass regime between the charm and the strange 
\begin{figure}[h]
% Figure removed
\caption{Continuum extrapolated $k{\mathrm{cot}}\delta_0$ or $A^{[0]}=-1/a_0$ estimates of the $D\bar B^*$ system 
as a function of $M_{ps}^2$ in units of $E_{D\bar B^*}$. The dotted vertical line close to the $y$-axis indicates 
$M_{ps}=M_{\pi}^{phys}$. The two star symbols represent the amplitude at $M_{ps} = M_{\pi}^{phys}$ and 
the critical $M_{ps}=M^{*}_{ps}$ above which the system becomes unbound.}
\eef{a0a1_separate}
quark masses to further constrain the dependence. Yet, our fits demonstrate near independence in 
the fit forms as can be observed from the consistency between the error bands from different fit 
forms. 

Based on the fit form $f_s(M_{ps})$ in the chiral regime, we find that the scattering length of the $D\bar B^*$ 
system at the physical light quark mass ($m_{u/d}^{phys}$), corresponding to $M_{ps}=M_{\pi}^{phys}$, to be
\beq
a_0^{phys} = 0.57(^{+4}_{-5})(17) \mbox{~fm}.
\eeq{scatlen}
The asymmetric errors indicate the statistical uncertainties, whereas the second parenthesis quotes 
the systematic uncertainties with the most dominant contribution arising from the chiral extrapolation 
fit forms. The positive value of the scattering length at $M_{ps}=M_{\pi}^{phys}$, at the level of
3$\sigma$ uncertainty, is an unambiguous evidence for the strength of the $D\bar B^*$ interaction 
potential to host a real $bc\bar u\bar d$ tetraquark bound state $T_{bc}$ with binding energy 
\beq
\delta m_{T_{bc}} = -43(^{+6}_{-7})(^{+14}_{-24}) \mbox{~MeV},
\eeq{betbc}
with respect to $E_{D\bar B^*}$. The first parenthesis indicates the statistical errors and the second one 
quantifies various systematic uncertainties added in quadrature. The pseudoscalar meson mass, corresponding 
to the critical light quark mass, where $a_0$ diverges, is found to be $M^{*}_{ps} = 2.73(21)(19) \mbox{~GeV}$. 
This critical point also signifies that QCD dynamics within such exotic systems is such that at a heavy 
light quark mass the system of quarks perhaps reaches the unitary gas limit, as indicated by the 
divergent scattering length \cite{Newton:1982qc}. For $M_{ps}\ge M^{*}_{ps}$, the $T_{bc}$ system 
remains unbound.

{\it Systematic uncertainties}: Our lattice setup together with the bare bottom and charm quark mass 
tuning procedure has been demonstrated to reproduce the $1S$ hyperfine splittings in quarkonia with 
uncertainties less than 6 MeV \cite{Mathur:2022ovu,Mathur:2016hsm}. We observe the effects of such a 
mistuning of either of the heavy quark mass on the energy splittings we extract are very small compared
to the statistical errors. Additionally, our strategy of evaluating the energy differences and working 
with mass ratios has also been shown to significantly mitigate the systematic uncertainties related to 
heavy quark masses \cite{Mathur:2018epb,Mathur:2022ovu}. This is observed to be the case in this study 
as well, leading to transparent signals for the ground state energy as shown in Figures \ref{fg:fitcompare},
\ref{fg:spectrum}, and \ref{fg:gsspectrum}. Our fitting procedure involves careful and conservative 
determination of statistical errors, and uncertainties related to the excited-state-contamination and 
fit-window errors. Additional checks using alternative quark smearing procedures also agree with our 
energy estimates, see Appendix A. The amplitude determination and followed extrapolations are performed 
with results from varying the fit-windows to evaluate the uncertainties propagated to our final results. 
The uncertainties related to the fit forms used in chiral extrapolations are observed to be the most 
dominant, as is evident from Figure \ref{fg:a0a1_separate}. We assume the partially quenched setup
involving ensembles with different sea pion masses, we utilize, have negligible effects on the energy 
splittings we extract for the explicitly exotic $T_{bc}$ tetraquark, similar to what was observed for 
heavy hadrons in Refs. \cite{Dowdall:2012ab,McNeile:2012qf}. Uncertainty related to scale setting is 
also found to be negligible in comparison to the statistical uncertainties in the energy splittings. 


%\begin{table*}[t]
\caption{Summary of the top-performing teams in each track of the RoboDepth Challenge.}
\centering\scalebox{1}{
\begin{tabular}{c|p{5cm}|p{5cm}}
\toprule
\textbf{Rank} & \textbf{\#1: Robust Self-Supervised MDE} & \textbf{\#2: Robust Supervised MDE}
\\\midrule\midrule
\multirow{13}{*}{\textcolor{robo_blue}{\textbf{1st Place}}} & \textbf{Team Name} & \textbf{Team Name}
\\
& \textcolor{robo_blue}{OpenSpaceAI} & \textcolor{robo_blue}{USTCxNetEaseFuxi}
\\
\cmidrule{2-3}
& \textbf{Team Members} & \textbf{Team Members}
\\
& Ruijie Zhu$^1$, Ziyang Song$^1$, Li Liu$^1$, Tianzhu Zhang$^{1,2}$ & Jun Yu$^1$, Mohan Jing$^1$, Pengwei Li$^1$, Xiaohua Qi$^1$, Cheng Jin$^2$, Yingfeng Chen$^2$, Jie Hou$^2$
\\
\cmidrule{2-3}
& \textbf{Affiliations} & \textbf{Affiliations}
\\
& $^1$University of Science and Technology of China, $^2$Deep Space Exploration Lab & $^1$University of Science and Technology of China, $^2$NetEase Fuxi
% \\
% \cmidrule{2-3}
% & \textbf{Approach} & \textbf{Approach}
% \\
% & IRUDepth with MPViT as depth encoder and PoseNet for camera poses and depth maps with AugMix& <...>
\\\cmidrule{2-3}
& \textbf{Contact} $\textrm{\Letter}$ & \textbf{Contact} $\textrm{\Letter}$
\\
& \texttt{ruijiezhu@mail.ustc.edu.cn} & \texttt{USTC\_IAT\_United@163.com}
\\\midrule\midrule
\multirow{17}{*}{\textcolor{robo_red}{\textbf{2nd Place}}} & \textbf{Team Name} & \textbf{Team Name}
\\
& \textcolor{robo_red}{USTC-IAT-United} & \textcolor{robo_red}{OpenSpaceAI}
\\
\cmidrule{2-3}
& \textbf{Team Members} & \textbf{Team Members}
\\
& Jun Yu$^1$, Xiaohua Qi$^1$, Jie Zhang$^2$, Mohan Jing$^1$, Pengwei Li$^1$, Zhen Kan$^1$, Qiang Ling$^1$, Liang Peng$^3$, Minglei Li$^3$, Di Xu$^3$, Changpeng Yang$^3$ & Li Liu$^1$, Ruijie Zhu$^1$, Ziyang Song$^1$, Tianzhu Zhang$^{1,2}$
\\
\cmidrule{2-3}
& \textbf{Affiliations} & \textbf{Affiliations}
\\
& $^1$University of Science and Technology of China, $^2$Central South University, $^3$Huawei Cloud Computing Technology Co., Ltd & $^1$University of Science and Technology of China, $^2$Deep Space Exploration Lab
\\
\cmidrule{2-3}
& \textbf{Contact} $\textrm{\Letter}$ & \textbf{Contact} $\textrm{\Letter}$
\\
& \texttt{USTC\_IAT\_United@163.com} & \texttt{liu\_li@mail.ustc.edu.cn}
\\\midrule\midrule
\multirow{11}{*}{\textcolor{robo_green}{\textbf{3rd Place}}} & \textbf{Team Name} & \textbf{Team Name}
\\
& \textcolor{robo_green}{YYQ} & \textcolor{robo_green}{GANCV}
\\
\cmidrule{2-3}
& \textbf{Team Members} & \textbf{Team Members}
\\
& Yuanqi Yao$^1$, Gang Wu$^1$, Jian Kuai$^1$, Xianming Liu$^1$, Junjun Jiang$^1$ & Jiamian Huang$^1$, Baojun Li$^1$
\\
\cmidrule{2-3}
& \textbf{Affiliations} & \textbf{Affiliations}
\\
& $^1$Harbin Institute of Technology & $^1$Individual Researcher
\\
\cmidrule{2-3}
& \textbf{Contact} $\textrm{\Letter}$ & \textbf{Contact} $\textrm{\Letter}$
\\
& \texttt{yuanqiyao@stu.hit.edu.cn} & \texttt{huang176368745@gmail.com}
\\\bottomrule
\end{tabular}
}
\label{tab:summary}
\end{table*}
{\it Summary}: We have performed a lattice QCD simulation of coupled $D\bar B^*$-$\bar BD^*$ scattering with 
explicitly exotic flavor $bc\bar u\bar d$ and $I(J^P) = 0(1^+)$. Following a rigorous extraction of 
finite-volume eigenenergies and continuum extrapolated elastic $D\bar B^*$ scattering amplitudes for 
the five light quark masses studied, we determine the light quark mass dependence of the elastic 
$D\bar B^*$ scattering length $a_0$. We observe unambiguous negative energy shifts between the interacting 
and non-interacting finite-volume energy levels. Our estimate for $a_0^{phys}$ (\eqn{scatlen}) is 
positive, indicating an attractive interaction between the $D$ and $\bar B^*$ mesons, which is strong 
enough to host a real bound state with binding energy $\delta m_{T_{bc}} = -43(^{+6}_{-7})(^{+14}_{-24})$ 
MeV. We find that the strength of interaction is such that this $bc\bar u\bar d$ tetraquark becomes 
unbound at $M^{*}_{ps}$, which is close to the $\eta_c$ meson mass. 

In this work, we make several important steps ahead to arrive at robust inference on the nature of interaction
between the $D$ and $\bar B^*$ mesons. Our main strategy has been to determine the signature of scattering length 
in $D\bar B^*$ interactions at the physical pion mass $a_0^{phys}$. Our results indicate that $a_0^{phys}$ is 
positive, which suggests that attractive $D\bar B^*$ interactions are strong enough to host a real bound state. 
Further theoretical investigations are desired to reduce the uncertainties in the binding energy of $T_{bc}$ 
with respect to $E_{D\bar B^*}$. Fully dynamical simulations on several more ensembles, with different volumes 
and improvized fermion actions, high statistics studies with lighter $m_{u/d}$, etc. are a few other 
improvisations that can further constrain the relevant scattering amplitude. Additionally, future works 
involving Hermitian correlation matrices at rest as well as in moving frames and those using bilocal two-meson 
interpolators with nonzero relative meson momenta aimed at reliable excited state extraction would be 
a few important steps ahead \cite{Padmanath:2018tuc,Padmanath:2022cvl,Chen:2022vpo,Wagner:2022bff}. We 
hope that our observations and inferences in this work will motivate more theoretical efforts and 
experimental searches for such states.



%%%%%%%%%%%%%%%%%%%%%%%%%%%%%%%%%%%%%%%%%%%%%%%%%%%%%%%%%%%%%%%%%%%%%%%%%%%%%%%%%
%% ACKNOWLEDGMENTS
\begin{acknowledgments}
%
This work is supported by the Department of Atomic Energy, Government of India, under Project Identification Number RTI 4002. We are thankful to the MILC collaboration and in particular to S. Gottlieb for providing us with the HISQ lattice ensembles. We thank Sara Collins for a careful reading of the manuscript. We thank the authors of Ref. \cite{Morningstar:2017spu} for making the {\it TwoHadronsInBox} package utilized in this work. We also thank Gunnar Bali, Parikshit Junnarkar and Sayantan Sharma for discussions. Computations were carried out on the Cray-XC30 of ILGTI, TIFR. Amplitude analyses were performed on Nandadevi computing cluster at IMSc Chennai. N. M. would also like to thank A. Salve and K. Ghadiali for computational support.
\end{acknowledgments}


\begin{acknowledgments}
This work is supported by the Department of Atomic Energy, Government of India, under Project Identification Number RTI 4002. M.P. gratefully acknowledges support from the Department of Science and Technology, India, SERB Start-up Research Grant No. SRG/2023/001235. We are thankful to the MILC collaboration and in particular to S. Gottlieb for providing us with the HISQ lattice ensembles. We thank Sara Collins for a careful reading of the manuscript. We thank the authors of Ref. \cite{Morningstar:2017spu} for making the {\it TwoHadronsInBox} package utilized in this work. We also thank Gunnar Bali, Parikshit Junnarkar, Alexey Nefediev, Sayantan Sharma, Stephen R. Sharpe, and Tanishk Shrimal for discussions. Computations were carried out on the Cray-XC30 of ILGTI, TIFR. Amplitude analyses were performed on Nandadevi computing cluster at IMSc Chennai. N. M. would also like to thank A. Salve and K. Ghadiali for computational support.
\end{acknowledgments}

%\cite{Dudek:2009qf,Padmanath:2013zfa,Padmanath:2015era,Francis:2016hui,Czarnecki:2017vco,Junnarkar:2018twb}

\textit{Appendix A: Ground state energy plateau.-}
In this work we have utilized a wall-source point-sink setup to construct the necessary two-point 
correlation functions. The use of such an asymmetric setup implies the effective energies 
$aE_{eff} = [ln(C(t)/C(t+\delta t))]/\delta t$ could approach their asymptotic values as rising-from-below, 
due to the nonpositive definite nature of the coefficients in a spectral decomposition, in 
contrast to a falling-from-above behaviour in a symmetric setup. In \fgn{boxsnk_plot}, we show 
the effective mass in wall-source point-sink setup with the brown-circle ($R^2 = 0$) which rises from below.

To avoid any ambiguity in selecting the plateau regions of effective masses of such correlators, we also 
employ a wall-source box-sink setup \cite{Hudspith:2020tdf}, which asymptotically approaches the symmetric 
limit. In the symmetric limit, the effective masses are expected to follow a conventional falling-from-above 
feature, modulo the statistical noise. To this end, we vary the smearing radius $R$ to investigate 
the time dependence of effective mass plateaus in the approach to the symmetric limit. In \fgn{boxsnk_plot}, 
we present a comparison of the effective energy (top) and effective energy splittings (bottom) determined 
using different quark sink smearing procedures for the case of $M_{ps}\sim700$ MeV on the finest ensemble. 
Clearly the rising-from-below behaviour is gradually disappearing in the approach to the symmetric limit. 
It is also evident that the results at the large time limit from point-sink and box-sink are very much 
consistent with each other affirming our assessment on effective mass plateau in choosing a fit range. Such 
a behavior of effective masses with varying smearing radii was also observed in Ref.~\cite{Hudspith:2020tdf}. 
In the large time limit, where the signal quality is still good, all of sink smearing cases suggest 
consistent negative energy shifts. This is evident from the large time behaviour of energy splittings 
presented in the bottom panel of \fgn{boxsnk_plot}, where the correlated statistical noise, not related to 
the excited state contamination, is suppressed between the numerator and denominator in the ratio correlators $R^n(t)$.

The agreement of energy splitting estimates from fits to $R^n(t)$ with those evaluated from separate fits 
to the GEVP eigenvalue correlators $\lambda^n(t)$ and the single-meson correlators $\mathcal{C}_{D/\bar B^*}$ 
at large times (see \fgn{fitcompare}) already rules out the usual concern of accidental partial cancellation 
of excited state contaminations in $R^n(t)$. The consistency at large times between ground state energy 
plateaus from different sink-smearing radii observed in top panel of \fgn{boxsnk_plot} further affirms 
the reliable isolation of the ground state plateau. Note also that the magnitude of such cancellations and 
the ground state saturation times could be different in different lattice QCD ensembles. All the ground 
state estimates for noncharm $M_{ps}$ values in our study are determined from the time intervals approximately 
between 1.5(2) fm [$t_{min}$] to 2.3(2) fm [$t_{max}$]. The consistent ground state saturation times across 
different ensembles with different specifications further imply the reliability of our ground state saturation, 
despite our asymmetric setup.

\bef[hbt]
% Figure removed
\caption{Comparison of effective energy (top) and effective energy splitting (bottom) for the ground state as 
determined using three different smearing radii applied on the quark fields at the sink timeslice. The legend 
indicates the smearing radius squared in units of the lattice spacing \cite{Hudspith:2020tdf}. The blue horizontal 
band indicates the final fit estimate for the energy and energy splitting. The results presented are for 
the case $M_{ps}\sim700$ MeV on the finest ensemble. }
\eef{boxsnk_plot}

\textit{Appendix B: Elastic $D\bar B^*$ excitations.-} Gaining access to higher lying elastic excitations 
in the $D\bar B^*$ channel is an important step ahead towards constraining the energy dependence of the 
amplitude over a long energy range. However, within the wall-smearing setup, all the nonzero momentum 
excitations are significantly suppressed. This suppression is exact in a free theory, and is empirically 
confirmed from the early plateauing and from the quality of signals in the interacting theory. While this 
suppression is advantageous in ground state energy determination (see Refs.\cite{Francis:2016hui,Francis:2018jyb,
Mathur:2018epb,Junnarkar:2018twb,Hudspith:2020tdf,Mathur:2022ovu} for details), the suppressed coupling to 
the nonzero momentum excitations implies that the access to higher two-meson elastic excitations with 
nonzero relative meson momenta are restricted in the wall-smearing setup. This implies other methodologies 
that facilitate the use of bilocal two-meson interpolators with separately momentum projected mesons are 
necessary in future studies \cite{HadronSpectrum:2009krc,Abdel-Rehim:2017dok,Wagner:2022bff}\footnote{While this 
letter was being reviewed, a preprint, Ref.~\cite{Alexandrou:2023cqg} appeared which utilizes bilocal 
two-meson interpolators in their analysis, utilizing the methods in Ref.~\cite{Abdel-Rehim:2017dok}.}. In 
this respect, it is informative to know the location of the lowest non-interacting level with nonzero relative 
meson momenta and whether it is close enough to influence the ground state energies in any substantial way. 
Considering this, in \fgn{gsspectrumee} we present the ground state eigenenergies along with the $D\bar B^*$ 
threshold and the next lowest elastic $D\bar B^*$ excitation with nonzero relative meson momentum determined 
using the continuum dispersion relation that is assumed in the finite-volume quantization condition 
\cite{Luscher:1990ux,Briceno:2014oea}. Clearly, the location of this first non-interacting elastic excitation 
is sufficiently high to have any nonnegligible effects on the extracted the ground state energies.

\begin{figure}[h]
% Figure removed
\caption{The ground state energy eigenvalues in the background of lowest two non-interacting $D\bar B^*$ 
finite-volume levels units of $E_{D\bar B^*}$ on all ensembles (see \tbn{lattice} for color-symbol 
conventions) for all $M_{ps}$ values (different vertical panels).}
\eef{gsspectrumee}

%%%%%%%%%%%%%%%%%%%%%%%%%%%%%%%%%%%%%%%%%%%%%%%%%%%%%%%%%%%%%%%%%%%%%%%%%%%%%%%%%
%% BIBILOGRAPHY
\bibliography{paper}

%%%%%%%%%%%%%%%%%%%%%%%%%%%%%%%%%%%%%%%%%%%%%%%%%%%%%%%%%%%%%%%%%%%%%%%%%%%%%%%%%

% balance columns at end of main text
\onecolumngrid
\clearpage
\onecolumngrid

\begin{center}
  {\Large \bf Supplemental material}
\end{center}

% turn on section numbers for the supplement, when it's included in the document
\makeatletter
\c@secnumdepth=4
\makeatother

\newif\ifsepsupp
\sepsuppfalse

\newpage
\appendix

\section{Proof of Lemma \ref{lemma, equivalence of two def of MDDO}}
\begin{proof}
For any ${\bs{\beta}}\in\mc H$, according to the definition of $G_{\bs s}$ (see Definition $\ref{def: MDDO}$), one has
\begin{align*}
\langle G_{\bs s},{\bs{\beta}}\rangle&=\int_{[0,1]} G_{\bs s}(t){\bs{\beta}}(t)~\mathrm{d}t=\int_{[0,1]}\mathrm{cov}\hspace{-0.9mm}\left(\bs{X}(t),\mathrm{e}^{\mi\langle \bs s,\Y\rangle}\right){\bs{\beta}}(t)~\mathrm{d}t\\
&=\int_{[0,1]}\mathrm{cov}\hspace{-0.9mm}\left(\bs{X}(t){\bs{\beta}}(t),\mathrm{e}^{\mi \langle \bs s,\Y\rangle}\right)~\mathrm{d}t.
\end{align*}
By Fubini theorem, under Assumption $\ref{as:joint distribution assumption}$, one can exchange the order of integration and covariance above and get that
\begin{align*}
 \langle G_{\bs s},{\bs{\beta}}\rangle&=\int_{[0,1]}\mathrm{cov}\hspace{-0.9mm}\left(\bs{X}(t){\bs{\beta}}(t),\mathrm{e}^{\mi \langle \bs s,\Y\rangle}\right)~\mathrm{d}t\\ &=\mathrm{cov}\hspace{-0.9mm}\left(\int_{[0,1]}\bs{X}(t){\bs{\beta}}(t)~\mathrm{d}t,\mathrm{e}^{\mi \langle \bs s,\Y\rangle}\right)=\mathrm{cov}\hspace{-0.9mm}\left(\langle \bs{X},{\bs{\beta}}\rangle,\mathrm{e}^{\mi \langle \bs s ,\Y\rangle}\right).
\end{align*}
Thus for any $\bs\alpha(t),{\bs{\beta}}(t)\in\mc H$, one can get
\begin{align*}
\big\langle \big(G_{\bs s}\otimes \overline{G}_{\bs s}\big)\bs\alpha,{\bs{\beta}}\big\rangle=\langle G_{\bs s},\bs\alpha\rangle\langle \overline{G}_{\bs s},{\bs{\beta}}\rangle=\mathrm{cov}\hspace{-0.9mm}\left(\langle \bs{X},\bs\alpha\rangle,\mathrm{e}^{\mi \langle \bs s,\Y\rangle}\right)\hspace{-0.9mm}\mathrm{cov}\hspace{-0.9mm}\left(\langle \bs{X},{\bs{\beta}}\rangle,\mathrm{e}^{-\mi\langle \bs s,\Y\rangle}\right)\\
=\mb{E}\hspace{-0.9mm}\left(\langle \bs{X},\bs\alpha\rangle\mathrm{e}^{\mi \langle \bs s,\Y\rangle}\right)\mb{E}\hspace{-0.8mm}\left(\langle \bs{X},{\bs{\beta}}\rangle\mathrm{e}^{-\mi \langle \bs s,\Y\rangle}\right)=\mb{E}\Big(\langle \bs{X},\bs\alpha\rangle\langle \bs{X}',{\bs{\beta}}\rangle\mathrm{e}^{\mi \langle \bs s,\Y-\Y'\rangle}\Big).
\end{align*}
Considering that $\mb{E}\big(\langle \bs{X},\alpha\rangle\langle \bs{X}',{\bs{\beta}}\rangle\big)=0$, one has
\begin{align*}
\big\langle \big(G_{\bs s}\otimes \overline{G}_{\bs s}\big)\bs\alpha,{\bs{\beta}}\big\rangle
=- \mb{E}\Big(\langle \bs{X},\bs\alpha\rangle\langle \bs{X}',{\bs{\beta}}\rangle\big(1-\mr{e}^{\mi \langle \bs s,\Y-\Y'\rangle}\big)\Big)&\\
=- \mb{E}\Big(\langle \bs{X},\bs\alpha\rangle\langle \bs{X}',{\bs{\beta}}\rangle\big[1-\cos\big(\langle \bs s,\Y-\Y'\rangle\big)\big]\Big)&\\
+\mi\mb{E}\Big(\langle \bs{X},\bs\alpha\rangle\langle \bs{X}',{\bs{\beta}}\rangle\big[\sin\big(\langle\bs s,\Y-\Y'\rangle\big)\big]\Big)&.
\end{align*}
It is easy to check that
\[\int_{\mb R^q}\frac{\sin \big(\langle\bs s,\Y-\Y'\rangle)\big)}{\|\bs s\|^{1+q}}~\mr{d}\bs s=\lim_{\varepsilon\to0^+}\int_{\bs s\in\mb{R}^q:\varepsilon\leqslant\|\bs s\|\leqslant \varepsilon^{-1}}\frac{\sin \big(\langle \bs s,\Y-\Y'\rangle\big)}{\|\bs s\|^{1+q}}~\mr{d}\bs s=0,\]
because the integrand is an odd function. By Lemma 1 in \cite{szekely2007measuring},  one can also get
\[\int_{\R^q}\frac{1-\cos\big(\langle \bs s,\Y-\Y'\rangle\big)}{\|\bs s\|^{1+q}}~\mr{d}\bs s=c_q\|\Y-\Y'\|.
\]
Combining above results with Definition $\ref{def: MDDO}$, one can obtain that 
\begin{align}\label{proof: lemma MDDO}
\langle\mathrm{MDDO}(\bs{X}|Y)\bs\alpha,{\bs{\beta}}\rangle=- \mb{E}\Big(\langle \bs{X},\bs\alpha\rangle\langle \bs{X}',{\bs{\beta}}\rangle\|\Y-\Y'\|\Big) .
\end{align}
Then by the arbitrariness of $\bs\alpha,{\bs{\beta}}\in\mc H$, the proof is completed. 
\end{proof}

\section{Proof of Theorem \ref{theorem, MDDO and conditional mean independence}}



According to \eqref{proof: lemma MDDO}, one can get the following useful lemma.
\begin{lemma}\label{lemma, MDDO and FMDD}
Under Assumption $\ref{as:joint distribution assumption}$, for all ${\bs{\beta}}\in\mathcal H$, $\|{\bs{\beta}}\|=1$, we have
\begin{align*}
\langle \mathrm{MDDO}(\boldsymbol{X}|\Y)({\bs{\beta}}),{\bs{\beta}}\rangle &=- \mathbb E\Big[ \langle \boldsymbol{X},{\bs{\beta}}\rangle \langle \boldsymbol{X}',{\bs{\beta}}\rangle \|\Y-\Y'\|\Big]\\
&=- \mathbb E\Big[\big\langle\langle \boldsymbol{X},{\bs{\beta}}\rangle{\bs{\beta}},\langle \boldsymbol{X}',{\bs{\beta}}\rangle{\bs{\beta}}\big\rangle\|\Y-\Y'\|\Big].
\end{align*}
\end{lemma}
This conclusion links MDDO with functional martingale
difference divergence  (FMDD, \citealt{lee2020testing}). 
Next we give the following two lemmas to finish the proof of Theorem $\ref{theorem, MDDO and conditional mean independence}$.
\begin{lemma}\label{lem: Txx=0tuiTx=0}If $T$ is a positive semi-definite operator on a Hilbert space $\wt{\mathcal{H}}$, then for all $x\in\wt{\mathcal{H}}$, one has $\langle Tx,x\rangle=0\Longleftrightarrow Tx=0$.
\end{lemma}
\begin{proof}
`$\Longleftarrow$': It is obvious.

`$\Longrightarrow$': It is easy to check that $f(a,b)=\langle Ta,b\rangle$ $(a,b\in\wt{\mc H})$ is a 
positive semi-definite Hermitian form. Thus, for any $y\in\wt{\mathcal{H}}$, one can use Cauchy inequality to get
\[|\langle Tx,y\rangle|^2\leqslant\langle Tx,x\rangle\langle Ty,y\rangle=0\Longrightarrow \langle Tx,y\rangle=0.\]
By the arbitrariness of $y\in\wt{\mc H}$, one has $Tx=0$.
\end{proof}

Our proof of Theorem $\ref{theorem, MDDO and conditional mean independence}$ is mainly inspired by the following property of
FMDD in \cite{lee2020testing}.
\begin{lemma}[Proposition 1 of \cite{lee2020testing}]\label{lem:prop1inlee}
If $\E[\|\X\|+\|\Y\|]<\infty$ and $\E[\|\bs X\|\|\Y\|]<\infty$, then we have
\[\E[\langle \X,\X'\rangle\|\Y-\Y'\|]=0\Longleftrightarrow \E[\X|\Y]=0\quad\text{almost surely},\]
where $(\X',\Y')$ is an i.i.d. copy of $(\X,\Y)$.
\end{lemma}
\paragraph{Proof of Theorem $\ref{theorem, MDDO and conditional mean independence}$}
\begin{proof}
Clearly, (ii) is a direct consequence of Lemma $\ref{lemma, equivalence of two def of MDDO}$ and the following lemma.

\begin{lemma}[Lemma 15 in \citealt{chen2023optimality}]\label{lem:cov TX}
If $T$ is an operator defined on $\mc H_1\to\mc H_2$ where $\mc H_i,i=1,2$ is a Hilbert space. $\bs X\in\mc H_1$ is a random element satisfying $\mb E[\bs X]=0$ . Then we have $\mr{var}(T\bs X)=T\mr{var}(\bs X)T^*$.
\end{lemma}

Now we start  to prove (i).
 First, one has
\begin{align*}\mathrm{MDDO}(\boldsymbol{X}|\Y)=0 &\Longleftrightarrow \mathrm{MDDO}(\boldsymbol{X}|\Y)({\bs{\beta}})=0,\quad\forall{\bs{\beta}}\in\mb{S}_{\mathcal H};\\
\mathbb E[\boldsymbol{X}|\Y]=0~~\text{a.s.}&\Longleftrightarrow\langle\mb E[\boldsymbol{X}|\Y],{\bs{\beta}}\rangle{\bs{\beta}}=0~~\text{a.s.} \quad\forall{\bs{\beta}}\in\mb{S}_{\mathcal H},
\end{align*}
where $\mb{S}_\mc{H}=\{{\bs{\beta}}\in\mc H:\|{\bs{\beta}}\|=1\}$. Second, from Lemma $\ref{lem: Txx=0tuiTx=0}$, one knows that
\begin{align*}
\mathrm{MDDO}(\boldsymbol{X}|\Y)({\bs{\beta}})=0&\Longleftrightarrow\langle\mathrm{MDDO}(\boldsymbol{X}|\Y)({\bs{\beta}}),{\bs{\beta}}\rangle=0.
\end{align*}
 Then under Assumption $\ref{as:joint distribution assumption}$, by Lemma $\ref{lemma, MDDO and FMDD}$ and $\ref{lem:prop1inlee}$, one has
\begin{align*}
&\langle\mathrm{MDDO}(\boldsymbol{X}|\Y)({\bs{\beta}}),{\bs{\beta}}\rangle=0\Longleftrightarrow\mathbb E[\big\langle\langle \boldsymbol{X},{\bs{\beta}}\rangle{\bs{\beta}},\langle \boldsymbol{X}',{\bs{\beta}}\rangle{\bs{\beta}}\rangle\|\Y-\Y'\|]=0\\
&\qquad\qquad\qquad\qquad\qquad\Longleftrightarrow\mathbb E[\langle \bs X,{\bs{\beta}}\rangle{\bs{\beta}}|\Y]=\langle \mb E[\boldsymbol{X}|\Y],{\bs{\beta}}\rangle{\bs{\beta}}=0~~\text{a.s.}
\end{align*}
This finishes the proof of Theorem $\ref{theorem, MDDO and conditional mean independence}$.
\end{proof}

% {\color{blue}\paragraph{Proof of Lemma \ref{lem:cov TX} (Repeated)}
% \begin{proof}
% For any $\u_1,\u_2\in\mc H_2$, we have
% \begin{align*}
% &\left\langle  T\mr{var}(\vX)T^*\u_1,\u_2  \right\rangle=\left\langle  T\mb E[\vX\otimes\vX]T^*\u_1,\u_2  \right\rangle
% =\left\langle  \mb E[\vX\otimes\vX]T^*\u_1,T^*\u_2  \right\rangle    
% \end{align*}
% since $\mb E[\vX]=0$. By the definition of convariance operator and expectation, we have 
% \begin{align*}
% \left\langle  \mb E[\vX\otimes\vX]T^*\u_1,T^*\u_2  \right\rangle=&\left\langle  \mb E[\left\langle\vX,  T^*\u_1 \right\rangle       \vX            ],T^*\u_2  \right\rangle
% =\mb E[  \left\langle\vX,  T^*\u_1 \right\rangle      \left\langle \vX            ,T^*\u_2  \right\rangle].
% \end{align*}
% Similarly, we have
% \begin{align*}
%  \left\langle  \mr{var}(T\vX)\u_1,\u_2  \right\rangle=\left\langle  \mb E[T\vX\otimes T\vX]\u_1,\u_2  \right\rangle=\mb E[  \left\langle T\vX,  \u_1 \right\rangle      \left\langle T\vX            ,\u_2  \right\rangle].\\    
% \end{align*}
% Then the proof is completed by noticing the following
% \begin{align*}
% \mb E[  \left\langle T\vX,  \u_1 \right\rangle      \left\langle T\vX            ,\u_2  \right\rangle]=\mb E[  \left\langle\vX,  T^*\u_1 \right\rangle      \left\langle \vX            ,T^*\u_2  \right\rangle].
% \end{align*}
% \end{proof}}



\section{Proof of Lemma \ref{lemma: SE=GammaS}}
Recall the following fact in FSIR.
\begin{lemma}\label{lemma, direct result of linearity condition}~\\
Under Assumption $\ref{as:Linearity condition and Coverage condition}~ \boldsymbol{\mathrm{{i)}}}$, we have $\mathcal S_{\mathbb E(\boldsymbol{X}|\Y)}\subseteq \Gamma \mc S_{\Y|\bs X}\subseteq \mc H$.
\end{lemma}
It is a trivial generalization of    \cite[Theorem 2.1]{ferre2003functional} from univariate response to multivariate response.
\paragraph{Proof of Lemma $\ref{lemma: SE=GammaS}$}
\begin{proof}
First, we prove that $\mathcal{S}_{\mathbb{E}(\bs X|\Y)}^\perp\subseteq \mathrm{Im}\{\mathrm{var(\mb{E}(\bs X|\Y))}\}^\perp$. For any ${\bs{\beta}}\in\mathcal{S}_{\mathbb{E}(\bs X|\Y)}^\perp$, one has $\langle{\bs{\beta}},\mb{E}(\bs X|\Y)\rangle=0$ a.s. Then for any $\bs\alpha\in\mathcal{H}$, one can get
\begin{align*}
\langle{\bs{\beta}},\mathrm{var}(\mb{E}(\bs X|\Y))\bs\alpha\rangle&=\langle{\bs{\beta}},\mb E\lmi\mb{E}(\bs X|\Y)\otimes \mb{E}(\bs X|\Y)\rmi\bs\alpha\rangle\\
&=\mb E\big(\langle\mb{E}(\bs X|\Y),\bs\alpha\rangle\langle{\bs{\beta}},\mb{E}(\bs X|\Y)\rangle\big)=0,
\end{align*}
which means that ${\bs{\beta}}\in\mathrm{Im}\{\mathrm{var}(\mb{E}(\bs X|\Y))\}^\perp$. Moreover, one has
\begin{align*}\mathcal{S}_{\mathbb{E}(\bs X|\Y)}^\perp\subseteq \mathrm{Im}\{\mathrm{var}(\mb{E}(\bs X|\Y))\}^\perp
%&\Rightarrow\left(\mathcal{S}_{\mathbb{E}(\bs X|Y)}^\perp\right)^\perp\supseteq \left(\mathrm{Im}\{\mathrm{var(\mb{E}(\bs X|Y))}\}^\perp\right)^\perp\\&
\Longrightarrow\overline{\mathcal{S}_{\mathbb{E}(\bs X|\Y)}}\supseteq\overline{\mathrm{Im}}\{\mathrm{var}(\mb{E}(\bs X|\Y))\}.
\end{align*}
Thus, $\overline{\mathrm{Im}}\{\mathrm{var}(\mb{E}(\bs X|\Y))\}\subseteq\overline{\mathcal{S}_{\mathbb{E}(\bs X|\Y)}}\subseteq\overline{\Gamma\mathcal{S}_{\Y|\bs X}}$ by Lemma $\ref{lemma, direct result of linearity condition}$. According to Assumption $\ref{as:Linearity condition and Coverage condition}$ \textbf{ii)}, one can get
\[\mathrm{dim}\left(\overline{\mathrm{Im}}\{\mathrm{var}(\mb{E}(\bs X|\Y))\}\right)=\mathrm{dim}\left(\overline{\mathcal{S}_{\mathbb{E}(\bs X|\Y)}}\right)=\mathrm{dim}(\overline{\Gamma\mathcal{S}_{\Y|\bs X}})=d.\]
One can complete the proof since finite dimension subspaces are closed.
\end{proof}

\section{Proof of Theorem \ref{theorem, MDDO and IRS}}
\begin{proof}
For convenience, we abbreviate $\mathrm{MDDO}(\boldsymbol{X}|\Y)$ to ${M}$. According to Theorem $\ref{theorem, MDDO and conditional mean independence}$ and Lemma $\ref{lem: Txx=0tuiTx=0}$, one can get
\begin{align*}{\bs{\beta}}\in\mathcal S_{\mb E(\boldsymbol{X}|\Y)}^\perp&\Longleftrightarrow\langle {\bs{\beta}},\mathbb E(\boldsymbol{X}|\Y)\rangle=0~~\text{a.s.}\Longleftrightarrow\mathbb E(\langle {\bs{\beta}},\boldsymbol{X}\rangle|\Y)=0~~\text{a.s.}\\
&\Longleftrightarrow\mathrm{MDDO}(\langle {\bs{\beta}},\boldsymbol{X}\rangle|\Y)=0\Longleftrightarrow\langle {M}{\bs{\beta}},{\bs{\beta}}\rangle=0\\
&\Longleftrightarrow{M}{\bs{\beta}}=0\Longleftrightarrow {\bs{\beta}}\in\mathrm{null}(M)=\overline{\mathrm{Im}}(M)^\perp,
\end{align*}
which means that $\mathcal S_{\mb E(\boldsymbol{X}|\Y)}^\perp=\overline{\mathrm{Im}}(M)^\perp$ and $\overline{\mathcal S_{\mb E(\boldsymbol{X}|\Y)}}=\overline{\mathrm{Im}}(M)$.
One can complete the proof since finite dimension subspaces are closed.
\end{proof}
\section{Proof of Lemma \ref{lemma, way of estimate truncate central subspace}}
Before proving Lemma $\ref{lemma, way of estimate truncate central subspace}$, we give the following lemma.
\begin{lemma}\label{lem: colPBP equal colPB operator}
Assume that $P$ is a bounded linear operator from a Hilbert space $\wt{\mc H}$ to itself and $B$ is a positive semi-definite operator from $\wt{\mc H}$ to itself. 
Then we have $\overline{\mathrm{Im}}(PBP^*)=\overline{\mathrm{Im}}(PB)$.
\end{lemma}
\begin{proof}
It suffices to show that $\mnull(BP^*)=\mnull(PBP^*)$. First, since $B$ is positive semi-definite, one has $\langle x,PBP^*x\rangle = \langle P^*x,BP^*x \rangle\geqslant 0~(\forall x\in\wt{\mc H})$. Thus $PBP^*$ is a positive semi-definite operator on $\wt{\H}$.
For any $y\in\wt{\H}$, we have 
\begin{align*}
PBP^*y=0\overset{(a)}{\Longleftrightarrow}\langle y,PBP^*y\rangle = \langle P^*y,BP^*y \rangle=0\overset{(b)}{\Longleftrightarrow} BP^*y=0. 
\end{align*}
where $(a)$ and $(b)$ come from Lemma $\ref{lem: Txx=0tuiTx=0}$.
Thus $\mnull(PBP^*)=\mnull(BP^*)$.
\end{proof}

\paragraph{Proof of Lemma $\ref{lemma, way of estimate truncate central subspace}$}
\begin{proof}
For convenience, we abbreviate $\mathrm{MDDO}(\boldsymbol{X}|\Y)$ and $\mathrm{MDDO}(\boldsymbol{X}^{(m)}|\Y)$ to ${M}$ and $M_m$ respectively. 

By Corollary $\ref{corollary, MDDO and central subspace}$, one can get $\Gamma\mathcal{S}_{\Y|\boldsymbol{X}}=\mathrm{Im}(M)$. Thus,
\begin{align}\label{eq: corollary, MDDO and central subspace}
\Pi_m\Gamma\mathcal{S}_{\Y|\boldsymbol{X}}=\Pi_m\mathrm{Im}(M)=\mathrm{Im}(\Pi_m M).
\end{align}
It is easy to check that
\begin{align}
\Gamma_m&:=\mathrm{var}(\bs X^{(m)})=\Pi_m\Gamma\Pi_m=\Pi_m\Gamma=\Gamma\Pi_m=\sum\limits_{i=1}^m\lambda_i\phi_i\otimes\phi_i.\label{eq: Gamma m def}
\end{align}
On the one hand, by the definition of $\mathcal{S}^{(m)}_{{\Y|\boldsymbol{X}}}$ and $\Gamma_m$ (see \eqref{def: truncated central subspace} and \eqref{eq: Gamma m def}), one can get
\begin{align}\label{eq:Pim Gamma S}
\Pi_m\Gamma\mathcal{S}_{\Y|\boldsymbol{X}}&=\Pi_m\Gamma\Pi_m\mathcal{S}_{\Y|\boldsymbol{X}}=(\Pi_m\Gamma)(\Pi_m\mathcal{S}_{\Y|\boldsymbol{X}})=\Gamma_m\mathcal{S}^{(m)}_{{\Y|\boldsymbol{X}}}.
\end{align}
On the other hand, one has $\overline{\mathrm{Im}}(\Pi_m M)=\overline{\mathrm{Im}}(\Pi_m M\Pi_m)$ by Lemma $\ref{lem: colPBP equal colPB operator}$. Since $\Pi_m M$ and $\Pi_m M\Pi_m$ are both of finite rank, one can further get
\begin{align*}
\mathrm{Im}(\Pi_m M)&=\overline{\mathrm{Im}}(\Pi_m M)=\overline{\mathrm{Im}}(\Pi_m M\Pi_m)=\mathrm{Im}(\Pi_mM\Pi_m).
\end{align*}
Then according to Theorem $\ref{theorem, MDDO and conditional mean independence}$(ii), one has
\begin{align}
\mathrm{Im}(\Pi_m M)=\mathrm{Im}(\Pi_mM\Pi_m)=\mathrm{Im}(M_m).\label{eq:Pim span M}
\end{align}
Combining \eqref{eq:Pim Gamma S}, \eqref{eq:Pim span M} with \eqref{eq: corollary, MDDO and central subspace}, one has $\Gamma_m\mathcal{S}^{(m)}_{{\Y|\boldsymbol{X}}}=\mathrm{Im}\{M_m\}$.
Finally, one can get $ \Gamma_m^\dagger\mathrm{Im}\{M_m\}=\Gamma_m^\dagger\Gamma_m\mathcal{S}^{(m)}_{{\Y|\boldsymbol{X}}}=\Pi_m\mathcal{S}^{(m)}_{{\Y|\boldsymbol{X}}}=\mathcal{S}^{(m)}_{{\Y|\boldsymbol{X}}}$.
\end{proof}

\section{Wely Inequality for a Self-adjoint and Compact Operator}\label{ap:Wely inequality for self-adjoint and compact operators}
First, we show the following three results in standard functional analysis textbook.
\begin{lemma}[Spectral theorem]\label{thm: Spectral theorem}Let $\wt{\mathcal{H}}$ be a Hilbert space and $A:\wt{\mc{H}}\to\wt{\mc{H}}$ be a compact, self-adjoint operator. There is an at most countable orthonormal basis $\{\wt e_j\}_{j\in J}$ ($J=\{1,\cdots,n\}$ or $\mathbb{Z}_{\geqslant1}$) of $\wt{\mathcal{H}}$ and eigenvalues $\{\wt\lambda_j\}_{j\in J}$ with $|\wt\lambda_1|\geqslant|\wt\lambda_2|\geqslant\cdots\geqslant0$ converging to zero, such that
\begin{align*}
x=\sum_{j\in J}\langle x,\wt e_j\rangle \wt e_j;\qquad Ax=\sum_{j\in J}\wt\lambda_j\langle x,\wt e_j\rangle \wt e_j,\qquad x \in\wt{\mathcal{H}}.
\end{align*}
\end{lemma}

\begin{lemma}[Rayleigh's principle]\label{lem:Rayleigh operator}Let $A$ be a compact, self-adjoint operator. If $\{\wt e_j\}_{j\in J}$ and $\{\wt\lambda_j\}_{j\in J}$ are eigenvectors and eigenvalues define in Lemma $\ref{thm: Spectral theorem}$ respectively. Then
\[|\wt\lambda_1|=\mathop{\sup\limits_{\|u\|=1}}|\langle Au,u\rangle|;\qquad|\wt\lambda_n|=\mathop{\sup\limits_{\|u\|=1}}_{u\in\{\wt e_1,\cdots,\wt e_{n-1}\}^\perp}|\langle Au,u\rangle|~(n\geqslant 2).\]
\end{lemma}
\begin{lemma}[Minimax theorem]\label{lem:minimax operator}
Assume that $A$ is a positive semi-definite and compact operator with its eigenvalues $\{\wt\lambda_i\}$ ordered as $\wt\lambda_1\geqslant\dots\geqslant \wt\lambda_n\geqslant\dots\geqslant 0$, then
$$
\wt\lambda_n=\inf_{E_{n-1}}\sup_{x\in E_{n-1}^\perp,\|x\|=1}\langle Ax,x\rangle
$$
where $E_{n-1}$ with dimension $n-1$ is a closed linear subspace of $\wt{\mc H}$.
\end{lemma}
Then we give the Wely inequality for a self-adjoint and compact operator.
\begin{proposition}\label{prop: wely operator}
Let $M=N+R$ where $M$, $N$ and $R$ are three self-adjoint and compact operators defined on a Hilbert space $\wt{\mc H}$. Also, $M$ and $N$ are positive semi-definite with their respective eigenvalues $\{\mu_i\},\{\nu_i\}$ ordered as follows
\begin{align*}
M:\mu_1\geqslant\dots\geqslant \mu_n\geqslant\dots\geqslant 0;\qquad
N:\nu_1\geqslant\dots\geqslant \nu_n\geqslant\dots\geqslant 0,
\end{align*}
while $R$'s eigenvalues are $\{\rho_i\}$ ordered as follows:
\[R:|\rho_1|\geqslant\dots\geqslant |\rho_n|\geqslant\dots\geqslant 0.\]
Then the following inequalities hold: $|\mu_k-\nu_k|\leqslant|\rho_1|=\|R\| $, $k\geqslant1$.
\end{proposition}
\begin{proof}
From Lemma $\ref{lem:minimax operator}$, we have:
\[\mu_n=\inf_{E_{n-1}}\sup_{x\in E_{n-1}^\perp,\|x\|=1}\langle Mx,x\rangle;\qquad\nu_n=\inf_{E_{n-1}}\sup_{x\in E_{n-1}^\perp,\|x\|=1}\langle Nx,x\rangle,\]
where $E_{n-1}$ with dimension $n-1$ is a closed linear subspace of $\wt{\mc H}$.
By Lemma $\ref{lem:Rayleigh operator}$, we have:
$$
\sup_{\|u\|=1}|\langle Ru,u\rangle|=|\rho_1|.
$$
Since $\langle Mu,u\rangle=\langle Nu,u\rangle+\langle Ru,u\rangle$, for any $\|u\|=1$, we have:
$$
\langle Nu,u\rangle-|\rho_1|\leqslant\langle Mu,u\rangle \leqslant \langle Nu,u\rangle+|\rho_1|.
$$
Then for any given $n-1$ dimensional closed linear subspace of $\wt{\mc H}$, we conclude
\begin{equation}\label{eq:max ineq}
\sup_{u\in E_{n-1}^\perp,\|u\|=1}\langle Nu,u\rangle-|\rho_1|\leqslant\sup_{u\in E_{n-1}^\perp,\|u\|=1}\langle Mu,u\rangle\leqslant \sup_{u\in E_{n-1}^\perp,\|u\|=1}\langle Nu,u\rangle+|\rho_1|.
\end{equation}
Take the infimum with respective to $E_{n-1}$ in \eqref{eq:max ineq}, we have
\[\nu_n-|\rho_1|\leqslant\mu_n\leqslant \nu_n+|\rho_1|\]
by Lemma $\ref{lem:minimax operator}$.
\end{proof}
The next result is a direct corollary of Proposition $\ref{prop: wely operator}$.
\begin{corollary}\label{coro:wely ineq operator}
Let $M$ and $N$ be two self-adjoint, positive semi-definite and compact operators defined on a Hilbert space $\wt{\mc H}$ with their respective eigenvalues $\{\mu_i\},\{\nu_i\}$ ordered as follows
\begin{align*}
M:\mu_1\geqslant\dots\geqslant \mu_n\geqslant\dots\geqslant 0\quad\text{and}\quad
N:\nu_1\geqslant\dots\geqslant \nu_n\geqslant\dots\geqslant 0.
\end{align*}
Then the following inequalities hold: $|\mu_k-\nu_k|\leqslant\|M-N\| $, $ k\geqslant1$.
\end{corollary}




\section{Proof of Proposition \ref{prop:bound hatMmd Mm}}
Before proving Proposition $\ref{prop:bound hatMmd Mm}$, we give the following conclusion, whose proof is deferred to the end of this section.
\begin{proposition}\label{proposition, concentration of MDDO}
Under Assumptions $\ref{as:joint distribution assumption}$ and $\ref{assumption: sub-Gaussian}$, for all $\gamma\in(0,1/2)$, there exist positive constants $D_0=D_0(\gamma,\sigma_0,\sigma_1)$, $D_1=D_1(\sigma_1)$, $D_2=D_2(\sigma_0,\sigma_1)$ and $n_0=n_0(\gamma,\sigma_0,\sigma_1)$ such that for all $n\geqslant n_0$ and
\[C\in \l D_0n^{\frac{2\gamma}{5}}-\ln\l D_1m^2n \r,D_2 n^{\frac{1}{5}}-\ln\l D_1m^2n \r \rmi,\]
we have
\begin{equation*}
\mathbb{P}\l\left\|\wh M_m- M_m\right\| <\l \frac{C+\ln( D_1m^2n)}{D_2}\r^{\frac52}\frac{12m}{\sqrt n}\r\geqslant 1-\exp(- C).
\end{equation*}
\end{proposition}
\paragraph{Proof of Proposition $\ref{prop:bound hatMmd Mm}$}
\begin{proof}




Using Corollary $\ref{coro:wely ineq operator}$, one can get
$
\lambda_i\l\wh M_m\r\leqslant \lno\wh M_m-M_m\rno +\lambda_i\l M_m\r
$. 
Since $\rank(M_m)=d$, one can get $\lambda_i(M_m)=0,~i\geqslant d+1$. Thus by Proposition $\ref{proposition, concentration of MDDO}$, one has
\begin{align}\label{eq:lambdai hat Mm upper bound}
\mathbb{P}\l\lambda_{d+1}(\wh M_m)<\l \frac{C+\ln\l D_1m^2n\r}{D_2}\r^{\frac52}\frac{12m}{\sqrt n}\r\geqslant 1-\exp(- C)\qquad(i\geq d+1). 
\end{align}
Notice that 
\begin{align*}\lno\wh M_m^d- M_m\rno &\leqslant\lno M_m-\wh M_m\rno +\lno\wh M_m-\wh M_m^d\rno ;\\
\lno\wh M_m-\wh M_m^d\rno &=\left\|\sum_{i=d+1}^\infty\wh\mu_i\wh\gamma_i\otimes \wh\gamma_i\right\| =\widehat{\lambda}_{d+1}=\lambda_{d+1}(\widehat{M}_m)
\end{align*}
by \eqref{wh M_m spectral decomposition}.
Then combing Proposition $\ref{proposition, concentration of MDDO}$ with \eqref{eq:lambdai hat Mm upper bound} can complete the proof.
\end{proof}


\paragraph{Proof of Proposition \ref{proposition, concentration of MDDO}}
\begin{proof}
Note that $\boldsymbol{X}^{(m)}=\sum\limits_{j=1}^m\langle \boldsymbol{X},\phi_j\rangle\phi_j$, then a simple calculation leads to
\begin{align*}
M_m&=-\sum_{i,j=1}^m\mathbb E\big[\langle \boldsymbol{X},\phi_i\rangle\langle \boldsymbol{X}',\phi_{j}\rangle\|\Y-\Y'\|\big]\phi_i\otimes\phi_j;\\
\widehat{M}_m&=-\sum_{i,j=1}^m\frac1{n^2}\sum_{k,\ell=1}^n\langle \boldsymbol{X}_k,\phi_i\rangle\langle \boldsymbol{X}_\ell,\phi_j\rangle\|\Y_k-\Y_\ell\|\phi_i\otimes\phi_j.
\end{align*}

For a operator $\Gamma'$ that can be expanded as $\Gamma':=\sum\limits_{i,j=1}^\infty a_{ij}\phi_i\otimes\phi_{j}$, let us define its maximal norm as $\|\Gamma'\|_{\mathrm{max}}=\sup\limits_{i,j}|a_{ij}|$.



\begin{lemma}\cite[Theorem 1]{mai2021slicing}\label{lemma, concentration of MDDOnm}
Under Assumptions $\ref{as:joint distribution assumption}$ and $\ref{assumption: sub-Gaussian}$, for all
$\gamma\in(0,1/2)$, there exist positive
constants $C_0=C_0(\gamma,\sigma_0,\sigma_1)$, $C_1=C_1(\sigma_1)$, $C_2 = C_2(\sigma_0;\sigma_1)$ and $n_0 = n_0(\gamma,\sigma_0,\sigma_1)$
such that for all $n\geqslant n_0$ and $\varepsilon\in(C_0 n^{-(1/2-\gamma)},1]$, we have
\begin{equation*}
\mathbb{P}\l\lno \widehat{M}_m-M_m\rno_{\max}>12\varepsilon\r\leqslant C_1 m^2n\exp\l- C_2\l\varepsilon^2 n\r^{1/5}\r.
\end{equation*}
\end{lemma}
\noindent Since $\lno\widehat{M}_m-M_m\rno \leqslant m\lno\widehat{M}_m-M_m\rno_{\mathrm{max}}$, one has
\begin{equation*}
\mathbb{P}\l\lno\widehat{M}_m-M_m\rno >12m\varepsilon\r\leqslant C_1 m^2n\exp\l-C_2\l\varepsilon^2 n\r^{1/5}\r.
\end{equation*}
Let $C=C_2\l\ve^2n\r^{1/5}-\ln\l C_1m^2n\r$ satisfying 
\begin{align*}
C\in\l C_2C_0^{2/5}n^{2\gamma/5}-\ln\l C_1m^2n\r,C_2n^{1/5}-\ln\l C_1m^2n\r\rmi,
\end{align*}
then one has
\begin{equation*}
\mathbb{P}\l\lno\widehat{M}_m-M_m\rno \leqslant\l \frac{C+\ln\l C_1m^2n\r}{C_2}\r^{\frac52}\frac{12m}{\sqrt{n}}\r>1- \exp(- C).
\end{equation*}
Then in order to complete the proof, one only need to choose $D_0$, $D_1$ and $D_2$ to be $C_2C_0^{2/5}$, $C_1$ and $C_2$ respectively. 
\end{proof}





\section{Properties of Sub-Gaussian Random Vectors}
We first review the definition of sub-Gaussian random variables.
\begin{definition}[Sub-Gaussian random variable and its upper-exponentially bounded constant]\label{def:sub gaussian variable}
A random variable $X$ is called a sub-Gaussian random variable if $X$ satisfies one of the following equivalent properties:
\begin{itemize}
 \item[1).] Tails. $\P(|X|>t)\leqslant \exp(1-t^{2}/K^{2}_{1})$ for any $t>0$;
 \item[2).] Moments. $\E[|X|^{p}]^{1/p}\leqslant K_{2}\sqrt{p}$ for any $p\geqslant 1$;
 \item[3).]Super-exponential moment: $\E[\exp(X^{2}/K^{2}_{3})]\leqslant \mr{e}$.

\noindent Moreover, if $\E[X]=0$, then the properties $1)-3)$ are also equivalent to the following one:
\item[4).] Moment generating function: $\E[\exp(tX)]\leqslant \exp(t^{2}K^{2}_{4})$ for all $t\in\R$.
\end{itemize}
Here $K_1$, $K_2$, $K_3$ and $K_4$ are four constants.
$K$ is called an upper-exponentially bounded constant of $X$ if 
$K\geqslant \max\{K_{1},K_{2},K_{3},K_{4}\}$.
\end{definition}
\begin{definition}[Sub-Gaussian random vector and its upper-exponentially bounded constant]\label{def,sub-Gaussian random vector,upper-exponentially bounded constant}
 ${X}\in\R^m$ is called a sub-Gaussian random vector if for all $x\in\R^m$, one-dimensional marginal $\langle{X},x\rangle$ is sub-Gaussian random variable. $K$ is called an upper-exponentially bounded constant of $X$ if $K$ satisfies:
 \begin{align*}
K\geqslant \sup_{x\in\mb{S}^{m-1}}K(\langle X,x\rangle) 
 \end{align*}
 where $K(\langle X,x\rangle)$ denotes an upper-exponentially bounded constant of $\langle X,x\rangle$.
Moreover, $K$ is called a uniform (about $m$) upper-exponentially bounded constant of $X$ if $K$ satisfies:
 \begin{align*}
K\geqslant \sup_m\sup_{x\in\mb{S}^{m-1}}K\l \langle X,x\rangle\r.
 \end{align*}
Furthermore, $X$ is called a uniform (about $m$) sun-Gaussian random vector.
 \end{definition}
The following is an application of sub-Gaussian random vectors.
\begin{lemma}[\citealt{vershynin2010introduction}]\label{lem:esgrm}
 Let $\M=[\bs m_1~\cdots~\bs m_n]$ be an $m\times n$ matrix ($n>m$) whose columns $\m_{i}$ are 
 independent centered sub-Gaussian random vectors with 
 covariance matrix $\mathbf{I}_{m}$. Let $\sigma^{+}_{\min}(\M)$ and $\sigma_{\max}(\M)$ be the infimum and supremum of positive singular values of $\M$ respectively. Then, for any $t>0$, with probability at least $1-2\exp(- C^{\prime}t^{2})$, we have
 \begin{equation*}
 \sqrt{n}-C_0\sqrt{m}-t\leqslant \sigma^{+}_{\min}(\M)\leqslant \sigma_{\max}(\M)\leqslant \sqrt{n}+C_0\sqrt{m}+t
 \end{equation*}
 where $C'$ and $C_0$ are two positive constants depending only on $K(\bs m_1)$:
 the upper-exponentially bounded constant of $\bs m_1$.
\end{lemma}
\noindent Let $t=\sqrt m$, then one can easily get
\begin{align}\label{equation, min max eval}
\begin{split}
\lambda_{\max}\left(\frac1n \M\M^\top\right)\leqslant \left(1+\frac{(C_0+1)\sqrt m}{\sqrt n}\right)^2;\\
\lambda_{\min}^+\left(\frac1n \M\M^\top\right)\geqslant \left(1-\frac{(C_0+1)\sqrt m}{\sqrt n}\right)^2, 
\end{split}
\end{align}
with probability at least $1-2\exp(- C'm)$ where $\lambda^{+}_{\min}(\cdot)$ and $\lambda_{\max}(\cdot)$ stands for the infimum and supremum of the positive spectrum respectively.



\begin{lemma}\label{lemma, estiamtion error of inverse sample cov}
Assume that $\x_1,\x_2,...,\x_n$ are $n$ i.i.d. samples from an $m$-dimensional centered sub-Gaussian vector with an invertible covariance matrix $\Sigma$. Let $\wh\Sigma:=\frac1n\sum_i \x_i\x_i^\top$.
Then there exists a positive constant $n_1'=n_1'(K(\bs m_1),c_1)$ ($c_1$ is defined in \eqref{eq: m n relationship}), such that when $n\geqslant n_1'$, we have
\begin{align*}
\lno\wh{\Sigma}-\Sigma\rno\hspace{-1.5mm}&\leqslant (C_0+2)^2\lambda_{\max}(\Sigma)\sqrt{\frac mn}~~\text{and}~~ \lno\wh{\Sigma}^{-1}-\Sigma^{-1}\rno\hspace{-1.5mm}\leqslant \frac{4(C_0+2)^2}{\lambda_{\min}(\Sigma)}\sqrt{\frac mn},
 \end{align*}
 with probability at least $1-2\exp(- C'm)$, where $C_0$ is defined in Lemma $\ref{lem:esgrm}$.
\end{lemma}
\begin{proof}
Let $\x_i=\Sigma^{\frac12}\m_i$ and $\bs{M}=[\bs m_1~\cdots~\bs m_n]$ where $\m_i$ is a centered sub-Gaussian random vector with covariance $\mathbf I_m$. Then one has 
\begin{align*}
\lno\wh\Sigma-\Sigma\rno&\leqslant\lno\Sigma^{\frac12}\rno\cdot\left\|\frac1n \M\M^\top-\mathbf I\right\|\cdot\lno\Sigma^{\frac12}\rno\\
&= \lambda_{\max}(\Sigma)\cdot\left[\lambda_{\max}\left(\frac1n \M\M^\top\right)-1\right]
\end{align*}
and 
\begin{align*}
\lno\wh{\Sigma}^{- 1}-\Sigma^{- 1}\rno
&\leqslant \lno\Sigma^{-\frac12}\rno\cdot\left\|\frac1n \M\M^\top-\mathbf I\right\|\cdot\lno\l\frac1n \M\M^\top\r^{-1}\rno\cdot\lno\Sigma^{-\frac12}\rno\\
&=\frac{1}{\lambda_{\min}(\Sigma)}\left[\lambda_{\max}\left(\frac1n \M\M^\top\right)-1\right]\cdot\lambda_{\min}\left(\frac1n \M\M^\top\right)^{-1}.
\end{align*}
By \eqref{equation, min max eval}, it is easy to check that
\begin{align*}&\lambda_{\max}\left(\frac1n \M\M^\top\right)-1\leqslant\left(1+\frac{(C_0+1)\sqrt m}{\sqrt n}\right)^2-1\leqslant\frac{(C_0+2)^2\sqrt m}{\sqrt n};\\
&\lambda_{\min}\left(\frac1n \M\M^\top\right)\geqslant \left(1-\frac{(C_0+1)\sqrt m}{\sqrt n}\right)^2\geqslant \frac14~\text{for}~n\geqslant [2(C_0+1)]^{\frac2{1-c_1}},
\end{align*}
with probability at least $1-2\exp(- C'm)$. Thus the proof is completed by choosing $n_1'(C_0,c_1):=[2(C_0+1)]^{\frac{2}{1-c_1}}$. 
\end{proof}

\section{Proof of Proposition \ref{prop:concentration Gammam dag Mmd}}\label{ap:concentration inequality}
We first give the following lemma whose proof is deferred to the end of this section.
\begin{lemma}\label{lem:PimTPimtoT}If $T$ is of finite rank, then we have $\lim\limits_{m\to \infty}\|\Pi_m T\Pi_m-T\| =0$.
\end{lemma}
A direct corollary of this lemma is as follows.
\begin{corollary}\label{lemma, M go to Mm}
%For any $\varepsilon>0$, one has $\|M-M_m\| <\varepsilon$ when $m$ is sufficiently large.
Under Assumptions $\ref{as:joint distribution assumption}$ and $\ref{as:Linearity condition and Coverage condition}$, we have $\lim\limits_{m\to\infty}\|M-M_m\| =0$.
\end{corollary}
\noindent We denote by $m_M(\varepsilon)$ the minimal integer $m_M$ satisfying $\|M-M_m\| \leqslant \varepsilon$ for all $m\geqslant m_M$.

Proposition $\ref{prop:concentration Gammam dag Mmd}$ is a direct corollary of the following Proposition.
\begin{proposition}
\label{prop:bound of finite estimate}
 Suppose that Assumptions $\ref{as:joint distribution assumption}$ to $\ref{assumption: rate-type condition}$ hold, then $\forall \gamma\in(0,1/2)$, there exist positive constants
 \begin{align*}
 n_1=n_1(\gamma,\sigma_0,\sigma_1,\bs K,m_M(1),c_1),\quad D_3=D_3(\|M\| ,\wt C,\bs K) 
 \end{align*}
and $C'=C'(\bs K)$
, such that when $n\geqslant n_1$, we have
\begin{equation*}
\begin{aligned}
\mb P\l \lno\widehat\Gamma_m^\dagger \widehat M_m^d-\Gamma_m^\dagger M_m\rno  \leqslant \left[\frac{C+\ln(D_1m^2n)}{D_2}\right]^{\frac52}\frac{24m^{\alpha_1+1}}{\wt C\sqrt n}+D_3\frac{m^{(2\alpha_1+1)/2}}{n^{1/2}} \r&\\
\geqslant 1-\exp(- C)-2\exp(- C'm).&
\end{aligned}
\end{equation*}
Here $D_1,D_2$ and $C$ are defined in Proposition $\ref{prop:bound hatMmd Mm}$ and $\bs K$ is the uniform upper-exponentially bounded constant of $(\sqrt{\lambda_1}w_1,\dots,\sqrt{\lambda_m}w_m)$. 
\end{proposition}
\begin{proof}
By triangle inequality, one has
\begin{align*}
&\lno\widehat{\Gamma}_m^\dagger \widehat M_m^d-\Gamma_m^\dagger M_m\rno 
=\lno\widehat\Gamma_m^\dagger \widehat M_m^d-\wh\Gamma_m^\dagger M_m+\wh\Gamma_m^\dagger M_m-\Gamma_m^\dagger M_m\rno 
\\&\qquad\leqslant\lno\Gamma_m^\dagger\rno \cdot \lno\widehat M_m^d-M_m\rno +\lno\widehat\Gamma_m^\dagger-\Gamma_m^\dagger\rno \cdot \lno M_m\rno .
\end{align*}
Thus one can bound $\lno\Gamma_m^{\dag}M_m-\widehat\Gamma_m^{\dag}\widehat M_m^d\rno $ by bound $\lno\Gamma_m^\dagger\rno $, $\lno\widehat\Gamma_m^\dagger-\Gamma_m^\dagger\rno $, $\lno\widehat M_m^d-M_m\rno $ and $\lno M_m\rno $ respectively.
\begin{itemize}
 \item\textbf{Bound of $\lno\Gamma_m^\dagger\rno $}: By Assumption $\ref{assumption: rate-type condition}$, one has 
\begin{align}\label{eq:bound Gammam dagger}
\lambda_j\geqslant \wt C j^{-\alpha_1}\Rightarrow\lno\Gamma_m^\dagger\rno =\lambda_m^{-1}\leqslant \wt{C}^{-1} m^{\alpha_1}. 
\end{align} 
 \item\textbf{Bound of $\lno\widehat\Gamma_m^\dagger-\Gamma_m^\dagger\rno $}:
 Let us define $\mc H_m:=\mathrm{span}\{\phi_1,\dots,\phi_m\}$ where $\{\phi_i\}$ is introduced in Equation $\eqref{eq:X expansion}$. It is easy to check that
$\lno\widehat\Gamma_m^\dagger-\Gamma_m^\dagger\rno =\lno(\widehat\Gamma_m^\dagger-\Gamma_m^\dagger)|_{\mc H_m}\rno $ since $\l\widehat\Gamma_m^\dagger-\Gamma_m^\dagger\r{\bs{\beta}}=0$ for any ${\bs{\beta}}\in\mc{H}_m^\perp$. 
Because $\l\widehat\Gamma_m^\dagger-\Gamma_m^\dagger\r|_{\mc H_m}$ can be represented by matrix $\widehat{\Sigma}^{-1}-\Sigma^{-1}$ defined in Lemma $\ref{lemma, estiamtion error of inverse sample cov}$ under orthonormal basis $\{\phi_i\}_{i=1}^m$, one can get $\lno\widehat\Gamma_m^\dagger-\Gamma_m^\dagger\rno =\|\widehat{\Sigma}^{-1}-\Sigma^{-1}\|$.
Similarly, one can also get $\lno\Gamma_m^\dagger\rno =\lno\Sigma^{-1}\rno=\lambda_{\min}^{-1}(\Sigma)$. Thus, by Lemma $\ref{lemma, estiamtion error of inverse sample cov}$ one has
\[\mb P\l\lno\widehat\Gamma_m^\dagger-\Gamma_m^\dagger\rno \leqslant {4(C_0+2)^2}\lno\Gamma^{\dag}_m\rno \sqrt{\frac mn}\r\geqslant 1-2\exp(- C'm)\]
for sufficiently large $n\geqslant n_1'(\bs K,c_1)$
. Combing with $\lno\Gamma_m^\dagger\rno \hspace{-1mm}\leqslant \wt{C}^{-1} m^{\alpha_1}$, one can get
\begin{equation}\label{eq: distance hat gamma m dagger hat gamma m dagger}
\mb P\l\lno\widehat\Gamma_m^\dagger-\Gamma_m^\dagger\rno \leqslant \frac{4(C_0+2)^2m^{(2\alpha_1+1)/2}}{\wt Cn^{1/2}}\r\geqslant 1-2\exp(- C'm)
\end{equation}
for sufficiently large $n\geqslant n_1'(\bs K,c_1)$.
 \item\textbf{Bound of $\lno\widehat M_m^d-M_m\rno $}:
 See Proposition $\ref{prop:bound hatMmd Mm}$.
 \item \textbf{Bound of $\lno M_m\rno $}: By Corollary $\ref{lemma, M go to Mm}$, $\|M-M_m\| \leqslant 1$ for sufficiently large $m\geqslant m_M(1)$. Then by triangle inequality, one can get
\[\|M_m\| -\|M\| \leqslant \|M-M_m\| \leqslant 1.\]
Hence,
\begin{align}\label{eq:Mm leq M C}
\|M_m\| \leqslant \|M\| +1.
\end{align}
\end{itemize}
Combing \eqref{eq:bound Gammam dagger}, \eqref{eq: distance hat gamma m dagger hat gamma m dagger}, Proposition $\ref{prop:bound hatMmd Mm}$ with \eqref{eq:Mm leq M C}, one can choose $D_3$ and $n_1$ to be $\frac{4(C_0+2)^2(\|M\| +1)}{\wt C}$ and $\max\{n_0,n_1'(\bs K,c_1),m_M(1)^{1/c_1}\}$ respectively to
complete the proof where $n_0$ is defined in Proposition $\ref{prop:bound hatMmd Mm}$.
\end{proof}

\paragraph{Proof of Lemma \ref{lem:PimTPimtoT}}
\begin{proof}By the triangle inequality and compatibility of operator norm, one has
\begin{align*}
\|\Pi_m T\Pi_m-T\| &\leqslant\|\Pi_mT\Pi_m-\Pi_mT\| +\|\Pi_mT-T\| \\
&\leqslant\|(\Pi_m-I)T^*\| +\|(\Pi_m-I)T\| 
\end{align*}
where $I=\sum\limits_{i=1}^\infty\phi_i\otimes\phi_i$ for $\{\phi_i\}_{i\in\mb{Z}_{\geqslant 1}}$ defined in \eqref{eq:X expansion} being an orthonormal basis of $\mc H$. 
% Since the adjoint of $M(\Pi_m-I)$ is $(\Pi_m-I)M$, we have
% \begin{align*}&\|M(\Pi_m-I)\| +\|(\Pi_m-I)M\| \\
% =&
% \end{align*}

Since $T$ is of finite rank, let us assume that $\{e_i\}_{i=1}^k$ is an orthonormal basis of $\mathrm{Im}(T)$ where $k=\mr{rank}(T)$. For any ${\bs{\beta}}\in\mathcal{H}$ such that $\|{\bs{\beta}}\|=1$, one has $\|T{\bs{\beta}}\|\leqslant\|T\| \|{\bs{\beta}}\|=\|T\| $, so one can assume that $T{\bs{\beta}}\in\mathrm{Im}(T)$ admits the following expansion under basis $\{e_i\}_{i=1}^k$:
\[T{\bs{\beta}}=\sum_{i=1}^k b_ie_i,\quad \sum_{i=1}^k b^2_i\leqslant\|T\| ^2<\infty.\]
Thus
\[\|(I-\Pi_m)T{\bs{\beta}}\|=\left\|\sum_{i=1}^k(I-\Pi_m) b_ie_i\right\|\leqslant\sum_{i=1}^k |b_i|\cdot\|(I-\Pi_m) e_i\|.\]
Clearly, $\|(\Pi_m-I)\alpha\|~(\forall\alpha\in\H)$ tends to $0$ as $m\to\infty$ since 
\[(I-\Pi_m)\alpha=\left(\sum_{i={m+1}}^\infty\phi_i\otimes\phi_i\right)\left(\sum\limits_{i=1}^\infty c_i\phi_i\right)=\sum_{i=m+1}^\infty c_i\phi_i\xrightarrow{m\to\infty} 0\]
where we have assumed that $\alpha=\sum\limits_{i=1}^\infty c_i\phi_i$ .

Thus $\forall\varepsilon>0$, there exists some $N_i>0$ such that $\forall m> N_i$ one has $\|(\Pi_m-I)e_i\|<\varepsilon$, $(\forall i=1,...,k)$. Let $N=\max\{N_1,\cdots,N_k\}$, then $\forall m>N$ one has
\[\|(I-\Pi_m)T{\bs{\beta}}\|\leqslant\sum_{i=1}^k |b_i|\cdot\|(I-\Pi_m) e_i\|\leqslant\sum_{i=1}^k |b_i|\varepsilon\leqslant k\varepsilon\|T\| ,\]
which means that $\forall m>N$, one has
\begin{align*}
\|(\Pi_m-I)T\| &=\sup_{\|{\bs{\beta}}\|=1}\|(\Pi_m-I)T{\bs{\beta}}\|\leqslant k\varepsilon\|T\| . 
\end{align*}
Thus $\lim\limits_{m\to\infty}\|(\Pi_m-I)T\| =0$. 

Similarly, one can also get $\lim\limits_{m\to\infty}\|(\Pi_m-I)T^*\| =0$. Then the proof of Lemma $\ref{lem:PimTPimtoT}$ is completed.
\end{proof}
% \section{Sin Theta Theorem}\label{ap:Sin Theta theorem}
% \subsection{Sin Theta Theorem for Self-adjoint Operators}
% \begin{lemma}[Proposition 2.3 in \cite{seelmann2014notes}]\label{lemma, sin theta of infinite dimension operator}
% Let $B$ be a self-adjoint operator on a separable Hilbert space $\widetilde{\mathcal{H}}$, and let ${V}\in\mathcal{L}(\widetilde{\mathcal{H}})$ be another self-adjoint operator where $\mathcal{L}\l\widetilde{\mc H}\r$ stands for the space of bounded linear operators from a Hilbert space $\widetilde{\mc H}$ to $\widetilde{\mc H}$.
% Write \[\mathrm{spec}( B)=\sigma\cup\Sigma\quad\text{and}\quad \mathrm{spec}( B+ V)=\omega\cup\Omega
% \]
% with $\sigma\cap\Sigma=\varnothing=\omega\cap\Omega$, and suppose that there is $\widehat d>0$ such that
% \[\mathrm{dist}(\sigma,\Omega)\geqslant \widehat d\quad\text{and}\quad\mathrm{dist}(\Sigma,\omega)\geqslant \wh d\]
% where $\mathrm dist(\sigma,\Sigma):=\min\{|a-b|:a\in\sigma,b\in\Omega\}$.
% Then, the operator angle $\Theta=\Theta(P_{ B}(\sigma),P_{ B+ V}(\omega))$ satisfies the bound
% \[\|\sin\Theta\|:=\|P_{{B}}(\sigma)-P_{{B}+{V}}(\omega)\| \leqslant\frac\pi2\frac{\| V\| }{\wh d}\]
% where $P_{ B}(\sigma)$ denotes the spectral projection for $ B$ associated with $\sigma$, i.e., 
% \[P_{B}(\sigma):=\frac{1}{2\pi\mathrm{i}}\oint_{\gamma}\frac{\mathrm{d}z}{z-B},\]
% where $\gamma$ is a contour on $\mathbb{C}$ that encloses $\sigma$ but no other elements of $\mathrm{spec}( B)$.
% \end{lemma}
% \begin{remark}
% We note that, 
% if further $ B$ is compact, 
% the spectral projection coincide with projection operator onto the closure of the space spanned by the eigenfunctions associated with the eigenvalues in $\sigma$.

% If $B$ is compact, by the spectral decomposition theorem one has
% \[B=\sum_{i=1}^\infty\mu_ie_i\otimes e_i\quad\text{and}\quad(z- B)^{-1}=\sum_{i=1}^\infty(z-\mu_i)^{-1}e_i\otimes e_i,\]
% where $\mr{spec}(B):=\{\mu_i\}_{i=1}^\infty$ satisfies $|\mu_i|\xrightarrow{i\to\infty} 0$.
% Then $\forall v\in \mathcal{H}$,
% \begin{align*}P_{B}(\sigma)v&=\frac{1}{2\pi\mathrm{i}}\oint_{\gamma}({z-B})^{-1}v~{\mathrm{d}z}=\frac{1}{2\pi\mathrm{i}}\oint_{\gamma}\sum_{i=1}^\infty(z-\mu_i)^{- 1}\langle e_i,v\rangle e_i~{\mathrm{d}z}\\
% &=\sum_{i=1}^\infty\left[\left(\frac{1}{2\pi\mathrm{i}}\oint_{\gamma}(z-\mu_i)^{-1}~{\mathrm{d}z}\right)\langle e_i,v\rangle e_i\right]=\sum_{i\in\{i:\mu_i\in\sigma\}}\langle e_i,v\rangle e_i.
% \end{align*}
% Especially, if $\sigma=\mr{spec}(B)\backslash\{0\}$, then $P_{B}(\sigma)$ is the projection operator onto the $\overline{\mathrm{Im}}(B)$.
% \end{remark}
% Splitting eigenvalues into nonzero part and zero part yields the following useful corollary.
% \begin{corollary}\label{cor: sin theta self adjoint}
% Let $B$ and $B'$ be two positive semi-definite {and compact} operators with finite rank on a separable Hilbert space $\widetilde{\mathcal{H}}$. Let $\lambda_{\min}^+( B)$ and $\lambda_{\min}^+(B')$ be the infimum of the positive eigenvalues of ${B}$ and ${B}'$ respectively. Then we have
% \[\left\|P_{ B}-P_{ B'}\right\| \leqslant\frac\pi2\frac{\| B- B'\| }{\min\{\lambda_{\min}^+( B),\lambda_{\min}^+( B')\}}.\]
% \end{corollary}
% \subsection{Sin Theta Theorem for General Operators}
% When ${B}$ and ${V}$ in Lemma $\ref{lemma, sin theta of infinite dimension operator}$ are not self-adjoint, we use the symmetrization trick, which mainly depends on the following Lemma.
% \begin{lemma}\label{lem:projection equality}
% $P_A=P_{AA^*}$ for any bounded linear operator $A$ from a Hilbert space $\wt\H$ to $\wt\H$. Especially, $P_A=P_{AA^{\top}}$ for any matrix $A$.
% \end{lemma}
% \begin{proof}This lemma is a direct corollary of Lemma $\ref{lem: colPBP equal colPB operator}$.
% \end{proof}

% Then we have the following Sin Theta theorem for general operator.
% \begin{lemma}\label{lemma, sin theta of nonadjoint operator}
% Let $ B,B'\in\mathcal{L}(\widetilde{\mathcal{H}})$ be two compact operators (not necessarily self-adjoint) with finite rank.
% Then we have
% \begin{align*}
% \left\|P_{ B}-P_{ B'}\right\| &\leqslant\frac\pi2\frac{\| B B^*- B'B'^*\| }{\min\lb\sigma_{\min}^+( B)^2,\sigma_{\min}^+(B')^2\rb}\\
% &\leqslant \frac\pi2\frac{\| B- B'\| ^2+2\| B- B'\| \| B'\| }{\min\lb\sigma_{\min}^+( B)^2,\sigma_{\min}^+( B')^2\rb}.
% \end{align*}
% \end{lemma}
% \begin{proof}By Lemma $\ref{lem:projection equality}$, one can get $\left\|P_{ B}-P_{ B'}\right\| =\left\|P_{ B B^*}-P_{ B' B'^*}\right\| $.
% Since $ BB^*, B'B'^*$ are both self-adjoint and compact, by Lemma $\ref{cor: sin theta self adjoint}$, one has
% \begin{align*}
% \left\|P_{ B B^*}-P_{ B' B'^*}\right\| \leqslant \frac{\pi}{2}\frac{\| B B^*- B' B'^*\| }{\min\lb\lambda_{\min}^+\l B B^*\r,\lambda_{\min}^+\l B' B'^*\r\rb}.
% \end{align*}
% Then the proof is completed in view of the following inequality:
% % of $\| B B^*- B' B'^*\| $:
% \begin{align}
% \lno B B^*- B' B'^*\rno &= \|( B- B')( B- B')^*\hspace{-0.5mm}+\hspace{-0.5mm}( B-B')(B')^*\hspace{-0.5mm}+\hspace{-0.5mm} B'( B- B')^*\| \nonumber\\
% &\leqslant \| B- B'\| ^2+2\| B- B'\| \| B'\| . \label{eq:sy ineq}
% \end{align}
% \end{proof}


\section{Sin Theta Theorem}\label{ap:Sin Theta theorem}
\subsection{Sin Theta Theorem for Self-adjoint Operators}
\begin{lemma}[Proposition 2.3 in \cite{seelmann2014notes}]\label{lemma, sin theta of infinite dimension operator}
Let $B$ be a self-adjoint operator on a separable Hilbert space $\widetilde{\mathcal{H}}$, and let ${V}\in\mathcal{L}(\widetilde{\mathcal{H}})$ be another self-adjoint operator where $\mathcal{L}\left(\widetilde{\mc H}\right)$ stands for the space of bounded linear operators from a Hilbert space $\widetilde{\mc H}$ to $\widetilde{\mc H}$.
Write the spectra of $B$ and $B+V$ as \[\mathrm{spec}( B)=\sigma\cup\Sigma\quad\text{and}\quad \mathrm{spec}( B+ V)=\omega\cup\Omega
\]
with $\sigma\cap\Sigma=\varnothing=\omega\cap\Omega$, and suppose that there is $\widehat d>0$ such that
\[\mathrm{dist}(\sigma,\Omega)\geqslant \widehat d\quad\text{and}\quad\mathrm{dist}(\Sigma,\omega)\geqslant \wh d\]
where $\mathrm dist(\sigma,\Sigma):=\min\{|a-b|:a\in\sigma,b\in\Omega\}$.
Then it holds that
\[\|P_{{B}}(\sigma)-P_{{B}+{V}}(\omega)\| \leqslant\frac\pi2\frac{\| V\| }{\wh d}\]
where $P_{ B}(\sigma)$ denotes the spectral projection for $ B$ associated with $\sigma$, i.e., 
\[P_{B}(\sigma):=\frac{1}{2\pi\mathrm{i}}\oint_{\gamma}\frac{\mathrm{d}z}{z-B},\]
where $\gamma$ is a contour on $\mathbb{C}$ that encloses $\sigma$ but no other elements of $\mathrm{spec}( B)$.
\end{lemma}
\begin{remark}
We note that, 
if further $ B$ is compact, 
the spectral projection coincide with projection operator onto the closure of the space spanned by the eigenfunctions associated with the eigenvalues in $\sigma$. 
% For more details, see, e.g., Remark 1 in \cite{chen2023optimality}.

Specifically, if $B$ is compact, by the spectral decomposition theorem one has
\[B=\sum_{i=1}^\infty\mu_ie_i\otimes e_i\quad\text{and}\quad(z- B)^{-1}=\sum_{i=1}^\infty(z-\mu_i)^{-1}e_i\otimes e_i,\]
where $\mr{spec}(B):=\{\mu_i\}_{i=1}^\infty$ satisfies $|\mu_i|\xrightarrow{i\to\infty} 0$.
Then $\forall v\in \mathcal{H}$, it holds that
\begin{align*}P_{B}(\sigma)v&=\frac{1}{2\pi\mathrm{i}}\oint_{\gamma}({z-B})^{-1}v~{\mathrm{d}z}=\frac{1}{2\pi\mathrm{i}}\oint_{\gamma}\sum_{i=1}^\infty(z-\mu_i)^{- 1}\langle e_i,v\rangle e_i~{\mathrm{d}z}\\
&=\sum_{i=1}^\infty\left[\left(\frac{1}{2\pi\mathrm{i}}\oint_{\gamma}(z-\mu_i)^{-1}~{\mathrm{d}z}\right)\langle e_i,v\rangle e_i\right]=\sum_{i\in\{i:\mu_i\in\sigma\}}\langle e_i,v\rangle e_i.
\end{align*}
In particular, if $\sigma=\mr{spec}(B)\backslash\{0\}$, then $P_{B}(\sigma)$ is the projection operator onto the $\overline{\mathrm{Im}}(B)$.
\end{remark}

Splitting eigenvalues into nonzero part and zero part yields the following useful corollary.
\begin{corollary}\label{cor: sin theta self adjoint}
Let $B$ and $B'$ be two positive semi-definite {and compact} operators with finite rank on a separable Hilbert space $\widetilde{\mathcal{H}}$. Let $\lambda_{\min}^+( B)$ and $\lambda_{\min}^+(B')$ be the infimum of the positive eigenvalues of ${B}$ and ${B}'$ respectively. Then we have
\[\left\|P_{ B}-P_{ B'}\right\| \leqslant\frac\pi2\frac{\| B- B'\| }{\min\{\lambda_{\min}^+( B),\lambda_{\min}^+( B')\}}.\]
\end{corollary}
\subsection{Sin Theta Theorem for General Operators}
When ${B}$ and ${V}$ in Lemma $\ref{lemma, sin theta of infinite dimension operator}$ are not self-adjoint, we use the symmetrization trick, which mainly depends on the following Lemma.
\begin{lemma}\label{lem:projection equality}
$P_A=P_{AA^*}$ for any bounded linear operator $A$ from a Hilbert space $\wt\H$ to $\wt\H$. Especially, $P_A=P_{AA^{\top}}$ for any matrix $A$.
\end{lemma}
\begin{proof}First we show that the null space of  $A^*$ is the same as the null space of $AA^*$.
On the one hand, 
\[x\in\mathrm{null}(A^*)\Longrightarrow
A^*x=0\Longrightarrow AA^*x=0\Longrightarrow x\in\mathrm{null}(AA^*); 
\]
One the other hand,
\begin{align*}x\in\mathrm{null}(AA^*)&\Longrightarrow
AA^*x=0\Longrightarrow \langle x,AA^*x\rangle=\langle A^*x,A^*x\rangle=\|A^*x\|^2=0\\
&\Longrightarrow A^*x=0\Longrightarrow x\in\mathrm{null}(A^*).
\end{align*}
Hence, we have $\mathrm{null}(A^*)=\mathrm{null}(AA^*)$. Take the orthogonal complement of the both sides of this equality, we can get
\[\mathrm{null}(A^*)^{\perp}=\mathrm{null}(AA^*)^{\perp}\Longrightarrow {\mathrm{Im}(A)}={\mathrm{Im}(AA^*)}.\]
\end{proof}
Then we have the following Sin Theta theorem for general operator.
\begin{lemma}\label{lemma, sin theta of nonadjoint operator}
Let $ B,B'\in\mathcal{L}(\widetilde{\mathcal{H}})$ be two compact operators (not necessarily self-adjoint) with finite rank.
Then we have
\begin{align*}
\left\|P_{ B}-P_{ B'}\right\| &\leqslant\frac\pi2\frac{\| B B^*- B'B'^*\| }{\min\left\{\sigma_{\min}^+( B)^2,\sigma_{\min}^+(B')^2\right\}}\\
&\leqslant \frac\pi2\frac{\| B- B'\| ^2+2\| B- B'\| \| B'\| }{\min\left\{\sigma_{\min}^+( B)^2,\sigma_{\min}^+( B')^2\right\}}.
\end{align*}
\end{lemma}
\begin{proof}By Lemma $\ref{lem:projection equality}$, one can get $\left\|P_{ B}-P_{ B'}\right\| =\left\|P_{ B B^*}-P_{ B' B'^*}\right\| $.
Since $ BB^*, B'B'^*$ are both self-adjoint and compact, by Lemma $\ref{cor: sin theta self adjoint}$, one has
\begin{align*}
\left\|P_{ B B^*}-P_{ B' B'^*}\right\| \leqslant \frac{\pi}{2}\frac{\| B B^*- B' B'^*\| }{\min\left\{\lambda_{\min}^+\left( B B^*\right),\lambda_{\min}^+\left( B' B'^*\right)\right\}}.
\end{align*}
Then the proof is completed in view of the following inequality:
% of $\| B B^*- B' B'^*\| $:
\begin{align}
\left\| B B^*- B' B'^*\right\| &= \|( B- B')( B- B')^*\hspace{-0.5mm}+\hspace{-0.5mm}( B-B')(B')^*\hspace{-0.5mm}+\hspace{-0.5mm} B'( B- B')^*\| \nonumber\\
&\leqslant \| B- B'\| ^2+2\| B- B'\| \| B'\| . \label{eq:sy ineq}
\end{align}
\end{proof}


\section{Proof of Theorem \ref{theorem, total convergence rate}}
Thanks to the triangle inequality, one can bound the subspace estimation error by bounding the error term (i): $\mathbf{ Loss}_1:=\left\|P_{\mc S_{\Y|\X}^{(m)}}-P_{ \widehat {\mc S}_{\Y|\X}^{(m)}}\right\| $ and error term (ii): $\mathbf{ Loss}_2:= \left\|P_{\mathcal S_{\Y|\boldsymbol{X}}}-P_{\mathcal S_{\Y|\boldsymbol{X}}^{(m)}}\right\| $ respectively.
\subsection{Upper bound of error term (i)}
We first give the following lemmas, whose proofs are all deferred to the end of this section.
\begin{lemma}\label{lem:Gammam dagger Mm uniformly bounded}
% Under Assumptions $\ref{as:joint distribution assumption}$ and $\ref{as:Linearity condition and Coverage condition}$,
% $\{\|\Gamma_m^\dagger M_m\| \}_{m=1}^\infty$ is uniformly (about $m$) bounded by $\|\Gamma^{-1}M\| $.
Under Assumptions $\ref{as:joint distribution assumption}$ and $\ref{as:Linearity condition and Coverage condition}$, it holds that $\|\Gamma_m^\dagger M_m\| \leq \|\Gamma^{-1}M\| (\forall m).$
% \begin{align*}
% \|\Gamma_m^\dagger M_m\| \leq \|\Gamma^{-1}M\| \quad\forall m.
% \end{align*}
% $\{\|\Gamma_m^\dagger M_m\| \}_{m=1}^\infty$ is uniformly (about $m$) bounded by $\|\Gamma^{-1}M\| $.
\end{lemma}
\begin{lemma}\label{lem: Gamma inverse M to Gammam dagger Mm}Under Assumptions $\ref{as:joint distribution assumption}$ and $\ref{as:Linearity condition and Coverage condition}$, we have \[\lim\limits_{m\to\infty}\lno\Gamma^{-1}M-\Gamma_m^\dagger M_m\rno =0.\]
\end{lemma}
\noindent We denote by $m_T(\varepsilon)$ the minimal integer $m_T$ satisfying $\lno\Gamma^{- 1}M-\Gamma_m^\dagger M_m\rno \hspace{-1mm}\leqslant \varepsilon$ for all $m\geqslant m_T$ and define an event 
$$\ttE:=\lb \left\|\widehat\Gamma_m^\dagger \widehat M_m^d-\Gamma_m^\dagger M_m\right\|  \leqslant\hspace{-0.5mm}\left(\tfrac{D_0+1}{D_2}\right)^{\frac52}\tfrac{24}{\wt C}n^{c_1(\alpha_1+1)+\gamma-\frac{1}{2}}+D_3n^{\frac{c_1(2\alpha_1+1)-1}{2}}\rb.$$
Then by taking $C$ to be $(D_0+1)n^{\frac{2\gamma}{5}}-\ln\l D_1m^2n \r$ in  Proposition \ref{prop:bound of finite estimate}, one has: for $n\geqslant \l\frac{D_0+1}{D_2}\r^{\frac{5}{1-2\gamma}}$,
$$\P(\ttE)\geq 1-D_1m^2n\exp\left[-(D_0+1)n^{\frac{2\gamma}{5}}\right] -2\exp(- C'm).$$
\begin{lemma}\label{lem:lower bound sigma min total}
Introducing $
\bigtriangleup :=\max\lb \frac{\sigma_d(\Gamma^{-1} M)}{2},\frac{\sigma_d(\Gamma^{-1} M)^2}{4\|\Gamma^{-1}M\| } \rb$.
Suppose that Assumptions $\ref{as:joint distribution assumption}$ to $\ref{assumption: rate-type condition}$ hold, $c_1(2\alpha_1+1)-1<0$ and $2(c_1(\alpha_1+1)+\gamma)-1<0$. Then there exists a positive constant
\begin{align*}
n_2'=n_2'\l\sigma_d(\Gamma^{-1}M),\|\Gamma^{-1}M\| , \gamma,\sigma_0,\sigma_1,\bs K,m_M(1),c_1,m_T\l \tfrac{\bigtriangleup}{2}\r,\wt C,\alpha_1\r
\end{align*}
such that when $n\geqslant n_2'$, we have
\begin{align}
\sigma_{\min}^+(\Gamma_m^{\dagger} M_m)^2\geqslant \tfrac{\sigma_d(\Gamma^{-1}M)^2}{2} \label{eq: lower bound of sigma min}. 
\end{align}
Furthermore, Conditioning on $\ttE$, we have
\begin{align}\label{eq: lower bound of sigma min hat}
&\sigma_{\min}^+(\wh\Gamma_m^{\dagger}\wh M_m^d)^2\geqslant \tfrac{\sigma_d(\Gamma^{-1}M)^2}{2}.
\end{align}
\end{lemma}
The following proposition is an upper bound of error term (i):
\begin{proposition}\label{proposition, estimation error}
Positive constants $D_1$, $D_2$ and $C'$  as in Proposition $\ref{prop:bound of finite estimate}$,
suppose that Assumptions $\ref{as:joint distribution assumption}$ to $\ref{assumption: rate-type condition}$ hold, then $\forall \gamma\in(0,1/2)$, if $c_1$ satisfies $2c_1(\alpha_1+1)+2\gamma-1<0$ and $c_1(2\alpha_1+1)-1<0$, there exists a positive constant $C_1:=C_1\l \|\Gamma^{-1}M\| ,\sigma_d(\Gamma^{-1}M) ,\wt C,\gamma,\sigma_0,\sigma_1\r$ such that
\begin{align*}
\P\l
\lno P_{\mc{S}_{\Y|\X}^{(m)}}-P_{ \widehat{\mc{S}}_{\Y|\X}^{(m)}}\rno \leqslant C_1\frac{m^{\alpha_1+1}}{n^{1/2-\gamma}}\r\geqslant1-2\exp(- C'm)&\\
- D_1m^2n\exp\l -(D_0+1)n^{\frac{2\gamma}{5}} \r&,
\end{align*}
when 
\begin{align*}
n\geqslant\max\Bigg\{ n_1,\l\tfrac{D_0+1}{D_2}\r^{\frac{5}{1-2\gamma}},\left[\tfrac{\|\Gamma^{-1}M\|  \wt C}{48}\l\tfrac{D_2}{D_0+1}\r^{\frac52}\right]^{\frac{2}{2(c_1(\alpha_1+1)+\gamma)-1}}&,\\
\l \tfrac{\|\Gamma^{-1}M\| }{2D_3}\r^{\frac{2}{c_1(2\alpha_1+1)-1}},n_2',\left[ \tfrac{D_3\wt C}{24}\l \tfrac{D_2}{D_0+1} \r^{\frac52} \right]^{\frac2{2\gamma+c_1}}&\Bigg\}
\end{align*}
where $n_2'$ is defined in Lemma $\ref{lem:lower bound sigma min total}$.
\end{proposition}
\begin{proof}
By Lemma $\ref{lemma, way of estimate truncate central subspace}$, $\eqref{def: estimator central subspace}$ and Lemma $\ref{lemma, sin theta of nonadjoint operator}$, one has
\begin{align}
&\left\|P_{\mc S_{\Y|\vX}^{(m)}}-P_{\wh{\mc{S}}_{\Y|\vX}^{(m)}}\right\| =\left\|P_{\Gamma_m^{\dagger}M_m}-P_{\wh\Gamma_m^{\dagger}\wh M_m^d}\right\| \nonumber\\
&\qquad\leqslant\frac{\pi}{2}\frac{\lno\widehat\Gamma_m^\dagger \widehat M_m^d-\Gamma_m^\dagger M_m\rno ^2+\lno\widehat\Gamma_m^\dagger \widehat M_m^d-\Gamma_m^\dagger M_m\rno \lno\Gamma_m^\dagger M_m\rno }{\min\lb\sigma_{\min}^+\l\wh\Gamma_m^\dagger \wh M_m^d\r^2,\sigma_{\min}^+\l\Gamma_m^\dagger M_m\r^2\rb}\label{eq: PS minus P hat S norm}.
% &\leqslant C_5\|\widehat\Gamma_m^\dagger \widehat M_m^d-\Gamma_m^\dagger M_m\|\\
% &=\widetilde O_{\mathbb{P}}\l\frac{m^{\alpha_1+1}}{n^{1/2}}\r,
\end{align}
% with probability at least $1-\exp(- C)-2\exp(- C'm)$.
Because of $c_1(2\alpha_1+1)-1<0$ and $2(c_1(\alpha_1+1)+\gamma)-1<0$, it is easy to check that when
\[n\geqslant\max\lb\left[\tfrac{\|\Gamma^{-1}M\|  \wt C}{48}\l\tfrac{D_2}{D_0+1}\r^{\frac52}\right]^{\frac{2}{2(c_1(\alpha_1+1)+\gamma)-1}},\l \tfrac{\|\Gamma^{-1}M\| }{2D_3}\r^{\frac{2}{c_1(2\alpha_1+1)-1}}\rb,\]
both $\l\tfrac{D_0+1}{D_2}\r^{\frac52}\tfrac{24}{\wt C}n^{c_1(\alpha_1+1)+\gamma-\frac{1}{2}}$ and $D_3n^{\frac{c_1(2\alpha_1+1)-1}{2}}$ are less than or equal to $\frac{\|\Gamma^{-1}M\| }{2}$. Thus, on the event $\ttE$,
\begin{align}\label{eq: high prob upper bound is Gamma minus 1 M}
\lno\widehat\Gamma_m^\dagger \widehat M_m^d-\Gamma_m^\dagger M_m\rno \leqslant \lno\Gamma^{-1}M\rno .
\end{align}
By Lemma $\ref{lem:Gammam dagger Mm uniformly bounded}$, inserting \eqref{eq: high prob upper bound is Gamma minus 1 M} into \eqref{eq: PS minus P hat S norm} leads to
$$
\lno P_{\mc{S}_{\Y|\X}^{(m)}}-P_{ \widehat{\mc{S}}_{\Y|\X}^{(m)}}\rno
\leqslant \frac{\pi\lno\widehat\Gamma_m^\dagger \widehat M_m^d-\Gamma_m^\dagger M_m\rno \lno\Gamma^{-1}M\rno }{\min\lb\sigma_{\min}^+\l\wh\Gamma_m^\dagger \wh M_m^d\r^2,\sigma_{\min}^+\l\Gamma_m^\dagger M_m\r^2\rb},
$$
on the event $\ttE$.
Furthermore, when $n\geqslant \left[ \frac{D_3\wt C}{24}\l \frac{D_2}{D_0+1} \r^{\frac52} \right]^{\frac2{2\gamma+c_1}}$ and $n\geq n_2'$, one can get
$\l \tfrac{D_0+1}{D_2}\r^{\frac52}\tfrac{24m^{\alpha_1+1}}{\wt C n^{1/2-\gamma}}$ is greater than or equal to $D_3\tfrac{m^{(2\alpha_1+1)/2}}{n^{1/2}}$
and then on the event $\ttE$,
\begin{align*}
\lno P_{\mc{S}_{\Y|\X}^{(m)}}-P_{ \widehat{\mc{S}}_{\Y|\X}^{(m)}}\rno \leqslant \tfrac{96\pi\|\Gamma^{-1}M\| }{\sigma_d(\Gamma^{-1}M)^2}\l \tfrac{D_0+1}{D_2}\r^{\frac52}\tfrac{m^{\alpha_1+1}}{\wt C n^{1/2-\gamma}}.
\end{align*}
 by
Lemma $\ref{lem:lower bound sigma min total}$.
Then choosing $C_1=\tfrac{96\pi\|\Gamma^{-1}M\| }{\wt C\sigma_d(\Gamma^{-1}M)^2}\l \tfrac{D_0+1}{D_2}\r^{\frac52}$ can complete the proof.
\end{proof}



\paragraph{Proof of Lemma \ref{lem:Gammam dagger Mm uniformly bounded}}
\begin{proof}
First, it is easy to check that:
\begin{align}
\Gamma^\dag_m=\Pi_m\Gamma^{-1}\Pi_m=\Pi_m\Gamma^{-1}=\Gamma^{-1}\Pi_m=\sum\limits_{i=1}^m\lambda_i^{-1}\phi_i\otimes\phi_i.\label{eq: Gamma m dag def}
\end{align}
According to \eqref{eq: Gamma m dag def} and $M_m=\Pi_mM\Pi_m$, it is easy to check that $\Gamma_m^\dagger M_m=\Pi_m \Gamma^{- 1}M\Pi_m$. Then by the compatibility of operator norm, one can get
\begin{align*}
\lno\Gamma_m^\dagger M_m\rno =\lno\Pi_m \Gamma^{-1}M\Pi_m\rno \leqslant \lno\Pi_m\rno  \lno\Gamma^{-1}M\rno \lno\Pi_m\rno =\lno\Gamma^{-1}M\rno .
\end{align*}
Note that $\Gamma^{-1}M$ is bounded since $\Gamma^{-1}M$ is of finite rank by Corollary $\ref{corollary, MDDO and central subspace}$. Thus the proof is completed. 
\end{proof}


\paragraph{Proof of Lemma \ref{lem: Gamma inverse M to Gammam dagger Mm}}
\begin{proof}
It is easy to check that
$\Gamma_m^\dagger M_m=\Pi_m\Gamma^{-1}M\Pi_m$ and $\Gamma^{-1}M$ is of finite rank by Corollary $\ref{corollary, MDDO and central subspace}$.
Thus the proof is completed by Lemma $\ref{lem:PimTPimtoT}$.
\end{proof}
\paragraph{Proof of Lemma \ref{lem:lower bound sigma min total}}
\begin{proof}
We first prove \eqref{eq: lower bound of sigma min}.
By Corollary $\ref{corollary, MDDO and central subspace}$ and Lemma $\ref{lem:projection equality}$, one has $\rank(\Gamma^{- 1}M)=\rank\l\Gamma^{- 1}M(\Gamma^{- 1}M)^*\r=d$. Thus
\begin{align*}
\sigma_{\min}^+(\Gamma^{-1}M)^2=\lambda_{\min}^+\l\Gamma^{-1}M(\Gamma^{-1}M)^*\r=\lambda_d\l \Gamma^{-1}M(\Gamma^{-1}M)^*\r. 
\end{align*}
 It is easy to see $\rank(\Gamma_m^\dagger M_m)=\rank\l \Gamma_m^\dagger M_m(\Gamma_m^\dagger M_m)^*\r\leqslant d$ by $\Gamma_m^\dagger M_m=\Pi_m \Gamma^{-1} M \Pi_m$ and Lemma $\ref{lem:projection equality}$, thus one can assume that 
 \begin{align*}
\sigma_{\min}^+(\Gamma^\dagger_m M_m)^2=\lambda_{\min}^+\l\Gamma_m^\dagger M_m(\Gamma_m^\dagger M_m)^*\r=\lambda_j\l \Gamma_m^\dagger M_m(\Gamma_m^\dagger M_m)^*\r
\end{align*}
for some $j\leqslant d$.
By Corollary $\ref{coro:wely ineq operator}$, $\eqref{eq:sy ineq}$ and
% (Notice that $M_m$ and $M$ are both compact and self-adjoint)
Lemma $\ref{lem: Gamma inverse M to Gammam dagger Mm}$
%and Lemma \ref{lem:Gammam dagger Mm uniformly bounded}
, one has
\begin{align*}
&\left|\sigma_{\min}^+(\Gamma^\dagger_m M_m)^2\hspace{-0.5mm}-\hspace{-0.5mm}\sigma_j(\Gamma^{-1} M)^2\right|\hspace{-0.5mm}=\hspace{-0.5mm}\left|\lambda_{j}\hspace{-1mm}\l\Gamma^\dagger_m M_m(\Gamma^\dagger_m M_m)^{*}\hspace{-0.5mm}\r\hspace{-0.5mm}-\hspace{-0.5mm}\lambda_j\hspace{-1mm}\l \Gamma^{-1} M(\Gamma^{-1} M)^*\hspace{-0.5mm}\r\right|\\
&\qquad\leqslant
\|\Gamma^{-1} M(\Gamma^{-1} M)^*- \Gamma_m^\dagger M_m(\Gamma_m^\dagger M_m)^*\| \\
&\qquad\leqslant \|\Gamma^{-1} M- \Gamma_m^\dagger M_m\| ^2+
\|\Gamma^{-1} M- \Gamma_m^\dagger M_m\| \cdot\|\Gamma^{-1} M\| \xrightarrow{m\to\infty} 0. 
% &\leqslant\|\Gamma^{-1} M- \Gamma_m^\dagger M_m\|\cdot3\|\Gamma^{-1} M\|
\end{align*}
Thus for 
$
n\geqslant m_T(\bigtriangleup)^{\frac1{c_1}}=m_T\l\max\lb\frac{\sigma_d(\Gamma^{-1} M)}{2},\frac{\sigma_d(\Gamma^{-1} M)^2}{4\|\Gamma^{-1}M\| }\rb\r^{\frac1{c_1}}, 
$
one has $\|\Gamma^{-1} M- \Gamma_m^\dagger M_m\| ^2$ and $\|\Gamma^{-1} M- \Gamma_m^\dagger M_m\| \cdot\|\Gamma^{-1} M\| $ are both less than or equal to $\frac{1}{4}\sigma_d(\Gamma^{-1} M)^2$. Hence one can get
$\left|\sigma_{\min}^+(\Gamma^\dagger_m M_m)^2-\sigma_j(\Gamma^{-1} M)^2\right|\leqslant\frac{1}{2}\sigma_d(\Gamma^{-1} M)^2$
% \begin{align*}\label{eq:sigma min Mm}
% \left|\sigma_{\min}^+(\Gamma^\dagger_m M_m)^2-\sigma_j(\Gamma^{-1} M)^2\right|\leqslant\frac{1}{2}\sigma_d(\Gamma^{-1} M)^2
% \|
% \lambda_j\l \Gamma_m^\dagger M_m\l\Gamma_m^\dagger M_m\r^*\r\geqslant \lambda_j\l \Gamma^{-1} M\l\Gamma^{-1} M\r^*\r-\frac{\lambda_d\l \Gamma^{-1} M\l\Gamma^{-1} M\r^*\r}{2}
% \geqslant\frac{\lambda_d\l \Gamma^{-1} M\l\Gamma^{-1} M\r^*\r}{2}. 
% \end{align*}
and
\begin{equation}
\sigma_{\min}^+(\Gamma^\dagger M_m)^2\geqslant \sigma_j(\Gamma^{-1} M)^2-\frac{1}{2}\sigma_d(\Gamma^{-1} M)^2\geqslant\frac{1}{2}\sigma_d(\Gamma^{-1} M)^2
\end{equation}
for sufficiently large $n$. This completes the proof of \eqref{eq: lower bound of sigma min}.





Next we prove $\eqref{eq: lower bound of sigma min hat}$. Combining Proposition $\ref{prop:bound of finite estimate}$ with Lemma $\ref{lem: Gamma inverse M to Gammam dagger Mm}$ leads to that on the event $\ttE$, 
$$
\lno\wh \Gamma_m^\dag\wh M^d_m- \Gamma^{-1}M\rno \leqslant\ve+\l\tfrac{D_0+1}{D_2}\r^{\frac52}\tfrac{24}{\wt C}n^{c_1(\alpha_1+1)+\gamma-\frac{1}{2}}+D_3n^{\frac{c_1(2\alpha_1+1)-1}{2}}
$$
for  $n\geqslant \max\{n_1,m_T( \ve)^{1/c_1}\}$.
Assuming that $c_1(2\alpha_1+1)-1<0$ and $2(c_1(\alpha_1+1)+\gamma)-1<0$, it is easy to check that when $$n\geqslant\max\lb\left[\frac{\bigtriangleup \wt C}{96}\l\frac{D_2}{D_0+1}\r^{\frac52}\right]^{\frac{2}{2(c_1(\alpha_1+1)+\gamma)-1}},\l \frac{\bigtriangleup}{4D_3}\r^{\frac{2}{c_1(2\alpha_1+1)-1}}\rb$$, both $\l\tfrac{D_0+1}{D_2}\r^{\frac52}\tfrac{24}{\wt C}n^{c_1(\alpha_1+1)+\gamma-\frac{1}{2}}$ and $D_3n^{\frac{c_1(2\alpha_1+1)-1}{2}}$ are less than or equal to $\frac{\bigtriangleup}{4}$. Letting $\varepsilon=\frac12\bigtriangleup$, one can get on the event $\ttE$,
when
\begin{align*}
n&\geqslant n_2'=n_2'\hspace{-0.5mm}\l\hspace{-0.5mm}\sigma_d(\Gamma^{-1}M),\|\Gamma^{-1}M\| , \gamma,\sigma_0,\sigma_1,\bs K,m_M(1),c_1,m_T\l \tfrac{\bigtriangleup}{2}\r,\wt C,\alpha_1\hspace{-0.5mm}\r\\
&:=\max\bigg\{ n_1,m_T\l \tfrac{\bigtriangleup}{2}\r^{1/c_1}, \left[\tfrac{\bigtriangleup \wt C}{96}\l\tfrac{D_2}{D_0+1}\r^{\frac52}\right]^{\frac{2}{2(c_1(\alpha_1+1)+\gamma)-1}},\l \tfrac{\bigtriangleup}{4D_3}\r^{\frac{2}{c_1(2\alpha_1+1)-1}}\bigg\},
\end{align*}
one has $\lno\wh \Gamma_m^\dag\wh M^d_m- \Gamma^{-1}M\rno \leqslant\bigtriangleup$ and further
$\sigma_{\min}^+(\wh\Gamma^\dagger \wh M^d_m)^2\hspace{-1mm}\geqslant\hspace{-1mm} \tfrac{\sigma_d(\Gamma^{-1} M)^2}{2}$ by the same argument as the proof of \eqref{eq: lower bound of sigma min}.
 This completes the proof of \eqref{eq: lower bound of sigma min hat}.
Considering that $m_T(\bigtriangleup)\leqslant m_{T}\l\frac\bigtriangleup2\r$, one can also get $\eqref{eq: lower bound of sigma min}$ when $n\geqslant n_2'$. Thus the proof is completed.
\end{proof}
\subsection{Upper bound of error term (ii)}\label{ap, subs, truncation error}
\begin{proposition}\label{proposition, truncation error}
Under Assumption $\ref{assumption: rate-type condition}$, there exists a positive constant $C_2:=C_2\l d,\wt C,\lambda_d(\mc{B}),\alpha_2\r$ where $\mc{B}:=\sum\limits_{i=1}^d {\bs{\beta}}_i\otimes{\bs{\beta}}_i$ for ${\bs{\beta}}_i$ defined in \eqref{def: central subspace}, such that when $n\geqslant \l \frac{\lambda_d({\mc{B}})}{4d\wt C^2}\sqrt{\frac{2\alpha_2-1}{\zeta(2\alpha_2)}}\r^{\frac{2}{c_1(1-2\alpha_2)}}$, we have
\begin{equation}\label{equation, truncation error}
 \left\|P_{\mathcal S_{\Y|\boldsymbol{X}}}-P_{\mathcal S_{\Y|\boldsymbol{X}}^{(m)}}\right\| \leqslant C_2m^{-\frac{2\alpha_2-1}{2}},
\end{equation}
where $\zeta(\cdot)$ is Riemann $\zeta$ function.
% \begin{equation}
% \|P_{\mathcal S_{Y|\boldsymbol{X}}}-P_{\mathcal S_{Y|\boldsymbol{X}}^{(m)}}\|\leqslant O_{\mathbb{P}}(dn^{-(\alpha_2-1)/(2\alpha_1+\alpha_2)}) 
% \end{equation}
\end{proposition}
\begin{proof}
Let ${\mc{B}^{(m)}}:=\sum\limits_{i=1}^d {\bs{\beta}}_i^{(m)}\otimes{\bs{\beta}}_i^{(m)}$ for ${\bs{\beta}}_i^{(m)}$ defined in \eqref{def: truncated central subspace}.
Combing with Equation $\eqref{def: central subspace}$, it is easy to check that $\left\|P_{\mathcal S_{\Y|\boldsymbol{X}}}-P_{\mathcal S_{\Y|\boldsymbol{X}}^{(m)}}\right\| =\|P_{\mc{B}}-P_{\mc{B}^{(m)}}\| $. By Corollary $\ref{cor: sin theta self adjoint}$, we have
\begin{align}\label{eq:sin theta for B Bm}
\|P_{\mc{B}}-P_{\mc{B}^{(m)}}\| \leqslant \frac{\pi}{2}\frac{\|{\mc{B}}-{\mc{B}^{(m)}}\| }{\min\{\lambda_{\min}^+({\mc{B}}),\lambda_{\min}^+({\mc{B}^{(m)}})\}}.
\end{align}

Note that ${\mc{B}}-{\mc{B}^{(m)}}$ is self-adjoint, then
\begin{align*}
&\lno{\mc{B}}-{\mc{B}^{(m)}}\rno =\sup_{{\bs{\beta}}\in\mathbb{S}_{ \mathcal H}}|\langle ({\mc{B}}-{\mc{B}^{(m)}})({\bs{\beta}}),{\bs{\beta}}\rangle|=\sup_{{\bs{\beta}}\in\mathbb{S}_{\mathcal H}}|\langle {\mc{B}}{\bs{\beta}},{\bs{\beta}}\rangle-\langle {\mc{B}^{(m)}}{\bs{\beta}},{\bs{\beta}}\rangle|\\
&~~=\sup_{{\bs{\beta}}\in\mathbb{S}_{\mathcal H}}\hspace{-0.9mm}\left|\sum_{i=1}^d\hspace{-0.9mm}\left[\langle{\bs{\beta}}_i,{\bs{\beta}}\rangle^2-\langle{\bs{\beta}}_i^{(m)},{\bs{\beta}}\rangle^2\right]\right|=\sup_{{\bs{\beta}}\in\mathbb{S}_{\mathcal H}}\hspace{-0.9mm}\left| \sum_{i=1}^d\langle{\bs{\beta}}_i-{\bs{\beta}}_i^{(m)},{\bs{\beta}}\rangle\langle{\bs{\beta}}_i+{\bs{\beta}}_i^{(m)},{\bs{\beta}}\rangle\right|\\
&~~\leqslant\sup_{{\bs{\beta}}\in\mathbb{S}_{\mathcal H}}\sum_{i=1}^d\left| \langle{\bs{\beta}}_i-{\bs{\beta}}_i^{(m)},{\bs{\beta}}\rangle\langle{\bs{\beta}}_i+{\bs{\beta}}_i^{(m)},{\bs{\beta}}\rangle\right|
\leqslant\sum_{i=1}^d\left\|{\bs{\beta}}_i-{\bs{\beta}}_i^{(m)}\right\|\left\|{\bs{\beta}}_i+{\bs{\beta}}_i^{(m)}\right\|,
\end{align*}
where the first inequality comes from the triangle inequality, and the 
second inequality comes from the Cauchy-Schwarz inequality and $\|{\bs{\beta}}\|=1$. 
 Then one has ${\bs{\beta}}_i=\sum\limits_{j=1}^\infty b_{ij}\phi_j$ and 
\[{\bs{\beta}}^{(m)}_i=\Pi_m{\bs{\beta}}_i=\sum_{j'=1}^m\phi_{j'}\otimes\phi_{j'}\sum_{j=1}^\infty b_{ij}\phi_j=\sum_{j'=1}^m\sum_{j=1}^\infty\langle\phi_{j'},\phi_j\rangle b_{ij}\phi_{j'}=\sum_{j=1}^mb_{ij}\phi_j.\]
According to Assumption $\ref{assumption: rate-type condition}$, one can get
\begin{align*}
\left\|{\bs{\beta}}_i-{\bs{\beta}}_i^{(m)}\right\|&=\left\|\sum_{j=m+1}^\infty b_{ij}\phi_j\right\|=\sqrt{\sum_{j=m+1}^\infty b_{ij}^2}\leqslant \wt C\sqrt{\sum_{j=m+1}^\infty j^{-2\alpha_2}};\\
\left\|{\bs{\beta}}_i+{\bs{\beta}}_i^{(m)}\right\|&\leqslant\|{\bs{\beta}}_i\|+\lno{\bs{\beta}}_i^{(m)}\rno\leqslant2\|{\bs{\beta}}_i\|=2\sqrt{\sum_{j=1}^\infty b_{ij}^2}\leqslant 2\wt C\sqrt{\sum_{j=1}^\infty j^{- 2\alpha_2}}.
\end{align*}
Because $\alpha_2>1/2$, one has
\[\sum\limits_{j=m+1}^\infty \frac{1}{j^{2\alpha_2}}\leqslant \frac{1}{2\alpha_2-1}\frac{1}{m^{2\alpha_2-1}};\qquad \sum_{j=1}^\infty \frac 1{j^{2\alpha_2}}=\zeta(2\alpha_2)\text{ is convergent},\]
where $\zeta(\cdot)$ is Riemann $\zeta$ function. Thus, one can get
\begin{equation}\label{eq: upper bound of operator norm of A minus B}
\lno{\mc{B}}-{\mc{B}^{(m)}}\rno \leqslant 2d\wt C^2\sqrt{\frac{\zeta(2\alpha_2)}{2\alpha_2-1}}m^{-\frac{2\alpha_2-1}{2}}.
\end{equation}

Furthermore, 
{since $\mr{rank}(\mc{B})=d$, one can get that $\lambda_{\min}^+(\mc{B})=\lambda_{d}(\mc{B})$. It is easy to see $\rank(\mc{B}^{(m)})\leqslant d$ by $\mc{B}^{(m)}=\Pi_m \mc{B} \Pi_m$, thus one can assume that $\lambda_{\min}^+(\mc{B}^{(m)})=\lambda_j( \mc{B}^{(m)})$ for some $j\leqslant d$.
By Corollary $\ref{coro:wely ineq operator}$
% (Notice that $M_m$ and $M$ are both compact and self-adjoint)
and \eqref{eq: upper bound of operator norm of A minus B}, one has:
$$
|\lambda_j( \mc{B}^{(m)})-\lambda_j\l \mc{B}\r|\leqslant\lno \mc{B}-\mc{B}^{(m)}\rno \leqslant 2d\wt C^2\sqrt{\frac{\zeta(2\alpha_2)}{2\alpha_2-1}}m^{-\frac{2\alpha_2-1}{2}}.
$$
Thus for sufficiently large {$n\geqslant \l \frac{\lambda_d({\mc{B}})}{4d\wt C^2}\sqrt{\frac{2\alpha_2-1}{\zeta(2\alpha_2)}} \r^{\frac{2}{c_1(1-2\alpha_2)}}$}, one has
\begin{align}
&\lambda_j\l \mc{B}^{(m)}\r\geqslant \lambda_j\l \mc{B}\r-\frac{\lambda_d\l \mc{B}\r}{2}
\geqslant\frac{\lambda_d\l \mc{B}\r}{2}\nonumber\\
&\qquad\Longrightarrow \min\{\lambda_{\min}^+({\mc{B}}),\lambda_{\min}^+({\mc{B}^{(m)}})\}\geqslant \frac{\lambda_d({\mc{B}})}{2}. \label{eq:lower bound lambda min plus B Bm}
\end{align}}
Inserting \eqref{eq: upper bound of operator norm of A minus B} and \eqref{eq:lower bound lambda min plus B Bm} into \eqref{eq:sin theta for B Bm} leads to
\begin{align*}
\left\|P_{\mathcal S_{\Y|\boldsymbol{X}}}-P_{\mathcal S_{\Y|\boldsymbol{X}}^{(m)}}\right\| \leqslant \frac{2\pi d\wt C^2}{\lambda_{d}(\mc{B})}\sqrt{\frac{\zeta(2\alpha_2)}{2\alpha_2-1}}m^{-\frac{2\alpha_2-1}{2}}.
\end{align*}
Then choosing $C_2:=\frac{2\pi d\wt C^2}{\lambda_d({\mc{B}})}\sqrt{\frac{\zeta(2\alpha_2)}{2\alpha_2-1}}$ can complete the proof.
\end{proof}



\subsection{Proof of Theorem \ref{theorem, total convergence rate}}
\begin{proof}
Note that
\begin{equation}
\begin{aligned}
\left\|P_{\mc{S}_{\Y|\X}}-P_{\widehat{\mc{S}}_{\Y|\X}^{(m)}}\right\| 
&\leqslant \left\|P_{\mc{S}_{\Y|\X}}-P_{\mc{S}_{\Y|\X}^{(m)}}\right\| +\left\|P_{\mc{S}_{\Y|\X}^{(m)}}-P_{ \widehat{\mc{S}}_{\Y|\X}^{(m)}}\right\| .\\
\end{aligned}
\end{equation}
Next we select $m$ to be $n^{\frac{1-2\gamma}{2\alpha_1+2\alpha_2+1}}$, i.e.,  $c_1:=\frac{1-2\gamma}{2\alpha_1+2\alpha_2+1}$. And it is easy to check that $c_1$ satisfies $2c_1(\alpha_1+1)+2\gamma-1=-\frac{(1-2\gamma)(2\alpha_2-1)}{2\alpha_1+2\alpha_2+1}<0$ and $c_1(2\alpha_1+1)-1=-\frac{2[\gamma(2\alpha_1+1)+\alpha_2]}{2\alpha_1+2\alpha_2+1}<0$.
Then combining Proposition $\ref{proposition, estimation error}$ with Proposition $\ref{proposition, truncation error}$ leads to
\begin{align*}
\P\left[\left\|P_{\mc S_{\Y|\X}}-P_{\widehat{\mc{S}}_{\Y|\X}^{(m)}}\right\| \leqslant\hspace{-0.5mm} (C_1+C_2)n^{-\frac{(2\alpha_2-1)(1-2\gamma)}{2(2\alpha_1+2\alpha_2+1)}}\right]\hspace{-1mm}\geqslant\hspace{-1mm} 1-2\exp\hspace{-0.5mm}\l\hspace{-1mm}- C'n^{\frac{1-2\gamma}{2\alpha_1+2\alpha_2+1}}\r&\\
-\exp\left[\ln\l D_1n^{\frac{2\alpha_1+2\alpha_2+3-4\gamma}{2\alpha_1+2\alpha_2+1}} \r-(D_0+1)n^{\frac{2\gamma}{5}}\right]&
\end{align*}
when $n\geqslant n_3'$, where
\begin{align*}
n_3'=\max\Bigg\{n_1,n_2',\left[\tfrac{\|\Gamma^{-1}M\|  \wt C}{48}\l\tfrac{D_2}{D_0+1}\r^{\frac52}\right]^{\frac{2}{2(c_1(\alpha_1+1)+\gamma)-1}}\hspace{-0.9mm},\l \tfrac{\|\Gamma^{-1}M\| }{2D_3}\r^{\frac{2}{c_1(2\alpha_1+1)-1}}\hspace{-0.9mm},\\
\l\tfrac{D_0+1}{D_2}\r^{\frac{5}{1-2\gamma}},\left[ \tfrac{D_3\wt C}{24}\l \tfrac{D_2}{D_0+1} \r^{\frac52} \right]^{\frac2{2\gamma+c_1}},\l\tfrac{\lambda_d(\mc{B})}{4d\wt C^2}\sqrt{\tfrac{{2\alpha_2-1}}{\zeta(2\alpha_2)}} \r^{\frac{2}{c_1(1-2\alpha_2)}}\Bigg\}
\end{align*}

It is easy to check that as long as $\frac{2\gamma}{5}<\frac{1-2\gamma}{2\alpha_1+2\alpha_2+1}\Longrightarrow\gamma<\frac{5}{4(\alpha_1+\alpha_2+3)}$, 
there exists a constant $n_3''=n_3''\l \gamma,\alpha_1,\alpha_2,D_0,D_1,C'\r$ such that when $n\geqslant n_3'$ further, we have 
\begin{align*}
\P\l\left\|P_{\mc S_{\Y|\X}}-P_{\widehat{\mc{S}}_{\Y|\X}^{(m)}}\right\| \leqslant (C_1+C_2)n^{-\frac{(2\alpha_2-1)(1-2\gamma)}{2(2\alpha_1+2\alpha_2+1)}} \r
\geqslant1-2\exp\l-\tfrac{D_0+1}{2}n^{\frac{2\gamma}{5}} \r.
\end{align*}
Thus one can choose $n_3=\max\{n_3',n_3''\}$ to get the following conclusion.
\begin{proposition}
Under Assumptions $\ref{as:joint distribution assumption}$ to $\ref{assumption: rate-type condition}$, for any $\gamma\in\l0,\tfrac{5}{4(\alpha_1+\alpha_2+3)}\r$, choosing 
$m=n^{\frac{1-2\gamma}{2\alpha_1+2\alpha_2+1}}$ (i.e.,  $c_1=\frac{1-2\gamma}{2\alpha_1+2\alpha_2+1}$) yields a positive constant
\begin{align*}
D_4:=D_4\l \|\Gamma^{-1}M\| ,\sigma_d(\Gamma^{-1}M) ,\gamma,\sigma_0,\sigma_1,d,\wt C,\lambda_d\l\sum\limits_{i=1}^d {\bs{\beta}}_i\otimes{\bs{\beta}}_i\r,\alpha_2\r 
\end{align*}
such that when $n$ is sufficiently large, we have:
\begin{align*}
\P\l\left\|P_{\mc{S}_{\Y|\X}}-P_{\widehat{\mc{S}}_{\Y|\X}^{(m)}}\right\| \leqslant D_4n^{-\frac{(2\alpha_2-1)(1-2\gamma)}{2(2\alpha_1+2\alpha_2+1)}} \r
\geqslant1-2\exp\l -\tfrac{D_0+1}{2}n^{\frac{2\gamma}{5}} \r,
\end{align*}
where $D_0$ and $D_1$ are defined in Proposition $\ref{prop:bound hatMmd Mm}$.
\end{proposition}
\noindent
% Theorem $\ref{theorem, total convergence rate}$ is a direct corollary of above proposition.
Define 
$$\mathtt F:=\left\{\left\|P_{\mc{S}_{\Y|\X}}-P_{\widehat{\mc{S}}_{\Y|\X}^{(m)}}\right\| \leqslant D_4n^{-\frac{(2\alpha_2-1)(1-2\gamma)}{2(2\alpha_1+2\alpha_2+1)}}\right\}.$$
Then 
\begin{align*}
 \mb E\left[\left\|P_{\mc{S}_{\Y|\X}}-P_{\widehat{ \mc{S}}_{\Y|\X}^{(m)}}\right\|^2\right] =&
  \mb E\left[\left\|P_{\mc{S}_{\Y|\X}}-P_{\widehat{ \mc{S}}_{\Y|\X}^{(m)}}\right\|^21_{\mathtt{F}}\right] +
   \mb E\left[\left\|P_{\mc{S}_{\Y|\X}}-P_{\widehat{ \mc{S}}_{\Y|\X}^{(m)}}\right\|^21_{\mathtt{F}^c}\right]\\ 
 \leqslant &
 D_4^2n^{-\frac{(2\alpha_2-1)(1-2\gamma)}{2\alpha_1+2\alpha_2+1}}+4\mb P\left( \mathtt F^c\right)\\
 \lesssim&n^{-\frac{(2\alpha_2-1)(1-2\gamma)}{2\alpha_1+2\alpha_2+1}}+\exp\l -\tfrac{D_0+1}{2}n^{\frac{2\gamma}{5}} \r\\
\lesssim&n^{-\frac{(2\alpha_2-1)(1-2\gamma)}{2\alpha_1+2\alpha_2+1}}.
\end{align*}
This completes the proof of  Theorem \ref{theorem, total convergence rate}.
\end{proof}







\section{Additional Simulation Results of Section \ref{sec:Synthetic}}
This section contains the additional  simulation results  of Sections \ref{sec:Synthetic}  when $\varepsilon\sim N(0,1)$.



We show the average $\mc D(\bs B;\bs{\wh B})$ with different $m$ or $\rho$ for three methods under $\mc M_1$ to $\mc M_3$ in Figure \ref{fig:error 3models,noise1},
where we mark minimal error in each model with red `$\times$'. The shaded areas represent the standard error associated with these estimates and all of them are less than  $0.01$. For FSFIR, the  minimal errors for $\mc M_1-\mc M_3$ are  $0.08,0.02,0.01$ respectively.
For TFSIR, the  minimal errors are  $0.08,0.02,0.01$ and for regularized FSIR,  the  minimal errors are $0.13,0.06,0.01$.  

% Figure environment removed


Figure \ref{fig:error 3models,noise1} shows that FSFIR attains the best performance among  all models. 
Moreover, FSFIR is easier to practice as it does not need a slice number $H$ in advance. 




 


\end{document}

