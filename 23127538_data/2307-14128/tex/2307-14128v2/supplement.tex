\setcounter{equation}{0}
\setcounter{figure}{0}
\setcounter{table}{0}

%%%%%%%%%%%%%%%%%%%%%%%%%%%%%%%%%%%%%%%%%%%%%%%%%%%%%%%%%%%%%%%%%%%%%%%%%%%%%%%%%
%\clearpage
\section{Low lying eigenenergies on all ensembles}\label{spectrum}
% Figure environment removed
In this section, we present the finite-volume GEVP eigenenergies of the isoscalar axialvector 
$bc{\bar{u}}{\bar{d}}$ channel that we extract on all four ensembles listed in Table I of 
the main article, at the five different $m_{u/d}$ values corresponding to $M_{ps}\sim$ 
0.5, 0.6, 0.7, 1.0, and 3.0 GeV. The eigenenergies shown in lattice units include 
the additive offsets related to the NRQCD-based dynamics of heavy bottom quarks. 
The non-interacting two-meson energy levels corresponding to $D\bar B^*$ and $\bar BD^*$ thresholds 
are indicated as dotted horizontal line segments for each lattice and each $M_{ps}$. 
The $D^*\bar B^*$ threshold in each case is also shown in the figure by dashed lines.
Note that the use of wall-smearing setup restricts any direct access to the elastic two-meson 
excitations with nonzero relative meson momenta. This means although a reliable ground state 
extraction could be made, the excited eigenenergies may not represent the real elastic excitations.


%%%%%%%%%%%%%%%%%%%%%%%%%%%%%%%%%%%%%%%%%%%%%%%%%%%%%%%%%%%%%%%%%%%%%%%%%%%%%%%%%
%\clearpage
\section{Operator-state overlaps}\label{sec:OSO}
\bef[htb]
% Figure removed
\caption{Modulus of normalized sink operator-state overlaps $|\tilde{Z}_i^n|$ for an eigenenergy 
indicated by $n={0, 1, 2}$ and an operator represented by $\mathcal{O}_i$, where $i={1, 2, 3}$ on 
the $L_1$ ensemble. The errors in the normalized overlap factors are smaller than the symbols 
and hence are suppressed. }
\eef{Zratiosl40}
In \fgn{Zratiosl40}, we present the modulus of normalized sink operator-state overlaps $|\tilde{Z}_i^n|$, 
normalized such that its largest value for any given operator $\mathcal{O}_i$ across all the eigenenergies
$\{n\}$ is unity \cite{Dudek:2009qf,Padmanath:2013zfa}. $\tilde{Z}_i^n$ quantifies the relative 
relevance of any given operator across all the eigenenergies. The $|\tilde{Z}_i^n|$ values are presented for 
all $M_{ps}$ cases on the $L_1$ ensemble. Each square marker corresponds to the $|\tilde{Z}_i^n|$ for 
a given operator $\mathcal{O}_i$ on to a given eigenenergy $n$. Each horizontal panel stands for an $M_{ps}$ 
indicated on the right-hand side, whereas the vertical lines in each horizontal panel part $|\tilde{Z}_i^n|$
for different operators indicated on the top panel. The $x$-axis ticks refer to the three finite-volume 
eigenenergies we have extracted. $\mathcal{O}_1$, the two-meson operator related to $D\bar B^*$ threshold, 
can be seen to have the largest overlap with the ground state and has significantly small overlaps with 
the excited eigenenergies. $\mathcal{O}_2$, the two-meson operator related to $\bar BD^*$ threshold, has the largest 
overlap with the first excited eigenenergy and a very small overlap with the ground state. $\mathcal{O}_2$ 
also have nonnegligible overlap factors with the second excited eigenenergy indicating $\bar BD^*$-type 
two-meson Fock component, which decreases with increasing $M_{ps}$. On the other hand, $\mathcal{O}_3$, 
the diquark-antidiquark type operator, have substantial overlap factors with all eigenenergies at the two 
lightest $M_{ps}$ values, whereas with an increased $M_{ps}$ its largest overlap is with the second excited 
eigenenergy. Note that $\mathcal{O}_3$ is Fierz related to two-meson interpolators \cite{Padmanath:2015era}, 
and the large $\tilde{Z}_3^n$ values of $\mathcal{O}_3$ for all $n$ could be related to this underlying 
connection between two-meson and diquark-antidiquark operators.

A summary from the above observations on overlap factors is as follows. $\mathcal{O}_1$ predominantly 
determines the ground state, whereas it has significantly small coupling with the excited eigenenergies. 
Similar patterns of overlap factors are also observed for other ensembles, all of which indicate 
that $\mathcal{O}_1$ predominantly determines the ground state. The two excited eigenenergies have strong 
two-meson and diquark-antidiquark Fock components in the two lightest $M_{ps}$ values. One could also 
evaluate and investigate the normalized source operator-state overlaps from the left eigenvectors of 
$\mathcal{C}$ in Eq. (1) in the main draft, which also leads to the same conclusions. 


%%%%%%%%%%%%%%%%%%%%%%%%%%%%%%%%%%%%%%%%%%%%%%%%%%%%%%%%%%%%%%%%%%%%%%%%%%%%%%%%%
%\clearpage
\section{Operator basis dependence}
\bef[tbh!]
% Figure removed
\caption{Operator basis dependence of the low lying eigenenergies of the $L_1$ ensemble and
$M_{ps}\sim$700 MeV for all possible operator basis that can be built out of the three operators
utilized in this work. }
\eef{basisdep}
In \fgn{basisdep}, we show the operator basis dependence as determined for $M_{ps}\sim$
700 MeV in the $L_1$ ensemble, for various operator basis build out of $\mathcal{O}_1$,
$\mathcal{O}_2$, and $\mathcal{O}_3$ operators as defined in Eq. (2) of the main draft. 
The digital indexing on the $x$-axis tick labels refers to various operator basis in the 
order $\{\mathcal{O}_1, \mathcal{O}_2, \mathcal{O}_3\}$, with an overline on the third 
index as a visual aid within the plot to highlight the diquark-antidiqaurk interpolator. 
1 (0) indicates an operator is included in (excluded from) the basis. The horizontal
lines refer to the $D\bar B^*$, $\bar BD^*$ and $\bar B^*D^*$ thresholds. The gray horizontal bands
refer to the two lowest levels in the full basis indicated by $11\overline{1}$. A level
below the threshold appears only when $\mathcal{O}_1$ is present in the basis. The first
excited eigenenergy in the full basis $11\overline{1}$ is faithfully reproduced in those bases
where $\mathcal{O}_2$ is included. $\mathcal{O}_3$ alone does not precisely determine
any level among the GEVP eigenenergies using full basis. Similar observations are also made
on other ensembles. In summary, the ground state in the full basis $11\overline{1}$ is 
reliably determined with $\mathcal{O}_1$ and is unaffected by the inclusion of $\mathcal{O}_2$ 
and $\mathcal{O}_3$ operators. 

%%%%%%%%%%%%%%%%%%%%%%%%%%%%%%%%%%%%%%%%%%%%%%%%%%%%%%%%%%%%%%%%%%%%%%%%%%%%%%%%%
%\clearpage
\bet[hbt!]
  \begin{center}
	  \begin{tabular}{p{2.0cm}p{2.0cm}p{1.5cm}>{\hfill\arraybackslash}p{2.cm}>{\hfill\arraybackslash}p{2.cm}>{\hfill\arraybackslash}p{2.5cm}>{\hfill\arraybackslash}p{2.cm}}
      \hline
		  %$M_{ps}$ [GeV] & $\chi^2/d.o.f$ & \multicolumn{2}{c}{$A^{[0]}/E_{DB^*}+aA^{[1]}/E_{DB^*}$} & \multicolumn{2}{c}{$A^{[0]}/E_{DB^*}+a^2A^{[2]}/E_{DB^*}$}\\\hline
		  $M_{ps}$ [GeV] & $\chi^2/d.o.f$ & \multicolumn{2}{c}{Linear} & $\chi^2/d.o.f$ & \multicolumn{2}{c}{Quadratic}\\
      
		  & & $A^{[0]}/E_{D\bar B^*}$ & $A^{[1]}/E_{D\bar B^*}$ && $A^{[0]}/E_{D\bar B^*}$ & $A^{[2]}/E_{D\bar B^*}$\\\hline
		  0.5 & 2.1/2 & $-0.05(1)$ & $~0.17(_{-11}^{+13})$ & 2.1/2 & $-0.045(6)$ & $0.9(_{-6}^{+7})$  \\\hline
		  0.6 & 0.5/2 & $-0.044(_{-8}^{+9})$ & $~0.10(_{-9}^{+9})$ & 0.5/2 & $-0.040(_{-5}^{+6})$ & $0.6(5)$ \\ \hline
		  0.7 & 3.0/2 & $-0.042(_{-6}^{+8})$ & $~0.09(_{-7}^{+6})$ & 3.7/2 & $-0.037(_{-4}^{+5})$ & $0.5(_{-4}^{+3})$ \\ \hline
		  1.0 & 2.9/2 & $-0.043(4)$ & $~0.11(_{-5}^{+5})$ & 2.9/2 & $-0.038(_{-3}^{+3})$  & $0.8(3)$  \\ \hline
		  3.0 & 3.6/2 & $~0.006(_{-5}^{+6})$ & $-0.20(_{-5}^{+4})$ & 3.6/2 & $-0.002(_{-3}^{+4})$ & $-1.2(_{-3}^{+2})$ \\ \hline
      \hline
  \end{tabular}
  \end{center}
\caption{Results from amplitude fits at different light quark mass cases indicated
in terms of $M_{ps}$ in the first column. The amplitude is approximated to be determined 
by the scattering length, with a linear or quadratic lattice spacing dependence as discussed 
in the main draft. The optimized parameter values in the table are presented in units of 
energy of the $D\bar B^*$ threshold, $E_{D\bar B^*}$. The $A^{[0]}$ parameter in either parametrization 
is negative of the inverse scattering length in the continuum limit. }
\eet{Ampfits1}

\begin{figure}[hbt!]
% Figure removed
% Figure removed
\caption{Left: $k{\mathrm{cot}}\delta_0$, in units of the elastic threshold $E_{D\bar B^*}$, versus $a$
(lattice spacing) for all $M_{ps}$ values. We follow the marker/color coding in Table I of 
the main draft for the data points referring to the simulated data. The colored/gray bands 
indicate the fit results with linear and quadratic lattice spacing dependence, respectively. 
Right: $k{\mathrm{cot}}\delta_0$ versus $k^2$ for all $M_{ps}$ values studied in units of 
the elastic threshold $E_{D\bar B^*}$. The dashed orange (cyan) curve indicates the constraint for 
the existence of a sub-threshold pole in the scattering amplitude. The horizontal bands are 
the continuum extrapolated estimates of $k{\mathrm{cot}}\delta_0$ for the respective $M_{ps}$. 
The black dashed vertical lines in the plots on the right indicate the location of the branch
point associated with the left hand cut arising from the $D\bar B\pi$ channel. }
\eef{pcotdelta_summary}

\section{Results on scattering amplitude}
In \tbn{Ampfits1}, we tabulate the results from different amplitude fits that were performed.
In \fgn{pcotdelta_summary}, we present the quality of these fits by comparing the fit results 
with the data points (see the figure caption for details). On the left of \fgn{a0_mpi2_fits_LQ}, 
we present the light quark mass dependence in the chiral regime determined from continuum 
extrapolated elastic $D\bar B^*$ scattering amplitudes following a linear and quadratic lattice 
spacing dependence. We present the final estimate (black star) from the linear fit form 
considering the presence of heavy quarks in our system, whereas the difference in the fit 
results are accounted in the systematics quoted in the main draft. On the right of \fgn{a0_mpi2_fits_LQ}, 
we present a comparison of the light quark mass dependence in the chiral regime between 
the fit involving all $M_{ps}$ datasets and the fit involving the lightest four $M_{ps}$ 
datasets. Either fitting procedures can be seen to be consistent with our final estimate in 
the chiral limit is shown by black star. 



\bef[hbt!]
% Figure removed
% Figure removed
\caption{Left: Comparison of light quark mass dependence of scattering amplitudes in the chiral regime 
determined from a linear (red band) and quadratic (green band) dependence of $k{\mathrm{cot}}\delta_0$ 
on the lattice spacing. Right: Comparison of light quark mass dependence of scattering amplitude 
$k{\mathrm{cot}}\delta_0$ in the chiral regime determined using the results from all five light quark 
masses (red band) and the results from four light light quark masses (black). The results in the physical 
limit in either cases can be seen to be consistent with the main result indicated by the star symbol. }
\eef{a0_mpi2_fits_LQ}

