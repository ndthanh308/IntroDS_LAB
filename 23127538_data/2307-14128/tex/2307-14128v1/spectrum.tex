%%%%%%%%%%%%%%%%%%%%%%%%%%%%%%%%%%%%%%%%%%%%%%%%%%%%%%%%%%%%%%
\section{Energy spectra in finite-volume}\label{fvresults}
%%%%%%%%%%%%%%%%%%%%%%%%%%%%%%%%%%%%%%%%%%%%%%%%%%%%%%%%%%%%%%
In this section, we present our results that we obtain from the finite-volume correlators. 
After presenting the energy spectrum extracted using variational techniques, we discuss 
the operator-state-overlaps and the operator basis dependence. In the final subsection, 
we describe our strategy for rebuilding the ground state energies that are corrected for 
the additive NRQCD offset and for using them in further amplitude fits. 

\subsection{Details of energy spectra}
% Figure environment removed
In \fgn{spectrum}, we present the finite-volume energy spectra of the isoscalar 
axialvector $bc{\bar{u}}{\bar{d}}$ channel that we extract on the four ensembles 
listed in \tbn{lattice}, at the five different $m_{u/d}$ values corresponding to 
$M_{ps}\sim$ 0.5, 0.6, 0.7, 1.0, and 3.0 GeV. The energy spectrum is shown in 
lattice units. Note that these levels are shown with unaccounted additive 
renormalization measures related to the NRQCD-based dynamics of the heavy bottom quarks. 
The noninteracting two-meson energy levels corresponding to $DB^*$ and $BD^*$ thresholds 
are indicated as dotted horizontal line segments for each lattice and each $M_{ps}$. 
A clear trend for negative energy shifts can be observed in all the cases, indicating 
a possible attractive interaction between the scattering particles involved \cite{scalarbc}.
The $B^*D^*$ threshold in each case is also shown in the figure by dashed lines. 

From the energy spectra in the lattices $L_1$, $S_2$ and $S_3$, it can be observed 
that a consistent pattern emerges with respect to the two-meson thresholds.
The relative positioning of the ground state energy with the elastic threshold in the 
$S_1$ ensemble is also consistent with the other three ensembles. This is an encouraging 
feature in the finite-volume spectrum, as our main interest is on reliable extraction of the 
ground state energies. It is this ground state energy from each ensemble that we 
later on employ to constrain the $DB^*$ scattering amplitude.  

The excited states in the $S_1$ ensemble for $M_{ps}$, other than at the charm point, 
indicate enhanced negative shifts compared to that on the other ensembles. This could be 
related to a combination of effects arising from various less attractive features of the 
$S_1$ lattice, which includes the coarsest lattice spacing, small spatial volume and 
possible insufficient statistics for the study at lighter $M_{ps}$. To this end, we 
perform two additional checks. First, we make an associated study of the $S_1$ and the 
$L_1$ ensembles at the level of variational analysis and fitting procedures to determine 
the low-lying spectra with emphasis on the ground and the first excited states. We discuss 
this part of the investigation in Appendix \ref{app:S1L1}. Secondly, we perform amplitude 
fits with and without results from the $S_1$ ensemble to verify the robustness in our estimates 
for the scattering length. We discuss this in detail in Section \ref{Ampfits}.

\subsection{Operator-state overlaps}\label{sec:OSO}
\bef[hbt!]
% Figure removed
\caption{Normalized operator-state overlaps $\tilde{Z}_i^n$ for a state indicated by $n={0, 1, 2}$ 
and an operator represented by $\mathcal{O}_i$, where $i={1, 2, 3}$ on the $L_1$ ensemble. 
The errors in the normalized overlap factors are smaller than the symbols and hence are 
suppressed. The five horizontal panes stand for the five different $M_{ps}$ values we 
investigate. The two vertical lines in each horizontal pane separate $\tilde{Z}_i^n$ for 
different operators $\mathcal{O}_i$. }
\eef{Zratiosl40}
Now we investigate the operator-state overlaps $Z_i^n$, as in \eqn{overlaps}, to evaluate 
the efficacy of the interpolators in determining the low-lying spectra. To this end, 
we define normalized operator-state overlaps $\tilde{Z}_i^n$ such that its largest value 
for any given operator $\mathcal{O}_i$ across all the states $\{n\}$ is unity \cite{Dudek:2009qf,Padmanath:2013zfa}.
$\tilde{Z}_i^n$ quantifies the relative relevance of any given operator across all the 
states. In \fgn{Zratiosl40}, we present $\tilde{Z}_i^n$ at all $M_{ps}$ values we have used
on the $L_1$ ensemble. Each square marker corresponds to the $\tilde{Z}_i^n$ for a given operator 
$\mathcal{O}_i$ on to a given state $n$. Each horizontal pane stands for an $M_{ps}$ indicated on the 
right-hand side, whereas the vertical lines in each horizontal pane part $\tilde{Z}_i^n$ 
for different operators indicated on the top pane. The $x$-axis ticks refer to the three low-lying 
states we have extracted. $\mathcal{O}_1$, the two-meson operator related to $DB^*$ threshold, 
can be seen to have the largest overlap with the ground state and has significantly small 
overlaps with the excited states. $\mathcal{O}_2$, the two-meson operator related to $BD^*$ 
threshold, has the largest overlap with the first excited state and a very small overlap with 
the ground state. $\mathcal{O}_2$ also have nonnegligible overlap factors with the second 
excited state indicating $BD^*$-type two-meson Fock component, which decreases with increasing 
$M_{ps}$. On the other hand, $\mathcal{O}_3$, the diquark-antidiquark type operator, have 
substantial overlap factors with all states at the two lightest $M_{ps}$ values, whereas with 
an increased $M_{ps}$ its largest overlap is with the second excited state. Note that 
$\mathcal{O}_3$ is Fierz related to two-meson interpolators \cite{Padmanath:2015era}, and 
the large $\tilde{Z}_3^n$ values of $\mathcal{O}_3$ for all $n$ could be related to this 
underlying connection between two-meson and diquark-antidiquark operators. 


A summary from the above observations on overlap factors is as follows. $\mathcal{O}_1$
predominantly determines the ground state, whereas it has a significantly small coupling with 
the excited states. Similar patterns of overlap factors are also observed for other ensembles, 
all of which indicate that $\mathcal{O}_1$ predominantly determines the ground state. 
The two excited states have strong two-meson and diquark-antidiquark Fock components 
in the two lightest $M_{ps}$ values. The two-meson Fock components in the second excited state 
and the diquark-antidiquark Fock components in the first excited state decreases with 
increasing $M_{ps}$. This is consistent with the phenomenological expectation, which suggests 
that the binding energy in doubly heavy tetraquarks increases with increasing 
relative heaviness for the heavy quarks with respect to the light quarks \cite{Francis:2016hui,
Czarnecki:2017vco,Junnarkar:2018twb}. A deeply bound state could be significantly compact 
and hence could have large Fock components of a compact object such as that of a 
diquark-antidiquark. In other words, the relevance of compact diquark-antiquark operators for 
the low-lying spectrum increases with decreasing light quark mass, as is evident from \fgn{Zratiosl40}. 

\subsection{Operator basis dependence}
\bef[tbh!]
% Figure removed
\caption{Operator basis dependence of the low energy spectra of the $L_1$ ensemble and 
$M_{ps}\sim$700 MeV for all possible operator basis that can be built out of the three operators 
discussed in Section \ref{sec:2ptIO}. The basis is presented in digital notation ($x$-axis tick labels) 
where the operators are arranged in the order $\{\mathcal{O}_1, \mathcal{O}_2, \mathcal{O}_3\}$.
The horizontal lines refer to the $DB^*$, $BD^*$, and $D^*B^*$ thresholds. The bands indicate the 
bounds of the ground and first excited state energy estimates from the full three-operator basis. }
\eef{basisdep}
Next, we look into the basis dependence of the finite-volume energy spectra presented in 
\fgn{spectrum}. In \fgn{basisdep}, we show this basis dependence as determined for $M_{ps}\sim$ 
700 MeV in the $L_1$ ensemble, for various operator basis build out of $\mathcal{O}_1$, 
$\mathcal{O}_2$, and $\mathcal{O}_3$ operators as defined in \eqn{mmops} and \eqn{dadops}. The digital 
indexing on the $x$-axis tick labels refers to various operator basis in the order 
$\{\mathcal{O}_1, \mathcal{O}_2, \mathcal{O}_3\}$, with an overline on the third index 
as a visual aid within the plot to highlight the diquark-antidiqaurk interpolator. 1 (0) 
indicates an operator is included in (excluded from) the basis. The horizontal 
lines refer to the $DB^*$, $BD^*$ and $B^*D^*$ thresholds. The gray horizontal bands 
refer to the two lowest levels in the full basis indicated by $11\overline{1}$. A level 
below the threshold appears only when $\mathcal{O}_1$ is present in the basis. The first 
excited state in the full basis $11\overline{1}$ is faithfully reproduced in those bases 
where $\mathcal{O}_2$ is included. $\mathcal{O}_3$ alone does not precisely determine 
any level in the energy spectrum using full basis. Similar observations are also made 
on other ensembles. 

In summary, the ground state in the full basis $11\overline{1}$ is reliably determined 
with $\mathcal{O}_1$ and is unaffected by the inclusion of $\mathcal{O}_2$ and 
$\mathcal{O}_3$ operators. The excited states have nonnegligible overlap factors with 
$\mathcal{O}_2$ and $\mathcal{O}_3$ operators. Given our setup with only a few energy 
levels, any assumption more complicated than a simple elastic $DB^*$ assumption for 
the amplitude fits is beyond the scope of this work. Such an assumption is justified within the 
isosymmetric limit as the lowest inelastic threshold ($BD^*$ at unphysically heavy 
$m_{u/d}$ or $BD\pi$ for $m_{u/d}^{phys}$) is significantly high. In light of all 
these observations above, we limit ourselves to using only ground states determined 
from all ensembles at various $M_{ps}$ values to constrain the elastic $S$-wave
$DB^*$ scattering amplitude.

\subsection{The ground state energies}

\begin{figure}[h]
% Figure removed
\caption{The ground state energies in units of the elastic threshold ($DB^*$) on all 
ensembles (see \tbn{lattice} for color-symbol conventions) for all $M_{ps}$ values
(different vertical panes).}
\eef{gsspectrum}

In this subsection, we discuss how we obtain ground state energies after adjusting the 
additive correction that is inherent to an NRQCD calculation. The set of numbers that 
we extract from our variational analysis and eigenvalue correlator fitting procedures 
are the single meson masses $M_{B}$, $M_{B^*}$, $M_{D}$, and $M_{D^*}$ and the energy 
splittings $\Delta E_n = E_n - M_{m_1} - M_{m_2}$ (see \eqn{ratio}). %for the interacting system from the reference 
%two meson level $m_1m_2$ determined from the ratio of correlators defined in \eqn{ratio}. 
First, we account for the NRQCD corrections in single meson masses (involving a bottom quark) as 
\beq
\tilde{M}_{B^{(*)}} = M_{B^{(*)}} - 0.5\overline M^{\bar bb}_{lat} + 0.5 \overline M^{\bar bb}_{phys},
\eeq{msnNRcor}
where $\overline M^{\bar bb}_{lat}(\overline M^{\bar bb}_{phys})$ refers to the spin averaged mass 
of the $1S$ bottomonium measured on the lattice (experiments). We follow this procedure as we 
have tuned the bottom quark mass through the spin average bottomonia at each ensemble.

For the interacting energy spectrum, the NRQCD offset is automatically canceled in the energy splittings
$\Delta E^n$. One can then build the energy estimates $\tilde{E}^n$ of interacting spectrum by adding 
the noninteracting level energy ($M_{m_1} + M_{m_2}$) with the energy splittings $\Delta E^n$ as,
\beq 
\tilde{E}^n = \Delta E^n + M_{m_1} + M_{m_2}.
\eeq{intNRcor}
If either $m_1$ or $m_2$ is a bottom meson, we use the corresponding corrected $\tilde{M}_{m_i}$\footnote{From 
the next section, for brevity we suppress the $\tilde{}$ notation indicating corrected masses and energies.} 
determined using \eqn{msnNRcor}, instead of $M_{m_i}$. 

In \fgn{gsspectrum}, we present the corrected ground state energy estimates, 
in units of the energy of elastic threshold $E_{DB^*}$, at various $M_{ps}$ and 
for all the ensembles we have employed. The spectrum clearly shows a trend of decreasing 
energy spitting, hence decreasing interaction strength, with increasing $M_{ps}$. 
Another feature worth noting here is that the lattice spacing dependence of the 
ground state energies on similar volume ensembles ($S_1$, $S_2$, $S_3$) for the 
non-charm $M_{\pi}$ are opposite to that at the charm point. We will revisit this 
point when we discuss extraction of $DB^*$ scattering amplitude using these energy levels. 

