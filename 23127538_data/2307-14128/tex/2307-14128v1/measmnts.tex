
\section{Measurements and interpolators}\label{sec:2ptIO}

Lattice determination of finite-volume spectrum follows through an evaluation or measurement of Euclidean 
two-point correlation functions $\mathcal{C}_{ij}(t)$, of interpolating operators 
$\mathcal{O}_i(\mathbf{x},t)$ with desired quantum numbers, given by
\beq
\mathcal{C}_{ij}(t) = \sum_{\mathbf{x}}\left<\mathcal{O}_i(\mathbf{x},t)\mathcal{O}_j^{\dagger}(\mathbf{0},0)\right> = \sum_n \frac{Z_i^nZ_j^{n\dagger}}{2E^n} e^{-E^nt}.
\eeq{c2pt}
Here the second equality suggests that $\mathcal{C}_{ij}(t)$ can be expressed as a sum of exponentials 
following a spectral decomposition. $Z_i^n = \bra{0}\mathcal{O}_i\ket{n}$ is the operator-state overlap that 
quantifies the efficacy of the interpolator $\mathcal{O}_i$ in determining the time evolution of the state $n$. 
The utilization of wall smearing for the quark sources effectively kills all the high-momentum modes
at the source, whereas a zero momentum projection at the sink time slice ($\sum_{\mathbf{x}}$), as shown
in \eqn{c2pt}, efficiently projects the correlation function to the rest frame.  

Our main focus is on the ground state in the $T_{1}^+$ irreducible representation (irrep) in the rest frame, 
which is the only relevant rest frame finite-volume irrep for studying states in the infinite-volume continuum 
with quantum numbers ($J^P = 1^+$). To this end, we use a similar set of operators in the $T_{1}^+$ irrep as 
was utilized in Ref. \cite{Francis:2018jyb} and we briefly discuss them below for completeness. Assuming isospin symmetry, 
the relevant low-lying two-meson thresholds in the order of increasing energy are $E_{DB^*} = M_{B^*}+M_{D}$, 
$E_{BD^*}=M_B+M_{D^*}$, and $E_{D^*B^*}=M_B^*{}+M_{D^*}$. Hence, we consider the following low-lying two-meson 
interpolators 
\beqa
\mathcal{O}_1(x) &=& [\bar u(x) \gamma_i b(x)][\bar d(x) \gamma_5 c(x)]  \nonumber \\&& - [\bar d(x) \gamma_i b(x)][\bar u(x) \gamma_5 c(x)] \nonumber \\
\mathcal{O}_2(x) &=& [\bar u(x) \gamma_5 b(x)][\bar d(x) \gamma_i c(x)]  \nonumber \\&& - [\bar d(x) \gamma_5 b(x)][\bar u(x) \gamma_i c(x)] \nonumber \\
\mathcal{O}'(x) &=& \epsilon_{ijk} [\bar u(x) \gamma_i b(x)][\bar d(x) \gamma_j c(x)] \nonumber \\&& - [\bar d(x) \gamma_i b(x)][\bar u(x) \gamma_j c(x)].
\eeqa{mmops}
We utilize $\mathcal{O}_1(x)$ and $\mathcal{O}_2(x)$ in the computation of correlation functions. 
$\mathcal{O}'(x)$ has its associated two-meson threshold sufficiently higher up in the energy 
spectrum compared to the other two thresholds and it was found to have no effects in the low-lying 
energy spectrum. Hence we disregard this operator
from the rest of our analysis. Note that the lowest three particle threshold $DB\pi$ is above $E_{BD^*}$
for all the considered heavier-than-physical light quark masses. At $m_{u/d}^{phys}$, the $BD\pi$ threshold is 
immediately below $E_{BD^*}$, yet it remains sufficiently above $E_{DB^*}$ 
to have any significant effects on the ground states that we extract. We also compute two-point 
correlation functions for $B$, $B^*$, $D$, and $D^*$ mesons, using standard local quark 
bilinear interpolators ($\overline Q~\Gamma~q$) with spin structures $\Gamma\sim\gamma_5$ and 
$\gamma_i$ for pseudoscalar and vector quantum numbers, respectively. 

Phenomenologically, doubly bottom tetraquark in the axialvector channel is expected to be deeply 
bound. Such a state is expected to be quite compact owing to its doubly heavy flavor content 
and deeply bound nature \cite{Francis:2016hui,Czarnecki:2017vco}. Consequently, a local 
diquark-antidiquark interpolator is naturally interesting. Such an operator has already 
been utilized in all lattice QCD studies of the doubly bottom as well as bottom charm tetraquarks 
in the past \cite{Bicudo:2015kna,Francis:2016hui,Bicudo:2017szl,Junnarkar:2018twb,Leskovec:2019ioa,
Francis:2018jyb,Hudspith:2020tdf,Meinel:2022lzo,Hudspith:2023loy} and we follow the same strategy. 
Along with operators in \eqn{mmops}, we employ a local diquark-antidiquark interpolator 
\beq
\mathcal{O}_3(x) = (\bar u(x)^T \Gamma_5 \bar d(x) - \bar d(x)^T \Gamma_5 \bar u(x))( b(x) \Gamma_i c(x)),
\eeq{dadops}
where $\Gamma_k = C\gamma_k$ with $C=i\gamma_y\gamma_t$ being the charge conjugation matrix and 
the diquarks (antidiquarks) in the color antitriplet (triplet) representations. 

Our final basis is composed of the above-mentioned three interpolators $\{\mathcal{O}_1(x), \mathcal{O}_2(x), \mathcal{O}_3(x)\}$, 
which is diverse enough to reliably determine the ground state in the energy spectra that we are interested in.. Using this basis we determine 
the correlation matrices, with elements evaluated as prescribed in \eqn{c2pt}. Then the correlation matrices $\mathcal{C}$
are analyzed following a variational approach \cite{Michael:1985ne} to determine the energy estimates for 
low-lying levels in the spectrum. In this procedure, we look for the solutions of the generalized eigenvalue 
problem (GEVP) given by 
\beq
\mathcal{C}(t)v^n(t) = \lambda^n(t) \mathcal{C}(t_0)v^n(t),
\eeq{gevp}
where $t_0$ is a reference timeslice at which the eigenvalues $\lambda^n$s are identically unity. 
\bef[h]
% Figure removed
\caption{Effective energy plot for the eigenvalue correlation function $\lambda^0(t)$ (square) and for the product 
of single-meson correlators (circle) representing the noninteracting two-meson correlation function 
($\mathcal{C}_{D}(t)\mathcal{C}_{B^*}(t)$). The data correspond to $M_{ps} \sim 700$ MeV in the finest ensemble. 
The bands shown are the energy fit estimates for the final chosen time intervals.}
\eef{effmass}
The eigensolutions in the large time limit represent the lowest $N$ eigenstates $E^n$, for which the time 
evolution is dictated by the eigenvalues as $\lim_{t\to\infty}\lambda^n(t) \sim A_ne^{-E^nt}$. 
The corresponding eigenvectors are represented by $v^n(t)$, which are related to the operator-state-overlaps as
\beq
Z_i^{n}=\bra{0}\mathcal{O}_i \ket{n} = \sqrt{2E^n}(V^{-1})_i^n e^{E^{n}(t_0)/2},
\eeq{overlaps}
where $V$ is a matrix built out of $v^n(t)$. $v^n(t)$ is expected to be time independent in the limit, 
where the signal in $\mathcal{C}$ is saturated by the lowest $N$ eigenstates of the system.  

Conventionally the signal in the two point correlator data $C(t)$ is first assessed based on the 
large time plateauing in effective energies defined as $aE_{eff} = [ln(C(t)/C(t+\delta t))]/\delta t$. In 
\fgn{effmass}, we present the effective energies as a function of time for the eigenvalue correlation 
function (squares) and the noninteracting two-meson ($\mathcal{C}_{D}(t)\mathcal{C}_{B^*}(t)$) 
correlation function (circles). These effective energies can be seen to saturate around timeslices 
24 to 28 in the example shown. The results presented correspond to the lowest eigenvalue correlator 
$\lambda^0(t)$ at the strange quark mass ($M_{ps}\sim0.7$ GeV) in the finest ensemble we study. 
Evidently, there is a negative shift in the energies in $\lambda^0(t)$ with respect to the 
noninteracting energies at all times, except at very large times where the signal-to-noise ratio 
degrades substantially. 

Extraction of the energy spectra proceeds via fitting the eigenvalue correlators, $\lambda_{n}(t)$, 
with the expected asymptotic exponential behaviour. Alternatively, one can fit the asymptotic time 
estimates for the ratio of correlators given by 
\beq
R^n(t)=\frac{\lambda^n(t)}{\mathcal{C}_{m_1}(t) \mathcal{C}_{m_2}(t)}, 
\eeq{ratio}
to a single exponential form ($Ae^{-\Delta E^nt}$), where $\Delta E^n$ is expected to saturate to 
$E^n-M_{m_1}-M_{m_2}$ at large times. Here, $\mathcal{C}_{m_i}$ is the correlation function for 
the meson $m_i$, and $M_{m_i}$ is its mass. Being a ratio, $R^n(t)$ is empirically known to efficiently 
mitigate correlated noise between the product of two meson correlators and the interacting correlator 
for the two-meson system \cite{Green:2021qol}. Note that the automatic cancellation of the additive 
mass renormalization, inherent to NRQCD formulation, is an added advantage in using \eqn{ratio} for 
the fits. In \fgn{fitcompare}, we present a representative plot showing the $t_{min}$ dependence of 
the $\Delta E^n$ fit estimates determined from the fits to $\lambda^n(t)$ and $R^n(t)$, respectively, 
where $t_{min}$ is the lower boundary of the time interval used for these fits for a fixed upper boundary 
timeslice for the time interval. The energy differences are evaluated from $\lambda^n(t)$ using the relation 
$\Delta E^n = E^n-M_{m_1}-M_{m_2}$, where $M_{m_1}$ and $M_{m_2}$ are mass estimates for individual 
mesons determined from separate fits to $\mathcal{C}_{m_1}(t)$ and $\mathcal{C}_{m_2}(t)$, respectively. 
The estimates from different procedures can be seen to agree asymptotically in time, based on which 
optimal $t_{min}$ values are chosen. Our final results are based on fitting the ratio correlators 
defined in \eqn{ratio}.

    
\bef[h]
% Figure removed
\caption{$t_{min}$ dependence of the $\Delta E^0$ fit estimates determined from the fits to $\lambda^0$
and $R^0(t)$ for the case $M_{ps} \sim 700$ MeV in the finest ensemble. Here the superscript 0 refers 
to the ground state. }
\eef{fitcompare}


