%%%%%%%%%%%%%%%%%%%%%%%%%%%%%%%%%%%%%%%%%%%%%%%%%%%%%%%%%%%%%%%%%%%%%%%%%%%%%%%%%
%% APPENDICES
\appendix

%%%%%%%%%%%%%%%%%%%%%%%%%%%%%%%%%%%%%%%%%%%%%%%%%%%%%%%%%%%%%
\section{Energy spectrum in $S_1$ and $L_1$ ensemble}\label{app:S1L1}
% Figure environment removed

% Figure environment removed
A dissimilar observation of enhanced negative energy shifts was observed 
for the excited energy levels at the non-charm pion masses in the $S_1$ 
ensemble in comparison with the other three ensembles. In this appendix, 
we assess the robustness of the ground and the first excited state determination 
within the $S_1$ ensemble in conjunction with the similar analysis made 
for the $L_1$ ensemble. We choose the $L_1$ ensemble for this purpose, as 
it has almost similar lattice spacing to that of the $S_1$ ensemble, yet have 
a larger spatial volume. This allows one to make justified comparison of 
the energy in lattice units between the two ensembles. 

In \fgn{l24l40efm}, we present the effective energies of the ground and first 
excited states in the $S_1$ and $L_1$ ensembles. The signal quality is clearly 
better in $L_1$, thanks to its large spatial volume. For the ground state, 
it can be seen that there is a consistent behaviour of energy estimate in $L_1$
to be below that for $S_1$. However, for the first excited state the energy 
estimate in $L_1$ consistently appear to above that for $S_1$ in the timeslice, 
where signal is not yet swamped by the noise. 

This observation can be further quantified by looking into the fit estimates 
for energies. In \fgn{l24l40tmin}, we present the $t_{min}$ dependence of the 
energy fit estimates for the same set of energy levels in terms of the energy 
splittings from ratio of correlators (\eqn{ratio}) with reference noninteracting 
two-meson level chosen as $DB^*$. Clear negative energy shifts can be observed 
for the ground state in the $L_1$ ensemble, whereas dominant noise in the $S_1$ 
ensemble results in a nearly consistent estimate with noninteracting scenario. 
The numbers for the first excited state in the $L_1$ ensemble clearly indicates 
a positive energy difference with respect to the $B^*D$ threshold, whereas 
the first excited state in the $S_1$ is significantly noisier and is consistent 
with the $DB^*$ threshold. 
 
\bef[bth!]
\vspace{0.5cm}
% Figure removed
\caption{Normalized operator-state overlaps $\tilde{Z}_i^n$ for a state 
indicated by $n={0, 1, 2}$ and an operator represented by $\mathcal{O}_i$, 
where $i={1, 2, 3}$ on the $S_1$ and $L_1$ ensembles and two values of $M_{ps}\sim0.5$
and 3.0 GeV. The uncertainties in the normalized overlaps are smaller than 
the size of the symbols, hence are suppressed.} 
\eef{l24l40overlap}

To understand the levels in $S_1$ and $L_1$ ensembles further, one could compare 
the normalized overlap factors defined in Section \ref{sec:OSO}. In \fgn{l24l40overlap}, 
we present the normalized overlap factors for all the energy levels in the $S_1$ and 
$L_1$ ensembles for the lightest and the heaviest pion masses. $\tilde{Z}_i^n$ for
the ground state in all the cases can be observed to be very similar, confirming 
the reliability of the corresponding estimate. The excited states at the charm point
also share similarities in the overlap factors confirming the state identity in terms 
of the interpolating operators. For the lightest pion mass, the overlap factors 
for the excited states are only partially consistent, possibly indicating their 
compounded nature. In summary, a below elastic threshold ground state is consistently 
observed with striking similarities also in the overlap factors. This observation 
gives confidence in using the ground states also from the $S_1$ ensemble to constrain 
the scattering amplitude. 




