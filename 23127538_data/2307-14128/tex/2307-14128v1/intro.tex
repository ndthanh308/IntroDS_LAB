\section{Introduction}
  \label{Intro}

One of the most active field of research in modern hadron physics is the study of exotic hadrons. Exotic 
hadrons are those hadrons, whose quantum numbers and/or valence quark contents, and/or their other properties 
(like decay channels, lifetimes, branching fractions, etc.) are in conflict with the widely accepted 
three-quark or quark-antiquark hadron interpretations. The existence of exotic hadrons have been proposed 
since the early days of quark model \cite{GellMann:1964nj,Jaffe:1976ig,Jaffe:2005zz} (see the recent reviews \cite{Lebed:2016hpi,Richard:2016eis,Ali:2017jda}).
Yet, the discoveries of such hadrons have been made only in recent years, predominantly with one or more
heavy quark content \cite{BESIII:2013ris,LHCb:2014zfx,LHCb:2015yax,LHCb:2019kea,LHCb:2021vvq,LHCb:2022xob}. 
Of these, the recent discovery of the doubly charmed tetraquark\footnote{We follow the nomenclature that 
a ``tetraquark" refers to any bound state or resonance with dominant four-quark Fock component, whether it 
is a compact four-quark object or a two-meson molecule or a mixture of both.} $T_{cc}$ marks an important 
milestone \cite{LHCb:2021vvq}, particularly for the fact that $T_{cc}$ is the longest lived exotic ever 
discovered and it has explicitly exotic flavor content $cc\bar u\bar d$. Phenomenologically, doubly heavy 
tetraquarks in the heavy quark limit are long hypothesized to form deeply bound states \cite{Ader:1981db,
Ballot:1983iv,Zouzou:1986qh,Heller:1986bt,Carlson:1987hh,Manohar:1992nd,Janc:2004qn,Ebert:2007rn,
Navarra:2007yw,Eichten:2017ffp,Karliner:2017qjm}. This implies a possible deeply bound doubly bottom 
tetraquark ($T_{bb} \equiv bb\bar u\bar d$). Recent lattice calculations have also reached a consensus 
on the existence of such a deeply bound state with isospin $I=0$, total angular momentum and parity $J^P = 1^+$ 
\cite{Bicudo:2015kna,Francis:2016hui,Bicudo:2017szl,Junnarkar:2018twb,Leskovec:2019ioa,Hudspith:2023loy}.  
The corresponding tetraquark in the charm sector is the above mentioned $T_{cc}$, which is observed to 
be just a few hundred keV below the $D^0D^{*+}$ threshold \cite{LHCb:2021vvq}. Naturally, the binding of 
a similar doubly heavy tetraquark, but with one bottom quark, one charm quark, and with a valence quark 
configuration $T_{bc} \equiv bc\bar u\bar d$ is interesting. In this work, we look for such bottom-charm 
tetraquark bound states with isoscalar axialvector [$I(J^P)=0(1^+)$] quantum numbers.

A deep binding in the doubly heavy tetraquarks has been most commonly proposed based on a compact heavy 
diquark-light antidiquark picture with a color antitriplet $\overline{3}_c$ heavy diquark and an isoscalar 
scalar [$I(J)=0(0)$] the light antidiquark in the triplet ($3_c$) color configuration. This phenomenological 
picture argues that the binding in the doubly heavy tetraquark increases with increasing heavy quark mass 
and decreasing light quark mass \cite{Francis:2016hui,Czarnecki:2017vco}. This implies the shallow binding 
energy of $T_{cc}$ could be a reflection of its dominant noncompact molecular nature \cite{Janc:2004qn,Agaev:2022ast}, 
although a compact nature has not been entirely ruled out \cite{Ortega:2022efc,Dai:2023kwv,Meng:2022ozq}. 
Assuming quantum mechanical solutions for simple attractive potentials between the $D$ and $D^*$ mesons, 
the variation in binding energy of $T_{cc}$ with the heavy and the light quark masses was discussed in Ref.
\cite{Padmanath:2022cvl}, see also Ref. \cite{Lyu:2023xro}. Bottom-charm tetraquarks form an intermediate 
platform, where there could be complicated interplay between the compact diquark-antidiquark picture and 
the molecular picture. Moreover, a bottom-charm tetraquark has two heavy quark mass scales in addition to 
that of the light quark, unlike its doubly bottom/charm counterparts. It is expected that both these 
distinctions would be reflected in their low energy spectra, when studied in conjunction with the doubly 
bottom and charm tetraquarks. In short, a collective and refined knowledge of low energy spectra in these 
three doubly heavy systems ($T_{bb}$, $T_{bc}$ and $T_{cc}$) could culminate in a deeper understanding of 
strong interaction dynamics across a wide quark mass regime spanning from charm to the bottom quarks.

The isoscalar bottom-charm tetraquarks with quantum numbers [$I(J^P) = 0(1^+)$] have been investigated 
previously both using lattice \cite{Francis:2018jyb,Hudspith:2020tdf,Meinel:2022lzo} and nonlattice methodologies 
\cite{Heller:1986bt,Carlson:1987hh,Janc:2004qn,Ebert:2007rn,Chen:2013aba,Sakai:2017avl,Eichten:2017ffp,
Karliner:2017qjm,Czarnecki:2017vco,Carames:2018tpe,Park:2018wjk,Deng:2018kly,Yang:2019itm,Agaev:2019kkz,Lu:2020rog,
Tan:2020ldi,Braaten:2020nwp}. Although all nonlattice procedures agree on the deeply bound nature of 
the doubly bottom tetraquarks, the predictions for bottom-charm tetraquarks are quite scattered from 
being unbound to deeply bound. An elaborate account of different nonlattice calculations and their 
predictions can be found in reviews \cite{Lebed:2016hpi,Richard:2016eis,Ali:2017jda}. Results from 
the three lattice QCD investigations for this tetraquark system are also inconclusive. The authors of 
Ref. \cite{Francis:2018jyb} reported a `strong'ly stable bottom-charm tetraquark. However, in a later 
publication, with a refined analysis and using a relatively large volume ensemble no such state was 
found \cite{Hudspith:2020tdf}. In Ref. \cite{Meinel:2022lzo}, no consistent negative energy shifts between 
the interacting and noninteracting finite-volume spectra was found across the various ensembles used. 
Consequently, they could not arrive at robust conclusions on whether a bound $bc\bar u\bar d$ tetraquark 
exists or not. 

In this work, we perform a lattice QCD simulation of coupled $DB^*$ and $BD^*$ two-meson 
channels\footnote{We work in the isosymmetric limit with no QED effects and $m_u=m_d$. Hence we choose 
to call the degenerate ($D^+B^-,~D^0\overline{B}^0$) threshold as $DB$, and equivalently for others 
like $DB^*$, $BD^*$ and $D^*B^*$.} that are the lowest two strong decay thresholds, in the order of 
increasing energies, relevant for isoscalar axialvector $bc\bar u\bar d$ tetraquarks. Our study 
includes five different light quark masses $m_{u/d}$ (approximately corresponding to the pseudoscalar 
meson masses $M_{ps}=$0.5, 0.6, 0.7, 1.0 and 3.0 GeV) for which we generate two point correlation 
matrices. Focusing on the interacting ground state energies extracted variationally from these 
correlation matrices, we search for any statistically significant energy shifts with respect to 
the noninteracting $DB^*$ threshold. The $DB^*$ scattering amplitudes are extracted following 
L\"uscher's finite-volume prescription within an elastic $DB^*$ assumption and are then extrapolated 
to the continuum limit. The light quark mass $m_{u/d}$ dependence of these continuum extrapolated 
amplitudes are studied to determine the fate of a possible $bc\bar u\bar d$ tetraquark pole in 
the $DB^*$ scattering amplitude at the physical pion mass. Towards the heavy quark limit, we also 
determine the critical pseudoscalar meson mass $M^*_{ps}$ at which $bc\bar u\bar d$ tetraquark 
would become unbound. 

The paper is organized as follows. In Section \ref{sec:lattice}, we briefly discuss our lattice setup, 
covering the ensembles and the quark actions used and the $b/c/s$ quark mass tuning.  Section 
\ref{sec:2ptIO} addresses our two point correlation matrix measurements and the interpolating operators 
used. Our results for the finite-volume energy spectrum, the operator-state-overlaps and operator basis 
dependence are presented in Section \ref{fvresults}. In Section \ref{Ampfits}, we elaborate on the 
amplitude fits, the followed continuum extrapolations, and chiral extrapolations. We discuss our 
findings, along with a brief discussion of other existing lattice results, in Section \ref{discuss}.  
Finally, we summarize the work  in Section \ref{summary}.






