%%%%%%%%%%%%%%%%%%%%%%%%%%%%%%%%%%%%%%%%%%%%%%%%%%%%%%%%%%%%%%%%%%%%%%%%%%%%%%%%%
\section{Discussion}
        \label{discuss}

We now discuss our results that we obtained, as mentioned in the previous section, in the perspective of other results obtained using lattice QCD. 
Two of the recent lattice QCD investigations \cite{Hudspith:2020tdf,Meinel:2022lzo} indicated an 
unbound isoscalar bottom-charm four quark system, although authors of Ref. \cite{Meinel:2022lzo} 
could not completely rule out the possibility of a weakly bound state. Note that in a previous 
calculation using a relatively small volume lattice QCD ensemble \cite{Francis:2018jyb}, the 
authors of Ref. \cite{Hudspith:2020tdf} reported a weakly bound state with binding energy 
between 15 to 61 MeV. The use of lattices with relatively coarse lattice spacings, larger 
uncertainties resulting in statistically zero energy shifts, and lack of a rigorous finite-volume 
amplitude analysis might have played significant role in concealing the physics in this 
bottom-charm system. In this work, we have tried to overcome these problems: our lattice 
setup involved four different lattice spacings, two different volumes, the wall smearing 
for quark sources leading to clean ground state signals, 
a variational analysis to extract the finite-volume energy spectra, L\"uscher's rigorous finite-volume 
procedure, the continuum extrapolation and the extraction of ($m_{u/d}$) dependence to precisely 
determine the scattering length at $m_{u/d}^{phys}$ (see \eqn{scatlen}). 
These have helped us to arrive at the first definitive conclusion on the existence of a real bound 
$bc\bar u\bar d$ tetraquark, with a binding energy $-43(^{+6}_{-7})(^{+14}_{-24})$ MeV. We believe 
our result will motivate the community to pursue more rigorous calculations involving a large set of 
interpolating operators, and multiple ensembles having multiple lattice volumes and spacings.


With our results it is now interesting to make a comparative study of the 
meson-meson interactions in different doubly heavy quark systems ($T_{bb}$, $T_{bc}$ and $T_{cc}$). 
In \fgn{scatlencomp}, we present the inverse scattering length ($1/a_0$) at the physical $M_{ps}$ 
determined for these three exotic systems from different lattice calculations \cite{Leskovec:2019ioa,
Lyu:2023xro,Aoki:2023nzp} and ours. The subscripts $(H)$ and $(L)$ in the $x$-axis tick labels refer 
to two distinct procedures, the HALQCD and L\"uscher-type finite-volume prescription followed respectively, in 
extracting the scattering length. The HALQCD procedure followed in Refs. \cite{Lyu:2023xro,Aoki:2023nzp} 
provides quite precise estimates, whereas the large uncertainty in the $BB^*$ scattering using
L\"uscher-type procedure obscures extracting a possible trend, if any exist. The large uncertainties 
in the estimates for $BB^*$ from L\"uscher-type procedure \cite{Leskovec:2019ioa} can be seen to be 
originating from the near threshold first excited state, which is the one of the two states used to 
constrain the two effective range parameters in the physical pion mass ensemble. Subduing these 
uncertainties require more finite-volume energy levels to constrain the amplitudes, which can be 
achieved either by extracting higher excited states, or by studying more ensembles at different 
volumes or at nonzero lab frame momenta. In short, more followup studies involving rigorous 
L\"uscher-type finite-volume treatments with precise estimates are highly desirable to make concrete 
procedure-independent statements on the bindings in different doubly heavy systems. 

\begin{figure}[h]
% Figure removed 
\caption{The inverse scattering length ($1/a_0$) in $DD^*$, $DB^*$ and $BB^*$ scattering at 
the physical pion mass as determined in Refs. \cite{Lyu:2023xro,Leskovec:2019ioa,Aoki:2023nzp} 
and in this work.}
\eef{scatlencomp}

Finally we would like to point out that this study is limited to the rest frame and focuses primarily 
on the ground states in the finite-volume. 
The determination of an effective range (within the elastic assumption) requires at least a 
reliable first excited state with possible dominant operator-state overlap with a $DB^*$ type operator 
with non-zero intrinsic meson momenta. Inclusion of such an operator in our wall source setup is not possible. 
Despite precise estimates for the ground states, for the preceding 
reasons, it is also difficult to comment on the bound state consistency check proposed in Ref. 
\cite{Iritani:2017rlk}. Future works involving construction of symmetric correlation matrices in rest 
as well as moving frames and those using bilocal two-meson interpolators with internal meson momenta 
would be very important step ahead \cite{Padmanath:2018tuc,Padmanath:2022cvl,Chen:2022vpo}. In this work, our main 
strategy has been to determine the signature of scattering length in $DB^*$ interactions at the 
physical pion mass $a_0^{phys}$, which indicates the strength of the interactions to host a real bound 
state. With $a_0^{phys}$ value presented in \eqn{scatlen}, it is evident that the $DB^*$ interaction 
potential at the physical pseudoscalar meson mass supports the existence of a real bound state with 
a binding energy of about 40 MeV with respect to $E_{DB^*}$, as given in \eqn{betbc}. 



