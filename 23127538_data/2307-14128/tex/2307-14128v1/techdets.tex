\section{Ensembles and fermion actions}\label{sec:lattice}

We use the same computational setup as in several of our previous publications \cite{Junnarkar:2019equ,
Junnarkar:2018twb,Basak:2014kma,Padmanath:2017lng,Basak:2012py,Basak:2013oya,Mathur:2016hsm,
Mathur:2018epb,Mathur:2018rwu,Junnarkar:2022yak,Mathur:2022ovu}, which we briefly summarize below for completeness. Four 
$N_f=2+1+1$ lattice QCD ensembles generated by the MILC collaboration are used in this study \cite{MILC:2012znn}, where
the dynamical quark flavors were simulated using Highly Improved Staggered Quark (HISQ) action on gauge fields 
that respect one-loop, tadpole-improved Symanzik gauge action with tuned coefficients through 
$\mathcal{O}(\alpha_sa^2, n_f\alpha_sa^2)$ \cite{Follana:2006rc}. The charm and strange quark masses are 
tuned to their respective physical values, whereas the dynamical light quarks are chosen such 
that $m_s/m_l\sim 5$. We list the relevant details of various lattice QCD ensembles used in \tbn{lattice}.

\bet[tbh]
  \begin{center}
	  \begin{tabular}{p{1.5cm}p{1.5cm}p{1.5cm}>{\hfill\arraybackslash}p{1.5cm}}
      \hline
Label & Symbol & $a~[fm]$     & $N_s^3\times N_t$ \\ \hline
$S_1$ & \pmb{\textcolor{red}{\tikz{\pgfsetplotmarksize{0.8ex}\pgfuseplotmark{diamond}}}} & 0.1207(11)   & $24^3\times64$ \\
$S_2$ & \pmb{\textcolor{magenta}{\tikz{\pgfsetplotmarksize{0.8ex}\pgfuseplotmark{pentagon}}}} & 0.0888(8)    & $32^3\times96$ \\
$S_3$ & \pmb{\textcolor{blue}{\tikz{\pgfsetplotmarksize{0.7ex}\pgfuseplotmark{o}}}} & 0.0582(4)    & $48^3\times144$ \\
$L_1$ & \pmb{\textcolor{OliveGreen}{\pgfsetplotmarksize{0.7ex}\tikz{\pgfuseplotmark{square}}}} & 0.1189(9)    & $40^3\times64$ \\   \hline
  \end{tabular}
  \end{center}
\caption{Relevant details of the lattice QCD ensembles used. The lattice spacing estimates 
are measured using the $r_1$ parameter \cite{MILC:2012znn}. $L$ in $L_1$ refers to large spatial volume, 
and $S$ in $S_1,~S_2$, and $S_3$ refer to small spatial volume. }
\eet{lattice}

The valence quark fields for the light, strange and charm flavors are realized using an overlap 
fermion action that is $\mathcal{O}(am)$ improved. To this end, we utilize the numerical 
implementation of the overlap action following Refs. \cite{Chen:2003im,xQCD:2010pnl}. Following 
the Fermilab prescription \cite{El-Khadra:1996wdx}, the bare charm quark mass on each ensemble was tuned 
using the kinetic mass of spin averaged $1S$ charmonia $\{a\overline M_{kin}^{\bar cc} = 0.75 aM_{kin}(J/\psi) + 0.25 aM_{kin}(\eta_c)\}$
determined for the respective ensembles. Further details on the tuning of charm quark mass, 
the tuned bare quark mass, and resulting discretization effects are discussed in Refs. \cite{Basak:2012py,Basak:2013oya}.
The bare strange quark mass is set by equating the lattice estimate for the fictitious pseudoscalar $\bar ss$ 
meson mass to 688.5 MeV \cite{Chakraborty:2014aca}. Additionally, we perform the quark propagator 
measurements in the valence sector using overlap fermion action for three other quark masses in 
all the ensembles corresponding to pseudoscalar masses of approximately 0.5, 0.6 and 1.0 GeV. 

We employ a nonrelativistic QCD (NRQCD) Hamiltonian \cite{Lepage:1992tx} for the bottom quark. 
We tuned the bottom quark mass using the Fermilab prescription \cite{El-Khadra:1996wdx}, by equating 
the lattice extracted kinetic mass of the spin averaged 1S bottomonia $\{\overline M_{kin}^{\bar bb} = 0.75 M_{kin}(\Upsilon) + 0.25 M_{kin}(\eta_b)\}$
to its experimental value, where the kinetic mass is evaluated from the dispersion relation 
$aM_{kin}^2 = ((ap)^2 - (a\Delta E)^2)/2a\Delta E$. The details of NRQCD Hamiltonian, the improvement 
coefficients, and bottom quark mass tuning on our setup are discussed in Ref. \cite{Mathur:2016hsm}.

\bef[tbh]
% Figure removed
\caption{A landscape plot of the pseudoscalar masses corresponding to the quark mass that we have utilized 
in this work for different lattice ensembles used. The horizontal gray bands indicate a representative 
$M_{ps}$ estimate to guide the eye for a similar pseudoscalar meson mass across all four ensembles.} 
\eef{mpiVslat}

In this work, we assume isospin symmetry ($m_u = m_d$), and then for the channel that study here, 
involves three quark masses: the bottom ($b$), the charm ($c$), and the light ($u/d$) quarks.
For the light quark mass, we investigate five 
different cases: three unphysical quark masses discussed above [referred in terms of the 
corresponding approximate pseudoscalar meson masses $M_{ps}\sim$0.5, 0.6, and 1.0 GeV], the 
strange quark mass [$M_{ps}\sim$0.7 GeV] and the charm quark mass [$M_{ps}\sim$3.0 GeV]. In 
\fgn{mpiVslat}, we present the landscape of the five light quark masses studied in terms of the 
corresponding $M_{ps}$ versus the ensembles used. Using this setup, we evaluate the finite-volume spectrum in the 
isoscalar axialvector channel with $bc\bar u\bar d$ flavor for all these five quark masses on all 
four ensembles, next investigate the scattering of $D$ and $B^*$ mesons in all five scenarios and then 
extract the $m_{u/d}$ (otherwise $M_{ps}$) dependence of the scattering parameters. We utilize a wall-smearing procedure for all 
our quark propagator measurements (see Refs. \cite{Mathur:2018epb,Junnarkar:2018twb,Mathur:2022ovu} 
for details), and our primary focus on the finite-volume spectrum is on the ground state in each case. 




