
%%%%%%%%%%%%%%%%%%%%%%%%%%%%%%%%%%%%%%%%%%%%%%%%%%%%%%%%%%%%%%
\section{$\mathbf{DB^*}$ scattering amplitude}\label{Ampfits}
%%%%%%%%%%%%%%%%%%%%%%%%%%%%%%%%%%%%%%%%%%%%%%%%%%%%%%%%%%%%%%

\subsection{Strategy}

The finite-volume energy splittings determined in the previous section are related to 
the infinite-volume scattering physics via L\"uscher's finite-volume prescription 
\cite{Luscher:1990ux} and its generalizations, e.g. \cite{Briceno:2014oea}. Assuming 
these energy splittings are purely described by an elastic scattering in the $DB^*$ 
system, we utilize them to constrain the associated $S$-wave scattering amplitude. Here 
we consider only the ground states in all ensembles for all quark mass scenarios, as 
the excited states are found to be affected by the inelastic $BD^*$ channel. 

It is interesting that even the excited states are also found to have statistically 
significant shifts with respect to the inelastic $BD^*$ threshold (see \fgn{spectrum}), 
possibly indicating nontrivial interactions between $B$ and $D^*$ mesons. If the $DB^*$ 
and $BD^*$ channels were totally decoupled, such shifts point to equivalent interactions 
in both channels \cite{bdsbc}. However, independent elastic analysis for the excited 
states is not well justified. On the other hand, the inclusion of excited states in our 
analysis demands an inelastic treatment involving more parameters than the available 
degrees of freedom in the amplitude fits, which is beyond the scope of this work. We 
also assume only negligible effects from higher partial waves or any off-shell pion 
exchange interactions that can induce coupling between $DB^*$ and $BD^*$ channels 
\cite{Du:2023hlu}, for the same reason. 

\subsection{Amplitude fits and continuum extrapolations}
For the scattering of a $D$ and a $B^*$ meson in the $S$-wave leading to total angular 
momentum and parity $J^P=1^+$, the scattering phase shifts $\delta_{l=0}(k)$ are related to 
the finite-volume energy spectrum through \cite{Luscher:1990ux}:
\beq
kcot[\delta_0(k)] = \frac{2Z_{00}[1;(\frac{kL}{2\pi})^2)]}{L\sqrt{\pi}},
\eeq{luscher}
where $k$ is the momentum of either mesons in the center of momentum frame corresponding 
to the center of momentum energy $E_{cm}=\sqrt{s}$. $k$ and $E_{cm}$ are related to each other through 
\beq
4sk^2 = (s-(M_{D}+M_{B^*})^2)(s-(M_{D}-M_{B^*})^2).
\eeq{k2cm}
A sub-threshold pole singularity in the $S$-wave scattering amplitude $t = ({\mathrm{cot}}\delta_0 - i)^{-1}$ 
occurs when $k{\mathrm{cot}}\delta_0 = \pm\sqrt{-k^2}$ 
\bet[hb]
  \begin{center}
          \begin{tabular}{p{2.0cm}p{2.0cm}p{2.0cm}>{\hfill\arraybackslash}p{2.cm}}
      \hline
      \hline
$M_{ps}$ [GeV] & $\chi^2/d.o.f$ & $A^{[0]}/E_{DB^*}$ & $A^{[1]}/E_{DB^*}$ \\\hline
\multirow{2}{*}{0.5} & 2.1/2 & $-0.05(1)$ & $~0.17(_{-11}^{+13})$ \\\cline{2-4} 
                     & 1.3/1 & $-0.05(1)$ & $~0.13(_{-12}^{+13})$ \\ \hline
\multirow{2}{*}{0.6} & 0.5/2 & $-0.044(_{-8}^{+9})$ & $~0.10(_{-9}^{+9})$ \\ \cline{2-4} 
                     & 0.3/1 & $-0.043(_{-8}^{+9})$ & $~0.09(_{-10}^{+9})$ \\ \hline
\multirow{2}{*}{0.7} & 3.0/2 & $-0.042(_{-6}^{+8})$ & $~0.09(_{-7}^{+6})$ \\ \cline{2-4} 
                     & 1.5/1 & $-0.040(_{-6}^{+8})$ & $~0.06(_{-8}^{+6})$ \\ \hline
\multirow{2}{*}{1.0} & 2.9/2 & $-0.043(4)$ & $~0.11(_{-5}^{+5})$ \\ \cline{2-4} 
                     & 0.4/1 & $-0.041(4)$ & $~0.14(_{-4}^{+5})$ \\ \hline
\multirow{2}{*}{3.0} & 3.6/2 & $~0.006(_{-5}^{+6})$ & $-0.20(_{-5}^{+4})$ \\ \cline{2-4} 
                     & 1.9/1 & $~0.010(_{-5}^{+6})$ & $-0.25(_{-5}^{+4})$ \\ \hline
      \hline
  \end{tabular}
  \end{center}
\caption{Results from amplitude fits for different light quark mass scenarios indicated 
in terms of $M_{ps}$ in the first column. For each $M_{ps}$, two independent fits are 
performed with (top row) and without (bottom row) the level from $S_1$ ensemble. All fits 
are performed with the parameterization in \eqn{linparam}, where the optimized parameter 
values in the table are presented in units of the $DB^*$ threshold, $E_{DB^*}$. }
\eet{Ampfits1}
for scattering in $S$-wave. We follow the procedure outlined in Appendix B of Ref. 
\cite{Padmanath:2022cvl} in constraining the amplitude, such that the parametrization of 
$k{\mathrm{cot}}\delta_0$ is tuned to satisfy Eq. (\ref{luscher}). The parametrized 
$k{\mathrm{cot}}\delta_0$ is then investigated for poles of $t$ in the complex energy plane.  

\begin{figure}[h]
% Figure removed
\caption{$k{\mathrm{cot}}\delta_0$, in units of the elastic threshold $E_{DB^*}$, versus $a$ 
(lattice spacing) for all $M_{\pi}$ values. We follow the marker/color coding in \tbn{lattice} 
for the data points referring to the simulated data. The colored/gray bands indicate the fit 
results to the continuum extrapolation fit form in \eqn{linparam} with/without the data from 
$S_1$ ensemble.} 
\eef{alatdep}
Since we use only the ground states for amplitude fits, we limit ourselves to a scattering 
amplitude parametrization that is completely described by scattering length $a_0$ in an effective 
range expansion near the threshold. Additionally, we also consider a lattice spacing dependence 
on the parametrization of $k{\mathrm{cot}}\delta_0$. We find that a linear functional form
given by 
\beq
k{\mathrm{cot}}\delta_0 = A^{[0]} + aA^{[1]}
\eeq{linparam}
provides acceptable fits to the scattering amplitudes. Such an $a$ dependence was also found to 
be necessary in our previous investigations using NRQCD framework as well \cite{Mathur:2022ovu}, 
and is consistent with the leading $a$ dependence of observables
involving an NRQCD evolution. In this form, $A^{[0]}=-1/a_0$, where $a_0$ is the scattering 
length in the continuum limit. 
We list the results from different amplitude fits in \tbn{Ampfits1}. In \fgn{alatdep}, we present 
the quality of these fits by comparing the fit results with the data points. The colored/gray 
bands indicate the fit results including/excluding results from the $S_1$ ensemble to the respective 
fits. It can be clearly seen that fit results are less affected by inputs from $S_1$ ensemble, 
which is obvious given the large uncertainties associated with them, in contrast to inputs from 
$L_1$, $S_2$, and $S_3$. In \fgn{pcotdelta_summary}, we present $k{\mathrm{cot}}\delta_0$ versus 
$k^2$ based on the ground state energies presented in \fgn{gsspectrum} following \eqn{luscher}. 
The colored/gray bands indicate continuum extrapolated results including/excluding results from 
the $S_1$ ensemble to the respective fits. Clearly, there are no statistically significant effects
from the inclusion/exclusion of the energy levels from the $S_1$ ensemble observed.
\begin{figure}[h]
% Figure removed
\caption{$k{\mathrm{cot}}\delta_0$ versus $k^2$ for all $M_{\pi}$ values studied in units of 
the elastic threshold $E_{DB^*}$. The data points refer to the simulated data and follow 
the color coding in \tbn{lattice}. The dashed orange (cyan) curve indicates the constraint 
for the existence of a sub-threshold pole in the scattering amplitude. The horizontal bands
are the continuum extrapolated estimates of $k{\mathrm{cot}}\delta_0$ for the respective
$M_{\pi}$ (see \fgn{alatdep}). }
\eef{pcotdelta_summary}

Our main aim is to reliably determine $A^{[0]}=-1/a_0$, the sign of which determines the fate 
of the near threshold pole, if there exists one. A negative (positive) value of $A^{[0]}$($a_0$) 
indicates that the interaction potential is strong enough to form a real bound state\cite{Landau:1991wop}. 
It can be seen from \tbn{Ampfits1} and \fgn{alatdep} that for the non-charm light quark masses, 
$A^{[0]}$, the continuum extrapolated value for $k{\mathrm{cot}}\delta_0$ is negative, which 
indicates a possibly strong attractive interaction sufficient enough to host a real bound state. 
Whereas at the charm point, despite the unambiguous negative energy shifts in the finite-volume 
ground state energies with respect to the elastic threshold, the attraction is weak to host any 
real bound state as suggested by the positive value of $k{\mathrm{cot}}\delta_0$ in the continuum 
limit. This observation goes in line with the phenomenological expectation for doubly heavy four 
quark ($QQ'l_1l_2$) systems with $m_{l_1}=m_{l_2}$ that the binding increases with increased 
relative heaviness of the heavy quarks with respect to its light quark content
\cite{Francis:2016hui,Czarnecki:2017vco,Junnarkar:2017sey}. 

Another interesting observation is related to the lattice spacing dependence of $k{\mathrm{cot}}\delta_0$
values. At the charm point, $A^{[1]}$ (see \eqn{linparam}) acquires a different signature in contrast 
to that for the light quark masses. This suggests that for a doubly heavy four quark ($QQ'l_1l_2$) system 
with $(m_{l_1} = m_{l_2}, ~m_{Q},m_{Q'}>>m_{l})$, the cut off effects weaken the finite-volume energy 
splitting of the ground state with the elastic threshold. On the other hand, at the charm point (where 
$m_{Q},m_{Q'}\sim m_{l}$) such effects enhance this energy splitting in the $QQ'l_1l_2$ system determined 
in a finite-volume. Relatively large errors at the noncharm $M_{ps}$ values partially obscure these effects, 
if any exist, while at the charm point such effects are clearly reflected. 

\subsection{Light quark mass dependence}
Following the individual amplitude fits to different light quark mass cases, now we investigate the light 
quark mass ($m_{u/d}$) or $M_{ps}$ dependence of the parameters $A^{[0]}$ and $A^{[1]}$. Due to leading order 
$M_{ps}^2$ terms in the chiral expansion, we assume the $M_{ps}$ dependence of hadron masses for 
light $m_{u/d}$ values ($m_q\lesssim\Lambda_{QCD}$) to be linear in $M_{ps}^2$. Whereas towards the heavy 
$m_{u/d}$ regime ($m_q>>\Lambda_{QCD}$) heavy hadron masses are expected to be proportional to the quark mass, 
hence to $M_{ps}$ \cite{Neubert:1993mb}. With these assumptions, we work with three following fit forms that 
could be useful. 
\beqa
	f_l(M_{ps}) &=& \alpha_c + \alpha_l M_{ps}, \nonumber \\
	f_s(M_{ps}) &=& \beta_c + \beta_s M_{ps}^2, \mbox{~~~and} \nonumber \\
	f_q(M_{ps}) &=& \theta_c + \theta_l M_{ps} + \theta_s M_{ps}^2.
\eeqa{mqdep}
Fits to determine the $M_{ps}$ dependence were made by minimizing a single cost function 
defined combinedly for $A^{[0]}$ and $A^{[1]}$ as %\cite{FullLuscher}
\beq
	\chi^2 =\sum_{\substack{x, y \\ \in \{A^{[j]}_{i}\}}}\left(f_x-f_{px}(M_{ps})\right)\tilde{\mathcal{C}}^{-1}_{xy}\left(f_y-f_{py}(M_{ps})\right),
\eeq{chi2mqdep}
where the summation runs over all fitted parameters $\{A^{[j]}_{i}\}$ with $j\in\{0, 1\}$ and $i$ 
referring to the five different light quark masses studied. In \tbn{Ampfits1}, we list the fit results 
for $f_{x,y}$. $\tilde{\mathcal{C}}_{ij}$ is the associated data covariance determined following Ref. \cite{Prelovsek:2020eiw}. 
$f_{pn}(M_{ps})$ are the fit forms incorporating the $m_{u/d}$ dependence in parameters $\{A^{[j]}_{i}\}$.  
In \fgn{a0a1_separate}, we show the fit results for $A^{[0]}=-1/a_0$ to the fit forms in \eqn{mqdep}. The large 
circles represent the $A^{[0]}$ values at different $M_{ps}$, the bands represent the fit results 
with different fit forms in \eqn{mqdep}, and the two stars represent $A^{[0]}$ at the physical 
$M_{ps}$ (equivalently the physical scattering length $a_0^{phys}$) and the critical $M_{ps}$ at which 
$A^{[0]}$ changes its sign (positive to negative), in other words, the system becomes unbound. It is 
indeed desired to have more points in the intermediate mass regime between the charm and the strange 
\begin{figure}[h]
% Figure removed
\caption{Continuum extrapolated $k{\mathrm{cot}}\delta_0$ or $A^{[0]}=-1/a_0$ estimates of the $DB^*$ system 
as a function of $M_{ps}^2$ in units of $E_{DB^*}$. The band indicates fit results to the simulated results. 
The legend carries info on the fit forms presented (see also \eqn{mqdep}) and the quality of fits. The dotted 
vertical line close to the $y$-axis indicates the physical $M_{ps}$. The two star symbols represent the 
amplitude at the physical $M_{ps}$ and the critical $M_{ps}$ at which the system becomes unbound.}
\eef{a0a1_separate}
quark masses to further constrain the dependence. Yet, our fits in this work demonstrate near independence in 
the fit forms as can be observed from the consistency between the error bands from different fit 
forms. 

\begin{figure}[h]
% Figure removed
\caption{The landscape of the continuum scattering length $A^{[0]}$ versus $A^{[1]}$ (see \eqn{linparam}) 
for all $M_{ps}$ values (indicated in the legend) studied. The central values are represented by black 
edged circles with color fillings, whereas the scattered points are the bootstrap samples. The band 
represents the correlated $M_{ps}$ dependence of the fitted parameters.} 
\eef{a0a1_combined}
Next we look at the correlated pion mass dependence in the parameters $A^{[0]}$ and $A^{[1]}$ (see 
\eqn{chi2mqdep} for the definition of the cost function) presented in \fgn{a0a1_combined}. The black 
bordered symbols are the central values of parameters determined for each $M_{ps}$, whereas scattered 
small circles indicate the bootstrap sample distribution in the $[A^{[0]},~A^{[1]}]$ landscape. The bands 
in the figure represent the uncertainty in the parameters, with the inner band quantifying the statistical 
errors, while the outer band also incorporates the systematic uncertainty arising from different fit forms 
added in quadrature symmetrically. A negative correlation can clearly be observed between the parameters 
across different quark masses studied, which is accounted in the fits through the data covariance matrix 
entering the cost function. This correlation can also be observed within the distribution of the bootstrap 
samplings at all quark masses. This observation clearly demonstrates the need for a careful treatment of 
cutoff errors, particularly in heavy hadron systems with interesting near threshold features, such as this. 

In the chiral regime ($m_{u/d}\lesssim\Lambda_{QCD}$), leading $m_{u/d}$ dependence in hadronic observables 
is assumed to go as linear in $M_{ps}^2$. Based on the fit form $f_s(M_{ps})$, we find that the scattering length 
of the $DB^*$ system at the physical light quark mass ($m_{u/d}^{phys}$) to be
\beq
a_0^{phys} = 0.57(^{+4}_{-5})(17) \mbox{~fm}.
\eeq{scatlen}
The asymmetric errors indicate the statistical uncertainties, whereas the second parenthesis quotes 
the systematic uncertainties with the most dominant contribution arising from the chiral extrapolation 
fit forms. We elaborate on various systematic uncertainties towards end of this section. The positive 
value of the scattering length is an unambiguous evidence for the ability/strength of the hadron-hadron 
interaction potential to host a real bound state (when $k~{\mathrm{cot}}\delta_0 = -\sqrt{-k^2}$). 
The observed scattering length at physical light quark mass suggests the presence of a real $bc\bar u\bar d$ 
tetraquark bound state $T_{bc}$ with binding energy 
\beq
\delta m_{T_{bc}} = -43(^{+6}_{-7})(^{+14}_{-24}) \mbox{~MeV},
\eeq{betbc}
with respect to $E_{DB^*}$. The systematic effects on the $a_0^{phys}$ and $\delta m_{T_{bc}}$ estimates 
of ignoring the charm point in the fits to the $m_{u/d}$ dependence are found to be very small, 
compared to the number quoted for systematic uncertainties in \eqn{scatlen}. 

Towards the heavy quark regime ($m_{u/d}>>\Lambda_{QCD}$), the heavy hadron masses can have 
leading linear dependence in $M_{ps}$ as $M_{ps}\propto m_{u/d}\sim m_{Q}$ \cite{Neubert:1993mb}. 
Following the fit form $f_l(M_{ps})$, which is linear in quark mass, the critical light quark mass 
$m_{u/d}^*$ at which the scattering length diverges, then changes its signature such that the 
interaction potential is not able host a real bound state, corresponds to the critical pseudoscalar 
meson mass given by 
\beq
M^{*}_{ps} = 2.73(21)(14) \mbox{~GeV}.
\eeq{unitary}
This corresponds to the star symbol at the zero crossing in the $x$-axis ($A^{[0]}=0$) in \fgn{a0a1_separate}. 
Once again the first parenthesis indicates the statistical errors and the second one quantifies various 
systematic uncertainties added in quadrature. 


Now we briefly comment on other possible sources of systematic uncertainties in this calculation. Our lattice setup, 
discussed in Section \ref{sec:lattice}, together with the bare bottom and charm quark mass tuning procedure 
has been demonstrated to reproduce the $1S$ hyperfine splittings in quarkonia with uncertainties less than 6 MeV
\cite{Mathur:2022ovu,Mathur:2016hsm}. Additionally, our strategy of evaluating the energy differences 
and working with mass ratios has also been shown to significantly mitigate the systematic uncertainties related 
heavy quark masses \cite{Mathur:2018epb,Mathur:2022ovu}. Our fitting procedure discussed in Section \ref{sec:2ptIO} 
involves careful and conservative determination of statistical errors, and uncertainties related to the 
excited-state-contamination and fit-window errors. The amplitude determination and followed extrapolations
are performed with results from varying the fit-windows to evaluate the uncertainties propagated to our final 
results. The uncertainties related to the fit forms used in chiral extrapolations are observed to be dominant,  
and the number in the second parenthesis in Eqs. \ref{scatlen}, \ref{betbc}, and \ref{unitary} are the total 
systematic uncertainties added in quadrature. Uncertainty related to scale setting are also found to be negligibly 
small in comparison to the statistical uncertainties \cite{Mathur:2018epb,Mathur:2022ovu}. 

