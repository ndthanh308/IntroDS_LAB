%%%%%%%%%%%%%%%%%%%%%%%%%%%%%%%%%%%%%%%%%%%%%%%%%%%%%%%%%%%%%%%%%%%%%%%%%%%%%%%%%
\section{Summary}
        \label{summary}

We have performed a lattice QCD simulation of coupled $DB^*$-$BD^*$ scattering with explicitly exotic flavor 
$bc\bar u\bar d$ and isoscalar axialvector quantum numbers [$I(J^P) = 0(1^+)$]. To this end, we utilize 
four lattice QCD ensembles generated by the MILC Collaboration with different lattice spacing, different 
volume, and $N_f=2+1+1$ dynamical quark fields realized with a HISQ action. The valence quarks with masses 
up to the charm quark mass are realized with an overlap action. For the bottom quarks, we use a lattice 
NRQCD formulation. 

Our finite-volume spectra are presented in Figures \ref{fg:spectrum} and \ref{fg:gsspectrum}, where the 
latter shows the ground state spectrum. Following a rigorous 
finite-volume amplitude analysis, we extract the elastic $DB^*$ scattering amplitudes, parametrized with 
a lattice spacing dependence, in the continuum limit for the five light quark masses studied (see 
\fgn{mpiVslat}). The quark mass dependence is then investigated to determine the elastic $DB^*$ scattering 
length $a_0^{phys}$ at the physical light quark mass and the critical pseudoscalar mass $M^{*}_{ps}$ at 
which $a_0$ diverges. Our estimate for $a_0^{phys}$, presented in \eqn{scatlen}, is positive, indicating an 
attractive interaction between the $D$ and $B^*$ mesons, which is strong enough to host a real bound state 
with binding energy $\delta m_{T_{bc}} = -43(^{+6}_{-7})(^{+13}_{-23})$ MeV. We find that the strength of 
interaction is such that this $bc\bar u\bar d$ tetraquark becomes unbound at $M^{*}_{ps}$, which is close 
to the $\eta_c$ meson mass (see \eqn{unitary}). 

In contrast to the inconclusive results from previous lattice calculations \cite{Francis:2018jyb,Hudspith:2020tdf,
Meinel:2022lzo}, we observe unambiguous negative energy shifts between the interacting and noninteracting 
finite-volume energy levels in the isoscalar axialvector quantum channel. In the physical limits, this leads 
to positive scattering length in elastic $DB^*$ scattering, which hints the possible existence of a real bound 
$T_{bc}$ state. We make several important steps ahead, with respect to the existing calculations, to arrive at 
robust inferences. Yet, future improvised investigations are desired to affirm our findings. Fully dynamical 
simulations on several more ensembles, with different fermion actions, high statistics studies with lighter 
$m_{u/d}$, etc. are a few other improvisations that can constrain the relevant scattering amplitude in 
a framework-independent way. 

We conclude this article with a brief remark on the experimental prospects relevant to this work. Bottom hadrons 
are known for low production rates in current day experiments due to the large centre of momentum energy requirements 
in producing the heavy $b$ quarks. Naturally, doubly heavy tetraquarks including bottom-charm tetraquarks are 
likely to have even low production rates. The recent discoveries of $\Xi_{cc}$ \cite{LHCb:2017iph}, $T_{cc}$ 
\cite{LHCb:2021vvq}, reports of tri-$J/\psi$ \cite{CMS:2021qsn}, associated $J/\psi\Upsilon$ \cite{LHCb:2023qgu}, 
and di-$\Upsilon$ \cite{CMS:2016liw} productions, and recent proposals of inclusive search strategies 
\cite{Gershon:2018gda,Qin:2021zqx} augment promising prospects for sooner discoveries in the bottom-charm sector.
We hope that our observations and inferences in this work will further motivate experimental searches for such 
states in future.
 







