\documentclass[%
 reprint,
superscriptaddress,
%groupedaddress,
%unsortedaddress,
%runinaddress,
%frontmatterverbose, 
%preprint,
%preprintnumbers,
%nofootinbib,
%nobibnotes,
%bibnotes,
 amsmath,amssymb,
 aps,
%pra,
%prb,
%rmp,
%prstab,
%prstper,
%floatfix,
one column
]{revtex4-2}

\usepackage{graphicx}
\usepackage{graphics}
\usepackage{amssymb}
\usepackage{amsmath}
\usepackage{epsfig}
\usepackage{latexsym}
\usepackage{color}
\usepackage{rotating}
\usepackage{subfigure}
\usepackage{hyperref}
\usepackage{times}
\usepackage[capitalise]{cleveref}
\usepackage[flushleft]{threeparttable}
%\renewcommand{\baselinestretch}{3}
%\usepackage{xspace}

\newcommand{\expect}[1]{\mbox{$\langle #1 \rangle$}}
\newcommand{\ket}[1]{\mbox{$|#1 \rangle$}}
\newcommand{\bra}[1]{\mbox{$\langle#1 |$}}
\newcommand{\half}{\mbox{$\frac{1}{2}$}}
\newcommand{\beq}{\begin{equation}}
\newcommand{\eeq}{\end{equation}}
\newcommand{\bea}{\begin{eqnarray}}
\newcommand{\eea}{\end{eqnarray}}
\newcommand{\bean}{\begin{eqnarray*}}
\newcommand{\eean}{\end{eqnarray*}}
\makeatother

%
\setlength{\abovecaptionskip}{7pt}
 \setlength{\belowcaptionskip}{2pt}
 \setlength{\floatsep}{2pt}
 \setlength{\textfloatsep}{4pt}
%\setlength{\topsep}{0pt}
\setlength{\partopsep}{0pt}

\begin{document}

\title{Supplementary information for massive quantum superpositions using magneto-mechanics}


\author{Sarath Raman Nair}
\thanks{Joint first authors}
\email[Corresponding author : ]{sarath.raman-nair@mq.edu.au}
\affiliation{School of Mathematical and Physical Sciences, Macquarie University, NSW 2109, Australia}
\affiliation{ARC Centre of Excellence for Engineered Quantum Systems (EQUS), Macquarie University, NSW 2109, Australia}

\author{Shilu Tian}
\thanks{Joint first authors}
\affiliation{Quantum Machines Unit, Okinawa Institute of Science and Technology Graduate University, Onna, Okinawa 904-0495, Japan}


\author{Gavin K. Brennen}
\affiliation{School of Mathematical and Physical Sciences, Macquarie University, NSW 2109, Australia}
\affiliation{ARC Centre of Excellence for Engineered Quantum Systems (EQUS), Macquarie University, NSW 2109, Australia}

\author{Sougato Bose}
\affiliation {Department of Physics and Astronomy, University College London, Gower Street, WC1E 6BT London, UK}

\author{Jason Twamley}
\email[Corresponding author : ]{ jason.twamley@oist.jp}
\affiliation{Quantum Machines Unit, Okinawa Institute of Science and Technology Graduate University, Onna, Okinawa 904-0495, Japan}


\vspace{10pt}

\maketitle

\section{Q-factor estimation}
%
% Figure environment removed
%

We assume that the main source of damping of the oscillator in our schemes is the collision of gas molecules surrounding the oscillator onto the oscillator.
%
Then the Quality factor which determines how long the oscillator can coherently oscillate in its ground state can be written generally as \cite{Wang2019DynamicsSuperconductor},
%
\begin{equation}
Q = \frac{\pi}{4} \rho \omega_{\mathrm{z}}^{2}A\frac{V}{\bar{F}_{\mathrm{g}}}
\label{eq:Qeq1}
\end{equation}
where $\rho$, $V$ is the volume of the oscillator respectively, $\omega_{\mathrm{z}}$ and $A$ are the frequency and amplitude of the oscillation respectively.
%
The term \(\bar{F}_{\mathrm{g}}\) is the drag force due to the collision of the oscillator with surrounding molecules.
%
This parameter depends on the geometry of the oscillator \cite{DeLimaBernardo2013DragGas}.%
%
In order to estimate this parameter \(\bar{F}_{\mathrm{g}}\), we first assume that the velocity of the oscillator moving in the gas is much smaller than the surrounding gas molecules as in \cite{Wang2019DynamicsSuperconductor}.
%
Then an equation, if the oscillator is a spherical body the Q-factor is already worked out in \cite{Wang2019DynamicsSuperconductor}.
%
In the case of flux qubit as the oscillator, we make the following assumption. The flux qubit is a toroid in a cross-sectional radius $r$ and a length of $2\pi R$, where $R$ is the radius of the flux qubit. We can assume this toroid as $N$ cylinders, each with a length of $2\pi R/N$. Then we can assume that these $N$ cylinders are moving in the gas along the direction perpendicular to the normal of the cross-sectional area. The collision of the gas molecules happens at the curved surfaces of the cylinders. The relative angle between the cylinders is not expected to make any difference. The total force is the force on each cylinder multiplied by the $N$. Which is equal to treating the toroid as a single cylinder with a length of $2\pi R$.
%
From references \cite{DeLimaBernardo2013DragGas, Wang2019DynamicsSuperconductor}, 
the $V/{\bar{F}_{\mathrm{g}}}$ can be written for both these two cases as,

\begin{equation}
\frac{V}{\bar{F}_{\mathrm{g}}} = \frac{2r}{3}\sqrt{\frac{3k_{\mathrm{B}}T}{M_{\mathrm{g}}}}\frac{1}{P_{\mathrm{g}}\omega_{\mathrm{z}}A}
\label{eq:Qeq2}
\end{equation}
where $P_{g}$ is the pressure of the gas, $k_{\mathrm{B}}$ is the Boltzmann's constant, $T$ is the temperature at which the experiments are performed, $M_{\mathrm{g}}$ is the mass of the gas molecule.
%
Here $r$ is the radius for the spherical resonator and for the toroid resonator, it is the cross-sectional radius.
%
Using \cref{eq:Qeq1} and \cref{eq:Qeq2}, we obtain the Q-factor as,
\begin{equation}
Q = \frac{\pi}{6}\frac{\rho r\omega_{\mathrm{z}}}{P_{\mathrm{g}}} \sqrt{\frac{3k_{\mathrm{B}}T}{M_{\mathrm{g}}}}
\label{eq:Qeq1}
\end{equation}
%
Assuming Helium as the gas is present in the surroundings we estimate the Q-factor for the optically levitated YIG sphere and the magnetically levitated flux qubit and the results are presented in \cref{fig:Q1}.


\section{Analytical modeling of motional superposition of a trapped crystalline magnetic microsphere\label{sec:appendixA}}

We describe the details of analytical modeling of the generation of a macroscopic quantum superposition using the flux-qubit magnetic actuation of a trapped YIG nanoparticle.
%

\subsection{Modelling the flux qubit as Solenoid}

%
% Figure environment removed
%


We model the two flux qubits as superconducting rings as mentioned in the main text.
%
For this we assume that these rings have radii of $R$ each and each of these flux qubits is symmetrically located at a distance $z^{FQ}_\pm=\pm \eta R$, along the z-axis around the origin, where $\eta$ is a dimensionless scaling factor. 
%
The flux qubits have supercurrents $I$ circulating in each ring.
%
We can write down the magnetic field generated by the two superconducting flux qubits at a point on the z-axis $\vec{r}=(0,0,z)$ as,
\begin{equation}
\Vec{B}(z) = \frac{\mu_{0} I R^{2}}{2}\left(\frac{1}{((z-\eta R)^{2}+R^{2})^{3/2}} - \frac{1}{((z+\eta R)^{2}+R^{2})^{3/2}}\right)\hat{z},
\label{eq:sm1}
\end{equation}
where $\mu_{0}$ is the vacuum permeability.
%
The gradient of the magnetic field can be obtained as,
\begin{equation}
\frac{\partial \Vec{B}}{\partial z} = -\frac{3\mu_{0} I R^{2}}{2}\left(\frac{(z-\eta R)}{((z-\eta R)^{2}+R^{2})^{5/2}} - \frac{(z+\eta R)}{((z+\eta R)^{2}+R^{2})^{5/2}}\right)\hat{z},
\label{eq:sm2}
\end{equation}

The numerical solutions for the magnetic field and its gradient from two superconducting flux qubits using equations (\ref{eq:sm1}) and (\ref{eq:sm2}) are shown in Figure \ref{fig:A1}. 
%
From the numerical solutions in \ref{fig:A1}, we can see that the magnetic field varies linearly and the gradient of the magnetic field is constant for small displacements. 
%

At the origin (z = 0) the magnetic field gradient is,
%
\begin{equation}
\frac{\partial \Vec{B}}{\partial z} = \frac{3\mu_{0} I}{R^{2}} \frac{\eta}{(1+\eta^{2})^{5/2}}\hat{z},
\label{eq:3}
\end{equation}
%
We find by maximizing $\eta/(1+\eta^{2})^{5/2}$ that this non-vanishing magnetic field gradient is maximum when the flux qubits are on each side of the YIG sphere with $\eta = 0.5$.

From \cref{fig:A1}, we can see that the magnetic field strength is linearly varying around $z=0$ such that we can approximate \(B \sim \frac{\partial B}{\partial z} z\) for small displacements.



\subsection{Analytical solution for the equilibrium position \label{sec:appendixB}}

We consider a spherical magnet (for example a YIG sphere) trapped in three dimensions.
%
The experimental demonstration of it via magnetic levitation can be found in the references \cite{Wang2019DynamicsSuperconductor}\cite{Gieseler2020Single-SpinMicromagnets}\cite{Vinante2020UltralowMicroparticles}\cite{Timberlake2019AccelerationOscillators}.
%
Assuming the trap to be a harmonic trap along all the axis, we can express the optical force on the z-axis defined in the main text as, \(\vec{F}_{\mathrm{TR}}=-\hat{z} \rho V\omega_{\mathrm{z}}^{2} z \) \cite{Gieseler2012}.
%
The magnetic force on the YIG sphere due to the flux qubit state using dipole model is, \(\Vec{F}^{|q\rangle}_{\mathrm{FQ}} = \vec{m}_{0} \cdot (\partial \vec{B}_z/ \partial z)_{|q\rangle}\).
%
We approximate the YIG nanosphere as a point dipole due to its spherical shape and thus can write its dipole moment as, $\Vec{m}_{0} =~\mu_{0}^{-1}{B_{\mathrm{r}}}V \hat{z}$.
%
Here the $B_{\mathrm{r}}$ is the remnant magnetic field of the YIG sphere.
%
Due to the trapping and magnetic actuation force on the YIG sphere, it moves to a new equilibrium position in the optical trap where it satisfies, $\Vec{F}^{|q\rangle}_{\mathrm{FQ}}+\Vec{F}_{\mathrm{TR}}=0$.
%
We can write down this condition as,

\begin{equation}
\frac{B_{\mathrm{r}}}{\mu_{0}} V  \left(\frac{\partial \vec{B}_z}{\partial z}\right)_{|q\rangle}-\rho V \omega^{2}_{\mathrm{z}} z_{\mathrm{eq}}=0
\label{eq:sm4}
\end{equation}
%
Using equation (\ref{eq:sm2}), for $|0\rangle$, we can write equation (\ref{eq:sm4}) as,
%
\begin{equation}
\frac{3 I B_{\mathrm{r}} R^{2}}{\rho \omega^{2}_{\mathrm{z}}}\left(\frac{(z_{\mathrm{eq}}-(R/2))}{((z_{\mathrm{eq}}-(R/2))^{2}+R^{2})^{5/2}} - \frac{(z_{\mathrm{eq}}+(R/2))}{((z_{\mathrm{eq}}+(R/2))^{2}+R^{2})^{5/2}}\right) + z_{\mathrm{eq}}=0,
\label{eq:sm5}
\end{equation}
where we considered $\eta = 1/2$.
%
Then for the superposition of $|0\rangle$ and $|1\rangle$, we can obtain spatial superpostion extent as, $\Delta z = 2 z_{\mathrm{eq}}$.
%
We can write \cref{eq:sm5} using the transformation $z_{\mathrm{eq}} \rightarrow (\Delta z/2)$ to obtain an equation in terms of \(\Delta z\) for numerical calculations as,
%
\begin{equation}
\frac{3 I B_{\mathrm{r}} R^{2}}{\rho \omega^{2}_{\mathrm{z}}}\left(\frac{((\Delta z/2)-(R/2))}{(((\Delta z/2)-(R/2))^{2}+R^{2})^{5/2}} - \frac{((\Delta z/2)+(R/2))}{(((\Delta z/2)+(R/2))^{2}+R^{2})^{5/2}}\right) + (\Delta z/2)=0.
\label{eq:sm6}
\end{equation}

\subsection{Magnetic field generated by spherical magnet}

%
% Figure environment removed
%

We assumed that the magnetic field produced by the spherical magnet at the location of the superconducting material is below the critical magnetic field of 9.8 mT for Aluminium. To verify this assumption, we have modeled a spherical magnet of radius 25 $\mu m$ (Other parameters are in the \cref{Table1} below), using Radia in Mathematica \cite{Chubar1998ADevices}. 
%
We applied linear anisotropic material for the spherical magnet in the Radia model, with the parallel and perpendicular magnetic field susceptibility of 1 and 0 respectively, with the remnant magnetic field of YIG along the z-axis.
%
We simulated the B-field along the x-axis at z = 30 $\mu m$ and z = 40 $\mu m$ for a sphere magnet at the origin and also for a sphere located at z = 1 $\mu m$ (Please see FIG. 1 in the main text). 
%
The results are shown in \cref{fig:Radia_B-field} and from these results, we can see that the assumption is correct for any shifts within the range presented here.

\section{Trapping superconducting flux qubit using spherical magnet}
In this section, we discuss the details of the scheme to generate spatial superposition using magnetically levitated flux qubit.
%
\subsection{Analytical solution for trapping superconducting flux qubit above the spherical magnet}
We approximate the levitation of flux qubit above the spherical magnet to the levitation of a superconducting ring using a point dipole as the spherical magnet can be approximated as a point dipole.
%
An analytical framework for estimating the trap frequency and equilibrium position for this problem of levitating a superconducting ring by a point dipole has been developed in reference \cite{Navau2021LevitationMagnetomechanics}.
%
We first assume a coordinate system where the magnet is at the origin of the coordinate system and the magnetic trapping axis of the flux qubit is along $+\hat{z}$.
%
Based on the analytical results by reference \cite{Navau2021LevitationMagnetomechanics}, the equilibrium position, $h$
%
for the flux qubit ring in the Meissner state levitated by the spherical micromagnet can then be obtained by solving the equation,
\begin{equation}
%
\frac{16 \pi^{2} r^{2} g \rho}{\mu_{0}^2 m_{mag}^2}R^{5}(\ln{[8R/r]}-2)-\frac{6(h/R)}{(1+(h/R)^2)^4}=0,
\label{eq:equilium position}
\end{equation}
where $R$, $\rho$, and $r$ are the radius, density, and cross-sectional radius of the flux qubit, respectively; and $m_{mag}$ is the magnetic moment of the magnet and $g$ is the gravity acceleration constant.
%
For a stable levitation, the dimensionless term $(16 \pi^{2} r^{2} g \rho/\mu_{0}^2 m_{mag}^2)R^{5}(\ln{[8R/r]}-2)$ in \ref{eq:equilium position}, should be less than 1.329 \cite{Navau2021LevitationMagnetomechanics}.
%
This sets an upper bound for the flux qubit radius $R$ for a fixed cross-sectional radius $r$.
%
Using the equilibrium position obtained from \ref{eq:equilium position}, we can estimate the trap frequency in Hz, along the trapping axis $\nu_{z}$ using the equation \cite{Navau2021LevitationMagnetomechanics},
%
\begin{equation}
\nu_{z}=\sqrt{\frac{3\mu_{0} m_{mag}^{2} (7(h/R)^2-1)}{128\pi^{4}r^{2}\rho R^4(\ln{[8R/r]}-2) (1+(h/R)^2)^5}}.
\label{eq:trap frequency}
\end{equation}


For the easiness of analysis, we shift the origin of the coordinate system from the spherical magnet to the equilibrium position,  $h$ of the levitated flux qubit with the transformation $(x, y, z) \rightarrow (x, y, z + h)$.

Driving the qubit inductively a supercurrent flow can be induced that we denote as $I$.
%
Then depending on the direction of flow qubit can be put into $|0\rangle$ or $|1\rangle$ state in the levitated flux qubits. 
%
Then the flux qubit acquires a resultant dipole moment, $\vec{m}_{\mathrm{sc}}= \hat{z} I \pi R^{2}$ ($\vec{m}_{\mathrm{sc}}= -\hat{z} I \pi R^{2}$),  which points downward (upward) the qubit is in the $|0\rangle$ ($|1\rangle$) state with $I$ flows clockwise (counterclockwise) direction. 
%
The levitated flux qubit with the magnetic moment $\vec{m}_{\mathrm{sc}}$, then experiences a force for being in one of the qubit states, $\vec{F}_{\mathrm{FQ}}=\vec{m}_{\mathrm{sc}} \cdot \left(\partial \vec{B}_{z}/\partial z)\right)$.
%
When the qubit is in $|0\rangle$ ($|1\rangle$) state, the flux qubit experiences a $\vec{F}_{\mathrm{FQ}}=\hat{z}3\mu_{0} m_{mag}I R^2/(2(z + h)^4)$ $\left(\vec{F}_{\mathrm{FQ}}=-\hat{z}3\mu_{0} m_{mag}I R^2/(2(z + h)^4)\right)$.
%
Due to this $\vec{F}_{\mathrm{FQ}}$ and the equilibrium position shifts to a position, $z_{\mathrm{eq}}$ where the $\vec{F}_{\mathrm{FQ}}$ is balanced by the magnetic trapping force $\vec{F}_{\mathrm{TR}} = -\hat{z} 8 \pi^{4} r^{2} \rho R \nu_{z}^{2} z_{\mathrm{eq}}$. 
%
At equilibrium position, $\vec{F}_{\mathrm{FQ}} + \vec{F}_{\mathrm{TR}} = 0$.
%
Please note that the effect of gravitational force is already included in the magnetic trapping force.
%
For $z_{\mathrm{eq}} << h$, the equilibrium position is obtained analytically as,
\begin{equation}
z_{\mathrm{eq}} \sim \frac{3\mu_{0} m_{mag} IR^2}{16 \pi^{2} r^{2} \rho \nu^2 h^4}.
\label{eq: equilibrium shift}
\end{equation}
%
Then the \(\Delta z = 2 z_{\mathrm{eq}}\).

\subsection{Trapping a superconducting sphere between two dipoles}

We show the back-action effect of the superconductor on the magnetic field used to trap it by considering the example of trapping a superconducting (SC) sphere using two dipoles in an anti-Helmholtz magnetic configuration (magnetic moments anti-parallel). This anti-Helmholtz magnetic configuration can be realised experimentally either using current carrying loops as in the section for introducing the first method
%
or using uniformly magnetised spherical magnets to produce an external magnetic field that is identical to a point magnetic dipole \cite{Edwards2017InteractionsSpheres}.

We first note that a number of authors have considered analytic treatments of the Meissner force of a SC sphere in anti-Helmholtz magnetic fields \cite{Romero-Isart2012,Hofer2019AnalyticField},
 as well as the force between a point dipole and a SC sphere \cite{Coffey2000LevitationSphere,Lin2006AnalyticCylinder}. Many of these works use the image method but there seems to be some concern in the literature regarding this method \cite{Perez-Diaz2007InterpretationLevitation}, and so we do not use the image method in what follows and solve for the magnetic fields and associated forces directly.

We consider the magnets and SC sphere to have centers that all lie on the z-axis, and visualise the geometry in the $x-z$ plane as shown in Figure \ref{fig:AC1}.
%
% %
% Figure environment removed
%
We analytically derive the induced magnetic field at the SC sphere due to the magnetic field of the two point dipoles, using scalar potentials of the dipoles and enforcing the appropriate boundary conditions at the surface of the SC sphere.
%
Then we derive an expression for the levitation force by considering the interaction between the resulting magnetic field and the two point dipoles.

%
We denote $\Phi_{+}$, and $\Phi_{-}$, to be the scalar potentials of the two dipoles located respectively at  $\vec{r}_{+}\sim (0, 0, d_{+})$ and $\vec{r}_{-}\sim (0, 0, -d_{-})$, at either side of the origin along the $z-$axis.
%
At the arbitrary point $\vec{r}\sim (x, y, z)$, these scalar potentials can be written as ~\cite{Coffey2000LevitationSphere, Coffey2002LondonSphere},
%
\begin{equation}
\Phi_{\pm}(\vec{r})  = \pm \frac{\mu_{0}}{4\pi}\frac{\vec{m}\cdot\vec{p}_{\pm}}{||\vec{p}_{\pm}||^{3}}\label{eq:AC1},
\end{equation}
%
%
where $\mu_{0}$ is the magnetic permeability of free space and $\vec{p}_{\pm}=\vec{r}-\vec{r}_{\pm}$.
%
In spherical coordinates with $\vec{r}$ radius, $\theta$ as polar angle and $\phi$ azimuthal angle, equations (\ref{eq:AC1}) can be rewritten as,
%
\begin{equation}
\Phi_{\pm}(\tilde{r}_{\pm}, \theta)  = \frac{\mu_{0} m}{4\pi {d^{2}_{\pm}}} \frac{( (\pm \tilde{r}_{\pm}) \cos{\theta} - 1)}{\Big((\pm {\tilde{r}_{\pm}})^{2}-2 (\pm {\tilde{r}_{\pm}}) \cos{\theta} + 1 \Big)^{\frac{3}{2}}}\label{eq:AC2},
\end{equation}
%
Here, $m = ||\vec{m}||$, $\tilde{r}_{\pm} = r/d_{\pm}$. There is no $\phi$ dependency due to the axial symmetry of the system.
%
Expanding using Legendre polynomials, (\ref{eq:AC2}) can be written as,
%
\begin{equation}
\Phi_{\pm}(\tilde{r}_{\pm}, \theta)  = -\frac{\mu_{0} m}{4\pi d_{\pm}^{2}} \sum_{n=0}^{\infty} (n+1)\left(\pm {\tilde{r}_{\pm}}\right)^{n} P_{n}(\cos\theta)\label{eq:AC3},\\
%
\end{equation}
%
Then the net scalar potential ($\Phi_{\mathrm{m}} \equiv \Phi_{+} + \Phi_{-}$), generated by the two dipoles can be written as,
%
\begin{equation}
\Phi_{\mathrm{m}}(r, \theta) = -\frac{\mu_{0} m}{4\pi} \sum_{n=0}^{\infty} (n+1) r^{n} P_{n}\left(\cos\theta\right) \left[\left(\frac{1}{d_{+}}\right)^{n+2} + \left(-\frac{1}{d_{-}}\right)^{n+2}\right].
\label{eq:AC4}
\end{equation}
%
The radial component of the magnetic field generated by the two dipoles is given by, $\vec{B}_{{\mathrm{r}}, {\mathrm{m}}} = -\hat{r} \frac{\partial \Phi_{\mathrm{m}}} {\partial r}$ and in our case we find,
%
\begin{equation}
\vec{B}_{\mathrm{r}, {\mathrm{m}}}(r, \theta) = \hat{r} \frac{\mu_{0} m}{4\pi} \sum_{n=0}^{\infty} n (n+1) r^{\left(n-1\right)} P_{n}\left(\cos\theta\right) \left[\left(\frac{1}{d_{+}}\right)^{n+2} + \left(-\frac{1}{d_{-}}\right)^{n+2}\right],\\
\label{eq:AC5}
\end{equation}
%
We use the subscript r to indicate that we only compute the radial component of $\vec{B}$ and the m subscript indicates that the magnetic field is due to the dipoles and without the SC sphere. 
%
Introducing the SC sphere results in the introduction of an additional effective magnetic field in response to the field due to the dipoles.
%
Taken together, the induced and dipole fields must satisfy the Meissner boundary conditions, that the total magnetic field has a vanishing orthogonal component at all superconducting surfaces.

We write down a general solution for the scalar potential corresponding to an induced magnetic field of the SC sphere, due to the Meissner effect as,

\begin{equation}
\Phi_{\mathrm{SC}}(r, \theta) = \sum_{n=0}^{\infty} \frac{C_{n}}{r^{\left(n+1\right)}} P_{n}\left(\cos\theta\right);~ r>\gamma,
\label{eq:AC6}
\end{equation}
%
where $C_{n}$ is a general coefficient that has to be determined and $\gamma$ is the radius of the SC sphere.
%
This gives a magnetic field in the radial direction as,
%
\begin{equation}
\vec{B}_{\mathrm{r}, \mathrm{SC}}(r, \theta) = \hat{r} \sum_{n=0}^{\infty} (n+1) \frac{C_{n}}{r^{\left(n+2\right)}} P_{n}\left(\cos\theta\right).
\label{eq:AC7}
\end{equation}
%

We now express the total radial component of the magnetic field as, $\vec{B}_{\mathrm{r}, \mathrm{T}}=\vec{B}_{\mathrm{r}, \mathrm{m}}+\vec{B}_{\mathrm{r}, \mathrm{SC}}$.
%
Considering the perfect Meissner state for the SC sphere, we must have $\vec{B}_{\mathrm{r}, \mathrm{T}}(r=\gamma, \theta) = 0~\forall \theta$. 
%
This permits us to fix $C_{n}$ to be,
%
\begin{equation}
C_{n} = - \frac{\mu_{0} m}{4\pi} n \gamma^{\left(2n+1\right)} \left[\left(\frac{1}{d_{+}}\right)^{n+2} + \left(-\frac{1}{d_{-}}\right)^{n+2}\right].
\label{eq:AC8}
\end{equation}
%
Inserting equation~(\ref{eq:AC8}) into equation~(\ref{eq:AC6}) and equation~(\ref{eq:AC7}), we obtain the scalar potential and the radial component of the induced magnetic field of SC sphere as,
%
\begin{equation}
\Phi_{\mathrm{SC}}(r, \theta) = - \frac{\mu_{0} m}{4\pi} \sum_{n=0}^{\infty} n \frac{\gamma^{\left(2n+1\right)}}{r^{\left(n+1\right)}} P_{n}\left(\cos\theta\right)\left[\left(\frac{1}{d_{+}}\right)^{n+2} + \left(-\frac{1}{d_{-}}\right)^{n+2}\right],
\label{eq:AC9}
\end{equation}
%
%
\begin{equation}
\vec{B}_{\mathrm{r}, \mathrm{SC}}(r, \theta) = -\hat{r} \frac{\mu_{0} m}{4\pi} \sum_{n=0}^{\infty} n (n+1) \frac{\gamma^{\left(2n+1\right)}}{r^{\left(n+2\right)}} P_{n}\left(\cos\theta\right) \left[\left(\frac{1}{d_{+}}\right)^{n+2} + \left(-\frac{1}{d_{-}}\right)^{n+2}\right].
\label{eq:AC10}
\end{equation}
% 
Choosing $\theta =0$ and $\pi$, the magnetic field at the location of the SC sphere, along the $z$ axis can be obtained from the radial component of the magnetic field. 
%
In order to compare them with the numerical simulations below, we now convert the magnetic field back to Cartesian coordinates and obtain,
%
\begin{equation}
\vec{B}^{\pm}_{\mathrm{m}}(z) = \pm \hat{k} \frac{\mu_{0} m}{4\pi} \sum_{n=0}^{\infty} n (n+1) \left(\pm z\right)^{\left(n-1\right)} P_{n}\left(\pm1\right)\left[\left(\frac{1}{d_{+}}\right)^{n+2} + \left(-\frac{1}{d_{-}}\right)^{n+2}\right],
\label{eq:AC11}
\end{equation}
%
where now $+$ and $-$ in $\pm$ correspond to respectively positive and negative values for the z axis and $\hat{k}$ is the unit vector along the z axis.
%
Similarly, we find for the induced magnetic field by the SC sphere, with radius $\gamma$ one has,
%
\begin{equation}
\vec{B}^{\pm}_{\mathrm{SC}} (z) = \mp \hat{k} \frac{\mu_{0} m}{4\pi} \sum_{n=0}^{\infty} n (n+1)\frac{\gamma^{\left(2n+1\right)}}{\left(\pm z\right)^{\left(n+2\right)}} P_{n}\left(\pm1\right) \left[\left(\frac{1}{d_{+}}\right)^{n+2} + \left(-\frac{1}{d_{-}}\right)^{n+2}\right].
\label{eq:AC12}
\end{equation}
%
From equations (\ref{eq:AC11}) and (\ref{eq:AC12}) we obtain,
\begin{eqnarray}
\vec{B}_{\mathrm{m}}(z) &= \vec{B}^{\pm}_{\mathrm{m}}(z) &= -\hat{k} \frac{\mu_{0} m}{2\pi} \left[\frac{1}{\left(z-d_{+}\right)^3}+\frac{1}{\left(z+d_{-}\right)^3}\right]
\label{eq:AC13},
\end{eqnarray}
%
\begin{eqnarray}
\vec{B}^{+}_{\mathrm{SC}}(z) &= -\vec{B}^{-}_{\mathrm{SC}}(z) &= \hat{k} \frac{\mu_{0} m}{2\pi} \gamma^{3}\left[\frac{1}{\left(\gamma^{2}-z d_{+}\right)^3}+\frac{1}{\left(\gamma^{2}+z d_{-}\right)^3}\right]\label{eq:AC14}.
\end{eqnarray}
%

Then the net magnetic field along the z axis between the two dipoles with the SC sphere is given by the piecewise function:
%
\begin{equation}
\vec{B}_{\mathrm{T}} (z) = 
\begin{cases}
\vec{B}_{\mathrm{m}}(z) + \vec{B}^{+}_{\mathrm{SC}}(z) & z > \gamma, \\
0 & -\gamma \leq z \leq \gamma,\\
\vec{B}_{\mathrm{m}}(z) + \vec{B}^{-}_{\mathrm{SC}}(z) & z < -\gamma.
\label{eq:AC15}
\end{cases}
\end{equation}
%
Without the SC sphere present between the two dipoles, $\vec{B}_{T} (z) = \vec{B}_{\mathrm{m}}(z)$.


We compute the z component of the force on the SC sphere and from the dipoles as,
\begin{equation}
 \vec{F}_{\mathrm{SC}}(d_{+},d_{-})=-\hat{k}\frac{\partial } {\partial z}\left(\vec{m}\cdot\vec{B}^{+}_{\mathrm{SC}}(z)\right)\Big|_{z=d_{+}}+\hat{k}\frac{\partial } {\partial z}\left(\vec{m}\cdot\vec{B}^{-}_{\mathrm{SC}}(z)\right)\Big|_{z=d_{-}},
 \label{eq:AC18}
\end{equation}
%
The final form of $\vec{F}_{\mathrm{SC}}$ is given by:
\begin{equation}
    \vec{F}_{\mathrm{SC}} = \hat{k}\frac{3}{2} \frac{\mu_{0} m^{2}}{2\pi} \gamma^{3}\left[\frac{2 d_{-}}{\left(\gamma^{2}-d_{-}^{2}\right)^4}-\frac{2 d_{+}}{\left(\gamma^{2}-d_{+}^{2}\right)^4}+\frac{d_{-}}{\left(\gamma^{2}+d_{+}d_{-}\right)^4}-\frac{d_{+}}{\left(\gamma^{2}+d_{+}d_{-}\right)^4}\right]\label{eq:AC19}.\\
\end{equation}

The properties of the trap can be explored by introducing displacements to the SC sphere.
%
For this, we consider two parameters the total gap between the centers of two magnets, $d = d_{+}+d_{-}$ and the displacement $2\delta = d_{-}-d_{+}$.
%
Then with the transformation $d_{\mp} = (d/2)\pm \delta$, we can write the $\vec{F}_{\mathrm{SC}}$ as a function of $\delta$.
%
We do a series expansion of such a function and from this series that consists only of odd terms we approximately neglect all the higher order terms assuming small displacements, but the first order term.
%
The resultant $\vec{F}_{\mathrm{SC}}(\delta)$ can be written as,
%
\begin{equation}
    \vec{F}_{\mathrm{SC}}(\delta) \sim -\hat{k}384\frac{\mu_{0} m^{2}}{\pi} \gamma^{3}\left[\frac{14d^{2}+8\gamma^{2}}{(d^{2}-4\gamma^{2})^{5}}-\frac{1}{(d^{2}+4\gamma^{2})^{4}}\right]\delta \label{eq:AC20}.\\
\end{equation}
%
For the separation between dipoles is much greater than the diameter of the SC sphere, $d>>2\gamma$, $\vec{F}_{\mathrm{SC}}(\delta)$ becomes,
%
\begin{equation}
    \vec{F}_{\mathrm{SC}}(\delta) \sim -\hat{k}4992\frac{\mu_{0} m^{2}}{\pi} \frac{\gamma^{3}}{d^{8}} \delta \label{eq:AC21}.\\
\end{equation}
%
The magnitude of force in (\ref{eq:AC21}) is of the form of $F_{\mathrm{SC}}(\delta) = - k_{\mathrm{z}} \delta$, where $k_{z}~=~4992\mu_{0} m^{2}\gamma^{3}/(\pi d^{8})$ is the spring constant or stiffness of the trap. 
%
It is also the negative slope of the force-displacement curve that can be drawn from equation (\ref{eq:AC21}).
%
The gravitational pull also puts an extra force on the SC sphere and displaces it from the origin along the negative z-axis. Then the total force on the SC sphere along the z-axis becomes, $- \hat{k} (k_{\mathrm{z}} \delta +m_{\mathrm{SC}}~g)$, where $m_{\mathrm{SC}}$ and $g$ are the mass of the SC sphere and acceleration due to gravity.
%
As a result, we can write the equilibrium position as, $\delta_{\mathrm{eq}} \sim ~-m_{\mathrm{SC}}~g/k_{\mathrm{z}}$
%
If the slope of the force-displacement curve is not going to vary from the origin to the equilibrium position and also approximates the trap as a harmonic one, the angular frequency in radian of the trap along the z axis at $\delta_{eq}$ is given as $\omega_{\mathrm{z}}=\sqrt{k_{\mathrm{z}}/m_{\mathrm{SC}}}$. 
%

In order to understand how much the back action of the SC sphere affected the dipole magnetic field and the force due to it, we treat the presence of the SC sphere in the magnetic field as a perturbation and do not consider any back action from the SC sphere on the magnetic fields. To identify this case from the previous case we denote the force, stiffness, equilibrium position, and trap frequency, with an over-line above the corresponding symbols. 

The force on the particle along the z-axis is then \cite{OBrien2019, Houlton2018AxisymmetricLevitation}, 
%
\begin{equation}
 \vec{\overline{F}}_{\mathrm{SC}} =\hat{k}\frac{\chi V}{\mu_{0}} B\frac{\partial B}{\partial z},
 \label{eq:AC22}
\end{equation}
%
where $B = \sqrt{B_{x}^{2}+B_{y}^{2}+B_{z}^{2}}$, with $B_{x/y/z}$ is the x/y/z component of the magnetic field. Since we are considering a perfect superconductor we can consider $\chi \sim -1$.
%
In contrast to the previous case, we consider $d_{+} = d_{-}$
%
In order to obtain  $B_{x/y/z}$, we write down $\Phi_{\mathrm{m}}$ directly in Cartesian coordinates from equation \ref{eq:AC1}  as.
\begin{equation}
\Phi_{\mathrm{m}}(x, y, z) = \frac{m}{4\pi}\left(\frac{(z-(d/2))}{(x^{2}+y^{2}+(z-(d/2))^{2})^{3/2}} -\frac{(z+(d/2))}{(x^{2}+y^{2}+(z+(d/2))^{2})^{3/2}}\right).
\label{eq:AC23}
\end{equation}
%
Then we can write $B_{\mathrm{x}}=-\mu_{0}\partial \Phi_{\mathrm{m}}(x, 0, 0)/\partial x$, $B_{\mathrm{y}}=-\mu_{0}\partial \Phi_{\mathrm{m}}(0, y, 0)/\partial y$, and $B_{\mathrm{z}}=-\mu_{0}\partial \Phi_{\mathrm{m}}(0, 0, z)/\partial z$.
%
The $B_{\mathrm{z}}$ is same as in equation (\ref{eq:AC13}), as $d_{+} = d_{-} = d/2$ 
%

Since we are interested in the force along the z-axis alone, where $x = y = 0$,
%
\begin{equation}
 \vec{\overline{F}}_{\mathrm{SC}}(z)=-\hat{k}\frac{V}{\mu_{0}} B_{\mathrm{m}}(z)\frac{\partial B_{\mathrm{m}}(z)}{\partial z},
 \label{eq:AC24}
\end{equation}
%
Then from equation \ref{eq:AC20} using series expansion and considering only the first order we get,

\begin{equation}
    \vec{\overline{F}}_{\mathrm{SC}} \sim -\hat{k} 3072\frac{\mu_{0} m^{2}}{\pi} \frac{\gamma^{3}}{d^{8}}\delta\label{eq:AC25}.\\
\end{equation}
%
Comparing equation \ref{eq:AC21} and \ref{eq:AC25}, we can see that the trap stiffness $k_{\mathrm{z}} \sim 1.625~\overline{k}_{\mathrm{z}}$, the equilibrium position $\delta_{\mathrm{eq}} \sim \overline{\delta}_{\mathrm{eq}}/1.625$, and the trap frequency $\omega_{\mathrm{z}} \sim 1.27~\overline{\omega}_{\mathrm{z}}$.

To validate this analytical method, we also built a COMSOL model with the same geometry as that in the above analysis (Figure \ref{fig:AC1}), except for magnet spheres instead of magnetic dipoles. The magnetic field of a magnet sphere with flux density  $Br_{magnet}$ and volume $V_{magnet}$ is identical to that of a magnetic dipole with moment $m=\frac{1}{\mu_{0}}Br_{magnet}V_{magnet}$, so that would not affect the result. In the model, we took the SC sphere radius $\gamma=100\ \mu m$, magnet spheres radius $R_{magnet}=1000\ \mu m$, the distance between two magnets $d=4000\ \mu m$, magnet flux density $Br_{magnet}=1\ T$. In the COMSOL simulation, the backaction of the superconductor to the magnetic field used to trap it is well considered. We get the vertical trap 
stiffness $k_{z\_COMSOL}= 0.32\ N/m$, which agrees well with $k_{z}=0.34\ N/m$ from the above backaction effect involved analytical method using the same parameters. Considering $k_{z} \sim 1.625~\overline{k}_{z}$, we can then conclude that the backaction effect of a superconductor in a trapping magnetic field is significant and it would change the vertical trap stiffness by $62.5\%$ in this case.

\subsection{FEM simulation in COMSOL}
%
We simulate the levitation of superconducting flux qubit using the spherical micro-magnet in a 2D axisymmetric model using the commercial finite element method (FEM) package COMSOL. 
%
In this FEM model, we consider the flux qubit as a superconducting ring with the ideal Meissner effect. 
%
We find that the London penetration has a negligible effect on the results due to the large size of the flux qubit.
%
Furthermore, by performing a fully 3D FEM COMSOL simulation we have determined that the levitated flux qubit is stable both vertically and horizontally, due to the loop shape of the flux qubit. We observe that the magnetic forces and torques will work to restore the qubit's equilibrium position and orientation when flux qubit horizontally shifts or tilts slightly. Thus the trapped flux qubit experiences complete rigid-body trapping.
%

To test the horizontal stability of the trapped flux qubit, we built a full 3D COMSOL model. In this model, we calculated the x component magnetic force ($F_x$) at the condition of horizontal displacement ($x$). We found that the magnetic force would be a restoring force when the flux qubit slightly shifts away from the balance position in the horizontal direction. Also, the torque on the flux qubit will restore it to the original position at the condition of tilting. It's also certain that the levitated flux qubit will be unstable when the horizontal displacement (or tilting) is too large. We have also calculated the horizontal trap frequency as around $19\:Hz$, and the tilting frequency as $9\:Hz$.

% Figure environment removed

\section{Table of variables and their values considered in the present study}

\begin{table*}
\begin{center}
\caption{System parameters and the values used to introduce the first method
%
\label{Table1}}
\begin{threeparttable}
\begin{tabular}{ p{8cm}p{1.5cm}p{3.5cm}p{1cm}}
\hline\hline
Quantity & Symbol & Value & Unit   \\ \hline
%
Density of  YIG \cite{Seberson2020SimulationGas}  & $\rho$ & $5110$ &{kg$/m^{3}$}\\
%
Remnant magnetic field of YIG \cite{Musa2017StructuralSynthesis}  & $B_{r}$ & $14.32 \times 10^{-3}$ & ${\rm T}$ \\
%
Magnitude of persistant current in SC flux qubit & $I$ & $1000$ & ${\rm nA}$ \\
Mass of the Helium gas molecule & $M_{\mathrm{g}}$ & $6.65 \times 10^{-27}$ & kg \\
Pressure of the gas surrounding the oscillator & $P_{\mathrm{g}}$ & $10^{-6}$ & kg/($m s^{2})$ \\
Boltzmann constant & $k_{\mathrm{B}}$ & $1.38 \times 10^{-23}$ & $m^{2}$ kg/($K s^{2})$ \\
%& & &      \\
\hline\hline
\end{tabular}

\end{threeparttable}
\end{center}
\vspace{0mm}
\end{table*}

\begin{table*}
\begin{center}
\caption{System parameters and the values used to introduce the second method
%
\label{BigTable}}
\begin{threeparttable}
\begin{tabular}{ p{8cm}p{1.5cm}p{3.5cm}p{1cm}}
\hline\hline
Quantity & Symbol & Value & Unit   \\ \hline
Radius of magnet sphere & $a$ & $12$ &${\rm \mu m}$ \\
Radius of SC flux qubit & $R$ & $180-183.4$ & ${\rm \mu m}$ \\
Radius of cross-section of SC flux qubit & $r$ & $5$ & ${\rm \mu m}$ \\
\hline
Magnetic moment of magnet sphere & $m_{mag}$ & $6.912\times10^{-9}$ & ${\rm A\cdot m^{2}}$ \\
Remanent flux density of magnet sphere & $Bz$ & $1.2$ & ${\rm T}$ \\
\hline
Mass of SC flux qubit & $M$ & $(2.41-2.45)\times10^{-10}$ & ${\rm kg}$  \\
Self inductance of SC flux qubit & $L$ & $(8.28-8.48)\times10^{-10}$ & ${\rm H}$ \\ 
\hline
Vacuum magnetic permeability & $\mu_{0}$ & $4\pi\times10^{-7}$ & ${\rm H/m}$ \\
Magnetic flux quantum & $\Phi_{0}$ & $2.068\times10^{-15}$ & ${\rm Wb}$ \\
\hline
Superposition current in SC flux qubit & $I$ & $\pm{1000}$ & ${\rm nA}$ \\
Magnetic moment due to superposition current & $M_{sc}$ & $(1.02-1.06)\times10^{-13}$ & ${\rm A\cdot m^{2}}$ \\
\hline
Equilibrium position of SC flux qubit & $h$ & $70-93$ & ${\rm \mu m}$ \\
%
Vertical trap frequency & $v_{z}$ & $10-43$ & ${\rm Hz}$ \\
\hline
Superposition states separation & $\Delta z$ & $7-100$ & ${\rm nm}$ \\
Zero point motion amplitude & $\delta z_{zpm}$ & $6\times10^{-14}$ & ${\rm m}$ \\
Superposition separation in terms of $\delta z_{zpm}$ & $\chi$ & $2\times10^{6}$ & NA \\
%
\hline\hline
\end{tabular}

\end{threeparttable}
\end{center}
\vspace{0mm}
\end{table*}

\bibliography{references}
\end{document}

