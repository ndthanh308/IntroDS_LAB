\section{Introduction}
Mobile apps are ubiquitous in our daily life for supporting different tasks such as reading, chatting, and banking.
Smartphones and tablets are the two types of portable devices with the most available apps~\cite{tabletShare}.
To conquer the market, one app is often available on both smartphones and tablets~\cite{majeed2015apps}.
Due to comparable functionalities, the smartphone and tablet versions of the same app have a highly similar Graphical User Interface (GUI).
Popular apps always share a similar GUI design between phone apps and tablet apps, for example, YouTube~\cite{youtube} and Spotify~\cite{spotify}.
If a tool can automatically recommend a GUI design for the appropriate tablet platform based on existing mobile GUIs, it can significantly minimise the developer's engineering effort and accelerate the development process.
From the user side, a comparable design could provide data or study on user preferences for consistency across different devices. 
It reduces the need for them to learn new navigation and interaction patterns~\cite{oulasvirta2020combinatorial}.
Therefore, automated GUI development tasks, such as the cross-platform conversion of GUI designs, GUI recommendations, etc., are gradually gaining attention from industry and academia~\cite{zhao2022code, li2022learning, chen2020wireframe}.
However, the field of automatic GUI development is still in a research bottleneck, lacking breakthroughs and widely recognized tools or methods.
If a present developer needs to develop a tablet-compatible version of their app, they usually start from scratch, resulting in needless costs increase and wasted existing design resources.

According to our observations, the growth of automated GUI development is hindered by two reasons.
First, as deep learning approaches, particularly generative models~\cite{theis2015note}, become more widespread in the field of automated GUI development, researchers increasingly need a pairwise, high-quality GUI dataset for training models, summarizing rules, and so on.
The current datasets, for example, Rico~\cite{deka2017rico}, ReDraw~\cite{moran2019redraw}, and Clay~\cite{li2022learning}, only include single GUI pages with UI metadata and UI screenshots, and there are no pairwise corresponding GUI page pairs.
Current datasets are only suitable for GUI component identification, GUI information summarising and GUI completion.
The lack of valid GUI pairs in current datasets have severely hindered the growth of GUI automated development.
Second, the collecting of GUI pairs between phones and tablets is more labor-intensive.
Individual GUI pages can be automatically collected and labeled by current GUI testing and exploration tools~\cite{hu2011automating, memon2002gui} to speed up the collection process.
However, due to the disparity in screen size and GUI design between phones and tablets, it is challenging to automatically align the content on both GUI pages.
Figure~\ref{fig:introExp} shows an example of a phone-tablet GUI pair of the app 'BBC News'~\cite{bbcNews}.
To accommodate tablet devices, the UI component group 1 in the phone GUI are converted to the parts marked as 1 in the tablet GUI.
We can find that these components not only change their positions and sizes, but also the layouts and types of GUIs.
To keep a consistent layout with the left GUI components in the tablet GUI, the UI component group 1 in the tablet design adds new contents that are not available in the mobile GUI (black box in the tablet GUI).
Another UI component group, which is marked as 2 in the tablet GUI, is not present in the phone GUI at all.
One tablet GUI page may correspond to the contents of multiple phone GUI pages due to the different screen size and design style.
The UI contents in group 2 of the tablet GUI correspond to other phone GUIs.

% Figure environment removed

In response to above challenges, we provide our dataset Papt, which is a \textbf{PA}irwise dataset for GUI conversion and retrieval between Android \textbf{P}hones and \textbf{T}ables.
It is the first pairwise GUI dataset of phones and tablets. 
The dataset contains 1,0035 corresponding phone-table GUI pairs, which are collected from 5,593 tablet-phone app pairs.
We first describe the data source, dataset collection process, collection approaches, and collection tools in this paper. 
We also open source our data collection tool for further related works.
Second, we describe the format of the dataset and the loading method. 
Finally, we perform preliminary experiments on the dataset for the GUI conversion task and GUI retrieval task. 
We present some of the GUI pages generated by the preliminary experiments.
We discuss the challenges faced by the current models on GUI conversion and GUI retrieval tasks.
This resource paper bridges the gap between smartphone GUIs and tablet GUIs.
Our goal is to provide an effective benchmark for GUI automation development and to encourage more academics to explore this field.

In summary, the novel contributions of this paper are the following:
\begin{itemize}
    \item We contribute the first pairwise GUI dataset between Android phones and tablets~\footnote{\url{https://github.com/huhanGitHub/papt}}.

    \item We provide the detailed procedure of data collection and open source our data collection tool.

    \item We perform preliminary experiments on the dataset and report the experimental results. We show some generated GUI designs and discuss the challenges faced by the current models on GUI conversion and GUI retrieval tasks.
\end{itemize}