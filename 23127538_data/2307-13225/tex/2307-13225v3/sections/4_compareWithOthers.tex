\subsection{Compare With Available Datasets}
\label{sec:dataCompare}
\begin{table*}[htbp]
\setlength{\abovecaptionskip}{0pt} 
\setlength{\belowcaptionskip}{5pt}
\caption{Comparison between our dataset and other GUI datasets}
\begin{adjustbox}{ width=\textwidth,center}
\centering
\begin{tabular}{ccccccc}
\hline
\textbf{Dataset}   & \textbf{GUI Platform} & \textbf{\#GUIs} & \textbf{\#Paired GUIs} & \textbf{\#Data Source App} & \textbf{Latest Updates} & \textbf{Mainly targeted tasks}\\
\hline
Rico~\cite{deka2017rico} & Phone & ~72,000 & 0 & ~9,700 & Sep. 2017 & UI Component Recognition, GUI completion \\
UI2code~\cite{chen2018ui} & Phone & ~185,277 & 0 & ~5,043 & June. 2018 & UI Skeleton Generation\\
Gallery D.C.~\cite{chen2019gallery} & Phone & 68,702 & 0 & 5,043 & Nov. 2019 & UI Search \\
LabelDroid~\cite{chen2020unblind} & Phone & 394,489 & 0 & 15,087 & May. 2020 & UI Component Prediction \\
UI5K~\cite{chen2020wireframe} & Phone & 54,987 & 0 & 7,748 & June. 2020 & UI Search \\
Enrico~\cite{leiva2020enrico} & Phone & ~1,460 & 0 & ~9,700 & Oct. 2020 & UI Layout Design Categorization \\
VINS~\cite{bunian2021vins} & Phone & ~2,740 & 0 & ~9,700 & May. 2021 & UI Search \\
Screen2Words~\cite{wang2021screen2words} & Phone & 22,417 & 0 & 6,269 & Oct. 2021 & UI screen summarization \\
Clay~\cite{li2022learning} & Phone & 59,555 & 0 & ~9,700 & May. 2022 &  UI Component Recognition, GUI completion \\
\textbf{Papt} & Phone, Tablet & 20,070 & 10,035 & 11,186 & Jan. 2023 & UI Component Recognition, GUI completion, GUI conversion, GUI search \\ 

\bottomrule
\end{tabular}
\end{adjustbox}
\label{tab:compare}
\vspace{-0.3cm}
\end{table*}




\subsubsection{Application to More Tasks}
Table~\ref{tab:compare} shows a summary of our and other GUI datasets.
First, since our data consist of phone-tablet pairs, we must manually locate the corresponding GUI pages between phones and tablets, resulting in a lesser number of pages than comparable datasets.
However, we now have a broader data source (including tablet GUIs), more supported tasks, and newer data.
Notably, it is the only available GUI dataset that contains phone-tablet pairwise GUIs. 
Our dataset addresses numerous significant gaps in existing GUI automated development and provides effective data support for the application of deep learning techniques in GUI generation, search, recommendation, and other domains.


\subsubsection{Data Accuracy}
Specifically, our data eliminates a large number of GUI visual mismatches that are frequent in current datasets like as Rico and Enrico.
Due to the limitations of the previous data collection tools, some GUIs have visual mismatches in the metadata and screenshots.
Figure~\ref{fig:mismatch} shows typical visual mismatch examples between hierarchy metadata and screenshots in current datasets.
Based on the bounding box coordinates of Android views provided in the metadata, we depict the location of the views in the metadata as a black dashed line in the screenshot.
We mark the obvious visual mismatches with a solid red line box, which do not correspond to any of the views in the rendered screenshot.
The metadata provides information on the UI elements behind the current layer, but these elements cannot be interacted with on the current screenshot.
Visual mismatches between the UI data in the metadata and the screenshot would result in the UI data in the metadata and the screenshot not corresponding one to the other.
Too many mismatch cases would have a negative impact on the efficiency of model generation and search. 
The selected UI collecting tool, UIautomator2, has optimised the GUI caption technique to avoid metadata and screenshots from containing inconsistent UI information~\cite{uiautomator2}.
During manual reviews, our volunteers also eliminated GUI pages with mismatched.
Compared to other datasets, such as rico, fewer mismatches give us a higher accuracy of our data.


% Figure environment removed
