% mnras_template.tex
%
% LaTeX template for creating an MNRAS paper
%
% v3.0 released 14 May 2015
% (version numbers match those of mnras.cls)
%
% Copyright (C) Royal Astronomical Society 2015
% Authors:
% Keith T. Smith (Royal Astronomical Society)

% Change log
%
% v3.0 May 2015
%    Renamed to match the new package name
%    Version number matches mnras.cls
%    A few minor tweaks to wording
% v1.0 September 2013
%    Beta testing only - never publicly released
%    First version: a simple (ish) template for creating an MNRAS paper

%%%%%%%%%%%%%%%%%%%%%%%%%%%%%%%%%%%%%%%%%%%%%%%%%%
% Basic setup. Most papers should leave these options alone.
\documentclass[fleqn,usenatbib]{mnras}

% MNRAS is set in Times font. If you don't have this installed (most LaTeX
% installations will be fine) or prefer the old Computer Modern fonts, comment
% out the following line
%\usepackage{newtxtext,newtxmath}
% Depending on your LaTeX fonts installation, you might get better results with one of these:
%\usepackage{mathptmx}
%\usepackage{txfonts}

% Use vector fonts, so it zooms properly in on-screen viewing software
% Don't change these lines unless you know what you are doing
\usepackage[T1]{fontenc}
\usepackage{ae,aecompl}
\usepackage[dvipsnames]{xcolor}
\usepackage{amssymb}
%\hypersetup{draft}
%%%%% AUTHORS - PLACE YOUR OWN PACKAGES HERE %%%%%

% Only include extra packages if you really need them. Common packages are:
\usepackage{graphicx}	% Including figure files
\usepackage{amsmath}	% Advanced maths commands
\usepackage{amssymb}	% Extra maths symbols
\usepackage{listings}
\usepackage{soul,xcolor}

%%%%%%%%%%%%%%%%%%%%%%%%%%%%%%%%%%%%%%%%%%%%%%%%%%

%%%%% AUTHORS - PLACE YOUR OWN COMMANDS HERE %%%%%

% Please keep new commands to a minimum, and use \newcommand not \def to avoid
% overwriting existing commands. Example:
%\newcommand{\pcm}{\,cm$^{-2}$}	% per cm-squared
\newcommand{\msun}{M$_\odot$}
\setstcolor{red}
\makeatletter
\setlength{\@fptop}{0pt}
\makeatother

%%%%%%%%%%%%%%%%%%%%%%%%%%%%%%%%%%%%%%%%%%%%%%%%%%

%%%%%%%%%%%%%%%%%%% TITLE PAGE %%%%%%%%%%%%%%%%%%%

\title[High-$z$ galaxies from bumpy power spectrum]{Excess of high-$z$ galaxies as a test for bumpy power spectrum of density perturbations
}
% High-$z$ evolution of the halo mass function in cosmological models with bumpy spectra of density perturbations
% Excess of haloes at high-$z$ as a test for bumpy power sectrum of density perturbations produced by inflation


\author[M.V. Tkachev, S.V. Pilipenko, E.V. Mikheeva, V.N. Lukash]{
M.V. Tkachev$^{1}$\thanks{mtkachev@asc.rssi.ru}
S.V. Pilipenko$^{1}$\thanks{spilipenko@asc.rssi.ru}
E.V. Mikheeva$^{1}$\thanks{helen@asc.rssi.ru}
V.N. Lukash$^{1}$\thanks{lukash@asc.rssi.ru}
\newauthor
\\
% List of institutions
$^{1}$Astro Space Center of P. N. Lebedev Physical Institute of RAS, Profsojuznaya 84/32, Moscow 117997, Russia\\
}

% These dates will be filled out by the publisher
\date{Accepted XXX. Received YYY; in original form ZZZ}

% Enter the current year, for the copyright statements etc.
\pubyear{2023}

% Don't change these lines
\begin{document}
\label{firstpage}
\pagerange{\pageref{firstpage}--\pageref{lastpage}}
\maketitle

% Abstract of the paper
\begin{abstract}
Modified matter power spectra with approximately Gaussian bump on sub-Mpc scales can be a result of a complex inflation. We consider five spectra with different amplitudes $A$ and locations $k_0$ and run N-body simulations in a cube $(5 Mpc/h)^3$ at $z>8$ to reveal the halo mass functions and their evolution with redshift. We have found that the ST formula provides a good approximation to a such kind of matter spectra. In the considered models the dark matter halo formation starts much more earlier than in $\Lambda$CDM, which in turn can result in an earlier star formation and a nuclear activity in galaxies. 
At $z=0$ the halo mass functions are hardly distinguishable from the standard $\Lambda$CDM, therefore the models with the bumpy spectra can be identified in observations by their excess in number of bright sources at high redshift only.

%Modified matter power spectra with the Gaussian bump on sub-Mpc scales are considered. 
%%to study the influence of the bump on halo mass function. 
%Each spectrum is the product of $\Lambda$CDM one and the Gaussian with some relative amplitudes $A$ and location at the scale $k_0$. 
%We tested five spectra with different values of $A$ ans $k_0$ for N-body simulations in a cube $(5 Mpc/h)^3$ to reveal the halo mass functions and their evolution with redshift.
%Also, we studied the applicability of the Sheth-Tormen approximation to a such kind of matter spectra and obtained a good agreement. 
%We found that in the investigated models the dark matter halo formation can start much more earlier than in $\Lambda$CDM, which in turn can result in an earlier star formation and a nuclear activity in galaxies. 
%At $z=0$ the halo mass functions are hardly distinguishable from the standard $\Lambda$CDM, therefore the considered models can be identified by their excess in number of bright submillimeter sources at high redshift only.   
\end{abstract}

\begin{keywords}
cosmology-simulations --- dark matter --- matter power spectrum
\end{keywords}


%%%%%%%%%%%%%%%%% BODY OF PAPER %%%%%%%%%%%%%%%%%%

\section{Introduction}  
\label{sec:intro}
%======================

It is commonly believed that an inflation predicts a spectrum of density perturbations similar to a power law or which can be approximated by a power law over the wide range of scales up to the current horizon value. Indeed, a smooth spectrum with barely discernible deviation is a common feature of many inflationary models based on the single scalar field theory. For example, the density perturbation spectrum in chaotic inflation with the single massive scalar field looks like 
$$
q_k\simeq\frac m\pi\ln\left[\frac{k_1}k\left(\ln\frac{ k_1}k\right)^{1/3}\right]
$$
in terms of $q$-scalar (see \cite{lukash1980}, %\cite{book}, 
\cite{bookrus}), where $k$ is a wave number, $k_1$ is the wave number at the end of inflation, and $m$ is a mass of inflaton. The running parameter of this spectrum cannot be detected at the available accuracy measurements.

During the inflationary stage of the Universe evolution, processes that disrupt the smooth spectrum of density perturbations could have taken place. Many such models have been proposed in the context of inflation with two fields, resonant amplification with oscillatory features, and others (see the numerous references in \cite{2023JCAP...04..011I}). 
% references 38-79
Those studies were motivated by the recent observational arguments in favour of primordial black holes as viable dark matter candidates. These arguments are the detection of gravitational waves from colliding black holes (see \cite{ligo}) and the absence of dark matter particle candidates in the Standard Model of particle physics (see \cite{Aleksandrov:2021}).   
% one more reference Невзоров Р Б "Феноменологические аспекты суперсимметричных расширений Стандартной модели" УФН 193 577–613 (2023) in Russian only till August-September
To generate primordial black holes it is necessary to amplify the density perturbation spectrum at some scales. However, unless one assumes the fine-tuning of related inflationary quantities, it follows that a peak-like features (also called a bump) can appear at other scales too (see, e.g. Fig.~2 of \cite{Ivanov94}). For example, they could exist at sub-Mpc one and be responsible for some unusual feature in halo mass function at low and high redshifts. This case is under consideration in this paper.

The power spectrum of density perturbations is now measured at comoving wavenumbers $k<1$~$h/$Mpc from the Cosmic Microwave Background (CMB), large scale structure of the Universe, Ly-$\alpha$ forest \citep{Chabanier19}. No deviations from the power-law primordial spectrum have been found. The deviations at smaller scales, $1<k<10^{3}$~$h/$Mpc, would affect galaxy formation. A simple and straightforward estimate using Press-Schechter (PS) formalism \citep{press} shows that addition of a peak-like feature at $k>1$~$h/$Mpc will result in amplification of the number density of galaxy or dwarf-sized haloes at high redshifts, while this amplification will be almost smoothed away at redshift $z=0$\footnote{Figure \ref{fig:sh_tor} demonstrates this behaviour using the ST approximation.}. This means that it is difficult to find a trace of the bump in the power spectrum at $z=0$, and observations at the Epoch of Reionization (EoR) are more suitable for that.

Recently surprisingly high number of massive galaxies at $z\geq 9$ has been discovered in \textit{James Webb Space Telescope} (JWST) data \citep{Naidu2022b,Castellano22,Finkelstein22,Donnan23,Labbe23}. It is actively debated now, whether these findings are in tension with the abundance of haloes predicted by the $\Lambda$CDM \citep{Boylan-Kolchin23,Lovell22,Chen23,Prada23,Shen23}. If the tension will be confirmed by further observations and their more careful interpretations, a bumpy power spectrum may be an explanation for these results. Our aim is to present the expected deviations from the $\Lambda$CDM halo abundance for cosmological models with the bump in the power spectrum. We demonstrate the use of bumpy power spectrum models by applying the Extreme Value Statistics (EVS) \citep{Gumbel58} to compare observations with theoretical predictions. Recently \cite{Lovell22} have shown using EVS that JWST observations can be explained by $\Lambda$CDM mass function only if one adopts very efficient star formation at the EoR: about 100\% of baryons in haloes must transform into stars.

The halo mass function, which is needed to construct the EVS, can be obtained with modifications of the PS formalism, e.g. the more precise Sheth-Tormen (ST) formula \citep{Sheth99} but there is a couple of possible caveats here.
As shown by \cite{Klypin_2011}, ST approximation gives
a fairly accurate fit for z = 0 for for a wide range of halo masses, however, it tends to overpredict the abundance of haloes at higher redshifts. This downside can be at least partially circumvented by adjusting the radius $R$ of the real space spherical top-hat window function $\hat{W}(kR)$ in the ST formula that we used 
 (for more accurate improvements over the ST approximation see e.g. \cite{Tinker_2008} or \cite{Behroozi_2013}).
Furthermore, while ST is known to reproduce the $\Lambda$CDM mass function quite well \citep{Jenkins2001,Reed2003}, it is not so good for cosmological models with more complicated spectra, e.g. Warm Dark Matter which has a sharp cut-off at small scales \citep{Angulo13}. So, before using ST or any other theoretical or empirical mass function for calculating the number density of haloes in cosmological models with bumpy spectra, we need to check that this technique is applicable. For this, we run a set of N-body cosmological simulations with different bumpy spectra, as well as a standard $\Lambda$CDM spectrum, and compare the simulated mass functions with the theoretical one.

Some investigations of halo formation in cosmological models with bumpy power spectra have been done in the literature. Authors of \cite{Knebe01} have analysed the impact of a positive or negative bump on the galaxy cluster mass function. They have found that PS formalism predicts the mass function quite well, but they have simulated different part of the power spectrum, $k<1$~$h/$Mpc. Since the slope of the power spectrum transfer function changes with $k$ and become more steep at higher $k$, it is not clear whether the results of \cite{Knebe01} are applicable for the range of $k$ we are interested in. Later \cite{Bagla09} have simulated a model with the bump at $k\approx 1$~$h/$Mpc, but the simulation resolution was not very high and the change of the mass function they found is moderate in comparison with the noise.
 
The paper is organised as follows:
in Section \ref{sec:cosmol} we describe the considered models;
in Section \ref{sec:simulations} we describe the details of our numerical simulations and provide the list of parameters used to run the N-body simulations.
In Section~\ref{sec:hmf} we present the ST mass functions 
adopted for modified matter spectra.
In Section \ref{sec:evs} we discuss the early halo formation realising in considered cosmological models.
Finally, we leave the Section \ref{sec:conclusions} for discussion of various implications.

%%%%%%%%%%%%%%%%%%%%%%%%%%%%%%%%%%%%%%%%%%%%%%%%%%%%%% fig1
% Figure environment removed
%%%%%%%%%%%%%%%%%%%%%%%%%%%%%%%%%%%%%%%%%%%%%%%%%%%%%%% end of fig1

%%%%%%%%%%%%%%%%%%%%%%%%%%%%%%%%%%%%%%%%%%%%%%%%%%%%%%%%%%%%
\section{Cosmological model}
\label{sec:cosmol}
%%%%%%%%%%%%%%%%%%%%%%%%%%%%%%%%%%%%%%%%%%%%%%%%%%%%%%%%%%%%
To study the influence of the peak-like features of density perturbations spectrum on the evolution of dark matter haloes we focused on the power spectra, which can be defined as a product of the standard $\Lambda$CDM spectrum and a Gaussian bump of the following form: 
%
\begin{equation}
    1 + A \cdot \exp \left( -\frac{(\log(k)-\log(k_0))^2}{\sigma_k^2} \right), 
    \label{bumps}
\end{equation}
%
where $k$ is a wave number, $A$, $k_0$, and $\sigma_k$ are bump parameters. We assume an equal value of $\sigma_k=0.1$ in all models. Numerical values of the parameters can be found in the Table~\ref{tab:sim}, whereas the Fig.~\ref{fig:spectra} illustrates the matter spectra. One can note that the shape (\ref{bumps}) is not predicted directly by simple modifications of the inflation model (see, e.g., \cite{2023JCAP...04..011I}). We consider the Gaussian shape as a universal approximation of the peak shape in various models.

The width of the peak, $\sigma_k$, does not play an important role in our studies: it is easy to show using theoretical halo mass functions (e.g., PS or ST) that the mass function does not change when $A\sigma_k = const$ for $\sigma_k \ll 1$. So there is enough to vary only one parameter, in our case the amplitude $A$. We do not consider bumps with $\sigma_k\gtrsim 1$ as this would made the study more complicated.

%==================================tab 1
\begin{table*}
\caption{Most relevant parameters of the simulation suite.} 
\centering
\begin{tabular}{llllllll}
\hline
Main suite:              & $\Lambda CDM$ & $gauss\_1$ & $gauss\_2$ & $gauss\_3$ & $gauss\_4$ & $gauss\_5$     \\ \hline
Box size $(Mpc/h)$       & 5.0           & 5.0        & 5.0        & 5.0        & 5.0        & 5.0            \\
N. particles $N_{total}$ & $512^3$       & $512^3$    & $512^3$    & $512^3$    & $512^3$    & $512^3$        \\
Initial redshift         & $300$         & $10^3$     & $10^3$     & $10^3$     & $10^3$     & $10^3$         \\
Final redshift           & $8$           & $8$        & $8$        & $8$        & $8$        & $8$         \\
$k_0$                    & --            & 7          & 15         & 30         & 80         & 80             \\
$A$                      & 0             & 20         & 20         & 20         & 10         & 3              \\
$\sigma$                 & --            & 0.1        & 0.1        & 0.1        & 0.1        & 0.1            \\ 
\hline
\end{tabular}
\label{tab:sim}
\end{table*}
%===============================end tab 1

All simulations share the same cosmological parameters in agreement with  the values obtained by the \citet{planck}, i.e.
$\Omega_m=0.31$, $\Omega_{\Lambda}=1-\Omega_m=0.69$,
$\Omega_b=0.048$, $h=0.67$, $n_s=0.96$.

%Вот тут непонятно: если спектры нормированы на анизотропию реликта, то $\sima_8$ у них у всех будет разный. Это противоречит тексту.
% Да, все так, разные. Там сигма-8 вообще в параметр файле не задается, а считается в момент генерации нач. условий для каждого конкретного случая. Надо вообще ее выпилить отсюда, чтобы глаз не мозолила. (выпилил)

As one can see from the Table~\ref{tab:sim}, all modified spectra from \textit{gauss\_1} to \textit{gauss\_4} have Gaussian bump maxima at different values of $k_0$. The last spectrum, \textit{gauss\_5}, is the exception, and has the same $k_0$ as \textit{gauss\_4}, but smaller amplitude $A$. This was done to study the case, where the amplitude of the Gaussian bump has small impact on the resulting halo mass function.

%%%%%%%%%%%%%%%%%%%%%%%%%%%%%%%%%%%%%%%%%%%%%%%%%%%%
\section{N-body Simulations}  
\label{sec:simulations}
%%%%%%%%%%%%%%%%%%%%%%%%%%%%%%%%%%%%%%%%%%%%%%%%%%%%%

We have run a series of 6 dark matter only simulations in a box size of $(5\,Mpc/h)^3$, $512^3$ particles each.
5 of simulations correspond to different modified matter power spectra, and one corresponds to the standard $\Lambda CDM$ one.

For our simulation suite we use the publicly available N-body code \texttt{GADGET-2}\footnote{http://wwwmpa.mpa-garching.mpg.de/~volker/gadget/} \citep{gadget}, which is widely used for cosmological simulations. This code uses a combined Tree + Particle Mesh (TreeMP) algorithm to estimate the gravitational accelerations for each particle by decomposing the gravitational forces into a long range term, computed from Particle-Mesh methods, and short scale interactions from the nearest neighbours using Tree methods, and can be used with periodic boundary conditions in the comoving frame. The code is designed to be MPI parallel, enabling it to efficiently utilise distributed computing resources by dividing the computational tasks among multiple processors, resulting in faster execution and scalability (i.e. $\cal{O}$$(N\log N)$), so it can handle a large number of particles with reasonable computational resources. 

All simulations with modified matter spectra start at $z = 1000$ in order to account for potential early formation of virialised structures,  while the simulation with $\Lambda CDM$ spectrum starts at $z = 300$. The final redshift of the simulations was set to $z = 8$, in order to reduce possible artefacts due to space periodicity of initial conditions in the relatively small-sized box.

We generate initial conditions for our simulations using publicly available code \texttt{ginnungagap}\footnote{https://github.com/ginnungagapgroup/ginnungagap}, the matter power spectrum is defined for each simulation individually by applying the transform function (\ref{bumps}), and where $\Lambda CDM$ power spectrum is generated with publicly available code CLASS \citep{CLASS}.

A total of 100 snapshots for each simulation were stored at redshift intervals equally-spaced in logarithmic scale, starting from $z = 35$ to $z = 8$. Halo analysis was performed with publicly available code \texttt{AHF}\footnote{http://popia.ft.uam.es/AHF/} \citep{AHF} with the assumption that each halo consists of no less than 50 particles.

%%%%%%%%%%%%%%%%%%%%%%%%%%%%%%%%%%%%%%%%%%%%%%%%%%%%%%%%%%
\section{Halo mass function}   
\label{sec:hmf}
%%%%%%%%%%%%%%%%%%%%%%%%%%%%%%%%%%%%%%%%%%%%%%%%%%%%%%%%%%
For each simulation we plotted a halo mass function (HMF) at two different redshifts: $z \simeq 13$ and $z \simeq 8$, where the former was chosen as a typical high-redshift value for the most distant galaxies found today.
%an arbitrary high-redshift value, and because at higher redshift there are not enough haloes to plot a smooth HMF for the cube $(5\, Mpc/h)^3$. 
While $z \simeq 8$ is the final redshift, up to which our simulation was calculated.

The comparison of simulated mass functions with the theoretical (ST) ones is shown in Fig.~\ref{fig:hmf}. Each panel (from left to right and from top to bottom) includes HMFs for $\Lambda CDM$ (\textit{black} lines) spectrum and one of the modified spectra (see legend for the specific colour). \textit{Solid} lines represent HMF for $z \simeq 8$, while \textit{dashed} lines do HMF with $z \simeq 13$. As expected, an increase in the power spectrum amplitude at certain wave numbers results in a corresponding increase in the abundance of haloes with masses associated with those wave numbers, as demonstrated by the Fig.~\ref{fig:hmf}.

In addition, each panel includes the corresponding HMFs, obtained from ST approximation\footnote{However, theoretical HMF with the \textit{gauss\_5} spectrum is not included in the right low panel of the Fig.~\ref{fig:hmf} to avoid clutter, and because it differs very slightly from \textit{gauss\_4}.}.
%\textbf{where exactly?} я чет не нашел ссылок на конкретные работы, но полагаю, что это что-то времен царя Гороха. Вот в ссылке от Валагеаса там про это тоже говорится, но ссылок неть =( Понятно
% надо у Сереги будет поспрашивать, когда он вернется
%it has been shown that HMFs, obtained with PS formalism, are in good agreement with the results from numerical cosmological simulations with the standard $\Lambda CDM$ spectrum \citep{Valageas_2009}.
One can see from Fig.~\ref{fig:hmf} that HMFs obtained from simulations with modified power spectrum appear to be in good accordance with the HMFs, obtained via ST approximation for the corresponding spectra. We compute ST HMF using the real space spherical top-hat window function. We have checked that it gives better coincidence with simulated HMFs than using Fourier-space top-hat or Gaussian window function for the bumpy power spectra. %However, just as in case with the standard $\Lambda CDM$ spectrum, PS slightly overestimates the low-mass tail and it underestimates the high-mass tail, compared with numerical results.

%=======================================fig 2
% Figure environment removed
%=========================================end fig 2

Considering that our simulations final redshift is $z = 8$, in Fig.~\ref{fig:sh_tor} we also compare the theoretical (ST) HMFs with the standard $\Lambda$CDM spectrum and the modified spectrum $gauss\_1$ at redshifts $z = 9$, 6, and 0. As one can see, in case of $z=6$ and $z=0$ lines, as redshift decreases, the difference between HMFs for $\Lambda CDM$ spectrum and HMFs for modified spectrum diminishes. This trend suggests that the impact of the modification on the HMF becomes less significant as the Universe evolves to lower redshifts.

Additionally, we had some concerns regarding the statistical adequacy of our simulation suite due to a relatively small size of simulation boxes $(5\,Mpc/h)^3$. Thus, we considered several snapshots from the simulation Extremely Small MultiDark Planck (\texttt{ESMD}) carried out by an international consortium within the framework of the MultiDark project\footnote{http://www.multidark.es}, and available at CosmoSim database\footnote{http://www.cosmosim.org}.
%ссылку Серега обещал дать, когда вернется -- там как-то хитро надо ссылаться, потому что она не в открытом доступе
\texttt{ESMD} has a much larger box size, $(64 Mpc/h)^3$. As one can see, HMFs derived from the \texttt{ESMD} simulation, represented by the \textit{dashed black} line on the Fig.~\ref{fig:sh_tor}, closely aligns with the theoretical ST HMFs for the $\Lambda$CDM spectrum, which indicates that ST HMFs that we plotted does not noticeably overestimate the number of haloes in the case of our simulation suite as well.

%============================================fig 3
% Figure environment removed
%==============================================end fig 3

Overall, the close agreement between the ST~HMFs and the simulated data in case of both $\Lambda$CDM spectrum and the modified spectra validates the utility of the ST approximation in capturing the halo mass distribution in cosmological simulations with modified spectra.

%%%%%%%%%%%%%%%%%%%%%%%%%%%%%%%%%%%%%%%%%%%%%%%%%
\section{Extreme value statistics with bumpy power spectra} \label{sec:evs}
%%%%%%%%%%%%%%%%%%%%%%%%%%%%%%%%%%%%%%%%%%%%%%%%%
Within this Section, we discuss how the effects of modified cosmological power spectrum might affect the observations of halo formation in the early Universe. We test how well different power spectra fit the observational data on $z>8$ galaxies with EVS approach.

The data we use are presented in the Table~\ref{tab:sources} and also displayed on the Fig.~\ref{fig:mass_vs_z}. Filled \textit{cyan} dots represent the first three sources from Table~\ref{tab:sources}, while filled \textit{pink} dots represent the rest of the sources from the Table~\ref{tab:sources}. When comparing observations to the theory, one needs to take into account the Eddington Bias \citep{Eddington13}. Due to the steepness of the HMF and galaxy stellar mass function, measurements of low mass objects can be easily upscattered. White-space dots with coloured border represent masses, corrected for Eddington Bias as follows:
%
\begin{align}
\ln M_{\mathrm{Edd}}=\ln M_{\mathrm{obs}}+\frac{1}{2} \epsilon \sigma_{\ln M}^2 \text {, }
\end{align}
%
where $\epsilon$ is the local slope of the underlying halo mass function, and $\sigma_{\ln M}$ is the uncertainty in the halo/stellar mass estimate.

%=================================================fig4
% Figure environment removed
%================================================end fig4

%In the \textit{left} panel of Fig.~\ref{fig:mass_vs_z}, we present an approximation of the most massive haloes using the Press-Schechter (PS) formalism. The PS mass function is utilised for a given spectrum, and we determine the approximated halo mass value at given redshift $z$ using the Extreme Value Statistics (EVS) approach, applied in \cite{Lovell22} \citep{Gumbel58}.
For the given area of the survey we compute the PDF of the most massive halo as a function of $z$ for $\cal{N}$ observations as:
%
\begin{align}
\Phi(M_{max}=m, \mathcal{N}) = \mathcal{N} f(m) \left[ F \right]^{\mathcal{N}-1}\,,
\label{eqn:evs}
\end{align}
%
where $f(m)$ and $F(m)$ for a fixed fraction of the sky $f_{sky}$ between redshifts $z_{\min}$ and $z_{\max}$ are calculated as follows:
%
\begin{align}
f(m) &= \frac{f_{\text{sky}}}{n_{\text{tot}}} \left[ \int_{z_{\min}}^{z_{\max}} dz \frac{dV}{dz} \frac{dn(m, z)}{dm} \right]\,, \\
F(m) &= \frac{f_{\text{sky}}}{n_{\text{tot}}} \left[ \int_{z_{\min}}^{z_{\max}} \int_{-\infty}^{m} dz \,dM \frac{dV}{dz} \frac{dn(M, z)}{dM} \right]\,,
\end{align}
%
where $n_{tot}$ is a normalisation factor giving the total number density of haloes:
%
\begin{align}
n_{\mathrm{tot}}=f_{\mathrm{sky}}\left[\int_{z_{\min}}^{z_{\max}} \int_{-\infty}^{\infty} d z\, d M \frac{d V}{d z} \frac{d n(M, z)}{d M}\right]\, .
\end{align}
%
Using the aforementioned equations along with the 
Eq.~(\ref{eqn:evs}), while considering an observational survey volume with an area of approximately 40 arcmin$^2$ as reported in \cite{Naidu2022a, Naidu2022b, Labbe23} one can find a EVS halo for a given survey. We plot the median of the PDF for $\mathcal{N}=1$, as well as the $3\sigma$ interval around the median in the \textit{left} panel of Fig.~\ref{fig:mass_vs_z}.

The thin solid coloured lines on the \textit{left} panel of Fig.~\ref{fig:mass_vs_z} correspond to the ST approximation, while the thick solid lines on the \textit{right} panel represent the most massive halo's mass in our numerical simulations for a given redshift. The lines are colour-coded in the same manner as in Fig.~\ref{fig:spectra} and Fig.~\ref{fig:hmf}. The $3\sigma$ range, obtained from EVS statistics calculations is depicted as filled \textit{light-grey} area and depicted in both \textit{left} and \textit{right} panels of the Fig.~\ref{fig:mass_vs_z} only for the $\Lambda$CDM model. The $3\sigma$ range for other models has similar width as for the $\Lambda$CDM model, i.e. from approximately $1/20$ to $20$ of the median. Additionally, both the masses approximated from theoretical HMF at the \textit{left} panel and the halo masses from simulations at the \textit{right} panel were adjusted by the baryon fraction, $f_b = 0.157$, while the stellar fraction $f_*$ was assumed $f_* = 1$. Note, that there is enough data to suggest that this quantity is at least an order of magnitude smaller and depends on both the redshift and the halo mass \citep{Silk2018}.

%=================INTERMEDIATE CONCLUSIONS=======================
%--1--
%As can be seen from Fig.~\ref{fig:mass_vs_z}, massive haloes start to appear at much earlier redshifts in the simulations with certain modified spectra (e.g., $gauss_1$, $gauss_2$, $gauss_3$) compared to the $\Lambda$CDM model. 
%This indicates that the enhanced power spectrum amplitude at certain wave numbers, as introduced in the modified spectra, might accelerate the formation of haloes. 
%Furthermore, we find that the masses of the most massive haloes in the simulations with modified spectra differ significantly from those in the $\Lambda$CDM model. Initially, the modified spectra simulations yield higher halo masses, but as redshift decreases, the halo masses tend to converge to the $\Lambda$CDM model, which is apparent for all modified spectra, except $gauss_4$ and $gauss_5$.\\
As can be seen from Fig.~\ref{fig:mass_vs_z}, the masses of the most massive haloes in the simulations with modified spectra differ significantly from those in the $\Lambda$CDM model above some redshift, which depends on the position of the bump. At smaller redshifts this quantity converges to the $\Lambda$CDM model. In particular, our model $gauss\_1$ fits the observations much better than the $\Lambda$CDM. In this model, all the observational points are within $2\sigma$ from the median.
%--2--

Both masses approximated from ST and the halo masses from simulations with modified spectra exhibit a similar trend with the deviations less than 30\% for the majority of spectra up to $z \approx 12$, although some discrepancy might be observed at higher redshift values (see the \textit{bottom} panel of Fig \ref{fig:mass_vs_z}). This implies that ST approximation can be used for calculating the EVS for bumpy power spectra.\\
%......................................................
%--3--
%Some mini-conclusions on observational data...
%......................................................
%--4-- small scales
% The influence of gravitational effects in models 4 and 5 can be important for objects with smaller masses, such as masses of 0-generation starts.
%=================================================table 2
\begin{table*}
\centering
\caption{Comparison of several observed galaxy candidates with $\Lambda$CDM simulations.}
\begin{tabular}{|l|l|c|l}
\hline
Source ID  & Redshift (z)         &  log$_{10}$($M_*$/\msun) & Refs.\\
\hline
CEERS-1749 & $16.0^{+0.6}_{-0.6}$ & $9.6^{+0.2}_{-0.2}$ & \cite{Naidu2022a} \\
GLASS-z12  & $12.4^{+0.1}_{-0.3}$ & $9.1^{+0.3}_{-0.4}$ & \cite{Naidu2022b} \\ 
GLASS-z10  & $10.4^{+0.4}_{-0.5}$ & $9.6^{+0.2}_{-0.4}$ & \cite{Naidu2022b} \\ 
%ID 1514    & $9.85^{+0.18}_{-0.12}$ & $9.8^{+0.2}_{-0.2}$ & \cite{Adams2022} \\
id-2859  & $8.11^{+0.75}_{-2.30}$ & $10.03^{+0.46}_{-0.75}$ & \cite{Labbe23}\\
id-13050 & $8.14^{+2.45}_{-2.33}$ & $10.14^{+0.45}_{-0.54}$ & \cite{Labbe23} \\
id-14924 & $8.83^{+0.67}_{-3.22}$ & $10.02^{+0.90}_{-1.63}$ & \cite{Labbe23} \\
id-16624 & $8.52^{+0.46}_{-0.80}$ & $9.30^{+0.72}_{-0.87}$ & \cite{Labbe23} \\
id-21834 & $8.54^{+1.52}_{-2.92}$ & $9.61^{+0.49}_{-1.50}$ & \cite{Labbe23} \\
id-35300 & $9.08^{+0.40}_{-3.50}$ & $10.40^{+0.60}_{-2.11}$ & \cite{Labbe23} \\
id-39575 & $8.62^{+0.45}_{-2.51}$ & $9.33^{+0.69}_{-1.11}$ & \cite{Labbe23} \\
\hline
\end{tabular}
\label{tab:sources}
\end{table*}
%================================================end table 2

%%%%%%%%%%%%%%%%%%%%%
%%%%%%%%%%%%%%%%%%%%%%%%%%%%%%%%%%%%
\section{Discussion and conclusions}   
\label{sec:conclusions}
%%%%%%%%%%%%%%%%%%%%%%%%%%%%%%%%%%%%

In the paper we studied a cosmological model with modified power spectrum of density perturbations. In this model a matter power spectrum has a Gaussian bump with a certain amplitude $A$ and a scale location $k_0$ (see eq.(\ref{bumps})). We considered five different models with parameter values summarised in the Table~\ref{tab:sim}. All models were studied with both numerical N-body simulations and analytical ST approximation. The latter supported the results of the simulations and was used to relate the observational data with our analysis using the EVS approach.

We have found that models with a bumpy spectra have higher abundance of haloes which could host first galaxies observed at $z>10$. At smaller redshifts the difference in the mass function with respect to the $\Lambda$CDM model vanishes. Models with the bumpy power spectrum can better explain the results recently obtained from the observations of JWST, if the stellar masses from the literature \citep{Naidu2022b,Labbe23} are used and converted into halo masses using the standard baryon fraction. In particular, a bump with the amplitude $A=20$ and position at $k_0=7$~$h$/Mpc can explain the presence of objects with the mass $\sim10^{10}$~M$_\odot$ at $z>15$. It is possible to fine-tune the bump for the particular value of the star formation efficiency, but we haven't done this exercise because the data on halo masses, to our opinion, is yet not enough reliable to derive the exact parameters of the bump from it. Our goal was to show that the models with the bumpy spectrum in principle can explain the overabundance of haloes at high redshifts and converge to the $\Lambda$CDM at low redshifts.

Furthermore, a bump at $k\sim7-10$~$h$/Mpc in the power spectrum can have several other observational consequences. As can be seen, e.g., from our Fig.~\ref{fig:mass_vs_z}, in the bumpy models haloes with masses below $10^{10}$~M$_\odot$ form earlier than in the $\Lambda$CDM. In particular, in the volume considered, haloes with $M=10^7$~M$_\odot$ appear at $z\approx20$ in the $\Lambda$CDM, and at $z\approx 35$ in $gauss\_1$ model. Thus, in the models with the bump the first stars would appear earlier, and there will be more time to seed and grow supermassive black holes. The next step of testing the models with the bump would be to predict the 21~cm signal from Reionization in these models (see, e.g., \citet{Munoz20}).

%Numerical simulations have demonstrated the enhanced power in halo mass function at high redshift. The Press-Schechter formalism has been tested for such unusual kind of spectra and demonstrated a good agreement with N-body results.  

%We found that in case of small $A$, the behaviour of halo mass functions aims for a standard $\Lambda$CDM one. 

So, the considered class of cosmological models can be responsible for early dark matter halo formation and results in early star formation and galaxy nuclei activity. 
%A theoretical basement of such a model is a multiple-stage inflation, which can also provide a creation of primordial black holes.  

%%%%%%%%%%%%%%%%%%%%%
%%%%%%%%%%%%%%%%%%%%%
\section*{Acknowledgements}
Authors thank A.G. Doroshkevich and P.B. Ivanov for fruitful discussions.

\section*{Data availability}
The data underlying this article will be shared on reasonable request to the corresponding author.

%%%%%%%%%%%%%%%%%%%%%%%%%%%%%%%%%%%%%%%%%%%%%%%%%%

%%%%%%%%%%%%%%%%%%%% REFERENCES %%%%%%%%%%%%%%%%%%

\bibliographystyle{mnras}
\bibliography{refs} 

%%%%%%%%%%%%%%%%% APPENDICES %%%%%%%%%%%%%%%%%%%%%



%%%%%%%%%%%%%%%%%%%%%%%%%%%%%%%%%%%%%%%%%%%%%%%%%


% Don't change these lines
\bsp	% typesetting comment
\label{lastpage}
\end{document}

% End of mnras_template.tex