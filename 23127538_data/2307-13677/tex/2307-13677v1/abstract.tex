\begin{abstract}
	Many data analytic systems have adopted a newly emerging compute resource, serverless (SL), to handle data analytics 
	queries in a timely and cost-efficient manner, i.e., serverless data analytics.
	While these systems can start processing queries quickly thanks to the agility and scalability of SL, 
	they may encounter performance- and cost-bottlenecks based on workloads due to SL's worse 
	performance and more expensive cost than traditional compute resources, e.g., virtual machine (VM).
	% need to mention trade
	In this paper, we introduce \textit{Smartpick}, a SL-enabled scalable data analytics 
	system that exploits SL and VM together to realize composite benefits, 
	i.e., agility from SL and better performance with reduced cost from VM.
	Smartpick uses a machine learning prediction scheme, decision-tree based Random Forest with Bayesian Optimizer, 
	to determine SL and VM configurations, i.e., how many SL and VM instances for queries, that meet cost-performance goals.
	%by predicting data analytics workloads.
	Smartpick offers a \textit{knob} for applications to allow them to explore a richer cost-performance tradeoff space opened by exploiting 
	SL and VM together. 
	To maximize the benefits of SL, Smartpick supports a simple but strong mechanism, called \textit{relay-instances}. 
	Smartpick also supports event-driven prediction model retraining to deal with workload dynamics.
	A Smartpick prototype was implemented on Spark and deployed on live test-beds, Amazon AWS and Google Cloud Platform. %, to illustrate its effectiveness.
	Evaluation results indicate 97.05\% and 83.49\% prediction accuracies respectively with up to 50\% cost reduction as opposed to the baselines. 
	%Furthermore, the model efficiently navigates the large cost-performance tradeoff space and handles dynamics in near real-time.
	The results also confirm that Smartpick allows data analytics applications to navigate the richer cost-performance tradeoff space efficiently and to handle workload dynamics effectively and automatically.%quickly.
\end{abstract}
\begin{CCSXML}
	<ccs2012>
	<concept>
	<concept_id>10010520.10010521.10010537.10003100</concept_id>
	<concept_desc>Computer systems organization~Cloud computing</concept_desc>
	<concept_significance>500</concept_significance>
	</concept>
	<concept>
	<concept_id>10010147.10010257.10010293</concept_id>
	<concept_desc>Computing methodologies~Machine learning approaches</concept_desc>
	<concept_significance>300</concept_significance>
	</concept>
	<concept>
	<concept_id>10010147.10010341.10010342</concept_id>
	<concept_desc>Computing methodologies~Model development and analysis</concept_desc>
	<concept_significance>300</concept_significance>
	</concept>
	<concept>
	<concept_id>10010147.10010919</concept_id>
	<concept_desc>Computing methodologies~Distributed computing methodologies</concept_desc>
	<concept_significance>300</concept_significance>
	</concept>
	</ccs2012>
\end{CCSXML}

\ccsdesc[500]{Computer systems organization~Cloud computing}
\ccsdesc[300]{Computing methodologies~Machine learning approaches}
\ccsdesc[300]{Computing methodologies~Model development and analysis}
\ccsdesc[300]{Computing methodologies~Distributed computing methodologies}

%%
%% Keywords. The author(s) should pick words that accurately describe
%% the work being presented. Separate the keywords with commas.
\keywords{serverless-enabled, machine learning, prediction model, cost-performance tradeoff, relay}