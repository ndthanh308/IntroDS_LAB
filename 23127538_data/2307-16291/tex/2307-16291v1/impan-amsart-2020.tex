\documentclass[12pt,intlimits]{amsart}

\synctex=1 
% Modif. February 26, 2020
% Send comments to publ@impan.pl

% Using pdflatex is preferred

\usepackage{amssymb}

%% Optional, but useful:
\usepackage{enumitem}




\makeatletter
\@namedef{subjclassname@2020}{%
  \textup{2020} Mathematics Subject Classification}
\makeatother


\allowdisplaybreaks

%% If you are using letters of the Polish alphabet, add 
\usepackage[T1]{fontenc}
%% E.g. the name "Zoladz" is then coded \.Zo{\l}\k{a}d\'z




%% Numbered objects of "theorem" style (text italicized).
%% Below, the optional parameters indicate that all objects are numbered together, and "by section".
%% However, you are welcome to use any other numbering system of your choice, as well as your own abbreviations.

\newtheorem{theorem}{Theorem}[section]
\newtheorem{corollary}[theorem]{Corollary}
\newtheorem{lemma}[theorem]{Lemma}
\newtheorem{proposition}[theorem]{Proposition}
\newtheorem{problem}[theorem]{Problem}

%% A numbered theorem with a fancy name:

\newtheorem{mainthm}[theorem]{Main Theorem}

%% Numbered objects of "non-theorem" style (text roman):

\theoremstyle{definition}
\newtheorem{definition}[theorem]{Definition}
\newtheorem{remark}[theorem]{Remark}
\newtheorem{example}[theorem]{Example}

%% An unnumbered object:

\newtheorem*{xrem}{Remark}


%% Equations numbered by section (optional):

\numberwithin{equation}{section}


%%%%%%%%%%% For IMPAN journals:

\frenchspacing

\textwidth=13.5cm
\textheight=23cm
\parindent=16pt
\oddsidemargin=-0.5cm
\evensidemargin=-0.5cm
\topmargin=-0.5cm

%%%%%%%%%%%%%%%%%%%%%%%%%%%%%%%%%%%
%%%%%%%%%%%%%%%%%%%%%%%%%%%%%%%%%%%

%%%% Put your macros here:

%%%% Here are two examples:

\DeclareMathOperator{\len}{length}

\newcommand{\obj}[3]{\mathcal{F}^{#1}\mathbb{S}^{#2}\mathbf{G}_{#3}}


%%%%%%%%%%%%%


%%% User defined commands

\newcommand{\essi}{\operatornamewithlimits{ess\,inf}}
\newcommand{\esss}{\operatornamewithlimits{ess\,sup}}


% Symbols for sets of numbers

\newcommand{\R}{\mathbb R}
\newcommand{\N}{\mathbb N}
\newcommand{\C}{\mathbb{C}}
\newcommand{\Z}{\mathbb Z}


\newcommand{\subR}{{\mathbb R}}
\newcommand{\subRn}{{{\mathbb R}^n}}


% Useful operators

\DeclareMathOperator{\supp}{supp}
\DeclareMathOperator{\dist}{dist}
\DeclareMathOperator{\diam}{diam}
\DeclareMathOperator*{\essinf}{ess\,inf}
\DeclareMathOperator*{\esssup}{ess\,sup}
\DeclareMathOperator{\sgn}{sgn}
\DeclareMathOperator{\osc}{\large{osc}}

% Abbreviations useful for variable Lebesgue spaces

\newcommand{\avf}{{\langle f\rangle}}
\newcommand{\avfa}{{\langle f_1\rangle}}
\newcommand{\avfb}{{\langle f_2\rangle}}
\newcommand{\avg}{{\langle g\rangle}}
\newcommand{\avh}{{\langle h\rangle}}
\newcommand{\avpaf}{{\langle f\rangle_{p_{1}(\cdot),Q}}}
\newcommand{\avpafa}{{\langle f_1\rangle_{p_{1}(\cdot),Q}}}
\newcommand{\avpbg}{{\langle g\rangle_{p_{2}(\cdot),Q}}}
\newcommand{\avpbfb}{{\langle f_2\rangle_{p_{2}(\cdot),Q}}}
\newcommand{\pap}{{p_{1}(\cdot)}}
\newcommand{\pbp}{{p_{2}(\cdot)}}
\newcommand{\cpap}{{p_{1}^{\prime}(\cdot)}}
\newcommand{\cpbp}{{p_{2}^{\prime}(\cdot)}}
\newcommand{\pjp}{{p_{j}(\cdot)}}
\newcommand{\cpjp}{{p_{j}^{\prime}(\cdot)}}
\newcommand{\pp}{{p(\cdot)}}
\newcommand{\cpp}{{p'(\cdot)}}
\newcommand{\Lp}{L^{p(\cdot)}}
\newcommand{\Lcp}{L^{p^{\prime}(\cdot)}}
\newcommand{\Lpa}{L^{p_{1}(\cdot)}}
\newcommand{\Lpb}{L^{p_{2}(\cdot)}}
\newcommand{\Lpl}{L_{loc}^{p(\cdot)}}
\newcommand{\Lpal}{L_{loc}^{p_{1}(\cdot)}}
\newcommand{\Lpbl}{L_{loc}^{p_{2}(\cdot)}}
\newcommand{\Pp}{\mathcal P}
\newcommand{\qq}{{q(\cdot)}}
\newcommand{\rr}{{r(\cdot)}}
\newcommand{\sst}{{s(\cdot)}}
\newcommand{\cqq}{{q'(\cdot)}}
\newcommand{\Lq}{L^{q(\cdot)}}
\newcommand{\Lr}{L^{r(\cdot)}}
\newcommand{\Ls}{L^{s(\cdot)}}
\newcommand{\ttt}{{t(\cdot)}}
\newcommand{\uu}{{u(\cdot)}}
\newcommand{\Lu}{L^{u(\cdot)}}
\newcommand{\Hp}{H^{p(\cdot)}}
\newcommand{\M}{\mathcal{M}}
\newcommand{\Q}{{Q_{j}^{k}}}
\newcommand{\vecpp}{{\vec{p}(\cdot)}}
\newcommand{\tfr}{\tau_{r_{m}}^{j}(f)(x)}
\newcommand{\tfR}{\tau_{R}^{j}(f)(x)}
\newcommand{\tfrxk}{\tau_{r_m}^{j}(f)(x_k)}
\newcommand{\tfRxk}{\tau_{R}^{j}(f)(x_k)}

\newcommand{\D}{\mathcal{D}}
\newcommand{\A}{\mathcal{A}}
\newcommand{\F}{\mathcal{F}}
\newcommand{\Rf}{\mathcal{R}}
\newcommand{\ps}{{p^*(\cdot)}}
\newcommand{\B}{\mathcal{B}}
\newcommand{\Ss}{\mathcal{S}}
\newcommand{\Qq}{{\mathcal{Q}}}

\newcommand{\sw}{{\mathcal{S}}}
\newcommand{\swp}{{\mathcal{S}'}}

\newcommand{\BV}{\ensuremath{\mathbf{B\!V}}}
\newcommand{\V}{\ensuremath{\mathbf{V}}}
\newcommand{\const}{\ensuremath{\mathrm{const}}}
\newcommand{\RBV}{\ensuremath{\mathbf{RBV}}}
\newcommand{\RBVv}{\ensuremath{\mathrm{RBV}}}
\newcommand{\AC}{\ensuremath{\mathbf{AC}}}
\newcommand{\Lip}{\ensuremath{\mathbf{Lip}}}
\newcommand{\euler}{e}
\newcommand{\odp}{\ensuremath{ \omega_{\delta}^{p(\cdot)}}   }

\DeclareMathAlphabet\EuRoman{U}{eur}{m}{n}
\SetMathAlphabet\EuRoman{bold}{U}{eur}{b}{n}
\renewcommand{\mathsf}{\EuRoman}

%\usepackage{showframe}

\newcommand{\comment}[1]{\vskip.3cm
	\fbox{%
		\parbox{0.93\linewidth}{\footnotesize #1}}
	\vskip.3cm}


\DeclareMathOperator*{\slim}{s-lim} % definition of strong limit 


\def\Xint#1{\mathchoice 
	{\XXint\displaystyle\textstyle{#1}}% 
	{\XXint\textstyle\scriptstyle{#1}}% 
	{\XXint\scriptstyle\scriptscriptstyle{#1}}% 
	{\XXint\scriptscriptstyle\scriptscriptstyle{#1}}% 
	\!\int} 
\def\XXint#1#2#3{{\setbox0=\hbox{$#1{#2#3}{\int}$} 
		\vcenter{\hbox{$#2#3$}}\kern-.5\wd0}} 
\def\ddashint{\Xint=} 
\def\dashint{\Xint-}
\def\avgint{\Xint-}


\usepackage{commath}
\usepackage{microtype}
\usepackage{mathtools}

\begin{document}


%%%%% To ease editing, for IMPAN journals add:

\baselineskip=17pt

%%%%%%%%%%%%%%%%



\title[Sobolev meets Riesz]{Sobolev meets Riesz: a characterization of weighted Sobolev spaces via weighted Riesz bounded variation spaces}

\author[D. Cruz-Uribe, OFS]{David Cruz-Uribe, OFS}
\address[D. Cruz-Uribe,OFS]{Departament of Mathematics\\ University of Alabama \\ Tuscaloosa, USA}
\email{dcruzuribe@ua.edu}

\author[O. Guzm\'an]{Oscar Guzm\'an}
\address[O. Guzm\'an]{Departament of Sciences and Humanities\\ University of America \\ Bogotá, Colombia and Department of Mathematics \\ United Arab Emirates University \\ Al Ain, United Arab Emirates}
\email{oscar.guzman@profesores.uamerica.edu.co}


\author[H. Rafeiro]{Humberto Rafeiro}
\address[H. Rafeiro]{Department of Mathematics \\ United Arab Emirates University \\ Al Ain, United Arab Emirates}
\email{rafeiro@uaeu.ac.ae}




\date{}


\begin{abstract}
  We introduce weighted Riesz bounded variation spaces
  defined on an open subset of the $n$-dimensional Euclidean space 
  and use them to characterize weighted Sobolev spaces when the weight
  belongs to the Muckenhoupt class.  As an application, using Rubio de
  Francia's extrapolation theory, a
  similar characterization of the variable exponent Sobolev spaces via
  variable exponent Riesz bounded variation spaces is obtained.
\end{abstract}



\subjclass[2010]{42B25,42B35}

\keywords{Riesz bounded variation spaces, Muckenhoupt $A_p$ weights, Sobolev spaces,  Rubio de Francia extrapolation} 
\maketitle



	
	\section{Introduction}
	\label{sec:intro}
	In the foundational paper \cite{Riesz1910}, F. Riesz proved
        (in modern terminology) that %given an interval $I \subset \subR  $
        an absolutely continuous
        function $f: I \to \subR$ % then  $f$
        belongs % (using contemporary terminology)
	to the Sobolev space $W^{1,p}(I)$, $1 < p < \infty$ and $I \subset \subR$ an interval, if and only if
        %
	\begin{equation}\label{eq:RieszVardefinition}
		\sup\sum_{j}\frac{|f(x_{j})-f(x_{j-1})|^{p}}{|x_{j}-x_{j-1}|^{p-1}}<\infty, 
	\end{equation}
	where the supremun is taken over all finite
        partitions %$\{[x_{j-1},x_{j}]\}$
	of $I$. The quantity in \eqref{eq:RieszVardefinition} is
        called the \emph{Riesz $p$-variation of $f$ on $I$}. We
        refer the interested readers to \cite{Appell-Merentes} for a
        comprehensive survey of the classical theory of bounded variation spaces.
	There has been progress  on
        generalizations of  Riesz bounded variation spaces, % based on other Banach function spaces besides Lebesgue spaces:
        for example, 
     weighted Riesz spaces~\cite{MR4154132}, and 
 variable exponent Riesz spaces \cite{castillo2016variable,Castillo2019,
   Kakochashvili_OnRiesz_2016}, to name just two.

 More recently, work has been done on extending the notion of Riesz bounded
 variation spaces to the case of  functions defined
 on general domains in $\R^n$, $n>1$, see, for instance, the
 works of Angeloni~\cite{Angeloni2017}, Barza and
 Lind~\cite{Barza-Lind2015}, 
 Bojarski~\cite{bojarski2011remarks}, and Mal\'y~\cite{Maly1999}.  We
 are particularly interested in \cite{Barza-Lind2015} since the space  
 $RBV^p(\Omega)$, $\Omega \subset \mathbb R^n$, is introduced and is defined as the set of all functions $f$ such that
 %
 \[ V_p(f; \Omega) = \sup \bigg[ \sum_{B_k\in \D} \bigg(
   \frac{\osc_{B_k}(f)}{r_k}\bigg)^p |B_k| \bigg]^{1/p} < \infty, \]
%nw
 where the supremum is taken over all countable
              collections $\D=\{B_k\}_{k=1}^\infty$ of disjoint balls of radius
              $r_{k}$ contained in $\Omega$.  It is shown that if
              $p>n$, then $f\in
              RBV^p(\Omega)$ if and only if  $f \in W^{1,p}(\Omega)$. % (that is, the space of all functions $f\in L^p(\Omega)$ with $\nabla f \in L^p(\Omega))$. 

              The goal of this paper is to extend this result and
              prove a version of the Riesz theorem characterizing
              weighted Sobolev spaces in $\R^n$,
              $W^{1,p}\left(\Omega, w\right)$, for weights $w$ in the
              Muckenhoupt class $A_p$, using an appropriate weighted version of $RBV^p$.  To state our main results, we give some
              essential definitions; we defer technical details and
              some additional definitions until the next section.
              Hereafter, $\Omega \subset \R^n$ will be an open set.
              Given a weight $w\in A_\infty$, %(that is, the union of the Muckenhoupt $A_p$ classes),
              let
              \begin{equation}\label{eq:rw}
                r_w := \inf\{q>1 : w\in A_q\}.
\end{equation}
                For
              $1\leq p<\infty$, let $W^{1,p}\left(\Omega, w\right)$
              denote the collection of functions
              $f\in L^{p}(\Omega,w) $ whose weak derivatives $D_{j}f$
              belong to $L^{p}\left(\Omega,w\right)$,
              $1\leqslant p <\infty$.  The space $W^{1,p}(\Omega,w)$
              is the collection of functions $f$ with weak derivatives
              endowed with the norm
        %
	\begin{equation}\label{def:Sobolev}
          \|f\|_{W^{1,p}(\Omega,w)}
          =  \left(\int_{\Omega}| f(x)|^{p}w(x)\dif{x}\right)^{1/p}
          +\left(\int_{\Omega}|\nabla f(x)|^{p}w(x)\dif{x}\right)^{1/p}.
              \end{equation}
              %
Weighted Sobolev spaces were introduced in~\cite{MR643158} and have
applications to the study of degenerate PDEs.  
              
              Given a measurable function $f : \Omega \rightarrow\R$ and $1\leq
              p<\infty$, we say that
              $f$ is of weighted Riesz bounded $p$-variation on
              $\Omega$, denoted by $f\in RBV^p(\Omega,w)$, if
              %
              \[ 	V_{p}\left(f;\Omega, w\right)
                \coloneqq
                \sup \bigg[\sum_{B_k \in
                    \D}\left(\frac{\osc_{B_k}(f)}{r_{k}}\right)^{p}w(B_k)\bigg]^{1/p}
                <\infty. \]
              %se
             
Our first main result, whose proof we postpone to Section \ref{sec:proof main theorem}, is the following. 
	\begin{theorem}\label{theo:RBVpq-W1p}
          Given $\Omega\subseteq \subRn$ an open set, assume that
          $p>n$. Let $w\in A_p$ be such that $p>n r_w$, with $r_w$ defined in \eqref{eq:rw}. Then
          $f\in W^{1,p}\left(\Omega,w\right)$ if and only if the function $f$ is
          continuous (perhaps after being redefined on a null set) % of measure $0$)
          and $f\in RBV^{p}\left(\Omega,w\right)\cap L^p(\Omega,w)$. Furthermore,
          %
		\begin{equation}\label{eq:Main_Inequality}
                  \|\nabla f\|_{L^{p}\left(\Omega,w\right)}
                  \lesssim  V_{p}\left(f;\Omega, w\right)
                  \lesssim \|\nabla f\|_{L^{p}\left(\Omega,w\right)},
		\end{equation}
                %
		where the implicit constants depend on $p, n$, and $[w]_{A_{p}}$. 	
              \end{theorem}
              
	\begin{remark}
          The unweighted version of Theorem~\ref{theo:RBVpq-W1p}
          (i.e., when $w=1$ so $r_w=1$) % and $p>n$)
          was
          proved by Barza and Lind in
          \cite{Barza-Lind2015}.   We note in passing an omission in
          the statement of their main result:  the
          hypothesis  that $f\in L^p(\Omega)$ is missing.  It is straightforward
          to give a counter-example without this assumption.  Let
          $\Omega=\R^n$ and let $f\equiv 1$.  Then $f$ is
          continuous and $f \in RBV^p(\R^n)$ for any $p>n$, but
          $f \not\in W^{1,p}(\R^n)$. % since it is not in $L^p(\R^n)$.  
	\end{remark}

        As a corollary to the proof of Theorem~\ref{theo:RBVpq-W1p}, we
        have that the left-hand side  inequality
        in~\eqref{eq:Main_Inequality} is always true.

        \begin{corollary} \label{cor:RBV-embed}
          Let $1\leq p<\infty$ and $w$ be a weight.  If $f\in
          RBV^p(\Omega,w)$, then $\|\nabla f\|_{L^p(\Omega,w)} \lesssim
          V_p(f; \Omega, w)$.
        \end{corollary}
        
        \begin{remark}
          Although we are primarily concerned with spaces defined with
              $A_{p}$ weights, Corollary~\ref{cor:RBV-embed} and
              Corollary~\ref{cor:doubling} below show that
              some of our results are true with
              weaker assumptions on the weights.  Many classical
              results for Sobolev spaces can be extended to the
              weighted case by using a wider class of weights: e.g.,  see
              \cite[Chapter 1]{Tero} for a discussion of Sobolev
              spaces defined using the so-called $p$-admissible weights.  It is
              an interesting question to determine the weakest
              hypotheses on the weights to define a \emph{rich} theory of weighted
              Riesz
              $p$-variation. 
            \end{remark}

            We can also give a weak-type estimate for the local
            Lipschitz constant of a function.  Recall that a function
            $f$ is said to be \textit{Lipschitz continuous} at $x$ if
	\[
	L_{f}(x)=\varlimsup_{y\rightarrow x}\frac{|f(x)-f(y)|}{|x-y|}<\infty.
	\]
        
	\begin{theorem}\label{prop:Main-Proposition}
		Assume that $w\in A_{p}$, $1\leq p<\infty$, and $f\in RBV^p(\R^n,w)$. Then for all
                $t>0$, 
		\[
                  w\left(\{x\in \subRn :  L_{f}(x)>t\}\right)
                  \lesssim \left(\frac{V_{p}(f; \R^n,w)}{t}\right)^{p}.
		\]
                %
                 The implicit constants depend on $p, n$, and $[w]_{A_{p}}$. 	
              \end{theorem}
              

              

As a corollary to the proof of Theorem~\ref{prop:Main-Proposition}, we
have that the $A_p$ condition can be significantly weakened.  Given any ball $B=B(x,r)$, let
$2B:=B(x,2r)$.   We say that a weight is \emph{doubling} if $w(2B) \leq Cw(B)$ with a constant $C$ independent of $B$.

 \begin{corollary} \label{cor:doubling} 
Theorem~\ref{prop:Main-Proposition}  remains true   %for $1\leq p<\infty$
when $w$ is only assumed to be doubling.
\end{corollary}

A straightforward consequence of
              Theorems~\ref{theo:RBVpq-W1p}
              and~\ref{prop:Main-Proposition} is the following result.

\begin{corollary}\label{cor:Almost-Differentiability-W1p}
  Let $w\in A_p$ and  $1\leq p<\infty$. If $f\in RBV^p(\R^n,w)$, then
  $f$ is differentiable almost everywhere.  Moreover, if  $p>nr_w$ and $f\in W^{1,p}(\subRn, w)$,
then $f$ is differentiable almost everywhere.
\end{corollary}



\begin{remark}
  In the unweighted case, the differentiability of functions in
  $W^{1,p}(\R^n)$, $p>n$, is due to Calder\'on~\cite{MR0045200} (see
  also~\cite[Theorem~6.17]{Heinonen2001}).   The unweighted version of
  Corollary~\ref{cor:Almost-Differentiability-W1p} was proved by Barza
  and Lind~\cite{Barza-Lind2015}; for the proof of the
  differentiability of functions in $RBV^p(\R^n)$ they assume that $p>n$, 
  but this is not needed in the proof.
\end{remark}


When $p=n$, Mal\'y~\cite{Maly1999} defined a space of functions of
bounded $n$-variation that is equivalent to $RBV^n(\Omega)$ and proved
that it is continuously embedded in $W^{1,n}(\Omega)$.  Furthermore, he
showed that the elements of $RBV^n(\Omega)$ are differentiable almost
everywhere.  By modifying the proofs of Theorem~\ref{theo:RBVpq-W1p}
and~\ref{prop:Main-Proposition} we can further extend this result to the
weighted case.



 \begin{theorem}\label{thm:embedding-n-variation-sobolev}
   Let $w \in A_n$ and 
   $f\in RBV^{n}(\Omega,w)\cap L^n(\Omega, w)$. Then
   $f\in W^{1,n}(\Omega,w)$ with $\|\nabla f\|_{L^p(\Omega,w)}\lesssim
   V_n(f; \Omega, w)$.  Moreover, $f $
   is differentiable almost everywhere.
 \end{theorem}


 \medskip
 
        As an application of Theorem~\ref{theo:RBVpq-W1p} and the
        Rubio de Francia extrapolation theory, we extend this result
        to the variable exponent Sobolev spaces.
        Define the variable Lebesgue space $\Lp(\Omega)$ to be all
        functions $f$ such that
        %
        \begin{equation} \label{eqn:var-norm}
          \|f\|_{L^{p(\cdot)}(\Omega)}
          =\inf  \Big\{\lambda>0 :
          \rho_{p(\cdot),\Omega}\left( f/\lambda \right)\leqslant 1 \Big\},
              \end{equation}
              %
	where 
	%
        \begin{equation} \label{eqn:var-modular}
	\rho_{p(\cdot),\Omega}(f):=\int_{\Omega}|f(x)|^{p(x)}\dif{x}.
      \end{equation}
      %
      The variable exponent Sobolev space $W^{1,\pp}(\Omega)$ consists of all functions $f\in
\Lp(\Omega)$ that are weakly differentiable and $\nabla f \in
\Lp(\Omega)$.
        We say $\pp \in LH(\Omega)$, $\pp : \Omega \rightarrow  [1,\infty)$,  if it is log-H\"older continuous
        locally and at infinity, i.e. it satisfies  \eqref{eq:local_log_Holder} and
        \eqref{eq:infinity_log_Holder}, respectively.
        
The space of functions of bounded $\pp$-variation,
$RBV^\pp(\Omega)$, is intuitively defined as the space
$RBV^p(\Omega)$, but with the constant exponent $p$ replaced by a
variable exponent $\pp$.  We defer the precise definition to
Section~\ref{sec:Variable-expoent-section}.  

\begin{theorem}\label{theo:Riesz_Variable-Exponent}
  Given an open set $\Omega \subset \R^n$, let $p(\cdot)\in
  LH(\Omega)$, $n<p_-\leqslant p_+<\infty$. Then 
  $f\in W^{1,p(\cdot)}(\Omega)$ if and only if $f$ is  a continuous function (possibly after being
                redefined on a null set)  and
                $f \in RBV^{p(\cdot)}(\Omega)\cap \Lp(\Omega)$.
                Furthermore,
                %
		\begin{equation}\label{eq: var-exponent_Riesz}
			\|f\|_{RBV^{p(\cdot)}(\Omega)} \approx \|\nabla f\|_{p(\cdot),\Omega}.
                      \end{equation}
                      %
	\end{theorem}    
	

              \medskip
             
	
	
              
	
	
	The remainder of this paper is organized as follows.  In
        Section \ref{Sec:Pre} we give some definitions and auxiliary
        results regarding $A_{p}$ weights, define and give some basic
        properties of variable Lebesgue spaces, and define the
        weighted Riesz bounded variation spaces
        $RBV^{p}(\Omega,w)$. Some embedding results in this new scale
        of functions are established as well. In
        Section~\ref{sec:proof main theorem} we prove
        Theorems~\ref{theo:RBVpq-W1p},~\ref{prop:Main-Proposition},
        and~\ref{thm:embedding-n-variation-sobolev}. Finally, in
        Section~\ref{sec:Variable-expoent-section} we define the
        variable exponent Riesz bounded variation space
        $RBV^{p(\cdot)}(\Omega)$ and prove
        Theorem~\ref{theo:Riesz_Variable-Exponent}.
        % The proof is an application of limited range, Rubio de Francia extrapolation, Theorem~\ref{theo:interpo_result}, as proved in~\cite{Cruz-Wang_2017}.
	
	
	\section{Preliminaries}\label{Sec:Pre}

        Throughout this paper $C,c$ will denote constants, depending
        on the underlying parameters, that may
        change their values even from line to line. Given $A,B>0$, we
        write $A\lesssim B$ if there exists $C>0$ such that
        $A\leqslant CB$. Additionally, if $A\lesssim B$ and
        $B\lesssim A$ simultaneously, we write $A\approx B$.  In the
        sequel we regard $\Omega$ as an open subset of $\subRn$. For
        an open ball with center $x$ and radius $r>0$ we write
        $B(x,r)$. Throughout this paper all the cubes $Q\subset \R^n$
        will have their sides parallel to the coordinate axis.  We
        denote the integral average of $f$ over the measurable subset
        $E$ by
	%
	\[ \langle f \rangle_E := \frac{1}{|E|}\int_E f(x) \dif  x = \avgint_E f(x)\dif x ,\,\, |E|>0,\]
	where $|E|$ denotes the Lebesgue measure of the set $E$. 
	
	
	\subsection{Muckenhoupt $A_p$ weights}
        By a weight we mean a non-negative, locally integrable
        function.  The measure of $E$ induced by the weight $w$ is  $w(E):=\int_{E}w(x)\dif{x}$.
	
A weight $w$ belongs to the Muckenhoupt $A_p$ class, $1<p<\infty$,
denoted $w\in A_p$,  if 
	\[[w]_{A_{p}}=\sup_Q \avgint_Q w\dif x \bigg(\avgint_Q w^{1/(1-p)}\dif x\bigg)^{p-1} <
	\infty, \]
	where the supremun is taken over all cubes (or balls) in
        $\subRn$. When $p=1$, we say that $w\in A_1$ if
        %
        \[ [w]_{A_1} = \sup_Q \esssup_{x\in Q} w(x)^{-1}\avgint_Q
          w\dif x < \infty.  \]
        %
The class $A_{\infty}$ is defined by
	\[
	A_\infty= \bigcup_{p\geqslant 1}A_p.
      \]
      For more information on $A_p$ weights, and for proofs of the
      following results which we will need below, 
      see~\cite{JavierDuBook,grafakosmodern}.
      %
      It follows from H\"older's inequality and the definition that if
      $1\leqslant p_1<p_2<\infty$, then $ A_{p_1} \subset A_{p_2}$.
      Moreover, the $A_p$ classes are left-open:  if $w\in A_p$ for
      some $p>1$, then there exists $\varepsilon>0$ such that $w\in
      A_{p-\varepsilon}$  (see~\cite[Corollary~7.6]{JavierDuBook}).
      Consequently, recalling that we defined
      \[ r_w = \inf\{ q>1 : w \in A_q \}, \]
      if $w\in A_p$, then $r_w<p$, and for every $q>r_w$, $w\in A_q$.
      

	
	\begin{lemma}\label{lemma: GeneralDoublingCon}
		Assume that  $w\in A_p$, $1\leqslant p < \infty$. Then 
		for every measurable subset $E$ of $Q$, 
		\begin{equation*}
			\left(\frac{|E|}{|Q|}\right)^{p}\leqslant [w]_{A_p}\frac{w(E)}{w(Q)}.	   
		\end{equation*}       
	\end{lemma}  
	
	We say that $w\in RH_{s}$, for some $s>1$, if 
	\begin{equation}\label{eq:RHIdef}
		[w]_{RH_{s}} = \sup_Q \frac{ \langle w ^s \rangle_Q ^{1/s}  }{ \langle w  \rangle_Q} < \infty, 
	\end{equation}
	where the supremun is taken over all cubes  in $\subRn$.
        
	\begin{lemma}\label{lemma:Ainfty_RHS}
		Assume that $w\in A_{\infty}$. Then there exists $s>1$, depending on $[w]_{A_p}$, such that $w \in RH_s$.
	\end{lemma}
	
	% \begin{lemma}\label{lemma:left-continuity}
	% 	Suppose  that $w\in A_p$, $1<p<\infty$. Then there exists $\varepsilon>0$ such that $w\in A_{p-\varepsilon}$.
	% \end{lemma}

        
	
	Finally, for $1\leq p<\infty$, $W^{1,p}\left(\Omega,w\right)$ is a Banach
      space (see \cite[Proposition 2.1.2]{Turesson}). Moreover, when
      $w\in A_p$, $W^{1,p}\left(\Omega,w\right)$ is the
      completion of
      $C^{\infty}(\Omega)\cap W^{1,p}\left(\Omega,w\right)$ with
      respect to the norm \eqref{def:Sobolev} (see ~\cite[Theorem
      2.5]{Tero}).
      
	
      \subsection{Variable exponent spaces}
            In this section, we recall the  definition of variable Lebesgue and Sobolev spaces.
      For further information on these spaces, 
      see~\cite{cruz-fiorenza-book,diening-harjulehto-hasto-ruzicka2010}.
      Let $\Pp(\Omega)$ denote the collection of all measurable exponent
      functions  $\pp: \Omega \to [1,\infty)$.  
      Denote the essential supremum and infimum of $\pp$ on a set
      $E\subseteq \subRn$ by $p_{+}(E)$ and $p_{-}(E)$,
      respectively. For the sake of simplicity, we write $p_{-}$ and
      $p_{+}$ when $E=\subRn$.  Hereafter, we will generally assume that
      $p_+<\infty$.   Recall the definition of the
      norm~\eqref{eqn:var-norm} and modular~\eqref{eqn:var-modular}
      associated to the exponent $\pp\in \Pp(\Omega)$.    The variable Lebesgue space
      $L^{p(\cdot)}(\Omega)$ is the collection of all measurable
      functions such that $\|f\|_{\Lp(\Omega)}<\infty$.  When there is
      no confusion about $\Omega$, we will sometimes write $\|f\|_\pp$
      for the norm. 

       Given $\pp\in \mathcal{P}(\Omega)$, we say that $\pp$ satisfies the
       local log-H\"older  continuity condition, and denote this by
       $\pp \in LH_{0}(\Omega)$  if there exists a constant $C_0$ such that
	\begin{equation}\label{eq:local_log_Holder}
		|p(x)-p(y)|\leqslant \frac{C_0}{-\log(|x-y|)},\quad x,y \in \Omega,\quad |x-y|<1/2,
	\end{equation}
	and we say that $\pp$ is  log-H\"older continuous at infinity,
        denoted by $\pp\in LH_{\infty}(\Omega)$ if there exist $p_{\infty}$ and $C_{\infty}>0$ such that
	\begin{equation}\label{eq:infinity_log_Holder}
		|p(x)-p_{\infty}|\leqslant \frac{C_{\infty}}{\log(e+|x|)},\quad x\in\Omega.
              \end{equation}
          %
              When $\pp$ satisfies \eqref{eq:local_log_Holder} and
              \eqref{eq:infinity_log_Holder} we say that $\pp$ is a
              log-H\"older continuous function and denote this by
              $\pp \in LH(\Omega)$. It is well known that the $LH$
              condition is a natural assumption in the variable
              exponent setting. For instance, $LH$ regularity is
              sufficient for the maximal operator to be bounded on
              $L^{p(\cdot)}(\subRn)$, see,  for example,~
              \cite{cruz-fiorenza-book,diening-harjulehto-hasto-ruzicka2010}.
              %The restriction of the domain to $\Omega$ can be eliminated: given any log-H\"older continuous function on a set $\Omega$, it can be extended to a function in $LH(\R^n)$ with the same supremum and infimum (see~\cite[Lemma~2.4]{cruz-fiorenza-book}.)
        

        Given a set $E$,  $0<|E|<\infty$, the harmonic mean of $\pp$ on $E$ is given by 
	\begin{equation*}
		\frac{1}{p_{E}}= \frac{1}{|E|}\int_E \frac{\dif
                  x}{p(x)}. 
	\end{equation*}
      

	
	% \begin{lemma}\label{lemma:modular-lp-norm-relation}
	% 	Let $p \in \mathcal{P}(\Omega)$. If we set 
	% 	\[  p_{*}=
	% 	\begin{cases}
	% 		p_{-},  &\text{if}\quad \|f\|_{p(\cdot),\Omega}<1,\\
	% 		p_{+}, &\text{if}\quad \|f\|_{p(\cdot),\Omega}>1,\\
	% 		1 &\text{if}\quad \|f\|_{p(\cdot),\Omega}=1.
	% 	\end{cases}
	% 	\]
	% 	Then  $\|f\|_{p(\cdot),\Omega}\leqslant \rho_{p(\cdot),\Omega}(f)^{1/p_{*}}.$   	      
	% \end{lemma}
	
	% \begin{lemma}
	% 	Let $p\in \mathcal{P}(\Omega)$ and $f \in L^{p(\cdot)}(\Omega)$. Then
	% 	\begin{equation*}
	% 		\|f\|_{p(\cdot),\Omega} \approx \sup\int_{\Omega}f(x)g(x)\dif{x},
	% 	\end{equation*}
	% 	where the supremum is taken over all   $g\in
        %         L^{p^{\prime}(\cdot)}(\Omega)$ such that
        %         $\|g\|_{p^{\prime}(\cdot),\Omega}\leqslant 1$.
        %         The implicit constants depend only on $\pp$.
	% \end{lemma}
	% \begin{lemma}\label{lemma:holder_ineq}
	% 	Given $p\in \mathcal{P}(\Omega)$,  let $f \in L^{p(\cdot)}(\Omega)$  and $g\in L^{p^{\prime}(\cdot)} (\Omega)$. Then 
	% 	\begin{equation*}
	% 		\int_{\Omega}|f(x)g(x)| \dif x\lesssim \|f\|_{p(\cdot),\Omega}\|g\|_{p^{\prime}(\cdot),\Omega}.
	% 	\end{equation*}
        %       \end{lemma}
              
	% The following result is well-known when $p(\cdot)=p$ is a
        % constant exponent (see \cite[Theorem
        % 13.44]{hewitt2013real}). Here prove it in the variable
        % exponent setting.  
	
	% \begin{lemma}\label{lemma:weak-bounded_Lp}
        %   Given $1<p_{-}\leqslant p_{+}<\infty$, let
        %   $\{f_{k}\}_{k=1}^\infty$ be a bounded sequence of functions
        %   in $L^{p(\cdot)}(\Omega)$ converging pointwise $a.e.$ to some
        %   $f \in L^{p(\cdot)}(\Omega)$. Then $f_{k}$ converges weakly
        %   to $f$ in $L^{p(\cdot)}(\Omega)$.
          
	% 	\begin{proof}
	% 		Since $\{f_k\}$ is uniformly bounded,  Fatou's lemma implies 
	% 		\begin{equation*}
	% 			\rho_{p(\cdot),\Omega}(f)=\int_{\Omega}\lim_{k\rightarrow\infty} |f_{k}(x)|^{p(x)}\dif{x}\leqslant 
	% 			\liminf_{k\rightarrow \infty}\int_{\Omega} |f_{k}(x)|^{p(x)}\dif{x}\leqslant C.
	% 		\end{equation*}
	% 		Hence, for any measurable set $A\subset \Omega$
	% 		\begin{equation}\label{eq:Weak_convergen1}
	% 			\rho_{p(\cdot),A}(f-f_k)\leqslant \rho_{p(\cdot),\Omega}(f-f_k)\leqslant 2^{p_+}(\rho_{p(\cdot),\Omega}(f)+\rho_{p(\cdot),\Omega}(f_k))\leqslant 2^{p_++1}C.
	% 		\end{equation}
	% 		Now, given $0<\varepsilon<1$, let $g\in L^{p^{\prime}(\cdot)}(\Omega)$. The absolute continuity of the integral implies that there exists $\delta>0$ such that, for every measurable set $E \subset \Omega$ for which $|E|<\delta$, we have $\rho_{p^{\prime}(\cdot),E}\left(g\right)<\varepsilon$. Next, we select a set $F\subset\subset \Omega$ such that $\rho_{p^{\prime}(\cdot),F^{c}}\left(g\right)<\varepsilon$. Since $|F|<\infty$ we apply Egorov's theorem to find  $F_{*}\subset F$ such that $|F\smallsetminus F_{*}|<\delta$ and $f_{k}$ converges  uniformly to $f$ on $F_{*}$. That is, there exist $k_{0}$ such that $k\geq k_{0}$ yields
	% 		\begin{equation*}
	% 			|f_{k}(x)-f(x)|<\varepsilon ,
	% 		\end{equation*}
	% 		for every $x\in F_{*}$. Then $\rho_{p(\cdot),F_{*}}\left(f_{k}-f\right)\leqslant \varepsilon^{p_{-}}|F_{*}|$. Observe that we can write $\Omega$ as the disjoint union $\Omega=F_{*}\cup (F\smallsetminus F_{*}) \cup (\Omega\smallsetminus F)$. Then we have
	% 		\begin{align*}
	% 			&\abs{\int_{\Omega}f_{k}(x)g(x)\dif{x}-\int_{\Omega}f(x)g(x)\dif{x}}\leqslant \int_{\Omega}|f_{k}(x)-f(x)||g(x)|\dif{x}\\
	% 			&\hspace{2cm}=\int_{F_{*}}|f_{k}(x)-f(x)||g(x)|\dif{x}+ \int_{F\smallsetminus F_{*}}|f_{k}(x)-f(x)||g(x)|\dif{x} \\
	% 			&\hspace{6.5cm}+\int_{\Omega\smallsetminus F}|f_{k}(x)-f(x)||g(x)|\dif{x}.
	% 		\end{align*}
	% 		By Hölder's inequality, Lemma \ref{lemma:modular-lp-norm-relation}, and the estimates for the modular in $F_{*}$  we arrive to                                   
	% 		\begin{align*}
	% 			\int_{F_{*}}|f_{k}(x)-f(x)||g(x)|\dif{x} &\leqslant  C\|f_k-f\|_{p(\cdot),F_{*}}\|g\|_{L^{p^{\prime}(\cdot)}(F_{*})}\\
	% 			& \leqslant C\left(\rho_{p(\cdot),F_{*}}(f_k-f)\right)^{1/p_{*}}\|g\|_{L^{p^{\prime}(\cdot)}(F_{*})}\\
	% 			&\leqslant C(\varepsilon^{p_{-}}|F_{*}|)^{1/p_{*}}=C(\varepsilon,p).
	% 		\end{align*}                           		    
	% 		Similarly, the estimate on $F\smallsetminus F_{*}$ follows from Hölder's inequality, \eqref{eq:Weak_convergen1}, and the absolute continuity of the integral,
	% 		\begin{align*}
	% 			\int_{F\smallsetminus F_{*}}|f_{k}(x)-f(x)||g(x)|\dif{x}& \leqslant  C\|f_k-f\|_{L^{p(\cdot)}(F\smallsetminus F_{*})}\|g\|_{L^{p^{\prime}(\cdot)}(F\smallsetminus F_{*})}\\
	% 			&\leqslant C \left(\rho_{p(\cdot),F\smallsetminus F_{*}}\left(f_k-f\right)\right)^{1/p_{*}}\left(\rho_{p^{\prime}(\cdot),F\smallsetminus F_{*}}\left(g\right)\right)^{1/p_{*}^{\prime}}\\
	% 			&\leqslant C(2^{p_{+}+1})^{1/p_{*}}\varepsilon^{1/p_{*}^{\prime}}= C(\varepsilon,p).
	% 		\end{align*}  
	% 		Finally, we estimate the integral over $\Omega\smallsetminus F$ as follows
	% 		\begin{align*}
	% 			\int_{\Omega\smallsetminus F}|f_{k}(x)-f(x)||g(x)|\dif{x}&\leqslant C\|f_k-f\|_{L^{p(\cdot)}(F^{c})}\|g\|_{L^{p^{\prime}(\cdot)}(F^{c})}\\
	% 			&\leqslant C \left(\rho_{p(\cdot), F^{c}}\left(f_k-f\right)\right)^{1/p_{*}}\left(\rho_{p^{\prime}(\cdot), F^{c}}\left(g\right)\right)^{1/p_{*}^{\prime}}\\
	% 			&\leqslant C(2^{p_{+}+1})^{1/p_{*}}\varepsilon^{1/p_{*}^{\prime}}= C(\varepsilon,p).           
	% 		\end{align*}     
	% 		Thus,  for $k\geq k_{0}$, we have 
	% 		\begin{equation*}
	% 			\abs{\int_{\Omega}f_{k}(x)g(x)\dif{x}-\int_{\Omega}f(x)g(x)\dif{x}}\leqslant C(\varepsilon,p),
	% 		\end{equation*}
	% 		with $C(\varepsilon,p) \to 0$ as $\varepsilon \to 0$, which in turn implies that $f_{k}$ converges weakly to $f$ in $L^{p(\cdot)}(\Omega)$.                        
	% 	\end{proof}
	% \end{lemma}
	
	% \begin{corollary}\label{cor:weak-convergence_Lp}
	% 	Given $1<p_{-}\leqslant p_{+}<\infty$, let $\{f_{k}\}$ be a bounded sequence of functions in $L^{p(\cdot)}(\Omega)$ converging a.e.  to some  $f \in L^{p(\cdot)}(\Omega)$. Then $f \in L^{p(\cdot)}(\Omega)$.
	% \end{corollary}
	% \begin{proof}
	% 	Take $g\in L^{p^{\prime}(\cdot)}(\Omega)$ with $\|g\|_{p^{\prime}(\cdot),\Omega}\leqslant 1$. By Lemma  \ref{lemma:weak-bounded_Lp} and  Lemma \ref{lemma:holder_ineq} we arrive to
	% 	\begin{align*}
	% 		\int_{\Omega}f(x)g(x)\dif{x}&=\int_{\Omega}\left[f(x)-f_{k}(x)+f_{k}(x)\right]g(x)\dif{x}\\
	% 		&\leqslant \abs{\int_{\Omega}(f(x)g(x)-f_{k}(x)g(x))\dif{x}}+\abs{\int_{\Omega}f_{k}(x)g(x)\dif{x}}\\
	% 		&\leqslant  1 + \|f_k\|_{L^{p(\cdot)}(\Omega)}\|g\|_{L^{p^{\prime}(\cdot)}(\Omega)}\leqslant 1+C\|g\|_{L^{p^{\prime}(\cdot)}(\Omega)},
	% 	\end{align*}
	% 	for $k$ large enough. Thus
	% 	\begin{equation*}
	% 		\|f\|_{L^{p(\cdot)}(\Omega)}=\sup_{\|g\|_{L^{p^{\prime}(\cdot)}(\Omega)}\leqslant 1} \int_{\Omega}f(x)g(x)\dif{x} < \infty.     \qedhere
	% 	\end{equation*}
	% \end{proof}
	
%We use the variable Lebesgue spaces to define the corresponding variable Sobolev space.
        Given a function $f\in
        L^{p(\cdot)}(\Omega)$, it belongs to the Sobolev space
        $W^{1,p(\cdot)}(\Omega)$ if its weak derivatives $D_{j}f$,
        $1\leqslant j\leqslant n$, exist and belong to
        $L^{p(\cdot)}(\Omega)$.  $W^{1,p(\cdot)}(\Omega)$  is a Banach space when
        equipped with the norm
        $\|f\|_{W^{1,p(\cdot)}(\Omega)}=\|f\|_{L^{p(\cdot)}(\Omega)}+\|\nabla
        f\|_{L^{p(\cdot)}(\Omega)}$ (see~\cite[Theorem~6.6]{cruz-fiorenza-book}).
        Though $C^{\infty}(\Omega)\cap
        W^{1,p(\cdot)}(\Omega)$ need not be dense in
        $W^{1,p(\cdot)}(\Omega)$ for general functions $\pp\in
        \mathcal{P}(\Omega)$, it is  when $\pp\in LH(\Omega)$, cf.~\cite[Theorem~6.14]{cruz-fiorenza-book}).

       
	\subsection{Weighted Riesz bounded variation spaces}
%In this section we defined the weighted generalization of the Riesz spaces of bounded $p$-variation.
        For  $f: \Omega \to \subR$,  its oscillation on a set  $E\subset \Omega$ is defined as
	\[
	\osc_{E}(f):= \sup_{x,y\in E}|f(x)-f(y)|.    
	\]
	Given a weight $w$ and $1\leq p<\infty$,  define the weighted
        Riesz $p$-variation of $f$ on the set $\Omega$ by
	\begin{equation}\label{def:Vpqspaces}
          V_{p}\left(f;\Omega,w\right)
          \coloneqq \sup \left[\sum_{B_k \in \D}
            \left(\frac{\osc_{B_k}(f)}{r_{k}}\right)^{p}w(B_k)\right]^{1/p},
	\end{equation}
	where the supremum is taken over all the countable collections
        $\D=\{B_k\}_{k=1}^\infty$ of disjoint balls of radius $r_{k}$ contained in
        $\Omega$. If $V_{p}\left(f;\Omega,w\right)<\infty$, we say
        that $f\in RBV^{p}\left(\Omega,w\right)$. 

\begin{remark} When $w=1$, 
        since
        $|B|\approx r^{n}$,  we recover the definition of Riesz variation given in
        ~\cite{Barza-Lind2015},  up to a dimensional constant.
      \end{remark}

       For
      functions $f$ and $g$, and $\alpha \in \R$, by Minkowski's inequality for sequence spaces, we have 
      %
      \[ V_{p}(f+g; \Omega,w) \leq  V_{p}(f; \Omega,w)
        +  V_{p}(g; \Omega,w),
\]
        and
  \[      V_{p}(\alpha f; \Omega,w)
        = |\alpha| V_{p}(f; \Omega,w); \]
      %
      hence         $V_{p}(f;\Omega,w)$ is a seminorm.  It is not a norm, since
        $V_{p}(f;\Omega,w)=0$ for every constant function
        $f$.
        
	
			
			
			\begin{lemma}\label{prop:Embedding-RBVp}
				Given $w$ a weight, assume that $\Omega$ is a bounded set. If $1<p_{1}<p_2<\infty$, then
				\[
				RBV^{p_2}\left(\Omega,w\right) \hookrightarrow RBV^{p_1}\left(\Omega,w\right).
				\]
                              \end{lemma}
                              
			\begin{proof}
                          Fix $f \in RBV^{p_2}\left(\Omega,w\right)$
                          and let $\D=\{B_k\}_{k=1}^\infty$ be a
                          disjont collection of countable balls
                          contained in $\Omega$.  Fix $s>1$ such
                          that $1/p_2+1/s=1/p_1$. Then, by H\"older's
                          inequality,
				\begin{align*}
					\sum_{B_k \in \D
                                  }\left(\frac{\osc_{B_k}(f)}{r_k}\right)^{p_1}\!\!\!w(B_k)
                                  &\leqslant \left(\sum_{B_k \in \D
                                    }{\left(\frac{\osc_{B_k}(f)}{r_k}\right)^{p_2}\!\!\!w(B_k)}\right)^{\frac{p_1}{p_2}}
\!                                    \left(\sum_{B_k \in \D }\!\!w(B_{k})\right)^{\frac{p_1}{s}}\\
					& \leqslant
                                   V_{p_2}\left(f;\Omega,w\right)^{p_{1}}w(\Omega)^{\frac{p_1}{s}}\\
                                  & <\infty.
				\end{align*}
			If we take the supremum over all such
                        collections $\D$, we get the desired embedding.
			\end{proof}

                        % If $w\in A_p$, $0<\delta<(p-r_w)/r_w$ and
                        % $q= p/(1+\delta)$, a straightforward
                        % calculation shows that $r_w<q< p$, thus
                        % $w\in A_{q}$ from Lemma
                        % \ref{lemma:increasing-Ap-Classes}.
		
		\section{Proof of
                  Theorems~\ref{theo:RBVpq-W1p},~\ref{prop:Main-Proposition},
                and~\ref{thm:embedding-n-variation-sobolev}}\label{sec:proof main theorem} 
		
		\begin{proof}[Proof of Theorem \ref{theo:RBVpq-W1p}]
                  Fix $w\in A_p$, $p>nr_w$.  We first prove that if
                  $f \in W^{1,p}(\Omega,w)$, then
                  $f\in RBV^p(\Omega,w)$.  To do so, we prove
                  a weighted version of the well-known \emph{Morrey
                    inequality} (see \cite[Theorem
                  9.12]{BrezisBook}).  Suppose that
                  $f\in C^{\infty}(\Omega)\cap W^{1,p}(\Omega,w)$. 
                  Fix $x_0\in \Omega$ and $R>0$ such that the open ball
                  $B(x_0,2R)$ is contained in
                  $\Omega$.   Fix $y,z\in  B(x_0,R)$  and let  $B_r= B(x,r)
                  \subset B(x_0,2R)$ be any ball containing
                  $y$ and $z$.    Then
			\begin{equation*}
				f(z)-f(y)=\int_{0}^{1}\nabla f(tz-(1-t)y) \cdot (z-y)\dif t.
			\end{equation*}
			Observe that $B(tz+(1-t)y,tr)\subset B(x,r)$
                        and $|z_j-y_j|\leqslant r$.  Since $p>nr_w$,
                        we can fix $\delta>0$ such that
                        $n<1+\delta< \frac{p}{r_w}$; therefore, if
                        $q=\frac{p}{1+\delta}$, then $nr_w<nq < p$.
                        Hence, by Fubini's theorem, a change of
                        variables, and Hölder's inequality applied
                        twice, we have 
			\begin{align*}
                          &|f(z)-\langle f\rangle_{B_{r}}|
                          \leqslant \avgint_{B_{r}}\int_{0}^{1}\sum_{j=1}^{n}\abs{D_{j} f(tz-(1-t)y)}|z_{j}-y_j|\dif t\dif y\\
				&\lesssim
				\sum_{j=1}^{n}\frac{1}{r^{n-1}}\int_{0}^{1}
				\left(\int_{B_r}|D_jf(y)|^{1+\delta}\dif y\right)^{1/(1+\delta)}|B(tx+(1-t)y,tr)|^{\delta/(1+\delta)} \frac{\dif t}{t^{n}}\\
				&\lesssim
				\sum_{j=1}^{n}r^{1-\frac{n}{1+\delta}}\int_{0}^{1}\left(\int_{B_r}\abs{D_{j}f(y)}^{1+\delta}\dif y\right)^{1/(1+\delta)} \frac{\dif t}{t^{n/(1+\delta)}}\\
				&\lesssim r^{1-\frac{nq}{p}}\sum_{j=1}^{n}\left(\int_{B_r}|D_{j}f(y)|^{p}w(y)\dif y\right)^{1/p}\left(\int_{B_r}w^{1/(1-q)}(y)\dif y\right)^{(q-1)/p}.
			\end{align*}
                        Note that in the final estimate we use that
                        $n<1+\delta$ to evaluate the integral.  
			Hence, if we  set $\sigma := w^{1/(1-q)}$, the preceding estimate for $ z,y \in B_r $ leads to
			\begin{multline*}
 |f(z)-f(y)| \leqslant |f(z)-f_{B_r}|+|f(y)-f_{B_r}| \\
 \lesssim  r^{1-n\frac{q}{p}}\sigma(B_r)^{(q-1)/p}\|\nabla
 f\|_{L^{p}(B_r,w)}
 \leq  r^{1-n\frac{q}{p}}\sigma(B_{2R})^{(q-1)/p}\|\nabla f\|_{L^{p}(B_R,w)}.
\end{multline*}
%
Moreover, since $w\in A_q$, then $\sigma(B_R)<\infty$.  This is true for
all such balls $B_r$, so if we fix the ball $B(x,r)$ so that
$r=2|z-y|$, we get that
%
\begin{equation} \label{eq:morrey1}
 |f(z)-f(y)| \leq
  C|x-y|^{1-n\frac{q}{p}}\sigma(B_{2R})^{(q-1)/p}\|\nabla
  f\|_{L^{p}(B_R,w)}. 
\end{equation}
% 
Since $ C^{\infty}(\Omega)\cap W^{1,p}(\Omega,w)$ is dense in
$W^{1,p}(\Omega,w)$ (see ~\cite[Theorem 1]{gol2009weighted}), a
standard argument shows that inequality~\eqref{eq:morrey1} holds for
every $f\in W^{1,p}\left(\Omega,w\right)$ and almost every
$y,\,z\in B(x_0,R)$.  Consequently, arguing as in the proof
of~\cite[Theorem 9.12]{BrezisBook}, we get that $f$ has a continuous
representative (that is to say, by redefining $f$ on a null set, we
have that $f$ is continuous on $B(x_0,R)$).  Since this is true for all
$x_0\in \Omega$, we have that $f$ has a continous representative on $\Omega$.

To estimate the $RBV^p(\Omega,w)$ norm of $f$, fix any collection
$\D=\{B_k\}_{k=1}^\infty$ of disjoint balls contained in $\Omega$.
Since $w\in A_q$,
			\[ 
			w(B_k)\sigma(B_k)^{q-1}\lesssim r_k^{nq}.  
			\]     
                        Therefore, by \eqref{eq:morrey1} we have that
                        %
			\begin{align*}
				\sum_{B_k
                          }{\left(\frac{\osc_{B_k}(f)}{r_k}\right)^{p}w(B_k)}
                          &=\sum_{B_k }{\left(\frac{\osc_{B_k}(f)}{r_k}\right)^{p}w(B_k)\frac{\sigma(B_k)^{q-1}}{\sigma(B_k)^{q-1}}}\\
				&\lesssim \sum_{B_k}\frac{\osc_{B_k}(f)^{p}}{r_k^{p-qn}}\frac{1}{\sigma(B_k)^{q-1}} \\
				& \lesssim \sum_{B_k}\|\nabla f\|_{L^{p}(B_k,w)}^{p} \\
				&\lesssim \|\nabla
                           f\|_{L^{p}(\Omega,w)}^{p}. 
			\end{align*}   
			If we take the supremum over all such
                        collections $\D$, we get that
                        $V_{p}(f;\Omega,w) \lesssim\|\nabla
                        f\|_{L^{p}(\Omega,w)}$. 
                        
			\medskip

                        To prove the converse, suppose $f\in
                        RBV^p(\Omega,w)$.  
                       We will adapt the construction in
                        \cite[Theorem~1.1]{Barza-Lind2015}.  Fix an open set
                        $\Omega_{0}\Subset\Omega$ and fix
                        \[
			0< R<\frac{d\left(\Omega_{0},\Omega^c\right)}{6\sqrt{n}}.
			\]
                        Let $\{\alpha_{k}\}_{k=1}^\infty$ be an
                        enumeration of the lattice
                        $2R\mathbb{Z}^{n}$. Form an index set
                        $K\in \N$ so that the collection of cubes
                        $\D=\{Q_k\}_{k\in K}$ consists of all cubes
                        such that each cube $Q_k$ has center
                        $\alpha_{k}$, side length $2R$, and satisfies
                        $Q_{k}\cap \Omega_{0}\neq \emptyset$.  We
                        claim that
                        %
                        \[ \Omega_ 0 \subset \bigcup_{k\in K}{Q_k}\subset
                          \Omega. \]
			%
                        The first inclusion is immediate.  If the
                        second did not hold, then for some $Q\in \D$
                        we would have that
                        $Q\cap \Omega^{c}\neq \emptyset$. Hence, there
                        exists $x\in Q\cap \Omega_{0}$ and
                        $y\in Q\cap \Omega^{c}$ such that
			\[
			d\left(\Omega_{0},\Omega^{c}\right)\leqslant |x-y|\leqslant 2\sqrt{n}R,
			\]       
			which is a contradiction. Denote by
                        $\B=\{B(\alpha_{k},3\sqrt{n}R)\}_{k\in K}$ the
                        collection of balls with center $\alpha_{k}$
                        and radius $3\sqrt{n}R$, and observe that,
                        arguing as before, we have that 
                        $Q_{k}\subset B(\alpha_{k}, 3\sqrt{n}R) \subset
                        \Omega$.   Since the centers of the balls in
                        $\B$ lie on a lattice and they have uniform
                        radii, the collection $\B$ can be
                        partitioned into  $N$ collections
                        $\{\B_{i}\}_{i=1}^{N}$ of disjoint balls,
                        $\B_i\subseteq \B$, where $N$ depends only on
                        the dimension.  For brevity, let
                        $B_k=B(\alpha_{k}, 3\sqrt{n}R)$. 

 
                        
                        Let $\varphi \in C_c^\infty(B(0,1))$ be
                        non-negative, radially descreasing function such that $\|\varphi\|_1=1$.
                        For $R>0$ define the sequence of mollifiers
                        $\varphi_{R}(x)=\frac{1}{R^{n}}\varphi(x/R)$.
                       For all 
                        $x\in Q_k$, $y\in B(0,R)$ implies $x-y\in
                        B_k$.    Moreover, we have that for $1\leq j
                        \leq n$, $D_j \varphi$ is uniformly bounded
                        and has $\int_{B(0,1)} D_j\varphi(x)\dif x =
                        0$  (the latter is well-known and follows,
                        for instance,
                        from the divergence theorem).    Therefore,
                        for each $j$ we can estimate as follows:
%
			\begin{align*}
				\norm{D_{j}
                          (\varphi_{R}*f)}_{L^{p}(\Omega_{0},w)}^{p} &\leqslant
                                                                      \sum_{k\in
                                                                      K}\int_{Q_{k}}|D_{j}\varphi_{R}*f(x)|^{p}w(x)\dif{x}\\
				& = \sum_{k\in K}\int_{Q_{k}}\abs{\int_{B(0,R)}D_{j}\varphi_{R}(y)(f(x-y)-f(x))\dif{y}}^{p}w(x)\dif{x}\\
				%             & = \sum_{k}\int_{Q_{k}}\abs{\int_{B(0,R)}D_j\varphi_{R}(y)f(x-y)\dif{y}}^{p}w(x)\dif{x}\\
				&\lesssim \sum_{k\in K}\int_{Q_{k}}R^{-(n+1)p}(\osc_{B_{k}}(f))^{p}|B(0,R)|^{p}w(x)\dif{x}\\
				&\lesssim \sum_{k\in K}\left(\frac{\osc_{B_k}(f)}{3\sqrt{n}R}\right)^{p}w(Q_{k}) \\
				& \leqslant \sum_{k\in K}\left(\frac{\osc_{B_k}(f)}{3\sqrt{n}R}\right)^{p}w(B_{k})\\
				& = \sum_{i=1}^{N}\sum_{B_{k}\in
                           \B_i}\left(\frac{\osc_{B_k}(f)}{3\sqrt{n}R}\right)^{p}w(B_{k}) \\
				& \lesssim V_{p}(f;\Omega,w)^{p}, %<\infty,
			\end{align*} 
                        where the last inequality follows since the
                        balls in $\B_i$ are disjoint.

Because $w\in A_p$, the Hardy-Littlewood maximal operator is bounded on
$L^p(\Omega,w)$,  which yields $\|\varphi_R*f\|_{L^p(\Omega_0,w)} \lesssim \|f\|_{L^p(\Omega,w)}$,  since $|\varphi_R*f(x)| \lesssim Mf(x)$.  
                        Thus,
                        $\{\varphi_{R}*f\}$ is a bounded sequence in
                        $W^{1,p}\left(\Omega_{0},w\right)$ 
			which converges pointwise almost everywhere to
                        $f$.  Thus, by Fatou's lemma, we have
                        that
                        %
                        \[ \|\nabla f\|_{L^p(\Omega_0,w)} \lesssim
                          V_{p}(f;\Omega,w). \]
                        %
                        Since the implicit constants do not depend on
                        $\Omega_0$, we can again apply Fatou's lemma
                        to get that the left-hand side 
                        inequality in ~\eqref{eq:Main_Inequality} holds.
                        Since we also have that $f\in
                        L^p(\Omega,w)$, it follows that  $f\in W^{1,p}\left(\Omega,w\right)$.  
 \end{proof}
                      
		\begin{proof}[Proof of Corollary~\ref{cor:RBV-embed}]
                  For this result it suffices to note that in the proof of the left-hand
                  side inequality in \eqref{eq:Main_Inequality} we did
                  not use the $A_{p}$ condition and this estimate
                  holds for any weight $w$ and for all $p$.  
                \end{proof}
                
		
		\begin{proof}[Proof of Theorem \ref{prop:Main-Proposition}]
                  Fix $t>0$.  By the definition of $L_{f}$, for any $x$ for which %such that
                  $L_{f}(x)>t$ there exists $\Delta_{x}\in \subRn$, $|\Delta_x|\leqslant 1$, such that
			\begin{equation}\label{eq:weak-Inequality}
				|f(x+\Delta_x)-f(x)| \geqslant t|\Delta_x|.
			\end{equation}    
                        Since the balls $\{ B(x,|\Delta_x|)\}$ cover the
                        set $\{ x : L_f(x)>t\}$ and have uniformly
                        bounded radii, by the Vitali
                        covering lemma there exists a subcollection
                        of disjoint balls $\{B_{k}\}_{k=1}^\infty$,
                        $B_{k}:=B(x_{k},|\Delta_{x_{k}}|)$, such that
			\[
			\{x\in \subRn : L_{f}(x)>t\}\subset \bigcup_{k} 5B_{k}.
			\]  
			By Lemma \ref{lemma: GeneralDoublingCon}
                        (applied to balls instead of cubes) we have
			\[
			\left(\frac{|B_{k}|}{|5B_{k}|}\right)^{p}\lesssim \frac{w(B_{k})}{w(5B_{k})}.
			\]       
			Hence,
                        %
			\begin{equation}\label{eq:weak-Inequality2}
w(\{x\in \subRn :  L_{f}(x)>t\}) \leq
w\bigg(\bigcup_{k}5B_{k}\bigg) \leq \sum_{k}w(5B_{k}) \lesssim \sum_{k}w(B_{k}).
\end{equation}
%
			Set $r_{k}=|\Delta_{x_k}|$. If we combine
                        \eqref{eq:weak-Inequality} and
                        \eqref{eq:weak-Inequality2} we get that 
                        %
			\begin{multline*}
                          t^{p}w(\{x\in\subRn : L_{f}(x)>t\})
                          \lesssim t^p\sum_{k}w(B_{k}) \lesssim 
                          \\ \lesssim
                          \sum_{k}\left(\frac{\osc_{B_k}(f)}{r_{k}}\right)^{p}w(B_{k})
                          \leqslant V_{p}(f;\Omega,w)^p,
			\end{multline*}
		which is the desired inequality.	
              \end{proof}

 \begin{proof}[Proof of Corollary~\ref{cor:doubling}]
 To see that this result is true, it suffices to note that in the proof of
 Theorem~\ref{prop:Main-Proposition}, we only use the $A_p$ condition
 to invoke Lemma~\ref{lemma: GeneralDoublingCon}, which in turn we
 only use to show that for each $k$, $w(5B_k)$ is uniformly bounded by $w(B_k)$. 
 Thus, it is enough to assume doubling.
\end{proof}

\begin{proof}[Proof of  Corollary~\ref{cor:Almost-Differentiability-W1p}]
Fix $f\in RBV^p(\Omega,w)$; then   by
Proposition~\ref{prop:Main-Proposition},  $|\{x\in \subRn : L_{f}(x)=\infty\}|=0$.
Given any bounded set $E$,
%
\[ |E| = \int_E w(x)^{1/p}w(x)^{-1/p} \dif x
  \leq \bigg(\int_E w(x)\dif x\bigg)^{1/p}
  \bigg(\int_E w(x)^{1-p'}\dif x\bigg)^{1/p'}; \]
%
since $w\in A_p$ the second integral is finite.  Therefore, given any
set $E$,  if
$w(E)=0$, it follows by a standard approximation argument that
$|E|=0$.   In particular, we have that $|\{x\in \subRn :
L_{f}(x)=\infty\}|=0$, and so by 
Stepanov's theorem \cite[Theorem 3.1.9]{Federer1969}, $f$ is differentiable almost everywhere.

If $p>nr_w$ and $f\in W^{1,p}(\Omega,w)$, then by Theorem
\ref{theo:RBVpq-W1p}, $f\in RBV^p(\Omega,w)$ and so again is
differentiable almost everywhere. 
\end{proof}

\begin{remark}
  By Corollary~\ref{cor:doubling}, if $w$ is doubling and mutually
  absolutely continuous with respect to Lebesgue measure, then $f\in
  RBV^p(\Omega,w)$ is again differentiable almost everywhere. 
\end{remark}

\begin{proof}[Proof of
  Theorem~\ref{thm:embedding-n-variation-sobolev}]
Given a weight $w$ and $f\in RBV^n(\Omega,w)\cap L^n(\Omega,w)$,  by
Corollary~\ref{cor:RBV-embed}, we have that  $f\in W^{1,n}(\Omega,w)$ and
the gradient estimate holds.  If $w\in A_n$, then by
Theorem~\ref{prop:Main-Proposition}, $f$ is differentiable almost
everywhere.
\end{proof}

		% \begin{remark}
		% 	An alternative proof for the Corollary \ref{cor:Almost-Differentiability-W1p} can be given adapting the classical proof of Calderón of the unweighted case  (see \cite[Theorem 6.17]{Heinonen2001}),  using the Lebesgue differentiation theorem for $w \dif x$ (see for example \cite[Theorem 8.4.6]{Benedetto2009}) and  the weighted estimate \eqref{eq:morrey1}. See also \cite[Corollary 1.6]{Bjorn2000} for the validity of Corollary \ref{cor:Almost-Differentiability-W1p} in the case of doubling measures.
		% \end{remark}
		%\begin{proof}[Proof of Corollary \ref{cor:Almost-Differentiability-W1p}]
		%	Choose a point $x_0$ such that 
		%	    \begin{equation}
			%	    \lim_{r\rightarrow 0}\frac{1}{w(B(x_0;r))}\int_{B(x_0;r)}|\nabla f(x_0)-\nabla f(x)|^{p}w\dif x=0,
			%    \end{equation}
		%and set $g(x)=f(x)-f(x_{0})-\nabla f(x_0)(x-x_0)$, for $x\in B(x_0;r)$. So, $\nabla g(x)= \nabla f(x)-\nabla f(x_0)$ a.e in B. Hence, applying \eqref{eq:Morrey-Ineq} to $g$, with $r=|y-x_0|$, and the $A_q$ condition we obtain
		%   	    \begin{align*}
			%  	    |f(y)-f(x_{0})-\nabla f(x_0)(y-x_0)|&= |g(y)-g(x_0)|\\
			%& \lesssim r\left(\frac{1}{w(B(x_0;r))}\int_{B(x_0;r)}|\nabla f(x_0)-\nabla f(x)|^{p}w\dif x\right)^{1/p},
			%\end{align*}
			%So, $y\rightarrow x_0$ implies that 
			%\[
			% \frac{|f(y)-f(x_{0})-\nabla f(x_0)(y-x_0)|}{|y-x_0|}\rightarrow 0,
			%\] 
			%and invoking the Lebesgue differentiation Theorem for the measure $wdx$ (see for example \cite[Theorem 8.4.6]{Benedetto2009} ) we conclude that $f$ is differentiable almost everywhere.                        	    
			%\end{proof}
			% \section{Weighted Bounded $n$-Variation}\label{sec:Weighted-n-variation} 
		
	
                \section{Proof of
                  Theorem~\ref{theo:Riesz_Variable-Exponent}
                }\label{sec:Variable-expoent-section}
			
                In this section, we prove
                Theorem~\ref{theo:Riesz_Variable-Exponent}, which
                  characterizes the  variable exponent
                Sobolev space $W^{1,p(\cdot)}(\Omega)$ in terms of
                the  variable exponent Riesz bounded variation spaces
                $RBV^{p(\cdot)}(\Omega)$.   We first give a formal
                definition of these spaces.  Given a  countable
                collection of disjoint balls $\D=\{B_k\}_{k=1}^\infty$ of radius $r_{k}$ contained in
                $\Omega$, define
                %
			\begin{equation*}
                          V_{\D}^{p(\cdot)}(f;\Omega)
                          =\sum_{k}\left(\frac{\osc_{B_{k}}(f)}{r_k}\right)^{p_{B_k}}
                          \|\chi_{B_{k}}\|_{p(\cdot)}^{p_{B_k}}.  
                              \end{equation*}
                              %
Use the Luxemburg norm to define the functional 
			\begin{equation} \label{eq:def_RBV_variable}
                          \|f\|_{RBV_{\D}^{p(\cdot)}(\Omega)}
                          = \inf\left \{ \lambda>0 :  V_{\D}^{p(\cdot)}(f/\lambda;\Omega)\leqslant 1 \right \}.
                        \end{equation}
                        A function $f$ belongs to $RBV^{p(\cdot)}(\Omega)$ if 
			\[
			\|f\|_{RBV^{p(\cdot)}(\Omega)}=\sup_\D\|f\|_{RBV_{\D}^{p(\cdot)}(\Omega)}< \infty,
                      \]
                      %
                      where the supremum is taken over all the
                      countable collections $\D$.   While not central
                      to our results, we have that
                      $\|\cdot\|_{RBV^{p(\cdot)}(\Omega)}$ is a
                      seminorm.  This can be proved using the standard
                      arguments used with the Luxemburg construction
                      to prove that the norms in Orlicz spaces and
                      variable Lebesgue spaces have the requisite
                      properties, 
                      see~\cite{cruz-fiorenza-book,MR1113700}.  
The fact that 
                      $\|\cdot\|_{RBV^{p(\cdot)}(\Omega)}$ is a
                      only  a seminorm is immediate, since 
                      $\|f\|_{RBV^{p(\cdot)}(\Omega)}=0$ whenever 
                      $f$ is a constant function. Finally, note that when
                      $\pp=p$ is constant, $1\leq p<\infty$, 
                      $ \|f\|_{RBV^{p(\cdot)}(\Omega)}=
                      V_{p}(f;\Omega)$, so the spaces
                      $RBV^\pp(\Omega)$ and $RBV^p(\Omega)$ are
                      equal.

                      To prove our main results in this section, we
                      first prove an alternative way to estimate the
                      seminorm in $RBV^\pp(\Omega)$.  Given an exponent
                      function $\pp$, a disjoint
                      collection $\D=\{B_k\}_{k=1}^\infty$ of balls
                      contained in $\Omega$, and a sequence
                      of real numbers $\{t_{B_{k}}\}_{B_{k}\in \D}$
                      associated to $\D$,  define the variable
                      exponent sequence space
			\begin{equation*}
                          \ell^{\D,\pp}
                          = \bigg \{\hat{t}=\{t_{B_{k}}\}_{B_{k}\in
                              \D} :
                            \sum_{B_k \in \D }|t_{B_k}|^{p_{B_{k}}}<\infty    \bigg \},
			\end{equation*}  
			and endow it with the Luxemburg norm    
			\[
                          \|\hat{t}\|_{\ell^{\D,\pp}}
                          =\inf \bigg \{ \lambda>0 :
                            \sum_{B_k \in \D }(|t_{B_k}|/\lambda)^{p_{B_{k}}}\leqslant 1 \bigg \},
                          \]
                          for more on variable sequence spaces,
                          see~\cite{diening-harjulehto-hasto-ruzicka2010}.  
                          The following result shows the connection
                          between these sequence spaces and the
                          variable Lebesgue space $L^\pp(\Omega)$
                          (for a proof,
                          see~\cite[Lemma~2.1]{kopaliani2008greediness}).
                          
			\begin{lemma}\label{theo:sequence_Lp_norm_ineq}
                          Let $\pp \in LH(\Omega)$ and
                          $\{e_{B_{k}}\}_{B_{k}\in \D}$ be the
                          canonical base for $\ell^{\D,\pp}$ (i.e., $e_{B_k}$
                          has entry 1 at the index $k$, and $0$
                          otherwise). Then for every disjoint
                          countable collection $\D$ of balls and every
                          real sequence $\{t_{B_k}\}$,
                          %
				\begin{equation}
				\bigg\|\sum_{B_k \in \D
                                          }t_{B_{k}}\chi_{B_{k}}\bigg\|_{p(\cdot)}
                                        \approx \bigg\|\sum_{B_k \in \D
                                          }t_{B_k}
                                          \|\chi_{B_k}\|_{p(\cdot)}e_{B_k}\bigg\|_{\ell^{\D,p}}.
				\end{equation}
                                %
                                The implicit constants depend only on $\pp$.
                              \end{lemma}
                              
                              Given a disjoint countable collection of
                              balls $\D=\{B_{k}\}_{k=1}^\infty$, define
                              the operator $G_{\D}$ by
                              %
			\begin{equation*}
				G_{\D} f(x)=\sum_{B_k\in \D}\frac{\osc_{B_k}(f)}{r_k}\chi_{B_{k}}(x). 
                              \end{equation*}
                              %
                              A key fact is that the 
                              $RBV_{\D}^{p(\cdot)}$ norm of $f$ and the
                              $L^{p(\cdot)}$ norm of the operator  $G_{\D} f$ are
                              comparable.
                              
			\begin{proposition}\label{cor:AveragingOp_RBV_equivalence}
				Given $\pp \in LH(\Omega)$, for every function
                                $f$ we have 
                                %
				\begin{equation}\label{eq:AveragingOp_RBVpnorm_equivalence}
                                  \|f\|_{RBV^{p(\cdot)}(\Omega)}
                                  \approx \sup_{\D}\|G_{\D}f\|_{L^{p(\cdot)}(\Omega)},
				\end{equation}
                                %
                                where the supremum is taken over every countable
                                collection $\D$ of disjoint balls in
                                $\Omega$.  
                                The implicit constants depend only on $\pp$.
                              \end{proposition}

                              
                              \begin{proof}
Fix a collection $\D$. Then by Lemma~\ref{theo:sequence_Lp_norm_ineq},
the definition of $\ell^{\D,\pp}$, and the definition of $RBV_\D^\pp$,
\eqref{eq:def_RBV_variable}, we have that
%
\begin{multline*}
  \|G_\D f\|_\pp
  = \bigg\| \sum_{B_k \in \D} \frac{\osc_{B_k}(f)}{r_k} \chi_{B_k}
  \bigg\|_\pp \approx \\
  \approx
  \bigg\| \sum_{B_k \in \D} \frac{\osc_{B_k}(f)}{r_k}
  \|\chi_{B_k}\|_\pp
  \bigg\|_{\ell^{\D,\pp}}
  = \|f\|_{RBV_\D^\pp(\Omega)}.
  \end{multline*}
  %
  If we take the supremum over every collection $\D$, we get
  \eqref{eq:AveragingOp_RBVpnorm_equivalence}.
 \end{proof}     
		
Finally, for our proof we need a version of Rubio de Francia
extrapolation into the scale of variable Lebesgue spaces.  More
precisely, we need a version of limited range extrapolation that was
proved in~\cite[Theorem 2.14]{Cruz-Wang_2017}.  This result is stated
in terms of an abstract family of extrapolation pairs $\F = \{
(f,g)\}$. 

 \begin{theorem}\label{theo:interpo_result}
  Given a family of extrapolation pairs $\F$,  let $1<q_{-}<q_+<\infty$ and assume that there exists ${p}$,
   $q_{-}<{p}<q_+,$ such that for every
   ${w}\in A_{{p}/q_-}\cap RH_{(q_+/{p})^{\prime}}$,
   \begin{equation}\label{eq:Extrapol_result2}
 \|f\|_{L^{{p}}({w})}\leqslant C\|g\|_{L^{{p}}({w})}
 \end{equation}
 %
 for every pair $(f,g)\in \mathcal{F}$ such
 that $\|f\|_{L^{{p}}({w})}<\infty$. Given
 $p(\cdot)\in LH(\Omega)$, suppose $q_{-}<p_{-}\leqslant p_{+}<q_{+}$. Then for
 $(f,g)\in \mathcal{F}$ such that $\|f\|_{p(\cdot)}<\infty$,
 %
 \begin{equation}\label{eq:Extrapol_result}
 \|f\|_{p(\cdot)}\leqslant C\|g\|_{p(\cdot)}.
 \end{equation}  
 %
\end{theorem}   
			
\begin{remark}
  We want to emphasize that the family of extrapolation pairs $\F$
  must be chosen so that the left-hand side of
  ~\eqref{eq:Extrapol_result2} and~\eqref{eq:Extrapol_result} are
  finite.  Given this assumption, to prove a desired inequality in general often requires an
  approximation argument; this is the case in our proof below.  On the
  other hand, we do not need to assume that the right-hand side of
  either of these inequalities is finite.
 For a complete discussion of this approach to
extrapolation, see~\cite{Cruz-Martell2011,Cruz-Wang_2017}.  
\end{remark}

\begin{proof}[Proof of Theorem \ref{theo:Riesz_Variable-Exponent}]
First assume that $f\in W^{1,\pp}(\Omega)$.  Let $B\subset \Omega$ be
any ball.  Then by the embedding of variable Lebesgue
spaces on compact domains~\cite[Corollary~2.48]{cruz-fiorenza-book},
%
\[ W^{1,\pp}(\Omega) \subset W^{1,\pp}(B) \subset W^{1,p_-}(B).  \]
%
Since $p_->n$, by~\cite[Theorem~1.1]{Barza-Lind2015} the function $f$ has a
continuous representative on $B$.  Since this is true for all balls
contained in $\Omega$, $f$ has a continuous representative on
$\Omega$. 

We will now prove that the right-hand side inequality in~\eqref{eq:
  var-exponent_Riesz} holds:  i.e., that $\|f\|_{RBV^\pp(\Omega)}
  \lesssim \|\nabla f\|_{L^\pp(\Omega)}$.  Fix a
collection $\D=\{B_{k}\}_{k=1}^\infty$ of disjoint balls contained in
$\Omega$. Since the balls are disjoint, given any weight $w$, we obtain
 %
\begin{equation} \label{eqn:G-norm}
  \begin{split}
\|G_{\D}f\|_{L^{p}(w,\Omega)}^{p}
   &=\int_{\Omega}(G_{\D}f(x))^{p}w(x)\dif{x} \\
   &= \sum_{B_k\in\D}\left(\frac{\osc_{B_k}(f)}{r_k}\right)^{p}w(B_k)\\
    &= V_p(f;\Omega,w)^p.
    \end{split}
 \end{equation}
 % 
 To apply extrapolation we need that the left-hand side
 of~\eqref{eq:Extrapol_result2} and~\eqref{eq:Extrapol_result} are
 finite.  To ensure this, we define a truncation of the operator
 $G_\D$.  For each $N\geq 1$, define
 $\D_{N}=\{B_{k} : B_{k}\in \D, 1\leqslant k \leqslant N\}$ and the
 truncated operator $G_{D_{N}}f$.  Define the family of extrapolation pairs
 %
 \begin{equation*}
   \mathcal{F}=
   \left\{ (G_{\D_N}f ,\nabla f) :   f\in  C^{\infty}(\Omega)\cap
     W^{1,p(\cdot)}(\Omega),\,
     N\geqslant 1 \right\}.
 \end{equation*}
 %
Fix $\pp\in LH(\Omega)$ such that $n<p_- \leq p_+<\infty$.  Let
$q_-=n$, and fix $q_+>p_+$.  Let $w$ be any weight in $A_{p_-/n}\cap
RH_{(q_+/p_-)'}$  (note that this class is not empty: if
we take $w\in A_{p_-/n}$, by Lemma~\ref{lemma:Ainfty_RHS} we have
$w\in RH_{(q_+/p_-)'}$ for all $q_+$ sufficiently large).  Since $w\in
A_{p_-/n}$, we have that $p_->nr_w$ and $w\in A_{p_-}$.  Fix $f\in C^{\infty}(\Omega)\cap
     W^{1,p(\cdot)}(\Omega)$.  For all $N$,
$G_{\D_N}f$ is compactly supported and in $L^\infty$ with $\|G_{\D_N}\|_{L^{p_-}(w)}
<\infty$, since $w$ is locally integrable.  Thus, by
Theorem~\ref{theo:RBVpq-W1p} and~\eqref{eqn:G-norm}
%
\[ \|G_{\D_N}f\|_{L^{p_{-}}(\Omega,w)}
  \leqslant C \|\nabla f\|_{L^{p_{-}}(\Omega,w)}. \]
%
Similarly, we have that
%
\[ \|G_{\D_N}f\|_\pp \leq \|G_{\D_N}f\|_\infty
  \|\supp(G_{\D_N}f)\|_\pp < \infty.  \]
%
Therefore, by Theorem~\ref{theo:interpo_result} applied to the family
$\F$, for every $N\geq 1$ and $f\in C^{\infty}(\Omega)\cap
     W^{1,p(\cdot)}(\Omega)$,
%
\begin{equation*}
  \|G_{\D_N}f\|_{L^{p(\cdot)}(\Omega)}
  \leqslant C \|\nabla f\|_{L^{p(\cdot)}(\Omega)}.
\end{equation*}
%
If we apply  Fatou's lemma in the variable Lebegue spaces (see
\cite[Theorem 2.61]{cruz-fiorenza-book}) we have  that for all $f\in
C^{\infty}(\Omega)\cap W^{1,p(\cdot)}(\Omega)$,
%
 \begin{equation} \label{eqn:dense-set}
   \| G_{\D}f\|_{L^{p(\cdot)}(\Omega)}
   \leqslant C \|\nabla f\|_{L^{p(\cdot)}(\Omega)}.
 \end{equation}

 To complete the proof, fix $f\in W^{1,p(\cdot)}(\Omega)$.  Recall
 that $C^{\infty}(\Omega)\cap W^{1,p(\cdot)}(\Omega)$ is dense in
 $W^{1,p(\cdot)}(\Omega)$ (see \cite[Theorem
 6.14]{cruz-fiorenza-book}).  Fix  a sequence $\{f_{k}\}_{k=1}^\infty$
 of functions in $C^{\infty}(\Omega)\cap W^{1,p(\cdot)}(\Omega)$ that converges
 to $f$ in the $W^{1,p(\cdot)}$ norm. Then we can find a subsequence (still denoted by $f_k$)  % If we pass to a subsequence, we may also assume that the sequence
 which converges pointwise to $f$
 (see \cite[Theorem 2.67]{cruz-fiorenza-book}).  Therefore, by Fatou's
 lemma in the variable Lebesgue spaces and by
 inequality~\eqref{eqn:dense-set},
 %
 \begin{equation*}
   \|G_{\D}f\|_{L^{p(\cdot)}(\Omega)}
   \leqslant  \liminf_{k\rightarrow \infty}\|G_{\D}f_k\|_{L^{p(\cdot)}(\Omega)}
   \lesssim \lim_{k\rightarrow\infty} \|\nabla f_k\|_{L^{p(\cdot)}(\Omega)}
   = \|\nabla f\|_{L^{p(\cdot)}(\Omega)}.
 \end{equation*}
%
 The desired inequality now follows from Proposition~\ref{cor:AveragingOp_RBV_equivalence}.             

\medskip			

We will now prove the converse.  Fix $f\in RBV^\pp(\Omega) \cap
L^\pp(\Omega)$.  We will prove that  $f\in W^{1,\pp}(\Omega)$ and the left-hand side inequality in ~\eqref{eq:
  var-exponent_Riesz} holds:  i.e., that $\|\nabla f\|_{L^\pp(\Omega)}
  \lesssim \|f\|_{RBV^\pp(\Omega)}$. 

  Let $\Omega_0\Subset  \Omega$; then we have the embedding
  %
 \begin{equation*}
   RBV^{p(\cdot)}(\Omega)\subset  RBV^{p(\cdot)}\left(\Omega_0\right)
   \subset RBV^{p_{-}}\left(\Omega_0\right).
 \end{equation*}
 %
 The first inclusion is immediate; the second one does not follow
 directly from the definition of $RBV^\pp$, but is a consequence of
 Corollary~\ref{cor:AveragingOp_RBV_equivalence} and the embedding of
 variable Lebesgue spaces on compact
 domains~\cite[Corollary~2.48]{cruz-fiorenza-book}.  Since $p_->n$,
 again by  \cite[Theorem
 1.1]{Barza-Lind2015} we have that
 $RBV^{p(\cdot)}(\Omega)\subset
 W^{1,p_{-}}\left(\Omega_0\right)$. Thus, weak derivatives are well defined in
 $RBV^{p(\cdot)}(\Omega)$.

 We now define  the family of extrapolation pairs
 \[
   \mathcal{F}=\big \{ (\langle\nabla f\rangle_{N}, G_{\D}f ) : f\in
     RBV^{p(\cdot)}(\Omega),\ N\geqslant 1, \D \big\},
 \] 
 where $\langle\nabla f\rangle_{N}=\min\{N,|\nabla f|\}$ and
 $\mathcal{D}$ is a countable family of disjoint
 balls contained in $\Omega$ that depends on $f$ and $N$; the exact
 choice of $\D$ will be made below.  Because $w$ is locally integrable, 
 $\| \langle\nabla f\rangle_{N} \|_{L^{p_-}(\Omega,w)}<\infty$;
 similarly, since $p_+<\infty$, $\| \langle\nabla f\rangle_{N}
 \|_{\Lp(\Omega)}<\infty$. 
We now argue as we did in the proof above, fixing $q_-,\,q_+$, and 
$w\in A_{\frac{p_{-}}{n}}\cap RH_{(\frac{q_{+}}{p_{-}})^{\prime}}$.
If $f\in RBV^{p_-}(\Omega,w)$, then by Theorem~\ref{theo:RBVpq-W1p}
and inequality~\eqref{eqn:G-norm}
%
\begin{equation} \label{eqn:saturate}
\| \langle\nabla f\rangle_{N} \|_{L^{p_-}(\Omega,w)}
  \leq \| \nabla f \|_{L^{p_-}(\Omega,w)}
  \lesssim V^{p_-}(f,\Omega,w)
  \leq 2\|G_\D f\|_{L^{p_-}(\Omega,w)}, 
\end{equation}
% 
where we choose $\D$ to saturate the supremum used to define
$V^{p_-}$.  On the other hand, if $f\not\in RBV^{p_-}(\Omega,w)$, then
$V^{p_-}(f,\Omega,w)=\infty$, so we can choose $\D$ so that
$\|G_\D f\|_{L^{p_-}(\Omega,w)}$ is either infinite or arbitrarily large; in particular, we can fix $\D$ so
that
$\| \langle\nabla f\rangle_{N} \|_{L^{p_-}(\Omega,w)} \lesssim \|G_\D
f\|_{L^{p_-}(\Omega,w)}$ with the same constant as
in~\eqref{eqn:saturate}.  Since the hypotheses of
Theorem~\ref{theo:interpo_result} are
satisfied for any pair  $(\langle\nabla f\rangle_{N}, G_{\D}f )$ in
$\mathcal{F}$, we conclude that
%
 \begin{equation*}
   \|\langle\nabla f\rangle_{N} \|_{L^{p(\cdot)}(\Omega)}
   \lesssim \| G_{\D} f\|_{L^{p(\cdot)}(\Omega)}
   \lesssim \|f\|_{RBV^\pp(\Omega)};
 \end{equation*}
 %
 the last inequality holds by
 Corollary~\ref{cor:AveragingOp_RBV_equivalence}. 
 If we again apply  Fatou's lemma in the variable Lebesgue spaces, we get
 \[
   \|\nabla f\|_{L^{p(\cdot)}(\Omega)}
   \leqslant \liminf_{N\rightarrow\infty} \|\langle \nabla
   f\rangle_{N}\|_{L^{p(\cdot)}(\Omega)}
   \lesssim  \|f\|_{RBV^{p(\cdot)}(\Omega)},
 \]      
which completes the proof.
\end{proof}

				%definition of $(p,w)-$variation, given a function $f$ and $\varepsilon>0$, there exists a collection of disjoint balls $\D(f)$ such that  
				%\[V_{p,w}(f,\Omega)\leqslant (1+\varepsilon)\|G_{\D(f)}f \|_{L^{p}(w,\Omega)}\leqslant 2 \|G_{\D(f)}f \|_{L^{p}(w,\Omega)}. \]
				%\color{blue}Case2. \color{black} 
				%fix $f\in C^{1}(\Omega)\cap RBV^{p(\cdot)}(\Omega)$, and let 
				%\[
				%\Omega_{N}=\{x\in\Omega:\  \dist(x,\partial \Omega)>1/N\}\cap B(0,N).
				%\]
				%We set the family 
				%\[
				%\mathcal{F}_2=\{ (\nabla f,\langle G_{\D(f)}f\rangle ):\ f\in  C_{0}^{1}(\Omega_{N})\cap RBV^{p(\cdot)}(\Omega_{N}),\ N\geq1 \},
				%\]  
				%Further, by Theorem \ref{theo:RBVpq-W1p} we have that  for $(p,n,w)$ chosen properly,
				%   \begin{equation}
					%\|\nabla f\|_{L^{p}(w)} \lesssim  \| G_{\D(f)}f \|_{L^{p}(w)}.
					% \end{equation}
				%Since $|\Omega_{N}|<\infty$, for every $N\geq 1$, the embedding $RBV^{p(\cdot)}(\Omega_{N}) \subset RBV^{p_{-}}(\Omega_{N}) $ follows from corollary []. Moreover, by virtue of remark [], there exists $\tilde{p}>1$ such that $RBV^{p_{-}}(\Omega_{N}) \subset W^{1,\tilde{p}}(\Omega_{N},w)$, $w \in A_{\tilde{p}}$.
				%and argue as before, taking into account that  $C_{0}^{\infty}(\Omega)\subset W_{0}^{1,p}(\Omega,w)$, the density of $C_{0}^{\infty}(\Omega)$ into $W_{0}^{1,p(\cdot)}(\Omega)$ and \eqref{eq:AveragingOp_RBVpnorm_equivalence} to conclude that 
				%the left inequality of \eqref{eq: var-exponent_Riesz} holds.

\begin{remark}
  Though our goal was to prove
  Theorem~\ref{theo:Riesz_Variable-Exponent} via extrapolation, we
  want to note that we can also prove the second half of this result
  by adapting the proof of Theorem \refeq{theo:RBVpq-W1p} to the
  variable exponent setting. Here we sketch the proof.

  We will follow the same construction and notation from the proof of
  Theorem \refeq{theo:RBVpq-W1p}. First, by homogeneity of the norm
  and by Corollary \refeq{cor:AveragingOp_RBV_equivalence}, without
  loss of generality, we may assume that
  $\|f\|_{RBV^{p(\cdot)}(\Omega)}=c_0$, for some constant that is
  sufficiently small that
  $\|G_{\D} f\|_{L^{p(\cdot)}(\Omega)}\leqslant 1$ for any collection
  of balls $\D$.  Therefore, by~\cite[Corollary
  2.22]{cruz-fiorenza-book},
  $\rho_{p(\cdot),\Omega}\left(G_{ \D}f\right)\leqslant 1$.  Since
  $p_{+}<\infty$ and $\mathcal{B}_{i}$ is a disjoint collection of
  balls, we get the estimate
  \begin{align*}
    \rho_{p(\cdot),\Omega_0} \left( D_{j} (\varphi_{R}*f)\right)
    &\leqslant \sum_{k\in K}\int_{Q_{k}}|D_{j}\varphi_{R}*f(x)|^{p(x)}\dif{x}\\
    & = \sum_{k\in K}\int_{Q_{k}}
      \abs{\int_{B(0,R)}D_{j}\varphi_{R}(y)(f(x-y)-f(x))\dif{y}}^{p(x)}\dif{x}\\
					%             & = \sum_{k}\int_{Q_{k}}\abs{\int_{B(0,R)}D_j\varphi_{R}(y)f(x-y)\dif{y}}^{p}w(x)\dif{x}\\
    &\lesssim \sum_{k\in K}\int_{Q_{k}}
      R^{-(n+1)p(x)}(\osc_{B_{k}}(f))^{p(x)}|B(0,R)|^{p(x)}\dif{x}\\
    &\lesssim \sum_{k\in K}\int_{B_k}
      \left(\frac{\osc_{B_k}(f)}{3\sqrt{n}R}\right)^{p(x)}\dif{x} \\
 & = \sum_{i=1}^{N}\sum_{B_{k}\in \B_i}\int_{B_k}\left(\frac{\osc_{B_k}(f)}{3\sqrt{n}R}\right)^{p(x)}\dif{x}\\
 &=
   \sum_{i=1}^{N}\int_{\Omega}\left(G_{\mathcal{B}_{i}}f\right)^{p(x)}\dif{x}
    \\
    & \leqslant
      \sum_{i=1}^{N}\rho_{p(\cdot),\Omega}\left(G_{\mathcal{B}_{i}}f\right)
    \\
    & \leqslant N.
 \end{align*}
 %
 So,
 $\|D_{j} (\varphi_{R}*f)\|_{L^{p(\cdot)}(\Omega_{0})}$ is uniformly bounded.
 Since $\pp \in LH(\Omega)$, the Hardy-Littlewood maximal operator is
 bounded on $L^\pp(\Omega)$, so
 $\|\varphi_{R}*f\|_{L^{p(\cdot)}(\Omega_{0})}$ is bounded.  
 %
 Hence, the sequence $\{\varphi_{R}*f\}_{R>0}$ is uniformly bounded in
 $W^{1,p(\cdot)}(\Omega_{0})$, and $\{\varphi_{R}*f\}_{R>0}$ converges
 pointwise to $f$.  We can adapt the proof
 of~\cite[Theorem~1.32]{Heinonen1993} to the variable exponent setting
 (again using the fact that $\pp \in LH(\Omega)$ and $p_+<\infty$) to
 conclude that $f\in W^{1,p(\cdot)}(\Omega_{0})$ for every
 $\Omega_{0}\Subset  \Omega$.  Then by Fatou's lemma, we get
 that $f\in W^{1,p(\cdot)}(\Omega)$.
 \end{remark}

 \begin{remark}
   Given the previous remark, it would be interesting to have a direct
   proof of the first half of
   Theorem~\ref{theo:Riesz_Variable-Exponent} that did not depend on
   extrapolation.  We conjecture that such a proof is possible, though
   we have not been able to find it.  A starting point would be the
   variants of the Morrey inquality that hold for variable Sobolev
   spaces, see, for example,~\cite[Theorem 6.36]{cruz-fiorenza-book}.
 \end{remark}
			
			
 \section*{Acknowledgments}
 The first author is partially supported by a Simons Foundation Travel
 Support for Mathematicians Grant. The second and third authors were
 supported by a Research Start-up Grant of United Arab Emirates
 University, UAE, via Grant G00002994.
			
			
			
			\bibliographystyle{plain}
			
			
			\bibliography{Weighted-RBVp_R_n}

                        \end{document}

\section{Introduction}
You can use this file as a template when submitting your paper to one of the IMPAN journals (except Dissertationes Mathematicae and Banach Center Publications, for which style files exist).

The format of this file is \textbf{not} the exact final printed format (for example, the latter is scaled down, and line breaks will most often be different), but it is convenient for editing purposes.


\section{Theorems etc.}
The statements of theorems, propositions etc. are set in
italics. In definitions, only the term being defined is emphasized. Remarks
and examples are set in roman type. 

\begin{definition}
A system $S$ is said to be \emph{self-extensional} if 
$S \in B$.
\end{definition}

\begin{problem}
Is $A+B=C$ true?
\end{problem}

%% Note that in the example below, the braces { } around \cite are necessary (due to nested optional parameters)

\begin{theorem}[Identity Principle, see also {\cite[Theorem 5]{HillDow}}]\label{T:1}
If $A=B$, then the following conditions are equivalent: 

\begin{enumerate}[label=\upshape(\roman*), leftmargin=*, widest=iii]
\item first item,\label{it:1}
\item second item,\label{it:2}
\item third item.\label{it:3}
\end{enumerate}
\end{theorem}

\begin{proof} We only prove \ref{it:1}$\Rightarrow$\ref{it:2}. Observe that
%
\begin{equation}\label{E:1}
\begin{aligned}[t]
\bigl(\tfrac{1}{2}(\obj{3}{b}{1} + \len\obj{a+i}{2}{4})\bigr)^2
&= \binom{a+b}{c-d}\\
&\quad + \Bigl(\prod_{i=1}^n A_i\Bigr)^2 + \biggl(\frac{u}{v}\biggr)^n\\
&\overset{\alpha}= 
\begin{cases}
\sqrt[3]{2/\!\sin x} &\text{if $x \in (0, \pi)$,}\\
0 &\text{otherwise.}
\end{cases}
\end{aligned}
\end{equation}
%
Now apply induction on $n$ to \eqref{E:1}.
\end{proof}

\begin{xrem}
Theorem~\ref{T:1} was independently proved in \cite{Kow}.
\end{xrem}


\begin{mainthm} 
Here comes the statement of a numbered theorem with a fancy name.
\end{mainthm}


Note that formulas in IMPAN journals are left-numbered. 
For many examples of codes of  multiline formulas, see
\[
\texttt{https://www.impan.pl/en/publishing-house/for-authors}.
\]
The \texttt{eqnarray} construction leads to well-known mistakes---if you have learnt it, just forget it.

Do not leave ``overflows'' in formulas; if the formula is too wide, break it yourself into lines or, e.g., shorten it by introducing some symbols. 

Do not re-invent {\LaTeX}; before using your own construction or creating a new symbol
look up Gr\"atzer \cite{Gratzer}---most probably, your intended construction or symbol is already there.

Add small spaces \texttt{\textbackslash,} only exceptionally, e.g. before differentials.



\section{Figures}
If you are including figures created outside {\LaTeX}, they should be prepared as pdf, jpg or eps files.
All figures will be printed black and white; colours will only appear in the online version.
If your original figures are coloured, check their black-and-white printouts;
you may wish to change (some of) the colours, or use shades of grey, to make some distinctions 
more visible.

Avoid very thin lines, and check whether all fonts used are embedded.

Remember that sometimes figures have to be scaled, and then the lettering is scaled too; therefore, very small lettering should be avoided.

% Figure environment removed

\subsection*{Acknowledgements}
Place all thanks and grant acknowledgements here.


%%%%%%%%%%% To ease editing, use normal size for the references:

\normalsize


\begin{thebibliography}{[HD82]}

%% Use the widest label as parameter above.
%% Reference items can be numbered or have labels of your choice, as below.
%% Arrange the items in the alphabetical order of names (and not in the order of labels).

%% In IMPAN journals, only the title is italicized; boldface is not used.
%% Do NOT give the issue number unless the issues are paginated separately, as in Uspekhi below.

%% To ease editing, add:

\normalsize
\baselineskip=17pt


%%%%%%%%%%%%%

\bibitem[G07]{Gratzer} G. Gr\"atzer,
\emph{More Math into \LaTeX},
4th ed., Springer, Berlin, 2007.


\bibitem[HD82]{HillDow}  R. Hill and A. Dow,
\emph{A ground-breaking achievement},
J.~Differential Equations 15 (1982), 197--211.

\bibitem[K74]{Kow}  J. Kowalski,
\emph{A very interesting paper},  
in: Algebra, Analysis and Beyond (Nowhere, 1973),   
E.~Fox et al. (eds.),
Lecture Notes in Math. 867, Springer, Berlin, 1974, 115--124.

\bibitem[N80]{Nov} A. S. Novikov,
\emph{Another fascinating article},  
Uspekhi Mat. Nauk 23 (1980), no.~3, 112--134 (in Russian); 
English transl.: Russian Math. Surveys 23 (1980), 572--595.

\bibitem[R]{Russ} B. Russell,
\emph{A new theorem},
arXiv:0612.9876 (2006).

\end{thebibliography}



\end{document}