\section{Introduction}

Compact binary coalescences (CBCs), astronomical occurrences marked by the merging of two distinct compact objects such as black holes (BHs) or neutron stars (NSs), present unique opportunities to study gravitational waves (GWs) \cite{gw_sources}. Since the advent of the Laser Interferometer Gravitational-wave Observatory (LIGO) and the subsequent detection of the first GW signal in 2015 \cite{abbott_2016}, our understanding of these cosmic phenomena has significantly expanded. The resultant waveforms from CBCs, or transient-modelled waveforms, encapsulate the dynamics of these merging systems and their study enables us to probe the nature of gravity itself. Confirming the predictions of General Relativity in the strong-field regime, such as the inspiral, merger, and ringdown phases of compact object mergers, allows us to test the limits of our current understanding and potentially uncover new physics \cite{Abram_1992}.

In our prior research, we introduced GravAD, a Python-based search pipeline for GW detection utilising automatic differentiation (AD) and JAX \cite{jax2018github}. GravAD's approach centres around dynamically generating and refining waveform templates, thereby improving their fit to incoming data with each iteration. This method not only enhances the efficiency of the detection process but also significantly reduces the number of templates required for data analysis \cite{Doyle_2023}.

This paper aims to further expand on the advancements made to GravAD since our initial publication. We have implemented significant enhancements to the system, driven by two primary motivations. Firstly, the escalating complexity of waveform templates and the increasing sensitivity of LIGO detectors necessitate continuous improvements in data analysis methods for GWs \cite{tedwards, abbott_2020}. The growth in detector sensitivity and the rise in GW detections underscore the need for resilient and efficient analytical tools. Secondly, with the emergence of new detectors capable of observing fresh CBC events, our analysis approach must become more comprehensive \cite{menge_2020}.

To address these motivations, we have integrated simulated signals into our pipeline, pushing GravAD's boundaries and expanding its range of detectable astrophysical sources. We have also further decreased the number of templates needed for data analysis, thus improving the efficiency of the detection process without compromising the quality of results. Acknowledging the trade-off between precision and computational resource requirements \cite{Coogan_2022}, we have explored alternative optimisation algorithms. For instance, the Adam optimisation algorithm, a method that adjusts the learning rate based on the estimated moment of gradients \cite{adam}, has been partially incorporated into GravAD, yielding substantial enhancements in performance and efficiency.

In this research, our primary objective is to develop our search pipeline by integrating simulated signals and innovative optimisation strategies. This includes the adoption of a callback mechanism reminiscent of TensorFlow's callback method \cite{tf}, which effectively halts optimisation processes once specific criteria have been fulfilled. Subsequently, we will discuss the outcomes and critically evaluate the implications of these modifications.
