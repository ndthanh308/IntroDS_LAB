 \section{Discussion}
 
The discussion section aims to provide a comprehensive analysis of the advancements made to the GravAD system, focusing on its rapid search functionality and accuracy in detecting GWs. By combining various techniques, including the integration of simulated signals and optimisation strategies, the pipeline demonstrates its effectiveness in efficiently processing GW data. 

One significant aspect to consider is the importance of efficient methods in allocating computational resources for other research. As the field of GW detection continues to evolve and the number of detections increases, computational hardware becomes a valuable and limited resource. The GravAD system addresses this challenge by implementing optimisation techniques that reduce the number of templates required in the search, allowing for the allocation of computational resources in other research areas.

However, it is crucial to acknowledge the limitations of GravAD. The system is limited by the capabilities of the ripple software. This limitation implies that the effectiveness of our algorithm is dependent on the capabilities and advancements of the software it relies on. Therefore, future improvements in waveform generation and differentiation will play a crucial role in enhancing the effectiveness of the GravAD pipeline.

By integrating simulated signals and optimisation strategies, the algorithm effectively processes GW data. The optimisation techniques, such as the use of a callback method, steer the search away from the un-needed exploration of the parameter space, improving the efficiency of the search process.

Moreover, the GravAD system's ability to accurately process a diverse range of mass ratios, as demonstrated in the simulated signals, reinforces its credibility. This proficiency underpins the robustness and wide-ranging applicability of the system.

 