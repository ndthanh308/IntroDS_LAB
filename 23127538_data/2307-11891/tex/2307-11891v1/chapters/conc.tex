 \section{Conclusion}
 
This study underscores the notable advancements in GravAD's functionality and efficiency in detecting gravitational waves. Leveraging a multitude of techniques, the system has improved its capability to process simulated signals, thereby enhancing the accuracy of gravitational wave detection. Despite minor discrepancies in individual mass predictions, GravAD adeptly preserved total mass values, showcasing its practicality across diverse mass ratios.

Our research also highlights the vital role of optimisation strategies in augmenting the efficacy and speed of GravAD's search process. The blend of SGD, SA, and momentum has proven effective, offering a balance between high average SNR and reasonable average iterations.

A pivotal achievement lies in GravAD's substantial reduction in the number of templates required for search processes, improving computational efficiency remarkably without compromising result quality. This major stride forward paves the way for the allocation of resources to other vital research areas, proving instrumental in the continual expansion of gravitational wave detection.

The success of GravAD remains tethered to the progression of ripple library. Future enhancements in the generation and differentiation of waveforms could further boost GravAD's effectiveness. Therefore, the continual evolution of our system necessitates paralleled advancements in the underlying technology.
