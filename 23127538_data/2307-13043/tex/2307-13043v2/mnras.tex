\documentclass[useAMS,usenatbib]{mn2e}
\usepackage{epsfig,graphicx,latexsym,amsmath,amssymb}
\usepackage[export]{adjustbox}

\usepackage{natbib}
\usepackage{hyperref}
\usepackage{mathrsfs}
\usepackage{lastpage}


\usepackage{color}
\newcommand{\pau}[1]{\textcolor{cyan}{{\bf Comment from Pau:} ``#1.''}}

\bibpunct[,]{(}{)}{;}{a}{}{,}



\title[Underluminous tidal disruptions]
      {Underluminous tidal disruption events}

\author[P. Amaro Seoane] % Running authors,
                                                    % use abbrev. names here
{Pau Amaro Seoane$^{1}$
                        \thanks{E-mail: amaro@riseup.net (PAS)}
                                % Thanks goes to 1rst author only
                                % Continue with other authors, full names
\\
Universitat Politècnica de València, Spain\\
Max-Planck-Institute for Extraterrestrial Physics, Garching, Germany\\
Higgs Centre for Theoretical Physics, Edinburgh, UK\\
Kavli Institute for Astronomy and Astrophysics, Beijing 100871, China 
}


\def\mnras{MNRAS}
\def\ssr{Space Sci. Rev.}
\def\apj{Astrophys. J.}
\def\apjl{Astrophys. J. Lett.}
\def\prd{Phys. Rev. D}
\def\prl{Phys. Rev. Lett.}
\def\pr{Phys. Rev.}
\def\aap{Astron. Astrophys.}
\def\planss{Planet. Space Sci.}
\def\aapr{Astron. Astrophys. Rev.}
\def\araa{Annu. Rev. Astron. Astrophys.}
\def\pasj{Publ. Astron. Soc. Japan}
\def\jcap{J. Cosmology Astropart. Phys.}
\def\aj{Astron. J.}
\def\baas{Bull. Am. Astron. Soc.}
\def\nat{Nature}


\begin{document}
\label{firstpage}


\date{draft \today}

\pagerange{\pageref{firstpage}--\pageref{lastpage}} \pubyear{}

\maketitle

\begin{abstract}
We have evidence of X-ray flares in several galaxies consistent with a a star
being tidally disrupted by a supermassive black hole (MBH).  If the star starts
on a nearly parabolic orbit relative to the MBH, one can derive that the
fallback rate follows a $t^{-5/3}$ decay in the bolometric luminosity. We have
modified the standard version of the smoothed-particle hydrodynamics (SPH) code
{\sc Gadget} to include a relativistic treatment of the gravitational forces.
We include non-spinning post-Newtonian corrections to incorpore the periapsis
shift and the spin-orbit coupling up to next-to-lowest order. We run a set of
simulations for different penetration factors in both the Newtonian- and the
relativistic regime.  We find that tidal disruptions around MBHs in the
relativistic cases are underluminous for values starting at $\beta \gtrapprox
2.25$; i.e. the fallback curves produced in the relativistic cases are
progressively lower compared to the Newtonian simulations as the penetration
parameter increases. This is due to the fact that, contrary to the Newtonian
cases, we find that all relativistic counterparts feature a survival core for
penetration factors going to values as high as $12.05$.  We derive a
relativistic calculation which shows that geodesics of the elements in the star
converge as compared to the Newtonian case, allowing for a core to survive the
tidal disruption. A survival core should consistently emerge from any TDE with
$\beta \gtrapprox 2.25$. The higher the value, the lower the colour
temperatures than derived from standard accretion models.
%We have evidence of X-ray flares in several galaxies consistent with a a star
%being tidally disrupted by a supermassive black hole (MBH).  If the star starts
%on a nearly parabolic orbit relative to the MBH, one can derive that the
%fallback rate follows a $t^{-5/3}$ decay in the bolometric luminosity, as it
%has been observed in the sampled light curves.  Depending on the penetration
%factor, $\beta$, a star will be torn apart differently. A deep $\beta$ will
%lead to the maximum compression point closer to the MBH and the debris can
%cross with itself at later times.  We have modified the standard version of the
%smoothed-particle hydrodynamics (SPH) code {\sc Gadget} to include a
%relativistic treatment of the gravitational forces between the gas particles of
%a main-sequence (MS) star and a MBH. We include non-spinning post-Newtonian
%corrections to incorpore the periapsis shift and the spin-orbit coupling up to
%next-to-lowest order. We run a set of simulations for different penetration
%factors in both the Newtonian- and the relativistic regime.  We find that tidal
%disruptions around MBHs in the relativistic cases are underluminous for values
%starting at $\beta \gtrapprox 2.25$; i.e. the fallback curves produced in the
%relativistic cases are progressively lower compared to the Newtonian
%simulations as the penetration parameter increases. This is due to the fact
%that, while the Newtonian cases display a total disruption, we find that all
%relativistic counterparts feature a survival core for penetration factors going
%to values as high as $12.05$.  We also derive a relativistic calculation which
%shows that we can indeed expect that in this case the geodesics of the elements
%in the star converge as compared to the Newtonian case, allowing for a core to
%survive the tidal disruption.  The measured colour temperatures of
%optically-discovered tidal disruption events are lower than predicted
%theoretically due to the fact that these arise from unbound orbits around MBHs
%when relativistic effects are taken into account. A survival core should
%consistently emerge from any TDE with $\beta \gtrapprox 2.25$. The higher the
%value, the lower the colour temperatures than derived from standard accretion
%models.
\end{abstract}

\begin{keywords}
transients: tidal disruption events -- relativistic processes -- quasars: supermassive black holes
\end{keywords}

\section{Motivation}
\label{sec.motivation}

A star passing very close to a massive black hole (MBH) may be torn apart
because of the tidal effects, and the interaction of the stellar debris in the
vicinity of the black hole will give rise to a burst of electromagnetic
radiation. The characteristics of this tidal disruption event (TDE), such as
its temperature, peak luminosity, and decay timescale, are functions of the
mass and spin of the central MBH.  The subsequent accretion of the debris gas
by the black hole produces additional emission, and lead to phases of bright
accretion that may reveal the presence of a MBH in an otherwise quiescent
galaxy \citep[see
e.g.][]{Wheeler1971,Hills75,FR76,CarterLuminet1982,CarterLuminet1983,Rees88,MCD91,MT99,SU99,FB02b,GezariEtal03,WM04},
with rates which vary depending on various factors, but of the order of
$10^{-5}-10^{-6}\,\textrm{yr}^{-1}$ (see e.g. \citealt{Rees1988} and
\citealt{StoneEtAl2020} for a recent review on the rates and characteristics).
These phenomena can be used as a probe of accretion physics close to the event
horizon \citep{Brenneman2013,Reynolds2014}.

Many disruption candidates have already been detected with ROSAT, Chandra,
Swift (see e.g. \href{https://tde.space}{https://tde.space}), and the ZTF
\citep{vanVelzenEtAl2020} and the number will surge with upcoming transient
surveys like the Large Synoptic Survey Telescope (LSST), SRG/eROSITA
\citep{KhabibullinEtAl2014}, as well as the ESA L2 mission Athena
\citep{Athena+Whitepaper}. 

A conondrum related to optically-discovered TDEs is that their color
temperatures are significantly below the values predicted by theoretical models
\citep{GezariEtAl2012,ChornockEtAl2014,HoloienEtAl2014,vanVelzenEtAl2014,ArcaviEtAl2014}.
Observations depict a temperature and bolometric luminosity well below
theoretical predictions based on accretion, and based on the same model, the
derived black-body emission radius implies an orbital motion below the expected
theoretical values. In general, the fallback model requires masses much less
than a solar mass in order to explain the difference in luminosity of the
observed flares and the theoretical expectations.

Different theoretical models have been put forward to explain this fact.  The
work of \cite{LiEtAl2002} suggested that the low luminosity may indicate that
the disrupted star is a brown dwarf or a planet.  An alternative explanation is
that the assumption that the gas immediately circularizes when it comes back
close the MBH is not accurate, and could trigger internal shocks that would
result into a reduced luminosity. In particular, \cite{PiranEtAl2015} suggested
that the released energy during the process of circularisation, and not that
accreted on to the MBH, is responsible for the observed optical TDE candidates.
More recently, \cite{ZhouEtAl2020} argued that the disk does not circularize
and remains eccentric, which as a consequence leads the orbital energy of the
stellar debris to be advected on to the MBH without being radiated away.

In this work we show with a set of smoothed-particle hydrodynamics (SPH)
simulations with relativistic corrections that unbound stars lead to partial
disruptions, which naturally explain the difference in the observed luminosity,
for penetration values as deep as $\beta=12.05$.

\section{Relativistic implementation}

Relativistic effects have been considered in the related literature by e.g.
\cite{TejedaEtAl2017}, which implemented a relativistic description of the
evolution of the hydrodynamical elements with a quasi-Newtonian treatment of
the fluid's self-gravity.  

Earlier this year, Ryu and collaborators presented a series of works which also
address TDEs in a relativistically fashion. For this approach they depart from
the intrinsically-conservative GR hydrodynamical numerical code of
\cite{NobleEtAl2009} designed to study magnetohydrodynamic (MHD) turbulence in
accretion disks around MBHs. To study TDEs, they assume that spacetime is
Schwarzschild plus contributions from the star self-gravity, and the dynamics
of the star is described by hydrodynamics, solving the general-relativistic
energy-momentum equations in the Schwarzschild background. This hence implies
that in the absence of material forces the fluid elements strictly follow
geodesics. The self-gravity of the star is described in the weak field, taking
into account only the Newtonian gravitational potential.  They then evolve the
hydrodynamical equations in a frame where the metric is nearly flat and  move
the whole system in a rigid way along the orbit using parallel transport of the
local frame. Hence, it is the trajectory of the system what ``sees'' the
Schwarzschild metric but the fluid elements almost live in a flat spacetime.

More precisely, they consider a modified metric, $\tilde{g}_{\mu\,\nu} =
g_{\mu\,\nu} + h_{\mu\,\nu}$, with $g_{\mu\,\nu}$ the Schwarzchild
metric and $h_{\mu\,\nu}$ accounts for the self-gravity of the star. They
assume the self-gravity is weak, so that the only non-zero component of the
self-gravity perturbations is the time-time one: $h_{tt} = -2 \Phi_{sg} /
c^2$, where $\Phi_{sg}$ is the Newtonian potential of the star, in the sense
that it satisfies a Poisson equation where the mass density is replaced by the
star proper rest-mass density. The procedure to incorporate this self-gravity
is a bit more intrincate than adding it to the Schwarzschild metric. The
assumptions made to compute the self-gravity contribution include that the
metric of Schwarzschild should not deviate from the Minkowski metric. Here is
where the intrincacy mentioned comes, since they need a parallel-transported
tetrad adapted to the star as mentioned before, so that in that frame the
assumptions made are valid. It is important to note that they separate the
problem of solving the hydrodynamical equations from the motion of the star
around the MBH. This can be envisaged as having a frame center at the (center
of mass) star where the metric, in an orthonormal basis, is exactly Minkowski
(deviating as we move from the center of mass). Then they solve the
hydrodynamical equations in this frame and the motion of the star is ``rigid''
(only the center of mass moves) according to the parallel-transport equations
for the orthonormal basis \citep[see][]{RyuEtAl2020b}.  In their calculations
all stars have net bound orbits by an amount of the order $\sim 10^{-10}c^2
\sim 10^{-3} (\sigma^2/2)$, where $\sigma$ is the bulge dispersion and $c$ the
speed of light.

With this scheme they investigate TDEs in four additional works. In
\cite{RyuEtAl2020a} they find that the critical pericenter distance for full
disruptions is enhanced by up to a factor of $\sim 3$ as compared to the
Newtonian case, and that it depends on the mass of the star in a non-trivial
way \citep[see][for previous work]{GuillochonRamirez-Ruiz2013}.  The results of
\cite{RyuEtAl2020c} regarding partial disruptions show that due to the little
mass distributed at low energies, late-time fallback is suppressed. The mass
return rate should then be $\propto t^{-p}$ with $p\in [2,\,5]$ in partial
disruptions.  In \cite{RyuEtAl2020d} they show that relativistic effects induce
width  delays in the debris energy so that the magnitude of the peak return
rate decreases.  These results had already been pointed out by the previous
work of \cite{IvanovChernyakova2006,Kesden2012,ServinKesden2017}, although Ryu
and collaborators provided quantitative corrections to these previous
treatments.  In \cite{KrolikEtAl2020} they discuss the event rates and the fate
of the rest of the star which is not disrupted (i.e. the amount of mass still
inside the computational box when they stopped the simulation), which might
interact with the MBH on a second periapsis passage or rejoin the stellar
cluster.

In this work we modify the acceleration computation of {\sc Gadget}
\citep{Springel2005} to include relativistic corrections, which are based on
the post-Newtonian (PN) formalism for the interaction between two bodies (in
our case each of the hydrodynamical particles and the MBH).  This means that we
simply add relativistic correcting terms to the Newtonian gravitational forces
calculated between the MBH and the hydrodynamical particles that form the star
during the whole simulation, which initially is set on a completly unbound
orbit.  This approximation allows us to capture the relativistic effects while
allowing us to study the evolution of the star to larger radii without any
other approximation than those inherent to SPH methods and the post-Newtonian
expansion, valid in this regime of low (but yet relativistic) velocities. In
this regard, our scheme is self-consistent and all phenomena related to
relativistic effects and hydrodynamics emerge naturally by integrating the
system.

The relative acceleration, in the center-of-mass form, including all PN
corrections can be written in the following way:

\begin{equation}
 \frac{d \vec v}{d t} = - \frac{G m}{r^2}[(1+A)\,\vec n + B \vec v \,] + \vec C_{\rm 1.5,SO}% + \vec C_{\rm 2,SS} + \vec C_{\rm 2.5,SO},
\label{eq.acc}
\end{equation}

\noindent
In this equation $\vec v = \vec v_1 - \vec v_2$ is the relative velocity
vector, $m = M_\bullet + m_{\rm gas}$ the total mass, with $M_\bullet$ the mass
of the MBH and $m_{\rm gas}$ the mass of a gas particle, $r$ the separation and
$\vec n = \vec r/r$. $A$ and $B$ are coefficients that can be found in
\cite{BlanchetIyer03}. Since we are modelling extremely light gas particles
around a MBH, we adopt the terms in the limit $m_{\rm gas} = 0$.  We consider
the leading order spin-orbit interaction, with the term $\vec C_{\rm 1.5,SO}$
in which the subscript SO stands for spin-orbit coupling, which can be found in
\citep{TagoshiEtAl01,FayeEtAl2006}, and the first post-Newtonian correction to
periapsis shift. We do not include dissipative terms because, contrary to an
extreme-mass ratio inspiral \citep{Amaro-SeoaneLRR2012}, the star only has one periapsis passage,
and the gravitational radiation can be neglected. All the PN interactions are only considered between a gas
particle and the MBH and are evaluated at all times during the whole integration. In all simulations with
spin we set the dimensionless spin vector to $\vec a = (0.7,\,0.7,\,0)$, so we
get maximum precession of the orbit of the star orbit lying in the X--Y plane.

The implementation of these relativistic terms follows the prescription given
in the work of \cite{KupiEtAl06}, which was the first work published about the
inclusion of post-Newtonian corrections in the context of stellar dynamics.
The addition of the spin to the problem was presented, also for the first time,
in a stellar-dynamics context in \cite{BremAmaro-SeoaneSpurzem2014}.  Both, the
periapsis shift and the spin terms have been tested in detail, and partially
published in the work of \cite{BremAmaro-SeoaneSpurzem2014} with a series of
comparisons with the semi-Keplerian approximation of \cite{Peters64}.



\section{Initial conditions}

In all simulations the mass of the MBH is $m_{\bullet}=10^6\,M_{\odot}$, the
mass of the star is $m_{\star}=1\,M_{\odot}$ and it is set on an unbound,
parabolic trajectory around the MBH, placed at the focus. It must be noted that
while bound orbits are less ``expensive'' computationally, the most natural
orbits are unbound ones, i.e. parabolic or hyperbolic, because we do not expect
the region of phase-space close to the MBH to produce bound orbits, at least in
a Milky Way-like galaxy\citep[see
e.g.][]{Amaro-SeoaneLRR2012,BaumgardtEtAl2018,SchoedelEtAl2018,Gallego-CanoEtAl2018}.

When the stars approach the MBH it will experience strong tidal forces whenever
the work exerted over the star by the tidal force exceeds its own binding
energy, (all energies are per unit mass), which is 

\begin{equation}
E_{\rm bind}=\frac{3\,G\,m_{\star}}{(5-n)r_{\star}},
\label{eq.Ebind}
\end{equation}

\noindent
where $n$ is the polytropic index \citep{Chandra42}, $m_{\star}$ the mass of
the star. This allows us to introduce a typical radius for this to happen, the
tidal radius $r_t$. Considering $r_{\star}\ll r_{\rm t}$,

\begin{equation}
r_t=\Bigg[\frac{(5-n)}{3}
\frac{{m}_{\bullet}}{m_{\star}}\Bigg]^{1/3} 2\,r_{\star}.
\label{eq.r_tid_bind}
\end{equation}

\noindent
With ${m}_{\bullet}$ the mass of the MBH. For a solar-type star, considering an $n=3$ 
polytrope, and ${m}_{\bullet}=10^6\,M_{\odot}$, we have that

\begin{equation}
r_{\rm t} = 110\,R_{\odot}\sim 0.51\,\textrm{AU}.
\label{eq.r_tib_bind_b}
\end{equation}

The initial distance of the star to the MBH is set to 20 times the axis of
symmetry of the parabola, i.e. the pericentric distance between the MBH and the
star. In order to investigate the fate of the bound material to
the star and the fallback rate, we have chosen a series of trajectories with
different penetration factors $\beta$, which is defined to be the ratio between the
tidal radius and the distance of periapsis, $1.5,\,2,\,3,\,4,\,5$ and $9$, and
run for each case (i) a Newtonian simulation, (ii) a relativistic simulation
taking into account only the periapsis shift of the SPH particles and (iii) a
relativistic simulation taking into account this effect and the spin
correction.

It must be noted that the value for the penetration factors has been estimated
by assuming a point-particle trajectory.  However, in the relativistic cases,
an initially assigned value for $\beta$ diverges as the star progresses in its
orbit towards the MBH as a function of the penetration factor. When the
extended star achieves the vertex of the parabola, the penetration factor has
differed from the initially chosen value. Hence, we initially set the star in
that point-particle trajectory for those specific $\beta$ values, and we derive
the real penetration factor when it reaches the verteces of the relativistic
cases.  These are $\beta=1.64,\,2.26,\,3.62,\,5.15,\,6.83$ and $12.05$, and we
use them in the Newtonian cases as well so as to be able to compare the
results. These values in turn correspond to the following distances of
periapsis: $r_{\rm peri} \sim
0.32,\,0.22,\,0.14,\,0.1,\,0.07,\,0.04~\textrm{AU}$. For a parabolic orbit, the
velocity of the star at periapsis is $v_{\rm peri}=\sqrt{2\,\mu/r_{\rm peri}}$,
with $\mu=G\,m_{\bullet}\simeq 1.3\times 10^{26} m^3\,s^{-2}$, so that the
corresponding velocities are $v_{\rm peri} \sim 7.37\times 10^7,\, 8.88 \times
10^7,\,1.11 \times 10^8,\,1.32\times 10^8,\,1.57\times 10^8,\,2.08\times
10^8\,m/s$ which in units of the speed of light $c$ are, respectively,
$0.24,\,0.29,\,0.37,\,0.44,\,0.52,\,0.69$. 

The stars are modelled as main-sequence (MS) stars with a polytrope of index
$3$ constructed initially following the method of \cite[][in particular, see
their on-line complements]{FB05}. We employ half a million particles to
construct the polytrope, which is enough to solve the tidal disruption process.
Enlarging this number by a number below at least an order of magnitude does not
necessary lead to a significant improvement of the simulation \citep[see section 3
of][]{Rasio2000}.


\section{Quantitative analysis}

One important aspect in the process is the evoltion of the star after the first
periapsis passage.  To determine which part of the gas particles in the
simulation is still bound or unbound to the star, we follow the following
prescription \citep[based in the work of][]{LaiEtAl1993,FulbrightEtAl1995}: So
as to decide whether a gas particle $i$ is bound to the star, we calculate the
specific energy of this particle relative to the star,

\begin{equation}
\epsilon_i = \upsilon_i + \frac{1}{2} v^2_{\rm rel} - \sum_j \frac{G m_i m_j}{r_{ij}},
\label{eq.specific}
\end{equation}

\noindent
where $\upsilon_i$ and $v_{\rm rel}$  are respectively
the internal energy and the relative velocity of gas particle $i$ to the center
of mass velocity of all the gas particles belonging to the star. The potential
part is summed up over all star particles $j$. If $\epsilon_i > 0$, the
particle is considered unbound from the star, otherwise bound. In the first
step of the iteration, all particles are assumed to be star particles. After
evaluating Eq. \ref{eq.specific}, particles are reassigned to be either still
part of the star or unbound. In the next step, the specific energy is evaluated
with respect to the reduced fraction of star particles. We stop the iteration
when reassignments to the unbound component cannot be made.  After the
iteration is complete we check which part of the gas that is no longer
gravitationally bound to the star is on Keplerian orbits around the MBH or
completely unbound from the system. This part of the gas is then considered for
the luminous fallback onto the MBH.

% Figure environment removed

In Fig.~(\ref{fig.Bound_Ejected_Mass_Beta1p64_to_3p62}) we show the fate of the
material stripped (bound) from (to) the star for the different penetration
factors mentioned before up to 3.62. The scaling of the first two panels in all
these figures is the same in order to be able to compare better.  In all
figures we also add a third panel showing the long-term evolution of the
Newtonian cases.  We can see from these figures that the amount of bound mass
to the star, i.e.  the survival star, is in all cases larger in the
relativistic simulations than in the Newtonian counterparts.  For the
relativistic cases, about $70\%,\,40\%,\,20\%$ of the star survives the
disruption after one day for the first three values of
$\beta=1.64,\,2.26,\,3.62$, while in the Newtonian simulations, this quantity
is only about $50\%,\,5\%,2\%$ These results are general in agreement with the
amount of mass of the survival core found by the work of \cite{RyuEtAl2020c}.

% Figure environment removed

In Figure~(\ref{fig.Bound_Ejected_Mass_Beta5p15_and_12p05}) we depict the same
as in Fig.~(\ref{fig.Bound_Ejected_Mass_Beta1p64_to_3p62}) for the two most
extreme penetration factors, $\beta=5.15,\,12.05$ (this last penetration factor
the star is still at a distance of two Schwarzschild radii from a Schwarzschild
MBH of mass $10^6\,M_{\odot}$). In this case, the amount of bound material to
the star in the Newtonian cases further decreases as compared to larger $\beta$
values. It is of about $1\%$ and $0\%$ respectively. However, the relativistic
cases show a much larger survival stellar object, with a mass of about $38\%$
for $\beta=5.15$ and of about $42\%$ for the most extreme case,
$\beta=12.05\%$; i.e. a larger quantity.  In the relativistic simulations the
amount of bound material to the star is larger than in their Newtonian
counterparts, even at the smallest value of $\beta$.

In this first case, for $\beta=1.64$, we observe in the Newtonian case of
Fig.(\ref{fig.Bound_Ejected_Mass_Beta1p64_to_3p62}) that a significant amount
of matter of the star survives the close interaction with the MBH.  Indeed,
\cite{GuillochonRamirez-Ruiz2013} estimated that (Newtonian) TDEs have 100\%
disruption only for penetration factors $\beta > 1.85$.  Recently,
\cite{MilesEtAl2020} have studied (Newtonian) partial disruptions that
corroborate the fallback rate proportion of $\propto t^{-9/4}$. Other
scenarios, like the progressive disruption of a star as result of a tidal
separation, however, lead to different eccentricities which predict a different
scaling of $\propto t^{-1.2}$ \cite{Amaro-SeoaneMillerKennedy2012}. Also,
\cite{RyuEtAl2020c} find $\propto t^{-p}$ with $p\in [2,\,5]$ depending on the mass of the star
and the role of their relativistic implementation.

In Fig.~(\ref{fig.SurvivalCoreBeta1p64_N}) we show the last snapshot of the
Newtonian simulation with $\beta=1.64$ using the visualization tool of
\cite{Price2007} to render the gas particles. Embedded in the figure we have
added a zoom of the area corresponding to what seems to be the remaining core
of the star.  However, this is a transient feature, as we can see in the
uppermost, right panel of Fig.~(\ref{fig.Bound_Ejected_Mass_Beta1p64_to_3p62}).
We integrated the system for up to some $\sim 18$ days from the starting point,
and we can see that the amount of mass decays very quickly with time, while the
amount of matter of the original star bound to the MBH is kept constant.  This
episodic core will not be bound to the star at later times. However, it appears
later as an enhanced fallback of debris in the first panel of
Fig.~(\ref{fig.FallbackBetas}) at later times, as we will explain in the next section.


% Figure environment removed



\section{Fallback rate}

% Figure environment removed

\vspace{0.3cm}


So as to test the implementation of the orbit and the behavior of the SPH star,
we calculate the fallback rate on to the MBH. For this, we calculate the
required time for the bound debris to come back again to periapsis by
estimating the specific energy of each particle, $E = G\left(M +
m\right)/(2\,a)$, with $G$ the gravitational constant, $M$ and $m$ the masses
of the MBH and one SPH particle, respectively.


From the angular momentum, one can derive that

\begin{equation}
e = \sqrt{1 + E \left(\frac{L}{\mu} \right)^2},
\label{eq.}
\end{equation}

\noindent
where we have introduced $\mu = G\,M$ and neglect the contribution of $m$. If we
define $\Delta t$ as the ellapsed time between the first periapsis passage and the
current position of the particle's position, at a radius $R$ from the MBH and time $T$, the
necessary time for the next periapsis passage is $t_{\rm peri}=T - \Delta t$.

This distance is

\begin{equation}
R = a\,\left(1-e\cdot \cos \epsilon\right),
\end{equation}

\noindent
with $\epsilon$
the eccentric anomaly; so that $\epsilon < \pi$ for outbound motion and
$\epsilon > \pi$ for inbound motion.  The mean anomaly can be calculated as


\begin{equation}
M = 2\pi \delta t/T = \epsilon -e \cdot \sin{\epsilon},
\end{equation}

\noindent
and allows us to calculate the ellapsed time since the last periapsis passage
(at $M=0$).  We integrate the star's orbit until its center of mass has
traveled far away from the tidal radius out to $3000 R_\odot$, i.e.  $\sim
12730\,R_{\rm Schw}$, with $R_{\rm Schw}$ the Schwarzschild radius of the MBH.
From this point on we assume the gas particles to be traveling on independent,
non-interacting Keplerian orbits solely determined by their orbital energy and
angular momentum. From these values we compute the classical time until the
subsequent pericenter passage, $t_{\rm peri}$.

We present the results as fallback curves which are mass histograms over
$t_{\rm peri}$ in Fig.~(\ref{fig.FallbackBetas}) for different values of the
penetration factor $\beta$.  We can see that the lower value of $\beta$ leads
to a drop in the Newtonian fallback between $10^2$ and $10^4$ days, which can
be envisaged as a result of matter falling back more quickly. This forms the
depression around the later enhanced fallback. This is matter which is bound to
the MBH, not to the original star, which has started to become bound to itself
after the evolution. The relativistic cases also display this feature but at
much earlier times and with much smaller depressions. This partial disruption
leads to fallback values in the Newtonian case which are similar to the
relativistic ones, as in the next value of $\beta=2.26$. From that value
upwards, the Newtonian cases lead to fallback values significantly higher than
the relativistic ones, starting with about half an order of magnitude up to
about five orders of magnitude for the deepest penetration and the spin case.
We can observe only a clear effect of the spin when we go to extreme
penetration values, the lowermost, right panel, with $\beta=12.05$. 

% Figure environment removed

In Fig.~(\ref{fig.SPH_mosaic_beta_2p26_N}) we show a mosaic with nine different
snapshots in the evolution of the Newtonian case of $\beta=2.26$. As we zoom
in, we can see that at later times, 5.73 days, no surviving core is left. This
situation changes completely when we consider the relativistic corrections, as
we can see in its counterpart, Fig.~(\ref{fig.SPH_mosaic_beta_2p26_R}), which
takes into account the post-Newtonian correcting terms. We can clearly see the
survival of a core which is bound to the original star, as shown in
Fig.~(\ref{fig.Bound_Ejected_Mass_Beta1p64_to_3p62}).

% Figure environment removed

This difference becomes even more evident when comparing the extreme case of
$\beta=12.05$, in Fig.~(\ref{fig.SPH_mosaic_beta_12p05_N}), the Newtonian case
and Fig.~(\ref{fig.SPH_mosaic_beta_12p05_R}).  Short after the passage through
periapsis, nothing is left from the original star in the Newtonian case, while
in the relativistic one we find a surviving core at much later times.

% Figure environment removed

% Figure environment removed

\section{An evaluation of the energy}

In order to understand the outcome of the results, we evaluate the internal
energy of the star in both the relativistic and Newtonian cases.  For
simplification, we approximate the star as a perfect fluid. The internal energy
of a perfect fluid is given by the equation:

\begin{equation}
U = \frac{3}{2} N k_B T,
\end{equation}

\noindent 
where \(N\) is the number of particles, \(k_B\) is the Boltzmann constant, and
\(T\) is the temperature. We can express \(N\) in terms of the mass of the star
\(m_{\star}\) and the mean molecular weight \(\mu\) as \(N = m_{\star} / \mu\). The mean
molecular weight is approximately \(\mu \approx 1\) for a fully ionized gas,
which is a reasonable approximation for the interior of a star. The temperature
of the star can be approximated as \(T \approx 10^7\) K, which is typical for
the core of a main-sequence star.

\noindent 
In the Newtonian case, the gravitational potential energy of the star is given by:

\begin{equation}
U_{\text{Newt}} = - \frac{3 G m_{\star}^2}{5 r_{\star}},
\end{equation}

\noindent 
where \(G\) is the gravitational constant, \(m_{\star}\) is the mass of the star, and
\(r_{\star}\) is the radius of the star. This equation comes from the standard formula
for the gravitational potential energy of a uniform sphere.

\noindent 
In the relativistic case, the gravitational potential energy of the star is
modified by the presence of the black hole. The gravitational potential energy
in the relativistic case can be approximated as:

\begin{equation}
U_{\text{Rel}} = - \frac{3 G m_{\star}^2}{5 r_{\star}} \left(1 + \frac{3 G m_{\star}}{2 c^2 r_{\star}}\right),
\end{equation}

\noindent 
where \(c\) is the speed of light. The term \(\frac{3 G m_{\star}}{2 c^2 r_{\star}}\) is the
post-Newtonian correction to the gravitational potential energy, which becomes
significant when the star is close to the black hole \citep[see e.g.][]{shapiro1983black}.  We can then calculate
the ratio of the internal energy in the relativistic case to the internal
energy in the Newtonian case as:

\begin{equation}
\text{Ratio} = \frac{U_{\text{Rel}}}{U_{\text{Newt}}}.
\end{equation}

\noindent 
Substituting the values for \(U_{\text{Rel}}\) and
\(U_{\text{Newt}}\) into this equation, we find that the ratio is
approximately \(2.37 \times 10^{6}\). This suggests that the internal energy of
the star in the relativistic case is about one million times larger than in
the Newtonian case.

The large energy ratio indicates that the internal energy of the star is
significantly higher in the relativistic case compared to the Newtonian case.
This is primarily due to the additional energy contributions from the
relativistic effects, such as the gravitational redshift and time dilation near
the black hole, which are not present in the Newtonian case.

The implications for the star are significant. In the Newtonian case, the
star's internal energy is not sufficient to counteract the tidal forces from
the black hole, leading to the star's disruption. However, in the relativistic
case, the higher internal energy could potentially allow the star to withstand
the tidal forces to a greater extent. This could result in a larger portion of
the star surviving the close encounter with the black hole.

\section{Geodesic convergence of particles on a parabolic orbit around a spinning massive black hole}

One better way to try to understand why a core survives in the relativistic
simulations is to calculate the geodesic convergence (or divergence) of the
elements of the star as it approaches the pericentre of the MBH. Once we have
the result for two test particles, we will extend the result to a group of them
distributed in a spherical fashion and try to derive whether they build up a
denser core in the relativistic case or not. I.e. we want to see whether the
particles tend to get closer in the relativistic case of this ``dust star'' as
compared to the Newtonian case. If this is the case, then we could understand
the outcome of the SPH simulations.  However, this is not an easy problem that
involves several steps and concepts. Here is a list of the steps we would need
to address to solve it.

We first need to calculate the geodesic deviation equation, which describes the
relative acceleration of nearby geodesics in a curved spacetime.  It can be used
to calculate the separation between particles in the dust star. The equation is
given by:

\begin{equation}
\frac{d^2 \xi^\alpha}{d\tau^2} = - R^\alpha_{\ \beta\mu\nu} u^\beta u^\mu \xi^\nu,
\label{eq.geo_dev}
\end{equation}

\noindent
where $D/d\tau$ is the covariant derivative along the geodesic,
$\xi^\alpha$ is the separation vector between the geodesics, $u^\alpha$ is
the four-velocity of the particles, and $R^\alpha_{\ \beta\mu\nu}$ is the
Riemann curvature tensor. 
 
We will also need to estimate the tidal tensor, given by the projection of the
Riemann tensor onto the observer's local rest frame. It describes the tidal
forces experienced by the dust star due to the black hole. In the Newtonian
limit, the tidal tensor is given by the second spatial derivatives of the
gravitational potential. In the relativistic case, it can be calculated from
the geodesic deviation equation. The relativistic corrections to the Newtonian
orbit can be calculated using the post-Newtonian approximation. The main
corrections are due to the perihelion shift and the frame-dragging effect
caused by the spin of the black hole. The perihelion shift can be calculated
using

\begin{equation}
\Delta\varphi = 6\pi \frac{GM}{c^2 a(1-e^2)},
\end{equation}

\noindent
where \(G\) is the gravitational constant, \(M\) is the mass of the black hole,
\(c\) is the speed of light, \(a\) is the semi-major axis of the orbit, and
\(e\) is the eccentricity. The frame-dragging effect can be calculated using
the Kerr metric, which describes the spacetime around a rotating black hole.
The change in the density distribution of the dust star can be calculated by
integrating the geodesic deviation equation over the volume of the star. This
will give the change in volume, and hence the change in density, as a function
of time.

\subsection{Two test particles}

The geodesic deviation equation describes the relative acceleration of nearby
geodesics in a curved spacetime. It can be used to calculate the separation
between the two particles as they approach the pericentre. The equation is
given by Eq.~(\ref{eq.geo_dev}). In the case of a Schwarzschild black hole, the
Riemann tensor has only one independent component, which can be written in
terms of the mass \(M\) of the black hole and the radial coordinate \(r\) as:

\begin{equation}
R_{trtr} = - \frac{2GM}{c^2 r^3}.
\end{equation}

\noindent
The geodesic deviation equation then simplifies to:

\begin{equation}
\frac{d^2 \xi^r}{d\tau^2} = - R_{trtr} u^t u^t \xi^r = \frac{2GM}{c^2 r^3} (u^t)^2 \xi^r,
\end{equation}

\noindent
where \(\xi^r\) is the radial component of the separation vector, and \(u^t\)
is the time component of the four-velocity.  Assuming that the two particles
are initially at rest with respect to each other, we have \(\xi^r = D\) and
\(u^t = dt/d\tau\), where \(t\) is the coordinate time and \(\tau\) is the
proper time. The equation then becomes:

\begin{equation}
\frac{d^2 D}{dt^2} = \frac{2GM}{c^2 r^3} \left(\frac{dt}{d\tau}\right)^2 D.
\end{equation}

\noindent 
This is a second-order differential equation for \(D\) as a function of \(t\).
The solution will depend on the initial conditions, which are given by \(D(0) =
D_0\) and \(D'(0) = 0\), where \(D_0\) is the initial separation between the
particles.  The solution to this equation will give the separation \(D\)
between the particles as a function of time as they approach the pericentre.
The value of \(D\) at the pericentre can then be obtained by evaluating the
solution at the time of pericentre passage.

Also, the spin of the black hole introduces an additional term in the
geodesic deviation equation, which accounts for the frame-dragging effect. This
effect is caused by the rotation of the black hole, which drags the surrounding
spacetime along with it.
The geodesic deviation equation in the presence of a rotating black hole (Kerr black hole) in Fermi normal coordinates is given by:

\begin{equation}
\frac{d^2D^i}{dt^2} = -R^i_{\phantom{i}0j0} D^j - 2 v^k \Omega_{kj} D^j,
\end{equation}

\noindent 
where \(v^k\) is the velocity of the test particle, \(\Omega_{kj}\) is the
angular velocity of the frame-dragging effect, and the other symbols have the
same meaning as before.

The second term on the right-hand side represents the effect of the black
hole's spin. This term causes a precession of the test particles' orbits, which
can lead to an increase in the separation between the particles, and hence an
increase in the density of the dust star.

However, solving this equation analytically to find the separation \(D(t)\) as
a function of time in the presence of a spinning black hole is a complex task
that would typically requires numerical methods.To simplify the problem, we can
make a few approximations. We will assume that the gravitational field of the
black hole is weak at the location of the particles. This is valid if the
particles are far from the event horizon of the black hole. Also, we will
assume that the particles are moving slowly compared to the speed of light.
This is valid if the kinetic energy of the particles is much less than their
rest mass energy.  The Fermi normal coordinates are a local coordinate system
in which the metric of spacetime is approximately Minkowskian. They are
particularly useful for calculating the tidal forces experienced by a small
body moving in a curved spacetime.

\noindent 
The Riemann tensor in Fermi normal coordinates can be calculated from the
metric. The only non-zero component is:

\begin{equation}
R_{trtr} = - \frac{2GM}{c^2 r^3}.
\end{equation}

\noindent 
The geodesic deviation equation then becomes:

\begin{equation}
\frac{d^2 \xi^r}{d\tau^2} = - R_{trtr} u^t u^t \xi^r = \frac{2GM}{c^2 r^3} (u^t)^2 \xi^r.
\end{equation}

\noindent 
where \(\xi^r\) is the radial component of the separation vector, and \(u^t\)
is the time component of the four-velocity.  Assuming that the two particles
are initially at rest with respect to each other, we have \(\xi^r = D\) and
\(u^t = dt/d\tau\), where \(t\) is the coordinate time and \(\tau\) is the
proper time. The equation then becomes:

\begin{equation}
\frac{d^2 D}{dt^2} = \frac{2GM}{c^2 r^3} \left(\frac{dt}{d\tau}\right)^2 D.
\end{equation}

\noindent 
This is a second-order differential equation for \(D\) as a function of \(t\).
The solution will depend on the initial conditions, which are given by \(D(0) =
D_0\) and \(D'(0) = 0\), where \(D_0\) is the initial separation between the
particles. I.e. we are simplifying the problem by assuming that initially
the particles have a relative velocity of zero, so as to be able to assess the
evolution of the system once it reaches pericentre.

The solution to this equation will give the separation \(D\) between the
particles as a function of time as they approach the pericentre. The value of
\(D\) at the pericentre can then be obtained by evaluating the solution at the
time of pericentre passage.  In the case of a Kerr black hole, the spacetime is
not only curved due to the mass of the black hole, but also twisted due to its
spin. This leads to an additional effect known as frame-dragging, which can
affect the motion of the test particles.

In the weak field and slow motion approximations, the geodesic deviation equation in the equatorial plane (\(\theta = \pi/2\)) of a Kerr black hole in Fermi normal coordinates is given by:

\begin{equation}
\frac{d^2 \xi^r}{dt^2} = \frac{2GM}{c^2 r^3} \left(\frac{dt}{d\tau}\right)^2 \xi^r - \frac{4GJ}{c^2 r^3} \left(\frac{dt}{d\tau}\right) \xi^\phi,
\end{equation}

\noindent 
where \(\xi^\phi\) is the azimuthal component of the separation vector, and \(J = aGM/c\) is the angular momentum of the black hole, with \(a\) being the spin parameter. This equation describes the radial tidal force experienced by the test particles, as well as the frame-dragging effect due to the spin of the black hole. The second term on the right-hand side is the frame-dragging term, which is proportional to the spin parameter.

Assuming that the two particles are initially at rest with respect to each other, we have \(\xi^r = D\) and \(\xi^\phi = 0\), and \(u^t = dt/d\tau\), where \(t\) is the coordinate time and \(\tau\) is the proper time. The equation then becomes:

\begin{equation}
\frac{d^2 D}{dt^2} = \frac{2GM}{c^2 r^3} \left(\frac{dt}{d\tau}\right)^2 D.
\end{equation}

\noindent 
This is a simple harmonic oscillator equation, and its solution is:

\begin{equation}
D(t) = D_0 \cos\left(\sqrt{\frac{2GM}{r^3}} t\right),
\end{equation}

\noindent 
where \(D_0\) is the initial separation between the particles. This solution indicates that the separation \(D\) between the particles oscillates with a frequency that increases as the particles approach the black hole (since \(r\) decreases), and its amplitude decreases.

At the pericentre, the radial coordinate \(r\) is equal to the pericentre distance \(r_p\), and the separation \(D\) is:

\begin{equation}
D(t_p) = D_0 \cos\left(\sqrt{\frac{2GM}{r_p^3}} t_p\right),
\end{equation}

\noindent 
where \(t_p\) is the time of pericentre passage. This shows that the separation between the particles decreases as they approach the pericentre, which is consistent with the expectation that the particles should be drawn closer together by the gravitational pull of the black hole.

For comparison, in the Newtonian case, the solution to the geodesic deviation equation is:

\begin{equation}
D(t) = D_0 \cos\left(\sqrt{\frac{GM}{r^3}} t\right),
\end{equation}

\noindent 
which oscillates with a slower frequency and larger amplitude than the relativistic case. The oscillation frequency of the separation between the particles is determined by the term inside the square root in the cosine function of the solution to the geodesic deviation equation. The factor of 2 in the relativistic case arises from the additional gravitational effects predicted by general relativity, which are not present in the Newtonian theory. This factor effectively doubles the strength of the gravitational field in the relativistic case, leading to a higher oscillation frequency.
In other words, the stronger gravitational field in the relativistic case causes the particles to be pulled closer together more quickly, leading to more rapid oscillations of their separation. Conversely, in the Newtonian case, the weaker gravitational field results in slower oscillations.
Therefore, at the pericentre, the separation \(D(t)\) in the relativistic case will be smaller than in the Newtonian case, assuming the same initial conditions. This implies that the effects of general relativity cause the particles to come closer together than they would under Newtonian gravity alone.

If we consider a distribution of test particles forming a ``dust star'', instead of only two particles,
in the Newtonian case, the density is uniform throughout the star. This is because the dust particles are not interacting with each other, and there are no forces acting on the particles other than gravity. Therefore, the density of the star is simply the total mass of the star divided by its volume, and it does not change with the radial distance from the center of the star.

In contrast, in the context of a relativistic dust star, the density is not uniform. The gravitational force is stronger near the center of the star, and this causes the dust particles to be drawn closer together, resulting in a higher density near the center of the star. This effect is enhanced by the relativistic effects of gravity, which cause the gravitational force to increase more rapidly with decreasing radial distance from the center of the star.

Therefore, at the pericentre, the density of a relativistic dust star is higher than that of a Newtonian dust star, assuming the same initial conditions. 

\subsection{The role of hydrodynamics}

We have seen that in the relativistic case, the gravitational force
is modified by corrections that become more significant as the
particles approach the black hole. These corrections include terms that depend
on the speed of the particles, which becomes larger as they get closer to the
black hole. If these particles were part of a star in a stellar interior, the
frame-dragging effect would help to stabilize the rotation of the star, while
the Lense-Thirring effect would cause a precession in the orientation of the
star's orbit around the center of mass. These effects would help to provide
additional stability to the structure of the star.

One could however argue that this is an approximation only valid for a
fictitious star made out of ``dust'' particles, and not representative of a
real star, where hydrodynamical phenomena might change the outcome.  We will
argue now that the test particle approach is fine for this study.

To calculate the time spent by the particles in the regime where 1PN and 1.5PN
post-Newtonian corrections dominate the evolution, we need to find the time at
which the Newtonian acceleration becomes of the same order as the 1PN and 1.5PN
corrections. At leading order, the Newtonian acceleration is given by

\begin{equation}
a_N = -\frac{GM}{r^2},
\end{equation}

\noindent 
where $r$ is the distance between the particles and $M$ is the mass of the
central black hole. The 1PN correction to the acceleration is given by
\citep{will2014}

\begin{equation}
a_{1PN} = -\frac{GM}{r^2}\left(\frac{3v^2}{c^2} + 2\right),
\end{equation}

\noindent 
and the 1.5PN correction is given by \citep{will2014}

\begin{equation}
a_{1.5PN} = \frac{2G^2M^2}{r^4c^4}(4v^4 + 12v^2c^2 - 3c^4).
\end{equation}

\noindent 
To find the associated timescale, we will use the distance between the
particles as $r$. Therefore, the acceleration at leading order becomes $a_N =
-GM/(0.3R_\odot)^2$, for an illustrative initial separation of $0.3R_\odot$. At
the point where the 1PN and 1.5PN corrections are of the same order as $a_N$,
we can write

\begin{equation}
a_{1PN} = a_{1.5PN} = \alpha \frac{GM}{(0.3R_\odot)^2},
\end{equation}

\noindent 
where $\alpha$ is a dimensionless constant that depends on the velocity of the particles. Solving for $r$, we find

\begin{equation}\label{eq:r_timescale}
r = \frac{\alpha^{1/2}}{2^{1/6}}\left(\frac{5GM}{3c^2}\right)^{1/3}.
\end{equation}

\noindent 
The time spent by the particles in the regime where post-Newtonian corrections
dominate can be found by dividing the distance in Eq. \eqref{eq:r_timescale} by
the velocity of the particles, $v$. Since we are interested in the time it
takes for the particles to reach periapsis, we can use the initial separation
between the particles, $0.3R_\odot$, as an estimate for the velocity.
Therefore, the timescale is given in seconds by

\begin{equation}\label{eq:time_scale}
t_{\rm scale} = \frac{r}{0.3R_\odot}\,\text{s}.
\end{equation}

\noindent 
We note that this is the time it takes for the particles to reach periapsis
assuming that they are moving at constant velocity. In reality they are not,
because they are approaching the periapsis and will be accelerating. Therefore,
we are being conservative about the estimate; in practise the time will be even
shorter but we can envisage this as an upper limit.

The hydrodynamical timescale is given by $\tau_{\rm hydro} \sim
\sqrt{{R_*^3}/({Gm_*})}$, where $R_*$ is the radius of the star and $m_*$ is its
mass. For a solar-type star, this timescale is on the order of $\tau_{\rm
hydro} \sim 1600$ seconds.  Comparing this timescale to the time spent in the
regime where the 1PN and 1.5PN corrections dominate, we can determine if the
hydrodynamical effects can be neglected during this time. The time spent in
this regime can be calculated by finding the time it takes for the terms
involving these corrections to become comparable to the Newtonian term. We can
write this time as:

\begin{align}
t_{\rm PN} = \frac{\gamma}{c^3}\frac{X^4 r_p^4}{GM} \left[\frac{3}{2} + \frac{7}{2}\cos(2\phi)\right], 
\label{eq:tpn}
\end{align}

\noindent 
where $m$ is the mass of the test particle, $D$ is the initial separation of
the test particles, and $\phi$ is the angle between the initial separation
vector and the spin vector of the black hole. To simplify the notation, we have
defined the dimensionless parameter $\gamma \equiv (11/8)(M/m)^{1/2}$, and we
have chosen a fraction $X$ of the pericentre distance $r_p$, such that $D =
Xr_p$.

\noindent 
To determine the maximum time spent in the post-Newtonian regime, we need to
find the maximum value of Eq. \eqref{eq:tpn} with respect to $\phi$. This
occurs when $\cos(2\phi) = 1$, so we obtain:

\begin{equation}
t_{\rm PN,max} = \frac{1}{c^3}\frac{X^4 r_p^4}{GM}\gamma \frac{5}{2} \label{eq:tpnmax}
\end{equation}

\noindent 
Using the values we calculated earlier for $r_p$ and $\gamma$, we can evaluate
the time spent by the particles in the regime where the 1PN and 1.5PN
post-Newtonian corrections become significant.  
From Eq.~\eqref{eq:tpn} and following \cite{Peters:1963ux}, the total amount of time spent in this 
regime is


\begin{align}
T_{\rm PN} & = \frac{5c^5r_p^4}{256G^2M^2(1-e^2)^{7/2}}\left(2 + 2e^2 + \frac{9}{4}(1+3e^2)\gamma \right.  \nonumber \\
           & \left. - 3(1+e^2)\beta - \frac{39}{8}(1+3e^2)\gamma^2 - \frac{3}{2}(1-e^2)\beta^2\right).
\end{align}


\noindent 
Substituting the values we obtained earlier for $r_p$ and $\gamma$, we get:

\begin{equation}
T_{\rm PN} = 3.72\times 10^{-4} \left(\frac{M_{\odot}}{M}\right)^2\text{ s}.
\end{equation}

\noindent 
This is much shorter than the hydrodynamical timescale of a typical star, which
is on the order of millions to billions of years, depending on the star's mass
and evolutionary stage \citep{cox1980principles}. Therefore, we can safely
neglect all hydrodynamical effects during the time in which the post-Newtonian
forces dominate.



\noindent 
However, we also need to consider the timescale for the star to readjust to a
perturbation caused by the gravitational interaction with the black hole. This
timescale is determined by the star's sound-crossing time, which is the time it
takes for a sound wave to travel across the star's diameter. For a typical
main-sequence star with a radius of $R_\odot \approx 6.96\times10^8$ m and a
sound speed of $c_s \approx 10^5$ m/s, the sound-crossing time is on the order
of a few minutes. Therefore, we need to ensure that the timescale for the
gravitational interaction with the black hole is longer than the star's
sound-crossing time.  For the specific example we have been considering, with
$M = 10^6M_{\odot}$ and $r_p = 2.84R_{\odot}$, we can calculate the ratio of
the post-Newtonian timescale to the sound-crossing time as:

\begin{equation}
\frac{T_{\rm PN}}{t_{\rm sc}} \approx 3.7\times 10^{-8},
\end{equation}

\noindent 
where $t_{\rm sc}$ is the sound-crossing time of the star. This confirms that
we can neglect hydrodynamical effects during the post-Newtonian regime.

\section*{More realistic stars}

Even if we just explained why conceptually it is fine to drop hydrodynamical
effects in our ``dust star'' thought experiment, let us try to understand what
a more realistic star would experiment so that we gain a bit more of intuition
about the tidal disruption allowing a core to survive.  Let us start with the
Tolman-Oppenheimer-Volkoff (TOV) equation for hydrostatic equilibrium in the
Newtonian approximation \citep[see e.g.][]{KW94},

\begin{equation}
\frac{dp}{dr} = -\frac{(p + \rho)(m + 4\pi r^3 p)}{r(r - 2m)},
\end{equation}

\noindent 
where \(p\) is the pressure, \(\rho\) is the density, \(m\) is the mass
enclosed within a radius \(r\), and \(r\) is the radial distance from the
center of the star.  To make the calculations simpler, we will consider a
homogeneous star, a gas blob, so that the mass enclosed within a radius \(r\)
is given by \(m(r) = \frac{4}{3}\pi r^3 \rho\). This equation describes the
balance between the gravitational force and the pressure gradient inside the
star. In the Newtonian case, the gravitational force is balanced by the
pressure gradient, which prevents the star from collapsing under its own
weight.

\noindent
Let us now include the 1PN and 1.5PN corrections to the TOV equation and solve
it to find the evolution of the density as the star approaches the pericentre.
The 1PN correction to the TOV equation is given by \citep[again, see][]{KW94}:

\begin{align}
\frac{dp}{dr} &= -\frac{(p + \rho)(m + 4\pi r^3 p)}{r(r - 2m)} - \nonumber \\
&  \frac{4\pi r^2 (9p + 5\rho + \rho \frac{dp}{dr}/p)}{r - 2m}.
\end{align}

\noindent 
This equation includes the additional pressure gradient due to the relativistic
correction to the gravitational force. The term \(\rho \frac{dp}{dr}/p\)
represents the change in density with respect to pressure.  Next, we need to
include the 1.5PN correction to the TOV equation. The 1.5PN correction accounts
for the effect of the black hole's spin on the gravitational force, and is 


\begin{align}
\frac{dp}{dr} & = -\frac{(p + \rho)(m + 4\pi r^3 p)}{r(r - 2m)} - \nonumber \\
&  \frac{4\pi r^2 (9p + 5\rho + \rho \frac{dp}{dr}/p)}{r - 2m} + \frac{4\pi r^2 (p + \rho)^2}{r - 2m}.
\end{align}

\noindent 
This equation includes the additional pressure gradient due to the relativistic
correction to the gravitational force, including the effect of the black hole's
spin. The term \((p + \rho)^2\) represents the square of the total energy
density in the star.

In order to derive the solution to find the evolution of the density as the
star approaches the pericentre, we need to find a solution to this system.
Even if we use a linear approximation for \(\rho(r)\), i.e. we assume that the
change in density is not large, the TOV equation with PN corrections is a
non-linear differential equation which is not straightforward to solve
analytically.  However, we can still gain some insight from the form of the
equation. The additional term, which is proportional to \((p + \rho)^2\),
suggests that the pressure and density inside the star increase due to the
relativistic effects. This is consistent with our earlier discussion that the
gravitational force increases as the star approaches the black hole, and the
pressure gradient must also increase to maintain hydrostatic equilibrium. 

\section{Conclusions}

In this work we have addressed the problem of TDEs being less luminous than
theoretically expected in the accretion disk model. We run a set of Newtonian
SPH simulations of an unbound star of one solar mass and a MBH of mass
$10^6\,M_{\odot}$ with penetration parameters ranging from $1.64$ to $12.05$.
We re-do the simulations with exactly the same $\beta$ parameters and initial
conditions but taking into account relativistic (post-Newtonian) corrections.
For this we consider two different sets of simulations - One which only
includes the first correction to periapsis shift in the equations of motion and
another which additionally takes into account the spin-orbit coupling
correction up to next-to-lowest order. For $\beta$ values starting at $\beta
\gtrapprox 2.25$, all relativistic simulations feature a surviving core of the
original star. The Newtonian simulations, however, do not. Only the lowest
value of $\beta=1.64$ in the Newtonian case displays a core which does not last
long bound to the original star. As a consequence, the fallback rates are lower
in the corresponding relativistic cases, and hence the luminosity is also
lower.  The deeper the TDE, the bigger the difference in luminosity between the
Newtonian and relativistic simulations. The effect of the spin only plays an
important role, as expected, for extremely deep penetration factors. This was
also noted by the work of \cite{GaftonRosswog2019}, who find that precession leads
to debris configurations which are absent in the Newtonian cases.

Moreover, in the relativistic cases the energy distribution is more spread out,
so that in each specific energy bin there is less matter.  Hence, the fall back
rate in every time step is lower; $dM /dt$ is closely related to $dM/dE$, with
$E$ the specific energy relative to the MBH. This can be seen in e.g. Fig. 3 of
\cite{EvansKochanek1989} and the work of \cite{RyuEtAl2020a,RyuEtAl2020b},
which shows that TDEs in the relativistic case has an energy distribution with
significant wings, as well as Fig. 2 of \cite{RyuEtAl2020c} and
\cite{RyuEtAl2020d} for a full disruption.  If $E$ is wider, $dM/dE$ will be
smaller and, thus, $dM/dt$ as well.  

The analysis has been primarily numerical but complemented by an analytical
investigation. For this, we examine the energy dynamics and the behavior of
geodesics in the context of a star approaching a black hole.  We considered the
distance between two geodesics, which represents the tidal deformation
experienced by the star as it approaches the black hole. We find that this
distance decreases as the star gets closer to the black hole, indicating that
the star is being stretched by the tidal forces. This stretching effect is a
key factor in the tidal disruption of the star.  We then extended our two-test
particle analysis to a ``dust star'', a simplified model of a star in which the
particles move along geodesics and there are no internal forces. We found that
the distance between the particles decreases as they approach the black hole,
confirming the stretching effect observed in the geodesic analysis. However, we
also found that this distance remains finite, indicating that the star does not
get completely disrupted but retains a core.

An important question in this dust star approach is what role plays
hydrodynamics.  For this, we first calculate the time spent in the interesting
regime; i.e. when the relativistic effects are important.  We found that these
corrections become significant at a certain distance from the black hole, which
we calculated using the Newtonian acceleration as a reference.  We then
considered the timescale for the star to traverse this regime, which we found
to be much shorter than the hydrodynamical timescale of a typical star. We then
considered the timescale for the star to readjust to a perturbation caused by
the gravitational interaction with the black hole. This timescale is determined
by the star's sound-crossing time, which is on the order of a few minutes for a
typical main-sequence star. We found that the timescale for the gravitational
interaction with the black hole is much shorter than the star's sound-crossing
time, confirming that we can neglect hydrodynamical effects during the
post-Newtonian regime.

Finally, we considered the effects of the black hole's gravity on the internal
structure of the star, using the Tolman-Oppenheimer-Volkoff (TOV) equation for
hydrostatic equilibrium. We included the 1PN and 1.5PN corrections to the TOV
equation and found that the pressure and density inside the star increase due
to the relativistic effects.  Our analytical findings confirm the numerical
results.  

Our results suggest that in Nature TDEs must have deeper penetration parameters
than previously thought to explain the observations. These orbits naturally
lead to the consequence of a reduced observed luminosity regardless of the
accretion disc, simply due to the fact that relativity allows a part of the
star to survive the disruption.  Future work should aim to solve the shocks
that occur during this process, in order to better understand the non-thermal
implications with the interstellar medium.

\section*{Acknowledgments}

The initial idea of this work was presented as a talk at the Al{\'a}jar meeting
in
2013\footnote{\href{http://astro-gr.org/alajar-meeting-2013-stellar-dynamics-growth-massive-black-holes/}{http://astro-gr.org/alajar-meeting-2013-stellar-dynamics-growth-massive-black-holes/}}
and later also as an invited talk at the workshop ``TDE17: Piercing the sphere
of
influence''\footnote{\href{https://www.ast.cam.ac.uk/meetings/2017/tde17.piercing.sphere.influence}{https://www.ast.cam.ac.uk/meetings/2017/tde17.piercing.sphere.influence}}
which took place in Cambridge. PAS thanks the participants and organisers for
encouraging him to publish the research and Enrico Ram{\'\i}rez-Ruiz, Pablo
Laguna, Zoltan Haiman, Julian Krolik, Nick Stone, Sterl Phinney, Ramesh
Narayan, and Stephan Rosswog for comments during the workshop. He is
particularly indebted to Julian Krolik for pointing him the publications by him
and his collaborators.

We acknowledge the funds from the ``European Union NextGenerationEU/PRTR'',
Programa de Planes Complementarios I+D+I (ref. ASFAE/2022/014).

\section*{Data Availability}    
Any data used in this analysis are available on reasonable request from the first author.

\begin{thebibliography}{64}
\expandafter\ifx\csname natexlab\endcsname\relax\def\natexlab#1{#1}\fi

\bibitem[{{Amaro-Seoane}(2018)}]{Amaro-SeoaneLRR2012}
{Amaro-Seoane} P., 2018, Living Reviews in Relativity, 21, 4

\bibitem[{{Amaro-Seoane} {et~al.}(2012){Amaro-Seoane}, {Miller}, \&
  {Kennedy}}]{Amaro-SeoaneMillerKennedy2012}
{Amaro-Seoane} P., {Miller} M.~C., {Kennedy} G.~F., 2012, MNRAS, 425, 2401

\bibitem[{{Arcavi} {et~al.}(2014){Arcavi}, {Gal-Yam}, {Sullivan}, {Pan},
  {Cenko}, {Horesh}, {Ofek}, {De Cia}, {Yan}, {Yang}, {Howell}, {Tal},
  {Kulkarni}, {Tendulkar}, {Tang}, {Xu}, {Sternberg}, {Cohen}, {Bloom},
  {Nugent}, {Kasliwal}, {Perley}, {Quimby}, {Miller}, {Theissen}, \&
  {Laher}}]{ArcaviEtAl2014}
{Arcavi} I., {Gal-Yam} A., {Sullivan} M., {Pan} Y.-C., {Cenko} S.~B., {Horesh}
  A., {Ofek} E.~O., {De Cia} A., {Yan} L., {Yang} C.-W., {Howell} D.~A., {Tal}
  D., {Kulkarni} S.~R., {Tendulkar} S.~P., {Tang} S., {Xu} D., {Sternberg} A.,
  {Cohen} J.~G., {Bloom} J.~S., {Nugent} P.~E., {Kasliwal} M.~M., {Perley}
  D.~A., {Quimby} R.~M., {Miller} A.~A., {Theissen} C.~A., {Laher} R.~R., 2014,
  ApJ, 793, 38

\bibitem[{{Baumgardt} {et~al.}(2018){Baumgardt}, {Amaro-Seoane}, \&
  {Sch{\"o}del}}]{BaumgardtEtAl2018}
{Baumgardt} H., {Amaro-Seoane} P., {Sch{\"o}del} R., 2018, A\&A, 609, A28

\bibitem[{{Blanchet} \& {Iyer}(2003)}]{BlanchetIyer03}
{Blanchet} L., {Iyer} B.~R., 2003, Classical and Quantum Gravity, 20, 755

\bibitem[{{Brem} {et~al.}(2013){Brem}, {Amaro-Seoane}, \&
  {Spurzem}}]{BremAmaro-SeoaneSpurzem2014}
{Brem} P., {Amaro-Seoane} P., {Spurzem} R., 2013, MNRAS, 434, 2999

\bibitem[{{Brenneman}(2013)}]{Brenneman2013}
{Brenneman} L., 2013, {Measuring the Angular Momentum of Supermassive Black
  Holes}

\bibitem[{{Carter} \& {Luminet}(1982)}]{CarterLuminet1982}
{Carter} B., {Luminet} J.~P., 1982, Nat, 296, 211

\bibitem[{{Carter} \& {Luminet}(1983)}]{CarterLuminet1983}
---, 1983, A\&A, 121, 97

\bibitem[{{Chandrasekhar}(1942)}]{Chandra42}
{Chandrasekhar} S., 1942, Physical Sciences Data

\bibitem[{{Chornock} {et~al.}(2014){Chornock}, {Berger}, {Gezari}, {Zauderer},
  {Rest}, {Chomiuk}, {Kamble}, {Soderberg}, {Czekala}, {Dittmann}, {Drout},
  {Foley}, {Fong}, {Huber}, {Kirshner}, {Lawrence}, {Lunnan}, {Marion},
  {Narayan}, {Riess}, {Roth}, {Sanders}, {Scolnic}, {Smartt}, {Smith},
  {Stubbs}, {Tonry}, {Burgett}, {Chambers}, {Flewelling}, {Hodapp}, {Kaiser},
  {Magnier}, {Martin}, {Neill}, {Price}, \& {Wainscoat}}]{ChornockEtAl2014}
{Chornock} R., {Berger} E., {Gezari} S., {Zauderer} B.~A., {Rest} A., {Chomiuk}
  L., {Kamble} A., {Soderberg} A.~M., {Czekala} I., {Dittmann} J., {Drout} M.,
  {Foley} R.~J., {Fong} W., {Huber} M.~E., {Kirshner} R.~P., {Lawrence} A.,
  {Lunnan} R., {Marion} G.~H., {Narayan} G., {Riess} A.~G., {Roth} K.~C.,
  {Sanders} N.~E., {Scolnic} D., {Smartt} S.~J., {Smith} K., {Stubbs} C.~W.,
  {Tonry} J.~L., {Burgett} W.~S., {Chambers} K.~C., {Flewelling} H., {Hodapp}
  K.~W., {Kaiser} N., {Magnier} E.~A., {Martin} D.~C., {Neill} J.~D., {Price}
  P.~A., {Wainscoat} R., 2014, ApJ, 780, 44

\bibitem[{Cox(1980)}]{cox1980principles}
Cox J.~P., 1980, Principles of Stellar Structure. Gordon and Breach Science
  Publishers

\bibitem[{{Evans} \& {Kochanek}(1989)}]{EvansKochanek1989}
{Evans} C.~R., {Kochanek} C.~S., 1989, ApJ Lett., 346, L13

\bibitem[{{Faye} {et~al.}(2006){Faye}, {Blanchet}, \&
  {Buonanno}}]{FayeEtAl2006}
{Faye} G., {Blanchet} L., {Buonanno} A., 2006, Ph. Rv. D., 74, 104033

\bibitem[{{Frank} \& {Rees}(1976)}]{FR76}
{Frank} J., {Rees} M.~J., 1976, MNRAS, 176, 633

\bibitem[{{Freitag} \& {Benz}(2002)}]{FB02b}
{Freitag} M., {Benz} W., 2002, A\&A, 394, 345

\bibitem[{{Freitag} \& {Benz}(2005)}]{FB05}
---, 2005, MNRAS, 358, 1133

\bibitem[{{Fulbright} {et~al.}(1995){Fulbright}, {Benz}, \&
  {Davies}}]{FulbrightEtAl1995}
{Fulbright} M.~S., {Benz} W., {Davies} M.~B., 1995, \apj, 440, 254

\bibitem[{{Gafton} \& {Rosswog}(2019)}]{GaftonRosswog2019}
{Gafton} E., {Rosswog} S., 2019, MNRAS, 487, 4790

\bibitem[{{Gallego-Cano} {et~al.}(2018){Gallego-Cano}, {Sch{\"o}del}, {Dong},
  {Nogueras-Lara}, {Gallego-Calvente}, {Amaro-Seoane}, \&
  {Baumgardt}}]{Gallego-CanoEtAl2018}
{Gallego-Cano} E., {Sch{\"o}del} R., {Dong} H., {Nogueras-Lara} F.,
  {Gallego-Calvente} A.~T., {Amaro-Seoane} P., {Baumgardt} H., 2018, A\&A, 609,
  A26

\bibitem[{{Gezari} {et~al.}(2012){Gezari}, {Chornock}, {Rest}, {Huber},
  {Forster}, {Berger}, {Challis}, {Neill}, {Martin}, {Heckman}, {Lawrence},
  {Norman}, {Narayan}, {Foley}, {Marion}, {Scolnic}, {Chomiuk}, {Soderberg},
  {Smith}, {Kirshner}, {Riess}, {Smartt}, {Stubbs}, {Tonry}, {Wood-Vasey},
  {Burgett}, {Chambers}, {Grav}, {Heasley}, {Kaiser}, {Kudritzki}, {Magnier},
  {Morgan}, \& {Price}}]{GezariEtAl2012}
{Gezari} S., {Chornock} R., {Rest} A., {Huber} M.~E., {Forster} K., {Berger}
  E., {Challis} P.~J., {Neill} J.~D., {Martin} D.~C., {Heckman} T., {Lawrence}
  A., {Norman} C., {Narayan} G., {Foley} R.~J., {Marion} G.~H., {Scolnic} D.,
  {Chomiuk} L., {Soderberg} A., {Smith} K., {Kirshner} R.~P., {Riess} A.~G.,
  {Smartt} S.~J., {Stubbs} C.~W., {Tonry} J.~L., {Wood-Vasey} W.~M., {Burgett}
  W.~S., {Chambers} K.~C., {Grav} T., {Heasley} J.~N., {Kaiser} N., {Kudritzki}
  R.-P., {Magnier} E.~A., {Morgan} J.~S., {Price} P.~A., 2012, Nat, 485, 217

\bibitem[{{Gezari} {et~al.}(2003){Gezari}, {Halpern}, {Komossa}, {Grupe}, \&
  {Leighly}}]{GezariEtal03}
{Gezari} S., {Halpern} J.~P., {Komossa} S., {Grupe} D., {Leighly} K.~M., 2003,
  ApJ, 592, 42

\bibitem[{{Guillochon} \& {Ramirez-Ruiz}(2013)}]{GuillochonRamirez-Ruiz2013}
{Guillochon} J., {Ramirez-Ruiz} E., 2013, ApJ, 767, 25

\bibitem[{{Hills}(1975)}]{Hills75}
{Hills} J.~G., 1975, Nat, 254, 295

\bibitem[{{Holoien} {et~al.}(2014){Holoien}, {Prieto}, {Bersier}, {Kochanek},
  {Stanek}, {Shappee}, {Grupe}, {Basu}, {Beacom}, {Brimacombe}, {Brown},
  {Davis}, {Jencson}, {Pojmanski}, \& {Szczygie{\l}}}]{HoloienEtAl2014}
{Holoien} T.~W.~S., {Prieto} J.~L., {Bersier} D., {Kochanek} C.~S., {Stanek}
  K.~Z., {Shappee} B.~J., {Grupe} D., {Basu} U., {Beacom} J.~F., {Brimacombe}
  J., {Brown} J.~S., {Davis} A.~B., {Jencson} J., {Pojmanski} G.,
  {Szczygie{\l}} D.~M., 2014, MNRAS, 445, 3263

\bibitem[{{Ivanov} \& {Chernyakova}(2006)}]{IvanovChernyakova2006}
{Ivanov} P.~B., {Chernyakova} M.~A., 2006, A\&A, 448, 843

\bibitem[{{Kesden}(2012)}]{Kesden2012}
{Kesden} M., 2012, Phys. Rev. D, 85, 024037

\bibitem[{{Khabibullin} {et~al.}(2014){Khabibullin}, {Sazonov}, \&
  {Sunyaev}}]{KhabibullinEtAl2014}
{Khabibullin} I., {Sazonov} S., {Sunyaev} R., 2014, MNRAS, 437, 327

\bibitem[{{Kippenhahn} \& {Weigert}(1994)}]{KW94}
{Kippenhahn} R., {Weigert} A., 1994, Stellar Structure and Evolution.
  Springer-Verlag Berlin Heidelberg

\bibitem[{{Krolik} {et~al.}(2020){Krolik}, {Piran}, \& {Ryu}}]{KrolikEtAl2020}
{Krolik} J., {Piran} T., {Ryu} T., 2020, arXiv e-prints, arXiv:2001.03234

\bibitem[{{Kupi} {et~al.}(2006){Kupi}, {Amaro-Seoane}, \&
  {Spurzem}}]{KupiEtAl06}
{Kupi} G., {Amaro-Seoane} P., {Spurzem} R., 2006, MNRAS, L77+

\bibitem[{{Lai} {et~al.}(1993){Lai}, {Rasio}, \& {Shapiro}}]{LaiEtAl1993}
{Lai} D., {Rasio} F.~A., {Shapiro} S.~L., 1993, ApJ, 412, 593

\bibitem[{{Li} {et~al.}(2002){Li}, {Narayan}, \& {Menou}}]{LiEtAl2002}
{Li} L.-X., {Narayan} R., {Menou} K., 2002, ApJ, 576, 753

\bibitem[{{Magorrian} \& {Tremaine}(1999)}]{MT99}
{Magorrian} J., {Tremaine} S., 1999, MNRAS, 309, 447

\bibitem[{{Miles} {et~al.}(2020){Miles}, {Coughlin}, \&
  {Nixon}}]{MilesEtAl2020}
{Miles} P.~R., {Coughlin} E.~R., {Nixon} C.~J., 2020, arXiv e-prints,
  arXiv:2006.09375

\bibitem[{{Murphy} {et~al.}(1991){Murphy}, {Cohn}, \& {Durisen}}]{MCD91}
{Murphy} B.~W., {Cohn} H.~N., {Durisen} R.~H., 1991, ApJ, 370, 60

\bibitem[{{Nandra} {et~al.}(2013){Nandra}, {Barret}, {Barcons}, {Fabian}, {den
  Herder}, {Piro}, {Watson}, {Adami}, {Aird}, {Afonso}, \&
  et~al.}]{Athena+Whitepaper}
{Nandra} K., {Barret} D., {Barcons} X., {Fabian} A., {den Herder} J.-W., {Piro}
  L., {Watson} M., {Adami} C., {Aird} J., {Afonso} J.~M., et~al., 2013, ArXiv
  e-prints

\bibitem[{{Noble} {et~al.}(2009){Noble}, {Krolik}, \& {Hawley}}]{NobleEtAl2009}
{Noble} S.~C., {Krolik} J.~H., {Hawley} J.~F., 2009, ApJ, 692, 411

\bibitem[{{Peters}(1964)}]{Peters64}
{Peters} P.~C., 1964, Physical Review, 136, 1224

\bibitem[{Peters \& Mathews(1963)}]{Peters:1963ux}
Peters P.~C., Mathews J., 1963, Physical Review, 131, 435

\bibitem[{{Piran} {et~al.}(2015){Piran}, {Svirski}, {Krolik}, {Cheng}, \&
  {Shiokawa}}]{PiranEtAl2015}
{Piran} T., {Svirski} G., {Krolik} J., {Cheng} R.~M., {Shiokawa} H., 2015, ApJ,
  806, 164

\bibitem[{{Price}(2007)}]{Price2007}
{Price} D.~J., 2007, Publications of the Astronomical Society of Australia, 24,
  159

\bibitem[{{Rasio}(2000)}]{Rasio2000}
{Rasio} F.~A., 2000, Progress of Theoretical Physics Supplement, 138, 609

\bibitem[{{Rees}(1988{\natexlab{a}})}]{Rees88}
{Rees} M.~J., 1988{\natexlab{a}}, Nat, 333, 523

\bibitem[{{Rees}(1988{\natexlab{b}})}]{Rees1988}
---, 1988{\natexlab{b}}, Nat, 333, 523

\bibitem[{{Reynolds}(2014)}]{Reynolds2014}
{Reynolds} C.~S., 2014, Space Science Reviews, 183, 277

\bibitem[{{Ryu} {et~al.}(2020{\natexlab{a}}){Ryu}, {Krolik}, {Piran}, \&
  {Noble}}]{RyuEtAl2020a}
{Ryu} T., {Krolik} J., {Piran} T., {Noble} S.~C., 2020{\natexlab{a}}, arXiv
  e-prints, arXiv:2001.03501

\bibitem[{{Ryu} {et~al.}(2020{\natexlab{b}}){Ryu}, {Krolik}, {Piran}, \&
  {Noble}}]{RyuEtAl2020b}
---, 2020{\natexlab{b}}, arXiv e-prints, arXiv:2001.03502

\bibitem[{{Ryu} {et~al.}(2020{\natexlab{c}}){Ryu}, {Krolik}, {Piran}, \&
  {Noble}}]{RyuEtAl2020c}
---, 2020{\natexlab{c}}, arXiv e-prints, arXiv:2001.03503

\bibitem[{{Ryu} {et~al.}(2020{\natexlab{d}}){Ryu}, {Krolik}, {Piran}, \&
  {Noble}}]{RyuEtAl2020d}
---, 2020{\natexlab{d}}, arXiv e-prints, arXiv:2001.03504

\bibitem[{{Sch{\"o}del} {et~al.}(2018){Sch{\"o}del}, {Gallego-Cano}, {Dong},
  {Nogueras-Lara}, {Gallego-Calvente}, {Amaro-Seoane}, \&
  {Baumgardt}}]{SchoedelEtAl2018}
{Sch{\"o}del} R., {Gallego-Cano} E., {Dong} H., {Nogueras-Lara} F.,
  {Gallego-Calvente} A.~T., {Amaro-Seoane} P., {Baumgardt} H., 2018, A\&A, 609,
  A27

\bibitem[{{Servin} \& {Kesden}(2017)}]{ServinKesden2017}
{Servin} J., {Kesden} M., 2017, Phys. Rev. D, 95, 083001

\bibitem[{Shapiro \& Teukolsky(1983)}]{shapiro1983black}
Shapiro S., Teukolsky S., 1983, Black holes, white dwarfs, and neutron stars:
  The physics of compact objects. Wiley-Interscience

\bibitem[{{Springel}(2005)}]{Springel2005}
{Springel} V., 2005, MNRAS, 364, 1105

\bibitem[{{Stone} {et~al.}(2020){Stone}, {Vasiliev}, {Kesden}, {Rossi},
  {Perets}, \& {Amaro-Seoane}}]{StoneEtAl2020}
{Stone} N.~C., {Vasiliev} E., {Kesden} M., {Rossi} E.~M., {Perets} H.~B.,
  {Amaro-Seoane} P., 2020, Space Science Reviews, 216, 35

\bibitem[{{Syer} \& {Ulmer}(1999)}]{SU99}
{Syer} D., {Ulmer} A., 1999, MNRAS, 306, 35

\bibitem[{{Tagoshi} {et~al.}(2001){Tagoshi}, {Ohashi}, \&
  {Owen}}]{TagoshiEtAl01}
{Tagoshi} H., {Ohashi} A., {Owen} B.~J., 2001, Ph. Rv. D., 63, 044006

\bibitem[{{Tejeda} {et~al.}(2017){Tejeda}, {Gafton}, {Rosswog}, \&
  {Miller}}]{TejedaEtAl2017}
{Tejeda} E., {Gafton} E., {Rosswog} S., {Miller} J.~C., 2017, MNRAS, 469, 4483

\bibitem[{{van Velzen} \& {Farrar}(2014)}]{vanVelzenEtAl2014}
{van Velzen} S., {Farrar} G.~R., 2014, ApJ, 792, 53

\bibitem[{{van Velzen} {et~al.}(2020){van Velzen}, {Gezari}, {Hammerstein},
  {Roth}, {Frederick}, {Ward}, {Hung}, {Cenko}, {Stein}, {Perley}, {Taggart},
  {Sollerman}, {Andreoni}, {Bellm}, {Brinnel}, {De}, {Dekany}, {Feeney},
  {Foley}, {Fremling}, {Giomi}, {Golkhou}, {Ho}, {Kasliwal}, {Kilpatrick},
  {Kulkarni}, {Kupfer}, {Laher}, {Mahabal}, {Masci}, {Nordin}, {Riddle},
  {Rusholme}, {Sharma}, {van Santen}, {Shupe}, \&
  {Soumagnac}}]{vanVelzenEtAl2020}
{van Velzen} S., {Gezari} S., {Hammerstein} E., {Roth} N., {Frederick} S.,
  {Ward} C., {Hung} T., {Cenko} S.~B., {Stein} R., {Perley} D.~A., {Taggart}
  K., {Sollerman} J., {Andreoni} I., {Bellm} E.~C., {Brinnel} V., {De} K.,
  {Dekany} R., {Feeney} M., {Foley} R.~J., {Fremling} C., {Giomi} M., {Golkhou}
  V.~Z., {Ho} A. Y.~Q., {Kasliwal} M.~M., {Kilpatrick} C.~D., {Kulkarni} S.~R.,
  {Kupfer} T., {Laher} R.~R., {Mahabal} A., {Masci} F.~J., {Nordin} J.,
  {Riddle} R., {Rusholme} B., {Sharma} Y., {van Santen} J., {Shupe} D.~L.,
  {Soumagnac} M.~T., 2020, arXiv e-prints, arXiv:2001.01409

\bibitem[{{Wang} \& {Merritt}(2004)}]{WM04}
{Wang} J., {Merritt} D., 2004, ApJ, 600, 149

\bibitem[{{Wheeler}(1971)}]{Wheeler1971}
{Wheeler} J., 1971, in Study Week on Nuclei of Galaxies, {O'Connell} D.~J.~K.,
  ed., p. 539

\bibitem[{Will(2018)}]{will2014}
Will C.~M., 2018, Theory and Experiment in Gravitational Physics, 2nd edn.
  Cambridge University Press

\bibitem[{{Zhou} {et~al.}(2020){Zhou}, {Liu}, {Komossa}, {Cao}, {Ho}, {Chen},
  \& {Li}}]{ZhouEtAl2020}
{Zhou} Z.~Q., {Liu} F.~K., {Komossa} S., {Cao} R., {Ho} L.~C., {Chen} X., {Li}
  S., 2020, arXiv e-prints, arXiv:2002.02267

\end{thebibliography}

\label{lastpage}
\end{document}
