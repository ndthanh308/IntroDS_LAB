
%%\documentclass[referee,sn-basic]{sn-jnl}% referee option is meant for double line spacing

%%=======================================================%%
%% to print line numbers in the margin use lineno option %%
%%=======================================================%%

%%\documentclass[lineno,sn-basic]{sn-jnl}% Basic Springer Nature Reference Style/Chemistry Reference Style

%%======================================================%%
%% to compile with pdflatex/xelatex use pdflatex option %%
%%======================================================%%

%%\documentclass[pdflatex,sn-basic]{sn-jnl}% Basic Springer Nature Reference Style/Chemistry Reference Style

%%\documentclass[sn-basic]{sn-jnl}% Basic Springer Nature Reference Style/Chemistry Reference Style
\documentclass[pdflatex,sn-mathphys]{sn-jnl}% Math and Physical Sciences Reference Style
%%\documentclass[sn-aps]{sn-jnl}% American Physical Society (APS) Reference Style
%%\documentclass[sn-vancouver]{sn-jnl}% Vancouver Reference Style
%%\documentclass[sn-apa]{sn-jnl}% APA Reference Style
%%\documentclass[sn-chicago]{sn-jnl}% Chicago-based Humanities Reference Style
%%\documentclass[sn-standardnature]{sn-jnl}% Standard Nature Portfolio Reference Style
%%\documentclass[default]{sn-jnl}% Default
%%\documentclass[default,iicol]{sn-jnl}% Default with double column layout




\jyear{2021}%

%% as per the requirement new theorem styles can be included as shown below
\theoremstyle{thmstyleone}%
\newtheorem{theorem}{Theorem}%  meant for continuous numbers
%%\newtheorem{theorem}{Theorem}[section]% meant for sectionwise numbers
%% optional argument [theorem] produces theorem numbering sequence instead of independent numbers for Proposition
\newtheorem{proposition}[theorem]{Proposition}% 
%%\newtheorem{proposition}{Proposition}% to get separate numbers for theorem and proposition etc.

\theoremstyle{thmstyletwo}%
\newtheorem{example}{Example}%
\newtheorem{remark}{Remark}%

\theoremstyle{thmstylethree}%
\newtheorem{definition}{Definition}%
\usepackage{siunitx}
\usepackage{verbatim}
\usepackage{setspace}

%%\unnumbered% uncomment this for unnumbered level heads
	\doublespacing
\usepackage{lineno}
 %\linenumbers

\begin{document}

\title[Lightwave Harmonic Mixer]{Lightwave Electronic Harmonic Frequency Mixing}
%On-chip Optical-field Up- and Down-sampling through a Nanoantenna Network


\author*[1]{\fnm{Matthew} \sur{Yeung}}\email{myeung@mit.edu}

\author[1,2]{\fnm{Lu-Ting} \sur{Chou}}
\author[1,3]{\fnm{Marco} \sur{Turchetti}}
\author[2]{\fnm{Shih-Hsuan} \sur{Chia}}  
\author[1]{\fnm{Karl K.} \sur{Berggren}}

\author*[1]{\fnm{Philip D.} \sur{Keathley}}\email{pdkeat2@mit.edu}

\affil*[1]{\orgdiv{Research Laboratory of Electronics}, \orgname{Massachusetts Institute of Technology}, \orgaddress{\street{77 Massachusetts Ave}, \city{Cambridge}, \postcode{02139}, \state{MA}, \country{USA}}}

\affil[2]{\orgdiv{Institute of Biophotonics}, \orgname{National Yang Ming Chiao Tung University}, \orgaddress{\street{Linong Street, Beitou District}, \city{Taipei City}, \postcode{112304}, \country{Taiwan}}}

\affil[3]{\orgdiv{IBM T.J. Watson Research Center}, \orgname{IBM Quantum}, \city{Yorktown Heights}, \postcode{10598}, \state{NY},  \country{USA}}


\abstract{Frequency mixers are fundamental building blocks in many electronic systems. They enable frequency conversion for signal detection and processing. While conventional electronic frequency mixers operate in the GHz or at best in the THz frequency range, a compact and scalable petahertz-scale electronic frequency mixer would enable practical field-resolved optical signal processing and readout without the need for nonlinear optical elements.  We used nanoscale antenna structures to experimentally demonstrate electronic harmonic mixing beyond 0.350 PHz. Through this harmonic mixing process, we demonstrate time-domain sampling of optical field waveforms spanning more than one octave of bandwidth. The high nonlinearity of the devices enabled field-resolved detection of spectral content outside of that contained within the local oscillator, greatly extending the range of detectable frequencies compared to conventional heterodyning techniques.  We anticipate these devices to provide compact, PHz-scale solid-state detection for integration with emerging technologies such as compact frequency combs, optical waveform synthesizers, and compact timing systems. Our work has important implications for various fields such as chemistry, physics, material science, and biology, where PHz signals exhibiting femtosecond-scale dynamics are of great interest for spectroscopic analysis or imaging.}

\keywords{integrated, lightwave electronics, optical, oscilloscope, ultrafast, spectroscopy}

\maketitle

\section{Introduction}\label{sec1}

Field-resolved optical detection techniques are of paramount importance to the advancement of ultrafast, light-based science and technology \cite{krauszAttosecondMetrologyElectron2014}.  By enabling the direct exploration of light-matter interaction dynamics, field-resolved optical detection is being increasingly used for the investigation of light-matter interactions in biological systems \cite{pupezaFieldresolvedInfraredSpectroscopy2020}, charge transfer dynamics in molecules and materials \cite{itoBuildupDephasingFloquet2023,dienstbierTracingAttosecondElectron2023, jinUltrafastDynamicsVan2018,reimannSubcycleObservationLightwavedriven2018,schmidTunableNonintegerHighharmonic2021,langerSymmetrycontrolledTemporalStructure2017}, near fields surrounding metasurfaces \cite{mamaikinElectricfieldresolvedNearinfraredMicroscopy2022}, higher-harmonic generation \cite{huangPolarizationControlIsolated2018,wangRolesPhotocarrierDoping2017,schubertSubcycleControlTerahertz2014,hohenleutnerRealtimeObservationInterfering2015}, and to inform the development of novel light sources based on optical waveform synthesis \cite{leskoSixoctaveOpticalFrequency2021,steinleitnerSinglecycleInfraredWaveform2022,wirthSynthesizedLightTransients2011}.  Optical field sampling provides the most direct field-resolved measurement of light-matter interaction dynamics in the time-domain, but has traditionally been limited to the terahertz to mid-infrared spectral regions (few to tens of terahertz) limiting its applicability.  

Significant efforts have been dedicated to extending optical field sampling techniques to the petahertz-scale to provide time-domain optical field readout without the need for spectral phase retrieval.  Notable techniques include electro-optic sampling (EOS) \cite{ridenteElectroopticCharacterizationSynthesized2022,ziminUltrabroadbandAllopticalSampling2022,keiberElectroopticSamplingNearinfrared2016}, tunneling ionization with a perturbation for the time-domain observation of an electric field (TIPTOE) \cite{parkDirectSamplingLight2018,blochlSpatiotemporalSamplingNearpetahertz2022,kimPetahertzOpticalOscilloscope2013,liuSingleshotMeasurementFewcycle2022}, and nonlinear photoconductive sampling in solids \cite{sederbergAttosecondOptoelectronicField2020,ziminPetahertzscaleNonlinearPhotoconductive2021}. Despite these efforts, such optical field sampling techniques operating at PHz frequencies often necessitate large pulse energies, undesirable for many materials \cite{borrego-varillasAttosecondSpectroscopyInvestigation2022} and organisms \cite{wangThreephotonNeuronalImaging2020}. Furthermore, these techniques typically rely on few-cycle carrier-envelope phase-stable pulses, requiring bulky and costly amplifiers operating at kHz repetition rates \cite{herbstRecentAdvancesPetahertz2022}. As a result, the practicality of petahertz-scale optical field sampling is severely limited. The ultimate desire is for a simple, inexpensive, easily fabricated device that can sense an optical field with sub-cycle time resolution: the optical equivalent of an off-the-shelf electronic oscilloscope.



% Giving background for closest related work
Recently, Bionta et al \cite{biontaOnchipSamplingOptical2021} experimentally demonstrated an on-chip sub-optical-cycle optical-field-driven tunneling device to measure degenerate (\textit{i.e.} gate and signal having the same central frequency) near-infrared fields down to 5 fJ using a Mach-Zender interferometer (MZI) using carrier-envelope phase (CEP) stable pulses.  While degenerate sampling was demonstrated in Ref.~\cite{biontaOnchipSamplingOptical2021}, non-degenerate (\textit{i.e.} gate and signal having different frequency) upsampling (\textit{i.e.} higher frequencies are sampled using a lower frequency) or downsampling (\textit{i.e.} lower frequencies are sampled using a higher frequency) was only shown in theory. Non-degenerate sampling would significantly broaden the available detection bandwidth, potentially enabling detectors capable of sampling fields spanning from the terahertz to the petahertz.  This increased bandwidth would allow for seamless amplitude- and phase-resolved characterization of nonlinear processes of interest, such as  solid-state higher harmonic generation \cite{wangRolesPhotocarrierDoping2017,keathleyUncoveringExtremeNonlinear2023}, coherent Raman scattering \cite{marksNonlinearInterferometricVibrational2004,kumarInvitedArticleComplex2018}, and multiphoton processes \cite{alvesChallengesProgressProspects2022,wenFutureChallengesHybrid2019,wangThreephotonNeuronalImaging2020,evansCoherentAntiStokesRaman2008}, without the need for nonlinear frequency conversion, spectral phase retrieval, or a spectrally-overlapped local oscillator reference.

In this work, we explore non-degenerate time-domain sampling of optical field waveforms using optical-field-driven tunneling from plasmonic nanoantennas.  An overview schematic is shown in Fig. \ref{fig:overview}.  We show that the nanoantenna devices can be conceptualized as petahertz-electronic harmonic frequency mixers as in Fig.~\ref{fig:overview}e and that the harmonic frequency mixing process enables sampling across octaves of bandwidth without the requirement of absolute carrier-envelop phase (CEP) stabilization or few-cycle pulse generation.  As an experimental demonstration, a non-CEP-stable, 10-cycle gate waveform having a central frequency of 0.177 PHz (1690 nm) was used to sample its second harmonic centered at 0.353 PHz (850 nm), further extending the known frequency limits of electronic harmonic frequency mixing for optical metrology~\cite{evensonExtensionAbsoluteFrequency1972}.  To demonstrate operation in the few-cycle limit without CEP locking and to experimentally investigate the temporal and spectral response of the mixing process and its dependence on the driving waveform, we used a broadband, few-cycle supercontinuum as the signal while driving the antennas with both a short (1.5-cycle) and long (10-cycle) gating waveform.  Accurate sampling was achieved in the few-cycle limit despite no CEP stabilization of the driving laser source.  In addition, we were able to observe bandwidth reduction and map the entire fundamental pass band of the sampling response under long-gate illumination when using the broadband supercontinuum source as the signal.

Given their demonstrated ability to operate as petahertz-electronic harmonic frequency mixers, we anticipate similar nanoantenna devices will be used as basic building blocks for electromagnetic signal detection and processing at optical frequencies.  


% Figure environment removed
 


\subsection{Principle of Sampling via Harmonic Mixing}\label{principles}
% Figure environment removed

Our approach shares similarities with the TIPTOE method \cite{parkDirectSamplingLight2018, choTemporalCharacterizationFemtosecond2019}, with the distinction that we used Fowler-Nordheim (FN) electron emission from the nanoantenna surface as our nonlinear medium.  This provides several key advantages.  First, the field enhancement and reduced work function of the antenna emitter (here taken to be gold) increase the field sensitivity by orders of magnitude relative to a gas-based medium, allowing for the use of relatively low-energy optical driving and sampling pulses.  Second, the geometry of the antenna can be changed to enable both full and half-wave rectification for control over the electronic frequency response of the devices \cite{ritzkowskyEngineeringFrequencyResponse2022}.  Third, the sub-wavelength nature of the devices removes the need for phase-matching and potentially provides spatial resolution.     

The sub-optical-cycle tunneling current is driven from the nanoantenna to a collector as seen in Fig. \ref{fig:cep_dependence}a. Note that for this work we always assume an asymmetric, diode-like configuration with a single antenna facing a flat collecting wire which provides a half-wave rectified response. A small additional signal (to be sampled), is superimposed and varied in time relative to the gate pulse. The time-averaged charge transferred across the nanoantenna gap is measured as a function of this delay.  The measured current is 
\begin{equation}
I(\tau) =  q\int_{-T_{rep}/2}^{T_{rep}/2} \Gamma [E_\mathrm{gate}(t-\tau)+E_\mathrm{signal}(t)] \,dt 
\end{equation}
 where $\Gamma$ is the Fowler-Nordheim equation. Provided that $E_\mathrm{signal}$ is sufficiently small, such that the electric field perturbation seen by the nanoantenna can be assumed to be linear and $E_\mathrm{gate}$ can be Taylor-expanded to the first order. This resulting expression is given by 
 \begin{equation}
 I(\tau) \approx  q\int_{-T_{rep}/2}^{T_{rep}/2} \Gamma[E_\mathrm{gate}(t-\tau)]+ [\frac{d\Gamma}{dE}\vert_{E_\mathrm{gate}(t-\tau)} \cdot E_\mathrm{signal}(t)] \,dt
 \end{equation}
 In this equation, the integral of the second term corresponds to the sampled electric field waveform.  This integral represents the small-signal cross-correlation between $\frac{d\Gamma}{dE}\vert_{E_\mathrm{gate}(t-\tau)}$ and $ E_\mathrm{signal}(t)$ which we denote as $I_{cc}(\tau)$.   Since this expression represents a cross-correlation, in general, the Fourier-transformed expression can be written as  
 $\tilde{I}_{cc}(\omega) \approx \mathcal{F}[\frac{d\Gamma}{dE}\vert_{E_\mathrm{gate}}]^* \cdot E_\mathrm{signal}(\omega) $ where $\tilde{H}(\omega) = \mathcal{F}[\frac{d\Gamma}{dE}\vert_{E_\mathrm{gate}}]^*$ is the full complex frequency response of the detector, which is plotted in Fig. \ref{fig:cep_dependence}b.  
 
Due to the high nonlinearity of the FN tunneling rate $\Gamma$ and rectified current response, $\tilde{H}(\omega)$ contains integer harmonics of $E_\mathrm{gate}$. As shown in Fig. \ref{fig:cep_dependence}b, integer harmonic frequency bands appear in the sampling response outside of the gate spectrum centered around $\omega_0$.  The appearance of these harmonics is a result of mixing between higher-order components of the electronic frequency response with higher-order signal fields. Thus, these devices can be conceptualized as electronic harmonic optical-frequency mixers (see left panel of Fig.~\ref{fig:overview}b). Due to the frequency mixing process, higher-order frequency components of the signal are shifted to the baseband (see right panel of Fig.~\ref{fig:overview}b), which enables highly broadband, phase-resolved readout of optical waveforms through standard electronic detection, and without the need for separate wave mixing using nonlinear optical elements which introduce phase-matching constraints and require larger driving pulse energies~\cite{ziminUltrabroadbandAllopticalSampling2022}.  We emphasize that while one can certainly obtain a mixing response via conventional heterodyning and homodyning using an $E^2$ detector (\textit{e.g.} a photodiode), $E^2$ nonlinearity without rectification is insufficient to measure a pulse's own phase function (\textit{i.e.} only spectral intensity and relative phase differences are provided) or any higher-order frequencies.   
 
Conceptualizing the devices as electronic optical frequency mixers also aids in describing important properties of their response.   In particular, it allows a transparent explanation of why absolute CEP locking of the gate and signal pulse is not a requirement for time-domain sampling, even for the case of signals comprised of higher-order harmonics, provided that they originate from the same laser source.   To illustrate why, consider the case of perfectly sinusoidal gate and signal functions, where the signal is a harmonic of the gate.  In this case, the gate becomes analogous to the local oscillator input of the mixer at central frequency $f_\text{LO}$, and the harmonic signal of the optical frequency input at central frequency $f_\text{OF} = k f_\text{LO}$ as represented in Fig.~\ref{fig:overview}e.  We can then represent $\frac{d\Gamma}{dE}\vert_{E_\mathrm{gate}(t-\tau)}$ as
 \begin{equation}
 \frac{d\Gamma}{dE}\vert_{E_\mathrm{gate}(t-\tau)} = h_0 + \frac{1}{2} \bigg ( \sum_{n = 1}^{\infty} \tilde{h}_n e^{i n \varphi} e^{i 2\pi n f_\text{LO}(t - \tau)} + \text{c.c.} \bigg ) 
 \end{equation}
 where $\varphi$ represents the absolute phase shift of the gate (LO), analogous to the CEP for the case of a pulsed gate.  Likewise, the signal is represented by 
 \begin{equation}
 E_\text{signal}(t) = \frac{1}{2} \tilde{a}_k e^{i k \varphi + \Delta\varphi} e^{i 2 \pi k f_\text{LO} t } + \text{c.c.}
 \end{equation}
where $\Delta\varphi$ represents any remaining phase difference in addition to $k\varphi$.  
The DC output response is then formed by the multiplication of conjugate and non-conjugate coefficients of $\tilde{h}_k$ and $\tilde{a}_k$ and is found to be
\begin{equation}
I_\text{cc}(\tau) = \frac{1}{4} \tilde{h}_k^{*}\tilde{a}_k e^{i 2 \pi k f_\text{LO} \tau} e^{i\Delta\varphi} + \text{c.c.} 
\end{equation}
Note that since the signal and current responses are both phase-locked to the local oscillator (gate), the absolute phase terms $\varphi$ always cancel and the harmonic mixing response is not sensitive to any fluctuations of the absolute phase $\varphi$. 

These behaviors translate directly to the case of pulsed gate and signal inputs.  A finite envelope of the gate pulse leads to a broadening of the harmonic pass bands.  When shifting the carrier-envelope phase of the gate, the phase of each passband in $\tilde{H}(\omega)$ increases as integer steps with the harmonic number as shown in Fig.~\ref{fig:cep_dependence}b.  These shifts are equal and opposite to those of optically-generated harmonic pulses meaning that so long as the source of harmonics and the nonlinear optical radiation originate from the same driving laser the sampled response does not depend on the absolute carrier-envelope phase.    
 
To further illustrate this absolute phase cancellation for the case of a pulsed gate and signal, we performed time-domain current cross-correlation (using Eqn. 1) simulations using a 10-cycle Gaussian pulse at 0.177 PHz as the gate and a 2-cycle signal at frequencies of various integer harmonic orders (details can be found in the Methods section). To investigate the effect of CEP on the sampled electric field waveform, we simultaneously increased the CEP of both the gate and signal pulses linearly from 0 to 2$\pi$ in individual simulations. We tracked the resulting sampling response for each simulation and found that, regardless of the CEP of the gate and signal pulses, the sampled electric field always had a CEP of 0, as shown in Fig. \ref{fig:cep_dependence}c. Next, we kept the same gate conditions (10-cycle Gaussian at 0.177 PHz) and changed the signal pulse to a two-cycle SHG, THG, and 4HG pulse, with the signal CEP being 2, 3, and 4 times the gate CEP, respectively. This resulted in a sampled electric field with a CEP of 0, as illustrated in Fig. \ref{fig:cep_dependence}d, e, f. These results can also be seen in the phase in the frequency domain in Fig. \ref{fig:cep_dependence}b, where the phase is constant. 

\section{Results and discussion}\label{sec2}
In this section, we present the results of the three types of measurements performed. 
First, we describe the degenerate optical field sampling experiments using 10-cycle near-infrared gate and signal waveforms both having a central frequency of 0.177 PHz as shown in Fig. \ref{fig:f_f} to illustrate the capability of multicycle sampling without the need for CEP stabilization.
We next discuss the demonstration of non-degenerate optical-field upsampling to demonstrate the capability of harmonic frequency mixing. We did this by using a 10-cycle 0.177 PHz gate to sample a signal pulse of a frequency of 0.353 PHz that was generated through the second harmonic generation of the gate waveform.
Finally, we demonstrate the sampling performance without CEP stabilization in the few-cycle limit and compare the frequency properties of the sampling response when gated using both a short, 1.5-cycle gate, and a long, 10-cycle gate.  To do this we used a broadband, self-compressed supercontinuum (SC) source with continuous spectral coverage from 0.13 to 0.35 PHz as the signal. 
%We thus were able to directly verify the broad  frequency response of the system even when multi-cycle gates were used. 
% Figure environment removed

%We first perform degenerate optical field sampling by illuminating the network of triangular nanoantennas which was fabricated by electron beam lithography on fused silica substrates with varying gap sizes between the triangle apex and collector wire as seen in
\subsection{Degenerate Sampling}
To perform degenerate optical field sampling with a 10-cycle (57 fs) 0.177 PHz (1690 nm) pulse (degenerate sampling of a 63-cycle pulse is also shown in SI Fig. \ref{fig:1030_sampling}), a network of triangular nanoantennas on fused silica substrates was illuminated with a commercial optical parametric amplifier, with varying gap sizes of 20-50 nm between the triangle apex and collector wire as depicted in Fig. \ref{fig:overview}b, c. The sampling was carried out using a gate and signal pulse pair generated with an MZI, with the pulse duration determined through frequency-resolved optical gating (FROG) measurements (See SI Fig. \ref{fig:f_f_schematic} for the measurement schematic, Fig. \ref{fig:fundamentalFrog} for FROG). 

We first confirmed that the devices were operated in the optical-field emission regime through the analysis of current scaling with intensity by measuring the output current as a function of the incident gate pulse energy (see SI Fig.~\ref{fig:currentVsPulseE}). It is worth noting that we observed a gradual decline in performance over time in most of our devices, resulting in the need to increase the input pulse energy to achieve the same current as the device was used over longer periods of time.  Similar effects have been noted and studied in past work and were attributed to small structural changes in the antennas under illumination resulting in reduced intensity enhancement~\cite{yangLightPhaseDetection2020}.   

Having confirmed operation in the optical-field emission regime, we then illuminated the antennas with the signal and gate pulse energies at 4.4 pJ (24 V/$\mu$m) and 5.4 nJ (0.85 V/nm), respectively. Note that the small-signal gain from the high nonlinearity of the optical-field emission response enables the sampling of signal pulse energies on the order of 1000$\times$ smaller than those of the gate.  The sampled electric field is shown in Fig. \ref{fig:f_f}a.  The measured optical period was 5.6 fs which matched the expected value for a frequency of 0.177 PHz. The pulse of the sampled field was 57 fs full-width at half maximum (FWHM), in close agreement with the 58 fs FWHM pulse duration retrieved FROG measurement (see SI Fig. \ref{fig:fundamentalFrog}c). The accuracy of the sampled field was further verified through spectral intensity and group delay analysis.  In Fig.~\ref{fig:f_f}b the squared modulus of the Fourier transformation of the sampled pulse (solid blue line) is shown to agree well with the spectrum as measured using a commercial grating-based indium gallium arsenide spectrometer (dashed blue line).  Also, the extracted group delay of the sampled pulse (solid orange line) is shown to be concave up and agrees well with the group delay retrieved from the FROG measurement (dashed orange line).  

We note that in these comparisons, we have ignored the impact of the antenna response.  To justify this, we simulated the electromagnetic response of the nanoantenna based on the experimentally fabricated geometry.  The field enhancement and phase are plotted as a function of frequency in Fig. \ref{fig:overview}c, d.
The results of the simulation show a nearly constant amplitude and group delay from 0.168 PHz to 0.191 PHz (1570 nm to 1784 nm), corresponding to the spectral extent of the signal pulse. The simulated field enhancement factor ranges from 17 to 22, and the group delay from -4.6$\times10^{-3}$ to -7.4$\times10^{-3}$ fs within this range, leading to negligible reshaping of the pulse.  


\subsection{Non-degenerate Sampling}

% Figure environment removed
In the previous section, we discussed degenerate sampling where the frequency of the gate matched that of the signal. Here, we demonstrate non-degenerate up-sampling of the second harmonic signal via harmonic mixing using the fundamental ($\omega$ = 0.177 PHz) as the gate.

We performed the non-degenerate up-sampling by adding lenses and a type 1 beta-barium borate ($\beta$-BaB2O4, BBO) crystal for generating the SHG (0.353 PHz) signal (see SI Fig. \ref{fig:f_2f_schematic} for measurement schematic). We illuminated the antennas on the device using 23 pJ (49 V/$\mu$m) for the SHG signal (after filtering the fundamental) and 4.9 nJ (0.80 V/nm) as the gate. Shown in Fig. \ref{fig:f_2f}a is the sampled SHG using the 10-cycle 0.177 PHz as a gate. The measured optical period was 2.8 fs matching the expected optical period for a center frequency of 0.353~ PHz.

To verify the accuracy of the sampled field, we performed FROG and use a commercial grating-based silicon spectrometer as a reference in the same way as for the case of degenerate sampling.  In Fig.~\ref{fig:f_2f}b we compare the corresponding squared modulus of the Fourier-transformed sampled fields (solid blue line) to the spectral intensity measured using the spectrometer (dashed blue line).  We can see in the lower-frequency portion of the frequency domain spectrum that no residual fundamental (0.177 PHz) was observed in the measurement. The pulse duration from our FROG measurement was found to be 49 fs FWHM which matches almost perfectly with the sampled pulse duration of 48 fs FWHM (see supplementary information Fig. \ref{fig:850Frog} for a comparison of FROG and sampled fields). The group delay of the sampled trace (solid orange curve) is concave down and matches relatively well with the FROG-retrieved group delay (dashed orange curve).  

As before, we ignored the amplitude and phase response of the nanoantenna.  Again looking at the spectral field enhancement and group delay of the antenna response as shown in Fig.~\ref{fig:overview}d, we see that over the spectral range of the signal from 0.333 PHz and 0.370 PHz (810 nm and 900 nm), the field enhancement amplitude and group delay were relatively constant, ranging from 8 to 12 and from -8.4 $\times10^{-3}$ fs radians to -2.6 $\times10^{-2}$ fs, respectively.  As for the degenerate case, this leads to negligible pulse reshaping in the time domain. 

There are several aspects of this measurement that are important to emphasize.  First, given that the second harmonic generation used to generate the signal was phase-locked to the second harmonic of the sampling response as noted in Section~1, there was no need for CEP stabilization of the gate.  Second, the nonlinearity of the mixing process enabled the phase-resolved, interferometric pulse readout without the need for optical generation of a spectrally-overlapped local oscillator as would typically be required for all-optical homodyning or heterodyning techniques~\cite{ziminUltrabroadbandAllopticalSampling2022}.  Third is that the direct cross-correlation output was an accurate representation of the signal pulse as retrieved using FROG without the need for any post-analysis, such as phase retrieval, despite the fact that the signal pulse was appreciably shorter than the gate (48~fs FWHM signal duration versus 60~fs FWHM gate pulse duration), and of a significantly higher carrier frequency.        
      


\subsection{Supercontinuum Sampling with Long and Short Gates}
% Figure environment removed

To demonstrate that CEP-stabilization is not required even in the few-cycle limit, we measured the sampling response  using a non-CEP-stable 1.5-cycle pulse as the gate and signal.  To further explore the spectral and temporal properties of the sampling response (and thus the electronic emission response), and investigate how these properties are influenced by the choice of gate pulse, we then measured the sampling response using the 10-cycle pulse as the gate while keeping the 1.5-cycle pulse as the signal. 

To start we used a compressed, 1.5-cycle supercontinuum source with a central frequency near 0.17 PHz having broad and continuous spectral coverage from below 0.13 to beyond 0.25 PHz as the gate and signal pulse. To generate the 1.5-cycle pulse, we used a commercially available large-mode-area photonic crystal fiber to generate a nearly transform-limited duration supercontinuum pulse at the fiber output based on soliton self-compression. The self-compression process significantly influences the direct output duration, which was predicted to be 6 fs based on generalized nonlinear Schrödinger (NLSE) simulation results. Additional details regarding the simulation and supercontinuum generation (SCG) process can be found in the methods section for further reference. To validate these findings, we conducted FROG measurements, which yielded a retrieved pulse duration of 9 fs (Fig. \ref{fig:scg}b and SI Fig. \ref{fig:scgFrog} for further details). Furthermore, using intensity autocorrelation (IAC) measurements, we observed an autocorrelation trace width of 13 fs FWHM. Accounting for a deconvolution factor of 1.54 and assuming a hyperbolic secant (sech) pulse shape, this corresponds to an estimated pulse duration of 8.4 fs FWHM (refer to SI Fig. \ref{fig:iac}). 

The short-gate-sampled trace was measured using the supercontinuum pulse as both gate and signal and is shown in Fig. \ref{fig:scg}a, with its spectrum shown in Fig.~\ref{fig:scg}c (blue curves). A gate pulse energy of 0.6 nJ (1.1 V/nm) was used for the gate pulse and a pulse energy of 80 fJ (37 V/$\mu$m) was used for the signal (see SI Fig. \ref{fig:scg_degenerate_schematic} for measurement schematic). The temporal shape of the short-gate-sampled trace is in good agreement with the retrieved FROG trace as shown in Fig.~\ref{fig:scg}b (for more details see SI Fig. \ref{fig:scgFrog}), with a central pulse duration of 8.5 fs FWHM (compare to 9 fs and 8.4 fs for the FROG and IAC results, respectively). Moreover, the measurement of the short-gate sampled trace using a non-CEP-stable laser highlights the mixing effect discussed earlier in Section 1. Since the CEP undergoes stochastic drift, one would anticipate substantial variations in the emitted current and resultant sampling response from pulse-to-pulse. However, these variations are averaged away as each sampling trace corresponds to integration over 1.8 $\times 10^7$ pulses, and the signal waveform is reproduced containing only the relative CEP difference as discussed in Section 1.  Our results demonstrate that the measurement is insensitive to shifts in the absolute CEP of the driving waveform in the few-cycle limit so long as the gate and signal exhibit relative CEP stability.


In Fig. \ref{fig:scg}b, we see that the spectrum of the short-gate-sampled trace spans from just below 0.16 PHz to roughly 0.24 PHz.  Near the central spectral peak, structure is observed with several peaks and more prominent dips at 0.167, 0.175, and 0.180 PHz. To understand the response, we compare the spectrum to the supercontinuum spectrum measured with a spectrometer (green curve), as well as a full simulation of the spectral response (purple curve) that incorporates our NLSE simulation result $\tilde{E}_\text{NLSE}(\omega)$, the sampling response $\tilde{H}_\text{SG}(\omega)$ (dashed blue curve) and the calculated antenna response $\tilde{H}_\text{pl}(\omega) = \tilde{E}_\text{tip}(\omega)/\tilde{E}_\text{inc}(\omega)$, where $\tilde{E}_\text{tip}$ is the excited field at the tip and $\tilde{E}_\text{inc}$ the field incident on the antenna.  The intensity enhancement at the antenna tip ${\vert \tilde{H}_\text{pl}(\omega) \vert}^2$ is shown as solid light blue curve for reference. We note that the NLSE simulation describes the details of the measured supercontinuum spectrum very accurately, with the only major difference being that the spectral structure near the center of the pulse is more pronounced as in our sampling result (for direct comparison, see SI Fig. \ref{fig:plasmonBroadening}).  

The spectral response of the short-gate-sampled trace generally agrees well with the full simulation.  While the spectral dips near 0.17 PHz are more pronounced in the sampled trace spectrum, their frequency locations align well with the NLSE simulation result.  Here $\tilde{H}_\text{SG}(\omega)$ was calculated assuming a perfect sech pulse with a peak field of 28 V/nm, corresponding to a field enhancement of 30.  We recognize that for the measured short-gate-sampled trace, the spectral cutoff at lower frequencies is a bit sharper than expected.  We attribute this discrepancy to two possible causes: (1) our limited scan range resulting in an underestimation of long-wavelength spectral components; and (2) weaker focusing of the long-wavelength components onto the nanoantenna detectors.  

To study the impact of the gate pulse on the sampling response, we next compare the case of the few-cycle gate to that using the 10-cycle pulse with a central frequency of 0.177 PHz as the gate while leaving the signal pulse fixed.  As touched on in Section~\ref{sec2}, the bandwidth of the integer harmonic peaks in $\tilde{H}(\omega)$ can be adjusted by varying the cycle count of the gate pulse. When the cycle count decreases (\textit{i.e.} the pulse duration is shorter), the frequency bandwidth of the pulse expands, leading to a wider bandwidth of the integer harmonic peaks in $\tilde{H}(\omega)$. Conversely, increasing the cycle count narrows the bandwidth of the integer harmonic peaks.  The field enhancement, nonlinearity of the emission response, and gate pulse waveform all contribute to impacting the precise shape of $\tilde{H}(\omega)$.  By maintaining the same broadband signal while altering the gate pulse waveforms, we can examine precisely how $\tilde{H}(\omega)$ impacts the resultant time and frequency content of the underlying electronic response, and how well our emission model captures this electronic response.   

The long-gate-sampled trace is shown in Fig. \ref{fig:scg}a with its spectrum in Fig.~\ref{fig:scg}c (solid orange curves).  For the measurements shown, we used a 6.6 nJ (0.94 V/nm) gate and a 2 pJ (52 V/$\mu$m) signal pulse.  Note that the peak field strengths of the gate and signal are comparable to those of the short-gate-sampled case when accounting for the increased pulse duration.  

Due to the longer gate pulse, a longer train of sub-cycle current bursts were generated causing the long-gate-sampled trace to have an increased pulse duration.  Along with the increased sampling response duration, we also see the expected bandwidth reduction in the spectral response shown in Fig. \ref{fig:scg}c.  While the central frequencies are faithfully retrieved, with a similar structure to the short-gate-sampled result, NLSE, and measured spectrum, there is a sharp dropoff moving towards higher and lower frequencies. We observe dips at 0.156, 0.186, and 0.190 PHz in the spectrum which is due to a combination of the gate pulse shape and localized fields at the tip. The dashed orange curve shows the sampling response ${\vert \tilde{H}_\text{LG}(\omega) \vert}^2$ of the long gate with a peak field of 30~V/nm and field enhancement of 32.  Through the simulation, we estimate the window of emitted current from the 10-cycle gate pulse to be roughly 32 fs under the conditions in the measurement, still significantly shorter than the original pulse duration of roughly 60~fs.  We emphasize that operating at lower peak field values or with materials having higher work function could further shorten this duration and increase the bandwidth response due to larger nonlinearities, at the cost of lower current yields and thus lower signal-to-noise ratios.

Lastly, to test the validity of our measurements and accuracy of our sampling response simulations, we took our experimentally sampled 10-cycle gate pulse from Fig. \ref{fig:f_f}a and the short-gate-sampled supercontinuum from Fig. \ref{fig:scg}a and used these to calculate the current cross-correlation $I_\text{cc}(\tau)$ using Eqn. 1.  We accounted for a peak gate field strength of 30~V/nm for the 10-cycle gate pulse and a signal to gate peak field ratio of 0.03. The simulated $I_\text{cc}(\tau)$ is shown in Fig. \ref{fig:scg}a, and its corresponding spectral intensity is shown in Fig. \ref{fig:scg}c (solid yellow curves). While certain details of the temporal tails of the calculated result are different in comparison to the long-gate-sampled measurement, the central lobe is faithfully reproduced.  The calculated response has a central lobe centered around t = 0 fs of duration 20 fs FWHM, whereas the measured long-gate-sampled supercontinuum has a central lobe duration of 18 fs FWHM. The calculated spectral response also reproduces the bandwidth limits and key features observed in the measurement, indicating that our simplified Fowler-Nordheim model provides a reasonable estimate of the underlying current response.   

\section{Concluding Remarks and Outlook}\label{sec4}
In this study, we used nanoantenna networks to demonstrate a broadband, on-chip electronic optical frequency harmonic mixer using optical-field-driven tunneling.  We showed how the harmonic frequency mixing process enables accurate optical field sampling of waveforms having central frequencies greater than that of the gate waveform using a commercial laser without the need for a few-cycle laser source or carrier-envelope phase locking.  In comparison to wave mixing in crystals, the optical-field-driven tunneling mechanism provides access to higher-order nonlinearities, and thus larger mixing bandwidths, while eliminating the need for phase-matching or a separate photodetection element.  Similar devices could be used to create compact and sensitive sampling optical oscilloscopes with bandwidths spanning multiple octaves.  We anticipate that such optical field oscilloscopes will provide needed time-domain detection tools that will help accelerate the development of ultrafast source technologies (e.g. compact frequency combs and optical waveform synthesizers), and enable new approaches to the investigation of nonlinear light-matter interactions. Beyond field sampling, electronic optical-frequency harmonic mixers could also be incorporated as fundamental components within future lightwave electronic systems for petahertz-scale communication and computation. 



\section{Methods}\label{sec5}
\subsection{Nanofabrication}
We started with 1 cm $\times$ 1 cm fused silica pieces (MTI Corp.) and cleaned them using piranha for 10 minutes prior to use. For the nanoantenna array fabrication, we spin-coated polymethyl methacrylate A2 (Microchem) at 2,750 revolutions per minute and baked at \SI{180}{\celsius} for 2 minutes. Afterward, DisCharge H2O X2 (DisChem Inc.) was spun at 3,000 revolutions per minute so that charging did not occur during the electron beam lithography write. The electron beam lithography was performed at 125 keV with a dose ranging from 4000-6000 $\mu$C/cm$^2$ with proximity effect correction. Development of the exposed polymethyl methacrylate samples was done at \SI{0}{\celsius} in a solution of 3:1 2-propanol to methyl isobutyl ketone for 50 seconds. Electron beam evaporation was performed at 2$\times$$10^{-6}$ Torr where we first deposited a 2 nm titanium adhesion layer, then 20 nm of gold. Lift-off was performed by submerging the samples in a \SI{65}{\celsius} solution of N-methyl pyrrolidone (Microchem) for one hour.

Contacts were made to the nanoantenna using photolithography. We spin coat nLOF 2035 at 3,000 revolutions per minute, then bake the resist at \SI{110}{\celsius} for 90 seconds. The exposure was performed using a maskless aligner with a wavelength of 375 nm and at a dose of 300 mJ/cm$^2$. After exposure, we do a post-exposure bake at \SI{110}{\celsius} for 90 seconds, then develop for 90 seconds in AZ726. We then use electron beam lithography to deposit 10 nm of a titanium adhesion layer and 50 nm of gold. Liftoff was performed at room temperature in a solution of acetone for at least 6 hours. The samples were ashed for 30 seconds, then mounted on a printed circuit board (PCB), and wire bonded.

\subsection{Laser and measurement Methods}
We used a LightConversion optical parametric amplifier pumped by a Yb:KGW laser pulse picked to 1 MHz (Cronus 3P) for our experiments. The idler output was compressed using a prism pair. The samples on the PCB were used as-is in ambient conditions and the output was connected to a trans-impedance amplifier with a gain of 1 V/nA (FEMTO). We measured the signal pulse-induced current through a lock-in measurement referenced by chopping the signal arm at $\sim$ 277 Hz. The x- and y-channels of the lock-in are output to an oscilloscope (Keysight) with a sampling rate of at least 25-50 kSa/s. Prior to sampling measurements, we illuminated the device with the gate pulse and ensured that the photocurrent remained constant for at least several minutes. To temporally control the delay between the signal and gate pulse, we placed the gate pulse on a closed-loop piezo stage (Piezosystem Jena) with $\pm$ 14 nm (0.05 fs) repeatability. 

In every measurement, we used neutral density (ND) filters to control the power of the signal and gate pulses. After recombining the two pulses, we placed a linear film polarizer with an extinction ratio of $\geq10^5$ (across all wavelengths) to ensure the two beams were horizontally polarized. Once collinearity was ensured between the gate and signal arm, both the signal and gate pulses were focused onto the nanoantenna chip through a reflective objective (Ealing). See SI Fig. \ref{fig:f_f_schematic} for the schematic. 

For frequency doubling, we used a type 1, 1.5 mm thick beta-barium borate ($\beta$-BaB$_2$O$_4$, BBO) crystal with $\theta$=24 $\phi$=90 to generate the second harmonic (SHG) of $\omega_{gate}$ = 0.177 PHz. The SHG (0.353 PHz) is filtered using a 0.207 PHz (1450 nm) high-frequency pass filter with ND 2 at 0.177 PHz along with a broadband achromatic half waveplate used to rotate the SHG polarization from vertical to horizontal, to match the gate pulse. Note that the integration time was crucial for accurate measurement of the higher frequencies.

For the non-degenerate supercontinuum measurement using a 10-cycle $\omega_{gate}$ = 0.177 PHz, we ensured that the phase induced by the optics was equivalent to when the degenerately sampled supercontinuum was measured. We also utilize the same device for degenerate and non-degenerate sampling.

For frequency-resolved optical gating (FROG) measurements, we used the RANA approach (see Ref. \cite{jafari100ReliableAlgorithm2019}) as a robust retrieval algorithm. The MATLAB code is available on the Trebino Group website (see \cite{trebinoMATLABCodeAvailable}).

We padded the retrieved waveforms with zeros before taking an FFT to improve spectral resolution.  This was justified as the spacing between consecutive pulses in time in the experiment is much larger (one microsecond) than the time window of the retrieved waveforms.

\subsection{Electromagnetic, Sampling, and Supercontinuum Generation Simulation}

For the field enhancement simulations, we used open-source finite-difference time-domain package (FDTD) PyMeep \cite{oskooiMeepFlexibleFreesoftware2010}. For the gold nanoantenna and silicon oxide substrate, we used the standard materials library included in the Python package. For setting up the FDTD conditions, we used periodic boundary conditions in the nanoantenna on a silicon oxide plane with perfectly matched layers in the direction of propagation of the plane-wave source to prevent multi-reflections affecting the simulation. To obtain realistic field enhancements, we ensured that the apex of the triangle had a radius of curvature of 10 nm, which we extracted through scanning electron microscopy of our devices.

In our sampling simulations, the Fowler-Nordheim equation, expressed as \newline $ \Gamma = \alpha \phi^{-1} E^2 \exp(-\beta \phi^{3/2}/E)$, was utilized. Here, $\alpha$ is 1.54$\times10^{-6}$ A eV V$^{-2}$, $\beta$ is 6.83 eV$^{-3/2}$V nm$^{-1}$, $\phi$ represents the work function (taken as 5.1 eV), and E denotes the electric field. With this equation, we numerically calculated the current cross-correlation by \[I_{CC}(\tau) \propto  \int \Gamma(E_\mathrm{gate}(t-\tau)+E_\mathrm{signal}(t)) \,dt \] using a ratio of signal to the gate of 0.03. 

For the simulated frequency responses $\tilde{H}(\omega)$ shown in Fig.~\ref{fig:scg}c, we used a sech centered at 0.17 PHz with a pulse duration of 8.5 fs with the same center frequency as the measured SCG and with a field enhancement of 30$\times$ for the short-gate-sampled SCG. For the long-gate-sampled SCG we used a 0.17 PHz Gaussian with a pulse duration of 60 fs and a field enhancement of 32$\times$.

We simulated the spectral broadening using a 1690-nm and 60-fs Gaussian pulse to evaluate the coupled pulse energy and fiber length for optimal spectral broadening and temporal compression based on a 12.2 $\mu$m endlessly single mode, large mode area photonic crystal fiber (NKT Photonics). The simulation solved a generalized nonlinear Schrödinger (NLSE) equation based on the split-step method \cite{dudleySupercontinuumGenerationOptical2010}, which considered the fiber dispersion, self-phase modulation, self-steepening, and the Raman response. We then experimentally controlled the fiber length to be 4 mm, coupling 100-nJ of pulse energy, and the corresponding measured spectrum was compared to the simulation using the same parameter.

Data and selected code used can be found in the following repositories: 
Data and selected code: https://github.com/qnngroup/manu-HarmonicMixer.git
FDTD of the nanoantenna, example simulation code is provided, https://github.com/qnngroup/pymeep-nanoantenna-simulator

\backmatter
\clearpage


\section*{Supplementary information}

\subsection{Limits of sampling a short signal pulse with a long gate pulse}
% Figure environment removed

Illuminating the nanoantennas with a gate pulse generates sub-optical-cycle current bursts, creating a comb-like frequency response with peaks centered at higher harmonics of the fundamental, with the bandwidth of these harmonics being dependent on the cycle count of the gate pulse. The device's limitations are determined by the gate frequency and pulse duration, with the latter being directly related to time resolution due to the electron burst's dependence on the optical period. To illustrate this, we performed the following simulations using a Gaussian 1-, 5-, and 10-cycle $\omega_\mathrm{gate}$ = 0.177 PHz and a single cycle $\omega_\mathrm{signal}$ = 0.353 PHz. Shown in Fig. \ref{fig:sim_diff_gate_sampling}a from top to bottom is a 1-cycle signal (2.9 fs full-width at half maximum, FWHM of the absolute modulus squared of the electric field) as a reference and the corresponding sampled electric field using a 1-, 5- and 10-cycle gate. When sampling is performed, the CEP of the signal does not change due to the harmonic mixing process as described in the main text. 

When we use a 1-cycle gate to measure a 1-cycle frequency-doubled signal, the sampled pulse duration is 2.9 fs, which is identical to the pulse duration of the signal. The corresponding Fourier-transformed amplitude is also shown in Fig. \ref{fig:sim_diff_gate_sampling}b and has a frequency bandwidth of $\Delta$f=0.125 PHz. As the cycle count of the gate pulse is increased to 5 cycles, it is seen that some of the frequencies are distorted in the time domain, resulting in a non-symmetric electric field waveform as shown in Fig. \ref{fig:sim_diff_gate_sampling}a, however, the sampled pulse duration is still 2.9 fs (see SI Fig. \ref{fig:sampledPulseDuration}). From the corresponding Fourier-transformed amplitude shown in Fig. \ref{fig:sim_diff_gate_sampling}b, we observe the expected frequency response from a 5-cycle gate, which has a bandwidth of 0.08 PHz, and the sampled spectrum is attenuated at the valleys centered at 0.263 PHz and 0.441 PHz. Finally, when the gate consists of 10 cycles, the non-symmetric electric field becomes even more distorted, resulting in a sampled pulse duration of 13.4 fs. The frequency response with a 10-cycle gate has a bandwidth of 0.04 PHz and causes the previous valleys, which were observed with a 5-cycle gate, to approach 0, resulting in a further broadening of the sampled electric field.

% Figure environment removed

In Fig. \ref{fig:sampledPulseDuration}, we show the difference in pulse duration, which is defined as the FWHM of the absolute modulus squared of the electric field. When a 1-cycle 1690 nm gate is used to sample a 1-cycle 850 nm signal, the pulse duration of the sample field is identical. When the 1-cycle gate is changed to a 5-cycle gate, the pulse sampled pulse duration is nearly the same but has two side lobes at $\pm$ 5.5 fs which have a normalized intensity that is 0.29. Lastly, when the gate cycle count is increased to 10, several additional side lobes appear at $\pm$ 5.5 fs, $\pm$ 11.4 fs, and $\pm$ 17.2 fs with normalized intensities of 0.73, 0.29, and 0.05, respectively.

% Figure environment removed

Shown in Fig. \ref{fig:sim_10cyc}a are two 10-cycle Gaussian pulses used as $\omega_\mathrm{gate}$, one with a carrier frequency corresponding to 0.353 PHz and another with 0.177 PHz. The corresponding Fowler-Nordheim electron emission rate for the two Gaussian pulses is shown in Fig. \ref{fig:sim_10cyc}b. Since there are several cycles, there are several bursts of electrons that are generated. The inset displays the main burst at t=0, which is modulated by the signal pulse, enabling small signal sampling. The current burst full-width at half maximum (FWHM) determines the time resolution and is  0.28 fs for $\omega_\mathrm{gate}$ = 0.353 PHz and 0.55 fs for $\omega_{gate}$ = 0.177 PHz. Lastly, the frequency response given the 10-cycle $\omega_\mathrm{gate}$ is shown in Fig. \ref{fig:sim_10cyc}c. Since we are using a 10-cycle pulse, there are multiple electron bursts in time, resulting in a reduction in the frequency bandwidth response. If we were using a single-cycle gate pulse, the frequency domain response would be the envelope shown in the black dashed lines. 

% Figure environment removed


Shown in Fig. \ref{fig:FE_dependence}a is the gate-dependent electron emission in time. The electron emission can be seen as the multiplication of a comb in time with the peaks separated depending on the device geometry. With an asymmetric design, the spacing is the optical period of the gate frequency. Mathematically, it can be described as the following: 
\begin{equation}
\begin{array}{cc}
     h(t) =f(t)\cdot g(t)  \\
     k(t) = h(t) \times i(t) 
\end{array} 
\end{equation} 
In time, an electronic comb (f) is generated and the amplitude of the comb follows an envelope function (g), resulting in h and is purely dependent on the gate-dependent localized electric fields at the tip. Then, h is convolved with the electron emission rate envelope with finite bandwidth (i) which is directly related to the optical period of the gate frequency.

Shown in Fig. \ref{fig:FE_dependence}b is the electron emission rate for 3 different field enhancements. When the field enhancement is small, the main electron emission burst centered at t = 0 plays the largest role in perturbative sampling. As the field enhancement increases by a factor of 10 (equivalently an intensity enhancement of 1000), the side electron emission bursts start to increase in amplitude and the smaller amplitude sub-optical-cycle oscillations in the electric field start to contribute to electron emission. 


\subsection{Supplementary information regarding the experiments}
% Figure environment removed

There are numerous papers demonstrating that such nanoantenna devices are operating in the optical field emission regime using 10s to 100s of pJ pulse energies with few-cycle pulses. To confirm our devices are operating in the Fowler-Nordheim regime with the 10-cycle $\omega$ = 0.177 PHz pulse, we measured the current output versus peak field shown in Fig. \ref{fig:currentVsPulseE} with a typical device that has never been used for sampling, denoted as "New Device", and a device which we used for more than 4 hours of illumination (denoted as "Used Device ($>$4 hr)").  We fit the low-pulse-energy section of the new device current vs peak field curve to the conventional multiphoton photoemission form, $E_{MP} \propto \vert F_0\vert^{N}$, where $F_0$ is the incident peak field. Using a gold nanostructure, we take the work function to be 5.1 eV with the incident photon energy to be 0.729 eV, so we expect N = 7, and experimentally, we find N = 7. 

The crossover from multiphoton photoemission to optical-field emission occurs at roughly 0.285 V/nm as shown in Fig. \ref{fig:currentVsPulseE}. We also show that after more than 4 hours of performing measurements on a device, the response slightly degrades and the field emission response settles into one that is well described by the Fowler-Nordheim tunneling rate as shown by the right curve in Fig.~\ref{fig:currentVsPulseE}.  The measured photocurrent and nonlinearity after long-term operation were sufficient to continue using the device for optical field sampling.

\subsubsection{Degenerate Sampling}

% Figure environment removed

For degenerate sampling of 0.177 PHz (1690 nm) as shown in SI. Fig. \ref{fig:f_f_schematic}, we used an MZI to generate a pulse pair. We used ND filters to attenuate the gate and signal pulses and place a chopper in the signal arm as a reference for the measurement of the current-induced change when the small signal perturbed the sub-optical cycle gate. After the pulse pair was recombined at the second beamsplitter, we used a linear polarizer with extinction coefficients greater than $10^5$ at 1690 nm.

% Figure environment removed

We performed frequency-resolved optical gating (FROG) measurements of the pulses in time to provide a reference to compare the sampled fields against.  In Fig. \ref{fig:fundamentalFrog}a, b we show the measured and retrieved spectrograms of the 10-cycle 1690 nm pulse used in the measurements. The degenerate sampled fields in Fig. \ref{fig:fundamentalFrog}c show relatively good agreement in the time domain with the FROG result.  In both traces a small side lobe is present in time near 65 fs. The retrieved pulse duration was 58 fs FWHM and the sampled pulse duration was 57 fs FWHM. Lastly, the frequency domain comparison between the sampled intensity, retrieved intensity, and the grating-based spectrometer is shown in Fig. \ref{fig:fundamentalFrog}d. The sampled group delay and retrieved group delay show very good agreement.  

% Figure environment removed

We also performed degenerate field sampling using a 63-cycle pulse with a center frequency of 0.291 PHz (1030 nm) using an air-cooled 6-watt LightConversion Carbide after characterizing the pulse using SHG FROG. To begin our measurement, we employed the built-in pulse picker to reduce the repetition rate down to 500 kHz from 1 MHz. We then directed the output of the laser through the MZI with a closed-loop linear piezo stage delay line (Smaract SLC2445-S with MCS2) with $\pm$ 40 nm ($\pm$ 0.13 fs) repeatability. Utilizing a pristine device, we measured the current as a function of the gate pulse energy. We proceeded with sampling once the total photocurrent reached approximately 20 pA (blue line in Fig. \ref{fig:1030_sampling}a), which is equivalent to 1.4 nJ (0.328 V/nm) for this specific device measured. After sampling, we remeasured the current as a function of the gate pulse energy and observed degradation (orange curve Fig. \ref{fig:1030_sampling}a) associated with long pulse durations and highly localized fields at the nanoantenna tips. We tried to obtain 20 pA of current after sampling, however, this specific device rapidly degraded at 2.3 nJ. Using a signal pulse energy of 0.02 nJ (38.6 V/um), we degenerately sampled the pulse as shown in Fig. \ref{fig:1030_sampling}b. 

In the top panel (1.76 ps x-axis range) of Fig. \ref{fig:1030_sampling}b, at $\pm$ 500 fs, small side lobes are seen. As we continue zooming in until the 10 fs x-axis range, the measured optical period is 3.4 fs and the expected optical period for a center frequency of 291 PHz is 3.4 fs. It is clear that the technique can be generalized beyond 10-cycle pulses, however, since the devices were not operating in the tunneling regime for these measurements, these results have been placed in the supplemental information and require future studies to better understand the mechanism.  

To verify the sampled field, we compared our measured results with SHG FROG. As seen in Fig. \ref{fig:1030_sampling}c, d is the measured and retrieved spectrograms of the 63-cycle 1030 nm pulse used in the measurements. The degenerate sampled fields in Fig. \ref{fig:1030_sampling}e show relatively good agreement in the time domain with the FROG result demonstrating that there is a small side lobe in time at $\pm$500 fs. The retrieved pulse duration is 218 fs and the sampled pulse duration is 216 fs. Lastly, shown in  Fig. \ref{fig:1030_sampling}f is the frequency domain comparison between the sampled intensity, retrieved intensity, and the grating-based spectrometer. The sampled group delay and retrieved group delay are also shown and show reasonable agreement.


\subsubsection{Non-degenerate Sampling}

% Figure environment removed

For non-degenerate sampling of 0.353 PHz (850 nm) using a 0.177 PHz (1690 nm) gate, we modified the previously shown setup by adding a telescope for the 1690 nm gate to control the beam spot size before focusing on the nanoantennas.  This modification was performed since when the SHG was generated and collimated using the type-1 1.5 mm thick BBO the beam spot was smaller in size before entering the objective. We also used a shortpass filter which was ND2 at wavelengths longer than 1450 nm as well as an achromatic half waveplate to rotate the SHG such that its polarization matched the gate pulse polarization. After the second beamsplitter, the pulse pair passed through a linear polarizer which had an extinction ratio that is $>10^5$ at 1690 nm and $>10^6$ at 850 nm.

% Figure environment removed

We performed FROG measurements on the SHG, and  the measured and retrieved spectrograms of the SHG of the 10-cycle 1690 nm pulse used in the measurements are shown in Fig. \ref{fig:850Frog}a, b. The non-degenerate sampled fields in Fig. \ref{fig:850Frog}c show relatively good agreement in the time domain with the FROG result. The retrieved pulse duration was 49 fs FWHM and the sampled pulse duration was 48 fs FWHM. Lastly, shown in  Fig. \ref{fig:850Frog}d is the frequency domain comparison between the sampled intensity, retrieved intensity, and the grating-based spectrometer. The sampled group delay and retrieved group delay also show very good agreement.


% Figure environment removed

To demonstrate non-degenerate downsampling and to show that lower frequencies that are not integer harmonics can still be sampled, we used a separate few-cycle supercontinuum source with details that can be found in \cite{putnamFewcycleCarrierEnvelopephasestable2019}. After supercontinuum generation, an SF10 prism pair was used to spatially disperse the beam.  After the prism pair, the Er pump was spatially filtered, and the soliton and dispersive wave contributions were spatially separated.  The higher-frequency dispersive wave portion of the supercontinuum was returned through the prism pair for compression and the removal of spatial chirp and was used as the gate.  We note that the lower-frequency soliton, which was used as the signal, was not returned through the prism pair leaving some residual spatial chirp. 

For the gate pulse, 19 pJ (0.21 V/nm at focus) of the 3-cycle (12 fs) dispersive wave (0.250 PHz, 1200 nm) was used.  For the signal, 30 pJ (0.15 V/nm at focus) of the 11-cycle (65 fs) soliton (0.166 PHz, 1600 nm) was used.  The sampling measurement configuration is shown in detail in Fig. \ref{fig:downsampling}a. Degenerate sampling of the dispersive wave was performed and is shown in Fig. \ref{fig:downsampling}b. Next, we performed non-degenerate downsampling of the soliton and measured an optical period of approximately 6.0 fs as expected as Fig. \ref{fig:downsampling}c. Lastly, the normalized Fourier-transformed time-domain sampled waveforms are shown in Fig. \ref{fig:downsampling}d.  This emphasizes the ability of the nonlinear mixing process to also sample lower frequency contributions at non-integer harmonics in the few-cycle regime without CEP locking.

\subsubsection{Supercontinuum Sampling}

For the supercontinuum measurements discussed in the main text, we pumped a commercially available endlessly single-mode, large-mode-area photonic crystal fiber with roughly 100 nJ of pulse energy to generate a supercontinuum spanning 0.13 to 0.35 PHz. This supercontinuum was then split and recombined using an identical pair of beamsplitters. Each arm had an ND filter with the same thickness of glass to control the pulse energy focused onto the nanoantennas. After the pulses were recombined, the pulse pair was passed through the same linear polarizer as in prior measurements. 


% Figure environment removed

% Figure environment removed

We performed FROG measurements on the SCG as shown in Fig. \ref{fig:scgFrog}.  Fig.~\ref{fig:scgFrog}a and b are the measured and retrieved spectrograms of the supercontinuum pulse used in the measurements. The degenerate sampled fields in Fig. \ref{fig:scgFrog}c show relatively good agreement with the pulses retrieved from the FROG measurements in the time domain. The retrieved pulse duration was 9 fs FWHM and the sampled pulse duration was 8.5 fs FWHM. Lastly, shown in  Fig. \ref{fig:scgFrog}d is the frequency domain comparison between the sampled intensity, retrieved intensity, simulated intensity spectrum, and the grating-based spectrometer. The sampled group delay and retrieved group delay are also shown and show very good agreement.

% Figure environment removed

For the interferometric autocorrelation measurements (IAC) (SI Fig. \ref{fig:iac}), we focused the light onto a 40 $\mu$m thick BBO and detected the IAC using a silicon photodiode after passing through a linear polarizer. We measured a duration of 13 fs FWHM, which corresponds to an 8.4-fs pulse width considering a deconvolution factor of 1.54 assuming a sech pulse shape. To validate the field sampling, we compared the measured IAC trace to the reconstructed IAC trace obtained using the sampled field shown in Fig. \ref{fig:scg}. The two traces share comparable features, both having a sub-two-cycle component in the center part of the IAC trace along with pronounced side lobes. The second-order dispersion-induced chirp was not observed to be significant in either trace. The IAC traces both show significant side lobes, which are explained by the long tails observed in the time domain of the sampled waveform.  These time-domain tails were also observed in the simulation of the nonlinear pulse propagation.  They are primarily induced by third-order dispersion accumulated during spectral broadening between normal and abnormal dispersion regimes as well as complex higher-order dispersion during nonlinear compression.

% Figure environment removed

We split the gate pulse into two arms such that one can be used to generate the exact continuum used for the degenerate sampling of the supercontinuum. We add a telescope on the gate arm to control the beam spot size difference before being focused on the nanoantennas. We kept the same ND filter and transmitted the beam through the same beamsplitter to ensure the pulses accumulates nearly the same phase as the degenerate supercontinuum sampling. After the two pulses recombine at BS2, they passed through the same linear polarizer as used before.


% Figure environment removed

We used the simulated supercontinuum fields in frequency and added the electromagnetically simulated gold field enhancement amplitude and phase to understand how the gold nanoantenna modified the supercontinuum. We add the gold field enhancement amplitude and phase into the frequency spectrum and normalized the squared modulus of the corresponding frequency spectrum which is shown in Fig. \ref{fig:plasmonBroadening}a. After incorporating the gold antenna response, we also looked at how the simulated frequency response plays a role in the simulated frequency spectrum. We observed that there is a reduction of intensity which is mainly attributed to the gate-dependent frequency response. Also, it can be seen that where the sampled (blue) and gold antenna intensity enhancement (teal) curves intersect at 0.217 PHz, the  sampled field starts to increase and is related to the intensity enhancement from the nanoantenna. Afterward, we took the inverse Fourier transform of the simulated frequency spectrum and normalized the squared modulus of the time-domain spectrum to understand how the gold nanoantenna affects the pulse duration which is shown in Fig. \ref{fig:plasmonBroadening}b. We observe that the gold nanoantenna does not significantly modify the supercontinuum in frequency or time since the devices are being operated away from their resonant wavelength.


\subsection{Regarding Alternative Materials}

% Figure environment removed

As these devices are based on resonant nanoantennas, we performed simulations using platinum, a metal that is not plasmonic, to demonstrate other nanoantenna materials could be used. It can be observed that toward DC frequencies (Fig. \ref{fig:FE_sim}a), the field enhancement converges. It can be easily seen in Fig. \ref{fig:FE_sim}b, where in the long wavelength field enhancement of the gold and platinum antennas converge, which is a characteristic of metal nanoantennas.


\section*{Acknowledgments}
We also graciously thank Prof. Sixian You for the use of the LightConversion Cronus 3P system used in this work. This work was carried out in part through the use of MIT.nano. We also graciously thank LightConversion for providing the 6-Watt Carbide used for testing degenerate sampling of a 63-cycle pulse. We would like to thank Matteo Castellani and Adina Bechofer for their comments and suggestions on the manuscript before submission. Finally, we would like to acknowledge the many helpful discussion and conversations with Felix Ritzkowsky regarding this work and the underlying topics.

This work was supported by the Air Force Office of Scientific Research under award number FA9550-18-1-0436. M. Y. Acknowledges support from the National Science Foundation Graduate Research Fellowship Program, Grant No. 1745302. L.-T. C acknowledges financial support from the Ministry of Education (R.O.C) for the Overseas Internship Program directed by Prof. Shih-Hsuan Chia at National Yang Ming Chiao Tung University and the National Science and Technology Council (R.O.C) for the doctoral fellowship program. S.-H. C acknowledges the support from the Young Scholar Fellowship Program by the National Science and Technology Council (R.O.C) under award number NSTC 112-2636-E-A49-004. The opinions and views expressed in this publication are from the authors and not necessarily from the National Science Foundation, the Air Force Office of Scientific Research, the Ministry of Education, or the National Science and Technology Council.


\section*{Declarations}

\begin{itemize}
\item Funding: This work was supported by the Air Force Office of Scientific Research under award number FA9550-18-1-0436. 
\item Conflict of interest/Competing interests: The authors do not declare any competing interests.
\item Availability of data and materials: Experimental data is provided on a public repository https://github.com/qnngroup/manu-HarmonicMixer.git
\item Code availability: Code used in this manuscript is provided on a public repository https://github.com/qnngroup/manu-HarmonicMixer.git. For the FDTD of the nanoantenna, an example simulation code is provided, https://github.com/qnngroup/pymeep-nanoantenna-simulator

\item Authors' contributions: M. Y., L. T. C., and P. D. K. conceived the experiments.  M. Y. designed the experiments along with contributions from L. T. C.  M. Y. performed the data analysis with input and contributions from P. D. K. and L. T. C. M. Y. performed the sampling and electromagnetic simulations with input from P. D. K. L. T. C. performed the retrieval and supercontinuum generation simulations with input from S. H. C. M. Y. and M. T. performed the device fabrication. M. Y. wrote the initial draft of the manuscript, with contributions in revision from K. K. B., L. T. C., and P. D. K.  Input was provided from all authors. The project was supervised by K. K. B. and P. D. K. 
\end{itemize}

\noindent

\bigskip


%% BioMed_Central_Bib_Style_v1.01

\begin{thebibliography}{44}
% BibTex style file: bmc-mathphys.bst (version 2.1), 2014-07-24
\ifx \bisbn   \undefined \def \bisbn  #1{ISBN #1}\fi
\ifx \binits  \undefined \def \binits#1{#1}\fi
\ifx \bauthor  \undefined \def \bauthor#1{#1}\fi
\ifx \batitle  \undefined \def \batitle#1{#1}\fi
\ifx \bjtitle  \undefined \def \bjtitle#1{#1}\fi
\ifx \bvolume  \undefined \def \bvolume#1{\textbf{#1}}\fi
\ifx \byear  \undefined \def \byear#1{#1}\fi
\ifx \bissue  \undefined \def \bissue#1{#1}\fi
\ifx \bfpage  \undefined \def \bfpage#1{#1}\fi
\ifx \blpage  \undefined \def \blpage #1{#1}\fi
\ifx \burl  \undefined \def \burl#1{\textsf{#1}}\fi
\ifx \doiurl  \undefined \def \doiurl#1{\url{https://doi.org/#1}}\fi
\ifx \betal  \undefined \def \betal{\textit{et al.}}\fi
\ifx \binstitute  \undefined \def \binstitute#1{#1}\fi
\ifx \binstitutionaled  \undefined \def \binstitutionaled#1{#1}\fi
\ifx \bctitle  \undefined \def \bctitle#1{#1}\fi
\ifx \beditor  \undefined \def \beditor#1{#1}\fi
\ifx \bpublisher  \undefined \def \bpublisher#1{#1}\fi
\ifx \bbtitle  \undefined \def \bbtitle#1{#1}\fi
\ifx \bedition  \undefined \def \bedition#1{#1}\fi
\ifx \bseriesno  \undefined \def \bseriesno#1{#1}\fi
\ifx \blocation  \undefined \def \blocation#1{#1}\fi
\ifx \bsertitle  \undefined \def \bsertitle#1{#1}\fi
\ifx \bsnm \undefined \def \bsnm#1{#1}\fi
\ifx \bsuffix \undefined \def \bsuffix#1{#1}\fi
\ifx \bparticle \undefined \def \bparticle#1{#1}\fi
\ifx \barticle \undefined \def \barticle#1{#1}\fi
\bibcommenthead
\ifx \bconfdate \undefined \def \bconfdate #1{#1}\fi
\ifx \botherref \undefined \def \botherref #1{#1}\fi
\ifx \url \undefined \def \url#1{\textsf{#1}}\fi
\ifx \bchapter \undefined \def \bchapter#1{#1}\fi
\ifx \bbook \undefined \def \bbook#1{#1}\fi
\ifx \bcomment \undefined \def \bcomment#1{#1}\fi
\ifx \oauthor \undefined \def \oauthor#1{#1}\fi
\ifx \citeauthoryear \undefined \def \citeauthoryear#1{#1}\fi
\ifx \endbibitem  \undefined \def \endbibitem {}\fi
\ifx \bconflocation  \undefined \def \bconflocation#1{#1}\fi
\ifx \arxivurl  \undefined \def \arxivurl#1{\textsf{#1}}\fi
\csname PreBibitemsHook\endcsname

%%% 1
\bibitem{krauszAttosecondMetrologyElectron2014}
\begin{barticle}
\bauthor{\bsnm{Krausz}, \binits{F.}},
\bauthor{\bsnm{Stockman}, \binits{M.I.}}:
\batitle{Attosecond metrology: From electron capture to future signal
  processing}.
\bjtitle{Nature Photon}
\bvolume{8}(\bissue{3}),
\bfpage{205}--\blpage{213}
(\byear{2014}).
\doiurl{10.1038/nphoton.2014.28}
\end{barticle}
\endbibitem

%%% 2
\bibitem{pupezaFieldresolvedInfraredSpectroscopy2020}
\begin{barticle}
\bauthor{\bsnm{Pupeza}, \binits{I.}},
\bauthor{\bsnm{Huber}, \binits{M.}},
\bauthor{\bsnm{Trubetskov}, \binits{M.}},
\bauthor{\bsnm{Schweinberger}, \binits{W.}},
\bauthor{\bsnm{Hussain}, \binits{S.A.}},
\bauthor{\bsnm{Hofer}, \binits{C.}},
\bauthor{\bsnm{Fritsch}, \binits{K.}},
\bauthor{\bsnm{Poetzlberger}, \binits{M.}},
\bauthor{\bsnm{Vamos}, \binits{L.}},
\bauthor{\bsnm{Fill}, \binits{E.}},
\bauthor{\bsnm{Amotchkina}, \binits{T.}},
\bauthor{\bsnm{Kepesidis}, \binits{K.V.}},
\bauthor{\bsnm{Apolonski}, \binits{A.}},
\bauthor{\bsnm{Karpowicz}, \binits{N.}},
\bauthor{\bsnm{Pervak}, \binits{V.}},
\bauthor{\bsnm{Pronin}, \binits{O.}},
\bauthor{\bsnm{Fleischmann}, \binits{F.}},
\bauthor{\bsnm{Azzeer}, \binits{A.}},
\bauthor{\bsnm{{\v Z}igman}, \binits{M.}},
\bauthor{\bsnm{Krausz}, \binits{F.}}:
\batitle{Field-resolved infrared spectroscopy of biological systems}.
\bjtitle{Nature}
\bvolume{577}(\bissue{7788}),
\bfpage{52}--\blpage{59}
(\byear{2020}).
\doiurl{10.1038/s41586-019-1850-7}
\end{barticle}
\endbibitem

%%% 3
\bibitem{itoBuildupDephasingFloquet2023}
\begin{barticle}
\bauthor{\bsnm{Ito}, \binits{S.}},
\bauthor{\bsnm{Sch{\"u}ler}, \binits{M.}},
\bauthor{\bsnm{Meierhofer}, \binits{M.}},
\bauthor{\bsnm{Schlauderer}, \binits{S.}},
\bauthor{\bsnm{Freudenstein}, \binits{J.}},
\bauthor{\bsnm{Reimann}, \binits{J.}},
\bauthor{\bsnm{Afanasiev}, \binits{D.}},
\bauthor{\bsnm{Kokh}, \binits{K.A.}},
\bauthor{\bsnm{Tereshchenko}, \binits{O.E.}},
\bauthor{\bsnm{G{\"u}dde}, \binits{J.}},
\bauthor{\bsnm{Sentef}, \binits{M.A.}},
\bauthor{\bsnm{H{\"o}fer}, \binits{U.}},
\bauthor{\bsnm{Huber}, \binits{R.}}:
\batitle{Build-up and dephasing of {{Floquet}}\textendash{{Bloch}} bands on
  subcycle timescales}.
\bjtitle{Nature}
\bvolume{616}(\bissue{7958}),
\bfpage{696}--\blpage{701}
(\byear{2023}).
\doiurl{10.1038/s41586-023-05850-x}
\end{barticle}
\endbibitem

%%% 4
\bibitem{dienstbierTracingAttosecondElectron2023}
\begin{barticle}
\bauthor{\bsnm{Dienstbier}, \binits{P.}},
\bauthor{\bsnm{Seiffert}, \binits{L.}},
\bauthor{\bsnm{Paschen}, \binits{T.}},
\bauthor{\bsnm{Liehl}, \binits{A.}},
\bauthor{\bsnm{Leitenstorfer}, \binits{A.}},
\bauthor{\bsnm{Fennel}, \binits{T.}},
\bauthor{\bsnm{Hommelhoff}, \binits{P.}}:
\batitle{Tracing attosecond electron emission from a nanometric metal tip}.
\bjtitle{Nature}
\bvolume{616}(\bissue{7958}),
\bfpage{702}--\blpage{706}
(\byear{2023}).
\doiurl{10.1038/s41586-023-05839-6}
\end{barticle}
\endbibitem

%%% 5
\bibitem{jinUltrafastDynamicsVan2018}
\begin{barticle}
\bauthor{\bsnm{Jin}, \binits{C.}},
\bauthor{\bsnm{Ma}, \binits{E.Y.}},
\bauthor{\bsnm{Karni}, \binits{O.}},
\bauthor{\bsnm{Regan}, \binits{E.C.}},
\bauthor{\bsnm{Wang}, \binits{F.}},
\bauthor{\bsnm{Heinz}, \binits{T.F.}}:
\batitle{Ultrafast dynamics in van der {{Waals}} heterostructures}.
\bjtitle{Nature Nanotech}
\bvolume{13}(\bissue{11}),
\bfpage{994}--\blpage{1003}
(\byear{2018}).
\doiurl{10.1038/s41565-018-0298-5}
\end{barticle}
\endbibitem

%%% 6
\bibitem{reimannSubcycleObservationLightwavedriven2018}
\begin{barticle}
\bauthor{\bsnm{Reimann}, \binits{J.}},
\bauthor{\bsnm{Schlauderer}, \binits{S.}},
\bauthor{\bsnm{Schmid}, \binits{C.P.}},
\bauthor{\bsnm{Langer}, \binits{F.}},
\bauthor{\bsnm{Baierl}, \binits{S.}},
\bauthor{\bsnm{Kokh}, \binits{K.A.}},
\bauthor{\bsnm{Tereshchenko}, \binits{O.E.}},
\bauthor{\bsnm{Kimura}, \binits{A.}},
\bauthor{\bsnm{Lange}, \binits{C.}},
\bauthor{\bsnm{G{\"u}dde}, \binits{J.}},
\bauthor{\bsnm{H{\"o}fer}, \binits{U.}},
\bauthor{\bsnm{Huber}, \binits{R.}}:
\batitle{Subcycle observation of lightwave-driven {{Dirac}} currents in a
  topological surface band}.
\bjtitle{Nature}
\bvolume{562}(\bissue{7727}),
\bfpage{396}--\blpage{400}
(\byear{2018}).
\doiurl{10.1038/s41586-018-0544-x}
\end{barticle}
\endbibitem

%%% 7
\bibitem{schmidTunableNonintegerHighharmonic2021}
\begin{barticle}
\bauthor{\bsnm{Schmid}, \binits{C.P.}},
\bauthor{\bsnm{Weigl}, \binits{L.}},
\bauthor{\bsnm{Gr{\"o}ssing}, \binits{P.}},
\bauthor{\bsnm{Junk}, \binits{V.}},
\bauthor{\bsnm{Gorini}, \binits{C.}},
\bauthor{\bsnm{Schlauderer}, \binits{S.}},
\bauthor{\bsnm{Ito}, \binits{S.}},
\bauthor{\bsnm{Meierhofer}, \binits{M.}},
\bauthor{\bsnm{Hofmann}, \binits{N.}},
\bauthor{\bsnm{Afanasiev}, \binits{D.}},
\bauthor{\bsnm{Crewse}, \binits{J.}},
\bauthor{\bsnm{Kokh}, \binits{K.A.}},
\bauthor{\bsnm{Tereshchenko}, \binits{O.E.}},
\bauthor{\bsnm{G{\"u}dde}, \binits{J.}},
\bauthor{\bsnm{Evers}, \binits{F.}},
\bauthor{\bsnm{Wilhelm}, \binits{J.}},
\bauthor{\bsnm{Richter}, \binits{K.}},
\bauthor{\bsnm{H{\"o}fer}, \binits{U.}},
\bauthor{\bsnm{Huber}, \binits{R.}}:
\batitle{Tunable non-integer high-harmonic generation in a topological
  insulator}.
\bjtitle{Nature}
\bvolume{593}(\bissue{7859}),
\bfpage{385}--\blpage{390}
(\byear{2021}).
\doiurl{10.1038/s41586-021-03466-7}
\end{barticle}
\endbibitem

%%% 8
\bibitem{langerSymmetrycontrolledTemporalStructure2017}
\begin{barticle}
\bauthor{\bsnm{Langer}, \binits{F.}},
\bauthor{\bsnm{Hohenleutner}, \binits{M.}},
\bauthor{\bsnm{Huttner}, \binits{U.}},
\bauthor{\bsnm{Koch}, \binits{S.W.}},
\bauthor{\bsnm{Kira}, \binits{M.}},
\bauthor{\bsnm{Huber}, \binits{R.}}:
\batitle{Symmetry-controlled temporal structure of high-harmonic carrier fields
  from a bulk crystal}.
\bjtitle{Nature Photon}
\bvolume{11}(\bissue{4}),
\bfpage{227}--\blpage{231}
(\byear{2017}).
\doiurl{10.1038/nphoton.2017.29}
\end{barticle}
\endbibitem

%%% 9
\bibitem{mamaikinElectricfieldresolvedNearinfraredMicroscopy2022}
\begin{barticle}
\bauthor{\bsnm{Mamaikin}, \binits{M.}},
\bauthor{\bsnm{Li}, \binits{Y.-L.}},
\bauthor{\bsnm{Ridente}, \binits{E.}},
\bauthor{\bsnm{Chen}, \binits{W.T.}},
\bauthor{\bsnm{Park}, \binits{J.-S.}},
\bauthor{\bsnm{Zhu}, \binits{A.Y.}},
\bauthor{\bsnm{Capasso}, \binits{F.}},
\bauthor{\bsnm{Weidman}, \binits{M.}},
\bauthor{\bsnm{Schultze}, \binits{M.}},
\bauthor{\bsnm{Krausz}, \binits{F.}},
\bauthor{\bsnm{Karpowicz}, \binits{N.}}:
\batitle{Electric-field-resolved near-infrared microscopy}.
\bjtitle{Optica, OPTICA}
\bvolume{9}(\bissue{6}),
\bfpage{616}--\blpage{622}
(\byear{2022}).
\doiurl{10.1364/OPTICA.454562}
\end{barticle}
\endbibitem

%%% 10
\bibitem{huangPolarizationControlIsolated2018}
\begin{barticle}
\bauthor{\bsnm{Huang}, \binits{P.-C.}},
\bauthor{\bsnm{{Hern{\'a}ndez-Garc{\'i}a}}, \binits{C.}},
\bauthor{\bsnm{Huang}, \binits{J.-T.}},
\bauthor{\bsnm{Huang}, \binits{P.-Y.}},
\bauthor{\bsnm{Lu}, \binits{C.-H.}},
\bauthor{\bsnm{Rego}, \binits{L.}},
\bauthor{\bsnm{Hickstein}, \binits{D.D.}},
\bauthor{\bsnm{Ellis}, \binits{J.L.}},
\bauthor{\bsnm{{Jaron-Becker}}, \binits{A.}},
\bauthor{\bsnm{Becker}, \binits{A.}},
\bauthor{\bsnm{Yang}, \binits{S.-D.}},
\bauthor{\bsnm{Durfee}, \binits{C.G.}},
\bauthor{\bsnm{Plaja}, \binits{L.}},
\bauthor{\bsnm{Kapteyn}, \binits{H.C.}},
\bauthor{\bsnm{Murnane}, \binits{M.M.}},
\bauthor{\bsnm{Kung}, \binits{A.H.}},
\bauthor{\bsnm{Chen}, \binits{M.-C.}}:
\batitle{Polarization control of isolated high-harmonic pulses}.
\bjtitle{Nature Photon}
\bvolume{12}(\bissue{6}),
\bfpage{349}--\blpage{354}
(\byear{2018}).
\doiurl{10.1038/s41566-018-0145-0}
\end{barticle}
\endbibitem

%%% 11
\bibitem{wangRolesPhotocarrierDoping2017}
\begin{barticle}
\bauthor{\bsnm{Wang}, \binits{Z.}},
\bauthor{\bsnm{Park}, \binits{H.}},
\bauthor{\bsnm{Lai}, \binits{Y.H.}},
\bauthor{\bsnm{Xu}, \binits{J.}},
\bauthor{\bsnm{Blaga}, \binits{C.I.}},
\bauthor{\bsnm{Yang}, \binits{F.}},
\bauthor{\bsnm{Agostini}, \binits{P.}},
\bauthor{\bsnm{DiMauro}, \binits{L.F.}}:
\batitle{The roles of photo-carrier doping and driving wavelength in high
  harmonic generation from a semiconductor}.
\bjtitle{Nat Commun}
\bvolume{8}(\bissue{1}),
\bfpage{1686}
(\byear{2017}).
\doiurl{10.1038/s41467-017-01899-1}
\end{barticle}
\endbibitem

%%% 12
\bibitem{schubertSubcycleControlTerahertz2014}
\begin{barticle}
\bauthor{\bsnm{Schubert}, \binits{O.}},
\bauthor{\bsnm{Hohenleutner}, \binits{M.}},
\bauthor{\bsnm{Langer}, \binits{F.}},
\bauthor{\bsnm{Urbanek}, \binits{B.}},
\bauthor{\bsnm{Lange}, \binits{C.}},
\bauthor{\bsnm{Huttner}, \binits{U.}},
\bauthor{\bsnm{Golde}, \binits{D.}},
\bauthor{\bsnm{Meier}, \binits{T.}},
\bauthor{\bsnm{Kira}, \binits{M.}},
\bauthor{\bsnm{Koch}, \binits{S.W.}},
\bauthor{\bsnm{Huber}, \binits{R.}}:
\batitle{Sub-cycle control of terahertz high-harmonic generation by dynamical
  {{Bloch}} oscillations}.
\bjtitle{Nature Photon}
\bvolume{8}(\bissue{2}),
\bfpage{119}--\blpage{123}
(\byear{2014}).
\doiurl{10.1038/nphoton.2013.349}
\end{barticle}
\endbibitem

%%% 13
\bibitem{hohenleutnerRealtimeObservationInterfering2015}
\begin{barticle}
\bauthor{\bsnm{Hohenleutner}, \binits{M.}},
\bauthor{\bsnm{Langer}, \binits{F.}},
\bauthor{\bsnm{Schubert}, \binits{O.}},
\bauthor{\bsnm{Knorr}, \binits{M.}},
\bauthor{\bsnm{Huttner}, \binits{U.}},
\bauthor{\bsnm{Koch}, \binits{S.W.}},
\bauthor{\bsnm{Kira}, \binits{M.}},
\bauthor{\bsnm{Huber}, \binits{R.}}:
\batitle{Real-time observation of interfering crystal electrons in
  high-harmonic generation}.
\bjtitle{Nature}
\bvolume{523}(\bissue{7562}),
\bfpage{572}--\blpage{575}
(\byear{2015}).
\doiurl{10.1038/nature14652}
\end{barticle}
\endbibitem

%%% 14
\bibitem{leskoSixoctaveOpticalFrequency2021}
\begin{barticle}
\bauthor{\bsnm{Lesko}, \binits{D.M.B.}},
\bauthor{\bsnm{Timmers}, \binits{H.}},
\bauthor{\bsnm{Xing}, \binits{S.}},
\bauthor{\bsnm{Kowligy}, \binits{A.}},
\bauthor{\bsnm{Lind}, \binits{A.J.}},
\bauthor{\bsnm{Diddams}, \binits{S.A.}}:
\batitle{A six-octave optical frequency comb from a scalable few-cycle erbium
  fibre laser}.
\bjtitle{Nat. Photonics}
\bvolume{15}(\bissue{4}),
\bfpage{281}--\blpage{286}
(\byear{2021}).
\doiurl{10.1038/s41566-021-00778-y}
\end{barticle}
\endbibitem

%%% 15
\bibitem{steinleitnerSinglecycleInfraredWaveform2022}
\begin{barticle}
\bauthor{\bsnm{Steinleitner}, \binits{P.}},
\bauthor{\bsnm{Nagl}, \binits{N.}},
\bauthor{\bsnm{Kowalczyk}, \binits{M.}},
\bauthor{\bsnm{Zhang}, \binits{J.}},
\bauthor{\bsnm{Pervak}, \binits{V.}},
\bauthor{\bsnm{Hofer}, \binits{C.}},
\bauthor{\bsnm{Hudzikowski}, \binits{A.}},
\bauthor{\bsnm{Sotor}, \binits{J.}},
\bauthor{\bsnm{Weigel}, \binits{A.}},
\bauthor{\bsnm{Krausz}, \binits{F.}},
\bauthor{\bsnm{Mak}, \binits{K.F.}}:
\batitle{Single-cycle infrared waveform control}.
\bjtitle{Nat. Photon.}
\bvolume{16}(\bissue{7}),
\bfpage{512}--\blpage{518}
(\byear{2022}).
\doiurl{10.1038/s41566-022-01001-2}
\end{barticle}
\endbibitem

%%% 16
\bibitem{wirthSynthesizedLightTransients2011}
\begin{barticle}
\bauthor{\bsnm{Wirth}, \binits{A.}},
\bauthor{\bsnm{Hassan}, \binits{M.{\relax Th}.}},
\bauthor{\bsnm{Grgura{\v s}}, \binits{I.}},
\bauthor{\bsnm{Gagnon}, \binits{J.}},
\bauthor{\bsnm{Moulet}, \binits{A.}},
\bauthor{\bsnm{Luu}, \binits{T.T.}},
\bauthor{\bsnm{Pabst}, \binits{S.}},
\bauthor{\bsnm{Santra}, \binits{R.}},
\bauthor{\bsnm{Alahmed}, \binits{Z.A.}},
\bauthor{\bsnm{Azzeer}, \binits{A.M.}},
\bauthor{\bsnm{Yakovlev}, \binits{V.S.}},
\bauthor{\bsnm{Pervak}, \binits{V.}},
\bauthor{\bsnm{Krausz}, \binits{F.}},
\bauthor{\bsnm{Goulielmakis}, \binits{E.}}:
\batitle{Synthesized {{Light Transients}}}.
\bjtitle{Science}
\bvolume{334}(\bissue{6053}),
\bfpage{195}--\blpage{200}
(\byear{2011}).
\doiurl{10.1126/science.1210268}
\end{barticle}
\endbibitem

%%% 17
\bibitem{ridenteElectroopticCharacterizationSynthesized2022}
\begin{barticle}
\bauthor{\bsnm{Ridente}, \binits{E.}},
\bauthor{\bsnm{Mamaikin}, \binits{M.}},
\bauthor{\bsnm{Altwaijry}, \binits{N.}},
\bauthor{\bsnm{Zimin}, \binits{D.}},
\bauthor{\bsnm{Kling}, \binits{M.F.}},
\bauthor{\bsnm{Pervak}, \binits{V.}},
\bauthor{\bsnm{Weidman}, \binits{M.}},
\bauthor{\bsnm{Krausz}, \binits{F.}},
\bauthor{\bsnm{Karpowicz}, \binits{N.}}:
\batitle{Electro-optic characterization of synthesized infrared-visible light
  fields}.
\bjtitle{Nat Commun}
\bvolume{13}(\bissue{1}),
\bfpage{1111}
(\byear{2022}).
\doiurl{10.1038/s41467-022-28699-6}
\end{barticle}
\endbibitem

%%% 18
\bibitem{ziminUltrabroadbandAllopticalSampling2022}
\begin{barticle}
\bauthor{\bsnm{Zimin}, \binits{D.A.}},
\bauthor{\bsnm{Yakovlev}, \binits{V.S.}},
\bauthor{\bsnm{Karpowicz}, \binits{N.}}:
\batitle{Ultra-broadband all-optical sampling of optical waveforms}.
\bjtitle{Science Advances}
\bvolume{8}(\bissue{51}),
\bfpage{1029}
(\byear{2022}).
\doiurl{10.1126/sciadv.ade1029}
\end{barticle}
\endbibitem

%%% 19
\bibitem{keiberElectroopticSamplingNearinfrared2016}
\begin{barticle}
\bauthor{\bsnm{Keiber}, \binits{S.}},
\bauthor{\bsnm{Sederberg}, \binits{S.}},
\bauthor{\bsnm{Schwarz}, \binits{A.}},
\bauthor{\bsnm{Trubetskov}, \binits{M.}},
\bauthor{\bsnm{Pervak}, \binits{V.}},
\bauthor{\bsnm{Krausz}, \binits{F.}},
\bauthor{\bsnm{Karpowicz}, \binits{N.}}:
\batitle{Electro-optic sampling of near-infrared waveforms}.
\bjtitle{Nature Photon}
\bvolume{10}(\bissue{3}),
\bfpage{159}--\blpage{162}
(\byear{2016}).
\doiurl{10.1038/nphoton.2015.269}
\end{barticle}
\endbibitem

%%% 20
\bibitem{parkDirectSamplingLight2018}
\begin{barticle}
\bauthor{\bsnm{Park}, \binits{S.B.}},
\bauthor{\bsnm{Kim}, \binits{K.}},
\bauthor{\bsnm{Cho}, \binits{W.}},
\bauthor{\bsnm{Hwang}, \binits{S.I.}},
\bauthor{\bsnm{Ivanov}, \binits{I.}},
\bauthor{\bsnm{Nam}, \binits{C.H.}},
\bauthor{\bsnm{Kim}, \binits{K.T.}}:
\batitle{Direct sampling of a light wave in air}.
\bjtitle{Optica, OPTICA}
\bvolume{5}(\bissue{4}),
\bfpage{402}--\blpage{408}
(\byear{2018}).
\doiurl{10.1364/OPTICA.5.000402}
\end{barticle}
\endbibitem

%%% 21
\bibitem{blochlSpatiotemporalSamplingNearpetahertz2022}
\begin{barticle}
\bauthor{\bsnm{Bl{\"o}chl}, \binits{J.}},
\bauthor{\bsnm{Sch{\"o}tz}, \binits{J.}},
\bauthor{\bsnm{Maliakkal}, \binits{A.}},
\bauthor{\bsnm{{\v S}reibere}, \binits{N.}},
\bauthor{\bsnm{Wang}, \binits{Z.}},
\bauthor{\bsnm{Rosenberger}, \binits{P.}},
\bauthor{\bsnm{Hommelhoff}, \binits{P.}},
\bauthor{\bsnm{Staudte}, \binits{A.}},
\bauthor{\bsnm{Corkum}, \binits{P.B.}},
\bauthor{\bsnm{Bergues}, \binits{B.}},
\bauthor{\bsnm{Kling}, \binits{M.F.}}:
\batitle{Spatiotemporal sampling of near-petahertz vortex fields}.
\bjtitle{Optica, OPTICA}
\bvolume{9}(\bissue{7}),
\bfpage{755}--\blpage{761}
(\byear{2022}).
\doiurl{10.1364/OPTICA.459612}
\end{barticle}
\endbibitem

%%% 22
\bibitem{kimPetahertzOpticalOscilloscope2013}
\begin{barticle}
\bauthor{\bsnm{Kim}, \binits{K.T.}},
\bauthor{\bsnm{Zhang}, \binits{C.}},
\bauthor{\bsnm{Shiner}, \binits{A.D.}},
\bauthor{\bsnm{Schmidt}, \binits{B.E.}},
\bauthor{\bsnm{L{\'e}gar{\'e}}, \binits{F.}},
\bauthor{\bsnm{Villeneuve}, \binits{D.M.}},
\bauthor{\bsnm{Corkum}, \binits{P.B.}}:
\batitle{Petahertz optical oscilloscope}.
\bjtitle{Nature Photon}
\bvolume{7}(\bissue{12}),
\bfpage{958}--\blpage{962}
(\byear{2013}).
\doiurl{10.1038/nphoton.2013.286}
\end{barticle}
\endbibitem

%%% 23
\bibitem{liuSingleshotMeasurementFewcycle2022}
\begin{barticle}
\bauthor{\bsnm{Liu}, \binits{Y.}},
\bauthor{\bsnm{Beetar}, \binits{J.E.}},
\bauthor{\bsnm{Nesper}, \binits{J.}},
\bauthor{\bsnm{{Gholam-Mirzaei}}, \binits{S.}},
\bauthor{\bsnm{Chini}, \binits{M.}}:
\batitle{Single-shot measurement of few-cycle optical waveforms on a chip}.
\bjtitle{Nat. Photon.}
\bvolume{16}(\bissue{2}),
\bfpage{109}--\blpage{112}
(\byear{2022}).
\doiurl{10.1038/s41566-021-00924-6}
\end{barticle}
\endbibitem

%%% 24
\bibitem{sederbergAttosecondOptoelectronicField2020}
\begin{barticle}
\bauthor{\bsnm{Sederberg}, \binits{S.}},
\bauthor{\bsnm{Zimin}, \binits{D.}},
\bauthor{\bsnm{Keiber}, \binits{S.}},
\bauthor{\bsnm{Siegrist}, \binits{F.}},
\bauthor{\bsnm{Wismer}, \binits{M.S.}},
\bauthor{\bsnm{Yakovlev}, \binits{V.S.}},
\bauthor{\bsnm{Floss}, \binits{I.}},
\bauthor{\bsnm{Lemell}, \binits{C.}},
\bauthor{\bsnm{Burgd{\"o}rfer}, \binits{J.}},
\bauthor{\bsnm{Schultze}, \binits{M.}},
\bauthor{\bsnm{Krausz}, \binits{F.}},
\bauthor{\bsnm{Karpowicz}, \binits{N.}}:
\batitle{Attosecond optoelectronic field measurement in solids}.
\bjtitle{Nat Commun}
\bvolume{11}(\bissue{1}),
\bfpage{430}
(\byear{2020}).
\doiurl{10.1038/s41467-019-14268-x}
\end{barticle}
\endbibitem

%%% 25
\bibitem{ziminPetahertzscaleNonlinearPhotoconductive2021}
\begin{barticle}
\bauthor{\bsnm{Zimin}, \binits{D.}},
\bauthor{\bsnm{Weidman}, \binits{M.}},
\bauthor{\bsnm{Sch{\"o}tz}, \binits{J.}},
\bauthor{\bsnm{Kling}, \binits{M.F.}},
\bauthor{\bsnm{Yakovlev}, \binits{V.S.}},
\bauthor{\bsnm{Krausz}, \binits{F.}},
\bauthor{\bsnm{Karpowicz}, \binits{N.}}:
\batitle{Petahertz-scale nonlinear photoconductive sampling in air}.
\bjtitle{Optica, OPTICA}
\bvolume{8}(\bissue{5}),
\bfpage{586}--\blpage{590}
(\byear{2021}).
\doiurl{10.1364/OPTICA.411434}
\end{barticle}
\endbibitem

%%% 26
\bibitem{borrego-varillasAttosecondSpectroscopyInvestigation2022}
\begin{barticle}
\bauthor{\bsnm{{Borrego-Varillas}}, \binits{R.}},
\bauthor{\bsnm{Lucchini}, \binits{M.}},
\bauthor{\bsnm{Nisoli}, \binits{M.}}:
\batitle{Attosecond spectroscopy for the investigation of ultrafast dynamics in
  atomic, molecular and solid-state physics}.
\bjtitle{Rep. Prog. Phys.}
\bvolume{85}(\bissue{6}),
\bfpage{066401}
(\byear{2022}).
\doiurl{10.1088/1361-6633/ac5e7f}
\end{barticle}
\endbibitem

%%% 27
\bibitem{wangThreephotonNeuronalImaging2020}
\begin{barticle}
\bauthor{\bsnm{Wang}, \binits{T.}},
\bauthor{\bsnm{Xu}, \binits{C.}}:
\batitle{Three-photon neuronal imaging in deep mouse brain}.
\bjtitle{Optica, OPTICA}
\bvolume{7}(\bissue{8}),
\bfpage{947}--\blpage{960}
(\byear{2020}).
\doiurl{10.1364/OPTICA.395825}
\end{barticle}
\endbibitem

%%% 28
\bibitem{herbstRecentAdvancesPetahertz2022}
\begin{barticle}
\bauthor{\bsnm{Herbst}, \binits{A.}},
\bauthor{\bsnm{Scheffter}, \binits{K.}},
\bauthor{\bsnm{Bidhendi}, \binits{M.M.}},
\bauthor{\bsnm{Kieker}, \binits{M.}},
\bauthor{\bsnm{Srivastava}, \binits{A.}},
\bauthor{\bsnm{Fattahi}, \binits{H.}}:
\batitle{Recent advances in petahertz electric field sampling}.
\bjtitle{J. Phys. B: At. Mol. Opt. Phys.}
\bvolume{55}(\bissue{17}),
\bfpage{172001}
(\byear{2022}).
\doiurl{10.1088/1361-6455/ac8032}
\end{barticle}
\endbibitem

%%% 29
\bibitem{biontaOnchipSamplingOptical2021}
\begin{barticle}
\bauthor{\bsnm{Bionta}, \binits{M.R.}},
\bauthor{\bsnm{Ritzkowsky}, \binits{F.}},
\bauthor{\bsnm{Turchetti}, \binits{M.}},
\bauthor{\bsnm{Yang}, \binits{Y.}},
\bauthor{\bsnm{Cattozzo~Mor}, \binits{D.}},
\bauthor{\bsnm{Putnam}, \binits{W.P.}},
\bauthor{\bsnm{K{\"a}rtner}, \binits{F.X.}},
\bauthor{\bsnm{Berggren}, \binits{K.K.}},
\bauthor{\bsnm{Keathley}, \binits{P.D.}}:
\batitle{On-chip sampling of optical fields with attosecond resolution}.
\bjtitle{Nat. Photonics}
\bvolume{15}(\bissue{6}),
\bfpage{456}--\blpage{460}
(\byear{2021}).
\doiurl{10.1038/s41566-021-00792-0}
\end{barticle}
\endbibitem

%%% 30
\bibitem{keathleyUncoveringExtremeNonlinear2023}
\begin{barticle}
\bauthor{\bsnm{Keathley}, \binits{P.D.}},
\bauthor{\bsnm{Jensen}, \binits{S.V.B.}},
\bauthor{\bsnm{Yeung}, \binits{M.}},
\bauthor{\bsnm{Bionta}, \binits{M.R.}},
\bauthor{\bsnm{Madsen}, \binits{L.B.}}:
\batitle{Uncovering extreme nonlinear dynamics in solids through time-domain
  field analysis}.
\bjtitle{Phys. Rev. B}
\bvolume{107}(\bissue{5}),
\bfpage{054302}
(\byear{2023}).
\doiurl{10.1103/PhysRevB.107.054302}
\end{barticle}
\endbibitem

%%% 31
\bibitem{marksNonlinearInterferometricVibrational2004}
\begin{barticle}
\bauthor{\bsnm{Marks}, \binits{D.L.}},
\bauthor{\bsnm{Boppart}, \binits{S.A.}}:
\batitle{Nonlinear {{Interferometric Vibrational Imaging}}}.
\bjtitle{Phys. Rev. Lett.}
\bvolume{92}(\bissue{12}),
\bfpage{123905}
(\byear{2004}).
\doiurl{10.1103/PhysRevLett.92.123905}
\end{barticle}
\endbibitem

%%% 32
\bibitem{kumarInvitedArticleComplex2018}
\begin{barticle}
\bauthor{\bsnm{Kumar}, \binits{V.}},
\bauthor{\bsnm{{De la Cadena}}, \binits{A.}},
\bauthor{\bsnm{Perri}, \binits{A.}},
\bauthor{\bsnm{Preda}, \binits{F.}},
\bauthor{\bsnm{Coluccelli}, \binits{N.}},
\bauthor{\bsnm{Cerullo}, \binits{G.}},
\bauthor{\bsnm{Polli}, \binits{D.}}:
\batitle{Invited {{Article}}: {{Complex}} vibrational susceptibility by
  interferometric {{Fourier}} transform stimulated {{Raman}} scattering}.
\bjtitle{APL Photonics}
\bvolume{3}(\bissue{9}),
\bfpage{092403}
(\byear{2018}).
\doiurl{10.1063/1.5034114}
\end{barticle}
\endbibitem

%%% 33
\bibitem{alvesChallengesProgressProspects2022}
\begin{barticle}
\bauthor{\bsnm{Alves}, \binits{J.}},
\bauthor{\bsnm{Feng}, \binits{J.}},
\bauthor{\bsnm{Nienhaus}, \binits{L.}},
\bauthor{\bsnm{Schmidt}, \binits{T.W.}}:
\batitle{Challenges, progress and prospects in solid state triplet fusion
  upconversion}.
\bjtitle{J. Mater. Chem. C}
\bvolume{10}(\bissue{20}),
\bfpage{7783}--\blpage{7798}
(\byear{2022}).
\doiurl{10.1039/D1TC05659J}
\end{barticle}
\endbibitem

%%% 34
\bibitem{wenFutureChallengesHybrid2019}
\begin{barticle}
\bauthor{\bsnm{Wen}, \binits{S.}},
\bauthor{\bsnm{Zhou}, \binits{J.}},
\bauthor{\bsnm{Schuck}, \binits{P.J.}},
\bauthor{\bsnm{Suh}, \binits{Y.D.}},
\bauthor{\bsnm{Schmidt}, \binits{T.W.}},
\bauthor{\bsnm{Jin}, \binits{D.}}:
\batitle{Future and challenges for hybrid upconversion nanosystems}.
\bjtitle{Nat. Photonics}
\bvolume{13}(\bissue{12}),
\bfpage{828}--\blpage{838}
(\byear{2019}).
\doiurl{10.1038/s41566-019-0528-x}
\end{barticle}
\endbibitem

%%% 35
\bibitem{evansCoherentAntiStokesRaman2008}
\begin{barticle}
\bauthor{\bsnm{Evans}, \binits{C.L.}},
\bauthor{\bsnm{Xie}, \binits{X.S.}}:
\batitle{Coherent {{Anti-Stokes Raman Scattering Microscopy}}: {{Chemical
  Imaging}} for {{Biology}} and {{Medicine}}}.
\bjtitle{Annual Review of Analytical Chemistry}
\bvolume{1}(\bissue{1}),
\bfpage{883}--\blpage{909}
(\byear{2008}).
\doiurl{10.1146/annurev.anchem.1.031207.112754}
\end{barticle}
\endbibitem

%%% 36
\bibitem{evensonExtensionAbsoluteFrequency1972}
\begin{barticle}
\bauthor{\bsnm{Evenson}, \binits{K.M.}},
\bauthor{\bsnm{Day}, \binits{G.W.}},
\bauthor{\bsnm{Wells}, \binits{J.S.}},
\bauthor{\bsnm{Mullen}, \binits{L.O.}}:
\batitle{Extension of {{Absolute Frequency Measurements}} to the cw
  {{He}}\textendash{{Ne Laser}} at 88 {{THz}} (3.39 {$\mu$})}.
\bjtitle{Applied Physics Letters}
\bvolume{20}(\bissue{3}),
\bfpage{133}--\blpage{134}
(\byear{1972}).
\doiurl{10.1063/1.1654077}
\end{barticle}
\endbibitem

%%% 37
\bibitem{choTemporalCharacterizationFemtosecond2019}
\begin{barticle}
\bauthor{\bsnm{Cho}, \binits{W.}},
\bauthor{\bsnm{Hwang}, \binits{S.I.}},
\bauthor{\bsnm{Nam}, \binits{C.H.}},
\bauthor{\bsnm{Bionta}, \binits{M.R.}},
\bauthor{\bsnm{Lassonde}, \binits{P.}},
\bauthor{\bsnm{Schmidt}, \binits{B.E.}},
\bauthor{\bsnm{Ibrahim}, \binits{H.}},
\bauthor{\bsnm{L{\'e}gar{\'e}}, \binits{F.}},
\bauthor{\bsnm{Kim}, \binits{K.T.}}:
\batitle{Temporal characterization of femtosecond laser pulses using tunneling
  ionization in the {{UV}}, visible, and mid-{{IR}} ranges}.
\bjtitle{Sci Rep}
\bvolume{9}(\bissue{1}),
\bfpage{16067}
(\byear{2019}).
\doiurl{10.1038/s41598-019-52237-y}
\end{barticle}
\endbibitem

%%% 38
\bibitem{ritzkowskyEngineeringFrequencyResponse2022}
\begin{bchapter}
\bauthor{\bsnm{Ritzkowsky}, \binits{F.}},
\bauthor{\bsnm{Bionta}, \binits{M.R.}},
\bauthor{\bsnm{Turchetti}, \binits{M.}},
\bauthor{\bsnm{Berggren}, \binits{K.K.}},
\bauthor{\bsnm{K{\"a}rtner}, \binits{F.X.}},
\bauthor{\bsnm{Keathley}, \binits{P.D.}}:
\bctitle{Engineering the {{Frequency Response}} of {{Petahertz-Electronic
  Nanoantenna Field-Sampling Devices}}}.
In: \bbtitle{Conference on {{Lasers}} and {{Electro-Optics}} (2022), Paper
  {{JW3A}}.56},
pp. \bfpage{3}--\blpage{56}.
\bpublisher{{Optica Publishing Group}},
\blocation{{San Jose, CA, USA}}
(\byear{2022}).
\doiurl{10.1364/CLEO\_AT.2022.JW3A.56}
\end{bchapter}
\endbibitem

%%% 39
\bibitem{yangLightPhaseDetection2020}
\begin{barticle}
\bauthor{\bsnm{Yang}, \binits{Y.}},
\bauthor{\bsnm{Turchetti}, \binits{M.}},
\bauthor{\bsnm{Vasireddy}, \binits{P.}},
\bauthor{\bsnm{Putnam}, \binits{W.P.}},
\bauthor{\bsnm{Karnbach}, \binits{O.}},
\bauthor{\bsnm{Nardi}, \binits{A.}},
\bauthor{\bsnm{K{\"a}rtner}, \binits{F.X.}},
\bauthor{\bsnm{Berggren}, \binits{K.K.}},
\bauthor{\bsnm{Keathley}, \binits{P.D.}}:
\batitle{Light phase detection with on-chip petahertz electronic networks}.
\bjtitle{Nat Commun}
\bvolume{11}(\bissue{1}),
\bfpage{3407}
(\byear{2020}).
\doiurl{10.1038/s41467-020-17250-0}
\end{barticle}
\endbibitem

%%% 40
\bibitem{jafari100ReliableAlgorithm2019}
\begin{barticle}
\bauthor{\bsnm{Jafari}, \binits{R.}},
\bauthor{\bsnm{Jones}, \binits{T.}},
\bauthor{\bsnm{Trebino}, \binits{R.}}:
\batitle{100\% reliable algorithm for second-harmonic-generation
  frequency-resolved optical gating}.
\bjtitle{Opt. Express, OE}
\bvolume{27}(\bissue{3}),
\bfpage{2112}--\blpage{2124}
(\byear{2019}).
\doiurl{10.1364/OE.27.002112}
\end{barticle}
\endbibitem

%%% 41
\bibitem{trebinoMATLABCodeAvailable}
\begin{botherref}
\oauthor{\bsnm{Trebino}, \binits{R.}}:
{{MATLAB}} Code Is Available on the {{Trebino Group}} Website.
https://frog.gatech.edu/code.html
\end{botherref}
\endbibitem

%%% 42
\bibitem{oskooiMeepFlexibleFreesoftware2010}
\begin{barticle}
\bauthor{\bsnm{Oskooi}, \binits{A.F.}},
\bauthor{\bsnm{Roundy}, \binits{D.}},
\bauthor{\bsnm{Ibanescu}, \binits{M.}},
\bauthor{\bsnm{Bermel}, \binits{P.}},
\bauthor{\bsnm{Joannopoulos}, \binits{J.D.}},
\bauthor{\bsnm{Johnson}, \binits{S.G.}}:
\batitle{Meep: {{A}} flexible free-software package for electromagnetic
  simulations by the {{FDTD}} method}.
\bjtitle{Computer Physics Communications}
\bvolume{181}(\bissue{3}),
\bfpage{687}--\blpage{702}
(\byear{2010}).
\doiurl{10.1016/j.cpc.2009.11.008}
\end{barticle}
\endbibitem

%%% 43
\bibitem{dudleySupercontinuumGenerationOptical2010}
\begin{bbook}
\beditor{\bsnm{Dudley}, \binits{J.M.}},
\beditor{\bsnm{Taylor}, \binits{J.R.}} (eds.):
\bbtitle{Supercontinuum {{Generation}} in {{Optical Fibers}}}.
\bpublisher{{Cambridge University Press}},
\blocation{{Cambridge}}
(\byear{2010}).
\doiurl{10.1017/CBO9780511750465}
\end{bbook}
\endbibitem

%%% 44
\bibitem{putnamFewcycleCarrierEnvelopephasestable2019}
\begin{barticle}
\bauthor{\bsnm{Putnam}, \binits{W.P.}},
\bauthor{\bsnm{Keathley}, \binits{P.D.}},
\bauthor{\bsnm{Cox}, \binits{J.A.}},
\bauthor{\bsnm{Liehl}, \binits{A.}},
\bauthor{\bsnm{Leitenstorfer}, \binits{A.}},
\bauthor{\bsnm{K{\"a}rtner}, \binits{F.X.}}:
\batitle{Few-cycle, carrier\textendash envelope-phase-stable laser pulses from
  a compact supercontinuum source}.
\bjtitle{J. Opt. Soc. Am. B, JOSAB}
\bvolume{36}(\bissue{2}),
\bfpage{93}--\blpage{97}
(\byear{2019}).
\doiurl{10.1364/JOSAB.36.000A93}
\end{barticle}
\endbibitem

\end{thebibliography}



\end{document}
