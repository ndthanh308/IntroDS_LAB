%\documentclass[aps, prl, english, showpacs, superscriptaddress, reprint]{revtex4-1}
\documentclass[reprint, showpacs, superscriptaddress, aps, prl, twocolumn, 10pt]{revtex4-1}
\usepackage{graphicx}
\usepackage{color}
\usepackage{amssymb}
\usepackage{amsmath}
\usepackage{tabularx}
\usepackage{bm}
\usepackage{hyperref}
\usepackage{ltxtable}
\usepackage{natbib} 
\usepackage{longtable}
\usepackage{enumitem}% http://ctan.org/pkg/enumitem
%\usepackage{subcaption}
%\captionsetup[figure]{font=small,labelfont=normalsize}
\usepackage{float}
\usepackage{mathtools}

\newcommand{\la}{\left\langle}
\newcommand{\ra}{\right\rangle}
\newcommand{\be}{\begin{equation}}
\newcommand{\ee}{\end{equation}}
\newcommand{\bse}{\begin{subequations}}
	\newcommand{\ese}{\end{subequations}}
\newcommand{\bea}{\begin{eqnarray}}
\newcommand{\eea}{\end{eqnarray}}
\newcommand{\ba}{\begin{array}}
	\newcommand{\ea}{\end{array}}
%


\usepackage[usenames,dvipsnames]{xcolor}
\hypersetup{
colorlinks,
citecolor=blue,
filecolor=blue,
linkcolor=blue,
urlcolor=blue}

\begin{document}

\title{Critical Prandtl number for Heat Transfer Enhancement in Rotating Convection}
\author{Mohammad Anas}
%\email{anas@iitk.ac.in}
\affiliation{Department of Mechanical Engineering, Indian Institute of Technology, Kanpur 208016, India}
\author{Pranav Joshi}
%\email{jpranavr@iitk.ac.in }
\affiliation{Department of Mechanical Engineering, Indian Institute of Technology, Kanpur 208016, India}

%\date{\today}

\begin{abstract}
Rotation can enhance the heat transfer in thermal convection at low and moderate Rayleigh number ($Ra$). However, there has been no evidence of such enhancement at high Rayleigh number ($Ra\gtrsim 10^{10}$), which is relevant for most large-scale natural phenomena. In this Letter, we show that rotation can enhance the heat transfer significantly even for high Rayleigh numbers ($\gtrsim 10^{10}$), provided the Prandtl number is greater than a critical value, $Pr_{cr}$, that increases with $Ra$. We also predict that heat transfer enhancement due to rotation not only would occur at $Ra>10^{10}$ but would also become more pronounced.
\end{abstract}

\maketitle
Thermal convection under the influence of background rotation manifests in various geophysical and astrophysical flows, such as flows occurring within the Earth’s atmosphere, oceans, and outer core~\cite{Gill:book, Glatzmaier:Nature1995, Marshall:RGP1999}, gaseous planets like Jupiter~\cite{Ingersoll:Science1990, Heimpel:Nature2005}, and solar interiors~\cite{Spiegel:ARAA1971}. Rotation, which introduces the Coriolis force into the system, significantly affects the characteristics of these flows, including heat and momentum transfer~\cite{Chandrasekhar:book:Instability, Rossby:JFM1969}. The canonical model to study the behavior of such systems is rotating Rayleigh-B\'enard convection (RBC), in which fluid motion occurs between a hot plate (at the bottom) and a cold plate (at the top) as a consequence of the thermal buoyancy while the system rotates along an axis parallel to the gravity~\cite{Chandrasekhar:book:Instability}. 

Rotating RBC is primarily governed by three dimensionless parameters: the Rayleigh number ($Ra$), which represents the strength of the buoyancy force over the dissipative forces, the Prandtl number ($Pr$), which represents the ratio of the momentum diffusivity to thermal diffusivity, and the Taylor number ($Ta$), which represents the strength of the Coriolis force relative to the viscous force. To characterize the relative strength of convection over rotation, convective Rossby number ($Ro=\sqrt{Ra/TaPr}$) is commonly used. When $Ro\gg 1$, the buoyancy force dominates over the Coriolis force, and the heat transfer characteristics of rotating RBC systems are similar to those of corresponding non-rotating RBC~\cite{King:Nature2008, King:JFM2012, Stevens:PRL2013}. On the other hand, when $Ro\ll 1$, rotation becomes dominant and the heat transfer in rotating RBC, as compared to that of non-rotating case, is severely suppressed. Such rotating RBC system exhibits similarities to geostrophic flow, which is characterized by a force balance between pressure gradient and the Coriolis force~\cite{King:Nature2008, Kunnen:JT2021, Ecke:ARFM2023}.

Rotation, which suppresses the intensity of flow, enhances the heat transfer in rotating RBC for a certain range of $Ra$, $Pr$, and $Ta$~\cite{Rossby:JFM1969, Zhong:JFM1993, Liu:PRL1997, Kunnen:PRE2006, Stevens:EJMFB2013, Ping:PRL2015, Zhong:PRL2009RotatingRBC, Stevens:PRL2013, Yang:PRF2020, Vorobieff:JFM2002, Stevens:PRL2009, Stevens:NJP2010, Stevens:POF2010, Stevens:PRE2011, Joshi:JFM2017, Weiss:PRL2010, Weiss:PRE2016, Chong:PRL2017}. This heat transfer enhancement in rotating RBC as compared to the non-rotating case is ascribed to Ekman pumping. Rotation generates columnar vortices aligned with the rotation axis in the flow, which in turn induce a secondary motion (parallel to the rotation) within the viscous boundary layer~\cite{Davidson:book:TurbulenceRotating}. This secondary motion facilitates the transport of hot fluid (at the bottom plate) and cold fluid (at the top plate) from the thermal boundary layers, leading to this enhancement in the heat transfer~\cite{Julien:JFM1996, Zhong:PRL2009RotatingRBC, Ecke:ARFM2023}.

Although ample evidence for this heat transfer enhancement in rotating RBC is found in earlier studies for moderate $Ra$~\cite{Rossby:JFM1969, Zhong:JFM1993, Liu:PRL1997, Kunnen:PRE2006, Stevens:EJMFB2013, Ping:PRL2015, Zhong:PRL2009RotatingRBC, Stevens:PRL2013, Yang:PRF2020, Vorobieff:JFM2002, Stevens:PRL2009, Stevens:NJP2010, Stevens:POF2010, Stevens:PRE2011, Joshi:JFM2017, Weiss:PRL2010, Weiss:PRE2016, Chong:PRL2017}, evidence for enhancement at high $Ra$ ($Ra\gtrsim 10^{10}$) does not exist. Thus, it is commonly expected that there is no (or insignificant) heat transfer enhancement at $Ra\gtrsim 10^{10}$ in rotating RBC~\cite{Kunnen:JT2021, Ecke:ARFM2023}. In this Letter, however, we demonstrate a clear evidence of significant heat transfer enhancement in rotating RBC even at $Ra\ge 10^{10}$. {We explore a very wide range of Prandtl numbers, including very high $Pr$ ($\sim \mathcal{O}(1000)$) that have not been studied earlier for rotating convection, to uncover the existence of a ‘critical’ Prandtl number, $Pr_{cr}$.} We show that for each $Ra$ (at least within the range of $Ra=2\times 10^4-2\times 10^{10}$ explored in the present work), heat transfer enhancement will occur only if the Prandtl number is greater than $Pr_{cr}$ that increases with increasing $Ra$. In this work, we also provide a precise definition of the optimal Prandtl number $Pr_{opt}$ for obtaining the maximum heat transfer enhancement at a given $Ra$ and show that $Pr_{opt}$ also increases with $Ra$. Importantly, the present findings predict that heat transfer enhancement due to rotation not only would occur but also would become more pronounced at $Ra>10^{10}$. 

For this study, we perform direct numerical simulations (DNS) of rotating RBC for a wide range of parameters: $Ra=g\beta\Delta H^3 /(\nu\kappa)=2\times 10^4-2\times 10^{10}$, $Pr=\nu/\kappa=1-1000$, and $Ta=4\Omega^2H^4/\nu^2=0-2\times 10^{12}$, and measure the heat transfer in terms of the Nusselt number $Nu=qH/(\lambda\Delta)$ . Here, $g$ is the acceleration due to gravity, $\beta$ is the thermal expansion coefficient, $\Delta$ is the temperature difference between the hot and cold plates, $H$ is the separation between the plates, $\nu$ is the kinematic viscosity, $\kappa$ is the thermal diffusivity, $\Omega$ is the system's rotation rate, $\lambda$ is the thermal conductivity of the fluid, and $q$ is the heat flux from the hot to cold plates. We perform simulations in a horizontally periodic rectangular domain of size $L\times L\times H$ ($L\times L$ in the horizontal directions) employing isothermal and no-slip (and impenetrable) boundary conditions at the top (cold) and bottom (hot) plates. For the simulations of non-rotating RBC ($Ta=0$) at moderate $Ra$, we use large aspect ratio ($\Gamma=L/H$) to avoid the effect of confinement on the Nusselt number~\cite{Huang:PRL2013}: $\Gamma=8$ for $Ra=2\times 10^4-10^6$ and $\Gamma=4$ for $Ra=10^7-10^8$. Considering the high computational cost at large $\Gamma$ for high $Ra$, we use $\Gamma=1$ for $Ra=5\times 10^8-2.3\times 10^9$ and $\Gamma= 0.5$ for $Ra=10^{10}$. We use $Nu\approx 0.12Ra^{0.30}$ (which we obtain by fitting the $Nu$ data for $Ra=10^6-10^{10}$ at $Pr=100$) to estimate $Nu\approx 148$ for $Ra=2\times 10^{10}$ and $Pr=100$. 

Since the horizontal length scale of the flow in rotating convection, $\ell_c$, decreases with $Ta$ as $\ell_c=2.4Ta^{-1/6}H$~\cite{Chandrasekhar:book:Instability}, we use relatively lower aspect ratios (half or one-fourth of those for the corresponding non-rotating RBC cases) for some simulations of rotating RBC. {In all simulations of rotating RBC, we ensure $\ell_c/L\lesssim 1/8$ to mitigate the effect of confinement on $Nu$~\cite{Kunnen:JFM2016, Julien:JFM1996}}. For more details about the simulations and the solver used in this study, please refer to the Supplementary Material~\cite{SM:ThisPaper}.

In Fig.~\ref{fig:Nu_Ta_Ro}, we show the variation of the normalized Nusselt number $Nu/Nu_0$ ($Nu_0$ is the Nusselt number for the non-rotating case) with the Taylor number, $Ta$, and the inverse of the Rossby number, $1/Ro$, for $Pr=1-1000$ at $Ra=[10^7, 10^8, 10^9, 10^{10}]$. We observe that when $Pr$ is not too small, $Nu/Nu_0$ first increases and then decreases as the rotation rate is increased. For each $Ra$, the maximum enhancement in the heat transfer as compared to the non-rotating case occurs in a certain range of $Pr$ and rotation rate. This maximum enhancement increases with increasing $Ra$ and can reach up to approximately $25\%$, $40\%$, and $55\%$ for $Ra=10^7$, $Ra=10^8$, and $Ra=10^9$, respectively. Note that we observe a significant heat transfer enhancement (more than $40\%$) even at $Ra=10^{10}$, which will be discussed later in greater detail. 

Interestingly, we observe (see Fig.~\ref{fig:Nu_Ta_Ro}) that the Taylor number $Ta$ serves as a better parameter than $1/Ro$ in representing the optimal rotation rate at which the maximum enhancement occurs. Unlike the optimal rotation rate represented in terms of the inverse of Rossby number, ($1/Ro_{opt}$), the optimal Taylor number, $Ta_{opt}$, is nearly independent of $Pr$ when significant enhancement is observed at a given $Ra$. Most earlier studies used $1/Ro_{opt}$ to represent the optimal rotation rate but found $1/Ro_{opt}$ to be strongly dependent on $Pr$~\cite{Zhong:PRL2009RotatingRBC, Stevens:NJP2010, Yang:PRF2020}. Since the heat transfer enhancement due to rotation is largely controlled by the dynamics of the Ekman boundary layer~\cite{Stellmach:PRL2014}, the thickness of which depends only on the Taylor number (which represents the ratio of Coriolis to viscous forces), $Ta$ can be expected to represent better the heat transfer enhancement than $1/Ro$ (which represents the ratio of Coriolis to buoyancy forces) at moderate and high rotation rates. This finding is also in line with the hypothesis of ~\citet{King:Nature2008} that the boundary layer controls the rotation-dominated regime in rotating RBC, rather than the balance between the buoyancy and Coriolis forces. Nonetheless, the beginning of the rotation-affected regime, i.e., the rotation rate at which $Nu/Nu_0$ deviates from $1$, is better represented by $1/Ro$ than by $Ta$, as seen from our results. Specifically, for $Ra = 10^7$ and $10^8$, the rotation-affected regime begins at $1/Ro\approx 0.2$, while for $Ra=10^9$, it begins at $1/Ro\approx 0.4$. These values are consistent with the findings of~\citet{Stevens:PRL2009}.

In Fig.~\ref{fig:Taopt_Ra}, we show the variation of $Ta_{opt}$ with $Ra$ for various Prandtl numbers. As discussed earlier, we observe that at a given $Ra$, the optimal Taylor number does not vary significantly with $Pr$. Also, $Ta_{opt}$ follows a power law close to $Ta_{opt}\propto Ra^{1.5}$ up to a certain $Ra$ and this limiting $Ra$ for the power law seems to increase with increasing $Pr$. Note that the Taylor number at which convection ceases completely ($Ta_{cs}$) also follows the scaling $Ta_{cs}\propto Ra^{1.5}$ for $Ta\gg 1$~\cite{Chandrasekhar:book:Instability}. 

In Fig.~\ref{fig:Nuopt_Pr}, we show the variation of the normalized maximum Nusselt number, $Nu_{max}/Nu_0$, with $Pr$ for $Ra=2\times 10^4-2\times 10^{10}$. At any given $Ra$ and $Pr$, $Nu_{max}$, by definition, corresponds to the Nusselt number at the optimal Taylor number for that $Ra$ and $Pr$. The maximum heat transfer enhancement represented by $Nu_{max}/Nu_0$ for each $Ra$ increases with $Pr$ up to a certain $Pr$ and then decreases. Note that for each $Ra$ there exists a Prandtl number (obtained by extrapolating the data for each $Ra$ to $Nu_{max}/Nu_0=1$) below which there will be no (or negligible) heat transfer enhancement at any rotation rate. We call this $Pr$ the critical Prandtl number $Pr_{cr}$. In rotating convection, columnar vortical structures are known to play an important role in the heat transfer by transporting the temperature anomaly from one wall to the other~\cite{Zhong:PRL2009RotatingRBC, Julien:JFM1996}. However, for $Pr<Pr_{cr}$, it is likely that the lateral diffusion of the heat/temperature anomaly away from the vortex columns restricts their ability to transport heat between the top and bottom walls: see Fig.~\ref{fig:TempContour} which shows the flow structure for different $Pr$ at $Ra=10^8$ and $Ta\approx Ta_{opt}$ ($Pr=2$ for Fig.~\ref{fig:TempContour}(a) is close to $Pr_{cr}$)~\cite{Stevens:NJP2010, Hartmann:arxiv2023}. As $Pr$ increases, this effect is expected to weaken {(e.g., see Fig.~\ref{fig:TempContour}(b) and \ref{fig:TempContour}(c))}, and so the heat transfer increases . However, the heat transfer enhancement decreases again at very large $Pr$. 
At any $Ra$, we define the Prandtl number at which $Nu_{max}/Nu_0$ reaches its maximum as the optimal Prandtl number $Pr_{opt}$ for that $Ra$. Some studies (e.g., \citet{Stevens:NJP2010}), comparing the heat transfer at a constant $Ro$, have proposed that at large $Pr$, the Ekman boundary layer ($\delta_u$) is much thicker than the thermal boundary layer ($\delta_\theta$); consequently, the columnar vortices do not reach the thermal boundary layer and the fluid entering them is not as hot (or as cold), leading to a decrease in the heat transfer enhancement at large $Pr$. However, we observe that the maximum enhancement (which occurs at $Ta_{opt}$) decreases at large Prandtl number even though $1.3\lesssim \delta_u/\delta_\theta\lesssim 1.5$ for $Pr>Pr_{opt}$ (see Supplementary Material~\cite{SM:ThisPaper}). We hypothesize that the higher viscous damping of the flow at very large $Pr$ weakens the vertical advection of heat by the columnar structures (see Fig.~\ref{fig:TempContour}(d) in which these columns are observed to have diffused significantly in the lateral direction), resulting in maximum heat transfer at an intermediate (optimal) $Pr$.

Here, we make an important observation. The results show a significant heat transfer enhancement even at $Ra\ge 10^{10}$: approximately $40\%$ at $Ra=2\times 10^{10}$ for $Pr=100$, and the trends indicate an even higher heat transfer enhancement for higher $Pr$. This finding challenges a common expectation that there will be no (or negligible) heat transfer enhancement due to rotation at $Ra\gtrsim 10^{10}$~\cite{Ecke:ARFM2023, Kunnen:JT2021}. The present results predict that heat transfer enhancement due to rotation is possible even for $Ra>10^{10}$ provided $Pr>Pr_{cr}$. For $Ra=10^{10}$, $Pr_{cr}\approx 10$. This is the reason why most earlier studies, which use $Pr<Pr_{cr}$, have reported no (or negligible) heat transfer enhancement for $Ra\gtrsim 10^{10}$: \citet{Niemela:JFM2010} (for $Pr=0.7-5.9$), \citet{Stellmach:PRL2014} (for $Pr\approx 1-7$), \citet{Kunnen:JFM2016} (for $Pr=1$), \citet{Ecke:PRL2014} (for $Pr=0.7$), and \citet{Hartmann:arxiv2023} (for $Pr = 4.38$ and $6.4$). Note that we also observe a significant heat transfer enhancement ($>10\%$) at $Ra=2\times 10^4$ (using $\Gamma=8$), in agreement with \citet{Rossby:JFM1969}'s experimental results for $\Gamma\gtrsim 6$. 

In Fig.~\ref{fig:PrcrProptRa}, we show the variation of $Pr_{cr}$ and $Pr_{opt}$ with $Ra$. Interestingly, both $Pr_{cr}$ and $Pr_{opt}$ increase monotonically with $Ra$ and approximately follow power-laws: $Pr_{cr}\approx 1.46\times 10^{-3}Ra^{0.375}$ and $Pr_{opt}\approx 8.18\times 10^{-3}Ra^{0.504}$. Considering the high computational cost at large $Pr$ and large $Ra$, we do not perform simulations at $Pr>200$ to find $Pr_{opt}$ for $Ra=10^{10}$, which is estimated to be $Pr_{opt}\approx 900$ by the above power-law fit. As Rayleigh number increases, the turbulent diffusion of heat is also expected to become stronger. At low $Pr$, this higher turbulent diffusion will combine with the large molecular thermal diffusivity to further increase the lateral diffusion of heat in the bulk, and hence, will decrease the ability of the vortex columns to transport heat. Thus, a correspondingly larger $Pr$ may be necessary to counter this effect of the enhanced turbulent thermal diffusivity to register any enhancement in the heat transfer, i.e., $Pr_{cr}$ will increase with increasing $Ra$. On the other hand, as the buoyancy forcing increases with increasing $Ra$, the viscous damping of the flow at large $Pr$ will become weaker and the heat transfer enhancement can be sustained until larger Prandtl numbers, i.e. $Pr_{opt}$ also increases as $Ra$ is increased. 

In Fig.~\ref{fig:NumaxRa}, we show the variation of $Nu_{max}(Pr_{opt})/Nu_0$ with $Ra$. Here, $Nu_{max}(Pr_{opt})$ is $Nu_{max}$ at $Pr=Pr_{opt}$. Again, we observe a clear monotonic increase of $Nu_{max}(Pr_{opt})/Nu_0$ with $Ra$, and the trend can be fitted by a power-law $Nu_{max}(Pr_{opt})/Nu_0\approx 0.62Ra^{0.044}$. This relationship predicts $Nu_{max}/Nu_0\approx 1.7$ (i.e., $70\%$ enhancement in heat transfer) for $Ra=10^{10}$ at $Pr_{opt}\approx 900$, and even higher enhancement at $Ra>10^{10}$ at higher $Pr$. Thus, the present trends predict that the maximum heat transfer enhancement increases with $Ra$, even for $Ra>10^{10}$, provided the Prandtl number is also increased commensurately. As discussed earlier, at a given $Ra$, the enhancement increases with $Pr$ up to $Pr_{opt}$ beyond which the viscous damping of the flow likely restricts the vertical advection of heat. However, as $Ra$ and hence buoyancy forcing is increased, the enhancement due to rotation can increase up to larger $Pr$ before the viscous damping effect becomes significant. Thus, the maximum enhancement can be expected to increase with $Ra$. 

Note that the effect of finite aspect ratio on $Nu$ in non-rotating simulations~\cite{Huang:PRL2013} may lead to some uncertainty in the values of $Pr_{cr}$ and $Nu_{max}/Nu_0$ for $Ra\ge 5\times 10^8$. However, this effect is expected to not alter any of the major findings of this study.


% Figure environment removed

% Figure environment removed

% Figure environment removed

% Figure environment removed

% Figure environment removed


% Figure environment removed

The present results show that as $Ra$ is increased in rotating RBC, $Pr_{cr}$,  $Pr_{opt}$, as well as the maximum heat transfer enhancement increase. In particular, not only enhancement is certainly possible for $Ra>10^{10}$, but is also expected to be higher than that for lower $Ra$ at the optimal Prandtl number. However, we do not know up to what $Ra$ these trends will persist. Simulations and experiments at significantly higher $Ra$ and $Pr$ than currently feasible may be required to answer this question. 

We thank Roshan Samuel for his valuable assistance in the development of the solver used in this study. We also thank Prof. M. K. Verma for inspiring us to utilize GPUs for scientific computing, and also for providing resources for the testing of the solver and for running some simulations. Mohammad Anas thanks Soumyadeep Chatterjee, Shadab Alam, and Manthan Verma for their useful discussions on the solver and this work. For all the simulations related to this work, we gratefully acknowledge the support and the resources provided by Param Sanganak under the National Supercomputing Mission, Government of India at the Indian Institute of Technology, Kanpur.

%\FloatBarrier

%\bibliography{bib/journal, bib/book}
%%%%%%%%%%%%%%%%%%%%% Appendix A %%%%%%%%%%%%%%%%%%%%% 

%\appendix
%\section{Appendix}
%\label{app:appendix}


%merlin.mbs apsrev4-1.bst 2010-07-25 4.21a (PWD, AO, DPC) hacked
%Control: key (0)
%Control: author (8) initials jnrlst
%Control: editor formatted (1) identically to author
%Control: production of article title (-1) disabled
%Control: page (0) single
%Control: year (1) truncated
%Control: production of eprint (0) enabled
\begin{thebibliography}{38}%
	\makeatletter
	\providecommand \@ifxundefined [1]{%
		\@ifx{#1\undefined}
	}%
	\providecommand \@ifnum [1]{%
		\ifnum #1\expandafter \@firstoftwo
		\else \expandafter \@secondoftwo
		\fi
	}%
	\providecommand \@ifx [1]{%
		\ifx #1\expandafter \@firstoftwo
		\else \expandafter \@secondoftwo
		\fi
	}%
	\providecommand \natexlab [1]{#1}%
	\providecommand \enquote  [1]{``#1''}%
	\providecommand \bibnamefont  [1]{#1}%
	\providecommand \bibfnamefont [1]{#1}%
	\providecommand \citenamefont [1]{#1}%
	\providecommand \href@noop [0]{\@secondoftwo}%
	\providecommand \href [0]{\begingroup \@sanitize@url \@href}%
	\providecommand \@href[1]{\@@startlink{#1}\@@href}%
	\providecommand \@@href[1]{\endgroup#1\@@endlink}%
	\providecommand \@sanitize@url [0]{\catcode `\\12\catcode `\$12\catcode
		`\&12\catcode `\#12\catcode `\^12\catcode `\_12\catcode `\%12\relax}%
	\providecommand \@@startlink[1]{}%
	\providecommand \@@endlink[0]{}%
	\providecommand \url  [0]{\begingroup\@sanitize@url \@url }%
	\providecommand \@url [1]{\endgroup\@href {#1}{\urlprefix }}%
	\providecommand \urlprefix  [0]{URL }%
	\providecommand \Eprint [0]{\href }%
	\providecommand \doibase [0]{http://dx.doi.org/}%
	\providecommand \selectlanguage [0]{\@gobble}%
	\providecommand \bibinfo  [0]{\@secondoftwo}%
	\providecommand \bibfield  [0]{\@secondoftwo}%
	\providecommand \translation [1]{[#1]}%
	\providecommand \BibitemOpen [0]{}%
	\providecommand \bibitemStop [0]{}%
	\providecommand \bibitemNoStop [0]{.\EOS\space}%
	\providecommand \EOS [0]{\spacefactor3000\relax}%
	\providecommand \BibitemShut  [1]{\csname bibitem#1\endcsname}%
	\let\auto@bib@innerbib\@empty
	%</preamble>
	\bibitem [{\citenamefont {Gill}(1982)}]{Gill:book}%
	\BibitemOpen
	\bibfield  {author} {\bibinfo {author} {\bibfnamefont {A.~E.}\ \bibnamefont
			{Gill}},\ }\href@noop {} {\emph {\bibinfo {title} {Atmosphere--ocean
				dynamics}}}\ (\bibinfo  {publisher} {Academic Press},\ \bibinfo {address}
	{New York},\ \bibinfo {year} {1982})\BibitemShut {NoStop}%
	\bibitem [{\citenamefont {Glatzmaier}\ and\ \citenamefont
		{Roberts}(1995)}]{Glatzmaier:Nature1995}%
	\BibitemOpen
	\bibfield  {author} {\bibinfo {author} {\bibfnamefont {G.~A.}\ \bibnamefont
			{Glatzmaier}}\ and\ \bibinfo {author} {\bibfnamefont {P.~H.}\ \bibnamefont
			{Roberts}},\ }\href@noop {} {\bibfield  {journal} {\bibinfo  {journal}
			{Nature}\ }\textbf {\bibinfo {volume} {377}},\ \bibinfo {pages} {203}
		(\bibinfo {year} {1995})}\BibitemShut {NoStop}%
	\bibitem [{\citenamefont {Marshall}\ and\ \citenamefont
		{Schott}(1999)}]{Marshall:RGP1999}%
	\BibitemOpen
	\bibfield  {author} {\bibinfo {author} {\bibfnamefont {J.}~\bibnamefont
			{Marshall}}\ and\ \bibinfo {author} {\bibfnamefont {F.}~\bibnamefont
			{Schott}},\ }\href@noop {} {\bibfield  {journal} {\bibinfo  {journal} {Rev.
				Geophy.}\ }\textbf {\bibinfo {volume} {37}},\ \bibinfo {pages} {1} (\bibinfo
		{year} {1999})}\BibitemShut {NoStop}%
	\bibitem [{\citenamefont {Ingersoll}(1990)}]{Ingersoll:Science1990}%
	\BibitemOpen
	\bibfield  {author} {\bibinfo {author} {\bibfnamefont {A.~P.}\ \bibnamefont
			{Ingersoll}},\ }\href@noop {} {\bibfield  {journal} {\bibinfo  {journal}
			{Science}\ }\textbf {\bibinfo {volume} {248}},\ \bibinfo {pages} {308}
		(\bibinfo {year} {1990})}\BibitemShut {NoStop}%
	\bibitem [{\citenamefont {Heimpel}\ \emph {et~al.}(2005)\citenamefont
		{Heimpel}, \citenamefont {Aurnou},\ and\ \citenamefont
		{Wicht}}]{Heimpel:Nature2005}%
	\BibitemOpen
	\bibfield  {author} {\bibinfo {author} {\bibfnamefont {M.}~\bibnamefont
			{Heimpel}}, \bibinfo {author} {\bibfnamefont {J.}~\bibnamefont {Aurnou}}, \
		and\ \bibinfo {author} {\bibnamefont {Wicht}},\ }\href@noop {} {\bibfield
		{journal} {\bibinfo  {journal} {Nature}\ }\textbf {\bibinfo {volume} {438}},\
		\bibinfo {pages} {193–196} (\bibinfo {year} {2005})}\BibitemShut {NoStop}%
	\bibitem [{\citenamefont {Spiegel}(1971)}]{Spiegel:ARAA1971}%
	\BibitemOpen
	\bibfield  {author} {\bibinfo {author} {\bibfnamefont {E.~A.}\ \bibnamefont
			{Spiegel}},\ }\href@noop {} {\bibfield  {journal} {\bibinfo  {journal} {Annu.
				Rev. Astron. Astrophys.}\ }\textbf {\bibinfo {volume} {9}},\ \bibinfo {pages}
		{323} (\bibinfo {year} {1971})}\BibitemShut {NoStop}%
	\bibitem [{\citenamefont
		{Chandrasekhar}(1961)}]{Chandrasekhar:book:Instability}%
	\BibitemOpen
	\bibfield  {author} {\bibinfo {author} {\bibfnamefont {S.}~\bibnamefont
			{Chandrasekhar}},\ }\href@noop {} {\emph {\bibinfo {title} {{Hydrodynamic and
					Hydromagnetic Stability}}}}\ (\bibinfo  {publisher} {Oxford University
		Press},\ \bibinfo {address} {Oxford},\ \bibinfo {year} {1961})\BibitemShut
	{NoStop}%
	\bibitem [{\citenamefont {Rossby}(1969)}]{Rossby:JFM1969}%
	\BibitemOpen
	\bibfield  {author} {\bibinfo {author} {\bibfnamefont {H.~T.}\ \bibnamefont
			{Rossby}},\ }\href@noop {} {\bibfield  {journal} {\bibinfo  {journal} {J.
				Fluid Mech.}\ }\textbf {\bibinfo {volume} {36}},\ \bibinfo {pages} {309}
		(\bibinfo {year} {1969})}\BibitemShut {NoStop}%
	\bibitem [{\citenamefont {King}\ \emph {et~al.}(2008)\citenamefont {King},
		\citenamefont {Stellmach}, \citenamefont {Noir}, \citenamefont {Hansen},\
		and\ \citenamefont {Aurnou}}]{King:Nature2008}%
	\BibitemOpen
	\bibfield  {author} {\bibinfo {author} {\bibfnamefont {E.~M.}\ \bibnamefont
			{King}}, \bibinfo {author} {\bibfnamefont {S.}~\bibnamefont {Stellmach}},
		\bibinfo {author} {\bibfnamefont {J.}~\bibnamefont {Noir}}, \bibinfo {author}
		{\bibfnamefont {U.}~\bibnamefont {Hansen}}, \ and\ \bibinfo {author}
		{\bibfnamefont {J.~M.}\ \bibnamefont {Aurnou}},\ }\href@noop {} {\bibfield
		{journal} {\bibinfo  {journal} {Nature}\ }\textbf {\bibinfo {volume} {457}},\
		\bibinfo {pages} {301} (\bibinfo {year} {2008})}\BibitemShut {NoStop}%
	\bibitem [{\citenamefont {King}\ \emph {et~al.}(2012)\citenamefont {King},
		\citenamefont {Stellmach},\ and\ \citenamefont {Aurnou}}]{King:JFM2012}%
	\BibitemOpen
	\bibfield  {author} {\bibinfo {author} {\bibfnamefont {E.~M.}\ \bibnamefont
			{King}}, \bibinfo {author} {\bibfnamefont {S.}~\bibnamefont {Stellmach}}, \
		and\ \bibinfo {author} {\bibfnamefont {J.~M.}\ \bibnamefont {Aurnou}},\
	}\href@noop {} {\bibfield  {journal} {\bibinfo  {journal} {J. Fluid Mech.}\
		}\textbf {\bibinfo {volume} {691}},\ \bibinfo {pages} {568} (\bibinfo {year}
		{2012})}\BibitemShut {NoStop}%
	\bibitem [{\citenamefont {Stevens}\ \emph
		{et~al.}(2009{\natexlab{a}})\citenamefont {Stevens}, \citenamefont {Zhong},
		\citenamefont {Clercx}, \citenamefont {Ahlers},\ and\ \citenamefont
		{Lohse}}]{Stevens:PRL2013}%
	\BibitemOpen
	\bibfield  {author} {\bibinfo {author} {\bibfnamefont {R.~J. A.~M.}\
			\bibnamefont {Stevens}}, \bibinfo {author} {\bibfnamefont {J.-Q.}\
			\bibnamefont {Zhong}}, \bibinfo {author} {\bibfnamefont {H.~J.~H.}\
			\bibnamefont {Clercx}}, \bibinfo {author} {\bibfnamefont {G.}~\bibnamefont
			{Ahlers}}, \ and\ \bibinfo {author} {\bibfnamefont {D.}~\bibnamefont
			{Lohse}},\ }\href@noop {} {\bibfield  {journal} {\bibinfo  {journal} {Phys.
				Rev. Lett.}\ }\textbf {\bibinfo {volume} {103}},\ \bibinfo {pages} {024503}
		(\bibinfo {year} {2009}{\natexlab{a}})}\BibitemShut {NoStop}%
	\bibitem [{\citenamefont {Kunnen}(2021)}]{Kunnen:JT2021}%
	\BibitemOpen
	\bibfield  {author} {\bibinfo {author} {\bibfnamefont {R.~P.~J.}\
			\bibnamefont {Kunnen}},\ }\href@noop {} {\bibfield  {journal} {\bibinfo
			{journal} {J. Turbulence}\ }\textbf {\bibinfo {volume} {22}},\ \bibinfo
		{pages} {267} (\bibinfo {year} {2021})}\BibitemShut {NoStop}%
	\bibitem [{\citenamefont {Ecke}\ and\ \citenamefont
		{Shishkina}(2023)}]{Ecke:ARFM2023}%
	\BibitemOpen
	\bibfield  {author} {\bibinfo {author} {\bibfnamefont {R.~E.}\ \bibnamefont
			{Ecke}}\ and\ \bibinfo {author} {\bibfnamefont {O.}~\bibnamefont
			{Shishkina}},\ }\href@noop {} {\bibfield  {journal} {\bibinfo  {journal}
			{Annual Review of Fluid Mechanics}\ }\textbf {\bibinfo {volume} {55}},\
		\bibinfo {pages} {603} (\bibinfo {year} {2023})}\BibitemShut {NoStop}%
	\bibitem [{\citenamefont {Zhong}\ \emph {et~al.}(1993)\citenamefont {Zhong},
		\citenamefont {Ecke},\ and\ \citenamefont {Steinberg}}]{Zhong:JFM1993}%
	\BibitemOpen
	\bibfield  {author} {\bibinfo {author} {\bibfnamefont {F.}~\bibnamefont
			{Zhong}}, \bibinfo {author} {\bibfnamefont {R.~E.}\ \bibnamefont {Ecke}}, \
		and\ \bibinfo {author} {\bibfnamefont {V.}~\bibnamefont {Steinberg}},\
	}\href@noop {} {\bibfield  {journal} {\bibinfo  {journal} {J. Fluid Mech.}\
		}\textbf {\bibinfo {volume} {249}},\ \bibinfo {pages} {135} (\bibinfo {year}
		{1993})}\BibitemShut {NoStop}%
	\bibitem [{\citenamefont {Liu}\ and\ \citenamefont {Ecke}(1997)}]{Liu:PRL1997}%
	\BibitemOpen
	\bibfield  {author} {\bibinfo {author} {\bibfnamefont {Y.}~\bibnamefont
			{Liu}}\ and\ \bibinfo {author} {\bibfnamefont {R.~E.}\ \bibnamefont {Ecke}},\
	}\href@noop {} {\bibfield  {journal} {\bibinfo  {journal} {Phys. Rev. Lett.}\
		}\textbf {\bibinfo {volume} {79}},\ \bibinfo {pages} {2257} (\bibinfo {year}
		{1997})}\BibitemShut {NoStop}%
	\bibitem [{\citenamefont {Kunnen}\ \emph {et~al.}(2006)\citenamefont {Kunnen},
		\citenamefont {Clercx},\ and\ \citenamefont {Geurts}}]{Kunnen:PRE2006}%
	\BibitemOpen
	\bibfield  {author} {\bibinfo {author} {\bibfnamefont {R.~P.~J.}\
			\bibnamefont {Kunnen}}, \bibinfo {author} {\bibfnamefont {H.~J.~H.}\
			\bibnamefont {Clercx}}, \ and\ \bibinfo {author} {\bibfnamefont {B.~J.}\
			\bibnamefont {Geurts}},\ }\href@noop {} {\bibfield  {journal} {\bibinfo
			{journal} {Phys. Rev. E}\ }\textbf {\bibinfo {volume} {74}},\ \bibinfo
		{pages} {056306} (\bibinfo {year} {2006})}\BibitemShut {NoStop}%
	\bibitem [{\citenamefont {Stevens}\ \emph {et~al.}(2013)\citenamefont
		{Stevens}, \citenamefont {Clercx},\ and\ \citenamefont
		{Lohse}}]{Stevens:EJMFB2013}%
	\BibitemOpen
	\bibfield  {author} {\bibinfo {author} {\bibfnamefont {R.~J.}\ \bibnamefont
			{Stevens}}, \bibinfo {author} {\bibfnamefont {H.~J.}\ \bibnamefont {Clercx}},
		\ and\ \bibinfo {author} {\bibfnamefont {D.}~\bibnamefont {Lohse}},\
	}\href@noop {} {\bibfield  {journal} {\bibinfo  {journal} {European J. of
				Mech. - B/Fluids}\ }\textbf {\bibinfo {volume} {40}},\ \bibinfo {pages} {41}
		(\bibinfo {year} {2013})}\BibitemShut {NoStop}%
	\bibitem [{\citenamefont {Wei}\ \emph {et~al.}(2015)\citenamefont {Wei},
		\citenamefont {Weiss},\ and\ \citenamefont {Ahlers}}]{Ping:PRL2015}%
	\BibitemOpen
	\bibfield  {author} {\bibinfo {author} {\bibfnamefont {P.}~\bibnamefont
			{Wei}}, \bibinfo {author} {\bibfnamefont {S.}~\bibnamefont {Weiss}}, \ and\
		\bibinfo {author} {\bibfnamefont {G.}~\bibnamefont {Ahlers}},\ }\href@noop {}
	{\bibfield  {journal} {\bibinfo  {journal} {Phys. Rev. Lett.}\ }\textbf
		{\bibinfo {volume} {114}},\ \bibinfo {pages} {114506} (\bibinfo {year}
		{2015})}\BibitemShut {NoStop}%
	\bibitem [{\citenamefont {Zhong}\ \emph {et~al.}(2009)\citenamefont {Zhong},
		\citenamefont {Stevens}, \citenamefont {Clercx}, \citenamefont {Verzicco},
		\citenamefont {Lohse},\ and\ \citenamefont
		{Ahlers}}]{Zhong:PRL2009RotatingRBC}%
	\BibitemOpen
	\bibfield  {author} {\bibinfo {author} {\bibfnamefont {J.-Q.}\ \bibnamefont
			{Zhong}}, \bibinfo {author} {\bibfnamefont {R.~J. A.~M.}\ \bibnamefont
			{Stevens}}, \bibinfo {author} {\bibfnamefont {H.~J.~H.}\ \bibnamefont
			{Clercx}}, \bibinfo {author} {\bibfnamefont {R.}~\bibnamefont {Verzicco}},
		\bibinfo {author} {\bibfnamefont {D.}~\bibnamefont {Lohse}}, \ and\ \bibinfo
		{author} {\bibfnamefont {G.}~\bibnamefont {Ahlers}},\ }\href@noop {}
	{\bibfield  {journal} {\bibinfo  {journal} {Phys. Rev. Lett.}\ }\textbf
		{\bibinfo {volume} {102}},\ \bibinfo {pages} {044502} (\bibinfo {year}
		{2009})}\BibitemShut {NoStop}%
	\bibitem [{\citenamefont {Yang}\ \emph {et~al.}(2020)\citenamefont {Yang},
		\citenamefont {Verzicco}, \citenamefont {Lohse},\ and\ \citenamefont
		{Stevens}}]{Yang:PRF2020}%
	\BibitemOpen
	\bibfield  {author} {\bibinfo {author} {\bibfnamefont {Y.}~\bibnamefont
			{Yang}}, \bibinfo {author} {\bibfnamefont {R.}~\bibnamefont {Verzicco}},
		\bibinfo {author} {\bibfnamefont {D.}~\bibnamefont {Lohse}}, \ and\ \bibinfo
		{author} {\bibfnamefont {R.~J. A.~M.}\ \bibnamefont {Stevens}},\ }\href@noop
	{} {\bibfield  {journal} {\bibinfo  {journal} {Phys. Rev. Fluids}\ }\textbf
		{\bibinfo {volume} {5}},\ \bibinfo {pages} {053501} (\bibinfo {year}
		{2020})}\BibitemShut {NoStop}%
	\bibitem [{\citenamefont {Vorobieff}\ and\ \citenamefont
		{Ecke}(2002)}]{Vorobieff:JFM2002}%
	\BibitemOpen
	\bibfield  {author} {\bibinfo {author} {\bibfnamefont {P.}~\bibnamefont
			{Vorobieff}}\ and\ \bibinfo {author} {\bibfnamefont {R.~E.}\ \bibnamefont
			{Ecke}},\ }\href@noop {} {\bibfield  {journal} {\bibinfo  {journal} {J. Fluid
				Mech.}\ }\textbf {\bibinfo {volume} {458}},\ \bibinfo {pages} {191–218}
		(\bibinfo {year} {2002})}\BibitemShut {NoStop}%
	\bibitem [{\citenamefont {Stevens}\ \emph
		{et~al.}(2009{\natexlab{b}})\citenamefont {Stevens}, \citenamefont {Zhong},
		\citenamefont {Clercx}, \citenamefont {Ahlers},\ and\ \citenamefont
		{Lohse}}]{Stevens:PRL2009}%
	\BibitemOpen
	\bibfield  {author} {\bibinfo {author} {\bibfnamefont {R.~J. A.~M.}\
			\bibnamefont {Stevens}}, \bibinfo {author} {\bibfnamefont {J.-Q.}\
			\bibnamefont {Zhong}}, \bibinfo {author} {\bibfnamefont {H.~J.~H.}\
			\bibnamefont {Clercx}}, \bibinfo {author} {\bibfnamefont {G.}~\bibnamefont
			{Ahlers}}, \ and\ \bibinfo {author} {\bibfnamefont {D.}~\bibnamefont
			{Lohse}},\ }\href@noop {} {\bibfield  {journal} {\bibinfo  {journal} {Phys.
				Rev. Lett.}\ }\textbf {\bibinfo {volume} {103}},\ \bibinfo {pages} {024503}
		(\bibinfo {year} {2009}{\natexlab{b}})}\BibitemShut {NoStop}%
	\bibitem [{\citenamefont {Stevens}\ \emph
		{et~al.}(2010{\natexlab{a}})\citenamefont {Stevens}, \citenamefont {Clercx},\
		and\ \citenamefont {Lohse}}]{Stevens:NJP2010}%
	\BibitemOpen
	\bibfield  {author} {\bibinfo {author} {\bibfnamefont {R.~J. A.~M.}\
			\bibnamefont {Stevens}}, \bibinfo {author} {\bibfnamefont {H.~J.~H.}\
			\bibnamefont {Clercx}}, \ and\ \bibinfo {author} {\bibfnamefont
			{D.}~\bibnamefont {Lohse}},\ }\href@noop {} {\bibfield  {journal} {\bibinfo
			{journal} {New J. Phys.}\ }\textbf {\bibinfo {volume} {12}},\ \bibinfo
		{pages} {075005} (\bibinfo {year} {2010}{\natexlab{a}})}\BibitemShut
	{NoStop}%
	\bibitem [{\citenamefont {Stevens}\ \emph
		{et~al.}(2010{\natexlab{b}})\citenamefont {Stevens}, \citenamefont {Clercx},\
		and\ \citenamefont {Lohse}}]{Stevens:POF2010}%
	\BibitemOpen
	\bibfield  {author} {\bibinfo {author} {\bibfnamefont {R.~J. A.~M.}\
			\bibnamefont {Stevens}}, \bibinfo {author} {\bibfnamefont {H.~J.~H.}\
			\bibnamefont {Clercx}}, \ and\ \bibinfo {author} {\bibfnamefont
			{D.}~\bibnamefont {Lohse}},\ }\href@noop {} {\bibfield  {journal} {\bibinfo
			{journal} {Phys. Fluids}\ }\textbf {\bibinfo {volume} {22}},\ \bibinfo
		{pages} {085103} (\bibinfo {year} {2010}{\natexlab{b}})}\BibitemShut
	{NoStop}%
	\bibitem [{\citenamefont {Stevens}\ \emph {et~al.}(2011)\citenamefont
		{Stevens}, \citenamefont {Overkamp}, \citenamefont {Lohse},\ and\
		\citenamefont {Clercx}}]{Stevens:PRE2011}%
	\BibitemOpen
	\bibfield  {author} {\bibinfo {author} {\bibfnamefont {R.~J. A.~M.}\
			\bibnamefont {Stevens}}, \bibinfo {author} {\bibfnamefont {J.}~\bibnamefont
			{Overkamp}}, \bibinfo {author} {\bibfnamefont {D.}~\bibnamefont {Lohse}}, \
		and\ \bibinfo {author} {\bibfnamefont {H.~J.~H.}\ \bibnamefont {Clercx}},\
	}\href@noop {} {\bibfield  {journal} {\bibinfo  {journal} {Phys. Rev. E}\
		}\textbf {\bibinfo {volume} {84}},\ \bibinfo {pages} {056313} (\bibinfo
		{year} {2011})}\BibitemShut {NoStop}%
	\bibitem [{\citenamefont {Joshi}\ \emph {et~al.}(2017)\citenamefont {Joshi},
		\citenamefont {Rajaei}, \citenamefont {Kunnen},\ and\ \citenamefont
		{Clercx}}]{Joshi:JFM2017}%
	\BibitemOpen
	\bibfield  {author} {\bibinfo {author} {\bibfnamefont {P.}~\bibnamefont
			{Joshi}}, \bibinfo {author} {\bibfnamefont {H.}~\bibnamefont {Rajaei}},
		\bibinfo {author} {\bibfnamefont {R.~P.~J.}\ \bibnamefont {Kunnen}}, \ and\
		\bibinfo {author} {\bibfnamefont {H.~J.~H.}\ \bibnamefont {Clercx}},\
	}\href@noop {} {\bibfield  {journal} {\bibinfo  {journal} {J. Fluid Mech.}\
		}\textbf {\bibinfo {volume} {830}},\ \bibinfo {pages} {R3} (\bibinfo {year}
		{2017})}\BibitemShut {NoStop}%
	\bibitem [{\citenamefont {Weiss}\ \emph {et~al.}(2010)\citenamefont {Weiss},
		\citenamefont {Stevens}, \citenamefont {Zhong}, \citenamefont {Clercx},
		\citenamefont {Lohse},\ and\ \citenamefont {Ahlers}}]{Weiss:PRL2010}%
	\BibitemOpen
	\bibfield  {author} {\bibinfo {author} {\bibfnamefont {S.}~\bibnamefont
			{Weiss}}, \bibinfo {author} {\bibfnamefont {R.~J. A.~M.}\ \bibnamefont
			{Stevens}}, \bibinfo {author} {\bibfnamefont {J.-Q.}\ \bibnamefont {Zhong}},
		\bibinfo {author} {\bibfnamefont {H.~J.~H.}\ \bibnamefont {Clercx}}, \bibinfo
		{author} {\bibfnamefont {D.}~\bibnamefont {Lohse}}, \ and\ \bibinfo {author}
		{\bibfnamefont {G.}~\bibnamefont {Ahlers}},\ }\href@noop {} {\bibfield
		{journal} {\bibinfo  {journal} {Phys. Rev. Lett.}\ }\textbf {\bibinfo
			{volume} {105}},\ \bibinfo {pages} {224501} (\bibinfo {year}
		{2010})}\BibitemShut {NoStop}%
	\bibitem [{\citenamefont {Weiss}\ \emph {et~al.}(2016)\citenamefont {Weiss},
		\citenamefont {Wei},\ and\ \citenamefont {Ahlers}}]{Weiss:PRE2016}%
	\BibitemOpen
	\bibfield  {author} {\bibinfo {author} {\bibfnamefont {S.}~\bibnamefont
			{Weiss}}, \bibinfo {author} {\bibfnamefont {P.}~\bibnamefont {Wei}}, \ and\
		\bibinfo {author} {\bibfnamefont {G.}~\bibnamefont {Ahlers}},\ }\href@noop {}
	{\bibfield  {journal} {\bibinfo  {journal} {Phys. Rev. E}\ }\textbf {\bibinfo
			{volume} {93}},\ \bibinfo {pages} {043102} (\bibinfo {year}
		{2016})}\BibitemShut {NoStop}%
	\bibitem [{\citenamefont {Chong}\ \emph {et~al.}(2017)\citenamefont {Chong},
		\citenamefont {Yang}, \citenamefont {Huang}, \citenamefont {Zhong},
		\citenamefont {Stevens}, \citenamefont {Verzicco}, \citenamefont {Lohse},\
		and\ \citenamefont {Xia}}]{Chong:PRL2017}%
	\BibitemOpen
	\bibfield  {author} {\bibinfo {author} {\bibfnamefont {K.~L.}\ \bibnamefont
			{Chong}}, \bibinfo {author} {\bibfnamefont {Y.}~\bibnamefont {Yang}},
		\bibinfo {author} {\bibfnamefont {S.-D.}\ \bibnamefont {Huang}}, \bibinfo
		{author} {\bibfnamefont {J.-Q.}\ \bibnamefont {Zhong}}, \bibinfo {author}
		{\bibfnamefont {R.~J. A.~M.}\ \bibnamefont {Stevens}}, \bibinfo {author}
		{\bibfnamefont {R.}~\bibnamefont {Verzicco}}, \bibinfo {author}
		{\bibfnamefont {D.}~\bibnamefont {Lohse}}, \ and\ \bibinfo {author}
		{\bibfnamefont {K.-Q.}\ \bibnamefont {Xia}},\ }\href@noop {} {\bibfield
		{journal} {\bibinfo  {journal} {Phys. Rev. Lett.}\ }\textbf {\bibinfo
			{volume} {119}},\ \bibinfo {pages} {064501} (\bibinfo {year}
		{2017})}\BibitemShut {NoStop}%
	\bibitem [{\citenamefont {Davidson}(2013)}]{Davidson:book:TurbulenceRotating}%
	\BibitemOpen
	\bibfield  {author} {\bibinfo {author} {\bibfnamefont {P.~A.}\ \bibnamefont
			{Davidson}},\ }\href@noop {} {\emph {\bibinfo {title} {{Turbulence in
					Rotating, Stratified and Electrically Conducting Fluids}}}}\ (\bibinfo
	{publisher} {Cambridge University Press},\ \bibinfo {address} {Cambridge},\
	\bibinfo {year} {2013})\BibitemShut {NoStop}%
	\bibitem [{\citenamefont {Julien}\ \emph {et~al.}(1996)\citenamefont {Julien},
		\citenamefont {Legg}, \citenamefont {Mcwilliams},\ and\ \citenamefont
		{Werne}}]{Julien:JFM1996}%
	\BibitemOpen
	\bibfield  {author} {\bibinfo {author} {\bibfnamefont {K.}~\bibnamefont
			{Julien}}, \bibinfo {author} {\bibfnamefont {S.}~\bibnamefont {Legg}},
		\bibinfo {author} {\bibfnamefont {J.}~\bibnamefont {Mcwilliams}}, \ and\
		\bibinfo {author} {\bibfnamefont {J.}~\bibnamefont {Werne}},\ }\href@noop {}
	{\bibfield  {journal} {\bibinfo  {journal} {J. Fluid Mech.}\ }\textbf
		{\bibinfo {volume} {322}},\ \bibinfo {pages} {243–273} (\bibinfo {year}
		{1996})}\BibitemShut {NoStop}%
	\bibitem [{\citenamefont {Huang}\ \emph {et~al.}(2013)\citenamefont {Huang},
		\citenamefont {Kaczorowski}, \citenamefont {Ni},\ and\ \citenamefont
		{Xia}}]{Huang:PRL2013}%
	\BibitemOpen
	\bibfield  {author} {\bibinfo {author} {\bibfnamefont {S.-D.}\ \bibnamefont
			{Huang}}, \bibinfo {author} {\bibfnamefont {M.}~\bibnamefont {Kaczorowski}},
		\bibinfo {author} {\bibfnamefont {R.}~\bibnamefont {Ni}}, \ and\ \bibinfo
		{author} {\bibfnamefont {K.-Q.}\ \bibnamefont {Xia}},\ }\href@noop {}
	{\bibfield  {journal} {\bibinfo  {journal} {Phys. Rev. Lett.}\ }\textbf
		{\bibinfo {volume} {111}},\ \bibinfo {pages} {104501} (\bibinfo {year}
		{2013})}\BibitemShut {NoStop}%
	\bibitem [{\citenamefont {Kunnen}\ \emph {et~al.}(2016)\citenamefont {Kunnen},
		\citenamefont {Ostilla-Mónico}, \citenamefont {van~der Poel}, \citenamefont
		{Verzicco},\ and\ \citenamefont {Lohse}}]{Kunnen:JFM2016}%
	\BibitemOpen
	\bibfield  {author} {\bibinfo {author} {\bibfnamefont {R.~P.~J.}\
			\bibnamefont {Kunnen}}, \bibinfo {author} {\bibfnamefont {R.}~\bibnamefont
			{Ostilla-Mónico}}, \bibinfo {author} {\bibfnamefont {E.~P.}\ \bibnamefont
			{van~der Poel}}, \bibinfo {author} {\bibfnamefont {R.}~\bibnamefont
			{Verzicco}}, \ and\ \bibinfo {author} {\bibfnamefont {D.}~\bibnamefont
			{Lohse}},\ }\href@noop {} {\bibfield  {journal} {\bibinfo  {journal} {J.
				Fluid Mech.}\ }\textbf {\bibinfo {volume} {799}},\ \bibinfo {pages}
		{413–432} (\bibinfo {year} {2016})}\BibitemShut {NoStop}%
	\bibitem [{SM:()}]{SM:ThisPaper}%
	\BibitemOpen
	\href@noop {} {\enquote {\bibinfo {title} {{See Supplemental Material at [URL
					will be inserted by publisher] for the discussion related to the simulations
					and the solvers}},}\ }\BibitemShut {NoStop}%
	\bibitem [{\citenamefont {Stellmach}\ \emph {et~al.}(2014)\citenamefont
		{Stellmach}, \citenamefont {Lischper}, \citenamefont {Julien}, \citenamefont
		{Vasil}, \citenamefont {Cheng}, \citenamefont {Ribeiro}, \citenamefont
		{King},\ and\ \citenamefont {Aurnou}}]{Stellmach:PRL2014}%
	\BibitemOpen
	\bibfield  {author} {\bibinfo {author} {\bibfnamefont {S.}~\bibnamefont
			{Stellmach}}, \bibinfo {author} {\bibfnamefont {M.}~\bibnamefont {Lischper}},
		\bibinfo {author} {\bibfnamefont {K.}~\bibnamefont {Julien}}, \bibinfo
		{author} {\bibfnamefont {G.}~\bibnamefont {Vasil}}, \bibinfo {author}
		{\bibfnamefont {J.~S.}\ \bibnamefont {Cheng}}, \bibinfo {author}
		{\bibfnamefont {A.}~\bibnamefont {Ribeiro}}, \bibinfo {author} {\bibfnamefont
			{E.~M.}\ \bibnamefont {King}}, \ and\ \bibinfo {author} {\bibfnamefont
			{J.~M.}\ \bibnamefont {Aurnou}},\ }\href@noop {} {\bibfield  {journal}
		{\bibinfo  {journal} {Phys. Rev. Lett.}\ }\textbf {\bibinfo {volume} {113}},\
		\bibinfo {pages} {254501} (\bibinfo {year} {2014})}\BibitemShut {NoStop}%
	\bibitem [{\citenamefont {Hartmann}\ \emph {et~al.}(2023)\citenamefont
		{Hartmann}, \citenamefont {Yerragolam}, \citenamefont {Verzicco},
		\citenamefont {Lohse},\ and\ \citenamefont {Stevens}}]{Hartmann:arxiv2023}%
	\BibitemOpen
	\bibfield  {author} {\bibinfo {author} {\bibfnamefont {R.}~\bibnamefont
			{Hartmann}}, \bibinfo {author} {\bibfnamefont {G.~S.}\ \bibnamefont
			{Yerragolam}}, \bibinfo {author} {\bibfnamefont {R.}~\bibnamefont
			{Verzicco}}, \bibinfo {author} {\bibfnamefont {D.}~\bibnamefont {Lohse}}, \
		and\ \bibinfo {author} {\bibfnamefont {R.~J. A.~M.}\ \bibnamefont
			{Stevens}},\ }\href@noop {} {\  (\bibinfo {year} {2023})},\ \Eprint
	{http://arxiv.org/abs/2305.02127} {arXiv:2305.02127 [physics.flu-dyn]}
	\BibitemShut {NoStop}%
	\bibitem [{\citenamefont {Niemela}\ \emph {et~al.}(2010)\citenamefont
		{Niemela}, \citenamefont {Babuin},\ and\ \citenamefont
		{Sreenivasan}}]{Niemela:JFM2010}%
	\BibitemOpen
	\bibfield  {author} {\bibinfo {author} {\bibfnamefont {J.~J.}\ \bibnamefont
			{Niemela}}, \bibinfo {author} {\bibfnamefont {S.}~\bibnamefont {Babuin}}, \
		and\ \bibinfo {author} {\bibfnamefont {K.~R.}\ \bibnamefont {Sreenivasan}},\
	}\href@noop {} {\bibfield  {journal} {\bibinfo  {journal} {J. Fluid Mech.}\
		}\textbf {\bibinfo {volume} {649}},\ \bibinfo {pages} {509} (\bibinfo {year}
		{2010})}\BibitemShut {NoStop}%
	\bibitem [{\citenamefont {Ecke}\ and\ \citenamefont
		{Niemela}(2014)}]{Ecke:PRL2014}%
	\BibitemOpen
	\bibfield  {author} {\bibinfo {author} {\bibfnamefont {R.~E.}\ \bibnamefont
			{Ecke}}\ and\ \bibinfo {author} {\bibfnamefont {J.~J.}\ \bibnamefont
			{Niemela}},\ }\href@noop {} {\bibfield  {journal} {\bibinfo  {journal} {Phys.
				Rev. Lett.}\ }\textbf {\bibinfo {volume} {113}},\ \bibinfo {pages} {114301}
		(\bibinfo {year} {2014})}\BibitemShut {NoStop}%
\end{thebibliography}%




\end{document}



