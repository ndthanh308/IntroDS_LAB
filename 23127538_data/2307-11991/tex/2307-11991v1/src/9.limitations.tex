

\section{Limitations and Future Works}


\subsection{Data Collections}

Several strategies can be implemented in future work to overcome the limitations in data collection. Firstly, to address the issue of anti-crawler rules on different websites, developing a more robust and adaptable crawler that can handle different anti-crawler mechanisms would be beneficial. This could involve implementing dynamic IP rotation or utilising proxies to avoid IP blocking. Additionally, employing machine learning techniques, such as automatic rule extraction or rule adaptation, could help automate the process of handling anti-crawler mechanisms.

Incorporating more advanced data-cleaning techniques can also improve the quality of the crawled data. This may involve NLP methods such as entity recognition, part-of-speech tagging, and named entity recognition, to identify and filter out irrelevant or noisy data. Additionally, leveraging machine learning algorithms, such as anomaly detection or outlier detection, can aid in identifying and removing low-quality or erroneous data points.
In terms of dataset standardisation, establishing a unified standard for data generation in the online domain would greatly facilitate the cleaning process. This could involve collaborating with website administrators or data providers to develop guidelines or formats for data representation. Furthermore, using human annotators or experts in the domain to review and clean a subset of the dataset manually can provide valuable insights and ensure a higher-quality dataset.

However, it is important to acknowledge that achieving a completely clean dataset is challenging, particularly when dealing with large-scale datasets. As such, future work should aim to strike a balance between manual review and automated cleaning techniques, while also considering the cost and scalability of the data cleaning process.

\subsection{Model Improvement}

Increasing the scale of model training by utilising larger models or ensembles of models can potentially enhance the performance and capabilities of the chatbot. Larger models can capture more nuanced patterns and relationships in the data, leading to more accurate and coherent responses.
Exploring different model architectures beyond autoregressive language models may provide valuable insights. Bidirectional models, such as the Transformer-XL, or models that incorporate external knowledge sources, such as knowledge graphs, can potentially improve the chatbot's contextual understanding and generate more informative responses.
Moreover, integrating feedback mechanisms into the training process can help iteratively improve the chatbot's performance. This could involve collecting user feedback on the generated responses and incorporating it into the model training through techniques like reinforcement learning or active learning.

Several disadvantages were also identified in the LLM architecture. Firstly, the maximum likelihood training approach of the \emph{WenZhong} model is susceptible to exposure bias, which occurs when samples are drawn from the target language distribution. This bias can lead to errors that researchers have yet to find effective solutions for. Additionally, training the \emph{WenZhong} model multiple times can result in a significant decrease in its quality.
Furthermore, the \emph{WenZhong} model follows an autoregressive architecture, focusing on modelling joint probability from left to right. This unidirectional training process limits its ability to capture information from all contexts, particularly hindering its performance in tasks requiring reading comprehension that rely on contextual background references.
Similar to the \emph{WenZhong} model, the \emph{PanGu} model also exhibits autoregressive characteristics. Although it inherits the ability to estimate the joint probability of language models, it suffers from the same limitations of unidirectional modelling. It lacks bidirectional context information and may produce duplicate results that require deduplication to resolve.

We also have reservations about the Jieba tokenizer used in the \emph{PanGu} model. Its performance and tokenization ability do not accurately handle complex Chinese tokenization. Furthermore, as neural networks and pre-trained models advance, Chinese NLP tasks increasingly demonstrate that tokenization is not always necessary. Large models can effectively learn character-to-character relationships without word segmentation. For instance, Google is considering discarding tokenization and using bytes directly. In our experience, adopting a more flexible tokenizer could make the model more suitable for various industrial applications, even at the cost of sacrificing some performance.


\subsection{User Experience and User Interface}
Enhancing the user experience and user interface of the chatbot can significantly impact its adoption and effectiveness. Future work should focus on improving the simplicity, intuitiveness, and accessibility of the website interface. This includes optimizing response times, refining the layout and design, and incorporating user-friendly features such as autocomplete suggestions or natural language understanding capabilities.

Furthermore, providing personalised recommendations and suggestions to users based on their preferences and previous interactions can enhance the user experience. Implementing techniques like collaborative filtering or user profiling can enable the chatbot to better understand and cater to individual user needs.
Usability testing and user feedback collection should be conducted regularly to gather insights on user preferences, pain points, and suggestions for improvement. Iterative design and development based on user-centered principles can ensure that the chatbot meets user expectations and effectively addresses their mental health support needs.


\subsection{Ethical Considerations and User Privacy}
As with any AI-based system, ethical considerations and user privacy are of utmost importance. Future work should prioritise addressing these concerns by implementing robust privacy protection mechanisms and ensuring transparency in data usage. This includes obtaining explicit user consent for data collection and usage, anonymizing sensitive user information, and implementing strict data access controls.
Developing mechanisms to handle potentially sensitive or harmful user queries is crucial. The chatbot should have appropriate safeguards and guidelines to avoid providing inaccurate or harmful advice. Integrating a reporting system where users can report problematic responses or seek human intervention can help mitigate potential risks.
Furthermore, ongoing monitoring and auditing of the chatbot's performance and behavior can help identify and rectify biases or discriminatory patterns. Regular evaluations by domain experts and user feedback analysis can improve the chatbot's reliability, fairness, and inclusivity.



While this project has made significant progress in developing an AI-based chatbot for mental health support, there are various limitations and areas for improvement. Overcoming challenges related to data quality, model performance, ethical considerations, and user experience will contribute to the overall effectiveness and reliability of the chatbot. By addressing these limitations and exploring future research directions, we can continue to advance the field of AI-powered mental health support systems and provide valuable assistance to individuals in need.


\section{Conclusion}

In conclusion, our project on Psy-LLM, an online psychological consultation platform, has been successfully completed and implemented. While there are areas identified for improvement based on specific evaluation indicators, we are confident that with improved equipment conditions, we can enhance the performance of this platform. We believe that the experimental results obtained from this project hold significant potential in contributing to the fields of supportive natural language generation and psychology, driving advancements at the intersection of these domains. We are optimistic that the deployment of such a system offers a practical approach to promoting the overall mental well-being of our society by providing timely responses and support to those who are in need.





