

\newcommand\citewithauthor[1] 
 {\citeauthor{#1}~(\citeyear{#1})~\citep{#1}}




\section{Related Works}

In recent years, there has been increasing interest in utilising AI for tackling difficult problems in traditional domains like 
adopting AI in the construction industry~\citep{regona2022opportunities},
localisation in robotic applications~\citep{lai2022slamreview},
assistance systems in the service sector~\citep{link2020use},
financial forecast~\citep{forexNonStationaryTimeSeries},
improving workflow in the oil and gas industry~\citep{koroteev2021artificial},
planning and scheduling~\citep{lai2022MEP},
monitoring ocean contamination~\citep{xu2022waterSedimentML},
remote sensing for search and rescue~\citep{lai2023UAV},
and even used in the life cycle of material discovery~\citep{li2020ai}.
Health care industry has been adopting AI-based machine-learning techniques for classifying medical images~\citep{castiglioni2021ai},
guiding cancer diagnosis~\citep{chugh2021survey},
as screening tools for diabetes~\citep{sensorsMLforDiabetes},
and ultimately improve the clinical workflow in the practice of medicine~\citep{brattain2018machine}.

One area of research focuses on using conversational agents, also known as chatbots, for mental health support. Chatbots have the potential to provide accessible and cost-effective assistance to individuals in need. 
For example, \citewithauthor{martinengo2022evaluation} qualitatively analysed user-conversational agent and found that these type of chapbots can offer anonymous, empathetic, and non-judgemental interactions that align face-to-face psychotherapy. chatbot can utilise NLP techniques to engage users in therapeutic conversations and provide personalised support. Results showed promising outcomes, indicating the potential effectiveness of chatbots in delivering mental health interventions~\citep{denecke2021artificial}.
Pre-trained language models have also gained attention in the field of mental health counselling. These models, such as GPT-3~\citep{brown2020language}, provide a foundation for generating human-like responses to user queries. \citewithauthor{wang2023prompt} explored the application of LLMs in providing mental health counselling. They found that LLMs demonstrated a certain level of understanding and empathy, providing responses that were perceived as helpful by users. However, limitations in controlling the model's output and ensuring ethical guidelines were highlighted.

Furthermore, there is a growing body of research on using NLP techniques to analyse mental health-related text data~\citep{gonzalez2017capturing}. Researchers have applied machine learning algorithms to detect mental health conditions~\citep{abd2020application}, predict suicidal ideation~\citep{ji2020suicidal}, and identify linguistic markers associated with psychological well-being~\citep{akstinaite2022identifying}. For instance, \citewithauthor{de2013predicting} analysed social media data to predict depression among individuals. By extracting linguistic features and using machine learning classifiers, they achieved promising results in identifying individuals at risk of depression.
Additionally, several studies have investigated the integration of modern technologies into existing mental health interventions. For instance, \citewithauthor{lui2017evidence} investigates the use of mobile applications to support the delivery of psychotherapy.

\citewithauthor{shaikh2022autonomous} developed a friendly AI-based chatbot using deep learning and artificial intelligence techniques. The chatbot aimed to help individuals with insomnia by addressing harmful feelings and increasing interactions with users as they experienced sadness and anxiety.
In another line of research, chatbots have been extensively studied in the domain of customer service. Many companies have adopted chatbots to assist customers in making purchases and understanding products. These chatbots provide prompt replies, enhancing customer satisfaction\citep{9885724}.
Furthermore, advancements in language models such as BERT and GPT have been influential in the development of conversational chatbots. Researchers have leveraged BERT-based question-answering models to improve the accuracy and efficiency of chatbot responses\citep{9652153}. The GPT models, including GPT-2 and GPT-3, have introduced innovations such as zero-shot and few-shot learning, significantly expanding their capabilities in generating human-like text\citep{Brown2020LanguageMA}. However, limitations in generating coherent and contextual responses and the interpretability of the models have been identified. The model incorporated a 48-layer Transformer stack and achieved a parameter count of 1.5 billion, resulting in enhanced generalisation abilities\citep{Brown2020LanguageMA}.

In summary, previous work in the field of AI and NLP for mental health support has demonstrated the potential of chatbots, pre-trained language models, and data analysis techniques. These approaches offer new avenues for delivering accessible and personalised mental health interventions. Nonetheless, further research is needed to address ethical, privacy, and reliability issues and to optimise the integration of AI technologies into existing counselling practices.





