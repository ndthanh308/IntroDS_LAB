\section{Introduction}

The field of AI utilising dialogue technology has witnessed significant growth, particularly in the domain of automatic chatbot and ticket support systems~\citep{handoyo2018ticketing}.
This application of dialogue technology has emerged as a cutting-edge and increasingly popular approach in the realm of AI-powered support systems.
With changing global dynamics, the severity of the ongoing pandemic, and an upsurge in psychological challenges faced by the public, the mental well-being of young individuals, in particular, is a cause for concern.
The pressures of urbanisation and the internet have led to various psychological issues~\citep{trivedi2008rapid}, including depression, procrastination, anxiety, obsessive-compulsive disorder, and social phobia~\citep{tian2020psychological}, which have become prevalent ailments of our time.

Psychological counselling involves the utilisation of psychological methods to provide assistance to individuals experiencing difficulties in psychological adaptation and seeking solutions.
The demand for psychological counselling has witnessed a significant surge in recent years~\citep{chen2022mental}, while the availability of professional psychological consultants remains insufficient.
The profession of psychological consulting imposes high standards and qualifications.
For instance, registered psychologists within Psychological Associations require students to possess a master's degree in psychology-related disciplines, undergo a minimum of 150 hours of direct counselling, and receive face-to-face supervision by registered supervisors for no less than 100 hours~\citep{gay2021school}.
Additionally, the burnout rate among mental health professionals further exacerbates this shortage~\citep{joshi2020burnout}.


In 2020, the global outbreak of COVID-19 exacerbated the need for timely and professional psychological counselling due to the tremendous stress it imposed on society~\citep{kontoangelos2020mental}.
Consequently, online psychological counselling through the Internet has progressively become the dominant mode of delivering counselling services~\citep{yurayat2023university}.
AI-based psychological counselling not only addresses the severe supply-demand gap in the consulting industry but also enhances the responsiveness of online psychological counselling services, thereby promoting the implementation of mental health strategies.

In light of these circumstances, our team is determined to develop a mental health consulting framework to serves as a constant source of support.
Creating an AI-powered framework can allow users to engage with it comfortably, given its non-human identity, thereby reducing feelings of shame among users~\citep{Prochaska_2021}. Amid the challenges posed by the pandemic, online psychological counselling has proven instrumental and has gradually become the predominant form of counselling. However, the growing disparity between supply and demand within our society's psychological consultation industry is a pressing concern. The application of AI technology to mental health and psychological counselling is an emerging and promising field. Conversation frameworks, chatbots, and virtual agents are computer programs that simulate human conversation~\citep{deryugina2010chatterbots}. They can engage in natural and effective interactions with individuals, providing them with emotional experiences through the incorporation of emotional and human-like characteristics. In practical terms, dialogue frameworks hold significant potential for replacing online consultations and addressing supply-demand imbalances.



In this study, we propose an AI-based \textbf{Psy}chological Support with \textbf{L}arge \textbf{L}anguage \textbf{M}odels (\textbf{Psy-LLM}) framework designed for question-answering, with the purpose of providing online consultation services to alleviate the demand for mental health professionals during pandemics and beyond.
Psy-LLM is an online psychological consultation model that is pre-trained with Large Language Models (LLMs) and futher trained with Q\&A from professional psychologists and large-scale crawled psychological articles.
The framework can give professional answers to users' requests for psychological consultation.
Our model is built upon large-scale pre-training corpus models, specifically \emph{PanGu}~\citep{Zeng2021PanGuLA} and \emph{WenZhong}~\citep{fengshenbang}. The \emph{PanGu} model, developed by Huawei's Pengcheng Laboratory, and the \emph{WenZhong} model, developed by the Idea Research Institute, served as the basis for our work. For data acquisition, we collected a substantial number of Chinese psychological articles from public websites. Additionally, we obtained permission from the Artificial Intelligence Research Institute of Tsinghua University to utilise the PsyQA dataset, which comprises a large number of question-answer pairs related to psychological counselling. Each answer in the dataset was reviewed and adjusted by professionals holding master's degrees or above in psychological counselling to ensure its quality. In downstream tasks, we fine-tuned the model using the acquired dataset and PsyQA~\citep{psyqa}. As part of the evaluation process, we established a dedicated website and deployed the fine-tuned model on a server, allowing users to provide timely ratings. Based on the scoring results, we iteratively refined and re-fine-tuned the model.

Our contribution includes proposing a framework for AI-based psychological consultation framework and an empirical study on its effectiveness. We have achieved the successful development of a mental health consulting model that effectively provides clear and professional responses to users' psychological inquiries. Empirically, we have tested deploying the model on a server and the model can provide responses to users within seconds.
Our framework has the potential to offer a practical tool for professionals to efficiently screen and promptly respond to individuals in urgent need of mental support, thereby addressing and alleviating pressing demands within the healthcare industry.
