\pdfoutput=1

\documentclass[dvipsnames]{article}
\usepackage{arxiv}

\usepackage{amsmath,amssymb,amsthm,amsfonts,mathtools}
% \usepackage{algorithmic}
\usepackage{graphicx}
\usepackage{textcomp}
\usepackage{braket}
\usepackage{booktabs}
\usepackage{multirow}
\usepackage{xspace}

\usepackage{bm}

\usepackage{csquotes}


% fonts math
\DeclareMathAlphabet{\mathdutchcal}{U}{dutchcal}{m}{n}
\SetMathAlphabet{\mathdutchcal}{bold}{U}{dutchcal}{b}{n}
\DeclareMathAlphabet{\mathdutchbcal}{U}{dutchcal}{b}{n}
%
\DeclareMathAlphabet{\mathpzc}{OT1}{pzc}{m}{it}


% \usepackage{thmtools}
% \declaretheoremstyle[
%     % spaceabove=6pt, spacebelow=6pt,
%     headfont=\scshape\bfseries,
%     notefont=\normalfont\bfseries, notebraces={(}{)},
%     bodyfont=\itshape,
%     postheadspace=1em,
%     qed=\qedsymbol
% ]{mystyle}
% \theoremstyle{mystyle}

    \newtheorem{definition}{Definition}
    \newtheorem{assumption}{Assumption}
    \newtheorem{problem}{Problem}
    \newtheorem{theorem}{Theorem}
    \newtheorem{lemma}{Lemma}
    \newtheorem{corollary}{Corollary}


% \usepackage{pgfplots}
% \usepackage{pgfplotstable}
% \pgfplotsset{compat=1.7}
% \usepackage{tikz}
% % ,pgfplots
% \usetikzlibrary{shapes,shapes.geometric,arrows.meta,positioning,matrix}
% % \tikzset{>=Stealth}

% %%%%%%%%%%%%%%%%%%%%%%%%%%%%%%%%%%%%%%%%%%%%%%%%%%%%%%%%%%%%
%   \NewSpotColorSpace{PANTONE}
%   \AddSpotColor{PANTONE} {PANTONE3015C} {PANTONE\SpotSpace 3015\SpotSpace C} {1 0.3 0 0.2}
%   \SetPageColorSpace{PANTONE}%
% %%%%%%%%%%%%%%%%%%%%%%%%%%%%%%%%%%%%%%%%%%%%%%%%%%%%%%%%%%%%

% %%%%%
% % fix ieee access missing color
%   \NewSpotColorSpace{PANTONE}
%   \AddSpotColor{PANTONE} {PANTONE3015C} {PANTONE\SpotSpace 3015\SpotSpace C} {1 0.3 0 0.2}
%   \SetPageColorSpace{PANTONE}%
% %%%%%


% \usepackage{caption,setspace} % for subfigure captions
% \captionsetup{font={sf,small,stretch=0.80},labelfont={bf,color=accessblue}} % for setting figure caption same style as ieee access
\makeatletter
\let\MYcaption\@makecaption
\makeatother
\usepackage{subcaption} % multiple figures
% \usepackage[font=footnotesize]{subcaption} % multiple figures
\makeatletter
\let\@makecaption\MYcaption
\makeatother

% \usepackage[caption=false]{subfig} % for subfigure





\usepackage[
    backend=biber
    % ,giveninits=false
  % ,giveninits=true                  % show only initials in ref
  ,url=false, isbn=false, doi=false % Remove unnecessary fields
  , maxnames=4
  , maxbibnames=99
  , minnames=3
  %,eprint=false            % Supress JSTOR, arXiv etc.
%   ,sortcites                        % sort multiple citations in text (eg. [3,1,6] =>[1,3,6])
  % ,sorting=none             % citation number appear as they are used
  ,date=year                % ony show year & ignore month or day field
  ,labeldate=year
  % ,style=ieee
  , dashed=false
  , uniquename=false
  , style=authoryear
  ]{biblatex}
% \AtEveryBibitem{                      % ONLY get rid of JSTOR (leave arXiv alone)
%   \iffieldequalstr{eprinttype}{jstor}
%   {\clearfield{eprint}}
%   {}
%   \clearfield{urlyear}
%   \clearfield{urlmonth}
%   \clearfield{url}
% }
\renewrobustcmd*{\bibinitdelim}{\,} % this controls the gap between each initials. "\," puts a thin space (def: ~ (full space))
\addbibresource{refs.bib}
% \addbibresource{ref.bib}
\newcommand*{\autociteauthor}[1]{\citeauthor*{#1}~\autocite{#1}} % combine autocite with author

% \newcommand{\cite}[1]{\autocite{#1}}

\usepackage{lipsum}

% \usepackage{algorithmic}
% \usepackage{graphicx} % for figure


%----------------------------------------------------------------------%
%  Custom algpseudocode commands                                       %
%----------------------------------------------------------------------%
\usepackage[boxed,ruled,vlined,linesnumbered]{algorithm2e}
\DontPrintSemicolon
\SetKwProg{Fn}{function}{}{}
\SetKwFunction{FnSampleFree}{SampleFree}
\SetKwFunction{FnRestartArm}{RestartArm}
\SetKwFunction{FnPickArm}{PickArm}
\SetKwFunction{FnRewire}{Rewire}
\SetKwFunction{FnNearest}{Nearest}
\SetKwFunction{FnRestartArm}{RestartArm}
\SetKwComment{Comment}{$\triangleright$\ }{}
\SetKwInput{KwInit}{Initialise}

%----------------------------------------------------------------------%
%  Custom cleveref commands                                       %
%----------------------------------------------------------------------%
\usepackage{cleveref}                                        % Better ref!
\crefname{assumption}{assumption}{assumptions}
\crefname{problem}{problem}{problems}
\crefname{algorithm}{Alg.}{Algs.}
\Crefname{algorithm}{Algorithm}{Algorithms}
\crefname{figure}{Fig.}{Figs.} % ieee figure must be cap
\crefformat{equation}{(#2#1#3)}
\crefrangeformat{equation}{(#3#1#4) to~(#5#2#6)}
\crefmultiformat{equation}{(#2#1#3)}%
{ and~(#2#1#3)}{, (#2#1#3)}{ and~(#2#1#3)}


\usepackage{microtype}


\microtypesetup{activate={true,nocompatibility},final,tracking=true,kerning=true,factor=1100,stretch=10,shrink=10}
% remove hyperref for non-numerical cite
\usepackage{etoolbox}
\makeatletter
\pretocmd{\NAT@citexnum}{\@ifnum{\NAT@ctype>\z@}{\let\NAT@hyper@\relax}{}}{}{}
\makeatother


%%%%%%%%%%%%%%%%%%%%%%%%%%%%%%%%%%%%%%%%%%%%%%%%%%%%%%%%%%%%%%%%%
% \input{_notations}
%%%%%%%%%%%%%%%%%%%%%%%%%%%%%%%%%%%%%%%%%%%%%%%%%%%%%%%%%%%%%%%%%








\begin{document}



\newcommand{\shortheadtitle}{
  Psy-LLM: Large Language Models for Mental Health Psychological Services
  }


\title{
  Psy-LLM: Scaling up Global Mental Health Psychological Services with \\ AI-based Large Language Models
  }

\author{
    Tin Lai\thanks{%
    Correspondence: \texttt{tin.lai@sydney.edu.au}
}%
\And
    Yukun Shi%
\And
    Zicong Du%
\And
    Jiajie Wu%
\And
    Ken Fu%
\And
    Yichao Dou%
\And
    Ziqi Wang% 
\AND%
\normalfont
  School of Computer Science\\
  The University of Sydney\\
  Australia
}



\maketitle

\begin{abstract}
    The demand for psychological counseling has grown significantly in recent years, particularly with the global outbreak of COVID-19, which has heightened the need for timely and professional mental health support.
    Online psychological counseling has emerged as the predominant mode of providing services in response to this demand.
    In this study, we propose the Psy-LLM framework, an AI-based system leveraging Large Language Models (LLMs) for question-answering in online psychological consultation.
    Our framework combines pre-trained LLMs with real-world professional Q\&A from psychologists and extensively crawled psychological articles.
    The Psy-LLM framework serves as a front-end tool for healthcare professionals, allowing them to provide immediate responses and mindfulness activities to alleviate patient stress.
    Additionally, it functions as a screening tool to identify urgent cases requiring further assistance.
    We evaluated the framework using intrinsic metrics, such as perplexity, and extrinsic evaluation metrics, with human participant assessments of response helpfulness, fluency, relevance, and logic.
    The results demonstrate the effectiveness of the Psy-LLM framework in generating coherent and relevant answers to psychological questions.
    This article concludes by discussing the potential of large language models to enhance mental health support through AI technologies in online psychological consultation.%
\end{abstract}


\newcommand{\citep}{\autocite}


% ######################################################
% Cycling
% ######################################################
To promote sustainability, cities worldwide are promoting a transition to public transportation and active transportation. From these, cycling has proven to provide numerous advantages, including benefits to health \cite{gotschi2016cycling}, economy \cite{clifton2013examining}, and reduction of carbon emissions \cite{NEVES2019130}. Despite these benefits, cycling numbers remain predominantly low in some cities. In contrast, barriers to cycling include hilliness, lack of cycling infrastructure, or appropriate bike storage or parking. Yet, the main deterrent to cycling relates to safety concerns \cite{aldred2015investigating, lawson2013perception, felix2019maturing}. If cyclists feel unsafe or are afraid to cycle, they will prefer other means of transportation. 


% ######################################################
% Perception of Safety
% ######################################################
Thus, for cities aiming to boost cycling numbers and the effectiveness of such strategies, it is increasingly important to understand what affects individuals' perceptions. Perception of cycling safety research explores how individuals subjectively experience cycling accident risk and what fears and events negatively impact one's perception of being involved in a cycling accident. Current research shows that infrastructure layout, fear of traffic, and distracted cycling are some aspects that influence this perception \cite{heinen2010commuting}. Most research focuses on surveys and in-loco and post-riding interviews to compare factors influencing perceptions \cite{sanders2015perceived}. Even though these approaches are vital to understanding cycling perception of safety, they need to be more scalable over space or time due to their high cost (human resources, time, and money). This prevents any analysis of perceptions over time, and qualitative non-scalable data analysis hampers any comparative study across cities or countries.


% !BIB TS-program =
\documentclass[12pt]{article}
\usepackage{color}
\usepackage{amsfonts,amssymb,amsmath}
\usepackage[export]{adjustbox}
\makeatletter
\setlength{\@fptop}{0pt}
 \makeatother
\usepackage{graphicx}
\usepackage[T1]{fontenc}
\usepackage[numbers,sort&compress]{natbib}
\graphicspath{ {./images/} } \textheight 9in \textwidth  6.5in
\topmargin -1cm \oddsidemargin -0.1in \evensidemargin -0.1in
\marginparwidth 17.57mm
%\renewcommand{\baselinestretch}{1.55}
\newcounter{tempeq}
\begin{document}
\title{\textsf{Enhancing the performance of an open quantum battery by adjusting its velocity}}
\author{B. Mojaveri\thanks{Email: bmojaveri@azaruniv.ac.ir; bmojaveri@gmail.com},
\hspace{2mm}R. Jafarzadeh Bahrbeig\thanks{Email:
r.jafarzadeh86@gmail.com},\hspace{2mm}M. A. Fasihi
\thanks{Email: ma-fasihi@azarunic.ac.ir}, and S. Babanzadeh\thanks{Email: s.babanzadeh@azaruniv.ac.ir}\\
{\small {Department of Physics, Azarbaijan Shahid Madani University,
PO Box 51745-406, Tabriz, Iran \,}}} \maketitle
\begin{abstract}
The performance of open quantum batteries (QBs) is severely limited
by decoherence due to the interaction with the surrounding
environment. So, protecting the charging processes against
decoherence is of great importance for realizing QBs. In this work
we address this issue by developing a charging process of a
qubit-based open QB composed of a qubit-battery and a qubit-charger,
where each qubit moves inside an independent cavity reservoir. Our
results show that, in both the Markovian and non-Markovian dynamics,
the charging characteristics, including the charging energy,
efficiency and ergotropy, regularly increase with increasing the
speed of charger and battery qubits. Interestingly, when the charger
and battery move with higher velocities, the initial energy of the
charger is completely transferred to the battery in the Markovian
dynamics. In this situation, it is possible to extract the total
stored energy as work for a long time. Our findings show that open
moving-qubit systems are robust and reliable QBs, thus making them a
promising candidate for experimental implementations.\\\\
{\bf Keywords:} Open quantum batter, Markovian and non-Markovian
charging process, Ergotropy, Atomic motion.
\end{abstract}
\section{introduction}
In recent years, with advancements in quantum thermodynamics, there
has been a radical change of perspective in the framework of energy
manipulation based on the electrochemical principles. The
possibility to create an alternative and efficient energy storage
device at small scale introduces the concept of the quantum battery
(QB), which was proposed by Alicki and Fennes in the 2013's
\cite{Alicki}, and  subsequently became into a significant field of
research. As their name indicates, QBs are finite dimensional
quantum systems that are able to temporarily store energy in their
quantum degrees of freedom for later use. The fundamental strategy
for developing the idea of QBs is based on their non-classical
features such as quantum coherence, entanglement and many-body
collective behaviors that can be cleverly exploited to achieve more
efficient and faster charging processes than the macroscopic
counterpart \cite{Strasberg, Vinjanampathy, Goold, Campisi,
Gelbwaser, Horodecki}. A QB is charged based on an interaction
protocol between QB itself with either an external field or a
quantum system which serves as a charger. It is then discharged into
a consumption hub based on the same protocol. When the battery
enters into an interaction with the charger, it transitions from a
lower energy level into the higher ones and will be charged. So far,
a variety of powerful charging protocols have been proposed in
different platforms, including two-level systems \cite{Farin, Zhang,
Fus}, harmonic oscillators \cite{Cata}, and hybrid light-matter
systems \cite{Maze, Manzo, Cond}. Some proposals have been also
devoted to implement QBs based on the two-level systems such as
trapped ions \cite{Forn, Lv}, cold atoms \cite{Bau} and
superconducting qubits \cite{Devoret}.

 Due to the fact that a real quantum system inevitably interacts with
its environment, studying QBs from the open quantum systems
perspective is attracting considerable interest. The interaction of
a QB with its surrounding environments causes the leakage of the
coherence of battery to the environment, leading to decoherence
effect in the battery. Such an adverse effect often plays a negative
role in the charging and discharging performance of QBs \cite{Camp,
Farin1, Carega}. Decoherence brought during the charging process
tends to lead QBs to a non-active (passive) equilibrium state in
which work extracting from the QBs is often impossible \cite{Barra}
in a cyclic unitary process. The environmental-induced noises also
affect QBs that are disconnected from both charger and consumption
hub and cause self-discharging of that QBs \cite{San0, Pedro,
Salimi}. Therefore, designing a more robust battery against the
environmental dissipations is valuable step for implementation of
QBs in the real-life. Recently, researchers have devoted efforts not
only to studying the effect of the environment on QBs, but also to
exploit non-classical effect as well as to developing open system
protocols to stabilize the charging cycle performance through
quantum control techniques. For example, Kamin et al \cite{Kamin1}
studied the charging performance of a qubit-based QB charged by the
mediation of a non-Markovian environment. They revealed the
non-Markovian property is beneficial for improving charging cycle
performance. In Ref. \cite{Squeezing}, the authors studied dynamics
of a continuous variable QB coupled weakly to the squeezed thermal
reservoir and managed to control the performance of the charging
process by boosting the quantum squeezing of reservoir. A feasible
route for harnessing loss-free dark states for stabilizing the
stored energy of a qubit-based open QB has been introduced in
\cite{Dark}. In addition to the above considerations, several other
protocols have been developed to protect the charging cycle of QBs
such as feedback control method \cite{Mitch, Shao, Ios}, convergent
iterative algorithm \cite{Borhan}, Bang-Bang modulation of the
intensity of an external Hamiltonian \cite{Franc}, inhiring an
auxiliary quantum system \cite{Behzadi}, modulating the detuning
between system and reservoir \cite{Yu0}, stimulated Raman adiabatic
passage technique \cite{Baris}, engineering quantum environments
\cite{Segal}, etc.

 On the other hand, according to the previous studies on
the Markovian and non-Markovian dynamics of open two-qubit systems,
translational motion of qubits provides novel insights for
stabilizing qubit-qubit entanglement against the environmental
induced dissipations by suitably adjusting the velocities of the
qubits \cite{Epjp0, morteza0, Chao0, sare0, Golkar1, Epjp1, MPLA,
Wang00}. We want here to use this safeguard capability of the
motional properties to improve the charging cycle performance of the
open qubit-based QBs. For this end, we consider a moving-biparticle
system composed of a qubit-battery and a qubit-charger that
independently interacts with their local environments. The battery
qubit here is charged with the help of the dipole-dipole interaction
with the charger qubit. We will investigate how the translational
motion of qubits affects the charging process of QB. Our results
show that translational motion of qubits always plays a constructive
role in protecting QB from decay induced by the environment. This
work is organized as follows: in Sec. 2, we introduce and describe
several figures of merit for characterizing the performance of QBs.
In Sec. 3, we illustrate our model and obtain explicit expressions
for the reduced density matrix of the QB and the charger. In Sec. 4
we present the results of our numerical simulations in the context
of their physical significance. Finally, Sec. 5 concludes this
paper.
\section{Figures of Merit}
Let us consider a QB modeled as a quantum system with d-dimensional
Hilbert space $\mathcal{H}$ and Hamiltonian $H_B$ such that
\renewcommand\theequation{\arabic{tempeq}\alph{equation}}
\setcounter{equation}{-1}
\addtocounter{tempeq}{1}\begin{eqnarray}\label{Bat}
H_B=\sum_{i=1}^{d} \varepsilon_i
|\varepsilon_i\rangle\langle\varepsilon_i|,
\end{eqnarray}
with non-degenerate energy levels $\varepsilon_i \leq
\varepsilon_{i+1}$. Internal energy of QB is given by $Tr(\rho_B
H_B)$, where $\rho_B$ is the state of the battery. Charging a QB
means brings the quantum system from a lower energy state $\rho_B$
to a higher energy state $\rho_B^\prime$, while discharging refers
to the inverse process, i.e., brings the quantum system from a
higher energy state $\rho_B^\prime$ to a lower one
$\rho_B^{\prime\prime}$:
\renewcommand\theequation{\arabic{tempeq}\alph{equation}}
\setcounter{equation}{-1}
\addtocounter{tempeq}{1}\begin{eqnarray}\label{den}
\texttt{Tr}\left\{\left(\rho_B^\prime-\rho_B\right) H_B\right\}\geq0,\qquad\qquad\qquad\qquad charging \nonumber \\
\texttt{Tr}\left\{\left(\rho_B^{\prime\prime}-\rho_B^\prime\right)
H_B\right\}\geq0.\qquad\qquad\qquad\quad \;\;discharging
\end{eqnarray}
Therefore, in a charging process, the actual stored energy of QB at
time $t$, regarding the initial energy, can be expressed as follows
\cite{Alicki}
\renewcommand\theequation{\arabic{tempeq}\alph{equation}}
\setcounter{equation}{-1} \addtocounter{tempeq}{1}\begin{equation}
\Delta E_B=\texttt{Tr}\{\rho_B(t) H_B\}-\texttt{Tr}\{\rho_B(0)
H_B\}.
\end{equation}
A complete converting the stored energy into valuable work is
impossible without dissipation of heat according to the second law
of thermodynamics. The maximum amount of energy extracted from a
given quantum state $\rho_B=\sum_{i} r_i |r_i\rangle\langle r_i|$,
($ r_i \geq r_{i+1}$) through a cyclic unitary operation is called
ergotropy \cite{Allahverdyan}. This quantity can be defined as
\cite{Allahverdyan, Franc0, Cakmak0}
\renewcommand\theequation{\arabic{tempeq}\alph{equation}}
\setcounter{equation}{-1}
\addtocounter{tempeq}{1}\begin{equation}\label{ergo}
\mathcal{W}=\texttt{Tr}\{\rho_B
H_B\}-\texttt{min}_U\,\texttt{Tr}\{U\rho_B U^{\dagger} H_B\},
\end{equation}
where the minimization is taken over all possible unitary
transformations acting locally on such system. It has been shown in
\cite{Allahverdyan} that no work can be extracted from the passive
counterpart of $\rho_B$ with the form $\sigma_{\rho_B}=\sum_{i} r_i
|\varepsilon_i\rangle\langle\varepsilon_i|$. The unique unitary
transformation $U=\sum_i |\varepsilon_i\rangle\langle r_i|$ on the
$\rho$ minimizes $\texttt{Tr}(U\rho_B U^{\dagger} H_B)$, and when
inserted in Eq. (\ref{ergo}) yields the following expression for the
ergotropy
\renewcommand\theequation{\arabic{tempeq}\alph{equation}}
\setcounter{equation}{-1} \addtocounter{tempeq}{1}\begin{equation}
\mathcal{W}=\sum_{i,j} r_j \varepsilon_i\left(|\langle
r_j|\varepsilon_i\rangle|^2-\delta_{ij}\right).
\end{equation}
In order to quantify the amount of extractable energy, the
efficiency $\eta$ is defined as the ratio between the ergotropy
$\mathcal{W}$ and the total charging energy $\Delta E_B$
\renewcommand\theequation{\arabic{tempeq}\alph{equation}}
\setcounter{equation}{-1} \addtocounter{tempeq}{1}\begin{equation}
\eta=\frac{\mathcal{W}}{\Delta E_B}.
\end{equation}% Figure environment removed
\section{Open Moving-Quantum Battery}
The open QB under consideration is composed of an atomic two-qubit
system, the qubit $A$ as a charger and the qubit $B$ as a quantum
battery, coupled to each other trough the dipole-dipole interaction.
The battery and charger qubits coupled locally to two independent
zero-temperature cavity reservoirs (see Fig. 1). We assume that each
qubit moves along the $z$-axis of its cavity at a constant
non-relativistic speed $v$. For simplicity we neglect here any
scattering \cite{Engl} or trapping \cite{Haro} effects and consider
the translational motion of the atom qubits being classically. Under
the dipole and rotating wave approximation, the entire system is
ruled by Hamiltonian (setting $\hbar=1$)
\renewcommand\theequation{\arabic{tempeq}\alph{equation}}
\setcounter{equation}{-1} \addtocounter{tempeq}{1}\begin{equation}
H=H_0+H_{int},
\end{equation}
with
\renewcommand\theequation{\arabic{tempeq}\alph{equation}}
\setcounter{equation}{-1}
\addtocounter{tempeq}{1}\begin{eqnarray}\label{Ham}
&&\hspace{-1.15cm}
H_0=H_A+H_B+H_{R_A}+H_{R_B}=\sum_{j=A,B}\left(\frac{\omega_0}{2}
\sigma_{z}^{j} + \sum_{k}\omega_{k}^j a_{k}^{j\dag} a_{k}^j\right),\nonumber\\
&&\hspace{-1.2cm}H_{int}=H_{A-B}+H_{A-R_A}+H_{B-R_B}=D\left(\sigma_{+}^{A}\sigma_{-}^{B}+\sigma_{-}^{A}
\sigma_{+}^{B}\right) +\sum_{j=A,B}\sum_{k} f_k^j(z)
\left(\mathfrak{g}_{k}^j \sigma_{+}^{j} a^j_k +H.c.\right).
\end{eqnarray}
Here, H.c. stands for Hermitian conjugate, $\sigma_z^j$,
$\sigma_+^j$, and $\sigma_-^j$ $(j=A,B)$ are, respectively, the
population inversion, raising and lowering operators of the $j$th
qubit with transition frequency $\omega_0$. $a_k^{j\dagger}$ and
$a^j_k$ are, respectively, the creation and annihilation operators
of the $k$th mode of the cavity reservoir $j$ with the frequency
$\omega_k^j$. Also, $D$ is coupling constant of the dipole-dipole
interaction between the battery and charger qubits, and
$\mathfrak{g}_{k}^j$ is the coupling constant between the $j$th
qubit and $k$th mode of in the cavity reservoir $j$. The effect of
translation motion of the battery and charger qubits has been
included in the model by introducing the $z$-dependent shape
function $f_k^j(z)$ in the Hamiltonian $H_{int}$. When the battery
and charger qubits are moving with same constant velocity $v$, the
shape function $f_k^j(z=vt)$ can be taken into account as
\renewcommand\theequation{\arabic{tempeq}\alph{equation}}
\setcounter{equation}{-1} \addtocounter{tempeq}{1}\begin{equation}
f_k^j(z)=\sin[\omega_k^j(\beta t-\Gamma)],\qquad\qquad j=A,B
\end{equation}
where, $\Gamma=L/c$ with $L$ being the size of the cavity. Also,
$\beta=v/c$ where $c$ refers to the speed of light in the vacuum
space. This particular form of the shape function can be obtained by
imposing an appropriate boundary condition on the cavity reservoirs
\cite{Lenard, morteza0}. Here we describe the translational motion
of both battery and charger qubits by classical mechanics ($z=vt$).
To this end, we will choose the values of the parameters in such a
way that the de Broglie wavelength of qubit $\lambda_B$ is
significantly smaller than the wavelength $\lambda_0$ associated
with the resonant transition $\omega_0=\omega_n$ ($\omega_n$ is the
central frequency of the cavity field mode) \cite{mortezapour,
Cook}. Furthermore, we consider a situation in which the photon
momentum is relatively small than the atomic momentum and thus we
neglect the atomic recoil caused by the interaction with the
electric field \cite{Wilkens}. In the optical regime, to ignore the
atomic recoil and consider the translational motion of atoms as
classical, the velocity of qubits should be $v\gg 10^{-3}$
\cite{morteza0}.

In the interaction picture (IP) generated by the unitary
transformation $U=e^{-iH_0t}$, the Hamiltonian (\ref{Ham}) can be
written as follows
\renewcommand\theequation{\arabic{tempeq}\alph{equation}}
\setcounter{equation}{-1}
\addtocounter{tempeq}{1}\begin{eqnarray}\label{HIP}
&&\hspace{-1.5cm}H_{IP}=D\left(\sigma_{+}^{A}
\sigma_{-}^{B}+\sigma_{-}^{A} \sigma_{+}^{B}\right)+
\sum_{j=A,B}\sum_{k} f_k^j(z)\left(\mathfrak{g}_{k}^j \sigma_{+}^{j}
a_k^{j} e^{i(\omega_0-\omega_k^j)t}+\mathfrak{g}_k^{j \ast}
\sigma_{-}^{j}a_{k}^{j\dag} e^{-i(\omega_0-\omega_k^j)t}\right).
\end{eqnarray}
It is straightforward to show that the total excitation operator
$\hat{\mathcal{N}}=\sum_{j=A,B}\left(\sum_k\hat{a_k}^{j\dagger}\hat{a_k}^j+
\frac{1}{2}\hat{\sigma}_{z}^{j}\right)+1$, commutes with the total
Hamiltonian, i.e. $[H,\hat{\mathcal{N}}]=0$ and therefor it is the
constant of the motion. This allows us to decompose Hilbert space of
the entire qubit-cavity system,
$\mathcal{H}=\mathcal{H}_q\otimes\mathcal{H}_R$ spanned by the basis
$\{\left|i_A,j_B\right\rangle\otimes\left|n_1,n_2, ...,n_k,
...\right\rangle_{R_A}|_{n_1,n_2,...=0}^{\infty}
\otimes\left|m_1,m_2, ...,m_k,
...\right\rangle_{R_B}|_{m_1,m_2,...=0}^{\infty}\}$
$\left(i,j=e,g\right)$ into the excitation subspaces, as follows
\renewcommand\theequation{\arabic{tempeq}\alph{equation}}
\setcounter{equation}{-1} \addtocounter{tempeq}{1}
\begin{eqnarray}
&&\hspace{-14mm} \mathcal{H}=\oplus_{n=0}^{\infty} \mathcal{H}_{n}.
\end{eqnarray}
As a result of this decomposition, the dynamics of the entire
qubit-reservoir system can be restricted to the excitation subspaces
labeled by the total excitation number $n$. Here we are interested
to explore dynamics of the entire system in the single-excitation
subspace $\mathcal{H}_1$ spanned by vectors
$\{\left|g_A,g_B\right\rangle\otimes\left|1_k\right\rangle_{R_A}\left|0_k\right\rangle_{R_B}|_{k=0}^\infty,
\left|g_A,g_B\right\rangle\otimes\left|0_k\right\rangle_{R_A}\left|1_k\right\rangle_{R_B}|_{k=0}^\infty,
\left|e_A,g_B\right\rangle\otimes\left|0_k\right\rangle_{R_A}\left|0_k\right\rangle_{R_B},
\left|g_A,e_B\right\rangle\otimes\left|0_k\right\rangle_{R_A}\left|0_k\right\rangle_{R_B}\}$
in which the single excitation is either in one of the qubits or in
the k-th mode of one of cavity reservoirs. We consider a normalized
initial state of entire qubit-reservoir as a superposition of
$\left|e_A,g_B\right\rangle\left|0_k\right\rangle_{R_A}\left|0_k\right\rangle_{R_B}$
and
$\left|g_A,e_B\right\rangle\left|0_k\right\rangle_{R_A}\left|0_k\right\rangle_{R_B}$
states with the following form
\renewcommand\theequation{\arabic{tempeq}\alph{equation}}
\setcounter{equation}{-1}
\addtocounter{tempeq}{1}\begin{eqnarray}\label{sai0}
|\Psi(0)\rangle=\big[c_1(0) |e_{A},g_{B}\rangle +c_2(0)
|g_{A},e_{B}\rangle\big]\otimes |0\rangle_{R_A}|0\rangle_{R_B}.
\end{eqnarray}
For times $t>0$, we expand the state vector $|\Psi(t)\rangle$ in
terms of the vector basis of the single-excitation subspace
$\mathcal{H}_1$ as
\renewcommand\theequation{\arabic{tempeq}\alph{equation}}
\setcounter{equation}{-1}
\addtocounter{tempeq}{1}{\footnotesize\begin{eqnarray}\label{sai}
&&\hspace{-3.5cm}\left|\Psi(t)\right\rangle=\big[c_1(t)\left |e_{A},
g_B\right\rangle +c_2(t) \left|g_A, e_B\right\rangle\big] \otimes
\left|0_k\right\rangle_{R_A}\left|0_k\right\rangle_{R_B}
\nonumber\\
&&\hspace{-2.35cm}+\left|g_A, g_B\right\rangle\otimes\sum_{k}
\big[d_{k}(t)\left|1_k\right\rangle_{R_A}\left|0_k\right\rangle_{R_B}+d_{k}^{\prime}(t)
\left|0_k\right\rangle_{R_A}\left|1_k\right\rangle_{R_B}\big],
\end{eqnarray}}
where the time-dependent amplitudes satisfy the normalization
requirement
\renewcommand\theequation{\arabic{tempeq}\alph{equation}}
\setcounter{equation}{-1} \addtocounter{tempeq}{1}\begin{eqnarray}
\sum_{i=1}^2|c_i(t)|^2+\sum_k(|d_{k}(t)|^2+|d_{k}^{\prime}(t)|^2)=1.
\end{eqnarray}
By taking the partial traces over the field modes and subsystem A
(B), the reduced time-dependent density operator for the battery
(charger) in the $\{\left|e\right\rangle, \left|g\right\rangle\}$
basis is obtained as
\renewcommand\theequation{\arabic{tempeq}\alph{equation}}
\setcounter{equation}{0} \addtocounter{tempeq}{1}\begin{eqnarray}
&&\hspace{-2cm}\rho_A(t)=|c_1(t)|^2\left|e_A\right\rangle\left\langle
e_A\right|-\left(1-|c_1(t)|^2\right)\left|g_A\right\rangle\left\langle
g_A\right|\label{rob2},\\
&&\hspace{-2cm}\rho_B(t)=|c_2(t)|^2\left|e_B\right\rangle\left\langle
e_B\right|-\left(1-|c_2(t)|^2\right)\left|g_B\right\rangle\left\langle
g_B\right|\label{rob1}.
\end{eqnarray}

 Inserting Eq. (\ref{sai}) into the time dependent Schr\"{o}dinger
equation $H_{IP}|\Psi(t)\rangle=i\frac{d}{d t}|\Psi(t)\rangle$, with
$H_{IP}$ given in (\ref{HIP}), leads to the following set of
differential equations for time-dependent amplitudes
\renewcommand\theequation{\arabic{tempeq}\alph{equation}}
\setcounter{equation}{0} \addtocounter{tempeq}{1}\begin{eqnarray}
&&\hspace{-4cm}i\dot{c_1}(t)=D c_2(t)+\sum_{k} \mathfrak{g}_{k}^A
f_k^A(z)d_{k}(t)e^{i(\omega_0-\omega_{k}^A)}\label{c1t},\\
&&\hspace{-4cm}i\dot{c_2}(t)= D c_1(t)+ \sum_{k} \mathfrak{g}_{k}^B
f_k^B(z)d_{k}^{\prime}(t)e^{i(\omega_0-\omega_{k}^B)}\label{c2t},\\
&&\hspace{-4cm}i\dot{d}_{k}(t)=\mathfrak{g}_k^{A\ast}f_k^A(z)
c_1(t)e^{-i(\omega_0-\omega_{k}^A)t},\label{d1t}\\
&&\hspace{-4cm}i\dot{d}_{k}^{\prime}(t)=
\mathfrak{g}_k^{B\ast}f_k^B(z)
c_2(t)e^{-i(\omega_0-\omega_{k}^B)t}\label{d2t}.
\end{eqnarray}
By integrating Eqs. (\ref{d1t}) and (\ref{d2t}) with the initial
condition $d_{k}(0)=0$ and $d_{k}^{\prime}(0)=0$ and putting their
solutions, respectively, in Eqs. (\ref{c1t}) and (\ref{c2t}), we get
the following integro-differential equations for the amplitudes
$c_1(t)$ and $c_2(t)$
\renewcommand\theequation{\arabic{tempeq}\alph{equation}}
\setcounter{equation}{0} \addtocounter{tempeq}{1}\begin{eqnarray}
&&\hspace{-2cm}\dot{c_1}(t)=-iDc_2(t)+\int_{0}^{t}F_A(t-t^\prime)c_1(t^\prime)dt^\prime,\label{mt}\\
&&\hspace{-2cm}\dot{c_2}(t)=-iDc_1(t)+\int_{0}^{t}F_B(t-t^\prime)c_2(t^\prime)dt^\prime,\label{nt}
\end{eqnarray}
where
\renewcommand\theequation{\arabic{tempeq}\alph{equation}}
\setcounter{equation}{0} \addtocounter{tempeq}{1}\begin{eqnarray}
&&\hspace{-2cm}F_{A}(t-t^\prime)=\sum_{k} |\mathfrak{g}_{k}^A|^2
e^{i(\omega_0-\omega_{k}^A)(t-t^\prime)}\sin[\omega_k^A(\beta^A
t-\Gamma)]\sin[\omega_k^A(\beta^A t^\prime-\Gamma)],\\
&&\hspace{-2cm}F_{B}(t-t^\prime)=\sum_{k} |\mathfrak{g}_{k}^B| ^2
e^{i(\omega_0-\omega_{k}^B)(t-t^\prime)}\sin[\omega_k^B(\beta^B
t-\Gamma)]\sin[\omega_k^B(\beta^B t^\prime-\Gamma)],
\end{eqnarray}
are the memory correlation function of the reservoirs $A$ and $B$,
respectively. For simplicity, we suppose
$F_{A}(t-t^\prime)=F_{B}(t-t^\prime)=F(t-t^\prime)$. In the limit of
a large number of modes ( in the continuum limit ), the correlation
function $F(t-t^\prime)$ takes the following form
\renewcommand\theequation{\arabic{tempeq}\alph{equation}}
\setcounter{equation}{-1}
\addtocounter{tempeq}{1}\begin{equation}\label{kernel}
F(t-t^\prime)=\int d\omega J(\omega)
e^{i(\omega_0-\omega)(t-t^\prime)}\sin[\omega(\beta
t-\Gamma)]\sin[\omega(\beta t^\prime-\Gamma)],
\end{equation}
in which $J(\omega)$ is the spectral density of the cavity
reservoirs and has the Lorentzian form \cite{Lenard, Breuer0}
\renewcommand\theequation{\arabic{tempeq}\alph{equation}}
\setcounter{equation}{-1}
\addtocounter{tempeq}{1}\begin{equation}\label{lorentz}
J(\omega)=\frac{1}{2\pi}\frac{\gamma\lambda^2}{(\omega_0-\omega-\Delta)^2+\lambda^2},
\end{equation}
where $\lambda$ defines the spectral width of the coupling which is
connected to the memory time $\tau_E$ by the relation
$\tau_E=\lambda^{-1}$ and $\gamma$ refers to the qubit-environment
coupling strength which is related to the relaxation time scale
$\tau_R$ by $\tau_R \approx \gamma^{-1}$. Also $\Delta$ is the
detuning of $\omega_0$ and the central frequency of the cavity. The
weak and strong coupling regimes can be distinguished by comparing
$\tau_E$ and  $\tau_R$, in other words with an increasing
$\frac{\tau_E}{\tau_R}=\frac{\gamma}{\lambda}$ ratio, the
interaction will transition into a strong coupling or a non-Markovian regime \cite{Breuer0}.\\
By inserting the Eq. (\ref{lorentz}) into the Eq. (\ref{kernel}) and
after some calculations, in the continuum limit ($\Gamma \rightarrow
\infty$), the correlation function is simplified as
\renewcommand\theequation{\arabic{tempeq}\alph{equation}}
\setcounter{equation}{-1}
\addtocounter{tempeq}{1}\begin{equation}\label{ft}
F(t-t^\prime)=\frac{\gamma \lambda}{4} \cosh[\beta
\overline{\lambda}(t-t^\prime)] e^{-(\lambda-i\Delta) |t-t^\prime|}
\end{equation}
with $\overline{\lambda}=\lambda+i(\omega_0-\Delta)$.\\
In view of (\ref{ft}), taking the Laplace transformations of both
sides of the differential Eqs. (\ref{mt}) and (\ref{nt}) and using
the convolution property
$\mathcal{L}[\int_{0}^{t}\mathbf{A}(t-t^\prime) \mathbf{B}(t^\prime)
dt^\prime]=\mathbf{A}(s)\mathbf{B}(s)$ yields
\renewcommand\theequation{\arabic{tempeq}\alph{equation}}
\setcounter{equation}{0} \addtocounter{tempeq}{1}\begin{eqnarray}
&&\hspace{-2cm}sc_1(s)-c_1(0)=-iDc_2(s)-F(s)c_1(s),\label{ms}\\
&&\hspace{-2cm}sc_2(s)-c_2(0)=-iDc_1(s)-F(s)c_2(s),\label{ns}
\end{eqnarray}
where the functions $c_1(s)$ and $c_2(s)$ are the Laplace
transformations of the $c_1(t)$ and $c_2(t)$, respectively, and
$F(s)$ is the Laplace transforms of $F(t-t^\prime)$ which has the
following explicit form
\renewcommand\theequation{\arabic{tempeq}\alph{equation}}
\setcounter{equation}{-1} \addtocounter{tempeq}{1}\begin{eqnarray}
F(s)=\frac{\gamma\lambda}{4}\frac{s+\overline{\lambda}}{(s+\overline{\lambda})^2-\beta^2\overline{\lambda}\,^2}.
\end{eqnarray}
By reformulating the Eqs. (\ref{ms}) and (\ref{ns}), we get a
general solution for $c_1(s)$ and $c_2(s)$ as follows
\renewcommand\theequation{\arabic{tempeq}\alph{equation}}
\setcounter{equation}{0} \addtocounter{tempeq}{1}\begin{eqnarray}
&&\hspace{-2cm}c_1(s)=\frac{s+F(s)}{\big(s+F(s)\big)^2+D^2}c_1(0)-i\frac{D}{(s+F(s))^2+D^2}c_2(0),\\
&&\hspace{-2cm}c_2(s)=\frac{s+F(s)}{\big(s+F(s)\big)^2+D^2}c_2(0)-i\frac{D}{(s+F(s))^2+D^2}c_1(0).
\end{eqnarray}
In continuation, by applying the inverse Laplace transformation on
the both side of the above equations, we obtain finally $c_1(t)$ and
$c_2(t)$, as
\renewcommand\theequation{\arabic{tempeq}\alph{equation}}
\setcounter{equation}{0} \addtocounter{tempeq}{1}\begin{eqnarray}
&&\hspace{-2cm}c_1(t)=\frac{1}{2}\bigg(c_1(0)\Re(\mathcal{M}(t))-ic_2(0)\Im(\mathcal{M}(t))\bigg)\label{ct12},\\
&&\hspace{-2cm}c_2(t)=\frac{1}{2}\bigg(c_2(0)\Re(\mathcal{M}(t))-ic_1(0)\Im(\mathcal{M}(t))\bigg)\label{ct122},
\end{eqnarray}
where, $\Re(x)$ ($\Im(x)$) is real (imaginary) part of $x$, and
\renewcommand\theequation{\arabic{tempeq}\alph{equation}}
\setcounter{equation}{-1} \addtocounter{tempeq}{1}\begin{equation}
\mathcal{M}(t)=\sum_{i,j,k=1}^3\varepsilon_{ijk}\frac{ e^{q_it}
(q_j-q_k)\bigg((q_i+\overline{\lambda})^2-\beta
^2\overline{\lambda}^2\bigg)}{\prod_{i=1}^{3}\prod_{j=i+1}^{3}(q_i-q_j)},
\end{equation}
with $\varepsilon_{ijk}$ is the Levi-Civita symbol and $q_i (i=  1,
2, 3)$ are the roots of
\renewcommand\theequation{\arabic{tempeq}\alph{equation}}
\setcounter{equation}{-1} \addtocounter{tempeq}{1}\begin{equation}
q^3+q^2(2 \overline{\lambda}-i \text{D} )+q \left(\frac{\gamma
\lambda }{4}+\overline{\lambda} (\overline{\lambda}-2 i
\text{D})-\beta ^2\overline{\lambda}^2\right)+\frac{\gamma  \lambda
\overline{\lambda}}{4}+i \text{D} \overline{\lambda}^2\left(\beta
^2-1\right)=0.
\end{equation}

 With substitution (\ref{ct12}) and (\ref{ct122}), respectively, into the reduced density matrices
(\ref{rob1}) and (\ref{rob2}), and then using the $\Delta
E_{A(B)}=\texttt{Tr}\{\rho_{A(B)}(t)
H_{A(B)}\}-\texttt{Tr}\{\rho_{A(B)}(0) H_{A(B)}\}$ , the internal
energy of the charger and battery are deduced as
\renewcommand\theequation{\arabic{tempeq}\alph{equation}}
\setcounter{equation}{-1} \addtocounter{tempeq}{1}\begin{equation}
\Delta
E_A=\omega_0\left(|c_1(t)|^2-|c_1(0)|^2\right),\quad\quad\Delta
E_B=\omega_0\left(|c_2(t)|^2-|c_2(0)|^2\right).
\end{equation}
On the other hand, one can obtain ergotropy of the battery by
substitution Eq. (\ref{rob1}) with Eq. (\ref{ergo}). So, we have
\renewcommand\theequation{\arabic{tempeq}\alph{equation}}
\setcounter{equation}{-1} \addtocounter{tempeq}{1}\begin{equation}
 W_B=\omega_0\left(2|c_2(t)|^2-1\right)\Theta
\left(|c_2(t)|^2-\frac{1}{2}\right),
\end{equation}
where $\Theta(x-x_0)$ is the Heaviside function, which satisfies
$\Theta(x-x_0)=0$ for $x<x_0$, $\Theta(x-x_0)=\frac{1}{2}$ for
$x=x_0$ and $\Theta(x-x_0)=1$ for $x>x_0$.
% Figure environment removed
% Figure environment removed
\section{Numerical Results and Discussion}
In this section, we will analyze the charging dynamics of the
introduced open moving-battery in the weak and strong coupling
regimes. In particular, we explore the role of the movement of QB on
the dynamical behavior of performance indicators including stored
energy, ergotropy and efficiency. In our following analysis, we
choose the optical regime parameters \cite{Hood, Pinkse} and
consider that qubit transition frequency as
$\omega_0=1.5\times10^{9}\lambda$. In what follows, we consider an
initial condition in which the battery is initially empty and the
charger has the maximum energy, i.e. $c_1(0)=0$, $c_2(0)=1$.
% Figure environment removed

 In Fig. 2, we plot the Markovian and non-Markovian dynamics of the stored energy $\Delta E_B$
for the initial state
$\left|\Psi(0)\right\rangle=\left|g\right\rangle_{A}\left|e\right\rangle_{B}\otimes
\left|0\right\rangle_{R_B}\left|0\right\rangle_{R_B}$, by
considering different values of the QB speed $\beta$. In panel (a),
the battery is charged in the Markovian dynamics with
$(\gamma=0.1\lambda)$, while in panel (b), it is charged in a
non-Markovian dynamics with $(\gamma=20\lambda)$. Here we consider a
situation at which the charger and battery's qubits are both in
resonance with the reservoir modes by setting $\Delta=0$. According
to this figure, the positive impact of the translational motion of
the charger and batter's qubits in controlling the stored energy of
battery is clearly visible in both Markovian and non-Markovian
charging processes. As can be seen in both Figs. 2(a) and (b), when
the charger and battery's qubits are at rest inside their cavity
reservoirs, the stored energy in the battery $\Delta E_B$ decays
into zero at sufficiently long times. However the rate of these
decays decreases regularly by gradual growth of the qubit velocity,
and therefore the energy stored in the battery and consequently the
charging process is strongly protected from the environmental
noises. Comparing Fig. 2(a) with Fig. 2(b) clearly reveals a
fundamental difference between Markovian and non-Markovian charging
processes. The maximal amount of stored energy in the Markovian
charging process is more than those of the non-Markovian charging
process. The reason stems from the nature of the qubit-cavity
coupling. In the non-Markovian charging process, the coupling
strength of charger's qubit to the cavity modes is greater than its
coupling to the battery's qubit, therefore, the initial internal
energy of charger has more tendency to evolve toward the reservoir
than to the battery. Moreover, since the motional effect of QB has
been included in battery-cavity and charger-cavity coupling
strength, it seems that increasing speed of QB decreases the
charger-cavity coupling strength in favor of to charger-battery
coupling strength, which increases the energy stored in the battery.

In order to get more insight to this area and a deeper understanding
of the relationship between the charger and battery energy, in Fig.
2 we have illustrated the energy stored in the battery at the end of
charging process as well as the energy that the charger loses at the
same time. Here $\Delta E_B$ and $|\Delta E_A|$ have been plotted as
a function of the dimensionless time $\lambda t$ for the qubit
velocities $\beta=0$ and $\beta=0.7\times 10^{-9}$ in the Markovian
and non-Markovian regimes. In the non-Markovian charging process,
$|\Delta E_A|$ is much more than $\Delta E_B$ for a given $\beta$ as
shown in Fig. 3(b). This implies that the internal energy of the
charger is not completely transferred to the battery. Fig. 3(b) also
shows that, when the charger and battery's qubits are at rest inside
their cavity reservoirs, the charger's qubit immediately loses a
large amount of its initial energy without being transferred to the
battery. However, increasing the qubit velocity (decreasing the
ratio of charger-cavity coupling strength to charger-battery
coupling strength) during the non-Markovian process, decreases the
initial loss-rate of the charger, and therefore improves the energy
transfer in the charging processes.

The relationship between the charger and battery energy in the
Markovian charging process is drastically different from that in the
non-Markovian charging process. One can infer from Fig. 3(a) that,
for the static battery-charger system ($\beta=0$), the total energy
of the charger can be transferred to the battery in the Markovian
short-charging process, where we have $|\Delta E_A|=\Delta E_B$.
Interestingly, when the qubits move with the velocity
$\beta=0.7\times10^{-9}$, $|\Delta E_A|=\Delta E_B$ holds at any
charging time. So, we conclude again that a robust Markovian
charging against the arisen dissipation can be achieved, when the
qubits move with higher velocities.
% Figure environment removed

 In the following, we examine the influence of translational motion
of the battery-charger system on the dynamics of ergotropy. In Fig.
4, we plot $W/W_{max}$ as a function of $\lambda t$ for the
different values of $\beta$ in the Markovian (Fig. 4(a)) and
non-Markovian (Fig. 4(b)) regimes. Our numerical results in Fig.
4(a) and (b) illustrate that, the effect of translational motion of
QB on the ergotropy is also constructive in both Markovian and
non-Markovian regimes. Fig. 4(b) shows that, in the non-Markovian
regime, in the cases of stationary ($\beta=0$) and slowly moving
($\beta=3\times10^{-9}$) qubits, we are not able to extract useful
work from the QB, but in this regime a considerable work can be
extracted, as the qubits move with a higher velocity
($\beta=0.8\times10^{-9}$). Our numerical results in Fig. 4(a)
illustrate that, the effect of translational motion of QB on the
ergotropy is more considerable in the Markovian case. We observe
that, in the Markovian regime, increasing the speed of QB $\beta$
(decreasing the qubit-reservoir coupling) not only boosts the
ergotropy, but also increases the number of time zones in which work
can be extracted. Accordingly, a strong robust charging process can
be established in the higher speed limit, in which the extractable
work approaches to its maximum value.

 Finally, we examine the effect of translational motion
of QB on the Markovian and non-Markovian charging efficiency. The
results for Markovian and non-Markovian charging processes are
presented in Fig. 5(a) and 5(b), respectively. Here we consider the
same parameter values as Fig. 4. Comparing Figs. 4 and 3 reveals
that both ergotropy and efficiency are positively affected by the
translational motion of QB. However the efficiency is influenced
more than the ergotropy; the amount of increment in efficiency is
more than the ergotropy in both Markovian and non-Markovian charging
processes.
\section{Outlook and summary}
To summarize, we proposed a mechanism for robust charging process of
an open qubit-based quantum battery (QB) whose robustness can be
well controlled by the translational motion of the charger and
battery in both Markovian and non-Markovian dynamical regimes. Both
the battery and charger's qubits move with a same speed inside two
separated identical environments, and are directly coupled by the
dipole-dipole interaction. We showed that the stored energy,
ergotropy and efficiency of the moving QB regularly increased with
the gradual growth of the charger and battery speed, thereby
improving its charging performance. The constructive role of the
translational movement of QB in controlling the charging process
arises from the attachment of qubits velocities to the
qubit-reservoir coupling strength (see Eq. (\ref{Ham})). According
to the adopted charging protocol, a weak qubit-reservoir coupling is
required for a strongly robust charging process which can be
fulfilled by adjusting $\beta$ to the higher velocities.

 Our results represent a further control strategy to have a robust QB with
a natural implementation in cavity-QED context. The strategy can be
easily implemented also in the circuit-QED setups where the qubit
position slowly varies linearly with time and also the qubit-cavity
interaction is tuned through a sinusoidal position-dependent
coupling \cite{Jones}.

  In perspective, we believe that this strategy can be used
to control the performance of the discharging of a qubit-based QB to
an available consumption hub. Further efforts in this field can be
devoted to use the proposed strategy for improving the performance
of the two-photon based charging process where the moving-QB is
coupled with a cavity reservoir by means of a two-photon
relaxation.\\\\
\textbf{\large{Data availability}}\\ The datasets used and analysed
during the current study available from the corresponding author on
reasonable request.
\begin{thebibliography}{99}
\bibitem{Alicki} R. Alicki and M. Fannes, Entanglement boost for extractable work from ensembles of quantum batteries, Phys. Rev. E 87, 042123 (2013).
\bibitem{Strasberg} P. Strasberg, G. Schaller, T. Brandes, and M. Esposito, Quantum and information thermodynamics: A unifying framework based on repeated interactions, Phys. Rev. X 7, 021003 (2016).
\bibitem{Vinjanampathy} S. Vinjanampathy and J. Anders, Quantum thermodynamics, Cont. Phy. 57, 545 (2016).
\bibitem{Goold} J. Goold, M. Huber, A. Riera, L. del Rio, and P. Skrzypczyk, The role of quantum information in thermodynamics: a topical review, J. Phys. A: Math. Theor. 49, 143001 (2016).
\bibitem{Campisi} M. Campisi, P. H\"{a}nggi, and P. Talkner, Colloquium: Quantum fluctuation relations: Foundations and applications, Rev. Mod. Phys. 83, 1653 (2011).
\bibitem{Gelbwaser} D. Gelbwaser-Klimovsky, W. Niedenzu and G. Kurizki, Thermodynamics of quantum systems under dynamical control, Adv. At. Mol. Opt. Phys., 64, 329 (2015).
\bibitem{Horodecki} M. Horodecki and J. Oppenheim,Fundamental limitations for quantum and nanoscale thermodynamics, Nature Comm. 4, 2059 (2013).
\bibitem{Farin} D. Farina, G. M. Andolina, A. Mari, M. Polini and V. Giovannetti, powerful charging of quantum batteries, Phys. Rev. B 99, 035421 (2019).
\bibitem{Zhang} Y-Y. Zhang, T-R. Yang, L. Fu and X. Wang, Powerful harmonic charging in a quantum battery, Phys. Rev. E 99, 052106 (2019).
\bibitem{Fus} L. Fusco, M. Paternostro, and G. D. Chiara, Work extraction and energy storage in the Dicke model, Phys. Rev. E 94, 052122 (2016).
\bibitem{Cata} R. R. Rodriguez, B. Ahmadi, P. Mazurek, S. Barzanjeh, R. Alicki and P. Horodecki, catalysis in charging quantum batteries, Phys. Rev. A 107, 042419 (2023).
\bibitem{Maze} J. Carrasco, J. R. Maze, C. Hermann-Avigliano and F. Barra, collective enhancement in dissipative quantum batteries, Phys. Rev. E. 105, 064119 (2022).
\bibitem{Manzo} M. Gumberidze, M. Kol\'{a}r and R. filip, Measurement induced Synthesis of coherent Quantum Batteries, Sci. Rep 9, 19628 (2019).
\bibitem{Cond} D. Ferraro, M. Campisi, G. M. Andolina, V. Pellegrini and M. Polini, High-power collective charging of a solid-state quantum battery, Phys. Rev. Lett. 120, 117702 (2018).
\bibitem{Forn} P. Forn-D\'{\i}laz, J. J. Garc\'{\i}la-Ripoll, B. Peropadre, J.-L. Orgiazzi, M. A. Yurtalan, R. Belyansky, C. M. Wilson, and A. Lupascu, Ultrastrong coupling of a single artificial atom to an electromagnetic continuum in the nonperturbative regime, Nat. Phys. 13, 39 (2016).
\bibitem{Lv} Bruzewicz, C.D.; Chiaverini, J.; McConnell, R.; Sage, J.M. Trapped-Ion Quantum Computing: Progress and Challenges. Appl. Phys. Rev. 2019, 6, 021314..
\bibitem{Bau} K. Baumann, C. Guerlin, F. Brennecke, and T. Esslinger, The dicke quantum phase transition with a superfluid gas in an optical cavity, Nature (London) 464, 1301 (2010)
\bibitem{Devoret} Devoret, M.H.; Schoelkopf, R. J. Superconducting Circuits for Quantum Information: An Outlook. Science 2013, 339, 1169
\bibitem{Farin1} D. Farina, G. M. Andolina, A. Mari, M. Polini, and V. Giovannetti, Charger-mediated energy transfer for quantum batteries: Anopen-system approach. Phys. Rev. B 99, 035421 (2019).
\bibitem{Camp} C. Ou, R. V. Chamberlin and S. Abe, Lindbladian operators, von Neumann entropy and energy conservation in time-dependent quantum open systems, Physica A 466, 450 (2017).
\bibitem{Carega} M. Carrega, A. Crescente, D. Ferraro, and M. Sassetti, Dissipative dynamics of an open quantum battery. New J. Phys. 22, 083085 (2020).
\bibitem{Barra} F. Barra, Dissipative charging of a quantum battery, Phys. Rev. Lett. 122, 210601 (2019).
\bibitem{San0} A. C. Santos, Quantum advantage of two-level batteries in
self-discharging process, Phys. Rev. E 103, 042118 (2021).
\bibitem{Pedro} L. P. Garcia-Pintos, A. Hamma, A. del Campo, Fluctuations in extractable work bound the charging power of quantum batteries. Phys. Rev. Lett. 125, 040601 (2020).
\bibitem{Salimi} F. H. Kamian, F. T. Tabesh, S. Salimi, F. Kheirandish, and A. C. Santos, Non-markovian effects on charging and selfdischarging processes of quantum batteries, New J. Phys. 22, 083007 (2020).
\bibitem{Kamin1} F. T. Tabesh, F. H. Kamin, and S. Salimi, Environmentmediated charging process of quantum batteries, Phys. Rev. A 102, 052223 (2020).
\bibitem{Squeezing} F. Centrone, L. Mancino, M. Paternostro, Charging batteries with quantum squeezing, https://doi.org/10.48550/arXiv.2106.07899.
\bibitem{Dark} J. Q. Quach and W. J. Munro, Using dark states to charge and stabilise open quantum batteries, Phys. Rev. Applied 14, 024092 (2020).
\bibitem{Mitch} M. T. Mitchison, J. Goold and J. Prior, Charging a quantum battery with linear feedback control, Quantum 5, 500 (2021).
\bibitem{Shao} Y. Yao and X. Q. Shao, Phys. Rev. E Optimal charging of open spin-chain quantum batteries via homodyne-based feedback control, 106, 014138 (2022).
\bibitem{Ios} S. Borisenok, Ergotropy of quantum battery controlled via target attractor feedback, J. Appl. Phys. 12, 43 (2020).
\bibitem {Borhan} R. R. Rodriguez, B. Ahmadi, G. Suarez, P. Mazurek, S. Barzanjeh, P. Horodecki, Optimal quantum control of charging quantum batteries, arXiv:2207.00094 [quant-ph].
\bibitem{Franc} F. Mazzoncini, V. Cavina, G. M. Andolina, P. A. Erdman and V. Giovannetti, Optimal control methods for quantum batteries, Phys. Rev. A 107 (2023) 032218.
\bibitem{Behzadi} N. Behzadi and H. Kassani, Mechanism of controlling robust and stable charging of open quantum batteries, J. Phys. A: Math. Theor. 55, 425303 (2022).
\bibitem{Yu0} J. L. Li, H. Z. Shen and X. X. Yi, Quantum batteries in non-Markovian reservoirs, Opt. Lett 21, 5614 (2022).
\bibitem{Baris} A. C. Santos, B. \c{C}akmak, S. Campbell and N.T. Zinner, Stable adiabatic quantum batteries, Phys. Rev. E 100, 032107 (2019).
\bibitem{Segal} J. Liu, D. Segal, Boosting quantum battery performance by structure engineering, arXiv:2104.06522 [quant-ph].
\bibitem{Epjp0} J. Taghipour, B. Mojaveri and A. Dehghani, Witnessing entanglement between two two-level atoms coupled to a leaky cavity via two-photon relaxation, Eur. Phys. J. Plus 137, 772 (2022).
\bibitem{morteza0} A. Mortezapour, M. A. Borji, and R. L. Franco, Protecting entanglement by adjusting the velocities of moving qubits inside non-Markovian environments, Laser Phys. Lett 14, 055201 (2017).
\bibitem{Chao0} W. Chao and F. Mao-Fa, The entanglement of two moving atoms interacting with a single-mode field via a three-photon process, Chin. Phys. B 19, 020309 (2010).
\bibitem{sare0} S. Golkar and M. K. Tavassoly and A. Nourmandipour, Entanglement dynamics of moving qubits in a common environment, J. Opt. Soc. Am. B 37, 400 (2020).
\bibitem{Golkar1} S. Golkar and M. K. Tavassoly And A. Nourmandipour, Qubit movement-assisted entanglement swapping, Chin. Phys. B. 29, 050304 (2020).
\bibitem{Epjp1} B. Mojaveri, A. Dehghani and J. Taghipour, Control of entanglement, single excited-state population and memory-assisted entropic uncertainty of two qubits moving in a cavity by using a classical driving field, Eur. Phys. J. Plus 137, 1065 (2022).
\bibitem{MPLA} J. Taghipour, B. Mojaveri and A. Dehghani, Witnessing entanglement between two two-level atoms moving inside a leaky cavity under classical control, Mod. Phys. Lett. A 37, 2250141 (2022).
\bibitem{Wang00} Q. Wang, R. Liu, H. M. Zou, D. Long and J. Wang, Entanglement dynamics of an open moving-biparticle system driven by classical-field, Phys. Scr. 97, 055101, (2022).
\bibitem{Allahverdyan} A. E. Allahverdyan, R. Balian and T. M. Nieuwenhuizen, Maximal work extraction from finite quantum systems. Eur. phys. Lett 67, 565 (2004).
\bibitem{Franc0} G. Francica, J. Goold, F. Plastina, and M. Paternostro, Daemonic ergotropy: enhanced work extraction from quantum correlations, npj Quantum Inf. 3, 12 (2017).
\bibitem{Cakmak0} B. \c{C}akmak, Ergotropy from coherences in an open quantum system, Phys. Rev. E 102, 042111 (2020).
\bibitem{Engl} B.G. Englert, J. Schwinger, A.O. Barut and M.O. Scully, Reflecting slow atoms from a micromaser field, Eur. Phys. Lett 14, 25 (1991).
\bibitem{Haro} S. Haroche, M. Brune and J.M. Raimond, Trapping atoms by the vacuum field in a cavity, Eur. Phys. Lett 14, 19 (1991).
\bibitem{Lenard} C. Leonardi and A. Vagliea, Non-markovian dynamics and spectrum of a moving atom strongly coupled to the field in a damped cavity, Opt. Commun 97, 130 (1993).
\bibitem{mortezapour} F. Nosrati, A. Mortezapour and R. Lo Franco, Validating and controlling quantum enhancement against noise by the motion of a qubit, Phys. Rev. A. 101, 012331 (2020).
\bibitem{Cook} R. J. Cook, Atomic motion in resonant radiation: An application of Ehrenfest's theorem, Phys. Rev. A. 20, 224 (1979).
\bibitem{Wilkens} M. Wilkens, Z. Bialynicka-Birula and P. Meystre, Spontaneous emission in a Fabry-P\'{e}rot cavity: The effects of atomic motion, Phys. Rev. A. 45, 477 (1992).
\bibitem{Breuer0} H. P. Breuer and F. Petruccione, \textit{The Theory of Open Quantum Systems} (Oxford University Press, Oxford, New York, 2002).
\bibitem{Hood} C. J. Hood et al., The Atom-Cavity Microscope: Single Atoms Bound in Orbit by Single Photons, Science 287, 1447 (2000).
\bibitem{Pinkse} P. W. H. Pinkse et al., Trapping an atom with single photons, Nature 404, 365 (2000).
\bibitem{Jones} P. J. Jones, J. A. M. Huhtam\"{a}ki, K. Y. Tan and M. M\"{o}tt\"{o}nen, Tunable electromagnetic environment for superconducting quantum bits, Sci. Rep. 3, 1987 (2013).
\end{thebibliography}
\end{document}

% ######################################################
% Pairwise Comparisons
% ######################################################
Studying such perceptions has traditionally been carried out using direct rating methods (users assign a score to each event or situation). This procedure requires a well-defined scale and user training and is particularly difficult to conduct when events or conditions substantially differ from one another \cite{perez2017practical}, which is the case when analyzing real-world environments. In contrast, using pairwise comparisons (users compare two situations and choose one of the two) is often simpler and faster to set up, well-suited for non-expert participants \cite{perez2017practical}, and presents lower measurement error compared to direct ratings \cite{shah2015estimation}. With this in mind, we employ pairwise comparisons to analyze cycling safety perceptions. Moreover, we draw current practice and knowledge from other research areas (e.g., sports outcome prediction and preference learning) about pairwise comparisons and how algorithms can be used to study cycling safety perceptions, something unexplored in cycling safety research. This paves the way to scale safety perception studies and ubiquitously understand how individuals perceive cycling risk.


% ######################################################
% Gap, Objectives & Contributions
% ######################################################
The main contributions of this paper are as follows. First, we draw knowledge from other research areas about pairwise comparisons and apply them to studying cycling safety perceptions. This novel approach
uses a survey showcasing images of two road environments and asking users which one they find safer, if any. % We use respondents' answers to compare different methodologies previously applied to sports prediction and preference learning, showcasing how these can be directly applied to our main goal: understanding cycling perception of safety.
With the respondents' answers, we compare different methodologies, previously applied to sports prediction and preference learning, and show how these can be directly applied to our main goal: understanding cycling perception of safety. Lastly, we draw from these results to objectively classify cycling environments based on urban characteristics and cycling environments. 


% ######################################################
% Outline of article
% ######################################################
We divide the article as follows. In the next section, we explore the current literature on pairwise comparisons and how traditional rating methods unravel such data. In Section \ref{sec:survey}, we detail our pairwise comparison survey and present different algorithms to rate cycling environments. Next, in Section \ref{sec:ranking}, we present the methodology, overviewing all pairwise ranking algorithms and environment classification. Section \ref{sec:results} presents the results and highlights what environments are perceived as safer or riskier. Finally, Section \ref{sec:conclusions} concludes the paper and draws possible paths forward.


\newcommand\citewithauthor[1] 
 {\citeauthor{#1}~(\citeyear{#1})~\citep{#1}}




\section{Related Works}

In recent years, there has been increasing interest in utilising AI for tackling difficult problems in traditional domains like 
adopting AI in the construction industry~\citep{regona2022opportunities},
localisation in robotic applications~\citep{lai2022slamreview},
assistance systems in the service sector~\citep{link2020use},
financial forecast~\citep{forexNonStationaryTimeSeries},
improving workflow in the oil and gas industry~\citep{koroteev2021artificial},
planning and scheduling~\citep{lai2022MEP},
monitoring ocean contamination~\citep{xu2022waterSedimentML},
remote sensing for search and rescue~\citep{lai2023UAV},
and even used in the life cycle of material discovery~\citep{li2020ai}.
Health care industry has been adopting AI-based machine-learning techniques for classifying medical images~\citep{castiglioni2021ai},
guiding cancer diagnosis~\citep{chugh2021survey},
as screening tools for diabetes~\citep{sensorsMLforDiabetes},
and ultimately improve the clinical workflow in the practice of medicine~\citep{brattain2018machine}.

One area of research focuses on using conversational agents, also known as chatbots, for mental health support. Chatbots have the potential to provide accessible and cost-effective assistance to individuals in need. 
For example, \citewithauthor{martinengo2022evaluation} qualitatively analysed user-conversational agent and found that these type of chapbots can offer anonymous, empathetic, and non-judgemental interactions that align face-to-face psychotherapy. chatbot can utilise NLP techniques to engage users in therapeutic conversations and provide personalised support. Results showed promising outcomes, indicating the potential effectiveness of chatbots in delivering mental health interventions~\citep{denecke2021artificial}.
Pre-trained language models have also gained attention in the field of mental health counselling. These models, such as GPT-3~\citep{brown2020language}, provide a foundation for generating human-like responses to user queries. \citewithauthor{wang2023prompt} explored the application of LLMs in providing mental health counselling. They found that LLMs demonstrated a certain level of understanding and empathy, providing responses that were perceived as helpful by users. However, limitations in controlling the model's output and ensuring ethical guidelines were highlighted.

Furthermore, there is a growing body of research on using NLP techniques to analyse mental health-related text data~\citep{gonzalez2017capturing}. Researchers have applied machine learning algorithms to detect mental health conditions~\citep{abd2020application}, predict suicidal ideation~\citep{ji2020suicidal}, and identify linguistic markers associated with psychological well-being~\citep{akstinaite2022identifying}. For instance, \citewithauthor{de2013predicting} analysed social media data to predict depression among individuals. By extracting linguistic features and using machine learning classifiers, they achieved promising results in identifying individuals at risk of depression.
Additionally, several studies have investigated the integration of modern technologies into existing mental health interventions. For instance, \citewithauthor{lui2017evidence} investigates the use of mobile applications to support the delivery of psychotherapy.

\citewithauthor{shaikh2022autonomous} developed a friendly AI-based chatbot using deep learning and artificial intelligence techniques. The chatbot aimed to help individuals with insomnia by addressing harmful feelings and increasing interactions with users as they experienced sadness and anxiety.
In another line of research, chatbots have been extensively studied in the domain of customer service. Many companies have adopted chatbots to assist customers in making purchases and understanding products. These chatbots provide prompt replies, enhancing customer satisfaction\citep{9885724}.
Furthermore, advancements in language models such as BERT and GPT have been influential in the development of conversational chatbots. Researchers have leveraged BERT-based question-answering models to improve the accuracy and efficiency of chatbot responses\citep{9652153}. The GPT models, including GPT-2 and GPT-3, have introduced innovations such as zero-shot and few-shot learning, significantly expanding their capabilities in generating human-like text\citep{Brown2020LanguageMA}. However, limitations in generating coherent and contextual responses and the interpretability of the models have been identified. The model incorporated a 48-layer Transformer stack and achieved a parameter count of 1.5 billion, resulting in enhanced generalisation abilities\citep{Brown2020LanguageMA}.

In summary, previous work in the field of AI and NLP for mental health support has demonstrated the potential of chatbots, pre-trained language models, and data analysis techniques. These approaches offer new avenues for delivering accessible and personalised mental health interventions. Nonetheless, further research is needed to address ethical, privacy, and reliability issues and to optimise the integration of AI technologies into existing counselling practices.






\section{Mental Health and Social Well-being in Overly Populated Cities}

The availability of mental health professionals has always been a major problem in overpopulated cities such as China.
The World Health Organisation has reported that the prevalence of depression in China exceeds 54 million people even before the onset of the COVID-19 pandemic~\citep{world2017depression}.
The situation has been exacerbated by the implementation of quarantine measures and social distancing, leading to a worsening condition~\citep{gou2022province}.
Unfortunately, only a small fraction of the affected population receives adequate medical treatment, as there are only 2 psychiatrists per 100,000 people in China~\citep{xiang2018rethinking}.
Consequently, there is a pressing need for a dynamic system that can assist patients effectively. Contemporary conversational chatbots have demonstrated their ability to emulate human-like conversations.

Hence, it is imperative to develop a user-friendly AI-based chatbot specifically designed to address anxiety and depression, with the aim of improving the user's emotional well-being by providing relevant and helpful responses.
This project aims to construct a Chinese psychological dialogue model capable of comprehending the semantic meaning of a consultant's request and offering appropriate advice, particularly to address the shortage of mental health workers during demanding periods. 
The trained model will be integrated into a website, featuring a user interface (UI) that ensures ease of operation, thereby enhancing the efficiency of psychological counselling.





\subsection{Research Questions}

Psychology is an intricate and advanced discipline gaining increasing significance as society progresses. However, due to its high barriers to entry, resources for psychological counselling have long been scarce. In numerous cases, individuals face challenges accessing adequate mental health support~\citep{19004}. Furthermore, the high cost of psychological counselling often prevents many individuals from prioritising their mental well-being. This issue is particularly prominent in China, a country with a large population where psychological problems have been historically overlooked. China needs a robust foundation for psychological counselling, including a deficient knowledge base and limited data. Consequently, intelligent assistance in psychology must be improved in the Chinese context.

Traditional psychological counselling primarily focuses on privacy and employs a one-on-one question-and-answer approach, inherently leading to inefficiencies. However, in today's high-pressure society, where mental health issues are pervasive, relying solely on scarce psychologists is arduous. Additionally, influenced by traditional culture, individuals often hesitate to acknowledge and address their psychological problems due to feelings of shame and perceiving such discussions as signs of weakness~\citep{sandhu}. This reluctance is especially prominent when conversing with real humans, let alone seeking assistance from unfamiliar psychologists.
Furthermore, with the advancement of modern Natural Language Processing (NLP) artificial intelligence models, there is a possibility to optimise the conventional and widely adopted question-and-answer model specifically for the field of psychology. When interacting with AI, people are more inclined to express their true thoughts and emotions without fear of prejudice and discrimination than real humans.
However, the lack of verbal cues and continuous monitoring of patients' emotional progress is also a cause of concern in an online psychological consultation context, even when performed by human counsellors~\citep {novella2022comparison}.


\subsection{Research Scope}
Through our project, we aim to make a meaningful contribution to the field of mental well-being. By leveraging our AI model and proposing a framework for an internet-accessible consultation, we intend to enhance the accessibility of mental health support, making it more affordable and providing an avenue for psychological question-and-answer interactions, particularly to address the shortage of human counsellors for mental health support.
To achieve this goal, we gather professional counselling question-and-answer data and psychologically relevant knowledge data to construct a robust question-and-answer model specific to this domain. The success of our project relies on the utilisation of high-quality question-and-answer models. We plan to employ established Chinese pre-trained models with exceptional human-computer interaction and communication skills, characterised by fluent language, logical reasoning, and semantic understanding. However, these existing pre-trained models need more specialised psychological expertise and emotional understanding for counselling purposes. To address this limitation, we intend to integrate two models, namely the \emph{WenZhong} model and the \emph{PanGu} model, and evaluate their performance to determine the more suitable choice as our final model.

Subsequently, it is imperative to make our model accessible to a broader audience. Leveraging the internet provides the most effective means to accomplish this objective. Together with the guidance of professional mental health experts, our model can provide an additional venue for the general public to access mental health support by easing the stress and demand on mental health staff through an open and online platform.


\section{Psy-LLM Framework}

Our project involves leveraging two large-scale pre-training models, namely \emph{WenZhong} and \emph{PanGu}, to develop the question-answering language model. The utilisation of pre-training models offers several advantages, including:
(1) Enhanced Language Representations: Pre-training on extensive unlabeled data enables the model to acquire more comprehensive language representations, which in turn can positively impact downstream tasks.
(2) Improved Initialisation Parameters: Pre-training provides a superior initialisation point for the model, facilitating better generalisation performance on the target task and expediting convergence during training.
(3) Effective Regularisation: Pre-training acts as an effective regularisation technique, mitigating the risk of overfitting when working with limited or small datasets. This is especially valuable as a randomly initialised deep model is susceptible to overfitting on such datasets.
By harnessing the advantages of pre-training models, we aim to enhance the performance and robustness of our question-answering language model for psychological counselling.


\subsection{PanGu Model}
\emph{PanGu} model is the first Chinese large-scale pre-training autoregressive language model with up to 200 billion~\citep{Zeng2021PanGuLA}. In an autoregressive model, the process of generating sentences can be likened to a Markov chain, where the prediction of a token is dependent on the preceding tokens. The \emph{PanGu} model, developed within the MindSpore framework, was trained using 2048 Ascend AI processors provided by Huawei and trained on a high-quality corpus of 1.1TB. It was officially released in April 2021 and has achieved the top rank in the Chinese Language Comprehension Benchmark (CLUE), a widely recognised benchmark for Chinese language comprehension~\citep{xu2020clue}.
% Figure environment removed

The architecture of the \emph{PanGu} model follows a similar structure to that of GPT-3, employing standard transformer layers~\cref{fig:PanGu}. Each transformer layer consists of two sub-layers: multi-head attention (MHA) and fully connected feed-forward network (FFN). The MHA involves three primary steps: calculating the similarity between the Query and Key, applying a softmax function to obtain attention scores, and multiplying the attention scores with the Value to obtain the attention output. The attention output then passes through a linear layer and undergoes softmax to generate the output embedding. The output embedding is further combined with the input of the FFN through a residual module. The FFN consists of two linear layers with a GeLU activation function in between. Both the MHA and FFN utilise the pre-layer normalisation scheme, facilitating faster and easier training of the Transformer model.

However, the last layer of the \emph{PanGu} model deviates from the standard transformer layer structure. Instead, it incorporates a query layer designed to predict the next token, thereby enhancing the model's positional awareness and improving generation effectiveness. The query layer serves as a narrow yet powerful decoder that solely relies on position information. The structure of the query layer is illustrated in~\cref{fig:QueryArc}. The primary distinction between the query layer and the transformer layer lies in the query input of self-attention. While the inputs of Query, Key, and Value in other self-attention layers of the transformer remain standard, the query layer introduces a query embedding, which functions similarly to position embedding, as the query input for self-attention in the last layer.


% Figure environment removed

The \emph{PanGu} model is available in four distinct variations, each characterised by different parameter sizes (\cref{table:pangu-setting}). These variations include \emph{PanGu} 350M, \emph{PanGu} 2.6B, \emph{PanGu} 13B, and \emph{PanGu} 200B (which is not open source). The parameter sizes differ across these models, reflecting their varying levels of complexity and capacity for language understanding and generation.


\begin{table}[b]
    \centering
    \caption{The parametric size of the various settings in the \emph{PanGu} model. \label{table:pangu-setting}}
    \begin{tabular}{@{}cccccc@{}}
    \toprule
    Model & Parameters & Layers & Hidden Size & Head & Seq Length \\ \midrule
    \emph{PanGu} 350M & 350M  & 24 & 1024 & 16 & 1024\\
    \emph{PanGu} 2.6B  & 2.6B & 32 & 2560 & 40 & 1024   \\
    \emph{PanGu} 13B & 13.1B & 40 & 5120 & 40 & 1024 \\
    \emph{PanGu} 13B & 207.0B & 64 & 16384 & 128 & 1024 \\
    \bottomrule
    \end{tabular}
\end{table}

\subsection{WenZhong Model}
In addition to the \emph{PanGu} model, we have also incorporated the \emph{WenZhong} model as one of the models used. The \emph{WenZhong} model is a pre-trained model based on the GPT-2 architecture and trained on a large-scale Chinese corpus. Over the past few years, pre-trained models have become the foundation of cognitive intelligence, enabling advancements in natural language processing and computer vision algorithms.

The scale of pre-trained models has been rapidly increasing, growing by a factor of 10 each year, starting from the initial BERT model with 100 million parameters to the more recent GPT models with over 100 billion parameters. Given the nature of our task, which requires a generation model with expertise in different professional domains, we have opted for the \emph{WenZhong} model.

For models like GPT, limited computing resources pose a challenge that hinders further progress in the field. Universities and research institutions often lack the necessary computing power to train and utilise large-scale pre-trained models. This limitation impedes the broader implementation of AI technologies. Hence, we have adopted the \emph{WenZhong} model, which is built upon a large pre-trained model trained on a Chinese corpus.

The \emph{WenZhong} model series consists of one-way language models dominated by a Decoder structure and a series of powerful generation models. The \emph{WenZhong}-3.5B model, with 3.5 billion parameters, employs 100G data and 256 A100 GPUs for 28 hours of training, exhibiting strong generation capabilities. Thus, the \emph{WenZhong} model is highly powerful, featuring 30 decoder layers and billions of parameters. Additionally, we have utilised the \emph{WenZhong}-GPT2-110M version in this project, which comprises 110 million parameters and 12 layers. It is important to note that the \emph{WenZhong} model has been pre-trained on the Wudao Corpus (300G version).




\subsection{Collecting Large Scale Dataset}
Two types of data sources were obtained for this project. The first dataset, called PsyQA~\citep{psyqa}, consisting of question and answer pairs, focuses on Chinese psychological health support. The authors granted us authorisation to use this dataset, which contains 22,000 questions and 56,000 well-structured, lengthy answers. The PsyQA dataset includes numerous high-quality questions and answers related to psychological support, and it had already undergone basic cleaning before we received it. For our experiments, we selected a test set of 5,000 samples from this PsyQA dataset.

\subsubsection{Data Crawling}
The second dataset was obtained by crawling various Chinese social media platforms, such as Tianya, Zhihu, and Yixinli. These platforms allow users to post topics or questions related to mental and emotional issues, and other users can provide responses. The Yixinli website specifically focuses on professional mental health support, but it only provided approximately 10,000 samples. Other types of datasets collected from these platforms included articles and conversations, which we converted into a question-and-answer format. However, we excluded the articles from our fine-tuning training due to the model's input limitations and the fact that our predictions focused on mental health support answers. The articles were often lengthy, and many of them were in PDF format, requiring additional time for conversion into a usable text format. Consequently, we only obtained around 5,000 article samples. In order to address the lack of emotional expression in the text of these articles, we incorporated text data from oral expressions. We crawled audio and video data from platforms like Qingting FM and Ximalaya, popular audio and video sharing forums in China. However, converting audio and video data into text format was a time-consuming task, resulting in a limited amount of this type of data in our dataset. We utilised the dataset obtained from websites for fine-tuning training. In the end, our entire dataset consisted of 400,000 samples, with each sample separated by a blank line, i.e., ``\textbackslash n\textbackslash n''.

\Cref{table:crawled-dataset} shows the time spent on data crawling from different websites. It is evident that a majority of the samples in this dataset were obtained from Tianya, resulting in a data size of approximately 2GB. The datasets from Zhihu and Yixinli were 500MB and 200MB, respectively. Overall, we spent approximately 70 hours on data collection. Although the data collected from the Internet was abundant and authentic, the cleaning process was challenging due to inconsistencies in the online data.

\begin{table}[tb]
    \centering
    \caption{Dataset crawled from different platforms. \label{table:crawled-dataset}}
    \begin{tabular}{@{}ccc@{}}
    \toprule
    Platform & Data Size & Crawling Time  \\ \midrule
    Tianya & 2GB & 40h+  \\
    Zhihu & 500Mb & 20h+  \\
    Yixinli & 200Mb & 8h+ \\
    \bottomrule
    \end{tabular}
\end{table}

To address the time-consuming nature of web crawling, we implemented a distributed crawl technology that utilised idle computers connected to the Internet or other networks, effectively harnessing additional processing power~\citep{Thelwall2001AWC}. Our approach involved obtaining sub-websites from the main website and saving them using custom crawling code. This code primarily relied on Python libraries such as ``requests'', ``BeautifulSoup'', and ``webdriver''. In addition, we employed dynamic web crawlers capable of collecting clickable elements, simulating user actions, comparing web page states, manipulating the DOM tree, and handling various user-invoked events~\citep{Li2018AutomaticallyCD}. Unlike static page structures that cannot handle dynamic local refresh and asynchronous loading~\citep{Yao2012AnAF}, dynamic crawlers were able to extract data from behind search interfaces.

The process of the dynamic crawler involved leveraging web developer tools within the browser to obtain XHR (XMLHttpRequest) information, which included requests containing headers, previews, and responses. By systematically searching through these layers of data and capturing network packets, we were able to acquire relevant files. After obtaining the sub-websites using both static and dynamic crawling methods, we distributed them across multiple idle computers. Each computer was assigned specific sub-websites, and we collected project-related data using a combination of static and dynamic crawling techniques. Ultimately, we utilised eight computers for the crawling process, which took approximately 70 hours.


\subsubsection{Data Cleaning}
In line with the \emph{PanGu} paper~\citep{Zeng2021PanGuLA}, we adopted the original data cleaning method utilised in the \emph{PanGu} model. Additionally, we incorporated some additional cleaning steps. The following are the cleaning steps we employed:
\begin{enumerate}
    \item \emph{Removal of duplicate samples}: We eliminated any duplicate samples present in the dataset to ensure data uniqueness.
    \item \emph{Removal of samples containing advertised keywords}: We excluded samples that contained specific keywords associated with advertisements or promotional content.
    \item \emph{Deletion of data with less than 150 characters}: Samples with less than 150 characters were removed from the dataset, as they were deemed insufficient for effective model training.
    \item \emph{Removal of URLs}: Any URLs present in the samples were eliminated to maintain the focus on text content.
    \item \emph{Removal of user names and post time}: User names, such as ``\texttt{@jack}'', and post timestamps were removed from the samples, as they were considered irrelevant to the text content.
    \item \emph{Removal of repeated punctuation}: Instances of repeated punctuation marks, such as ``!!!'' or ``.....'', were removed from the samples to ensure cleaner and more concise text.
    \item \emph{Conversion of traditional Chinese to simplified Chinese}: All traditional Chinese characters were converted to simplified Chinese characters to standardise the text.
\end{enumerate}

Following the data cleaning process, the dataset could be directly inputted into the \emph{PanGu} model. However, for training with the \emph{WenZhong} model, the samples needed further processing. Specifically, all punctuations were removed, and the samples were tokenized to ensure a consistent length of 1000 tokens for compatibility with the \emph{WenZhong} model.


\subsubsection{Data Analysis}
Data analysis plays a crucial role in understanding the fundamental characteristics of textual data. In the context of Chinese language, the exploratory data analysis (EDA) methods may not be as diverse compared to those used for English. In this study, we primarily employed two common methods: word frequency analysis and sentence length analysis, to gain insights into the dataset.

To analyse the distribution of characters in each sample, we referred to the character number data presented in~\cref{table:dataset-distribution}. By visualizing this information using a box chart, we examined the range of character counts across the samples.
There are some samples that are empty after the data cleaning, of which we then prune from our dataset.
\Cref{fig:data-distribution} displays the distribution of sample lengths, indicating that the majority of samples fall within the range of 10,000 characters.

Overall, these preliminary analyses allowed us to gain initial insights into the dataset and provided a foundation for further exploration and understanding of the textual data.

\begin{table}[tb]
    \centering
    \caption{Data distribution of the length of each sample. \label{table:dataset-distribution}}
    \begin{tabular}{@{}cccccccc@{}}
    \toprule
    Count & Mean & Std & Min & 25\% & 50\% & 70\% & Max \cr \midrule
    371,434 & 5,343 & 11,335 & 0 & 653 & 1,835 & 6,039 & 454,611 \cr
    \bottomrule
    \end{tabular}
\end{table}


% Figure environment removed

To examine the word frequency in our dataset, we conducted an analysis after removing the stop words. The word cloud visualisation in~\cref{fig:frequent-word} illustrates the most frequent words in the dataset. Notably, the prominent words observed include ``Anxiety'', ``Appearance'', ``Marriage'', ``Relationship'', ``Family'', and ``Stressful''. These words are highly relevant to the topic of mental health, indicating that our dataset is robust and aligns well with the focus of our task.

The presence of these mental health-related terms further underscores the suitability of our data for addressing the objectives of our study. It suggests that our dataset encompasses significant content related to psychological aspects, allowing us to effectively explore and address relevant topics in the context of our research.

% Figure environment removed




\subsection{Model Training}
\textbf{Model Size} We plan to use \emph{PanGu} 350M to generate language considering the computational power, which contains 350 million parameters, 24 layers, 1024 hidden size and 16 attention heads. Besides, we also want to train the \emph{WenZhong}-110M model that contain 12 layers and has 110M parameter.

\textbf{Training Data} We employ the 2.85GB psychology corpus data which is crawled from psychology platforms like Yixinli and Tianya, to train the original \emph{PanGu} 350M model. After that,  we use 5,6000 question-answer pairs from PsyQA dataset to fine-tune the model.

\textbf{Training Platform} We train the \emph{PanGu} model on OpenI platform with a free V100 graphics card GPU, because OpenI is the open source platform of \emph{PanGu} model, convenient for us to deploy the required files, image and GPU. The batch size is set to 8 and the training iteration is set to 100,000, because we found that 50,000 iteration is not enough  for the model’s loss to converge.
 Besides, we train the \emph{WenZhong} model in jupyter notebook .To fine-tune this model, we need to tokenize the data, which can transform words into the token. Besides, we also need to isolate the max length of each sentence as 500.





\subsection{Dataset Evaluation}

Determining the cleaning rules and data filtering thresholds are important aspects of the data cleaning process. To evaluate the quality of the dataset obtained from website crawling, we employed a data quality evaluation method that combined both manual assessment and model-based evaluation.

For the model-based evaluation, we utilised the \emph{PanGu} 350M model and calculated the perplexity metric after each stage of data cleaning. A lower perplexity value indicates a more effective cleaning process and higher dataset quality.
In addition to the model-based evaluation, we sought input from experts in the field of psychology. We invited two members from our University's School of Psychology, Faculty of Science to perform a random sample check on the dataset after it had undergone the cleaning process. While this method does not cover the entire corpus comprehensively, it does provide valuable insights and plays a role in both data cleaning and data quality evaluation.

The evaluation process involved the following steps: First, we provided the experts with a sample of the cleaned dataset and asked them to assess its quality based on their expertise and domain knowledge. They evaluated the dataset for accuracy, relevance, and coherence, providing feedback and suggestions for further improvements.

Next, we conducted a comparative analysis between the model-based evaluation and the expert evaluation. We examined the perplexity scores obtained from the \emph{PanGu} 350M model and compared them with the feedback provided by the experts. This allowed us to identify any discrepancies or areas of improvement in the dataset.

Overall, the combination of model-based evaluation and expert assessment provided a comprehensive evaluation of the dataset quality. It allowed us to identify and address any issues or shortcomings in the data cleaning process, ensuring that the final dataset used for training and evaluation was of high quality and suitable for our research purposes.

\subsection{Model Training Setting}

\emph{Models:} For our training, we utilise the \emph{PanGu} 350M model, considering the available computational resources. This model consists of 350 million parameters, 24 layers, a hidden size of 1024, and 16 attention heads. Additionally, we target the \emph{WenZhong}-110M model, which contains 12 layers and 110 million parameters.

\emph{Training Data:} We collected a psychology corpus dataset, totaling 2.85GB, which was crawled from psychology platforms such as Yixinli and Tianya. This dataset was used for training the original \emph{PanGu} 350M model. Subsequently, we fine-tuned the model using 56,000 question-answer pairs from the PsyQA dataset.

\emph{Training Platform:} The \emph{PanGu} model was trained on the OpenI platform, utilising a free 1 V100 graphics card GPU. OpenI is an open-source platform specifically designed for the \emph{PanGu} model, allowing us to easily deploy the necessary files, images, and GPU resources.
For training with the V100 graphics card (32 GB memory), the minimum recommended configuration is one card, while the recommended configuration is two cards. The graphics card requirements can be adjusted based on the memory size (for example, a 16GB memory card would require twice as many cards as the V100). If the dataset is large, increasing the number of graphics cards can help improve the training speed
We set the batch size to 8 and performed training for 100,000 iterations, as we observed that 50,000 iterations were insufficient for the model's loss to converge.
For the \emph{WenZhong} model, we used Jupyter Notebook to run the pre-trained model and fine-tuned it on a system with 64GB memory and an RTX3060 graphics card. The version details of the hardware and software components are listed in~\cref{table:hardware-software-version}.



\begin{table}[tb]
    \centering
    \caption{Hardware and Software Versions}
    \label{table:hardware-software-version}
    \begin{tabular}{@{}cc@{}}
    \toprule
    Hardware and Software & Version \\
    \midrule
    Operating System & Windows 10 \\
    numpy & 1.18.5 \\
    pandas & 1.3.4 \\
    torch & 1.11.0 \\
    tokenizers & 0.13.1 \\
    tqdm & 4.64.1 \\
    jupyter & 1.0.0 \\
    transformers & 4.23.1 \\
    \bottomrule
    \end{tabular}
\end{table}









\subsubsection{Training Process}
Training Process. According to the guide of training \emph{PanGu} model with GPU, the first step is environment configuration. We prepared Pytorch, \emph{PanGu} image, one V100 graphics card and some \emph{PanGu} model files like vocabulary. The second step is data preprocessing. We put the training corpus into txt file, and each sample is a paragraph separated by two newlines (Figure 6). Then converted it into bin file, because it is the required input format of training \emph{PanGu} model. The third step is model training. We uploaded the \emph{PanGu} model and the bin file to the OpenI platform and set some parameters like iteration to train it (Figure 7). The training of \emph{PanGu} model contains two steps, which is also one of the highlights of the model. Firstly, we train the original \emph{PanGu} 350M model with all the crawling data for 100,000 iterations, because we have tried to train 50,000 iterations but the loss of the model did not converge. We can see from Figure 8, when iterations are about 60,000, the loss converges, so we downloaded the model trained 60,000 iterations. This model have learned psychology domain knowledge. Secondly, we fine-tune it with the PsyQA dataset. Figure 9 reveals that the best iteration is about 9,000, so that is the final best \emph{PanGu} model after training and fine-tuning. The fourth is to download the best trained \emph{PanGu} model based on the loss, perplexity from the output log (Figure 10) and manual testing. Finally, we used the GPU environment to inference.

We used the early stop method to choose appropriate iterations. Stopping the training of the network before the validation loss started to increase effectively prevented the model from overfitting. For example, in Figure 9, when the model has more than 9,000 training iterations, the validation loss of the model starts to rise, which means the phenomenon of overfitting occurs.

In \emph{WenZhong} model,before we train the model, we need to import \emph{WenZhong} model and tokenizer. However, when we run the code in colab, there is an error called ‘ CUDA out of memory ‘. Therefore, we have to use a new way to calculate the gradient. First of all, it can sum up the gradient in every step. Second, this way can get the average loss by dividing the gradient by the steps, which means less accumulation. After solving this problem, we can free to set the batch size, the numbers of epochs and so on.

% Figure environment removed







\subsection{Model Evaluation}

In this section, we assess the performance and effectiveness of our proposed language model for online psychological consultation. We employ a combination of intrinsic evaluation metrics and human evaluation to comprehensively evaluate the model's capabilities. We begin by utilising perplexity, ROUGE-L, and Distinct-n metrics to measure the model's language generation quality, similarity to the reference text, and diversity. Additionally, we recognise the limitations of these metrics and emphasise the importance of human evaluation in providing subjective assessments of the model's outputs, considering factors such as coherence, relevance, and overall quality. Through this comprehensive evaluation approach, we aim to gain a comprehensive understanding of our model's strengths, weaknesses, and suitability for its intended purpose in the context of online psychological consultation.

\subsubsection{Metric-based Evaluation}

Perplexity is a widely used intrinsic evaluation metric that provides a measure of how well a language model predicts a given sample. Mathematically, perplexity is defined as the reciprocal of the average probability assigned to each token in the dataset by the language model~\citep{Chen1998EvaluationMF}. In simpler terms, a lower perplexity value indicates better performance of the language model. Since perplexity is based on the average log-likelihood of the dataset, it can be computed quickly and is statistically robust, as it is not easily affected by outliers.

The formula for calculating perplexity is given by
 \begin{align}
    PP(W)
    &= \mathbb{P}\left( w_1 w_2 \ldots w_N \right)^{{-1}/{N}} \\
    &=\sqrt[N]{ \prod_{i=1}^N \frac{1}{\mathbb{P}(w_i)} }
 \end{align}
where $PP$ is the perplexity, $\mathbb{P}$ is the probability of the $i^{th}$ word, and $N$ is the length of a sentence.
It is important to note that as the dataset size increases, perplexity tends to decrease, indicating better performance.

However, it is crucial to understand that low perplexity does not necessarily equate to high accuracy. Perplexity is primarily used as a preliminary measure and should not be solely relied upon for evaluating model accuracy. Additionally, comparing the performance of models on different datasets with varying word distributions can be challenging~\citep{Chen1998EvaluationMF}. Therefore, while perplexity provides valuable insights into model performance, it should be complemented with other evaluation metrics and considerations when assessing model accuracy.



ROUGE-L (Longest Common Subsequence) is an evaluation metric that measures the number of overlapping units between the predicted text generated by a language model and the actual reference text~\citep{Lin2004ROUGEAP}. By quantifying the similarity between the predicted and reference texts, ROUGE-L provides a measure of how closely the generated text matches the desired output.

Distinct-1 and Distinct-2 are evaluation metrics that assess the diversity of the generated text. Distinct-1 calculates the number of distinct unigrams (individual words) divided by the total number of generated words, while Distinct-2 calculates the number of distinct bigrams (pairs of adjacent words) divided by the total number of generated bigrams~\citep{Li2016ADO}. These metrics reflect the degree of diversity in the generated text by quantifying the presence of unique unigrams and bigrams.

The formulas for calculating Distinct-n are as follows:

\begin{equation}
    \text{Distinct-n} := Distinct(n)=\frac{Count(\texttt{unique}, \texttt{n-gram})}{Count(\texttt{word})}
\end{equation}

Here, $Count(\texttt{unique}, \texttt{n-gram})$ represents the number of $n$-grams that are not repeated in a reply, and $Count(\texttt{word})$ indicates the total number of $n$-gram words in the reply. A higher value of $Distinct(n)$ indicates a greater diversity in the distinct generations.

Together, these evaluation metrics, including perplexity, ROUGE-L, Distinct-1, and Distinct-2, provide insights into the quality, similarity, and diversity of the generated text by the language model. They serve as valuable tools for assessing the performance and effectiveness of the model in generating accurate and diverse outputs.

While metrics like perplexity and Distinct-n provide insights into the language model's performance in terms of language generation, they do not necessarily indicate high accuracy. Therefore, in order to evaluate models more convincingly, human evaluation is still necessary. Human evaluators can provide subjective assessments of the generated text, taking into account factors such as coherence, relevance, and overall quality, which are important aspects that cannot be fully captured by automated evaluation metrics alone.


\subsubsection{Human evaluation}


For the purpose of human evaluation, we have developed an online marking system to assess the performance of our language model in the context of online psychological consultation. This evaluation system aims to streamline the process and ensure effective assessment by focusing on four key metrics: Helpfulness, Fluency, Relevance, and Logic. Each metric is scored on a scale of 1 to 5, allowing evaluators to provide a quantitative assessment of each aspect.
The four metrics are defined as follows:

\begin{enumerate}
    \item \textbf{Helpfulness:} This metric evaluates whether the generated response is helpful for patients seeking psychological support.
    \item \textbf{Fluency:} Fluency refers to the degree of coherence and naturalness exhibited in the generated response.
    \item \textbf{Relevance:} Relevance assesses the extent to which the content of the response is directly related to the posed question.
    \item \textbf{Logic:} Logic examines the logical consistency and coherence of the meaning conveyed in the generated response.
\end{enumerate}


To conduct the human evaluation, we invited six students from the psychological faculty to assess a set of 200 question-answer pairs generated by our model. We employed two evaluation methods to obtain a comprehensive understanding of the model's performance.

In the first method, evaluators compared responses generated by both the \emph{PanGu} model and the \emph{WenZhong} model in response to the same question. They assigned scores to these answers based on the predetermined metrics, allowing for a direct comparison between the two models.
The second method involved incorporating the actual answers alongside the predicted responses as a whole, allowing evaluators to assess the differences and similarities between the generated responses and the actual ones.

By employing these human evaluation methods, we aim to gain valuable insights into the performance of our language model, particularly in terms of the disparities between predicted and actual responses. This comprehensive evaluation approach will provide a deeper understanding of the model's capabilities and guide further improvements in its performance for online psychological consultation.

\section{Experimental Results}



In this section, we present the findings and outcomes of the evaluation and experimentation conducted to assess the performance and effectiveness of our proposed language model for online psychological consultation. This section provides a comprehensive analysis of the model's performance based on intrinsic and human evaluation metrics. We discuss the results obtained from metrics such as perplexity, ROUGE-L, and Distinct-n, which shed light on language generation quality, similarity to reference text, and diversity of the generated responses. Additionally, we present the outcomes of the human evaluation, which includes scores given by evaluators based on metrics such as Helpfulness, Fluency, Relevance, and Logic. Through these rigorous evaluations, we aim to provide an in-depth understanding of the strengths and weaknesses of our language model and its suitability for the task of online psychological consultation.


\subsection{Result of Intrinsic Evaluation}

The results of the intrinsic evaluation comparing the performance of the \emph{PanGu} model and the \emph{WenZhong} model are presented in \cref{table:intrinsic-metric}. The metrics used for evaluation include perplexity, ROUGE-L, Distinct-1, and Distinct-2.

As shown in~\cref{table:intrinsic-metric}, the \emph{PanGu} model outperforms the \emph{WenZhong} model across all metrics. The \emph{PanGu} model achieves a lower perplexity value of 34.56 compared to 38.40 for the \emph{WenZhong} model, indicating that the \emph{PanGu} model better predicts the sample probabilities in the dataset.

Furthermore, the ROUGE-L score, which measures the similarity between the generated responses and the reference text, is higher for the \emph{PanGu} model (28.18) than the \emph{WenZhong} model (23.56). This suggests that the \emph{PanGu} model generates responses more aligned with the expected answers.

In terms of diversity in generated responses, the \emph{PanGu} model also exhibits higher Distinct-1 (4.57) and Distinct-2 (12.74) scores compared to the \emph{WenZhong} model (3.55 and 9.67, respectively). This indicates that the \emph{PanGu} model produces a wider range of distinct unigrams and bigrams in its responses, leading to a more diverse language generation.

The intrinsic evaluation results consistently demonstrate that the \emph{PanGu} model performs better than the \emph{WenZhong} model across all metrics. These findings highlight the superior language generation capabilities of the \emph{PanGu} model in the context of online psychological consultation.


\begin{table}[tb]
    \centering
    \caption{The result of Intrinsic Evaluation between two models. \label{table:intrinsic-metric}}
    \begin{tabular}{@{}ccccc@{}}
    \toprule
    Model & Perplexity & Rouge-L & Distinct1 & Distinct2  \\ \midrule
    WenZhong & 38.40  & 23.56 & 3.55 & 9.67\\
    PanGu & 34.56 & 28.18 & 4.57 & 12.74  \\
    \bottomrule
    \end{tabular}
\end{table}


\subsection{Case Study: Examining Psy-LLM's responses on coping with depression}

In the following, we will qualitatively evaluate the outputs of both models by asking our Psy-LLM a common question for individuals suffering from depression.

\subsubsection{PanGu Model}
To demonstrate the capabilities of the \emph{PanGu} model, we provide an example of question answering using the model on coping with depression (\cref{fig:pangu-demo}), which we have translated into English. The question posed to the model is,
\begin{displayquote}
\emph{``Question: How to deal with depression?''},
\end{displayquote}
and the model's response is as follows:
\begin{displayquote}
\emph{``Answer: Hello, first let me give you a small hug. I hope it brings you some warmth. Typically, individuals with depression are often misunderstood and even scolded by their families, which can be really disheartening. I understand that you are feeling very sad and disappointed with your family situation. I want you to know that you are a remarkable person, and you are not at fault. Remember, change starts with you, and you are incredibly strong and sensible.''}
\end{displayquote}

This response showcases the model's fluency and ability to employ different strategies.
Analysing the response in~\cref{fig:pangu-demo}, we can observe that the model utilises various strategies. The initial part, offering a virtual hug, demonstrates a reassurance strategy~\cite{psyqa}. It provides emotional support and encouragement to individuals seeking consultation, helping them relax. The second part, addressing family dynamics, employs a restatement strategy by clearly reiterating the question to provide a more realistic understanding of the individual's mental health condition. The final part utilises a guidance strategy, advising the individual on steps to improve their situation. Overall, the model's response is coherent and consistent with normal human logic, making it highly valuable.

The training of the \emph{PanGu} model consists of two steps. Initially, the model is trained with a vast amount of crawled data, enabling it to learn extensive domain knowledge in psychology. Subsequently, the model undergoes fine-tuning using the downstream PsyQA dataset. During this stage, the model primarily focuses on learning the pattern of \emph{``Question:''} and \emph{``Answer:''} to generate responses in the same format as the downstream tasks. When we input a question in the format \emph{``Question: $\cdots$''} to the model, the model outputs a response in the format \emph{``Answer: $\cdots$''}. This demonstrates that the model has successfully learned the desired pattern and can generate appropriate responses.

% Figure environment removed

\subsubsection{WenZhong Model}


% Figure environment removed


In \emph{WenZhong} model, after fine-tuning the model using Psychology data, the model can output very suitable answer.
We also utilised the \emph{WenZhong} model to answer the same question (\cref{fig:WenZhong-demo}), and the generated response was:




\begin{displayquote}
\emph{``Our experience tells us that there are many times when you just have to do things by your own will, then there will be a lot of discomfort. For example, if you are studying, the time of day, the place of study, the attitude of study, the ability to study, etc., these must be treated carefully, especially these should be avoided as much as possible, for example, then if you do not want to do something negative, leave it alone and do as much as possible to do some preventive measures for depression, so that it will not lead to the development of this disease.''}
\end{displayquote}


This example demonstrates that the \emph{WenZhong} model can provide suitable answers in the field of psychology. However, one challenge we encountered is that some generated outputs do not directly address the question. This issue may be attributed to the limited fine-tuning data specific to psychology. In order to further improve the performance of the \emph{WenZhong} model in psychology-related tasks, a larger and more diverse dataset from the field of psychology could be incorporated during the fine-tuning process.




\subsection{Human Evaluation}

To conduct an empirical evaluation of Psy-LLM's effectiveness, we enlisted the participation of six students from the psychological faculty to assess a set of 200 question-answer pairs generated by our language model. In order to obtain a comprehensive understanding of the model's performance, we employed two evaluation methods for the participants to provide ratings on the responses.
We have created a web front-end for users to access our Psy-LLM platform, and their technical details are discussed in~\cref{sec:web}.

The first method directly compared responses generated by both the \emph{PanGu} and the \emph{WenZhong} models in response to the same question. Evaluators assigned scores to these answers based on predetermined metrics, enabling a clear and direct comparison between the two models' performance.
In the second method, we presented evaluators with a combined set of predicted and actual responses. This allowed them to evaluate and assess the differences and similarities between the generated responses and the ground truth answers.

By utilising these human evaluation methods, we aim to gain valuable insights into the performance of our language model, particularly in terms of the disparities between predicted and actual responses. This comprehensive evaluation approach will provide a deeper understanding of the model's capabilities and guide further improvements in its performance for online psychological consultation.


The human evaluation results, using two different methods, are presented in~\cref{table:human-eval-1,table:human-eval-2}. These evaluate human-perceived metrics of \emph{Helpfulness, Fluency, Relevance, and Logic}.
\Cref{table:human-eval-1} shows the results of the first human evaluation method, where evaluators provided scores for each metric. Consistent with the findings from the intrinsic evaluation, the \emph{PanGu} model outperforms the \emph{WenZhong} model in terms of \emph{Helpfulness} (3.87 vs. 3.56), \emph{Fluency} (4.36 vs. 4.14), \emph{Relevance} (4.09 vs. 3.87), and \emph{Logic} (3.83 vs. 3.63). These results indicate that human evaluators generally consider the \emph{PanGu} model's generated responses more helpful, fluent, relevant, and logical than the \emph{WenZhong} model.

However, a notable observation is made when comparing the scores obtained in~\cref{table:human-eval-1} with the scores from~\cref{table:human-eval-2}. \Cref{table:human-eval-2} presents the scores for the predicted answers of both models as well as the actual answers. Interestingly, the scores for the actual answers are significantly higher than those for the predicted answers of both models across all metrics. This discrepancy suggests that the evaluators, who had the opportunity to compare the actual answers with the predicted answers, marked the predicted answers relatively lower. This finding highlights the importance of incorporating human evaluation in assessing the performance of language models and the need for further improvement in generating more accurate and satisfactory responses.

In summary, the human evaluation results align with the intrinsic evaluation findings, indicating that the \emph{PanGu} model performs better than the \emph{WenZhong} model. However, it is important to note that the scores for the actual answers are considerably higher than those for the predicted answers, implying room for improvement in the generated responses of the language models.






\begin{table}[tb]
    \centering
    \caption{Average Human ratings of Psy-LLM responses, only with the two AI-powered versions. \label{table:human-eval-1}}
    \begin{tabular}{@{}ccc@{}}
    \toprule
    Metrics & WenZhong & PanGU  \\ \midrule
    Helpfulness & 3.56  & 3.87\\
    Fluency  & 4.14 & 4.36   \\
    Relevance & 3.87 & 4.09 \\
    Logic & 3.63 & 3.83 \\
    \bottomrule
    \end{tabular}
\end{table}

\begin{table}[tb]
    \centering
    \caption{Average Human ratings of Psy-LLM responses, alongside the ground-truths from datasets. \label{table:human-eval-2}}
    \begin{tabular}{@{}cccc@{}}
    \toprule
    Rating Metrics & WenZhong & PanGU & Ground Truth \\ \midrule
    Helpfulness & 3.45  & 3.54 & 4.52\\
    Fluency  & 3.95 & 4.12 & 4.83   \\
    Relevance & 3.77 & 3.96 & 4.72 \\
    Logic & 3.61 & 3.75 & 4.56 \\
    \bottomrule
    \end{tabular}
\end{table}




\section{Web-Interface for Accessible Online Consultation}\label{sec:web}

One of the primary objectives was to explore the provision of online AI-powered consultation and question-and-answer services in psychology. We adopted a distributed architecture, separating the model's front-end, back-end, and computing servers into modular components. Each module was developed with distinct responsibilities, allowing for easier upgrades and interchangeability of combinations. Communication between the modules was achieved through API interactions, enabling them to function independently without relying on the internal functionality of other modules.

Furthermore, we placed a strong emphasis on security during the design process. We implemented measures to encrypt and protect our modular systems at a product level. The common API interface was productised and encrypted, ensuring secure communication between the components. Additionally, we implemented the HTTPS web system architecture, enhancing security by encrypting each cloud server with TLS (SSL).
By adopting a distributed and modular approach and prioritising security, we aimed to address the challenges of hosting a large-scale online consultation service model. These design choices allowed for flexibility, scalability, and enhanced security in our system architecture, contributing to our project's overall success and reliability.


\subsection{Web Technologies}

We utilised the following services and technologies for our website development:

\begin{itemize}
    \item \emph{ReactJS}: ReactJS was our front-end framework due to its extensive library support. ReactJS offers a wide range of reusable components and follows a modular, component-based architecture, making designing and enhancing the front end easier. ReactJS is responsive and provides excellent cross-platform support.
    \item \emph{AWS Amplify}: AWS Amplify is a rapid front-end deployment service provided by Amazon. It enables us to quickly deploy the front end of our website and seamlessly communicate with other system components. Amplify provides fully managed CI/CD (Continuous Integration/Continuous Deployment) and hosting, ensuring fast, secure, and reliable encryption services.
    \item \emph{Google Domain}: We utilised Google Domain services for secure encapsulation of our EC2 host DNS.
    \item \emph{Amazon EC2}: EC2 provides virtual server instances with highly available underlying designs. It offers reliable, scalable, and flexible access in terms of cost and performance. EC2 provides powerful computing resources and pre-configured environments, making it an excellent choice for running large models. Its robust network performance and high-performance computing clusters allow for high throughput and low-latency online processing. We used simple Flask-based scripts to handle concurrent requests.
    \item \emph{Python \& Flask}: We used Python as our scripting language to run the models and build APIs. Flask, a web framework written in Python, was used for creating API endpoints and handling request-response interactions.
    \item \emph{Apache}: We used Apache, an open-source web server software, for configuring port forwarding, reverse proxies, and listening.
    \item \emph{Let's Encrypt \& Certbot}: We employed Let's Encrypt and Certbot for TLS (Transport Layer Security) encryption, ensuring secure communication between the website and users.
\end{itemize}




% Figure environment removed
The diagram in~\cref{fig:web-arc} illustrates the architecture of our independently developed web system for the cloud-based site. The website's user interface is accessible through the front end, deployed on the AWS Amplify service. Built on the ReactJS framework, the front-end communicates with the back-end database through an internal API, enabling storage of user evaluation data for model effectiveness optimisation. The database is hosted within Amplify Hosting. The standalone website interacts with the model runtime server via a public API. To ensure privacy and protect the host address, we register a public domain name through Google Domains and link it to the host server's DNS.

The pre-trained model is deployed on an Amazon EC2 instance host configured as an AWS Linux virtual server. The model uses Python code and Flask scripts, allowing for local server calls. Apache is used for HTTP reverse proxy communication, forwarding external model input data to the local server where the model is waiting and generating results.
To provide secure HTTPS encryption for the web products deployed on AWS Amplify, we employ TSL encryption for the EC2 instance DNS addresses. This is achieved using Let's Encrypt and Certbot as cost-effective alternatives to commercial SSL certificates.

The website's front end is designed with simplicity, featuring an input box for users to enter Chinese questions, as depicted in~\cref{fig:web-init}. Upon submission, the system communicates with the back-end model through the API. It awaits the completion of model processing (\cref{fig:web-load}) before returning the results to the output box, as depicted in (\cref{fig:web-done}). Users can rate the results using the built-in rating system, and there is a link to an additional evaluation site at the bottom of the page.

% Figure environment removed


% Figure environment removed





\section{Discussion}

The discussion section provides a comprehensive analysis of the project outcome, product perspective, website perspective, model perspective, and evaluation perspective.


\subsection{Project Outcome}

We have successfully developed and implemented an effective chatbot for mental health counselling. Through the training and fine-tuning large-scale Chinese pre-training models on mental health datasets, the chatbot has acquired valuable knowledge in psychology, enhancing its ability to provide counselling services. The deployment of the chatbot on a website interface has created a convenient and accessible platform for users seeking mental health support. Although the chatbot is currently in its prototype stage, our project demonstrates the feasibility of building an AI-based counselling system. It is a valuable reference for future research and development in this area.

From a model perspective, our evaluation results demonstrate the superiority of the \emph{PanGu} model over the \emph{WenZhong} model, as expected due to its larger size and advanced architecture. The \emph{PanGu} model's design contributes to its outperformance, particularly its incremental learning ability and enhanced natural language understanding capabilities. However, both models fall short of achieving human-level performance, which can be attributed to the quality of the training dataset and the inherent limitations of autoregressive language models. Enhancing the dataset quality and exploring alternative language model architectures hold promise for addressing these limitations and further improving performance.


\subsection{Evaluation Perspective}

The human evaluation results indicate that both the \emph{PanGu} model and the \emph{WenZhong} model have yet to reach human-level performance. Even though training the models on our dataset crawled from websites, the predicted answers strongly focus on psychological content but lack logical coherence. One potential reason for this is the quality of our dataset, which may need to be higher to provide comprehensive and reliable training examples. Although we conducted human evaluation during the data cleaning stage, the sheer volume of data made it challenging to cover every instance. To address this, we recommend thoroughly evaluating the website data before crawling to ensure a higher-quality dataset.

Another factor impacting human evaluation is the limited computing conditions during model training. Our model requires a specific training environment and numerous parameters, making it time-consuming to adjust and fine-tune it effectively. We could not optimise the parameters and achieve optimal testing results with our current resources. Consequently, the model's performance may have been hindered by these limitations.
Furthermore, the autoregressive nature of both the \emph{PanGu} model and the \emph{WenZhong} model poses challenges in comprehending contextual information. As autoregressive language models, their training processes are unidirectional, focused on modelling the joint probability from left to right. The next predicted word is solely based on the preceding word, limiting their ability to capture information from broader contexts. This lack of contextual background reference makes it difficult for language models to handle reading comprehension tasks like humans.

In summary, the evaluation results shed light on the areas where improvements can be made. Enhancing the dataset quality through pre-evaluation and addressing the limitations of our computing conditions are crucial steps toward advancing the model's performance. Additionally, exploring alternative language model architectures that can effectively capture contextual information may contribute to bridging the gap between model-generated responses and human-level performance.

\subsection{Product and Practicality Perspective}

The performance of the online consultation service indicates its significant potential for streamlining mental health support with minimal resources. The user experience has been a priority in product design, and the cloud infrastructure deployment ensures easy access via mobile devices. As part of our future improvements, we plan to incorporate an automatic emotion recognition system into the website, enabling the identification of users in distress and facilitating timely intervention. The design and development of our product hold substantial societal value in the mental health support field, providing a promising avenue for further exploration and refinement.

Regarding the website, we have designed and implemented a modern, cloud-based network architecture that boasts lightweight, scalable, and highly secure features. This architecture allows for low-cost, large-scale model computing sites, enabling widespread accessibility to AI-based question-and-answer services. Our approach serves as a reference for small organisations and enterprises with limited resources, showcasing the possibilities of deploying AI capabilities effectively.





\section{Limitations and Future Works}

While we have presented promising results with our Psy-LLM model for usage in assisting mental health workers, our study is exploratory in nature, and hence, there exist numerous limitations that we would like to raise in the following.

\subsection{Model Capability and Usage in Real-World}
While there are numerous benefits in deploying an AI-powered Large Language Model for supporting the demand in the mental health sector, one should consider several ethical and practical issues.
Firstly, as a language-based model, the model's output is based purely on the input text. 
However, studies have shown that nonverbal communication is one of the key factors in counselling outcome~\citep{hill1981nonverbal}.
In fact, a well-trained counsellor can often pick up subtle cues even when there is a lack of response from the patient.
Standalone LLM models like Psy-LLM cannot address such an issue (unless techniques like facial emotion detection from the computer vision community are integrated as a unified system~\citep{jaiswal2020facial}).
Furthermore, rapport-building with clients is often a crucial step in clinical psychology.
However, an AI-based model would face severe difficulties in building trusted client relationships.
As a result, it is critical to realise that such an AI-powered system cannot replace real-world counselling setups.
A practical approach would be to pair the model output under the supervision of a trained counsellor as a good psychoeducational tool.
The model output can be used as an initial guideline or suggestion for assisting human counsellors in providing useful and trusted consultations with patients.

\subsection{Data Collections}

Several strategies can be implemented in future work to overcome the limitations in data collection. Firstly, to address the issue of anti-crawler rules on different websites, developing a more robust and adaptable crawler that can handle different anti-crawler mechanisms would be beneficial. The access limitation could involve implementing dynamic IP rotation or utilising proxies to avoid IP blocking. Machine learning techniques, such as automatic rule extraction or rule adaptation, could also help automate handling anti-crawler mechanisms.

Incorporating more advanced data-cleaning techniques can also improve the quality of the crawled data. More advanced data-cleaning procedures may involve NLP methods, such as entity recognition, part-of-speech tagging, and named entity recognition to identify and filter out irrelevant or noisy data. Additionally, leveraging machine learning algorithms, such as anomaly detection or outlier detection, can aid in identifying and removing low-quality or erroneous data points.
In terms of dataset standardisation, establishing a unified standard for data generation in the online domain would greatly facilitate the cleaning process. This could involve collaborating with website administrators or data providers to develop guidelines or formats for data representation. Furthermore, using human annotators or experts in the domain to manually review and clean a subset of the dataset can provide valuable insights and ensure a higher-quality dataset.

However, it is important to acknowledge that achieving a completely clean dataset is challenging, particularly when dealing with large-scale datasets. As such, future work should strike a balance between the manual review and automated cleaning techniques while also considering the cost and scalability of the data cleaning process.

\subsection{Model Improvement}

Increasing the scale of model training by utilising larger models or ensembles of models can enhance the performance and capabilities of the chatbot. Larger models can capture more nuanced patterns and relationships in the data, leading to more accurate and coherent responses.
Exploring different model architectures beyond autoregressive language models may provide valuable insights. Bidirectional models (e.g. Transformer-XL) or models that incorporate external knowledge sources (e.g. knowledge graphs) can improve the chatbot's contextual understanding and generate more informative responses.
Moreover, integrating feedback mechanisms into the training process can help iteratively improve the chatbot's performance. This could involve collecting user feedback on the generated responses and incorporating it into the model training through reinforcement learning or active learning.

Several disadvantages were also identified in the LLM architecture. Firstly, the maximum likelihood training approach of the \emph{WenZhong} model is susceptible to exposure bias, which occurs when samples are drawn from the target language distribution. This bias can lead to errors for which researchers have yet to find effective solutions. Additionally, training the \emph{WenZhong} model multiple times can significantly decrease its quality.
Furthermore, the \emph{WenZhong} model follows an autoregressive architecture, which models joint probability from left to right. This unidirectional training process limits its ability to capture information from all contexts, particularly hindering its performance in tasks requiring reading comprehension that rely on contextual background references.
Similar to the \emph{WenZhong} model, the \emph{PanGu} model also exhibits autoregressive characteristics. Although it inherits the ability to estimate the joint probability of language models, it suffers from the same limitations of unidirectional modelling. It lacks bidirectional context information and may produce duplicate results requiring resolve deduplication.

We also have reservations about the Jieba tokeniser used in the \emph{PanGu} model. Its performance and tokenisation ability need to handle complex Chinese tokenisation accurately. Furthermore, as neural networks and pre-trained models advance, Chinese NLP tasks increasingly demonstrate that tokenisation is only sometimes necessary. Large models can effectively learn character-to-character relationships without word segmentation. For instance, Google is considering discarding tokenisation and using bytes directly. Adopting a more flexible tokeniser could make the model more suitable for various industrial applications, even when sacrificing some performance.


\subsection{User Experience and User Interface}
Enhancing the chatbot's user experience and user interface can significantly impact its adoption and effectiveness. Future work should focus on improving the simplicity, intuitiveness, and accessibility of the website interface. This includes optimising response times, refining the layout and design, and incorporating user-friendly features such as autocomplete suggestions or natural language understanding capabilities.

Furthermore, personalised recommendations and suggestions to users based on their preferences and previous interactions can enhance the user experience. Techniques like collaborative filtering or user profiling can enable the chatbot to understand better and cater to individual user needs.
Usability testing and user feedback collection should be conducted regularly to gather insights on user preferences, pain points, and suggestions for improvement. Iterative design and development based on user-centred principles can ensure that the chatbot meets user expectations and effectively addresses their mental health support needs.


\subsection{Ethical Considerations and User Privacy}
As with any AI-based system, ethical considerations and user privacy are paramount. Future work should address these concerns by implementing robust privacy protection mechanisms and ensuring transparency in data usage. This includes obtaining explicit user consent for data collection and usage, anonymising sensitive user information, and implementing strict data access controls.
Developing mechanisms to handle potentially sensitive or harmful user queries is crucial. The chatbot should have appropriate safeguards and guidelines to avoid providing inaccurate or harmful advice. Integrating a reporting system where users can report problematic responses or seek human intervention can help mitigate potential risks.
Furthermore, monitoring and auditing the chatbot's performance and behaviour can help identify and rectify biases or discriminatory patterns. Regular evaluations by domain experts and user feedback analysis can improve the chatbot's reliability, fairness, and inclusivity.



While this project has made significant progress in developing an AI-based chatbot for mental health support, there are various limitations and areas for improvement. Overcoming challenges related to data quality, model performance, ethical considerations, and user experience will contribute to the overall effectiveness and reliability of the chatbot. By addressing these limitations and exploring future research directions, we can continue to advance the field of AI-powered mental health support systems and provide valuable assistance to individuals in need.


\section{Conclusion}

In conclusion, our project on Psy-LLM, an exploratory study on using Large Language Models as an assistive mental health tool, has been successfully completed and implemented. While there are areas identified for improvement based on specific evaluation indicators, we are confident that with improved equipment conditions, we can enhance the performance of this platform. The experimental results obtained from this project hold significant potential to contribute to the fields of supportive natural language generation and psychology, driving advancements at the intersection of these domains. The deployment of such a system offers a practical approach to promoting the overall mental well-being of our society by providing timely responses and support to those in need.









\printbibliography

\end{document}
