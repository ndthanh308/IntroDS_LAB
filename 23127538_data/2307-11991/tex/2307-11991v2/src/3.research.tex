\section{Mental Health and Social Well-being in Overly Populated Cities}

The availability of mental health professionals has always been a major problem in overpopulated cities such as China.
The World Health Organisation has reported that the prevalence of depression in China exceeds 54 million people even before the onset of the COVID-19 pandemic~\citep{world2017depression}.
The situation has been exacerbated by the implementation of quarantine measures and social distancing, leading to a worsening condition~\citep{gou2022province}.
Unfortunately, only a small fraction of the affected population receives adequate medical treatment, as there are only 2 psychiatrists per 100,000 people in China~\citep{xiang2018rethinking}.
Consequently, there is a pressing need for a dynamic system that can assist patients effectively. Contemporary conversational chatbots have demonstrated their ability to emulate human-like conversations.

Hence, it is imperative to develop a user-friendly AI-based chatbot specifically designed to address anxiety and depression, with the aim of improving the user's emotional well-being by providing relevant and helpful responses.
This project aims to construct a Chinese psychological dialogue model capable of comprehending the semantic meaning of a consultant's request and offering appropriate advice, particularly to address the shortage of mental health workers during demanding periods. 
The trained model will be integrated into a website, featuring a user interface (UI) that ensures ease of operation, thereby enhancing the efficiency of psychological counselling.





\subsection{Research Questions}

Psychology is an intricate and advanced discipline gaining increasing significance as society progresses. However, due to its high barriers to entry, resources for psychological counselling have long been scarce. In numerous cases, individuals face challenges accessing adequate mental health support~\citep{19004}. Furthermore, the high cost of psychological counselling often prevents many individuals from prioritising their mental well-being. This issue is particularly prominent in China, a country with a large population where psychological problems have been historically overlooked. China needs a robust foundation for psychological counselling, including a deficient knowledge base and limited data. Consequently, intelligent assistance in psychology must be improved in the Chinese context.

Traditional psychological counselling primarily focuses on privacy and employs a one-on-one question-and-answer approach, inherently leading to inefficiencies. However, in today's high-pressure society, where mental health issues are pervasive, relying solely on scarce psychologists is arduous. Additionally, influenced by traditional culture, individuals often hesitate to acknowledge and address their psychological problems due to feelings of shame and perceiving such discussions as signs of weakness~\citep{sandhu}. This reluctance is especially prominent when conversing with real humans, let alone seeking assistance from unfamiliar psychologists.
Furthermore, with the advancement of modern Natural Language Processing (NLP) artificial intelligence models, there is a possibility to optimise the conventional and widely adopted question-and-answer model specifically for the field of psychology. When interacting with AI, people are more inclined to express their true thoughts and emotions without fear of prejudice and discrimination than real humans.
However, the lack of verbal cues and continuous monitoring of patients' emotional progress is also a cause of concern in an online psychological consultation context, even when performed by human counsellors~\citep {novella2022comparison}.


\subsection{Research Scope}
Through our project, we aim to make a meaningful contribution to the field of mental well-being. By leveraging our AI model and proposing a framework for an internet-accessible consultation, we intend to enhance the accessibility of mental health support, making it more affordable and providing an avenue for psychological question-and-answer interactions, particularly to address the shortage of human counsellors for mental health support.
To achieve this goal, we gather professional counselling question-and-answer data and psychologically relevant knowledge data to construct a robust question-and-answer model specific to this domain. The success of our project relies on the utilisation of high-quality question-and-answer models. We plan to employ established Chinese pre-trained models with exceptional human-computer interaction and communication skills, characterised by fluent language, logical reasoning, and semantic understanding. However, these existing pre-trained models need more specialised psychological expertise and emotional understanding for counselling purposes. To address this limitation, we intend to integrate two models, namely the \emph{WenZhong} model and the \emph{PanGu} model, and evaluate their performance to determine the more suitable choice as our final model.

Subsequently, it is imperative to make our model accessible to a broader audience. Leveraging the internet provides the most effective means to accomplish this objective. Together with the guidance of professional mental health experts, our model can provide an additional venue for the general public to access mental health support by easing the stress and demand on mental health staff through an open and online platform.