% Figure environment removed

For locomotion onboard Nadia, we rely on controlling the instantaneous capture point (ICP) \cite{pratt2006capture}.
The ICP is a linear combination of the CoM position and velocity defined as 
\begin{equation}
    \mathbf{\xi} = \mathbf{x} + \frac{1}{\omega} \dot{\mathbf{x}},
\end{equation}
where $\mathbf{\xi}$ is the ICP position, $\mathbf{x}$ and $\mathbf{\dot{x}}$ are the CoM position and velocity, and $\omega = \sqrt{g / \Delta z_{com}}$ is the natural frequency of the inverted pendulum. 
The ICP dynamics are
\begin{equation}
    \dot{\mathbf{\xi}} = \omega \left( \mathbf{\xi} - \mathbf{r}_{\text{ecmp}}\right),
    \label{eqn:icp_dynamics}
\end{equation}
where $\mathbf{r}_{\text{ecmp}}$ is the enhanced centroidal moment pivot (eCMP) \cite{englsberger2017smooth}, which directly controls the ICP. 
The main concept of ICP control is to control the divergent dynamics with the eCMP location through either foot placement, ankle torques, or angular torque about the CoM,  so that the convergent dynamics of the CoM are indirectly stabilized. 
By placing the eCMP directly at the location of the ICP, the ICP has zero velocity, allowing the CoM position to converge over time. 

Because the ICP dynamics are first order, Eq. \ref{eqn:icp_dynamics} has the solution
\begin{equation}
    \mathbf{\xi}(t) = e^{\omega t} \left( \mathbf{\xi}_0 - \mathbf{r}_{\text{ecmp}} \right) + \mathbf{r}_{\text{ecmp}},
    \label{eqn:icp_trajectory}
\end{equation}
assuming $\mathbf{r}_{\text{ecmp}}$ is held constant throughout $t$. 
This is important, as it provides a closed form solution for where the ICP will be when the step is finished at time $T_r$. 
This, then, sets the required step location for the robot to maintain stability.

However, unless the robot has point feet and is a point mass, it is not constrained to using a fixed eCMP location, which is required by Eq. \ref{eqn:icp_trajectory}.
Instead, the robot has an allowable set of control inputs available, $\mathbf{r}_{\text{ecmp}} \in \mathcal{U}$. 
If the angular momentum rate is assumed to be zero and there is no change in height, this is equivalent to saying $\mathbf{r}_{\text{ecmp}}$ must remain within the foot, which is defined as a convex hull in practice.
In doing so, we can easily calculate the set of possible future ICP positions for all possible control inputs for time $t \in \left[ t_{min}, \infty \right)$.
This defines the \textit{one step capture region}, $\mathcal{C}_1$, shown in Fig. \ref{fig:one_step_capture_region}, as the region in which the robot must step to come to a stop in a single step.
Computing $\mathcal{C}_1$ is straightforward (see \cite{pratt2012capturability} for details), and 
can be used in some step adjustment algorithm that places the current step in $\mathcal{C}_1$.
As $C_1$ is, by construction, convex, this algorithm can be as simple as an orthogonal projection of the nominal step position onto $C_1$.

Using the ICP dynamics and a reference eCMP trajectory from from the desired steps, we can compute a reference ICP trajectory \cite{seyde2018inclusion}.
We can define a general CoM trajectory as
\begin{equation}
\begin{array}{l}
    \mathbf{x}(t) = \mathbf{c}_{0} e^{\omega t} + \mathbf{c}_{1} e^{-\omega t} + \mathbf{c}_{2} t^3 + \mathbf{c}_{3} t^2 + \mathbf{c}_{4} t + \mathbf{c}_{5},
    \end{array}
    \label{eqn:com_trajectory}
\end{equation}
which is directly the solution to the inverted pendulum dynamics \cite{kajita20013d}, assuming a  cubic eCMP trajectory.
The unknown coefficients in Eq. \ref{eqn:com_trajectory} are found by solving a constrained linear system.
These constraints can be defined as initial and final eCMP positions and velocities from the eCMP trajectory, CoM position and velocity continuity constraints at each knot point, and an initial CoM position constraint for the first segment and a terminal ICP position for the last.
This is notably similar to the approach presented in\cite{englsberger2017smooth}, but instead solving for the coefficients simultaneously as opposed to recursively, which has the benefit of flexibility constraint definition.
