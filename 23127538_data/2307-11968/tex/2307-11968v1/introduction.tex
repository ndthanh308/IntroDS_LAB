The ability to maintain stability while walking is an inherent requirement for humanoid robots to successfully perform any task that uses locomotion.
Balancing is a skill at which humans are exceptional, with people capable of climbing sheer cliff faces, walking tightropes, running over stepping stones, and performing parkour. 
For bipedal balance specifically, several distinct strategies have been observed: shifting the Center of Pressure within the base of support (the ``ankle strategy") \cite{horak1986central}, generating angular momentum (the ``hip strategy") \cite{pijnappels2010armed}, and changing the base of support (the ``step strategy") \cite{maki1997role}. 
Utilizing these individual mechanisms for balance regulation, along with additional strategies such as varying the timing of contacts, is essential for humanoid robots to properly navigate the world in a stable fashion.

% Figure environment removed

Much of the work during the DARPA Robotics Challenge era for humanoid robots centered around properly tracking the Center of Mass (CoM) dynamics to stably execute desired footsteps \cite{Koolen_2016}.
A subsequent focus on increasing the dynamic capabilities of humanoids and closing the gap with their biological counterparts has lead to the development of a number of strategies for step adjustment for stability.
Many approaches have leveraged reduced order models to compute the best step for stability, such as the Linear Inverted Pendulum (LIP) \cite{feng2016robust}, the Instantaneous Capture Point (ICP) \cite{pratt2006capture, pratt2012capturability}, the Divergent Component of Motion (DCM) \cite{griffin2017walking, englsberger2017smooth,khadiv2020walking}, and more recently, the Angular LIP (ALIP)\cite{gong2021one}.
Model predictive control (MPC) has also become a popular technique to encode this step adjustment, leveraging reduced order models for simplicity \cite{di2018dynamic} but also using the full dynamics \cite{meduri2023biconmp}.

However, algorithms designed to leverage step adjustment often consider simple constraints on the adjustment, with little attention paid to the option for cross-over. 
One unique approach proposes uses Linear Temporal Logic to determine where to step from the LIP dynamics, including possible cross-over, and then full body kinematic optimization to avoid collisions \cite{gu2022reactive}.
Other works using MPC can generate footsteps that \textit{may} be capable of emergent cross-over due to their ability to avoid self-collisions \cite{khazoom2022humanoid}.
Alternatively, the reachable regions with cross-over have been used in a convex MPC by decomposing the non-convex region into convex regions, which are then handled as feasibility constraints in the MPC \cite{habib2022handling}.
This MPC also includes ankle torques for shifting the robot's weight.
Notably, none of these MPC-based works have demonstrated successful results on a hardware platform.
This leaves a gap in the understanding of the benefits of cross-over recovery mode is for bipedal stability. 
In fact, there's little understanding of how the inclusion of cross-over may affect stability on robots. 
Recent human studies, however, have found a strong link between the use of cross-over stepping and degradation of tracking (via disturbance) towards the end of the swing phase, indicating this recovery mode (along with jumping) is a significant mechanism employed by people\cite{leestma2023linking}.



While less explored than step adjustment strategies, step timing adaptation has garnered significant recent interest. 
However, the nonlinearity of the system dynamics with respect to time make this challenging.
It has been shown to be sufficient to vary only the next step time and position in a simple walking pattern generator with a point foot to maintain stability \cite{khadiv2020walking}, demonstrating the power of time as a control input.
The simple point foot allows treating the nonlinear term in the  dynamics as a scalar function of time. 
Timing has also been included as a decision variable in MPC \cite{aftab2012ankle, kryczka2015online, carpentier2016versatile}.
However this results in a non-convex and nonlinear problem that can be challenging to solve on real hardware with computational constraints.
The MPC problem can also be posed using a time-optimal parameterization \cite{caron2017make}, where, instead of formulating the timing between control ticks using direct transcription, the MPC parameterizes the resulting trajectory as a function of time.
Contact timing optimization can be avoided by using contact implicit trajectory optimization, where the actual event of contact is encoded in complimentarity conditions for interaction forces with the environment \cite{cleac2021fast,  posa2014direct}.
Outside of MPC, explicitly considering step timing has been limited to \textit{swing} time, with little-to-no attention provided to the effects of varying \textit{transfer} time, which is often omitted in controllers based on point foot models \cite{khadiv2020walking}.

In this work, we propose a number of enhancements that enable the robot to achieve both specific  footholds in the world while also adjusting the step positions and times as necessary. 
First, we present a novel controller for momentum shaping using the ankle and hip strategy in a compact QP that is decoupled from step adjustment. 
We next present enhancements to the calculation of capture regions for bipedal locomotion to better consider how constraints on where the robot can step may affect the ability to recover. 
We then highlight a new strategy for performing cross-over steps to maintain stability, which greatly enhances the variety of tracking error from which the robot may recover.
Our last contribution is an adaptation of our previous work on adjusting the swing duration \cite{griffin2017walking} to the transfer phase, enabling the robot to effectively ``skip" this period of double support when the stability of the system is compromised. 

