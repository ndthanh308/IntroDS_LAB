While the reachability-aware capture regions provide a better model for step recovery, they do highlight the fundamental limitations encountered by preventing cross-over steps.
Our robot, Nadia, was specifically designed with a range of motion at the hip roll joint that would allow for cross-over, as shown in Fig. \ref{fig:nadia-crossover}.
To include this in our capture region calculation, however, we need to redefine the reachability constraint shown in Fig. \ref{fig:reachability_constraint}.

% Figure environment removed

To address this, we can create a new reachability constraint that allows for cross-over, shown in Fig. \ref{fig:crossover_reachability}.
We can define the shape of this region by imposing a maximum forward and backward cross-over distance, $w_\text{fwd}$ and $w_\text{bwd}$, as well as a cross-over angle $\theta_\text{fwd}$ and $\theta_\text{bwd}$.
This allows different cross-over amounts in the forward and backward directions, with the angle enabling stance leg collision avoidance.
However, this new reachability region is non-convex, making much of the computation and constraint formulation significantly more complex.
Instead, similar to \cite{habib2022handling}, we can decompose this non-convex region into three convex regions: one for forward cross-over $\mathcal{R}_\text{fwd}$, one for backward cross-over $\mathcal{R}_\text{bwd}$, and the original reachability constraint with no cross-over $\mathcal{R}_b$.

% Figure environment removed

While this methodology is not dissimilar to previous works \cite{habib2022handling}, instead of selecting the region based on the effect when applied to a MPC, we define a set of rules to determine the appropriate constraint based on $\mathcal{C}_N$. 
The first rule is if there is overlap between $\mathcal{C}_N$ and  $\mathcal{R}_{\text{b}}$, apply this constraint for $\mathcal{R}_1$.
The second rule is, if there is no intersection between $\mathcal{R}_\text{b}$ and $\mathcal{C}_N$,  pick the region between $\mathcal{R}_{\text{fwd}}$ and $\mathcal{R}_{\text{bwd}}$ that has the most intersection with $\mathcal{C}_N$.
This can be thought of as selecting the reachability constraint that provides the largest factor of safety.
This prioritization of the base reachability constraint before using either of the cross-over constraints ensures that cross-over is only used when $\mathcal{R}_\text{b}$ is insufficient to stabilize the system.



The third and final rule is, if there is no intersection between any reachability constraint and $\mathcal{C}_N$, select the reachability constraint region that is closest to $\mathcal{C}_N$. 
From the definition of the capture region, there being no intersection between $\mathcal{C}_N$ and $\mathcal{R}_1$  strictly means that there is no way for the robot to recover, as it is impossible to step in $\mathcal{C}_N$.
However, as mentioned in Sec. \ref{sec:multi_step_capture_regions}, when we calculate $\mathcal{C}_N$, we do not consider the ``ankle" or ``hip" strategies as part of the feedback, or the ability to step more quickly. 
This means that $\mathcal{C}_N$ is a conservative estimate of the \textit{real} multi-step capture region.
Because of this, we enable the robot to continue to try to regain balance, even if there are no explicitly feasible recovery steps available.

