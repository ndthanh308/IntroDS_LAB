% Figure environment removed

We used the presented walking controller in both simulation and hardware experiments on our robot, Nadia.
For our first experiment, to investigate how the different stabilization mechanisms affect the robot's balance, we ran simulations with different feedback mechanisms enabled in the following combinations:
only ICP control from Sec. \ref{sec:capture_point_control}; ICP control and step adjustment; ICP control with step and swing duration adjustment; ICP control with step, swing, and transfer duration adjustment; and all mechanisms--ICP control, step with cross-over, swing, and transfer duration adjustment.
The robot was told to walk in place, and then a disturbance was applied to the pelvis at 0.25\% of the way through swing with the right foot for 0.1s.
We then recorded the magnitude of the maximum recoverable disturbance, and mapped this to a change in velocity to normalize for push duration and robot weight. The results are shown in Fig. \ref{fig:sim_recoverable_disturbances}.
The step timings were 0.7 s in swing and 0.3 s in transfer, similar to what is used on hardware. For reachability, $l_\text{max} = l_\text{min} = 1.0m, w_\text{min} = 0.125m, w_\text{max} = 0.8m, w_\text{nom} = 0.25m, w_\text{fwd}=0.1m, w_\text{bwd} = -0.05m, \theta_\text{fwd} = 20^\circ$ and $\theta_\text{bwd}=30^\circ$.
As can be seen in Fig. \ref{fig:sim_recoverable_disturbances}, the addition of each new control mode all increase the disturbance that the robot can recover from. 
The only exception is that including transfer adaptation does not seem to increase recovery when the push is slightly towards the outside and mostly forward.
When comparing pushing towards the inside to outside, only ICP control alone shows greater recovery, likely due to the increased support polygon from the next step.
From these plots, it is easy to see the expansion of the recovery region via cross-over steps results in a greatly increased ability to recover from inward disturbances.
The exception is when they're slightly backward, which would require stepping directly through the stance foot, which is not allowed in the reachability constraint.


% Figure environment removed

As a feasibility proof for the ICP controller in Sec. \ref{sec:capture_point_control} to stabilize the robot on rough terrain, we set up a cinder block field, shown in Fig. \ref{fig:nadia_rough_terrain}, for the robot to traverse. 
Footsteps were provided by an operator for the robot to execute. 
%Step timings were \hl{what were the timings?}.
As shown in Fig. \ref{fig:nadia_rough_terrain} and in the supplemental video, the robot was able to stably execute these steps to climb this terrain.

In the first push recovery hardware experiment, we commanded the robot to walk forward, and applied disturbances to the torso, as shown in Fig. \ref{fig:nadia_push_recovery}.
In this experiment, the robot is using a swing duration of 0.6s and transfer duration of 0.2s, and the disturbance shown is applied at the very end of the right swing. 
The subsequent transfer duration is executed in 0.03s, the next swing in 0.4s, the following transfer in 0.17s. 
At this point, the robot starts to stabilize, executing the second step in 0.49s.
The robot made a backward step adjustment of approximately 30cm to balance.
The step adjustment employed by the robot is shown in Fig. \ref{fig:nadia_data_treadmill_stepadjustment}.
This also shows how the robot moved the eCMP in the base of support to try to recover balance.
This demonstrates the combined techniques of swing duration, transfer duration, and step position adjustment are effective for regaining balance.

% Figure environment removed


% Figure environment removed

To test the benefits of using cross-over steps on the robot hardware, we directed the robot to walk in place and applied disturbances to the torso towards the inside of the step. 
Images from this experiment can be seen in Fig. \ref{fig:nadia_crossover_recovery}, with data in Fig. \ref{fig:nadia_crossover_data}.
After the disturbance, the robot crossed the swing leg in front of the stance leg within the reachability constraint, and then adjusted outward on the subsequent step, to then resume stepping in place.
Additionally, in Fig. \ref{fig:nadia_crossover_data}, it can be seen that the feedback eCMP saturates  to a maximum distance outside the right side of the support polygon to help maintain stability.
If the robot had been unable to perform the cross-over step, the left recovery step would have been placed several (approx. 10) centimeters further left, requiring an extremely large subsequent step, as every additional centimeter the foot moves requires $10cm$ of step adjustment based on the $~0.73s$ swing duration used.
From this, it is highly unlikely that the robot would have been able to recover without the use of cross-over steps.


% Figure environment removed


