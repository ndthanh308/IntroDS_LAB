\documentclass[letterpaper, 10 pt, conference]{ieeeconf}  % Comment this line out if you need a4paper

%\documentclass[a4paper, 10pt, conference]{ieeeconf}      % Use this line for a4 paper

\IEEEoverridecommandlockouts                              % This command is only needed if 
                                                          % you want to use the \textbf{Definition:}\thanks command

\overrideIEEEmargins                                      % Needed to meet printer requirements.

\usepackage{graphics} % for pdf, bitmapped graphics files
\usepackage{epsfig} % for postscript graphics files
%\usepackage{mathptmx} % assumes new font selection scheme installed
\usepackage{times} % assumes new font selection scheme installed
\usepackage{amsmath} % assumes amsmath package installed
\usepackage{amssymb}  % assumes amsmath package installed
\usepackage{multirow}
\usepackage{color,soul}
\usepackage{algorithm}
\usepackage{algpseudocode}

\usepackage{hyperref}
%\usepackage[natbib=true]{biblatex}
\usepackage[comma,numbers]{natbib}
\renewcommand*{\bibfont}{\footnotesize}
\DeclareMathOperator*{\argmin}{arg\,min}


\newtheorem{definition}{Definition}[section]

\bibliographystyle{IEEEtran}

\addtolength{\topmargin}{4pt}

\title{\LARGE \bf
Reachability Aware Capture Regions with Time Adjustment and Cross-Over for Step Recovery}

\author{Robert Griffin$^{1,2,3}$, James Foster$^{2}$, Stefan Fasano$^{1}$, Brandon Shrewsbury$^{1,3}$, Sylvain Bertrand$^{1}$% <-this % stops a space
\thanks{This work was funded through ONR Grant N00014-19-1-2023, NASA Grant No. 80NSSC20M0197, and DAC Cooperative Agreement W911NF-21-2-0241. Email: \url{rgriffin@ihmc.org}.}% <-this % stops a space
\thanks{$^{1}$Author is with the Florida Institute for Human and Machine Cognition.}%
\thanks{$^{2}$Author is with the University of West Florida.}%
\thanks{$^{3}$Author is with Boardwalk Robotics.}} 
%
\usepackage{amsmath}
\usepackage{mathtools}
\usepackage{thmtools}
\usepackage{cancel}
\usepackage{wrapfig}


\newcommand{\cmark}{\ding{51}}%
\newcommand{\xmark}{\ding{55}}%

\usepackage{tcolorbox}
\tcbset{boxsep=0mm,boxrule=0pt,colframe=white,arc=0mm,left=0.5mm,right=0.5mm}

\newcommand\SC{\mathcal{S}}
\newcommand{\antonio}[1]{{\color{magenta} Antonio: ``#1''}}

\DeclareMathOperator*{\argmax}{arg\,max}
\DeclareMathOperator*{\argmin}{arg\,min}

\newcommand{\theHalgorithm}{\arabic{algorithm}}

\usepackage[capitalize,noabbrev]{cleveref}


\DeclareMathOperator{\LRU}{LRU}

\newcommand\A{\mathbf{A}}
\newcommand\V{\mathbf{V}}
\newcommand\B{\mathbf{B}}
\newcommand\C{\mathbb{C}}
\newcommand\Exp{\mathbb{E}}
\newcommand\R{\mathbb{R}}
\newcommand\calM{\mathcal{M}}
\newcommand\calR{\mathcal{R}}
\newcommand\rank{\operatorname{rank}}
\newcommand\eps{\varepsilon}
\newcommand\h{h}
\newcommand\bound{b}
\DeclareMathOperator{\tr}{tr}
\DeclareMathOperator{\vect}{vec}
\DeclareMathOperator{\diag}{diag}




\begin{document}

\bstctlcite{IEEEexample:BSTcontrol}

\maketitle
\thispagestyle{empty}
\pagestyle{empty}

\begin{abstract}
For humanoid robots to live up to their potential utility, they must be able to robustly recover from instabilities.
In this work, we propose a number of balance enhancements to enable the robot to both achieve specific, desired footholds in the world and adjusting the step positions and times as necessary while leveraging ankle and hip.
This includes improving the calculation of capture regions for bipedal locomotion to better consider how step constraints affect the ability to recover. 
We then explore a new strategy for performing cross-over steps to maintain stability, which greatly enhances the variety of tracking error from which the robot may recover.
Our last contribution is a strategy for time adaptation during the transfer phase for recovery. 
We then present these results on our humanoid robot, Nadia, in both simulation and hardware, showing the robot walking over rough terrain, recovering from external disturbances, and taking cross-over steps to maintain balance.
\end{abstract}

\section{Introduction}
\label{introduction}

\section{Introduction}
Current quantum hardware is unable to carry out universal quantum computations due to the buildup of errors that occur during the computation. 
The magnitude of the individual error is currently above the value that the Threshold Theorem requires in order to kick-start quantum error correction and fault-tolerant quantum computation~\cite[Section 10.6]{nielsen_chuang_2010}. 
Although the experimentally achieved fidelity rates are promising and the error bounds are inching closer to the required threshold, we will have to work for the foreseeable future with quantum hardware with errors that build-up during the computation.  This implies that we can only do a limited number of steps before the output of the computation has become completely uncorrelated with the intended one.

For fault-tolerant quantum computing, we repeat four steps: 
1) We apply a number of single and two-qubit quantum gates, in parallel whenever possible; 
2) We perform a syndrome measurement on a subset of the qubits; 
3) We perform fast classical computations to determine which errors have occurred and how to correct them; 
and, 4) We apply correction terms based on the classical computations.
We then repeat these four steps with a next sequence of gates. 
These four steps are essential to fault-tolerant quantum computing. 


The starting point of this work is to use the four steps outlined above, not to carry out error correction and fault-tolerant computation, but to enhance short, constant-depth, {\em uncorrected} quantum circuits that perform single qubit gates and {\em nearest-neighbor} two qubit gates. 
Since in the long run we will have to implement error-correction and fault-tolerant computation anyhow, and this is done by such a four-step process, why not make other use of this architecture? Moreover, on some of the quantum hardware platforms, these operations are already in place.
Embracing this idea we naturally arrive at the question: what is the computational power of \textit{low-depth} quantum-classical circuits organized as in the four steps outlined above? 
We thus investigate circuits that execute a small, ideally constant, number of stages, where at each stage we may apply, in parallel, single qubit gates and {\em nearest-neighbor} two qubit gates, followed by measurements, followed by low-depth classical computations of which the outcome can control quantum gates in later stages. 
It is not clear, at first, whether such circuits, especially with constant depth, can do anything remotely useful. 
But we will see that this is indeed the case: many quantum computations can be done by such circuits in constant depth. 
By parallelizing quantum computations in this way, we improve the overall computational capabilities of these circuits, as we do not incur errors on qubits that are idle, simply because qubits are not idle for a very long time. 
Furthermore, reducing the depth of quantum circuits, at the cost of increasing width, allows the circuit to be run faster even if errors occur.

The first usage of such a four-step layout, not to do error correction, but to perform computations, can be found in the paradigm of measurement-based quantum computing~\cite{gottesman1999demonstrating,raussendorf2001one,jozsa2006introduction,clark2007generalised}: 
A universal form of quantum computing where a quantum state is prepared and operations are performed by measuring qubits in different bases, depending on previous measurements and intermediate measurements.

\citeauthor{PhamSvore2013} were the first to formalize the four-step protocol for performing computations~\cite{PhamSvore2013}. They included specific hardware topologies by considering two-dimensional graphs for imposing constraints on qubit interactions. In their model, they develop circuits for particularly useful multi-qubit gates, including specifying costs in the width, number of qubits, depth, number of concurrent time steps, size, and total number of non-Identity operations.
As a result, they find an algorithm that factors integers in polylogarithmic depth.
\citeauthor{Browne:2011} showed that the main tool in the work by \citeauthor{PhamSvore2013}, the fan-out gate, can also be replaced by additional log-depth classical computations in the measurement-based quantum computing setting~\cite{Browne:2011}.

More recently, \citeauthor{Cirac:2021} introduced a scheme to implement unitary operations involving quantum circuits combined with Local Operations and Classical Communication ($\mathsf{LOCC}$) channels: $\mathsf{LOCC}$-assisted quantum circuits~\cite{Cirac:2021}. Similarly to the four-step scheme we just described, they allow for a short depth circuit to be run on the qubits, followed by one round of $\mathsf{LOCC}$, in which ancilla qubits are measured and local unitaries are applied based on the measurement outcomes. They show that in this model any 1D transitionally invariant matrix-product state (MPS) with fixed bond dimension is in the same phase of matter as the trivial state. Similar ideas can be found in~\cite{TVV_NonAbelianTopologicalOrder_2022, tantivasadakarn2021long}.

In this work, we introduce a new model, called \textit{Local Alternating Quantum-Classical Computations} ($\LAQCC$). In this model we alternate between running quantum circuits (constrained by locality), ending in the measurement of a subset of qubits, and fast classical computations based on the measurement results. The outcome of the classical computations are then used to control future quantum circuits. We allow for flexibility in this model, by giving different constraints to the power of both the quantum circuits and the classical circuits as well as the number of alternations between them. 
Most attention will be given to $\LAQCC$ containing quantum circuits of constant depth, classical circuits of logarithmic depth and at most a constant number of alternations between them. 
Any circuit constructed in this model is considered to be of constant depth. 
We restrict ourselves to logarithmic depth classical computations, as this is the first natural and non-trivial extension beyond constant-depth classical computations. 
Constant-depth classical computations do however also have an equivalent constant-depth quantum implementation.

The definition of $\LAQCC$ sharpens the original definition of \citeauthor{PhamSvore2013} by adding constraints to the intermediate classical computations. This allows us to bound the power of $\LAQCC$ from above. 

The main result of \citeauthor{Cirac:2021}, that 1D translational invariant MPS with fixed bond dimension can be prepared by $\mathsf{LOCC}$-assisted circuits, relies on local symmetries of the MPS. These symmetries allow them to prepare local states (on a constant number of qubits) and glue them together by doing one round of the appropriate entangling measurement and corrections, after which they run a round of local unitaries to get the desired result. This general scheme for preparing states that exhibit an MPS description with the appropriate local symmetries requires only geometrically local unitaries and one round of measurement and corrections an therefore is accessible in $\LAQCC$. Studying different local symmetries, known as Symmetry Protected Topological (SPT) phases of matter, to find measurement-based constant depth circuits for states is a broad ongoing field of research~\cite{TVV_NonAbelianTopologicalOrder_2022, tantivasadakarn2021long, smith2023deterministic}. 
All these schemes have a $\LAQCC$ implementation.

%$\LAQCC$-circuits also exist for general schemes of preparing local states, based on the local tensors, and gluing them together using one round of entangled measurement and corrections, based on the local symmetry. 
%The main result of \citeauthor{Cirac:2021}, that 1D translational invariant MPS with fixed bond dimension can be prepared by $\mathsf{LOCC}$-assisted circuits, relies heavily on local symmetries of the MPS and as a result also has an equivalent $\LAQCC$ implementation. 
%The corrections applied after the measurement round are local unitaries depending on the local symmetries of the MPS. 

 

%This general scheme of preparing local states, based on the local tensors, and gluing it together by doing one round of entangled measurement and corrections, based on the local symmetry, is accessible in $\LAQCC$.
Note however that \citeauthor{Cirac:2021} also suggest a circuit for the $W$-state.
This circuit uses sequentially and dependent measurement-based corrections of the ancilla qubits. 
These dependent measurements translate to sequential alternations between the quantum and classical circuits and therefore increase the total depth to linear depth, exceeding the constant-depth constraints imposed by $\LAQCC$-circuits. 

We study the power of the $\LAQCC$ model with respect to state preparation, showing that even with only constant quantum-depth and logarithmic classical depth it remains possible to prepare states with long-range entanglement.
Another surprising result is that it is unlikely that $\LAQCC$ circuits are classically simulatable. We show that any instantaneous quantum polynomial-time (IQP) circuit~\cite{Bremner2010,Shepherd2009} has an $\LAQCC$ implementation.
Classical simulation of IQP circuits implies the collapse of the polynomial hierarchy to the third level, which is not believed to be true~\cite{Bremner2017}. Therefore, we expect that $\LAQCC$ circuits are unlikely to be classically simulatable. We bound the power of $\LAQCC$ by showing that it is contained in $\QNC^1$, the class of polynomial-size, log-depth circuits.

Next, we also study the power that intermediate classical calculations can add to quantum computations, by considering a new model that alternates between polynomially many polynomial-depth quantum circuits and unbounded classical computations
We study this model by doing a complexity theoretical analysis, where we draw inspiration from the notions of complexity given by \citeauthor{RosenthalYuen:2022}, \citeauthor{MetgerYuen:2023}, and \citeauthor{Aaronson:2004}.
All three complexity notions are based on the notion of state preparation, instead of more traditional definition of complexity such as the decidability of a computational problem. 
The first two consider classes based on sequences of quantum states preparable by a polynomial-sized quantum circuit, where the circuits are uniformly generated by a computational class, for instance, the class $\mathsf{PSPACE}$, which results in the complexity class $\mathsf{StatePSPACE}$~\cite{RosenthalYuen:2022,MetgerYuen:2023}.
The third notion considers a relative complexity, where the complexity is measured between two given states, and is measured by the number of gates, from a given gate-set, required to transform one state in another state~\cite{Aaronson:2004}. 
For our definition of state preparation complexity, we drop the uniformity constraint from~\cite{RosenthalYuen:2022,MetgerYuen:2023} and define a class as $\mathsf{StateX}$, which refers to states preparable by circuits of type $\mathsf{X}$. 
As an example, if $\mathsf{X} = \QNC^0$, this results in the class $\mathsf{StateQNC^0}$, which is the set of states preparable from the $\ket{0}^n$ state by poly-size constant-depth circuits. 
This notion is similar to the relative complexity from~\cite{Aaronson:2004}, where one state is the  $\ket{0}^n$ state and instead of counting the number of gates we consider the set of states preparable by a fixed number of gates. Using this notion of complexity we show that any state preparable by an $\LAQCC^*$ circuit is also preparable by a $\mathsf{PostQPoly}$ circuit, the class of circuits of polynomial depth with an additional post-selection gate. 

All Clifford circuits have a constant-depth $\LAQCC$ implementation, implying that any stabilizer state can be implemented by a constant-depth $\LAQCC$ circuit, see Section~\ref{sec:clifford_circuits} for a proof of this statement. 
Efficient circuits for stabilizer states have been known already through measurement-based quantum computing. Therefore this paper focuses on the preparation of non-stabilizer states, and as a surprising result we find novel constant-depth protocols for four very natural classes of non-stabilizer states.
Despite the extensive research into these four classes of non-stabilizer states and the many applications of them, no efficient constant- or low-depth state preparation protocols are known yet. We specifically consider these four classes as they are all often used as initial states in other algorithms.

The first state is a uniform superposition over an arbitrary number of states. 
This state finds applications in many quantum algorithms, as they often start with a uniform superposition over multiple states. 
This superposition is often achieved by applying Hadamard gates to every qubit due to its simplicity to prepare. 
Yet, the analysis of many algorithms, such as Shor's algorithm~\cite{Shor:1997}, would benefit from a different initial superposition. 
The circuit to prepare the uniform superposition over an arbitrary number of states uses an exact version of Grover search as a subroutine, that turns a probabilistic circuit, with a known constant probability of success, into a deterministic circuit. 
We use the circuit for preparing a uniform superposition over an arbitrary number of states as a subroutine in the next two quantum state preparation protocols. 

The second state is the $W$-state, the uniform superposition over all computational basis states of Hamming-weight~$1$, a natural long-ranged entangled state that displays a fundamentally nonequivalent type of entanglement from the Greenberger–Horne–Zeilinger state~\cite{WState:2000}, for which $\LAQCC$-type constant-depth circuits were previously known~\cite{PhamSvore2013, Cirac:2021}. 
The $W$-state is often used as benchmark for new quantum hardware~\cite{Haffner2005,Neeley2010,GarciaPerez:2021}. 
A novel way to prepare the $W$-state therefore gives a new way to benchmark different quantum devices with each other. 
A circuit for preparing the $W$-state was given in~\cite{Cirac:2021}, but this implementation requires sequentially alternating measurements followed by local unitaries, which in the $\LAQCC$ model is not considered to be of constant depth. 
We improve this protocol by giving an $\LAQCC$ implementation of the $W$-state, based on a compress-uncompress method that links the one-hot and binary encoding of integers.

The third state considered is the Dicke state, a generalization of the $W$-state, a superposition over all computational basis states with Hamming-weight $k$~\cite{Dicke:1954}. 
Dicke states have relevance in various practical settings.
For instance, for quantum game theory~\cite{zdemir2007}, quantum storage~\cite{Bacon_Compress:2006,Plesch:2010}, quantum error correction~\cite{ouyang2014permutation}, quantum metrology~\cite{toth2012multipartite}, and quantum networking~\cite{prevedel2009experimental}. 
Dicke states have been used as a starting state for variational optimization algorithms, most notably Quantum Alternating Operator Ansatz (QAOA)~\cite{Hadfield2019}, to find solutions to problems such as Maximum k-vertex Cover~\cite{Brandhofer2022,cook2020quantum}.
The ground states of physical Hamiltonians describing one-dimensional chains tend to show a resemblance to Dicke states such as states resulting from the Bethe ansatz, making them an ideal starting state when investigating the ground state behavior of these Hamiltonians~\cite{TDL_BetheAnsatzDerivation:2010,B_ExcitedStateQuantumPhaseTransitions:2013,DickeTransitions:2021}. 
For instance, the algorithm by \citeauthor{van2021preparing}, who give an algorithm to prepare the Bethe ansatz eigenstates of the spin-1/2 XXZ spin chain, starts by first preparing a Dicke state~\cite{van2021preparing}. 
A Dicke-state preparation protocol based on the compress-uncompress methodology used in the $W$-state furthermore finds applications in entanglement distillation, where the entanglement of a large state is concentrated on only a few qubits. 
Efficient deterministic circuits for preparing Dicke states have been proposed by \citeauthor{bartschi2019deterministic}~\cite{bartschi2019deterministic, bartschi2022deterministic_short_depth}. 
They provide a quantum circuit of depth $\mathO(k \log(\frac{n}{k}))$, allowing arbitrary connectivity, to prepare a Dicke state, which they conjecture to be optimal when $k$ is constant. 
In this work, we provide a constant-depth $\LAQCC$ circuit below their conjectured bound already for constant $k$. 
However, this does not directly disprove their conjecture, as we allow for intermediate measurements and classical computations. 
More significantly, we even construct constant-depth $\LAQCC$ circuits for $k = \mathO(\sqrt{n})$ greatly improving their bound.
This construction extends the compress-uncompress method for the $W$-state combined with additional subroutines. 

We continue with a log-depth state preparation protocol for the Dicke-state for arbitrary $k$. 
This protocol implements an efficient transformation between the factoradic number representation and the combinatorial number representation of a positive integer. 
The combinatorial number representation relates directly to the Dicke state. 
The provided efficient transformation between number representation systems might be of independent interest. 

We conclude by modifying our protocol for preparing a Dicke-state to a protocol that prepares quantum many-body scar states in constant-depth. 
These states have low entanglement and longer coherence times than states with similar energy density.
These characteristics make many-body scar states interesting to analyze and relevant within physics.
Many-body scar states appear for instance in the AKLT model~\cite{AKLT:1987,MRBAR:2018,MRB:2018} and different spin models~\cite{SI:2019,MOBFR:2020}.
Known methods for preparing these states have polynomial-depth~\cite{Gustafson:2023}, whereas our circuit has constant depth. 

% We conclude by studying the power that intermediate classical calculations can add to quantum computations. 
% In this study, we define a new model that relaxes constant-depth quantum circuits to polynomial depth quantum circuits, log-depth classical calculations to unbounded classical computations and a constant number of alternations to a polynomial number of alternations. 
% We call this model $\LAQCC^*$. 
% We study this model by doing a complexity theoretical analysis, where we draw inspiration from the notions of complexity given by \citeauthor{RosenthalYuen:2022}, \citeauthor{MetgerYuen:2023}, and \citeauthor{Aaronson:2004}.
% All three complexity notions are based on the notion of state preparation, instead of more traditional definition of complexity such as the decidability of a computational problem. 
% The first two consider classes based on sequences of quantum states preparable by a polynomial-sized quantum circuit, where the circuits are uniformly generated by a computational class, for instance, the class $\mathsf{PSPACE}$, which results in the complexity class $\mathsf{StatePSPACE}$~\cite{RosenthalYuen:2022,MetgerYuen:2023}.
% The third notion considers a relative complexity, where the complexity is measured between two given states, and is measured by the number of gates, from a given gate-set, required to transform one state in another state~\cite{Aaronson:2004}. 
% For our definition of state preparation complexity, we drop the uniformity constraint from~\cite{RosenthalYuen:2022,MetgerYuen:2023} and define a class as $\mathsf{StateX}$, which refers to states preparable by circuits of type $\mathsf{X}$. 
% As an example, if $\mathsf{X} = \QNC^0$, this results in the class $\mathsf{StateQNC^0}$, which is the set of states preparable from the $\ket{0}^n$ state by poly-size constant-depth circuits. 
% This notion is similar to the relative complexity from~\cite{Aaronson:2004}, where one state is the  $\ket{0}^n$ state and instead of counting the number of gates we consider the set of states preparable by a fixed number of gates. Using this notion of complexity we show that any state preparable by an $\LAQCC^*$ circuit is also preparable by a $\mathsf{PostQPoly}$ circuit, the class of circuits of polynomial depth with an additional post-selection gate. 

\paragraph{Summary of results}
\begin{itemize}
    \item We give a new definition of a computational model that captures the power of the four step process: applying a constant number of layers of one- and two-qubit gates; performing a syndrome measurement; perform a fast classical computation determining corrections; apply corrections. We call this model \emph{Local Alternating Quantum Classical Computations}, or $\LAQCC$ for short. In this model we bound the allowed quantum operations, intermediate classical calculations, and number of rounds separately. In Section~\ref{sec:LAQCC_model} we define this model and give a list of operations based on results from literature contained in this computational model. In some of these operations we explicitly use that we allow for multiple, but at most constant, rounds  of corrections.
    \item  We show show that there exist $\LAQCC$ circuits that can not be weakly simulated in Section~\ref{sec:IQP_in_LAQCC}. We further show that for every $\LAQCC$ circuit there exists a $\QNC^1$ circuit simulating it perfectly, in Section~\ref{sec:LAQCC_in_QNC1}.
    \item We introduce a new type computational complexity for preparing states and show that the extension of $\LAQCC$ where we allow a polynomial number of rounds and unbounded classical computation, is contained in $\mathsf{PostQPoly}$, the class of polynomial circuits with post-selection, in Section~\ref{sec:Complexity results}.
    \item We show a protocol to prepare the uniform superposition state of size $q$ in $\LAQCC$ using $\mathO(\ceil{\log_2(q)}^2)$ qubits in Section~\ref{sec:superposition_modulo_q}. 
    \item We show a protocol to prepare the $W_n$ state in $\LAQCC$ using $\mathO(n\log(n))$ qubits in Section~\ref{sec:W_state_in_LAQCC}.
    \item We show two ways of preparing the Dicke-$(n,k)$ state. The first method is in $\LAQCC$, works up to $k = \mathO(\sqrt{n})$, uses $\mathO(n^2\log(n))$ qubits, and is found in Section~\ref{sec:dicke:small_k}. The second method is in $\LAQCC\text{-}\mathsf{LOG}$ (an extension of $\LAQCC$ allowing for logarithmic number of alterations instead of constant), works for any $k$, uses $\mathO(\text{poly}(n))$ qubits, and is found in Section~\ref{sec:Dicke_in_LAQCC_LOG}. 
    \item We extend on our $\LAQCC$ method of generating Dicke-$(n,k)$ states for $k = \mathO(\sqrt{n})$ and show a protocol to generate many-body scar states for a particular Hamiltonian in $\LAQCC$ (Section~\ref{sec:many_body_scar}). 
\end{itemize}
Summarized in a table, we provide the following state generation protocols:
\begin{table}[htb]
\centering
\begin{tabular}{l|l|l|l}
\textbf{State description} & \textbf{Width} & \textbf{Depth} & \textbf{Implementation}\\
\hline 
Uniform superposition mod $q$: $\frac{1}{\sqrt{q}} \sum_{i = 0}^{q-1}\ket{i}$ & $\mathO(\ceil{\log^2 q})$ & $\mathO(1)$ & Section~\ref{sec:superposition_modulo_q}\\

$W$-state: $\frac{1}{\sqrt{n}}\sum_{i = 0}^{n-1}\ket{e_i}$ & $\mathO(n \log n)$ & $\mathO(1)$ & Section~\ref{sec:W_state_in_LAQCC}\\

Dicke-$(n,k)$, $k = \mathO(\sqrt{n})$: $\binom{n}{k}^{-1/2}\sum_{x \in \{0,1\}^n: |x| = k} \ket{x}$ &  $\mathO(n^2\log n)$ & $\mathO(1)$ 
&Section~\ref{sec:dicke:small_k}\\

Dicke-$(n,k)$: $\binom{n}{k}^{-1/2}\sum_{x \in \{0,1\}^n: |x| = k} \ket{x}$ & $\mathO(\text{poly}(n))$ & $\mathO(\log n)$ &Section~\ref{sec:Dicke_in_LAQCC_LOG}\\

QMBS: $\ket{S_k} = \frac{1}{k! \sqrt{\mathcal N(n,k)}}(Q^\dagger)^k \ket{\Omega}$ &  $\mathO(n^2\log n)$ & $\mathO(1)$  &  Section~\ref{sec:many_body_scar}
\end{tabular}
\caption{Summary of state preparation protocols given in this paper.}
\label{tab:sate_prep}
\end{table}
In the entry for the quantum many-body scar state $Q$ denotes the raising operator and $\mathcal N(n,k)=\binom{n-k-1}{k}$. 
Section~\ref{sec:many_body_scar} will provide more details on the variables and the implementation. 

\paragraph{Organization of the paper}
\noindent We first introduce relevant preliminaries in Section~\ref{sec:preliminaries}. 
In Section~\ref{sec:LAQCC_model} we formally define the class of Local Alternating Quantum-Classical Computations ($\LAQCC$). We also show that any Clifford circuit can be implemented in constant depth $\LAQCC$ (a result based on a result from measurement-based quantum computing~\cite{jozsa2006introduction}). 
This result allows us to give many useful multi-qubit gates and routines in Section~\ref{sec:gates_created_in_LAQCC}. 
Beyond that we show that constant depth $\LAQCC$ circuits are contained in $\QNC^1$ and that any $\mathsf{IQP}$ circuit has an $\LAQCC$ implementation.
We conclude this section with an analysis of a more powerful instantiation of $\LAQCC$ and show an inclusion with respect to the class $\mathsf{PostQPoly}$, which is the class of circuits of polynomial depth with one additional post-selection gate. 
In Section~\ref{sec:state_prep_in_LAQCC} we give $\LAQCC$ circuit implementations for preparing the uniform superposition over an arbitrary number of states, the $W$-state and the Dicke state up to $k = \mathO(\sqrt{n})$. We furthermore give a log-depth circuit implementation for preparing the Dicke state for any $k$. We conclude by showing a $\LAQCC$ circuit for generating many body scar states of a particular type of Hamiltonian.



\section{Capture Point and Capture Region}
% Figure environment removed

For locomotion onboard Nadia, we rely on controlling the instantaneous capture point (ICP) \cite{pratt2006capture}.
The ICP is a linear combination of the CoM position and velocity defined as 
\begin{equation}
    \mathbf{\xi} = \mathbf{x} + \frac{1}{\omega} \dot{\mathbf{x}},
\end{equation}
where $\mathbf{\xi}$ is the ICP position, $\mathbf{x}$ and $\mathbf{\dot{x}}$ are the CoM position and velocity, and $\omega = \sqrt{g / \Delta z_{com}}$ is the natural frequency of the inverted pendulum. 
The ICP dynamics are
\begin{equation}
    \dot{\mathbf{\xi}} = \omega \left( \mathbf{\xi} - \mathbf{r}_{\text{ecmp}}\right),
    \label{eqn:icp_dynamics}
\end{equation}
where $\mathbf{r}_{\text{ecmp}}$ is the enhanced centroidal moment pivot (eCMP) \cite{englsberger2017smooth}, which directly controls the ICP. 
The main concept of ICP control is to control the divergent dynamics with the eCMP location through either foot placement, ankle torques, or angular torque about the CoM,  so that the convergent dynamics of the CoM are indirectly stabilized. 
By placing the eCMP directly at the location of the ICP, the ICP has zero velocity, allowing the CoM position to converge over time. 

Because the ICP dynamics are first order, Eq. \ref{eqn:icp_dynamics} has the solution
\begin{equation}
    \mathbf{\xi}(t) = e^{\omega t} \left( \mathbf{\xi}_0 - \mathbf{r}_{\text{ecmp}} \right) + \mathbf{r}_{\text{ecmp}},
    \label{eqn:icp_trajectory}
\end{equation}
assuming $\mathbf{r}_{\text{ecmp}}$ is held constant throughout $t$. 
This is important, as it provides a closed form solution for where the ICP will be when the step is finished at time $T_r$. 
This, then, sets the required step location for the robot to maintain stability.

However, unless the robot has point feet and is a point mass, it is not constrained to using a fixed eCMP location, which is required by Eq. \ref{eqn:icp_trajectory}.
Instead, the robot has an allowable set of control inputs available, $\mathbf{r}_{\text{ecmp}} \in \mathcal{U}$. 
If the angular momentum rate is assumed to be zero and there is no change in height, this is equivalent to saying $\mathbf{r}_{\text{ecmp}}$ must remain within the foot, which is defined as a convex hull in practice.
In doing so, we can easily calculate the set of possible future ICP positions for all possible control inputs for time $t \in \left[ t_{min}, \infty \right)$.
This defines the \textit{one step capture region}, $\mathcal{C}_1$, shown in Fig. \ref{fig:one_step_capture_region}, as the region in which the robot must step to come to a stop in a single step.
Computing $\mathcal{C}_1$ is straightforward (see \cite{pratt2012capturability} for details), and 
can be used in some step adjustment algorithm that places the current step in $\mathcal{C}_1$.
As $C_1$ is, by construction, convex, this algorithm can be as simple as an orthogonal projection of the nominal step position onto $C_1$.

Using the ICP dynamics and a reference eCMP trajectory from from the desired steps, we can compute a reference ICP trajectory \cite{seyde2018inclusion}.
We can define a general CoM trajectory as
\begin{equation}
\begin{array}{l}
    \mathbf{x}(t) = \mathbf{c}_{0} e^{\omega t} + \mathbf{c}_{1} e^{-\omega t} + \mathbf{c}_{2} t^3 + \mathbf{c}_{3} t^2 + \mathbf{c}_{4} t + \mathbf{c}_{5},
    \end{array}
    \label{eqn:com_trajectory}
\end{equation}
which is directly the solution to the inverted pendulum dynamics \cite{kajita20013d}, assuming a  cubic eCMP trajectory.
The unknown coefficients in Eq. \ref{eqn:com_trajectory} are found by solving a constrained linear system.
These constraints can be defined as initial and final eCMP positions and velocities from the eCMP trajectory, CoM position and velocity continuity constraints at each knot point, and an initial CoM position constraint for the first segment and a terminal ICP position for the last.
This is notably similar to the approach presented in\cite{englsberger2017smooth}, but instead solving for the coefficients simultaneously as opposed to recursively, which has the benefit of flexibility constraint definition.


\section{Capture Point Control}
\label{sec:capture_point_control}
To achieve locomotion over complex terrain, one approach is to track a reference capture point location, $\mathbf{\xi}_r$, which comes from Eq. \ref{eqn:com_trajectory}. 
This can be done through momentum shaping with an ICP based on LIPM dynamics:
\begin{equation}
\dot{\mathbf{l}}_d = m \left( \omega^2 \left( \mathbf{x} - \mathbf{r}_{\text{ecmp},d}\right) + \mathbf{g} \right),
\label{eqn:momentum_law}
\end{equation}
where $\dot{\mathbf{l}}_d$ that is the net linear momentum rate objective and $\mathbf{r}_{\text{ecmp},d}$ is the desired eCMP position from the controller. 
The use of the eCMP allows the robot to use both linear  and angular momentum for balance (the ``ankle" and ``hip" strategies, respectively) \cite{englsberger2017smooth}.
The challenge is to balance the use of these tasks. 
We also specifically want to decouple this from the step adjustment mechanism, such that step adjustment does its best to maintain balance, and ICP control does its best to maintain tracking.

At the highest level, we can slightly redefine a standard proportional ICP feedback controller \cite{hopkins2014humanoid, griffin2017walking} as
\begin{equation}
\mathbf{r}_{\text{ecmp},d} = \mathbf{k}_p \left( \mathbf{\xi} - \mathbf{\xi}_r \right) + \mathbf{r}_{\text{ecmp},r}, \ \ \ \mathbf{r}_{\text{ecmp},r} = \mathbf{r}_{\text{cop},r} + \mathbf{\kappa}_r,
\end{equation}
where $\mathbf{r}_{\text{cop},r}$ and $\mathbf{r}_{\text{ecmp},r}$ are the reference CoP and eCMP positions values from  Eq. \ref{eqn:com_trajectory} and $\kappa_r$ is the reference difference between the two. 
This feedback task can be written as
\begin{equation}
        \mathbf{\delta} + \mathbf{\kappa} = \mathbf{k}_p \mathbf{\xi}_e,
        \label{eqn:feedback_task}
\end{equation}
where $\delta$ and $\kappa$ encode the CoP and eCMP feedback, respectively, and $\mathbf{\xi}_e = \mathbf{\xi} - \mathbf{\xi}_r$.
This definition results in the optimal controller output, $\mathbf{r}_{\text{cop,d}} = \mathbf{r}_{\text{cop,r}} + \mathbf{\delta}^*$ and $\mathbf{r}_{\text{ecmp,d}} = \mathbf{r}_{\text{cop,d}} + \mathbf{\kappa}^*$.



%% Figure environment removed


 
While these feedback terms can be found directly from Eq. \ref{eqn:feedback_task}, this does not provide a mechanism to balance the use of the CoP and eCMP. 
To do so, we define a basic QP as:
\begin{equation}
\begin{aligned}
    \min_{\kappa, \delta} \quad &  \left\|  \mathbf{\delta} + \mathbf{\kappa} - \mathbf{k}_p \mathbf{\xi}_e \right\|_{Q_e} + \left\|  \left( \mathbf{\delta} + \mathbf{\kappa}\right)^T \mathbf{k}_p \mathbf{\xi}_e \right\|_{Q_\perp} + \\
    & \left\| \delta \right\|_{R_\delta} + \left\| \kappa - \kappa_r \right\|_{R_\kappa} + \left\|  \mathbf{\delta} + \mathbf{\kappa} - \mathbf{\delta}_p - \mathbf{\kappa}_p \right\|_{R_p} 
    \\
\text{s.t.} \quad & \mathbf{A}_{\text{foot}} \left( \mathbf{r}_{\text{cop,r}} + \delta \right) \le \mathbf{b}_{\text{foot}}, \\
& \kappa_{\text{min}} \le \kappa \le \kappa_{\text{max}}.
\label{eqn:qp}
\end{aligned}
\end{equation}
The $Q_e$ task tries to achieve the desired feedback magnitude, the $Q_\perp$ task forces the feedback to be along the appropriate direction, $R_\delta$ and $R_\kappa$ regularize the solution, and $R_p$ penalizes deviations from the previous solution $\delta_p$ and $\kappa_p$ for consistency.
The first constraint forces the output CoP to lie within the support polygon using a half-space formulation, while the second constraint bounds the  angular momentum.

% Figure environment removed

The output regulation $R_p$ is a distinct advantage of the QP-based approach. 
While feedback can be filtered using  direct rate limits or low-pass filters to the output on the feedback controller in Eq. \ref{eqn:feedback_task}, this prohibits rapid changes of force distribution for stability in the face of a sudden constraint change or disturbance.
Penalizing changes in feedback as a cost in the QP, instead, allows leveraging large feedback when necessary but avoiding small changes when not.

Once the desired eCMP is obtained, we can compute $\dot{\mathbf{l}}_d$ using Eq. \ref{eqn:momentum_law}. 
This is then supplied as a task to a whole-body controller along with other motion tasks \cite{Koolen_2016}.
In this case, a desired angular momentum rate is encoded entirely in the $\dot{\mathbf{l}}_d$ task and the other spatial acceleration tasks.
The whole-body controller balances the trade off between the necessary angular momentum rate to achieve $\dot{\mathbf{l}}_d$ and these other accelerations, allowing for perfect tracking if there are no redundant commands during nominal walking, and generating motions like pitching the torso and windmilling the arms when necessary to achieve $\dot{\mathbf{l}}_d$.



\section{Multi-Step Capture Regions}
\label{sec:multi_step_capture_regions}
% Figure environment removed

Significantly degraded tracking from external disturbances, unstructured terrain, or slips may require multiple steps for the robot to recover.
Limiting the robot to deadbeat step adjustment using only a single step for recovery correspondingly places limits on the recoverable error.
On top of this, as the walking speed increases, the reference dynamics are often \textit{not} one step capturable, so requiring one step capturability limits the overall locomotion speed.
To address this, we consider multiple steps for recovery using multi-step capture regions \cite{koolen2012capturability}.
However, as there are reachability constraints limiting the step adjustment, these multi-step capture regions must be enhanced to consider these possibly complex restrictions.

The $N$-step capture region (referred to as $\mathcal{C}_N$) is the region in which the robot must step to avoid falling in $N$ or fewer steps. 
This region is constructed using a series of nested regions, with the innermost being $\mathcal{C}_1$, and growing increasingly larger until $\mathcal{C}_N$ is reached, with $\mathcal{C}_{N-1} \subset \mathcal{C}_N$.
By definition, after a step is taken in region $\mathcal{C}_N$, the system is then in a $N-1$ capturable state, such that the next step can be taken in region $\mathcal{C}_{N-1}$ to recover \cite{koolen2012capturability}.

To compute $\mathcal{C}_N$, we can first define a reachability constraint that restricts the position of step $n$, $\mathbf{r}_{\text{foot}, n}$, to some reachable area, $\mathbf{r}_{\text{foot}, n} \in \mathcal{R}_n$, which can be scaled in size by $\alpha$ and translated by $\delta$, $\alpha \mathcal{R} + \delta$.
From the ICP dynamics, we know that the ICP state at the end of step $n$ is 
\begin{equation}
    \mathbf{\xi}_n = e^{\omega T_s} \left( \mathbf{\xi}_{n-1} - \mathbf{r}_{\text{foot},n-1} \right) +  \mathbf{r}_{\text{foot},n-1},
\end{equation}
where $T_s$ is the step duration. As $\mathbf{r}_{\text{foot}, n} \in \mathcal{R}_n$, we can define the remaining error after step $n$ is taken, $\mathbf{\xi}_{e,n} = \mathbf{\xi}_n - \mathbf{r}_{\text{foot},n}$, as the shortest vector from $\mathbf{\xi}_n$ to $\mathcal{R}_n$, or
\begin{equation}
\mathbf{\xi}_{e,n} = \mathbf{d} \left(\mathbf{\xi}_n, \mathcal{R}_n \right).
\end{equation}
From Eq. \ref{eqn:icp_dynamics}, if the state can be captured on the $N^{th}$ step,  $\mathbf{\xi}_{e,N} = 0$, or
\begin{equation}
    \mathbf{d} \left( \xi_{e, N},  \mathcal{R}_{N}\right) = 0.
    \label{eqn:captured_definition1}
\end{equation}
As $\mathcal{R}_N$ is always defined centered at $\mathbf{r}_{foot,N}$, we can define everything assuming $\mathbf{r}_{foot,N} = 0$. From this, it follows that $\mathbf{\xi}_{e,N} = e^{\omega T_s}$. 
This transforms Eq. \ref{eqn:captured_definition1} to
\begin{equation}
    \mathbf{d} \left( \xi_{e, N-1}, e^{-\omega T_s} \mathcal{R}_{N}\right) = 0.
\end{equation}
This is equivalent to saying that, to be capturable in $N$ steps,  
\begin{equation}
\xi_{e,N-1} \in e^{-\omega T_s} \mathcal{R}_N.
\label{eqn:error_capture}
\end{equation}
This allows us to calculate the maximum amount of additional error that can be rejected by adding step $n$ for recovery, recursing back in time to the end of the current step with $e^{\omega T_s (n-1)}$.

From here, we can define $\mathcal{C}_N$ as the set of all footstep locations that drive the state, $\xi$, to $\mathcal{C}_{N-1}$.
Region $\mathcal{C}_N$, then, can be found as the set of all points $\mathbf{r}$ such that the scaled reachable region intersects the previous capture region, or
\begin{equation} 
\mathcal{C}_N = \left\{ \mathbf{r} \ | \ \left( e^{-\omega T_s \left( N - 1 \right)} \mathcal{R}_N + \mathbf{r} \right) \  \cap \  \mathcal{C}_{N-1} \right\},
\label{eqn:capture_region}
\end{equation}
where $N-1$ captures the recursive nature of this operation.
Then, if the current step is taken in the $C_N$ region, the robot can recover in $N$ steps, with each step being reachable, $\mathbf{r}_n \in \mathcal{R}_n, \forall n = 1, \dots, N$. 

This is a subtly different definition than that presented previously \cite{koolen2012capturability}, as it is designed for arbitrary reachability constraints $\mathcal{R}$.
However, if a simple reachable set is used, defined only by a maximum step length, $l_\text{max}$, where $\mathcal{R}_{l_\text{max}} = \left\{ \mathbf{r} \ | \ \| \mathbf{r} \| \le l_{\text{max}} \right\}$, the resulting capture regions are the same. 
In this case, $\mathcal{C}_N$ is simply a direct expansion of $\mathcal{C}_{N-1}$ by a distance of $e^{-\omega T_s (N-1)} l_{\text{max}}$, resulting in $\mathcal{C}_{1,2,3}$ shown on the top in Fig. \ref{fig:multi_step_capture_regions}. 


% Figure environment removed


It should be noted that, when computing $\mathcal{C}_N$, there are additional control inputs available to the robot besides step position, namely ankle and hip torques, as well as step timing adjustment (as in \cite{griffin2017walking}) that have been ignored in our approach for calculating $\mathcal{C}_N$.
This makes the presented method an interior estimation of the true $\mathcal{C}_N$, and leads to more step adjustment on each step than may be strictly necessary for recovery.
In practice, we prefer this conservative approach, as it provides a better factor of safety for the adjustment. 
Additionally, by using more step adjustment, the robot's gait is less likely to lead to foot tipping or severe torso angles from saturating the ankle and hip control strategies, respectively.

A major limitation of the use of the simple reachability constraint $\mathcal{R}_{l_\text{max}}$, is the assumption that the robot can step anywhere within $l_{\text{max}}$ of its stance foot.
In practice, it is often useful to define some convex reachability constraint for the robot to prevent a number of undesirable events from occurring, including stepping too wide, stepping on its own feet, or colliding with the stance leg. 
To address this, we use an ellipsoidal constraint on the swing foot relative to the stance foot, defined as $\mathcal{R}_b$ and shown in Fig. \ref{fig:reachability_constraint}, which allows constraining the minimum and maximum step width and length.
We approximate this ellipse as a polygon.

% Figure environment removed

We can then compute the capture region using this new ellipsoidal reachable region in Eq. \ref{eqn:capture_region}.
From this, the boundary of $\mathcal{C}_N$ is defined as the foot position at the intersection of the transformed $\mathcal{R}_N$ and the boundary of $\mathcal{C}_{N-1}$. 
As both $\mathcal{R}_N$ and $\mathcal{C}_{N-1}$ are polygons, $\mathcal{C}_N$ is by construction a polygon, and the outer vertices of $\mathcal{C}_N$ come from the intersection between vertices of the transformed $R_N$ and $\mathcal{C}_{N-1}$.
Algorithmically, this is equivalent to calculating $\mathcal{C}_N$ as the valid foot positions when sweeping the scaled $\mathcal{R}_N$ around the perimeter of $\mathcal{C}_{N-1}$, as shown in Fig. \ref{fig:two_step_capture_regions}.

From Fig. \ref{fig:multi_step_capture_regions}, including knowledge of this reachable set strongly affects the shape of $\mathcal{C}_N$, with the results of using $\mathcal{R}_b$ to compute the $\mathcal{C}_N$ shown in the middle.
It is apparent that, because of the minimum width constraint, additional steps do not provide much additional stability when the error is towards the inside of the foot, as recovery requires almost completely stepping in $\mathcal{C}_1$.
This can be seen by the lack of additional area in the inward direction when comparing $\mathcal{C}_2$ to $\mathcal{C}_1$ in the middle of Fig. \ref{fig:multi_step_capture_regions}.
This makes sense; the original, simple reachable set assumes the robot can take a subsequent step of width $l_\text{max}$, when in actuality, by prohibiting cross-over, the closest width for the next step is $-w_{\text{min}}$. 
The second step, then, provides no corrective control in the inward direction as the control authority (defined by $\mathcal{R}_b$) is actually \textit{negative}.
Thus, if $\mathcal{R}_{l_\text{max}}$ was used to compute $\hat{\mathcal{C}}_N$, the robot may have stepped outside the actual $\mathcal{C}_N$, leading to a fall! 

\section{Cross-Over Reachability}
While the reachability-aware capture regions provide a better model for step recovery, they do highlight the fundamental limitations encountered by preventing cross-over steps.
Our robot, Nadia, was specifically designed with a range of motion at the hip roll joint that would allow for cross-over, as shown in Fig. \ref{fig:nadia-crossover}.
To include this in our capture region calculation, however, we need to redefine the reachability constraint shown in Fig. \ref{fig:reachability_constraint}.

% Figure environment removed

To address this, we can create a new reachability constraint that allows for cross-over, shown in Fig. \ref{fig:crossover_reachability}.
We can define the shape of this region by imposing a maximum forward and backward cross-over distance, $w_\text{fwd}$ and $w_\text{bwd}$, as well as a cross-over angle $\theta_\text{fwd}$ and $\theta_\text{bwd}$.
This allows different cross-over amounts in the forward and backward directions, with the angle enabling stance leg collision avoidance.
However, this new reachability region is non-convex, making much of the computation and constraint formulation significantly more complex.
Instead, similar to \cite{habib2022handling}, we can decompose this non-convex region into three convex regions: one for forward cross-over $\mathcal{R}_\text{fwd}$, one for backward cross-over $\mathcal{R}_\text{bwd}$, and the original reachability constraint with no cross-over $\mathcal{R}_b$.

% Figure environment removed

While this methodology is not dissimilar to previous works \cite{habib2022handling}, instead of selecting the region based on the effect when applied to a MPC, we define a set of rules to determine the appropriate constraint based on $\mathcal{C}_N$. 
The first rule is if there is overlap between $\mathcal{C}_N$ and  $\mathcal{R}_{\text{b}}$, apply this constraint for $\mathcal{R}_1$.
The second rule is, if there is no intersection between $\mathcal{R}_\text{b}$ and $\mathcal{C}_N$,  pick the region between $\mathcal{R}_{\text{fwd}}$ and $\mathcal{R}_{\text{bwd}}$ that has the most intersection with $\mathcal{C}_N$.
This can be thought of as selecting the reachability constraint that provides the largest factor of safety.
This prioritization of the base reachability constraint before using either of the cross-over constraints ensures that cross-over is only used when $\mathcal{R}_\text{b}$ is insufficient to stabilize the system.



The third and final rule is, if there is no intersection between any reachability constraint and $\mathcal{C}_N$, select the reachability constraint region that is closest to $\mathcal{C}_N$. 
From the definition of the capture region, there being no intersection between $\mathcal{C}_N$ and $\mathcal{R}_1$  strictly means that there is no way for the robot to recover, as it is impossible to step in $\mathcal{C}_N$.
However, as mentioned in Sec. \ref{sec:multi_step_capture_regions}, when we calculate $\mathcal{C}_N$, we do not consider the ``ankle" or ``hip" strategies as part of the feedback, or the ability to step more quickly. 
This means that $\mathcal{C}_N$ is a conservative estimate of the \textit{real} multi-step capture region.
Because of this, we enable the robot to continue to try to regain balance, even if there are no explicitly feasible recovery steps available.



\section{Transfer Time Adjustment}
% Figure environment removed

An additional control mechanism is to modify the swing duration to allow the robot to take the current step more quickly \cite{griffin2017walking,khadiv2020walking}. However, when a disturbance requires more than a single step to recover, relying only on speeding up the swing is no longer sufficient when the gait includes a double-support ``transfer" phase.
Many robots that rely primarily on step-adjustment for stability avoid the complexity of the added transfer phase by omitting it from the gait entirely\cite{khadiv2020walking, gong2021one}.
For navigating rough terrain, though, the inclusion of a transfer phase can be important to facilitate proper weight shifting to unload the feet.
When performing step adjustment, however, the goal is to change the base of support as quickly as possible to get the eCMP into the necessary position for the robot to become capturable. 
Any included transfer phase delays this shifting, making the robot less stable.
Thus, it is necessary to determine an appropriate strategy for ``speeding up" the transfer phase.

Mathematically, time adjustment can be described as finding the time $t^*$ that minimizes the difference between the reference capture point $\mathbf{\xi}_r(t^*)$ from Eq. \ref{eqn:com_trajectory} 
 and the current state $\mathbf{\xi}$.
This is equivalent to solving
\begin{equation}
    t^* = \argmin_\tau \left\| \mathbf{\xi}_r(\tau) - \mathbf{\xi} \right\|,
\label{eqn:time_optimization}
\end{equation}
but noting that $\mathbf{\xi}_r(\tau)$ is generally highly nonlinear.



If the eCMP is assumed to be a constant value, the dynamics of Eq. \ref{eqn:com_trajectory} simplify to those in Eq. \ref{eqn:icp_trajectory}, and methods such as the one previously used in \cite{griffin2017walking} become possible. 
In this approach, we know that the ICP evolves along a line from its current value at $\xi_r = \xi(t)$ to the final value at $\xi_T = \xi(T_s)$. 
We then know that the closest ICP along the plan, satisfying Eq. \ref{eqn:time_optimization}, will occur at the orthogonal projection of $\mathbf{\xi}$ onto the line $\mathbf{\xi}_T - \mathbf{\xi}_r$, resulting in $\mathbf{\xi}_p$. 
This then leads to the time adjustment
\begin{equation}
    \Delta t = \frac{1}{\omega} \log \left( \frac{\xi_p - \mathbf{r}_{\text{ecmp},r}}{\xi_r - \mathbf{r}_{\text{ecmp},r}}\right),
    \label{eqn:time_delta}
\end{equation}
where $t^* = t + \Delta t$.

This assumption of constant eCMP value is not representative of the transfer phase, however, where the robot is shifting the weight from one foot to the next. 
While transfer still has a closed form definition from Eq. \ref{eqn:com_trajectory} (see  \cite{englsberger2017smooth}), the optimization for time becomes much more challenging.
Instead, in this work we choose to simply apply a discount rate $\gamma$ to the adaption in Eq. \ref{eqn:time_delta}, making the update law $t^* = t + \gamma \Delta t$.
By setting $\gamma$ to be sufficiently small, this result converges to the actual $t^*$ as the adjustment is solved iteratively from one control tick to the next. 

Even when not performing step recovery, the ability to adjust the time in transfer has been found to be beneficial. 
When using just the feedback controller in Sec. \ref{sec:capture_point_control}, if the actual ICP is leading the reference ICP, the robot will heavily shift the eCMP towards to the upcoming foot to ``brake" the dynamics and converge back to the plan. It will then shift the eCMP back to the trailing foot and ``push" to resume the plan.
If, instead, the time is simply adjusted forward, the amount of feedback is decreased and this ``brake then push" phenomena is avoided. 
As we are also applying the time adjustment law in \cite{griffin2017walking} to the swing phase, this type of control becomes akin to using the actual ICP position as a monotonically increasing phase variable for determining $\mathbf{\xi}_r$.

\section{Results}
\section{Experimental Results}\label{sec:results}
    \subsection{General Results}
        The basic ResSAN model is used to determine reference results which our expanded model can be compared to as it is structurally similar to ResLAN but does not possess the Lidar adaptive components of it. Further, we compare with the full-size PackNet-SAN and the unmodified NLSPN architecture. 
        As it can be seen from Tab.\,\ref{tab:sota-results}, our LiDAR-adaptive ResLAN achieves competitive performance compared to state-of-the-art standard depth completion methods, which are specialized to the unfiltered 64-beam-LiDAR. The performance differences are in the range of a few centimetres in terms of MAE, which is acceptable given the practical advantage that ResLAN can generalize to different beam patterns as will be shown below.

        Furthermore, we compared the architectures for a set of three different input types that contained 64, 32 or 16 LiDAR channels using both filter types on the metrics from the KITTI benchmark. The NLSPN model was trained for the standard depth completion task and then evaluated with different input data. As for the ResSAN models, we trained one model for each input type and tested it for the corresponding one which serve serve as the \emph{Baseline} in Tab.\,\ref{tab:overall-results}. Our ResLAN model was jointly trained for all three settings. As listed in Tab.\,\ref{tab:overall-results}, the ResLAN models outperform the challenging baseline in all metrics for FOV filtering and all but one for sparse filtering. This implies that our LiDAR adaptive model is able to outperform dedicated models in case of very sparse input depth. Fig.\,\ref{fig:comp-plot} shows this is indeed the case for 32 and even more for 16 channels. For FOV-filtered inputs with 16 channels, the ResLAN exhibits approx. $10\%$ smaller MAE than the baseline. As for the NLSPN, it becomes apparent that it is not capable of generalizing to other input types since it shows clearly worse results. The difference is especially pronounced for the FOV filtering where on average more than every fourth predicted pixel is more than $25 \%$ deviating from the ground truth\,($\delta_{1.25}$). Therefore, using a weight-adapting network in combination with differently filtered input depths allows us to train models that outperform their non-adaptive counterparts.

        \begin{table}[]
            \centering
    	    \small
            \vspace{0.4cm}
            \caption{\textbf{Depth estimation result for standard depth completion} when the ResSAN model was only trained for 64 channels and the ResLAN model for multiple tasks. The PackNet-SAN and NLSPN models were trained with the setup that was also used for our model architecture.}
            \footnotesize
            \setlength{\tabcolsep}{5pt}
            \begin{tabular}{@{}lrrrrl@{}}
            \toprule
            \multicolumn{6}{c}{\textbf{Standard LiDAR Depth Completion}}                                                                                                                         \\ \midrule
            \multicolumn{1}{l|}{Method}          & RMSE $\downarrow$            & MAE  $\downarrow$            & iRMSE $\downarrow$             & iMAE $\downarrow$ & $\delta_{1.25}$ $\uparrow$ \\
            \multicolumn{1}{l|}{}                & \multicolumn{1}{l}{{[}mm{]}} & \multicolumn{1}{l}{{[}mm{]}} & \multicolumn{1}{l}{{[}1/km{]}} & {[}1/km{]}        &                            \\ \midrule
            \multicolumn{1}{l|}{PackNet-SAN}     &  914                            &  298                            &  2.78                              &  1.4                 &  99.65 \%                          \\
            \multicolumn{1}{l|}{NLSPN}           &  \textbf{889}                            &   \textbf{263}                           &  \textbf{2.62}                              &   \textbf{1.3}                &   \textbf{99.61} \%                         \\ \midrule
            \multicolumn{1}{l|}{ResSAN (Ours)}   & 948                             &  275                            &  2.75                              &    1.4               &   99.58 \%                         \\
            \multicolumn{1}{l|}{ResLAN (Ours)} &   969                           &  283                            &   2.83                             &   1.4                &  99.56 \%                          \\ \bottomrule
            \end{tabular}
            \vspace{0.2cm}
            \label{tab:sota-results}
        \end{table}

        \begin{table}[]
    	    \centering
    	    \small
    	    \caption{\textbf{Depth estimation results of the two baseline setups and the explicit and implicit ResSAN} when evaluated on a combination of 16, 32 and 64 channel depth inputs. Please note that Specialist Methods need to train three specialized networks, one for each of the three types of inputs while our method only uses one network.}
            \footnotesize
            \setlength{\tabcolsep}{4.8pt}
            \begin{tabular}{@{}lrrrrl@{}}
                \toprule
                \multicolumn{6}{c}{\textbf{Sparse Channel Filter}}                                                                                                                                  \\ \midrule
                \multicolumn{1}{l|}{Method}        & RMSE $\downarrow$            & MAE  $\downarrow$            & iRMSE $\downarrow$             & iMAE $\downarrow$ & $\delta_{1.25}$ $\uparrow$  \\
                \multicolumn{1}{l|}{}              & \multicolumn{1}{l}{{[}mm{]}} & \multicolumn{1}{l}{{[}mm{]}} & \multicolumn{1}{l}{{[}1/km{]}} & {[}1/km{]}        &                             \\ \midrule
                \multicolumn{1}{l|}{NLSPN}         &  1396                            &  437                            & 5.54                               &  2.2                 &  98.82 \%                           \\
                \multicolumn{1}{l|}{Baseline}      & \textbf{1207}                             &  381                            & 4.41                               &  1.8                 &  \textbf{99.37} \%                           \\
                \multicolumn{1}{l|}{ResLAN (Ours)} &  1215                            &  \textbf{378}                            &  \textbf{4.27}                              &  \textbf{1.7}                 &  99.31 \%                           \\ \toprule
                \multicolumn{6}{c}{\textbf{Field-of-View Filter}}                                                                                                                                   \\ \midrule
                \multicolumn{1}{l|}{Method}        & RMSE $\downarrow$            & MAE  $\downarrow$            & iRMSE $\downarrow$             & iMAE $\downarrow$ & $\delta_{1.25}$ $\uparrow$ \\
                \multicolumn{1}{l|}{}              & \multicolumn{1}{l}{{[}mm{]}} & \multicolumn{1}{l}{{[}mm{]}} & \multicolumn{1}{l}{{[}1/km{]}} & {[}1/km{]}        &                             \\ \midrule
                \multicolumn{1}{l|}{NLSPN}         &  2738                            &  1702                            & 12.3                              &  4.3                 &  74.69 \%                           \\
                \multicolumn{1}{l|}{Baseline}      &  1556                            &  525                            &  6.8                              &  3.0                 & 98.14 \%                            \\
                \multicolumn{1}{l|}{ResLAN (Ours)} &  \textbf{1548}                            &  \textbf{519}                            &  \textbf{6.44}                              &  \textbf{2.8}                 & \textbf{98.52 \%}                            \\ \bottomrule
            \end{tabular}
            \label{tab:overall-results}
        \end{table}

        
        
        % Figure environment removed
        
        % Figure environment removed

    \subsection{Filter Effects}
        Comparing the effect of the two different types of depth input filters on the model performance, it becomes apparent that FOV filtering is the more challenging task. In that setting, reducing LiDAR channels is more detrimental to the performance than sparse filtering as it creates regions where no depth information is available. Effectively, the model is forced to perform depth prediction in these regions. These effects are highlighted in the depth images in Fig.\,\ref{fig:dense-maps} where the effect of a 16-channel sparse depth filter and a 16-channel FOV can be compared.

    \subsection{Generalization Capabilities}
        We trained three models for both filter types eaach, so the combinations and number of filtered depth inputs they receive are different. This serves the purpose of testing the generalization capabilities of the ResLAN architecture as well as the robustness to different filter settings. After training, the models were evaluated for the depth input settings they were trained for, as well as for ones they weren't exposed to. Overall, ResLAN shows good generalization capabilities. As one can gather from Fig.\,\ref{fig:explicit-comp} and Fig.\,\ref{fig:implicit-comp}, the consequences of slightly varying sets of input depth settings are limited. The most considerable deviations can be seen when the model is tasked to extrapolate. For instance, the model $\{64, 32, 16\}$ shows a noticeably higher MAE for eight-channel depth inputs than the model that was trained for it. Similar behaviour can be seen for the FOV filtering case as well for the model $\{64, 48, 32\}$ when tasked to generalize for a 16-channel input. There is no such pronounced effect for generalization tasks that lie between two filter settings the model was trained for. At most, it can be observed that models that were trained for a smaller range of filter values perform slightly better than ones that have to cover a wider range. The number of filter settings used in a fixed range does not relevantly influence the model performance, as can be seen, when comparing the two models in Fig.\,\ref{fig:implicit-comp}, which are both trained for a range of 64 to 32 channels but one with three filter settings and the other one with five.
    
    % Figure environment removed
    
    
    % Figure environment removed

\section{Conclusion}
\section{Conclusion and Future Work}
In this work, I design corruption-robust algorithms for the Lipschitz contextual search problem. I present the \emph{agnostic checking} technique and demonstrate its effectiveness in designing corruption-robust algorithms. There are several open problems for future research. First, in the algorithm I propose for pricing loss, the schedule for agnostic checks is fixed upfront. Can the learner design an adaptive checking schedule for the pricing loss? Second, this work assumes the learner has knowledge of the Lipschitz constant $L$. Can the learner design efficient no-regret algorithms without knowledge of $L$? 

\bibliography{mybib}

\end{document}