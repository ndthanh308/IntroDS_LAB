% Figure environment removed

An additional control mechanism is to modify the swing duration to allow the robot to take the current step more quickly \cite{griffin2017walking,khadiv2020walking}. However, when a disturbance requires more than a single step to recover, relying only on speeding up the swing is no longer sufficient when the gait includes a double-support ``transfer" phase.
Many robots that rely primarily on step-adjustment for stability avoid the complexity of the added transfer phase by omitting it from the gait entirely\cite{khadiv2020walking, gong2021one}.
For navigating rough terrain, though, the inclusion of a transfer phase can be important to facilitate proper weight shifting to unload the feet.
When performing step adjustment, however, the goal is to change the base of support as quickly as possible to get the eCMP into the necessary position for the robot to become capturable. 
Any included transfer phase delays this shifting, making the robot less stable.
Thus, it is necessary to determine an appropriate strategy for ``speeding up" the transfer phase.

Mathematically, time adjustment can be described as finding the time $t^*$ that minimizes the difference between the reference capture point $\mathbf{\xi}_r(t^*)$ from Eq. \ref{eqn:com_trajectory} 
 and the current state $\mathbf{\xi}$.
This is equivalent to solving
\begin{equation}
    t^* = \argmin_\tau \left\| \mathbf{\xi}_r(\tau) - \mathbf{\xi} \right\|,
\label{eqn:time_optimization}
\end{equation}
but noting that $\mathbf{\xi}_r(\tau)$ is generally highly nonlinear.



If the eCMP is assumed to be a constant value, the dynamics of Eq. \ref{eqn:com_trajectory} simplify to those in Eq. \ref{eqn:icp_trajectory}, and methods such as the one previously used in \cite{griffin2017walking} become possible. 
In this approach, we know that the ICP evolves along a line from its current value at $\xi_r = \xi(t)$ to the final value at $\xi_T = \xi(T_s)$. 
We then know that the closest ICP along the plan, satisfying Eq. \ref{eqn:time_optimization}, will occur at the orthogonal projection of $\mathbf{\xi}$ onto the line $\mathbf{\xi}_T - \mathbf{\xi}_r$, resulting in $\mathbf{\xi}_p$. 
This then leads to the time adjustment
\begin{equation}
    \Delta t = \frac{1}{\omega} \log \left( \frac{\xi_p - \mathbf{r}_{\text{ecmp},r}}{\xi_r - \mathbf{r}_{\text{ecmp},r}}\right),
    \label{eqn:time_delta}
\end{equation}
where $t^* = t + \Delta t$.

This assumption of constant eCMP value is not representative of the transfer phase, however, where the robot is shifting the weight from one foot to the next. 
While transfer still has a closed form definition from Eq. \ref{eqn:com_trajectory} (see  \cite{englsberger2017smooth}), the optimization for time becomes much more challenging.
Instead, in this work we choose to simply apply a discount rate $\gamma$ to the adaption in Eq. \ref{eqn:time_delta}, making the update law $t^* = t + \gamma \Delta t$.
By setting $\gamma$ to be sufficiently small, this result converges to the actual $t^*$ as the adjustment is solved iteratively from one control tick to the next. 

Even when not performing step recovery, the ability to adjust the time in transfer has been found to be beneficial. 
When using just the feedback controller in Sec. \ref{sec:capture_point_control}, if the actual ICP is leading the reference ICP, the robot will heavily shift the eCMP towards to the upcoming foot to ``brake" the dynamics and converge back to the plan. It will then shift the eCMP back to the trailing foot and ``push" to resume the plan.
If, instead, the time is simply adjusted forward, the amount of feedback is decreased and this ``brake then push" phenomena is avoided. 
As we are also applying the time adjustment law in \cite{griffin2017walking} to the swing phase, this type of control becomes akin to using the actual ICP position as a monotonically increasing phase variable for determining $\mathbf{\xi}_r$.