% Figure environment removed

Significantly degraded tracking from external disturbances, unstructured terrain, or slips may require multiple steps for the robot to recover.
Limiting the robot to deadbeat step adjustment using only a single step for recovery correspondingly places limits on the recoverable error.
On top of this, as the walking speed increases, the reference dynamics are often \textit{not} one step capturable, so requiring one step capturability limits the overall locomotion speed.
To address this, we consider multiple steps for recovery using multi-step capture regions \cite{koolen2012capturability}.
However, as there are reachability constraints limiting the step adjustment, these multi-step capture regions must be enhanced to consider these possibly complex restrictions.

The $N$-step capture region (referred to as $\mathcal{C}_N$) is the region in which the robot must step to avoid falling in $N$ or fewer steps. 
This region is constructed using a series of nested regions, with the innermost being $\mathcal{C}_1$, and growing increasingly larger until $\mathcal{C}_N$ is reached, with $\mathcal{C}_{N-1} \subset \mathcal{C}_N$.
By definition, after a step is taken in region $\mathcal{C}_N$, the system is then in a $N-1$ capturable state, such that the next step can be taken in region $\mathcal{C}_{N-1}$ to recover \cite{koolen2012capturability}.

To compute $\mathcal{C}_N$, we can first define a reachability constraint that restricts the position of step $n$, $\mathbf{r}_{\text{foot}, n}$, to some reachable area, $\mathbf{r}_{\text{foot}, n} \in \mathcal{R}_n$, which can be scaled in size by $\alpha$ and translated by $\delta$, $\alpha \mathcal{R} + \delta$.
From the ICP dynamics, we know that the ICP state at the end of step $n$ is 
\begin{equation}
    \mathbf{\xi}_n = e^{\omega T_s} \left( \mathbf{\xi}_{n-1} - \mathbf{r}_{\text{foot},n-1} \right) +  \mathbf{r}_{\text{foot},n-1},
\end{equation}
where $T_s$ is the step duration. As $\mathbf{r}_{\text{foot}, n} \in \mathcal{R}_n$, we can define the remaining error after step $n$ is taken, $\mathbf{\xi}_{e,n} = \mathbf{\xi}_n - \mathbf{r}_{\text{foot},n}$, as the shortest vector from $\mathbf{\xi}_n$ to $\mathcal{R}_n$, or
\begin{equation}
\mathbf{\xi}_{e,n} = \mathbf{d} \left(\mathbf{\xi}_n, \mathcal{R}_n \right).
\end{equation}
From Eq. \ref{eqn:icp_dynamics}, if the state can be captured on the $N^{th}$ step,  $\mathbf{\xi}_{e,N} = 0$, or
\begin{equation}
    \mathbf{d} \left( \xi_{e, N},  \mathcal{R}_{N}\right) = 0.
    \label{eqn:captured_definition1}
\end{equation}
As $\mathcal{R}_N$ is always defined centered at $\mathbf{r}_{foot,N}$, we can define everything assuming $\mathbf{r}_{foot,N} = 0$. From this, it follows that $\mathbf{\xi}_{e,N} = e^{\omega T_s}$. 
This transforms Eq. \ref{eqn:captured_definition1} to
\begin{equation}
    \mathbf{d} \left( \xi_{e, N-1}, e^{-\omega T_s} \mathcal{R}_{N}\right) = 0.
\end{equation}
This is equivalent to saying that, to be capturable in $N$ steps,  
\begin{equation}
\xi_{e,N-1} \in e^{-\omega T_s} \mathcal{R}_N.
\label{eqn:error_capture}
\end{equation}
This allows us to calculate the maximum amount of additional error that can be rejected by adding step $n$ for recovery, recursing back in time to the end of the current step with $e^{\omega T_s (n-1)}$.

From here, we can define $\mathcal{C}_N$ as the set of all footstep locations that drive the state, $\xi$, to $\mathcal{C}_{N-1}$.
Region $\mathcal{C}_N$, then, can be found as the set of all points $\mathbf{r}$ such that the scaled reachable region intersects the previous capture region, or
\begin{equation} 
\mathcal{C}_N = \left\{ \mathbf{r} \ | \ \left( e^{-\omega T_s \left( N - 1 \right)} \mathcal{R}_N + \mathbf{r} \right) \  \cap \  \mathcal{C}_{N-1} \right\},
\label{eqn:capture_region}
\end{equation}
where $N-1$ captures the recursive nature of this operation.
Then, if the current step is taken in the $C_N$ region, the robot can recover in $N$ steps, with each step being reachable, $\mathbf{r}_n \in \mathcal{R}_n, \forall n = 1, \dots, N$. 

This is a subtly different definition than that presented previously \cite{koolen2012capturability}, as it is designed for arbitrary reachability constraints $\mathcal{R}$.
However, if a simple reachable set is used, defined only by a maximum step length, $l_\text{max}$, where $\mathcal{R}_{l_\text{max}} = \left\{ \mathbf{r} \ | \ \| \mathbf{r} \| \le l_{\text{max}} \right\}$, the resulting capture regions are the same. 
In this case, $\mathcal{C}_N$ is simply a direct expansion of $\mathcal{C}_{N-1}$ by a distance of $e^{-\omega T_s (N-1)} l_{\text{max}}$, resulting in $\mathcal{C}_{1,2,3}$ shown on the top in Fig. \ref{fig:multi_step_capture_regions}. 


% Figure environment removed


It should be noted that, when computing $\mathcal{C}_N$, there are additional control inputs available to the robot besides step position, namely ankle and hip torques, as well as step timing adjustment (as in \cite{griffin2017walking}) that have been ignored in our approach for calculating $\mathcal{C}_N$.
This makes the presented method an interior estimation of the true $\mathcal{C}_N$, and leads to more step adjustment on each step than may be strictly necessary for recovery.
In practice, we prefer this conservative approach, as it provides a better factor of safety for the adjustment. 
Additionally, by using more step adjustment, the robot's gait is less likely to lead to foot tipping or severe torso angles from saturating the ankle and hip control strategies, respectively.

A major limitation of the use of the simple reachability constraint $\mathcal{R}_{l_\text{max}}$, is the assumption that the robot can step anywhere within $l_{\text{max}}$ of its stance foot.
In practice, it is often useful to define some convex reachability constraint for the robot to prevent a number of undesirable events from occurring, including stepping too wide, stepping on its own feet, or colliding with the stance leg. 
To address this, we use an ellipsoidal constraint on the swing foot relative to the stance foot, defined as $\mathcal{R}_b$ and shown in Fig. \ref{fig:reachability_constraint}, which allows constraining the minimum and maximum step width and length.
We approximate this ellipse as a polygon.

% Figure environment removed

We can then compute the capture region using this new ellipsoidal reachable region in Eq. \ref{eqn:capture_region}.
From this, the boundary of $\mathcal{C}_N$ is defined as the foot position at the intersection of the transformed $\mathcal{R}_N$ and the boundary of $\mathcal{C}_{N-1}$. 
As both $\mathcal{R}_N$ and $\mathcal{C}_{N-1}$ are polygons, $\mathcal{C}_N$ is by construction a polygon, and the outer vertices of $\mathcal{C}_N$ come from the intersection between vertices of the transformed $R_N$ and $\mathcal{C}_{N-1}$.
Algorithmically, this is equivalent to calculating $\mathcal{C}_N$ as the valid foot positions when sweeping the scaled $\mathcal{R}_N$ around the perimeter of $\mathcal{C}_{N-1}$, as shown in Fig. \ref{fig:two_step_capture_regions}.

From Fig. \ref{fig:multi_step_capture_regions}, including knowledge of this reachable set strongly affects the shape of $\mathcal{C}_N$, with the results of using $\mathcal{R}_b$ to compute the $\mathcal{C}_N$ shown in the middle.
It is apparent that, because of the minimum width constraint, additional steps do not provide much additional stability when the error is towards the inside of the foot, as recovery requires almost completely stepping in $\mathcal{C}_1$.
This can be seen by the lack of additional area in the inward direction when comparing $\mathcal{C}_2$ to $\mathcal{C}_1$ in the middle of Fig. \ref{fig:multi_step_capture_regions}.
This makes sense; the original, simple reachable set assumes the robot can take a subsequent step of width $l_\text{max}$, when in actuality, by prohibiting cross-over, the closest width for the next step is $-w_{\text{min}}$. 
The second step, then, provides no corrective control in the inward direction as the control authority (defined by $\mathcal{R}_b$) is actually \textit{negative}.
Thus, if $\mathcal{R}_{l_\text{max}}$ was used to compute $\hat{\mathcal{C}}_N$, the robot may have stepped outside the actual $\mathcal{C}_N$, leading to a fall! 