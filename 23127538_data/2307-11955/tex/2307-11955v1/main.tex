

\documentclass{article}


\usepackage{microtype}
\usepackage{graphicx}
\usepackage{subfigure}
\usepackage{booktabs} 


\usepackage{hyperref}
\usepackage{enumerate}


\newcommand{\theHalgorithm}{\arabic{algorithm}}




% If accepted, instead use the following line for the camera-ready submission:
\usepackage[accepted]{icml2023_dp4ml}

% For theorems and such
\usepackage{amsmath}
\usepackage{amssymb}
\usepackage{mathtools}
\usepackage{amsthm}

\renewcommand{\Pr}{\field{P}}
\DeclareMathOperator{\Tr}{Tr}

\newcommand{\ba}{\boldsymbol{a}}
\newcommand{\bb}{\boldsymbol{b}}
\newcommand{\bc}{\boldsymbol{c}}
\newcommand{\be}{\boldsymbol{e}}
\newcommand{\bh}{\boldsymbol{h}}
\newcommand{\bj}{\boldsymbol{j}}
\newcommand{\bl}{\boldsymbol{l}}
\newcommand{\bp}{\boldsymbol{p}}
\newcommand{\bq}{\boldsymbol{q}}
\newcommand{\bg}{\boldsymbol{g}}
\newcommand{\bs}{\boldsymbol{s}}
\newcommand{\bx}{\boldsymbol{x}}
\newcommand{\bu}{\boldsymbol{u}}
\newcommand{\by}{\boldsymbol{y}}
\newcommand{\bA}{\boldsymbol{A}}
\newcommand{\bB}{\boldsymbol{B}}
\newcommand{\bH}{\boldsymbol{H}}
\newcommand{\bI}{\boldsymbol{I}}
\newcommand{\bS}{\boldsymbol{S}}
\newcommand{\bX}{\boldsymbol{X}}
\newcommand{\bY}{\boldsymbol{Y}}
\newcommand{\bxi}{\boldsymbol{\xi}}
\newcommand{\sY}{\mathcal{Y}}
\newcommand{\bhatY}{\boldsymbol{\hat{Y}}}
\newcommand{\bbary}{\boldsymbol{\bar{y}}}
\newcommand{\bz}{\boldsymbol{z}}
\newcommand{\bZ}{\boldsymbol{Z}}
\newcommand{\bbarZ}{\boldsymbol{\bar{Z}}}
\newcommand{\bbarz}{\boldsymbol{\bar{z}}}
\newcommand{\bhatZ}{\boldsymbol{\hat{Z}}}
\newcommand{\bhatz}{\boldsymbol{\hat{z}}}
\newcommand{\barz}{\bar{z}}
\newcommand{\bbarS}{\boldsymbol{\bar{S}}}
\newcommand{\bw}{\boldsymbol{w}}
\newcommand{\bW}{\boldsymbol{W}}
\newcommand{\bU}{\boldsymbol{U}}
\newcommand{\bv}{\boldsymbol{v}}
\newcommand{\bzero}{\boldsymbol{0}}
\newcommand{\balpha}{\boldsymbol{\alpha}}
\newcommand{\bbeta}{\boldsymbol{\beta}}
\newcommand{\bmu}{\boldsymbol{\mu}}
\newcommand{\bpi}{\boldsymbol{\pi}}
\newcommand{\bSigma}{\boldsymbol{\Sigma}}
\newcommand{\btheta}{\boldsymbol{\theta}}
\newcommand{\bTheta}{\boldsymbol{\Theta}}
\newcommand{\sA}{\mathcal{A}}
\newcommand{\sC}{\mathcal{C}}
\newcommand{\sX}{\mathcal{X}}
\newcommand{\sS}{\mathcal{S}}
\newcommand{\sZ}{\mathcal{Z}}
\newcommand{\sbarZ}{\bar{\mathcal{Z}}}
\newcommand{\fbag}{\bold{F}}


\newcommand{\argmin}{\mathop{\mathrm{argmin}}}
\newcommand{\argmax}{\mathop{\mathrm{argmax}}}
\newcommand{\conv}{\mathop{\mathrm{conv}}}
\newcommand{\interior}{\mathop{\mathrm{int}}}
\newcommand{\dom}{\mathop{\mathrm{dom}}}
\newcommand{\bdry}{\mathop{\mathrm{bdry}}}
%\newcommand{\argmin}[1]{\underset{#1}{\operatorname{argmin}}}
%\newcommand{\argmax}[1]{\underset{#1}{\operatorname{argmax}}}

%\newcommand{\todo}[1]{\textcolor{red}{TODO: #1}}
\newcommand{\fixme}[1]{\textcolor{red}{FIXME: #1}}

\newcommand{\field}[1]{\mathbb{#1}}
\newcommand{\fY}{\field{Y}}
\newcommand{\fX}{\field{X}}
\newcommand{\fH}{\field{H}}

\newcommand{\R}{\field{R}}
\newcommand{\Nat}{\field{N}}
\newcommand{\E}{\field{E}}
\newcommand{\Var}{\mathrm{Var}}


\newcommand{\bbartheta}{\boldsymbol{\bar{\theta}}}
\newcommand\theset[2]{ \left\{ {#1} \,:\, {#2} \right\} }
\newcommand\inn[2]{ \left\langle {#1} \,,\, {#2} \right\rangle }
\newcommand\RE[2]{ D\left({#1} \| {#2}\right) }
\newcommand\Ind[1]{ \left\{{#1}\right\} }
\newcommand{\norm}[1]{\left\|{#1}\right\|}
\newcommand{\diag}[1]{\mbox{\rm diag}\!\left\{{#1}\right\}}

\newcommand{\defeq}{\stackrel{\rm def}{=}}
\newcommand{\sgn}{\mbox{\sc sgn}}
\newcommand{\scI}{\mathcal{I}}
\newcommand{\scO}{\mathcal{O}}

\newcommand{\dt}{\displaystyle}
\renewcommand{\ss}{\subseteq}
\newcommand{\wh}{\widehat}
\newcommand{\ve}{\varepsilon}
\newcommand{\hlambda}{\wh{\lambda}}
\newcommand{\yhat}{\wh{y}}

\newcommand{\hDelta}{\wh{\Delta}}
\newcommand{\hdelta}{\wh{\delta}}
\newcommand{\spin}{\{-1,+1\}}

%\newcommand{\theHalgorithm}{\arabic{algorithm}}

\newcommand{\reals}{\mathbb{R}}
\newcommand{\sign}{{\rm sign}}

%Color Definitions
\newcommand{\blue}{\color{blue}}
\newcommand{\red}{\color{red}}
\newcommand{\green}{\color{OliveGreen}}
\newcommand{\violet}{\color{violet}}

\DeclareMathOperator*{\Exp}{\mathbf{E}}
\DeclareMathOperator{\Regret}{Regret}
\DeclareMathOperator{\Wealth}{Wealth}
\DeclareMathOperator{\Reward}{Reward}
\DeclareMathOperator{\Risk}{Risk}
\DeclareMathOperator{\Prox}{Prox}

\newcommand{\KL}[2]{\operatorname{KL}\left({#1};{#2}\right)}  % KL divergence
\newcommand{\indicator}{\mathbf{1}}


% if you use cleveref..
\usepackage[capitalize,noabbrev]{cleveref}


\theoremstyle{plain}
\newtheorem{theorem}{Theorem}[section]
\newtheorem{proposition}[theorem]{Proposition}
\newtheorem{lemma}[theorem]{Lemma}
\newtheorem{corollary}[theorem]{Corollary}
\theoremstyle{definition}
\newtheorem{definition}[theorem]{Definition}
\newtheorem{assumption}[theorem]{Assumption}
\theoremstyle{remark}
\newtheorem{remark}[theorem]{Remark}

\defcitealias{ChenO23}{Chen \& Orabona (ICML 2023)}


\usepackage[textsize=tiny]{todonotes}


\icmltitlerunning{ Implicit Interpretation of Importance Weight Aware Updates}

\begin{document}

\twocolumn[
\icmltitle{Implicit Interpretation of Importance Weight Aware Updates}


\icmlsetsymbol{equal}{*}

\begin{icmlauthorlist}
\icmlauthor{Keyi Chen}{sch}
\icmlauthor{Francesco Orabona}{sch}

\end{icmlauthorlist}


\icmlaffiliation{sch}{Boston University,  Boston, US}

\icmlcorrespondingauthor{Keyi Chen}{keyichen@bu.edu}
\icmlcorrespondingauthor{Francesco Orabona}{francesco@orabona.com}


\icmlkeywords{Machine Learning, ICML}

\vskip 0.3in
]


\printAffiliationsAndNotice{}  

\begin{abstract}
Due to its speed and simplicity, subgradient descent is one of the most used optimization algorithms in convex machine learning algorithms. However, tuning its learning rate is probably its most severe bottleneck to achieve consistent good performance.
A common way to reduce the dependency on the learning rate is to use implicit/proximal updates. One such variant is the Importance Weight Aware (IWA) updates, which consist of infinitely many infinitesimal updates on each loss function. However, IWA updates' empirical success is not completely explained by their theory.
In this paper, we show for the first time that IWA updates have a strictly better regret upper bound than plain gradient updates in the online learning setting. Our analysis is based on the new framework by \citetalias{ChenO23} to analyze generalized implicit updates using a \textbf{dual formulation}. In particular, our results imply that IWA updates can be considered as approximate implicit/proximal updates.
\end{abstract}


% Figure environment removed

\section{Introduction}
Automatic 3D reconstruction of clothed humans using image inputs has gained increasing significance due to its potential applications in a wide array of AR/VR scenarios. High-fidelity reconstructions typically depend on sophisticated capture systems, which are developed with dense camera arrays~\cite{collet2015high,joo2015panoptic,joo2018total}, programmable light-stages~\cite{Vlasic2009, guo2019relightables}, and depth sensors~\cite{newcombe2011kinectfusion,DoubleFusion,BodyFusion,dou2016fusion4d,newcombe2015dynamicfusion}. However, stringent capture environments equipped with complex hardware pose significant challenges for consumer-level applications.


In this context, considerable research effort has been dedicated to developing methods that allow for more flexible capture configurations, such as utilizing a few RGB inputs. Among these works, learning implicit functions \cite{iccv2020PIFu, saito2020pifuhd, hong2021stereopifu} has proven effective in achieving highly detailed reconstructions by integrating the advancements of deep neural networks. These methods employ large multi-layer perceptrons (MLPs) to predict the occupancy probability or truncated signed distance function (TSDF) value of every queried 3D point based on its associated local feature, which is extracted from images. They can recover a continuous surface at arbitrary resolutions without topology restrictions.


However, in typical MLP-based implicit networks, the occupancy or TSDF value at each location is solved independently with planar image features, rendering them less capable of addressing challenging cases such as occlusions. Consequently, these methods suffer from generalization and robustness issues, particularly when tackling strong occlusions caused by large motion or multiple interacting humans. 
Some follow-up studies  \cite{zheng2021deepmulticap,zheng2021pamir,huang2020arch} utilize an extra geometric model, SMPL~\cite{Loper2015}, to improve robustness by introducing strong shape priors. 
Their success typically relies on the assumption of geometrical similarity \cite{huang2020arch} between the shape prior and target reconstruction, making them intractable for handling complex cases with loose clothes and sensitive to errors in SMPL model fitting.



%\ping{this paragraph sounds like `TSDF is better than MLP/SMPL, and we use TSDF to solve the problem'. But in Sec 3, we are telling a different story, saying `MLP needs a 3D convolutional encoder'. We need to make these two sections consistent.}\sicong{I think in this paragraph we claim that the TSDF}


%We opt for Trucated Signed Distance Funtion (TSDF) volumetric representations as they are naturally suitable for convolution operations, which have shown remarkable performance for learning hierarchical features on 2D visual perception tasks \cite{SunXLW19}. 
%Meanwhile, TSDF also describes the gradual geometry change around shape surface, which is not reflected by occupancy volume. 

We instead revisit the 3D volumetric representation and resort to 3D convolutional neural networks (CNNs) for feature learning, due to their impressive performance in feature learning and the ability to incorporate spatial context. However, volumetric methods and 3D convolution involve discretization, which might raise concerns regarding whether a discretized volume can preserve subtle geometric details as continuous representations learned in implicit functions. We investigate the relationship between volume resolution and quantization error on synthetic data by converting target mesh objects to TSDF volumes, as shown in Figure~\ref{fig:quantization_error}. We observe that the quantization errors are significantly reduced by increasing volume resolution and become nearly negligible when reaching a relatively high resolution (e.g., 512 or higher). In other words, achieving fine-detailed reconstruction is not supposed to be restricted by the use of volume representations as long as a proper volume resolution is utilized. Therefore, we present a method with high-resolution feature volumes, e.g., 256 and 512, while traditional volumetric methods \cite{varol18_bodynet,gilbert2018volumetric} are often limited to much lower resolutions, such as 32 or 128.



On the other hand, an increase in volume resolution may lead to a cubic growth of memory overhead \cite{8100085}. Reducing memory costs while guaranteeing the granularity of volumetric representations is necessary for pursuing high-quality reconstruction. Thus, we adopt a coarse-to-fine approach and cull away irrelevant voxels to build a sparse high-resolution feature volume. At the coarse level, the network computes an initial TSDF by applying a U-Net with sparse 3D CNN \cite{3DSemanticSegmentationWithSubmanifoldSparseConvNet} on the sparse feature volume, which is carved by a visual hull. Through our experiments, it turns out that more than 95\% of the volume grids are discarded by the visual hull culling, making the sparse 3D CNN efficient. At the fine level, the network focuses on a narrow band near the zero-level set of the initial TSDF and discretizes the narrow band with smaller voxels. By employing this narrow-band culling, we further shrink the sampling space, resulting in a relatively small range of grid numbers (usually 300K--500K in our experiments) even with a high volume resolution of 512. The remaining voxels in the narrow band are associated with features that fuse high-frequency information from the computed normal maps upon the low-frequency shape from the coarse level to compute the TSDF at high resolution. The final mesh is then extracted from the TSDF using the Marching-Cube algorithm ~\cite{Lorensen87marchingcubes}.
% Different from the u-net sturcture to preserve global topology context, we then apply a shallow 3dcnn to compute the final TSDF $D_{final}$ which contain more local geometry detail.




% \ping{this paragraph can be expanded. It is an important contribution and often ignored by other works. stress on the novel idea of regressing blending weights instead of colors}

In addition to geometry, high-quality mesh texture is also a crucial factor contributing to visual appearance. Directly computing a color field in 3D space, as in \cite{iccv2020PIFu}, struggles to capture high-frequency texture details, while the neural radiance field (NeRF) \cite{yu2020pixelnerf} or the DoubleField~\cite{shao2022doublefield} require expensive per-instance optimization and are often unstable for sparse input images. In contrast, we adopt an image-based rendering approach to compute a texture atlas map, which is efficient and widely supported in existing computer graphics tools. 
Specifically, we compute a blending weight at each 3D point on the mesh surface to determine its color as a weighted average of the colors at its image projections. The blending weights can be computed at a relatively coarse resolution, e.g., 512 volume resolution in our case, and leave texture details to the high-resolution images, such as 1K or 2K. Unlike previous methods that generate blurry texturing results under sparse input, our method generalizes well on both synthetic and real data with just a few input views. 
Figure~\ref{fig:teaser} shows two examples reconstructed by our method. Despite the challenging garment, pose, and occlusion, our method recovers faithful shape, normal, and texture on the right.

%with a wide variety of poses and clothing styles, and it is also adaptive to handle input image with arbitrary resolutions.
%\sicong{For this concern we claim that when the resolution of dicretized volume meets certain threshold (which is 256 in our experiment), the quantization error can be neglected.} 



In summary, the main contributions of this paper are as follows:
\begin{itemize}
\vspace{-0.1in}
  \item 
  We revisit the 3D volumetric representation and demonstrate that it can support clothed human reconstruction with equal or even better performance compared to implicit representation. 
  \item 
  We develop a memory and computation-efficient method for high-resolution volumetric reconstruction using sophisticated sparse 3D CNN, coarse-to-fine estimation, and voxel culling by visual hull and narrow bands. 
  \item 
  We introduce a novel method to compute a texture atlas map, which captures rich appearance details from high-resolution input images.
  \item 
  We achieve impressive results on standard benchmark datasets Twindom and MultiHuman, significantly reducing the point-2-surface (P2S) precision to approximately 0.2cm from just six input views, with more than $50\%$ error reduction compared to the state-of-the-art methods, including DoubleField~\cite{shao2022doublefield} and PIFuHD~\cite{saito2020pifuhd}.
\end{itemize}
\paragraph{Related Work}
While IWA updates~\citep{KarampatziakisL11} were motivated by the use of importance weights, they end up being similar to the implicit~\citep{KivinenW97,KulisB10,CampolongoO20} and proximal~\citep{Rockafellar76} updates. In particular, for some losses like the hinge loss, the IWA update and the implicit/proximal update coincide. In this view, they are also similar to the aProx updates~\citep{AsiD19}, which take an implicit/proximal step on truncated linear models. However, as far as we know, no explicit relationship between implicit/proximal updates and IWA updates was known till now. Moreover, the best guarantee for IWA updates just shows that in some restricted cases the regret upper bound is not too much worse than the one of OGD~\citep{KarampatziakisL11}. Finally, all the previous analysis of implicit updates in online learning are conducted in the primal space, while our analysis is done completely in the dual space.


% While there are many works on implicit mirror descent in both the online and offline setting~\citep[see, e.g.,][]{Moreau65,Martinet70,Rockafellar76,KivinenW97,ParikhB14,CampolongoO20,Shtoff22}, the number of works that deal with implicit updates for FTRL is quite limited. We are only aware of \citet{McMahan10}, which quantifies a gain only for specific regularizers. However, the framework in \citet{McMahan10} is non-constructive in the sense that it is difficult to see how to generalize implicit updates. \citet{JoulaniGS17} extends this last result, but it does not provide a link with the maximization of the dual function that governs the regret upper bound.
% 
% The closest approach to our framework is the one of \citet{Shalev-ShwartzS07,ShalevS07},  which develop a theory of FTRL updates as maximization of a dual function. However, their framework is limited to a specific shape of regularizers and it does not deal with implicit updates.

%For implicit OMD, \citet{CampolongoO20} showed that implicit
%updates give rise to regret guarantees that depend on the temporal variability of the losses, so that constant regret is achievable if the variability of the losses is zero. They suggest that FTRL with full losses can achieve the same guarantee, but they also point out that given its computational complexity it would be ``not worth pursuing.''
%Here, we show how to achieve the same bound of implicit OMD with our generalized implicit FTRL, while retaining the same computational complexity of implicit OMD.

%Proximal updates on surrogate functions were introduced in \citet{AsiD19} for the OMD algorithm.
%\citet{ChenCO22} have tried to incorporate the same idea in an FTRL-based parameter-free algorithm~\cite{OrabonaP16}, however, their approach is ad-hoc is it seems difficult to generalize it.



\newcommand{\rednote}[1] {{\color{red}$\blacktriangleright${#1}$\blacktriangleleft$}}

%---------------------%
%       References    %
%---------------------%
\newcommand{\fig}[1]{Fig.~\ref{#1}}
\newcommand{\sect}[1]{Sec.~\ref{#1}}
\newcommand{\apd}[1]{Appendix~\ref{#1}}
\newcommand{\eq}[1]{(\ref{#1})}

%

\newtheorem{theorem}{Theorem}
\newtheorem{definition}{Definition}
\newtheorem{corollary}{Corollary}
\newtheorem{lemma}{Lemma}
\newtheorem{proof}{Proof}

% Symbols definitions

\newcommand{\Real}[1]{\Re \left \{ #1\right \}}
\newcommand{\en} {E}
\newcommand{\td}[1] {\tilde{#1}}
\newcommand{\lt}[1] {{\td{\lambda}}_{#1}}
\newcommand{\bl}[1] {\text{\boldmath ${\lambda}$}_{#1}}
\newcommand{\blt}[1] {\text{\boldmath $\td{\lambda}$}_{#1}}
\newcommand{\vg}[1] {{\mbox{{\boldmath ${#1}$}}}}
\newcommand{\vgs}[2] {\vg{#1}_{#2}}

\newcommand{\fourier}[1]{\mathcal{F} \left [ #1\right ]}
\newcommand{\invfourier}[1]{\mathcal{F}^{-1} \left [ #1\right ]}

\newcommand{\PX}[1] {{\mathbb{P}}\left\{{#1}\right\}}
\newcommand{\EX}[1] {{\mathbb{E}}\left\{{#1}\right\}}
\newcommand{\EXs}[2] {{\mathbb{E}}_{{#1}}\!\!\left\{{#2}\right\}}
\newcommand{\Var}[1] {{\text{Var}}\left ({#1}\right )}

\newcommand{\pX}[1] {{\mathbf{p}}\left\{{#1}\right\}}


\newcommand{\Hone} {\mathcal{H}_1}
\newcommand{\Hzero}{\mathcal{H}_0}
\newcommand{\hatHone} {\hat{\mathcal{H}}_1}
\newcommand{\hatHzero}{\hat{\mathcal{H}}_0}

\newcommand{\Mu} {\mathcal{M}_u}
\newcommand{\Mc}{\mathcal{M}_c}

\newcommand{\boldA} {{\bf{A}}}
\newcommand{\boldg} {{\bf{g}}}
\newcommand{\bolds} {{\bf{s}}}
\newcommand{\boldf} {{\bf{f}}}
\newcommand{\bolda} {{\bf{a}}}
\newcommand{\boldb} {{\bf{b}}}
\newcommand{\boldp} {{\bf{p}}}
\newcommand{\bolde} {{\bf{e}}}
\newcommand{\boldk} {{\bf{k}}}
\newcommand{\boldK} {{\bf{K}}}
\newcommand{\boldu} {{\bf{u}}}
\newcommand{\boldc} {{\bf{c}}}
\newcommand{\boldV} {{\bf{V}}}
\newcommand{\boldX} {{\bf{X}}}
\newcommand{\boldY} {{\bf{Y}}}
\newcommand{\boldW} {{\bf{W}}}
\newcommand{\boldU} {{\bf{U}}}
\newcommand{\boldE} {{\bf{E}}}
\newcommand{\boldJ} {{\bf{J}}}
\newcommand{\boldH} {{\bf{H}}}
\newcommand{\boldm} {{\bf{m}}}
\newcommand{\boldP} {{\bf{P}}}
\newcommand{\boldF} {{\bf{F}}}
\newcommand{\boldG} {{\bf{G}}}
\newcommand{\boldR} {{\bf{R}}}
\newcommand{\boldC} {{\bf{C}}}
\newcommand{\boldB} {{\bf{B}}}
\newcommand{\boldD} {{\bf{D}}}
\newcommand{\boldLambda} {{\bf{\Lambda}}}
\newcommand{\boldq} {{\bf{q}}}

%\newcommand{\boldT} {{\bf{T}}}
%\newcommand{\boldF} {\bf{F}}
\newcommand{\boldI} {{\bf{I}}}
\newcommand{\boldr} {{\bf{r}}}
\newcommand{\meanr} {{\overline{r}}}
\newcommand{\meanboldr} {{\overline{\bf{r}}}}
\newcommand{\boldn} {{\bf{n}}}
\newcommand{\boldx} {{\bf{x}}}
\newcommand{\boldy} {{\bf{y}}}
\newcommand{\boldh} {{\bf{h}}}
\newcommand{\boldz} {{\bf{z}}}
\newcommand{\boldw} {{\bf{w}}}
\newcommand{\boldv} {{\bf{v}}}

\newcommand{\boldt} {{\bf{t}}}
\newcommand{\meanw} {{\overline{w}}}
\newcommand{\meanboldw} {{\overline{\bf{w}}}}
\newcommand{\boldd} {{\bf{d}}}
\newcommand{\boldalpha} {\bf{\alpha}}
\newcommand{\boldbeta} {\bf{\beta}}
\newcommand{\boldgamma} {\bf{\gamma}}
\newcommand{\boldrho} {\bf{\rho}}
\newcommand{\boldhc} {\bf{h}_{\text{c}}}

\newcommand{\Pd} {P_{\text{d}}}
\newcommand{\Pf} {P_{\text{f}}}

\newcommand{\Pb} {P_{\text{b}}}

\newcommand{\Rb} {R_{\text{b}}}
\newcommand{\Ep} {E_{\text{p}}}


\newcommand{\Tp} {T_{\text{p}}}
\newcommand{\Td} {T_{\text{d}}}
\newcommand{\fc} {f_{\text{c}}}
\newcommand{\ts} {t_{\text{s}}}
\newcommand{\Ta} {T_{\text{a}}}
\newcommand{\Ti} {T_{\text{i}}}
\newcommand{\Np} {N_{\text{p}}}
\newcommand{\Nps} {N_{\text{ps}}}
\newcommand{\tp} {\tau_{\text{p}}}
\newcommand{\Es} {E_{\text{s}}}
\newcommand{\Eb} {E_{\text{b}}}
\newcommand{\Ts} {T_{\text{s}}}
\newcommand{\Tf} {T_{\text{f}}}
\newcommand{\Tc} {T_{\text{c}}}
\newcommand{\Th} {T_{\text{h}}}
\newcommand{\Tb} {T_{\text{b}}}
\newcommand{\Tob} {T_{\text{ob}}}
\newcommand{\Nc} {N_{\text{c}}}
\newcommand{\Ns} {N_{\text{s}}}
\newcommand{\Na} {N_{\text{A}}}

\newcommand{\Nsym} {N_{\text{sym}}}
\newcommand{\tint} {T_{\text{int}}}
\newcommand{\TX}[1] {{\mathbb{T}}\left [{#1}\right ]}
\newcommand{\Prob}[1] {\text{P}\left\{{#1}\right\}}
\newcommand{\Q}[1] {Q \left ( #1 \right )}
\newcommand{\Nch} {N_{\text{ch}}}
\newcommand{\Lp} {L_{\text{p}}}
\newcommand{\dref} {d_{\text{ref}}}
\newcommand{\wref} {w_{\text{ref}}}
\newcommand{\Wref} {W_{\text{ref}}}
\newcommand{\Href} {H_{\text{ref}}}
\newcommand{\ZA} {Z_{\text{A}}}
\newcommand{\taup} {\tau_{\text{f}}}
%\newcommand{\etaup} {\hat{\tau}_{\text{f}}}
\newcommand{\taud} {{\tau_{\text{d}}}}
\newcommand{\etaud} {\hat{\tau}_{\text{d}}}
\newcommand{\toa} {\tau}
\newcommand{\etoa} {\hat{\tau}}
\newcommand{\Beff} {B_{\text{eff}}}

\newcommand{\Pric} {P_{\text{r}}}
\newcommand{\Thetai} {{\bf \Theta}^{\text{inc}}}
\newcommand{\Thetar} {{\bf \Theta}^{\text{ref}}}
\newcommand{\Thetat} {{\bf \Theta}^{\text{t}}}
\newcommand{\Prc}{P_{r}^{metal\,can}}
\newcommand{\Prw}{P_{r}^{bottle\,water}}


\newcommand{\floor}[1] {f \left ({#1} \right )}
%\newcommand{\rect}[1] {\text{rect} \left ({#1} \right )}
\newcommand{\sinc}[1] {\text{sinc} \left ({#1} \right )}

\def\dsp{\displaystyle}

\def\rect{{\text{rect}}}
\def\erfc{{\text{erfc}}}
\def\erf{{\text{erf}}}
\def\inve{{\text{inverfc}}}
\def\teq{\triangleq}
\def\bs{$\blacksquare$}
%\def\sinc{{\text{sinc}}}
\newcommand{\tr}[1]{{\rm tr} \left ( #1 \right ) }
\newcommand{\rank}[1]{{\rm rank} \left ( #1 \right )}
\newcommand{\diag}[1]{{\rm diag} \left ( #1 \right )}

\newcommand{\Cfunc}[1]{{C^{(\text{#1})}}}
\newcommand{\cvect}[1]{{\mathbf{c}^{(\text{#1})}}}
\newcommand{\ucvect}[1]{{\underline{{\mathbf{c}}}^{(\text{#1})}}}
\newcommand{\epsilonvect}[1]{{\mathbf{\varepsilon}^{(\text{#1})}}}
\newcommand{\uchat}[1]{{\underline{\hat{\mathbf{c}}}^{(\text{#1})}}}
\newcommand{\chat}[1]{{\hat{\mathbf{c}}^{(\text{#1})}}}

\newcommand{\dEve} {d_{\text{Eve}}}
\newcommand{\NEve} {N_{\text{Eve}}}
\newcommand{\MEve} {M_{\text{Eve}}}

\newcommand{\Dset} {\mathcal{D}}
\newcommand{\GP}[1] {\mathcal{GP}\left ( #1 \right )}

\newcommand{\argmax}[1]{\underset{{#1}}{\operatorname{argmax}}}
\newcommand{\argmin}[1]{\underset{{#1}}{\operatorname{argmin}}}


% operazioni
\newcommand{\convZ}{*}
\newcommand{\conjZ}{^{\dag}}
\newcommand{\argmaxZ}[1]{\operatorname*{argmax}_{#1}}
\newcommand{\argminZ}[1]{\operatorname*{argmin}_{#1}}
\newcommand{\sincZ}{\text{sinc}}

% operatori, trasformate, funzioni lineari
\newcommand{\EXZ}[1] {{\mathbb{E}}\left\{{#1}\right\}}
\newcommand{\EXZBig} {\mathbb{E}}
%\newcommand{\V}[1] {\left|#1\right|}


\newcommand{\0}{\mathbf{0}}

% insiemi

\newcommand{\RZ}{\mathbb{R}^2}
%\newcommand{\C}{\mathbb{C}}
\newcommand{\A}{\mathcal{A}}
\newcommand{\Nset}{\mathcal{N}}
\newcommand{\Vset}{\mathcal{V}}

\newcommand{\rc}{r_{\text{c}}}
\newcommand{\Pe}{P_{\text{e}}}
%\newcommand{\SNR}{\mathsf{SNR}}
\newcommand{\SNR}{\text{SNR}}

\newcommand{\bx} {{\bf{x}}}
\newcommand{\bX} {{\bf{X}}}
\newcommand{\bW} {{\bf{W}}}
\newcommand{\bw} {{\bf{w}}}
\newcommand{\bY} {{\bf{Y}}}
\newcommand{\boldeta} {{\boldsymbol{\eta}}}



\newcommand{\Nr} {N_{\text{R}}}
\newcommand{\Nt} {N_{\text{T}}}
\newcommand{\Nx} {N_{\text{X}}}
\newcommand{\Ny} {N_{\text{Y}}}
\newcommand{\Nmin} {N_{\text{min}}}
\newcommand{\Lr} {L_{\text{R}}}
\newcommand{\Lt} {L_{\text{T}}}


\newcommand{\alp}{\alpha_{nm}}
\newcommand{\CRLB}{\text{CRLB}}
\newcommand{\K}{\text{K}}
\newcommand{\Banda}{B_\text{eff}}
\newcommand{\Bnm}{\beta_\text{nm}}
\newcommand{\Ptx} {P_{\text{T}}}
\newcommand{\Prx} {P_{\text{R}}}
\newcommand{\Psigma}{P_{\sigma_x}}
\newcommand{\fstart}{f_{\text{start}}}
\newcommand{\fstop}{f_{\text{stop}}}
\newcommand{\Gt} {G_{\text{T}}}
\newcommand{\Gr} {G_{\text{R}}}
\newcommand{\Gc} {G_{\text{C}}}
\newcommand{\Glis} {G_{\text{LIS}}}
\newcommand{\Gap} {G_{\text{AP}}}
\newcommand{\Gscm} {G_{\text{SCM}}}
\newcommand{\Fap} {F_{\text{AP}}}
\newcommand{\fb} {f_{\text{b}}}
\newcommand{\Fscm} {F_{\text{SCM}}}
\newcommand{\Pn} {P_{\text{N}}}
\newcommand{\pBS}{\textbf{p}_\text{BS}}
\newcommand{\phix} {\phi_{\text{inc}}}
\newcommand{\phiy} {\phi_{\text{Y}}}
\newcommand{\phiz} {\phi_{0}}
\newcommand{\phit} {\phi_{\text{T}}}
\newcommand{\phir}{ \phi_{\text{R}}}
\newcommand{\pu}{\mathbf{p}_\text{u}}
\newcommand{\thetax} {\theta_{\text{inc}}}
\newcommand{\thetak}{\theta^{(k)}}
\newcommand{\Thetax} {\Theta_{\text{inc}}}
\newcommand{\thetay} {\theta_{\text{Y}}}
\newcommand{\thetaz} {\theta_{0}}
\newcommand{\thetat} {\theta_{\text{T}}}
\newcommand{\thetar}{ \theta_{\text{R}}}
\newcommand{\Gam}{\Gamma_{nm}}
\newcommand{\znm}{Z_{\text{nm}}(f)}
\newcommand{\deltaT}{\Delta_\text{T}}
\newcommand{\W}{\text{W}}
\newcommand{\Thetak}{\Theta^{(k)}}
\newcommand{\nk}{n^{(k)}}
\newcommand{\sk}{s^{(k)}}
\newcommand{\vk}{v^{(k)}}
\newcommand{\wk}{w^{(k)}}
\newcommand{\xk}{x^{(k)}}
\newcommand{\yk}{y^{(k)}}
\newcommand{\h}{h_{nm}^{(k)}}
\newcommand{\g}{g_{nm}^{(k)}}
\newcommand{\reflec}{r_{nm}^{(k)}}
\subsection{Generalized Implicit FTRL}
\label{sec:mainres}

In this section, we summarize the generalized formulation of the implicit FTRL algorithm from \citet{ChenO23}.
The main idea is to depart from the implicit or linearized updates,  and directly design updates that improve the upper bound on the regret. 
More in detail, the basic analysis of most of the online learning algorithms is based on the definition of subgradients:
\begin{equation}
\label{eq:subgradient_ineq}
\ell_t(\bx_t) - \ell_t(\bu)
\leq \langle \bg_t, \bx_t-\bu\rangle, \ \forall \bg_t \in \partial \ell_t(\bx_t)~.
\end{equation}
This allows to study the regret on the linearized losses as a proxy for the regret on the losses $\ell_t$.
Instead, \citet{ChenO23} introduce a new fundamental and more general strategy: using the Fenchel-Young inequality, we have
\[
\ell_t(\bx_t) - \ell_t(\bu)  
\leq  \ell_t(\bx_t) - \langle \bz_t,\bu\rangle + \ell_t^\star(\bz_t), \ \forall \bz_t~.
\]
In particular, the algorithm will choose $\bz_t$ from the dual space to make a certain upper bound involving this quantity to be tighter.
This is a better inequality than \eqref{eq:subgradient_ineq} because when we select $\bz_t=\bg_t \in \partial \ell_t(\bx_t)$, using Theorem~\ref{thm:props_fenchel}, we recover \eqref{eq:subgradient_ineq}. So, this inequality subsumes the standard one for subgradients, but, using $\bz_t \in \ell_t(\bx_{t+1})$, it also subsumes the similar inequality used in the implicit case.
%, as we show in Section~\ref{sec:prox}. Moreover, we will see in Section~\ref{sec:aprox} that it covers cases where $\bz_t$ is \emph{not} a subgradient of $\ell_t$. 

The analysis in \citet{ChenO23} shows that the optimal setting of $\bz_t$ is the one that minimizes the sum of two conjugate functions:
\begin{equation}
\label{eq:h}
H_t(\bz)
\triangleq\psi^\star_{t+1,V}(\btheta_{t}-\bz) + \ell^\star_t(\bz),
\end{equation}
%or
%\begin{equation}
%\label{eq:hprime}
%H'_t(\bz)
%\triangleq\psi^\star_{t,V}(\btheta_{t}-\bz_t) + \ell^\star_t(\bz),
%\end{equation}
where $\psi_{t,V}$ is the restriction of the regularizer used at time $t$ on the feasible set $V$, i.e., $\psi_{t,V}\triangleq\psi_t+i_V$.
However, we can show that any setting of $\bz_t$ that satisfies $H(\bz_t)< H(\bg_t)$ 
%(or $H'(\bz_t)< H'(\bg_t)$)
guarantees a strict improvement in the worst-case regret w.r.t. using the linearized losses.
The presence of conjugate functions should not be surprising because we are looking for a surrogate gradient $\bz_t$ that lives in the dual space.

%One might wonder why the need for two different updates using $H_t$ or $H'_t$. The reason is that when using time-varying regularizers that depend on the data, like in the FTRL version of AdaGrad~\citep{McMahanS10,DuchiHS11}, if $\lambda_{t+1}$ depends on $\bz_t$ it might make the calculation of the update particularly difficult. This can be avoided using the update involving $H'_t$.

\begin{algorithm}[t]
\caption{Generalized Implicit FTRL}
\label{alg:giftrl}
\begin{algorithmic}[1]
{
    \REQUIRE{Non-empty closed set $V\subseteq \R^d$, a sequence of regularizers $\psi_1, \dots, \psi_T : \R^d \rightarrow (-\infty, +\infty]$}
    \STATE{$\btheta_1=\boldsymbol{0}$}
    \FOR{$t=1$ {\bfseries to} $T$}
    \STATE{Output $\bx_t \in \argmin_{\bx \in V} \ \psi_t(\bx) - \langle \btheta_t, \bx\rangle$}
    \STATE{Receive $\ell_t:V \rightarrow \R$ and pay $\ell_t(\bx_t)$}
    \STATE{Set $\bg_t \in \partial \ell_t(\bx_t)$}
    \STATE{Set $\bz_t$ such that $H_t(\bz_t)\leq H_t(\bg_t)$
    %or $H'_t(\bz_t)\leq H'_t(\bg_t)$
    where $H_t$
    %and $H'_t$
    is defined in \eqref{eq:h}}
    %and \eqref{eq:hprime}}
    %\[
    %\psi^\star_{t,V}(\btheta_{t}-\bz_t) + \ell^\star_t(\bz_t)
    %\leq \psi^\star_{t,V}(\btheta_{t}-\bg_t) + \ell^\star_t(\bg_t)
    %\]
    %or such that
    %\[
    %\psi^\star_{t+1,V}(\btheta_{t}-\bz_t) + \ell^\star_t(\bz_t)
    %\leq \psi^\star_{t+1,V}(\btheta_{t}-\bg_t) + \ell^\star_t(\bg_t)
    %\]}
    \STATE{Set $\btheta_{t+1}=\btheta_t-\bz_t$}
    \ENDFOR
}
\end{algorithmic}
\end{algorithm}

Once we have the $\bz_t$, we treat them as the gradient of surrogate linear losses.
So, putting it all together, Algorithm~\ref{alg:giftrl} shows the final algorithm.
\citet{ChenO23} prove the following general theorem for it.

%We will use one of the degrees of freedom in the FTRL equality lemma~\cite{Orabona19}. In particular, there will be two different loss functions in the analysis and the lemma will be instantiated with one of them.
%
\begin{theorem}
\label{thm:main}
Let $V\subseteq \R^d$ be closed and non-empty and $\psi_t:V \rightarrow \R$.
With the notation in Algorithm~\ref{alg:giftrl}, define by $F_t(\bx) = \psi_{t}(\bx) + \sum_{i=1}^{t-1} \langle \bz_i, \bx\rangle$, so that $\bx_t \in \argmin_{\bx \in V} \ F_{t}(\bx)$. 
Finally, assume that $\argmin_{\bx \in V} \ F_{t}(\bx)$ and $\partial \ell_t(\bx_t)$ are not empty for all $t$.
%\begin{itemize}
%\vspace{-0.3cm}
%\item
For any $\bz_t \in\R^d$ and any $\bu \in \R^d$, we have
%\begin{align*}
%&\psi^\star_{t+1,V}(\btheta_{t}-\bz_t) + \ell^\star_t(\bz_t) \\
%&\quad \leq \psi^\star_{t+1,V}(\btheta_{t}-\bg_t) + \ell^\star_t(\bg_t) \\
%&\quad = \psi^\star_{t+1,V}(\btheta_{t}-\bg_t) + \langle \bx_t, \bg_t\rangle - \ell_t(\bx_t),
%\end{align*}
%for any $\bg_t \in \partial \ell_t(\bx_t)$.
\begin{align*}
&\Regret_T(\bu) \leq \psi_{T+1}(\bu) - \min_{\bx \in V} \ \psi_{1}(\bx)\\
&\quad +\sum_{t=1}^T [\psi^\star_{t+1,V}(\btheta_{t}-\bg_t) - \psi^\star_{t,V}(\btheta_t) + \langle \bx_t, \bg_t\rangle-\delta_t] \\
&\quad + F_{T+1}(\bx_{T+1}) - F_{T+1}(\bu),
\end{align*}
where $\delta_t \triangleq H_t(\bg_t)-H_t(\bz_t)$.
%\item If $\psi_{t+1}(\bx) \geq \psi_t(\bx)$ for any $\bx \in V$, then, for any $\bz_t \in \R^d$, we have
% \begin{align*}
% &\Regret_T(\bu)\leq\psi_{T+1}(\bu) - \min_{\bx \in V} \ \psi_{1}(\bx)\\
% &\quad   +\sum_{t=1}^T [\psi^\star_{t,V}(\btheta_{t}-\bg_t) - \psi^\star_{t,V}(\btheta_t) + \langle \bx_t, \bg_t\rangle-\delta_t] \\
% &\quad + F_{T+1}(\bx_{T+1}) - F_{T+1}(\bu),
% \end{align*}
% where $\delta'_t \triangleq H'_t(\bg_t)-H'_t(\bz_t)$.
%\end{itemize}
%\vspace{-0.3cm}
\end{theorem}
%
% \begin{proof}
% %Define $\psi_{t,V}\triangleq\psi_t+i_V$, the restriction of $\psi_t$ to $V$.
% The proof is composed of simple but not obvious steps.
% The first important observation is that the definition of $\bx_t$ in the algorithm corresponds exactly to the one of FTRL on the linear losses $\langle \bz_t, \cdot\rangle$. Hence, we can use the FTRL equality in~\citet[Lemma 7.1]{Orabona19}:
% \begin{align*}
% -\sum_{t=1}^T \langle \bz_t,\bu\rangle
% &= \psi_{T+1}(\bu) - \min_{\bx \in V} \ \psi_{1}(\bx) + \sum_{t=1}^T [F_t(\bx_t) \\
% &\quad- F_{t+1}(\bx_{t+1})]  + F_{T+1}(\bx_{T+1}) - F_{T+1}(\bu),
% \end{align*}
% where we have simplified the terms $\langle \bz_t, \bx_t\rangle$ on both sides.
% 
% Now, use Fenchel-Young inequality, to have $\langle \bz_t,\bu\rangle \leq \ell_t(\bu) + \ell_t^\star(\bz_t)$.
% Hence, we have
% \begin{align*}
% -\sum_{t=1}^T \ell_t(\bu)
% &\leq  \sum_{t=1}^T [F_t(\bx_t) - F_{t+1}(\bx_{t+1}) + \ell_t^\star(\bz_t)] \\
% &\quad +\psi_{T+1}(\bu) - \min_{\bx \in V} \ \psi_{1}(\bx) \\
% &\quad + F_{T+1}(\bx_{T+1}) - F_{T+1}(\bu)~.
% \end{align*}
% Observe that 
% \begin{align*}
% F_t(\bx_t) 
% &= \min_{\bx \in V} \ \psi_{t}(\bx) + \sum_{i=1}^{t-1} \langle \bz_i, \bx\rangle\\
% &= - \max_{\bx \in V} \ \langle \btheta_t, \bx\rangle - \psi_{t}(\bx) 
% = - \psi^\star_{t,V}(\btheta_t)~.
% \end{align*}
% In the same way, we have $-F_{t+1}(\bx_{t+1}) = \psi^\star_{t+1,V}(\btheta_{t+1})$.
% Also, for any $\bg_t \in \partial \ell_t(\bx_t)$, by Theorem~\ref{thm:props_fenchel} we have 
% $\ell_t^\star(\bg_t) = \langle \bx_t, \bg_t\rangle - \ell_t(\bx_t)$.
% Hence, each term in the sum can be written as
% \begin{align*}
% &F_t(\bx_t) - F_{t+1}(\bx_{t+1}) + \ell_t^\star(\bz_t) \\
% &\quad = \psi^\star_{t+1,V}(\btheta_{t+1}) - \psi^\star_{t,V}(\btheta_t) + \ell_t^\star(\bz_t)\\
% &\quad = H_t(\bz_t) - \psi^\star_{t,V}(\btheta_t)~.
% \end{align*}
% Now, we just add and subtract $H_t(\bg_t) = \psi^\star_{t+1,V}(\btheta_t - \bg_t) +\langle \bg_t,\bx_t\rangle - \ell_t(\bx_t) $ to obtain the stated bound.
% 
% The second case is similar. We just have to observe that if $\psi_{t+1,V} \geq \psi_{t,V}$, then $\psi^\star_{t+1,V} \leq \psi^\star_{t,V}$.
% Hence, each term in the sum can be upper bounded as
% \begin{align*}
% &F_t(\bx_t) - F_{t+1}(\bx_{t+1}) + \ell_t^\star(\bz_t) \\
% &\quad \leq \psi^\star_{t,V}(\btheta_{t+1}) - \psi^\star_{t,V}(\btheta_t) + \ell_t^\star(\bz_t)\\
% &\quad = H'_t(\bz_t) - \psi^\star_{t,V}(\btheta_t)~.
% \end{align*}
% As before, adding and subtracting $H'_t(\bg_t) = \psi^\star_{t,V}(\btheta_{t}-\bg_t) + \langle \bx_t, \bg_t\rangle - \ell_t(\bx_t)$ gives the stated bound.
% \end{proof}

The Theorem~\ref{thm:main} is stated with very weak assumptions to show its generality, but it is immediate to obtain concrete regret guarantees just assuming, for example, strongly convex regularizers and convex and Lipschitz losses and using well-known methods, such as \citet[Lemma~7.8]{Orabona19}

However, we can already understand why this is an interesting guarantee. Let's first consider the case that $\bz_t=\bg_t$ and the constant regularizer $\frac{1}{2\eta}\|\bx\|^2_2$. In this case, we recover the OGD algorithm. Even the guarantee in the Theorem exactly recovers the best known one~\citep[Corollary 7.9]{Orabona19}, with $\delta_t=0$. 
Instead, if we set $\bz_t$ to be the minimizer of $H_t$, \citet{ChenO23} shows that we recover the implicit/proximal update.
Finally, if we set $\bz_t$ such that $H_t(\bz_t)< H_t(\bg_t)$ we will have that $\delta_t>0$. Hence, in each single term of the sum we have a negative factor that makes the regret bound smaller and we can interpret the resulting update as an \emph{approximate implicit/proximal update}.
%While it might be difficult to give a lower bound to $\delta_t$ and $\delta'_t$ without additional assumptions, the main value of this analysis is in giving a \emph{unifying way to design generalized implicit updates for FTRL}. In the next section we show how to recover IWA.


%\subsection{Comparison with Implicit Online Mirror Descent}
%\label{sec:prox}
%
%In this section, we show that when $\bz_t$ is set to minimize $H_t(\bz)$ or $H'_t(\bz)$, we recover different variants of implicit updates.
%
%Assume that the $\ell_t$ are closed and convex. Also, assume that $\psi^\star_{t,V}$ is differentiable, that is true, for example, when $\psi_t$ is strongly convex by Theorem~\ref{thm:duality}. Then, observe that by the first-order optimality condition and Theorem~\ref{thm:props_fenchel}, we have
%\begin{align}
%\bz_t &= \argmin_{\bz} \ H_t(\bz) \nonumber \\
%&\Leftrightarrow \nabla \psi^\star_{t+1,V}(\btheta_{t}-\bz_t) \in \partial \ell^\star_t(\bz_t) \nonumber \\
%& \Leftrightarrow \bz_t \in  \partial \ell_t(\nabla \psi^\star_{t+1,V}(\btheta_{t}-\bz_t)) 
%= \partial \ell_t(\bx_{t+1})~. \label{eq:cond_update_exact}
%\end{align}
%Hence, in this case, we have that the optimal $\bz_t$ is the gradient at the \emph{next} point $\bx_{t+1}$. This is exactly what happens in the implicit updates. 
%
%Under the same assumptions, we also have
%\begin{align}
%\bz_t = \argmin_{\bz} \ H'_t(\bz) 
%&\Leftrightarrow \nabla \psi^\star_{t,V}(\btheta_{t}-\bz_t) \in \partial \ell^\star_t(\bz_t) \nonumber \\
%& \Leftrightarrow \bz_t \in  \partial \ell_t(\nabla \psi^\star_{t,V}(\btheta_{t+1}))~. \label{eq:cond_update_inexact}
%%= \partial \ell_t(\nabla \psi^\star_{t,V}(\btheta_{t+1}))
%\end{align}
%In this other case, the update also has an implicit flavor but the subgradient is queried on a point different from the next point, where the difference depends on how much $\nabla \psi^\star_{t, V}$ differs from $\nabla \psi^\star_{t+1, V}$.
%
%Let's see this connection even more precisely, considering \emph{proximal updates}.
%Hence, for simplicity, let's consider the case that $V=\R^d$, similar considerations hold in the constrained case.
%Consider the case that $\psi_t(\bx)=\frac{\lambda_t}{2}\|\bx\|_2^2$. In this case, the update can be written with the \emph{proximal operator} of the loss functions. In particular, the proximal operator of $\eta f$, is defined as
%\[
%\Prox_{\eta f}(\by)
%\triangleq \argmin_{\bx \in \R^d} \ \frac{1}{2}\|\bx-\by\|_2^2 + \eta f(\bx)~.
%\]
%If the function $f$ is differentiable we have that $\Prox_{\eta f}(\by) = \by - \eta \nabla f(\Prox_{\eta f}(\by))$. In words, the proximal update moves by a quantity that depends on the gradient on the updated point. The implicit nature of these updates justifies the name ``implicit updates'' used in the online learning literature.
%More generally, we have that $\Prox_{\eta f}(\by) \in \by - \eta \partial f(\Prox_{\eta f}(\by))$. We list some common proximal operators in Appendix~\ref{sec:updates}.
%
%Assuming $\lambda_{t+1}$ does not depend on $\bz_t$, using the proximal operator we can rewrite the update in \eqref{eq:cond_update_exact} as
%%Assume that $\psi_{t+1}(\bx)$ to be strictly convex and not depending on $\bz_t$. In this case we have that the minimizer of the bound satisties
%%\[
%%\boldsymbol{0} \in -\nabla \psi^\star_{t+1}(\theta_t-\bz_t)+ \partial \ell^\star_t(\bz_t)
%%\Leftrightarrow \bz_t \in \partial \ell_t(\bx_{t+1}),
%%\]
%%where we used the assumption that $\ell_t$ is convex, proper, closed. Hence, in this case $\bz_t$ is exactly a subradient of the current loss in the next prediction, exactly as in the implicit updates.
%%Let's consider a particular example: $\psi_t(\bx)=\frac{\lambda_t}{2}\|\bx\|_2^2$.
%\begin{align}
%\bx_{t+1}
%&=\frac{\btheta_{t+1}}{\lambda_{t+1}}
%= \Prox_{\frac{\ell_t}{\lambda_{t+1}}}\left(\frac{\btheta_t}{\lambda_{t+1}}\right) \nonumber \\
%&= \Prox_{\frac{\ell_t}{\lambda_{t+1}}}\left(\frac{\lambda_t \bx_t}{\lambda_{t+1}}\right)~. \label{eq:prox_ftrl_2}
%\end{align}
%%and
%%\begin{align}
%%\bx_{t+1}
%%= \Prox_{\frac{\ell_t}{\lambda_{t+1}}}\left(\frac{\btheta_t}{\lambda_{t+1}}\right)
%%= \Prox_{\frac{\ell_t}{\lambda_{t+1}}}\left(\frac{\lambda_t \bx_t}{\lambda_{t+1}}\right)~. \label{eq:prox_ftrl_2}
%%\end{align}
%
%Similarly, we can rewrite the update in \eqref{eq:cond_update_inexact} as
%\begin{align*}
%\frac{\btheta_{t+1}}{\lambda_t}
%&= \frac{\btheta_{t}}{\lambda_t} - \frac{\bz_t}{\lambda_t}
%= \bx_t - \frac{\bz_t}{\lambda_t}
%\in \bx_t - \frac{1}{\lambda_t}\partial \ell_t(\nabla \psi^\star_{t,V}(\btheta_{t+1})) \\
%&= \bx_t - \frac{1}{\lambda_t}\partial \ell_t\left(\frac{\btheta_{t+1}}{\lambda_t}\right)~.
%\end{align*}
%Hence, we have that $\frac{\btheta_{t+1}}{\lambda_t} = \Prox_{\frac{\ell_t}{\lambda_t}} (\bx_t)$ and we get
%\begin{equation}
%\label{eq:prox_ftrl_1}
%\bx_{t+1}
%= \frac{\btheta_{t+1}}{\lambda_{t+1}} 
%= \frac{\lambda_t}{\lambda_{t+1}}\Prox_{\frac{\ell_t}{\lambda_t}} (\bx_t)~.
%\end{equation}
%
%% Moreover, setting $\tilde{\bx}_{t+1} = \frac{\btheta_t-\bz_t}{\lambda_t} = \bx_t - \frac{\bz_t}{\lambda_t}$, we have that 
%% \begin{equation}
%% \label{eq:less_tight_eq1}
%% \bz_t \in \partial \ell_t(\tilde{\bx}_{t+1})~.
%% \end{equation}
%% Hence, from the monotone property of subgradients, we have
%% \[
%% \langle \bz_t, \tilde{\bx}_{t+1}-\bx_t\rangle \geq \langle \bg_t, \tilde{\bx}_{t+1}-\bx_t\rangle,
%% \]
%% that by Cauchy-Schwarz implies that $\|\bz_t\|_2\leq \|\bg_t\|_2$. If $\ell_t$ is strictly convex, then the inequality between $\|\bg_t\|_2$ and $\|\bz_t\|_2$ is strict. Hence, if the worst-case regret guarantees depends on $\|\bz_t\|_2$, then it will be strictly better than the one that depends on $\bg_t$.
%
%
%It is instructive to compare both updates with the one of Implicit Online Mirror Descent using $\psi(\bx)=\frac{1}{2}\|\bx\|^2_2$ as distance generating function and stepsizes $\frac{1}{\lambda_t}$. In this case, we would update with 
%\begin{align}
%\bx_{t+1} 
%&= \argmin_{\bx} \frac12 \|\bx_t-\bx\|_2^2 + \frac{1}{\lambda_t}\ell_t(\bx) \nonumber \\
%&= \Prox_\frac{\ell_t}{\lambda_t}(\bx_t)~. \label{eq:prox_omd}
%\end{align}
%Comparing \eqref{eq:cond_update_exact} and \eqref{eq:cond_update_inexact} to \eqref{eq:prox_omd}, we see, when $\lambda_t \leq \lambda_{t+1}$ as it is usual, the two updates above shrink a bit towards the zero vector, that is the initial point $\bx_1$, before or after the proximal operator. This shrinking is given by the FTRL update and it is the key difference with 
%Implicit OMD update.
%The different update also corresponds to a different guarantee: the regret of the generalized implicit FTRL holds for unbounded domains too, while in Implicit OMD with time-varying stepsizes can have linear regret on unbounded domains~\citep{OrabonaP18}. Interestingly, a similar shrinking has been proposed in \citet{FangHPF20} to fix the unbounded issue in OMD.
%Clearly, the updates \eqref{eq:cond_update_exact} and \eqref{eq:cond_update_inexact} become equivalent to  \eqref{eq:prox_omd} for $\lambda_t$ constant in $t$, that is exactly the only case when implicit/proximal online mirror descent works for unbounded domains.

\section{Importance Weight Aware Updates}

The IWA updates were motivated by the failure of OGD to deal with arbitrarily large importance weights.
In fact, the standard approach to use importance weights in OGD is to simply multiply the gradient by the importance weight. However, when the importance weight is large, we might have an update that is far beyond what is necessary to attain a small loss on it. \citet{KarampatziakisL11} proposed IWA, a computationally efficient way to use importance weights without damaging the convergence properties of OGD.
In particular, IWA updates are motivated by the following invariance property: an example with importance weight $h \in \Nat$ should be treated as if it is an unweighted example appearing $h$ times in the dataset.

More formally, IWA updates are designed for importance weighted convex losses over linear predictors. So, let $\bq_t\in \mathbb{R}^d$ be the $t^\text{th}$ sample and $h_t\in \R_{+}$ its importance weight. Each loss function $\ell_t:\R^d \to \R$ is defined as $\ell_t(\bx)\triangleq \hat{\ell}_t(\langle \bq_t, \bx\rangle)$, where $\bx$ is the predictor, $\langle \bq_t, \bx\rangle$ is the forecast on sample $\bq_t$ of the linear predictor $\bx$, and $\hat{\ell}_t:\R\to\R$ is the $h_t$-weighted convex loss function. For example $\hat{\ell}_t(p) = \frac{h_t}{2} (p-y_t)^2$ for linear regression with square loss,  $\hat{\ell}_t(p)=h_t \ln (1+e^{-y_tp})$, and $\hat{\ell}_t(p)=h_t \max(1-p y_t,0)$ for linear classification with hinge loss.
%As standard in online learning, the goal is to minimize the regret on the losses $\ell_t$ over $T$ rounds. As we said before, minimizing the regret also maximizes the convergence rate in case the samples are drawn i.i.d. from a fixed distribution.

The key idea of \citet{KarampatziakisL11} is performing a sequence of $N$ updates on each loss function $\ell_t$, each of them with learning rate $\eta/N$, and take $N\to\infty$.
Given the particular shape of the loss functions, all the gradients for a given sample $\bq_t$ points in the same direction: $\nabla \ell_t(\bx) = \hat{\ell}'_t(\langle \bq_t, \bx\rangle) \bq_t$. Therefore, the cumulative effect of performing $N$ consecutive updates in a row on each sample $\bq_t$ amounts to a single update in the direction of $\bq_t$ rescaled by a single scalar. Hence, we just have to find this scalar. More in details, the effect of doing a sequence of infinitesimal updates can be modelled by an ordinary differential equation (ODE), as detailed in the following theorem.
%
\begin{theorem}[{\citep[Theorem~1]{KarampatziakisL11}}]
\label{th:iwaupdate}
Let $\hat{\ell}$ to be continuously differentiable. Then, the limit for $N\to\infty$ of the OGD update with $N$ updates on the same loss function with learning rate $\eta/N$ is equal to the update
\[
\bx_{t+1} = \bx_t - s_t(1)\bq_t,
\]
where the scaling function $s_t:\R\to \R$ satisfies $s_t(0)=0$ and the differential equation
\[
s'_t(h)
= \eta \hat{\ell}'_t(\langle\bq_t, \bx_t - s_t(h)\bq_t\rangle)~.
\]
\end{theorem}

\paragraph{IWA Updates are Generalized Implicit Updates}
As we said above, IWA updates do not have strong theoretical guarantees. Indeed, we do not even know if they give the same performance of the plain gradient updates in the worst case.
Here, we show that IWA is an instantiation of the generalized implicit FTRL. This implies that its regret upper bound is better than the one of online gradient descent. Moreover, this also gives a way to interpret IWA updates as approximate proximal/implicit updates.

Denote by $p_t \triangleq \langle \bq_t, \bx_t\rangle$, $\bx_t(h) \triangleq \bx_t - s_t(h)\bq_t$,  $p_t(h) \triangleq \langle \bq_t, \bx_t(h)\rangle$, and $\bg_t(h) \triangleq \hat{\ell}'_t(p_t(h)) \bq_t$. Consider the generalized implicit FTRL with regularization $\psi_t(\bx) = \frac{1}{2\eta}\|\bx\|^2_2$ and $V=\R^d$.
Set $\bz_t$ as
\begin{align}
\bz_t 
&\triangleq \int_{0}^1 \! \bg_t(h) \, \mathrm{d}h \nonumber \\
&= \frac{1}{\eta} \left(\int_{0}^1 \! \eta \hat{\ell}_t'(\langle\bq_t, \bx_t-s_t(h)\bq_t\rangle) \, \mathrm{d}h \right) \bq_t \nonumber \\
&=  \frac{1}{\eta} s_t(1) \bq_t~. \label{eq:iwa_z}
\end{align}
Then, the iterates of Algorithm~\ref{alg:giftrl} are the same as the iterates of IWA updates 
\[
\bx_{t+1} 
= \frac{\btheta_t-\bz_t}{1/\eta} 
= \bx_t - \frac{\bz_t}{1/\eta} 
= \bx_t - s_t(1) \bq_t~.
\]
In words, we can now analyze IWA updates as an instantiation of generalized implicit updates.

In particular, Theorem~\ref{th:iwa} shows sufficient conditions on the loss $\hat{\ell}_t$ to guarantee that IWA updates are as good as the subgradient $\bg_t$ by proving that $\bz_t$ satisfy $H_t(\bz_t) \leq H_t(\bg_t)$.
\begin{theorem}
\label{th:iwa}
Assume $s'_t(h)$ to be continuous in $[0,1]$. If $\forall h\in[0,1]$, $\hat{\ell}'_t(p_t(h))$ satisfies one of the following requirements:
\begin{itemize}
\item $\hat{\ell}'_t(p_t(h))\geq0$, $\hat{\ell}'''_t(p_t(h)) \geq 0$
\item $\hat{\ell}'_t(p_t(h))\leq0$, $\hat{\ell}'''_t(p_t(h)) \leq 0$
\end{itemize}
then $\bz_t=\int_{0}^1 \! \bg_t(h) \, \mathrm{d}h$ satisfies
\begin{equation}
\label{eq:iwa}
H_t(\bz_t)\leq H_t(\bg_t)~.
%\psi^\star\left(\btheta_t-\bz_t \right) + \ell_t^\star \left(\bz_t\right) 
%\leq \psi^\star(\btheta_t-\bg_t) + \ell_t^\star (\bg_t)~.
\end{equation}
\end{theorem}

Before proving it, we will need the following technical lemmas.

\begin{lemma}
\label{lemma:ell}
Let $\hat{\ell}_t:\R\to \R$ to be three times differentiable.
\begin{itemize}
\item If $\hat{\ell}'(p_t(h))\geq0, \hat{\ell}'''(p_t(h))\geq 0$, then $s'_t(h)$ is non-negative, non-increasing, convex.
\item If $\hat{\ell}_t'(p_t(h))\leq0, \hat{\ell}_t'''(p_t(h)) \leq 0$, then $s'_t(h)$ is non-positive, non-decreasing, concave.
\end{itemize}
\end{lemma}
%
\begin{proof}
First, observe that
\begin{align*}
s'_t(h) &= \eta\hat{\ell}'(\langle\bx_t-s_t(h)\bq_t,\bq_t\rangle)=\eta \hat{\ell}'(p_t (h))\\
s''(h)  &= \eta\hat{\ell}''(p_t (h))(-\| \bq_t\|^2)s'_t(h)\\
s'''(h) &= \eta\hat{\ell}'''(p_t (h))\| \bq_t\|^4 (s'_t(h))^2\\
&\quad +\eta \hat{\ell}''(p_t (h))(-\| \bq_t\|^2_2)s''(h)~.
\end{align*}

\textbf{Case 1:} $\hat{\ell}'(p)\geq0, \hat{\ell}'''(p)\geq 0$.
In this case, $s'_t(h)\geq0$, $s''(h)\leq0$,  $s'''(h)\geq0$. That is, $s'_t(h)$ is non-negative, non-increasing, and convex.

\textbf{Case 2:} $\hat{\ell}'(p)\leq0, \hat{\ell}'''(p) \leq 0$.
In this case, $s'_t(h)\leq0$, $s''(h)\geq0$,  $s'''(h)\leq0$. That is, $s'_t(h)$ is non-positive, non-decreasing, and concave.
\end{proof}

\begin{lemma}
\label{lemma:int}
Let $s'_t(h)$ to be continuous in $[0,1]$.
\begin{itemize}
\item If $s'_t(h)$ is convex and non-negative, then $\forall h \in [0,1]$ we have
\[
\frac{1}{2}(s'_t(0)+s'_t(h)) \geq \frac{h}{2}(s'_t(0)+s'_t(h)) \geq s_t(h)~.
\]
\item If $s'_t(h)$ is concave and non-positive, then $\forall h \in [0,1]$ we have
\[
\frac{1}{2}(s'_t(0)+s'_t(h)) \leq \frac{h}{2}(s'_t(0)+s'_t(h)) \leq s_t(h)~.
\]
\end{itemize}
\end{lemma}
%
\begin{proof}
Given that $s'_t(h)$ is non-negative, we have
$\frac{1}{2}(s'_t(0)+s'_t(h)) \geq \frac{h}{2}(s'_t(0)+s'_t(h))$.
Now, observe that $\frac{h}{2}(s'_t(0)+s'_t(h))$ is the area of the trapezium with first base $s'_t(0)$, second base $s'_t(h)$, and height $h$. Given that the function is convex and non-negative, this area is bigger than the integral of $s'_t$ between 0 and $h$, that is equal to $s_t(h)$, that proves the statement.

We can prove the other case in a similarly way. 
\end{proof}

We can now prove Theorem~\ref{th:iwa}.
\begin{proof}[Proof of Theorem~\ref{th:iwa}]
The left hand side of \eqref{eq:iwa} is equal to
\[
\psi^\star\left(\int_{0}^1 \! \btheta_t- \bg_t(h) \, \mathrm{d}h\right) + \ell_t^\star \left(\int_{0}^1 \! \bg_t(h) \, \mathrm{d}h\right)~.
\]
Since $\psi^\star$ and $\ell_t^\star$ are convex, applying Jensen's inequality, the left hand side of \eqref{eq:iwa} is upper bounded by
\[
\int_{0}^1 \! \psi^\star( \btheta_t- \bg_t(h)) \, \mathrm{d}h + \int_{0}^1 \! \ell_t^\star(\bg_t(h)) \, \mathrm{d}h~.
\]
Moreover, the right hand side of \eqref{eq:iwa} is equal to
\[
\int_{0}^1 \! \psi^\star(\btheta_t- \bg_t) \, \mathrm{d}h + \int_{0}^1 \! \ell_t^\star(\bg_t) \, \mathrm{d}h~.
\]
So, if we can prove that 
\begin{align*}
&\int_{0}^1 \! \psi^\star(\btheta_t- \bg_t(h)) \, \mathrm{d}h + \int_{0}^1 \! \ell_t^\star(\bg_t(h)) \, \mathrm{d}h\\ 
&\quad \leq \int_{0}^1 \! \psi^\star(\btheta_t- \bg_t) \, \mathrm{d}h + \int_{0}^1 \! \ell_t^\star(\bg_t) \, \mathrm{d}h,
\end{align*}
then \eqref{eq:iwa} is proved. 

For this, it is sufficient to prove that $\forall h\in[0,1]$, we have
\[
\psi^\star(\btheta_t- \bg_t(h)) + \ell_t^\star(\bg_t(h))
\leq \psi^\star(\btheta_t- \bg_t) + \ell_t^\star(\bg_t)~.
\]
Given that $\psi_t^\star(\btheta) = \frac{1}{2\lambda}\|\btheta\|^2_2$, where $\lambda = 1/\eta$, and by using the fact that $\langle \bg, \bx \rangle = \ell_t(\bx)+\ell_t^\star(\bg)$, for any pair of $\bx$, $\bg$ satisfying $\bg \in \partial \ell_t(\bx)$, the inequality above can be written as
\begin{align*}
\frac{1}{2\lambda} \| \btheta_t &-\bg_t(h) \|^2_2 + \langle \bg_t(h), \bx_t(h) \rangle - \ell_t(\bx_t(h)) \\
&\quad \leq \frac{1}{2\lambda}\| \btheta_t -\bg_t \|^2_2 + \langle \bg_t,\bx_t \rangle - \ell_t(\bx_t)~.
%\Leftrightarrow &-\frac{1}{\lambda} \langle \btheta_t, \bg_t(h) \rangle + \frac{1}{2\lambda} \| \bg_t(h)\|^2 + \langle \bg_t(h), \bx_t(h) \rangle - \ell_t(\bx_t(h))~.
\end{align*}

Simplifying this inequality, we have
\begin{align*}
&\frac{1}{2\lambda} \| \bg_t(h)\|^2_2 - \frac{1}{2\lambda} \| \bg_t\|^2_2\\
&\quad \leq \ell_t(\bx_t(h)) - \ell_t(\bx_t) -  \langle \bg_t(h), \bx_t(h) -\bx_t \rangle~.
\end{align*}
Since $\ell_t(\bx)$ is convex, we obtain
\begin{align*}
&\ell_t(\bx_t(h)) - \ell_t(\bx_t) -  \langle \bg_t(h), \bx_t(h) -\bx_t \rangle \\
&\quad \geq \langle \bg_t, \bx_t(h) -\bx_t \rangle - \langle \bg_t(h), \bx_t(h) -\bx_t \rangle \\
&\quad = \langle \bg_t(h)-  \bg_t , \bx_t - \bx_t(h) \rangle~.
\end{align*}

So, we just need to prove that 
\[
\frac{1}{2\lambda} \| \bg_t(h)\|^2_2 - \frac{1}{2\lambda} \| \bg_t\|^2_2
\leq \langle \bg_t(h)-  \bg_t , \bx_t - \bx_t(h) \rangle~.
\]
Using $\| \ba\|^2_2 - \|\bb\|^2_2 = \langle \ba-\bb, \ba+\bb\rangle$, the inequality above can be rewritten as
\begin{align*}
\frac{1}{2\lambda} \langle \bg_t(h) -\bg_t, \bg_t(h) + \bg_t \rangle 
&\leq \langle \bg_t(h)-  \bg_t , \bx_t - \bx_t(h) \rangle \\
&= \langle \bg_t(h)-  \bg_t , s_t(h) \bq_t \rangle~.
\end{align*}
That is,
\begin{align*}
\frac{1}{2}& \left( \hat{\ell}'_t(p_t(h)) - \hat{\ell}'_t(p_t)  \right)\left( \frac{1}{\lambda} \hat{\ell}'_t(p_t(h)) + \frac{1}{\lambda} \hat{\ell}'_t(p_t)  \right) \| \bq_t\|^2_2 \\
&\leq \left( \hat{\ell}'_t(p_t(h)) - \hat{\ell}'_t(p_t)  \right) s_t(h) \| \bq_t\|^2_2~.
\end{align*}
Since $s'_t(h) = \frac{1}{\lambda} \hat{\ell}'_t(p_t(h))$, multiplying both side by $1/\lambda$, the above inequality becomes
\begin{align}
&\frac{1}{2}(s'_t(h) - s'_t(0))(s'_t(0)+s'_t(h)) \nonumber \\
&\quad \leq (s'_t(h) - s'_t(0))s_t(h)~. \label{eq:iwa_s}
\end{align}

Now, we consider two cases.

\textbf{Case 1:} $\hat{\ell}'_t(p_t(h))\geq0$ and $\hat{\ell}_t'''(p_t(h))\geq 0$.
By Lemma~\ref{lemma:ell}, $s'_t(h)$ is non-negative, non-increasing, and convex. So, in particular we have $s'_t(h) - s'_t(0)\leq 0$.
In this case, by Lemma~\ref{lemma:int}, we have $\frac{1}{2}(s'_t(0)+s'_t(h)) \geq s_t(h)$.
%So, \eqref{eq:iwa_s} is proved. 

\textbf{Case 2:} $\hat{\ell}'_t(p_t(h))\leq0$, and $\hat{\ell}_t'''(p_t(h)) \leq 0$.
By Lemma~\ref{lemma:ell}, $s'_t(h)$ is non-positive, non-decreasing, concave. So, in particular, we have $s'_t(h) -s'_t(0) \geq 0$.
In this case,  by Lemma~\ref{lemma:int}, we have $\frac{1}{2}(s'_t(0)+s'_t(h)) \leq s_t(h)$.
%Equation~\eqref{eq:iwa_s} is proved. 

Combining the two cases, we conclude that \eqref{eq:iwa_s} is true that implies that \eqref{eq:iwa} is true as well.
\end{proof}



\subsection{Examples of Losses and IWA Updates}

Now, we present some examples of loss functions that satisfy the requirements of Theorem~\ref{th:iwa} and their corresponding IWA updates from \citet{KarampatziakisL11}. In all the following examples, the prediction of the algorithm on sample $\bq$ is $\langle \bq, \bx\rangle$.

\textbf{Logistic loss: $\hat{\ell}(p) = h \ln(1+e^{-y p})$}.

The IWA update is $\frac{W(e^{h \eta \|\bq\|^2_2 + y p +e^{yp}})-h \eta \|\bq\|^2_2-e^{y p}}{y\|\bq\|^2_2}$ for $y \in \{-1, 1\}$, where $W(x)$ is the Lambert function.
We have that
\[
\hat{\ell}' (p) = \frac{-y h}{1+e^{py}}, \quad 
\hat{\ell}''' (p) = h y^3(-e^{-py-1})~.
\]
When $y \geq 0$, $\hat{\ell}'(p) \leq 0$ and $\hat{\ell}''' (p)\leq 0$. When $y\leq 0$, $\hat{\ell}'(p) \geq 0$ and $\hat{\ell}''' (p)\geq 0$. 

\textbf{Exponential loss: $\hat{\ell}(p)=e^{-yp}$}.

The IWA update is $\frac{p y -\ln(h \eta \|\bq\|_2^2+e^{p y})}{\|\bq\|_2^2 y}$ for $y \in \{-1, 1\}$.
We have that
\[
\hat{\ell}' (p) = y(-e^{-py}), \quad 
\hat{\ell}''' (p) = y^3(-e^{-py})~.
\]
When $y \geq 0$, $\hat{\ell}l'(p) \leq 0$ and $\hat{\ell}''' (p)\leq 0$.  
When $y\leq 0$, $\hat{\ell}'(p) \geq 0$ and $\hat{\ell}''' (p)\geq 0$.


\textbf{Logarithmic loss: $\hat{\ell}(p) = y \ln(y/p) + (1-y) \ln((1-y)/(1-p))$}.

The IWA update is $\frac{p -1 + \sqrt{(p-1)^2+2h \eta \|\bq\|^2_2}}{\|\bq\|_2^2}$ for $y=0$, and $\frac{p - \sqrt{p^2+2h \eta \|\bq\|^2_2}}{\|\bq\|_2^2}$ for $y=1$.
\begin{itemize}
\item if y=0
\[
\hat{\ell}' (p)=\frac{1}{1-p},\quad
\hat{\ell}''' (p)= -\frac{2}{(p-1)^3}~.
\]
\item if y=1
\[
\hat{\ell}' (p)=-\frac{1}{p}, \quad
\hat{\ell}''' (p)= -\frac{2}{p^3}~.
\]
\end{itemize}
For both cases, $\hat{\ell}'(p)$ and $\hat{\ell}'''(p)$ will have the same sign.
 
% \textbf{Hinge loss: $\hat{\ell}(p)=\max(0, 1-yp)$}.
% 
% The IWA update in this case coincides with the proximal update: $-y \min(h \eta, \frac{1-y p}{\|\bq\|^2_2})$ for $y\in \{-1,1\}$.
% When $py\leq 1$,
% \[
% \hat{\ell}' (p)=-y,\quad \hat{\ell}''' (p)=0~.
% \]
% Since the IWA updates will not overshoot the hinge corner, the requirements are satisfied. 

\textbf{Squared loss: $\hat{\ell}(p) = \frac{1}{2}(y-p)^2$.}
The IWA update is $\frac{p-y}{\|\bq\|^2_2}(1-e^{-h \eta \|\bq\|^2_2})$.
\[
\hat{\ell}' (p)=p-y,\quad \hat{\ell}''' (p)=0~.
\]
In this case, the sign of the first derivative can change. However, according to Section 4.2 of \citet{KarampatziakisL11}, for the squared loss IWA will not overshoot the minimum. This means that for any $h \in [0,1]$, $p_t(h)-y$ will always have the same sign, so the conditions are verified.

\begin{table}[t]
	\centering
	\caption{Preallocation strategy results with $3$ machines per tool group and $10$ operations per lot}
	\label{tab:table}
	\figspace\scriptsize
	%	\resizebox{15.5cm}{!}{
		\begin{tabular}{|l%r
				cl||rr|rr|rr|rr|}
			%			\hline
			%			&                    &                      & %        &
			%			 \multicolumn{8}{c}{\textbf{M = 9}} \\
			\hline
			& \multicolumn{1}{@{\hspace{-3mm}}c@{\hspace{-3mm}}}{\textbf{9 Machines}}                   &                      & % &
			\multicolumn{2}{r|}{\textbf{70 Operations}}                 & \multicolumn{2}{r|}{\textbf{80 Operations}}                 & \multicolumn{2}{r|}{\textbf{90 Operations}}                 & \multicolumn{2}{r|}{\textbf{100 Operations}}                 \\
			& Size % \multicolumn{2}{c}{\textbf{Parameters}}            
			&        &
			Lot                         & Step                        & Lot                         & Step                        & Lot          & Step         & Lot          & Step         \\
			%			& size              % & setup % idx
			%			                  &         & 0                           & 1                           & 0                           & 1                           & 0            & 1            & 0            & 1            \\
			%			&                    & setup                &         &                             &                             &                             &                             &              &              &              &              \\
			\hline\hline
			\multirow{3}{*}{\textbf{Fixed}}    & \multirow{3}{*}{1} & % \multirow{3}{*}{0/1} &
			Makespan    & 483                         & 428                         & 489                         & 440                         & 486          & 531          & 592          & 553         \\
			&                    & %                     &
			Setup/Batch & 6/12                        & 2/12                        & 5/14                        & 0/13                        & 5/14         & 3/12         & 3/12         & 0/16         \\
			&                    & %                     &
			1\ts{st}/2\ts{nd} Stage & 2/1                         & TO/27                          & 6/2                        & TO/13                          & 11/13         & TO           & TO/78           & TO           \\
			\midrule
			\multirow{6}{*}{\textbf{Flexible}} & \multirow{3}{*}{2} & % \multirow{6}{*}{0}   &
			Makespan    & 483                         & 475                         & 592                         & 592                         & 592          & 539          & 745          & 698          \\
			&                    & %                     &
			Setup/Batch & 2/8                        & 0/9                        & 1/8                        & 1/8                        & 1/10         & 0/11          & 0/12          & 0/15          \\
			&                    & %                     &
			1\ts{st}/2\ts{nd} Stage & 5/1                         & TO                          & TO/114                          & TO/1                          & TO/130           & TO           & TO           & TO          \\
			\cline{2-11}
			%			& & & & & & & & & & &   \\
			& \multirow{3}{*}{3} & %                     &
			Makespan    & 559                         & --                          & 815                         & --                          & 1357 & -- & 1486 & -- \\ % \multicolumn{4}{c|}{\multirow{3}{*}{Assignment issue}}     \\
			&                    & %                     &
			Setup/Batch & 0/8                         & --                          & 0/8                        & --                          & 0/10 & -- & 10/18 & -- \\ %\multicolumn{4}{c|}{}                                      \\
			&                    & %                     &
			1\ts{st}/2\ts{nd} Stage & TO                       & --                          & TO/140                          & --                          & TO/79 & -- & TO & -- \\ %\multicolumn{4}{c|}{}                                      \\
			\midrule
			\multirow{6}{*}{\textbf{Setup}}    & \multirow{3}{*}{2} & % \multirow{6}{*}{1}   &
			Makespan    & 483                         & 475                         & 592                         & 592                         & 592          & 536          & 745          & 683          \\
			&                    & %                     &
			Setup/Batch & 2/8                        & 0/9                        & 1/8                        & 1/8                        & 1/10         & 0/12          & 0/13          & 0/16          \\
			&                    & %                     &
			1\ts{st}/2\ts{nd} Stage & 2/1                        & TO                          & TO/21                          & TO/25                          & TO/22           & TO           & TO/76           & TO           \\
			%			& & & & & & & & & & &   \\
			\cline{2-11}
			& \multirow{3}{*}{3} & %                     &
			Makespan    & \textbf{334}                         & --                          & \textbf{345}                         & --                          & \textbf{434}          & --           & \textbf{555}          & --           \\
			&                    & %                     &
			Setup/Batch & 0/8                         & --                          & 0/8                         & --                          & 0/11          & --           & 0/12          & --           \\
			&                    & %                     &
			1\ts{st}/2\ts{nd} Stage & TO/20                       & --                          & TO/123                          & --                          & TO           & --           & TO/73           & --           \\
			\hline
		\end{tabular}
		%	}
\end{table}
%
We constructed a scalable set of benchmark instances, focusing on sub-routes of
$10$ production operations for two product types from the SMT2020 simulation scenario~\cite{kopp2020smt2020}.
The $10$ operations in both sub-routes are processed by machines
belonging to three tool groups and do thus involve re-entrant flow,
as a lot visits the same tool group multiple times.
Moreover, the operations incorporate batching and specific setups, and machines undergo periodic maintenance operations.
In the following, we concentrate on instances with $9$ machines, i.e., $3$ per
tool group, and gradually increasing number of lots.
Further smaller- and larger-scale instances along with our implementation are
available online.\footref{foo:online}

We ran our experiments with \clingodl\ (version 1.4.0) on an Intel® Core™i7-8650U CPU Dell Latitude 5590 machine under Windows 10, imposing two time limits per run:
the first stage for makespan minimization is aborted at $450$ seconds, in which case the best schedule found so far % (if any) 
is taken as upper bound on the makespan for proceeding to minimize setup and batch violations with 
another $150$ seconds time limit.

Table~\ref{tab:table} reports the quality of best schedules obtained within the time limits for both optimization stages, split into `Makespan' and `Setup/Batch'
values, while two runtimes or `TO' for a timeout, respectively, are given in the
`1\ts{st}/2\ts{nd} Stage' rows, only listing a single `TO' entry in case both stages timed out.
The `Size' column provides the value taken for the constant \lstinline{sub_size},
limiting the number of machines in subgroups to which the operations are preallocated.
For the latter, the `Lot' columns include results with value \lstinline{0} for the constant \lstinline{lot_step}, where a common subgroup takes all operations for a lot, or for value \lstinline{1} in the `Step' columns, leading to their distribution among subgroups.

The `Size' value 1 necessarily leads to a fixed machine assignment, for which the
quality indicators clearly show that the `Step' strategy yields better schedules,
although it incurs more timeouts and thus fewer certain optima because operations on different lots increase the flexibility of execution sequences and thus search complexity.
While flexibility within subgroups by setting their `Size' to 2 or 3 in principle allows for improved schedules, we observe a deterioration due to sharply increasing instantiation size and search effort, as already observed in \cite{ali2023flexible}.
The setup strategy to differentiate operations and machines within subgroups,
activated by changing the constant \lstinline{by_setup},
aims to cut down the scheduling complexity in line with the optimization objectives by reducing the need for setup changes.
This leads to significantly improved schedules with `Size' 3, where the
`Lot' and `Step' preallocation strategies are indifferent and redundant results for the latter are omitted, up to a critical size reached with $100$~operations.

With our preliminary approach~\cite{ali2023flexible}, using a more naive and less feature-rich encoding of either fixed or fully flexible machine assignments, the
threshold at which problem size and combinatorics get prohibitive was reached at less than $50$ operations already.
Despite gearing up to double that size, our benchmark instances still represent small excerpts of the large-scale semiconductor fabs with more than $100$ tool groups and from $242$ to $543$ production operations per lot modeled by~\cite{kopp2020smt2020}.
%
The elevated complexity in comparison to basic settings like the traditional FJSP is mainly due to sophisticated setup and maintenance operations, requiring a detailed analysis of execution sequences on machines for SMSP.
We conjecture that similar scalability limits would also be encountered with MIP or CP encodings, yet the first-order modeling language of ASP with difference logic facilitates rapid prototyping and experimentation.
In fact, our performance evaluation aims to explore the feasibility of search and optimization, in order to come up with strategies for breaking down large SMSP instances into more manageable portions, e.g., focusing on some bottleneck tool groups or re-entrant flow of operations.

% This section will show the experimental results performed by applying the machine assignment strategies mentioned before, with several instances ranging from $30$ to $130$ steps and $6$ to $12$ machines. All experiments are run using an Intel\textsuperscript{\textregistered} Core\texttrademark{} i7-8650U CPU Dell Latitude 5590 machine under Windows 10. Our timeout limit is $600$ seconds, splitted to $450$ seconds for the makespan and $150$ seconds for the setup and batching. 

% We considered three tool groups for all generated instances in which batch processing, time/counter-based maintenance, and setup are considered. For generating the instances, we started with a small instance containing $30$ steps and $6$ machines where each tool group has $2$ machines and then we generate the next instance by adding one more lot, which has $10$ steps. We kept the tool group size till the fixed machine assignment strategy could not reach the optimum within the time limit. We created $3$ parameters \textit{size, idx} and \textit{setup} to activate a specific machine assignment strategy. The size determines the size of a sub-group in each tool group. The $idx$ defines the Job/Step-based indexing of all steps in the same tool group where all steps of the same lot will have the same index if the $idx = 0$ and Hence, they are assigned to the same sub-group/machine. If $idx = 1$, then each step in the tool group will have an identical index. The last parameter setup is to activate the setup strategy or not. If the $setup = 1$, then the setup strategy is applied; if $setup = 0$ then it's not applied.

% % To continue tomorrow isA :)
% Table \ref{tab:table01} shows the results of the instances with $2$ machines in each toll group. The first column refers to the strategy applied for the machine assignment. The second and third columns show the parameters for selecting a particular strategy. The assignment is fully flexible if the \textit{size} is greater than or equal to the number of machines in a tool group. Otherwise, the assignment is partially flexible. In the fourth column, we list our optimization criteria and the time limit for the makespan and setup/batching represented by 1st/2nd call. Each following two consecutive columns illustrate the results of an instance when the Job/Step-based indexing is selected. From the \ref{tab:table01}, we observed that the best-obtained results were achieved by the full flexible assignment in the first three instances and for the last instance, the setup strategy was the best. The fixed/setup strategies terminated within the time limit except for only one case.

% \begin{table}[h]
% 	\centering
% 	\caption{Comparison between the allocation strategies with 2 machines per tool group}
% 	\label{tab:table01}
% %	\resizebox{15.5cm}{!}{
% 		\begin{tabular}{|l%r
% 			cl||rr|rr|rr|rr|}
% 			\hline
% %			&                    &                      &         & \multicolumn{8}{c}{\textbf{M = 6}} \\
% %			\hline
% 			& \textbf{M = 6}                   & %                     &
% 			  & \multicolumn{2}{r|}{\textbf{Instance 01}}                 & \multicolumn{2}{r|}{\textbf{Instance 02}}                 & \multicolumn{2}{r|}{\textbf{Instance 03}}                 & \multicolumn{2}{r|}{\textbf{Instance 04}}                 \\
% 			& Size % \multicolumn{2}{c}{\textbf{Parameters}}            
% 			 &			         & Job                         & Step                        & Job                         & Step                        & Job          & Step         & Job          & Step         \\
% 			\hline
% %			& size               & setup %idx
% %			                  &         & 0                           & 1                           & 0                           & 1                           & 0            & 1            & 0            & 1            \\
% %			&                    & setup                &         &                              &                             &                             &                             &              &              &              &              \\
% 			\hline
% 			\multirow{3}{*}{\textbf{Fixed}}    & \multirow{3}{*}{1} & % \multirow{3}{*}{0/1} &
% 			 Makespan    & 409                         & 353                         & 409                         & 409                         & 525          & 424          & 525          & 493          \\
% 			&                    & %                     &
% 			 Setup/Batch & 5/6                         & 4/6                         & 4/8                         & 4/8                         & 4/9          & 1/9          & 3/11          & 2/10          \\
% 			&                    & %                     &
% 			 1\ts{st}/2\ts{nd}-Call & \textless{}1/\textless{}1 & \textless{}1/\textless{}1 & \textless{}1/\textless{}1 & \textless{}1/\textless{}1 & 31/1         & 137/6        & 37/11          & TO/53           \\
% 			\midrule
% 			\multirow{3}{*}{\textbf{Flexible}} & \multirow{3}{*}{2} & % \multirow{3}{*}{0}   &
% 			 Makespan   & \textbf{233}                         & --                          & \textbf{281}                         & --                          & \textbf{365}          & --           & 587          & --           \\
% 			&                    & %                     &
% 			 Setup/Batch & 0/5                         & --                          & 0/6                         & --                          & 0/8          & --           & 3/9          & --           \\
% 			&                    & %                     &
% 			 1\ts{st}/2\ts{nd}-Call & 7/0                         & --                          & TO/6                          & --                          & TO/83           & --           & TO           & --           \\
% 			\midrule
% 			\multirow{3}{*}{\textbf{Setup}}    & \multirow{3}{*}{2} & % \multirow{3}{*}{1}   &
% 			 Makespan  & 277                         & --                          & 321                         & --                          & 381          & --           & \textbf{419}          & --           \\
% 			&                    & %                     &
% 			 Setup/Batch & 0/4                         & --                          & 0/6                         & --                          & 0/8          & --           & 0/9          & --           \\
% 			&                    & %                     &
% 			 1\ts{st}/2\ts{nd}-Call & \textless{}1/\textless{}1 & --                          & 25/1                         & --                          & TO/12        & --           & TO/122           & -- \\
% 			 \hline
% 		\end{tabular}
% %	}
% \end{table}

% Table~\ref{tab:table02} summarizes the results of the subsequent $4$ instances where each tool group has $3$ machines. In this instances group, we can split the machines into sub-group by setting the \textit{size} parameter to $2$; in that case, we have two sub-groups in each tool group. The experiments demonstrated that the fixed strategy has the same or better performance than the flexible. In addition, the flexible strategy could not find a feasible solution for instances $7$ and $8$ when all machines were in the same group. On the other hand, the setup strategy performed better than the other two strategies when all machines were in one group, in addition to reaching the optimal value of the setup for all instances. 

% \begin{table}[h]
% 	\centering
% 	\caption{Comparison between the allocation strategies with 3 machines per tool group}
% 	\label{tab:table02}
% %	\resizebox{15.5cm}{!}{
% 		\begin{tabular}{|l%r
% 			cl||rr|rr|rr|rr|}
% %			\hline
% %			&                    &                      & %        &
% %			 \multicolumn{8}{c}{\textbf{M = 9}} \\
% 			\hline
% 			& \textbf{M = 9}                   &                      & % &
% 			 \multicolumn{2}{r|}{\textbf{Instance 05}}                 & \multicolumn{2}{r|}{\textbf{Instance 06}}                 & \multicolumn{2}{r|}{\textbf{Instance 07}}                 & \multicolumn{2}{r|}{\textbf{Instance 08}}                 \\
% 			& Size % \multicolumn{2}{c}{\textbf{Parameters}}            
% 			&        &
% 			 Job                         & Step                        & Job                         & Step                        & Job          & Step         & Job          & Step         \\
% %			& size              % & setup % idx
% %			                  &         & 0                           & 1                           & 0                           & 1                           & 0            & 1            & 0            & 1            \\
% %			&                    & setup                &         &                             &                             &                             &                             &              &              &              &              \\
% 			\hline\hline
% 			\multirow{3}{*}{\textbf{Fixed}}    & \multirow{3}{*}{1} & % \multirow{3}{*}{0/1} &
% 			 Makespan    & 525                         & 433                         & 525                         & 452                         & 525          & 521          & 643          & \textbf{559}          \\
% 			&                    & %                     &
% 			 Setup/Batch & 6/13                        & 1/13                        & 5/15                        & 0/14                        & 5/16         & 6/16         & 6/12         & 3/12         \\
% 			&                    & %                     &
% 			 1\ts{st}/2\ts{nd}-Call & 30/3                         & TO/153                          & 24/8                        & TO/63                          & 231/81         & TO           & TO           & TO           \\
% 			\midrule
% 			\multirow{6}{*}{\textbf{Flexible}} & \multirow{3}{*}{2} & % \multirow{6}{*}{0}   &
% 			 Makespan    & 525                         & 475                         & 650                         & 650                         & 650          & 595          & 745          & 742          \\
% 			&                    & %                     &
% 			 Setup/Batch & 2/11                        & 0/11                        & 1/12                        & 1/12                        & 6/13         & 4/14          & 3/17          & n/a          \\
% 			&                    & %                     &
% 			 1\ts{st}/2\ts{nd}-Call & 26/7                         & TO                          & TO/12                          & TO                          & TO           & TO           & TO           & TO           \\
% 			\cline{2-11}
% %			& & & & & & & & & & &   \\
% 			& \multirow{3}{*}{3} & %                     &
% 			 Makespan    & 744                         & --                          & 1206                         & --                          & 1698 & -- & n/a & -- \\ % \multicolumn{4}{c|}{\multirow{3}{*}{Assignment issue}}     \\
% 			&                    & %                     &
% 			 Setup/Batch & 2/12                         & --                          & n/a                        & --                          & 8/15 & -- & n/a & -- \\ %\multicolumn{4}{c|}{}                                      \\
% 			&                    & %                     &
% 			 1\ts{st}/2\ts{nd}-Call & TO                       & --                          & TO                          & --                          & TO & -- & TO & -- \\ %\multicolumn{4}{c|}{}                                      \\
% 			\midrule
% 			\multirow{6}{*}{\textbf{Setup}}    & \multirow{3}{*}{2} & % \multirow{6}{*}{1}   &
% 			 Makespan    & 525                         & 475                         & 650                         & 650                         & 643          & 553          & 745          & 642          \\
% 			&                    & %                     &
% 			 Setup/Batch & 2/11                        & 0/11                        & 1/12                        & 1/12                        & 1/14         & 0/13          & 1/14          & 1/16          \\
% 			&                    & %                     &
% 			 1\ts{st}/2\ts{nd}-Call & 44/2                        & TO                          & TO/4                          & TO/2                          & TO           & TO/7           & TO           & TO           \\
% %			& & & & & & & & & & &   \\
% 			\cline{2-11}
% 			& \multirow{3}{*}{3} & %                     &
% 			 Makespan    & \textbf{346}                         & --                          & \textbf{373}                         & --                          & \textbf{429}          & --           & 820          & --           \\
% 			&                    & %                     &
% 			 Setup/Batch & n/a                         & --                          & n/a                         & --                          & n/a          & --           & n/a          & --           \\
% 			&                    & %                     &
% 			 1\ts{st}/2\ts{nd}-Call & TO                       & --                          & TO                          & --                          & TO           & --           & TO           & --           \\
% 			\hline
% 		\end{tabular}
% %	}
% \end{table}

% Table~\ref{tab:table03} considers $4$ machines in each tool group and the flexible strategy obtained the best result for the first instance. However, it had the same feasibility issue when all machines were in the same group. For the rest instances, the setup strategy dominated when the machines were equally distributed into sub-groups. 

% From the conducted experiments, we can conclude that 
% \begin{itemize}
% 	\item The flexible assignment performed well on the small-scale.
% 	\item While increasing the scale, the setup strategy dominates in the most cases
% 	\item Assigning the steps of the same lot independently with the fixed assignment leads to better performance
% 	\item The Setup strategy has a significant impact in minimizing the setup objective through all instances
% 	\item The full flexible assignment has an assignment issue while increasing the number of machines
% \end{itemize}

% \begin{table}[h]
% 	\centering
% 	\caption{Comparison between the allocation strategies with 4 machines per tool group}
% 	\label{tab:table03}
% %	\resizebox{15.5cm}{!}{%
% 		\begin{tabular}{|l%r
% 			cl||rr|rr|rr|rr|}
% 			\hline
% %			&                    &                      &  &  \multicolumn{8}{c}{\textbf{M = 12}} 
% %			\\ \hline
% 			& \textbf{M = 12}                   & %                     & 
% 			 & \multicolumn{2}{r|}{\textbf{Instance 09}}                 & \multicolumn{2}{r|}{\textbf{Instance 10}}                 & \multicolumn{2}{r|}{\textbf{Instance 11}}                 & \multicolumn{2}{r|}{\textbf{Instance 12}}                 \\
% 			& Size % \multicolumn{2}{l}{\textbf{Parameters}}            
% 			 &			 &			 Job                    & Step                   & Job                    & Step                   & Job                    & Step                   & Job                    & Step                   \\
% %			& Size               & setup % idx
% %			                  &  & 0                      & 1                      & 0                      & 1                      & 0                      & 1                      & 0                      & 1                      \\
% %			&                    & setup                &  &  &                        &                        &                        &                        &                        &                        &                                               \\
% 			\hline\hline
% 			\multirow{3}{*}{\textbf{Fixed}}    & \multirow{3}{*}{1} & % \multirow{3}{*}{0/1} &
% 			 Makespan                 & 525                    & 453                    & 525                    & 452                    & 525                    & 493                    & 643                    & 561                    \\
% 			&                    & %                     &
% 			 Setup/Batch              & 7/19                   & 3/20                   & 7/20                  & n/a                   & 6/22                   & 4/20                   & 4/22                   & n/a                   \\
% 			&                    & %                     &
% 			 1\ts{st}/2\ts{nd}-Call              & 124/5                 & TO & 25/17                 & TO & 25/53                 & TO/142 & TO & TO \\
% 			\midrule
% 			\multirow{9}{*}{\textbf{Flexible}} & \multirow{3}{*}{2} & % \multirow{9}{*}{0}   &
% 			 Makespan                 & \textbf{373}                    & 503                    & 491                    & 778                    & 569                    & 569                    & 765                    & 1673                   \\
% 			&                    & %                     &
% 			 Setup/Batch              & n/a                    & 6/17                    & n/a                   & n/a                    & n/a                    & n/a                   & n/a                    & 12/24                  \\
% 			&                    & %                     &
% 			 1\ts{st}/2\ts{nd}-Call              & TO & TO & TO & TO & TO & TO & TO & TO \\
% 			\cline{2-11}
% %			& & & & & & & & & & &   \\
% 			& \multirow{3}{*}{3} & %                     &
% 			 Makespan                 & 709                    & 688                    & 800                    & 907                    & 876                    & 876                    & 905                    & 1643                   \\
% 			&                    & %                     &
% 			 Setup/Batch              & 5/17                    & n/a                   & 3/18                   & 5/19                   & n/a                   & n/a                   & n/a                  & 15/24                    \\
% 			&                    & %                     &
% 			 1st/2nd              & TO & TO & TO & TO & TO & TO & TO & TO \\
% 			\cline{2-11}
% %			& & & & & & & & & & &   \\
% 			& \multirow{3}{*}{4} & %                     &
% 			 Makespan                 & n/a & -- & n/a & -- & n/a & -- & n/a & -- \\ %\multicolumn{8}{c|}{\multirow{3}{*}{Assignment issue}}                                                                                                                                                 \\
% 			&                    & %                     &
% 			 Setup/Batch              & n/a & -- & n/a & -- & n/a & -- & n/a & -- \\ %\multicolumn{8}{c|}{}                                                                                                                                                                                  \\
% 			&                    & %                     &
% 			 1\ts{st}/2\ts{nd}-Call              & TO & -- & TO & -- & TO & -- & TO & -- \\ %\multicolumn{8}{c|}{}                                                                                                                                                                                  \\
% 			\midrule
% 			\multirow{9}{*}{\textbf{Setup}}    & \multirow{3}{*}{2} & % \multirow{9}{*}{1}   &
% 			 Makespan                 & 401                    & 396                    & 419                    & \textbf{416}                    & \textbf{419}                    & \textbf{419}                    & \textbf{457}                    & 471                    \\
% 			&                    & %                     &
% 			 Setup/Batch              & 0/15                   & 0/14                   & 0/16                   & 0/16                   & n/a                   & n/a                   & 0/21                    & n/a                    \\
% 			&                    & %                     &
% 			 1\ts{st}/2\ts{nd}-Call              & TO & TO/92 & TO & TO & TO & TO & TO & TO \\
% 			\cline{2-11}
% %			& & & & & & & & & & &   \\
% 			& \multirow{3}{*}{3} & %                     &
% 			 Makespan                 & 706                    & 642                    & 792                    & 753                    & 942                    & 942                    & 939                    & 894                    \\
% 			&                    & %                     &
% 			 Setup/Batch              & 1/14                    & n/a                    & 2/16                    & n/a                   & n/a                   & n/a                    & n/a                    & 1/22                    \\
% 			&                    & %                     &
% 			 1\ts{st}/2\ts{nd}-Call              & TO & TO & TO & TO & TO & TO & TO & TO \\
% 			\cline{2-11}
% %			& & & & & & & & & & &   \\
% 			& \multirow{3}{*}{4} & %                     &
% 			 Makespan                 & 679                    & -- & 1725                    & -- & n/a                    & -- & n/a                    & -- \\
% 			&                    & %                     &
% 			 Setup/Batch              & n/a                   & -- & n/a                    & -- & n/a                   & -- & n/a                   & -- \\
% 			&                    & %                     &
% 			 1st/2nd              & TO & -- & TO & -- & TO & -- & TO & -- \\
% 			\hline
% 		\end{tabular}%
% %	}
% \end{table}
\section{Conclusion}
\label{sec:conc}
In this paper, we propose a fast leakage and variability-aware thermal simulation method that also captures the temperature dependence of conductivity. We derive a closed-form of the Green's function considering all these effects using novel insights and algebraic techniques. Our approach provides fast and accurate solutions for both the steady-state and the transient thermal profile and has been validated with a wide variety of test cases. As device dimensions continue to shrink, process variation has become a serious problem. The methods proposed in this work can equip designers to tackle this problem and may spawn further research in this area.


\section*{Acknowledgements}
Francesco Orabona is supported by the National Science Foundation under the grant no. 2046096 ``CAREER: Parameter-free Optimization Algorithms for Machine Learning''.



\bibliography{../../../../learning}
\bibliographystyle{icml2023}


\end{document}

