In this work, we have explored a novel methodology to analyze the perception of cycling safety using pairwise image comparisons. We explore and compare different popular covariate-free paired models to rate cycling environments according to individuals' perceptions, achieving good accuracies for the total number of comparisons. In addition, we explore how binary classification can be used to classify environments as being perceived as safe or unsafe directly from image characteristics. The results show this methodology's potential for widely comparing cycling environments and understanding how these environments impact individuals' perceptions of risk. Moreover, even with few comparisons, the information extracted is very relevant. This knowledge is critical as perceptions of safety significantly impact cycling adoption, potentially hindering any city's strategy to increase cycling numbers if safety perceptions are not encompassed. 

In the future, we plan to expand the work here started. One possible way forward is to use identifiable image characteristics (e.g., using image segmentation or object detection) as predictors to rate environments' perception of safety scores directly. In turn, this approach would improve scaling even further, as environment characteristics and their impacts on the perception of safety could be computed without further comparisons from individuals. A second approach can be using this same information in covariate-based ranking methods. Third, an analysis can be made if different typologies of individuals (i.e., Geller's cycling profiles) have different perceptions of safety, which can help cycling promotion strategies to more accurately and effectively target some populations' needs. 