Aimed at capturing individuals' perceptions of risk, we create a two-part survey. The first part aimed to collect information regarding the user's cycling profile and followed a slightly modified version survey on cyclists' typologies \cite{dill2013four}. Going forward, we focus solely on the second part, which employs pairwise image comparisons of cycling environments. Instituto Superior Técnico's Ethics Committee evaluated the survey, which we then deployed online. The survey took about 10-15 minutes to complete. 

\input{images/4.images_example}

%%%%%%%%%%%%%%%%%%%%%%%%%%%%%%%%%%%%%%%%%%%%%%%%%%%%%%%%%%%%%%%%%%%%%%%%%%%%%%%%
%% Pairwise image comparisons
%%%%%%%%%%%%%%%%%%%%%%%%%%%%%%%%%%%%%%%%%%%%%%%%%%%%%%%%%%%%%%%%%%%%%%%%%%%%%%%%
%\subsection{Pairwise image comparisons}
We repeatedly present respondents with two road environment pictures and ask them to select the one they perceive as safer for cycling (Figure \ref{fig:pairwise_survey}). We randomly sampled street-view images of road environments from Mapillary (https://www.mapillary.com/) from Berlin, Germany. 
The selected array of pictures captures a wide range of urban environments, including different infrastructure layouts, dedicated cycle lanes, urban characteristics, street furniture, vegetation, and varying degrees of other road users and pedestrians. 
We collected a set of 4481 total images across Berlin. Figure \ref{fig:images_example} shows some cycling environment images. To show pairs of pictures to respondents, we preprocess each image to extract key attributes about the depicted environment. We employ a partial factorial design, randomly selecting two photos with the same level of features, while others are free to vary, e.g., both images have the same level of vegetation, or both include a cycle lane. We ask respondents to complete 65 paired comparisons, but they can stop at an earlier number.

We collect responses from 251 users, averaging 3.25 comparisons per image and 29 comparisons per respondent. Of the respondents, 123 identified as males, 71 as females, with the remaining preferring not to disclose their gender. Agewise, 86 were between the ages of 18-30, 64 between 31-40, 30 between 41-50, 16 mentioned they were older than 51, and the remaining did not specify any age. Overall, individuals could be classified according to Geller's cycling profiles \cite{geller2006four} as No Way, No How (5.1\%); Interested, but Concerned (51.5\%); Enthused \& Confident (38.3\%); and Strong \& Fearless (4.1\%). 

