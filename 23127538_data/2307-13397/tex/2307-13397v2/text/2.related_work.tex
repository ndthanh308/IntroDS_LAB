%%%%%%%%%%%%%%%%%%%%%%%%%%%%%%%%%%%%%%%%%%%%%%%%%%%%%%%%%%%%%%%%%%%%%%%%%%%%%%%%
%% PERCEPTION OF CYCLING SAFETY
%%%%%%%%%%%%%%%%%%%%%%%%%%%%%%%%%%%%%%%%%%%%%%%%%%%%%%%%%%%%%%%%%%%%%%%%%%%%%%%%
\subsection{Subjective cycling safety}
Perceived or subjective safety relates to the feeling of safety of an individual, i.e., how individuals subjectively experience accident risk. Measuring this is vital for municipalities and decision-makers to make informed decisions and adequately provide cyclists with environments they feel safe to cycle in. 
Previous research has found many characteristics that relate to the sense of risk, such as cycling helmets and clothing \cite{aldred2015reframing}, sense of traffic \cite{sanders2015perceived}, urban roads and compliance with road rules \cite{lawson2013perception}, and infrastructure layout \cite{chataway2014safety}.
In effect, urban features can be indexed to measure perceived risk objectively. Indicators and scales, such as the Bicycle Stress Level \cite{sorton1994bicycle}, the Level of Traffic Stress \cite{mekuria2012low} or its updated form \cite{furth2017level}, help planners and researchers to compare contexts and analyze cycling environments.


%%%%%%%%%%%%%%%%%%%%%%%%%%%%%%%%%%%%%%%%%%%%%%%%%%%%%%%%%%%%%%%%%%%%%%%%%%%%%%%%
%% METHODS FOR STUDYING PERCEPTION OF CYCLING SAFETY
%%%%%%%%%%%%%%%%%%%%%%%%%%%%%%%%%%%%%%%%%%%%%%%%%%%%%%%%%%%%%%%%%%%%%%%%%%%%%%%%
The need to acquire such vital data has led researchers to employ qualitative \textit{in situ} or online surveys and interviews to understand what urban features may trigger or negatively arouse individuals \cite{sanders2015perceived, aldred2015reframing}. Naturalistic and semi-naturalistic approaches are often used. These approaches focus on more quantitative methods to capture human responses to risky environments, such as using physiological data using wearable sensors \cite{zeile2016urban}, showcasing cycling videos \cite{Parkin2007}, use of virtual reality \cite{von2018risk}, or eye tracking devices \cite{schmidt2018risk}. 
Yet, these approaches are often not scalable as they are time-consuming and resource-intensive, require precise preparation and monitoring of special devices, or may require individual training.

Recently, some methods have been proposed to counter this. For example, \cite{von2022safe} used a Likert-scale-based survey using 1900 images of cycling environments to generalize recommendations regarding best practices regarding subjectively safe cycling lanes. 
\cite{ito2021assessing} have used computer vision to index bikeability utilizing several automatically extracted features from street-view images (SVI) to compare Tokyo and Singapore. 
Likewise, although not applied to cycling safety, machine learning, and other data processing methodologies have explored how individuals perceive different environments, enabling faster, easier, and automatic evaluations for different perceptions \cite{Naik2014, dubey2016deep, ramirez2021measuring}.




%%%%%%%%%%%%%%%%%%%%%%%%%%%%%%%%%%%%%%%%%%%%%%%%%%%%%%%%%%%%%%%%%%%%%%%%%%%%%%%%
%% PAIRWISE COMPARISONS
%%%%%%%%%%%%%%%%%%%%%%%%%%%%%%%%%%%%%%%%%%%%%%%%%%%%%%%%%%%%%%%%%%%%%%%%%%%%%%%%
\subsection{Pairwise comparisons}
Pairwise comparison models aim to predict the outcome of comparing two items, i.e., when items $A$ and $B$ are compared, would a user prefer item $A$, item $B$, or would they be perceived equally (tie)? These models were first proposed in psychophysics and marketing research and have typically followed the seminal works of Thurstone \cite{thurstone1927law} and Bradley–Terry \cite{bradley1952rank}. In the past decades, paired comparison models have been explored and applied to many domains, including sports skill ranking and game prediction \cite{maystre2019pairwise, chau2023spectral}, image quality analysis \cite{xu2016pairwise}, and city perceptions \cite{Naik2014, costa2019citysafe}.


%%%%%%%%%%%%%%%%%%%%%%%%%%%%%%%%%%%%%%%%%%%%%%%%%%%%%%%%%%%%%%%%%%%%%%%%%%%%%%%%
%% RATING METHODS FROM PAIRWISE COMPARISONS
%%%%%%%%%%%%%%%%%%%%%%%%%%%%%%%%%%%%%%%%%%%%%%%%%%%%%%%%%%%%%%%%%%%%%%%%%%%%%%%%
Typical models assume that there is a latent score $s_i$ for each item $i$ and the outcome probability on a comparison between items $i$ and $j$ is a function of the difference between their scores, e.g., $\theta(s_i-s_j)$. Models' usual underlying goal is to estimate the latent scores $s_i$ from the data to obtain an interpretable and comparable score for each item. If $s_i > s_j$, a user would have a greater probability of picking item $i$. The function $\theta$ can have many forms but usually follows a Gaussian or logistic distribution initially proposed by Thurstone \cite{thurstone1927law} and Bradley–Terry \cite{bradley1952rank}, respectively.

Several methodologies have been proposed to extend comparison models, including iterative algorithms, Bayesian-based models, and covariate-based or covariate-free models. Covariate-based models often allow for new items to be added to the comparison set seamlessly without any prior comparison involving new items. Yet, these methods require having said covariates and do not rely entirely on the outputs of paired comparisons. For this work, we focus on covariate-free models requiring only results from pairwise comparisons.
For iterative algorithms, probably the most well-known case is the Elo rating \cite{elo1978rating}, which has been used to rank chess players by FIDE\footnote{https://ratings.fide.com/calc.phtml?page=change}, by FIFA to rank women's national football teams\footnote{https://www.fifa.com/fifa-world-ranking/procedure-women}, or by FiveThirtyEight to rank NFL teams\footnote{https://fivethirtyeight.com/methodology/how-our-nfl-predictions-work/}. Elo uses a simple online stochastic update rule based on an item's scores and the expected outcome of one item winning over the other. Despite its simplicity, Elo has remained one of the most used procedures since it is tractable and can easily adjust to diverse situations and scenarios. For Bayesian models, both Glicko \cite{glickman1999parameter} and TrueSkill \cite{herbrich2006trueskill} have been put forward as probabilistic methods that measure not only the latent scores $s_i$ but also the uncertainty associated with each score, which is often valuable.

More recently, other approaches have been suggested using alternative methodologies. These include spectral ranking that (usually) involves computing the pairwise comparison matrix leading eigenvalues and eigenvectors \cite{chau2023spectral}, convex problem formulation that usually penalizes wrongly or contradictory answers \cite{xu2016pairwise, costa2019citysafe}, or Gaussian processes to model different data dynamics \cite{maystre2019pairwise, chu2005preference}. 

In this work, we study paired comparison models to analyze cycling perception of safety. To the best of our knowledge, this has not been explored before and can potentially help researchers analyze the impact of the cycling environments on individuals' perceptions, enabling faster and continuous evaluations of such effects.
