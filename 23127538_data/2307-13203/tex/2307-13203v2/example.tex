As a concrete motivating example, consider the house shown in Figure~\ref{PlanWorld}.
Suppose that it is to be turned, via automation, into a `smart home' to serve
as an assisted living space for elderly persons. Assume that sensors can be
placed in each labelled contiguous area 
(\pool, \study, \bedroom, \bathroom, \kitchen, \ld, \byard, \fyard).

Suppose that it is lunchtime and the person it...

Consider, for example, the home whose architecture diagram, along with its graph representation is given in Figure~\ref{PlanWorld}.
It is easy to imagine constructing a world graph from it by replacing each of the undirected edges by two directed ones and placing an occupancy sensor in each area (a world state), as we wish, to satisfy the constraints we elaborate on as follows.
In this setup, we assume that a robot assists an elderly person to move around their house, and thus the agent consists of the robot-human pair.

It is almost lunch time and the agent asserts that they will move around the
house and then finally enter the dining room to eat and take essential
medication. Thus, we must be able to understand whether or not the agent
actually ended their walk in the dining room. Of course, in order to do this,
we only need to activate a single occupancy sensor in the dining room.
However, since the pollen is higher in the current season, we also specify a
desired behavior -- namely, that the agent never venture outside the house
\emph{and} also end their walk in the dining room. As it turns out, satisfying
the constraints in this type of problem requires the activation of two
additional sensors -- those in the front and back yards. This makes sense.
After all, we are now asking for more information about the how the agent moves
than before. Notice the 3 kinds of behavior that we can now discriminate
between --- ones that are both safe and desirable (staying in the house and
ending in the dining room), ones that are safe, but undesirable (going outside
the house, but ending in the dining room), and ones that are not safe (not
ending in the dining room).


Another salient feature that distinguishes our work, as mentioned previously, is the ability to consider privacy constraints. Even though 
minimizing the number of selected sensors decreases the amount of information obtained, it is not always clear 
that the information which \emph{can} be obtained satisfies particular requirements. Though it may initially sound counter-intuitive, satisfying specific privacy requirements may necessitate the activation of more sensors than is needed for discrimination alone. 
The point is made clear by another short example that follows.

Continuing the example with the robot which assists the elderly person, now assume that it is almost dinner time, and 
the robot is tasked with ensuring that the assisted human ends their walk in the dining room after spending some time in the kitchen and master bedroom.
The minimum sensor selection for this scenario quite simply involves the
activation of the occupancy sensors in the kitchen, master bedroom, and dining
rooms respectively.  However, now, the assisted human is concerned that third
parties without a need-to-know requirement may compromise their privacy by
finding out whether they are in the kitchen or master bedroom. Their concern is
not misplaced, given the increasing number of attacks on cloud services in
recent years~\cite{chou2013security} which the outside system might use to
store its data. 
Therefore, we additionally require that any walk the agent takes which ends in the kitchen be indistinguishable to the sensor system from that which ends in the master bedroom. When this requirement is added, it turns out that the minimum sensor selection now includes 5 occupancy sensors --- one each in the front and back yards, one each in the bathroom and study, and one in the dining room. 
Though simplistic, this example provides a strong foundation for the idea that it is not merely enough to reduce the number of sensors to increase privacy, but that sometimes it may be necessary to activate a different and higher cardinality combination of sensors to satisfy all constraints. 


