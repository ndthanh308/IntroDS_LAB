\newpage

\section{Additional Background and Notation}\label{sec:app-notations}
We stick to the following notations and their shorthands in the appendix.
\begin{table}[H]
\centering
{
    \setlength\doublerulesep{1pt}

\ifmain\begin{tabular}{p{3cm} p{11cm}}\fi
\ifneurips\begin{tabular}{p{4cm} p{8cm}}\fi
    \toprule[2pt]\midrule[0.5pt]
    Notations & Interpretations \\ \toprule[1.5pt]
    $u_h, r_h, P_h$ & the leader's and the follower's rewards and the transition kernel for the MDP.\\\midrule
    $M$, $M^*$, $\cM$ & a model for the leader's and follower's rewards and also the transition kernel. $M^*$ is the true model. The  model class is $\cM$.\\\midrule
    $\theta$, $\Theta$ & We use $\theta$ to denote the part of the model (follower's reward if myopic, and follower's reward and transition kernel if nonmyopic) that determines the follower's quantal response mapping $\pi\rightarrow\nu^\pi$. The model class is $\Theta$.
    \\\midrule
    $\pi, \nu^{\pi, M}, \nu^{\pi,\theta}$ & $\pi$ is the leader's policy, $\nu^{\pi,M}$ is the follower's response under $(\pi, M)$, $\nu^{\pi, \theta}$ has the same meaning as $\nu^{\pi,M}$.
    \\\midrule
    $J(\pi,M)$ & the total utility collected by the leader as the follower responses is $\nu^{\pi, M}$.\\\midrule
    $Q_h^{\pi, M}, V_h^{\pi, M}, A_h^{\pi, M}$, $U_h^{\pi,M}, W_h^{\pi,M}$ & (Q, V, A)-functions for the follower and the (U, W) functions for the leader under $(\pi, \nu^{\pi, M}, P^M, r^M, u^M)$ at step $h$.\\\midrule
    % $Q_h, V_h, A_h$ or $U_h, W_h$ & (Q, V, A)-functions for the follower/leader under some $\pi$ and the true model $M^*$ at step $h$, where $\pi$ is specified in the context.\\\midrule[.25pt]
    % $\tQ_h, \tV_h, \tA_h$ or $\tilde U_h, \tilde W_h$ & (Q, V, A)-functions for the follower/leader under some $\pi$ and an alternative model $\tilde M$ at step $h$, where $\pi$ is specified in the context.\\\midrule[.25pt]
    $\EE^{\pi, M}, \EE$ & the expectation $\EE^{\pi, M}$ is taken with respect to the trajectory generated by $(\pi,\nu^{\pi, M}, T^M)$, $\EE$ is a always a short hand for $\EE^{\pi, M^*}$.\\\midrule
    $\EE_{s_h, b_h}, \EE_{s_h}$ & $\EE_{s_h, b_h}$ is a short hand of $\EE^{\pi, M^*}\sbr{\cdot\given s_h, b_h}$ and $\EE_{s_h}$ is a short hand of $\EE^{\pi, M^*}\sbr{\cdot\given s_h}$.\\\midrule
    % $\nu_h, \tilde\nu_h$ & the follower's best response under $(\pi, M^*)$ and $(\pi, \tM)$, respectively, where $\pi$ is specified in the context.\\\midrule
    % $P_h,  \tilde P_h$ & the transition kernel/operator given by $M^*$ and $\tilde M$, respectively.\\\midrule
    $r_h^\pi(s_h, b_h)$ & $r_h^\pi = \inp[]{r_h(s_h, \cdot, b_h)}{\pi_h(\cdot\given s_h, b_h)}_\cA$ 
    % and the same for $ \tilde r$.
    \\\midrule
    $ P_h^\pi(s_{h+1}\given s_h, b_h)$ & $P_h^\pi = \inp[]{P_h(s_{h+1}\given s_h, \cdot, b_h)}{\pi_h(\cdot\given s_h, b_h)}_\cA$.\\\midrule
    $\pi^*$, $\pi^{*,M}$ & $\pi^*$ is the best policy for the leader, and $\pi^{*, M}$ is the best policy for the leader under model $M$, i.e. $\pi^{*,M} = \argmax_{\pi\in\Pi} J(\pi, M)$,\\\midrule
    $\Upsilon_h^\pi$ & the operator for the QRE defined in \eqref{eq:Upsilon}
    \\\midrule
    $T_h^{\pi,\nu}$, $T_h^{\pi}$, $T_h^{\pi, \theta}$, $T_h^{*}$, $T_h^{*, \theta}$ & $T_h^{\pi,\nu}$, $T_h^{\pi}$, $T_h^{\pi, \theta}$ are integral operators defined in \eqref{eq:operator_T_pi_nu}, \eqref{eq:operator_T_pi}, \eqref{eq:operator_T_pi_theta}, respectively. $T_h^*$ and $T_h^{*,\theta}$ are optimality integral operators defined in \eqref{eq:greedy} and \eqref{eq:optimistic-operator}, respectively.
    \\\midrule
    $\TT_h^{\pi}$, $\TT_h^{\pi,\theta}$, $\TT_h^*$, $\TT_h^{*,\theta}$ & 
    These are Bellman (optimality) operators defined in \eqref{eq:TT_pi}, \eqref{eq:TT_pi_theta}, \eqref{eq:bellman_opt_oper}, \eqref{eq:TT_*_theta}, respectively.\\\midrule
    $B_A$ & $B_A$ upper bounds the follower's Q-, V-, and A-functions and is specified in \eqref{eq:define_BA}.\\

    \bottomrule[2pt]
\end{tabular}
}
\caption{Table for notations.}\label{tab:notation}
\end{table}



% \begin{table}[h!]
% \centering
% {\setlength\doublerulesep{1pt}
% \begin{tabular}{p{4cm} p{8cm}}
%     \toprule[2pt]\midrule[0.5pt]
%     Notations & Interpretations \\ \toprule[1.5pt]
%         \\
%     \midrule\\

%     \bottomrule[2pt]
% \end{tabular}
% }
% \caption{Table for notations.}\label{tab:}
% \end{table}


% \newpage

% {\ifmain
% \todo{In the model based case, I think that the kernel MDP (in the general form) can also have small quantal eluder dimension. This is because given the offline data, we can already gurantee small TV distance in the transition kernel with respect to the \say{data distribution}, it doesn't matter what $V_{l+1}$ we plug in for the QRE. In fact, we always switch to $V_{l+1}^{\pi^i, M^t}$ in \eqref{eq:OnN-1st-g-sq} of \Cref{sec:proof-farsighted MDP} for putting a guarantee on $M^t$ with a cost of $\beta$.
% If we use kernel $\phi(s, a, b, s')$, we just imagine that $V_{l+1}^{\pi^t, M^t}$ is determined by $\pi^t$ (given that $\pi^t$ and $M^t$ is a pair). Therefore, we have $\sum_{i\in[t-1]}\la \sum_{s'}\PP^{\pi^i}_l(s, a, b, s')\cdot V_{l+1}^{\pi^i, M^i}(s'), \mu^t- \mu^*\ra^2 \le \beta$, which gives small error in $\la \sum_{s'}\PP^{\pi^t}_l(s, a, b, s')\cdot V_{l+1}^{\pi^t, M^t}(s'), \mu^t- \mu^*\ra$
% }
% This is \say{trick of switching}, because the data provides us with a stronger TV guarantee.

% \todo{Can we extend the model based algorithm to a model free version with general approximation version?
% The answer is yes, because we just introduce an additional second order error in the transition for each step, which amounts to no more than $\log(T)$.
% }\fi
% }



In the following, we introduce additional background knowledge that will be useful in the proofs. 
% We define the quantal Bellman operator $\BB_h^\pi:\cF(\cS\times\cB)\rightarrow \cF(\cS\times\cB)$ as
% \begin{align}
%     (\BB_h^\pi g)(s_h, b_h) = r_h^\pi(s_h, a_h) + \gamma \eta^{-1} \EE_{s_{h+1}\sim P_h^\pi(\cdot\given s_h, b_h)}\sbr{\log\rbr{\sum _{b'\in\cB} \exp\rbr{\eta g(s_{h+1}, b' )} }}. \label{eq:bellman_operator_follower} 
% \end{align}
Let $\Theta$ be the set of model parameters that determines the follower's quantal response model. 
In particular, when the follower is myopic, i.e., $\gamma = 0$, 
$\theta$ is the parameter of the follower's reward function.
In this case, we assume that there exists some $\theta^* \in \Theta$ such that $r_h = r_h^{\theta^*}$ for all $h \in [H]$. 
In addition, when the follower is farsighted with $\gamma >0$, we parameterize both the follower's reward function and transition kernel and write $\{ r^{\theta}, P^{\theta} \}_{\theta \in \Theta}$. 
In this case, we assume that there exists $\theta^* \in \Theta$ such that 
$r_h = r_h^{\theta^*}$ and $P_h = P_h^{\theta^*}$ for all $h \in [ H ] $. 
We note that each $\theta$ uniquely specifies a quantal response mapping according to \eqref{eq:quantal_response_policy}.
We let $\nu^{\pi, \theta} $ denote the quantal response of a  leader's policy $\pi$ under the model with parameter $\theta$. 

Moreover, recall that we denote the leader's state- and action-value function by $W$ and $U$ respectively. 
To simplify the notation, 
for any leader's policy $\pi$ and follower's policy $\nu$,  we define integral operators $ T_h^{\pi, \nu }, T_h ^{\pi} \colon  \cF(\cS\times\cA \times \cB)\rightarrow \cF(\cS ) $ by letting 
\#
\bigl ( T_h^{\pi , \nu }   U_{h} \bigr)(s_{h}) = \la U_{h}(s_{h}, \cdot, \cdot), \pi_{h}\otimes \nu_{h}  (\cdot, \cdot\given s_{h})\ra _{\cA\times \cB}, \label{eq:operator_T_pi_nu}\\
\bigl ( T_h^{\pi }   U_{h} \bigr)(s_{h}) = \la U_{h}(s_{h}, \cdot, \cdot), \pi_{h}\otimes \nu_{h}^{\pi} (\cdot, \cdot\given s_{h})\ra _{\cA\times \cB} ,\label{eq:operator_T_pi} 
\#
where $\nu^{\pi} $ is the quantal response of $\pi$ under the true model. 
Then the quantal  Bellman operator $\TT^{\pi}_h$ defined in \eqref{eq:bellman_operator_leader}  can be equivalently written as 
\begin{align}\label{eq:TT_pi}
\bigl( \TT_h^{\pi } f \bigr)  (s_h, a_h, b_h) = u_h(s_h, a_h, b_h) + \EE_{s_{h+1}\sim P_h(\cdot\given s_h, a_h, b_h)} \sbr{\bigl ( T_h^{\pi} f\bigr) (s_{h+1})}.
\end{align}
Similarly, for any $\theta \in \Theta$, we define $T^{\pi, \theta}_h $ by letting 
\begin{align}\label{eq:operator_T_pi_theta}
\big ( T_h^{\pi, \theta} U_{h} \bigr) (s_{h}) = \la U_{h}(s_{h}, \cdot, \cdot), \pi_{h}\otimes \nu_{h}^{\pi,\theta}(\cdot, \cdot\given s_{h})
\ra _{\cA\times \cB} . 
\end{align}
We similarly define a quantal Bellman operator $\TT_h^{\theta}$ by letting 
\begin{align}\label{eq:TT_pi_theta}
    \bigl( \TT_h^{\pi,\theta} f \bigr)  (s_h, a_h, b_h) = u_h(s_h, a_h, b_h) + \EE_{s_{h+1}\sim P_h(\cdot\given s_h, a_h, b_h)} \bigl[ \bigl ( T_h^{\pi,\theta} f\bigr) (s_{h+1})\bigr] .
\end{align}

{\noindent \bf Bellman Optimality  Equation in Myopic Case.}
Furthermore, recall that we let $\Pi = \{ \Pi_h \}_{h\in [ H]} $ denote the class of leader's policies. 
Specifically
 for the case where the follower is 
 myopic, we define an operator 
$T_h^*  \colon \cF(\cS\times\cA\times\cB)\rightarrow \cF(\cS)$ as 
\begin{align}
   \bigl( T_{h}^*  f\bigr) (s_h) = \max_{\pi_h\in\Pi_h} \dotp{f(s_h,\cdot,\cdot)}{\pi_h\otimes \nu^{\pi}_h(\cdot, \cdot\given s_h)}_{\cA\times \cB }. \label{eq:greedy}
\end{align}
In other words, \eqref{eq:greedy} can be regarded as finding the  ``greedy'' policy of the leader, assuming leader's reward function is $f$ and the follower is myopic. 
Based on $T_h^*$, we define the Bellman optimality operator $\TT_h^*\colon \cF(\cS\times\cA\times\cB)\rightarrow \cF(\cS\times\cA\times\cB)
$ 
by letting 
\begin{align}
    \bigl ( \TT_h^{*} f \bigr )  (s_h, a_h, b_h) = u_h(s_h, a_h, b_h) + \EE_{s_{h+1}\sim P_h(\cdot\given s_h, a_h, b_h)} \bigl[ \bigl ( T_{h+1}^{*} f\bigr) (s_{h+1}) \bigr] ,\label{eq:bellman_opt_oper}
\end{align}
Note that we let $\pi^*$  denote the optimal policy of the leader.
Let $W^*$ and $U^*$ denote $W^{\pi^*}$ and $U^{\pi^*}$ respectively, which are the value functions of the leader at QSE. 
Using the Bellman operator defined in \eqref{eq:bellman_opt_oper}, we obtain the \emph{Bellman optimality equation} for the leader: 
\#\label{eq:bellman_opt_eqn}
U_h^* (s,a,b) = \bigl ( \TT_h^* U_{h+1}^* \bigr )(s,a,b), \qquad 
W_h^*(s) =  \bigl( T_{h}^*  U_h^* \bigr) (s),\qquad \forall (s,a,b,h) ,
\# 
where we stick to the convention that $U_{H+1} ^* = \mathbf{0}$. 

The key message conveyed in the  Bellman equation \eqref{eq:bellman_opt_eqn}  is that, when the model is known and the follower is myopic, the optimal policy of the follower can be computed via dynamic programming, which is similar to the case of an MDP.  
However, such a benign property cannot be extended to the farsighted case where $\gamma >0 $. 
The main reason for such dichotomy is that, in the myopic case, for each time step $h$, the quantal response $\nu_h^{\pi}$ only depends on $\pi$ through $\pi_h$. 
Thus, whenIn this case, 
we can regard the leader's problem with an auxiliary MDP with the action space at step $h$ being $\Pi_h $. 
The reward function $\tilde r_h$ and the transition kernel $\tilde P_h$ of such an auxiliary MDP are given by  
\$
\tilde r_h (s_h, \pi_h) & =  \la u_h(s_h, \cdot, \cdot), \pi_h \otimes \nu_h^{\pi}  (\cdot, \cdot \given s_h) \ra _{\cS \times \cA }, \notag \\
\tilde P_h (s_{h+1} \given s_h , \pi_h ) & =  \la  (s_{h+1}  \given s_h, \cdot, \cdot ) , \pi_h \otimes \nu_h^{\pi}  (\cdot, \cdot \given s_h) \ra _{\cS \times \cA },  
\$
where $\pi_h \in \Pi_h$ is an action,   and $u_h$ is the reward of the leader and $P_h$ is the transition kernel of the leader-follower Markov game. 
The optimal policy $\pi^*$ of the leader is exactly the optimal policy of the auxiliary MDP, and thus can be found via dynamic programming. 
In contrast, for a farsighted follower,
for each timestep $h$, 
the quantal response policy $\nu_h^{\pi}$ 
depends on $\pi$ through $\{ \pi_\ell  \}_{\ell \geq h }.$
As a result, the quantal response is a complicated mapping of the leader's policy, which prohibits dynamic programming.\footnote{The  \emph{feedback Stackelberg equilibrium} for leader-follower games with farsighted followers can be solved via dynamic programming. See, e.g., \cite{bacsar1998dynamic} for details. Our notion of Stackelberg equilibrium corresponds to the global Stackelberg equilibrium \citep{bacsar1998dynamic}, which does not admits a dynamic programming formulation in general.} 

Finally, we can also define  the Bellman optimality operator when the follower is critic and has reward $r^{\theta} $. 
In this case, similar to $T_h^*$ defined in \eqref{eq:greedy}, 
we define an operator $T_h^{* , \theta}$   by letting 
\begin{align} \label{eq:optimistic-operator}
\bigl(  T_{h}^{*, \theta } f\bigr) (s_h) = \max_{\pi_h\in\Pi_h} \big \la f(s_h,\cdot,\cdot), \pi_h\otimes \nu^{\pi, \theta}_h (\cdot, \cdot\given s_h)\bigr \ra _{\cA\times \cB },
\end{align}
where $\nu^{\theta} _h $ is the quantal response based on  reward $r^{\theta}_h $.  
Based on $T_{h}^{*, \theta} $, we define  
the  Bellman optimality operator for the leader 
$\TT_h^{*, \theta}:\cF(\cS\times\cA\times\cB)\rightarrow \cF(\cS\times\cA\times\cB)$ by letting 
\begin{align}\label{eq:TT_*_theta}
    \bigl ( \TT_h^{*, \theta} f \bigr )  (s_h, a_h, b_h) = u_h(s_h, a_h, b_h) + \EE_{s_{h+1}\sim P_h(\cdot\given s_h, a_h, b_h)} \bigl[ \bigl ( T_{h+1}^{*, \theta}f\bigr) (s_{h+1}) \bigr] ,
\end{align}
The Bellman optimality equation corresponding to $\TT^{*, \theta}$ can be established similar to \eqref{eq:bellman_opt_eqn}. 


\vspace{5pt} 
{\noindent \bf Additional Notation.} 
In the sequel, we let $B_A$ be the global upper bound for the follower's advantage function. We derive an explicit form for $B_A$ in the sequel.
Specifically, for the follower's true advantage functions, we have $\nbr{A_h^\pi}_\infty = \nbr{Q_h^\pi - V_h^\pi}_\infty$.
Here, an upper bound for $V_h^\pi$ can be derived by
\begin{align*}
    \abr{V_h^\pi(s_h)}  = \biggabr{\eta^{-1} \log\biggrbr{\sum_{b_h\in\cB} \exp\rbr{\eta Q_h^\pi(s_h, b_h)}} } \le \eta^{-1} \log\abr{\cB} + \nbr{Q_h^\pi(s_h,\cdot)}_\infty,
\end{align*}
where $\abr{\cB}$ is the number of follower's actions if $\cB$ is discrete, and $\abr{\cB}$ is the length of $\cB$ on the real line for continuous action case. 
Moreover, in the continuous action case, we can always normalize $\cB$ to the unit interval on the real line, i.e., $\cB=[0,1]$, which helps us get rid of the $\eta^{-1}\log|\cB|$ term.  
In the sequel, we just keep this term in the upper bound and remind the readers what $\abr{\cB}$ stands for here.
Therefore, we have that 
\begin{align*}
    \nbr{Q_h^\pi}_\infty \le \nbr{r_h^\pi}_\infty  + \gamma \nbr{V_{h+1}^\pi}_\infty\le \rbr{\gamma \eta^{-1} \log |\cB| + 1} + \gamma \nbr{Q_{h+1}^\pi}_\infty.
\end{align*}
By a recursive argument, we have for the Q function that 
\begin{align}
    \nbr{Q_h^\pi}_\infty \le \eff_{H-h+1}(\gamma)\cdot \rbr{\gamma \eta^{-1}\log\abr{\cB}+1} \eqdef B_Q, \quad \forall h\in[H], \pi\in\Pi, \label{eq:def B_Q}
\end{align}
where we define $\eff_h(\gamma) = (1-\gamma^h)/(1-\gamma)$ as the effective horizon for the follower truncated for $h$ steps and $B_Q$ as the upper bound for the follower's Q function.
Therefore, for the follower's advantage function, we have 
\begin{align*}
    \nbr{A_h^\pi}_\infty &= \nbr{Q_h^\pi - V_h^\pi}_\infty \nend
    &= \max_{(s_h, b_h)\in\cS\times\cB}\Bigabr{Q_h^\pi(s_h, b_h) -  \max_{\nu'\in\Delta(\cB)}\rbr{\dotp{Q_h^\pi(s_h,\cdot)}{\nu'(\cdot)} + \eta^{-1}\cH(\nu')}
    }\nend
    & \le \nbr{Q_h^\pi}_\infty + \eta^{-1} \log\abr{\cB}\nend
    &\le \rbr{1+\eff_{H-h+1}(\gamma) }\cdot  \rbr{\eta^{-1}\log\abr{\cB}+1}. 
\end{align*}
Therefore, it suffices to set 
\#\label{eq:define_BA}
B_A = \rbr{1+\eff_{H}(\gamma) }\cdot  \bigrbr{\eta^{-1}\log\abr{\cB}+1}.
\# 
Note that this $B_A$ also boundes $Q_h$ and $V_h$ by our derivation. Hence, we also denote by $B_A$ the upper bounds for the Q and the V functions for the follower.
\vspace{5pt}

\subsection{Function Classes and Covering Number}\label{sec:covering number}
We define several function class with their corresponding covering numbers that will be used in establishing our learning guarantees.

\paragraph{General Model Class $\cM$.}

we consider $\cM$ to be the model class where each $M\in\cM$ uniquely specifies 
the  environment of a Markov game for both the leader and the follower. 
Specifically, for any $M\in\cM$, with slight abuse of notation, we write  $M=(u^M, r^M, P^M)$, which  specifies the leader's and the follower's reward as well as the transition kernel. We consider the covering number of $\cM$ with respect to the following distance:
\begin{align*}
    \varrho(M,\tilde M)\defeq \max_{h\in[H], \atop 
    (s_h, a_h, b_h)\in\cS\times\cA\times\cB}
    \cbr{\bignbr{u_h- \tilde u_h}_\infty, \bignbr{r_h- \tilde r_h}_\infty, \bignbr{P_h(\cdot\given s_h, a_h, b_h)-\tilde P_h(\cdot\given s_h, a_h, b_h)}_1}, 
\end{align*}
where $(r, u, P)$ is a short hand of $(r^M, u^M, P^M)$ and $(\tilde r, \tilde u, \tilde P)$ is a short hand of $(r^{\tilde M}, u^{\tilde M}, P^{\tilde M})$.
With  this definition, we have the following proposition that bounds the error betweeen policies and value functions in terms of $\varrho (\cdot, \cdot)$.
\begin{proposition} \label{prop:error-prop}
    For any two $M,\tilde M\in\cM$ such that $\rho(M, \tilde M)\le \epsilon$, and two policy $\pi, \tilde\pi$ such that $\bignbr{\pi_h - \tilde\pi_h}_1\le \epsilon$, 
    we let $(U, W, Q, V,   \nu)$ be the value functions and quantal response associated with $\pi$  under $M$,  , and let $(\tilde U, \tilde W, \tilde Q, \tilde V,   \tilde \nu)$ be corresponding terms associated with $\tilde \pi$ under $\tilde M$. 
    Then,  
    for all  $h\in[H]$ and $ (s_h, a_h, b_h)\in\cS\times\cA\times\cB$, we have 
    \begin{align*}
    \bignbr{Q_h - \tilde Q_h}_\infty &\le 2\epsilon\cdot (1+\gamma B_A) \cdot \eff_{H}(\gamma), \nend
    %%%%%%%%%%%%%%
    \bignbr{U_h - \tilde U_h}_\infty &\le \epsilon H \cdot \bigrbr{4\eta H (1+\gamma B_A) \cdot \eff_H(\gamma) + 1 + 2H}, \nend
    %%%%%%%%%%%%%%
    \bignbr{V_h - \tilde V_h }_\infty 
    &\le \bignbr{Q_h - \tilde Q_h}_\infty \le 2\epsilon\cdot (1+\gamma B_A) \cdot \eff_{H}(\gamma), \nend
    %%%%%%%%%%%%%
    \bignbr{W_h - \tilde W_h}_\infty &\le \epsilon (H+1) \cdot \bigrbr{4\eta H (1+\gamma B_A)\cdot \eff_H(\gamma) + 1 + 2H}. 
    \end{align*}
    Meanwhile, 
   for the quantal response, 
    \begin{align*}
        D_\H^2(\tilde \nu_h(\cdot\given s_h), \nu_h(\cdot\given s_h)) \le \nbr{\nu_h(\cdot\given s_h) - \tilde \nu_h(\cdot\given s_h)}_1 \le 8 \eta \epsilon \cdot (1+\gamma B_A)  \cdot \eff_{H}(\gamma).
    \end{align*}
\end{proposition}
\begin{proof}
    See Appendix \ref{proof:prop:error-prop} for a detailed proof.
\end{proof}
Therefore, we see it suffices to control $\varrho(\cdot, \cdot)$ in order for the value functions for both the follower and the leader to be under control. Moreover, the quantal response is also under control.
Therefore, we define a new distance $\rho$ for $\cM$ as 
\begin{equation}\label{eq:rho-cM}
    \rho(M,\tilde M)\defeq 6 \!\!\!\!\!\!\!\!\!\!\!\!\max_{\pi\in\Pi, h\in[H], \atop 
    (s_h, a_h, b_h)\in\cS\times\cA\times\cB}\!\!
    \cbr{\begin{aligned}
        &\bignbr{u_h- \tilde u_h}_\infty, \bignbr{r_h- \tilde r_h}_\infty, D_\H\orbr{P_h(\cdot\given s_h, a_h, b_h), \tilde P_h(\cdot\given s_h, a_h, b_h)}\\
        & \bignbr{Q_h^\pi - \tilde Q_h^\pi}_\infty, \bignbr{U_h^\pi - \tilde U_h^\pi}_\infty, D_\H(\tilde \nu_h^\pi(\cdot\given s_h), \nu_h^\pi(\cdot\given s_h)) 
    \end{aligned}}
\end{equation}
And we denote by $\cN(\cM)=\cN_\rho(\cM, T^{-1})$ the covering number for model class $\cM$. Note that $\cN_\rho(\cM,\epsilon)$ is related to $\cN_\varrho(\cM, \epsilon)$ only by a change of $\epsilon$ according to \Cref{prop:error-prop} where we just take $\pi=\tilde\pi$.



\paragraph{Response Model Class $\Theta$.}
Let $\Theta$ be the set of model parameters that determines the follower's quantal response model. 
In particular, when the follower is myopic, i.e., $\gamma = 0$, 
$\theta$ is the parameter of the follower's reward function.
In this case, we assume that there exists some $\theta^* \in \Theta$ such that $r_h = r_h^{\theta^*}$ for all $h \in [H]$. 
In addition, when the follower is farsighted with $\gamma >0$, we parameterize both the follower's reward function and transition kernel and write $\{ r^{\theta}, P^{\theta} \}_{\theta \in \Theta}$. 
In this case, we assume that there exists $\theta^* \in \Theta$ such that 
$r_h = r_h^{\theta^*}$ and $P_h = P_h^{\theta^*}$ for all $h \in [ H ] $. 
We note that each $\theta$ uniquely specifies a quantal response mapping according to \eqref{eq:quantal_response_policy}.
% We let $\nu^{\pi, \theta} $ denote the quantal response of a  leader's policy $\pi$ under the model with parameter $\theta$. 
We consider distance $\rho$ for model $\Theta$ as
\begin{equation}\label{eq:rho-Theta}
    \rho(\theta,\tilde\theta)\defeq \max_{\pi\in\Pi, h\in[H], \atop 
    (s_h, a_h, b_h)\in\cS\times\cA\times\cB}
    \cbr{\begin{aligned}
        &\bignbr{r_h- \tilde r_h}_\infty, \bignbr{P_h(\cdot\given s_h, a_h, b_h) - \tilde P_h(\cdot\given s_h, a_h, b_h)}_1\\
        & \bignbr{Q_h^\pi - \tilde Q_h^\pi}_\infty, D_\H(\tilde \nu_h^\pi(\cdot\given s_h), \nu_h^\pi(\cdot\given s_h)) 
    \end{aligned}}, 
\end{equation}
where $(r, P, Q, \nu)$ is given under $\theta$ and $(\tilde r, \tilde P, \tilde Q, \tilde \nu)$ is given under $\tilde \theta$.
We note that $\theta$ is just a subclass of $\cM$. We denote by $\cN_\rho(\Theta, \epsilon)$ the covering number of $\Theta$ with respect to this $\rho$.

Notably, when the follower is myopic, we just need to cover each $\Theta_h$, where $\Theta_h$ only contains the parameters for $r_h^\theta$, with respect to the following distance 
\begin{align}\label{eq:rho-Theta_h}
    \rho(\theta_h,\tilde\theta_h) \defeq \max_{s_h\in\cS} \cbr{(1+\eta)\|r_h-\tilde r_h\|_\infty, D_\H(\tilde \nu_h^\pi(\cdot\given s_h), \nu_h^\pi(\cdot\given s_h)) }.
\end{align}
Here, the additional $(1+\eta)$ term is needed by \Cref{lem:MLE-formal}.
A covering number for $\Theta_h$ can be thus denoted by $\cN_\rho(\Theta_h, \epsilon)$.
Only for the myopic case, we denote by 
\begin{align}\label{eq:cN-Theta-myopic}
    \cN_\rho(\Theta, \epsilon) = \max_{h\in[H]} \cN_\rho(\Theta_h, \epsilon).
\end{align}
For the nonmyopic case, this covering number $\cN_\rho(\Theta, \epsilon)$ is with respect to the distance defined in \eqref{eq:rho-Theta}.

We calculate this covering number in \eqref{eq:cN-Theta-myopic} for a given step $h\in[H]$ where the follower's reward parameter space is $\Theta_h$ in the linear myopic case.
Specifically, we don't need to consider the transition for myopic follower, and the Q function is simply the reward function. Therefore, we just need to bound $\|r_h-\tilde r_h\|_\infty\le \min \cbr{\epsilon^2 /(8 \eta), \epsilon /(1+\eta)} $, and by \Cref{prop:error-prop}, we have $D_\H(\tilde \nu_h^\pi(\cdot\given s_h), \nu_h^\pi(\cdot\given s_h)) <\epsilon$. Therefore, a covering number for $\Theta_h$ is given by 
\begin{align}\label{eq:cN-Theta_h}
    \log \cN_\rho(\Theta_h, \epsilon) \le d \log\rbr{1+\frac{2 B_\Theta B_\phi}{\min \cbr{\epsilon^2 /(8 \eta), \epsilon /(1+\eta)}}} \lesssim d\log\rbr{1+ \eta /\epsilon^2 + (1+\eta)/\epsilon}.
\end{align}
where $B_\Theta$ bounds the $\Theta_h$ class in the 2 norm and $B_\phi$ bounds the feature mapping $\phi$ in the 2 norm for each $(s_h, a_h, b_h)$. The covering number for $\Theta$ is just $H\cN_\rho(\Theta_h, \epsilon)$.

We remark that $\cM$ is strictly larger than $\Theta$ in the sense that $\cM$ also contains the leader's reward.
We introduce $\cM$ for model-based learning for the leader in \Cref{sec:farsighted}, and $\Theta$ is used for learning the quantal response mapping via model-based maximum likelihood estimation as we have discussed before.


\paragraph{Leader's Value Function Class  $\cU$.}
For both online and offline learning the leader's value function with general function approximation and myopic follower, we introduce function class $\cU:\cS\times\cA\times\cB\rightarrow \RR$, which we assume is uniformly bounded by $H$ and the following completeness and realizability assumption holds. 
\begin{condition}\label{cond:real-comp}
We say that $\cU$ satisfies the realizability and the completeness conditions if the followings hold, 
\begin{itemize}
    \item[(i)] (\textit{Realizability}) There exists $\theta^*\in\Theta$ such that $r_h^{\theta^*}=r_h$ for any $h\in[H]$. For any $\pi\in\Pi, \theta\in\Theta$, there exists $U\in\cU$ such that $U_h = \TT_h^{\pi,\theta} U_{h+1}$ for any $h\in[H]$;
    \item[(ii)] (\textit{Completeness}) For any $U\in\cU$, $\pi\in\Pi$, $\theta\in\Theta$, and $h\in[H]$, there exists $U'\in\cU$ such that $U'=\TT_h^{\pi,\theta} U_{h+1}$. 
\end{itemize}
\end{condition}
We consider the covering number of $\cU$ with respect to $\rho(U, \tilde U) = \onbr{U-\tilde U}_\infty$. Specifically, we denote by $\cN(\cU)=\cN_\rho(\cU, T^{-1})$ the covering number of $\cU$. In the following, we characterize two joint function classes including $\cU$ and $\Theta$ that will be used under the general function approximation setting with myopic follower. 


\paragraph{Joint Class $\cY$.}
For studying the offline case with myopic follower using general function approximation, we have a joint class $\cY_h = \Theta_{h+1}\times \Pi_{h+1}\times \cU^2$, where $\Theta_h$ only contains the follower's reward at step $h$. 
This class is used for studying the suboptimality of the MLE-BCP in \Cref{alg:MLE-BCP}, where $\cU^2$ is used to approximate the leader's value function $U_h^{\pi, \theta}, U_{h+1}^{\pi,\theta}$, $\Theta_{h+1}$ is class for the follower's reward at step $h+1$, and $\Pi_{h+1}$ is the leader's policy class at step $h+1$.
We consider the following distance, 
% \Zhuoran{Explain the meaning of $y$}
% \Siyu{to be redefined because we need to cover the squared loss of $r$ and the squared loss of $U$. We should include a coefficient before the supremum.}
  \begin{equation}\label{eq:rho-cY}
    \rho(y, \tilde y) = \max_{s_{h+1}\in\cS}\cbr{\begin{aligned}
        &\bignbr{U_h-\tilde U_h}_\infty,  \bignbr{U_{h+1}-\tilde U_{h+1}}_\infty, \\
        &\sup_{s_{h+1}\in\cS}\bignbr{(\pi_{h+1}\otimes \nu_{h+1}^{\pi, \theta}-\tilde \pi_{h+1}\otimes \nu_{h+1}^{\tilde\pi, \tilde\theta})(\cdot, \cdot\given s_{h+1})}_1
    \end{aligned}}.
\end{equation}
We let $\cN(\cY)=\cN_\rho(\cY, T^{-1})$. Define $\cN_\rho(\Pi, \epsilon)$ as the $\epsilon$-covering number of the policy class $\Pi$ with respect to distance $\rho(\pi_h, \tilde\pi_h) = \max_{(s_h, b_h)\in\cS\times\cB} \nbr{(\pi_h - \tilde\pi_h)(\cdot\given s_h, b_h)}_1$.
It is easy to see that $\cN_\rho(\cY_h, \epsilon)\le \cN_\rho(\cU,\epsilon)\cdot \cN_\rho(\Pi_{h+1}, \epsilon')\cdot\cN_\rho(\Theta_{h+1}, \epsilon') $ following the result in \Cref{prop:error-prop}, where for the myopic follower, 
$$\epsilon' = \frac{\epsilon}{8\eta (1+\gamma B_A)\eff_H(\gamma)} = \frac{\epsilon}{8\eta}.$$
In particular, we let 
\begin{align}\label{eq:cN-cY}
    \cN_\rho(\cY, \epsilon) = \max_{h\in[H]} \cN_\rho(\cY_h, \epsilon).
\end{align}

\paragraph{Joint Class $\cZ$. }
For the online myopic setting with general function approximation $\cU$, we consider a joint function class $\cZ_h=\cU^2\times\Theta_{h+1}$. 
This class is used for studying the regret of the MLE-GOLF in \Cref{alg:MLE-GOLF}, where $\cU^2$ is used to approximate the leader's optimistic value function $U_h^{*,\theta}, U_{h+1}^{*,\theta}$, $\Theta_{h+1}$ is class for the follower's reward at step $h+1$.
We consider the following distance for $z, \tilde z\in\cZ_h$, 
    \begin{align}\label{eq:rho-cZ}
        &\rho\orbr{z, \tilde z}  = \max_{h\in[H]}\cbr{\bignbr{U_h-\tilde U_h}_\infty, \bignbr{ T_{h+1}^{*,\theta} U_{h+1} (\cdot) -  T_{h+1}^{*,\tilde\theta} \tilde U_{h+1} (\cdot)}_\infty }, 
    \end{align} 
    where the optimistic operator $T_{h+1}^{*, \theta}$ is defined in \eqref{eq:optimistic-operator}.
    We denote by $\cN(\cZ_h)=\cN_\rho(\cZ_h, T^{-1})$ the covering number of the smallest $T^{-1}$-covering net for $\cZ_h$ with respect to this distance $\rho(\cdot,\cdot)$. One should notice that for any $\bignbr{U_{h+1} - \tilde U_{h+1}}_\infty\le \epsilon$, $\onbr{r_{h+1}^\theta - r_{h+1}^{\tilde \theta}}_\infty \le \epsilon$, we have
    \$
    &\bignbr{ T_{h+1}^{*,\theta} U_{h+1} (\cdot) -  T_{h+1}^{*,\tilde\theta} \tilde U_{h+1} (\cdot)}_\infty \nend
    &\quad \le \max_{\pi\in\Pi}\cbr{\bignbr{T_{h+1}^{\pi,\theta} U_{h+1} (\cdot) -  T_{h+1}^{\pi,\tilde\theta} \tilde U_{h+1} (\cdot)}_\infty, \bignbr{T_{h+1}^{\tilde\pi,\theta} U_{h+1} (\cdot) -  T_{h+1}^{\tilde\pi,\tilde\theta} \tilde U_{h+1} (\cdot)}_\infty}\nend
    &\quad \le \max_{\pi\in\Pi} \bignbr{W_{h+1}^\pi - \tilde W_{h+1}^\pi}_\infty \nend
    &\quad \le \epsilon (H+1) \rbr{4\eta H + 1 + 2H}
    \$
    for the myopic case, 
    where we let $\pi$ be the optimal policy under $\theta, U$ and $\tilde\pi$ be the optimal policy under $\tilde \theta, \tilde U$. Here, the last inequality comes from \Cref{prop:error-prop} with $\lambda=0$.
    Therefore, $\cN_\rho(\cZ_h, \epsilon)\le \cN_\rho(\cU, \epsilon) \cdot \cN_\rho(\cU, \epsilon') \cdot \cN_\rho(\Theta_{h+1}, \epsilon')$, where 
    $$\epsilon' = \frac{\epsilon}{(H+1)(4\eta H +1 + 2H)}. $$

This definition will be used in \Cref{lem:MLE} for the farsighted follower case where we directly incorporate MLE algorithm using $\cM$.
In particular, we let
\begin{align}\label{eq:cN-cZ}
    \cN_\rho(\cZ, \epsilon) = \max_{h\in[H]} \cN_\rho(\cZ_h, \epsilon). 
\end{align}


% Remind general function approximation. 
% $\Theta$, $\Pi$, $\cU$. 
% $N_{\mathrm{cov}}$ stands for the covering number of $\Theta$, $\Pi$, $\cU$.
% Covering number 
% $d_{\Theta}$, $d_{\pi}$, $d_{\cU}$ stand for log-covering number of $\Theta$ in terms of what norm, with error $1/T$. 

\subsection{Eluder Dimension}\label{sec:eluder dimension}
we present the definition of (distributional) eluder dimension 
% and a follow-up lemma on bounding the cumulative error with the (distributional) eluder dimension 
that will be useful for our online learning purpose.

\begin{definition}[Eluder Dimension]\label{def:Eluder dimension}
    Let $\cG$ be a function class defined on $\cX$.
	The  eluder dimension $\dim_\E(\cG,\epsilon)$ is the length of the longest sequence $\{x_1, \ldots, x_n\} \subset \cX$ such that there exists $\epsilon'\ge\epsilon$ where for all $m \in [n]$, there exists $g_m\in\cG$ such that
    \begin{align*}
        \sum_{i=1}^{m-1} \rbr{g_m(x_i)}^2 \le \epsilon', \quad \abr{g_m(x_m)} > \epsilon'.
    \end{align*}
\end{definition}
We can similarly define an eluder dimension for signed measure.
\begin{definition}[Eluder Dimension for Signed Measures]
    \label{def:DE}
    Let $\cG$ be a function class defined on measurable space $\cX$, and $\sP$ be a family of signed measures over $\cX$. Suppose that any $g\in\cG$ is integrable with respect to any $\rho\in\sP$.
	The eluder dimension for signed measures with respect to $\cG$ and $\sP$ is denoted by $\dim_\DE(\cG,\sP,\epsilon)$, which is the length of the longest sequence $\{\rho_1, \ldots, \rho_n\} \subset \sP$ satisfying the following condition:  
    there exists $\epsilon'\ge\epsilon$ where for any $m\in[n]$, there exists $g_m\in\cG$ such that  
    \begin{align*}
        \sum_{i=1}^{m-1} \rbr{\int_\cX g_m \rd \rho_i}^2 \le \epsilon', \quad \abr{\int_\cX  g_m \rd\rho_m} > \epsilon'.
    \end{align*}
\end{definition}
% For any function class $\cG$ that has a finite distributional Eluder dimension over probability measures $\sP$, we have the following lemma.


In the sequel, we will denote by $\dim(\cG)=\dim_\DE(\cG, \sP, T^{-1/2})$ the eluder dimension with respect to the class of signed measures $\sP$. Note that the standard distributional eluder dimension is a special case for the eluder dimension for signed measures.

\subsubsection{Eluder Dimensions in Myopic Case}\label{sec:eluder myopic}
We discuss the function classes and their corresponding eluder dimensions (with respect to signed measures) that characterize the hardness of leader's exploration problem in the face of a myopic  follower. 
Such a complexity measure will be used for analyzing the regret of the MLE-GOLF algorithm.
Detailed proofs of the regret upper bound in terms of  these eluder dimensions  are available in \Cref{sec:proof-Online-MG}.
\paragraph{Eluder Dimension for Leader's Bellman Error.}
As is shown in the proof \Cref{sec:proof-Online-MG}, the leader's Bellman error we are dealing with is 
\begin{align*}
    \sum_{t=1}^T \EE^{\hat\pi^t} [(\hat U_h^t - \TT_h^{*, \hat\theta^t} \hat U_{h+1}^t)(s_h, a_h, b_h)], 
\end{align*}
where $\hat U^t, \hat\theta^t$ are just the optimistic estimators obtained at episode $t$.
We define a class of functions that corresponds to this error
\begin{align*}
    \cG_L=\cbr{g:\Pi\times\cU^2\times\Theta\rightarrow \RR: g=\EE^\pi [U_h-\TT_h^{*,\theta} U_{h+1}]}. 
\end{align*}
For the leader's Bellman error, we have the following configurations:
\begin{itemize}[leftmargin=20pt]
    \item[(i)] Define function class $\cG_{h,L} = \bigcbr{g:\cS\times\cA\times\cB\rightarrow \RR\given  g = U_h-\TT_h^{*, \theta} U_{h+1}, \exists U\in\cU, \theta\in\Theta}$. 
    Moreover, consider sequence $\{g_h^i = \hat U_h^i - \TT_h^{*, \hat\theta^i} \hat U_{h+1}^i\}_{i\in[T]}$, where $\hat\theta^i, \hat U^i$ are the estimated $\theta$, $U$ at episode $i$.  
    It is obvious that $g_h^i\in\cG_{h,L}$;
    \item[(ii)] Define the class of probability measures as 
    $$\sP_{h,L} = \cbr{\rho\in\Delta(\cS\times\cA\times\cB)\given\rho(\cdot)=\PP^\pi((s_h, a_h, b_h)=\cdot), \pi\in\Pi}.$$
    Moreover, consider sequence $\{\rho^i(\cdot) = \PP^{\hat\pi^i}((s_h, a_h, b_h)=\cdot)\}_{i\in[T]}$, where $\hat\pi^i$ is the optimistic policy used at episode $i$;
\end{itemize}
Under these definitions, we let low rank property for $\cG_L$ that $\dim(\cG_L) = \max_{h\in[H]}\dim_\DE\rbr{\cG_{h, L}, \sP_{h, L}, T^{-1/2}}$.
This configuration works because for the chosen sequences, we have
\begin{align*}
    \EE_{\rho_h^i}[g_h^t] = \EE^{i} \sbr{(\hat U_h^t - \TT_h^{*, \hat\theta^t} \hat U_{h+1}^t)(s_h,a_h,b_h)}.
\end{align*}
If $i=t$, this will be the leader's Bellman error we aim to bound.
Moreover, the online guarantee will be
$$\sum_{i=1}^{t-1} \rbr{\EE_{\rho^i}\sbr{g_h^t}}^2 = \sum_{i=1}^{t-1}\EE^{i}[(\hat U_{h}^{t} - \TT_{h}^{*, \hat\theta^t} \hat U_{h + 1}^t)^2]\lesssim H^2\beta$$ for any $t\in[T]$, which enables us to use \Cref{lem:de-regret}.
In particular, for the 
linear Markov Game in \Cref{def:linear MDP}, we have $\dim(\cG_L)\lesssim d$ as is discussed in \citet{jin2021bellman}.

\paragraph{Eluder Dimension for Follower's Quantal Response Error.}
As is shown in the proof of \Cref{sec:proof-Online-MG}, 
the follower's quantal response error is characterized by the following term, 
\begin{align*}
    \sum_{t=1}^T \bigrbr{\Upsilon_h^{\hat\pi^t} (r_h^{\hat\theta^t}-r_h)}(s_h^t,b_h^t)= \sum_{t=1}^T\QRE(s_h^t, b_h^t;\hat\theta^t, \hat\pi^t).
\end{align*}
We define 
\begin{align*}
    \cG_F = \cbr{g:\cS\times\cB\times\Pi\times\Theta\rightarrow \RR: g=\Upsilon_h^{\pi}(r_h^\theta - r_h)(s_h, b_h)},
\end{align*}
where we  recall the linear operator $\Upsilon_h^\pi:\cF(\cS\times\cA\times\cB)\rightarrow \cF(\cS\times\cB)$ defined as 
\begin{align*}
    \rbr{\Upsilon_h^\pi f}(s_h, b_h) = \dotp{\pi_h(\cdot\given s_h, b_h)}{f(s_h, \cdot, b_h)} - \dotp{\pi_h\otimes \nu_h^{\pi}(\cdot,\cdot\given s_h)}{f(s_h,\cdot,\cdot)}. 
\end{align*}
Here, $(\Upsilon_h ^{\pi} f ) (s_h,b_h) $ quantifies  how far $b_h$ is from being the quantal response of $\pi$ at state $s_h$, measured in terms of  $f$. One can also think of $(\Upsilon_h^\pi f)(s_h, b_h)$ as the \say{advantage} of the reward induced by action $b_h$ compared to the reward induced by the quantal response. 
% It suffices to use $\Upsilon_h^\pi $ instead for a myopic follower where $\alpha\in\sA$ is a state-wise prescription.
We have the following configurations for the follower's quantal response error:
\begin{itemize}[leftmargin=20pt]
    \item[(i)] We define function class on $\cS\times\cA\times\cB$ as,
    \begin{align*}
    \cG_{h, F} = \cbr{g:\cS\times\cA\times\cB\rightarrow \RR: \exists \theta\in\Theta, g= {r_h^\theta - r_h}}.
    \end{align*}
    In addition, we consider a sequence $\{g_h^i = r_h^{\hat\theta^i} - r_h\}_{i\in[T]}$. 
    \item[(ii)] We define a class of signed measures on $\cS\times\cA\times\cB$ as
    \begin{align*}
        \sP_{h, F} &= \{\rho(\cdot) = \PP^\pi(a_h=\cdot\given s_h, b_h)\delta_{(s_h, b_h)}(\cdot) 
        - \PP^\pi((a_h, b_h)=\cdot\given s_h)\delta_{(s_h)}(\cdot)\nend
        &\qqquad \biggiven \pi\in\Pi, (s_h, b_h)\in\cS\times\cB\}, 
    \end{align*}
    where $\delta_{(s_h, b_h)}(\cdot)$ is the measure that assigns measure $1$ to state-action pair $(s_h, b_h)$. 
    In addition, we consider a sequence 
    $$\cbr{\rho_h^i(\cdot) = \PP^{\hat\pi^i}(a_h=\cdot\given s_h, b_h)\delta_{(s_h, b_h)}(\cdot) 
    - \PP^{\hat\pi^i}((a_h, b_h)=\cdot\given s_h)\delta_{(s_h)}(\cdot)}_{i\in[T]}.$$
\end{itemize}
Under these definitions, we let $\dim(\cG_F) = \max_{h\in[H]}\dim_\DE\rbr{\cG_{h, F}, \sP_{h, F}, T^{-1/2}}$.
For simplicity, we denote by $g_h^t(s_h^i, b_h^i, \pi^i)$ the integral of $g_h^t$ with respect to the signed measure $\rho_h^i$ since $\rho_h^i$ is uniquely determined by $(s_h^i, b_h^i, \pi^i)$. 
    It is easy to check that $$g_h^t(s_h^i, b_h^i, \pi^i) = \bigrbr{\Upsilon_h^{\pi^i}(r_h^{\hat\theta^t}-r_h)}(s_h^i, b_h^i) = \QRE(s_h^i, b_h^i; \hat\theta^t, \pi^i).$$
This configurations work because when we take $t=i$, $g_h^t(s_h^t, b_h^t, \pi^t)$ will be exactly the quantal response error, and the online guarantee for these two sequences are 
    \begin{align*}
        \sum_{i=1}^{t-1} (g_h^t(s_h^i, b_h^i, \pi^i))^2 \lesssim C_\eta^2\beta, 
    \end{align*}
by our MLE guarantee, which enables us to use \Cref{lem:de-regret}.
Moreover, we remark that under the linear Markov game setting, i.e., $r_h(s_h, a_h,b_h) = \la \phi_h(s_h, a_h,b_h), \theta\ra_\cB$, we simply have $g_h^t(s_h^i, b_h^i, \pi^i) = \la (\Upsilon_h^{\pi^i}\phi_h)(s_h^i, b_h^i), \hat\theta_h^{t} - \theta_h^*\ra_\cB$, which has a bilinear form and we simply have for the eluder dimension $\dim(\cG_{F})\lesssim d$.
% or $\dim(\cG)=\dim_\E(\cG, T^{-1/2})$ for a standard Eluder dimension.

\subsubsection{Eluder Dimensions in Farsighted Case}
\label{sec:eluder farsighted}
We discuss the function classes and their corresponding eluder dimensions (with respect to signed measures) that characterize the hardness of leader's exploration problem in the face of a farsighted follower. 
Such a complexity measure will be used for analyzing the regret of the OMLE algorithm. 
Detailed proof using the eluder dimensions of these classes are available in \Cref{sec:proof-farsighted MDP}.

\paragraph{Eluder Dimension for the Leader's Bellman Error.}
As is shown in \Cref{sec:proof-farsighted MDP}, the leader's Bellman error is 
\begin{align*}
    \sum_{t=1}^T \EE^{\pi^t} \sbr{\bigrbr{\tilde U_h^t - u_h}(s_h, a_h, b_h) - \tilde W_{h+1}^t(s_{h+1})}, 
\end{align*}
where $\tilde U^t=U^{\pi^t, M^t}$, $\tilde W^t=W^{\pi^t, M^t}$, and $M^t,\pi^t$ are the optimistic model estimator and optimistic policy obtained at episode $t$. 
We define the following error class
\begin{align*}
    \cG_L = \cbr{\cM\times\Pi\rightarrow\RR: \EE^\pi\sbr{\bigrbr{U_h^{\pi^M, M} - u_h}(s_h, a_h, b_h) - W_{h+1}^{\pi^M,M}(s_{h+1})}}, 
\end{align*}
where $\pi^M=\argmax_{\pi\in\Pi} J(\pi, M)$ is the optimistic policy corresponding to model $M$.
To formally define the eluder dimension for $\cG_L$, we consider the following configurations for step $h\in[H]$.
\begin{itemize}[leftmargin=20pt]
    \item[(i)] Define function class $\cG_{h,L}$ as
    \begin{align*}
        \cG_{h, L} &= \Big\{g:\cS\times\cA\times\cB\rightarrow \RR \Biggiven g={\bigrbr{U_h^{\pi^{\tilde M},\tilde M} - u - P_h W_{h+1}^{\pi^{\tilde M},\tilde M}}(s_h, a_h, b_h)}, \exists \tilde M\in\cM\Big\}, 
    \end{align*}
    where we define $\pi^M = \argmax_{\pi\in\Pi}J(\pi, M)$.
    Specifically, the expectation is taken under $\pi$ and the true model. 
    Consider a sequence of function $\{g_h^i = (\tilde U_h^{i} - u - P_h \tilde W_{h+1}^i)\}_{i\in[T]}$. We have $g_h^i\in\cG_{h, L}$ since we define $\tilde U^i = U^{\pi^i, M^i}$ and we have by the optimism in the algorithm that $\pi^i = \pi^{M^i}$. The same also holds for $\tilde W^i$.
    \item[(ii)] Define a class of probability measures over $\cS\times\cA\times\cB$ as $$\sP_{h, L}=\{\PP^\pi((s_h, a_h, b_h)=\cdot), \forall \pi\in\Pi\}.$$
    Consider a sequence of probability measures $\{\rho_h^i(\cdot)=\PP^{\pi^i}((s_h, a_h, b_h)=\cdot)\}_{i\in[T]}$, where $\pi^i$ is the policy used at episode $i$.
    \item[(iii)] Under these two sequences, we denote by $g_h^t(\pi^i) = \EE_{\rho_h^i}[g_h^t]$ for simplicity. 
    We have 
    $$g_h^t(\pi^i) =\EE^{\pi^i}\bigsbr{{\bigrbr{\tilde U_h^t - u - P_h \tilde W_{h+1}^t}(s_h,a_h,b_h)}}, $$
    which should be bounded by $3H$.
\end{itemize}
We thus denote by $\dim(\cG_L) = \max_{h\in[H]}\dim_\DE(\cG_{h, L}, \sP_{h, L}, T^{-1/2})$. This is actually the distributional eluder dimension of $\cG_{h, L}$, which satisfies $\dim(\cG_L)\lesssim d$ for the linear Markov game defined in \Cref{def:linear MDP} because for any $g\in\cG_{h, L}$ and $\rho\in\sP_{h, L}$, the integral of $g$ with respect to $\rho$ always admits a bilinear form $\EE_\rho[g]= \la \EE_\rho[\phi_h(s_h, a_h, b_h)], \zeta^g\ra$. 

%  or the linear kernel MDP \citep{zhou2021provably}.
% , low rank MDP \citep{agarwal2020flambe}.

\paragraph{Eluder Dimension for the  First-Order Error of the Follower's Quantal Response.}
As we will show in the proof \Cref{sec:proof-farsighted MDP}, the first order term in the quantal response error is just
\begin{align*}
    \sum_{h=1}^H\sum_{t=1}^T |\tilde\Delta_h^{1, t}(s_h^t, b_h^t)|,
\end{align*}
where $\tilde\Delta_h^{1, t}$ is defined as
\begin{align*}
    \tilde \Delta^{(1, t)}_h(s_h, b_h) &=  \rbr{\EE_{s_h, b_h}^t -\EE_{s_h}^t}\Biggsbr{\sum_{l=h}^H \gamma^{l-h}{\rbr{\tilde Q_l^t - r_l^{\pi^t} - \gamma P_l^{\pi^t} \tilde V_{l+1}^t}(s_l, b_l)}}, 
\end{align*} 
Here, $\tilde r_l^t = r_l^{M^t}$, $\tilde P_l^t = P_l^{M^t}$, $\tilde V^t = V^{\pi^t, M^t}$, and $\tilde Q^t=Q^{\pi^t, M^t}$. $M^t$ is the estimated model and $\pi^t$ is the optimistic policy at episode $t$. In particular, $\pi^t = \argmax_{\pi\in\Pi}J(\pi, M^t) \eqdef \pi^{M^t}$.
We define 
\begin{align*}
    \cG_F^1 &= \bigg\{\Pi\times\cM\times\cS\times\cB\rightarrow \RR: \nend
    &\qqquad \ts
    (\EE_{s_h, b_h}^\pi - \EE_{s_h}^\pi)\Bigsbr{\sum_{l=h}^H \gamma^{l-h}\bigrbr{r_l^M- r_l + \gamma (P_l^{M} - P_l) V_{l+1}^{\pi^M, M}}(s_l, a_l, b_l)}, \forall h\in[H]\bigg\} . 
\end{align*}
We consider the following configurations for the definition of the eluder dimension of $\cG_F^1$. 
% {\red Check here}
\begin{itemize}[leftmargin=20pt]
    \item[(i)] Define function class $\cG_{h, F}^1$ as 
    \begin{align*}
        \cG_{h, F}^1 &= \Bigg\{g:(\cS\times\cA\times\cB)^{H-h+1}\rightarrow \RR \bigggiven \exists M\in\cM, \nend
        &\qqquad g((s_l, a_l, b_l)_{l=h}^H) = {\sum_{l=h}^H \gamma^{l-h}\bigrbr{r_l^M- r_l + \gamma (P_l^{M} - P_l) V_{l+1}^{\pi^M, M}}(s_l, a_l, b_l)}
        \Bigg\},
    \end{align*}
    where we remind the readers that $\pi^M = \argmax_{\pi\in\Pi}J(\pi, M)$ only depends on $M$. Consider 
    sequences 
    $$\cbr{g_h^t=\sum_{l=h}^H \gamma^{l-h}\orbr{\tilde r_l^t- r_l + \gamma (\tilde P_l^t - P_l) \tilde V_{l+1}^t}}_{t\in[T]}. $$
    It is clear that $g_h^t\in\cG_{h, F}^1$ since $\tilde V_h^t = V_h^{\pi^t, M^t}$ and $\pi^t = \argmax_{\pi\in\Pi}J(\pi, M^t) = \pi^{M^t}$.
    % $r_h^{\pi}(s_h, b_h) = \dotp{r_h(s_h, \cdot, b_h)}{\pi_h(\cdot\given s_h, b_h)}_\cA$ and the same holds for $P_h^\pi$. 
    % One can see immediately that $\cG_F^1$ is a bilinear class, where the expectation part depends on $\pi$ and the follower's Bellman error part depends on $M$.
    
    \item [(ii)] Define a class of signed measures over $(\cS\times\cA\times\cB)^{H-h+1}$ as 
    \begin{equation*}
        \sP_{h, F}^1 = \cbr{\begin{aligned}
            &\PP^\pi(((s_l, a_l, b_l)_{l=h+1}^H , a_h)=\cdot \given s_h, b_h)\delta_{(s_h, b_h)}(\cdot) \nend
            &\quad - \PP^\pi(((s_l, a_l, b_l)_{l=h+1}^H , a_h, b_h)=\cdot \given s_h)\delta_{(s_h)}(\cdot) 
        \end{aligned}
        \bigggiven \pi\in\Pi, (s_h, b_h)\in\cS\times\cB}, 
    \end{equation*}
    where $\delta_{s_h, b_h}$ is the measure that puts measure $1$ on a single state-action pair $(s_h, b_h)$, and the conditional is well defined by the Markov property. Also, consider the following sequence, 
    \begin{align*}
        \Big\{\rho_h^t(\cdot)&=\PP^{\pi^t}(((s_l, a_l, b_l)_{l=h+1}^H , a_h)=\cdot \given s_h^t, b_h^t)\delta_{(s_h^t, b_h^t)}(\cdot) \nend
        &\qquad - \PP^{\pi^t}(((s_l, a_l, b_l)_{l=h+1}^H , a_h, b_h)=\cdot \given s_h^t)\delta_{(s_h^t)}(\cdot)\Big\}_{t\in[T]},
    \end{align*}
    and we also have $\rho_h^t\in\sP_{h, F}^1$.
    \item   [(iii)] 
    In particular, we define $g_h^t(s_h^i, b_h^i, \pi^i)$ as the integral of $g_h^t$ with respect to $\rho_h^i$, which is given by
    $$g_h^t(s_h^i, b_h^i, \pi^i)= \rbr{\EE_{s_h^i, b_h^i}^{\pi^i}-\EE_{s_h^i}^{\pi^i}} \sbr{\sum_{l=h}^H \gamma^{l-h}\Bigrbr{\tilde r_l^t- r_l + \gamma (\tilde P_l^t - P_l) \tilde V_{l+1}^t}(s_l, a_l, b_l)},$$
    % One can check that $g_h^t\in\cG_F^1$ 
    Note that the sequence of signed measures is uniquely determined by $\{(s_h^t, b_h^t, \pi^t)\}_{t\in[T]}$. Moreover, we have $g_h^t(s_h^i, b_h^i, \pi^i)$ bounded by $\eff_H(\gamma)(2\nbr{r_h}_\infty + 2\nbr{V_{h+1}}_\infty) \le 4B_A \eff_H(\gamma)$, where the definition of $B_A$ is available in \eqref{eq:define_BA};  
\end{itemize}
We define the maximal eluder dimension of $\cG_{h, F}^1$ with respect to $\sP_{h,F}^1$ as $$\dim(\cG_F^1) =\max_{h\in[H]} \dim_\DE(\cG_{h, F}^1,\sP_{h, F}^1, T^{-1/2}). $$
We remark that the linear Markov game have eluder dimension $\dim(\cG_{h, F}^2)\lesssim Hd$.
Note that for each $l\in\{h, \dots, H\}$ in the expression of $g_h^t(s_h^i, b_h^i, \pi^i)$, we have 
\begin{align*}
    &\rbr{\EE_{s_h^i, b_h^i}^{\pi^i}-\EE_{s_h^i}^{\pi^i}} \sbr{\sum_{l=h}^H \gamma^{l-h}\Bigrbr{\tilde r_l^t- r_l + \gamma (\tilde P_l^t - P_l) \tilde V_{l+1}^t}(s_l, a_l, b_l)}\nend
    &\quad  = \sum_{l=h}^H \gamma^{l-h} \Bigdotp{\bigrbr{\EE_{s_h^i, b_h^i}^{\pi^i}-\EE_{s_h^i}^{\pi^i}}\sbr{\phi_l(s_l,a_l,b_l)}}{\hat\theta_h^t - \theta_h + \sum_{s_{l+1}}(\hat\mu_l^t - \mu_l)(s_{l+1})\tilde V_{l+1}^t(s_{l+1})}, 
\end{align*}
which admits a bilinear form. 
Since we can stack $H$ vectors in the summation together for both sides of bilinear form, which creates a bilinear form with dimension $Hd$, we claim that $\dim(\cG_{F}^1)\lesssim Hd$ in this case.


\paragraph{Eluder Dimension for the Second-Order Error of the Follower's Quantal Response. }
According to \Cref{sec:proof-farsighted MDP}, the second order term in the quantal response error is just 
\begin{align*}
    \sum_{t=1}^T \EE^t\sbr{ \rbr{\rbr{\tilde Q_h^t - r_h^{\pi^t} - \gamma P_h^{\pi^t} \tilde V_{h+1}^t}(s_h, b_h)}^2}.
\end{align*}
We denote by $\cG_{F}^2$ the class of functions corresponding to this second-order QRE, 
\begin{align*}
    \cG_F^2=\bigcbr{\EE_{s_h, b_h}^{\pi} \osbr{\orbr{r_h^M- r_h + \gamma (P_h^{M} - P_h) V_{l+1}^{*, M}}(s_h, a_h, b_h)}}, 
\end{align*}
for all $\pi\in\Pi, M\in\cM, (s_h, a_h, b_h)\in\cS\times\cA\times\cB$ and $h\in[H]$.
We consider the following configurations.
\begin{itemize}[leftmargin =20pt]
    \item[(i)] We take the same function class $\cG_F^2$ as
    \begin{align*}
        \cG_{h, F}^2 &= \Bigg\{g:\cS\times\cA\times\cB\rightarrow \RR: \exists M\in\cM, h\in[H] \nend
        &\qqquad g(s_h, b_h, a_h) = {\bigrbr{r_h^M- r_h + \gamma (P_h^{M} - P_h) V_{l+1}^{\pi^M, M}}(s_h, a_h, b_h)}
        \Bigg\},
    \end{align*}
    where $\cG_F^2$ is bounded by $4B_A$. 
    \item[(ii)] We define a class of probability measures on $\cS\times\cA\times\cB$ as $$\sP_{h, F}^2 = \cbr{\PP^\pi(a_h=\cdot\given s_h, b_h)\delta_{(s_h, b_h)}(\cdot)\given \pi\in\Pi, (s_h,b_h)\in\cS\times\cB}, $$
    where $\delta_{(s_h,b_h)}(\cdot)$ is the measure that assigns $1$ to the state-action pair $(s_h, b_h)$. 
    \item[(iii)] We take a sequence of functions $\{g_h^t\}_{t\in[T]}$ as $\{g_h^t = \tilde r_h^{t}- r_h + \gamma (\tilde P_h^t - P_h) \tilde V_{l+1}^{t}\}_{t\in[T]}$, and take a sequence of probability measures as $\{\rho_h^t(\cdot) = \PP^{\pi^t}(a_h=\cdot\given s_h^t, b_h^t)\delta_{(s_h^t, b_h^t)}(\cdot)\}_{t\in[T]}$, where we define $\tilde r_h^t = r_h^{M^t}$ and $\tilde P_h^t = P_h^{M^t}$.
    One can check that $g_h^t\in\cG_{h,F}^1$ since $\tilde V_h^t = V_h^{\pi^t, M^t}$ and we have $\pi^t = \argmax_{\pi\in\Pi}J(\pi, M^t) = \pi^{M^t}$. In addition, we define $g_h^t(s_h^i, b_h^i, \pi^i)$ as the integral of $g_h^t$ with respect to $\rho_h^i$, which is given by
    \begin{align*}
        g_h^t (s_h^i, b_h^i, \pi^i) = \EE_{s_h^i, b_h^i}^{\pi^i} \sbr{\bigrbr{\tilde r_h^{t}- r_h + \gamma (\tilde P_h^t - P_h) \tilde V_{l+1}^{t}}(s_h, a_h, b_h)},
    \end{align*}
    Note that the sequence of probability measures is uniquely determined by $\{(s_h^t, b_h^t, \pi^t)\}_{t\in[T]}$.
\end{itemize}
We let $\dim(\cG_F^2)=\max_{h\in[H]}\dim_\DE(\cG_{h,F}^2,\sP_{h, F}^2, T^{-1/2})$ be the eluder dimension.
Similar to the previous discussion, we also have a bilinear for $g_h^t(s_h^i, b_h^i, \pi^i)$ under the linear Markov game. Hence, $\dim(\cG_{h, F}^2)\lesssim d$.

We remark that 
for the linear matrix MDP \citep{zhou2021provably}, we just have $P(s_{h+1}|s_h, a_h, b_h) = \psi_h(s_h, a_h, b_h)^\top W_h \varphi_h(s')$ for some unknown $\RR^{d\times d}$ matrix $W$. 
By viewing $W_h\varphi_h(s')=\mu_h(s')$, our discussions hold for the linear matrix MDP as well.
For the linear mixture MDP \citep{chen2022unified}, the rewards and the transition kernel share the same parameter $\theta_h$, but the transition kernel is given by $P_h(s_{h+1}\given s_h,a_h, b_h)=\la \psi_h(s_{h+1}, s_h, a_h, b_h), \theta_h\ra$. 
We remark that we can instead take 
$$g_h^t(\pi^i) =\EE^{\pi^i}\bigsbr{{\bigrbr{(u_h^{M^t} - u_h) + (P_h^{M^t} - P_h) \tilde W_{h+1}^i}(s_h,a_h,b_h)}}, $$ 
for the leader's Bellman error, 
$$g_h^t(s_h^i, b_h^i, \pi^i)= \rbr{\EE_{s_h^i, b_h^i}^{\pi^i}-\EE_{s_h^i}^{\pi^i}} \sbr{\sum_{l=h}^H \gamma^{l-h}\Bigrbr{\tilde r_l^t- r_l + \gamma (\tilde P_l^t - P_l) \tilde V_{l+1}^i}(s_l, a_l, b_l)},$$
for the first order quantal response error, and 
\begin{align*}
    g_h^t (s_h^i, b_h^i, \pi^i) = \EE_{s_h^i, b_h^i}^{\pi^i} \sbr{\bigrbr{\tilde r_h^{t}- r_h + \gamma (\tilde P_h^t - P_h) \tilde V_{l+1}^{i}}(s_h, a_h, b_h)},
\end{align*}
for the second order quantal response error, these errors still admit a low rank factorization. In fact, the online guarantees of these three errors are already implied by \eqref{eq:OnN-guarantee-Bellmanerror}, \eqref{eq:OnN-1st-g-sq}, and \eqref{eq:OnN-guarantee-2ndQRE}, where a change of the V- or the W-functions does not matter because we have strong guarantee on the TV distance for the transition kernel, and the change of V- or W-functions only introduces a $\cO(\beta)$ terms in the online guarantee which is caused by the squared TV distance for the transition kernel.
% In this work, we focus on the Eluder dimension of the following function classes.

% \Zhuoran{Complete}
% \begin{itemize}
%     \item[(i)] function class $$\cG_L = \cbr{U_h-\TT_h^{\theta} U_{h+1}: U\in\cU, \theta\in\Theta}$$ 
%     defined on $\cX=\cS\times\cA\times\cB$, which is bounded by $B_{\cG_L}\le 2B_U$. We consider the class of probability measures $\sP_L = \cbr{\rho\in\Delta(\cS\times\cA\times\cB): \rho(\cdot)=\PP^\pi((s_h, a_h, b_h)=\cdot)}$ with $\rho^i(\cdot) = \PP^{\hat\pi^i}((s_h, a_h, b_h)=\cdot)$;
% \end{itemize}




{\ifneurips 
\input{neurips/algorithm_detail_neurips.tex}
\fi}
% $\mathrm{dim}_{\cG}$




% Eluder dimension and covering number. 



\section{More Details of Technical Ingredients}\label{sec:app-major-tech}

In this section, we summarize and provide  proofs of the  important techniques used for analyzing the QSE. The following is a table of addition constants used in this section. 
\begin{table}[h!]
    \centering
    {\setlength\doublerulesep{1pt}
    \begin{tabular}{p{2cm}|p{10.5cm}}
        \toprule[2pt]\midrule[0.5pt]
        Notations & Interpretations \\ \toprule[1.5pt]
        $C^{(0)}$   & $C^{(0)}=2\eta H$ \\\midrule
        $C^{(1)}$ & $C^{(1)}=\eta^2 H   \bigrbr{1+ 4 \eff_H(\gamma)}\cdot \exp(2\eta B_A)$ \\\midrule
        $C^{(2)}$ & $
           C^{(2)}   =  2  \eta^2 H^2\cdot \exp\orbr{6\eta B_A} \cdot(1+4 \eff_H(\gamma)) \cdot \rbr{\eff_H(\exp(2\eta B_A)\gamma)}^2$ 
        \\\midrule
        $C^{(3)}$ & $C^{(3)}={\eta^2 \exp(2\eta B_A)}\bigrbr{2+\eta B_A \cdot  \exp\rbr{2\eta B_A}}/2$
        \\\midrule
        $L^{(1)}$ & $L^{(1)} = 6(\eta^{-1}+2 B_A)\cdot {\eff_H\rbr{\gamma}}$
        \\\midrule 
        $L^{(2)}$ & $L^{(2)} = c H^2 \eff_H(c_2)^2 \kappa^2 \exp\rbr{8\eta B_A} (\eta^{-1}+B_A)^2$, where 
        $c_2 = \gamma(2\exp(2\eta B_A)+\kappa\exp(4\eta B_A))$, and $c$ is a universal constant.\\
        \bottomrule[2pt]
    \end{tabular}
    }
    \caption{Constants used for \Cref{sec:app-major-tech}}\label{tab:}
\end{table}

\subsection{Performance Difference Lemma for for QSE}\label{sec:app-subopt-decompose}

In this subsection, we further elaborate on the performance difference lemma introduced in \S\ref{sec:subopt decomposition}. 
For generality, we consider the farsighted case.  
In the following,  
we consider a fixed policy $\pi$ and let its quantal response under the true model be $\nu^{\pi}$. 
Let $\tilde \nu$ be an estimate of $\nu^{\pi}$ and let $\tilde U$ and $\tilde W$ be any estimates of  $U^{\pi} $ and $  W^{\pi}$, which are defined respectively in    \eqref{eq:U_function} and \eqref{eq:W_function}.
We note that $\tilde W$ and $\tilde U$ not necessarily satisfy $
\tilde W_h(s ) = \la \tilde U_h (s, \cdot , \cdot ) , \pi _h \otimes \tilde \nu_h^{\pi} (\cdot , \cdot \given s) \ra . 
$.
We present a slightly more general version of   the performance difference lemma, which directly implies \eqref{eq:performance diff-1}.

% {\main Lemma \ref{lem:performance diff informal}\fi} {\neurips \eqref{eq:performance diff-1}\fi}. 


\begin{lemma}[Performance Difference]\label{lem:subopt-decomposition} 
For any fixed policy $\pi$, 
let $\tilde \nu$ be an estimate of the quantal response $\nu^{\pi}$ and let $\tilde U$ and $\tilde W$ be estimates of $U^\pi$ and $W^{\pi}$ respectively. 
Based on $\tilde U$ and $\tilde W$, we can estimate $J(\pi)$ defined in \eqref{eq:J} by 
\$
\EE_{s_1\sim\rho_0}\bigl [ \orbr{T_1^{\pi,\tilde\nu}\tilde U_1} (s_1 )\bigr ]  \qquad \textrm{and} \qquad \EE_{s_1\sim\rho_0}\bigl [ \tilde W_1(s_1)\bigr]  , 
\$
where we operator $T_h^{\pi,\tilde\nu}$ is defined in \eqref{eq:operator_T_pi_nu}.
The error or these estimators can be bounded as follows: 
\begin{align}
    &\EE_{s_1\sim\rho_0}\bigsbr{ \orbr{T_1^{\pi,\tilde\nu}\tilde U_1}(s_1 )} - J(\pi ) \nend
    %%%%%%
    % &\quad = {\sum_{h=1}^H \EE\sbr{\bigrbr{\tilde U_h - u_h}(s_h, a_h, b_h)- \tilde W_{h+1}(s_{h+1})}} + \sum_{h=1}^H \EE\sbr{\tilde W_h(s_h)-\tilde U_h(s_h, a_h, b_h)}\nend
    %%%%%%
    &\quad\le\underbrace{{\sum_{h=1}^H \EE\Bigsbr{\orbr{\tilde U_h - u_h}(s_h, a_h, b_h)-  \orbr{T_{h+1}^{\pi,\tilde\nu} \tilde U_{h+1}}(s_{h+1})}}}_{\dr \text{Leader's Bellman error}}+ \underbrace{\sum_{h=1}^H   H  \cdot \EE \bigsbr{ \nbr{\rbr{\tilde \nu_h-\nu_h^{\pi }}(\cdot\given s_h)}_1} }_{\dr \text{Quantal response error}}\label{eq:perform-diff-general}, \\
    & \EE_{s_1\sim\rho_0}\bigsbr{\tilde W_1(s_1)}  - J(\pi)\nend
    %%%%%%%%
    &\quad=\underbrace{{\sum_{h=1}^H \EE\bigsbr{\orbr{\tilde U_h - u_h}(s_h, a_h, b_h)- \tilde W_{h+1}(s_{h+1})}}}_{\dr \text{Leader's Bellman error}} + \underbrace{\sum_{h=1}^H \EE\bigsbr{\tilde W_h(s_h) -  \orbr{T_{h}^{\pi,\tilde\nu} \tilde U_{h}} (s_{h})}}_{\ds\text{Value mismatch error}}\nend
    &\qqquad+\underbrace{\sum_{h=1}^H  H  \cdot  \EE  \bigsbr{ \nbr{\rbr{\tilde \nu_h-\nu_h^{\pi }}(\cdot\given s_h)}_1} }_{\dr \text{Quantal response error}}, 
    \label{eq:perform-diff-linear} 
\end{align}
where  the expectation is taken with respect to the randomness of the trajectory generated by $(\pi, \nu^{\pi})$ on the true model $M^*$. 

 

\end{lemma}



Here, we provide two ways to estimate $J(\pi)$.
The first inequality \eqref{eq:perform-diff-general} is useful for  setting with general function approximation, and the second inequality \eqref{eq:perform-diff-linear} becomes handy particularly in the   setting with linear function approximation, where we  additionally introduced an algorithm with penalty/bonus terms. 

This lemma shows that the estimation error of $\EE_{s_1\sim\rho_0}\bigl [ \orbr{T_1^{\pi,\tilde\nu}\tilde U_1} (s_1 )\bigr ] $ can be decomposed into a sum of three terms --- leader's Bellman error, the error of the estimated quantal response, and an additional term that measures the mismatch between $\tilde W$ and $\tilde U$, which is included here for generality and will be used in the analysis of both \Cref{alg:PMLE} and \Cref{alg:MLE-OVI} where some penalties/bonuses are included in the estimation of $\hat W$.

Furthermore,  by the definition of 
$T_h^{\pi,\tilde\nu}$ in \eqref{eq:operator_T_pi_nu},
%  {\main\eqref{eq:relation_tileWU}\fi} {\neurips $
%  \tilde W_h(s ) = \la \tilde U_h (s, \cdot , \cdot ) , \pi _h \otimes \tilde \nu_h^{\pi} (\cdot , \cdot \given s) \ra . 
%  $\fi}  
 $
 \tilde W_h(s ) = \la \tilde U_h (s, \cdot , \cdot ) , \pi _h \otimes \tilde \nu_h^{\pi} (\cdot , \cdot \given s) \ra
 $
 is equivalent to $\tilde W_ h = T_h^{\pi, \tilde \nu} \tilde U_h$. 
Thus, 
\eqref{eq:perform-diff-linear} directly implies \eqref{eq:performance diff-1}
% {\main Lemma \ref{lem:performance diff informal}\fi} {\neurips \eqref{eq:performance diff-1}\fi} 
as a special case.

% \begin{align}
%     &\EE_{s_1\sim\rho_0}\sbr{ T_1^{\pi,\tilde\nu}\tilde U_1(s_1, a_1, b_1)} - J(\pi ) \nend
%     %%%%%%
%     % &\quad = {\sum_{h=1}^H \EE\sbr{\bigrbr{\tilde U_h - u_h}(s_h, a_h, b_h)- \tilde W_{h+1}(s_{h+1})}} + \sum_{h=1}^H \EE\sbr{\tilde W_h(s_h)-\tilde U_h(s_h, a_h, b_h)}\nend
%     %%%%%%
%     &\quad\le\underbrace{{\sum_{h=1}^H \EE\sbr{\bigrbr{\tilde U_h - u_h}(s_h, a_h, b_h)-  T_{h+1}^{\pi,\tilde\nu} \tilde U_{h+1}(s_{h+1})}}}_{\dr \text{Leader's Bellman error}}+ \underbrace{\sum_{h=1}^H  H \EE \nbr{\rbr{\tilde \nu_h-\nu_h}(\cdot\given s_h)}_1}_{\dr \text{Quantal response error}}\label{eq:perform-diff-general}\\
%     %%%%%%%%
%     &\quad=\underbrace{{\sum_{h=1}^H \EE\sbr{\bigrbr{\tilde U_h - u_h}(s_h, a_h, b_h)- \tilde W_{h+1}(s_{h+1})}}}_{\dr \text{Leader's Bellman error}} + \underbrace{\sum_{h=1}^H \EE\sbr{\tilde W_h(s_h) -  T_{h}^{\pi,\tilde\nu} \tilde U_{h}(s_{h})}}_{\ds\text{Value mismatch error}}\nend
%     &\qqquad+\underbrace{\sum_{h=1}^H  H \EE \nbr{\rbr{\tilde \nu_h-\nu_h}(\cdot\given s_h)}_1}_{\dr \text{Quantal response error}}, 
%     \label{eq:perform-diff-linear}
% \end{align}
 




\begin{proof}
By the definitions of $U^{\pi}$ and $W^{\pi}$ in \eqref{eq:U_function} and \eqref{eq:W_function}, $J(\pi)$ can be written as 
\$
J(\pi) = \EE_{s_1 \sim \rho_0 } [ W_1 ^{\pi} (s_1) ] = \EE_{s_1 \sim \rho_0 } [  \orbr{T_1 ^{\pi, \nu}U_1 ^{\pi} } (s_1) ],
\$ 
where we write $\nu = \nu^{\pi}$ to simplify the notation.
Recall that we define the quantal Bellman operator $\TT_h^{\pi}$ in \eqref{eq:bellman_operator_leader}, whose fixed point is $U^{\pi}$.
Then, by direct calculation, we have 
\#\label{eq:subopt-decompose-equality-0}
&\EE_{s_1\sim\rho_0}\bigsbr{ \orbr{T_1^{\pi,\tilde\nu}\tilde U_1}(s_1 )} - J(\pi )    \nend 
& \quad  =    \EE_{s_1\sim\rho_0}\bigsbr{\orbr{T_1^{\pi,\tilde\nu} - T_1^{\pi,\nu}} \tilde U_1(s_1)}  + \EE_{s_1\sim\rho_0}\bigsbr{\orbr{  T_1^{\pi,\nu}  \tilde U_1 - T_1^{\pi,\nu}  U_1^{\pi} } (s_1)}. 
\#
Furthermore, using the Bellman equation of $U^{\pi}$, we have 
\#\label{eq:subopt-decompose-equality-01}
& \EE_{s_1\sim\rho_0}\bigsbr{\orbr{  T_1^{\pi,\nu}  \tilde U_1 - T_1^{\pi,\nu}  U_1^{\pi} } (s_1)} 
= \EE \bigsbr{  \tilde U_1(s_1, s_1, a_1) - U_1 ^{\pi} (s_1, a_1, b_1)}\nend 
& \quad  = \EE \bigsbr{  \orbr{\tilde U_1 -u_1} (s_1, s_1, a_1) - T_{2}^{\pi,\nu} \tilde U_{2}(s_2 ) }  +  \EE\bigsbr{ T_{2}^{\pi,\nu} \tilde U_{2} (s_2) - T_{2}^{\pi,\nu}U_{2}^{\pi}  (s_2) }. 
\# 
Here the second equality follows from the Bellman equation, and the expectation is taken with respect to the randomness of the trajectory generated by $(\pi, \nu^{\pi})$ on the true model $M^*$. . 
Furthermore, by replacing $T_{2}^{\pi,\nu} \tilde U_{2}$ by $T_{2}^{\pi, \tilde \nu} \tilde U_{2}$ in \eqref{eq:subopt-decompose-equality-01}, and combining \eqref{eq:subopt-decompose-equality-0}, we obtain that 
\# 
&\EE_{s_1\sim\rho_0}\bigsbr{ \orbr{T_1^{\pi,\tilde\nu}\tilde U_1}(s_1 )} - J(\pi )    \nend 
     & \quad 
     =  {\sum_{h=1}^H \EE\bigsbr{\bigrbr{\tilde U_h - u_h}(s_h, a_h, b_h)-  \orbr{T_{h+1}^{\pi,\tilde\nu} \tilde U_{h+1}}(s_{h+1})}} + \sum_{h=1}^H \EE\bigsbr{\orbr{ T_h^{\pi,\tilde\nu}-  T_h^{\pi,\nu}} \tilde U_h(s_h)} ,\label{eq:subopt-decompose-equality-1}
\# 
where we apply recursion over all $h\in [H]$. 
Finally, note that $\tilde U_h $ is bounded by $H$ in the $\ell_{\infty}$-norm. Using H\"older's inequality, we have 
\#
\EE\bigsbr{\orbr{ T_h^{\pi,\tilde\nu}-  T_h^{\pi,\nu}} \tilde U_h(s_h)} \leq H \cdot \EE \bigsbr{ \nbr{\rbr{\tilde \nu_h-\nu_h}(\cdot\given s_h)}_1}.\label{eq:subopt-decompose-equality-12}
\#
Combining \eqref{eq:subopt-decompose-equality-1} and \eqref{eq:subopt-decompose-equality-12}, we establish \eqref{eq:perform-diff-general}.

 
It remains to prove  \eqref{eq:perform-diff-linear}. 
To this end, it suffices to incorporate the value mismatch error in \eqref{eq:perform-diff-linear} into \eqref{eq:perform-diff-general}.
Specifically,  for any $h \in [H]$, we have 
\#\label{eq:subopt-decompose-equality-2}
& \EE\bigsbr{\bigrbr{\tilde U_h - u_h}(s_h, a_h, b_h)-  T_{h+1}^{\pi,\tilde\nu} \tilde U_{h+1}(s_{h+1})} \notag \\
& \quad    ={\sum_{h=1}^H \EE\bigsbr{\bigrbr{\tilde U_h - u_h}(s_h, a_h, b_h)- \tilde W_{h+1}(s_{h+1})}} + \sum_{h=2}^H \EE\bigsbr{\tilde W_h(s_h) - T_h^{\pi,\tilde\nu}\tilde U_h(s_h)}. 
\#
Combining \eqref{eq:perform-diff-general} and \eqref{eq:subopt-decompose-equality-2}
we have 
\$
& \EE_{s_1\sim\rho_0}\bigsbr{ \tilde W_1(s_1)} - J(\pi) \notag \\
& \quad = \EE\bigsbr{\tilde W_1(s_1) - \orbr{T_1^{\pi,\tilde\nu}\tilde U_1}(s_1)} +  \EE\bigsbr{ \orbr{T_1^{\pi,\tilde\nu}\tilde U_1} (s_1)} - J(\pi) 
\notag \\
&  \quad\le {\sum_{h=1}^H \EE\bigsbr{\bigrbr{\tilde U_h - u_h}(s_h, a_h, b_h)- \tilde W_{h+1}(s_{h+1})}} + \sum_{h=1}^H \EE\bigsbr{\tilde W_h(s_h) - \orbr{T_h^{\pi,\tilde\nu}\tilde U_h }(s_h)}\nend
&\qqquad+ \sum_{h=1}^H  H \cdot  \EE \bigsbr{ \nbr{\rbr{\tilde \nu_h-\nu_h}(\cdot\given s_h)}_1} ,
\$ 
which gives us \eqref{eq:perform-diff-linear}. 
    Hence, we complete the proof.
% where $H$ upper bound $\tilde U$. The first equality holds by a switch between $T_1^{\pi, \tilde\nu}$ and $T_1^{\pi, \nu}$, the second equality holds by doing this switching recursively and noting that $\nu_{H+1}=\tilde\nu_{H+1}$ since it is out of the horizon, and the last inequality holds by the difinition of $\nbr{\cdot}_1$-norm. Hence, we obtain \eqref{eq:perform-diff-general}. If we include $\tilde W$, we just decompose all the $T_{h+1}^{\pi,\tilde\nu}\tilde U_{h+1}(s_{h+1})$ terms for $h=1,\dots, H$ on the righthand side of \eqref{eq:perform-diff-general} and also the $T_1^{\pi,\tilde\nu}\tilde U_1(s_1, a_1, b_1)$ term on the left hand side and obtain 
%     \begin{align}
%         &\EE_{s_1\sim\rho_0}\sbr{ \tilde W_1(s_1)} - J(\pi) \nend
%     %%%%%%
%     &\quad =  {\sum_{h=1}^H \EE\sbr{\bigrbr{\tilde U_h - u_h}(s_h, a_h, b_h)-  T_{h+1}^{\pi,\tilde\nu} \tilde U_{h+1}(s_{h+1})}} + \sum_{h=1}^H \EE\sbr{\rbr{ T_h^{\pi,\tilde\nu}-  T_h^{\pi,\nu}} \tilde U_h(s_h)}\nend
%     &\qqquad + \EE\sbr{\tilde W_1(s_1) - T_1^{\pi,\tilde\nu}\tilde U_1(s_1)}\nend
%     %%%%%%%%
%     &\quad={\sum_{h=1}^H \EE\sbr{\bigrbr{\tilde U_h - u_h}(s_h, a_h, b_h)- \tilde W_{h+1}(s_{h+1})}} + \sum_{h=1}^H \EE\sbr{\tilde W_h(s_h) - T_h^{\pi,\tilde\nu}\tilde U_h(s_h)}\nend
%     &\qqquad+  \sum_{h=1}^H \EE\sbr{\rbr{ T_h^{\pi,\tilde\nu}-  T_h^{\pi,\nu}} \tilde U_h(s_h)}\label{eq:subopt-decompose-equality-2}\\
%     %%%%%%%%%%
%     &\quad\le {\sum_{h=1}^H \EE\sbr{\bigrbr{\tilde U_h - u_h}(s_h, a_h, b_h)- \tilde W_{h+1}(s_{h+1})}} + \sum_{h=1}^H \EE\sbr{\tilde W_h(s_h) - T_h^{\pi,\tilde\nu}\tilde U_h(s_h)}\nend
%     &\qqquad+ \sum_{h=1}^H  H \EE \nbr{\rbr{\tilde \nu_h-\nu_h}(\cdot\given s_h)}_1, \notag
%     \end{align}
%     which gives us \eqref{eq:perform-diff-linear}. 
%     Hence, we complete the proof.
\end{proof}



Recall that we estimate the quantal response mapping via   model-based maximum likelihood  estimation. 
In particular, we approximate the true quantal response policy $\nu^{\pi}$ within class $\{ \nu^{\pi, \theta}\}_{\theta \in \Theta}$, where 
$\nu^{\pi, \theta}$ can be written as 
\$
\nu_h^{\pi, \theta} (b_h\given s_h) &= \exp\bigl ( \eta \cdot A_h^{\pi, \theta} (s_h, b_h) \bigr), \quad\text{where}\quad A_h^{\pi, \theta} (s_h, b_h) = Q_h^{\pi, \theta} (s_h, b_h) - V_{   h}^{\pi, \theta} (s_h) ,
\$
Here $A^{\pi, \theta}$, $Q^{\pi, \theta}$, and $V^{\pi, \theta}$ are the advantage function, and value functions corresponding to policy $\pi$, under the model $\{ r^{\theta} , P^{\theta}\}$. 
In the following, we present a lemma which relates the error of quantal response mapping in \eqref{eq:perform-diff-general} and \eqref{eq:perform-diff-linear} to  estimation errors of the follower's value functions.
To simplify the notation, 
we consider a fixed policy $\pi$, and define  
$r_h^\pi(s_h,b_h) =\inp{r_h(s_h, \cdot, b_h)}{\pi(\cdot\given s_h, b_h)}_{\cA}$ and $P_h^\pi(s_{h+1}\given s_h,b_h) =\inp{P_h(s_{h+1}\given s_h, \cdot, b_h)}{\pi(\cdot\given s_h, b_h)}_{\cA} $.
To simplify the notation, 
in the rest of subsection, 
we let $\EE=\EE^{\pi,M^*}$ and $\EE_{z} [\cdot]=\EE^{\pi,M^*}[\cdot\given z]$ for any variable $z$.
 

%Our first group of techniques concern the suboptimality decomposition of the QSE with strategic follower. 
%   , and denote by $(\tilde U, \tilde W, \tilde Q, \tilde V, \tilde A)$ an alternative satisfying $\tilde V_h = \eta^{-1}\log \int \exp(\eta \tilde Q_h)$, $\tilde A_h = \tilde Q_h -\tilde V_h$. 
% Moreover, we let $\tilde \nu_h = \exp(\eta \tilde A_h)$ as the response under $\tilde A$. In the sequel, we let $\EE=\EE^{\pi,M^*}$ and $\EE_{z} [\cdot]=\EE^{\pi,M^*}[\cdot\given z]$ if without special reminder. We ignore the superscript $\pi, M^*$ for simplicity. We also let $r_h^\pi(s_h,b_h) =\inp{r_h(s_h, \cdot, b_h)}{\pi(\cdot\given s_h, b_h)}$ and $P_h^\pi(s_{h+1}\given s_h,b_h) =\inp{P_h(s_{h+1}\given s_h, \cdot, b_h)}{\pi(\cdot\given s_h, b_h)}$.  
% Based on \eqref{eq:performance diff-1},






% We consider a fixed policy $\pi$,  denote by $(U, W, Q, V, A, \nu)$ the ground truth under $\pi$ and the true model $M^*$, and denote by $(\tilde U, \tilde W, \tilde Q, \tilde V, \tilde A, \tilde \nu)$ an alternative satisfying $\tilde V_h = \eta^{-1}\log \int \exp(\eta \tilde Q_h)$, $\tilde A_h = \tilde Q_h -\tilde V_h$, and $\tilde \nu_h = \exp(\eta \tilde A_h)$. We then have for the U function that



% We next characterize the quantal response error in a Taylor expansion flavor.
\begin{lemma}[Response Model Error]
    \label{lem:performance diff}
    We consider a fixed policy $\pi$
and   let $\tilde Q$    be an estimate  of $Q^{\pi}$. 
We define a V-function  $\tilde V$ and an advantage function $\tilde A$ by letting 
\#\label{eq:tilde_functions}
\tilde V_h (s) = \frac{1}{\eta} \cdot \log \bigg(  \sum_{b \in \cB} \exp( \eta \cdot \tilde Q_h(s, b) ) \biggr), \qquad \textrm{and}\qquad \tilde A_h(s,a) = \tilde Q_h (s,b) - \tilde V_h (s). 
\# 
Furthermore, we define a follower's policy $\tilde \nu$ by letting $\tilde \nu_h(b \given s) = \exp( \eta\cdot \tilde A_h(s,b))$. 
Then the difference between $\tilde \nu$ and $\nu^{\pi}$ can be bounded by 
\begin{align}
    &\sum_{h=1}^H  H \cdot  \EE \bigsbr{\nbr{\rbr{\tilde \nu_h-\nu_h^{\pi} }(\cdot\given s_h)}_1} \nend
    &\quad\le C^{(0)} \cdot \sum_{h=1}^H \underbrace{\EE\bigsbr{\bigabr{\tilde \Delta^{(1)}_h(s_h, b_h)}}}_{\ds\text{1st-order error}}  + C^{(1)} \cdot 
    \sum_{h=1}^H \underbrace{\EE\bigsbr{ \bigabr{(\tilde A_h  - A_h^{\pi})(s_h, b_h)}^2}}_{\ds\text{2nd-order error}} \label{eq:taylor-myopic}\\
    &\quad\le 
    C^{(0)} \cdot 
    \sum_{h=1}^H \underbrace{\EE\bigsbr{\bigabr{\tilde \Delta^{(1)}_h(s_h, b_h)}}}_{\ds\text{1st-order error}}  + C^{(2)} \cdot 
    \max_{h\in [H]} \underbrace{\EE\bigsbr{ \bigabr{\orbr{\tilde Q_h - r_h^\pi - \gamma P_h^\pi \tilde V_{h+1}}(s_h, b_h)}^2}}_{\ds\text{2nd-order error}},\label{eq:taylor-farsighted} 
\end{align}
where $\tilde \Delta^{(1)}_h(s_h, b_h)$ is defined as
\begin{align*}
    \tilde \Delta^{(1)}_h(s_h, b_h) &=  \rbr{\EE_{s_h, b_h} -\EE_{s_h}}\Biggsbr{\sum_{l=h}^H \gamma^{l-h}\underbrace{\rbr{\tilde Q_l - r_l^\pi - \gamma P_l^\pi \tilde V_{l+1}}(s_l, b_l)}_{\ds\text{Follower's Bellman error}}}. 
    % \label{eq:def Delta^1}
\end{align*}
Furthermore, the constants $C^{(0)}$, $C^{(1)}$, and $C^{(2)}$ are given by
\#\label{eq:define_constants}
\begin{split}
    C^{(0)}&=2\eta H , \qquad   C^{(1)}=\eta^2 H   \bigrbr{1+ 4 \eff_H(\gamma)}\cdot \exp(2\eta B_A), \\
C^{(2)}  & =  2  \eta^2 H^2\cdot \exp\orbr{6\eta B_A} \cdot(1+4 \eff_H(\gamma)) \cdot \rbr{\eff_H(\exp(2\eta B_A)\gamma)}^2,
\end{split}
\#
where $B_A$ defined in \eqref{eq:define_BA} is an upper bound on the magnitude of the advantage function, and we define $\eff_H(x) = (1-x^H)/(1-x)$ as the \say{effective}  horizon with respect to $x$.
% \begin{proof}
%     See \Cref{lem:performance diff} for a detailed proof.
% \end{proof}



    % denote by $(U, W, Q, V, A)$ the ground truth under $\pi$, and denote by $(\tilde U, \tilde W, \tilde Q, \tilde V, \tilde A)$ an alternative satisfying $\tilde V_h = \eta^{-1}\log \int \exp(\eta \tilde Q_h)$, $\tilde A_h = \tilde Q_h -\tilde V_h$. Suppose that $\tilde\nu_h=\exp\orbr{\eta \tilde A_h}$ and $\nu$ is the real quantal response under policy $\pi$. The behavior model error in \eqref{eq:perform-diff-linear} can be upper bounded by
\end{lemma}
\begin{proof}
    See \Cref{sec:proof-performance diff} for a detailed proof.
\end{proof}

If we view the quantal response as a functional of the advantage function (as shown in \eqref{eq:quantal_response_policy}),  this lemma plays the role of Taylor expansion of the quantal response into the first- and second-order errors. 
In particular, the first-order error corresponds to the Bellman error of the follower's problem, and the second-order term is mean-squared error of the advantage function (as in \eqref{eq:taylor-myopic}) or the sum of squares of the Bellman error (as in \eqref{eq:taylor-farsighted}). 
Here we establish two versions of upper bounds because \eqref{eq:taylor-myopic} is handy for the myopic case while the second \eqref{eq:taylor-farsighted} is more useful for the farsighted case.
 
In the case with a  myopic follower,
the $Q$-function is reduced to the reward function $r$ of the follower. 
We have the following corollary. 
\begin{corollary}[Response Model Error for Myopic Case]\label{cor:response-diff-myopic}
    Let $r$ be the true reward function and let $\tilde r$ be an estimated reward. 
Let $\pi \colon \cS \times \cB \rightarrow \Delta(\cA )$ be a fixed policy. 
Let $\nu$ and $\tilde \nu$ be the quantal response function based on $r$ and $\tilde r$, respectively, i.e.,
\$
\nu(b\given s) \propto \exp \big (  \eta\cdot  r^{\pi} (s,b)\big ) ,\qquad \nu(b\given s) \propto \exp\big ( \eta \cdot \tilde r^{\pi} (s,b)\big ).
\$
Here we define $r^{\pi} $ by letting $r^{\pi} (s,b) = \la r(s,\cdot, b), \pi(\cdot \given s, b)\ra$, and define $\tilde r^{\pi}$ similarly. 
Then for any state $s\in \cS$, we have 
    \begin{align}
        D_\TV\rbr{\nu(\cdot\given s), \tilde\nu(\cdot\given s)}
        &\le  \eta  \EE_s\sbr{\abr{(\tilde r^\pi(s, b) - r^\pi(s, b)) - \EE_s\bigsbr{\tilde r^\pi(s, b) - r^\pi(s, b)}}} \nend
        &\qquad + C^{(3)}\EE_s\bigsbr{\rbr{\rbr{\tilde r^\pi(s, b) -r^\pi(s, b)} - \EE_s\sbr{\tilde r^\pi(s, b) -r^\pi(s, b)}}^2}, \label{eq:TV-for-myopic}  
        % \notag 
    \end{align}
    where $C^{(3)}={\eta^2 \exp(2\eta B_A)}\bigrbr{2+\eta B_A \cdot  \exp\rbr{2\eta B_A}}/2$ and $B_A$  defined in \eqref{eq:define_BA} is $2 + 2 \log |\cB| / \eta  $. 
    Here the expectation $\EE_s$ is only taken with respect to $b \sim \nu(\cdot \given s)$ when conditioned on this $s$.
    \begin{proof}
        See \Cref{sec:proof-response-diff-myopic} for a detailed proof.
    \end{proof}
\end{corollary}
In the following, we present the result for a special case where 
the follower is myopic with a linear reward function. 
In this case, we write 
  $Q^{\pi, \theta}(s, b) = r^{\pi, \theta}(s, b) = \inp[]{\phi^\pi(s, b)}{\theta}$ for some $\RR^d$ kernel $\phi:\cS\times\cA\times\cB\rightarrow\RR^d$ with $\phi^\pi(s, b)=\inp{\phi(s, \cdot, b)}{\pi(\cdot\given s, b)}_\cA$ and parameter $\theta\in\RR^d$.
  The quantal response policy is given by $\nu^{\pi, \theta} (b \given s) \propto \exp(\eta \cdot \la \phi^{\pi}(s,b), \theta \ra)$. 
In particular, we show that the 1st- and the 2nd-order QRE decomposition in \eqref{eq:QRE-decompose} of \Cref{sec:learning QSE} is just a direct result of \Cref{cor:response-diff-myopic}. Recall by definition of the $\QRE$ in \eqref{eq:QRE}, 
\begin{align*}
    \QRE(s_h, b_h;\tilde\theta,\pi) &=  (\Upsilon_h^{\pi}(\tilde r_h - r_h))\orbr{s_h, b_h}\nend
    &= \dotp{\pi_h(\cdot\given s_h, b_h)}{(\tilde r_h - r_h)(s_h, \cdot, b_h)} - \dotp{\pi_h\otimes \nu_h^{\pi}(\cdot,\cdot\given s_h)}{(\tilde r_h - r_h)(s_h,\cdot,\cdot)}\nend
    & = \sbr{{(\tilde r_h^\pi(s_h, b_h) - r_h^\pi(s_h, b_h)) - \EE_{s_h}\bigsbr{\tilde r_h^\pi(s_h, b_h) - r_h^\pi(s_h, b_h)}}}. 
\end{align*}
We plug in the linear representation of the follower's reward $r_h^\pi(s_h, b_h) = \la\phi_h^\pi(s_h, b_h), \theta_h^*\ra$, which implies that 
\begin{align*}
    \EE_{s_h}\QRE(s_h, b_h;\tilde\theta, \pi)^2 = \Cov_{s_h}^{\pi,\theta^*} \sbr{(\tilde r_h^\pi - r_h^\pi)(s_h, b_h)} = \onbr{\tilde \theta_h -\theta_h}_{\Sigma_{s_h}^{\pi,\theta^*}}^2,
\end{align*}
where $\Sigma_{s_h}^{\pi, \theta^*} = \Cov_{s_h}^{\pi, \theta^*}[\phi_h^\pi(s_h, b_h)]$. Moreover, For the first term on the right hand side of \eqref{eq:TV-for-myopic}, 
\begin{align*}
    &\EE_{s_h}\sbr{\abr{(\tilde r^\pi(s_h, b_h) - r^\pi(s_h, b_h)) - \EE_s\bigsbr{\tilde r^\pi(s_h, b_h) - r^\pi(s_h, b_h)}}} \nend
    &\quad \le \sqrt{\EE_{s_h}\bigsbr{\rbr{\rbr{\tilde r^\pi(s_h, b_h) -r^\pi(s_h, b_h)} - \EE_{s_h}\sbr{\tilde r^\pi(s_h, b_h) -r^\pi(s_h, b_h)}}^2}} \nend
    &\quad = \EE_{s_h}\QRE(s_h, b_h;\tilde\theta, \pi)^2,
\end{align*}
where the inequality holds by the Cauchy-Schwarz inequality. The second term follows similarly.
We further generalize the above argument to the following corollary. 

\begin{corollary}[Response Model Error for Linear and  Myopic Case]\label{lem:response diff-myopic-linear}
    Under the setting of \Cref{cor:response-diff-myopic}, we assume the reward function of the myopic follower is a linear function of $\phi$. 
    Let $\theta^*$ be the parameter of the true reward function and let 
    $\tilde \theta \in \Theta$ be another parameter. 
    Let $\pi \in \Pi$ be any policy and let $s \in \cS$ be any state. 
   We define a matrix  $\Sigma_s^{\pi,\theta}$  as $ \Cov_s^{\pi,\theta}[\phi^\pi(s, b)], $ i.e.,
   \#\label{eq:define_sigma_s}
   \Sigma_s^{\pi,\theta} = \EE_s \bigl [ (\phi^\pi(s, b) - \EE_s [\phi^\pi(s, b)] \bigr )(\phi^\pi(s, b) - \EE_s [\phi^\pi(s, b)] \bigr )^\top \bigr ],
   \#
   where the expectation is taken with respect to $\nu^{\pi,\theta}(\cdot \given s)$.
    Then we have 
    \begin{align}
        &D_\TV\rbr{\nu^{\pi,\theta^*}(\cdot\given s) , \nu^{\pi,\tilde\theta}(\cdot\given s)} \nend
        &\quad \le \min\Bigcbr{f\Bigrbr{\sqrt{\trace\bigrbr{{\Psi}^{\dagger} \Sigma_{ s}^{\pi, \tilde\theta}}}\cdot \bignbr{\theta^*- \tilde\theta}_{{\Psi}}}, f\Bigrbr{\sqrt{\trace\bigrbr{{\Psi}^{\dagger} \Sigma_{ s}^{\pi, \theta^*}}}\cdot \bignbr{\theta^*- \tilde\theta}_{{\Psi}}}},
        % \label{eq:TV-for-myopic-linear}
    \end{align}
    where $\Psi\in \SSS_+^{d}$ can be any fixed nonnegative definite matrix, $\Psi^{\dagger}$ is the pseudo-inverse of $\Psi$,  and the univariate function $f$ is defined as
    $
        f(x) = \eta x + C^{(3)} x^2 
    $ with 
    $$C^{(3)}={\eta^2 \exp(2\eta B_A)}\bigrbr{2+\eta B_A \cdot  \exp\rbr{2\eta B_A}}/2.$$
\end{corollary}

\begin{proof}
    See \Cref{sec:proof-response-diff-myopic_linear} for a detailed proof. 
\end{proof}
    % $\Gamma^{(2)}_h$ is defined as
    % \begin{align*}
    %     \Gamma^{(2)}_h(s_h;\alpha,\theta_h) = \sqrt{\EE_{s_h}^{{\alpha,\theta_h}}\sbr{\phi_h^{\alpha}(s_h,b_h)^\top {\Psi}^{\dagger}\phi_h^{\alpha}(s_h,b_h)} -\bignbr{\EE_{s_h}^{{\alpha,\theta_h}}\phi_h^{\alpha}(s_h,b_h)}_{{\Psi}^{\dagger}}^2}\cdot \bignbr{\theta_h^*- \theta_h}_{{\Psi}}, 
    % \end{align*}
    % and $\Gamma^{(3)}$ is defined as 
    % \begin{align*}
    %     \Gamma^{(3)}_h(s_h;\alpha,\theta_h) = \sqrt{\EE_{s_h}^{{\alpha,\theta_h^*}}\sbr{\phi_h^{\alpha}(s_h,b_h)^\top {\Psi}^{\dagger}\phi_h^{\alpha}(s_h,b_h)} -\bignbr{\EE_{s_h}^{{\alpha,\theta_h^*}}\phi_h^{\alpha}(s_h,b_h)}_{{\Psi}^{\dagger}}^2}\cdot \bignbr{\theta_h^*- \theta_h}_{{\Psi}}
    % \end{align*}
    
We first show that the uncertainty quantifier $\Gamma^{(2)}_h(s_h;\pi, \theta_h)$ used in \eqref{eq:Gamma^2} of \S\ref{sec:offline-ML} {\neurips and \eqref{eq:Gamma^2-neurips}} can be derived directly from \Cref{lem:response diff-myopic-linear}. 
Recall by definition, 
\begin{align*}
    \Gamma^{(2)}_h(s_h;\pi_h , \theta_h) = 2 H(\eta  \xi(s_h;\pi,\theta_h)  + C^{(3)}  \xi(s_h;\pi,\theta_h)^2 ), 
\end{align*}
where $\xi(s_h;\pi,\theta_h)^2$ is defined as
\begin{equation*}
    \xi(s_h;\pi, \theta_h)^ 2 =  \trace\bigrbr{\bigrbr{T\Sigma_{h,\cD}^{\theta} + I_d}^\dagger \Sigma_{s_h}^{\pi , \theta}}  \cdot  \bigl( 2 (\eta^{-1}+B_A)^2 \beta + 4B_\Theta^2 \bigr).
\end{equation*}
One can easily check that for any $\theta_h\in\CI_{h,\Theta}(\beta)$ where $\CI_{h,\Theta}(\beta)$ is the offline confidence set, 
\begin{align*}
    \Gamma^{(2)}_h(s_h;\pi_h,\theta_h) 
    &= 2 H \cdot f\Bigrbr{\sqrt{\trace\bigrbr{\bigrbr{T\Sigma_{h,\cD}^{\theta}+I_d}^\dagger \Sigma_{s_h}^{\pi,\theta}}} (2C_\eta^2 \beta + 4 B_\Theta^2)}\nend
    &\ge 2 H \cdot f\Bigrbr{\sqrt{\trace\bigrbr{\bigrbr{T\Sigma_{h,\cD}^{\theta}+I_d}^\dagger \Sigma_{s_h}^{\pi,\theta}}} (T\nbr{\theta_h -\theta_h^*}_{\Sigma_{h,\cD}^{\theta}}^2 + \nbr{\theta_h -\theta_h^*}_{I_d}^2)}\nend
    & \ge 2 H D_\TV\bigrbr{\nu_h^{\pi,\theta^*}(\cdot\given s_h), \nu_h^{\pi,\theta}(\cdot\given s_h)}, 
\end{align*}
where we use definition $C_\eta^2 = \eta^{-1}+B_A$ in the first equality, and in the first inequality, $2C_\eta^2\beta T^{-1} \ge \nbr{\theta_h-\theta_h^*}_{\Sigma_{h,\cD}^\theta}^2$ holds by \Cref{cor:formal-MLE confset-linear myopic} if $\CI_{h,\Theta}(\beta)$ is a valid confidence set, and $4B_\Theta^2\ge \nbr{\theta_h-\theta_h^*}_{I_d}^2$ holds by noting that $\nbr{\theta_h}^2\le B_\Theta$ for any $\theta_h\in\Theta_h$. 
The last inequality holds just by \Cref{lem:response diff-myopic-linear} where we plug in $\Psi = T\Sigma_{h,\cD}^{\theta}+I_d$.
% Note that $\bignbr{\theta^*-\tilde\theta}_{\Sigma_{s_h}^{\pi,\theta^*}}$



\subsection{Learning Quantal Response via MLE}\label{sec:app-MLE}
In \S\ref{sec:MLE for behavior model}, we introduce how to learn the follower's  quantal response model from the follower's feedbacks via maximum likelihood estimation. 
In the following, we provide a slightly stronger lemma that implies  \Cref{sec:proof-MLE-general} in \S\ref{sec:MLE for behavior model}. 
In the following, we follow the notation used in \S\ref{sec:MLE for behavior model}, where 
we let $\theta = \{ \theta_h \}_{h \in [H]} \in \Theta $ denote the parameters of the follower's response model,
where $\Theta = \Theta_1 \times \ldots \times \Theta_H$ is the set of all parameters. 
Here we assume $\theta_h \in \Theta_h$ for all $h\in [H]$.
In specific, each $\theta$ is associated with a model, denoted by  ${ r^{\theta}, P^{\theta}}$, 
which is   shorthand notation for 
\$
\{ r_1^{\theta_1}, P_1^{\theta_1},  \ldots, r_H ^{\theta_H}, P_H^{\theta_H}\}. 
\$ 
Moreover, when the follower is myopic, $\theta$ only parameterize a reward function $r^{\theta}$. 
We assume that the parametric model of the follower captures the true model. That is, there exists $\theta^* \in \Theta $ such that 
$\{r^{\theta}, P^{\theta^*}\}$ is the true environment. 


%we assume that  
%$\Theta$ is the set of the  parameters that determines the follower's response model. 


For any $\theta \in \Theta$ and any policy $\pi$ of the leader, we let $\nu^{\pi, \theta}$, $A^{\pi, \theta}$, $Q^{\pi, \theta } $ and $V^{\pi, \theta }$ denote the quantal response of $\pi$, advantage function, and Q- and V-functions under model $\{ r^{\theta}, P^{\theta} \}$, which are defined according to  \eqref{eq:quantal_response_policy}--{\main\eqref{eq:v_pi_qr}\fi}{\neurips \eqref{eq:qv_pi_qr}\fi}.
 The quantal response   under the true model is $  \nu^{\pi, \theta^*}$.
Thus, given a  (possibly adaptive) dataset $\cD = \{(s_h ^i, a_h ^i, b_h ^i, \pi_h ^i)\}_{i\in[t-1], h\in [H]}$,
the negative log-likelihood at step $h$ is given by
\#\label{eq:loglikelihood-1}
\cL_h^t  (\theta  )  & = -  \sum_{i = 1}^{t-1}  
\log \nu^{\pi^i, \theta} (b_h^i \given s_h^i) 
  = -  \sum_{i=1}^{t-1}   \eta  \cdot A_h^{\pi^i , \theta}(s^i,  b^i), 
\# 
 where   $\nu^{\pi^i, \theta} (b_h^i \given s_h^i) $ is the probability of observing the follower's action $b_h^i$ when the model parameter is $\theta$, state is $s_h^i$, and the leader announces $\pi^i$, and the second equality in \eqref{eq:loglikelihood-1} is due to \eqref{eq:quantal_response_policy}.   
 Here we assume the data $\cD$ satisfies the compliance property, i.e., $\PP^{\pi^t}_\cD(b_h^t\given s_h^t, (s_j^t, a_j^t, b_j^t, u_j^t)_{j\in[h-1]}, \tau^{1:t-1})  = \nu_h^{\pi^t}(b_h^t\given s_h^t), \forall h\in[H], t\in[T]$. 
 Such a property is satisfied 
by the online setting and the offline setting where behavior policies are adaptive. 
 Therefore, the MLE estimator of $\theta^*$ can be obtained by minimizing $\cL_h^t(\theta)$ over $\Theta$. 
Moreover, based on $\cL_h$, we can construct a confidence set for $\theta^*$: 
\begin{align}
    \confset_{h,\Theta}^t(\beta)=\cbr{\theta\in\Theta: \cL_h^t(\theta)\le \inf_{\theta'\in\Theta}\cL_h^t(\theta') + \beta}, \label{eq:behavior_model_confset-1}
\end{align}
where  $\cL_h^t(\theta) = \sum_{i=1}^{t-1} \eta A_h^{\pi^i, \theta}(s_h^i, b_h^i)$.


We remark that $\cL_h^t(\theta)$ \eqref{eq:loglikelihood-1} is a function of $\theta = \{ \theta_h \}_{h\in [H]}$. 
The reason is that the follower's quantal response is obtained by solving the optimal policy of an entropy-regularized MDP, whose optimal policy depends on the reward and transitions across all $H$ steps. 
But if the follower is myopic, then $\cL_h^t(\theta)$ only depends on $\theta_h$. In this case, we can regard $\cL_h^t$ as a function on $\Theta_h$, and replace $\Theta$ in \eqref{eq:behavior_model_confset-1} by $\Theta_h$.  We will leverage such observation in \Cref{sec:learning QSE}.


\begin{lemma}[Confidence Set]\label{lem:MLE-formal}
We define a distance $\rho  $ on $\Theta$ by letting 
\begin{align}\label{eq:rho for MLE}
 \rho(\theta, \tilde \theta) \defeq \max_{\pi\in\Pi, s_h\in\cS, h\in[H]} \cbr{D_\H\orbr{\nu_h^{\pi, \theta}(\cdot\given s_h), \nu_h^{\pi, \tilde \theta}(\cdot\given s_h)}, (1+\eta) \cdot\bignbr{Q_h^{\pi, \theta} - Q_h^{\pi, \tilde\theta}}_\infty}, 
\end{align}
where $B_A$ upper bounds the follower's A-, Q-, and V-functions and is specified in \eqref{eq:define_BA}. 
Let $\cN_\rho(\Theta,\epsilon)$ be the $\epsilon$-covering number of $\Theta$ with respect to the distance $\rho$.
That is, $\cN_\rho(\Theta,\epsilon)$ is the smallest $N \geq 1$ with the following property: there exists $\{ \theta^i\}_{i\in [N]} \subseteq \Theta$ such that, for any $\theta \in \Theta$, there exists $\theta^i$ such that $\rho(\theta, \theta^i) \leq \epsilon$.
For any $\delta \in (0,1)$, we set $\beta \ge  2\log(e^3 H\cdot \cN_\rho(\Theta, T^{-1})/\delta)$.  
Then  with probability at least $1-\delta$,   the following properties hold for $\confset_{h, \Theta}^t (\beta)$ defined in \eqref{eq:behavior_model_confset-1}:
    \begin{itemize}
    \item [(i)]  (Validity) $\theta^*\in\confset_{h, \Theta}^t(\beta)$; 
    \item [(ii)] (Accuracy) For any $\theta\in\Theta, t\in[T], h\in[H]$, it holds that 
    \#
        &
        \sum_{i=1}^{t-1} D_\H^2\bigrbr{\nu_h^{\pi^i, \theta}(\cdot\given s_h^i), \nu_h^{\pi^i, \theta^*}(\cdot\given s_h^i)} \le  \frac {1}{2}\rbr{\cL_h^t(\theta) - \cL_h^t(\theta^*)+ \beta}, \label{eq:MLE-guarantee-hellinger-1}\\
        &
        \sum_{i=1}^{t-1} \EE^{\pi^i }D_\H^2\bigrbr{\nu_h^{\pi^i, \theta}, \nu_h^{\pi^i, \theta^*}} \le  \frac {1}{2}\rbr{\cL_h^t(\theta) - \cL_h^t(\theta^*)+ \beta}\label{eq:MLE-guarantee-hellinger-2}.
 \#
 \end{itemize} 
 Furthermore, the above two inequalities ensure respectively that for 
   $\forall \theta'\in\bigcbr{\theta^*, \theta}, \theta\in\Theta, \forall h\in[H]$,
    \begin{align}
        &\sum_{i=1}^{t-1} {\Var_{s_h^i}^{\pi^i, \theta'} \bigsbr{Q_h^{\pi^i, \theta}(s_h, b_h) - Q_h^{\pi^i, \theta^*}(s_h, b_h)}} \le 4 C_\eta^2 \rbr{\cL_h^t(\theta) - \cL_h^t(\theta^*)+ \beta}, \label{eq:MLE_guarantee_Q-1} \\
        & \sum_{i=1}^{t-1} \EE^{\pi^i }{\Var_{s_h}^{\pi^i, \theta'} \bigsbr{Q_h^{\pi^i, \theta}(s_h, b_h) - Q_h^{\pi^i, \theta^*}(s_h, b_h)}} \le  4 C_\eta^2 \rbr{\cL_h^t(\theta) - \cL_h^t(\theta^*)+ \beta},\label{eq:MLE_guarantee_Q}
    \end{align}
    where $\Var_s^{\pi, \theta}[Z] = \Var^{\pi, \theta}[Z\given s] = \EE^{\pi,\theta}[(Z - \EE^{\pi, \theta}[Z\given s])^2\given s]$, $C_\eta =\eta^{-1}+B_A$. Moreover, for any $\theta\in\CI_\Theta(\beta), h\in[H]$, and a given  $t\in[T]$, we have with probability at least $1-\delta$ that
\begin{align}
&\sum_{i=1}^{t-1} 
    \rbr{\rbr{Q_h^{\pi^i, \theta} - Q_h^{\pi^i, \theta^*}}(s_h^i, b_h^i)
    -\EE_{s_h^i}^{\pi^i,\theta^*}\sbr{ \bigrbr{Q_h^{\pi^i, \theta} - Q_h^{\pi^i, \theta^*}}(s_h, b_h)}}^2 
    \le  \cO(C_\eta^2 \beta) .  \label{eq:MLE-guarantee-Q-3}
\end{align}
 
% and $\cN_\varrho (\Theta, \epsilon)$ is the covering number of the smallest $\epsilon$-covering numebr of $\Theta$ with respect to distance 
% \begin{align*}
%     \varrho(\theta, \tilde\theta) \defeq \max_{\pi\in\Pi, h\in[H]}\nbr{Q_h^{\pi, \theta} - Q_h^{\pi, \tilde\theta}}_\infty.
% \end{align*}
    \begin{proof}
        See \Cref{sec:proof-MLE-general} for a detailed proof.
    \end{proof}
\end{lemma}
We remark that if $\theta\in\CI_\Theta^t(\beta)$, the guarantees in the accuracy results \eqref{eq:MLE_guarantee_Q} and \eqref{eq:MLE_guarantee_Q-1} are just $8C_\eta^2 \beta$ since $\cL_h^t(\theta)-\cL_h^t(\theta^*)\le \cL_h^t(\theta) - \inf_{\theta'\in\Theta}\cL_h^t(\theta') \le \beta$ for all $h\in[H]$.
Moreover, the covering number defined here is a special case of a more general version given by \eqref{eq:rho-Theta} and also in the myopic case it is given by \eqref{eq:rho-Theta_h}. 
Next, we comment on the myopic case, where it suffices to consider a covering for $\Theta_h$ (which only contains the parameters for the follower's reward function at step $h$) instead of the whole $H$-step class $\Theta$.
\begin{remark}[Myopic case for \Cref{lem:MLE-formal}]\label{rmk:MLE-formal-myopic}
    If the follower is myopic, it suffices to use the distance $\rho$ for class $\Theta_h$ defined in \eqref{eq:rho-Theta_h}, and let $\cN_\rho(\Theta, \epsilon) = \max_{h\in[H]} \cN_\rho(\Theta_h, \epsilon)$. The conclusion in \Cref{lem:MLE-formal} still applies.
\end{remark}

% {\main
% Here, the first expectation-type guarantee is useful for the online setting while the non-expectation-type guarantee is used for the offline setting.
% For the latter,  we allow the data to be dependent across samples or there might be no stationary distribution for the data collection process.
% To bridge the result in \Cref{lem:MLE-formal} to \eqref{eq:bandit-ub-1}, we define $\hat\theta_\MLE=\argmin_{\theta\in\Theta}\cL^t(\theta)$, where $\cL^t(\theta)=\sum_{h=1}^H \cL_h^t(\theta)$. We also let $\beta'=H\beta$, and it holds that  $$\cL^t(\theta^*) - \inf_{\theta'\in\Theta}\cL^t(\theta')= \sum_{h=1}^H \cL_h^t(\theta^*) - \cL_h^t(\hat\theta_\MLE) \le \sum_{h=1}^H \cL_h^t(\theta^*) - \inf_{\theta'\in\Theta}\cL_h^t(\theta') \le H \beta =\beta', $$ 
% which shows that the confidence set in \eqref{eq:behavior_model_confset} is valid with significance level $\beta'$.
% For the accuracy result, we note that \eqref{eq:MLE-guarantee-hellinger-2} holds for any $\theta\in\Theta$ and we just sum them up for all $h\in[H]$ and use the fact \begin{align*}
%     \sum_{i=1}^{t-1} \EE^{\pi^i }D_\H^2\bigrbr{\nu_h^{\pi^i, \theta}, \nu_h^{\pi^i, \theta^*}} 
%     &\le  \sum_{h=1}^H \sum_{i=1}^{t-1} \EE^{\pi^i }D_\H^2\bigrbr{\nu_h^{\pi^i, \theta}, \nu_h^{\pi^i, \theta^*}} \nend
%     & \le \frac {1}{2}\rbr{\cL_h^t(\theta) - \cL_h^t(\theta^*)} + H \log\rbr{\frac{d H\cN_\rho(\Theta, T^{-1})}{\delta}}\nend
%     &\le \frac 1 2 \rbr{\cL^t(\theta)-\cL^t(\hat\theta_\MLE)} + H \beta \le 2\beta', 
% \end{align*}
% for $\theta\in \{\theta:\cL^t(\theta)-\cL^t(\hat\theta_\MLE)\le \beta'$\}. 
% Therefore, the conclusion in \Cref{lem:bandit} just follows from \eqref{eq:MLE-guarantee-hellinger-2} of \Cref{lem:MLE-formal}.
% \fi}

Next, we give a proof on \Cref{eq:bandit-ub-1} 
% {\main\Cref{cor:MLE confset-linear myopic}\fi}{\neurips \eqref{eq:bandit-ub-1}\fi}
, which borrows \Cref{rmk:MLE-formal-myopic} and the covering number of $\Theta_h$ given in \eqref{eq:cN-Theta_h}.
\begin{remark}[Formal statement of \eqref{eq:bandit-ub-1}
    % {\ifmain \Cref{cor:MLE confset-linear myopic}\fi}{\ifneurips \eqref{eq:bandit-ub-1}\fi}
    ]\label{cor:formal-MLE confset-linear myopic}
    Consider the myopic and linear case, Suppose that $\beta\ge C d\log(H(1+\eta T^2 + (1+\eta)T)\delta^{-1})$ for some universal constant $C>0$,  \eqref{eq:MLE_guarantee_Q-1} further implies that for all $h\in[H]$
        \begin{align}
            \max\cbr{\bignbr{\hat\theta_h-\theta_h^*}_{\Sigma_{h, t}^{\theta^*}}^2, \bignbr{\hat\theta_h-\theta_h^*}_{\Sigma_{h, t}^{\hat\theta}}^2, 
            \EE^{\pi^i, M^*}\bignbr{\hat\theta_h-\theta_h^*}_{\Sigma_{h, t}^{\theta^*}}^2, 
            \EE^{\pi^i, M^*}\bignbr{\hat\theta_h-\theta_h^*}_{\Sigma_{h, t}^{\hat\theta}}^2 } \le 8 C_{\eta}^2 \beta,
            \label{eq:app-bandit-ub-1}
        \end{align}
        where $C_\eta =\eta^{-1}+B_A$, $B_A$ is specified in \eqref{eq:define_BA}, and $\Sigma_{h, t}^{\theta}$ is a data-dependent covariance matrix defined as 
        \begin{align}\label{eq:app-cov matrix}
            \Sigma_{h, t}^{\theta}= \sum_{i=1}^{t-1}
            {\Cov_{s_h^i}^{\pi^i, \theta}\bigsbr{\phi^{\pi^i}(s_h, b_h)}}, 
        \end{align}
    where $\Cov_{s_h}^{\pi, \theta}[\phi^{\pi}(s_h, b_h)]$ represents the covariance matrix of the feature $\phi^\pi$ with respect to $\nu_h^{\pi,\theta}(\cdot\given s_h)$.
    % \begin{proof}
    %     We use \eqref{eq:MLE_guarantee_Q-1} and obtain by noting that $(Q^{\pi^i, \theta}-Q^{\pi^i, \theta^*})(s, b) = \phi^{\pi^i}(s, b)^\top(\theta_h-\theta_h^*)$, which gives the result.
    % \end{proof}
\end{remark}

We next consider a special case where each trajectory in the offline data is independently collected. 
We have the following lemma for the MLE guarantee with independent dataset.


\begin{lemma}[{Confidence in Hellinger Distance with Independent Data}]
    \label{lem:MLE-indep-data}
    Suppose the conditions in \Cref{lem:MLE-formal} hold.
    Suppose each trajectory in the offline dataset $\cD=\{\tau^t\}_{t\in[T]}$ is independently collected. 
    For the confidence set $\CI_\Theta(\beta)$ defined in \Cref{eq:behavior_model_confset-1} with $\beta$ properly chosen according to \Cref{lem:MLE-formal},
    with probability at least $1-2\delta$, it holds that (i) $\theta^*\in\CI_\Theta(\beta)$; (ii) for any $\theta\in\CI_\Theta(\beta)$ , $h\in[H]$, 
    \begin{align*}
        \sum_{i=1}^T\EE_\cD\sbr{D_\H^2\rbr{\nu_h^{\pi^i, \theta}(\cdot\given s_h^i), \nu_h^{\pi^i, \theta^*}(\cdot\given s_h^i)}}  \le \cO(\beta),
    \end{align*}
    where the expectation is taken for  the randomness in both the trajectories and in the leader's policy choices.
    \begin{proof}
        See \ref{sec:proof-MLE-indep-data} for a detailed proof.
    \end{proof}
\end{lemma}
Up to now, we have obtain all the guarantees we  need from the MLE of the follower's quantal response.

\subsection{Learning Leader's Value Function}\label{sec:app-value function}
In this section, we study the problem of learning the leader's value function for both the offline and the online setting. 

\paragraph{Learning Leader's Value Function in Offline Setting.}
for each $\pi$ and estimated follower's response model $\theta$. 
%The results in this section mainly follows \citet{xie2021bellman,lyu2022pessimism}.
We only focus on myopic follower in this section. Recall the Bellman loss we defined in \Cref{sec:offline-myopic}, which is defined as
\begin{align*}
    &\ell_h(U_h', U_{h+1}, \theta, \pi) = \sum_{i=1}^T \rbr{U_h'(s_h^i, a_h^i, b_h^i) - u_h^i -   T^{\pi, \theta}_h U_{h+1}(s_{h+1}^i)}^2.
\end{align*}
where we define $ T_h^{\pi, \theta}U_{h+1}(s_{h+1}) = \inp[]{U_{h+1}(s_{h+1}, \cdot, \cdot)}{\pi_{h+1}\otimes \nu_{h+1}^{\pi, \theta}(\cdot, \cdot\given s_{h+1})}$. 
We aim to characterize the confidence set 
\begin{align*}
    \CI_{\cU}^{\pi,\theta}(\beta) = \cbr{U\in\cU: \ell_h(U_h, U_{h+1},\theta, \pi) - \inf_{U_h'\in\cU} \ell_h(U_h', U_{h+1}, \theta,\pi)\le \beta, \forall h\in[H]}.
\end{align*}
Recall the definition of the Bellman operator for the leader in \eqref{eq:bellman_operator_leader}. Similar to this definition, we define $\TT_h^{\pi,\theta}:\sF(\cS\times\cA\times\cB)\rightarrow \sF(\cS\times\cA\times\cB)$ as 
\begin{align*}
    \rbr{\TT_h^{\pi,\theta} f} (s_h, a_h, b_h) = u_h(s_h, a_h, b_h) + \EE_{s_{h+1}\sim P_h(\cdot\given s_h, a_h, b_h)} \sbr{\rbr{ T_{h+1}^{\pi,\theta}f}(s_{h+1})},
\end{align*}
and we add $\theta$ to the superscription to remind ourselves that the expectation within $\TT_h^{\pi,\theta}$ is taken with respect to the follower's quantal behavior guided by both $\pi$ and $\theta$.
In the sequel, we denote by $U^{\pi,\theta}$ the leader's U function defined similar to \eqref{eq:U_function} but with respect to policy $\pi$ and the response model $\theta$, 
\begin{align*}
    U_h^{\pi,\theta}(s_h, a_h, b_h) & = u_h(s_h, a_h, b_h) + \rbr{ P_h \circ T_{h+1}^{\pi,\theta} \circ U_{h+1}^{\pi,\theta}}(s_h, a_h, b_h) = \TT_h^{\pi,\theta} U_{h+1}^{\pi,\theta}(s_h, a_h, b_h).
\end{align*}
We clarify that $\theta$  only contains the estimated reward for myopic follower. 
Now, we present the following corollary on the validity and accuracy of the confidence set $\CI_\cU^{\pi,\theta}(\beta)$. 
\begin{lemma}[Confidence Set $\CI_{\cU}^{\pi,\theta}(\beta)$]\label{lem:leader-bellman-loss}
    Suppose that each trajectory in the data is independently collected and the function class $\cU$ satisfies the realizability and the completeness assumption given by \Cref{thm:Offline-MG}. Suppose we have 
    % \todo{
    $\beta \ge 
    {110 H^2\cdot\log(H \cN_\rho(\cY, T^{-1})\delta^{-1}) }  + (45 H^2 + 60 H )
  $, where the covering number is defined by \eqref{eq:cN-cY}.
%   }  
  Here, we have a joint class $\cY_h = \Theta_{h+1}\times \Pi_{h+1}\times \cU^2$ and the $\epsilon$-covering number $\cN_\rho(\cY_h,\epsilon)$ is with respect to the distance $\rho$ specified in \eqref{eq:rho-cY}.
%   following distance:
%   \begin{align*}
%     &\nbr{y-\tilde y}_\infty \nend
%     &\quad = \max_{h\in[H]}\cbr{\bignbr{U_h-\tilde U_h}_\infty,  \bignbr{U_{h+1}-\tilde U_{h+1}}_\infty, \sup_{s_{h+1}\in\cS}\bignbr{(\pi_{h+1}\otimes \nu_{h+1}^{\pi, \theta}-\tilde \pi_{h+1}\otimes \nu_{h+1}^{\tilde\pi, \tilde\theta})(\cdot, \cdot\given s_{h+1})}_\TV}.
% \end{align*}
Then for any $h\in[H], \theta\in\Theta, \pi\in\Pi$, we have with probability at least $1-\delta$: (i) $U^{\pi, \theta}\in \CI_\cU^{\pi,\theta}(\beta)$; (ii) for any $\tilde U\in\CI_\cU^{\pi,\theta}(\beta)$, $\EE_\cD[\|\tilde U_{h} - \TT_{h}^{\pi,\theta}\tilde U_{h + 1}\|^2]\le 4\beta T^{-1}$, where $\EE_\cD$ is the expectation taken with respect to the data generating distribution.

% \Zhuoran{Comment on linear case.} 


% $ \cN_{\mathrm{cov}} =  \cN  ( 1/T)$ 
    \begin{proof}
        See \Cref{sec:proof-leader-bellman-loss} for a detailed proof.
    \end{proof}
\end{lemma}


% \begin{corollary}[\textit{Offline guarantee for the confidence set of the leader's value function}]\label{cor:CI-U}
%     Suppose that each trajectory in the data is independently collected and the function class $\cU$ satisfies the realizability and the completeness assumption given by \Cref{thm:Offline-MG}.  By selecting $\beta = \epsilon_S$ where $\epsilon_S$ in given in \Cref{lem:leader-bellman-loss}, 
%     \begin{proof}
       
%     \end{proof}
% \end{corollary}

\paragraph{Learning Leader's Value Function in Online Setting.}
In this subsection, we provide gurantee for online learning the leader's value function. The analysis in this subsection maily follows \citet{jin2021bellman}.
Recall the online Bellman loss given by \eqref{eq:online-MG-bellman loss} in \Cref{sec:myopic-online}, 
\begin{align*}
    &\ell_h^t(U_h', U_{h+1}, \theta_{h+1}) \nend
    &\quad = \sum_{i=1}^{t-1}  \rbr{U_h'(s_h^i, a_h^i, b_h^i) - u_h^i -  \max_{\pi_{h+1}\in\sA}\inp[\Big]{U_{h+1}(s_{h+1}^i, \cdot, \cdot)}{\pi_{h+1}\otimes \nu_{h+1}^{\pi, \theta}(\cdot, \cdot\given s_{h+1}^i)}}^2.
\end{align*}
We define confidence set for each $\theta\in\Theta, t\in[T]$ as 
\begin{align*}
    \CI_\cU^{t, \theta}(\beta) = \cbr{U\in\cU: \ell_h^t(U_h, U_{h+1}, \theta_{h+1}, \pi) - \inf_{U'\in\cU_h} \ell_h^t(U', U_{h+1}, \theta_{h+1}, \pi)\le \beta, \forall h\in[H]}.
\end{align*}
In the sequel, we define $U^{*, \theta}$ as the optimal value function if the follower's true response model is $\theta$. 
Specifically, $U^{*, \theta}$ satisfies
\begin{align*}
    U_h^{*, \theta}(s_h, a_h, b_h) = u_h(s_h, a_h, b_h) + \bigrbr{\bigrbr{P_h\circ  T_{h+1}^{*, \theta}} U_{h+1}^{*, \theta}} (s_h, a_h, b_h), 
\end{align*}
where $ T_h^{*,\theta}:\cF(\cS\times\cA\times\cB)\rightarrow \cF(\cS)$ is the policy optimization operator defined as
\begin{align*}
     T_{h}^{*,\theta} f(s_h) = \max_{\pi_h\in\sA} \dotp{f(s_h,\cdot,\cdot)}{\pi_h\otimes \nu^{\pi, \theta}(\cdot, \cdot\given s_h)}.
\end{align*}
With respect to $ T^{*,\theta}$, we define the optimistic Bellman operator for the leader 
$\TT_h^{*,\theta}:\sF(\cS\times\cA\times\cB)\rightarrow \sF(\cS\times\cA\times\cB)$ as 
\begin{align*}
    \big(\TT_h^{*,\theta} f \bigr) 
    (s_h, a_h, b_h) = u_h(s_h, a_h, b_h) + \EE_{s_{h+1}\sim P_h(\cdot\given s_h, a_h, b_h)} \bigsbr{\bigrbr{ T_{h+1}^{*,\theta}f}(s_{h+1})}.
\end{align*}
We have the following guarantee on the confidence set.
\begin{lemma}[\textrm{Online guarantee for the confidence set of the leader's value function}]\label{lem:CI-U-online}
    For the online setting and the function class $\cU$ that satisfies the realizability and the completeness assumption given by \Cref{thm:Online-MG}. We consider a joint function class $\cZ_h=\cU^2\times\Theta_{h+1}$ and denote by $\cN_\rho(\cZ_{h}, \epsilon)$ the covering number of the smallest $\epsilon$-covering net for $\cZ_h$ with respect to this distance $\rho$ defined in \eqref{eq:rho-cZ}.
    By selecting $\beta \ge \epsilon_S=c H^2 \allowbreak \log(HT\cN_\rho(\cZ, T^{-1})\delta^{-1}) + (45 H^2 + 60 B_u)$ for some universal constant $c$ where the covering number is defined by \eqref{eq:cN-cZ}, for any $t\in[T], h\in[H], \theta\in\Theta$, we have with probability at least $1-\delta$: (i) $U^{*, \theta}\in \CI_\cU^{t,\theta}(\beta)$; (ii) for any $\tilde U\in\CI_\cU^{t,\theta}(\beta)$, $\sum_{i=1}^{t-1}\EE^{\pi^i}[\orbr{\orbr{\tilde U_{h} - \TT_{h}^{*,\theta}\tilde U_{h + 1}}(s_h, a_h, b_h)}^2]\le 4\beta$ and $\sum_{i=1}^{t-1} \orbr{\orbr{\tilde U_{h} - \TT_{h}^{*, \theta}\tilde U_{h + 1}}(s_h^i, a_h^i, b_h^i)}^2 \le 4 \beta$.
    \begin{proof}
        See \Cref{sec:proof-CI-U-online} for a detailed proof.
    \end{proof}
\end{lemma}




\subsection{Putting Everything Together:  Bounding Leader's Suboptimality}\label{sec:app-connection}

Here, we give a summary of the results presented in this section and show how the results in the previous parts are connected with each other for obtaining a guarantee of the suboptimality.

\vspace{5pt} 
{\noindent \bf Controlling Leader's Bellman Error.}
In \Cref{sec:app-subopt-decompose}, we study the suboptimality decomposition for the leader. From \Cref{lem:subopt-decomposition}, we learn that the suboptimality comprises two major terms, namely the leader's Bellman error and the follower's response error. The leader's Bellman error is simply given by 
\begin{align*}
    \text{Leader's Bellman error} = \sum_{h=1}^H \EE\sbr{\bigrbr{\tilde U_h - u_h}(s_h, a_h, b_h)-  T_{h+1}^{\pi,\tilde\nu} \tilde U_{h+1}(s_{h+1})} = \sum_{h=1}^H \EE\sbr{\tilde U_h - \TT_h^{\pi,\tilde\theta} \tilde U_{h+1}}, 
\end{align*}
where a list of definitions for $\TT_h$ and $ T_h$ can be found in \Cref{sec:app-notations}. For the offline setting with independent collected data, we use the guarantee from \Cref{lem:leader-bellman-loss} that $\EE_\cD[\|\tilde U_{h} - \TT_{h}^{\pi,\theta} \tilde U_{h + 1}\|^2]\le 4\beta T^{-1}$ if $\tilde U$ if properly chosen from the confidence set $\CI_\cU^{\pi,\theta}(\beta)$. 
For the online setting, we employ  \Cref{lem:CI-U-online} to show that $\sum_{i=1}^{t-1}\EE^{\pi^i}[\|\tilde U_{h} - \TT_{h}^{\theta}\tilde U_{h + 1}\|^2]\le 4\beta$ if $\tilde U$ is properly chosen such that  $\tilde U\in\CI_\cU^{t,\theta}(\beta)$. 
Moreover, in the online setting, $\pi$ is just the optimistic policy and $\TT_h^{\pi,\theta} = \TT_h^{\theta}$. Hence, the leader's Bellman error and the value function guarantee matches and we can control the leader's Bellman error by a distribution shift argument leveraging concentrability coefficients in the offline setting or via the eluder dimension of a proper function class that captures the complexity of this Bellman error in the online setting.
% \Zhuoran{How to control? via analysis of prediction eror and  distributional shift or offlien data (offline), eluder dimension of a proper function class that captures the complexity of the Bellman error (online)}
%how to control the leader's Bellman error is straightforward.

\vspace{5pt} 
{\noindent \bf Controlling Myopic Follower's Quantal Response Error.}
We first look at the easier setting where we aim to control a myopic follower's quantal response error, which is given by the TV distance betweem $\nu$ and $\tilde \nu$.
Recall from \Cref{cor:response-diff-myopic} that for a given state $s\in\cS$,
\begin{align*}
    \EE D_\TV\rbr{\nu(\cdot\given s), \tilde\nu(\cdot\given s)}
        &\le  \eta  \EE\sbr{\abr{(\tilde r^\pi(s, b) - r^\pi(s, b)) - \EE\bigsbr{\tilde r^\pi(s, b) - r^\pi(s, b)}}} \nend
        &\qquad + C^{(3)}\EE\sbr{\rbr{\rbr{\tilde r^\pi(s, b) -r^\pi(s, b)} - \EE\sbr{\tilde r^\pi(s, b) -r^\pi(s, b)}}^2}.
\end{align*}
If we look at the guarantee of MLE in \eqref{eq:MLE_guarantee_Q} of \Cref{lem:MLE-formal} for the myopic case, we directly have 
\begin{align*}
    \sum_{i=1}^{t-1} \EE^{\pi^i, M^*}{\Var_{s_h}^{\pi^i, \theta^*} \bigsbr{r^{\pi^i, \theta}(s, b) - r^{\pi^i, \theta^*}(s, b)}} \le  2 C_\eta^2 \beta, 
\end{align*}
for both the online and the offline cases, which gives control to both the first order and the second order terms in the TV distance upper bound.


\vspace{5pt} 
{\noindent \bf Controlling Farsighted Follower's Quantal Response Error.}
The last part is a more challenging case for a farsighted follower. Using the result in \Cref{lem:performance diff}, 
\begin{align*}
    \text{Quantal response error} \le C^{(0)}
    \sum_{h=1}^H \underbrace{\EE\sbr{\abr{\tilde \Delta^{(1)}_h(s_h, b_h)}}}_{\ds\text{1st-order error}}  + C^{(2)}
    \max_{h\in [H]} \underbrace{\EE\sbr{ \rbr{\tilde Q_h - r_h^\pi - \gamma P_h^\pi \tilde V_{h+1}}^2}}_{\ds\text{2nd-order error}}, 
\end{align*}
with $\tilde\Delta^{(1)}$ given by 
\begin{align*}
    \tilde \Delta^{(1)}_h(s_h, b_h) &=  \rbr{\EE_{s_h, b_h} -\EE_{s_h}}\Biggsbr{\sum_{l=h}^H \gamma^{l-h}\underbrace{\rbr{\tilde Q_l - r_l^\pi - \gamma P_l^\pi \tilde V_{l+1}}(s_l, b_l)}_{\ds\text{Follower's Bellman error}}}, 
\end{align*}
it is not straightforward to see how to bound these two terms by guarantee of the MLE in \Cref{lem:MLE-formal}. 
Fortunately,  we have the following two lemmas that  bound  the first-order error and the second-order error  separately.
\begin{lemma}[Bounding First-Order Error]\label{lem:1st-ub}
    For any $\pi\in\Pi$ and $(\tilde U, \tilde W, \tilde Q, \tilde V, \tilde A, \tilde \nu)$ satisfying the conditions in \Cref{lem:subopt-decomposition}, we have for all $h\in[H]$ that
    \begin{align*}
        \EE\sbr{\abr{\tilde \Delta^{(1)}_h(s_h, b_h)}} &\le   L^{(1)} \cdot
        \max_{h\in[H]} \EE \bigl [  D_\H(\nu_h(\cdot\given s_h),\tilde\nu_h(\cdot\given s_h)) \bigr ] , 
    \end{align*}
    where $L^{(1)} = 6(\eta^{-1}+2 B_A)\cdot {\eff_H\rbr{\gamma}}$ and  $\eff_H(\gamma) = \orbr{1-\gamma^H}/\orbr{1-\gamma}$ is the effective foresight of the follower. For the second order, we have
    \begin{align}\label{eq:1st-ub-2}
        &\bigrbr{\tilde \Delta_h^{(1)}(s_h, b_h)}^2  \nend
        &\quad \le 2 \rbr{\rbr{\EE_{s_h, b_h}-\EE_{s_h}} \bigsbr{\orbr{Q_h - \tilde Q_h}(s_h, b_h)}}^2 \\
        &\qqquad + 16 \gamma^2  \rbr{\eta^{-1} +2 B_A}^2\eff_H(\gamma) \sum_{l=h+1}^H \gamma^{l-h-1} {\rbr{\EE_{s_h}+\EE_{s_h, b_h}}\sbr{D_\H^2(\nu_l(\cdot\given s_l), \tilde\nu_l(\cdot\given s_l))}}. \notag 
    \end{align}
        \begin{proof}
            See \Cref{sec:proof-1st-ub} for a detailed proof.
        \end{proof}
    \end{lemma}
An important observation is that the first order term is bounded only by the follower's quantal response distance without invoking any transition model error, even for farsighted follower. This is because that the first order term only captures parts of the follower's response error. We next bound the second order term.
\begin{lemma}[Bounding Second-Order Error] \label{lem:2nd-ub}
    For any $\pi\in\Pi$ and $(\tilde U, \tilde W, \tilde Q, \tilde V, \tilde A, \tilde \nu)$ satisfying the conditions in \Cref{lem:subopt-decomposition}, we additionally assume $\tilde Q_h(s_h, b_h) = \tilde r_h^\pi(s_h, b_h) + (\tilde P_h^{\pi} \tilde V_{h+1})(s_h, b_h)$ for estimated reward $\tilde r$ and transition kernel $\tilde P$. Suppose that the follower's reward at each state satisfies a linear constraint $\dotp{x}{r_h(s_h, a_h, \cdot)} = \varsigma$ at every $(s_h,a_h)\in\cS\times\cA$ for some $x:\cB\rightarrow \RR$ such that $\inp{\ind}{x}\neq 0$ and $\varsigma\in\RR$. Define ratio $\kappa = \nbr{x}_\infty/|\inp{x}{\ind}|$. Then we have that
    \begin{align*}
        &\max_{h\in[H]}\EE\sbr{ \rbr{\rbr{\tilde Q_h - r_h^\pi - \gamma P_h^\pi \tilde V_{h+1}}(s_h, b_h)}^2} \nend
        &\quad \le L^{(2)} \max_{h\in[H]}\cbr{\EE D_\H^2(\nu_h(\cdot\given s_h),\tilde\nu_h(\cdot\given s_h))+\EE D_\TV^2(P_h^\pi(\cdot\given s_h, b_h),\tilde P_h^\pi(\cdot\given s_h, b_h))}, 
    \end{align*}
    where $L^{(2)} = c H^2 \eff_H(c_2)^2 \kappa^2 \exp\rbr{8\eta B_A} (\eta^{-1}+B_A)^2$ for some absolute constant $c>0$, and 
    $$c_2 = \gamma \rbr{2\exp\rbr{2\eta B_A}+\kappa\exp\rbr{4\eta B_A} }.$$
    \begin{proof}
        See \Cref{sec:proof-2nd-ub} for a detailed proof.
    \end{proof}
\end{lemma}

In the farsighted follower case, we will intensively turn to these two lemmas to control the first- and second-order terms both online and offline. Specifically, the first term in \eqref{eq:1st-ub-2} can be controlled by \eqref{eq:MLE-guarantee-Q-3} in \Cref{lem:MLE-formal}, while the second term is just the Hellinger distance, which can be controlled by our guarantee in \eqref{eq:MLE-guarantee-hellinger-1}. The argument for \Cref{lem:2nd-ub} is quite the same, while both the Hellinger distance of the quantal response and the TV distance of the transition kernel is controllable, as we will see in \Cref{lem:MLE}.

% \subsection{Proof for \Cref{sec:app-major-tech}}

% \subsubsection{Proof of \Cref{lem:performance diff}}\label{sec:proof-performance diff}
% 
In the following, we prove \Cref{lem:performance diff}, which relates the estimation error of quantal response policy to a few estimation errors involving the follower's value functions. 
To simplify the notation, we let $\nu$, 
$Q, V, A$ denote $\nu$, $Q^{\pi}$, $V^{\pi}$, and $A^{\pi}$, respectively, which are quantities computed under the true model $M^*$.  
Note that we have  
$\nu_h(b\given s) = \exp(\eta \cdot A_h (s, b) )$ and 
$\tilde \nu_h (b \given s) = \exp(\eta \cdot \tilde A_h (s,b) )$.
By the upper bound in \eqref{eq:nu-tv-ub-0} of \Cref{lem:response diff},
we have 
\begin{align}
    \dr{(i)} &\defeq \sum_{h=1}^H  H \cdot  \EE \bigsbr{ \nbr{\tnu_h(\cdot\given s_h)-\nu_h(\cdot\given s_h)}_1} = \sum_{h=1}^H  2 H \cdot  \EE \bigsbr{  D_\TV \bigrbr{\nu(\cdot\given s_h), \tilde \nu(\cdot\given s_h)}} \nend
    &\le 2 \eta  H \cdot  \underbrace{\sum_{h=1}^H \EE\bigsbr{\bigabr{\orbr{A_h-\tilde A_h}(s_h, b_h)}}}_{\dr (ii)} \nend
    &\qqquad + \eta^2  H \cdot \sum_{h=1}^H \EE\Bigsbr{ \exp\bigrbr{\eta\bigabr{\orbr{A_h-\tilde A_h}(s_h, b_h)}}\cdot \bigabr{\orbr{A_h-\tilde A_h}(s_h, b_h)}^2}. \label{eq:(i)}
\end{align}
% In this section, we characterize the performance difference that arises from model misspecification, namely the difference in the leader's total reward under a given policy $\pi$ for a misspecified model $\tilde M$ against the true model $M^*$. In the following, we use $(Q_h, V_h, A_h, \nu_h, U_h, W_h)$ for the follower/leader under the true model $M^*$, and $(\tQ_h,\tV_h,\tA_h, \tnu_h, \tilde U_h, \tilde W_h)$ for the follower/leader under the alternative model $\tilde M$. 
% We let
% \begin{align}
%     \dr{(i)} &\defeq \sum_{h=1}^H  H \EE \nbr{\tnu_h(\cdot\given s_h)-\nu_h(\cdot\given s_h)}_1\nend
%     &\le 2 \eta  H \underbrace{\sum_{h=1}^H \EE\sbr{\abr{\rbr{A_h-\tilde A_h}(s_h, a_h)}}}_{\dr (ii)} \nend
%     &\qqquad + \eta^2  H \sum_{h=1}^H \EE\sbr{ \exp\rbr{\eta\abr{\rbr{A_h-\tilde A_h}(s_h, a_h)}}\cdot \abr{\rbr{A_h-\tilde A_h}(s_h, a_h)}^2}, \label{eq:(i)}
% \end{align}
%where the first inequality holds by noting that $\bignbr{\tilde U_h}_\infty \le  H$, and the second inequality uses the upper bound for the TV distance between two difference logistic responses given by \Cref{lem:response diff}.
In the following, we let 
\$
    \tilde\Delta_h^{(1)} (s_h, b_h)&\defeq  \rbr{\EE_{s_h, b_h} - \EE_{s_h}}\sbr{\sum_{l=h}^H \gamma^{l-h}\bigrbr{\orbr{\tilde Q_l - r_l^\pi - \gamma P_l^\pi \tilde V_{l+1}}(s_l, b_l)}}, \\
    \tilde\Delta_h^{(2)}(s_h) &\defeq \EE_{s_h}\sbr{\sum_{l=h}^H \gamma^{l-h} \kl\infdivx[\big]{\nu_l(\cdot\given s_l)}{\tilde\nu_l(\cdot\given s_l)}}. 
\$
We note that we denote $\EE_{z} [\cdot]=\EE^{\pi,M^*}[\cdot\given z]$ for any variable $z$.
Here the expectations in $\tilde\Delta^{(1)}_h$ and $\tilde\Delta^{(2)}_h$ are taken with respect to the randomness of the trajectory generated by $\{ \pi, \nu^{\pi}\}$, given $s_h$ or $(s_h, b_h)$.
%Note that $\tilde\Delta^{(1)}_h$ is actually a short hand of $\tilde\Delta^{(1)}_{h,\pi, M}(s_h, b_h)$.
We can further bound   (ii) defined in \eqref{eq:(i)} by invoking \Cref{lem:AQV-func diff}, 
which implies that 
\$
\dr{(ii)}&= \sum_{h=1}^H \EE\sbr{\abr{\rbr{\EE_{s_h, b_h}-\EE_{s_h}} \bigsbr{\tilde\Delta_h^{(1)}(s_h, b_h) - \gamma\eta^{-1}\tilde\Delta_{h+1}^{(2)} (s_{h+1})} + \eta^{-1}\kl\infdivx[\big]{\nu_h(\cdot\given s_h)}{\tilde \nu_h(\cdot\given s_h)}}}.
\$
By the law of total expectation, 
we have 
\$
\EE \Bigsbr{\EE _{s_h, b_h} \bigsbr{   \tilde\Delta_{h+1}^{(2)} (s_{h+1})}} = \EE \Bigsbr{\EE _{s_h } \bigsbr{   \tilde\Delta_{h+1}^{(2)} (s_{h+1})}}  \geq 0.
\$
Besides, by the definition of $\tilde \Delta_h^{(2)}$, we have 
\$
\EE\bigsbr{\tilde\Delta_h^{(2)}(s_h)} = \EE\bigsbr{ \kl\infdivx[\big]{\nu_h(\cdot\given s_h)}{\tilde \nu_h(\cdot\given s_h)} + \gamma \cdot \tilde\Delta_{h+1}^{(2)} (s_{h+1}) }. 
\$ 
Thus, by triangle inequality, we have 
\begin{align}
    \dr{(ii)}  &\le \sum_{h=1}^H\EE\Bigsbr{\bigabr{\rbr{\EE_{s_h, b_h}-\EE_{s_h}}\bigsbr{\tilde\Delta_h^{(1)}(s_h, b_h)}}} + 2\eta^{-1} \sum_{h=1}^H\EE\bigsbr{\tilde\Delta_h^{(2)}(s_h)}, \label{eq:(ii)}
\end{align}
Furthermore, for the second term on the right-hand side of \eqref{eq:(ii)}, we  apply the inequality between KL divergence and the $\chi^2$ divergence to each term $\kl\infdivx[]{\nu_l(\cdot\given s_l)}{\tilde\nu_l(\cdot\given s_l)}$ and obtain that 
\begin{align}
\kl\infdivx[\big]{\nu_l(\cdot\given s_l)}{\tilde\nu_l(\cdot\given s_l)}
&\le \chi^2\infdivx[\big]{\nu_l(\cdot\given s_l)}{\tilde\nu_l(\cdot\given s_l)}\nend
&= \inp[\Bigg]{\nu_l(\cdot\given s_l)}{\biggrbr{\sqrt{\frac{\nu_l(\cdot\given s_l)}{\tilde\nu_l(\cdot\given s_l)}}-\sqrt{\frac{\tilde\nu_l(\cdot\given s_l)}{\nu_l(\cdot\given s_l)}}}^2}_{\cB}\nend
&\le \eta^2 \cdot \EE_{s_l}\sbr{\exp\bigrbr{\eta  \cdot \bigabr{\orbr{A_l-\tilde A_l}(s_l,b_l)}}\cdot \bigrbr{\orbr{A_l -\tilde A_l}(s_l, b_l)}^2}, \label{eq:kl-ub}
\end{align}
where last expectation is with respect to $b_ l \sim \nu_{l } (\cdot \given s_l)$. 
Here the inequality holds by noting that 
 $\sqrt{\nu_l/\tilde \nu_l}=\exp(\eta(A_l-\tilde A_l)/2)$ and the basic inequality $| \exp( x ) - \exp(y)| \leq \exp ( | x-y|) \cdot |x -y|$.
Plugging \eqref{eq:kl-ub} back into \eqref{eq:(ii)}, we obtain
\begin{align}
    \dr{(ii)} &\le \sum_{h=1}^H\EE\Bigsbr{\bigabr{\rbr{\EE_{s_h, b_h}-\EE_{s_h}}\bigsbr{\tilde\Delta_h^{(1)}(s_h, b_h)}}}  + 2\eta^{-1} \sum_{h=1}^H \EE\sbr{\sum_{l=h}^H \gamma^{l-h} \cdot \kl\infdivx[\big]{\nu_l(\cdot\given s_l)}{\tilde\nu_l(\cdot\given s_l)}}\nend
    &\le \sum_{h=1}^H\EE\sbr{\bigabr{\rbr{\EE_{s_h, b_h}-\EE_{s_h}}\bigsbr{\tilde\Delta_h^{(1)}(s_h, b_h)}}} \nend
    &\qquad + 2\eta \sum_{h=1}^H \sum_{l=h}^{H} \gamma^{l-h} \cdot  \EE\sbr{\exp\bigrbr{\eta  \cdot \bigabr{\orbr{A_l-\tilde A_l}(s_l,b_l)}}\cdot \bigabr{\orbr{A_l -\tilde A_l}(s_l, b_l)}^2}\nend
    &\le \sum_{h=1}^H\EE\sbr{\bigabr{\rbr{\EE_{s_h, b_h}-\EE_{s_h}}\bigsbr{\tilde\Delta_h^{(1)}(s_h, b_h)}}} \nend
    &\qquad + \frac{2\eta(1-\gamma^H)}{1-\gamma}\cdot \sum_{h=1}^H  \EE\sbr{\exp\bigrbr{\eta  \cdot \bigabr{\orbr{A_h-\tilde A_h}(s_h,b_h)}}\cdot \bigabr{\orbr{A_h -\tilde A_h}(s_h, b_h)}^2} .
     \label{eq:(ii)-2}
\end{align}
Recall that we  define $\eff_H(x) = (1-x^H)/(1-x)$ as the \say{effective}  horizon with respect to $x$.


Plugging \eqref{eq:(ii)-2} back into \eqref{eq:(i)}, we conclude that
\begin{align}
    \dr{(i)}&\le 2\eta  H \cdot \sum_{h=1}^H  \EE\sbr{\abr{\rbr{\EE_{s_h, b_h}-\EE_{s_h}}\bigsbr{\tilde\Delta_h^{(1)}(s_h, b_h)}}}   \nend
    &\qquad + \eta^2  H  \bigrbr{1+ 4  \cdot \eff_H(\gamma) } \cdot \sum_{h=1}^H  \EE\sbr{\exp\bigrbr{\eta  \cdot \bigabr{\orbr{A_h-\tilde A_h}(s_h,b_h)}}\cdot \bigabr{\orbr{A_h -\tilde A_h}(s_h, b_h)}^2}  . \label{eq:(ii)-21}
\end{align}
Note that we define  $C^{(1)}$ in \eqref{eq:define_constants}. 
Since $\oabr{\orbr{A_h-\tilde A_h}(s_h,b_h)} \leq 2 B_{A}$, 
by  \eqref{eq:(ii)-21}  and inequality 
\$
\exp\bigrbr{\eta  \cdot \bigabr{\orbr{A_h-\tilde A_h}(s_h,b_h)}}  \leq \exp(2\eta B_{A}), 
\$
we conclude the proof of \eqref{eq:taylor-myopic}. 

It remains to prove \eqref{eq:taylor-farsighted}. 
Notice that 
\#\label{eq:f_2-01}
\begin{split}
    V_h (s_h) = \max_{\nu' \in \Delta (\cB) }\bigl\{ \inp{\nu' }{Q_h (s_h, \cdot )}_{\cB } +\eta^{-1} \cH(\nu')\bigr\} , \\
    \tilde V_h (s_h) = \max_{\nu' \in \Delta (\cB) }\bigl\{ \inp{\nu' }{\tilde Q_h (s_h, \cdot )}_{\cB } +\eta^{-1} \cH(\nu')\bigr\} ,  
\end{split}
\#
where the maximizers are $\nu_h (\cdot \given s_h)$ and $\tilde \nu_h (\cdot \given s_h)$, respectively.
Then, by  \eqref{eq:f_2-01}
we have 
\#
& \bigabr{\orbr{V_h-\tilde V_h}(s_h)}   \notag \\
& \quad \le \max\Bigcbr{\inp[\big]{\nu_h(\cdot \given s_h) }  { \bigabr{Q_h(s_h, \cdot )-\tilde Q_h(s_h, \cdot )} }_{\cB} }, ~\Bigabr{\inp[\big]{\tilde\nu_h(\cdot \given s_h) }{\bigabr{ Q_h(s_h, \cdot )-\tilde Q_h(s_h, \cdot )} }_{\cB} } \notag\\
& \quad  =   \max\Bigcbr{  \EE_{s_h} \bigl [  \bigl | (Q_h - \tilde Q_h) (s_h, b_h ) \big | \bigr ] , ~   \EE_{s_h} \bigl [ \big |  (Q_h - \tilde Q_h)  (s_h, b_h )\bigr |  \cdot \tilde \nu_h (b_h \given s_h) / \nu_h (b_h \given s_h ) \bigr ]  }   \notag \\
& \quad  \leq    \exp(2 \eta B_A)  \cdot    \EE_{s_h} \bigl [ \big |  (Q_h - \tilde Q_h) (s_h, b_h )\big | \bigr ]   , \label{eq:f_2-1} 
\# 
where the expectation is taken with respect to $b_h \sim \nu_h (\cdot \given s_h)$. 
Here the first inequality is obtained from the optimality condition of \eqref{eq:f_2-01}, and the  last inequality holds because  $\nbr{\tilde\nu_h/\nu_h}_\infty \le \exp(2\eta B_A)$. 
Note that $\tilde A = \tilde Q - \tilde V$ and $A = Q - V$.
By triangle inequality, we have 
\begin{align}
    &\abr{\orbr{A_h-\tilde A_h}(s_h, b_h)}  
  \le  \bigabr{\orbr{Q_h-\tilde Q_h}(s_h, b_h)} + \exp\orbr{2\eta B_A} \cdot  \EE_{s_h} \bigl [  \big |  (Q_h - \tilde Q_h) (s_h, b_h )\big |  \bigr ]  .  \label{eq:f_2-11} 
\end{align}
%   
%  
%  
% For the left side, we first notice 
% \begin{align*}
%     \EE_{s_h}\rbr{A_h^\pi-\tilde A_h^\pi}^2 
%     &= \EE_{s_h}\sbr{\rbr{\rbr{\EE_{s_h, b_h} -\EE_{s_h}}\sbr{A_h^\pi-\tilde A_h^\pi}}^2 }+ \rbr{\EE_{s_h}\sbr{A_h^\pi-\tilde A_h^\pi}}^2\nend
%     & = \EE_{s_h}\sbr{\rbr{\rbr{\EE_{s_h, b_h}-\EE_{s_h}}\sbr{\tilde\Delta_h^{(1)}(s_h, b_h) - \gamma\eta^{-1}\tilde\Delta_{h+1}^{(2)} (s_{h+1})}}^2} + \rbr{\EE_{s_h}\abr{A_h^\pi-\tilde A_h^\pi}}^2,
% \end{align*}
% where the first equality follows from a standard mean-variance decomposition, and the inequality holds by \eqref{eq:A diff-1} in \Cref{lem:AQV-func diff} and noting that $\eta^{-1}\kl\infdivx{\nu_h}{\tilde\nu_h} = \EE_{s_h, b_h}\bigsbr{A_h-\tilde A_h}$.
Now for $\oabr{   (Q_h - \tilde Q_h) (s_h, b_h ) }$, by the Bellman equation $Q_h = r^{\pi}_h + \gamma P_h^{\pi} V_{h+1}$, we have  
\begin{align}
    &\bigabr{ \orbr{Q_h-\tilde Q_h}(s_h, b_h)} \nend
    &\quad \le  
    \bigabr{ 
        \orbr{\tilde Q_h - r_h^\pi - \gamma P_h^\pi \tilde V_{h+1}}(s_h, b_h)
    } 
         + \gamma  \cdot \bigabr{\bigrbr{P_h^\pi \orbr{V_{h+1}-\tilde V_{h+1}}}(s_h, b_h)}
         \nend
    &\quad \le \bigabr{\orbr{\tilde Q_h - r_h^\pi - \gamma P_h^\pi \tilde V_{h+1}}
    (s_h, b_h)} 
    + \gamma \cdot \exp\rbr{2\eta B_A}\EE_{s_h, b_h}\bigsbr{\oabr{Q_{h+1}^\pi-\tilde Q_{h+1}^\pi}},\label{eq:f_2-Q-ub}
\end{align}
where the first inequality holds by a standard decomposition and the second inequality is obtained by applying the  same upper bound for $V_{h}-\tilde V_h$ in \eqref{eq:f_2-1} to $V_{h+1} - \tilde V_{h+1}$. 
By recursion, we have 
\begin{align}
    \bigabr{\orbr{Q_h-\tilde Q_h}(s_h, b_h)} &\le  \sum_{l=h}^H \big (\gamma \cdot \exp(2\eta B_A  ) \big) ^{l-h} \cdot  \EE_{s_h, b_h}\bigsbr{\bigabr{\orbr{\tilde Q_l - r_l^\pi - \gamma P_l^\pi \tilde V_{l+1}}(s_l, b_l)}}.\label{eq:f_2-Q-telo}
    % \nend
    % &\qqquad +  \sum_{l=h}^H \exp(2\eta B_A (l-h+1)) \gamma^{l-h} \rbr{\EE_{s_h}\abr{r_l^\pi-\tilde r_l^\pi} + \gamma \EE_{s_h}\abr{\rbr{P_l^\pi-\tildeP_l^\pi}\tilde V_{h+1}}}.
\end{align}
Now, by the boundedness of $A_h$ and $\tilde A_h$, and \eqref{eq:f_2-11}, 
we have 
\$
& \EE\sbr{\exp\bigrbr{\eta  \cdot \bigabr{\orbr{A_h-\tilde A_h}(s_h,b_h)}}\cdot \bigabr{\orbr{A_h -\tilde A_h}(s_h, b_h)}^2} \notag \\
& \quad \leq \exp\bigrbr{2\eta B_A}\cdot \EE\bigsbr{ \bigabr{\orbr{A_h-\tilde A_h}(s_h, a_h)}^2}\nend
&\quad \le 2\exp\bigrbr{6\eta B_A}\cdot {\EE\bigsbr{ \bigabr{\orbr{\tilde Q_h - Q_h}(s_h, b_h)}^2}},
\$ 
where in the  last inequality we use the basic inequality $(a + b)^2 \leq 2 a ^2 + 2 b^2 $. 
Combining  with \eqref{eq:f_2-Q-telo}, we obtain that 
\begin{align*}
    &\EE\sbr{\exp\bigrbr{\eta  \cdot \bigabr{\orbr{A_h-\tilde A_h}(s_h,b_h)}}\cdot \bigabr{\orbr{A_h -\tilde A_h}(s_h, b_h)}^2} \nend
    &\quad \le 2\exp\bigrbr{6\eta B_A} \cdot \EE \biggsbr { \biggrbr{\sum_{l=h}^H (\gamma \cdot \exp(2\eta B_A  ) \big) ^{l-h} \cdot  \EE_{s_h, b_h}\bigsbr{\bigabr{\orbr{\tilde Q_l - r_l^\pi - \gamma P_l^\pi \tilde V_{l+1}}(s_l, b_l)}}}^2 } \nend
    &\quad \le 2\exp\orbr{6\eta B_A} \cdot \bigrbr{\eff_H(\exp\orbr{2\eta B_A}\gamma)}^2  \cdot \max_{l\in\{h, \dots, H\}}\EE\bigsbr{\bigabr{\orbr{\tilde Q_l - r_l^\pi - \gamma P_l^\pi \tilde V_{l+1}}(s_l, b_l)}^2}.
\end{align*}
Recall that we define  $C^{(2)} = 2 \eta^2  H^2 \cdot \exp\rbr{6\eta B_A}  \cdot \rbr{1+ 4 \eff_H(\gamma)} \cdot \rbr{\eff_H(\exp\cbr{2\eta B_A}\gamma)}^2$. By \eqref{eq:(ii)-21}, we establish \eqref{eq:perform-diff-linear}. 
Therefore, we 
  complete the proof of \Cref{lem:performance diff}.

% \subsubsection{Proof of \Cref{cor:response-diff-myopic}}
% \label{sec:proof-response-diff-myopic}
% 


Since we consider any fixed state $s \in \cS$, in this proof, we omit $s$ 
to simplify the notation. 
To apply Lemma \ref{lem:performance diff}, 
we note that 
$Q$ and $\tilde Q$ in   Lemma \ref{lem:performance diff} becomes $r^{\pi}$ and $\tilde r^{\pi}$ in the myopic case. 
To make the proof consistent with that of Lemma \ref{lem:performance diff}, we use notation $\{ Q, \tilde Q, V, \tilde V, A, \tilde A\}$ in the sequel. 

 
To begin with, 
we invoke \eqref{eq:nu-tv-ub-0} in \Cref{lem:response diff} and obtain that 
\begin{align*}
    D_\TV\orbr{\nu, \tilde \nu}
    &\le \eta \cdot \inp[\big]{\nu} {\oabr{\tilde A-A} + \frac \eta 2 \exp\bigrbr{\eta\oabr{\tilde A-A}} \cdot \orbr{\tilde A-A}^2}_{\cB } \nend
    &\le \eta \cdot \inp[\big]{\nu} {\oabr{\tilde A-A} + \frac \eta 2 \exp\orbr{2\eta B_A} \orbr{\tilde A-A}^2}_{\cB }\nend
    &= \eta \cdot \EE\bigsbr{\oabr{\tilde A-A}} + \frac {\eta^2} {2} \exp\bigrbr{2\eta B_A} \cdot \Bigrbr{\Var\orbr{\tilde A-A}+ \bigrbr{\EE\osbr{\tilde A-A}}^2},
\end{align*}
where the last equality holds by the  variance-mean decomposition. 
Here the expectation and variance are taken with respect to $   b \sim \nu(\cdot \given s ) $. 
Now, using \Cref{lem:AQV-func diff} to the myopic case, we have
\begin{align*}
    \EE\bigsbr{\oabr{\tilde A - A}} 
    &\le \EE\bigsbr{\bigabr{(\tilde Q - Q) - \EE  \osbr{\tilde Q - Q}}} + \eta^{-1}\kl\infdivx[]{\nu}  {\tilde\nu}.  
\end{align*}
For the variance term, we have
\begin{align*}
    \Var\orbr{\tilde A -A} & = \EE\bigsbr{\bigrbr{\orbr{\tilde A -A} - \EE\orbr{\tilde A -A}}^2}  = \EE\bigsbr{\bigrbr{\orbr{\tilde Q -Q} - \EE\osbr{\tilde Q -Q}}^2}, 
\end{align*}
where the last equality holds because $V $ and $\tilde V $ do not involve $b $. 
Furthermore, 
note that $$\eta\EE\osbr{A-\tilde A} = \kl\infdivx[]{\nu}{\tilde\nu} \leq 2 \eta B_A.$$ 
Thus, combining the inequalities above, we have 
  we have
\begin{align}
    & D_\TV(\nu,\tilde\nu)\nend 
    & \quad \le   \eta \cdot   \EE\bigsbr{\bigabr{(\tilde Q - Q) - \EE  \osbr{\tilde Q - Q}}} + \frac{\eta^2}{2} \cdot \exp\rbr{2\eta B_A} \cdot \EE\bigsbr{\bigrbr{\orbr{\tilde Q -Q} - \EE\osbr{\tilde Q -Q}}^2}\nend
    &\qquad + \kl\infdivx[]{\nu}{\tilde\nu} + \exp\rbr{2\eta B_A}/ 2 \cdot  \bigrbr{\kl\infdivx[]{\nu}{\tilde\nu}}^2\nend
    &\quad \le\eta \cdot   \EE\bigsbr{\bigabr{(\tilde Q - Q) - \EE  \osbr{\tilde Q - Q}}}  + \frac{\eta^2}{2} \cdot \exp\rbr{2\eta B_A} \cdot  \EE\bigsbr{\bigrbr{\orbr{\tilde Q -Q} - \EE\osbr{\tilde Q -Q}}^2}\nend
    &\qquad + \bigrbr{1 + \eta B_A \exp\rbr{2\eta B_A} }\cdot \kl\infdivx[]{\nu}{\tilde\nu}, \label{eq:TV-ub-taylor}
\end{align}
where the  last inequality holds by noting that $\kl\infdivx[]{\nu}{\tilde\nu}\le 2\eta B_A$.
% Now, suppose that $r$ is parameterized by $\theta$ and $\tilde r$ is parameterized by $\tilde\theta$.

In the following, we  handle the KL divergence term.
We calculate the derivative of $\eta^{-2}\kl\infdivx{\nu}{\tilde\nu}$ with respect to $\tilde Q$ and obtain
\begin{align*}
    \partial_{\tilde Q}\rbr{\eta^{-2}\kl\infdivx{\nu}{\tilde\nu}} = \eta^{-1}\partial_{\tilde Q} \bigrbr{\EE\osbr{A-\tilde A}} = \eta^{-1}{\bigrbr{\partial_{\tilde Q} \tilde V - \nu}} = \eta^{-1}\rbr{\tilde \nu - \nu},
\end{align*}
where $\ind$ denote the all one vector of length $|\cB|$ is $\cB$ is discrete.
Here the first equality follows from $\eta\EE\osbr{A-\tilde A} = \kl\infdivx[]{\nu}{\tilde\nu}$, the second equality holds because $\nu$ and $A$ do not depend on $\tilde Q$, and $\tilde A = \tilde Q - \tilde V$. 
Moreover, the last equality holds because 
$$\tilde V(s)  = \eta^{-1} \log \bigg(\sum_{b \in \cB} \exp \big( \eta \cdot  \tilde Q(s, b)\bigr) \biggr), $$
and also $\partial_{\tilde Q}\EE[\tilde Q] = \nu$.
We further take a second-order derivative and obtain
\begin{align*}
    \partial^2_{\tilde Q \tilde Q} \rbr{\eta^{-2}\kl\infdivx{\nu}{\tilde\nu}} = \eta^{-1}\partial_{\tilde Q} \tilde \nu = \diag(\tilde \nu) -\tilde \nu \tilde\nu^\top\eqdef \H, 
\end{align*}
where the last equality holds for the vector case. 
% For a continuous action space, we have the Hessian represented as $\partial^2_{\tilde Q \tilde Q} \rbr{\eta^{-2}\kl\infdivx{\nu}{\tilde\nu}}(b, b') = \delta(b-b') - \tilde \nu(b)\tilde\nu(b')$. For simplicity, we just stick to the notation for the discrete case while the generalization to the continuous case is just a matter of change of notations. 
Note that the Hessian is upper and lower bounded by $\L$ where $\L = \diag(\nu )-\nu \nu^\top$, which is proved by  the following proposition. 
\begin{proposition}\label{prop:Hessian-ulb}
Let $\H = \diag(\tilde\nu) -\tilde\nu \tilde\nu^\top$ and $\L=\diag(\nu)-\nu \nu^\top$ where $\nu=\exp\orbr{\eta A}$ and $\tilde\nu=\exp\orbr{\eta \tilde A}$ are two quantal response over $\cB$ with $\nbr{A}_\infty\le B_A, \onbr{\tilde A}_\infty\le B_A$. Then   for any vector  $g\in \RR^{|\cB| }$, we have  
\begin{align}
    \exp\rbr{2 \eta B_A} \cdot g^\top \L g \ge x^\top \H x \ge \exp\rbr{-2 \eta B_A} \cdot g^\top \L g.\label{eq:Hessian ub lb}
\end{align}
\end{proposition}
\begin{proof}
Note that $\exp\rbr{-2 \eta B_A}\le  \tilde \nu(b) / \nu(b) \le\exp\rbr{ 2 \eta B_A}$ for any $b\in \cB$. 
Let $\EE^{\nu}$ and $\Var^{\nu}$ denote the expectation and variance under distribution $\nu$. 
Then we have 
\begin{align*}
    g^\top \L g & = \Var^\nu[g(b)]
   = \EE^\nu\bigsbr{\bigrbr{g(b) - \EE^\nu[g(b)]}^2},\notag \\
   g^\top \H g & = \Var^{\tilde \nu}[g(b)]
   = \EE^{\tilde \nu}\bigsbr{\bigrbr{g(b) - \EE^\nu[g(b)]}^2}.
\end{align*}
By direct computation, we have 
\begin{align*}
    & \exp\rbr{-\eta B_A} \cdot \EE^{\tilde\nu}\bigsbr{\bigrbr{g(b) - \EE^{\tilde\nu}[g(b)]}^2}
    \notag \\
    & \quad \le \exp\rbr{-\eta B_A}\cdot \EE^{\tilde\nu}\bigsbr{\bigrbr{g(b) - \EE^{\nu}[g(b)]}^2} 
     \le \EE^\nu\sbr{\rbr{g(b) - \EE^\nu[g(b)]}^2} 
    %%%%%%%%
\end{align*}
where the first inequality is true because changing $\EE^{\tilde\nu}[g(b)]$ to $\EE^{ \nu}[g(b)]$ incurs additional bias, and the second inequality is true because $\tilde \nu (b) / \nu(b)$ 
Similarly, we have 
\begin{align*}
    %%%%%%%%
     \EE^\nu\sbr{\rbr{g(b) - \EE^\nu[g(b)]}^2} 
    %%%%%%%%
    &\le \EE^{\nu}\sbr{\rbr{g(b) - \EE^{\tilde\nu}[g(b)]}^2}  
    %%%%%%%%
     \le  \exp\rbr{\eta B_A}  \cdot \EE^{\tilde\nu}\sbr{\rbr{g(b) - \EE^{\tilde\nu}[g(b)]}^2}.
\end{align*}
Therefore, we conclude that \eqref{eq:Hessian ub lb} holds. 
\end{proof}


Using the lower bound in \eqref{eq:Hessian ub lb}, we have for the KL divergence that
\begin{align*}
    \eta^{-2}\kl\infdivx[]{\nu}{\tilde\nu} &\le 1/2 \cdot (\tilde Q - Q)^\top  \H (\tilde Q - Q)  \le  \exp\rbr{2\eta B_A}/ 2 \cdot (\tilde Q - Q)^\top  \L (\tilde Q - Q) \nend
    &= \frac{\exp\rbr{2\eta B_A}}{2} \cdot (\tilde Q - Q)^\top  \bigrbr{\diag(\nu)-\nu\nu^\top} (\tilde Q - Q) , 
\end{align*}
where the first inequality holds by noting that the derivative of the KL-divergence at $\tilde\nu=\nu$ is zero, and we upper bound the KL-divergence  only by the second order term. Furthermore, the second inequality holds because  $\H\preceq \exp(2\eta B_A)\cdot \L$, which is proved  by \Cref{prop:Hessian-ulb}. 
% where the last inequality holds by applying \eqref{eq:Hessian ub lb} to the Hessian of the KL divergence evaluated at $\nu$. 
%Note that one can also plug in $\tilde\nu$ in the last inequality. 
Therefore, we conclude for \eqref{eq:TV-ub-taylor} that
\begin{align*}
    D_\TV \rbr{\nu, \tilde \nu} &\le \eta \cdot   \EE\bigsbr{\bigabr{(\tilde Q - Q) - \EE\osbr{\tilde Q - Q}}} + \frac{\eta^2}{2} \exp\rbr{2\eta B_A} \cdot \EE\bigsbr{\bigrbr{\orbr{\tilde Q -Q} - \EE\osbr{\tilde Q -Q}}^2}\nend
    &\qquad + \bigrbr{1 + \eta B_A \cdot \exp\rbr{2\eta B_A} }\cdot \kl\infdivx[]{\nu}{\tilde\nu}\nend
    &\le \eta \cdot   \EE\bigsbr{\bigabr{(\tilde Q - Q) - \EE\osbr{\tilde Q - Q}}} \nend
    &\qquad + \frac{\eta^2 \exp(2\eta B_A)}{2} \bigrbr{2+\eta B_A \cdot  \exp\rbr{2\eta B_A}} \cdot  \EE\bigsbr{\bigrbr{\orbr{\tilde Q -Q} - \EE\osbr{\tilde Q -Q}}^2}, 
\end{align*}
which finishes the proof of \Cref{cor:response-diff-myopic}.


% % \subsubsection{Proof of \Cref{lem:response diff-myopic}}
% % \label{sec:proof-formal-response diff-linear}
% % % We first invoke the upper bound for this TV distance in \eqref{eq:response decomposition} of \Cref{lem:performance diff} where we take $H=1$ for myopic follower and swap the position of $\theta^*$ and $\tilde\theta$ (by the exchangeability of the TV distance), 
% \begin{align}
%     &D_\TV\rbr{\nu^{\pi,\theta^*}(\cdot\given s) , \nu^{\pi,\tilde\theta}(\cdot\given s)} \nend
%     &\quad \le \eta \cdot {\EE_{s}^{\pi, \tilde\theta}\sbr{\abr{\rbr{\EE^{\pi, \tilde\theta}_{s, b}-\EE^{\pi, \tilde\theta}_{s}}\bigsbr{ \inp{\phi^{\pi}(s, b)}{\theta^*-\tilde\theta} }}}} \nend
%     & \qquad\quad + \frac 1 2 \eta^2  \cdot 
%     {\EE_{s}^{\pi, \tilde\theta}\sbr{ \exp\bigrbr{\eta\bigabr{A^{\pi,\tilde\theta}-A^{\pi,\theta^*}}}\cdot \bigabr{A^{\pi,\tilde\theta} - A^{\pi, \theta^*}}^2}}\nend
%     &\quad \le \eta  \cdot \underbrace{\sqrt{\EE_{s}^{\pi, \tilde\theta}\sbr{\rbr{\rbr{\EE^{\pi, \tilde\theta}_{s, b}-\EE^{\pi, \tilde\theta}_{s}}\bigsbr{ \inp{\phi^{\pi}(s, b)}{\theta^*-\tilde\theta} }}^2 }}}_{\dr (i)} \nend
%     & \qquad\quad + \frac 1 2 \eta^2  \exp\rbr{2\eta B_A} \cdot 
%     {\EE_{s}^{\pi, \tilde\theta}\sbr{ \bigrbr{A^{\pi,\tilde\theta} - A^{\pi, \theta^*}}^2}},\label{eq:TV-ub-MLE} 
% \end{align}
% where the second inequality holds by using the Cauchy-Schwartz inequality and bounding the exponential term by its maximal value.
% By a standard variance and mean decomposition in the last quadratic term of \eqref{eq:TV-ub-MLE}, we obtain
% \begin{align}
%     \EE_{s}^{\pi, \tilde\theta}\sbr{ \bigrbr{A^{\pi,\tilde\theta} - A^{\pi, \theta^*}}^2}
%     & \le \EE_{s}^{\pi, \tilde\theta}\sbr{ \rbr{ \rbr{\EE^{\pi, \tilde\theta}_{s, b}-\EE^{\pi, \tilde\theta}_{s}}\sbr{A^{\pi,\tilde\theta} - A^{\pi, \theta^*}}}^2} + \rbr{\EE^{\pi, \tilde\theta}_{s}\sbr{A^{\pi,\tilde\theta} - A^{\pi, \theta^*}}}^2\nend
%     & =\underbrace{\EE_{s}^{\pi, \tilde\theta}\sbr{ \rbr{ \rbr{\EE^{\pi, \tilde\theta}_{s, b}-\EE^{\pi, \tilde\theta}_{s}}\sbr{\inp[\big]{\phi^{\pi}(s, b)}{\theta^*-\tilde\theta} }}^2}}_{\dr (i)^2} + \underbrace{\rbr{\EE^{\pi, \tilde\theta}_{s}\sbr{A^{\pi,\tilde\theta} - A^{\pi, \theta^*}}}^2}_{\dr (ii)}.
%     % &\le \EE_{s}^{\pi, \tilde\theta}\sbr{ \rbr{ \rbr{\EE^{\pi, \tilde\theta}_{s, b}-\EE^{\pi, \tilde\theta}_{s}}\sbr{\inp[\big]{\phi^{\pi}(s, b)}{\theta^*-\tilde\theta} }}^2}
%     % &\qquad + \underbrace{\rbr{ \rbr{\EE^{\pi, \theta^*}_{s}-\EE^{\pi, \tilde\theta}_{s}}\sbr{\inp[\big]{\phi^{\pi}(s, b)}{\theta^*-\tilde\theta} }}^2}_{\dr (iii)},
%     \label{eq:A-square-ub}
% \end{align}
% where the equality follows from \Cref{lem:AQV-func diff} on the difference in the advantage function
% % , and the last inequality holds by \Cref{lem:KL-ub} where we notice that $\kl\infdivx[\big]{\nu^{\pi, \tilde\theta}}{\nu^{\pi, \theta^*}} = \EE^{\pi, \tilde\theta}_{s}\bigsbr{A^{\pi,\tilde\theta} - A^{\pi, \theta^*}}$.
% Combining \eqref{eq:A-square-ub} with \eqref{eq:TV-ub-MLE}, we conclude that
% \begin{align}
%     D_\TV\rbr{\nu^{\pi,\theta^*}(\cdot\given s) , \nu^{\pi,\tilde\theta}(\cdot\given s)} &\le \eta \cdot {\dr(i)} + \frac{\eta^2\exp\rbr{2\eta B_A}}{2} \cdot \rbr{{\dr(i)}^2 + {\dr(ii)}}.\label{eq:TV-ub-offline}
% \end{align}
% It is straightforward to bound term (i) by guarantee of MLE.  By \eqref{eq:bandit-ub-2} in \Cref{lem:bandit}, we have
% \begin{align}
%     \bignbr{\theta^*-\tilde\theta_{\MLE}}_{{\Psi}}^2 \le \underbrace{\min\cbr{\frac{\lambda_d(\tilde\Phi) }{\lambda_1(\Sigma_{\cD})}, \frac{|\cB|}{\min_{t\in[T]}\lambda_2(\Xi^{t, \tilde\theta})}}}_{\ds Z_{\cD}} C_\eta^2 \cdot 
%         \frac 1 T\log\rbr{\frac{\cN(\Theta, 1/T)}{\delta}}, \label{eq:MLE-guarantee-offline}
% \end{align}
% where $C_\eta= {B_A}/\rbr{1-\exp\rbr{-\eta B_A}}$, $\Sigma_{\cD}=T^{-1}\cdot \sum_{t=1}^T\EE_{s^t}^{\nu^{\pi^t, \tilde\theta}}[\psi^{t, \tilde\theta} {\psi^{t, \tilde\theta}}^\top]$, $\psi^{t, \theta} =\phi^{t}(b) - \EE_{s^t}^{\nu^{\pi^t, \theta}}[\phi^t(b')]$, and 
% $\tilde\Phi^t=\int_{\cB}\phi^t(b)\phi^t(b)^\top \rd b$.
% Note if $\cB$ has infinitely many actions, the second term in $Z_{\cD}$ is meaningless and only the first term matters.
% If $\cB$ has finite actions, we define $\tilde\Phi^t=\sum_{b\in\cB}\phi^t(b)\phi^t(b)^\top$ and $\Xi^{t, \tilde\theta} = \diag(\nu^{\pi^t, \tilde\theta}(\cdot\given s^t)) - (\nu^{\pi^t, \tilde\theta}(\cdot\given s^t))(\nu^{\pi^t, \tilde\theta}(\cdot\given s^t))^\top$.

We have by \Cref{cor:response-diff-myopic} that 
\begin{align*}
    D_\TV\rbr{\nu(\cdot\given s), \tilde\nu(\cdot\given s)}
        &\le  \eta  \EE\sbr{\abr{(\tilde r^\pi(s, b) - r^\pi(s, b)) - \EE\bigsbr{\tilde r^\pi(s, b) - r^\pi(s, b)}}} \nend
        &\qquad + C^{(3)}\EE\sbr{\rbr{\rbr{\tilde r^\pi(s, b) -r^\pi(s, b)} - \EE\sbr{\tilde r^\pi(s, b) -r^\pi(s, b)}}^2}.
\end{align*}
% We define the weighted covariance matrix as
% \begin{align*}
%     \Sigma_{ s}^{\pi, \theta} \defeq \EE_{s}^{\pi, \theta} \sbr{\psi^{\pi,\theta}(s, b)\psi^{\pi,\theta}(s, b)^\top}\quad\text{where}\quad \psi^{\pi,\theta}(s, b) = \phi^{\pi}(s, b) - \EE_{s}^{\pi,\theta}\phi^{\pi}(s, \cdot).
% \end{align*}
% Under this definition, we introduce a nonnegative definite matrix $\Psi\in \SSS_+^{d}$ and write down the covariance term as
% \begin{align*}
%     &\EE\sbr{\rbr{\rbr{\tilde r^\pi(s, b) -r^\pi(s, b)} - \EE\sbr{\tilde r^\pi(s, b) -r^\pi(s, b)}}^2} \nend
%     &\quad = \bignbr{\theta^*-\tilde\theta}_{\Sigma_{ s}^{\pi, \tilde\theta}} = \nbr{\sqrt{\Psi}^{\dagger}\sqrt{\Psi}\rbr{\theta^*-\tilde\theta}}_{\Sigma_{ s}^{\pi, \tilde\theta}} \le \sqrt{\Bignbr{\Psi^{\dagger} \Sigma_{ s}^{\pi, \tilde\theta}}_\oper }\cdot \bignbr{\theta^*-\tilde\theta}_{\Psi} \le \sqrt{\trace\rbr{{\Psi}^{\dagger} \Sigma_{ s}^{\pi, \tilde\theta}}} \cdot \bignbr{\theta^*-\tilde\theta}_{\Psi}.
% \end{align*}
% where we recall from \Cref{lem:bandit} the definition of the weighted Laplacian of the comparison feature graph with respect to the offline data $\cD$ and parameter $\tilde\theta$ as $ \Sigma_\cD^{\theta}\defeq T^{-1} \sum_{t=1}^T
% \EE_{s^t}^{\pi^t, \theta}\bigsbr{\psi^{\pi^t, \theta}(s, b)\psi^{\pi^t, \theta}(s, b)^\top}$.
% Note that the operator norm is further bounded by the trace,
% \begin{align*}
%     \sqrt{\Bignbr{{\Psi}^{\dagger} \Sigma_{ s}^{\pi, \tilde\theta}}_\oper} 
%     &\le \sqrt{\trace\rbr{{\Psi}^{\dagger} \Sigma_{ s}^{\pi, \tilde\theta}}} \nend
%     &= \sqrt{\trace\rbr{\EE_{s}^{\pi, \tilde\theta}\sbr{ {\Psi}^\dagger \psi^{\pi,\tilde\theta}(s, b)\psi^{\pi,\tilde\theta}(s, b)^\top}}}\nend
%     & = \sqrt{\EE_{s}^{\pi, \tilde\theta}\sbr{\psi^{\pi,\tilde\theta}(s, b)^\top {\Psi}^\dagger  \psi^{\pi,\tilde\theta}(s, b)}}.
% \end{align*}
% We plug in the definition $\psi^{\pi,\tilde\theta}(s, b)=\phi^{\pi}(s, b) - \EE_{s}^{\pi,\tilde\theta}\phi^{\pi}(s, \cdot)$ and obtain
% \begin{align*}
%     &\sqrt{\EE_{s}^{\pi, \tilde\theta}\sbr{\psi^{\pi,\tilde\theta}(s, b)^\top {\Psi}^\dagger  \psi^{\pi,\tilde\theta}(s, b)}} = \underbrace{\sqrt{\EE_{s}^{{\pi,\tilde\theta}}\sbr{\phi^{\pi}(s,b)^\top {\Psi}^{\dagger}\phi^{\pi}(s,b)} -\bignbr{\EE_{s}^{{\pi,\tilde\theta}}\phi^{\pi}(s, b)}_{{\Psi}^{-1}}^2}}_{\ds \Upsilon_{s}^{\pi,\tilde\theta}}.
% \end{align*}
% Hence, we have for term (i) that  ${\dr (i)}\le 
% \Upsilon_{s}^{\pi,\tilde\theta}\cdot \bignbr{\theta^*-\tilde\theta}_{{\Psi}}$.
% % \begin{align}
% %     {\dr (i)} 
% %     &\le 
% %     \Upsilon_{s}^{\pi,\tilde\theta}\cdot \bignbr{\theta^*-\tilde\theta}_{{\Psi}}
% %     \le 2 \Upsilon_{s}^{\pi,\tilde\theta}\cdot \underbrace{\sqrt{Z_{\cD} C_\eta^2  \cdot \frac 1 T\log\rbr{\frac{\cN(\Theta, 1/T)}{\delta}} + \bignbr{\tilde\theta - \tilde\theta_{\MLE}}_{{\Psi}}^2}}_{\ds \zeta^{\tilde\theta}},\label{eq:myopic-(i)-ub}
% % \end{align}
% % on the success of \Cref{lem:bandit}. 
% % Here, the second inequality follows from the triangular inequaltiy, and the last inequality holds from \eqref{eq:MLE-guarantee-offline}.
% Now, we have addressed term (i) of \eqref{eq:TV-ub-offline} and it remains to show the upper bound for term (ii). Observe that under the quantal response model with logistic preference, term (ii) is nothing but just the squared KL divergence $\eta^{-2}\kl\infdivx[]{\nu^{\pi, \tilde\theta}(\cdot\given s)}{\nu^{\pi, \theta^*}(\cdot\given s)}^2$. Let $\Delta \theta=\tilde\theta - \theta^*$ and consider $\tilde\theta$ to be fixed. The Hessian of the KL divergence with respect to parameters in the second position is 
% \begin{align*}
%     \eta^{-2}\nabla_{\theta}^2 \kl\infdivx[]{\nu^{\pi, \tilde\theta}(\cdot\given s)}{\nu^{\pi, \theta}(\cdot\given s)} = \EE_s^{\pi,\theta} \sbr{\phi^\pi(s, b)\phi^\pi(s, b)^\top} - \EE_s^{\pi,\theta}\sbr{\phi^\pi(s, b)} \cdot \EE_s^{\pi,\theta}\sbr{\phi^\pi(s, b)}^\top = \Sigma_s^{\pi,\theta}.
% \end{align*}
% We observe that the KL divergence is strongly convex in the second parameter. Actually, for any test $x\in\RR^d$ and let $g(b)=\phi^\pi(s, \cdot)^\top x$, we have $x^\top \rbr{\eta^{-2}\nabla_{\theta}^2 \kl\infdivx[]{\nu^{\pi, \tilde\theta}(\cdot\given s)}{\nu^{\pi, \theta}(\cdot\given s)}} x = \Var^{\pi,\theta}[g(b)]$ and furthermore,
% \begin{align*}
%      \Var^{\pi,\theta}[g(b)]
%     \ge \exp(-2\eta B_A) \EE_s^{\pi, \tilde\theta}\sbr{\rbr{g(b) - \EE_s^{\pi,\theta}[g(b)]}^2} \ge \exp\rbr{-2\eta B_A} \Var^{\pi, \tilde\theta}[g(b)],
% \end{align*}
% which suggests that $\eta^{-2}\nabla_{\theta}^2 \kl\infdivx[]{\nu^{\pi, \tilde\theta}(\cdot\given s)}{\nu^{\pi, \theta}(\cdot\given s)}=\Sigma_s^{\pi,\theta}\succeq \exp(-2\eta B_A) \Sigma_s^{\pi,\tilde\theta}$. Actually, we can also swap $\theta$ and $\tilde\theta$ and obtain 
% \begin{align*}
%     \eta\exp(2\eta B_A) \Sigma_s^{\pi,\tilde\theta} \succeq \eta^{-}\nabla_{\theta}^2 \kl\infdivx[]{\nu^{\pi, \tilde\theta}(\cdot\given s)}{\nu^{\pi, \theta}(\cdot\given s)} \succeq \eta\exp(-2\eta B_A) \Sigma_s^{\pi,\tilde\theta} , \quad \forall \theta\in\Theta.
% \end{align*}
% Therefore, we can bound term (ii) simply by
% \begin{align*}
%     {\dr(ii)}\le \frac{\eta^2}{4} \exp\rbr{4\eta B_A} \bignbr{\theta^*-\tilde\theta}_{\Sigma_s^{\pi,\tilde\theta}}^4 \le  \frac{\eta^2}{4} \exp\rbr{4\eta B_A} \rbr{\Upsilon_{s}^{\pi,\tilde\theta}\cdot \bignbr{\theta^*-\tilde\theta}_{{\Psi}}}^4,
% \end{align*}
% where the last inequality holds by the same upper bound for term (i).
% Combining our results for both term (i) (ii), we have for \eqref{eq:TV-ub-offline} that
% \begin{align*}
%     &D_\TV\rbr{\nu^{\pi,\theta^*}(\cdot\given s) , \nu^{\pi,\tilde\theta}(\cdot\given s)} \nend
%     &\quad\le \eta \cdot {\dr(i)} + \frac{\eta^2\exp\rbr{2\eta B_A}}{2} \cdot \rbr{{\dr(i)}^2 + {\dr(ii)}}\nend
%     &\quad\le \eta \Upsilon_{s}^{\pi,\tilde\theta} \bignbr{\theta^*-\tilde\theta}_{{\Psi}} + \frac{\eta^2\exp\rbr{2\eta B_A}}{2} \rbr{\rbr{\Upsilon_{s}^{\pi,\tilde\theta} \bignbr{\theta^*-\tilde\theta}_{{\Psi}}}^2 + \frac{\eta^2}{4} \exp\rbr{4\eta B_A} \rbr{\Upsilon_{s}^{\pi,\tilde\theta} \bignbr{\theta^*-\tilde\theta}_{{\Psi}}}^4}.
% \end{align*}
% Now, we let $\Gamma^{(2)}(s;\pi_{s}, \tilde\theta) = \Upsilon_{s}^{\pi,\tilde\theta} \bignbr{\theta^*-\tilde\theta}_{{\Psi}} $ and 
% \begin{align*}
%     f(x) = \eta x + \frac{\eta^2\exp(2\eta B_A)}{2} x^2 + \frac{\eta^4\exp(6\eta B_A)}{8} x^4, 
% \end{align*}
% and just conclude that $D_\TV\bigrbr{\nu^{\pi_{s},\theta^*}(\cdot\given s) , \nu^{\pi_{s},\tilde\theta}(\cdot\given s)} \le f(\Gamma^{(2)}(s;\pi_{s}, \tilde\theta))$, 
% which completes the proof of \Cref{lem:response diff-myopic}. 
% % One can swap $\theta^*$ and $\$


% % \subsection{Follow up Discussion on \Cref{thm:PMLE-VI-myopic}}\label{sec:follow up on myopic offline}
% % \todo{comment on the use of the Laplacian, including why a standard $\sum\phi\phi^\top$ would fail. Also, comment on the use of $L_\cD$. }

% \subsubsection{Proof of \Cref{lem:MLE-formal}}
% \label{sec:proof-MLE-general}
% % The key idea is replacing the negative log likelihood $\cL^{(1)}_\cD(\cdot)$ with some curve with constant Hessian $\EE_{\nu^{\pi^t, \theta^*}}[\psi^{t, \theta^*} {\psi^{t, \theta^*}}^\top]$ in the neighbourhood of $\theta^*$. Specifically, we first lower bound $\cL^{(1)}_\cD(\hat\theta_\MLE)-\cL^{(1)}_\cD(\theta^*)$ by the Hellinger distance $T^{-1}\cdot\sum_{t=1}^T D_\H^2(\nu^{\pi^t, \hat\theta_\MLE}, \nu^{\pi^t, \theta^*})$ and some $\cO(T^{-1})$ term. Then, a careful scrutiny of the Hellinger distance will show that $$D_\H^2(\nu^{\pi^t, \theta}, \nu^{\pi^t, \theta^*})\ge B_A^{-2} (\theta-\theta^*)^\top\EE^{\nu^{\pi^t, \theta^*}}[\psi^{t, \theta^*} {\psi^{t, \theta^*}}^\top](\theta-\theta^*).$$
% To make the above discussion rigorous, we first invoke the following concentration Lemma.
\begin{proof} 
We prove this lemma by leveraging 
 \Cref{lem:freeman-variation} with $X_t = (-\cL_{h,t}(\theta) + \cL_{h, t}(\theta^*))/2$ where $\cL_{h}^t(\theta)=-\sum_{i=1}^{t-1} \eta A_h^{\pi^i, \theta}(s_h^i,b_h^i)$. 
We choose filtration $\cF_{h, t-1}=\sigma(\tau^{1:t-1})$ where $\sigma(X)$ denotes the $\sigma$-algebra generated by $X$ and $\tau^{1:t-1}$ is just the history up to $t-1$.
Let $\cN_\rho(\Theta, \epsilon)$ be  the covering number for the $\epsilon$-covering net of $\Theta$  with respect to norm $\rho$ defined in \eqref{eq:rho for MLE}. 
Let $\Theta_\epsilon$ be  the $\epsilon$-covering net of $\Theta$. 
To simplify the notation, we define  $\iota =  \log\rbr{ H\cN_\rho(\Theta, \epsilon) / \delta}$.
Then, for all $\theta \in \Theta_{\epsilon}$, 
we have with probability $1-\delta$ for all $h\in[H], t\in[T]$ that
\begin{align}
    &\frac 1 2 \rbr{-\cL_h^t(\theta) + \cL_h^t(\theta^*)} \nend
    &\quad\le \sum_{i=1}^{t-1} \log\EE^{\pi^i   }\sbr{\sqrt{ \nu_h^{\pi^i,\theta}(\cdot\given s_h) \big / \nu_h^{\pi^i, \theta^*}(\cdot\given s_h)}}  + \log\rbr{ H\cN_\rho(\Theta, \epsilon) / \delta} \nend
    &\quad\le -  \sum_{i=1}^{t-1} \EE^{\pi^i }\Bigl [ D_\H^2\bigrbr{\nu_h^{\pi^i, \theta}(\cdot\given s_h ), \nu_h^{\pi^i, \theta^*}(\cdot\given s_h )} \Bigr ] + \log\rbr{ H\cN_\rho(\Theta, \epsilon) / \delta}, \label{eq:MLE-to-hellinger-1}
\end{align}
where the expectation is taken with respect to the true model. 
Here, the first inequality holds by applying  \Cref{lem:freeman-variation} and   taking a union bound over the  $\epsilon$-covering net. 
% Note that $\log(\cN_\rho(\Theta, \epsilon))$ only grows with 
% $\log(\eta T)$ since $\cL_{h,t}(\theta)$ is $2\eta$-Lipschitz with respect to $\theta$. 
The second inequality holds by noting that $\log(x)\le x-1$ and by the definition of the Hellinger distance.


Meanwhile, by the definition of the distance $\rho$ in   \eqref{eq:rho for MLE},
for any $\theta, \tilde \theta \in \Theta$, we have 
\begin{align*}
&|D_\H^2(\nu_h^{\pi,\theta}, \nu_h^{\pi, \theta^*}) - D_\H^2(\nu_h^{\pi,\tilde\theta}, \nu_h^{\pi, \theta^*})|\nend
&\quad \le (D_\H^2(\nu_h^{\pi,\theta}, \nu_h^{\pi, \theta^*}) + D_\H^2(\nu_h^{\pi,\tilde\theta}, \nu_h^{\pi, \theta^*})) \cdot 
\bigabr{D_\H^2(\nu_h^{\pi,\theta}, \nu_h^{\pi, \theta^*}) - D_\H^2(\nu_h^{\pi,\tilde\theta}, \nu_h^{\pi, \theta^*})}\nend
&\quad \le 2 D_\H(\nu_h^{\pi,\tilde\theta}, \nu_h^{\pi,\theta})\nend
&\quad \le 2\rho(\theta, \tilde\theta), 
\end{align*} 
where  the second inequality holds by noting that the Hellinger distance does not exceed 1, and that the Hellinger distance satisfies the triangle inequality as a norm, and the last inequality holds by definition of $\rho$. 
Moreover, by noting that $\cL_h^t(\theta)=-\sum_{i=1}^t \eta A_h^{\pi^i, \theta}(s_h^i,b_h^i)$, we have
\begin{align*}
\bigl | \cL_h^t (\theta ) - \cL_h^t(\tilde \theta ) \bigr |
&\le \eta T \max_{i\in[t-1]}\bigabr{ A_h^{\pi^i, \theta}(s_h^i,b_h^i) - A_h^{\pi^i, \tilde\theta}(s_h^i,b_h^i)}\nend
&\le 2\eta T \max_{i\in[t-1]}\bignbr{ Q_h^{\pi^i, \theta}- Q_h^{\pi^i, \tilde\theta}}_\infty \nend
&\leq 2 T \cdot \rho(\theta, \tilde \theta),
\end{align*}
where the second inequality holds by noting that $|(V_h^{\pi, \theta} - V_h^{\pi, \tilde\theta})(s_h)| \le \onbr{Q_h^{\pi,\theta}-Q_h^{\pi,\tilde\theta}}_\infty$ by the same argument in \eqref{eq:f_2-01} and \eqref{eq:f_2-1}, and the last inequality holds by noting that $\eta\le (\gamma B_A + 1 + \eta)$.
% \Zhuoran{Derive the second one, STOP HERE}
Therefore, 
adding an extra term $3T\epsilon$ to the right-hand side of \eqref{eq:MLE-to-hellinger-1} extends the result to any $\theta\in\Theta$ by definition of the covering net $\Theta_\epsilon$.
We thus obtain for all $\theta\in\Theta, h\in[H], t\in[T]$ with probability $1-\delta$,
\begin{align}
    \frac 1 2 \rbr{-\cL_h^t(\theta) + \cL_h^t(\theta^*)} 
    &\le -  \sum_{i=1}^{t-1} \EE^{\pi^i, \theta^*}D_\H^2\rbr{\nu_h^{\pi^i, \theta}(\cdot\given s_h^i), \nu_h^{\pi^i, \theta^*}(\cdot\given s_h^i)} \nend
    &\qquad + \log\rbr{ H\cN_\rho(\Theta, \epsilon) / \delta} + 3T\epsilon. \label{eq:KL-2-D_H-online}
\end{align}
% As an alternative, we can also take the filtration as $\cF_{h, t-1} = \sigma(\{s_{h}^i \}_{i\in[t]}, \{b_h^i\}_{i\in[t-1]})$ and obtain a similar result
% \begin{align}
%     \frac 1 2 \rbr{-\cL_h^t(\theta) + \cL_h^t(\theta^*)} 
%     &\le - \sum_{i=1}^{t-1} D_\H^2\rbr{\nu_h^{\pi^i, \theta}(\cdot\given s_h^i), \nu_h^{\pi^i, \theta^*}(\cdot\given s_h^i)} + \iota. \label{eq:KL-2-D_H-offline}
% \end{align}
In the sequel, we take $\epsilon=T^{-1}$ and  let $\iota= \log\rbr{ H\cN_\rho(\Theta, T^{-1}) / \delta}+ 3$.
Now, we plug in $\hat\theta_{h,\MLE}=\argmin_{\theta'\in\Theta} \cL_h^t(\theta')$ in \eqref{eq:KL-2-D_H-online}
%  and \eqref{eq:KL-2-D_H-offline},  
and obtain by the nonnegativity of the Hellinger distance that
\begin{align*}
    \cL_h^t(\theta^*) \le \inf_{\theta'\in\Theta} \cL_h^t(\theta' ) + 2\iota \le  \cL_h^t(\hat\theta_{h,\MLE}) + 2\iota  , 
\end{align*}
which guarantees that our confidence set is indeed valid by letting 
$$\beta \ge  2\iota = 2\log(e^3 H\cdot \cN_\rho(\Theta, T^{-1})/\delta). $$
Next, we show that our confidence set is also accurate.
For \eqref{eq:KL-2-D_H-online}, we have that
\begin{align}
    \sum_{i=1}^{t-1} \EE^{\pi^i, \theta^*}D_\H^2\rbr{\nu_h^{\pi^i, \theta}(\cdot\given s_h^i), \nu_h^{\pi^i, \theta^*}(\cdot\given s_h^i)} 
    &\le  \frac 1 2\rbr{\cL_h^t(\theta) - \cL_h^t(\theta^*)} +\iota\nend
    &\le \frac 1 2\rbr{\cL_h^t(\theta) - \cL_h^t(\theta^*)+\beta}, \label{eq:MLE-to-hellinger-2}
\end{align}
% and also 
% \begin{align*}
%     \sum_{i=1}^{t-1} D_\H^2\rbr{\nu_h^{\pi^i, \theta}(\cdot\given s_h^i), \nu_h^{\pi^i, \theta^*}(\cdot\given s_h^i)} \le  \frac 1 2\rbr{\cL_h^t(\theta) - \cL_h^t(\theta^*)} + \log\rbr{\frac{e^2H\cN_\rho(\Theta, T^{-1})}{\delta}}, 
% \end{align*}
Now, if $\theta\in\CI_{h,\Theta}^t(\beta)$, it follows directly from \eqref{eq:MLE-to-hellinger-2} that
\begin{align*}
    \sum_{i=1}^{t-1} \EE^{\pi^i, \theta^*}D_\H^2\rbr{\nu_h^{\pi^i, \theta}(\cdot\given s_h^i), \nu_h^{\pi^i, \theta^*}(\cdot\given s_h^i)} \le \beta, 
\end{align*}
which shows that our confidence set is also valid and gives \eqref{eq:MLE-guarantee-hellinger-2}.
%  since the righthand side can be replaced by $3/2 \beta$ using the fact that $\cL_h^t(\theta) - \cL_h^t(\theta^*)\le \cL_h^t(\theta) - \inf_{\theta'\in\Theta}\cL_h^t(\theta')\le \beta$  for any $\theta\in\CI_\Theta(\beta)$.
We next show how to derive the bound for the Q function. Invoking \Cref{lem:D_H-2-A^2}, we have that
\begin{align}
    8 D_\H^2\rbr{\nu_h^{\pi, \hat\theta}(\cdot\given s_h), \nu_h^{\pi, \theta^*}(\cdot\given s_h)} 
    &\ge \rbr{\frac{\eta}{1+\eta B_A}}^2\cdot \inp[\Big]{\nu_h^{\pi, \theta^*}}{ \orbr{A_h^{\pi, \hat\theta}-A_h^{\pi, \theta^*}}^2} \nend
    &\ge \rbr{\frac{\eta}{1+\eta B_A}}^2\cdot \EE_{s_h}^{\pi, \theta^*}\rbr{\bigrbr{\EE_{s_h, b_h}^{\pi, \theta^*} - \EE_{s_h}^{\pi, \theta^*}}\osbr{A_h^{\pi, \hat\theta}-A_h^{\pi, \theta^*}}}^2\nend
    &= \rbr{\frac{\eta}{1+\eta B_A}}^2\cdot \EE_{s_h}^{\pi, \theta^*}\Bigrbr{\orbr{\EE_{s_h, b_h}^{\pi, \theta^*} - \EE_{s_h}^{\pi, \theta^*}}\osbr{Q_h^{\pi, \hat\theta}-Q_h^{\pi, \theta^*}}}^2, \label{eq:hellinger-2-Q}
\end{align}
where the second inequality follows from the Jensen's inequality, and the last inequality holds by invoking \eqref{eq:A diff-1} in \Cref{lem:AQV-func diff}. One can also swap $\theta^*$ and $\hat\theta$ by the exchangeability of the Hellinger distance and obtain another version.  Note that $C_\eta = \eta^{-1}+B_A$. 
Plugging \eqref{eq:hellinger-2-Q} with $C_\eta$ into the previous accuracy guarantees gives \eqref{eq:MLE_guarantee_Q}.
%  and \eqref{eq:MLE_guarantee_Q-1}. 
Therefore, we have proved \eqref{eq:MLE-guarantee-hellinger-2} and \eqref{eq:MLE_guarantee_Q}. 
For deriving \eqref{eq:MLE-guarantee-hellinger-1} and \eqref{eq:MLE_guarantee_Q-1}, we just change our filtration to $\cF_{h, t-1} = \sigma((s_h^i, b_h^i)_{i\in[t-1]}, s_h^t)$ and everything follows. 

Lastly, we prove the guarantee in \eqref{eq:MLE-guarantee-Q-3}. 
We use \eqref{eq:MLE_guarantee_Q-1} with $\theta'$ replaced by $\theta^*$ and for all $\theta\in\CI_{h, \Theta}^t(\beta)$, 
\begin{align*}
    \sum_{i=1}^{t-1} {\Var_{s_h^i}^{\pi^i, \theta^*} \bigsbr{Q_h^{\pi^i, \theta}(s_h, b_h) - Q_h^{\pi^i, \theta^*}(s_h, b_h)}} \le 4 C_\eta^2 \rbr{\cL_h^t(\theta) - \cL_h^t(\theta^*)+ \beta} \le 8 C_\eta^2 \beta, 
    % \label{eq:MLE_guarantee_Q-1} 
\end{align*}
which is equivalent to saying that for all $h\in[H], \theta\in\CI_{h, \Theta}^t(\beta)$,
\begin{align}
    \sum_{i=1}^{t-1} 
    \EE_{s_h^i}^{\pi^i,\theta^*}\rbr{\rbr{Q_h^{\pi^i, \theta} - Q_h^{\pi^i, \theta*}}(s_h^i, b_h^i)
    -\EE_{s_h^i}^{\pi^i,\theta^*}\sbr{ \rbr{Q_h^{\pi^i, \theta} - Q_h^{\pi^i, \theta^*}}(s_h, b_h)}}^2 \le 8 C_\eta^2 \beta. \label{eq:MLE-Q-ub-1}
\end{align}
Recall the covering net $\Theta_\epsilon$ we constructed before. For all $\theta\in\Theta_\epsilon \cap \CI_{h,\Theta}^t(\beta), h\in[H]$, and a given $t\in[T]$, we have  by a standard martingale concentration in \Cref{cor:martigale concentration} that with probability at least $1-2\delta$,
\begin{align*}
    &\sum_{i=1}^{t-1} 
    \rbr{\rbr{Q_h^{\pi^i, \theta} - Q_h^{\pi^i, \theta^*}}(s_h^i, b_h^i)
    -\EE_{s_h^i}^{\pi^i,\theta^*}\sbr{ \rbr{Q_h^{\pi^i, \theta} - Q_h^{\pi^i, \theta^*}}(s_h, b_h)}}^2 
    \nend
    &\quad \le \frac 3 2\sum_{i=1}^{t-1}\EE_{s_h^i}^{\pi^i,\theta^*} 
    \rbr{\rbr{Q_h^{\pi^i, \theta} - Q_h^{\pi^i, \theta^*}}(s_h^i, b_h)
    -\EE_{s_h^i}^{\pi^i,\theta^*}\sbr{ \rbr{Q_h^{\pi^i, \theta} - Q_h^{\pi^i, \theta^*}}(s_h, b_h)}}^2 \nend
    &\qqquad + 32 B_A^2 \log\rbr{2H \cN_\rho(\Theta, \epsilon)\delta^{-1} }\nend
    &\quad \le 12 C_\eta^2 \beta  +32 B_A^2 \log\rbr{2H \cN_\rho(\Theta, \epsilon)\delta^{-1} }\le 28 C_\eta^2 \beta 
\end{align*}
where the second inequality follows from \eqref{eq:MLE-Q-ub-1}. 
Moreover, we note that
\begin{align*}
    &\rbr{\rbr{Q_h^{\pi^i, \theta} - Q_h^{\pi^i, \theta^*}}(s_h^i, b_h)
    -\EE_{s_h^i}^{\pi^i,\theta^*}\sbr{ \rbr{Q_h^{\pi^i, \theta} - Q_h^{\pi^i, \theta^*}}(s_h, b_h)}}^2 \nend
    &\qqquad - \rbr{\rbr{Q_h^{\pi^i, \tilde\theta} - Q_h^{\pi^i, \theta*}}(s_h^i, b_h)
    -\EE_{s_h^i}^{\pi^i,\theta^*}\sbr{ \rbr{Q_h^{\pi^i, \tilde\theta} - Q_h^{\pi^i, \theta^*}}(s_h, b_h)}}^2\nend
    &\quad \le 8 \cdot \max_{\pi\in\Pi,\theta\in\Theta}\bignbr{Q_h^{\pi,\theta}}_\infty
    \cdot 2 \bignbr{Q_h^{\pi^i, \theta} -  Q_h^{\pi^i, \tilde\theta}}_\infty \le 16 B_A \rho(\theta, \tilde\theta),
\end{align*}
where the last inequality holds by noting that $B_A$ upper bounds $\max_{\theta\in\Theta, \pi\in\Pi, h\in[H]}\onbr{Q_h^{\pi, \theta}}_\infty$ and using the definition of $\rho$. 
% and $\cN_\varrho (\Theta, \epsilon)$ is the covering number of the smallest $\epsilon$-covering number of $\Theta$ with respect to distance 
% \begin{align*}
%     \varrho(\theta, \tilde\theta) \defeq \max_{\pi\in\Pi, h\in[H]}\nbr{Q_h^{\pi, \theta} - Q_h^{\pi, \tilde\theta}}_\infty.
% \end{align*}
As a result, for all $\theta\in\CI_\Theta(\beta), h\in[H]$, and a given $t\in[T]$, we conclude that with probability at least $1-2\delta$,
\begin{align*}
    &\sum_{i=1}^{t-1} 
    \rbr{\rbr{Q_h^{\pi^i, \theta} - Q_h^{\pi^i }}(s_h^i, b_h^i)
    -\EE_{s_h^i}^{\pi^i,\theta^*}\sbr{ \rbr{Q_h^{\pi^i, \theta} - Q_h^{\pi^i, \theta^*}}(s_h, b_h)}}^2  \le 28 C_\eta^2 \beta  + 16 B_A.
\end{align*}
Replace $\delta$ by $\delta/2$, we have for all $\theta\in\CI_\Theta(\beta), h\in[H]$ and a given $t\in[T]$ with probability at least $1-\delta$ that 
\begin{align*}
    &\sum_{i=1}^{t-1} 
    \rbr{\rbr{Q_h^{\pi^i, \theta} - Q_h^{\pi^i }}(s_h^i, b_h^i)
    -\EE_{s_h^i}^{\pi^i,\theta^*}\sbr{ \rbr{Q_h^{\pi^i, \theta} - Q_h^{\pi^i, \theta^*}}(s_h, b_h)}}^2 
    \nend
    &\quad \le 28 C_\eta^2 \beta + 56 C_\eta^2\log 2 +16 B_A =  \cO(C_\eta^2 \beta).
\end{align*}
which 
finishes the proof of \Cref{lem:MLE-formal}.
\end{proof}

% \subsubsection{Proof of \Cref{lem:MLE-indep-data}}\label{sec:proof-MLE-indep-data}
% 
Here, we show the guarantee for the MLE with independently collected dataset.
Since each trajectory is independently collected, we are able to use the Bernstein inequality for indepedent random variables $Z_h^i = D_\H^2\orbr{\nu_h^{\pi^i, \theta}(\cdot\given s_h^i), \nu_h^{\pi^i, \theta^*}(\cdot\given s_h^i)}$, 
\#
    \abr{\frac 1 T \sum_{i=1}^T Z_h^i - \EE_\cD\sbr{Z_h^i}} &\le \sqrt{\frac{4\sum_{i=1}^T\Var[Z_h^i]\log(2\delta^{-1})}{T^2}} + \frac{4\log(2\delta^{-1})}{3T} \nend
    &\le \sqrt{\frac{4\sum_{i=1}^T \EE_\cD[Z_h^i]\log(2\delta^{-1})}{T^2}} + \frac{4\log(2\delta^{-1})}{3T}\nend
    &\le \frac{1}{2T}\sum_{i=1}^T \EE_\cD [Z_h^i] + \frac{2 \log(2\delta^{-1})}{T} + \frac{4\log(2\delta^{-1})}{3T},\notag
\#
where the second inequality holds by noting that $\Var[Z_h^i] \le \EE_\cD[(Z_h^i)^2]\le \EE_\cD[Z_h^i]$ by using the property that Hellinger distance is always upper bounded by $1$. We now conclude by further taking a union bound over $h\in[H]$ and $\theta\in\Theta$ that
\#
\frac{1}{T} \sum_{i=1}^T \EE_\cD[Z_h^i] &\le \frac{2}{T} \sum_{i=1}^T Z_h^i  + \frac{8\log(2H \cN_\rho(\Theta, \epsilon)\delta^{-1})}{T} + 6\epsilon\nend
&\le \frac{2}{T} \sum_{i=1}^T Z_h^i  + \frac{8\log(2eH \cN_\rho(\Theta, T^{-1})\delta^{-1})}{T},\notag
\#
where the last inequality holds by taking $\epsilon=T^{-1}$. Plug in the definition of $Z_h^i$, we have
\begin{align*}
    \frac 1 T\sum_{i=1}^T\EE_\cD\sbr{D_\H^2\rbr{\nu_h^{\pi^i, \theta}(\cdot\given s_h^i), \nu_h^{\pi^i, \theta^*}(\cdot\given s_h^i)}} &\le \frac{2}{T} \sum_{i=1}^T D_\H^2\rbr{\nu_h^{\pi^i, \theta}(\cdot\given s_h^i), \nu_h^{\pi^i, \theta^*}(\cdot\given s_h^i)}  \nend
    &\qquad + \frac{8\log(2eH \cN_\rho(\Theta, T^{-1})\delta^{-1})}{T}.
\end{align*}
Using \eqref{eq:MLE-guarantee-hellinger-1} in \Cref{lem:MLE-formal} for any $\theta\in\cC_{\Theta}(\beta)$, we give 
\begin{align}
    \sum_{i=1}^{T} D_\H^2\bigrbr{\nu_h^{\pi^i, \theta}(\cdot\given s_h^i), \nu_h^{\pi^i, \theta^*}(\cdot\given s_h^i)} 
        %%%%%%%%%%
        &\le  \frac 1 2\rbr{\cL_h(\theta) - \cL_h(\theta^*)} + \log\rbr{\frac{eH\cN_\rho(\Theta, T^{-1})}{\delta}} \nend
        %%%%%%%%%
        &\le \frac 1 2\rbr{\cL_h(\theta) - \inf_{\theta'\in\Theta}\cL_h(\theta')} + \log\rbr{\frac{eH\cN_\rho(\Theta, T^{-1})}{\delta}}\nend
        &\le \frac 3 2 \beta, \label{eq:OffGM-nu-hellinger-1}
\end{align}
where the last inequality is just by definition of $\CI_\Theta(\beta)$ in \Cref{eq:behavior_model_confset-1}. Therefore, 
\begin{align*}
    \sum_{i=1}^T\EE_\cD\sbr{D_\H^2\rbr{\nu_h^{\pi^i, \theta}(\cdot\given s_h^i), \nu_h^{\pi^i, \theta^*}(\cdot\given s_h^i)}}  \le 3\beta + {8\log(2eH \cN_\rho(\Theta, T^{-1})\delta^{-1})} \le 11\beta, 
\end{align*}
with probability at least $1-2\delta$ for all $h\in[H]$ and $\theta\in\CI_\Theta(\beta)$. The validity guarantee is already shown in \Cref{lem:MLE-formal}.
We complete the proof of \Cref{lem:MLE-indep-data}.

% \Cref{thm:11_6_gyorfi}, where we take $Z_h^i = D_\H^2\rbr{\nu_h^{\pi^i, \theta}(\cdot\given s_h^i), \nu_h^{\pi^i, \theta^*}(\cdot\given s_h^i)}$ and take $g(Z)=Z$. One can verify that $g\in[0, 1]$ with covering number $\cN_\infty(\epsilon, \cG)=1$.
% Hence, we conclude with $\epsilon = 1/3$, $\alpha=120\log(4/\delta) T^{-1}$ that for any fixed $\theta\in\Theta$, 
% \begin{align*}
%     \PP\rbr{\frac 1 T \sum_{i=1}^T Z_h^i > 2 \EE_\cD Z_h^i  + \frac{60\log(4/\delta)}{T}} \le \delta,
% \end{align*}
% Now, we take a union bound over the $\epsilon$-covering net for $\Theta$ with respect to $\rho$ and also over $h\in[H]$ and obtain with probability at least $1-\delta$ that for any $\theta\in\Theta$, $h\in[H]$ 
% \begin{align*}
%     \frac 1 T \sum_{i=1}^T D_\H^2\rbr{\nu_h^{\pi^i, \theta}(\cdot\given s_h^i), \nu_h^{\pi^i, \theta^*}(\cdot\given s_h^i)} \le 2 \EE_\cD D_\H^2\rbr{\nu_h^{\pi, \theta}(\cdot\given s_h), \nu_h^{\pi, \theta^*}(\cdot\given s_h)} + \frac{1+ 60\log(4\cN_\rho(\Theta, T^{-1})/\delta)}{T}, 
% \end{align*}
% where we can use the same covering number for $\Theta$ since $\rho(\theta, \tilde\theta)$ can still bound the difference in the squared Hellinger distance. Here, the expectation on the right hand side is taken with respect to both the randomness in $\pi$ and $s_h$.

% \subsubsection{Proof of \Cref{lem:leader-bellman-loss}}\label{sec:proof-leader-bellman-loss}
% In the following proof, we always consider the expectation to be taken with respect to the data generating distribution.
We first prove the following concentration result: for any $h\in[H]$ and any $y=(\theta_{h+1}, \pi_{h+1}, U_{h+1}, U_{h})\in \cY_h = \Theta_{h+1}\times \Pi_{h+1}\times \cU^2$, 
it holds with probability at least $1-\delta$ that 
      \begin{align}
        &\abr{T\EE[(U_{h} - \TT_{h}^{\pi,\theta}U_{h + 1})^2] - \ell_{h}(U_{h}, U_{h + 1}, \theta, \pi) + \ell_{h}(\TT_{h}^{\pi,\theta}U_{h + 1}, U_{h + 1}, \theta, \pi)} \notag\\
        &\qquad \le \epsilon_S + \frac{T}{2} \EE[(U_{h} - \TT_{h}^{\pi,\theta}U_{h + 1})^2].\label{eq:cY-confset-1}
      \end{align}
     where
    $
    {110 B_U^2\cdot\log(H \max_{h\in[H]}\cN_\rho(\cY_h, T^{-1})\delta^{-1}) } \cdot {T^{-1}} + (45 B_U^2 + 60 B_U )T^{-1}
    $ and $B_U=H$ is the upper bound for the function class $U$.

\paragraph{Concentration. }
Our proof is an adaptation of Lemma D.2 in \citep{lyu2022pessimism}, although we simplify a little bit by directly using the covering number for a joint class $\cY_h = \Pi_{h+1}\times\Theta_{h+1}\times\cU^2$. We take an $\epsilon$-covering net $\cY_\epsilon$ for $\cY$ with respect to distance $\rho$ specified by \eqref{eq:rho-cY}. 
We first use \Cref{lem:bernstein},  
where we take 
\begin{align*}
    {Z_h^i} &= \ell_{h}(U'_{h}, U_{h + 1}, \theta, \pi) - \ell_{h}(\TT_{h}^{\pi,\theta}U_{h + 1}, U_{h + 1}, \theta, \pi)\nend
    & = \rbr{U_h(s_h^i, a_h^i, b_h^i) - u_h^i -  T_{h+1}^{\pi,\theta} U_{h+1}(s_{h+1}^i)}^2  - \rbr{\TT_h^{\pi, \theta} U_{h+1}(s_h^i, a_h^i, b_h^i) - u_h^i - T_{h+1}^{\pi,\theta} U_{h+1}(s_{h+1}^i)}^2.
\end{align*}
Here, we recall the definition of $\TT_h^{\pi}$ given by \eqref{eq:bellman_operator_leader}.
One can verify that $|{Z_h^i}|\le 9B_U^2$ where $B_U$ bounds both the leader's reward and the value function class $\cU$.
We first calculate the expectation of $Z_h^i$, 
\begin{align*}
    \EE[{Z_h^i}] 
    &= \EE \bigg[\EE_{s_h^i, a_h^i, b_h^i} \Big[\rbr{U_h(s_h^i, a_h^i, b_h^i) - u_h^i -  T_{h+1}^{\pi,\theta} U_{h+1}(s_{h+1}^i)}^2  \nend
    &\qquad - \rbr{\TT_h^{\pi, \theta} U_{h+1}(s_h^i, a_h^i, b_h^i) - u_h^i - T_{h+1}^{\pi,\theta} U_{h+1}(s_{h+1}^i)}^2\Big]\bigg]\nend
    & = \EE\bigg[\Bigrbr{U_h(s_h^i, a_h^i, b_h^i)- \TT_h^{\pi, \theta} U_{h+1}(s_h^i, a_h^i, b_h^i)}\nend
    &\qquad \cdot \EE_{s_h^i, a_h^i, b_h^i}\sbr{U_h(s_h^i, a_h^i, b_h^i) + \TT_h^{\pi, \theta} U_{h+1}(s_h^i, a_h^i, b_h^i)- 2u_h^i - 2T_{h+1}^{\pi,\theta} U_{h+1}(s_{h+1}^i) }\bigg]\nend
    & = \EE\sbr{\Bigrbr{U_h(s_h^i, a_h^i, b_h^i)- \TT_h^{\pi, \theta} U_{h+1}(s_h^i, a_h^i, b_h^i)}^2}, 
\end{align*}
where $\EE_{x}[\cdot]$ is a short hand of $\EE[\cdot\given x]$ and the expectation is taken with respect to the data generating distribution. Here, the second equality holds by the law of total expectation, and the third equality holds by noting that $\EE_{s_h^i, a_h^i, b_h^i}\osbr{u_h^i  + T_h^{\pi,\theta} U_{h+1}(s_{h+1}^i)} = \TT_h^{\pi, \theta} U_{h+1} (s_h^i, a_h^i, b_h^i)$.
Next, we calculate the variance, 
\begin{align*}
    \Var[{Z_h^i}] &\le \EE[{Z_h^i}^2] \nend
    &\le \EE\bigg[\Bigrbr{U_h(s_h^i, a_h^i, b_h^i)- \TT_h^{\pi, \theta} U_{h+1}(s_h^i, a_h^i, b_h^i)}^2\nend
    &\qquad \cdot \EE_{s_h^i, a_h^i, b_h^i}\sbr{\rbr{U_h(s_h^i, a_h^i, b_h^i) + \TT_h^{\pi, \theta} U_{h+1}(s_h^i, a_h^i, b_h^i)- 2u_h^i - 2T_{h+1}^{\pi,\theta} U_{h+1}(s_{h+1}^i)}^2 } \bigg]\nend
    &\le 49 B_U^2 \EE\sbr{\Bigrbr{U_h(s_h^i, a_h^i, b_h^i)- \TT_h^{\pi, \theta} U_{h+1}(s_h^i, a_h^i, b_h^i)}^2} = 49 B_U^2 \EE[{Z_h^i}].
\end{align*}
Now, by \Cref{lem:bernstein}, we have for each $y\in\cY_\epsilon$ that
\begin{align*}
    \abr{\frac 1 T \sum_{i=1}^T {Z_h^i} - \EE\sbr{{Z_h^i}}} &\le \frac{1}{2T}\sum_{i=1}^T \EE [{Z_h^i}] + \frac{110 B_U^2\cdot\log(2\delta^{-1})}{T}.
\end{align*}
Now, we extend the result to $\cY$, where we notice that for any two $y, \tilde y\in\cY$ such that $\rho(y, \tilde y) \le \epsilon$, 
\begin{align*}
    &\bigrbr{U_h(s_h^i, a_h^i, b_h^i) - u_h^i -  T_{h+1}^{\pi,\theta} U_{h+1}(s_{h+1}^i)}^2  - \bigrbr{\TT_h^{\pi, \theta} U_{h+1}(s_h^i, a_h^i, b_h^i) - u_h^i - T_{h+1}^{\pi,\theta} U_{h+1}(s_{h+1}^i)}^2 \nend
    &\qquad - \rbr{\bigrbr{\tilde U_h(s_h^i, a_h^i, b_h^i) - u_h^i -  T_{h+1}^{\tilde\pi,\tilde\theta} \tilde U_{h+1}(s_{h+1}^i)}^2  - \bigrbr{\TT_h^{\tilde\pi, \tilde\theta} \tilde U_{h+1}(s_h^i, a_h^i, b_h^i) - u_h^i - T_{h+1}^{\tilde\pi,\tilde\theta} \tilde U_{h+1}(s_{h+1}^i)}^2}\nend
    %%%%%%%%%%%%%
    &\quad \le 6 B_U \rbr{\onbr{U-\tilde U}_\infty + \bignbr{(T_{h+1}^{\pi,\theta}-T_{h+1}^{\tilde\pi,\tilde\theta} )\tilde U_{h+1}}_\infty + \bignbr{T_{h+1}^{\pi,\theta}(U_{h+1} - \tilde U_{h+1})}_\infty}\nend
    &\qquad + 6 B_U \cdot 2\rbr{\bignbr{(T_{h+1}^{\pi,\theta}-T_{h+1}^{\tilde\pi,\tilde\theta} )\tilde U_{h+1}}_\infty + \bignbr{T_{h+1}^{\pi,\theta}(U_{h+1} - \tilde U_{h+1})}}\nend
    &\quad\le 6 B_U(2\epsilon + B_U \epsilon ) + 12 B_U(B_U\epsilon + \epsilon)\nend
    &\quad\le (18 B_U^2 + 24 B_U)\epsilon, 
\end{align*}
where the second inequality follows from the definition of the covering net with respect to distance $\rho$ defined in \eqref{eq:rho-cY}. 
We obtain with probability at least $1-\delta$ that for any $y\in\cY_h$, $h\in[H]$, 
\begin{align*}
    \abr{\frac 1 T \sum_{i=1}^T {Z_h^i} - \EE\sbr{{Z_h^i}}} &\le \inf_{\epsilon>0} \frac{1}{2T}\sum_{i=1}^T \EE [{Z_h^i}] + \frac{110 B_U^2\cdot\log(H \cN_\rho(\cY_h, \epsilon)\delta^{-1})}{T} +2.5\cdot (18B_U^2 + 24 B_U)\epsilon\nend
    &\le \frac{1}{2T}\sum_{i=1}^T \EE [{Z_h^i}] + {110 B_U^2\cdot\log( H \cN_\rho(\cY_h, T^{-1})\delta^{-1}) } \cdot {T^{-1}} + (45 B_U^2 + 60 B_U )T^{-1}\nend
    & = \frac 1 T \rbr{\epsilon_S + \frac{T}{2} \EE[(U_{h} - \TT_{h}^{\pi,\theta}U_{h + 1})^2]}, 
\end{align*}
where $\epsilon_S = {110 B_U^2\cdot\log(H \max_{h\in[H]}\cN_\rho(\cY_h, T^{-1})\delta^{-1}) }  + (45 B_U^2 + 60 B_U )$, 
which proves our claim in \eqref{eq:cY-confset-1}.

\paragraph{Guarantee of the Confidence Set $\CI_{\cU}^{\pi,\theta}(\beta)$.}
We give a brief proof for the validity and the accuracy of the confidence set. For any $U_{h+1}\in\cU, \theta\in\Theta, \pi\in\Pi, h\in[H]$, on the one hand,
\begin{align}
    \ell_h(U_h, U_{h+1},\theta, \pi) - \inf_{U_h'\in\cU} \ell_h(U_h', U_{h+1}, \theta,\pi) \le \epsilon_S - \frac{T}{2} \EE[\orbr{U_{h} - \TT_{h}^{\pi,\theta}U_{h + 1}}^2], \label{eq:Bellman-loss-guarantee-1}
\end{align}
on the other hand,
\begin{align}
    \ell_h(U_h, U_{h+1},\theta, \pi) - \inf_{U_h'\in\cU} \ell_h(U_h', U_{h+1}, \theta,\pi) 
    &\ge \ell_h(U_h, U_{h+1},\theta, \pi) - \ell_h(\TT_h^{\pi, \theta} U_{h+1},  U_{h+1}, \theta,\pi) \nend
    &\ge - \epsilon_S  + \frac{T}{2} \EE[\orbr{U_{h} - \TT_{h}^{\pi,\theta}U_{h + 1}}^2], \label{eq:Bellman-loss-guarantee-2}
\end{align}
where the first inequality holds by the completeness assumption. 
For \eqref{eq:Bellman-loss-guarantee-1}, we plug in $U=U^{\pi,\theta}$ (realizability) and obtain 
$$\ell_h(U^{\pi,\theta}_h, U^{\pi,\theta}_{h+1},\theta, \pi) - \inf_{U_h'\in\cU} \ell_h(U_h', U^{\pi,\theta}_{h+1}, \theta,\pi) \le \epsilon_S.$$
Therefore, by having $\beta \ge \epsilon_S$, we have $U^{\pi,\theta}\in\CI_\cU^{\pi,\theta}(\beta)$. For the second one in \eqref{eq:Bellman-loss-guarantee-2}, we plug in any $U\in\CI_{\cU}^{\pi,\theta}(\beta)$ and obtain $\EE[\|U_{h} - \TT_{h}^{\pi,\theta}U_{h + 1}\|^2]\le T^{-1}\cdot (2\beta +2\epsilon_S)\le 4\beta T^{-1}$. Hence, we complete our proof of \Cref{lem:leader-bellman-loss}.

% \subsubsection{Proof of \Cref{lem:CI-U-online}}\label{sec:proof-CI-U-online}
% Our proof follows a similar scheme as in the proof of \Cref{lem:leader-bellman-loss}.

\paragraph{Concentration.}
We first take
\begin{align*}
    {Z_h^i} &= \ell_{h}(U'_{h}, U_{h + 1}, \theta_{h+1}, \pi) - \ell_{h}(\TT_{h}^{*,\theta}U_{h + 1}, U_{h + 1}, \theta_{h+1}, \pi)\nend
    & = \rbr{U_h(s_h^i, a_h^i, b_h^i) - u_h^i -  T_{h+1}^{*, \theta} U_{h+1}(s_{h+1}^i)}^2  - \rbr{\TT_h^{*, \theta} U_{h+1}(s_h^i, a_h^i, b_h^i) - u_h^i - T_{h+1}^{*, \theta} U_{h+1}(s_{h+1}^i)}^2.
\end{align*}
For the online setting, we have $(s_h^i, a_h^i, b_h^i, s_{h+1}^i)$ adapted to some filtration $\cF_h^i$. One choice of the filtration is $\cF_h^j=\sigma\rbr{\tau^{1:j-1}}$. Another choice of the filtration is $\cF_h^j=\sigma\rbr{(s_h^j, a_h^j, b_h^j), (s_h^i, a_h^i, b_h^i, s_{h+1}^i)_{i\in[j-1]}}$, where $\sigma(X)$ is the sigma-algebra of $X$.
They will both work for our proof.
We let $\EE^{i}_{x}[\cdot]$ be a short hand of $\EE^{i}[\cdot\given x, \cF_h^i]$ in the following proof.
We first calculate the expectation, 
\begin{align*}
    \EE^{i}[{Z_h^i}] 
    &= \EE^{i} \bigg[\EE^{i}_{s_h^i, a_h^i, b_h^i} \Big[\rbr{U_h(s_h^i, a_h^i, b_h^i) - u_h^i -  T_{h+1}^{*, \theta} U_{h+1}(s_{h+1}^i)}^2  \nend
    &\qquad - \rbr{\TT_h^{*, \theta} U_{h+1}(s_h^i, a_h^i, b_h^i) - u_h^i - T_{h+1}^{*, \theta} U_{h+1}(s_{h+1}^i)}^2\Big]\bigg]\nend
    & = \EE^{i}\bigg[\Bigrbr{U_h(s_h^i, a_h^i, b_h^i)- \TT_h^{*, \theta} U_{h+1}(s_h^i, a_h^i, b_h^i)}\nend
    &\qquad \cdot \EE^{i}_{s_h^i, a_h^i, b_h^i}\sbr{U_h(s_h^i, a_h^i, b_h^i) + \TT_h^{*, \theta} U_{h+1}(s_h^i, a_h^i, b_h^i)- 2u_h^i - 2T_{h+1}^{*, \theta} U_{h+1}(s_{h+1}^i) }\bigg]\nend
    & = \EE^{i}\sbr{\Bigrbr{U_h(s_h^i, a_h^i, b_h^i)- \TT_h^{*, \theta} U_{h+1}(s_h^i, a_h^i, b_h^i)}^2}, 
\end{align*}
where the second equality holds by the law of total expectation, and the third equality holds by noting that $\EE^{i}_{s_h^i, a_h^i, b_h^i}\osbr{u_h^i  + T_{h+1}^{*, \theta} U_{h+1}(s_{h+1}^i)} = \TT_h^{*, \theta} U_{h+1} (s_h^i, a_h^i, b_h^i)$.
Next, we calculate the variance, 
\begin{align*}
    \Var^{i}[{Z_h^i}^2] &\le \EE^{i}[{Z_h^i}^2]\nend
    &\le \EE^{i}\bigg[\Bigrbr{U_h(s_h^i, a_h^i, b_h^i)- \TT_h^{*, \theta} U_{h+1}(s_h^i, a_h^i, b_h^i)}^2\nend
    &\qquad \cdot \EE^{i}_{s_h^i, a_h^i, b_h^i}\sbr{\rbr{U_h(s_h^i, a_h^i, b_h^i) + \TT_h^{*, \theta} U_{h+1}(s_h^i, a_h^i, b_h^i)- 2u_h^i - 2T_{h+1}^{*, \theta} U_{h+1}(s_{h+1}^i)}^2 } \bigg]\nend
    &\le 49 B_U^2 \EE^{i}\sbr{\Bigrbr{U_h(s_h^i, a_h^i, b_h^i)- \TT_h^{*, \theta} U_{h+1}(s_h^i, a_h^i, b_h^i)}^2} = 49 B_U^2 \EE^{i}[{Z_h^i}].
\end{align*}
Also, one can verify that $|{Z_h^i}|\le 9B_U^2$ where $B_U$ bounds both the leader's reward and the value function class $\cU$.
We next take a $\epsilon$-covering of the class $\cZ_h=\cU^2\times\Theta_{h+1}$ with respect to the following distance defined in \eqref{eq:rho-cZ},
\begin{align*}
    &\rho\orbr{z, \tilde z}  = \max_{h\in[H]}\cbr{\bignbr{U_h-\tilde U_h}_\infty, \bignbr{ T_{h+1}^{*,\theta} U_{h+1} (\cdot) -  T_{h+1}^{*,\tilde\theta} \tilde U_{h+1} (\cdot)}_\infty }, 
\end{align*} 
We invoke the Freedman inequality \Cref{lem:freedman} for the martingale sequence $Z_h^i -\EE^i[Z_h^i]$, which says that for all $t\in [T], h\in[H], z\in\cZ_\epsilon$, it holds with probability at least $1-\delta$
\begin{align*}
    \sum_{i=1}^t \rbr{Z_h^i - \EE^i[Z_h^i]}
    &\le \frac{\lambda(e-2)}{9 B_U^2} \sum_{i=1}^t \EE^i\sbr{\rbr{Z_h^i - \EE^i[Z_h^i]}^2} + 9 \lambda^{-1} B_U^2\log(TH\cN_\rho(\cZ_h, \epsilon)\delta^{-1})\nend
    &\le \frac{\lambda(e-2) 49 }{9 } \sum_{i=1}^t \EE^i\sbr{Z_h^i} + 9 \lambda^{-1} B_U^2\log(TH\cN_\rho(\cZ_h, \epsilon)\delta^{-1}), \quad \forall \lambda\in(0, 1).
\end{align*}
Here, we plug in $\lambda = 9/98(e-2)$, $\epsilon=T^{-1}$ and notice that the above inequality also holds for $-Z_h^i + \EE^i[Z_h^i]$, which gives us for all $z\in\cZ_\epsilon, h\in[H], t\in[T]$ that
\begin{align}\label{eq:leader-Bellman-1}
    \abr{\sum_{i=1}^t \rbr{Z_h^i - \EE^i[Z_h^i]}} \le \frac 1 2 \sum_{i=1}^t \EE^i\sbr{Z_h^i} + c B_U^2\log(TH\cN_\rho(\cZ_h, T^{-1})\delta^{-1}), 
\end{align}
where we plug in $\epsilon=T^{-1}$ and $c=98(e-2)$ should be a universal constant. 
Next, we notice that
\begin{align*}
    &\bigrbr{U_h(s_h^i, a_h^i, b_h^i) - u_h^i -  T_{h+1}^{*,\theta} U_{h+1}(s_{h+1}^i)}^2  - \bigrbr{\TT_h^{*, \theta} U_{h+1}(s_h^i, a_h^i, b_h^i) - u_h^i - T_{h+1}^{*,\theta} U_{h+1}(s_{h+1}^i)}^2 \nend
    &\qquad - \rbr{\bigrbr{\tilde U_h(s_h^i, a_h^i, b_h^i) - u_h^i -  T_{h+1}^{*,\tilde\theta} \tilde U_{h+1}(s_{h+1}^i)}^2  - \bigrbr{\TT_h^{*, \tilde\theta} \tilde U_{h+1}(s_h^i, a_h^i, b_h^i) - u_h^i - T_{h+1}^{*,\tilde\theta} \tilde U_{h+1}(s_{h+1}^i)}^2}\nend
    %%%%%%%%%%%%%
    &\quad \le 6 B_U \rbr{\onbr{U-\tilde U}_\infty + \bignbr{(T_{h+1}^{*,\theta}-T_{h+1}^{*,\tilde\theta} )\tilde U_{h+1}}_\infty + \bignbr{T_{h+1}^{*,\theta}(U_{h+1} - \tilde U_{h+1})}_\infty}\nend
    &\qquad + 6 B_U \cdot 2\rbr{\bignbr{(T_{h+1}^{*,\theta}-T_{h+1}^{*,\tilde\theta} )\tilde U_{h+1}}_\infty + \bignbr{T_{h+1}^{*,\theta}(U_{h+1} - \tilde U_{h+1})}}\nend
    &\quad\le 6 B_U(2\epsilon + B_U \epsilon ) + 12 B_U(B_U\epsilon + \epsilon) \le (18 B_U^2 + 24 B_U)\epsilon, 
\end{align*}
Now, we let $\epsilon_S=c B_U^2 \allowbreak \log(TH\cN_\rho(\cZ_h, T^{-1})\delta^{-1}) + (45 B_U^2 + 60 B_u)$ and extend the result in \eqref{eq:leader-Bellman-1} to the whole class $\cY$, 
\begin{align*}
    \abr{\sum_{i=1}^t \rbr{Z_h^i - \EE^i[Z_h^i]}} 
    &\le \frac 1 2 \sum_{i=1}^t \EE^i\sbr{Z_h^i} + c B_U^2\log(TH\cN_\rho(\cZ, T^{-1})\delta^{-1}) + 2.5\cdot(18 B_U^2 + 24 B_U)\nend
    &\le \frac 1 2 \sum_{i=1}^t \EE^i\sbr{Z_h^i} + \epsilon_S.
\end{align*}
the following argument follows exactly the same as \eqref{eq:Bellman-loss-guarantee-1} and \eqref{eq:Bellman-loss-guarantee-2}, 
where we use the realizability assumption that $U^{*,\theta}\in \cU$ and the completeness assumption that there exists $U'\in\cU$ such that $U'=\TT_h^{*, \theta} U$ for any $U\in\cU, \theta\in\Theta$. 
We finish our proof of \Cref{lem:CI-U-online}.

% \subsubsection{Proof of \Cref{lem:1st-ub}}
% \label{sec:proof-1st-ub}
% Recall by definition, 
\begin{align*}
    \tilde \Delta^{(1)}_h(s_h, b_h) &=  \rbr{\EE_{s_h, b_h} -\EE_{s_h}}\Biggsbr{\sum_{l=h}^H \gamma^{l-h}\underbrace{\rbr{\tilde Q_l - r_l^\pi - \gamma P_l^\pi \tilde V_{l+1}}(s_l, b_l)}_{\ds\text{Follower's Bellman error}}}.
\end{align*}
In this section, we will bound $\EE\sbr{\abr{\tilde \Delta^{(1)}_h(s_h, b_h)}}$ by the KL distance in the following way.
% \paragraph{For Small $\eta \bigabr{\tilde A- A}$.}
% On the one hand, we have by the Cauchy-Schwartz inequality that
% \begin{align*}
%     \EE \exp\rbr{\eta \bigabr{\tilde A- A}} \cdot \EE \sbr{\exp\rbr{- \eta \bigabr{\tilde A- A}}\cdot \bigabr{\tilde A-A}^2} \ge \rbr{\EE\sbr{\bigabr{\tilde A-A}}}^2.
% \end{align*}
% On the other hand, the second term on the left hand side can be bounded by the Hellinger distance, 
% \begin{align*}
%     \EE D_\H^2(\nu, \tilde\nu) = \EE \inp[\bigg]{\nu}{\biggrbr{1-\sqrt{\frac{\tilde\nu}{\nu}}}^2} \ge \EE \sbr{\eta^2 \exp\rbr{- \eta \bigabr{\tilde A- A}}\cdot \bigabr{\tilde A-A}^2}.
% \end{align*}
% Hence, we conclude that
% \begin{align*}
%     \EE \exp\rbr{\eta \bigabr{\tilde A- A}} \cdot \EE D_\H^2(\nu, \tilde\nu)\ge \eta^2  \cdot \rbr{\EE\bigabr{\tilde A-A}}^2,
% \end{align*}
% and also
% \begin{align}
%     \sum_{l=h}^H \gamma^{l-h}\EE \exp\rbr{\eta \bigabr{\tilde A_l- A_l}} \cdot \sum_{l=h}^H \gamma^{l-h}\EE D_\H^2 \rbr{\nu_l, \tilde\nu_l} \ge \eta^2 \rbr{\sum_{l=h}^H \gamma^{l-h} \EE \bigabr{\tilde A_l-A_l}}^2.\label{eq:A-cauchy}
% \end{align}
We follow from the decomposition of the A-function in \Cref{lem:AQV-func diff},
\begin{align*}
    \abr{\tilde\Delta_h^{(1)}(s_h, b_h)} 
    &\le \bigabr{\bigrbr{A_h-\tilde A_h}(s_h, b_h)} + 2\eta^{-1} \Delta_h^{(2)}(s_h) \nend
    &\le \bigabr{\bigrbr{A_h-\tilde A_h}(s_h, b_h)} + 2\EE_{s_h}\sbr{\sum_{l=h}^H \gamma^{l-h} \inp[]{\nu_l(\cdot\given s_l)}{(A_l-\tilde A_l)(s_l, b_l)}}, 
\end{align*}
where we use the definition $\Delta_h^{(2)}(s_h)\defeq \EE_{s_h}\sbr{\sum_{l=h}^H \gamma^{l-h} \kl\infdivx[]{\nu_l}{\tilde\nu_l}}$ and the last inequality holds by noting that $\kl\infdivx[]{\nu}{\tilde\nu} = \eta\inp[]{\nu}{A-\tilde A}$.
Therefore, we conclude that
\begin{align}
    \EE\sbr{\abr{\tilde \Delta^{(1)}_h(s_h, b_h)}} 
    &\le  \EE\bigabr{\bigrbr{A_h-\tilde A_h}(s_h, b_h)} + 2\EE\sbr{\sum_{l=h}^H \gamma^{l-h} \inp[]{\nu_l(\cdot\given s_l)}{(A_l-\tilde A_l)(s_l, \cdot)}}
    \nend
    &
    \le 3 \sum_{l=h}^H \gamma^{l-h} \EE\bigabr{(A_l-\tilde A_l)(s_l, b_l)} \label{eq:Delta-A}
    % \\
    % &\le 3 \eta^{-1} \sqrt{\sum_{l=h}^H \gamma^{l-h}\EE \exp\rbr{\eta \bigabr{\tilde A_l- A_l}}} \cdot D_\RL(M^*,\tilde M;\pi),\nonumber
\end{align}
% where the last inequality is a direct result of \eqref{eq:A-cauchy}.
% \paragraph{For Large $\eta \bigabr{\tA - A}$. }
We now invoke the lower bound \eqref{eq:nu-tv-lb-1} in \Cref{lem:response diff} and obtain
\begin{align*}
    D_\TV(\nu_h, \tnu_h) &\ge \frac{1-\exp\rbr{-2\eta B_A}}{4 B_A} \cdot {\EE_{s_h}\bigabr{(\tA_h-A_h)(s_h, b_h)} } \nend
    &\ge  \frac{\eta}{2(1+ 2\eta B_A)} \cdot {\EE_{s_h}\bigabr{(\tA_h-A_h)(s_h, b_h)} }.
\end{align*}
Combining these results, we obtain
\begin{align*}
    \EE\sbr{\abr{\tilde \Delta^{(1)}_h(s_h, b_h)}} &\le 3\sum_{l=h}^H \gamma^{l-h} \EE\bigabr{(\tA_l-A_l)(s_l,b_l)} \nend
    &\le 3 \cdot \rbr{\frac{\eta}{2(1+ 2\eta B_A)}}^{-1} \sum_{l=h}^H \gamma^{l-h}\EE D_\TV(\nu_l(\cdot, s_l),\tilde\nu_l(\cdot, s_l)) \nend
    &\le 6(1+2\eta B_A)\cdot  {\frac{1-\gamma^H}{1-\gamma}}  \cdot \eta^{-1} \max_{h\in[H]} \EE D_\TV(\nu_h(\cdot, s_h),\tilde\nu_h(\cdot, s_h)),
\end{align*}
where the last inequality follows from from the fact that $(1-\exp(-x))/2x\ge 1/2(1+x)$.
% Hence, we have
% \begin{align*}
%     \EE\sbr{\abr{\tilde \Delta^{(1)}_h(s_h, b_h)}} &\le   3 \cdot \frac{4 \eta B_A}{1-\exp\rbr{-2\eta B_A}}\cdot
%     \sqrt{\frac{1-\gamma^H}{1-\gamma}}  \cdot
%     \eta^{-1}D_\RL(M^*,\tilde M;\pi)\nend
%     &\le 6(1+2\eta B_A)\cdot  \sqrt{\frac{1-\gamma^H}{1-\gamma}}  \cdot \eta^{-1}D_\RL(M^*,\tilde M;\pi),
% \end{align*}
Hence, we complete the proof of the first order of $\tilde \Delta^{(1)}$ in \Cref{lem:1st-ub}.

In the sequel, we will study how to upper bound $\bigrbr{\tilde \Delta_h^{(1)}(s_h, b_h)}^2$. We first have by \Cref{lem:AQV-func diff} that
\begin{align*}
    \bigrbr{\tilde \Delta_h^{(1)}(s_h, b_h)}^2 
    &= 2 \rbr{\rbr{\EE_{s_h, b_h}-\EE_{s_h}} \bigsbr{\orbr{A_h - \tilde A_h}(s_h, b_h)}}^2 + 2 \gamma^2 \eta^{-2} \rbr{\rbr{\EE_{s_h, b_h}-\EE_{s_h}}\bigsbr{\Delta_{h+1}^{(2)}(s_{h+1})}}^2\nend
    & \le 2 \rbr{\rbr{\EE_{s_h, b_h}-\EE_{s_h}} \bigsbr{\orbr{Q_h - \tilde Q_h}(s_h, b_h)}}^2 + 4 \gamma^2 \eta^{-2} \rbr{\EE_{s_h, b_h}\bigsbr{\Delta_{h+1}^{(2)}(s_{h+1})}}^2  \nend
    &\qquad + 4 \gamma^2 \eta^{-2} \rbr{\EE_{s_h}\bigsbr{\Delta_{h+1}^{(2)}(s_{h+1})}}^2,
\end{align*}
where the last inequality holds by using \eqref{eq:A diff-1} and note that $\eta^{-1}\kl\infdivx[]{\nu_h}{\tilde\nu_h} = \EE_{s_h}[A_h-\tilde A_h]$. By definition of $\Delta_h^{(2)}(s_h)$, we just focus on the second term and obtain 
\begin{align*}
    \rbr{\eta^{-1} \EE_{s_h, b_h}\bigsbr{\Delta_{h+1}^{(2)}(s_{h+1})} }^2
    &= \rbr{\eta^{-1}\EE_{s_h, b_h}\sbr{\sum_{l=h+1}^H \gamma^{l-h-1} \kl\infdivx[]{\nu_l(\cdot\given s_l)}{\tilde\nu_l(\cdot\given s_l)}}}^2\nend
    & = \rbr{\EE_{s_h, b_h}\sbr{\sum_{l=h+1}^H \gamma^{l-h-1} \inp[\big]{\nu_l(\cdot\given s_l)}{(A_l-\tilde A_l)(s_l, \cdot)}_\cB}}^2\nend
    &\le \eff_H(\gamma) \sum_{l=h+1}^H \gamma^{l-h-1}\rbr{\EE_{s_h, b_h}\sbr{\inp[\big]{\nu_l(\cdot\given s_l)}{\bigabr{(A_l-\tilde A_l)(s_l, \cdot)}}_\cB}}^2,
\end{align*}
where the last inequality follows from the Cauchy-Schwartz inequaltiy and we recall $\eff_H(\gamma) = (1-\gamma^H)/(1-\gamma)$.
We now invoke the lower bound \eqref{eq:nu-tv-lb-1} in \Cref{lem:response diff} and obtain
\begin{align*}
    D_\TV(\nu_h, \tnu_h) \ge \frac{\eta}{2(1+2\eta B_A)} \cdot \inp[\big]{\nu_h(\cdot\given s_h)}{\bigabr{
    \orbr{\tilde A - A}(s_h,\cdot)}}.
\end{align*}
Combining these results, we obtain
\begin{align*}
    &\rbr{\eta^{-1} \EE_{s_h, b_h}\bigsbr{\Delta_{h+1}^{(2)}(s_{h+1})} }^2 \nend
    &\quad \le 4\rbr{\eta^{-1} +2 B_A}^2\eff_H(\gamma) \sum_{l=h+1}^H \gamma^{l-h-1} {\EE_{s_h, b_h}\sbr{D_\H^2(\nu_l(\cdot\given s_l), \tilde\nu_l(\cdot\given s_l))}}, 
\end{align*}
where the inequality holds by using the Jensen's  inequality and move the expectation outside of the square. As a result, 
\begin{align*}
    &\bigrbr{\tilde \Delta_h^{(1)}(s_h, b_h)}^2  \nend
    &\quad \le 2 \rbr{\rbr{\EE_{s_h, b_h}-\EE_{s_h}} \bigsbr{\orbr{Q_h - \tilde Q_h}(s_h, b_h)}}^2 \nend
    &\qqquad + 16 \gamma^2  \rbr{\eta^{-1} +2 B_A}^2\eff_H(\gamma) \sum_{l=h+1}^H \gamma^{l-h-1} {\rbr{\EE_{s_h}+\EE_{s_h, b_h}}\sbr{D_\H^2(\nu_l(\cdot\given s_l), \tilde\nu_l(\cdot\given s_l))}}, 
\end{align*}
which completes our proof of \Cref{lem:1st-ub}.

% \subsubsection{Proof of \Cref{lem:2nd-ub}}
% \label{sec:proof-2nd-ub}
% In this proof, we remind readers that $Q, A, r^\pi$ are functions from $\cS\times\cB$ to $\RR$, $V:\cS\times\RR$ and $P_h^\pi:\cS\times\cB\rightarrow\Delta(\cS)$. In the sequel, we will neglect the dependence on $s_h, b_h$ for simplicity.
The major part in this proof is to upper bound $\EE\osbr{\orbr{( P_h^\pi-\tilde P_h^{\pi})\tilde V_{h+1}}^2}$ and $\EE\osbr{\orbr{r_h^\pi-\tilde r_h^\pi}^2}$ by $D_{\RL, h}^2$ separately. 
Moreover, we use $B_A$ in \eqref{eq:define_BA} to bound the follower's Q- and A-function. 
We will leave out the dependence on $(s_h, b_h)$ most of the times in the following proof when it does not cause any confusion in the context.

Note that we only have guarantee for $D_\TV^2(\nu_h, \tilde\nu_h)$ by MLE, which cannot directly guarantee that the true utility is identifiable since a constant shift does not change the follower's behavior at all. For the reward to be identifiable, we need an additional linear constraint, namely $\inp{x}{r_h(s_h, a_h, \cdot)}=\varsigma$.
We start with the easier part with the transition kernel.
\begin{align*}
    \EE\osbr{\orbr{( P_h^\pi-\tilde P_h^{\pi})\tilde V_{h+1}}^2}\le 2^2 B_A^2 \EE\sbr{D_\TV^2( P_h^\pi, \tilde  P_h^\pi)}.
\end{align*}
For the follower's reward, we take a real number $\xi$ and have the following decomposition
\begin{align*}
    &\inf_{\xi\in\RR}\EE_{s_h}\abr{r_h^\pi-\tilde r_h^\pi - \xi}  \nend
    &\quad = \inf_{\xi\in\RR}\EE_{s_h}\bigabr{Q_h-\tilde Q_h - \xi - \gamma\bigrbr{ P_h^\pi-\tilde P_h^\pi}\tilde V_{h+1} - \gamma  P_h^\pi\bigrbr{V_{h+1}-\tilde V_{h+1}}}\nend
    &\quad \le \inf_{\xi\in\RR}\EE_{s_h}\bigabr{Q_h-\tilde Q_h - \xi} + \gamma \EE_{s_h}\bigabr{\bigrbr{ P_h^\pi-\tilde P_h^\pi}\tilde V_{h+1} } + \gamma\exp\rbr{2\eta B_A}\EE_{s_h}\bigabr{Q_{h+1} - \tilde Q_{h+1}}, \nend
    &\quad \le \EE_{s_h}\bigabr{A_h-\tilde A_h} + \gamma\EE_{s_h}\bigabr{\bigrbr{ P_h^\pi-\tilde P_h^\pi}\tilde V_{h+1} } + \gamma\exp\rbr{2\eta B_A}\EE_{s_h}\bigabr{Q_{h+1} - \tilde Q_{h+1}}
\end{align*}
where the first inequality holds by the same argument for $V-\tilde V$ in \eqref{eq:f_2-1}, and the second inequality holds simply by plugging $\xi = V_h(s_h) - \tilde V_h(s_h)$. Now, we can plug in the bound for $\EE_{s_h}\oabr{A_h-\tilde A_h}$ in \Cref{lem:response diff} and obtain
\begin{align}
    &\inf_{\xi\in\RR}\EE_{s_h}\abr{r_h^\pi-\tilde r_h^\pi - \xi} \nend
    &\quad \le \underbrace{2(\eta^{-1}+2B_A) D_\TV(\nu_h, \tilde\nu_h) + 2 \gamma B_A \EE_{s_h} D_\TV( P_h^\pi, \tilde P_h^\pi)}_{\ds \sD_h} + \gamma\exp\rbr{2\eta B_A}\EE_{s_h}\bigabr{Q_{h+1} - \tilde Q_{h+1}}.
    % &\le \rbr{2\eta^{-1}(1+2\eta B_A)+2\gamma B_U} D_{\RL,h}(\tilde M,M^*;\pi) + \gamma\exp\rbr{2\eta B_A}\EE_{s_h}\abr{Q_{h+1} - \tilde Q_{h+1}}, 
    \label{eq:f_2-r-diff}
\end{align}
We next show what we can say about the utility when combining the guarantee of \eqref{eq:f_2-r-diff} with the linear constraint $\inp{x}{r_h(s_h, a_h, \cdot)}=\varsigma$. Specifically, we have the following lemma.
\begin{lemma}[Identification of the follower's utility]\label{lem:identification}
    Suppose for $r, \tilde r:\cB\rightarrow \RR$, for some distribution $\nu\in\Delta(\cB)$ such that $\nu>0$, we have $\inf_{\xi\in\RR}\inp{\nu}{\abr{r-\tilde r-\xi}}\le \varepsilon$ and $\inp{x}{r-\tilde r}=0$ hold at the same time for some $x:\cB\rightarrow \RR$ such that $\inp{\ind}{x}\neq 0$. We have
    \begin{align*}
        \inp{\nu}{\abr{r-\tilde r}}\le\rbr{1 + \nbr{\frac x \nu}_\infty \cdot \frac{1}{\abr{\inp{x}{\ind}}}} \epsilon
    \end{align*}
    \begin{proof}
        See \Cref{sec:proof-identification} for a detailed proof.
    \end{proof}
\end{lemma}
With \Cref{lem:identification}, we conclude with $\nbr{\nu}_\infty\ge \exp\rbr{-\eta B_A}$ and $\kappa = \nbr{x}_\infty/|\la x, \ind\ra|$ that
\begin{align*}
    \EE_{s_h}\abr{r_h^\pi-\tilde r_h^\pi} &\le \rbr{1+\exp\rbr{2\eta B_A} \kappa} \bigrbr{\sD_h + \gamma \exp\rbr{2\eta B_A}\cdot\EE_{s_h}\bigabr{Q_{h+1} - \tilde Q_{h+1}}}.
\end{align*}
On the other hand, for the Q-function, we have by \eqref{eq:f_2-Q-ub} that 
\begin{align*}
    \EE_{s_h}\bigabr{Q_h - \tilde Q_h}
    &\le \EE_{s_h}\bigabr{r_h^\pi-\tilde r_h^\pi + \gamma \bigrbr{ P_h^\pi-\tilde P_h^\pi}\tilde V_{h+1}} + \gamma \exp\rbr{2\eta B_A} {\EE_{s_h}\bigabr{Q_{h+1}-\tilde Q_{h+1}}}\nend
    &\le \underbrace{2\rbr{1+\exp\rbr{2\eta B_A} \kappa} }_{\ds c_1}\cdot \sD_h \nend
    &\qquad + \underbrace{\rbr{2+\exp\rbr{2\eta B_A} \kappa}\gamma \exp\rbr{2\eta B_A}}_{\ds c_2}\cdot\EE_{s_h}\bigabr{Q_{h+1} - \tilde Q_{h+1}}, 
\end{align*}
where in the last inequality, we directly upper bound $\EE_{s_h}|\gamma(P_h^\pi - \tilde P_h^\pi)\tilde V_{h+1}|$ by $\sD_h$. 
Therefore, we have by a recursive argument that 
\begin{align*}
    \EE_{s_h}\bigabr{Q_h -\tilde Q_h} \le \sum_{l=h}^H c_2^{l-h} c_1 \EE_{s_h}\sD_l.
\end{align*}
For now, we are able to deal with $(\EE_{s_h}\abr{r_h^\pi-\tilde r_h^\pi})^2$. However, note that what we actually want to get is the version with the square within the expectation $\EE_{s_h}$, i.e.,  $\EE\rbr{r_h^\pi-\tilde r_h^\pi}^2$. Therefore, we need a variance-mean decomposition,
\begin{align*}
    \EE\rbr{r_h^\pi-\tilde r_h^\pi}^2&= \EE\bigrbr{Q_h-\tilde Q_h - \gamma \bigrbr{ P_h^\pi -\tilde P_h^\pi}\tilde V_{h+1} -\gamma  P_h^\pi \bigrbr{V_{h+1}-\tilde V_{h+1}}}^2\nend
    &\le 4\EE\bigsbr{\bigrbr{A_h -\tilde A_h}^2 + \bigrbr{V_h-\tilde V_h}^2 + \gamma^2 \bigrbr{\bigrbr{ P_h^\pi -\tilde P_h^\pi}\tilde V_{h+1}}^2 +\gamma^2 \bigrbr{V_{h+1}-\tilde V_{h+1}}^2}\nend
    &\le 4\EE\bigsbr{\bigrbr{A_h -\tilde A_h}^2} + 16\gamma^2 B_A^2\EE\bigsbr{D_\TV( P_h^\pi,\tilde  P_h^\pi)^2}\nend
    &\qquad + 4\exp\rbr{4\eta B_A}\EE\bigsbr{\bigrbr{\EE_{s_h}\bigabr{Q_h-\tilde Q_h}}^2+ \bigrbr{\EE_{s_{h+1}}\bigabr{Q_{h+1}-\tilde Q_{h+1}}}^2}, 
\end{align*}
where in the first inequality, we use $Q=A+V$, and use the Jensen's inequality to derive the last term.
The last inequality holds by noting the upper bound for difference in the V-function used in \Cref{eq:f_2-1}.
We notice that the first term can be upper bounded by the squared Hellinger distance,
\begin{align*}
    D_\H^2\rbr{\nu_h,\tilde\nu_h} &= \dotp{\nu_h}{\rbr{1-\sqrt\frac{\tilde \nu_h}{\nu_h}}^2}_\cB\nend
    &= \Bigdotp{\nu_h}{\rbr{1-\exp\rbr{\frac \eta 2 (\tilde A_h - A_h)}}^2}_\cB\nend
    &\ge \rbr{\frac{1-\exp\rbr{-\eta B_A}}{2 B_A}}^2  \cdot \bigdotp{\nu_h}{\bigrbr{A_h-\tilde A_h}^2}_\cB \ge \rbr{\frac{\eta}{2}}^2  \cdot \bigdotp{\nu_h}{\bigrbr{A_h-\tilde A_h}^2}_\cB,
\end{align*}
where the first inequality holds by noting that $|1-\exp(x)|\ge (1-\exp(-B))|x|/B$ for any $|x|\le B$.
The last inequality uses the inequality $(1-\exp(-x))\ge x/(1+x)$ for all $x>0$. 
Therefore, we have the follower's squared reward difference bounded by
\begin{align*}
    &\EE\rbr{r_h^\pi -\tilde r_h^\pi}^2 \nend
    &\quad \le 16\eta^{-2} \EE D_\H^2(\nu_h,\tilde\nu_h) + 16\gamma^2 B_A^2\EE D_\TV^2( P_h^\pi,\tilde  P_h^\pi)\nend
    &\qqquad + 4 \exp\rbr{4\eta B_A} \EE\sbr{\rbr{\sum_{l=h}^H c_2^{l-h} c_1 \EE_{s_h}\sD_l}^2 + \rbr{\sum_{l=h+1}^H c_2^{l-h} c_1 \EE_{s_{h+1}}\sD_l}^2}\nend
    &\quad \le 16\eta^{-2} \EE D_\H^2(\nu_h,\tilde\nu_h) + 16\gamma^2 B_A^2\EE D_\TV^2( P_h^\pi,\tilde  P_h^\pi)\nend
    &\qqquad + 8 H \eff_H(c_2)^2 c_1^2 \exp\rbr{4\eta B_A} \max_{h\in[H]}\EE\sbr{{\sD_h}^2}\nend
    &\quad \le 16\eta^{-2} \EE D_\H^2(\nu_h,\tilde\nu_h) + 16\gamma^2 B_A^2\EE D_\TV^2( P_h^\pi,\tilde  P_h^\pi)\nend
    &\qqquad + 8 H \eff_H(c_2)^2 c_1^2\exp\rbr{4\eta B_A} \max_{h\in[H]}\EE\sbr{\rbr{{2(\eta^{-1}+2B_A) D_\TV(\nu_h, \tilde\nu_h) + 2 \gamma B_A \EE_{s_h} D_\TV( P_h^\pi, \tilde P_h^\pi)}}^2}, 
\end{align*}
where in the second inequality, we uses the Cauchy-Schwartz inequality that $\EE(\sum a_l x_l)^2\le \sum a_l \cdot \EE\sum a_l x_l^2 \le (\sum a_l)^2 \cdot \max_l \EE b_l^2$ for constant sequence $a_l>0$. 
In summary, we have
\begin{align*}
    \EE\rbr{r_h^\pi -\tilde r_h^\pi}^2
    &\le {32 H^2 \eff_H(c_2)^2 c_1^2\exp\rbr{4\eta B_A} \rbr{4(\eta^{-1}+2B_A)^2+4\gamma^2 B_A^2 }}  \nend
    & \qquad \cdot \max_{h\in[H]}\cbr{\EE D_\H^2(\nu_h,\tilde\nu_h)+\EE D_\TV^2(P_h^\pi,\tilde P_h^\pi)}\nend
    &\le \underbrace{640 H^2 \eff_H(c_2)^2 c_1^2\exp\rbr{4\eta B_A} (\eta^{-1}+B_A)^2}_{\ds c_3/4}  \nend
    & \qquad \cdot \max_{h\in[H]}\cbr{\EE D_\H^2(\nu_h,\tilde\nu_h)+\EE D_\TV^2(P_h^\pi,\tilde P_h^\pi)}
\end{align*}
Therefore, we conclude that
\begin{align*}
    \max_{h\in[H]}\EE\sbr{ \bigrbr{{\tilde Q_h - r_h^\pi - \gamma P_h^\pi \tilde V_{h+1}}}^2} 
    &\le 2 \max_{h\in[H]} \EE\sbr{\rbr{\tilde r_h^\pi - r_h^\pi}^2} + 2 \gamma^2\max_{h\in[H]} \EE\sbr{\bigrbr{\bigrbr{\tilde P_h^\pi - P_h^\pi}\tilde V_{h+1}}^2}\nend
    &\le c_3 \max_{h\in[H]}\cbr{\EE D_\H^2(\nu_h,\tilde\nu_h)+\EE D_\TV^2(P_h^\pi,\tilde P_h^\pi)}, 
\end{align*}
which completes our proof of \Cref{lem:2nd-ub}
% where the first inequality holds by noting that $C_\eta\le 2(1+2\eta B_A)$. 
% \newpage
% which completes the proof of \Cref{cor:online linear}.


\subsubsection{Proof of \Cref{lem:identification}}\label{sec:proof-identification}
For condition $\inf_{\xi\in\RR}\inp{\nu}{\abr{r-\tilde r-\xi}}\le \varepsilon$, we assume that the infimum is achieved at $\xi^*$. Let $r^* = r -\xi^*$ and we have
\begin{align*}
\abr{\inp{r^*-\tilde r}{x}} \le \inp{\abr{r^*-\tilde r}}{\abr{x}} \le \inp{\abr{r^*-\tilde r}}{\nu} \cdot \nbr{\frac{x}{\nu}}_\infty\le \varepsilon\nbr{\frac{x}{\nu}}_\infty,
\end{align*}
where the second inequality is just a distribution shift and the last inequality is given by the condition. Furthermore, for our target,
\begin{align*}
    \inp{\abr{r-\tilde r}}{\nu} \le \inp{\abr{r^*-\tilde r}}{\nu} + \abr{\xi^*} = \varepsilon + \abr{\inp{r^*-r}{x}} \cdot \frac{1}{\abr{\inp{x}{\ind}}},
\end{align*}
where the inequality follows from the triangle inequality and the equality holds by noting that $\inp{x}{\ind}\neq 0$ and $\inp{\abr{r^*-\tilde r}}{\nu}\le \varepsilon$. We bridge these two inequalities by noting that
\begin{align*}
    \abr{\inp{r^*-\tilde r}{x}} = \abr{\inp{r-\tilde r}{x} + \inp{r^*-r}{x} } = \abr{\inp{r^*-r}{x} },
\end{align*}
where the second inequality holds by noting that $\inp{r-\tilde r}{x}=0$. Combining these results and we have
\begin{align*}
    \inp{\abr{r-\tilde r}}{\nu} \le \epsilon + \abr{\inp{r^*-r}{x}} \cdot \frac{1}{\abr{\inp{x}{\ind}}} \le \rbr{1 + \nbr{\frac x \nu}_\infty \cdot \frac{1}{\abr{\inp{x}{\ind}}}} \epsilon, 
\end{align*}
which completes the proof on \Cref{lem:identification}.


% \section{Proof of \Cref{lem:bandit} on MLE Estimator}\label{sec:proof-bandit}
Recall that $\hat\theta_\MLE$ minimizes the negative log likelihood 
\begin{align*}
    \cL^{(1)}_\cD(\theta) \defeq - \frac 1 T \sum_{t=1}^T \underbrace{\rbr{\eta \inp[]{\phi^t(b^t)} {\theta} - \log \rbr{\sum_{b'\in\cB}\exp\rbr{ \eta\inp[]{\phi^t(b')}{\theta} }}}}_{\ds \ell^t(\theta)},  
\end{align*}
where we abbreviate $\phi^{\alpha^t}(s^t, b)$ to $\phi^t(b)$ since both $\alpha^t$ and $s^t$ are uniquely determined by $t$. Showing the convergence of $\hat\theta_\MLE$ by directly working with $\cL^{(1)}_\cD(\theta)$ is not satisfactory since the Hessian of $\cL^{(1)}_\cD(\theta)$ is given by a direct computation
\begin{align*}
    \nabla^2\cL^{(1)}_\cD(\theta) = \frac 1 T \sum_{t=1}^T \eta^2 \EE_{b\sim\nu^{\pi^t, \theta}}\Bigsbr{
        \underbrace{\rbr{\phi^t(b) - \EE^{\nu^{\pi^t, \theta}}\phi^t}}_{\ds \psi^{t,\theta}(b)}
        \rbr{\phi^t(b) - \EE^{\nu^{\pi^t, \theta}}\phi^t}^\top }.
\end{align*}
Note that $\psi^{t, \theta}$ also depends on the choice of $\theta$, which makes the Hessian highly nonlinear. The strong convexity of this $\cL^{(1)}_\cD(\theta)$ is thus determined by the \textit{worst} $\theta$. Current techniques for lower bounding this Hessian hence suffers from a coefficient exponential in $\eta$. See Lemma Theorem 4 in \citet{shah2015estimation} where the upper bound depends on $\lambda_2(H)^{-1}$ and this eigenvalue can be $\exp(-\eta B_r)$. Or one can check from the proof of Lemma 3.1 in \citet{zhu2023principled} where the upper bound depends on $\gamma^{-1}$ with $\gamma=1/(2+\exp(-\eta B_r)+\exp(\eta B_r))$. This is not ideal for our setting since we may have a rather large $\eta$ if the follower has a higher level of rationality.
Therefore, our question is whether we can come up with an upper bound that only depends polynomial (later we will show that it is possible for linearity dependence) on $\eta$.

The key idea is replacing the negative log likelihood $\cL^{(1)}_\cD(\cdot)$ with some curve with constant Hessian $\EE_{\nu^{\pi^t, \theta^*}}[\psi^{t, \theta^*} {\psi^{t, \theta^*}}^\top]$ in the neighbourhood of $\theta^*$. Specifically, we first lower bound $\cL^{(1)}_\cD(\hat\theta_\MLE)-\cL^{(1)}_\cD(\theta^*)$ by the Hellinger distance $T^{-1}\cdot\sum_{t=1}^T D_\H^2(\nu^{\pi^t, \hat\theta_\MLE}, \nu^{\pi^t, \theta^*})$ and some $\cO(T^{-1})$ term. Then, a careful scrutiny of the Hellinger distance will show that $$D_\H^2(\nu^{\pi^t, \theta}, \nu^{\pi^t, \theta^*})\ge B_A^{-2} (\theta-\theta^*)^\top\EE^{\nu^{\pi^t, \theta^*}}[\psi^{t, \theta^*} {\psi^{t, \theta^*}}^\top](\theta-\theta^*).$$
To make the above discussion rigorous, we first invoke the following concentration Lemma.



Using \Cref{lem:freeman-variation} with $X_t = (-\ell^t(\theta) + \ell^t(\theta^*))/2$ and taking a union bound over $\theta\in\Theta$, we have with probability $1-\delta$ and for all $\theta\in\Theta$ that
\begin{align}
    \frac 1 2 \rbr{-\cL^{(1)}_\cD(\theta) + \cL^{(1)}_\cD(\theta^*)} &\le \frac 1 T \sum_{t=1}^T \log\EE^{\nu^{\pi^t, \theta^*}}\sbr{\sqrt\frac{\nu^{\pi^t,\theta}}{\nu^{\pi^t, \theta^*}}}  + \frac 1 T\log\rbr{\frac{\cN(\Theta, \epsilon)}{\delta}} + \epsilon\nend
    &\le - \frac 1 T \sum_{t=1}^T D_\H^2\rbr{\nu^{\pi^t, \theta}, \nu^{\pi^t, \theta^*}} + \frac 1 T\log\rbr{\frac{\cN(\Theta, \epsilon)}{\delta}} + \epsilon, \label{eq:KL-2-D_H}
\end{align}
where the first inequality holds since we take a $\epsilon$-covering net for each $\ell^t(\theta)$ over $\Theta$. Note that $\log(\cN(\Theta, \epsilon))$ only grows with $\log(\eta T)$ since $\ell^t(\theta)$ is $2\eta$-Lipschitz with respect to $\theta$. 
The second inequality holds by noting that $\log(x)\le x-1$ and by the definition of the Hellinger distance.
Note that the left hand side of \eqref{eq:KL-2-D_H} is lower bounded by $-\cL_\cD(\hat\theta)+\cL_\cD(\hat\theta_\MLE)$ if we plug in $\hat\theta\in \Theta$ by definition, which yields
\begin{align*}
    \frac 1 T \sum_{t=1}^T D_\H^2\rbr{\nu^{\pi^t, \hat\theta}, \nu^{\pi^t, \theta^*}} \le \frac 1 T\log\rbr{\frac{\cN(\Theta, \epsilon)}{\delta}} + \epsilon + \cL_\cD(\hat\theta)-\cL_\cD(\hat\theta_\MLE)
\end{align*}

We next lower bound the Hellinger distance with a quadratic form of $\hat\theta-\theta^*$.Invoking \Cref{lem:D_H-2-A^2}, we have that
\begin{align*}
    D_\H^2\rbr{\nu^{\pi^t, \hat\theta}, \nu^{\pi^t, \theta^*}} \ge \rbr{\frac{\eta}{1+\eta B_A}}^2\cdot \inp[\Big]{\nu^{\pi^t, \theta^*}}{ \rbr{A^{\pi^t, \hat\theta}-A^{\pi^t, \theta^*}}^2}, 
\end{align*}
where $B_A$ upper bound the advantage function $A$ and $A^{\pi^t, \theta}$ is a short hand of $A^{\pi^t, \theta}(s^t,)$. We further have
\begin{align*}
    \inp[\Big]{\nu^{\pi^t, \theta^*}}{ \rbr{A^{\pi^t, \hat\theta}-A^{\pi^t, \theta^*}}^2}
    &\ge \EE_{s^t}^{\pi^t, \theta^*}\rbr{\rbr{\EE_{s^t, b}^{\pi^t, \theta^*} - \EE_{s^t}^{\pi^t, \theta^*}}\sbr{A^{\pi^t, \hat\theta}-A^{\pi^t, \theta^*}}}^2\nend
    &= \EE_{s^t}^{\pi^t, \theta^*}\rbr{\rbr{\EE_{s^t, b}^{\pi^t, \theta^*} - \EE_{s^t}^{\pi^t, \theta^*}}\sbr{\inp[\big]{\phi^t(b)}{\hat\theta-\theta^*}}}^2, 
\end{align*}
where the inequality holds by the Jensen's inequality and the equality follows by 
invoking \Cref{lem:AQV-func diff} with $H=1$ and $\gamma=1$. Combining everything together, we have
\begin{align*}
    &(\hat\theta-\theta^*)^\top\Bigrbr{\underbrace{\frac 1 T\sum_{t=1}^T\EE^{\nu^{\pi^t, \theta^*}}[\psi^{t, \theta^*} {\psi^{t, \theta^*}}^\top]}_{\ds \Sigma_\cD^{\theta^*}}}
     (\hat\theta-\theta^*)\nend
    &\quad \le \bigrbr{\underbrace{\eta^{-1}+B_A}_{\ds C_\eta}}^2 \cdot 
     \rbr{\frac 1 T\log\rbr{\frac{\cN(\Theta, \epsilon)}{\delta}} + \epsilon+\cL_\cD(\hat\theta)-\cL_\cD(\hat\theta_\MLE)}, 
\end{align*}
Note that the Hellinger satisfies the exchangeability property and by swapping $\hat\theta$ with $\theta^*$, we also obtain
\begin{align*}
    &(\hat\theta-\theta^*)^\top\Bigrbr{\underbrace{\frac 1 T\sum_{t=1}^T\EE^{\nu^{\pi^t, \hat\theta}}[\psi^{t, \hat\theta} {\psi^{t, \hat\theta}}^\top]}_{\ds \Sigma_\cD^{\hat\theta}}}
    (\hat\theta-\theta^*)\nend
    &\quad \le \biggrbr{\underbrace{\frac{B_A}{1-\exp\rbr{-\eta B_A}}}_{\ds C_\eta}}^2 \cdot 
    \rbr{\frac 1 T\log\rbr{\frac{\cN(\Theta, \epsilon)}{\delta}} + \epsilon+\cL_\cD(\hat\theta)-\cL_\cD(\hat\theta_\MLE)}, 
\end{align*}
which proves the first statement \eqref{eq:bandit-ub-1}.
Note that for a finite follower's action set $\cB$ with $\abr{\cB}=K$, we actually have $\Sigma_\cD=\eta^{-2}\nabla^2\cL^{(1)}_\cD(\theta^*)= T^{-1}\sum_{t=1}^T \Phi^t \Xi^{t, \theta^*} (\Phi^t)^\top$ where $\Phi^t=[\phi^t(b_1),\dots, \phi^t(b_K)]$ and $\Xi^{t, \theta^*} = \diag(\nu^{\pi^t, \theta^*}) - (\nu^{\pi^t, \theta^*})(\nu^{\pi^t, \theta^*})^\top$.
For the Laplacian $L$ defined as
\begin{align*}
    L=T^{-1}\sum_{t=1}^T \Phi^t(KI-\ind\ind^\top)(\Phi^t)^\top, 
\end{align*}
we directly have for any $z\in\RR^d$ that
\begin{align*}
    z^\top \bar\Sigma_\cD z\ge \min\cbr{\frac{\lambda_d(\varPhi)}{\lambda_1(\bar\Sigma_\cD)}, \frac{K}{\min_{t\in[T]}\lambda_2(\Xi^{t, \theta^*})}} z^\top  L z, 
\end{align*}
where $\lambda_i(\cdot)$ is the $i$-th smallest eigen value and $\varPhi = T^{-1}\sum_{t=1}^T\Phi^t(\Phi^t)^\top$. 
Here, the first coefficient $\lambda_d(\varPhi)/\lambda_1(\bar\Sigma_\cD)$ is direct since $\lambda_d(L)\le \lambda_d(\varPhi)$, and the second coefficient follows by noting that $K\Xi^{t, \theta^*}/\lambda_2(\Xi^{t, \theta^*}) \succeq L$ where both $\Xi^{t, \theta^*}$ and $L$ shares the same null space $\rm{span}\{\ind\}$.
Hence, we complete the proof of \Cref{lem:bandit} by further plugging in $\epsilon=T^{-1}$.

\subsection{Follow up Discussion on \Cref{lem:bandit}}\label{sec:follow up on bandit}
We define $L_\cD$ as the standard Laplacian of the comparison feature graph, 
\begin{align*}
    L_\cD \defeq \frac 1 T \sum_{t=1}^T
    \rbr{\int_{\cB}\phi^t(b)\phi^t(b)^\top\rd b - \int_{\cB}\phi^t(b)\rd b \cdot \int_{\cB}\phi^t(b)^\top \rd b}.
\end{align*}
In particular, if $\cB$ is a finite action space with $K$ actions, we have $L_\cD = T^{-1}\sum_{t=1}^T \Phi^t(KI-\ind\ind^\top)(\Phi^t)^\top$ with $\Phi^t=[\phi^t(b_1),\dots, \phi^t(b_K)]$.
Moreover, for the standard Laplacian $L_\cD$, we have with probability at least $1-\delta$ that
    \begin{align}
        \bignbr{\hat\theta_\MLE-\theta^*}_{L_\cD}^2 \le \underbrace{\min\cbr{\frac{\lambda_d(\tilde\Phi) }{\lambda_1(\Sigma_\cD)}, \frac{K}{\min_{t\in[T]}\lambda_2(\Xi^{t, \theta^*})}}}_{\ds Z_\cD} C_\eta^2 \cdot 
        \frac 1 T\log\rbr{\frac{\cN(\Theta, 1/T)}{\delta}}, \label{eq:bandit-ub-2}
    \end{align}
    where $\Xi^{t, \theta^*} = \diag(\nu^{\pi^t, \theta^*}(\cdot\given s^t)) - (\nu^{\pi^t, \theta^*}(\cdot\given s^t))(\nu^{\pi^t, \theta^*}(\cdot\given s^t))^\top$ and $\tilde\Phi = T^{-1}\sum_{t=1}^T\Phi^t(\Phi^t)^\top$. If $\cB$ is a bounded continuous space, we still have guarantee for $Z_\cD= {\lambda_d(\tilde\Phi) }/{\lambda_1(\Sigma_\cD)}$ with $\tilde\Phi=\int_{\cB}\phi^t(b)\phi^t(b)^\top \rd b$.


To make a fair comparison with results in multiclass logistic regression, we consider $\pi^t$ and $s^t$ to be fixed across samples and the follower only has $K$ choices, which can be thought of as leader using the same strategy and querying the same state all the time. As it turns out, our upper bound \eqref{eq:bandit-ub-2} improves upon the previous results in terms of the choice number $K$ and the coefficient regarding the inverse temperature $\eta$. Note that $\Xi^{t, \theta^*}=\eta^{-2}\nabla^2 \cL^{(1)}_\cD(\theta^*) = \nabla^2(-\log \sigma(\inp[]{\phi^{t}}{\theta^*}))$ with $\sigma(x)=\exp(x_1)/\sum_{k=1}^K\exp(x_k)$. Our construction of $\Xi^{t, \theta^*}$ thus matches the definition of $H$ in \citet{shah2015estimation}, which enables a direct comparison as the following.
\textbf{Firstly}, in terms of $K$, our result only depends linearly in  $K/\lambda_2(\Xi^{t, \theta^*})$ while previous results have a quadratic dependence, e.g., Theorem 4 in \citet{shah2015estimation} or Theorem 4.1 in \citet{zhu2023principled}.
\textbf{Secondly}, in terms of $\eta$, the coefficient in \eqref{eq:bandit-ub-2} matches the $\eta^{-2}$ one given by the lower bound as $\eta$ approaches $0$ \citep[Theorem 1, Note that as $\eta\rightarrow 0$, $1-\exp(-\eta B_A)\rightarrow B_A\eta$  and $\lambda_2(\Xi^{t, \theta^*})\rightarrow K^{-1}$ since $\nu^{\pi^t, \theta^*}$ tends to a uniform distribution]{shah2015estimation}. 
On the other hand, for a large $\eta$, we have $C_\eta\rightarrow 1$ and the dependence on $\eta$ actually comes from $\lambda_2(\Xi^{t, \theta^*})^{-1}$ which may be of order $\exp(\eta B_A)$ in the worst case. However, we remark that $\lambda_2(\Xi^{t, \theta^*})$ only depends on the \textit{true parameter} $\theta^*$, which makes significantly improvements over the $\lambda_2(H)$ used in Theorem 4 of \citet{shah2015estimation} where they require $\nabla^2(-\log\sigma(\inp[]{\phi^t}{\theta}))\succ H$ for all $\theta\in\Theta$.
% for two arm comparison 
\textbf{Thirdly}, we notice that allowing the leader to change her strategy is a blessing rather than a detriment for purpose of learning the follower's reward, which may at first looks conterintuitive. To make the point clear, we consider a fixed $\pi^t=\pi^1$ against a changing $\pi^t$ across samples while we fix the state $s^t=s^1$ and let $\phi^{\alpha^1}(s^1, b)=I$.
If we fix the leader's policy $\pi=\pi^1$, the first term in $Z_\cD$ blows up since $\Sigma_\cD$ has a null space $\spn(\ind)$ by noting that $\Sigma_\cD=\Xi^{t,\theta^*}$. Thus, the second term of $Z_\cD$ dominates and we have $Z_\cD=K\exp(-\eta B_A)$ in the worst case. However, if we allow $\pi^t$ to change across samples, it is much more likely that $\Sigma_\cD$ no longer has this null space $\spn(\ind)$ and if $\Sigma_\cD$ further has full coverage over the $d$-dimensional space, the minimum taken in $Z_\cD$ will ensure that the coefficient does not scale with $\eta$ anymore (or at least no worst than just considering the second term in $Z_\cD$ in the fixed $\pi$ case.) \todo{put this discussion in the appendix and give a brief comparison.}

Lastly, we remark on why we should care about the dependence on the follower's action space size and the inverse temperature. \textbf{Firstly}, although it has been shown that it is difficult for humans to comprehense multiple choices at the same time \citep{saaty2003magic, miller1956magical, kiger1984depth}, it is on the contrary a much simpler task for human to make a selection in a simple continuous choice space \todo{to be added}. \textbf{Secondly}, if we consider an MDP with multi-step choices, the rate $\exp(\eta B_A) T^{-1}$ that appears in \citet{zhu2023principled,shah2015estimation} becomes intolerable for large $H$ and $\eta$ when $B_A\approx H B_r$. 
Thankfully, the analysis in \Cref{lem:bandit} suggests a much tighter upper bound given by \eqref{eq:bandit-ub-1}, which is obtained by studying the localized curvature of $\cL^{(1)}_\cD(\cdot)$ around the true parameter $\theta^*$ instead of looking at the global convexity of $\cL^{(1)}_\cD(\cdot)$.  Given that $\Sigma_\cD$ is the Hessian of the log-likelihood evaluated at $\theta^*$ up to a polynomial factor in $\eta$, \eqref{eq:bandit-ub-1} indicates an information-theoretically flavored upper bound.

%!TEX root =../neurips_main.tex
\ifmain
\section{Offline Learning with Myopic Follower}\label{sec:offline-myopic}
\fi
\ifneurips
\subsection{Offline Learning with Myopic Follower}\label{sec:offline-myopic}
\fi
In this section, we study the problem of offline learning the optimal policy for the leader when the follower is myopic. 
In the offline setting, the offline dataset is collected as $\cD = \{(s_h^i, a_h^i, b_h^i, u_h^i,\pi_h^i)_{h=1}^H\}_{i\in[T]}$.
Here, $\pi^i$ should be thought of as a random variable.
% We consider a data generating process in which the leader's policies across episodes are indepedent.
% , but the prescriptions used  within an episode can be dependent across steps. This can be viewed equivalently as having $T$ pairs of leaders (can be strategic) and followers independently playing this quantal stackelberg game. 
% Under this setting, we suppose that $\tau^i$ has a marginal distribution $\mu$.
We let $\EE_\cD$ denote the expectation with respect to the data generating distribution (also over the randomness of $\pi^i$). 
We study two function approximation schemes, namely the linear function approximation and the general function approximation.

\ifmain
\subsection{Offline Learning for Linear Markov Game}\label{sec:offline-ML}
\fi
\ifneurips
\subsubsection{Offline Learning for Linear MDP}\label{sec:offline-ML}
\fi
% Previously, we study the Offline-MG algorithm for learning the QSE with general function class, which is done by constructing a valid and accurate confidence set for both the environment model and the behavior model.
In this subsection, we  develop a computationally efficient and value iteration-based algorithm for the linear Markov game setting which is defined in \Cref{def:linear MDP}.
{\iffalse
We first give a definition for our linear MDP setting.
\begin{definition}[{Linear MDP}]\label{def:linear MDP}
    For the episodic leader-follower  Markov game, we call it linear if there exists maps $\varpi^*_h:\cS\times\RR^d$ and features  $\phi_h(\cdot,\cdot):\cS\times\cA\times\cB\rightarrow\RR^d$ for any $h\in[H]$ such that
    \begin{align*}
        P_h(s_{h+1}'\given s_h, a_h, b_h) = \inp[]{\phi_h(s_h, a_h, b_h)}{\varpi^*_h(s_{h+1})}_{\RR^d}.
    \end{align*}
    Furthermore, the leader's and the follower's utility admit the following linear factorizations
    \begin{align*}
        u_h(s_h, a_h, b_h) = \inp[]{\phi_h(s_h, a_h, b_h)}{\vartheta^*_h}_{\RR^d}, \quad r_h(s_h, a_h, b_h) = \inp[]{\phi_h(s_h, a_h, b_h)}{\theta^*_h}_{\RR^d},
    \end{align*}
    where $\vartheta^*_h\in\RR^d$ and $\theta^*_h\in\RR^d$ are parameters. 
\end{definition}
\fi}
Recall the guarantee we have for the confidence set based on the negative log-likelihood. A blessing of the myopic follower case is that at each state $s_h$, the follower's quantal response is only a function of the policy $\pi_h$ and the reward $r_h$ with model parameter $\theta_h\in\Theta_h$ at the same step. Therefore, the negative log-likelihood for the follower's behavior at step $h$ is given by
\begin{align}
    \cL_{h}(\theta_h) = - \sum_{i=1}^T \rbr{\eta r_h^{\pi^i, \theta}(s_h^i, b_h^i) - \log \rbr{\sum_{b'\in\cB} \exp\rbr{\eta r_h^{\pi^i, \theta}(s_h^i, b')}}}. \label{eq:myopic-offline-general-MLE loss}
\end{align}
Here, the follower's reward function $r_h^{\pi^i,\theta}$ only depends on $\theta_h$. 
One can 
% write down the negative log-likelihood as \eqref{eq:myopic-offline-general-MLE loss}
% {\color{purple} 
% where $\phi_h^i(b_h)=\inp{\phi_h(s_h^i, b_h^i, \cdot)}{\pi_h^i(\cdot\given b_h^i)}_{\cA_h}$ and $\theta_h\in\RR^d$. }
% and 
construct confidence sets 
$$\CI_{h,\Theta}(\beta)=\cbr{\theta_h\in\Theta_h: \cL_h(\theta_h)\le \inf_{\theta'\in\RR^d}\cL_h(\theta')+\beta}, \quad \forall h\in[H]$$
following the same manner as \eqref{eq:behavior_model_confset}.
Here, we can let each parameter class $\Theta_h$ be a bounded subset of $\RR^d$ with $\nbr{\theta_h}_2\le B_\Theta$.  
In the following, we seperately discuss how to deal with the uncertainty in the environment model and the behavior model by adding penalties in the leader's value functions.

\paragraph{Environment Model Uncertainty Quantification.}
The value interation follows a very similar idea as \citet{zhong2021can, jin2020provably}, but the main difference is that we need to handle the uncertainty in the behavior model parameter $\theta_h$. 
We first give the update of the state-action value functions $\hat U_h(\cdot,\cdot,\cdot)$ at each step.
The idea is to exploit the linear structure $U_h(s_h, a_h, b_h) = u_h(s_h, a_h, b_h) + \TT_h W_{h+1}(s_h, a_h, b_h)=\inp{\phi_h(s_h, a_h, b_h)}{\omega_h}$ with $\omega_h = \vartheta_h^* + \sum_{s_{h+1}}\mu_h(s_{h+1})W_{h+1}(s_{h+1})$,
and solve for the state-action value function $\hat U_h(\cdot,\cdot,\cdot)$ by the following ridge regression,
\#
    & \hat\omega_h  =\argmin_{\omega\in\RR^d} \sum_{i=1}^T \rbr{\phi_h(s_h^i, a_h^i, b_h^i)^\top \omega - u_h^i - \hat W_h(s_{h+1}^i)}^2 + \nbr{\omega}_2^2, \label{eq:linear ridge}\\
    & \hat U_h(s_h, a_h, b_h)  = \phi_h(s_h,a_h,b_h)^\top \hat\omega_h - \Gamma^{(1)}_h(s_h, a_h, b_h),\nonumber
\# 
where $\Gamma^{(1)}_h$ is an uncertainty quantifier \citep{jin2021pessimism,zhong2021can} for the uncertainty in the environment model, and this term is included to ensure pessimism. Here, we can choose $\Gamma^{(1)}_h(s_h, a_h, b_h)= \tilde \cO\big(\sqrt{\phi_h(s_h, a_h, b_h)^\top\Lambda_h^{\dagger}\phi_h(s_h, a_h, b_h)}\big)$ where $$\Lambda_h=\sum_{i=1}^T \phi_h(s_h^i, a_h^i, b_h^i) \phi_h(s_h^i, a_h^i, b_h^i)^\top \allowbreak+ I_d$$ is the kernel obtained from the ridge regression problem \eqref{eq:linear ridge}. One should also be aware that the ridge regression problem \eqref{eq:linear ridge} has a closed form solution $\hat \omega_h\leftarrow \Lambda_h^{-1} (\sum_{i=1}^T \phi_h(s_h^i, a_h^i, b_h^i) (u_h^i + \hat W_{h+1}(s_{h+1}^i))) $. Plugging this closed form solution into \eqref{eq:linear ridge}, we get an update for the leader's U function.

\paragraph{Behavior Model Uncertainty Quantification.}
We next show how to deal with the behavior model uncertainty and find a good policy for the leader.
Recall the confidence set $\CI_{h,\Theta}(\beta)$ we construct for the follower's behavior model. Given the fact that the  follower is myopic, the behavior model at step $h$ is fully characterized by $\theta_h$ and the leader can decides on what policy to use simply by looking at $\hat U_h$ given by \eqref{eq:linear ridge} and $\CI_{h,\Theta}(\beta)$, and take the policy that maximizes the one-step value subject to the \emph{pessimistic} estimation, which gives us the first scheme
\begin{align}
    \textbf{S1:}\quad  \hat\pi_h(s_h)  = \argmax_{\pi_h(s_h) \in\sA} \min_{\theta_h\in\confset_{h, \Theta}(\beta)} \inp[\big]{\hat U_h(s_h, \cdot,\cdot)}{\pi_h \otimes\nu_h^{\pi, \theta}(\cdot,\cdot\given s_h)}_{\cA\times\cB},\label{eq:scheme-1}
\end{align}
where $\hat W_h(s_h)$ is just the optimal value to the maximin problem and we remind the readers that $\sA$ is the prescription space.
% The guarantee for \eqref{eq:scheme-1} follows almost directly from \Cref{thm:Offline-MG}.
% Here, we should choose $\beta\ge T^{-1}\log(H\cN(\Theta_h, T^{-1})/\delta)$, which is given by \Cref{lem:bandit} and by further taking a union bound over $h\in[H]$. 
However, we should note that the problem is highly nonlinear and it is often hard to compute this maximin problem. Note that the inner minimization is included just to ensure pessimism. We should ask ourselves if we can turn the inner minimization problem into a maximization problem by adding penalties for the uncertainty in the behavior model $\theta_h$. By an analysis of the TV distance $D_\TV(\nu_h^{\pi,\theta_h}(\cdot\given s_h), \nu_h^{\pi,\theta_h^*}(\cdot\given s_h))$
% $D_\TV\bigrbr{\nu_h^{\pi_h,\theta_h}, \nu_h^{\pi_h, \theta_h^*}}$ very similar to the one given in \Cref{lem:performance diff informal}
specialized to the linear case,\footnote{See \Cref{lem:response diff-myopic-linear} and \Cref{cor:formal-MLE confset-linear myopic} for more details.} we are able to include a penalty term to ensure pessimism and turn the problem into a maximization for $\forall s_h\in\cS$.
\begin{align}
    \textbf{S2:}\quad  \hat\pi_h(s_h) = \argmax_{\pi_h(s_h) \in\sA, \atop \theta_h\in\confset_{h,\Theta}(\beta)} \inp[\big]{\hat U_h(s_h, \cdot,\cdot)}{\pi_h \otimes\nu_h^{\pi, \theta}(\cdot,\cdot\given s_h)}_{\cA\times\cB} - \Gamma^{(2)}_h(s_h;\pi_h , \theta_h),\label{eq:scheme-2}
\end{align}
where $
\Gamma^{(2)}_h(s_h;\pi_h , \theta_h) = 2 B_U(\eta \xi(s_h;\pi_h, \theta_h)  + C^{(3)} \xi(s_h;\pi_h, \theta_h)^2 )
$ with 
$$C^{(3)}=\rbr{2+\exp\rbr{2\eta B_A}\eta B_A} \eta^2 \exp\rbr{2\eta B_A}/{2}, $$ 
and
\begin{equation}
    \xi(s_h;\pi_h, \theta_h) = \sqrt{\trace\Bigrbr{\bigrbr{T\Sigma_{h,\cD}^{\theta_h} + I_d}^\dagger \Sigma_{s_h}^{\pi_h , \theta_h}}} \cdot \sqrt{8 C_{\eta}^2 \beta + 4B_\Theta^2}.\label{eq:Gamma^2}
\end{equation}
Here, $\Sigma_{h,\cD}^{\theta} = T^{-1} \sum_{i=1}^T \Cov_{s_h^i}^{\pi^i, \theta} \allowbreak [\phi_h^{\pi^i}(s_h^i, b_h)]$ is the data-dependent covariance matrix defined in \eqref{eq:cov matrix},  and $\Sigma_{s_h}^{\pi,\theta}=\Cov_{s_h}^{\pi,\theta} \allowbreak [\phi_h^{\pi_h}(s_h, b_h)]$ is the covariance matrix that actually only depends on $\pi_h(s_h)$ and  parameter $\theta_h$. 
% Moreover, recall that $\nbr{\theta_h}_2 \le B_\Theta$ for all $\theta_h\in\Theta_h$.
Here, we remark that $\Gamma^{(2)}_h$ is a valid uncertainty quantifier which also captures the nonlinear effect of the TV distance $D_\TV(\nu_h^{\pi,\theta_h}(\cdot\given s_h), \nu_h^{\pi,\theta_h^*}(\cdot\given s_h))$ by the second order term and the analysis of $\Gamma^{(2)}_h$ is available in \Cref{lem:response diff-myopic-linear}.
% {\main
% If we look back at \eqref{eq:scheme-2}, another way to think is viewing $\theta_h$ as parts of the leader's \say{policy} but subject to a confidence set constraint.
% \fi}
%  and the target function in the maximization problem is just a pessimistic evaluation of the leader's utility under the \say{joint policy} $(\pi_h, \theta_h)$.

Although Scheme 2 already avoids the maximin optimization in Scheme 1, we are still not satisfied since the  data-dependent covariance matrix $\Sigma_{h,\cD}^{\theta}$ also depends on the optimization variable $\theta_h$, which poses challenges in computation. 
% Also, the confidence set constraint is not easy to satisfy since $\cL_h(\cdot)$ is highly nonlinear. 
One way to deal with the problem is considering a fixed $\theta_h$. As a matter of fact, we are able to replace the $\theta_h$ in \eqref{eq:scheme-2} with the MLE estimator $\hat\theta_{h,\MLE} = \argmin_{\theta_h\in\Theta} \cL_{h}(\theta_h)$, which gives the following scheme,
\begin{align}
    \textbf{S3:}\quad  \hat\pi_h(s_h) = \argmax_{\pi_h(s_h) \in\sA}  \inp[\big]{\hat U_h(s_h, \cdot,\cdot)}{\pi_h \otimes\nu^{\pi, \hat\theta_{h,\MLE}}(\cdot,\cdot\given s_h)}_{\cA\times\cB} - \Gamma^{(2)}_h(s_h;\pi_h , \hat\theta_{h, \MLE}). \label{eq:scheme-3}
\end{align}  
Here, the uncertainty quantifier $\Gamma_h^{(2)}$ is still needed to ensure pessimism.
To bridge the estimation error of $\hat \theta_{h,\MLE}$ in Scheme 3 to the previous two schemes, one can show that both $\Sigma_{h,\cD}^{\hat\theta_{h,\MLE}}$ and $\Sigma_{s_h}^{\pi,\hat\theta_{h,\MLE}}$ are upper and lower bounded by their correspondences with $\theta^*$ plugged in. 
The analysis is available in \Cref{prop:Hessian-ulb}.
% where we define $\Gamma^{(3)}_h(s_h;\pi_h)$ as $
% \Gamma^{(3)}_h(s_h;\pi_h) = 2 B_U(\eta \varrho  + C^{(3)} \varrho^2 )
% $ with $C^{(3)}=\rbr{1+\exp\rbr{2\eta B_A}+\eta B_A \exp\rbr{2\eta B_A}}/{2}$ and, 
% \begin{align*}
%     \varrho = \sqrt{\trace\rbr{\rbr{T\Sigma_{h,\cD}^{\hat\theta_{h,\MLE}} + I_d}^\dagger \Sigma_{s_h}^{\pi_h , \hat\theta_{h,\MLE}}}} \cdot \sqrt{2 C_{\eta}^2 \beta + 4B_\Theta^2}.
% \end{align*}
%
% {\color{purple} commented.
% where $\Psi\in \SSS_+^{d}$ can be any chosen nonnegative definite matrix
% and $\Upsilon_h(\cdot;\cdot,\cdot,\cdot)$ is defined as
% \begin{align}
%     \Upsilon_h(s_h;\pi , \theta_h, \Psi) \defeq \sqrt{\EE_{s_h}^{{\pi ,\theta_h}}\sbr{\phi_h^{\pi }(s_h,b_h)^\top {\Psi}^{\dagger}\phi_h^{\pi }(s_h,b_h)} -\bignbr{\EE_{s_h}^{{\pi ,\theta_h}}\phi_h^{\pi }(s_h,b_h)}_{{\Psi}^{\dagger}}^2}. \label{eq:Upsilon}
% \end{align}
% We remark that the higher order terms in $f$ stem from the nonlinearity of the TV distance between two different responses given by different $\theta_h$.
% Note that $\Upsilon_h(s_h;\pi , \theta_h, \Psi) = \sqrt{\trace(\Psi^\dagger \Sigma_{s_h}^{\pi , \theta_h})}$, where $\Sigma_{s_h}^{\pi ,\theta_h}=\EE_{s_h}^{\pi , \theta_h}\bigsbr{\phi_h^{\pi }{\phi_h^{\pi }}^\top} - \EE_{s_h}^{\pi ,\theta_h}\bigsbr{\phi_h^{\pi }} \cdot \EE_{s_h}^{\pi ,\theta_h}\bigsbr{\phi_h^{\pi }}^\top$.
% The definition of $\Upsilon_h$ is analogue to the previous penalty term $\Gamma^{(1)}_h$, except that $\Upsilon_h$ deals with the uncertainty in the follower's response using the weighted Laplacian $\Sigma_{s_h}^{\pi , \theta_h}$.
% % Thus, it is natural for us to choose $\Psi$
% % Note that the terms with coefficient $\exp(\eta B_A)$ in $f(\cdot)$ are actually of higher order (roughly speaking, $\Gamma^{(2)}_h=\cO(T^{-1/2})$), which suggests a $\cO(\eta\sqrt T)$ rate in our results. 
% Thus, a natural choice for $\hat\theta_h$  would be the MLE estimator $\hat\theta_{h, \MLE}$ as we wish $\hat\theta_h$ to be as close to $\theta_h^*$ as possible and the corresponding choice for $\Psi$ would be the weighted Laplacian $\Psi = \Sigma_{\cD}^{\hat\theta_{h,\MLE}} + T^{-1} I_d$ corresponding to the data \footnote{One can also consider using $\Psi=L_\cD$ which is the standard Laplacian (see \Cref{sec:follow up on bandit} for details). The only difference will be in the concentrability coefficient (as is shown in \eqref{eq:bandit-ub-2}) in the suboptimality. }. 
% See \Cref{sec:follow up on myopic offline} for more details concerning the use of the Laplacians. 
% Thanks to the guarantee we have in \Cref{lem:bandit}, the term $\bignbr{\theta_h^*-\hat\theta_h}_{\Psi}$ could be replaced by its corresponding upperbound $C_\eta^2 T^{-1}\log(H\cN(\Theta_h, T^{-1})/\delta)+T^{-1} B_\Theta$ without knowing $\theta_h^*$ by \Cref{lem:bandit} \footnote{Note that we plug in $\delta/H$ for a union upper bound over $h\in[H]$.}.
%
% The benefit of Scheme 2 is in computation since $\Psi$ and $\hat\theta_{h,\MLE}$ is fixed in this policy optimization step. 
% One issue concerning the second scheme is that we need $\Sigma_\cD^{\hat\theta_{h, \MLE}}$ to have sufficient coverage, which might not be true even if $\Sigma_{\cD}^{\theta_h^*}$ has sufficient coverage since $\hat\theta_{h,\MLE}$ is correlated with the data
% \footnote{In the worst case, the concentrability coefficient can be $\exp(2\eta B_A)$ larger than Scheme 1 due to the distribution shift $\onbr{\nu_h^{\pi , \theta_h^*}/\nu_h^{\pi , \hat\theta_h}}_\infty$.}. 
% To have a better offline guarantee, one thing we can do is to optimize the pessimistic lower bound given in \eqref{eq:scheme-2} over all possible choices of $\theta_h\in\confset_h(\beta)$ 
% instead of randomly picking one, which gives us the third scheme,
%
% \begin{align}
%     \textbf{S3:}\quad  \hat\pi  = \argmax_{\pi \in\sA}  \inp[\big]{\hat U_h(s_h, \cdot,\cdot)}{\pi \otimes\nu^{\pi , \hat\theta_h}(\cdot,\cdot\given s_h)}_{\cA\times\cB} - 2 B_U f\rbr{\Gamma^{(2)}_h(s_h;\pi , \hat\theta_h)}, 
%     % \label{eq:scheme-3}
% \end{align}  
% where the choice for $\Psi=\Sigma_\cD^{\theta_h} + T^{-1} I_d$ remains the same. Here, we should upper bound $\bignbr{\theta_h^*-\hat\theta_h}_{\Psi}$ in $\Gamma^{(2)}$ with $C_\eta^2 T^{-1}\log(H\cN(\Theta_h, T^{-1})/\delta)+T^{-1} B_\Theta^2 + \beta$ by \eqref{eq:bandit-ub-1} in \Cref{lem:bandit}. We remark that Scheme 3 will have the same kind of guarantee as Scheme 1, while no computation of maximin is required. 
% }
Finally, the \textbf{M}aximal \textbf{L}ikelihood \textbf{E}stimation with \textbf{P}essimistic \textbf{V}alue \textbf{I}teration (MLE-PVI) algorithm is summarized as the following.
\begin{algorithm}[H]
    \begin{algorithmic}[1]
    \Require {$\eta, \cD$}
    \State Initialize $\hat W_{H+1}=0$.
    \For{$h=H, H-1,\dots, 1$}
    \State Obtain kernel $\Lambda_h\leftarrow \sum_{i=1}^T \phi_h(s_h^i, a_h^i, b_h^i) \phi_h(s_h^i, a_h^i, b_h^i)^\top + I$. 
    \State Solve the ridge regression for $\hat \omega_h\leftarrow \Lambda_h^{-1} \rbr{\sum_{i=1}^T \phi_h(s_h^i, a_h^i, b_h^i) \bigrbr{u_h^i + \hat W_{h+1}(s_{h+1}^i)}} $.
    \State Update $\hat U_h(\cdot,\cdot,\cdot)\leftarrow \phi_h(\cdot,\cdot,\cdot)^\top \hat \omega_h - \Gamma_h^{(1)}(\cdot,\cdot,\cdot)$ and truncate to $[0, H-h+1]$.
    \State Compute $(\hat W_h(s_h), \hat\pi_h(s_h) )$ as the optimal value and optimal solution to S1 \eqref{eq:scheme-1}, S2 \eqref{eq:scheme-2}, or S3 \eqref{eq:scheme-3} for each $s_h\in\cS$.
    \EndFor
    \Ensure $\hat\pi=(\hat\pi_h)_{h\in[H]}$.
    \end{algorithmic}
    \caption{Offline MLE-PVI Algorithm for Myopic Follower under Linear Markov Game}
    \label{alg:PMLE}
\end{algorithm}
We have the following theoretical guarantee for \Cref{alg:PMLE}.
% \Siyu{satisfies the compliance condition, i.e., $\PP_\cD(u_h^i=u, s_{h+1}^i=s\given \tau^{i-1}, \{s_{h'}^i, \pi_{h'}^i, a_{h'}^i, b_{h'}^i\}_{h'\in[h]}) =\PP(u_h=u, s_{h+1}=s\given s_h=s_h^i, a_h=a_h^i, b_h=b_h^i)$ \citep{jin2021pessimism, zhong2022pessimistic} where $\tau^{i-1}$ is the historical data up to episode $i-1$. We remark that $\xi$ should satisfies $\xi \ge \rbr{2 C_{\eta}^2 \beta + B_\Theta^2}$.}
\begin{theorem}[{Suboptimality for MLE-PVI}]\label{thm:PMLE-VI-myopic}
    Suppose the data compliance condition \begin{align}\label{eq:data compliance}
        &\PP_\cD(u_h^i=u, s_{h+1}^i=s\given \tau^{i-1}, \{s_{h'}^i, \pi_{h'}^i, a_{h'}^i, b_{h'}^i\}_{h'\in[h]}) \nend
        &\quad =\PP(u_h=u, s_{h+1}=s\given s_h=s_h^i, a_h=a_h^i, b_h=b_h^i), \quad \forall h\in[H], i\in[T], 
    \end{align} 
    holds.
    We choose $\Gamma^{(1)}_h(\cdot,\cdot,\cdot) \ge C_1 d H \allowbreak \sqrt{\log(2d H T/\delta)}\cdot \sqrt{\phi_h(\cdot,\cdot,\cdot)^\top \Lambda_h^{-1}\phi_h(\cdot,\cdot,\cdot)}$ for some universal constant $C_1>0$ and $\beta \ge C_2 d\log(H(1+\eta T^2)\delta^{-1})$ for some universal constant $C_2>0$. For the PMLE-VI algorithm under the above three schemes, we have with probability at least $1-2\delta$ that
    \begin{align*}
        \subopt(\hat\pi) \le \sum_{h=1}^H 2\EE^{\pi^*, \nu^{\pi^*}} \sbr{\Gamma^{(1)}_h(s_h, a_h, b_h) + \Gamma^{(2)}_h(s_h;\pi_h^*, \theta_h')},
    \end{align*}
    where $\theta_h' = \theta_h^*$ for Scheme 1 and Scheme 2, and $\theta_h' = \hat\theta_{h, \MLE}$ for Scheme 3.
    % where the second term is given by the following configurations of these schemes,
    % \begin{itemize}
    %     \item[(i)] By using Scheme 1 with confidence set width $\beta\ge T^{-1}\log(H\cN(\Theta_h, T^{-1})/\delta)$, we have
    %     $$ \zeta_h = 2 B_U C_{\eta}
    %     \sqrt{\rbr{\frac 1 T\log\rbr{\frac{H \cN(\Theta, 1/T)}{\delta}} + \beta}}\cdot \EE^{\pi^*}\sbr{\Upsilon_h\rbr{s_h; \pi_h^*, \theta_h^*, \Sigma_\cD^{\theta_h^*}}}$$;
    %     \item[(ii)] By using Scheme 2 with  $\hat\theta_h = \hat\theta_{h,\MLE}$, $\Psi = \Sigma_{\cD}^{\hat\theta_{h,\MLE}} + T^{-1} I_d$, and replacing $\bignbr{\theta_h^*-\hat\theta_h}_{\Psi}$ in \eqref{eq:Gamma^2} by $\iota_h \ge \sqrt{C_\eta^2 T^{-1}\log(H\cN(\Theta_h, T^{-1})/\delta) + T^{-1}B_\Theta^2}$, we have
    %     $$ \zeta_h = 4 B_U \iota_h \EE^{\pi^*}\sbr{\Upsilon_h\rbr{s_h; \pi_h^*, \hat\theta_{h, \MLE}, \Sigma_\cD^{\hat\theta_{h,\MLE}}}} ;$$
    %     \item[(iii)] By using Scheme 3 with $\Psi = \Sigma_{\cD}^{\theta_{h}} + T^{-1} I_d$, $\beta\ge T^{-1}\log\rbr{H\cN(\Theta_h, T^{-1})/\delta}$ and replacing $\bignbr{\theta_h^*-\hat\theta_h}_{\Psi}$ in \eqref{eq:Gamma^2} by $ \iota_h \ge \sqrt{C_\eta^2 (T^{-1}\log(H\cN(\Theta_h, T^{-1})/\delta)+\beta + T^{-1}B_\Theta^2)}$, we have 
    %     $$ \zeta_h = 2 B_U \iota_h \EE^{\pi^*}\sbr{\Upsilon_h\rbr{s_h; \pi_h^*, \theta_h^*, \Sigma_\cD^{\theta_h^*}}}. $$
    % \end{itemize}
    % Here, $\Upsilon_h(\cdot;\cdot,\cdot,\cdot)$ is defined in \eqref{eq:Upsilon}
    \begin{proof}
        See \Cref{sec:proof-PMLE-VI-myopic} for a detailed proof.
    \end{proof}
\end{theorem}
We give the following corollary that characterizes the distribution shift issue. 
\begin{corollary}[Distribution shift]\label{rmk:MLE-PVI-dist-shift}
    Suppose for the leader's side, we have with high probability that 
    $
    % \begin{align}
    \Lambda_h\succeq I + c_1 T \EE^{\pi^*,\nu^{\pi^*}}[\phi_h\phi_h^\top]
    % \label{eq:MLE-PVI-coverage-1}
    % \end{align}
    $ for some constant $c_1>0$, 
    and the for the follower's side, we have with high probability 
    \begin{align}
    &I + {\ts\sum_{t=1}^T }\EE^{\pi^i, \nu^{\pi^i}} \bigsbr{(\Upsilon_h^{\pi^i} \phi_h)  (\Upsilon_h^{\pi^i} \phi_h)^\top \given s_h^i}  \succeq I + c_2 T\EE^{\pi^*, \nu^{\pi^*}} \bigsbr{(\Upsilon_h^{\pi^*} \phi_h) (\Upsilon_h^{\pi^*} \phi_h)^\top} , 
    \label{eq:MLE-PVI-coverage}
    \end{align}
    for some constant $c_2>0$,
    where $\Upsilon_h^{\pi}\phi $ is a short hand of $(\Upsilon_h^{\pi}\phi)(s_h, b_h)$. We then have for Scheme 1 and Scheme 2, 
    \begin{align*}
        {\subopt(\hat\pi) 
        \lesssim \frac{d^{3/2}H^2} {\sqrt{c_1 T}} +  \eta C_\eta H^{2}d \cdot \sqrt{\frac{1}{c_2 T}} +  e^{4\eta B_A} (\eta C_\eta)^3 H^2  d^2 \cdot  \frac{1}{c_2 T}, }
    \end{align*}
    and for Scheme 3, 
    \begin{align*}
    { \subopt(\hat\pi) 
    \lesssim \frac{d^{3/2}H^2} {\sqrt{c_1 T}} + e^{2\eta B_A} \eta C_\eta H^{2}d \cdot \sqrt{\frac{1}{c_2 T}} +  e^{8\eta B_A} (\eta C_\eta)^3 H^2  d^2 \cdot  \frac{1}{c_2 T}.}
    \end{align*}
\end{corollary}
\begin{proof}
    See \Cref{sec:proof-MLE-PVI-dist-shift} for a detailed proof.
\end{proof}
We note that \eqref{eq:MLE-PVI-coverage} is similar to the standard sufficient coverage condition in linear MDP but 
customized for linear QRE, where the operator $\Upsilon_h^{\pi^*}$  defined in \eqref{eq:Upsilon} plays a key role in the distribution shift. In particular, \eqref{eq:MLE-PVI-coverage} not only requires coverage over the trajectory induced by $\pi^*$, but also requires richness in the leader's prescription $\pi^i(s_h)$ at those states visited under $\pi^*$.
To understand this point, we note that if the leader announces the same policy $\pi^0$ for all the time, the follower always acts according to the same reward $r_h^{\pi^0} (s_h, b_h)= \la r_h(s_h,\cdot,b_h), \pi^0_h(\cdot\given s_h, b_h)\ra_\cB$, which is only a linear subspace of the reward function and the leader cannot anticipate the follower's quantal response for other policies. 

We next understand the first order terms, i.e., $\cO(T^{-1/2})$ terms in the suboptimality. The first term characterizes the leader's Bellman error, which is standard for RL problems. 
The second term characterizes the follower's first-order quantal response error (QRE). The first-order QRE term suffers from an $\exp(2\eta B_A)$ coefficient only in Scheme 3. We remark that this is because we fix the follower's quantal response using the MLE estimator in \eqref{eq:scheme-3} while Scheme 1 and Scheme 2 allow us to pick a more refined estimator $\hat\theta_h$ in the confidence set at the cost of heavier computation.
% \Cref{thm:PMLE-VI-myopic-neurips}
% We next show an ensurance result which says that the condition \eqref{eq:MLE-PVI-coverage} is not any stronger than \eqref{eq:MLE-PVI-coverage-1}. 
% % $\Gamma^{(2)}_h(s_h;\pi,\hat\theta_{h,\MLE})\le \exp\rbr{4\eta B_A} \Gamma^{(2)}_h(s_h;\pi, \theta_h^*)$. 
% We let $\L=\diag(u)-u u^\top$. We can rewrite $\Sigma_{h,\cD}^u$ and $\Sigma_{s_h}^{\pi^*, u}$ as 
% \begin{align*}
%     \Sigma_{h,\cD}^u = T^{-1}\sum_{i=1}^T \phi_h^{\pi^i}(s_h^i, \cdot) \L \phi_h^{\pi^i}(s_h^i, \cdot)^\top,\qquad  \Sigma_{s_h}^{\pi^*, u} = \phi_h^{\pi^*}(s_h, \cdot)\L \phi_h^{\pi^*}(s_h, \cdot)^\top.
% \end{align*} 
% We define $\H^{\pi,\theta_h} = \diag(\nu_h^{\pi,\theta_h}(\cdot\given s_h)) - \nu_h^{\pi,\theta_h}(\cdot\given s_h)\nu_h^{\pi,\theta_h}(\cdot\given s_h)^\top$ 
% and rewrite $\Sigma_{h,\cD}^{\theta_h^*}$ and $\Sigma_{s_h}^{\pi^*, \theta_h^*}$ as 
% \begin{align*}
%     \Sigma_{h,\cD}^{\theta_h^*} = T^{-1}\sum_{i=1}^T \phi_h^{\pi^i}(s_h^i, \cdot) \H^{\pi^i,\theta_h^*} \phi_h^{\pi^i}(s_h^i, \cdot)^\top,\qquad  \Sigma_{s_h}^{\pi^*, u} = \phi_h^{\pi^*}(s_h, \cdot) \H^{\pi^*, \theta_h^*} \phi_h^{\pi^*}(s_h, \cdot)^\top,
% \end{align*}
% We next see what guarantee we have for the the distribution shift induced by $\Gamma^{(2)}_h$.
% Suppose that 
% \begin{align*}
%     T \Sigma_{h,\cD}^{\theta_h^*}+I = \sum_{i=1}^T \Sigma_{s_h^i}^{\pi^i, \theta_h^*} + I \succeq c T \EE_\cD \sbr{\Sigma_{s_h^i}^{\pi^i, \theta_h^*}} + I, 
% \end{align*}
% with high probability, 
% where the expectation is taken for both $\pi^i$ and $s_h^i$.


% Since $\Sigma_{h,\cD}^u$ now only depends on $\phi_h^{\pi^i}$ and $s_h^i$, 
% and $\Sigma_{s_h}^{\pi^*,u}$



% \todo{to be discussed, results implication, and dependent data.}

% \Zhuoran{Corrollary}
\ifneurips
\subsubsection{Offline Learning with General Function Class}\label{sec:Offline-MG}
\fi
\ifmain
\subsection{Offline Learning with General Function Class}\label{sec:Offline-MG}
\fi
In this subsection, we carry out the offline learning scheme with general function approximation.
For the leader's side, we propose to learn the environment model by minimizing the squared loss of the Bellman error over the U function for each policy $\pi$. 
For consistency, we still assume that the follower's reward function at step $h\in[H]$ lies in some general function class parameterized by $\theta_h\in\Theta_h$. We let $\Theta=\{\Theta_h\}_{h\in[H]}$.
Suppose $\cU:\cS\times\cA\times\cB\rightarrow\RR$ is a given function class for the leader's state-action value function. 
Following the idea of Bellman-consistent pessimism \citep{xie2021bellman}, we define a loss function for the environment model error as
\begin{align}
    &\ell_h(U_h', U_{h+1}, \theta_{h+1}, \pi) \nend
    &\quad = \sum_{i=1}^T \rbr{U_h'(s_h^i, a_h^i, b_h^i) - u_h^i -  \inp[\big]{U_{h+1}(s_{h+1}^i, \cdot, \cdot)}{\pi_{h+1}\otimes \nu_{h+1}^{\pi, \theta}(\cdot, \cdot\given s_{h+1}^i)}}^2. \label{eq:myopic-offline-general-Bellman loss}
\end{align}
Intuitively, the loss $\ell_h$ going to zero means no Bellman error for the value functions between step $h$ and step $h+1$ under $\pi$ and $\nu^{\pi,\theta}$.
Note that the unknown parameters are $\theta=\{\theta_h\}_{h\in[H]}\in\Theta$ and $\{U_h\}_{h\in[H]}\in\cU^{\otimes H}$.
% where we take $\cU$ as a the function class for the leader's U function.
Based on the loss functions defined in \eqref{eq:myopic-offline-general-MLE loss} and \eqref{eq:myopic-offline-general-Bellman loss}, we can construct a confidence set for each leader's policy $\pi$ as
\begin{align}
    &\CI_{\cU, \Theta}^\pi(\beta) \nend
    &\quad= \cbr{
    (U,\theta)\in\cU^{\otimes H}\times\Theta:
    \rbr{ \ds
        \cL_h(\theta_h)-\inf_{\theta'\in\Theta_h}\cL_h(\theta') \le \beta 
    \atop \ds
        \ell_h(U_h, U_{h+1}, \theta_{h+1}, \pi) - \inf_{U'\in\cU_h} \ell_h(U', U_{h+1}, \theta_{h+1}, \pi)\le H^2\beta}, 
    \forall h\in[H]}. \label{eq:myopic-offline-general-confset}
\end{align}
The first condition in \eqref{eq:myopic-offline-general-confset} characterizes a valid and accurate confidence set for the follower's behavior model as we have done in \eqref{eq:bandit-ub-1}. For the second condition in \eqref{eq:myopic-offline-general-confset}, if certain realizability and completeness conditions are satisfied, we have guarantee on small leader's Bellman errors \citep{xie2021bellman, lyu2022pessimism}, which characterizes the uncertainty in the environment model.
Combining these two guarantees, we can therefore expect $\CI_{\cU, \Theta}^\pi (\beta)$ to be a valid and accurate confidence set for both the environment and the behavior models.
Following the principle of pessimism, we can output the policy $\hat\pi$ as, 
\begin{align}
    \hat\pi=\argmax_{\pi\in\Pi} \min_{(U, \theta)\in\CI_{\cU,\Theta}^\pi(\beta)} \EE_{s_1\sim\rho_0} \sbr{\inp[\big]{U_1(s_1, \cdot, \cdot)}{\pi_1\otimes\nu_1^{\pi, \theta}(\cdot,\cdot\given s_1)}_{\cA \times \cB}}.\label{eq:offline-MG-pi^hat}
\end{align}
% Then, we just greadily take $\hat\pi(s_h) = \argmax_{\pi_h\in\sA} \inp[\big]{\hat U_h(s_h, \cdot, \cdot)}{\pi_h\otimes\nu_h^{\pi_h, \hat\theta_h}(\cdot,\cdot\given s_h)}$.
To present our results, we first define an optimistic Bellman operator $\TT_h^{*, \theta}:\cF(\cS\times\cA\times\cB)\rightarrow \cF(\cS\times\cA\times\cB)$ for the leader as
\begin{align}\label{eq:define optimistic Bellman opt}
    &\bigrbr{\TT_h^{*,\theta} f} (s_h, a_h, b_h) \nend
    &\quad = u_h(s_h, a_h, b_h) +  \EE\sbr{\max_{\pi_{h+1}(s_{h+1})\in\sA}\bigdotp{f(s_{h+1}, \cdot, \cdot)}{\pi_{h+1}\otimes\nu_{h+1}^{\pi,\theta}(\cdot,\cdot\given s_{h+1})}_{\cA\times\cB}\Biggiven s_h, a_h, b_h}.
\end{align}
Here, the expectation is taken with respect to $s_{h+1}\sim P_h(\cdot\given s_h, a_h, b_h)$. 
We now summarize our offline policy learning with \textbf{M}aximum \textbf{L}ikelihood with \textbf{B}ellman \textbf{C}onsistent \textbf{P}essimism (MLE-BCP) algorithm together with its theoretical guarantee.
\begin{algorithm}[H]
    \begin{algorithmic}[1]
    \Require {$\eta, \cD$}
    \State Construct confidence set $C_{\cU,\Theta}^\pi(\beta)$ by \eqref{eq:myopic-offline-general-confset}. 
    \State Solve for the policy $\hat\pi$ with Bellman consistent pessimism in \eqref{eq:offline-MG-pi^hat}. 
    \Ensure $\hat\pi=(\hat\pi_h)_{h\in[H]}$.
    \end{algorithmic}
    \caption{Offline MLE-BCP for Myopic Follower under General Function Approximation}
    \label{alg:MLE-BCP}
\end{algorithm}
\begin{theorem}[{Suboptimality for MLE-BCP}]\label{thm:Offline-MG}
    Suppose that each trajectory in the offline dataset is independently collected.
    Suppose that the following conditions hold for model class $\Theta$ and function class $\cU$:
\begin{itemize}
    \item[(i)] (\textit{Realizability}) There exists $\theta^*\in\Theta$ such that $r_h^{\theta^*}=r_h$ for any $h\in[H]$. For any $\pi\in\Pi, \theta\in\Theta$, there exists $U\in\cU$ such that $U_h = \TT_h^{\pi,\theta} U_{h+1}$ for any $h\in[H]$;
    \item[(ii)] (\textit{Completeness}) For any $U\in\cU$ and $\pi\in\Pi, \theta\in\Theta$, there exists $U'\in\cU$ such that $U'=\TT_h^{\pi,\theta} U_{h+1}$ for any $h\in[H]$. 
\end{itemize}
    By choosing $\beta = c\cdot \max\cbr{\log(H \cN_\rho(\cY, T^{-1})\delta^{-1}), \log(H\cN_\rho(\Theta, T^{-1})/\delta)}$ for some universal constant $c>0$, where the covering number for $\Theta$ and $\cY$ are defined in \eqref{eq:cN-Theta-myopic} and \eqref{eq:cN-cY}, respectively, 
    we have for the offline algorithm \eqref{eq:offline-MG-pi^hat} that 
    \begin{align*}
        \subopt(\hat\pi) 
        &\lesssim \max_{U\in\cU,\theta\in\Theta, h\in[H]}
            \sqrt{\frac{{{\bignbr{{ U_h  -  \TT_h^{\pi^*,\theta}  U_{h+1}} }_{2, d^{\pi^*}}^2}}}{{{\bignbr{ U_{h} - \TT_{h}^{\pi^*,\theta}  U_{h + 1}}_{2,\cD}^2}}}} 
        \cdot H^2\sqrt{\beta T^{-1}}  \nend
        &\qquad + \max_{\theta\in\Theta, h\in[H]}\sqrt{ \frac{{\bignbr{\Upsilon_h^{\pi^*} (r_h^\theta - r_h^{\theta^*})}_{2, d^{\pi^*}}^2}}{\bignbr{{\Upsilon_h^{\pi^i} (r_h^\theta - r_h^{\theta^*})}}_{2,\cD}^2}} \cdot  H^2 \eta C_\eta \sqrt{\beta T^{-1}}\nend
        &\qquad  +   \max_{\theta\in\Theta, h\in[H]} {
            \frac{{\bignbr{\Upsilon_h^{\pi^*} (r_h^\theta - r_h^{\theta^*})}_{2, d^{\pi^*}}^2}}{\bignbr{{\Upsilon_h^{\pi^i} (r_h^\theta - r_h^{\theta^*})}}_{2,\cD}^2}
        }\cdot  H^2 \exp(4\eta B_A) (\eta C_\eta)^3 \beta T^{-1},
    \end{align*}
    where $C_\eta = \eta^{-1}+B_A$ and $\lesssim$ only hides universal constants.
    \begin{proof}
        See \Cref{sec:proof-offline-MG} for a detailed proof.
    \end{proof}
\end{theorem}
\Cref{thm:Offline-MG} establishes the suboptimality for offline learning the optimal policy using general function approximation.
Similar to the linear case, the first two terms characterizes the leader's Bellman error and the follower's first order quantal response error, respectively. 
The only exponential term appears in the follower's second order quantal response error term, which is roughly of order $\cO(T^{-1})$.
In particular, the concentrability coefficients that address the distribution shift issue are characterized by the ratio in both the Bellman error and the QRE error.


%! TEXroot =main.tex
\ifmain
\section{Online Learning with Myopic Follower} \label{sec:myopic-online}
\fi 
\ifneurips
\subsection{Online Learning with Myopic Follower} \label{sec:myopic-online}
\fi
In the previous section, we address the problem of offline learning the QSE with myopic follower under both general function class and linear MDP setting. In this section, we move on to the online scenario with myopic follower. Specifically, the game proceeds as the following. At state $s_h$ in episode $t$, the leader announces her prescription $\alpha_h^t:\cB\rightarrow\Delta(\cA)$,  and the myopic follower picks an action $b_h^t$. The leader then pick an action $a_h^t\sim \alpha_h^t(\cdot\given b_h^t)$ and the state then transits to $s_{h+1}^t$. The game at episode $t$ ends at the $H$-th step and a new episode begins next. We also study the online problem both under the general function class and the linear MDP setting.
% \todo{Define operator $\varPsi_h^\pi:\cF(\cS\times\cA\times\cB)\rightarrow \cF(\cS\times\cB)$ as 
% \begin{align*}
%     \rbr{\varPsi_h^\pi f}(s_h, b_h) = \dotp{\pi_h(\cdot\given s_h, b_h)}{f(s_h, \cdot, b_h)} - \dotp{\pi_h\otimes \nu_h^{\pi}(\cdot,\cdot\given s_h)}{f(s_h,\cdot,\cdot)}. 
% \end{align*}
% }

\ifmain
\subsection{Online Learning for Linear Markov Game}
\label{sec:online-myopic-linear}
\fi
\ifneurips
\subsubsection{Online Learning for Linear MDP}
\label{sec:online-myopic-linear}
\fi
The case for online learning with linear MDP is not so different from the offline one, except for the fact that we have to incorporate optimism for exploration. This is nothing much but just flipping the sign of the penalties and turn them into bonuses. Following the same spirit of \Cref{sec:offline-ML}, we simply present our algorithm here for the sake of completeness. 
For updating the U function, we add a bonus to the ridge regression result,
\begin{gather}
    \hat \omega_h^t\leftarrow (\Lambda_h^t)^{-1} \rbr{\sum_{i=1}^{t-1} \phi_h(s_h^i, a_h^i, b_h^i) \bigrbr{u_h^i + \hat W_{h+1}(s_{h+1}^i)}}, \label{eq:linear ridge}\\
    \hat U_h^t(s_h, a_h, b_h) = \phi_h(s_h,a_h,b_h)^\top \hat\omega_h^t + \Gamma^{(1,t)}_h(s_h, a_h, b_h),\nonumber
\end{gather}
where we choose $\Gamma^{(1,t)}_h(s_h, a_h, b_h)= \cO(\sqrt{\phi_h(s_h, a_h, b_h)^\top\Lambda_h^{\dagger}\phi_h(s_h, a_h, b_h)})$ where $\Lambda_h=\sum_{i=1}^T  \phi_h(s_h^i, a_h^i, b_h^i)\allowbreak \phi_h(s_h^i, a_h^i, b_h^i)^\top \allowbreak+ I_d$. For updating the W function, 
we still define the negative loglikelihood as the one given in \eqref{eq:myopic-offline-general-MLE loss},
\begin{align}
    \cL_{h}^t(\theta_h) = - \sum_{i=1}^{t-1} \rbr{\eta r_h^{\pi^i, \theta}(s_h^i, b_h^i) - \log \rbr{\sum_{b'\in\cB} \exp\rbr{\eta r_h^{\pi^i, \theta}(s_h^i, b')}}}. \label{eq:online-MG-MLE loss}
\end{align}
We obtain an optimistic policy via the following two schemes,
\begin{align}
    \textbf{S4:}\quad  \hat\pi_h^t(s_h)  &= \argmax_{\pi_h(s_h) \in\sA
    \atop \theta_h\in\confset_{h, \Theta}^t(\beta)} \inp[\big]{\hat U_h^t(s_h, \cdot,\cdot)}{\pi_h \otimes\nu_h^{\pi , \theta}(\cdot,\cdot\given s_h)}_{\cA\times \cB}, \quad \forall s_h\in\cS_h,\label{eq:scheme-4}\\
    \textbf{S5:}\quad  \hat\pi_h^t(s_h)  &= \argmax_{\pi_h(s_h) \in\sA}  \inp[\big]{\hat U_h^t(s_h, \cdot,\cdot)}{\pi_h \otimes\nu^{\pi , \hat\theta_{h,\MLE}^t}(\cdot,\cdot\given s_h)}_{\cA\times \cB} + \Gamma^{(2,t)}_h(s_h;\pi_h , \hat\theta_{h, \MLE}^t). \label{eq:scheme-5}
\end{align}
which follow from \eqref{eq:scheme-1} and \eqref{eq:scheme-3}, respectively. Here, we denote by $\CI_{h,\Theta}^t(\beta) = \{\theta_h\in\Theta_h: \cL_h^t(\theta)\le \min_{\theta_h'}\cL_h^t(\theta_h')+\beta\}$ the confidence set at step $t$ with $\cL_h^t(\theta)$ defined in \eqref{eq:online-MG-MLE loss}. Moreover, $\hat\theta_{h,\MLE}^t$ is the MLE estimator that minimizes the negative log-likelihood $\cL_h^t(\cdot)$, and $\Gamma_h^{(2,t)}(s_h;\pi_h,\theta_h)=2 H(\eta \xi  + C^{(3)} \xi^2 )
$ with $C^{(3)}=\eta^2 \exp(2\eta B_A) (2+\eta B_A \exp(2\eta B_A))/2$ and, 
\begin{equation}
    \xi = \sqrt{\trace\rbr{\bigrbr{\Sigma_{h, t}^{\theta_h} + I_d}^\dagger \Sigma_{s_h}^{\pi_h , \theta_h}}} \cdot \sqrt{8 C_{\eta}^2 \beta + 4B_\Theta^2}.\label{eq:Gamma^2-online}
\end{equation}
where $\Sigma_{h,t}^{\theta_h} = \sum_{i=1}^{t-1} \Cov_{s_h^i}^{\pi_h^i, \theta_h} \allowbreak [\phi_h^{\pi_h^i}(s_h^i, b_h)]$ is the covariate matrix and $\Sigma_{s_h}^{\pi,\theta}=\Cov_{s_h}^{\pi,\theta} \allowbreak [\phi_h^{\pi_h}(s_h, b_h)]$. We summary the \textbf{M}aximal \textbf{L}ikelihood \textbf{E}stimation with \textbf{O}ptimistic \textbf{V}alue \textbf{I}teration (MLE-OVI) algorithm as the following.
\begin{algorithm}[H]
    \begin{algorithmic}[1]
    \Require {$\eta, \cD$}
    \State Initialize $\cD=\emptyset$.
    \For{$t=1,\dots, T$}
    \State Initialize $\hat W_{H+1}^t=0$
    \For{$h=H, H-1,\dots, 1$}.
    \State Obtain kernel $\Lambda_h^t\leftarrow \sum_{i=1}^{t-1} \phi_h(s_h^i, a_h^i, b_h^i) \phi_h(s_h^i, a_h^i, b_h^i)^\top + I$. 
    \State Solve the ridge regression for $\hat \omega_h^t\leftarrow \rbr{\Lambda_h^t}^{-1} \rbr{\sum_{i=1}^{t-1} \phi_h(s_h^i, a_h^i, b_h^i) \bigrbr{u_h^i + \hat W_{h+1}^t (s_{h+1}^i)}} $.
    \State Update $\hat U_h^t(\cdot,\cdot,\cdot)\leftarrow \phi_h(\cdot,\cdot,\cdot)^\top \hat \omega_h^t + \Gamma_h^{(1, t)}(\cdot,\cdot,\cdot)$ and truncate to $[0, H-h+1]$.
    \State Compute $\hat W_h^t(s_h)$ and $\hat\pi_h^t(s_h)$ as the optimal value and optimal solution to S4 \eqref{eq:scheme-4} or S5 \eqref{eq:scheme-5}.
    \EndFor
    \State Announce $\hat\pi^t$ and collect a trajectory $\tau^t = \{(s_h^t, a_h^t, b_h^t, \hat\pi_h^t)\}_{h\in[H]}$. 
    \State $\cD\leftarrow \cD\cup \{\tau^t\}$. 
    \EndFor
    % \Ensure $\hat\pi=(\hat\pi )_{h\in[H]}$.
    \end{algorithmic}
    \caption{Online MLE-OVI for Myopic Follower under Linear Markov Game}
    \label{alg:MLE-OVI}
\end{algorithm}

We also provide theoretical guarantee for the MLE-OVI algorithm.
\begin{theorem}[{Regret for MLE-OVI}]\label{thm:Online-ML}
    We choose 
    $$\Gamma^{(1, t)}_h(\cdot,\cdot,\cdot) = C_1 d H \allowbreak \sqrt{\log(2d H T^2/\delta)}\cdot \sqrt{\phi_h(\cdot,\cdot,\cdot)^\top (\Lambda_h^t)^{-1}\phi_h(\cdot,\cdot,\cdot)}$$ for some universal constant $C_1>0$ and $\beta = C_2 d\log(HT(1+\eta T^2)\delta^{-1})$ for some universal constant $C_2>0$. 
    For Scheme 4, 
\begin{align*}
    \Reg(T)\lesssim d H^2 \sqrt{d T} + \eta C_\eta H^2 d \sqrt{T} + \exp(4\eta B_A) (\eta C_\eta)^3 H^2 d^2 \log T, 
\end{align*}
and for Scheme 5, 
\begin{align*}
    \Reg(T)\lesssim d H^2 \sqrt{d T} + \exp(4\eta B_A)\eta C_\eta H^2 d \sqrt{T} + \exp(8\eta B_A) (\eta C_\eta)^3 H^2 d^2 \log T. 
\end{align*} 
    \begin{proof}
        See \Cref{sec:proof-Online-ML} for a detailed proof.
    \end{proof}
\end{theorem}

We observe from \Cref{thm:Online-ML} that the the optimistic value iteration methods proposed by both Scheme 4 and Scheme 5 achieve sublinear online regret. 
Scheme 5 suffers from an additional exponential term in the second term, which corresponds to the first order quantal response error. 
What happens here resembles the offline suboptimality bound for Scheme 3 in \Cref{rmk:MLE-PVI-dist-shift} and the reason is quite similar given that we directly use the MLE estimator $\hat\theta^t_{h, \MLE}$ for Scheme 5 instead of exploiting the confidence set for $\theta$.




\ifmain
\subsection{Online Learning with General Function Class}\label{sec:online-myopic-general}
\fi 
\ifneurips
\subsubsection{Online Learning with General Function Class}\label{sec:online-myopic-general}
\fi 
We develop an online learning algorithm very similar to the one given in \Cref{sec:Offline-MG} for the offline general function class, despite that we use optimism in leader's policy learning.
However, for the leader's Bellman loss which is defined by \eqref{eq:myopic-offline-general-Bellman loss} in the offline case, we use a slightly different version that directly incorporates the rule of optimism, 
\begin{align}
    &\!\!\! \ell_h^t(U_h', U_{h+1}, \theta_{h+1}) \nend
    &\!\!\!\quad = \sum_{i=1}^{t-1}  \rbr{U_h'(s_h^i, a_h^i, b_h^i) \!-\! u_h^i - \!\!\!\! \max_{\pi_{h+1}(s_{h+1})\in\sA}\inp[\big]{U_{h+1}(s_{h+1}^i, \cdot, \cdot)}{\pi_{h+1}\otimes \nu_{h+1}^{\pi, \theta}(\cdot, \cdot\given s_{h+1}^i)}_{\cA\times\cB}}^2.\label{eq:online-MG-bellman loss}
\end{align}
Here, there is a slight abuse of notation and we distinguish this loss from its correspondence in the offline case by the superscript $t$. 
We would like to remark that this loss function assembles the one used in GOLF of \citet{jin2021bellman}.
We build the confidence set for both the environment model and the behavior model as
\begin{align}
    &\CI_{\cU, \Theta}^t(\beta) \nend
    &\quad= \cbr{
    (U,\theta)\in\cU^{\otimes H}\times\Theta:
    \rbr{ \ds
        \cL_h^t(\theta_h)-\inf_{\theta_h'\in\Theta_h}\cL_h^t(\theta_h') \le \beta 
    \atop \ds
        \ell_h^t(U_h, U_{h+1}, \theta_{h+1}) - \inf_{U'\in\cU_h} \ell_h^t(U', U_{h+1}, \theta_{h+1})\le H^2\beta}, 
    \forall h\in[H]}. \label{eq:myopic-online-general-confset}
\end{align}
Following the principle of optimism, we can output a pair of optimistic model parameter, 
\begin{align}
    (\hat U^t, \hat\theta^t)=\argmax_{\pi_1\in\Pi_1 \atop (U, \theta)\in\CI_{\cU,\Theta}^t(\beta)} \EE_{s_1\sim\rho_0} \sbr{\inp[\big]{U_1(s_1, \cdot, \cdot)}{\pi_1\otimes\nu_1^{\pi, \theta}(\cdot,\cdot\given s_1)}_{\cA\times\cB}}.\label{eq:online-MG-opt-parameter}
\end{align}
Specifically, for any given state $s_h$, an optimistic policy is then given greedily by 
\begin{align}
    \hat\pi^t(s_h)=\argmax_{\pi_{h}(s_h)\in\sA}\inp[\big]{U_{h}(s_{h}, \cdot, \cdot)}{\pi_{h}\otimes \nu_{h}^{\pi, \theta}(\cdot, \cdot\given s_{h})}_{\cA\times\cB}. \label{eq:online-MG-hat pi}
\end{align}
We summarize the above \textbf{M}aximum \textbf{L}ikelihood \textbf{E}stimation with \textbf{G}lobal \textbf{O}ptimism based on \textbf{L}ocal \textbf{F}itting (MLE-GOLF) algorithm as the following.
\begin{algorithm}[H]
    \begin{algorithmic}[1]
    \Require {$\eta$}
    \State Initiate $\cD =\emptyset$.
    \For{$t=1,\dots,T$}
    \State Construct confidence set $\CI_{\cU, \Theta}^t(\beta)$ by \eqref{eq:myopic-online-general-confset}. 
    \State Solve for $\hat U^t, \hat\theta^t$ by \eqref{eq:online-MG-opt-parameter}.
    \State Solve for the greedy policy $\hat\pi^t$ by \eqref{eq:online-MG-hat pi}. 
    \State Deploy $\hat\pi^t$ and collect a trajectory $\tau^t = \{(s_h^t, a_h^t, b_h^t, \hat\pi_h^t)\}_{h\in[H]}$. 
    \State $\cD\leftarrow \cD\cup \{\tau^t\}$. 
    \EndFor
    \end{algorithmic}
    \caption{Online MLE-GOLF for Myopic Follower under General Function Approximation}
    \label{alg:MLE-GOLF}
\end{algorithm}
% \todo{
% \begin{itemize}
%     \item function class $\cG_L = \cbr{U_h-\TT_h^{\theta} U_{h+1}: U\in\cU, \theta\in\Theta}$ defined on $\cX=\cS\times\cA\times\cB$, which is bounded by $B_{\cG_L}\le 2B_U$; the class of probability measures $\sP_L = \cbr{\rho\in\Delta(\cS\times\cA\times\cB): \rho(\cdot)=\PP^\pi((s_h, a_h, b_h)=\cdot)}$ with $\rho^i(\cdot) = \PP^{\hat\pi^i}((s_h, a_h, b_h)=\cdot)$; Under these definitions, we let $d_{\cG_L} = \dim_\DE\rbr{\cG_L, \sP_L, T^{-1/2}}$ be the distributional Eluder dimension for $\cG_L$.
%     \item function class defined on $\cX=\cS\times\cB\times\sA$,
%     \begin{align*}
%     \cG_F = \cbr{g:\cX\rightarrow \RR: \exists \theta\in\Theta, g(s_h, b_h, \pi)= \rbr{\varPsi_h^{\pi}\rbr{r_h^\theta - r_h}}(s_h, b_h)};
%     \end{align*}
%     which is bounded by $4B_r$ where $B_r$ boundes the reward function induced by $\theta\in\Theta$. Linear operator $\varPsi_h^\pi:\cF(\cS\times\cA\times\cB)\rightarrow \cF(\cS\times\cB)$ defined as 
% \begin{align*}
%     \rbr{\varPsi_h^\pi f}(s_h, b_h) = \dotp{\pi_h(\cdot\given s_h, b_h)}{f(s_h, \cdot, b_h)} - \dotp{\pi_h\otimes \nu_h^{\pi}(\cdot,\cdot\given s_h)}{f(s_h,\cdot,\cdot)}. 
% \end{align*}
% We let $d_{\cG_F}=\dim_\E(\cG_F, T^{-1/2})$ be the Eluder dimension of $\cG_F$.
% \end{itemize}
% }

We identify two function classes whose complexities determine the online learning hardness. The first one is the Bellman residuals $\cG_L:\cS\times\cA\times\cB\rightarrow \RR$ defined as 
$$\cG_L=\{U_h-\TT_h^{*,\theta} U_{h+1}, U\in\cU, \theta\in\Theta, h\in[H]\}, $$
where $\TT_h^{*,\theta}:\cF(\cS\times\cA\times\cB)\rightarrow \cF(\cS\times\cA\times\cB)$ is the Bellman optimality operator defined in \eqref{eq:define optimistic Bellman opt}.
The second one is the QRE defined in \eqref{eq:QRE} where we define the class of QREs $\cG_F:\cS\times\cB\rightarrow \RR$  as
$$
\cG_F =  \{\Upsilon_h^{\pi}(r_h^{\theta} - r_h), \pi\in\Pi, \theta\in\Theta, h\in[H]\}, 
$$
where recall the operator $\Upsilon_h^\pi:\cF(\cS\times\cA\times\cB)\rightarrow \cF(\cS\times\cB)$ defined as 
\begin{align*}
    \rbr{\Upsilon_h^\pi f}(s_h, b_h) = \dotp{\pi_h(\cdot\given s_h, b_h)}{f(s_h, \cdot, b_h)} - \dotp{\pi_h\otimes \nu_h^{\pi}(\cdot,\cdot\given s_h)}{f(s_h,\cdot,\cdot)}.
\end{align*}
In the sequel, we let $\dim(\cG_L)=\dim_\E(\cG_L, T^{-1/2})$ and $\dim(\cG_F)=\dim_\E(\cG_F, T^{-1/2})$ be the eluder dimensions for these two function classes. \footnote{See \Cref{sec:eluder dimension} for definition of the eluder dimension.} 

\begin{theorem}[{Regret for MLE-GOLF}]\label{thm:Online-MG}
    Suppose that the following conditions hold for model class $\Theta$ and function class $\cU$:
\begin{itemize}
    \item[(i)] (\textit{Realizability}) There exists $\theta^*\in\Theta$ such that $r_h^{\theta^*}=r_h$ for any $h\in[H]$. For $\theta^*\in\Theta$, there exists $U\in\cU$ such that $U_h = U_h^{*}$ for any $h\in[H]$;
    \item[(ii)] (\textit{Completeness}) For any $U\in\cU$ and $\theta\in\Theta$, there exists $U'\in\cU$ such that $U'=\TT_h^{*,\theta} U$ for any $h\in[H]$. 
\end{itemize}
    By choosing $\beta\ge c \max\cbr{ \log(HT\cN_\infty(\cZ, T^{-1})\delta^{-1}), \log(HT \cN_\rho(\Theta, T^{-1})/\delta)}$ for some universal constant $c>0$, where $\cN_\rho(\cZ,\epsilon)$ and $\cN_\rho(\Theta, \epsilon)$ are the maximal (over $h\in[H]$) $\epsilon$-covering number defined in \eqref{eq:cN-cZ}, \eqref{eq:cN-Theta-myopic} for the joint function class $\cZ_h = \Theta_{h+1}\times\cU^2$ and  $\Theta_h$, respectively. 
    We have for the online algorithm that 
    \begin{align*}
        \Reg(T)\lesssim H^2 \sqrt{\dim(\cG_L)\beta T} + H^2 \eta C_\eta \sqrt{\dim(\cG_F)\beta T} + H^2(\eta C_\eta)^3\exp(4\eta B_A) \beta \log T, 
    \end{align*}
    where $\lesssim$ only hides universal constants.
    \begin{proof}
        See \Cref{sec:proof-Online-MG} for a detailed proof.
    \end{proof}
\end{theorem}

\Cref{thm:Online-MG} characterizes the online learning complexity in terms of the eluder dimensions of function classes $\cG_L$ and $\cG_F$.
In particular, we do not suffer from any $\exp(2\eta B_A)$ coefficient in the $\sqrt{T}$ term thanks to optimism, though the $\log(T)$ term still has $C^{(3)}=\cO(\eta B_A\exp(2\eta B_A))$. 
We remark that when applied to the linear function approximation, we can reproduce the result in \Cref{thm:Online-ML} with $\dim(\cG_L) \lesssim d$ and $\dim(\cG_F) \lesssim d$. 

\section{Proofs for Farsighted Case}
Before we dive into the proof, we first present the following guarantee for the MLE method used in \S\Cref{sec:farsighted}.
\begin{lemma}[Guarantee of MLE]\label{lem:MLE}
    By choosing $\beta\ge  \allowbreak 9\log(3e^2H \cN_\rho(\cM,T^{-1})\delta^{-1})$, where $\cN_\rho(\cM,\epsilon)$ is the minimal size of an $\epsilon$-optimistic covering net of $\cM$. Here, an $\epsilon$-optimistic covering net $\cM_\epsilon\subset \cM$ is a finite subset such that for any $M\in\cM$, there exists $\tilde M\in\cM_\epsilon$ satisfying the following conditions:
    \begin{itemize}[leftmargin=20pt]
        \item[(i)] $D_\H\orbr{\nu_h^{\pi, M}(\cdot\given s_h), \nu_h^{\pi, \tilde M}(\cdot\given s_h)}\le \epsilon$, $D_\H\orbr{P_h^M(\cdot\given s_h, a_h, b_h), \allowbreak P_h^{\tilde M}(\cdot\given s_h, a_h, b_h)}\le \epsilon$, $\bigabr{(u_h^{M}-u_h^{\tilde M})(s_h,a_h,b_h)} \le \epsilon$, and $|(A_h^{\pi, M}-A_h^{\pi, \tilde M})(s_h, b_h)|\le \eta^{-1} \epsilon$ for all $\pi\in\Pi$, $h\in[H]$, $(s_h, a_h, b_h)\in\cS\times\cA\times\cB$;
        \item[(ii)] $P_h^{M}(s_{h+1}\given s_h, a_h, b_h)\le \exp(\epsilon)P_h^{\tilde M} (s_{h+1}\given s_h, a_h, b_h)$ for all $h\in[H]$ and $(s_{h+1}, s_h, a_h, b_h)\in\cS^2 \times \cA\times\cB$.
    \end{itemize}
%  \begin{align*}
%     &\rho(M, \tilde M) \nend
%     &\quad\defeq 6\max_{\pi\in\Pi, h\in[H]\atop (s_h, a_h, b_h)\in\cS\times\cA\times\cB}\Big\{D_\H\orbr{\nu_h^{\pi, M}(\cdot\given s_h), \nu_h^{\pi, \tilde M}(\cdot\given s_h)} , 
%     D_\H\orbr{P_h^M(\cdot\given s_h, a_h, b_h), \allowbreak P_h^{\tilde M}(\cdot\given s_h, a_h, b_h)} , \nend
%     &\qqquad \qqquad \qqquad \bigabr{(u_h^{M}-u_h^{\tilde M})(s_h,a_h,b_h)}, \bigabr{\orbr{Q_h^{\pi,M} - Q_h^{\pi,\tilde M}}(s_h,b_h)}\Big\}.
% \end{align*} 
The confidence set $\CI_\cM^t(\beta)$ satisfies the following with probability at least $1-\delta$: for any given $t\in[T]$, $M^*\in\confset_\cM^t(\beta)$, and it also holds for $\forall h\in[H], \forall M\in\confset^t(\beta)$ that
    \begin{align*}
        \sum_{i=1}^{t-1}D_{\RL, h}^2\rbr{M, M^*,\pi^i}
        \le 4\beta,\quad 
        \sum_{i=1}^{t-1}\hat D_{\RL, h,i}^2\rbr{M, M^*}
        \le 4\beta, 
    \end{align*}
 where 
    \begin{align*}
        D_{\RL,h}^2 (M,  M^*;\pi) &=   \EE^{\pi, M^*}D_\H^2\rbr{\nu_h^{\pi,M}, \nu_h^{\pi,M^*}} +
        \EE^{\pi, M^*} D_\H^2(P_h^{M}, P_h^{M^*}) +
        \EE^{\pi, M^*}\rbr{u_h^{M^*}-u_h^M}^2,\nend
        %%%%%%%%%%%
        D_{\RL,h,i}^2(M,M^*) &= D_\H^2\orbr{\nu_h^{\pi^i, M}(\cdot\given s_h^i), \nu_h^{\pi^i, \tilde M}(\cdot\given s_h^i)} +
        D_\H^2\orbr{P_h^M(\cdot\given s_h^i, a_h^i, b_h^i), \allowbreak P_h^{\tilde M}(\cdot\given s_h^i, a_h^i, b_h^i)} \nend
        &\qquad + \bigrbr{(u_h^{M}-u_h^{\tilde M})(s_h^i,a_h^i,b_h^i)}^2,
    \end{align*}
    and $\EE^{\pi, M}$ is taken under policy $\pi$ and the model $M$.
    \begin{proof}
        See \Cref{sec:proof-MLE} for a detailed proof.
    \end{proof}
\end{lemma}
\Cref{lem:MLE} guarantees that the confidence set $\confset^t(\beta)$ is valid in the sense that $M^*\in\confset^t(\beta)$ and any $M\in\confset^t(\beta)$ has $D_\RL$ bounded by $\cO(\beta)$. 
For the optimistic covering net, we remark that constraints in the first conditions are discussed in \eqref{eq:rho-cM}. The second condition requires $P_h^M$ to be dominated above by $P_h^{\tilde M}$, which is needed to control the difference in the log-likelihood.

\subsection{Proof of \Cref{thm:PMLE} on PMLE wiith Farsighted Follower}
\label{sec:proof-PMLE}
In this subsection, we provide a formal proof to \Cref{thm:PMLE}.
The proof is carried out as the following.

\paragraph{Step 1. Offline Suboptimality Decomposition}
By \Cref{lem:MLE}, we have with high probability that $M^*\in\CI_\cM(\beta)$, and we have the following suboptimality decomposition,
\begin{align*}
    J(\pi^*) - J(\hat\pi) 
    &= J(\pi^*) - J(\pi^*, M^{\pi^*}) + J(\pi^*, M^{\pi^*}) - J(\hat\pi, M^{\hat\pi}) + J(\hat\pi, M^{\hat\pi}) - J(\hat\pi)\nend
    &\le J(\pi^*) - J(\pi^*, M^{\pi^*}) +J(\pi^*, M^{\pi^*}) - J(\hat\pi, M^{\hat\pi})\nend
    &\le J(\pi^*) - J(\pi^*, M^{\pi^*}) ,
\end{align*}
where we define $M^\pi = \argmin_{M\in\CI_\cM(\beta)}J(\pi, M)$ as the pessimistic estimated model for $\pi$. Here, the first inequality holds by noting that $J(\hat\pi) = J(\hat\pi, M^*) \ge J(\hat\pi, M^{\hat\pi})$ by the validity of the confidence set $\CI_\cM(\beta)$ and the definition of $M^{\hat\pi}$, and the  second inequality is a direct result of policy optimization. Now, we further decompose the suboptimality using \Cref{lem:subopt-decomposition}. We define $\tilde M = M^{\pi^*}$, and let $\tilde U, \tilde W$ be the follower's value functions under policy $\pi^*$ and the estimated model $\tilde M$. Let $\tilde \nu$ be the estimated quantal response under $\pi^*$ and $\tilde M$. We have for $\pi^*, \tilde U, \tilde W, \tilde\nu$ that 
\begin{align*}
    &J(\pi^* ) - J(\pi^*, \tilde M)\nend
    &\quad \le \sum_{h=1}^H \EE \sbr{\rbr{\tilde U_h - u_h}(s_h, a_h, b_h) -  \tilde W_{h+1}(s_{h+1})} + \sum_{h=1}^H 2 H  \EE \sbr{D_\TV\rbr{\tilde \nu_h(\cdot\given s_h), \nu_h(\cdot\given s_h)}}\nend
    &\quad \le \sum_{h=1}^H \underbrace{\EE \sbr{\rbr{\tilde U_h - u_h}(s_h, a_h, b_h) -  \tilde W_{h+1}(s_{h+1})}}_{\dr\text{Leader's Bellman error}} \nend
    &\qqquad +  
    C^{(0)} \cdot 
    \sum_{h=1}^H \underbrace{\EE\bigsbr{\bigabr{\tilde \Delta^{(1)}_h(s_h, b_h)}}}_{\ds\text{1st-order error}}  + C^{(2)} \cdot 
    \max_{h\in [H]} \underbrace{\EE\bigsbr{ \bigrbr{\orbr{\tilde Q_h - r_h^{\pi^*} - \gamma P_h^{\pi^*} \tilde V_{h+1}}(s_h, b_h)}^2}}_{\ds\text{2nd-order error}}
\end{align*}
where the expectation is taken under $\pi^*$ and the true model $M^*$, and we define $\tilde Q, \tilde V$ as the follower's value functions under policy $\pi^*$ and model $\tilde M$. 
Here, the first inequality holds by \eqref{eq:perform-diff-linear} where we notice that $J(\pi^*, \tilde M) = \EE[\tilde W_1(s_1)]$ and also that $\tilde W_h = T_h^{\pi^*, \tilde \nu}\tilde U_h$. The second inequality holds by using \Cref{lem:performance diff}. Moreover, the definition of $\tilde\Delta_H^{(1)}(s_h, b_h) $ is given by 
\begin{align*}
    \tilde \Delta^{(1)}_h(s_h, b_h) &=  \rbr{\EE_{s_h, b_h} -\EE_{s_h}}\Biggsbr{\sum_{l=h}^H \gamma^{l-h}\underbrace{\rbr{\tilde Q_l - r_l^{\pi^*} - \gamma P_l^{\pi^*} \tilde V_{l+1}}(s_l, b_l)}_{\ds\text{Follower's Bellman error}}}. 
    % \label{eq:def Delta^1}
\end{align*}
In the sequel, we separately bound these three terms.

\paragraph{Step 2. Bounding the Leader's Bellman Error.}
We present the gurantee we have for the Leader's Bellman error on the samples. Define $\EE^i=\EE^{\pi^i}$, which is the expectation taken under $\pi^i$ and the true model $M^*$. We have
\begin{align*}
    &\sum_{i=1}^T 
    \EE^i \sbr{\rbr{\bigrbr{\tilde U_h - u_h}(s_h, a_h, b_h) - \tilde W_{h+1}(s_{h+1})}^2}\nend
    &\quad =\sum_{i=1}^T \EE^i \sbr{\rbr{{\bigrbr{\tilde u_h - u_h}(s_h, a_h, b_h) + \bigrbr{\tilde P_h - P_h} \tilde W_{h+1}(s_{h+1})}}^2}\nend
    &\quad \le 2\sum_{i=1}^T \EE^i \sbr{\rbr{\rbr{\tilde u_h - u_h}(s_h,a_h,b_h)}^2} + 8 H^2 \sum_{i=1}^T \EE^i D_\TV^2\rbr{\tilde P_h(\cdot\given s_h, a_h, b_h), P_h(\cdot\given s_h, a_h, b_h)}\nend
    &\quad \lesssim  H^2 \beta, 
\end{align*}
where the first equality holds by the definition of $\tilde U$, and the first inequality holds by the Jensen's inequality, and the last inequality holds by the MLE guarantee in \Cref{lem:MLE}, where we hide some universal constants by \say{$\lesssim$}.
Therefore, we have that
\begin{align*}
    &\sum_{h=1}^H \EE \sbr{\rbr{\tilde U_h - u_h}(s_h, a_h, b_h) -  \tilde W_{h+1}(s_{h+1})}\nend
    &\quad \le \sum_{h=1}^H \sqrt{\EE \Bigsbr{\Bigrbr{\bigrbr{\tilde U_h - u_h}(s_h, a_h, b_h) -  \tilde W_{h+1}(s_{h+1})}^2}}\nend
    &\quad \lesssim H\sqrt{H^2 \beta} \cdot 
    \sqrt{\frac{\EE \Bigsbr{\Bigrbr{\bigrbr{\tilde U_h - u_h}(s_h, a_h, b_h) -  \tilde W_{h+1}(s_{h+1})}^2}}{\sum_{i=1}^T 
    \EE^i \Bigsbr{\rbr{\bigrbr{\tilde U_h - u_h}(s_h, a_h, b_h) - \tilde W_{h+1}(s_{h+1})}^2}}}\nend
    &\quad \le H^2\sqrt{ \beta} \cdot \max_{M\in\cM, h\in[H]} \sqrt{\frac{\EE\sbr{\rbr{\bigrbr{U_h^{\pi^*, M}-\bigrbr{u_h+P_h W_{h+1}^{\pi^*, M}}}(s_h, a_h, b_h)}^2 }}{\sum_{i=1}^T \EE^i\sbr{\rbr{\bigrbr{U_h^{\pi^*, M}-\bigrbr{u_h+P_h W_{h+1}^{\pi^*, M}}}(s_h, a_h, b_h)}^2 }}}.
\end{align*}

\paragraph{Step 3. Bound the first-order Error of the Follower's Response.}
We first study the following guarantee for the 1st order term over the offline samples,
\begin{align*}
    &\sum_{i=1}^T \EE^i \sbr{\rbr{\rbr{\EE_{s_h, b_h}^i -\EE_{s_h}^i}\sbr{\sum_{l=h}^H \gamma^{l-h}\rbr{\tilde r_l - r_l + \gamma \bigrbr{\tilde P_l - P_l} \tilde V_{l+1}}(s_l,a_l,  b_l)}}^2}\nend
    %%%%%%%%%%%%
    &\quad \lesssim \sum_{i=1}^T \EE^i \sbr{\rbr{\rbr{\EE_{s_h, b_h}^i -\EE_{s_h}^i}\sbr{\sum_{l=h}^H \gamma^{l-h}\rbr{\tilde r_l - r_l + \gamma \bigrbr{\tilde P_l - P_l} V_{l+1}^{\pi^i, \tilde M}}(s_l,a_l,  b_l)}}^2}\nend
    &\qqquad + \sum_{i=1}^T \EE^i \sbr{\rbr{\rbr{\EE_{s_h, b_h}^i -\EE_{s_h}^i}\sbr{\sum_{l=h}^H \gamma^{l-h}\rbr{\gamma \bigrbr{\tilde P_l - P_l} \bigrbr{\tilde V_{l+1} - V_{l+1}^{\pi^i,\tilde M}}}(s_l,a_l,  b_l)}}^2} \nend
    &\quad \lesssim \sum_{i=1}^T \EE^i \sbr{\rbr{\rbr{\EE_{s_h, b_h}^i -\EE_{s_h}^i}\Biggsbr{\sum_{l=h}^H \gamma^{l-h}\rbr{\tilde r_l - r_l + \gamma \bigrbr{\tilde P_l - P_l} V_{l+1}^{\pi^i, \tilde M} }(s_l, a_l, b_l)}}^2}\nend
    &\qqquad + B_A^2 \sum_{i=1}^T   \EE^i\rbr{\sum_{l=h}^H \gamma^{l-h}\rbr{\EE_{s_h,b_h}^i + \EE_{s_h}^i}  \sbr{D_\TV\rbr{\tilde P_l(\cdot\given s_l, a_l, b_l), P_l(\cdot\given s_l, a_l, b_l)}}}^2\nend
    &\quad \lesssim \sum_{i=1}^T \EE^i \biggsbr{\biggrbr{\underbrace{\rbr{\EE_{s_h, b_h}^i -\EE_{s_h}^i}\Biggsbr{\sum_{l=h}^H \gamma^{l-h}\rbr{\tilde r_l - r_l + \gamma \bigrbr{\tilde P_l - P_l} V_{l+1}^{\pi^i, \tilde M} }(s_l, a_l, b_l)}}_{\ds \tilde \Delta^{(1)}_{h, \pi^i, \tilde M}(s_h, b_h) }}^2}\nend
    &\qqquad + B_A^2 \sum_{i=1}^T   \eff_H(\gamma) \sum_{l=h}^H \gamma^{l-h} \EE^i D_\TV^2\rbr{\tilde P_l(\cdot\given s_l, a_l, b_l), P_l(\cdot\given s_l, a_l, b_l)}
\end{align*}
where the first inequality holds by the Jensen's inequality, where we add a $(\tilde P_l - P_l) V_{l+1}^{\pi^i, \tilde M}$ term and substract it, which gives us a separate $(\tilde P_l - P_l) (\tilde V_{l+1} - V_{l+1}^{\pi^i, \tilde M})$ term. 
In the second inquality, we upper bound $(\tilde P_l - P_l) (\tilde V_{l+1} - V_{l+1}^{\pi^i, \tilde M})$ by the TV distance between $\tilde P_l$ and $P_l$ multiplied by the infinity norm $\bignbr{\tilde V_{l+1} - V_{l+1}^{\pi^i, \tilde M}}_\infty$, which is bounded by $2 B_A$ by our argument in \Cref{sec:app-notations}. Since the TV distance is always nonnegative, we can safely flips the sign between $\EE_{s_h, b_h}^i - \EE_{s_h}^i $. The above steps give us the first inequality. 
The second inequality simply holds by using the Cauchy-Schwartz inequality where we move the square inside the expectation for the summation of the TV distance, and the $\eff_H(\gamma)$ is just a byproduct produced when applying the Cauchy-Schwartz inequality.

Next, we show how to control this $\tilde \Delta_{h,\pi^i, \tilde M}(s_h, b_h)$ term. We first notice that $\tilde\Delta^{(1)}_{h, \pi^i, \tilde M}(s_h, b_h)$ is nothing but just $\tilde\Delta^{(1)}_h(s_h, b_h)$ plugged in with $\pi^i$ as the policy $\pi$ and $\tilde Q^{\pi^i}=Q^{\pi^i, \tilde M}, \tilde V^{\pi^i} = V^{\pi^i, \tilde M}$ as the follower's value functions $\tilde Q, \tilde V$. 
Hence, we can invoke \Cref{lem:1st-ub} which says that 
\begin{align*}
    &\bigrbr{\tilde \Delta_{h, \pi^i,\tilde M}^{(1)}(s_h, b_h)}^2  \nend
        &\quad \le 2 \rbr{\rbr{\EE_{s_h, b_h}^i-\EE_{s_h}^i} \bigsbr{\orbr{Q_h^{\pi^i} - \tilde Q_h^{\pi^i}}(s_h, b_h)}}^2 \nend
        &\qqquad + 16 \gamma^2  \rbr{\eta^{-1} +2 B_A}^2\eff_H(\gamma) \sum_{l=h+1}^H \gamma^{l-h-1} {\rbr{\EE_{s_h}^i+\EE_{s_h, b_h}^i}\sbr{D_\H^2(\nu_l^{\pi^i}(\cdot\given s_l), \tilde\nu_l^{\pi^i}(\cdot\given s_l))}}
\end{align*}
where we define $\tilde \nu_l^{\pi^i} = \nu_l^{\pi^i, \tilde M}$ as the quantal response under $\pi^i$ and the estimated model $\tilde M$. This is true by our definition of $\tilde Q^{\pi^i}, \tilde V^{\pi^i}$ that they are the follower's value functions under $\pi^i$ and model $\tilde M$.
Therefore, we have that 
\begin{align*}
    &\sum_{i=1}^T \EE^i\bigrbr{\tilde \Delta_{h, \pi^i,\tilde M}^{(1)}(s_h, b_h)}^2  \nend
    &\quad \lesssim \sum_{i=1}^T \EE^i\rbr{\rbr{\EE_{s_h, b_h}^i-\EE_{s_h}^i} \bigsbr{\orbr{Q_h^{\pi^i} - \tilde Q_h^{\pi^i}}(s_h, b_h)}}^2 \nend
    &\qqquad + \gamma^2  \rbr{\eta^{-1} +2 B_A}^2\eff_H(\gamma) \sum_{i=1}^T \sum_{l=h+1}^H \gamma^{l-h-1} {\EE^i\sbr{D_\H^2(\nu_l^{\pi^i}(\cdot\given s_l), \tilde\nu_l^{\pi^i}(\cdot\given s_l))}} \nend
    &\quad\lesssim \rbr{(\eta^{-2}+B_A^2) + \gamma^2  \rbr{\eta^{-1} +2 B_A}^2\eff_H(\gamma) H} \beta, 
\end{align*}
where the last inequality holds by both \eqref{eq:MLE-guarantee-Q-3} in \Cref{lem:MLE-formal} for the Q difference term and the MLE guarantee in \Cref{lem:MLE} for the Hellinger term. Here, we upper bound $\gamma^{l-h-1}$ by $1$ and take a summation over $h\in[H]$. Therefore, we conclude that 
\begin{align*}
    &
    \sum_{i=1}^T \EE^i \sbr{\rbr{\rbr{\EE_{s_h, b_h}^i -\EE_{s_h}^i}\sbr{\sum_{l=h}^H \gamma^{l-h}\rbr{\tilde r_l - r_l + \gamma \bigrbr{\tilde P_l - P_l} \tilde V_{l+1}}(s_l,a_l,  b_l)}}^2} \nend
    &\quad\lesssim  
    \sum_{i=1}^T \EE^i\bigrbr{\tilde \Delta_{h, \pi^i,\tilde M}^{(1)}(s_h, b_h)}^2 \nend
    &\qqquad + 
    B_A^2 \sum_{i=1}^T   \eff_H(\gamma) \sum_{l=h}^H \gamma^{l-h} \EE^i D_\TV^2\rbr{\tilde P_l(\cdot\given s_l, a_l, b_l), P_l(\cdot\given s_l, a_l, b_l)}\nend
    &\quad \le  \rbr{(\eta^{-2}+B_A^2) + \gamma^2  \rbr{\eta^{-1} +2 B_A}^2\eff_H(\gamma) H} \beta  + B_A^2 \eff_H(\gamma) H \beta \nend
    &\quad \lesssim \rbr{\eta^{-2} + B_A^2 }\eff_H(\gamma)H \beta. 
\end{align*}
As a result, we have that
\begin{align*}
    &C^{(0)}\sum_{h=1}^H \EE\sbr{\abr{\tilde \Delta_h^{(1)}(s_h, b_h)}}\nend
    &\quad\le C^{(0)} \sum_{h=1}^H \sqrt{\EE\sbr{\tilde \Delta_h^{(1)}(s_h, b_h)^2}}\nend
    %%%%%%%%%%%%%
    &\quad \lesssim C^{(0)} H \sqrt{ \rbr{\eta^{-2} + B_A^2 }\eff_H(\gamma)H \beta} \nend
    &\qqquad \cdot \max_{h\in[H]}\sqrt\frac{\EE \sbr{\rbr{\rbr{\EE_{s_h, b_h} -\EE_{s_h}}\sbr{\sum_{l=h}^H \gamma^{l-h}\rbr{\tilde r_l - r_l + \gamma \bigrbr{\tilde P_l - P_l} \tilde V_{l+1}}(s_l,a_l,  b_l)}}^2}}{\sum_{i=1}^T \EE^i \sbr{\rbr{\rbr{\EE_{s_h, b_h}^i -\EE_{s_h}^i}\sbr{\sum_{l=h}^H \gamma^{l-h}\rbr{\tilde r_l - r_l + \gamma \bigrbr{\tilde P_l - P_l} \tilde V_{l+1}}(s_l,a_l,  b_l)}}^2}}\nend
    %%%%%%%%%%%%
    &\quad \lesssim \rbr{1 + \eta B_A }H^2\sqrt{ H \eff_H(\gamma)\beta} \nend
    &\quad \cdot \max_{M\in\cM, h\in[H]}\sqrt\frac{\EE \sbr{\rbr{\rbr{\EE_{s_h, b_h} -\EE_{s_h}}\sbr{\sum_{l=h}^H \gamma^{l-h}\rbr{r_l^M - r_l + \gamma \bigrbr{P_l^M - P_l} V_{l+1}^{\pi^*, M}}(s_l,a_l,  b_l)}}^2}}{\sum_{i=1}^T \EE^i \sbr{\rbr{\rbr{\EE_{s_h, b_h}^i -\EE_{s_h}^i}\sbr{\sum_{l=h}^H \gamma^{l-h}\rbr{ r_l^M - r_l + \gamma \bigrbr{P_l^M - P_l} V_{l+1}^{\pi^*, M}}(s_l,a_l,  b_l)}}^2}}, 
\end{align*}
where we notice that $C^{(0)}=2\eta H$.

\paragraph{Step 4. Bound the second-order Error in the Follower's Response.}
The last thing to do is controlling the second order term. We first expand the second order term in terms of $r_h, P_h, V_{h+1}$ by definitions and have the following guarantee for the second order term over the samples, 
\begin{align*}
    &\sum_{i=1}^T \EE^i\sbr{ \rbr {\EE_{s_h, b_h}^i\sbr{{\bigrbr{\tilde r_h - r_h + \gamma \bigrbr{\tilde P_h - P_h} \tilde V_{h+1}}(s_h,a_h, b_h)}}}^2}\nend
    &\quad \lesssim \sum_{i=1}^T \EE^i\sbr{ \rbr {\EE_{s_h, b_h}^i\sbr{{\bigrbr{\tilde r_h - r_h + \gamma \bigrbr{\tilde P_h - P_h} V_{h+1}^{\pi^i, \tilde M}}(s_h,a_h, b_h)}}}^2} \nend
    &\qqquad + \sum_{i=1}^T \EE^i\sbr{ \rbr {\EE_{s_h, b_h}^i\sbr{{\gamma \bigrbr{\tilde P_h - P_h} \bigrbr{\tilde V_{h+1} - V_{h+1}^{\pi^i, \tilde M} }(s_h,a_h, b_h)}}}^2}\nend
    %%%%%% split %%%%%%
    &\quad \lesssim \sum_{i=1}^T \EE^i\sbr{ \rbr {\EE_{s_h, b_h}^i\sbr{{\bigrbr{\tilde r_h - r_h + \gamma \bigrbr{\tilde P_h - P_h} V_{h+1}^{\pi^i, \tilde M}}(s_h,a_h, b_h)}}}^2} \nend
    &\qqquad + \gamma^2B_A^2 \sum_{i=1}^T \EE^i\sbr{ D_\TV^2\rbr{\tilde P_h(\cdot\given s_h, a_h, b_h), P_h(\cdot\given s_h, a_h, b_h)}}, 
\end{align*}
where the first inequality holds by the Jensen's inequality, the second inequality holds by upper bounding the difference in $(\tilde P-P)(\tilde V-V^{\pi^i, \tilde M})$ by the TV distance and the upper bound for the follower's V function as $B_A$.
We now invoke \Cref{lem:2nd-ub} for $(\pi^i, Q_h^{\pi^i,\tilde M}, V_{h+1}^{\pi^i, \tilde M})$, which gives us 
\begin{align}
    &\sum_{i=1}^T \max_{h\in[H]}\EE^i\sbr{ \rbr {\EE_{s_h, b_h}^i\sbr{{\bigrbr{\tilde r_h - r_h + \gamma \bigrbr{\tilde P_h - P_h} V_{h+1}^{\pi^i, \tilde M}}(s_h,a_h, b_h)}}}^2} \nend
    &\quad =\sum_{i=1}^T\max_{h\in[H]}\EE^i\sbr{ \rbr{\rbr{Q_h^{\pi^i,\tilde M} - r_h^{\pi^i} - \gamma P_h^{\pi^i}  V_{h+1}^{\pi^i, \tilde M}}(s_h, b_h)}^2} \nend
    &\quad \le \sum_{i=1}^T L^{(2)} \sum_{h=1}^H \cbr{\EE D_\H^2(\nu_h^{\pi^i}(\cdot\given s_h),\nu_h^{\pi^i, \tilde M}(\cdot\given s_h))+\EE D_\TV^2(P_h^{\pi^i}(\cdot\given s_h, b_h),P_h^{\pi^i, \tilde M}(\cdot\given s_h, b_h))}\nend
    &\quad \lesssim L^{(2)} H \beta .\label{eq:PMLE-1}
\end{align}
where in the first inequality, we additionally replace the maximum over $h\in[H]$ as a summation over $h\in[H]$, and the second inequlity holds by the MLE guarantee in \Cref{lem:MLE}. Hence, we conclude that 
\begin{align*}
    &\sum_{i=1}^T \EE^i\sbr{ \rbr {\EE_{s_h, b_h}^i\sbr{{\bigrbr{\tilde r_h - r_h + \gamma \bigrbr{\tilde P_h - P_h} \tilde V_{h+1}}(s_h,a_h, b_h)}}}^2}\nend
    %%%%%%%% split %%%%
    &\quad \lesssim \sum_{i=1}^T \EE^i\sbr{ \rbr {\EE_{s_h, b_h}^i\sbr{{\bigrbr{\tilde r_h - r_h + \gamma \bigrbr{\tilde P_h - P_h} V_{h+1}^{\pi^i, \tilde M}}(s_h,a_h, b_h)}}}^2} \nend
    &\qqquad + \gamma^2B_A^2 \sum_{i=1}^T \EE^i\sbr{ D_\TV^2\rbr{\tilde P_h(\cdot\given s_h, a_h, b_h), P_h(\cdot\given s_h, a_h, b_h)}}\nend
    &\quad \lesssim L^{(2)} H \beta + \gamma^2 B_A^2 \beta, 
\end{align*}
where the last inequality holds by using \eqref{eq:PMLE-1} for the first term and the MLE guarantee in \Cref{lem:MLE} for the second term.
Now, we invoke \Cref{prop:de-regret-prop} and obtain 
\begin{align*}
    &C^{(2)} \cdot 
    \max_{h\in[H]}\EE\bigsbr{ \bigrbr{\orbr{\tilde Q_h - r_h^{\pi^*} - \gamma P_h^{\pi^*} \tilde V_{h+1}}(s_h, b_h)}^2} \nend
    &\quad = C^{(2)} \cdot \max_{h\in[H]}
    \EE\sbr{ \rbr{\EE_{s_h, b_h}\sbr{\bigrbr{\tilde r_h - r_h + \gamma \bigrbr{\tilde P_h - P_h} \tilde V_{h+1}}(s_h, a_h, b_h)}}^2} \nend
    &\quad \lesssim C^{(2)} \bigrbr{L^{(2)} H \beta + \gamma^2 B_A^2 \beta} \nend
    &\qqquad\cdot \max_{h\in[H]}\frac{\EE\sbr{ \rbr{\EE_{s_h, b_h}\sbr{\bigrbr{\tilde r_h - r_h + \gamma \bigrbr{\tilde P_h - P_h} \tilde V_{h+1}}(s_h, a_h, b_h)}}^2}}{\sum_{i=1}^T \EE^i\sbr{ \rbr {\EE_{s_h, b_h}^i\sbr{{\bigrbr{\tilde r_h - r_h + \gamma \bigrbr{\tilde P_h - P_h} \tilde V_{h+1}}(s_h,a_h, b_h)}}}^2}}\nend
    &\quad \lesssim C^{(2)} \bigrbr{L^{(2)} H \beta + \gamma^2 B_A^2 \beta} \nend
    &\qqquad\cdot \max_{h\in[H], M\in\cM}\frac{\EE\sbr{ \rbr{\EE_{s_h, b_h}\sbr{\bigrbr{r_h^{M} - r_h + \gamma \bigrbr{P_h^M - P_h}  V_{h+1}^{\pi^*, M}}(s_h, a_h, b_h)}}^2}}{\sum_{i=1}^T \EE^i\sbr{ \rbr {\EE_{s_h, b_h}^i\sbr{{\bigrbr{r_h^{M} - r_h + \gamma \bigrbr{P_h^M - P_h}  V_{h+1}^{\pi^*, M}}(s_h,a_h, b_h)}}}^2}}, 
\end{align*}
where the first inequality is just a distribution, and the last inequality takes a maximum over $\cM$. 

In summary, for the leader's Bellman error, 
\begin{align*}
    \text{LBE} \lesssim H^2 \sqrt{\beta} \cdot \max_{M\in\cM, h\in[H]} \sqrt{\frac{\EE\sbr{\rbr{\bigrbr{U_h^{\pi^*, M}-\bigrbr{u_h+P_h W_{h+1}^{\pi^*, M}}}(s_h, a_h, b_h)}^2 }}{\sum_{i=1}^T \EE^i\sbr{\rbr{\bigrbr{U_h^{\pi^*, M}-\bigrbr{u_h+P_h W_{h+1}^{\pi^*, M}}}(s_h, a_h, b_h)}^2 }}},
\end{align*}
for the first order term in the follower's quantal response error, 
\begin{align*}
    &\text{1st-QRE} \nend
    &\quad \lesssim  \eta C_\eta H^2\sqrt{ H \eff_H(\gamma)\beta} \nend
    &\qquad \cdot \max_{M\in\cM, h\in[H]}\sqrt\frac{\EE \sbr{\rbr{\rbr{\EE_{s_h, b_h} -\EE_{s_h}}\sbr{\sum_{l=h}^H \gamma^{l-h}\rbr{r_l^M - r_l + \gamma \bigrbr{P_l^M - P_l} V_{l+1}^{\pi^*, M}}(s_l,a_l,  b_l)}}^2}}{\sum_{i=1}^T \EE^i \sbr{\rbr{\rbr{\EE_{s_h, b_h}^i -\EE_{s_h}^i}\sbr{\sum_{l=h}^H \gamma^{l-h}\rbr{ r_l^M - r_l + \gamma \bigrbr{P_l^M - P_l} V_{l+1}^{\pi^*, M}}(s_l,a_l,  b_l)}}^2}}, 
\end{align*}
and for the second order term in the follower's quantal response error, 
\begin{align*}
    &\text{2nd-QRE}\nend
    &\quad \lesssim C^{(2)} L^{(2)} H \beta \nend
    &\qqquad\cdot \max_{h\in[H], M\in\cM}\frac{\EE\sbr{ \rbr{\EE_{s_h, b_h}\sbr{\bigrbr{r_h^{M} - r_h + \gamma \bigrbr{P_h^M - P_h}  V_{h+1}^{\pi^*, M}}(s_h, a_h, b_h)}}^2}}{\sum_{i=1}^T \EE^i\sbr{ \rbr {\EE_{s_h, b_h}^i\sbr{{\bigrbr{r_h^{M} - r_h + \gamma \bigrbr{P_h^M - P_h}  V_{h+1}^{\pi^*, M}}(s_h,a_h, b_h)}}}^2}}. 
\end{align*}
Hence, we complete the proof of \Cref{thm:PMLE}

\subsection{Proof of \Cref{thm:OMLE-farsighted} on OMLE with Farsighted Follower}\label{sec:proof-farsighted MDP}
In this section, we provide formal proofs for the result to Theorem \ref{thm:OMLE-farsighted}.
The proof relies on a novel decomposition of the online regret in the Taylor-series form and utilizes the techniques for analyzing the OMLE in \citep{chen2022unified, foster2021statistical,jin2021bellman}. Before diving into the proof, we present the following key lemma that provides guarantees for the confidence set $\confset_\cM^t(\beta)$ given by the algorithm.

The proof is carried out as the following. We recall that $\beta\ge  \allowbreak 9\log(3e^2TH \cN_\rho(\cM,T^{-1})\delta^{-1})$, where we additionally include an $\log T$ term to ensure a union bound over $t\in[T]$.

\paragraph{Step 1. Online Regret Decompostion. }
By \Cref{lem:MLE}, we have with high probability that $M^*\in\confset_\cM^t(\beta)$. Hence, we can upper bound the online regret by
\begin{align*}
    \Reg(T) = \sum_{t=1}^T J(\pi^*, M^*) - J(\pi^t, M^*) \le \sum_{t=1}^T J(\pi^t, M^t) - J(\pi^t, M^*), 
\end{align*}
where the inequality holds by additionally noting that OMLE produces the pair $(\pi^t, M^t)$ that maximizes $J$ within the confidence set $\confset^t(\beta)$.
The key in studying the regret in this MDP with strategic follower is decomposing the performance difference into both the follower's temporal difference and the follower's response difference. 
We invoke \Cref{lem:subopt-decomposition} with $\pi^t, \tilde U^t, \tilde W^t, \tilde \nu^t$, where $\tilde U^t, \tilde W^t, \tilde \nu^t$ are given under policy $\pi^t$ and the estimated model $\tilde M^t$, and they should satisfy $\tilde W_h^t(s_h)=(T_h^{\pi^t, \tilde \nu^t}\tilde U_h^t) (s_h)$. We additionally define $\nu^t = \nu^{\pi^t, M^*}$. We have that
\begin{align*}
    &J(\pi^t, \tilde M^t) - J(\pi^t, M^*)\nend
    &\quad \le \sum_{h=1}^H \EE^t \sbr{\rbr{\tilde U_h^t - u_h}(s_h, a_h, b_h) -  \tilde W_{h+1}^t(s_{h+1})} + \sum_{h=1}^H 2 H  \EE^t \sbr{D_\TV\rbr{\tilde \nu_h^t(\cdot\given s_h), \nu_h^t(\cdot\given s_h)}}\nend
    &\quad \le \sum_{h=1}^H \underbrace{\EE^t \sbr{\rbr{\tilde U_h^t - u_h}(s_h, a_h, b_h) - \tilde W_{h+1}^t(s_{h+1})}}_{\dr \text{Leader's Bellman error}} \nend
    &\qqquad +   C^{(0)}
    \sum_{h=1}^H \underbrace{\EE^t\sbr{\abr{\tilde \Delta^{(1,t)}_h(s_h, b_h)}}}_{\ds\text{1st-order error}}  + C^{(2)}
    \max_{h\in [H]} \underbrace{\EE^t\sbr{ \rbr{\rbr{\tilde Q_h^t - r_h^{\pi^t} - \gamma P_h^{\pi^t} \tilde V_{h+1}^t}(s_h, b_h)}^2}}_{\ds\text{2nd-order error}}
\end{align*}
where the expectation $\EE^t$ is taken with respect to $\pi^t$ and the true model $M^*$, 
$C^{(0)}=2\eta H$, 
$C^{(2)} = 2 H\eta^2 H (1+4 \eff_H(\gamma))\exp\rbr{6\eta B_A}\cdot \rbr{\eff_H(\exp(2\eta B_A)\gamma)}^2$ with $\eff_H(x) = (1-x^H)/(1-x)$ as the \say{effective}  horizon with respect to $x$, 
and $\tilde\Delta_j^{(1, t)}(s_h, b_h)$ is defined as
\begin{align*}
    \tilde \Delta^{(1, t)}_h(s_h, b_h) &=  \rbr{\EE_{s_h, b_h}^t -\EE_{s_h}^t}\Biggsbr{\sum_{l=h}^H \gamma^{l-h}\underbrace{\rbr{\tilde Q_l^t - r_l^{\pi^t} - \gamma P_l^{\pi^t} \tilde V_{l+1}^t}(s_l, b_l)}_{\ds\text{Follower's Bellman error}}}.
\end{align*}
Here, the second inequality comes from \Cref{lem:performance diff} and uses the definition that $\tilde V_h^t = \eta^{-1}\log \int \exp(\eta \tilde Q_h^t)$, $\tilde A_h^t = \tilde Q_h^t -\tilde V_h^t$,  and that $\tilde\nu_h^t=\exp\orbr{\eta \tilde A_h^t}$ under the alternative model $\tilde M^t$. In the sequel, we will bound these three terms separately.

\paragraph{Step 2. Bounding the Leader's Bellman Error.}
We first show that the leader's Bellman error is controllable when summed up for $T$ steps. Specifically, consider the following configurations for step $h\in[H]$,
\begin{itemize}[leftmargin=20pt]
    \item[(i)] Define function class $\cG_{h,L}$ as
    \begin{align*}
        \cG_L^h &= \Big\{g:\cS\times\cA\times\cB\rightarrow \RR \Biggiven g={\bigrbr{U_h^{\pi^{\tilde M},\tilde M} - u - P_h W_{h+1}^{\pi^{\tilde M},\tilde M}}(s_h, a_h, b_h)}, \exists \tilde M\in\cM\Big\}, 
    \end{align*}
    where we define $\pi^M = \argmax_{\pi\in\Pi}J(\pi, M)$.
    Specifically, the expectation is taken under $\pi$ and the true model. Consider a sequence of function $\{g_h^i = (\tilde U_h^{i} - u - P_h \tilde W_{h+1}^i)\}_{i\in[T]}$. We see directly that $g_h^i\in\cG_{h, L}$ since $\tilde U^i = U^{\pi^i, M^i}$ and we have by the optimism in the algorithm that $\pi^i = \pi^{M^i}$. The same also holds for $\tilde W^i$.
    \item[(ii)] Define a class of probability measures over $\cS\times\cA\times\cB$ as $$\sP_{h, L}=\{\PP^\pi((s_h, a_h, b_h)=\cdot), \forall \pi\in\Pi\}.$$
    Consider a sequence of probability measures $\{\rho_h^i(\cdot)=\PP^{\pi^i}((s_h, a_h, b_h)=\cdot)\}_{i\in[T]}$.
    % \item Define a class of probability measures over space $\cS\times\cA\times\cB$ as \begin{align*}
    %     \sP_L = \cbr{\rho\in\Delta(\cS\times\cA\times\cB): \exists h\in[H], \pi\in\Pi, \rho(\cdot)=\PP^\pi((s_h, a_h, b_h)=\cdot)}.
    % \end{align*}
    \item[(iii)] Under this two sequences, we denote by $g_h^t(\pi^i) = \EE_{\rho_h^i}[g_h^t]$ for simplicity. 
    We have 
    $$g_h^t(\pi^i) =\EE^i\bigsbr{{\bigrbr{\tilde U_h^t - u - P_h \tilde W_{h+1}^t}(s_h,a_h,b_h)}}, $$
    which should be bounded by $3H$.
\end{itemize}
We denote by $\dim(\cG_L) = \max_{h\in[H]}\dim_\DE(\cG_{h, L}, \sP_{h, L}, T^{-1/2})$ in the sequel.
Our guarantee for the sequence $\{g_h^i\}_{i\in[T]}$ and $\{\pi^i\}_{i\in[T]}$ is 
\begin{align}\label{eq:OnN-guarantee-Bellmanerror}
    \sum_{i=1}^{t-1} \rbr{g_h^t(\pi^i)}^2 
    &= \sum_{i=1}^t \rbr{\EE^i\bigsbr{{\bigrbr{\tilde U_h^t - u - P_h \tilde W_{h+1}^t}(s_h,a_h,b_h)}}}^2 \nend
    &\le \sum_{i=1}^t \EE^i\sbr{\rbr{\bigrbr{\tilde U_h^t - u_h - P_h \tilde W_{h+1}^t}(s_h,a_h,b_h)}^2}\nend
    &\le \sum_{i=1}^t 2 \EE^i\sbr{\rbr{\rbr{\tilde u_h^t - u_h}(s_h, a_h, b_h)}^2} + 8 H^2 \EE^i D_\TV^2\rbr{P_h(\cdot\given s_h, a_h, b_h), \tilde P_h^t(\cdot\given s_h, a_h, b_h)}\nend
    &\le 8 H^2 \cdot 4 \beta,
\end{align}
where we define $\tilde u_h^t = u_h^{M^t}$ and $\tilde P_h^t = P_h^{M^t}$. The first ineqality holds by the Cauchy-Schwartz inequality ,  the second inequality holds by noting that $\tilde U_h^t = \tilde u_h^t + \tilde P_h^t \tilde W_h^t$, and the last inequality holds by invoking the guarantee in \Cref{lem:MLE}. We then have by the first order argument in \Cref{lem:de-regret} that 
\begin{align*}
    \sum_{i=1}^T \abr{g_h^t(\pi^t)} \le 2\sqrt{\dim\rbr{\cG_L} 32 H^2 \beta T} + 3 H\min\cbr{T, \dim\rbr{\cG_L}}  + \sqrt T, 
\end{align*}
which implies that the leader's Bellman error is upper bounded by $\cO( H^2 \sqrt{\dim\rbr{\cG_L} H \beta T})$.

\paragraph{Step 3. Bounding the first-order Term.}
we next show that the first-order term in the follower's response error is also under control. 
\begin{itemize}[leftmargin=20pt]
    \item[(i)] Define function class $\cG_{h, F}^1$ as 
    \begin{align*}
        \cG_{h, F}^1 &= \Bigg\{g:(\cS\times\cA\times\cB)^{H-h+1}\rightarrow \RR \bigggiven \exists M\in\cM, \nend
        &\qqquad g((s_l, a_l, b_l)_{l=h}^H) = {\sum_{l=h}^H \gamma^{l-h}\bigrbr{r_l^M- r_l + \gamma (P_l^{M} - P_l) V_{l+1}^{\pi^M, M}}(s_l, a_l, b_l)}
        \Bigg\},
    \end{align*}
    where we remind the readers that $\pi^M = \argmax_{\pi\in\Pi}J(\pi, M)$ only depends on $M$. Consider 
    sequences 
    $$\cbr{g_h^t=\sum_{l=h}^H \gamma^{l-h}\orbr{\tilde r_l^t- r_l + \gamma (\tilde P_l^t - P_l) \tilde V_{l+1}^t}}_{t\in[T]}, $$
    where we define $\tilde r_l^t = r_l^{M^t}$ and $\tilde P_l^t = P_l^{M^t}$.
    It is obvious that $g_h^t\in\cG_{h, F}^1$ since $\tilde V_h^t = V_h^{\pi^t, M^t}$ and $\pi^t = \argmax_{\pi\in\Pi}J(\pi, M^t) = \pi^{M^t}$.
    % $r_h^{\pi}(s_h, b_h) = \dotp{r_h(s_h, \cdot, b_h)}{\pi_h(\cdot\given s_h, b_h)}_\cA$ and the same holds for $P_h^\pi$. 
    % One can see immediately that $\cG_F^1$ is a bilinear class, where the expectation part depends on $\pi$ and the follower's Bellman error part depends on $M$.
    
    \item Define a class of signed measures over $(\cS\times\cA\times\cB)^{H-h+1}$ as 
    \begin{equation*}
        \sP_{h, F}^1 = \cbr{\begin{aligned}
            &\PP^\pi(((s_l, a_l, b_l)_{l=h+1}^H , a_h)=\cdot \given s_h, b_h)\delta_{(s_h, b_h)}(\cdot) \nend
            &\quad - \PP^\pi(((s_l, a_l, b_l)_{l=h+1}^H , a_h, b_h)=\cdot \given s_h)\delta_{(s_h)}(\cdot) 
        \end{aligned}
        \bigggiven \pi\in\Pi, (s_h, b_h)\in\cS\times\cB}, 
    \end{equation*}
    where $\delta_{s_h, b_h}$ is the measure that puts measure $1$ on a single state-action pair $(s_h, b_h)$, and the conditional is well defined by the Markov property. Also, consider the following sequence, 
    \begin{align*}
        \Big\{\rho_h^t(\cdot)&=\PP^{\pi^t}(((s_l, a_l, b_l)_{l=h+1}^H , a_h)=\cdot \given s_h^t, b_h^t)\delta_{(s_h^t, b_h^t)}(\cdot) \nend
        &\qquad - \PP^{\pi^t}(((s_l, a_l, b_l)_{l=h+1}^H , a_h, b_h)=\cdot \given s_h^t)\delta_{(s_h^t)}(\cdot)\Big\}_{t\in[T]},
    \end{align*}
    and we also have $\rho_h^t\in\sP_{h, F}^1$.
    \item  
    In particular, we define $g_h^t(s_h^i, b_h^i, \pi^i)$ as the integral of $g_h^t$ with respect to $\rho_h^i$, which is given by
    $$g_h^t(s_h^i, b_h^i, \pi^i)= \rbr{\EE_{s_h^i, b_h^i}^{\pi^i}-\EE_{s_h^i}^{\pi^i}} \sbr{\sum_{l=h}^H \gamma^{l-h}\Bigrbr{\tilde r_l^t- r_l + \gamma (\tilde P_l^t - P_l) \tilde V_{l+1}^t}(s_l, a_l, b_l)},$$
    % One can check that $g_h^t\in\cG_F^1$ 
    Note that the sequence of signed measures is uniquely determined by $\{(s_h^t, b_h^t, \pi^t)\}_{t\in[T]}$. Moreover, we have $g_h^t(s_h^i, b_h^i, \pi^i)$ bounded by $\eff_H(\gamma)(2\nbr{r_h}_\infty + 2\nbr{V_{h+1}}_\infty) \le 4B_A \eff_H(\gamma)$, where we the definition of $B_A$ is available in \eqref{eq:define_BA};  
\end{itemize}
We define the maximal eluder dimension of $\cG_{h, F}^1$ with respect to $\sP_{h,F}^1$ as $$\dim(\cG_F^1) =\max_{h\in[H]} \dim_\DE(\cG_{h, F}^1,\sP_{h, F}^1, T^{-1/2}).$$
We first see what guarantee we have on the given sequences $\{g_h^t\}_{t\in[T]}$ and $\{(s_h^t, b_h^t, \pi^t)\}_{t\in[T]}$, 
\begin{align}
    &\sum_{i=1}^{t-1} \rbr{g_h^t(s_h^i, b_h^i, \pi^i)}^2 \nend
    & \quad =\sum_{i=1}^{t-1} \rbr{\rbr{\EE_{s_h^i, b_h^i}^{i}-\EE_{s_h^i}^{i}} \sbr{\sum_{l=h}^H \gamma^{l-h}\Bigrbr{\tilde r_l^t- r_l + \gamma (\tilde P_l^t - P_l) \tilde V_{l+1}^t}(s_l, a_l, b_l)}}^2 \nend
    & \quad \le 2\sum_{i=1}^{t-1} \rbr{\rbr{\EE_{s_h^i, b_h^i}^{i}-\EE_{s_h^i}^{i}} \sbr{\sum_{l=h}^H \gamma^{l-h}\Bigrbr{\tilde r_l^t- r_l + \gamma (\tilde P_l^t - P_l) V_{l+1}^{\pi^i, M^t}}(s_l, a_l, b_l)}}^2\nend
    &\qqquad + 2\sum_{i=1}^{t-1} \rbr{\rbr{\EE_{s_h^i, b_h^i}^{i}-\EE_{s_h^i}^{i}}\sbr{\sum_{l=h}^H \gamma^{l-h+1} (\tilde P_l^t - P_l) \bigrbr{V_{l+1}^{\pi^i, M^t} - \tilde V_{l+1}^t}(s_l, a_l,b_l)}}^2 \nend
    & \quad \le 2\sum_{i=1}^{t-1} \Biggrbr{\underbrace{
        \rbr{\EE_{s_h^i, b_h^i}^{i}-\EE_{s_h^i}^{i}} \sbr{\sum_{l=h}^H \gamma^{l-h}\Bigrbr{\tilde r_l^t- r_l + \gamma (\tilde P_l^t - P_l) V_{l+1}^{\pi^i, M^t}}(s_l, a_l, b_l)}
    }_{\ds\tilde\Delta^{(1)}_{h, \pi^i, M^t}(s_h, b_h)}}^2\nend
    &\qquad + \underbrace{C B_A^2 \eff_{H}(\gamma)\sum_{i=1}^{t-1} \rbr{\EE_{s_h^i, b_h^i}^i + \EE_{s_h^i}^i} \sbr{\sum_{l=h}^H \gamma^{l-h+1} D_\TV^2 \rbr{\tilde P_l^t(\cdot\given s_l, a_l,b_l), P_l(\cdot\given s_l, a_l,b_l)}}}_{\dr (i)}, \label{eq:OnN-1st-g-sq}
\end{align}
where the first inequaltiy holds by the Jensen's inequality, and the second inequality holds by upper bounding the difference $\tilde P_l^t - P_l$ by the TV distance. Here, we are able to use $B_A$ as the upper bound for the V functions by our discussion in \Cref{sec:app-notations}, and $C$ hides some universal constant.
Applying \Cref{lem:1st-ub} with $\tilde \Delta^{(1)}_h(s_h, b_h)$ replaced by
\begin{align*}
    \tilde \Delta^{(1)}_{h,\pi^i, M^t}(s_h, b_h) 
    &=  \rbr{\EE_{s_h, b_h}^{i} -\EE_{s_h}^{i}}\Biggsbr{\sum_{l=h}^H \gamma^{l-h} 
    {\rbr{Q_l^{\pi^i, M^t} - r_l^{\pi^i} - \gamma P_l^{\pi^i} V_{l+1}^{\pi^i, M^t}}(s_l, b_l)}}\nend
    & = \rbr{\EE_{s_h, b_h}^{i} -\EE_{s_h}^{i}}\Biggsbr{\sum_{l=h}^H \gamma^{l-h} 
    {\rbr{\tilde r_l^t - r_l + \gamma (\tilde P_l^t - P_l) V_{l+1}^{\pi^i, M^t}}(s_l, a_l, b_l)}},
\end{align*}
we obtain for all $t\in[T], h\in[H]$,
\begin{align}
    &\sum_{i=1}^{t-1}\bigrbr{\tilde \Delta_{h, \pi^i, M^t}^{(1)}(s_h^i, b_h^i)}^2  \nend
    &\quad \le 2 \sum_{i=1}^{t-1}\rbr{\rbr{\EE_{s_h^i, b_h^i}^i-\EE_{s_h^i}^i} \bigsbr{\orbr{Q_h^{\pi^i, M^t} -  Q_h^{\pi^i}}(s_h, b_h)}}^2 \nend
    &\qqquad + C \gamma^2  \rbr{\eta^{-1} +2 B_A}^2\eff_H(\gamma) \sum_{l=h+1}^H \gamma^{l-h-1} {\rbr{\EE_{s_h^i}^i+\EE_{s_h^i, b_h^i}^i}\sbr{D_\H^2(\nu_l^{\pi^i}(\cdot\given s_l), \nu_l^{\pi^i, M^t}(\cdot\given s_l))}}\nend
    &\quad\le  8 C_\eta^2 \beta  + 64 B_Q^2 \log\rbr{TH\cN_\infty(\cM, T^{-1})\delta^{-1}} + C \gamma^2  \rbr{\eta^{-1} +2 B_A}^2\eff_H(\gamma)^2  \beta \nend
    &\quad \le \cO\rbr{ (\eta^{-1}+2B_A)^2 \eff_H(\gamma)^2  \beta},\label{eq:OnN-1st-Delta1-sq}
\end{align}
where the second inequality holds by \eqref{eq:MLE-guarantee-Q-3} in \Cref{lem:MLE-formal}, and the covering number $\cN_\varrho(\cM, \epsilon)$ is with respect to the infinite norm of the Q function. Here, $\cO$ only hides universal constant independent of $H,\eta, T$. 
Meanwhile, for the TV distance term in \eqref{eq:OnN-1st-g-sq}, we have also by \Cref{lem:MLE} that 
\begin{align*}
    {\dr (i)}\le 8 B_V^2 \eff_{H}(\gamma){\sum_{l=h}^H \gamma^{l-h+1} 8\beta }\le \cO(B_A^2 \eff_H(\gamma)^2 \beta),
\end{align*}
where $\cO$ only hides some universal constants.
Hence, we conclude that 
\begin{align*}
    \sum_{i=1}^{t-1} \rbr{g_h^t(s_h^i, b_h^i, \pi^i)}^2  \le \cO\rbr{\rbr{\eta^{-1}+B_A}^2 \eff_H(\gamma)^2 \beta}.
\end{align*}
Now, for the first-order term, we have  
\begin{align*}
    &C^{(0)}
    \sum_{t=1}^T \sum_{h=1}^H{\EE^t\sbr{\abr{\tilde \Delta^{(1,t)}_h(s_h, b_h)}}}\nend
    &\quad \le 2 C^{(0)} \sum_{h=1}^H \underbrace{\sum_{t=1}^T \abr{\tilde\Delta^{(1, t)}_h(s_h^t, b_h^t)}}_{\ts \sum_{t=1}^T |g_h^t(s_h^t, b_h^t, \pi^t)|} + H C^{(0)} \cdot 4 \bignbr{\tilde \Delta^{(1)}_h}_\infty \log\rbr{H\cN_\rho(\cM, T^{-1})\delta^{-1}} \nend
    &\quad \le \cO\rbr{ HC^{(0)} \eff_H(\gamma)\sqrt{\dim(\cG_F^1)\rbr{\eta^{-1}+B_A}^2  \beta T }}\nend
    &\quad\le \cO\rbr{ H^2 \eff_H(\gamma)\rbr{1+\eta B_A} \sqrt{\dim(\cG_F^1)\beta T}},
\end{align*}
where the first inequality follows from a standard martingale concentration in \Cref{cor:martigale concentration}, and the second inequality holds by using the first order regret bound in \Cref{lem:de-regret}, and the last inequality holds by $C^{(0)}=2\eta H$.

\paragraph{Step 3. Bounding the Second-Order Term.}
Previously, we decompose the online regret and obtain a second-order term, which we referred to as (ii), 
\begin{align*}
    {\dr (ii)}\defeq C^{(2)}
    \sum_{t=1}^T \max_{h\in [H]} {\EE^t\sbr{ \rbr{\rbr{\tilde Q_h^t - r_h^{\pi^t} - \gamma P_h^{\pi^t} \tilde V_{h+1}^t}(s_h, b_h)}^2}}.
\end{align*}
For this term, we specify the function class to use for our purpose, 

\begin{itemize}[leftmargin =20pt]
    \item[(i)] We take the same function class $\cG_F^2$ as
    \begin{align*}
        \cG_{h, F}^2 &= \Bigg\{g:\cS\times\cA\times\cB\rightarrow \RR: \exists M\in\cM, h\in[H] \nend
        &\qqquad g(s_h, b_h, a_h) = {\bigrbr{r_h^M- r_h + \gamma (P_h^{M} - P_h) V_{l+1}^{\pi^M, M}}(s_h, a_h, b_h)}
        \Bigg\},
    \end{align*}
    where $\cG_F^2$ is bounded by $4B_A$. 
    \item[(ii)] We define a class of probability measures on $\cS\times\cA\times\cB$ as $$\sP_{h, F}^2 = \cbr{\PP^\pi(a_h=\cdot\given s_h, b_h)\delta_{(s_h, b_h)}(\cdot)\given \pi\in\Pi, (s_h,b_h)\in\cS\times\cB}, $$
    where $\delta_{(s_h,b_h)}(\cdot)$ is the measure that assigns $1$ to the state-action pair $(s_h, b_h)$. 
    \item[(iii)] We take a sequence of functions $\{g_h^t\}_{t\in[T]}$ as $\{g_h^t = \tilde r_h^{t}- r_h + \gamma (\tilde P_h^t - P_h) \tilde V_{l+1}^{t}\}_{t\in[T]}$, and take a sequence of probability measures as $\{\rho_h^t(\cdot) = \PP^{\pi^t}(a_h=\cdot\given s_h^t, b_h^t)\delta_{(s_h^t, b_h^t)}(\cdot)\}_{t\in[T]}$, where we define $\tilde r_h^t = r_h^{M^t}$ and $\tilde P_h^t = P_h^{M^t}$.
    One can check that $g_h^t\in\cG_{h,F}^1$ since $\tilde V_h^t = V_h^{\pi^t, M^t}$ and we have $\pi^t = \argmax_{\pi\in\Pi}J(\pi, M^t) = \pi^{M^t}$. In addition, we define $g_h^t(s_h^i, b_h^i, \pi^i)$ as the integral of $g_h^t$ with respect to $\rho_h^i$, which is given by
    \begin{align*}
        g_h^t (s_h^i, b_h^i, \pi^i) = \EE_{s_h^i, b_h^i}^{\pi^i} \sbr{\bigrbr{\tilde r_h^{t}- r_h + \gamma (\tilde P_h^t - P_h) \tilde V_{l+1}^{t}}(s_h, a_h, b_h)},
    \end{align*}
    Note that the sequence of probability measures is uniquely determined by $\{(s_h^t, b_h^t, \pi^t)\}_{t\in[T]}$.
\end{itemize}
We let $\dim(\cG_F^2)=\max_{h\in[H]}\dim_\DE(\cG_{h,F}^2,\sP_{h, F}^2, T^{-1/2})$ be the eluder dimension.
We next establish guarantee for $\sum_{i=1}^{t-1} (g_h^t(s_h^i, b_h^i, \pi^i))^2$.
\begin{align}\label{eq:OnN-guarantee-2ndQRE}
    &\sum_{i=1}^{t-1} (g_h^t(s_h^i, b_h^i, \pi^i))^2 \nend
    &\quad = \sum_{i=1}^{t-1} \rbr{\EE_{s_h^i, b_h^i}^{i} \sbr{\bigrbr{\tilde r_h^{t}- r_h + \gamma (\tilde P_h^t - P_h) \tilde V_{h+1}^{t}}(s_h, a_h, b_h)}}^2\nend
    &\quad  \le 2\sum_{i=1}^{t-1} \rbr{\EE_{s_h^i, b_h^i}^{i} \sbr{\bigrbr{\tilde r_h^{t}- r_h + \gamma (\tilde P_h^t - P_h) V_{h+1}^{\pi^i, M^t}}(s_h, a_h, b_h)}}^2 \nend
    &\qqquad + 2\cdot 16B_A^2 \sum_{i=1}^{t-1} \rbr{\EE_{s_h^i, b_h^i}^i\sbr{D_\TV^2\rbr{\tilde P_h^t(\cdot\given s_h, a_h, b_h), P_h(\cdot\given s_h, a_h, b_h)}}}\nend
    &\quad \lesssim 3 \underbrace{\sum_{i=1}^{t-1}\EE^i\rbr{\EE_{s_h, b_h}^{i} \sbr{\bigrbr{\tilde r_h^{t}- r_h + \gamma (\tilde P_h^t - P_h) V_{h+1}^{\pi^i, M^t}}(s_h, a_h, b_h)}}^2}_{\dr (iii)} + 32 B_A^2 \log\rbr{TH\cN_\rho(\cM,T^{-1}\delta^{-1})}\nend
    &\qqquad + 48 B_A^2 \underbrace{\sum_{i=1}^{t-1} \EE^i\sbr{D_\TV^2\rbr{\tilde P_h^t(\cdot\given s_h, a_h, b_h), P_h(\cdot\given s_h, a_h, b_h)}}}_{\dr (iv)} + 64 B_A^2 \log\rbr{TH\cN_\rho(\cM, T^{-1})}
\end{align}
where in the first inequality, we use $2B_A$ to upper bound $\|\tilde V_{h+1}^t - V_{h+1}^{\pi^i, M^t}\|_\infty$, and uses the Cauchy-Schwartz inequality to move the square into the expectation.
Here, the second inequality uses a standard martingale concentration result in \Cref{cor:martigale concentration} for both terms, and we invoke the same upper bound $B_A$ for the V functions.
Now, we invoke \Cref{lem:2nd-ub}, which says that term (iii) enjoys the following upper bound, 
\begin{align*}
    {\dr (iii)} &=\sum_{i=1}^{t-1}\EE^i\rbr{\EE_{s_h, b_h}^{i} \sbr{\bigrbr{\tilde r_h^{t}- r_h + \gamma (\tilde P_h^t - P_h) V_{h+1}^{\pi^i, M^t}}(s_h, a_h, b_h)}}^2 \nend
    & = \sum_{i=1}^{t-1}\EE^i\rbr{{\bigrbr{Q_h^{\pi^i, M^t} - r_h^{\pi^i} - \gamma P_h^{\pi^i} V_{h+1}^{\pi^i, M^t}}(s_h, a_h, b_h)}}^2\nend
    &\le L^{(2)} \sum_{i=1}^{t-1} \max_{h\in[H]} \cbr{\EE^i D_\H^2\bigrbr{\nu_h^{\pi^i, M^t}(\cdot\given s_h), \nu_h^{\pi^i}(\cdot\given s_h)}+ \EE^i D_\TV^2\bigrbr{P_h^{\pi^i, M^t}(\cdot\given s_h, b_h), P_h^{\pi^i}(\cdot\given s_h, b_h)}}\nend
    &\le L^{(2)} 4H \beta , 
\end{align*}
where the last inequality uses the guarantee in \Cref{lem:MLE}. To enable a direct use of the MLE guarantee, we replace the maximum by a sum over all $h\in[H]$ and upper bound the TV distance by the Hellinger distance. Here, $L^{(2)}$ is defined in \Cref{lem:2nd-ub}. For term (iv), we use the same guarantee in \Cref{lem:MLE} and the same bounding the TV distance by the Hellinger distance argument and obtain ${\dr (iv)}\le 4\beta$. Therefore, we conclude that 
\begin{align*}
    \sum_{i=1}^{t-1} (g_h^t(s_h^i, b_h^i, \pi^i))^2 \lesssim {\bigrbr{L^{(2)} H + B_A^2} \beta},
\end{align*}
where $\lesssim$ only hides some universal constant. As a result of the second order regret in \Cref{lem:de-regret}, we have 
\begin{align*}
    \sum_{t=1}^{T} (g_h^t(s_h^t, b_h^t, \pi^t))^2 
    &\lesssim {\dim(\cG_F^2) \bigrbr{L^{(2)} H + B_A^2} \beta  + \min\cbr{T, \dim(\cG_F^2)} 16 B_A^2+1}\nend
    &\lesssim {H \dim(\cG_F^2) \beta  L^{(2)}},
\end{align*}
where $\lesssim$ hides some universal constant. Therefore, we conclude that
\begin{align*}
    {\dr (ii)}
    &\defeq C^{(2)}
    \sum_{t=1}^T \max_{h\in [H]} {\EE^t\sbr{ \rbr{\rbr{\tilde Q_h^t - r_h^{\pi^t} - \gamma P_h^{\pi^t} \tilde V_{h+1}^t}(s_h, b_h)}^2}}\nend
    &\le C^{(2)} \sum_{t=1}^T \sum_{h=1}^H \EE^t \sbr{ \rbr{\rbr{\tilde Q_h^t - r_h^{\pi^t} - \gamma P_h^{\pi^t} \tilde V_{h+1}^t}(s_h, b_h)}^2} \nend
    & \le 2C^{(2)} \sum_{t=1}^T \sum_{h=1}^H \EE^t_{s_h^t, b_h^t} \sbr{ \rbr{\rbr{\tilde Q_h^t - r_h^{\pi^t} - \gamma P_h^{\pi^t} \tilde V_{h+1}^t}(s_h, b_h)}^2} \nend
    &\qquad + 4 H C^{(2)} 9 B_A^2 \log\rbr{H \cN_\rho(\cM,T^{-2})\delta^{-1}}. 
\end{align*}
where in the first inequality, we replace the maximum by the summation over $h\in[H]$ and in the second inequality, we invoke the martingale concentration in \Cref{cor:martigale concentration}.
Hence, we establish our bound for the second-order term as
\begin{align*}
    {\dr(ii)} 
    &\le 2 C^{(2)} \sum_{h=1}^H \sum_{t=1}^T \rbr{g_h^t(s_h^t, b_h^t, \pi^t)}^2 + 4 H C^{(2)} 9 B_A^2 \log\rbr{H \cN_\rho(\cM,T^{-2})\delta^{-1}}\nend
    &\lesssim {H^2 C^{(2)} \dim(\cG_F^2) \beta  L^{(2)}} +   { H C^{(2)} \beta}
    % \nend
    % &\le \cO\rbr{H^6\eff_H(\gamma)\rbr{\eff_H(\exp(2\eta B_A)\gamma)}^2 (\iota^{2H}+1) \kappa^2 \exp\rbr{14\eta B_A} (1+\eta^2B_A^2)  \dim(\cG_F^2) \beta }
\end{align*}
where $L^{(2)} = c H^2 \eff_H(c_2)^2 \kappa^2 \exp\rbr{8\eta B_A} C_\eta^2$ with 
$c_2 = \gamma(2\exp(2\eta B_A)+\kappa\exp(4\eta B_A))$, and $C^{(2)} = 2 H^2\eta^2 (1+4 \eff_H(\gamma))\exp\rbr{6\eta B_A}\cdot \rbr{\eff_H(\exp(2\eta B_A)\gamma)}^2$.

In summary, for the leader's Bellman error, we have
\begin{align*}
    \text{LBE}\lesssim H(2\sqrt{\dim\rbr{\cG_L} H^2 \beta T} + 3 H\min\cbr{T, \dim\rbr{\cG_L}}  + \sqrt T) \lesssim H^2 \sqrt{\dim\rbr{\cG_L} \beta T}, 
\end{align*}
for the first-order term in the follower's QRE,
\begin{align*}
    {\text{1st-QRE}}\lesssim  H^2 \eff_H(\gamma) \eta C_\eta \sqrt{\dim(\cG_F^1)\beta T}, 
\end{align*}
where $C_\eta = \eta^{-1}+B_A$,
and for the second-order term in the follower's QRE, 
\begin{align*}
\text{2nd-QRE}\lesssim H^2 C^{(2)} L^{(2)} \dim(\cG_F^2) \beta \log T
\end{align*}
which completes the proof for \Cref{thm:OMLE-farsighted}.

% Specifically, for any leader's policy $\pi\in\Pi$, the true model $M^*$ and an alternative model $\tilde M\in\cM$, we have
% \begin{align}
%     &J(\pi, \tM) - J(\pi, M^*) \nend
%     &\quad = {\sum_{h=1}^H \EE^{\pi, M^*}\sbr{\tilde U_h(s_h, a_h, b_h) - u_h(s_h, a_h, b_h)- \tilde W_{h+1}(s_{h+1})}} + \sum_{h=1}^H \EE^{\pi, M^*}\sbr{\tilde W_h(s_h)-\tilde U_h(s_h, a_h, b_h)}\nend
%     &\quad\le\underbrace{{\sum_{h=1}^H \EE^{\pi, M^*}\sbr{\tilde U_h(s_h, a_h, b_h) - u_h(s_h, a_h, b_h)- \tilde W_{h+1}(s_{h+1})}}}_{\dr \text{temporal difference}}+ \underbrace{\sum_{h=1}^H  H \EE \nbr{\tnu_h(\cdot\given s_h)-\nu_h(\cdot\given s_h)}_1}_{\dr \text{response difference}}, \label{eq:performance diff}
% \end{align}
% where $\tilde U_h, \tilde W_h, \tilde\nu_h$ are the correspondences of $U_h, W_h, \nu_h$ under model $\tilde M$.
% In the following, we denote the above $h$-step temporal difference by $\cE^{(0)}_h(\tM, M^*;\pi)$, which is a standard result from classical MDP.
% We call the second term response difference since the second term characterize the TV distance between the follower's true response and an estimated response learned by the leader.
% Here, we have to invoke the TV bound since the follower's utility is not aligned with the leader's. 
% A unique challenge of our problem is to characterize the response difference in the face of a strategic follower.
% The following lemma states that the  follower's response difference admits a \say{Taylor expansion} flavored decomposition. 
% \begin{lemma}[\textit{Response difference for farsighted follower}]
%     % \label{lem:performance diff}
% For any given leader's policy $\pi$, the true model $M^*$ and an alternative model $\tM\in\cM$, using the denotions specified in \Cref{tab:notation}, the response difference term in \eqref{eq:performance diff} can be upper bounded by
% \begin{align}
%     &\sum_{h=1}^H  H \EE \nbr{\tnu_h(\cdot\given s_h)-\nu_h(\cdot\given s_h)}_1 \nend
%     &\quad \le 2\eta  H \cdot 
%     \sum_{h=1}^H \underbrace{\EE^{\pi, M^*}\sbr{\abr{\rbr{\EE^{\pi, M^*}_{s_h, b_h}-\EE^{\pi, M^*}_{s_h}}\bigsbr{\Delta^{(1)}_{h,\pi, \tM}(s_h, b_h)}}}}_{\ds \cE^{(1)}_h(\tM, M^*;\pi)} \nend
%     & \qqquad + \underbrace{\eta^2  H  \rbr{1+ 4 \frac{1-\gamma^H}{1-\gamma}}}_{\ds C^{(1)}} \cdot 
%     \sum_{h=1}^H \underbrace{\EE^{\pi, M^*}\sbr{ \exp\bigrbr{\eta\bigabr{A_h-\tilde A_h}}\cdot \bigabr{\tilde A_h - A_h}^2}}_{\ds \cE^{(2)}_h(\tM, M^*;\pi)}, \label{eq:response decomposition}
% \end{align}
% where $ H = \max_{h\in[H], \pi\in\Pi, M\in\cM}\bignbr{U_h^{\pi, M}}_\infty$ and $\Delta^{(1)}_{h,\pi, M}(s_h, b_h)$ is defined as
% \begin{align}
%     \Delta^{(1)}_{h,\pi, \tM}(s_h, b_h) &\defeq  \EE_{s_h, b_h}\sbr{\sum_{i=h}^H \gamma^{i-h}\rbr{r_i^\pi(s_h, b_h)-\tilde r_i^\pi(s_h, b_h) + \gamma \rbr{\bigrbr{\TT_i^\pi -\tilde\TT_i^\pi}\tilde V_{i+1}}(s_h, b_h)}}.\label{eq:def Delta^1}
% \end{align}
% \begin{proof}
%     See \Cref{sec:performance diff} for a detailed proof.
% \end{proof}
% \end{lemma}

% Therefore, the online regret can be reorganized into
% \begin{align*}
%     \Reg(T)\le \sum_{t=1}^T \Biggrbr{\underbrace{\sum_{h=1}^H\cE^{(0)}_h\bigrbr{M^t, M^*;\pi^t} + 2\eta H\sum_{h=1}^H \cE^{(1)}_h\bigrbr{M^t, M^*;\pi^t}}_{\ds I_t: \text{1st-order terms}} + \underbrace{C^{(1)} \sum_{h=1}^H \cE^{(2)}_h\bigrbr{M^t, M^*;\pi^t}}_{\ds J_t: \text{ 2nd-order term}}}.
% \end{align*}
% Here, we notice that $\cE^{(0)}$ and $\cE^{(1)}$ are first order terms in the sense that both $\cE^{(0)}$ and $\cE^{(1)}$ have a linear form while $\cE^{(2)}$ is the second order term and depends quadratically on $|\tilde A_h - A_h|$. In the sequel, we aim to bound them separately.

% \paragraph{Step 2. Bounding the First Order Terms.}
% A key point for bounding the first order terms is relating the guarantee of MLE in \Cref{lem:MLE} to the first order error $\cE^{(0)}, \cE^{(1)}$, which is done by a characterization in terms of the Eluder dimensions.
% % \begin{lemma}[\textit{Eluder dimension bound, Lemma 41 in \citet{jin2021bellman}}] \label{lem:1st-eluder}
% %     Given a function class $\cF$ defined on $\cX$ with $\abr{f(x)}<C$ for all $(f,x)\in \cF\times \cX$. Suppose sequences $\{f_t\}_{t=1}^T$ and $\{x_t\}_{t=1}^T$ satisfy that for all $t\in[T]$, $\sum_{j=1}^{t-1} f_j(x_t)^2\le \beta$. Then for all $t\in[T]$ and $w>0$, we have the first order guarantee
% %     \begin{align*}
% %         \sum_{j=1}^t \abr{f_j(x_j)} \le 2\sqrt{\dimE(\cF, \cX, w)\beta t} + \min\cbr{t, \dimE(\cF, \cX, w)} C + t w, 
% %     \end{align*}
% %     and also the second order guarantee
% %     \begin{align*}
% %         \sum_{j=1}^t f_{j}(x_j)^2 \le \dimE(\cF, \cX, w) \beta \log t + \min\cbr{t, \dimE(\cF, \cX, w)} C^2 + t w^2.
% %     \end{align*}
% %     \begin{proof}
% %         \todo{to be added.}
% %     \end{proof}
% % \end{lemma} 
% Note that \Cref{lem:1st-eluder} is slightly different from the original version in the sense that we include a second order guarantee, which is an adaptation of the original proof. 
% Recall that 
% \begin{align*}
%     \cE^{(0)}_h(\tilde M, M^*;\pi) &= \EE^{\pi, M^*}\sbr{\tilde U_h(s_h, a_h, b_h) - u_h(s_h, a_h, b_h)- \tilde W_{h+1}(s_{h+1})}\nend
%     & =  \EE^{\pi, M^*}\sbr{\tilde u_h  - u_h + \rbr{\tilde \TT_h - \TT_h} \tilde W_{h+1}} ,
% \end{align*}
% and by definition of $\Delta^{(1)}$ in \eqref{eq:def Delta^1} that
% \begin{align*}
%     \cE^{(1)}_h(\tilde M, M^*;\pi)
%     &=\EE^{\pi, M^*}\sbr{\abr{\rbr{\EE^{\pi, M^*}_{s_h, b_h}-\EE^{\pi, M^*}_{s_h}}\bigsbr{\Delta^{(1)}_{h,\pi, \tM}(s_h, b_h)}}}\nend
%     &= \EE^{\pi, M^*}\abr{\sum_{i=h}^H\gamma^{i-h}\rbr{\EE_{s_h, b_h}^{\pi, M^*} - \EE_{s_h}^{\pi, M^*}}\sbr{\tilde r_i-r_i+\gamma \rbr{\tilde\TT_i - \TT_i}\tilde V_{i+1}}}.
% \end{align*}
% Now, the task boils down to finding a $f$ such that: (i) the first order terms $\cE^{(0)}$ and $\cE^{(1)}$ are bounded by this $f$; (ii) $\sum_{j=1}^{t-1} f_j^2\le \cO(\beta)$ can be guaranteed by the MLE.
% Following this spirit, we construct $f_{0,h}^M, f_{1, h}^M$ as
% \begin{align*}
%     f_{0,h}^M(\pi) &\defeq  \EE^{\pi, M^*}\sbr{u_h^M  - u_h^{M^*} + \rbr{\TT_h^M - \TT_h^{M^*}} W_{h+1}^{\pi_\opt^M, M} }, \nend
%     f_{1,h}^M(\pi) & \defeq \EE^{\pi, M^*}\abr{\sum_{i=h}^H\gamma^{i-h}\rbr{\EE_{s_h, b_h}^{\pi, M^*} - \EE_{s_h}^{\pi, M^*}}\sbr{ r_i^M-r_i^{M^*}+\gamma \rbr{\TT_i^M - \TT_i^{M^*}}V_{i+1}^{\pi_\opt^M, M}}}, 
% \end{align*}
% where $\pi^M_\opt$ is the optimistic policy for the leader under model $M$.
% One can easily check that $f_{0,h}^{M^t}(\pi^t)= \cE^{(0)}_h\bigrbr{M^t, M^*;\pi^t}$ for the temporal difference error and $f_{1,h}^{M^t}(\pi^t) = \cE^{(1)}_h\bigrbr{M^t, M^*;\pi^t}$ for the first order term in the response difference error. We would like to point out that the definitions of $f_0$ and $f_1$ are different from $\cE^{(0)}, \cE^{(1)}$, and the above equality relationship between the function $f$ and the error $\cE$ only holds if $\pi^t=\pi_\opt^{M^t}$, which makes $\pi^t$ a function of $M^t$.
% In the sequel, let $\cF_{0,h}=\{f_{0,h}^M(\cdot):M\in\cM\}$ and $\cF_{1,h}=\{f_{1,h}^M(\cdot):M\in\cM\}$.
% % Now that the first condition that $f$ bounds the error at each round $t$ holds.
% For the second condition of \Cref{lem:1st-eluder} that $\sum_{j=1}^{t-1} f^2_j(x_t)\le \cO(\beta)$, we can check that
% \begin{align*}
%     f_{0,h}^{M}(\pi) &\le \EE\sbr{\abr{u_h^M-u_h^{M^*}} + 2 H D_\TV \rbr{T_h^M, T_h^{M^*}}} 
%     % &\le H \sum_{h=1}^H \EE\sbr{\rbr{u_h^M-u_h^{M^*}}^2 + 4 H^2 D_\H^2\rbr{T_h^M, T_h^{M^*}}}\nend
%     \le (2 H +1) D_{\RL,h}(M, M^*;\pi).
% \end{align*}
% where 
% % the first inequality holds by the Cauchy-Schwartz inequality, and
% the second inequality holds by noting that the TV distance is upper bounded by the Hellinger distance used in $D_\RL$. 
% Therefore, we have $\sum_{j=1}^{t-1}f_{0,h}^{M^t}(\pi^j)^2 \le (2H+1)^2 \sum_{j=1}^{t-1} D_{\RL,h}^2(M^t, M^*;\pi^j)\le 4(2H+1)^2 \beta$ by \Cref{lem:MLE}.
% The case for $f_1$ is more complicated since we have to control the follower's reward difference via the follower's behavior difference. We invoke the following lemma.
% % \begin{lemma}[\textit{Bounding $f_1$ by $D_\RL$}]\label{lem:1st-ub}
% % For any $\eta>0$, $\pi\in\Pi$ and $\tM\in\cM$, we have
% % \begin{align*}
% %     f_{1,h}^{\tM}(\pi) \le   \underbrace{6(1+2\eta B_A)\cdot
% %     \sqrt{C_{\gamma, H}}}_{\ds L^{(1)}}  \cdot
% %     \eta^{-1}\max_{h\in[H]} \EE^{\pi, M^*} D_\TV(\nu_h,\tilde\nu_h), 
% % \end{align*}
% % where $C_{\gamma, H} = \orbr{1-\gamma^H}/\orbr{1-\gamma}$ is the effective foresight of the follower.
% %     \begin{proof}
% %         See \Cref{sec:1st-ub} for a detailed proof.
% %     \end{proof}
% % \end{lemma}
% We see from \Cref{lem:1st-ub} that $\sum_{j=1}^{t-1} f_{1,h}^{M^t}(\pi^j)^2\le \eta^{-2}\rbr{L^{(1)}}^2 \sum_{j=1}^{t-1} D_{\RL,h}^2(M^t, M^*;\pi^j)\le 4\eta^{-2}(L^{(1)})^2 \beta$. Hence, $f_2$ also satisifies the condition in \Cref{lem:1st-eluder}.
% Combining the discussions for $f_0$ and $f_1$ and invoking \Cref{lem:1st-eluder} together with $\sum_{s<t}D_\RL^2(M^t, M^*;\pi^s)\le 4\beta$ in \Cref{lem:MLE}, we have (for $w=1/\sqrt T$ in \Cref{lem:1st-eluder}) that
% \begin{align*}
%     \sum_{t=1}^T I_t &\le  H\rbr{2\sqrt{\dimE(\cF_0, \Pi, 1/\sqrt T) (2H+1)^2 \cdot 4\beta \cdot T} + \min\cbr{T, \dimE(\cF_0, \Pi, 1/\sqrt T)} B_{\cF_0} + \sqrt T}\nend
%     &\qquad + 2\eta H  H \cdot \Big( 2L^{(1)} \eta^{-1}\sqrt{\max_h\dimE(\cF_{1,h}, \Pi, 1/\sqrt T)  \cdot 4\beta \cdot T} \nend
%     &\qquad + \min\cbr{T, \max_h\dimE(\cF_{1,h}, \Pi, 1/\sqrt T)} B_{\cF_1} + \sqrt T \Big)\nend
%     &\le 
%     \rbr{ 4H (2H+1) \sqrt{5 d_0 \beta} + 8 H  H L^{(1)} \sqrt{d_1\beta} + 2} \sqrt{T} + \min\cbr{T, d_1, d_2} \rbr{B_{\cF_0} + 2\eta H  H B_{\cF_1}}, 
% \end{align*}
% where we let $d_0 = \max_{h\in[H]}\dimE(\cF_{0,h}, \Pi, 1/\sqrt T)$, $d_1=\max_h\dimE(\cF_{1,h}, \Pi, 1/\sqrt T)$, $B_{\cF_0}=\max_{f\in\cF, \pi\in\Pi}|f_0(\pi)|$, and $B_{\cF_1}=\max_{h\in[H],f\in\cF, \pi\in\Pi}|f_{1,h}(\pi)|$. Keeping the dominant terms for simplicity, we have
% \begin{align*}
%     \sum_{t=1}^T I_t\le \cO\rbr{H^{2} \sqrt{d_0\beta T} + (1+\eta B_A) C_{\gamma, H} H^2\sqrt{d_1\beta T}}.
% \end{align*}
% \paragraph{Step 3. Bounding the 2nd-Order Term.}
% Recall the second order error
% \begin{align*}
%     \cE_h^{(2)}(\tilde M, M^*, \pi) = \EE^{\pi, M^*}\sbr{ \exp\bigrbr{\eta\bigabr{A_h-\tilde A_h}}\cdot \bigabr{\tilde A_h - A_h}^2}\le \exp\bigrbr{2\eta B_A}\cdot \EE^{\pi, M^*}\sbr{ \bigabr{\tilde A_h - A_h}^2}.
% \end{align*}
% For the second order term, suppose that we can take some $f_{2,h}^M(\pi)$ such that $f_{2,h}^M(\pi)\ge \EE^{\pi, M^*}\bigabr{A_h^{\pi, M}-A_h^{\pi,M^*}}$ and $f_{2,h}^M(\pi)\le L^{(2)} D_\RL(M, M^*;\pi)$ for some $L^{(2)}>0$.
% Following the second result in \Cref{lem:1st-eluder}, we let $w=1/\sqrt T$ and obtain
% \begin{align*}
%     \sum_{t=1}^T J_t &\le C^{(1)} \exp\rbr{2\eta B_A}H \cdot \Big(4L^{(2)}\max_h\dimE(\cF_{2, h}, \cX, 1/\sqrt T) \beta \log T\nend 
%     &\qquad + \min\cbr{T, \max_h\dimE(\cF_{2 ,h}, \cX, 1/\sqrt T)} B_{\cF_{2}}^2 + 1\Big)\nend
%     &\le \rbr{4 L^{(2)} d_2 \beta \log T + \max\cbr{T, d_2} B_{\cF_2}^2 + 1} 5 \eta^2  H   \frac{1-\gamma^H}{1-\gamma} H\exp\rbr{2\eta B_A}, 
% \end{align*}
% where $d_2=\max_h\dimE(\cF_{2,h}, \Pi, 1/\sqrt T)$, $B_{\cF_1}=\max_{h\in[H],f\in\cF, \pi\in\Pi}|f_{2,h}(\pi)|$. Note that the second order term only scales with $\log(T)$. Hence, the second order term is not significant as $T$ grows large. For simplicity, we have
% \begin{align*}
%     \sum_{t=1}^T J_t\le \cO\rbr{\eta^2 H^2 C_{\gamma, H} \exp\rbr{2\eta B_A} L^{(2)} d_2 \beta \log T}.
% \end{align*}
% Combining the results from the first order terms and the second order term, we finish our proof of \Cref{thm:OMLE-farsighted}.





\subsection{Proof of \Cref{lem:MLE} on the Guarantee of MLE}\label{sec:proof-MLE}
The following proof mainly follows the proof of Theorem E.1 in \citet{chen2022unified}.
Recall the MLE function is defined as
\begin{align*}
    \cL^t_h(M)&\defeq -\sum_{i=1}^{t-1}  \bigg(\eta \rbr{Q_{h}^{\pi^i, M}(s_h^i, b_h^i)-V_{h}^{\pi^i, M}(s_h^i)} + \log P_h^M(s_{h+1}^i\given s_h^i, a_h^i, b_h^i)  \nend
    &\qquad - \rbr{u_{h}^i - u_{h}^M(s_h^i, a_h^i, b_h^i)}^2\bigg)\nend
    & = -\sum_{i=1}^{t-1} \rbr{\log \nu_h^{\pi^i, M}(b_h^i\given s_h^i) + \log P_h^M(s_{h+1}^i\given s_h^i, a_h^i, b_h^i) - \rbr{u_{h}^i - u_{h}^M(s_h^i, a_h^i, b_h^i)}^2},
\end{align*}
and the RL distance $D_{\RL, h}$ is defined as
\begin{align*}
    D_{\RL,h}^2 (M,  M^*;\pi) =   \EE^{\pi, M^*}D_\H^2\rbr{\nu_h^{\pi,M}, \nu_h^{\pi,M^*}} +
    \EE^{\pi, M^*} D_\H^2(P_h^{M}, P_h^{M^*}) +
    \EE^{\pi, M^*}\rbr{u_h^{M^*}-u_h^M}^2,
\end{align*}
We take an $\epsilon$-optimistic covering net of $\cM$ and denote the covering number as $\cN_\rho(\cM,\epsilon)$
By \Cref{lem:freeman-variation}, and take the filtration as $\sF_{i-1}=\sigma(\tau^{1:i-1})$, we have with probability at least $1-\delta/3H$, for all $t\in[T]$ and $M$ belonging to a $\epsilon$-covering net $\cM_\epsilon$  that
\begin{align}
    \frac{1}{2} \sum_{i=1}^{t-1} \log \frac{\nu_h^{\pi^i, M}}{\nu_h^{\pi^i, M^*}} 
    &\le \sum_{i=1}^{t-1} \log \EE^{\pi^i}\sbr{\exp\rbr{ \frac{1}{2}\cdot \log \frac{\nu_h^{\pi^i, M}}{\nu_h^{\pi^i, M^*}}}} + \log\rbr{\frac{3H\cN_\rho(\cM,\epsilon)}{\delta}}\nend
    & \le -\sum_{i=1}^{t-1} \EE^{\pi^i}\sbr{1 -  \sqrt{\frac{\nu_h^{\pi^i, M}}{\nu_h^{\pi^i, M^*}}}} + \log\rbr{\frac{3H\cN_\rho(\cM,\epsilon)}{\delta}}\nend
    &= -\frac{1}{2} \sum_{i=1}^{t-1} \EE^{\pi^i}D_\H^2\rbr{\nu_h^{\pi^i,M}(\cdot\given s_h), \nu_h^{\pi^i,M^*}(\cdot\given s_h)} + \log\rbr{\frac{3H\cN_\rho(\cM,\epsilon)}{\delta}}, \label{eq:MLE-nu}
\end{align}
where the first inequality is a direct result of \Cref{lem:freeman-variation}, the second inequality holds by using the inequality $\log(x)\le x-1$, and the last equality holds by the definition of the Hellinger distance.
Following the same argument, we have with probability at least $1-\delta/3H$ and for all $t\in[T]$, $M\in\cM_\epsilon$ that
\begin{align}
    \frac{1}{2}\sum_{i=1}^{t-1} \log \frac{P_h^M}{P_h^{M^*}} \le -\frac{1}{2} \sum_{i=1}^{t-1} \EE^{\pi^i} D_\H^2(P_h^{M}, P_h^{M^*}) + \log\rbr{\frac{3H\cN_\rho(\cM,\epsilon)}{\delta}}.\label{eq:MLE-T}
\end{align}
For the reward, we view $u_h^i$ as a random variable and have with the same argument that it holds with probability at least $1-\delta/3H$ that
\begin{align}
    &\frac 1 3 \sum_{i=1}^{t-1} \rbr{\rbr{u_h^i - u_h^{M^*}}^2 - \rbr{u_h^i - u_h^{M}}^2}\nend
    &\quad \le \sum_{i=1}^{t-1} \log \EE^{\pi^i}\sbr{\exp\rbr{\frac 1 3\rbr{\rbr{u_h^i - u_h^{M^*}}^2 - \rbr{u_h^i - u_h^{M}}^2}}}+\log\rbr{\frac{3H\cN_\rho(\cM,\epsilon)}{\delta}}\nend
    &\quad \le -\sum_{i=1}^{t-1} \frac 1 9 \EE^{\pi^i}\rbr{u_h^{M^*}-u_h^M}^2 + \log\rbr{\frac{3H\cN_\rho(\cM,\epsilon)}{\delta}},\label{eq:MLE-u}
\end{align}
where the second inequality holds by using the inequality $\EE[\exp(\lambda ((r-\EE r)^2-(r - \hat r)^2))]\le \exp\bigrbr{-\lambda(1-2\sigma^2\lambda)\rbr{\EE r - \hat r}^2}$ for any $\lambda\in\RR$, $\hat r\in\RR$ and $\sigma^2$-sub-Gaussian random variable $r$. Here, we notice that $\sup_{h\in[H]}\nbr{u_h}_\infty\le 1$ by our model assumption.
Suppose that for $M\in\cM$, the nearest point to $M$ in the $\epsilon$-covering net is $\tilde M$. 
Note that for the pair $(M, \tilde M)$, we have
\begin{align}\label{eq:D_RL-covering bound}
    &\sup_{\pi\in\Pi, h\in[H]}\abr{D_{\RL,h}^2(M, M^*;\pi) - D_{\RL,h}^2(\tilde M, M^*;\pi)}\nend
    &\quad \le 2\sup_{\pi\in\Pi, h\in[H]}\EE^{\pi, M^*}\sbr{D_\H\rbr{\nu_h^{\pi, M}, \nu_h^{\pi, \tilde M}} + D_\H\rbr{P_h^M, P_h^{\tilde M}} + \abr{u_h^{M}-u_h^{\tilde M}}}\nend
    &\quad \le 6\epsilon, 
\end{align}
where the inequality holds by noting that the Hellinger distance satisfies the triangle inequality and the fact that $D_\H(\cdot,\cdot)\le 1$ and $|u_h|\le 1$. 
Moreover, for the negative log-likelihood function, 
\begin{align}\label{eq:cL-covering bound}
    \cL_h^t(\tilde M) - \cL_h^t(M) &= \sum_{i=1}^{t-1} \rbr{\log \frac{\nu_h^{\pi^i, M}(b_h^i\given s_h^i)}{\nu_h^{\pi^i, \tilde M}(b_h^i\given s_h^i)} + \log \frac{P_h^{M}(s_{h+1}^i\given s_h^i, a_h^i, b_h^i)}{P_h^{\tilde M}(s_{h+1}^i\given s_h^i, a_h^i, b_h^i)}} \nend
    &\qquad + \sum_{i=1}^{t-1} \bigrbr{u_{h}^i - u_{h}^{\tilde M}(s_h^i, a_h^i, b_h^i)}^2 - \rbr{u_{h}^i - u_{h}^M(s_h^i, a_h^i, b_h^i)}^2\nend
    &\le \eta T \sup_{\pi\in\Pi, h\in H} \bignbr{A_h^{\pi,M} - A_h^{\pi,\tilde M}}_\infty + T \log(\exp(\epsilon)) + 2T \sup_{ h\in H} \bignbr{u_h^{M} - u_h^{\tilde M}}_\infty\nend
    &\le 4 T, 
\end{align}
where the first inequality holds by noting the optimistic covering condition $P_h^{M}(s_{h+1}\given s_h, a_h, b_h)\le \exp(\epsilon)P_h^{\tilde M} (s_{h+1}\given s_h, a_h, b_h)$.
Now, using \eqref{eq:MLE-nu}, \eqref{eq:MLE-T}, and \eqref{eq:MLE-u}, we conclude that with probability at least $1-\delta$, for all $M\in \cM$ and $h\in[H]$,
\begin{align*}
    &\sum_{i=1}^{t-1}D_{\RL, h}^2\rbr{M, M^*;\pi^i}\nend
    &\quad \le {\sum_{i=1}^{t-1} \EE^{\pi^i}D_\H^2\rbr{\nu_h^{\pi^i,\tilde M}, \nu_h^{\pi^i,M^*}} +
    \sum_{i=1}^{t-1} \EE^{\pi^i} D_\H^2(P_h^{\tilde M}, P_h^{M^*}) +
    \sum_{i=1}^{t-1} \EE^{\pi^i}\rbr{u_h^{M^*}-u_h^{\tilde M}}^2} + 6\epsilon T\nend
    &\quad \le -3 \rbr{ \sum_{i=1}^{t-1} \log \frac{\nu_h^{\pi^i, \tilde M}}{\nu_h^{\pi^i, M^*}} + 
    \sum_{i=1}^{t-1} \log \frac{P_h^{\tilde M}}{P_h^{M^*}} + 
    \sum_{i=1}^{t-1} \rbr{\rbr{u_h^i - u_h^{M^*}}^2 - \rbr{u_h^i - u_h^{\tilde M}}^2}} \nend
    &\qqquad + 9\log\rbr{\frac{3H\cN_\rho(\cM,\epsilon)}{\delta}}+6T\epsilon \nend
    &\quad\le 3 \rbr{\cL^t_h(\tilde M)- \cL^t_h(M^*)} + 9\log\rbr{\frac{3H\cN_\rho(\cM,\epsilon)}{\delta}} + 6T\epsilon, 
\end{align*}
where the first inequality holds by definition of $D_\RL$ and \eqref{eq:D_RL-covering bound}, the second inequality holds by taking a union bound over the success of \eqref{eq:MLE-nu}, \eqref{eq:MLE-T}, and \eqref{eq:MLE-u}, and the last inequality holds by definition of $\cL^t_h$.
Furthermore, we notice by \eqref{eq:cL-covering bound} that
\begin{align}\label{eq:MLE-D_RL}
    \sum_{i=1}^{t-1}D_{\RL, h}^2\rbr{M, M^*;\pi^i}
    &\le 3 \rbr{\cL^t_h(\tilde M) -  \cL^t_h(M) + \cL^t_h(M)- \cL^t_h(M^*)} + 9\log\rbr{\frac{3H\cN_\rho(\cM,\epsilon)}{\delta}} + 6T\epsilon\nend
    &\le 3 \rbr{\cL^t_h(M)- \cL^t_h(M^*)} + 9\log\rbr{\frac{3H\cN_\rho(\cM,\epsilon)}{\delta}} + 18T\epsilon.
\end{align}
Now, we replace $M$ by $\hat M_{h,\MLE} =\argmin_{M\in\cM} \cL^t_h(M)$, $\epsilon$ by $T^{-1}$ in \eqref{eq:MLE-D_RL} and obtain with probability $1-\delta$ for all $h\in[H], t\in[T]$ that 
\begin{align*}
    \cL^t_h(M^*) - \inf_{M\in\cM}\cL^t_h(M) &\le -\frac 1 3 \sum_{i=1}^{t-1}D_{\RL, h}^2\rbr{M, M^*;\pi^i} + 9\log\rbr{\frac{3H\cN_\rho(\cM,\epsilon)}{\delta}} + 18T\epsilon\nend
    &\le 9\log\rbr{\frac{3H\cN_\rho(\cM,T^{-1})}{\delta}} + 18 = \beta, 
\end{align*}
which shows that $M^*\in \confset^t(\beta)$ with high probability. On the other hand, we plug in any $M\in\confset_\cM^t(\beta)$ and obtain for any $h\in[H]$, $t\in[T]$, and $M\in\confset_\cM^t(\beta)$ that
\begin{align*}
    \sum_{i=1}^{t-1}D_{\RL, h}^2\rbr{M, M^*;\pi^i}
    &\le 3 \rbr{\cL^t_h(M)- \cL^t_h(M^*)} + 9\log\rbr{\frac{3e^2H\cN_\rho(\cM,\epsilon)}{\delta}} \le 4\beta,
\end{align*}
where the last inequality holds by definition of the confidence set. Hence, we complete our proof of \Cref{lem:MLE}.
Lastly, we can also take filtration $\sF_{i-1} = \sigma((s_h^j, a_h^j, b_h^j)_{j\in[i-1]}, s_h^i)$ for \eqref{eq:MLE-nu} and $\sF_{i-1} = \sigma((s_h^j, a_h^j, b_h^j)_{j\in[i]})$ for \eqref{eq:MLE-T} and \eqref{eq:MLE-u}, follows exactly the same steps, and obtain
\begin{align*}
    \sum_{i=1}^{t-1} \hat D_{\RL,h,i}^2(M,M^*) 
    % &\le \frac 3 2  \sum_{i=1}^{t-1}D_{\RL, h}^2\rbr{M, M^*;\pi^i} + 6 \log (eH\cN_\rho(\cM,T^{-1})\delta^{-1}) 
    \le 4 \beta,
\end{align*}
which finishes the proof of \Cref{lem:MLE}.

% \subsection{Proof of \Cref{lem:1st-ub} on Bounding the First Order Term by $D_\RL$} \label{sec:1st-ub}
% In this section, we will bound $f_{1, h}^\tM(\pi)$ by $D_\RL$ in the following way.
% % \paragraph{For Small $\eta \bigabr{\tilde A- A}$.}
% % On the one hand, we have by the Cauchy-Schwartz inequality that
% % \begin{align*}
% %     \EE \exp\rbr{\eta \bigabr{\tilde A- A}} \cdot \EE \sbr{\exp\rbr{- \eta \bigabr{\tilde A- A}}\cdot \bigabr{\tilde A-A}^2} \ge \rbr{\EE\sbr{\bigabr{\tilde A-A}}}^2.
% % \end{align*}
% % On the other hand, the second term on the left hand side can be bounded by the Hellinger distance, 
% % \begin{align*}
% %     \EE D_\H^2(\nu, \tilde\nu) = \EE \inp[\bigg]{\nu}{\biggrbr{1-\sqrt{\frac{\tilde\nu}{\nu}}}^2} \ge \EE \sbr{\eta^2 \exp\rbr{- \eta \bigabr{\tilde A- A}}\cdot \bigabr{\tilde A-A}^2}.
% % \end{align*}
% % Hence, we conclude that
% % \begin{align*}
% %     \EE \exp\rbr{\eta \bigabr{\tilde A- A}} \cdot \EE D_\H^2(\nu, \tilde\nu)\ge \eta^2  \cdot \rbr{\EE\bigabr{\tilde A-A}}^2,
% % \end{align*}
% % and also
% % \begin{align}
% %     \sum_{i=h}^H \gamma^{i-h}\EE \exp\rbr{\eta \bigabr{\tilde A_i- A_i}} \cdot \sum_{i=h}^H \gamma^{i-h}\EE D_\H^2 \rbr{\nu_i, \tilde\nu_i} \ge \eta^2 \rbr{\sum_{i=h}^H \gamma^{i-h} \EE \bigabr{\tilde A_i-A_i}}^2.\label{eq:A-cauchy}
% % \end{align}
% To bridge the difference in the A-function to $f_{1,h}^\tM(\pi)$, we follow from the decomposition of the A-function in \Cref{lem:AQV-func diff},
% \begin{align*}
%     \abr{\rbr{\EE_{s_h, b_h}-\EE_{s_h}} \Delta_h^{(1)}} 
%     &\le \bigabr{A_h-\tilde A_h} + 2\eta^{-1} \Delta_h^{(2)}(s_h) \le \bigabr{A_h-\tilde A_h} + 2\EE_{s_h}\sbr{\sum_{i=h}^H \gamma^{i-h} \inp[]{\nu_i}{A_i-\tilde A_i}}, 
% \end{align*}
% where we use the definition $\Delta_h^{(2)}(s_h)\defeq \EE_{s_h}\sbr{\sum_{i=h}^H \gamma^{i-h} \kl\infdivx[]{\nu_i}{\tilde\nu_i}}$ and the last inequality holds by noting that $\kl\infdivx[]{\nu}{\tilde\nu} = \eta\inp[]{\nu}{A-\tilde A}$.
% Therefore, we conclude that
% \begin{align}
%     f_{1, h}^\tM(\pi) 
%     &\le  \EE\bigabr{A_h-\tilde A_h} + 2\EE\sbr{\sum_{i=h}^H \gamma^{i-h} \inp[]{\nu_i}{A_i-\tilde A_i}}
%     % \nend
%     % &
%     \le 3 \sum_{i=h}^H \gamma^{i-h} \EE\bigabr{A_i-\tilde A_i} \label{eq:Delta-A}
%     % \\
%     % &\le 3 \eta^{-1} \sqrt{\sum_{i=h}^H \gamma^{i-h}\EE \exp\rbr{\eta \bigabr{\tilde A_i- A_i}}} \cdot D_\RL(M^*,\tilde M;\pi),\nonumber
% \end{align}
% % where the last inequality is a direct result of \eqref{eq:A-cauchy}.
% % \paragraph{For Large $\eta \bigabr{\tA - A}$. }
% We now invoke the lower bound \eqref{eq:nu-tv-lb-1} in \Cref{lem:response diff} and obtain
% \begin{align*}
%     D_\TV(\nu_h, \tnu_h) \ge \frac{1-\exp\rbr{-2\eta B_A}}{4 B_A} \cdot {\EE_{s_h}\bigabr{\tA_h-A_h} }.
% \end{align*}
% Combining these results, we obtain
% \begin{align*}
%     f_{1,h}^{\tM}(\pi) &\le 3\sum_{i=h}^H \gamma^{i-h} \EE\bigabr{\tA_i-A_i} \nend
%     &\le 3 \cdot \rbr{\frac{1-\exp\rbr{-2\eta B_A}}{4 B_A}}^{-1} \sum_{i=h}^H \gamma^{i-h}\EE D_\TV(\nu_h,\tilde\nu_h) \nend
%     &\le 6(1+2\eta B_A)\cdot  \sqrt{\frac{1-\gamma^H}{1-\gamma}}  \cdot \eta^{-1} \max_{h\in[H]} \EE D_\TV(\nu_h,\tilde\nu_h),
% \end{align*}
% where the last inequality follows from from the fact that $(1-\exp(-x))/2x\ge 1/2(1+x)$.
% % Hence, we have
% % \begin{align*}
% %     f_{1,h}^{\tM}(\pi) &\le   3 \cdot \frac{4 \eta B_A}{1-\exp\rbr{-2\eta B_A}}\cdot
% %     \sqrt{\frac{1-\gamma^H}{1-\gamma}}  \cdot
% %     \eta^{-1}D_\RL(M^*,\tilde M;\pi)\nend
% %     &\le 6(1+2\eta B_A)\cdot  \sqrt{\frac{1-\gamma^H}{1-\gamma}}  \cdot \eta^{-1}D_\RL(M^*,\tilde M;\pi),
% % \end{align*}
% Hence, we complete the proof of \Cref{lem:1st-ub}.
% \section{Proof of \Cref{cor:online linear} }\label{sec:proof-OMLE-linear}
In this proof, we apply \Cref{thm:OMLE-farsighted} to linear MDP. 
To do so, we need to (i) specify the choice of $\cF_{2, h}$ under the linear constraint of the follower's utility; (ii) compute the Eluder dimension of $\cF_0, \cF_{1,h}$ and $\cF_{2, h}$.

\paragraph{Specification of $\cF_{2,h}$. }
We remind the reader of the condition for $f_{2, h}$, 
\begin{align*}
    \EE_{s_h, b_h}^{\pi, M^*}\bigabr{A_h^{\pi, M}-A_h^{\pi,M^*}} \le  \abr{f_{2,h}^M(\pi)} \le L^{(2)} D_\RL(M, M^*;\pi).
\end{align*}

Therefore, we see directly that by taking $f_{2, h}^M(\pi)$ as 
\begin{align*}
    f_{2, h}^M(\pi) = \EE^{\pi, M^*}\sbr{\rbr{r_h^{\pi, M}- r_h^{\pi,M^*} + \gamma \bigrbr{\TT_h^{\pi, M} - \TT_h^{\pi, M^*}} V_{h+1}^{\pi_{\opt}^M, M}}^2}, 
\end{align*}
we have for all $h\in[H]$ that
\begin{align*}
    \cE_{h}^{(2)}(M^t, M^*;\pi^t)\le C^{(2)} \sum_{h=1}^H f_{2,h}^{M^t}(\pi^t).
\end{align*}
We next see how to upper bound $\sum_{i=1}^T f_{2,h}^{M^i}(\pi^i)$ by the guarantee of MLE. Specifically, we need to upper bound $\EE\osbr{\orbr{(\TT_h^\pi-\tilde\TT_h^{\pi})\tilde V_{h+1}}^2}$ and $\EE\osbr{\orbr{r_h^\pi-\tilde r_h^\pi}^2}$ by $D_{\RL, h}^2$ seperately. 

\paragraph{Bounding $f_{2,h}$ with $D_\RL$.}
Note that we only have guarantee for $D_\TV^2(\nu_h, \tilde\nu_h)$ by MLE, which cannot directly guarantee that the true utility is identifiable since a constant shift does not change the follower's behavior at all. For the reward to be identifiable, we need an additional linear constraint, namely $\inp{x}{r_h(s_h, a_h, \cdot)}=\varsigma$.
We start with the easier part with the transition kernel.
\begin{align*}
    \EE\osbr{\orbr{(\TT_h^\pi-\tilde\TT_h^{\pi})\tilde V_{h+1}}^2}\le 2^2 B_U^2 \EE\sbr{D_\TV^2(\TT_h, \tilde \TT_h)} \le 4 B_U^2 D_{\RL, h}^2(\tilde M, M^*;\pi).
\end{align*}
For the utility, we have the following decomposition
\begin{align*}
    \inf_{\xi\in\RR}\EE_{s_h}\abr{r_h^\pi-\tilde r_h^\pi - \xi}  &= \inf_{\xi\in\RR}\EE_{s_h}\abr{Q_h^\pi-\tilde Q_h^\pi - \xi - \gamma\rbr{\TT_h^\pi-\tilde\TT_h^\pi}\tilde V_{h+1} - \gamma \TT_h^\pi\rbr{V_{h+1}-\tilde V_{h+1}}}\nend
    &\le \inf_{\xi\in\RR}\EE_{s_h}\abr{Q_h^\pi-\tilde Q_h^\pi - \xi} + \gamma \EE_{s_h}\abr{\rbr{\TT_h^\pi-\tilde\TT_h^\pi}\tilde V_{h+1} } + \gamma\exp\rbr{2\eta B_A}\EE_{s_h}\abr{Q_{h+1}^\pi - \tilde Q_{h+1}^\pi}, \nend
    &\le \EE_{s_h}\abr{A_h^\pi-\tilde A_h^\pi} + \gamma\EE_{s_h}\abr{\rbr{\TT_h^\pi-\tilde\TT_h^\pi}\tilde V_{h+1} } + \gamma\exp\rbr{2\eta B_A}\EE_{s_h}\abr{Q_{h+1}^\pi - \tilde Q_{h+1}^\pi}
\end{align*}
where the first inequality holds by the same argument for $V-\tilde V$ in \eqref{eq:f_2-1}, and the second inequality holds simply by plugging $\xi = V_h^\pi(s_h) - \tilde V_h^\pi(s_h)$. Now, we can plug in the bound for $\EE_{s_h}\oabr{A_h^\pi-\tilde A_h^\pi}$ in \Cref{lem:response diff} and obtain
\begin{align}
    \inf_{\xi\in\RR}\EE_{s_h}\abr{r_h^\pi-\tilde r_h^\pi - \xi} &\le \underbrace{C_\eta^{-1} \eta D_\TV(\nu_h, \tilde\nu_h) + 2 \gamma B_U \EE_{s_h} D_\TV(\TT_h, \tilde\TT_h)}_{\ds \sD_h(\tilde M, M^*;\pi)} + \gamma\exp\rbr{2\eta B_A}\EE_{s_h}\abr{Q_{h+1}^\pi - \tilde Q_{h+1}^\pi}
    % &\le \rbr{2\eta^{-1}(1+2\eta B_A)+2\gamma B_U} D_{\RL,h}(\tilde M,M^*;\pi) + \gamma\exp\rbr{2\eta B_A}\EE_{s_h}\abr{Q_{h+1}^\pi - \tilde Q_{h+1}^\pi}, 
    \label{eq:f_2-r-diff}
\end{align}
where the second inequality holds by noting that $C_\eta\le 2(1+2\eta B_A)$. We next show what we can say about the utility when combining the guarantee of \eqref{eq:f_2-r-diff} with the linear constraint $\inp{x}{r_h(s_h, a_h, \cdot)}=\varsigma$. Specifically, we have the following lemma.
% \begin{lemma}[\textit{Identification of the follower's utility}]\label{lem:identification}
%     Suppose for $r, \tilde r:\cB\rightarrow \RR$, for some distribution $\nu\in\Delta(\cB)$ such that $\nu>0$, we have $\inf_{\xi\in\RR}\inp{\nu}{\abr{r-\tilde r-\xi}}\le \varepsilon$ and $\inp{x}{r-\tilde r}=0$ hold at the same time for some $x:\cB\rightarrow \RR$ such that $\inp{\ind}{x}\neq 0$. We have
%     \begin{align*}
%         \inp{\nu}{\abr{r-\tilde r}}\le\rbr{1 + \nbr{\frac x \nu}_\infty \cdot \frac{1}{\abr{\inp{x}{\ind}}}} \epsilon
%     \end{align*}
%     \begin{proof}
%         See \Cref{sec:proof-identification} for a detailed proof.
%     \end{proof}
% \end{lemma}
With \Cref{lem:identification}, we conclude that
\begin{align*}
    \EE_{s_h}\abr{r_h^\pi-\tilde r_h^\pi} &\le \rbr{1+\exp\rbr{2\eta B_A} \kappa} \rbr{\sD_h(\tilde M, M^*;\pi) + \gamma \exp\rbr{2\eta B_A}\cdot\EE_{s_h}\abr{Q_{h+1}^\pi - \tilde Q_{h+1}^\pi}}.
\end{align*}
On the other hand, for the Q-function, we have by \eqref{eq:f_2-Q update} that 
\begin{align*}
    \EE_{s_h}\abr{Q_h^\pi - \tilde Q_h^\pi}
    &\le \EE_{s_h}\abr{r_h^\pi-\tilde r_h^\pi + \gamma \rbr{\TT_h^\pi-\tilde\TT_h^\pi}\tilde V_{h+1}} + \gamma \exp\rbr{2\eta B_A} {\EE_{s_h}\abr{Q_{h+1}^\pi-\tilde Q_{h+1}^\pi}}\nend
    &\le \underbrace{2\rbr{1+\exp\rbr{2\eta B_A} \kappa} }_{\ds c_1}\cdot \sD_h(\tilde M,M^*;\pi) \nend
    &\qquad + \underbrace{\rbr{2+\exp\rbr{2\eta B_A} \kappa}\gamma \exp\rbr{2\eta B_A}}_{\ds c_2}\cdot\EE_{s_h}\abr{Q_{h+1}^\pi - \tilde Q_{h+1}^\pi}.
\end{align*}
Therefore, we have by a recursive argument that 
\begin{align*}
    \EE_{s_h}\abr{Q_h^\pi -\tilde Q_h^\pi} \le \sum_{l=h}^H c_2^{l-h} c_1 \EE_{s_h}\sD_l(\tilde M, M^*;\pi).
\end{align*}
For now, we are able to deal with $(\EE_{s_h}\abr{r_h^\pi-\tilde t_h^\pi})^2$. However, note that  the square appears within the expection $\EE_{s_h}$ in terms of the follower's utility error $\EE\rbr{r_h^\pi-\tilde r_h^\pi}^2$. Therefore, we need a variance-mean decomposition,
\begin{align*}
    \EE\rbr{r_h^\pi-\tilde r_h^\pi}^2&= \EE\rbr{Q_h^\pi-\tilde Q_h^\pi - \gamma \rbr{\TT_h^\pi -\tilde\TT_h^\pi}\tilde V_{h+1} -\gamma \TT_h^\pi \rbr{V_{h+1}-\tilde V_{h+1}}}^2\nend
    &\le 4\EE\sbr{\rbr{A_h^\pi -\tilde A_h^\pi}^2 + \rbr{V_h-\tilde V_h}^2 + \gamma^2 \rbr{\rbr{\TT_h^\pi -\tilde\TT_h^\pi}\tilde V_{h+1}}^2 +\gamma^2 \rbr{V_{h+1}-\tilde V_{h+1}}^2}\nend
    &\le 4\EE\sbr{\rbr{A_h^\pi -\tilde A_h^\pi}^2} + 16\gamma^2 B_U^2\EE\sbr{D_\TV(\TT_h,\tilde \TT_h)^2}\nend
    &\qquad + 4\exp\rbr{4\eta B_A}\EE\sbr{\rbr{\EE_{s_h}\abr{Q_h^\pi-\tilde Q_h^\pi}}^2+ \rbr{\EE_{s_{h+1}}\abr{Q_{h+1}^\pi-\tilde Q_{h+1}^\pi}}^2}, 
\end{align*}
where the last inequality holds by noting the upper bound for difference in the V-function used in \Cref{eq:f_2-1}.
We notice that the first term can be upper bounded by the squared Hellinger distance,
\begin{align*}
    D_\H^2\rbr{\nu_h,\tilde\nu_h} &= \dotp{\nu_h}{\rbr{1-\sqrt\frac{\tilde \nu_h}{\nu_h}}^2}\nend
    &= \dotp{\nu_h}{\rbr{1-\exp\rbr{\frac \eta 2 (\tilde A_h - A_h)}}^2}\nend
    &\ge \rbr{\frac{1-\exp\rbr{\eta B_A}}{2 B_A}}^2  \cdot \dotp{\nu_h}{\rbr{A_h-\tilde A_h}^2},
\end{align*}
where the inequality holds by noting that $|1-\exp(x)|\ge (1-\exp(-B))|x|/B$ for any $|x|\le B$.
Therefore, we have the squared utility difference bounded by
\begin{align*}
    \EE\rbr{r_h^\pi -\tilde r_h^\pi}^2 &\le 4 C_\eta^{2} \eta^{-2} \EE D_\H^2(\nu_h,\tilde\nu_h) + 16\gamma^2 B_U^2\EE D_\TV^2(\TT_h,\tilde \TT_h)\nend
    &\qquad + 4 \exp\rbr{4\eta B_A} \EE\sbr{\rbr{\sum_{l=h}^H c_2^{l-h} c_1 \EE_{s_h}\sD_l(\tilde M, M^*;\pi)}^2 + \rbr{\sum_{l=h+1}^H c_2^{l-h} c_1 \EE_{s_{h+1}}\sD_l(\tilde M, M^*;\pi)}^2}\nend
    &\le 4 C_\eta^{-2} \eta^2 \EE D_\H^2(\nu_h,\tilde\nu_h) + 16\gamma^2 B_U^2\EE D_\TV^2(\TT_h,\tilde \TT_h)\nend
    &\qquad + 8 H \max\cbr{c_2^{2H}, 1} c_1^2\exp\rbr{4\eta B_A} \EE\sbr{\sum_{l=h}^H {\sD_l(\tilde M, M^*;\pi)}^2}\nend
    &\le 4 C_\eta^{-2} \eta^2 \EE D_\H^2(\nu_h,\tilde\nu_h) + 16\gamma^2 B_U^2\EE D_\TV^2(\TT_h,\tilde \TT_h)\nend
    &\qquad + 8 H \max\cbr{c_2^{2H}, 1} c_1^2\exp\rbr{4\eta B_A} \EE\sbr{\sum_{h=1}^H \rbr{{C_\eta^{-1} \eta D_\TV(\nu_h, \tilde\nu_h) + 2 \gamma B_U \EE_{s_h} D_\TV(\TT_h, \tilde\TT_h)}}^2}.
\end{align*}
In summary, we have
\begin{align*}
    \EE\rbr{r_h^\pi -\tilde r_h^\pi}^2\le 32 H^2 \max\cbr{c_2^{2H}, 1} c_1^2\exp\rbr{4\eta B_A} \rbr{4(\eta^{-1}+2B_A)^2+4\gamma^2 B_U^2 } \max_{h\in[H]} D_{\RL,h}^2(\tilde M, M^*;\pi).
\end{align*}
where the first inequality holds by noting that $C_\eta\le 2(1+2\eta B_A)$. 
% \newpage
% which completes the proof of \Cref{cor:online linear}.


\subsection{Proof of \Cref{lem:identification}}\label{sec:proof-identification}
For condition $\inf_{\xi\in\RR}\inp{\nu}{\abr{r-\tilde r-\xi}}\le \varepsilon$, we assume that the infimum is achieved at $\xi^*$. Let $r^* = r -\xi^*$ and we have
\begin{align*}
\abr{\inp{r^*-\tilde r}{x}} \le \inp{\abr{r^*-\tilde r}}{\abr{x}} \le \inp{\abr{r^*-\tilde r}}{\nu} \cdot \nbr{\frac{x}{\nu}}_\infty\le \varepsilon\nbr{\frac{x}{\nu}}_\infty,
\end{align*}
where the second inequality is just a distribution shift and the last inequality is given by the condition. Furthermore, for our target,
\begin{align*}
    \inp{\abr{r-\tilde r}}{\nu} \le \inp{\abr{r^*-\tilde r}}{\nu} + \abr{\xi^*} = \varepsilon + \abr{\inp{r^*-r}{x}} \cdot \frac{1}{\abr{\inp{x}{\ind}}},
\end{align*}
where the inequality follows from the triangle inequality and the equality holds by noting that $\inp{x}{\ind}\neq 0$ and $\inp{\abr{r^*-\tilde r}}{\nu}\le \varepsilon$. We bridge these two inequalities by noting that
\begin{align*}
    \abr{\inp{r^*-\tilde r}{x}} = \abr{\inp{r-\tilde r}{x} + \inp{r^*-r}{x} } = \abr{\inp{r^*-r}{x} },
\end{align*}
where the second inequality holds by noting that $\inp{r-\tilde r}{x}=0$. Combining these results and we have
\begin{align*}
    \inp{\abr{r-\tilde r}}{\nu} \le \epsilon + \abr{\inp{r^*-r}{x}} \cdot \frac{1}{\abr{\inp{x}{\ind}}} \le \rbr{1 + \nbr{\frac x \nu}_\infty \cdot \frac{1}{\abr{\inp{x}{\ind}}}} \epsilon, 
\end{align*}
which completes the proof on \Cref{lem:identification}.


\section{Proofs of the Auxiliary Results in  \Cref{sec:app-major-tech}}
In this section, we provide proof for the lemmas introduced  in \Cref{sec:app-major-tech}. 

% \subsection{Proof for \Cref{sec:app-major-tech}}

\subsection{Proof of \Cref{lem:performance diff}}\label{sec:proof-performance diff}

In the following, we prove \Cref{lem:performance diff}, which relates the estimation error of quantal response policy to a few estimation errors involving the follower's value functions. 
To simplify the notation, we let $\nu$, 
$Q, V, A$ denote $\nu$, $Q^{\pi}$, $V^{\pi}$, and $A^{\pi}$, respectively, which are quantities computed under the true model $M^*$.  
Note that we have  
$\nu_h(b\given s) = \exp(\eta \cdot A_h (s, b) )$ and 
$\tilde \nu_h (b \given s) = \exp(\eta \cdot \tilde A_h (s,b) )$.
By the upper bound in \eqref{eq:nu-tv-ub-0} of \Cref{lem:response diff},
we have 
\begin{align}
    \dr{(i)} &\defeq \sum_{h=1}^H  H \cdot  \EE \bigsbr{ \nbr{\tnu_h(\cdot\given s_h)-\nu_h(\cdot\given s_h)}_1} = \sum_{h=1}^H  2 H \cdot  \EE \bigsbr{  D_\TV \bigrbr{\nu(\cdot\given s_h), \tilde \nu(\cdot\given s_h)}} \nend
    &\le 2 \eta  H \cdot  \underbrace{\sum_{h=1}^H \EE\bigsbr{\bigabr{\orbr{A_h-\tilde A_h}(s_h, b_h)}}}_{\dr (ii)} \nend
    &\qqquad + \eta^2  H \cdot \sum_{h=1}^H \EE\Bigsbr{ \exp\bigrbr{\eta\bigabr{\orbr{A_h-\tilde A_h}(s_h, b_h)}}\cdot \bigabr{\orbr{A_h-\tilde A_h}(s_h, b_h)}^2}. \label{eq:(i)}
\end{align}
% In this section, we characterize the performance difference that arises from model misspecification, namely the difference in the leader's total reward under a given policy $\pi$ for a misspecified model $\tilde M$ against the true model $M^*$. In the following, we use $(Q_h, V_h, A_h, \nu_h, U_h, W_h)$ for the follower/leader under the true model $M^*$, and $(\tQ_h,\tV_h,\tA_h, \tnu_h, \tilde U_h, \tilde W_h)$ for the follower/leader under the alternative model $\tilde M$. 
% We let
% \begin{align}
%     \dr{(i)} &\defeq \sum_{h=1}^H  H \EE \nbr{\tnu_h(\cdot\given s_h)-\nu_h(\cdot\given s_h)}_1\nend
%     &\le 2 \eta  H \underbrace{\sum_{h=1}^H \EE\sbr{\abr{\rbr{A_h-\tilde A_h}(s_h, a_h)}}}_{\dr (ii)} \nend
%     &\qqquad + \eta^2  H \sum_{h=1}^H \EE\sbr{ \exp\rbr{\eta\abr{\rbr{A_h-\tilde A_h}(s_h, a_h)}}\cdot \abr{\rbr{A_h-\tilde A_h}(s_h, a_h)}^2}, \label{eq:(i)}
% \end{align}
%where the first inequality holds by noting that $\bignbr{\tilde U_h}_\infty \le  H$, and the second inequality uses the upper bound for the TV distance between two difference logistic responses given by \Cref{lem:response diff}.
In the following, we let 
\$
    \tilde\Delta_h^{(1)} (s_h, b_h)&\defeq  \rbr{\EE_{s_h, b_h} - \EE_{s_h}}\sbr{\sum_{l=h}^H \gamma^{l-h}\bigrbr{\orbr{\tilde Q_l - r_l^\pi - \gamma P_l^\pi \tilde V_{l+1}}(s_l, b_l)}}, \\
    \tilde\Delta_h^{(2)}(s_h) &\defeq \EE_{s_h}\sbr{\sum_{l=h}^H \gamma^{l-h} \kl\infdivx[\big]{\nu_l(\cdot\given s_l)}{\tilde\nu_l(\cdot\given s_l)}}. 
\$
We note that we denote $\EE_{z} [\cdot]=\EE^{\pi,M^*}[\cdot\given z]$ for any variable $z$.
Here the expectations in $\tilde\Delta^{(1)}_h$ and $\tilde\Delta^{(2)}_h$ are taken with respect to the randomness of the trajectory generated by $\{ \pi, \nu^{\pi}\}$, given $s_h$ or $(s_h, b_h)$.
%Note that $\tilde\Delta^{(1)}_h$ is actually a short hand of $\tilde\Delta^{(1)}_{h,\pi, M}(s_h, b_h)$.
We can further bound   (ii) defined in \eqref{eq:(i)} by invoking \Cref{lem:AQV-func diff}, 
which implies that 
\$
\dr{(ii)}&= \sum_{h=1}^H \EE\sbr{\abr{\rbr{\EE_{s_h, b_h}-\EE_{s_h}} \bigsbr{\tilde\Delta_h^{(1)}(s_h, b_h) - \gamma\eta^{-1}\tilde\Delta_{h+1}^{(2)} (s_{h+1})} + \eta^{-1}\kl\infdivx[\big]{\nu_h(\cdot\given s_h)}{\tilde \nu_h(\cdot\given s_h)}}}.
\$
By the law of total expectation, 
we have 
\$
\EE \Bigsbr{\EE _{s_h, b_h} \bigsbr{   \tilde\Delta_{h+1}^{(2)} (s_{h+1})}} = \EE \Bigsbr{\EE _{s_h } \bigsbr{   \tilde\Delta_{h+1}^{(2)} (s_{h+1})}}  \geq 0.
\$
Besides, by the definition of $\tilde \Delta_h^{(2)}$, we have 
\$
\EE\bigsbr{\tilde\Delta_h^{(2)}(s_h)} = \EE\bigsbr{ \kl\infdivx[\big]{\nu_h(\cdot\given s_h)}{\tilde \nu_h(\cdot\given s_h)} + \gamma \cdot \tilde\Delta_{h+1}^{(2)} (s_{h+1}) }. 
\$ 
Thus, by triangle inequality, we have 
\begin{align}
    \dr{(ii)}  &\le \sum_{h=1}^H\EE\Bigsbr{\bigabr{\rbr{\EE_{s_h, b_h}-\EE_{s_h}}\bigsbr{\tilde\Delta_h^{(1)}(s_h, b_h)}}} + 2\eta^{-1} \sum_{h=1}^H\EE\bigsbr{\tilde\Delta_h^{(2)}(s_h)}, \label{eq:(ii)}
\end{align}
Furthermore, for the second term on the right-hand side of \eqref{eq:(ii)}, we  apply the inequality between KL divergence and the $\chi^2$ divergence to each term $\kl\infdivx[]{\nu_l(\cdot\given s_l)}{\tilde\nu_l(\cdot\given s_l)}$ and obtain that 
\begin{align}
\kl\infdivx[\big]{\nu_l(\cdot\given s_l)}{\tilde\nu_l(\cdot\given s_l)}
&\le \chi^2\infdivx[\big]{\nu_l(\cdot\given s_l)}{\tilde\nu_l(\cdot\given s_l)}\nend
&= \inp[\Bigg]{\nu_l(\cdot\given s_l)}{\biggrbr{\sqrt{\frac{\nu_l(\cdot\given s_l)}{\tilde\nu_l(\cdot\given s_l)}}-\sqrt{\frac{\tilde\nu_l(\cdot\given s_l)}{\nu_l(\cdot\given s_l)}}}^2}_{\cB}\nend
&\le \eta^2 \cdot \EE_{s_l}\sbr{\exp\bigrbr{\eta  \cdot \bigabr{\orbr{A_l-\tilde A_l}(s_l,b_l)}}\cdot \bigrbr{\orbr{A_l -\tilde A_l}(s_l, b_l)}^2}, \label{eq:kl-ub}
\end{align}
where last expectation is with respect to $b_ l \sim \nu_{l } (\cdot \given s_l)$. 
Here the inequality holds by noting that 
 $\sqrt{\nu_l/\tilde \nu_l}=\exp(\eta(A_l-\tilde A_l)/2)$ and the basic inequality $| \exp( x ) - \exp(y)| \leq \exp ( | x-y|) \cdot |x -y|$.
Plugging \eqref{eq:kl-ub} back into \eqref{eq:(ii)}, we obtain
\begin{align}
    \dr{(ii)} &\le \sum_{h=1}^H\EE\Bigsbr{\bigabr{\rbr{\EE_{s_h, b_h}-\EE_{s_h}}\bigsbr{\tilde\Delta_h^{(1)}(s_h, b_h)}}}  + 2\eta^{-1} \sum_{h=1}^H \EE\sbr{\sum_{l=h}^H \gamma^{l-h} \cdot \kl\infdivx[\big]{\nu_l(\cdot\given s_l)}{\tilde\nu_l(\cdot\given s_l)}}\nend
    &\le \sum_{h=1}^H\EE\sbr{\bigabr{\rbr{\EE_{s_h, b_h}-\EE_{s_h}}\bigsbr{\tilde\Delta_h^{(1)}(s_h, b_h)}}} \nend
    &\qquad + 2\eta \sum_{h=1}^H \sum_{l=h}^{H} \gamma^{l-h} \cdot  \EE\sbr{\exp\bigrbr{\eta  \cdot \bigabr{\orbr{A_l-\tilde A_l}(s_l,b_l)}}\cdot \bigabr{\orbr{A_l -\tilde A_l}(s_l, b_l)}^2}\nend
    &\le \sum_{h=1}^H\EE\sbr{\bigabr{\rbr{\EE_{s_h, b_h}-\EE_{s_h}}\bigsbr{\tilde\Delta_h^{(1)}(s_h, b_h)}}} \nend
    &\qquad + \frac{2\eta(1-\gamma^H)}{1-\gamma}\cdot \sum_{h=1}^H  \EE\sbr{\exp\bigrbr{\eta  \cdot \bigabr{\orbr{A_h-\tilde A_h}(s_h,b_h)}}\cdot \bigabr{\orbr{A_h -\tilde A_h}(s_h, b_h)}^2} .
     \label{eq:(ii)-2}
\end{align}
Recall that we  define $\eff_H(x) = (1-x^H)/(1-x)$ as the \say{effective}  horizon with respect to $x$.


Plugging \eqref{eq:(ii)-2} back into \eqref{eq:(i)}, we conclude that
\begin{align}
    \dr{(i)}&\le 2\eta  H \cdot \sum_{h=1}^H  \EE\sbr{\abr{\rbr{\EE_{s_h, b_h}-\EE_{s_h}}\bigsbr{\tilde\Delta_h^{(1)}(s_h, b_h)}}}   \nend
    &\qquad + \eta^2  H  \bigrbr{1+ 4  \cdot \eff_H(\gamma) } \cdot \sum_{h=1}^H  \EE\sbr{\exp\bigrbr{\eta  \cdot \bigabr{\orbr{A_h-\tilde A_h}(s_h,b_h)}}\cdot \bigabr{\orbr{A_h -\tilde A_h}(s_h, b_h)}^2}  . \label{eq:(ii)-21}
\end{align}
Note that we define  $C^{(1)}$ in \eqref{eq:define_constants}. 
Since $\oabr{\orbr{A_h-\tilde A_h}(s_h,b_h)} \leq 2 B_{A}$, 
by  \eqref{eq:(ii)-21}  and inequality 
\$
\exp\bigrbr{\eta  \cdot \bigabr{\orbr{A_h-\tilde A_h}(s_h,b_h)}}  \leq \exp(2\eta B_{A}), 
\$
we conclude the proof of \eqref{eq:taylor-myopic}. 

It remains to prove \eqref{eq:taylor-farsighted}. 
Notice that 
\#\label{eq:f_2-01}
\begin{split}
    V_h (s_h) = \max_{\nu' \in \Delta (\cB) }\bigl\{ \inp{\nu' }{Q_h (s_h, \cdot )}_{\cB } +\eta^{-1} \cH(\nu')\bigr\} , \\
    \tilde V_h (s_h) = \max_{\nu' \in \Delta (\cB) }\bigl\{ \inp{\nu' }{\tilde Q_h (s_h, \cdot )}_{\cB } +\eta^{-1} \cH(\nu')\bigr\} ,  
\end{split}
\#
where the maximizers are $\nu_h (\cdot \given s_h)$ and $\tilde \nu_h (\cdot \given s_h)$, respectively.
Then, by  \eqref{eq:f_2-01}
we have 
\#
& \bigabr{\orbr{V_h-\tilde V_h}(s_h)}   \notag \\
& \quad \le \max\Bigcbr{\inp[\big]{\nu_h(\cdot \given s_h) }  { \bigabr{Q_h(s_h, \cdot )-\tilde Q_h(s_h, \cdot )} }_{\cB} }, ~\Bigabr{\inp[\big]{\tilde\nu_h(\cdot \given s_h) }{\bigabr{ Q_h(s_h, \cdot )-\tilde Q_h(s_h, \cdot )} }_{\cB} } \notag\\
& \quad  =   \max\Bigcbr{  \EE_{s_h} \bigl [  \bigl | (Q_h - \tilde Q_h) (s_h, b_h ) \big | \bigr ] , ~   \EE_{s_h} \bigl [ \big |  (Q_h - \tilde Q_h)  (s_h, b_h )\bigr |  \cdot \tilde \nu_h (b_h \given s_h) / \nu_h (b_h \given s_h ) \bigr ]  }   \notag \\
& \quad  \leq    \exp(2 \eta B_A)  \cdot    \EE_{s_h} \bigl [ \big |  (Q_h - \tilde Q_h) (s_h, b_h )\big | \bigr ]   , \label{eq:f_2-1} 
\# 
where the expectation is taken with respect to $b_h \sim \nu_h (\cdot \given s_h)$. 
Here the first inequality is obtained from the optimality condition of \eqref{eq:f_2-01}, and the  last inequality holds because  $\nbr{\tilde\nu_h/\nu_h}_\infty \le \exp(2\eta B_A)$. 
Note that $\tilde A = \tilde Q - \tilde V$ and $A = Q - V$.
By triangle inequality, we have 
\begin{align}
    &\abr{\orbr{A_h-\tilde A_h}(s_h, b_h)}  
  \le  \bigabr{\orbr{Q_h-\tilde Q_h}(s_h, b_h)} + \exp\orbr{2\eta B_A} \cdot  \EE_{s_h} \bigl [  \big |  (Q_h - \tilde Q_h) (s_h, b_h )\big |  \bigr ]  .  \label{eq:f_2-11} 
\end{align}
%   
%  
%  
% For the left side, we first notice 
% \begin{align*}
%     \EE_{s_h}\rbr{A_h^\pi-\tilde A_h^\pi}^2 
%     &= \EE_{s_h}\sbr{\rbr{\rbr{\EE_{s_h, b_h} -\EE_{s_h}}\sbr{A_h^\pi-\tilde A_h^\pi}}^2 }+ \rbr{\EE_{s_h}\sbr{A_h^\pi-\tilde A_h^\pi}}^2\nend
%     & = \EE_{s_h}\sbr{\rbr{\rbr{\EE_{s_h, b_h}-\EE_{s_h}}\sbr{\tilde\Delta_h^{(1)}(s_h, b_h) - \gamma\eta^{-1}\tilde\Delta_{h+1}^{(2)} (s_{h+1})}}^2} + \rbr{\EE_{s_h}\abr{A_h^\pi-\tilde A_h^\pi}}^2,
% \end{align*}
% where the first equality follows from a standard mean-variance decomposition, and the inequality holds by \eqref{eq:A diff-1} in \Cref{lem:AQV-func diff} and noting that $\eta^{-1}\kl\infdivx{\nu_h}{\tilde\nu_h} = \EE_{s_h, b_h}\bigsbr{A_h-\tilde A_h}$.
Now for $\oabr{   (Q_h - \tilde Q_h) (s_h, b_h ) }$, by the Bellman equation $Q_h = r^{\pi}_h + \gamma P_h^{\pi} V_{h+1}$, we have  
\begin{align}
    &\bigabr{ \orbr{Q_h-\tilde Q_h}(s_h, b_h)} \nend
    &\quad \le  
    \bigabr{ 
        \orbr{\tilde Q_h - r_h^\pi - \gamma P_h^\pi \tilde V_{h+1}}(s_h, b_h)
    } 
         + \gamma  \cdot \bigabr{\bigrbr{P_h^\pi \orbr{V_{h+1}-\tilde V_{h+1}}}(s_h, b_h)}
         \nend
    &\quad \le \bigabr{\orbr{\tilde Q_h - r_h^\pi - \gamma P_h^\pi \tilde V_{h+1}}
    (s_h, b_h)} 
    + \gamma \cdot \exp\rbr{2\eta B_A}\EE_{s_h, b_h}\bigsbr{\oabr{Q_{h+1}^\pi-\tilde Q_{h+1}^\pi}},\label{eq:f_2-Q-ub}
\end{align}
where the first inequality holds by a standard decomposition and the second inequality is obtained by applying the  same upper bound for $V_{h}-\tilde V_h$ in \eqref{eq:f_2-1} to $V_{h+1} - \tilde V_{h+1}$. 
By recursion, we have 
\begin{align}
    \bigabr{\orbr{Q_h-\tilde Q_h}(s_h, b_h)} &\le  \sum_{l=h}^H \big (\gamma \cdot \exp(2\eta B_A  ) \big) ^{l-h} \cdot  \EE_{s_h, b_h}\bigsbr{\bigabr{\orbr{\tilde Q_l - r_l^\pi - \gamma P_l^\pi \tilde V_{l+1}}(s_l, b_l)}}.\label{eq:f_2-Q-telo}
    % \nend
    % &\qqquad +  \sum_{l=h}^H \exp(2\eta B_A (l-h+1)) \gamma^{l-h} \rbr{\EE_{s_h}\abr{r_l^\pi-\tilde r_l^\pi} + \gamma \EE_{s_h}\abr{\rbr{P_l^\pi-\tildeP_l^\pi}\tilde V_{h+1}}}.
\end{align}
Now, by the boundedness of $A_h$ and $\tilde A_h$, and \eqref{eq:f_2-11}, 
we have 
\$
& \EE\sbr{\exp\bigrbr{\eta  \cdot \bigabr{\orbr{A_h-\tilde A_h}(s_h,b_h)}}\cdot \bigabr{\orbr{A_h -\tilde A_h}(s_h, b_h)}^2} \notag \\
& \quad \leq \exp\bigrbr{2\eta B_A}\cdot \EE\bigsbr{ \bigabr{\orbr{A_h-\tilde A_h}(s_h, a_h)}^2}\nend
&\quad \le 2\exp\bigrbr{6\eta B_A}\cdot {\EE\bigsbr{ \bigabr{\orbr{\tilde Q_h - Q_h}(s_h, b_h)}^2}},
\$ 
where in the  last inequality we use the basic inequality $(a + b)^2 \leq 2 a ^2 + 2 b^2 $. 
Combining  with \eqref{eq:f_2-Q-telo}, we obtain that 
\begin{align*}
    &\EE\sbr{\exp\bigrbr{\eta  \cdot \bigabr{\orbr{A_h-\tilde A_h}(s_h,b_h)}}\cdot \bigabr{\orbr{A_h -\tilde A_h}(s_h, b_h)}^2} \nend
    &\quad \le 2\exp\bigrbr{6\eta B_A} \cdot \EE \biggsbr { \biggrbr{\sum_{l=h}^H (\gamma \cdot \exp(2\eta B_A  ) \big) ^{l-h} \cdot  \EE_{s_h, b_h}\bigsbr{\bigabr{\orbr{\tilde Q_l - r_l^\pi - \gamma P_l^\pi \tilde V_{l+1}}(s_l, b_l)}}}^2 } \nend
    &\quad \le 2\exp\orbr{6\eta B_A} \cdot \bigrbr{\eff_H(\exp\orbr{2\eta B_A}\gamma)}^2  \cdot \max_{l\in\{h, \dots, H\}}\EE\bigsbr{\bigabr{\orbr{\tilde Q_l - r_l^\pi - \gamma P_l^\pi \tilde V_{l+1}}(s_l, b_l)}^2}.
\end{align*}
Recall that we define  $C^{(2)} = 2 \eta^2  H^2 \cdot \exp\rbr{6\eta B_A}  \cdot \rbr{1+ 4 \eff_H(\gamma)} \cdot \rbr{\eff_H(\exp\cbr{2\eta B_A}\gamma)}^2$. By \eqref{eq:(ii)-21}, we establish \eqref{eq:perform-diff-linear}. 
Therefore, we 
  complete the proof of \Cref{lem:performance diff}.

\subsection{Proof of \Cref{cor:response-diff-myopic}}
\label{sec:proof-response-diff-myopic}



Since we consider any fixed state $s \in \cS$, in this proof, we omit $s$ 
to simplify the notation. 
To apply Lemma \ref{lem:performance diff}, 
we note that 
$Q$ and $\tilde Q$ in   Lemma \ref{lem:performance diff} becomes $r^{\pi}$ and $\tilde r^{\pi}$ in the myopic case. 
To make the proof consistent with that of Lemma \ref{lem:performance diff}, we use notation $\{ Q, \tilde Q, V, \tilde V, A, \tilde A\}$ in the sequel. 

 
To begin with, 
we invoke \eqref{eq:nu-tv-ub-0} in \Cref{lem:response diff} and obtain that 
\begin{align*}
    D_\TV\orbr{\nu, \tilde \nu}
    &\le \eta \cdot \inp[\big]{\nu} {\oabr{\tilde A-A} + \frac \eta 2 \exp\bigrbr{\eta\oabr{\tilde A-A}} \cdot \orbr{\tilde A-A}^2}_{\cB } \nend
    &\le \eta \cdot \inp[\big]{\nu} {\oabr{\tilde A-A} + \frac \eta 2 \exp\orbr{2\eta B_A} \orbr{\tilde A-A}^2}_{\cB }\nend
    &= \eta \cdot \EE\bigsbr{\oabr{\tilde A-A}} + \frac {\eta^2} {2} \exp\bigrbr{2\eta B_A} \cdot \Bigrbr{\Var\orbr{\tilde A-A}+ \bigrbr{\EE\osbr{\tilde A-A}}^2},
\end{align*}
where the last equality holds by the  variance-mean decomposition. 
Here the expectation and variance are taken with respect to $   b \sim \nu(\cdot \given s ) $. 
Now, using \Cref{lem:AQV-func diff} to the myopic case, we have
\begin{align*}
    \EE\bigsbr{\oabr{\tilde A - A}} 
    &\le \EE\bigsbr{\bigabr{(\tilde Q - Q) - \EE  \osbr{\tilde Q - Q}}} + \eta^{-1}\kl\infdivx[]{\nu}  {\tilde\nu}.  
\end{align*}
For the variance term, we have
\begin{align*}
    \Var\orbr{\tilde A -A} & = \EE\bigsbr{\bigrbr{\orbr{\tilde A -A} - \EE\orbr{\tilde A -A}}^2}  = \EE\bigsbr{\bigrbr{\orbr{\tilde Q -Q} - \EE\osbr{\tilde Q -Q}}^2}, 
\end{align*}
where the last equality holds because $V $ and $\tilde V $ do not involve $b $. 
Furthermore, 
note that $$\eta\EE\osbr{A-\tilde A} = \kl\infdivx[]{\nu}{\tilde\nu} \leq 2 \eta B_A.$$ 
Thus, combining the inequalities above, we have 
  we have
\begin{align}
    & D_\TV(\nu,\tilde\nu)\nend 
    & \quad \le   \eta \cdot   \EE\bigsbr{\bigabr{(\tilde Q - Q) - \EE  \osbr{\tilde Q - Q}}} + \frac{\eta^2}{2} \cdot \exp\rbr{2\eta B_A} \cdot \EE\bigsbr{\bigrbr{\orbr{\tilde Q -Q} - \EE\osbr{\tilde Q -Q}}^2}\nend
    &\qquad + \kl\infdivx[]{\nu}{\tilde\nu} + \exp\rbr{2\eta B_A}/ 2 \cdot  \bigrbr{\kl\infdivx[]{\nu}{\tilde\nu}}^2\nend
    &\quad \le\eta \cdot   \EE\bigsbr{\bigabr{(\tilde Q - Q) - \EE  \osbr{\tilde Q - Q}}}  + \frac{\eta^2}{2} \cdot \exp\rbr{2\eta B_A} \cdot  \EE\bigsbr{\bigrbr{\orbr{\tilde Q -Q} - \EE\osbr{\tilde Q -Q}}^2}\nend
    &\qquad + \bigrbr{1 + \eta B_A \exp\rbr{2\eta B_A} }\cdot \kl\infdivx[]{\nu}{\tilde\nu}, \label{eq:TV-ub-taylor}
\end{align}
where the  last inequality holds by noting that $\kl\infdivx[]{\nu}{\tilde\nu}\le 2\eta B_A$.
% Now, suppose that $r$ is parameterized by $\theta$ and $\tilde r$ is parameterized by $\tilde\theta$.

In the following, we  handle the KL divergence term.
We calculate the derivative of $\eta^{-2}\kl\infdivx{\nu}{\tilde\nu}$ with respect to $\tilde Q$ and obtain
\begin{align*}
    \partial_{\tilde Q}\rbr{\eta^{-2}\kl\infdivx{\nu}{\tilde\nu}} = \eta^{-1}\partial_{\tilde Q} \bigrbr{\EE\osbr{A-\tilde A}} = \eta^{-1}{\bigrbr{\partial_{\tilde Q} \tilde V - \nu}} = \eta^{-1}\rbr{\tilde \nu - \nu},
\end{align*}
where $\ind$ denote the all one vector of length $|\cB|$ is $\cB$ is discrete.
Here the first equality follows from $\eta\EE\osbr{A-\tilde A} = \kl\infdivx[]{\nu}{\tilde\nu}$, the second equality holds because $\nu$ and $A$ do not depend on $\tilde Q$, and $\tilde A = \tilde Q - \tilde V$. 
Moreover, the last equality holds because 
$$\tilde V(s)  = \eta^{-1} \log \bigg(\sum_{b \in \cB} \exp \big( \eta \cdot  \tilde Q(s, b)\bigr) \biggr), $$
and also $\partial_{\tilde Q}\EE[\tilde Q] = \nu$.
We further take a second-order derivative and obtain
\begin{align*}
    \partial^2_{\tilde Q \tilde Q} \rbr{\eta^{-2}\kl\infdivx{\nu}{\tilde\nu}} = \eta^{-1}\partial_{\tilde Q} \tilde \nu = \diag(\tilde \nu) -\tilde \nu \tilde\nu^\top\eqdef \H, 
\end{align*}
where the last equality holds for the vector case. 
% For a continuous action space, we have the Hessian represented as $\partial^2_{\tilde Q \tilde Q} \rbr{\eta^{-2}\kl\infdivx{\nu}{\tilde\nu}}(b, b') = \delta(b-b') - \tilde \nu(b)\tilde\nu(b')$. For simplicity, we just stick to the notation for the discrete case while the generalization to the continuous case is just a matter of change of notations. 
Note that the Hessian is upper and lower bounded by $\L$ where $\L = \diag(\nu )-\nu \nu^\top$, which is proved by  the following proposition. 
\begin{proposition}\label{prop:Hessian-ulb}
Let $\H = \diag(\tilde\nu) -\tilde\nu \tilde\nu^\top$ and $\L=\diag(\nu)-\nu \nu^\top$ where $\nu=\exp\orbr{\eta A}$ and $\tilde\nu=\exp\orbr{\eta \tilde A}$ are two quantal response over $\cB$ with $\nbr{A}_\infty\le B_A, \onbr{\tilde A}_\infty\le B_A$. Then   for any vector  $g\in \RR^{|\cB| }$, we have  
\begin{align}
    \exp\rbr{2 \eta B_A} \cdot g^\top \L g \ge x^\top \H x \ge \exp\rbr{-2 \eta B_A} \cdot g^\top \L g.\label{eq:Hessian ub lb}
\end{align}
\end{proposition}
\begin{proof}
Note that $\exp\rbr{-2 \eta B_A}\le  \tilde \nu(b) / \nu(b) \le\exp\rbr{ 2 \eta B_A}$ for any $b\in \cB$. 
Let $\EE^{\nu}$ and $\Var^{\nu}$ denote the expectation and variance under distribution $\nu$. 
Then we have 
\begin{align*}
    g^\top \L g & = \Var^\nu[g(b)]
   = \EE^\nu\bigsbr{\bigrbr{g(b) - \EE^\nu[g(b)]}^2},\notag \\
   g^\top \H g & = \Var^{\tilde \nu}[g(b)]
   = \EE^{\tilde \nu}\bigsbr{\bigrbr{g(b) - \EE^\nu[g(b)]}^2}.
\end{align*}
By direct computation, we have 
\begin{align*}
    & \exp\rbr{-\eta B_A} \cdot \EE^{\tilde\nu}\bigsbr{\bigrbr{g(b) - \EE^{\tilde\nu}[g(b)]}^2}
    \notag \\
    & \quad \le \exp\rbr{-\eta B_A}\cdot \EE^{\tilde\nu}\bigsbr{\bigrbr{g(b) - \EE^{\nu}[g(b)]}^2} 
     \le \EE^\nu\sbr{\rbr{g(b) - \EE^\nu[g(b)]}^2} 
    %%%%%%%%
\end{align*}
where the first inequality is true because changing $\EE^{\tilde\nu}[g(b)]$ to $\EE^{ \nu}[g(b)]$ incurs additional bias, and the second inequality is true because $\tilde \nu (b) / \nu(b)$ 
Similarly, we have 
\begin{align*}
    %%%%%%%%
     \EE^\nu\sbr{\rbr{g(b) - \EE^\nu[g(b)]}^2} 
    %%%%%%%%
    &\le \EE^{\nu}\sbr{\rbr{g(b) - \EE^{\tilde\nu}[g(b)]}^2}  
    %%%%%%%%
     \le  \exp\rbr{\eta B_A}  \cdot \EE^{\tilde\nu}\sbr{\rbr{g(b) - \EE^{\tilde\nu}[g(b)]}^2}.
\end{align*}
Therefore, we conclude that \eqref{eq:Hessian ub lb} holds. 
\end{proof}


Using the lower bound in \eqref{eq:Hessian ub lb}, we have for the KL divergence that
\begin{align*}
    \eta^{-2}\kl\infdivx[]{\nu}{\tilde\nu} &\le 1/2 \cdot (\tilde Q - Q)^\top  \H (\tilde Q - Q)  \le  \exp\rbr{2\eta B_A}/ 2 \cdot (\tilde Q - Q)^\top  \L (\tilde Q - Q) \nend
    &= \frac{\exp\rbr{2\eta B_A}}{2} \cdot (\tilde Q - Q)^\top  \bigrbr{\diag(\nu)-\nu\nu^\top} (\tilde Q - Q) , 
\end{align*}
where the first inequality holds by noting that the derivative of the KL-divergence at $\tilde\nu=\nu$ is zero, and we upper bound the KL-divergence  only by the second order term. Furthermore, the second inequality holds because  $\H\preceq \exp(2\eta B_A)\cdot \L$, which is proved  by \Cref{prop:Hessian-ulb}. 
% where the last inequality holds by applying \eqref{eq:Hessian ub lb} to the Hessian of the KL divergence evaluated at $\nu$. 
%Note that one can also plug in $\tilde\nu$ in the last inequality. 
Therefore, we conclude for \eqref{eq:TV-ub-taylor} that
\begin{align*}
    D_\TV \rbr{\nu, \tilde \nu} &\le \eta \cdot   \EE\bigsbr{\bigabr{(\tilde Q - Q) - \EE\osbr{\tilde Q - Q}}} + \frac{\eta^2}{2} \exp\rbr{2\eta B_A} \cdot \EE\bigsbr{\bigrbr{\orbr{\tilde Q -Q} - \EE\osbr{\tilde Q -Q}}^2}\nend
    &\qquad + \bigrbr{1 + \eta B_A \cdot \exp\rbr{2\eta B_A} }\cdot \kl\infdivx[]{\nu}{\tilde\nu}\nend
    &\le \eta \cdot   \EE\bigsbr{\bigabr{(\tilde Q - Q) - \EE\osbr{\tilde Q - Q}}} \nend
    &\qquad + \frac{\eta^2 \exp(2\eta B_A)}{2} \bigrbr{2+\eta B_A \cdot  \exp\rbr{2\eta B_A}} \cdot  \EE\bigsbr{\bigrbr{\orbr{\tilde Q -Q} - \EE\osbr{\tilde Q -Q}}^2}, 
\end{align*}
which finishes the proof of \Cref{cor:response-diff-myopic}.


% \subsection{Proof of \Cref{lem:response diff-myopic}}
% \label{sec:proof-formal-response diff-linear}
% % We first invoke the upper bound for this TV distance in \eqref{eq:response decomposition} of \Cref{lem:performance diff} where we take $H=1$ for myopic follower and swap the position of $\theta^*$ and $\tilde\theta$ (by the exchangeability of the TV distance), 
% \begin{align}
%     &D_\TV\rbr{\nu^{\pi,\theta^*}(\cdot\given s) , \nu^{\pi,\tilde\theta}(\cdot\given s)} \nend
%     &\quad \le \eta \cdot {\EE_{s}^{\pi, \tilde\theta}\sbr{\abr{\rbr{\EE^{\pi, \tilde\theta}_{s, b}-\EE^{\pi, \tilde\theta}_{s}}\bigsbr{ \inp{\phi^{\pi}(s, b)}{\theta^*-\tilde\theta} }}}} \nend
%     & \qquad\quad + \frac 1 2 \eta^2  \cdot 
%     {\EE_{s}^{\pi, \tilde\theta}\sbr{ \exp\bigrbr{\eta\bigabr{A^{\pi,\tilde\theta}-A^{\pi,\theta^*}}}\cdot \bigabr{A^{\pi,\tilde\theta} - A^{\pi, \theta^*}}^2}}\nend
%     &\quad \le \eta  \cdot \underbrace{\sqrt{\EE_{s}^{\pi, \tilde\theta}\sbr{\rbr{\rbr{\EE^{\pi, \tilde\theta}_{s, b}-\EE^{\pi, \tilde\theta}_{s}}\bigsbr{ \inp{\phi^{\pi}(s, b)}{\theta^*-\tilde\theta} }}^2 }}}_{\dr (i)} \nend
%     & \qquad\quad + \frac 1 2 \eta^2  \exp\rbr{2\eta B_A} \cdot 
%     {\EE_{s}^{\pi, \tilde\theta}\sbr{ \bigrbr{A^{\pi,\tilde\theta} - A^{\pi, \theta^*}}^2}},\label{eq:TV-ub-MLE} 
% \end{align}
% where the second inequality holds by using the Cauchy-Schwartz inequality and bounding the exponential term by its maximal value.
% By a standard variance and mean decomposition in the last quadratic term of \eqref{eq:TV-ub-MLE}, we obtain
% \begin{align}
%     \EE_{s}^{\pi, \tilde\theta}\sbr{ \bigrbr{A^{\pi,\tilde\theta} - A^{\pi, \theta^*}}^2}
%     & \le \EE_{s}^{\pi, \tilde\theta}\sbr{ \rbr{ \rbr{\EE^{\pi, \tilde\theta}_{s, b}-\EE^{\pi, \tilde\theta}_{s}}\sbr{A^{\pi,\tilde\theta} - A^{\pi, \theta^*}}}^2} + \rbr{\EE^{\pi, \tilde\theta}_{s}\sbr{A^{\pi,\tilde\theta} - A^{\pi, \theta^*}}}^2\nend
%     & =\underbrace{\EE_{s}^{\pi, \tilde\theta}\sbr{ \rbr{ \rbr{\EE^{\pi, \tilde\theta}_{s, b}-\EE^{\pi, \tilde\theta}_{s}}\sbr{\inp[\big]{\phi^{\pi}(s, b)}{\theta^*-\tilde\theta} }}^2}}_{\dr (i)^2} + \underbrace{\rbr{\EE^{\pi, \tilde\theta}_{s}\sbr{A^{\pi,\tilde\theta} - A^{\pi, \theta^*}}}^2}_{\dr (ii)}.
%     % &\le \EE_{s}^{\pi, \tilde\theta}\sbr{ \rbr{ \rbr{\EE^{\pi, \tilde\theta}_{s, b}-\EE^{\pi, \tilde\theta}_{s}}\sbr{\inp[\big]{\phi^{\pi}(s, b)}{\theta^*-\tilde\theta} }}^2}
%     % &\qquad + \underbrace{\rbr{ \rbr{\EE^{\pi, \theta^*}_{s}-\EE^{\pi, \tilde\theta}_{s}}\sbr{\inp[\big]{\phi^{\pi}(s, b)}{\theta^*-\tilde\theta} }}^2}_{\dr (iii)},
%     \label{eq:A-square-ub}
% \end{align}
% where the equality follows from \Cref{lem:AQV-func diff} on the difference in the advantage function
% % , and the last inequality holds by \Cref{lem:KL-ub} where we notice that $\kl\infdivx[\big]{\nu^{\pi, \tilde\theta}}{\nu^{\pi, \theta^*}} = \EE^{\pi, \tilde\theta}_{s}\bigsbr{A^{\pi,\tilde\theta} - A^{\pi, \theta^*}}$.
% Combining \eqref{eq:A-square-ub} with \eqref{eq:TV-ub-MLE}, we conclude that
% \begin{align}
%     D_\TV\rbr{\nu^{\pi,\theta^*}(\cdot\given s) , \nu^{\pi,\tilde\theta}(\cdot\given s)} &\le \eta \cdot {\dr(i)} + \frac{\eta^2\exp\rbr{2\eta B_A}}{2} \cdot \rbr{{\dr(i)}^2 + {\dr(ii)}}.\label{eq:TV-ub-offline}
% \end{align}
% It is straightforward to bound term (i) by guarantee of MLE.  By \eqref{eq:bandit-ub-2} in \Cref{lem:bandit}, we have
% \begin{align}
%     \bignbr{\theta^*-\tilde\theta_{\MLE}}_{{\Psi}}^2 \le \underbrace{\min\cbr{\frac{\lambda_d(\tilde\Phi) }{\lambda_1(\Sigma_{\cD})}, \frac{|\cB|}{\min_{t\in[T]}\lambda_2(\Xi^{t, \tilde\theta})}}}_{\ds Z_{\cD}} C_\eta^2 \cdot 
%         \frac 1 T\log\rbr{\frac{\cN(\Theta, 1/T)}{\delta}}, \label{eq:MLE-guarantee-offline}
% \end{align}
% where $C_\eta= {B_A}/\rbr{1-\exp\rbr{-\eta B_A}}$, $\Sigma_{\cD}=T^{-1}\cdot \sum_{t=1}^T\EE_{s^t}^{\nu^{\pi^t, \tilde\theta}}[\psi^{t, \tilde\theta} {\psi^{t, \tilde\theta}}^\top]$, $\psi^{t, \theta} =\phi^{t}(b) - \EE_{s^t}^{\nu^{\pi^t, \theta}}[\phi^t(b')]$, and 
% $\tilde\Phi^t=\int_{\cB}\phi^t(b)\phi^t(b)^\top \rd b$.
% Note if $\cB$ has infinitely many actions, the second term in $Z_{\cD}$ is meaningless and only the first term matters.
% If $\cB$ has finite actions, we define $\tilde\Phi^t=\sum_{b\in\cB}\phi^t(b)\phi^t(b)^\top$ and $\Xi^{t, \tilde\theta} = \diag(\nu^{\pi^t, \tilde\theta}(\cdot\given s^t)) - (\nu^{\pi^t, \tilde\theta}(\cdot\given s^t))(\nu^{\pi^t, \tilde\theta}(\cdot\given s^t))^\top$.

We have by \Cref{cor:response-diff-myopic} that 
\begin{align*}
    D_\TV\rbr{\nu(\cdot\given s), \tilde\nu(\cdot\given s)}
        &\le  \eta  \EE\sbr{\abr{(\tilde r^\pi(s, b) - r^\pi(s, b)) - \EE\bigsbr{\tilde r^\pi(s, b) - r^\pi(s, b)}}} \nend
        &\qquad + C^{(3)}\EE\sbr{\rbr{\rbr{\tilde r^\pi(s, b) -r^\pi(s, b)} - \EE\sbr{\tilde r^\pi(s, b) -r^\pi(s, b)}}^2}.
\end{align*}
% We define the weighted covariance matrix as
% \begin{align*}
%     \Sigma_{ s}^{\pi, \theta} \defeq \EE_{s}^{\pi, \theta} \sbr{\psi^{\pi,\theta}(s, b)\psi^{\pi,\theta}(s, b)^\top}\quad\text{where}\quad \psi^{\pi,\theta}(s, b) = \phi^{\pi}(s, b) - \EE_{s}^{\pi,\theta}\phi^{\pi}(s, \cdot).
% \end{align*}
% Under this definition, we introduce a nonnegative definite matrix $\Psi\in \SSS_+^{d}$ and write down the covariance term as
% \begin{align*}
%     &\EE\sbr{\rbr{\rbr{\tilde r^\pi(s, b) -r^\pi(s, b)} - \EE\sbr{\tilde r^\pi(s, b) -r^\pi(s, b)}}^2} \nend
%     &\quad = \bignbr{\theta^*-\tilde\theta}_{\Sigma_{ s}^{\pi, \tilde\theta}} = \nbr{\sqrt{\Psi}^{\dagger}\sqrt{\Psi}\rbr{\theta^*-\tilde\theta}}_{\Sigma_{ s}^{\pi, \tilde\theta}} \le \sqrt{\Bignbr{\Psi^{\dagger} \Sigma_{ s}^{\pi, \tilde\theta}}_\oper }\cdot \bignbr{\theta^*-\tilde\theta}_{\Psi} \le \sqrt{\trace\rbr{{\Psi}^{\dagger} \Sigma_{ s}^{\pi, \tilde\theta}}} \cdot \bignbr{\theta^*-\tilde\theta}_{\Psi}.
% \end{align*}
% where we recall from \Cref{lem:bandit} the definition of the weighted Laplacian of the comparison feature graph with respect to the offline data $\cD$ and parameter $\tilde\theta$ as $ \Sigma_\cD^{\theta}\defeq T^{-1} \sum_{t=1}^T
% \EE_{s^t}^{\pi^t, \theta}\bigsbr{\psi^{\pi^t, \theta}(s, b)\psi^{\pi^t, \theta}(s, b)^\top}$.
% Note that the operator norm is further bounded by the trace,
% \begin{align*}
%     \sqrt{\Bignbr{{\Psi}^{\dagger} \Sigma_{ s}^{\pi, \tilde\theta}}_\oper} 
%     &\le \sqrt{\trace\rbr{{\Psi}^{\dagger} \Sigma_{ s}^{\pi, \tilde\theta}}} \nend
%     &= \sqrt{\trace\rbr{\EE_{s}^{\pi, \tilde\theta}\sbr{ {\Psi}^\dagger \psi^{\pi,\tilde\theta}(s, b)\psi^{\pi,\tilde\theta}(s, b)^\top}}}\nend
%     & = \sqrt{\EE_{s}^{\pi, \tilde\theta}\sbr{\psi^{\pi,\tilde\theta}(s, b)^\top {\Psi}^\dagger  \psi^{\pi,\tilde\theta}(s, b)}}.
% \end{align*}
% We plug in the definition $\psi^{\pi,\tilde\theta}(s, b)=\phi^{\pi}(s, b) - \EE_{s}^{\pi,\tilde\theta}\phi^{\pi}(s, \cdot)$ and obtain
% \begin{align*}
%     &\sqrt{\EE_{s}^{\pi, \tilde\theta}\sbr{\psi^{\pi,\tilde\theta}(s, b)^\top {\Psi}^\dagger  \psi^{\pi,\tilde\theta}(s, b)}} = \underbrace{\sqrt{\EE_{s}^{{\pi,\tilde\theta}}\sbr{\phi^{\pi}(s,b)^\top {\Psi}^{\dagger}\phi^{\pi}(s,b)} -\bignbr{\EE_{s}^{{\pi,\tilde\theta}}\phi^{\pi}(s, b)}_{{\Psi}^{-1}}^2}}_{\ds \Upsilon_{s}^{\pi,\tilde\theta}}.
% \end{align*}
% Hence, we have for term (i) that  ${\dr (i)}\le 
% \Upsilon_{s}^{\pi,\tilde\theta}\cdot \bignbr{\theta^*-\tilde\theta}_{{\Psi}}$.
% % \begin{align}
% %     {\dr (i)} 
% %     &\le 
% %     \Upsilon_{s}^{\pi,\tilde\theta}\cdot \bignbr{\theta^*-\tilde\theta}_{{\Psi}}
% %     \le 2 \Upsilon_{s}^{\pi,\tilde\theta}\cdot \underbrace{\sqrt{Z_{\cD} C_\eta^2  \cdot \frac 1 T\log\rbr{\frac{\cN(\Theta, 1/T)}{\delta}} + \bignbr{\tilde\theta - \tilde\theta_{\MLE}}_{{\Psi}}^2}}_{\ds \zeta^{\tilde\theta}},\label{eq:myopic-(i)-ub}
% % \end{align}
% % on the success of \Cref{lem:bandit}. 
% % Here, the second inequality follows from the triangular inequaltiy, and the last inequality holds from \eqref{eq:MLE-guarantee-offline}.
% Now, we have addressed term (i) of \eqref{eq:TV-ub-offline} and it remains to show the upper bound for term (ii). Observe that under the quantal response model with logistic preference, term (ii) is nothing but just the squared KL divergence $\eta^{-2}\kl\infdivx[]{\nu^{\pi, \tilde\theta}(\cdot\given s)}{\nu^{\pi, \theta^*}(\cdot\given s)}^2$. Let $\Delta \theta=\tilde\theta - \theta^*$ and consider $\tilde\theta$ to be fixed. The Hessian of the KL divergence with respect to parameters in the second position is 
% \begin{align*}
%     \eta^{-2}\nabla_{\theta}^2 \kl\infdivx[]{\nu^{\pi, \tilde\theta}(\cdot\given s)}{\nu^{\pi, \theta}(\cdot\given s)} = \EE_s^{\pi,\theta} \sbr{\phi^\pi(s, b)\phi^\pi(s, b)^\top} - \EE_s^{\pi,\theta}\sbr{\phi^\pi(s, b)} \cdot \EE_s^{\pi,\theta}\sbr{\phi^\pi(s, b)}^\top = \Sigma_s^{\pi,\theta}.
% \end{align*}
% We observe that the KL divergence is strongly convex in the second parameter. Actually, for any test $x\in\RR^d$ and let $g(b)=\phi^\pi(s, \cdot)^\top x$, we have $x^\top \rbr{\eta^{-2}\nabla_{\theta}^2 \kl\infdivx[]{\nu^{\pi, \tilde\theta}(\cdot\given s)}{\nu^{\pi, \theta}(\cdot\given s)}} x = \Var^{\pi,\theta}[g(b)]$ and furthermore,
% \begin{align*}
%      \Var^{\pi,\theta}[g(b)]
%     \ge \exp(-2\eta B_A) \EE_s^{\pi, \tilde\theta}\sbr{\rbr{g(b) - \EE_s^{\pi,\theta}[g(b)]}^2} \ge \exp\rbr{-2\eta B_A} \Var^{\pi, \tilde\theta}[g(b)],
% \end{align*}
% which suggests that $\eta^{-2}\nabla_{\theta}^2 \kl\infdivx[]{\nu^{\pi, \tilde\theta}(\cdot\given s)}{\nu^{\pi, \theta}(\cdot\given s)}=\Sigma_s^{\pi,\theta}\succeq \exp(-2\eta B_A) \Sigma_s^{\pi,\tilde\theta}$. Actually, we can also swap $\theta$ and $\tilde\theta$ and obtain 
% \begin{align*}
%     \eta\exp(2\eta B_A) \Sigma_s^{\pi,\tilde\theta} \succeq \eta^{-}\nabla_{\theta}^2 \kl\infdivx[]{\nu^{\pi, \tilde\theta}(\cdot\given s)}{\nu^{\pi, \theta}(\cdot\given s)} \succeq \eta\exp(-2\eta B_A) \Sigma_s^{\pi,\tilde\theta} , \quad \forall \theta\in\Theta.
% \end{align*}
% Therefore, we can bound term (ii) simply by
% \begin{align*}
%     {\dr(ii)}\le \frac{\eta^2}{4} \exp\rbr{4\eta B_A} \bignbr{\theta^*-\tilde\theta}_{\Sigma_s^{\pi,\tilde\theta}}^4 \le  \frac{\eta^2}{4} \exp\rbr{4\eta B_A} \rbr{\Upsilon_{s}^{\pi,\tilde\theta}\cdot \bignbr{\theta^*-\tilde\theta}_{{\Psi}}}^4,
% \end{align*}
% where the last inequality holds by the same upper bound for term (i).
% Combining our results for both term (i) (ii), we have for \eqref{eq:TV-ub-offline} that
% \begin{align*}
%     &D_\TV\rbr{\nu^{\pi,\theta^*}(\cdot\given s) , \nu^{\pi,\tilde\theta}(\cdot\given s)} \nend
%     &\quad\le \eta \cdot {\dr(i)} + \frac{\eta^2\exp\rbr{2\eta B_A}}{2} \cdot \rbr{{\dr(i)}^2 + {\dr(ii)}}\nend
%     &\quad\le \eta \Upsilon_{s}^{\pi,\tilde\theta} \bignbr{\theta^*-\tilde\theta}_{{\Psi}} + \frac{\eta^2\exp\rbr{2\eta B_A}}{2} \rbr{\rbr{\Upsilon_{s}^{\pi,\tilde\theta} \bignbr{\theta^*-\tilde\theta}_{{\Psi}}}^2 + \frac{\eta^2}{4} \exp\rbr{4\eta B_A} \rbr{\Upsilon_{s}^{\pi,\tilde\theta} \bignbr{\theta^*-\tilde\theta}_{{\Psi}}}^4}.
% \end{align*}
% Now, we let $\Gamma^{(2)}(s;\pi_{s}, \tilde\theta) = \Upsilon_{s}^{\pi,\tilde\theta} \bignbr{\theta^*-\tilde\theta}_{{\Psi}} $ and 
% \begin{align*}
%     f(x) = \eta x + \frac{\eta^2\exp(2\eta B_A)}{2} x^2 + \frac{\eta^4\exp(6\eta B_A)}{8} x^4, 
% \end{align*}
% and just conclude that $D_\TV\bigrbr{\nu^{\pi_{s},\theta^*}(\cdot\given s) , \nu^{\pi_{s},\tilde\theta}(\cdot\given s)} \le f(\Gamma^{(2)}(s;\pi_{s}, \tilde\theta))$, 
% which completes the proof of \Cref{lem:response diff-myopic}. 
% % One can swap $\theta^*$ and $\$


% % \subsection{Follow up Discussion on \Cref{thm:PMLE-VI-myopic}}\label{sec:follow up on myopic offline}
% % \todo{comment on the use of the Laplacian, including why a standard $\sum\phi\phi^\top$ would fail. Also, comment on the use of $L_\cD$. }

\subsection{Proof of \Cref{lem:response diff-myopic-linear}}
\label{sec:proof-response-diff-myopic_linear}  
    % $\Gamma^{(2)}_h$ is defined as
    % \begin{align*}
    %     \Gamma^{(2)}_h(s_h;\alpha,\theta_h) = \sqrt{\EE_{s_h}^{{\alpha,\theta_h}}\sbr{\phi_h^{\alpha}(s_h,b_h)^\top {\Psi}^{\dagger}\phi_h^{\alpha}(s_h,b_h)} -\bignbr{\EE_{s_h}^{{\alpha,\theta_h}}\phi_h^{\alpha}(s_h,b_h)}_{{\Psi}^{\dagger}}^2}\cdot \bignbr{\theta_h^*- \theta_h}_{{\Psi}}, 
    % \end{align*}
    % and $\Gamma^{(3)}$ is defined as 
    % \begin{align*}
    %     \Gamma^{(3)}_h(s_h;\alpha,\theta_h) = \sqrt{\EE_{s_h}^{{\alpha,\theta_h^*}}\sbr{\phi_h^{\alpha}(s_h,b_h)^\top {\Psi}^{\dagger}\phi_h^{\alpha}(s_h,b_h)} -\bignbr{\EE_{s_h}^{{\alpha,\theta_h^*}}\phi_h^{\alpha}(s_h,b_h)}_{{\Psi}^{\dagger}}^2}\cdot \bignbr{\theta_h^*- \theta_h}_{{\Psi}}
    % \end{align*}
    \begin{proof}
        % We define the weighted covariance matrix as
% \begin{align*}
%     \Sigma_{ s}^{\pi, \theta} \defeq \EE_{s}^{\pi, \theta} \sbr{\psi^{\pi,\theta}(s, b)\psi^{\pi,\theta}(s, b)^\top}\quad\text{where}\quad \psi^{\pi,\theta}(s, b) = \phi^{\pi}(s, b) - \EE_{s}^{\pi,\theta}\phi^{\pi}(s, \cdot).
% \end{align*}
Similar to the proof of \Cref{cor:response-diff-myopic}, we omit the state $s$ to simplify the notation. 
Note that $\tilde Q$ and $Q$ in \Cref{cor:response-diff-myopic}
correspond to $\tilde r^{\pi} = \la \phi^{\pi}, \tilde \theta\ra$ and $r^{\pi} = \la \phi^{\pi}, \theta^*$, respectively.  
Let $\EE_s $ denote the expectation taken with respect to $\nu^{\pi} (\cdot \given s)$.
Then we have 
\begin{align}
    &\EE_s\Bigsbr{\Bigrbr{\rbr{\tilde r^\pi(s, b) -r^\pi(s, b)} - \EE \sbr{\tilde r^\pi(s, b) -r^\pi(s, b)}}^2} = \bignbr{\theta^*-\tilde\theta}_{\Sigma_{ s}^{\pi, \theta}}^2  \nend
    &\quad 
    = 
    \trace\Bigrbr{ \Sigma_{ s}^{\pi, \theta^*} \cdot (\theta ^* - \tilde \theta)(\theta ^* - \tilde \theta)^\top }  
      \le {\trace\bigrbr{{\Psi}^{\dagger} \Sigma_{ s}^{\pi, \theta^*}}} \cdot \bignbr{\theta^*-\tilde\theta}_{\Psi}^2  , \label{eq:reward-covarance-bounded-by-Sigma}
\end{align}
where ${\Psi}^{\dagger}$ is the pseudo-inverse of $\Psi$.
Moreover, by Jensen's inequality,
we have 
\#
& \EE_s\Bigsbr{\Bigabr{\rbr{\tilde r^\pi(s, b) -r^\pi(s, b)} - \EE \sbr{\tilde r^\pi(s, b) -r^\pi(s, b)}} }  \label{eq:reward-abs-bounded-by-Sigma}  \\
&\quad  \leq \biggl(  \EE_s\Bigsbr{\Bigrbr{\rbr{\tilde r^\pi(s, b) -r^\pi(s, b)} - \EE \sbr{\tilde r^\pi(s, b) -r^\pi(s, b)}}^2} \biggr) \leq \sqrt{{\trace\bigrbr{{\Psi}^{\dagger} \Sigma_{ s}^{\pi, \theta^*}}}}  \cdot \bignbr{\theta^*-\tilde\theta}_{\Psi}.  \notag
\# 
Finally, combining \eqref{eq:reward-covarance-bounded-by-Sigma}, \eqref{eq:reward-abs-bounded-by-Sigma}, and 
  \Cref{cor:response-diff-myopic}, we conclude the proof. 
  We note that  a similar result also holds if we swap $\theta^*$ and $\tilde\theta$.
    \end{proof}


\subsection{Proof of \Cref{lem:MLE-formal}}
\label{sec:proof-MLE-general}
% The key idea is replacing the negative log likelihood $\cL^{(1)}_\cD(\cdot)$ with some curve with constant Hessian $\EE_{\nu^{\pi^t, \theta^*}}[\psi^{t, \theta^*} {\psi^{t, \theta^*}}^\top]$ in the neighbourhood of $\theta^*$. Specifically, we first lower bound $\cL^{(1)}_\cD(\hat\theta_\MLE)-\cL^{(1)}_\cD(\theta^*)$ by the Hellinger distance $T^{-1}\cdot\sum_{t=1}^T D_\H^2(\nu^{\pi^t, \hat\theta_\MLE}, \nu^{\pi^t, \theta^*})$ and some $\cO(T^{-1})$ term. Then, a careful scrutiny of the Hellinger distance will show that $$D_\H^2(\nu^{\pi^t, \theta}, \nu^{\pi^t, \theta^*})\ge B_A^{-2} (\theta-\theta^*)^\top\EE^{\nu^{\pi^t, \theta^*}}[\psi^{t, \theta^*} {\psi^{t, \theta^*}}^\top](\theta-\theta^*).$$
% To make the above discussion rigorous, we first invoke the following concentration Lemma.
\begin{proof} 
We prove this lemma by leveraging 
 \Cref{lem:freeman-variation} with $X_t = (-\cL_{h,t}(\theta) + \cL_{h, t}(\theta^*))/2$ where $\cL_{h}^t(\theta)=-\sum_{i=1}^{t-1} \eta A_h^{\pi^i, \theta}(s_h^i,b_h^i)$. 
We choose filtration $\cF_{h, t-1}=\sigma(\tau^{1:t-1})$ where $\sigma(X)$ denotes the $\sigma$-algebra generated by $X$ and $\tau^{1:t-1}$ is just the history up to $t-1$.
Let $\cN_\rho(\Theta, \epsilon)$ be  the covering number for the $\epsilon$-covering net of $\Theta$  with respect to norm $\rho$ defined in \eqref{eq:rho for MLE}. 
Let $\Theta_\epsilon$ be  the $\epsilon$-covering net of $\Theta$. 
To simplify the notation, we define  $\iota =  \log\rbr{ H\cN_\rho(\Theta, \epsilon) / \delta}$.
Then, for all $\theta \in \Theta_{\epsilon}$, 
we have with probability $1-\delta$ for all $h\in[H], t\in[T]$ that
\begin{align}
    &\frac 1 2 \rbr{-\cL_h^t(\theta) + \cL_h^t(\theta^*)} \nend
    &\quad\le \sum_{i=1}^{t-1} \log\EE^{\pi^i   }\sbr{\sqrt{ \nu_h^{\pi^i,\theta}(\cdot\given s_h) \big / \nu_h^{\pi^i, \theta^*}(\cdot\given s_h)}}  + \log\rbr{ H\cN_\rho(\Theta, \epsilon) / \delta} \nend
    &\quad\le -  \sum_{i=1}^{t-1} \EE^{\pi^i }\Bigl [ D_\H^2\bigrbr{\nu_h^{\pi^i, \theta}(\cdot\given s_h ), \nu_h^{\pi^i, \theta^*}(\cdot\given s_h )} \Bigr ] + \log\rbr{ H\cN_\rho(\Theta, \epsilon) / \delta}, \label{eq:MLE-to-hellinger-1}
\end{align}
where the expectation is taken with respect to the true model. 
Here, the first inequality holds by applying  \Cref{lem:freeman-variation} and   taking a union bound over the  $\epsilon$-covering net. 
% Note that $\log(\cN_\rho(\Theta, \epsilon))$ only grows with 
% $\log(\eta T)$ since $\cL_{h,t}(\theta)$ is $2\eta$-Lipschitz with respect to $\theta$. 
The second inequality holds by noting that $\log(x)\le x-1$ and by the definition of the Hellinger distance.


Meanwhile, by the definition of the distance $\rho$ in   \eqref{eq:rho for MLE},
for any $\theta, \tilde \theta \in \Theta$, we have 
\begin{align*}
&|D_\H^2(\nu_h^{\pi,\theta}, \nu_h^{\pi, \theta^*}) - D_\H^2(\nu_h^{\pi,\tilde\theta}, \nu_h^{\pi, \theta^*})|\nend
&\quad \le (D_\H^2(\nu_h^{\pi,\theta}, \nu_h^{\pi, \theta^*}) + D_\H^2(\nu_h^{\pi,\tilde\theta}, \nu_h^{\pi, \theta^*})) \cdot 
\bigabr{D_\H^2(\nu_h^{\pi,\theta}, \nu_h^{\pi, \theta^*}) - D_\H^2(\nu_h^{\pi,\tilde\theta}, \nu_h^{\pi, \theta^*})}\nend
&\quad \le 2 D_\H(\nu_h^{\pi,\tilde\theta}, \nu_h^{\pi,\theta})\nend
&\quad \le 2\rho(\theta, \tilde\theta), 
\end{align*} 
where  the second inequality holds by noting that the Hellinger distance does not exceed 1, and that the Hellinger distance satisfies the triangle inequality as a norm, and the last inequality holds by definition of $\rho$. 
Moreover, by noting that $\cL_h^t(\theta)=-\sum_{i=1}^t \eta A_h^{\pi^i, \theta}(s_h^i,b_h^i)$, we have
\begin{align*}
\bigl | \cL_h^t (\theta ) - \cL_h^t(\tilde \theta ) \bigr |
&\le \eta T \max_{i\in[t-1]}\bigabr{ A_h^{\pi^i, \theta}(s_h^i,b_h^i) - A_h^{\pi^i, \tilde\theta}(s_h^i,b_h^i)}\nend
&\le 2\eta T \max_{i\in[t-1]}\bignbr{ Q_h^{\pi^i, \theta}- Q_h^{\pi^i, \tilde\theta}}_\infty \nend
&\leq 2 T \cdot \rho(\theta, \tilde \theta),
\end{align*}
where the second inequality holds by noting that $|(V_h^{\pi, \theta} - V_h^{\pi, \tilde\theta})(s_h)| \le \onbr{Q_h^{\pi,\theta}-Q_h^{\pi,\tilde\theta}}_\infty$ by the same argument in \eqref{eq:f_2-01} and \eqref{eq:f_2-1}, and the last inequality holds by noting that $\eta\le (\gamma B_A + 1 + \eta)$.
% \Zhuoran{Derive the second one, STOP HERE}
Therefore, 
adding an extra term $3T\epsilon$ to the right-hand side of \eqref{eq:MLE-to-hellinger-1} extends the result to any $\theta\in\Theta$ by definition of the covering net $\Theta_\epsilon$.
We thus obtain for all $\theta\in\Theta, h\in[H], t\in[T]$ with probability $1-\delta$,
\begin{align}
    \frac 1 2 \rbr{-\cL_h^t(\theta) + \cL_h^t(\theta^*)} 
    &\le -  \sum_{i=1}^{t-1} \EE^{\pi^i, \theta^*}D_\H^2\rbr{\nu_h^{\pi^i, \theta}(\cdot\given s_h^i), \nu_h^{\pi^i, \theta^*}(\cdot\given s_h^i)} \nend
    &\qquad + \log\rbr{ H\cN_\rho(\Theta, \epsilon) / \delta} + 3T\epsilon. \label{eq:KL-2-D_H-online}
\end{align}
% As an alternative, we can also take the filtration as $\cF_{h, t-1} = \sigma(\{s_{h}^i \}_{i\in[t]}, \{b_h^i\}_{i\in[t-1]})$ and obtain a similar result
% \begin{align}
%     \frac 1 2 \rbr{-\cL_h^t(\theta) + \cL_h^t(\theta^*)} 
%     &\le - \sum_{i=1}^{t-1} D_\H^2\rbr{\nu_h^{\pi^i, \theta}(\cdot\given s_h^i), \nu_h^{\pi^i, \theta^*}(\cdot\given s_h^i)} + \iota. \label{eq:KL-2-D_H-offline}
% \end{align}
In the sequel, we take $\epsilon=T^{-1}$ and  let $\iota= \log\rbr{ H\cN_\rho(\Theta, T^{-1}) / \delta}+ 3$.
Now, we plug in $\hat\theta_{h,\MLE}=\argmin_{\theta'\in\Theta} \cL_h^t(\theta')$ in \eqref{eq:KL-2-D_H-online}
%  and \eqref{eq:KL-2-D_H-offline},  
and obtain by the nonnegativity of the Hellinger distance that
\begin{align*}
    \cL_h^t(\theta^*) \le \inf_{\theta'\in\Theta} \cL_h^t(\theta' ) + 2\iota \le  \cL_h^t(\hat\theta_{h,\MLE}) + 2\iota  , 
\end{align*}
which guarantees that our confidence set is indeed valid by letting 
$$\beta \ge  2\iota = 2\log(e^3 H\cdot \cN_\rho(\Theta, T^{-1})/\delta). $$
Next, we show that our confidence set is also accurate.
For \eqref{eq:KL-2-D_H-online}, we have that
\begin{align}
    \sum_{i=1}^{t-1} \EE^{\pi^i, \theta^*}D_\H^2\rbr{\nu_h^{\pi^i, \theta}(\cdot\given s_h^i), \nu_h^{\pi^i, \theta^*}(\cdot\given s_h^i)} 
    &\le  \frac 1 2\rbr{\cL_h^t(\theta) - \cL_h^t(\theta^*)} +\iota\nend
    &\le \frac 1 2\rbr{\cL_h^t(\theta) - \cL_h^t(\theta^*)+\beta}, \label{eq:MLE-to-hellinger-2}
\end{align}
% and also 
% \begin{align*}
%     \sum_{i=1}^{t-1} D_\H^2\rbr{\nu_h^{\pi^i, \theta}(\cdot\given s_h^i), \nu_h^{\pi^i, \theta^*}(\cdot\given s_h^i)} \le  \frac 1 2\rbr{\cL_h^t(\theta) - \cL_h^t(\theta^*)} + \log\rbr{\frac{e^2H\cN_\rho(\Theta, T^{-1})}{\delta}}, 
% \end{align*}
Now, if $\theta\in\CI_{h,\Theta}^t(\beta)$, it follows directly from \eqref{eq:MLE-to-hellinger-2} that
\begin{align*}
    \sum_{i=1}^{t-1} \EE^{\pi^i, \theta^*}D_\H^2\rbr{\nu_h^{\pi^i, \theta}(\cdot\given s_h^i), \nu_h^{\pi^i, \theta^*}(\cdot\given s_h^i)} \le \beta, 
\end{align*}
which shows that our confidence set is also valid and gives \eqref{eq:MLE-guarantee-hellinger-2}.
%  since the righthand side can be replaced by $3/2 \beta$ using the fact that $\cL_h^t(\theta) - \cL_h^t(\theta^*)\le \cL_h^t(\theta) - \inf_{\theta'\in\Theta}\cL_h^t(\theta')\le \beta$  for any $\theta\in\CI_\Theta(\beta)$.
We next show how to derive the bound for the Q function. Invoking \Cref{lem:D_H-2-A^2}, we have that
\begin{align}
    8 D_\H^2\rbr{\nu_h^{\pi, \hat\theta}(\cdot\given s_h), \nu_h^{\pi, \theta^*}(\cdot\given s_h)} 
    &\ge \rbr{\frac{\eta}{1+\eta B_A}}^2\cdot \inp[\Big]{\nu_h^{\pi, \theta^*}}{ \orbr{A_h^{\pi, \hat\theta}-A_h^{\pi, \theta^*}}^2} \nend
    &\ge \rbr{\frac{\eta}{1+\eta B_A}}^2\cdot \EE_{s_h}^{\pi, \theta^*}\rbr{\bigrbr{\EE_{s_h, b_h}^{\pi, \theta^*} - \EE_{s_h}^{\pi, \theta^*}}\osbr{A_h^{\pi, \hat\theta}-A_h^{\pi, \theta^*}}}^2\nend
    &= \rbr{\frac{\eta}{1+\eta B_A}}^2\cdot \EE_{s_h}^{\pi, \theta^*}\Bigrbr{\orbr{\EE_{s_h, b_h}^{\pi, \theta^*} - \EE_{s_h}^{\pi, \theta^*}}\osbr{Q_h^{\pi, \hat\theta}-Q_h^{\pi, \theta^*}}}^2, \label{eq:hellinger-2-Q}
\end{align}
where the second inequality follows from the Jensen's inequality, and the last inequality holds by invoking \eqref{eq:A diff-1} in \Cref{lem:AQV-func diff}. One can also swap $\theta^*$ and $\hat\theta$ by the exchangeability of the Hellinger distance and obtain another version.  Note that $C_\eta = \eta^{-1}+B_A$. 
Plugging \eqref{eq:hellinger-2-Q} with $C_\eta$ into the previous accuracy guarantees gives \eqref{eq:MLE_guarantee_Q}.
%  and \eqref{eq:MLE_guarantee_Q-1}. 
Therefore, we have proved \eqref{eq:MLE-guarantee-hellinger-2} and \eqref{eq:MLE_guarantee_Q}. 
For deriving \eqref{eq:MLE-guarantee-hellinger-1} and \eqref{eq:MLE_guarantee_Q-1}, we just change our filtration to $\cF_{h, t-1} = \sigma((s_h^i, b_h^i)_{i\in[t-1]}, s_h^t)$ and everything follows. 

Lastly, we prove the guarantee in \eqref{eq:MLE-guarantee-Q-3}. 
We use \eqref{eq:MLE_guarantee_Q-1} with $\theta'$ replaced by $\theta^*$ and for all $\theta\in\CI_{h, \Theta}^t(\beta)$, 
\begin{align*}
    \sum_{i=1}^{t-1} {\Var_{s_h^i}^{\pi^i, \theta^*} \bigsbr{Q_h^{\pi^i, \theta}(s_h, b_h) - Q_h^{\pi^i, \theta^*}(s_h, b_h)}} \le 4 C_\eta^2 \rbr{\cL_h^t(\theta) - \cL_h^t(\theta^*)+ \beta} \le 8 C_\eta^2 \beta, 
    % \label{eq:MLE_guarantee_Q-1} 
\end{align*}
which is equivalent to saying that for all $h\in[H], \theta\in\CI_{h, \Theta}^t(\beta)$,
\begin{align}
    \sum_{i=1}^{t-1} 
    \EE_{s_h^i}^{\pi^i,\theta^*}\rbr{\rbr{Q_h^{\pi^i, \theta} - Q_h^{\pi^i, \theta*}}(s_h^i, b_h^i)
    -\EE_{s_h^i}^{\pi^i,\theta^*}\sbr{ \rbr{Q_h^{\pi^i, \theta} - Q_h^{\pi^i, \theta^*}}(s_h, b_h)}}^2 \le 8 C_\eta^2 \beta. \label{eq:MLE-Q-ub-1}
\end{align}
Recall the covering net $\Theta_\epsilon$ we constructed before. For all $\theta\in\Theta_\epsilon \cap \CI_{h,\Theta}^t(\beta), h\in[H]$, and a given $t\in[T]$, we have  by a standard martingale concentration in \Cref{cor:martigale concentration} that with probability at least $1-2\delta$,
\begin{align*}
    &\sum_{i=1}^{t-1} 
    \rbr{\rbr{Q_h^{\pi^i, \theta} - Q_h^{\pi^i, \theta^*}}(s_h^i, b_h^i)
    -\EE_{s_h^i}^{\pi^i,\theta^*}\sbr{ \rbr{Q_h^{\pi^i, \theta} - Q_h^{\pi^i, \theta^*}}(s_h, b_h)}}^2 
    \nend
    &\quad \le \frac 3 2\sum_{i=1}^{t-1}\EE_{s_h^i}^{\pi^i,\theta^*} 
    \rbr{\rbr{Q_h^{\pi^i, \theta} - Q_h^{\pi^i, \theta^*}}(s_h^i, b_h)
    -\EE_{s_h^i}^{\pi^i,\theta^*}\sbr{ \rbr{Q_h^{\pi^i, \theta} - Q_h^{\pi^i, \theta^*}}(s_h, b_h)}}^2 \nend
    &\qqquad + 32 B_A^2 \log\rbr{2H \cN_\rho(\Theta, \epsilon)\delta^{-1} }\nend
    &\quad \le 12 C_\eta^2 \beta  +32 B_A^2 \log\rbr{2H \cN_\rho(\Theta, \epsilon)\delta^{-1} }\le 28 C_\eta^2 \beta 
\end{align*}
where the second inequality follows from \eqref{eq:MLE-Q-ub-1}. 
Moreover, we note that
\begin{align*}
    &\rbr{\rbr{Q_h^{\pi^i, \theta} - Q_h^{\pi^i, \theta^*}}(s_h^i, b_h)
    -\EE_{s_h^i}^{\pi^i,\theta^*}\sbr{ \rbr{Q_h^{\pi^i, \theta} - Q_h^{\pi^i, \theta^*}}(s_h, b_h)}}^2 \nend
    &\qqquad - \rbr{\rbr{Q_h^{\pi^i, \tilde\theta} - Q_h^{\pi^i, \theta*}}(s_h^i, b_h)
    -\EE_{s_h^i}^{\pi^i,\theta^*}\sbr{ \rbr{Q_h^{\pi^i, \tilde\theta} - Q_h^{\pi^i, \theta^*}}(s_h, b_h)}}^2\nend
    &\quad \le 8 \cdot \max_{\pi\in\Pi,\theta\in\Theta}\bignbr{Q_h^{\pi,\theta}}_\infty
    \cdot 2 \bignbr{Q_h^{\pi^i, \theta} -  Q_h^{\pi^i, \tilde\theta}}_\infty \le 16 B_A \rho(\theta, \tilde\theta),
\end{align*}
where the last inequality holds by noting that $B_A$ upper bounds $\max_{\theta\in\Theta, \pi\in\Pi, h\in[H]}\onbr{Q_h^{\pi, \theta}}_\infty$ and using the definition of $\rho$. 
% and $\cN_\varrho (\Theta, \epsilon)$ is the covering number of the smallest $\epsilon$-covering number of $\Theta$ with respect to distance 
% \begin{align*}
%     \varrho(\theta, \tilde\theta) \defeq \max_{\pi\in\Pi, h\in[H]}\nbr{Q_h^{\pi, \theta} - Q_h^{\pi, \tilde\theta}}_\infty.
% \end{align*}
As a result, for all $\theta\in\CI_\Theta(\beta), h\in[H]$, and a given $t\in[T]$, we conclude that with probability at least $1-2\delta$,
\begin{align*}
    &\sum_{i=1}^{t-1} 
    \rbr{\rbr{Q_h^{\pi^i, \theta} - Q_h^{\pi^i }}(s_h^i, b_h^i)
    -\EE_{s_h^i}^{\pi^i,\theta^*}\sbr{ \rbr{Q_h^{\pi^i, \theta} - Q_h^{\pi^i, \theta^*}}(s_h, b_h)}}^2  \le 28 C_\eta^2 \beta  + 16 B_A.
\end{align*}
Replace $\delta$ by $\delta/2$, we have for all $\theta\in\CI_\Theta(\beta), h\in[H]$ and a given $t\in[T]$ with probability at least $1-\delta$ that 
\begin{align*}
    &\sum_{i=1}^{t-1} 
    \rbr{\rbr{Q_h^{\pi^i, \theta} - Q_h^{\pi^i }}(s_h^i, b_h^i)
    -\EE_{s_h^i}^{\pi^i,\theta^*}\sbr{ \rbr{Q_h^{\pi^i, \theta} - Q_h^{\pi^i, \theta^*}}(s_h, b_h)}}^2 
    \nend
    &\quad \le 28 C_\eta^2 \beta + 56 C_\eta^2\log 2 +16 B_A =  \cO(C_\eta^2 \beta).
\end{align*}
which 
finishes the proof of \Cref{lem:MLE-formal}.
\end{proof}

\subsection{Proof of \Cref{lem:MLE-indep-data}}\label{sec:proof-MLE-indep-data}

Here, we show the guarantee for the MLE with independently collected dataset.
Since each trajectory is independently collected, we are able to use the Bernstein inequality for indepedent random variables $Z_h^i = D_\H^2\orbr{\nu_h^{\pi^i, \theta}(\cdot\given s_h^i), \nu_h^{\pi^i, \theta^*}(\cdot\given s_h^i)}$, 
\#
    \abr{\frac 1 T \sum_{i=1}^T Z_h^i - \EE_\cD\sbr{Z_h^i}} &\le \sqrt{\frac{4\sum_{i=1}^T\Var[Z_h^i]\log(2\delta^{-1})}{T^2}} + \frac{4\log(2\delta^{-1})}{3T} \nend
    &\le \sqrt{\frac{4\sum_{i=1}^T \EE_\cD[Z_h^i]\log(2\delta^{-1})}{T^2}} + \frac{4\log(2\delta^{-1})}{3T}\nend
    &\le \frac{1}{2T}\sum_{i=1}^T \EE_\cD [Z_h^i] + \frac{2 \log(2\delta^{-1})}{T} + \frac{4\log(2\delta^{-1})}{3T},\notag
\#
where the second inequality holds by noting that $\Var[Z_h^i] \le \EE_\cD[(Z_h^i)^2]\le \EE_\cD[Z_h^i]$ by using the property that Hellinger distance is always upper bounded by $1$. We now conclude by further taking a union bound over $h\in[H]$ and $\theta\in\Theta$ that
\#
\frac{1}{T} \sum_{i=1}^T \EE_\cD[Z_h^i] &\le \frac{2}{T} \sum_{i=1}^T Z_h^i  + \frac{8\log(2H \cN_\rho(\Theta, \epsilon)\delta^{-1})}{T} + 6\epsilon\nend
&\le \frac{2}{T} \sum_{i=1}^T Z_h^i  + \frac{8\log(2eH \cN_\rho(\Theta, T^{-1})\delta^{-1})}{T},\notag
\#
where the last inequality holds by taking $\epsilon=T^{-1}$. Plug in the definition of $Z_h^i$, we have
\begin{align*}
    \frac 1 T\sum_{i=1}^T\EE_\cD\sbr{D_\H^2\rbr{\nu_h^{\pi^i, \theta}(\cdot\given s_h^i), \nu_h^{\pi^i, \theta^*}(\cdot\given s_h^i)}} &\le \frac{2}{T} \sum_{i=1}^T D_\H^2\rbr{\nu_h^{\pi^i, \theta}(\cdot\given s_h^i), \nu_h^{\pi^i, \theta^*}(\cdot\given s_h^i)}  \nend
    &\qquad + \frac{8\log(2eH \cN_\rho(\Theta, T^{-1})\delta^{-1})}{T}.
\end{align*}
Using \eqref{eq:MLE-guarantee-hellinger-1} in \Cref{lem:MLE-formal} for any $\theta\in\cC_{\Theta}(\beta)$, we give 
\begin{align}
    \sum_{i=1}^{T} D_\H^2\bigrbr{\nu_h^{\pi^i, \theta}(\cdot\given s_h^i), \nu_h^{\pi^i, \theta^*}(\cdot\given s_h^i)} 
        %%%%%%%%%%
        &\le  \frac 1 2\rbr{\cL_h(\theta) - \cL_h(\theta^*)} + \log\rbr{\frac{eH\cN_\rho(\Theta, T^{-1})}{\delta}} \nend
        %%%%%%%%%
        &\le \frac 1 2\rbr{\cL_h(\theta) - \inf_{\theta'\in\Theta}\cL_h(\theta')} + \log\rbr{\frac{eH\cN_\rho(\Theta, T^{-1})}{\delta}}\nend
        &\le \frac 3 2 \beta, \label{eq:OffGM-nu-hellinger-1}
\end{align}
where the last inequality is just by definition of $\CI_\Theta(\beta)$ in \Cref{eq:behavior_model_confset-1}. Therefore, 
\begin{align*}
    \sum_{i=1}^T\EE_\cD\sbr{D_\H^2\rbr{\nu_h^{\pi^i, \theta}(\cdot\given s_h^i), \nu_h^{\pi^i, \theta^*}(\cdot\given s_h^i)}}  \le 3\beta + {8\log(2eH \cN_\rho(\Theta, T^{-1})\delta^{-1})} \le 11\beta, 
\end{align*}
with probability at least $1-2\delta$ for all $h\in[H]$ and $\theta\in\CI_\Theta(\beta)$. The validity guarantee is already shown in \Cref{lem:MLE-formal}.
We complete the proof of \Cref{lem:MLE-indep-data}.

% \Cref{thm:11_6_gyorfi}, where we take $Z_h^i = D_\H^2\rbr{\nu_h^{\pi^i, \theta}(\cdot\given s_h^i), \nu_h^{\pi^i, \theta^*}(\cdot\given s_h^i)}$ and take $g(Z)=Z$. One can verify that $g\in[0, 1]$ with covering number $\cN_\infty(\epsilon, \cG)=1$.
% Hence, we conclude with $\epsilon = 1/3$, $\alpha=120\log(4/\delta) T^{-1}$ that for any fixed $\theta\in\Theta$, 
% \begin{align*}
%     \PP\rbr{\frac 1 T \sum_{i=1}^T Z_h^i > 2 \EE_\cD Z_h^i  + \frac{60\log(4/\delta)}{T}} \le \delta,
% \end{align*}
% Now, we take a union bound over the $\epsilon$-covering net for $\Theta$ with respect to $\rho$ and also over $h\in[H]$ and obtain with probability at least $1-\delta$ that for any $\theta\in\Theta$, $h\in[H]$ 
% \begin{align*}
%     \frac 1 T \sum_{i=1}^T D_\H^2\rbr{\nu_h^{\pi^i, \theta}(\cdot\given s_h^i), \nu_h^{\pi^i, \theta^*}(\cdot\given s_h^i)} \le 2 \EE_\cD D_\H^2\rbr{\nu_h^{\pi, \theta}(\cdot\given s_h), \nu_h^{\pi, \theta^*}(\cdot\given s_h)} + \frac{1+ 60\log(4\cN_\rho(\Theta, T^{-1})/\delta)}{T}, 
% \end{align*}
% where we can use the same covering number for $\Theta$ since $\rho(\theta, \tilde\theta)$ can still bound the difference in the squared Hellinger distance. Here, the expectation on the right hand side is taken with respect to both the randomness in $\pi$ and $s_h$.

\subsection{Proof of \Cref{lem:leader-bellman-loss}}\label{sec:proof-leader-bellman-loss}
In the following proof, we always consider the expectation to be taken with respect to the data generating distribution.
We first prove the following concentration result: for any $h\in[H]$ and any $y=(\theta_{h+1}, \pi_{h+1}, U_{h+1}, U_{h})\in \cY_h = \Theta_{h+1}\times \Pi_{h+1}\times \cU^2$, 
it holds with probability at least $1-\delta$ that 
      \begin{align}
        &\abr{T\EE[(U_{h} - \TT_{h}^{\pi,\theta}U_{h + 1})^2] - \ell_{h}(U_{h}, U_{h + 1}, \theta, \pi) + \ell_{h}(\TT_{h}^{\pi,\theta}U_{h + 1}, U_{h + 1}, \theta, \pi)} \notag\\
        &\qquad \le \epsilon_S + \frac{T}{2} \EE[(U_{h} - \TT_{h}^{\pi,\theta}U_{h + 1})^2].\label{eq:cY-confset-1}
      \end{align}
     where
    $
    {110 B_U^2\cdot\log(H \max_{h\in[H]}\cN_\rho(\cY_h, T^{-1})\delta^{-1}) } \cdot {T^{-1}} + (45 B_U^2 + 60 B_U )T^{-1}
    $ and $B_U=H$ is the upper bound for the function class $U$.

\paragraph{Concentration. }
Our proof is an adaptation of Lemma D.2 in \citep{lyu2022pessimism}, although we simplify a little bit by directly using the covering number for a joint class $\cY_h = \Pi_{h+1}\times\Theta_{h+1}\times\cU^2$. We take an $\epsilon$-covering net $\cY_\epsilon$ for $\cY$ with respect to distance $\rho$ specified by \eqref{eq:rho-cY}. 
We first use \Cref{lem:bernstein},  
where we take 
\begin{align*}
    {Z_h^i} &= \ell_{h}(U'_{h}, U_{h + 1}, \theta, \pi) - \ell_{h}(\TT_{h}^{\pi,\theta}U_{h + 1}, U_{h + 1}, \theta, \pi)\nend
    & = \rbr{U_h(s_h^i, a_h^i, b_h^i) - u_h^i -  T_{h+1}^{\pi,\theta} U_{h+1}(s_{h+1}^i)}^2  - \rbr{\TT_h^{\pi, \theta} U_{h+1}(s_h^i, a_h^i, b_h^i) - u_h^i - T_{h+1}^{\pi,\theta} U_{h+1}(s_{h+1}^i)}^2.
\end{align*}
Here, we recall the definition of $\TT_h^{\pi}$ given by \eqref{eq:bellman_operator_leader}.
One can verify that $|{Z_h^i}|\le 9B_U^2$ where $B_U$ bounds both the leader's reward and the value function class $\cU$.
We first calculate the expectation of $Z_h^i$, 
\begin{align*}
    \EE[{Z_h^i}] 
    &= \EE \bigg[\EE_{s_h^i, a_h^i, b_h^i} \Big[\rbr{U_h(s_h^i, a_h^i, b_h^i) - u_h^i -  T_{h+1}^{\pi,\theta} U_{h+1}(s_{h+1}^i)}^2  \nend
    &\qquad - \rbr{\TT_h^{\pi, \theta} U_{h+1}(s_h^i, a_h^i, b_h^i) - u_h^i - T_{h+1}^{\pi,\theta} U_{h+1}(s_{h+1}^i)}^2\Big]\bigg]\nend
    & = \EE\bigg[\Bigrbr{U_h(s_h^i, a_h^i, b_h^i)- \TT_h^{\pi, \theta} U_{h+1}(s_h^i, a_h^i, b_h^i)}\nend
    &\qquad \cdot \EE_{s_h^i, a_h^i, b_h^i}\sbr{U_h(s_h^i, a_h^i, b_h^i) + \TT_h^{\pi, \theta} U_{h+1}(s_h^i, a_h^i, b_h^i)- 2u_h^i - 2T_{h+1}^{\pi,\theta} U_{h+1}(s_{h+1}^i) }\bigg]\nend
    & = \EE\sbr{\Bigrbr{U_h(s_h^i, a_h^i, b_h^i)- \TT_h^{\pi, \theta} U_{h+1}(s_h^i, a_h^i, b_h^i)}^2}, 
\end{align*}
where $\EE_{x}[\cdot]$ is a short hand of $\EE[\cdot\given x]$ and the expectation is taken with respect to the data generating distribution. Here, the second equality holds by the law of total expectation, and the third equality holds by noting that $\EE_{s_h^i, a_h^i, b_h^i}\osbr{u_h^i  + T_h^{\pi,\theta} U_{h+1}(s_{h+1}^i)} = \TT_h^{\pi, \theta} U_{h+1} (s_h^i, a_h^i, b_h^i)$.
Next, we calculate the variance, 
\begin{align*}
    \Var[{Z_h^i}] &\le \EE[{Z_h^i}^2] \nend
    &\le \EE\bigg[\Bigrbr{U_h(s_h^i, a_h^i, b_h^i)- \TT_h^{\pi, \theta} U_{h+1}(s_h^i, a_h^i, b_h^i)}^2\nend
    &\qquad \cdot \EE_{s_h^i, a_h^i, b_h^i}\sbr{\rbr{U_h(s_h^i, a_h^i, b_h^i) + \TT_h^{\pi, \theta} U_{h+1}(s_h^i, a_h^i, b_h^i)- 2u_h^i - 2T_{h+1}^{\pi,\theta} U_{h+1}(s_{h+1}^i)}^2 } \bigg]\nend
    &\le 49 B_U^2 \EE\sbr{\Bigrbr{U_h(s_h^i, a_h^i, b_h^i)- \TT_h^{\pi, \theta} U_{h+1}(s_h^i, a_h^i, b_h^i)}^2} = 49 B_U^2 \EE[{Z_h^i}].
\end{align*}
Now, by \Cref{lem:bernstein}, we have for each $y\in\cY_\epsilon$ that
\begin{align*}
    \abr{\frac 1 T \sum_{i=1}^T {Z_h^i} - \EE\sbr{{Z_h^i}}} &\le \frac{1}{2T}\sum_{i=1}^T \EE [{Z_h^i}] + \frac{110 B_U^2\cdot\log(2\delta^{-1})}{T}.
\end{align*}
Now, we extend the result to $\cY$, where we notice that for any two $y, \tilde y\in\cY$ such that $\rho(y, \tilde y) \le \epsilon$, 
\begin{align*}
    &\bigrbr{U_h(s_h^i, a_h^i, b_h^i) - u_h^i -  T_{h+1}^{\pi,\theta} U_{h+1}(s_{h+1}^i)}^2  - \bigrbr{\TT_h^{\pi, \theta} U_{h+1}(s_h^i, a_h^i, b_h^i) - u_h^i - T_{h+1}^{\pi,\theta} U_{h+1}(s_{h+1}^i)}^2 \nend
    &\qquad - \rbr{\bigrbr{\tilde U_h(s_h^i, a_h^i, b_h^i) - u_h^i -  T_{h+1}^{\tilde\pi,\tilde\theta} \tilde U_{h+1}(s_{h+1}^i)}^2  - \bigrbr{\TT_h^{\tilde\pi, \tilde\theta} \tilde U_{h+1}(s_h^i, a_h^i, b_h^i) - u_h^i - T_{h+1}^{\tilde\pi,\tilde\theta} \tilde U_{h+1}(s_{h+1}^i)}^2}\nend
    %%%%%%%%%%%%%
    &\quad \le 6 B_U \rbr{\onbr{U-\tilde U}_\infty + \bignbr{(T_{h+1}^{\pi,\theta}-T_{h+1}^{\tilde\pi,\tilde\theta} )\tilde U_{h+1}}_\infty + \bignbr{T_{h+1}^{\pi,\theta}(U_{h+1} - \tilde U_{h+1})}_\infty}\nend
    &\qquad + 6 B_U \cdot 2\rbr{\bignbr{(T_{h+1}^{\pi,\theta}-T_{h+1}^{\tilde\pi,\tilde\theta} )\tilde U_{h+1}}_\infty + \bignbr{T_{h+1}^{\pi,\theta}(U_{h+1} - \tilde U_{h+1})}}\nend
    &\quad\le 6 B_U(2\epsilon + B_U \epsilon ) + 12 B_U(B_U\epsilon + \epsilon)\nend
    &\quad\le (18 B_U^2 + 24 B_U)\epsilon, 
\end{align*}
where the second inequality follows from the definition of the covering net with respect to distance $\rho$ defined in \eqref{eq:rho-cY}. 
We obtain with probability at least $1-\delta$ that for any $y\in\cY_h$, $h\in[H]$, 
\begin{align*}
    \abr{\frac 1 T \sum_{i=1}^T {Z_h^i} - \EE\sbr{{Z_h^i}}} &\le \inf_{\epsilon>0} \frac{1}{2T}\sum_{i=1}^T \EE [{Z_h^i}] + \frac{110 B_U^2\cdot\log(H \cN_\rho(\cY_h, \epsilon)\delta^{-1})}{T} +2.5\cdot (18B_U^2 + 24 B_U)\epsilon\nend
    &\le \frac{1}{2T}\sum_{i=1}^T \EE [{Z_h^i}] + {110 B_U^2\cdot\log( H \cN_\rho(\cY_h, T^{-1})\delta^{-1}) } \cdot {T^{-1}} + (45 B_U^2 + 60 B_U )T^{-1}\nend
    & = \frac 1 T \rbr{\epsilon_S + \frac{T}{2} \EE[(U_{h} - \TT_{h}^{\pi,\theta}U_{h + 1})^2]}, 
\end{align*}
where $\epsilon_S = {110 B_U^2\cdot\log(H \max_{h\in[H]}\cN_\rho(\cY_h, T^{-1})\delta^{-1}) }  + (45 B_U^2 + 60 B_U )$, 
which proves our claim in \eqref{eq:cY-confset-1}.

\paragraph{Guarantee of the Confidence Set $\CI_{\cU}^{\pi,\theta}(\beta)$.}
We give a brief proof for the validity and the accuracy of the confidence set. For any $U_{h+1}\in\cU, \theta\in\Theta, \pi\in\Pi, h\in[H]$, on the one hand,
\begin{align}
    \ell_h(U_h, U_{h+1},\theta, \pi) - \inf_{U_h'\in\cU} \ell_h(U_h', U_{h+1}, \theta,\pi) \le \epsilon_S - \frac{T}{2} \EE[\orbr{U_{h} - \TT_{h}^{\pi,\theta}U_{h + 1}}^2], \label{eq:Bellman-loss-guarantee-1}
\end{align}
on the other hand,
\begin{align}
    \ell_h(U_h, U_{h+1},\theta, \pi) - \inf_{U_h'\in\cU} \ell_h(U_h', U_{h+1}, \theta,\pi) 
    &\ge \ell_h(U_h, U_{h+1},\theta, \pi) - \ell_h(\TT_h^{\pi, \theta} U_{h+1},  U_{h+1}, \theta,\pi) \nend
    &\ge - \epsilon_S  + \frac{T}{2} \EE[\orbr{U_{h} - \TT_{h}^{\pi,\theta}U_{h + 1}}^2], \label{eq:Bellman-loss-guarantee-2}
\end{align}
where the first inequality holds by the completeness assumption. 
For \eqref{eq:Bellman-loss-guarantee-1}, we plug in $U=U^{\pi,\theta}$ (realizability) and obtain 
$$\ell_h(U^{\pi,\theta}_h, U^{\pi,\theta}_{h+1},\theta, \pi) - \inf_{U_h'\in\cU} \ell_h(U_h', U^{\pi,\theta}_{h+1}, \theta,\pi) \le \epsilon_S.$$
Therefore, by having $\beta \ge \epsilon_S$, we have $U^{\pi,\theta}\in\CI_\cU^{\pi,\theta}(\beta)$. For the second one in \eqref{eq:Bellman-loss-guarantee-2}, we plug in any $U\in\CI_{\cU}^{\pi,\theta}(\beta)$ and obtain $\EE[\|U_{h} - \TT_{h}^{\pi,\theta}U_{h + 1}\|^2]\le T^{-1}\cdot (2\beta +2\epsilon_S)\le 4\beta T^{-1}$. Hence, we complete our proof of \Cref{lem:leader-bellman-loss}.

\subsection{Proof of \Cref{lem:CI-U-online}}\label{sec:proof-CI-U-online}
Our proof follows a similar scheme as in the proof of \Cref{lem:leader-bellman-loss}.

\paragraph{Concentration.}
We first take
\begin{align*}
    {Z_h^i} &= \ell_{h}(U'_{h}, U_{h + 1}, \theta_{h+1}, \pi) - \ell_{h}(\TT_{h}^{*,\theta}U_{h + 1}, U_{h + 1}, \theta_{h+1}, \pi)\nend
    & = \rbr{U_h(s_h^i, a_h^i, b_h^i) - u_h^i -  T_{h+1}^{*, \theta} U_{h+1}(s_{h+1}^i)}^2  - \rbr{\TT_h^{*, \theta} U_{h+1}(s_h^i, a_h^i, b_h^i) - u_h^i - T_{h+1}^{*, \theta} U_{h+1}(s_{h+1}^i)}^2.
\end{align*}
For the online setting, we have $(s_h^i, a_h^i, b_h^i, s_{h+1}^i)$ adapted to some filtration $\cF_h^i$. One choice of the filtration is $\cF_h^j=\sigma\rbr{\tau^{1:j-1}}$. Another choice of the filtration is $\cF_h^j=\sigma\rbr{(s_h^j, a_h^j, b_h^j), (s_h^i, a_h^i, b_h^i, s_{h+1}^i)_{i\in[j-1]}}$, where $\sigma(X)$ is the sigma-algebra of $X$.
They will both work for our proof.
We let $\EE^{i}_{x}[\cdot]$ be a short hand of $\EE^{i}[\cdot\given x, \cF_h^i]$ in the following proof.
We first calculate the expectation, 
\begin{align*}
    \EE^{i}[{Z_h^i}] 
    &= \EE^{i} \bigg[\EE^{i}_{s_h^i, a_h^i, b_h^i} \Big[\rbr{U_h(s_h^i, a_h^i, b_h^i) - u_h^i -  T_{h+1}^{*, \theta} U_{h+1}(s_{h+1}^i)}^2  \nend
    &\qquad - \rbr{\TT_h^{*, \theta} U_{h+1}(s_h^i, a_h^i, b_h^i) - u_h^i - T_{h+1}^{*, \theta} U_{h+1}(s_{h+1}^i)}^2\Big]\bigg]\nend
    & = \EE^{i}\bigg[\Bigrbr{U_h(s_h^i, a_h^i, b_h^i)- \TT_h^{*, \theta} U_{h+1}(s_h^i, a_h^i, b_h^i)}\nend
    &\qquad \cdot \EE^{i}_{s_h^i, a_h^i, b_h^i}\sbr{U_h(s_h^i, a_h^i, b_h^i) + \TT_h^{*, \theta} U_{h+1}(s_h^i, a_h^i, b_h^i)- 2u_h^i - 2T_{h+1}^{*, \theta} U_{h+1}(s_{h+1}^i) }\bigg]\nend
    & = \EE^{i}\sbr{\Bigrbr{U_h(s_h^i, a_h^i, b_h^i)- \TT_h^{*, \theta} U_{h+1}(s_h^i, a_h^i, b_h^i)}^2}, 
\end{align*}
where the second equality holds by the law of total expectation, and the third equality holds by noting that $\EE^{i}_{s_h^i, a_h^i, b_h^i}\osbr{u_h^i  + T_{h+1}^{*, \theta} U_{h+1}(s_{h+1}^i)} = \TT_h^{*, \theta} U_{h+1} (s_h^i, a_h^i, b_h^i)$.
Next, we calculate the variance, 
\begin{align*}
    \Var^{i}[{Z_h^i}^2] &\le \EE^{i}[{Z_h^i}^2]\nend
    &\le \EE^{i}\bigg[\Bigrbr{U_h(s_h^i, a_h^i, b_h^i)- \TT_h^{*, \theta} U_{h+1}(s_h^i, a_h^i, b_h^i)}^2\nend
    &\qquad \cdot \EE^{i}_{s_h^i, a_h^i, b_h^i}\sbr{\rbr{U_h(s_h^i, a_h^i, b_h^i) + \TT_h^{*, \theta} U_{h+1}(s_h^i, a_h^i, b_h^i)- 2u_h^i - 2T_{h+1}^{*, \theta} U_{h+1}(s_{h+1}^i)}^2 } \bigg]\nend
    &\le 49 B_U^2 \EE^{i}\sbr{\Bigrbr{U_h(s_h^i, a_h^i, b_h^i)- \TT_h^{*, \theta} U_{h+1}(s_h^i, a_h^i, b_h^i)}^2} = 49 B_U^2 \EE^{i}[{Z_h^i}].
\end{align*}
Also, one can verify that $|{Z_h^i}|\le 9B_U^2$ where $B_U$ bounds both the leader's reward and the value function class $\cU$.
We next take a $\epsilon$-covering of the class $\cZ_h=\cU^2\times\Theta_{h+1}$ with respect to the following distance defined in \eqref{eq:rho-cZ},
\begin{align*}
    &\rho\orbr{z, \tilde z}  = \max_{h\in[H]}\cbr{\bignbr{U_h-\tilde U_h}_\infty, \bignbr{ T_{h+1}^{*,\theta} U_{h+1} (\cdot) -  T_{h+1}^{*,\tilde\theta} \tilde U_{h+1} (\cdot)}_\infty }, 
\end{align*} 
We invoke the Freedman inequality \Cref{lem:freedman} for the martingale sequence $Z_h^i -\EE^i[Z_h^i]$, which says that for all $t\in [T], h\in[H], z\in\cZ_\epsilon$, it holds with probability at least $1-\delta$
\begin{align*}
    \sum_{i=1}^t \rbr{Z_h^i - \EE^i[Z_h^i]}
    &\le \frac{\lambda(e-2)}{9 B_U^2} \sum_{i=1}^t \EE^i\sbr{\rbr{Z_h^i - \EE^i[Z_h^i]}^2} + 9 \lambda^{-1} B_U^2\log(TH\cN_\rho(\cZ_h, \epsilon)\delta^{-1})\nend
    &\le \frac{\lambda(e-2) 49 }{9 } \sum_{i=1}^t \EE^i\sbr{Z_h^i} + 9 \lambda^{-1} B_U^2\log(TH\cN_\rho(\cZ_h, \epsilon)\delta^{-1}), \quad \forall \lambda\in(0, 1).
\end{align*}
Here, we plug in $\lambda = 9/98(e-2)$, $\epsilon=T^{-1}$ and notice that the above inequality also holds for $-Z_h^i + \EE^i[Z_h^i]$, which gives us for all $z\in\cZ_\epsilon, h\in[H], t\in[T]$ that
\begin{align}\label{eq:leader-Bellman-1}
    \abr{\sum_{i=1}^t \rbr{Z_h^i - \EE^i[Z_h^i]}} \le \frac 1 2 \sum_{i=1}^t \EE^i\sbr{Z_h^i} + c B_U^2\log(TH\cN_\rho(\cZ_h, T^{-1})\delta^{-1}), 
\end{align}
where we plug in $\epsilon=T^{-1}$ and $c=98(e-2)$ should be a universal constant. 
Next, we notice that
\begin{align*}
    &\bigrbr{U_h(s_h^i, a_h^i, b_h^i) - u_h^i -  T_{h+1}^{*,\theta} U_{h+1}(s_{h+1}^i)}^2  - \bigrbr{\TT_h^{*, \theta} U_{h+1}(s_h^i, a_h^i, b_h^i) - u_h^i - T_{h+1}^{*,\theta} U_{h+1}(s_{h+1}^i)}^2 \nend
    &\qquad - \rbr{\bigrbr{\tilde U_h(s_h^i, a_h^i, b_h^i) - u_h^i -  T_{h+1}^{*,\tilde\theta} \tilde U_{h+1}(s_{h+1}^i)}^2  - \bigrbr{\TT_h^{*, \tilde\theta} \tilde U_{h+1}(s_h^i, a_h^i, b_h^i) - u_h^i - T_{h+1}^{*,\tilde\theta} \tilde U_{h+1}(s_{h+1}^i)}^2}\nend
    %%%%%%%%%%%%%
    &\quad \le 6 B_U \rbr{\onbr{U-\tilde U}_\infty + \bignbr{(T_{h+1}^{*,\theta}-T_{h+1}^{*,\tilde\theta} )\tilde U_{h+1}}_\infty + \bignbr{T_{h+1}^{*,\theta}(U_{h+1} - \tilde U_{h+1})}_\infty}\nend
    &\qquad + 6 B_U \cdot 2\rbr{\bignbr{(T_{h+1}^{*,\theta}-T_{h+1}^{*,\tilde\theta} )\tilde U_{h+1}}_\infty + \bignbr{T_{h+1}^{*,\theta}(U_{h+1} - \tilde U_{h+1})}}\nend
    &\quad\le 6 B_U(2\epsilon + B_U \epsilon ) + 12 B_U(B_U\epsilon + \epsilon) \le (18 B_U^2 + 24 B_U)\epsilon, 
\end{align*}
Now, we let $\epsilon_S=c B_U^2 \allowbreak \log(TH\cN_\rho(\cZ_h, T^{-1})\delta^{-1}) + (45 B_U^2 + 60 B_u)$ and extend the result in \eqref{eq:leader-Bellman-1} to the whole class $\cY$, 
\begin{align*}
    \abr{\sum_{i=1}^t \rbr{Z_h^i - \EE^i[Z_h^i]}} 
    &\le \frac 1 2 \sum_{i=1}^t \EE^i\sbr{Z_h^i} + c B_U^2\log(TH\cN_\rho(\cZ, T^{-1})\delta^{-1}) + 2.5\cdot(18 B_U^2 + 24 B_U)\nend
    &\le \frac 1 2 \sum_{i=1}^t \EE^i\sbr{Z_h^i} + \epsilon_S.
\end{align*}
the following argument follows exactly the same as \eqref{eq:Bellman-loss-guarantee-1} and \eqref{eq:Bellman-loss-guarantee-2}, 
where we use the realizability assumption that $U^{*,\theta}\in \cU$ and the completeness assumption that there exists $U'\in\cU$ such that $U'=\TT_h^{*, \theta} U$ for any $U\in\cU, \theta\in\Theta$. 
We finish our proof of \Cref{lem:CI-U-online}.

\subsection{Proof of \Cref{lem:1st-ub}}
\label{sec:proof-1st-ub}
Recall by definition, 
\begin{align*}
    \tilde \Delta^{(1)}_h(s_h, b_h) &=  \rbr{\EE_{s_h, b_h} -\EE_{s_h}}\Biggsbr{\sum_{l=h}^H \gamma^{l-h}\underbrace{\rbr{\tilde Q_l - r_l^\pi - \gamma P_l^\pi \tilde V_{l+1}}(s_l, b_l)}_{\ds\text{Follower's Bellman error}}}.
\end{align*}
In this section, we will bound $\EE\sbr{\abr{\tilde \Delta^{(1)}_h(s_h, b_h)}}$ by the KL distance in the following way.
% \paragraph{For Small $\eta \bigabr{\tilde A- A}$.}
% On the one hand, we have by the Cauchy-Schwartz inequality that
% \begin{align*}
%     \EE \exp\rbr{\eta \bigabr{\tilde A- A}} \cdot \EE \sbr{\exp\rbr{- \eta \bigabr{\tilde A- A}}\cdot \bigabr{\tilde A-A}^2} \ge \rbr{\EE\sbr{\bigabr{\tilde A-A}}}^2.
% \end{align*}
% On the other hand, the second term on the left hand side can be bounded by the Hellinger distance, 
% \begin{align*}
%     \EE D_\H^2(\nu, \tilde\nu) = \EE \inp[\bigg]{\nu}{\biggrbr{1-\sqrt{\frac{\tilde\nu}{\nu}}}^2} \ge \EE \sbr{\eta^2 \exp\rbr{- \eta \bigabr{\tilde A- A}}\cdot \bigabr{\tilde A-A}^2}.
% \end{align*}
% Hence, we conclude that
% \begin{align*}
%     \EE \exp\rbr{\eta \bigabr{\tilde A- A}} \cdot \EE D_\H^2(\nu, \tilde\nu)\ge \eta^2  \cdot \rbr{\EE\bigabr{\tilde A-A}}^2,
% \end{align*}
% and also
% \begin{align}
%     \sum_{l=h}^H \gamma^{l-h}\EE \exp\rbr{\eta \bigabr{\tilde A_l- A_l}} \cdot \sum_{l=h}^H \gamma^{l-h}\EE D_\H^2 \rbr{\nu_l, \tilde\nu_l} \ge \eta^2 \rbr{\sum_{l=h}^H \gamma^{l-h} \EE \bigabr{\tilde A_l-A_l}}^2.\label{eq:A-cauchy}
% \end{align}
We follow from the decomposition of the A-function in \Cref{lem:AQV-func diff},
\begin{align*}
    \abr{\tilde\Delta_h^{(1)}(s_h, b_h)} 
    &\le \bigabr{\bigrbr{A_h-\tilde A_h}(s_h, b_h)} + 2\eta^{-1} \Delta_h^{(2)}(s_h) \nend
    &\le \bigabr{\bigrbr{A_h-\tilde A_h}(s_h, b_h)} + 2\EE_{s_h}\sbr{\sum_{l=h}^H \gamma^{l-h} \inp[]{\nu_l(\cdot\given s_l)}{(A_l-\tilde A_l)(s_l, b_l)}}, 
\end{align*}
where we use the definition $\Delta_h^{(2)}(s_h)\defeq \EE_{s_h}\sbr{\sum_{l=h}^H \gamma^{l-h} \kl\infdivx[]{\nu_l}{\tilde\nu_l}}$ and the last inequality holds by noting that $\kl\infdivx[]{\nu}{\tilde\nu} = \eta\inp[]{\nu}{A-\tilde A}$.
Therefore, we conclude that
\begin{align}
    \EE\sbr{\abr{\tilde \Delta^{(1)}_h(s_h, b_h)}} 
    &\le  \EE\bigabr{\bigrbr{A_h-\tilde A_h}(s_h, b_h)} + 2\EE\sbr{\sum_{l=h}^H \gamma^{l-h} \inp[]{\nu_l(\cdot\given s_l)}{(A_l-\tilde A_l)(s_l, \cdot)}}
    \nend
    &
    \le 3 \sum_{l=h}^H \gamma^{l-h} \EE\bigabr{(A_l-\tilde A_l)(s_l, b_l)} \label{eq:Delta-A}
    % \\
    % &\le 3 \eta^{-1} \sqrt{\sum_{l=h}^H \gamma^{l-h}\EE \exp\rbr{\eta \bigabr{\tilde A_l- A_l}}} \cdot D_\RL(M^*,\tilde M;\pi),\nonumber
\end{align}
% where the last inequality is a direct result of \eqref{eq:A-cauchy}.
% \paragraph{For Large $\eta \bigabr{\tA - A}$. }
We now invoke the lower bound \eqref{eq:nu-tv-lb-1} in \Cref{lem:response diff} and obtain
\begin{align*}
    D_\TV(\nu_h, \tnu_h) &\ge \frac{1-\exp\rbr{-2\eta B_A}}{4 B_A} \cdot {\EE_{s_h}\bigabr{(\tA_h-A_h)(s_h, b_h)} } \nend
    &\ge  \frac{\eta}{2(1+ 2\eta B_A)} \cdot {\EE_{s_h}\bigabr{(\tA_h-A_h)(s_h, b_h)} }.
\end{align*}
Combining these results, we obtain
\begin{align*}
    \EE\sbr{\abr{\tilde \Delta^{(1)}_h(s_h, b_h)}} &\le 3\sum_{l=h}^H \gamma^{l-h} \EE\bigabr{(\tA_l-A_l)(s_l,b_l)} \nend
    &\le 3 \cdot \rbr{\frac{\eta}{2(1+ 2\eta B_A)}}^{-1} \sum_{l=h}^H \gamma^{l-h}\EE D_\TV(\nu_l(\cdot, s_l),\tilde\nu_l(\cdot, s_l)) \nend
    &\le 6(1+2\eta B_A)\cdot  {\frac{1-\gamma^H}{1-\gamma}}  \cdot \eta^{-1} \max_{h\in[H]} \EE D_\TV(\nu_h(\cdot, s_h),\tilde\nu_h(\cdot, s_h)),
\end{align*}
where the last inequality follows from from the fact that $(1-\exp(-x))/2x\ge 1/2(1+x)$.
% Hence, we have
% \begin{align*}
%     \EE\sbr{\abr{\tilde \Delta^{(1)}_h(s_h, b_h)}} &\le   3 \cdot \frac{4 \eta B_A}{1-\exp\rbr{-2\eta B_A}}\cdot
%     \sqrt{\frac{1-\gamma^H}{1-\gamma}}  \cdot
%     \eta^{-1}D_\RL(M^*,\tilde M;\pi)\nend
%     &\le 6(1+2\eta B_A)\cdot  \sqrt{\frac{1-\gamma^H}{1-\gamma}}  \cdot \eta^{-1}D_\RL(M^*,\tilde M;\pi),
% \end{align*}
Hence, we complete the proof of the first order of $\tilde \Delta^{(1)}$ in \Cref{lem:1st-ub}.

In the sequel, we will study how to upper bound $\bigrbr{\tilde \Delta_h^{(1)}(s_h, b_h)}^2$. We first have by \Cref{lem:AQV-func diff} that
\begin{align*}
    \bigrbr{\tilde \Delta_h^{(1)}(s_h, b_h)}^2 
    &= 2 \rbr{\rbr{\EE_{s_h, b_h}-\EE_{s_h}} \bigsbr{\orbr{A_h - \tilde A_h}(s_h, b_h)}}^2 + 2 \gamma^2 \eta^{-2} \rbr{\rbr{\EE_{s_h, b_h}-\EE_{s_h}}\bigsbr{\Delta_{h+1}^{(2)}(s_{h+1})}}^2\nend
    & \le 2 \rbr{\rbr{\EE_{s_h, b_h}-\EE_{s_h}} \bigsbr{\orbr{Q_h - \tilde Q_h}(s_h, b_h)}}^2 + 4 \gamma^2 \eta^{-2} \rbr{\EE_{s_h, b_h}\bigsbr{\Delta_{h+1}^{(2)}(s_{h+1})}}^2  \nend
    &\qquad + 4 \gamma^2 \eta^{-2} \rbr{\EE_{s_h}\bigsbr{\Delta_{h+1}^{(2)}(s_{h+1})}}^2,
\end{align*}
where the last inequality holds by using \eqref{eq:A diff-1} and note that $\eta^{-1}\kl\infdivx[]{\nu_h}{\tilde\nu_h} = \EE_{s_h}[A_h-\tilde A_h]$. By definition of $\Delta_h^{(2)}(s_h)$, we just focus on the second term and obtain 
\begin{align*}
    \rbr{\eta^{-1} \EE_{s_h, b_h}\bigsbr{\Delta_{h+1}^{(2)}(s_{h+1})} }^2
    &= \rbr{\eta^{-1}\EE_{s_h, b_h}\sbr{\sum_{l=h+1}^H \gamma^{l-h-1} \kl\infdivx[]{\nu_l(\cdot\given s_l)}{\tilde\nu_l(\cdot\given s_l)}}}^2\nend
    & = \rbr{\EE_{s_h, b_h}\sbr{\sum_{l=h+1}^H \gamma^{l-h-1} \inp[\big]{\nu_l(\cdot\given s_l)}{(A_l-\tilde A_l)(s_l, \cdot)}_\cB}}^2\nend
    &\le \eff_H(\gamma) \sum_{l=h+1}^H \gamma^{l-h-1}\rbr{\EE_{s_h, b_h}\sbr{\inp[\big]{\nu_l(\cdot\given s_l)}{\bigabr{(A_l-\tilde A_l)(s_l, \cdot)}}_\cB}}^2,
\end{align*}
where the last inequality follows from the Cauchy-Schwartz inequaltiy and we recall $\eff_H(\gamma) = (1-\gamma^H)/(1-\gamma)$.
We now invoke the lower bound \eqref{eq:nu-tv-lb-1} in \Cref{lem:response diff} and obtain
\begin{align*}
    D_\TV(\nu_h, \tnu_h) \ge \frac{\eta}{2(1+2\eta B_A)} \cdot \inp[\big]{\nu_h(\cdot\given s_h)}{\bigabr{
    \orbr{\tilde A - A}(s_h,\cdot)}}.
\end{align*}
Combining these results, we obtain
\begin{align*}
    &\rbr{\eta^{-1} \EE_{s_h, b_h}\bigsbr{\Delta_{h+1}^{(2)}(s_{h+1})} }^2 \nend
    &\quad \le 4\rbr{\eta^{-1} +2 B_A}^2\eff_H(\gamma) \sum_{l=h+1}^H \gamma^{l-h-1} {\EE_{s_h, b_h}\sbr{D_\H^2(\nu_l(\cdot\given s_l), \tilde\nu_l(\cdot\given s_l))}}, 
\end{align*}
where the inequality holds by using the Jensen's  inequality and move the expectation outside of the square. As a result, 
\begin{align*}
    &\bigrbr{\tilde \Delta_h^{(1)}(s_h, b_h)}^2  \nend
    &\quad \le 2 \rbr{\rbr{\EE_{s_h, b_h}-\EE_{s_h}} \bigsbr{\orbr{Q_h - \tilde Q_h}(s_h, b_h)}}^2 \nend
    &\qqquad + 16 \gamma^2  \rbr{\eta^{-1} +2 B_A}^2\eff_H(\gamma) \sum_{l=h+1}^H \gamma^{l-h-1} {\rbr{\EE_{s_h}+\EE_{s_h, b_h}}\sbr{D_\H^2(\nu_l(\cdot\given s_l), \tilde\nu_l(\cdot\given s_l))}}, 
\end{align*}
which completes our proof of \Cref{lem:1st-ub}.

\subsection{Proof of \Cref{lem:2nd-ub}}
\label{sec:proof-2nd-ub}
In this proof, we remind readers that $Q, A, r^\pi$ are functions from $\cS\times\cB$ to $\RR$, $V:\cS\times\RR$ and $P_h^\pi:\cS\times\cB\rightarrow\Delta(\cS)$. In the sequel, we will neglect the dependence on $s_h, b_h$ for simplicity.
The major part in this proof is to upper bound $\EE\osbr{\orbr{( P_h^\pi-\tilde P_h^{\pi})\tilde V_{h+1}}^2}$ and $\EE\osbr{\orbr{r_h^\pi-\tilde r_h^\pi}^2}$ by $D_{\RL, h}^2$ separately. 
Moreover, we use $B_A$ in \eqref{eq:define_BA} to bound the follower's Q- and A-function. 
We will leave out the dependence on $(s_h, b_h)$ most of the times in the following proof when it does not cause any confusion in the context.

Note that we only have guarantee for $D_\TV^2(\nu_h, \tilde\nu_h)$ by MLE, which cannot directly guarantee that the true utility is identifiable since a constant shift does not change the follower's behavior at all. For the reward to be identifiable, we need an additional linear constraint, namely $\inp{x}{r_h(s_h, a_h, \cdot)}=\varsigma$.
We start with the easier part with the transition kernel.
\begin{align*}
    \EE\osbr{\orbr{( P_h^\pi-\tilde P_h^{\pi})\tilde V_{h+1}}^2}\le 2^2 B_A^2 \EE\sbr{D_\TV^2( P_h^\pi, \tilde  P_h^\pi)}.
\end{align*}
For the follower's reward, we take a real number $\xi$ and have the following decomposition
\begin{align*}
    &\inf_{\xi\in\RR}\EE_{s_h}\abr{r_h^\pi-\tilde r_h^\pi - \xi}  \nend
    &\quad = \inf_{\xi\in\RR}\EE_{s_h}\bigabr{Q_h-\tilde Q_h - \xi - \gamma\bigrbr{ P_h^\pi-\tilde P_h^\pi}\tilde V_{h+1} - \gamma  P_h^\pi\bigrbr{V_{h+1}-\tilde V_{h+1}}}\nend
    &\quad \le \inf_{\xi\in\RR}\EE_{s_h}\bigabr{Q_h-\tilde Q_h - \xi} + \gamma \EE_{s_h}\bigabr{\bigrbr{ P_h^\pi-\tilde P_h^\pi}\tilde V_{h+1} } + \gamma\exp\rbr{2\eta B_A}\EE_{s_h}\bigabr{Q_{h+1} - \tilde Q_{h+1}}, \nend
    &\quad \le \EE_{s_h}\bigabr{A_h-\tilde A_h} + \gamma\EE_{s_h}\bigabr{\bigrbr{ P_h^\pi-\tilde P_h^\pi}\tilde V_{h+1} } + \gamma\exp\rbr{2\eta B_A}\EE_{s_h}\bigabr{Q_{h+1} - \tilde Q_{h+1}}
\end{align*}
where the first inequality holds by the same argument for $V-\tilde V$ in \eqref{eq:f_2-1}, and the second inequality holds simply by plugging $\xi = V_h(s_h) - \tilde V_h(s_h)$. Now, we can plug in the bound for $\EE_{s_h}\oabr{A_h-\tilde A_h}$ in \Cref{lem:response diff} and obtain
\begin{align}
    &\inf_{\xi\in\RR}\EE_{s_h}\abr{r_h^\pi-\tilde r_h^\pi - \xi} \nend
    &\quad \le \underbrace{2(\eta^{-1}+2B_A) D_\TV(\nu_h, \tilde\nu_h) + 2 \gamma B_A \EE_{s_h} D_\TV( P_h^\pi, \tilde P_h^\pi)}_{\ds \sD_h} + \gamma\exp\rbr{2\eta B_A}\EE_{s_h}\bigabr{Q_{h+1} - \tilde Q_{h+1}}.
    % &\le \rbr{2\eta^{-1}(1+2\eta B_A)+2\gamma B_U} D_{\RL,h}(\tilde M,M^*;\pi) + \gamma\exp\rbr{2\eta B_A}\EE_{s_h}\abr{Q_{h+1} - \tilde Q_{h+1}}, 
    \label{eq:f_2-r-diff}
\end{align}
We next show what we can say about the utility when combining the guarantee of \eqref{eq:f_2-r-diff} with the linear constraint $\inp{x}{r_h(s_h, a_h, \cdot)}=\varsigma$. Specifically, we have the following lemma.
\begin{lemma}[Identification of the follower's utility]\label{lem:identification}
    Suppose for $r, \tilde r:\cB\rightarrow \RR$, for some distribution $\nu\in\Delta(\cB)$ such that $\nu>0$, we have $\inf_{\xi\in\RR}\inp{\nu}{\abr{r-\tilde r-\xi}}\le \varepsilon$ and $\inp{x}{r-\tilde r}=0$ hold at the same time for some $x:\cB\rightarrow \RR$ such that $\inp{\ind}{x}\neq 0$. We have
    \begin{align*}
        \inp{\nu}{\abr{r-\tilde r}}\le\rbr{1 + \nbr{\frac x \nu}_\infty \cdot \frac{1}{\abr{\inp{x}{\ind}}}} \epsilon
    \end{align*}
    \begin{proof}
        See \Cref{sec:proof-identification} for a detailed proof.
    \end{proof}
\end{lemma}
With \Cref{lem:identification}, we conclude with $\nbr{\nu}_\infty\ge \exp\rbr{-\eta B_A}$ and $\kappa = \nbr{x}_\infty/|\la x, \ind\ra|$ that
\begin{align*}
    \EE_{s_h}\abr{r_h^\pi-\tilde r_h^\pi} &\le \rbr{1+\exp\rbr{2\eta B_A} \kappa} \bigrbr{\sD_h + \gamma \exp\rbr{2\eta B_A}\cdot\EE_{s_h}\bigabr{Q_{h+1} - \tilde Q_{h+1}}}.
\end{align*}
On the other hand, for the Q-function, we have by \eqref{eq:f_2-Q-ub} that 
\begin{align*}
    \EE_{s_h}\bigabr{Q_h - \tilde Q_h}
    &\le \EE_{s_h}\bigabr{r_h^\pi-\tilde r_h^\pi + \gamma \bigrbr{ P_h^\pi-\tilde P_h^\pi}\tilde V_{h+1}} + \gamma \exp\rbr{2\eta B_A} {\EE_{s_h}\bigabr{Q_{h+1}-\tilde Q_{h+1}}}\nend
    &\le \underbrace{2\rbr{1+\exp\rbr{2\eta B_A} \kappa} }_{\ds c_1}\cdot \sD_h \nend
    &\qquad + \underbrace{\rbr{2+\exp\rbr{2\eta B_A} \kappa}\gamma \exp\rbr{2\eta B_A}}_{\ds c_2}\cdot\EE_{s_h}\bigabr{Q_{h+1} - \tilde Q_{h+1}}, 
\end{align*}
where in the last inequality, we directly upper bound $\EE_{s_h}|\gamma(P_h^\pi - \tilde P_h^\pi)\tilde V_{h+1}|$ by $\sD_h$. 
Therefore, we have by a recursive argument that 
\begin{align*}
    \EE_{s_h}\bigabr{Q_h -\tilde Q_h} \le \sum_{l=h}^H c_2^{l-h} c_1 \EE_{s_h}\sD_l.
\end{align*}
For now, we are able to deal with $(\EE_{s_h}\abr{r_h^\pi-\tilde r_h^\pi})^2$. However, note that what we actually want to get is the version with the square within the expectation $\EE_{s_h}$, i.e.,  $\EE\rbr{r_h^\pi-\tilde r_h^\pi}^2$. Therefore, we need a variance-mean decomposition,
\begin{align*}
    \EE\rbr{r_h^\pi-\tilde r_h^\pi}^2&= \EE\bigrbr{Q_h-\tilde Q_h - \gamma \bigrbr{ P_h^\pi -\tilde P_h^\pi}\tilde V_{h+1} -\gamma  P_h^\pi \bigrbr{V_{h+1}-\tilde V_{h+1}}}^2\nend
    &\le 4\EE\bigsbr{\bigrbr{A_h -\tilde A_h}^2 + \bigrbr{V_h-\tilde V_h}^2 + \gamma^2 \bigrbr{\bigrbr{ P_h^\pi -\tilde P_h^\pi}\tilde V_{h+1}}^2 +\gamma^2 \bigrbr{V_{h+1}-\tilde V_{h+1}}^2}\nend
    &\le 4\EE\bigsbr{\bigrbr{A_h -\tilde A_h}^2} + 16\gamma^2 B_A^2\EE\bigsbr{D_\TV( P_h^\pi,\tilde  P_h^\pi)^2}\nend
    &\qquad + 4\exp\rbr{4\eta B_A}\EE\bigsbr{\bigrbr{\EE_{s_h}\bigabr{Q_h-\tilde Q_h}}^2+ \bigrbr{\EE_{s_{h+1}}\bigabr{Q_{h+1}-\tilde Q_{h+1}}}^2}, 
\end{align*}
where in the first inequality, we use $Q=A+V$, and use the Jensen's inequality to derive the last term.
The last inequality holds by noting the upper bound for difference in the V-function used in \Cref{eq:f_2-1}.
We notice that the first term can be upper bounded by the squared Hellinger distance,
\begin{align*}
    D_\H^2\rbr{\nu_h,\tilde\nu_h} &= \dotp{\nu_h}{\rbr{1-\sqrt\frac{\tilde \nu_h}{\nu_h}}^2}_\cB\nend
    &= \Bigdotp{\nu_h}{\rbr{1-\exp\rbr{\frac \eta 2 (\tilde A_h - A_h)}}^2}_\cB\nend
    &\ge \rbr{\frac{1-\exp\rbr{-\eta B_A}}{2 B_A}}^2  \cdot \bigdotp{\nu_h}{\bigrbr{A_h-\tilde A_h}^2}_\cB \ge \rbr{\frac{\eta}{2}}^2  \cdot \bigdotp{\nu_h}{\bigrbr{A_h-\tilde A_h}^2}_\cB,
\end{align*}
where the first inequality holds by noting that $|1-\exp(x)|\ge (1-\exp(-B))|x|/B$ for any $|x|\le B$.
The last inequality uses the inequality $(1-\exp(-x))\ge x/(1+x)$ for all $x>0$. 
Therefore, we have the follower's squared reward difference bounded by
\begin{align*}
    &\EE\rbr{r_h^\pi -\tilde r_h^\pi}^2 \nend
    &\quad \le 16\eta^{-2} \EE D_\H^2(\nu_h,\tilde\nu_h) + 16\gamma^2 B_A^2\EE D_\TV^2( P_h^\pi,\tilde  P_h^\pi)\nend
    &\qqquad + 4 \exp\rbr{4\eta B_A} \EE\sbr{\rbr{\sum_{l=h}^H c_2^{l-h} c_1 \EE_{s_h}\sD_l}^2 + \rbr{\sum_{l=h+1}^H c_2^{l-h} c_1 \EE_{s_{h+1}}\sD_l}^2}\nend
    &\quad \le 16\eta^{-2} \EE D_\H^2(\nu_h,\tilde\nu_h) + 16\gamma^2 B_A^2\EE D_\TV^2( P_h^\pi,\tilde  P_h^\pi)\nend
    &\qqquad + 8 H \eff_H(c_2)^2 c_1^2 \exp\rbr{4\eta B_A} \max_{h\in[H]}\EE\sbr{{\sD_h}^2}\nend
    &\quad \le 16\eta^{-2} \EE D_\H^2(\nu_h,\tilde\nu_h) + 16\gamma^2 B_A^2\EE D_\TV^2( P_h^\pi,\tilde  P_h^\pi)\nend
    &\qqquad + 8 H \eff_H(c_2)^2 c_1^2\exp\rbr{4\eta B_A} \max_{h\in[H]}\EE\sbr{\rbr{{2(\eta^{-1}+2B_A) D_\TV(\nu_h, \tilde\nu_h) + 2 \gamma B_A \EE_{s_h} D_\TV( P_h^\pi, \tilde P_h^\pi)}}^2}, 
\end{align*}
where in the second inequality, we uses the Cauchy-Schwartz inequality that $\EE(\sum a_l x_l)^2\le \sum a_l \cdot \EE\sum a_l x_l^2 \le (\sum a_l)^2 \cdot \max_l \EE b_l^2$ for constant sequence $a_l>0$. 
In summary, we have
\begin{align*}
    \EE\rbr{r_h^\pi -\tilde r_h^\pi}^2
    &\le {32 H^2 \eff_H(c_2)^2 c_1^2\exp\rbr{4\eta B_A} \rbr{4(\eta^{-1}+2B_A)^2+4\gamma^2 B_A^2 }}  \nend
    & \qquad \cdot \max_{h\in[H]}\cbr{\EE D_\H^2(\nu_h,\tilde\nu_h)+\EE D_\TV^2(P_h^\pi,\tilde P_h^\pi)}\nend
    &\le \underbrace{640 H^2 \eff_H(c_2)^2 c_1^2\exp\rbr{4\eta B_A} (\eta^{-1}+B_A)^2}_{\ds c_3/4}  \nend
    & \qquad \cdot \max_{h\in[H]}\cbr{\EE D_\H^2(\nu_h,\tilde\nu_h)+\EE D_\TV^2(P_h^\pi,\tilde P_h^\pi)}
\end{align*}
Therefore, we conclude that
\begin{align*}
    \max_{h\in[H]}\EE\sbr{ \bigrbr{{\tilde Q_h - r_h^\pi - \gamma P_h^\pi \tilde V_{h+1}}}^2} 
    &\le 2 \max_{h\in[H]} \EE\sbr{\rbr{\tilde r_h^\pi - r_h^\pi}^2} + 2 \gamma^2\max_{h\in[H]} \EE\sbr{\bigrbr{\bigrbr{\tilde P_h^\pi - P_h^\pi}\tilde V_{h+1}}^2}\nend
    &\le c_3 \max_{h\in[H]}\cbr{\EE D_\H^2(\nu_h,\tilde\nu_h)+\EE D_\TV^2(P_h^\pi,\tilde P_h^\pi)}, 
\end{align*}
which completes our proof of \Cref{lem:2nd-ub}
% where the first inequality holds by noting that $C_\eta\le 2(1+2\eta B_A)$. 
% \newpage
% which completes the proof of \Cref{cor:online linear}.


\subsubsection{Proof of \Cref{lem:identification}}\label{sec:proof-identification}
For condition $\inf_{\xi\in\RR}\inp{\nu}{\abr{r-\tilde r-\xi}}\le \varepsilon$, we assume that the infimum is achieved at $\xi^*$. Let $r^* = r -\xi^*$ and we have
\begin{align*}
\abr{\inp{r^*-\tilde r}{x}} \le \inp{\abr{r^*-\tilde r}}{\abr{x}} \le \inp{\abr{r^*-\tilde r}}{\nu} \cdot \nbr{\frac{x}{\nu}}_\infty\le \varepsilon\nbr{\frac{x}{\nu}}_\infty,
\end{align*}
where the second inequality is just a distribution shift and the last inequality is given by the condition. Furthermore, for our target,
\begin{align*}
    \inp{\abr{r-\tilde r}}{\nu} \le \inp{\abr{r^*-\tilde r}}{\nu} + \abr{\xi^*} = \varepsilon + \abr{\inp{r^*-r}{x}} \cdot \frac{1}{\abr{\inp{x}{\ind}}},
\end{align*}
where the inequality follows from the triangle inequality and the equality holds by noting that $\inp{x}{\ind}\neq 0$ and $\inp{\abr{r^*-\tilde r}}{\nu}\le \varepsilon$. We bridge these two inequalities by noting that
\begin{align*}
    \abr{\inp{r^*-\tilde r}{x}} = \abr{\inp{r-\tilde r}{x} + \inp{r^*-r}{x} } = \abr{\inp{r^*-r}{x} },
\end{align*}
where the second inequality holds by noting that $\inp{r-\tilde r}{x}=0$. Combining these results and we have
\begin{align*}
    \inp{\abr{r-\tilde r}}{\nu} \le \epsilon + \abr{\inp{r^*-r}{x}} \cdot \frac{1}{\abr{\inp{x}{\ind}}} \le \rbr{1 + \nbr{\frac x \nu}_\infty \cdot \frac{1}{\abr{\inp{x}{\ind}}}} \epsilon, 
\end{align*}
which completes the proof on \Cref{lem:identification}.


%%%%-----------------Definition/Theorems------------------%%%%
\theoremstyle{definition} %%upright text, extra space above and below
    \newtheorem{definition}{Definition}

\theoremstyle{plain} %% italic text, extra space above and below
    \newtheorem{theorem}[definition]{Theorem}
    \newtheorem{proposition}[definition]{Proposition}
    \newtheorem{lemma}[definition]{Lemma}
    \newtheorem{corollary}[definition]{Corollary}
    \newtheorem{claim}[definition]{Claim}
    \newtheorem{assumption}[definition]{Assumption}

\theoremstyle{remark} %% upright text, no extra space above or below
    \newtheorem{remark}[definition]{Remark}



%%%%-----------------Bibliography------------------%%%%

\usepackage[bibstyle=alphabetic,citestyle=alphabetic,useprefix,giveninits=true, sorting=ynt, sortcites, minbibnames=99,maxbibnames=99,backend=biber]{biblatex}  %%other styles: numeric-comp, authoryear 
    %% sorting: ynt in the text, nyt in the bibliography with additional command there
\renewbibmacro{in:}{} %% removes in: before journals
\bibliography{bibliography}  %% Name of the file with the bibliography
\emergencystretch=1em %% Adjusts the overfulls in the bibliography by allowing more space between words

%%% 
\DeclareSourcemap{
  \maps[datatype=bibtex]{
    \map[overwrite]{ %%If DOI is present, doesn't print arXiv
      \step[fieldsource=doi, final]
      \step[fieldset=url, null]
      \step[fieldset=eprint, null]
      }
     \map[overwrite]{ %% If only arXiv is present, doens't print pages and eid (eliminates duplicates)
      \step[fieldsource=eprint, final]
      \step[fieldset=pages, null]
      \step[fieldset=eid, null]
      \step[fieldset=journal, null]
    }  
  }
}

%%%%-----------------Headers------------------%%%%
    % \usepackage{fancyhdr} %% Headers
    %     \pagestyle{fancy}
    %     \fancyhf{}
    %     \fancyhead[LE,RO]{\thepage}
    %     \fancyhead[RE]{ \nouppercase{\leftmark} }
    %     \fancyhead[LO]{ \nouppercase{\rightmark} }
    %     \setlength{\headheight}{15pt}
    % \usepackage{emptypage}



%%%%-----------------Typing shortcuts------------------%%%%

    \newcommand{\Tr}[0]{\text{Tr}}
    \newcommand{\indices}{}
%% tilde variables 
    \renewcommand{\Tilde}{\widetilde}   
    \newcommand{\tc}{\widetilde{c}}
    \newcommand{\tom}{\widetilde{\omega}}
    \newcommand{\tg}{\widetilde{g}}
    \newcommand{\te}{\widetilde{e}}
    \newcommand{\txi}[1]{\widetilde{\xi}^{#1}}
    \newcommand{\tedl}{\widetilde{\underline{e}}^\dag}
    
%% Bold variables
    \newcommand{\bxi}{\boldsymbol{\xi}}
    \newcommand{\bc}{\mathbf{c}}
    \newcommand{\bom}{\boldsymbol{\omega}}
    
%% double tilde variables
    \newcommand{\ttc}{\widetilde{\tc}}
    \newcommand{\tte}{\widetilde{\te}}
    \newcommand{\ttom}{\widetilde{\tom}}
    \newcommand{\ttxi}[1]{\widetilde{\widetilde{\xi^{#1}}}}
    
%% Generic Math
    \DeclareMathOperator{\Ima}{Im}
    \newcommand{\oloc}{\Omega_{\mathrm{loc}}}
    \newcommand{\paral}{\slash\!\slash}
    \newcommand{\Ker}[1]{\mathrm{Ker}{(#1)}}
    \newcommand{\comp}[1]{\langle #1 \rangle}
    \newcommand{\X}[1]{(X_{#1})}
    \newcommand{\qsp}[2]{\,\ensuremath{\raise.5ex\hbox{$#1$}\big\slash\raise-.5ex\hbox{$#2$}}} 
    \newcommand{\pard}[2]{\frac{\delta#1}{\delta#2}}

%% BV shortcuts
    \newcommand{\FF}{\mathfrak{F}}
    \newcommand{\AKSZ}{\textsf{\tiny AKSZ}}
    \DeclareMathOperator{\BFV}{\mathit{BFV}}
    \DeclareMathOperator{\BV}{\mathit{AKSZ}}
    \newcommand{\dd}{\mathrm{d}}
    \newcommand{\ndash}{\nobreakdash-\hspace{0pt}}

%% Commands for AKSZ
    \newcommand{\Fp}[2]{\mathcal{F}_{#2}^\partial(#1)}
    \newcommand{\Sp}[2]{S_{#2}^\partial(#1)}
    \newcommand{\Qp}[2]{Q_{#2}^\partial(#1)}
    \newcommand{\varp}[2]{\varpi_{#2}^\partial(#1)}
    \newcommand{\alp}[2]{\alpha_{#2}^\partial(#1)}
    \newcommand{\uF}[1]{{\mathcal{F}}_{R}(#1)}
    \newcommand{\uFF}[1]{{\mathfrak{F}}_{R}(#1)}
    \newcommand{\uS}[1]{{S}_{R}(#1)}
    \newcommand{\uV}[1]{{\varpi}_{R}(#1)}
    \newcommand{\uQ}[1]{{Q}_{R}(#1)}
    \newcommand{\UQ}{{Q}_R}

%%filtration symbols
    \newcommand{\filt}[1]{(#1)}
    \newcommand{\filtint}[1]{M^{(#1)}}
    \newcommand{\filtBF}[1]{(#1 )}
    \newcommand{\filtintBF}[1]{M^{(#1 )}}

%%% BF non degenerate
    \newcommand{\BFnd}{BF_{*}}

%%%%-----------------For long computations------------------%%%%
    \makeatletter
        \newcommand{\zzlabel}[1]{\ifmeasuring@\else\ltx@label{#1}\fi} %%new label (necessary if amsmath is present)
    \makeatother

    \newcounter{terms}[equation] %%counter for terms in a equation
    \newcommand{\unl}[2]{\underline{#1}_{\refstepcounter{terms} \zzlabel{#2} \theterms}} %%underlines, counts a term and put the corresponding number. Put the term in the first slot and a label in the second

    \newcommand{\reft}[2]{(\ref{#1}.\ref{#2})} %%refers to a term in a equation as (equation.term)

    %\showlabels[\color{blue}]{zzlabel}  %%shows the labels of the terms



% In probability theory, a filtration refers to a sequence of sigma-algebras that represent the information available at different points in time in a stochastic process.

% Formally, let $(\Omega, \mathcal{F}, (\mathcal{F}t){t\geq 0}, \mathbb{P})$ be a filtered probability space, where $(\Omega, \mathcal{F}, \mathbb{P})$ is a probability space and $(\mathcal{F}t){t\geq 0}$ is a family of sigma-algebras (i.e., collections of events) that satisfy the following conditions:

% $\mathcal{F}_0$ contains all the certain events (i.e., ${\emptyset, \Omega}$).
% $\mathcal{F}_t \subseteq \mathcal{F}$ for all $t \geq 0$ (i.e., each sigma-algebra is a subset of the underlying probability space).
% $\mathcal{F}_s \subseteq \mathcal{F}_t$ for all $0 \leq s \leq t$ (i.e., each sigma-algebra contains all the information in the previous sigma-algebras).
% In this context, a stochastic process $(X_t)_{t\geq 0}$ is said to be adapted to the filtration $(\mathcal{F}t){t\geq 0}$ if for all $t\geq 0$, the random variable $X_t$ is $\mathcal{F}_t$-measurable (i.e., the value of $X_t$ can be determined based on the information available at time $t$).

% Therefore, when we say that $(X_1, X_2, \ldots, X_N)$ is adapted to a filtration $(\mathcal{F}t){t\geq 0}$, it means that each random variable $X_i$ depends only on the information available up to time $t_i$, and not on any future information. In other words, the information available at time $t_i$ is sufficient to determine the value of $X_i$.
\subsection{Other Auxiliary Lemmas}
\begin{lemma}[Lemma A.4 in \citet{foster2021statistical}]\label{lem:freeman-variation}
  For any sequence of real-valued random variables $(X_t)_{t\in[T]}$ adapted to a filtration $(\cF_t)_{t\in[T]}$, it holds with probability at least $1-\delta$ for all $t\in[T]$ that 
  \begin{align*}
      \sum_{i=1}^{t}X_i\le \sum_{i=1}^t \log\rbr{\EE\sbr{e^{X_i}\biggiven \sF_{i-1}}} + \log(\delta^{-1}).
  \end{align*}
\end{lemma}

\begin{lemma}[Bernstein inequality]\label{lem:bernstein}
  For independent random variables $Z_1,\dots, Z_i,\dots, Z_T$ such that $|Z_i-\EE Z_i|\le B$ and $\Var[Z_i]\le L \EE[Z_i]$, we have with probability at least $1-\delta$ that
  \begin{align*}
    \abr{\frac 1 T \sum_{i=1}^T Z_i - \EE\sbr{Z_i}} &\le \frac{1}{2T}\sum_{i=1}^T \EE [Z_i] + \frac{(2L +4B/3)\log(2\delta^{-1})}{T}.
  \end{align*}
\end{lemma}
  \begin{proof}
    By a standard Bernstein inequality, 
\begin{align*}
  \abr{\frac 1 T \sum_{i=1}^T Z_i - \EE\sbr{Z_i}} &\le \sqrt{\frac{4\sum_{i=1}^T\Var[Z_i]\log(2\delta^{-1})}{T^2}} + \frac{4 B \log(2\delta^{-1})}{3T} \nend
    &\le \sqrt{\frac{4 L\sum_{i=1}^T \EE[Z_i]\log(2\delta^{-1})}{T^2}} + \frac{4 B\log(2\delta^{-1})}{3T}\nend
    &\le \frac{1}{2T}\sum_{i=1}^T \EE [Z_i] + \frac{(2L +4B/3)\log(2\delta^{-1})}{T}.
  \end{align*}
  Therefore, we conclude the proof.
\end{proof}


\begin{lemma}[Freedman's inequality, Lemma 9 in \citet{agarwal2014taming}]\label{lem:freedman}
  Let $X_1, X_2, \dotsc, X_T$ be a sequence of real-valued random variables adapted to a filtration $\{\sF_t\}_{t\in[T]}$.
  Assume for all $t \in \{1,2,\dotsc,T\}$, $X_t \leq R$ and
  $\EE[X_t|\sF_{t-1}] = 0$.
  Define $S := \sum_{t=1}^T X_t$ and $V := \sum_{t=1}^T
  \EE[X_t^2|\sF_{t-1}]$.
  For any $\delta \in (0,1)$
  and $\lambda \in [0,1/R]$,
  with probability at least $1-\delta$,
  \begin{equation}
    S \leq (e-2)\lambda V + \frac{\log(1/\delta)}{\lambda}\le \lambda V + \frac{\log(1/\delta)}{\lambda}.\label{eq:freedman ineq}
  \end{equation}
  \end{lemma}

  \begin{corollary}[Martingale concentration derived from Freedman's inequality] \label{cor:martigale concentration}
    Let $X_1, X_2, \dotsc, X_T$ be a sequence of real-valued random variables adapted to a filtration $\{\cF_t\}_{t\in[T]}$.
    Suppose that $|X_t - \EE\sbr{X_t\given \sF_{t-1}}| \le R$ for all $t\in[T]$.
    For any $\delta \in (0,1)$,
  with probability at least $1-\delta$,
  \begin{align}
    \abr{\sum_{t=1}^T X_t - \EE\sbr{X_t\given \sF_{t-1}}} \le  R\sqrt{T} \log(2e\delta^{-1}). \label{eq:martingale-1}
  \end{align}
  Moreover, suppose that $X_t\in[0, R]$ for any $t\in[T]$, with probability at least $1-\delta$, 
  \begin{align}
    \frac 1 2\sum_{t=1}^T \EE\sbr{X_t\given \sF_{t-1}} - 2R\log(2\delta^{-1}) \le \sum_{t=1}^T X_t \le \frac 3 2 \sum_{t=1}^T \EE\sbr{X_t\given \sF_{t-1}} + 2R\log(2\delta^{-1}). \label{eq:martingale-2} 
  \end{align}
  \end{corollary}

\begin{proof}
  We let $Z_t = X_t - \EE[X_t\given \sF_{t-1}]$ and it is obvious that $\EE[Z_t\given \sF_{t-1}] = 0$ and $|Z_t|\le R$.
  For \eqref{eq:martingale-1}, we use \eqref{eq:freedman ineq} for sequence $\{Z_t\}_{t\in[T]}$, and it holds with probability at least $1-\delta/2$
  \begin{align}
    \sum_{t=1}^T X_t - \EE\sbr{X_t\given \sF_{t-1}} 
    &\le  \lambda \sum_{t=1}^T \EE[Z_t^2\given \sF_{t-1}] + \lambda^{-1} \log(2\delta^{-1})\label{eq:martingale-proof-1}\\
    &\le  \lambda T R^2 + \lambda^{-1} \log(2\delta^{-1}). \notag
  \end{align}
  We let $\lambda = (R\sqrt T)^{-1}$ an have with probability $1-\delta/2$ that
  \begin{align*}
    \sum_{t=1}^T X_t - \EE\sbr{X_t\given \sF_{t-1}} \le  R\sqrt{T} \log(2e\delta^{-1}), 
  \end{align*}
  Using \eqref{eq:freedman ineq} again for $\{-Z_t\}_{t\in[T]}$, we obtain the opposite side, which implies that \eqref{eq:martingale-1} holds with probability at least $1-\delta$. 

  For \eqref{eq:martingale-2}, we notice from \eqref{eq:martingale-proof-1} that with probability at least $1-\delta/2$,
  \begin{align*}
    \sum_{t=1}^T X_t - \EE\sbr{X_t\given \sF_{t-1}} 
    &\le  \lambda \sum_{t=1}^T \EE[Z_t^2\given \sF_{t-1}] + \lambda^{-1} \log(2\delta^{-1})\nend
    &\le  \lambda \sum_{t=1}^T \EE[X_t^2\given \sF_{t-1}] + \lambda^{-1} \log(2\delta^{-1})\nend
    & \le \lambda R \sum_{t=1}^T \EE[X_t\given \sF_{t-1}] + \lambda^{-1} \log(2\delta^{-1}), 
  \end{align*}
  where the last inequality holds by the nonnegativity of $X_t$. Applying this inequality to $\{-Z_t\}_{t\in[T]}$ as well, we conclude with probability at least $1-\delta$ that
  \begin{align*}
    \abr{\sum_{t=1}^T X_t - \EE\sbr{X_t\given \sF_{t-1}}} \le \lambda R \sum_{t=1}^T \EE[X_t\given \sF_{t-1}] + \lambda^{-1} \log(2\delta^{-1}).
  \end{align*}
  We take $\lambda = 1/(2R)$, which gives the result in \eqref{eq:martingale-2}.
\end{proof}
% \begin{theorem}[Adaptation of Theorem 11.6 from~\citet{gyorfi2002distribution}]
%     \label{thm:11_6_gyorfi}
%     Let $B \geq 1$ and let $\cG$ be a class of functions $g: \RR^{d} \to [0, B]$. Let $Z_{1}, Z_{2}, \ldots, Z_{K}$ be i.i.d.~$\RR^{d}$-valued random variables. Assume $\alpha > 0$, $0 < \epsilon < 1$, and $K \geq 1$. Then
%     \[
%       \Pr\rbr{\sup_{g \in \cG}\frac{\frac{1}{K}\sum_{j = 1}^{K}g(Z_{j}) - \EE[Z_{j}]}{\alpha + \frac{1}{K}\sum_{j = 1}^{K}g(Z_{j}) + \EE[Z_{j}]} > \epsilon} \leq 4\cN_{\infty}\rbr{\frac{\alpha\epsilon}{5}, \cG}\exp\rbr{-\frac{3\epsilon^{2}\alpha K}{40B}}.
%     \]
%     \end{theorem}

% \begin{lemma}[\textit{Concentration of Self-Normalized Processes \citep{abbasi2011improved}}]
%   Let $\{\cF_t \}^\infty_{t=0}$ be a filtration and $\{\epsilon_t\}^\infty_{t=1}$ be an $\RR$-valued stochastic process such that $\epsilon_t$ is $\sF_{t} $-measurable for all $t\geq 1$.
%   Moreover, suppose that conditioning on $\sF_{t-1}$, 
%    $\epsilon_t $ is a  zero-mean and $\sigma$-sub-Gaussian random variable for all $t\geq 1$, that is,  
%    \$%\label{eq:def_subgaussian} 
%     \EE[\epsilon_t\given \sF_{t-1}]=0,\qquad \EE\bigl[ \exp(\lambda \epsilon_t) \biggiven \sF_{t-1}\bigr]\leq \exp(\lambda^2\sigma^2/2) , \qquad \forall \lambda \in \RR. 
%     \$
%    Meanwhile, let $\{\phi_t\}_{t=1}^\infty$ be an $\RR^d$-valued stochastic process such that  $\phi_t $  is $\sF_{t -1}$-measurable for all $ t\geq 1$. 
%   Also, let  $M_0 \in \RR^{d\times d}$ be a  deterministic positive-definite matrix and 
%   \$
%   M_t = M_0 + \sum_{s=1}^t \phi_s\phi_s^\top
%   \$ for all $t\geq 1$. For all $\delta>0$, it holds that
%   \begin{equation*}
%   \Big\| \sum_{s=1}^t \phi_s \epsilon_s \Big\|_{ M_t ^{-1}}^2 \leq 2\sigma^2\cdot  \log \Bigl( \frac{\det(M_t)^{1/2}\cdot \det(M_0)^{- 1/2}}{\delta} \Bigr)
%   \end{equation*}
%   for all $t\ge1$ with probability at least $1-\delta$.
%   \label{lem:concen_self_normalized}
%   \end{lemma}

\begin{lemma}[Elliptical potential  lemma for vectors, Proposition 1 in \citet{carpentier2020elliptical}]\label{lem:elliptical potential}
Let $u_1,\ldots,u_T$ be a sequence of arbitrary vectors in $\mathbb{R}^d$ such that $\|u_i\|_2 \leq 1$. For any $1\leq t\leq T$, we define
\[
V_t = \sum_{s=1}^{t-1} u_s u_s^\top +\lambda I\, .
\]
It then holds that
\[
  \sum_{t=1}^T \|u_t\|_{V_{t}^{-1}} \le \sum_{t=1}^T \|u_t\|_{V_{t+1}^{-1}} \leq \sqrt{Td\log\left(\frac{T+d\lambda}{d\lambda}\right)}, 
\]
and also that
\[
  \sum_{t=1}^T \|u_t\|_{V_{t}^{-1}}^2 \le 2 \log\rbr{{\det(V_T)}}
\]
\end{lemma}

\begin{lemma}[Elliptical Potential Lemma for matrices]
  \label{lem:potential-matrix}
  Suppose $U_0 = \lambda I_d$, $U_t = U_{t-1} + X_t$ for $X_t\in\SSS_+^d$, and $\trace\rbr{X_t} \leq L$, then
  \begin{equation*}
    \sum_{t=1}^T \sqrt{\trace\rbr{U_{t-1}^{-1} X_t}} \leq \sqrt{\frac{LT/\lambda}{\log\rbr{1+L/\lambda}}\cdot d \log\rbr{1+\frac{LT}{\lambda d}}}, 
  \end{equation*}
  and also
  \begin{align*}
    \sum_{t=1}^T {\trace\rbr{U_{t-1}^{-1} X_t}} \le \frac{L/\lambda}{\log\rbr{1+L/\lambda}}\cdot d \log\rbr{1+\frac{LT}{\lambda d}}
  \end{align*}
\end{lemma}

\begin{proof}
  First, we have the following decomposition,
  \[
      U_t = U_{t-1} + X_tX_t^{\top} = U_{t-1}^{\frac{1}{2}}(I + U_{t-1}^{-1/2}X_tU_{t-1}^{-1/2})U_{t-1}^{\frac{1}{2}}.
  \]

  Taking the determinant on both sides, we get
  \[
  \det(U_t) = \det(U_{t-1}) \det(I + U_{t-1}^{-1/2}X_tU_{t-1}^{-1/2}),
  \]
  which follows from $\det(I_d + X)\ge 1+\trace(X) $ for $X\in\SSS_+^d$ that
  \[
     \det(U_t) = \det(U_{t-1}) \rbr{1 + \trace\rbr{U_{t-1}^{-1} X_t}}.
\]
  By taking advantage of the telescope structure, we have
  \[
    \begin{split}
           & \sum_{t=1}^T \log\rbr{1 + \trace\rbr{U_{t-1}^{-1} X_t}} \leq \log \frac{\det(U_T)}{\det(U_0)} \leq d\log \left(1 + \frac{LT}{\lambda d}\right),
    \end{split}
  \]
  where the last inequality follows from the fact that $\mbox{Tr}(U_T) \leq \mbox{Tr}(U_0) + LT = \lambda d + LT$, and thus $\det(U_T) \leq (\lambda + LT/d)^d$. Moreover, by noting that $\log(1+x)\ge \log(1+B)\cdot x/B$, we have 
  \begin{align*}
    \sum_{t=1}^T \trace\rbr{U_{t-1}^{-1} X_t} &\le \frac{L/\lambda}{\log\rbr{1+L/\lambda}} \cdot \sum_{t=1}^T \log\rbr{1+\trace\rbr{U_{t-1}^{-1} X_t}} \nend
    &\le \frac{L/\lambda}{\log\rbr{1+L/\lambda}}\cdot d \log\rbr{1+\frac{LT}{\lambda d}}.
  \end{align*} 
  Therefore, Cauchy-Schwarz inequality implies,
  \[
    \sum_{t=1}^T \sqrt{\trace\rbr{U_{t-1}^{-1} X_t}} \leq \sqrt{T\sum_{t=1}^T \trace\rbr{U_{t-1}^{-1} X_t}} \leq \sqrt{\frac{LT/\lambda}{\log\rbr{1+L/\lambda}}\cdot d \log\rbr{1+\frac{LT}{\lambda d}}}.
  \]
  Therefore, we conclude the proof.
\end{proof}


\begin{lemma}[Bounding Cumulative Error with Eluder Dimension, adapted from Lemma 41 in \citet{jin2021bellman}]
  \label{lem:de-regret}
  Given a function class $\cG$ defined on $\cX$ with $|g(x)|\le B$ for all $(g,x)\in\cG\times\cX$, and a family of finite signed measures $\sP$ over $\cX$. 
  Suppose sequence $\{g_t\}_{t=1}^{T}\subset \cG$ and $\{\rho_t\}_{t=1}^{T}\subset\sP$ satisfy that for all $t\in[T]$,
  $\sum_{i=1}^{t-1} (\EE_{\rho_i} [g_t])^2 \le \beta$. Then for all $t\in[T]$ and $\omega>0$ we have for the first order cumulative error that
  \begin{align}\label{eq:DE error-1st order}
      \sum_{i=1}^{t} \abr{\int_{\cX} g_i \rd \rho_i} \le 2\sqrt{\dim_\DE (\cG,\sP,\omega)\beta t}+\min\{t,\dim_\DE (\cG,\sP,\omega)\}B +t\omega, 
  \end{align} 
  and for the second order cumulative error that
  \begin{align}\label{eq:DE error-2nd order}
      \sum_{i=1}^t \rbr{\int_{\cX} g_i \rd \rho_i}^2 \le \dim_\DE(\cG,\sP,\omega)\beta \log(eT) + \min\{t,\dim_\DE (\cG,\sP,\omega)\}B^2 + t\omega^2.
  \end{align}
  \begin{proof}
      We first invoke the following proposition (extended to signed measures) from \citet{jin2021bellman} whose proof can be found in the proof of Proposition 43 in \citet{jin2021bellman}.
      \begin{proposition}\label{prop:ed-sequence length}
          Given a function class $\cG$ defined on $\cX$, and a family of signed measures $\sP$ over $\cX$. 
          Suppose sequences $\{\phi_i\}_{i=1}^{T}\subset \cG$ and $\{\rho_i\}_{i=1}^{T}\subset\sP$ satisfy that for all $t\in[T]$,
          $\sum_{i=1}^{t-1} (\int_\cX \phi_t \rd \rho_i)^2 \le \beta$. Then for all $t\in[T]$,
      $$
          \sum_{i=1}^{t} \ind \cbr{\abr{\int_\cX \phi_i \rd \rho_i} > \epsilon } \leq \rbr{\frac{\beta}{\epsilon^2}+1}\dim_\DE (\cG,\sP,\epsilon).
      $$
      \label{prop:de-regret-prop}
      \end{proposition}
  The following proof is largely the same as those in \citet{jin2021bellman}. For completeness, we still present them here.
  Fix $t \in [T]$ and let $d = \dim_\DE (\cG,\sP,\omega)$. Sort the sequence $\{|\int_{\cX}\phi_1 \rd \rho_1|,\dots,|\int_{\cX}\phi_t \rd \rho_t|\}$ in a decreasing order and denote it by $\{e_1,\dots,e_t\}$ such that $e_1 \geq e_2 \geq \dots \geq e_t$. We have for the cumulative reward of order $k\in\{1, 2\}$ that
$$
  \sum_{i=1}^t |\EE_{\rho_i} [\phi_i]|^k = \sum_{i=1}^t e_i^k = \sum_{i=1}^t e_i^k \ind\big \{e_i \leq \omega \big \} + \sum_{i=1}^t e_i^k \ind\big \{e_i > \omega\big\} \leq t\omega^k + \sum_{i=1}^t e_i^k \ind\big \{e_i > \omega\big\}.
$$
For $i \in [t]$, we want to prove that if $e_i > \omega$, then we have $e_i \leq \min\{\sqrt{{d\beta}/\rbr{i-d}},B\}$. Assume $i\in[t]$ satisfies $e_i >  \omega$. 
Then there exists $\alpha$ such that $e_i >\alpha\ge  \omega$.
By Proposition \ref{prop:de-regret-prop}, we have
$$
i \le 
\sum_{i=1}^t \ind\big \{e_i > \alpha\big\} 
\le \rbr{ \frac{\beta}{\alpha^2} + 1 } \dim_\DE (\cG,\sP,\alpha)
\le 
\rbr{ \frac{\beta}{\alpha^2} + 1 } \dim_\DE (\cG,\sP,\omega),
$$
  where the last inequality holds by definition of the distributional eluder dimension.
The inequality directly implies that $\alpha\le \sqrt{{d\beta}/\rbr{i-d}}$.
Besides, recall $e_i \leq B$, so we have   $e_i \leq \min\{\sqrt{{d\beta}/\rbr{i-d}},B\}$.
Finally, we have
\begin{equation*}
\begin{aligned}
    & \sum_{i=1}^t e_i^k \ind\big \{e_i > \omega\big\}  \nend 
          &\quad \leq \min\{d,t\}B^k+\sum_{i=d+1}^t \rbr{\frac{d\beta}{i-d}}^{k/2} \nend
          &\quad \leq \min\{d,t\}B^k + \ind\cbr{k=1}\cdot \sqrt{d\beta}\int_{0}^t \frac{1}{\sqrt{x}} dx +\ind\cbr{k=2} \cdot  \rbr{d\beta \int_{1}^t x^{-1} \rd x + d\beta}\\
    &\quad \leq \min\{d,t\}B^k + \ind\cbr{k=1}\cdot 2\sqrt{d\beta t} + \ind\cbr{k=2} \cdot d\beta \rbr{\log t +1},
\end{aligned}
\end{equation*}
which completes the proof.
  \end{proof}
\end{lemma}
Note that the above proof as well as \Cref{prop:ed-sequence length} should hold for both the eluder dimension and the distributional eluder dimension.