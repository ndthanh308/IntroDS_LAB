\section{Proofs for Farsighted Case}
Before we dive into the proof, we first present the following guarantee for the MLE method used in \S\Cref{sec:farsighted}.
\begin{lemma}[Guarantee of MLE]\label{lem:MLE}
    By choosing $\beta\ge  \allowbreak 9\log(3e^2H \cN_\rho(\cM,T^{-1})\delta^{-1})$, where $\cN_\rho(\cM,\epsilon)$ is the minimal size of an $\epsilon$-optimistic covering net of $\cM$. Here, an $\epsilon$-optimistic covering net $\cM_\epsilon\subset \cM$ is a finite subset such that for any $M\in\cM$, there exists $\tilde M\in\cM_\epsilon$ satisfying the following conditions:
    \begin{itemize}[leftmargin=20pt]
        \item[(i)] $D_\H\orbr{\nu_h^{\pi, M}(\cdot\given s_h), \nu_h^{\pi, \tilde M}(\cdot\given s_h)}\le \epsilon$, $D_\H\orbr{P_h^M(\cdot\given s_h, a_h, b_h), \allowbreak P_h^{\tilde M}(\cdot\given s_h, a_h, b_h)}\le \epsilon$, $\bigabr{(u_h^{M}-u_h^{\tilde M})(s_h,a_h,b_h)} \le \epsilon$, and $|(A_h^{\pi, M}-A_h^{\pi, \tilde M})(s_h, b_h)|\le \eta^{-1} \epsilon$ for all $\pi\in\Pi$, $h\in[H]$, $(s_h, a_h, b_h)\in\cS\times\cA\times\cB$;
        \item[(ii)] $P_h^{M}(s_{h+1}\given s_h, a_h, b_h)\le \exp(\epsilon)P_h^{\tilde M} (s_{h+1}\given s_h, a_h, b_h)$ for all $h\in[H]$ and $(s_{h+1}, s_h, a_h, b_h)\in\cS^2 \times \cA\times\cB$.
    \end{itemize}
%  \begin{align*}
%     &\rho(M, \tilde M) \nend
%     &\quad\defeq 6\max_{\pi\in\Pi, h\in[H]\atop (s_h, a_h, b_h)\in\cS\times\cA\times\cB}\Big\{D_\H\orbr{\nu_h^{\pi, M}(\cdot\given s_h), \nu_h^{\pi, \tilde M}(\cdot\given s_h)} , 
%     D_\H\orbr{P_h^M(\cdot\given s_h, a_h, b_h), \allowbreak P_h^{\tilde M}(\cdot\given s_h, a_h, b_h)} , \nend
%     &\qqquad \qqquad \qqquad \bigabr{(u_h^{M}-u_h^{\tilde M})(s_h,a_h,b_h)}, \bigabr{\orbr{Q_h^{\pi,M} - Q_h^{\pi,\tilde M}}(s_h,b_h)}\Big\}.
% \end{align*} 
The confidence set $\CI_\cM^t(\beta)$ satisfies the following with probability at least $1-\delta$: for any given $t\in[T]$, $M^*\in\confset_\cM^t(\beta)$, and it also holds for $\forall h\in[H], \forall M\in\confset^t(\beta)$ that
    \begin{align*}
        \sum_{i=1}^{t-1}D_{\RL, h}^2\rbr{M, M^*,\pi^i}
        \le 4\beta,\quad 
        \sum_{i=1}^{t-1}\hat D_{\RL, h,i}^2\rbr{M, M^*}
        \le 4\beta, 
    \end{align*}
 where 
    \begin{align*}
        D_{\RL,h}^2 (M,  M^*;\pi) &=   \EE^{\pi, M^*}D_\H^2\rbr{\nu_h^{\pi,M}, \nu_h^{\pi,M^*}} +
        \EE^{\pi, M^*} D_\H^2(P_h^{M}, P_h^{M^*}) +
        \EE^{\pi, M^*}\rbr{u_h^{M^*}-u_h^M}^2,\nend
        %%%%%%%%%%%
        D_{\RL,h,i}^2(M,M^*) &= D_\H^2\orbr{\nu_h^{\pi^i, M}(\cdot\given s_h^i), \nu_h^{\pi^i, \tilde M}(\cdot\given s_h^i)} +
        D_\H^2\orbr{P_h^M(\cdot\given s_h^i, a_h^i, b_h^i), \allowbreak P_h^{\tilde M}(\cdot\given s_h^i, a_h^i, b_h^i)} \nend
        &\qquad + \bigrbr{(u_h^{M}-u_h^{\tilde M})(s_h^i,a_h^i,b_h^i)}^2,
    \end{align*}
    and $\EE^{\pi, M}$ is taken under policy $\pi$ and the model $M$.
    \begin{proof}
        See \Cref{sec:proof-MLE} for a detailed proof.
    \end{proof}
\end{lemma}
\Cref{lem:MLE} guarantees that the confidence set $\confset^t(\beta)$ is valid in the sense that $M^*\in\confset^t(\beta)$ and any $M\in\confset^t(\beta)$ has $D_\RL$ bounded by $\cO(\beta)$. 
For the optimistic covering net, we remark that constraints in the first conditions are discussed in \eqref{eq:rho-cM}. The second condition requires $P_h^M$ to be dominated above by $P_h^{\tilde M}$, which is needed to control the difference in the log-likelihood.

\subsection{Proof of \Cref{thm:PMLE} on PMLE wiith Farsighted Follower}
\label{sec:proof-PMLE}
In this subsection, we provide a formal proof to \Cref{thm:PMLE}.
The proof is carried out as the following.

\paragraph{Step 1. Offline Suboptimality Decomposition}
By \Cref{lem:MLE}, we have with high probability that $M^*\in\CI_\cM(\beta)$, and we have the following suboptimality decomposition,
\begin{align*}
    J(\pi^*) - J(\hat\pi) 
    &= J(\pi^*) - J(\pi^*, M^{\pi^*}) + J(\pi^*, M^{\pi^*}) - J(\hat\pi, M^{\hat\pi}) + J(\hat\pi, M^{\hat\pi}) - J(\hat\pi)\nend
    &\le J(\pi^*) - J(\pi^*, M^{\pi^*}) +J(\pi^*, M^{\pi^*}) - J(\hat\pi, M^{\hat\pi})\nend
    &\le J(\pi^*) - J(\pi^*, M^{\pi^*}) ,
\end{align*}
where we define $M^\pi = \argmin_{M\in\CI_\cM(\beta)}J(\pi, M)$ as the pessimistic estimated model for $\pi$. Here, the first inequality holds by noting that $J(\hat\pi) = J(\hat\pi, M^*) \ge J(\hat\pi, M^{\hat\pi})$ by the validity of the confidence set $\CI_\cM(\beta)$ and the definition of $M^{\hat\pi}$, and the  second inequality is a direct result of policy optimization. Now, we further decompose the suboptimality using \Cref{lem:subopt-decomposition}. We define $\tilde M = M^{\pi^*}$, and let $\tilde U, \tilde W$ be the follower's value functions under policy $\pi^*$ and the estimated model $\tilde M$. Let $\tilde \nu$ be the estimated quantal response under $\pi^*$ and $\tilde M$. We have for $\pi^*, \tilde U, \tilde W, \tilde\nu$ that 
\begin{align*}
    &J(\pi^* ) - J(\pi^*, \tilde M)\nend
    &\quad \le \sum_{h=1}^H \EE \sbr{\rbr{\tilde U_h - u_h}(s_h, a_h, b_h) -  \tilde W_{h+1}(s_{h+1})} + \sum_{h=1}^H 2 H  \EE \sbr{D_\TV\rbr{\tilde \nu_h(\cdot\given s_h), \nu_h(\cdot\given s_h)}}\nend
    &\quad \le \sum_{h=1}^H \underbrace{\EE \sbr{\rbr{\tilde U_h - u_h}(s_h, a_h, b_h) -  \tilde W_{h+1}(s_{h+1})}}_{\dr\text{Leader's Bellman error}} \nend
    &\qqquad +  
    C^{(0)} \cdot 
    \sum_{h=1}^H \underbrace{\EE\bigsbr{\bigabr{\tilde \Delta^{(1)}_h(s_h, b_h)}}}_{\ds\text{1st-order error}}  + C^{(2)} \cdot 
    \max_{h\in [H]} \underbrace{\EE\bigsbr{ \bigrbr{\orbr{\tilde Q_h - r_h^{\pi^*} - \gamma P_h^{\pi^*} \tilde V_{h+1}}(s_h, b_h)}^2}}_{\ds\text{2nd-order error}}
\end{align*}
where the expectation is taken under $\pi^*$ and the true model $M^*$, and we define $\tilde Q, \tilde V$ as the follower's value functions under policy $\pi^*$ and model $\tilde M$. 
Here, the first inequality holds by \eqref{eq:perform-diff-linear} where we notice that $J(\pi^*, \tilde M) = \EE[\tilde W_1(s_1)]$ and also that $\tilde W_h = T_h^{\pi^*, \tilde \nu}\tilde U_h$. The second inequality holds by using \Cref{lem:performance diff}. Moreover, the definition of $\tilde\Delta_H^{(1)}(s_h, b_h) $ is given by 
\begin{align*}
    \tilde \Delta^{(1)}_h(s_h, b_h) &=  \rbr{\EE_{s_h, b_h} -\EE_{s_h}}\Biggsbr{\sum_{l=h}^H \gamma^{l-h}\underbrace{\rbr{\tilde Q_l - r_l^{\pi^*} - \gamma P_l^{\pi^*} \tilde V_{l+1}}(s_l, b_l)}_{\ds\text{Follower's Bellman error}}}. 
    % \label{eq:def Delta^1}
\end{align*}
In the sequel, we separately bound these three terms.

\paragraph{Step 2. Bounding the Leader's Bellman Error.}
We present the gurantee we have for the Leader's Bellman error on the samples. Define $\EE^i=\EE^{\pi^i}$, which is the expectation taken under $\pi^i$ and the true model $M^*$. We have
\begin{align*}
    &\sum_{i=1}^T 
    \EE^i \sbr{\rbr{\bigrbr{\tilde U_h - u_h}(s_h, a_h, b_h) - \tilde W_{h+1}(s_{h+1})}^2}\nend
    &\quad =\sum_{i=1}^T \EE^i \sbr{\rbr{{\bigrbr{\tilde u_h - u_h}(s_h, a_h, b_h) + \bigrbr{\tilde P_h - P_h} \tilde W_{h+1}(s_{h+1})}}^2}\nend
    &\quad \le 2\sum_{i=1}^T \EE^i \sbr{\rbr{\rbr{\tilde u_h - u_h}(s_h,a_h,b_h)}^2} + 8 H^2 \sum_{i=1}^T \EE^i D_\TV^2\rbr{\tilde P_h(\cdot\given s_h, a_h, b_h), P_h(\cdot\given s_h, a_h, b_h)}\nend
    &\quad \lesssim  H^2 \beta, 
\end{align*}
where the first equality holds by the definition of $\tilde U$, and the first inequality holds by the Jensen's inequality, and the last inequality holds by the MLE guarantee in \Cref{lem:MLE}, where we hide some universal constants by \say{$\lesssim$}.
Therefore, we have that
\begin{align*}
    &\sum_{h=1}^H \EE \sbr{\rbr{\tilde U_h - u_h}(s_h, a_h, b_h) -  \tilde W_{h+1}(s_{h+1})}\nend
    &\quad \le \sum_{h=1}^H \sqrt{\EE \Bigsbr{\Bigrbr{\bigrbr{\tilde U_h - u_h}(s_h, a_h, b_h) -  \tilde W_{h+1}(s_{h+1})}^2}}\nend
    &\quad \lesssim H\sqrt{H^2 \beta} \cdot 
    \sqrt{\frac{\EE \Bigsbr{\Bigrbr{\bigrbr{\tilde U_h - u_h}(s_h, a_h, b_h) -  \tilde W_{h+1}(s_{h+1})}^2}}{\sum_{i=1}^T 
    \EE^i \Bigsbr{\rbr{\bigrbr{\tilde U_h - u_h}(s_h, a_h, b_h) - \tilde W_{h+1}(s_{h+1})}^2}}}\nend
    &\quad \le H^2\sqrt{ \beta} \cdot \max_{M\in\cM, h\in[H]} \sqrt{\frac{\EE\sbr{\rbr{\bigrbr{U_h^{\pi^*, M}-\bigrbr{u_h+P_h W_{h+1}^{\pi^*, M}}}(s_h, a_h, b_h)}^2 }}{\sum_{i=1}^T \EE^i\sbr{\rbr{\bigrbr{U_h^{\pi^*, M}-\bigrbr{u_h+P_h W_{h+1}^{\pi^*, M}}}(s_h, a_h, b_h)}^2 }}}.
\end{align*}

\paragraph{Step 3. Bound the first-order Error of the Follower's Response.}
We first study the following guarantee for the 1st order term over the offline samples,
\begin{align*}
    &\sum_{i=1}^T \EE^i \sbr{\rbr{\rbr{\EE_{s_h, b_h}^i -\EE_{s_h}^i}\sbr{\sum_{l=h}^H \gamma^{l-h}\rbr{\tilde r_l - r_l + \gamma \bigrbr{\tilde P_l - P_l} \tilde V_{l+1}}(s_l,a_l,  b_l)}}^2}\nend
    %%%%%%%%%%%%
    &\quad \lesssim \sum_{i=1}^T \EE^i \sbr{\rbr{\rbr{\EE_{s_h, b_h}^i -\EE_{s_h}^i}\sbr{\sum_{l=h}^H \gamma^{l-h}\rbr{\tilde r_l - r_l + \gamma \bigrbr{\tilde P_l - P_l} V_{l+1}^{\pi^i, \tilde M}}(s_l,a_l,  b_l)}}^2}\nend
    &\qqquad + \sum_{i=1}^T \EE^i \sbr{\rbr{\rbr{\EE_{s_h, b_h}^i -\EE_{s_h}^i}\sbr{\sum_{l=h}^H \gamma^{l-h}\rbr{\gamma \bigrbr{\tilde P_l - P_l} \bigrbr{\tilde V_{l+1} - V_{l+1}^{\pi^i,\tilde M}}}(s_l,a_l,  b_l)}}^2} \nend
    &\quad \lesssim \sum_{i=1}^T \EE^i \sbr{\rbr{\rbr{\EE_{s_h, b_h}^i -\EE_{s_h}^i}\Biggsbr{\sum_{l=h}^H \gamma^{l-h}\rbr{\tilde r_l - r_l + \gamma \bigrbr{\tilde P_l - P_l} V_{l+1}^{\pi^i, \tilde M} }(s_l, a_l, b_l)}}^2}\nend
    &\qqquad + B_A^2 \sum_{i=1}^T   \EE^i\rbr{\sum_{l=h}^H \gamma^{l-h}\rbr{\EE_{s_h,b_h}^i + \EE_{s_h}^i}  \sbr{D_\TV\rbr{\tilde P_l(\cdot\given s_l, a_l, b_l), P_l(\cdot\given s_l, a_l, b_l)}}}^2\nend
    &\quad \lesssim \sum_{i=1}^T \EE^i \biggsbr{\biggrbr{\underbrace{\rbr{\EE_{s_h, b_h}^i -\EE_{s_h}^i}\Biggsbr{\sum_{l=h}^H \gamma^{l-h}\rbr{\tilde r_l - r_l + \gamma \bigrbr{\tilde P_l - P_l} V_{l+1}^{\pi^i, \tilde M} }(s_l, a_l, b_l)}}_{\ds \tilde \Delta^{(1)}_{h, \pi^i, \tilde M}(s_h, b_h) }}^2}\nend
    &\qqquad + B_A^2 \sum_{i=1}^T   \eff_H(\gamma) \sum_{l=h}^H \gamma^{l-h} \EE^i D_\TV^2\rbr{\tilde P_l(\cdot\given s_l, a_l, b_l), P_l(\cdot\given s_l, a_l, b_l)}
\end{align*}
where the first inequality holds by the Jensen's inequality, where we add a $(\tilde P_l - P_l) V_{l+1}^{\pi^i, \tilde M}$ term and substract it, which gives us a separate $(\tilde P_l - P_l) (\tilde V_{l+1} - V_{l+1}^{\pi^i, \tilde M})$ term. 
In the second inquality, we upper bound $(\tilde P_l - P_l) (\tilde V_{l+1} - V_{l+1}^{\pi^i, \tilde M})$ by the TV distance between $\tilde P_l$ and $P_l$ multiplied by the infinity norm $\bignbr{\tilde V_{l+1} - V_{l+1}^{\pi^i, \tilde M}}_\infty$, which is bounded by $2 B_A$ by our argument in \Cref{sec:app-notations}. Since the TV distance is always nonnegative, we can safely flips the sign between $\EE_{s_h, b_h}^i - \EE_{s_h}^i $. The above steps give us the first inequality. 
The second inequality simply holds by using the Cauchy-Schwartz inequality where we move the square inside the expectation for the summation of the TV distance, and the $\eff_H(\gamma)$ is just a byproduct produced when applying the Cauchy-Schwartz inequality.

Next, we show how to control this $\tilde \Delta_{h,\pi^i, \tilde M}(s_h, b_h)$ term. We first notice that $\tilde\Delta^{(1)}_{h, \pi^i, \tilde M}(s_h, b_h)$ is nothing but just $\tilde\Delta^{(1)}_h(s_h, b_h)$ plugged in with $\pi^i$ as the policy $\pi$ and $\tilde Q^{\pi^i}=Q^{\pi^i, \tilde M}, \tilde V^{\pi^i} = V^{\pi^i, \tilde M}$ as the follower's value functions $\tilde Q, \tilde V$. 
Hence, we can invoke \Cref{lem:1st-ub} which says that 
\begin{align*}
    &\bigrbr{\tilde \Delta_{h, \pi^i,\tilde M}^{(1)}(s_h, b_h)}^2  \nend
        &\quad \le 2 \rbr{\rbr{\EE_{s_h, b_h}^i-\EE_{s_h}^i} \bigsbr{\orbr{Q_h^{\pi^i} - \tilde Q_h^{\pi^i}}(s_h, b_h)}}^2 \nend
        &\qqquad + 16 \gamma^2  \rbr{\eta^{-1} +2 B_A}^2\eff_H(\gamma) \sum_{l=h+1}^H \gamma^{l-h-1} {\rbr{\EE_{s_h}^i+\EE_{s_h, b_h}^i}\sbr{D_\H^2(\nu_l^{\pi^i}(\cdot\given s_l), \tilde\nu_l^{\pi^i}(\cdot\given s_l))}}
\end{align*}
where we define $\tilde \nu_l^{\pi^i} = \nu_l^{\pi^i, \tilde M}$ as the quantal response under $\pi^i$ and the estimated model $\tilde M$. This is true by our definition of $\tilde Q^{\pi^i}, \tilde V^{\pi^i}$ that they are the follower's value functions under $\pi^i$ and model $\tilde M$.
Therefore, we have that 
\begin{align*}
    &\sum_{i=1}^T \EE^i\bigrbr{\tilde \Delta_{h, \pi^i,\tilde M}^{(1)}(s_h, b_h)}^2  \nend
    &\quad \lesssim \sum_{i=1}^T \EE^i\rbr{\rbr{\EE_{s_h, b_h}^i-\EE_{s_h}^i} \bigsbr{\orbr{Q_h^{\pi^i} - \tilde Q_h^{\pi^i}}(s_h, b_h)}}^2 \nend
    &\qqquad + \gamma^2  \rbr{\eta^{-1} +2 B_A}^2\eff_H(\gamma) \sum_{i=1}^T \sum_{l=h+1}^H \gamma^{l-h-1} {\EE^i\sbr{D_\H^2(\nu_l^{\pi^i}(\cdot\given s_l), \tilde\nu_l^{\pi^i}(\cdot\given s_l))}} \nend
    &\quad\lesssim \rbr{(\eta^{-2}+B_A^2) + \gamma^2  \rbr{\eta^{-1} +2 B_A}^2\eff_H(\gamma) H} \beta, 
\end{align*}
where the last inequality holds by both \eqref{eq:MLE-guarantee-Q-3} in \Cref{lem:MLE-formal} for the Q difference term and the MLE guarantee in \Cref{lem:MLE} for the Hellinger term. Here, we upper bound $\gamma^{l-h-1}$ by $1$ and take a summation over $h\in[H]$. Therefore, we conclude that 
\begin{align*}
    &
    \sum_{i=1}^T \EE^i \sbr{\rbr{\rbr{\EE_{s_h, b_h}^i -\EE_{s_h}^i}\sbr{\sum_{l=h}^H \gamma^{l-h}\rbr{\tilde r_l - r_l + \gamma \bigrbr{\tilde P_l - P_l} \tilde V_{l+1}}(s_l,a_l,  b_l)}}^2} \nend
    &\quad\lesssim  
    \sum_{i=1}^T \EE^i\bigrbr{\tilde \Delta_{h, \pi^i,\tilde M}^{(1)}(s_h, b_h)}^2 \nend
    &\qqquad + 
    B_A^2 \sum_{i=1}^T   \eff_H(\gamma) \sum_{l=h}^H \gamma^{l-h} \EE^i D_\TV^2\rbr{\tilde P_l(\cdot\given s_l, a_l, b_l), P_l(\cdot\given s_l, a_l, b_l)}\nend
    &\quad \le  \rbr{(\eta^{-2}+B_A^2) + \gamma^2  \rbr{\eta^{-1} +2 B_A}^2\eff_H(\gamma) H} \beta  + B_A^2 \eff_H(\gamma) H \beta \nend
    &\quad \lesssim \rbr{\eta^{-2} + B_A^2 }\eff_H(\gamma)H \beta. 
\end{align*}
As a result, we have that
\begin{align*}
    &C^{(0)}\sum_{h=1}^H \EE\sbr{\abr{\tilde \Delta_h^{(1)}(s_h, b_h)}}\nend
    &\quad\le C^{(0)} \sum_{h=1}^H \sqrt{\EE\sbr{\tilde \Delta_h^{(1)}(s_h, b_h)^2}}\nend
    %%%%%%%%%%%%%
    &\quad \lesssim C^{(0)} H \sqrt{ \rbr{\eta^{-2} + B_A^2 }\eff_H(\gamma)H \beta} \nend
    &\qqquad \cdot \max_{h\in[H]}\sqrt\frac{\EE \sbr{\rbr{\rbr{\EE_{s_h, b_h} -\EE_{s_h}}\sbr{\sum_{l=h}^H \gamma^{l-h}\rbr{\tilde r_l - r_l + \gamma \bigrbr{\tilde P_l - P_l} \tilde V_{l+1}}(s_l,a_l,  b_l)}}^2}}{\sum_{i=1}^T \EE^i \sbr{\rbr{\rbr{\EE_{s_h, b_h}^i -\EE_{s_h}^i}\sbr{\sum_{l=h}^H \gamma^{l-h}\rbr{\tilde r_l - r_l + \gamma \bigrbr{\tilde P_l - P_l} \tilde V_{l+1}}(s_l,a_l,  b_l)}}^2}}\nend
    %%%%%%%%%%%%
    &\quad \lesssim \rbr{1 + \eta B_A }H^2\sqrt{ H \eff_H(\gamma)\beta} \nend
    &\quad \cdot \max_{M\in\cM, h\in[H]}\sqrt\frac{\EE \sbr{\rbr{\rbr{\EE_{s_h, b_h} -\EE_{s_h}}\sbr{\sum_{l=h}^H \gamma^{l-h}\rbr{r_l^M - r_l + \gamma \bigrbr{P_l^M - P_l} V_{l+1}^{\pi^*, M}}(s_l,a_l,  b_l)}}^2}}{\sum_{i=1}^T \EE^i \sbr{\rbr{\rbr{\EE_{s_h, b_h}^i -\EE_{s_h}^i}\sbr{\sum_{l=h}^H \gamma^{l-h}\rbr{ r_l^M - r_l + \gamma \bigrbr{P_l^M - P_l} V_{l+1}^{\pi^*, M}}(s_l,a_l,  b_l)}}^2}}, 
\end{align*}
where we notice that $C^{(0)}=2\eta H$.

\paragraph{Step 4. Bound the second-order Error in the Follower's Response.}
The last thing to do is controlling the second order term. We first expand the second order term in terms of $r_h, P_h, V_{h+1}$ by definitions and have the following guarantee for the second order term over the samples, 
\begin{align*}
    &\sum_{i=1}^T \EE^i\sbr{ \rbr {\EE_{s_h, b_h}^i\sbr{{\bigrbr{\tilde r_h - r_h + \gamma \bigrbr{\tilde P_h - P_h} \tilde V_{h+1}}(s_h,a_h, b_h)}}}^2}\nend
    &\quad \lesssim \sum_{i=1}^T \EE^i\sbr{ \rbr {\EE_{s_h, b_h}^i\sbr{{\bigrbr{\tilde r_h - r_h + \gamma \bigrbr{\tilde P_h - P_h} V_{h+1}^{\pi^i, \tilde M}}(s_h,a_h, b_h)}}}^2} \nend
    &\qqquad + \sum_{i=1}^T \EE^i\sbr{ \rbr {\EE_{s_h, b_h}^i\sbr{{\gamma \bigrbr{\tilde P_h - P_h} \bigrbr{\tilde V_{h+1} - V_{h+1}^{\pi^i, \tilde M} }(s_h,a_h, b_h)}}}^2}\nend
    %%%%%% split %%%%%%
    &\quad \lesssim \sum_{i=1}^T \EE^i\sbr{ \rbr {\EE_{s_h, b_h}^i\sbr{{\bigrbr{\tilde r_h - r_h + \gamma \bigrbr{\tilde P_h - P_h} V_{h+1}^{\pi^i, \tilde M}}(s_h,a_h, b_h)}}}^2} \nend
    &\qqquad + \gamma^2B_A^2 \sum_{i=1}^T \EE^i\sbr{ D_\TV^2\rbr{\tilde P_h(\cdot\given s_h, a_h, b_h), P_h(\cdot\given s_h, a_h, b_h)}}, 
\end{align*}
where the first inequality holds by the Jensen's inequality, the second inequality holds by upper bounding the difference in $(\tilde P-P)(\tilde V-V^{\pi^i, \tilde M})$ by the TV distance and the upper bound for the follower's V function as $B_A$.
We now invoke \Cref{lem:2nd-ub} for $(\pi^i, Q_h^{\pi^i,\tilde M}, V_{h+1}^{\pi^i, \tilde M})$, which gives us 
\begin{align}
    &\sum_{i=1}^T \max_{h\in[H]}\EE^i\sbr{ \rbr {\EE_{s_h, b_h}^i\sbr{{\bigrbr{\tilde r_h - r_h + \gamma \bigrbr{\tilde P_h - P_h} V_{h+1}^{\pi^i, \tilde M}}(s_h,a_h, b_h)}}}^2} \nend
    &\quad =\sum_{i=1}^T\max_{h\in[H]}\EE^i\sbr{ \rbr{\rbr{Q_h^{\pi^i,\tilde M} - r_h^{\pi^i} - \gamma P_h^{\pi^i}  V_{h+1}^{\pi^i, \tilde M}}(s_h, b_h)}^2} \nend
    &\quad \le \sum_{i=1}^T L^{(2)} \sum_{h=1}^H \cbr{\EE D_\H^2(\nu_h^{\pi^i}(\cdot\given s_h),\nu_h^{\pi^i, \tilde M}(\cdot\given s_h))+\EE D_\TV^2(P_h^{\pi^i}(\cdot\given s_h, b_h),P_h^{\pi^i, \tilde M}(\cdot\given s_h, b_h))}\nend
    &\quad \lesssim L^{(2)} H \beta .\label{eq:PMLE-1}
\end{align}
where in the first inequality, we additionally replace the maximum over $h\in[H]$ as a summation over $h\in[H]$, and the second inequlity holds by the MLE guarantee in \Cref{lem:MLE}. Hence, we conclude that 
\begin{align*}
    &\sum_{i=1}^T \EE^i\sbr{ \rbr {\EE_{s_h, b_h}^i\sbr{{\bigrbr{\tilde r_h - r_h + \gamma \bigrbr{\tilde P_h - P_h} \tilde V_{h+1}}(s_h,a_h, b_h)}}}^2}\nend
    %%%%%%%% split %%%%
    &\quad \lesssim \sum_{i=1}^T \EE^i\sbr{ \rbr {\EE_{s_h, b_h}^i\sbr{{\bigrbr{\tilde r_h - r_h + \gamma \bigrbr{\tilde P_h - P_h} V_{h+1}^{\pi^i, \tilde M}}(s_h,a_h, b_h)}}}^2} \nend
    &\qqquad + \gamma^2B_A^2 \sum_{i=1}^T \EE^i\sbr{ D_\TV^2\rbr{\tilde P_h(\cdot\given s_h, a_h, b_h), P_h(\cdot\given s_h, a_h, b_h)}}\nend
    &\quad \lesssim L^{(2)} H \beta + \gamma^2 B_A^2 \beta, 
\end{align*}
where the last inequality holds by using \eqref{eq:PMLE-1} for the first term and the MLE guarantee in \Cref{lem:MLE} for the second term.
Now, we invoke \Cref{prop:de-regret-prop} and obtain 
\begin{align*}
    &C^{(2)} \cdot 
    \max_{h\in[H]}\EE\bigsbr{ \bigrbr{\orbr{\tilde Q_h - r_h^{\pi^*} - \gamma P_h^{\pi^*} \tilde V_{h+1}}(s_h, b_h)}^2} \nend
    &\quad = C^{(2)} \cdot \max_{h\in[H]}
    \EE\sbr{ \rbr{\EE_{s_h, b_h}\sbr{\bigrbr{\tilde r_h - r_h + \gamma \bigrbr{\tilde P_h - P_h} \tilde V_{h+1}}(s_h, a_h, b_h)}}^2} \nend
    &\quad \lesssim C^{(2)} \bigrbr{L^{(2)} H \beta + \gamma^2 B_A^2 \beta} \nend
    &\qqquad\cdot \max_{h\in[H]}\frac{\EE\sbr{ \rbr{\EE_{s_h, b_h}\sbr{\bigrbr{\tilde r_h - r_h + \gamma \bigrbr{\tilde P_h - P_h} \tilde V_{h+1}}(s_h, a_h, b_h)}}^2}}{\sum_{i=1}^T \EE^i\sbr{ \rbr {\EE_{s_h, b_h}^i\sbr{{\bigrbr{\tilde r_h - r_h + \gamma \bigrbr{\tilde P_h - P_h} \tilde V_{h+1}}(s_h,a_h, b_h)}}}^2}}\nend
    &\quad \lesssim C^{(2)} \bigrbr{L^{(2)} H \beta + \gamma^2 B_A^2 \beta} \nend
    &\qqquad\cdot \max_{h\in[H], M\in\cM}\frac{\EE\sbr{ \rbr{\EE_{s_h, b_h}\sbr{\bigrbr{r_h^{M} - r_h + \gamma \bigrbr{P_h^M - P_h}  V_{h+1}^{\pi^*, M}}(s_h, a_h, b_h)}}^2}}{\sum_{i=1}^T \EE^i\sbr{ \rbr {\EE_{s_h, b_h}^i\sbr{{\bigrbr{r_h^{M} - r_h + \gamma \bigrbr{P_h^M - P_h}  V_{h+1}^{\pi^*, M}}(s_h,a_h, b_h)}}}^2}}, 
\end{align*}
where the first inequality is just a distribution, and the last inequality takes a maximum over $\cM$. 

In summary, for the leader's Bellman error, 
\begin{align*}
    \text{LBE} \lesssim H^2 \sqrt{\beta} \cdot \max_{M\in\cM, h\in[H]} \sqrt{\frac{\EE\sbr{\rbr{\bigrbr{U_h^{\pi^*, M}-\bigrbr{u_h+P_h W_{h+1}^{\pi^*, M}}}(s_h, a_h, b_h)}^2 }}{\sum_{i=1}^T \EE^i\sbr{\rbr{\bigrbr{U_h^{\pi^*, M}-\bigrbr{u_h+P_h W_{h+1}^{\pi^*, M}}}(s_h, a_h, b_h)}^2 }}},
\end{align*}
for the first order term in the follower's quantal response error, 
\begin{align*}
    &\text{1st-QRE} \nend
    &\quad \lesssim  \eta C_\eta H^2\sqrt{ H \eff_H(\gamma)\beta} \nend
    &\qquad \cdot \max_{M\in\cM, h\in[H]}\sqrt\frac{\EE \sbr{\rbr{\rbr{\EE_{s_h, b_h} -\EE_{s_h}}\sbr{\sum_{l=h}^H \gamma^{l-h}\rbr{r_l^M - r_l + \gamma \bigrbr{P_l^M - P_l} V_{l+1}^{\pi^*, M}}(s_l,a_l,  b_l)}}^2}}{\sum_{i=1}^T \EE^i \sbr{\rbr{\rbr{\EE_{s_h, b_h}^i -\EE_{s_h}^i}\sbr{\sum_{l=h}^H \gamma^{l-h}\rbr{ r_l^M - r_l + \gamma \bigrbr{P_l^M - P_l} V_{l+1}^{\pi^*, M}}(s_l,a_l,  b_l)}}^2}}, 
\end{align*}
and for the second order term in the follower's quantal response error, 
\begin{align*}
    &\text{2nd-QRE}\nend
    &\quad \lesssim C^{(2)} L^{(2)} H \beta \nend
    &\qqquad\cdot \max_{h\in[H], M\in\cM}\frac{\EE\sbr{ \rbr{\EE_{s_h, b_h}\sbr{\bigrbr{r_h^{M} - r_h + \gamma \bigrbr{P_h^M - P_h}  V_{h+1}^{\pi^*, M}}(s_h, a_h, b_h)}}^2}}{\sum_{i=1}^T \EE^i\sbr{ \rbr {\EE_{s_h, b_h}^i\sbr{{\bigrbr{r_h^{M} - r_h + \gamma \bigrbr{P_h^M - P_h}  V_{h+1}^{\pi^*, M}}(s_h,a_h, b_h)}}}^2}}. 
\end{align*}
Hence, we complete the proof of \Cref{thm:PMLE}

\subsection{Proof of \Cref{thm:OMLE-farsighted} on OMLE with Farsighted Follower}\label{sec:proof-farsighted MDP}
In this section, we provide formal proofs for the result to Theorem \ref{thm:OMLE-farsighted}.
The proof relies on a novel decomposition of the online regret in the Taylor-series form and utilizes the techniques for analyzing the OMLE in \citep{chen2022unified, foster2021statistical,jin2021bellman}. Before diving into the proof, we present the following key lemma that provides guarantees for the confidence set $\confset_\cM^t(\beta)$ given by the algorithm.

The proof is carried out as the following. We recall that $\beta\ge  \allowbreak 9\log(3e^2TH \cN_\rho(\cM,T^{-1})\delta^{-1})$, where we additionally include an $\log T$ term to ensure a union bound over $t\in[T]$.

\paragraph{Step 1. Online Regret Decompostion. }
By \Cref{lem:MLE}, we have with high probability that $M^*\in\confset_\cM^t(\beta)$. Hence, we can upper bound the online regret by
\begin{align*}
    \Reg(T) = \sum_{t=1}^T J(\pi^*, M^*) - J(\pi^t, M^*) \le \sum_{t=1}^T J(\pi^t, M^t) - J(\pi^t, M^*), 
\end{align*}
where the inequality holds by additionally noting that OMLE produces the pair $(\pi^t, M^t)$ that maximizes $J$ within the confidence set $\confset^t(\beta)$.
The key in studying the regret in this MDP with strategic follower is decomposing the performance difference into both the follower's temporal difference and the follower's response difference. 
We invoke \Cref{lem:subopt-decomposition} with $\pi^t, \tilde U^t, \tilde W^t, \tilde \nu^t$, where $\tilde U^t, \tilde W^t, \tilde \nu^t$ are given under policy $\pi^t$ and the estimated model $\tilde M^t$, and they should satisfy $\tilde W_h^t(s_h)=(T_h^{\pi^t, \tilde \nu^t}\tilde U_h^t) (s_h)$. We additionally define $\nu^t = \nu^{\pi^t, M^*}$. We have that
\begin{align*}
    &J(\pi^t, \tilde M^t) - J(\pi^t, M^*)\nend
    &\quad \le \sum_{h=1}^H \EE^t \sbr{\rbr{\tilde U_h^t - u_h}(s_h, a_h, b_h) -  \tilde W_{h+1}^t(s_{h+1})} + \sum_{h=1}^H 2 H  \EE^t \sbr{D_\TV\rbr{\tilde \nu_h^t(\cdot\given s_h), \nu_h^t(\cdot\given s_h)}}\nend
    &\quad \le \sum_{h=1}^H \underbrace{\EE^t \sbr{\rbr{\tilde U_h^t - u_h}(s_h, a_h, b_h) - \tilde W_{h+1}^t(s_{h+1})}}_{\dr \text{Leader's Bellman error}} \nend
    &\qqquad +   C^{(0)}
    \sum_{h=1}^H \underbrace{\EE^t\sbr{\abr{\tilde \Delta^{(1,t)}_h(s_h, b_h)}}}_{\ds\text{1st-order error}}  + C^{(2)}
    \max_{h\in [H]} \underbrace{\EE^t\sbr{ \rbr{\rbr{\tilde Q_h^t - r_h^{\pi^t} - \gamma P_h^{\pi^t} \tilde V_{h+1}^t}(s_h, b_h)}^2}}_{\ds\text{2nd-order error}}
\end{align*}
where the expectation $\EE^t$ is taken with respect to $\pi^t$ and the true model $M^*$, 
$C^{(0)}=2\eta H$, 
$C^{(2)} = 2 H\eta^2 H (1+4 \eff_H(\gamma))\exp\rbr{6\eta B_A}\cdot \rbr{\eff_H(\exp(2\eta B_A)\gamma)}^2$ with $\eff_H(x) = (1-x^H)/(1-x)$ as the \say{effective}  horizon with respect to $x$, 
and $\tilde\Delta_j^{(1, t)}(s_h, b_h)$ is defined as
\begin{align*}
    \tilde \Delta^{(1, t)}_h(s_h, b_h) &=  \rbr{\EE_{s_h, b_h}^t -\EE_{s_h}^t}\Biggsbr{\sum_{l=h}^H \gamma^{l-h}\underbrace{\rbr{\tilde Q_l^t - r_l^{\pi^t} - \gamma P_l^{\pi^t} \tilde V_{l+1}^t}(s_l, b_l)}_{\ds\text{Follower's Bellman error}}}.
\end{align*}
Here, the second inequality comes from \Cref{lem:performance diff} and uses the definition that $\tilde V_h^t = \eta^{-1}\log \int \exp(\eta \tilde Q_h^t)$, $\tilde A_h^t = \tilde Q_h^t -\tilde V_h^t$,  and that $\tilde\nu_h^t=\exp\orbr{\eta \tilde A_h^t}$ under the alternative model $\tilde M^t$. In the sequel, we will bound these three terms separately.

\paragraph{Step 2. Bounding the Leader's Bellman Error.}
We first show that the leader's Bellman error is controllable when summed up for $T$ steps. Specifically, consider the following configurations for step $h\in[H]$,
\begin{itemize}[leftmargin=20pt]
    \item[(i)] Define function class $\cG_{h,L}$ as
    \begin{align*}
        \cG_L^h &= \Big\{g:\cS\times\cA\times\cB\rightarrow \RR \Biggiven g={\bigrbr{U_h^{\pi^{\tilde M},\tilde M} - u - P_h W_{h+1}^{\pi^{\tilde M},\tilde M}}(s_h, a_h, b_h)}, \exists \tilde M\in\cM\Big\}, 
    \end{align*}
    where we define $\pi^M = \argmax_{\pi\in\Pi}J(\pi, M)$.
    Specifically, the expectation is taken under $\pi$ and the true model. Consider a sequence of function $\{g_h^i = (\tilde U_h^{i} - u - P_h \tilde W_{h+1}^i)\}_{i\in[T]}$. We see directly that $g_h^i\in\cG_{h, L}$ since $\tilde U^i = U^{\pi^i, M^i}$ and we have by the optimism in the algorithm that $\pi^i = \pi^{M^i}$. The same also holds for $\tilde W^i$.
    \item[(ii)] Define a class of probability measures over $\cS\times\cA\times\cB$ as $$\sP_{h, L}=\{\PP^\pi((s_h, a_h, b_h)=\cdot), \forall \pi\in\Pi\}.$$
    Consider a sequence of probability measures $\{\rho_h^i(\cdot)=\PP^{\pi^i}((s_h, a_h, b_h)=\cdot)\}_{i\in[T]}$.
    % \item Define a class of probability measures over space $\cS\times\cA\times\cB$ as \begin{align*}
    %     \sP_L = \cbr{\rho\in\Delta(\cS\times\cA\times\cB): \exists h\in[H], \pi\in\Pi, \rho(\cdot)=\PP^\pi((s_h, a_h, b_h)=\cdot)}.
    % \end{align*}
    \item[(iii)] Under this two sequences, we denote by $g_h^t(\pi^i) = \EE_{\rho_h^i}[g_h^t]$ for simplicity. 
    We have 
    $$g_h^t(\pi^i) =\EE^i\bigsbr{{\bigrbr{\tilde U_h^t - u - P_h \tilde W_{h+1}^t}(s_h,a_h,b_h)}}, $$
    which should be bounded by $3H$.
\end{itemize}
We denote by $\dim(\cG_L) = \max_{h\in[H]}\dim_\DE(\cG_{h, L}, \sP_{h, L}, T^{-1/2})$ in the sequel.
Our guarantee for the sequence $\{g_h^i\}_{i\in[T]}$ and $\{\pi^i\}_{i\in[T]}$ is 
\begin{align}\label{eq:OnN-guarantee-Bellmanerror}
    \sum_{i=1}^{t-1} \rbr{g_h^t(\pi^i)}^2 
    &= \sum_{i=1}^t \rbr{\EE^i\bigsbr{{\bigrbr{\tilde U_h^t - u - P_h \tilde W_{h+1}^t}(s_h,a_h,b_h)}}}^2 \nend
    &\le \sum_{i=1}^t \EE^i\sbr{\rbr{\bigrbr{\tilde U_h^t - u_h - P_h \tilde W_{h+1}^t}(s_h,a_h,b_h)}^2}\nend
    &\le \sum_{i=1}^t 2 \EE^i\sbr{\rbr{\rbr{\tilde u_h^t - u_h}(s_h, a_h, b_h)}^2} + 8 H^2 \EE^i D_\TV^2\rbr{P_h(\cdot\given s_h, a_h, b_h), \tilde P_h^t(\cdot\given s_h, a_h, b_h)}\nend
    &\le 8 H^2 \cdot 4 \beta,
\end{align}
where we define $\tilde u_h^t = u_h^{M^t}$ and $\tilde P_h^t = P_h^{M^t}$. The first ineqality holds by the Cauchy-Schwartz inequality ,  the second inequality holds by noting that $\tilde U_h^t = \tilde u_h^t + \tilde P_h^t \tilde W_h^t$, and the last inequality holds by invoking the guarantee in \Cref{lem:MLE}. We then have by the first order argument in \Cref{lem:de-regret} that 
\begin{align*}
    \sum_{i=1}^T \abr{g_h^t(\pi^t)} \le 2\sqrt{\dim\rbr{\cG_L} 32 H^2 \beta T} + 3 H\min\cbr{T, \dim\rbr{\cG_L}}  + \sqrt T, 
\end{align*}
which implies that the leader's Bellman error is upper bounded by $\cO( H^2 \sqrt{\dim\rbr{\cG_L} H \beta T})$.

\paragraph{Step 3. Bounding the first-order Term.}
we next show that the first-order term in the follower's response error is also under control. 
\begin{itemize}[leftmargin=20pt]
    \item[(i)] Define function class $\cG_{h, F}^1$ as 
    \begin{align*}
        \cG_{h, F}^1 &= \Bigg\{g:(\cS\times\cA\times\cB)^{H-h+1}\rightarrow \RR \bigggiven \exists M\in\cM, \nend
        &\qqquad g((s_l, a_l, b_l)_{l=h}^H) = {\sum_{l=h}^H \gamma^{l-h}\bigrbr{r_l^M- r_l + \gamma (P_l^{M} - P_l) V_{l+1}^{\pi^M, M}}(s_l, a_l, b_l)}
        \Bigg\},
    \end{align*}
    where we remind the readers that $\pi^M = \argmax_{\pi\in\Pi}J(\pi, M)$ only depends on $M$. Consider 
    sequences 
    $$\cbr{g_h^t=\sum_{l=h}^H \gamma^{l-h}\orbr{\tilde r_l^t- r_l + \gamma (\tilde P_l^t - P_l) \tilde V_{l+1}^t}}_{t\in[T]}, $$
    where we define $\tilde r_l^t = r_l^{M^t}$ and $\tilde P_l^t = P_l^{M^t}$.
    It is obvious that $g_h^t\in\cG_{h, F}^1$ since $\tilde V_h^t = V_h^{\pi^t, M^t}$ and $\pi^t = \argmax_{\pi\in\Pi}J(\pi, M^t) = \pi^{M^t}$.
    % $r_h^{\pi}(s_h, b_h) = \dotp{r_h(s_h, \cdot, b_h)}{\pi_h(\cdot\given s_h, b_h)}_\cA$ and the same holds for $P_h^\pi$. 
    % One can see immediately that $\cG_F^1$ is a bilinear class, where the expectation part depends on $\pi$ and the follower's Bellman error part depends on $M$.
    
    \item Define a class of signed measures over $(\cS\times\cA\times\cB)^{H-h+1}$ as 
    \begin{equation*}
        \sP_{h, F}^1 = \cbr{\begin{aligned}
            &\PP^\pi(((s_l, a_l, b_l)_{l=h+1}^H , a_h)=\cdot \given s_h, b_h)\delta_{(s_h, b_h)}(\cdot) \nend
            &\quad - \PP^\pi(((s_l, a_l, b_l)_{l=h+1}^H , a_h, b_h)=\cdot \given s_h)\delta_{(s_h)}(\cdot) 
        \end{aligned}
        \bigggiven \pi\in\Pi, (s_h, b_h)\in\cS\times\cB}, 
    \end{equation*}
    where $\delta_{s_h, b_h}$ is the measure that puts measure $1$ on a single state-action pair $(s_h, b_h)$, and the conditional is well defined by the Markov property. Also, consider the following sequence, 
    \begin{align*}
        \Big\{\rho_h^t(\cdot)&=\PP^{\pi^t}(((s_l, a_l, b_l)_{l=h+1}^H , a_h)=\cdot \given s_h^t, b_h^t)\delta_{(s_h^t, b_h^t)}(\cdot) \nend
        &\qquad - \PP^{\pi^t}(((s_l, a_l, b_l)_{l=h+1}^H , a_h, b_h)=\cdot \given s_h^t)\delta_{(s_h^t)}(\cdot)\Big\}_{t\in[T]},
    \end{align*}
    and we also have $\rho_h^t\in\sP_{h, F}^1$.
    \item  
    In particular, we define $g_h^t(s_h^i, b_h^i, \pi^i)$ as the integral of $g_h^t$ with respect to $\rho_h^i$, which is given by
    $$g_h^t(s_h^i, b_h^i, \pi^i)= \rbr{\EE_{s_h^i, b_h^i}^{\pi^i}-\EE_{s_h^i}^{\pi^i}} \sbr{\sum_{l=h}^H \gamma^{l-h}\Bigrbr{\tilde r_l^t- r_l + \gamma (\tilde P_l^t - P_l) \tilde V_{l+1}^t}(s_l, a_l, b_l)},$$
    % One can check that $g_h^t\in\cG_F^1$ 
    Note that the sequence of signed measures is uniquely determined by $\{(s_h^t, b_h^t, \pi^t)\}_{t\in[T]}$. Moreover, we have $g_h^t(s_h^i, b_h^i, \pi^i)$ bounded by $\eff_H(\gamma)(2\nbr{r_h}_\infty + 2\nbr{V_{h+1}}_\infty) \le 4B_A \eff_H(\gamma)$, where we the definition of $B_A$ is available in \eqref{eq:define_BA};  
\end{itemize}
We define the maximal eluder dimension of $\cG_{h, F}^1$ with respect to $\sP_{h,F}^1$ as $$\dim(\cG_F^1) =\max_{h\in[H]} \dim_\DE(\cG_{h, F}^1,\sP_{h, F}^1, T^{-1/2}).$$
We first see what guarantee we have on the given sequences $\{g_h^t\}_{t\in[T]}$ and $\{(s_h^t, b_h^t, \pi^t)\}_{t\in[T]}$, 
\begin{align}
    &\sum_{i=1}^{t-1} \rbr{g_h^t(s_h^i, b_h^i, \pi^i)}^2 \nend
    & \quad =\sum_{i=1}^{t-1} \rbr{\rbr{\EE_{s_h^i, b_h^i}^{i}-\EE_{s_h^i}^{i}} \sbr{\sum_{l=h}^H \gamma^{l-h}\Bigrbr{\tilde r_l^t- r_l + \gamma (\tilde P_l^t - P_l) \tilde V_{l+1}^t}(s_l, a_l, b_l)}}^2 \nend
    & \quad \le 2\sum_{i=1}^{t-1} \rbr{\rbr{\EE_{s_h^i, b_h^i}^{i}-\EE_{s_h^i}^{i}} \sbr{\sum_{l=h}^H \gamma^{l-h}\Bigrbr{\tilde r_l^t- r_l + \gamma (\tilde P_l^t - P_l) V_{l+1}^{\pi^i, M^t}}(s_l, a_l, b_l)}}^2\nend
    &\qqquad + 2\sum_{i=1}^{t-1} \rbr{\rbr{\EE_{s_h^i, b_h^i}^{i}-\EE_{s_h^i}^{i}}\sbr{\sum_{l=h}^H \gamma^{l-h+1} (\tilde P_l^t - P_l) \bigrbr{V_{l+1}^{\pi^i, M^t} - \tilde V_{l+1}^t}(s_l, a_l,b_l)}}^2 \nend
    & \quad \le 2\sum_{i=1}^{t-1} \Biggrbr{\underbrace{
        \rbr{\EE_{s_h^i, b_h^i}^{i}-\EE_{s_h^i}^{i}} \sbr{\sum_{l=h}^H \gamma^{l-h}\Bigrbr{\tilde r_l^t- r_l + \gamma (\tilde P_l^t - P_l) V_{l+1}^{\pi^i, M^t}}(s_l, a_l, b_l)}
    }_{\ds\tilde\Delta^{(1)}_{h, \pi^i, M^t}(s_h, b_h)}}^2\nend
    &\qquad + \underbrace{C B_A^2 \eff_{H}(\gamma)\sum_{i=1}^{t-1} \rbr{\EE_{s_h^i, b_h^i}^i + \EE_{s_h^i}^i} \sbr{\sum_{l=h}^H \gamma^{l-h+1} D_\TV^2 \rbr{\tilde P_l^t(\cdot\given s_l, a_l,b_l), P_l(\cdot\given s_l, a_l,b_l)}}}_{\dr (i)}, \label{eq:OnN-1st-g-sq}
\end{align}
where the first inequaltiy holds by the Jensen's inequality, and the second inequality holds by upper bounding the difference $\tilde P_l^t - P_l$ by the TV distance. Here, we are able to use $B_A$ as the upper bound for the V functions by our discussion in \Cref{sec:app-notations}, and $C$ hides some universal constant.
Applying \Cref{lem:1st-ub} with $\tilde \Delta^{(1)}_h(s_h, b_h)$ replaced by
\begin{align*}
    \tilde \Delta^{(1)}_{h,\pi^i, M^t}(s_h, b_h) 
    &=  \rbr{\EE_{s_h, b_h}^{i} -\EE_{s_h}^{i}}\Biggsbr{\sum_{l=h}^H \gamma^{l-h} 
    {\rbr{Q_l^{\pi^i, M^t} - r_l^{\pi^i} - \gamma P_l^{\pi^i} V_{l+1}^{\pi^i, M^t}}(s_l, b_l)}}\nend
    & = \rbr{\EE_{s_h, b_h}^{i} -\EE_{s_h}^{i}}\Biggsbr{\sum_{l=h}^H \gamma^{l-h} 
    {\rbr{\tilde r_l^t - r_l + \gamma (\tilde P_l^t - P_l) V_{l+1}^{\pi^i, M^t}}(s_l, a_l, b_l)}},
\end{align*}
we obtain for all $t\in[T], h\in[H]$,
\begin{align}
    &\sum_{i=1}^{t-1}\bigrbr{\tilde \Delta_{h, \pi^i, M^t}^{(1)}(s_h^i, b_h^i)}^2  \nend
    &\quad \le 2 \sum_{i=1}^{t-1}\rbr{\rbr{\EE_{s_h^i, b_h^i}^i-\EE_{s_h^i}^i} \bigsbr{\orbr{Q_h^{\pi^i, M^t} -  Q_h^{\pi^i}}(s_h, b_h)}}^2 \nend
    &\qqquad + C \gamma^2  \rbr{\eta^{-1} +2 B_A}^2\eff_H(\gamma) \sum_{l=h+1}^H \gamma^{l-h-1} {\rbr{\EE_{s_h^i}^i+\EE_{s_h^i, b_h^i}^i}\sbr{D_\H^2(\nu_l^{\pi^i}(\cdot\given s_l), \nu_l^{\pi^i, M^t}(\cdot\given s_l))}}\nend
    &\quad\le  8 C_\eta^2 \beta  + 64 B_Q^2 \log\rbr{TH\cN_\infty(\cM, T^{-1})\delta^{-1}} + C \gamma^2  \rbr{\eta^{-1} +2 B_A}^2\eff_H(\gamma)^2  \beta \nend
    &\quad \le \cO\rbr{ (\eta^{-1}+2B_A)^2 \eff_H(\gamma)^2  \beta},\label{eq:OnN-1st-Delta1-sq}
\end{align}
where the second inequality holds by \eqref{eq:MLE-guarantee-Q-3} in \Cref{lem:MLE-formal}, and the covering number $\cN_\varrho(\cM, \epsilon)$ is with respect to the infinite norm of the Q function. Here, $\cO$ only hides universal constant independent of $H,\eta, T$. 
Meanwhile, for the TV distance term in \eqref{eq:OnN-1st-g-sq}, we have also by \Cref{lem:MLE} that 
\begin{align*}
    {\dr (i)}\le 8 B_V^2 \eff_{H}(\gamma){\sum_{l=h}^H \gamma^{l-h+1} 8\beta }\le \cO(B_A^2 \eff_H(\gamma)^2 \beta),
\end{align*}
where $\cO$ only hides some universal constants.
Hence, we conclude that 
\begin{align*}
    \sum_{i=1}^{t-1} \rbr{g_h^t(s_h^i, b_h^i, \pi^i)}^2  \le \cO\rbr{\rbr{\eta^{-1}+B_A}^2 \eff_H(\gamma)^2 \beta}.
\end{align*}
Now, for the first-order term, we have  
\begin{align*}
    &C^{(0)}
    \sum_{t=1}^T \sum_{h=1}^H{\EE^t\sbr{\abr{\tilde \Delta^{(1,t)}_h(s_h, b_h)}}}\nend
    &\quad \le 2 C^{(0)} \sum_{h=1}^H \underbrace{\sum_{t=1}^T \abr{\tilde\Delta^{(1, t)}_h(s_h^t, b_h^t)}}_{\ts \sum_{t=1}^T |g_h^t(s_h^t, b_h^t, \pi^t)|} + H C^{(0)} \cdot 4 \bignbr{\tilde \Delta^{(1)}_h}_\infty \log\rbr{H\cN_\rho(\cM, T^{-1})\delta^{-1}} \nend
    &\quad \le \cO\rbr{ HC^{(0)} \eff_H(\gamma)\sqrt{\dim(\cG_F^1)\rbr{\eta^{-1}+B_A}^2  \beta T }}\nend
    &\quad\le \cO\rbr{ H^2 \eff_H(\gamma)\rbr{1+\eta B_A} \sqrt{\dim(\cG_F^1)\beta T}},
\end{align*}
where the first inequality follows from a standard martingale concentration in \Cref{cor:martigale concentration}, and the second inequality holds by using the first order regret bound in \Cref{lem:de-regret}, and the last inequality holds by $C^{(0)}=2\eta H$.

\paragraph{Step 3. Bounding the Second-Order Term.}
Previously, we decompose the online regret and obtain a second-order term, which we referred to as (ii), 
\begin{align*}
    {\dr (ii)}\defeq C^{(2)}
    \sum_{t=1}^T \max_{h\in [H]} {\EE^t\sbr{ \rbr{\rbr{\tilde Q_h^t - r_h^{\pi^t} - \gamma P_h^{\pi^t} \tilde V_{h+1}^t}(s_h, b_h)}^2}}.
\end{align*}
For this term, we specify the function class to use for our purpose, 

\begin{itemize}[leftmargin =20pt]
    \item[(i)] We take the same function class $\cG_F^2$ as
    \begin{align*}
        \cG_{h, F}^2 &= \Bigg\{g:\cS\times\cA\times\cB\rightarrow \RR: \exists M\in\cM, h\in[H] \nend
        &\qqquad g(s_h, b_h, a_h) = {\bigrbr{r_h^M- r_h + \gamma (P_h^{M} - P_h) V_{l+1}^{\pi^M, M}}(s_h, a_h, b_h)}
        \Bigg\},
    \end{align*}
    where $\cG_F^2$ is bounded by $4B_A$. 
    \item[(ii)] We define a class of probability measures on $\cS\times\cA\times\cB$ as $$\sP_{h, F}^2 = \cbr{\PP^\pi(a_h=\cdot\given s_h, b_h)\delta_{(s_h, b_h)}(\cdot)\given \pi\in\Pi, (s_h,b_h)\in\cS\times\cB}, $$
    where $\delta_{(s_h,b_h)}(\cdot)$ is the measure that assigns $1$ to the state-action pair $(s_h, b_h)$. 
    \item[(iii)] We take a sequence of functions $\{g_h^t\}_{t\in[T]}$ as $\{g_h^t = \tilde r_h^{t}- r_h + \gamma (\tilde P_h^t - P_h) \tilde V_{l+1}^{t}\}_{t\in[T]}$, and take a sequence of probability measures as $\{\rho_h^t(\cdot) = \PP^{\pi^t}(a_h=\cdot\given s_h^t, b_h^t)\delta_{(s_h^t, b_h^t)}(\cdot)\}_{t\in[T]}$, where we define $\tilde r_h^t = r_h^{M^t}$ and $\tilde P_h^t = P_h^{M^t}$.
    One can check that $g_h^t\in\cG_{h,F}^1$ since $\tilde V_h^t = V_h^{\pi^t, M^t}$ and we have $\pi^t = \argmax_{\pi\in\Pi}J(\pi, M^t) = \pi^{M^t}$. In addition, we define $g_h^t(s_h^i, b_h^i, \pi^i)$ as the integral of $g_h^t$ with respect to $\rho_h^i$, which is given by
    \begin{align*}
        g_h^t (s_h^i, b_h^i, \pi^i) = \EE_{s_h^i, b_h^i}^{\pi^i} \sbr{\bigrbr{\tilde r_h^{t}- r_h + \gamma (\tilde P_h^t - P_h) \tilde V_{l+1}^{t}}(s_h, a_h, b_h)},
    \end{align*}
    Note that the sequence of probability measures is uniquely determined by $\{(s_h^t, b_h^t, \pi^t)\}_{t\in[T]}$.
\end{itemize}
We let $\dim(\cG_F^2)=\max_{h\in[H]}\dim_\DE(\cG_{h,F}^2,\sP_{h, F}^2, T^{-1/2})$ be the eluder dimension.
We next establish guarantee for $\sum_{i=1}^{t-1} (g_h^t(s_h^i, b_h^i, \pi^i))^2$.
\begin{align}\label{eq:OnN-guarantee-2ndQRE}
    &\sum_{i=1}^{t-1} (g_h^t(s_h^i, b_h^i, \pi^i))^2 \nend
    &\quad = \sum_{i=1}^{t-1} \rbr{\EE_{s_h^i, b_h^i}^{i} \sbr{\bigrbr{\tilde r_h^{t}- r_h + \gamma (\tilde P_h^t - P_h) \tilde V_{h+1}^{t}}(s_h, a_h, b_h)}}^2\nend
    &\quad  \le 2\sum_{i=1}^{t-1} \rbr{\EE_{s_h^i, b_h^i}^{i} \sbr{\bigrbr{\tilde r_h^{t}- r_h + \gamma (\tilde P_h^t - P_h) V_{h+1}^{\pi^i, M^t}}(s_h, a_h, b_h)}}^2 \nend
    &\qqquad + 2\cdot 16B_A^2 \sum_{i=1}^{t-1} \rbr{\EE_{s_h^i, b_h^i}^i\sbr{D_\TV^2\rbr{\tilde P_h^t(\cdot\given s_h, a_h, b_h), P_h(\cdot\given s_h, a_h, b_h)}}}\nend
    &\quad \lesssim 3 \underbrace{\sum_{i=1}^{t-1}\EE^i\rbr{\EE_{s_h, b_h}^{i} \sbr{\bigrbr{\tilde r_h^{t}- r_h + \gamma (\tilde P_h^t - P_h) V_{h+1}^{\pi^i, M^t}}(s_h, a_h, b_h)}}^2}_{\dr (iii)} + 32 B_A^2 \log\rbr{TH\cN_\rho(\cM,T^{-1}\delta^{-1})}\nend
    &\qqquad + 48 B_A^2 \underbrace{\sum_{i=1}^{t-1} \EE^i\sbr{D_\TV^2\rbr{\tilde P_h^t(\cdot\given s_h, a_h, b_h), P_h(\cdot\given s_h, a_h, b_h)}}}_{\dr (iv)} + 64 B_A^2 \log\rbr{TH\cN_\rho(\cM, T^{-1})}
\end{align}
where in the first inequality, we use $2B_A$ to upper bound $\|\tilde V_{h+1}^t - V_{h+1}^{\pi^i, M^t}\|_\infty$, and uses the Cauchy-Schwartz inequality to move the square into the expectation.
Here, the second inequality uses a standard martingale concentration result in \Cref{cor:martigale concentration} for both terms, and we invoke the same upper bound $B_A$ for the V functions.
Now, we invoke \Cref{lem:2nd-ub}, which says that term (iii) enjoys the following upper bound, 
\begin{align*}
    {\dr (iii)} &=\sum_{i=1}^{t-1}\EE^i\rbr{\EE_{s_h, b_h}^{i} \sbr{\bigrbr{\tilde r_h^{t}- r_h + \gamma (\tilde P_h^t - P_h) V_{h+1}^{\pi^i, M^t}}(s_h, a_h, b_h)}}^2 \nend
    & = \sum_{i=1}^{t-1}\EE^i\rbr{{\bigrbr{Q_h^{\pi^i, M^t} - r_h^{\pi^i} - \gamma P_h^{\pi^i} V_{h+1}^{\pi^i, M^t}}(s_h, a_h, b_h)}}^2\nend
    &\le L^{(2)} \sum_{i=1}^{t-1} \max_{h\in[H]} \cbr{\EE^i D_\H^2\bigrbr{\nu_h^{\pi^i, M^t}(\cdot\given s_h), \nu_h^{\pi^i}(\cdot\given s_h)}+ \EE^i D_\TV^2\bigrbr{P_h^{\pi^i, M^t}(\cdot\given s_h, b_h), P_h^{\pi^i}(\cdot\given s_h, b_h)}}\nend
    &\le L^{(2)} 4H \beta , 
\end{align*}
where the last inequality uses the guarantee in \Cref{lem:MLE}. To enable a direct use of the MLE guarantee, we replace the maximum by a sum over all $h\in[H]$ and upper bound the TV distance by the Hellinger distance. Here, $L^{(2)}$ is defined in \Cref{lem:2nd-ub}. For term (iv), we use the same guarantee in \Cref{lem:MLE} and the same bounding the TV distance by the Hellinger distance argument and obtain ${\dr (iv)}\le 4\beta$. Therefore, we conclude that 
\begin{align*}
    \sum_{i=1}^{t-1} (g_h^t(s_h^i, b_h^i, \pi^i))^2 \lesssim {\bigrbr{L^{(2)} H + B_A^2} \beta},
\end{align*}
where $\lesssim$ only hides some universal constant. As a result of the second order regret in \Cref{lem:de-regret}, we have 
\begin{align*}
    \sum_{t=1}^{T} (g_h^t(s_h^t, b_h^t, \pi^t))^2 
    &\lesssim {\dim(\cG_F^2) \bigrbr{L^{(2)} H + B_A^2} \beta  + \min\cbr{T, \dim(\cG_F^2)} 16 B_A^2+1}\nend
    &\lesssim {H \dim(\cG_F^2) \beta  L^{(2)}},
\end{align*}
where $\lesssim$ hides some universal constant. Therefore, we conclude that
\begin{align*}
    {\dr (ii)}
    &\defeq C^{(2)}
    \sum_{t=1}^T \max_{h\in [H]} {\EE^t\sbr{ \rbr{\rbr{\tilde Q_h^t - r_h^{\pi^t} - \gamma P_h^{\pi^t} \tilde V_{h+1}^t}(s_h, b_h)}^2}}\nend
    &\le C^{(2)} \sum_{t=1}^T \sum_{h=1}^H \EE^t \sbr{ \rbr{\rbr{\tilde Q_h^t - r_h^{\pi^t} - \gamma P_h^{\pi^t} \tilde V_{h+1}^t}(s_h, b_h)}^2} \nend
    & \le 2C^{(2)} \sum_{t=1}^T \sum_{h=1}^H \EE^t_{s_h^t, b_h^t} \sbr{ \rbr{\rbr{\tilde Q_h^t - r_h^{\pi^t} - \gamma P_h^{\pi^t} \tilde V_{h+1}^t}(s_h, b_h)}^2} \nend
    &\qquad + 4 H C^{(2)} 9 B_A^2 \log\rbr{H \cN_\rho(\cM,T^{-2})\delta^{-1}}. 
\end{align*}
where in the first inequality, we replace the maximum by the summation over $h\in[H]$ and in the second inequality, we invoke the martingale concentration in \Cref{cor:martigale concentration}.
Hence, we establish our bound for the second-order term as
\begin{align*}
    {\dr(ii)} 
    &\le 2 C^{(2)} \sum_{h=1}^H \sum_{t=1}^T \rbr{g_h^t(s_h^t, b_h^t, \pi^t)}^2 + 4 H C^{(2)} 9 B_A^2 \log\rbr{H \cN_\rho(\cM,T^{-2})\delta^{-1}}\nend
    &\lesssim {H^2 C^{(2)} \dim(\cG_F^2) \beta  L^{(2)}} +   { H C^{(2)} \beta}
    % \nend
    % &\le \cO\rbr{H^6\eff_H(\gamma)\rbr{\eff_H(\exp(2\eta B_A)\gamma)}^2 (\iota^{2H}+1) \kappa^2 \exp\rbr{14\eta B_A} (1+\eta^2B_A^2)  \dim(\cG_F^2) \beta }
\end{align*}
where $L^{(2)} = c H^2 \eff_H(c_2)^2 \kappa^2 \exp\rbr{8\eta B_A} C_\eta^2$ with 
$c_2 = \gamma(2\exp(2\eta B_A)+\kappa\exp(4\eta B_A))$, and $C^{(2)} = 2 H^2\eta^2 (1+4 \eff_H(\gamma))\exp\rbr{6\eta B_A}\cdot \rbr{\eff_H(\exp(2\eta B_A)\gamma)}^2$.

In summary, for the leader's Bellman error, we have
\begin{align*}
    \text{LBE}\lesssim H(2\sqrt{\dim\rbr{\cG_L} H^2 \beta T} + 3 H\min\cbr{T, \dim\rbr{\cG_L}}  + \sqrt T) \lesssim H^2 \sqrt{\dim\rbr{\cG_L} \beta T}, 
\end{align*}
for the first-order term in the follower's QRE,
\begin{align*}
    {\text{1st-QRE}}\lesssim  H^2 \eff_H(\gamma) \eta C_\eta \sqrt{\dim(\cG_F^1)\beta T}, 
\end{align*}
where $C_\eta = \eta^{-1}+B_A$,
and for the second-order term in the follower's QRE, 
\begin{align*}
\text{2nd-QRE}\lesssim H^2 C^{(2)} L^{(2)} \dim(\cG_F^2) \beta \log T
\end{align*}
which completes the proof for \Cref{thm:OMLE-farsighted}.

% Specifically, for any leader's policy $\pi\in\Pi$, the true model $M^*$ and an alternative model $\tilde M\in\cM$, we have
% \begin{align}
%     &J(\pi, \tM) - J(\pi, M^*) \nend
%     &\quad = {\sum_{h=1}^H \EE^{\pi, M^*}\sbr{\tilde U_h(s_h, a_h, b_h) - u_h(s_h, a_h, b_h)- \tilde W_{h+1}(s_{h+1})}} + \sum_{h=1}^H \EE^{\pi, M^*}\sbr{\tilde W_h(s_h)-\tilde U_h(s_h, a_h, b_h)}\nend
%     &\quad\le\underbrace{{\sum_{h=1}^H \EE^{\pi, M^*}\sbr{\tilde U_h(s_h, a_h, b_h) - u_h(s_h, a_h, b_h)- \tilde W_{h+1}(s_{h+1})}}}_{\dr \text{temporal difference}}+ \underbrace{\sum_{h=1}^H  H \EE \nbr{\tnu_h(\cdot\given s_h)-\nu_h(\cdot\given s_h)}_1}_{\dr \text{response difference}}, \label{eq:performance diff}
% \end{align}
% where $\tilde U_h, \tilde W_h, \tilde\nu_h$ are the correspondences of $U_h, W_h, \nu_h$ under model $\tilde M$.
% In the following, we denote the above $h$-step temporal difference by $\cE^{(0)}_h(\tM, M^*;\pi)$, which is a standard result from classical MDP.
% We call the second term response difference since the second term characterize the TV distance between the follower's true response and an estimated response learned by the leader.
% Here, we have to invoke the TV bound since the follower's utility is not aligned with the leader's. 
% A unique challenge of our problem is to characterize the response difference in the face of a strategic follower.
% The following lemma states that the  follower's response difference admits a \say{Taylor expansion} flavored decomposition. 
% \begin{lemma}[\textit{Response difference for farsighted follower}]
%     % \label{lem:performance diff}
% For any given leader's policy $\pi$, the true model $M^*$ and an alternative model $\tM\in\cM$, using the denotions specified in \Cref{tab:notation}, the response difference term in \eqref{eq:performance diff} can be upper bounded by
% \begin{align}
%     &\sum_{h=1}^H  H \EE \nbr{\tnu_h(\cdot\given s_h)-\nu_h(\cdot\given s_h)}_1 \nend
%     &\quad \le 2\eta  H \cdot 
%     \sum_{h=1}^H \underbrace{\EE^{\pi, M^*}\sbr{\abr{\rbr{\EE^{\pi, M^*}_{s_h, b_h}-\EE^{\pi, M^*}_{s_h}}\bigsbr{\Delta^{(1)}_{h,\pi, \tM}(s_h, b_h)}}}}_{\ds \cE^{(1)}_h(\tM, M^*;\pi)} \nend
%     & \qqquad + \underbrace{\eta^2  H  \rbr{1+ 4 \frac{1-\gamma^H}{1-\gamma}}}_{\ds C^{(1)}} \cdot 
%     \sum_{h=1}^H \underbrace{\EE^{\pi, M^*}\sbr{ \exp\bigrbr{\eta\bigabr{A_h-\tilde A_h}}\cdot \bigabr{\tilde A_h - A_h}^2}}_{\ds \cE^{(2)}_h(\tM, M^*;\pi)}, \label{eq:response decomposition}
% \end{align}
% where $ H = \max_{h\in[H], \pi\in\Pi, M\in\cM}\bignbr{U_h^{\pi, M}}_\infty$ and $\Delta^{(1)}_{h,\pi, M}(s_h, b_h)$ is defined as
% \begin{align}
%     \Delta^{(1)}_{h,\pi, \tM}(s_h, b_h) &\defeq  \EE_{s_h, b_h}\sbr{\sum_{i=h}^H \gamma^{i-h}\rbr{r_i^\pi(s_h, b_h)-\tilde r_i^\pi(s_h, b_h) + \gamma \rbr{\bigrbr{\TT_i^\pi -\tilde\TT_i^\pi}\tilde V_{i+1}}(s_h, b_h)}}.\label{eq:def Delta^1}
% \end{align}
% \begin{proof}
%     See \Cref{sec:performance diff} for a detailed proof.
% \end{proof}
% \end{lemma}

% Therefore, the online regret can be reorganized into
% \begin{align*}
%     \Reg(T)\le \sum_{t=1}^T \Biggrbr{\underbrace{\sum_{h=1}^H\cE^{(0)}_h\bigrbr{M^t, M^*;\pi^t} + 2\eta H\sum_{h=1}^H \cE^{(1)}_h\bigrbr{M^t, M^*;\pi^t}}_{\ds I_t: \text{1st-order terms}} + \underbrace{C^{(1)} \sum_{h=1}^H \cE^{(2)}_h\bigrbr{M^t, M^*;\pi^t}}_{\ds J_t: \text{ 2nd-order term}}}.
% \end{align*}
% Here, we notice that $\cE^{(0)}$ and $\cE^{(1)}$ are first order terms in the sense that both $\cE^{(0)}$ and $\cE^{(1)}$ have a linear form while $\cE^{(2)}$ is the second order term and depends quadratically on $|\tilde A_h - A_h|$. In the sequel, we aim to bound them separately.

% \paragraph{Step 2. Bounding the First Order Terms.}
% A key point for bounding the first order terms is relating the guarantee of MLE in \Cref{lem:MLE} to the first order error $\cE^{(0)}, \cE^{(1)}$, which is done by a characterization in terms of the Eluder dimensions.
% % \begin{lemma}[\textit{Eluder dimension bound, Lemma 41 in \citet{jin2021bellman}}] \label{lem:1st-eluder}
% %     Given a function class $\cF$ defined on $\cX$ with $\abr{f(x)}<C$ for all $(f,x)\in \cF\times \cX$. Suppose sequences $\{f_t\}_{t=1}^T$ and $\{x_t\}_{t=1}^T$ satisfy that for all $t\in[T]$, $\sum_{j=1}^{t-1} f_j(x_t)^2\le \beta$. Then for all $t\in[T]$ and $w>0$, we have the first order guarantee
% %     \begin{align*}
% %         \sum_{j=1}^t \abr{f_j(x_j)} \le 2\sqrt{\dimE(\cF, \cX, w)\beta t} + \min\cbr{t, \dimE(\cF, \cX, w)} C + t w, 
% %     \end{align*}
% %     and also the second order guarantee
% %     \begin{align*}
% %         \sum_{j=1}^t f_{j}(x_j)^2 \le \dimE(\cF, \cX, w) \beta \log t + \min\cbr{t, \dimE(\cF, \cX, w)} C^2 + t w^2.
% %     \end{align*}
% %     \begin{proof}
% %         \todo{to be added.}
% %     \end{proof}
% % \end{lemma} 
% Note that \Cref{lem:1st-eluder} is slightly different from the original version in the sense that we include a second order guarantee, which is an adaptation of the original proof. 
% Recall that 
% \begin{align*}
%     \cE^{(0)}_h(\tilde M, M^*;\pi) &= \EE^{\pi, M^*}\sbr{\tilde U_h(s_h, a_h, b_h) - u_h(s_h, a_h, b_h)- \tilde W_{h+1}(s_{h+1})}\nend
%     & =  \EE^{\pi, M^*}\sbr{\tilde u_h  - u_h + \rbr{\tilde \TT_h - \TT_h} \tilde W_{h+1}} ,
% \end{align*}
% and by definition of $\Delta^{(1)}$ in \eqref{eq:def Delta^1} that
% \begin{align*}
%     \cE^{(1)}_h(\tilde M, M^*;\pi)
%     &=\EE^{\pi, M^*}\sbr{\abr{\rbr{\EE^{\pi, M^*}_{s_h, b_h}-\EE^{\pi, M^*}_{s_h}}\bigsbr{\Delta^{(1)}_{h,\pi, \tM}(s_h, b_h)}}}\nend
%     &= \EE^{\pi, M^*}\abr{\sum_{i=h}^H\gamma^{i-h}\rbr{\EE_{s_h, b_h}^{\pi, M^*} - \EE_{s_h}^{\pi, M^*}}\sbr{\tilde r_i-r_i+\gamma \rbr{\tilde\TT_i - \TT_i}\tilde V_{i+1}}}.
% \end{align*}
% Now, the task boils down to finding a $f$ such that: (i) the first order terms $\cE^{(0)}$ and $\cE^{(1)}$ are bounded by this $f$; (ii) $\sum_{j=1}^{t-1} f_j^2\le \cO(\beta)$ can be guaranteed by the MLE.
% Following this spirit, we construct $f_{0,h}^M, f_{1, h}^M$ as
% \begin{align*}
%     f_{0,h}^M(\pi) &\defeq  \EE^{\pi, M^*}\sbr{u_h^M  - u_h^{M^*} + \rbr{\TT_h^M - \TT_h^{M^*}} W_{h+1}^{\pi_\opt^M, M} }, \nend
%     f_{1,h}^M(\pi) & \defeq \EE^{\pi, M^*}\abr{\sum_{i=h}^H\gamma^{i-h}\rbr{\EE_{s_h, b_h}^{\pi, M^*} - \EE_{s_h}^{\pi, M^*}}\sbr{ r_i^M-r_i^{M^*}+\gamma \rbr{\TT_i^M - \TT_i^{M^*}}V_{i+1}^{\pi_\opt^M, M}}}, 
% \end{align*}
% where $\pi^M_\opt$ is the optimistic policy for the leader under model $M$.
% One can easily check that $f_{0,h}^{M^t}(\pi^t)= \cE^{(0)}_h\bigrbr{M^t, M^*;\pi^t}$ for the temporal difference error and $f_{1,h}^{M^t}(\pi^t) = \cE^{(1)}_h\bigrbr{M^t, M^*;\pi^t}$ for the first order term in the response difference error. We would like to point out that the definitions of $f_0$ and $f_1$ are different from $\cE^{(0)}, \cE^{(1)}$, and the above equality relationship between the function $f$ and the error $\cE$ only holds if $\pi^t=\pi_\opt^{M^t}$, which makes $\pi^t$ a function of $M^t$.
% In the sequel, let $\cF_{0,h}=\{f_{0,h}^M(\cdot):M\in\cM\}$ and $\cF_{1,h}=\{f_{1,h}^M(\cdot):M\in\cM\}$.
% % Now that the first condition that $f$ bounds the error at each round $t$ holds.
% For the second condition of \Cref{lem:1st-eluder} that $\sum_{j=1}^{t-1} f^2_j(x_t)\le \cO(\beta)$, we can check that
% \begin{align*}
%     f_{0,h}^{M}(\pi) &\le \EE\sbr{\abr{u_h^M-u_h^{M^*}} + 2 H D_\TV \rbr{T_h^M, T_h^{M^*}}} 
%     % &\le H \sum_{h=1}^H \EE\sbr{\rbr{u_h^M-u_h^{M^*}}^2 + 4 H^2 D_\H^2\rbr{T_h^M, T_h^{M^*}}}\nend
%     \le (2 H +1) D_{\RL,h}(M, M^*;\pi).
% \end{align*}
% where 
% % the first inequality holds by the Cauchy-Schwartz inequality, and
% the second inequality holds by noting that the TV distance is upper bounded by the Hellinger distance used in $D_\RL$. 
% Therefore, we have $\sum_{j=1}^{t-1}f_{0,h}^{M^t}(\pi^j)^2 \le (2H+1)^2 \sum_{j=1}^{t-1} D_{\RL,h}^2(M^t, M^*;\pi^j)\le 4(2H+1)^2 \beta$ by \Cref{lem:MLE}.
% The case for $f_1$ is more complicated since we have to control the follower's reward difference via the follower's behavior difference. We invoke the following lemma.
% % \begin{lemma}[\textit{Bounding $f_1$ by $D_\RL$}]\label{lem:1st-ub}
% % For any $\eta>0$, $\pi\in\Pi$ and $\tM\in\cM$, we have
% % \begin{align*}
% %     f_{1,h}^{\tM}(\pi) \le   \underbrace{6(1+2\eta B_A)\cdot
% %     \sqrt{C_{\gamma, H}}}_{\ds L^{(1)}}  \cdot
% %     \eta^{-1}\max_{h\in[H]} \EE^{\pi, M^*} D_\TV(\nu_h,\tilde\nu_h), 
% % \end{align*}
% % where $C_{\gamma, H} = \orbr{1-\gamma^H}/\orbr{1-\gamma}$ is the effective foresight of the follower.
% %     \begin{proof}
% %         See \Cref{sec:1st-ub} for a detailed proof.
% %     \end{proof}
% % \end{lemma}
% We see from \Cref{lem:1st-ub} that $\sum_{j=1}^{t-1} f_{1,h}^{M^t}(\pi^j)^2\le \eta^{-2}\rbr{L^{(1)}}^2 \sum_{j=1}^{t-1} D_{\RL,h}^2(M^t, M^*;\pi^j)\le 4\eta^{-2}(L^{(1)})^2 \beta$. Hence, $f_2$ also satisifies the condition in \Cref{lem:1st-eluder}.
% Combining the discussions for $f_0$ and $f_1$ and invoking \Cref{lem:1st-eluder} together with $\sum_{s<t}D_\RL^2(M^t, M^*;\pi^s)\le 4\beta$ in \Cref{lem:MLE}, we have (for $w=1/\sqrt T$ in \Cref{lem:1st-eluder}) that
% \begin{align*}
%     \sum_{t=1}^T I_t &\le  H\rbr{2\sqrt{\dimE(\cF_0, \Pi, 1/\sqrt T) (2H+1)^2 \cdot 4\beta \cdot T} + \min\cbr{T, \dimE(\cF_0, \Pi, 1/\sqrt T)} B_{\cF_0} + \sqrt T}\nend
%     &\qquad + 2\eta H  H \cdot \Big( 2L^{(1)} \eta^{-1}\sqrt{\max_h\dimE(\cF_{1,h}, \Pi, 1/\sqrt T)  \cdot 4\beta \cdot T} \nend
%     &\qquad + \min\cbr{T, \max_h\dimE(\cF_{1,h}, \Pi, 1/\sqrt T)} B_{\cF_1} + \sqrt T \Big)\nend
%     &\le 
%     \rbr{ 4H (2H+1) \sqrt{5 d_0 \beta} + 8 H  H L^{(1)} \sqrt{d_1\beta} + 2} \sqrt{T} + \min\cbr{T, d_1, d_2} \rbr{B_{\cF_0} + 2\eta H  H B_{\cF_1}}, 
% \end{align*}
% where we let $d_0 = \max_{h\in[H]}\dimE(\cF_{0,h}, \Pi, 1/\sqrt T)$, $d_1=\max_h\dimE(\cF_{1,h}, \Pi, 1/\sqrt T)$, $B_{\cF_0}=\max_{f\in\cF, \pi\in\Pi}|f_0(\pi)|$, and $B_{\cF_1}=\max_{h\in[H],f\in\cF, \pi\in\Pi}|f_{1,h}(\pi)|$. Keeping the dominant terms for simplicity, we have
% \begin{align*}
%     \sum_{t=1}^T I_t\le \cO\rbr{H^{2} \sqrt{d_0\beta T} + (1+\eta B_A) C_{\gamma, H} H^2\sqrt{d_1\beta T}}.
% \end{align*}
% \paragraph{Step 3. Bounding the 2nd-Order Term.}
% Recall the second order error
% \begin{align*}
%     \cE_h^{(2)}(\tilde M, M^*, \pi) = \EE^{\pi, M^*}\sbr{ \exp\bigrbr{\eta\bigabr{A_h-\tilde A_h}}\cdot \bigabr{\tilde A_h - A_h}^2}\le \exp\bigrbr{2\eta B_A}\cdot \EE^{\pi, M^*}\sbr{ \bigabr{\tilde A_h - A_h}^2}.
% \end{align*}
% For the second order term, suppose that we can take some $f_{2,h}^M(\pi)$ such that $f_{2,h}^M(\pi)\ge \EE^{\pi, M^*}\bigabr{A_h^{\pi, M}-A_h^{\pi,M^*}}$ and $f_{2,h}^M(\pi)\le L^{(2)} D_\RL(M, M^*;\pi)$ for some $L^{(2)}>0$.
% Following the second result in \Cref{lem:1st-eluder}, we let $w=1/\sqrt T$ and obtain
% \begin{align*}
%     \sum_{t=1}^T J_t &\le C^{(1)} \exp\rbr{2\eta B_A}H \cdot \Big(4L^{(2)}\max_h\dimE(\cF_{2, h}, \cX, 1/\sqrt T) \beta \log T\nend 
%     &\qquad + \min\cbr{T, \max_h\dimE(\cF_{2 ,h}, \cX, 1/\sqrt T)} B_{\cF_{2}}^2 + 1\Big)\nend
%     &\le \rbr{4 L^{(2)} d_2 \beta \log T + \max\cbr{T, d_2} B_{\cF_2}^2 + 1} 5 \eta^2  H   \frac{1-\gamma^H}{1-\gamma} H\exp\rbr{2\eta B_A}, 
% \end{align*}
% where $d_2=\max_h\dimE(\cF_{2,h}, \Pi, 1/\sqrt T)$, $B_{\cF_1}=\max_{h\in[H],f\in\cF, \pi\in\Pi}|f_{2,h}(\pi)|$. Note that the second order term only scales with $\log(T)$. Hence, the second order term is not significant as $T$ grows large. For simplicity, we have
% \begin{align*}
%     \sum_{t=1}^T J_t\le \cO\rbr{\eta^2 H^2 C_{\gamma, H} \exp\rbr{2\eta B_A} L^{(2)} d_2 \beta \log T}.
% \end{align*}
% Combining the results from the first order terms and the second order term, we finish our proof of \Cref{thm:OMLE-farsighted}.




