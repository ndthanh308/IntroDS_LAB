\section{More Details of Technical Ingredients}\label{sec:app-major-tech}

In this section, we summarize and provide  proofs of the  important techniques used for analyzing the QSE. The following is a table of addition constants used in this section. 
\begin{table}[h!]
    \centering
    {\setlength\doublerulesep{1pt}
    \begin{tabular}{p{2cm}|p{10.5cm}}
        \toprule[2pt]\midrule[0.5pt]
        Notations & Interpretations \\ \toprule[1.5pt]
        $C^{(0)}$   & $C^{(0)}=2\eta H$ \\\midrule
        $C^{(1)}$ & $C^{(1)}=\eta^2 H   \bigrbr{1+ 4 \eff_H(\gamma)}\cdot \exp(2\eta B_A)$ \\\midrule
        $C^{(2)}$ & $
           C^{(2)}   =  2  \eta^2 H^2\cdot \exp\orbr{6\eta B_A} \cdot(1+4 \eff_H(\gamma)) \cdot \rbr{\eff_H(\exp(2\eta B_A)\gamma)}^2$ 
        \\\midrule
        $C^{(3)}$ & $C^{(3)}={\eta^2 \exp(2\eta B_A)}\bigrbr{2+\eta B_A \cdot  \exp\rbr{2\eta B_A}}/2$
        \\\midrule
        $L^{(1)}$ & $L^{(1)} = 6(\eta^{-1}+2 B_A)\cdot {\eff_H\rbr{\gamma}}$
        \\\midrule 
        $L^{(2)}$ & $L^{(2)} = c H^2 \eff_H(c_2)^2 \kappa^2 \exp\rbr{8\eta B_A} (\eta^{-1}+B_A)^2$, where 
        $c_2 = \gamma(2\exp(2\eta B_A)+\kappa\exp(4\eta B_A))$, and $c$ is a universal constant.\\
        \bottomrule[2pt]
    \end{tabular}
    }
    \caption{Constants used for \Cref{sec:app-major-tech}}\label{tab:}
\end{table}

\subsection{Performance Difference Lemma for for QSE}\label{sec:app-subopt-decompose}

In this subsection, we further elaborate on the performance difference lemma introduced in \S\ref{sec:subopt decomposition}. 
For generality, we consider the farsighted case.  
In the following,  
we consider a fixed policy $\pi$ and let its quantal response under the true model be $\nu^{\pi}$. 
Let $\tilde \nu$ be an estimate of $\nu^{\pi}$ and let $\tilde U$ and $\tilde W$ be any estimates of  $U^{\pi} $ and $  W^{\pi}$, which are defined respectively in    \eqref{eq:U_function} and \eqref{eq:W_function}.
We note that $\tilde W$ and $\tilde U$ not necessarily satisfy $
\tilde W_h(s ) = \la \tilde U_h (s, \cdot , \cdot ) , \pi _h \otimes \tilde \nu_h^{\pi} (\cdot , \cdot \given s) \ra . 
$.
We present a slightly more general version of   the performance difference lemma, which directly implies \eqref{eq:performance diff-1}.

% {\main Lemma \ref{lem:performance diff informal}\fi} {\neurips \eqref{eq:performance diff-1}\fi}. 


\begin{lemma}[Performance Difference]\label{lem:subopt-decomposition} 
For any fixed policy $\pi$, 
let $\tilde \nu$ be an estimate of the quantal response $\nu^{\pi}$ and let $\tilde U$ and $\tilde W$ be estimates of $U^\pi$ and $W^{\pi}$ respectively. 
Based on $\tilde U$ and $\tilde W$, we can estimate $J(\pi)$ defined in \eqref{eq:J} by 
\$
\EE_{s_1\sim\rho_0}\bigl [ \orbr{T_1^{\pi,\tilde\nu}\tilde U_1} (s_1 )\bigr ]  \qquad \textrm{and} \qquad \EE_{s_1\sim\rho_0}\bigl [ \tilde W_1(s_1)\bigr]  , 
\$
where we operator $T_h^{\pi,\tilde\nu}$ is defined in \eqref{eq:operator_T_pi_nu}.
The error or these estimators can be bounded as follows: 
\begin{align}
    &\EE_{s_1\sim\rho_0}\bigsbr{ \orbr{T_1^{\pi,\tilde\nu}\tilde U_1}(s_1 )} - J(\pi ) \nend
    %%%%%%
    % &\quad = {\sum_{h=1}^H \EE\sbr{\bigrbr{\tilde U_h - u_h}(s_h, a_h, b_h)- \tilde W_{h+1}(s_{h+1})}} + \sum_{h=1}^H \EE\sbr{\tilde W_h(s_h)-\tilde U_h(s_h, a_h, b_h)}\nend
    %%%%%%
    &\quad\le\underbrace{{\sum_{h=1}^H \EE\Bigsbr{\orbr{\tilde U_h - u_h}(s_h, a_h, b_h)-  \orbr{T_{h+1}^{\pi,\tilde\nu} \tilde U_{h+1}}(s_{h+1})}}}_{\dr \text{Leader's Bellman error}}+ \underbrace{\sum_{h=1}^H   H  \cdot \EE \bigsbr{ \nbr{\rbr{\tilde \nu_h-\nu_h^{\pi }}(\cdot\given s_h)}_1} }_{\dr \text{Quantal response error}}\label{eq:perform-diff-general}, \\
    & \EE_{s_1\sim\rho_0}\bigsbr{\tilde W_1(s_1)}  - J(\pi)\nend
    %%%%%%%%
    &\quad=\underbrace{{\sum_{h=1}^H \EE\bigsbr{\orbr{\tilde U_h - u_h}(s_h, a_h, b_h)- \tilde W_{h+1}(s_{h+1})}}}_{\dr \text{Leader's Bellman error}} + \underbrace{\sum_{h=1}^H \EE\bigsbr{\tilde W_h(s_h) -  \orbr{T_{h}^{\pi,\tilde\nu} \tilde U_{h}} (s_{h})}}_{\ds\text{Value mismatch error}}\nend
    &\qqquad+\underbrace{\sum_{h=1}^H  H  \cdot  \EE  \bigsbr{ \nbr{\rbr{\tilde \nu_h-\nu_h^{\pi }}(\cdot\given s_h)}_1} }_{\dr \text{Quantal response error}}, 
    \label{eq:perform-diff-linear} 
\end{align}
where  the expectation is taken with respect to the randomness of the trajectory generated by $(\pi, \nu^{\pi})$ on the true model $M^*$. 

 

\end{lemma}



Here, we provide two ways to estimate $J(\pi)$.
The first inequality \eqref{eq:perform-diff-general} is useful for  setting with general function approximation, and the second inequality \eqref{eq:perform-diff-linear} becomes handy particularly in the   setting with linear function approximation, where we  additionally introduced an algorithm with penalty/bonus terms. 

This lemma shows that the estimation error of $\EE_{s_1\sim\rho_0}\bigl [ \orbr{T_1^{\pi,\tilde\nu}\tilde U_1} (s_1 )\bigr ] $ can be decomposed into a sum of three terms --- leader's Bellman error, the error of the estimated quantal response, and an additional term that measures the mismatch between $\tilde W$ and $\tilde U$, which is included here for generality and will be used in the analysis of both \Cref{alg:PMLE} and \Cref{alg:MLE-OVI} where some penalties/bonuses are included in the estimation of $\hat W$.

Furthermore,  by the definition of 
$T_h^{\pi,\tilde\nu}$ in \eqref{eq:operator_T_pi_nu},
%  {\main\eqref{eq:relation_tileWU}\fi} {\neurips $
%  \tilde W_h(s ) = \la \tilde U_h (s, \cdot , \cdot ) , \pi _h \otimes \tilde \nu_h^{\pi} (\cdot , \cdot \given s) \ra . 
%  $\fi}  
 $
 \tilde W_h(s ) = \la \tilde U_h (s, \cdot , \cdot ) , \pi _h \otimes \tilde \nu_h^{\pi} (\cdot , \cdot \given s) \ra
 $
 is equivalent to $\tilde W_ h = T_h^{\pi, \tilde \nu} \tilde U_h$. 
Thus, 
\eqref{eq:perform-diff-linear} directly implies \eqref{eq:performance diff-1}
% {\main Lemma \ref{lem:performance diff informal}\fi} {\neurips \eqref{eq:performance diff-1}\fi} 
as a special case.

% \begin{align}
%     &\EE_{s_1\sim\rho_0}\sbr{ T_1^{\pi,\tilde\nu}\tilde U_1(s_1, a_1, b_1)} - J(\pi ) \nend
%     %%%%%%
%     % &\quad = {\sum_{h=1}^H \EE\sbr{\bigrbr{\tilde U_h - u_h}(s_h, a_h, b_h)- \tilde W_{h+1}(s_{h+1})}} + \sum_{h=1}^H \EE\sbr{\tilde W_h(s_h)-\tilde U_h(s_h, a_h, b_h)}\nend
%     %%%%%%
%     &\quad\le\underbrace{{\sum_{h=1}^H \EE\sbr{\bigrbr{\tilde U_h - u_h}(s_h, a_h, b_h)-  T_{h+1}^{\pi,\tilde\nu} \tilde U_{h+1}(s_{h+1})}}}_{\dr \text{Leader's Bellman error}}+ \underbrace{\sum_{h=1}^H  H \EE \nbr{\rbr{\tilde \nu_h-\nu_h}(\cdot\given s_h)}_1}_{\dr \text{Quantal response error}}\label{eq:perform-diff-general}\\
%     %%%%%%%%
%     &\quad=\underbrace{{\sum_{h=1}^H \EE\sbr{\bigrbr{\tilde U_h - u_h}(s_h, a_h, b_h)- \tilde W_{h+1}(s_{h+1})}}}_{\dr \text{Leader's Bellman error}} + \underbrace{\sum_{h=1}^H \EE\sbr{\tilde W_h(s_h) -  T_{h}^{\pi,\tilde\nu} \tilde U_{h}(s_{h})}}_{\ds\text{Value mismatch error}}\nend
%     &\qqquad+\underbrace{\sum_{h=1}^H  H \EE \nbr{\rbr{\tilde \nu_h-\nu_h}(\cdot\given s_h)}_1}_{\dr \text{Quantal response error}}, 
%     \label{eq:perform-diff-linear}
% \end{align}
 




\begin{proof}
By the definitions of $U^{\pi}$ and $W^{\pi}$ in \eqref{eq:U_function} and \eqref{eq:W_function}, $J(\pi)$ can be written as 
\$
J(\pi) = \EE_{s_1 \sim \rho_0 } [ W_1 ^{\pi} (s_1) ] = \EE_{s_1 \sim \rho_0 } [  \orbr{T_1 ^{\pi, \nu}U_1 ^{\pi} } (s_1) ],
\$ 
where we write $\nu = \nu^{\pi}$ to simplify the notation.
Recall that we define the quantal Bellman operator $\TT_h^{\pi}$ in \eqref{eq:bellman_operator_leader}, whose fixed point is $U^{\pi}$.
Then, by direct calculation, we have 
\#\label{eq:subopt-decompose-equality-0}
&\EE_{s_1\sim\rho_0}\bigsbr{ \orbr{T_1^{\pi,\tilde\nu}\tilde U_1}(s_1 )} - J(\pi )    \nend 
& \quad  =    \EE_{s_1\sim\rho_0}\bigsbr{\orbr{T_1^{\pi,\tilde\nu} - T_1^{\pi,\nu}} \tilde U_1(s_1)}  + \EE_{s_1\sim\rho_0}\bigsbr{\orbr{  T_1^{\pi,\nu}  \tilde U_1 - T_1^{\pi,\nu}  U_1^{\pi} } (s_1)}. 
\#
Furthermore, using the Bellman equation of $U^{\pi}$, we have 
\#\label{eq:subopt-decompose-equality-01}
& \EE_{s_1\sim\rho_0}\bigsbr{\orbr{  T_1^{\pi,\nu}  \tilde U_1 - T_1^{\pi,\nu}  U_1^{\pi} } (s_1)} 
= \EE \bigsbr{  \tilde U_1(s_1, s_1, a_1) - U_1 ^{\pi} (s_1, a_1, b_1)}\nend 
& \quad  = \EE \bigsbr{  \orbr{\tilde U_1 -u_1} (s_1, s_1, a_1) - T_{2}^{\pi,\nu} \tilde U_{2}(s_2 ) }  +  \EE\bigsbr{ T_{2}^{\pi,\nu} \tilde U_{2} (s_2) - T_{2}^{\pi,\nu}U_{2}^{\pi}  (s_2) }. 
\# 
Here the second equality follows from the Bellman equation, and the expectation is taken with respect to the randomness of the trajectory generated by $(\pi, \nu^{\pi})$ on the true model $M^*$. . 
Furthermore, by replacing $T_{2}^{\pi,\nu} \tilde U_{2}$ by $T_{2}^{\pi, \tilde \nu} \tilde U_{2}$ in \eqref{eq:subopt-decompose-equality-01}, and combining \eqref{eq:subopt-decompose-equality-0}, we obtain that 
\# 
&\EE_{s_1\sim\rho_0}\bigsbr{ \orbr{T_1^{\pi,\tilde\nu}\tilde U_1}(s_1 )} - J(\pi )    \nend 
     & \quad 
     =  {\sum_{h=1}^H \EE\bigsbr{\bigrbr{\tilde U_h - u_h}(s_h, a_h, b_h)-  \orbr{T_{h+1}^{\pi,\tilde\nu} \tilde U_{h+1}}(s_{h+1})}} + \sum_{h=1}^H \EE\bigsbr{\orbr{ T_h^{\pi,\tilde\nu}-  T_h^{\pi,\nu}} \tilde U_h(s_h)} ,\label{eq:subopt-decompose-equality-1}
\# 
where we apply recursion over all $h\in [H]$. 
Finally, note that $\tilde U_h $ is bounded by $H$ in the $\ell_{\infty}$-norm. Using H\"older's inequality, we have 
\#
\EE\bigsbr{\orbr{ T_h^{\pi,\tilde\nu}-  T_h^{\pi,\nu}} \tilde U_h(s_h)} \leq H \cdot \EE \bigsbr{ \nbr{\rbr{\tilde \nu_h-\nu_h}(\cdot\given s_h)}_1}.\label{eq:subopt-decompose-equality-12}
\#
Combining \eqref{eq:subopt-decompose-equality-1} and \eqref{eq:subopt-decompose-equality-12}, we establish \eqref{eq:perform-diff-general}.

 
It remains to prove  \eqref{eq:perform-diff-linear}. 
To this end, it suffices to incorporate the value mismatch error in \eqref{eq:perform-diff-linear} into \eqref{eq:perform-diff-general}.
Specifically,  for any $h \in [H]$, we have 
\#\label{eq:subopt-decompose-equality-2}
& \EE\bigsbr{\bigrbr{\tilde U_h - u_h}(s_h, a_h, b_h)-  T_{h+1}^{\pi,\tilde\nu} \tilde U_{h+1}(s_{h+1})} \notag \\
& \quad    ={\sum_{h=1}^H \EE\bigsbr{\bigrbr{\tilde U_h - u_h}(s_h, a_h, b_h)- \tilde W_{h+1}(s_{h+1})}} + \sum_{h=2}^H \EE\bigsbr{\tilde W_h(s_h) - T_h^{\pi,\tilde\nu}\tilde U_h(s_h)}. 
\#
Combining \eqref{eq:perform-diff-general} and \eqref{eq:subopt-decompose-equality-2}
we have 
\$
& \EE_{s_1\sim\rho_0}\bigsbr{ \tilde W_1(s_1)} - J(\pi) \notag \\
& \quad = \EE\bigsbr{\tilde W_1(s_1) - \orbr{T_1^{\pi,\tilde\nu}\tilde U_1}(s_1)} +  \EE\bigsbr{ \orbr{T_1^{\pi,\tilde\nu}\tilde U_1} (s_1)} - J(\pi) 
\notag \\
&  \quad\le {\sum_{h=1}^H \EE\bigsbr{\bigrbr{\tilde U_h - u_h}(s_h, a_h, b_h)- \tilde W_{h+1}(s_{h+1})}} + \sum_{h=1}^H \EE\bigsbr{\tilde W_h(s_h) - \orbr{T_h^{\pi,\tilde\nu}\tilde U_h }(s_h)}\nend
&\qqquad+ \sum_{h=1}^H  H \cdot  \EE \bigsbr{ \nbr{\rbr{\tilde \nu_h-\nu_h}(\cdot\given s_h)}_1} ,
\$ 
which gives us \eqref{eq:perform-diff-linear}. 
    Hence, we complete the proof.
% where $H$ upper bound $\tilde U$. The first equality holds by a switch between $T_1^{\pi, \tilde\nu}$ and $T_1^{\pi, \nu}$, the second equality holds by doing this switching recursively and noting that $\nu_{H+1}=\tilde\nu_{H+1}$ since it is out of the horizon, and the last inequality holds by the difinition of $\nbr{\cdot}_1$-norm. Hence, we obtain \eqref{eq:perform-diff-general}. If we include $\tilde W$, we just decompose all the $T_{h+1}^{\pi,\tilde\nu}\tilde U_{h+1}(s_{h+1})$ terms for $h=1,\dots, H$ on the righthand side of \eqref{eq:perform-diff-general} and also the $T_1^{\pi,\tilde\nu}\tilde U_1(s_1, a_1, b_1)$ term on the left hand side and obtain 
%     \begin{align}
%         &\EE_{s_1\sim\rho_0}\sbr{ \tilde W_1(s_1)} - J(\pi) \nend
%     %%%%%%
%     &\quad =  {\sum_{h=1}^H \EE\sbr{\bigrbr{\tilde U_h - u_h}(s_h, a_h, b_h)-  T_{h+1}^{\pi,\tilde\nu} \tilde U_{h+1}(s_{h+1})}} + \sum_{h=1}^H \EE\sbr{\rbr{ T_h^{\pi,\tilde\nu}-  T_h^{\pi,\nu}} \tilde U_h(s_h)}\nend
%     &\qqquad + \EE\sbr{\tilde W_1(s_1) - T_1^{\pi,\tilde\nu}\tilde U_1(s_1)}\nend
%     %%%%%%%%
%     &\quad={\sum_{h=1}^H \EE\sbr{\bigrbr{\tilde U_h - u_h}(s_h, a_h, b_h)- \tilde W_{h+1}(s_{h+1})}} + \sum_{h=1}^H \EE\sbr{\tilde W_h(s_h) - T_h^{\pi,\tilde\nu}\tilde U_h(s_h)}\nend
%     &\qqquad+  \sum_{h=1}^H \EE\sbr{\rbr{ T_h^{\pi,\tilde\nu}-  T_h^{\pi,\nu}} \tilde U_h(s_h)}\label{eq:subopt-decompose-equality-2}\\
%     %%%%%%%%%%
%     &\quad\le {\sum_{h=1}^H \EE\sbr{\bigrbr{\tilde U_h - u_h}(s_h, a_h, b_h)- \tilde W_{h+1}(s_{h+1})}} + \sum_{h=1}^H \EE\sbr{\tilde W_h(s_h) - T_h^{\pi,\tilde\nu}\tilde U_h(s_h)}\nend
%     &\qqquad+ \sum_{h=1}^H  H \EE \nbr{\rbr{\tilde \nu_h-\nu_h}(\cdot\given s_h)}_1, \notag
%     \end{align}
%     which gives us \eqref{eq:perform-diff-linear}. 
%     Hence, we complete the proof.
\end{proof}



Recall that we estimate the quantal response mapping via   model-based maximum likelihood  estimation. 
In particular, we approximate the true quantal response policy $\nu^{\pi}$ within class $\{ \nu^{\pi, \theta}\}_{\theta \in \Theta}$, where 
$\nu^{\pi, \theta}$ can be written as 
\$
\nu_h^{\pi, \theta} (b_h\given s_h) &= \exp\bigl ( \eta \cdot A_h^{\pi, \theta} (s_h, b_h) \bigr), \quad\text{where}\quad A_h^{\pi, \theta} (s_h, b_h) = Q_h^{\pi, \theta} (s_h, b_h) - V_{   h}^{\pi, \theta} (s_h) ,
\$
Here $A^{\pi, \theta}$, $Q^{\pi, \theta}$, and $V^{\pi, \theta}$ are the advantage function, and value functions corresponding to policy $\pi$, under the model $\{ r^{\theta} , P^{\theta}\}$. 
In the following, we present a lemma which relates the error of quantal response mapping in \eqref{eq:perform-diff-general} and \eqref{eq:perform-diff-linear} to  estimation errors of the follower's value functions.
To simplify the notation, 
we consider a fixed policy $\pi$, and define  
$r_h^\pi(s_h,b_h) =\inp{r_h(s_h, \cdot, b_h)}{\pi(\cdot\given s_h, b_h)}_{\cA}$ and $P_h^\pi(s_{h+1}\given s_h,b_h) =\inp{P_h(s_{h+1}\given s_h, \cdot, b_h)}{\pi(\cdot\given s_h, b_h)}_{\cA} $.
To simplify the notation, 
in the rest of subsection, 
we let $\EE=\EE^{\pi,M^*}$ and $\EE_{z} [\cdot]=\EE^{\pi,M^*}[\cdot\given z]$ for any variable $z$.
 

%Our first group of techniques concern the suboptimality decomposition of the QSE with strategic follower. 
%   , and denote by $(\tilde U, \tilde W, \tilde Q, \tilde V, \tilde A)$ an alternative satisfying $\tilde V_h = \eta^{-1}\log \int \exp(\eta \tilde Q_h)$, $\tilde A_h = \tilde Q_h -\tilde V_h$. 
% Moreover, we let $\tilde \nu_h = \exp(\eta \tilde A_h)$ as the response under $\tilde A$. In the sequel, we let $\EE=\EE^{\pi,M^*}$ and $\EE_{z} [\cdot]=\EE^{\pi,M^*}[\cdot\given z]$ if without special reminder. We ignore the superscript $\pi, M^*$ for simplicity. We also let $r_h^\pi(s_h,b_h) =\inp{r_h(s_h, \cdot, b_h)}{\pi(\cdot\given s_h, b_h)}$ and $P_h^\pi(s_{h+1}\given s_h,b_h) =\inp{P_h(s_{h+1}\given s_h, \cdot, b_h)}{\pi(\cdot\given s_h, b_h)}$.  
% Based on \eqref{eq:performance diff-1},






% We consider a fixed policy $\pi$,  denote by $(U, W, Q, V, A, \nu)$ the ground truth under $\pi$ and the true model $M^*$, and denote by $(\tilde U, \tilde W, \tilde Q, \tilde V, \tilde A, \tilde \nu)$ an alternative satisfying $\tilde V_h = \eta^{-1}\log \int \exp(\eta \tilde Q_h)$, $\tilde A_h = \tilde Q_h -\tilde V_h$, and $\tilde \nu_h = \exp(\eta \tilde A_h)$. We then have for the U function that



% We next characterize the quantal response error in a Taylor expansion flavor.
\begin{lemma}[Response Model Error]
    \label{lem:performance diff}
    We consider a fixed policy $\pi$
and   let $\tilde Q$    be an estimate  of $Q^{\pi}$. 
We define a V-function  $\tilde V$ and an advantage function $\tilde A$ by letting 
\#\label{eq:tilde_functions}
\tilde V_h (s) = \frac{1}{\eta} \cdot \log \bigg(  \sum_{b \in \cB} \exp( \eta \cdot \tilde Q_h(s, b) ) \biggr), \qquad \textrm{and}\qquad \tilde A_h(s,a) = \tilde Q_h (s,b) - \tilde V_h (s). 
\# 
Furthermore, we define a follower's policy $\tilde \nu$ by letting $\tilde \nu_h(b \given s) = \exp( \eta\cdot \tilde A_h(s,b))$. 
Then the difference between $\tilde \nu$ and $\nu^{\pi}$ can be bounded by 
\begin{align}
    &\sum_{h=1}^H  H \cdot  \EE \bigsbr{\nbr{\rbr{\tilde \nu_h-\nu_h^{\pi} }(\cdot\given s_h)}_1} \nend
    &\quad\le C^{(0)} \cdot \sum_{h=1}^H \underbrace{\EE\bigsbr{\bigabr{\tilde \Delta^{(1)}_h(s_h, b_h)}}}_{\ds\text{1st-order error}}  + C^{(1)} \cdot 
    \sum_{h=1}^H \underbrace{\EE\bigsbr{ \bigabr{(\tilde A_h  - A_h^{\pi})(s_h, b_h)}^2}}_{\ds\text{2nd-order error}} \label{eq:taylor-myopic}\\
    &\quad\le 
    C^{(0)} \cdot 
    \sum_{h=1}^H \underbrace{\EE\bigsbr{\bigabr{\tilde \Delta^{(1)}_h(s_h, b_h)}}}_{\ds\text{1st-order error}}  + C^{(2)} \cdot 
    \max_{h\in [H]} \underbrace{\EE\bigsbr{ \bigabr{\orbr{\tilde Q_h - r_h^\pi - \gamma P_h^\pi \tilde V_{h+1}}(s_h, b_h)}^2}}_{\ds\text{2nd-order error}},\label{eq:taylor-farsighted} 
\end{align}
where $\tilde \Delta^{(1)}_h(s_h, b_h)$ is defined as
\begin{align*}
    \tilde \Delta^{(1)}_h(s_h, b_h) &=  \rbr{\EE_{s_h, b_h} -\EE_{s_h}}\Biggsbr{\sum_{l=h}^H \gamma^{l-h}\underbrace{\rbr{\tilde Q_l - r_l^\pi - \gamma P_l^\pi \tilde V_{l+1}}(s_l, b_l)}_{\ds\text{Follower's Bellman error}}}. 
    % \label{eq:def Delta^1}
\end{align*}
Furthermore, the constants $C^{(0)}$, $C^{(1)}$, and $C^{(2)}$ are given by
\#\label{eq:define_constants}
\begin{split}
    C^{(0)}&=2\eta H , \qquad   C^{(1)}=\eta^2 H   \bigrbr{1+ 4 \eff_H(\gamma)}\cdot \exp(2\eta B_A), \\
C^{(2)}  & =  2  \eta^2 H^2\cdot \exp\orbr{6\eta B_A} \cdot(1+4 \eff_H(\gamma)) \cdot \rbr{\eff_H(\exp(2\eta B_A)\gamma)}^2,
\end{split}
\#
where $B_A$ defined in \eqref{eq:define_BA} is an upper bound on the magnitude of the advantage function, and we define $\eff_H(x) = (1-x^H)/(1-x)$ as the \say{effective}  horizon with respect to $x$.
% \begin{proof}
%     See \Cref{lem:performance diff} for a detailed proof.
% \end{proof}



    % denote by $(U, W, Q, V, A)$ the ground truth under $\pi$, and denote by $(\tilde U, \tilde W, \tilde Q, \tilde V, \tilde A)$ an alternative satisfying $\tilde V_h = \eta^{-1}\log \int \exp(\eta \tilde Q_h)$, $\tilde A_h = \tilde Q_h -\tilde V_h$. Suppose that $\tilde\nu_h=\exp\orbr{\eta \tilde A_h}$ and $\nu$ is the real quantal response under policy $\pi$. The behavior model error in \eqref{eq:perform-diff-linear} can be upper bounded by
\end{lemma}
\begin{proof}
    See \Cref{sec:proof-performance diff} for a detailed proof.
\end{proof}

If we view the quantal response as a functional of the advantage function (as shown in \eqref{eq:quantal_response_policy}),  this lemma plays the role of Taylor expansion of the quantal response into the first- and second-order errors. 
In particular, the first-order error corresponds to the Bellman error of the follower's problem, and the second-order term is mean-squared error of the advantage function (as in \eqref{eq:taylor-myopic}) or the sum of squares of the Bellman error (as in \eqref{eq:taylor-farsighted}). 
Here we establish two versions of upper bounds because \eqref{eq:taylor-myopic} is handy for the myopic case while the second \eqref{eq:taylor-farsighted} is more useful for the farsighted case.
 
In the case with a  myopic follower,
the $Q$-function is reduced to the reward function $r$ of the follower. 
We have the following corollary. 
\begin{corollary}[Response Model Error for Myopic Case]\label{cor:response-diff-myopic}
    Let $r$ be the true reward function and let $\tilde r$ be an estimated reward. 
Let $\pi \colon \cS \times \cB \rightarrow \Delta(\cA )$ be a fixed policy. 
Let $\nu$ and $\tilde \nu$ be the quantal response function based on $r$ and $\tilde r$, respectively, i.e.,
\$
\nu(b\given s) \propto \exp \big (  \eta\cdot  r^{\pi} (s,b)\big ) ,\qquad \nu(b\given s) \propto \exp\big ( \eta \cdot \tilde r^{\pi} (s,b)\big ).
\$
Here we define $r^{\pi} $ by letting $r^{\pi} (s,b) = \la r(s,\cdot, b), \pi(\cdot \given s, b)\ra$, and define $\tilde r^{\pi}$ similarly. 
Then for any state $s\in \cS$, we have 
    \begin{align}
        D_\TV\rbr{\nu(\cdot\given s), \tilde\nu(\cdot\given s)}
        &\le  \eta  \EE_s\sbr{\abr{(\tilde r^\pi(s, b) - r^\pi(s, b)) - \EE_s\bigsbr{\tilde r^\pi(s, b) - r^\pi(s, b)}}} \nend
        &\qquad + C^{(3)}\EE_s\bigsbr{\rbr{\rbr{\tilde r^\pi(s, b) -r^\pi(s, b)} - \EE_s\sbr{\tilde r^\pi(s, b) -r^\pi(s, b)}}^2}, \label{eq:TV-for-myopic}  
        % \notag 
    \end{align}
    where $C^{(3)}={\eta^2 \exp(2\eta B_A)}\bigrbr{2+\eta B_A \cdot  \exp\rbr{2\eta B_A}}/2$ and $B_A$  defined in \eqref{eq:define_BA} is $2 + 2 \log |\cB| / \eta  $. 
    Here the expectation $\EE_s$ is only taken with respect to $b \sim \nu(\cdot \given s)$ when conditioned on this $s$.
    \begin{proof}
        See \Cref{sec:proof-response-diff-myopic} for a detailed proof.
    \end{proof}
\end{corollary}
In the following, we present the result for a special case where 
the follower is myopic with a linear reward function. 
In this case, we write 
  $Q^{\pi, \theta}(s, b) = r^{\pi, \theta}(s, b) = \inp[]{\phi^\pi(s, b)}{\theta}$ for some $\RR^d$ kernel $\phi:\cS\times\cA\times\cB\rightarrow\RR^d$ with $\phi^\pi(s, b)=\inp{\phi(s, \cdot, b)}{\pi(\cdot\given s, b)}_\cA$ and parameter $\theta\in\RR^d$.
  The quantal response policy is given by $\nu^{\pi, \theta} (b \given s) \propto \exp(\eta \cdot \la \phi^{\pi}(s,b), \theta \ra)$. 
In particular, we show that the 1st- and the 2nd-order QRE decomposition in \eqref{eq:QRE-decompose} of \Cref{sec:learning QSE} is just a direct result of \Cref{cor:response-diff-myopic}. Recall by definition of the $\QRE$ in \eqref{eq:QRE}, 
\begin{align*}
    \QRE(s_h, b_h;\tilde\theta,\pi) &=  (\Upsilon_h^{\pi}(\tilde r_h - r_h))\orbr{s_h, b_h}\nend
    &= \dotp{\pi_h(\cdot\given s_h, b_h)}{(\tilde r_h - r_h)(s_h, \cdot, b_h)} - \dotp{\pi_h\otimes \nu_h^{\pi}(\cdot,\cdot\given s_h)}{(\tilde r_h - r_h)(s_h,\cdot,\cdot)}\nend
    & = \sbr{{(\tilde r_h^\pi(s_h, b_h) - r_h^\pi(s_h, b_h)) - \EE_{s_h}\bigsbr{\tilde r_h^\pi(s_h, b_h) - r_h^\pi(s_h, b_h)}}}. 
\end{align*}
We plug in the linear representation of the follower's reward $r_h^\pi(s_h, b_h) = \la\phi_h^\pi(s_h, b_h), \theta_h^*\ra$, which implies that 
\begin{align*}
    \EE_{s_h}\QRE(s_h, b_h;\tilde\theta, \pi)^2 = \Cov_{s_h}^{\pi,\theta^*} \sbr{(\tilde r_h^\pi - r_h^\pi)(s_h, b_h)} = \onbr{\tilde \theta_h -\theta_h}_{\Sigma_{s_h}^{\pi,\theta^*}}^2,
\end{align*}
where $\Sigma_{s_h}^{\pi, \theta^*} = \Cov_{s_h}^{\pi, \theta^*}[\phi_h^\pi(s_h, b_h)]$. Moreover, For the first term on the right hand side of \eqref{eq:TV-for-myopic}, 
\begin{align*}
    &\EE_{s_h}\sbr{\abr{(\tilde r^\pi(s_h, b_h) - r^\pi(s_h, b_h)) - \EE_s\bigsbr{\tilde r^\pi(s_h, b_h) - r^\pi(s_h, b_h)}}} \nend
    &\quad \le \sqrt{\EE_{s_h}\bigsbr{\rbr{\rbr{\tilde r^\pi(s_h, b_h) -r^\pi(s_h, b_h)} - \EE_{s_h}\sbr{\tilde r^\pi(s_h, b_h) -r^\pi(s_h, b_h)}}^2}} \nend
    &\quad = \EE_{s_h}\QRE(s_h, b_h;\tilde\theta, \pi)^2,
\end{align*}
where the inequality holds by the Cauchy-Schwarz inequality. The second term follows similarly.
We further generalize the above argument to the following corollary. 

\begin{corollary}[Response Model Error for Linear and  Myopic Case]\label{lem:response diff-myopic-linear}
    Under the setting of \Cref{cor:response-diff-myopic}, we assume the reward function of the myopic follower is a linear function of $\phi$. 
    Let $\theta^*$ be the parameter of the true reward function and let 
    $\tilde \theta \in \Theta$ be another parameter. 
    Let $\pi \in \Pi$ be any policy and let $s \in \cS$ be any state. 
   We define a matrix  $\Sigma_s^{\pi,\theta}$  as $ \Cov_s^{\pi,\theta}[\phi^\pi(s, b)], $ i.e.,
   \#\label{eq:define_sigma_s}
   \Sigma_s^{\pi,\theta} = \EE_s \bigl [ (\phi^\pi(s, b) - \EE_s [\phi^\pi(s, b)] \bigr )(\phi^\pi(s, b) - \EE_s [\phi^\pi(s, b)] \bigr )^\top \bigr ],
   \#
   where the expectation is taken with respect to $\nu^{\pi,\theta}(\cdot \given s)$.
    Then we have 
    \begin{align}
        &D_\TV\rbr{\nu^{\pi,\theta^*}(\cdot\given s) , \nu^{\pi,\tilde\theta}(\cdot\given s)} \nend
        &\quad \le \min\Bigcbr{f\Bigrbr{\sqrt{\trace\bigrbr{{\Psi}^{\dagger} \Sigma_{ s}^{\pi, \tilde\theta}}}\cdot \bignbr{\theta^*- \tilde\theta}_{{\Psi}}}, f\Bigrbr{\sqrt{\trace\bigrbr{{\Psi}^{\dagger} \Sigma_{ s}^{\pi, \theta^*}}}\cdot \bignbr{\theta^*- \tilde\theta}_{{\Psi}}}},
        % \label{eq:TV-for-myopic-linear}
    \end{align}
    where $\Psi\in \SSS_+^{d}$ can be any fixed nonnegative definite matrix, $\Psi^{\dagger}$ is the pseudo-inverse of $\Psi$,  and the univariate function $f$ is defined as
    $
        f(x) = \eta x + C^{(3)} x^2 
    $ with 
    $$C^{(3)}={\eta^2 \exp(2\eta B_A)}\bigrbr{2+\eta B_A \cdot  \exp\rbr{2\eta B_A}}/2.$$
\end{corollary}

\begin{proof}
    See \Cref{sec:proof-response-diff-myopic_linear} for a detailed proof. 
\end{proof}
    % $\Gamma^{(2)}_h$ is defined as
    % \begin{align*}
    %     \Gamma^{(2)}_h(s_h;\alpha,\theta_h) = \sqrt{\EE_{s_h}^{{\alpha,\theta_h}}\sbr{\phi_h^{\alpha}(s_h,b_h)^\top {\Psi}^{\dagger}\phi_h^{\alpha}(s_h,b_h)} -\bignbr{\EE_{s_h}^{{\alpha,\theta_h}}\phi_h^{\alpha}(s_h,b_h)}_{{\Psi}^{\dagger}}^2}\cdot \bignbr{\theta_h^*- \theta_h}_{{\Psi}}, 
    % \end{align*}
    % and $\Gamma^{(3)}$ is defined as 
    % \begin{align*}
    %     \Gamma^{(3)}_h(s_h;\alpha,\theta_h) = \sqrt{\EE_{s_h}^{{\alpha,\theta_h^*}}\sbr{\phi_h^{\alpha}(s_h,b_h)^\top {\Psi}^{\dagger}\phi_h^{\alpha}(s_h,b_h)} -\bignbr{\EE_{s_h}^{{\alpha,\theta_h^*}}\phi_h^{\alpha}(s_h,b_h)}_{{\Psi}^{\dagger}}^2}\cdot \bignbr{\theta_h^*- \theta_h}_{{\Psi}}
    % \end{align*}
    
We first show that the uncertainty quantifier $\Gamma^{(2)}_h(s_h;\pi, \theta_h)$ used in \eqref{eq:Gamma^2} of \S\ref{sec:offline-ML} {\neurips and \eqref{eq:Gamma^2-neurips}} can be derived directly from \Cref{lem:response diff-myopic-linear}. 
Recall by definition, 
\begin{align*}
    \Gamma^{(2)}_h(s_h;\pi_h , \theta_h) = 2 H(\eta  \xi(s_h;\pi,\theta_h)  + C^{(3)}  \xi(s_h;\pi,\theta_h)^2 ), 
\end{align*}
where $\xi(s_h;\pi,\theta_h)^2$ is defined as
\begin{equation*}
    \xi(s_h;\pi, \theta_h)^ 2 =  \trace\bigrbr{\bigrbr{T\Sigma_{h,\cD}^{\theta} + I_d}^\dagger \Sigma_{s_h}^{\pi , \theta}}  \cdot  \bigl( 2 (\eta^{-1}+B_A)^2 \beta + 4B_\Theta^2 \bigr).
\end{equation*}
One can easily check that for any $\theta_h\in\CI_{h,\Theta}(\beta)$ where $\CI_{h,\Theta}(\beta)$ is the offline confidence set, 
\begin{align*}
    \Gamma^{(2)}_h(s_h;\pi_h,\theta_h) 
    &= 2 H \cdot f\Bigrbr{\sqrt{\trace\bigrbr{\bigrbr{T\Sigma_{h,\cD}^{\theta}+I_d}^\dagger \Sigma_{s_h}^{\pi,\theta}}} (2C_\eta^2 \beta + 4 B_\Theta^2)}\nend
    &\ge 2 H \cdot f\Bigrbr{\sqrt{\trace\bigrbr{\bigrbr{T\Sigma_{h,\cD}^{\theta}+I_d}^\dagger \Sigma_{s_h}^{\pi,\theta}}} (T\nbr{\theta_h -\theta_h^*}_{\Sigma_{h,\cD}^{\theta}}^2 + \nbr{\theta_h -\theta_h^*}_{I_d}^2)}\nend
    & \ge 2 H D_\TV\bigrbr{\nu_h^{\pi,\theta^*}(\cdot\given s_h), \nu_h^{\pi,\theta}(\cdot\given s_h)}, 
\end{align*}
where we use definition $C_\eta^2 = \eta^{-1}+B_A$ in the first equality, and in the first inequality, $2C_\eta^2\beta T^{-1} \ge \nbr{\theta_h-\theta_h^*}_{\Sigma_{h,\cD}^\theta}^2$ holds by \Cref{cor:formal-MLE confset-linear myopic} if $\CI_{h,\Theta}(\beta)$ is a valid confidence set, and $4B_\Theta^2\ge \nbr{\theta_h-\theta_h^*}_{I_d}^2$ holds by noting that $\nbr{\theta_h}^2\le B_\Theta$ for any $\theta_h\in\Theta_h$. 
The last inequality holds just by \Cref{lem:response diff-myopic-linear} where we plug in $\Psi = T\Sigma_{h,\cD}^{\theta}+I_d$.
% Note that $\bignbr{\theta^*-\tilde\theta}_{\Sigma_{s_h}^{\pi,\theta^*}}$



\subsection{Learning Quantal Response via MLE}\label{sec:app-MLE}
In \S\ref{sec:MLE for behavior model}, we introduce how to learn the follower's  quantal response model from the follower's feedbacks via maximum likelihood estimation. 
In the following, we provide a slightly stronger lemma that implies  \Cref{sec:proof-MLE-general} in \S\ref{sec:MLE for behavior model}. 
In the following, we follow the notation used in \S\ref{sec:MLE for behavior model}, where 
we let $\theta = \{ \theta_h \}_{h \in [H]} \in \Theta $ denote the parameters of the follower's response model,
where $\Theta = \Theta_1 \times \ldots \times \Theta_H$ is the set of all parameters. 
Here we assume $\theta_h \in \Theta_h$ for all $h\in [H]$.
In specific, each $\theta$ is associated with a model, denoted by  ${ r^{\theta}, P^{\theta}}$, 
which is   shorthand notation for 
\$
\{ r_1^{\theta_1}, P_1^{\theta_1},  \ldots, r_H ^{\theta_H}, P_H^{\theta_H}\}. 
\$ 
Moreover, when the follower is myopic, $\theta$ only parameterize a reward function $r^{\theta}$. 
We assume that the parametric model of the follower captures the true model. That is, there exists $\theta^* \in \Theta $ such that 
$\{r^{\theta}, P^{\theta^*}\}$ is the true environment. 


%we assume that  
%$\Theta$ is the set of the  parameters that determines the follower's response model. 


For any $\theta \in \Theta$ and any policy $\pi$ of the leader, we let $\nu^{\pi, \theta}$, $A^{\pi, \theta}$, $Q^{\pi, \theta } $ and $V^{\pi, \theta }$ denote the quantal response of $\pi$, advantage function, and Q- and V-functions under model $\{ r^{\theta}, P^{\theta} \}$, which are defined according to  \eqref{eq:quantal_response_policy}--{\main\eqref{eq:v_pi_qr}\fi}{\neurips \eqref{eq:qv_pi_qr}\fi}.
 The quantal response   under the true model is $  \nu^{\pi, \theta^*}$.
Thus, given a  (possibly adaptive) dataset $\cD = \{(s_h ^i, a_h ^i, b_h ^i, \pi_h ^i)\}_{i\in[t-1], h\in [H]}$,
the negative log-likelihood at step $h$ is given by
\#\label{eq:loglikelihood-1}
\cL_h^t  (\theta  )  & = -  \sum_{i = 1}^{t-1}  
\log \nu^{\pi^i, \theta} (b_h^i \given s_h^i) 
  = -  \sum_{i=1}^{t-1}   \eta  \cdot A_h^{\pi^i , \theta}(s^i,  b^i), 
\# 
 where   $\nu^{\pi^i, \theta} (b_h^i \given s_h^i) $ is the probability of observing the follower's action $b_h^i$ when the model parameter is $\theta$, state is $s_h^i$, and the leader announces $\pi^i$, and the second equality in \eqref{eq:loglikelihood-1} is due to \eqref{eq:quantal_response_policy}.   
 Here we assume the data $\cD$ satisfies the compliance property, i.e., $\PP^{\pi^t}_\cD(b_h^t\given s_h^t, (s_j^t, a_j^t, b_j^t, u_j^t)_{j\in[h-1]}, \tau^{1:t-1})  = \nu_h^{\pi^t}(b_h^t\given s_h^t), \forall h\in[H], t\in[T]$. 
 Such a property is satisfied 
by the online setting and the offline setting where behavior policies are adaptive. 
 Therefore, the MLE estimator of $\theta^*$ can be obtained by minimizing $\cL_h^t(\theta)$ over $\Theta$. 
Moreover, based on $\cL_h$, we can construct a confidence set for $\theta^*$: 
\begin{align}
    \confset_{h,\Theta}^t(\beta)=\cbr{\theta\in\Theta: \cL_h^t(\theta)\le \inf_{\theta'\in\Theta}\cL_h^t(\theta') + \beta}, \label{eq:behavior_model_confset-1}
\end{align}
where  $\cL_h^t(\theta) = \sum_{i=1}^{t-1} \eta A_h^{\pi^i, \theta}(s_h^i, b_h^i)$.


We remark that $\cL_h^t(\theta)$ \eqref{eq:loglikelihood-1} is a function of $\theta = \{ \theta_h \}_{h\in [H]}$. 
The reason is that the follower's quantal response is obtained by solving the optimal policy of an entropy-regularized MDP, whose optimal policy depends on the reward and transitions across all $H$ steps. 
But if the follower is myopic, then $\cL_h^t(\theta)$ only depends on $\theta_h$. In this case, we can regard $\cL_h^t$ as a function on $\Theta_h$, and replace $\Theta$ in \eqref{eq:behavior_model_confset-1} by $\Theta_h$.  We will leverage such observation in \Cref{sec:learning QSE}.


\begin{lemma}[Confidence Set]\label{lem:MLE-formal}
We define a distance $\rho  $ on $\Theta$ by letting 
\begin{align}\label{eq:rho for MLE}
 \rho(\theta, \tilde \theta) \defeq \max_{\pi\in\Pi, s_h\in\cS, h\in[H]} \cbr{D_\H\orbr{\nu_h^{\pi, \theta}(\cdot\given s_h), \nu_h^{\pi, \tilde \theta}(\cdot\given s_h)}, (1+\eta) \cdot\bignbr{Q_h^{\pi, \theta} - Q_h^{\pi, \tilde\theta}}_\infty}, 
\end{align}
where $B_A$ upper bounds the follower's A-, Q-, and V-functions and is specified in \eqref{eq:define_BA}. 
Let $\cN_\rho(\Theta,\epsilon)$ be the $\epsilon$-covering number of $\Theta$ with respect to the distance $\rho$.
That is, $\cN_\rho(\Theta,\epsilon)$ is the smallest $N \geq 1$ with the following property: there exists $\{ \theta^i\}_{i\in [N]} \subseteq \Theta$ such that, for any $\theta \in \Theta$, there exists $\theta^i$ such that $\rho(\theta, \theta^i) \leq \epsilon$.
For any $\delta \in (0,1)$, we set $\beta \ge  2\log(e^3 H\cdot \cN_\rho(\Theta, T^{-1})/\delta)$.  
Then  with probability at least $1-\delta$,   the following properties hold for $\confset_{h, \Theta}^t (\beta)$ defined in \eqref{eq:behavior_model_confset-1}:
    \begin{itemize}
    \item [(i)]  (Validity) $\theta^*\in\confset_{h, \Theta}^t(\beta)$; 
    \item [(ii)] (Accuracy) For any $\theta\in\Theta, t\in[T], h\in[H]$, it holds that 
    \#
        &
        \sum_{i=1}^{t-1} D_\H^2\bigrbr{\nu_h^{\pi^i, \theta}(\cdot\given s_h^i), \nu_h^{\pi^i, \theta^*}(\cdot\given s_h^i)} \le  \frac {1}{2}\rbr{\cL_h^t(\theta) - \cL_h^t(\theta^*)+ \beta}, \label{eq:MLE-guarantee-hellinger-1}\\
        &
        \sum_{i=1}^{t-1} \EE^{\pi^i }D_\H^2\bigrbr{\nu_h^{\pi^i, \theta}, \nu_h^{\pi^i, \theta^*}} \le  \frac {1}{2}\rbr{\cL_h^t(\theta) - \cL_h^t(\theta^*)+ \beta}\label{eq:MLE-guarantee-hellinger-2}.
 \#
 \end{itemize} 
 Furthermore, the above two inequalities ensure respectively that for 
   $\forall \theta'\in\bigcbr{\theta^*, \theta}, \theta\in\Theta, \forall h\in[H]$,
    \begin{align}
        &\sum_{i=1}^{t-1} {\Var_{s_h^i}^{\pi^i, \theta'} \bigsbr{Q_h^{\pi^i, \theta}(s_h, b_h) - Q_h^{\pi^i, \theta^*}(s_h, b_h)}} \le 4 C_\eta^2 \rbr{\cL_h^t(\theta) - \cL_h^t(\theta^*)+ \beta}, \label{eq:MLE_guarantee_Q-1} \\
        & \sum_{i=1}^{t-1} \EE^{\pi^i }{\Var_{s_h}^{\pi^i, \theta'} \bigsbr{Q_h^{\pi^i, \theta}(s_h, b_h) - Q_h^{\pi^i, \theta^*}(s_h, b_h)}} \le  4 C_\eta^2 \rbr{\cL_h^t(\theta) - \cL_h^t(\theta^*)+ \beta},\label{eq:MLE_guarantee_Q}
    \end{align}
    where $\Var_s^{\pi, \theta}[Z] = \Var^{\pi, \theta}[Z\given s] = \EE^{\pi,\theta}[(Z - \EE^{\pi, \theta}[Z\given s])^2\given s]$, $C_\eta =\eta^{-1}+B_A$. Moreover, for any $\theta\in\CI_\Theta(\beta), h\in[H]$, and a given  $t\in[T]$, we have with probability at least $1-\delta$ that
\begin{align}
&\sum_{i=1}^{t-1} 
    \rbr{\rbr{Q_h^{\pi^i, \theta} - Q_h^{\pi^i, \theta^*}}(s_h^i, b_h^i)
    -\EE_{s_h^i}^{\pi^i,\theta^*}\sbr{ \bigrbr{Q_h^{\pi^i, \theta} - Q_h^{\pi^i, \theta^*}}(s_h, b_h)}}^2 
    \le  \cO(C_\eta^2 \beta) .  \label{eq:MLE-guarantee-Q-3}
\end{align}
 
% and $\cN_\varrho (\Theta, \epsilon)$ is the covering number of the smallest $\epsilon$-covering numebr of $\Theta$ with respect to distance 
% \begin{align*}
%     \varrho(\theta, \tilde\theta) \defeq \max_{\pi\in\Pi, h\in[H]}\nbr{Q_h^{\pi, \theta} - Q_h^{\pi, \tilde\theta}}_\infty.
% \end{align*}
    \begin{proof}
        See \Cref{sec:proof-MLE-general} for a detailed proof.
    \end{proof}
\end{lemma}
We remark that if $\theta\in\CI_\Theta^t(\beta)$, the guarantees in the accuracy results \eqref{eq:MLE_guarantee_Q} and \eqref{eq:MLE_guarantee_Q-1} are just $8C_\eta^2 \beta$ since $\cL_h^t(\theta)-\cL_h^t(\theta^*)\le \cL_h^t(\theta) - \inf_{\theta'\in\Theta}\cL_h^t(\theta') \le \beta$ for all $h\in[H]$.
Moreover, the covering number defined here is a special case of a more general version given by \eqref{eq:rho-Theta} and also in the myopic case it is given by \eqref{eq:rho-Theta_h}. 
Next, we comment on the myopic case, where it suffices to consider a covering for $\Theta_h$ (which only contains the parameters for the follower's reward function at step $h$) instead of the whole $H$-step class $\Theta$.
\begin{remark}[Myopic case for \Cref{lem:MLE-formal}]\label{rmk:MLE-formal-myopic}
    If the follower is myopic, it suffices to use the distance $\rho$ for class $\Theta_h$ defined in \eqref{eq:rho-Theta_h}, and let $\cN_\rho(\Theta, \epsilon) = \max_{h\in[H]} \cN_\rho(\Theta_h, \epsilon)$. The conclusion in \Cref{lem:MLE-formal} still applies.
\end{remark}

% {\main
% Here, the first expectation-type guarantee is useful for the online setting while the non-expectation-type guarantee is used for the offline setting.
% For the latter,  we allow the data to be dependent across samples or there might be no stationary distribution for the data collection process.
% To bridge the result in \Cref{lem:MLE-formal} to \eqref{eq:bandit-ub-1}, we define $\hat\theta_\MLE=\argmin_{\theta\in\Theta}\cL^t(\theta)$, where $\cL^t(\theta)=\sum_{h=1}^H \cL_h^t(\theta)$. We also let $\beta'=H\beta$, and it holds that  $$\cL^t(\theta^*) - \inf_{\theta'\in\Theta}\cL^t(\theta')= \sum_{h=1}^H \cL_h^t(\theta^*) - \cL_h^t(\hat\theta_\MLE) \le \sum_{h=1}^H \cL_h^t(\theta^*) - \inf_{\theta'\in\Theta}\cL_h^t(\theta') \le H \beta =\beta', $$ 
% which shows that the confidence set in \eqref{eq:behavior_model_confset} is valid with significance level $\beta'$.
% For the accuracy result, we note that \eqref{eq:MLE-guarantee-hellinger-2} holds for any $\theta\in\Theta$ and we just sum them up for all $h\in[H]$ and use the fact \begin{align*}
%     \sum_{i=1}^{t-1} \EE^{\pi^i }D_\H^2\bigrbr{\nu_h^{\pi^i, \theta}, \nu_h^{\pi^i, \theta^*}} 
%     &\le  \sum_{h=1}^H \sum_{i=1}^{t-1} \EE^{\pi^i }D_\H^2\bigrbr{\nu_h^{\pi^i, \theta}, \nu_h^{\pi^i, \theta^*}} \nend
%     & \le \frac {1}{2}\rbr{\cL_h^t(\theta) - \cL_h^t(\theta^*)} + H \log\rbr{\frac{d H\cN_\rho(\Theta, T^{-1})}{\delta}}\nend
%     &\le \frac 1 2 \rbr{\cL^t(\theta)-\cL^t(\hat\theta_\MLE)} + H \beta \le 2\beta', 
% \end{align*}
% for $\theta\in \{\theta:\cL^t(\theta)-\cL^t(\hat\theta_\MLE)\le \beta'$\}. 
% Therefore, the conclusion in \Cref{lem:bandit} just follows from \eqref{eq:MLE-guarantee-hellinger-2} of \Cref{lem:MLE-formal}.
% \fi}

Next, we give a proof on \Cref{eq:bandit-ub-1} 
% {\main\Cref{cor:MLE confset-linear myopic}\fi}{\neurips \eqref{eq:bandit-ub-1}\fi}
, which borrows \Cref{rmk:MLE-formal-myopic} and the covering number of $\Theta_h$ given in \eqref{eq:cN-Theta_h}.
\begin{remark}[Formal statement of \eqref{eq:bandit-ub-1}
    % {\ifmain \Cref{cor:MLE confset-linear myopic}\fi}{\ifneurips \eqref{eq:bandit-ub-1}\fi}
    ]\label{cor:formal-MLE confset-linear myopic}
    Consider the myopic and linear case, Suppose that $\beta\ge C d\log(H(1+\eta T^2 + (1+\eta)T)\delta^{-1})$ for some universal constant $C>0$,  \eqref{eq:MLE_guarantee_Q-1} further implies that for all $h\in[H]$
        \begin{align}
            \max\cbr{\bignbr{\hat\theta_h-\theta_h^*}_{\Sigma_{h, t}^{\theta^*}}^2, \bignbr{\hat\theta_h-\theta_h^*}_{\Sigma_{h, t}^{\hat\theta}}^2, 
            \EE^{\pi^i, M^*}\bignbr{\hat\theta_h-\theta_h^*}_{\Sigma_{h, t}^{\theta^*}}^2, 
            \EE^{\pi^i, M^*}\bignbr{\hat\theta_h-\theta_h^*}_{\Sigma_{h, t}^{\hat\theta}}^2 } \le 8 C_{\eta}^2 \beta,
            \label{eq:app-bandit-ub-1}
        \end{align}
        where $C_\eta =\eta^{-1}+B_A$, $B_A$ is specified in \eqref{eq:define_BA}, and $\Sigma_{h, t}^{\theta}$ is a data-dependent covariance matrix defined as 
        \begin{align}\label{eq:app-cov matrix}
            \Sigma_{h, t}^{\theta}= \sum_{i=1}^{t-1}
            {\Cov_{s_h^i}^{\pi^i, \theta}\bigsbr{\phi^{\pi^i}(s_h, b_h)}}, 
        \end{align}
    where $\Cov_{s_h}^{\pi, \theta}[\phi^{\pi}(s_h, b_h)]$ represents the covariance matrix of the feature $\phi^\pi$ with respect to $\nu_h^{\pi,\theta}(\cdot\given s_h)$.
    % \begin{proof}
    %     We use \eqref{eq:MLE_guarantee_Q-1} and obtain by noting that $(Q^{\pi^i, \theta}-Q^{\pi^i, \theta^*})(s, b) = \phi^{\pi^i}(s, b)^\top(\theta_h-\theta_h^*)$, which gives the result.
    % \end{proof}
\end{remark}

We next consider a special case where each trajectory in the offline data is independently collected. 
We have the following lemma for the MLE guarantee with independent dataset.


\begin{lemma}[{Confidence in Hellinger Distance with Independent Data}]
    \label{lem:MLE-indep-data}
    Suppose the conditions in \Cref{lem:MLE-formal} hold.
    Suppose each trajectory in the offline dataset $\cD=\{\tau^t\}_{t\in[T]}$ is independently collected. 
    For the confidence set $\CI_\Theta(\beta)$ defined in \Cref{eq:behavior_model_confset-1} with $\beta$ properly chosen according to \Cref{lem:MLE-formal},
    with probability at least $1-2\delta$, it holds that (i) $\theta^*\in\CI_\Theta(\beta)$; (ii) for any $\theta\in\CI_\Theta(\beta)$ , $h\in[H]$, 
    \begin{align*}
        \sum_{i=1}^T\EE_\cD\sbr{D_\H^2\rbr{\nu_h^{\pi^i, \theta}(\cdot\given s_h^i), \nu_h^{\pi^i, \theta^*}(\cdot\given s_h^i)}}  \le \cO(\beta),
    \end{align*}
    where the expectation is taken for  the randomness in both the trajectories and in the leader's policy choices.
    \begin{proof}
        See \ref{sec:proof-MLE-indep-data} for a detailed proof.
    \end{proof}
\end{lemma}
Up to now, we have obtain all the guarantees we  need from the MLE of the follower's quantal response.

\subsection{Learning Leader's Value Function}\label{sec:app-value function}
In this section, we study the problem of learning the leader's value function for both the offline and the online setting. 

\paragraph{Learning Leader's Value Function in Offline Setting.}
for each $\pi$ and estimated follower's response model $\theta$. 
%The results in this section mainly follows \citet{xie2021bellman,lyu2022pessimism}.
We only focus on myopic follower in this section. Recall the Bellman loss we defined in \Cref{sec:offline-myopic}, which is defined as
\begin{align*}
    &\ell_h(U_h', U_{h+1}, \theta, \pi) = \sum_{i=1}^T \rbr{U_h'(s_h^i, a_h^i, b_h^i) - u_h^i -   T^{\pi, \theta}_h U_{h+1}(s_{h+1}^i)}^2.
\end{align*}
where we define $ T_h^{\pi, \theta}U_{h+1}(s_{h+1}) = \inp[]{U_{h+1}(s_{h+1}, \cdot, \cdot)}{\pi_{h+1}\otimes \nu_{h+1}^{\pi, \theta}(\cdot, \cdot\given s_{h+1})}$. 
We aim to characterize the confidence set 
\begin{align*}
    \CI_{\cU}^{\pi,\theta}(\beta) = \cbr{U\in\cU: \ell_h(U_h, U_{h+1},\theta, \pi) - \inf_{U_h'\in\cU} \ell_h(U_h', U_{h+1}, \theta,\pi)\le \beta, \forall h\in[H]}.
\end{align*}
Recall the definition of the Bellman operator for the leader in \eqref{eq:bellman_operator_leader}. Similar to this definition, we define $\TT_h^{\pi,\theta}:\sF(\cS\times\cA\times\cB)\rightarrow \sF(\cS\times\cA\times\cB)$ as 
\begin{align*}
    \rbr{\TT_h^{\pi,\theta} f} (s_h, a_h, b_h) = u_h(s_h, a_h, b_h) + \EE_{s_{h+1}\sim P_h(\cdot\given s_h, a_h, b_h)} \sbr{\rbr{ T_{h+1}^{\pi,\theta}f}(s_{h+1})},
\end{align*}
and we add $\theta$ to the superscription to remind ourselves that the expectation within $\TT_h^{\pi,\theta}$ is taken with respect to the follower's quantal behavior guided by both $\pi$ and $\theta$.
In the sequel, we denote by $U^{\pi,\theta}$ the leader's U function defined similar to \eqref{eq:U_function} but with respect to policy $\pi$ and the response model $\theta$, 
\begin{align*}
    U_h^{\pi,\theta}(s_h, a_h, b_h) & = u_h(s_h, a_h, b_h) + \rbr{ P_h \circ T_{h+1}^{\pi,\theta} \circ U_{h+1}^{\pi,\theta}}(s_h, a_h, b_h) = \TT_h^{\pi,\theta} U_{h+1}^{\pi,\theta}(s_h, a_h, b_h).
\end{align*}
We clarify that $\theta$  only contains the estimated reward for myopic follower. 
Now, we present the following corollary on the validity and accuracy of the confidence set $\CI_\cU^{\pi,\theta}(\beta)$. 
\begin{lemma}[Confidence Set $\CI_{\cU}^{\pi,\theta}(\beta)$]\label{lem:leader-bellman-loss}
    Suppose that each trajectory in the data is independently collected and the function class $\cU$ satisfies the realizability and the completeness assumption given by \Cref{thm:Offline-MG}. Suppose we have 
    % \todo{
    $\beta \ge 
    {110 H^2\cdot\log(H \cN_\rho(\cY, T^{-1})\delta^{-1}) }  + (45 H^2 + 60 H )
  $, where the covering number is defined by \eqref{eq:cN-cY}.
%   }  
  Here, we have a joint class $\cY_h = \Theta_{h+1}\times \Pi_{h+1}\times \cU^2$ and the $\epsilon$-covering number $\cN_\rho(\cY_h,\epsilon)$ is with respect to the distance $\rho$ specified in \eqref{eq:rho-cY}.
%   following distance:
%   \begin{align*}
%     &\nbr{y-\tilde y}_\infty \nend
%     &\quad = \max_{h\in[H]}\cbr{\bignbr{U_h-\tilde U_h}_\infty,  \bignbr{U_{h+1}-\tilde U_{h+1}}_\infty, \sup_{s_{h+1}\in\cS}\bignbr{(\pi_{h+1}\otimes \nu_{h+1}^{\pi, \theta}-\tilde \pi_{h+1}\otimes \nu_{h+1}^{\tilde\pi, \tilde\theta})(\cdot, \cdot\given s_{h+1})}_\TV}.
% \end{align*}
Then for any $h\in[H], \theta\in\Theta, \pi\in\Pi$, we have with probability at least $1-\delta$: (i) $U^{\pi, \theta}\in \CI_\cU^{\pi,\theta}(\beta)$; (ii) for any $\tilde U\in\CI_\cU^{\pi,\theta}(\beta)$, $\EE_\cD[\|\tilde U_{h} - \TT_{h}^{\pi,\theta}\tilde U_{h + 1}\|^2]\le 4\beta T^{-1}$, where $\EE_\cD$ is the expectation taken with respect to the data generating distribution.

% \Zhuoran{Comment on linear case.} 


% $ \cN_{\mathrm{cov}} =  \cN  ( 1/T)$ 
    \begin{proof}
        See \Cref{sec:proof-leader-bellman-loss} for a detailed proof.
    \end{proof}
\end{lemma}


% \begin{corollary}[\textit{Offline guarantee for the confidence set of the leader's value function}]\label{cor:CI-U}
%     Suppose that each trajectory in the data is independently collected and the function class $\cU$ satisfies the realizability and the completeness assumption given by \Cref{thm:Offline-MG}.  By selecting $\beta = \epsilon_S$ where $\epsilon_S$ in given in \Cref{lem:leader-bellman-loss}, 
%     \begin{proof}
       
%     \end{proof}
% \end{corollary}

\paragraph{Learning Leader's Value Function in Online Setting.}
In this subsection, we provide gurantee for online learning the leader's value function. The analysis in this subsection maily follows \citet{jin2021bellman}.
Recall the online Bellman loss given by \eqref{eq:online-MG-bellman loss} in \Cref{sec:myopic-online}, 
\begin{align*}
    &\ell_h^t(U_h', U_{h+1}, \theta_{h+1}) \nend
    &\quad = \sum_{i=1}^{t-1}  \rbr{U_h'(s_h^i, a_h^i, b_h^i) - u_h^i -  \max_{\pi_{h+1}\in\sA}\inp[\Big]{U_{h+1}(s_{h+1}^i, \cdot, \cdot)}{\pi_{h+1}\otimes \nu_{h+1}^{\pi, \theta}(\cdot, \cdot\given s_{h+1}^i)}}^2.
\end{align*}
We define confidence set for each $\theta\in\Theta, t\in[T]$ as 
\begin{align*}
    \CI_\cU^{t, \theta}(\beta) = \cbr{U\in\cU: \ell_h^t(U_h, U_{h+1}, \theta_{h+1}, \pi) - \inf_{U'\in\cU_h} \ell_h^t(U', U_{h+1}, \theta_{h+1}, \pi)\le \beta, \forall h\in[H]}.
\end{align*}
In the sequel, we define $U^{*, \theta}$ as the optimal value function if the follower's true response model is $\theta$. 
Specifically, $U^{*, \theta}$ satisfies
\begin{align*}
    U_h^{*, \theta}(s_h, a_h, b_h) = u_h(s_h, a_h, b_h) + \bigrbr{\bigrbr{P_h\circ  T_{h+1}^{*, \theta}} U_{h+1}^{*, \theta}} (s_h, a_h, b_h), 
\end{align*}
where $ T_h^{*,\theta}:\cF(\cS\times\cA\times\cB)\rightarrow \cF(\cS)$ is the policy optimization operator defined as
\begin{align*}
     T_{h}^{*,\theta} f(s_h) = \max_{\pi_h\in\sA} \dotp{f(s_h,\cdot,\cdot)}{\pi_h\otimes \nu^{\pi, \theta}(\cdot, \cdot\given s_h)}.
\end{align*}
With respect to $ T^{*,\theta}$, we define the optimistic Bellman operator for the leader 
$\TT_h^{*,\theta}:\sF(\cS\times\cA\times\cB)\rightarrow \sF(\cS\times\cA\times\cB)$ as 
\begin{align*}
    \big(\TT_h^{*,\theta} f \bigr) 
    (s_h, a_h, b_h) = u_h(s_h, a_h, b_h) + \EE_{s_{h+1}\sim P_h(\cdot\given s_h, a_h, b_h)} \bigsbr{\bigrbr{ T_{h+1}^{*,\theta}f}(s_{h+1})}.
\end{align*}
We have the following guarantee on the confidence set.
\begin{lemma}[\textrm{Online guarantee for the confidence set of the leader's value function}]\label{lem:CI-U-online}
    For the online setting and the function class $\cU$ that satisfies the realizability and the completeness assumption given by \Cref{thm:Online-MG}. We consider a joint function class $\cZ_h=\cU^2\times\Theta_{h+1}$ and denote by $\cN_\rho(\cZ_{h}, \epsilon)$ the covering number of the smallest $\epsilon$-covering net for $\cZ_h$ with respect to this distance $\rho$ defined in \eqref{eq:rho-cZ}.
    By selecting $\beta \ge \epsilon_S=c H^2 \allowbreak \log(HT\cN_\rho(\cZ, T^{-1})\delta^{-1}) + (45 H^2 + 60 B_u)$ for some universal constant $c$ where the covering number is defined by \eqref{eq:cN-cZ}, for any $t\in[T], h\in[H], \theta\in\Theta$, we have with probability at least $1-\delta$: (i) $U^{*, \theta}\in \CI_\cU^{t,\theta}(\beta)$; (ii) for any $\tilde U\in\CI_\cU^{t,\theta}(\beta)$, $\sum_{i=1}^{t-1}\EE^{\pi^i}[\orbr{\orbr{\tilde U_{h} - \TT_{h}^{*,\theta}\tilde U_{h + 1}}(s_h, a_h, b_h)}^2]\le 4\beta$ and $\sum_{i=1}^{t-1} \orbr{\orbr{\tilde U_{h} - \TT_{h}^{*, \theta}\tilde U_{h + 1}}(s_h^i, a_h^i, b_h^i)}^2 \le 4 \beta$.
    \begin{proof}
        See \Cref{sec:proof-CI-U-online} for a detailed proof.
    \end{proof}
\end{lemma}




\subsection{Putting Everything Together:  Bounding Leader's Suboptimality}\label{sec:app-connection}

Here, we give a summary of the results presented in this section and show how the results in the previous parts are connected with each other for obtaining a guarantee of the suboptimality.

\vspace{5pt} 
{\noindent \bf Controlling Leader's Bellman Error.}
In \Cref{sec:app-subopt-decompose}, we study the suboptimality decomposition for the leader. From \Cref{lem:subopt-decomposition}, we learn that the suboptimality comprises two major terms, namely the leader's Bellman error and the follower's response error. The leader's Bellman error is simply given by 
\begin{align*}
    \text{Leader's Bellman error} = \sum_{h=1}^H \EE\sbr{\bigrbr{\tilde U_h - u_h}(s_h, a_h, b_h)-  T_{h+1}^{\pi,\tilde\nu} \tilde U_{h+1}(s_{h+1})} = \sum_{h=1}^H \EE\sbr{\tilde U_h - \TT_h^{\pi,\tilde\theta} \tilde U_{h+1}}, 
\end{align*}
where a list of definitions for $\TT_h$ and $ T_h$ can be found in \Cref{sec:app-notations}. For the offline setting with independent collected data, we use the guarantee from \Cref{lem:leader-bellman-loss} that $\EE_\cD[\|\tilde U_{h} - \TT_{h}^{\pi,\theta} \tilde U_{h + 1}\|^2]\le 4\beta T^{-1}$ if $\tilde U$ if properly chosen from the confidence set $\CI_\cU^{\pi,\theta}(\beta)$. 
For the online setting, we employ  \Cref{lem:CI-U-online} to show that $\sum_{i=1}^{t-1}\EE^{\pi^i}[\|\tilde U_{h} - \TT_{h}^{\theta}\tilde U_{h + 1}\|^2]\le 4\beta$ if $\tilde U$ is properly chosen such that  $\tilde U\in\CI_\cU^{t,\theta}(\beta)$. 
Moreover, in the online setting, $\pi$ is just the optimistic policy and $\TT_h^{\pi,\theta} = \TT_h^{\theta}$. Hence, the leader's Bellman error and the value function guarantee matches and we can control the leader's Bellman error by a distribution shift argument leveraging concentrability coefficients in the offline setting or via the eluder dimension of a proper function class that captures the complexity of this Bellman error in the online setting.
% \Zhuoran{How to control? via analysis of prediction eror and  distributional shift or offlien data (offline), eluder dimension of a proper function class that captures the complexity of the Bellman error (online)}
%how to control the leader's Bellman error is straightforward.

\vspace{5pt} 
{\noindent \bf Controlling Myopic Follower's Quantal Response Error.}
We first look at the easier setting where we aim to control a myopic follower's quantal response error, which is given by the TV distance betweem $\nu$ and $\tilde \nu$.
Recall from \Cref{cor:response-diff-myopic} that for a given state $s\in\cS$,
\begin{align*}
    \EE D_\TV\rbr{\nu(\cdot\given s), \tilde\nu(\cdot\given s)}
        &\le  \eta  \EE\sbr{\abr{(\tilde r^\pi(s, b) - r^\pi(s, b)) - \EE\bigsbr{\tilde r^\pi(s, b) - r^\pi(s, b)}}} \nend
        &\qquad + C^{(3)}\EE\sbr{\rbr{\rbr{\tilde r^\pi(s, b) -r^\pi(s, b)} - \EE\sbr{\tilde r^\pi(s, b) -r^\pi(s, b)}}^2}.
\end{align*}
If we look at the guarantee of MLE in \eqref{eq:MLE_guarantee_Q} of \Cref{lem:MLE-formal} for the myopic case, we directly have 
\begin{align*}
    \sum_{i=1}^{t-1} \EE^{\pi^i, M^*}{\Var_{s_h}^{\pi^i, \theta^*} \bigsbr{r^{\pi^i, \theta}(s, b) - r^{\pi^i, \theta^*}(s, b)}} \le  2 C_\eta^2 \beta, 
\end{align*}
for both the online and the offline cases, which gives control to both the first order and the second order terms in the TV distance upper bound.


\vspace{5pt} 
{\noindent \bf Controlling Farsighted Follower's Quantal Response Error.}
The last part is a more challenging case for a farsighted follower. Using the result in \Cref{lem:performance diff}, 
\begin{align*}
    \text{Quantal response error} \le C^{(0)}
    \sum_{h=1}^H \underbrace{\EE\sbr{\abr{\tilde \Delta^{(1)}_h(s_h, b_h)}}}_{\ds\text{1st-order error}}  + C^{(2)}
    \max_{h\in [H]} \underbrace{\EE\sbr{ \rbr{\tilde Q_h - r_h^\pi - \gamma P_h^\pi \tilde V_{h+1}}^2}}_{\ds\text{2nd-order error}}, 
\end{align*}
with $\tilde\Delta^{(1)}$ given by 
\begin{align*}
    \tilde \Delta^{(1)}_h(s_h, b_h) &=  \rbr{\EE_{s_h, b_h} -\EE_{s_h}}\Biggsbr{\sum_{l=h}^H \gamma^{l-h}\underbrace{\rbr{\tilde Q_l - r_l^\pi - \gamma P_l^\pi \tilde V_{l+1}}(s_l, b_l)}_{\ds\text{Follower's Bellman error}}}, 
\end{align*}
it is not straightforward to see how to bound these two terms by guarantee of the MLE in \Cref{lem:MLE-formal}. 
Fortunately,  we have the following two lemmas that  bound  the first-order error and the second-order error  separately.
\begin{lemma}[Bounding First-Order Error]\label{lem:1st-ub}
    For any $\pi\in\Pi$ and $(\tilde U, \tilde W, \tilde Q, \tilde V, \tilde A, \tilde \nu)$ satisfying the conditions in \Cref{lem:subopt-decomposition}, we have for all $h\in[H]$ that
    \begin{align*}
        \EE\sbr{\abr{\tilde \Delta^{(1)}_h(s_h, b_h)}} &\le   L^{(1)} \cdot
        \max_{h\in[H]} \EE \bigl [  D_\H(\nu_h(\cdot\given s_h),\tilde\nu_h(\cdot\given s_h)) \bigr ] , 
    \end{align*}
    where $L^{(1)} = 6(\eta^{-1}+2 B_A)\cdot {\eff_H\rbr{\gamma}}$ and  $\eff_H(\gamma) = \orbr{1-\gamma^H}/\orbr{1-\gamma}$ is the effective foresight of the follower. For the second order, we have
    \begin{align}\label{eq:1st-ub-2}
        &\bigrbr{\tilde \Delta_h^{(1)}(s_h, b_h)}^2  \nend
        &\quad \le 2 \rbr{\rbr{\EE_{s_h, b_h}-\EE_{s_h}} \bigsbr{\orbr{Q_h - \tilde Q_h}(s_h, b_h)}}^2 \\
        &\qqquad + 16 \gamma^2  \rbr{\eta^{-1} +2 B_A}^2\eff_H(\gamma) \sum_{l=h+1}^H \gamma^{l-h-1} {\rbr{\EE_{s_h}+\EE_{s_h, b_h}}\sbr{D_\H^2(\nu_l(\cdot\given s_l), \tilde\nu_l(\cdot\given s_l))}}. \notag 
    \end{align}
        \begin{proof}
            See \Cref{sec:proof-1st-ub} for a detailed proof.
        \end{proof}
    \end{lemma}
An important observation is that the first order term is bounded only by the follower's quantal response distance without invoking any transition model error, even for farsighted follower. This is because that the first order term only captures parts of the follower's response error. We next bound the second order term.
\begin{lemma}[Bounding Second-Order Error] \label{lem:2nd-ub}
    For any $\pi\in\Pi$ and $(\tilde U, \tilde W, \tilde Q, \tilde V, \tilde A, \tilde \nu)$ satisfying the conditions in \Cref{lem:subopt-decomposition}, we additionally assume $\tilde Q_h(s_h, b_h) = \tilde r_h^\pi(s_h, b_h) + (\tilde P_h^{\pi} \tilde V_{h+1})(s_h, b_h)$ for estimated reward $\tilde r$ and transition kernel $\tilde P$. Suppose that the follower's reward at each state satisfies a linear constraint $\dotp{x}{r_h(s_h, a_h, \cdot)} = \varsigma$ at every $(s_h,a_h)\in\cS\times\cA$ for some $x:\cB\rightarrow \RR$ such that $\inp{\ind}{x}\neq 0$ and $\varsigma\in\RR$. Define ratio $\kappa = \nbr{x}_\infty/|\inp{x}{\ind}|$. Then we have that
    \begin{align*}
        &\max_{h\in[H]}\EE\sbr{ \rbr{\rbr{\tilde Q_h - r_h^\pi - \gamma P_h^\pi \tilde V_{h+1}}(s_h, b_h)}^2} \nend
        &\quad \le L^{(2)} \max_{h\in[H]}\cbr{\EE D_\H^2(\nu_h(\cdot\given s_h),\tilde\nu_h(\cdot\given s_h))+\EE D_\TV^2(P_h^\pi(\cdot\given s_h, b_h),\tilde P_h^\pi(\cdot\given s_h, b_h))}, 
    \end{align*}
    where $L^{(2)} = c H^2 \eff_H(c_2)^2 \kappa^2 \exp\rbr{8\eta B_A} (\eta^{-1}+B_A)^2$ for some absolute constant $c>0$, and 
    $$c_2 = \gamma \rbr{2\exp\rbr{2\eta B_A}+\kappa\exp\rbr{4\eta B_A} }.$$
    \begin{proof}
        See \Cref{sec:proof-2nd-ub} for a detailed proof.
    \end{proof}
\end{lemma}

In the farsighted follower case, we will intensively turn to these two lemmas to control the first- and second-order terms both online and offline. Specifically, the first term in \eqref{eq:1st-ub-2} can be controlled by \eqref{eq:MLE-guarantee-Q-3} in \Cref{lem:MLE-formal}, while the second term is just the Hellinger distance, which can be controlled by our guarantee in \eqref{eq:MLE-guarantee-hellinger-1}. The argument for \Cref{lem:2nd-ub} is quite the same, while both the Hellinger distance of the quantal response and the TV distance of the transition kernel is controllable, as we will see in \Cref{lem:MLE}.

% \subsection{Proof for \Cref{sec:app-major-tech}}

% \subsubsection{Proof of \Cref{lem:performance diff}}\label{sec:proof-performance diff}
% 
In the following, we prove \Cref{lem:performance diff}, which relates the estimation error of quantal response policy to a few estimation errors involving the follower's value functions. 
To simplify the notation, we let $\nu$, 
$Q, V, A$ denote $\nu$, $Q^{\pi}$, $V^{\pi}$, and $A^{\pi}$, respectively, which are quantities computed under the true model $M^*$.  
Note that we have  
$\nu_h(b\given s) = \exp(\eta \cdot A_h (s, b) )$ and 
$\tilde \nu_h (b \given s) = \exp(\eta \cdot \tilde A_h (s,b) )$.
By the upper bound in \eqref{eq:nu-tv-ub-0} of \Cref{lem:response diff},
we have 
\begin{align}
    \dr{(i)} &\defeq \sum_{h=1}^H  H \cdot  \EE \bigsbr{ \nbr{\tnu_h(\cdot\given s_h)-\nu_h(\cdot\given s_h)}_1} = \sum_{h=1}^H  2 H \cdot  \EE \bigsbr{  D_\TV \bigrbr{\nu(\cdot\given s_h), \tilde \nu(\cdot\given s_h)}} \nend
    &\le 2 \eta  H \cdot  \underbrace{\sum_{h=1}^H \EE\bigsbr{\bigabr{\orbr{A_h-\tilde A_h}(s_h, b_h)}}}_{\dr (ii)} \nend
    &\qqquad + \eta^2  H \cdot \sum_{h=1}^H \EE\Bigsbr{ \exp\bigrbr{\eta\bigabr{\orbr{A_h-\tilde A_h}(s_h, b_h)}}\cdot \bigabr{\orbr{A_h-\tilde A_h}(s_h, b_h)}^2}. \label{eq:(i)}
\end{align}
% In this section, we characterize the performance difference that arises from model misspecification, namely the difference in the leader's total reward under a given policy $\pi$ for a misspecified model $\tilde M$ against the true model $M^*$. In the following, we use $(Q_h, V_h, A_h, \nu_h, U_h, W_h)$ for the follower/leader under the true model $M^*$, and $(\tQ_h,\tV_h,\tA_h, \tnu_h, \tilde U_h, \tilde W_h)$ for the follower/leader under the alternative model $\tilde M$. 
% We let
% \begin{align}
%     \dr{(i)} &\defeq \sum_{h=1}^H  H \EE \nbr{\tnu_h(\cdot\given s_h)-\nu_h(\cdot\given s_h)}_1\nend
%     &\le 2 \eta  H \underbrace{\sum_{h=1}^H \EE\sbr{\abr{\rbr{A_h-\tilde A_h}(s_h, a_h)}}}_{\dr (ii)} \nend
%     &\qqquad + \eta^2  H \sum_{h=1}^H \EE\sbr{ \exp\rbr{\eta\abr{\rbr{A_h-\tilde A_h}(s_h, a_h)}}\cdot \abr{\rbr{A_h-\tilde A_h}(s_h, a_h)}^2}, \label{eq:(i)}
% \end{align}
%where the first inequality holds by noting that $\bignbr{\tilde U_h}_\infty \le  H$, and the second inequality uses the upper bound for the TV distance between two difference logistic responses given by \Cref{lem:response diff}.
In the following, we let 
\$
    \tilde\Delta_h^{(1)} (s_h, b_h)&\defeq  \rbr{\EE_{s_h, b_h} - \EE_{s_h}}\sbr{\sum_{l=h}^H \gamma^{l-h}\bigrbr{\orbr{\tilde Q_l - r_l^\pi - \gamma P_l^\pi \tilde V_{l+1}}(s_l, b_l)}}, \\
    \tilde\Delta_h^{(2)}(s_h) &\defeq \EE_{s_h}\sbr{\sum_{l=h}^H \gamma^{l-h} \kl\infdivx[\big]{\nu_l(\cdot\given s_l)}{\tilde\nu_l(\cdot\given s_l)}}. 
\$
We note that we denote $\EE_{z} [\cdot]=\EE^{\pi,M^*}[\cdot\given z]$ for any variable $z$.
Here the expectations in $\tilde\Delta^{(1)}_h$ and $\tilde\Delta^{(2)}_h$ are taken with respect to the randomness of the trajectory generated by $\{ \pi, \nu^{\pi}\}$, given $s_h$ or $(s_h, b_h)$.
%Note that $\tilde\Delta^{(1)}_h$ is actually a short hand of $\tilde\Delta^{(1)}_{h,\pi, M}(s_h, b_h)$.
We can further bound   (ii) defined in \eqref{eq:(i)} by invoking \Cref{lem:AQV-func diff}, 
which implies that 
\$
\dr{(ii)}&= \sum_{h=1}^H \EE\sbr{\abr{\rbr{\EE_{s_h, b_h}-\EE_{s_h}} \bigsbr{\tilde\Delta_h^{(1)}(s_h, b_h) - \gamma\eta^{-1}\tilde\Delta_{h+1}^{(2)} (s_{h+1})} + \eta^{-1}\kl\infdivx[\big]{\nu_h(\cdot\given s_h)}{\tilde \nu_h(\cdot\given s_h)}}}.
\$
By the law of total expectation, 
we have 
\$
\EE \Bigsbr{\EE _{s_h, b_h} \bigsbr{   \tilde\Delta_{h+1}^{(2)} (s_{h+1})}} = \EE \Bigsbr{\EE _{s_h } \bigsbr{   \tilde\Delta_{h+1}^{(2)} (s_{h+1})}}  \geq 0.
\$
Besides, by the definition of $\tilde \Delta_h^{(2)}$, we have 
\$
\EE\bigsbr{\tilde\Delta_h^{(2)}(s_h)} = \EE\bigsbr{ \kl\infdivx[\big]{\nu_h(\cdot\given s_h)}{\tilde \nu_h(\cdot\given s_h)} + \gamma \cdot \tilde\Delta_{h+1}^{(2)} (s_{h+1}) }. 
\$ 
Thus, by triangle inequality, we have 
\begin{align}
    \dr{(ii)}  &\le \sum_{h=1}^H\EE\Bigsbr{\bigabr{\rbr{\EE_{s_h, b_h}-\EE_{s_h}}\bigsbr{\tilde\Delta_h^{(1)}(s_h, b_h)}}} + 2\eta^{-1} \sum_{h=1}^H\EE\bigsbr{\tilde\Delta_h^{(2)}(s_h)}, \label{eq:(ii)}
\end{align}
Furthermore, for the second term on the right-hand side of \eqref{eq:(ii)}, we  apply the inequality between KL divergence and the $\chi^2$ divergence to each term $\kl\infdivx[]{\nu_l(\cdot\given s_l)}{\tilde\nu_l(\cdot\given s_l)}$ and obtain that 
\begin{align}
\kl\infdivx[\big]{\nu_l(\cdot\given s_l)}{\tilde\nu_l(\cdot\given s_l)}
&\le \chi^2\infdivx[\big]{\nu_l(\cdot\given s_l)}{\tilde\nu_l(\cdot\given s_l)}\nend
&= \inp[\Bigg]{\nu_l(\cdot\given s_l)}{\biggrbr{\sqrt{\frac{\nu_l(\cdot\given s_l)}{\tilde\nu_l(\cdot\given s_l)}}-\sqrt{\frac{\tilde\nu_l(\cdot\given s_l)}{\nu_l(\cdot\given s_l)}}}^2}_{\cB}\nend
&\le \eta^2 \cdot \EE_{s_l}\sbr{\exp\bigrbr{\eta  \cdot \bigabr{\orbr{A_l-\tilde A_l}(s_l,b_l)}}\cdot \bigrbr{\orbr{A_l -\tilde A_l}(s_l, b_l)}^2}, \label{eq:kl-ub}
\end{align}
where last expectation is with respect to $b_ l \sim \nu_{l } (\cdot \given s_l)$. 
Here the inequality holds by noting that 
 $\sqrt{\nu_l/\tilde \nu_l}=\exp(\eta(A_l-\tilde A_l)/2)$ and the basic inequality $| \exp( x ) - \exp(y)| \leq \exp ( | x-y|) \cdot |x -y|$.
Plugging \eqref{eq:kl-ub} back into \eqref{eq:(ii)}, we obtain
\begin{align}
    \dr{(ii)} &\le \sum_{h=1}^H\EE\Bigsbr{\bigabr{\rbr{\EE_{s_h, b_h}-\EE_{s_h}}\bigsbr{\tilde\Delta_h^{(1)}(s_h, b_h)}}}  + 2\eta^{-1} \sum_{h=1}^H \EE\sbr{\sum_{l=h}^H \gamma^{l-h} \cdot \kl\infdivx[\big]{\nu_l(\cdot\given s_l)}{\tilde\nu_l(\cdot\given s_l)}}\nend
    &\le \sum_{h=1}^H\EE\sbr{\bigabr{\rbr{\EE_{s_h, b_h}-\EE_{s_h}}\bigsbr{\tilde\Delta_h^{(1)}(s_h, b_h)}}} \nend
    &\qquad + 2\eta \sum_{h=1}^H \sum_{l=h}^{H} \gamma^{l-h} \cdot  \EE\sbr{\exp\bigrbr{\eta  \cdot \bigabr{\orbr{A_l-\tilde A_l}(s_l,b_l)}}\cdot \bigabr{\orbr{A_l -\tilde A_l}(s_l, b_l)}^2}\nend
    &\le \sum_{h=1}^H\EE\sbr{\bigabr{\rbr{\EE_{s_h, b_h}-\EE_{s_h}}\bigsbr{\tilde\Delta_h^{(1)}(s_h, b_h)}}} \nend
    &\qquad + \frac{2\eta(1-\gamma^H)}{1-\gamma}\cdot \sum_{h=1}^H  \EE\sbr{\exp\bigrbr{\eta  \cdot \bigabr{\orbr{A_h-\tilde A_h}(s_h,b_h)}}\cdot \bigabr{\orbr{A_h -\tilde A_h}(s_h, b_h)}^2} .
     \label{eq:(ii)-2}
\end{align}
Recall that we  define $\eff_H(x) = (1-x^H)/(1-x)$ as the \say{effective}  horizon with respect to $x$.


Plugging \eqref{eq:(ii)-2} back into \eqref{eq:(i)}, we conclude that
\begin{align}
    \dr{(i)}&\le 2\eta  H \cdot \sum_{h=1}^H  \EE\sbr{\abr{\rbr{\EE_{s_h, b_h}-\EE_{s_h}}\bigsbr{\tilde\Delta_h^{(1)}(s_h, b_h)}}}   \nend
    &\qquad + \eta^2  H  \bigrbr{1+ 4  \cdot \eff_H(\gamma) } \cdot \sum_{h=1}^H  \EE\sbr{\exp\bigrbr{\eta  \cdot \bigabr{\orbr{A_h-\tilde A_h}(s_h,b_h)}}\cdot \bigabr{\orbr{A_h -\tilde A_h}(s_h, b_h)}^2}  . \label{eq:(ii)-21}
\end{align}
Note that we define  $C^{(1)}$ in \eqref{eq:define_constants}. 
Since $\oabr{\orbr{A_h-\tilde A_h}(s_h,b_h)} \leq 2 B_{A}$, 
by  \eqref{eq:(ii)-21}  and inequality 
\$
\exp\bigrbr{\eta  \cdot \bigabr{\orbr{A_h-\tilde A_h}(s_h,b_h)}}  \leq \exp(2\eta B_{A}), 
\$
we conclude the proof of \eqref{eq:taylor-myopic}. 

It remains to prove \eqref{eq:taylor-farsighted}. 
Notice that 
\#\label{eq:f_2-01}
\begin{split}
    V_h (s_h) = \max_{\nu' \in \Delta (\cB) }\bigl\{ \inp{\nu' }{Q_h (s_h, \cdot )}_{\cB } +\eta^{-1} \cH(\nu')\bigr\} , \\
    \tilde V_h (s_h) = \max_{\nu' \in \Delta (\cB) }\bigl\{ \inp{\nu' }{\tilde Q_h (s_h, \cdot )}_{\cB } +\eta^{-1} \cH(\nu')\bigr\} ,  
\end{split}
\#
where the maximizers are $\nu_h (\cdot \given s_h)$ and $\tilde \nu_h (\cdot \given s_h)$, respectively.
Then, by  \eqref{eq:f_2-01}
we have 
\#
& \bigabr{\orbr{V_h-\tilde V_h}(s_h)}   \notag \\
& \quad \le \max\Bigcbr{\inp[\big]{\nu_h(\cdot \given s_h) }  { \bigabr{Q_h(s_h, \cdot )-\tilde Q_h(s_h, \cdot )} }_{\cB} }, ~\Bigabr{\inp[\big]{\tilde\nu_h(\cdot \given s_h) }{\bigabr{ Q_h(s_h, \cdot )-\tilde Q_h(s_h, \cdot )} }_{\cB} } \notag\\
& \quad  =   \max\Bigcbr{  \EE_{s_h} \bigl [  \bigl | (Q_h - \tilde Q_h) (s_h, b_h ) \big | \bigr ] , ~   \EE_{s_h} \bigl [ \big |  (Q_h - \tilde Q_h)  (s_h, b_h )\bigr |  \cdot \tilde \nu_h (b_h \given s_h) / \nu_h (b_h \given s_h ) \bigr ]  }   \notag \\
& \quad  \leq    \exp(2 \eta B_A)  \cdot    \EE_{s_h} \bigl [ \big |  (Q_h - \tilde Q_h) (s_h, b_h )\big | \bigr ]   , \label{eq:f_2-1} 
\# 
where the expectation is taken with respect to $b_h \sim \nu_h (\cdot \given s_h)$. 
Here the first inequality is obtained from the optimality condition of \eqref{eq:f_2-01}, and the  last inequality holds because  $\nbr{\tilde\nu_h/\nu_h}_\infty \le \exp(2\eta B_A)$. 
Note that $\tilde A = \tilde Q - \tilde V$ and $A = Q - V$.
By triangle inequality, we have 
\begin{align}
    &\abr{\orbr{A_h-\tilde A_h}(s_h, b_h)}  
  \le  \bigabr{\orbr{Q_h-\tilde Q_h}(s_h, b_h)} + \exp\orbr{2\eta B_A} \cdot  \EE_{s_h} \bigl [  \big |  (Q_h - \tilde Q_h) (s_h, b_h )\big |  \bigr ]  .  \label{eq:f_2-11} 
\end{align}
%   
%  
%  
% For the left side, we first notice 
% \begin{align*}
%     \EE_{s_h}\rbr{A_h^\pi-\tilde A_h^\pi}^2 
%     &= \EE_{s_h}\sbr{\rbr{\rbr{\EE_{s_h, b_h} -\EE_{s_h}}\sbr{A_h^\pi-\tilde A_h^\pi}}^2 }+ \rbr{\EE_{s_h}\sbr{A_h^\pi-\tilde A_h^\pi}}^2\nend
%     & = \EE_{s_h}\sbr{\rbr{\rbr{\EE_{s_h, b_h}-\EE_{s_h}}\sbr{\tilde\Delta_h^{(1)}(s_h, b_h) - \gamma\eta^{-1}\tilde\Delta_{h+1}^{(2)} (s_{h+1})}}^2} + \rbr{\EE_{s_h}\abr{A_h^\pi-\tilde A_h^\pi}}^2,
% \end{align*}
% where the first equality follows from a standard mean-variance decomposition, and the inequality holds by \eqref{eq:A diff-1} in \Cref{lem:AQV-func diff} and noting that $\eta^{-1}\kl\infdivx{\nu_h}{\tilde\nu_h} = \EE_{s_h, b_h}\bigsbr{A_h-\tilde A_h}$.
Now for $\oabr{   (Q_h - \tilde Q_h) (s_h, b_h ) }$, by the Bellman equation $Q_h = r^{\pi}_h + \gamma P_h^{\pi} V_{h+1}$, we have  
\begin{align}
    &\bigabr{ \orbr{Q_h-\tilde Q_h}(s_h, b_h)} \nend
    &\quad \le  
    \bigabr{ 
        \orbr{\tilde Q_h - r_h^\pi - \gamma P_h^\pi \tilde V_{h+1}}(s_h, b_h)
    } 
         + \gamma  \cdot \bigabr{\bigrbr{P_h^\pi \orbr{V_{h+1}-\tilde V_{h+1}}}(s_h, b_h)}
         \nend
    &\quad \le \bigabr{\orbr{\tilde Q_h - r_h^\pi - \gamma P_h^\pi \tilde V_{h+1}}
    (s_h, b_h)} 
    + \gamma \cdot \exp\rbr{2\eta B_A}\EE_{s_h, b_h}\bigsbr{\oabr{Q_{h+1}^\pi-\tilde Q_{h+1}^\pi}},\label{eq:f_2-Q-ub}
\end{align}
where the first inequality holds by a standard decomposition and the second inequality is obtained by applying the  same upper bound for $V_{h}-\tilde V_h$ in \eqref{eq:f_2-1} to $V_{h+1} - \tilde V_{h+1}$. 
By recursion, we have 
\begin{align}
    \bigabr{\orbr{Q_h-\tilde Q_h}(s_h, b_h)} &\le  \sum_{l=h}^H \big (\gamma \cdot \exp(2\eta B_A  ) \big) ^{l-h} \cdot  \EE_{s_h, b_h}\bigsbr{\bigabr{\orbr{\tilde Q_l - r_l^\pi - \gamma P_l^\pi \tilde V_{l+1}}(s_l, b_l)}}.\label{eq:f_2-Q-telo}
    % \nend
    % &\qqquad +  \sum_{l=h}^H \exp(2\eta B_A (l-h+1)) \gamma^{l-h} \rbr{\EE_{s_h}\abr{r_l^\pi-\tilde r_l^\pi} + \gamma \EE_{s_h}\abr{\rbr{P_l^\pi-\tildeP_l^\pi}\tilde V_{h+1}}}.
\end{align}
Now, by the boundedness of $A_h$ and $\tilde A_h$, and \eqref{eq:f_2-11}, 
we have 
\$
& \EE\sbr{\exp\bigrbr{\eta  \cdot \bigabr{\orbr{A_h-\tilde A_h}(s_h,b_h)}}\cdot \bigabr{\orbr{A_h -\tilde A_h}(s_h, b_h)}^2} \notag \\
& \quad \leq \exp\bigrbr{2\eta B_A}\cdot \EE\bigsbr{ \bigabr{\orbr{A_h-\tilde A_h}(s_h, a_h)}^2}\nend
&\quad \le 2\exp\bigrbr{6\eta B_A}\cdot {\EE\bigsbr{ \bigabr{\orbr{\tilde Q_h - Q_h}(s_h, b_h)}^2}},
\$ 
where in the  last inequality we use the basic inequality $(a + b)^2 \leq 2 a ^2 + 2 b^2 $. 
Combining  with \eqref{eq:f_2-Q-telo}, we obtain that 
\begin{align*}
    &\EE\sbr{\exp\bigrbr{\eta  \cdot \bigabr{\orbr{A_h-\tilde A_h}(s_h,b_h)}}\cdot \bigabr{\orbr{A_h -\tilde A_h}(s_h, b_h)}^2} \nend
    &\quad \le 2\exp\bigrbr{6\eta B_A} \cdot \EE \biggsbr { \biggrbr{\sum_{l=h}^H (\gamma \cdot \exp(2\eta B_A  ) \big) ^{l-h} \cdot  \EE_{s_h, b_h}\bigsbr{\bigabr{\orbr{\tilde Q_l - r_l^\pi - \gamma P_l^\pi \tilde V_{l+1}}(s_l, b_l)}}}^2 } \nend
    &\quad \le 2\exp\orbr{6\eta B_A} \cdot \bigrbr{\eff_H(\exp\orbr{2\eta B_A}\gamma)}^2  \cdot \max_{l\in\{h, \dots, H\}}\EE\bigsbr{\bigabr{\orbr{\tilde Q_l - r_l^\pi - \gamma P_l^\pi \tilde V_{l+1}}(s_l, b_l)}^2}.
\end{align*}
Recall that we define  $C^{(2)} = 2 \eta^2  H^2 \cdot \exp\rbr{6\eta B_A}  \cdot \rbr{1+ 4 \eff_H(\gamma)} \cdot \rbr{\eff_H(\exp\cbr{2\eta B_A}\gamma)}^2$. By \eqref{eq:(ii)-21}, we establish \eqref{eq:perform-diff-linear}. 
Therefore, we 
  complete the proof of \Cref{lem:performance diff}.

% \subsubsection{Proof of \Cref{cor:response-diff-myopic}}
% \label{sec:proof-response-diff-myopic}
% 


Since we consider any fixed state $s \in \cS$, in this proof, we omit $s$ 
to simplify the notation. 
To apply Lemma \ref{lem:performance diff}, 
we note that 
$Q$ and $\tilde Q$ in   Lemma \ref{lem:performance diff} becomes $r^{\pi}$ and $\tilde r^{\pi}$ in the myopic case. 
To make the proof consistent with that of Lemma \ref{lem:performance diff}, we use notation $\{ Q, \tilde Q, V, \tilde V, A, \tilde A\}$ in the sequel. 

 
To begin with, 
we invoke \eqref{eq:nu-tv-ub-0} in \Cref{lem:response diff} and obtain that 
\begin{align*}
    D_\TV\orbr{\nu, \tilde \nu}
    &\le \eta \cdot \inp[\big]{\nu} {\oabr{\tilde A-A} + \frac \eta 2 \exp\bigrbr{\eta\oabr{\tilde A-A}} \cdot \orbr{\tilde A-A}^2}_{\cB } \nend
    &\le \eta \cdot \inp[\big]{\nu} {\oabr{\tilde A-A} + \frac \eta 2 \exp\orbr{2\eta B_A} \orbr{\tilde A-A}^2}_{\cB }\nend
    &= \eta \cdot \EE\bigsbr{\oabr{\tilde A-A}} + \frac {\eta^2} {2} \exp\bigrbr{2\eta B_A} \cdot \Bigrbr{\Var\orbr{\tilde A-A}+ \bigrbr{\EE\osbr{\tilde A-A}}^2},
\end{align*}
where the last equality holds by the  variance-mean decomposition. 
Here the expectation and variance are taken with respect to $   b \sim \nu(\cdot \given s ) $. 
Now, using \Cref{lem:AQV-func diff} to the myopic case, we have
\begin{align*}
    \EE\bigsbr{\oabr{\tilde A - A}} 
    &\le \EE\bigsbr{\bigabr{(\tilde Q - Q) - \EE  \osbr{\tilde Q - Q}}} + \eta^{-1}\kl\infdivx[]{\nu}  {\tilde\nu}.  
\end{align*}
For the variance term, we have
\begin{align*}
    \Var\orbr{\tilde A -A} & = \EE\bigsbr{\bigrbr{\orbr{\tilde A -A} - \EE\orbr{\tilde A -A}}^2}  = \EE\bigsbr{\bigrbr{\orbr{\tilde Q -Q} - \EE\osbr{\tilde Q -Q}}^2}, 
\end{align*}
where the last equality holds because $V $ and $\tilde V $ do not involve $b $. 
Furthermore, 
note that $$\eta\EE\osbr{A-\tilde A} = \kl\infdivx[]{\nu}{\tilde\nu} \leq 2 \eta B_A.$$ 
Thus, combining the inequalities above, we have 
  we have
\begin{align}
    & D_\TV(\nu,\tilde\nu)\nend 
    & \quad \le   \eta \cdot   \EE\bigsbr{\bigabr{(\tilde Q - Q) - \EE  \osbr{\tilde Q - Q}}} + \frac{\eta^2}{2} \cdot \exp\rbr{2\eta B_A} \cdot \EE\bigsbr{\bigrbr{\orbr{\tilde Q -Q} - \EE\osbr{\tilde Q -Q}}^2}\nend
    &\qquad + \kl\infdivx[]{\nu}{\tilde\nu} + \exp\rbr{2\eta B_A}/ 2 \cdot  \bigrbr{\kl\infdivx[]{\nu}{\tilde\nu}}^2\nend
    &\quad \le\eta \cdot   \EE\bigsbr{\bigabr{(\tilde Q - Q) - \EE  \osbr{\tilde Q - Q}}}  + \frac{\eta^2}{2} \cdot \exp\rbr{2\eta B_A} \cdot  \EE\bigsbr{\bigrbr{\orbr{\tilde Q -Q} - \EE\osbr{\tilde Q -Q}}^2}\nend
    &\qquad + \bigrbr{1 + \eta B_A \exp\rbr{2\eta B_A} }\cdot \kl\infdivx[]{\nu}{\tilde\nu}, \label{eq:TV-ub-taylor}
\end{align}
where the  last inequality holds by noting that $\kl\infdivx[]{\nu}{\tilde\nu}\le 2\eta B_A$.
% Now, suppose that $r$ is parameterized by $\theta$ and $\tilde r$ is parameterized by $\tilde\theta$.

In the following, we  handle the KL divergence term.
We calculate the derivative of $\eta^{-2}\kl\infdivx{\nu}{\tilde\nu}$ with respect to $\tilde Q$ and obtain
\begin{align*}
    \partial_{\tilde Q}\rbr{\eta^{-2}\kl\infdivx{\nu}{\tilde\nu}} = \eta^{-1}\partial_{\tilde Q} \bigrbr{\EE\osbr{A-\tilde A}} = \eta^{-1}{\bigrbr{\partial_{\tilde Q} \tilde V - \nu}} = \eta^{-1}\rbr{\tilde \nu - \nu},
\end{align*}
where $\ind$ denote the all one vector of length $|\cB|$ is $\cB$ is discrete.
Here the first equality follows from $\eta\EE\osbr{A-\tilde A} = \kl\infdivx[]{\nu}{\tilde\nu}$, the second equality holds because $\nu$ and $A$ do not depend on $\tilde Q$, and $\tilde A = \tilde Q - \tilde V$. 
Moreover, the last equality holds because 
$$\tilde V(s)  = \eta^{-1} \log \bigg(\sum_{b \in \cB} \exp \big( \eta \cdot  \tilde Q(s, b)\bigr) \biggr), $$
and also $\partial_{\tilde Q}\EE[\tilde Q] = \nu$.
We further take a second-order derivative and obtain
\begin{align*}
    \partial^2_{\tilde Q \tilde Q} \rbr{\eta^{-2}\kl\infdivx{\nu}{\tilde\nu}} = \eta^{-1}\partial_{\tilde Q} \tilde \nu = \diag(\tilde \nu) -\tilde \nu \tilde\nu^\top\eqdef \H, 
\end{align*}
where the last equality holds for the vector case. 
% For a continuous action space, we have the Hessian represented as $\partial^2_{\tilde Q \tilde Q} \rbr{\eta^{-2}\kl\infdivx{\nu}{\tilde\nu}}(b, b') = \delta(b-b') - \tilde \nu(b)\tilde\nu(b')$. For simplicity, we just stick to the notation for the discrete case while the generalization to the continuous case is just a matter of change of notations. 
Note that the Hessian is upper and lower bounded by $\L$ where $\L = \diag(\nu )-\nu \nu^\top$, which is proved by  the following proposition. 
\begin{proposition}\label{prop:Hessian-ulb}
Let $\H = \diag(\tilde\nu) -\tilde\nu \tilde\nu^\top$ and $\L=\diag(\nu)-\nu \nu^\top$ where $\nu=\exp\orbr{\eta A}$ and $\tilde\nu=\exp\orbr{\eta \tilde A}$ are two quantal response over $\cB$ with $\nbr{A}_\infty\le B_A, \onbr{\tilde A}_\infty\le B_A$. Then   for any vector  $g\in \RR^{|\cB| }$, we have  
\begin{align}
    \exp\rbr{2 \eta B_A} \cdot g^\top \L g \ge x^\top \H x \ge \exp\rbr{-2 \eta B_A} \cdot g^\top \L g.\label{eq:Hessian ub lb}
\end{align}
\end{proposition}
\begin{proof}
Note that $\exp\rbr{-2 \eta B_A}\le  \tilde \nu(b) / \nu(b) \le\exp\rbr{ 2 \eta B_A}$ for any $b\in \cB$. 
Let $\EE^{\nu}$ and $\Var^{\nu}$ denote the expectation and variance under distribution $\nu$. 
Then we have 
\begin{align*}
    g^\top \L g & = \Var^\nu[g(b)]
   = \EE^\nu\bigsbr{\bigrbr{g(b) - \EE^\nu[g(b)]}^2},\notag \\
   g^\top \H g & = \Var^{\tilde \nu}[g(b)]
   = \EE^{\tilde \nu}\bigsbr{\bigrbr{g(b) - \EE^\nu[g(b)]}^2}.
\end{align*}
By direct computation, we have 
\begin{align*}
    & \exp\rbr{-\eta B_A} \cdot \EE^{\tilde\nu}\bigsbr{\bigrbr{g(b) - \EE^{\tilde\nu}[g(b)]}^2}
    \notag \\
    & \quad \le \exp\rbr{-\eta B_A}\cdot \EE^{\tilde\nu}\bigsbr{\bigrbr{g(b) - \EE^{\nu}[g(b)]}^2} 
     \le \EE^\nu\sbr{\rbr{g(b) - \EE^\nu[g(b)]}^2} 
    %%%%%%%%
\end{align*}
where the first inequality is true because changing $\EE^{\tilde\nu}[g(b)]$ to $\EE^{ \nu}[g(b)]$ incurs additional bias, and the second inequality is true because $\tilde \nu (b) / \nu(b)$ 
Similarly, we have 
\begin{align*}
    %%%%%%%%
     \EE^\nu\sbr{\rbr{g(b) - \EE^\nu[g(b)]}^2} 
    %%%%%%%%
    &\le \EE^{\nu}\sbr{\rbr{g(b) - \EE^{\tilde\nu}[g(b)]}^2}  
    %%%%%%%%
     \le  \exp\rbr{\eta B_A}  \cdot \EE^{\tilde\nu}\sbr{\rbr{g(b) - \EE^{\tilde\nu}[g(b)]}^2}.
\end{align*}
Therefore, we conclude that \eqref{eq:Hessian ub lb} holds. 
\end{proof}


Using the lower bound in \eqref{eq:Hessian ub lb}, we have for the KL divergence that
\begin{align*}
    \eta^{-2}\kl\infdivx[]{\nu}{\tilde\nu} &\le 1/2 \cdot (\tilde Q - Q)^\top  \H (\tilde Q - Q)  \le  \exp\rbr{2\eta B_A}/ 2 \cdot (\tilde Q - Q)^\top  \L (\tilde Q - Q) \nend
    &= \frac{\exp\rbr{2\eta B_A}}{2} \cdot (\tilde Q - Q)^\top  \bigrbr{\diag(\nu)-\nu\nu^\top} (\tilde Q - Q) , 
\end{align*}
where the first inequality holds by noting that the derivative of the KL-divergence at $\tilde\nu=\nu$ is zero, and we upper bound the KL-divergence  only by the second order term. Furthermore, the second inequality holds because  $\H\preceq \exp(2\eta B_A)\cdot \L$, which is proved  by \Cref{prop:Hessian-ulb}. 
% where the last inequality holds by applying \eqref{eq:Hessian ub lb} to the Hessian of the KL divergence evaluated at $\nu$. 
%Note that one can also plug in $\tilde\nu$ in the last inequality. 
Therefore, we conclude for \eqref{eq:TV-ub-taylor} that
\begin{align*}
    D_\TV \rbr{\nu, \tilde \nu} &\le \eta \cdot   \EE\bigsbr{\bigabr{(\tilde Q - Q) - \EE\osbr{\tilde Q - Q}}} + \frac{\eta^2}{2} \exp\rbr{2\eta B_A} \cdot \EE\bigsbr{\bigrbr{\orbr{\tilde Q -Q} - \EE\osbr{\tilde Q -Q}}^2}\nend
    &\qquad + \bigrbr{1 + \eta B_A \cdot \exp\rbr{2\eta B_A} }\cdot \kl\infdivx[]{\nu}{\tilde\nu}\nend
    &\le \eta \cdot   \EE\bigsbr{\bigabr{(\tilde Q - Q) - \EE\osbr{\tilde Q - Q}}} \nend
    &\qquad + \frac{\eta^2 \exp(2\eta B_A)}{2} \bigrbr{2+\eta B_A \cdot  \exp\rbr{2\eta B_A}} \cdot  \EE\bigsbr{\bigrbr{\orbr{\tilde Q -Q} - \EE\osbr{\tilde Q -Q}}^2}, 
\end{align*}
which finishes the proof of \Cref{cor:response-diff-myopic}.


% % \subsubsection{Proof of \Cref{lem:response diff-myopic}}
% % \label{sec:proof-formal-response diff-linear}
% % % We first invoke the upper bound for this TV distance in \eqref{eq:response decomposition} of \Cref{lem:performance diff} where we take $H=1$ for myopic follower and swap the position of $\theta^*$ and $\tilde\theta$ (by the exchangeability of the TV distance), 
% \begin{align}
%     &D_\TV\rbr{\nu^{\pi,\theta^*}(\cdot\given s) , \nu^{\pi,\tilde\theta}(\cdot\given s)} \nend
%     &\quad \le \eta \cdot {\EE_{s}^{\pi, \tilde\theta}\sbr{\abr{\rbr{\EE^{\pi, \tilde\theta}_{s, b}-\EE^{\pi, \tilde\theta}_{s}}\bigsbr{ \inp{\phi^{\pi}(s, b)}{\theta^*-\tilde\theta} }}}} \nend
%     & \qquad\quad + \frac 1 2 \eta^2  \cdot 
%     {\EE_{s}^{\pi, \tilde\theta}\sbr{ \exp\bigrbr{\eta\bigabr{A^{\pi,\tilde\theta}-A^{\pi,\theta^*}}}\cdot \bigabr{A^{\pi,\tilde\theta} - A^{\pi, \theta^*}}^2}}\nend
%     &\quad \le \eta  \cdot \underbrace{\sqrt{\EE_{s}^{\pi, \tilde\theta}\sbr{\rbr{\rbr{\EE^{\pi, \tilde\theta}_{s, b}-\EE^{\pi, \tilde\theta}_{s}}\bigsbr{ \inp{\phi^{\pi}(s, b)}{\theta^*-\tilde\theta} }}^2 }}}_{\dr (i)} \nend
%     & \qquad\quad + \frac 1 2 \eta^2  \exp\rbr{2\eta B_A} \cdot 
%     {\EE_{s}^{\pi, \tilde\theta}\sbr{ \bigrbr{A^{\pi,\tilde\theta} - A^{\pi, \theta^*}}^2}},\label{eq:TV-ub-MLE} 
% \end{align}
% where the second inequality holds by using the Cauchy-Schwartz inequality and bounding the exponential term by its maximal value.
% By a standard variance and mean decomposition in the last quadratic term of \eqref{eq:TV-ub-MLE}, we obtain
% \begin{align}
%     \EE_{s}^{\pi, \tilde\theta}\sbr{ \bigrbr{A^{\pi,\tilde\theta} - A^{\pi, \theta^*}}^2}
%     & \le \EE_{s}^{\pi, \tilde\theta}\sbr{ \rbr{ \rbr{\EE^{\pi, \tilde\theta}_{s, b}-\EE^{\pi, \tilde\theta}_{s}}\sbr{A^{\pi,\tilde\theta} - A^{\pi, \theta^*}}}^2} + \rbr{\EE^{\pi, \tilde\theta}_{s}\sbr{A^{\pi,\tilde\theta} - A^{\pi, \theta^*}}}^2\nend
%     & =\underbrace{\EE_{s}^{\pi, \tilde\theta}\sbr{ \rbr{ \rbr{\EE^{\pi, \tilde\theta}_{s, b}-\EE^{\pi, \tilde\theta}_{s}}\sbr{\inp[\big]{\phi^{\pi}(s, b)}{\theta^*-\tilde\theta} }}^2}}_{\dr (i)^2} + \underbrace{\rbr{\EE^{\pi, \tilde\theta}_{s}\sbr{A^{\pi,\tilde\theta} - A^{\pi, \theta^*}}}^2}_{\dr (ii)}.
%     % &\le \EE_{s}^{\pi, \tilde\theta}\sbr{ \rbr{ \rbr{\EE^{\pi, \tilde\theta}_{s, b}-\EE^{\pi, \tilde\theta}_{s}}\sbr{\inp[\big]{\phi^{\pi}(s, b)}{\theta^*-\tilde\theta} }}^2}
%     % &\qquad + \underbrace{\rbr{ \rbr{\EE^{\pi, \theta^*}_{s}-\EE^{\pi, \tilde\theta}_{s}}\sbr{\inp[\big]{\phi^{\pi}(s, b)}{\theta^*-\tilde\theta} }}^2}_{\dr (iii)},
%     \label{eq:A-square-ub}
% \end{align}
% where the equality follows from \Cref{lem:AQV-func diff} on the difference in the advantage function
% % , and the last inequality holds by \Cref{lem:KL-ub} where we notice that $\kl\infdivx[\big]{\nu^{\pi, \tilde\theta}}{\nu^{\pi, \theta^*}} = \EE^{\pi, \tilde\theta}_{s}\bigsbr{A^{\pi,\tilde\theta} - A^{\pi, \theta^*}}$.
% Combining \eqref{eq:A-square-ub} with \eqref{eq:TV-ub-MLE}, we conclude that
% \begin{align}
%     D_\TV\rbr{\nu^{\pi,\theta^*}(\cdot\given s) , \nu^{\pi,\tilde\theta}(\cdot\given s)} &\le \eta \cdot {\dr(i)} + \frac{\eta^2\exp\rbr{2\eta B_A}}{2} \cdot \rbr{{\dr(i)}^2 + {\dr(ii)}}.\label{eq:TV-ub-offline}
% \end{align}
% It is straightforward to bound term (i) by guarantee of MLE.  By \eqref{eq:bandit-ub-2} in \Cref{lem:bandit}, we have
% \begin{align}
%     \bignbr{\theta^*-\tilde\theta_{\MLE}}_{{\Psi}}^2 \le \underbrace{\min\cbr{\frac{\lambda_d(\tilde\Phi) }{\lambda_1(\Sigma_{\cD})}, \frac{|\cB|}{\min_{t\in[T]}\lambda_2(\Xi^{t, \tilde\theta})}}}_{\ds Z_{\cD}} C_\eta^2 \cdot 
%         \frac 1 T\log\rbr{\frac{\cN(\Theta, 1/T)}{\delta}}, \label{eq:MLE-guarantee-offline}
% \end{align}
% where $C_\eta= {B_A}/\rbr{1-\exp\rbr{-\eta B_A}}$, $\Sigma_{\cD}=T^{-1}\cdot \sum_{t=1}^T\EE_{s^t}^{\nu^{\pi^t, \tilde\theta}}[\psi^{t, \tilde\theta} {\psi^{t, \tilde\theta}}^\top]$, $\psi^{t, \theta} =\phi^{t}(b) - \EE_{s^t}^{\nu^{\pi^t, \theta}}[\phi^t(b')]$, and 
% $\tilde\Phi^t=\int_{\cB}\phi^t(b)\phi^t(b)^\top \rd b$.
% Note if $\cB$ has infinitely many actions, the second term in $Z_{\cD}$ is meaningless and only the first term matters.
% If $\cB$ has finite actions, we define $\tilde\Phi^t=\sum_{b\in\cB}\phi^t(b)\phi^t(b)^\top$ and $\Xi^{t, \tilde\theta} = \diag(\nu^{\pi^t, \tilde\theta}(\cdot\given s^t)) - (\nu^{\pi^t, \tilde\theta}(\cdot\given s^t))(\nu^{\pi^t, \tilde\theta}(\cdot\given s^t))^\top$.

We have by \Cref{cor:response-diff-myopic} that 
\begin{align*}
    D_\TV\rbr{\nu(\cdot\given s), \tilde\nu(\cdot\given s)}
        &\le  \eta  \EE\sbr{\abr{(\tilde r^\pi(s, b) - r^\pi(s, b)) - \EE\bigsbr{\tilde r^\pi(s, b) - r^\pi(s, b)}}} \nend
        &\qquad + C^{(3)}\EE\sbr{\rbr{\rbr{\tilde r^\pi(s, b) -r^\pi(s, b)} - \EE\sbr{\tilde r^\pi(s, b) -r^\pi(s, b)}}^2}.
\end{align*}
% We define the weighted covariance matrix as
% \begin{align*}
%     \Sigma_{ s}^{\pi, \theta} \defeq \EE_{s}^{\pi, \theta} \sbr{\psi^{\pi,\theta}(s, b)\psi^{\pi,\theta}(s, b)^\top}\quad\text{where}\quad \psi^{\pi,\theta}(s, b) = \phi^{\pi}(s, b) - \EE_{s}^{\pi,\theta}\phi^{\pi}(s, \cdot).
% \end{align*}
% Under this definition, we introduce a nonnegative definite matrix $\Psi\in \SSS_+^{d}$ and write down the covariance term as
% \begin{align*}
%     &\EE\sbr{\rbr{\rbr{\tilde r^\pi(s, b) -r^\pi(s, b)} - \EE\sbr{\tilde r^\pi(s, b) -r^\pi(s, b)}}^2} \nend
%     &\quad = \bignbr{\theta^*-\tilde\theta}_{\Sigma_{ s}^{\pi, \tilde\theta}} = \nbr{\sqrt{\Psi}^{\dagger}\sqrt{\Psi}\rbr{\theta^*-\tilde\theta}}_{\Sigma_{ s}^{\pi, \tilde\theta}} \le \sqrt{\Bignbr{\Psi^{\dagger} \Sigma_{ s}^{\pi, \tilde\theta}}_\oper }\cdot \bignbr{\theta^*-\tilde\theta}_{\Psi} \le \sqrt{\trace\rbr{{\Psi}^{\dagger} \Sigma_{ s}^{\pi, \tilde\theta}}} \cdot \bignbr{\theta^*-\tilde\theta}_{\Psi}.
% \end{align*}
% where we recall from \Cref{lem:bandit} the definition of the weighted Laplacian of the comparison feature graph with respect to the offline data $\cD$ and parameter $\tilde\theta$ as $ \Sigma_\cD^{\theta}\defeq T^{-1} \sum_{t=1}^T
% \EE_{s^t}^{\pi^t, \theta}\bigsbr{\psi^{\pi^t, \theta}(s, b)\psi^{\pi^t, \theta}(s, b)^\top}$.
% Note that the operator norm is further bounded by the trace,
% \begin{align*}
%     \sqrt{\Bignbr{{\Psi}^{\dagger} \Sigma_{ s}^{\pi, \tilde\theta}}_\oper} 
%     &\le \sqrt{\trace\rbr{{\Psi}^{\dagger} \Sigma_{ s}^{\pi, \tilde\theta}}} \nend
%     &= \sqrt{\trace\rbr{\EE_{s}^{\pi, \tilde\theta}\sbr{ {\Psi}^\dagger \psi^{\pi,\tilde\theta}(s, b)\psi^{\pi,\tilde\theta}(s, b)^\top}}}\nend
%     & = \sqrt{\EE_{s}^{\pi, \tilde\theta}\sbr{\psi^{\pi,\tilde\theta}(s, b)^\top {\Psi}^\dagger  \psi^{\pi,\tilde\theta}(s, b)}}.
% \end{align*}
% We plug in the definition $\psi^{\pi,\tilde\theta}(s, b)=\phi^{\pi}(s, b) - \EE_{s}^{\pi,\tilde\theta}\phi^{\pi}(s, \cdot)$ and obtain
% \begin{align*}
%     &\sqrt{\EE_{s}^{\pi, \tilde\theta}\sbr{\psi^{\pi,\tilde\theta}(s, b)^\top {\Psi}^\dagger  \psi^{\pi,\tilde\theta}(s, b)}} = \underbrace{\sqrt{\EE_{s}^{{\pi,\tilde\theta}}\sbr{\phi^{\pi}(s,b)^\top {\Psi}^{\dagger}\phi^{\pi}(s,b)} -\bignbr{\EE_{s}^{{\pi,\tilde\theta}}\phi^{\pi}(s, b)}_{{\Psi}^{-1}}^2}}_{\ds \Upsilon_{s}^{\pi,\tilde\theta}}.
% \end{align*}
% Hence, we have for term (i) that  ${\dr (i)}\le 
% \Upsilon_{s}^{\pi,\tilde\theta}\cdot \bignbr{\theta^*-\tilde\theta}_{{\Psi}}$.
% % \begin{align}
% %     {\dr (i)} 
% %     &\le 
% %     \Upsilon_{s}^{\pi,\tilde\theta}\cdot \bignbr{\theta^*-\tilde\theta}_{{\Psi}}
% %     \le 2 \Upsilon_{s}^{\pi,\tilde\theta}\cdot \underbrace{\sqrt{Z_{\cD} C_\eta^2  \cdot \frac 1 T\log\rbr{\frac{\cN(\Theta, 1/T)}{\delta}} + \bignbr{\tilde\theta - \tilde\theta_{\MLE}}_{{\Psi}}^2}}_{\ds \zeta^{\tilde\theta}},\label{eq:myopic-(i)-ub}
% % \end{align}
% % on the success of \Cref{lem:bandit}. 
% % Here, the second inequality follows from the triangular inequaltiy, and the last inequality holds from \eqref{eq:MLE-guarantee-offline}.
% Now, we have addressed term (i) of \eqref{eq:TV-ub-offline} and it remains to show the upper bound for term (ii). Observe that under the quantal response model with logistic preference, term (ii) is nothing but just the squared KL divergence $\eta^{-2}\kl\infdivx[]{\nu^{\pi, \tilde\theta}(\cdot\given s)}{\nu^{\pi, \theta^*}(\cdot\given s)}^2$. Let $\Delta \theta=\tilde\theta - \theta^*$ and consider $\tilde\theta$ to be fixed. The Hessian of the KL divergence with respect to parameters in the second position is 
% \begin{align*}
%     \eta^{-2}\nabla_{\theta}^2 \kl\infdivx[]{\nu^{\pi, \tilde\theta}(\cdot\given s)}{\nu^{\pi, \theta}(\cdot\given s)} = \EE_s^{\pi,\theta} \sbr{\phi^\pi(s, b)\phi^\pi(s, b)^\top} - \EE_s^{\pi,\theta}\sbr{\phi^\pi(s, b)} \cdot \EE_s^{\pi,\theta}\sbr{\phi^\pi(s, b)}^\top = \Sigma_s^{\pi,\theta}.
% \end{align*}
% We observe that the KL divergence is strongly convex in the second parameter. Actually, for any test $x\in\RR^d$ and let $g(b)=\phi^\pi(s, \cdot)^\top x$, we have $x^\top \rbr{\eta^{-2}\nabla_{\theta}^2 \kl\infdivx[]{\nu^{\pi, \tilde\theta}(\cdot\given s)}{\nu^{\pi, \theta}(\cdot\given s)}} x = \Var^{\pi,\theta}[g(b)]$ and furthermore,
% \begin{align*}
%      \Var^{\pi,\theta}[g(b)]
%     \ge \exp(-2\eta B_A) \EE_s^{\pi, \tilde\theta}\sbr{\rbr{g(b) - \EE_s^{\pi,\theta}[g(b)]}^2} \ge \exp\rbr{-2\eta B_A} \Var^{\pi, \tilde\theta}[g(b)],
% \end{align*}
% which suggests that $\eta^{-2}\nabla_{\theta}^2 \kl\infdivx[]{\nu^{\pi, \tilde\theta}(\cdot\given s)}{\nu^{\pi, \theta}(\cdot\given s)}=\Sigma_s^{\pi,\theta}\succeq \exp(-2\eta B_A) \Sigma_s^{\pi,\tilde\theta}$. Actually, we can also swap $\theta$ and $\tilde\theta$ and obtain 
% \begin{align*}
%     \eta\exp(2\eta B_A) \Sigma_s^{\pi,\tilde\theta} \succeq \eta^{-}\nabla_{\theta}^2 \kl\infdivx[]{\nu^{\pi, \tilde\theta}(\cdot\given s)}{\nu^{\pi, \theta}(\cdot\given s)} \succeq \eta\exp(-2\eta B_A) \Sigma_s^{\pi,\tilde\theta} , \quad \forall \theta\in\Theta.
% \end{align*}
% Therefore, we can bound term (ii) simply by
% \begin{align*}
%     {\dr(ii)}\le \frac{\eta^2}{4} \exp\rbr{4\eta B_A} \bignbr{\theta^*-\tilde\theta}_{\Sigma_s^{\pi,\tilde\theta}}^4 \le  \frac{\eta^2}{4} \exp\rbr{4\eta B_A} \rbr{\Upsilon_{s}^{\pi,\tilde\theta}\cdot \bignbr{\theta^*-\tilde\theta}_{{\Psi}}}^4,
% \end{align*}
% where the last inequality holds by the same upper bound for term (i).
% Combining our results for both term (i) (ii), we have for \eqref{eq:TV-ub-offline} that
% \begin{align*}
%     &D_\TV\rbr{\nu^{\pi,\theta^*}(\cdot\given s) , \nu^{\pi,\tilde\theta}(\cdot\given s)} \nend
%     &\quad\le \eta \cdot {\dr(i)} + \frac{\eta^2\exp\rbr{2\eta B_A}}{2} \cdot \rbr{{\dr(i)}^2 + {\dr(ii)}}\nend
%     &\quad\le \eta \Upsilon_{s}^{\pi,\tilde\theta} \bignbr{\theta^*-\tilde\theta}_{{\Psi}} + \frac{\eta^2\exp\rbr{2\eta B_A}}{2} \rbr{\rbr{\Upsilon_{s}^{\pi,\tilde\theta} \bignbr{\theta^*-\tilde\theta}_{{\Psi}}}^2 + \frac{\eta^2}{4} \exp\rbr{4\eta B_A} \rbr{\Upsilon_{s}^{\pi,\tilde\theta} \bignbr{\theta^*-\tilde\theta}_{{\Psi}}}^4}.
% \end{align*}
% Now, we let $\Gamma^{(2)}(s;\pi_{s}, \tilde\theta) = \Upsilon_{s}^{\pi,\tilde\theta} \bignbr{\theta^*-\tilde\theta}_{{\Psi}} $ and 
% \begin{align*}
%     f(x) = \eta x + \frac{\eta^2\exp(2\eta B_A)}{2} x^2 + \frac{\eta^4\exp(6\eta B_A)}{8} x^4, 
% \end{align*}
% and just conclude that $D_\TV\bigrbr{\nu^{\pi_{s},\theta^*}(\cdot\given s) , \nu^{\pi_{s},\tilde\theta}(\cdot\given s)} \le f(\Gamma^{(2)}(s;\pi_{s}, \tilde\theta))$, 
% which completes the proof of \Cref{lem:response diff-myopic}. 
% % One can swap $\theta^*$ and $\$


% % \subsection{Follow up Discussion on \Cref{thm:PMLE-VI-myopic}}\label{sec:follow up on myopic offline}
% % \todo{comment on the use of the Laplacian, including why a standard $\sum\phi\phi^\top$ would fail. Also, comment on the use of $L_\cD$. }

% \subsubsection{Proof of \Cref{lem:MLE-formal}}
% \label{sec:proof-MLE-general}
% % The key idea is replacing the negative log likelihood $\cL^{(1)}_\cD(\cdot)$ with some curve with constant Hessian $\EE_{\nu^{\pi^t, \theta^*}}[\psi^{t, \theta^*} {\psi^{t, \theta^*}}^\top]$ in the neighbourhood of $\theta^*$. Specifically, we first lower bound $\cL^{(1)}_\cD(\hat\theta_\MLE)-\cL^{(1)}_\cD(\theta^*)$ by the Hellinger distance $T^{-1}\cdot\sum_{t=1}^T D_\H^2(\nu^{\pi^t, \hat\theta_\MLE}, \nu^{\pi^t, \theta^*})$ and some $\cO(T^{-1})$ term. Then, a careful scrutiny of the Hellinger distance will show that $$D_\H^2(\nu^{\pi^t, \theta}, \nu^{\pi^t, \theta^*})\ge B_A^{-2} (\theta-\theta^*)^\top\EE^{\nu^{\pi^t, \theta^*}}[\psi^{t, \theta^*} {\psi^{t, \theta^*}}^\top](\theta-\theta^*).$$
% To make the above discussion rigorous, we first invoke the following concentration Lemma.
\begin{proof} 
We prove this lemma by leveraging 
 \Cref{lem:freeman-variation} with $X_t = (-\cL_{h,t}(\theta) + \cL_{h, t}(\theta^*))/2$ where $\cL_{h}^t(\theta)=-\sum_{i=1}^{t-1} \eta A_h^{\pi^i, \theta}(s_h^i,b_h^i)$. 
We choose filtration $\cF_{h, t-1}=\sigma(\tau^{1:t-1})$ where $\sigma(X)$ denotes the $\sigma$-algebra generated by $X$ and $\tau^{1:t-1}$ is just the history up to $t-1$.
Let $\cN_\rho(\Theta, \epsilon)$ be  the covering number for the $\epsilon$-covering net of $\Theta$  with respect to norm $\rho$ defined in \eqref{eq:rho for MLE}. 
Let $\Theta_\epsilon$ be  the $\epsilon$-covering net of $\Theta$. 
To simplify the notation, we define  $\iota =  \log\rbr{ H\cN_\rho(\Theta, \epsilon) / \delta}$.
Then, for all $\theta \in \Theta_{\epsilon}$, 
we have with probability $1-\delta$ for all $h\in[H], t\in[T]$ that
\begin{align}
    &\frac 1 2 \rbr{-\cL_h^t(\theta) + \cL_h^t(\theta^*)} \nend
    &\quad\le \sum_{i=1}^{t-1} \log\EE^{\pi^i   }\sbr{\sqrt{ \nu_h^{\pi^i,\theta}(\cdot\given s_h) \big / \nu_h^{\pi^i, \theta^*}(\cdot\given s_h)}}  + \log\rbr{ H\cN_\rho(\Theta, \epsilon) / \delta} \nend
    &\quad\le -  \sum_{i=1}^{t-1} \EE^{\pi^i }\Bigl [ D_\H^2\bigrbr{\nu_h^{\pi^i, \theta}(\cdot\given s_h ), \nu_h^{\pi^i, \theta^*}(\cdot\given s_h )} \Bigr ] + \log\rbr{ H\cN_\rho(\Theta, \epsilon) / \delta}, \label{eq:MLE-to-hellinger-1}
\end{align}
where the expectation is taken with respect to the true model. 
Here, the first inequality holds by applying  \Cref{lem:freeman-variation} and   taking a union bound over the  $\epsilon$-covering net. 
% Note that $\log(\cN_\rho(\Theta, \epsilon))$ only grows with 
% $\log(\eta T)$ since $\cL_{h,t}(\theta)$ is $2\eta$-Lipschitz with respect to $\theta$. 
The second inequality holds by noting that $\log(x)\le x-1$ and by the definition of the Hellinger distance.


Meanwhile, by the definition of the distance $\rho$ in   \eqref{eq:rho for MLE},
for any $\theta, \tilde \theta \in \Theta$, we have 
\begin{align*}
&|D_\H^2(\nu_h^{\pi,\theta}, \nu_h^{\pi, \theta^*}) - D_\H^2(\nu_h^{\pi,\tilde\theta}, \nu_h^{\pi, \theta^*})|\nend
&\quad \le (D_\H^2(\nu_h^{\pi,\theta}, \nu_h^{\pi, \theta^*}) + D_\H^2(\nu_h^{\pi,\tilde\theta}, \nu_h^{\pi, \theta^*})) \cdot 
\bigabr{D_\H^2(\nu_h^{\pi,\theta}, \nu_h^{\pi, \theta^*}) - D_\H^2(\nu_h^{\pi,\tilde\theta}, \nu_h^{\pi, \theta^*})}\nend
&\quad \le 2 D_\H(\nu_h^{\pi,\tilde\theta}, \nu_h^{\pi,\theta})\nend
&\quad \le 2\rho(\theta, \tilde\theta), 
\end{align*} 
where  the second inequality holds by noting that the Hellinger distance does not exceed 1, and that the Hellinger distance satisfies the triangle inequality as a norm, and the last inequality holds by definition of $\rho$. 
Moreover, by noting that $\cL_h^t(\theta)=-\sum_{i=1}^t \eta A_h^{\pi^i, \theta}(s_h^i,b_h^i)$, we have
\begin{align*}
\bigl | \cL_h^t (\theta ) - \cL_h^t(\tilde \theta ) \bigr |
&\le \eta T \max_{i\in[t-1]}\bigabr{ A_h^{\pi^i, \theta}(s_h^i,b_h^i) - A_h^{\pi^i, \tilde\theta}(s_h^i,b_h^i)}\nend
&\le 2\eta T \max_{i\in[t-1]}\bignbr{ Q_h^{\pi^i, \theta}- Q_h^{\pi^i, \tilde\theta}}_\infty \nend
&\leq 2 T \cdot \rho(\theta, \tilde \theta),
\end{align*}
where the second inequality holds by noting that $|(V_h^{\pi, \theta} - V_h^{\pi, \tilde\theta})(s_h)| \le \onbr{Q_h^{\pi,\theta}-Q_h^{\pi,\tilde\theta}}_\infty$ by the same argument in \eqref{eq:f_2-01} and \eqref{eq:f_2-1}, and the last inequality holds by noting that $\eta\le (\gamma B_A + 1 + \eta)$.
% \Zhuoran{Derive the second one, STOP HERE}
Therefore, 
adding an extra term $3T\epsilon$ to the right-hand side of \eqref{eq:MLE-to-hellinger-1} extends the result to any $\theta\in\Theta$ by definition of the covering net $\Theta_\epsilon$.
We thus obtain for all $\theta\in\Theta, h\in[H], t\in[T]$ with probability $1-\delta$,
\begin{align}
    \frac 1 2 \rbr{-\cL_h^t(\theta) + \cL_h^t(\theta^*)} 
    &\le -  \sum_{i=1}^{t-1} \EE^{\pi^i, \theta^*}D_\H^2\rbr{\nu_h^{\pi^i, \theta}(\cdot\given s_h^i), \nu_h^{\pi^i, \theta^*}(\cdot\given s_h^i)} \nend
    &\qquad + \log\rbr{ H\cN_\rho(\Theta, \epsilon) / \delta} + 3T\epsilon. \label{eq:KL-2-D_H-online}
\end{align}
% As an alternative, we can also take the filtration as $\cF_{h, t-1} = \sigma(\{s_{h}^i \}_{i\in[t]}, \{b_h^i\}_{i\in[t-1]})$ and obtain a similar result
% \begin{align}
%     \frac 1 2 \rbr{-\cL_h^t(\theta) + \cL_h^t(\theta^*)} 
%     &\le - \sum_{i=1}^{t-1} D_\H^2\rbr{\nu_h^{\pi^i, \theta}(\cdot\given s_h^i), \nu_h^{\pi^i, \theta^*}(\cdot\given s_h^i)} + \iota. \label{eq:KL-2-D_H-offline}
% \end{align}
In the sequel, we take $\epsilon=T^{-1}$ and  let $\iota= \log\rbr{ H\cN_\rho(\Theta, T^{-1}) / \delta}+ 3$.
Now, we plug in $\hat\theta_{h,\MLE}=\argmin_{\theta'\in\Theta} \cL_h^t(\theta')$ in \eqref{eq:KL-2-D_H-online}
%  and \eqref{eq:KL-2-D_H-offline},  
and obtain by the nonnegativity of the Hellinger distance that
\begin{align*}
    \cL_h^t(\theta^*) \le \inf_{\theta'\in\Theta} \cL_h^t(\theta' ) + 2\iota \le  \cL_h^t(\hat\theta_{h,\MLE}) + 2\iota  , 
\end{align*}
which guarantees that our confidence set is indeed valid by letting 
$$\beta \ge  2\iota = 2\log(e^3 H\cdot \cN_\rho(\Theta, T^{-1})/\delta). $$
Next, we show that our confidence set is also accurate.
For \eqref{eq:KL-2-D_H-online}, we have that
\begin{align}
    \sum_{i=1}^{t-1} \EE^{\pi^i, \theta^*}D_\H^2\rbr{\nu_h^{\pi^i, \theta}(\cdot\given s_h^i), \nu_h^{\pi^i, \theta^*}(\cdot\given s_h^i)} 
    &\le  \frac 1 2\rbr{\cL_h^t(\theta) - \cL_h^t(\theta^*)} +\iota\nend
    &\le \frac 1 2\rbr{\cL_h^t(\theta) - \cL_h^t(\theta^*)+\beta}, \label{eq:MLE-to-hellinger-2}
\end{align}
% and also 
% \begin{align*}
%     \sum_{i=1}^{t-1} D_\H^2\rbr{\nu_h^{\pi^i, \theta}(\cdot\given s_h^i), \nu_h^{\pi^i, \theta^*}(\cdot\given s_h^i)} \le  \frac 1 2\rbr{\cL_h^t(\theta) - \cL_h^t(\theta^*)} + \log\rbr{\frac{e^2H\cN_\rho(\Theta, T^{-1})}{\delta}}, 
% \end{align*}
Now, if $\theta\in\CI_{h,\Theta}^t(\beta)$, it follows directly from \eqref{eq:MLE-to-hellinger-2} that
\begin{align*}
    \sum_{i=1}^{t-1} \EE^{\pi^i, \theta^*}D_\H^2\rbr{\nu_h^{\pi^i, \theta}(\cdot\given s_h^i), \nu_h^{\pi^i, \theta^*}(\cdot\given s_h^i)} \le \beta, 
\end{align*}
which shows that our confidence set is also valid and gives \eqref{eq:MLE-guarantee-hellinger-2}.
%  since the righthand side can be replaced by $3/2 \beta$ using the fact that $\cL_h^t(\theta) - \cL_h^t(\theta^*)\le \cL_h^t(\theta) - \inf_{\theta'\in\Theta}\cL_h^t(\theta')\le \beta$  for any $\theta\in\CI_\Theta(\beta)$.
We next show how to derive the bound for the Q function. Invoking \Cref{lem:D_H-2-A^2}, we have that
\begin{align}
    8 D_\H^2\rbr{\nu_h^{\pi, \hat\theta}(\cdot\given s_h), \nu_h^{\pi, \theta^*}(\cdot\given s_h)} 
    &\ge \rbr{\frac{\eta}{1+\eta B_A}}^2\cdot \inp[\Big]{\nu_h^{\pi, \theta^*}}{ \orbr{A_h^{\pi, \hat\theta}-A_h^{\pi, \theta^*}}^2} \nend
    &\ge \rbr{\frac{\eta}{1+\eta B_A}}^2\cdot \EE_{s_h}^{\pi, \theta^*}\rbr{\bigrbr{\EE_{s_h, b_h}^{\pi, \theta^*} - \EE_{s_h}^{\pi, \theta^*}}\osbr{A_h^{\pi, \hat\theta}-A_h^{\pi, \theta^*}}}^2\nend
    &= \rbr{\frac{\eta}{1+\eta B_A}}^2\cdot \EE_{s_h}^{\pi, \theta^*}\Bigrbr{\orbr{\EE_{s_h, b_h}^{\pi, \theta^*} - \EE_{s_h}^{\pi, \theta^*}}\osbr{Q_h^{\pi, \hat\theta}-Q_h^{\pi, \theta^*}}}^2, \label{eq:hellinger-2-Q}
\end{align}
where the second inequality follows from the Jensen's inequality, and the last inequality holds by invoking \eqref{eq:A diff-1} in \Cref{lem:AQV-func diff}. One can also swap $\theta^*$ and $\hat\theta$ by the exchangeability of the Hellinger distance and obtain another version.  Note that $C_\eta = \eta^{-1}+B_A$. 
Plugging \eqref{eq:hellinger-2-Q} with $C_\eta$ into the previous accuracy guarantees gives \eqref{eq:MLE_guarantee_Q}.
%  and \eqref{eq:MLE_guarantee_Q-1}. 
Therefore, we have proved \eqref{eq:MLE-guarantee-hellinger-2} and \eqref{eq:MLE_guarantee_Q}. 
For deriving \eqref{eq:MLE-guarantee-hellinger-1} and \eqref{eq:MLE_guarantee_Q-1}, we just change our filtration to $\cF_{h, t-1} = \sigma((s_h^i, b_h^i)_{i\in[t-1]}, s_h^t)$ and everything follows. 

Lastly, we prove the guarantee in \eqref{eq:MLE-guarantee-Q-3}. 
We use \eqref{eq:MLE_guarantee_Q-1} with $\theta'$ replaced by $\theta^*$ and for all $\theta\in\CI_{h, \Theta}^t(\beta)$, 
\begin{align*}
    \sum_{i=1}^{t-1} {\Var_{s_h^i}^{\pi^i, \theta^*} \bigsbr{Q_h^{\pi^i, \theta}(s_h, b_h) - Q_h^{\pi^i, \theta^*}(s_h, b_h)}} \le 4 C_\eta^2 \rbr{\cL_h^t(\theta) - \cL_h^t(\theta^*)+ \beta} \le 8 C_\eta^2 \beta, 
    % \label{eq:MLE_guarantee_Q-1} 
\end{align*}
which is equivalent to saying that for all $h\in[H], \theta\in\CI_{h, \Theta}^t(\beta)$,
\begin{align}
    \sum_{i=1}^{t-1} 
    \EE_{s_h^i}^{\pi^i,\theta^*}\rbr{\rbr{Q_h^{\pi^i, \theta} - Q_h^{\pi^i, \theta*}}(s_h^i, b_h^i)
    -\EE_{s_h^i}^{\pi^i,\theta^*}\sbr{ \rbr{Q_h^{\pi^i, \theta} - Q_h^{\pi^i, \theta^*}}(s_h, b_h)}}^2 \le 8 C_\eta^2 \beta. \label{eq:MLE-Q-ub-1}
\end{align}
Recall the covering net $\Theta_\epsilon$ we constructed before. For all $\theta\in\Theta_\epsilon \cap \CI_{h,\Theta}^t(\beta), h\in[H]$, and a given $t\in[T]$, we have  by a standard martingale concentration in \Cref{cor:martigale concentration} that with probability at least $1-2\delta$,
\begin{align*}
    &\sum_{i=1}^{t-1} 
    \rbr{\rbr{Q_h^{\pi^i, \theta} - Q_h^{\pi^i, \theta^*}}(s_h^i, b_h^i)
    -\EE_{s_h^i}^{\pi^i,\theta^*}\sbr{ \rbr{Q_h^{\pi^i, \theta} - Q_h^{\pi^i, \theta^*}}(s_h, b_h)}}^2 
    \nend
    &\quad \le \frac 3 2\sum_{i=1}^{t-1}\EE_{s_h^i}^{\pi^i,\theta^*} 
    \rbr{\rbr{Q_h^{\pi^i, \theta} - Q_h^{\pi^i, \theta^*}}(s_h^i, b_h)
    -\EE_{s_h^i}^{\pi^i,\theta^*}\sbr{ \rbr{Q_h^{\pi^i, \theta} - Q_h^{\pi^i, \theta^*}}(s_h, b_h)}}^2 \nend
    &\qqquad + 32 B_A^2 \log\rbr{2H \cN_\rho(\Theta, \epsilon)\delta^{-1} }\nend
    &\quad \le 12 C_\eta^2 \beta  +32 B_A^2 \log\rbr{2H \cN_\rho(\Theta, \epsilon)\delta^{-1} }\le 28 C_\eta^2 \beta 
\end{align*}
where the second inequality follows from \eqref{eq:MLE-Q-ub-1}. 
Moreover, we note that
\begin{align*}
    &\rbr{\rbr{Q_h^{\pi^i, \theta} - Q_h^{\pi^i, \theta^*}}(s_h^i, b_h)
    -\EE_{s_h^i}^{\pi^i,\theta^*}\sbr{ \rbr{Q_h^{\pi^i, \theta} - Q_h^{\pi^i, \theta^*}}(s_h, b_h)}}^2 \nend
    &\qqquad - \rbr{\rbr{Q_h^{\pi^i, \tilde\theta} - Q_h^{\pi^i, \theta*}}(s_h^i, b_h)
    -\EE_{s_h^i}^{\pi^i,\theta^*}\sbr{ \rbr{Q_h^{\pi^i, \tilde\theta} - Q_h^{\pi^i, \theta^*}}(s_h, b_h)}}^2\nend
    &\quad \le 8 \cdot \max_{\pi\in\Pi,\theta\in\Theta}\bignbr{Q_h^{\pi,\theta}}_\infty
    \cdot 2 \bignbr{Q_h^{\pi^i, \theta} -  Q_h^{\pi^i, \tilde\theta}}_\infty \le 16 B_A \rho(\theta, \tilde\theta),
\end{align*}
where the last inequality holds by noting that $B_A$ upper bounds $\max_{\theta\in\Theta, \pi\in\Pi, h\in[H]}\onbr{Q_h^{\pi, \theta}}_\infty$ and using the definition of $\rho$. 
% and $\cN_\varrho (\Theta, \epsilon)$ is the covering number of the smallest $\epsilon$-covering number of $\Theta$ with respect to distance 
% \begin{align*}
%     \varrho(\theta, \tilde\theta) \defeq \max_{\pi\in\Pi, h\in[H]}\nbr{Q_h^{\pi, \theta} - Q_h^{\pi, \tilde\theta}}_\infty.
% \end{align*}
As a result, for all $\theta\in\CI_\Theta(\beta), h\in[H]$, and a given $t\in[T]$, we conclude that with probability at least $1-2\delta$,
\begin{align*}
    &\sum_{i=1}^{t-1} 
    \rbr{\rbr{Q_h^{\pi^i, \theta} - Q_h^{\pi^i }}(s_h^i, b_h^i)
    -\EE_{s_h^i}^{\pi^i,\theta^*}\sbr{ \rbr{Q_h^{\pi^i, \theta} - Q_h^{\pi^i, \theta^*}}(s_h, b_h)}}^2  \le 28 C_\eta^2 \beta  + 16 B_A.
\end{align*}
Replace $\delta$ by $\delta/2$, we have for all $\theta\in\CI_\Theta(\beta), h\in[H]$ and a given $t\in[T]$ with probability at least $1-\delta$ that 
\begin{align*}
    &\sum_{i=1}^{t-1} 
    \rbr{\rbr{Q_h^{\pi^i, \theta} - Q_h^{\pi^i }}(s_h^i, b_h^i)
    -\EE_{s_h^i}^{\pi^i,\theta^*}\sbr{ \rbr{Q_h^{\pi^i, \theta} - Q_h^{\pi^i, \theta^*}}(s_h, b_h)}}^2 
    \nend
    &\quad \le 28 C_\eta^2 \beta + 56 C_\eta^2\log 2 +16 B_A =  \cO(C_\eta^2 \beta).
\end{align*}
which 
finishes the proof of \Cref{lem:MLE-formal}.
\end{proof}

% \subsubsection{Proof of \Cref{lem:MLE-indep-data}}\label{sec:proof-MLE-indep-data}
% 
Here, we show the guarantee for the MLE with independently collected dataset.
Since each trajectory is independently collected, we are able to use the Bernstein inequality for indepedent random variables $Z_h^i = D_\H^2\orbr{\nu_h^{\pi^i, \theta}(\cdot\given s_h^i), \nu_h^{\pi^i, \theta^*}(\cdot\given s_h^i)}$, 
\#
    \abr{\frac 1 T \sum_{i=1}^T Z_h^i - \EE_\cD\sbr{Z_h^i}} &\le \sqrt{\frac{4\sum_{i=1}^T\Var[Z_h^i]\log(2\delta^{-1})}{T^2}} + \frac{4\log(2\delta^{-1})}{3T} \nend
    &\le \sqrt{\frac{4\sum_{i=1}^T \EE_\cD[Z_h^i]\log(2\delta^{-1})}{T^2}} + \frac{4\log(2\delta^{-1})}{3T}\nend
    &\le \frac{1}{2T}\sum_{i=1}^T \EE_\cD [Z_h^i] + \frac{2 \log(2\delta^{-1})}{T} + \frac{4\log(2\delta^{-1})}{3T},\notag
\#
where the second inequality holds by noting that $\Var[Z_h^i] \le \EE_\cD[(Z_h^i)^2]\le \EE_\cD[Z_h^i]$ by using the property that Hellinger distance is always upper bounded by $1$. We now conclude by further taking a union bound over $h\in[H]$ and $\theta\in\Theta$ that
\#
\frac{1}{T} \sum_{i=1}^T \EE_\cD[Z_h^i] &\le \frac{2}{T} \sum_{i=1}^T Z_h^i  + \frac{8\log(2H \cN_\rho(\Theta, \epsilon)\delta^{-1})}{T} + 6\epsilon\nend
&\le \frac{2}{T} \sum_{i=1}^T Z_h^i  + \frac{8\log(2eH \cN_\rho(\Theta, T^{-1})\delta^{-1})}{T},\notag
\#
where the last inequality holds by taking $\epsilon=T^{-1}$. Plug in the definition of $Z_h^i$, we have
\begin{align*}
    \frac 1 T\sum_{i=1}^T\EE_\cD\sbr{D_\H^2\rbr{\nu_h^{\pi^i, \theta}(\cdot\given s_h^i), \nu_h^{\pi^i, \theta^*}(\cdot\given s_h^i)}} &\le \frac{2}{T} \sum_{i=1}^T D_\H^2\rbr{\nu_h^{\pi^i, \theta}(\cdot\given s_h^i), \nu_h^{\pi^i, \theta^*}(\cdot\given s_h^i)}  \nend
    &\qquad + \frac{8\log(2eH \cN_\rho(\Theta, T^{-1})\delta^{-1})}{T}.
\end{align*}
Using \eqref{eq:MLE-guarantee-hellinger-1} in \Cref{lem:MLE-formal} for any $\theta\in\cC_{\Theta}(\beta)$, we give 
\begin{align}
    \sum_{i=1}^{T} D_\H^2\bigrbr{\nu_h^{\pi^i, \theta}(\cdot\given s_h^i), \nu_h^{\pi^i, \theta^*}(\cdot\given s_h^i)} 
        %%%%%%%%%%
        &\le  \frac 1 2\rbr{\cL_h(\theta) - \cL_h(\theta^*)} + \log\rbr{\frac{eH\cN_\rho(\Theta, T^{-1})}{\delta}} \nend
        %%%%%%%%%
        &\le \frac 1 2\rbr{\cL_h(\theta) - \inf_{\theta'\in\Theta}\cL_h(\theta')} + \log\rbr{\frac{eH\cN_\rho(\Theta, T^{-1})}{\delta}}\nend
        &\le \frac 3 2 \beta, \label{eq:OffGM-nu-hellinger-1}
\end{align}
where the last inequality is just by definition of $\CI_\Theta(\beta)$ in \Cref{eq:behavior_model_confset-1}. Therefore, 
\begin{align*}
    \sum_{i=1}^T\EE_\cD\sbr{D_\H^2\rbr{\nu_h^{\pi^i, \theta}(\cdot\given s_h^i), \nu_h^{\pi^i, \theta^*}(\cdot\given s_h^i)}}  \le 3\beta + {8\log(2eH \cN_\rho(\Theta, T^{-1})\delta^{-1})} \le 11\beta, 
\end{align*}
with probability at least $1-2\delta$ for all $h\in[H]$ and $\theta\in\CI_\Theta(\beta)$. The validity guarantee is already shown in \Cref{lem:MLE-formal}.
We complete the proof of \Cref{lem:MLE-indep-data}.

% \Cref{thm:11_6_gyorfi}, where we take $Z_h^i = D_\H^2\rbr{\nu_h^{\pi^i, \theta}(\cdot\given s_h^i), \nu_h^{\pi^i, \theta^*}(\cdot\given s_h^i)}$ and take $g(Z)=Z$. One can verify that $g\in[0, 1]$ with covering number $\cN_\infty(\epsilon, \cG)=1$.
% Hence, we conclude with $\epsilon = 1/3$, $\alpha=120\log(4/\delta) T^{-1}$ that for any fixed $\theta\in\Theta$, 
% \begin{align*}
%     \PP\rbr{\frac 1 T \sum_{i=1}^T Z_h^i > 2 \EE_\cD Z_h^i  + \frac{60\log(4/\delta)}{T}} \le \delta,
% \end{align*}
% Now, we take a union bound over the $\epsilon$-covering net for $\Theta$ with respect to $\rho$ and also over $h\in[H]$ and obtain with probability at least $1-\delta$ that for any $\theta\in\Theta$, $h\in[H]$ 
% \begin{align*}
%     \frac 1 T \sum_{i=1}^T D_\H^2\rbr{\nu_h^{\pi^i, \theta}(\cdot\given s_h^i), \nu_h^{\pi^i, \theta^*}(\cdot\given s_h^i)} \le 2 \EE_\cD D_\H^2\rbr{\nu_h^{\pi, \theta}(\cdot\given s_h), \nu_h^{\pi, \theta^*}(\cdot\given s_h)} + \frac{1+ 60\log(4\cN_\rho(\Theta, T^{-1})/\delta)}{T}, 
% \end{align*}
% where we can use the same covering number for $\Theta$ since $\rho(\theta, \tilde\theta)$ can still bound the difference in the squared Hellinger distance. Here, the expectation on the right hand side is taken with respect to both the randomness in $\pi$ and $s_h$.

% \subsubsection{Proof of \Cref{lem:leader-bellman-loss}}\label{sec:proof-leader-bellman-loss}
% In the following proof, we always consider the expectation to be taken with respect to the data generating distribution.
We first prove the following concentration result: for any $h\in[H]$ and any $y=(\theta_{h+1}, \pi_{h+1}, U_{h+1}, U_{h})\in \cY_h = \Theta_{h+1}\times \Pi_{h+1}\times \cU^2$, 
it holds with probability at least $1-\delta$ that 
      \begin{align}
        &\abr{T\EE[(U_{h} - \TT_{h}^{\pi,\theta}U_{h + 1})^2] - \ell_{h}(U_{h}, U_{h + 1}, \theta, \pi) + \ell_{h}(\TT_{h}^{\pi,\theta}U_{h + 1}, U_{h + 1}, \theta, \pi)} \notag\\
        &\qquad \le \epsilon_S + \frac{T}{2} \EE[(U_{h} - \TT_{h}^{\pi,\theta}U_{h + 1})^2].\label{eq:cY-confset-1}
      \end{align}
     where
    $
    {110 B_U^2\cdot\log(H \max_{h\in[H]}\cN_\rho(\cY_h, T^{-1})\delta^{-1}) } \cdot {T^{-1}} + (45 B_U^2 + 60 B_U )T^{-1}
    $ and $B_U=H$ is the upper bound for the function class $U$.

\paragraph{Concentration. }
Our proof is an adaptation of Lemma D.2 in \citep{lyu2022pessimism}, although we simplify a little bit by directly using the covering number for a joint class $\cY_h = \Pi_{h+1}\times\Theta_{h+1}\times\cU^2$. We take an $\epsilon$-covering net $\cY_\epsilon$ for $\cY$ with respect to distance $\rho$ specified by \eqref{eq:rho-cY}. 
We first use \Cref{lem:bernstein},  
where we take 
\begin{align*}
    {Z_h^i} &= \ell_{h}(U'_{h}, U_{h + 1}, \theta, \pi) - \ell_{h}(\TT_{h}^{\pi,\theta}U_{h + 1}, U_{h + 1}, \theta, \pi)\nend
    & = \rbr{U_h(s_h^i, a_h^i, b_h^i) - u_h^i -  T_{h+1}^{\pi,\theta} U_{h+1}(s_{h+1}^i)}^2  - \rbr{\TT_h^{\pi, \theta} U_{h+1}(s_h^i, a_h^i, b_h^i) - u_h^i - T_{h+1}^{\pi,\theta} U_{h+1}(s_{h+1}^i)}^2.
\end{align*}
Here, we recall the definition of $\TT_h^{\pi}$ given by \eqref{eq:bellman_operator_leader}.
One can verify that $|{Z_h^i}|\le 9B_U^2$ where $B_U$ bounds both the leader's reward and the value function class $\cU$.
We first calculate the expectation of $Z_h^i$, 
\begin{align*}
    \EE[{Z_h^i}] 
    &= \EE \bigg[\EE_{s_h^i, a_h^i, b_h^i} \Big[\rbr{U_h(s_h^i, a_h^i, b_h^i) - u_h^i -  T_{h+1}^{\pi,\theta} U_{h+1}(s_{h+1}^i)}^2  \nend
    &\qquad - \rbr{\TT_h^{\pi, \theta} U_{h+1}(s_h^i, a_h^i, b_h^i) - u_h^i - T_{h+1}^{\pi,\theta} U_{h+1}(s_{h+1}^i)}^2\Big]\bigg]\nend
    & = \EE\bigg[\Bigrbr{U_h(s_h^i, a_h^i, b_h^i)- \TT_h^{\pi, \theta} U_{h+1}(s_h^i, a_h^i, b_h^i)}\nend
    &\qquad \cdot \EE_{s_h^i, a_h^i, b_h^i}\sbr{U_h(s_h^i, a_h^i, b_h^i) + \TT_h^{\pi, \theta} U_{h+1}(s_h^i, a_h^i, b_h^i)- 2u_h^i - 2T_{h+1}^{\pi,\theta} U_{h+1}(s_{h+1}^i) }\bigg]\nend
    & = \EE\sbr{\Bigrbr{U_h(s_h^i, a_h^i, b_h^i)- \TT_h^{\pi, \theta} U_{h+1}(s_h^i, a_h^i, b_h^i)}^2}, 
\end{align*}
where $\EE_{x}[\cdot]$ is a short hand of $\EE[\cdot\given x]$ and the expectation is taken with respect to the data generating distribution. Here, the second equality holds by the law of total expectation, and the third equality holds by noting that $\EE_{s_h^i, a_h^i, b_h^i}\osbr{u_h^i  + T_h^{\pi,\theta} U_{h+1}(s_{h+1}^i)} = \TT_h^{\pi, \theta} U_{h+1} (s_h^i, a_h^i, b_h^i)$.
Next, we calculate the variance, 
\begin{align*}
    \Var[{Z_h^i}] &\le \EE[{Z_h^i}^2] \nend
    &\le \EE\bigg[\Bigrbr{U_h(s_h^i, a_h^i, b_h^i)- \TT_h^{\pi, \theta} U_{h+1}(s_h^i, a_h^i, b_h^i)}^2\nend
    &\qquad \cdot \EE_{s_h^i, a_h^i, b_h^i}\sbr{\rbr{U_h(s_h^i, a_h^i, b_h^i) + \TT_h^{\pi, \theta} U_{h+1}(s_h^i, a_h^i, b_h^i)- 2u_h^i - 2T_{h+1}^{\pi,\theta} U_{h+1}(s_{h+1}^i)}^2 } \bigg]\nend
    &\le 49 B_U^2 \EE\sbr{\Bigrbr{U_h(s_h^i, a_h^i, b_h^i)- \TT_h^{\pi, \theta} U_{h+1}(s_h^i, a_h^i, b_h^i)}^2} = 49 B_U^2 \EE[{Z_h^i}].
\end{align*}
Now, by \Cref{lem:bernstein}, we have for each $y\in\cY_\epsilon$ that
\begin{align*}
    \abr{\frac 1 T \sum_{i=1}^T {Z_h^i} - \EE\sbr{{Z_h^i}}} &\le \frac{1}{2T}\sum_{i=1}^T \EE [{Z_h^i}] + \frac{110 B_U^2\cdot\log(2\delta^{-1})}{T}.
\end{align*}
Now, we extend the result to $\cY$, where we notice that for any two $y, \tilde y\in\cY$ such that $\rho(y, \tilde y) \le \epsilon$, 
\begin{align*}
    &\bigrbr{U_h(s_h^i, a_h^i, b_h^i) - u_h^i -  T_{h+1}^{\pi,\theta} U_{h+1}(s_{h+1}^i)}^2  - \bigrbr{\TT_h^{\pi, \theta} U_{h+1}(s_h^i, a_h^i, b_h^i) - u_h^i - T_{h+1}^{\pi,\theta} U_{h+1}(s_{h+1}^i)}^2 \nend
    &\qquad - \rbr{\bigrbr{\tilde U_h(s_h^i, a_h^i, b_h^i) - u_h^i -  T_{h+1}^{\tilde\pi,\tilde\theta} \tilde U_{h+1}(s_{h+1}^i)}^2  - \bigrbr{\TT_h^{\tilde\pi, \tilde\theta} \tilde U_{h+1}(s_h^i, a_h^i, b_h^i) - u_h^i - T_{h+1}^{\tilde\pi,\tilde\theta} \tilde U_{h+1}(s_{h+1}^i)}^2}\nend
    %%%%%%%%%%%%%
    &\quad \le 6 B_U \rbr{\onbr{U-\tilde U}_\infty + \bignbr{(T_{h+1}^{\pi,\theta}-T_{h+1}^{\tilde\pi,\tilde\theta} )\tilde U_{h+1}}_\infty + \bignbr{T_{h+1}^{\pi,\theta}(U_{h+1} - \tilde U_{h+1})}_\infty}\nend
    &\qquad + 6 B_U \cdot 2\rbr{\bignbr{(T_{h+1}^{\pi,\theta}-T_{h+1}^{\tilde\pi,\tilde\theta} )\tilde U_{h+1}}_\infty + \bignbr{T_{h+1}^{\pi,\theta}(U_{h+1} - \tilde U_{h+1})}}\nend
    &\quad\le 6 B_U(2\epsilon + B_U \epsilon ) + 12 B_U(B_U\epsilon + \epsilon)\nend
    &\quad\le (18 B_U^2 + 24 B_U)\epsilon, 
\end{align*}
where the second inequality follows from the definition of the covering net with respect to distance $\rho$ defined in \eqref{eq:rho-cY}. 
We obtain with probability at least $1-\delta$ that for any $y\in\cY_h$, $h\in[H]$, 
\begin{align*}
    \abr{\frac 1 T \sum_{i=1}^T {Z_h^i} - \EE\sbr{{Z_h^i}}} &\le \inf_{\epsilon>0} \frac{1}{2T}\sum_{i=1}^T \EE [{Z_h^i}] + \frac{110 B_U^2\cdot\log(H \cN_\rho(\cY_h, \epsilon)\delta^{-1})}{T} +2.5\cdot (18B_U^2 + 24 B_U)\epsilon\nend
    &\le \frac{1}{2T}\sum_{i=1}^T \EE [{Z_h^i}] + {110 B_U^2\cdot\log( H \cN_\rho(\cY_h, T^{-1})\delta^{-1}) } \cdot {T^{-1}} + (45 B_U^2 + 60 B_U )T^{-1}\nend
    & = \frac 1 T \rbr{\epsilon_S + \frac{T}{2} \EE[(U_{h} - \TT_{h}^{\pi,\theta}U_{h + 1})^2]}, 
\end{align*}
where $\epsilon_S = {110 B_U^2\cdot\log(H \max_{h\in[H]}\cN_\rho(\cY_h, T^{-1})\delta^{-1}) }  + (45 B_U^2 + 60 B_U )$, 
which proves our claim in \eqref{eq:cY-confset-1}.

\paragraph{Guarantee of the Confidence Set $\CI_{\cU}^{\pi,\theta}(\beta)$.}
We give a brief proof for the validity and the accuracy of the confidence set. For any $U_{h+1}\in\cU, \theta\in\Theta, \pi\in\Pi, h\in[H]$, on the one hand,
\begin{align}
    \ell_h(U_h, U_{h+1},\theta, \pi) - \inf_{U_h'\in\cU} \ell_h(U_h', U_{h+1}, \theta,\pi) \le \epsilon_S - \frac{T}{2} \EE[\orbr{U_{h} - \TT_{h}^{\pi,\theta}U_{h + 1}}^2], \label{eq:Bellman-loss-guarantee-1}
\end{align}
on the other hand,
\begin{align}
    \ell_h(U_h, U_{h+1},\theta, \pi) - \inf_{U_h'\in\cU} \ell_h(U_h', U_{h+1}, \theta,\pi) 
    &\ge \ell_h(U_h, U_{h+1},\theta, \pi) - \ell_h(\TT_h^{\pi, \theta} U_{h+1},  U_{h+1}, \theta,\pi) \nend
    &\ge - \epsilon_S  + \frac{T}{2} \EE[\orbr{U_{h} - \TT_{h}^{\pi,\theta}U_{h + 1}}^2], \label{eq:Bellman-loss-guarantee-2}
\end{align}
where the first inequality holds by the completeness assumption. 
For \eqref{eq:Bellman-loss-guarantee-1}, we plug in $U=U^{\pi,\theta}$ (realizability) and obtain 
$$\ell_h(U^{\pi,\theta}_h, U^{\pi,\theta}_{h+1},\theta, \pi) - \inf_{U_h'\in\cU} \ell_h(U_h', U^{\pi,\theta}_{h+1}, \theta,\pi) \le \epsilon_S.$$
Therefore, by having $\beta \ge \epsilon_S$, we have $U^{\pi,\theta}\in\CI_\cU^{\pi,\theta}(\beta)$. For the second one in \eqref{eq:Bellman-loss-guarantee-2}, we plug in any $U\in\CI_{\cU}^{\pi,\theta}(\beta)$ and obtain $\EE[\|U_{h} - \TT_{h}^{\pi,\theta}U_{h + 1}\|^2]\le T^{-1}\cdot (2\beta +2\epsilon_S)\le 4\beta T^{-1}$. Hence, we complete our proof of \Cref{lem:leader-bellman-loss}.

% \subsubsection{Proof of \Cref{lem:CI-U-online}}\label{sec:proof-CI-U-online}
% Our proof follows a similar scheme as in the proof of \Cref{lem:leader-bellman-loss}.

\paragraph{Concentration.}
We first take
\begin{align*}
    {Z_h^i} &= \ell_{h}(U'_{h}, U_{h + 1}, \theta_{h+1}, \pi) - \ell_{h}(\TT_{h}^{*,\theta}U_{h + 1}, U_{h + 1}, \theta_{h+1}, \pi)\nend
    & = \rbr{U_h(s_h^i, a_h^i, b_h^i) - u_h^i -  T_{h+1}^{*, \theta} U_{h+1}(s_{h+1}^i)}^2  - \rbr{\TT_h^{*, \theta} U_{h+1}(s_h^i, a_h^i, b_h^i) - u_h^i - T_{h+1}^{*, \theta} U_{h+1}(s_{h+1}^i)}^2.
\end{align*}
For the online setting, we have $(s_h^i, a_h^i, b_h^i, s_{h+1}^i)$ adapted to some filtration $\cF_h^i$. One choice of the filtration is $\cF_h^j=\sigma\rbr{\tau^{1:j-1}}$. Another choice of the filtration is $\cF_h^j=\sigma\rbr{(s_h^j, a_h^j, b_h^j), (s_h^i, a_h^i, b_h^i, s_{h+1}^i)_{i\in[j-1]}}$, where $\sigma(X)$ is the sigma-algebra of $X$.
They will both work for our proof.
We let $\EE^{i}_{x}[\cdot]$ be a short hand of $\EE^{i}[\cdot\given x, \cF_h^i]$ in the following proof.
We first calculate the expectation, 
\begin{align*}
    \EE^{i}[{Z_h^i}] 
    &= \EE^{i} \bigg[\EE^{i}_{s_h^i, a_h^i, b_h^i} \Big[\rbr{U_h(s_h^i, a_h^i, b_h^i) - u_h^i -  T_{h+1}^{*, \theta} U_{h+1}(s_{h+1}^i)}^2  \nend
    &\qquad - \rbr{\TT_h^{*, \theta} U_{h+1}(s_h^i, a_h^i, b_h^i) - u_h^i - T_{h+1}^{*, \theta} U_{h+1}(s_{h+1}^i)}^2\Big]\bigg]\nend
    & = \EE^{i}\bigg[\Bigrbr{U_h(s_h^i, a_h^i, b_h^i)- \TT_h^{*, \theta} U_{h+1}(s_h^i, a_h^i, b_h^i)}\nend
    &\qquad \cdot \EE^{i}_{s_h^i, a_h^i, b_h^i}\sbr{U_h(s_h^i, a_h^i, b_h^i) + \TT_h^{*, \theta} U_{h+1}(s_h^i, a_h^i, b_h^i)- 2u_h^i - 2T_{h+1}^{*, \theta} U_{h+1}(s_{h+1}^i) }\bigg]\nend
    & = \EE^{i}\sbr{\Bigrbr{U_h(s_h^i, a_h^i, b_h^i)- \TT_h^{*, \theta} U_{h+1}(s_h^i, a_h^i, b_h^i)}^2}, 
\end{align*}
where the second equality holds by the law of total expectation, and the third equality holds by noting that $\EE^{i}_{s_h^i, a_h^i, b_h^i}\osbr{u_h^i  + T_{h+1}^{*, \theta} U_{h+1}(s_{h+1}^i)} = \TT_h^{*, \theta} U_{h+1} (s_h^i, a_h^i, b_h^i)$.
Next, we calculate the variance, 
\begin{align*}
    \Var^{i}[{Z_h^i}^2] &\le \EE^{i}[{Z_h^i}^2]\nend
    &\le \EE^{i}\bigg[\Bigrbr{U_h(s_h^i, a_h^i, b_h^i)- \TT_h^{*, \theta} U_{h+1}(s_h^i, a_h^i, b_h^i)}^2\nend
    &\qquad \cdot \EE^{i}_{s_h^i, a_h^i, b_h^i}\sbr{\rbr{U_h(s_h^i, a_h^i, b_h^i) + \TT_h^{*, \theta} U_{h+1}(s_h^i, a_h^i, b_h^i)- 2u_h^i - 2T_{h+1}^{*, \theta} U_{h+1}(s_{h+1}^i)}^2 } \bigg]\nend
    &\le 49 B_U^2 \EE^{i}\sbr{\Bigrbr{U_h(s_h^i, a_h^i, b_h^i)- \TT_h^{*, \theta} U_{h+1}(s_h^i, a_h^i, b_h^i)}^2} = 49 B_U^2 \EE^{i}[{Z_h^i}].
\end{align*}
Also, one can verify that $|{Z_h^i}|\le 9B_U^2$ where $B_U$ bounds both the leader's reward and the value function class $\cU$.
We next take a $\epsilon$-covering of the class $\cZ_h=\cU^2\times\Theta_{h+1}$ with respect to the following distance defined in \eqref{eq:rho-cZ},
\begin{align*}
    &\rho\orbr{z, \tilde z}  = \max_{h\in[H]}\cbr{\bignbr{U_h-\tilde U_h}_\infty, \bignbr{ T_{h+1}^{*,\theta} U_{h+1} (\cdot) -  T_{h+1}^{*,\tilde\theta} \tilde U_{h+1} (\cdot)}_\infty }, 
\end{align*} 
We invoke the Freedman inequality \Cref{lem:freedman} for the martingale sequence $Z_h^i -\EE^i[Z_h^i]$, which says that for all $t\in [T], h\in[H], z\in\cZ_\epsilon$, it holds with probability at least $1-\delta$
\begin{align*}
    \sum_{i=1}^t \rbr{Z_h^i - \EE^i[Z_h^i]}
    &\le \frac{\lambda(e-2)}{9 B_U^2} \sum_{i=1}^t \EE^i\sbr{\rbr{Z_h^i - \EE^i[Z_h^i]}^2} + 9 \lambda^{-1} B_U^2\log(TH\cN_\rho(\cZ_h, \epsilon)\delta^{-1})\nend
    &\le \frac{\lambda(e-2) 49 }{9 } \sum_{i=1}^t \EE^i\sbr{Z_h^i} + 9 \lambda^{-1} B_U^2\log(TH\cN_\rho(\cZ_h, \epsilon)\delta^{-1}), \quad \forall \lambda\in(0, 1).
\end{align*}
Here, we plug in $\lambda = 9/98(e-2)$, $\epsilon=T^{-1}$ and notice that the above inequality also holds for $-Z_h^i + \EE^i[Z_h^i]$, which gives us for all $z\in\cZ_\epsilon, h\in[H], t\in[T]$ that
\begin{align}\label{eq:leader-Bellman-1}
    \abr{\sum_{i=1}^t \rbr{Z_h^i - \EE^i[Z_h^i]}} \le \frac 1 2 \sum_{i=1}^t \EE^i\sbr{Z_h^i} + c B_U^2\log(TH\cN_\rho(\cZ_h, T^{-1})\delta^{-1}), 
\end{align}
where we plug in $\epsilon=T^{-1}$ and $c=98(e-2)$ should be a universal constant. 
Next, we notice that
\begin{align*}
    &\bigrbr{U_h(s_h^i, a_h^i, b_h^i) - u_h^i -  T_{h+1}^{*,\theta} U_{h+1}(s_{h+1}^i)}^2  - \bigrbr{\TT_h^{*, \theta} U_{h+1}(s_h^i, a_h^i, b_h^i) - u_h^i - T_{h+1}^{*,\theta} U_{h+1}(s_{h+1}^i)}^2 \nend
    &\qquad - \rbr{\bigrbr{\tilde U_h(s_h^i, a_h^i, b_h^i) - u_h^i -  T_{h+1}^{*,\tilde\theta} \tilde U_{h+1}(s_{h+1}^i)}^2  - \bigrbr{\TT_h^{*, \tilde\theta} \tilde U_{h+1}(s_h^i, a_h^i, b_h^i) - u_h^i - T_{h+1}^{*,\tilde\theta} \tilde U_{h+1}(s_{h+1}^i)}^2}\nend
    %%%%%%%%%%%%%
    &\quad \le 6 B_U \rbr{\onbr{U-\tilde U}_\infty + \bignbr{(T_{h+1}^{*,\theta}-T_{h+1}^{*,\tilde\theta} )\tilde U_{h+1}}_\infty + \bignbr{T_{h+1}^{*,\theta}(U_{h+1} - \tilde U_{h+1})}_\infty}\nend
    &\qquad + 6 B_U \cdot 2\rbr{\bignbr{(T_{h+1}^{*,\theta}-T_{h+1}^{*,\tilde\theta} )\tilde U_{h+1}}_\infty + \bignbr{T_{h+1}^{*,\theta}(U_{h+1} - \tilde U_{h+1})}}\nend
    &\quad\le 6 B_U(2\epsilon + B_U \epsilon ) + 12 B_U(B_U\epsilon + \epsilon) \le (18 B_U^2 + 24 B_U)\epsilon, 
\end{align*}
Now, we let $\epsilon_S=c B_U^2 \allowbreak \log(TH\cN_\rho(\cZ_h, T^{-1})\delta^{-1}) + (45 B_U^2 + 60 B_u)$ and extend the result in \eqref{eq:leader-Bellman-1} to the whole class $\cY$, 
\begin{align*}
    \abr{\sum_{i=1}^t \rbr{Z_h^i - \EE^i[Z_h^i]}} 
    &\le \frac 1 2 \sum_{i=1}^t \EE^i\sbr{Z_h^i} + c B_U^2\log(TH\cN_\rho(\cZ, T^{-1})\delta^{-1}) + 2.5\cdot(18 B_U^2 + 24 B_U)\nend
    &\le \frac 1 2 \sum_{i=1}^t \EE^i\sbr{Z_h^i} + \epsilon_S.
\end{align*}
the following argument follows exactly the same as \eqref{eq:Bellman-loss-guarantee-1} and \eqref{eq:Bellman-loss-guarantee-2}, 
where we use the realizability assumption that $U^{*,\theta}\in \cU$ and the completeness assumption that there exists $U'\in\cU$ such that $U'=\TT_h^{*, \theta} U$ for any $U\in\cU, \theta\in\Theta$. 
We finish our proof of \Cref{lem:CI-U-online}.

% \subsubsection{Proof of \Cref{lem:1st-ub}}
% \label{sec:proof-1st-ub}
% Recall by definition, 
\begin{align*}
    \tilde \Delta^{(1)}_h(s_h, b_h) &=  \rbr{\EE_{s_h, b_h} -\EE_{s_h}}\Biggsbr{\sum_{l=h}^H \gamma^{l-h}\underbrace{\rbr{\tilde Q_l - r_l^\pi - \gamma P_l^\pi \tilde V_{l+1}}(s_l, b_l)}_{\ds\text{Follower's Bellman error}}}.
\end{align*}
In this section, we will bound $\EE\sbr{\abr{\tilde \Delta^{(1)}_h(s_h, b_h)}}$ by the KL distance in the following way.
% \paragraph{For Small $\eta \bigabr{\tilde A- A}$.}
% On the one hand, we have by the Cauchy-Schwartz inequality that
% \begin{align*}
%     \EE \exp\rbr{\eta \bigabr{\tilde A- A}} \cdot \EE \sbr{\exp\rbr{- \eta \bigabr{\tilde A- A}}\cdot \bigabr{\tilde A-A}^2} \ge \rbr{\EE\sbr{\bigabr{\tilde A-A}}}^2.
% \end{align*}
% On the other hand, the second term on the left hand side can be bounded by the Hellinger distance, 
% \begin{align*}
%     \EE D_\H^2(\nu, \tilde\nu) = \EE \inp[\bigg]{\nu}{\biggrbr{1-\sqrt{\frac{\tilde\nu}{\nu}}}^2} \ge \EE \sbr{\eta^2 \exp\rbr{- \eta \bigabr{\tilde A- A}}\cdot \bigabr{\tilde A-A}^2}.
% \end{align*}
% Hence, we conclude that
% \begin{align*}
%     \EE \exp\rbr{\eta \bigabr{\tilde A- A}} \cdot \EE D_\H^2(\nu, \tilde\nu)\ge \eta^2  \cdot \rbr{\EE\bigabr{\tilde A-A}}^2,
% \end{align*}
% and also
% \begin{align}
%     \sum_{l=h}^H \gamma^{l-h}\EE \exp\rbr{\eta \bigabr{\tilde A_l- A_l}} \cdot \sum_{l=h}^H \gamma^{l-h}\EE D_\H^2 \rbr{\nu_l, \tilde\nu_l} \ge \eta^2 \rbr{\sum_{l=h}^H \gamma^{l-h} \EE \bigabr{\tilde A_l-A_l}}^2.\label{eq:A-cauchy}
% \end{align}
We follow from the decomposition of the A-function in \Cref{lem:AQV-func diff},
\begin{align*}
    \abr{\tilde\Delta_h^{(1)}(s_h, b_h)} 
    &\le \bigabr{\bigrbr{A_h-\tilde A_h}(s_h, b_h)} + 2\eta^{-1} \Delta_h^{(2)}(s_h) \nend
    &\le \bigabr{\bigrbr{A_h-\tilde A_h}(s_h, b_h)} + 2\EE_{s_h}\sbr{\sum_{l=h}^H \gamma^{l-h} \inp[]{\nu_l(\cdot\given s_l)}{(A_l-\tilde A_l)(s_l, b_l)}}, 
\end{align*}
where we use the definition $\Delta_h^{(2)}(s_h)\defeq \EE_{s_h}\sbr{\sum_{l=h}^H \gamma^{l-h} \kl\infdivx[]{\nu_l}{\tilde\nu_l}}$ and the last inequality holds by noting that $\kl\infdivx[]{\nu}{\tilde\nu} = \eta\inp[]{\nu}{A-\tilde A}$.
Therefore, we conclude that
\begin{align}
    \EE\sbr{\abr{\tilde \Delta^{(1)}_h(s_h, b_h)}} 
    &\le  \EE\bigabr{\bigrbr{A_h-\tilde A_h}(s_h, b_h)} + 2\EE\sbr{\sum_{l=h}^H \gamma^{l-h} \inp[]{\nu_l(\cdot\given s_l)}{(A_l-\tilde A_l)(s_l, \cdot)}}
    \nend
    &
    \le 3 \sum_{l=h}^H \gamma^{l-h} \EE\bigabr{(A_l-\tilde A_l)(s_l, b_l)} \label{eq:Delta-A}
    % \\
    % &\le 3 \eta^{-1} \sqrt{\sum_{l=h}^H \gamma^{l-h}\EE \exp\rbr{\eta \bigabr{\tilde A_l- A_l}}} \cdot D_\RL(M^*,\tilde M;\pi),\nonumber
\end{align}
% where the last inequality is a direct result of \eqref{eq:A-cauchy}.
% \paragraph{For Large $\eta \bigabr{\tA - A}$. }
We now invoke the lower bound \eqref{eq:nu-tv-lb-1} in \Cref{lem:response diff} and obtain
\begin{align*}
    D_\TV(\nu_h, \tnu_h) &\ge \frac{1-\exp\rbr{-2\eta B_A}}{4 B_A} \cdot {\EE_{s_h}\bigabr{(\tA_h-A_h)(s_h, b_h)} } \nend
    &\ge  \frac{\eta}{2(1+ 2\eta B_A)} \cdot {\EE_{s_h}\bigabr{(\tA_h-A_h)(s_h, b_h)} }.
\end{align*}
Combining these results, we obtain
\begin{align*}
    \EE\sbr{\abr{\tilde \Delta^{(1)}_h(s_h, b_h)}} &\le 3\sum_{l=h}^H \gamma^{l-h} \EE\bigabr{(\tA_l-A_l)(s_l,b_l)} \nend
    &\le 3 \cdot \rbr{\frac{\eta}{2(1+ 2\eta B_A)}}^{-1} \sum_{l=h}^H \gamma^{l-h}\EE D_\TV(\nu_l(\cdot, s_l),\tilde\nu_l(\cdot, s_l)) \nend
    &\le 6(1+2\eta B_A)\cdot  {\frac{1-\gamma^H}{1-\gamma}}  \cdot \eta^{-1} \max_{h\in[H]} \EE D_\TV(\nu_h(\cdot, s_h),\tilde\nu_h(\cdot, s_h)),
\end{align*}
where the last inequality follows from from the fact that $(1-\exp(-x))/2x\ge 1/2(1+x)$.
% Hence, we have
% \begin{align*}
%     \EE\sbr{\abr{\tilde \Delta^{(1)}_h(s_h, b_h)}} &\le   3 \cdot \frac{4 \eta B_A}{1-\exp\rbr{-2\eta B_A}}\cdot
%     \sqrt{\frac{1-\gamma^H}{1-\gamma}}  \cdot
%     \eta^{-1}D_\RL(M^*,\tilde M;\pi)\nend
%     &\le 6(1+2\eta B_A)\cdot  \sqrt{\frac{1-\gamma^H}{1-\gamma}}  \cdot \eta^{-1}D_\RL(M^*,\tilde M;\pi),
% \end{align*}
Hence, we complete the proof of the first order of $\tilde \Delta^{(1)}$ in \Cref{lem:1st-ub}.

In the sequel, we will study how to upper bound $\bigrbr{\tilde \Delta_h^{(1)}(s_h, b_h)}^2$. We first have by \Cref{lem:AQV-func diff} that
\begin{align*}
    \bigrbr{\tilde \Delta_h^{(1)}(s_h, b_h)}^2 
    &= 2 \rbr{\rbr{\EE_{s_h, b_h}-\EE_{s_h}} \bigsbr{\orbr{A_h - \tilde A_h}(s_h, b_h)}}^2 + 2 \gamma^2 \eta^{-2} \rbr{\rbr{\EE_{s_h, b_h}-\EE_{s_h}}\bigsbr{\Delta_{h+1}^{(2)}(s_{h+1})}}^2\nend
    & \le 2 \rbr{\rbr{\EE_{s_h, b_h}-\EE_{s_h}} \bigsbr{\orbr{Q_h - \tilde Q_h}(s_h, b_h)}}^2 + 4 \gamma^2 \eta^{-2} \rbr{\EE_{s_h, b_h}\bigsbr{\Delta_{h+1}^{(2)}(s_{h+1})}}^2  \nend
    &\qquad + 4 \gamma^2 \eta^{-2} \rbr{\EE_{s_h}\bigsbr{\Delta_{h+1}^{(2)}(s_{h+1})}}^2,
\end{align*}
where the last inequality holds by using \eqref{eq:A diff-1} and note that $\eta^{-1}\kl\infdivx[]{\nu_h}{\tilde\nu_h} = \EE_{s_h}[A_h-\tilde A_h]$. By definition of $\Delta_h^{(2)}(s_h)$, we just focus on the second term and obtain 
\begin{align*}
    \rbr{\eta^{-1} \EE_{s_h, b_h}\bigsbr{\Delta_{h+1}^{(2)}(s_{h+1})} }^2
    &= \rbr{\eta^{-1}\EE_{s_h, b_h}\sbr{\sum_{l=h+1}^H \gamma^{l-h-1} \kl\infdivx[]{\nu_l(\cdot\given s_l)}{\tilde\nu_l(\cdot\given s_l)}}}^2\nend
    & = \rbr{\EE_{s_h, b_h}\sbr{\sum_{l=h+1}^H \gamma^{l-h-1} \inp[\big]{\nu_l(\cdot\given s_l)}{(A_l-\tilde A_l)(s_l, \cdot)}_\cB}}^2\nend
    &\le \eff_H(\gamma) \sum_{l=h+1}^H \gamma^{l-h-1}\rbr{\EE_{s_h, b_h}\sbr{\inp[\big]{\nu_l(\cdot\given s_l)}{\bigabr{(A_l-\tilde A_l)(s_l, \cdot)}}_\cB}}^2,
\end{align*}
where the last inequality follows from the Cauchy-Schwartz inequaltiy and we recall $\eff_H(\gamma) = (1-\gamma^H)/(1-\gamma)$.
We now invoke the lower bound \eqref{eq:nu-tv-lb-1} in \Cref{lem:response diff} and obtain
\begin{align*}
    D_\TV(\nu_h, \tnu_h) \ge \frac{\eta}{2(1+2\eta B_A)} \cdot \inp[\big]{\nu_h(\cdot\given s_h)}{\bigabr{
    \orbr{\tilde A - A}(s_h,\cdot)}}.
\end{align*}
Combining these results, we obtain
\begin{align*}
    &\rbr{\eta^{-1} \EE_{s_h, b_h}\bigsbr{\Delta_{h+1}^{(2)}(s_{h+1})} }^2 \nend
    &\quad \le 4\rbr{\eta^{-1} +2 B_A}^2\eff_H(\gamma) \sum_{l=h+1}^H \gamma^{l-h-1} {\EE_{s_h, b_h}\sbr{D_\H^2(\nu_l(\cdot\given s_l), \tilde\nu_l(\cdot\given s_l))}}, 
\end{align*}
where the inequality holds by using the Jensen's  inequality and move the expectation outside of the square. As a result, 
\begin{align*}
    &\bigrbr{\tilde \Delta_h^{(1)}(s_h, b_h)}^2  \nend
    &\quad \le 2 \rbr{\rbr{\EE_{s_h, b_h}-\EE_{s_h}} \bigsbr{\orbr{Q_h - \tilde Q_h}(s_h, b_h)}}^2 \nend
    &\qqquad + 16 \gamma^2  \rbr{\eta^{-1} +2 B_A}^2\eff_H(\gamma) \sum_{l=h+1}^H \gamma^{l-h-1} {\rbr{\EE_{s_h}+\EE_{s_h, b_h}}\sbr{D_\H^2(\nu_l(\cdot\given s_l), \tilde\nu_l(\cdot\given s_l))}}, 
\end{align*}
which completes our proof of \Cref{lem:1st-ub}.

% \subsubsection{Proof of \Cref{lem:2nd-ub}}
% \label{sec:proof-2nd-ub}
% In this proof, we remind readers that $Q, A, r^\pi$ are functions from $\cS\times\cB$ to $\RR$, $V:\cS\times\RR$ and $P_h^\pi:\cS\times\cB\rightarrow\Delta(\cS)$. In the sequel, we will neglect the dependence on $s_h, b_h$ for simplicity.
The major part in this proof is to upper bound $\EE\osbr{\orbr{( P_h^\pi-\tilde P_h^{\pi})\tilde V_{h+1}}^2}$ and $\EE\osbr{\orbr{r_h^\pi-\tilde r_h^\pi}^2}$ by $D_{\RL, h}^2$ separately. 
Moreover, we use $B_A$ in \eqref{eq:define_BA} to bound the follower's Q- and A-function. 
We will leave out the dependence on $(s_h, b_h)$ most of the times in the following proof when it does not cause any confusion in the context.

Note that we only have guarantee for $D_\TV^2(\nu_h, \tilde\nu_h)$ by MLE, which cannot directly guarantee that the true utility is identifiable since a constant shift does not change the follower's behavior at all. For the reward to be identifiable, we need an additional linear constraint, namely $\inp{x}{r_h(s_h, a_h, \cdot)}=\varsigma$.
We start with the easier part with the transition kernel.
\begin{align*}
    \EE\osbr{\orbr{( P_h^\pi-\tilde P_h^{\pi})\tilde V_{h+1}}^2}\le 2^2 B_A^2 \EE\sbr{D_\TV^2( P_h^\pi, \tilde  P_h^\pi)}.
\end{align*}
For the follower's reward, we take a real number $\xi$ and have the following decomposition
\begin{align*}
    &\inf_{\xi\in\RR}\EE_{s_h}\abr{r_h^\pi-\tilde r_h^\pi - \xi}  \nend
    &\quad = \inf_{\xi\in\RR}\EE_{s_h}\bigabr{Q_h-\tilde Q_h - \xi - \gamma\bigrbr{ P_h^\pi-\tilde P_h^\pi}\tilde V_{h+1} - \gamma  P_h^\pi\bigrbr{V_{h+1}-\tilde V_{h+1}}}\nend
    &\quad \le \inf_{\xi\in\RR}\EE_{s_h}\bigabr{Q_h-\tilde Q_h - \xi} + \gamma \EE_{s_h}\bigabr{\bigrbr{ P_h^\pi-\tilde P_h^\pi}\tilde V_{h+1} } + \gamma\exp\rbr{2\eta B_A}\EE_{s_h}\bigabr{Q_{h+1} - \tilde Q_{h+1}}, \nend
    &\quad \le \EE_{s_h}\bigabr{A_h-\tilde A_h} + \gamma\EE_{s_h}\bigabr{\bigrbr{ P_h^\pi-\tilde P_h^\pi}\tilde V_{h+1} } + \gamma\exp\rbr{2\eta B_A}\EE_{s_h}\bigabr{Q_{h+1} - \tilde Q_{h+1}}
\end{align*}
where the first inequality holds by the same argument for $V-\tilde V$ in \eqref{eq:f_2-1}, and the second inequality holds simply by plugging $\xi = V_h(s_h) - \tilde V_h(s_h)$. Now, we can plug in the bound for $\EE_{s_h}\oabr{A_h-\tilde A_h}$ in \Cref{lem:response diff} and obtain
\begin{align}
    &\inf_{\xi\in\RR}\EE_{s_h}\abr{r_h^\pi-\tilde r_h^\pi - \xi} \nend
    &\quad \le \underbrace{2(\eta^{-1}+2B_A) D_\TV(\nu_h, \tilde\nu_h) + 2 \gamma B_A \EE_{s_h} D_\TV( P_h^\pi, \tilde P_h^\pi)}_{\ds \sD_h} + \gamma\exp\rbr{2\eta B_A}\EE_{s_h}\bigabr{Q_{h+1} - \tilde Q_{h+1}}.
    % &\le \rbr{2\eta^{-1}(1+2\eta B_A)+2\gamma B_U} D_{\RL,h}(\tilde M,M^*;\pi) + \gamma\exp\rbr{2\eta B_A}\EE_{s_h}\abr{Q_{h+1} - \tilde Q_{h+1}}, 
    \label{eq:f_2-r-diff}
\end{align}
We next show what we can say about the utility when combining the guarantee of \eqref{eq:f_2-r-diff} with the linear constraint $\inp{x}{r_h(s_h, a_h, \cdot)}=\varsigma$. Specifically, we have the following lemma.
\begin{lemma}[Identification of the follower's utility]\label{lem:identification}
    Suppose for $r, \tilde r:\cB\rightarrow \RR$, for some distribution $\nu\in\Delta(\cB)$ such that $\nu>0$, we have $\inf_{\xi\in\RR}\inp{\nu}{\abr{r-\tilde r-\xi}}\le \varepsilon$ and $\inp{x}{r-\tilde r}=0$ hold at the same time for some $x:\cB\rightarrow \RR$ such that $\inp{\ind}{x}\neq 0$. We have
    \begin{align*}
        \inp{\nu}{\abr{r-\tilde r}}\le\rbr{1 + \nbr{\frac x \nu}_\infty \cdot \frac{1}{\abr{\inp{x}{\ind}}}} \epsilon
    \end{align*}
    \begin{proof}
        See \Cref{sec:proof-identification} for a detailed proof.
    \end{proof}
\end{lemma}
With \Cref{lem:identification}, we conclude with $\nbr{\nu}_\infty\ge \exp\rbr{-\eta B_A}$ and $\kappa = \nbr{x}_\infty/|\la x, \ind\ra|$ that
\begin{align*}
    \EE_{s_h}\abr{r_h^\pi-\tilde r_h^\pi} &\le \rbr{1+\exp\rbr{2\eta B_A} \kappa} \bigrbr{\sD_h + \gamma \exp\rbr{2\eta B_A}\cdot\EE_{s_h}\bigabr{Q_{h+1} - \tilde Q_{h+1}}}.
\end{align*}
On the other hand, for the Q-function, we have by \eqref{eq:f_2-Q-ub} that 
\begin{align*}
    \EE_{s_h}\bigabr{Q_h - \tilde Q_h}
    &\le \EE_{s_h}\bigabr{r_h^\pi-\tilde r_h^\pi + \gamma \bigrbr{ P_h^\pi-\tilde P_h^\pi}\tilde V_{h+1}} + \gamma \exp\rbr{2\eta B_A} {\EE_{s_h}\bigabr{Q_{h+1}-\tilde Q_{h+1}}}\nend
    &\le \underbrace{2\rbr{1+\exp\rbr{2\eta B_A} \kappa} }_{\ds c_1}\cdot \sD_h \nend
    &\qquad + \underbrace{\rbr{2+\exp\rbr{2\eta B_A} \kappa}\gamma \exp\rbr{2\eta B_A}}_{\ds c_2}\cdot\EE_{s_h}\bigabr{Q_{h+1} - \tilde Q_{h+1}}, 
\end{align*}
where in the last inequality, we directly upper bound $\EE_{s_h}|\gamma(P_h^\pi - \tilde P_h^\pi)\tilde V_{h+1}|$ by $\sD_h$. 
Therefore, we have by a recursive argument that 
\begin{align*}
    \EE_{s_h}\bigabr{Q_h -\tilde Q_h} \le \sum_{l=h}^H c_2^{l-h} c_1 \EE_{s_h}\sD_l.
\end{align*}
For now, we are able to deal with $(\EE_{s_h}\abr{r_h^\pi-\tilde r_h^\pi})^2$. However, note that what we actually want to get is the version with the square within the expectation $\EE_{s_h}$, i.e.,  $\EE\rbr{r_h^\pi-\tilde r_h^\pi}^2$. Therefore, we need a variance-mean decomposition,
\begin{align*}
    \EE\rbr{r_h^\pi-\tilde r_h^\pi}^2&= \EE\bigrbr{Q_h-\tilde Q_h - \gamma \bigrbr{ P_h^\pi -\tilde P_h^\pi}\tilde V_{h+1} -\gamma  P_h^\pi \bigrbr{V_{h+1}-\tilde V_{h+1}}}^2\nend
    &\le 4\EE\bigsbr{\bigrbr{A_h -\tilde A_h}^2 + \bigrbr{V_h-\tilde V_h}^2 + \gamma^2 \bigrbr{\bigrbr{ P_h^\pi -\tilde P_h^\pi}\tilde V_{h+1}}^2 +\gamma^2 \bigrbr{V_{h+1}-\tilde V_{h+1}}^2}\nend
    &\le 4\EE\bigsbr{\bigrbr{A_h -\tilde A_h}^2} + 16\gamma^2 B_A^2\EE\bigsbr{D_\TV( P_h^\pi,\tilde  P_h^\pi)^2}\nend
    &\qquad + 4\exp\rbr{4\eta B_A}\EE\bigsbr{\bigrbr{\EE_{s_h}\bigabr{Q_h-\tilde Q_h}}^2+ \bigrbr{\EE_{s_{h+1}}\bigabr{Q_{h+1}-\tilde Q_{h+1}}}^2}, 
\end{align*}
where in the first inequality, we use $Q=A+V$, and use the Jensen's inequality to derive the last term.
The last inequality holds by noting the upper bound for difference in the V-function used in \Cref{eq:f_2-1}.
We notice that the first term can be upper bounded by the squared Hellinger distance,
\begin{align*}
    D_\H^2\rbr{\nu_h,\tilde\nu_h} &= \dotp{\nu_h}{\rbr{1-\sqrt\frac{\tilde \nu_h}{\nu_h}}^2}_\cB\nend
    &= \Bigdotp{\nu_h}{\rbr{1-\exp\rbr{\frac \eta 2 (\tilde A_h - A_h)}}^2}_\cB\nend
    &\ge \rbr{\frac{1-\exp\rbr{-\eta B_A}}{2 B_A}}^2  \cdot \bigdotp{\nu_h}{\bigrbr{A_h-\tilde A_h}^2}_\cB \ge \rbr{\frac{\eta}{2}}^2  \cdot \bigdotp{\nu_h}{\bigrbr{A_h-\tilde A_h}^2}_\cB,
\end{align*}
where the first inequality holds by noting that $|1-\exp(x)|\ge (1-\exp(-B))|x|/B$ for any $|x|\le B$.
The last inequality uses the inequality $(1-\exp(-x))\ge x/(1+x)$ for all $x>0$. 
Therefore, we have the follower's squared reward difference bounded by
\begin{align*}
    &\EE\rbr{r_h^\pi -\tilde r_h^\pi}^2 \nend
    &\quad \le 16\eta^{-2} \EE D_\H^2(\nu_h,\tilde\nu_h) + 16\gamma^2 B_A^2\EE D_\TV^2( P_h^\pi,\tilde  P_h^\pi)\nend
    &\qqquad + 4 \exp\rbr{4\eta B_A} \EE\sbr{\rbr{\sum_{l=h}^H c_2^{l-h} c_1 \EE_{s_h}\sD_l}^2 + \rbr{\sum_{l=h+1}^H c_2^{l-h} c_1 \EE_{s_{h+1}}\sD_l}^2}\nend
    &\quad \le 16\eta^{-2} \EE D_\H^2(\nu_h,\tilde\nu_h) + 16\gamma^2 B_A^2\EE D_\TV^2( P_h^\pi,\tilde  P_h^\pi)\nend
    &\qqquad + 8 H \eff_H(c_2)^2 c_1^2 \exp\rbr{4\eta B_A} \max_{h\in[H]}\EE\sbr{{\sD_h}^2}\nend
    &\quad \le 16\eta^{-2} \EE D_\H^2(\nu_h,\tilde\nu_h) + 16\gamma^2 B_A^2\EE D_\TV^2( P_h^\pi,\tilde  P_h^\pi)\nend
    &\qqquad + 8 H \eff_H(c_2)^2 c_1^2\exp\rbr{4\eta B_A} \max_{h\in[H]}\EE\sbr{\rbr{{2(\eta^{-1}+2B_A) D_\TV(\nu_h, \tilde\nu_h) + 2 \gamma B_A \EE_{s_h} D_\TV( P_h^\pi, \tilde P_h^\pi)}}^2}, 
\end{align*}
where in the second inequality, we uses the Cauchy-Schwartz inequality that $\EE(\sum a_l x_l)^2\le \sum a_l \cdot \EE\sum a_l x_l^2 \le (\sum a_l)^2 \cdot \max_l \EE b_l^2$ for constant sequence $a_l>0$. 
In summary, we have
\begin{align*}
    \EE\rbr{r_h^\pi -\tilde r_h^\pi}^2
    &\le {32 H^2 \eff_H(c_2)^2 c_1^2\exp\rbr{4\eta B_A} \rbr{4(\eta^{-1}+2B_A)^2+4\gamma^2 B_A^2 }}  \nend
    & \qquad \cdot \max_{h\in[H]}\cbr{\EE D_\H^2(\nu_h,\tilde\nu_h)+\EE D_\TV^2(P_h^\pi,\tilde P_h^\pi)}\nend
    &\le \underbrace{640 H^2 \eff_H(c_2)^2 c_1^2\exp\rbr{4\eta B_A} (\eta^{-1}+B_A)^2}_{\ds c_3/4}  \nend
    & \qquad \cdot \max_{h\in[H]}\cbr{\EE D_\H^2(\nu_h,\tilde\nu_h)+\EE D_\TV^2(P_h^\pi,\tilde P_h^\pi)}
\end{align*}
Therefore, we conclude that
\begin{align*}
    \max_{h\in[H]}\EE\sbr{ \bigrbr{{\tilde Q_h - r_h^\pi - \gamma P_h^\pi \tilde V_{h+1}}}^2} 
    &\le 2 \max_{h\in[H]} \EE\sbr{\rbr{\tilde r_h^\pi - r_h^\pi}^2} + 2 \gamma^2\max_{h\in[H]} \EE\sbr{\bigrbr{\bigrbr{\tilde P_h^\pi - P_h^\pi}\tilde V_{h+1}}^2}\nend
    &\le c_3 \max_{h\in[H]}\cbr{\EE D_\H^2(\nu_h,\tilde\nu_h)+\EE D_\TV^2(P_h^\pi,\tilde P_h^\pi)}, 
\end{align*}
which completes our proof of \Cref{lem:2nd-ub}
% where the first inequality holds by noting that $C_\eta\le 2(1+2\eta B_A)$. 
% \newpage
% which completes the proof of \Cref{cor:online linear}.


\subsubsection{Proof of \Cref{lem:identification}}\label{sec:proof-identification}
For condition $\inf_{\xi\in\RR}\inp{\nu}{\abr{r-\tilde r-\xi}}\le \varepsilon$, we assume that the infimum is achieved at $\xi^*$. Let $r^* = r -\xi^*$ and we have
\begin{align*}
\abr{\inp{r^*-\tilde r}{x}} \le \inp{\abr{r^*-\tilde r}}{\abr{x}} \le \inp{\abr{r^*-\tilde r}}{\nu} \cdot \nbr{\frac{x}{\nu}}_\infty\le \varepsilon\nbr{\frac{x}{\nu}}_\infty,
\end{align*}
where the second inequality is just a distribution shift and the last inequality is given by the condition. Furthermore, for our target,
\begin{align*}
    \inp{\abr{r-\tilde r}}{\nu} \le \inp{\abr{r^*-\tilde r}}{\nu} + \abr{\xi^*} = \varepsilon + \abr{\inp{r^*-r}{x}} \cdot \frac{1}{\abr{\inp{x}{\ind}}},
\end{align*}
where the inequality follows from the triangle inequality and the equality holds by noting that $\inp{x}{\ind}\neq 0$ and $\inp{\abr{r^*-\tilde r}}{\nu}\le \varepsilon$. We bridge these two inequalities by noting that
\begin{align*}
    \abr{\inp{r^*-\tilde r}{x}} = \abr{\inp{r-\tilde r}{x} + \inp{r^*-r}{x} } = \abr{\inp{r^*-r}{x} },
\end{align*}
where the second inequality holds by noting that $\inp{r-\tilde r}{x}=0$. Combining these results and we have
\begin{align*}
    \inp{\abr{r-\tilde r}}{\nu} \le \epsilon + \abr{\inp{r^*-r}{x}} \cdot \frac{1}{\abr{\inp{x}{\ind}}} \le \rbr{1 + \nbr{\frac x \nu}_\infty \cdot \frac{1}{\abr{\inp{x}{\ind}}}} \epsilon, 
\end{align*}
which completes the proof on \Cref{lem:identification}.

