Recall by definition, 
\begin{align*}
    \tilde \Delta^{(1)}_h(s_h, b_h) &=  \rbr{\EE_{s_h, b_h} -\EE_{s_h}}\Biggsbr{\sum_{l=h}^H \gamma^{l-h}\underbrace{\rbr{\tilde Q_l - r_l^\pi - \gamma P_l^\pi \tilde V_{l+1}}(s_l, b_l)}_{\ds\text{Follower's Bellman error}}}.
\end{align*}
In this section, we will bound $\EE\sbr{\abr{\tilde \Delta^{(1)}_h(s_h, b_h)}}$ by the KL distance in the following way.
% \paragraph{For Small $\eta \bigabr{\tilde A- A}$.}
% On the one hand, we have by the Cauchy-Schwartz inequality that
% \begin{align*}
%     \EE \exp\rbr{\eta \bigabr{\tilde A- A}} \cdot \EE \sbr{\exp\rbr{- \eta \bigabr{\tilde A- A}}\cdot \bigabr{\tilde A-A}^2} \ge \rbr{\EE\sbr{\bigabr{\tilde A-A}}}^2.
% \end{align*}
% On the other hand, the second term on the left hand side can be bounded by the Hellinger distance, 
% \begin{align*}
%     \EE D_\H^2(\nu, \tilde\nu) = \EE \inp[\bigg]{\nu}{\biggrbr{1-\sqrt{\frac{\tilde\nu}{\nu}}}^2} \ge \EE \sbr{\eta^2 \exp\rbr{- \eta \bigabr{\tilde A- A}}\cdot \bigabr{\tilde A-A}^2}.
% \end{align*}
% Hence, we conclude that
% \begin{align*}
%     \EE \exp\rbr{\eta \bigabr{\tilde A- A}} \cdot \EE D_\H^2(\nu, \tilde\nu)\ge \eta^2  \cdot \rbr{\EE\bigabr{\tilde A-A}}^2,
% \end{align*}
% and also
% \begin{align}
%     \sum_{l=h}^H \gamma^{l-h}\EE \exp\rbr{\eta \bigabr{\tilde A_l- A_l}} \cdot \sum_{l=h}^H \gamma^{l-h}\EE D_\H^2 \rbr{\nu_l, \tilde\nu_l} \ge \eta^2 \rbr{\sum_{l=h}^H \gamma^{l-h} \EE \bigabr{\tilde A_l-A_l}}^2.\label{eq:A-cauchy}
% \end{align}
We follow from the decomposition of the A-function in \Cref{lem:AQV-func diff},
\begin{align*}
    \abr{\tilde\Delta_h^{(1)}(s_h, b_h)} 
    &\le \bigabr{\bigrbr{A_h-\tilde A_h}(s_h, b_h)} + 2\eta^{-1} \Delta_h^{(2)}(s_h) \nend
    &\le \bigabr{\bigrbr{A_h-\tilde A_h}(s_h, b_h)} + 2\EE_{s_h}\sbr{\sum_{l=h}^H \gamma^{l-h} \inp[]{\nu_l(\cdot\given s_l)}{(A_l-\tilde A_l)(s_l, b_l)}}, 
\end{align*}
where we use the definition $\Delta_h^{(2)}(s_h)\defeq \EE_{s_h}\sbr{\sum_{l=h}^H \gamma^{l-h} \kl\infdivx[]{\nu_l}{\tilde\nu_l}}$ and the last inequality holds by noting that $\kl\infdivx[]{\nu}{\tilde\nu} = \eta\inp[]{\nu}{A-\tilde A}$.
Therefore, we conclude that
\begin{align}
    \EE\sbr{\abr{\tilde \Delta^{(1)}_h(s_h, b_h)}} 
    &\le  \EE\bigabr{\bigrbr{A_h-\tilde A_h}(s_h, b_h)} + 2\EE\sbr{\sum_{l=h}^H \gamma^{l-h} \inp[]{\nu_l(\cdot\given s_l)}{(A_l-\tilde A_l)(s_l, \cdot)}}
    \nend
    &
    \le 3 \sum_{l=h}^H \gamma^{l-h} \EE\bigabr{(A_l-\tilde A_l)(s_l, b_l)} \label{eq:Delta-A}
    % \\
    % &\le 3 \eta^{-1} \sqrt{\sum_{l=h}^H \gamma^{l-h}\EE \exp\rbr{\eta \bigabr{\tilde A_l- A_l}}} \cdot D_\RL(M^*,\tilde M;\pi),\nonumber
\end{align}
% where the last inequality is a direct result of \eqref{eq:A-cauchy}.
% \paragraph{For Large $\eta \bigabr{\tA - A}$. }
We now invoke the lower bound \eqref{eq:nu-tv-lb-1} in \Cref{lem:response diff} and obtain
\begin{align*}
    D_\TV(\nu_h, \tnu_h) &\ge \frac{1-\exp\rbr{-2\eta B_A}}{4 B_A} \cdot {\EE_{s_h}\bigabr{(\tA_h-A_h)(s_h, b_h)} } \nend
    &\ge  \frac{\eta}{2(1+ 2\eta B_A)} \cdot {\EE_{s_h}\bigabr{(\tA_h-A_h)(s_h, b_h)} }.
\end{align*}
Combining these results, we obtain
\begin{align*}
    \EE\sbr{\abr{\tilde \Delta^{(1)}_h(s_h, b_h)}} &\le 3\sum_{l=h}^H \gamma^{l-h} \EE\bigabr{(\tA_l-A_l)(s_l,b_l)} \nend
    &\le 3 \cdot \rbr{\frac{\eta}{2(1+ 2\eta B_A)}}^{-1} \sum_{l=h}^H \gamma^{l-h}\EE D_\TV(\nu_l(\cdot, s_l),\tilde\nu_l(\cdot, s_l)) \nend
    &\le 6(1+2\eta B_A)\cdot  {\frac{1-\gamma^H}{1-\gamma}}  \cdot \eta^{-1} \max_{h\in[H]} \EE D_\TV(\nu_h(\cdot, s_h),\tilde\nu_h(\cdot, s_h)),
\end{align*}
where the last inequality follows from from the fact that $(1-\exp(-x))/2x\ge 1/2(1+x)$.
% Hence, we have
% \begin{align*}
%     \EE\sbr{\abr{\tilde \Delta^{(1)}_h(s_h, b_h)}} &\le   3 \cdot \frac{4 \eta B_A}{1-\exp\rbr{-2\eta B_A}}\cdot
%     \sqrt{\frac{1-\gamma^H}{1-\gamma}}  \cdot
%     \eta^{-1}D_\RL(M^*,\tilde M;\pi)\nend
%     &\le 6(1+2\eta B_A)\cdot  \sqrt{\frac{1-\gamma^H}{1-\gamma}}  \cdot \eta^{-1}D_\RL(M^*,\tilde M;\pi),
% \end{align*}
Hence, we complete the proof of the first order of $\tilde \Delta^{(1)}$ in \Cref{lem:1st-ub}.

In the sequel, we will study how to upper bound $\bigrbr{\tilde \Delta_h^{(1)}(s_h, b_h)}^2$. We first have by \Cref{lem:AQV-func diff} that
\begin{align*}
    \bigrbr{\tilde \Delta_h^{(1)}(s_h, b_h)}^2 
    &= 2 \rbr{\rbr{\EE_{s_h, b_h}-\EE_{s_h}} \bigsbr{\orbr{A_h - \tilde A_h}(s_h, b_h)}}^2 + 2 \gamma^2 \eta^{-2} \rbr{\rbr{\EE_{s_h, b_h}-\EE_{s_h}}\bigsbr{\Delta_{h+1}^{(2)}(s_{h+1})}}^2\nend
    & \le 2 \rbr{\rbr{\EE_{s_h, b_h}-\EE_{s_h}} \bigsbr{\orbr{Q_h - \tilde Q_h}(s_h, b_h)}}^2 + 4 \gamma^2 \eta^{-2} \rbr{\EE_{s_h, b_h}\bigsbr{\Delta_{h+1}^{(2)}(s_{h+1})}}^2  \nend
    &\qquad + 4 \gamma^2 \eta^{-2} \rbr{\EE_{s_h}\bigsbr{\Delta_{h+1}^{(2)}(s_{h+1})}}^2,
\end{align*}
where the last inequality holds by using \eqref{eq:A diff-1} and note that $\eta^{-1}\kl\infdivx[]{\nu_h}{\tilde\nu_h} = \EE_{s_h}[A_h-\tilde A_h]$. By definition of $\Delta_h^{(2)}(s_h)$, we just focus on the second term and obtain 
\begin{align*}
    \rbr{\eta^{-1} \EE_{s_h, b_h}\bigsbr{\Delta_{h+1}^{(2)}(s_{h+1})} }^2
    &= \rbr{\eta^{-1}\EE_{s_h, b_h}\sbr{\sum_{l=h+1}^H \gamma^{l-h-1} \kl\infdivx[]{\nu_l(\cdot\given s_l)}{\tilde\nu_l(\cdot\given s_l)}}}^2\nend
    & = \rbr{\EE_{s_h, b_h}\sbr{\sum_{l=h+1}^H \gamma^{l-h-1} \inp[\big]{\nu_l(\cdot\given s_l)}{(A_l-\tilde A_l)(s_l, \cdot)}_\cB}}^2\nend
    &\le \eff_H(\gamma) \sum_{l=h+1}^H \gamma^{l-h-1}\rbr{\EE_{s_h, b_h}\sbr{\inp[\big]{\nu_l(\cdot\given s_l)}{\bigabr{(A_l-\tilde A_l)(s_l, \cdot)}}_\cB}}^2,
\end{align*}
where the last inequality follows from the Cauchy-Schwartz inequaltiy and we recall $\eff_H(\gamma) = (1-\gamma^H)/(1-\gamma)$.
We now invoke the lower bound \eqref{eq:nu-tv-lb-1} in \Cref{lem:response diff} and obtain
\begin{align*}
    D_\TV(\nu_h, \tnu_h) \ge \frac{\eta}{2(1+2\eta B_A)} \cdot \inp[\big]{\nu_h(\cdot\given s_h)}{\bigabr{
    \orbr{\tilde A - A}(s_h,\cdot)}}.
\end{align*}
Combining these results, we obtain
\begin{align*}
    &\rbr{\eta^{-1} \EE_{s_h, b_h}\bigsbr{\Delta_{h+1}^{(2)}(s_{h+1})} }^2 \nend
    &\quad \le 4\rbr{\eta^{-1} +2 B_A}^2\eff_H(\gamma) \sum_{l=h+1}^H \gamma^{l-h-1} {\EE_{s_h, b_h}\sbr{D_\H^2(\nu_l(\cdot\given s_l), \tilde\nu_l(\cdot\given s_l))}}, 
\end{align*}
where the inequality holds by using the Jensen's  inequality and move the expectation outside of the square. As a result, 
\begin{align*}
    &\bigrbr{\tilde \Delta_h^{(1)}(s_h, b_h)}^2  \nend
    &\quad \le 2 \rbr{\rbr{\EE_{s_h, b_h}-\EE_{s_h}} \bigsbr{\orbr{Q_h - \tilde Q_h}(s_h, b_h)}}^2 \nend
    &\qqquad + 16 \gamma^2  \rbr{\eta^{-1} +2 B_A}^2\eff_H(\gamma) \sum_{l=h+1}^H \gamma^{l-h-1} {\rbr{\EE_{s_h}+\EE_{s_h, b_h}}\sbr{D_\H^2(\nu_l(\cdot\given s_l), \tilde\nu_l(\cdot\given s_l))}}, 
\end{align*}
which completes our proof of \Cref{lem:1st-ub}.