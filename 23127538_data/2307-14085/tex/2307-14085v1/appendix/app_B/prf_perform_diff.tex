
In the following, we prove \Cref{lem:performance diff}, which relates the estimation error of quantal response policy to a few estimation errors involving the follower's value functions. 
To simplify the notation, we let $\nu$, 
$Q, V, A$ denote $\nu$, $Q^{\pi}$, $V^{\pi}$, and $A^{\pi}$, respectively, which are quantities computed under the true model $M^*$.  
Note that we have  
$\nu_h(b\given s) = \exp(\eta \cdot A_h (s, b) )$ and 
$\tilde \nu_h (b \given s) = \exp(\eta \cdot \tilde A_h (s,b) )$.
By the upper bound in \eqref{eq:nu-tv-ub-0} of \Cref{lem:response diff},
we have 
\begin{align}
    \dr{(i)} &\defeq \sum_{h=1}^H  H \cdot  \EE \bigsbr{ \nbr{\tnu_h(\cdot\given s_h)-\nu_h(\cdot\given s_h)}_1} = \sum_{h=1}^H  2 H \cdot  \EE \bigsbr{  D_\TV \bigrbr{\nu(\cdot\given s_h), \tilde \nu(\cdot\given s_h)}} \nend
    &\le 2 \eta  H \cdot  \underbrace{\sum_{h=1}^H \EE\bigsbr{\bigabr{\orbr{A_h-\tilde A_h}(s_h, b_h)}}}_{\dr (ii)} \nend
    &\qqquad + \eta^2  H \cdot \sum_{h=1}^H \EE\Bigsbr{ \exp\bigrbr{\eta\bigabr{\orbr{A_h-\tilde A_h}(s_h, b_h)}}\cdot \bigabr{\orbr{A_h-\tilde A_h}(s_h, b_h)}^2}. \label{eq:(i)}
\end{align}
% In this section, we characterize the performance difference that arises from model misspecification, namely the difference in the leader's total reward under a given policy $\pi$ for a misspecified model $\tilde M$ against the true model $M^*$. In the following, we use $(Q_h, V_h, A_h, \nu_h, U_h, W_h)$ for the follower/leader under the true model $M^*$, and $(\tQ_h,\tV_h,\tA_h, \tnu_h, \tilde U_h, \tilde W_h)$ for the follower/leader under the alternative model $\tilde M$. 
% We let
% \begin{align}
%     \dr{(i)} &\defeq \sum_{h=1}^H  H \EE \nbr{\tnu_h(\cdot\given s_h)-\nu_h(\cdot\given s_h)}_1\nend
%     &\le 2 \eta  H \underbrace{\sum_{h=1}^H \EE\sbr{\abr{\rbr{A_h-\tilde A_h}(s_h, a_h)}}}_{\dr (ii)} \nend
%     &\qqquad + \eta^2  H \sum_{h=1}^H \EE\sbr{ \exp\rbr{\eta\abr{\rbr{A_h-\tilde A_h}(s_h, a_h)}}\cdot \abr{\rbr{A_h-\tilde A_h}(s_h, a_h)}^2}, \label{eq:(i)}
% \end{align}
%where the first inequality holds by noting that $\bignbr{\tilde U_h}_\infty \le  H$, and the second inequality uses the upper bound for the TV distance between two difference logistic responses given by \Cref{lem:response diff}.
In the following, we let 
\$
    \tilde\Delta_h^{(1)} (s_h, b_h)&\defeq  \rbr{\EE_{s_h, b_h} - \EE_{s_h}}\sbr{\sum_{l=h}^H \gamma^{l-h}\bigrbr{\orbr{\tilde Q_l - r_l^\pi - \gamma P_l^\pi \tilde V_{l+1}}(s_l, b_l)}}, \\
    \tilde\Delta_h^{(2)}(s_h) &\defeq \EE_{s_h}\sbr{\sum_{l=h}^H \gamma^{l-h} \kl\infdivx[\big]{\nu_l(\cdot\given s_l)}{\tilde\nu_l(\cdot\given s_l)}}. 
\$
We note that we denote $\EE_{z} [\cdot]=\EE^{\pi,M^*}[\cdot\given z]$ for any variable $z$.
Here the expectations in $\tilde\Delta^{(1)}_h$ and $\tilde\Delta^{(2)}_h$ are taken with respect to the randomness of the trajectory generated by $\{ \pi, \nu^{\pi}\}$, given $s_h$ or $(s_h, b_h)$.
%Note that $\tilde\Delta^{(1)}_h$ is actually a short hand of $\tilde\Delta^{(1)}_{h,\pi, M}(s_h, b_h)$.
We can further bound   (ii) defined in \eqref{eq:(i)} by invoking \Cref{lem:AQV-func diff}, 
which implies that 
\$
\dr{(ii)}&= \sum_{h=1}^H \EE\sbr{\abr{\rbr{\EE_{s_h, b_h}-\EE_{s_h}} \bigsbr{\tilde\Delta_h^{(1)}(s_h, b_h) - \gamma\eta^{-1}\tilde\Delta_{h+1}^{(2)} (s_{h+1})} + \eta^{-1}\kl\infdivx[\big]{\nu_h(\cdot\given s_h)}{\tilde \nu_h(\cdot\given s_h)}}}.
\$
By the law of total expectation, 
we have 
\$
\EE \Bigsbr{\EE _{s_h, b_h} \bigsbr{   \tilde\Delta_{h+1}^{(2)} (s_{h+1})}} = \EE \Bigsbr{\EE _{s_h } \bigsbr{   \tilde\Delta_{h+1}^{(2)} (s_{h+1})}}  \geq 0.
\$
Besides, by the definition of $\tilde \Delta_h^{(2)}$, we have 
\$
\EE\bigsbr{\tilde\Delta_h^{(2)}(s_h)} = \EE\bigsbr{ \kl\infdivx[\big]{\nu_h(\cdot\given s_h)}{\tilde \nu_h(\cdot\given s_h)} + \gamma \cdot \tilde\Delta_{h+1}^{(2)} (s_{h+1}) }. 
\$ 
Thus, by triangle inequality, we have 
\begin{align}
    \dr{(ii)}  &\le \sum_{h=1}^H\EE\Bigsbr{\bigabr{\rbr{\EE_{s_h, b_h}-\EE_{s_h}}\bigsbr{\tilde\Delta_h^{(1)}(s_h, b_h)}}} + 2\eta^{-1} \sum_{h=1}^H\EE\bigsbr{\tilde\Delta_h^{(2)}(s_h)}, \label{eq:(ii)}
\end{align}
Furthermore, for the second term on the right-hand side of \eqref{eq:(ii)}, we  apply the inequality between KL divergence and the $\chi^2$ divergence to each term $\kl\infdivx[]{\nu_l(\cdot\given s_l)}{\tilde\nu_l(\cdot\given s_l)}$ and obtain that 
\begin{align}
\kl\infdivx[\big]{\nu_l(\cdot\given s_l)}{\tilde\nu_l(\cdot\given s_l)}
&\le \chi^2\infdivx[\big]{\nu_l(\cdot\given s_l)}{\tilde\nu_l(\cdot\given s_l)}\nend
&= \inp[\Bigg]{\nu_l(\cdot\given s_l)}{\biggrbr{\sqrt{\frac{\nu_l(\cdot\given s_l)}{\tilde\nu_l(\cdot\given s_l)}}-\sqrt{\frac{\tilde\nu_l(\cdot\given s_l)}{\nu_l(\cdot\given s_l)}}}^2}_{\cB}\nend
&\le \eta^2 \cdot \EE_{s_l}\sbr{\exp\bigrbr{\eta  \cdot \bigabr{\orbr{A_l-\tilde A_l}(s_l,b_l)}}\cdot \bigrbr{\orbr{A_l -\tilde A_l}(s_l, b_l)}^2}, \label{eq:kl-ub}
\end{align}
where last expectation is with respect to $b_ l \sim \nu_{l } (\cdot \given s_l)$. 
Here the inequality holds by noting that 
 $\sqrt{\nu_l/\tilde \nu_l}=\exp(\eta(A_l-\tilde A_l)/2)$ and the basic inequality $| \exp( x ) - \exp(y)| \leq \exp ( | x-y|) \cdot |x -y|$.
Plugging \eqref{eq:kl-ub} back into \eqref{eq:(ii)}, we obtain
\begin{align}
    \dr{(ii)} &\le \sum_{h=1}^H\EE\Bigsbr{\bigabr{\rbr{\EE_{s_h, b_h}-\EE_{s_h}}\bigsbr{\tilde\Delta_h^{(1)}(s_h, b_h)}}}  + 2\eta^{-1} \sum_{h=1}^H \EE\sbr{\sum_{l=h}^H \gamma^{l-h} \cdot \kl\infdivx[\big]{\nu_l(\cdot\given s_l)}{\tilde\nu_l(\cdot\given s_l)}}\nend
    &\le \sum_{h=1}^H\EE\sbr{\bigabr{\rbr{\EE_{s_h, b_h}-\EE_{s_h}}\bigsbr{\tilde\Delta_h^{(1)}(s_h, b_h)}}} \nend
    &\qquad + 2\eta \sum_{h=1}^H \sum_{l=h}^{H} \gamma^{l-h} \cdot  \EE\sbr{\exp\bigrbr{\eta  \cdot \bigabr{\orbr{A_l-\tilde A_l}(s_l,b_l)}}\cdot \bigabr{\orbr{A_l -\tilde A_l}(s_l, b_l)}^2}\nend
    &\le \sum_{h=1}^H\EE\sbr{\bigabr{\rbr{\EE_{s_h, b_h}-\EE_{s_h}}\bigsbr{\tilde\Delta_h^{(1)}(s_h, b_h)}}} \nend
    &\qquad + \frac{2\eta(1-\gamma^H)}{1-\gamma}\cdot \sum_{h=1}^H  \EE\sbr{\exp\bigrbr{\eta  \cdot \bigabr{\orbr{A_h-\tilde A_h}(s_h,b_h)}}\cdot \bigabr{\orbr{A_h -\tilde A_h}(s_h, b_h)}^2} .
     \label{eq:(ii)-2}
\end{align}
Recall that we  define $\eff_H(x) = (1-x^H)/(1-x)$ as the \say{effective}  horizon with respect to $x$.


Plugging \eqref{eq:(ii)-2} back into \eqref{eq:(i)}, we conclude that
\begin{align}
    \dr{(i)}&\le 2\eta  H \cdot \sum_{h=1}^H  \EE\sbr{\abr{\rbr{\EE_{s_h, b_h}-\EE_{s_h}}\bigsbr{\tilde\Delta_h^{(1)}(s_h, b_h)}}}   \nend
    &\qquad + \eta^2  H  \bigrbr{1+ 4  \cdot \eff_H(\gamma) } \cdot \sum_{h=1}^H  \EE\sbr{\exp\bigrbr{\eta  \cdot \bigabr{\orbr{A_h-\tilde A_h}(s_h,b_h)}}\cdot \bigabr{\orbr{A_h -\tilde A_h}(s_h, b_h)}^2}  . \label{eq:(ii)-21}
\end{align}
Note that we define  $C^{(1)}$ in \eqref{eq:define_constants}. 
Since $\oabr{\orbr{A_h-\tilde A_h}(s_h,b_h)} \leq 2 B_{A}$, 
by  \eqref{eq:(ii)-21}  and inequality 
\$
\exp\bigrbr{\eta  \cdot \bigabr{\orbr{A_h-\tilde A_h}(s_h,b_h)}}  \leq \exp(2\eta B_{A}), 
\$
we conclude the proof of \eqref{eq:taylor-myopic}. 

It remains to prove \eqref{eq:taylor-farsighted}. 
Notice that 
\#\label{eq:f_2-01}
\begin{split}
    V_h (s_h) = \max_{\nu' \in \Delta (\cB) }\bigl\{ \inp{\nu' }{Q_h (s_h, \cdot )}_{\cB } +\eta^{-1} \cH(\nu')\bigr\} , \\
    \tilde V_h (s_h) = \max_{\nu' \in \Delta (\cB) }\bigl\{ \inp{\nu' }{\tilde Q_h (s_h, \cdot )}_{\cB } +\eta^{-1} \cH(\nu')\bigr\} ,  
\end{split}
\#
where the maximizers are $\nu_h (\cdot \given s_h)$ and $\tilde \nu_h (\cdot \given s_h)$, respectively.
Then, by  \eqref{eq:f_2-01}
we have 
\#
& \bigabr{\orbr{V_h-\tilde V_h}(s_h)}   \notag \\
& \quad \le \max\Bigcbr{\inp[\big]{\nu_h(\cdot \given s_h) }  { \bigabr{Q_h(s_h, \cdot )-\tilde Q_h(s_h, \cdot )} }_{\cB} }, ~\Bigabr{\inp[\big]{\tilde\nu_h(\cdot \given s_h) }{\bigabr{ Q_h(s_h, \cdot )-\tilde Q_h(s_h, \cdot )} }_{\cB} } \notag\\
& \quad  =   \max\Bigcbr{  \EE_{s_h} \bigl [  \bigl | (Q_h - \tilde Q_h) (s_h, b_h ) \big | \bigr ] , ~   \EE_{s_h} \bigl [ \big |  (Q_h - \tilde Q_h)  (s_h, b_h )\bigr |  \cdot \tilde \nu_h (b_h \given s_h) / \nu_h (b_h \given s_h ) \bigr ]  }   \notag \\
& \quad  \leq    \exp(2 \eta B_A)  \cdot    \EE_{s_h} \bigl [ \big |  (Q_h - \tilde Q_h) (s_h, b_h )\big | \bigr ]   , \label{eq:f_2-1} 
\# 
where the expectation is taken with respect to $b_h \sim \nu_h (\cdot \given s_h)$. 
Here the first inequality is obtained from the optimality condition of \eqref{eq:f_2-01}, and the  last inequality holds because  $\nbr{\tilde\nu_h/\nu_h}_\infty \le \exp(2\eta B_A)$. 
Note that $\tilde A = \tilde Q - \tilde V$ and $A = Q - V$.
By triangle inequality, we have 
\begin{align}
    &\abr{\orbr{A_h-\tilde A_h}(s_h, b_h)}  
  \le  \bigabr{\orbr{Q_h-\tilde Q_h}(s_h, b_h)} + \exp\orbr{2\eta B_A} \cdot  \EE_{s_h} \bigl [  \big |  (Q_h - \tilde Q_h) (s_h, b_h )\big |  \bigr ]  .  \label{eq:f_2-11} 
\end{align}
%   
%  
%  
% For the left side, we first notice 
% \begin{align*}
%     \EE_{s_h}\rbr{A_h^\pi-\tilde A_h^\pi}^2 
%     &= \EE_{s_h}\sbr{\rbr{\rbr{\EE_{s_h, b_h} -\EE_{s_h}}\sbr{A_h^\pi-\tilde A_h^\pi}}^2 }+ \rbr{\EE_{s_h}\sbr{A_h^\pi-\tilde A_h^\pi}}^2\nend
%     & = \EE_{s_h}\sbr{\rbr{\rbr{\EE_{s_h, b_h}-\EE_{s_h}}\sbr{\tilde\Delta_h^{(1)}(s_h, b_h) - \gamma\eta^{-1}\tilde\Delta_{h+1}^{(2)} (s_{h+1})}}^2} + \rbr{\EE_{s_h}\abr{A_h^\pi-\tilde A_h^\pi}}^2,
% \end{align*}
% where the first equality follows from a standard mean-variance decomposition, and the inequality holds by \eqref{eq:A diff-1} in \Cref{lem:AQV-func diff} and noting that $\eta^{-1}\kl\infdivx{\nu_h}{\tilde\nu_h} = \EE_{s_h, b_h}\bigsbr{A_h-\tilde A_h}$.
Now for $\oabr{   (Q_h - \tilde Q_h) (s_h, b_h ) }$, by the Bellman equation $Q_h = r^{\pi}_h + \gamma P_h^{\pi} V_{h+1}$, we have  
\begin{align}
    &\bigabr{ \orbr{Q_h-\tilde Q_h}(s_h, b_h)} \nend
    &\quad \le  
    \bigabr{ 
        \orbr{\tilde Q_h - r_h^\pi - \gamma P_h^\pi \tilde V_{h+1}}(s_h, b_h)
    } 
         + \gamma  \cdot \bigabr{\bigrbr{P_h^\pi \orbr{V_{h+1}-\tilde V_{h+1}}}(s_h, b_h)}
         \nend
    &\quad \le \bigabr{\orbr{\tilde Q_h - r_h^\pi - \gamma P_h^\pi \tilde V_{h+1}}
    (s_h, b_h)} 
    + \gamma \cdot \exp\rbr{2\eta B_A}\EE_{s_h, b_h}\bigsbr{\oabr{Q_{h+1}^\pi-\tilde Q_{h+1}^\pi}},\label{eq:f_2-Q-ub}
\end{align}
where the first inequality holds by a standard decomposition and the second inequality is obtained by applying the  same upper bound for $V_{h}-\tilde V_h$ in \eqref{eq:f_2-1} to $V_{h+1} - \tilde V_{h+1}$. 
By recursion, we have 
\begin{align}
    \bigabr{\orbr{Q_h-\tilde Q_h}(s_h, b_h)} &\le  \sum_{l=h}^H \big (\gamma \cdot \exp(2\eta B_A  ) \big) ^{l-h} \cdot  \EE_{s_h, b_h}\bigsbr{\bigabr{\orbr{\tilde Q_l - r_l^\pi - \gamma P_l^\pi \tilde V_{l+1}}(s_l, b_l)}}.\label{eq:f_2-Q-telo}
    % \nend
    % &\qqquad +  \sum_{l=h}^H \exp(2\eta B_A (l-h+1)) \gamma^{l-h} \rbr{\EE_{s_h}\abr{r_l^\pi-\tilde r_l^\pi} + \gamma \EE_{s_h}\abr{\rbr{P_l^\pi-\tildeP_l^\pi}\tilde V_{h+1}}}.
\end{align}
Now, by the boundedness of $A_h$ and $\tilde A_h$, and \eqref{eq:f_2-11}, 
we have 
\$
& \EE\sbr{\exp\bigrbr{\eta  \cdot \bigabr{\orbr{A_h-\tilde A_h}(s_h,b_h)}}\cdot \bigabr{\orbr{A_h -\tilde A_h}(s_h, b_h)}^2} \notag \\
& \quad \leq \exp\bigrbr{2\eta B_A}\cdot \EE\bigsbr{ \bigabr{\orbr{A_h-\tilde A_h}(s_h, a_h)}^2}\nend
&\quad \le 2\exp\bigrbr{6\eta B_A}\cdot {\EE\bigsbr{ \bigabr{\orbr{\tilde Q_h - Q_h}(s_h, b_h)}^2}},
\$ 
where in the  last inequality we use the basic inequality $(a + b)^2 \leq 2 a ^2 + 2 b^2 $. 
Combining  with \eqref{eq:f_2-Q-telo}, we obtain that 
\begin{align*}
    &\EE\sbr{\exp\bigrbr{\eta  \cdot \bigabr{\orbr{A_h-\tilde A_h}(s_h,b_h)}}\cdot \bigabr{\orbr{A_h -\tilde A_h}(s_h, b_h)}^2} \nend
    &\quad \le 2\exp\bigrbr{6\eta B_A} \cdot \EE \biggsbr { \biggrbr{\sum_{l=h}^H (\gamma \cdot \exp(2\eta B_A  ) \big) ^{l-h} \cdot  \EE_{s_h, b_h}\bigsbr{\bigabr{\orbr{\tilde Q_l - r_l^\pi - \gamma P_l^\pi \tilde V_{l+1}}(s_l, b_l)}}}^2 } \nend
    &\quad \le 2\exp\orbr{6\eta B_A} \cdot \bigrbr{\eff_H(\exp\orbr{2\eta B_A}\gamma)}^2  \cdot \max_{l\in\{h, \dots, H\}}\EE\bigsbr{\bigabr{\orbr{\tilde Q_l - r_l^\pi - \gamma P_l^\pi \tilde V_{l+1}}(s_l, b_l)}^2}.
\end{align*}
Recall that we define  $C^{(2)} = 2 \eta^2  H^2 \cdot \exp\rbr{6\eta B_A}  \cdot \rbr{1+ 4 \eff_H(\gamma)} \cdot \rbr{\eff_H(\exp\cbr{2\eta B_A}\gamma)}^2$. By \eqref{eq:(ii)-21}, we establish \eqref{eq:perform-diff-linear}. 
Therefore, we 
  complete the proof of \Cref{lem:performance diff}.