In this proof, we remind readers that $Q, A, r^\pi$ are functions from $\cS\times\cB$ to $\RR$, $V:\cS\times\RR$ and $P_h^\pi:\cS\times\cB\rightarrow\Delta(\cS)$. In the sequel, we will neglect the dependence on $s_h, b_h$ for simplicity.
The major part in this proof is to upper bound $\EE\osbr{\orbr{( P_h^\pi-\tilde P_h^{\pi})\tilde V_{h+1}}^2}$ and $\EE\osbr{\orbr{r_h^\pi-\tilde r_h^\pi}^2}$ by $D_{\RL, h}^2$ separately. 
Moreover, we use $B_A$ in \eqref{eq:define_BA} to bound the follower's Q- and A-function. 
We will leave out the dependence on $(s_h, b_h)$ most of the times in the following proof when it does not cause any confusion in the context.

Note that we only have guarantee for $D_\TV^2(\nu_h, \tilde\nu_h)$ by MLE, which cannot directly guarantee that the true utility is identifiable since a constant shift does not change the follower's behavior at all. For the reward to be identifiable, we need an additional linear constraint, namely $\inp{x}{r_h(s_h, a_h, \cdot)}=\varsigma$.
We start with the easier part with the transition kernel.
\begin{align*}
    \EE\osbr{\orbr{( P_h^\pi-\tilde P_h^{\pi})\tilde V_{h+1}}^2}\le 2^2 B_A^2 \EE\sbr{D_\TV^2( P_h^\pi, \tilde  P_h^\pi)}.
\end{align*}
For the follower's reward, we take a real number $\xi$ and have the following decomposition
\begin{align*}
    &\inf_{\xi\in\RR}\EE_{s_h}\abr{r_h^\pi-\tilde r_h^\pi - \xi}  \nend
    &\quad = \inf_{\xi\in\RR}\EE_{s_h}\bigabr{Q_h-\tilde Q_h - \xi - \gamma\bigrbr{ P_h^\pi-\tilde P_h^\pi}\tilde V_{h+1} - \gamma  P_h^\pi\bigrbr{V_{h+1}-\tilde V_{h+1}}}\nend
    &\quad \le \inf_{\xi\in\RR}\EE_{s_h}\bigabr{Q_h-\tilde Q_h - \xi} + \gamma \EE_{s_h}\bigabr{\bigrbr{ P_h^\pi-\tilde P_h^\pi}\tilde V_{h+1} } + \gamma\exp\rbr{2\eta B_A}\EE_{s_h}\bigabr{Q_{h+1} - \tilde Q_{h+1}}, \nend
    &\quad \le \EE_{s_h}\bigabr{A_h-\tilde A_h} + \gamma\EE_{s_h}\bigabr{\bigrbr{ P_h^\pi-\tilde P_h^\pi}\tilde V_{h+1} } + \gamma\exp\rbr{2\eta B_A}\EE_{s_h}\bigabr{Q_{h+1} - \tilde Q_{h+1}}
\end{align*}
where the first inequality holds by the same argument for $V-\tilde V$ in \eqref{eq:f_2-1}, and the second inequality holds simply by plugging $\xi = V_h(s_h) - \tilde V_h(s_h)$. Now, we can plug in the bound for $\EE_{s_h}\oabr{A_h-\tilde A_h}$ in \Cref{lem:response diff} and obtain
\begin{align}
    &\inf_{\xi\in\RR}\EE_{s_h}\abr{r_h^\pi-\tilde r_h^\pi - \xi} \nend
    &\quad \le \underbrace{2(\eta^{-1}+2B_A) D_\TV(\nu_h, \tilde\nu_h) + 2 \gamma B_A \EE_{s_h} D_\TV( P_h^\pi, \tilde P_h^\pi)}_{\ds \sD_h} + \gamma\exp\rbr{2\eta B_A}\EE_{s_h}\bigabr{Q_{h+1} - \tilde Q_{h+1}}.
    % &\le \rbr{2\eta^{-1}(1+2\eta B_A)+2\gamma B_U} D_{\RL,h}(\tilde M,M^*;\pi) + \gamma\exp\rbr{2\eta B_A}\EE_{s_h}\abr{Q_{h+1} - \tilde Q_{h+1}}, 
    \label{eq:f_2-r-diff}
\end{align}
We next show what we can say about the utility when combining the guarantee of \eqref{eq:f_2-r-diff} with the linear constraint $\inp{x}{r_h(s_h, a_h, \cdot)}=\varsigma$. Specifically, we have the following lemma.
\begin{lemma}[Identification of the follower's utility]\label{lem:identification}
    Suppose for $r, \tilde r:\cB\rightarrow \RR$, for some distribution $\nu\in\Delta(\cB)$ such that $\nu>0$, we have $\inf_{\xi\in\RR}\inp{\nu}{\abr{r-\tilde r-\xi}}\le \varepsilon$ and $\inp{x}{r-\tilde r}=0$ hold at the same time for some $x:\cB\rightarrow \RR$ such that $\inp{\ind}{x}\neq 0$. We have
    \begin{align*}
        \inp{\nu}{\abr{r-\tilde r}}\le\rbr{1 + \nbr{\frac x \nu}_\infty \cdot \frac{1}{\abr{\inp{x}{\ind}}}} \epsilon
    \end{align*}
    \begin{proof}
        See \Cref{sec:proof-identification} for a detailed proof.
    \end{proof}
\end{lemma}
With \Cref{lem:identification}, we conclude with $\nbr{\nu}_\infty\ge \exp\rbr{-\eta B_A}$ and $\kappa = \nbr{x}_\infty/|\la x, \ind\ra|$ that
\begin{align*}
    \EE_{s_h}\abr{r_h^\pi-\tilde r_h^\pi} &\le \rbr{1+\exp\rbr{2\eta B_A} \kappa} \bigrbr{\sD_h + \gamma \exp\rbr{2\eta B_A}\cdot\EE_{s_h}\bigabr{Q_{h+1} - \tilde Q_{h+1}}}.
\end{align*}
On the other hand, for the Q-function, we have by \eqref{eq:f_2-Q-ub} that 
\begin{align*}
    \EE_{s_h}\bigabr{Q_h - \tilde Q_h}
    &\le \EE_{s_h}\bigabr{r_h^\pi-\tilde r_h^\pi + \gamma \bigrbr{ P_h^\pi-\tilde P_h^\pi}\tilde V_{h+1}} + \gamma \exp\rbr{2\eta B_A} {\EE_{s_h}\bigabr{Q_{h+1}-\tilde Q_{h+1}}}\nend
    &\le \underbrace{2\rbr{1+\exp\rbr{2\eta B_A} \kappa} }_{\ds c_1}\cdot \sD_h \nend
    &\qquad + \underbrace{\rbr{2+\exp\rbr{2\eta B_A} \kappa}\gamma \exp\rbr{2\eta B_A}}_{\ds c_2}\cdot\EE_{s_h}\bigabr{Q_{h+1} - \tilde Q_{h+1}}, 
\end{align*}
where in the last inequality, we directly upper bound $\EE_{s_h}|\gamma(P_h^\pi - \tilde P_h^\pi)\tilde V_{h+1}|$ by $\sD_h$. 
Therefore, we have by a recursive argument that 
\begin{align*}
    \EE_{s_h}\bigabr{Q_h -\tilde Q_h} \le \sum_{l=h}^H c_2^{l-h} c_1 \EE_{s_h}\sD_l.
\end{align*}
For now, we are able to deal with $(\EE_{s_h}\abr{r_h^\pi-\tilde r_h^\pi})^2$. However, note that what we actually want to get is the version with the square within the expectation $\EE_{s_h}$, i.e.,  $\EE\rbr{r_h^\pi-\tilde r_h^\pi}^2$. Therefore, we need a variance-mean decomposition,
\begin{align*}
    \EE\rbr{r_h^\pi-\tilde r_h^\pi}^2&= \EE\bigrbr{Q_h-\tilde Q_h - \gamma \bigrbr{ P_h^\pi -\tilde P_h^\pi}\tilde V_{h+1} -\gamma  P_h^\pi \bigrbr{V_{h+1}-\tilde V_{h+1}}}^2\nend
    &\le 4\EE\bigsbr{\bigrbr{A_h -\tilde A_h}^2 + \bigrbr{V_h-\tilde V_h}^2 + \gamma^2 \bigrbr{\bigrbr{ P_h^\pi -\tilde P_h^\pi}\tilde V_{h+1}}^2 +\gamma^2 \bigrbr{V_{h+1}-\tilde V_{h+1}}^2}\nend
    &\le 4\EE\bigsbr{\bigrbr{A_h -\tilde A_h}^2} + 16\gamma^2 B_A^2\EE\bigsbr{D_\TV( P_h^\pi,\tilde  P_h^\pi)^2}\nend
    &\qquad + 4\exp\rbr{4\eta B_A}\EE\bigsbr{\bigrbr{\EE_{s_h}\bigabr{Q_h-\tilde Q_h}}^2+ \bigrbr{\EE_{s_{h+1}}\bigabr{Q_{h+1}-\tilde Q_{h+1}}}^2}, 
\end{align*}
where in the first inequality, we use $Q=A+V$, and use the Jensen's inequality to derive the last term.
The last inequality holds by noting the upper bound for difference in the V-function used in \Cref{eq:f_2-1}.
We notice that the first term can be upper bounded by the squared Hellinger distance,
\begin{align*}
    D_\H^2\rbr{\nu_h,\tilde\nu_h} &= \dotp{\nu_h}{\rbr{1-\sqrt\frac{\tilde \nu_h}{\nu_h}}^2}_\cB\nend
    &= \Bigdotp{\nu_h}{\rbr{1-\exp\rbr{\frac \eta 2 (\tilde A_h - A_h)}}^2}_\cB\nend
    &\ge \rbr{\frac{1-\exp\rbr{-\eta B_A}}{2 B_A}}^2  \cdot \bigdotp{\nu_h}{\bigrbr{A_h-\tilde A_h}^2}_\cB \ge \rbr{\frac{\eta}{2}}^2  \cdot \bigdotp{\nu_h}{\bigrbr{A_h-\tilde A_h}^2}_\cB,
\end{align*}
where the first inequality holds by noting that $|1-\exp(x)|\ge (1-\exp(-B))|x|/B$ for any $|x|\le B$.
The last inequality uses the inequality $(1-\exp(-x))\ge x/(1+x)$ for all $x>0$. 
Therefore, we have the follower's squared reward difference bounded by
\begin{align*}
    &\EE\rbr{r_h^\pi -\tilde r_h^\pi}^2 \nend
    &\quad \le 16\eta^{-2} \EE D_\H^2(\nu_h,\tilde\nu_h) + 16\gamma^2 B_A^2\EE D_\TV^2( P_h^\pi,\tilde  P_h^\pi)\nend
    &\qqquad + 4 \exp\rbr{4\eta B_A} \EE\sbr{\rbr{\sum_{l=h}^H c_2^{l-h} c_1 \EE_{s_h}\sD_l}^2 + \rbr{\sum_{l=h+1}^H c_2^{l-h} c_1 \EE_{s_{h+1}}\sD_l}^2}\nend
    &\quad \le 16\eta^{-2} \EE D_\H^2(\nu_h,\tilde\nu_h) + 16\gamma^2 B_A^2\EE D_\TV^2( P_h^\pi,\tilde  P_h^\pi)\nend
    &\qqquad + 8 H \eff_H(c_2)^2 c_1^2 \exp\rbr{4\eta B_A} \max_{h\in[H]}\EE\sbr{{\sD_h}^2}\nend
    &\quad \le 16\eta^{-2} \EE D_\H^2(\nu_h,\tilde\nu_h) + 16\gamma^2 B_A^2\EE D_\TV^2( P_h^\pi,\tilde  P_h^\pi)\nend
    &\qqquad + 8 H \eff_H(c_2)^2 c_1^2\exp\rbr{4\eta B_A} \max_{h\in[H]}\EE\sbr{\rbr{{2(\eta^{-1}+2B_A) D_\TV(\nu_h, \tilde\nu_h) + 2 \gamma B_A \EE_{s_h} D_\TV( P_h^\pi, \tilde P_h^\pi)}}^2}, 
\end{align*}
where in the second inequality, we uses the Cauchy-Schwartz inequality that $\EE(\sum a_l x_l)^2\le \sum a_l \cdot \EE\sum a_l x_l^2 \le (\sum a_l)^2 \cdot \max_l \EE b_l^2$ for constant sequence $a_l>0$. 
In summary, we have
\begin{align*}
    \EE\rbr{r_h^\pi -\tilde r_h^\pi}^2
    &\le {32 H^2 \eff_H(c_2)^2 c_1^2\exp\rbr{4\eta B_A} \rbr{4(\eta^{-1}+2B_A)^2+4\gamma^2 B_A^2 }}  \nend
    & \qquad \cdot \max_{h\in[H]}\cbr{\EE D_\H^2(\nu_h,\tilde\nu_h)+\EE D_\TV^2(P_h^\pi,\tilde P_h^\pi)}\nend
    &\le \underbrace{640 H^2 \eff_H(c_2)^2 c_1^2\exp\rbr{4\eta B_A} (\eta^{-1}+B_A)^2}_{\ds c_3/4}  \nend
    & \qquad \cdot \max_{h\in[H]}\cbr{\EE D_\H^2(\nu_h,\tilde\nu_h)+\EE D_\TV^2(P_h^\pi,\tilde P_h^\pi)}
\end{align*}
Therefore, we conclude that
\begin{align*}
    \max_{h\in[H]}\EE\sbr{ \bigrbr{{\tilde Q_h - r_h^\pi - \gamma P_h^\pi \tilde V_{h+1}}}^2} 
    &\le 2 \max_{h\in[H]} \EE\sbr{\rbr{\tilde r_h^\pi - r_h^\pi}^2} + 2 \gamma^2\max_{h\in[H]} \EE\sbr{\bigrbr{\bigrbr{\tilde P_h^\pi - P_h^\pi}\tilde V_{h+1}}^2}\nend
    &\le c_3 \max_{h\in[H]}\cbr{\EE D_\H^2(\nu_h,\tilde\nu_h)+\EE D_\TV^2(P_h^\pi,\tilde P_h^\pi)}, 
\end{align*}
which completes our proof of \Cref{lem:2nd-ub}
% where the first inequality holds by noting that $C_\eta\le 2(1+2\eta B_A)$. 
% \newpage
% which completes the proof of \Cref{cor:online linear}.


\subsubsection{Proof of \Cref{lem:identification}}\label{sec:proof-identification}
For condition $\inf_{\xi\in\RR}\inp{\nu}{\abr{r-\tilde r-\xi}}\le \varepsilon$, we assume that the infimum is achieved at $\xi^*$. Let $r^* = r -\xi^*$ and we have
\begin{align*}
\abr{\inp{r^*-\tilde r}{x}} \le \inp{\abr{r^*-\tilde r}}{\abr{x}} \le \inp{\abr{r^*-\tilde r}}{\nu} \cdot \nbr{\frac{x}{\nu}}_\infty\le \varepsilon\nbr{\frac{x}{\nu}}_\infty,
\end{align*}
where the second inequality is just a distribution shift and the last inequality is given by the condition. Furthermore, for our target,
\begin{align*}
    \inp{\abr{r-\tilde r}}{\nu} \le \inp{\abr{r^*-\tilde r}}{\nu} + \abr{\xi^*} = \varepsilon + \abr{\inp{r^*-r}{x}} \cdot \frac{1}{\abr{\inp{x}{\ind}}},
\end{align*}
where the inequality follows from the triangle inequality and the equality holds by noting that $\inp{x}{\ind}\neq 0$ and $\inp{\abr{r^*-\tilde r}}{\nu}\le \varepsilon$. We bridge these two inequalities by noting that
\begin{align*}
    \abr{\inp{r^*-\tilde r}{x}} = \abr{\inp{r-\tilde r}{x} + \inp{r^*-r}{x} } = \abr{\inp{r^*-r}{x} },
\end{align*}
where the second inequality holds by noting that $\inp{r-\tilde r}{x}=0$. Combining these results and we have
\begin{align*}
    \inp{\abr{r-\tilde r}}{\nu} \le \epsilon + \abr{\inp{r^*-r}{x}} \cdot \frac{1}{\abr{\inp{x}{\ind}}} \le \rbr{1 + \nbr{\frac x \nu}_\infty \cdot \frac{1}{\abr{\inp{x}{\ind}}}} \epsilon, 
\end{align*}
which completes the proof on \Cref{lem:identification}.

