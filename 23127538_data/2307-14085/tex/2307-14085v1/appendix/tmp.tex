
% In probability theory, a filtration refers to a sequence of sigma-algebras that represent the information available at different points in time in a stochastic process.

% Formally, let $(\Omega, \mathcal{F}, (\mathcal{F}t){t\geq 0}, \mathbb{P})$ be a filtered probability space, where $(\Omega, \mathcal{F}, \mathbb{P})$ is a probability space and $(\mathcal{F}t){t\geq 0}$ is a family of sigma-algebras (i.e., collections of events) that satisfy the following conditions:

% $\mathcal{F}_0$ contains all the certain events (i.e., ${\emptyset, \Omega}$).
% $\mathcal{F}_t \subseteq \mathcal{F}$ for all $t \geq 0$ (i.e., each sigma-algebra is a subset of the underlying probability space).
% $\mathcal{F}_s \subseteq \mathcal{F}_t$ for all $0 \leq s \leq t$ (i.e., each sigma-algebra contains all the information in the previous sigma-algebras).
% In this context, a stochastic process $(X_t)_{t\geq 0}$ is said to be adapted to the filtration $(\mathcal{F}t){t\geq 0}$ if for all $t\geq 0$, the random variable $X_t$ is $\mathcal{F}_t$-measurable (i.e., the value of $X_t$ can be determined based on the information available at time $t$).

% Therefore, when we say that $(X_1, X_2, \ldots, X_N)$ is adapted to a filtration $(\mathcal{F}t){t\geq 0}$, it means that each random variable $X_i$ depends only on the information available up to time $t_i$, and not on any future information. In other words, the information available at time $t_i$ is sufficient to determine the value of $X_i$.