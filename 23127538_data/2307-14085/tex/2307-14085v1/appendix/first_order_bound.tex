% \subsection{Proof of \Cref{lem:1st-ub} on Bounding the First Order Term by $D_\RL$} \label{sec:1st-ub}
% In this section, we will bound $f_{1, h}^\tM(\pi)$ by $D_\RL$ in the following way.
% % \paragraph{For Small $\eta \bigabr{\tilde A- A}$.}
% % On the one hand, we have by the Cauchy-Schwartz inequality that
% % \begin{align*}
% %     \EE \exp\rbr{\eta \bigabr{\tilde A- A}} \cdot \EE \sbr{\exp\rbr{- \eta \bigabr{\tilde A- A}}\cdot \bigabr{\tilde A-A}^2} \ge \rbr{\EE\sbr{\bigabr{\tilde A-A}}}^2.
% % \end{align*}
% % On the other hand, the second term on the left hand side can be bounded by the Hellinger distance, 
% % \begin{align*}
% %     \EE D_\H^2(\nu, \tilde\nu) = \EE \inp[\bigg]{\nu}{\biggrbr{1-\sqrt{\frac{\tilde\nu}{\nu}}}^2} \ge \EE \sbr{\eta^2 \exp\rbr{- \eta \bigabr{\tilde A- A}}\cdot \bigabr{\tilde A-A}^2}.
% % \end{align*}
% % Hence, we conclude that
% % \begin{align*}
% %     \EE \exp\rbr{\eta \bigabr{\tilde A- A}} \cdot \EE D_\H^2(\nu, \tilde\nu)\ge \eta^2  \cdot \rbr{\EE\bigabr{\tilde A-A}}^2,
% % \end{align*}
% % and also
% % \begin{align}
% %     \sum_{i=h}^H \gamma^{i-h}\EE \exp\rbr{\eta \bigabr{\tilde A_i- A_i}} \cdot \sum_{i=h}^H \gamma^{i-h}\EE D_\H^2 \rbr{\nu_i, \tilde\nu_i} \ge \eta^2 \rbr{\sum_{i=h}^H \gamma^{i-h} \EE \bigabr{\tilde A_i-A_i}}^2.\label{eq:A-cauchy}
% % \end{align}
% To bridge the difference in the A-function to $f_{1,h}^\tM(\pi)$, we follow from the decomposition of the A-function in \Cref{lem:AQV-func diff},
% \begin{align*}
%     \abr{\rbr{\EE_{s_h, b_h}-\EE_{s_h}} \Delta_h^{(1)}} 
%     &\le \bigabr{A_h-\tilde A_h} + 2\eta^{-1} \Delta_h^{(2)}(s_h) \le \bigabr{A_h-\tilde A_h} + 2\EE_{s_h}\sbr{\sum_{i=h}^H \gamma^{i-h} \inp[]{\nu_i}{A_i-\tilde A_i}}, 
% \end{align*}
% where we use the definition $\Delta_h^{(2)}(s_h)\defeq \EE_{s_h}\sbr{\sum_{i=h}^H \gamma^{i-h} \kl\infdivx[]{\nu_i}{\tilde\nu_i}}$ and the last inequality holds by noting that $\kl\infdivx[]{\nu}{\tilde\nu} = \eta\inp[]{\nu}{A-\tilde A}$.
% Therefore, we conclude that
% \begin{align}
%     f_{1, h}^\tM(\pi) 
%     &\le  \EE\bigabr{A_h-\tilde A_h} + 2\EE\sbr{\sum_{i=h}^H \gamma^{i-h} \inp[]{\nu_i}{A_i-\tilde A_i}}
%     % \nend
%     % &
%     \le 3 \sum_{i=h}^H \gamma^{i-h} \EE\bigabr{A_i-\tilde A_i} \label{eq:Delta-A}
%     % \\
%     % &\le 3 \eta^{-1} \sqrt{\sum_{i=h}^H \gamma^{i-h}\EE \exp\rbr{\eta \bigabr{\tilde A_i- A_i}}} \cdot D_\RL(M^*,\tilde M;\pi),\nonumber
% \end{align}
% % where the last inequality is a direct result of \eqref{eq:A-cauchy}.
% % \paragraph{For Large $\eta \bigabr{\tA - A}$. }
% We now invoke the lower bound \eqref{eq:nu-tv-lb-1} in \Cref{lem:response diff} and obtain
% \begin{align*}
%     D_\TV(\nu_h, \tnu_h) \ge \frac{1-\exp\rbr{-2\eta B_A}}{4 B_A} \cdot {\EE_{s_h}\bigabr{\tA_h-A_h} }.
% \end{align*}
% Combining these results, we obtain
% \begin{align*}
%     f_{1,h}^{\tM}(\pi) &\le 3\sum_{i=h}^H \gamma^{i-h} \EE\bigabr{\tA_i-A_i} \nend
%     &\le 3 \cdot \rbr{\frac{1-\exp\rbr{-2\eta B_A}}{4 B_A}}^{-1} \sum_{i=h}^H \gamma^{i-h}\EE D_\TV(\nu_h,\tilde\nu_h) \nend
%     &\le 6(1+2\eta B_A)\cdot  \sqrt{\frac{1-\gamma^H}{1-\gamma}}  \cdot \eta^{-1} \max_{h\in[H]} \EE D_\TV(\nu_h,\tilde\nu_h),
% \end{align*}
% where the last inequality follows from from the fact that $(1-\exp(-x))/2x\ge 1/2(1+x)$.
% % Hence, we have
% % \begin{align*}
% %     f_{1,h}^{\tM}(\pi) &\le   3 \cdot \frac{4 \eta B_A}{1-\exp\rbr{-2\eta B_A}}\cdot
% %     \sqrt{\frac{1-\gamma^H}{1-\gamma}}  \cdot
% %     \eta^{-1}D_\RL(M^*,\tilde M;\pi)\nend
% %     &\le 6(1+2\eta B_A)\cdot  \sqrt{\frac{1-\gamma^H}{1-\gamma}}  \cdot \eta^{-1}D_\RL(M^*,\tilde M;\pi),
% % \end{align*}
% Hence, we complete the proof of \Cref{lem:1st-ub}.