


\section{Auxiliary Results and Their Proofs}


In this section, 
we present and prove 
the auxiliary lemmas that helps theoretical analysis. 

\subsection{Auxiliary Results for Learning Quantal Response Mapping}

In this subsection, we introduce some technical lemmas used for learning follower's quantal response mapping. 
To simplify the notation, similar to the statement of \Cref{lem:performance diff}, we fix a 
policy $\pi$, and let $\tilde Q$ be an estimate of $Q^{\pi}$. 
Moreover, for Lemmas \ref{lem:response diff}--\ref{lem:KL-ub},  we consider a fixed state $s$ at a fixed timestep $h\in [H]$, and with slight abuse of notation, omit it in the value functions. 
In particular, for $\tilde V$ and $\tilde A$ defined in \eqref{eq:tilde_functions}, in the sequel, we write 
$\tilde Q = \tilde Q_h (s, \cdot )$, $\tilde V = \tilde V_h(s)$, and $\tilde A = \tilde A_h(s, \cdot)$. 
Furthermore, we write  $Q = Q^{\pi}_h (s, \cdot )$, $V = V^{\pi}_h (s )$, and $A = A^{\pi}_h (s, \cdot )$. 
Finally, 
we  write $\nu^{\pi} _h(\cdot \given s)$ and $\tilde \nu_h (\cdot \given s)$   as $\nu \in \Delta(\cB)$ and $\tilde \nu \in \Delta(\cB)$ respectively.
Using this notation, 
$Q, \tilde Q,$,$A $, and $\tilde A$ can all be regarded as vectors in $\RR^{|\cB|}$, $ V, \tilde V \in \RR$, and we have 
\#\label{eq:define_nu_vecs}
\nu(b) = \exp \big (\eta \cdot A(b) \bigr ), \qquad \tilde \nu (b) = \exp\big (\eta \cdot \tilde A(b) \bigr ), \forall b\in \cB. 
\#
The following lemma establishes upper and lower bounds on the total variation (TV) distance between $\nu $ and $\tilde \nu$. 

 

\begin{lemma}[TV-Distance Between Quantal Responses]\label{lem:response diff}
Let $\nu, \tilde \nu \in \Delta(\cB)$ be defined in \eqref{eq:define_nu_vecs}, where $\tilde A = \tilde Q - \tilde V$ and $A = Q - V$. 
The TV distance  between $\nu$ and $\tilde\nu$ enjoys the following upper bound:  
\begin{align*}
    D_\TV\orbr{\nu, \tilde \nu}\le \eta \cdot \inp[\Big]{\nu} {\bigabr{\tilde Q-Q-\xi \cdot \ind } + \frac \eta 2 \exp\bigrbr{\eta\bigabr{\tilde Q-Q-\xi \cdot \ind}} \cdot \bigrbr{\tilde Q-Q-\xi \cdot \ind}^2}_{\cB }, \quad \forall \xi\in\RR.
\end{align*}
Here $\ind \in \RR^{|\cB|}$ is an all-one vector. 
In particular, setting $\xi = \tilde V - V$, we have 
\begin{align}\label{eq:nu-tv-ub-0}
    D_\TV\orbr{\nu, \tilde \nu}\le \eta \cdot \inp[\Big]{\nu} {\oabr{\tilde A - A  } + \frac \eta 2 \exp\bigrbr{\eta\oabr{\tilde A - A }}\cdot \orbr{\tilde A - A }^2}_{\cB }, \quad \forall \xi\in\RR.
\end{align}
Besides, 
the following  lower bounds hold: 
\begin{align}
    D_\TV & \orbr{\nu, \tilde \nu} \ge \frac{\eta}{2(1+2\eta B_A)} \cdot \inp[\big]{\nu}{\oabr{\tilde A - A}}_{\cB}, \label{eq:nu-tv-lb-1}\\
    D_\TV & \orbr{\nu, \tilde \nu}\ge \frac 1 2 \inp[\big]{\nu} { \eta \exp\bigrbr{- \eta \oabr{\tilde A- A}}\cdot \oabr{\tilde A-A}}_{\cB }.  \label{eq:nu-tv-lb-2}
\end{align}
Here $B_{A}$ is an upper bound on $\max \{ \| A \|_{\infty}, \| \tilde A \|_{\infty} \} $, which, for example, can be set as in \eqref{eq:define_BA}.
\end{lemma}

%Suppose that $\nu$ is the follower's quantal response for Q-function $Q:\cB\rightarrow\RR$, i.e., $\nu=\argmax_{\nu'\in\Delta(\cB)}\inp[]{\nu'}{Q}_\cB + \eta^{-1} H(\nu)$.  Then, for any two Q-functions $Q$ and $\tilde Q$ and their corresponding quantal responses $\nu$ and $\tilde\nu$, the TV distance between $\nu$ and $\tilde\nu$ enjoys the following upper bound, 

%and also the following lower boundeds
% \begin{gather}
%     D_\TV\bigrbr{\nu, \tilde \nu} \ge \frac{\eta}{2(1+2\eta B_A)} \cdot \inp[\big]{\nu}{\bigabr{\tilde A - A}}, \label{eq:nu-tv-lb-1}\\
%     D_\TV\bigrbr{\nu, \tilde \nu}\ge \frac 1 2 \inp[\Big]{\nu} { \eta \exp\rbr{- \eta \bigabr{\tilde A- A}}\cdot \bigabr{\tilde A-A}}, \label{eq:nu-tv-lb-2}
% \end{gather}
% where $A$ and $\tilde A$ are the A-function for $Q$ and $\tilde Q$, respectively.
\begin{proof}
In this proof, to simplify the notation, we omit the subscript in $\la \cdot , \cdot \ra_{\cB}$ without causing confusion. 
We first prove  the lower bounds. 
Using the relation between TV distance and $\ell_1$-norm, 
we have 
\begin{align}
    D_\TV\bigrbr{\nu, \tilde \nu} = \frac{1}{2} \cdot \| \nu - \tilde \nu \|_1 = \frac{1}{2} \cdot \inp[\Big]{\nu}{\Bigabr{\ind -\frac{\tilde \nu}{\nu}}}  = \frac 1 2 \cdot \inp[\big]{\nu} {\bigabr{\ind -\exp\bigrbr{\eta\orbr{\tilde A - A }}}}. \label{eq:TV-1}
\end{align}
We note that $|\exp(x)-1|\ge (1-\exp(-B)) / B \cdot |x|$ holds for all $x \in [-B, B]$.  
Thus, for any $b \in \cB$, 
we have 
\#\label{eq:TV-11}
\Bigabr{1  -\exp\Bigrbr{\eta\bigrbr{\tilde A  (b) -A (b)  }}} \geq  \frac{1-\exp\rbr{-2 \eta B_A} } {2\eta B_A } \cdot \eta \cdot | \tilde A(b) - A(b) | .
\#
Hence, combining \eqref{eq:TV-1} and \eqref{eq:TV-11},  we have
\begin{align*}
    &\frac 1 2 \cdot \inp[\big]{\nu} {\bigabr{\ind -\exp\bigrbr{\eta\orbr{\tilde A - A }}}}
    \ge \frac \eta 2 \cdot \frac{1-\exp\bigrbr{-2 \eta B_A}}{2\eta B_A} \cdot \inp[]{\nu}{\oabr{\tilde A - A}} \geq \frac \eta 2 \cdot \frac{1}{1+2\eta B_A}\cdot \inp[]{\nu}{\oabr{\tilde A - A}}.
\end{align*}
where the last inequality holds by noting that 
$$
1 - \exp(-x ) = \frac{\exp(x)-1}{\exp(x)-1 + 1}\ge \frac{x}{1+x}, \quad \forall x>0.
$$
Therefore, we establish \eqref{eq:nu-tv-lb-1}.

Meanwhile,  we note that $ | \exp(x) - 1 | \geq \exp( -  | x| ) \cdot x $ holds for all $x \in \RR$, which indicates that
\begin{align*}
    &\frac 1 2 \cdot \inp[\big]{\nu} {\bigabr{\ind -\exp\bigrbr{\eta\orbr{\tilde A - A }}}}\ge 
    \frac 1 2 \inp[\big]{\nu} { \eta \exp\bigrbr{- \eta \oabr{\tilde A- A}}\cdot \oabr{\tilde A-A}}.
\end{align*}
Thus, we establish \eqref{eq:nu-tv-lb-2}.

It remains to establish an upper bound 
for the right hand side of \eqref{eq:TV-1}. Note that 
$A = Q - V$ and $\tilde A = \tilde Q - \tilde V$. 
Then we have 
\begin{align*}
    &\frac 1 2 \cdot \inp[\big]{\nu} {\bigabr{\ind -\exp\bigrbr{\eta\orbr{\tilde A - A }}}}  \nend
    &\quad = \frac 1 2 \cdot \inp[\big]{\nu} {\bigabr{\ind -\exp\bigrbr{\eta\orbr{\tilde Q - Q - \tilde V +  V }}}}  \nend
    &\quad \le \frac 1 2 \bigdotp{\nu}{\bigabr{\ind - \exp\bigrbr{\eta\bigrbr{\tilde Q - Q}}}} + \frac 1 2 \bigdotp{\nu}{\bigabr{\exp\bigrbr{\eta\bigrbr{\tilde Q - Q}} - \exp\bigrbr{\eta\bigrbr{\tilde Q - Q -\tilde V+V}}}}\nend
    &\quad = \frac 1 2 \cdot \inp[\big]{\nu} {\bigabr{\ind - \exp\bigrbr{\eta\orbr{\tilde Q- Q}}}} + \frac 1 2 \inp[\big]{\ind}{\bigabr{\exp\bigrbr{\eta\bigrbr{\tilde Q -V}}-\exp\bigrbr{\eta\bigrbr{\tilde Q -\tilde V}}}},
\end{align*}
where the inequality is just a split of terms and the last equality holds by the definition $\nu=\exp\rbr{\eta(Q-V)}$. Using the equality $\inp[]{\ind}{\exp(\eta\tilde Q)}=\exp(\eta \tilde V)$ by definition of $\tilde V$, and noting that $V$ and $\tilde V$ does not depend on $\cB$, we further obtain
\begin{align}
    &\frac 1 2 \cdot \inp[\big]{\nu} {\bigabr{\ind -\exp\bigrbr{\eta\orbr{\tilde A - A }}}}  \nend
    &\quad \le \frac 1 2 \cdot \inp[\big]{\nu} {\bigabr{\ind - \exp\bigrbr{\eta\orbr{\tilde Q- Q}}}} + \frac 1 2 \inp[\Big]{\exp\bigrbr{\tilde Q}}{\bigabr{\exp\bigrbr{\eta\bigrbr{-V}}-\exp\bigrbr{\eta\bigrbr{-\tilde V}}}} \nend
    &\quad = \frac 1 2 \cdot \inp[\big]{\nu} {\bigabr{1- \exp\bigrbr{\eta\bigrbr{\tilde Q- Q}}}} + \frac 1 2 \exp\bigrbr{\eta \tilde V}\cdot \bigabr{\exp\rbr{-\eta V} - \exp\bigrbr{-\eta\tilde V}}, \label{eq:TV-2}
\end{align}
Moreover, the second term on the right hand side of \eqref{eq:TV-2} can be bounded by the same trick,
\begin{align}
    \frac 1 2 \exp\bigrbr{\eta \tilde V}\cdot \bigabr{\exp\bigrbr{-\eta V} - \exp\bigrbr{-\eta\tilde V}}
    &= \frac 1 2 \exp\rbr{-\eta V}\cdot \bigabr{\exp\bigrbr{\eta \tilde V} - \exp\rbr{\eta V}}\nend
    &=\frac 1 2 \exp\bigrbr{-\eta V}\cdot \bigabr{\inp[\big]{\ind}{\exp\bigrbr{\eta \tilde Q}-\exp\bigrbr{\eta Q}}}\nend
    &= \frac 1 2 \bigabr{\inp[\big]{\nu}{\exp\bigrbr{\eta\tilde Q-\eta Q}-1}}, \label{eq:TV-3}
\end{align}
where the last equality uses the fact that $\nu = \exp(\eta Q -\eta V)$. 
Plugging \eqref{eq:TV-3} into \eqref{eq:TV-2} gives
% \begin{align*}
%     D_\TV\orbr{\nu, \tilde \nu}
%     \le \inp[\big]{\nu} {\bigabr{\ind - \exp\bigrbr{\eta\orbr{\tilde Q- Q}}}}
%     \le \eta \cdot \exp\rbr{\eta \bignbr{\tilde Q-Q}_\infty}\cdot \inp[\Big]{\nu} {\bigabr{\tilde Q-Q}}, 
% \end{align*}
% where the inequality holds by the Lipschitz continuity relationship.
% Also, using Taylor expansion of $\exp(x)-1$ at $x=0$ gives 
\begin{align*}
    &\frac 1 2 \cdot \inp[\big]{\nu} {\bigabr{\ind -\exp\bigrbr{\eta\orbr{\tilde A - A }}}} \nend
    &\quad \le
    \inp[\big]{\nu} {\bigabr{\ind - \exp\bigrbr{\eta\orbr{\tilde Q- Q}}}}\nend
    &\quad \le\eta \cdot \inp[\big]{\nu} {\oabr{\tilde Q-Q} + \frac \eta 2 \cdot \exp\orbr{\eta\oabr{\tilde Q-Q}}\oabr{\tilde Q-Q}^2},
\end{align*}
where the last inequality holds by a Taylor expansion of $\exp(x)-1$ at $x=0$.
Moreover, note that adding a constant shift $\xi\ind$ to both $\tilde Q-Q$ does not change $\nu$ nor $\tilde \nu$. Hence, we finish the proof of the upper bound.
\end{proof} 




The next lemma establishes a lower bound on the Hellinger distance between $\tilde \nu$ and $\nu$.

\begin{lemma}[Hellinger distance Between Quantal Responses]\label{lem:D_H-2-A^2}


    Let $\nu $ and $\tilde \nu$ be defined in \eqref{eq:define_nu_vecs}. 
    We let $D_\H (\cdot, \cdot)$ denote the Hellinger distance between probability distributions. 
    Then we  have
    \begin{align*}
        D_\H^2\rbr{\nu, \tilde\nu} \ge  \frac{\eta^2}{8(1+\eta B_A)^2} \cdot \inp[\big]{\nu}{ (\tilde A-A)^2}_{\cB}.
    \end{align*}
\end{lemma}
    \begin{proof}
        In this proof, to simplify the notation, we omit the subscript in $\la \cdot , \cdot \ra_{\cB}$.
         By the  definition of the Hellinger distance,  we have 
        \begin{align*}
            D_\H^2(\nu, \tilde\nu) &= \frac 1 2 \inp[\big]{\nu}{\bigrbr{1-\sqrt{ \tilde\nu / \nu}}^2} 
            =\frac 1 2 \inp[\big]{\nu}{\bigabr{\ind -\exp\bigrbr{\eta/2 \cdot \orbr{\tilde A-A} } } ^2}\nend
            &\ge \frac 1 8 \rbr{\frac{1-\exp\rbr{-\eta B_A}}{B_A}}^2\cdot \inp[\big]{\nu}{ (\tilde A-A)^2} \ge  \frac{\eta^2}{8(1+\eta B_A)^2} \cdot \inp[\big]{\nu}{ (\tilde A-A)^2}.
        \end{align*}
        where the first inequality follows from $|\exp(x)-1|\ge (1-\exp(-B))|x|/B$ for any $|x|\le B$, and the last inequality holds by noting that 
        $$
        1 - \exp(-x ) = \frac{\exp(x)-1}{\exp(x)-1 + 1}\ge \frac{x}{1+x}, \quad \forall x>0.
        $$
        Therefore, we conclude the proof. 
    \end{proof} 

    The next lemma establishes an upper bound on the KL-divergence between $\tilde \nu$ and $\nu$ in terms of difference of Q-functions. 


\begin{lemma}[KL-Divergence Between Quantal Responses]\label{lem:KL-ub}
    Let $\nu $ and $\tilde \nu$ be defined in \eqref{eq:define_nu_vecs}. We have
    \begin{align*}
      \kl\infdivx[]{\nu}{\tilde\nu} \le \eta \cdot \inp[\big]{\nu - \tilde\nu}{Q-\tilde Q}_{\cB}.
    \end{align*}
\end{lemma}
    \begin{proof}
        To simplify the notation, we omit the subscript in $\la \cdot , \cdot \ra_{\cB}$.
        By the  definition of the KL divergence, we have 
        \begin{align*}
            \eta^{-1}\kl\infdivx[]{\nu}{\tilde\nu} &= \inp[\big]{\nu}{A-\tilde A}\nend
            & = \inp[\big]{\nu}{Q-\tilde Q} - \orbr{V-\tilde V}\nend
            & = \inp[\big]{\nu}{Q-\tilde Q} -\bigcbr{ \max_{\nu'}\inp{\nu'}{Q} + \eta^{-1} \cH(\nu')} + \bigcbr{ \max_{\nu'}\inp{\nu'}{\tilde Q} + \eta^{-1} \cH(\nu')},
        \end{align*}
where the second equality holds because $V$ and $\tilde V$ are real numbers, and the last equality follows from the optimality condition of the quantal response. 
Note that the optimal solutions to the optimization problems are $\nu$ and $\tilde \nu$, respectively. 
Thus, we have 
\$
-\bigcbr{ \max_{\nu'}\inp{\nu'}{Q} + \eta^{-1} \cH(\nu')} + \bigcbr{ \max_{\nu'}\inp{\nu'}{\tilde Q} + \eta^{-1} \cH(\nu')} \leq \la \tilde \nu, - Q + \tilde Q\ra, 
\$
where we replace $\tilde \nu'$ by $\tilde \nu$ in the first 
 maximization. 
 Therefore, combining the above equations above, we complete  the proof.
    \end{proof}







As shown in Lemma \ref{lem:response diff}, 
the estimation error  of the follower's A-function plays a central role in the analysis of quantal response. 
The next lemma  characterizes  $A_h-\tilde A_h$. 
To the following, we follow the same  notation as  in \Cref{lem:performance diff}. Besides, 
to simplify the presentation, we define  
\begin{equation*}
\begin{aligned}
    \Delta_h^{(1)} (s_h, b_h)&\defeq  \EE_{s_h, b_h}\sbr{\sum_{i=h}^H \gamma^{i-h}\bigrbr{r_i^\pi + \gamma P_{i}^\pi\tilde V_{i+1} - \tilde Q_i}(s_i,b_i)}, \\
    \Delta_h^{(2)}(s_h) &\defeq \EE_{s_h}\sbr{\sum_{i=h}^H \gamma^{i-h} \kl\infdivx[]{\nu_i(\cdot\given s_i)}{\tilde\nu_i(\cdot\given s_i)}}, 
\end{aligned}
\end{equation*}
where $\EE_{s_h, b_h}$ is a short hand of $\EE^{\pi, M^*}\sbr{\cdot\given s_h, b_h}$ and $\EE_{s_h}$ is a short hand of $\EE^{\pi, M^*}\sbr{\cdot\given s_h}$, and $\EE^{\pi, M^*}$ denotes the expectation with respect to the randomness of the trajectory induced by $(pi, \nu^{pi})$ under the true model $M^*$. 
\begin{lemma}[Difference  in  A-Functions]\label{lem:AQV-func diff}
We follow the same notation  as in \cref{lem:performance diff}. 
For any $h \in [H]$ and $(s_h, b_h) \in \cS\times \cB$, we have  
    \begin{align*}
        &A_h(s_h,b_h)-\tilde A_h(s_h, b_h) = \rbr{\EE_{s_h, b_h}-\EE_{s_h}} \sbr{\Delta_h^{(1)}(s_h, b_h) - \gamma\eta^{-1}\Delta_{h+1}^{(2)} (s_{h+1})} + \eta^{-1}\kl\infdivx[]{\nu_h}{\tilde \nu_h}, 
    \end{align*}
    where  $\kl\infdivx[]{\nu_h}{\tilde\nu_h} = \eta\cdot \EE_{s_h}\bigsbr{A_h-\tilde A_h}$. 
    In particular, in the myopic case, 
    we have 
    \$
    A_h(s_h,b_h)-\tilde A_h(s_h, b_h) = \rbr{\EE_{s_h, b_h}-\EE_{s_h}} \sbr{( r_h^{\pi} - \tilde r_h^{\pi} ) (s_h, b_h)} + \eta^{-1}\kl\infdivx[]{\nu_h}{\tilde \nu_h},
    \$ 
    where $\tilde r$ is an estimate of $r$ that generates $\tilde \nu_h$. 
\begin{proof}
    To simplify the notation, we omit  $s_h, b_h$ in the functions and  omit the subscript in $\la \cdot , \cdot \ra_{\cB}$.
By the optimality of $\nu_h$ and $\tilde \nu_h$, we can write the 
    the difference of V-functions  as follows:
    \begin{align*}
        V_h-\tilde V_h &= \inp[]{\nu_h}{Q_h} + \eta^{-1} \cH(\nu_h) - \inp[]{\tilde \nu_h}{\tilde Q_h} - \eta^{-1} \cH(\tilde \nu_h)\nend
        & =\inp[]{\nu_h}{Q_h-\tilde Q_h} +\inp[]{\nu_h-\tilde\nu_h}{\tilde Q_h} - \inp[]{\nu_h}{Q_h-V_h} + \inp[]{\tilde\nu_h}{\tilde Q_h -\tilde V_h}, 
    \end{align*}
where we use the fact that     
$\eta^{-1}\cH(\nu_h) = - \eta^{-1}\dotp{\nu_h}{\log \nu_h}= -\dotp{\nu_h}{Q_h - V_h}$. 
Note that here both $\tilde V_h$ and $V_h$ are real numbers. 
Then, by direct calculation, 
we have 
\$
\inp[]{\nu_h-\tilde\nu_h}{\tilde Q_h} - \inp[]{\nu_h}{Q_h-V_h} + \inp[]{\tilde\nu_h}{\tilde Q_h -\tilde V_h} = - \inp[\big]{\nu_h}{Q_h - V_h - \bigrbr{\tilde Q_h -\tilde V_h}}, 
\$ 
where we use the fact that   $\inp[]{\tilde \nu_h-\nu_h}{\tilde V_h}=0$ since $\tilde V_h$ does not depend on $b_h$,
Hence, we can write $V_h-\tilde V_h $ as 
    \#  
    V_h-\tilde V_h &= \inp[]{\nu_h}{Q_h-\tilde Q_h} - \inp[\big]{\nu_h}{Q_h - V_h - \bigrbr{\tilde Q_h -\tilde V_h}}\nend
        & = \inp[]{\nu_h}{Q_h-\tilde Q_h} - \eta^{-1} \kl\infdivx[]{\nu_h}{\tilde \nu_h}, \label{eq:V diff}
    \# 
   where the  equality holds by noting that $\kl\infdivx[]{\nu_h}{\tilde\nu_h}= \eta \odotp{\nu_h}{A_h -\tilde A_h} $.
   Thus, by \eqref{eq:V diff}, 
   For the A-function, we have
    \begin{align}
        A_h -\tilde A_h & =Q_h - V_h - \bigrbr{\tilde Q_h -\tilde V_h}= \orbr{\EE_{s_h, b_h}-\EE_{s_h}}\sbr{Q_h-\tilde Q_h}+\eta^{-1}\kl\infdivx[]{\nu_h}{\tilde \nu_h}.  \label{eq:A diff-1}
    \end{align}
Meanwhile, by the Bellman equation $Q_h = r_h^{\pi} + \gamma \cdot P_h ^{\pi} V_{h+1}$, 
    we have 
    \begin{align}
        Q_h -\tilde Q_h &= \EE_{s_h, b_h}\sbr{r_h^\pi + \gamma \tilde V_{h+1} - \tilde Q_h + \gamma\orbr{V_{h+1} - \tilde V_{h+1}}}\nend
        &=\EE_{s_h, b_h}\sbr{r_h^\pi + \gamma \tilde V_{h+1} - \tilde Q_h + \gamma \orbr{Q_{h+1}-\tilde Q_{h+1}} - \gamma \eta^{-1} \cdot \kl\infdivx[]{\nu_{h+1}}{\tilde\nu_{h+1}}}\nend
        &= \EE_{s_h, b_h}\sbr{\sum_{i=h}^{H}\gamma^{i-h}\bigrbr{r_i^\pi + \gamma \tilde V_{i+1} - \tilde Q_i - \gamma\eta^{-1}\cdot \kl\infdivx[]{\nu_{i+1}}{\tilde\nu_{i+1}}}},  \label{eq:Q diff}
    \end{align}
where the second equality follows from 
 \eqref{eq:V diff}, and the 
last equality follows from an recursive argument from step $h$ to $H$.
    Plugging \eqref{eq:Q diff} into \eqref{eq:A diff-1}, we have  
    \begin{align*}
        A_h - \tilde A_h &= \rbr{\EE_{s_h, b_h}-\EE_{s_h}}\sbr{\sum_{i=h}^{H}\gamma^{i-h}\bigrbr{r_i^\pi + \gamma \tilde V_{i+1} - \tilde Q_i - \gamma\eta^{-1}\kl\infdivx[]{\nu_{i+1}}{\tilde\nu_{i+1}}}} \nend
        &\qquad + \eta^{-1}\kl\infdivx[]{\nu_h}{\tilde \nu_h}\nend
        & = \rbr{\EE_{s_h, b_h}-\EE_{s_h}} \sbr{\Delta_h^{(1)}(s_h, b_h) - \gamma\eta^{-1}\Delta_{h+1}^{(2)} (s_{h+1})} + \eta^{-1}\kl\infdivx[]{\nu_h}{\tilde \nu_h}, 
    \end{align*}
    which completes the proof.
\end{proof}
\end{lemma}


\subsection{Proof of \Cref{prop:error-prop}}
 \label{proof:prop:error-prop}

 In this subsection, we prove \Cref{prop:error-prop}, which relates the difference between value functions and quantal response policies to the distance between models and leader's policies. 


\begin{proof}
    Recall that 
    $(\tilde Q, \tilde V, \tilde U, \tilde W, \tilde \nu)$ are the value functions and 
quantal response policy associated with $\tilde \pi$, under model $\tilde M$. 
Similarly, $ ( Q,  V,  U,  W,  \nu)$ are the corresponding terms associated with a policy $\pi$ under model $M$.  

By Bellman's equation of the follower and triangle inequality, we have 
    % \begin{align*}
    %     Q_h(s_h, a_h, b_h) &= \rbr{r_h + \gamma P_h V_{h+1}}(s_h, a_h, b_h), \nend
    %     U_h(s_h, a_h, b_h) & = \rbr{u_h + P_h W_{h+1}}(s_h, a_h, b_h), 
    % \end{align*}
    % which establishes that
    \begin{align*}
        \onbr{Q_h - \tilde Q_h}_\infty 
        &\le \bignbr{r_h - \tilde r_h}_\infty + \gamma \cdot \onbr{\tilde V_{h+1}}_\infty \cdot \sup_{s_h, a_h, b_h\in\cS\times\cA\times\cB}\bignbr{P_h(\cdot\given s_h, a_h, b_h)-\tilde P_h(\cdot\given s_h, a_h, b_h)}_1 \nend
        &\qquad + \gamma\cdot  \bignbr{V_{h+1} -\tilde V_{h+1}}_\infty + \bignbr{\tilde r_h + \gamma \tilde P_h \tilde V_{h+1}}_\infty \cdot \sup_{s_h, b_h\in\cS\times\cB}\nbr{\pi_h(\cdot\given s_h, b_h) - \tilde \pi_h(\cdot\given s_h, b_h)}_1\nend
        &\le 2(1 + \gamma B_A) \epsilon + \gamma \cdot \onbr{V_{h+1} -\tilde V_{h+1}}_\infty, 
    \end{align*}
where the last inequality follows from the fact that $
\onbr{\tilde V_{h+1}}_\infty  \leq B_A, \onbr{\tilde r_h+\gamma \tilde P_h\tilde V_{h+1}}_\infty\le 1 + \gamma B_A$, where for the upper bound of the V function, we 
Moreover, by the leader's Bellman equation, we have 
    \begin{align*}
        \bignbr{U_h - \tilde U_h}_\infty 
        &\le \bignbr{u_h - \tilde u_h}_\infty + \bignbr{\tilde W}_\infty \cdot \bignbr{P_h(\cdot\given s_h, a_h, b_h)-\tilde P_h(\cdot\given s_h, a_h, b_h)}_1 + \bignbr{W_{h+1} -\tilde W_{h+1}}_\infty \nend
        &\le (1 + H) \epsilon + \bignbr{W_{h+1} -\tilde W_{h+1}}_\infty.
    \end{align*}
    We next look at the quantal response difference, where we invoke equation (53) of Lemma 5.1 in \citet{chen2022adaptive}, which says that
    \begin{align}
        \nbr{\nu_h(\cdot\given s_h) - \tilde \nu_h(\cdot\given s_h)}_1 \le 4\eta \bignbr{Q_h(s_h, \cdot) -\tilde Q_h(s_h, \cdot) }_\infty.\label{eq:error-prop-nu}
    \end{align}
    Moreover, for the update of the V function, we have
    \begin{align*}
        \bigabr{V_h(s_h) -\tilde V_h(s_h)} &\le \max\cbr{\bigabr{\bigdotp{\nu_h(\cdot\given s_h)}{Q_h(s_h, \cdot) - \tilde Q_h(s_h, \cdot)}_\cB}, \bigabr{\bigdotp{\tilde\nu_h(\cdot\given s_h)}{Q_h(s_h, \cdot) - \tilde Q_h(s_h, \cdot)}_\cB}}\nend
        &\le \bignbr{Q_h(s_h, \cdot) -\tilde Q_h(s_h, \cdot)}_\infty, 
    \end{align*}
    where the first inequality holds by noting that $V_h(s_h) \defeq \max_{\nu'\in\Delta(\cB)} \dotp{\nu'}{Q_h (s_h,\cdot)}_\cB + \eta^{-1}\cH(\nu')$. Therefore, we have $\bignbr{V_h-\tilde V_h}_\infty \le \bignbr{Q_h-\tilde Q_h}_\infty$.
    And for the W function, we have
    \begin{align*}
        \bignbr{W_h - \tilde W_h}_\infty 
        &\le \bignbr{U_h -\tilde U_h}_\infty + \bignbr{\tilde U_h}_\infty \cdot \sup_{s_h\in\cS} \bignbr{\nu_h \otimes \pi_h (\cdot,\cdot\given s_h)- \tilde\nu_h \otimes\tilde\pi_h(\cdot,\cdot\given s_h)}_1\nend
        &\le \bignbr{U_h -\tilde U_h}_\infty + \bignbr{\tilde U_h}_\infty \cdot \rbr{\bignbr{\nu_h- \tilde\nu_h }_1 + \bignbr{\pi_h- \tilde\pi_h }_1}\nend
        &\le \bignbr{U_h -\tilde U_h}_\infty + 4\eta H  \bignbr{Q_h -\tilde Q_h}_\infty + H \epsilon, 
    \end{align*}
    where the last inequality holds by using \eqref{eq:error-prop-nu} and note that $\tilde U_h$ is bounded by $H$. As a result, we have that 
    \begin{align*}
        \begin{bmatrix}
            \bignbr{U_h - \tilde U_h}_\infty \\
            \bignbr{Q_h - \tilde Q_h}_\infty 
        \end{bmatrix}
        \le 
        \begin{bmatrix}
            1 & 4\eta H \\
            0 & \gamma
        \end{bmatrix}
        \begin{bmatrix}
            \bignbr{U_{h+1} - \tilde U_{h+1}}_\infty \\
            \bignbr{Q_{h+1} - \tilde Q_{h+1}}_\infty 
        \end{bmatrix}
        + \epsilon
        \begin{bmatrix}
            1+2H \\
            2(1+\gamma B_A)
        \end{bmatrix}. 
    \end{align*}
Solving this matrix inequality, we have 
\begin{align*}
    \bignbr{Q_h - \tilde Q_h}_\infty &\le \epsilon 2(1+\gamma B_A) \eff_{H-h+1}(\gamma) \le \epsilon 2(1+\gamma B_A) \eff_{H}(\gamma), \nend
    \bignbr{U_h - \tilde U_h}_\infty &\le \epsilon H \rbr{4\eta H 2(1+\gamma B_A)\eff_H(\gamma) + 1 + 2H}.
\end{align*}
Furthermore, we can establish that 
\begin{align*}
    \bignbr{V_h - \tilde V_h }_\infty 
    &\le \bignbr{Q_h - \tilde Q_h}_\infty \le \epsilon 2(1+\gamma B_A) \eff_{H}(\gamma), \nend
    \bignbr{W_h - \tilde W_h}_\infty &\le \epsilon (H+1) \rbr{4\eta H 2(1+\gamma B_A)\eff_H(\gamma) + 1 + 2H}, 
\end{align*}
and also for the quantal response, 
\begin{align*}
    D_\H^2(\tilde \nu_h(\cdot\given s_h), \nu_h(\cdot\given s_h)) \le \nbr{\nu_h(\cdot\given s_h) - \tilde \nu_h(\cdot\given s_h)}_1 \le 4 \eta \epsilon 2(1+\gamma B_A) \eff_{H}(\gamma).
\end{align*}
Therefore, we conclude the proof. 
\end{proof}