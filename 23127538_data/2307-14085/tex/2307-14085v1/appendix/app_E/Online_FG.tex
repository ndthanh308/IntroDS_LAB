In this section, we provide formal proofs for the result to Theorem \ref{thm:OMLE-farsighted}.
The proof relies on a novel decomposition of the online regret in the Taylor-series form and utilizes the techniques for analyzing the OMLE in \citep{chen2022unified, foster2021statistical,jin2021bellman}. Before diving into the proof, we present the following key lemma that provides guarantees for the confidence set $\confset_\cM^t(\beta)$ given by the algorithm.

The proof is carried out as the following. We recall that $\beta\ge  \allowbreak 9\log(3e^2TH \cN_\rho(\cM,T^{-1})\delta^{-1})$, where we additionally include an $\log T$ term to ensure a union bound over $t\in[T]$.

\paragraph{Step 1. Online Regret Decompostion. }
By \Cref{lem:MLE}, we have with high probability that $M^*\in\confset_\cM^t(\beta)$. Hence, we can upper bound the online regret by
\begin{align*}
    \Reg(T) = \sum_{t=1}^T J(\pi^*, M^*) - J(\pi^t, M^*) \le \sum_{t=1}^T J(\pi^t, M^t) - J(\pi^t, M^*), 
\end{align*}
where the inequality holds by additionally noting that OMLE produces the pair $(\pi^t, M^t)$ that maximizes $J$ within the confidence set $\confset^t(\beta)$.
The key in studying the regret in this MDP with strategic follower is decomposing the performance difference into both the follower's temporal difference and the follower's response difference. 
We invoke \Cref{lem:subopt-decomposition} with $\pi^t, \tilde U^t, \tilde W^t, \tilde \nu^t$, where $\tilde U^t, \tilde W^t, \tilde \nu^t$ are given under policy $\pi^t$ and the estimated model $\tilde M^t$, and they should satisfy $\tilde W_h^t(s_h)=(T_h^{\pi^t, \tilde \nu^t}\tilde U_h^t) (s_h)$. We additionally define $\nu^t = \nu^{\pi^t, M^*}$. We have that
\begin{align*}
    &J(\pi^t, \tilde M^t) - J(\pi^t, M^*)\nend
    &\quad \le \sum_{h=1}^H \EE^t \sbr{\rbr{\tilde U_h^t - u_h}(s_h, a_h, b_h) -  \tilde W_{h+1}^t(s_{h+1})} + \sum_{h=1}^H 2 H  \EE^t \sbr{D_\TV\rbr{\tilde \nu_h^t(\cdot\given s_h), \nu_h^t(\cdot\given s_h)}}\nend
    &\quad \le \sum_{h=1}^H \underbrace{\EE^t \sbr{\rbr{\tilde U_h^t - u_h}(s_h, a_h, b_h) - \tilde W_{h+1}^t(s_{h+1})}}_{\dr \text{Leader's Bellman error}} \nend
    &\qqquad +   C^{(0)}
    \sum_{h=1}^H \underbrace{\EE^t\sbr{\abr{\tilde \Delta^{(1,t)}_h(s_h, b_h)}}}_{\ds\text{1st-order error}}  + C^{(2)}
    \max_{h\in [H]} \underbrace{\EE^t\sbr{ \rbr{\rbr{\tilde Q_h^t - r_h^{\pi^t} - \gamma P_h^{\pi^t} \tilde V_{h+1}^t}(s_h, b_h)}^2}}_{\ds\text{2nd-order error}}
\end{align*}
where the expectation $\EE^t$ is taken with respect to $\pi^t$ and the true model $M^*$, 
$C^{(0)}=2\eta H$, 
$C^{(2)} = 2 H\eta^2 H (1+4 \eff_H(\gamma))\exp\rbr{6\eta B_A}\cdot \rbr{\eff_H(\exp(2\eta B_A)\gamma)}^2$ with $\eff_H(x) = (1-x^H)/(1-x)$ as the \say{effective}  horizon with respect to $x$, 
and $\tilde\Delta_j^{(1, t)}(s_h, b_h)$ is defined as
\begin{align*}
    \tilde \Delta^{(1, t)}_h(s_h, b_h) &=  \rbr{\EE_{s_h, b_h}^t -\EE_{s_h}^t}\Biggsbr{\sum_{l=h}^H \gamma^{l-h}\underbrace{\rbr{\tilde Q_l^t - r_l^{\pi^t} - \gamma P_l^{\pi^t} \tilde V_{l+1}^t}(s_l, b_l)}_{\ds\text{Follower's Bellman error}}}.
\end{align*}
Here, the second inequality comes from \Cref{lem:performance diff} and uses the definition that $\tilde V_h^t = \eta^{-1}\log \int \exp(\eta \tilde Q_h^t)$, $\tilde A_h^t = \tilde Q_h^t -\tilde V_h^t$,  and that $\tilde\nu_h^t=\exp\orbr{\eta \tilde A_h^t}$ under the alternative model $\tilde M^t$. In the sequel, we will bound these three terms separately.

\paragraph{Step 2. Bounding the Leader's Bellman Error.}
We first show that the leader's Bellman error is controllable when summed up for $T$ steps. Specifically, consider the following configurations for step $h\in[H]$,
\begin{itemize}[leftmargin=20pt]
    \item[(i)] Define function class $\cG_{h,L}$ as
    \begin{align*}
        \cG_L^h &= \Big\{g:\cS\times\cA\times\cB\rightarrow \RR \Biggiven g={\bigrbr{U_h^{\pi^{\tilde M},\tilde M} - u - P_h W_{h+1}^{\pi^{\tilde M},\tilde M}}(s_h, a_h, b_h)}, \exists \tilde M\in\cM\Big\}, 
    \end{align*}
    where we define $\pi^M = \argmax_{\pi\in\Pi}J(\pi, M)$.
    Specifically, the expectation is taken under $\pi$ and the true model. Consider a sequence of function $\{g_h^i = (\tilde U_h^{i} - u - P_h \tilde W_{h+1}^i)\}_{i\in[T]}$. We see directly that $g_h^i\in\cG_{h, L}$ since $\tilde U^i = U^{\pi^i, M^i}$ and we have by the optimism in the algorithm that $\pi^i = \pi^{M^i}$. The same also holds for $\tilde W^i$.
    \item[(ii)] Define a class of probability measures over $\cS\times\cA\times\cB$ as $$\sP_{h, L}=\{\PP^\pi((s_h, a_h, b_h)=\cdot), \forall \pi\in\Pi\}.$$
    Consider a sequence of probability measures $\{\rho_h^i(\cdot)=\PP^{\pi^i}((s_h, a_h, b_h)=\cdot)\}_{i\in[T]}$.
    % \item Define a class of probability measures over space $\cS\times\cA\times\cB$ as \begin{align*}
    %     \sP_L = \cbr{\rho\in\Delta(\cS\times\cA\times\cB): \exists h\in[H], \pi\in\Pi, \rho(\cdot)=\PP^\pi((s_h, a_h, b_h)=\cdot)}.
    % \end{align*}
    \item[(iii)] Under this two sequences, we denote by $g_h^t(\pi^i) = \EE_{\rho_h^i}[g_h^t]$ for simplicity. 
    We have 
    $$g_h^t(\pi^i) =\EE^i\bigsbr{{\bigrbr{\tilde U_h^t - u - P_h \tilde W_{h+1}^t}(s_h,a_h,b_h)}}, $$
    which should be bounded by $3H$.
\end{itemize}
We denote by $\dim(\cG_L) = \max_{h\in[H]}\dim_\DE(\cG_{h, L}, \sP_{h, L}, T^{-1/2})$ in the sequel.
Our guarantee for the sequence $\{g_h^i\}_{i\in[T]}$ and $\{\pi^i\}_{i\in[T]}$ is 
\begin{align}\label{eq:OnN-guarantee-Bellmanerror}
    \sum_{i=1}^{t-1} \rbr{g_h^t(\pi^i)}^2 
    &= \sum_{i=1}^t \rbr{\EE^i\bigsbr{{\bigrbr{\tilde U_h^t - u - P_h \tilde W_{h+1}^t}(s_h,a_h,b_h)}}}^2 \nend
    &\le \sum_{i=1}^t \EE^i\sbr{\rbr{\bigrbr{\tilde U_h^t - u_h - P_h \tilde W_{h+1}^t}(s_h,a_h,b_h)}^2}\nend
    &\le \sum_{i=1}^t 2 \EE^i\sbr{\rbr{\rbr{\tilde u_h^t - u_h}(s_h, a_h, b_h)}^2} + 8 H^2 \EE^i D_\TV^2\rbr{P_h(\cdot\given s_h, a_h, b_h), \tilde P_h^t(\cdot\given s_h, a_h, b_h)}\nend
    &\le 8 H^2 \cdot 4 \beta,
\end{align}
where we define $\tilde u_h^t = u_h^{M^t}$ and $\tilde P_h^t = P_h^{M^t}$. The first ineqality holds by the Cauchy-Schwartz inequality ,  the second inequality holds by noting that $\tilde U_h^t = \tilde u_h^t + \tilde P_h^t \tilde W_h^t$, and the last inequality holds by invoking the guarantee in \Cref{lem:MLE}. We then have by the first order argument in \Cref{lem:de-regret} that 
\begin{align*}
    \sum_{i=1}^T \abr{g_h^t(\pi^t)} \le 2\sqrt{\dim\rbr{\cG_L} 32 H^2 \beta T} + 3 H\min\cbr{T, \dim\rbr{\cG_L}}  + \sqrt T, 
\end{align*}
which implies that the leader's Bellman error is upper bounded by $\cO( H^2 \sqrt{\dim\rbr{\cG_L} H \beta T})$.

\paragraph{Step 3. Bounding the first-order Term.}
we next show that the first-order term in the follower's response error is also under control. 
\begin{itemize}[leftmargin=20pt]
    \item[(i)] Define function class $\cG_{h, F}^1$ as 
    \begin{align*}
        \cG_{h, F}^1 &= \Bigg\{g:(\cS\times\cA\times\cB)^{H-h+1}\rightarrow \RR \bigggiven \exists M\in\cM, \nend
        &\qqquad g((s_l, a_l, b_l)_{l=h}^H) = {\sum_{l=h}^H \gamma^{l-h}\bigrbr{r_l^M- r_l + \gamma (P_l^{M} - P_l) V_{l+1}^{\pi^M, M}}(s_l, a_l, b_l)}
        \Bigg\},
    \end{align*}
    where we remind the readers that $\pi^M = \argmax_{\pi\in\Pi}J(\pi, M)$ only depends on $M$. Consider 
    sequences 
    $$\cbr{g_h^t=\sum_{l=h}^H \gamma^{l-h}\orbr{\tilde r_l^t- r_l + \gamma (\tilde P_l^t - P_l) \tilde V_{l+1}^t}}_{t\in[T]}, $$
    where we define $\tilde r_l^t = r_l^{M^t}$ and $\tilde P_l^t = P_l^{M^t}$.
    It is obvious that $g_h^t\in\cG_{h, F}^1$ since $\tilde V_h^t = V_h^{\pi^t, M^t}$ and $\pi^t = \argmax_{\pi\in\Pi}J(\pi, M^t) = \pi^{M^t}$.
    % $r_h^{\pi}(s_h, b_h) = \dotp{r_h(s_h, \cdot, b_h)}{\pi_h(\cdot\given s_h, b_h)}_\cA$ and the same holds for $P_h^\pi$. 
    % One can see immediately that $\cG_F^1$ is a bilinear class, where the expectation part depends on $\pi$ and the follower's Bellman error part depends on $M$.
    
    \item Define a class of signed measures over $(\cS\times\cA\times\cB)^{H-h+1}$ as 
    \begin{equation*}
        \sP_{h, F}^1 = \cbr{\begin{aligned}
            &\PP^\pi(((s_l, a_l, b_l)_{l=h+1}^H , a_h)=\cdot \given s_h, b_h)\delta_{(s_h, b_h)}(\cdot) \nend
            &\quad - \PP^\pi(((s_l, a_l, b_l)_{l=h+1}^H , a_h, b_h)=\cdot \given s_h)\delta_{(s_h)}(\cdot) 
        \end{aligned}
        \bigggiven \pi\in\Pi, (s_h, b_h)\in\cS\times\cB}, 
    \end{equation*}
    where $\delta_{s_h, b_h}$ is the measure that puts measure $1$ on a single state-action pair $(s_h, b_h)$, and the conditional is well defined by the Markov property. Also, consider the following sequence, 
    \begin{align*}
        \Big\{\rho_h^t(\cdot)&=\PP^{\pi^t}(((s_l, a_l, b_l)_{l=h+1}^H , a_h)=\cdot \given s_h^t, b_h^t)\delta_{(s_h^t, b_h^t)}(\cdot) \nend
        &\qquad - \PP^{\pi^t}(((s_l, a_l, b_l)_{l=h+1}^H , a_h, b_h)=\cdot \given s_h^t)\delta_{(s_h^t)}(\cdot)\Big\}_{t\in[T]},
    \end{align*}
    and we also have $\rho_h^t\in\sP_{h, F}^1$.
    \item  
    In particular, we define $g_h^t(s_h^i, b_h^i, \pi^i)$ as the integral of $g_h^t$ with respect to $\rho_h^i$, which is given by
    $$g_h^t(s_h^i, b_h^i, \pi^i)= \rbr{\EE_{s_h^i, b_h^i}^{\pi^i}-\EE_{s_h^i}^{\pi^i}} \sbr{\sum_{l=h}^H \gamma^{l-h}\Bigrbr{\tilde r_l^t- r_l + \gamma (\tilde P_l^t - P_l) \tilde V_{l+1}^t}(s_l, a_l, b_l)},$$
    % One can check that $g_h^t\in\cG_F^1$ 
    Note that the sequence of signed measures is uniquely determined by $\{(s_h^t, b_h^t, \pi^t)\}_{t\in[T]}$. Moreover, we have $g_h^t(s_h^i, b_h^i, \pi^i)$ bounded by $\eff_H(\gamma)(2\nbr{r_h}_\infty + 2\nbr{V_{h+1}}_\infty) \le 4B_A \eff_H(\gamma)$, where we the definition of $B_A$ is available in \eqref{eq:define_BA};  
\end{itemize}
We define the maximal eluder dimension of $\cG_{h, F}^1$ with respect to $\sP_{h,F}^1$ as $$\dim(\cG_F^1) =\max_{h\in[H]} \dim_\DE(\cG_{h, F}^1,\sP_{h, F}^1, T^{-1/2}).$$
We first see what guarantee we have on the given sequences $\{g_h^t\}_{t\in[T]}$ and $\{(s_h^t, b_h^t, \pi^t)\}_{t\in[T]}$, 
\begin{align}
    &\sum_{i=1}^{t-1} \rbr{g_h^t(s_h^i, b_h^i, \pi^i)}^2 \nend
    & \quad =\sum_{i=1}^{t-1} \rbr{\rbr{\EE_{s_h^i, b_h^i}^{i}-\EE_{s_h^i}^{i}} \sbr{\sum_{l=h}^H \gamma^{l-h}\Bigrbr{\tilde r_l^t- r_l + \gamma (\tilde P_l^t - P_l) \tilde V_{l+1}^t}(s_l, a_l, b_l)}}^2 \nend
    & \quad \le 2\sum_{i=1}^{t-1} \rbr{\rbr{\EE_{s_h^i, b_h^i}^{i}-\EE_{s_h^i}^{i}} \sbr{\sum_{l=h}^H \gamma^{l-h}\Bigrbr{\tilde r_l^t- r_l + \gamma (\tilde P_l^t - P_l) V_{l+1}^{\pi^i, M^t}}(s_l, a_l, b_l)}}^2\nend
    &\qqquad + 2\sum_{i=1}^{t-1} \rbr{\rbr{\EE_{s_h^i, b_h^i}^{i}-\EE_{s_h^i}^{i}}\sbr{\sum_{l=h}^H \gamma^{l-h+1} (\tilde P_l^t - P_l) \bigrbr{V_{l+1}^{\pi^i, M^t} - \tilde V_{l+1}^t}(s_l, a_l,b_l)}}^2 \nend
    & \quad \le 2\sum_{i=1}^{t-1} \Biggrbr{\underbrace{
        \rbr{\EE_{s_h^i, b_h^i}^{i}-\EE_{s_h^i}^{i}} \sbr{\sum_{l=h}^H \gamma^{l-h}\Bigrbr{\tilde r_l^t- r_l + \gamma (\tilde P_l^t - P_l) V_{l+1}^{\pi^i, M^t}}(s_l, a_l, b_l)}
    }_{\ds\tilde\Delta^{(1)}_{h, \pi^i, M^t}(s_h, b_h)}}^2\nend
    &\qquad + \underbrace{C B_A^2 \eff_{H}(\gamma)\sum_{i=1}^{t-1} \rbr{\EE_{s_h^i, b_h^i}^i + \EE_{s_h^i}^i} \sbr{\sum_{l=h}^H \gamma^{l-h+1} D_\TV^2 \rbr{\tilde P_l^t(\cdot\given s_l, a_l,b_l), P_l(\cdot\given s_l, a_l,b_l)}}}_{\dr (i)}, \label{eq:OnN-1st-g-sq}
\end{align}
where the first inequaltiy holds by the Jensen's inequality, and the second inequality holds by upper bounding the difference $\tilde P_l^t - P_l$ by the TV distance. Here, we are able to use $B_A$ as the upper bound for the V functions by our discussion in \Cref{sec:app-notations}, and $C$ hides some universal constant.
Applying \Cref{lem:1st-ub} with $\tilde \Delta^{(1)}_h(s_h, b_h)$ replaced by
\begin{align*}
    \tilde \Delta^{(1)}_{h,\pi^i, M^t}(s_h, b_h) 
    &=  \rbr{\EE_{s_h, b_h}^{i} -\EE_{s_h}^{i}}\Biggsbr{\sum_{l=h}^H \gamma^{l-h} 
    {\rbr{Q_l^{\pi^i, M^t} - r_l^{\pi^i} - \gamma P_l^{\pi^i} V_{l+1}^{\pi^i, M^t}}(s_l, b_l)}}\nend
    & = \rbr{\EE_{s_h, b_h}^{i} -\EE_{s_h}^{i}}\Biggsbr{\sum_{l=h}^H \gamma^{l-h} 
    {\rbr{\tilde r_l^t - r_l + \gamma (\tilde P_l^t - P_l) V_{l+1}^{\pi^i, M^t}}(s_l, a_l, b_l)}},
\end{align*}
we obtain for all $t\in[T], h\in[H]$,
\begin{align}
    &\sum_{i=1}^{t-1}\bigrbr{\tilde \Delta_{h, \pi^i, M^t}^{(1)}(s_h^i, b_h^i)}^2  \nend
    &\quad \le 2 \sum_{i=1}^{t-1}\rbr{\rbr{\EE_{s_h^i, b_h^i}^i-\EE_{s_h^i}^i} \bigsbr{\orbr{Q_h^{\pi^i, M^t} -  Q_h^{\pi^i}}(s_h, b_h)}}^2 \nend
    &\qqquad + C \gamma^2  \rbr{\eta^{-1} +2 B_A}^2\eff_H(\gamma) \sum_{l=h+1}^H \gamma^{l-h-1} {\rbr{\EE_{s_h^i}^i+\EE_{s_h^i, b_h^i}^i}\sbr{D_\H^2(\nu_l^{\pi^i}(\cdot\given s_l), \nu_l^{\pi^i, M^t}(\cdot\given s_l))}}\nend
    &\quad\le  8 C_\eta^2 \beta  + 64 B_Q^2 \log\rbr{TH\cN_\infty(\cM, T^{-1})\delta^{-1}} + C \gamma^2  \rbr{\eta^{-1} +2 B_A}^2\eff_H(\gamma)^2  \beta \nend
    &\quad \le \cO\rbr{ (\eta^{-1}+2B_A)^2 \eff_H(\gamma)^2  \beta},\label{eq:OnN-1st-Delta1-sq}
\end{align}
where the second inequality holds by \eqref{eq:MLE-guarantee-Q-3} in \Cref{lem:MLE-formal}, and the covering number $\cN_\varrho(\cM, \epsilon)$ is with respect to the infinite norm of the Q function. Here, $\cO$ only hides universal constant independent of $H,\eta, T$. 
Meanwhile, for the TV distance term in \eqref{eq:OnN-1st-g-sq}, we have also by \Cref{lem:MLE} that 
\begin{align*}
    {\dr (i)}\le 8 B_V^2 \eff_{H}(\gamma){\sum_{l=h}^H \gamma^{l-h+1} 8\beta }\le \cO(B_A^2 \eff_H(\gamma)^2 \beta),
\end{align*}
where $\cO$ only hides some universal constants.
Hence, we conclude that 
\begin{align*}
    \sum_{i=1}^{t-1} \rbr{g_h^t(s_h^i, b_h^i, \pi^i)}^2  \le \cO\rbr{\rbr{\eta^{-1}+B_A}^2 \eff_H(\gamma)^2 \beta}.
\end{align*}
Now, for the first-order term, we have  
\begin{align*}
    &C^{(0)}
    \sum_{t=1}^T \sum_{h=1}^H{\EE^t\sbr{\abr{\tilde \Delta^{(1,t)}_h(s_h, b_h)}}}\nend
    &\quad \le 2 C^{(0)} \sum_{h=1}^H \underbrace{\sum_{t=1}^T \abr{\tilde\Delta^{(1, t)}_h(s_h^t, b_h^t)}}_{\ts \sum_{t=1}^T |g_h^t(s_h^t, b_h^t, \pi^t)|} + H C^{(0)} \cdot 4 \bignbr{\tilde \Delta^{(1)}_h}_\infty \log\rbr{H\cN_\rho(\cM, T^{-1})\delta^{-1}} \nend
    &\quad \le \cO\rbr{ HC^{(0)} \eff_H(\gamma)\sqrt{\dim(\cG_F^1)\rbr{\eta^{-1}+B_A}^2  \beta T }}\nend
    &\quad\le \cO\rbr{ H^2 \eff_H(\gamma)\rbr{1+\eta B_A} \sqrt{\dim(\cG_F^1)\beta T}},
\end{align*}
where the first inequality follows from a standard martingale concentration in \Cref{cor:martigale concentration}, and the second inequality holds by using the first order regret bound in \Cref{lem:de-regret}, and the last inequality holds by $C^{(0)}=2\eta H$.

\paragraph{Step 3. Bounding the Second-Order Term.}
Previously, we decompose the online regret and obtain a second-order term, which we referred to as (ii), 
\begin{align*}
    {\dr (ii)}\defeq C^{(2)}
    \sum_{t=1}^T \max_{h\in [H]} {\EE^t\sbr{ \rbr{\rbr{\tilde Q_h^t - r_h^{\pi^t} - \gamma P_h^{\pi^t} \tilde V_{h+1}^t}(s_h, b_h)}^2}}.
\end{align*}
For this term, we specify the function class to use for our purpose, 

\begin{itemize}[leftmargin =20pt]
    \item[(i)] We take the same function class $\cG_F^2$ as
    \begin{align*}
        \cG_{h, F}^2 &= \Bigg\{g:\cS\times\cA\times\cB\rightarrow \RR: \exists M\in\cM, h\in[H] \nend
        &\qqquad g(s_h, b_h, a_h) = {\bigrbr{r_h^M- r_h + \gamma (P_h^{M} - P_h) V_{l+1}^{\pi^M, M}}(s_h, a_h, b_h)}
        \Bigg\},
    \end{align*}
    where $\cG_F^2$ is bounded by $4B_A$. 
    \item[(ii)] We define a class of probability measures on $\cS\times\cA\times\cB$ as $$\sP_{h, F}^2 = \cbr{\PP^\pi(a_h=\cdot\given s_h, b_h)\delta_{(s_h, b_h)}(\cdot)\given \pi\in\Pi, (s_h,b_h)\in\cS\times\cB}, $$
    where $\delta_{(s_h,b_h)}(\cdot)$ is the measure that assigns $1$ to the state-action pair $(s_h, b_h)$. 
    \item[(iii)] We take a sequence of functions $\{g_h^t\}_{t\in[T]}$ as $\{g_h^t = \tilde r_h^{t}- r_h + \gamma (\tilde P_h^t - P_h) \tilde V_{l+1}^{t}\}_{t\in[T]}$, and take a sequence of probability measures as $\{\rho_h^t(\cdot) = \PP^{\pi^t}(a_h=\cdot\given s_h^t, b_h^t)\delta_{(s_h^t, b_h^t)}(\cdot)\}_{t\in[T]}$, where we define $\tilde r_h^t = r_h^{M^t}$ and $\tilde P_h^t = P_h^{M^t}$.
    One can check that $g_h^t\in\cG_{h,F}^1$ since $\tilde V_h^t = V_h^{\pi^t, M^t}$ and we have $\pi^t = \argmax_{\pi\in\Pi}J(\pi, M^t) = \pi^{M^t}$. In addition, we define $g_h^t(s_h^i, b_h^i, \pi^i)$ as the integral of $g_h^t$ with respect to $\rho_h^i$, which is given by
    \begin{align*}
        g_h^t (s_h^i, b_h^i, \pi^i) = \EE_{s_h^i, b_h^i}^{\pi^i} \sbr{\bigrbr{\tilde r_h^{t}- r_h + \gamma (\tilde P_h^t - P_h) \tilde V_{l+1}^{t}}(s_h, a_h, b_h)},
    \end{align*}
    Note that the sequence of probability measures is uniquely determined by $\{(s_h^t, b_h^t, \pi^t)\}_{t\in[T]}$.
\end{itemize}
We let $\dim(\cG_F^2)=\max_{h\in[H]}\dim_\DE(\cG_{h,F}^2,\sP_{h, F}^2, T^{-1/2})$ be the eluder dimension.
We next establish guarantee for $\sum_{i=1}^{t-1} (g_h^t(s_h^i, b_h^i, \pi^i))^2$.
\begin{align}\label{eq:OnN-guarantee-2ndQRE}
    &\sum_{i=1}^{t-1} (g_h^t(s_h^i, b_h^i, \pi^i))^2 \nend
    &\quad = \sum_{i=1}^{t-1} \rbr{\EE_{s_h^i, b_h^i}^{i} \sbr{\bigrbr{\tilde r_h^{t}- r_h + \gamma (\tilde P_h^t - P_h) \tilde V_{h+1}^{t}}(s_h, a_h, b_h)}}^2\nend
    &\quad  \le 2\sum_{i=1}^{t-1} \rbr{\EE_{s_h^i, b_h^i}^{i} \sbr{\bigrbr{\tilde r_h^{t}- r_h + \gamma (\tilde P_h^t - P_h) V_{h+1}^{\pi^i, M^t}}(s_h, a_h, b_h)}}^2 \nend
    &\qqquad + 2\cdot 16B_A^2 \sum_{i=1}^{t-1} \rbr{\EE_{s_h^i, b_h^i}^i\sbr{D_\TV^2\rbr{\tilde P_h^t(\cdot\given s_h, a_h, b_h), P_h(\cdot\given s_h, a_h, b_h)}}}\nend
    &\quad \lesssim 3 \underbrace{\sum_{i=1}^{t-1}\EE^i\rbr{\EE_{s_h, b_h}^{i} \sbr{\bigrbr{\tilde r_h^{t}- r_h + \gamma (\tilde P_h^t - P_h) V_{h+1}^{\pi^i, M^t}}(s_h, a_h, b_h)}}^2}_{\dr (iii)} + 32 B_A^2 \log\rbr{TH\cN_\rho(\cM,T^{-1}\delta^{-1})}\nend
    &\qqquad + 48 B_A^2 \underbrace{\sum_{i=1}^{t-1} \EE^i\sbr{D_\TV^2\rbr{\tilde P_h^t(\cdot\given s_h, a_h, b_h), P_h(\cdot\given s_h, a_h, b_h)}}}_{\dr (iv)} + 64 B_A^2 \log\rbr{TH\cN_\rho(\cM, T^{-1})}
\end{align}
where in the first inequality, we use $2B_A$ to upper bound $\|\tilde V_{h+1}^t - V_{h+1}^{\pi^i, M^t}\|_\infty$, and uses the Cauchy-Schwartz inequality to move the square into the expectation.
Here, the second inequality uses a standard martingale concentration result in \Cref{cor:martigale concentration} for both terms, and we invoke the same upper bound $B_A$ for the V functions.
Now, we invoke \Cref{lem:2nd-ub}, which says that term (iii) enjoys the following upper bound, 
\begin{align*}
    {\dr (iii)} &=\sum_{i=1}^{t-1}\EE^i\rbr{\EE_{s_h, b_h}^{i} \sbr{\bigrbr{\tilde r_h^{t}- r_h + \gamma (\tilde P_h^t - P_h) V_{h+1}^{\pi^i, M^t}}(s_h, a_h, b_h)}}^2 \nend
    & = \sum_{i=1}^{t-1}\EE^i\rbr{{\bigrbr{Q_h^{\pi^i, M^t} - r_h^{\pi^i} - \gamma P_h^{\pi^i} V_{h+1}^{\pi^i, M^t}}(s_h, a_h, b_h)}}^2\nend
    &\le L^{(2)} \sum_{i=1}^{t-1} \max_{h\in[H]} \cbr{\EE^i D_\H^2\bigrbr{\nu_h^{\pi^i, M^t}(\cdot\given s_h), \nu_h^{\pi^i}(\cdot\given s_h)}+ \EE^i D_\TV^2\bigrbr{P_h^{\pi^i, M^t}(\cdot\given s_h, b_h), P_h^{\pi^i}(\cdot\given s_h, b_h)}}\nend
    &\le L^{(2)} 4H \beta , 
\end{align*}
where the last inequality uses the guarantee in \Cref{lem:MLE}. To enable a direct use of the MLE guarantee, we replace the maximum by a sum over all $h\in[H]$ and upper bound the TV distance by the Hellinger distance. Here, $L^{(2)}$ is defined in \Cref{lem:2nd-ub}. For term (iv), we use the same guarantee in \Cref{lem:MLE} and the same bounding the TV distance by the Hellinger distance argument and obtain ${\dr (iv)}\le 4\beta$. Therefore, we conclude that 
\begin{align*}
    \sum_{i=1}^{t-1} (g_h^t(s_h^i, b_h^i, \pi^i))^2 \lesssim {\bigrbr{L^{(2)} H + B_A^2} \beta},
\end{align*}
where $\lesssim$ only hides some universal constant. As a result of the second order regret in \Cref{lem:de-regret}, we have 
\begin{align*}
    \sum_{t=1}^{T} (g_h^t(s_h^t, b_h^t, \pi^t))^2 
    &\lesssim {\dim(\cG_F^2) \bigrbr{L^{(2)} H + B_A^2} \beta  + \min\cbr{T, \dim(\cG_F^2)} 16 B_A^2+1}\nend
    &\lesssim {H \dim(\cG_F^2) \beta  L^{(2)}},
\end{align*}
where $\lesssim$ hides some universal constant. Therefore, we conclude that
\begin{align*}
    {\dr (ii)}
    &\defeq C^{(2)}
    \sum_{t=1}^T \max_{h\in [H]} {\EE^t\sbr{ \rbr{\rbr{\tilde Q_h^t - r_h^{\pi^t} - \gamma P_h^{\pi^t} \tilde V_{h+1}^t}(s_h, b_h)}^2}}\nend
    &\le C^{(2)} \sum_{t=1}^T \sum_{h=1}^H \EE^t \sbr{ \rbr{\rbr{\tilde Q_h^t - r_h^{\pi^t} - \gamma P_h^{\pi^t} \tilde V_{h+1}^t}(s_h, b_h)}^2} \nend
    & \le 2C^{(2)} \sum_{t=1}^T \sum_{h=1}^H \EE^t_{s_h^t, b_h^t} \sbr{ \rbr{\rbr{\tilde Q_h^t - r_h^{\pi^t} - \gamma P_h^{\pi^t} \tilde V_{h+1}^t}(s_h, b_h)}^2} \nend
    &\qquad + 4 H C^{(2)} 9 B_A^2 \log\rbr{H \cN_\rho(\cM,T^{-2})\delta^{-1}}. 
\end{align*}
where in the first inequality, we replace the maximum by the summation over $h\in[H]$ and in the second inequality, we invoke the martingale concentration in \Cref{cor:martigale concentration}.
Hence, we establish our bound for the second-order term as
\begin{align*}
    {\dr(ii)} 
    &\le 2 C^{(2)} \sum_{h=1}^H \sum_{t=1}^T \rbr{g_h^t(s_h^t, b_h^t, \pi^t)}^2 + 4 H C^{(2)} 9 B_A^2 \log\rbr{H \cN_\rho(\cM,T^{-2})\delta^{-1}}\nend
    &\lesssim {H^2 C^{(2)} \dim(\cG_F^2) \beta  L^{(2)}} +   { H C^{(2)} \beta}
    % \nend
    % &\le \cO\rbr{H^6\eff_H(\gamma)\rbr{\eff_H(\exp(2\eta B_A)\gamma)}^2 (\iota^{2H}+1) \kappa^2 \exp\rbr{14\eta B_A} (1+\eta^2B_A^2)  \dim(\cG_F^2) \beta }
\end{align*}
where $L^{(2)} = c H^2 \eff_H(c_2)^2 \kappa^2 \exp\rbr{8\eta B_A} C_\eta^2$ with 
$c_2 = \gamma(2\exp(2\eta B_A)+\kappa\exp(4\eta B_A))$, and $C^{(2)} = 2 H^2\eta^2 (1+4 \eff_H(\gamma))\exp\rbr{6\eta B_A}\cdot \rbr{\eff_H(\exp(2\eta B_A)\gamma)}^2$.

In summary, for the leader's Bellman error, we have
\begin{align*}
    \text{LBE}\lesssim H(2\sqrt{\dim\rbr{\cG_L} H^2 \beta T} + 3 H\min\cbr{T, \dim\rbr{\cG_L}}  + \sqrt T) \lesssim H^2 \sqrt{\dim\rbr{\cG_L} \beta T}, 
\end{align*}
for the first-order term in the follower's QRE,
\begin{align*}
    {\text{1st-QRE}}\lesssim  H^2 \eff_H(\gamma) \eta C_\eta \sqrt{\dim(\cG_F^1)\beta T}, 
\end{align*}
where $C_\eta = \eta^{-1}+B_A$,
and for the second-order term in the follower's QRE, 
\begin{align*}
\text{2nd-QRE}\lesssim H^2 C^{(2)} L^{(2)} \dim(\cG_F^2) \beta \log T
\end{align*}
which completes the proof for \Cref{thm:OMLE-farsighted}.