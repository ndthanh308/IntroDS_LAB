\section{Proof of \Cref{cor:online linear} }\label{sec:proof-OMLE-linear}
In this proof, we apply \Cref{thm:OMLE-farsighted} to linear MDP. 
To do so, we need to (i) specify the choice of $\cF_{2, h}$ under the linear constraint of the follower's utility; (ii) compute the Eluder dimension of $\cF_0, \cF_{1,h}$ and $\cF_{2, h}$.

\paragraph{Specification of $\cF_{2,h}$. }
We remind the reader of the condition for $f_{2, h}$, 
\begin{align*}
    \EE_{s_h, b_h}^{\pi, M^*}\bigabr{A_h^{\pi, M}-A_h^{\pi,M^*}} \le  \abr{f_{2,h}^M(\pi)} \le L^{(2)} D_\RL(M, M^*;\pi).
\end{align*}

Therefore, we see directly that by taking $f_{2, h}^M(\pi)$ as 
\begin{align*}
    f_{2, h}^M(\pi) = \EE^{\pi, M^*}\sbr{\rbr{r_h^{\pi, M}- r_h^{\pi,M^*} + \gamma \bigrbr{\TT_h^{\pi, M} - \TT_h^{\pi, M^*}} V_{h+1}^{\pi_{\opt}^M, M}}^2}, 
\end{align*}
we have for all $h\in[H]$ that
\begin{align*}
    \cE_{h}^{(2)}(M^t, M^*;\pi^t)\le C^{(2)} \sum_{h=1}^H f_{2,h}^{M^t}(\pi^t).
\end{align*}
We next see how to upper bound $\sum_{i=1}^T f_{2,h}^{M^i}(\pi^i)$ by the guarantee of MLE. Specifically, we need to upper bound $\EE\osbr{\orbr{(\TT_h^\pi-\tilde\TT_h^{\pi})\tilde V_{h+1}}^2}$ and $\EE\osbr{\orbr{r_h^\pi-\tilde r_h^\pi}^2}$ by $D_{\RL, h}^2$ seperately. 

\paragraph{Bounding $f_{2,h}$ with $D_\RL$.}
Note that we only have guarantee for $D_\TV^2(\nu_h, \tilde\nu_h)$ by MLE, which cannot directly guarantee that the true utility is identifiable since a constant shift does not change the follower's behavior at all. For the reward to be identifiable, we need an additional linear constraint, namely $\inp{x}{r_h(s_h, a_h, \cdot)}=\varsigma$.
We start with the easier part with the transition kernel.
\begin{align*}
    \EE\osbr{\orbr{(\TT_h^\pi-\tilde\TT_h^{\pi})\tilde V_{h+1}}^2}\le 2^2 B_U^2 \EE\sbr{D_\TV^2(\TT_h, \tilde \TT_h)} \le 4 B_U^2 D_{\RL, h}^2(\tilde M, M^*;\pi).
\end{align*}
For the utility, we have the following decomposition
\begin{align*}
    \inf_{\xi\in\RR}\EE_{s_h}\abr{r_h^\pi-\tilde r_h^\pi - \xi}  &= \inf_{\xi\in\RR}\EE_{s_h}\abr{Q_h^\pi-\tilde Q_h^\pi - \xi - \gamma\rbr{\TT_h^\pi-\tilde\TT_h^\pi}\tilde V_{h+1} - \gamma \TT_h^\pi\rbr{V_{h+1}-\tilde V_{h+1}}}\nend
    &\le \inf_{\xi\in\RR}\EE_{s_h}\abr{Q_h^\pi-\tilde Q_h^\pi - \xi} + \gamma \EE_{s_h}\abr{\rbr{\TT_h^\pi-\tilde\TT_h^\pi}\tilde V_{h+1} } + \gamma\exp\rbr{2\eta B_A}\EE_{s_h}\abr{Q_{h+1}^\pi - \tilde Q_{h+1}^\pi}, \nend
    &\le \EE_{s_h}\abr{A_h^\pi-\tilde A_h^\pi} + \gamma\EE_{s_h}\abr{\rbr{\TT_h^\pi-\tilde\TT_h^\pi}\tilde V_{h+1} } + \gamma\exp\rbr{2\eta B_A}\EE_{s_h}\abr{Q_{h+1}^\pi - \tilde Q_{h+1}^\pi}
\end{align*}
where the first inequality holds by the same argument for $V-\tilde V$ in \eqref{eq:f_2-1}, and the second inequality holds simply by plugging $\xi = V_h^\pi(s_h) - \tilde V_h^\pi(s_h)$. Now, we can plug in the bound for $\EE_{s_h}\oabr{A_h^\pi-\tilde A_h^\pi}$ in \Cref{lem:response diff} and obtain
\begin{align}
    \inf_{\xi\in\RR}\EE_{s_h}\abr{r_h^\pi-\tilde r_h^\pi - \xi} &\le \underbrace{C_\eta^{-1} \eta D_\TV(\nu_h, \tilde\nu_h) + 2 \gamma B_U \EE_{s_h} D_\TV(\TT_h, \tilde\TT_h)}_{\ds \sD_h(\tilde M, M^*;\pi)} + \gamma\exp\rbr{2\eta B_A}\EE_{s_h}\abr{Q_{h+1}^\pi - \tilde Q_{h+1}^\pi}
    % &\le \rbr{2\eta^{-1}(1+2\eta B_A)+2\gamma B_U} D_{\RL,h}(\tilde M,M^*;\pi) + \gamma\exp\rbr{2\eta B_A}\EE_{s_h}\abr{Q_{h+1}^\pi - \tilde Q_{h+1}^\pi}, 
    \label{eq:f_2-r-diff}
\end{align}
where the second inequality holds by noting that $C_\eta\le 2(1+2\eta B_A)$. We next show what we can say about the utility when combining the guarantee of \eqref{eq:f_2-r-diff} with the linear constraint $\inp{x}{r_h(s_h, a_h, \cdot)}=\varsigma$. Specifically, we have the following lemma.
% \begin{lemma}[\textit{Identification of the follower's utility}]\label{lem:identification}
%     Suppose for $r, \tilde r:\cB\rightarrow \RR$, for some distribution $\nu\in\Delta(\cB)$ such that $\nu>0$, we have $\inf_{\xi\in\RR}\inp{\nu}{\abr{r-\tilde r-\xi}}\le \varepsilon$ and $\inp{x}{r-\tilde r}=0$ hold at the same time for some $x:\cB\rightarrow \RR$ such that $\inp{\ind}{x}\neq 0$. We have
%     \begin{align*}
%         \inp{\nu}{\abr{r-\tilde r}}\le\rbr{1 + \nbr{\frac x \nu}_\infty \cdot \frac{1}{\abr{\inp{x}{\ind}}}} \epsilon
%     \end{align*}
%     \begin{proof}
%         See \Cref{sec:proof-identification} for a detailed proof.
%     \end{proof}
% \end{lemma}
With \Cref{lem:identification}, we conclude that
\begin{align*}
    \EE_{s_h}\abr{r_h^\pi-\tilde r_h^\pi} &\le \rbr{1+\exp\rbr{2\eta B_A} \kappa} \rbr{\sD_h(\tilde M, M^*;\pi) + \gamma \exp\rbr{2\eta B_A}\cdot\EE_{s_h}\abr{Q_{h+1}^\pi - \tilde Q_{h+1}^\pi}}.
\end{align*}
On the other hand, for the Q-function, we have by \eqref{eq:f_2-Q update} that 
\begin{align*}
    \EE_{s_h}\abr{Q_h^\pi - \tilde Q_h^\pi}
    &\le \EE_{s_h}\abr{r_h^\pi-\tilde r_h^\pi + \gamma \rbr{\TT_h^\pi-\tilde\TT_h^\pi}\tilde V_{h+1}} + \gamma \exp\rbr{2\eta B_A} {\EE_{s_h}\abr{Q_{h+1}^\pi-\tilde Q_{h+1}^\pi}}\nend
    &\le \underbrace{2\rbr{1+\exp\rbr{2\eta B_A} \kappa} }_{\ds c_1}\cdot \sD_h(\tilde M,M^*;\pi) \nend
    &\qquad + \underbrace{\rbr{2+\exp\rbr{2\eta B_A} \kappa}\gamma \exp\rbr{2\eta B_A}}_{\ds c_2}\cdot\EE_{s_h}\abr{Q_{h+1}^\pi - \tilde Q_{h+1}^\pi}.
\end{align*}
Therefore, we have by a recursive argument that 
\begin{align*}
    \EE_{s_h}\abr{Q_h^\pi -\tilde Q_h^\pi} \le \sum_{l=h}^H c_2^{l-h} c_1 \EE_{s_h}\sD_l(\tilde M, M^*;\pi).
\end{align*}
For now, we are able to deal with $(\EE_{s_h}\abr{r_h^\pi-\tilde t_h^\pi})^2$. However, note that  the square appears within the expection $\EE_{s_h}$ in terms of the follower's utility error $\EE\rbr{r_h^\pi-\tilde r_h^\pi}^2$. Therefore, we need a variance-mean decomposition,
\begin{align*}
    \EE\rbr{r_h^\pi-\tilde r_h^\pi}^2&= \EE\rbr{Q_h^\pi-\tilde Q_h^\pi - \gamma \rbr{\TT_h^\pi -\tilde\TT_h^\pi}\tilde V_{h+1} -\gamma \TT_h^\pi \rbr{V_{h+1}-\tilde V_{h+1}}}^2\nend
    &\le 4\EE\sbr{\rbr{A_h^\pi -\tilde A_h^\pi}^2 + \rbr{V_h-\tilde V_h}^2 + \gamma^2 \rbr{\rbr{\TT_h^\pi -\tilde\TT_h^\pi}\tilde V_{h+1}}^2 +\gamma^2 \rbr{V_{h+1}-\tilde V_{h+1}}^2}\nend
    &\le 4\EE\sbr{\rbr{A_h^\pi -\tilde A_h^\pi}^2} + 16\gamma^2 B_U^2\EE\sbr{D_\TV(\TT_h,\tilde \TT_h)^2}\nend
    &\qquad + 4\exp\rbr{4\eta B_A}\EE\sbr{\rbr{\EE_{s_h}\abr{Q_h^\pi-\tilde Q_h^\pi}}^2+ \rbr{\EE_{s_{h+1}}\abr{Q_{h+1}^\pi-\tilde Q_{h+1}^\pi}}^2}, 
\end{align*}
where the last inequality holds by noting the upper bound for difference in the V-function used in \Cref{eq:f_2-1}.
We notice that the first term can be upper bounded by the squared Hellinger distance,
\begin{align*}
    D_\H^2\rbr{\nu_h,\tilde\nu_h} &= \dotp{\nu_h}{\rbr{1-\sqrt\frac{\tilde \nu_h}{\nu_h}}^2}\nend
    &= \dotp{\nu_h}{\rbr{1-\exp\rbr{\frac \eta 2 (\tilde A_h - A_h)}}^2}\nend
    &\ge \rbr{\frac{1-\exp\rbr{\eta B_A}}{2 B_A}}^2  \cdot \dotp{\nu_h}{\rbr{A_h-\tilde A_h}^2},
\end{align*}
where the inequality holds by noting that $|1-\exp(x)|\ge (1-\exp(-B))|x|/B$ for any $|x|\le B$.
Therefore, we have the squared utility difference bounded by
\begin{align*}
    \EE\rbr{r_h^\pi -\tilde r_h^\pi}^2 &\le 4 C_\eta^{2} \eta^{-2} \EE D_\H^2(\nu_h,\tilde\nu_h) + 16\gamma^2 B_U^2\EE D_\TV^2(\TT_h,\tilde \TT_h)\nend
    &\qquad + 4 \exp\rbr{4\eta B_A} \EE\sbr{\rbr{\sum_{l=h}^H c_2^{l-h} c_1 \EE_{s_h}\sD_l(\tilde M, M^*;\pi)}^2 + \rbr{\sum_{l=h+1}^H c_2^{l-h} c_1 \EE_{s_{h+1}}\sD_l(\tilde M, M^*;\pi)}^2}\nend
    &\le 4 C_\eta^{-2} \eta^2 \EE D_\H^2(\nu_h,\tilde\nu_h) + 16\gamma^2 B_U^2\EE D_\TV^2(\TT_h,\tilde \TT_h)\nend
    &\qquad + 8 H \max\cbr{c_2^{2H}, 1} c_1^2\exp\rbr{4\eta B_A} \EE\sbr{\sum_{l=h}^H {\sD_l(\tilde M, M^*;\pi)}^2}\nend
    &\le 4 C_\eta^{-2} \eta^2 \EE D_\H^2(\nu_h,\tilde\nu_h) + 16\gamma^2 B_U^2\EE D_\TV^2(\TT_h,\tilde \TT_h)\nend
    &\qquad + 8 H \max\cbr{c_2^{2H}, 1} c_1^2\exp\rbr{4\eta B_A} \EE\sbr{\sum_{h=1}^H \rbr{{C_\eta^{-1} \eta D_\TV(\nu_h, \tilde\nu_h) + 2 \gamma B_U \EE_{s_h} D_\TV(\TT_h, \tilde\TT_h)}}^2}.
\end{align*}
In summary, we have
\begin{align*}
    \EE\rbr{r_h^\pi -\tilde r_h^\pi}^2\le 32 H^2 \max\cbr{c_2^{2H}, 1} c_1^2\exp\rbr{4\eta B_A} \rbr{4(\eta^{-1}+2B_A)^2+4\gamma^2 B_U^2 } \max_{h\in[H]} D_{\RL,h}^2(\tilde M, M^*;\pi).
\end{align*}
where the first inequality holds by noting that $C_\eta\le 2(1+2\eta B_A)$. 
% \newpage
% which completes the proof of \Cref{cor:online linear}.


\subsection{Proof of \Cref{lem:identification}}\label{sec:proof-identification}
For condition $\inf_{\xi\in\RR}\inp{\nu}{\abr{r-\tilde r-\xi}}\le \varepsilon$, we assume that the infimum is achieved at $\xi^*$. Let $r^* = r -\xi^*$ and we have
\begin{align*}
\abr{\inp{r^*-\tilde r}{x}} \le \inp{\abr{r^*-\tilde r}}{\abr{x}} \le \inp{\abr{r^*-\tilde r}}{\nu} \cdot \nbr{\frac{x}{\nu}}_\infty\le \varepsilon\nbr{\frac{x}{\nu}}_\infty,
\end{align*}
where the second inequality is just a distribution shift and the last inequality is given by the condition. Furthermore, for our target,
\begin{align*}
    \inp{\abr{r-\tilde r}}{\nu} \le \inp{\abr{r^*-\tilde r}}{\nu} + \abr{\xi^*} = \varepsilon + \abr{\inp{r^*-r}{x}} \cdot \frac{1}{\abr{\inp{x}{\ind}}},
\end{align*}
where the inequality follows from the triangle inequality and the equality holds by noting that $\inp{x}{\ind}\neq 0$ and $\inp{\abr{r^*-\tilde r}}{\nu}\le \varepsilon$. We bridge these two inequalities by noting that
\begin{align*}
    \abr{\inp{r^*-\tilde r}{x}} = \abr{\inp{r-\tilde r}{x} + \inp{r^*-r}{x} } = \abr{\inp{r^*-r}{x} },
\end{align*}
where the second inequality holds by noting that $\inp{r-\tilde r}{x}=0$. Combining these results and we have
\begin{align*}
    \inp{\abr{r-\tilde r}}{\nu} \le \epsilon + \abr{\inp{r^*-r}{x}} \cdot \frac{1}{\abr{\inp{x}{\ind}}} \le \rbr{1 + \nbr{\frac x \nu}_\infty \cdot \frac{1}{\abr{\inp{x}{\ind}}}} \epsilon, 
\end{align*}
which completes the proof on \Cref{lem:identification}.

