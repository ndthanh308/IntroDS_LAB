\documentclass[11pt]{article}
\usepackage[OT1]{fontenc}
\usepackage{smile}
\renewcommand{\baselinestretch}{1.05}
%\usepackage{authblk}

%\renewcommand\Authsep{ }

%%%%%%%%%%%%-- Comments --%%%%%%%%%%%%
\newcommand{\Zhuoran}[1]{{\cosor{blue}[Zhuoran: #1]}}
\newcommand{\Siyu}[1]{{\color{purple}[Siyu: #1]}}
%%%%%%%%%%%%-- new command --%%%%%%%%%%%%
\def\est{{\rm est}}
\def\RL{{\rm RL}}
\def\H{{\rm H}}
\def\L{{\rm L}}
\def\tV{{\tilde V}}
\def\tQ{{\tilde Q}}
\def\tA{{\tilde A}}
\def\tQ{{\tilde Q}}
\def\tR{{\tilde R}}
\def\tT{{\tilde T}}
\def\tbQ{{\tilde\bQ}}
\def\tbV{{\tilde\bV}}
\def\tbR{{\tilde\bR}}
\def\tM{{\tilde M}}
\def\tnu{{\tilde \nu}}
\def\dimE{{\dim_{\rm E}}}
\def\opt{{\rm opt}}
\def\MLE{{\rm MLE}}
\DeclareMathOperator{\spn}{span}
\def\confset{{\cC}}
\def\eff{{\mathrm{eff}}}
\def\tmp{{\rm *todo*}}

\DeclareMathAlphabet{\pazocal}{OMS}{zplm}{m}{n}
\newcommand{\Unif}{\pazocal{U}}
\def\CI{{\cC}}
\def\DE{{\rm DE}}
\def\E{{\rm E}}
% \def{\PI}{{\Pi}}
\def\QRE{{\mathtt{QRE}}}

\newif\ifmain
\maintrue
\newif\ifneurips
\neuripsfalse
%%%%%%%%%%%%-- start --%%%%%%%%%%%%%%%
\title{\huge Actions Speak What You Want:
Provably Sample-Efficient Reinforcement Learning of the Quantal Stackelberg Equilibrium from Strategic Feedbacks}

% \author{}
\author{Siyu Chen\thanks{Department of Statistics and Data Science, Yale University. Email: \texttt{siyu.chen.sc3226@yale.edu}.} \qquad Mengdi Wang\thanks{Department of Electrical and Computer Engineering, Princeton University. Email: \texttt{mengdiw@princeton.edu}.} \qquad Zhuoran Yang\thanks{Department of Statistics and Data Science, Yale University. Email: \texttt{zhuoran.yang@yale.edu}.}}

\iffalse
\affil[1]{
% \footnotesize 
\small
\textit{Department of Statistics and Data Science, Yale University}}
\affil[2]{
% \footnotesize 
\small
\textit{Center for Statistics and Machine Learning, Department of Electrical and Computer Engineering, Department of Computer Science, Princeton University}}
\fi 


\date{}
\begin{document}
\maketitle
\begin{abstract}
\iffalse
We investigate the challenge of learning a Quantal Stackelberg Equilibrium (QSE) in a leader-follower Markov game with episodic interactions, where the follower has bounded rationality. At the outset of the game, the leader announces her policy to the follower and commits to it. The follower observes the leader's policy and, in turn, adopts a quantal response policy by solving an entropy-regularized policy optimization. The goal of the leader is to find a QSE, which involves determining her optimal policy that maximizes her total return in expectation, as well as the corresponding quantal response of the follower.
We propose reinforcement learning (RL) algorithms for learning QSE on behalf of the leader under both online and offline settings, where only the follower's actions are observable, but not their rewards. RL for QSE poses unique difficulties due to the need to learn the quantal response mapping from the follower's feedback data, as well as the bilevel optimization nature of Stackelberg games. However, we have successfully overcome these difficulties and developed sample-efficient RL algorithms that achieve sublinear suboptimality and regret bounds under offline and online settings, respectively, using general function approximation techniques.
Our algorithms rely on maximum likelihood estimation of the quantal response mapping and the principle of optimism/pessimism. Our technical analysis includes uncertainty quantification of the estimated quantal response via sublevel sets of the likelihood function, and a new performance difference lemma that links the performance of the leader's policy to the error incurred in learning the quantal response. To our knowledge, this is the first set of provably sample-efficient RL algorithms for solving a QSE without the leader observing the follower's reward.



END




    
   In particular, 
Each episode involves the leader committing to an $H$-step policy, followed by the follower selecting a logit quantal response, either myopically or nonmyopically.  
    We take into account information asymmetry that the follower's reward remains unobservable throughout the game
    , which requires the leader to infer the follower's quantal response model from the follower's past behaviors.
    For the case of a myopic follower,
    we develop sample-efficient RL algorithms based on Maximum Likelihood Estimator (MLE) for both offline and online solving a QSE strategy of the leader.
    Theoretical analysis shows that our algorithms achieve suboptimality (offline) and sublinear regret (online)
    using either general function approximation or linear function approximation. 
    For the case of a nonmyopic follower, we introduce model-based offline/online algorithms that are sample efficient under the condition that the leader can uniquely identify the follower's reward with certain prior knowledge. 
    To the best of our knowledge, we establish the first provably efficient RL algorithms for solving for QSEs without the leader observing the follower's reward. This problem is also closely related to reinforcement learning with human feedback (RLHF), and we address the unique challenge of unobservable rewards confronting inherent incentive misalignment within a Stackelberg game.
    \fi


    We study reinforcement learning (RL) for    learning a Quantal Stackelberg Equilibrium (QSE) in an episodic Markov game with a leader-follower structure. 
In specific, at the outset of the game, the leader announces her policy to the follower and commits to it.
The follower observes the leader's policy and, in turn, adopts a quantal response policy by solving an entropy-regularized policy optimization problem induced by leader's policy. 
The goal of the leader is to find her optimal policy, which  yields the optimal expected total return, by interacting with the follower and learning from data. 
A key challenge of this problem is that the leader cannot observe the follower's reward, and needs to infer the follower's quantal response model from his actions against leader's policies. 
We propose sample-efficient algorithms for both the online and offline settings, in the context of function approximation. 
Our algorithms are based on (i) learning the quantal response model via maximum likelihood estimation and (ii) model-free or model-based RL for solving the leader's decision making problem, and we show that they achieve sublinear regret upper bounds. Moreover, we quantify the uncertainty of these estimators and leverage the uncertainty to implement optimistic and pessimistic algorithms for online and offline settings. 
Besides, when specialized to the linear and myopic setting, our algorithms are also computationally efficient. 
Our theoretical analysis features a novel performance-difference lemma which incorporates the error of quantal response model, which might be of independent interest.  
\end{abstract}

{
  \hypersetup{linkcolor=black}
  \tableofcontents
}
\newpage 
%%%%%%%%%%%%-- main content --%%%%%%%%%%
% Figure environment removed

\section{Introduction}
Automatic 3D reconstruction of clothed humans using image inputs has gained increasing significance due to its potential applications in a wide array of AR/VR scenarios. High-fidelity reconstructions typically depend on sophisticated capture systems, which are developed with dense camera arrays~\cite{collet2015high,joo2015panoptic,joo2018total}, programmable light-stages~\cite{Vlasic2009, guo2019relightables}, and depth sensors~\cite{newcombe2011kinectfusion,DoubleFusion,BodyFusion,dou2016fusion4d,newcombe2015dynamicfusion}. However, stringent capture environments equipped with complex hardware pose significant challenges for consumer-level applications.


In this context, considerable research effort has been dedicated to developing methods that allow for more flexible capture configurations, such as utilizing a few RGB inputs. Among these works, learning implicit functions \cite{iccv2020PIFu, saito2020pifuhd, hong2021stereopifu} has proven effective in achieving highly detailed reconstructions by integrating the advancements of deep neural networks. These methods employ large multi-layer perceptrons (MLPs) to predict the occupancy probability or truncated signed distance function (TSDF) value of every queried 3D point based on its associated local feature, which is extracted from images. They can recover a continuous surface at arbitrary resolutions without topology restrictions.


However, in typical MLP-based implicit networks, the occupancy or TSDF value at each location is solved independently with planar image features, rendering them less capable of addressing challenging cases such as occlusions. Consequently, these methods suffer from generalization and robustness issues, particularly when tackling strong occlusions caused by large motion or multiple interacting humans. 
Some follow-up studies  \cite{zheng2021deepmulticap,zheng2021pamir,huang2020arch} utilize an extra geometric model, SMPL~\cite{Loper2015}, to improve robustness by introducing strong shape priors. 
Their success typically relies on the assumption of geometrical similarity \cite{huang2020arch} between the shape prior and target reconstruction, making them intractable for handling complex cases with loose clothes and sensitive to errors in SMPL model fitting.



%\ping{this paragraph sounds like `TSDF is better than MLP/SMPL, and we use TSDF to solve the problem'. But in Sec 3, we are telling a different story, saying `MLP needs a 3D convolutional encoder'. We need to make these two sections consistent.}\sicong{I think in this paragraph we claim that the TSDF}


%We opt for Trucated Signed Distance Funtion (TSDF) volumetric representations as they are naturally suitable for convolution operations, which have shown remarkable performance for learning hierarchical features on 2D visual perception tasks \cite{SunXLW19}. 
%Meanwhile, TSDF also describes the gradual geometry change around shape surface, which is not reflected by occupancy volume. 

We instead revisit the 3D volumetric representation and resort to 3D convolutional neural networks (CNNs) for feature learning, due to their impressive performance in feature learning and the ability to incorporate spatial context. However, volumetric methods and 3D convolution involve discretization, which might raise concerns regarding whether a discretized volume can preserve subtle geometric details as continuous representations learned in implicit functions. We investigate the relationship between volume resolution and quantization error on synthetic data by converting target mesh objects to TSDF volumes, as shown in Figure~\ref{fig:quantization_error}. We observe that the quantization errors are significantly reduced by increasing volume resolution and become nearly negligible when reaching a relatively high resolution (e.g., 512 or higher). In other words, achieving fine-detailed reconstruction is not supposed to be restricted by the use of volume representations as long as a proper volume resolution is utilized. Therefore, we present a method with high-resolution feature volumes, e.g., 256 and 512, while traditional volumetric methods \cite{varol18_bodynet,gilbert2018volumetric} are often limited to much lower resolutions, such as 32 or 128.



On the other hand, an increase in volume resolution may lead to a cubic growth of memory overhead \cite{8100085}. Reducing memory costs while guaranteeing the granularity of volumetric representations is necessary for pursuing high-quality reconstruction. Thus, we adopt a coarse-to-fine approach and cull away irrelevant voxels to build a sparse high-resolution feature volume. At the coarse level, the network computes an initial TSDF by applying a U-Net with sparse 3D CNN \cite{3DSemanticSegmentationWithSubmanifoldSparseConvNet} on the sparse feature volume, which is carved by a visual hull. Through our experiments, it turns out that more than 95\% of the volume grids are discarded by the visual hull culling, making the sparse 3D CNN efficient. At the fine level, the network focuses on a narrow band near the zero-level set of the initial TSDF and discretizes the narrow band with smaller voxels. By employing this narrow-band culling, we further shrink the sampling space, resulting in a relatively small range of grid numbers (usually 300K--500K in our experiments) even with a high volume resolution of 512. The remaining voxels in the narrow band are associated with features that fuse high-frequency information from the computed normal maps upon the low-frequency shape from the coarse level to compute the TSDF at high resolution. The final mesh is then extracted from the TSDF using the Marching-Cube algorithm ~\cite{Lorensen87marchingcubes}.
% Different from the u-net sturcture to preserve global topology context, we then apply a shallow 3dcnn to compute the final TSDF $D_{final}$ which contain more local geometry detail.




% \ping{this paragraph can be expanded. It is an important contribution and often ignored by other works. stress on the novel idea of regressing blending weights instead of colors}

In addition to geometry, high-quality mesh texture is also a crucial factor contributing to visual appearance. Directly computing a color field in 3D space, as in \cite{iccv2020PIFu}, struggles to capture high-frequency texture details, while the neural radiance field (NeRF) \cite{yu2020pixelnerf} or the DoubleField~\cite{shao2022doublefield} require expensive per-instance optimization and are often unstable for sparse input images. In contrast, we adopt an image-based rendering approach to compute a texture atlas map, which is efficient and widely supported in existing computer graphics tools. 
Specifically, we compute a blending weight at each 3D point on the mesh surface to determine its color as a weighted average of the colors at its image projections. The blending weights can be computed at a relatively coarse resolution, e.g., 512 volume resolution in our case, and leave texture details to the high-resolution images, such as 1K or 2K. Unlike previous methods that generate blurry texturing results under sparse input, our method generalizes well on both synthetic and real data with just a few input views. 
Figure~\ref{fig:teaser} shows two examples reconstructed by our method. Despite the challenging garment, pose, and occlusion, our method recovers faithful shape, normal, and texture on the right.

%with a wide variety of poses and clothing styles, and it is also adaptive to handle input image with arbitrary resolutions.
%\sicong{For this concern we claim that when the resolution of dicretized volume meets certain threshold (which is 256 in our experiment), the quantization error can be neglected.} 



In summary, the main contributions of this paper are as follows:
\begin{itemize}
\vspace{-0.1in}
  \item 
  We revisit the 3D volumetric representation and demonstrate that it can support clothed human reconstruction with equal or even better performance compared to implicit representation. 
  \item 
  We develop a memory and computation-efficient method for high-resolution volumetric reconstruction using sophisticated sparse 3D CNN, coarse-to-fine estimation, and voxel culling by visual hull and narrow bands. 
  \item 
  We introduce a novel method to compute a texture atlas map, which captures rich appearance details from high-resolution input images.
  \item 
  We achieve impressive results on standard benchmark datasets Twindom and MultiHuman, significantly reducing the point-2-surface (P2S) precision to approximately 0.2cm from just six input views, with more than $50\%$ error reduction compared to the state-of-the-art methods, including DoubleField~\cite{shao2022doublefield} and PIFuHD~\cite{saito2020pifuhd}.
\end{itemize}
% !TEX program = pdflatex
% !TEX root = main.tex


\section{The Model}

We represent a series of interactions between $N$ individuals as a sequence of weighted directed networks with adjacency matrix $A^t$ for $t=0,1,2,\ldots,T$. For each $t$, its entry $A_{ij}^t$ is the outcome of interactions $i \rightarrow j$ suggesting that $i$ is ranked above $j$. This allows both cardinal and ordinal inputs. For instance, in team sports, $A_{ij}^t$ could be the number of points by which team $i$ beat team $j$, or we could simply set $A_{ij}^t=1$ to indicate that $i$ won and $j$ lost. We can include the case where individuals interact multiple times at time $t$ by summing the corresponding entries.

We assume that the values of $A_{ij}^t$ are influenced by a vector of real-valued ranks $\v{s}^t=(s_{1}^t,\dots, s_{N}^t)$, where $s_i^t$ is $i$'s skill, strength or prestige at time $t$.
To model these interactions, we follow SpringRank's approach of imagining the network as a physical system~\cite{de2018physical}. Specifically, each node $i$ is embedded in $\mathbb{R}$ at position $s_i^t$, and each directed edge $i \rightarrow j$ becomes an oriented spring with a non-zero resting length and displacement $s_i^t-s_j^t$. Since we are free to rescale latent space and the energy scale, we set the spring constant and resting length to $1$. The spring corresponding to an edge $i \rightarrow j$ at time $t$ then has energy
\be\label{eqn:staticH}
H_{ij}(s_i^t,s_j^t)=\f{1}{2} \bup{s_i^t-s_j^t-1}^{2} \, .
\ee
If there were no other effects, the total energy of the system at time $t$ would then be 
\be\label{eqn:totalstaticH}
H^t(\v{s}^t) = \sum_{i,j=1}^{N} A_{ij}^t \,H_{ij}(s_i^t,s_j^t) \, .
\ee
If we determined $\v{s}^t$ by minimizing $H^t$ for each $t$ separately, we would simply be applying the static SpringRank model separately to each ``snapshot'' of the network. This would ignore all previous (and future) interactions, and ignore the hypothesis that ranks change smoothly from one time-step to the next.

% Figure environment removed

To model this smoothness, we also assume a dependence between ranks at successive time-steps. Specifically, we extend the Hamiltonian~\eqref{eqn:totalstaticH} with an extra term that models the \emph{self-interaction} between past and current ranks,
\begin{equation}\label{eqn:selfH}
\Hself^t(\v{s}^t,\v{s}^{t-1}) 
= \frac{\kself}{2} \sum_{i=1}^N (s_i^t-s_i^{t-1})^2 \, .
\end{equation}
This can be seen as a set of additional ``self-springs'' that connect the rank of each individual with its own previous rank. The spring constant $\kself$ parametrizes how smoothly we want the ranks to change from one step to the next. In inference terms, $\kself$ is a hyperparameter which we tune using cross-validation.

Summing over all time-steps $0 < t \le T$ and adding this to the pairwise interactions at each time-step then gives a total energy

\begin{align}\label{eqn:fullH}
\Htotal(\{\v{s}^t\}) = \sum_{t=0}^T H^t(\v{s}^t) + \sum_{t=1}^T \Hself^t(\v{s}^t,\v{s}^{t-1}) \, .
\end{align}
We call this the dynamical SpringRank Hamiltonian. The optimal ranks $\v{s}^0,\v{s}^1,\ldots,\v{s}^T$ are those that minimize it.


There are two ways to minimize $\Htotal$. One is to proceed in an online way, moving forward in time. In this approach, we use the static SpringRank model Eq.~\eqref{eqn:totalstaticH} to find the initial ranks $\v{s}^0$ by minimizing $H^0(\v{s}^0)$. As in Ref.~\cite{de2018physical}, the energy is unchanged if we add a constant to all the ranks; we can break this translational symmetry by setting the mean initial rank $(1/N) \sum_{i=1}^N v_i^0$ to zero.
Then, at each subsequent time-step $t \ge 1$, we update the ranks by taking into account both the new pairwise interactions and the self-springs connecting the ranks with their previous values. Namely, given $\v{s}^{t-1}$ and $A^t$, we find the ranks $\v{s}^t$ that minimize $H^t(\v{s}^t) + \Hself^t(\v{s}^t,\v{s}^{t-1})$.

Since this is a convex function of $\v{s}^t$, we can find its minimum by setting its gradient to zero, or equivalently by balancing all the forces $v_i^t$. This yields a system of linear equations:
\begin{align}\label{eqn:fullsolution}
\rup{ D^{out,t}+D^{in,t}- \bup{A^t + (A^t)^\dagger}+\kself\id} \,\v{s}^t
&=\rup{D^{out,t}-D^{in,t}}\v{1} \nonumber \\& +\kself\, \v{s}^{t-1} \, . 
\end{align}

Here 
$D^{out,t}$ and $D^{in,t}$ are diagonal matrices whose entries are the weighted out- and in-degrees $D^{out,t}_{ii}=\sum_{j}A^t_{ij}$ and $D^{in,t}_{ii}=\sum_{j}A^t_{ji}$; 
$\dagger$ denotes the transpose; 
$\id$ is the identity matrix; 
and $\v{1}$ is the all-ones vector.

The matrix on the left side of~\Cref{eqn:fullsolution} is invertible if $\kself > 0$. In particular, its eigenvector $\v{1}$ has eigenvalue $N \kself$. Thus for each $A^t$ and each $\v{s}^{t-1}$, Eq.~\eqref{eqn:fullsolution} has a unique solution $\v{s}^t$. Overall, Eq.~\eqref{eqn:fullsolution} is similar to the regularized version of SpringRank~\cite{de2018physical} with regularization parameter $\alpha= \kself$. However, unlike the static model, there is a term on the right-hand side containing the previous ranks $\v{s}^{t-1}$, creating a Markovian dependence between successive time-steps. We refer to this model as \dsrfull\ (\dsr).

Importantly the online DSR approach does not actually minimize $\Htotal$, instead solving a sequence of minimization problems, one for each time step. To minimize $\Htotal$ instead, we set $\nabla \Htotal(\v{s}^t) = 0$, solving for the minimizers $\v{s}^t$ over all $N(T+1)$ ranks simultaneously, yielding the following system of equations (SI \Cref{sec:h_total_derive}):

\begin{align}\label{eqn:h_total}
\rup{ D^{out,t}+D^{in,t} - \bup{A^t+(A^t)^\dagger} + 2\kself\id}\,\v{s}^t 
&=\rup{D^{out,t}-D^{in,t}}\v{1} \nonumber\\ 
& +\kself \,\bup{\v{s}^{t-1} + \v{s}^{t+1}} \, . 
\end{align}
This differs from \Cref{eqn:fullH} in that the right-hand side now includes both past and future ranks (which doubles the contribution of $\kself$ on the left). We remove the terms $\v{s}^{t-1}$ and $\v{s}^{t+1}$ for $t=0$ and $t=T$ respectively. This entire system has translational symmetry, since the energy Eq.~\eqref{eqn:fullH} remains the same if we add the same constant to all ranks at all times, but we can again break this symmetry by setting the mean rank to zero.

Additionally, in contrast to \Cref{eqn:fullsolution}, the ranks at $t$ now depend on both $t-1$ and $t+1$, which themselves depend on ranks at adjacent time-steps, so that ranks are affected by interactions in both the past and the future. In computer science, methods like this where the entire history is provided to the algorithm are called \emph{offline}, to distinguish them from \emph{online} approaches that update their results in real time as data becomes available. Thus we refer to this model as \nmdsrfull\ (\nmdsr).  

The cost of solving \Cref{eqn:fullsolution} for a single time-step is the same as static SpringRank with only one additional parameter to be tuned using cross-validation, and there are $T$ such $N$-dimensional equations to be solved successively. On the other hand, \Cref{eqn:h_total} requires solving a single  system of dimension $NT$, whose operator consists of $T$ blocks, each of dimension $N\times N$. While these two approaches feature numbers of non-zero entries that are fundamentally determined by the number of total edges across all time steps, the cost of solving \dsr vs \nmdsr will depend on the particular choice of linear solver~\cite{peng2021solving}.

Philosophically, Eqns.~\eqref{eqn:fullsolution} and~\eqref{eqn:h_total} are trying to do two different things. If we are given all the data $A^0,A^1,\ldots,A^T$ and we want to infer retrospectively how each individual's rank changed over time, it makes sense to include both past and future interactions as in~\eqref{eqn:h_total} so that $s_i^t$ is affected by $i$'s entire history. 

In contrast, \eqref{eqn:fullsolution} can be viewed as modeling each individual's perceived rank at the time, based only on the interactions that have occurred so far.

In principle, one could envisage other ways to formally incorporate an explicit dependence on  $\v{s}^{t-1}$ into the model, and we provide one example in SI \Cref{sec:sidynl}. However, we found that the approaches presented in this Section provide a natural interpretation, result in good prediction performance on both real and synthetic datasets (see \Cref{sec:results}) and are computationally scalable. 

We close this section with two possible extensions to these models. First, in some settings we might have timestamps $t$ that are not successive integers $0,1,\ldots,T$. In this case, if the time interval between two successive times is $\Delta t$, one could scale the spring constant of the self-springs between time-steps as $\kself/\Delta t$. This corresponds to the fact that if we have $\Delta$ identical springs in series, each of which is stretched by $(s^t-s^{t-1})/\Delta$, their total energy is $(1/2)(\kself/\Delta)(s^t-s^{t-1})^2$. The same expression applies if the timestamps are real-valued so that $\Delta$ is not an integer.

Second, if we believe that not just the ranks themselves but their rates of change behave smoothly over time, one could add a momentum term to the Hamiltonian which is quadratic in the discrete second derivative of the ranks. Since
\begin{gather*}
\left( (s^{t+1}-s^t) - (s^t-s^{t-1}) \right)^2
= \left( s^{t+1} - 2 s^t + s^{t-1} \right)^2 \\
= 2 (s^t-s^{t-1})^2 + 2 (s^{t+1}-s^t)^2 - (s^{t+1} - s^{t-1})^2 \, ,
\end{gather*}
this is equivalent to adding a repulsive force, i.e., a spring with negative spring constant, between ranks two time-steps apart. Note that the system nevertheless remains convex: this momentum term is positive semidefinite, so adding it to~\eqref{eqn:fullH} keeps the coupling matrix positive definite except for translational symmetry. Of course, these terms are second-order in time. In the online approach, one would have to determine $\v{s}^0$ from the static model, $\v{s}^1$ from the first-order model~\eqref{eqn:fullsolution}, and then use the model including this momentum term for $\v{s}^t$ for $t \ge 2$. We have not pursued this here, but it may make sense for certain datasets.


\subsection{Moving-window SpringRank}\label{subsec:mwsr}

Before we test the various versions of \dsrfull\ defined above, we consider a simpler model as a baseline. 
The simplest way to extend SpringRank to a dynamical context is to apply the static model to the interactions in a series of ``windows,'' where in each window we sum the interactions over a series of consecutive time-steps. For instance, we can compute $\v{s}^t$ for each $t$ by applying the static model to a window of width $\tau$, i.e., replacing $A^t$ with $\sum_{t'=t}^{t+\tau-1} A^{t'}$. Since these windows overlap, the resulting estimates $\v{s}^t$ will be smooth to some extent, even without imposing an explicit dependence between $\v{s}^t$ and $\v{s}^{t-1}$. We use this method, which we call \mwsrfull\ (\mwsr), as a baseline to compare with the dynamical models presented above.

Roughly speaking, a larger $\tau$ is like a larger self-spring constant $\kself$, since it induces more overlap between windows and thus a stronger correlation between the inferred ranks. However, like a decaying-history approach, \mwsr\ assumes a particular kernel for the importance of past time-steps: namely, that all $t'$ in the window are equally important. In contrast, \dsrfull\ infers the importance of past time-steps by coupling $\v{s}^t$ with $\v{s}^{t-1}$.

However, both models have a free parameter that needs to be tuned, i.e., $\kself$ and $\tau$. A shorter window $\tau$ or smaller spring constant $\kself$ allows the ranks to respond quickly to new interactions, while a longer window or larger spring constant more tightly couples nearby estimates. This trade-off suggests the existence of an optimal window length $\tau_{\opt}$. We tune $\tau$ using a cross-validation procedure as explained in SI \Cref{sisec:tuning}.


\subsection{Generative Model and Synthetic Data}
\label{sec:genmod}

Analogous to a model presented in~\cite{de2018physical}, we propose a probabilistic generative model for dynamic data. It takes as input the ranks $\v{s}^t$ and generates a sequence of weighted directed networks with adjacency matrix $A^t$ at time $t$. One can also imagine models that generate the ranks, for instance with a random walk with Gaussian steps whose log-probability is the self-spring Hamiltonian~\eqref{eqn:selfH}, but we treat $\v{s}^t$ as an input since we want the user of this model to have control over how the ground-truth ranks vary with time.  For instance, in our experiments below we generate synthetic data where the ranks vary sinusoidally.

The generative model has two real-valued parameters: a signal-to-noise ratio or inverse temperature $\beta$, and an overall density of edges $c$. Given the ranks $\v{s}^t$, it generates weighted, directed edges between each pair of nodes $i,j$ independently, as follows. The probability $P_{ij}^t(\beta)$ of $i$ ``beating'' $j$ at time $t$, giving a directed edge $i \to j$, is a logistic function as in~\cite{de2018physical} or the Bradley-Terry-Luce model~\cite{bradley1952,luce1959}:
\bea
\nonumber P_{ij}^t(\beta)=\frac{1}{1+\e^{-2\beta(s_i^t-s_j^t)}} \, .
\eea
The number of such edges, which gives the integer weight $A_{ij}^t$, is then drawn from a Poisson distribution whose mean $\lambda_{ij}^t$ is $cP^t_{ij}\,(\beta)$: 
\be
\label{generative_poiss}
A^t_{ij} \sim \Poi\left(\lambda_{ij}^t=\frac{c}{1+\e^{-2\beta(s_i^t-s_j^t)}}\right).
\ee
Since $P_{ij}^t(\beta) + P_{ji}^t(\beta)=1$, for any pair $i,j$ the total number of interactions $A_{ij}^t + A_{ji}^t$ is Poisson-distributed with mean $c$. The rank differences $s_i^t-s_j^t$ are used only to choose the directions of these edges. This  is equivalent to a model where we define a random multigraph where the number of edges between $i$ and $j$ is $\Poi(c)$, and then we choose the direction of each edge independently according to $P_{ij}^t$.

This is different from the generative model proposed in the static case in~\cite{de2018physical}. In that model the probability that $i$ and $j$ interact depends on $s_i-s_j$ so that nodes are more likely to interact if their ranks are fairly close. This is consistent with SpringRank's assumption that if $i$ beats $j$ then $j$ is below $i$, but not too far below it (since the springs have resting length $1$). This assumption makes sense for some datasets but not for others. By generating synthetic data without this dependence, our intent is to pose a greater challenge to SpringRank by modeling (for example) round-robin tournaments where every team plays each other.

\subsection{Model Evaluation}
\label{sec:testing}

Assessing a ranking model on real datasets is not straightforward since we do not know the true values of the underlying ranks. Nevertheless, we may measure the extent to which inferred ranks are accurate in the sense that they can predict the outcome of new observations. 

There are several performance metrics that can be used for prediction evaluation. From coarse-grained measures capable of predicting the likely winner to more fine-grained measures that also estimate odds, we consider four main metrics in our experiments, detailed in \Cref{sisec:evaluation}. We measure prediction performance using a cross-validation protocol where datasets are divided into training and test sets. The training set is used for hyperparameter tuning and parameter estimation while performance is evaluated on the test set. In order to preserve the chronological ordering of the data, the test set contains future observations, i.e., observations that chronologically follow those used in training. Hyperparameters for each method are tuned using grid-search in order to maximize the performance metrics as described in SI \Cref{sisec:tuning}.





%%% Local Variables:
%%% mode: latex
%%% TeX-master: "main"
%%% End:

\section{Scaling Law implies Slingshot Generalization}
\label{sec:slingshot}
This section is a warmup for our theory, highlighting that the Scaling Law  implies a strong inductive bias ---learning capability that can seem shockingly high from naive application of learning theory. We use a simplistic thought experiment that assumes the Scaling Law continues to hold for text data derived from multiple sources. (This seems roughly true  in practice.) 

 Assume there are $S$ elementary skills in language, and each piece of text applies exactly one of these skills. % they are  viewed as a disjoint union of $k$ equal-sized sub-languages, where the text in the $i$th language is representative of application of the $i$'th skill. 
Then  training and test datasets consists of $S$ (roughly) equal portions, one per sub-language. 
If $D$ is the total dataset size then the size per sub-language is $D/S$. Section~\ref{sec:excessentropy} suggests that excess cross entropy roughly captures the rate of learning, so the average error on applying the {\em union} of $S$ skills on test tasks (involving any of the $S$ skills) is the per-word excess cross-entropy of the model trained on the full language, which scales as $1/D^{0.28}$ according to the Scaling Law. (While this is the average per word, for at least $1/2$ of the the sub-languages the per-word cross-entropy is at most twice of this quantity.) 
By contrast, if we were to train the model  using just data from a single skill, the scaling would be $S^{0.28}/D^{0.28}$, which is worse by a multiplicative
factor $S^{0.28}$, provided the model size is the same in all cases.   Since $S$ is presumably large, training on the full dataset gives hugely better performance on individual skills than if we trained separate models on individual skills.
We call this effect  {\em  Slingshot Generalization}\footnote{Economists might call this phenomenon {\em increasing returns to scale}.} and it requires
no mechanistic understanding of gradient descent or transformers. 

The above effect is closely related to  {\em Effective Data Transfer}, a concept used to empirically quantify efficacy of pre-training. 
Suppose we use a small labeled dataset of size $n$ for supervised  training of a classifier by fine-tuning a pretrained LLM. This usually yields final accuracy much better than training the classifier using vanilla supervised training on  the labeled dataset. The improvement can be thought of as an effective transfer of knowledge (= additional equivalent labeled data) from the model's pretraining. Our discussion here  begins to make this precise as a form of inductive bias. 

\textbf{View from Learning theory:}  Naive application of classical learning theory implies that if the model is trained on a single skill (i.e., sub-language) then its  error on downstream tasks on that skill should be at least the reciprocal of the square root of dataset size, which is $1/\sqrt{D/S}$. However, the above analysis suggests that when the model is trained on the {\em union} of $S$ sub-languages, Scaling Law makes the error on the average skill decrease as  $1/D^{0.28}$. When $S \gg D^{0.44}$, we have $1/\sqrt{D/S} \gg 1/D^{0.28}$, which suggests at first sight that the Scaling Law must violate learning theory, at least in this toy example. This inference is erroneous, however.  Learning theory carves out an exception for inductive bias of the model --- the error estimate of
$1/\sqrt{D/S}$ for training on a single sub-language assumes no inductive bias\footnote{To give a trivial example, if we were to commence training with a model that already has minimum training and test loss, then for sure its error is zero, not $1/\sqrt{D/S}.$}.
The correct conclusion is that training using the union of  $S$ datasets provides a strong inductive bias for learning on each individual dataset. 
This inductive bias presumably arises from the model's architecture and the fact that its parameters can aggregate structural information from all $S$ sub-languages, which improves learning on individual languages. (This effect is intuitively clear to humans: learning your $6$th language is easier than learning your $2$nd.)




%!TEX root =../neurips_main.tex
\ifmain
\section{Offline Learning with Myopic Follower}\label{sec:offline-myopic}
\fi
\ifneurips
\subsection{Offline Learning with Myopic Follower}\label{sec:offline-myopic}
\fi
In this section, we study the problem of offline learning the optimal policy for the leader when the follower is myopic. 
In the offline setting, the offline dataset is collected as $\cD = \{(s_h^i, a_h^i, b_h^i, u_h^i,\pi_h^i)_{h=1}^H\}_{i\in[T]}$.
Here, $\pi^i$ should be thought of as a random variable.
% We consider a data generating process in which the leader's policies across episodes are indepedent.
% , but the prescriptions used  within an episode can be dependent across steps. This can be viewed equivalently as having $T$ pairs of leaders (can be strategic) and followers independently playing this quantal stackelberg game. 
% Under this setting, we suppose that $\tau^i$ has a marginal distribution $\mu$.
We let $\EE_\cD$ denote the expectation with respect to the data generating distribution (also over the randomness of $\pi^i$). 
We study two function approximation schemes, namely the linear function approximation and the general function approximation.

\ifmain
\subsection{Offline Learning for Linear Markov Game}\label{sec:offline-ML}
\fi
\ifneurips
\subsubsection{Offline Learning for Linear MDP}\label{sec:offline-ML}
\fi
% Previously, we study the Offline-MG algorithm for learning the QSE with general function class, which is done by constructing a valid and accurate confidence set for both the environment model and the behavior model.
In this subsection, we  develop a computationally efficient and value iteration-based algorithm for the linear Markov game setting which is defined in \Cref{def:linear MDP}.
{\iffalse
We first give a definition for our linear MDP setting.
\begin{definition}[{Linear MDP}]\label{def:linear MDP}
    For the episodic leader-follower  Markov game, we call it linear if there exists maps $\varpi^*_h:\cS\times\RR^d$ and features  $\phi_h(\cdot,\cdot):\cS\times\cA\times\cB\rightarrow\RR^d$ for any $h\in[H]$ such that
    \begin{align*}
        P_h(s_{h+1}'\given s_h, a_h, b_h) = \inp[]{\phi_h(s_h, a_h, b_h)}{\varpi^*_h(s_{h+1})}_{\RR^d}.
    \end{align*}
    Furthermore, the leader's and the follower's utility admit the following linear factorizations
    \begin{align*}
        u_h(s_h, a_h, b_h) = \inp[]{\phi_h(s_h, a_h, b_h)}{\vartheta^*_h}_{\RR^d}, \quad r_h(s_h, a_h, b_h) = \inp[]{\phi_h(s_h, a_h, b_h)}{\theta^*_h}_{\RR^d},
    \end{align*}
    where $\vartheta^*_h\in\RR^d$ and $\theta^*_h\in\RR^d$ are parameters. 
\end{definition}
\fi}
Recall the guarantee we have for the confidence set based on the negative log-likelihood. A blessing of the myopic follower case is that at each state $s_h$, the follower's quantal response is only a function of the policy $\pi_h$ and the reward $r_h$ with model parameter $\theta_h\in\Theta_h$ at the same step. Therefore, the negative log-likelihood for the follower's behavior at step $h$ is given by
\begin{align}
    \cL_{h}(\theta_h) = - \sum_{i=1}^T \rbr{\eta r_h^{\pi^i, \theta}(s_h^i, b_h^i) - \log \rbr{\sum_{b'\in\cB} \exp\rbr{\eta r_h^{\pi^i, \theta}(s_h^i, b')}}}. \label{eq:myopic-offline-general-MLE loss}
\end{align}
Here, the follower's reward function $r_h^{\pi^i,\theta}$ only depends on $\theta_h$. 
One can 
% write down the negative log-likelihood as \eqref{eq:myopic-offline-general-MLE loss}
% {\color{purple} 
% where $\phi_h^i(b_h)=\inp{\phi_h(s_h^i, b_h^i, \cdot)}{\pi_h^i(\cdot\given b_h^i)}_{\cA_h}$ and $\theta_h\in\RR^d$. }
% and 
construct confidence sets 
$$\CI_{h,\Theta}(\beta)=\cbr{\theta_h\in\Theta_h: \cL_h(\theta_h)\le \inf_{\theta'\in\RR^d}\cL_h(\theta')+\beta}, \quad \forall h\in[H]$$
following the same manner as \eqref{eq:behavior_model_confset}.
Here, we can let each parameter class $\Theta_h$ be a bounded subset of $\RR^d$ with $\nbr{\theta_h}_2\le B_\Theta$.  
In the following, we seperately discuss how to deal with the uncertainty in the environment model and the behavior model by adding penalties in the leader's value functions.

\paragraph{Environment Model Uncertainty Quantification.}
The value interation follows a very similar idea as \citet{zhong2021can, jin2020provably}, but the main difference is that we need to handle the uncertainty in the behavior model parameter $\theta_h$. 
We first give the update of the state-action value functions $\hat U_h(\cdot,\cdot,\cdot)$ at each step.
The idea is to exploit the linear structure $U_h(s_h, a_h, b_h) = u_h(s_h, a_h, b_h) + \TT_h W_{h+1}(s_h, a_h, b_h)=\inp{\phi_h(s_h, a_h, b_h)}{\omega_h}$ with $\omega_h = \vartheta_h^* + \sum_{s_{h+1}}\mu_h(s_{h+1})W_{h+1}(s_{h+1})$,
and solve for the state-action value function $\hat U_h(\cdot,\cdot,\cdot)$ by the following ridge regression,
\#
    & \hat\omega_h  =\argmin_{\omega\in\RR^d} \sum_{i=1}^T \rbr{\phi_h(s_h^i, a_h^i, b_h^i)^\top \omega - u_h^i - \hat W_h(s_{h+1}^i)}^2 + \nbr{\omega}_2^2, \label{eq:linear ridge}\\
    & \hat U_h(s_h, a_h, b_h)  = \phi_h(s_h,a_h,b_h)^\top \hat\omega_h - \Gamma^{(1)}_h(s_h, a_h, b_h),\nonumber
\# 
where $\Gamma^{(1)}_h$ is an uncertainty quantifier \citep{jin2021pessimism,zhong2021can} for the uncertainty in the environment model, and this term is included to ensure pessimism. Here, we can choose $\Gamma^{(1)}_h(s_h, a_h, b_h)= \tilde \cO\big(\sqrt{\phi_h(s_h, a_h, b_h)^\top\Lambda_h^{\dagger}\phi_h(s_h, a_h, b_h)}\big)$ where $$\Lambda_h=\sum_{i=1}^T \phi_h(s_h^i, a_h^i, b_h^i) \phi_h(s_h^i, a_h^i, b_h^i)^\top \allowbreak+ I_d$$ is the kernel obtained from the ridge regression problem \eqref{eq:linear ridge}. One should also be aware that the ridge regression problem \eqref{eq:linear ridge} has a closed form solution $\hat \omega_h\leftarrow \Lambda_h^{-1} (\sum_{i=1}^T \phi_h(s_h^i, a_h^i, b_h^i) (u_h^i + \hat W_{h+1}(s_{h+1}^i))) $. Plugging this closed form solution into \eqref{eq:linear ridge}, we get an update for the leader's U function.

\paragraph{Behavior Model Uncertainty Quantification.}
We next show how to deal with the behavior model uncertainty and find a good policy for the leader.
Recall the confidence set $\CI_{h,\Theta}(\beta)$ we construct for the follower's behavior model. Given the fact that the  follower is myopic, the behavior model at step $h$ is fully characterized by $\theta_h$ and the leader can decides on what policy to use simply by looking at $\hat U_h$ given by \eqref{eq:linear ridge} and $\CI_{h,\Theta}(\beta)$, and take the policy that maximizes the one-step value subject to the \emph{pessimistic} estimation, which gives us the first scheme
\begin{align}
    \textbf{S1:}\quad  \hat\pi_h(s_h)  = \argmax_{\pi_h(s_h) \in\sA} \min_{\theta_h\in\confset_{h, \Theta}(\beta)} \inp[\big]{\hat U_h(s_h, \cdot,\cdot)}{\pi_h \otimes\nu_h^{\pi, \theta}(\cdot,\cdot\given s_h)}_{\cA\times\cB},\label{eq:scheme-1}
\end{align}
where $\hat W_h(s_h)$ is just the optimal value to the maximin problem and we remind the readers that $\sA$ is the prescription space.
% The guarantee for \eqref{eq:scheme-1} follows almost directly from \Cref{thm:Offline-MG}.
% Here, we should choose $\beta\ge T^{-1}\log(H\cN(\Theta_h, T^{-1})/\delta)$, which is given by \Cref{lem:bandit} and by further taking a union bound over $h\in[H]$. 
However, we should note that the problem is highly nonlinear and it is often hard to compute this maximin problem. Note that the inner minimization is included just to ensure pessimism. We should ask ourselves if we can turn the inner minimization problem into a maximization problem by adding penalties for the uncertainty in the behavior model $\theta_h$. By an analysis of the TV distance $D_\TV(\nu_h^{\pi,\theta_h}(\cdot\given s_h), \nu_h^{\pi,\theta_h^*}(\cdot\given s_h))$
% $D_\TV\bigrbr{\nu_h^{\pi_h,\theta_h}, \nu_h^{\pi_h, \theta_h^*}}$ very similar to the one given in \Cref{lem:performance diff informal}
specialized to the linear case,\footnote{See \Cref{lem:response diff-myopic-linear} and \Cref{cor:formal-MLE confset-linear myopic} for more details.} we are able to include a penalty term to ensure pessimism and turn the problem into a maximization for $\forall s_h\in\cS$.
\begin{align}
    \textbf{S2:}\quad  \hat\pi_h(s_h) = \argmax_{\pi_h(s_h) \in\sA, \atop \theta_h\in\confset_{h,\Theta}(\beta)} \inp[\big]{\hat U_h(s_h, \cdot,\cdot)}{\pi_h \otimes\nu_h^{\pi, \theta}(\cdot,\cdot\given s_h)}_{\cA\times\cB} - \Gamma^{(2)}_h(s_h;\pi_h , \theta_h),\label{eq:scheme-2}
\end{align}
where $
\Gamma^{(2)}_h(s_h;\pi_h , \theta_h) = 2 B_U(\eta \xi(s_h;\pi_h, \theta_h)  + C^{(3)} \xi(s_h;\pi_h, \theta_h)^2 )
$ with 
$$C^{(3)}=\rbr{2+\exp\rbr{2\eta B_A}\eta B_A} \eta^2 \exp\rbr{2\eta B_A}/{2}, $$ 
and
\begin{equation}
    \xi(s_h;\pi_h, \theta_h) = \sqrt{\trace\Bigrbr{\bigrbr{T\Sigma_{h,\cD}^{\theta_h} + I_d}^\dagger \Sigma_{s_h}^{\pi_h , \theta_h}}} \cdot \sqrt{8 C_{\eta}^2 \beta + 4B_\Theta^2}.\label{eq:Gamma^2}
\end{equation}
Here, $\Sigma_{h,\cD}^{\theta} = T^{-1} \sum_{i=1}^T \Cov_{s_h^i}^{\pi^i, \theta} \allowbreak [\phi_h^{\pi^i}(s_h^i, b_h)]$ is the data-dependent covariance matrix defined in \eqref{eq:cov matrix},  and $\Sigma_{s_h}^{\pi,\theta}=\Cov_{s_h}^{\pi,\theta} \allowbreak [\phi_h^{\pi_h}(s_h, b_h)]$ is the covariance matrix that actually only depends on $\pi_h(s_h)$ and  parameter $\theta_h$. 
% Moreover, recall that $\nbr{\theta_h}_2 \le B_\Theta$ for all $\theta_h\in\Theta_h$.
Here, we remark that $\Gamma^{(2)}_h$ is a valid uncertainty quantifier which also captures the nonlinear effect of the TV distance $D_\TV(\nu_h^{\pi,\theta_h}(\cdot\given s_h), \nu_h^{\pi,\theta_h^*}(\cdot\given s_h))$ by the second order term and the analysis of $\Gamma^{(2)}_h$ is available in \Cref{lem:response diff-myopic-linear}.
% {\main
% If we look back at \eqref{eq:scheme-2}, another way to think is viewing $\theta_h$ as parts of the leader's \say{policy} but subject to a confidence set constraint.
% \fi}
%  and the target function in the maximization problem is just a pessimistic evaluation of the leader's utility under the \say{joint policy} $(\pi_h, \theta_h)$.

Although Scheme 2 already avoids the maximin optimization in Scheme 1, we are still not satisfied since the  data-dependent covariance matrix $\Sigma_{h,\cD}^{\theta}$ also depends on the optimization variable $\theta_h$, which poses challenges in computation. 
% Also, the confidence set constraint is not easy to satisfy since $\cL_h(\cdot)$ is highly nonlinear. 
One way to deal with the problem is considering a fixed $\theta_h$. As a matter of fact, we are able to replace the $\theta_h$ in \eqref{eq:scheme-2} with the MLE estimator $\hat\theta_{h,\MLE} = \argmin_{\theta_h\in\Theta} \cL_{h}(\theta_h)$, which gives the following scheme,
\begin{align}
    \textbf{S3:}\quad  \hat\pi_h(s_h) = \argmax_{\pi_h(s_h) \in\sA}  \inp[\big]{\hat U_h(s_h, \cdot,\cdot)}{\pi_h \otimes\nu^{\pi, \hat\theta_{h,\MLE}}(\cdot,\cdot\given s_h)}_{\cA\times\cB} - \Gamma^{(2)}_h(s_h;\pi_h , \hat\theta_{h, \MLE}). \label{eq:scheme-3}
\end{align}  
Here, the uncertainty quantifier $\Gamma_h^{(2)}$ is still needed to ensure pessimism.
To bridge the estimation error of $\hat \theta_{h,\MLE}$ in Scheme 3 to the previous two schemes, one can show that both $\Sigma_{h,\cD}^{\hat\theta_{h,\MLE}}$ and $\Sigma_{s_h}^{\pi,\hat\theta_{h,\MLE}}$ are upper and lower bounded by their correspondences with $\theta^*$ plugged in. 
The analysis is available in \Cref{prop:Hessian-ulb}.
% where we define $\Gamma^{(3)}_h(s_h;\pi_h)$ as $
% \Gamma^{(3)}_h(s_h;\pi_h) = 2 B_U(\eta \varrho  + C^{(3)} \varrho^2 )
% $ with $C^{(3)}=\rbr{1+\exp\rbr{2\eta B_A}+\eta B_A \exp\rbr{2\eta B_A}}/{2}$ and, 
% \begin{align*}
%     \varrho = \sqrt{\trace\rbr{\rbr{T\Sigma_{h,\cD}^{\hat\theta_{h,\MLE}} + I_d}^\dagger \Sigma_{s_h}^{\pi_h , \hat\theta_{h,\MLE}}}} \cdot \sqrt{2 C_{\eta}^2 \beta + 4B_\Theta^2}.
% \end{align*}
%
% {\color{purple} commented.
% where $\Psi\in \SSS_+^{d}$ can be any chosen nonnegative definite matrix
% and $\Upsilon_h(\cdot;\cdot,\cdot,\cdot)$ is defined as
% \begin{align}
%     \Upsilon_h(s_h;\pi , \theta_h, \Psi) \defeq \sqrt{\EE_{s_h}^{{\pi ,\theta_h}}\sbr{\phi_h^{\pi }(s_h,b_h)^\top {\Psi}^{\dagger}\phi_h^{\pi }(s_h,b_h)} -\bignbr{\EE_{s_h}^{{\pi ,\theta_h}}\phi_h^{\pi }(s_h,b_h)}_{{\Psi}^{\dagger}}^2}. \label{eq:Upsilon}
% \end{align}
% We remark that the higher order terms in $f$ stem from the nonlinearity of the TV distance between two different responses given by different $\theta_h$.
% Note that $\Upsilon_h(s_h;\pi , \theta_h, \Psi) = \sqrt{\trace(\Psi^\dagger \Sigma_{s_h}^{\pi , \theta_h})}$, where $\Sigma_{s_h}^{\pi ,\theta_h}=\EE_{s_h}^{\pi , \theta_h}\bigsbr{\phi_h^{\pi }{\phi_h^{\pi }}^\top} - \EE_{s_h}^{\pi ,\theta_h}\bigsbr{\phi_h^{\pi }} \cdot \EE_{s_h}^{\pi ,\theta_h}\bigsbr{\phi_h^{\pi }}^\top$.
% The definition of $\Upsilon_h$ is analogue to the previous penalty term $\Gamma^{(1)}_h$, except that $\Upsilon_h$ deals with the uncertainty in the follower's response using the weighted Laplacian $\Sigma_{s_h}^{\pi , \theta_h}$.
% % Thus, it is natural for us to choose $\Psi$
% % Note that the terms with coefficient $\exp(\eta B_A)$ in $f(\cdot)$ are actually of higher order (roughly speaking, $\Gamma^{(2)}_h=\cO(T^{-1/2})$), which suggests a $\cO(\eta\sqrt T)$ rate in our results. 
% Thus, a natural choice for $\hat\theta_h$  would be the MLE estimator $\hat\theta_{h, \MLE}$ as we wish $\hat\theta_h$ to be as close to $\theta_h^*$ as possible and the corresponding choice for $\Psi$ would be the weighted Laplacian $\Psi = \Sigma_{\cD}^{\hat\theta_{h,\MLE}} + T^{-1} I_d$ corresponding to the data \footnote{One can also consider using $\Psi=L_\cD$ which is the standard Laplacian (see \Cref{sec:follow up on bandit} for details). The only difference will be in the concentrability coefficient (as is shown in \eqref{eq:bandit-ub-2}) in the suboptimality. }. 
% See \Cref{sec:follow up on myopic offline} for more details concerning the use of the Laplacians. 
% Thanks to the guarantee we have in \Cref{lem:bandit}, the term $\bignbr{\theta_h^*-\hat\theta_h}_{\Psi}$ could be replaced by its corresponding upperbound $C_\eta^2 T^{-1}\log(H\cN(\Theta_h, T^{-1})/\delta)+T^{-1} B_\Theta$ without knowing $\theta_h^*$ by \Cref{lem:bandit} \footnote{Note that we plug in $\delta/H$ for a union upper bound over $h\in[H]$.}.
%
% The benefit of Scheme 2 is in computation since $\Psi$ and $\hat\theta_{h,\MLE}$ is fixed in this policy optimization step. 
% One issue concerning the second scheme is that we need $\Sigma_\cD^{\hat\theta_{h, \MLE}}$ to have sufficient coverage, which might not be true even if $\Sigma_{\cD}^{\theta_h^*}$ has sufficient coverage since $\hat\theta_{h,\MLE}$ is correlated with the data
% \footnote{In the worst case, the concentrability coefficient can be $\exp(2\eta B_A)$ larger than Scheme 1 due to the distribution shift $\onbr{\nu_h^{\pi , \theta_h^*}/\nu_h^{\pi , \hat\theta_h}}_\infty$.}. 
% To have a better offline guarantee, one thing we can do is to optimize the pessimistic lower bound given in \eqref{eq:scheme-2} over all possible choices of $\theta_h\in\confset_h(\beta)$ 
% instead of randomly picking one, which gives us the third scheme,
%
% \begin{align}
%     \textbf{S3:}\quad  \hat\pi  = \argmax_{\pi \in\sA}  \inp[\big]{\hat U_h(s_h, \cdot,\cdot)}{\pi \otimes\nu^{\pi , \hat\theta_h}(\cdot,\cdot\given s_h)}_{\cA\times\cB} - 2 B_U f\rbr{\Gamma^{(2)}_h(s_h;\pi , \hat\theta_h)}, 
%     % \label{eq:scheme-3}
% \end{align}  
% where the choice for $\Psi=\Sigma_\cD^{\theta_h} + T^{-1} I_d$ remains the same. Here, we should upper bound $\bignbr{\theta_h^*-\hat\theta_h}_{\Psi}$ in $\Gamma^{(2)}$ with $C_\eta^2 T^{-1}\log(H\cN(\Theta_h, T^{-1})/\delta)+T^{-1} B_\Theta^2 + \beta$ by \eqref{eq:bandit-ub-1} in \Cref{lem:bandit}. We remark that Scheme 3 will have the same kind of guarantee as Scheme 1, while no computation of maximin is required. 
% }
Finally, the \textbf{M}aximal \textbf{L}ikelihood \textbf{E}stimation with \textbf{P}essimistic \textbf{V}alue \textbf{I}teration (MLE-PVI) algorithm is summarized as the following.
\begin{algorithm}[H]
    \begin{algorithmic}[1]
    \Require {$\eta, \cD$}
    \State Initialize $\hat W_{H+1}=0$.
    \For{$h=H, H-1,\dots, 1$}
    \State Obtain kernel $\Lambda_h\leftarrow \sum_{i=1}^T \phi_h(s_h^i, a_h^i, b_h^i) \phi_h(s_h^i, a_h^i, b_h^i)^\top + I$. 
    \State Solve the ridge regression for $\hat \omega_h\leftarrow \Lambda_h^{-1} \rbr{\sum_{i=1}^T \phi_h(s_h^i, a_h^i, b_h^i) \bigrbr{u_h^i + \hat W_{h+1}(s_{h+1}^i)}} $.
    \State Update $\hat U_h(\cdot,\cdot,\cdot)\leftarrow \phi_h(\cdot,\cdot,\cdot)^\top \hat \omega_h - \Gamma_h^{(1)}(\cdot,\cdot,\cdot)$ and truncate to $[0, H-h+1]$.
    \State Compute $(\hat W_h(s_h), \hat\pi_h(s_h) )$ as the optimal value and optimal solution to S1 \eqref{eq:scheme-1}, S2 \eqref{eq:scheme-2}, or S3 \eqref{eq:scheme-3} for each $s_h\in\cS$.
    \EndFor
    \Ensure $\hat\pi=(\hat\pi_h)_{h\in[H]}$.
    \end{algorithmic}
    \caption{Offline MLE-PVI Algorithm for Myopic Follower under Linear Markov Game}
    \label{alg:PMLE}
\end{algorithm}
We have the following theoretical guarantee for \Cref{alg:PMLE}.
% \Siyu{satisfies the compliance condition, i.e., $\PP_\cD(u_h^i=u, s_{h+1}^i=s\given \tau^{i-1}, \{s_{h'}^i, \pi_{h'}^i, a_{h'}^i, b_{h'}^i\}_{h'\in[h]}) =\PP(u_h=u, s_{h+1}=s\given s_h=s_h^i, a_h=a_h^i, b_h=b_h^i)$ \citep{jin2021pessimism, zhong2022pessimistic} where $\tau^{i-1}$ is the historical data up to episode $i-1$. We remark that $\xi$ should satisfies $\xi \ge \rbr{2 C_{\eta}^2 \beta + B_\Theta^2}$.}
\begin{theorem}[{Suboptimality for MLE-PVI}]\label{thm:PMLE-VI-myopic}
    Suppose the data compliance condition \begin{align}\label{eq:data compliance}
        &\PP_\cD(u_h^i=u, s_{h+1}^i=s\given \tau^{i-1}, \{s_{h'}^i, \pi_{h'}^i, a_{h'}^i, b_{h'}^i\}_{h'\in[h]}) \nend
        &\quad =\PP(u_h=u, s_{h+1}=s\given s_h=s_h^i, a_h=a_h^i, b_h=b_h^i), \quad \forall h\in[H], i\in[T], 
    \end{align} 
    holds.
    We choose $\Gamma^{(1)}_h(\cdot,\cdot,\cdot) \ge C_1 d H \allowbreak \sqrt{\log(2d H T/\delta)}\cdot \sqrt{\phi_h(\cdot,\cdot,\cdot)^\top \Lambda_h^{-1}\phi_h(\cdot,\cdot,\cdot)}$ for some universal constant $C_1>0$ and $\beta \ge C_2 d\log(H(1+\eta T^2)\delta^{-1})$ for some universal constant $C_2>0$. For the PMLE-VI algorithm under the above three schemes, we have with probability at least $1-2\delta$ that
    \begin{align*}
        \subopt(\hat\pi) \le \sum_{h=1}^H 2\EE^{\pi^*, \nu^{\pi^*}} \sbr{\Gamma^{(1)}_h(s_h, a_h, b_h) + \Gamma^{(2)}_h(s_h;\pi_h^*, \theta_h')},
    \end{align*}
    where $\theta_h' = \theta_h^*$ for Scheme 1 and Scheme 2, and $\theta_h' = \hat\theta_{h, \MLE}$ for Scheme 3.
    % where the second term is given by the following configurations of these schemes,
    % \begin{itemize}
    %     \item[(i)] By using Scheme 1 with confidence set width $\beta\ge T^{-1}\log(H\cN(\Theta_h, T^{-1})/\delta)$, we have
    %     $$ \zeta_h = 2 B_U C_{\eta}
    %     \sqrt{\rbr{\frac 1 T\log\rbr{\frac{H \cN(\Theta, 1/T)}{\delta}} + \beta}}\cdot \EE^{\pi^*}\sbr{\Upsilon_h\rbr{s_h; \pi_h^*, \theta_h^*, \Sigma_\cD^{\theta_h^*}}}$$;
    %     \item[(ii)] By using Scheme 2 with  $\hat\theta_h = \hat\theta_{h,\MLE}$, $\Psi = \Sigma_{\cD}^{\hat\theta_{h,\MLE}} + T^{-1} I_d$, and replacing $\bignbr{\theta_h^*-\hat\theta_h}_{\Psi}$ in \eqref{eq:Gamma^2} by $\iota_h \ge \sqrt{C_\eta^2 T^{-1}\log(H\cN(\Theta_h, T^{-1})/\delta) + T^{-1}B_\Theta^2}$, we have
    %     $$ \zeta_h = 4 B_U \iota_h \EE^{\pi^*}\sbr{\Upsilon_h\rbr{s_h; \pi_h^*, \hat\theta_{h, \MLE}, \Sigma_\cD^{\hat\theta_{h,\MLE}}}} ;$$
    %     \item[(iii)] By using Scheme 3 with $\Psi = \Sigma_{\cD}^{\theta_{h}} + T^{-1} I_d$, $\beta\ge T^{-1}\log\rbr{H\cN(\Theta_h, T^{-1})/\delta}$ and replacing $\bignbr{\theta_h^*-\hat\theta_h}_{\Psi}$ in \eqref{eq:Gamma^2} by $ \iota_h \ge \sqrt{C_\eta^2 (T^{-1}\log(H\cN(\Theta_h, T^{-1})/\delta)+\beta + T^{-1}B_\Theta^2)}$, we have 
    %     $$ \zeta_h = 2 B_U \iota_h \EE^{\pi^*}\sbr{\Upsilon_h\rbr{s_h; \pi_h^*, \theta_h^*, \Sigma_\cD^{\theta_h^*}}}. $$
    % \end{itemize}
    % Here, $\Upsilon_h(\cdot;\cdot,\cdot,\cdot)$ is defined in \eqref{eq:Upsilon}
    \begin{proof}
        See \Cref{sec:proof-PMLE-VI-myopic} for a detailed proof.
    \end{proof}
\end{theorem}
We give the following corollary that characterizes the distribution shift issue. 
\begin{corollary}[Distribution shift]\label{rmk:MLE-PVI-dist-shift}
    Suppose for the leader's side, we have with high probability that 
    $
    % \begin{align}
    \Lambda_h\succeq I + c_1 T \EE^{\pi^*,\nu^{\pi^*}}[\phi_h\phi_h^\top]
    % \label{eq:MLE-PVI-coverage-1}
    % \end{align}
    $ for some constant $c_1>0$, 
    and the for the follower's side, we have with high probability 
    \begin{align}
    &I + {\ts\sum_{t=1}^T }\EE^{\pi^i, \nu^{\pi^i}} \bigsbr{(\Upsilon_h^{\pi^i} \phi_h)  (\Upsilon_h^{\pi^i} \phi_h)^\top \given s_h^i}  \succeq I + c_2 T\EE^{\pi^*, \nu^{\pi^*}} \bigsbr{(\Upsilon_h^{\pi^*} \phi_h) (\Upsilon_h^{\pi^*} \phi_h)^\top} , 
    \label{eq:MLE-PVI-coverage}
    \end{align}
    for some constant $c_2>0$,
    where $\Upsilon_h^{\pi}\phi $ is a short hand of $(\Upsilon_h^{\pi}\phi)(s_h, b_h)$. We then have for Scheme 1 and Scheme 2, 
    \begin{align*}
        {\subopt(\hat\pi) 
        \lesssim \frac{d^{3/2}H^2} {\sqrt{c_1 T}} +  \eta C_\eta H^{2}d \cdot \sqrt{\frac{1}{c_2 T}} +  e^{4\eta B_A} (\eta C_\eta)^3 H^2  d^2 \cdot  \frac{1}{c_2 T}, }
    \end{align*}
    and for Scheme 3, 
    \begin{align*}
    { \subopt(\hat\pi) 
    \lesssim \frac{d^{3/2}H^2} {\sqrt{c_1 T}} + e^{2\eta B_A} \eta C_\eta H^{2}d \cdot \sqrt{\frac{1}{c_2 T}} +  e^{8\eta B_A} (\eta C_\eta)^3 H^2  d^2 \cdot  \frac{1}{c_2 T}.}
    \end{align*}
\end{corollary}
\begin{proof}
    See \Cref{sec:proof-MLE-PVI-dist-shift} for a detailed proof.
\end{proof}
We note that \eqref{eq:MLE-PVI-coverage} is similar to the standard sufficient coverage condition in linear MDP but 
customized for linear QRE, where the operator $\Upsilon_h^{\pi^*}$  defined in \eqref{eq:Upsilon} plays a key role in the distribution shift. In particular, \eqref{eq:MLE-PVI-coverage} not only requires coverage over the trajectory induced by $\pi^*$, but also requires richness in the leader's prescription $\pi^i(s_h)$ at those states visited under $\pi^*$.
To understand this point, we note that if the leader announces the same policy $\pi^0$ for all the time, the follower always acts according to the same reward $r_h^{\pi^0} (s_h, b_h)= \la r_h(s_h,\cdot,b_h), \pi^0_h(\cdot\given s_h, b_h)\ra_\cB$, which is only a linear subspace of the reward function and the leader cannot anticipate the follower's quantal response for other policies. 

We next understand the first order terms, i.e., $\cO(T^{-1/2})$ terms in the suboptimality. The first term characterizes the leader's Bellman error, which is standard for RL problems. 
The second term characterizes the follower's first-order quantal response error (QRE). The first-order QRE term suffers from an $\exp(2\eta B_A)$ coefficient only in Scheme 3. We remark that this is because we fix the follower's quantal response using the MLE estimator in \eqref{eq:scheme-3} while Scheme 1 and Scheme 2 allow us to pick a more refined estimator $\hat\theta_h$ in the confidence set at the cost of heavier computation.
% \Cref{thm:PMLE-VI-myopic-neurips}
% We next show an ensurance result which says that the condition \eqref{eq:MLE-PVI-coverage} is not any stronger than \eqref{eq:MLE-PVI-coverage-1}. 
% % $\Gamma^{(2)}_h(s_h;\pi,\hat\theta_{h,\MLE})\le \exp\rbr{4\eta B_A} \Gamma^{(2)}_h(s_h;\pi, \theta_h^*)$. 
% We let $\L=\diag(u)-u u^\top$. We can rewrite $\Sigma_{h,\cD}^u$ and $\Sigma_{s_h}^{\pi^*, u}$ as 
% \begin{align*}
%     \Sigma_{h,\cD}^u = T^{-1}\sum_{i=1}^T \phi_h^{\pi^i}(s_h^i, \cdot) \L \phi_h^{\pi^i}(s_h^i, \cdot)^\top,\qquad  \Sigma_{s_h}^{\pi^*, u} = \phi_h^{\pi^*}(s_h, \cdot)\L \phi_h^{\pi^*}(s_h, \cdot)^\top.
% \end{align*} 
% We define $\H^{\pi,\theta_h} = \diag(\nu_h^{\pi,\theta_h}(\cdot\given s_h)) - \nu_h^{\pi,\theta_h}(\cdot\given s_h)\nu_h^{\pi,\theta_h}(\cdot\given s_h)^\top$ 
% and rewrite $\Sigma_{h,\cD}^{\theta_h^*}$ and $\Sigma_{s_h}^{\pi^*, \theta_h^*}$ as 
% \begin{align*}
%     \Sigma_{h,\cD}^{\theta_h^*} = T^{-1}\sum_{i=1}^T \phi_h^{\pi^i}(s_h^i, \cdot) \H^{\pi^i,\theta_h^*} \phi_h^{\pi^i}(s_h^i, \cdot)^\top,\qquad  \Sigma_{s_h}^{\pi^*, u} = \phi_h^{\pi^*}(s_h, \cdot) \H^{\pi^*, \theta_h^*} \phi_h^{\pi^*}(s_h, \cdot)^\top,
% \end{align*}
% We next see what guarantee we have for the the distribution shift induced by $\Gamma^{(2)}_h$.
% Suppose that 
% \begin{align*}
%     T \Sigma_{h,\cD}^{\theta_h^*}+I = \sum_{i=1}^T \Sigma_{s_h^i}^{\pi^i, \theta_h^*} + I \succeq c T \EE_\cD \sbr{\Sigma_{s_h^i}^{\pi^i, \theta_h^*}} + I, 
% \end{align*}
% with high probability, 
% where the expectation is taken for both $\pi^i$ and $s_h^i$.


% Since $\Sigma_{h,\cD}^u$ now only depends on $\phi_h^{\pi^i}$ and $s_h^i$, 
% and $\Sigma_{s_h}^{\pi^*,u}$



% \todo{to be discussed, results implication, and dependent data.}

% \Zhuoran{Corrollary}
\ifneurips
\subsubsection{Offline Learning with General Function Class}\label{sec:Offline-MG}
\fi
\ifmain
\subsection{Offline Learning with General Function Class}\label{sec:Offline-MG}
\fi
In this subsection, we carry out the offline learning scheme with general function approximation.
For the leader's side, we propose to learn the environment model by minimizing the squared loss of the Bellman error over the U function for each policy $\pi$. 
For consistency, we still assume that the follower's reward function at step $h\in[H]$ lies in some general function class parameterized by $\theta_h\in\Theta_h$. We let $\Theta=\{\Theta_h\}_{h\in[H]}$.
Suppose $\cU:\cS\times\cA\times\cB\rightarrow\RR$ is a given function class for the leader's state-action value function. 
Following the idea of Bellman-consistent pessimism \citep{xie2021bellman}, we define a loss function for the environment model error as
\begin{align}
    &\ell_h(U_h', U_{h+1}, \theta_{h+1}, \pi) \nend
    &\quad = \sum_{i=1}^T \rbr{U_h'(s_h^i, a_h^i, b_h^i) - u_h^i -  \inp[\big]{U_{h+1}(s_{h+1}^i, \cdot, \cdot)}{\pi_{h+1}\otimes \nu_{h+1}^{\pi, \theta}(\cdot, \cdot\given s_{h+1}^i)}}^2. \label{eq:myopic-offline-general-Bellman loss}
\end{align}
Intuitively, the loss $\ell_h$ going to zero means no Bellman error for the value functions between step $h$ and step $h+1$ under $\pi$ and $\nu^{\pi,\theta}$.
Note that the unknown parameters are $\theta=\{\theta_h\}_{h\in[H]}\in\Theta$ and $\{U_h\}_{h\in[H]}\in\cU^{\otimes H}$.
% where we take $\cU$ as a the function class for the leader's U function.
Based on the loss functions defined in \eqref{eq:myopic-offline-general-MLE loss} and \eqref{eq:myopic-offline-general-Bellman loss}, we can construct a confidence set for each leader's policy $\pi$ as
\begin{align}
    &\CI_{\cU, \Theta}^\pi(\beta) \nend
    &\quad= \cbr{
    (U,\theta)\in\cU^{\otimes H}\times\Theta:
    \rbr{ \ds
        \cL_h(\theta_h)-\inf_{\theta'\in\Theta_h}\cL_h(\theta') \le \beta 
    \atop \ds
        \ell_h(U_h, U_{h+1}, \theta_{h+1}, \pi) - \inf_{U'\in\cU_h} \ell_h(U', U_{h+1}, \theta_{h+1}, \pi)\le H^2\beta}, 
    \forall h\in[H]}. \label{eq:myopic-offline-general-confset}
\end{align}
The first condition in \eqref{eq:myopic-offline-general-confset} characterizes a valid and accurate confidence set for the follower's behavior model as we have done in \eqref{eq:bandit-ub-1}. For the second condition in \eqref{eq:myopic-offline-general-confset}, if certain realizability and completeness conditions are satisfied, we have guarantee on small leader's Bellman errors \citep{xie2021bellman, lyu2022pessimism}, which characterizes the uncertainty in the environment model.
Combining these two guarantees, we can therefore expect $\CI_{\cU, \Theta}^\pi (\beta)$ to be a valid and accurate confidence set for both the environment and the behavior models.
Following the principle of pessimism, we can output the policy $\hat\pi$ as, 
\begin{align}
    \hat\pi=\argmax_{\pi\in\Pi} \min_{(U, \theta)\in\CI_{\cU,\Theta}^\pi(\beta)} \EE_{s_1\sim\rho_0} \sbr{\inp[\big]{U_1(s_1, \cdot, \cdot)}{\pi_1\otimes\nu_1^{\pi, \theta}(\cdot,\cdot\given s_1)}_{\cA \times \cB}}.\label{eq:offline-MG-pi^hat}
\end{align}
% Then, we just greadily take $\hat\pi(s_h) = \argmax_{\pi_h\in\sA} \inp[\big]{\hat U_h(s_h, \cdot, \cdot)}{\pi_h\otimes\nu_h^{\pi_h, \hat\theta_h}(\cdot,\cdot\given s_h)}$.
To present our results, we first define an optimistic Bellman operator $\TT_h^{*, \theta}:\cF(\cS\times\cA\times\cB)\rightarrow \cF(\cS\times\cA\times\cB)$ for the leader as
\begin{align}\label{eq:define optimistic Bellman opt}
    &\bigrbr{\TT_h^{*,\theta} f} (s_h, a_h, b_h) \nend
    &\quad = u_h(s_h, a_h, b_h) +  \EE\sbr{\max_{\pi_{h+1}(s_{h+1})\in\sA}\bigdotp{f(s_{h+1}, \cdot, \cdot)}{\pi_{h+1}\otimes\nu_{h+1}^{\pi,\theta}(\cdot,\cdot\given s_{h+1})}_{\cA\times\cB}\Biggiven s_h, a_h, b_h}.
\end{align}
Here, the expectation is taken with respect to $s_{h+1}\sim P_h(\cdot\given s_h, a_h, b_h)$. 
We now summarize our offline policy learning with \textbf{M}aximum \textbf{L}ikelihood with \textbf{B}ellman \textbf{C}onsistent \textbf{P}essimism (MLE-BCP) algorithm together with its theoretical guarantee.
\begin{algorithm}[H]
    \begin{algorithmic}[1]
    \Require {$\eta, \cD$}
    \State Construct confidence set $C_{\cU,\Theta}^\pi(\beta)$ by \eqref{eq:myopic-offline-general-confset}. 
    \State Solve for the policy $\hat\pi$ with Bellman consistent pessimism in \eqref{eq:offline-MG-pi^hat}. 
    \Ensure $\hat\pi=(\hat\pi_h)_{h\in[H]}$.
    \end{algorithmic}
    \caption{Offline MLE-BCP for Myopic Follower under General Function Approximation}
    \label{alg:MLE-BCP}
\end{algorithm}
\begin{theorem}[{Suboptimality for MLE-BCP}]\label{thm:Offline-MG}
    Suppose that each trajectory in the offline dataset is independently collected.
    Suppose that the following conditions hold for model class $\Theta$ and function class $\cU$:
\begin{itemize}
    \item[(i)] (\textit{Realizability}) There exists $\theta^*\in\Theta$ such that $r_h^{\theta^*}=r_h$ for any $h\in[H]$. For any $\pi\in\Pi, \theta\in\Theta$, there exists $U\in\cU$ such that $U_h = \TT_h^{\pi,\theta} U_{h+1}$ for any $h\in[H]$;
    \item[(ii)] (\textit{Completeness}) For any $U\in\cU$ and $\pi\in\Pi, \theta\in\Theta$, there exists $U'\in\cU$ such that $U'=\TT_h^{\pi,\theta} U_{h+1}$ for any $h\in[H]$. 
\end{itemize}
    By choosing $\beta = c\cdot \max\cbr{\log(H \cN_\rho(\cY, T^{-1})\delta^{-1}), \log(H\cN_\rho(\Theta, T^{-1})/\delta)}$ for some universal constant $c>0$, where the covering number for $\Theta$ and $\cY$ are defined in \eqref{eq:cN-Theta-myopic} and \eqref{eq:cN-cY}, respectively, 
    we have for the offline algorithm \eqref{eq:offline-MG-pi^hat} that 
    \begin{align*}
        \subopt(\hat\pi) 
        &\lesssim \max_{U\in\cU,\theta\in\Theta, h\in[H]}
            \sqrt{\frac{{{\bignbr{{ U_h  -  \TT_h^{\pi^*,\theta}  U_{h+1}} }_{2, d^{\pi^*}}^2}}}{{{\bignbr{ U_{h} - \TT_{h}^{\pi^*,\theta}  U_{h + 1}}_{2,\cD}^2}}}} 
        \cdot H^2\sqrt{\beta T^{-1}}  \nend
        &\qquad + \max_{\theta\in\Theta, h\in[H]}\sqrt{ \frac{{\bignbr{\Upsilon_h^{\pi^*} (r_h^\theta - r_h^{\theta^*})}_{2, d^{\pi^*}}^2}}{\bignbr{{\Upsilon_h^{\pi^i} (r_h^\theta - r_h^{\theta^*})}}_{2,\cD}^2}} \cdot  H^2 \eta C_\eta \sqrt{\beta T^{-1}}\nend
        &\qquad  +   \max_{\theta\in\Theta, h\in[H]} {
            \frac{{\bignbr{\Upsilon_h^{\pi^*} (r_h^\theta - r_h^{\theta^*})}_{2, d^{\pi^*}}^2}}{\bignbr{{\Upsilon_h^{\pi^i} (r_h^\theta - r_h^{\theta^*})}}_{2,\cD}^2}
        }\cdot  H^2 \exp(4\eta B_A) (\eta C_\eta)^3 \beta T^{-1},
    \end{align*}
    where $C_\eta = \eta^{-1}+B_A$ and $\lesssim$ only hides universal constants.
    \begin{proof}
        See \Cref{sec:proof-offline-MG} for a detailed proof.
    \end{proof}
\end{theorem}
\Cref{thm:Offline-MG} establishes the suboptimality for offline learning the optimal policy using general function approximation.
Similar to the linear case, the first two terms characterizes the leader's Bellman error and the follower's first order quantal response error, respectively. 
The only exponential term appears in the follower's second order quantal response error term, which is roughly of order $\cO(T^{-1})$.
In particular, the concentrability coefficients that address the distribution shift issue are characterized by the ratio in both the Bellman error and the QRE error.


%! TEXroot =main.tex
\ifmain
\section{Online Learning with Myopic Follower} \label{sec:myopic-online}
\fi 
\ifneurips
\subsection{Online Learning with Myopic Follower} \label{sec:myopic-online}
\fi
In the previous section, we address the problem of offline learning the QSE with myopic follower under both general function class and linear MDP setting. In this section, we move on to the online scenario with myopic follower. Specifically, the game proceeds as the following. At state $s_h$ in episode $t$, the leader announces her prescription $\alpha_h^t:\cB\rightarrow\Delta(\cA)$,  and the myopic follower picks an action $b_h^t$. The leader then pick an action $a_h^t\sim \alpha_h^t(\cdot\given b_h^t)$ and the state then transits to $s_{h+1}^t$. The game at episode $t$ ends at the $H$-th step and a new episode begins next. We also study the online problem both under the general function class and the linear MDP setting.
% \todo{Define operator $\varPsi_h^\pi:\cF(\cS\times\cA\times\cB)\rightarrow \cF(\cS\times\cB)$ as 
% \begin{align*}
%     \rbr{\varPsi_h^\pi f}(s_h, b_h) = \dotp{\pi_h(\cdot\given s_h, b_h)}{f(s_h, \cdot, b_h)} - \dotp{\pi_h\otimes \nu_h^{\pi}(\cdot,\cdot\given s_h)}{f(s_h,\cdot,\cdot)}. 
% \end{align*}
% }

\ifmain
\subsection{Online Learning for Linear Markov Game}
\label{sec:online-myopic-linear}
\fi
\ifneurips
\subsubsection{Online Learning for Linear MDP}
\label{sec:online-myopic-linear}
\fi
The case for online learning with linear MDP is not so different from the offline one, except for the fact that we have to incorporate optimism for exploration. This is nothing much but just flipping the sign of the penalties and turn them into bonuses. Following the same spirit of \Cref{sec:offline-ML}, we simply present our algorithm here for the sake of completeness. 
For updating the U function, we add a bonus to the ridge regression result,
\begin{gather}
    \hat \omega_h^t\leftarrow (\Lambda_h^t)^{-1} \rbr{\sum_{i=1}^{t-1} \phi_h(s_h^i, a_h^i, b_h^i) \bigrbr{u_h^i + \hat W_{h+1}(s_{h+1}^i)}}, \label{eq:linear ridge}\\
    \hat U_h^t(s_h, a_h, b_h) = \phi_h(s_h,a_h,b_h)^\top \hat\omega_h^t + \Gamma^{(1,t)}_h(s_h, a_h, b_h),\nonumber
\end{gather}
where we choose $\Gamma^{(1,t)}_h(s_h, a_h, b_h)= \cO(\sqrt{\phi_h(s_h, a_h, b_h)^\top\Lambda_h^{\dagger}\phi_h(s_h, a_h, b_h)})$ where $\Lambda_h=\sum_{i=1}^T  \phi_h(s_h^i, a_h^i, b_h^i)\allowbreak \phi_h(s_h^i, a_h^i, b_h^i)^\top \allowbreak+ I_d$. For updating the W function, 
we still define the negative loglikelihood as the one given in \eqref{eq:myopic-offline-general-MLE loss},
\begin{align}
    \cL_{h}^t(\theta_h) = - \sum_{i=1}^{t-1} \rbr{\eta r_h^{\pi^i, \theta}(s_h^i, b_h^i) - \log \rbr{\sum_{b'\in\cB} \exp\rbr{\eta r_h^{\pi^i, \theta}(s_h^i, b')}}}. \label{eq:online-MG-MLE loss}
\end{align}
We obtain an optimistic policy via the following two schemes,
\begin{align}
    \textbf{S4:}\quad  \hat\pi_h^t(s_h)  &= \argmax_{\pi_h(s_h) \in\sA
    \atop \theta_h\in\confset_{h, \Theta}^t(\beta)} \inp[\big]{\hat U_h^t(s_h, \cdot,\cdot)}{\pi_h \otimes\nu_h^{\pi , \theta}(\cdot,\cdot\given s_h)}_{\cA\times \cB}, \quad \forall s_h\in\cS_h,\label{eq:scheme-4}\\
    \textbf{S5:}\quad  \hat\pi_h^t(s_h)  &= \argmax_{\pi_h(s_h) \in\sA}  \inp[\big]{\hat U_h^t(s_h, \cdot,\cdot)}{\pi_h \otimes\nu^{\pi , \hat\theta_{h,\MLE}^t}(\cdot,\cdot\given s_h)}_{\cA\times \cB} + \Gamma^{(2,t)}_h(s_h;\pi_h , \hat\theta_{h, \MLE}^t). \label{eq:scheme-5}
\end{align}
which follow from \eqref{eq:scheme-1} and \eqref{eq:scheme-3}, respectively. Here, we denote by $\CI_{h,\Theta}^t(\beta) = \{\theta_h\in\Theta_h: \cL_h^t(\theta)\le \min_{\theta_h'}\cL_h^t(\theta_h')+\beta\}$ the confidence set at step $t$ with $\cL_h^t(\theta)$ defined in \eqref{eq:online-MG-MLE loss}. Moreover, $\hat\theta_{h,\MLE}^t$ is the MLE estimator that minimizes the negative log-likelihood $\cL_h^t(\cdot)$, and $\Gamma_h^{(2,t)}(s_h;\pi_h,\theta_h)=2 H(\eta \xi  + C^{(3)} \xi^2 )
$ with $C^{(3)}=\eta^2 \exp(2\eta B_A) (2+\eta B_A \exp(2\eta B_A))/2$ and, 
\begin{equation}
    \xi = \sqrt{\trace\rbr{\bigrbr{\Sigma_{h, t}^{\theta_h} + I_d}^\dagger \Sigma_{s_h}^{\pi_h , \theta_h}}} \cdot \sqrt{8 C_{\eta}^2 \beta + 4B_\Theta^2}.\label{eq:Gamma^2-online}
\end{equation}
where $\Sigma_{h,t}^{\theta_h} = \sum_{i=1}^{t-1} \Cov_{s_h^i}^{\pi_h^i, \theta_h} \allowbreak [\phi_h^{\pi_h^i}(s_h^i, b_h)]$ is the covariate matrix and $\Sigma_{s_h}^{\pi,\theta}=\Cov_{s_h}^{\pi,\theta} \allowbreak [\phi_h^{\pi_h}(s_h, b_h)]$. We summary the \textbf{M}aximal \textbf{L}ikelihood \textbf{E}stimation with \textbf{O}ptimistic \textbf{V}alue \textbf{I}teration (MLE-OVI) algorithm as the following.
\begin{algorithm}[H]
    \begin{algorithmic}[1]
    \Require {$\eta, \cD$}
    \State Initialize $\cD=\emptyset$.
    \For{$t=1,\dots, T$}
    \State Initialize $\hat W_{H+1}^t=0$
    \For{$h=H, H-1,\dots, 1$}.
    \State Obtain kernel $\Lambda_h^t\leftarrow \sum_{i=1}^{t-1} \phi_h(s_h^i, a_h^i, b_h^i) \phi_h(s_h^i, a_h^i, b_h^i)^\top + I$. 
    \State Solve the ridge regression for $\hat \omega_h^t\leftarrow \rbr{\Lambda_h^t}^{-1} \rbr{\sum_{i=1}^{t-1} \phi_h(s_h^i, a_h^i, b_h^i) \bigrbr{u_h^i + \hat W_{h+1}^t (s_{h+1}^i)}} $.
    \State Update $\hat U_h^t(\cdot,\cdot,\cdot)\leftarrow \phi_h(\cdot,\cdot,\cdot)^\top \hat \omega_h^t + \Gamma_h^{(1, t)}(\cdot,\cdot,\cdot)$ and truncate to $[0, H-h+1]$.
    \State Compute $\hat W_h^t(s_h)$ and $\hat\pi_h^t(s_h)$ as the optimal value and optimal solution to S4 \eqref{eq:scheme-4} or S5 \eqref{eq:scheme-5}.
    \EndFor
    \State Announce $\hat\pi^t$ and collect a trajectory $\tau^t = \{(s_h^t, a_h^t, b_h^t, \hat\pi_h^t)\}_{h\in[H]}$. 
    \State $\cD\leftarrow \cD\cup \{\tau^t\}$. 
    \EndFor
    % \Ensure $\hat\pi=(\hat\pi )_{h\in[H]}$.
    \end{algorithmic}
    \caption{Online MLE-OVI for Myopic Follower under Linear Markov Game}
    \label{alg:MLE-OVI}
\end{algorithm}

We also provide theoretical guarantee for the MLE-OVI algorithm.
\begin{theorem}[{Regret for MLE-OVI}]\label{thm:Online-ML}
    We choose 
    $$\Gamma^{(1, t)}_h(\cdot,\cdot,\cdot) = C_1 d H \allowbreak \sqrt{\log(2d H T^2/\delta)}\cdot \sqrt{\phi_h(\cdot,\cdot,\cdot)^\top (\Lambda_h^t)^{-1}\phi_h(\cdot,\cdot,\cdot)}$$ for some universal constant $C_1>0$ and $\beta = C_2 d\log(HT(1+\eta T^2)\delta^{-1})$ for some universal constant $C_2>0$. 
    For Scheme 4, 
\begin{align*}
    \Reg(T)\lesssim d H^2 \sqrt{d T} + \eta C_\eta H^2 d \sqrt{T} + \exp(4\eta B_A) (\eta C_\eta)^3 H^2 d^2 \log T, 
\end{align*}
and for Scheme 5, 
\begin{align*}
    \Reg(T)\lesssim d H^2 \sqrt{d T} + \exp(4\eta B_A)\eta C_\eta H^2 d \sqrt{T} + \exp(8\eta B_A) (\eta C_\eta)^3 H^2 d^2 \log T. 
\end{align*} 
    \begin{proof}
        See \Cref{sec:proof-Online-ML} for a detailed proof.
    \end{proof}
\end{theorem}

We observe from \Cref{thm:Online-ML} that the the optimistic value iteration methods proposed by both Scheme 4 and Scheme 5 achieve sublinear online regret. 
Scheme 5 suffers from an additional exponential term in the second term, which corresponds to the first order quantal response error. 
What happens here resembles the offline suboptimality bound for Scheme 3 in \Cref{rmk:MLE-PVI-dist-shift} and the reason is quite similar given that we directly use the MLE estimator $\hat\theta^t_{h, \MLE}$ for Scheme 5 instead of exploiting the confidence set for $\theta$.




\ifmain
\subsection{Online Learning with General Function Class}\label{sec:online-myopic-general}
\fi 
\ifneurips
\subsubsection{Online Learning with General Function Class}\label{sec:online-myopic-general}
\fi 
We develop an online learning algorithm very similar to the one given in \Cref{sec:Offline-MG} for the offline general function class, despite that we use optimism in leader's policy learning.
However, for the leader's Bellman loss which is defined by \eqref{eq:myopic-offline-general-Bellman loss} in the offline case, we use a slightly different version that directly incorporates the rule of optimism, 
\begin{align}
    &\!\!\! \ell_h^t(U_h', U_{h+1}, \theta_{h+1}) \nend
    &\!\!\!\quad = \sum_{i=1}^{t-1}  \rbr{U_h'(s_h^i, a_h^i, b_h^i) \!-\! u_h^i - \!\!\!\! \max_{\pi_{h+1}(s_{h+1})\in\sA}\inp[\big]{U_{h+1}(s_{h+1}^i, \cdot, \cdot)}{\pi_{h+1}\otimes \nu_{h+1}^{\pi, \theta}(\cdot, \cdot\given s_{h+1}^i)}_{\cA\times\cB}}^2.\label{eq:online-MG-bellman loss}
\end{align}
Here, there is a slight abuse of notation and we distinguish this loss from its correspondence in the offline case by the superscript $t$. 
We would like to remark that this loss function assembles the one used in GOLF of \citet{jin2021bellman}.
We build the confidence set for both the environment model and the behavior model as
\begin{align}
    &\CI_{\cU, \Theta}^t(\beta) \nend
    &\quad= \cbr{
    (U,\theta)\in\cU^{\otimes H}\times\Theta:
    \rbr{ \ds
        \cL_h^t(\theta_h)-\inf_{\theta_h'\in\Theta_h}\cL_h^t(\theta_h') \le \beta 
    \atop \ds
        \ell_h^t(U_h, U_{h+1}, \theta_{h+1}) - \inf_{U'\in\cU_h} \ell_h^t(U', U_{h+1}, \theta_{h+1})\le H^2\beta}, 
    \forall h\in[H]}. \label{eq:myopic-online-general-confset}
\end{align}
Following the principle of optimism, we can output a pair of optimistic model parameter, 
\begin{align}
    (\hat U^t, \hat\theta^t)=\argmax_{\pi_1\in\Pi_1 \atop (U, \theta)\in\CI_{\cU,\Theta}^t(\beta)} \EE_{s_1\sim\rho_0} \sbr{\inp[\big]{U_1(s_1, \cdot, \cdot)}{\pi_1\otimes\nu_1^{\pi, \theta}(\cdot,\cdot\given s_1)}_{\cA\times\cB}}.\label{eq:online-MG-opt-parameter}
\end{align}
Specifically, for any given state $s_h$, an optimistic policy is then given greedily by 
\begin{align}
    \hat\pi^t(s_h)=\argmax_{\pi_{h}(s_h)\in\sA}\inp[\big]{U_{h}(s_{h}, \cdot, \cdot)}{\pi_{h}\otimes \nu_{h}^{\pi, \theta}(\cdot, \cdot\given s_{h})}_{\cA\times\cB}. \label{eq:online-MG-hat pi}
\end{align}
We summarize the above \textbf{M}aximum \textbf{L}ikelihood \textbf{E}stimation with \textbf{G}lobal \textbf{O}ptimism based on \textbf{L}ocal \textbf{F}itting (MLE-GOLF) algorithm as the following.
\begin{algorithm}[H]
    \begin{algorithmic}[1]
    \Require {$\eta$}
    \State Initiate $\cD =\emptyset$.
    \For{$t=1,\dots,T$}
    \State Construct confidence set $\CI_{\cU, \Theta}^t(\beta)$ by \eqref{eq:myopic-online-general-confset}. 
    \State Solve for $\hat U^t, \hat\theta^t$ by \eqref{eq:online-MG-opt-parameter}.
    \State Solve for the greedy policy $\hat\pi^t$ by \eqref{eq:online-MG-hat pi}. 
    \State Deploy $\hat\pi^t$ and collect a trajectory $\tau^t = \{(s_h^t, a_h^t, b_h^t, \hat\pi_h^t)\}_{h\in[H]}$. 
    \State $\cD\leftarrow \cD\cup \{\tau^t\}$. 
    \EndFor
    \end{algorithmic}
    \caption{Online MLE-GOLF for Myopic Follower under General Function Approximation}
    \label{alg:MLE-GOLF}
\end{algorithm}
% \todo{
% \begin{itemize}
%     \item function class $\cG_L = \cbr{U_h-\TT_h^{\theta} U_{h+1}: U\in\cU, \theta\in\Theta}$ defined on $\cX=\cS\times\cA\times\cB$, which is bounded by $B_{\cG_L}\le 2B_U$; the class of probability measures $\sP_L = \cbr{\rho\in\Delta(\cS\times\cA\times\cB): \rho(\cdot)=\PP^\pi((s_h, a_h, b_h)=\cdot)}$ with $\rho^i(\cdot) = \PP^{\hat\pi^i}((s_h, a_h, b_h)=\cdot)$; Under these definitions, we let $d_{\cG_L} = \dim_\DE\rbr{\cG_L, \sP_L, T^{-1/2}}$ be the distributional Eluder dimension for $\cG_L$.
%     \item function class defined on $\cX=\cS\times\cB\times\sA$,
%     \begin{align*}
%     \cG_F = \cbr{g:\cX\rightarrow \RR: \exists \theta\in\Theta, g(s_h, b_h, \pi)= \rbr{\varPsi_h^{\pi}\rbr{r_h^\theta - r_h}}(s_h, b_h)};
%     \end{align*}
%     which is bounded by $4B_r$ where $B_r$ boundes the reward function induced by $\theta\in\Theta$. Linear operator $\varPsi_h^\pi:\cF(\cS\times\cA\times\cB)\rightarrow \cF(\cS\times\cB)$ defined as 
% \begin{align*}
%     \rbr{\varPsi_h^\pi f}(s_h, b_h) = \dotp{\pi_h(\cdot\given s_h, b_h)}{f(s_h, \cdot, b_h)} - \dotp{\pi_h\otimes \nu_h^{\pi}(\cdot,\cdot\given s_h)}{f(s_h,\cdot,\cdot)}. 
% \end{align*}
% We let $d_{\cG_F}=\dim_\E(\cG_F, T^{-1/2})$ be the Eluder dimension of $\cG_F$.
% \end{itemize}
% }

We identify two function classes whose complexities determine the online learning hardness. The first one is the Bellman residuals $\cG_L:\cS\times\cA\times\cB\rightarrow \RR$ defined as 
$$\cG_L=\{U_h-\TT_h^{*,\theta} U_{h+1}, U\in\cU, \theta\in\Theta, h\in[H]\}, $$
where $\TT_h^{*,\theta}:\cF(\cS\times\cA\times\cB)\rightarrow \cF(\cS\times\cA\times\cB)$ is the Bellman optimality operator defined in \eqref{eq:define optimistic Bellman opt}.
The second one is the QRE defined in \eqref{eq:QRE} where we define the class of QREs $\cG_F:\cS\times\cB\rightarrow \RR$  as
$$
\cG_F =  \{\Upsilon_h^{\pi}(r_h^{\theta} - r_h), \pi\in\Pi, \theta\in\Theta, h\in[H]\}, 
$$
where recall the operator $\Upsilon_h^\pi:\cF(\cS\times\cA\times\cB)\rightarrow \cF(\cS\times\cB)$ defined as 
\begin{align*}
    \rbr{\Upsilon_h^\pi f}(s_h, b_h) = \dotp{\pi_h(\cdot\given s_h, b_h)}{f(s_h, \cdot, b_h)} - \dotp{\pi_h\otimes \nu_h^{\pi}(\cdot,\cdot\given s_h)}{f(s_h,\cdot,\cdot)}.
\end{align*}
In the sequel, we let $\dim(\cG_L)=\dim_\E(\cG_L, T^{-1/2})$ and $\dim(\cG_F)=\dim_\E(\cG_F, T^{-1/2})$ be the eluder dimensions for these two function classes. \footnote{See \Cref{sec:eluder dimension} for definition of the eluder dimension.} 

\begin{theorem}[{Regret for MLE-GOLF}]\label{thm:Online-MG}
    Suppose that the following conditions hold for model class $\Theta$ and function class $\cU$:
\begin{itemize}
    \item[(i)] (\textit{Realizability}) There exists $\theta^*\in\Theta$ such that $r_h^{\theta^*}=r_h$ for any $h\in[H]$. For $\theta^*\in\Theta$, there exists $U\in\cU$ such that $U_h = U_h^{*}$ for any $h\in[H]$;
    \item[(ii)] (\textit{Completeness}) For any $U\in\cU$ and $\theta\in\Theta$, there exists $U'\in\cU$ such that $U'=\TT_h^{*,\theta} U$ for any $h\in[H]$. 
\end{itemize}
    By choosing $\beta\ge c \max\cbr{ \log(HT\cN_\infty(\cZ, T^{-1})\delta^{-1}), \log(HT \cN_\rho(\Theta, T^{-1})/\delta)}$ for some universal constant $c>0$, where $\cN_\rho(\cZ,\epsilon)$ and $\cN_\rho(\Theta, \epsilon)$ are the maximal (over $h\in[H]$) $\epsilon$-covering number defined in \eqref{eq:cN-cZ}, \eqref{eq:cN-Theta-myopic} for the joint function class $\cZ_h = \Theta_{h+1}\times\cU^2$ and  $\Theta_h$, respectively. 
    We have for the online algorithm that 
    \begin{align*}
        \Reg(T)\lesssim H^2 \sqrt{\dim(\cG_L)\beta T} + H^2 \eta C_\eta \sqrt{\dim(\cG_F)\beta T} + H^2(\eta C_\eta)^3\exp(4\eta B_A) \beta \log T, 
    \end{align*}
    where $\lesssim$ only hides universal constants.
    \begin{proof}
        See \Cref{sec:proof-Online-MG} for a detailed proof.
    \end{proof}
\end{theorem}

\Cref{thm:Online-MG} characterizes the online learning complexity in terms of the eluder dimensions of function classes $\cG_L$ and $\cG_F$.
In particular, we do not suffer from any $\exp(2\eta B_A)$ coefficient in the $\sqrt{T}$ term thanks to optimism, though the $\log(T)$ term still has $C^{(3)}=\cO(\eta B_A\exp(2\eta B_A))$. 
We remark that when applied to the linear function approximation, we can reproduce the result in \Cref{thm:Online-ML} with $\dim(\cG_L) \lesssim d$ and $\dim(\cG_F) \lesssim d$. 
\ifmain
\section{Extension to Learning with Farsighted Follower} \label{sec:farsighted}
\fi
\ifneurips
\subsection{Extension to Learning with Farsighted Follower} \label{sec:farsighted}
\fi
In this section, we explore the possibility of learning the behavior model for a farsighted follower in both the offline and online settings.
We extend our previous techniques to  online learning the QSE with nonmyopic follower. 
To ease our presentation, we use $M=(r^M, u^M, P^M)\in\cM$ to denote the model and all definitions previously introduced with $\theta$ can be naturally extended to this larger model class $M$. In particular, we let $M^*$ denote the true model. We suppose that the follower's reward in our model (which contains the true model) satisfies a linear constraint $\la x(\cdot), r_h^M(s_h, a_h, \cdot)\ra_{\cB} = \varsigma$ for all $(s_h, a_h)\in\cS\times\cA$ and $h\in[H]$, where $x:\cB\rightarrow\RR$ is a known function and $\varsigma\in\RR$ is a known constant. In particular, such a constraint rules out a free dimension in the follower's reward and is introduced to ensure that $r$ can be uniquely identified in the quantal response model. 
For instance, the linear constraint can be that the reward averaged over $\cB$ is $0.5$. We remark that such an assumption is not without loss of generality for a farsighted follower.


% \paragraph{Data Collection.}
% The offline data collection process is similar to the one described in \Cref{sec:offline-myopic}, except that the leader announces her $H$-step policy $\pi^t$ at the beginning of episode $t$ and the data collect the announced joint policy $\pi^t$ rather than just the prescription at each visited state.
% We assume the policy to be indepedent across episode for the offline setting.
% For the online setting, the game is played in a similar way except that the leader's future policy can depend on the past data. 

% \paragraph{Learning the Behavior Model.}
\paragraph{Offline Algorithm.}
Although we have gained success in learning the behavior model for the myopic case with both general function approximation and linear function approximation, previous results does not generalize to nonmyopic follower easily. A unique challenge that we face in the farsighted follower case is that the follower's choice model has a long term dependency, which means that the uncertainty in the transition kernel also enters the estimation error of the follower's behavior model. Another challenge is that the follower's choice is a joint effect of the follower's future expected total utilities, and it takes additional efforts to decompose it stepwise so as to utilize the knowledge for planning.

We propose a model based method for jointly learning the environment model and the behavior model. Recall from \Cref{sec:markov_game_def} the model is given by $M=\cbr{P_h^M, r_h^M, u_h^M}_{h\in[H]}$. We consider the following negative generalized likelihood 
\begin{align*}
    \cL_h(M) = - \sum_{i=1}^T \rbr{\eta A_h^{\pi^i, M}(s_h^i, b_h^i) - \log P_h^M(s_{h+1}^i\given s_h^i, a_h^i, b_h^i) - (u_h^i - u_h^M(s_h^i, a_h^i, b_h^i))^2}, 
\end{align*}
where the first term $\eta A_h^{\pi^i, M}$ comes from the follower's quantal response under model $M$ and policy $\pi^i$, and the second term is the likelihood of the transition kernel. 
Note that computing this $A_h^{\pi^i, M}$ actually requires computing the best response for step $h+1,\dots,H$ and update the follower's Q- and V-functions according to \eqref{eq:qv_pi_qr}. 
Using this generalized likelihood, one can directly obtain a confidence set for model $M$ and do planning using pessimism, 
\begin{align*}
    \hat\pi &= \argmax_{\pi\in\Pi} \min_{M\in\CI_M(\beta)} J(\pi, M), \quad \text{where} \\
    \confset_\cM(\beta) &= \cbr{M\in\cM\given \cL_h (M)\le \inf_{M'\in\cM} \cL_h(M') + \beta, \quad \forall h\in[H]}, 
\end{align*}
and
$
    J(\pi, M)
    %  \defeq \EE^{\pi, \nu^{\pi,M}}\osbr{\sum_{h=1}^H u_{h}^M(s_h, a_h, b_h)}.
$ is just the total reward for the leader evaluated under an estimated model $M$.
We have the following \textbf{P}essimistic \textbf{M}aximum \textbf{L}ikelihood \textbf{E}stimation algorithm for the offline setting. 
\begin{algorithm}[H]
    \begin{algorithmic}[1]
    \Require {$\eta, \cD$}
    \State Build confidence set $\confset_\cM(\beta) = \cbr{M\in\cM: \cL_h(M)-\inf_{M'\in\cM}\cL_h(M')\le \beta, \forall h\in[H]}$.
    \State Find the pessimistic policy $\hat\pi = \argmax_{\pi\in\Pi}\min_{ M\in\confset_\cM(\beta)} J(\pi, M)$. 
    \Ensure {$\hat\pi$}
    \end{algorithmic}
    \caption{Pessimistic MLE (PMLE) for Offline Learning with Farsighted Follower}
    \label{alg:real-PMLE}
\end{algorithm}
\begin{theorem}[Suboptimality for PMLE]\label{thm:PMLE}
We let $\beta\ge  \allowbreak 9\log(3e^2H \cN_\rho(\cM,T^{-1})\delta^{-1})$, where $\cN_\rho(\cM,\epsilon)$ is the minimal size of an $\epsilon$-optimistic covering net of $\cM$, which is defined in \Cref{lem:MLE}. 
Suppose $\la x, r_h^M(s_h, a_h, \cdot)\ra_\cB = \varsigma$ for all $M\in\cM$, $(s_h, a_h)\in\cS\times\cA$ and $h\in[H]$, and we let $\kappa = \nbr{x}_\infty/|\la x, \ind\ra_\cB|$.
Then with probability at least $1-\delta$, we have for the PMLE algorithm that 
\begin{align*}
    \subopt(\hat\pi) \lesssim C_1^{\pi^*} H^2 \sqrt{ \beta T^{-1}} + C_2^{\pi^*} \eta C_\eta H^{5/2}\sqrt{ \eff_H(\gamma)\beta T^{-1/2}} + C_3^{\pi^*}\cdot C^{(2)} L^{(2)} H \beta T^{-1}, 
\end{align*}
where 
\[C_1^{\pi^*} = \max_{M\in\cM, h\in[H]} \sqrt{\frac{\EE\Bigsbr{\Bigrbr{\bigrbr{U_h^{\pi^*, M}-\bigrbr{u_h+P_h W_{h+1}^{\pi^*, M}}}(s_h, a_h, b_h)}^2 }}{T^{-1}\sum_{i=1}^T \EE^i\Bigsbr{\Bigrbr{\bigrbr{U_h^{\pi^*, M}-\bigrbr{u_h+P_h W_{h+1}^{\pi^*, M}}}(s_h, a_h, b_h)}^2 }}}, \]
\[C_2^{\pi^*} =\max_{M\in\cM, h\in[H]}\sqrt\frac{\EE \Bigsbr{\rbr{\rbr{\EE_{s_h, b_h} -\EE_{s_h}}\sbr{\sum_{l=h}^H \gamma^{l-h}\rbr{r_l^M - r_l + \gamma \bigrbr{P_l^M - P_l} V_{l+1}^{\pi^*, M}}(s_l,a_l,  b_l)}}^2}}{T^{-1}\sum_{i=1}^T \EE^i \Bigsbr{\rbr{\rbr{\EE_{s_h, b_h}^i -\EE_{s_h}^i}\sbr{\sum_{l=h}^H \gamma^{l-h}\rbr{ r_l^M - r_l + \gamma \bigrbr{P_l^M - P_l} V_{l+1}^{\pi^*, M}}(s_l,a_l,  b_l)}}^2}}, \]
\[C_3^{\pi^*} = \max_{h\in[H], M\in\cM}{\frac{\EE\sbr{ \rbr{\EE_{s_h, b_h}\sbr{\bigrbr{r_h^{M} - r_h + \gamma \bigrbr{P_h^M - P_h}  V_{h+1}^{\pi^*, M}}(s_h, a_h, b_h)}}^2}}{T^{-1}\sum_{i=1}^T \EE^i\sbr{ \rbr {\EE_{s_h, b_h}^i\sbr{{\bigrbr{r_h^{M} - r_h + \gamma \bigrbr{P_h^M - P_h}  V_{h+1}^{\pi^*, M}}(s_h,a_h, b_h)}}}^2}}},\]
where the expectation $\EE$ without superscript is taken with respect to $(\pi^*, \nu^{\pi^*})$, and we have constants 
\begin{align*}
    L^{(2)} &= c H^2 \eff_H(c_2)^2 \kappa^2 \exp\rbr{8\eta B_A} (\eta^{-1}+B_A)^2, \\
    C^{(2)}  & =  2  \eta^2 H^2\cdot \exp\orbr{6\eta B_A} \cdot(1+4 \eff_H(\gamma)) \cdot \rbr{\eff_H(\exp(2\eta B_A)\gamma)}^2.
\end{align*}
\end{theorem}
\begin{proof}
    See \Cref{sec:proof-PMLE} for a detailed proof.
\end{proof}

Compared to \Cref{thm:Offline-MG}, the suboptimality for a farsighted follower becomes more complicated in the sense that concentrability coefficient $C_2^{\pi^*}$ for the follower's quantal response error also depend on estimation errors for both the follower's reward $r_h$ and the transition $P_h$ along the optimal trajectory.
In addition, the second order quantal response error, i.e., the last term in the suboptimality, suffers from a $\exp(\cO(\eta H))$ coefficient, which occurs as a result of the quantal response model and the coefficient also grows exponentially with respect to the inverse temperature $\eta$.
Such a result suggests for small data size $T$, the second order QRE terms dominates. 

% We first address the problem of learning the long term utilities of the follower. Note that the follower's Q function satisfies the Bellman update $Q_h^\pi = \BB_h^\pi Q_{h+1}^\pi$. In line with the Bellman loss $\ell$ constructed in \eqref{eq:myopic-offline-general-Bellman loss} for learning the Bellman update for the leader, we construct Bellman loss $\ell^F_{h,\cD}$ as 
% \begin{align*}
%     \ell^F_{h}(Q_h', Q_{h+1}, \theta, \pi) = \sum_{i=1}^T \rbr{\rbr{Q_h'-r_h^{\pi, \theta}}(s_h^i, b_h^i) - \inp[\Big]{\nu_{h+1}^{Q_{h+1}}(\cdot\given s_{h+1}^i)}{Q_{h+1}(s_{h+1}^i, \cdot)}}, 
% \end{align*}
% where we note that $\nu$ is a function of $Q$. Using this Bellman loss, we are able to learn the transition and obtain an estimation of the Q function for each $\theta$ and policy $\pi^i$. Specifically, we consider the following confidence set
% \begin{equation}\label{eq:online-FG-Q}
%     \CI_Q^\theta(\beta) =\cbr{
%         \begin{aligned}
%         &\cbr{Q^i}_{i\in[T]}\in\cQ^{\times(H\times T)}:\\
%         &\qquad \ell_h^F(Q_{h}^i, Q_{h+1}^i, \theta, \pi^i) - \inf_{Q'\in\cQ}  \ell_h^F(Q', Q_{h+1}^i, \theta, \pi^i) \le \beta, \forall (h,i)\in[H]\times[T]
%         \end{aligned}
%     }.
% \end{equation}
% \Cref{eq:online-FG-Q} characterizes a confidence set for the Q function at each episode for each $\theta$. We can expect the confidence set to be valid and accurate since the estimation of each $Q^i$ incorporates the transition information in all trajectories and we just need to ensure a $\beta$ by taking a union bound over the $T$ episodes.
% Since we can estimate the Q function for each episode in the offline data, we are then able to incorporate MLE with the negative log-likelihood using the estimated Q function,
% \begin{align}\label{eq:online-FG-MLE}
%     \cL_h\rbr{Q^{(H,T)}} = - \sum_{i=1}^T \rbr{\eta Q_h^i(s_h^i, b_h^i) - \log \rbr{\int_{b'\in\cB} \exp\rbr{\eta Q_h^i(s_h^i, b')}\rd b'}}, 
% \end{align}
% where we abbreviate $\cbr{Q_h^i}_{i\in[T]}$ to $Q^{(H,T)}$. Using the likelihood and the estimated Q function, 
% we then obtain the following confidence set for $\theta$,
% \begin{align*}
%     \CI_\Theta(\beta) = \cbr{\theta\in\Theta: \cL_h\rbr{\cbr{Q_h^i}_{i\in[T]}} - \inf_{}\cL_h\rbr{\cbr{Q_h^i}_{i\in[T]}}}
% \end{align*}
\paragraph{Online Algorithm.}
Similar to the offline setting, 
the negative generalized-likelihood for the model $M=\cbr{r_h^M, P_h^M, u_h^M}_{h\in[H]}$ at step $h$ and time $t$ is given by
\begin{align}
    \cL_h^t(M)& = -\sum_{i=1}^{t-1} \bigg(\eta A_{h}^{\pi^i, M}(s_h^i, b_h^i) + \log P_h^M(s_{h+1}^i\given s_h^i, a_h^i, b_h^i)   - \rbr{u_{h}^i - u_{h}^M(s_h^i, a_h^i, b_h^i)}^2\bigg),  \label{eq:MLE}
\end{align}
% where $\tau^{t-1}=\cbr{(s_h^i, a_h^i, b_h^i, \pi_h^i)_{h\in[H]}}_{i\in[t-1]}$ is the history up to step $t-1$.
The OMLE algorithm plays the greedy policy with respect to the leader's most favorable model in the $\beta$-superlevel set of the above log-likelihood $\cL^t(\cdot)$, 
\begin{align}
    \pi^t=\argmax_{\pi\in\Pi, M\in\confset_\cM^t(\beta)} J(\pi, M), \quad \st \quad \confset_\cM^t(\beta) = \cbr{M\in\cM\given \cL_h^t (M)\le \inf_{M'\in\cM} \cL_h^t(M') + \beta, \forall h\in[H]}, \label{eq:OMLE}
\end{align}
where
$
    J(\pi, M) 
    % \defeq \EE^{\pi, \nu^{\pi,M}}\osbr{\sum_{h=1}^H u_{h}^M(s_h, a_h, b_h)}.
$
is just the total reward for the leader evaluated under an estimated model $M$.
We have the following \textbf{O}ptimistic \textbf{M}aximum \textbf{L}ikelihood \textbf{E}stimation algorithm for the online setting. 
\begin{algorithm}[H]
    \begin{algorithmic}[1]
    \Require {$\eta, T$}
    \State Initiate $\cD=\emptyset$.
    \For{$t=1,\dots,T$}
    \State Build confidence set $\confset^t_\cM(\beta) = \cbr{M\in\cM: \cL_h^t(M)-\inf_{M'\in\cM}\cL_h^t(M')\le \beta, \forall h\in[H]}$.
    \State Find the optimistic policy $\pi^t = \argmax_{\pi\in\Pi, M\in\confset_\cM^t(\beta)} J(\pi, M)$. 
    \State Announce $\pi^t$ and observe a trajectory $\tau^t = \ocbr{\orbr{s_h^t, a_h^t, b_h^t, \pi_h^t}}_{h\in[H]}$.
    \State Update $\cD \leftarrow \cD \cup \{\tau^t\}$.
    \EndFor
    \end{algorithmic}
    \caption{Optimistic MLE (OMLE) for Online Farsighted Follower}
    \label{alg:OMLE}
\end{algorithm}
% We construct $f_0^M, f_{1, h}^M$ as
% \begin{align*}
%     f_{0,h}^M(\pi) &\defeq \EE^{\pi, M^*}\sbr{u_h^M  - u_h^{M^*} + \rbr{\PP_h^M - \PP_h^{M^*}} W_h^{\pi_\opt^M, M} },\nend
%     f_{1,h}^M(\pi) & \defeq \EE^{\pi, M^*}\abr{\sum_{i=h}^H\gamma^{i-h}\rbr{\EE_{s_h, b_h}^{\pi, M^*} - \EE_{s_h}^{\pi, M^*}}\sbr{ r_i^M-r_i^{M^*}+\gamma \rbr{\PP_i^M - \PP_i^{M^*}}V_i^{\pi_\opt^M, M}}} .
% \end{align*}
% % \todo{There is an error that $f_{1, h}$'s Eluder dimension is not $d$ when the $\abr{\cdot}$ exists. We should deal with the squared Hellinger distance directly in \Cref{lem:1st-ub}.}
% We also consider the following function classes $\cF_{0,h}=\{f_{0,h}^M(\cdot):M\in\cM\}$ and $\cF_{1,h}=\{f_{1,h}^M(\cdot):M\in\cM \}$.
% Let $d_0 = \max_h\dimE(\cF_{0,h}, \Pi, 1/\sqrt T)$, $d_1=\max_h\dimE(\cF_{1,h}, \Pi, 1/\sqrt T)$ be the (maximal) Eluder dimensions for these function classes.
% Similar to \citet{chen2022unified}, we define

In the sequel, we define three types of errors the corresponding (distributional) eluder dimensions for characterizing the online learning complexities.
The first is the leader's Bellman residuals and the class is defined as 
$\cG_L=\{\EE^{\pi}[(U_h^{*, M} - u_h -  W_{h+1}^{*, M})(s_h,a_h,b_h)], M\in\cM, h\in[H]\}$, where we define $U^{*, M}, W^{*, M}$ as the leader's optimistic U- and W-functions under $\pi^{M} = \argmax_{\pi\in\Pi}J(\pi, M)$ and $M$.
% and the Bellman optimality operator $\TT_h^{*, M}$ is defined the same way as \eqref{eq:def-TT-opt-neurips} where we only replace $T_{h+1}^{*, \theta}$ by $T_{h+1}^{*, M}$, which is just a change of notations.
The second function class is an analogy to the $\QRE$ in \eqref{eq:QRE} for farsighted follower defined as
% \ifneurips{\abovebelowskip{.5}{.5}
% \begin{align}\label{eq:GF1-neurips}
%     {\ts \cG_F^1 = \ocbr{(\EE_{s_h, b_h}^{\pi} - \EE_{s_h}^{\pi})\osbr{\sum_{l=h}^H \gamma^{l-h}\orbr{r_l^M- r_l + \gamma (P_l^{M} - P_h) V_{l+1}^{*, M}}(s_l, a_l, b_l)}}}, 
% \end{align}
% }\hspace{-5pt}\fi
\begin{align}\label{eq:GF1-neurips}
     \cG_F^1 = \cbr{(\EE_{s_h, b_h}^{\pi} - \EE_{s_h}^{\pi})\sbr{\sum_{l=h}^H \gamma^{l-h}\bigrbr{r_l^M- r_l + \gamma (P_l^{M} - P_h) V_{l+1}^{*, M}}(s_l, a_l, b_l)}}, 
\end{align}
for all $(\pi, M, h)\in(\Pi, \cM, [H])$ and $(s_h, b_h)\in\cS\times\cB$. Here, we let $\EE_{s_h, b_h}^\pi[\cdot] = \EE^{\pi, \nu^\pi}[\cdot\given s_h, b_h]$ and $V^{*, M}$ is the follower's optimistic V-function under $\pi^M$ and $M$. In particular, when we take $\gamma=0$ for myopic follower, \eqref{eq:GF1-neurips} reduces to the $\QRE$ defined in \eqref{eq:QRE}.
The last function class is unique for the farsighted case, which captures the second order term in the QRE as the follower's Bellman error, 
% \ifneurips{\abovebelowskip{.5}{.5}
% \begin{align*}
%     \cG_F^2=\ocbr{\EE_{s_h, b_h}^{\pi} \osbr{\orbr{r_h^M- r_h + \gamma (P_h^{M} - P_h) V_{l+1}^{*, M}}(s_h, a_h, b_h)}},
% \end{align*}
% }\hspace{-5pt}\fi
\ifmain
\begin{align*}
    \cG_F^2=\cbr{\EE_{s_h, b_h}^{\pi} \sbr{\bigrbr{r_h^M- r_h + \gamma (P_h^{M} - P_h) V_{l+1}^{*, M}}(s_h, a_h, b_h)}},
\end{align*}
\fi
for all $(\pi, M, h)\in(\Pi, \cM, [H])$ and $(s_h, b_h)\in\cS\times\cB$.
% Here, $V^{*, M}$ is the V-function defined in \eqref{eq:qv_pi_qr} but under $\pi^{*, M}$ and $M$. 
We denote by $\dim(\cG_L)$, $\dim(\cG_F^1)$, and $\dim(\cG_F^2)$ their eluder dimensions\footnote{See \Cref{sec:eluder farsighted} for definitions.} with properly selected parameters. 

% \begin{align*}
%     D_{\RL,h}^2 (M,  M^*;\pi) =   \EE^{\pi, M^*}D_\H^2\rbr{\nu_h^{\pi,M}, \nu_h^{\pi,M^*}} +
%     \EE^{\pi, M^*} D_\H^2(P_h^{M}, P_h^{M^*}) +
%     \EE^{\pi, M^*}\rbr{u_h^{M^*}-u_h^M}^2,
% \end{align*}
% and $P^{\pi, M}$ is the distribution of $(s_h,a_h,b_h)_{h\in[H]}$ under policy $\pi$ and model $M$.
\begin{theorem}[{Regret for OMLE}]\label{thm:OMLE-farsighted}
    We let $\beta\ge  \allowbreak 9\log(3e^2H T\cN_\rho(\cM,T^{-1})\delta^{-1})$, where $\cN_\rho(\cM,\epsilon)$ is the minimal size of an $\epsilon$-optimistic covering net of $\cM$, which is defined in \Cref{lem:MLE}. 
    Suppose $\la x, r_h^M(s_h, a_h, \cdot)\ra_\cB = \varsigma$ for all $M\in\cM$, $(s_h, a_h)\in\cS\times\cA$ and $h\in[H]$, and we let $\kappa = \nbr{x}_\infty/|\la x, \ind\ra_\cB|$.
    Then with probability at least $1-\delta$, we have for the OMLE algorithm that 
    \begin{align*}
        \Reg(T) \le H^2\sqrt{\dim(\cG_L) \beta T} + \eta C_\eta  H^2\eff_H(\gamma) \sqrt{\dim(\cG_F^1) \beta T} + H^2 C^{(2)} L^{(2)} \dim(\cG_F^2)\beta \log(T),
    \end{align*}
    where 
    $L^{(2)} = c H^2 \eff_H(c_2)^2 \kappa^2 \exp\rbr{8\eta B_A}  C_\eta^2$ for some universal constant $c>0$ and for $c_2 = \gamma(2\exp(2\eta B_A)+\kappa\exp(4\eta B_A))$, and $C^{(2)} = 2  \eta^2 H^2  \exp\orbr{6\eta B_A}  (1+4 \eff_H(\gamma)) \cdot \rbr{\eff_H(\exp(2\eta B_A)\gamma)}^2$.
    % \begin{align*}
    %     \EE^{\pi, M^*}\bigrbr{A_h^{\pi, M}-A_h^{\pi,M^*}}^2 \le  \abr{f_{2,h}^M(\pi)} \le L^{(2)} \max_{h\in[H]} D_{\RL,h}^2(M, M^*;\pi), \quad \forall h\in[H].
    % \end{align*}
    % holds for any $\pi\in\Pi$ and $M\in\cM$. The OMLE algorithm for the MDP with farsighted follower
    % achieves the following with probability at least $1-\delta$, 
    % \begin{align*}
    %     \Reg(T)&\le \cO\rbr{H^{2} \sqrt{d_0\beta T} + (1+\eta B_A) C_{\gamma, H} H^2\sqrt{d_1\beta T}} + \cO\rbr{\eta^2 H^2 C_{\gamma, H} \exp\rbr{2\eta B_A} L^{(2)} d_2 \beta \log T}
    % \end{align*}
    % where $
    %     C_{\gamma, H} =\rbr{1-\gamma^H}/\rbr{1-\gamma}
    % $ is the effective foresight for the follower.
    \begin{proof}
        See \Cref{sec:proof-farsighted MDP} for a detailed proof.
    \end{proof}
\end{theorem}

We remark that our theorem handles a wide range of MDP classes, e.g., linear Markov game in \Cref{def:linear MDP}, linear matrix MDP \citep{zhou2021provably}, or linear mixture MDP \citep{chen2022unified},
% , and low rank MDP \citep{agarwal2020flambe}, 
where we have
$\dim(\cG_L) = \dim(\cG_F^2) \lesssim d$ and $\dim(\cG_F^1)\lesssim Hd$. 
See \Cref{sec:eluder farsighted} for more details.
We also note that the first order terms (with $\sqrt T$ regret) only depends polynomially on $\eta, H$ and the exponential effect lies in the $\log(T)$ term, which indicates that we can handle a follower with more rationality. Additionally, in the best case we have $\kappa = |\cB|^{-1}$ by taking $x=\ind$ and how to relax this $\la x(\cdot), r_h^M(s_h, a_h, \cdot)\ra_{\cB} = \varsigma$ constraint leaves for future work.

% \todo{another error concerning the use of $D_\RL$, the definition should be corrected and check the dependence on $H$.}
% \todo{to be discussed.}
% Note that $\cF_{2,h}$ only enters the theoretical results without need of specification in the OMLE algorithm. 
% In the following, we apply the learning result in \Cref{thm:OMLE-farsighted} to the linear MDP case and specify the choice of $\cF_{2,h}$ when the follower's reward satisifes some linear constraint.
% \begin{corollary}[\textit{Online Learning for Linear MDP}]\label{cor:online linear}
% Consider a linear MDP under \Cref{def:linear MDP}. Suppose that the follower's reward at each state satisfies a linear constraint $\dotp{x}{r_h(s_h, a_h, \cdot)} = \varsigma$ for some $x:\cB\rightarrow \RR$ such that $\inp{\ind}{x}\neq 0$ and $\varsigma\in\RR$. Define ratio $\kappa = \nbr{x}_\infty/|\inp{x}{\ind}|$. Then, \Cref{alg:OMLE} achieves the following with probability at least $1-\delta$,
% \begin{proof}
%     See \Cref{sec:proof-OMLE-linear} for a detailed proof.
% \end{proof}
% \end{corollary}
\section{Conclusion and Future Work}
In this work, I design corruption-robust algorithms for the Lipschitz contextual search problem. I present the \emph{agnostic checking} technique and demonstrate its effectiveness in designing corruption-robust algorithms. There are several open problems for future research. First, in the algorithm I propose for pricing loss, the schedule for agnostic checks is fixed upfront. Can the learner design an adaptive checking schedule for the pricing loss? Second, this work assumes the learner has knowledge of the Lipschitz constant $L$. Can the learner design efficient no-regret algorithms without knowledge of $L$? 
%%%%%%%%%%%%-- reference --%%%%%%%%%%%
\newpage
\bibliographystyle{ims}
\bibliography{reference}

%%%%%%%%%%%% -- appendix -- %%%%%%%%%%%
\newpage 
\appendix

\begin{comment}
\section{System Architecture}
\label{appendix:architecture}
\system has a novel modularized system architecture with three key components: 
\emph{StreamManager}, 
\emph{TxnManager} and \emph{TxnScheduler}. 
These components are instantiated in each thread locally.
The execution outline of \system is presented in Algorithm~\ref{alg:algo}.
Transactional stream processing is continuous and potentially never ends (Line 1$\sim$8).
The dependency resolution and execution of state transactions are separated into two non-overlapping phases by punctuations~\cite{Tucker:2003:EPS:776752.776780} (Line 2 and 5), which guarantees that no subsequent input event will have a smaller timestamp. 
Effectively, a batch of state transactions is collected during the first phase, and processed during the second phase.

In the first phase (i.e., stream processing phase), 
the \emph{StreamManager} conducts preprocessing for every input event ($e$). Similar to some prior works~\cite{tstream}, state transactions may be issued but not immediately processed during preprocessing (Line 3).
The \emph{pre\_processing} and \emph{post\_processing} functions are exposed as APIs to users.
The \emph{TxnManager} handles dependency resolution (Line 4) among state transactions and insert decomposed operations to construct a \tpg. We discuss the detailed two-phase \tpg construction process in Section~\ref{subsec:construction}.

In the second phase  (i.e., transaction processing phase), 
the \emph{TxnManager} is first involved again to refine (Line 6) the constructed \tpg with further dependency resolution.
The \emph{TxnScheduler} 
schedules operations for concurrent execution based on the constructed \tpg according to the three dimensions of scheduling decisions (Line 7). 
In particular, a scheduling decision model $M$ is instantiated based on the constructed \tpg (Line 14).
\textbf{\circled{1}} Guided by $M$, execution threads adopt an exploration strategy (Section~\ref{subsec:explore}) to explore the constructed \tpg for operations available to be scheduled constrained by dependencies. 
\textbf{\circled{2}} 
During exploration, one or multiple operations may be treated as the 
% basic 
unit of scheduling (Section~\ref{subsec:granularity}). 
Subsequently, \textbf{\circled{3}} every thread executes operation(s) in the unit of scheduling with various abort handling mechanisms (Section~\ref{subsec:abort_handling}).
Only when state transactions are processed (i.e., committed or aborted) can the associated input events be postprocessed (Line 8) by the \emph{StreamManager} based on transaction processing results.
\end{comment}

\begin{comment}
\begin{algorithm}
\footnotesize
    \KwData{$e$ \tcp{Input event}}
    \KwData{$txn_{ts}$ \tcp{State transaction}}
    \KwData{$G$ \tcp{The currently constructed TPG}}
    \While{!finish processing of input streams}{
        \eIf(\tcp*[h]{Phase 1}){\text{$e$ is not a $punctuation$}}{
                $txn_{ts}$ $\gets$ PRE\_Processing($e$)\;
                \textbf{TPG\_Construction}($G$, $txn_{ts}$)\; 
          }(\tcp*[h]{Phase 2}){
                \textbf{TPG\_Refinement}($G$)\; 
                \textbf{TXN\_Scheduling}($G$)\; 
                POST\_Processing()\;
          }
    }
    
    \SetKwFunction{FMain}{TPG\_Construction}
    \SetKwProg{Fn}{Function}{:}{}
    \Fn{\FMain{$G$, $txn_{ts}$}}{
        $O_{1..k}$ $\gets$ \textbf{Partition} $txn_{ts}$\;
        \ForEach{\text{operation $O_{i}$ $\in$ $O_{1..k}$}}{
            \textbf{Identify} its \ld\;
            $G$ $\gets$ $G$ + $O_{i}$ \;
        }
    }
    \SetKwFunction{FMain}{TPG\_Refinement}
    \SetKwProg{Fn}{Function}{:}{}
    \Fn{\FMain{$G$}}{
        \ForEach{\text{vertex $e_{i}$ $\in$ $G$}}{
            \textbf{Identify} its \td, \pd\;
        }
    }
    
    \SetKwFunction{FMain}{TXN\_Scheduling}
    \SetKwProg{Fn}{Function}{:}{}
    \Fn{\FMain{$G$}}{
        $M$ $\gets$ Instantiated with $G$;\tcp{A decision model}
        \While{!finish scheduling of $G$
        }{
          \textbf{\circled{2}} $Scheduling Unit$ $\gets$ \textbf{\circled{1}} \emph{Explore}($G$, $M$)\; 
            \textbf{\circled{3}} \emph{Execute with Abort Handling} ($Scheduling Unit$)\; 
        }
    }
  \caption{Execution Outline of \system}
  \label{alg:algo}
\end{algorithm}
\end{comment}
\end{document}
