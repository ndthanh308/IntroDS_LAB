\documentclass[amsmath,amssymb,superscriptaddress,preprint,floatfix,footinbib]{revtex4-1}
\usepackage[]{graphicx}
\usepackage{tabularx}
\usepackage[usenames,dvipsnames]{color}
\usepackage{soul}
\usepackage{bm}

\usepackage{times}
\usepackage{amsfonts}
\usepackage{mathrsfs}
\usepackage{graphicx}% Include figure files
\usepackage{dcolumn}% Align table columns on decimal point
\usepackage{bm}% bold math
\usepackage{color}
\usepackage[colorlinks,bookmarks=false,citecolor=blue,linkcolor=red,urlcolor=blue]{hyperref}
\bibliographystyle{apsrev}
\usepackage{multirow}
\usepackage{physics}

\renewcommand{\thefigure}{S\arabic{figure}}

\begin{document}


\title{The electronic structure of intertwined kagome, honeycomb, and triangular sublattices of the intermetallics MCo$_2$Al$_9$}

\author{Chiara Bigi}\email{chiara.bigi@synchrotron-soleil.fr}
\affiliation{SUPA, School of Physics and Astronomy, University of St Andrews, St Andrews KY16 9SS, UK}\affiliation{Synchrotron SOLEIL, F-91190 Saint-Aubin, France}

\author{Sahar Pakdel}
\affiliation{CAMD, Department of Physics, Technical University of Denmark, 2800 Kgs. Lyngby, Denmark}

\author{Micha{\l} J. Winiarski}
\affiliation{Faculty of Applied Physics and Mathematics, Advanced Materials Centre, Gdansk University of Technology, Narutowicza 11/12, 80-233, Gdansk, Poland}

\author{Pasquale Orgiani}
\affiliation{Istituto Officina dei Materiali (IOM)-CNR, Laboratorio TASC, Area Science Park, S.S.14, Km 163.5, 34149 Trieste, Italy}

\author{Ivana Vobornik}
\affiliation{Istituto Officina dei Materiali (IOM)-CNR, Laboratorio TASC, Area Science Park, S.S.14, Km 163.5, 34149 Trieste, Italy}

\author{Jun Fujii}
\affiliation{Istituto Officina dei Materiali (IOM)-CNR, Laboratorio TASC, Area Science Park, S.S.14, Km 163.5, 34149 Trieste, Italy}

\author{Giorgio Rossi}
\affiliation{Istituto Officina dei Materiali (IOM)-CNR, Laboratorio TASC, Area Science Park, S.S.14, Km 163.5, 34149 Trieste, Italy}\affiliation{Department of Physics, University of Milano, 20133 Milano, Italy}

\author{Vincent Polewczyk}
\affiliation{Istituto Officina dei Materiali (IOM)-CNR, Laboratorio TASC, Area Science Park, S.S.14, Km 163.5, 34149 Trieste, Italy}

\author{Phil D.C. King}
\affiliation{SUPA, School of Physics and Astronomy, University of St Andrews, St Andrews KY16 9SS, UK}

\author{Giancarlo Panaccione}
\affiliation{Istituto Officina dei Materiali (IOM)-CNR, Laboratorio TASC, Area Science Park, S.S.14, Km 163.5, 34149 Trieste, Italy}

\author{Tomasz Klimczuk} 
\affiliation{Faculty of Applied Physics and Mathematics, Advanced Materials Centre, Gdansk University of Technology, Narutowicza 11/12, 80-233, Gdansk, Poland}

\author{Kristian Sommer Thygesen}\email{thygesen@fysik.dtu.dk}
\affiliation{CAMD, Department of Physics, Technical University of Denmark, 2800 Kgs. Lyngby, Denmark}

\author{Federico Mazzola}\email{federico.mazzola@unive.it}
\affiliation{Department of Molecular Sciences and Nanosystems, Ca’ Foscari University of Venice, 30172 Venice, Italy}\affiliation{Istituto Officina dei Materiali (IOM)-CNR, Laboratorio TASC, Area Science Park, S.S.14, Km 163.5, 34149 Trieste, Italy}

\date{\today}
\maketitle
%\newpage
\section*{Methods}
{\bf Tight binding model.} The electronic structure for Al s-orbitals arranged in a kagome lattice in Fig.1e was obtained for on-site energy E$_0=0$~eV and hopping $\sigma$ bond V$_ss=-1$~eV. The nearest neighbour hopping distance was set equal to the Al-Al kagome net in the $ab$-plane of MCo$_2$Al$_9$ system (a=b=$7.9$~\AA). An impurity scattering broadening term Im($\Sigma$)~=~$100$~meV was introduced to simulate the finite lifetime of a realistic spectral function.\\

{\bf Experimental details.} Single crystals of SrCo$_2$Al$_9$ and BaCo$_2$Al$_9$ were grown employing the self-flux technique \cite{Canfield_1992} using alumina crucibles and a frit-disc \cite{Canfield_2016}. Sr ($99.95\%$ pure, Onyxmet) or Ba dendrites ($99.95\%$, Onyxmet), Co scraps ($99.9\%+$, Alfa Aesar) and Al slug ($99.999\%$, Onyxmet) were put in an alumina crucible at the atomic ratio of $2:3:60$ (ca. $1.2$~g total for each sample) and sealed in a quartz tube, evacuated and backfilled with Ar to dilute the Al vapor attacking the tube walls. The ampoules were placed in a box furnace, heated to $1120^{\circ}$~C, held for $6$~hours and then slowly cooled ($2^{\circ}$~C/h) to $900^{\circ}$~C (SrCo$_2$Al$_9$) or $845^{\circ}$C (BaCo$_2$Al$_9$). The temperature program and sample compositions were adjusted (based on recent crystal growth reports \cite{Ryzynska_2020,Meier_2021}) to yield single crystals free of SrAl$_4$/BaAl$_4$ contamination and of sufficient size for ARPES measurements. At the final temperature, each ampoule was centrifugated to remove the excess Al-rich flux. The growth process yields numerous hexagonal needle-like single crystals, the largest being ca. $1$~mm wide and ca. $5$~mm long. The remaining droplets of solidified Al were etched from the crystal surfaces using ca. $0.1$~M NaOH solution in an ultrasonic bath. Selected single crystals were fine-ground and examined using a powder x-ray diffractometer (Bruker D2 Phaser instrument, CuK$\alpha$ source, LynxEye XE-T detector) to confirm the phase composition. Obtained powder patterns were consistent with previous reports \cite{Ryzynska_2020,Turban_1975}. The ARPES measurements were carried out at the synchrotron radiation source ELETTRA by using the DA30 hemispherical analyser of the APE-LE end-station (Trieste). The energy and momentum resolutions were better than $12$~meV and $0.018$~\AA$^{-1}$, respectively. The sample temperature was kept constant at $77$~K throughout the experiment. Radial momentum distribution curves (MDC) were extracted every $5$ degrees over $180^{\circ}$ range from the SrCo$_2$Al$_9$ Fermi surface and fitted to Lorentzian peak to extract the projected electron carrier density. A factor two was introduced to account for the two bands forming the Fermi contour as resulting from DFT calculations.\\

{\bf DFT.} First-principles calculations are performed using density functional theory (DFT) \cite{PhysRev.140.A1133} within the projector-augmented wave (PAW) formalism \cite{PhysRevB.50.17953} as implemented in the GPAW code \cite{Enkovaara_2010, PhysRevB.71.035109}. All the calculations were performed using the Atomic Simulation Recipes (ASR) framework \cite{GJERDING2021110731} in combination with the atomic simulation environment (ASE) \cite{Larsen_2017}. The exchange correlation potentials were described through the Perdew-Burke-Ernzerhof (PBE) functional within the generalized gradient approximation (GGA) formalism \cite{perdew}. A plane-wave cutoff energy of 800 eV is employed for all calculations and the width of the Fermi-Dirac smearing is kept at 50 meV. For geometry optimization, 6/$\AA^{-1}$ kmesh density is used and forces are converged until 0.01 eV/$\AA^{-1}$. After lattice relaxation, the electron density is determined self-consistently on a uniform k-point grid of density 12.0/$\AA^{-1}$. From this density, the PBE band structure is computed non-self consistently at 400 k-points distributed along the band path. Spin-orbit coupling (SOC) is added non-self consistently.\\

\clearpage
% Figure environment removed

% Figure environment removed

% Figure environment removed

% Figure environment removed
\clearpage
% Figure environment removed

\bibliography{references.bib}

\end{document}