\section{REG Corpora} \label{sec:corpus}

In the following, we explain the corpora used in this work. These corpora are English-language corpora.
\subsection{The \msr and \negc Corpora}
In the current study, we only use the articles from the training sets of these corpora (see the number of documents in Table \ref{tab:corpora}). Following the same approach as \citet{castro-ferreira-etal-2018-neuralreg}, we created a version of the GREC corpora for the End-to-end (E2E) REG modelling. For the classic ML models, we reproduced the models using the feature sets from the studies mentioned in Section \ref{subsec:systems}.

\subsection{The \wsj Corpus}

As mentioned earlier, the WSJ portion of the OntoNotes corpus \citep{weischedel2013ontonotes} is our third data source.\footnote{We used Ontonotes 5.0 licensed by the Linguistic Data Consortium (LDC) \url{https://catalog.ldc.upenn.edu/LDC2013T19}.} We use the version of the corpus that \citet{same-etal-2022-non} developed for E2E REG modeling.\footnote{Note that \wsj was used in~\citet{same-etal-2022-non}, but no corpus analysis or comparison was provided.} Since empty pronouns are not annotated in \wsj, we decided to also exclude them from the two GREC corpora and focus on a 3-label classification task. The labels considered in this study are \emph{pronoun}, \emph{description},  and \emph{proper name}. Table \ref{tab:corpora} presents a detailed overview of these corpora.

\begin{table}[tbp]
\small
\begin{tabular}{lccc}
\toprule
 & \msr & \negc & \wsj  \\ \hline
number of documents            & 1655 & 808   & 582   \\ 
word/doc (mean)           & 148 & 129   & 530   \\ 
sent/doc (mean)      & 7.1 & 5.8   & 25   \\ 
par/doc (mean)      & 2.3 & 2.2   & 10.8   \\ 
referent/doc (mean) & 1 & 2.6   & 15   \\ 
%{|l|}{\textbf{M length sentence}} & 25.8 & 25.8   & 29.5   \\ \hline
number of RE & 11705 & 8378  & 25400  \\ 
%{|l|}{Mean N of mentions / chain}        & 7   & -   \\ \hline
description \%               & 13.84\% & 4\%   & 38.29\%   \\ 
proper name \%                 & 38.09\% & 40.79\%   & 34.57\%   \\ 
pronoun \%              & 41.79\% & 48.75\%   & 27.14\%   \\ 
empty \%              & 6.28\% & 6.47\%   & -   \\ \hline
\end{tabular}
\caption{\label{tab:corpora}
Comparison of the \msr, \negc, and \wsj corpora in terms of their length-related characteristics and distribution of REs. \textit{Doc}, \textit{sent} and \textit{par} stands for \textit{documents}, \textit{sentences} and \textit{paragraphs}.
}
\end{table}

\paragraph{Data Splits.} We have made a document-wise split of the data. We split the \wsj~data in accordance with the CoNLL 2012 Shared Task \citep{pradhan-etal-2012-conll}. Our \wsj training, development, and test sets contain 20275, 2831, and 2294 samples, respectively. We did an 85-5-10 split of the GREC datasets in accordance with \citet{belz2009generating}. After excluding empty pronouns, the \msr~training, development, and test sets contain 9413, 519, 1038 instances, and the \negc~training, development, and test sets contain 6681, 259, 896 instances.

\paragraph{Proportion of Referring Expressions} As shown in Table \ref{tab:corpora}, pronouns and proper names make up 80\% and 89.5\% of the referential instances in \msr and \negc, respectively. This implies that the other two referential forms, namely descriptions and empty references, account for approximately 20\% of the cases in \msr and about 10\% in \negc. Given this imbalance in the frequency of different forms within the two corpora, we question its potential effect on algorithm performance. Specifically, we are wondering if forms with lower frequencies are accurately predicted by the algorithms.