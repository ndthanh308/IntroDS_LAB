\section{Research Questions} \label{sec:rq}

15 years after the GREC shared tasks, we were curious to know to what extent the conclusions from GREC still ``stand''. We, therefore, came up with the following research questions.

In the first place, we are interested in \emph{the impact of the choice of corpus on the performance of REG algorithms} ($\mathcal{R}_1$). GREC uses only the introductory part of Wikipedia articles (see Section~\ref{sec:grec}), which represents only one genre of human language use. Considering that a good REG algorithm needs to model the general use of reference, a better evaluation framework should include texts from multiple genres. Therefore, we also include the WSJ corpus in the study
(see Section~\ref{sec:corpus} for more details) and conduct a correlation analysis to quantify how the choice of corpus impacts the evaluation results. 

Second, previous studies suggested that classic machine learning (ML) based REG algorithms perform on par with most recent neural methods~\citep{same-etal-2022-non}. However, their study has three limitations: (1) they did not incorporate pre-trained language models (PLMs); (2) they focused on the surface forms of REs, which partly depend on the performance of surface realisation; (3) they did not assess the models based on the intuition that a model with good explanatory power should be less influenced by the choice of corpus. Therefore, we adopt PLMs to the task of REG-in-context (see Section~\ref{sec:algorithm} for more details) and investigate \emph{how good is the explanatory power of PLM-based REG models compared to classic ML-based models} ($\mathcal{R}_2$) using the enhanced GREC framework.

Finally, as previously mentioned, one of the primary theoretical objectives of GREC was to computationally explore the contribution of factors that originate from linguistic studies to the choice of referential forms. It is reasonable to expect that such contributions may change depending on the choice of corpus. In this study, we conduct an importance analysis to investigate \emph{whether the importance ranking of linguistic factors changes when we use different corpora} ($\mathcal{R}_3$).
