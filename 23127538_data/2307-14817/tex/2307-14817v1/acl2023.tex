% This must be in the first 5 lines to tell arXiv to use pdfLaTeX, which is strongly recommended.
\pdfoutput=1
% In particular, the hyperref package requires pdfLaTeX in order to break URLs across lines.

\documentclass[11pt]{article}

% Remove the "review" option to generate the final version.
\usepackage{ACL2023}
%\usepackage[subtle]{savetrees}
%\usepackage[margin=2cm]{geometry}
\usepackage{tikz,amsmath, amssymb,bm,color, amsthm,amsfonts}
\usetikzlibrary{positioning, calc,chains,fit,shapes}
%\usetikzlibrary{circuits.logic.US,circuits.logic.IEC,fit}
\usepackage{enumerate}
\usepackage{comment}
\usepackage{tikz}
\usepackage{graphics}
%\usepackage[cm]{fullpage}
\usepackage{longtable}
\usepackage{mdframed}
\usepackage{caption}
\usepackage{subcaption}
\usepackage{slashbox}
\usepackage{url}
\usepackage{framed}
\usepackage{array}
\usepackage{tabu}
\usepackage{lscape}
\usepackage{multirow}
\usepackage{ulem}
\usepackage{multicol}
\usepackage{placeins}
\usepackage{cite}
\usepackage{enumitem}
\usepackage{mathtools}
%\usepackage[numbers]{natbib}
%\usepackage{mathtools}
%\usepackage{authblk}

\mdfsetup{skipabove=2pt,skipbelow=2pt}
%\setlenght {\marginparwidth }{2cm}
%\usepackage{todonotes}

%\usepackage{floatrow}
%\usepackage{adjustbox}
%\setlength{\extrarowheight}{.05ex}
%\renewcommand\thesubfigure{\roman{subfigure}}


%\newtheorem{theorem}{Theorem}[section]
%\newtheorem{lemma}[theorem]{Lemma}
%\newtheorem{observation}[theorem]{Observation}
%\newtheorem{corollary}[theorem]{Corollary}
%\newtheorem{proposition}[theorem]{Proposition}
%\newtheorem{definition}[theorem]{Definition}
\newtheorem{construction}{Construction}
%\newtheorem{conjecture}{Conjecture}
%\newtheorem{remark}[theorem]{Remark}

\newcommand{\pname}[1]{\textnormal{\textsc{#1}}}
\newcommand{\cclass}[1]{\textnormal{\textsf{#1}}}
\newcommand{\nog}{nine} % no of members in the gang!
\newcommand{\nogd}{nineteen} % no of members in the gang - for deletion/completion
\newcommand{\nogl}{eighteen} % no of members in the larger gang - for editing
\newcommand{\nogld}{thirty eight} % no of members in the larger gang - for deletion/completion
\newcommand{\diffnog}{ten} %
%\newcommand{\dominatedby}{dominated by} %
%\newcommand{\dominatingset}{dominating set} %
%\newcommand{\dominates}{dominates} %
\newcommand{\simulates}{simulates} %
\newcommand{\baseset}{base} %
\newcommand{\issimulatedby}{is simulated by} %

\newcommand{\StarSAT}{\pname{8-SAT$_{\geq 6}$}}
\newcommand{\FSAT}{\pname{4-SAT$_{\geq 2}$}}
\newcommand{\FISAT}{\pname{5-SAT$_{\geq 3}$}}
\newcommand{\SIXSAT}{\pname{6-SAT$_{\geq 4}$}}
\newcommand{\ESAT}{\pname{8-SAT$_{\geq 6}$}}
\newcommand{\KSAT}{\pname{$k$-SAT$_{\geq {k-2}}$}}
\newcommand{\KSATO}{\pname{$k$-SAT}}
\newcommand{\ESATO}{\pname{8-SAT}}
\newcommand{\FSATO}{\pname{4-SAT}}
\newcommand{\FISATO}{\pname{5-SAT}}
\newcommand{\TSAT}{\pname{3-SAT}}
\newcommand{\HED}{\pname{${H}$-free Edge Deletion}}
\newcommand{\AEE}{\pname{${A}$-free Edge Editing}}
\newcommand{\AED}{\pname{${A}$-free Edge Deletion}}
\newcommand{\TSED}{\pname{$t$-star-free Edge Deletion}}
\newcommand{\ATSED}{\pname{Annotated $t$-star-free Edge Deletion}}
\newcommand{\AFSED}{\pname{Annotated $4$-star-free Edge Deletion}}
\newcommand{\FSED}{\pname{$4$-star-free Edge Deletion}}
\newcommand{\FVSED}{\pname{$5$-star-free Edge Deletion}}
\newcommand{\HEE}{\pname{${H}$-free Edge Editing}}
\newcommand{\HEC}{\pname{${H}$-free Edge Completion}}
\newcommand{\HDEE}{\pname{${H'}$-free Edge Editing}}
\newcommand{\HDDEE}{\pname{${H''}$-free Edge Editing}}
\newcommand{\HDED}{\pname{${H'}$-free Edge Deletion}}
\newcommand{\HDEC}{\pname{${H'}$-free Edge Completion}}
\newcommand{\HBEE}{\pname{${\overline{H}}$-free Edge Editing}}
\newcommand{\HBED}{\pname{${\overline{H}}$-free Edge Deletion}}
\newcommand{\HBEC}{\pname{${\overline{H}}$-free Edge Completion}}
\newcommand{\HOEDCE}{\pname{${H_1}$-free Edge Deletion(Completion/Editing)}}
\newcommand{\HEDCE}{\pname{${H}$-free Edge Deletion(Completion/Editing)}}
\newcommand{\HEEDC}{\pname{${H}$-free Edge Editing(Deletion/Completion)}}
\newcommand{\HDEEDC}{\pname{${H'}$-free Edge Editing(Deletion/Completion)}}
\newcommand{\BFED}{\pname{Bow-free Edge Deletion}}
\newcommand{\ABFED}{\pname{Annotated Bow-free Edge Deletion}}
\newcommand{\DTIS}{\pname{Distance-3 Independent Set}}
\newcommand{\SVC}{\pname{Strong Vertex Cover}}
\newcommand{\CLIQUE}{\pname{Clique}}
\newcommand{\IS}{\pname{Independent Set}}
\newcommand{\PFS}{\pname{Propagational-$f$ Satisfiability}}
\newcommand{\RHED}{\pname{Restricted ${H}$-free Edge Deletion}}
\newcommand{\RHEC}{\pname{Restricted ${H}$-free Edge Completion}}
\newcommand{\RHDED}{\pname{Restricted ${H'}$-free Edge Deletion}}
\newcommand{\RHDEC}{\pname{Restricted ${H'}$-free Edge Completion}}
\newcommand{\RHEE}{\pname{Restricted ${H}$-free Edge Editing}}
\newcommand{\PH}{$\cclass{NP} \subseteq \cclass{coNP/poly}$}
\newcommand{\NOPH}{$\cclass{NP} \not\subseteq \cclass{coNP/poly}$}
\newcommand{\LG}{\mathcal{W}}
\newcommand{\LGD}{\mathcal{W}'}
\newcommand{\LGDD}{\mathcal{W}''}


%\let\oldvee\vee
\renewcommand\vee{\boxtimes}

\newcommand\addvmargin[1]{
  \node[fit=(current bounding box),inner ysep=#1,inner xsep=0]{};
}
\setlength{\fboxrule}{0pt}

\newcommand{\defstage}[2]{% PGD Version
  \hfill\\\smallskip\noindent%
  \begin{tabularx}{\textwidth}{|l X|}%
    \hline%
    \multicolumn{2}{|l|}{\textbf{#1}}\\%
    &#2\\\hline%
  \end{tabularx}%
%  \smallskip%
}
\setlength\extrarowheight{15pt}

\newcounter{rowcntr}[table]
\renewcommand{\therowcntr}{\thetable.\arabic{rowcntr}}

% A new columntype to apply automatic stepping
\newcolumntype{N}{>{\refstepcounter{rowcntr}\therowcntr}c}

% Reset the rowcntr counter at each new tabular
\AtBeginEnvironment{longtabu}{\setcounter{rowcntr}{0}}

\newcounter{rowcntra}[table]
\renewcommand{\therowcntra}{\arabic{rowcntra}}

% A new columntype to apply automatic stepping
\newcolumntype{M}{>{\refstepcounter{rowcntra}\therowcntra}c}

% Reset the rowcntr counter at each new tabular
\AtBeginEnvironment{tabular}{\setcounter{rowcntra}{0}}

\newcommand{\NPC}{NP-Complete}


\newcommand{\highlight}[1]{\textcolor{blue}{#1}}
\newcommand{\dhanya}[1]{\textcolor{blue}{dhanya: #1}}


%\newcommand{\XCD1}[1]{\pname{$\chi_{cd}$\ensuremath{(#1)}}}
\newcommand{\XCD}{\pname{$\chi_{cd}$}}
\newcommand{\SC}{\pname{$\omega_{s}$}}

\newcommand{\CDC}{\textsc{CD-coloring}}
\newcommand{\SCP}{\textsc{Separated-Cluster}}
\newcommand{\TD}{\textsc{Total Domination}}
\newcommand{\ISP}{\textsc{Independent Set}}
\newcommand{\CC}{\textsc{Clique Cover}}
\newcommand{\TETHS}{Further, the problem cannot be solved in time \ensuremath{2^{o(|V(G)|)}}, unless the ETH fails}
%\usetikzlibrary{positioning,chains,shapes,calc}
\usetikzlibrary{fit}
\thispagestyle{empty}
\usetikzlibrary{
  graphs,
  graphs.standard
}
% Standard package includes
\usepackage{times}
\usepackage{latexsym}
\usepackage{booktabs}
\usepackage{multirow}
\usepackage{makecell}
\usepackage{xspace}

% For proper rendering and hyphenation of words containing Latin characters (including in bib files)
\usepackage[T1]{fontenc}
% For Vietnamese characters
% \usepackage[T5]{fontenc}
% See https://www.latex-project.org/help/documentation/encguide.pdf for other character sets

% This assumes your files are encoded as UTF8
\usepackage[utf8]{inputenc}

% This is not strictly necessary, and may be commented out.
% However, it will improve the layout of the manuscript,
% and will typically save some space.
\usepackage{microtype}

% This is also not strictly necessary, and may be commented out.
% However, it will improve the aesthetics of text in
% the typewriter font.
\usepackage{inconsolata}


% If the title and author information does not fit in the area allocated, uncomment the following
%
%\setlength\titlebox{<dim>}
%
% and set <dim> to something 5cm or larger.

\title{Models of reference production: How do they withstand the test of time?}

% Author information can be set in various styles:
% For several authors from the same institution:
% \author{Author 1 \and ... \and Author n \\
%         Address line \\ ... \\ Address line}
% if the names do not fit well on one line use
%         Author 1 \\ {\bf Author 2} \\ ... \\ {\bf Author n} \\
% For authors from different institutions:
% \author{Author 1 \\ Address line \\  ... \\ Address line
%         \And  ... \And
%         Author n \\ Address line \\ ... \\ Address line}
% To start a seperate ``row'' of authors use \AND, as in
% \author{Author 1 \\ Address line \\  ... \\ Address line
%         \AND
%         Author 2 \\ Address line \\ ... \\ Address line \And
%         Author 3 \\ Address line \\ ... \\ Address line}

\author{Fahime Same\textsuperscript{$\heartsuit$}, 
Guanyi Chen\textsuperscript{$\spadesuit$}, \and
Kees van Deemter\textsuperscript{$\spadesuit$}\\
\textsuperscript{$\heartsuit$}Department of Linguistics, University of Cologne\\
\textsuperscript{$\spadesuit$}Department of Information and Computing Sciences, Utrecht University\\
\texttt{f.same@uni-koeln.de, g.chen@ccnu.edu.cn, c.j.vandeemter@uu.nl}}

\begin{document}
\maketitle
\begin{abstract}
In recent years, many NLP studies have focused solely on performance improvement. In this work, we focus on the linguistic and scientific aspects of NLP. We use the task of generating referring expressions in context (REG-in-context) as a case study and start our analysis from GREC, a comprehensive set of shared tasks in English that addressed this topic over a decade ago. We ask what the performance of models would be if we assessed them (1) on more realistic datasets, and (2) using more advanced methods. 
We test the models using different evaluation metrics and feature selection experiments. We conclude that GREC can no longer be regarded as offering a reliable assessment of
models' ability to mimic human reference production, because the results are highly impacted by the choice of corpus and evaluation metrics.
Our results also suggest that pre-trained language models are less dependent on the choice of corpus than classic Machine Learning models, and therefore make more robust class predictions.
\end{abstract}

% Figure environment removed

\section{Introduction}
Automatic 3D reconstruction of clothed humans using image inputs has gained increasing significance due to its potential applications in a wide array of AR/VR scenarios. High-fidelity reconstructions typically depend on sophisticated capture systems, which are developed with dense camera arrays~\cite{collet2015high,joo2015panoptic,joo2018total}, programmable light-stages~\cite{Vlasic2009, guo2019relightables}, and depth sensors~\cite{newcombe2011kinectfusion,DoubleFusion,BodyFusion,dou2016fusion4d,newcombe2015dynamicfusion}. However, stringent capture environments equipped with complex hardware pose significant challenges for consumer-level applications.


In this context, considerable research effort has been dedicated to developing methods that allow for more flexible capture configurations, such as utilizing a few RGB inputs. Among these works, learning implicit functions \cite{iccv2020PIFu, saito2020pifuhd, hong2021stereopifu} has proven effective in achieving highly detailed reconstructions by integrating the advancements of deep neural networks. These methods employ large multi-layer perceptrons (MLPs) to predict the occupancy probability or truncated signed distance function (TSDF) value of every queried 3D point based on its associated local feature, which is extracted from images. They can recover a continuous surface at arbitrary resolutions without topology restrictions.


However, in typical MLP-based implicit networks, the occupancy or TSDF value at each location is solved independently with planar image features, rendering them less capable of addressing challenging cases such as occlusions. Consequently, these methods suffer from generalization and robustness issues, particularly when tackling strong occlusions caused by large motion or multiple interacting humans. 
Some follow-up studies  \cite{zheng2021deepmulticap,zheng2021pamir,huang2020arch} utilize an extra geometric model, SMPL~\cite{Loper2015}, to improve robustness by introducing strong shape priors. 
Their success typically relies on the assumption of geometrical similarity \cite{huang2020arch} between the shape prior and target reconstruction, making them intractable for handling complex cases with loose clothes and sensitive to errors in SMPL model fitting.



%\ping{this paragraph sounds like `TSDF is better than MLP/SMPL, and we use TSDF to solve the problem'. But in Sec 3, we are telling a different story, saying `MLP needs a 3D convolutional encoder'. We need to make these two sections consistent.}\sicong{I think in this paragraph we claim that the TSDF}


%We opt for Trucated Signed Distance Funtion (TSDF) volumetric representations as they are naturally suitable for convolution operations, which have shown remarkable performance for learning hierarchical features on 2D visual perception tasks \cite{SunXLW19}. 
%Meanwhile, TSDF also describes the gradual geometry change around shape surface, which is not reflected by occupancy volume. 

We instead revisit the 3D volumetric representation and resort to 3D convolutional neural networks (CNNs) for feature learning, due to their impressive performance in feature learning and the ability to incorporate spatial context. However, volumetric methods and 3D convolution involve discretization, which might raise concerns regarding whether a discretized volume can preserve subtle geometric details as continuous representations learned in implicit functions. We investigate the relationship between volume resolution and quantization error on synthetic data by converting target mesh objects to TSDF volumes, as shown in Figure~\ref{fig:quantization_error}. We observe that the quantization errors are significantly reduced by increasing volume resolution and become nearly negligible when reaching a relatively high resolution (e.g., 512 or higher). In other words, achieving fine-detailed reconstruction is not supposed to be restricted by the use of volume representations as long as a proper volume resolution is utilized. Therefore, we present a method with high-resolution feature volumes, e.g., 256 and 512, while traditional volumetric methods \cite{varol18_bodynet,gilbert2018volumetric} are often limited to much lower resolutions, such as 32 or 128.



On the other hand, an increase in volume resolution may lead to a cubic growth of memory overhead \cite{8100085}. Reducing memory costs while guaranteeing the granularity of volumetric representations is necessary for pursuing high-quality reconstruction. Thus, we adopt a coarse-to-fine approach and cull away irrelevant voxels to build a sparse high-resolution feature volume. At the coarse level, the network computes an initial TSDF by applying a U-Net with sparse 3D CNN \cite{3DSemanticSegmentationWithSubmanifoldSparseConvNet} on the sparse feature volume, which is carved by a visual hull. Through our experiments, it turns out that more than 95\% of the volume grids are discarded by the visual hull culling, making the sparse 3D CNN efficient. At the fine level, the network focuses on a narrow band near the zero-level set of the initial TSDF and discretizes the narrow band with smaller voxels. By employing this narrow-band culling, we further shrink the sampling space, resulting in a relatively small range of grid numbers (usually 300K--500K in our experiments) even with a high volume resolution of 512. The remaining voxels in the narrow band are associated with features that fuse high-frequency information from the computed normal maps upon the low-frequency shape from the coarse level to compute the TSDF at high resolution. The final mesh is then extracted from the TSDF using the Marching-Cube algorithm ~\cite{Lorensen87marchingcubes}.
% Different from the u-net sturcture to preserve global topology context, we then apply a shallow 3dcnn to compute the final TSDF $D_{final}$ which contain more local geometry detail.




% \ping{this paragraph can be expanded. It is an important contribution and often ignored by other works. stress on the novel idea of regressing blending weights instead of colors}

In addition to geometry, high-quality mesh texture is also a crucial factor contributing to visual appearance. Directly computing a color field in 3D space, as in \cite{iccv2020PIFu}, struggles to capture high-frequency texture details, while the neural radiance field (NeRF) \cite{yu2020pixelnerf} or the DoubleField~\cite{shao2022doublefield} require expensive per-instance optimization and are often unstable for sparse input images. In contrast, we adopt an image-based rendering approach to compute a texture atlas map, which is efficient and widely supported in existing computer graphics tools. 
Specifically, we compute a blending weight at each 3D point on the mesh surface to determine its color as a weighted average of the colors at its image projections. The blending weights can be computed at a relatively coarse resolution, e.g., 512 volume resolution in our case, and leave texture details to the high-resolution images, such as 1K or 2K. Unlike previous methods that generate blurry texturing results under sparse input, our method generalizes well on both synthetic and real data with just a few input views. 
Figure~\ref{fig:teaser} shows two examples reconstructed by our method. Despite the challenging garment, pose, and occlusion, our method recovers faithful shape, normal, and texture on the right.

%with a wide variety of poses and clothing styles, and it is also adaptive to handle input image with arbitrary resolutions.
%\sicong{For this concern we claim that when the resolution of dicretized volume meets certain threshold (which is 256 in our experiment), the quantization error can be neglected.} 



In summary, the main contributions of this paper are as follows:
\begin{itemize}
\vspace{-0.1in}
  \item 
  We revisit the 3D volumetric representation and demonstrate that it can support clothed human reconstruction with equal or even better performance compared to implicit representation. 
  \item 
  We develop a memory and computation-efficient method for high-resolution volumetric reconstruction using sophisticated sparse 3D CNN, coarse-to-fine estimation, and voxel culling by visual hull and narrow bands. 
  \item 
  We introduce a novel method to compute a texture atlas map, which captures rich appearance details from high-resolution input images.
  \item 
  We achieve impressive results on standard benchmark datasets Twindom and MultiHuman, significantly reducing the point-2-surface (P2S) precision to approximately 0.2cm from just six input views, with more than $50\%$ error reduction compared to the state-of-the-art methods, including DoubleField~\cite{shao2022doublefield} and PIFuHD~\cite{saito2020pifuhd}.
\end{itemize}

\section{The GREC Shared Tasks}\label{sec:grec}

In this section, we summarise the GREC task, the corpora used by GREC, and its conclusions.

\subsection{The GREC Task and its Corpora} \label{subsec:greccorpora}

According to \citeauthor{belz2009generating}, ``\emph{the GREC tasks are about how to generate appropriate references to an entity in the context of a piece of discourse longer than a sentence}" \citeyearpar[p.~297]{belz2009generating}. The main task 
was to predict the referential form, namely whether to use a pronoun, proper name, description or an empty reference at a given point in discourse.

The GREC challenges use two corpora, both created from the introductory sections of Wikipedia articles: (1) GREC-2.0 (henceforth \msr, as it was used in the GREC-MSR shared tasks of 2008 and 2009) consists of 1941 introductory sections of the articles across five domains (people, river, mountain, city, and country); and (2) GREC-People (henceforth \negc as it was used in the GREC-NEG shared task in 2009) contains 1000 introductory sections from Wikipedia articles about composers, chefs, and inventors. Here is an example from \negc:
% 

\enumsentence{\underline{\textbf{David Chang}} (born 1977) is a noted American chef. \underline{\textbf{He}} is chef/owner of Momofuku Noodle Bar, Momofuku Ko and Momofuku Ssäm Bar in New York City. \underline{\textbf{Chang}} attended Trinity College, where \underline{\textbf{he}} majored in religious studies. In 2003, \underline{\textbf{Chang}} opened \underline{\textbf{his}} first restaurant, Momofuku Noodle Bar, in the East Village.}\label{ex:davidchang}

A key difference between \msr and \negc lies in their RE annotation practices. In \msr, only those REs that refer to the main topic of the article are annotated, while in \negc, mentions of all \textit{human} referents are annotated. For instance, in a document about David Chang, \msr will only annotate REs referring to David Chang, while \negc will include annotations for all human referents, including David Chang and others.

\subsection{REG Algorithms Submitted to GREC} \label{subsec:systems}

Various REG algorithms were submitted to the GREC challenges. 
These consist of feature-based ML algorithms: \cnts~\citep{hendrickx-etal-2008-cnts}, 
\icsi~\citep{favre-bohnet-2009-icsi}, \isg~\citep{bohnet-2008-g}, \osu~\citet{jamison-mehay-2008-osu} and \udel~\citet{greenbacker-mccoy-2009-udel}, and an algorithm that mixes feature-based ML and rules: \texttt{JUNLG}~\citep{gupta2009junlg}.
Table \ref{tab:grecsystems} presents the details of each model, including the ML method, and the original reported accuracy on \msr (cf. \citet{belz2009generating} for details).

\begin{table}[tbp]
\centering
\begin{tabular}{lccc}
\hline
Name & GREC ST & ALG & Acc   \\ \hline
\udel & \msr’09  & C5.0   & 77.71 \\ 
\icsi & \msr’09  & CRF    & 75.16 \\ 
\cnts & \msr’08  & MBL    & 72.61 \\ 
\isg & \msr’08  & MLP    & 70.78 \\ 
\osu & \msr’08  & MaxEnt & 69.82 \\ 
\texttt{JUNLG} & \msr’09 & Rule & 75.40 \\
\hline
\end{tabular}
\caption{\label{tab:grecsystems}
An overview of the algorithms submitted to GREC. The first column contains the name of the respective algorithm. The column GREC ST presents the name of the \msr shared task to which the algorithm was submitted. The third column, ALG, lists the algorithms used, where abbreviations from top to bottom are C5.0 decision tree, conditional random field, memory-based learning, multi-layer perceptron, maximum entropy, and frequency-based rules. The fourth column, Acc, reports the original accuracy of the algorithms, as reported in \citet{belz2009generating}. Note that \udel, \icsi, and \texttt{JUNLG} were submitted to both the \msr’08 and \msr'09 shared tasks, and we only present the newest results here.
}
\end{table}

\subsection{Feature Selection}

The GREC Tasks were designed to find out \emph{what kind of information is useful for making choices between different kinds of referring expressions in context} \citep[p. 297]{belz2009generating}. However, the original paper does not consider the factors that contributed to the RE choice in the systems submitted to GREC. In a follow-up study, \citet{greenbacker2009feature} conducted a feature selection study informed by psycholinguistics. They experimented with various feature subsets derived from their system, known as UDel, which had previously been submitted to the GREC. Additionally, they incorporated selected features from another REG system, CNTS \citep{hendrickx-etal-2008-cnts}, into their study. They show that features motivated by psycholinguistic studies and certain sentence construction features have a positive impact on the performance of REG models. Follow-up feature-selection studies including \citet{kibrik2016referential} and \citet{same-van-deemter-2020-linguistic} also emphasise the contribution of factors such as recency and grammatical role to the choice of RE form.


\section{Research Questions} \label{sec:rq}

15 years after the GREC shared tasks, we were curious to know to what extent the conclusions from GREC still ``stand''. We, therefore, came up with the following research questions.

In the first place, we are interested in \emph{the impact of the choice of corpus on the performance of REG algorithms} ($\mathcal{R}_1$). GREC uses only the introductory part of Wikipedia articles (see Section~\ref{sec:grec}), which represents only one genre of human language use. Considering that a good REG algorithm needs to model the general use of reference, a better evaluation framework should include texts from multiple genres. Therefore, we also include the WSJ corpus in the study
(see Section~\ref{sec:corpus} for more details) and conduct a correlation analysis to quantify how the choice of corpus impacts the evaluation results. 

Second, previous studies suggested that classic machine learning (ML) based REG algorithms perform on par with most recent neural methods~\citep{same-etal-2022-non}. However, their study has three limitations: (1) they did not incorporate pre-trained language models (PLMs); (2) they focused on the surface forms of REs, which partly depend on the performance of surface realisation; (3) they did not assess the models based on the intuition that a model with good explanatory power should be less influenced by the choice of corpus. Therefore, we adopt PLMs to the task of REG-in-context (see Section~\ref{sec:algorithm} for more details) and investigate \emph{how good is the explanatory power of PLM-based REG models compared to classic ML-based models} ($\mathcal{R}_2$) using the enhanced GREC framework.

Finally, as previously mentioned, one of the primary theoretical objectives of GREC was to computationally explore the contribution of factors that originate from linguistic studies to the choice of referential forms. It is reasonable to expect that such contributions may change depending on the choice of corpus. In this study, we conduct an importance analysis to investigate \emph{whether the importance ranking of linguistic factors changes when we use different corpora} ($\mathcal{R}_3$).

\begin{algorithm}[!t]
    \caption{\method{}}\label{alg: iterative training}
    \begin{algorithmic}
        \State Input: dataset $\gD$, oracle $\oracle$, balanced synthetic dataset size $N$
        \State $i \leftarrow 0$
        \State $\theta_i \leftarrow \argmin_\theta \gL^{\text{DM}}_\theta$ \Comment{train baseline DM, \cref{eq:score-matching}}
        \State ${s_i} \leftarrow s_{\theta_i}(\rvx_t; t)$
        \While{not done}
            \State $\synth^+_i,\synth^-_i \leftarrow$ generate samples from DM with score function ${s_i}$ and label with $\oracle$
            \While{$\min(|\synth^+_i|,|\synth^-_i|)<N$}
             \State $\synth^+,\synth^- \leftarrow$ generate more samples  from DM with score function ${s_i}$ and label with $\oracle$
             \State $\synth^+_i \leftarrow \synth^+_i \cup \synth^+$, $\synth^-_i \leftarrow \synth^-_i \cup \synth^-$
            \EndWhile
            \State $\alpha_i \leftarrow |\synth^+_i| / (|\synth^+_i| + |\synth^-_i|)$ \Comment{Estimate class prior probabilities for Bayes optimal classifier} 
            \State $\synth^+_i \leftarrow \subsample(N,\synth^+_i), \synth^-_i \leftarrow \subsample(N,\synth^-_i)$ \Comment{balance dataset for IS classifier training}
            \State $\phi_i \leftarrow \argmin_\phi \hat{\gL}^{\text{cls}}_\phi(\alpha_i, \synth^+_i, \synth^-_i)$ \Comment{train guidance classifier, \cref{eq: classifier loss}}
            \State $i \leftarrow i+1$

            \If{distill}
                \State $\psi \leftarrow \argmin_\psi \gL^{\text{dtl}}_\psi$\Comment{distill into single DM, \cref{eq: distillation}}
                \State ${s_i} \leftarrow  s_{\psi}(\rvx_t; t)$ 
            \Else  \Comment{``stack'' guidance classifiers}
                \State ${s_i} \leftarrow {s_{i-1}} + \nabla_{\rvx_t}\log C_{\phi_i}(\rvx_t; t)$ \Comment{See \cref{eq:gen-neg-score}}
            \EndIf
        \EndWhile
        \State \Return DM score function $s_i$
    \end{algorithmic}
\end{algorithm}
\section{Dataset Description}
\label{sec:dataset}


In this section, we describe our dataset. 
\dataset contains \num{1209} unique roots.
A root refers to the first element in a package name. 
For example, in the package name "com.example.mypackage", "com" is the root.

We also collected data on the number of fields in the package names of the apps in our dataset. 
A field refers to a dot-separated element in a package name.
For example, in the package name "com.example.mypackage", there are three fields: "com", "example", and "mypackage".
Results are visible in Table~\ref{table:num_of_fields}.
There are 733 package names with only one field, \num{14368} package names with two fields, etc.

\begin{table}
    \centering
    \caption{Number of package names per field in \dataset}
    \begin{adjustbox}{width=.7\columnwidth,center}
        \begin{tabular}{lr|lr}
            \hline
            Fields & Count & Fields & Count \\ \hline
            with 1 field & \num{733} & with 5 fields & \num{100}\\ 
            with 2 fields & \num{14368} & with 6 fields & \num{6}\\ 
            with 3 fields & \num{4231} & with 7 fields & \num{3}\\ 
            with 4 fields & \num{720} & with 8 fields & \num{1}\\
            \hline \hline
            \multicolumn{2}{l|}{Total} & \multicolumn{2}{r}{\libsAfterRefinement}
        \end{tabular}
    \end{adjustbox}
    \label{table:num_of_fields}
\end{table}



There are significantly fewer package names with four or more fields. 
The number of package names with one field is relatively low compared to the others.
This suggests that many package names in the dataset follow a standard naming convention with a domain name followed by one or more subpackages.
The presence of package names with four or more fields may indicate the use of more complex or specialized naming conventions\footnote{Examples of libraries are:  
\href{https://mvnrepository.com/artifact/riddley/riddley}{riddley}, 
\href{https://mvnrepository.com/artifact/jakarta.annotation}{jakarta.annotation}, 
\href{https://mvnrepository.com/artifact/com.vogle.sbpayment}{com.vogle.sbpayment}, 
\href{https://mvnrepository.com/artifact/pl.robakowski.jersey.bootstrap}{pl.robakowski.jersey.bootstrap}, 
\href{https://mvnrepository.com/search?q=de.tudresden.inf.lat.jsexp}{de.tudresden.inf.lat.jsexp}, 
\href{https://mvnrepository.com/artifact/de.hs_rm.cs.vs.tools.vocabularygenerator}{de.hs\_rm.cs.vs.tools.vocabularygenerator}, 
\href{https://mvnrepository.com/artifact/eu.adlogix.com.google.api.ads.dfp}{eu.adlogix.com.google.api.ads.dfp}, 
\href{https://mvnrepository.com/artifact/us.gov.dot.faa.ang.c55/huggs}{us.gov.dot.faa.ang.c55.gradle.huggs}.
}.


Table~\ref{table:top_ten} presents the top 10 most frequent roots and the top 10 most frequent fields found. 
In the first two columns, we can see that the root "com" is by far the most frequent, with more than \num{10000} occurrences. 
The second most frequent root is "net", with \num{1265} occurrences. 
In the second two columns, which represents the most frequent fields, including the roots, we can see that the field "com" is still the most frequent. 
The second two columns do not differ much from the first two columns, except for the two fields "gradle" and "android" that now appear.
This could indicate that Android libraries are prevalent in the dataset.
It is confirmed in the last two columns, which represent the most frequent fields without the roots. 
After "gradle" and "android", the third most frequent field is "sdk", with 138 occurrences.
We see a shift in the most prevalent fields. 
Instead of roots, we now see fields such as "sdk", "maven", "plugin(s)", "api", "tools", and "common".
This may be indicative of the types of libraries.
Overall, the results suggest that most package names are from the "com" domain and that Android libraries are well represented.


\begin{table}
    \centering
    \caption{Top 10 roots and fields present in \dataset}
    \begin{adjustbox}{width=.9\columnwidth,center}
        \begin{tabular}{c|c|c|c|c|c}
            \hline
            \multicolumn{2}{c|}{\textbf{Top 10 roots}} & \multicolumn{2}{c|}{\textbf{Top 10 most used fields}} & \multicolumn{2}{c}{\textbf{Top 10 most used fields w/o roots}}\\ \hline
            \textbf{Root} & \textbf{Count} & \textbf{Field} & \textbf{Count} & \textbf{Field} & \textbf{Count} \\ \hline \hline
            com & \num{10520} & com & \num{10551} & gradle & \num{250} \\ \hline
            net & \num{1265} & net & \num{1273} & android & \num{239} \\ \hline
            de & \num{917} & de & \num{918} & sdk & \num{138} \\ \hline
            cn & \num{663} & cn & \num{663} & plugin & \num{120} \\ \hline
            dev & \num{458} & dev & \num{465} & maven & \num{108} \\ \hline
            me & \num{411} & me & \num{413} & plugins & \num{93} \\ \hline
            eu & \num{223} & gradle & \num{254} & api & \num{60} \\ \hline
            ru & \num{202} & android & \num{240} & oss & \num{57} \\ \hline
            fr & \num{188} & eu & \num{224} & tools & \num{54} \\ \hline
            ch & \num{181} & ru & \num{203} & common & \num{39} \\ \hline
        \end{tabular}
    \end{adjustbox}
    \label{table:top_ten}
\end{table}
\section{Evaluation} \label{sec:evaluation}

\begin{table*}[tbp]
\centering
\small
\begin{tabular}{cccccccccc}
\toprule
& \multicolumn{3}{c}{\msr} & \multicolumn{3}{c}{\negc} & \multicolumn{3}{c}{\wsj} \\
& Acc. & F1 & wF1 & Acc. & F1 & wF1 & Acc. & F1 & wF1 \\ \cmidrule(lr){2-4} \cmidrule(lr){5-7} \cmidrule(lr){8-10} 
\udel & 66.86 & 56.76 & 64.3 & \textbf{80.80} & 55.45 & 77.9 & 63.74 & 64.23 & 63.2 \\
\icsi & \underline{71.19} & 64.73 & 70.4 & 80.36 & 64.53 & \underline{78.6} & 64.62 & 64.15 & 63.4 \\
\cnts & 68.59 & 61.39 & 67.2 & 78.68 & 61.62 & 76.8 & 64.31 & 64.59 & 64.4 \\
\osu & 68.02 & 60.28 & 66.6 & 79.24 & 57.04 & 76.5 & 69.20 & 69.63 & 68.9 \\
\isg & 67.05 & 58.83 & 65.3 & 77.34 & 59.52 & 75.6 & 69.15 & 69.35 & 69.2 \\ \midrule
\bert & \textbf{71.68} & \underline{66.70} & \textbf{71.4} & 77.79 & \underline{72.87} & 77.7 & \underline{80.95} & \underline{80.93} & \underline{80.9} \\
\roberta & 70.91 & \textbf{67.53} & \underline{70.7} & \textbf{80.80} & \textbf{77.29} & \textbf{80.7} & \textbf{82.61} & \textbf{82.70} & \textbf{82.6} \\ \midrule
Average & 69.19 & 62.32 & 67.99 & 79.29 & 64.05 & 77.69 & 70.65 & 70.80 & 70.37 \\
\bottomrule
\end{tabular}
\caption{\label{tab:performance} Overall accuracy (Acc.), macro-averaged F1 (F1), and weighted-macro F1 (wF1) scores of the algorithms depicted in Section~\ref{sec:algorithm}. For instance, \msr-\udel refers to a C5.0 classifier trained on the \msr~corpus, using the feature set mentioned in \citet{greenbacker-mccoy-2009-udel}.}
%Its Acc., F1 and wF1 of this model are 66.86, 56.76, and 64.3, respectively.}
\end{table*}


In this section, we introduce the evaluation protocol and report the performance of the models.

\subsection{Implementation Details} \label{sec:implementation}

For \bert and \roberta, we used \textit{bert-base-cased} and \textit{roberta-base}, both from Hugging Face. For fine-tuning, we set the batch size to 16, the learning rate to 1e-3, the dropout rate to 0.5, and the size of the output layer to 256. We ran each model for 20 epochs and used the one that achieved the highest F1 score on the development set. The implementation details of the classic ML-based models can be found in Appendix~\ref{sec:appendixML}.

\subsection{Evaluation Protocol} \label{sec:protocol}

The main evaluation metric in the GREC-MSR shared tasks was accuracy. 
In addition to accuracy, we also report macro-F1 and weighted-macro F1. We argue that different metrics evaluate algorithms from different perspectives and provide us with different meaningful insights. 
For pragmatic tasks like REG, it makes sense to ask how well an algorithm performs on naturally distributed data which is often imbalanced. For these cases, reporting accuracy and weighted F1 are logical. 
Furthermore, analogous to other classification tasks, minority categories should not be overlooked. Take as an example the class \emph{description} in the \negc corpus, which occurs only 4\%. If a model fails to produce this class, the produced document might sound unnatural. Therefore, it is important to ensure that an algorithm is not over- or under-generating certain classes. Looking into accuracy and macro-F1 together provides insights into such cases.

\subsection{Performance of the Models}\label{subsec:overallacc}

The overall accuracy of the models, their macro F1, and their weighted-macro F1 are presented in Table \ref{tab:performance}. 
We also present the ranking of the models based on these scores in Appendix~\ref{sec:app_rank}. 


\paragraph{PLM-based Models.} The best-performing models across all corpora and metrics are PLM-based models.  In six out of nine rankings, \bert and \roberta are ranked as the top two models. The sole exception is \negc, where \bert is the second worst model. The benefit of using PLMs is the largest on the \wsj corpus. For example, \roberta improves the macro F1 score from 69.63 (i.e., the performance of the best ML-based model) to 82.70.


\paragraph{ML-based Models.} In contrast to the robust performance of the PLM models, the performance of the classic ML models is more corpus-dependent. In the case of \msr and \negc, \icsi is the best-performing model, while in the case of \wsj, it is at the bottom section of the rankings. Another interesting observation is the performance of the \udel models. In terms of accuracy, \udel has the highest performance in \negc, while it has the lowest performance in both \msr and \wsj. In terms of macro-F1 rankings, the \negc \udel model dropped from first to last place, whereas \bert improved from penultimate place to second place. In general, our ML models yielded lower scores than the original models used in the GREC study \citep{belz2009generating}. This could be attributed to a variety of factors, including differences in feature engineering and model parameters.

\paragraph{Comparing Different Metrics.} 

Upon comparing average scores across the three metrics, we observe that for \msr and \negc, PLMs are clear winners only when macro-F1 is the metric in question. However, for \wsj, PLMs are winners on all three metrics. This may be because the distribution of categories in \wsj is much more balanced than in the other two corpora.
\subsection{Performance Indicators}\label{subsection:performance_indicators_definition}
We use three categories of indicators to analyze the performance of a scenario definition:

\textbf{Inefficiency Rate}: it is the difference in fuel spent between a traffic scenario in which the aircraft comply to DAA resolution advisories (RA), and the analogous scenario in which each vehicle follows its optimal path, without observation of traffic separation rules. In case of vertical deviations, a higher fuel rate is required for climb maneuvers, a lower fuel rate in descent maneuver, which increase the net total for the mission. The resulting value of this indicator is the average value over all traffic configurations in a scenario set. 

\textbf{Loss of Separation (LoS) Rate}: despite there being several ways of defining traffic separation, we examine just the simplest one, which is checking whether or not the vehicles are separated by at least a fixed \emph{minimum distance}. In order to include both DAIDALUS and ACAS sXu in the same tables, we use one separation distance from each one, respectively: 4,000 ft, which is the Horizontal Miss Distance, or HMD, used in DO-365B \cite{DO_365} to define the so-called \emph{Hazard Alert Zone} (HAZ), associated to the DAA Well-Clear (DWC) concept of separation; and 2,000 ft, used in DO-396 \cite{DO_396}, that defines the Loss of Well-Clear (LoWC) event in relation to large UAVs or manned aircraft. These indicators will denote the rate of scenarios, in a scenario set, where the distance between any aircraft pair fell below the afore mentioned threshold values. 

\textbf{Timeout Rate}: this indicates, in a scenario set, the rate of scenarios where any aircraft exceeded a maximum time without reaching its destination point. As pointed out in section \ref{section:scenario_definitions}, this phenomenon occurs because DAA Resolution Advisories (RAs) cause long chains of maneuvers that extend beyond the energy/fuel allowance of the vehicle, due to shortcomings in coordination. We use a time threshold of 1,000 seconds but, in practice, the timeout would be determined by the energy/fuel capacity of the vehicle.

\textbf{Scenario Computing Time}: the time needed to simulate a single scenario instance, in a single core of an Intel Xeon CPU, discounted the fact that multiple scenario instances can be run in parallel in a multi-core CPU. In our simulated scenarios, the DAA algorithm is called at least each 2 seconds, for each aircraft, but when the aircraft is in avoidance mode, that can happen more often. In the case of ACAS sXu, the requirement of receiving various messages to update a single track contribute to result in multiple calls per simulated second.


\subsection{Scenario Specifications and Labels}
In this study, a scenario specification is defined by features such as: the DAA algorithm used, the dimensionality (2-D or 3-D), if it uses extrinsic priorities or not, the target separation parameter, and possibly other features. The scenario labels used in this section encode these attributes:
\begin{itemize}
\item \texttt{dai\_ip\_2d\_4k}: DAIDALUS without extrinsic priorities, 2-D maneuvering and regular 4 kft Horizontal Miss Distance (HMD);
\item \texttt{dai\_ep\_2d\_4k}: similar to the above, with extrinsic priorities;
\item \texttt{sxu\_ip\_2d\_2k}: ACAS sXu with intrinsic priorities only, 2-D maneuvering and regular 2 kft LoWC threshold;
\item \texttt{sxu\_ep\_2d\_2k}: similar to the above, with extrinsic priorities;
\item \texttt{dai\_ep\_3d\_4k}: similar to \texttt{dai\_ep\_2d\_4k}, with 3-D maneuvering;
\item \texttt{dai\_ep\_2d\_2k}: similar to \texttt{dai\_ep\_2d\_4k}, with HMD reduced to 2 kft.
\end{itemize}

And there are other features and labels that will be mentioned below as needed.

\subsection{Performance Analysis}
The analysis of selected scenario specifications is summarized in table~\ref{table:performance_analysis}. The first notorious observation in this table is the effect of extrinsic priorities to decrease inefficiency. With regards to safety indicators, their effect is mixed, and we have to observe each case separately. In the case of DAIDALUS, priorities decreased the 2 kft LoS rate and, most drastically, the timeout rate, while increased the 4 kft LoS rate. In the case of ACAS sXu, priorities increased both LoS rate indicators, but decreased the timeout rate drastically. Based on our rule of thumb assessment, it can be said that DAIDALUS works better with extrinsic priorities, while ACAS sXu works better without them. We conjecture that the following reasons explain this fact: i) that DAIDALUS has more built-in symmetries than ACAS sXu; ii) that ACAS sXu has already built-in priority rules for multi-aircraft encounters, and extrinsic priorities may contradict with them.

\begin{table}[h]
    \caption{Summary of closed-loop performance indicators per scenario.}
    \label{table:performance_analysis}
    \begin{center}
    \begin{tabularx}{\columnwidth}{|p{0.20\columnwidth}|p{0.11\columnwidth}|p{0.09\columnwidth}|p{0.09\columnwidth}|p{0.1\columnwidth}|p{0.105\columnwidth}|}
        \hline
    Scenario spec. & Ineffici-ency rate & LoS rate 4~kft & LoS rate 2~kft & Timeout rate & Scenario comp. time (s)\\
    \hline
    \texttt{dai\_ip\_2d\_4k} & 9.71\% & 1.4E-2 & 6.5E-5 & 1.6E-2 & 6.5E-2\\
    \texttt{dai\_ep\_2d\_4k} & 4.83\% & 2.4E-2 & 4.1E-5 & 0 & 5.1E-2\\
    \texttt{sxu\_ip\_2d\_2k} & 20.7\% & 8.7E-1 & 4.1E-2 & 1.6E-3 & 7.8E+1\\
    \texttt{sxu\_ep\_2d\_2k} & 10.9\% & 9.0E-1 & 2.0E-1 & 1.8E-5 & 7.8E+1\\
    \texttt{dai\_ep\_3d\_4k} & 4.38\% & 8.9E-6 & 0 & 0 & 7.4E-2\\
    \texttt{dai\_ep\_2d\_2k} & 1.3\% & 9.0E-1 & 1.1E-1 & 0 & 3.9E-2\\
    \hline
    \end{tabularx}
    \end{center}
\end{table}

According to a line of reasoning, it would be expected, that, in the more efficient scenarios, the aircraft fly closer to each other and, therefore, there should be a higher probability of losing separation. But this is not the only principle at play, because, if the aircraft perform deviations with the least extra distance, while keeping separation, they stay less in the air and decrease the total number of conflicts. This becomes more understandable when we compare \texttt{dai\_ep\_2d\_4k} with \texttt{dai\_ep\_3d\_4k}, where the latter achieved a small advantage in efficiency, but a huge one in safety. \texttt{dai\_ep\_3d\_4k} is capable of shortening the total distances, but has a residual cost associated to vertical maneuvers, where the climb maneuvers spend fuel at higher rates. 

It cannot escape from observation that DAIDALUS performed much better than ACAS sXu in almost all indicators. In our opinion, it would be reasonable to expect that ACAS sXu would not excel in the LoS rates, especially that of 4 kft, because its first protection criterion is 2 kft, as a built-in feature. However, with smaller protection volumes, the deviations should be smaller and, by this reasoning, its expected inefficiency would be lower than that of DAIDALUS. But our results show otherwise when we compare the cases of DAIDALUS with those of ACAS sXu. The only case in which ACAS sXu obtained an advantage was for the LoS rates comparison between \texttt{sxu\_ip\_2d\_2k} and \texttt{dai\_ep\_2d\_2k}, which have the same separation target. In any case, the Los rate obtained for ACAS sXu meets the performance requirement of ASTM F3442 \cite{ASTM}, which uses the definition of LoWC Ratio (LR), which is the ratio between the LoS rate of 2 kft shown in table~\ref{table:performance_analysis}, with DAA active, and corresponding LoS rate with DAA inactive, the latter being 0.785 according  to our simulations. Thus, the resulting LR scores for ACAS sXu here are 0.052 and 0.253, respectively for the two ACAS sXu specs, which are well below the value of 0.4 from \cite{ASTM}, and consistent with the performance analysis of \cite{DO_396}. We conjecture that these scores would be lower in a future 3-D scenario spec of a ACAS sXu, by following the same improvement obtained with DAIDALUS. 

\subsection{Possible approximations to the closed-loop behavior}
Trying to alleviate the heavy computational load to simulate large numbers of different traffic configurations, in this multi-aircraft, closed-loop setup, we considered some approximated solutions, such as the use of Deep Neural Networks to emulate the ACAS Xu/sXu behavior, in the lines followed by \cite{Julian2018,Bak2022}. However, the existing solutions that we found available were developed for just one intruder aircraft, so they were not suitable for our study. Another possibility would be the exploitation of symmetry transformations \cite{Sibai2020}, however the history-dependent nature of the ACAS Xu/sXu algorithms, associated to the present closed-loop setup, make this possibility unpractical. Thus, we started exploring simpler ways of deducing closed-loop behavior without having to perform the full simulation of a scenario. So far, we tried to analyze correlations between measures of open-loop maneuvers and the closed loop performance. The features that we explored are:
\begin{itemize}
    \item \textbf{Distance flown until the end of the first deviation maneuver} ($\overline{M/D}$): we consider the total distance flown until a ``Clear-of-Conflict'' (CoC) event happens, that is, after one or more divergent maneuvers start in a scenario instance, we stop the scenario when the first divergent maneuver of any aircraft finishes and that individual aircraft is clear of conflict, the moment from which some decision must be made on how to continue the mission. We count the total number of maneuvers started, and divide it by the sum of the flown distances, across all scenario instances in an execution set associated to a scenario spec.
    \item \textbf{Average angle deviation maneuver} ($\overline{\alpha}$): using the same stopping rule of above, we account the angle difference between the heading angles of the aircraft at the beginning of the divergent maneuver and at the stopping moment. 
\end{itemize}
For each of these measures, we ran the 122,416 traffic configuration instances with the open-loop stopping rule. Here, all the scenario specifications are 2-dimensional, and we use abbreviated lables to achieve a better display in the graph legends. Namely, the scenario specifications in this subsection are defined as:
\begin{itemize}
    \item \texttt{D1}: DAIDALUS without extrinsic priorities and with deterministic sensor data;
    \item \texttt{D2}: DAIDALUS with extrinsic priorities and deterministic sensor data;
    \item \texttt{D3}: DAIDALUS with extrinsic priorities and Sensor Uncertainty Mitigation (SUM). This is a design feature \cite{Narkawicz2018} to mitigate uncertainty in sensor data, as for example, to determine the position of an intruder aircraft;
    \item \texttt{D4}: DAIDALUS with its Horizontal Miss Distance (HMD) set to 200 ft (the standard is 4,000 ft) and uncertain sensor data;
    \item \texttt{X1}: ACAS sXu without extrinsic priorities;
    \item \texttt{X2}: ACAS sXu with extrinsic priorities;
    \item \texttt{X3}: ACAS sXu with scenario downscaled to speed of 43 knots and cell radius of 1 km.
\end{itemize}
We used the values of $\overline{M/D}$ and $\overline{\alpha}$ obtained for each of the specs above as inputs to a linear regressor of inefficiency, as defined in section~\ref{subsection:performance_indicators_definition}, and generated a plot with the pairs of (true, predicted) values from this regression, as shown in fig.~\ref{fig:inefficiency_regression}. The trend line in the figure, which depicts the regressor, seems to represent a strong correlation, which is confirmed by the value of $R^2$. When considering each of the regression inputs separately, we obtain $R^2=0.73$ for $\overline{M/D}$ and $R^2=0.77$ for $\overline{\alpha}$, which show that they contribute with approximately equal predicting power. 
% Figure environment removed

We performed a similar analysis for the LoS indicators, but we found very little correlation, as $R^2=0.11$ for the 2~kft LoS indicator. Nevertheless, it can be concluded that these open-loop measurements are a good proxy for the closed-loop inefficiency, with the advantage that the predictor discounts the bias that may have been introduced by the closed-loop mission management system, which is not part of the DAA specification. A rough estimate for the computing time saved is of 68\% in the inefficiency case.      

\section{Discussion}
\label{sec: discussion}
\kmsdelete{In this work} We study \kmsreplace{Fairness-Aware PAC learning}{Fair-ERM} in the malicious noise model, and  in some cases allow 
the learner to maintain optimal overall accuracy despite the signal in Group $B$ being almost entirely washed out.
%when we allow learners to use the
%$\PQ$ randomized expansion of the hypothesis class $\mathcal{H}$
In particular we show that different fairness constraints have fundamentally different behavior in the presence of Malicious Noise, in terms of the amount of accuracy loss that a given level of Malicious Noise could cause a fairness-constrained learner to incur. 
The key to achieving our results, which are more optimistic than those in \cite{lampert}, is allowing for improper learners using the (P,Q)-randomized expansions of the given class $\mathcal{H}$.
%We \kmsreplace{present a picture of the}{prove upper and lower bounds on}
%accuracy loss for a range of fairness notions, given \kmsreplace{this simple randomization step.}{learning over $\PQ$.
%In general our results indicate Fair-ERM (given learning over $\PQ$) is more robust than claimed in \cite{lampert}.
The type of smoothness we create by using $\PQ$ seems to be a natural property that is likely shared by many natural hypothesis classes.

Fairness notions are motivated as a response to learned disparities when there is \kmsdelete{data corruption or} systemic error affecting \kmsdelete{the data for}
one group. 
Fairness notions are supposed to mitigate this by ruling out classifiers that have worse performance on a sub-group. 
This can peg both classifiers at a lower level of performance \kmsdelete{(e.g that the lower subgroup)} in order to \emph{motivate} \cite{hardt16} improving the data collection or labelling process to obtain more reliable performance. 
%So in \kmsreplace{some}{a} sense, sensitivity of the fairness notion to poor sub-group performance caused by malicious noise is the \textit{point} of fairness constraints! 
However, it also desirable that fairness constraints perform gracefully when subject to Malicious Noise because fairness constraints will be used in contexts where the data is unreliable and noisy and this might not be known to the learner.
This tension, exposed by our work, motivates 
%a revisiting of fairness notions from first principles approach and trying to axiomatize the 
%desired properties of a fairness intervention a la cryptography and privacy. \footnote{Work in multi-calibration \cite{multicalib} is a viable direction for this problem but it is unclear how 
%that and related notions behave with unreliable data. }
on going work studying the sensitivity level of fairness constraints. 
%If we we are to take a view, if a classifier is deployed 


\paragraph*{\textit{Ethics Statement:}}

Regarding potential biases, in addition to the biases present in text-based datasets, biases can also be introduced by the pre-trained language models~\cite{bender2021dangers} used in this work. In other words, the REG algorithms we developed in this study may make different predictions with respect to different genders, for instance. In the future, we plan to investigate this phenomenon and find ways to mitigate it.

\paragraph*{\textit{Supplementary Materials Availability Statement:}}
All associated data, source code, output files, scripts, documentation, and other relevant material to this paper are publicly available and can be accessed on our GitHub repository: \url{https://github.com/fsame/REG_GREC-WSJ}, \href{https://doi.org/10.5281/zenodo.8182689}{DOI: 10.5281/zenodo.8182689}.

\paragraph*{\textit{Acknowledgements:}}
We thank the anonymous reviewers for their helpful comments.
Fahime Same is supported by the German Research Foundation (DFG)– Project-ID 281511265 – SFB 1252 ``Prominence in Language”.
\bibliography{anthology,custom}
\bibliographystyle{acl_natbib}

\newpage
\appendix
\begin{comment}
\section{System Architecture}
\label{appendix:architecture}
\system has a novel modularized system architecture with three key components: 
\emph{StreamManager}, 
\emph{TxnManager} and \emph{TxnScheduler}. 
These components are instantiated in each thread locally.
The execution outline of \system is presented in Algorithm~\ref{alg:algo}.
Transactional stream processing is continuous and potentially never ends (Line 1$\sim$8).
The dependency resolution and execution of state transactions are separated into two non-overlapping phases by punctuations~\cite{Tucker:2003:EPS:776752.776780} (Line 2 and 5), which guarantees that no subsequent input event will have a smaller timestamp. 
Effectively, a batch of state transactions is collected during the first phase, and processed during the second phase.

In the first phase (i.e., stream processing phase), 
the \emph{StreamManager} conducts preprocessing for every input event ($e$). Similar to some prior works~\cite{tstream}, state transactions may be issued but not immediately processed during preprocessing (Line 3).
The \emph{pre\_processing} and \emph{post\_processing} functions are exposed as APIs to users.
The \emph{TxnManager} handles dependency resolution (Line 4) among state transactions and insert decomposed operations to construct a \tpg. We discuss the detailed two-phase \tpg construction process in Section~\ref{subsec:construction}.

In the second phase  (i.e., transaction processing phase), 
the \emph{TxnManager} is first involved again to refine (Line 6) the constructed \tpg with further dependency resolution.
The \emph{TxnScheduler} 
schedules operations for concurrent execution based on the constructed \tpg according to the three dimensions of scheduling decisions (Line 7). 
In particular, a scheduling decision model $M$ is instantiated based on the constructed \tpg (Line 14).
\textbf{\circled{1}} Guided by $M$, execution threads adopt an exploration strategy (Section~\ref{subsec:explore}) to explore the constructed \tpg for operations available to be scheduled constrained by dependencies. 
\textbf{\circled{2}} 
During exploration, one or multiple operations may be treated as the 
% basic 
unit of scheduling (Section~\ref{subsec:granularity}). 
Subsequently, \textbf{\circled{3}} every thread executes operation(s) in the unit of scheduling with various abort handling mechanisms (Section~\ref{subsec:abort_handling}).
Only when state transactions are processed (i.e., committed or aborted) can the associated input events be postprocessed (Line 8) by the \emph{StreamManager} based on transaction processing results.
\end{comment}

\begin{comment}
\begin{algorithm}
\footnotesize
    \KwData{$e$ \tcp{Input event}}
    \KwData{$txn_{ts}$ \tcp{State transaction}}
    \KwData{$G$ \tcp{The currently constructed TPG}}
    \While{!finish processing of input streams}{
        \eIf(\tcp*[h]{Phase 1}){\text{$e$ is not a $punctuation$}}{
                $txn_{ts}$ $\gets$ PRE\_Processing($e$)\;
                \textbf{TPG\_Construction}($G$, $txn_{ts}$)\; 
          }(\tcp*[h]{Phase 2}){
                \textbf{TPG\_Refinement}($G$)\; 
                \textbf{TXN\_Scheduling}($G$)\; 
                POST\_Processing()\;
          }
    }
    
    \SetKwFunction{FMain}{TPG\_Construction}
    \SetKwProg{Fn}{Function}{:}{}
    \Fn{\FMain{$G$, $txn_{ts}$}}{
        $O_{1..k}$ $\gets$ \textbf{Partition} $txn_{ts}$\;
        \ForEach{\text{operation $O_{i}$ $\in$ $O_{1..k}$}}{
            \textbf{Identify} its \ld\;
            $G$ $\gets$ $G$ + $O_{i}$ \;
        }
    }
    \SetKwFunction{FMain}{TPG\_Refinement}
    \SetKwProg{Fn}{Function}{:}{}
    \Fn{\FMain{$G$}}{
        \ForEach{\text{vertex $e_{i}$ $\in$ $G$}}{
            \textbf{Identify} its \td, \pd\;
        }
    }
    
    \SetKwFunction{FMain}{TXN\_Scheduling}
    \SetKwProg{Fn}{Function}{:}{}
    \Fn{\FMain{$G$}}{
        $M$ $\gets$ Instantiated with $G$;\tcp{A decision model}
        \While{!finish scheduling of $G$
        }{
          \textbf{\circled{2}} $Scheduling Unit$ $\gets$ \textbf{\circled{1}} \emph{Explore}($G$, $M$)\; 
            \textbf{\circled{3}} \emph{Execute with Abort Handling} ($Scheduling Unit$)\; 
        }
    }
  \caption{Execution Outline of \system}
  \label{alg:algo}
\end{algorithm}
\end{comment}

\end{document}
