\section{SRIM simulation conditions and data handling} \label{SRIMcondition}


\begin{table} [tb!]
	\centering
	\begin{tabular}{cccccc}
		\hline\hline
		Ion & H & He & Al & Fe & Au\\
		\hline
		$E_0$ (MeV) & 1 & 1 & 3 & 5 & 10\\ 
		\hline\hline	
	\end{tabular}
	\caption{Projectile ions and corresponding incident energies $E_0$.}
	\label{tab:ions}
\end{table}

\begin{table} [tb!]
	\centering
	\begin{tabular}{ccccccc}
		\hline\hline
		Target & Fe & Ni & Cu & Pd & W & Pt\\
		\hline
		$E_d$ (eV) \cite{Nordlund-2015-ID597} & 40 & 40 & 29 & 41 & 90 & 44 \\
		$b$ \cite{Nordlund-2018-Improvingatomicdis} & -0.568 & -1.01 & -0.68 & -0.88 & -0.56 & -1.12\\ 
		$c$ \cite{Nordlund-2018-Improvingatomicdis} & 0.286 & 0.23 & 0.16 & 0.15 & 0.12 & 0.11\\
		\hline\hline	
	\end{tabular}
	\caption{Displacement threshold, $E_d$, and arc-dpa model parameters, $(b, c)$, of simulated targets.}
	\label{tab:parameters}
\end{table} 

\renewcommand{\arraystretch}{1.75}

\begin{table*} [tb!]
	\centering
	\begin{threeparttable}[t]
		\begin{tabular}{lccc}
			\hline\hline
			Quantity & Symbol 
			& \parbox[t][1cm]{3cm}{
				Method 1 (M1)\\COLLISON.txt
			}
			& \parbox[t]{3cm}{ \centering
				Method 2 (M2)\\VACANCY.txt
			} 
			\\
			
			\hline
			
			PKAs per ion & $N_{\text{PKA}}$ 
			& $N_{\text{rows}} / N_{\text{ions}}$ 
			& $\sum_k{ \left[ \nu_i \right]_k\, \Delta x}$ \tnote{\dag} 
			\\
			
			\hline
			
			& &\multicolumn{2}{c}{NRT-dpa model} \\
			
			Displacements per ion & $N_{d}$ 
			& $ \langle \nu_d \rangle  \cdot N_{\text{PKA}} $
			& $\sum_k{\left[ \nu_i + \nu_r \right]_k\, \Delta x} $ \tnote{\dag} 
			\\
								
			Mean displacements per PKA & $ \langle \nu_d \rangle $ 
			& $N_{\text{rows}}^{-1}\,\sum_k{\left[ \nu_d\right]_k}$ \tnote{\ddag}
			& $ N_d / N_{\text{PKA}} $ 
			\\
			
			\hline
			
			& &\multicolumn{2}{c}{arc-dpa model} \\
			
			Displacements per ion & $N_{d,arc}$ 
			& \multicolumn{2}{c}{$\langle \nu_{d,arc} \rangle \cdot N_{\text{PKA}} $}\\
			
			Mean displacements per PKA & $ \langle \nu_{d,arc} \rangle $ 
			& $N_{\text{rows}}^{-1}\,\sum_k{
				\left[ \nu_d\right]_k \xi\left( \left[ \nu_d\right]_k \right)
				}$ \tnote{\ddag}
			& \parbox[c]{3cm}{
			\begin{center}
				eq. \eqref{eq:ndarc_approx} with $\langle \nu_d \rangle $\\as above
			\end{center}}
			\\

			\hline\hline	
		\end{tabular}
		
		
		\begin{tablenotes}
			\item [\dag]  $\left[ \nu_i \right]_k$ and $\left[ \nu_r \right]_k$ are the ``vacancies by ions'' 
			              and ``vacancies by recoils'', respectively, in the $k$-th target depth bin, with  
						  $\Delta x$ denoting the bin width.
			\item [\ddag]  $\left[ \nu_d \right]_k$ denotes the number of vacancies 
			               estimated by SRIM for the $k$-th PKA event. The sum is over all events.
			
		\end{tablenotes}
	\end{threeparttable}
	\caption{Calculation of damage parameters from SRIM output files}
	\label{tab:quantities}
\end{table*} 
\renewcommand{\arraystretch}{1.0}

All simulations were performed utilizing SRIM-2013 and employing the option "Ion distribution and Quick calculation of damage" (Q-C). Lattice and surface binding energies were set equal to zero according to the recommendation in \cite{Stoller-2013-ID110}. 
A range of projectile ions were employed, with atomic numbers varying from $Z=1$ (H) to 79 (Au)
and energies ranging from $E_0=1$ to 10 MeV, similarly to the work of \citet{Agarwal-2021-OntheuseofSRIMf}. The ions and corresponding energies are listed in
Table \ref{tab:ions}. Table \ref{tab:parameters} shows all the targets that we tested, which are essentially all materials whose arc-dpa parameters were estimated in \cite{Nordlund-2018-Improvingatomicdis}. Target
thickness was chosen appropriately in order to ensure that
the impinging ions stop within the examined
region. 
The target displacement energies, $E_d$, are based on internationally recommended standard values and are also given in Table \ref{tab:parameters}.
In the case of Fe self-ion irradiation, an extra simulation with $E_0=78.7$~keV was also performed in order to directly compare with results from \cite{Nordlund-2015-ID597}. For each ion/target combination 10,000 ion histories were run.

Damage parameters were extracted from the SRIM output files, either VACANCY.txt or COLLISON.txt. Table \ref{tab:quantities} lists all quantities of interest and the way they are calculated depending on the damage model and the output file used. 

The number of PKAs per ion, $N_{\text{PKA}}$, is obtained by integrating the 2nd data column of VACANCY.txt (``vacancies by ions'',  $\nu_i$) or by dividing the number of data rows, $N_{\text{rows}}$, in COLLISON.txt by the number of simulated ions, $N_{\text{ions}}$. $N_{\text{PKA}}$ is independent of the damage model. 

The NRT displacements per ion, $N_d$, is obtained as follows. In the case of VACANCY.txt, $N_d$ is found by summing the $2^{nd}$ and $3^{rd}$ column of the data table, i.e., ``vacancies by ions'', $\nu_i$, and "vacancies by recoils", $\nu_r$, respectively. 
Regarding the COLLISON.txt file, $N_d$ is calculated by adding up the "Target vacancies",  $\nu_d$, of all PKAs and dividing by $N_{\text{ions}}$. Finally, the average displacements per PKA, $\langle \nu_d \rangle$, is equal to $N_d / N_{\text{PKA}}$. 

The calculation of arc-dpa damage parameters is described in the next section.

All evaluations and the parsing of SRIM output files were performed in the OCTAVE computing environment \cite{Eaton2022}. The open source python code PYSRIM \cite{pysrim} was employed to automate the SRIM calculations. All relevant data and code are available in \cite{mitsi_e_2023_8116031}.

\section{Methods and Results}
In this section, we present the two different methods to obtain arc-dpa damage parameters from SRIM output.

\subsection{Method 1 (M1)}
This method utilizes the COLLISON.txt output file. In SRIM Q-C mode, this file lists all simulated PKA scattering events and reports, among other data, the number of displacements, $\nu_d$, generated per event. These $\nu_d$ values, labelled "Target vacancies", are calculated according to the NRT model, eq. \eqref{eq:NRT}, with the damage energy, $T_d$, obtained from the approximate LSS theory \cite{Ziegler-2008-ID880}. For the $\nu_d$ values in COLLISON.txt that satisfy $\nu_d > 1$, we can easily recover the LSS damage energy by multiplying $\nunrt$ with the cascade multiplication factor, $L$ (cf. eq. \eqref{eq:NRT}). Then, the obtained $T_d$ can be used in eq. \eqref{eq:arc} to evaluate the displacements according to the arc-dpa model. This is essentially what is done in M1, however, instead of actually evaluating $T_d$ we employ $\nunrt$ directly in the equivalent arc-dpa definition, eq. \eqref{eq:arc2}. Thus, the steps to calculate the arc-dpa parameters are as follows: 
\begin{enumerate}
	\item Run SRIM with the "Quick calculation of damage" (Q-C) option.
	\item Parse the COLLISON.txt output file to obtain the NRT displacements per PKA event, $\nunrt$.
	\item Calculate the corresponding $\nuarc$ per PKA from eq. \eqref{eq:arc2}, $\nuarc = \nunrt\cdot \xi(\nunrt)$.
	\item Take the average of the $\nuarc$ values to obtain the mean displacements per PKA according to the arc-dpa model, $\langle\nuarc\rangle$ (cf. Table \ref{tab:quantities}).
	\item Multiply by the number of PKAs per ion, $N_{PKA}$, to obtain the number of displacements per ion, $N_{d,arc} = \langle\nuarc\rangle \cdot N_{\text{PKA}}$
\end{enumerate}

M1 is very similar to the method proposed by \citet{Nordlund-2015-ID597}. The main difference lies in the derivation of damage energy. In \cite{Nordlund-2015-ID597}, $T_d$ is obtained by separate SRIM simulations employing the "Detailed Calculation with Full Damage Cascades" (F-C) option. In this case, SRIM utilizes detailed stopping power calculations for all secondary recoils in the PKA cascade, thus, the value of $T_d$ is potentially more accurate. \citet{Agarwal-2021-OntheuseofSRIMf} have made a detailed comparison of SRIM damage calculations in Q-C and F-C modes. They found that there is a difference of up to $\pm 25\%$ in the amount of NRT vacancies predicted by the two modes, when vacancy production is estimated by the SRIM damage energy. The authors attributed the difference to the use of the LSS approximation in Q-C mode. It is expected that also in the present case, where the arc-dpa damage estimation in M1 is based on the Q-C damage energy, there will be similar differences with respect to the procedure described in \cite{Nordlund-2015-ID597}, where the F-C mode was employed.  

To make a quantitative comparison of the two approaches, we repeated the simulation of 78.7 keV Fe ions incident on an Fe target that was employed in \cite{Nordlund-2015-ID597}. 
Table \ref{tab:comparison} shows the results from the two approaches. It is seen that there is only a small 2\% difference in the NRT parameters, $N_d$ and $\langle\nunrt\rangle$, obtained with the present method in comparison to the values reported in \cite{Nordlund-2015-ID597}, while the corresponding arc-dpa parameters almost coincide. We attribute the good agreement to the low damage energies occurring in this simulation.  
To have a more meaningful comparison, we simulated self-ion Fe irradiation with a much higher projectile energy, $E_0=5$~MeV, and evaluated the results with both our proposed method M1 and the one by \citet{Nordlund-2015-ID597}.  
In the latter case, we used the data from their fig. 1.2 to extend the interpolation of $T_d$ to target recoil energies up to 10~MeV. 
The results are also listed in Table \ref{tab:comparison}. As seen from the table, there is a 10\% difference between the NRT parameters obtained by our M1 and the evaluation according to \cite{Nordlund-2015-ID597}. This difference is comparable to the observations of \cite{Agarwal-2021-OntheuseofSRIMf} and thus can be attributed to the use of approximate LSS damage energy in the Q-C simulation mode. The corresponding arc-dpa parameters exhibit a similar but slightly lower difference of about 8\%. This is due to the fact that the arc-dpa efficiency lowers the significance of high energy damage events, where the errors due to the LSS approximation are more pronounced.

\begin{table*} [tb!]
	\centering
	\begin{threeparttable}[t]
	\begin{tabular}{lcccccc}
		\hline\hline
		{} & $E_0$ & $N_{\text{PKA}}$ & $N_{d}$ & $\langle\nunrt\rangle$ 
		& $N_{d, arc}$  &$\langle\nuarc\rangle$ \\
		\hline
		\citet{Nordlund-2015-ID597} & \multirow{3}{*}{78.7 keV} & \multirow{3}{*}{44.1} & 
		539 & 12.2\tnote{\dag} & 217 & 4.93\tnote{\dag} \\
		This study - Method 1 & {} & {} & 
		530 & 12.0 & 217 & 4.92 \\ 
		This study - Method 2 & {} & {} & 
		530 & 12.0 & 209 & 4.74 \\
		\hline
		Method of \citet{Nordlund-2015-ID597} & \multirow{3}{*}{5 MeV} & \multirow{3}{*}{442} & 
		8800 & 20.0 & 3150 & 7.14 \\
		This study - Method 1 & {} & {} & 
		7870 & 17.9 & 2900 & 6.56 \\ 
		This study - Method 2 & {} & {} & 
		7870 & 17.9 & 2890 & 6.54 \\
		\hline\hline	
	\end{tabular}

	\begin{tablenotes}
		\item [\dag] Mean values are calculated by dividing $N_d$ and $N_{d,arc}$ from \cite{Nordlund-2015-ID597} by $N_{\text{PKA}}$ as obtained in the present study.
		 
	\end{tablenotes}
	\end{threeparttable}

	\caption{Damage parameters obtained by different methods for the irradiation of an Fe target with Fe ions of energy $E_0$.}
	\label{tab:comparison}
\end{table*}



\subsection*{Method 2 (M2)}

% Figure environment removed

The objective of M2 is to provide a quick estimate of the arc-dpa damage parameters, without having to resort to the cumbersome processing of COLLISON.txt. For this, we note that from eq. \eqref{eq:arc2} the average arc-dpa can be written:
\begin{equation}\label{eq:arc3}
\langle\nuarc\rangle = (1-c)\langle\nunrt^{1+b}\rangle + c \cdot \langle\nunrt\rangle.
\end{equation}
Thus, to obtain $\langle\nuarc\rangle$ the value of $\langle\nunrt^{1+b}\rangle$ is needed. We performed an approximate calculation of this quantity, employing a power-law cross-section for the ion-target atom interaction and ignoring the effect of ionization losses, i.e., setting $T_d \approx T$. As shown in \ref{apdx1}, the following approximation
\begin{equation}\label{eq:nuapprox}
\langle\nunrt^{1+b}\rangle \approx \langle\nunrt\rangle^{\lambda (1+b)},
\end{equation}
where $ \lambda\approx 0.56 $, gives adequate results for a wide range of incident ion energies and ion-target combinations. 
This can be seen in fig. \ref{fig1}, where
$ \langle\nunrt^{1+b}\rangle $ 
is plotted as a function of 
$ \langle\nunrt\rangle^{1+b} $ for all the ion/target combinations simulated in the current work. The data shown in the figure have been obtained by taking the $\nunrt$ values per PKA event listed in COLLISON.txt and evaluating the required averages. As seen from the figure, the data from all simulated targets lie within $\pm 10\%$ of the approximate eq. \eqref{eq:nuapprox}, which is depicted by the dashed line.

Utilizing the above approximation, the arc-dpa damage parameters can be obtained by the following prescription:
\begin{enumerate}
	\item Run SRIM with the "Quick calculation of damage" (Q-C) option.
	\item Calculate the NRT-$\langle \nunrt \rangle$ from VACANCY.txt as described in Table \ref{tab:quantities}.
	\item Obtain $\langle\nuarc\rangle$ from eq. \eqref{eq:arc3}, substituting the approximate relation \eqref{eq:nuapprox}:
	\begin{equation}\label{eq:ndarc_approx}
	\langle\nuarc\rangle \approx (1-c)\langle\nunrt\rangle^{0.56(1+b)} + c \cdot \langle\nunrt\rangle
	\end{equation}
	\item The number of displacements per ion is $\langle\nu_{d,arc}\rangle \cdot N_{\text{PKA}}$
\end{enumerate}

Fig. \ref{fig2} depicts the ratio of $\langle\nuarc\rangle$ calculated by the two methods, M2 and M1, respectively, for all simulated ion/target combinations. It is seen that the results of the approximate method M2 deviate by at most 3\% from those of M1.
A similar small deviation can be observed in Table \ref{tab:comparison} between the arc-dpa damage parameters obtained by methods M1 and M2 in the two simulated Fe self-ion irradiations. In the low energy case the arc-dpa parameters obtained by M2 are 4\% lower than those of M1 while in the high energy example the two methods produce essentially equivalent results. Thus, the method M2 can be employed for a quick, approximate evaluation of arc-dpa damage, introducing an error of not more than a few percent compared to the more detailed method M1.

% Figure environment removed

\section{Conclusions}

In this work, we present two methods for evaluating arc-dpa damage parameters in ion irradiations employing the SRIM simulation code with the option ``Quick calculation of damage'' (Q-C). 

The first method is based on SRIM's COLLISON.txt output file, which lists the NRT displacements, $\nunrt$, produced in each simulated primary knock-on atom (PKA) recoil event. The $\nunrt$ values are converted to the corresponding arc-dpa model prediction, $\nuarc$, by means of eq. \eqref{eq:arc2} and then averaged to obtain the total damage parameters. This procedure is similar to the one proposed by \citet{Nordlund-2015-ID597} only in our case the damage energy is essentially obtained by the LSS approximation employed in SRIM's Q-C mode, whereas in \cite{Nordlund-2015-ID597} the damage energy was interpolated from the results of separate detailed SRIM simulations. Thus, our method gains in simplicity but can lead to errors due to the approximation in the damage energy calculation. The errors in the estimated damage could be up to $\sim 30\%$ \cite{Agarwal-2021-OntheuseofSRIMf}. 

In the second method, we devise an approximate relation, which gives $\langle\nuarc\rangle$ directly as a function of $\langle\nunrt\rangle$. Thus, the cumbersome processing of the COLLISON.txt file is not needed since the NRT damage parameter $\langle\nunrt\rangle$ can be easily obtained from VACANCY.txt. 
We found that the arc-dpa parameters obtained by this approximate method differ by not more than a few percent from those calculated by the first method. 