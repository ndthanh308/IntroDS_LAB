\section{Introduction}

In studies of radiation effects in materials it is generally desirable to have a standardized parameter to quantify radiation damage exposure, that would provide a
common basis for comparison of data obtained under different irradiation conditions in terms of impinging particle type and energy. 
Currently, the internationally accepted standard parameter for this purpose is the number of displacements per atom (dpa) 
calculated according to the Norgett-Robinson-Torrens (NRT) model \cite{Norgett-1975-ID480}. 
In the case of ion irradiation, one of the most widely used software tools for estimating the NRT-dpa exposure is the Monte Carlo code SRIM (Stopping and Range of Ions in Matter) \cite{Ziegler-2010-ID479}. SRIM incorporates the NRT model and readily provides the NRT-dpa value for a given ion irradiation. Its popularity is based on the fact that it employs accurate ion stopping powers and on its user-friendly interface. Several authors \cite{Stoller-2013-ID110,Li-2015-IM3DAparallelMon,Weber-2019-ID1179,Crocombette-2019-Quickcalculationof,Stoller-2019-ID1182,Agarwal-2021-OntheuseofSRIMf}
have discussed the application of SRIM for accurate NRT-dpa calculations.

Recently, a modification to the NRT model has been proposed, the athermal recombination corrected dpa (arc-dpa) \cite{Nordlund-2015-ID597, Nordlund-2018-ID1137}. It addresses a well known issue of NRT, namely, the overestimation of the number of stable defects generated by high energy displacement cascades. The arc-dpa model is based on evidence from experimental studies and computer simulations, which indicates that significant defect recombination takes place during the cascade cool-down phase leading to reduced numbers of remaining stable defects. Currently, there is no standardized way to compute arc-dpa exposure in ion irradiations as the model has not yet been implemented in any of the widely used software tools. In their original publication introducing the new model, \citet{Nordlund-2015-ID597} already proposed a method to indirectly estimate the arc-dpa parameter based on the output of standard SRIM simulations. 
Their implementation consisted of two main parts. 
First, a series of SRIM simulations were performed to evaluate the energy deposited by primary knock-on atom (PKA) recoils as target displacements. This is also called damage energy, $T_d$, and must be obtained as a function of the initial PKA recoil energy, $E_R$, for a given target material. In \cite{Nordlund-2015-ID597} this was done for Fe and an interpolating function was devised to obtain $T_d$ as a function of $E_R$ continuously for recoil energies up to 300~keV. 
In the second part of the calculation, the information obtained on $T_d$ is used in post-processing of the SRIM output file "COLLISON.TXT" to finally obtain the arc-dpa values.

In this paper, we propose two alternative methods to calculate arc-dpa exposure using SRIM. The first one is also based on the COLLISON.txt file, similarly to the method in \cite{Nordlund-2015-ID597}. However, instead of separately computing $T_d$ by interpolation, we use the damage energy values that are internally calculated in SRIM with the Lind­hard-Scharff-Schiøtt (LSS) approximation \cite{Lindhard-1963-RANGECONCEPTSANDH}. Thus, the damage energy  interpolation for different target materials is not required. The second method is based on an approximate formula that we propose, which can be employed to estimate directly the arc-dpa exposure based on the corresponding NRT-dpa value. Thus, the cumbersome handling of the COLLISON.txt file is avoided. The two methods are tested on all targets for which arc-dpa model parameters are available and for a range of projectile ions. 




\section{Radiation Damage Models}
The NRT model gives the number of stable displacements, $\nunrt$, produced by a PKA recoil with damage energy $T_d$ as:
\begin{equation}\label{eq:NRT}
\nunrt(T_d)=
	\begin{cases}
		  0             & \text{for } T_d \le E_d\\
		  1 			& \text{for } E_d < T_d \le L\\
T_d/L & \text{for } T_d > L
	\end{cases}	
\end{equation} \\
where $E_d$ is the displacement threshold energy, i.e., the minimum energy required to displace an atom from its lattice position. $L=2E_d/0.8$ denotes the cascade multiplication threshold above which more than one stable displacements are generated by the PKA.

In the arc-dpa model, the 3$^{rd}$ branch of \eqref{eq:NRT} is multiplied by an energy dependent efficiency factor, $\xi \leq 1$. The model definition is summarized in the following two relations:
\begin{equation}\label{eq:arc}
	\nuarc(T_d)=
	\begin{cases}
		0             & \text{for } T_d \le E_d,\\
		1 			& \text{for } E_d < T_d \le L,\\
	\xi(T_d/L)\cdot T_d/L & \text{for } T_d>L,
	\end{cases}	
\end{equation}
\begin{equation}\label{eq:xi}
	\xi(x) = (1-c)\, x^b + c, \quad \text{for } x \geq 1.
\end{equation}
The parameters \textit{b} and \textit{c} are material constants that have been determined for a number of target materials by \citet{Nordlund-2018-Improvingatomicdis}. Their values are given in table \ref{tab:parameters}.

We note that for damage energies above the displacement threshold, $T_d>E_d$, $\nuarc(T_d)$ can be compactly written as
\begin{equation}\label{eq:arc2}
\nuarc(T_d) = \nunrt(T_d)\cdot \xi\left[ \nunrt(T_d) \right].
\end{equation}
This definition will be utilized in the following paragraphs.


 

