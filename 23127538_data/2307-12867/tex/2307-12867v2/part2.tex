\section{Methods}
In this section, we present the two different methods to obtain arc-dpa damage parameters from SRIM output.

\subsection{Method 1 (M1)}
This method utilizes the COLLISON.txt output file. In SRIM Q-C mode, this file lists all simulated PKA scattering events and reports, among other data, the number of displacements, $\nu_d$, generated per event. These $\nu_d$ values, labelled "Target vacancies", are calculated according to the NRT model, eq. \eqref{eq:NRT}, with the damage energy, $T_d$, obtained from the approximate LSS theory \cite{Ziegler-2008-ID880}. For $\nu_d > 1$, we can easily recover the LSS damage energy by multiplying $\nunrt$ with the cascade multiplication factor, $L$ (cf. eq. \eqref{eq:NRT}). Then, the obtained $T_d$ can be used in eq. \eqref{eq:arc} to evaluate the displacements according to the arc-dpa model. Equivalently, $\nuarc$ can be obtained by plugging $\nunrt$ directly into the alternative arc-dpa definition, eq. \eqref{eq:arc2}. The steps to calculate the arc-dpa damage parameters are as follows: 
\begin{enumerate}
	\item Run SRIM with the "Quick calculation of damage" (Q-C) option.
	\item Parse the COLLISON.txt output file to obtain the NRT displacements per PKA event, $\nunrt$.
	\item Calculate the corresponding $\nuarc$ per PKA from eq. \eqref{eq:arc2}, $\nuarc = \nunrt\cdot \xi(\nunrt)$.
	\item Take the average of the $\nuarc$ values to obtain the mean displacements per PKA according to the arc-dpa model, $\langle\nuarc\rangle$ (cf. Table \ref{tab:quantities}).
	\item Multiply by the number of PKAs per ion, $N_{PKA}$, to obtain the number of displacements per ion, $N_{d,arc} = \langle\nuarc\rangle \cdot N_{\text{PKA}}$
\end{enumerate}

\subsection*{Method 2 (M2)}

% Figure environment removed

The objective of M2 is to provide a quick estimate of the arc-dpa damage parameters, without having to resort to the cumbersome processing of COLLISON.txt. For this, we note that from eq. \eqref{eq:arc2} the average arc-dpa can be written:
\begin{equation}\label{eq:arc3}
\langle\nuarc\rangle = (1-c)\langle\nunrt^{1+b}\rangle + c \cdot \langle\nunrt\rangle.
\end{equation}
Thus, to obtain $\langle\nuarc\rangle$ the value of $\langle\nunrt^{1+b}\rangle$ is needed. We performed an approximate calculation of this quantity, employing a power-law cross-section for the ion-target atom interaction and ignoring the effect of ionization losses, i.e., setting $T_d \approx T$. As shown in \ref{apdx1}, the following approximation
\begin{equation}\label{eq:nuapprox}
\langle\nunrt^{1+b}\rangle \approx \langle\nunrt\rangle^{\lambda (1+b)},
\end{equation}
where $ \lambda\approx 0.56 $, gives adequate results for a wide range of incident ion energies and ion-target combinations. 
This can be seen in fig. \ref{fig1}, where
$ \langle\nunrt^{1+b}\rangle $ 
is plotted as a function of 
$ \langle\nunrt\rangle^{1+b} $ for all the ion/target combinations simulated in the current work. The data shown in the figure have been obtained by taking the $\nunrt$ values per PKA event listed in COLLISON.txt and evaluating the required averages. As seen from the figure, the data from all simulated targets lie within $\pm 10\%$ of the approximate eq. \eqref{eq:nuapprox}, which is depicted by the dashed line.

Utilizing the above approximation, the arc-dpa damage parameters can be obtained by the following prescription:
\begin{enumerate}
	\item Run SRIM with the "Quick calculation of damage" (Q-C) option.
	\item Calculate the NRT-$\langle \nunrt \rangle$ from VACANCY.txt as described in Table \ref{tab:quantities}.
	\item Obtain $\langle\nuarc\rangle$ from eq. \eqref{eq:arc3}, substituting the approximate relation \eqref{eq:nuapprox}:
	\begin{equation}\label{eq:ndarc_approx}
	\langle\nuarc\rangle \approx (1-c)\langle\nunrt\rangle^{0.56(1+b)} + c \cdot \langle\nunrt\rangle
	\end{equation}
	\item The number of displacements per ion is $\langle\nu_{d,arc}\rangle \cdot N_{\text{PKA}}$
\end{enumerate}



\section{Results and discussion}

\begin{table*} [tb!]
	\centering
	\begin{threeparttable}[t]
	\begin{adjustbox}{max width=\textwidth}
	\begin{tabular}{lcccScS}
		\hline\hline
		{} & $E_0$ & $N_{\text{PKA}}$ & {$N_{d}$} & {$\langle\nunrt\rangle$} 
		& {$N_{d, arc}$}  & {$\langle\nuarc\rangle$} \\
		\hline
		\citet{Nordlund-2015-ID597} & \multirow{3}{*}{78.7 keV} & \multirow{3}{*}{44.1} & 
		539 & 12.2\tnote{\dag} & 217 & 4.93\tnote{\dag} \\
		This study - Method 1 & {} & {} & 
		530 & 12.0 & 217 & 4.92 \\ 
		This study - Method 2 & {} & {} & 
		530 & 12.0 & 209 & 4.74 \\
		\hline
		Method of \citet{Nordlund-2015-ID597} & \multirow{3}{*}{5 MeV} & \multirow{3}{*}{442} & 
		8800 & 20.0 & 3150 & 7.14 \\
		This study - Method 1 & {} & {} & 
		7870 & 17.9 & 2900 & 6.56 \\ 
		This study - Method 2 & {} & {} & 
		7870 & 17.9 & 2890 & 6.54 \\
		\hline\hline	
	\end{tabular}
	\end{adjustbox}

	\begin{tablenotes}
		\item [\dag] $\langle\nunrt\rangle$ and $\langle\nuarc\rangle$ are calculated by dividing $N_d$ and $N_{d,arc}$ from \cite{Nordlund-2015-ID597}, respectively, by $N_{\text{PKA}}$ as obtained in the present study.
		 
	\end{tablenotes}
	\end{threeparttable}

	\caption{Damage parameters obtained by different methods for the irradiation of an Fe target with Fe ions of energy $E_0$.}
	\label{tab:comparison}
\end{table*}

Damage parameters obtained by SRIM according to method M1 are compared to the results of \citet{Nordlund-2015-ID597}.
First, we repeated the simulation of 78.7 keV Fe ions incident on an Fe target that was reported in \cite{Nordlund-2015-ID597}. The results of both methods are given in Table \ref{tab:comparison}.
As seen from the table, there is a small 2\% difference in the NRT parameters, $N_d$ and $\langle\nunrt\rangle$, between our M1 and the results of \cite{Nordlund-2015-ID597}. The corresponding arc-dpa parameters almost coincide. 
As the damage energies occurring in this example are relatively low, we simulated self-ion Fe irradiation with a much higher projectile energy, $E_0=5$~MeV, and evaluated the results with both our proposed method M1 and the one described in \citet{Nordlund-2015-ID597}.  
In the latter case, we used the data from their fig.~1.2 to extend the interpolation of $T_d$ to target recoil energies up to 5~MeV. 
The resulting damage parameters, also listed in Table \ref{tab:comparison}, show that there is a 10\% difference between the NRT parameters obtained by our method M1 and the evaluation according to \cite{Nordlund-2015-ID597}. 
The corresponding arc-dpa parameters exhibit a similar but slightly lower discrepancy of about 8\%.
We attribute the differences in damage parameters to the distinct way the damage energy is evaluated in the two methods. In the present work we employ the Q-C mode, while \citet{Nordlund-2015-ID597} used SRIM's "Detailed Calculation with Full Damage Cascades" (F-C) option. In the latter case, SRIM utilizes detailed stopping power calculations for all secondary recoils in the PKA cascade, thus, the value of $T_d$ is potentially more accurate than in the Q-C mode where the LSS approximation is employed. \citet{Agarwal-2021-OntheuseofSRIMf} have made a detailed comparison of SRIM damage calculations in Q-C and F-C modes. They found differences up to $\pm 25\%$ in the damage energy predicted by the two modes. The discrepancies we observe here between M1 and the method of \cite{Nordlund-2015-ID597} are of comparable magnitude. The smaller discrepancy observed in the arc-dpa parameters is due to the fact that the arc-dpa damage function lowers the significance of high energy damage events, where the errors due to the LSS approximation are more pronounced.

\begin{table*} [tb!]
	\centering
	\begin{adjustbox}{max width=\textwidth}
	\begin{tabular}{lSSS@{\hspace{1cm}}SSS@{\hspace{1cm}}SSS}
		\hline\hline
		Projectile &
		$\langle\nunrt\rangle$ & 
		\multicolumn{2}{c@{\hspace{1cm}}}{$\langle\nuarc\rangle$} & 
			$\langle\nunrt\rangle$ & 
		\multicolumn{2}{c@{\hspace{1cm}}}{$\langle\nuarc\rangle$} & 
		$\langle\nunrt\rangle$ & 
		\multicolumn{2}{c}{$\langle\nuarc\rangle$} \\
		
		{} & {}
		& {M1} & {M2} &
		& {M1} & {M2} &
		& {M1} & {M2} \\
		\hline
		& \multicolumn{3}{c@{\hspace{1cm}}}{Fe target}
		& \multicolumn{3}{c@{\hspace{1cm}}}{Ni target}
		& \multicolumn{3}{c}{Cu target} \\		
	
		1 MeV H & 1.91 & 1.38 & 1.38 & 1.92 & 1.21 & 1.21 & 2.02 & 1.27 & 1.28 \\
		1 MeV He & 2.47 & 1.62 & 1.60 & 2.54 & 1.35 & 1.35 & 2.76 & 1.45 & 1.45 \\
		3 MeV Al  & 10.4 & 4.24 & 4.22 & 10.6 & 3.21 & 3.21 & 12.6 & 3.29 & 3.35 \\
		5 MeV Fe & 17.9 & 6.56 & 6.54 & 18.6 & 5.04 & 5.04 & 22.3 & 4.95 & 5.04 \\
		10 MeV Au  & 44.8 & 14.8 & 14.6 & 47.7 & 11.7 & 11.7 & 60.4 & 11.4 & 11.4 \\
		
		\hline
		& \multicolumn{3}{c@{\hspace{1cm}}}{Pd target}
		& \multicolumn{3}{c@{\hspace{1cm}}}{W target}
		& \multicolumn{3}{c}{Pt target} \\
			
		1 MeV H & 1.89 & 1.17 & 1.17 & 1.61 & 1.20 & 1.18 & 1.78 & 1.06 & 1.05  \\
		1 MeV He & 2.47 & 1.27 & 1.27 & 1.94 & 1.29 & 1.27 & 2.37 & 1.11 & 1.10  \\
		3 MeV Al & 11.6 & 2.70 & 2.73 & 8.06 & 2.51 & 2.44 & 11.7 & 2.08 & 2.04  \\
		5 MeV Fe & 22.7 & 4.39 & 4.45 & 16.7 & 3.85 & 3.76 & 25.6 & 3.59 & 3.53  \\
		10 MeV Au & 67.0 & 11.1 & 11.2 & 52.8 & 8.88 & 8.67 & 87.4 & 10.4 & 10.3  \\

		\hline\hline
	\end{tabular}
	\end{adjustbox}			
	\caption{Mean displacements per primary knock-on atom for both NRT- and arc-dpa models, $\langle\nunrt\rangle$ and $\langle\nuarc\rangle$, respectively, as obtained by SRIM. For $\langle\nuarc\rangle$ the results of both calculation methods M1 and M2 are given. }
	\label{tab:all_data}
\end{table*} 

% Figure environment removed

Table \ref{tab:comparison} shows also the results of the approximate method M2 for the two Fe self-irradiation simulations. The NRT damage parameters of methods M1 and M2, obtained from the files COLLISON.txt and VACANCY.txt, respectively, are identical as expected. The corresponding arc-dpa parameters exhibit a discrepancy of 4\% and 0.3\% in the low and high energy simulation, respectively. Table \ref{tab:all_data} lists the values of NRT mean displacements per PKA, $\langle \nunrt \rangle$, and the corresponding arc-dpa value, $\langle \nuarc \rangle$, as calculated by both methods M1 and M2 for all test cases that we simulated in the current work. 
Comparing the two models, it is observed that the arc-dpa predicts significantly fewer displacements per PKA at high projectile energies compared to NRT, while at low projectile energies the difference is not so large. This is exactly the behavior anticipated for the arc-dpa model \cite{Nordlund-2015-ID597,Nordlund-2018-Improvingatomicdis,Nordlund-2018-ID1137}. The values of $\langle \nuarc \rangle$ calculated by methods M1 and M2 are in all cases very similar.
Fig. \ref{fig2} depicts the ratio of $\langle\nuarc\rangle$ obtained by M2 and M1, respectively. As seen from the figure, the approximate method M2 deviates by at most 3\% from the results of M1.
Thus, the method M2 can be employed for an approximate evaluation of arc-dpa damage, introducing an error of not more than a few percent compared to the more detailed method M1. 

\subsection{Depth-dependent calculations}
In many applications the depth-dependent damage profile is also of interest. Both methods M1 and M2 can be employed for obtaining the arc-dpa damage profile from SRIM output.
This is most straightforward in the case of M2, were the calculations shown in Table \ref{tab:quantities} can be applied line-by-line to the data of VACANCY.txt and thus obtain $\langle \nuarc \rangle$ and $N_{d,arc}$ as a function of depth.
On the other hand, for method M1 extra processing is required to select from COLLISON.txt those PKA events which occur within a certain target depth bin and then perform the calculations of Table \ref{tab:quantities} to obtain the arc-dpa parameters in this particular bin. By iterating this procedure over all depth bins we finally obtain the damage profile.

Indicative results of depth-dependent application of methods M1 and M2 are depicted in Fig. \ref{AuonW}. The figure shows the damage profiles predicted by SRIM for a 10~MeV Au irradiation of W in the standard NRT-dpa model and the arc-dpa model as obtained by methods M1 and M2. As seen in the figure, there is a peak in vacancy production at about 0.5~$\mu$m in both NRT- and arc-dpa data. The arc-dpa profiles obtained by M1 and M2 are almost identical.

% Figure environment removed

\section{Conclusions}

In this work, we present two methods for evaluating arc-dpa damage parameters in ion irradiations employing the SRIM simulation code. The methods are based on the ``Quick calculation of damage'' (Q-C) option to obtain an initial estimate of displacement damage compatible with the NRT standard. 

The first method employs SRIM's COLLISON.txt output file, which lists the NRT displacements, $\nunrt$, produced in each simulated primary knock-on atom (PKA) recoil event. The $\nunrt$ values are converted to the corresponding arc-dpa model prediction, $\nuarc$, by means of eq. \eqref{eq:arc2} and then averaged to obtain the total damage parameters. This procedure is similar to the one proposed by \citet{Nordlund-2015-ID597} only in our case the damage energy, $T_d$, is essentially obtained by the LSS approximation employed in SRIM's Q-C mode, whereas in \cite{Nordlund-2015-ID597} the damage energy was interpolated from the results of separate detailed SRIM simulations. Thus, our method gains in simplicity but can lead to errors due to the approximation in the damage energy calculation. According to \citet{Agarwal-2021-OntheuseofSRIMf} the discrepancy in $T_d$ obtained by Q-C and F-C modes, respectively, could be up to $\sim 25\%$. It is expected that this would be also the upper limit for the discrepancy in arc-dpa damage. 

In the second method, we devise an approximate relation, which gives $\langle\nuarc\rangle$ directly as a function of $\langle\nunrt\rangle$. Thus, the cumbersome processing of the COLLISON.txt file is not needed since $\langle\nunrt\rangle$ can be easily obtained from VACANCY.txt. 
We found that the arc-dpa parameters obtained by this approximate method differ by not more than a few percent from those calculated by the first method. 

Both methods can also be employed for depth-dependent arc-dpa damage calculations. 

Finally, it is noted that if the arc-dpa model is expanded to more complex systems in the future, as, e.g., concentrated alloys, the use of the ``quick damage'' option may have to be re-evaluated, as it does not handle properly multi-elemental targets.