\subsection{Depth-dependent calculations}
In many applications the depth-dependent damage profile is also of interest. Both methods M1 and M2 can be employed for obtaining the arc-dpa damage profile from SRIM output.
This is most straightforward in the case of M2, were the calculations shown in Table \ref{tab:quantities} can be applied line-by-line to the data of VACANCY.txt and thus obtain $\langle \nuarc \rangle$ and $N_{d,arc}$ as a function of depth.
On the other hand, for method M1 extra processing is required to select from COLLISON.txt those PKA events which occur within a certain target depth bin and then perform the calculations of Table \ref{tab:quantities} to obtain the arc-dpa parameters in this particular bin. By iterating this procedure over all depth bins we finally obtain the damage profile.

Indicative results of depth-dependent application of methods M1 and M2 are depicted in Fig. \ref{AuonW}. The figure shows the damage profiles predicted by SRIM for a 10~MeV Au irradiation of W in the standard NRT-dpa model and the arc-dpa model as obtained by methods M1 and M2. As seen in the figure, there is a peak in vacancy production at about 0.5~$\mu$m in both NRT- and arc-dpa data. The arc-dpa profiles obtained by M1 and M2 are almost identical.

% Figure environment removed

\section{Conclusions}

In this work, we present two methods for evaluating arc-dpa damage parameters in ion irradiations employing the SRIM simulation code. The methods are based on the ``Quick calculation of damage'' (Q-C) option to obtain an initial estimate of displacement damage compatible with the NRT standard. 

The first method employs SRIM's COLLISON.txt output file, which lists the NRT displacements, $\nunrt$, produced in each simulated primary knock-on atom (PKA) recoil event. The $\nunrt$ values are converted to the corresponding arc-dpa model prediction, $\nuarc$, by means of eq. \eqref{eq:arc2} and then averaged to obtain the total damage parameters. This procedure is similar to the one proposed by \citet{Nordlund-2015-ID597} only in our case the damage energy, $T_d$, is essentially obtained by the LSS approximation employed in SRIM's Q-C mode, whereas in \cite{Nordlund-2015-ID597} the damage energy was interpolated from the results of separate detailed SRIM simulations. Thus, our method gains in simplicity but can lead to errors due to the approximation in the damage energy calculation. According to \citet{Agarwal-2021-OntheuseofSRIMf} the discrepancy in $T_d$ obtained by Q-C and F-C modes, respectively, could be up to $\sim 25\%$. It is expected that this would be also the upper limit for the discrepancy in arc-dpa damage. 

In the second method, we devise an approximate relation, which gives $\langle\nuarc\rangle$ directly as a function of $\langle\nunrt\rangle$. Thus, the cumbersome processing of the COLLISON.txt file is not needed since $\langle\nunrt\rangle$ can be easily obtained from VACANCY.txt. 
We found that the arc-dpa parameters obtained by this approximate method differ by not more than a few percent from those calculated by the first method. 

Both methods can also be employed for depth-dependent arc-dpa damage calculations. 

Finally, it is noted that if the arc-dpa model is expanded to more complex systems in the future, as, e.g., concentrated alloys, the use of the ``quick damage'' option may have to be re-evaluated, as it does not handle properly multi-elemental targets.