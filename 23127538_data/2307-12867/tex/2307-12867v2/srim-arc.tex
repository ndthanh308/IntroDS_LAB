%% 
%% Copyright 2007-2019 Elsevier Ltd
%% 
%% This file is part of the 'Elsarticle Bundle'.
%% ---------------------------------------------
%% 
%% It may be distributed under the conditions of the LaTeX Project Public
%% License, either version 1.2 of this license or (at your option) any
%% later version.  The latest version of this license is in
%%    http://www.latex-project.org/lppl.txt
%% and version 1.2 or later is part of all distributions of LaTeX
%% version 1999/12/01 or later.
%% 
%% The list of all files belonging to the 'Elsarticle Bundle' is
%% given in the file `manifest.txt'.
%% 
%% Template article for Elsevier's document class `elsarticle'
%% with harvard style bibliographic references

%\documentclass[review,12pt,sort&compress]{elsarticle}
\documentclass[final,5p,sort&compress]{elsarticle}

%% Use the option review to obtain double line spacing
%% \documentclass[preprint,review,12pt]{elsarticle}

%% Use the options 1p,twocolumn; 3p; 3p,twocolumn; 5p; or 5p,twocolumn
%% for a journal layout:
%% \documentclass[final,1p,times]{elsarticle}
%% \documentclass[final,1p,times,twocolumn]{elsarticle}
%% \documentclass[final,3p,times]{elsarticle}
%% \documentclass[final,3p,times,twocolumn]{elsarticle}
%% \documentclass[final,5p,times]{elsarticle}
%% \documentclass[final,5p,times,twocolumn]{elsarticle}

%% For including figures, graphicx.sty has been loaded in
%% elsarticle.cls. If you prefer to use the old commands
%% please give \usepackage{epsfig}

%% The amssymb package provides various useful mathematical symbols
\usepackage{amssymb}
\usepackage{amsmath}
%% The amsthm package provides extended theorem environments
%% \usepackage{amsthm}

% for tables
\usepackage{multirow}
\usepackage{threeparttable}
\usepackage{adjustbox}
\usepackage{siunitx}

%% The lineno packages adds line numbers. Start line numbering with
%% \begin{linenumbers}, end it with \end{linenumbers}. Or switch it on
%% for the whole article with \linenumbers.
\usepackage{lineno}

% hyperlinks
% \usepackage{xcolor}
\usepackage[colorlinks]{hyperref}
\usepackage{todonotes}


\journal{NIMB}

%\newcommand{\nunrt}{{\nu_{\text{NRT}}}}
%\newcommand{\nuarc}{{\nu_{\text{ARC}}}}

\newcommand{\nunrt}{\nu_{d}}
\newcommand{\nuarc}{\nu_{d,\text{arc}}}

\begin{document}

\begin{frontmatter}

%% Title, authors and addresses

%% use the tnoteref command within \title for footnotes;
%% use the tnotetext command for theassociated footnote;
%% use the fnref command within \author or \address for footnotes;
%% use the fntext command for theassociated footnote;
%% use the corref command within \author for corresponding author footnotes;
%% use the cortext command for theassociated footnote;
%% use the ead command for the email address,
%% and the form \ead[url] for the home page:
%% \title{Title\tnoteref{label1}}
%% \tnotetext[label1]{}
%% \author{Name\corref{cor1}\fnref{label2}}
%% \ead{email address}
%% \ead[url]{home page}
%% \fntext[label2]{}
%% \cortext[cor1]{}
%% \address{Address\fnref{label3}}
%% \fntext[label3]{}

\title{On the use of SRIM for calculating arc-dpa exposure\\
{\small 
\textit{Published in Nucl. Instrum. Methods Phys. Res., Sect. B,} 
\href{https://doi.org/10.1016/j.nimb.2023.165145}{doi:10.1016/j.nimb.2023.165145} 
} 
}

%% use optional labels to link authors explicitly to addresses:
%% \author[label1,label2]{}
%% \address[label1]{}
%% \address[label2]{}

\author[1,2]{E. Mitsi\corref{cor1}}
\ead{elmitsi@ipta.demokritos.gr}
\author[2]{K. Koutsomitis}
\author[2]{G. Apostolopoulos}

\address[1]{Department of Physics, National Technical University of Athens, Zografou Campus, EL-15780 Athens, Greece}
\address[2]{Institute of Nuclear and Radiological Science and Technology, Energy, and Safety,\\  N.C.S.R. ``Demokritos'', EL-15310 Agia Paraskevi, Greece}
\cortext[cor1]{Corresponding author}

\begin{abstract}
We propose two methods for evaluating athermal recombination corrected (arc) displacement damage parameters in ion irradiations employing the computer code SRIM (Stopping and Range of Ions in Matter). The first method consists of post-processing the detailed SRIM output for all simulated damage events and re-calculating according to the arc damage model. In the second method, an approximate empirical formula is devised which gives the average displacements in the arc damage model as a function of the corresponding quantity according to the standard Norgett-Robinson-Torrens model, which is readily obtained from SRIM. 

\end{abstract}

%%Graphical abstract
%\begin{graphicalabstract}
%%% Figure removed
%\end{graphicalabstract}

%%Research highlights
%\begin{highlights}
%\item Research highlight 1
%\item Research highlight 2
%\end{highlights}

\begin{keyword}
Displacements per atom (dpa) \sep
Athermal recombination corrected dpa \sep
Ion irradiation \sep
SRIM
\end{keyword}

\end{frontmatter}

%\linenumbers

%% main text
%\section{}
%\label{}

\section{Introduction}

In studies of radiation effects in materials it is generally desirable to have a standardized parameter to quantify radiation damage exposure, that would provide a
common basis for comparison of data obtained under different irradiation conditions in terms of impinging particle type and energy. 
Currently, the internationally accepted standard parameter for this purpose is the number of displacements per atom (dpa) 
calculated according to the Norgett-Robinson-Torrens (NRT) model \cite{Norgett-1975-ID480}. 
Recently, a modification to the NRT model has been proposed, the athermal recombination corrected dpa (arc-dpa) \cite{Nordlund-2015-ID597, Nordlund-2018-Improvingatomicdis, Nordlund-2018-ID1137}. It addresses a well known issue of NRT, namely, the overestimation of the number of stable defects generated by high energy displacement cascades. The arc-dpa model is based on evidence from experimental studies and computer simulations, which indicates that significant defect recombination takes place during the cascade cool-down phase leading to reduced numbers of remaining stable defects.
In its present formulation, the model can be applied to a limited set of monoatomic metallic target materials, for which the required material-specific model parameters have been obtained. It is expected that in the future, as new experimental and simulation data becomes available, the arc-dpa model will be applicable to a wider range of metallic materials including both dilute and concentrated alloys \cite{Nordlund-2018-Improvingatomicdis}.

There are several software tools available for estimating radiation damage exposure. In the case of ion irradiation, one of the most widely used such tools is the Monte Carlo code SRIM (Stopping and Range of Ions in Matter) \cite{Ziegler-2010-ID479}. Its popularity is based on the fact that it employs accurate ion stopping powers and provides a user-friendly interface. Several authors \cite{Stoller-2013-ID110,Li-2015-IM3DAparallelMon,Weber-2019-ID1179,Crocombette-2019-Quickcalculationof,Stoller-2019-ID1182,Agarwal-2021-OntheuseofSRIMf}
have discussed the application of SRIM for accurate damage calculations. The program offers different options for the simulation of damage events. Many authors have noted that the option ``Quick calculation of damage'' (Q-C) may be preferable in order to obtain results comparable to the NRT model \cite{Stoller-2013-ID110,Weber-2019-ID1179}. However, the Q-C mode implements certain approximations and thus may be less accurate, especially for multi-elemental targets \cite{Weber-2019-ID1179,Crocombette-2019-Quickcalculationof}. On the other hand, the more detailed ``Full damage cascades'' (F-C) option tends to significantly overestimate damage production compared to the Q-C mode \cite{Stoller-2013-ID110,Agarwal-2021-OntheuseofSRIMf}. \citet{Agarwal-2021-OntheuseofSRIMf} suggested employing SRIM in F-C mode to obtain the average of the damage energy, $T_d$, i.e., the ion energy deposited to target displacements. The average $T_d$ is then inserted in the NRT formula to calculate the average damage produced. This is most accurate at high damage energies, where the NRT damage function is linear with respect to $T_d$.

Regarding the arc-dpa model, there is currently no standardized way to compute damage exposure in ion irradiations as the model has not yet been implemented in any of the widely used software tools. 
Since the arc-dpa damage function is strongly non-linear, knowledge of just the average value of $T_d$ is not enough to correctly estimate the damage, as done, e.g., for NRT-dpa with the SRIM damage energy method of \citet{Agarwal-2021-OntheuseofSRIMf}.  
In their original publication introducing the new damage model, \citet{Nordlund-2015-ID597} already discussed the application of SRIM for indirect estimation of arc-dpa damage parameters. 
They presented a method of calculation in two steps. 
First, a series of SRIM simulations were performed to evaluate $T_d$ as a function of the initial PKA recoil energy, $E_R$, for a given target material. In \cite{Nordlund-2015-ID597}, this was done for Fe and an interpolating function was devised to obtain $T_d$ continuously as a function of recoil energy up to 300~keV. 
In the second part of the calculation, the information obtained on $T_d$ is used for post-processing the SRIM output file "COLLISON.txt" to finally obtain the arc-dpa values.

In this paper, we propose two alternative methods to calculate arc-dpa exposure using SRIM. 
They are based on the ``Quick calculation of damage'' (Q-C) option, which provides a NRT-compatible damage estimate as a starting point. Since the arc-dpa model refers currently only to monoatomic metals, the known limitation of the Q-C mode regarding multi-elemental targets is not relevant.   
The first of the proposed methods utilizes the SRIM output file COLLISON.txt as previously discussed in \cite{Nordlund-2015-ID597}. However, instead of separately computing $T_d$ by interpolation, we use the damage energy values that are internally calculated by SRIM with the Lind­hard-Scharff-Schiøtt (LSS) approximation \cite{Lindhard-1963-RANGECONCEPTSANDH}. Thus, the damage energy interpolation for different target materials is not required. The second method is based on an approximate formula that we propose, which can be employed for direct estimation of arc-dpa exposure based on the corresponding NRT-dpa value. Thus, the cumbersome handling of the COLLISON.txt file is avoided. The two methods are tested on all targets for which arc-dpa model parameters are available and for a range of projectile ions. 

\section{Radiation Damage Models}
The NRT model gives the number of stable displacements, $\nunrt$, produced by a PKA recoil with damage energy $T_d$ as:
\begin{equation}\label{eq:NRT}
\nunrt(T_d)=
	\begin{cases}
		  0             & \text{for } T_d \le E_d\\
		  1 			& \text{for } E_d < T_d \le L\\
T_d/L & \text{for } T_d > L
	\end{cases}	
\end{equation} \\
where $E_d$ is the displacement threshold energy, i.e., the minimum energy required to displace an atom from its lattice position. $L=2E_d/0.8$ denotes the cascade multiplication threshold above which more than one stable displacements are generated by the PKA.

In the arc-dpa model, the 3$^{rd}$ branch of \eqref{eq:NRT} is multiplied by an energy dependent efficiency factor, $\xi \leq 1$. The model definition is summarized in the following two relations:
\begin{equation}\label{eq:arc}
	\nuarc(T_d)=
	\begin{cases}
		0             & \text{for } T_d \le E_d,\\
		1 			& \text{for } E_d < T_d \le L,\\
	\xi(T_d/L)\cdot T_d/L & \text{for } T_d>L,
	\end{cases}	
\end{equation}
\begin{equation}\label{eq:xi}
	\xi(x) = (1-c)\, x^b + c, \quad \text{for } x \geq 1.
\end{equation}
The parameters \textit{b} and \textit{c} are material constants that have been determined for a number of target materials by \citet{Nordlund-2018-Improvingatomicdis}. Their values are given in table \ref{tab:parameters}.

We note that for damage energies above the displacement threshold, $T_d>E_d$, $\nuarc(T_d)$ can be compactly written as
\begin{equation}\label{eq:arc2}
\nuarc(T_d) = \nunrt(T_d)\cdot \xi\left[ \nunrt(T_d) \right].
\end{equation}
This definition will be utilized in the following paragraphs.


\section{SRIM simulation conditions and data handling} \label{SRIMcondition}


\begin{table} [tb!]
	\centering
	\begin{tabular}{cccccc}
		\hline\hline
		Ion & H & He & Al & Fe & Au\\
		\hline
		$E_0$ (MeV) & 1 & 1 & 3 & 5 & 10\\ 
		\hline\hline	
	\end{tabular}
	\caption{Projectile ions and corresponding incident energies $E_0$.}
	\label{tab:ions}
\end{table}

\begin{table} [tb!]
	\centering
	\begin{tabular}{ccccccc}
		\hline\hline
		Target & Fe & Ni & Cu & Pd & W & Pt\\
		\hline
		$E_d$ (eV) \cite{Nordlund-2015-ID597} & 40 & 40 & 29 & 41 & 90 & 44 \\
		$b$ \cite{Nordlund-2018-Improvingatomicdis} & -0.568 & -1.01 & -0.68 & -0.88 & -0.56 & -1.12\\ 
		$c$ \cite{Nordlund-2018-Improvingatomicdis} & 0.286 & 0.23 & 0.16 & 0.15 & 0.12 & 0.11\\
		\hline\hline	
	\end{tabular}
	\caption{Displacement threshold, $E_d$, and arc-dpa model parameters, $(b, c)$, of simulated targets.}
	\label{tab:parameters}
\end{table} 

\renewcommand{\arraystretch}{1.75}

\begin{table*} [tb!]
	\centering

	\begin{adjustbox}{max width=\textwidth}
	\begin{threeparttable}[t]
		
		\begin{tabular}{lccc}
			\hline\hline
			Quantity & Symbol 
			& \parbox[c][1cm]{3cm}{
				Method 1 (M1)\\COLLISON.txt
			}
			& \parbox[c]{3cm}{ \centering
				Method 2 (M2)\\VACANCY.txt
			} 
			\\
			
			\hline
			
			PKAs per ion & $N_{\text{PKA}}$ 
			& $N_{\text{rows}} / N_{\text{ions}}$ 
			& $\sum_k{ \left[ \nu_i \right]_k\, \Delta x}$ \tnote{\dag} 
			\\
			
			\hline
			
			& &\multicolumn{2}{c}{NRT-dpa model} \\
			
			Displacements per ion & $N_{d}$ 
			& $N_{\text{ions}}^{-1}\,\sum_k{\left[ \nu_d\right]_k}$ \tnote{\ddag}
			& $\sum_k{\left[ \nu_i + \nu_r \right]_k\, \Delta x} $ \tnote{\dag} 
			\\
								
			Mean displacements per PKA & $ \langle \nu_d \rangle $ 
			& \multicolumn{2}{c}{$ N_d / N_{\text{PKA}}$}\\
			
			\hline
			
			& &\multicolumn{2}{c}{arc-dpa model} \\
			
			Displacements per ion & $N_{d,arc}$ 
			& \multicolumn{2}{c}{$\langle \nu_{d,arc} \rangle \cdot N_{\text{PKA}} $}\\
			
			Mean displacements per PKA & $ \langle \nu_{d,arc} \rangle $ 
			& $N_{\text{rows}}^{-1}\,\sum_k{
				\left[ \nu_d\right]_k \xi\left( \left[ \nu_d\right]_k \right)
				}$ \tnote{\ddag}
			& \parbox[c]{3cm}{
			\begin{center}
				eq. \eqref{eq:ndarc_approx} with $\langle \nu_d \rangle $\\as above
			\end{center}}
			\\

			\hline\hline	
		\end{tabular}
		
		
		
		\begin{tablenotes}
			\item [\dag]  $\left[ \nu_i \right]_k$ and $\left[ \nu_r \right]_k$ are the ``vacancies by ions'' 
			              and ``vacancies by recoils'', respectively, in the $k$-th target depth bin, with  
						  $\Delta x$ denoting the bin width.
			\item [\ddag]  $\left[ \nu_d \right]_k$ denotes the number of vacancies 
			               estimated by SRIM for the $k$-th PKA event. The sum is over all events.
			
		\end{tablenotes}

		

	\end{threeparttable}
	\end{adjustbox}
	\caption{Calculation of damage parameters from SRIM output files}
	\label{tab:quantities}
\end{table*} 
\renewcommand{\arraystretch}{1.0}

All simulations were performed utilizing SRIM-2013 and employing the option "Ion distribution and Quick calculation of damage" (Q-C). 
% Lattice and surface binding energies were set equal to zero according to the recommendation in \cite{Stoller-2013-ID110}. 
A range of projectile ions were employed, with atomic numbers varying from $Z=1$ (H) to 79 (Au)
and energies ranging from $E_0=1$ to 10 MeV, similarly to the work of \citet{Agarwal-2021-OntheuseofSRIMf}. The ions and corresponding energies are listed in
Table \ref{tab:ions}. Table \ref{tab:parameters} shows all the targets that we tested, which are essentially all materials whose arc-dpa parameters were estimated in \cite{Nordlund-2018-Improvingatomicdis}. Target
thickness was chosen appropriately in order to ensure that
the impinging ions stop within the examined
region. 
The target displacement energies, $E_d$, are based on internationally recommended standard values and are also given in Table \ref{tab:parameters}.
In the case of Fe self-ion irradiation, an extra simulation with $E_0=78.7$~keV was also performed in order to directly compare with results from \cite{Nordlund-2015-ID597}. For each ion/target combination 10,000 ion histories were run.

Damage parameters are extracted from the SRIM output files, either VACANCY.txt or COLLISON.txt. Table \ref{tab:quantities} lists all quantities of interest and the way they are calculated depending on the damage model and the output file used. 

The number of PKAs per ion, $N_{\text{PKA}}$, is obtained by integrating the 2nd data column of VACANCY.txt (``vacancies by ions'',  $\nu_i$) or by dividing the number of data rows, $N_{\text{rows}}$, in COLLISON.txt by the number of simulated ions, $N_{\text{ions}}$. $N_{\text{PKA}}$ is independent of the damage model. 

The NRT displacements per ion, $N_d$, is obtained as follows. In the case of VACANCY.txt, $N_d$ is found by summing the $2^{nd}$ and $3^{rd}$ column of the data table, i.e., ``vacancies by ions'', $\nu_i$, and "vacancies by recoils", $\nu_r$, respectively, as suggested by previous authors \cite{Stoller-2013-ID110,Nordlund-2015-ID597}. 
Regarding the COLLISON.txt file, $N_d$ is calculated by adding up the "Target vacancies",  $\nu_d$, of all PKAs and dividing by $N_{\text{ions}}$. Finally, the average displacements per PKA, $\langle \nu_d \rangle$, is equal to $N_d / N_{\text{PKA}}$. 

The calculation of arc-dpa damage parameters is described in the next section.

All evaluations and the parsing of SRIM output files were performed in the OCTAVE computing environment \cite{Eaton2022}. The open source python code PYSRIM \cite{pysrim} was employed to automate the SRIM calculations. All relevant data and code are available in \cite{mitsi_e_2023_8116031}.


 


\section{SRIM simulation conditions and data handling} \label{SRIMcondition}


\begin{table} [tb!]
	\centering
	\begin{tabular}{cccccc}
		\hline\hline
		Ion & H & He & Al & Fe & Au\\
		\hline
		$E_0$ (MeV) & 1 & 1 & 3 & 5 & 10\\ 
		\hline\hline	
	\end{tabular}
	\caption{Projectile ions and corresponding incident energies $E_0$.}
	\label{tab:ions}
\end{table}

\begin{table} [tb!]
	\centering
	\begin{tabular}{ccccccc}
		\hline\hline
		Target & Fe & Ni & Cu & Pd & W & Pt\\
		\hline
		$E_d$ (eV) \cite{Nordlund-2015-ID597} & 40 & 40 & 29 & 41 & 90 & 44 \\
		$b$ \cite{Nordlund-2018-Improvingatomicdis} & -0.568 & -1.01 & -0.68 & -0.88 & -0.56 & -1.12\\ 
		$c$ \cite{Nordlund-2018-Improvingatomicdis} & 0.286 & 0.23 & 0.16 & 0.15 & 0.12 & 0.11\\
		\hline\hline	
	\end{tabular}
	\caption{Displacement threshold, $E_d$, and arc-dpa model parameters, $(b, c)$, of simulated targets.}
	\label{tab:parameters}
\end{table} 

\renewcommand{\arraystretch}{1.75}

\begin{table*} [tb!]
	\centering
	\begin{threeparttable}[t]
		\begin{tabular}{lccc}
			\hline\hline
			Quantity & Symbol 
			& \parbox[t][1cm]{3cm}{
				Method 1 (M1)\\COLLISON.txt
			}
			& \parbox[t]{3cm}{ \centering
				Method 2 (M2)\\VACANCY.txt
			} 
			\\
			
			\hline
			
			PKAs per ion & $N_{\text{PKA}}$ 
			& $N_{\text{rows}} / N_{\text{ions}}$ 
			& $\sum_k{ \left[ \nu_i \right]_k\, \Delta x}$ \tnote{\dag} 
			\\
			
			\hline
			
			& &\multicolumn{2}{c}{NRT-dpa model} \\
			
			Displacements per ion & $N_{d}$ 
			& $ \langle \nu_d \rangle  \cdot N_{\text{PKA}} $
			& $\sum_k{\left[ \nu_i + \nu_r \right]_k\, \Delta x} $ \tnote{\dag} 
			\\
								
			Mean displacements per PKA & $ \langle \nu_d \rangle $ 
			& $N_{\text{rows}}^{-1}\,\sum_k{\left[ \nu_d\right]_k}$ \tnote{\ddag}
			& $ N_d / N_{\text{PKA}} $ 
			\\
			
			\hline
			
			& &\multicolumn{2}{c}{arc-dpa model} \\
			
			Displacements per ion & $N_{d,arc}$ 
			& \multicolumn{2}{c}{$\langle \nu_{d,arc} \rangle \cdot N_{\text{PKA}} $}\\
			
			Mean displacements per PKA & $ \langle \nu_{d,arc} \rangle $ 
			& $N_{\text{rows}}^{-1}\,\sum_k{
				\left[ \nu_d\right]_k \xi\left( \left[ \nu_d\right]_k \right)
				}$ \tnote{\ddag}
			& \parbox[c]{3cm}{
			\begin{center}
				eq. \eqref{eq:ndarc_approx} with $\langle \nu_d \rangle $\\as above
			\end{center}}
			\\

			\hline\hline	
		\end{tabular}
		
		
		\begin{tablenotes}
			\item [\dag]  $\left[ \nu_i \right]_k$ and $\left[ \nu_r \right]_k$ are the ``vacancies by ions'' 
			              and ``vacancies by recoils'', respectively, in the $k$-th target depth bin, with  
						  $\Delta x$ denoting the bin width.
			\item [\ddag]  $\left[ \nu_d \right]_k$ denotes the number of vacancies 
			               estimated by SRIM for the $k$-th PKA event. The sum is over all events.
			
		\end{tablenotes}
	\end{threeparttable}
	\caption{Calculation of damage parameters from SRIM output files}
	\label{tab:quantities}
\end{table*} 
\renewcommand{\arraystretch}{1.0}

All simulations were performed utilizing SRIM-2013 and employing the option "Ion distribution and Quick calculation of damage" (Q-C). Lattice and surface binding energies were set equal to zero according to the recommendation in \cite{Stoller-2013-ID110}. 
A range of projectile ions were employed, with atomic numbers varying from $Z=1$ (H) to 79 (Au)
and energies ranging from $E_0=1$ to 10 MeV, similarly to the work of \citet{Agarwal-2021-OntheuseofSRIMf}. The ions and corresponding energies are listed in
Table \ref{tab:ions}. Table \ref{tab:parameters} shows all the targets that we tested, which are essentially all materials whose arc-dpa parameters were estimated in \cite{Nordlund-2018-Improvingatomicdis}. Target
thickness was chosen appropriately in order to ensure that
the impinging ions stop within the examined
region. 
The target displacement energies, $E_d$, are based on internationally recommended standard values and are also given in Table \ref{tab:parameters}.
In the case of Fe self-ion irradiation, an extra simulation with $E_0=78.7$~keV was also performed in order to directly compare with results from \cite{Nordlund-2015-ID597}. For each ion/target combination 10,000 ion histories were run.

Damage parameters were extracted from the SRIM output files, either VACANCY.txt or COLLISON.txt. Table \ref{tab:quantities} lists all quantities of interest and the way they are calculated depending on the damage model and the output file used. 

The number of PKAs per ion, $N_{\text{PKA}}$, is obtained by integrating the 2nd data column of VACANCY.txt (``vacancies by ions'',  $\nu_i$) or by dividing the number of data rows, $N_{\text{rows}}$, in COLLISON.txt by the number of simulated ions, $N_{\text{ions}}$. $N_{\text{PKA}}$ is independent of the damage model. 

The NRT displacements per ion, $N_d$, is obtained as follows. In the case of VACANCY.txt, $N_d$ is found by summing the $2^{nd}$ and $3^{rd}$ column of the data table, i.e., ``vacancies by ions'', $\nu_i$, and "vacancies by recoils", $\nu_r$, respectively. 
Regarding the COLLISON.txt file, $N_d$ is calculated by adding up the "Target vacancies",  $\nu_d$, of all PKAs and dividing by $N_{\text{ions}}$. Finally, the average displacements per PKA, $\langle \nu_d \rangle$, is equal to $N_d / N_{\text{PKA}}$. 

The calculation of arc-dpa damage parameters is described in the next section.

All evaluations and the parsing of SRIM output files were performed in the OCTAVE computing environment \cite{Eaton2022}. The open source python code PYSRIM \cite{pysrim} was employed to automate the SRIM calculations. All relevant data and code are available in \cite{mitsi_e_2023_8116031}.

\section{Methods and Results}
In this section, we present the two different methods to obtain arc-dpa damage parameters from SRIM output.

\subsection{Method 1 (M1)}
This method utilizes the COLLISON.txt output file. In SRIM Q-C mode, this file lists all simulated PKA scattering events and reports, among other data, the number of displacements, $\nu_d$, generated per event. These $\nu_d$ values, labelled "Target vacancies", are calculated according to the NRT model, eq. \eqref{eq:NRT}, with the damage energy, $T_d$, obtained from the approximate LSS theory \cite{Ziegler-2008-ID880}. For the $\nu_d$ values in COLLISON.txt that satisfy $\nu_d > 1$, we can easily recover the LSS damage energy by multiplying $\nunrt$ with the cascade multiplication factor, $L$ (cf. eq. \eqref{eq:NRT}). Then, the obtained $T_d$ can be used in eq. \eqref{eq:arc} to evaluate the displacements according to the arc-dpa model. This is essentially what is done in M1, however, instead of actually evaluating $T_d$ we employ $\nunrt$ directly in the equivalent arc-dpa definition, eq. \eqref{eq:arc2}. Thus, the steps to calculate the arc-dpa parameters are as follows: 
\begin{enumerate}
	\item Run SRIM with the "Quick calculation of damage" (Q-C) option.
	\item Parse the COLLISON.txt output file to obtain the NRT displacements per PKA event, $\nunrt$.
	\item Calculate the corresponding $\nuarc$ per PKA from eq. \eqref{eq:arc2}, $\nuarc = \nunrt\cdot \xi(\nunrt)$.
	\item Take the average of the $\nuarc$ values to obtain the mean displacements per PKA according to the arc-dpa model, $\langle\nuarc\rangle$ (cf. Table \ref{tab:quantities}).
	\item Multiply by the number of PKAs per ion, $N_{PKA}$, to obtain the number of displacements per ion, $N_{d,arc} = \langle\nuarc\rangle \cdot N_{\text{PKA}}$
\end{enumerate}

M1 is very similar to the method proposed by \citet{Nordlund-2015-ID597}. The main difference lies in the derivation of damage energy. In \cite{Nordlund-2015-ID597}, $T_d$ is obtained by separate SRIM simulations employing the "Detailed Calculation with Full Damage Cascades" (F-C) option. In this case, SRIM utilizes detailed stopping power calculations for all secondary recoils in the PKA cascade, thus, the value of $T_d$ is potentially more accurate. \citet{Agarwal-2021-OntheuseofSRIMf} have made a detailed comparison of SRIM damage calculations in Q-C and F-C modes. They found that there is a difference of up to $\pm 25\%$ in the amount of NRT vacancies predicted by the two modes, when vacancy production is estimated by the SRIM damage energy. The authors attributed the difference to the use of the LSS approximation in Q-C mode. It is expected that also in the present case, where the arc-dpa damage estimation in M1 is based on the Q-C damage energy, there will be similar differences with respect to the procedure described in \cite{Nordlund-2015-ID597}, where the F-C mode was employed.  

To make a quantitative comparison of the two approaches, we repeated the simulation of 78.7 keV Fe ions incident on an Fe target that was employed in \cite{Nordlund-2015-ID597}. 
Table \ref{tab:comparison} shows the results from the two approaches. It is seen that there is only a small 2\% difference in the NRT parameters, $N_d$ and $\langle\nunrt\rangle$, obtained with the present method in comparison to the values reported in \cite{Nordlund-2015-ID597}, while the corresponding arc-dpa parameters almost coincide. We attribute the good agreement to the low damage energies occurring in this simulation.  
To have a more meaningful comparison, we simulated self-ion Fe irradiation with a much higher projectile energy, $E_0=5$~MeV, and evaluated the results with both our proposed method M1 and the one by \citet{Nordlund-2015-ID597}.  
In the latter case, we used the data from their fig. 1.2 to extend the interpolation of $T_d$ to target recoil energies up to 10~MeV. 
The results are also listed in Table \ref{tab:comparison}. As seen from the table, there is a 10\% difference between the NRT parameters obtained by our M1 and the evaluation according to \cite{Nordlund-2015-ID597}. This difference is comparable to the observations of \cite{Agarwal-2021-OntheuseofSRIMf} and thus can be attributed to the use of approximate LSS damage energy in the Q-C simulation mode. The corresponding arc-dpa parameters exhibit a similar but slightly lower difference of about 8\%. This is due to the fact that the arc-dpa efficiency lowers the significance of high energy damage events, where the errors due to the LSS approximation are more pronounced.

\begin{table*} [tb!]
	\centering
	\begin{threeparttable}[t]
	\begin{tabular}{lcccccc}
		\hline\hline
		{} & $E_0$ & $N_{\text{PKA}}$ & $N_{d}$ & $\langle\nunrt\rangle$ 
		& $N_{d, arc}$  &$\langle\nuarc\rangle$ \\
		\hline
		\citet{Nordlund-2015-ID597} & \multirow{3}{*}{78.7 keV} & \multirow{3}{*}{44.1} & 
		539 & 12.2\tnote{\dag} & 217 & 4.93\tnote{\dag} \\
		This study - Method 1 & {} & {} & 
		530 & 12.0 & 217 & 4.92 \\ 
		This study - Method 2 & {} & {} & 
		530 & 12.0 & 209 & 4.74 \\
		\hline
		Method of \citet{Nordlund-2015-ID597} & \multirow{3}{*}{5 MeV} & \multirow{3}{*}{442} & 
		8800 & 20.0 & 3150 & 7.14 \\
		This study - Method 1 & {} & {} & 
		7870 & 17.9 & 2900 & 6.56 \\ 
		This study - Method 2 & {} & {} & 
		7870 & 17.9 & 2890 & 6.54 \\
		\hline\hline	
	\end{tabular}

	\begin{tablenotes}
		\item [\dag] Mean values are calculated by dividing $N_d$ and $N_{d,arc}$ from \cite{Nordlund-2015-ID597} by $N_{\text{PKA}}$ as obtained in the present study.
		 
	\end{tablenotes}
	\end{threeparttable}

	\caption{Damage parameters obtained by different methods for the irradiation of an Fe target with Fe ions of energy $E_0$.}
	\label{tab:comparison}
\end{table*}



\subsection*{Method 2 (M2)}

% Figure environment removed

The objective of M2 is to provide a quick estimate of the arc-dpa damage parameters, without having to resort to the cumbersome processing of COLLISON.txt. For this, we note that from eq. \eqref{eq:arc2} the average arc-dpa can be written:
\begin{equation}\label{eq:arc3}
\langle\nuarc\rangle = (1-c)\langle\nunrt^{1+b}\rangle + c \cdot \langle\nunrt\rangle.
\end{equation}
Thus, to obtain $\langle\nuarc\rangle$ the value of $\langle\nunrt^{1+b}\rangle$ is needed. We performed an approximate calculation of this quantity, employing a power-law cross-section for the ion-target atom interaction and ignoring the effect of ionization losses, i.e., setting $T_d \approx T$. As shown in \ref{apdx1}, the following approximation
\begin{equation}\label{eq:nuapprox}
\langle\nunrt^{1+b}\rangle \approx \langle\nunrt\rangle^{\lambda (1+b)},
\end{equation}
where $ \lambda\approx 0.56 $, gives adequate results for a wide range of incident ion energies and ion-target combinations. 
This can be seen in fig. \ref{fig1}, where
$ \langle\nunrt^{1+b}\rangle $ 
is plotted as a function of 
$ \langle\nunrt\rangle^{1+b} $ for all the ion/target combinations simulated in the current work. The data shown in the figure have been obtained by taking the $\nunrt$ values per PKA event listed in COLLISON.txt and evaluating the required averages. As seen from the figure, the data from all simulated targets lie within $\pm 10\%$ of the approximate eq. \eqref{eq:nuapprox}, which is depicted by the dashed line.

Utilizing the above approximation, the arc-dpa damage parameters can be obtained by the following prescription:
\begin{enumerate}
	\item Run SRIM with the "Quick calculation of damage" (Q-C) option.
	\item Calculate the NRT-$\langle \nunrt \rangle$ from VACANCY.txt as described in Table \ref{tab:quantities}.
	\item Obtain $\langle\nuarc\rangle$ from eq. \eqref{eq:arc3}, substituting the approximate relation \eqref{eq:nuapprox}:
	\begin{equation}\label{eq:ndarc_approx}
	\langle\nuarc\rangle \approx (1-c)\langle\nunrt\rangle^{0.56(1+b)} + c \cdot \langle\nunrt\rangle
	\end{equation}
	\item The number of displacements per ion is $\langle\nu_{d,arc}\rangle \cdot N_{\text{PKA}}$
\end{enumerate}

Fig. \ref{fig2} depicts the ratio of $\langle\nuarc\rangle$ calculated by the two methods, M2 and M1, respectively, for all simulated ion/target combinations. It is seen that the results of the approximate method M2 deviate by at most 3\% from those of M1.
A similar small deviation can be observed in Table \ref{tab:comparison} between the arc-dpa damage parameters obtained by methods M1 and M2 in the two simulated Fe self-ion irradiations. In the low energy case the arc-dpa parameters obtained by M2 are 4\% lower than those of M1 while in the high energy example the two methods produce essentially equivalent results. Thus, the method M2 can be employed for a quick, approximate evaluation of arc-dpa damage, introducing an error of not more than a few percent compared to the more detailed method M1.

% Figure environment removed

\section{Conclusions}

In this work, we present two methods for evaluating arc-dpa damage parameters in ion irradiations employing the SRIM simulation code with the option ``Quick calculation of damage'' (Q-C). 

The first method is based on SRIM's COLLISON.txt output file, which lists the NRT displacements, $\nunrt$, produced in each simulated primary knock-on atom (PKA) recoil event. The $\nunrt$ values are converted to the corresponding arc-dpa model prediction, $\nuarc$, by means of eq. \eqref{eq:arc2} and then averaged to obtain the total damage parameters. This procedure is similar to the one proposed by \citet{Nordlund-2015-ID597} only in our case the damage energy is essentially obtained by the LSS approximation employed in SRIM's Q-C mode, whereas in \cite{Nordlund-2015-ID597} the damage energy was interpolated from the results of separate detailed SRIM simulations. Thus, our method gains in simplicity but can lead to errors due to the approximation in the damage energy calculation. The errors in the estimated damage could be up to $\sim 30\%$ \cite{Agarwal-2021-OntheuseofSRIMf}. 

In the second method, we devise an approximate relation, which gives $\langle\nuarc\rangle$ directly as a function of $\langle\nunrt\rangle$. Thus, the cumbersome processing of the COLLISON.txt file is not needed since the NRT damage parameter $\langle\nunrt\rangle$ can be easily obtained from VACANCY.txt. 
We found that the arc-dpa parameters obtained by this approximate method differ by not more than a few percent from those calculated by the first method. 


\section*{Acknowledgement}

This work has been carried out within the framework of the EUROfusion Consortium, funded by the European Union via the Euratom Research and Training Programme (Grant Agreement No 101052200 — EUROfusion). Views and opinions expressed are however those of the author(s) only and do not necessarily reflect those of the European Union or the European Commission. Neither the European Union nor the European Commission can be held responsible for them. The funding from the Hellenic General Secretariat for Research and Innovation for the Greek National Programme of the Controlled Thermonuclear Fusion is also acknowledged.

\begin{comment}
\section{System Architecture}
\label{appendix:architecture}
\system has a novel modularized system architecture with three key components: 
\emph{StreamManager}, 
\emph{TxnManager} and \emph{TxnScheduler}. 
These components are instantiated in each thread locally.
The execution outline of \system is presented in Algorithm~\ref{alg:algo}.
Transactional stream processing is continuous and potentially never ends (Line 1$\sim$8).
The dependency resolution and execution of state transactions are separated into two non-overlapping phases by punctuations~\cite{Tucker:2003:EPS:776752.776780} (Line 2 and 5), which guarantees that no subsequent input event will have a smaller timestamp. 
Effectively, a batch of state transactions is collected during the first phase, and processed during the second phase.

In the first phase (i.e., stream processing phase), 
the \emph{StreamManager} conducts preprocessing for every input event ($e$). Similar to some prior works~\cite{tstream}, state transactions may be issued but not immediately processed during preprocessing (Line 3).
The \emph{pre\_processing} and \emph{post\_processing} functions are exposed as APIs to users.
The \emph{TxnManager} handles dependency resolution (Line 4) among state transactions and insert decomposed operations to construct a \tpg. We discuss the detailed two-phase \tpg construction process in Section~\ref{subsec:construction}.

In the second phase  (i.e., transaction processing phase), 
the \emph{TxnManager} is first involved again to refine (Line 6) the constructed \tpg with further dependency resolution.
The \emph{TxnScheduler} 
schedules operations for concurrent execution based on the constructed \tpg according to the three dimensions of scheduling decisions (Line 7). 
In particular, a scheduling decision model $M$ is instantiated based on the constructed \tpg (Line 14).
\textbf{\circled{1}} Guided by $M$, execution threads adopt an exploration strategy (Section~\ref{subsec:explore}) to explore the constructed \tpg for operations available to be scheduled constrained by dependencies. 
\textbf{\circled{2}} 
During exploration, one or multiple operations may be treated as the 
% basic 
unit of scheduling (Section~\ref{subsec:granularity}). 
Subsequently, \textbf{\circled{3}} every thread executes operation(s) in the unit of scheduling with various abort handling mechanisms (Section~\ref{subsec:abort_handling}).
Only when state transactions are processed (i.e., committed or aborted) can the associated input events be postprocessed (Line 8) by the \emph{StreamManager} based on transaction processing results.
\end{comment}

\begin{comment}
\begin{algorithm}
\footnotesize
    \KwData{$e$ \tcp{Input event}}
    \KwData{$txn_{ts}$ \tcp{State transaction}}
    \KwData{$G$ \tcp{The currently constructed TPG}}
    \While{!finish processing of input streams}{
        \eIf(\tcp*[h]{Phase 1}){\text{$e$ is not a $punctuation$}}{
                $txn_{ts}$ $\gets$ PRE\_Processing($e$)\;
                \textbf{TPG\_Construction}($G$, $txn_{ts}$)\; 
          }(\tcp*[h]{Phase 2}){
                \textbf{TPG\_Refinement}($G$)\; 
                \textbf{TXN\_Scheduling}($G$)\; 
                POST\_Processing()\;
          }
    }
    
    \SetKwFunction{FMain}{TPG\_Construction}
    \SetKwProg{Fn}{Function}{:}{}
    \Fn{\FMain{$G$, $txn_{ts}$}}{
        $O_{1..k}$ $\gets$ \textbf{Partition} $txn_{ts}$\;
        \ForEach{\text{operation $O_{i}$ $\in$ $O_{1..k}$}}{
            \textbf{Identify} its \ld\;
            $G$ $\gets$ $G$ + $O_{i}$ \;
        }
    }
    \SetKwFunction{FMain}{TPG\_Refinement}
    \SetKwProg{Fn}{Function}{:}{}
    \Fn{\FMain{$G$}}{
        \ForEach{\text{vertex $e_{i}$ $\in$ $G$}}{
            \textbf{Identify} its \td, \pd\;
        }
    }
    
    \SetKwFunction{FMain}{TXN\_Scheduling}
    \SetKwProg{Fn}{Function}{:}{}
    \Fn{\FMain{$G$}}{
        $M$ $\gets$ Instantiated with $G$;\tcp{A decision model}
        \While{!finish scheduling of $G$
        }{
          \textbf{\circled{2}} $Scheduling Unit$ $\gets$ \textbf{\circled{1}} \emph{Explore}($G$, $M$)\; 
            \textbf{\circled{3}} \emph{Execute with Abort Handling} ($Scheduling Unit$)\; 
        }
    }
  \caption{Execution Outline of \system}
  \label{alg:algo}
\end{algorithm}
\end{comment} 

%\section{BibTeX}
%The bibliography is handled automatically with the BibTeX system.
%
%The file \texttt{srim.bib} contains all papers from project "SRIM/Radiation damage".
%
%Citation to papers can be done either author-number:
%
%The arc-dpa model has been formulated by  \citet{Nordlund-2018-Improvingatomicdis} 
%
%or just numeric: 
%
%The arc-dpa model has been formulated in \citep{Nordlund-2018-Improvingatomicdis}.
%
%For bulleted lists use:
%\begin{itemize}
%	\item One
%	\item Two
%\end{itemize}





%% For citations use: 
%%       \citet{<label>} ==> Jones et al. [21]
%%       \citep{<label>} ==> [21]
%%

%% If you have bibdatabase file and want bibtex to generate the
%% bibitems, please use
%%
\bibliographystyle{elsarticle-num-names} 
\bibliography{srim-arc,pysrim,zenodo}

%% else use the following coding to input the bibitems directly in the
%% TeX file.


\end{document}

\endinput
%%
%% End of file `elsarticle-template-num-names.tex'.
