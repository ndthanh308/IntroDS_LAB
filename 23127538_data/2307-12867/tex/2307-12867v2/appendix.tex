%% The Appendices part is started with the command \appendix;
%% appendix sections are then done as normal sections
\appendix 
\section{Approximation of $\langle \nu_d^{1+b} \rangle$}
\label{apdx1}
The general expression for the average $\langle \nu_d^n \rangle$ is given by 
\begin{equation}\label{eq:gennu}
\langle\nu_d^n\rangle = \frac {\int_{Ed}^{T_{m}} [\nu_d(T_d)]^n \,  d \sigma(E,T)}{\int_{Ed}^{T_{m}} d \sigma(E,T)},
\end{equation}
where $d\sigma(E,T)$ denotes the cross-section for scattering of an ion with initial energy $E$ producing a PKA with recoil energy $T$. $T_m$ is the maximum PKA recoil energy. Making the following assumptions:
\begin{enumerate}[(i)]
	\item A power-law cross-section, $d\sigma(E,T) \propto dT/T^{1+p}$, where $p$ ranges from 0.5 (heavy ions) to 1 (light ions) \cite{Was-2017-ID904}
	\item Ionization losses can be ignored ($T_d \approx T$)
\end{enumerate}
and performing the integrations in eq. \eqref{eq:gennu} we obtain the following analytical expression: 
\begin{equation}\label{eq:nu}
\langle\nu_d^n\rangle	= 
\frac{(L/E_d)^{p} - 1 + \frac{p}{p-n}\left[ 1 - (L/T_m)^{p-n} \right]}
{(L/E_d)^{p} - (L/T_m)^{p}} \; ,
\end{equation}
which is valid for $T_m\geq L$ and $n\neq p$. In the special case $n=p$ it becomes
\begin{equation}\label{eq:nu1}
\langle\nu_d^n\rangle	= 
\frac{(L/E_d)^{n} - 1 - n \log(L/T_m) }
{(L/E_d)^{n} - (L/T_m)^{n}} \; .
\end{equation}

Based on eqs. \eqref{eq:nu}-\eqref{eq:nu1} we calculate $\langle\nu_d^{1+b}\rangle$ for several representative $(b,p)$ combinations and for $T_m$ values in the range $L < T_m < 10^4 L$. This corresponds to a maximum $T_m$ of $\sim 10^6$~eV in Fe and similar values for other metals. The results are shown in fig. \ref{fig:nu_theory} as a function of $\langle\nu_d\rangle^{1+b}$ in a double-logarithmic plot. $\langle\nu_d\rangle$ is also obtained from \eqref{eq:nu}.
It is apparent from the figure that all curves follow roughly a central line. 
Fitting a power law of the form: 
\begin{equation}\label{eq:powerlaw}
\langle\nu_d^{1+b}\rangle \approx 
A \, \langle\nu_d\rangle^{\lambda\, (1+b)},
\end{equation}
to the data, with $A$ and $\lambda$ as adjustable parameters, we obtain the values $\lambda \approx 0.56$ and $A \approx 1.0$. This is denoted by the dashed line in fig. \ref{fig:nu_theory}. The deviation of the analytically calculated $\langle\nu_d^{1+b}\rangle$ from the fitted power law is within $\pm 20\%$, which corresponds to the shaded area in fig. \ref{fig:nu_theory}. 
% Figure environment removed
