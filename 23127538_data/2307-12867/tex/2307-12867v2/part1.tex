\section{Introduction}

In studies of radiation effects in materials it is generally desirable to have a standardized parameter to quantify radiation damage exposure, that would provide a
common basis for comparison of data obtained under different irradiation conditions in terms of impinging particle type and energy. 
Currently, the internationally accepted standard parameter for this purpose is the number of displacements per atom (dpa) 
calculated according to the Norgett-Robinson-Torrens (NRT) model \cite{Norgett-1975-ID480}. 
Recently, a modification to the NRT model has been proposed, the athermal recombination corrected dpa (arc-dpa) \cite{Nordlund-2015-ID597, Nordlund-2018-Improvingatomicdis, Nordlund-2018-ID1137}. It addresses a well known issue of NRT, namely, the overestimation of the number of stable defects generated by high energy displacement cascades. The arc-dpa model is based on evidence from experimental studies and computer simulations, which indicates that significant defect recombination takes place during the cascade cool-down phase leading to reduced numbers of remaining stable defects.
In its present formulation, the model can be applied to a limited set of monoatomic metallic target materials, for which the required material-specific model parameters have been obtained. It is expected that in the future, as new experimental and simulation data becomes available, the arc-dpa model will be applicable to a wider range of metallic materials including both dilute and concentrated alloys \cite{Nordlund-2018-Improvingatomicdis}.

There are several software tools available for estimating radiation damage exposure. In the case of ion irradiation, one of the most widely used such tools is the Monte Carlo code SRIM (Stopping and Range of Ions in Matter) \cite{Ziegler-2010-ID479}. Its popularity is based on the fact that it employs accurate ion stopping powers and provides a user-friendly interface. Several authors \cite{Stoller-2013-ID110,Li-2015-IM3DAparallelMon,Weber-2019-ID1179,Crocombette-2019-Quickcalculationof,Stoller-2019-ID1182,Agarwal-2021-OntheuseofSRIMf}
have discussed the application of SRIM for accurate damage calculations. The program offers different options for the simulation of damage events. Many authors have noted that the option ``Quick calculation of damage'' (Q-C) may be preferable in order to obtain results comparable to the NRT model \cite{Stoller-2013-ID110,Weber-2019-ID1179}. However, the Q-C mode implements certain approximations and thus may be less accurate, especially for multi-elemental targets \cite{Weber-2019-ID1179,Crocombette-2019-Quickcalculationof}. On the other hand, the more detailed ``Full damage cascades'' (F-C) option tends to significantly overestimate damage production compared to the Q-C mode \cite{Stoller-2013-ID110,Agarwal-2021-OntheuseofSRIMf}. \citet{Agarwal-2021-OntheuseofSRIMf} suggested employing SRIM in F-C mode to obtain the average of the damage energy, $T_d$, i.e., the ion energy deposited to target displacements. The average $T_d$ is then inserted in the NRT formula to calculate the average damage produced. This is most accurate at high damage energies, where the NRT damage function is linear with respect to $T_d$.

Regarding the arc-dpa model, there is currently no standardized way to compute damage exposure in ion irradiations as the model has not yet been implemented in any of the widely used software tools. 
Since the arc-dpa damage function is strongly non-linear, knowledge of just the average value of $T_d$ is not enough to correctly estimate the damage, as done, e.g., for NRT-dpa with the SRIM damage energy method of \citet{Agarwal-2021-OntheuseofSRIMf}.  
In their original publication introducing the new damage model, \citet{Nordlund-2015-ID597} already discussed the application of SRIM for indirect estimation of arc-dpa damage parameters. 
They presented a method of calculation in two steps. 
First, a series of SRIM simulations were performed to evaluate $T_d$ as a function of the initial PKA recoil energy, $E_R$, for a given target material. In \cite{Nordlund-2015-ID597}, this was done for Fe and an interpolating function was devised to obtain $T_d$ continuously as a function of recoil energy up to 300~keV. 
In the second part of the calculation, the information obtained on $T_d$ is used for post-processing the SRIM output file "COLLISON.txt" to finally obtain the arc-dpa values.

In this paper, we propose two alternative methods to calculate arc-dpa exposure using SRIM. 
They are based on the ``Quick calculation of damage'' (Q-C) option, which provides a NRT-compatible damage estimate as a starting point. Since the arc-dpa model refers currently only to monoatomic metals, the known limitation of the Q-C mode regarding multi-elemental targets is not relevant.   
The first of the proposed methods utilizes the SRIM output file COLLISON.txt as previously discussed in \cite{Nordlund-2015-ID597}. However, instead of separately computing $T_d$ by interpolation, we use the damage energy values that are internally calculated by SRIM with the Lind­hard-Scharff-Schiøtt (LSS) approximation \cite{Lindhard-1963-RANGECONCEPTSANDH}. Thus, the damage energy interpolation for different target materials is not required. The second method is based on an approximate formula that we propose, which can be employed for direct estimation of arc-dpa exposure based on the corresponding NRT-dpa value. Thus, the cumbersome handling of the COLLISON.txt file is avoided. The two methods are tested on all targets for which arc-dpa model parameters are available and for a range of projectile ions. 

\section{Radiation Damage Models}
The NRT model gives the number of stable displacements, $\nunrt$, produced by a PKA recoil with damage energy $T_d$ as:
\begin{equation}\label{eq:NRT}
\nunrt(T_d)=
	\begin{cases}
		  0             & \text{for } T_d \le E_d\\
		  1 			& \text{for } E_d < T_d \le L\\
T_d/L & \text{for } T_d > L
	\end{cases}	
\end{equation} \\
where $E_d$ is the displacement threshold energy, i.e., the minimum energy required to displace an atom from its lattice position. $L=2E_d/0.8$ denotes the cascade multiplication threshold above which more than one stable displacements are generated by the PKA.

In the arc-dpa model, the 3$^{rd}$ branch of \eqref{eq:NRT} is multiplied by an energy dependent efficiency factor, $\xi \leq 1$. The model definition is summarized in the following two relations:
\begin{equation}\label{eq:arc}
	\nuarc(T_d)=
	\begin{cases}
		0             & \text{for } T_d \le E_d,\\
		1 			& \text{for } E_d < T_d \le L,\\
	\xi(T_d/L)\cdot T_d/L & \text{for } T_d>L,
	\end{cases}	
\end{equation}
\begin{equation}\label{eq:xi}
	\xi(x) = (1-c)\, x^b + c, \quad \text{for } x \geq 1.
\end{equation}
The parameters \textit{b} and \textit{c} are material constants that have been determined for a number of target materials by \citet{Nordlund-2018-Improvingatomicdis}. Their values are given in table \ref{tab:parameters}.

We note that for damage energies above the displacement threshold, $T_d>E_d$, $\nuarc(T_d)$ can be compactly written as
\begin{equation}\label{eq:arc2}
\nuarc(T_d) = \nunrt(T_d)\cdot \xi\left[ \nunrt(T_d) \right].
\end{equation}
This definition will be utilized in the following paragraphs.


\section{SRIM simulation conditions and data handling} \label{SRIMcondition}


\begin{table} [tb!]
	\centering
	\begin{tabular}{cccccc}
		\hline\hline
		Ion & H & He & Al & Fe & Au\\
		\hline
		$E_0$ (MeV) & 1 & 1 & 3 & 5 & 10\\ 
		\hline\hline	
	\end{tabular}
	\caption{Projectile ions and corresponding incident energies $E_0$.}
	\label{tab:ions}
\end{table}

\begin{table} [tb!]
	\centering
	\begin{tabular}{ccccccc}
		\hline\hline
		Target & Fe & Ni & Cu & Pd & W & Pt\\
		\hline
		$E_d$ (eV) \cite{Nordlund-2015-ID597} & 40 & 40 & 29 & 41 & 90 & 44 \\
		$b$ \cite{Nordlund-2018-Improvingatomicdis} & -0.568 & -1.01 & -0.68 & -0.88 & -0.56 & -1.12\\ 
		$c$ \cite{Nordlund-2018-Improvingatomicdis} & 0.286 & 0.23 & 0.16 & 0.15 & 0.12 & 0.11\\
		\hline\hline	
	\end{tabular}
	\caption{Displacement threshold, $E_d$, and arc-dpa model parameters, $(b, c)$, of simulated targets.}
	\label{tab:parameters}
\end{table} 

\renewcommand{\arraystretch}{1.75}

\begin{table*} [tb!]
	\centering

	\begin{adjustbox}{max width=\textwidth}
	\begin{threeparttable}[t]
		
		\begin{tabular}{lccc}
			\hline\hline
			Quantity & Symbol 
			& \parbox[c][1cm]{3cm}{
				Method 1 (M1)\\COLLISON.txt
			}
			& \parbox[c]{3cm}{ \centering
				Method 2 (M2)\\VACANCY.txt
			} 
			\\
			
			\hline
			
			PKAs per ion & $N_{\text{PKA}}$ 
			& $N_{\text{rows}} / N_{\text{ions}}$ 
			& $\sum_k{ \left[ \nu_i \right]_k\, \Delta x}$ \tnote{\dag} 
			\\
			
			\hline
			
			& &\multicolumn{2}{c}{NRT-dpa model} \\
			
			Displacements per ion & $N_{d}$ 
			& $N_{\text{ions}}^{-1}\,\sum_k{\left[ \nu_d\right]_k}$ \tnote{\ddag}
			& $\sum_k{\left[ \nu_i + \nu_r \right]_k\, \Delta x} $ \tnote{\dag} 
			\\
								
			Mean displacements per PKA & $ \langle \nu_d \rangle $ 
			& \multicolumn{2}{c}{$ N_d / N_{\text{PKA}}$}\\
			
			\hline
			
			& &\multicolumn{2}{c}{arc-dpa model} \\
			
			Displacements per ion & $N_{d,arc}$ 
			& \multicolumn{2}{c}{$\langle \nu_{d,arc} \rangle \cdot N_{\text{PKA}} $}\\
			
			Mean displacements per PKA & $ \langle \nu_{d,arc} \rangle $ 
			& $N_{\text{rows}}^{-1}\,\sum_k{
				\left[ \nu_d\right]_k \xi\left( \left[ \nu_d\right]_k \right)
				}$ \tnote{\ddag}
			& \parbox[c]{3cm}{
			\begin{center}
				eq. \eqref{eq:ndarc_approx} with $\langle \nu_d \rangle $\\as above
			\end{center}}
			\\

			\hline\hline	
		\end{tabular}
		
		
		
		\begin{tablenotes}
			\item [\dag]  $\left[ \nu_i \right]_k$ and $\left[ \nu_r \right]_k$ are the ``vacancies by ions'' 
			              and ``vacancies by recoils'', respectively, in the $k$-th target depth bin, with  
						  $\Delta x$ denoting the bin width.
			\item [\ddag]  $\left[ \nu_d \right]_k$ denotes the number of vacancies 
			               estimated by SRIM for the $k$-th PKA event. The sum is over all events.
			
		\end{tablenotes}

		

	\end{threeparttable}
	\end{adjustbox}
	\caption{Calculation of damage parameters from SRIM output files}
	\label{tab:quantities}
\end{table*} 
\renewcommand{\arraystretch}{1.0}

All simulations were performed utilizing SRIM-2013 and employing the option "Ion distribution and Quick calculation of damage" (Q-C). 
% Lattice and surface binding energies were set equal to zero according to the recommendation in \cite{Stoller-2013-ID110}. 
A range of projectile ions were employed, with atomic numbers varying from $Z=1$ (H) to 79 (Au)
and energies ranging from $E_0=1$ to 10 MeV, similarly to the work of \citet{Agarwal-2021-OntheuseofSRIMf}. The ions and corresponding energies are listed in
Table \ref{tab:ions}. Table \ref{tab:parameters} shows all the targets that we tested, which are essentially all materials whose arc-dpa parameters were estimated in \cite{Nordlund-2018-Improvingatomicdis}. Target
thickness was chosen appropriately in order to ensure that
the impinging ions stop within the examined
region. 
The target displacement energies, $E_d$, are based on internationally recommended standard values and are also given in Table \ref{tab:parameters}.
In the case of Fe self-ion irradiation, an extra simulation with $E_0=78.7$~keV was also performed in order to directly compare with results from \cite{Nordlund-2015-ID597}. For each ion/target combination 10,000 ion histories were run.

Damage parameters are extracted from the SRIM output files, either VACANCY.txt or COLLISON.txt. Table \ref{tab:quantities} lists all quantities of interest and the way they are calculated depending on the damage model and the output file used. 

The number of PKAs per ion, $N_{\text{PKA}}$, is obtained by integrating the 2nd data column of VACANCY.txt (``vacancies by ions'',  $\nu_i$) or by dividing the number of data rows, $N_{\text{rows}}$, in COLLISON.txt by the number of simulated ions, $N_{\text{ions}}$. $N_{\text{PKA}}$ is independent of the damage model. 

The NRT displacements per ion, $N_d$, is obtained as follows. In the case of VACANCY.txt, $N_d$ is found by summing the $2^{nd}$ and $3^{rd}$ column of the data table, i.e., ``vacancies by ions'', $\nu_i$, and "vacancies by recoils", $\nu_r$, respectively, as suggested by previous authors \cite{Stoller-2013-ID110,Nordlund-2015-ID597}. 
Regarding the COLLISON.txt file, $N_d$ is calculated by adding up the "Target vacancies",  $\nu_d$, of all PKAs and dividing by $N_{\text{ions}}$. Finally, the average displacements per PKA, $\langle \nu_d \rangle$, is equal to $N_d / N_{\text{PKA}}$. 

The calculation of arc-dpa damage parameters is described in the next section.

All evaluations and the parsing of SRIM output files were performed in the OCTAVE computing environment \cite{Eaton2022}. The open source python code PYSRIM \cite{pysrim} was employed to automate the SRIM calculations. All relevant data and code are available in \cite{mitsi_e_2023_8116031}.


 

