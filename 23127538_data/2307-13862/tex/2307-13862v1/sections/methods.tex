Here we describe our methodology for estimating the confidence region of a cosmological data-set using Neyman construction. To aid the reader, a glossary of terms used throughout is provided in Table~\ref{tab:glossary}.

\begin{table}
    \centering
    \begin{tabular}{p{0.15\linewidth}|p{0.75\linewidth}}
        Parameter & Description  \\
        $\Omega_{M}^{best}$, $w^{best}$ & The best-fitting output cosmology for our experiment data-set. Described in Section~\ref{sec:experiment}. \\
        $\Omega_{M}'$, $w'$ & A strategically selected input cosmology from which 150 realisations will be drawn. Described in Section~\ref{sec:percentile}. \\
        $\vec{\Omega_{M}}$, $\vec{w}$ & A distribution of 150 best-fitting cosmologies, produced by processing the 150 realisations of $\Omega_{M}'$, $w'$ with Pippin. Described in Section~\ref{sec:percentile}. \\
        $w^{*}(\Omega_{M})$ & A one dimensional function that approximates $\vec{\Omega_{M}}$, $\vec{w}$. Found by fitting a Gaussian process through $\vec{\Omega_{M}}$, $\vec{w}$. Described in Section~\ref{sec:ellipse}.
    \end{tabular}
    \caption{A glossary of terms used in our methodology, which are defined throughout the text}
    \label{tab:glossary}
\end{table}

For a given input cosmology, the Neyman construction provides a prescription for using simulations to calculate the percentile contour, or the boundary of a confidence region, that input cosmology lies on. By calculating these percentile contours for a grid of input cosmologies, the confidence region can be estimated as the set of input cosmologies which lie on percentile contours less than or equal to the desired confidence level. The extensive compute time of supernova simulations makes it difficult to densely sample the parameter space, so we instead create an approximate Neyman construction by strategically choosing a small number of input cosmologies at representative locations on the 68\% contour.

In Section~\ref{sec:percentile} we describe how to calculate the percentile contour for a single cosmological input. In Section~\ref{sec:confidence_region} we describe how we find cosmological inputs which lie on the 68\% percentile contour, and how we estimate the confidence region.

\subsection{Calculating the percentile contour}\label{sec:percentile}
To calculate the percentile contour at any given cosmology ($\Omega_{M}'$, $w'$, which as noted above is chosen to lie at representative locations of the 68\% contour), we simulate 150 SNe Ia data-sets with $\Omega_{M}'$, $w'$ as the true cosmological input, using the procedure described in Section~\ref{sec:experiment}. \added{Each data-set is produced with a different random seed, allowing for statistical fluctuations between realisations.} We analyse each of these data-sets with Pippin to produce a distribution of best fitting cosmologies ($\vec{\Omega_{M}}$, $\vec{w}$) in the space of measured $\Omega_{M}$ and $w$. The Neyman construction predicts that the percentile contour for $\Omega_{M}'$, $w'$ is the percentage of these best fitting cosmologies encompassed by a coverage ellipse of this distribution. The coverage ellipse is defined to be centered on $\Omega_{M}'$, $w'$ and to intersect $\Omega_{M}^{best}$, $w^{best}$, which were calculated in Section~\ref{sec:experiment}, Equation~\ref{eq:best}. This ellipse represents the probability of a data-set with true cosmology $\Omega_{M}'$, $w'$ having a best fitting cosmology equal to $\Omega_{M}^{best}$, $w^{best}$. Figure~\ref{fig:Likelihood} shows an example of this calculation for $\Omega_{M}'=0.188$, $w'=-0.783$, which lies at one extreme end of the 68\% contour we are testing. \added{Figure~\ref{fig:Hubble} presents an example Hubble Diagram for both $\Omega_{M}^{best}$, $w^{best}$ and $\Omega_{M}'$, $w'$.}

% Figure environment removed

% Figure environment removed
\subsubsection{Fitting the coverage ellipse}\label{sec:ellipse}
The distribution of best fitting cosmologies about $\Omega_{M}'$, $w'$ usually follows the "banana" distribution that is typical of supernova cosmology, and is due to the inherent degeneracy between $\Omega_{M}$ and $w$. This contour shape makes it difficult to determine an accurate coverage ellipse around this distribution. To determine a coverage ellipse, we first fit a Gaussian Process (GP) through $\vec{\Omega_{M}}$, $\vec{w}$ to produce the one-dimensional function $w^{*}(\Omega_{M})$, which approximates $w$ as a function of $\Omega_{M}$ in the plane of $\vec{\Omega_{M}}$, $\vec{w}$. Next we subtract $w^{*}(\vec{\Omega_{M}})$ from $\vec{w}$ to transform the distribution of best fitting cosmologies into a more elliptical distribution. We fit a coverage ellipse to this transformed distribution which is centered on $\Omega_{M}'$, $w'$ and intersects $\Omega_{M}^{best}$, $w^{best}$.

The percentile contour for $\Omega_{M}'$, $w'$ is then the percentage of the best fitting cosmologies covered by this coverage ellipse. Figure~\ref{fig:GPE} shows an example of this transformation and ellipse fitting technique for $\Omega_{M}'=0.188$, $w'=-0.783$, the same cosmology as shown in Figure~\ref{fig:Likelihood}.

The uncertainty in the computed percentage is estimated by performing 1000 bootstrap resamples of the distribution of best fitting cosmologies, and typically results in an uncertainty in the percentile contour of $\pm4\%$.

% Figure environment removed

\subsection{Estimating the confidence region}\label{sec:confidence_region}
We now have a statistically rigorous method of calculating the percentile contour over the cosmological parameter space. Using this method, we could compute a Neyman construction by computing the percentile contour across a grid that covers the cosmological parameter space, and from this determine a confidence region. However, each evaluation of the percentile contour requires 150 simulated data-sets, which is computationally expensive.
To more efficiently determine a confidence region, we develop an approximate Neyman construction method which requires far fewer simulations. Instead of evaluating the percentile contour over the entire cosmological parameter space, we use the bisection method to iteratively find the input cosmologies that lie on the 68\% percentile contour. These cosmologies define the edge of the 68\% confidence region.

We first calculate the percentile contour for cosmologies at several representative locations on the 68\% contour. We select two input cosmologies which are at the furthest extent of the 68\% contour, and two input cosmologies which are at the closest region of the 68\% contour to $\Omega_{M}^{best}$, $w^{best}$. This enables us to probe the consistency of the 68\% contour across its entire span. The cosmological inputs used in defining the approximate Neyman construction, and the percentile contour for each input are shown in Figure~\ref{fig:Neyman}, and detailed in Table~\ref{tab:InputPercentile}. Inputs 1a, 1b, 1c, and 1d were chosen to lie on the 68\% contour of the original experiment cosmology posterior. A second set of inputs (2a, 2b, 2c, and 2d) were chosen to compensate for how far the previous set of coverage ellipses differed from 68\% coverage, as described below.

% Figure environment removed

If the input cosmologies lie at a percentile contour of $<68\%$ confidence, we select a new cosmology further from $\Omega_{M}^{best}$, $w^{best}$, and conversely select a cosmology closer to $\Omega_{M}^{best}$, $w^{best}$ if the initial percentile contour is $>68\%$ confidence. This allows us to find an input cosmology which lies on a percentile contour within one standard deviation of 68\%, as measured by bootstrap resampling. Though the iterative method typically converges to a percentile contour within one standard deviation of $68\%$ within 2 or 3 iterations, converging to exactly 68\% would take significantly more iterations. As such, once we are within one standard deviation of 68\%, we linearly interpolate or extrapolate to find an input cosmology which lies exactly on the 68\% percentile contour.

Figure~\ref{fig:Approximate} shows this approximate Neyman Construction technique for $\Omega_{M}'=0.188$, $w'=-0.783$, the same cosmology shown in Figures~\ref{fig:Likelihood} and~\ref{fig:GPE}. This cosmology was found to lie on the $46\%\pm4\%$ percentile contour, and after iteration the cosmology $\Omega_{M}'=0.145$, $w'=-0.725$ was found to lie on the $65\%\pm4\%$ percentile contour, which is within one standard deviation of 68\%. We linearly extrapolated from these two input cosmologies to find that $\Omega_{M}'=0.138$, $w'=-0.716$ lay on the 68\% percentile contour. Figure~\ref{fig:Ellipse} shows the coverage ellipses fit to each cosmological input.

% Figure environment removed

\begin{table}
    \centering
    \begin{tabular}{l|l|l|l}
         Cosmology          & Input         & Input         & Percentile            \\
         Input              & $\Omega_{M}$  & $w$           & Contour               \\\hline
         Experiment         & 0.3           & -1.0          & -                     \\
         1a                 & 0.188         & -0.783        & $46\%\pm4\%$          \\
         1b                 & 0.38          & -1.25         & $74\%\pm4\%$          \\
         1c                 & 0.307         & -0.977        & $47\%\pm4\%$          \\
         1d                 & 0.292         & -1.02         & $62\%\pm4\%$          \\
         2a                 & 0.145         & -0.725        & $65\%\pm4\%$          \\
         2b                 & 0.37          & -1.204        & $66\%\pm4\%$          \\
         2c                 & 0.3075        & -0.9765       & $67\%\pm4\%$          \\
         2d                 & 0.2917        & -1.0205       & $70\%\pm4\%$
    \end{tabular}
    \caption{The input $\Omega_{M}$ and $w$ values for the experiment and approximate Neyman construction input cosmologies, as well as the percentile contour each cosmological input lies on.}
    \label{tab:InputPercentile}
\end{table}

% Figure environment removed