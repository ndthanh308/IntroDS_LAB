In the previous section we describe how we computed the location of the 68\% confidence region at four strategically selected cosmological inputs in the $\Omega_{M}-w$ plane using an approximate Neyman construction. The results are presented in Table~\ref{tab:PercentDiff} and shown in Figure~\ref{fig:Final}.

We see the largest difference between the experiment cosmology 68\% contour and the 68\% confidence region in the first input cosmology, which lies in the top left quadrant of the cosmological contour. We find a shift of $\sim$0.05 in $\Omega_{M}$ and a shift of $\sim$0.07 in $w$. The second input cosmology, which lies in the bottom right quadrant has a shift of $\sim$0.007 in $\Omega_{M}$ and $\sim$0.03 in $w$. These input cosmologies lie at the furthest extent of the 68\% cosmological contour from $\Omega_{M}^{best}$ and $w^{best}$. In contrast, the third and fourth input cosmologies, which lie much closer to the experiment input cosmology, have a shift of $\lesssim$0.001 in both $\Omega_{M}$ and $w$. The increase in discrepancy as we probe parameter space that is further from the experiment input cosmology is expected, \replaced{as the BBC bias correction is only performed at a single point in cosmological parameter space, the best-fitting cosmology. The bias correction depends on the input cosmology, thus the BBC bias correction produces a contour which is accurate close to the best-fitting cosmology, but induces an offset in the contour at parameter space further from the best-fitting cosmology.}{as the bias correction used by BBC is most accurate near the input cosmology, and performs a lower quality correction further from the input.}\added{By contrast, our Neyman construction method applies this bias correction in a cosmologically dependent manner across the entire parameter space, removing this offset.}

This offset is important to consider, especially when combining supernova contours with other cosmological probes such as the cosmic microwave background. Fortunately, it is the region of posterior space close to the input cosmology which overlaps with the contours of other cross-cutting probes. As such this offset is unlikely to significantly affect multi-probe cosmological analyses.

Where this offset could be significant is in the investigation of cosmological tensions, where accurate uncertainties \added{of the tails} are vital to successfully assess the significance of the tension. In these cases, it may be useful to use a method like our approximate Neyman construction to produce more accurate and statistically rigorous measure of the uncertainty in a cosmological fit.
% Figure environment removed

\begin{table}
    \centering
    \begin{tabular}{l|l|l|l|l}
        Cosmology       & $\Omega_{M}'$     & $w'$      & $\Omega_{M}$ Absolute & $w$ Absolute  \\
        Input           &                   &           & Difference            & Difference    \\\hline
        1               & 0.138             & -0.716    & 0.05                  & 0.07          \\
        2               & 0.373             & -1.216    & 0.007                 & 0.03          \\
        3               & 0.308             & -0.976    & 0.001                 & 0.001         \\
        4               & 0.2918            & -1.02     & 0.0002                & 0.0
    \end{tabular}
    \caption{Comparison between the 68\% confidence region determined from our approximate Neyman construction, and the 68\% contour of the experiment cosmology. The absolute difference is the difference between the cosmologies at the edge of the 68\% contour produced by Pippin, and the cosmologies at the edge of the 68\% confidence region produced by our approximate Neyman construction.}
    \label{tab:PercentDiff}
\end{table}
\subsection{Effect of Bias Correction Input Cosmology}
We repeat our analysis with bias correction simulations that use a cosmology that is nearer to the data-set they are attempting to correct, rather than a single bias correction shared amongst all realisations. Figure~\ref{fig:BCFinal}, and Table~\ref{tab:BCPercentDiff} show the results of this repeat analysis. Our results are very similar to the case when we shared the bias correction simulation amongst all realisations, with the first and second cosmological input deviating the most. Overall, these results reinforce our suggestion that the offset present between the cosmological contour and confidence region is caused by the BBC bias correction, however the choice of bias correction does not significantly affect our consistency test. If computational cost is a concern, using only one bias correction simulation shared amongst all realisations will significantly reduce the computational cost of our approximate Neyman construction method, without significantly reducing the quality of the consistency test.
% Figure environment removed
\begin{table}
    \centering
    \begin{tabular}{l|l|l|l|l}
         Cosmology  & $\Omega_{M}'$      & $w'$      & $\Omega_{M}$ Absolute     & $w$ Absolute      \\
         Input      &                    &           & Difference                & Difference        \\\hline
         1          & 0.125              & -0.698    & 0.063                     & 0.085             \\
         2          & 0.371              & -1.21     & 0.009                     & 0.004             \\
         3          & 0.308              & -0.977    & 0.001                     & 0.0               \\ 
         4          & 0.292              & -1.02     & 0.0                       & 0.0
    \end{tabular}
    \caption{As for Table~\ref{tab:PercentDiff}, but varying the bias correction simulation to match the input cosmology.}
    \label{tab:BCPercentDiff}
\end{table}