In this paper we present a statistically rigorous method for checking the consistency of contours produced in a  cosmological analysis. To achieve this, we implement an approximate Neyman construction which requires far less computation than a true Neyman construction. This approximate Neyman construction is then used to define the 68\% confidence region for a single cosmological realisation. We use this confidence region to test the consistency of the 68\% contour produced by the BBC framework, as integrated in Pippin, although this method can be used to test the consistency of any cosmological parameter estimation method. This represents the first time the BBC framework has been tested with a statistically rigorous methodology. 

Our analysis showed that, for a DES-3YR like dataset, Pippin is producing reasonable, consistent parameter estimates. There was some discrepancy between the CR and the cosmological contour when considering the farthest extent of the 68\% contour. This discrepancy was, at maximum, a shift of $\sim$0.05 in $\Omega_{M}$, and $\sim$0.07 in $w$, and was likely due to the accuracy of BBC's bias correction being best when close to the input cosmology, and degrading in regions of parameter space which are far from the input cosmology. It is also important to recognise that this does not correspond to an equivalent error in the reported maximum posterior cosmological parameters. When considering cosmological inputs close to the experiment cosmology input, the confidence region and cosmological contour had near perfect agreement. As such any overall discrepancy is unlikely to significantly effect the results of a cosmological analysis, especially when multiple cross-cutting probes are combined. However, this shift is important when considering analyses concerned with assessing cosmological tensions - where the precise shape and size of the contour are vitally important to the analysis.

We see very similar results when each realisation had its own bias correction simulation, rather than sharing one bias correction simulation amongst all realisations, indicating that a sensible choice of bias correction is not likely to significantly effect our consistency checks.

Overall, we believe our method for consistency checking cosmological contours with an approximate Neyman construction represents an important improvement in the statistical rigour applied to cosmological analyses, and should \replaced{become a standard step in all cosmological analyses}{be considered carefully when a survey analysis is performed}. Our methodology can also be used to rigorously test cosmological contours for other cosmological probes, which have similarly complex pipelines. We believe this method will be particularly useful for future analyses, such as the DES 5-year supernova analysis, and the upcoming LSST survey. We plan to repeat this analysis using simulations that match the DES 5-year supernova analysis to test the consistency of those results.