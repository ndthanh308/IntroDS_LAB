\documentclass[preprint,superscriptaddress,notitlepage,endfloat]{revtex4-2}
\usepackage{graphicx,bm,mathtools,color}
\usepackage[pdfstartview=FitH, CJKbookmarks=true, bookmarksnumbered=true, bookmarksopen=true, colorlinks, pdfborder=001, linkcolor=blue, anchorcolor=blue, citecolor=blue]{hyperref}
%\usepackage{lineno}
%\linenumbers
%% \usepackage[utf8]{inputenc} % allow utf-8 input
%\usepackage[T1]{fontenc}    % use 8-bit T1 fonts
%\usepackage{hyperref}       % hyperlinks
\usepackage{url}            % simple URL typesetting
\usepackage{booktabs}       % professional-quality tables
\usepackage{multirow}    
\usepackage{amsfonts}       % blackboard math symbols
\usepackage{nicefrac}       % compact symbols for 1/2, etc.
\usepackage{microtype}      % microtypogrhy
% \usepackage{natbib}
\usepackage{enumerate}
%\usepackage{enumitem}
\usepackage{hhline}
\usepackage{makecell}
\usepackage{pifont}

% use Times
%\usepackage{times}
% For figures
\usepackage{graphicx} % more modern
%\usepackage{epsfig} % less modern
%\usepackage{subfigure}
\usepackage{caption}
\usepackage{subcaption}
% For citations
\usepackage{amsmath}
\usepackage{amsthm}
\usepackage{amssymb}
\usepackage{tikz}
\usepackage{xcolor}
\usetikzlibrary{arrows}

\allowdisplaybreaks

%for fonts
\usepackage{mathrsfs}

% For algorithms
\usepackage{algorithm}
\usepackage{algorithmic}
% \usepackage{algpseudocode}
% \usepackage[noend]{algpseudocode}
\usepackage{hyperref}
\usepackage{bm}
%\usepackage{todonotes}

%For theorems
\allowdisplaybreaks

%for convinience
\newcommand{\RR}{\mathbb{R}}
\newcommand{\vct}{\boldsymbol }
%\newcommand{\mat}{\mathbf}
\newcommand{\rnd}{\mathsf}
\newcommand{\ud}{\mathrm d}
\newcommand{\nml}{\mathcal{N}}
\newcommand{\loss}{\mathcal{L}}
\newcommand{\hinge}{\mathcal{R}}
\newcommand{\kl}{\mathrm{KL}}
\newcommand{\cov}{\mathrm{cov}}
\newcommand{\dir}{\mathrm{Dir}}
\newcommand{\mult}{\mathrm{Mult}}
\newcommand{\err}{\mathrm{err}}
\newcommand{\sgn}{\mathrm{sgn}}
%\renewcommand{\span}{\mathrm{span}}
% \newcommand{\argmin}{\mathrm{argmin}}
% \newcommand{\argmax}{\mathrm{argmax}}
\newcommand{\poly}{\mathrm{poly}}
% \newcommand{\rank}{\mathrm{rank}}
% \newcommand{\conv}{\mathrm{conv}}
%\newcommand{\E}{\mathbb{E}}
% \newcommand{\diag}{\mat{diag}}
\newcommand{\acc}{\mathrm{acc}}

\newcommand{\labs}{\left\vert}
\newcommand{\rabs}{\right\vert}
\newcommand{\lnorm}{\left\Vert}
\newcommand{\rnorm}{\right\Vert}

\newcommand{\aff}{\mathrm{aff}}
% \newcommand{\range}{\mathrm{Range}}
\newcommand{\Sgn}{\mathrm{sign}}

\newcommand{\hit}{\mathrm{hit}}
\newcommand{\cross}{\mathrm{cross}}
\newcommand{\Left}{\mathrm{left}}
\newcommand{\Right}{\mathrm{right}}
\newcommand{\Mid}{\mathrm{mid}}
\newcommand{\bern}{\mathrm{Bernoulli}}
\newcommand{\ols}{\mathrm{ols}}
\newcommand{\tr}{\operatorname{tr}}
\newcommand{\opt}{\mathrm{opt}}
%\newcommand{\ridge}{\mathrm{ridge}}
\newcommand{\unif}{\mathrm{Unif}}
\newcommand{\Image}{\mathrm{im}}
\newcommand{\Kernel}{\mathrm{ker}}
\newcommand{\supp}{\mathrm{supp}}
\newcommand{\pred}{\mathrm{pred}}
\newcommand{\distequal}{\stackrel{\mathbf{P}}{=}}
%\newcommand{\gege}{\textcircled{1}}
\newcommand{\gege}{{A(\vect{w},\vect{w}_*)}}
\newcommand{\gele}{{A(\vect{w},-\vect{w}_*)}}
\newcommand{\lele}{{A(-\vect{w},-\vect{w}_*)}}
\newcommand{\lege}{{A(-\vect{w},\vect{w}_*)}}
\newcommand{\firstlayer}{\mathbf{W}}
\newcommand{\firstlayerWN}{v}
\newcommand{\secondlayer}{a}
\newcommand{\inputvar}{\vect{x}}
\newcommand{\anglemat}{\mathbf{\Phi}}
\newcommand{\holder}{H\"{o}lder }
\newcommand{\real}{\mathbb{R}}
\newcommand{\approxerr}{\delta}

\def\R{\mathbb{R}}
\def\Z{\mathbb{Z}}
\def\cA{\mathcal{A}}
\def\cB{\mathcal{B}}
\def\cD{\mathcal{D}}
\def\cE{\mathcal{E}}
\def\cF{\mathcal{F}}
\def\cG{\mathcal{G}}
\def\cH{\mathcal{H}}
\def\cS{\mathcal{S}}
\def\cI{\mathcal{I}}
\def\cL{\mathcal{L}}
\def\cM{\mathcal{M}}
\def\cN{\mathcal{N}}
\def\cP{\mathcal{P}}
\def\cS{\mathcal{S}}
\def\cT{\mathcal{T}}
\def\cV{\mathcal{V}}
\def\cW{\mathcal{W}}
\def\cZ{\mathcal{Z}}
\def\SS{\mathbb{S}}
\def\NN{\mathbb{N}}
\def\bP{\mathbf{P}}
\def\TV{\mathrm{TV}}
\def\MSE{\mathrm{MSE}}

\def\vw{\mathbf{w}}
\def\va{\mathbf{a}}
\def\vZ{\mathbf{Z}}

\newcommand{\mat}[1]{#1}
\newcommand{\vect}[1]{#1}
\newcommand{\norm}[1]{\left\|#1\right\|}
\newcommand{\normop}[1]{\left\|#1\right\|_{\mathrm{op}}}
\newcommand{\simplex}{\triangle}
\newcommand{\abs}[1]{\left|#1\right|}
\newcommand{\expect}{\mathbb{E}}
\newcommand{\prob}{\mathbb{P}}
\newcommand{\proj}{\gP}
% \newcommand{\prox}[2]{\textbf{Prox}_{#1}\left\{#2\right\}}
\newcommand{\event}[1]{\mathscr{#1}}
\newcommand{\set}[1]{#1}
\newcommand{\diff}{\text{d}}
\newcommand{\difference}{\triangle}
\newcommand{\inputdist}{\mathcal{Z}}
\newcommand{\indict}{\mathbb{I}}
\newcommand{\rotmat}{\mathbf{R}}
\newcommand{\normalize}[1]{\overline{#1}}
\newcommand{\vectorize}[1]{\text{vec}\left(#1\right)}
\newcommand{\vclass}{\mathcal{G}}
\newcommand{\pclass}{\Pi}
\newcommand{\qclass}{\mathcal{Q}}
\newcommand{\rclass}{\mathcal{R}}
\newcommand{\classComplexity}[2]{N_{class}(#1,#2)}
\newcommand{\cclass}{\mathcal{F}}
\newcommand{\gclass}{\mathcal{G}}
\newcommand{\pthres}{p_{thres}}
\newcommand{\ethres}{\epsilon_{thres}}
\newcommand{\eclass}{\epsilon_{class}}
\newcommand{\states}{\mathcal{S}}
\newcommand{\trans}{P}
\newcommand{\lowprobstate}{\psi}
\newcommand{\actions}{\mathcal{A}}
\newcommand{\contexts}{\mathcal{X}}
\newcommand{\edges}{\mathcal{E}}
\newcommand{\variance}{\text{Var}}
\newcommand{\params}{\vect{w}}

\newcommand{\relu}[1]{\sigma\left(#1\right)}
\newcommand{\reluder}[1]{\sigma'\left(#1\right)}
\newcommand{\act}[1]{\sigma\left(#1\right)}

\newtheorem{thm}{Theorem}
% \newtheorem{thm}{Theorem}
\newtheorem{lem}{Lemma}
% Thm -> corollary 
\newtheorem{cor}{Corollary}
\newtheorem{prop}{Proposition}
\newtheorem{asmp}{Assumption}
\newtheorem{defn}{Definition}
\newtheorem{oracle}{Oracle}
\newtheorem{fact}{Fact}
\newtheorem{conj}{Conjecture}
\newtheorem{rem}{Remark}
\newtheorem{example}{Example}
\newtheorem{condition}{Condition}
\newtheorem{exercise}{Exercise}
\newtheorem{mess}{Message}
\newtheorem{claim}{Claim}
\newtheorem{ec}{Empirical Conclusion}






\usepackage[capitalize,noabbrev]{cleveref}
% \usepackage{cleveref}
\crefname{thm}{Theorem}{Theorems}
\crefname{lem}{Lemma}{Lemmas}
\crefname{cor}{Corollary}{Corollaries}
\crefname{prop}{Proposition}{Propositions}
\crefname{asmp}{Assumption}{Assumptions}
\crefname{defn}{Definition}{Definitions}
\crefname{oracle}{Oracle}{Oracles}
\crefname{fact}{Fact}{Facts}
\crefname{conj}{Conjecture}{Conjectures}
\crefname{rem}{Remark}{Remarks}
\crefname{claim}{Claim}{Claims}
\crefname{ec}{Empirical Observation}{Empirical Observations}


\renewcommand{\algorithmicrequire}{\textbf{Input:}}
\renewcommand{\algorithmicensure}{\textbf{Output:}}


\definecolor{red}{rgb}{1, 0, 0}
\newcommand{\RED}[1]{{\color{red} #1}}

\definecolor{green}{rgb}{0, 1, 0}
\definecolor{darkgreen}{rgb}{0.0, 0.2, 0.13}
\definecolor{darkseagreen}{rgb}{0.56, 0.74, 0.56}
\definecolor{officegreen}{rgb}{0.0, 0.5, 0.0}


\newcommand{\GREEN}[1]{{\color{green} #1}}

\definecolor{blue}{rgb}{0, 0, 1}
\newcommand{\BLUE}[1]{{\color{blue} #1}}

\definecolor{orange}{rgb}{1, 0.4, 0.0}
\newcommand{\ORANGE}[1]{{\color{orange} #1}}


\newcommand{\ket}[1]{\left\vert #1\right\rangle}
\newcommand{\red}[1]{{\color{red} #1}}
\newcommand{\blue}[1]{{\color{blue} #1}}
\newcommand{\be}{\begin{equation}}
	\newcommand{\ee}{\end{equation}}
\newcommand{\ua}{\uparrow}
\newcommand{\da}{\downarrow}
\newcommand{\LNO}{La$_3$Ni$_2$O$_7$ }
\newcommand{\Tc}{$T_{\rm c}$ }


\begin{document}
	
	
	\renewcommand{\abstractname}{} 
	\title{High-temperature superconductivity with zero-resistance and strange metal behaviour in La$_{3}$Ni$_{2}$O$_{7}$}
	
	\author{Yanan Zhang}
	\thanks{These authors contributed equally to this work.}
	\author{Dajun Su} 
	\thanks{These authors contributed equally to this work.}
	\affiliation  {Center for Correlated Matter and School of Physics, Zhejiang University, Hangzhou 310058, China}
	
	\author{Yanen Huang}
	\affiliation  {Center for Correlated Matter and School of Physics, Zhejiang University, Hangzhou 310058, China}
	
	
	\author{Hualei Sun}
	\affiliation  {Center for Neutron Science and Technology, Guangdong Provincial Key Laboratory of Magnetoelectric Physics and Devices, School of Physics, Sun Yat-Sen University, Guangzhou, Guangdong 510275, China}
	
	\author{Mengwu Huo}
	\affiliation  {Center for Neutron Science and Technology, Guangdong Provincial Key Laboratory of Magnetoelectric Physics and Devices, School of Physics, Sun Yat-Sen University, Guangzhou, Guangdong 510275, China}
	
	
	\author{Zhaoyang Shan}
	\affiliation  {Center for Correlated Matter and School of Physics, Zhejiang University, Hangzhou 310058, China}
	
	
	\author{Kaixin Ye}
	\affiliation  {Center for Correlated Matter and School of Physics, Zhejiang University, Hangzhou 310058, China}
	
	\author{Zihan Yang}
	\affiliation  {Center for Correlated Matter and School of Physics, Zhejiang University, Hangzhou 310058, China}
	
	\author{Rui Li}
	\affiliation  {Center for Correlated Matter and School of Physics, Zhejiang University, Hangzhou 310058, China}
	
	
	\author{Michael Smidman}
	\affiliation  {Center for Correlated Matter and School of Physics, Zhejiang University, Hangzhou 310058, China}
	
	\author{Meng Wang}
	\email[Corresponding author: ]{wangmeng5@mail.sysu.edu.cn}
	\affiliation  {Center for Neutron Science and Technology, Guangdong Provincial Key Laboratory of Magnetoelectric Physics and Devices, School of Physics, Sun Yat-Sen University, Guangzhou, Guangdong 510275, China}
	
	
	
	\author{Lin Jiao}
	\email[Corresponding author: ]{lin.jiao@zju.edu.cn}
	\affiliation  {Center for Correlated Matter and School of Physics, Zhejiang University, Hangzhou 310058, China}
	
	
	\author{Huiqiu Yuan}
	\email[Corresponding author: ]{hqyuan@zju.edu.cn}
	\affiliation{Center for Correlated Matter and School of Physics, Zhejiang University, Hangzhou 310058, China}
	\affiliation  {State Key Laboratory of Silicon and Advanced Semiconductor Materials, Zhejiang University, Hangzhou 310058, China}
 	\affiliation  {Zhejiang Province Key Laboratory of Quantum Technology and Device,
School of Physics, Zhejiang University, Hangzhou 310058, China}
	\affiliation  {Collaborative Innovation Center of Advanced Microstructures, Nanjing 210093, China}
	
	\date{\today}
	
	\begin{abstract}
		\textbf{Recently signatures of superconductivity were observed close to 80 K in \LNO under pressure \cite{Wang2023arxiv}. This discovery positions \LNO as the first bulk nickelate with high-temperature superconductivity, but the lack of zero resistance presents a significant drawback for validating the findings. Here we report pressure measurements up to over 30 GPa using a liquid pressure medium and show that \LNO does exhibit zero resistance.
 We find that \LNO remains metallic under applied pressures, suggesting the absence of a metal-insulator transition proximate to the superconductivity. Analysis of the normal state $T$-linear resistance suggests an intricate link between this strange metal behaviour and superconductivity, whereby at high pressures both the linear resistance coefficient and superconducting transition are slowly suppressed by pressure, while at intermediate pressures both the superconductivity and strange metal behaviour appear disrupted, possibly due to a nearby structural instability. The association between strange metal behaviour and high-temperature superconductivity is very much in line with diverse classes of unconventional superconductors, including the cuprates and Fe-based superconductors \cite{Cooper2009,Jiang2023NP,Taillefer2010,Greene2020,Phillips2022}. Understanding the superconductivity of \LNO evidently requires further revealing the interplay of strange metal behaviour, superconductivity, as well as possible competing electronic or structural phases.}
	\end{abstract}
	
	
	
	\maketitle
	
	Quasi-two-dimensional layered transition metal oxide structures are a common motif for high-temperature unconventional superconductivity~\cite{Norman2014,stewart2017}, yet this elusive phenomenon has only been realized in a handful of material classes, primarily the Cu-based cuprate \cite{Muller1986, Chu1987} and Fe-based  superconductors \cite{Hosono2008,Chen2008}. Very recently, signatures of superconductivity up to nearly 80~K were revealed in bulk samples of the nickelate \LNO above 14~GPa \cite{Wang2023arxiv}. This is in contrast to the infinite-layer nickelates whereby superconductivity with \Tc of 5-15~K is found in thin-films \cite{,Li2019nature,Ariando2020PRL,Harold2020PRM,Ariando2022scienceadvance,Ding2023nature}, increasing to 30~K under pressure \cite{JGCheng2022NC}, but superconductivity is not observed in bulk samples \cite{WenHaiHu2020CM,Phelan2020PRM}. While the Ni$^+$ of the infinite-layer nickelates has the same $3d^9$ configuration as the Cu$^{2+}$ of the cuprates, the presence of apical oxygens in \LNO leads to a formal $3d^{7.5}$ configuration. It is suggested that interlayer coupling between Ni $3d_{z^2}$  and apical oxygen $p$ orbitals at high pressures leads to partially occupied $3d_{x^2-y^2}$ orbitals \cite{Wang2023arxiv}, resembling the physics of the cuprates \cite{Lee2006,Keimer2015}, but the relevant orbitals and pairing instabilities for the superconductivity remain to be clarified \cite{Nakata,Philipp2023ARXIV,ZhangGuangMing2023ARXIV,HuJiangping2023ARXIV,WANGQIANGHUA2023ARXIV,Eremin2023ARXIV}.
	
	Although signatures of high-temperature superconductivity are reported from a drop of the electrical resistivity, as well as a diamagnetic response of the ac susceptibility \cite{Wang2023arxiv}, crucially zero resistance was not reached in electrical resistivity measurements, which is a vital criterion for establishing the occurrence of superconductivity. Here we report electrical resistance measurements of \LNO under pressure, which were performed in both a piston-cylinder cell (PCC) and a diamond anvil cell (DAC) (Extended Data Fig. 2), where the use of a liquid pressure-transmitting-medium leads to a greatly improved hydrostaticity. As displayed in Fig.~\ref{Fig1}, at a pressure of 20.5~GPa a clear superconducting transition is observed in $R(T)$ which onsets below 66~K, and  close to 40~K  zero resistance is reached within the noise level of the measurement system. Above $T_{\rm c}$, $R(T)$ exhibits a linear temperature dependence, characteristic of strange metal behaviour \cite{Wang2023arxiv}. Such a zero-resistance state at temperatures around 40~K clearly demonstrates high-temperature superconductivity in La$_3$Ni$_2$O$_7$. We note that these  
	samples have the same chemical composition as Ref.~\cite{Wang2023arxiv} where zero resistance was not observed, but those measurements were performed either with no pressure-transmitting medium, or with a solid medium, suggesting that more hydrostatic pressures realized with a liquid medium are crucial for its observation. Furthermore, the small sample dimensions (of order 10-100~$\mu$m) used in our DAC measurements may largely reduce the effects of sample inhomogeneities, allowing for the observation of a sharp superconducting transition with zero resistance.  
	
	The temperature dependence of the resistance $R(T)$ is displayed in Fig.~\ref{Fig2} for various pressures up to 29.2~GPa. At ambient pressure, $R(T)$ is metallic across the entire temperature range, and there is an anomaly around $T^*=130$~K. This temperature scale is close to the charge-density wave (CDW) transition observed in previous studies \cite{Wang2022sciencechina, Hundley2001PRB}. When a small pressure is applied, $R(T)$ remains metallic while $T^*$ decreases slightly at 1.0~GPa, and is not detected at 2.0~GPa. 
	\LNO being consistently metallic under pressure is in contrast to Ref.~\cite{Wang2023arxiv} where weakly insulating behaviour is induced by a small pressure, which in turn disappears following the structural transition above 10~GPa. 
	Our findings therefore are much more in line with reports showing that insulating behaviour in \LNO is associated with oxygen deficient rather than stoichiometric samples \cite{ZHANG1994,Hundley2001PRB,Masatoshi1995JPSJ}. Note that the DAC  measurements of the low-pressure structural phase (P $<$ 13~GPa) exhibit a large contact resistance between the sample and Au leads, which increases significantly with decreasing temperature, and could give rise to spurious insulating behaviour \cite{Blankenship1988YBCO} (Extended Data Fig. 3). This sizeable contact resistance is absent in DAC measurements performed above 13 GPa, and it is avoided in our piston-cylinder cell measurements below 2.5 GPa by increasing the area of the contacts and by avoiding heating the sample.
	
	At pressures above 13~GPa, superconducting transitions are observed in $R(T)$, which reach zero resistance below the transition (inset of Fig.~\ref{Fig2}b). Upon increasing the pressure, the onset of the superconducting transition $T^{\rm onset}_{\rm c}$ increases from 37.5~K at 13.7~GPa to a maximum of 66~K at 20.5~GPa. Further increasing the pressure (up to 30~GPa) results in a gradual decrease of $T^{\rm onset}_{\rm c}$, with d$T_{\rm c}$/d$P$ $\approx$ $-$0.8~K/GPa. A similar pressure dependence of
	$T^{\rm onset}_{\rm c}$  is measured in another sample (Extended Data Fig. 4), where the broader transition is likely a consequence of sample inhomogeneity. Strange metal behaviour is also observed across this pressure range, which corresponds to where there is a linear temperature dependence of $R(T)$. At 13.7~GPa, where there is evidence of a possible structural transition at higher temperatures (Extended Data Fig. 5), the linear $R(T)$ extends only up to around 100~K, but at 20.5~GPa this reaches at least 270~K. From analyzing with $R$($T$) = $R$(0) + $A^{\prime}T$ (dashed green lines in Fig.~\ref{Fig2}), it is found that the $T$-linear coefficient $A^{\prime}$ decreases with increasing pressure, as discussed below.
	
	Figures~\ref{Fig3}a and \ref{Fig3}b display  $R(T)$ in different applied magnetic fields under pressures of 20.5 and 26.6~GPa, respectively. The derived upper critical fields $H_{c2}(T)$ (defined from $T_c^{onset}$) versus temperature are shown in Fig~\ref{Fig3}c, where the initial slopes d$\mu_0H_{c2}(T)$/d$T$ are 0.55~T/K and 0.57~T/K for 20.5 and 26.6~GPa, respectively. By fitting $H_{c2}(T)$ with a Ginzburg-Landau model, respective zero-temperature values of 97~T and 83~T are obtained. Moreover, there is a gradual broadening of the transition with increasing field, indicating thermally activated flux flow (TAFF) in La$_{3}$Ni$_{2}$O$_{7}$. As shown by the Arrhenius plots in Fig~\ref{Fig3}d, the resistance data near the transition can be analyzed based on the simplified TAFF model: $U_0$($H$) = $-$dLn$R$/d(1/$T$), where $U_0$ is the thermal activation energy \cite{Vinokur1994RMP}. It is noted that the Arrhenius scaling holds over 3-4 orders until the signal reaches the noise floor of our instrument. These behaviours very much resemble iron-pnictide \cite{Wen2008} and chalcogenides \cite{Jiao2012}, as well as high $T_c$ cuprates \cite{Waszczak1990PRB}. Fitting to the TAFF model  derives an activation energy of $U_0$(1~T) = 702~K, which is much smaller than that for many cuprate and Fe-based superconductors, reflecting relatively weak pinning forces in La$_{3}$Ni$_{2}$O$_{7}$. Upon increasing the applied magnetic field, $U_0$($H$) follows a power law behaviour ($\propto H^{-n}$), with a small exponent of $n$ = 0.12, indicating a weak field-dependence of $U_0$.

	
	The temperature-pressure phase diagram in Fig.~\ref{Fig4} shows the main results from the current pressure study. Superconductivity with zero-resistance appears above 13~GPa, and is rapidly enhanced by pressure, reaching a maximum at 20.5~GPa, followed by a gradual decrease upon further pressure increases. The strange metal behaviour in the normal state is also highlighted by the color plot, in which the green regions correspond to where there is close to a $T$-linear resistance, while the pressure dependence of the $T$-linear coefficient $A^{\prime}$ is shown in the upper panel.  At 13.7 GPa, there is a relatively narrow temperature range over which there may be a $T$-linear $R(T)$ (up to around 100~K), and \Tc is correspondingly lower. On the other hand, the significant enhancement of \Tc at 16~GPa coincides with a marked expansion of the strange metal region, where the temperature range of the $T$-linear behaviour is largest near to the pressure where  \Tc is a maximum, at which there is only a 1\% deviation from linearity. At higher pressures, both $A^{\prime}$ and \Tc decrease with pressure, and the temperature range of the strange metal region is again reduced, as shown by the dashed lines in Fig.~\ref{Fig4}b. 
	
	Our observations suggest a close relationship between the strange metal behaviour and  high-temperature superconductivity in La$_{3}$Ni$_{2}$O$_{7}$, where the decrease of \Tc with pressure is associated with a reduction of $A^{\prime}$. On the other hand, at 13.7~GPa there does not appear to be a correspondingly reduced $A^{\prime}$, but at this pressure a weak signature of a possible structural transition is detected at higher temperatures (Extended Data Fig. 5), which appears to disrupt both the superconductivity and strange metal state. This suggests that understanding the superconductivity  of \LNO requires revealing the various competing structural and electronic phases, which are highlighted when the phase diagram is extended to low pressures in Fig.~\ref{Fig4}c. Here a possible CDW transition is suppressed by a moderate pressure, while at higher pressures there is a transition to a structural phase favouring both high-temperature superconductivity and strange metal behaviour. The rich interplay of these different ground states is very much in line with the cuprates and Fe-based superconductors \cite{Fernandes2014,Proust2019}. In particular, the initimate link between the strange metal phase and high-temperature superconductivity in \LNO is also reminiscent of the enigmatic relationship between these phenomena in the cuprates and Fe-based superconductors \cite{Cooper2009,Jin2011,Taillefer2010,Licciardello2019,Zhao2022,Jiang2023NP,Taillefer2010,Greene2020}, whereby superconducting properties such as \Tc and the superfluid density are often correlated with $A^{\prime}$,  reaching a peak at optimal doping. Since Landau quasiparticles are anticipated to be absent in the strange metal phase, these in turn suggest that unconventional superconductivity could emerge via non-quasiparticle states \cite{Phillips2022}. As such, these results hint at an unconventional nature of the high-temperature superconductivity in La$_{3}$Ni$_{2}$O$_{7}$, while underscoring the close association between strange metals and superconductivity across various materials systems, highlighting the nickelate superconductors as a new platform for examining this interplay.
	
	\clearpage
	
%	\noindent \textbf{Methods} \\
%	\noindent Single crystals of \LNO were synthesized using a vertical optical image floating zone furnace, as described in Ref. \cite{Wang2022sciencechina} (Extended Data Fig. 1). In this work, we studied five samples cut from the same growth. Sample S1, S2, and S3 were measured in a piston-cylinder cell, while S1-1 and S2-1, which were small samples cut from  S1 and S2, were measured in a diamond anvil cell. The data presented in the main text is from S1 and S1-1. Resistivity measurements using the four-probe method under pressure were carried out utilizing a piston-cylinder-type pressure cell up to 2.0 GPa, with Daphne 7373 used as the pressure-transmitting medium. The applied pressure was determined by the shift in $T_{\rm c}$ of a high-quality Pb single crystal \cite{Eiling1981}.  For the  measurements at higher pressures, single crystals of La$_{3}$Ni$_{2}$O$_{7}$ were polished and cut to approximate dimensions $120 \times 80 \times 20~\mu \mathrm{m}^3$, and were then loaded into a BeCu diamond anvil cell with a 400-$\mu$m-diameter culet. A 100-$\mu$m-thick pre-indented rhenium gasket was covered with boron nitride for electrical insulation and a 200-$\mu$m-diameter hole was drilled as the sample chamber. Daphne oil 7373 was used as the pressure transmitting medium, in order to obtain good hydrostaticity. The DAC was loaded together with several small ruby balls for pressure determination at room temperature using the ruby fluorescence method \cite{mao1986rubycalibration}. For electrical transport measurements,  15 $\mu$m diameter gold wires were attached to the samples using silver epoxy paste and the resistance was measured using a four-probe method. The resistance measurements under pressure were performed in a Teslatron-PT system with an Oxford $^{3}$He refrigerator and a Quantum Design Physical Property Measurement System (PPMS).
	
	
	
	
	\noindent \textbf{Acknowledgments} 
	Work at Zhejiang University was supported by the National Key R\&D Program of China (Grant No. 2022YFA1402200), the Key R\&D Program of Zhejiang Province, China (Grant No. 2021C01002), the National Natural Science Foundation of China (Grants No. 12274364, No. 11974306, No. 12034017, No. 12174332, and No. 12204408), and the Zhejiang Provincial Natural Science Foundation of China (Grant No. LR22A040002). Work at Sun Yat-sen University was supported by the National Natural Science Foundation of China (Grant No. 12174454), Guangdong Basic and Applied Basic Research Funds (Grant No. 2021B1515120015), Guangzhou Basic and Applied Basic Research Funds (Grant No. 202201011123), and Guangdong Provincial Key Laboratory of Magnetoelectric Physics and Devices (Grant No. 2022B1212010008).
	\\
	\\
	\textbf{Additional information} Correspondence and requests for materials should be addressed to H. Q. Yuan (hqyuan@zju.edu.cn), L. Jiao (lin.jiao@zju.edu.cn), or M. Wang (wangmeng5@mail.sysu.edu.cn).
	\\
	\\
	\textbf{Author contributions} 
	H.Y. and L.J. conceived the experiments. The single crystals were provided by H.S., M.H., and M.W.. Y.Z. and D.S. conducted the measurements with the help of Y.H., Z.S., K.Y., Z.Y., and R.L.. M.S., Y.Z., L.J., and H.Y. wrote the paper with input from all authors.
	\\
	\textbf{Competing financial interests} The authors declare no competing financial interests.
	
	
	
	\bibliographystyle{naturemag}
	\bibliography{ref}
	
	
	
	
	\clearpage
	
		% Figure environment removed
	
	
	% Figure environment removed
	
	% Figure environment removed

		% Figure environment removed
	
	
	
	
	
\end{document}
