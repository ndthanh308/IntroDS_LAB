\section{Conclusions and Future Work}
In this paper, we conduct a survey for hierarchical multi-label text classification, including the data sets, main approaches, learning strategies, evaluation metrics and challenges.  We find that early hierarchical multi label classification models were based on a tree structure, and more recent approaches rely on deep learning models, especially Transformer and BERT.  Currently, the graph neural network based approach is the main stream, where the hierarchical label information is modeled together with the input text in a graph and a deep graph neural network is learned. The current state of art uses an ensemble approach, where a set of classifiers based on the hierarchical structure of labels are learned, and a global classifier is then developed to predict all the labels. The ensemble approach is quite effective but few formal results are published in academia. We consider a few future research directions: i) Leverage instruction learning for hierarchical multi-label text classification, as prompt learning has shown quite strong performance. Designing appropriate prompt for hierarchical multi-label classifiers can be regarded as a weakly supervised learning problem.  ii) Deal with the label imbalance and sparsity problem.  iii) Inject domain knowledge into the classifier, this is often efficient when labeled data is scare and domain knowledge is strait-forward.  iv) As recent big language models like ChatGPT and GPT4 are released, the ability of zero-shot learning of these language models remains as question. v) Incremental hierarchical multi-label learning, the label hierarchy can evolve with more labels and different hierarchy, continual learning of the classifier is a key problem in the real world scenario. 

%\\In the future, how to design a good classifier is a research direction. A good classifier can make good use of the relationship between text and layers of labels, as well as the dependency relationship between layers of labels. Secondly, finding a more suitable text encoding method can be regarded as a research direction, and whether there is a better text encoding method that can enable the network to better understand the information within the text. At the same time, as research continues to deepen, the problem of extreme multi label classification will become increasingly prominent, and the problem of long tailed data distribution will become increasingly prominent. Therefore, how to solve these two issues is also one of the future research directions.