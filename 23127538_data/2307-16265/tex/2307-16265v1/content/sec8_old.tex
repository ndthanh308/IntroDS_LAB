\section{Conclusions and Future Work}

In this article, we summarize several methods for dealing with hierarchical multi label classification problems. We divided the model into four major categories, summarized the dataset and evaluation indicators, and proposed the challenges faced by hierarchical multi label tasks.\\
\\Through research on model classification, it can be seen that early hierarchical multi label classification was based on a tree structure, with the parent label serving as the root node of the tree or subtree, and the child label serving as the leaf node of the tree or subtree. During classification, the prediction starts from the root node, and the prediction of related node labels is performed layer by layer according to the tree structure, until the leaf node. Modeling using tree structures is relatively complex and computationally expensive, so after the continuous development of deep learning, especially the introduction of transformer and bert, the method of using trees has gradually withdrawn from the stage. The current mainstream model is a serial model. Most models now divide the model into different levels based on the hierarchical structure of labels. For each level, a classifier can be set to classify labels, or information can be collected at each level. Finally, a global classifier is used to uniformly predict all labels. In engineering, there are also methods that use ensemble, which classify each category separately, and then talk about information fusion. However, this method is relatively rare in the field of scientific research. Another method is to input hierarchical structure information into the model as known information. Generally speaking, hierarchical structure is modeled through a graph model. When making predictions, the model can use the input hierarchical structure information to make predictions.\\
\\In the future, how to design a good classifier is a research direction. A good classifier can make good use of the relationship between text and layers of labels, as well as the dependency relationship between layers of labels. Secondly, finding a more suitable text encoding method can be regarded as a research direction, and whether there is a better text encoding method that can enable the network to better understand the information within the text. At the same time, as research continues to deepen, the problem of extreme multi label classification will become increasingly prominent, and the problem of long tailed data distribution will become increasingly prominent. Therefore, how to solve these two issues is also one of the future research directions.