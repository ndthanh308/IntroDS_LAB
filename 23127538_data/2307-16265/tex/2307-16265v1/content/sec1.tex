\section{Introduction}
The recent advances in deep learning, large language modeling, instruction learning and human-AI alignment has revolutionized the NLP community. Among all these different kinds of NLP tasks, text classification is still regarded as the backbone for most downstream NLP applications. In the real world scenario, most text classification tasks are multi-label ones. For example, the hierarchical categorization of scientific literature in arXiv, the grouping of products in online shopping sites, as well as the management of papers in e-libraries. Conceptually,  hierarchical multi-label text classification means that text are associated with multiple labels, for which the labels are organized in a structured hierarchy. Mathmatically, hierarchical multi-label classification can be defined as a function: $ f: X \rightarrow Y^*$,  where $X$ is the sample space and $Y^*$ is the power set of all possible label sets. Each element in $Y^*$ is an ordered tag sequence $(y_1, y_2,..., y_k)$, representing the hierarchical path from the root node to the leaf node. The goal of hierarchical multi-label classification is to learn a function $f$, so that for any given sample $x \in X$, $f (x)$ can output one or more labels which are most relevant to $x$. Traditionally, the one-vs-rest or one-vs-one learning strategy is a common way to learn hierarchical multi-label classifiers, but it ignores the information of the label hierarchy and the learning process is expensive. Modern approaches utilize a deep learning encoder based on a Binary Cross Entropy loss and do the training in a end-to-end manner. In addition, the label information can also be leveraged in the learning process. 

Currently, there are more advanced models and learning approaches for hierarchical multi label tasks, but few literature has summarized these methods. Therefore, we conduct a survey of recent advances on hierarchical multi-label text classification. This survey is organized in the following way: Section 2 summarizes the data sets used for hierarchical multi-label classification. Section 3 mainly discusses the models and methods used for hierarchical multi-label text classification, which are mainly divided into four major categories. In section 4, we discuss some machine learning strategies for hierarchical multi-label text classification. Subsequently, section 5 show the widely used evaluation metrics and section 6 will mention some existing challenges and the last section is the conclusion and some suggestion  for future work.