\section{Data sets}
In this section, we will introduce the common hierarchical multi-label classification data sets and divide them into three categories: emotion categories, articles categories, and topic categories. We count 17 commonly used data sets and analyze their source, content, hierarchy and other information. Among them, there are four emotion data sets, 7  article datasets and 6 topic datasets.
\subsection{Emotion  data sets}
Such data sets mainly include user emotion labels,Using for performing the emotion-label classification,being common in a variety of food reviews, film reviews and other data sets.\\
\\
\textbf{YELP} The YELP data set is sourced from the US restaurant review website and consists of 60,000 merchants, 8.63 million comments and 200,000 images. This data set shows the different scores of users for the store, which is divided into two sub-data sets: YELP-2 and YELP-5. User reviews of YELP-2 mainly use positive and negative emotion labels, while YELP-5 reviews have five different scores.\\
\\
\textbf{Amazon Product Reviews} This dataset contains a large number of user product reviews of Amazon, with about 8 million reviews. Emotional labels are also divided into binary categories and multi-class categories, just like YELP. Its common sub-datasets include Amazon-670k, Amazon-3m, etc.\\
\\
\textbf{IMBD} IMBD is a binary sentiment classification dataset containing both positive and negative ratings. It has 50k samples and the testing set and the training set both have 25k of them. All data are obtained from film reviews on the IMBD website.

\begin{table}
\centering
\caption{Emotional data sets}\label{tab1}
\begin{tabular}{|c|c|c|c|}
\hline
Name &  Field & Number of categories & Level depth\\
\hline
YELP & Dining reviews & 539 & -\\
Amazon 670k & Product reviews & 670091 & 9\\
Amazon 3M & Product reviews & 2812281 & -\\
IMBD & movie reviews & - & -\\
\hline
\end{tabular}
\end{table}


\subsection{Article classification data sets}
These data sets have information on a large number of different articles that may originate from the same websites or journals. The content of the article may involve various aspects.\\
\\
\textbf{WOS}  The WOS dataset contains web of science articles, with 134 catalogues and 46,985 instances, covering all aspects. This data set is a common data set for hierarchical multi-label classification. There are three categories: WOS-5736, WOS-11967, WOS-46985. WOS-5736 and WOS-11967 are subsets of WOS-46985.\\
\\
\textbf{Arxiv}  The Arxiv dataset has about 1.7million articles, divided into 14 different features, such as authors, titles et al. It is divided into 62,055 subcategories according to different disciplines, which basically includes most of the papers on the arxiv website.\\
\\
\textbf{Pubmed} The Pubmed dataset is a dataset about biomedicine and has its own retrieval function. The data set contains approximately 20,000 instances, and all of them are scientific research papers on diabetes. Each instance in the dataset is represented by a weighted word vector. \\
\\
\textbf{Reuters Corpora}  This data set is a collection of Reuters news and articles and have two versions. RCV1 is modified on the Reuters-21578 dataset and has 8 million instances in English. RCV2 has nearly 5 million instances in 13 different languages.Both are very commonly used text classification data sets.\\
\\
\textbf{NYTimes}  This dataset contains 1.8 million articles from New York Times during 1987-2007. It is a very large dataset,which widely used in NLP studies such as document categorization, entity extraction, etc.
\begin{table}
\centering
\caption{Emotional data sets}\label{tab1}
\begin{tabular}{|c|c|c|c|}
\hline
Name &  Field & Number of categories & Level depth\\
\hline
WOS  & Academic articles & 141 & 2\\
WOS-11967 & Academic articles & 40 &2\\
Arxiv & Academic issues & -& -\\
Pubmed & Biomedical articles & 17693 & 15\\
Reuters  & News & 101 & 3\\
RCV1 &Reuters news &103 &6\\
NYTimes & News & 2318 & 10\\
\hline
\end{tabular}
\end{table}
\subsection{Topic classification data sets}
Each data set is mainly about a certain topic.The data set includes information about one specific topic,such as patents,book advertising,bookmarks,etc.\\
\\
\textbf{HUPC}  This dataset covers 4.5 million patent documents in English. The patents come from the USPTO website which all applied in 2004-2018. This dataset provides the structured metadata for each application in order to be better utilized by researchers.\\
\\
\textbf{BGC-EN}  Blurb Genre Collection (BGC) is a data set of book advertising descriptions, and each bulbar is divided into one or several categories. This data set consists of 91,892 instances, including the book’s title, the author, url, ISBN, etc.\\
\\
\textbf{WIPO}  This dataset is a very typical patent dataset, similar to HUPC. This dataset contains 74,650 samples from the World Intellectual Property Organization (WIPO). It contains the ID, author, class and other information of the patent, of which 45324 samples are used for the training set and 28926 are used for the test set.
\\
\\
\textbf{DBpedia} This is a very common data set in NLP research. The information in this data set is extracted from Wikipedia. DBpedia is updated monthly to add or delete some data and adjust the categories. It has a lot of sub-datasets, for example: DBpedia Ontology Dataset, which has 630k instances; Dataset 3.8, which has 3.77 million samples.\\
\\
\textbf{Delicious}  Delicious is a social platform for sharing and discovering bookmarks. Data for Delicious dataset comes from this site and contains the users’ ID and friendship between the users. In the two available CSV files, there are 103144 nodes and 1419519 edges.\\
\\
\textbf{Enron} The dataset, collected by the CALO project, contains data from approximately 150 users. There are 0.5M samples in the dataset, divided into 56 labels.\\
\begin{table}
\centering
\caption{Emotional data sets}\label{tab1}
\begin{tabular}{|c|c|c|c|}
\hline
Name &  Field & Number of categories & Level depth\\
\hline
HUPC  & Patent information & - & -\\
BGC-EN  & books’ introduction & 146 & 4\\
WIPO-alpha & Patent information &5229 &4\\
DBpedia & Wikipedia data & - & -\\
Delicious & User Information & 983 & -\\Enron & Email information & 56 & 3\\
\hline
\end{tabular}
\end{table}