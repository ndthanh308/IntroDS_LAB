\section{Background}
With the rapid development of the information age, the number of articles is increasing at an explosive rate. Manual tagging of relevant articles has become very time-consuming, and there is a certain impact of subjective consciousness in manual tagging. Therefore, text classification has more practical significance. With the progress of scientific research on text classification, people gradually notice that the same article may have different labels, However, there may be some associations between different tags. For example, some tags can be classified as child tags or parent tags of other tags in some scenarios. Specifically, a news article can be classified into three categories: "sports", "basketball" and "warrior", of which "basketball" is a subclass of "sports" and "warrior" is a subclass of "basketball". The relationship between the class labels is described as hierarchical relationship. The text classification task with this relationship is called hierarchical multi-label text classification task. In mathematics, hierarchical multi-label classification can be defined as a function:
$
f: X \rightarrow Y^*
$
where $X$ is the sample space and $Y^*$ is the power set of all possible label sets. Each element in Y * is an ordered tag sequence $(y_1, y_2,..., y_k)$, representing the hierarchical path from the root node to the leaf node. For example, (sports, basketball, warrior) is an effective tag sequence. The goal of hierarchical multi-label classification is to learn a function $f$, so that for any given sample $x \in X$, $f (x)$ can output one or more tag sequences most relevant to $x$.\\
The hierarchical multi-label classification task is mainly to calculate the dependency relationship between labels and find labels at different levels in a large number of label data sets. At present, there are many researches in this field. The overall idea is to obtain the data set, preprocess the text, and then put it into a specific classifier through different models to achieve the output of different categories of text labels. Researchers focus on creating good models to learn the mapping of instances to tags and complete classification. Multi-level classification will use graph relations to separate labels and document metadata, which can make the whole process faster. \\
\\
In 2014, Yoon Kim first applied the CNN structure to the text classification task\cite{kim-2014-convolutional}, followed by the TextRNN model, followed by the hierarchical attention network model HAN \cite{yang-etal-2016-hierarchical}, and added the attention mechanism to TextRNN for the first time, using a hierarchical structure to represent a text, which is well interpretable. In order to solve the label masking problem, the BERT model \cite{2018arXiv181004805D}for sequence generation was proposed. In addition, RNN network is also very popular. RNN-GRU uses GRU to make the original model have a better classification starting point. At the same time, there are many label classification methods based on graph, tree and other data structures. Although the ultimate purpose of hierarchical multi-label classification is the same as that of traditional text multi-label classification, because it introduces the hierarchical structure of labels, multi-label text classification also faces problems such as inaccurate labels and unclear definitions of dependencies between labels. In this article, we will summarize different models and methods in hierarchical multi-label text classification on different dimensions, and we will also analyze the challenges in future work.