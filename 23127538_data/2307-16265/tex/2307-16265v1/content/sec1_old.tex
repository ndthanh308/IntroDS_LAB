\section{Introduction}
As a fundamental problem in natural language processing, multi-label text classification means assigning documents to multiple labels. However, people gradually notice that the hierarchical relationships between different labels is also an extra important information which can help classify. Thus Hierarchical Multi-label Text Classification refers to that texts are associated with multiple class labels which are organized in a structured label hierarchy. Currently, there are many and complex models for hierarchical multi label tasks, and few literature has summarized such topics. Therefore, we have conducted a survey of the literature on hierarchical multi-label text classification in recent years. This survey is mainly composed of six parts. The second part introduces the background and development status of hierarchical multi-label tasks. The third part summarizes the data set used for hierarchical multi-label classification. The fourth part mainly discusses the models and methods used for hierarchical multi-label text classification in the literature we have investigated, and divides these models into four major categories. The fifth part shows some learning strategies used in this task. The sixth part mainly summarizes the evaluation indicators. The seventh part proposes some challenges for the current hierarchical multi-label classification task. The last part is the conclusions drawn and suggestions for future work.