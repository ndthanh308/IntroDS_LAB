\documentclass{article}
%
%        \setcounter{tocdepth}{2}
%        \makeatletter
%        \renewcommand*\l@author[2]{}
%        \renewcommand*\l@title[2]{}
%        \makeatletter
%
\usepackage{siunitx}
\usepackage{csquotes}
\usepackage[inline]{enumitem}
\usepackage{graphicx}
\usepackage{xspace}
\usepackage{amssymb}
\usepackage{amsmath}
\usepackage{ntheorem}
\usepackage{stmaryrd}
\usepackage{hyperref}

%---------------Shortcuts by martin sasieta

%---------------General
\newcommand{\be}{\begin{equation}}
\newcommand{\ee}{\end{equation}}
\newcommand{\bea}{\begin{eqnarray}}
\newcommand{\eea}{\end{eqnarray}}
\newcommand{\bg}{\begin{gather}}
\newcommand{\eg}{\end{gather}}



%---------------Paper

\newcommand{\tlambda}{\tilde{\lambda}}
\newcommand{\Ri}{\mathsf{R}}
\newcommand{\aux}{\mathsf{A}}
\newcommand{\scrv}{\mathscr{V}}
\newcommand{\Ws}{\mathcal{W}}
\newcommand{\sign}{\text{sign}}
\newcommand{\Co}{\mathsf{c}}
\newcommand{\Op}{\mathcal{O}}
\newcommand{\Le}{\mathsf{L}}
\newcommand{\lef}{\mathsf{l}}
\newcommand{\bulk}{\text{bulk}}
\newcommand{\Sch}{\mathsf{S}}
\newcommand{\na}{\mathsf{n}}
\newcommand{\ri}{\mathsf{r}}
\newcommand{\rad}{\mathsf{Rad}}
\newcommand{\rhohat}{\hat{\rho}}
\newcommand{\trho}{\tilde{\rho}}
%\newcommand{\max}{\text{max}}


\newcommand{\nvec}{\mathbf{n}}

\newcommand{\Lim}[1]{\raisebox{0.5ex}{\scalebox{0.8}{$\displaystyle \lim_{#1}\;$}}}
\newcommand{\tbeta}{\tilde{\beta}}

\newcommand{\MS}[1]{ { \color{blue} \footnotesize (\textsf{MS}) \textsf{\textsl{#1}} } }

\newcommand*\pFqskip{8mu}
\catcode`,\active
\newcommand*\pFq{\begingroup
        \catcode`\,\active
        \def ,{\mskip\pFqskip\relax}%
        \dopFq
}
\catcode`\,12
\def\dopFq#1#2#3#4#5{%
        {}_{#1}F_{#2}\biggl[\genfrac..{0pt}{}{#3}{#4};#5\biggr]%
        \endgroup
}
\renewcommand{\subsectionautorefname}{Sect.}
\renewcommand{\subsubsectionautorefname}{Sect.}
\renewcommand{\sectionautorefname}{Sect.}
\renewcommand{\figureautorefname}{Fig.}

\newcommand{\nlred}{-2mm}
\newcommand{\indred}{-1mm}
\newcounter{acounter}

\newcommand{\assumption}[2]{
	\refstepcounter{acounter}\label{ass:#1}\small\textbf{Assumption \arabic{acounter}:}\normalsize\enspace #2 
}

\newtheorem*{definition}{Definition}
\newenvironment{mdefinition}[1]
{\begin{definition}[#1]}
{\end{definition}}
\providecommand*{\texdefinitionautorefname}{Def.}

\newtheorem*{texproofsketch}{Proof Sketch}
\newenvironment{proofsketch}[1]
{\begin{texproofsketch}[#1]\enspace}
{\it \proofbox\end{texproofsketch}}

\newtheorem*{texproof}{Proof}
\newenvironment*{mproof}[1]
{\begin{texproof}[#1]}
{\it \proofbox\end{texproof}}


\newtheorem{ttheorem}{Theorem}
\newenvironment{mtheorem}
{\begin{ttheorem}}{\end{ttheorem}}
\providecommand*{\ttheoremautorefname}{Thm.}


\newtheorem*{proposition}{Proposition}
\newtheorem{tproposition}{Proposition}
\newenvironment{mproposition}[1]
{\begin{proposition}[#1]}{\end{proposition}}
\providecommand*{\ttheoremautorefname}{Prop.}


\newtheorem{texample}{Example}
\newenvironment{mexample}[1]
{\begin{texample}\it (#1)\hspace{3mm}}
	{\ple\rm
\end{texample}}
\providecommand*{\texampleautorefname}{Exp.}





\newtheorem{texcurs}{Excursus}
\newenvironment{excursus}[1]
{\begin{texcurs}\it (#1)\hspace{3mm}}
	{\ple\rm
\end{texcurs}}
\providecommand*{\texcursautorefname}{Exc.}


\newcommand\FramedBox[3]{%
  \setlength\fboxsep{4pt}
  \fbox{\parbox[t][#1][l]{#2}{\vspace*{0mm}#3}}}


\hyphenation{pro-mise}
\hyphenation{Au-to-no-mous-Decisiveness}

\title{Framing Relevance for Safety-Critical Autonomous Systems}
\author{Astrid Rakow\thanks{This research was supported by the German Research Council (DFG) in the PIRE Projects SD-SSCPS and ISCE-ACPS under grant no. DA 206/11-1.}\\a.rakow@uol.de\\Carl von Ossietzky University of Oldenburg}
\begin{document}
\maketitle
\begin{abstract}We are in the process of building complex highly autonomous systems that have build-in beliefs, perceive their environment and exchange information. 
These systems construct their respective world view and based on it they plan their future manoeuvres, i.e., they choose their actions in order to establish their goals based on their prediction of the possible futures. 
Usually these systems face an overwhelming flood of information provided by a variety of sources where by far not everything is relevant.
The goal of our work is to develop a formal approach to determine what is relevant for a safety critical autonomous system (\HAS) at its current mission, i.e., what information suffices to build an appropriate world view to accomplish its mission goals.
\end{abstract}
\tableofcontents
\section{Introduction}
Current quantum hardware is unable to carry out universal quantum computations due to the buildup of errors that occur during the computation. 
The magnitude of the individual error is currently above the value that the Threshold Theorem requires in order to kick-start quantum error correction and fault-tolerant quantum computation~\cite[Section 10.6]{nielsen_chuang_2010}. 
Although the experimentally achieved fidelity rates are promising and the error bounds are inching closer to the required threshold, we will have to work for the foreseeable future with quantum hardware with errors that build-up during the computation.  This implies that we can only do a limited number of steps before the output of the computation has become completely uncorrelated with the intended one.

For fault-tolerant quantum computing, we repeat four steps: 
1) We apply a number of single and two-qubit quantum gates, in parallel whenever possible; 
2) We perform a syndrome measurement on a subset of the qubits; 
3) We perform fast classical computations to determine which errors have occurred and how to correct them; 
and, 4) We apply correction terms based on the classical computations.
We then repeat these four steps with a next sequence of gates. 
These four steps are essential to fault-tolerant quantum computing. 


The starting point of this work is to use the four steps outlined above, not to carry out error correction and fault-tolerant computation, but to enhance short, constant-depth, {\em uncorrected} quantum circuits that perform single qubit gates and {\em nearest-neighbor} two qubit gates. 
Since in the long run we will have to implement error-correction and fault-tolerant computation anyhow, and this is done by such a four-step process, why not make other use of this architecture? Moreover, on some of the quantum hardware platforms, these operations are already in place.
Embracing this idea we naturally arrive at the question: what is the computational power of \textit{low-depth} quantum-classical circuits organized as in the four steps outlined above? 
We thus investigate circuits that execute a small, ideally constant, number of stages, where at each stage we may apply, in parallel, single qubit gates and {\em nearest-neighbor} two qubit gates, followed by measurements, followed by low-depth classical computations of which the outcome can control quantum gates in later stages. 
It is not clear, at first, whether such circuits, especially with constant depth, can do anything remotely useful. 
But we will see that this is indeed the case: many quantum computations can be done by such circuits in constant depth. 
By parallelizing quantum computations in this way, we improve the overall computational capabilities of these circuits, as we do not incur errors on qubits that are idle, simply because qubits are not idle for a very long time. 
Furthermore, reducing the depth of quantum circuits, at the cost of increasing width, allows the circuit to be run faster even if errors occur.

The first usage of such a four-step layout, not to do error correction, but to perform computations, can be found in the paradigm of measurement-based quantum computing~\cite{gottesman1999demonstrating,raussendorf2001one,jozsa2006introduction,clark2007generalised}: 
A universal form of quantum computing where a quantum state is prepared and operations are performed by measuring qubits in different bases, depending on previous measurements and intermediate measurements.

\citeauthor{PhamSvore2013} were the first to formalize the four-step protocol for performing computations~\cite{PhamSvore2013}. They included specific hardware topologies by considering two-dimensional graphs for imposing constraints on qubit interactions. In their model, they develop circuits for particularly useful multi-qubit gates, including specifying costs in the width, number of qubits, depth, number of concurrent time steps, size, and total number of non-Identity operations.
As a result, they find an algorithm that factors integers in polylogarithmic depth.
\citeauthor{Browne:2011} showed that the main tool in the work by \citeauthor{PhamSvore2013}, the fan-out gate, can also be replaced by additional log-depth classical computations in the measurement-based quantum computing setting~\cite{Browne:2011}.

More recently, \citeauthor{Cirac:2021} introduced a scheme to implement unitary operations involving quantum circuits combined with Local Operations and Classical Communication ($\mathsf{LOCC}$) channels: $\mathsf{LOCC}$-assisted quantum circuits~\cite{Cirac:2021}. Similarly to the four-step scheme we just described, they allow for a short depth circuit to be run on the qubits, followed by one round of $\mathsf{LOCC}$, in which ancilla qubits are measured and local unitaries are applied based on the measurement outcomes. They show that in this model any 1D transitionally invariant matrix-product state (MPS) with fixed bond dimension is in the same phase of matter as the trivial state. Similar ideas can be found in~\cite{TVV_NonAbelianTopologicalOrder_2022, tantivasadakarn2021long}.

In this work, we introduce a new model, called \textit{Local Alternating Quantum-Classical Computations} ($\LAQCC$). In this model we alternate between running quantum circuits (constrained by locality), ending in the measurement of a subset of qubits, and fast classical computations based on the measurement results. The outcome of the classical computations are then used to control future quantum circuits. We allow for flexibility in this model, by giving different constraints to the power of both the quantum circuits and the classical circuits as well as the number of alternations between them. 
Most attention will be given to $\LAQCC$ containing quantum circuits of constant depth, classical circuits of logarithmic depth and at most a constant number of alternations between them. 
Any circuit constructed in this model is considered to be of constant depth. 
We restrict ourselves to logarithmic depth classical computations, as this is the first natural and non-trivial extension beyond constant-depth classical computations. 
Constant-depth classical computations do however also have an equivalent constant-depth quantum implementation.

The definition of $\LAQCC$ sharpens the original definition of \citeauthor{PhamSvore2013} by adding constraints to the intermediate classical computations. This allows us to bound the power of $\LAQCC$ from above. 

The main result of \citeauthor{Cirac:2021}, that 1D translational invariant MPS with fixed bond dimension can be prepared by $\mathsf{LOCC}$-assisted circuits, relies on local symmetries of the MPS. These symmetries allow them to prepare local states (on a constant number of qubits) and glue them together by doing one round of the appropriate entangling measurement and corrections, after which they run a round of local unitaries to get the desired result. This general scheme for preparing states that exhibit an MPS description with the appropriate local symmetries requires only geometrically local unitaries and one round of measurement and corrections an therefore is accessible in $\LAQCC$. Studying different local symmetries, known as Symmetry Protected Topological (SPT) phases of matter, to find measurement-based constant depth circuits for states is a broad ongoing field of research~\cite{TVV_NonAbelianTopologicalOrder_2022, tantivasadakarn2021long, smith2023deterministic}. 
All these schemes have a $\LAQCC$ implementation.

%$\LAQCC$-circuits also exist for general schemes of preparing local states, based on the local tensors, and gluing them together using one round of entangled measurement and corrections, based on the local symmetry. 
%The main result of \citeauthor{Cirac:2021}, that 1D translational invariant MPS with fixed bond dimension can be prepared by $\mathsf{LOCC}$-assisted circuits, relies heavily on local symmetries of the MPS and as a result also has an equivalent $\LAQCC$ implementation. 
%The corrections applied after the measurement round are local unitaries depending on the local symmetries of the MPS. 

 

%This general scheme of preparing local states, based on the local tensors, and gluing it together by doing one round of entangled measurement and corrections, based on the local symmetry, is accessible in $\LAQCC$.
Note however that \citeauthor{Cirac:2021} also suggest a circuit for the $W$-state.
This circuit uses sequentially and dependent measurement-based corrections of the ancilla qubits. 
These dependent measurements translate to sequential alternations between the quantum and classical circuits and therefore increase the total depth to linear depth, exceeding the constant-depth constraints imposed by $\LAQCC$-circuits. 

We study the power of the $\LAQCC$ model with respect to state preparation, showing that even with only constant quantum-depth and logarithmic classical depth it remains possible to prepare states with long-range entanglement.
Another surprising result is that it is unlikely that $\LAQCC$ circuits are classically simulatable. We show that any instantaneous quantum polynomial-time (IQP) circuit~\cite{Bremner2010,Shepherd2009} has an $\LAQCC$ implementation.
Classical simulation of IQP circuits implies the collapse of the polynomial hierarchy to the third level, which is not believed to be true~\cite{Bremner2017}. Therefore, we expect that $\LAQCC$ circuits are unlikely to be classically simulatable. We bound the power of $\LAQCC$ by showing that it is contained in $\QNC^1$, the class of polynomial-size, log-depth circuits.

Next, we also study the power that intermediate classical calculations can add to quantum computations, by considering a new model that alternates between polynomially many polynomial-depth quantum circuits and unbounded classical computations
We study this model by doing a complexity theoretical analysis, where we draw inspiration from the notions of complexity given by \citeauthor{RosenthalYuen:2022}, \citeauthor{MetgerYuen:2023}, and \citeauthor{Aaronson:2004}.
All three complexity notions are based on the notion of state preparation, instead of more traditional definition of complexity such as the decidability of a computational problem. 
The first two consider classes based on sequences of quantum states preparable by a polynomial-sized quantum circuit, where the circuits are uniformly generated by a computational class, for instance, the class $\mathsf{PSPACE}$, which results in the complexity class $\mathsf{StatePSPACE}$~\cite{RosenthalYuen:2022,MetgerYuen:2023}.
The third notion considers a relative complexity, where the complexity is measured between two given states, and is measured by the number of gates, from a given gate-set, required to transform one state in another state~\cite{Aaronson:2004}. 
For our definition of state preparation complexity, we drop the uniformity constraint from~\cite{RosenthalYuen:2022,MetgerYuen:2023} and define a class as $\mathsf{StateX}$, which refers to states preparable by circuits of type $\mathsf{X}$. 
As an example, if $\mathsf{X} = \QNC^0$, this results in the class $\mathsf{StateQNC^0}$, which is the set of states preparable from the $\ket{0}^n$ state by poly-size constant-depth circuits. 
This notion is similar to the relative complexity from~\cite{Aaronson:2004}, where one state is the  $\ket{0}^n$ state and instead of counting the number of gates we consider the set of states preparable by a fixed number of gates. Using this notion of complexity we show that any state preparable by an $\LAQCC^*$ circuit is also preparable by a $\mathsf{PostQPoly}$ circuit, the class of circuits of polynomial depth with an additional post-selection gate. 

All Clifford circuits have a constant-depth $\LAQCC$ implementation, implying that any stabilizer state can be implemented by a constant-depth $\LAQCC$ circuit, see Section~\ref{sec:clifford_circuits} for a proof of this statement. 
Efficient circuits for stabilizer states have been known already through measurement-based quantum computing. Therefore this paper focuses on the preparation of non-stabilizer states, and as a surprising result we find novel constant-depth protocols for four very natural classes of non-stabilizer states.
Despite the extensive research into these four classes of non-stabilizer states and the many applications of them, no efficient constant- or low-depth state preparation protocols are known yet. We specifically consider these four classes as they are all often used as initial states in other algorithms.

The first state is a uniform superposition over an arbitrary number of states. 
This state finds applications in many quantum algorithms, as they often start with a uniform superposition over multiple states. 
This superposition is often achieved by applying Hadamard gates to every qubit due to its simplicity to prepare. 
Yet, the analysis of many algorithms, such as Shor's algorithm~\cite{Shor:1997}, would benefit from a different initial superposition. 
The circuit to prepare the uniform superposition over an arbitrary number of states uses an exact version of Grover search as a subroutine, that turns a probabilistic circuit, with a known constant probability of success, into a deterministic circuit. 
We use the circuit for preparing a uniform superposition over an arbitrary number of states as a subroutine in the next two quantum state preparation protocols. 

The second state is the $W$-state, the uniform superposition over all computational basis states of Hamming-weight~$1$, a natural long-ranged entangled state that displays a fundamentally nonequivalent type of entanglement from the Greenberger–Horne–Zeilinger state~\cite{WState:2000}, for which $\LAQCC$-type constant-depth circuits were previously known~\cite{PhamSvore2013, Cirac:2021}. 
The $W$-state is often used as benchmark for new quantum hardware~\cite{Haffner2005,Neeley2010,GarciaPerez:2021}. 
A novel way to prepare the $W$-state therefore gives a new way to benchmark different quantum devices with each other. 
A circuit for preparing the $W$-state was given in~\cite{Cirac:2021}, but this implementation requires sequentially alternating measurements followed by local unitaries, which in the $\LAQCC$ model is not considered to be of constant depth. 
We improve this protocol by giving an $\LAQCC$ implementation of the $W$-state, based on a compress-uncompress method that links the one-hot and binary encoding of integers.

The third state considered is the Dicke state, a generalization of the $W$-state, a superposition over all computational basis states with Hamming-weight $k$~\cite{Dicke:1954}. 
Dicke states have relevance in various practical settings.
For instance, for quantum game theory~\cite{zdemir2007}, quantum storage~\cite{Bacon_Compress:2006,Plesch:2010}, quantum error correction~\cite{ouyang2014permutation}, quantum metrology~\cite{toth2012multipartite}, and quantum networking~\cite{prevedel2009experimental}. 
Dicke states have been used as a starting state for variational optimization algorithms, most notably Quantum Alternating Operator Ansatz (QAOA)~\cite{Hadfield2019}, to find solutions to problems such as Maximum k-vertex Cover~\cite{Brandhofer2022,cook2020quantum}.
The ground states of physical Hamiltonians describing one-dimensional chains tend to show a resemblance to Dicke states such as states resulting from the Bethe ansatz, making them an ideal starting state when investigating the ground state behavior of these Hamiltonians~\cite{TDL_BetheAnsatzDerivation:2010,B_ExcitedStateQuantumPhaseTransitions:2013,DickeTransitions:2021}. 
For instance, the algorithm by \citeauthor{van2021preparing}, who give an algorithm to prepare the Bethe ansatz eigenstates of the spin-1/2 XXZ spin chain, starts by first preparing a Dicke state~\cite{van2021preparing}. 
A Dicke-state preparation protocol based on the compress-uncompress methodology used in the $W$-state furthermore finds applications in entanglement distillation, where the entanglement of a large state is concentrated on only a few qubits. 
Efficient deterministic circuits for preparing Dicke states have been proposed by \citeauthor{bartschi2019deterministic}~\cite{bartschi2019deterministic, bartschi2022deterministic_short_depth}. 
They provide a quantum circuit of depth $\mathO(k \log(\frac{n}{k}))$, allowing arbitrary connectivity, to prepare a Dicke state, which they conjecture to be optimal when $k$ is constant. 
In this work, we provide a constant-depth $\LAQCC$ circuit below their conjectured bound already for constant $k$. 
However, this does not directly disprove their conjecture, as we allow for intermediate measurements and classical computations. 
More significantly, we even construct constant-depth $\LAQCC$ circuits for $k = \mathO(\sqrt{n})$ greatly improving their bound.
This construction extends the compress-uncompress method for the $W$-state combined with additional subroutines. 

We continue with a log-depth state preparation protocol for the Dicke-state for arbitrary $k$. 
This protocol implements an efficient transformation between the factoradic number representation and the combinatorial number representation of a positive integer. 
The combinatorial number representation relates directly to the Dicke state. 
The provided efficient transformation between number representation systems might be of independent interest. 

We conclude by modifying our protocol for preparing a Dicke-state to a protocol that prepares quantum many-body scar states in constant-depth. 
These states have low entanglement and longer coherence times than states with similar energy density.
These characteristics make many-body scar states interesting to analyze and relevant within physics.
Many-body scar states appear for instance in the AKLT model~\cite{AKLT:1987,MRBAR:2018,MRB:2018} and different spin models~\cite{SI:2019,MOBFR:2020}.
Known methods for preparing these states have polynomial-depth~\cite{Gustafson:2023}, whereas our circuit has constant depth. 

% We conclude by studying the power that intermediate classical calculations can add to quantum computations. 
% In this study, we define a new model that relaxes constant-depth quantum circuits to polynomial depth quantum circuits, log-depth classical calculations to unbounded classical computations and a constant number of alternations to a polynomial number of alternations. 
% We call this model $\LAQCC^*$. 
% We study this model by doing a complexity theoretical analysis, where we draw inspiration from the notions of complexity given by \citeauthor{RosenthalYuen:2022}, \citeauthor{MetgerYuen:2023}, and \citeauthor{Aaronson:2004}.
% All three complexity notions are based on the notion of state preparation, instead of more traditional definition of complexity such as the decidability of a computational problem. 
% The first two consider classes based on sequences of quantum states preparable by a polynomial-sized quantum circuit, where the circuits are uniformly generated by a computational class, for instance, the class $\mathsf{PSPACE}$, which results in the complexity class $\mathsf{StatePSPACE}$~\cite{RosenthalYuen:2022,MetgerYuen:2023}.
% The third notion considers a relative complexity, where the complexity is measured between two given states, and is measured by the number of gates, from a given gate-set, required to transform one state in another state~\cite{Aaronson:2004}. 
% For our definition of state preparation complexity, we drop the uniformity constraint from~\cite{RosenthalYuen:2022,MetgerYuen:2023} and define a class as $\mathsf{StateX}$, which refers to states preparable by circuits of type $\mathsf{X}$. 
% As an example, if $\mathsf{X} = \QNC^0$, this results in the class $\mathsf{StateQNC^0}$, which is the set of states preparable from the $\ket{0}^n$ state by poly-size constant-depth circuits. 
% This notion is similar to the relative complexity from~\cite{Aaronson:2004}, where one state is the  $\ket{0}^n$ state and instead of counting the number of gates we consider the set of states preparable by a fixed number of gates. Using this notion of complexity we show that any state preparable by an $\LAQCC^*$ circuit is also preparable by a $\mathsf{PostQPoly}$ circuit, the class of circuits of polynomial depth with an additional post-selection gate. 

\paragraph{Summary of results}
\begin{itemize}
    \item We give a new definition of a computational model that captures the power of the four step process: applying a constant number of layers of one- and two-qubit gates; performing a syndrome measurement; perform a fast classical computation determining corrections; apply corrections. We call this model \emph{Local Alternating Quantum Classical Computations}, or $\LAQCC$ for short. In this model we bound the allowed quantum operations, intermediate classical calculations, and number of rounds separately. In Section~\ref{sec:LAQCC_model} we define this model and give a list of operations based on results from literature contained in this computational model. In some of these operations we explicitly use that we allow for multiple, but at most constant, rounds  of corrections.
    \item  We show show that there exist $\LAQCC$ circuits that can not be weakly simulated in Section~\ref{sec:IQP_in_LAQCC}. We further show that for every $\LAQCC$ circuit there exists a $\QNC^1$ circuit simulating it perfectly, in Section~\ref{sec:LAQCC_in_QNC1}.
    \item We introduce a new type computational complexity for preparing states and show that the extension of $\LAQCC$ where we allow a polynomial number of rounds and unbounded classical computation, is contained in $\mathsf{PostQPoly}$, the class of polynomial circuits with post-selection, in Section~\ref{sec:Complexity results}.
    \item We show a protocol to prepare the uniform superposition state of size $q$ in $\LAQCC$ using $\mathO(\ceil{\log_2(q)}^2)$ qubits in Section~\ref{sec:superposition_modulo_q}. 
    \item We show a protocol to prepare the $W_n$ state in $\LAQCC$ using $\mathO(n\log(n))$ qubits in Section~\ref{sec:W_state_in_LAQCC}.
    \item We show two ways of preparing the Dicke-$(n,k)$ state. The first method is in $\LAQCC$, works up to $k = \mathO(\sqrt{n})$, uses $\mathO(n^2\log(n))$ qubits, and is found in Section~\ref{sec:dicke:small_k}. The second method is in $\LAQCC\text{-}\mathsf{LOG}$ (an extension of $\LAQCC$ allowing for logarithmic number of alterations instead of constant), works for any $k$, uses $\mathO(\text{poly}(n))$ qubits, and is found in Section~\ref{sec:Dicke_in_LAQCC_LOG}. 
    \item We extend on our $\LAQCC$ method of generating Dicke-$(n,k)$ states for $k = \mathO(\sqrt{n})$ and show a protocol to generate many-body scar states for a particular Hamiltonian in $\LAQCC$ (Section~\ref{sec:many_body_scar}). 
\end{itemize}
Summarized in a table, we provide the following state generation protocols:
\begin{table}[htb]
\centering
\begin{tabular}{l|l|l|l}
\textbf{State description} & \textbf{Width} & \textbf{Depth} & \textbf{Implementation}\\
\hline 
Uniform superposition mod $q$: $\frac{1}{\sqrt{q}} \sum_{i = 0}^{q-1}\ket{i}$ & $\mathO(\ceil{\log^2 q})$ & $\mathO(1)$ & Section~\ref{sec:superposition_modulo_q}\\

$W$-state: $\frac{1}{\sqrt{n}}\sum_{i = 0}^{n-1}\ket{e_i}$ & $\mathO(n \log n)$ & $\mathO(1)$ & Section~\ref{sec:W_state_in_LAQCC}\\

Dicke-$(n,k)$, $k = \mathO(\sqrt{n})$: $\binom{n}{k}^{-1/2}\sum_{x \in \{0,1\}^n: |x| = k} \ket{x}$ &  $\mathO(n^2\log n)$ & $\mathO(1)$ 
&Section~\ref{sec:dicke:small_k}\\

Dicke-$(n,k)$: $\binom{n}{k}^{-1/2}\sum_{x \in \{0,1\}^n: |x| = k} \ket{x}$ & $\mathO(\text{poly}(n))$ & $\mathO(\log n)$ &Section~\ref{sec:Dicke_in_LAQCC_LOG}\\

QMBS: $\ket{S_k} = \frac{1}{k! \sqrt{\mathcal N(n,k)}}(Q^\dagger)^k \ket{\Omega}$ &  $\mathO(n^2\log n)$ & $\mathO(1)$  &  Section~\ref{sec:many_body_scar}
\end{tabular}
\caption{Summary of state preparation protocols given in this paper.}
\label{tab:sate_prep}
\end{table}
In the entry for the quantum many-body scar state $Q$ denotes the raising operator and $\mathcal N(n,k)=\binom{n-k-1}{k}$. 
Section~\ref{sec:many_body_scar} will provide more details on the variables and the implementation. 

\paragraph{Organization of the paper}
\noindent We first introduce relevant preliminaries in Section~\ref{sec:preliminaries}. 
In Section~\ref{sec:LAQCC_model} we formally define the class of Local Alternating Quantum-Classical Computations ($\LAQCC$). We also show that any Clifford circuit can be implemented in constant depth $\LAQCC$ (a result based on a result from measurement-based quantum computing~\cite{jozsa2006introduction}). 
This result allows us to give many useful multi-qubit gates and routines in Section~\ref{sec:gates_created_in_LAQCC}. 
Beyond that we show that constant depth $\LAQCC$ circuits are contained in $\QNC^1$ and that any $\mathsf{IQP}$ circuit has an $\LAQCC$ implementation.
We conclude this section with an analysis of a more powerful instantiation of $\LAQCC$ and show an inclusion with respect to the class $\mathsf{PostQPoly}$, which is the class of circuits of polynomial depth with one additional post-selection gate. 
In Section~\ref{sec:state_prep_in_LAQCC} we give $\LAQCC$ circuit implementations for preparing the uniform superposition over an arbitrary number of states, the $W$-state and the Dicke state up to $k = \mathO(\sqrt{n})$. We furthermore give a log-depth circuit implementation for preparing the Dicke state for any $k$. We conclude by showing a $\LAQCC$ circuit for generating many body scar states of a particular type of Hamiltonian.


\section{Related work on REG in context}\label{sec:litreview}

Deciding about the form of a referring expression and determining its content are two different steps of the Referring Expression Generation (REG) task \citep{comreg2019}. In the current article, our focus is on the first step, namely determining the form of a referring expression. We will expand thiw work to the second task, namelz the content realisation of the REs in future work.
 

\subsection{Different RF categories} \label{subsec:refcateories}

As \citet{kibrik2016referential} put it, the \textit{basic} and binary referential choice is between the choice of a pronoun versus a variety of NPs. Studies addressing pronominalization, such as \citet{mccoy1999generating,poesio2004centering} and \citet{henschel2000pronominalization}, often focus on this binary distinction. More recent studies 
have looked at a wider range of referring expression (RE) types. For instance, the GREC shared tasks \citep{belz2009generating} exploited four RE types, namely pronoun, proper name, common noun and covert (empty) reference. \citet{kibrik2016referential} focused on a three-way distinction between the choice of a pronoun, proper name and common noun; while \citet{castro-ferreira-etal-2016-towards} classified REs into five categories of pronoun, proper name, common noun, demonstrative NP and empty reference.

\subsection{Different approaches to REG in context} \label{subsec:regapproach} 


Different methods are used to predict the referential choice in context. Rule-based approaches, such as \citet{passonneau1996using}, \citet{mccoy1999generating}, \citet{henschel2000pronominalization} and \citet{krahmer2002efficient}, employ different algorithms to predict RF choice-taking, for instance, centering rules or salience-based accounts into consideration. The GREC shared-task challenges, as one of the first systematic studies on the generation of REs in context, introduced new
feature-based Machine Learning (ML) solutions to this task (e.g. \citet{greenbacker-mccoy-2009-udel, hendrickx-etal-2008-cnts, bohnet-2008-g, favre-bohnet-2009-icsi}, among others). 
Following these shared tasks, \citet{kibrik2016referential} trained decision trees and regression models on the WSJ MoRA, a corpus of Wall Street Journal articles, using a large number of factors. In a more recent feature-based study, \citet{castro-ferreira-etal-2016-towards} trained Naive Bayes and Recurrent Neural Network (RNN) algorithms on the VaREG corpus, taking individual differences in the generation of REs into account. Over the past few years, deep learning approaches have been pre-dominantly used for an end-to-end generation of REs in context, predicting type and content of expressions altogether \citep{castro-ferreira-etal-2018-neuralreg,cao-cheung-2019-referring,cunha-etal-2020-referring,same-etal-2022-non}. \citet{chen-etal-2021-neural-referential} have used pre-trained language models for the choice of RF, but they only use the benchmark NLG dataset, namely WebNLG \citep{gardent-etal-2017-webnlg, castro-ferreira-etal-2018-enriching}, in their study and only use \bert. 

\section{A game-theoretic, doxastic framework}\label{sec:relFrameIR}
Above in \autoref{sec:designproc}, we describe when in the design process of a safety-critical autonomous system \HAS our approach of determining relevance can be helpful. 
In this section we are concerned with the formal ingredients of the framework. 
We discuss  the decisions taken in the design of the framework and point to related work.

In the terms of the IR literature, our relevance notion can be described as 
\emph{situational} --the circumstances of \HAS are taken into account--, 
\emph{subjective} --relevance is determined from the view point of \HAS--, 
\emph{goal-implied} --the goals of \HAS determine whether \HAS misses something relevant--, 
\emph{temporal and spacial} --the performance of \HAS during a maneuver is examined within space and time as captured in \universeD.
%and  \emph{context-dependent} --\HAS is considered with the context of \universeD--. 
The framework integrates these different dimensions, so that we can apply game theory to determine what observations and knowledge is necessary.

\emph{How does the framework integrate so many dimensions of relevance? How can a decision-procedure answer whether something is relevant?}

In short, we model beliefs on the one hand and we use a model of the application domain, \universeD, as ground truth on the other hand. 
We link the two via a two-player dynamic game -- one player is the autonomous system \HAS and the other player is the environment.	

%%%%%%%%%%%%%%%%%%%%%%%%%%%%%%%%%%%%%%%%%%%%%%%%%%%%%%%%%%%%%%%%%%%%%%%%%%
%
\subsection{Scope of the Framework}
%
%%%%%%%%%%%%%%%%%%%%%%%%%%%%%%%%%%%%%%%%%%%%%%%%%%%%%%%%%%%%%%%%%%%%%%%%%%
We aim to support the development of safety-critical autonomous systems that can partially observe their environment. 
Their perceptions may be perturbed or may be contradicting each other. 
We assume that a system \HAS additionally uses its knowledge base to construct its beliefs. 
The knowledge base holds insights about the application domain, that an engineer provided at design time, as well as statements that \HAS gets from trusted sources during its mission.

By asking \enquote{What knowledge and what observations are necessary to build beliefs upon which \HAS can achieve its goals?} we treat belief formation as a central ingredient of our framework. 
Accordingly we use a \emph{doxastic model}, that is a model that captures beliefs explicitly.

A system \HAS necessarily builds approximating beliefs since its environment is vastly complex while its resources are limited (cf. \autoref{fig:beliefDeviation}).
A system \HAS aims for beliefs that capture the \emph{relevant} aspects.
Allowing the most freedom in building such beliefs provides the greatest potential for saving resources. 
% Figure environment removed

A belief describes what a system \HAS thinks is currently possible. 
To this end we use the possible world semantics \cite{reasoningA}. Accordingly, a belief is a set of possible worlds. 
Since our worlds capture their believed history, current state and future  we call them alternative realities (cf. \autoref{fig:relevanceRelationHAS}).% 
\footnote{The possible worlds semantics is often captured via Kripke structure $K$ where $K$'s states represent the worlds and $K$'s state transitions represent the accessibility relation, \ie $s\rightarrow s'$ means in world $s$ $s'$ is a possible world.},
We model that \HAS judges the best action based on its current belief. It does this by simulating whether the action will lead to a mission success in the future  of the believed realities.

Since we aim to characterize whether the system \HAS achieves its goals when choosing its actions based on its believes, we link the belief formation to ground truth \universeD, as illustrated in  \autoref{fig:worlds}. 
The feedback loop of \enquote{A system \HAS builds its beliefs based on its perceptions of \universeD.},
\enquote{A system \HAS chooses its actions based on its beliefs.} and
\enquote{A system \HAS's actions influence the state of \universeD.} establishes this link.

% Figure environment removed
Since we want to determine whether a sufficient belief can be formed by approximation of the ground truth \universeD,  we explicitly support beliefs that are structurally distinct from \universeD (cf. \autoref{fig:beliefDeviation}).
Therefore, the ground truth \universeD and the beliefs of \HAS are two separate structures in our framework.

In the framework, we model that \HAS has a knowledge base that captures insights built-in by engineers or received by trusted sources. 
The entries of the knowledge base represent \emph{believed knowledge}, that is \HAS thinks that the entries are true. 
But it is possible that the statements are false.
Our motivation of modelling a base of believed knowledge is, that \HAS will be equipped with rules approximating the reality. 
In order to detect rules and insights that are too coarse, we have to be able to model them in our framework. 
In the sequel we will often refer to the entries of the knowledge base simply as knowledge.

Given a belief formation, we use game theory to determine whether an autonomous safety-critical system \HAS will be successful in \universeD. We also use game theory to determine whether \HAS can form beliefs such that it will be successful. 
We regard a maneuver of \HAS as a dynamic game of the player \HAS and the surrounding world, which might include other agents. 
The system \HAS can control its actions while concurrently the environment chooses from its actions.  The combined actions determine the state change of the \universeD. 
We hence can examine evolutions along \HAS's maneuver in time and space with evolving context.  


At its core relevance is a relationship, as mentioned in \autoref{sec:relIR}. 
We examine what knowledge and observations of the world are relevant for \HAS.
\enquote{$X$ is relevant} entails \enquote{having made observation $X$/knowing $X$ makes a difference to \HAS} and not knowing/observing $X$ would hinder \HAS in achieving its goals in \universeD. 
We capture this aspect by defining knowledge/observations $X$ to be relevant, if there is no \enquote{smaller} $X'$ which enables \HAS achieving its goal (cf.~\autoref{def:wr}). 
We do this analogously to \cite{DF11}, where the minimal perimeter of a world model is determined. 
In a nutshell, we explore how well a system \HAS performs when we omit knowledge and observations.
If omission leads to a worse performance the omitted was relevant. 


The framework is concerned with \HAS's assessment of the environment via its sensors. 
To this end it only models first-order beliefs but not higher-order beliefs, i.e., beliefs about beliefs. 
Thus system $A$ cannot argue: \enquote{System $B$ will slow down -- I think that $B$ thinks there is a speed limit} or \enquote{System $B$ will slow down -- I  think that $B$ thinks that I think that $B$ should slow down}.  
Including higher-order beliefs will increase the overall complexity of the model. 
There are certainly application where modelling high-order beliefs is essential. We imagine that higher-order beliefs are essential when designing entertaining or comfort functions. 
There the mental state of a user has be taken into account and the system aims to optimally support the user rather guaranteeing goals. 
In contrast, safety-critical systems usually take decisions based on conservative approximations in order to be on the safe side. 



In game theory  \emph{rationality} is an central notion. 
Basically, a rational agent $A$ does what promises to result in the outcome $R$ that is best for $A$.
Different notions of rationality exist in literature varying in how to precisely and appropriately capture this notion for a given application.
We assume that \HAS chooses the action that it thinks will lead to the best result.
The system \HAS simulates the effect of its actions in its mind, i.e. it examines the effect on the current set of possible worlds.
So \HAS takes rationally belief-based decisions. 
We do not assume though, that \HAS rationally forms beliefs. 
For instance, we allow  that \HAS believes an object to be red, although according to its observations it is blue, we also allow \HAS to believe that an object is a house at one time instance and at the next time instance \HAS believes it is a tree.
We decided not to constrain the belief formation because of the way beliefs are constructed in autonomous systems. 
The belief of \HAS may be determined by a composition of different components, and there may not necessarily be an entity that ensures that the resulting belief is rational\footnote{What rational in this context should capture, would have to be discussed first.}. 

Our framework, nevertheless, supports the study of different kinds of belief formation functions and we consider it future work. 
We imagine that during the design, requirements regarding the belief formation might be specified. 
So, whether a belief formation exists, that satisfies the requirements, might be valuable insight when developing safety-critical autonomous systems. 
In this line of research, we are also interested in the formalisation of classes of requirements on the belief formation.
In particular, we are interested in belief formations satisfying certain robustness or stability criteria. 
A notion of robustness of  belief formation might express that a given rate of object misclassification can be tolerated.
A stability criterion might express that the beliefs are formed such that replanning is rare and triggered sufficiently early.


%%%%%%%%%%%%%%%%%%%%%%%%%%%%%%%%%%%%%%%%%%%%%%%%%%%%%%%%%%%%%%%%%%%%%%%%%%%%%%%%%%%
%
\subsection{Works related to the Formal Approach \cite{doxFrame}}
%
%%%%%%%%%%%%%%%%%%%%%%%%%%%%%%%%%%%%%%%%%%%%%%%%%%%%%%%%%%%%%%%%%%%%%%%%%%%%%%%%%%%
Epistemology is the theory of knowledge and concerned with information-processing and cognitive success \cite{CollinsEpi,Sheffield}.
Doxastic means \enquote{relating to belief}~\cite{Collins}. 
By using the term \enquote{doxastic}, we want to stress that our formalism focuses on beliefs.
%
%
In the epistemic logic literature, the semantics of doxastic languages are often given via \emph{doxastic models}, that are special Kripke structures \cite{EpistemicsLogics}.
A doxastic model $(S,v,\rightarrow_i)$ consists of a set of nodes $S$ representing possible worlds $w$, a valuation function $v: S\rightarrow 2^{\Props}$ for the set of atomic facts \Props and a belief relation $\rightarrow_i$  for each player $i$, that specifies \enquote{$i$ deems $w'$ possible in $w$} if \enquote{$w$ $\rightarrow_i$ $w'$}. With other words, the belief of $i$ at $w$ is defined as the worlds accessible via the agent $i$'s belief relation, $\rightarrow_i$ \cite{EpistemicsLogics,DoxasticMeyer2003}. 
%
%


In this paper, we use complex possible worlds instead of the plain nodes of a Kripke structure. 
In our framework, each possible world is a Kripke structure itself, called alternative reality.
It encodes the believed histories, the current states and possible futures. 
%
A system \HAS uses alternative realities to simulate its strategy in order to decide on its current action. 
In our framework, a reality constitutes an extensive form two-player zero-sum game, where the winning condition is defined by the list of linear temporal logic (LTL) goals of \HAS.
The belief formation is based on partial observations and the currently available knowledge/strong beliefs.

A couple of epistemic temporal logics have been suggested for specifying aspects of knowledge throughout time for multi-agent systems. These logics combine temporal logics with knowledge operators, like KCTL \cite{CTLKMC}, KCTL$^*$ or HyperCTL$^{*}_{lp}$~\cite{HyperCTL}. They are interpreted over Kripke structures. 
But since an agent $i$ has its local view, only certain propositions are assumed to be observable, so that an observational equivalence relation $\sim_i$ on the traces arises. 
\enquote{Agent $i$ knows $\varphi$} then means that $\varphi$ holds on all $i$-equivalent initial traces. 
%
The alternating time temporal logics (ATL)~\cite{ATL} has been developed for reasoning about what agents can achieve by themselves or in groups throughout time. 
In ATL, the path quantifiers of CTL are replaced by modalities that allow to quantify paths in the control of groups of agents.
ATL is interpreted over concurrent game structures~(CGS), which are labelled state transition systems. 
By adding a knowledge operator, ATL has been extended to an epistemic variant, ATEL~\cite{ATEL}. 
To this end the concurrent game is extended by an observational equivalence relation per agent modelling the agent's limited view.
%

Just like the logics above, we assume that \HAS can only partially observe the ground truth.  
Our beliefs, however, cannot straightforwardly be expressed in terms of an equivalence on the ground truth, since an alternative reality may be a distinct Kripke structure and a belief does not have to include the ground truth.
%
In contrast to be above logics, we use in our framework a variant of LTL to specify constraints on the beliefs. 
A  so-called  BLTL formula is therefore interpreted on a belief \belief, i.e. a set of alternative realities. 
Since the set of possible beliefs \Beliefs is finite, a formula $\K\varphi$ means the finite conjunction $\bigwedge_{\reality\in\belief}\reality\models\varphi$. 


The field of epistemic planning is concerned with computing plans (\enquote{a finite succession of events} \cite{DELStrat}) that achieve the desirable state of knowledge from a given current state of knowledge~\cite{GIntroDEL}.  
DEL, dynamic epistemic logic, is a formalism to describe planning tasks succinctly by a semantic and action model based approach. 
Epistemic models capture the knowledge state of the agents, and epistemic action models describe how these are transformed. 
An evolution results from  a stepwise application of the available actions. 
In \cite{DELStrat} distributed synthesis of observational-strategies for multiplayer games are considered.  
%
While ATEL and DEL allow for reasoning about a combination of knowledge and strategies, we are interested in the belief \emph{formation}. 
We ask whether there exists a belief formation that justifies a strategy that successfully achieves temporal goals within a given ground truth world. 

Properties of belief formation are studied in the field of belief revision and update. 
Belief revision is done when a new piece of information contradicts the current information, and it aims to determine a consistent belief set. 
Belief updates may be necessary when the world is dynamic~\cite{BelRevIntro}. 
The works in this field are concerned with rational belief formation, following e.g. some guiding principle like making minimal changes~\cite{BelRevIntro}. 
%
In our work, we consider very general belief formation functions, since we focus on safety-critical autonomous systems.

BDI agents are rational agents with the mental attitudes of belief~(B), desire~(D) and intention~(I)~\cite{BDI}.
Beliefs describe what information the agent has, 
desires represent the agent's motivational state and specify what the agent would like to achieve, 
while intentions represent the currently chosen course of action. 
These attitudes allow an agent balancing between deliberation about its course of action and its commitment to the chosen course of action.
%
In our framework, an agent deliberates about its course of action at each state. 
We do not enforce commitment to a certain course of action, as we are interested in whether some belief formation exists.
Nevertheless, the framework conceptually allows capturing notions of commitment, and we plan to examine these in future work. 
Basically, a chosen action represents a set of believed best possible world strategies. These can be considered as the current intent. 
So, a notion of commitment could require that (some) strategies of the previous belief are still best strategies in the current belief.
An engineer may then specify when a system should be committed.



% !TeX root = ../MVFD_arxiv.tex


\section{The framework}\label{sec2}
In this section we  present a formal mathematical setup for the local regularity  for bivariate   stochastic processes  (also called scalar random fields, or simply random fields) and the data observed for such processes.

\subsection{Data}\label{sec:data}
Consider $N$ independent realizations, also called sheets, $X^{(1)},\ldots,X^{(j)},\ldots X^{(N)}$   of a stochastic process $X $ defined on a continuous domain $\cT\in\mathbb R^2$. For simplicity, we here focus on domains $\mathcal T$ in the plane, the extension to higher dimensions would not raise different challenges. For the purpose of describing our methodology, we distinguish three observational scenarios of the $N$ realizations. 
First, the ideal, infeasible situation where the sheets $X^{(j)}$ are \emph{completely observed}, that is without error over the entire domain $\cT$.  
The second case is the one where the $X^{(j)}$ are observed (measured) at some \emph{discrete points} in the domain $\cT$, \emph{without noise}. The domain points can be fixed to be the same for all the $X^{(i)}$'s (common design), or can be randomly drawn for each sheets separately (independent design). Finally, the most realistic scenario is the one where in the second case we admit that the realizations of $X$  are observed at discrete domain points \emph{with noise}. 



 To formally describe the second and third scenarios, let  $M_1, \dotsc,M_N$ be an independent sample of an integer-valued random variable $M$ with expectation $\EE[M]=\Mmu$. 
% which increases with $N$. 
In the independent design case, for each $1\leq j \leq N$, and given  $M_j$, let $\Tnm\in\cT$,  $1\leq m \leq  M_j$, be a random sample of a random vector $\TT\in\cT$. The $\Tnm$'s represent the observation points for the realization $\Xp{j}$. We assume that the realizations of $X$, $M$ and $\TT$ are mutually independent. 
In the common design case, $M\equiv \mathfrak m$ and the $\Tnm$'s are the same for all $j$.  Let $\mathcal T_{obs}^{(j)} $ denote the set of observation times $ \Tnm $, $1\leq m \leq M_j$, on the sheet $\Xp{j}$. With common design,  $\mathcal T_{obs}^{(j)} $ does not depend on $j$, while with independent design the expected cardinal of $\mathcal T_{obs}^{(j)} $ can be random with mean $\mathfrak m$.   The following presentation  includes both independent design and common design cases. \color{black}  Finally, the data  consist of  the pairs  $(\Ynm , \Tnm ) \in\mathbb R \times \cT $ where $\Ynm$ is defined as
\begin{equation}\label{model_eq}
	\Ynm = \Xn (\Tnm) + \enm, \quad\text{with}  \quad  \enm = \sigma(\Tnm,\Xn(\Tnm)) \unm, 
	\quad 1\leq i \leq N,  \; 1\leq m \leq M_j.
\end{equation}
Here, the $\unm  \in\mathbb R $ are independent copies of a centered variable $e$ with unit variance, and $\sigma^2(\cdot,\cdot)\geq 0$ is some unknown, bounded conditional variance  function which account for possibly heteroscedastic measurement errors. The case  $\sigma(t,x)\equiv 0$ corresponds to our second scenario, while in the third scenario we have positive conditional variance. 

For each $1\leq j \leq N$,  let $\widetilde X^{(j)}$ denote an observable approximation of $X^{(j)}$. If   the sheets $X^{(j)}$ were  completely observed, as in our infeasible first scenario, $\widetilde X^{(j)} = X^{(j)}$. 
When $X^{(j)}$ are observed  only at  some discrete points $\Tnm$,  arbitrary  $\widetilde X^{(j)}(\Tt)$ can be obtained by simple interpolation or defined equal to the value of $\widetilde X^{(j)}$ at the nearest neighbor of $\Tt$.  Finally, with noisy, discretely observed sheets,   $\widetilde X^{(j)}$  is a pilot nonparametric estimator  of $X^{(j)}$, such as kernel smoothing, splines \emph{etc}.  


Let us next introduce a general class of stochastic processes (random fields)  $X$ with  irregular realizations $X^{(j)}$, for which the regularity can vary over the domain $\mathcal T$. 


\subsection{A class of multivariate processes}
Let $\mathcal{T}$ be an open, bounded bi-dimensional rectangle with the closure included in $(0,\infty)^2$. In the following, $H_1,H_2 : \mathcal T \to (0,1)$ are two continuously differentiable functions such that 
\begin{equation}\label{low_thres}
\min_{i=1,2} \inf_{t\in\mathcal T} H_i(\Tt) >0.
\end{equation} 	
Let $\overline{H} = \max\{H_1 ,H_2\}$.
We also consider the vector-valued function $\mathbf{L}=(L_1^{(1)},L_2^{(1)},L_1^{(2)},L_2^{(2)}),$  where the components  are  non-negative, Lipschitz  continuous functions defined on $\mathcal{T}$ such that 
\begin{equation}\label{id_L}
	L_i^{(1)}(\Tt) +L_i^{(2)}(\Tt) >0,\qquad \forall \Tt\in\mathcal T, \; i=1,2.
\end{equation}
%We denote $\mathbf{L}=(L_1^{(1)},L_2^{(1)},L_1^{(2)},L_2^{(2)}),$ and assume that a constant $C_{\mathbf L}$ exists such that 
%\begin{equation}\label{up_thres}
%C_{\mathbf{L}}=\max_{i,j=1,2} \sup_{t\in\mathcal T} L_j^{(i)}(\Tt) <\infty.
%\end{equation} 	

Let $X$ be a real-valued, second order stochastic process defined on $(0,\infty)^2$. Let $(e_1,e_2)$ be the canonical basis of $\mathbb R ^2,$ and, for sufficiently small scalars $\Delta$, let   
$$
\theta_{\Tt}^{(i)}(\Delta)=\EE\left[\left\{X\left(\Tt-\frac{\Delta}{2}e_i\right)-X\left(\Tt+\frac{\Delta}{2}e_i\right)\right\}^2\right],\quad i=1,2.
$$ 


\begin{definition}\label{def}
Let $H_1$, $H_2$ satisfy \eqref{low_thres}.	The class $\mathcal {H}^{H_1,H_2}(\mathbf{L},\mathcal{T})$ is the set of stochastic processes $X$ satisfying the following condition:  constants $  \Delta_0, C,\beta>0$ exist such that 
	for any $\Tt\in \mathcal T$ and $ 0<\Delta\leq\Delta_0$,  
	\begin{equation}\label{as_repr}
		\left|\theta_{\Tt}^{(i)}(\Delta)-L_1^{(i)}(\Tt)\Delta^{2H_1(\Tt)} -L_2^{(i)}(\Tt)\Delta^{2H_2(\Tt)}\right|\leq C\Delta^{2 \overline{H}(\Tt)+\beta}, \quad i=1,2.
	\end{equation}
	Let  
	$$
	\mathcal{H}^{H_1,H_2} %=\mathcal{H}^{H_1,H_2}(\mathcal{T})
	=\bigcup_{\mathbf{L}}\mathcal {H}^{H_1,H_2}(\mathbf{L},\mathcal{T}) ,
	$$
	where the union is taken over the set of four-dimensional functions $\mathbf{L}$ with non negative positives Lipschitz  continuous components satisfying \eqref{id_L} 
	%and \eqref{up_thres}. 
	The functions $H_1,H_2$ define the local regularity of the process, while $\mathbf{L}$ represent  the local Hölder constants.
\end{definition}

Definition \ref{def}  is general, and extends the local regularity notion considered by \cite{GKP} for processes defined on a compact interval on the real line.  A main example we have in mind is the  multi-fractional Brownian  sheet (MfBs) with a time-deformation. MfBs  is a generalization of the standard fractional Brownian sheet, where the Hurst parameter is allowed to vary along the  domain. The definition of this general class of  processes and some of their properties are provided in Section \ref{BfMs}.




%%%%%%%%%%%%%%%%%%%%%%%%%%%%%%%%%%%%%%%%%%%%%%%%%%%%%%%%%%%%%%%%%%%%%%%%%%%%%%%%%%%%
%%%%%%%%%%%%%%%%%%%%%%%%%%%%%%%%%%%%%%%%%%%%%%%%%%%%%%%%%%%%%%%%%%%%%%%%%%%%%%%%%%%%%
\section{Autonomous Decisions\cite{doxFrame}}\label{sec:autonom}
%%%%%%%%%%%%%%%%%%%%%%%%%%%%%%%%%%%%%%%%%%%%%%%%%%%%%%%%%%%%%%%%%%%%%%%%%%%%%%%%%%%%%
%%%%%%%%%%%%%%%%%%%%%%%%%%%%%%%%%%%%%%%%%%%%%%%%%%%%%%%%%%%%%%%%%%%%%%%%%%%%%%%%%%%%%
%
\normalsize
In this section, we formalize a notion of \emph{autonomous decision} and then characterize when a system exists that can take autonomous decisions to accomplish their goals.
%

Notions of autonomy are discussed in various scientific fields, as we outline in \autoref{sec:autoSys}.
Our formalization aims at capturing that \texttt{(bel\label{as:belief})} an autonomous system must take decisions based on its internal world view (i.e. its belief) and that \texttt{(rat\label{as:rational})} the system chooses the choice alternative that promises the best outcome, i.e. the system is somehow rational. 
We distinguish autonomous decisions from \emph{automatic decisions}, that play out rule-determined choices and are not the rational consequence \wrt the belief content.
The difference between autonomous and automatic decisions is illustrated by \autoref{ex:autonAutom} below.
%
\begin{mexample}{Autonomous decisions \vs automatic decisions}\label{ex:autonAutom}
	Suppose that \ego has a permanently broken sensor that flips the colors (cf. \autoref{fig:switched}) and that \ego believes in its sensors. 

	% Figure environment removed
		So, if \other is red, \ego thinks that \other is blue and vice versa.
	Suppose moreover, that \ego knows that a red car is hasty and a blue car is slow.
	If \ego decides autonomously, then it will follow a strategy highlighted by bold arrows, \ie it will go straight on, if a red car is approaching and it will take the turn when a blue car is approaching. This strategy promises the best outcome \wrt \ego's beliefs.
	 Since it beliefs in its sensors and chooses rationally, it takes the worst possible decisions. 

	Let us now consider a system that plays out automatic decisions. Suppose an engineer is aware, that the sensor switches colours\footnote{Maybe just under certain conditions and hence the engineer did not bother to change the belief formation and decided to patch this problem by implementing a rule}. He hence equips \ego with a rule, that switches the chosen actions accordingly: \enquote{Do not take the turn, when you think a red car is approaching. Takes the turn, when you think a blue car is approaching}. 
		This choice does not make sense to \ego, it is not rational \wrt to its belief content, but, in our scenario it is better when evaluated on the design time world.
\end{mexample}
%
	The above example \autoref{ex:autonAutom} highlights, that autonomous decisions are not necessarily better than automatic decisions. 
	Since the system \HAS is missing some relevant aspect of the design time world, it takes the wrong decisions. 
	We characterize in Def.~\ref{def:optimal} when a system \HAS can act successfully.
	
\begin{excursus}{Autonomous vs Automatic}
	Although, in the above example the automatic system outperforms the autonomous system,
	autonomous systems promise to be more capable of dealing with new situations. 
	An autonomous system chooses the best possible option based on extrapolation of the system's world view.	
	So once a robust world model for extrapolation has been build, an autonomous system will take \enquote{sensible} decisions.
	Hence a design task when building a system \HAS, is to determine the relevant aspects of the world models and to assess the impact of missing information and sensor perturbations.
	The quality of this extrapolation can be validated by means of runtime monitoring.

Since an automatic system plays out rules, an unforeseen event in the real world might result in situations where no rule applies anymore. 
	The challenge for developing an automatic system hence lies in defining a robust set of rules.  
	This set of rules also has to be evaluated regarding all possible scenarios. 
	It has to be evaluated whether all cases have been identified, when the rule should trigger and whether in these situations, the encoded behaviour is appropriate.
	The expected occurrence a situations satisfying the rule's antecedent are often quite rare, what makes the validation during runtime more difficult.
\end{excursus}

In order to formalize when an \emph{autonomous decision} can be taken successfully, we contrast 
\begin{enumerate}
	\item \emph{truth-observing strategies} (strategies that have access to ground-truth) with 
	\item \emph{doxastic strategies} (strategies that can only observe the formed beliefs) and 
	\item \emph{possible-world strategies} (strategies that run as simulation within the beliefs).
\end{enumerate}
The best truth-observing strategy represents what any system can possibly achieve.
The best doxastic strategy represents what a system with a given belief formation can possibly achieve. 
If the best doxastic strategy performs as good as the best truth-observing strategy, we say that an autonomous system is successful. 
Since our systems choose rationally, they choose what seems to be the best choice according to their belief content. 
The best possible-world strategy describes  what the system believes to be the best strategy in all possible worlds.

The different strategy notions are introduced step by step in the following and an overview of the notions is given in \autoref{tab:nutshell} on page \pageref{tab:nutshell}.
%


%%%%%%%%%%%%%%%%%%%%%%%%%%%%%%%%%%%%%%%%%%%%%%%%%%%%%%%%	%
%%%%%%%%%%%%%%%%%%%%%%%%%%%%%%%%%%%%%%%%%%%%%%%%%%%%%%%%	%
\subsection{Truth-Observing Strategy} 
%%%%%%%%%%%%%%%%%%%%%%%%%%%%%%%%%%%%%%%%%%%%%%%%%%%%%%%%	%
%%%%%%%%%%%%%%%%%%%%%%%%%%%%%%%%%%%%%%%%%%%%%%%%%%%%%%%%	%
In our framework, we use truth-observing strategies as reference of what would be achievable, if \ego could directly access the ground truth \universeD via a set of propositions $\props\subseteq\Props_{d}$. 
%%%
To this end, we say \ego implements a \emph{$\props$-truth-observing strategy} $\strats:(2^{\props})^+\rightarrow\Act_{\ego}$, if \ego chooses its actions based on the history of values of $\props$ as observed in the ground-truth model \universeD. 
When \ego is at state \state of \universeD, a state that was reached via path \Path with $\Label(\Path_{\leq i})|_{\props}=\history$ and $\state=\Path(i)$, it chooses ${\strats}(\history)$. 
A truth-observing strategy \strats together with a sequence of environment actions $\e\in\Act_{\env}^{\omega}$ determines a set of traces, $\Cmp(\e,\strats)$.
Formally, $\Cmp(\e,\strats) = \{\cmp_0 \cmp_1\ldots \in (2^{\Props_D})^\omega |
\exists \textit{ path } \Path \textit{ from } \Init_D, \forall i\geq 0:
\cmp_i = \Label_D(\Path(i)) \land 
\act_i:=\strats(\Label_D(\Path_{\leq i})|_{\props}) \land
(\act_i,\e(i))\in\Label_D(\Path_i,\Path_{i+1})\}$.

%%%%%%%%%%%%%%%%%%%%%%%%%%%%%%%%%%%%%%%%%%%%%%%%%%%%%%%%	%
%%%%%%%%%%%%%%%%%%%%%%%%%%%%%%%%%%%%%%%%%%%%%%%%%%%%%%%%	%
\subsection{Doxastic Strategy} Since a system \HAS has no direct access to the ground truth,  it has to decide based on its history of beliefs. 
%%%%%%%%%%%%%%%%%%%%%%%%%%%%%%%%%%%%%%%%%%%%%%%%%%%%%%%%	%
%%%%%%%%%%%%%%%%%%%%%%%%%%%%%%%%%%%%%%%%%%%%%%%%%%%%%%%%	%
We formalize this by the notion of doxastic strategy.
At a state $\state=\Path(i)$ in \universeD \ego takes a decision based on the history of its beliefs $\bhist_0\ldots\bhist_i$ that \ego has built along $\Path_{\leq i}$.
So to implement the \emph{doxastic strategy} $\stratb:\Beliefs^{+}\rightarrow\Act_{\ego}$ on $\universeD$,  \ego chooses $\stratb(\LabelB(\Path_{\leq i}))$.
%
A strategy \stratb together with a  sequence of environment actions $\e\in\Act_{\env}^{\omega}$ determines a set of traces in \universeD, just like for truth-observing strategies. 
The set of traces is $\Cmp(\e,\stratb) = \{\cmp_0 \cmp_1\ldots \in (2^{\Props_D})^\omega |
\exists \textit{ path } \Path \textit{ from } \Init_\epi, \forall i\geq 0:
\cmp_i = \Label_\epi(\Path(i)) \land 
\act_i:=\stratb(\LabelB(\Path_{\leq i})) \land
(\act_i,\e(i))\in\Label_\epi(\Path_i,\Path_{i+1})\}$.

Note that  doxastic strategy indirectly depends on what is observable:  the belief formation \LabelB (cf. Def.~\ref{def:bform}) observes only a certain set of observations.\\

\myparagraph{Dominance, $\strat'\leq_{\universe,\psi}\strat$}Since truth-observing and doxastic strategies both determine traces for a given sequence of environment actions,  we can compare them straight forwardly:  A strategy \strat achieves a goal list \goalList up to $n$ on \universe, if no matter what the environment does, \strat achieves $\goalList$ up to $n$, \ie the set $\bigcup_{\e\in\Act_{\env}^{\omega}}\cmp\in\Cmp(\e,\strat)$ satisfies $\goalList$ up to $n$ (cf. page \pageref{def:goals}).
%
A strategy \emph{\strat \goalList-dominates a strategy $\strat'$ on $\universe$}, $\strat'\leq_{\universe,\psi}\strat$, iff $\strat'$ achieves \goalList up to $n'$ and \strat up to $n$ where $n'\leq n$.
We also say \emph{$\strat'$ $\varphi$-dominates $\strat$}, $\strat'\leq_{\universe,\varphi}\strat$, for an LTL property $\varphi$, iff $\strat'\leq_{\universe,\psi}\strat$ for the goal list $\psi$ with the singleton goal set $\Phi=\{\varphi\}$.
We omit $\universe$ if it is clear from the context.
%
\begin{mexample}{Truth-Observing and Doxastic Strategies}\label{ex:tods}
	As an example of a dominant \props-truth-observing strategy, let us consider 
	\begin{itemize}[leftmargin=4mm]
		\item the goal list of \autoref{ex:goals}  on page \pageref{ex:goals}, ($\varphi_c$, \ie no collisions, is more important than $\varphi_t$,\ie do a turn), 
		\item the world model in \autoref{fig:setup}(b) on page \pageref{fig:setup}, 
		\item the propositions $\props:=\{\xego,\textsf{s},\textsf{h}\}$ to be observable by \strats and 
		\item the strategy \strats that chooses to drive straight on, if \other is hasty, and that chooses to turn, if \other is slow\\
	(it maps \Small{$1s\mapsto f$, $1s,2s\mapsto f$, $1s,2s,3s \mapsto f$, $1s,2s,3s,4s\mapsto f$, and  $1h\mapsto f$, $1h,2h\mapsto t$, \ldots}).
	\end{itemize}
	Strategy \strats achieves \goalList only up to $\varphi_c$, \ie collision-freedom, and \strats is a dominant ($\props$-truth-observing) strategy, since in all cases collision-freedom is guaranteed and
	in case the car is slow, no other strategy can do better, \ie realize both, collision-freedom and the turn.

	Let us now consider a doxastic strategy \stratb.
	\begin{itemize}
		\item We consider the same goal list and world model as for \strats.
		\item We take the konwledge-consistent belief formation \LabelB as sketched in \autoref{fig:bellab} on page \pageref{fig:bellab}, \ie \ego always believes in its sensor readings and its sensor initially switches colours.
			
			Its set of observables is $\Obs:=\Props_{x_e}\cup\Props_{y_e}\cup\{\textsf{undef, \bp, \rp}\}$ (cf. \autoref{ex:hist}, p.~\pageref{ex:hist}). The knowledge base is defined on p.~\pageref{ex:kb},~\autoref{ex:kb}, as  $\{$\fs
			$\varphi_z$ (\other is at most at $x=4$),
					{{$\varphi_t$}} (a turn is only possible at $x=2$),
					$\varphi_i$ (\ego starts at $x=1$), 
					$\varphi_{ct}$ (the initial car type does not change)\}\normalsize.

				\item Let \stratb be a doxastic strategy with \Small{$B_{0,1}\mapsto f$, $B_{0,1}\,B_{1,1}\mapsto f$, $\ldots$} and \Small{$B_{0,2}\mapsto f$, $B_{0,2}\,B_{1,2}\mapsto t$, $\ldots$}. \stratb is illustrated in \autoref{fig:belstrat}.				
	\end{itemize}
	% Figure environment removed

	Just like \strats,  \stratb chooses to turn when \other is hasty (\Small{$B_{0,2}\,B_{1,2}\mapsto t$}), and it chooses to drive straight on, if \other is slow (\Small{$B_{0,1}\,B_{1,1}\mapsto f$}). 
	As there is \enquote{no better} strategy, \stratb is dominant. 
\end{mexample}
%

Next, we want to capture that \ego chooses its actions based on the \emph{content} of its beliefs. In order to motivate our formalization, let us consider the following example, where \ego does not choose its actions based on its belief content.
\begin{mexample}{Decisions Not Based on the Belief Content}\label{ex:noCon}
	We modify our running example slightly: Let us assume the colour perception is severely broken and permanently switches red to blue and vice versa.	
	In \autoref{fig:bellab2} the changed world model is given along with a belief formation that relies on the colour perception, i.e., if the sensors say the other car is red (blue), then \ego believes the other car is red (blue).
	% Figure environment removed

	Let $\stratb'$ be a doxastic strategy with \Small{$B_{0,1}\mapsto f$, $B_{0,1},B_{1,2}\mapsto f$, $\ldots$} and \Small{$B_{0,2}\mapsto f$, $B_{0,2},B_{1,1}\mapsto t$, $\ldots$}. 
	Just as \strats and \stratb,  the strategy $\stratb'$ realizes a turn on \universeD, if \other is hasty (\Small{$B_{0,2},B_{1,1}\mapsto t$}), and \ego drives straight on, if \other is slow (\Small{$B_{0,1},B_{1,2}\mapsto f$}). 
	So $\stratb'$ is a dominant strategy, but $\stratb'$ makes no sense from \ego's perspective. 
	In case \other is hasty, \ego believes that \other is slow, since it trusts its sensors and \ego extrapolates that doing the turn  would cause a collision. 
	But in this case, $\stratb'$ demands to take the turn.
	Vice versa,  $\stratb'$ chooses to drive straight on, when \ego believes \other is hasty and it extrapolates that taking the turn is alright. 
\end{mexample}
%
%%%%%%%%%%%%%%%%%%%%%%%%%%%%%%%%%%%%%%%%%%%%%%%%%%%%%%%%%%%%%%%%%%%%%%%%%%%%%%%%
%%%%%%%%%%%%%%%%%%%%%%%%%%%%%%%%%%%%%%%%%%%%%%%%%%%%%%%%	%
\subsection{Possible-worlds Strategy}
%%%%%%%%%%%%%%%%%%%%%%%%%%%%%%%%%%%%%%%%%%%%%%%%%%%%%%%%	%
%%%%%%%%%%%%%%%%%%%%%%%%%%%%%%%%%%%%%%%%%%%%%%%%%%%%%%%%%%%%%%%%%%%%%%%%%%%%%%%%
\autoref{ex:noCon} motivates, what it means that \ego decides based on the content of its belief. 
We will formalize this as \enquote{\Ego always chooses an action, that a dominant strategy in \ego's current belief \belief would also choose at the believed current state}.
%
To capture this formally, we introduce the notion of \emph{possible-worlds strategy}. 

A \emph{possible-worlds strategy}  is a function $\stratw:{(2^{\Props_\belief})^{+}}\rightarrow\Act_\ego$ and it is applied to the alternative realities of \ego's current belief \belief. 
This results in believed traces.
We define this set of traces in an alternative reality $\reality=(\world,\cStates)\in\belief$ for a (believed) sequence of environment actions $\e\in\Act_{\env}^{\omega}(\world)$ as 
$\Cmp(\e,\stratw,\reality) = \{\cmp_0 \cmp_1\ldots \in (2^{\Props})^\omega | \exists \textit{ path } \Path\textit{ in }\world\textit{ from }\Init:  \forall i\geq 0:
\cmp_i = \Labelw(\Path(i)) \land 
\act_i:=\stratw(\Labelw(\Path_{\leq i})) \land
(\act_i,e(i))\in\Labelw({\Path}_{i},\Path_{i+1}) \}$.
We generalize the notion of \goalList-dominance to possible-worlds strategies. 
A possible-worlds strategy \stratw \goalList-dominates a possible-worlds strategy $\stratw'$ in \belief, if \stratw \goalList-dominates $\stratw'$ in all realities $\reality\in\belief$.
%
\begin{mexample}{Possible-Worlds Strategy}\label{ex:pws}
	Consider the possible-worlds strategy \stratw  that chooses to turn, if \other is hasty, and to drive straight on, if \other is slow, i.e., we consider \stratw with  \Small{$114hr\mapsto f$, $114hr, 212hr\mapsto t$,  $114hr, 212hr,221hr\mapsto f$, \ldots, $114sb\mapsto f$, $114sb,213sbb_p\mapsto f$, \ldots}. \stratw is sketched for the case that \other is hasty via bold arcs in \autoref{fig:bels}. 
	Note that \autoref{fig:bels} shows the excerpts of $B_{0,1}$ and $B_{1,2}$ as in \eg \autoref{fig:bellab2}.  
	$B_{0,1}$  expresses that \ego thinks that it is at the initial state \Small{$114hr$}. 
	\Ego follows \stratw by choosing \Small{$\stratb(114hr)=f$} when having this belief. 
	% Figure environment removed
	The belief $B_{1,2}$ (cf. \autoref{fig:bels} b) captures that \ego thinks to have already made one move and is now at state \Small{$212hr$}. According to \stratw, \ego has to choose $t$, since \Small{$\stratw(114hr,212hr)=t$}. \Ego hence has to choose $t$ when currently having the belief $B_{1,2}$.
\end{mexample}


\begin{table}
	\FramedBox{6.8cm}{\textwidth}{
		\small
		Strategy tyes:\\[\nlred]
		\begin{itemize}
			\item\small \emph{truth-observing strategy} $\strats:(2^{\props})^+\rightarrow\Act_{\ego}$\\ 
				observes the  ground truth world $\universeD$ via $\props\subseteq\Props_{\universeD}$ and takes decisions based on their history; serves as comparative reference of what is achievable given $\props$ could be observed directly\\[\nlred]
			\item\small \emph{doxastic strategy} $\stratb:\Beliefs^{+}\rightarrow\Act_{\ego}$\\
				observes the beliefs to take decisions and takes decisions based on their history; represents the decision making of autonomous and automatic systems\\[\nlred]
			\item\small \emph{possible-worlds strategy} $\stratw:{(2^{\Props_\belief})^{+}}\rightarrow\Act_\ego$\\
				captures how a system \HAS \enquote{simulates} its strategies within the alternative realities; decisions are taken based on the believed history within the respective alternative reality\\[\nlred]
		\end{itemize}
		A strategy \strat  \emph{\goalList-dominates} $\strat'$, $\strat'\leq\strat$, iff $\strat'$ achieves the goal list \goalList up to priority $m'$ but \strat achieves \goalList up to priority $m'$ with  $m'\leq m$.\\
		\normalsize
		}
		\caption{Strategy types \& dominance in a nutshell}\label{tab:nutshell}
\end{table}

%

A dominant possible-worlds strategy determines what is the best to do, given a  belief. 
So in order to express that \HAS chooses the action, that it thinks is currently the best, we refer to what a dominant possible-worlds strategy would choose for a given belief.

A peculiarity of possible-worlds strategies is, that they can be \emph{indecisive} for a belief \belief. 
That is, a possible-worlds strategy might determine two or more different actions for the set of believed current states.
More precisely,  \stratw is called current-state indecisive, if there are two paths, $\Path_1, \Path_2$, in \belief leading to believed current states and if \stratw chooses the action $\act_1$ at $\Path_1$ while it chooses $\act_2$ at $\Path_2$:  
%
\begin{mdefinition}{current-state (in)decisive}
	We call a possible-worlds strategy \stratw \emph{current-state indecisive} in belief \belief iff $\exists\reality_1,\reality_2\in\belief\land \bigwedge_{i\in\{1,2\}}\exists_i\Path_i\in\Paths(\reality_i):\bigwedge_{i\in\{1,2\}}\last(\Path_i)\in\cStates(\reality_{i})\land\stratw(\Label_{\reality_1}(\Path_1))\not=\stratw(\Label_{\reality_2}(\Path_2))$. 


	\stratw is \emph{current-state decisive} in \belief iff it is not current-state indecisive in \belief.
\end{mdefinition}
%

The indecisiveness may result from uncertainties of \ego. 
\Ego might be missing information that would allow it to determine the current situation sufficiently.
Since this information is missing, \ego instead forms a belief with a multitude of realities. 
That way a belief can encode even contradictory information. 
%
\begin{mexample}{Lack of information and indecisiveness}\label{ex:unknownGoal}
Let us assume that \ego has to get to a filling station on the shortest possible route. 
It is currently not sure where the filling station is.
It hence forms a belief $\belief$ of two realities, $\reality_1$ and $\reality_2$.
In reality $\reality_1$ the filling station is to its left, while in $\reality_2$ the filling station is to its right. 
In $\reality_1$ \ego must to move left while it must move right in $\reality_2$. 
Since \ego deems both realities possible, it cannot decide whether to turn right or left. 
\end{mexample}
%
\begin{mexample}{Indecisive possible-worlds strategy}\label{ex:undec}
Another example where a possible-world strategy is not able to determine a unique best choice, is given by $B_2$, the belief depicted in \autoref{fig:undec}. 
	% Figure environment removed	
	$B_2$ may be formed because \ego's sensor does not give any information about \other's colour, so that \ego believes that both colors are possible. 
	$B_2$ moreover captures that \ego believes to have made one step, i.e., it believes to be in state $s_2$ of reality $r_1$ or in state $u_2$ of reality $r_2$. 
	The strategy \stratw determines $f$ as the best option due to $s_2$ and $t$ as the best option due to $u_2$ of $r_2$. 
	Hence it is not obvious, whether to choose at the current state $t$ or $f$, given $B_2$.
\end{mexample}
We call the set of actions that a possible-worlds strategy chooses at the set of current states, its current-state choices, $\cAct(\stratw,\belief)$:
%
\begin{mdefinition}{current-state choices of \stratw, \Small{$\cAct(\stratw,\belief)$} }\label{def:cdec}
	Let \Small{$\cAct(\stratw,\belief)$} be the set with

	
	$\act \in \cAct(\stratw,\belief)\Leftrightarrow \exists \reality=(\world,\cStates)\in\belief: \exists \textit{ path }\pi\textit{ in }\world: \pi(0)\in\Init\land\textit{last}(\pi)\in\cStates\land\stratw(\pi)=\act$.
	
	We call \Small{$\cAct(\stratw,\belief)$} the current-state choices of \stratw. 
\end{mdefinition}
%
Given a strategy is decisive at the set of current states, we call it current-state decisive:
\begin{mdefinition}{current-state decisive possible-worlds strategy}\label{def:cdec}
	A possible-worlds strategy \stratw is \emph{current-state decisive in a belief \belief}, if \Small{$\cAct(\stratw,\belief)$} is a singleton.
\end{mdefinition}
Note, that the examples \autoref{ex:unknownGoal} and \autoref{ex:undec} illustrate that the existence of a current-state decisive possible-worlds strategy is not guaranteed.
\begin{mproposition}{Existence of a Current-state Decisive Strategy}\label{prop:ex-csds}
	We can decide whether there is a current-state decisive possible-worlds strategy \stratw achieving an LTL property $\psi$ in a given belief \belief.
	If it exists, we can synthesize such a strategy.
\end{mproposition}
\begin{proofsketch}{Prop.~\ref{prop:ex-csds}}
	We sketch how a current-state decisive possible-worlds strategy for a belief \belief can be synthesized.
	Given a belief \Small{$\belief=\{\reality_1,\ldots,\reality_n\}\in\Beliefs$}, we first build a single reality $\reality_{\belief}$ by the disjoint union of all alternative realities \Small{$\reality_{\belief}:=(($
	$\dbigcup_{r_i}\{\States_i\}$, 
	$\dbigcup_{r_i}\Edges_i$, 
	$\dbigcup_{r_i}\Label_i$,  
	$\dbigcup_{r_i}\Init_i)$, 
	$\dbigcup_{r_i}\{{\cStates}_i\})$}.
	If necessary, we can make the realities disjoint by renaming their states but keeping their structure.
	

 	We iterate through the list of \ego's actions.
	In iteration $i$, we modify $\reality_{\belief}$ to create the $\reality_i$ where the current action $\act_i$ becomes the only possible choice at all current states $\state_c$. 
	More precisely, in $\reality_i$ all transition that orginate at a current state $\state_c$ and that are labelled with an action $\act\not=\act_i$ are (re)directed to lead to $\sundef$.
	Using \cite{LTLSynth}, we synthesize a winning strategy \strat for $\psi \land \Box \neg \sundef$\footnote{The complexity is 2EXPTIME-complete}. If it exists, we stop iterating. 
	The synthesized winning strategy \strat applyies the same action at all current states, by construction. 
	It obiviously is also a winning strategy of \belief and current-state decisive.
	Since we check for all actions, whether such a strategy \strat exists, the algorithm is guaranteed to find a current-state decisive possible-worlds strategy, if it exists. Since there are only finitely many actions, the algoirithm terminates.
\end{proofsketch}
We consider actions that are the current-state choices of a dominant possible-worlds strategy as  rationally justified choices of an autonomous system. We therefore define the set of current-state choices of a belief:
%
\begin{mdefinition}{best choices in \belief, \Small{$\bAct(\belief)$} }\label{def:choiceb}
	Let a goal list $\goalList$ be given. Let \Small{$\bAct(\belief)$} be the set with
	
	$\act\in\Small{\bAct(\belief)}\Leftrightarrow \exists\tit{$\goalList$-dominant }\stratw\tit{ in }\belief:\act\in\Small{\cAct(\stratw,\belief)}$. 
\\
	We call \Small{$\bAct(\belief)$} the current-state choices in \belief for $\psi$. 
\end{mdefinition}
%  
The following example illustrates that in a belief \belief there can be several dominant current-state decisive possible-worlds strategies.
%
\begin{mexample}{current-state decisive and multiple action choices}\label{ex:undecMulti}
	Let us assume as in \autoref{ex:unknownGoal} that \ego has to get to a filling station on the shortest possible route. 
Let us assume \ego forms a belief $\belief$ of only one reality, $\reality$.
In reality $\reality$ there is a filling station to its left and to its right. 
	Hence \ego can turn right --let this be strategy $\stratw_1$-- and it can turn left --strategy $\stratw_2$. 
So \ego can choose to turn right or left according to  $\stratw_1$ and $\stratw_2$, respectively, and both strategies are current-state decisive.
\end{mexample}
%
\begin{mproposition}{Determining the best choices \Small{$\bAct(\belief)$}}\label{prop:actBel}
	Let a goal list $\goalList$ and a belief $\belief$ be given. 
	
	We can determine the set of best choices \Small{$\bAct(\belief)$} for $\goalList$ in \belief.
\end{mproposition}
\begin{proofsketch}{Prop.~\ref{prop:actBel}}	
	According to Def.~\ref{def:choiceb} \Small{$\bAct(\belief)$}  are the actions that are current-state choices of some $\psi$-dominant startegy $\stratw$ in \belief.
	We first determine the maximal $\mn$ of the goal list $\goalList$ that is achievable by any  possible-worlds strategy in \belief. With other words, a possible worlds strategy is dominant, if it achieves \goalList up to \mn.  
	To this end we check whether we can synthesize, \cite{LTLSynth}, a strategy in $\reality_{\belief}$\footnote{$\reality_{\belief}$ is the disjoint union of all alternative realities as defined in the proof of Prop.~\ref{prop:ex-csds} on page \pageref{prop:ex-csds}} that achieves $\psi$ for priority \mn, starting with $\mn=|\goalList|$ down to $\mn=0$.
%
	We then proceed as in the proof of Prop.~\ref{prop:ex-csds} on page \pageref{prop:ex-csds}, \ie by examining whether there is a possible-worlds strategy that achieves \goalList up to $\mn$ in the modified reality $\reality_i$ for action $i$, but we do not stop as soon as one could be synthesized but instead we examine all actions \Act.
\end{proofsketch}
%%%%%%%%%%%%%%%%%%%%%%%%%%%%%%%%%%%%%%%%%%%%%%%%%%%%%%%%	%
%%%%%%%%%%%%%%%%%%%%%%%%%%%%%%%%%%%%%%%%%%%%%%%%%%%%%%%%	%
\subsection{Autonomous Decision}
%%%%%%%%%%%%%%%%%%%%%%%%%%%%%%%%%%%%%%%%%%%%%%%%%%%%%%%%	%
%%%%%%%%%%%%%%%%%%%%%%%%%%%%%%%%%%%%%%%%%%%%%%%%%%%%%%%%	%
In this section we develop our notion of an autonomous decision and we define when systems are autonomous-decisive.\\ %

\emph{For the following let a doxastic model $\epi=(\universeD,\goalList, \LabelK,\Obs, \Beliefs, \LabelB)$ of a system \HAS be given.
	Let \Path be a finite initial path in \universeD, \history the observed history along $\Path$ and $\belief=\LabelB(\history)$ the formed belief.}\\



We call a system that decides based on its beliefs a \emph{doxastic system}.
Its decisions are determined by a doxastic strategy $\stratb$. 
\begin{mdefinition}{doxastic system}\label{def:doxSys}
	A \emph{doxastic system} $\sys$ is a pair $\sys=(\epi,\stratb)$ of a doxastic model $\epi$ and a doxastic strategy $\stratb$, $\stratb:\Beliefs^{+}\rightarrow\Act_{\ego}$ on $\universeD$.  
	For all finite paths \Path in \universeD, the system chooses $\stratb(\LabelB(\Path))$.
\end{mdefinition}
Doxastic systems do not base their decisions on the ground truth or on observations, but on their beliefs.
It is not constrained how they come to a decision though. 
Hence doxastic systems can be \eg automatic or autonomous systems (\cf \autoref{ex:autonAutom}). 

We regard autonomous systems as special doxastic systems, whose decisions are rational \wrt the content of the current belief.
So, $\stratb$ should choose actions that are the current-state choices of a dominant possible-worlds strategy \stratw.
Moreover, we require that \stratw should be current-state decisive.
When no current-state decisive strategy exists, this means that there is  no way to rationally avoid an unwanted consequence.
Then the current-state choice set \cAct(\stratw,\belief) means a gamble: $\act_1\in\cAct(\stratw,\belief)$ might achieve the targeted goal or another action $\act_2\in\cAct(\stratw,\belief), \act_1\not=\act_2,$ would be the right choice.


We hence regard it a design goal to develop systems that form beliefs where a current-state decisive possible-worlds strategy exists -- with other words, we strive to build a system \HAS that always builds a belief where it can determine a choice achieving its goals.
If a belief \belief is formed where no current-state decisive possible-worlds strategy exists, an engineer can adjusting the system's goals (\eg by weakening the goals to \enquote{if you are uncertain, choose the safe option}) or by improving the formed beliefs -- adding additional knowledge or adding sensors/observables.\\
\assumption{csdec}{In the following we assume that the belief formation \LabelB forms only beliefs \belief in which a current-state decisive strategy exists.}\\





A system that is not autonomous-decisive cannot rationally determine which action is currently appropriate. A goal for the  design of \HAS is hence to ensure that a system is autonomous-decisive. 


To summarize, we call a decision autonomous, if it is the rational choice for the current belief, that is 
 $\stratb$ chooses actions that are the current-state choices of a dominant, current-state decisive possible-worlds strategy:
%
\begin{mdefinition}{autonomous decision}\label{def:autonomDec}
	The system \HAS decides autonomously at \Path,  if it chooses an action $\act\in\bAct(\belief)$.
\end{mdefinition}
%
A system that always decides autonomously, follows a special doxastic strategy $\stratbb$ that always chooses an action $\act\in\bAct(\belief)$ when on a path \Path in \universeD, where $\belief=\LabelB(\Label_{\States}({\Path})|_{ \Obs},\LabelK(\last(\Path)))$ is the belief formed after the observed history along \Path and while having the believed current knowledge $\LabelK(\last(\Path))$.
Note that such a system follows a memoryless doxastic strategy.
The system's memory is \enquote{shifted} into the beliefs. 
The framework thus can capture how a system \HAS deals with the  finite memory also \wrt encoding the relevant.
%
\begin{mdefinition}{autonomous strategy}\label{def:autonomDec}
	A doxastic strategy $\stratbb : \Beliefs^{+} \rightarrow \Act$ is called an \emph{autonomous strategy}  iff 
		for all belief histories $\bar\belief\in\Beliefs^+$ it holds that {$\stratbb(\bar\belief)\in\bAct(\last(\bar\belief))$}.
\end{mdefinition}
%


We say that a system \sys autonomous-decisively achieves the goal list $\goalList$ up to $n$, if it implements an autonomous strategy $\stratbb$, \ie $\sys=(\epi,\stratbb)$, and \stratbb achieves $\psi$ up to $n$.

So far we do not require that an autonomous-decisive system \sys behaves appropriately in a given setting.
It is only guaranteed, that \sys acts rationally \wrt its beliefs. 
Its belief formation does not have to reflect the real world though.
Def. \ref{def:optimal} closes the gap.


By Def.~\ref{def:optimal} we basically enforce the belief formation \LabelB to form beliefs, so that \sys is as successful as the best system with direct access to the ground-truth of the design-time world \universeD.
\begin{mdefinition}{Optimal autonomous-decisive system}\label{def:optimal}
	The autonomous system $\sys=(\universeD,\goalList, \LabelK,\Obs, \Beliefs, \LabelB,\stratbb)$ is an \emph{optimal} autonomous-decisive system, 
	if the autonomous strategy $\stratbb$ is not $\goalList$-dominated by any \Props-truth-observing strategy.
\end{mdefinition}
%
In the following we are focusing on \emph{optimal} autonomous-decisive systems. 
For brevity we usually just speak of \emph{autonomous systems}.
We will discuss the relation of our notion of optimal autonomous-decisive systems with the notion of autonomous systems in \autoref{sec:autoSys}.


Def.~\ref{def:optimal} requires that the belief formation captures the gist of observations \wrt \sys's goals. 
It is a rather flexible way of constraining the belief formation: \LabelB has to preserve the \emph{relevant aspects} of \universeD. 
A more direct way to anchor beliefs in the ground-truth is given by the knowledge base. 

We say that $\sys$ is an \emph{optimal} autonomous-decisive system, 
	if the autonomous strategy $\stratbb$ is not $\goalList$-dominated by any \Props-truth-observing strategy\footnote{an \Props-truth-observing strategy is based on perfect observations of \universeD, cf. Tab.~\ref{tab:nutshell}}.
$\sys$'s belief formation \LabelB then builds beliefs, such that \sys is as successful as the best system with direct access to the complete ground truth, \universeD.
%

We can decide for a given doxastic model without belief formation, $\epi^{-}=(\universeD,\goalList,\LabelK,\Obs,\Beliefs,.)$, whether there is a knowledge-consistent belief formation \LabelB and
an autonomous strategy \stratbb, and we can synthesize the two (cf.~\autoref{th:auto}).  
%
\begin{mtheorem}[Autonomous Decisiveness~\cite{doxFrame}]\label{th:auto}
	Let  $\epi^{-}=(\universeD,\goalList,\LabelK,\Obs,\Beliefs,.)$ be a doxastic model without belief formation.

	We can decide whether there is a knowledge-consistent belief formation \LabelB and a  doxastic strategy \stratb such that $\sys=(\epi^{-},\LabelB,\stratb)$ is an optimal autonomous-decisive system. 
	If such \LabelB and \stratb exist, we can synthesize them. 
\end{mtheorem}
\begin{proofsketch}{\autoref{th:auto}}
	The proof can be sketched as follows. 
	We build a Kripke structure $\universeD'$ such that any \Obs-truth-observing strategy \strats in $\universeD'$ encodes (i) a belief formation $\LabelB$ and (ii) an autonomous strategy \stratbb, such that (a) \LabelB is knowledge-consistent and (b) if \strats achieves \goalList up to $n$, also $\stratbb$ does. 
	%
	The idea for the construction of $\universeD'$ is as follows. In $\universeD'$ the strategy \strats does not choose actions but beliefs.	
	%When a belief \belief is chosen in state \state, then all actions of the respective current-state choice $\Act(\belief)$ represent a rational and possible choice of \stratbb. 
	Therefore,  the transitions in \universeD are copied to $\universeD'$ and get relabelled with the belief \belief that justifies the action of \ego as a rational choice.
	We sketch the major steps of the construction as (i)-(iv) below:
	(i) We determine the current-state choices $\Act(\belief)$ for all beliefs in $\Beliefs$ by Prop.~\ref{prop:actBel}.
%
	(ii) We build the modified Kripke structure $\universeD'$: 
	Therefore we copy the state set \States of \universeD to become the state set $\States'$ of $\universeD'$.
	We then iterate over all states $\state\in\States$ of $\universeD$.
	If there is a transition from $\state$ via action $\act=(\act_1,\act_2)$ to $\state_2$ but no knowledge-consistent belief justifies \ego's action $\act_1$, \ie $\emptyset=\Beliefs_{\act,\state}:=\{\belief\in\Beliefs\mid \act_1\in{\Act(\belief)}$ and $\belief\models\LabelK(\state)\}$, we add a transition from \state to state \sundef and label this transition with $\act=(\perp,\act_2)$ to express that \ego will not choose this action, since it is no rational choice. %
	If $\Beliefs_{\act,\state}\not=\emptyset$, we iterate over all beliefs $\belief\in\Beliefs_{\act,\state}$ and introduce a transition from $\state'$ via $(\belief,\act_2)$ to $\state'_2$ in $\universeD'$, \ie we replace $\act_1$ by $\belief$.
	
	(iii) In order to judge how well the doxastic strategy \stratb has to perform for an autonomous-decisive system, we determine the maximal $\mn$ up to which $\goalList$ can be achieved by any \Props-truth-observing strategy in \universeD by iteratively applying strategy synthesis for LTL properties \cite{LTLSynth} starting from the maximum priority goals. 
%
	(iv) We then synthesize an \Obs-truth-observing strategy $\strats$ on {$\universeD'$}~\cite{LTLSynth} for the goal list $\goalList$ and priority $\mn$.
	In case $\strats$ achieves $\mn$, we define \LabelB by $\LabelB(\history):=\strats(\history)|_{\Beliefs}$, \ie the ${\LabelB}(\history)$ chooses the belief that labels the chosen transition.
	\stratbb may choose any action that is justified by $\LabelB(\history)$.
	Then $\sys=(\epi,\LabelB,\stratbb)$ is an optimal autonomous-decisive system. 
	If $\strats$ cannot achieve $\mn$, the truth-observing strategies on \universeD perform better, so no knowledge-consistent belief-formation for an autonomous optimal strategy exist for $\epi^{-}$.
\end{proofsketch}


%

To summarize, according to Theorem \autoref{th:auto} when designing an autonomous system \HAS, we can specify
\begin{itemize}
	\item the application domain via $\universeD$, 
	\item the list of goals \Goals, 
	\item the believed knowledge that the system \HAS will have, 
	\item what observations \HAS  can make and 
	\item how its internal representation the world is, \ie the possible worlds,
\end{itemize}
	and then we can determine whether it is at all possible to form beliefs such that the system \HAS is able to autonomously-decide and succeed as if it knew the ground truth. 
	Moreover, we can synthesize an appropriate belief labelling, so that the corresponding autonomous strategy is optimal for its goals. 

	Since we consider a quite liberal notion of autonomous system, it means that if the above check fails, it if often not possible to build an autonomous system with the given input and resources.  

%
Given we provide our system under construction full observability and let its beliefs reflect \universeD precisely and do not provide false believe knowledge, then an autonomous system \sys is guaranteed to exist.





The following example illustrates that the beliefs of an autonomous-decisive system \sys can be rather loosely linked to reality, observations are not (directly) represented in the beliefs and the possible worlds differ substantially from the ground-truth. 
Nevertheless, \sys can be successful. 

\begin{mexample}{Freedom of beliefs}
	In \autoref{fig:good} we sketch a belief formation where \Ego believes all the time, that \other has the wrong colour.
	The possible worlds do not reflect the ground truth world, \universeD, well, \eg in the possible worlds of $B_{0,1}$ and $B_{0,2}$ \other is red and the dominant strategy is not to turn, while in \universeD the dominant strategy is to do the turn, when \other is red. 
	Furthermore, the belief formation does rather abruptly update its beliefs (especially from $B_{0,2}$ to $B_{1,2}$).
	% Figure environment removed
	Although the formed beliefs may seem degenerate and inappropriately capturing the reality, the belief formation allows \ego to behave as good as when it would know the ground truth. 
	This freedom of belief formation allows efficient and compact encodings of the perceptions comprised to the relevant aspects. 
	A system \sys that is optimal according to Def.~\ref{def:optimal} will have a belief formation that captures the application domain \universeD as closely as necessary to satisfy the system's goals.   

	If \universeD has to be captured even closer than necessary for \sys's goals, this can be enforced by means of the knowledge base.
\end{mexample}


\begin{mexample}{Autonomous, Non-Autonomous,  Automatic}\label{ex:autonAutom2}
	Let us consider an example of an autonomous \ego in the  setting of \autoref{fig:bellab}, where the sensor only initially switches colours, and consider the possible-worlds strategy \stratw of \autoref{ex:pws} (turn, if \other is hasty, and drive on, if \other is slow). 
	%
	When \ego initially evaluates its situation in $s_1$ of \universeD, it believes that the situation is as described by {$B_{0,1}$}, i.e. \other is a hasty, red car.
	\Ego can decide to follow \stratw in {$B_{0,1}$}, as it seems a good choice -- \stratw is dominant and current-state decisive in {$B_{0,1}$}. 
	According to its extrapolation, it would move one step forward, and then it would successfully take the turn.
	%
	After actually moving forward, \ego evaluates the situation in $s_2$.
	In $s_2$ \ego believes in {$B_{1,1}$} reflecting that \ego now truthfully perceives \other's colour as blue (cf. \autoref{ex:knowledgeBelief}).  
	Again, \stratw is a dominant current-state decisive possible-worlds strategy and determines \textsf{f} as the next move.
	Along this line, it is easy to see that \ego can implement a doxastic strategy \stratb that chooses the action that \stratw determines for the respective $\LabelB_{\tsf{auton}}(h)$.


	For an example for where no autonomous \ego exists but an automatic \ego can be built, we modify our running example slightly.
	Let us assume that \ego is unsure of its own initial position, thinking that it may initially be at \Small{$x=1$} or \Small{$x=2$}, as sketched in \autoref{fig:bellab3}. 
	Let a belief formation $\LabelB_{\tsf{autom}}$ be given that evolves the initial beliefs {$B_{0,3}$} and {$B_{0,4}$} analogously to \autoref{ex:knowledgeBelief}, that is, \ego perceives the correct colour after moving one step forward.
	The possible-worlds strategy \stratw is still a dominant strategy but not current-state decisive, since for example in {$B_{0,3}$} \ego would do the turn at $s1$ due to reality $r_2$, and it would also drive straight on due to $r_1$. Hence, there is no dominant possible-worlds strategy in {$B_{0,3}$} that is able to determine one action. \Ego cannot decide autonomously.
	Nevertheless, we can specify a dominant doxastic strategy \stratb for this case, but its actions are not chosen based on the belief content: \Ego chooses to turn after one step when its initial belief was {$B_{0,4}$} (\Small{$B_{0,4},B_{1,4}\mapsto t$}), otherwise it drives straight on.
	This strategy is dominant and could be used to build an \emph{automatic} system, where \ego just plays out \stratb. Such a strategy might be useful when an engineer knows that \ego will start from \Small{$x=1$} but did not equip \ego with this information.
\end{mexample}

A system that is not autonomous-decisive cannot rationally determine by itself which action is currently appropriate. A goal for the  design of a system \HAS is hence to ensure that a system is autonomous-decisive. 
%\subsection{The Notion of Autonomous System}
\subsection{The Notion of Autonomous System}\label{sec:autoSys}
Above we have introduced our notion of \emph{autonomous-decisive system} as a system that takes rational decisions based on its current believes.
In Def.~\ref{def:optimal} we defined an \emph{optimal autonomous-decisive system}, as an autonomous-decisive system that performs at least as well as if it knew the ground-truth.
We do not intend these terms to characterize general autonomous systems nor do we aim  to capture aspects of free-will, independence or the ability of reflection; rather, we focus on the decision making of autonomous systems.
In the following we briefly discuss the notion of autonomous system and then relate our notions to it.

\paragraph{Autonomous Systems} The literal meaning of \emph{autonomy} is derived from \emph{auto = self} and \emph{nomos = law}.
Autonomy thus means self-governance \cite{merriamAutonomous} and the concept of autonomy can be found in different kinds of sciences~\cite{LitAutom2016}. 
For systems engineering the word autonomy describes the ability of a system to make its own decisions about its actions without the need for the
involvement of an outside supervisor~\cite{AutomStage}.


Although the terms \emph{automation} and \emph{autonomy} are sometimes used interchangeably~\cite{LitAutom2016},  
a significant difference
between the term autonomous and automatic is that an automatic system will do exactly as programmed while an autonomous system can make choices \cite{huAuto}. 
  
Several level of automation (LoA) have been suggested and discussed in literature.
Many of these see autonomy as the ultimate level of automation.~\cite{LitAutom2016} 
For instance, Parasuraman, Sherdian and Wickens list  in \cite{interactionLevel2000} ten levels of automation. 
Their levels target four broad classes of functions: information acquisition, information analysis, decision \& action selection and action implementation.
At the lowest level, humans must make all decisions and control all actions; at higher levels of automation, the automatic system increasingly takes over while humans receive less and less information about its operations. At level 10 the system decides everything and acts autonomously.\footnote{
In contrast, SAE J3016 defines a taxonomy for six levels of driving automation the SAE Levels of Driving Automation$^{\textsf{TM}}$ avoiding the term autonomy. 
They range from Level 0 (no driving automation) to Level 5 (full driving automation) in the context of motor vehicles and their operation on roadways \cite{SAE}.}

According to J. Sifakis, \cite{AutoSysSifakis}, the main characteristic of autonomous systems is their
ability to handle knowledge and adaptively respond to environment changes.
Autonomous systems have to operate for extended periods of time under
significant uncertainties in the environment and they have to compensate a certain amount of system failures, both without external intervention \cite{TwIntAutCS1989}. 
Many agree with ~\cite{AutoSysSifakis} that autonomy combines perception, reflection, goal management, planning and self-adaptation~\cite{AutoSysSifakis}.
Often autonomous systems are discussed with a focus on artificial intelligence and learning \cite{industAuto,AutoSysSifakis}. 


\paragraph{Autonomous-decisive Systems} In what follows, we argue that our notion of optimal autonomous-decisive system fits well with the notion of autonomous system as outlined above.

The key asset of our notion is formalizing an \enquote{epistemic goal-directedness} for autonomous systems.
Our notion is based  on the fact that a system perceives its environment and maintains a varying knowledge base.
We introduce the explicit requirement that the devised plans have to be rational with respect to its internal world view.  
Thereby we can also distinguish autonomous systems from automatic systems in a way that is compatible with \eg \cite{huAuto}.

What about reflection, goal management, planning and self-adaptation? 
Our work is primarily concerned with decision making. We see you framework as a first step towards formalizing and studying a couple of interesting properties of autonomous systems, as we sketch below.
 
 The formal framework does not constrain what kind of information the \HAS's beliefs encode nor 
what kind of actions an autonomous systems can perform or how it perceives feedback regarding its actions' effect. 
We hence believe that the framework allows to study systems that have explorational awareness, \ie that explore the environment gathering information as part of their strategy.
For instance a robot can explore unknown paths of maze by keeping track where it has been.

Similarly, we believe that reflection can be treated within this framework. 
To this end, the \HAS's believes have to encode believes on believes -- not only represent the believed factual world.
We imagine that finite believe hierarchies could be encoded similar to \cite{perea2012}.
While the treatment of explorative system seems within the framework seems to be straight-forward, modeling believe-hierarchies is considered as future work. 

The framework as such that does not have a notion of goals or is concerned with goal management. 
We believe that the framework could be extended by a notion of subgoals, in such a way that it is possible to analyze whether there is a strategy for subgoal selection.
Conceptually, goals of autonomous system seem to be a mean for breaking down a complex global goal to more easily treatable goals.
So instead of subgoal selection, we can examine whether there is a global strategy that depends only on a certain limited simulation and planning horizon.

Finally, we want to remark that we can consider the design time universe as a training set where certain known aspects of the world are captured. A deployed \HAS then has to be equipped with a belief formation that is able to deal with unexpected events. 
Studying formal robust properties of the belief formation seems an important and interesting endeavour, \eg \enquote{Given a belief formation, how much can the real world deviate from the design-time world?} or \enquote{How much timing tolerance does a certain belief formation have?}.  


\section{Relevance}\label{sec:relevance}
We consider the combination (\LabelK,\Obs,\Beliefs) of labeled knowledge, observations and possible beliefs as important set screws for an engineer to develop an optimal autonomous-decisive system.
 \Small{{{\RE}\,{\LabelK}}}, he can equip the system \HAS with prior knowledge and implement mechanisms to update \HAS's knowledge base during the mission, 
 \Small{{{\re}\,{\Obs}}}, he can provide more sensing capabilities and, 
 \Small{{{\re}\,{\Beliefs}}}, he can increase the resources for the internal representation of the world model.
In the following we denote (\LabelK,\Obs,\Beliefs) also as \kob.

To support an engineer, we characterise whether a tuple \kob is sufficient for a given setting.
The basic idea is: If a system is an \emph{optimal} autonomous system, then its formed beliefs conserve the relevant aspects of \universeD. 
Hence the \kob is sufficient if a relevance conserving belief formation exists. 
To answer whether \kob are relevant, we test whether it is possible to build an optimal autonomous system with less knowledge, observations or beliefs. 

For the following we consider a doxastic model $\epi$ to be given with $(\universeD,\goalList,\LabelK,\Obs,$ $\Beliefs,\LabelB)$ with a knowledge-consistent belief formation \LabelB and a system $\sys=(\epi,\stratb)$ with a doxastic strategy.\\


%%%%%%%%%%%%%%%%%%%%%%%%%%%%%%%%%%%%%%%%%%%%%%%%%%%%%%%%%%%%%%%%%%%%%%%%%%%%%%%%%%%%%%
%%%%%%%%%%%%%%%%%%%%%%%%%%%%%%%%%%%%%%%%%%%%%%%%%%%%%%%%%%%%%%%%%%%%%%%%%%%%%%%%%%%%%%
\subsection{Conservation of the Relevant}
%%%%%%%%%%%%%%%%%%%%%%%%%%%%%%%%%%%%%%%%%%%%%%%%%%%%%%%%%%%%%%%%%%%%%%%%%%%%%%%%%%%%%%
%%%%%%%%%%%%%%%%%%%%%%%%%%%%%%%%%%%%%%%%%%%%%%%%%%%%%%%%%%%%%%%%%%%%%%%%%%%%%%%%%%%%%%
We first define when the relevant is conserved. 
Therefore we compare \ego's (doxastic and autonomous-decisive) performance with the performance that \ego could have when it would access the ground-truth, \universeD. 


We first develop a notion of relevance conservation for doxastic systems in order to highlight that the requirements for autonomous systems are more demanding. \\

We say that the belief formation \LabelB conserves the relevant of \universeD, 
if \epi can perform based on its beliefs as successful as it could when directly and truthfully observing the ground-truth \universeD. %
%
\begin{mdefinition}{Relevance Conservation for Doxastic Systems}\label{def:wrelcon}
Let $\Obs_D\subseteq\Props$ be a set of propositions.
	The belief formation $\LabelB$ of a doxastic model $\epi=(\universeD,\goalList,\LabelK,\Obs,\Beliefs,\LabelB)$ \emph{conserves the relevant of a $\Obs$-observable \universeD}, if there exists a doxastic strategy \stratb  for \epi that is dominant \wrt all $\Props$-observing strategies $\strats$.
\end{mdefinition}
%
When $\LabelB$ is conserving the relevant of  completely observable design-time model \universeD. 
then \ego could --by implementing \stratb of Def.~\ref{def:wrelcon}-- perform as well as possible when the ground-truth \universeD would be completely observable. 
Def.~\ref{def:wrelcon} captures this aspect by comparing the performance of \ego that is observing \Obs with the performance on the ground-truth \universeD that is observable via $\Props$.  

But what does it mean that a belief formation $\LabelB$ conserves the relevant? Intuitively, it means that $\LabelB$ preserves \universeD in sufficient detail to map the history of beliefs \enquote{somehow} to the best action.
The choice of action does not have to be plausible \wrt the content of a system's beliefs though. 
It is up to the engineer to choose which strategy $\stratb$ will be implemented by the system.\\

The choice of action must be plausible \wrt the belief content though, when it comes to autonomously-decisive systems.
An autonomous-decisive system chooses at all times actions \act that are justified in the respective current belief \belief, \ie $\act\in\Small{\Act(\belief)}$ (cf. Def.~\ref{def:choiceb}).
We hence say that the belief formation conserves the relevant for autonomous-decisiveness, if at all times the \enquote{best actions} \wrt the belief content are chosen. 
%
\begin{mdefinition}{Relevance Conservation for Autonomous-Decisiveness}\label{def:relcon}
	Let $\Stratsbb$ be the set of autonomous strategies that exist for $\epi=(\universeD,\goalList,\LabelK,\Obs,\Beliefs,\LabelB)$.


	The belief formation $\LabelB$ of \epi \emph{conserves the relevant of a $\Obs$-observable design-time world \universeD for autonomous-decisiveness}, 
	if all $\stratbb\in\Stratsbb$ are dominant \wrt $\Props$-observing strategies $\strats$ on \universeD.
\end{mdefinition}
%
Note, that we assume (\cf Ass. \ref{ass:csdec}, p.~\pageref{ass:csdec}) that the belief formation \LabelB forms only beliefs \belief in which a dominant current-state decisive strategy exists. Hence  $\Stratsbb$ contains at least one strategy. 
 
 
 If \LabelB conserves the relevant for autonomous-decisiveness as defined in Def.~\ref{def:relcon}, then any of the autonomous strategies \stratbb of \epi observing \universeD via \Obs will perform  as successful as possible when directly accessing \universeD via $\Obs'$. 
 It may seem surprising, that Def.~\ref{def:relcon} refers to \emph{all} autonomous strategies $\stratbb\in\Stratsbb$. The reason is, that the final decision on the chosen action lies with the autonomous system. 
%

\begin{mexample}{Conservation of the Relevant}\label{ex:conserve}
	As an example of a belief formation that conserves the relevant for autonomous-decisiveness, we refer the reader back to  \autoref{ex:autonAutom2} on page \pageref{ex:autonAutom2}. 
There we sketched a setting where the sensors are initially broken but when the decision has to taken the sensors provide the relevant information. 
The resulting belief formation $\LabelB_{\textsf{auton}}$ allows a system to perform as well as when knowing the ground-truth, \ie not having initially disturbed sensor readings. 
	In  \autoref{ex:autonAutom} on page \pageref{ex:autonAutom} we saw an example of a belief formation that conserves the relevant for doxastic systems but not relevance for autonomous-decisiveness. 
	In the example, \ego's sensors are permanently switching colours and \ego has a knowledge base that forces it to believe that a read car is fast and a slow car is blue. 
	Consequently, \ego cannot autonomously determine what is best to do in \universeD. 
	But the belief-formation is such that an engineer can choose a strategy for \ego that deals with the color readings appropriately, \ie \enquote{switch them back}.
\end{mexample}

Conserving the relevant for autonomous-decisiveness is stronger than conserving the relevant for doxastic systems: 
%
%
\begin{proposition}[Relevance Conservation]\label{th:refine}
	\enspace\\[-2mm]\enspace
	\begin{enumerate}
		\item\label{it:autDox} If \LabelB conserves the relevant for autonomous-decisiveness, then \LabelB conserves the relevant for doxastic systems.
	
		\item\label{it:doxAut} If \LabelB conserves the relevant for doxastic systems, then \LabelB does not necessarily conserve the relevant for autonomous-decisiveness.
	\end{enumerate}
	\end{proposition} 
%
\begin{proofsketch}{Prop. \ref{th:refine}}
	Prop.~\ref{th:refine}(\ref{it:autDox}) follows directly from Def.~\ref{def:wrelcon} and Def.~\ref{def:relcon}. Prop.~\ref{th:refine}(\ref{it:doxAut}) follows from the example \ref{ex:conserve}.
\end{proofsketch}


The next proposition is concerned with systems, where the belief formation is captured via a set of rules. 
Such autonomous systems still play an important role especially in safety critical applications, although artificial intelligence systems, that intransparently build their beliefs, gain more and more importance.

Since the resources of a system \HAS are limited, we consider belief formation functions that can be represented by a finite number of regular expressions. 
%%%%%%%%%%%%%%%%%%%%%%%%%%%%%%%%%%%%%%%%%%%%%%%%%%%%%%%%%%%%
\begin{mdefinition}{Regular Belief Formation}
We say \LabelB is regular, if $\LabelB$ can be defined via a finite number $n$ of regular expressions $\regEx_i$, i.e., 
for all observable histories $\history\in\histories_{\Obs}$ of \universeD it holds, that there is an $i, 1\leq i\leq n$ such that $\LabelB(\history)=\belief_i$ iff $\history\models\regEx_i$.
\end{mdefinition}
%%%%%%%%%%%%%%%%%%%%%%%%%%%%%%%%%%%%%%%%%%%%%%%%%%%%%%%%%%%%%%

Given a doxastic model with a regular belief formation \LabelB, we can decide whether \LabelB conserves the relevant for autonomous-decisiveness:
%
\begin{proposition}[Conservation of Relevance]\label{th:check}
	Given a regular belief formation \LabelB, we can decide whether \LabelB conserves the relevant for autonomous-decisiveness.
\end{proposition}
%
\begin{mproof}{Prop. \ref{th:check}}
	We first determine the maximal priority $\mn$ up to which the goal list $\goalList$ can be achieved on \universeD by applying iteratively strategy synthesis for LTL properties~\cite{LTLSynth} starting with the maximum goal list and then iteratively decreasing $\mn$. 	
	\LabelB of $\epi$ conserves the relevant, if all autonomous strategies $\stratbb\in\Stratsbb$ achieve at least $\mn$ (\cf~Def.~\ref{def:relcon}).
	We then construct an automaton $\aut_{\Act()\times\universeD}$, in which the environment is unconstrained and \ego chooses its actions from $\Act(\LabelB(h))$ after observing history $h$.
	It holds that iff $\aut_{\Act()\times\universeD}$ satisfies $\goalList$ up to $\mn$, then  \LabelB conserves the relevant for autonomous systems.

	Construction of $\aut_{\Act()\times\universeD}$: For each belief $\belief\in\Beliefs$, we can determine the current-state choices $\bAct(\belief)$ (Prop.~\ref{prop:actBel}). 
	Thus, the belief formation \LabelB can be considered as an \Obs-observing strategy assigning $\Act(\belief)$ to an observed history \history with $\belief=\LabelB(\history)$: 
	Since \LabelB is regular, we can build a mealy automaton $\aut_{\Act()}$ that determines $\Act(\LabelB(\history))$ for an observed history \history.
	When $\aut_{\Act()}$ transitions to an accepting state because of \history, this transition gets labelled with the current-state choices $\Act(\LabelB(\history))$.
	We derive a composed automaton ${\aut_{\Act()\times\universeD}}$ by parallel composition of the design-time world \universeD and $\aut_{\Act()}$.  
	In $\aut_{\Act()\times\universeD}$, \ego can take an action \act only if  $\aut_{\Act()}$ allows this, i.e. it is a current-state choice for the observed history. 
	If \ego may not take $\act_1$ in state \state, the combined action $\act=(\act_1,\act_2)$ for all $\act_2\in \Act_{\env}$, leads to the state \sundef. 
\end{mproof}
%

In the next section we will characterise what knowledge, observations and beliefs are relevant.
Thereby we turn to questions like \enquote{Can we do with less observations?}, \enquote{Can we do with less detailed beliefs?} or \enquote{Can we compensate missing observations by adding knowledge?}.
%
%%%%%%%%%%%%%%%%%%%%%%%%%%%%%%%%%%%%%%%%%%%%%%%%%%%%%%%%%%%%%%%%%%%%%%%%%%%%%%%
\subsection{Relevance of \kob}
%%%%%%%%%%%%%%%%%%%%%%%%%%%%%%%%%%%%%%%%%%%%%%%%%%%%%%%%%%%%%%%%%%%%%%%%%%%%%%%
%
Our notion of \emph{relevance conservation} characterises combinations of \kob that allow  a system to form beliefs that are \emph{sufficiently precise} for the system to be optimal. 
In this section we define that \kob is \emph{relevant}, if it conserves the relevant (\ie is sufficient), and in additional also \emph{\enquote{minimal}}. 
%we examine what combinations of \kob, \ie believes knowledge, observations, possible beliefs, are \emph{necessary} for \ego to be an optimal autonomous system.

The three dimensions of \kob are of course interrelated. Intuitively, knowledge (\LabelK) about the world can replace observations that a system \HAS needs otherwise. 
Having more resources for the representation of the inner world model (\Beliefs) allows a system \HAS to store more of the made observations and allows it  to make finer predictions.
More observations (\Obs) vice versa allow \HAS to forget more  and thus have a simpler model of the history and future or to have less knowledge.
We hence expect that often several incomparable minima can be determined. 

To define a \emph{minimal \kob}, we first define partial order relations on  the set of knowledge labeling functions, the set of observations and the set of possible beliefs. 
We then infer a partial order to order tuples \kob. 


We chose the partial orders to reflect the decisions an engineer has to make during the design:
\begin{enumerate}[label=\small(PO\arabic*)\normalsize,itemindent=4mm]
	\item\label{i:obs}  $\Obs\leq\Obs'$ $:\Leftrightarrow$ $\Obs\subseteq\Obs'$\\
For this paper we assume that a greater set of observations means that more sensors are necessary. We are hence interested in determining the minimal set of required observations. 
	\item\label{i:bel}  $\Beliefs\leq\Beliefs'$ $:\Leftrightarrow$ $\Beliefs\subseteq\Beliefs'$\\
		%A belief $\belief\in\Beliefs$ represents in this paper merely the current-state choices. 
		%So, the minimal set of required beliefs represents the simplest encoding of different current-state choices.
		%this is not true due to the possibility to link the belief to the ground-truth
		For the design of a \HAS the size of the set of possible beliefs \Beliefs corresponds to the resources that are necessary to encode the beliefs.
	\item\label{i:know} $\LabelK\leq\LabelK'$ $:\Leftrightarrow$ $\forall \state\in\States: \LabelK(\state) \leq \LabelK'(\state)$ $:\Leftrightarrow$ $\forall \state\in\States: [\LabelK'(\state)]\subseteq [\LabelK(\state)]$,\\
		where [{\priorK}] denotes the set of traces on all possible worlds, $\Worlds$, that satisfy the believed knowledge $\priorK$. 
		As we deal with knowledge-consistent belief formations here, $\priorK\leq\priorK'$ means that \priorK' constrains the beliefs that can be formed less. 
		In other words, the system \HAS knows less since it has more uncertainty.
		
		$\LabelK\leq\LabelK'$ means that $\LabelK'$ declares more knowledge at least at one state of \universeD and it declares not less knowledge than \LabelK in all other states. 
		An engineer can provide prior knowledge, \eg she can hard-code the believed knowledge into \HAS, and she can implement the knowledge labelling, i.e. ensure that mechanisms are in place that will update the knowledge base during \HAS's missions.
	\item\label{i:kob} $\kob \leq \kobp$ $:\Leftrightarrow$ \ref{i:obs}-\ref{i:know} hold.

\end{enumerate}
By  \ref{i:kob} we now  define the notion of \emph{weak relevance}. 
A tuple \kob is weak relevant, if we cannot find a strictly smaller tuple \kobp.
We call this \emph{weak}, since there can be other tuples \kob that are incomparable with \kob.
Hence the question \enquote{Is \kob relevant} does not have a definite answer.
Nevertheless the notion of weak relevance allows to answer, whether a system \HAS can do with more observations in exchange for less knowledge or fewer possible beliefs or whether more knowledge allows \HAS to have fewer possible beliefs or less observations and so on. 

%
%
\begin{definition}[Weak Relevance]\label{def:wr}
	Let a design-time world $\universeD$ and a prioritised list of goals $\goalList$ be given.

	\kob is weakly relevant for $(\universeD,\goalList)$, if 
	\begin{enumerate}
		\item\label{def:cons} there is a belief formation \LabelB of $\epi:=(\universeD,\goalList,\LabelK,\Obs,\Beliefs,\LabelB)$ that conserves the relevant for autonomous systems and 
		\item\label{def:min} for all $\kobp\not=\kob$ with 
			$\LabelK'\leq\LabelK$, $\Obs'\leq\Obs$ and $\Beliefs'\leq\Beliefs$
			$\kobp$  there is no knowledge-consistent belief formation \LabelB' of $\epi':=(\universeD,\goalList,\LabelK',\Obs',\Beliefs',\LabelB')$ that conserves the relevant for autonomous systems.
	\end{enumerate}
	\LabelK is weakly relevant if there are \Obs and \Beliefs, such that \kob is weakly relevant. 
	Analogously we define that \Obs (\Beliefs) is weakly relevant if there are \LabelB, \LabelK and \Beliefs (\Obs). 
\end{definition}
To illustrate the notion, we consider an example. 
\begin{mexample}{Weak Relevance}
Let us assume \ego observes its position $\pos$, a time stamp $t$ and its speed $v$. Its goal is to determine its past average acceleration $\acc$.%
%
\footnote{We assume finite domains and hence finite encodings of numerical values. The computations will be rounded appropriately. \Ego's actions are computation steps.} 
%
Moreover, let us assume that the perception of position is flawed when it is raining while the speed is still correctly measured.  
Then only $\{v,t\}$ is weakly relevant, that is, they suffice to determine the average acceleration. 
	Neither the set $\{\pos,v,t\}$ is weakly relevant nor the set $\{\pos,t\}$. The further is not minimal, the latter does not conserve the relevant, since $\acc$ cannot be determined while it is raining. 

	
	Given \ego has the knowledge \enquote{it will not rain} both $\{\pos,t\}$ and $\{v,t\}$ are weakly relevant. 
\end{mexample} 

Let us now turn to questions  like \enquote{Is \Obs relevant, given \LabelK and \Beliefs?}, i.e. we assume tow component of the triple are known.
The question of relevance ten might have a definite answer, but not necessarily. We hence consider it an interesting notion.
In Def.~\ref{def:relevance} we define \LabelK (\Obs, \Beliefs) to be relevant, if there is no alternative minimal choice, i.e., the system \HAS has to have \LabelK (\Obs, \Beliefs) in order to be able to perform autonomously optimal.
\begin{mdefinition}{Relevance}\label{def:relevance}
	\Obs is relevant for (\universeD,\goalList) with (\LabelK,\Beliefs) iff
	\begin{enumerate}
		\item \Obs is weakly relevant and 
		\item there is no other \Obs' that is weakly relevant.
	\end{enumerate}
	Likewise we define \LabelK and  \Beliefs are relevant for (\universeD,\goalList) with (\Obs,\Beliefs) and respectively (\LabelK,\Obs). 
\end{mdefinition}
%
%
\begin{mtheorem}[Relevance]\label{th:relevant}
	Given a doxastic model   $\epi=(\universeD,\goalList,\priorK,\Obs,\LabelB)$ of \HAS within its environment, we can decide whether \ko is (weakly) relevant for \LabelB in \epi. 
\end{mtheorem}	
\begin{mproof}{\autoref{th:relevant}}
	To show \autoref{def:cons} of Def.~\ref{def:wr} we check whether there is a \Obs-observing strategy in \universeD. 
	To check \autoref{def:min} of Def.~\ref{def:wr} we build the \enquote{lesser} pairs $(\priorK',\Obs')$, i.e. $\priorK'\leq\priorK$, $\Obs'\leq\Obs$ and $(\priorK,\Obs)\not=(\priorK',\Obs')$, and check whether there is a belief labelling \LabelB' that conserves the relevant again by checking whether there a $\Obs'$-observing strategy in \universeD.
\end{mproof}

\bibliographystyle{plain}
\bibliography{refs}
\end{document}
\typeout{get arXiv to do 4 passes: Label(s) may have changed. Rerun}
