
\section{The Sinkhorn Diffusion}\label{sec:diffmirr}
%\marginpar{\color{blue} [YH. As of July 19, It would be great if we improve \ref{rmk:mirror-Langevin}.
%I agree with the structure of this section and general idea of the proof. 
%I just need to check the proof of Theorem 3.2 for possible minor fixes. 
%]}
 It is a natural question whether there exists a diffusion such that the family of marginal distributions at each time point gives the Sinkhorn flow \eqref{eq:velocity}.  A classic example of this correspondence is  %betweee
 the Langevin diffusion (see \cite{lemons1997paul}) whose time marginals satisfy the Fokker-Planck equation. Such stochastic processes are useful in many applications including optimization (see \cite{roberts1996exponential,durmus2019analysis,chizat2022trajectory}) and sampling (see \cite{sohl2015deep,song2020score}). In this section we construct such a stochastic process inspired from a natural Markov chain embedded in the Sinkhorn algorithm, see \cref{prop:mchn} later in the paper. %{\red [Are we going to cut off some part of Section 4 on Markov chains?]}

%{\color{blue} We introduce the following family of stochastic processes,  whose distributions are the mirror flows \eqref{eq:velocity}.}

\begin{defn}[Sinkhorn diffusion] The Sinkhorn diffusion is   the solution to the following stochastic differential equation (SDE):
\begin{equation}\label{eq:diffSDE}
    dX_t=\left(-\frac{\partial f\hfill}{\partial \xsut}(X_t)-\frac{\partial g\hfill}{\partial\xsut}\left(X_t^{u_t}\right)+\frac{\partial h_t\hfill}{\partial \xsut}(X_t)\right)\,dt+\sqrt{2\frac{\partial X_t\hfill}{\partial X_t^{u_t}}}dB_t,
\end{equation}

%{\red {\bf Here what I wrote previously below was wrong as I did not include the effect of the difficution matrix after it is defferentiated. The above expression of SDE is consistent with Ito's formula. } \marginpar{YH. Probably I am misunderstanding something here... Please let me know} [Probability it would be easier for the reader, if we  add a remark that \eqref{eq:diffSDE} is nothing but the same as the usual Langevin diffusion type equation, corresponding to the entropic gradient flow, where in that case the drift part is given by the Wasserstein gradient. In our case, I think we are just multiplying $\frac{\partial x^{u_t}}{\partial x}$ to the drift vector, and use the Brownian motion of the underlying geometry of the metric $\frac{\partial X_t\hfill}{\partial X_t^{u_t}}$. In other words, that \eqref{eq:diffSDE} is nothing but the diffusion version of the entropic gradient flow, with the `mirror' metric given by $\frac{\partial x \hfill}{\partial x^{u_t}}$. Is this a correct understanding? (By the way, this correspondence is different from that one in Theorem~\ref{thm:existpropX}.) For the convergence of such a process, the point is that the underlyng metric also changes along the time and the flow, so, it is *not* straightforward as we see below. But, probably this point of view (Langevin diffusion with respect to the mirror metric, deforming the drift and the Brownian motion by the mirror metric) may put the whole theorems in a more natural way? Probably, what I am saying here is that it would be better to organize the discussion of this section from the point of view of Remark~\ref{rmk:mirror-Langevin}.]}
where
\begin{enumerate}[(i)]

\item $X_0$ is distributed according to an initial density $\rho_0$. At each subsequent time $t$, $X_t$ admits a density $\rho_t=e^{-h_t}$. 

\item $u_t$ is a convex function whose gradient is the Brenier map transporting $\rho_t$ to $e^{-g}$. That is, $(\nabla u_t)_{\#} \rho_t=e^{-g}$. At each time $t$, this leads to a mirror coordinate system $x \mapsto x^{u_t}$. 

\item $\frac{\partial f\hfill}{\partial \xsut}(x)$ refers to the derivative of $x \mapsto f(x)$ with respect to the dual variable $x^{u_t}$. Same for $\frac{\partial h_t\hfill}{\partial \xsut}(x)$.

\item $\frac{\partial g\hfill}{\partial \xsut}(x^{u_t})$ is the gradient of the map $y \mapsto g(y)$ evaluated at $y=x^{u_t}$. 

\item $(B_t,\; t\geq 0)$ is a standard $d$-dimensional  Brownian motion and the diffusion matrix 
$\displaystyle 2\frac{\partial X_t\hfill}{\partial X_t^{u_t}}$ 
at time $t$ is  
\[
2\frac{\partial x\hfill}{\partial x^{u_t}}=2\left( \nabla^2 u_t(x) \right)^{-1},
\]
evaluated at $X_t=x$. 
\end{enumerate}
\end{defn}


Sinkhorn diffusion is an example of Mckean-Vlasov family of diffusions \cite{mckean1966class}.  The study of such systems originated from the probabilistic study of the Boltzmann and Vlasov equations due to Kac~\cite{kac1956foundations}, McKean~\cite{mckean75}, Dobrushin~\cite{dobrushin79}, Tanaka~\cite{tanaka78} and many others. For modern surveys, see Sznitman~\cite{SznitmanSF}, Villani~\cite{villani12notes}, Chaintron and Diez~\cite{ChaintronDiez} and Jabin~\cite{Jabin14}.  

%\SP{SP: not sure how the following references fit here. Are there processes related to the Sinkhorn diffusion?}

%Other references: Stein's variational gradient descent (see~\cite{liu2016stein,Liu2017}), maximum mean discrepancy gradient flow (see \cite{aubin2022mirror}), mirror flows with application to EM algorithm (see \cite{arbel2019maximum}).


We will show that a weak solution of the SDE exists and is unique under suitable assumptions. Towards this, suppose that solution of the PMA \eqref{eq:pma} exists which satisfies Assumption \ref{asn:solcon} with the initial condition $u_0$.
In fact, Assumption \ref{asn:solcon} gives sufficient regularity to the mirroring map $\nabla u_t$, which gives the mirror map $x\mapsto x^{u_t}$. %$=2\left( \nabla^2 u_t(x) \right)^{-1},
The Sinkhorn diffusion always has a dual diffusion process, say $Y$,  given via this mirror map, i.e., $Y_t=X_t^{u_t}$.   As we will show in \cref{thm:existpropX} below $Y$ satisfies the following SDE: 
\begin{equation}\label{eq:dualdiffSDE} 
\begin{split}
dY_t&= -\nabla h_t\left(Y^{w_t}_t\right)dt +\sqrt{2\frac{\partial Y_t\hfill}{\partial Y_t^{w_t}}}d B_t,
\end{split}
\end{equation}
where $\rho_t=e^{-h_t}= \left(\nabla w_t\right)_{\#} e^{-g}$ is the pushforward of $e^{-g}$ by the map $y \mapsto \nabla w_t(y)$. 
Here $\nabla h_t(Y_t^{w_t})$ is the gradient of $h_t$ with respect to its argument, evaluated at $Y_t^{w_t}$. 

%{\red 
%[Isn't $e^{-h_t}$ nothing but $\rho_t$?\\
%Also, isn't $\nabla h_t\left(Y^{w_t}_t\right)$ nothing but$\nabla g\left(Y_t\right) $? $\leftarrow$ I think I was wrong.
%Or is the gradient $\nabla$ in \eqref{eq:dualdiffSDE} for the variable $Y^{w_t}$, not $Y$? 
%]
%}

\begin{thm}\label{thm:existprop}
Let
\[
b(t,y):=- \nabla h_t(y^{w_t}), \quad \sigma(t,y):=\sqrt{2 \frac{\partial y\hfill}{\partial y^{w_t}}}= \sqrt{2  \left(\nabla^{2}\left( w_t(y)\right)\right)^{-1}}.
\]

Additionally, suppose the standard global Lipschitz and linear growth conditions hold,  namely, for some $K>0$, $b, \sigma$ are uniformly $K$-Lipschitz functions on $\rr^d$ and,  
    \begin{equation}\label{eq:growth}
    \norm{b(t,y)}^2 + \norm{\sigma(t,y)}^2 \le K\left(1 + \norm{y}^2 \right)
    \end{equation}
for all $t\ge 0$. 
Then, if the initial distribution is square-integrable, the SDE \eqref{eq:dualdiffSDE} admits a unique strong solution such that every subsequent $Y_t$ is also square-integrable.  

The infinitesimal generator of the process at time $t$, acting on a $\diffcont^2$ function $\phi$, is given by 
\[
\mathcal{L}_t^Y \phi = e^{g} \div \left( e^{-g} \nabla_{y^{w_t}}\phi \right).
\]
Consequently $e^{-g}$ is a stationary distribution for this process.
\end{thm}
%{\red 
%[For this theorem, I think it would be better to separate the general result about SDE with the general $b$ and $\sigma$. Probably we want to put   \cref{asn:solcon} in the theorem. ]
%}

\begin{remark}\label{rem:suffcon}
    The condition \eqref{eq:growth} holds under \cref{asn:solcon} via elementary computations. In particular, this implies that under \cref{asn:solcon}, the conclusion in \cref{thm:existprop} holds.
\end{remark}


\begin{proof}[Proof of~\cref{thm:existprop}] The claim about existence, uniqueness (pathwise and in law) and square-integrability follow from \cite[Theorem 5.2.9]{karatzas1991brownian}.

It remains to compute the infinitesimal generator. Let $\varphi\in \diffcont^2$. Then, by It\^o's formula 
\[
\begin{split}
    \mathcal{L}^Y_t\varphi &= - \frac{\partial \varphi}{\partial y} \cdot \frac{\partial h_t\hfill}{\partial y^{w_t}}(y^{w_t}) + \sum_{i=1}^d \sum_{l=1}^d \frac{\partial y_l\hfill}{\partial y^{w_t}_i} \frac{\partial^2 \varphi}{\partial y_i \partial y_l}.
\end{split}
\]
By the formula for change of measures
\[
-h_t(y^{w_t})= -g(y) - \log \det \frac{\partial y\hfill}{\partial y^{w_t}}.
\]
Taking gradients with respect to $y^{w_t}$ on both sides and using formula \eqref{eq:tensorelpf} we get 
\[
\begin{split}
  \mathcal{L}^Y_t\varphi &=  -\frac{\partial \varphi}{\partial y} \cdot \frac{\partial g\hfill}{\partial y^{w_t}}(y) + \sum_{i=1}^d\sum_{l=1}^d \frac{\partial \varphi}{\partial y_l} \frac{\partial^2 y_l}{\partial y_i \partial y^{w_t}_i} + \sum_{i=1}^d \sum_{l=1}^d  \frac{\partial^2 \varphi}{\partial y_i \partial y_l}\frac{\partial y_l\hfill}{\partial y^{w_t}_i}\\
  &= -\frac{\partial \varphi}{\partial y} \cdot \frac{\partial g\hfill}{\partial y^{w_t}}(y) + \sum_{i=1}^d \frac{\partial}{\partial y_i}\left[ \sum_{l=1}^d \frac{\partial \varphi}{\partial y_l} \frac{\partial y_l\hfill}{\partial y^{w_t}_i} \right]\\
  &=-\frac{\partial \varphi}{\partial y} \cdot \frac{\partial g\hfill}{\partial y^{w_t}}(y) + \sum_{i=1}^d \frac{\partial^2 \varphi}{\partial y_i \partial y^{w_t}_i}= e^{g}\div\left( e^{-g} \nabla_{y^{w_t}} \varphi\right).
\end{split}
\]

That $e^{-g}$ is a stationary measure follows immediately, since
\[
\int \mathcal{L}^Y_{t} \varphi(y) e^{-g(y)}dy = \int \div\left( e^{-g} \nabla_{y^{w_t}} \varphi\right) dy=0.
\]
\end{proof}

\begin{comment}
\begin{proof}
    Consider an SDE of the form
\begin{equation}\label{eq:sdegen}
dY_t = b(t,Y_t)dt + \sigma(t, Y_t) dB_t,
\end{equation}
where $B$ is a standard multidimensional Brownian motion and with an initial condition $Y_0=y_0$ ({\color{blue} In our case $Y_0$ is non-degenerate. Does that change anything?}). 


Stroock and Varadhan proved that (see \cite[Theorem 5.11]{klebaner}) that if $(y,t)\mapsto \sigma(y,t)$ is continuous, positive and if, for each $T>0$, there is a constant $K_T>0$ such that 
\begin{equation}\label{eq:weakexist}
\abs{b(y,t)} + \abs{\sigma(y,t)} \le K_T(1+ \abs{y}),
\end{equation}
for all $(y,t)\in \R^d \times [0,T]$, then the SDE \eqref{eq:sdegen} admits a unique weak solution for any $y_0$ which is also strong Markov. By stopping the process when it hits a ball of radius $R$, we can replace \eqref{eq:weakexist} by a local linear growth criterion: weak existence and uniqueness holds if for every $T>0$ and every $R>0$, there is a constant $K(T,R)$ such that   
\begin{equation}\label{eq:weakexistnew}
\sup_{\abs{y}\le R,\; 0\le t\le T }\left[ \abs{b(y,t)} + \abs{\sigma(y,t)}\right] \le K(T,R)(1+ \abs{y}),
\end{equation}

In the case of \eqref{eq:dualdiffSDE}, 
\[
b(y,t)= -\frac{\partial h_t\hfill}{\partial y^{w_t}}(y^{w_t}),\quad \sigma^2(y,t)= 2\frac{\partial y\hfill}{\partial y^{w_t}}. 
\]
As $(\nabla u_t)_{\#}\rho_t=\exp(-g)$ and $\rho_t=\exp(-h_t)$, \eqref{eq:pma} implies $$h_t(x)=\frac{\partial}{\partial t}u_t(x)-f(x).$$
Then writing both $b(y,t)$ and $\sigma(y,t)$ in terms of $u_t$, we get:
\[
b(y,t)= -\nabla^2 u_t(y^{w_t})\left(\frac{\partial}{\partial y}\frac{\partial}{\partial t}u_t(y^{w_t})-\frac{\partial}{\partial y}f(y^{w_t})\right),\quad \sigma^2(y,t)= 2\nabla^2 u_t(y^{w_t}). 
\]
By Assumptions \ref{asn:smoothfg} and \ref{asn:solcon}, both $b(\cdot,\cdot)$ and $\sigma(\cdot,\cdot)$ are continuous functions on $\{(y,t): |y|\le R,\ 0\le t\le T\}$, which in turn, yields \eqref{eq:weakexistnew}.

\end{proof}
\end{comment}
%{\blue \bf [I understand this theorem is a standard result but, where is the proof of this theorem? ]}
%\item Suppose \bu0(⋅)\bu_0(\cdot) is such that ∇\bu0#μ=ν\nabla \bu_0\#\mu=\nu. Then \eqref{eq:diffSDE} reduces to the following diffusion SDE:
%\begin{equation}
%\begin{aligned}\label{eq:diffmirSDE} 
%dX_t=-\frac{\partial}{\partial\xszt}& g(\bxszt)\,dt+ %\sqrt{2\seczx}\,dB_t, \\ &\nabla \bu_0(X_0)\sim \nu.
%\end{aligned}
%\end{equation}

\begin{remark}
    The above diffusion generator is a time-inhomogeneous analog of the one described by \cite[Section 2]{Kolesnikov2012HessianMC}. 
\end{remark}



\begin{thm}\label{thm:existpropX}
Let $Y$ be a solution of \eqref{eq:dualdiffSDE}. Define $X_t:=Y_t^{w_t}$. Then $(X_t,\; t\geq 0)$ solves \eqref{eq:diffSDE}. Conversely, for any solution $X$ of \eqref{eq:diffSDE}, the transformed process $Y_t=X_t^{u_t}$ is a solution of \eqref{eq:dualdiffSDE}. Hence, under the assumptions of Theorem \ref{thm:existprop}, a weak solution of \eqref{eq:diffSDE} exists and is unique. The marginals of $X_t$ so constructed satisfy \eqref{eq:velocity}. 
\end{thm}


Of course, one may also impose global Lipschitzness and linear growth property on the drift and diffusion coefficients of $X$ to obtain a strong solution. The reason we used the $Y$ process is one, that it has fewer terms in the drift, and two, we only require the $Y$ process to run in stationarity to obtain a weak solution for $X$.
%\marginpar{Good point! Now I see why $Y_t$ was put first.}

In order to prove \cref{thm:existpropX}, we need the well-known Ito's lemma \cite{karatzas1991brownian}, which we note down here for easy reference. 

%\marginpar{It is not a good idea to use the notation $Y_t$ in this lemma.}
\begin{lmm}\label{lem:ito}
Consider an SDE of the form
$$d X_t=b(t,X_t)\,dt+\sigma(t,X_t)\,dB_t.$$
Here $b:[0,\infty)\times \R^d\to \R^d$ and $\sigma:[0,\infty)\times \R^d \rightarrow \R^d \times \R^d$ are progressively measurable functions. Let $\phi:[0,\infty)\times \R^d\to\R$ be an element in  $\diffcont^{1,2}$. Define $S_t=\phi(t,X_t)$. Then 
\begin{align}\label{eq:ito}
d S_t= \frac{\partial}{\partial x}  \phi(t,X_t)^{\top}& \sigma(t,X_t)\,dB_t+\left[\frac{\partial}{\partial t}\phi(t,X_t) + \frac{\partial}{\partial x} \phi(t,X_t)^{\top} b(t,X_t)\right] dt \nonumber\\
&+\frac{1}{2}\trc\left(\sigma(t,X_t) \sigma(t,X_t)^{\top}\nabla^2_x \phi(t,X_t)\right) dt.
\end{align}
\end{lmm}

%\marginpar{[I guess there could be a simplification of this proof. Let me think about it.]}
\begin{proof}[Proof of Theorem \ref{thm:existpropX}]
By \cref{thm:existprop}, there exists a strong Markov process $(Y_t,\ t\ge 0)$ which is a weak solution to the SDE in \eqref{eq:dualdiffSDE}. 
Consider $X_t=Y_t^{w_t}$. By \cref{asn:solcon} (iv), we can apply It\^{o}'s formula in \cref{lem:ito} with
$$b(t,y)=-\frac{\partial h_t \hfill}{\partial y^{w_t}}(y^{w_t}),\quad \sigma^2(t,y)=2\frac{\partial y}{\partial y^{w_t}},\quad \phi(t,y)=y^{w_t},$$ 
to get that $(X_t,\ t\ge 0)$ is Markov and:
\begin{align}\label{eq:dual2}
dX_t&= \frac{\partial Y_t^{w_t}}{\partial Y_t\hfill}  \sqrt{2\frac{\partial Y_t\hfill}{\partial Y_t^{w_t}}}\,dB_t+\left[\left(\frac{\partial}{\partial t}\nabla w_t\right)(Y_t) - \frac{\partial Y_t^{w_t}}{\partial Y_t}  \frac{\partial h_t\hfill}{\partial y^{w_t}}(Y_t^{w_t})\right] dt\nonumber \\ &+\frac{1}{2}\trc\left(2\frac{\partial Y_t\hfill}{\partial Y_t^{w_t}}\nabla^2_y (Y_t)^{w_t}_i\right)_{i\in [d]} dt.
\end{align}
By using \cref{lem:dualPMA} and the multivariate chain rule, the second term on the right hand side above reduces to
\begin{align}\label{eq:dual1}
&\;\;\;\;\;\left(\frac{\partial}{\partial t}\nabla w_t\right)(Y_t) - \frac{\partial Y_t^{w_t}}{\partial Y_t}\frac{\partial h_t\hfill}{\partial y^{w_t}}(Y_t^{w_t})\nonumber \\&=\frac{\partial g}{\partial y}(Y_t)-\frac{\partial f}{\partial y}(Y_t^{w_t})+\frac{\partial}{\partial y}\ldet \left(\frac{\partial Y_t^{w_t}}{\partial Y_t\hfill}\right)-\frac{\partial h_t}{\partial y\hfill}(Y_t^{w_t})\nonumber \\&=\frac{\partial g\hfill}{\partial x^{u_t}}(X_t^{u_t})-\frac{\partial f\hfill}{\partial x^{u_t}}(X_t)-\frac{\partial}{\partial x^{u_t}}\ldet\left(\frac{\partial X_t\hfill}{\partial X_t^{u_t}\hfill}\right)-\frac{\partial h_t\hfill}{\partial x^{u_t}}(X_t) \nonumber \\ &=-\frac{\partial f\hfill}{\partial x^{u_t}}(X_t).
\end{align}
In the third display above, we have used that $Y_t=X_t^{u_t}$. In the fourth display, we use \cref{lem:jacobian} with $\phi=u_t$, $a=h_t$ and $b=g$. Next let us simplify the third term on the right hand side of \eqref{eq:dual2}, for each $i\in [d]$.
\begin{align*}
\trc\left(\frac{\partial Y_t\hfill}{\partial Y_t^{w_t}}\nabla^2_y (Y_t)^{w_t}_i\right)&=\sum_{k} \frac{\partial^2}{\partial y^{w_t}_k \partial y_k}(Y_t)^{w_t}_i=\sum_{k}\frac{\partial^2}{\partial x_k\partial x^{u_t}_k}(X_t)_i.
\end{align*}
In the above display, we have used once again that $Y_t=X_t^{u_t}$. By combining the above observation with  \cref{lem:tensorel}, we then have:
\begin{align}\label{eq:dual3}
&\;\;\;\;\;\trc\left(\frac{\partial Y_t\hfill}{\partial Y_t^{w_t}}\nabla^2_y (Y_t)^{w_t}_i\right)\nonumber \\&=-\frac{\partial}{\partial x^{u_t}_i}\ldet\left(\frac{\partial X_t^{u_t}}{\partial X_t\hfill}\right)=\frac{\partial h_t\hfill}{\partial x^{u_t}_i}(X_t)-\frac{\partial g\hfill}{\partial x^{u_t}_i}(X_t^{u_t}).
\end{align}
By combining \eqref{eq:dual2}, \eqref{eq:dual1} and \eqref{eq:dual3}, we get that $(X_t,\ t\ge 0)$ is a strong Markov process which is a weak solution of \eqref{eq:diffSDE}.

\bigskip 

\noindent Note that there exists a unique strong Markov process which is a weak solution to \eqref{eq:dualdiffSDE} by \cref{thm:existprop}. In order to establish uniqueness in \cref{thm:existpropX}, it suffices to show that given any strong Markov process $(X_t,\ t\ge 0)$ which is a weak solution of \eqref{eq:diffSDE}, the process $(Y_t=X_t^{u_t},\ t\ge 0)$ is a weak solution to \eqref{eq:dualdiffSDE}. Once again, we use It\^{o}'s lemma \ref{lem:ito}, this time with 
$$b(t,x)=-\frac{\partial f\hfill}{\partial \xsut}(x)-\frac{\partial g\hfill}{\partial\xsut}\left(x^{u_t}\right)+\frac{\partial h_t\hfill}{\partial \xsut}(x), \quad \sigma^2(t,x)=2\frac{\partial x\hfill}{\partial x^{u_t}},\quad \phi(t,x)=x^{u_t}.$$
This gives
\begin{align}\label{eq:dual4}
dY_t= \frac{\partial X_t^{u_t}\hfill}{\partial X_t\hfill}  \sqrt{2\frac{\partial X_t\hfill}{\partial X_t^{u_t}}}\,dB_t&+\Bigg[\left(\frac{\partial}{\partial t}\nabla u_t\right)(X_t) -\frac{\partial X_t^{u_t}}{\partial X_t\hfill}\Bigg(\frac{\partial f\hfill}{\partial \xsut}(X_t)+\frac{\partial g\hfill}{\partial\xsut}\left(X_t^{u_t}\right)\nonumber \\ &-\frac{\partial h_t\hfill}{\partial \xsut}(X_t)\Bigg)\Bigg] dt+\frac{1}{2}\trc\left(2\frac{\partial X_t\hfill}{\partial X_t^{u_t}}\nabla^2_x (X_t)^{u_t}_i\right)_{i\in [d]} dt.
\end{align}
By using \eqref{eq:pma}, we get:
\begin{align*}
&\;\;\;\;\left(\frac{\partial}{\partial t}\nabla u_t\right)(X_t) -\frac{\partial X_t^{u_t}}{\partial X_t\hfill}\left(\frac{\partial f\hfill}{\partial \xsut}(X_t)+\frac{\partial g\hfill}{\partial\xsut}\left(X_t^{u_t}\right)-\frac{\partial h_t\hfill}{\partial \xsut}(X_t)\right)\\ &=-2\frac{\partial g}{\partial x}(X_t^{u_t})+\frac{\partial h_t}{\partial x}(X_t)+\frac{\partial}{\partial x}\ldet\left(\frac{\partial X_t^{u_t}}{\partial X_t\hfill}\right)=-\frac{\partial g}{\partial x}(X_t^{u_t}).
\end{align*}
Here the last equality follows by invoking \cref{lem:jacobian} with $\phi=u_t$, $a=h_t$ and $b=g$. Finally, fix $i\in [d]$ and note that by the same computation as in \eqref{eq:dual3}, we have:
\begin{align*}
\trc\left(\frac{\partial X_t\hfill}{\partial X_t^{u_t}}\nabla^2_x (X_t)^{u_t}_i\right)=\frac{\partial}{\partial x_i}\ldet\left(\frac{\partial X_t^{u_t}}{\partial X_t\hfill}\right)=\frac{\partial g\hfill}{\partial x_i}(X_t^{u_t})-\frac{\partial h_t}{\partial x_i}(X_t).
\end{align*}
Combining the two displays above with \eqref{eq:dual4} and using that $Y_t=X_t^{u_t}$, we get that $(Y_t,\ t\ge 0)$ is a strong Markov process which is a weak solution of \eqref{eq:dualdiffSDE}.

\noindent We will use Ito's rule to establish the flow of the marginals. Pick a smooth, compactly supported real-valued function $\phi(\cdot)$, then by invoking Ito's rule in~\eqref{eq:ito} with 
\begin{align*}
b(t,x)=-\frac{\partial f\hfill}{\partial x^{u_t}}(x)-\frac{\partial g\hfill}{\partial x^{u_t}}(x^{u_t})+\frac{\partial h_t\hfill}{\partial x^{u_t}}(x),\quad \quad \sigma^2(t,x)=2\frac{\partial x\hfill}{\partial x^{u_t}},
\end{align*}
the expectation of the generator is given by:
\begin{align*}
&=\E[\mathcal{L}(\phi)(X_t)]\\ &=\int \left\langle\frac{\partial\phi}{\partial x}(x),b(t,x)\right\rangle \exp(-h_t(x))\,dx + \sum_{i,j}\int \frac{\partial}{\partial x_i}\left(\frac{\partial \phi\hfill}{\partial x_j}(x)\right)\left(\frac{\partial x\hfill}{\partial x^{u_t}}\right)_{i,j}\exp(-h_t(x))\,dx\\ &=\int \left\langle\frac{\partial\phi}{\partial x}(x),b(t,x)\right\rangle \exp(-h_t(x))\,dx + \sum_{j}\int \frac{\partial}{\partial x^{u_t}_j}\left(\frac{\partial \phi\hfill}{\partial x_j}(x)\right)\exp(-h_t(x))\,dx\\ &=\int \left\langle\frac{\partial\phi}{\partial x}(x),b(t,x)\right\rangle \exp(-h_t(x))\,dx + \sum_{j}\int \frac{\partial\phi\hfill}{\partial x_{j}}(x)\frac{\partial}{\partial x^{u_t}_j}\exp(-h_t(x))\,dx\\ &=\int \left\langle\frac{\partial\phi}{\partial x}(x),b(t,x)+\frac{\partial}{\partial x^{u_t}}(g(x^{u_t}))\right\rangle \exp(-h_t(x))\,dx\\ &=\int \left\langle\frac{\partial\phi}{\partial x}(x),-\frac{\partial}{\partial x^{u_t}}(f-h_t)(x)\right\rangle \exp(-h_t(x))\,dx. 
\end{align*}
By the absolute continuity of $(\rho_t)$, there exists a velocity field $v_t(\cdot)\in \R^d$ such that the continuity equation 
\begin{equation}\label{eq:continuity}
\frac{\partial\rho_t}{\partial_t\hfill} +\div{(v_t \rho_t)}=0
\end{equation}
is satisfied in the sense that 
\begin{align*}
\E[\mathcal{L}(\phi)(X_t)]=\int \left\langle \frac{\partial\phi}{\partial x\hfill}(x),v_t(x)\right\rangle \exp(-h_t(x))\,dx.
\end{align*}
As the above displays hold for all smooth $\phi(\cdot,\cdot)$, by comparing them, we get:
$$v_t(x)=-\frac{\partial}{\partial x^{u_t}}(f-h_t)(x).$$


\noindent This completes the proof.
\end{proof} 
%\marginpar{YH. From my point of view, this remark~\ref{rmk:mirror-Langevin} is a natural starting point of the discussin in this section 3.}
\begin{remark}\label{rmk:mirror-Langevin}
    A very special case of \eqref{eq:dualdiffSDE} is when $t\mapsto u_t$ is constant, i.e., $u_t\equiv u_0=:u$. Then both $X$ and $Y$ processes are stationary with respect to $e^{-f}$ and $e^{-g}$, respectively. The SDE \eqref{eq:diffSDE} and \eqref{eq:dualdiffSDE}, respectively, reduce to  
    \begin{equation}\label{eq:diffmirSDE}
    \begin{split}
    dX_t &=-\frac{\partial g\hfill}{\partial x^u}\left(X_t^{u}\right)dt+\sqrt{2\frac{\partial X_t\hfill}{\partial X_t^{u}}}dB_t\\
    dY_t &=-\frac{\partial f\hfill}{\partial y^w}\left(Y_t^{w}\right)dt+\sqrt{2\frac{\partial Y_t\hfill}{\partial Y_t^{w}}}dB_t,
    \end{split}
     \end{equation}
    where $w=u^*$, the convex conjugate.
    
    Note that \eqref{eq:diffmirSDE} is equivalent to the mirror Langevin diffusion (see \cite{ahn2021efficient,zhang2020wasserstein}). In the case where $\mu=\nu$, \eqref{eq:diffmirSDE} reduces to the standard Langevin diffusion (see \cite{jordan1998variational,durmus2019analysis}).
\end{remark}


%\subsection{Evolution PDE/continuity equation} 

%\begin{lmm}\label{thm:conteq}
 %The time marginals of the Sinkhorn diffusion $X_t$ in \eqref{eq:diffSDE} is given by the Sinkhorn PDE from Theorem \ref{thm:existlin}. 
 %\end{lmm}





%\begin{prop}[Connection to linearized optimal transport]\label{prop:linot1}
    %Suppose Assumptions \ref{asn:solcon} and \ref{asn:smoothfg} hold. Recall the definition of $v_t(\cdot)$ from \eqref{eq:velocity}. Then we have:
    %\end{prop}

%\begin{proof}   
%\end{proof}
 
%