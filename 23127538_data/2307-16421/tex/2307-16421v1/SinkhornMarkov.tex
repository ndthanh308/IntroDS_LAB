\documentclass[preprint]{amsart}
\usepackage{amsmath, amsfonts, amssymb, amsbsy, bigstrut, graphicx, enumerate,  upref, longtable, comment, bbm, physics, bm}
\usepackage{scalerel}
\usepackage{pstricks,csquotes}
\usepackage[breaklinks]{hyperref}

\hypersetup{
colorlinks=true,%
citecolor=blue,%
filecolor=blue,%
linkcolor=red,%
urlcolor=blue
}

\usepackage[numbers, sort&compress]{natbib} 
\usepackage{tikz}
\usetikzlibrary{decorations.markings}
%\usepackage[notcite,notref]{showkeys}
\usepackage[noabbrev,capitalize]{cleveref}
\usepackage{calc}
\usepackage{longtable}
\usepackage{accents}
\newcommand{\dbtilde}[1]{\accentset{\approx}{#1}}

% Macros

\newcommand{\E}{\mathbb{E}}
\newcommand{\R}{\mathbb{R}}
\newcommand{\argmax}{\operatorname{argmax}}
\newcommand{\argmin}{\operatorname{argmin}}
\newcommand{\xsph}{x^{\phi}}
\newcommand{\xsut}{x^{u_t}}
\newcommand{\emn}{\mathrm{EOT}(\mu,\nu;\vep)}
\newcommand{\xszt}{x^{\bu_0}}
\newcommand{\px}{p_{\scaleto{X}{3.5pt}}}
\newcommand{\py}{p_{\scaleto{Y}{3.5pt}}}
\newcommand{\ysdt}{y^{\bst}}
\newcommand{\bxsut}{(X_t)^{\bu_t}}
\newcommand{\bxszt}{(X_t)^{\bu_0}}
\newcommand{\bu}{\bar{u}}
\newcommand{\et}{\rho_t}
\newcommand{\mcx}{\mathcal{X}}
\newcommand{\mcxp}{\mathcal{X}'}
\newcommand{\ctx}{C_{T,\mcxp}}
\newcommand{\mcy}{\mathcal{Y}}
\newcommand{\secbx}{\frac{\partial}{\partial x} x^{\bu_t}}
\newcommand{\eh}{h_t}
\newcommand{\bmd}{\Delta}
\newcommand{\trc}{\mathrm{Trace}}
\newcommand{\mcI}{\mathcal{I}}
\newcommand{\mcJ}{\mathcal{J}}
\newcommand{\mcD}{\mathcal{D}}
\newcommand{\mc}{\mathcal{C}^{\vep}}
\newcommand{\secox}{\frac{\partial x^{u_t}}{\partial x \hfill}}
\newcommand{\secphx}{\frac{\partial\xsph}{\partial x\hfill}}
\newcommand{\secphxil}{\frac{\partial\xsph_i}{\partial x_{\ell}}}
\newcommand{\secphxim}{\frac{\partial \xsph_i}{\partial x_{m}}}
\newcommand{\bst}{\bu^*_t}
\newcommand{\bysut}{(Y_t)^{\bst}}
\newcommand{\ldet}{\log{\mathrm{det}}}
\newcommand{\secpx}{\frac{\partial X_t \hfill}{\partial\bxsut}}
\newcommand{\seczx}{\frac{\partial X_t\hfill}{\partial\bxszt}}
\newcommand{\secpy}{\frac{\partial Y_t\hfill}{\partial\bysut }}
\newcommand{\secpxs}{\frac{\partial x\hfill}{\partial\xsut }}
\newcommand{\secph}{\frac{\partial x\hfill}{\partial\xsph}}
\newcommand{\secphlj}{\frac{\partial x_{\ell}}{\partial\xsph_j}}
\newcommand{\secphjl}{\frac{\partial x_{j}}{\partial\xsph_{\ell}}}
\newcommand{\secphmk}{\frac{\partial x_{m}}{\partial\xsph_k}}
\newcommand{\secphmi}{\frac{\partial x_{m}}{\partial\xsph_i}}
\newcommand{\secpys}{\frac{\partial \bysut}{\partial Y_t} }
\newcommand{\lmn}{\lambda_{\mathrm{min}}}
\newcommand{\lmx}{\lambda_{\mathrm{max}}}
\newcommand{\mI}{KL}
\newcommand{\lsi}[1]{C_{#1}}
\newcommand{\ptac}{\mathcal{P}_2^{\mathrm{ac}}(\R^d)}
\newcommand{\sfe}{\sigma_F^2(t)}
\newcommand{\sse}{\sigma_S^2(t)}

\newcommand{\diffcont}{\mathcal{C}}
\newcommand{\iprod}[1]{\left\langle #1 \right\rangle}
\newcommand{\probspace}{\mathcal{P}_2\left( \R^d\right)}
\newcommand{\tanspace}{\mathrm{Tan}}
\newcommand{\ltwo}{\mathbf{L}^2}
\newcommand{\wass}{\mathbb{W}}
%\newcommand{\wass}{W}
\newcommand{\diffgen}{\mathcal{L}}
\newcommand{\vep}{\varepsilon}
\newcommand{\gpp}[1]{\tilde{\gamma}_{#1}^{\vep}}
\newcommand{\gnp}[1]{\gamma_{#1}^{\vep}}
\newcommand{\mk}[1]{\rho_{#1}^{\vep}}
\newcommand{\nk}[1]{\eta_{#1}^{\vep}}
\newcommand{\gvp}{\gamma^{\vep}}
\newcommand{\SP}[1]{\textcolor{purple}{SP:#1}}
\newcommand{\ND}[1]{\textcolor{blue}{#1}}
\newcommand{\wt}[3]{\wass_2^{#1}({#2},{#3})}
\newcommand{\dlt}[4]{d^{#1}_{\mathrm{LOT},{#2}}({#3},{#4})}
\newcommand{\lot}[1]{\mathrm{LOT}^2_{#1}}
\newcommand{\opV}{\mathcal{V}^{\vep}}
\newcommand{\opU}{\mathcal{U}^{\vep}}
\newcommand{\opS}{\mathcal{S}^{\vep}}
\newcommand{\rv}{\rho^{\vep}}
\newcommand{\opD}{\mathcal{D}}
\newcommand{\ophS}{\hat{\mathcal{S}}^{\vep}}
\newcommand{\hmn}{\hat{\mu}_n}
\newcommand{\hnn}{\hat{\nu}_n}
\newcommand{\mtd}{\mathbb{T}^d}
\newcommand{\cdt}{c_{\mtd}}
\newcommand{\NN}{\mathbb{N}}
\newcommand{\rr}{\mathbb{R}}
\newcommand{\ent}{\mathrm{Ent}}
\newcommand{\omap}{\mathbf{t}}
\newcommand{\id}{\mathbf{id}}
\newcommand{\txi}{\widetilde{\xi}}
\newcommand{\KL}[2]{\mathrm{KL}(#1 \parallel  #2)}
\newcommand{\mgf}[3]{\mathcal{M}_{#1}[#2:#3]}
\newcommand{\mfR}[3]{\mathrm{Rem}\big[#1\big](#2;#3)}
\newcommand{\Var}{\mathrm{Var}}
\newcommand{\fil}{\mathcal{F}}
\newcommand{\opP}{\bm{P}^{\vep}}
\newcommand{\opQ}{\bm{Q}^{\vep}}
\newcommand{\tpP}[1]{\bm{\tilde{P}}_{#1\vep}}
\newcommand{\tpQ}[1]{\bm{\tilde{Q}}_{#1\vep}}
\newcommand{\opR}{\bm{R}^{\vep}}
\newcommand{\fv}{\mathrm{FV}}
\newcommand{\hs}{\mathrm{HS}}



\renewcommand{\subjclassname}{\textup{2010} Mathematics Subject Classification} 
\allowdisplaybreaks
% Auxiliary 

\newtheorem{thm}{Theorem}
\newtheorem{lmm}[thm]{Lemma}
\newtheorem{cor}[thm]{Corollary}
\newtheorem{prop}[thm]{Proposition}
\newtheorem{defn}[thm]{Definition}
\newtheorem{assm}{Assumption}
\newtheorem{problem}[thm]{Problem}
\theoremstyle{definition}
\newtheorem{remark}[thm]{Remark}
\newtheorem{ex}[thm]{Example}

\numberwithin{thm}{section}
%\numberwithin{remark}{section}
%\numberwithin{ex}{section}
\numberwithin{assm}{section}
%\numberwithin{defn}{section}
\numberwithin{equation}{section}



\begin{document}

\title[Wasserstein Mirror Gradient Flows]{Wasserstein Mirror Gradient Flow as the limit of the Sinkhorn algorithm}% and Scaling Limit of the Sinkhorn Algorithm}



\author{Nabarun Deb}
\address{Nabarun Deb\\ Department of Mathematics \\ University of British Columbia\\ Vancouver, Canada\\ {Email: ndeb@math.ubc.ca}}
\author{Young-Heon Kim}
\address{Young-Heon Kim\\ Department of Mathematics \\ University of British Columbia\\ Vancouver, Canada\\ {Email: yhkim@math.ubc.ca}}
\author{Soumik Pal}
\address{Soumik Pal\\ Department of Mathematics \\ University of Washington\\ Seattle WA 98195, USA\\ {Email: soumik@uw.edu}}
\author{Geoffrey Schiebinger}
\address{Geoffrey Schiebinger\\ Department of Mathematics \\ University of British Columbia\\ Vancouver, Canada\\ {Email: geoff@math.ubc.ca}}


\begin{abstract}
 	We prove that the sequence of marginals obtained from the iterations of the Sinkhorn algorithm or the iterative proportional fitting procedure (IPFP) on joint densities, converges to an absolutely continuous curve on the $2$-Wasserstein space, as the regularization parameter $\vep$ goes to zero and the number of iterations is scaled as $1/\vep$ (and other technical assumptions). This limit, which we call the Sinkhorn flow, is an example of a Wasserstein mirror gradient flow, a concept we introduce here inspired by the well-known Euclidean mirror gradient flows. In the case of Sinkhorn, the gradient is that of the relative entropy functional with respect to one of the marginals and the mirror is half of the squared Wasserstein distance functional from the other marginal. Interestingly, the norm of the velocity field of this flow can be interpreted as the metric derivative with respect to the linearized optimal transport (LOT) distance. An equivalent description of this flow is provided by the parabolic Monge-Amp\`{e}re PDE whose connection to the Sinkhorn algorithm was noticed by Berman (2020). We derive conditions for exponential convergence for this limiting flow. We also construct a Mckean-Vlasov diffusion whose  marginal distributions follow the Sinkhorn flow.
\end{abstract}

\keywords{Entropy regularized optimal transport, Mckean-Vlasov diffusion, mirror descent, parabolic Monge-Amp\`ere, Sinkhorn Algorithm, Wasserstein gradient flows}
	
\subjclass[2000]{49N99, 49Q22,  60J60}


\thanks{Thanks to PIMS Kantorovich Initiative for facilitating this collaboration supported through a PIMS PRN and the NSF Infrastructure grant DMS 2133244.  Pal is supported by NSF grants DMS-2052239 and DMS-2134012.  Kim and Schiebinger are supported in part by New Frontier Research Funds (NFRF) of Canada as well as NSERC Discovery grant. Schiebinger is also supported by a MSHR Scholar Award, a CASI from the Burroughs Wellcome Fund and a CIHR Project Grant. Nabarun Deb is supported by the PIMS PRN postdoc fellowship.}


\maketitle

%\tableofcontents
%\marginpar{We can remove the table of contents after finishing the draft}
The problem of the presence or absence of phase transition is central in statistical mechanics. To prove the existence of phase transition, the standard idea is to define a notion of contour and use \textit{Peierls' argument} \cite{Peierls.1936}. In the usual Ising model \cite{Ising_25}, particles of the system interact only with their nearest-neighbors. On ferromagnetic long-range Ising models \cite{Anderson_Yuval_69}, there is interaction between each pair of spins in the lattice. The Hamiltonian of the model is given formally by
\begin{equation*}
    H(\sigma) = - \sum_{x,y\in \Z^d}J_{xy}\sigma_x\sigma_y,
\end{equation*}
where $J_{xy}=J|x-y|^{-\alpha}$, $J>0$, $\alpha > d$. It is well-known that the Peierls' argument in dimension 2 implies phase transition for Ising models with nearest-neighbors or long-range interactions when $d\geq 2$, using correlation inequalities. For the unidimensional lattice, it was known that short-range models do not present phase transition. In the long-range case, a different behavior was expected depending on the exponent $\alpha$ (see \cite{Kac_Thompson_69}), but the problem was challenging since contours were first created as multidimensional objects.

In dimension $d=1$, phase transition was proved first in 1969 by Dyson \cite{Dyson.69}, for $\alpha \in (1,2)$, by proving phase transition in an auxiliary model and then using correlation inequalities. In 1982, Fr{\"o}hlich and Spencer \cite{Frohlich.Spencer.82} introduced a notion of one-dimensional contours and then applied the Peierls' argument to show phase transition for the critical value $\alpha = 2$. These contours were inspired by the multiscale techniques previously introduced to study the Berezinskii-Kosterlitz-Thouless transition in two-dimensional continuous spin systems \cite{FS81}. Later, Cassandro, Ferrari, Merola and Presutti  \cite{Cassandro.05} extended the contour argument previously available for $\alpha=2$ to exponents $\alpha\in (3-\frac{\ln 3}{\ln 2}, 2)$, with the additional restriction that the nearest-neighbor interaction is strong, i.e.,  ${J(1)\gg 1}$; this restriction was removed for a subclass of interactions in \cite{Bissacot.Endo.18}. Further results were obtained using contour arguments, such as the decay of correlations, cluster expansions, phase transition with random interactions, etc; some references with these results are \cite{ Cassandro.Merola.Picco.17, Cassandro.Merola.Picco.Rozikov.14, Imbrie.82, Imbrie.Newman.88, Johansson.91}. 

In the multidimensional setting ($d\geq 2$), Ginibre, Grossmann, and Ruelle, in \cite{Ginibre.Grossmann.Ruelle.66}, proved the phase transition for $\alpha > d+1$, using an enhanced version of Peierls' argument and the usual contours. Park proposed a different notion of contour for long-range systems in \cite{Park.88.I, Park.88.II}, extending the Pirogov-Sinai theory available for short-range interactions assuming $\alpha > 3d+1$, although he can also consider Potts models with his methods. Some results in the literature suggest that truly long-range effects appear only when $d < \alpha \leq d+1$, see for instance, \cite{Biskup_Chayes_Kivelson_07}. Recently, Affonso, Bissacot, Endo and Handa \cite{Affonso.2021}, inspired by the ideas from Fr{\"o}hlich and Spencer in \cite{FS81, Frohlich.Spencer.82}, introduced a version of multiscale multidimensional contour and proved phase transition by a contour argument in the whole region $\alpha > d$. They can consider long-range Ising models with deterministic decaying fields, first introduced in the context of nearest-neighbor interactions in \cite{Bissacot_Cioletti_10}. For these models, the lack of analyticity of the free energy does not imply phase transition since these models have the same free energy as the models with zero field. It is expected that fields decaying slowly imply uniqueness. In this setting, a contour argument is useful for proofs of phase transitions as well for uniqueness, some papers with models with deterministic decaying fields are \cite{Aoun_Ott_Velenik_23, Bissacot_Cass_Cio_Pres_15, Bissacot.Endo.18, Cioletti_Vila_2016}.

The Random Field Ising model (RFIM) \cite{Imry.Ma.75} is the nearest-neighbor Ising model with an additional external field acting on each site $(h_x)_{x\in\Z^d}$ that is a family of i.i.d. Gaussian random variable with mean 0 and variance 1. Formally, the Hamiltonian of the model is given by
\begin{equation*}
    H(\sigma) = - \sum_{\substack{x,y\in \Z^d \\|x-y|=1}}J\sigma_x\sigma_y  - \varepsilon\sum_{x\in\Z^d}h_x\sigma_x,
\end{equation*}
where $J>0$, $\varepsilon>0$, $\alpha > d$ and $d \geq 1$. A detailed account of the history of the phase transition problem for this model, as well as detailed proofs, was given in \cite{Bovier.06}. Here we present a brief overview.

During the 1980s, the question of the specific dimension where phase transition for the RFIM should happen attracted much attention and was a topic of heated debate. Two convincing arguments were dividing the physics community. One of them, due to Imry and Ma \cite{Imry.Ma.75}, was a non-rigorous application of the Peierls' argument together with the use of the isoperimetric inequality. The key idea of Peierls' argument is to define a notion of contour and calculate the energy cost of "erasing" each contour, i.e., the energy cost of flipping all spins inside the contour. When there is no external field, that energy necessary to flip the spins in a region $A\subset \Z^d$ is of the order of the boundary $|\partial A|$. When we add an external field, we get an extra cost depending on this field. Imry and Ma argued that this cost should be approximately $\sqrt{|A|}$, which is smaller than $|\partial A|$ for all regions only when $d\geq 3$, so this should be the region where phase transition occurs. The other argument, due to Parisi and Sourlas \cite{Parisi.Sourlas.79}, based on dimensional reduction, predicted that the $d$-dimensional RFIM would behave like the $d-2$-dimensional nearest-neighbor Ising model, therefore presenting phase transition only when $d\geq 4$. 

The question was settled by two celebrated papers showing that Imry and Ma's prediction was correct. First, in 1988, Bricmont and Kupiainen \cite{Bricmont.Kupiainen.88} showed that there is phase transition almost surely in $d\geq3$, for low temperatures and variance $\varepsilon$ small enough. Their proof uses a rigorous renormalization group analysis for the short-range case and it is considered involved. Still, they claimed that the result works for any model with a suitable contour representation and centered sub-gaussian external field. Later on, Aizenman and Wehr \cite{Aizenman.Wehr.90} proved uniqueness for $d\leq 2$. For detailed proofs of these results, we refer the reader to \cite{Bovier.06} (see also \cite{Berretti.85, Camia.18, Frohlich.Imbre.84,  Klein.Masooman.97} for more uniqueness results). 

Recently, Ding and Zhuang, see \cite{Ding2021}, provided a simpler proof of the phase transition, not using RGM. And in  \cite{Ding.Liu.Xia.22}, Ding, Liu and Xia proved that if $\beta_c(d)$ is the critical inverse of the temperature of the Ising model with no field, for all $\beta>\beta_c(d)$ there exists a critical value $\varepsilon_0(d, \beta)$ such that the RFIM with $\varepsilon \leq \varepsilon_0$ presents phase transition. 

In the present paper, we are considering a long-range Ising model with a random field, whose Hamiltonian is given formally by
\begin{equation*}
    H(\sigma) = - \sum_{x,y\in \Z^d}J_{xy}\sigma_x\sigma_y - \varepsilon\sum_{x\in\Z^d}h_x\sigma_x,
\end{equation*}
where $J_{xy}=J|x-y|^{-\alpha}$, $J, \varepsilon>0$, $\alpha > d$ and $h_x\in\mathbb{R}$, $d\geq 3$.
Until now, the only known result in the long-range setting is for the one-dimensional long-range Ising model with a random field, by Cassandro, Orlandi, and Picco \cite{Cassandro.Picco.09}. They used the contours of \cite{Cassandro.05} to show the phase transition for the model when $\alpha\in (3-\frac{\ln 3}{\ln 2}, \frac{3}{2})$, under the assumption $J(1) \gg 1$. We stress that, as remarked by Aizenman, Greenblatt, and Lebowitz \cite{Aizenman_Greenblatt_Lebowitz_2012}, although their argument does not work for the whole region for the exponent $\alpha$, the phase transition holds for values close to the critical value $\alpha=3/2$, since by the Aizenman-Wehr theorem we know that there is uniqueness for $\alpha>3/2$.

The argument from Ding and Zhuang in \cite{Ding2021}, for $d\geq3$, involves controlling the probability of a bad event, which is closely related to controlling the quantity $$\sup_{\substack{0\in A\subset\Z^d \\ A \text{ connected }}}\frac{\sum_{x\in A}h_x}{|\partial A|},$$ known as the greedy animal lattice normalized by the boundary. The greedy animal lattice normalized by the size, instead of the boundary, was extensively studied for general distributions of $(h_x)_{x\in\Z^d}$, see \cite{Cox_Gandolfi_Griffin_Kesten_93, Gandolfi_Kesten_94, Hammond_06, Martin_02}. When we normalize by the boundary, an argument by Fisher, Fr\"{o}hlich and Spencer \cite{FFS84} shows that the expected value of the greedy animal lattice is constant. In dimension $d=2$, the expected value is not finite, see \cite{Ding.Wirth.20}. The supremum is taken over connected regions containing the origin since the interiors of the usual Peierls contours are of this form.


For the long-range model, the interior of contours is not necessarily connected. In fact, long-range contours may have considerably large diameters with respect to their size, so their interiors can be very sparse. To avoid this, we define contours, strongly inspired by the $(M,a,r)$-partition in \cite{Affonso.2021}, using a multiscaled procedure that assures that the contours have no cluster with small density.  With them, we generalize the arguments by Fisher-Fr\"{o}hlich-Spencer \cite{FFS84}, and prove that the expected value of the greedy animal lattice is constant, even considering regions not necessarily connected in the supremum. Then, we prove the phase transition for $d\geq 3$. The main result of this paper is the following.
\begin{theorem*}Given $d\geq 3$, $\alpha>d$, there exists $\beta_c\coloneqq\beta(d, \alpha)$ and $\varepsilon_c\coloneqq\varepsilon(d, \alpha)$ such that, for $\beta >\beta_c$ and $\varepsilon\leq \varepsilon_c$, the extremal Gibbs measures $\mu_{\beta, \varepsilon}^+$ and $\mu_{\beta, \varepsilon}^-$ are distinct, that is, $\mu_{\beta, \varepsilon}^+ \neq \mu_{\beta, \varepsilon}^-$ $\mathbb{P}$-almost surely. Therefore the long-range random field Ising model presents phase transition.
\end{theorem*}

This paper is divided as follows. In Section 2, we define the model and the contours, and suitable generalizations to the constructions in \cite{Affonso.2021} are introduced.  In Section 3, we define two bad events of the external field and prove that they occur with a small probability.  In Section 4, we present the proof of the phase transition.




\section{Mirror gradient flows: Euclidean and Wasserstein}\label{sec:mirror}

\subsection{Mirror descent and mirror gradient flow in Euclidean spaces}

Mirror descent was originally proposed by Nemirovsky and Yudin in \cite{nemirovskii83} as a generalization of gradient descent. For a modern account and its connections to other gradient based optimization schemes, see \cite[Sections 2.1.3 and B.2]{Wilson18}. 




%it is an iterative procedure to do gradient descent on a function $F:\R^d\rightarrow \R$ on the dual coordinates with respect to $u$. There are several closely related and essentially equivalent versions of this algorithm. We choose the one that resembles most the Euler scheme for Euclidean gradient descent.

%Fix a positive step size $\tau \approx 0$. Start at $x_0$. The explicit mirror descent algorithm determines $x_1$ by
%\[
%x^u_1 = x^u_0 - \tau \nabla_x F(x_0). 
%\]
%Thereafter, iteratively, 
%\[
%x^u_{k+1} = x^u_k - \tau \nabla_x F(x_k). 
%\]
%The implicit scheme utilizes the Bregman divergence associated with $u$ given by 
%\[
%D_u(x \mid y)= u(x) - u(y) - \iprod{\nabla u(y), x-y}. 
%\]
%Iteratively, given $x_0, \ldots, x_k$, the next step is given by  \[
%x_{k+1}= \mathrm{argmin}\left[ F(x) + \frac{1}{\tau}D_u(x \mid x_k) \right].
%\]

%To see its connection with the regular gradient descent, note that when $u(z)=\frac{1}{2}\norm{z}^2$, the explicit and the implicit mirror descent schemes are the usual Euler explicit and implicit schemes for gradient descent of $F$.

Assume that $u:\R^d\rightarrow \R$ is a differentiable convex function such that $\nabla u: \R^d \rightarrow \R^d$ is a diffeomorphism. This diffeomorphism induces two global coordinate charts on $\R^d$, namely $x\leftrightarrow x^u:= \nabla u(x)$ (also see \cref{def:pdcor}). This map is called the \textit{mirror map} in the literature. Given a differentiable function $F: \rr^d \rightarrow \rr$, its Euclidean mirror gradient flow can be described by the ODE (with a given initial condition)
\[
\frac{d}{dt} x^u_t = - \frac{\partial}{\partial x} F(x_t),\; t\ge 0.
\]
Applying chain rule to the LHS, we get the equivalent alternative ODE
\begin{equation}\label{eq:mirrorflowEuclid}
\frac{d}{dt} x_t = -\frac{\partial x\hfill}{\partial x^u}(x^u_t) \frac{\partial F}{\partial x}(x_t)= - \frac{\partial F\hfill}{\partial x^u}(x_t). 
\end{equation}
In the last equality we are utilizing the idea that $x \mapsto x^u$ is a diffeomorphism on $\R^d$ and gradients of functions can be taken with respect to either coordinate system. When $u(x)=\frac{1}{2} \norm{x}^2$, both ODEs coincide and we recover the usual Euclidean gradient flow. 

The Euclidean mirror gradient flow can be interpreted as a ``true'' gradient flow if we change the manifold structure of $\R^d$ to make it a so-called \textit{Hessian Riemannian manifold}. See \cite{Hessian97} and \cite[Section B.2.4]{Wilson18}. Roughly, at a point $x\in \R^d$, consider the positive definite Hessian matrix $\nabla^{2} u(x)$. For any tangent vectors $y$ at $x$, change the Riemannian metric to $y^T \nabla^{2} u(x) y$. This induces a new Riemannian manifold on $\R^d$. The mirror gradient flow \eqref{eq:mirrorflowEuclid} is the gradient flow of $F$ on this Hessian Riemannian manifold. 

\subsubsection{Examples.} Consider the following examples where we take $d=1$, $F(x)=\frac{1}{2}x^2$, and different choices of the mirror function $u$. Notice that the function $F$ admits a unique minimizer at zero. For each of the following flows, the initial condition is $x_0=1$.

\begin{enumerate}[(i)]
    \item $u(x)=\frac{1}{2}x^2$. In this case $x^u=x$. Thus, we get back our usual gradient flow equation $\dot{x}_t=-\nabla_x F(x_t)=-x_t$. For our given initial condition, the solution is $x_t=\exp(-t)$, $t\ge 0$. It converges to zero exponentially fast. 
    \item $u(x)=x^4$. Thus $x^u=4x^3$ and $\frac{\partial x^u}{\partial x}=12x^2$. Thus, 
    \[
    \dot{x}_t= -\nabla_{x^u} F(x_t)= -\frac{1}{12 x_t^2} x_t= -\frac{1}{12 x_t}.
    \]
    The solution of this ODE, with $x_0=1$, is $x_t=\sqrt{1-t/6}$, $0\le t \le 6$. The solution does not extend beyond $[0,6]$ due to a singularity at $t=6$ when $x_6=0$, the minimizer of $F$. The flow converges to the minimizer in finite time, unlike the previous example. 
    \item $u(x)=1/x$ for $x>0$. Thus $x^u=-\frac{1}{x^2}$ and $\frac{\partial x^u}{\partial x\hfill}=\frac{2}{x^3}$. The mirror flow equation is given by $\dot{x}_t= -\frac{1}{2} x_t^4$. The unique solution with $x_0=1$ is $x_t=\left(1 + 3t/2 \right)^{-1/3}$. The solution is well defined for all $t\ge 0$ and converges to the minimizer of $F$ just polynomially. 
\end{enumerate}

%For a different set of example, consider again $d=1$, the same mirror function $u(x)=e^x$ and the family of convex functions $F_\lambda(x)=e^{\lambda x}$, for $\lambda \neq 0$. These functions do not admit any minimizer on the real line. However, on the extended real line, the minimizer is $-\infty$ for $\lambda >0$ and $+\infty$ for $\lambda <0$.  

%Take the initial condition $x_0=0$ throughout. 
%The usual gradient flow equation is $\dot{x}_t=-\lambda e^{\lambda x_t}$, $x_0=1$, which admits a unique solution
%\[
%x_t= -\frac{1}{\lambda}\log\left( \lambda^2 t + 1\right).
%\]
%The mirror gradient ODE is given by 
%\[
%\dot{x}_t = -\frac{1}{u''(x_t)} F'_\lambda(x_t)=-\lambda e^{(\lambda-1)x_t}, \quad x_0=0.
%\]
%The unique solution for $\lambda \neq 0$ is
%\[
%x_t=\begin{cases}
%-\frac{1}{\lambda-1} \log\left(\lambda(\lambda-1) t +1 \right), & %\text{when $\lambda \in (1, \infty) \cup (-\infty, 0)$},\\
%-\lambda t, & \text{when $\lambda =1$},\\
%\frac{1}{1-\lambda} \log\left(\lambda(1-\lambda) t +1 \right), & %\text{when $\lambda \in (0,1)$}.
%\end{cases}
%\]

%As one can see the impact of the mirror becomes pronounced in the interval $\lambda \in (0,1]$, with $\lambda=1$ giving a drastically different behavior and faster convergence.

In all these examples it is clear that the behavior of the Hessian of the mirror function $u$ in a neighborhood of the minimizer of $F$ plays a very important role. If this Hessian is very close to zero, the mirror flow speeds up significantly more than the gradient flow of $F$. This is an intuition that we expect to carry over to the Wasserstein set-up as well. 


\subsection{An informal description of Wasserstein mirror gradient flows}\label{sec:wassmirrorflowformal} 


%\marginpar{ In this Section 2.2, we need some citations; we can put [Villani, Topics on OT]. e.g. After (25), cite [Villani’s book] for the expression of Wasserstein Gradient. 
%}
Recall that $\ptac$ denote the set of Lebesgue absolutely continuous Borel probability measures on $\R^d$ with finite second moments. Equip this space with the Wasserstein $2$ metric. 
We take the point of view (originally due to Otto \cite{Otto_2001}) that this Wasserstein space can be thought of as an ``infinite dimensional Riemannian manifold'' in the following sense \cite[Chapter 8]{ambrosio2005gradient}. At any $\rho \in \ptac$ define the tangent space by $\tanspace_\rho = \overline{\left\{ \nabla g,\; g \in \diffcont_c^\infty \right\}}$,
where the closure is taken in $L^2(\rho)$ \cite[Section 8.4]{ambrosio2005gradient}. The metric tensor is given by the $L^2$  norm. 

%\marginpar{Otto calculus is usually for absolutely continuous (ac) measures, so on $P_{2,ac}$. }

For suitable functions $F:\ptac\rightarrow \R \cup \{\infty\}$, the Wasserstein gradient at an absolutely continuous probability measure with a $\mathcal{C}^1$ density is the element $\nabla_{\wass}F(\rho)\in \tanspace_\rho$, characterized by the following identity \cite[Lemma 10.4.1]{ambrosio2005gradient} that holds for all $v\in \tanspace_\rho$:
\[
\iprod{\nabla_{\wass}F(\rho), v }_{\ltwo(\rho)}= \int_{\R^d} \frac{\delta F}{\delta \rho}(x) \dot{\rho}(x) dx,
\]
where $\dot{\rho}= - \div(v\rho)$. The Wasserstein gradient flow is given by the continuity equation \cite[Section 11.1]{ambrosio2005gradient} $\frac{\partial}{\partial t} \rho_t = \div{\left(\nabla_{\wass}F(\rho_t) \rho_t\right)}$.


%Recall the notions of a generalized geodesic \cite[Definition 9.2.2]{ambrosio2005gradient} and generalized geodesically convex functions \cite[Definition 9.2.4]{ambrosio2005gradient}. Let $U:\probspace\rightarrow \R \cup \{\infty\}$ denote a generalized geodesically convex function. 
%\marginpar{ Before (2.2), generalized geodesics should be defined, etc. } 

Let 
\begin{equation}\label{eq:mirror}
U(\rho)=\frac{1}{2}\wass_2^2\left(\rho, e^{-g}\right),
\end{equation}
which is known to be convex over generalized geodesics with base $e^{-g}$ (\cite[Lemma 9.2.1 and Definition 9.2.2]{ambrosio2005gradient}). 
We will use $U$ as a ``mirror'' to generate a ``mirror potential'' given by 
\[
\rho \mapsto \rho^U:=\nabla_{\wass}U(\rho).
\]
Notice that while $\rho$ is a measure, its mirror potential $\rho^U$ is a function in $\ltwo(\rho)$. For the special case of $U(\rho)=\frac{1}{2}\wass_2^2\left(\rho, e^{-g}\right)$, $\rho^U$ is the  Kantorovich map (gradient of the Kantorovich potential) transporting $\rho$ to $e^{-g}$ \cite[Theorem 10.4.12]{ambrosio2005gradient}. 

By analogy with the Euclidean space, for a suitable function $F: \probspace\rightarrow \R\cup \{\infty\}$, one may define the Wasserstein mirror gradient flow by two equivalent PDEs. The first one takes time derivative in the mirror potential.
\begin{equation}\label{eq:mirrorgradflow}
\frac{\partial}{\partial t} \rho^U_t = - \nabla_{\wass} F(\rho_t),
\end{equation}
which represents an equality of two elements in $\tanspace_{\rho_t}$ viewed as a subspace of $\ltwo(\rho_t)$. 


Let $\nabla u$ denote the Brenier map transporting $\rho$ to $e^{-g}$ for a convex function $u$. Then 
\begin{equation}\label{eq:KPtoBP}
\nabla u(x) =  x - \rho^U(x).
\end{equation}
In terms of the convex Brenier potentials, \eqref{eq:mirrorgradflow} can be equivalently written as 
\begin{equation}\label{eq:mirrorgradflow2}
\frac{\partial}{\partial t} \nabla u_t = \nabla_{\wass} F(\rho_t).
\end{equation}
When $\nabla_{\wass} F(\rho_t)= \nabla \frac{\delta F}{\delta \rho_t}$, the gradient of the first variation of $F$ at $\rho_t$ \cite[Lemma 10.4.1]{ambrosio2005gradient}, by removing the gradients on both sides of the above, we get the PDE.
\begin{equation}\label{eq:mirrorgradflow3}
\frac{\partial u_t}{\partial t}(x)= \frac{\delta F}{\delta \rho_t}(x), \quad \text{where $u_t$ is convex and} \; (\nabla u_t)_{\#} \rho_t=e^{-g}.
\end{equation}
As shown in the following Section \ref{sec:pmasec}, this family of PDEs includes the parabolic Monge-Amp\`ere \eqref{eq:pma2}.

The second equivalent way of describing the Wasserstein mirror gradient flow is to take the time derivative of the curve in the space of measures, i.e., specify a continuity equation for the curve on Wasserstein space.  

Fix an absolutely continuous measure $\rho$. The Hessian matrix $\nabla^2 u$ is defined $\rho$-a.s. in the Alexandrov sense and is positive semidefinite. Define a new metric tensor on $\tanspace_{\rho}$ by 
\begin{equation}\label{eq:hessian-metric}
\iprod{v_1, v_2}_U= \iprod{v_1, \left(\nabla^2 u\right) v_2}_{\ltwo(\rho)}, \quad v_1, v_2 \in \tanspace_\rho. 
\end{equation}
This defines a new Riemannian gradient on the Wasserstein space $\nabla_{\wass}^U$ and a new gradient flow equation
\begin{equation}\label{eq:newgradflow}
\frac{\partial}{\partial t} \rho_t = \div\left( \nabla_{\wass}^U F(\rho_t) \rho_t \right).
\end{equation}
When it exists, we call this flow the Wasserstein mirror gradient flow of $F$ with respect to $U(\rho)=\frac{1}{2}\wass_2^2\left(\rho, e^{-g}\right)$. While $\nabla_{\wass} F(\rho)$ is given by the gradient of the first variation of $F$, $\nabla_x \left(\frac{\delta F}{\delta \rho}\right)$, for $\nabla_{\wass}^U F(\rho_t)$ we get
\begin{equation}\label{eq:expgrad}
\nabla_{\wass}^U F(\rho_t)= \nabla_{x^{u_t}}\left(\frac{\delta F}{\delta \rho_t}\right), \quad \text{where $u_t$ is convex and}\; (\nabla u_t)_{\#} \rho_t= e^{-g}.
\end{equation}
Since $\frac{\partial x\hfill}{\partial x^u} = \left(\nabla^2 u \right)^{-1}$, \eqref{eq:expgrad} can also be written as \[
\nabla_{\wass}^U F(\rho_t)=  \left(\nabla^2 u \right)^{-1} \nabla_{\wass} F(\rho)\]
which is consistent with the metric \eqref{eq:hessian-metric}.
Thus \eqref{eq:newgradflow} and \eqref{eq:expgrad} describe the Wasserstein mirror gradient flow as a continuity equation. 


The connection between \eqref{eq:newgradflow} and \eqref{eq:mirrorgradflow3} is that if $(u_t,\; t\ge 0)$ is a solution of the latter, then $\rho_t:=\left( \nabla u_t\right)^{-1}_{\#} e^{-g}$ 
 satisfies the former. We show this below assuming that the solution of \eqref{eq:mirrorgradflow3} satisfies, for each $t$, $\nabla^2 u_t(x)$ is invertible and the inverse is continuous in $x$, $\rho_t$ a.s.. 

 Let $T_t=\nabla u_t$ denote the transport map. Then, by the chain rule,  
\begin{align}\label{eq:inverse_derivative}
    \left[\frac{\partial}{\partial t} (T_t)^{-1} \right]( T_t(x)) = - [\nabla T_t(x)]^{-1}   \left[\frac{\partial}{\partial t} T_t \right](x).
\end{align}
Notice that $[\nabla T_t(x)]^{-1} =\left(\nabla^2u_t(x)\right)^{-1}$.
Thus, from \eqref{eq:mirrorgradflow2}, 
\[
\left[\frac{\partial}{\partial t} (T_t)^{-1} \right]( T_t(x)) = - \left(\nabla^2u_t(x)\right)^{-1}\nabla \frac{\delta F}{\delta \rho_t}=-\nabla_{x^{u_t}}\frac{\delta F}{\delta \rho_t}=- \nabla_{\wass}^UF(\rho_t). 
\]

Let $\xi$ be a smooth, compactly supported test function. Then 
\[
\begin{split}
\frac{d}{dt} &\int \xi(x) \rho_t(x)dx = \frac{d}{dt} \int \xi\left(T_t^{-1}(y)\right)e^{-g(y)}dy\\
&=\int \nabla \xi\left(T_t^{-1}(y)\right)\cdot  \left[\frac{\partial}{\partial t} (T_t)^{-1} \right](y) e^{-g(y)}dy  \\
&=\int \nabla \xi(x) \left[\frac{\partial}{\partial t} (T_t)^{-1} \right](T_t(x)) \rho_t(x)dx=- \int \iprod{\nabla \xi(x), \nabla_{\wass}^UF(\rho_t)} \rho_t(x)dx.
\end{split}
\]
This proves that $(\rho_t,\; t\ge 0)$ is a weak solution of the continuity equation \eqref{eq:newgradflow}. 

Although, to the best of our knowledge, this particular mirror gradient flow on the Wasserstein space is new in the literature, some related ideas that involve modifications to the usual Wasserstein geometry and/or considering gradient flows on them can be found in other recent papers such as \cite{SVGD17,RankinWong23, WassersteinNewton}.  

\begin{remark}\label{rmk:hori-ver-corr}
    The paragraph around  \eqref{eq:inverse_derivative} gives a  geometric way to understand the connection between \eqref{eq:mirrorgradflow3} and 
    \eqref{eq:newgradflow}. Namely, if we consider the graph of $\nabla u_t$ (the mirror map) in the product space $\mathbb{R}^d \times \mathbb{R}^d$, the vector field $\nabla_{\wass} F(\rho_t)$ gives the vertical variation of the graph, while the vector field $-\nabla_{\wass}^U F(\rho_t)=-  \left(\nabla^2 u \right)^{-1} \nabla_{\wass} F(\rho)$ gives its horizontal counterpart, which gives the same variation of the graph. The equation \eqref{eq:inverse_derivative} gives the precise relation between the horizontal and vertical variation.
    Note that the distribution $\rho_t$ is the result of the drift $\nabla_{\wass}^U F(\rho_t)$ via the continuity equation. The difference between the mirror flow and the usual Wasserstein gradient flow, is that the vector field $\nabla_{\wass} F(\rho_t)$ provides variation on $\rho_t$ not directly in the continuity equation but indirectly through the aforementioned vertical-horizontal correspondence, which in turn is provided by the mirror map. 
\end{remark}


\subsection{Parabolic Monge-Amp\`{e}re and the mirror gradient flow of Kullback-Leibler divergence}\label{sec:pmasec}



 %, especially for the general cost functions. } % and further studied by \cite{kitagawa2012parabolic} for domains with boundary.
The parabolic Monge-Amp\`{e}re (PMA) is the following partial differential equation (PDE): 
\begin{equation}\label{eq:pma}
    \frac{\partial u_t}{\partial t\hfill} (x)=f(x)-g(x^{u_t})+\ldet\left( \frac{\partial x^{u_t}}{\partial x\hfill} \right),
\end{equation}
for some initial function $u_0:\R^d\to\R$. The idea is that, as $t\rightarrow \infty$, the LHS of \eqref{eq:pma} should converge to zero, whereby $u_t$ should converge to the solution of the Monge-Amp\`{e}re equation for transporting $e^{-f}$ to $e^{-g}$. Although the PMA has been around in the literature, there has been some recent appearances in the context of optimal transport. For example,  \cite{kim2012parabolic} studied it for general cost functions on closed manifolds, while 
\cite{kitagawa2012parabolic} studies it for bounded domains. On the other hand, \cite{deb2025no} recently used it for generative modeling and variational inference. 



There is a more convenient form for us that we will utilize frequently. This needs the well-known change of measure lemma (see \cite[Theorem 1.6.9]{durrettprob}) which we note here for easy reference.
%\marginpar{ Probably this lemma should be stated before \eqref{eq:pma2}?}
\begin{lmm}\label{lem:jacobian}(Change of measure)
    Let $e^{-a}$ be a probability density function on $\R^d$ for some function $a:\R^d\rightarrow \R$. For a $\diffcont^2$ strictly convex function $\phi:\R^d\rightarrow \R$, let $e^{-b}$ be the pushforward $(\nabla \phi)_{\#} e^{-a}$ Then,
    \begin{equation}\label{eq:jacobian}
    b(x^\phi) = a(x) + \ldet \frac{\partial x^\phi}{\partial x\hfill}.
    \end{equation}
\end{lmm}


Now, let $h_t:\R^d\rightarrow \R$ be such that $e^{-h_t}$ is the pushforward of the measure $e^{-g}$ by the inverse of the map $x\mapsto x^{u_t}$. Then, by Lemma \ref{lem:jacobian}, it follows that \eqref{eq:pma} can be alternatively written as 
\begin{equation}\label{eq:pma2}
\frac{\partial u_t}{\partial t}(x)= f(x) - h_t(x) = \log \left( \frac{e^{-h_t(x)}}{e^{-f}}. \right).
\end{equation}

Consider the function on the Wasserstein space 
\begin{equation}\label{eq:choiceofF}
F(\rho):=\begin{cases}
\KL{\rho}{e^{-f}},& \text{when $\rho$ is dominated by {\color{red}$e^{-f}$}}, \\
+\infty,& \text{otherwise}.  
\end{cases}
\end{equation}
When $f$ is convex, it is well-known that $F$ is geodesically (and generalized geodesically) convex and lower semicontinuous. 

The first variation of $F$ (see \cite[Lemma 10.4.1]{ambrosio2005gradient}) at $\rho_t=e^{-h_t}$ is given by $\frac{\delta F}{\delta \rho_t}= f(x) - h_t(x)$, 
(ignoring an additive constant which does not affect its gradient $\nabla_{\wass} F = \nabla_x  \frac{\delta F}{\delta \rho}$).
Thus, \eqref{eq:pma2} may be written as $\dot{u}_t=\frac{\delta F}{\delta \rho_t}$ which is exactly \eqref{eq:mirrorgradflow3} and, by taking a gradient on both sides, we recover  \eqref{eq:mirrorgradflow2}. In our language, \textit{the PMA is the evolution of the mirror potential for the mirror gradient flow of relative entropy with respect to $e^{-f}$, where the mirror is generated by the function in \eqref{eq:mirror}}. 


By \eqref{eq:newgradflow} and \eqref{eq:expgrad}, the mirror gradient flow itself satisfies the continuity equation 
 \begin{equation*}
 \partial_t \et+\div{(\et v_t)}=0, \quad v_t(x)=-\frac{\partial\hfill}{\partial x^{u_t}} (f-h_t)(x),
 \end{equation*}
 where $(u_t,\; t\ge 0)$ solves the PMA \eqref{eq:pma}. This is the same equation we introduced in \eqref{eq:velocity}. We call $(\rho_t,\; t\ge 0)$ the Sinkhorn flow, and we will show that this is the limit of iterates of the Sinkhorn algorithm. Note that, alternatively, from \eqref{eq:pma2}, $u_t$ may also be expressed as
 \[
 u_t(x)= u_0(x) + \int_0^t (f(x) + \log \rho_s(x))ds, \quad t\ge 0.
 \]

\noindent Let us now formalize the existence of a solution to \eqref{eq:velocity}. To state this, we need some assumptions on the solution of the PMA \eqref{eq:pma}. 
\begin{assm}\label{asn:solcon}
Assume that $f,g,u_0$ are such that the PMA \eqref{eq:pma} admits a solution $\left( u_t,\; t\ge 0 \right)$. Additionally,
\begin{enumerate}[(i)]
\item Given any $T>0$, there exists constants $A_T>0$ and $B_T>0$ such that 
\begin{align}\label{eq:curvbd}
\inf_x \inf_{t\in [0,T]}\lmn\left(\frac{\partial x^{u_t}}{\partial x\hfill}\right)\geq A_T,\quad \sup_x \sup_{t\in [0,T]} \lmx\left(\frac{\partial x^{u_t}}{\partial x\hfill}\right)\le B_T.
\end{align}
\item The map $(t,x)\mapsto u_t(x)$  lies in $\diffcont^{1,2}([0,\infty)\times \R^d)$\footnote{The reader is warned that the  $\diffcont^{k,\ell}$ notation is not to be confused with the H\"{o}lder class of functions}. Further, for every $T>0$, all the corresponding mixed partial derivatives (in space and time) are bounded on $[0,T]\times \R^d$.
%\item The map $(t,y)\mapsto y^{u_t^*}$ (where $u_t^*$ is the convex conjugate of $u_t$) lies in $\diffcont^{1,2}([0,\infty)\times \R^d)$ (where the function $y^{u_t^*}$ is viewed componentwise). 
%\item (Moment assumptions) Let $Y\sim \exp(-g)$ and assume that for all $t,T>0$,
%$$\E\left[\bigg\lVert \nabla\frac{\partial u_t}{\partial t\hfill}(Y^{u_t^*})\bigg\rVert^2\lmx^2\left(\frac{\partial Y^{u_t^*}}{\partial Y\hfill}\right)\right]<\infty,$$
%$$\sup_{t\in [0,T]} \E\left(\frac{\partial u_t}{\partial t\hfill}(Y^{u_t^*})\right)^2<\infty.$$
%{\color{red} Circle back to check if (iii) is now needed.}
\item The first two derivatives of $f(\cdot)$ and $g(\cdot)$ are bounded and uniformly continuous. Further, the first four spatial derivatives of $\{u_t\}_{t\in [0,T]}$ and $\{u_t^*\}_{t\in [0,T]}$ are bounded and uniformly continuous, for all $T>0$.
 \end{enumerate}
\end{assm}

Given the solution of the PMA $(u_t,\; t\ge 0)$ (as in \eqref{eq:pma}), note that $\left(w_t=u_t^*,\; t\ge 0 \right)$ denotes the corresponding process of convex conjugates. By \cref{asn:solcon}, part (i), both $u_t$ and $w_t$'s are all strictly convex $\diffcont^2$ functions. Therefore, $\nabla u_t(\cdot)$ and $\nabla w_t(\cdot)$ are both diffeomorphisms on $\R^d$. In view of~\cref{def:pdcor}, we then have a system of dual coordinates on $\R^d$ given by $x\mapsto \xsut$ (or equivalently $y\mapsto y^{w_t}$), one for each $t\geq 0$. We show in Lemma \ref{lem:dualPMA} that the family $(w_t)_{t\ge 0}$ is also a solution of a PMA that we call the \text{dual PMA}. The following observation is an immediate consequence of \cref{asn:solcon} which we note as a remark below. 

\begin{remark}\label{rem:dualasn}
 If \cref{asn:solcon} holds for the solution of the PMA $(u_t)_{t\ge 0}$, then  the assumptions (i)--(iii)  also hold for $(w_t=u_t^*)_{t\ge 0}$, as $\nabla u_t = (\nabla w_t)^{-1}$, and vice versa. This is relevant for \cref{lem:dualPMA} below.
\end{remark}
  %\marginpar{I got stuck at this remark, what PME do we have for $w_t$? I understand $w_t$ corresponds to some flow, but, not clearly seeing its PME.. {\color{red} Wouldn't \eqref{eq:dualPMA} work?}} 


\begin{remark}\label{rem:bermanver}
Sufficient conditions for \cref{asn:solcon} have been studied in the literature under different assumptions on the supports of the probability measures $e^{-f}$ and $e^{-g}$, and the initializer $u_0$ for \eqref{eq:pma}. For example, in \cite[Proposition 4.5]{berman2020}, that if $u_0$ is four times continuously  differentiable, $f$ and $g$ are twice continuously differentiable, and $e^{-f}$, $e^{-g}$ are supported on the torus, then \cref{asn:solcon} is satisfied.  Similar results have also been established for compact Riemannian manifolds (see \cite[Theorem 1.1]{kim2012parabolic}), and general bounded convex domains (see \cite[Theorem 3.1]{kitagawa2012parabolic}, \cite[Theorem A]{Tang2013}, and \cite[Theorem 1.1]{Abedin2020}. A simple case when \cref{asn:solcon} holds for probability measures supported on $\R^d$ is where both $\mu$ and $\nu$ are Gaussian probability densities (see Examples \ref{ex:loc} and \ref{ex:scale}).

The most critical part of \cref{asn:solcon} is the curvature bound in \eqref{eq:curvbd}. When the probability measures are supported on $\R^d$, in a recent paper \cite[Lemma 2.2]{chiarini2024semiconcavity}, the authors show that the Sinkhorn potentials (see \eqref{eq:twostepit}) satisfy the curvature condition \eqref{eq:curvbd}, uniformly for all small enough $\vep$, provided that the Hessians of $f$, $g$ are bounded above and below by positive constants times identity, and both the initial potentials are strongly convex. Given the connections between the Sinkhorn potentials and the PMA \eqref{eq:pma} established in \cite[Lemma 4.4]{berman2020}, this suggests that the PMA \eqref{eq:pma} should also satisfy \eqref{eq:curvbd} under the additional log-concavity assumptions. A rigorous proof of this is left for future research.

\end{remark}


We are now in position to state our first main result which features the existence of a strong solution to \eqref{eq:velocity}. \begin{thm}\label{thm:existlin}
Fix $T>0$ and suppose that \cref{asn:solcon} holds. Recall $w_t=u_t^*$ from Remark \ref{rem:dualasn}. Consider the push-forward $\rho_t=(\nabla w_t)_{\#}e^{-g}$. Then $(\rho_t)_{t\in [0,T]}$ is an absolutely continuous curve in the $2$-Wasserstein space and it is a strong solution to \eqref{eq:velocity}. 
\end{thm}

The proof of \cref{thm:existlin} depends on the following change of coordinate lemma that will be used multiple times in this paper.

\begin{lmm}\label{lem:tensorel}
    Suppose that $\phi:\R^d \to\R$ is strictly convex such that the function $\nabla \phi: \R^d \rightarrow \R^d$ is a $\diffcont^2$ diffeomorphism. That is both  $\nabla \phi$ and its inverse $\nabla \phi^*$ are twice continuously differentiable. Then, 
    \begin{equation}\label{eq:tensorelpf}
    \frac{\partial}{\partial \xsph_j}\left(\log\det\left(\frac{\partial x^\phi}{\partial x\hfill}\right)\right)=-\sum_{\ell=1}^d \frac{\partial^2 x_j}{\partial x_{\ell}\partial \xsph_{\ell}},
    \end{equation}
    for all $j\in [d]$ and $x\in\R^d$.
\end{lmm}

The proof of \cref{lem:tensorel} is deferred to~\cref{sec:pfres}. 

\begin{proof}[Proof of \cref{thm:existlin}]
By invoking \cref{lem:jacobian} with $\phi=w_t$, $a=g$, we observe that $\rho_t$ is strictly positive and 
 $h_t:=-\log \rho_t$
 is well-defined. 
 Observe that 
 \begin{align}\label{eq:com1}
 h_t(x)=g(x^{u_t})-\ldet\left(\frac{\partial x^{u_t}}{\partial x\hfill}\right)=f-\frac{\partial u_t}{\partial t\hfill}
 \end{align}
 where the first equality is by \eqref{eq:jacobian} and the second equality is by \eqref{eq:pma}. 
 
Now starting with the first equality above along with the chain rule, we get:
\begin{align*}
\frac{\partial h_t}{\partial t\hfill}(x)\nonumber &=\frac{\partial}{\partial t}\left(g(x^{u_t})-\ldet\left(\frac{\partial x^{u_t}}{\partial x\hfill}\right)\right)\nonumber \\ &=\bigg\langle \frac{\partial x^{u_t}}{\partial t\hfill},\frac{\partial g\hfill}{\partial x^{u_t}} (x^{u_t})\bigg\rangle - \frac{\partial}{\partial t}\left(\ldet\left(\frac{\partial x^{u_t}}{\partial x\hfill}\right)\right)
\end{align*}
For the second term, we will use \cite[Section A.4.1]{boyd2004convex} to note the fact that the derivative of $\ldet(A)$ with respect to the entries of $A$ is given by $A^{-1}$. By taking $A=\frac{\partial x^{u_t}}{\partial x\hfill}$ (therefore $A^{-1}=\frac{\partial x\hfill}{\partial x^{u_t}}$) and applying the chain rule again, we get
\begin{align}\label{eq:rhot1}
    -\frac{1}{\rho_t(x)}\frac{\partial \rho_t}{\partial t}(x)=\frac{\partial h_t}{\partial t\hfill}(x)=\bigg\langle \frac{\partial x^{u_t}}{\partial t\hfill},\frac{\partial g\hfill}{\partial x^{u_t}} (x^{u_t})\bigg\rangle-\bigg\langle \frac{\partial x\hfill}{\partial x^{u_t}}, \frac{\partial}{\partial t}\left(\frac{\partial x^{u_t}}{\partial x\hfill}\right)\bigg\rangle.\end{align}
%\nonumber \\ &=\bigg\langle \frac{\partial x^{u_t}}{\partial t\hfill},-\frac{\partial g}{\partial x} (x^{u_t})\bigg\rangle + \sum_{i,j} \left(\frac{\partial x}{\partial x^{u_t}}\right)_{i,j} \frac{\partial}{\partial x}\left(\frac{\partial x^{u_t}}{\partial t\hfill}\right)_{i,j}.

Take 
\begin{align}\label{eq:vtalt}
v_t=\frac{\partial\hfill}{\partial x^{u_t}}((h_t-f)(x))=-\frac{\partial\hfill}{\partial x^{u_t}}\left(\frac{\partial u_t}{\partial t\hfill}(x)\right).
\end{align}
The first equality is the definition of $v_t$ as in  \eqref{eq:velocity}, and the second equality is the PMA \eqref{eq:pma}. Then by the product rule we have:
\begin{align}\label{eq:rhot2}
&\frac{\div(v_t\rho_t)}{\rho_t}(x)\nonumber =\langle v_t(x),\nabla \log{\rho_t(x)}\rangle + \div v_t(x)\nonumber \\ 
&=\bigg\langle -\frac{\partial}{\partial x^{u_t}}\left(\frac{\partial u_t}{\partial t\hfill}(x)\right),-\frac{\partial h_t}{\partial x\hfill}(x)\bigg\rangle - \div{\left(\frac{\partial}{\partial x^{u_t}}\left(\frac{\partial u_t}{\partial t\hfill}(x)\right)\right)},
\end{align}
where the last line uses \eqref{eq:vtalt}. We will now simplify each of the terms above. For the first term, observe that:

\begin{align*}
    \bigg\langle -\frac{\partial}{\partial x^{u_t}}\left(\frac{\partial u_t}{\partial t\hfill}(x)\right),-\frac{\partial h_t}{\partial x\hfill}(x)\bigg\rangle &=\bigg\langle \frac{\partial}{\partial x}\frac{\partial u_t}{\partial t\hfill}(x)\left(\frac{\partial x}{\partial x^{u_t}\hfill}\right),\frac{\partial h_t}{\partial x}(x)\bigg\rangle\\ &=\bigg\langle \frac{\partial x^{u_t}}{\partial t\hfill}, \left(\frac{\partial x}{\partial x^{u_t}\hfill}\right)\frac{\partial h_t}{\partial x}(x)\bigg\rangle =\bigg\langle \frac{\partial x^{u_t}}{\partial t\hfill},\frac{\partial h_t\hfill}{\partial x^{u_t}}(x)\bigg\rangle.
\end{align*}
The first and third equalities above use the chain rule (see \eqref{eq:notcal} for clarity of notation). For the second equality, we have interchanged the time and the space derivatives in the term $\frac{\partial}{\partial x}\frac{\partial u_t}{\partial t\hfill}(x)$ using \cref{asn:solcon}, part (ii). We now move on to the second term of \eqref{eq:rhot2}. Note that
\begin{align*}
    \div{\left(\frac{\partial}{\partial x^{u_t}}\left(\frac{\partial u_t}{\partial t\hfill}(x)\right)\right)}&=\div{\left(\frac{\partial x^{u_t}}{\partial t}\left(\frac{\partial x}{\partial x^{u_t}}\right)\right)}\\ &=\sum_{i,j=1}^d \frac{\partial}{\partial x_i}\left(\left(\frac{\partial x^{u_t}}{\partial t\hfill}\right)_j\left(\frac{\partial x\hfill}{\partial x^{u_t}}\right)_{j,i}\right).
\end{align*}
In the first equality, we have used the chain rule as before and the second display uses the definition of divergence. We will now  use the product rule to obtain 
\begin{align*}
    &\;\;\;\;\sum_{i,j=1}^d \frac{\partial}{\partial x_i}\left(\left(\frac{\partial x^{u_t}}{\partial t\hfill}\right)_j\left(\frac{\partial x\hfill}{\partial x^{u_t}}\right)_{j,i}\right)\\ &=\bigg\langle \frac{\partial}{\partial t}\left(\frac{\partial x^{u_t}}{\partial x\hfill}\right),\frac{\partial x\hfill}{\partial x^{u_t}}\bigg\rangle+\sum_{j=1}^d \left(\frac{\partial x^{u_t}}{\partial t\hfill}\right)_j\left(\sum_{i=1}^d \frac{\partial^2 x_i}{\partial x_i\partial x_i^{u_t}}\right)\\ &=\bigg\langle \frac{\partial x\hfill}{\partial x^{u_t}},\frac{\partial}{\partial t}\left(\frac{\partial x^{u_t}}{\partial x\hfill}\right)\bigg\rangle-\bigg\langle \left(\frac{\partial x^{u_t}}{\partial t\hfill}\right),\frac{\partial}{\partial x^{u_t}}\left(\ldet\left(\frac{\partial x^{u_t}}{\partial x\hfill}\right)\right)\bigg\rangle.
\end{align*}
In the second inequality above, we have used \eqref{eq:tensorelpf} on the second term. We now combine our observations on the two terms of \eqref{eq:rhot2} to get: 
\begin{small}
\begin{align*}
    \frac{\div(v_t\rho_t)}{\rho_t}(x)&=\bigg\langle \left(\frac{\partial x^{u_t}}{\partial t\hfill}\right),\frac{\partial}{\partial x^{u_t}}\left(h_t(x)+\ldet\left(\frac{\partial x^{u_t}}{\partial x\hfill}\right)\right)\bigg\rangle-\bigg\langle \frac{\partial x\hfill}{\partial x^{u_t}},\frac{\partial}{\partial t}\left(\frac{\partial x^{u_t}}{\partial x\hfill}\right)\bigg\rangle.
\end{align*}
\end{small}
Observe that, by applying $\frac{\partial}{\partial x_j^{u_t}}$ on both sides of \eqref{eq:com1} we get 
\begin{align*}
    \frac{\partial}{\partial x^{u_t}}\left(h_t(x)+\ldet\left(\frac{\partial x^{u_t}}{\partial x\hfill}\right)\right)=\frac{\partial g\hfill}{\partial x^{u_t}}(x^{u_t}).
\end{align*}
Therefore, 
\begin{align}\label{eq:rhot3}
\frac{\div(v_t\rho_t)}{\rho_t}(x)=\bigg\langle \left(\frac{\partial x^{u_t}}{\partial t\hfill}\right),\frac{\partial g\hfill}{\partial x^{u_t}}(x^{u_t})\bigg\rangle-\bigg\langle \frac{\partial x\hfill}{\partial x^{u_t}},\frac{\partial}{\partial t}\left(\frac{\partial x^{u_t}}{\partial x\hfill}\right)\bigg\rangle.
\end{align}

By adding \eqref{eq:rhot1} and \eqref{eq:rhot3}, it follows that $\frac{\partial \rho_t}{\partial t\hfill}+\div{(\rho_t v_t)}=0$.

\noindent Next, we establish absolute continuity of $(\rho_t)_{t\in [0,T]}$. By invoking \cite[Theorem 8.3.1]{ambrosio2005gradient}, it suffices to show that $\sup_{x,t\in [0,T]} \lVert v_t(x)\rVert<\infty$. By the representation of $v_t$ in \eqref{eq:vtalt} and the PMA \eqref{eq:pma}, we observe that for $t\in [0,T]$,
\begin{align*}
\lVert v_t(x)\rVert&=\left\lVert \left(\frac{\partial x}{\partial x^{u_t}}\right)\frac{\partial\hfill}{\partial x}\left(f(x)-g(x^{u_t})+\ldet\left(\frac{\partial x^{u_t}}{\partial x\hfill}\right)\right)\right\rVert\\ &\le B_T \left\lVert \frac{\partial\hfill}{\partial x}\left(f(x)-g(x^{u_t})+\ldet\left(\frac{\partial x^{u_t}}{\partial x\hfill}\right)\right) \right\rVert.
\end{align*}
Here in the first equality, we have again used the chain rule and in the following inequality, we have used the upper bound in \cref{asn:solcon}, part (i). Next, note that by \cref{asn:solcon}, parts (i) and (iii), we have:
$$\sup_x \ \left\lVert \frac{\partial f}{\partial x}(x)\right\rVert<\infty, \quad \sup_x \left\lVert \frac{\partial\hfill}{\partial x}(g(x^{u_t}))\right\rVert\le \sup_x \left\lVert \frac{\partial g\hfill}{\partial x^{u_t}}(x^{u_t})\right\rVert \ \sup_x \left\lVert \frac{\partial x^{u_t}}{\partial x\hfill}\right\rVert<\infty.$$
Therefore, to prove $\sup_{x,t\in [0,T]} \lVert v_t(x)\rVert$ is bounded, it suffices to control the norm of $\frac{\partial}{\partial x}\left(\ldet\left(\frac{\partial x^{u_t}}{\partial x\hfill}\right)\right)$. To wit, we will once again use \cite[Section A.4.1]{boyd2004convex} (as we did earlier while obtaining \eqref{eq:rhot1} above). For $i\in [d]$, we then get
\begin{align*}
    \frac{\partial}{\partial x}\left(\ldet\left(\frac{\partial x^{u_t}}{\partial x\hfill}\right)\right)=\bigg\langle \frac{\partial x\hfill}{\partial x^{u_t}},\frac{\partial\hfill}{\partial x_i}\left(\frac{\partial x^{u_t}}{\partial x\hfill}\right)\bigg\rangle.
\end{align*}
Now observe that the entries of $\frac{\partial x\hfill}{\partial x^{u_t}}$ are uniformly bounded (in $x$ and $t\in [0,T]$) in terms of $A_T$ by using \cref{asn:solcon}, part (i). Also the entries of $\frac{\partial}{\partial x_i}\left(\frac{\partial x^{u_t}}{\partial x\hfill}\right)$ are uniformly bounded by \cref{asn:solcon}, part (iii), where we have assumed boundedness of the third derivative tensor of $u_t$'s. Therefore, 
$$\sup_{x,t\in [0,T]} \left\lVert \frac{\partial}{\partial x}\left(\ldet\left(\frac{\partial x^{u_t}}{\partial x\hfill}\right)\right)\right\rVert<\infty.$$
This readily implies $\sup_{x,t\in [0,T]} \lVert v_t(x)\rVert<\infty$ and completes the proof.
\end{proof}

The velocity field $(v_t)$ appearing in the Sinkhorn flow \eqref{eq:velocity} is not a gradient in $x$. Thus, it may not lie in the $2$-Wasserstein tangent space at $\rho_t$, and the metric derivative (as defined in \cite[Theorem 1.1.2]{ambrosio2005gradient}) of the curve at time $t$ may not be $\norm{v_t}_{L^2(\rho_t)}$, which is the case for usual gradient flows in an appropriate sense (see \cite[Theorem 8.3.1]{ambrosio2005gradient}). 

However, as we show below, if we replace the $2$-Wasserstein distance (see \eqref{eq:2wass}) with the $\dlt{}{e^{-g}}{\cdot}{\cdot}$ metric (see \eqref{eq:2linot}), the metric derivative of the curve is indeed given by $\norm{v_t}_{L^2(\rho_t)}$. 

\begin{thm}\label{prop:metderlin}
Suppose \cref{asn:solcon} holds. Then with $\rho_t$ as in \cref{thm:existlin}, for any $t\ge 0$, we have: 
\begin{align}\label{eq:linot}
\lim_{\delta\to 0} \frac{1}{\delta}\dlt{}{e^{-g}}{\rho_{t+\delta}}{\rho_t}=\lVert v_t\rVert_{L^2(\rho_t)},
\end{align}
and 
\begin{align}\label{eq:linotsec}
    \dlt{}{e^{-g}}{\rho_{t+\delta}}{(\nabla w_t+\delta v_t(\nabla w_t))_{\#} e^{-g}}=o(\delta).
\end{align}
\end{thm}

The conclusion in \eqref{eq:linotsec} shows that if we appropriately perturb $\nabla w_t$ in the direction of the velocity $v_t$, then the corresponding push-forward measure $(\nabla w_t+\delta v_t(\nabla w_t))_{\#} e^{-g}$ yields a first order approximation to $\rho_{t+\delta}$. Such results are very popular in the literature on continuity equations and usual gradient flows.  Similar results for Euclidean gradient flows imply that, following the tangent vector of a smooth curve $\{x_t\}$ at a point $t$ over a small timestep $\delta$ is ``close" up to the first order (in the Euclidean metric) to $x_{t+\delta}$. Alternatively, it means that the trajectory of a particle moving along a smooth curve can be approximated by piecewise constant velocity curves. This is also true for usual Wasserstein gradient flows (see \cite[Proposition 8.4.6]{ambrosio2005gradient}). There the authors show that if $(\rho_t)_{t\ge 0}$ satisfies a continuity equation with velocity $v_t$ which is a gradient in $x$ (not the case in our setting), then $\wass_2(\rho_{t+\delta},(\mathrm{Id}+\delta v_t)_{\#}\rho_t)=o(\delta)$. \cref{prop:metderlin} establishes a similar result for Wasserstein mirror flows with the Hilbertian linear optimal transport distance (see \eqref{eq:2linot}). Note that the Wasserstein distance is \emph{smaller than} the linear optimal transport distance. In this sense, the approximation of $\rho_{t+\delta}$ in \eqref{eq:linotsec} is stronger than that in \cite[Proposition 8.4.6]{ambrosio2005gradient} (albeit under stronger regularity conditions).


In order to prove \cref{prop:metderlin} we need some additional results. First is an important conjugacy relationship for the PMA. 
Remark \ref{rmk:hori-ver-corr} suggests that there is a dual flow because one can flip the choice of the horizontal and vertical directions. Indeed, recall that the potential $u_t$ in \eqref{eq:KPtoBP} gives the flow $\rho_t= (\nabla u_t)^{-1}_\# e^{-g}$. We can consider the family of convex conjugates $w_t=u_t^*$, whereby $\nabla w_t  = (\nabla u_t)^{-1}$. Viewed as vector fields the time derivative $\partial_t \nabla u_t$
    is equivalent to the vertical variation $\nabla_{\wass} F(\rho_t)$ while 
    $\partial_t \nabla w_t$  is equivalent to   the horizontal variation
   $- \nabla_{\wass}^U F(\rho_t).$
   By \eqref{eq:pma}, $(u_t)$ solves a PMA. We show in Lemma \ref{lem:dualPMA} that $(w_t)$ also solves a (different) PMA. This conjugacy is also evident from the Sinkhorn algorithm which comes with a pair of potentials at each step. The PMA \eqref{eq:pma} is the limit (in $\vep$) of one of these sequence of potentials. So, by symmetry, it is only natural that the other sequence of potentials has a PMA limit as well and the corresponding potentials in the two limiting PMAs are related by convex duality. 
  
   
   %One may understand the flow of $w_t$ as a mirror flow where the mirror potential is given by the (time-dependent) function 
   %$W_t(\cdot) =\frac{1}{2} \wass_2^2 ( \cdot, \rho_t)$.  
    
   %{\color{blue}[YH. I am not completely satisfied the latter half of this paragraph. ]
%\end{remark}

%\marginpar{\red
%[YH. As in Lemma~\ref{lem:dualPMA}, the mirror flow has its conjugate. Probably we want to say something about it in this subsection too? I tried it here. 
%Simply, for the conjugate $w$, \[\left(\nabla^2 w \right)^{-1} \nabla_{\wass}^U F(\rho_t)=   \nabla_{\wass} F(\rho)\]
%]


\begin{lmm}\label{lem:dualPMA}
Suppose \cref{asn:solcon} holds. Then the process $\left( w_t=u_t^*,\; t\ge 0 \right)$  is also the solution of a PMA:
\begin{equation}\label{eq:dualPMA}
\frac{\partial w_t}{\partial t}(y)= g(y) - f(y^{w_t}) + \ldet \left( \frac{\partial y^{w_t}}{\partial y\hfill}\right),
\end{equation}
with the initial condition $w_0=u_0^*$. 
\end{lmm}        

The next result is a property of the velocity field $v_t$ in \eqref{eq:velocity}.

\begin{lmm}\label{lem:convexcall}
Suppose \cref{asn:solcon} holds. Then, for $\delta>0$ small enough, the function $y\mapsto y^{w_t}+\delta v_t(y^{w_t})$ is the gradient in $y$ of the convex function $y\mapsto w_t(y)-\delta (f(y^{w_t})-h_t(y^{w_t}))$.
\end{lmm}

\begin{proof}[Proof of \cref{prop:metderlin}]

We first prove \eqref{eq:linot}. As $w_t=u_t^*$ satisfies $(\nabla w_t)_{\#} e^{-g}=\rho_t=e^{-h_t}$, we have:
    \begin{align}\label{eq:simlin}
    \frac{1}{\delta^2}\dlt{2}{e^{-g}}{\rho_{t+\delta}}{\rho_t}&=\int \lVert \delta^{-1}(\nabla w_{t+\delta}(y)-\nabla w_t(y))\rVert^2 e^{-g(y)}\,dy.
    \end{align}
    By \cref{asn:solcon} (also see \cref{rem:dualasn}), we note that
    $$\sup_y \bigg|\delta^{-1}(\nabla w_{t+\delta}(y)-\nabla w_t(y))-\nabla \frac{\partial w_t}{\partial t\hfill}(y)\bigg|\le \sup_{y,s\in [t,t+\delta]}\bigg|\nabla \frac{\partial w_s}{\partial s\hfill}(y)-\nabla \frac{\partial w_t}{\partial t\hfill}(y)\bigg|=o(1).$$
    By combining the above display with \eqref{eq:dualPMA}, we get:
    \begin{align*}
        \sup_y \bigg|\delta^{-1}(\nabla w_{t+\delta}(y)-\nabla w_t(y))- \nabla\left(g(y)-f(y^{w_t})+\ldet\left(\frac{\partial y^{w_t}}{\partial y\hfill}\right)\right)\bigg|=o(1).
    \end{align*}
    Here $o(1)$ is with respect to $\delta\to 0$. Next we use the first equality in \eqref{eq:vtalt} to observe that 
    \begin{align}\label{eq:labelgrad}
        v_t(y^{w_t})=\frac{\partial}{\partial y}((h_t-f)(y^{w_t}))=\frac{\partial}{\partial y}\left(g(y)-f(y^{w_t})+\ldet\left(\frac{\partial y^{w_t}}{\partial y\hfill}\right)\right).
    \end{align}
    In the second equality, we have used \eqref{eq:com1}. By combining the two displays above, we get:
    \begin{align}\label{eq:vtcall2}
        \sup_y \bigg|\delta^{-1}(\nabla w_{t+\delta}(y)-\nabla w_t(y))- v_t(y^{w_t})\bigg|=o(1).
    \end{align}
    By combining the above display with \eqref{eq:simlin}, we get
    \begin{align*}
    \frac{1}{\delta^2}\dlt{2}{e^{-g}}{\rho_{t+\delta}}{\rho_t}&=\int \lVert v_t(y^{w_t})\rVert^2 e^{-g(y)}\,dy+o(1).
    \end{align*}
    A final change of variable with $x=y^{w_t}$ establishes \eqref{eq:linot}. 

\vspace{0.05in}

We next prove \eqref{eq:linotsec}. By \cref{lem:convexcall}, there exists $\delta>0$ small enough such that the function $y\mapsto y^{w_t}+\delta v_t(y^{w_t})$ is the gradient of a convex function. With such $\delta>0$, by McCann's Theorem (see \cite{McCann1995}), $y^{w_t}+\delta v_t(y^{w_t})$ is the optimal transport map from $e^{-g}$ to $(y^{w_t}+\delta v_t(y^{w_t}))_{\#} e^{-g}$. As $y^{w_{t+\delta}}$ is the optimal transport map from $e^{-g}$ to $\rho_{t+\delta}$. With these observations, we get:
\begin{align*}
    &\;\;\;\;\dlt{}{e^{-g}}{\rho_{t+\delta}}{(\nabla w_t+\delta v_t(\nabla w_t))_{\#} e^{-g}}\\ &=\left(\int \lVert \nabla w_{t+\delta}(y)-\nabla w_t(y)-\delta v_t(y^{w_t})\rVert^2 e^{-g(y)}\,dy\right)^{\frac{1}{2}}=o(\delta), 
\end{align*}
where the last equality follows from \eqref{eq:vtcall2}. This proves \eqref{eq:linotsec}.
\end{proof}

\begin{remark}[Mirror descent for fixed $\vep>0$]\label{rem:Flavian}
In \cite[Theorem 2]{leger2021gradient}, the author showed that for fixed $\vep>0$, the marginals of the Sinkhorn algorithm \eqref{eq:sinkupdt} can be viewed as iterations of an explicit mirror descent algorithm. In particular, one can (informally) write 
\begin{align*}
\rho_{k+1}^{\vep}&=\argmin_{\mu\in\mathcal{P}_2(\R^d)}\bigg[\KL{\rho_{k}^{\vep}}{e^{-f}}+\bigg\langle \mu-\rho_{k}^{\vep}, \frac{\partial \hfill}{\partial \mu}\KL{\mu}{e^{-f}}\big|_{\mu=\rho_{k}^{\vep}}\bigg\rangle\\ &\quad\quad + F^*(\mu)-F^*(\rho_{k,\vep})-\bigg\langle \mu-\rho_{k}^{\vep},\frac{\partial \hfill}{\partial \mu}F^*(\mu)\bigg|_{\mu=\rho_{k}^{\vep}}\bigg\rangle,  
\end{align*}
where $F^*$ is the convex conjugate of the function $F$ defined by 
$$F(\phi)=\langle \opV[\phi],e^{-g}\rangle,$$
for ``smooth" functions $\phi:\R^d\to\R^d$ and $\opV$ defined in \eqref{eq:basedef} later. 

While \cite{leger2021gradient} focuses on the $\vep>0$ case, we take the limiting perspective ($\vep\to 0$) which allows us to explicitly identify the mirror function (see \eqref{eq:mirror}) and velocity vector field of the limiting (Sinkhorn) PDE \eqref{eq:velocity}. To our understanding, these explicit quantities cannot be derived from the proof techniques used in \cite{leger2021gradient}.
\end{remark}


\subsection{Exponential convergence of the 
Sinkhorn flow}\label{sec:sinkflow}

As \eqref{eq:velocity} 
arises out of the mirror flow of the entropy functional $\KL{\rho}{e^{-f}}$, it is natural to ask the following: does $\rho_t\to e^{-f}$ as $t\to\infty$? If so, then in what sense and what is the speed of convergence? 

To address this, we will now establish the exponential convergence of the Sinkhorn flow to $e^{-f}$. To begin, let us define the log-Sobolev inequality.

\begin{defn}\label{def:isid}
    We say that a probability measure $\xi\in \ptac$ satisfies a logarithmic Sobolev inequality (LSI) with constant $\lsi{\xi}>0$ if for all $\rho\in\ptac$, 
    \[
    \KL{\rho}{\xi}\le \frac{1}{2\lsi{\xi}}I(\rho|\xi),
    \]
    where $I(\rho|\xi)$ is the \emph{relative Fisher information} defined by
    \[
    I(\rho|\xi):=\int \left\lVert \nabla\log\frac{d\rho}{d\xi}\right\rVert^2\,d\rho.
    \]
\end{defn}

\begin{thm}\label{lem:expcon}
    Suppose there exists $\lsi{f}>0$ such that $e^{-f}$ satisfies LSI with constant $\lsi{f}$. Also assume that there exists a positive continuous function $h(\cdot)$ on $[0,\infty)$ such that
    \begin{equation}\label{eq:uniflb}
    \inf_{x\in\R^d}\lmn\left(\frac{\partial x\hfill}{\partial x^{u_t}}\right)\ge h(t).
    \end{equation}
    Then we have
    $$\KL{\rho_t}{e^{-f}}\le \KL{\rho_0}{e^{-f}}\exp(-2\lsi{f}H(t)),$$
    and 
    $$\wass_2(\rho_t,e^{-f})\le \sqrt{\frac{2\KL{\rho_0}{e^{-f}}}{\lsi{f}}}\exp(-\lsi{f} H(t)),$$
    where $H(t):=\int_0^t h(s)\,ds$.
\end{thm}

%{\red YH. The condition \eqref{eq:uniflb}  depends on the mirror-gradient flow itself. So, without providing a condition where it holds the whole theorem may not look attractive. 
%We can remark the equivalence between the mirror-flow and the PME, and use Assumption \ref{asn:solcon}. On the other hand, under the same assumption the PME solution converges exponentially, so this theorem is nothing but a translation of it. }

%\SP{ Part of this is, of course, unavoidable. Any condition for the convergence of the flow will be, vice versa, a condition on the convergence of the PMA.}

%{\color{blue} ND: Also \cite{berman2020}, only mentions exponential convergence of $u_t$ to $u_{\infty}$ in uniform norm. It is not clear to me if exponential rates in terms of KL divergence for $\rho_t$ to $e^{-f}$ follow from that. Given that Sinkhorm is the mirror flow of KL, I think exponential rates for KL as provided by \cref{lem:expcon} is what we want.}

\begin{proof}[Proof of \cref{lem:expcon}]
    For convenience, we define $m:=\KL{\rho_0}{e^{-f}}$. By using the standard terminology of analysis in the $2$-Wasserstein space (see \cite[Chapters 9 and 10]{ambrosio2005gradient}), we note that the function $\rho\mapsto \KL{\rho}{e^{-f}}$ admits a subdifferential $\nabla \log\rho_t+\nabla f$.  
    %\marginpar{ [YH. Don't we need convexity of $f$ for such displacement convexity? For the proof we only need LSI.] } 
 This implies
    \begin{align*}
        \frac{d}{dt}\KL{\rho_t}{e^{-f}}&=\int \iprod{\nabla \log{\rho_t(x)}+\nabla f(x),v_t(x)}\,d\rho_t(x)\\ &=-\int \left(\nabla \log{\rho_t(x)}+\nabla f(x)\right)^{\top}\frac{\partial x\hfill}{\partial x^{u_t}\hfill}\left(\nabla \log{\rho_t(x)}+\nabla f(x)\right)\,d\rho_t(x).
    \end{align*}
    In the last display, we have used the representation of $v_t$ from \eqref{eq:velocity} coupled with the chain rule as illustrated in \eqref{eq:notcal}. Next, by using \eqref{eq:uniflb}, we get:
    \begin{align*}
        &\;\;\;\;-\int \left(\nabla \log{\rho_t(x)}+\nabla f(x)\right)^{\top}\frac{\partial x\hfill}{\partial x^{u_t}\hfill}\left(\nabla \log{\rho_t(x)}+\nabla f(x)\right)\,d\rho_t(x)\\ &\le -h(t)\int \lVert \nabla \log{\rho_t(x)}+\nabla f(x)\rVert^2\,d\rho_t(x)\\ &=-h(t) I(\rho_t|e^{-f})\le -2\lsi{f}h(t)\KL{\rho_t}{e^{-f}}
    \end{align*}
    In the final display here we have used the logarithmic Sobolev inequality (LSI) assumption, see \cref{def:isid}. 
    Next by invoking Gronwall's inequality (see \cite[Theorem II]{walter2012differential}), we get:
    $$\KL{\rho_t}{e^{-f}}\le \KL{\rho_0}{e^{-f}}\exp(-2\lsi{f}\int_0^t h(s)\,ds).$$
    Next we use the HWI inequality (see \cite[Theorem 1]{Otto2000}) to get:
    $$\wass_2^2(\rho_t,e^{-f})\le \frac{2}{\lsi{f}}\KL{\rho_t}{e^{-f}}\le \frac{2}{\lsi{f}}\KL{\rho_0}{e^{-f}}\exp(-2\lsi{f}\int_0^t h(s)\,ds).$$
    This completes the proof.
\end{proof}

The LSI assumption (see~\cref{def:isid}) is standard in the literature on rates of convergence of flows and plays a pivotal role in establishing information geometric inequalities; see \cite{Anton2001,Otto2000} and the references therein. In particular, the LSI condition can be verified in many popular examples. We cite two of them below.

\begin{enumerate}
    \item If $\inf_{x\in\R^d}\lmn(\nabla^2 f(x))\ge c$ for some constant $c>0$, then $e^{-f}$ satisfies LSI with constant $c>0$ (see \cite{Bakry1985}).
    \item Suppose $e^{-\tilde{f}}$ satisfies LSI with constant $\tilde{c}$. Let $\bar{f}:=f-\tilde{f}$ and assume $\bar{f}\in L^{\infty}(\R^d)$. Then $e^{-f}$ satisfies LSI with constant $c:=\tilde{c}\exp(\inf \bar{f}-\sup \bar{f})$ (see \cite{Holley1987}).
\end{enumerate}
We refer the reader to \cite{Cattiaux2010,Chen2021,Wang2001} for other conditions under which  the LSI condition can be established.

%\SP{The following set of assumptions should be removed from here and stated before the main theorems.} \ND{In \cref{sec:mcconst}. We might need to use some assumption here that guarantees $\nabla u_t$'s are diffeomorphisms.}

\subsection{Other examples of Wasserstein mirror gradient flows}\label{sec:othexamp}

It is natural to be curious about Wasserstein mirror gradient flows with other choices of function $F$. We give a few examples below. The reader should be careful that the PDEs and flows that we calculate have not been shown to exist or be well-behaved. 
\medskip

%\noindent\textbf{Example 1: Potential energy.} 
\begin{ex}[Potential energy]
Consider the function $F(\rho)=\frac{1}{2}\int \norm{x}^2 \rho(dx)$. It is strictly geodesically convex and has a unique minimizer at $\delta_0$. 

Given $e^{-g}$ and the mirror function \eqref{eq:mirror}, let us compute the time evolution of the mirror gradient flow. The evolution of the mirror potential is given by 
\[
\frac{\partial u_t}{\partial t}(x)= \frac{\delta F}{\delta \rho_t}(x)= \frac{1}{2}\norm{x}^2.  
\]
Thus $u_t= u_0 + \frac{t}{2} \norm{x}^2$, $t\ge 0$. A simple but instructive special case is when $u_0(x)=\frac{1}{2} \norm{x}^2$, i.e. $\rho_0=e^{-g}$, itself. Then $\nabla u_t= (1+t)\mathbf{id}$. Thus $\rho_t$ is the law of $Y/(1+t)$, where $Y \sim e^{-g}$. Clearly $\lim_{t\rightarrow \infty}\rho_t=\delta_0$ in $\wass_2$.
\end{ex}

\medskip

%\noindent\textbf{Example 2: Entropy.} 
\begin{ex}[Entropy]
Let $F(\rho)= \int \rho(x) \log \rho(x) dx$ denote the entropy function. The function is defined to be $+\infty$ when the measure is not absolutely continuous. From \cite[Lemma 10.4.1]{ambrosio2005gradient}, $\frac{\delta F}{\delta \rho}=\log \rho + 1$. Thus, given $e^{-g}$, the time evolution of the mirror potential is given by the PDE
\begin{equation}\label{eq:entropymirror}
\frac{\partial u_t}{\partial t}(x)=  \log \rho_t(x) +1, \quad \text{i.e.},\quad \nabla u_t(x)= \nabla u_0(x) + \int_0^t \nabla \log \rho_s ds.
\end{equation}
The flow on the other hand is given by the continuity equation $\dot{\rho}_t + \nabla \cdot (v_t \rho_t)=0$ where
\[
v_t(x) = - \frac{\partial\hfill}{\partial x^{u_t}} \log \rho_t(x).  
\]
A closed form solution is available for the Gaussian family. Let $\rho_0, e^{-g}$ be both standard normal. Thus $\nabla u_0(x)= x$. Then $\rho_t=N(0, (1+t)^2I)$ is a solution of the mirror gradient flow. This can be easily verified from the fact that 
\[
\nabla \log \rho_t(x)= -\frac{x}{(1+t)^2}, \quad \nabla u_t(x)= \frac{x}{(1+t)},\quad \nabla \frac{\partial u_t}{\partial t}(x)= -\frac{x}{(1+t)^2}.
\]
This system is a solution to \eqref{eq:entropymirror}. 

It is, of course, well-known that the Wasserstein gradient flow of entropy is the heat flow which admits the solution $N(0, (1+t)I)$ at time $t$ when started with standard normal. Thus, in this case, the mirror gradient flow ``converges'' faster than the usual gradient flow, although in both cases the solution diffuses as $t\rightarrow \infty$.  
\end{ex}

\medskip

%\noindent\textbf{R\'enyi entropy.} 
\begin{ex}[R\'{e}nyi entropy]
In \cite{Otto_2001}, Otto identified the solution of the porous medium equation as the Wasserstein gradient flow of the following functional that is related to the R\'{e}nyi entropy:
\[
F(\rho)= \frac{1}{m-1}\int \rho^m(x)dx,  
\]
where $m \ge 1 - \frac{1}{d}$, $m> \frac{d}{d+2}$ and $m\neq 1$. The function is defined to be $+\infty$ if $\rho$ is not absolutely continuous.

To find its mirror gradient flow, fix $e^{-g}$ to generate the mirror potential. From \cite[Lemma 10.4.1]{ambrosio2005gradient}, $\frac{\delta F}{\delta \rho}=\frac{m}{m-1} \rho^{m-1}$. Then, the evolution of the mirror potential is given by the PDE 
\[
\frac{\partial u_t}{\partial t}(x)= \frac{m}{m-1} \rho_t^{m-1}, \quad \text{where}\quad \rho_t= (\nabla u^*_t)_{\#} e^{-g},
\]
and the continuity equation of the mirror gradient flow is given by the velocity
\[
v_t(x)=-m\rho_t^{m-2}\nabla_{x^{u_t}} \rho_t(x).
\]
It is not immediate how these new flows compare with the traditional ones. 
\end{ex}


\section{The Sinkhorn Diffusion}\label{sec:diffmirr}
%\marginpar{\color{blue} [YH. As of July 19, It would be great if we improve \ref{rmk:mirror-Langevin}.
%I agree with the structure of this section and general idea of the proof. 
%I just need to check the proof of Theorem 3.2 for possible minor fixes. 
%]}
 It is a natural question whether there exists a diffusion such that the family of marginal distributions at each time point gives the Sinkhorn flow \eqref{eq:velocity}.  A classic example of this correspondence is  %betweee
 the Langevin diffusion (see \cite{lemons1997paul}) whose time marginals satisfy the Fokker-Planck equation. Such stochastic processes are useful in many applications including optimization (see \cite{roberts1996exponential,durmus2019analysis,chizat2022trajectory}) and sampling (see \cite{sohl2015deep,song2020score}). In this section we construct such a stochastic process inspired from a natural Markov chain embedded in the Sinkhorn algorithm, see \cref{prop:mchn} later in the paper. %{\red [Are we going to cut off some part of Section 4 on Markov chains?]}

%{\color{blue} We introduce the following family of stochastic processes,  whose distributions are the mirror flows \eqref{eq:velocity}.}

\begin{defn}[Sinkhorn diffusion] The Sinkhorn diffusion is   the solution to the following stochastic differential equation (SDE):
\begin{equation}\label{eq:diffSDE}
    dX_t=\left(-\frac{\partial f\hfill}{\partial \xsut}(X_t)-\frac{\partial g\hfill}{\partial\xsut}\left(X_t^{u_t}\right)+\frac{\partial h_t\hfill}{\partial \xsut}(X_t)\right)\,dt+\sqrt{2\frac{\partial X_t\hfill}{\partial X_t^{u_t}}}dB_t,
\end{equation}

%{\red {\bf Here what I wrote previously below was wrong as I did not include the effect of the difficution matrix after it is defferentiated. The above expression of SDE is consistent with Ito's formula. } \marginpar{YH. Probably I am misunderstanding something here... Please let me know} [Probability it would be easier for the reader, if we  add a remark that \eqref{eq:diffSDE} is nothing but the same as the usual Langevin diffusion type equation, corresponding to the entropic gradient flow, where in that case the drift part is given by the Wasserstein gradient. In our case, I think we are just multiplying $\frac{\partial x^{u_t}}{\partial x}$ to the drift vector, and use the Brownian motion of the underlying geometry of the metric $\frac{\partial X_t\hfill}{\partial X_t^{u_t}}$. In other words, that \eqref{eq:diffSDE} is nothing but the diffusion version of the entropic gradient flow, with the `mirror' metric given by $\frac{\partial x \hfill}{\partial x^{u_t}}$. Is this a correct understanding? (By the way, this correspondence is different from that one in Theorem~\ref{thm:existpropX}.) For the convergence of such a process, the point is that the underlyng metric also changes along the time and the flow, so, it is *not* straightforward as we see below. But, probably this point of view (Langevin diffusion with respect to the mirror metric, deforming the drift and the Brownian motion by the mirror metric) may put the whole theorems in a more natural way? Probably, what I am saying here is that it would be better to organize the discussion of this section from the point of view of Remark~\ref{rmk:mirror-Langevin}.]}
where
\begin{enumerate}[(i)]

\item $X_0$ is distributed according to an initial density $\rho_0$. At each subsequent time $t$, $X_t$ admits a density $\rho_t=e^{-h_t}$. 

\item $u_t$ is a convex function whose gradient is the Brenier map transporting $\rho_t$ to $e^{-g}$. That is, $(\nabla u_t)_{\#} \rho_t=e^{-g}$. At each time $t$, this leads to a mirror coordinate system $x \mapsto x^{u_t}$. 

\item $\frac{\partial f\hfill}{\partial \xsut}(x)$ refers to the derivative of $x \mapsto f(x)$ with respect to the dual variable $x^{u_t}$. Same for $\frac{\partial h_t\hfill}{\partial \xsut}(x)$.

\item $\frac{\partial g\hfill}{\partial \xsut}(x^{u_t})$ is the gradient of the map $y \mapsto g(y)$ evaluated at $y=x^{u_t}$. 

\item $(B_t,\; t\geq 0)$ is a standard $d$-dimensional  Brownian motion and the diffusion matrix 
$\displaystyle 2\frac{\partial X_t\hfill}{\partial X_t^{u_t}}$ 
at time $t$ is  
\[
2\frac{\partial x\hfill}{\partial x^{u_t}}=2\left( \nabla^2 u_t(x) \right)^{-1},
\]
evaluated at $X_t=x$. 
\end{enumerate}
\end{defn}


Sinkhorn diffusion is an example of Mckean-Vlasov family of diffusions \cite{mckean1966class}.  The study of such systems originated from the probabilistic study of the Boltzmann and Vlasov equations due to Kac~\cite{kac1956foundations}, McKean~\cite{mckean75}, Dobrushin~\cite{dobrushin79}, Tanaka~\cite{tanaka78} and many others. For modern surveys, see Sznitman~\cite{SznitmanSF}, Villani~\cite{villani12notes}, Chaintron and Diez~\cite{ChaintronDiez} and Jabin~\cite{Jabin14}.  

%\SP{SP: not sure how the following references fit here. Are there processes related to the Sinkhorn diffusion?}

%Other references: Stein's variational gradient descent (see~\cite{liu2016stein,Liu2017}), maximum mean discrepancy gradient flow (see \cite{aubin2022mirror}), mirror flows with application to EM algorithm (see \cite{arbel2019maximum}).


We will show that a weak solution of the SDE exists and is unique under suitable assumptions. Towards this, suppose that solution of the PMA \eqref{eq:pma} exists which satisfies Assumption \ref{asn:solcon} with the initial condition $u_0$.
In fact, Assumption \ref{asn:solcon} gives sufficient regularity to the mirroring map $\nabla u_t$, which gives the mirror map $x\mapsto x^{u_t}$. %$=2\left( \nabla^2 u_t(x) \right)^{-1},
The Sinkhorn diffusion always has a dual diffusion process, say $Y$,  given via this mirror map, i.e., $Y_t=X_t^{u_t}$.   As we will show in \cref{thm:existpropX} below $Y$ satisfies the following SDE: 
\begin{equation}\label{eq:dualdiffSDE} 
\begin{split}
dY_t&= -\nabla h_t\left(Y^{w_t}_t\right)dt +\sqrt{2\frac{\partial Y_t\hfill}{\partial Y_t^{w_t}}}d B_t,
\end{split}
\end{equation}
where $\rho_t=e^{-h_t}= \left(\nabla w_t\right)_{\#} e^{-g}$ is the pushforward of $e^{-g}$ by the map $y \mapsto \nabla w_t(y)$. 
Here $\nabla h_t(Y_t^{w_t})$ is the gradient of $h_t$ with respect to its argument, evaluated at $Y_t^{w_t}$. 

%{\red 
%[Isn't $e^{-h_t}$ nothing but $\rho_t$?\\
%Also, isn't $\nabla h_t\left(Y^{w_t}_t\right)$ nothing but$\nabla g\left(Y_t\right) $? $\leftarrow$ I think I was wrong.
%Or is the gradient $\nabla$ in \eqref{eq:dualdiffSDE} for the variable $Y^{w_t}$, not $Y$? 
%]
%}

\begin{thm}\label{thm:existprop}
Let
\[
b(t,y):=- \nabla h_t(y^{w_t}), \quad \sigma(t,y):=\sqrt{2 \frac{\partial y\hfill}{\partial y^{w_t}}}= \sqrt{2  \left(\nabla^{2}\left( w_t(y)\right)\right)^{-1}}.
\]

Additionally, suppose the standard global Lipschitz and linear growth conditions hold,  namely, for some $K>0$, $b, \sigma$ are uniformly $K$-Lipschitz functions on $\rr^d$ and,  
    \begin{equation}\label{eq:growth}
    \norm{b(t,y)}^2 + \norm{\sigma(t,y)}^2 \le K\left(1 + \norm{y}^2 \right)
    \end{equation}
for all $t\ge 0$. 
Then, if the initial distribution is square-integrable, the SDE \eqref{eq:dualdiffSDE} admits a unique strong solution such that every subsequent $Y_t$ is also square-integrable.  

The infinitesimal generator of the process at time $t$, acting on a $\diffcont^2$ function $\phi$, is given by 
\[
\mathcal{L}_t^Y \phi = e^{g} \div \left( e^{-g} \nabla_{y^{w_t}}\phi \right).
\]
Consequently $e^{-g}$ is a stationary distribution for this process.
\end{thm}
%{\red 
%[For this theorem, I think it would be better to separate the general result about SDE with the general $b$ and $\sigma$. Probably we want to put   \cref{asn:solcon} in the theorem. ]
%}

\begin{remark}\label{rem:suffcon}
    The condition \eqref{eq:growth} holds under \cref{asn:solcon} via elementary computations. In particular, this implies that under \cref{asn:solcon}, the conclusion in \cref{thm:existprop} holds.
\end{remark}


\begin{proof}[Proof of~\cref{thm:existprop}] The claim about existence, uniqueness (pathwise and in law) and square-integrability follow from \cite[Theorem 5.2.9]{karatzas1991brownian}.

It remains to compute the infinitesimal generator. Let $\varphi\in \diffcont^2$. Then, by It\^o's formula 
\[
\begin{split}
    \mathcal{L}^Y_t\varphi &= - \frac{\partial \varphi}{\partial y} \cdot \frac{\partial h_t\hfill}{\partial y^{w_t}}(y^{w_t}) + \sum_{i=1}^d \sum_{l=1}^d \frac{\partial y_l\hfill}{\partial y^{w_t}_i} \frac{\partial^2 \varphi}{\partial y_i \partial y_l}.
\end{split}
\]
By the formula for change of measures
\[
-h_t(y^{w_t})= -g(y) - \log \det \frac{\partial y\hfill}{\partial y^{w_t}}.
\]
Taking gradients with respect to $y^{w_t}$ on both sides and using formula \eqref{eq:tensorelpf} we get 
\[
\begin{split}
  \mathcal{L}^Y_t\varphi &=  -\frac{\partial \varphi}{\partial y} \cdot \frac{\partial g\hfill}{\partial y^{w_t}}(y) + \sum_{i=1}^d\sum_{l=1}^d \frac{\partial \varphi}{\partial y_l} \frac{\partial^2 y_l}{\partial y_i \partial y^{w_t}_i} + \sum_{i=1}^d \sum_{l=1}^d  \frac{\partial^2 \varphi}{\partial y_i \partial y_l}\frac{\partial y_l\hfill}{\partial y^{w_t}_i}\\
  &= -\frac{\partial \varphi}{\partial y} \cdot \frac{\partial g\hfill}{\partial y^{w_t}}(y) + \sum_{i=1}^d \frac{\partial}{\partial y_i}\left[ \sum_{l=1}^d \frac{\partial \varphi}{\partial y_l} \frac{\partial y_l\hfill}{\partial y^{w_t}_i} \right]\\
  &=-\frac{\partial \varphi}{\partial y} \cdot \frac{\partial g\hfill}{\partial y^{w_t}}(y) + \sum_{i=1}^d \frac{\partial^2 \varphi}{\partial y_i \partial y^{w_t}_i}= e^{g}\div\left( e^{-g} \nabla_{y^{w_t}} \varphi\right).
\end{split}
\]

That $e^{-g}$ is a stationary measure follows immediately, since
\[
\int \mathcal{L}^Y_{t} \varphi(y) e^{-g(y)}dy = \int \div\left( e^{-g} \nabla_{y^{w_t}} \varphi\right) dy=0.
\]
\end{proof}

\begin{comment}
\begin{proof}
    Consider an SDE of the form
\begin{equation}\label{eq:sdegen}
dY_t = b(t,Y_t)dt + \sigma(t, Y_t) dB_t,
\end{equation}
where $B$ is a standard multidimensional Brownian motion and with an initial condition $Y_0=y_0$ ({\color{blue} In our case $Y_0$ is non-degenerate. Does that change anything?}). 


Stroock and Varadhan proved that (see \cite[Theorem 5.11]{klebaner}) that if $(y,t)\mapsto \sigma(y,t)$ is continuous, positive and if, for each $T>0$, there is a constant $K_T>0$ such that 
\begin{equation}\label{eq:weakexist}
\abs{b(y,t)} + \abs{\sigma(y,t)} \le K_T(1+ \abs{y}),
\end{equation}
for all $(y,t)\in \R^d \times [0,T]$, then the SDE \eqref{eq:sdegen} admits a unique weak solution for any $y_0$ which is also strong Markov. By stopping the process when it hits a ball of radius $R$, we can replace \eqref{eq:weakexist} by a local linear growth criterion: weak existence and uniqueness holds if for every $T>0$ and every $R>0$, there is a constant $K(T,R)$ such that   
\begin{equation}\label{eq:weakexistnew}
\sup_{\abs{y}\le R,\; 0\le t\le T }\left[ \abs{b(y,t)} + \abs{\sigma(y,t)}\right] \le K(T,R)(1+ \abs{y}),
\end{equation}

In the case of \eqref{eq:dualdiffSDE}, 
\[
b(y,t)= -\frac{\partial h_t\hfill}{\partial y^{w_t}}(y^{w_t}),\quad \sigma^2(y,t)= 2\frac{\partial y\hfill}{\partial y^{w_t}}. 
\]
As $(\nabla u_t)_{\#}\rho_t=\exp(-g)$ and $\rho_t=\exp(-h_t)$, \eqref{eq:pma} implies $$h_t(x)=\frac{\partial}{\partial t}u_t(x)-f(x).$$
Then writing both $b(y,t)$ and $\sigma(y,t)$ in terms of $u_t$, we get:
\[
b(y,t)= -\nabla^2 u_t(y^{w_t})\left(\frac{\partial}{\partial y}\frac{\partial}{\partial t}u_t(y^{w_t})-\frac{\partial}{\partial y}f(y^{w_t})\right),\quad \sigma^2(y,t)= 2\nabla^2 u_t(y^{w_t}). 
\]
By Assumptions \ref{asn:smoothfg} and \ref{asn:solcon}, both $b(\cdot,\cdot)$ and $\sigma(\cdot,\cdot)$ are continuous functions on $\{(y,t): |y|\le R,\ 0\le t\le T\}$, which in turn, yields \eqref{eq:weakexistnew}.

\end{proof}
\end{comment}
%{\blue \bf [I understand this theorem is a standard result but, where is the proof of this theorem? ]}
%\item Suppose \bu0(⋅)\bu_0(\cdot) is such that ∇\bu0#μ=ν\nabla \bu_0\#\mu=\nu. Then \eqref{eq:diffSDE} reduces to the following diffusion SDE:
%\begin{equation}
%\begin{aligned}\label{eq:diffmirSDE} 
%dX_t=-\frac{\partial}{\partial\xszt}& g(\bxszt)\,dt+ %\sqrt{2\seczx}\,dB_t, \\ &\nabla \bu_0(X_0)\sim \nu.
%\end{aligned}
%\end{equation}

\begin{remark}
    The above diffusion generator is a time-inhomogeneous analog of the one described by \cite[Section 2]{Kolesnikov2012HessianMC}. 
\end{remark}



\begin{thm}\label{thm:existpropX}
Let $Y$ be a solution of \eqref{eq:dualdiffSDE}. Define $X_t:=Y_t^{w_t}$. Then $(X_t,\; t\geq 0)$ solves \eqref{eq:diffSDE}. Conversely, for any solution $X$ of \eqref{eq:diffSDE}, the transformed process $Y_t=X_t^{u_t}$ is a solution of \eqref{eq:dualdiffSDE}. Hence, under the assumptions of Theorem \ref{thm:existprop}, a weak solution of \eqref{eq:diffSDE} exists and is unique. The marginals of $X_t$ so constructed satisfy \eqref{eq:velocity}. 
\end{thm}


Of course, one may also impose global Lipschitzness and linear growth property on the drift and diffusion coefficients of $X$ to obtain a strong solution. The reason we used the $Y$ process is one, that it has fewer terms in the drift, and two, we only require the $Y$ process to run in stationarity to obtain a weak solution for $X$.
%\marginpar{Good point! Now I see why $Y_t$ was put first.}

In order to prove \cref{thm:existpropX}, we need the well-known Ito's lemma \cite{karatzas1991brownian}, which we note down here for easy reference. 

%\marginpar{It is not a good idea to use the notation $Y_t$ in this lemma.}
\begin{lmm}\label{lem:ito}
Consider an SDE of the form
$$d X_t=b(t,X_t)\,dt+\sigma(t,X_t)\,dB_t.$$
Here $b:[0,\infty)\times \R^d\to \R^d$ and $\sigma:[0,\infty)\times \R^d \rightarrow \R^d \times \R^d$ are progressively measurable functions. Let $\phi:[0,\infty)\times \R^d\to\R$ be an element in  $\diffcont^{1,2}$. Define $S_t=\phi(t,X_t)$. Then 
\begin{align}\label{eq:ito}
d S_t= \frac{\partial}{\partial x}  \phi(t,X_t)^{\top}& \sigma(t,X_t)\,dB_t+\left[\frac{\partial}{\partial t}\phi(t,X_t) + \frac{\partial}{\partial x} \phi(t,X_t)^{\top} b(t,X_t)\right] dt \nonumber\\
&+\frac{1}{2}\trc\left(\sigma(t,X_t) \sigma(t,X_t)^{\top}\nabla^2_x \phi(t,X_t)\right) dt.
\end{align}
\end{lmm}

%\marginpar{[I guess there could be a simplification of this proof. Let me think about it.]}
\begin{proof}[Proof of Theorem \ref{thm:existpropX}]
By \cref{thm:existprop}, there exists a strong Markov process $(Y_t,\ t\ge 0)$ which is a weak solution to the SDE in \eqref{eq:dualdiffSDE}. 
Consider $X_t=Y_t^{w_t}$. By \cref{asn:solcon} (iv), we can apply It\^{o}'s formula in \cref{lem:ito} with
$$b(t,y)=-\frac{\partial h_t \hfill}{\partial y^{w_t}}(y^{w_t}),\quad \sigma^2(t,y)=2\frac{\partial y}{\partial y^{w_t}},\quad \phi(t,y)=y^{w_t},$$ 
to get that $(X_t,\ t\ge 0)$ is Markov and:
\begin{align}\label{eq:dual2}
dX_t&= \frac{\partial Y_t^{w_t}}{\partial Y_t\hfill}  \sqrt{2\frac{\partial Y_t\hfill}{\partial Y_t^{w_t}}}\,dB_t+\left[\left(\frac{\partial}{\partial t}\nabla w_t\right)(Y_t) - \frac{\partial Y_t^{w_t}}{\partial Y_t}  \frac{\partial h_t\hfill}{\partial y^{w_t}}(Y_t^{w_t})\right] dt\nonumber \\ &+\frac{1}{2}\trc\left(2\frac{\partial Y_t\hfill}{\partial Y_t^{w_t}}\nabla^2_y (Y_t)^{w_t}_i\right)_{i\in [d]} dt.
\end{align}
By using \cref{lem:dualPMA} and the multivariate chain rule, the second term on the right hand side above reduces to
\begin{align}\label{eq:dual1}
&\;\;\;\;\;\left(\frac{\partial}{\partial t}\nabla w_t\right)(Y_t) - \frac{\partial Y_t^{w_t}}{\partial Y_t}\frac{\partial h_t\hfill}{\partial y^{w_t}}(Y_t^{w_t})\nonumber \\&=\frac{\partial g}{\partial y}(Y_t)-\frac{\partial f}{\partial y}(Y_t^{w_t})+\frac{\partial}{\partial y}\ldet \left(\frac{\partial Y_t^{w_t}}{\partial Y_t\hfill}\right)-\frac{\partial h_t}{\partial y\hfill}(Y_t^{w_t})\nonumber \\&=\frac{\partial g\hfill}{\partial x^{u_t}}(X_t^{u_t})-\frac{\partial f\hfill}{\partial x^{u_t}}(X_t)-\frac{\partial}{\partial x^{u_t}}\ldet\left(\frac{\partial X_t\hfill}{\partial X_t^{u_t}\hfill}\right)-\frac{\partial h_t\hfill}{\partial x^{u_t}}(X_t) \nonumber \\ &=-\frac{\partial f\hfill}{\partial x^{u_t}}(X_t).
\end{align}
In the third display above, we have used that $Y_t=X_t^{u_t}$. In the fourth display, we use \cref{lem:jacobian} with $\phi=u_t$, $a=h_t$ and $b=g$. Next let us simplify the third term on the right hand side of \eqref{eq:dual2}, for each $i\in [d]$.
\begin{align*}
\trc\left(\frac{\partial Y_t\hfill}{\partial Y_t^{w_t}}\nabla^2_y (Y_t)^{w_t}_i\right)&=\sum_{k} \frac{\partial^2}{\partial y^{w_t}_k \partial y_k}(Y_t)^{w_t}_i=\sum_{k}\frac{\partial^2}{\partial x_k\partial x^{u_t}_k}(X_t)_i.
\end{align*}
In the above display, we have used once again that $Y_t=X_t^{u_t}$. By combining the above observation with  \cref{lem:tensorel}, we then have:
\begin{align}\label{eq:dual3}
&\;\;\;\;\;\trc\left(\frac{\partial Y_t\hfill}{\partial Y_t^{w_t}}\nabla^2_y (Y_t)^{w_t}_i\right)\nonumber \\&=-\frac{\partial}{\partial x^{u_t}_i}\ldet\left(\frac{\partial X_t^{u_t}}{\partial X_t\hfill}\right)=\frac{\partial h_t\hfill}{\partial x^{u_t}_i}(X_t)-\frac{\partial g\hfill}{\partial x^{u_t}_i}(X_t^{u_t}).
\end{align}
By combining \eqref{eq:dual2}, \eqref{eq:dual1} and \eqref{eq:dual3}, we get that $(X_t,\ t\ge 0)$ is a strong Markov process which is a weak solution of \eqref{eq:diffSDE}.

\bigskip 

\noindent Note that there exists a unique strong Markov process which is a weak solution to \eqref{eq:dualdiffSDE} by \cref{thm:existprop}. In order to establish uniqueness in \cref{thm:existpropX}, it suffices to show that given any strong Markov process $(X_t,\ t\ge 0)$ which is a weak solution of \eqref{eq:diffSDE}, the process $(Y_t=X_t^{u_t},\ t\ge 0)$ is a weak solution to \eqref{eq:dualdiffSDE}. Once again, we use It\^{o}'s lemma \ref{lem:ito}, this time with 
$$b(t,x)=-\frac{\partial f\hfill}{\partial \xsut}(x)-\frac{\partial g\hfill}{\partial\xsut}\left(x^{u_t}\right)+\frac{\partial h_t\hfill}{\partial \xsut}(x), \quad \sigma^2(t,x)=2\frac{\partial x\hfill}{\partial x^{u_t}},\quad \phi(t,x)=x^{u_t}.$$
This gives
\begin{align}\label{eq:dual4}
dY_t= \frac{\partial X_t^{u_t}\hfill}{\partial X_t\hfill}  \sqrt{2\frac{\partial X_t\hfill}{\partial X_t^{u_t}}}\,dB_t&+\Bigg[\left(\frac{\partial}{\partial t}\nabla u_t\right)(X_t) -\frac{\partial X_t^{u_t}}{\partial X_t\hfill}\Bigg(\frac{\partial f\hfill}{\partial \xsut}(X_t)+\frac{\partial g\hfill}{\partial\xsut}\left(X_t^{u_t}\right)\nonumber \\ &-\frac{\partial h_t\hfill}{\partial \xsut}(X_t)\Bigg)\Bigg] dt+\frac{1}{2}\trc\left(2\frac{\partial X_t\hfill}{\partial X_t^{u_t}}\nabla^2_x (X_t)^{u_t}_i\right)_{i\in [d]} dt.
\end{align}
By using \eqref{eq:pma}, we get:
\begin{align*}
&\;\;\;\;\left(\frac{\partial}{\partial t}\nabla u_t\right)(X_t) -\frac{\partial X_t^{u_t}}{\partial X_t\hfill}\left(\frac{\partial f\hfill}{\partial \xsut}(X_t)+\frac{\partial g\hfill}{\partial\xsut}\left(X_t^{u_t}\right)-\frac{\partial h_t\hfill}{\partial \xsut}(X_t)\right)\\ &=-2\frac{\partial g}{\partial x}(X_t^{u_t})+\frac{\partial h_t}{\partial x}(X_t)+\frac{\partial}{\partial x}\ldet\left(\frac{\partial X_t^{u_t}}{\partial X_t\hfill}\right)=-\frac{\partial g}{\partial x}(X_t^{u_t}).
\end{align*}
Here the last equality follows by invoking \cref{lem:jacobian} with $\phi=u_t$, $a=h_t$ and $b=g$. Finally, fix $i\in [d]$ and note that by the same computation as in \eqref{eq:dual3}, we have:
\begin{align*}
\trc\left(\frac{\partial X_t\hfill}{\partial X_t^{u_t}}\nabla^2_x (X_t)^{u_t}_i\right)=\frac{\partial}{\partial x_i}\ldet\left(\frac{\partial X_t^{u_t}}{\partial X_t\hfill}\right)=\frac{\partial g\hfill}{\partial x_i}(X_t^{u_t})-\frac{\partial h_t}{\partial x_i}(X_t).
\end{align*}
Combining the two displays above with \eqref{eq:dual4} and using that $Y_t=X_t^{u_t}$, we get that $(Y_t,\ t\ge 0)$ is a strong Markov process which is a weak solution of \eqref{eq:dualdiffSDE}.

\noindent We will use Ito's rule to establish the flow of the marginals. Pick a smooth, compactly supported real-valued function $\phi(\cdot)$, then by invoking Ito's rule in~\eqref{eq:ito} with 
\begin{align*}
b(t,x)=-\frac{\partial f\hfill}{\partial x^{u_t}}(x)-\frac{\partial g\hfill}{\partial x^{u_t}}(x^{u_t})+\frac{\partial h_t\hfill}{\partial x^{u_t}}(x),\quad \quad \sigma^2(t,x)=2\frac{\partial x\hfill}{\partial x^{u_t}},
\end{align*}
the expectation of the generator is given by:
\begin{align*}
&=\E[\mathcal{L}(\phi)(X_t)]\\ &=\int \left\langle\frac{\partial\phi}{\partial x}(x),b(t,x)\right\rangle \exp(-h_t(x))\,dx + \sum_{i,j}\int \frac{\partial}{\partial x_i}\left(\frac{\partial \phi\hfill}{\partial x_j}(x)\right)\left(\frac{\partial x\hfill}{\partial x^{u_t}}\right)_{i,j}\exp(-h_t(x))\,dx\\ &=\int \left\langle\frac{\partial\phi}{\partial x}(x),b(t,x)\right\rangle \exp(-h_t(x))\,dx + \sum_{j}\int \frac{\partial}{\partial x^{u_t}_j}\left(\frac{\partial \phi\hfill}{\partial x_j}(x)\right)\exp(-h_t(x))\,dx\\ &=\int \left\langle\frac{\partial\phi}{\partial x}(x),b(t,x)\right\rangle \exp(-h_t(x))\,dx + \sum_{j}\int \frac{\partial\phi\hfill}{\partial x_{j}}(x)\frac{\partial}{\partial x^{u_t}_j}\exp(-h_t(x))\,dx\\ &=\int \left\langle\frac{\partial\phi}{\partial x}(x),b(t,x)+\frac{\partial}{\partial x^{u_t}}(g(x^{u_t}))\right\rangle \exp(-h_t(x))\,dx\\ &=\int \left\langle\frac{\partial\phi}{\partial x}(x),-\frac{\partial}{\partial x^{u_t}}(f-h_t)(x)\right\rangle \exp(-h_t(x))\,dx. 
\end{align*}
By the absolute continuity of $(\rho_t)$, there exists a velocity field $v_t(\cdot)\in \R^d$ such that the continuity equation 
\begin{equation}\label{eq:continuity}
\frac{\partial\rho_t}{\partial_t\hfill} +\div{(v_t \rho_t)}=0
\end{equation}
is satisfied in the sense that 
\begin{align*}
\E[\mathcal{L}(\phi)(X_t)]=\int \left\langle \frac{\partial\phi}{\partial x\hfill}(x),v_t(x)\right\rangle \exp(-h_t(x))\,dx.
\end{align*}
As the above displays hold for all smooth $\phi(\cdot,\cdot)$, by comparing them, we get:
$$v_t(x)=-\frac{\partial}{\partial x^{u_t}}(f-h_t)(x).$$


\noindent This completes the proof.
\end{proof} 
%\marginpar{YH. From my point of view, this remark~\ref{rmk:mirror-Langevin} is a natural starting point of the discussin in this section 3.}
\begin{remark}\label{rmk:mirror-Langevin}
    A very special case of \eqref{eq:dualdiffSDE} is when $t\mapsto u_t$ is constant, i.e., $u_t\equiv u_0=:u$. Then both $X$ and $Y$ processes are stationary with respect to $e^{-f}$ and $e^{-g}$, respectively. The SDE \eqref{eq:diffSDE} and \eqref{eq:dualdiffSDE}, respectively, reduce to  
    \begin{equation}\label{eq:diffmirSDE}
    \begin{split}
    dX_t &=-\frac{\partial g\hfill}{\partial x^u}\left(X_t^{u}\right)dt+\sqrt{2\frac{\partial X_t\hfill}{\partial X_t^{u}}}dB_t\\
    dY_t &=-\frac{\partial f\hfill}{\partial y^w}\left(Y_t^{w}\right)dt+\sqrt{2\frac{\partial Y_t\hfill}{\partial Y_t^{w}}}dB_t,
    \end{split}
     \end{equation}
    where $w=u^*$, the convex conjugate.
    
    Note that \eqref{eq:diffmirSDE} is equivalent to the mirror Langevin diffusion (see \cite{ahn2021efficient,zhang2020wasserstein}). In the case where $\mu=\nu$, \eqref{eq:diffmirSDE} reduces to the standard Langevin diffusion (see \cite{jordan1998variational,durmus2019analysis}).
\end{remark}


%\subsection{Evolution PDE/continuity equation} 

%\begin{lmm}\label{thm:conteq}
 %The time marginals of the Sinkhorn diffusion $X_t$ in \eqref{eq:diffSDE} is given by the Sinkhorn PDE from Theorem \ref{thm:existlin}. 
 %\end{lmm}





%\begin{prop}[Connection to linearized optimal transport]\label{prop:linot1}
    %Suppose Assumptions \ref{asn:solcon} and \ref{asn:smoothfg} hold. Recall the definition of $v_t(\cdot)$ from \eqref{eq:velocity}. Then we have:
    %\end{prop}

%\begin{proof}   
%\end{proof}
 
%

\section{Sinkhorn Markov chain and Limiting Dynamics}\label{sec:mcconst}

\noindent Fix $\epsilon>0$. Following \cite{berman2020}, we will track the evolution of the Sinkhorn algorithm  using the following  maps $\opV:\diffcont(\R^d)\to \diffcont(\R^d)$ and $\opU:\diffcont(\R^d)\to \diffcont(\R^d)$, where 
\begin{equation}\label{eq:basedef}
    \begin{split}
        \opV[u](y)&:= \vep \log\int \exp\left(\frac{1}{\vep}\langle x,y\rangle-\frac{1}{\vep}u(x)-f(x)\right)\,dx,\quad u \in \diffcont(\R^d)\\
        \opU[v](x)&:= \vep \log\int  \exp\left(\frac{1}{\vep}\langle x,y\rangle-\frac{1}{\vep}v(y)-g(y)\right)\,dy, \quad v \in \diffcont(\R^d).
    \end{split}
\end{equation}



Next, consider the new operator $\opS:\diffcont(\R^d)\to \diffcont(\R^d)$ defined by:
\begin{equation}\label{eq:Sep}
\opS[u]:=\opU\circ \opV[u].    
\end{equation}
In terms of the Sinkhorn algorithm $\opS$ tracks the potential after two successive steps.

The following proposition from \cite[eqn. (2.1.4)]{berman2020} shows an useful property of the increment $\opS[u]-u$ which will be useful in the sequel. 

\begin{prop}\label{cl:normalize}
For any $u\in \diffcont(\R^d)$, consider the nonnegative function \begin{equation}\label{eq:claim12}
\rv[u](x):=\exp\left(\frac{1}{\vep}\left(\opS[u](x)- u(x)\right)\right).
\end{equation}
Then, 
\[
\int \rv[u](x)\exp(-f(x))\,dx=1.
\]
That is, $\rv[u]\exp(-f)$ is a probability measure. 
\end{prop}

The operator $\opS$ yields a natural iteration on $\diffcont(\R^d)$. Starting with some $u_0\in \diffcont(\R^d)$, consider
\begin{equation}\label{eq:twostepit}
u^{\vep}_{k+1}:=\opS[u_k],\quad \mk{k+1}:=\rv[u_k^{\vep}]\exp(-f),\quad k\ge 0.
\end{equation}
The $u_k^{\vep}$s are usually called \emph{Sinkhorn potentials}. By~\cref{cl:normalize}, we have
\begin{equation}\label{eq:discdiff}
\frac{u^{\vep}_{k+1}-u^{\vep}_k}{\vep}-f=\log{\mk{k+1}},\quad \mbox{where}\, \int_x \mk{k+1}(x)\,dx=1.
\end{equation}
Consequently, the average increment satisfies
\[
\int_x (u^{\vep}_{k+1}(x)-u^{\vep}_k(x))\,e^{-f(x)}\,dx=-\vep \KL{e^{-f}}{{\color{purple}\mk{k+1}}}\leq 0,
\]
where $\mI$ denotes the appropriate Kullback-Leibler divergence.

Note how the LHS of \eqref{eq:discdiff} looks like a discrete time derivative if iterations are indexed in units of $\epsilon$. That is, we replace $k$ by $k \vep$ and $(k+1)$ by $(k+1)\vep$. This observation will be useful later when we take scaling limit by sending $k\vep \rightarrow t>0$. 
\par 

%\subsection{Sinkhorn Markov chain} 

Define the following sequence of probability densities on $\R^d\times\R^d$ for $k\ge 0$:
\begin{align}\label{eq:coupling}
\gvp_{k+1}(x,y):=\exp\left(\frac{1}{\vep}\langle x,y\rangle-\frac{1}{\vep}u^{\vep}_k(x)-\frac{1}{\vep}\opV[u^{\vep}_{k}](y)-f(x)-g(y)\right).
\end{align}
From the definitions of $\opV$, $\opU$ and $\opS$ in \eqref{eq:basedef} and \eqref{eq:claim12}, it is easy to check that
$$\int_{\R^d\times\R^d}\gvp_{k+1}(x,y)\,dx\,dy=1,\quad \forall \ k\ge 0,$$
and further
\begin{equation}\label{eq:curtaincall}
p_X \gvp_{k+1}=\mk{k},\quad p_Y \gvp_{k+1}=\exp(-g),\quad \quad \forall\ k\ge 0.
\end{equation}
Therefore $\gamma_{k}^{\vep}$ is the Schr\"{o}dinger Bridge (see \cite{Schrodinger1932}) coupling between $\mk{k}$ and $e^{-g}$.  
{\color{black}As the $Y$-marginals of all $\gvp_{k+1}$s remain stationary at $\exp(-g)$, one can construct a natural Markov chain using $\gvp_{k+1}$s. This is elucidated in the following definition.

\begin{defn}[Sinkhorn Markov chain]\label{prop:mchn}
Let $\gamma_0^{\vep}$ be some arbitrary joint distribution where $p_Y \gamma_0^\vep =e^{-g}$. For $k\ge 1$, consider the family of joint distributions $\gamma_k^\vep$ from \eqref{eq:coupling}. Then, the transition probabilities for the Sinkhorn Markov chain can be defined inductively as follows. For any $k\ge 0$, suppose $(X_k^{\vep},Y_k^{\vep})=(x,y)$. Then sample $X_{k+1}^{\vep}$ from the conditional distribution of $X|Y=y$ under $\gvp_{k+1}$. Now  suppose $X_{k+1}^\vep=x'$. Then, sample $Y^{\vep}_{k+1}$ similarly from the other conditional, i.e., $Y|X=x'$ under $\gvp_{k+1}$. Then $(X_k^{\vep},\ k\ge 0)$ forms a time-inhomogeneous Markov chain which we will call the \emph{Sinkhorn Markov chain}.
\end{defn}

The transition kernel for the Markov chain in \cref{prop:mchn} can be written out in terms of the conditional distributions under $(\gvp_{k+1},\ k\ge 0)$. In particular, note that 
\begin{equation}\label{eq:con1}
p_{Y|X}\gvp_{k+1}(y|x)=\exp\left(\frac{1}{\vep}\langle x,y\rangle-\frac{1}{\vep}\opV[u_k^{\vep}](y)-\frac{1}{\vep}\opS[u_k^{\vep}](x)-g(y)\right),
\end{equation}
and 
\begin{equation}\label{eq:con2}
p_{X|Y}\gvp_{k+1}(x|y)=\exp\left(\frac{1}{\vep}\langle x,y\rangle-\frac{1}{\vep}\opV[u_k^{\vep}](y)-\frac{1}{\vep}u_k^{\vep}(x)-f(x)\right).
\end{equation}
Then the transition kernel can be written as
\begin{align}\label{eq:tranker}
    \Pr(X^{\vep}_{k+1}\in \,dx|X^{\vep}_k=z)&=\int_{\R^d} p_{Y|X}\gvp_{k}(y|z)p_{X|Y}\gvp_{k+1}(x|y)\,dy.
\end{align}

The following is easy to check. We provide a brief proof below.

\begin{prop}\label{prop:Sinmar}
    The marginals of the \emph{Sinkhorn Markov chain} are distributed according to $\mk{k}$ for $k\ge 1$.
\end{prop}

\begin{proof}
    By \eqref{eq:tranker}, for $k\ge 1$, we get:
    \begin{align*}
        \mathbb{P}(X_{k}^{\vep}\in \,dx)&=\int \mathbb{P}(X_{k}^{\vep}\in \, dx| X_{k-1}^{\vep}=z)p_{X}\gvp_{k-1}(z)\,dz\\ &=\int \int p_{Y|X}\gvp_{k-1}(y|z)p_{X|Y}\gvp_{k}(x|y) p_{X}\gvp_{k-1}(z)\,dy\,dz\\ &=\int p_{X|Y}\gvp_{k}(x|y)\left(\int \gvp_{k-1}(z,y)\,dz\right)\,dy\\ &=\int \exp\left(\frac{1}{\vep}\langle x,y\rangle - \frac{1}{\vep}\opV[u_{k-1}^{\vep}](y)-\frac{1}{\vep}u_{k-1}^{\vep}(x)-f(x)-g(y)\right)\,dy\\ &=\exp\left(\frac{1}{\vep}(u_{k}^{\vep}(x)-u_{k-1}^{\vep}(x))-f(x)\right)=\mk{k}(x).
    \end{align*}
    For the fourth equality, we use \eqref{eq:con2}. The fifth equality uses \eqref{eq:curtaincall}. The rest follows using elementary manipulations of conditional probabilities.

    %Next assume that the conclusion holds for $k=k_0\ge 1$, i.e., we assume that the following holds: 
    %\begin{align}\label{eq:indstep}
     %   \mathbb{P}(X_{k_0}^{\vep}\in dx)=\mk{k_0}(x).
    %\end{align}
    %We aim to show that the same conclusion holds for $k=k_0+1$. To wit, observe that  
    %\begin{align*}
     %   \mathbb{P}(X_{k_0+1}^{\vep}\in \,dx)&=\int \mathbb{P}(X_{k_0+1}^{\vep}\in \, dx| X_{k_0}^{\vep}=z)p_{X}\gvp_{k_0}(z)\,dz\\ &=\int \int \mathbb{P}(X_{k_0+1}^{\vep}\in x| Y_{k_0}^{\vep}=y, X_{k_0}^{\vep}=z)p_{Y|X}\gvp_{k_0}(y|z) p_X \gvp_{k_0}(z)\,dz\,dy\\ &=\int p_{X|Y}\gvp_{k_0+1}(x|y)\left(\int \gvp_{k_0}(z,y)\,dz\right)\,dy\\ &=\int \exp\left(\frac{1}{\vep}\langle x,y\rangle - \frac{1}{\vep}\opV[u_{k_0}^{\vep}](y)-\frac{1}{\vep}u_{k_0}^{\vep}(x)-f(x)-g(y)\right)\,dy\\ &=\exp\left(\frac{1}{\vep}(u_{k_0+1}^{\vep}(x)-u_{k_0}^{\vep}(x))-f(x)\right)=\mk{k_0+1}(x).
   % \end{align*}
\end{proof}

It is natural to ask if the Sinkhorn Markov chain in \cref{prop:mchn} has a diffusive limit as $\vep\rightarrow 0+$. In fact, the drift and the diffusion coefficients in the SDE \eqref{eq:diffSDE} correspond to the leading order terms of the conditional expectation and the conditional variance of the one-step increment of this Markov chain. Hence, it can be conjectured that the Sinkhorn diffusion is indeed the limit of the Sinkhorn Markov chain, under suitably strong assumptions. However, the proof of this convergence appears challenging and is not taken up here. The theorem below proves the weaker statement of one-dimensional marginal convergence, i.e., the convergence of $\rho_k^{\vep}$s according to \cref{prop:Sinmar}. 




\begin{thm}\label{thm:convergence}
    Suppose that Assumption \ref{asn:solcon} holds. Fix $T>0$. For $t\in [0,T]$, recall from \cref{thm:existlin} that $\rho_{t}=\exp(-h_t)=(\nabla w_t)_{\#} e^{-g}$ satisfies the Sinkhorn flow in \eqref{eq:velocity}. Then the following holds:
    
    \begin{equation}\label{eq:step1show}
    \sup_{k\ge 1\ : \ k\vep\le T}\wass_2^2(\rho_{k\vep},\mk{k})\le C_T \vep,
    \end{equation}
    where $C_T$ is a constant depending on $T$ and the constants implicit in Assumption \ref{asn:solcon}. This implies, in particular,  
    $$ \mk{\lfloor T/\vep \rfloor}\to \rho_T, \qquad \mbox{as}\ \vep\to 0$$
    for every fixed $T>0$, in the topology of weak convergence.
\end{thm}

\begin{proof}
{\color{black}    
We set up some notation first. Let $(u_{k\vep})_{k\ge 0}$ and $(w_{k\vep}=u^*_{k\vep})_{k\ge 0}$ denote sequences corresponding to the PMA process \eqref{eq:pma} and the dual PMA process \eqref{eq:dualPMA}, restricted to time points $t=0,\vep,2\vep,\ldots $. Throughout this proof, we will write $C>0$ for a generic constant depending on $f$, $g$, $T$, $d$, but crucially not on $\vep$. Note that this constant may change from one line to another.

To establish \eqref{eq:step1show}, we also define
\[
\tilde{\xi}_{k\vep}:=\left( \nabla w_{k\vep},\mathrm{id}\right)_{\#} e^{-g},
\]
which is the law of $(Y^{w_t},Y)$ where $Y \sim e^{-g}$. Therefore $p_X \tilde{\xi}_{k\vep}=\rho_{k\vep}$. Finally, recall the definition of $\gamma^{\vep}_{k+1}(x,y)$ from \eqref{eq:coupling}.  

\vspace{0.1in}

\emph{Proof of \eqref{eq:step1show}.} In this proof, we will establish the following result:

\begin{align}\label{eq:jointcon}
\wass_2^2(\gamma_{k+1}^{\vep},\tilde{\xi}_{k\vep})\le C\vep.
\end{align}
As the $2$-Wasserstein distance between the marginals is smaller, by \eqref{eq:jointcon}, we have $\wass_2^2(p_X \gamma_{k+1}^{\vep}, \ p_X \tilde{\xi}_{k\vep})\le C\vep$. Further, As $p_X \gvp_{k+1}=\rho_k^{\vep}$ (by \eqref{eq:curtaincall}) and $p_X \tilde{\xi}_{k\vep}=\rho_{k\vep}$ (as argued above), the conclusion in \eqref{eq:step1show} will follow. Therefore, it only remains to show \eqref{eq:jointcon}.

\vspace{0.05in}

\emph{Proof of \eqref{eq:jointcon}.} We will break this proof down into $3$ steps. 

\noindent \emph{Step (a).} We will show that 
\begin{align}\label{eq:stepa}
\wass_2^2(\gamma_{k+1}^{\vep},\tilde{\xi}_{k\vep})\le \E_{\gamma_{k+1}^{\vep}} \lVert X-Y^{w_{k\vep}}\rVert^2.
\end{align}

\noindent \emph{Step (b).} The right hand side of \eqref{eq:stepa} is technically challenging to work with as it involves $\gvp_{k+1}$, which in turn involves the Sinkhorn potentials $u_k^{\vep}$'s which are not smooth as $\vep\to 0$. To circumvent this, we define a surrogate probability measure replacing $u_k^{\vep}$'s in the right hand side of \eqref{eq:stepa} with $u_{k\vep}$'s from the PMA. To wit, define 
$$
\xi^{\vep}_{k+1}(x,y):=\exp\left(\frac{1}{\vep}\langle x,y\rangle - \frac{1}{\vep}u_{k\vep}(x)-\frac{1}{\vep}\opV[u_{k\vep}](y)-f(x)-g(y)\right),
$$
We will show in step (b) that: 
\begin{align}\label{eq:stepb}
\E_{\gamma_{k+1}^{\vep}} \lVert X-Y^{w_{k\vep}}\rVert^2\le C\E_{\xi_{k+1}^{\vep}} \lVert X-Y^{w_{k\vep}}\rVert^2.
\end{align}

\noindent \emph{Step (c).} In the final step, we will leverage the smoothness of $(u_{k\vep})$'s from \cref{asn:solcon} to show that 

\begin{align}\label{eq:stepc}
\sup_{k:\ k\vep\le T} \E_{\xi_{k+1}^{\vep}} \lVert X- Y^{w_{k\vep}}\rVert^2\le C\vep.
\end{align}

Clearly, combining \eqref{eq:stepa}, \eqref{eq:stepb} and \eqref{eq:stepc} establishes \eqref{eq:jointcon}, thereby completing the proof. 

\vspace{0.05in}

\emph{Proof of step (a).} Recall that $p_Y \gvp_{k+1}=e^{-g}$ (from \eqref{eq:curtaincall}) and $p_Y \xi_{k+1}^{\vep}=e^{-g}$ (by definition). With this in mind, construct a coupling $\tilde{\pi}_{k+1}^{\vep}$ between $\gamma_{k+1}^{\vep}$ and $\tilde{\xi}_{k\vep}$ as follows: sample $(X,Y)\sim \gamma_{k+1}^{\vep}$ and let $\tilde{\pi}_{k+1}^{\vep}$ be the law of $(X,Y,Y^{w_{k\vep}},Y)$. By definition of $2$-Wasserstein distance (see \eqref{eq:2wass}), we then have:
\begin{align}\label{eq:jointcon1}
    \wass_2^2(\gamma_{k+1}^{\vep},\tilde{\xi}_{k\vep})\le \E_{\tilde{\pi}_{k+1}^{\vep}} \lVert X-Y^{w_{k\vep}}\rVert^2=\E_{\gamma_{k+1}^{\vep}} \lVert X-Y^{w_{k\vep}}\rVert^2.
\end{align}
This establishes step (a).

\vspace{0.05in} 

\emph{Proof of step (b).} For proving step (b), we need the following  preparatory lemma. 

\begin{lmm}\label{lem:erboundmain}
Suppose \cref{asn:solcon} holds. Define $$a_k^{\vep}(x):=\frac{1}{\vep}(u_k^{\vep}-u_{k\vep})(x),\quad b_k^{\vep}(y):=\frac{1}{\vep}(\opV[u_k^{\vep}]-\opV[u_{k\vep}])(y).$$
Then there exists $C>0$ such that 
\begin{align*}
    \sup_{k:\ k\vep\le T} \left(\lVert a_k^{\vep}\rVert_{\infty}+\lVert b_k^{\vep}\rVert_{\infty}\right)\le C.
\end{align*}
\end{lmm}

From the above definitions of $a_k^{\vep}$ and $b_k^{\vep}$, the following relation is immediate:
\begin{align}\label{eq:funrel}
\gamma^{\vep}_{k+1}(x,y)= \xi^{\vep}_{k+1}(x,y) \exp\left(-a_k^\vep(x) - b_k^\vep(y)\right).
\end{align}
An application of \cref{lem:erboundmain} then yields
$$\gamma_{k+1}^{\vep}\le C\xi_{k+1}^{\vep}.$$
This readily implies \eqref{eq:stepb}. We remind the reader here that the constant $C$ is changing from line to line. Therefore, $C$ in the above display is not the same as that in \cref{lem:erboundmain}.

\vspace{0.05in}


\emph{Proof of step (c).} To establish step (c), we need the following lemma. 
\begin{lmm}\label{lem:pmaboundmain}
Suppose \cref{asn:solcon} holds and $\vep \in (0,1)$.  
Then there exists $C>0$ such that 
\begin{align*}
    \sup_{\substack{k:\ k\vep\le T,\\ y\in \R^d}} \left| \opV[u_{k\vep}](y)-w_{k\vep}(y)-\frac{\vep d}{2}\log{(2\pi\vep)}+\vep f(y^{w_{k\vep}})-\frac{\vep}{2}\ldet\left(\frac{\partial y^{w_{k\vep}}}{\partial y\hfill}\right) \right|\le C\vep^2.
\end{align*}
\end{lmm}

The following function (often referred to as the \emph{Bregman divergence function}) will help us simplify notation in the proof of step (c).
\begin{equation}\label{eq:bregdiv}
\mathcal{D}[u_{k\vep}](x|y):=u_{k\vep}(x)+w_{k\vep}(y)-\langle x,y\rangle\ge 0.
\end{equation}

Now the left hand side of \eqref{eq:stepc} (barring the supremum) simplifies as 

\begin{align*}
&\;\;\;\;\E_{\xi_{k+1}^{\vep}} \lVert X-Y^{w_{k\vep}}\rVert^2\nonumber \\ &=\int \int \lVert x-y^{w_{k\vep}}\rVert^2 \exp\left(\frac{1}{\vep}\langle x,y\rangle - \frac{1}{\vep} u_{k\vep}(x)-\frac{1}{\vep}\opV[u_{k\vep}](y)-f(x)-g(y)\right)\,dx\,dy\nonumber \\ &\le C \int \int \frac{1}{(2\pi\vep)^{d/2}}\sqrt{\mathrm{det}\left(\frac{\partial y\hfill}{\partial y^{w_{k\vep}}}\right)}\lVert x-y^{w_{k\vep}}\rVert^2\times \\ \;\;\;\;&\exp\bigg(-\frac{1}{\vep}\mathcal{D}[u_{k\vep}](x|y)+f(y^{w_{k\vep}})-f(x)-g(y)\bigg)\,dx\,dy
\end{align*}
where the last inequality follows from \cref{lem:pmaboundmain}. 

\noindent By a Taylor expansion of $u_{k\vep}(x)$ around the point $y^{w_{k\vep}}$, coupled with \cref{asn:solcon}, (i), we get:
\begin{equation}\label{eq:estimpf1}
\mathcal{D}[u_{k\vep}](x|y)\ge \frac{A_T}{2}\lVert x-y^{w_{k\vep}}\rVert^2.
\end{equation}

Combining the two displays above with \cref{asn:solcon}, part (ii),  it follows that: 

\begin{align*}
&\;\;\;\;\;\E_{\xi_{k+1}^{\vep}} \lVert X-Y^{w_{k\vep}}\rVert^2\\ &\le \frac{C B_T^{\frac{d}{2}}}{(2\pi\vep)^{\frac{d}{2}}} \int \int \lVert x-y^{w_{k\vep}}\rVert^2 \exp\left(-\frac{A_T}{2\vep}\lVert x-y^{w_{k\vep}}\rVert^2+f(y^{w_{k\vep}})-f(x)-g(y)\right)\,dx\,dy.
\end{align*}

Next, we focus on the inner integral above (the one with respect to $x$). We drop the $e^{-g(y)}$ term and adjust some constants, all of which are free of $x$ and reproduce the rest below:

\begin{align*}
    &\;\;\;\;\frac{1}{(2\pi\vep A_T)^{\frac{d}{2}}} \int \lVert x-y^{w_{k\vep}}\rVert^2 \exp\left(-\frac{A_T}{2\vep}\lVert x-y^{w_{k\vep}}\rVert^2+f(y^{w_{k\vep}})-f(x)\right)\,dx\\ &\le \frac{1}{(2\pi\vep A_T)^{\frac{d}{2}}} \int \lVert x-y^{w_{k\vep}}\rVert^2 \exp\left(-\frac{A_T}{2\vep}\lVert x-y^{w_{k\vep}}\rVert^2+\lVert x-y^{w_{k\vep}}\rVert_1 \lVert \nabla f\rVert_{\infty}\right)\,dx\\ &=\E\left[\lVert Z_{\vep}-y^{w_{k\vep}}\rVert^2 \exp\left(\lVert Z_{\vep}-y^{w_{k\vep}}\rVert_1 \lVert \nabla f\rVert_{\infty}\right)\right],
\end{align*}
where $Z_{\vep,y}\sim N(y^{w_{k\vep}},\vep A_T^{-1} \mathrm{I}_d)$. 

As $\lVert Z_{\vep,y}-y^{w_{k\vep}}\rVert$ is distributed according to a $\sqrt{\vep A_T^{-1}}\chi_d$ random variable. By a standard $\chi_d$ tail bound (see example \cite[Lemma 1]{laurent2000adaptive}), it follows that:
$$\E\left[\lVert Z_{\vep}-y^{w_{k\vep}}\rVert^2 \exp\left(\lVert Z_{\vep}-y^{w_{k\vep}}\rVert_1 \lVert \nabla f\rVert_{\infty}\right)\right]\le C\vep.$$
By combining the above observations, we then get:
$$\E_{\xi_{k+1}^{\vep}} \lVert X-Y^{w_{k\vep}}\rVert^2\le C\vep \int \exp(-g(y))\,dy=C\vep.$$
As the above bound is uniform over $k$ such that $k\vep\le T$, the conclusion in \eqref{eq:stepc} follows.
}
\end{proof}

\begin{comment}
 Since $\sup_n\mathrm{KL}(\gamma_{k_n}', \xi_{k_n}')< \infty$, it follows that $\sup_n\mathbb{W}_1\left( \gamma_{k_n}', \xi_{k_n}' \right)< \infty$. In particular, $(\gamma_{k_n}',\; n \in \NN)$ is a tight sequence of probability measures. Thus it is relatively compact in weak topology. 

Consider any limit point $\gamma_\infty'$.
Since KL divergence is lower semicontinuous in its arguments in the weak topology, it follows that 
\[
\mathrm{KL}\left( \gamma_\infty', \xi_\infty' \right)\le \liminf_n \mathrm{KL}(\gamma_{k_n}', \xi_{k_n}')< \infty.
\]
Since the $Y$ marginal of $\gamma'_\infty$ must be the limit of the $Y$ marginals of $\gamma'_{k_n}$, it is again $e^{-g}$. Thus 
\begin{enumerate}
\item $\xi_\infty'$ must be supported on the graph of a function, i.e., it must be of the form $(T, \mathrm{id})_{\# e^{-g}}$. 
\item Since the only way the pushforward by two graphs $(\mathrm{id}, T_1)$ and $(\mathrm{id}, T_2)$ of the same measure $e^{-g}$ can have finite KL for the graphs to be the same $e^{-g}$ almost surely.
\end{enumerate}

Thus, necessarily $\gamma_\infty'=\xi_\infty'$ as measures. Hence the marginal law of the $X$ coordinates are also the same. Thus $\lim_{n\rightarrow \infty} \rho^{\vep}_{k_n}= \rho_t$, and we are done. 
\end{comment}

%\begin{remark}
%A similar proof technique would also show convergence of $\rho_k^{\vep}$ to $\rho_{k\vep}$ in the Kullback-Leibler divergence (see \eqref{eq:KLdef}).
%\end{remark}

\noindent In other words, \cref{thm:convergence} shows that the marginals from the Sinkhorn algorithm \eqref{eq:sinkupdt} converge to the marginals of the limiting dynamics governed by the Sinkhorn PDE (see \eqref{eq:velocity}). In light of \cref{sec:mirror}, the above result shows the connection between the Sinkhorn algorithm and the Wasserstein mirror flow of entropy (in continuum) with respect to the mirror $U(\cdot)$ which maps $\rho\mapsto \frac{1}{2}\wass_2^2\left(\rho, e^{-g}\right)$ (see \eqref{eq:mirror}).

\begin{remark}
    Recall that $\gamma_{k+1}^{\vep}$ is the Schr\"{o}dinger Bridge coupling between $\mk{k}$ and $e^{-g}$. From the proof of \cref{thm:convergence} (see in particular \eqref{eq:jointcon}), we actually obtain a quantitative convergence bound on this coupling too. In particular, it holds that $\wass_2^2(\gamma_{k+1}^{\vep},(\nabla w_{k\vep},\mathrm{id})_{\#} e^{-g})\le \vep$ uniformly over $1\le k\le \lfloor T/\vep\rfloor$.
\end{remark}


\begin{comment}
\section{Deviation from the PMA}\label{sec:error}


In this section we generalize the proof of Berman to quantify deviations of Sinkhorn from the PMA.

For this analysis let $(u_t,\; t\ge 0)$ be the solution of the PMA. Also, consider the following two transition probabilities. 
\[
\begin{split}
P^u_\vep(x \mid y) &= \frac{1}{(2\pi \vep)^{d/2}}\exp\left(  -\frac{1}{\vep} u(x) - f(x) - \frac{1}{2\vep} \norm{x-y}^2 - \frac{1}{\vep}\opV[u](y) \right)\\
\end{split}
\]
And
\[
Q^v_{\vep}(y \mid z) = \frac{1}{(2\pi \vep)^{d/2}}\exp\left(  -\frac{1}{\vep} v(y) - g(y) - \frac{1}{2\vep} \norm{y-z}^2 - \frac{1}{\vep}\opU[v](z) \right).
\]


When $u=u_t$, for some $t$, under appropriate regularity assumptions on the PMA, we have shown the following Gaussian approximations. 
For example,
\begin{equation}\label{eq:gapproxp}
\begin{split}
P^{\opS[u]}_\vep(\cdot\mid y) &\approx 
N\left( y^*-\vep  \left(2 \frac{\partial f}{\partial y}(y^*) - \frac{\partial h}{\partial y}(y^*)\right), \; \vep \left(\frac{\partial y^*}{\partial y}\right) \right),
\end{split}
\end{equation}
Thus the Markov transition kernel 
\begin{equation}\label{eq:gapproxq}
Q^{\opV[u]}_\vep(\cdot \mid z) \approx N\left( z_* + \vep \frac{\partial}{\partial z} f(z)- \frac{\vep}{2} \frac{\partial}{\partial z} g(z_*)- \frac{\vep}{2} \frac{\partial}{\partial z} h(z), \vep \left(\frac{\partial z_*}{\partial z}\right)\right).
\end{equation}
The errors for both these approximations are $o(\epsilon)$.

\subsection{From Log Sinkhorn Iterations to Sinkhorn}

We will now perform a \textit{change of measure} tilting to the log-Sinkhorn iterates. We proceed by induction. 

Suppose after $k$ iterations, the joint density in Sinkhorn is given by 
\[
\gamma_k'(x,y)=\frac{1}{(2\pi \vep)^{d/2}}\exp\left(  - \frac{1}{2\vep} \norm{y-x}^2 - \frac{1}{\vep}u^\vep_k(x) - \frac{1}{\vep} v^\vep_{k+1}(y)-f(x) - g(y)\right),
\]
where $v^\vep_{k+1}=\opV[u^\vep_k]$. Thus, the $Y$ marginal is exactly $e^{-g}$, while we denote the $X$ marginal by $\rho_k$. We are going to compare the $X$ marginal of this joint density with another one computed from the PMA. 

Recursively define the following quantities which are functions of the PMA. 
\[
\hat{u}_{(k+1)\vep}:= \opS[u_{k\vep}], \quad \hat{v}_{(k+1)\vep}:= \opV[u_{k\vep}].
\]
Let $\xi_k'$ be the joint density 
\[
\xi_k'(x,y)= \frac{1}{(2\pi \vep)^{d/2}} \exp\left( -\frac{1}{2\vep}\norm{x-y}^2 - \frac{1}{\vep} {u}_{k\vep}(x) - \frac{1}{\vep} \hat{v}_{(k+1)\vep}(y) - f(x) - g(y) \right).
\]
It is easy to see that the above joint density has $Y$-marginal $e^{-g}$ and two conditional densities given by $q_\vep^{\opV[u_{k\vep}]}(\cdot \mid x)$ and $p^{u_{k\vep}}(\cdot \mid y)$. 
These are the densities of the probability measures \eqref{eq:gapproxq} and \eqref{eq:gapproxp}, respectively. The $X$-marginal of $\xi_k'$ is given by 
\[
\rho^*_k(x)=\exp\left( \frac{1}{\vep}\hat{u}_{(k+1)\vep} - \frac{1}{\vep} u_{k\vep} - f(x) \right),
\]
while we denote the $X$-marginal of $\gamma_k'$ by $\rho^\vep_k$. 

Now, there exists $(a^\vep_k, b^\vep_k)$ such that 
\[
\gamma_k'(x,y)= \xi_k'(x,y) e^{-a^\vep_k(x) - b^\vep_k(y)}. 
\]
This is because if we let $a^\vep_k(x):=\frac{1}{\vep}\left(u^\vep_k(x)- u_{k\vep}(x)\right)$, then
\[
\begin{split}
b^\vep_k(y):=\frac{1}{\vep}\left(v^\vep_{k+1}(y)-\hat{v}_{(k+1)\vep}\right)&= \log \int e^{-a^\vep_k(x)} p_\vep^{u_{k\vep}}(x\mid y)dx. 
\end{split}
\]


Thus, the $X$-marginal is given by 
\[
\begin{split}
\rho^\vep_k(x)&= e^{-a^\vep_k(x)} \int e^{-b^\vep_k(y)} \xi_k'(x,y)dy= e^{-a^\vep_k(x)} \rho_k^*(x) \int e^{-b^\vep_k(y)} q_\vep^{\hat{v}_{k+1}}(y\mid x)dy\\
&={\color{red}\rho_k^*(x) e^{-a^\vep_k(x)}\int e^{-a^\vep_k(x)}\left( \int p_\vep^{u_{k\vep}}(x\mid y) q_\vep^{\hat{v}_{k+1}}(y\mid x)dy\right) dx}\\
&= e^{\hat{a}^\vep_{k+1}(x) - a^\vep_k(x)} \rho_k^*(x),
\end{split}
\]
where
{\color{red}\[
\hat{a}^\vep_{k+1}(x)=\log\int e^{-a^\vep_k(x)} \int p_\vep^{u_{k\vep}}(x\mid y) q_\vep^{\hat{v}_{k+1}}(y\mid x)dydx.
\]}

{\color{red} The integral with $p_\vep^{u_{k\vep}}(x\mid y)$ should be in the denominator??} 

Now, the $X$ marginal density from the PMA is given by approximately 
\[
\rho_{k\vep}= \exp\left( \frac{1}{\vep}\left( u_{(k+1)\vep} - u_{k\vep} \right) - f(x) + o_\vep(1)\right).
\]
Thus 
\[
\rho^\vep_k(x)= \exp\left( \hat{a}^\vep_{k+1}(x) - a^\vep_k(x) + \frac{1}{\vep}\left( \hat{u}_{(k+1)\vep} - u_{(k+1)\vep}\right) +o_\vep(1) \right) \rho_{k\vep}
\]


\textbf{Goal:} Our goal is to show that the term $\rho_k(x) \rightarrow \rho_{k\vep}$ as $\vep \rightarrow 0+$ and $k=O(1/\vep)$.  



Towards this goal we compare between $a^\vep_{k+1}$ and $\hat{a}^\vep_{k+1}$. 
By definition
\[
\begin{split}
a^\vep_{k+1}(x)&= \frac{1}{\vep} \left( u^\vep_{k+1}(x) - u_{(k+1)\vep}(x)\right)=\frac{1}{\vep} \left( \opU[v^\vep_{k+1}](x) - u_{(k+1)\vep}(x)\right).
\end{split}
\]
By a similar calculation as above
\[
\begin{split}
\opU[v^\vep_{k+1}](x)&=\vep \log \int \frac{1}{(2\pi \vep)^{d/2}} \exp\left( -\frac{1}{2\vep}\norm{y-x}^2 - \frac{1}{\vep}v^\vep_{k+1}(y) - g(y)\right)dy\\
=&\vep \log \int \frac{1}{(2\pi \vep)^{d/2}} \exp\left( -\frac{1}{2\vep}\norm{y-x}^2 - \frac{1}{\vep}\hat{v}^\vep_{k+1}(y) - b^\vep_k(y) - g(y)\right)dy\\
=& \hat{u}_{(k+1)\vep}(x) + \vep \log \int e^{-b^\vep_k(y)} q_\vep^{\hat{v}^\vep_{k+1}}(y\mid x)dy\\
=& {\color{red}\hat{u}_{(k+1)\vep}(x) + \vep (\hat{a}_{k+1}(x) - a^\vep_k(x))}. 
\end{split}
\]
{\color{red} Should it just be $\hat{a}_{k+1}^{\vep}(x)$ instead of $(\hat{a}^{\vep}_{k+1}(x) - a^\vep_k(x))$??}
On the other hand, the terms below are purely functions of the PMA. 
\[
\frac{1}{\vep}\left( \hat{u}_{k+1}(x) - u_{(k+1)\vep}(x) \right)=\frac{1}{\vep}\left( \opS[u_{(k\vep)}] - u_{(k+1)\vep} \right).
\]
Thus, by adding the two terms above, we have derived
\[
a^\vep_{k+1}(x)= (\hat{a}^\vep_{k+1}(x) - a^\vep_k(x)) - \frac{1}{\vep}\left( u_{(k+1)\vep} - \opS[u_{(k\vep)}] \right).
\]
Thus
\[
\rho^\vep_k(x)= e^{a_{k+1}^\vep(x) + o(1)} \rho_{k\vep}(x).
\]

The strategy is now to replicate the proof of Berman with these lower order potentials $(a^\vep_k, b^\vep_k)$. 
Start with the initial condition $a^\vep_0=0$.
We show that this new sequence of potentials $(a^\vep_k,b^\vep_k)$ satisfies a new recursive formula that converges to zero as $\vep \rightarrow 0$.  We show below that both sequences are $o(\vep)$. 

Let 
\[
\delta_{(k+1)\vep}:= \frac{1}{\vep}\left( u_{(k+1)\vep} - \opS[u_{(k\vep)}] \right).
\]
We know that $\delta_{(k+1)\vep}=o(\vep)$. Since $a^\vep_0=0$, then $\hat{a}^\vep_1=0$. Thus 
\[
a^\vep_1=-\delta_{\vep}=o(\vep). 
\]
It follows by induction that $a^\vep_k=k o(\vep)$. Thus, for $k=O(1/\vep)$, $a^\vep_k\rightarrow 0$. 

It is also immediate that the argument works if $a^\vep_0=o(\vep)$, so it should also give us multidimensional convergence. 

\end{comment}













\section{Proof of technical results}\label{sec:pfres}
In this section we will prove the auxiliary results from the main paper whose proofs we deferred, except for Lemmas \ref{lem:erboundmain} and \ref{lem:pmaboundmain} (which will be proved in the next section). 
\begin{proof}[Proof of~\cref{lem:tensorel}]
By~\eqref{eq:conjrel}, the following matrix relationship holds:
\begin{align*}
\left(\secphx\right)\left(\secph\right)=I_d.
\end{align*}
Therefore, by expanding the matrix product, for any $i,j\in [d]^2$, we get:
\begin{align}\label{eq:tensorel1}
\sum_{\ell} \secphxil \cdot \secphlj=\sum_{\ell} \left(\secphx\right)_{i,\ell}\left(\secph\right)_{\ell,j}=\delta_{i,j},
\end{align}
where $\delta_{i,j}=0$ if $i\neq j$ and $\delta_{i,i}=1$. 

\noindent Fix $k\in [d]$. By differentiating the LHS  of~\eqref{eq:tensorel1} with respect to $\xsph_k$, i.e., applying the operator $\frac{\partial}{\partial \xsph_{k}}$, we get:
\begin{align}\label{eq:tensorel2}
\frac{\partial}{\partial \xsph_{k}}\left(\sum_{\ell} \secphxil \cdot \secphlj\right)&=\sum_{\ell} \frac{\partial}{\partial \xsph_{k}} \left(\secphxil\right)\cdot \secphlj+\sum_{\ell} \secphxil \cdot \frac{\partial}{\partial \xsph_k} \left(\secphlj\right)\nonumber \\
&=\sum_{\ell,m} \frac{\partial}{\partial x_{m}}\left(\secphxil\right)\cdot\secphmk \cdot \secphlj+\sum_{\ell} \secphxil \cdot \frac{\partial}{\partial \xsph_k} \left(\secphlj\right)\nonumber \\&=\sum_{\ell,m} \frac{\partial}{\partial x_{\ell}}\left(\secphxim\right)\cdot\secphmk \cdot \secphlj+\sum_{\ell} \secphxil \cdot \frac{\partial}{\partial \xsph_k} \left(\secphjl\right).
\end{align}
Here the first equality uses the product rule of derivatives. The second equality uses the chain rule of derivatives on the first term (the second term is intact here). Finally the third display follows by noting that $\frac{\partial}{\partial x_{m}}\left(\secphxil\right)=\frac{\partial}{\partial x_{\ell}}\left(\secphxim\right)$ and $\frac{\partial x}{\partial x^{\phi}}$ is a symmetric matrix (both of which follow from the $\diffcont^2$ diffeomorphism assumption on $\nabla \phi$). Next note that the derivative of the right hand side of \eqref{eq:tensorel1} with respect to $x_k^{\phi}$ is $0$. With this observation, by  combining~\eqref{eq:tensorel1}~and~\eqref{eq:tensorel2}, we get:
\begin{align*}
\sum_{\ell} \secphxil \cdot \frac{\partial}{\partial \xsph_k} \left(\secphjl\right)=-\sum_{\ell,m} \frac{\partial}{\partial x_{\ell}}\left(\secphxim\right)\cdot\secphmk \cdot \secphlj.
\end{align*}
By choosing $k=i$, summing up over $i$, we get:
\begin{align}\label{eq:tenso1}
\sum_{\ell} \frac{\partial}{\partial x_{\ell}}\left(\secphjl\right)=-\sum_{i,\ell,m} \frac{\partial}{\partial x_{\ell}}\left(\secphxim\right)\cdot\secphmi \cdot \secphlj.
\end{align}

\noindent Note that the left hand side of \eqref{eq:tenso1} is same as the right hand side of \eqref{eq:tensorelpf}. Let us now show that the right hand side of \eqref{eq:tenso1} matches the left hand side of \eqref{eq:tensorelpf}.

\noindent To achieve this, note that for a symmetric positive definite matrix $A$, by \cite[Section A.4.1]{boyd2004convex}, we have $\nabla_A \ldet(A)= A^{-1}$. Here the gradient is computed entry-wise. By using the above observation with $A=\frac{\partial x^{\phi}}{\partial x\hfill}$ (so $A^{-1}=\frac{\partial x\hfill}{\partial x^{\phi}}$), we get
\begin{align*}
\frac{\partial}{\partial \xsph_j} \ldet \left(\secphx\right)&= \sum_{i,m} \frac{\partial}{\partial \xsph_j} \left( \secphxim \right) \secphmi &= \sum_{i,\ell,m} \frac{\partial}{\partial x_{\ell}}\left(\secphxim\right)\cdot\secphlj\cdot\secphmi .
\end{align*}
The last equality above follows using the chain rule of derivatives. Coupling the above observation with \eqref{eq:tenso1}  completes the proof.
\end{proof}

\begin{proof}[Proof of \cref{lem:dualPMA}]
As $w_t=u_t^*$ with $u_t$ being the solution from \eqref{eq:pma}, given $y\in\R^d$, the following holds:
$$w_t(y)+u_t(y^{w_t})-\langle y, y^{w_t}\rangle=0.$$
By taking partial derivatives with respect to $t$ on both sides above, we further have:
\begin{align}\label{eq:dualPM1}
    \frac{\partial w_t}{\partial t\hfill}(y)+\frac{\partial}{\partial t}(u_t(y^{w_t}))-\iprod{y,\frac{\partial y^{w_t}}{\partial t\hfill}}=0
\end{align}
The second term in the left hand side of \eqref{eq:dualPM1} needs further simplification. To wit, by using the chain rule of derivatives, we get:
$$\frac{\partial}{\partial t}(u_t(y^{w_t}))=\frac{\partial}{\partial t}u_t(y^{w_t})+\iprod{(y^{w_t})^{u_t},\frac{\partial y^{w_t}}{\partial t\hfill}}=\frac{\partial}{\partial t}u_t(y^{w_t})+\iprod{y,\frac{\partial y^{w_t}}{\partial t\hfill}}.$$
In the last equality above, we have used the observation that $\nabla u_t(\nabla w_t(y))=y$. By coupling the above observation with \eqref{eq:dualPM1} we get
$$\frac{\partial w_t}{\partial t\hfill}(y)=-\frac{\partial}{\partial t}u_t(y^{w_t})=g((y^{w_t})^{u_t})-f(y^{w_t})-\ldet\left(\frac{\partial x^{u_t}}{\partial x\hfill}\right)\bigg|_{x=y^{w_t}}.$$
In the last equality we have used the PMA \eqref{eq:pma}. The conclusion of the lemma now follows by combining the above display with the following observations:
$$(y^{w_t})^{u_t}=y,\quad\quad \ldet\left(\frac{\partial x^{u_t}}{\partial x\hfill}\right)\bigg|_{x=y^{w_t}}=-\ldet\left(\frac{\partial y^{w_t}}{\partial y\hfill}\right).$$
\end{proof}

\begin{proof}[Proof of  \cref{lem:convexcall}]
By \eqref{eq:labelgrad}, $\nabla w_t(y)+\delta v_t(y^{w_t})$ is the gradient of the following function with respect to $y$
$$\Lambda_t(y):=w_t(y)+\delta \left(g(y)-f(y^{w_t})+\ldet\left(\frac{\partial y^{w_t}}{\partial y\hfill}\right)\right).$$
It remains to prove that $y\mapsto \Lambda_t(y)$ is convex. Towards this direction, we first assume the following claim and complete the proof.
\begin{align}\label{eq:eigbound}
    \sup_y \lVert \nabla^2 \Lambda_t(y)\rVert_{\mathrm{op}}<\infty,
\end{align}
where $\lVert \cdot \rVert_{\mathrm{op}}$ denotes the $L^2$ operator norm of a matrix (see~\cref{tab:table3}). Recall from \cref{asn:solcon}, part (i) that $\inf_y \lmn\left(\frac{\partial y^{w_t}}{\partial y\hfill}\right)>0$. Therefore there exists $\delta>0$ such that $$\delta < \ \frac{\inf_y \lmn\left(\frac{\partial y^{w_t}}{\partial y\hfill}\right)}{2\sup_y \lVert \nabla^2 \Lambda_t(y)\rVert_{\mathrm{op}}}.$$ 
By using Weyl's inequality (see \cite[Theorem 3.3.16]{horn1994topics}), we then get:
$$\inf_y \lmn(\Lambda_t(y))\ge \inf_y \lmn\left(\frac{\partial y^{w_t}}{\partial y\hfill}\right)-\delta \sup_y\lVert \Lambda_t(y)\rVert_{\mathrm{op}}>\frac{1}{2}\inf_y \lmn\left(\frac{\partial y^{w_t}}{\partial y\hfill}\right)>0.$$

This establishes the convexity of $y\mapsto \Lambda_t(y)$ and  completes the proof of \cref{lem:convexcall}. Therefore, it only remains to prove \eqref{eq:eigbound}. 

\emph{Proof of \eqref{eq:eigbound}.} By \cref{asn:solcon}, parts (i) and (iii), $\sup_y \lVert w_t(y)\rVert_{\mathrm{op}}<\infty$ and $\sup_y \lVert \nabla^2 g(y)\rVert_{\mathrm{op}}<\infty$. Similarly for $i,j\in [d]^2$, we get 
$$\frac{\partial^2}{\partial y_i\partial y_j}(f(y^{w_t}))=\sum_{\ell}\frac{\partial^3}{\partial y_i \partial y_j \partial y_{\ell}} w_t(y) \frac{\partial}{\partial y_{\ell}} f(y^{w_t})+\sum_{\ell,m}\left(\frac{\partial y_{\ell}^{w_t}}{\partial y_j\hfill}\right)\left(\frac{\partial y_m^{w_t}}{\partial y_i\hfill}\right)\frac{\partial^2}{\partial y_m\partial y_{\ell}}f(y^{w_t}).$$
Here we have used a combination of the product rule and the chain rule. Similar calculations were done in the proofs of \cref{thm:existlin} in the main paper, and so we skip the details for brevity. By the uniform boundedness of the first two derivatives of $f$ and the first $3$ derivatives of $w_t$ from \cref{asn:solcon}, part (iii) (also see \cref{rem:dualasn}), we immediately get:
$$\max_{i,j\in [d]^2}\sup_{y}\bigg|\frac{\partial^2}{\partial y_i\partial y_j}(f(y^{w_t}))\bigg|<\infty.$$
A similar computation also yields
\begin{align*}
    \frac{\partial^2}{\partial y_i \partial y_j}\left(\ldet\left(\frac{\partial y^{w_t}}{\partial y\hfill}\right)\right)&=\sum_{k,\ell}\left(\frac{\partial y_k\hfill}{\partial y_{\ell}^{w_t}}\right)\frac{\partial^4}{\partial y_i\partial y_j\partial y_k\partial y_{\ell}} w_t(y)\\ &\;\;\;+\sum_{k,\ell,m}\frac{\partial^3}{\partial y_k\partial y_{\ell}\partial y_m}u_t(y^{w_t})\frac{\partial^3}{\partial y_k\partial y_{\ell}\partial y_j}w_t(y)\left(\frac{\partial y_m^{w_t}}{\partial y_i}\right).
\end{align*}
By the uniform boundedness of the first two derivatives of $f$ and the first $4$ derivatives of $w_t$ from \cref{asn:solcon}, part (iii) (also see \cref{rem:dualasn}), we immediately get:
$$\max_{i,j\in [d]^2}\sup_y \Bigg|\frac{\partial^2}{\partial y_i \partial y_j}\left(\ldet\left(\frac{\partial y^{w_t}}{\partial y\hfill}\right)\right)\Bigg|<\infty.$$
Combining these observations, \eqref{eq:eigbound} follows.
\end{proof}

%\begin{proof}[Proof of \cref{cl:normalize}]
%Note that 
%\begin{align}\label{eq:claim11}
%\rho[u](x)&=\int \exp\left(-\frac{1}{\vep}u(x)-\frac{1}{2\vep}\norm{ x-y}^2\right)\exp\left(-\frac{1}{\vep}\opV[u](y)\right)d\nu(dy)\nonumber \\ 
%&=\int_y \frac{\exp(-(2\vep)^{-1}\norm{ x-y}^2 -(\vep)^{-1}u(x))}{\int_z \exp(-(2\vep)^{-1}\lVert y-z\rVert^2-(\vep)^{-1}u(z))\, d\mu(z)}\,d\nu(y).
%\end{align}
%By Tonelli's Theorem, interchanging the order of integration, we get:
%\begin{align*}
%\int \rho[u](x)\, d\mu(x)&=\int_y \frac{\int_x\exp(-(2\vep)^{-1}\lVert x-y\rVert^2-u(x))\, d\mu(x)}{\int_z \exp(-(2\vep)^{-1}\lVert y-z\rVert^2/2-u(z))\, d\mu(z)} d\nu(y)=\int_y  d\nu(y)=1.
%\end{align*}
%This completes the proof of the Proposition.
%\end{proof}

\section{Proof of Lemmas \ref{lem:erboundmain} and \ref{lem:pmaboundmain}}\label{sec:mainresultlems}
Recall the definitions of $\opU$ and $\opV$ from \eqref{eq:basedef}. 
\cref{lem:pmaboundmain} is a Laplace approximation applied to the integral operator $\opV$. A similar estimate can be found in \cite[Lemma 4.2]{berman2020} under different assumptions. On the other hand, \cref{lem:erboundmain} is a triangular approximation argument which is motivated from the proof of \cite[Lemma 4.4]{berman2020}. We will actually prove Lemmas \ref{lem:erboundmain} and \ref{lem:pmaboundmain} in the reverse order. We need to set up some notation first. For $k\ge 0$, consider the Bregman divergence from \eqref{eq:bregdiv} for the convex function $u_{k\vep}(\cdot)$ and $x,y\in\R^d$, 
\[
\mcD[u_{k\vep}](x|y) = u_{k\vep}(x) + u_{k\vep}^*(y) - \langle x,y\rangle.
\]
Note that $\mcD[u_{k\vep}^*](y|x)=\mcD[u_{k\vep}](x|y)$. Also given a sufficiently smooth function $h:\R^d\to\R$ and any $r\in \mathbb{N}$, define 
    \begin{align}\label{eq:taylornot}
        T[h:r](x|y):=\sum_{|\alpha|=r}\frac{D^{\alpha} h(y)}{\alpha!}(x-y)^{\alpha},
    \end{align}
    and 
    $$R[h:r](x|y):=\sum_{|\beta|=r}\frac{r}{\beta!}\left(\int_0^1 (1-t)^{r-1} D^{\beta} h(y+t(x-y))\,dt\right)(x-y)^{\beta}-T[h:r](x|y).$$
    Here, the $D^{\alpha}$ (or $D^{\beta}$) operators denote the standard multivariate differential operators given a nonnegative multi-index $\alpha=(\alpha_1,\ldots ,\alpha_d)$ (respectively $\beta=(\beta_1,\ldots ,\beta_d)$). Also $|\alpha|=\sum_{i=1}^d \alpha_i$, $|\beta|=\sum_{i=1}^d \beta_i$ and, as usual, $(x-y)^\alpha=\prod_{i=1}^d (x_i-y_i)^{\alpha_i}$ and so on.
    
    In other words, $T[h:r](x|y)$ denotes the $r$-th polynomial in the Taylor series expansion of $h$ at the point $x$ around the point $y$. On the other hand, $R[h:r](x|y)$ denotes the corresponding remainder term after a Taylor expansion of the function $h$ up to the $r$-th order term (at the point $x$  around the point $y$). 

    
    Finally, we note the following canonical estimates:
    \begin{equation}\label{eq:gradbound3}
    |T[h:r](x|y)|\le  C_r\lVert \nabla^r h\rVert_{\infty}\lVert x-y\rVert^{r},
    \end{equation}
    and
    \begin{equation}\label{eq:gbd4}
    |R[h:r](x|y)|\le C_r \lVert x-y\rVert^r \sup_{\lVert z_1-z_2\lVert \le \lVert x-y\rVert}\lVert \nabla^r h(z_1)-\nabla^r h(z_2)\rVert_{\infty},
    \end{equation}
    where $C_r>1$ is a universal constant (depending on $d$) that is free of $h$, $x$, and $y$. Here $\lVert \nabla^r h\rVert_{\infty}$ denotes the maximum of the supremum norms of every component function in the $r$-th order multiderivative. 


\begin{proof}[Proof of \cref{lem:pmaboundmain}]
Throughout the proof we always assume $k\vep\le T$. Further, $C$ will denote constants (possibly different in various steps) depending only on $d$ and all the other constants implicit in \cref{asn:solcon}. 

Let $x,y\in\R^d$. Recall from \eqref{eq:estimpf1} that 
$\mathcal{D}[u_{k\vep}](x|y)\ge \frac{A_T}{2}\lVert x-y^{w_{k\vep}}\rVert^2$, for some positive constant $A_T$. We are now in a position to simplify $\opV$. 
    
    \vspace{0.1in}
    
    By an algebraic identity using the definition of $\opV$ from \eqref{eq:basedef}, we observe that 
    \begin{align}\label{eq:simpl1}
       &\;\;\;\;\frac{\sqrt{\mathrm{det}(\nabla^2 u_{k\vep}(y^{w_{k\vep}}))}}{(2\pi\vep)^{\frac{d}{2}}}\exp\left(\frac{1}{\vep}\opV[u_{k\vep}](y)-\frac{1}{\vep}w_{k\vep}(y)+f(y^{w_{k\vep}})\right)\nonumber \\ &=\frac{\sqrt{\mathrm{det}(\nabla^2 u_{k\vep}(y^{w_{k\vep}}))}}{(2\pi\vep)^{\frac{d}{2}}}\int \exp\bigg(-\frac{1}{\vep}\mcD[u_{k\vep}](x|y)-f(x)+f(y^{w_{k\vep}})\bigg)\,dx.
    \end{align}
    Define 
    $$r_{\vep}:=\sqrt{-40d A_T^{-1} \vep\log{\vep}},\qquad \mbox{for}\ \vep\in (0,1/2).$$
    We now split the integral in \eqref{eq:simpl1} into two complementary domains: the first one is an integral over $B_{r_{\vep}}(y^{w_{k\vep}})$ and the second one is over $B^c_{r_{\vep}}(y^{w_{k\vep}})$. We will show that 
    \begin{equation}\label{eq:showsmall}
        \sup_y \frac{\sqrt{\mathrm{det}(\nabla^2 u_{k\vep}(y^{w_{k\vep}}))}}{(2\pi\vep)^{\frac{d}{2}}}\int\limits_{B^c_{r_{\vep}}(y^{w_{k\vep}})} \exp\bigg(-\frac{1}{\vep}\mcD[u_{k\vep}](x|y)-f(x)+f(y^{w_{k\vep}})\bigg)\,dx \le C \vep^{10},
    \end{equation}
    and 
    \begin{equation}\label{eq:showlarge}
        \sup_y \bigg|\frac{\sqrt{\mathrm{det}(\nabla^2 u_{k\vep}(y^{w_{k\vep}}))}}{(2\pi\vep)^{\frac{d}{2}}}\int\limits_{B_{r_{\vep}}(y^{w_{k\vep}})} \exp\bigg(-\frac{1}{\vep}\mcD[u_{k\vep}](x|y)-f(x)+f(y^{w_{k\vep}})\bigg)\,dx - 1\bigg|\le C \vep.
    \end{equation}
    Let us complete the proof by assuming \eqref{eq:showsmall} and \eqref{eq:showlarge} first. To wit, by combining \eqref{eq:showsmall} and \eqref{eq:showlarge}, with \eqref{eq:simpl1}, we get:
    $$\sup_y \bigg|\frac{\sqrt{\mathrm{det}(\nabla^2 u_{k\vep}(y^{w_{k\vep}}))}}{(2\pi\vep)^{\frac{d}{2}}}\exp\left(\frac{1}{\vep}\opV[u_{k\vep}](y)-\frac{1}{\vep}w_{k\vep}(y)+f(y^{w_{k\vep}})\right)-1\bigg|\le C \vep.$$
    The conclusion of \cref{lem:pmaboundmain} then follows by using the elementary inequality $|\log{(1+x)}|\le 2|x|$ for $|x|\le 1/2$. It only remains to show \eqref{eq:showsmall} and \eqref{eq:showlarge}.
    
    \vspace{0.1in}
    
    \emph{Proof of \eqref{eq:showsmall}.} We note the following sequence of displays with line-by-line explanations to follow:
    \begin{align}\label{eq:gradbound4}
        &\;\;\;\;\frac{\sqrt{\mathrm{det}(\nabla^2 u_{k\vep}(y^{w_{k\vep}}))}}{(2\pi\vep)^{\frac{d}{2}}}\int\limits_{B^c_{r_{\vep}}(y^{w_{k\vep}})} \exp\bigg(-\frac{1}{\vep}\mcD[u](x|y)-f(x)+f(y^{w_{k\vep}})\bigg)\,dx\nonumber \\ & \le \frac{C}{(2\pi\vep)^{\frac{d}{2}}}\int\limits_{B^c_{r_{\vep}}(y^{w_{k\vep}})} \exp\bigg(-\frac{A_T}{2\vep }\lVert x-y^{w_{k\vep}}\rVert^2-f(x)+f(y^{w_{k\vep}})\bigg)\,dx\nonumber \\ &= C\E\left[\exp\left(f(y^{w_{k\vep}})-f\left(y^{w_{k\vep}}+\sqrt{\vep A_T^{-1}} Z\right)\right)\mathbf{1}\left(\sqrt{\vep A_T^{-1}}Z\in B_{r_{\vep}}^c(0)\right)\right]\nonumber \\ &\le C  \E\left[\exp\left(\sqrt{\vep A_T^{-1}}\lVert \nabla f\rVert_{\infty}\sum_{i=1}^d |Z_i|\right)\mathbf{1}\left(\sqrt{\vep A_T^{-1}}Z\in B_{r_{\vep}}^c(0)\right)\right]
    \end{align}
    Above, in the first inequality, we have used  \eqref{eq:estimpf1} and \cref{asn:solcon}, part (i) to bound $\sqrt{\mathrm{det}(\nabla^2 u_{k\vep}(y^{w_{k\vep}}))}$ from above. In the following equality, we have adjusted some constants to rewrite the integral in terms of an expectation of a standard multivariate Gaussian random variable $Z$. The next inequality follows from the elementary observation that 
    \begin{align*}
    \left|f\left(y^{w_{k\vep}}+\sqrt{\vep A_T^{-1}}Z\right)-f(y^{w_{k\vep}})\right|&=\int_0^1 \left\langle \nabla f\left(y^{w_{k\vep}}+t\sqrt{\vep A_T^{-1}}Z\right),\sqrt{\vep A_T^{-1}}Z\right\rangle\,dt\\ &\le \sqrt{\vep A_T^{-1}}\lVert \nabla f\rVert_{\infty}\sum_{i=1}^d |Z_i|.
    \end{align*}
    Next, we apply the Cauchy-Schwartz inequality in \eqref{eq:gradbound4} to get:
    \begin{align*}
    &\;\;\;\;\frac{\sqrt{\mathrm{det}(\nabla^2 u_{k\vep}(y^{w_{k\vep}}))}}{(2\pi\vep)^{\frac{d}{2}}}\int\limits_{B^c_{r_{\vep}}(y^{w_{k\vep}})} \exp\bigg(-\frac{1}{\vep}\mcD[u](x|y)-f(x)+f(y^{w_{k\vep}})\bigg)\,dx\\ &\le
    C \sqrt{\E\exp\left(\sqrt{\vep A_T^{-1}}\lVert \nabla f\rVert_{\infty} \sum_{i=1}^d |Z_i|\right)}\sqrt{\Pr(\sqrt{\vep A_T^{-1}}\lVert Z\rVert\ge r_{\vep})}\\ &\le C \sqrt{\left(\E\exp\left(\sqrt{\vep A_T^{-1}}\lVert \nabla f\rVert_{\infty}  |Z_1|\right)\right)^d}\sqrt{d\Pr\left(|Z_1|\ge r_{\vep}/\sqrt{\vep d A_T^{-1}}\right)}\le C \vep^{10}.
    \end{align*}
    To understand the inequalities in the last line above, note that by using standard Gaussian tail bounds, we have 
    $$\E\exp\left(\sqrt{\vep A_T^{-1}}\lVert \nabla f\rVert_{\infty}  |Z_1|\right)\le \E\exp\left(\sqrt{A_T^{-1}}\lVert \nabla f\rVert_{\infty}|Z_1|\right)\le C,$$
    as $\vep\le 1$. Moreover, by the union bound,
     \begin{align*}
        \Pr&(\sqrt{\vep A_T^{-1}}\lVert Z\rVert\ge r_{\vep})\le \sum_{i=1}^d \Pr\left(|Z_i|\ge r_{\vep}/\sqrt{\vep d A_T^{-1}}\right)=d \Pr\left(|Z_1|\ge r_{\vep}/\sqrt{\vep d A_T^{-1}}\right)\\ &\le 2d \exp(-r_{\vep}^2/(2\vep d A_T^{-1}))=2d \exp(20 \log{\vep})=2d \vep^{20}.
    \end{align*}
\vspace{0.1in}

   \emph{Proof of \eqref{eq:showlarge}.} 
    Set $$\tilde{Z}^{(\vep)}_{w_{k\vep},y}\sim y^{w_{k\vep}}+\sqrt{\vep}Z_{w_{k\vep},y}, \quad \mbox{where} \quad Z_{w_{k\vep},y}\sim N(0,\nabla^2 w_{k\vep}(y)).$$
    By a third order Taylor series expansion, for any $k\ge 0$:
    \begin{small}
    \begin{align}\label{eq:taylor1}
        &\mcD[u_{k\vep}](x|y)=\frac{1}{2}(x-y^{w_{k\vep}})^{\top}\nabla^2 u_{k\vep}(y^{w_{k\vep}})(x-y^{w_{k\vep}}) + T[u_{k\vep}:3](x|y^{w_{k\vep}}) + R[u_{k\vep}:3](x|y^{w_{k\vep}}).
    \end{align}
    \end{small}
    Also, by a first order Taylor expansion to the function $f$:
    \begin{align}\label{eq:taylor2}
       f(x)&=f(y^{w_{k\vep}})+T[f:1](x|y^{w_{k\vep}})+R[f:1](x|y^{w_{k\vep}}).
    \end{align}
    By \eqref{eq:taylor1} and \eqref{eq:taylor2},
    \begin{align}\label{eq:largge}
        &\;\;\;\;\frac{\sqrt{\mathrm{det}(\nabla^2 u_{k\vep}(y^{w_{k\vep}}))}}{(2\pi\vep)^{\frac{d}{2}}}\int\limits_{B_{r_{\vep}}(y^{w_{k\vep}})} \exp\left(-\frac{1}{\vep}\mcD[u_{k\vep}](x|y)-f(x)+f(y^{w_{k\vep}})\right)\,dx \nonumber \\ &=\frac{\sqrt{\mathrm{det}(\nabla^2 u_{k\vep}(y^{w_{k\vep}}))}}{(2\pi\vep)^{\frac{d}{2}}}\int\limits_{B_{r_{\vep}}(y^{w_{k\vep}})} \exp\bigg(-\frac{1}{2\vep}(x-y^{w_{k\vep}})^{\top}\nabla^2 u_{k\vep}(y^{w_{k\vep}})(x-y^{w_{k\vep}})\bigg)\times \nonumber \\ 
        &  \exp\left(-\frac{1}{\vep}T[u_{k\vep}:3](x|y^{w_{k\vep}})- \frac{1}{\vep} R[u_{k\vep}:3](x|y^{w_{k\vep}})-T[f:1](x|y^{w_{k\vep}})-R[f:1](x|y^{w_{k\vep}})\right)\,dx.
    \end{align}

    Now in order to prove \eqref{eq:showlarge}, it suffices to show that the final equality above is $1+O(\vep)$. Let us sketch the rest of the argument first. Note that in the second line above, we have the density of a $N(y^{w_{k\vep}},\vep\nabla^2 w_{k\vep}(y))$ random variable, i.e., $\tilde{Z}^{(\vep)}_{w_{k\vep},y}$ declared above. This enables us to rewrite the above integral as a Gaussian integral as follows:

    \begin{align}\label{eq:large1}
    &\E\bigg[\exp\bigg(-\frac{1}{\vep}T[u_{k\vep}:3](\tilde{Z}^{(\vep)}_{w_{k\vep},y}|y^{w_{k\vep}})- \frac{1}{\vep} R[u_{k\vep}:3](\tilde{Z}^{(\vep)}_{w_{k\vep},y}|y^{w_{k\vep}})-T[f:1](\tilde{Z}^{(\vep)}_{w_{k\vep},y}|y^{w_{k\vep}})\nonumber \\ &-R[f:1](\tilde{Z}^{(\vep)}_{w_{k\vep},y}|y^{w_{k\vep}})\bigg)\mathbf{1}(\sqrt{\vep}Z_{w_{k\vep},y}\in B_{r_{\vep}}(0))\bigg].
    \end{align}

    Next we will use two simple approximations of the exponential function as follows: for $|z|\le M$, 
    \begin{align}\label{eq:taylorapp}
        \bigg|\exp(z)-1-z\bigg|\le \frac{z^2}{2}\exp(M),\qquad \mbox{and} \qquad \bigg|\exp(z)-1\bigg|\le |z|\exp(M).
    \end{align}
    The idea is to use \eqref{eq:taylorapp} to approximate the exponential terms in \eqref{eq:large1}. In particular, the terms in \eqref{eq:large1} featuring the Taylor polynomials in the exponent, namely $\exp\big(-\frac{1}{\vep}T[u_{k\vep}:3](\tilde{Z}^{(\vep)}_{w_{k\vep},y}|y^{w_{k\vep}})\big)$ and $\exp\big(-T[f:1](\tilde{Z}^{(\vep)}_{w_{k\vep},y}|y^{w_{k\vep}})\big)$, will be approximated using the first inequality in \eqref{eq:taylorapp}. The remainder terms $\exp\big(\frac{1}{\vep} R[u_{k\vep}:3](\tilde{Z}^{(\vep)}_{w_{k\vep},y}|y^{w_{k\vep}})\big)$ and $\exp\big(-R[f:1](\tilde{Z}^{(\vep)}_{w_{k\vep},y}|y^{w_{k\vep}})\big)$ on the other hand, will be approximated using the second inequality in \eqref{eq:taylorapp}. The role of $M$ in \eqref{eq:taylorapp} will be played by an appropriate function of $r_{\vep}$ thanks to the indicator term in \eqref{eq:large1}. Let us illustrate the above description concretely using one of the aforementioned terms. To wit, note that by \eqref{eq:gradbound3}, we have:
    \begin{align}\label{eq:large2}
    &\;\;\;\;\bigg|\frac{1}{\vep}T[u_{k\vep}:3](\tilde{Z}^{(\vep)}_{w_{k\vep},y}|y^{w_{k\vep}})\mathbf{1}(\sqrt{\vep}Z_{w_{k\vep},y}\in B_{r_{\vep}}(0))\bigg|\nonumber \\ &\le \frac{C}{\vep} \lVert \sqrt{\vep}Z_{w_{k\vep},y}\rVert^3 \mathbf{1}(\sqrt{\vep}Z_{w_{k\vep},y}\in B_{r_{\vep}}(0))\le \frac{C}{\vep} r_{\vep}^3 = C\sqrt{\vep}\left(\log{\left(\frac{1}{\vep}\right)}\right)^{3/2}.
    \end{align}
    Define
    \begin{align*}
    \vartheta^{(1)}_{\vep}(\tilde{Z}^{(\vep)}_{w_{k\vep},y}|y^{w_{k\vep}})&:=\exp\left(-\frac{1}{\vep}T[u_{k\vep}:3](\tilde{Z}^{(\vep)}_{w_{k\vep},y}|y^{w_{k\vep}})\right)-1+\frac{1}{\vep}T[u_{k\vep}:3](\tilde{Z}^{(\vep)}_{w_{k\vep},y}|y^{w_{k\vep}}).
    \end{align*}
    By invoking the first inequality in \eqref{eq:taylorapp} with $M=C\sqrt{\vep}\left(\log{\left(\frac{1}{\vep}\right)}\right)^{3/2}$ as in \eqref{eq:large2}, we get:
    \begin{align}\label{eq:large11}
    &\;\;\;\;\sup_y \big|\vartheta^{(1)}_{\vep}(\tilde{Z}^{(\vep)}_{w_{k\vep},y}|y^{w_{k\vep}})\big|\mathbf{1}(\sqrt{\vep}Z_{w_{k\vep},y}\in B_{r_{\vep}}(0))\nonumber \\ &\le \sup_y \left(\frac{1}{\vep}T[u_{k\vep}:3](\tilde{Z}^{(\vep)}_{w_{k\vep},y}|y^{w_{k\vep}})\right)^2 \mathbf{1}(\sqrt{\vep}Z_{w_{k\vep},y}\in B_{r_{\vep}}(0)) \exp\left(C\sqrt{\vep}\left(\log{\left(\frac{1}{\vep}\right)}\right)^{3/2}\right)\nonumber \\&\le C \vep \left(\log{\left(\frac{1}{\vep}\right)}\right)^3.
    \end{align}
    In the last inequality we have bounded the term $\exp(C\sqrt{\vep}(\log{(1/\vep)})^{3/2})$ by a constant using $\vep\in (0,1)$. In the same vein, 
    \begin{align}\label{eq:large12}
    &\;\;\;\;\sup_y \E\left[\big|\vartheta^{(1)}_{\vep}(\tilde{Z}^{(\vep)}_{w_{k\vep},y}|y^{w_{k\vep}})\big|\mathbf{1}(\sqrt{\vep}Z_{w_{k\vep},y}\in B_{r_{\vep}}(0))\right]\nonumber \\ &\le \sup_y \E\left[\left(\frac{1}{\vep}T[u_{k\vep}:3](\tilde{Z}^{(\vep)}_{w_{k\vep},y}|y^{w_{k\vep}})\right)^2 \mathbf{1}(\sqrt{\vep}Z_{w_{k\vep},y}\in B_{r_{\vep}}(0))\right] \exp\left(C\sqrt{\vep}\left(\log{\left(\frac{1}{\vep}\right)}\right)^{3/2}\right)\nonumber \\&\le  C \sup_y \E\left[\left(\frac{1}{\vep}T[u_{k\vep}:3](\tilde{Z}^{(\vep)}_{w_{k\vep},y}|y^{w_{k\vep}})\right)^2 \right]\le C \vep .
    \end{align}
    In the last display, we have additionally used standard Gaussian tail bounds.

    \vspace{0.1in}

    We can carry out the same line of argument for the other terms in the exponential as seen in \eqref{eq:large1}. We simply produce the corresponding definitions and bounds noting that they can be obtained similarly as the bounds for $\vartheta^{(1)}_{\vep}(\tilde{Z}^{(\vep)}_{w_{k\vep},y}|y^{w_{k\vep}})$ above.

    Define 
    \begin{align*}
    \vartheta^{(2)}_{\vep}(\tilde{Z}^{(\vep)}_{w_{k\vep},y}|y^{w_{k\vep}})&:=\exp\left(- \frac{1}{\vep} R[u_{k\vep}:3](\tilde{Z}^{(\vep)}_{w_{k\vep},y}|y^{w_{k\vep}})\right)-1.
    \end{align*}
    It satisfies 
    \begin{equation}\label{eq:large21}
        \sup_y \big|\vartheta^{(2)}_{\vep}(\tilde{Z}^{(\vep)}_{w_{k\vep},y}|y^{w_{k\vep}})\big|\mathbf{1}(\sqrt{\vep}Z_{w_{k\vep},y}\in B_{r_{\vep}}(0))\le C \vep \left(\log{\left(\frac{1}{\vep}\right)}\right)^2,
    \end{equation}
    and
    \begin{equation}\label{eq:large22}
        \sup_y \E\left[\big|\vartheta^{(2)}_{\vep}(\tilde{Z}^{(\vep)}_{w_{k\vep},y}|y^{w_{k\vep}})\big|\mathbf{1}(\sqrt{\vep}Z_{w_{k\vep},y}\in B_{r_{\vep}}(0))\right]\le C \vep.
    \end{equation}
    Define 
    \begin{align*}
    \vartheta^{(3)}_{\vep}(\tilde{Z}^{(\vep)}_{w_{k\vep},y}|y^{w_{k\vep}})&:=\exp\left(- T[f:1](\tilde{Z}^{(\vep)}_{w_{k\vep},y}|y^{w_{k\vep}})\right)-1+T[f:1](\tilde{Z}^{(\vep)}_{w_{k\vep},y}|y^{w_{k\vep}}).
    \end{align*}
    It satisfies 
    \begin{equation}\label{eq:large31}
        \sup_y \big|\vartheta^{(3)}_{\vep}(\tilde{Z}^{(\vep)}_{w_{k\vep},y}|y^{w_{k\vep}})\big|\mathbf{1}(\sqrt{\vep}Z_{w_{k\vep},y}\in B_{r_{\vep}}(0))\le C \vep \log{\left(\frac{1}{\vep}\right)},
    \end{equation}
    and
    \begin{equation}\label{eq:large32}
        \sup_y \E\left[\big|\vartheta^{(3)}_{\vep}(\tilde{Z}^{(\vep)}_{w_{k\vep},y}|y^{w_{k\vep}})\big|\mathbf{1}(\sqrt{\vep}Z_{w_{k\vep},y}\in B_{r_{\vep}}(0))\right]\le C \vep.
    \end{equation}
    Finally, define 
    \begin{align*}
    \vartheta^{(4)}_{\vep}(\tilde{Z}^{(\vep)}_{w_{k\vep},y}|y^{w_{k\vep}})&:=\exp\left(-  R[f:1](\tilde{Z}^{(\vep)}_{w_{k\vep},y}|y^{w_{k\vep}})\right)-1.
    \end{align*}
    It satisfies 
    \begin{equation}\label{eq:large41}
        \sup_y \big|\vartheta^{(4)}_{\vep}(\tilde{Z}^{(\vep)}_{w_{k\vep},y}|y^{w_{k\vep}})\big|\mathbf{1}(\sqrt{\vep}Z_{w_{k\vep},y}\in B_{r_{\vep}}(0))\le C \vep \log{\left(\frac{1}{\vep}\right)},
    \end{equation}
    and
    \begin{equation}\label{eq:large42}
        \sup_y \E\left[\big|\vartheta^{(4)}_{\vep}(\tilde{Z}^{(\vep)}_{w_{k\vep},y}|y^{w_{k\vep}})\big|\mathbf{1}(\sqrt{\vep}Z_{w_{k\vep},y}\in B_{r_{\vep}}(0))\right]\le C \vep.
    \end{equation}
    It is worth noting that the bounds \eqref{eq:large31} and \eqref{eq:large32} (which correspond to approximating terms involving Taylor polynomials) will require \eqref{eq:gradbound3} in the same way as the bounds \eqref{eq:large11} and \eqref{eq:large12} (which also correspond to approximating terms involving Taylor polynomials). On the other hand, the same step will require \eqref{eq:gbd4} while obtaining the bounds \eqref{eq:large21}, \eqref{eq:large22}, \eqref{eq:large41}, and \eqref{eq:large42} (which correspond to approximating remainder terms after suitable Taylor series approximations). 


    We will now use the above bounds in conjunction with \eqref{eq:largge} and \eqref{eq:large1}. First we use the definitions of $\vartheta^{(i)}_{\vep}(\tilde{Z}^{(\vep)}_{w_{k\vep},y}|y^{w_{k\vep}})$, $i=1,2,3,4$, in \eqref{eq:largge} and \eqref{eq:large1} to note that:

    \begin{align}\label{eq:gaussbreak}
        &\;\;\;\;\frac{\sqrt{\mathrm{det}(\nabla^2 u_{k\vep}(y^{w_{k\vep}}))}}{(2\pi\vep)^{\frac{d}{2}}}\int\limits_{B_{r_{\vep}}(y^{w_{k\vep}})} \exp\left(-\frac{1}{\vep}\mcD[u_{k\vep}](x|y)-f(x)+f(y^{w_{k\vep}})\right)\,dx\nonumber \\ 
        &=\E\bigg[\left(1-\frac{1}{\vep}T[u_{k\vep}:3](\tilde{Z}^{(\vep)}_{w_{k\vep},y}|y^{w_{k\vep}})+\vartheta^{(1)}_{\vep}(\tilde{Z}^{(\vep)}_{w_{k\vep},y}|y^{w_{k\vep}})\right)\left(1+\vartheta^{(2)}_{\vep}(\tilde{Z}^{(\vep)}_{w_{k\vep},y}|y^{w_{k\vep}})\right)\nonumber \\ &\qquad \left(1-T[f:1](\tilde{Z}^{(\vep)}_{w_{k\vep},y}|y^{w_{k\vep}})+\vartheta^{(3)}_{\vep}(\tilde{Z}^{(\vep)}_{w_{k\vep},y}|y^{w_{k\vep}})\right)\left(1+\vartheta^{(4)}_{\vep}(\tilde{Z}^{(\vep)}_{w_{k\vep},y}|y^{w_{k\vep}})\right)\nonumber \\ &\qquad \qquad \mathbf{1}(\sqrt{\vep}Z_{w_{k\vep},y}\in B_{r_{\vep}}(0))\bigg].
    \end{align}

We will now expand out all the brackets above. Let us first try to isolate the terms which are $o(\vep)$. To wit, note that if a term has a product of at least two of $\vartheta_{\vep}^{(i)}(\tilde{Z}^{(\vep)}_{w_{k\vep},y}|y^{w_{k\vep}})$ and $\vartheta^{(j)}_{\vep}(\tilde{Z}^{(\vep)}_{w_{k\vep},y}|y^{w_{k\vep}})$, then it is $o(\vep)$. 
This is because by \eqref{eq:large11}, \eqref{eq:large21}, \eqref{eq:large31}, and \eqref{eq:large41}, for $i\neq j$, we have \begin{align*}&\;\;\;\;\sup_y \big|\vartheta_{\vep}^{(i)}(\tilde{Z}^{(\vep)}_{w_{k\vep},y}|y^{w_{k\vep}})\vartheta^{(j)}_{\vep}(\tilde{Z}^{(\vep)}_{w_{k\vep},y}|y^{w_{k\vep}})\mathbf{1}(\sqrt{\vep}Z_{w_{k\vep},y}\in B_{r_{\vep}}(0))\big|\\ &\le C\vep^2\left(\log{\left(\frac{1}{\vep}\right)}\right)^{6}=o(\vep).
\end{align*}
Same goes for terms having product of some $\vartheta_{\vep}^{(i)}(\tilde{Z}^{(\vep)}_{w_{k\vep},y}|y^{w_{k\vep}})$ with either of $\frac{1}{\vep}T[u_{k\vep}:3](\tilde{Z}^{(\vep)}_{w_{k\vep},y}|y^{w_{k\vep}})$ or $T[f:1](\tilde{Z}^{(\vep)}_{w_{k\vep},y}|y^{w_{k\vep}})$. For example, by \eqref{eq:large2} and \eqref{eq:large21}, we have:
\begin{align*}
&\;\;\;\;\sup_y \bigg|\vartheta_{\vep}^{(i)}(\tilde{Z}^{(\vep)}_{w_{k\vep},y}|y^{w_{k\vep}})\frac{1}{\vep}T[u_{k\vep}:3](\tilde{Z}^{(\vep)}_{w_{k\vep},y}|y^{w_{k\vep}})\bigg|\mathbf{1}(\sqrt{\vep}Z_{w_{k\vep},y}\in B_{r_{\vep}}(0))\\ &\le C\vep^{3/2}\left(\log{\left(\frac{1}{\vep}\right)}\right)^{7/2}=o(\vep).
\end{align*}
Therefore, we can easily isolate the terms that potentially contribute $O(\vep)$ or higher. By doing this in \eqref{eq:gaussbreak}, we get that:
    
    \begin{align*}
        &\;\;\;\;\frac{\sqrt{\mathrm{det}(\nabla^2 u_{k\vep}(y^{w_{k\vep}}))}}{(2\pi\vep)^{\frac{d}{2}}}\int\limits_{B_{r_{\vep}}(y^{w_{k\vep}})} \exp\left(-\frac{1}{\vep}\mcD[u_{k\vep}](x|y)-f(x)+f(y^{w_{k\vep}})\right)\,dx\nonumber \\ 
        &=1-\E\bigg[\bigg(\frac{1}{\vep}T[u_{k\vep}:3](\tilde{Z}^{(\vep)}_{w_{k\vep},y}|y^{w_{k\vep}})+\frac{1}{\vep}T[u_{k\vep}:3](\tilde{Z}^{(\vep)}_{w_{k\vep},y}|y^{w_{k\vep}})T[f:1](\tilde{Z}^{(\vep)}_{w_{k\vep},y}|y^{w_{k\vep}})\\ &-T[f:1](\tilde{Z}^{(\vep)}_{w_{k\vep},y}|y^{w_{k\vep}})+\sum_{i=1}^4 \vartheta_{\vep}^{(i)}(\tilde{Z}^{(\vep)}_{w_{k\vep},y}|y^{w_{k\vep}})\bigg)\mathbf{1}(\sqrt{\vep}Z_{w_{k\vep},y}\in B_{r_{\vep}}(0))\bigg]+o(\vep).
    \end{align*}
    Here the $o(\vep)$ term, as argued above, is uniform in $y$. Next, we use the symmetry of Gaussians to note that 
    $$\E\left[\frac{1}{\vep}T[u_{k\vep}:3](\tilde{Z}^{(\vep)}_{w_{k\vep},y}|y^{w_{k\vep}})\mathbf{1}(\sqrt{\vep}Z_{w_{k\vep},y}\in B_{r_{\vep}}(0))\right]=0,$$
    and 
    $$\E\left[T[f:1](\tilde{Z}^{(\vep)}_{w_{k\vep},y}|y^{w_{k\vep}})\mathbf{1}(\sqrt{\vep}Z_{w_{k\vep},y}\in B_{r_{\vep}}(0))\right]=0.$$
    Therefore, the second and the fourth term above are both $0$. For the third term, note that by using \cref{asn:solcon}, part (iii), we get:
    \begin{align*}
        &\;\;\;\sup_y\E\left[\frac{1}{\vep}T[u_{k\vep}:3](\tilde{Z}^{(\vep)}_{w_{k\vep},y}|y^{w_{k\vep}})T[f:1](\tilde{Z}^{(\vep)}_{w_{k\vep},y}|y^{w_{k\vep}})\mathbf{1}(\sqrt{\vep}Z_{w_{k\vep},y}\in B_{r_{\vep}}(0))\right]\\ &\le \sup_y C\frac{1}{\vep}\E\lVert \tilde{Z}^{(\vep)}_{w_{k\vep},y}-y^{w_{k\vep}}\rVert^4\le C \vep.
    \end{align*}
    Finally, by \eqref{eq:large12}, \eqref{eq:large22}, \eqref{eq:large32}, and \eqref{eq:large42}, we get:
    $$\sup_y \sum_{i=1}^4 \E\left[\bigg|\vartheta_{\vep}^{(i)}(\tilde{Z}^{(\vep)}_{w_{k\vep},y}|y^{w_{k\vep}})\mathbf{1}(\sqrt{\vep}Z_{w_{k\vep},y}\in B_{r_{\vep}}(0))\bigg|\right]\le C\vep.$$
    Therefore, all the requisite terms are $O(\vep)$ or of a lower order. This establishes \eqref{eq:showlarge}, thereby completing the proof of \cref{lem:pmaboundmain}.
\end{proof}

In order to prove \cref{lem:erboundmain}, we consider the following proposition which will be used multiple times in the sequel. 

\begin{prop}\label{prop:contra}
Given $\phi_1, \phi_2\in \diffcont(\R^d)$, we have
$$\lVert\opV[\phi_1]-\opV[\phi_2]\rVert_{\infty}\le \lVert \phi_1-\phi_2\rVert_{\infty},\qquad \lVert \opU[\phi_1]-\opU[\phi_2]\rVert_{\infty}\le \lVert \phi_1-\phi_2\rVert_{\infty}.$$
\end{prop}

\begin{proof}
    By the variational representation of KL divergence, we have:
\begin{equation}\label{eq:dualRE}
\opV[\phi_1](y) = \vep\sup_{\nu:\ \KL{\nu}{e^{-f}}<\infty}\left[ \int \left(\frac{1}{\vep}\langle x,y\rangle - \frac{1}{\vep}\phi_1(x)\right) d\nu(x) - \KL{\nu}{e^{-f}}\right].
\end{equation}
Clearly the same representation as in \eqref{eq:dualRE} also holds with $\phi_1$ replacing $\phi_2$. Now, for any probability measure $\nu$ satisfying $\KL{\nu}{e^{-f}}<\infty$, we have:
\begin{align*}
\bigg| &\;\;\;\; \left[\int \left(\frac{1}{\vep}\langle x,y\rangle - \frac{1}{\vep}\phi_1(x)\right) d\nu(x) - \KL{\nu}{\mu}\right] \\ & -  \int \left[\int \left(\frac{1}{\vep}\langle x,y\rangle - \frac{1}{\vep}\phi_1(x)\right) d\nu(x) - \KL{\nu}{\mu}\right]\bigg| \le \norm{\phi_1 - \phi_2}_\infty. 
\end{align*}
Invoking \eqref{eq:dualRE} completes the proof for $\opV$s. The same strategy also works for $\opU$'s.
\end{proof}

We are now in position to prove the key lemma.
%\SP{Not sure you want to claim this is a "main" result. Perhaps a "key lemma"?}{\color{blue} Edited.}

\vspace{0.1in} 

\emph{Proof of \cref{lem:erboundmain}.} By the uniform Lipschitz property of $\opV$s as established in \cref{prop:contra}, we have: 
$$\lVert b_k^{\vep}\rVert_{\infty}=\frac{1}{\vep}\lVert \opV[u_k^{\vep}]-\opV[u_{k\vep}]\rVert_{\infty}\le \frac{1}{\vep}\lVert u_k^{\vep}-u_{k\vep}\rVert_{\infty}=\lVert a_k^{\vep}\rVert_{\infty}.$$
Therefore, it suffices to only work with $a_k^{\vep}$s as defined in \cref{lem:erboundmain}. 
%\SP{What does the above senetence mean? Are you saying that the proof is similar for b?} {\color{blue} Edited.}
Recall that $u_{k\vep}$'s are the solutions of the PMA \eqref{eq:pma} restricted to time points $t=\vep,2\vep,\ldots$. Consider the following surrogate sequence of functions 
$\bar{u}_{k+1}^{\vep}:=\opS[u_{k\vep}],$
and define,
\begin{equation}\label{eq:approx}
R_T(\vep):=\sup_{k\vep\le T}\lVert \bar{u}_{k+1}^{\vep}-u_{(k+1)\vep}\rVert_{\infty}.
\end{equation}

With the above notation, the proof of \cref{lem:erboundmain} will proceed in two steps. Once again throughout this proof, we will use $C$ to denote a generic constant (which might change from one line to the next) free of $\vep$. 

\emph{Step (a).} We will show that 
$$R_T(\vep)\le C \vep^2.$$

\emph{Step (b).} We then show that 
$$\lVert a_k^{\vep}\rVert_{\infty}\le (k/\vep) R_T(\vep),$$
for all $k$ such that $k\vep\le T$.

By combining steps (a) and (b), we get:
$$\sup_{k:\ k\vep\le T} \lVert a_k^{\vep}\rVert_{\infty}\le \frac{T}{\vep^2}R_T(\vep)\le CT.$$

This proves \cref{lem:erboundmain}. We now focus our attention on proving steps (a) and (b). 

\emph{Proof of step (a).} Recall the time derivatives of the $u_{k\vep}$s as in the PMA~\eqref{eq:pma}. By \cref{asn:solcon}, part (ii), we have:
\begin{align}\label{eq:pmareg}
    &\;\;\;\;\sup_{k:\ k\vep\le T}\sup_x \Bigg|u_{(k+1)\vep}(x)-u_{k\vep}(x)-\vep\left(f(x)-g(x^{u_{k\vep}})+\ldet\left(\frac{\partial x^{u_{k\vep}}}{\partial x\hfill}\right)\right)\Bigg|\nonumber \\ &=\sup_{k:\ k\vep\le T}\sup_x \Bigg|u_{(k+1)\vep}(x)-u_{k\vep}(x)-\vep\frac{\partial}{\partial t}u_t(x)\big|_{t=k\vep}\Bigg|\le C\vep.
\end{align}

%\SP{Unsure. Assumption 2.1 (ii) is about continuity. How does that lead to (6.25)?} {\color{blue} Edited. Earlier Assumption 2.1 (ii) had a uniformity missing as I figured the notation $\diffcont^{k,\ell}$ already incorporates that. I believe I have corrected it now.}


Next, we will show the following:
\begin{equation}\label{eq:stepa1}
    \sup_{k: \ k\vep\le T}\sup_x \bigg|\exp\left(\frac{\bar{u}_{k+1}^{\vep}(x)-u_{k\vep}(x)}{\vep}-\left(f(x)-g(x^{u_{k\vep}})+\ldet\left(\frac{\partial x^{u_{k\vep}}}{\partial x\hfill}\right)\right)\right)-1\bigg|\le C\vep,
\end{equation}
By taking logarithms in \eqref{eq:stepa1} and combining it with \eqref{eq:pmareg}, we easily get Step (a). Therefore, it only remains to prove \eqref{eq:stepa1}.

\emph{Proof of \eqref{eq:stepa1}.} First let us define 
$$\vartheta_{k\vep}(y):=\vep^{-1}\left(\opV[u_{k\vep}](y)-w_{k\vep}(y)-\frac{\vep d}{2}\log{(2\pi\vep)}+\vep f(y^{w_{k\vep}})-\frac{\vep}{2}\ldet\left(\frac{\partial y^{w_{k\vep}}}{\partial y\hfill}\right)\right).$$
The following estimate is an immediate consequence of \cref{lem:pmaboundmain}.
\begin{align}\label{eq:estcon}
    \sup_y \bigg|\exp\left(\frac{\vartheta_{k\vep}(y)}{\vep}\right)-1\bigg|\le C\vep.
\end{align}
We also note the following algebraic identity:
\begin{align*}
    &\;\;\;\;\exp\left(\frac{\bar{u}_{k+1}^{\vep}(x)-u_{k\vep}(x)}{\vep}-\left(f(x)-g(x^{u_{k\vep}})+\ldet\left(\frac{\partial x^{u_{k\vep}}}{\partial x\hfill}\right)\right)\right)\\ &=\frac{\sqrt{\mathrm{det}\left(\frac{\partial x\hfill}{\partial x^{u_{k\vep}}\hfill}\right)}}{(2\pi\vep)^{\frac{d}{2}}}\int \exp\Bigg(\frac{1}{\vep}\langle x,y\rangle-\frac{1}{\vep}w_{k\vep}(y)-\frac{1}{\vep}u_{k\vep}(x)+f(y^{w_{k\vep}})-f(x)-g(y)+g(x^{u_{k\vep}})\\ &\qquad -\frac{1}{2}\ldet\left(\frac{\partial y^{w_{k\vep}}}{\partial y\hfill}\right) + \frac{1}{2}\ldet\left(\frac{\partial x\hfill}{\partial x^{u_{k\vep}}\hfill}\right)\Bigg)\exp\left(-\frac{\vartheta_{k\vep}(y)}{\vep}\right)\,dy
\end{align*}
By combining the above display with \eqref{eq:estcon}, we have:
\begin{align*}
    &\;\;\;\;\sup_{k:\ k\vep\le T}\sup_x \Bigg|\exp\left(\frac{\bar{u}_{k+1}^{\vep}(x)-u_{k\vep}(x)}{\vep}-\left(f(x)-g(x^{u_{k\vep}})+\ldet\left(\frac{\partial x^{u_{k\vep}}}{\partial x\hfill}\right)\right)\right)\\ &\qquad\qquad - \frac{\sqrt{\mathrm{det}\left(\frac{\partial x\hfill}{\partial x^{u_{k\vep}}}\right)}}{(2\pi\vep)^{\frac{d}{2}}}\int \exp\Bigg(\frac{1}{\vep}\langle x,y\rangle-\frac{1}{\vep}w_{k\vep}(y)-\frac{1}{\vep}u_{k\vep}(x)+f(y^{w_{k\vep}})-f(x)\\ &\qquad\qquad -g(y)+g(x^{u_{k\vep}})-\frac{1}{2}\ldet\left(\frac{\partial y^{w_{k\vep}}}{\partial y\hfill}\right) + \frac{1}{2}\ldet\left(\frac{\partial x\hfill}{\partial x^{u_{k\vep}}\hfill}\right)\Bigg)\,dy\Bigg|\\ &\le (C\vep) \sup_{k:\ k\vep\le T}\sup_x \frac{\sqrt{\mathrm{det}\left(\frac{\partial x\hfill}{\partial x^{u_{k\vep}}}\right)}}{(2\pi\vep)^{\frac{d}{2}}}\int \exp\Bigg(\frac{1}{\vep}\langle x,y\rangle-\frac{1}{\vep}w_{k\vep}(y)-\frac{1}{\vep}u_{k\vep}(x)+f(y^{w_{k\vep}})-f(x)\\ &\qquad -g(y)+g(x^{u_{k\vep}})-\frac{1}{2}\ldet\left(\frac{\partial y^{w_{k\vep}}}{\partial y\hfill}\right) + \frac{1}{2}\ldet\left(\frac{\partial x\hfill}{\partial x^{u_{k\vep}}\hfill}\right)\Bigg)\,dy.
\end{align*}
Therefore in order to prove \eqref{eq:stepa1}, it suffices to prove: 
\begin{align*}
    &\;\;\;\;\sup_{k:\ k\vep\le T}\sup_x\bigg|\frac{\sqrt{\mathrm{det}\left(\frac{\partial x\hfill}{\partial x^{u_{k\vep}}}\right)}}{(2\pi\vep)^{\frac{d}{2}}}\int \exp\Bigg(\frac{1}{\vep}\langle x,y\rangle-\frac{1}{\vep}w_{k\vep}(y)-\frac{1}{\vep}u_{k\vep}(x)+f(y^{w_{k\vep}})-f(x)\nonumber\\ &\qquad -g(y)+g(x^{u_{k\vep}})-\frac{1}{2}\ldet\left(\frac{\partial y^{w_{k\vep}}}{\partial y\hfill}\right) + \frac{1}{2}\ldet\left(\frac{\partial x\hfill}{\partial x^{u_{k\vep}}\hfill}\right)\Bigg)\,dy-1\bigg|\le C\vep.
\end{align*}
Once again by taking logarithms on both sides above, it suffices to show:
\begin{align}\label{eq:stepa2}
&\sup_{k:\ k\vep\le T} \sup_x \bigg|\log\int \exp\left(\frac{1}{\vep}\langle x,y\rangle-\frac{1}{\vep}w_{k\vep}(y)+f(y^{w_{k\vep}})-g(y)-\frac{1}{2}\ldet\left(\frac{\partial y^{w_{k\vep}}}{\partial y\hfill}\right)\right)\,dy\nonumber \\ &\qquad - \frac{\vep d}{2}\log{(2\pi\vep)} - f(x) + g(x^{u_{k\vep}})-\ldet\left(\frac{\partial x^{u_{k\vep}}}{\partial x\hfill}\right)\bigg|\le C\vep. 
\end{align}
Now $$\widetilde{\opV}[w_{k\vep}](x):=\vep\log\int \exp\left(\frac{1}{\vep}\langle x,y\rangle-\frac{1}{\vep}w_{k\vep}(y)+f(y^{w_{k\vep}})-g(y)-\frac{1}{2}\ldet\left(\frac{\partial y^{w_{k\vep}}}{\partial y\hfill}\right)\right)\,dy$$
has the same form as $\opV[u_{k\vep}]$ with $u_{k\vep}$ replaced by $w_{k\vep}$ and $f$ replaced by the function $\tilde{f}$ given by 
$$\tilde{f}(y)=-f(y^{w_{k\vep}})+g(y)+\frac{1}{2}\ldet\left(\frac{\partial y^{w_{k\vep}}}{\partial y\hfill}\right).$$
By \cref{asn:solcon}, part (iii), $\tilde{f}$ satisfies the same assumptions as $f$ required in \cref{lem:pmaboundmain}. 
Therefore, reworking the same proof as \cref{lem:pmaboundmain} with $\tilde{f}$, we get:
%\SP{Are you invoking Lemma 4.5 or reworking the proof on a different function $\tilde{f}$ that satisfies the same assumption as $f$?} {\color{blue} Edited. I believe it is the latter. I was viewing these as being equivalent. }
$$\sup_{k:\ k\vep\le T}\sup_x \bigg| \widetilde{\opV}[w_{k\vep}](x)-\frac{\vep d}{2}\log{(2\pi\vep)}+\vep \tilde{f}(x^{u_{k\vep}})-\frac{\vep}{2}\ldet\left(\frac{\partial x^{u_{k\vep}}}{\partial x\hfill}\right)\bigg|\le C\vep^2.$$
As $\tilde{f}(x^{u_{k\vep}})=-f(x)+g(x^{u_{k\vep}})-\frac{1}{2}\ldet\left(\frac{\partial x^{u_{k\vep}}}{\partial x\hfill}\right)$, the above display establishes \eqref{eq:stepa2}. This establishes \eqref{eq:stepa1} and consequently, also establishes Step (a).

\vspace{0.1in}

\emph{Proof of Step (b).} To establish step (b), the crucial tool will be \cref{prop:contra}. To wit, note that for all $k$ such that $k\vep\le T$, we have
%\SP{small t or large T?} {\color{blue} Edited}
\begin{align*}
    \lVert a_k^{\vep}\rVert_{\infty}&\le \frac{1}{\vep}\lVert u_{k}^{\vep}-\bar{u}_{k}^{\vep}\rVert_{\infty}+\frac{1}{\vep} \lVert \bar{u}_{k}^{\vep}-u_{k\vep}\rVert_{\infty}\\ &\le \frac{1}{\vep}\lVert \opS[u_{k-1}^{\vep}]-\opS[u_{(k-1)\vep}]\rVert_{\infty}+\frac{1}{\vep}R_t(\vep)\\ &\le \lVert a_{k-1}^{\vep}\rVert_{\infty}+\frac{1}{\vep}R_t(\vep).
\end{align*}
The last inequality follows by noting that $\opS$ is $1$-Lipschitz in the uniform norm which in turn follows from the fact that $\opS=\opU\circ\opV$ and both $\opU$, $\opV$ are $1$-Lipschitz in the uniform norm by \cref{prop:contra}. Now, in order to prove the conclusion in Step (b), we will use the above inequality recursively, i.e., 
$$\lVert a_k^{\vep}\rVert_{\infty}\le \lVert a_{k-1}^{\vep}\rVert_{\infty}+\frac{1}{\vep}R_t(\vep)\le \lVert a_{k-2}^{\vep}\rVert_{\infty}+\frac{2}{\vep}R_t(\vep)\le \ldots \le \lVert a_0^{\vep}\rVert_{\infty}+\frac{k}{\vep}R_T(\vep).$$
Now, by definition, $a_0^{\vep}=\vep^{-1}(u_0^{\vep}-u_0)=0$ as we use the same initializer for the Sinkhorn algorithm \eqref{eq:sinkupdt} and the PMA \eqref{eq:pma}. This implies, using the above display coupled with step (a), 
$$\sup_{k:\ k\vep\le T}\lVert a_k^{\vep}\rVert_{\infty}\le \sup_{k:\ k\vep\le T} \frac{k}{\vep}R_T(\vep)\le \sup_{k:\ k\vep\le T} \frac{k}{\vep}\cdot C\vep^2\le CT.$$
This establishes step (b).
%\SP{Can you expand this last line? Write down a two-line argument using math symbols.} {\color{blue} Edited.}
\begin{comment}
\begin{lmm}\label{lem:stab2}
Consider the Sinkhorn algorithm initialized according to \eqref{eq:init}. Then for $t\ge 0$, and $k\vep \le t$, the following holds: 
$$\lVert u^{\vep}_{k}-\tilde{u}_{k\vep,\vep}\rVert_{\infty}\leq k R_t(\vep).$$
\end{lmm}

\begin{proof}
Once again, the proof proceeds by induction. The result is trivial for $k=0$ according to \eqref{eq:init}. Assume it holds upto some $k\equiv k_0$. To establish the bound at $k+1$, note that
$$\lVert u^{\vep}_{k+1}-\tilde{u}_{(k+1)\vep,\vep}\rVert_{\infty}\leq \lVert u^{\vep}_{k+1}-\bar{u}_{(k+1)\vep,\vep}\rVert_{\infty}+R_t(\vep)\leq \lVert u^{\vep}_{k}-\tilde{u}_{k\vep,\vep}\rVert_{\infty}+R_t(\vep).$$
Here the last inequality follows from~\cref{lem:stab1}. By the induction hypothesis $\lVert u^{\vep}_{k}-\tilde{u}_{k\vep,\vep}\rVert_{\infty}\leq k R_t(\vep)$. Combining this observation with the above display completes the induction. 
\end{proof}

We are now in the position to state the main lemma of this section which should lead to the desired conclusion. 

\begin{lmm}\label{lem:erbd}
    Under \cref{asn:solcon}, we have
    $$R_t(\vep)=O(\vep^3).$$
\end{lmm}

\begin{lmm}\label{lem:prelimestim}
 Given a convex function $u(\cdot)$, set $w:=u^*$. For any $\vep\in (0,1/2)$, consider a function $G(\cdot)$ on $\R^d$ which is smooth. Assume that 
 $$A_1:=\big(\inf_{x\in\R^d}\lmn(\nabla^2 u(x))\big)^{-1}\vee \big(\sup_{x\in\R^d} \lmx(\nabla^2 u(x))\big) \in (0,\infty).$$
 Also 
 \begin{align*}
 A_2&:=\max_{r=3}^6 \big(\lVert \nabla^r u\rVert_{\infty}\vee \lVert \nabla^r w\rVert_{\infty}\big)+\sum_{r=1}^4 \lVert \nabla^r G\rVert_{\infty} <\infty,
 \end{align*}
 and define 
 \begin{align*}
 \omega(\delta)&:=\sup_{\lVert x-y\rVert\le\delta} \left(\lVert \nabla^4 G(x)-\nabla^4 G(y)\rVert_{\infty}+\lVert \nabla^6 u(x)-\nabla^6 u(y)\rVert_{\infty}\right) <\infty,
 \end{align*}
 for all $\delta>0$. Assume $\omega(\delta)\to 0$ as $\delta\to 0$. Set $A:=A_1\vee A_2$. 
For any $\vep\in (0,1/2)$ and any $y\in\R^d$, define a probability measure with density
 $$ \exp\left(\frac{1}{\vep} \langle x,y\rangle -\frac{1}{\vep}u(x)-G(x)-\frac{1}{\vep}\mc(y)\right)$$
 where $\mc(y)$ is the (scaled) log-normalizing constant  for all $\vep>0$.  Then there exists a constant $C>0$ (depending on $A$, $d$, and $\omega(\cdot)$ such that the following estimates hold:
 \begin{small}
 \begin{align}\label{eq:prestim1}
     &\bigg|\mc(y)-w(y)+\vep \big(G(y^w)-\frac{d}{2}\log{(2\pi\vep)}-(1/2)\ldet(\nabla^2 w(y))\big)-\vep^2M[u,G](y) \bigg| \le C \vep^3
 \end{align}
 \end{small}
 where 
 \begin{align}\label{eq:gaussdef}
     M[u,G](y) &:=  \frac{1}{2}\E\big(T[u:3](y^w+Z_{w,y}|y^w)\big)^2-\E T[u:4](y^w+Z_{w,y}|y^w)\nonumber \\ &\qquad - \E T[G:2](y^w+Z_{w,y}|y^w) + \frac{1}{2}\E(T[G:1](y^w+Z_{w,y}|y^w))^2\nonumber \\ &\qquad + \E T[u:3](y^w+Z_{w,y}|y^w) T[G:1](y^w+Z_{w,y}|y^w).
 \end{align}
\end{lmm}
 %\begin{align}\label{eq:prestim2}
  %   &\;\;\;\;\big\lVert \E_{\rho_y}(X)-y^{w}-\frac{\vep}{2}\nabla(\ldet(\nabla^2 w))(y)+\vep\nabla(G(\nabla w))(y)\big\rVert \nonumber \\ &\le C_A \vep^{3/2}\big((-\log{\vep})^{3/2}+(\omega_1\vee\omega_2)(\sqrt{\vep\log{(1/\vep)}})\big).
 %\end{align}
 %and 
 %\begin{align}\label{eq:prestim3}
  %   &\;\;\;\;\big\lVert \mbox{Var}_{\rho_y}(X)-\vep \nabla^2 w(y)-\frac{\vep^2}{2}\nabla^2(\ldet(\nabla^2 w))(y)+\vep^2 \nabla^2 (G(\nabla w))(y) \big\rVert_{\infty}\nonumber \\ &\le C_A \left(\vep^2 (\omega_1\vee \omega_2)\big(\sqrt{\vep\log{(1/\vep)}}\big)+\vep^{5/2}\big(-\log{\vep})^{9/2}\big)\right).
 %\end{align}
 %\end{comment}


%\begin{remark}\label{rem:prest}
%We note here that $\nabla \mc(y)=\E_{\rho_y}(X)$ and $\vep\nabla^2 \mc(y)=\mbox{Var}_{\rho_y}(X)$.    
%\end{remark}

\begin{proof}
    
    By a multivariate Taylor series expansion, we have for any $k\ge 0$:
    \begin{align}\label{eq:taylor1}
        &\;\;\;\;\mcD[u](x|y)\nonumber \\ &=-\frac{1}{2}(x-y^{w})^{\top}\nabla^2 u(y^{w})(x-y^{w}) - \sum_{r=3}^4 T[u:r](x|y^{w})-R[u:4](x|y^{w}).
    \end{align}
    We obtain equivalent expansions for the functions  $G(\cdot)$ and $\tilde{G}_{\vep}(\cdot)$, to get:
    \begin{align}\label{eq:taylor2}
       G(x)&=G(y^{w})+T[G:1](x|y^{w})+T[G:2](x|y^w)+R[G:2](x|y^{w}).
    \end{align}
    Another observation we will use in this proof is as follows:
    \begin{equation}\label{eq:estimpf1}
\mcD[u](x|y) \le -\frac{1}{2A_1}\lVert x-y^{w}\rVert^2.
\end{equation}
    
    \noindent We are now in a position to simplify $\mc(\cdot)$. 
    
    \vspace{0.1in}
    
    \emph{Proof of \eqref{eq:prestim1}.} Observe that 
    \begin{align}\label{eq:simpl1}
       &\;\;\;\;\frac{\sqrt{\mathrm{det}(\nabla^2 u(y^{w}))}}{(2\pi\vep)^{\frac{d}{2}}}\exp\left(\frac{1}{\vep}\mc(y)-\frac{1}{\vep}w(y)+G(y^w)\right)\nonumber \\ &=\frac{\sqrt{\mathrm{det}(\nabla^2 u(y^{w}))}}{(2\pi\vep)^{\frac{d}{2}}}\int \exp\bigg(\frac{1}{\vep}\mcD[u](x|y)-G(x)+G(y^w)-\vep\tilde{G}_{\vep}(x)+\vep\tilde{G}_{\vep}(y^w)\bigg)\,dx.
    \end{align}
    Define 
    $$r_{\vep}:=\sqrt{-40d A_1 \vep\log{\vep}},\qquad \mbox{for}\ \vep\in (0,1/2).$$
    For the rest of the proof, $C$ will denote changing constants depending only on $d$, $A$, and $\omega(\cdot)$. Observe that by \eqref{eq:estimpf1}, we have:
    \begin{align}\label{eq:gradbound4}
        &\;\;\;\;\frac{\sqrt{\mathrm{det}(\nabla^2 u(y^{w}))}}{(2\pi\vep)^{\frac{d}{2}}}\int\limits_{B^c_{r_{\vep}}(y^{w})} \exp\bigg(\frac{1}{\vep}\mcD[u](x|y)-G(x)+G(y^{w})\bigg)\,dx\nonumber \\ &\le C\exp(G(y^{w}))\E\left[\exp\left(-G(y^{w}+\sqrt{\vep A_1} Z)\right)\mathbf{1}(\sqrt{\vep A_1}Z\in B_{r_{\vep}}^c(0))\right]\nonumber \\ &\le C \sqrt{\prod_{i=1}^d \E\exp\left(\sqrt{\vep A_1}\lVert \nabla G\rVert_{\infty} |Z_i|\right)}\sqrt{d\Pr(|Z_1|\ge r_{\vep}/\sqrt{\vep d A_1})}\le C \vep^{10}.
    \end{align}
    In the last step, we have used standard Gaussian tail bounds. Next, we set $$\tilde{Z}^{(\vep)}_{w,y}\sim y^{w}+\sqrt{\vep}Z_{w,y}.$$ 
    We now focus on the integral inside $B_{r_{\vep}}(y^{w})$. 
    \begin{align}\label{eq:gaussbreak}
        &\;\;\;\;\frac{\sqrt{\mathrm{det}(\nabla^2 u(y^{w}))}}{(2\pi\vep)^{\frac{d}{2}}}\int\limits_{B_{r_{\vep}}(y^{w})} \exp\left(\frac{1}{\vep}\mcD[u](x|y)-G(x)+G(y^{w})\right)\,dx\nonumber \\ &=\E\bigg[\left(\sum_{m=0}^{\infty}\frac{1}{\vep^m m!}(-T[u:3](\tilde{Z}^{(\vep)}_{w,y}|y^{w}))^m\right)\left(\sum_{m=0}^{\infty}\frac{1}{\vep^m m!}(- T[u:4](\tilde{Z}^{(\vep)}_{w,y}|y^{w}))^m\right)\nonumber \\ &\qquad \left(\sum_{m=0}^{\infty}\frac{1}{\vep^m m!}(-R[u:4](\tilde{Z}^{(\vep)}_{w,y}|y^{w}))^m\right)\left(\sum_{m=0}^{\infty}\frac{1}{m!}(-T[G:1](\tilde{Z}^{(\vep)}_{w,y}|y^{w}))^m\right)\nonumber \\  &\qquad \left(\sum_{m=0}^{\infty}\frac{1}{m!}(-T[G:2](\tilde{Z}^{(\vep)}_{w,y}|y^{w}))^m\right)\left(\sum_{m=0}^{\infty}\frac{1}{m!}(-R[G:2](\tilde{Z}^{(\vep)}_{w,y}|y^{w}))^m\right)\nonumber \\  &\qquad \mathbf{1}(\sqrt{\vep}Z_{w,y}\in B_{r_{\vep}}(0))\bigg].
    \end{align}
    We will now simplify the right hand side above. By invoking \eqref{eq:gradbound3} on the set $\{\sqrt{\vep}Z_{w,y}\in B_{r_{\vep}}(0)\}$, we get:
    $$\bigg|\sum_{m=3}^{\infty}\frac{1}{\vep^m m!}(-T[u:3](\tilde{Z}^{(\vep)}_{w,y}|y^{w}))^m\bigg|\le C \vep^{3/2}(-\log{\vep})^{9/2}.$$
    Similarly, we also have:
    $$\bigg|\sum_{m=2}^{\infty}\frac{1}{\vep^m m!}(- T[u:4](\tilde{Z}^{(\vep)}_{w,y}|y^{w}))^m\bigg|\le  C \vep^2 (-\log{\vep})^{4},$$
    $$\bigg|\sum_{m=2}^{\infty}\frac{1}{\vep^m m!}(-R[u:4](\tilde{Z}^{(\vep)}_{w,y}|y^{w}))^m\bigg|\le C \vep^2 (-\log{\vep})^{4} ,$$
    $$\bigg|\sum_{m=3}^{\infty}\frac{1}{m!}(-T[G:1](\tilde{Z}^{(\vep)}_{w,y}|y^{w}))^m\bigg|\le C\vep^{3/2}(-\log{\vep})^{3/2} ,$$
    $$\bigg|\sum_{m=2}^{\infty}\frac{1}{m!}(-T[G:2](\tilde{Z}^{(\vep)}_{w,y}|y^{w}))^m\bigg|\le  C \vep^2 (\log{\vep})^2,$$
    $$\bigg|\sum_{m=2}^{\infty}\frac{1}{m!}(-R[G:2](\tilde{Z}^{(\vep)}_{w,y}|y^{w}))^m\bigg|\le  C \vep^2 (\log{\vep})^2,$$
    on the set $\{\sqrt{\vep}Z_{w,y}\in B_{r_{\vep}}(0)\}$. 
    To handle the other summands in the right hand side of \eqref{eq:gaussbreak}, we observe the following:
    $$\E(T[u:3](\tilde{Z}^{(\vep)}_{w,y}|y^w))=\E(T[G:1](\tilde{Z}^{(\vep)}_{w,y}|y^w))=0,$$
    $$\E\left[\frac{1}{\vep}|R[u:4](\tilde{Z}^{(\vep)}_{w,y}|y^{w})|\mathbf{1}(\sqrt{\vep}Z_{w,y}\in B_{r_{\vep}}(0))\right]\le C\vep \omega(r_{\vep}),$$
    $$\E\left[|R[G:2](\tilde{Z}^{(\vep)}_{w,y}|y^{w})|\mathbf{1}(\sqrt{\vep}Z_{w,y}\in B_{r_{\vep}}(0))\right]\le C\vep \omega(r_{\vep}).$$
    Also,
    \begin{align*}
        &\;\;\;\;\E\bigg[\left(1-\frac{1}{\vep}T[u:3](\tilde{Z}^{(\vep)}_{w,y}|y^{w})+\frac{1}{2\vep^2}\big(T[u:3](\tilde{Z}^{(\vep)}_{w,y}|y^{w})\big)^2\right)\left(1-\frac{1}{\vep}T[u:4](\tilde{Z}^{(\vep)}_{w,y}|y^{w})\right)\\ & \quad \left(1-T[G:1](\tilde{Z}^{(\vep)}_{w,y}|y^{w})+\frac{1}{2}\big(T[G:1](\tilde{Z}^{(\vep)}_{w,y}|y^{w})\big)^2\right)\left(1-T[G:2](\tilde{Z}^{(\vep)}_{w,y}|y^{w})\right)\\ & \quad \mathbf{1}(\sqrt{\vep}Z_{w,y}\in B_{r_{\vep}}(0))\bigg]\\ &=1 + \frac{1}{2}\E\big(T[u:3](y^w+Z_{w,y}|y^w)\big)^2-\E T[u:4](y^w+Z_{w,y}|y^w) \nonumber \\ &\qquad + \frac{1}{2}\E(T[G:1](y^w+Z_{w,y}|y^w))^2 - \E T[G:2](y^w+Z_{w,y}|y^w)  \nonumber \\ & \qquad + \E T[u:3](y^w+Z_{w,y}|y^w) T[G:1](y^w+Z_{w,y}|y^w) \\ &= 1+\vep M[u,G](y) + R_{\vep}(y),
    \end{align*}
    where $\lVert R_{\vep}\rVert_{\infty}\le C\vep^{3/2}(\log{(1/\vep)})^{9/2}$. Combining the above observations and using them in  \eqref{eq:gaussbreak}, we get:
    \begin{align*}
        &\;\;\;\;\bigg|\frac{\sqrt{\mathrm{det}(\nabla^2 u(y^{w}))}}{(2\pi\vep)^{\frac{d}{2}}}\int\limits_{B_{r_{\vep}}(y^{w})} \exp\bigg(\frac{1}{\vep}D[u](x|y)-G(x)+G(y^{w})\bigg)\,dx-1 -\vep M[u,G](y)\bigg| \\ &\quad \le C\left(\vep^{3/2}(\log{(1/\vep)})^{9/2}+ \vep \omega(r_{\vep})\right).
    \end{align*}
    Combining the above observation with \eqref{eq:simpl1} and \eqref{eq:gradbound4}, establishes \eqref{eq:prestim1}.
\end{proof}

Another related result will be useful to establish convergence of the Markov chain. The proof of this new result is simpler than the proof of \cref{lem:prelimestim}. Hence, we skip the details of the proof for brevity.

\begin{lmm}\label{lem:prelimestim2}
Consider the same setting as in \cref{lem:prelimestim}. Then 
\begin{align*}
&\;\;\;\;\sqrt{\mathrm{det}\left(\frac{\partial x\hfill}{\partial x^u}\right)}\frac{1}{(2\pi\vep)^{\frac{d}{2}}}\int (y-x^u)\exp\left(\frac{1}{\vep}\langle x,y\rangle-\frac{1}{\vep}u(x)-\frac{1}{\vep}w(y)-G(y)+G(x^u)\right)\,dy\\ &=\vep \frac{\partial}{\partial x}\left(-G(x^u)+\frac{1}{2}\ldet\left(\frac{\partial x^u}{\partial x\hfill}\right)\right)+o(\vep),
\end{align*}
and 
\begin{align*}
&\;\;\;\;\sqrt{\mathrm{det}\left(\frac{\partial x\hfill}{\partial x^u}\right)}\frac{1}{(2\pi\vep)^{\frac{d}{2}}}\int (y-x^u)(y-x^u)^{\top}\exp\bigg(\frac{1}{\vep}\langle x,y\rangle-\frac{1}{\vep}u(x)-\frac{1}{\vep}w(y)\\ &-G(y)+G(x^u)\bigg)\,dy=\vep \left(\frac{\partial x^u}{\partial x\hfill}\right)+o(\vep),
\end{align*}
where the $o(\cdot)$ term is in the uniform norm.
\end{lmm}
\end{comment}

\begin{longtable}{|p{2cm}|p{9.8cm}|}
\caption{Notation chart}
\label{tab:table3}\\
\hline
\textbf{Notation} & \textbf{Meaning} \\
\hline
$\mathbb{N}$ & Set of natural numbers \\
\hline
$[n]$, $n\in\mathbb{N}$ & The set $\{1,2,\ldots ,n\}$ \\
\hline
$\lmn(A)$ & The minimum eigenvalue of a square matrix $A$ \\
\hline 
$\lmx(A)$ & The maximum eigenvalue of a square matrix $A$ \\
\hline 
$\trc(A)$ & The trace of a square matrix $A$ \\
\hline 
$\lVert A\rVert_{\hs}$ & The Frobenius norm of a square matrix $A$ \\ 
\hline 
$\lVert A\rVert_{\mathrm{op}}$ & The $L^2$ operator norm of a square matrix $A$\\
\hline
$\sqrt{A}$ & Cholesky square root of a symmetric and positive definite matrix $A$, i.e., $\sqrt{A}=S$ if and only if $A=S^2$
\\ 
\hline 
$A\preceq B$ & $B-A$ is non-negative definite \\ 
\hline
$I_d$ & The $d\times d$ identity matrix. We will drop the $d$ from the subscript when the dimension is obvious\\ 
\hline 
$\mathrm{Id}$ & The identity function on $\R^d$\\ \hline 
$|x|$ & The Euclidean norm of a vector $x$ \\ 
\hline 
$T_{\#}\mu$ & Given a probability measures $\mu$ on $\R^d$ and a function $T:\R^d \rightarrow \R^d$, this is the push-forward of $\mu$ by $T$, i.e., the probability distribution of $T(X)$ where $X\sim\mu$ \\ \hline 
$\px \gamma$ & The $X$ marginal density of the joint density $\gamma$ on $\R^d\times \R^d$\\ \hline 
$\py \gamma$ & The $Y$ marginal density of the joint density $\gamma$ on $\R^d\times \R^d$
\\ \hline 
$p_{X|Y}\gamma(\cdot|\cdot)$ & The conditional density of $X$ given $Y$ under the joint density $\gamma$\\ \hline 
$p_{Y|X}\gamma(\cdot|\cdot)$ & The conditional density of $Y$ given $X$ under the joint density $\gamma$\\ \hline 
$\diffcont(\R^d)$ & The space of continuous functions on $\R^d$\\ \hline 
$\diffcont^k(\R^d)$ & The space of $k$ times continuously differentiable functions on $\R^d$ \\ \hline 
$\diffcont^{k,\ell}(\R^d)$ & The space of functions on $[0,\infty)\times \R^d$ (time $\times$ space) with uniformly continuous mixed derivatives up to order $k$ in time and $\ell$ in space \\ \hline 
$B_r(x)$ & The Euclidean ball centered at $x$ with radius $r$ \\ \hline
$\div$ & The Divergence operator\\ \hline 
$\bmd$ & The Laplacian operator\\ \hline
$\nabla$ or $\frac{\partial}{\partial x}$ & The gradient or partial derivative with respect to the usual coordinate chart $x$\\ \hline 
$\nabla^r$ & The $r$-th order multi-derivative with respect to the coordinate chart $x$\\ \hline 
$\int$ & The integral notation without any domain specified will always imply that the integral is over $\R^d$\\ \hline
$\nabla^{-2}$ & The inverse of the Hessian matrix with respect to the coordinate chart $x$\\ \hline 
$\frac{\delta}{\delta \rho}$ & The first variation of a function with respect to a probability measure $\rho$\\ \hline
$\lVert \cdot\rVert_{L^2(\rho)}$ & The $L^2$ norm of a function computed with respect to a probability measure $\rho$\\ \hline
\end{longtable}

%\begin{proof}[Proof of~\cref{thm:conteq}]
%\end{proof}

\begin{comment}

Let 
$\bar{I}(\mu,\Sigma;A):=P( N(\mu,\Sigma)\in A^c).$
Then the above observation yields a further lower bound of 
\begin{align*}
\frac{(2\pi\vep)^{\frac{d}{2}}}{\sqrt{\det(\nabla^2 u(y^{u^*})+\eta I_d)}}\exp\left(\frac{1}{\vep}u^*(y)-f(y^{u^*})+\omega^I_f(y^{u^*};r_y)\right)\left(1-\bar{I}(y^{u^*},\Sigma_{\vep}(y);B_{r_y}(y^{u^*}))\right).
\end{align*}
From elementary computations, it further holds that 
$$\bar{I}(y^{u^*},\vep (\nabla^2 u(y^{u^*})+\eta I_d)^{-1};B_{r_y}(y^{u^*}))\le P(\chi^2_d\ge \vep^{-1}r_y^2\eta).$$
Combining the above observations, we then get:
\begin{align}\label{eq:son1}
   \exp\left(\frac{1}{\vep}\opV[u](y)\right)\ge \frac{(2\pi\vep)^{\frac{d}{2}}}{\sqrt{\det(\nabla^2 u(y^{u^*})+\eta I_d)}}\exp\left(\frac{1}{\vep}u^*(y)-f(y^{u^*})\right)R^{(1)}_{\vep}(y),
\end{align}
where 
$$R^{(1)}_{\vep}(y):=\exp\left(\omega^I_f(y^{u^*};r_y)\right)\left(1-P(\chi^2_d\ge \vep^{-1}r_y^2\eta)\right).$$
Therefore, for $x\in\R^d$, we have:
\begin{align*}
    \exp\left(\frac{1}{\vep}\opS[u](x)\right)\le \frac{1}{(2\pi\vep)^{\frac{d}{2}}}\int \frac{\sqrt{\det(\nabla^2 u(y^{u^*})+\eta I_d)}}{R_{\vep}^{(1)}(y)}\exp\left(\frac{1}{\vep}\langle x,y\rangle-\frac{1}{\vep}u^*(y)-g(y)+f(y^{u^*})\right)\,dy
\end{align*}

$$-\eta I_d\preceq \nabla^2 u(\tilde{y})-\nabla^2 u(y^{u^*})\preceq \eta I_d.$$ 
{\color{blue} We can choose $r_y$ to be free of $y$ if we are on compact domains or on the torus.} 
Then the following holds for all $z\in B_{r_y}(y^{u^*})$:
\begin{align*}
    \langle z,y\rangle - u(z)\ge u^*(y) - \frac{1}{2}(z-y^{u^*})^{\top}\left(\nabla^2 u(y^{u^*})+\eta I_d\right)(z-y^{u^*}).
\end{align*}
For $\delta>0$ and $x,z\in\R^d$, also define 
$$\omega^I_f(z;\delta):=-\delta\sup_{x\in B_{\delta}(z)}\lVert \nabla f(x)\rVert\le \inf_{x\in B_{\delta}(z)} (f(z)-f(x)),$$
and 
$$\Sigma_{\vep}(y):=\vep\left(\nabla^2 u(y^{u^*})+\eta I_d\right)^{-1}.$$

Let $m_{\vep}:=\sqrt{-10\vep\log{\vep}}$. Consequently, 
\begin{align*}
    &\;\;\;\;\;\exp\left(\frac{1}{\vep}\opS[u](x)-\frac{1}{\vep}u(x)\right)\nonumber \\&\le \frac{(C_T)^{\frac{d}{2}}}{(\pi\vep)^{\frac{d}{2}}}\int_{B^c_{m_{\vep}}(x^u)} \frac{1}{R_{\vep}^{(1)}(y)}\exp\left(-\frac{1}{4\vep C_T}\lVert y-x^u\rVert^2-g(y)+f(y^{u^*})\right)\,dy\nonumber \\ &+\frac{1}{(2\pi\vep)^{\frac{d}{2}}}\int_{B_{m_{\vep}}(x^u)} \frac{\sqrt{\det(\nabla^2 u(y^{u^*})+\eta I_d)}}{R_{\vep}^{(1)}(y)}\exp\left(\frac{1}{\vep}(\langle x,y\rangle-u^*(y)-u(x))-g(y)+f(y^{u^*})\right)\,dy\\ &=: T_1+T_2.
\end{align*}
Let $Z_{\vep}\sim N(0,2\vep C_T I_d)$. We focus on the first term $T_1$ above to get:
\begin{align*}
    T_1=2^{\frac{d}{2}} (C_T)^d \E\left[\frac{1}{R_{\vep}^{(1)}(x^u+Z_{\vep})}\exp\left(-g(x^u+Z_{\vep})+f((x^u+Z_{\vep})^{u^*})\right)\bm{1}(\lVert Z_{\vep}\rVert \ge m_{\vep})\right].
\end{align*}
Assume that there exists $\vep_0$ (free of $x$) and $\theta_0:\R^d\to\R$ such that 
\begin{align*}
    \left\{\E\left[\left(\frac{1}{R_{\vep}^{(1)}(x^u+Z_{\vep})}\right)^2\exp\left(-2g(x^u+Z_{\vep})+2f((x^u+Z_{\vep})^{u^*})\right)\right]\right\}^{\frac{1}{2}}\le \theta_0(x). 
\end{align*}
Then
$$T_1\le 2^{\frac{d}{2}}(C_T)^d\sqrt{\theta_0(x)}P\left(\chi^2_d\ge \frac{m_{\vep}^2}{2\vep C_T}\right).$$
By \eqref{eq:basedef}, we have:
\begin{align}\label{eq:temp1}
    &\;\;\;\;\;\exp\left(\frac{1}{\vep}\left(\opV[u_k^{\vep}](y)-\opV[\opS[u_{(k-1)\vep}]](y)\right)\right)\nonumber\\ &=\int\exp\left(-\frac{1}{\vep}(u_k^{\vep}(x)-\opS[u_{(k-1)\vep](x)})\right)\opP[\opS[u_{(k-1)\vep}]](x|y)\,dx,
\end{align}
Next, ideally, we want to look at and simplify
\begin{align*}
&\;\;\;\;\;\int \exp\left(\frac{1}{\vep}\left(\opV[u_k^{\vep}](y)-\opV[\opS[u_{(k-1)\vep}]](y)\right)\right)\opQ[\opV[u_{(k-1)\vep}]](y|z)\,dy\\ &=\int_x \exp\left(-\frac{1}{\vep}(u_k^{\vep}(x)-\opS[u_{(k-1)\vep}](x))\right)\opR[u_{(k-1)\vep}](x|z)\,dx.
\end{align*}
In the last display, we have used \eqref{eq:temp1}. Instead of the term in the first display above, in Soumik da's notes, the term is 
$$\int\exp\left({\color{red}\frac{1}{\vep}\left(\opV[u_k^{\vep}](y)-\opV[u_{k\vep}](y)\right)}\right)\opQ[\opV[u_{k\vep}]](y|z)\,dy.$$
Note that in $\opQ[\opV[u_{k\vep}]](y|z)$, the term $(1/\vep)\opV[u_{k\vep}]$ comes with a negative sign and so, there is no immediate cancellation. Instead, if we had 
\begin{align*}
    &\;\;\;\;\;\int\exp\left({\color{red}\frac{1}{\vep}\left(\opV[u_{k\vep}](y)-\opV[u_k^{\vep}](y)\right)}\right)\opQ[\opV[u_{k\vep}]](y|z)\,dy\\ &=\exp\left(\frac{1}{\vep}\left(u_{k+1}^{\vep}(x)-\opS[u_{k\vep}](x)\right)\right).
\end{align*}
The main issue is not so much with the indexing but the signs going the other way in the exponential.

\medskip

\hrule

\medskip

\SP{SP: New calculations}

\bigskip

We assume nice properties of the two PMA solutions $(u_t,\; t\ge 0)$ and $(v_t=u_t^*,\; t\ge 0)$ and Gaussian approximations for the transition kernels obtained from these solutions. 

Consider the titling of log-Sinkhorn iterates and use the notation $v^\vep_{k+1}:=\opV[u_k^\vep]$.  

\begin{equation}\label{eq:vtilt}
   \frac{1}{\vep}\left( v^\vep_{k+1}(y) - \opV[u_{k\vep}](y) \right)=  \log\int \exp\left( -\frac{1}{\vep} \left( u_k^\vep(x) - u_{k\vep}(x)\right) \right) p^{\vep}[u_{k\vep}](x\mid y)dx.  
\end{equation}

Similarly, with $u_{k+1}^\vep:=\opU[v^\vep_{k+1}]$,  
\begin{equation}\label{eq:utilt}
     \frac{1}{\vep}\left( u_{k+1}^\vep(z) - \opU[v_{k\vep}](z) \right)= \log\int \exp\left( -\frac{1}{\vep} \left( v_{k+1}^\vep(y) - v_{k\vep}(y)\right) \right) q^{\vep}[v_{k\vep}](y\mid z)dy.
\end{equation}

We wish to compare $v^\vep_{k+1}$ with $v_{(k+1)\vep}$ and $u^\vep_{k+1}$ with $u_{(k+1)\vep}$. Rearranging the above identities slightly,
\begin{equation}\label{eq:uvtilt2}
\begin{split}
    \frac{1}{\vep}\left( v^\vep_{k+1}(y) - v_{(k+1)\vep}(y) \right)&=  \frac{1}{\vep}\left(\opV[u_{k\vep}](y)- v_{(k+1)\vep}(y)\right)\\
    + &\log \int \exp\left( -\frac{1}{\vep} \left( u_k^\vep(x) - u_{k\vep}(x)\right) \right) p^{\vep}[u_{k\vep}](x\mid y)dx\\
    \frac{1}{\vep}\left( u_{k+1}^\vep(z) - u_{(k+1)\vep} \right)&=  \frac{1}{\vep}\left(\opU[v_{k\vep}](z) - u_{(k+1)\vep}(z)\right)\\
    + &\log \int \exp\left( -\frac{1}{\vep} \left( v_{k+1}^\vep(y) - v_{k\vep}(y)\right) \right) q^{\vep}[v_{k\vep}](y\mid z)dy.
\end{split}
\end{equation}


Now assume that $u_0^\vep=u_0$. For the induction hypothesis, we will assume all four of these below. 
\begin{enumerate}[(i)]
\item A global growth estimate:
\[
\frac{1}{\vep}\left( u^{\vep}_{k}(x) - u_{k\vep}(x) \right) \sim o_\vep(1) \abs{x} + o(\vep), 
\]
This is a shorthand for the claim that there exist two pairs of functions $\gamma_1, \gamma_2$ and $\delta_1, \delta_2:(0, \infty) \rightarrow \R$ such that 
\begin{equation}\label{eq:ihypothesis}
\lim_{\vep \rightarrow 0+} \gamma_i(\vep)=0,\; \lim_{\vep \rightarrow 0+} \vep^{-1} \delta_i(\vep)=0,
\end{equation}
for $i=1,2$, and 
\[
\gamma_1(\vep) \abs{x} + \delta_1(\vep) \le \frac{1}{\vep}\left( u^\vep_{k}(x) - u_{k\vep}(x) \right) \le  \gamma_2(\vep) \abs{x} + \delta_2(\vep).
\]
In what follows below, the functions $\gamma_i, \delta_i$ will be assumed to be uniform in $k \in [T/\vep]$ for a given $T>0$. 

\item A local estimate:
\[
\int \frac{1}{\vep}\left( u^\vep_{k}(x) - u_{k\vep}(x) \right) p^\vep_{u_{k\vep}}(x \mid y)dx= o(\vep)
\]
\item A global estimate on PMA:
\[
\frac{1}{\vep}\left( \opV[u_{k\vep}](y) - v_{k\vep}(y) \right) \sim o_\vep(1) \abs{y} + o(\vep), 
\]
{\color{blue}We should probably have $v_{(k+1)\vep}$ above as we are comparing $\opV[u_0]$ and $v_{\vep}$ in the first step. So e.g.,
\begin{equation}\label{eq:centering}
\frac{1}{\vep}\left( \opV[u_{0}](y) - v_{\vep}(y) \right)=o_{\vep}(1)+\frac{1}{\vep}\left(\opV[u_0](y)-w_0-\vep {w_0}\right)=o_{\vep}(1)-g(y)-\frac{1}{2}\ldet\frac{\partial y^{w_0}}{\partial y\hfill}.
\end{equation}
}
\item A local estimate on PMA:
\[
\int \frac{1}{\vep}\left( \opV[u_{k\vep}](y) - v_{k\vep}(y) \right) q^\vep[v_{k\vep}](y\mid z) dy= o(\vep). 
\]
{\color{blue}Then \eqref{eq:centering} suggests that we should center the above integral with the term 
$$g(z^{u_0})+\frac{1}{2}\ldet\left(\frac{\partial y^{w_0}}{\partial y\hfill}\right)\bigg|_{y=z^{u_0}}.$$}
\end{enumerate}



By assumption \eqref{eq:ihypothesis} is true for $k=0$. To make the following calculations simpler, we will replace each integral with respect to $p^\vep[u_{k\vep}]$ and $q^\vep[v_{k\vep}]$ by their corresponding normal approximation. This will add an $e^{o(\vep)}$ correction term, which will be absorbed by the integrand. Moreover, we use the well known identity that if $Z$ is a multidimensional normal random variable with mean $O(\vep)$ and variance $O(\vep)$, then 
\begin{equation}\label{eq:normalmgf}
\log \E(e^{ o_\vep(1)\abs{Z} + o(\vep)})= o(\vep^2) + o(\vep)= o(\vep). 
\end{equation}  

Now, by the first identity in \eqref{eq:uvtilt2} for $k=0$, we get
\[
\frac{1}{\vep}\left( v^\vep_{1}(y) - v_\vep(y) \right)= \frac{1}{\vep}\left( \opV[u_0](y) - v_\vep(y) \right) \sim o_\vep(1)\abs{y}.
\]




Assume that the induction hypotheses hold for $(u_k^\vep, u_{k\vep})$ pair. Then, we prove that it holds for the pairs $(v_{k+1}^\vep, v_{k\vep})$ and then $(u_{k+1}^\vep, u_{(k+1)\vep})$. First by \eqref{eq:uvtilt2} and our assumption, 
\[
\begin{split}
 \frac{1}{\vep}\left( v^\vep_{k+1}(y) - v_{(k+1)\vep}(y) \right)&=  \frac{1}{\vep}\left(\opV[u_{k\vep}](y)- v_{(k+1)\vep}(y)\right)\\
    + &\log \int \exp\left( -\frac{1}{\vep} \left( u_k^\vep(x) - u_{k\vep}(x)\right) \right) p^{\vep}[u_{k\vep}](x\mid y)dx\\
    &\sim o_\vep(1) \abs{y} + o(\vep) + \log \int \exp\left( o_\vep(1) \abs{x} + o(\vep)\right) p^\vep_{u_{k\vep}}(x\mid y)dx\\
    &\sim o_\vep(1) \abs{y} + o(\vep).
\end{split}
\]
The final $o(\vep)$ comes from our established bound that under $p^\vep_{u_{k\vep}}$, the random variable (\SP{has to be properly centered}) has mean $O(\vep)$ and variance $O(\vep)$.

    \emph{Proof of \eqref{eq:prestim2}.} We proceed in the same way as in the proof of \eqref{eq:prestim1}. Observe that 
    \begin{align*}
    &\;\;\;\;\E_{\rho_y}(X)-y^w\\ &=\frac{\frac{\sqrt{\mathrm{det}(\nabla^2 u(y^w))}}{(2\pi\vep)^{d/2}}\int (x-y^w)\exp\left(\frac{1}{\vep}D[u](x|y) - G(x) + G(y^w)\right)\,dx}{\frac{\sqrt{\mathrm{det}(\nabla^2 u(y^w))}}{(2\pi\vep)^{d/2}}\int \exp\left(\frac{1}{\vep}D[u](x|y) - G(x) + G(y^w)\right)\,dx}.
    \end{align*}
    We begin with the numerator in the above expression. Note that the denominator has already been dealt with in the proof of \eqref{eq:prestim1}. Note that: 
    \begin{align}\label{eq:gaussbreak2}
        &\;\;\;\;\frac{\sqrt{\mathrm{det}(\nabla^2 u(y^{w}))}}{(2\pi\vep)^{\frac{d}{2}}}\int\limits_{B_{r_{\vep}}(y^{w})} (x-y^w)\exp\left(\frac{1}{\vep}D[u](x|y)-G(x)+G(y^{w})\right)\,dx\nonumber \\ &=\E\bigg[\sqrt{\vep}Z^{(1)}_y\left(\sum_{m=0}^{\infty}\frac{1}{\vep^m m!}(-T[u:3](\tilde{Z}^{(1)}_{\vep,y}|y^{w}))^m\right)\left(\sum_{m=0}^{\infty}\frac{1}{\vep^m m!}(- T[u:4](\tilde{Z}^{(1)}_{\vep,y}|y^{w}))^m\right)\nonumber \\ &\qquad \left(\sum_{m=0}^{\infty}\frac{1}{m!}(-R[h:4](\tilde{Z}^{(1)}_{\vep,y}|y^{w}))^m\right)\left(\sum_{m=0}^{\infty}\frac{1}{m!}(-T[G:1](\tilde{Z}^{(1)}_{\vep,y}|y^{w}))^m\right)\nonumber \\  &\qquad \left(\sum_{m=0}^{\infty}\frac{1}{m!}(-T[G:2](\tilde{Z}^{(1)}_{\vep,y}|y^{w}))^m\right)\left(\sum_{m=0}^{\infty}\frac{1}{m!}(-R[G:2](\tilde{Z}^{(1)}_{\vep,y}|y^{w}))^m\right)\nonumber \\ &\qquad\mathbf{1}(\sqrt{\vep}Z^{(1)}_{y}\in B_{r_{\vep}}(0))\bigg].
    \end{align}
    We focus on two terms in the above expectation as provided below:
    \begin{align*}
        &\;\;\;\;\E\left[\sqrt{\vep}Z_y^{(1)}\left(1-\frac{1}{\vep}T[u:3](\tilde{Z}_{\vep,y}^{(1)}|y^w)\right)\left(1-T[G:1]\right)\right]
        \\ &=-\vep \E\big(Z_y^{(1)}T[u:3](Z^{(1)}_y|0)\big)-\E \big(Z_y^{(1)} T[G:1](Z^{(1)}_y|0)\big)+O(\vep^{3/2}(-\log{\vep})^{3/2})\\ &=\frac{\vep}{2}\nabla(\ldet(\nabla^2 w))(y)-\vep\nabla(G(\nabla w))(y)+O(\vep^{3/2}(-\log{\vep})^{3/2}).
    \end{align*}
    We can repeat the same techniques as in the proof of \eqref{eq:prestim1} to handle the other terms in the aforementioned expansion, as well as the integral over $B_{r_{\vep}}(y^w)$. This implies:
     \begin{align*}
        &\;\;\;\;\frac{\sqrt{\mathrm{det}(\nabla^2 u(y^{w}))}}{(2\pi\vep)^{\frac{d}{2}}}\int\limits_{B_{r_{\vep}}(y^{w})} (x-y^w)\exp\left(\frac{1}{\vep}D[u](x|y)-G(x)+G(y^{w})\right)\,dx \\ &=\frac{\vep}{2}\nabla(\ldet(\nabla^2 w))(y)-\vep\nabla(G(\nabla w))(y)+O(\vep^{3/2}(-\log{\vep})^{3/2})+O(\vep^{3/2}(\omega_1\vee\omega_2)(r_{\vep})).
    \end{align*}
    Combining the above observation with \eqref{eq:prestim1} completes the proof of \eqref{eq:prestim2}.
Similarly, 
\[
\int \frac{1}{\vep}\left( v^\vep_{k+1}(y) - v_{(k+1)\vep}(y) \right) q^{\vep}_{v_{k\vep}}(y \mid z) dy = \int \frac{1}{\vep}\left(\opV[u_{k\vep}](y)- v_{(k+1)\vep}(y)\right) q^{\vep}_{v_{k\vep}}(y \mid z) dy + o(\vep).
\]
That the entire integral above is $o(\vep)$ now follows from the induction hypothesis. 

Repeating the same argument from $v_k^\vep$ to $v^\vep_{k+1}$ gives us the induction step. 

{\color{blue} KL and more.}

TBA. 
    
    Combining the above equality with \eqref{eq:gradbound4} and \cref{lem:proxim}, we get:
    $$\sup_y \bigg|\opV[u_0](y)-w_0(y)-\frac{\vep d}{2}\log{(2\pi\vep)}+\vep f(y^{w_0})-\frac{\vep}{2}\ldet(\nabla^2 w_0(y))+\vep^2 M^{(1)}_{0,y}\bigg|=O(\vep^{9/4}).$$
    We now move on to the gradient of $\opV[u_0](\cdot)$. We first observe that 
    \begin{align*}
    &\;\;\;\;\nabla \opV[u_0](y)-y^{w_0}\\ &=\int (x-y^{w_0})\exp\left(\frac{1}{\vep}\langle x,y\rangle-\frac{1}{\vep}u_0(x)-\frac{1}{\vep}\opV[u_0](y)-f(x)\right)\,dx \\ &=\exp(-\mcI_0(y))\frac{\sqrt{\mathrm{det}(\nabla^2 u_0(y^{w_0}))}}{(2\pi\vep)^{\frac{d}{2}}}\int (x-y^{w_0})\exp\bigg(\frac{1}{\vep}D[u_0](x|y)-f(x)+f(y^{w_0})\bigg)\,dx\\ &=\left(-\vep \nabla (f(\nabla w_0))(y)+\frac{\vep}{2}\nabla(\ldet(\nabla^2 w_0))(y)+O(\vep^2)\right)\exp(-\mcI_0(y)).
    \end{align*}
    Consequently, 
    $$\lVert \nabla \mcI_0(y)\rVert_{\infty}=O(\vep).$$
    Next we discuss the Hessian. Observe that:
    \begin{align*}
        &\;\;\;\;\nabla^2 \opV[u_0](y)\\ &=\frac{1}{\vep}\int (x-y^{w_0})(x-y^{w_0})^{\top}\exp\left(\frac{1}{\vep}\langle x,y\rangle-\frac{1}{\vep}u_0(x)-\frac{1}{\vep}\opV[u_0](y)-f(x)\right)\,dx\\ &\qquad -\frac{1}{\vep}(\nabla \opV[u_0](y)-y^{w_0})(\nabla \opV[u_0](y)-y^{w_0})^{\top}\\ &=\nabla^2 w_0(y)+O(\vep).
    \end{align*}
    This implies 
    $$\lVert \nabla^2 \mcI_0(y)\rVert_{\infty}=O(1).$$
    We now do similar computations for $\mcJ_{\vep}(x)$, $\nabla \mcJ_{\vep}(x)$, and $\nabla^2 \mcJ_{\vep}(x)$. Towards this direction, observe that 
    \begin{align*}
        &\;\;\;\;\;\exp\left(\mcJ_{\vep}(x)+\mcI_0(x^{u_0})\right)\\ &=\frac{1}{(2\pi\vep)^{\frac{d}{2}}\sqrt{\mathrm{det}(\nabla^2 u_{0}(x))}}\int \exp\bigg(\frac{1}{\vep}\langle x,y\rangle - \frac{1}{\vep}u_{0}(x)-\frac{1}{\vep} w_{0}(y)-\log\sqrt{\frac{\mathrm{det}(\nabla^2 w_{0}(y))}{\mathrm{det}(\nabla^2 w_{0}(x^{u_{0}}))}}\nonumber \\ &\qquad +f(y^{w_{0}})-f(x)-g(y)+g(x^{u_{0}})-\mcI_{0}(y)+\mcI_0(x^{u_0})\bigg)\,dy\\ &=1+O(\vep)(1+\lVert \nabla \mathcal{I}_0\rVert_{\infty}^2+\lVert \nabla^2\mathcal{I}_0\rVert_{\infty}).
    \end{align*}
    Consequently,
    $$\lVert \mathcal{J}_{\vep}(x)+\mathcal{I}_0(x^{u_0})\rVert_{\infty}\le O(\vep)(1+\lVert \nabla \mathcal{I}_0\rVert_{\infty}^2+\lVert \nabla^2\mathcal{I}_0\rVert_{\infty}).$$
    Next, we observe that
    \begin{align*}
        &\;\;\;\;\nabla u_1^{\vep}(x)-x^{u_0}\\ &=\int (y-x^{u_0})\exp\left(\frac{1}{\vep}\langle x,y\rangle-\frac{1}{\vep}u_1^{\vep}(x)-\frac{1}{\vep}\opV[u_0](y)-g(y)\right)\,dy\\ &=\exp\left(-\mcJ_{\vep}(x)-\mcI_0(x^{u_0})\right)\frac{\sqrt{\mathrm{det}(\nabla^2 w_0(x^{u_0}))}}{(2\pi\vep)^{\frac{d}{2}}}\int (y-x^{u_0})\exp\bigg(\frac{1}{\vep}\langle x,y\rangle - \frac{1}{\vep} u_0(x) - \frac{1}{\vep} w_0(y) \\ & \qquad - f(x) + f(y^{w_0})  -\log\sqrt{\frac{\ldet(\nabla^2 w_0(y))}{\ldet(\nabla^2 w_0(x^{u_0}))}}-g(y)+g(x^{u_0})-\mcI_0(y)+\mcI_0(x^{u_0})\bigg)\,dy\\ &=\left(1+O(\vep)(\lVert \nabla \mathcal{I}_0\rVert_{\infty}^2+\lVert \nabla^2\mathcal{I}_0\rVert_{\infty})\right)\big(\vep\nabla(f(x)-g(x^{u_0})+\ldet(\nabla^2 u_0(x)))+O(\vep)\lVert \nabla \mcI_0\rVert_{\infty}\big).
    \end{align*}
    In a similar vein, we also have:
    \begin{align*}
        &\;\;\;\;\nabla^2 u_1^{\vep}(x)\\ &=\frac{1}{\vep}\int (y-x^{u_0})(y-x^{u_0})^{\top}\exp\left(\frac{1}{\vep}\langle x,y\rangle-\frac{1}{\vep}u_1^{\vep}(x)-\frac{1}{\vep}\opV[u_0](y)-g(y)\right)\,dy\\ &\qquad -\frac{1}{\vep}\big(\nabla u_1^{\vep}(x)-\nabla u_0(x)\big)\big(\nabla u_1^{\vep}(x)-\nabla u_0(x)\big)^{\top}\\ &=\nabla^2 u_0(x)+O(\vep)(\lVert \nabla \mathcal{I}_0\rVert_{\infty}^2+\lVert \nabla^2\mathcal{I}_0\rVert_{\infty}).
    \end{align*}

\noindent Generate $X_0^{\vep}\sim (\nabla u_0^*)_{\# e^{-g}}$. Given $X_0^{\vep}=x$, sample $Y_1^{\vep}\sim \opQ[\opV[u_0^{\vep}]](\cdot|x)$ and given $Y_1^{\vep}=y$, sample $X_1^{\vep}\sim \opP[\opS[u_0]](\cdot|y)$. In generally, given $X^{\vep}_{i}=x$, sample $Y_{i+1}^{\vep}\sim \opQ[\opV[u^{\vep}_i]](\cdot|x)=:\opQ_i(\cdot|x)$ and given $Y_{i+1}^{\vep}=y$, sample $X_{i+1}^{\vep}\sim \opP[\opS[u_i^{\vep}]](\cdot|y)=:\opP_i(\cdot|y)$. We proceed similarly to construct the Markov chain $(X_0^{\vep},Y_1^{\vep},X_1^{\vep},\ldots ,X_{k}^{\vep})$.

\noindent We generate the corresponding Markov chain with the PMA (see \eqref{eq:pma}). Given $X_{k\vep}=x$, sample 
$$Y_{(k+1)\vep}\ |\ X_{k\vep}=x\sim  N\left(x^{u_{k\vep}}+\vep\left(\frac{\partial f}{\partial x}(x)-\frac{1}{2}\frac{\partial g}{\partial x}(x^{u_{k\vep}})-\frac{1}{2}\frac{\partial h_{k\vep}}{\partial x\hfill}(x)\right),\ \vep \left(\frac{\partial x^{u_{k\vep}}}{\partial x\hfill}\right)\right).$$
We write the corresponding transition kernel by $\tpQ{k}(\cdot|x)$. Recall that $w_t=u_t^*$. Given $Y_{(k+1)\vep}=y$, sample 
$$X_{(k+1)\vep}\ |\ Y_{(k+1)\vep}=y\sim  N\left(y^{w_{k\vep}}-\vep\left(2\frac{\partial f}{\partial y}(y^{w_{k\vep}})-\frac{\partial h_{k\vep}}{\partial y}(y^{w_{k\vep}})\right),\ \vep \left(\frac{\partial y^{w_{k\vep}}}{\partial y\hfill}\right)\right).$$
We write the corresponding transition kernel by $\tpP{k}(\cdot|y)$. We start with $X_0=X_0^{\vep}$ as in the preceding paragraph. This yields the Markov chain $(X_0,Y_{\vep},X_{\vep},\ldots , X_{k\vep})$. 


\noindent Throughout this part, we assume that there exists fixed $T\in (0,\infty)$ such that $0<k\vep\le T$.

\noindent The goal is to show that 
\begin{equation}\label{eq:kldgoal}
\KL{(X_0,Y_{\vep},X_{\vep},\ldots , X_{k\vep})}{(X_0^{\vep},Y_1^{\vep},X_1^{\vep},\ldots ,X_{k}^{\vep})}=o_{\vep}(1),
\end{equation}
where the above limit is to be understood as $\vep\to 0$ and $k\vep\le T$. Recall that $X_0$ and $X_0^{\vep}$ have the same distribution. We want to eventually proceed inductively to prove \eqref{eq:kldgoal}. 

\vspace{0.05in}

\noindent \emph{$k=1$ case}. We begin with 
\begin{align*}
&\;\;\;\;\;\KL{(X_0,Y_{\vep},X_{\vep})}{(X_0^{\vep},Y_{1}^{\vep},X_{1}^{\vep})}\\ &=\E_{Y\sim Y_{\vep}} \left[\KL{\tpP{ }(\cdot|Y)}{\opP_1(\cdot|Y)}\right]+\E_{X\sim X_0}\left[\KL{\tpQ{ }(\cdot|X)}{\opQ_1(\cdot|X)}\right]
\end{align*}
The following lemma provides a series of recursions that express the Sinkhorn potentials \eqref{eq:twostepit}  explicitly in terms of the PMA \eqref{eq:pma} and the dual PMA \eqref{eq:dualPMA}. We set up some notation first. Define 
$$\mcI_0(y):=\log\frac{1}{(2\pi\vep)^{\frac{d}{2}}\sqrt{\mathrm{det}(\nabla^2 w_0(y))}}\int \exp\left(\frac{1}{\vep}\langle x,y\rangle - \frac{1}{\vep}u_0(x)-\frac{1}{\vep}w_0(y)-f(x)+f(y^{w_0})\right)\,dx,$$
and $\mcJ_{0}(x)=0$. 
Finally for $k\ge 0$, define:
$$\tilde{u}_{(k+1)\vep}(x)=u_{k\vep}(x)+\vep(f(x)-g(x^{u_{k\vep}})+\ldet(\nabla^2 u_{k\vep}(x))),$$
\begin{align*}
    \mcJ_{(k+1)\vep}(x)&:=\log\frac{1}{(2\pi\vep)^{\frac{d}{2}}\sqrt{\mathrm{det}(\nabla^2 u_{k\vep}(x))}}\int \exp\bigg(\frac{1}{\vep}\langle x,y\rangle - \frac{1}{\vep}u_{k\vep}(x)-\frac{1}{\vep} w_{k\vep}(y)\nonumber \\ &\qquad -\log\sqrt{\frac{\mathrm{det}(\nabla^2 w_{k\vep}(y))}{\mathrm{det}(\nabla^2 w_{k\vep}(x^{u_{k\vep}}))}}+f(y^{w_{k\vep}})-f(x)-g(y)+g(x^{u_{k\vep}})-\mcI_{k\vep}(y)\bigg)\,dy,
\end{align*}
\begin{align*}
    \mcI_{(k+1)\vep}(y)&:=\log\frac{1}{(2\pi\vep)^{\frac{d}{2}}\sqrt{\mathrm{det}(\nabla^2 w_{(k+1)\vep}(y))}}\int \exp\bigg(\frac{1}{\vep}\langle x,y\rangle - \frac{1}{\vep}\tilde{u}_{(k+1)\vep}(x)-\frac{1}{\vep}w_{(k+1)\vep}(y)\nonumber \\ &\qquad -f(x)+f(y^{w_{(k+1)\vep}})-\mcJ_{(k+1)\vep}(x)\bigg)\,dy,
\end{align*}

\begin{remark}
Note that both the $\mcI_{k\vep}$s and $\mcJ_{k\vep}$s are all purely functions of the PMA \eqref{eq:pma} and the dual PMA \eqref{eq:dualPMA}.
\end{remark}

\begin{lmm}\label{lem:proxim}
Suppose $u_0(\cdot)$ is the initial potential for both the Sinkhorn algorithm and the PMA. Then the following relations hold for all $k\ge 0$ and $x,y\in\R^d$:
\begin{equation}\label{eq:system3}
\begin{split}
\opV[u_{k}^{\vep}](y)=w_{k\vep}(y)+\frac{\vep d}{2}\log{(2\pi\vep)}-\vep f(y^{w_{k\vep}})+\frac{\vep}{2}\ldet(\nabla^2 w_{k\vep}(y))+\vep \mcI_{k\vep}(y),
\end{split}
\end{equation}
and 
\begin{equation}\label{eq:system4}
    u_{k+1}^{\vep}(x)=\tilde{u}_{(k+1)\vep}(x)+\vep \mcJ_{(k+1)\vep}(x).
\end{equation}
\end{lmm}


\begin{proof}
    The proof of \eqref{eq:system3} and \eqref{eq:system4} proceeds via induction. \par 

    First consider the case when $k=0$. Observe that 
    \begin{align*}
        \opV[u_0](y)&=\vep\log\int \exp\left(\frac{1}{\vep}\langle x,y\rangle - \frac{1}{\vep}u_0(x)-f(x)\right)\,dx\\ &=w_0(y)+\frac{\vep d}{2}\log{(2\pi\vep)}-\vep f(y^{w_0})-\frac{\vep}{2}\ldet(\nabla^2 u_0(y^{w_0})) \\ & + \vep \log\frac{\sqrt{\mathrm{det}(\nabla^2 u_0(y^{w_0}))}}{(2\pi\vep)^{\frac{d}{2}}}\int \exp\left(\frac{1}{\vep}\langle x,y\rangle - \frac{1}{\vep}w_0(y)-\frac{1}{\vep}u_0(x)-f(x)+f(y^{w_0})\right)\,dx.
    \end{align*}
    Noting that  $\nabla^2 u_0(y^{w_0})=\nabla^{-2}w_0(y)$, establishes \eqref{eq:system3} for $k=1$. Next, 
    \begin{align*}
        &\;\;\;\;u_1^{\vep}(x)-\tilde{u}_{\vep}(x)\\ &=\vep\log\int \exp\left(\frac{1}{\vep}\langle x,y\rangle - \frac{1}{\vep}\opV[u_0](y)-\frac{1}{\vep}u_{0}(x)-g(y)-f(x)+g(x^{u_0})-\ldet(\nabla^2 u_0(x))\right)\,dy\\ &=\vep\log\frac{1}{(2\pi\vep)^{\frac{d}{2}}}\int \exp\bigg(\frac{1}{\vep}\langle x,y\rangle - \frac{1}{\vep}w_0(y)-\frac{1}{\vep}u_0(x)+f(y^{w_0})-\frac{1}{2}\ldet(\nabla^2 w_0(y))\\ &\qquad\qquad -g(y)-f(x)+g(x^{u_0})-\ldet(\nabla^2 u_0(x))-\mcI_0(y)\bigg)\,dy\\ &=\vep\log\frac{1}{(2\pi\vep)^{\frac{d}{2}}\sqrt{\mathrm{det}(\nabla^2 u_{0}(x))}}\int \exp\bigg(\frac{1}{\vep}\langle x,y\rangle - \frac{1}{\vep}u_{0}(x)-\frac{1}{\vep}w_0(y)-g(y)+g(x^{u_0})\\ &\qquad +f(y^{w_0})-f(x)-\log\sqrt{\frac{\mathrm{det}(\nabla^2 w_{0}(y))}{\mathrm{det}(\nabla^2 w_{0}(x^{u_{0}}))}}-\mcI_{0}(y)\bigg)\,dy\\ &=\vep \mcJ_{\vep}(x).
    \end{align*}
    where we have used $\nabla^2 w_{\vep}(x^{u_{\vep}})=\nabla^{-2} u_{\vep}(x)$. This establishes \eqref{eq:system4} for the case $k=1$.\par 

    We now assume \eqref{eq:system3} and \eqref{eq:system4} for all $k\le k_0$. For $k=k_0+1$, we then have from the induction hypothesis:
    \begin{align*}
        \opV[u_{k_0+1}^{\vep}](y)&=\vep\log\int\exp\left(\frac{1}{\vep}\langle x,y\rangle - \frac{1}{\vep}\tilde{u}_{k_0+1}^{\vep}(x)-f(x)\right)\,dx\\ &=\vep \log\int \exp\left(\frac{1}{\vep}\langle x,y\rangle - \frac{1}{\vep}\tilde{u}_{(k_0+1)\vep}(x)-f(x)-\mcJ_{(k_0+1)\vep}(x)\right)\,dx\\ &=w_{(k_0+1)\vep}(y)+\frac{\vep d}{2}\log{(2\pi\vep)}-\vep f(y^{w_{(k_0+1)\vep}})+\frac{\vep}{2}\ldet(\nabla^2 w_{(k_0+1)\vep}(y))\\ &+\vep \log\frac{1}{(2\pi\vep)^{\frac{d}{2}}\sqrt{\mathrm{det}(\nabla^2 w_{(k_0+1)\vep}(y))}}\int \exp\bigg(\frac{1}{\vep}\langle x,y\rangle - \frac{1}{\vep}\tilde{u}_{(k_0+1)\vep}(x)\nonumber \\ &-\frac{1}{\vep}w_{(k_0+1)\vep}(y) -f(x)+f(y^{w_{(k_0+1)\vep}(y)})-\mcJ_{(k_0+1)\vep}(x)\bigg)\,dy
    \end{align*}
    This establishes \eqref{eq:system3} for $k=k_0+1$. A similar computation shows \eqref{eq:system4} holds for $k=k_0+1$ as well, thereby completing the proof.
\end{proof}

\noindent The following technical result is a version of Laplace's method of approximating integrals. We skip the details of the proof for brevity.

\begin{lmm}
  Recall that $\mu(dx)=\exp(-f(x))$ is a probability measure, and assume that $\lVert \nabla f\rVert_{\infty}\vee \lVert \nabla^2 f\rVert_{\infty}<\infty$. Also assume that $u:\R^d\to\R^d$ is a smooth function with $\inf_{x}\lmn(\nabla^2(u(x)))>0$, and $\lVert \nabla^3 u\rVert_{\infty}\vee \lVert \nabla^4 u\rVert_{\infty}<\infty$. Define 
  $$h_{\vep}[u](y)=\int \exp\left(\frac{1}{\vep}\langle x,y\rangle - \frac{1}{\vep} u(x)-f(x)\right)\,dx.$$
  Then there exists a constant 
  $$C\equiv C\bigg(\lVert \nabla f\rVert_{\infty},\ \lVert \nabla^2 f\rVert_{\infty},\ \inf_{x}\lmn(\nabla^2(u(x))),\ \lVert \nabla^3 u\rVert_{\infty},\ \lVert \nabla^4 u\rVert_{\infty}\bigg),$$
  such that for all $0<\vep\le 1/2$, then,
  $$\sup_{y}\bigg|\exp\bigg(-\frac{u^*(y)}{\vep}+f(y^{u^*})\bigg)\frac{h_{\vep}[u](y)}{(2\pi\vep)^{\frac{d}{2}}}\sqrt{\det(\nabla^2 u(y^{u^*}))}-1\bigg|\le C\vep \left(\log{\left(\frac{1}{\vep}\right)}\right)^3.$$
\end{lmm}

\noindent The following technical lemma is a refinement of \cite[Proposition 1.11, page 287]{Revuzyor}

\begin{lmm}\label{lem:Revyor1}
Given a function $\phi:\R^d\to\R$ which is twice continuously differentiable and satisfies 
$$\sup_{x\in\R^d} \lVert \nabla^2\phi(x)\rVert_{\infty}<\infty.$$
Also assume that $\nabla^2\phi(\cdot)$ is a uniformly continuous function on $\R^d$. 
Then,
$$\lim_{t\to 0} \bigg\lVert\frac{e^{-4t\lVert \nabla\phi\rVert^2}}{1+t^{1/4}\lVert \nabla \phi\rVert^3}\bigg|\frac{e^{\phi}P_t(e^{-\phi})-1}{t}-\frac{1}{2}\left(-\Delta \phi+\lVert \nabla \phi\rVert^2\right)\bigg|\bigg\rVert_{\infty}=0.$$
Here $$P_t(e^{-\phi})(x)=\E e^{-\phi(x+\sqrt{t}Z)},$$
where $Z\sim N(0,\mathrm{I_d})$.
\end{lmm}

\begin{proof}
     Assume $C:=\sup_{x\in\R^d} \lVert \nabla^2\phi(x)\rVert_{\infty}<\infty$ and define 
    $g(\delta):=\sup_{\lVert x-y\rVert\le \delta}\lVert \nabla^2 \phi(x)-\nabla^2\phi(y)\rVert_{\mathrm{op}}$. By our assumption, $g(\delta)\to 0$ as $\delta\to 0$.  Throughout this proof, we will hide constants depending on $d$ by the generic notation $C_A$ which could change from one line to the next. The idea is to keep taking the maximums of the constants that arise in each step and keep redefining $C_A$ as this maximum. Note that $C_A$ does not depend on $\phi(\cdot)$.\par

    
    By a second order Taylor approximation, we have:
    \begin{align*}
        \phi(x+\sqrt{t}Z)=\phi(x)+\sqrt{t}\nabla \langle\phi(x),Z\rangle+\frac{t}{2}Z^{\top}\nabla^2\phi(x)Z+\frac{t}{2}Z^{\top}(D_t-\nabla^2\phi(x))Z,
    \end{align*}
    where
    $$D_t:=2\int_0^1 (1-\lambda)\nabla^2\phi(x+\lambda\sqrt{t}Z)\,d\lambda.$$
    The above integral is to be interpreted entrywise. We hide the dependence on $x,Z,\phi$ in our notation for simplicity. Note that $D_t$ is a measurable function in $Z$ under our assumptions. Further, given $R>0$ and on the set $\lVert Z\rVert\le R$, the following estimate holds: 
    \begin{align}\label{eq:calllate}
    \sup_{x\in\R^d}\lVert D_t-\nabla^2\phi(x)\rVert_{\mathrm{op}}\le C_A g(\sqrt{t}R).
    \end{align}
    Consequently, we can write:
    \begin{align*}
    e^{\phi(x)}P_t(e^{-\phi})(x)-1=\E\big[e^{-\sqrt{t}\langle Z,\nabla \phi(x)\rangle-\frac{t}{2}Z^{\top}\nabla^2\phi(x)Z-\frac{t}{2}Z^{\top}(D_t-\nabla^2\phi(x))Z}\big]-1.
    \end{align*}

    \SP{Don't follow the calculation below. Please expand. Before you take expectation, you need to make sure that $\xi$ is a measurable function of $Z$. Is that obvious?} \ND{Switching to integral form of remainder  where measurability seems to follow from continuity and boundedness of Hessian. More steps added.}
    
    By a full Taylor series expansion of the exponential function, we get:
    \begin{align*}
        &\;\;\;\;\;e^{\phi(x)}P_t(e^{-\phi})(x)-1\\ &=\E\bigg[e^{-\sqrt{t}\langle Z,\nabla \phi(x)\rangle-\frac{t}{2}Z^{\top}\nabla^2\phi(x)Z}\bigg(\sum_{k\ge 0} \frac{(-t)^k}{2^k k!}(Z^{\top}(D_t-\nabla^2\phi(x))Z)^k\bigg)\bigg]-1\\ &=\E\bigg[e^{-\sqrt{t}\langle Z,\nabla \phi(x)\rangle-\frac{t}{2}Z^{\top}\nabla^2\phi(x)Z}\bigg(\sum_{k\ge 1} \frac{(-t)^k}{2^k k!}(Z^{\top}(D_t-\nabla^2\phi(x))Z)^k\bigg)\bigg]\\ &\qquad\qquad +\sum_{k\ge 1} \E\bigg[\frac{\big(-\sqrt{t}\langle Z,\nabla \phi(x)\rangle-\frac{t}{2}Z^{\top}\nabla^2\phi(x)Z\big)^k}{k!}\bigg]\\ &=\E\bigg[e^{-\sqrt{t}\langle Z,\nabla \phi(x)\rangle-\frac{t}{2}Z^{\top}\nabla^2\phi(x)Z}\bigg(\sum_{k\ge 1} \frac{(-t)^k}{2^k k!}(Z^{\top}(D_t-\nabla^2\phi(x))Z)^k\bigg)\bigg]\\ &\qquad\qquad-\frac{t}{2}\Delta \phi(x)+\frac{t}{2}\lVert \nabla \phi(x)\rVert^2+\frac{t^2}{8}\big((\Delta \phi(x))^2+\lVert \nabla^2 \phi(x)\rVert_{\hs}^2\big)\\ &\qquad\qquad +\sum_{k\ge 3} \E\bigg[\frac{\big(-\sqrt{t}\langle Z,\nabla \phi(x)\rangle-\frac{t}{2}Z^{\top}\nabla^2\phi(x)Z\big)^k}{k!}\bigg].
    \end{align*}
    Therefore,
    \begin{align}\label{eq:boundreq}
    &\;\;\;\;\;\bigg|\frac{e^{\phi(x)}P_t(e^{-\phi})(x)-1}{t}-\frac{1}{2}\left(-\Delta \phi(x)+\lVert \nabla \phi(x)\rVert^2\right)\bigg|\nonumber \\&\le \bigg|\E\bigg[e^{-\sqrt{t}\langle Z,\nabla \phi(x)\rangle-\frac{t}{2}Z^{\top}\nabla^2\phi(x)Z}\bigg(\sum_{k\ge 1} \frac{(-1)^k t^{k-1}}{2^k k!}(Z^{\top}(D_t-\nabla^2\phi(x))Z)^k\bigg)\bigg]\bigg|+tC_A\nonumber \\&+\frac{1}{2t}\sum_{k\ge 3}\frac{(2\sqrt{t})^k}{k!}\E|\langle Z,\nabla \phi(x)\rangle|^k+\frac{1}{2}\sum_{k\ge 3}\frac{t^{k-1}}{k!}\E\big[\big(\lVert Z\rVert^2 Cd\big)^k\big].
    \end{align}
    We bound each of the terms on the right hand side of \eqref{eq:boundreq}. Straightforward computations using Gaussian moment generating functions imply, for $t\le (1\wedge (Cd^2)^{-1})/256$, that:
    \begin{align}\label{eq:boundreq1}
    \frac{1}{2t}\sum_{k\ge 3}\frac{(2\sqrt{t})^k}{k!}\E|\langle Z,\nabla \phi(x)\rangle|^k\le C_A\sqrt{t}\lVert \nabla \phi(x)\rVert^3\exp(4t\lVert \nabla \phi(x)\rVert^2),
    \end{align}
    and 
    \begin{align}\label{eq:boundreq2}
    \frac{1}{2}\sum_{k\ge 3}\frac{t^{k-1}}{k!}\big(\lVert Z\rVert^2 Cd\big)^k\le C_A t^2.
    \end{align}
    Next let $$Z_{x,t}=(I+t\nabla^2\phi(x))^{-1}\sqrt{t}\nabla \phi(x)+(I+t\nabla^2\phi(x))^{-1/2}Z.$$
    Then by a change of variable argument, we get:
    \begin{align}\label{eq:boundreq2}
        &\;\;\;\;\;\bigg|\E\bigg[e^{-\sqrt{t}\langle Z,\nabla \phi(x)\rangle-\frac{t}{2}Z^{\top}\nabla^2\phi(x)Z}\bigg(\sum_{k\ge 1} \frac{(-1)^k t^{k-1}}{2^k k!}(Z^{\top}(D_t-\nabla^2\phi(x))Z)^k\bigg)\bigg]\bigg|\nonumber \\ &=\frac{e^{\frac{t}{2}\lVert \nabla \phi(x)\rVert^2}}{2\sqrt{\mbox{det}(I+t\nabla^2\phi(x))}}\E\left[Z_{x,t}^{\top}(D_t-\nabla^2\phi(x))Z_{x,t}e^{\frac{t}{2}Z_{x,t}^{\top}(D_t-\nabla^2\phi(x))Z_{x,t}}\right]\nonumber \\ &\le C_A e^{\frac{t}{2}\lVert \nabla \phi(x)\rVert^2}\E\left[\big|Z_{x,t}^{\top}(D_t-\nabla^2\phi(x))Z_{x,t}\big|e^{tCd\lVert Z_{x,t}\rVert^2}\right]\nonumber \\ &\le C_Ae^{\frac{t}{2}\lVert \nabla \phi(x)\rVert^2}\E\left[\lVert D_t-\nabla^2\phi(x)\rVert_{\mathrm{op}}\lVert Z_{x,t}\rVert^2 e^{tCd\lVert Z_{x,t}\rVert^2}\right].
    \end{align}
We will break the above expectation into two parts with indicators $\lVert Z\rVert\le R$ and $\lVert Z\rVert> R$. On the set $\lVert Z\rVert\le R$, the following holds:
\begin{equation}\label{eq:calate}
\lVert Z_{x,t}\rVert^2\le 8t\lVert \nabla\phi(x)\rVert^2+4\lVert Z\rVert^2.
\end{equation}
Therefore, 
\begin{align}\label{eq:boundreq3}
    &\;\;\;\;e^{\frac{t}{2}\lVert \nabla \phi(x)\rVert^2}\E\left[\lVert D_t-\nabla^2\phi(x)\rVert_{\mathrm{op}}\lVert Z_{x,t}\rVert^2 e^{tCd\lVert Z_{x,t}\rVert^2}\mathbf{1}(\lVert Z\rVert\le R)\right]\nonumber \\&\le C_A\exp(t\lVert \nabla \phi(x)\rVert^2)g(\sqrt{t}R)\left(t\lVert \nabla \phi(x)\rVert^2\E[e^{4tCd\lVert Z\rVert^2}]+\E[\lVert Z\rVert^2e^{4tCd\lVert Z\rVert^2}]\right)\nonumber \\ &\le C_A(1+t\lVert\nabla\phi(x)\rVert^2)e^{t\lVert\nabla\phi(x)\rVert^2}g(\sqrt{t}R).
\end{align}
In the first bound above, we have used \eqref{eq:calllate} and \eqref{eq:calate}. Next, observe that, 
\begin{align}\label{eq:boundreq4}
    &\;\;\;\;e^{\frac{t}{2}\lVert \nabla \phi(x)\rVert^2}\E\left[\lVert D_t-\nabla^2\phi(x)\rVert_{\mathrm{op}}\lVert Z_{x,t}\rVert^2 e^{tCd\lVert Z_{x,t}\rVert^2}\mathbf{1}(\lVert Z\rVert> R)\right]\nonumber \\&\le C_A\exp(t\lVert \nabla \phi(x)\rVert^2)\left(t\lVert \nabla \phi(x)\rVert^2\E[e^{4tCd\lVert Z\rVert^2}\mathbf{1}(\lVert Z\rVert>R)]+\E[\lVert Z\rVert^2e^{4tCd\lVert Z\rVert^2}\mathbf{1}(\lVert Z\rVert>R)]\right)\nonumber \\ &\le C_A(1+t\lVert\nabla\phi(x)\rVert^2)e^{t\lVert\nabla\phi(x)\rVert^2}\exp(-R^2/4d).
\end{align}
In the first bound above, we have used \eqref{eq:calate} and the boundedness of the Hessian. 

We next note the elementary inequality $\lVert \nabla\phi(x)\rVert^2\le 1+\lVert \nabla\phi(x)\rVert^3$.  By combining \eqref{eq:boundreq}, \eqref{eq:boundreq1}, \eqref{eq:boundreq2}, \eqref{eq:boundreq3}, and \eqref{eq:boundreq4}, we have the following bound:
\begin{align*}
    &\;\;\;\;\frac{C_A^{-1} e^{-4t\lVert \nabla\phi(x)\rVert^2}}{1+t^{1/4}\lVert \nabla \phi(x)\rVert^3}\bigg|\frac{e^{\phi(x)}P_t(e^{-\phi})(x)-1}{t}-\frac{1}{2}\left(-\Delta \phi(x)+\lVert \nabla \phi(x)\rVert^2\right)\bigg|\\ &\le \frac{t(1+t)}{1+t^{1/4}\lVert \nabla\phi(x)\rVert^3}+\frac{\sqrt{t}\lVert \nabla \phi(x)\rVert^3}{1+t^{1/4}\lVert \nabla\phi(x)\rVert^3}+\frac{1+t(1+\lVert \nabla\phi(x)\rVert^3)}{1+t^{1/4}\lVert \nabla\phi(x)\rVert^3}\bigg(\exp\bigg(\frac{-R^2}{4d}\bigg)+g(\sqrt{t}R)\bigg)\\ &\le t(1+t)+t^{1/4}+2\bigg(\exp\bigg(\frac{-R^2}{4d}\bigg)+g(\sqrt{t}R)\bigg).
\end{align*}
We can now arrive at the conclusion by taking supremum over $x$ and limits as $t\to 0$ followed by $R\to\infty$.
\end{proof}

For the next result, we define some basic notation first. 
For $y,z\in\R^d$ and $v\in C(\R^d)$, let us define
\begin{equation*}\label{eq:con3}
    \opQ[v](y|z):=\exp\left(\frac{1}{\vep}\langle y,z\rangle-\frac{1}{\vep}v(y)-\frac{1}{\vep}\opU[v](z)-g(y)\right).
\end{equation*}
It is easy to see that, for every given $z$, $\opQ[v](\cdot|z)$ is a probability density. Hence, it is a Markov transition density. 

Similarly, for $x,y\in\R^d$ and $u\in C(\R^d)$, we define
\begin{equation*}
    \opP[u](x|y):=\exp\left(\frac{1}{\vep}\langle x,y\rangle-\frac{1}{\vep}u(x)-\frac{1}{\vep}\opV[u](y)-f(x)\right).
\end{equation*}
As before, $\opP[u](\cdot|y)$ is a Markov transition density for every given $y$. 

We then define the two-step Markov transition density by combining $\opP$ and $\opQ$. For $x,z\in\R^d$ and $u\in C(\R^d)$, as follows:
\begin{equation*}
    \opR[u](x|z):=\int \opQ[\opV[u]](y|z)\opP[\opS[u]](x|y)\,dy.
\end{equation*}
Note that by \eqref{eq:con1} and \eqref{eq:con2}, the following identities hold:
\begin{align*}
    p_{Y|X}\gvp_{k}(y|z)=\opQ[\opV[u_{k-1}^{\vep}]](y|z),\quad p_{X|Y}\gvp_{k+1}(x|y)=\opP[\opS[u_{k-1}^{\vep}]](x|y).
\end{align*}

\begin{lmm}\label{lem:recursion}
Define 
$$\xi_{k,\vep}^{(2)}(x):=\frac{1}{\vep}u_{k\vep}(x)-\frac{1}{\vep}u_{k}^{\vep}(x).$$
For $x_1,x_2\in\R^d$ and any function $\lambda:\R^d\to\R$, define 
$$\mfR{\lambda}{x_1}{x_2}:=\lambda(x_1)-\lambda(x_2).$$
Then the following holds for all $k\ge 0$, $\vep>0$, and $x\in\R^d$:
\begin{align}\label{eq:step15}
\xi_{k+1,\vep}^{(2)}(x)-\xi_{k,\vep}^{(2)}(x)&=\frac{1}{\vep}\left(u_{(k+1)\vep}(x)-\opS[u_{k\vep}](x)\right)\nonumber \\&-\log{\E_{Y\sim \opQ[\opV[u_{k\vep}]](\cdot|x)}\left[\frac{1}{\E_{Z\sim \opP[u_{k\vep}](\cdot|Y)} \exp\left(\mfR{\xi_{k,\vep}^{(2)}}{Z}{x}\right)}\right]}.
\end{align}
\end{lmm}

\begin{proof}
    Note that
\begin{align}\label{eq:step12}
    &\;\;\;\;\;\frac{1}{\vep}\opU[\opV[u_k^{\vep}]](x)-\frac{1}{\vep}\opU[\opV[u_{k\vep}]](x)\nonumber \\ &=\log\int \exp\left(\frac{1}{\vep}\langle x,y\rangle - \frac{1}{\vep}\opU[\opV[u_{k\vep}]](x)-\frac{1}{\vep}\opV[u_k^{\vep}](y)-g(y)\right)\,dy\nonumber \\ &=\log\E_{Y\sim \opQ[\opV[u_{k\vep}]](\cdot|x)} \exp\left(\frac{1}{\vep}\left(\opV[u_{k\vep}](Y)-\opV[u_k^{\vep}](Y)\right)\right)\nonumber \\ &=\xi_{k,\vep}^{(1)}(x^{u_{k\vep}})+\log\E_{Y\sim \opQ[\opV[u_{k\vep}]](\cdot|x)} \exp\left(\mfR{\xi_{k,\vep}^{(1)}}{Y}{x^{u_{k\vep}}}\right),
\end{align}
where for all $y\in\R^d$, we define
$$\xi_{k,\vep}^{(1)}(y):=\frac{1}{\vep}\opV[u_{k\vep}](y)-\frac{1}{\vep}\opV[u_k^{\vep}](y).$$
Next, let us fix $y\in\R^d$, and recall that $w_{k\vep}=u_{k\vep}^*$. Note that by a similar computation as above, we can write:
\begin{align}\label{eq:step13}
    -\xi_{k,\vep}^{(1)}(y)&=\frac{1}{\vep}\opV[u_k^{\vep}](y)-\frac{1}{\vep}\opV[u_{k\vep}](y)\nonumber \\ &=\xi_{k,\vep}^{(2)}(y^{w_{k\vep}})+\log\E_{Z\sim \opP[u_{k\vep}](\cdot|y)} \exp\left(\mfR{\xi_{k,\vep}^{(2)}}{Z}{y^{w_{k\vep}}}\right).
\end{align}

By combining \eqref{eq:step12} and \eqref{eq:step13}, we then get that for any $x\in\R^d$
\begin{align}\label{eq:step14}
    &\;\;\;\;\;\frac{1}{\vep}u_{k+1}^{\vep}(x)-\frac{1}{\vep}\opS[u_{k\vep}](x)\nonumber \\ &=-\xi_{k,\vep}^{(2)}(x)-\log\E_{Z\sim \opP[u_{k\vep}](\cdot|x^{u_{k\vep}})} \exp\left(\mfR{\xi_{k,\vep}^{(2)}}{Z}{x}\right)\\ \nonumber &\quad\quad+\log\E_{Y\sim \opQ[\opV[u_{k\vep}]](\cdot|x)} \exp\left(\mfR{\xi_{k,\vep}^{(1)}}{Y}{x^{u_{k\vep}}}\right).
\end{align}
To simplify the right hand side above further, fix any $y\in\R^d$ and use \eqref{eq:step13} to get that:
\begin{align*}
    \mfR{\xi_{k,\vep}^{(1)}}{y}{x^{u_{k\vep}}}&=\xi_{k,\vep}^{(2)}(x)-\xi_{k,\vep}^{(2)}(y^{w_{k\vep}})-\log\E_{Z\sim \opP[u_{k\vep}](\cdot|y)} \exp\left(\mfR{\xi_{k,\vep}^{(2)}}{Z}{y^{w_{k\vep}}}\right)\\ &\quad\quad+\log\E_{Z\sim \opP[u_{k\vep}](\cdot|x^{u_{k\vep}})} \exp\left(\mfR{\xi_{k,\vep}^{(2)}}{Z}{x}\right).
\end{align*}
Therefore,
\begin{align*}
&\;\;\;\;\log{\E_{Y\sim \opQ[\opV[u_{k\vep}]](\cdot|x)} \exp\left(\mfR{\xi_{k,\vep}^{(1)}}{Y}{x^{u_{k\vep}}}\right)}\\ &=\log{\E_{Y\sim \opQ[\opV[u_{k\vep}]](\cdot|x)}\left[\frac{1}{\E_{Z\sim \opP[u_{k\vep}](\cdot|Y)} \exp\left(\mfR{\xi_{k,\vep}^{(2)}}{Z}{x}\right)}\right]}\\ &\quad\quad+\log{\E_{Z\sim \opP[u_{k\vep}](\cdot|x^{u_{k\vep}})} \exp\left(\mfR{\xi_{k,\vep}^{(2)}}{Z}{x}\right)}.
\end{align*}
By combining the above observation with \eqref{eq:step14}, the conclusion follows.

\end{proof}

\begin{lmm}\label{lem:grabound}
    Under the assumptions of ... we have:
    
\end{lmm}


Also,
    \begin{align*}
        \mathcal{R}_{k\vep}(y)&:=w_{k\vep}(y)+\frac{\vep d}{2}\log{(2\pi\vep)}-\vep f(y^{w_{k\vep}})+\frac{\vep}{2}\ldet(\nabla^2 w_{k\vep}(y))\\ &\qquad -\vep^2 M[u_{k\vep},f](y)-\vep^2 \sum_{j=0}^{k-1} \mathcal{M}[u_{j\vep}](y^{w_{k\vep}}),
    \end{align*}
    
    \begin{align*}
        \mathcal{R}_{k\vep}(y;x)&:=w_{k\vep}(y)+\frac{\vep d}{2}\log{(2\pi\vep)}-\vep f(y^{w_{k\vep}})+\frac{\vep}{2}\ldet(\nabla^2 w_{k\vep}(y))\\ &\qquad -\vep^2 M[u_{k\vep},f](x^{u_{k\vep}})-\vep^2 \sum_{j=0}^{k-1} \mathcal{M}[u_{j\vep}](x).
    \end{align*}
    We define the remainder terms:
    \begin{align*}
        a_{k\vep}:=\sup_{y}\frac{1}{\vep}\bigg|\opV[u_k^{\vep}](y)-\mathcal{R}_{k\vep}(y)\bigg|,
    \end{align*}
    %and 
    %\begin{align*}
    %    \tilde{a}_{k\vep}:=\sup_{(y,x):\ \lVert y-x^{u_{k\vep}}\rVert \le r_{\vep}}\frac{1}{\vep}\bigg|\opV[u_k^{\vep}](y)-\mathcal{R}_{k\vep}(y;x)\bigg|.
    %\end{align*}
    The corresponding terms for approximating the $\opU$ operator are as follows:
    \begin{align*}
        \tilde{\mathcal{R}}_{k\vep}(x)&:=u_{k\vep}(x)+\vep^2 \sum_{j=0}^{k-1} \mathcal{M}[u_{j\vep}](x),
    \end{align*}
    
    %\begin{align*}
    %    \tilde{\mathcal{R}}_{k\vep}(x;y)&:=u_{k\vep}(x)+\vep^2 \sum_{j=0}^{k-1} \mathcal{M}[u_{j\vep}](y^{w_{k\vep}}).
    %\end{align*}
    We define the remainder terms:
    \begin{align*}
        b_{k\vep}:=\sup_{x}\frac{1}{\vep}\bigg|u_k^{\vep}(x)-\tilde{\mathcal{R}}_{k\vep}(x)\bigg|,
    \end{align*}
    %and 
    %\begin{align*}
    %    \tilde{b}_{k\vep}:=\sup_{(x,y):\ \lVert x-y^{w_{k\vep}}\rVert \le r_{\vep}}\frac{1}{\vep}\bigg|u_k^{\vep}(x)-\tilde{\mathcal{R}}_{k\vep}(x;y)\bigg|.
    %\end{align*}

 Note that for $k\ge 0$, 

\begin{align*}
    \Lambda & :=\exp\left(\frac{1}{\vep}\opV[u_k^{\vep}](y)-\frac{1}{\vep}\mathcal{R}_{k\vep}(y)\right)\\ &=\int \exp\left(\frac{1}{\vep}\langle x,y\rangle-\frac{1}{\vep}u_k^{\vep}(x)-\frac{1}{\vep}\mathcal{R}_{k\vep}(y)-f(x)\right)\,dx \\& = \int_{B_{r_{\vep}}(y^{w_{k\vep}})} \exp\left(\frac{1}{\vep}\langle x,y\rangle-\frac{1}{\vep}\tilde{\mathcal{R}}_{k\vep}(x;y)-\frac{1}{\vep}\mathcal{R}_{k\vep}(y)-f(x)-\frac{1}{\vep}u_k^{\vep}(x)+\frac{1}{\vep}\tilde{\mathcal{R}}_{k\vep}(x;y)\right)\,dx\\ &+\int_{B^c_{r_{\vep}}(y^{w_{k\vep}})} \exp\left(\frac{1}{\vep}\langle x,y\rangle-\frac{1}{\vep}\tilde{\mathcal{R}}_{k\vep}(x)-\frac{1}{\vep}\mathcal{R}_{k\vep}(y)-f(x)-\frac{1}{\vep}u_k^{\vep}(x)+\frac{1}{\vep}\tilde{\mathcal{R}}_{k\vep}(x)\right)\\ &=: \Lambda_1+\Lambda_2.
\end{align*}

We begin with $\Lambda_1$. Observe that 
\begin{align*}
&\;\;\;\;\frac{1}{\vep}\langle x,y\rangle-\frac{1}{\vep}\tilde{\mathcal{R}}_{k\vep}(x)-\frac{1}{\vep}\mathcal{R}_{k\vep}(y)-f(x)-\frac{1}{\vep}u_k^{\vep}(x)+\frac{1}{\vep}\tilde{\mathcal{R}}_{k\vep}(x;y)\\ &=\frac{1}{\vep}\mathcal{D}[u_{k\vep}](x|y)-f(x)+f(y^{w_{k\vep}})-\frac{1}{2}\ldet(\nabla^2 w_{k\vep}(y))+\vep M[u_{k\vep},f](y)-\frac{d}{2}\log{(2\pi\vep)}\\ &\qquad -\frac{1}{\vep}u_k^{\vep}(x)+\frac{1}{\vep}\tilde{\mathcal{R}}_{k\vep}(x;y).
\end{align*}
Therefore, with 
$$\Lambda_1^{(1)}:=\exp\left(f(y^{w_{k\vep}})+\vep M[u_{k\vep},f](y)\right)\frac{\sqrt{\mathrm{det}(\nabla^2 u_{k\vep}(y^{w_{k\vep}}))}}{(2\pi\vep)^{d/2}} \int\limits_{B_{r_{\vep}}(y^{w_{k\vep}})} \exp\left(\frac{1}{\vep}\mathcal{D}[u_{k\vep}](x|y)-f(x)\right)\,dx,$$
we have:
$$\exp(-\tilde{b}_{k\vep})\le \frac{\Lambda_1}{\Lambda_1^{(1)}}\le \exp(\tilde{b}_{k\vep}).$$
By \cref{lem:prelimestim}, we have:
\begin{align*}
    \big|\Lambda_1^{(1)}-1\big|\le \eta_t \left(\vep \xi_t(r_{\vep})+\vep^{3/2}(\log{(1/\vep)})^{9/2}\right).
\end{align*}
We now move to $\Lambda_2$. By invoking \cref{lem:prelimestim}, we get:
$$\Lambda_2\le \exp(b_{k\vep})\eta_t \vep^{10}.$$
Combining the above observations, we get:
\begin{align*}
    \exp\left(\frac{1}{\vep}\opV[u_k^{\vep}](y)-\frac{1}{\vep}\mathcal{R}_{k\vep}(y)\right)\ge \exp(-\tilde{b}_{k\vep})\left(1-\eta_t \left(\vep \xi_t(r_{\vep})+\vep^{3/2}(\log{(1/\vep)})^{9/2}\right)\right).
\end{align*}
Similarly, we have:
\begin{align*}
    &\;\;\;\;\exp\left(\frac{1}{\vep}\opV[u_k^{\vep}](y)-\frac{1}{\vep}\mathcal{R}_{k\vep}(y)\right)\\ &\le \exp(\tilde{b}_{k\vep})\left(1+\eta_t \left(\vep \xi_t(r_{\vep})+\vep^{3/2}(\log{(1/\vep)})^{9/2}\right)\right)+\eta_t\vep^{10}\exp(b_{k\vep}).
\end{align*}
By combining the two observations above, for $\vep>0$ small enough (depending only on $\eta_t$, we get: 
$$a_{k\vep}\le \tilde{b}_{k\vep}+\eta_t \left(\vep \xi_t(r_{\vep})+\vep^{3/2}(\log{(1/\vep)})^{9/2}\right)+\eta_t \vep^{10}\exp(b_{k\vep}).$$

We now fix $y,z$ such that $\lVert y-z^{u_{k\vep}}\rVert\le r_{\vep}$ and note that:


For our next lemma, we need to discuss further notation. 
Fix $0\le s\le t$. Recall \eqref{eq:gaussdef} and set
    \begin{align*}
    M[u_{s},f](y)&\equiv M[u_s,f,0](y)\nonumber \\ & =\frac{1}{2}\E\big(T[u_s:3](y^{w_s}+Z_{w_s,y}|y^{w_s})\big)^2 -\E T[u_s:4](y^{w_s}+Z_{w_s,y}|y^{w_s}) \\ & - \E T[f:2](y^{w_s}+Z_{w_s,y}|y^{w_s}) + \frac{1}{2}\E(T[f:1](y^{w_s}+Z_{w_s,y}|y^{w_s}))^2\nonumber \\ & + \E T[u_s:3](y^{w_s}+Z_{w_s,y}|y^{w_s}) T[f:1](y^{w_s}+Z_{w_s,y}|y^{w_s}),
    \end{align*}
    where we used the same notation from \cref{lem:prelimestim}. Also given any $\tilde{G}_{\vep}$ satisfying the conditions of \cref{lem:prelimestim}, define
    \begin{align}\label{eq:rem1}
        \tilde{M}[u_s,f,\vep \tilde{G}_{\vep}](y):=M[u_s,f,\vep \tilde{G}_{\vep}](y)-M[u_s,f](y).
    \end{align}
    Next up, we set 
    \begin{align}\label{eq:subfun}
    G_s(y):=-f(y^{w_s})+g(y)+\frac{1}{2}\ldet(\nabla^2 w_s(y)).
    \end{align}
    Note that as $g(\cdot)$ has bounded and uniformly continuous derivatives of the second order, and $u_s$, $w_s$ have bounded and uniformly continuous derivatives of the sixth order as per the assumptions of \cref{thm:inftheo}, $G_s(\cdot)$ as defined above has bounded and absolutely continuous derivatives of the fourth order. Once again, note that 
    \begin{align*}
     M[w_{s},G_s](x)&\equiv M[w_s,G_s,0](x)\nonumber \\ & =\frac{1}{2}\E\big(T[w_s:3](x^{u_s}+Z_{u_s,x}|x^{u_s})\big)^2 -\E T[w_s:4](x^{u_s}+Z_{u_s,x}|x^{u_s}) \\ & - \E T[G_s:2](x^{u_s}+Z_{u_s,x}|x^{u_s}) + \frac{1}{2}\E(T[G_s:1](x^{u_s}+Z_{u_s,x}|x^{u_s}))^2\nonumber \\ & + \E T[u_s:3](x^{u_s}+Z_{u_s,x}|x^{u_s}) T[G_s:1](x^{u_s}+Z_{u_s,x}|x^{u_s}).
    \end{align*}
    In a similar vein as in \eqref{eq:rem1}, we have:
    \begin{align}\label{eq:rem2}
       \tilde{M}[w_s,G_s,\vep \tilde{G}_{\vep}])(x):=M[w_s,G_s,\vep \tilde{G}_{\vep}](x)-M[w_s,G_s](x). 
    \end{align}
    Define further 
    \begin{align}\label{eq:tosee}\mathcal{M}[u_s](x):=M[w_{s},G_s](x)-M[u_s,f](x^{u_s})-\frac{1}{2}\frac{\partial^2}{\partial s^2}u_s(x),
    \end{align}
    and 
    \begin{align}
    \Theta_{k\vep}(x):=\sum_{r=0}^{k-1} \mathcal{M}[u_{r\vep}](x),\qquad \Theta_{0}\equiv 0,
    \end{align}
    for $k\ge 1$. The final two definitions are as follows: 
    \begin{align*}
        \Theta^*_{1,k\vep}(x):=\sum_{r=0}^{k-1} \tilde{M}[w_{r\vep},G_{r\vep},-\vep \Theta_{r\vep}(\nabla w_{r\vep})](x),
    \end{align*}
    and 
    \begin{align*}
        \Theta^*_{2,k\vep}(x):=-\sum_{r=1}^{k-1} \tilde{M}[u_{r\vep},f,-\vep \Theta_{r\vep}](x^{u_{r\vep}}),
    \end{align*}
    When $k=0$, the above summand above is set to $0$.
    The next lemma is elementary and accordingly, we skip the algebraic details of the proof.

    \begin{lmm}\label{eq:estimmt}
        Fix $t>0$. Under \cref{asn:solcon}, given $k\ge 1$, $\vep>0$, satisfying $k\vep \le t$, there exists a constant $C_t>0$ and a function $\omega_t(\cdot)$ with $\lim_{\delta\to 0} \omega_t(\delta)=0$, such that 
        \begin{align*}
            \sup_{k\le \lceil \frac{t}{\vep}\rceil} \vep\left(\lVert \Theta_{k\vep}\rVert_{\infty}\ +\ \lVert \nabla \Theta_{k\vep}\rVert_{\infty}\ + \ \lVert \nabla^2 \Theta_{k\vep}\rVert_{\infty}\right)\le C_t.
        \end{align*}
    \end{lmm}

    We are now in the position to state and prove our main result.
    
    \begin{lmm}
        Fix $t>0$. Under \cref{asn:solcon} with $k\ge 1$, $\vep>0$ such that $k\vep \le t$, we have:
        $$\lVert u_k^{\vep}-u_{k\vep}-\vep^2 \Theta_{(k-1)\vep}-\vep^2 \Theta^*_{1,(k-1)\vep}-\vep^2 \Theta^*_{2,(k-1)\vep}\rVert_{\infty}=o(\vep)$$
    \end{lmm}

    \begin{proof}
        The proof proceeds by induction.
        \vspace{0.1in}

        \noindent{Base case $k=1$}: 
    \end{proof}
    
\vspace{0.1in}

    \emph{$k=1$ case}: 
     and $C_{D,1}$ depends on the first $4$ derivatives of $u_0$, the first $2$ derivatives of $f$ and their  modulus of continuities. Also define

    $$\omega[u_0,f](\delta):=\sup_{\lVert z_1-z_2\rVert \le \delta} \big|M[u_0,f](z_1)-M[u_0,f](z_2)\big|.$$
    Using this observation, we will invoke \cref{lem:prelimestim} for each iteration of the Sinkhorn algorithm. By applying \cref{lem:prelimestim} with $u\equiv u_0$, $G\equiv f$, and $\tilde{G}\equiv 0$, we get:
    \begin{align*}
        \lVert\mathcal{R}^{\vep}_0-\opV[u_0]\rVert_{\infty}\le C_{0} \vep^2 \left(\omega[u_0,f](\sqrt{\vep \log{(1/\vep)}})+\sqrt{\vep}(\log{(1/\vep)})^{9/2}\right)=:E_{\vep,0},
    \end{align*}
    
    Note that 
    \begin{align*}
        &\;\;\;\;\exp\left(\frac{1}{\vep}u_1^{\vep}(x)-\frac{1}{\vep}\tilde{u}_{\vep}(x)-\vep M[u_0,f](x^{u_0})\right)\\ &=\frac{\sqrt{\mathrm{det}(\nabla^2 w_0(x^{u_0}))}}{(2\pi\vep)^{d/2}}\int_{B_{r_{\vep}}^c(x^{u_0})}\exp\bigg(\frac{1}{\vep}\mcD[w_0](y|x)+G_0(y)-G_0(x^{u_0})\nonumber \\ &\qquad \qquad +\vep(M[u_0,f](y)-M[u_0,f](x^{u_0}))-\frac{1}{\vep}(\opV[u_0](y)-\mathcal{R}^{\vep}_0(y))\bigg)\,dy\\ &+\frac{\sqrt{\mathrm{det}(\nabla^2 w_0(x^{u_0}))}}{(2\pi\vep)^{d/2}}\int_{B_{r_{\vep}}(x^{u_0})}\exp\bigg(\frac{1}{\vep}\mcD[w_0](y|x)+G_0(y)-G_0(x^{u_0})\nonumber \\ &\qquad \qquad -\frac{1}{\vep}(\opV[u_0](y)-\mathcal{R}^{\vep}_0(y;x))\bigg)\,dy
    \end{align*}
    By using \eqref{eq:gradbound4}, the first term above is bounded by 
    $$C_1 \exp\left(\vep C_1+\vep E_{0,\vep}+2\vep \lVert M[u_0,f]\rVert_{\infty}\right)\vep^{10}.$$
    For the second term, note that 
    \begin{align*}
        &\;\;\;\;\log\frac{\int_{B_{r_{\vep}}(x^{u_0})}\exp\bigg(\frac{1}{\vep}\mcD[w_0](y|x)+G_0(y)-G_0(x^{u_0})-\frac{1}{\vep}(\opV[u_0](y)-\mathcal{R}^{\vep}_0(y;x))\bigg)\,dy}{\int_{B_{r_{\vep}}(x^{u_0})}\exp\bigg(\frac{1}{\vep}\mcD[w_0](y|x)+G_0(y)-G_0(x^{u_0})\bigg)\,dy}\\ &\in \left(-\sup_{y\in B_{r_{\vep}}(x^{u_0})} \frac{1}{\vep}\big|\opV[u_0](y)-\mathcal{R}^{\vep}_0(y;x)\big|,\ \sup_{y\in B_{r_{\vep}}(x^{u_0})} \frac{1}{\vep}\big|\opV[u_0](y)-\mathcal{R}^{\vep}_0(y;x)\big|\right).
    \end{align*}
    By \cref{lem:prelimestim}, we then have:
    \begin{align*}
        &\;\;\;\;\bigg|\int_{B_{r_{\vep}}(x^{u_0})}\exp\bigg(\frac{1}{\vep}\mcD[w_0](y|x)+G_0(y)-G_0(x^{u_0})\bigg)\,dy-1+\vep M[w_0,G_0](x)\bigg|\\ &\le C_2\left(\vep^{3/2}(\log{(1/\vep)})^{9/2}+\vep \omega_2(r_{\vep})\right).
    \end{align*}
    Consequently, we get:
    $$\big|u_1^{\vep}(x)-\tilde{u}_{\vep}(x)-\vep^2 M[u_0,f](x^{u_0})+\vep^2 M[w_0,G_0](x)\big|\le .$$




\hrule 

\begin{proof}
    We begin the proof with some definitions. 
\begin{align*}
    \xi_t&:=\sup_{k\le \lceil t/\vep\rceil} \vep \sum_{j=0}^k \lVert \mathcal{M}[u_{j\vep}]\rVert_{\infty} + \sup_{s\le t} \, (\lVert \nabla G_s\rVert_{\infty}+\lVert \nabla^2 G_s\rVert_{\infty})\\ &\quad +\lVert \nabla f\rVert_{\infty}+\lVert \nabla^2 f\rVert + \sup_{s\le t} \, (\lVert \nabla^2 u_s\rVert_{\infty}+\lVert \nabla^3 u_s\rVert_{\infty}+\lVert \nabla^4 u_s\rVert_{\infty})\infty,
\end{align*}
and the following function:
\begin{align*}
    \xi_t(\delta)&:=\sup_{s\le t} \sup_{\lVert z_1-z_2\rVert\le \delta} \bigg(|\mathcal{M}[u_s](z_1)-\mathcal{M}[u_s](z_2)|+\lVert \nabla^2 G_s(z_1)-\nabla^2 G_s(z_2)\rVert_{\infty}\\ &\qquad +\lVert \nabla^4 u_s(z_1)-\nabla^4 u_s(z_2)\rVert_{\infty} +\lVert \nabla^2 f(z_1)-\nabla^2 f(z_2)\rVert_{\infty}\bigg)<\infty.
\end{align*}

Observe that 
\begin{align*}
    \exp(\mathcal{I}_{k\vep}(y))&=\exp\left(\frac{1}{\vep}\opV[u_k^{\vep}](y)-\frac{1}{\vep}w_{k\vep}(y)-\frac{d}{2}\log{(2\pi\vep)}+f(y^{w_{k\vep}})-\frac{1}{2}\ldet(\nabla^2 w_{k\vep}(y))\right)\\ &=\frac{\sqrt{\mathrm{det}(\nabla^2 u_{k\vep}(y^{w_{k\vep}})}}{(2\pi\vep)^{d/2}}\int \exp\left(\frac{1}{\vep}\langle x,y\rangle - \frac{1}{\vep} w_{k\vep}(y)-\frac{1}{\vep}u_k^{\vep} + f(y^{w_{k\vep}}) - f(x)\right)\,dx
\end{align*}
Therefore, 
\begin{align*}
    &\;\;\;\;\frac{\exp(\mathcal{I}_{k\vep}(y))}{\frac{\sqrt{\mathrm{det}(\nabla^2 u_{k\vep}(y^{w_{k\vep}})}}{(2\pi\vep)^{d/2}}\int \exp\left(\frac{1}{\vep}\langle x,y\rangle - \frac{1}{\vep} w_{k\vep}(y)-\frac{1}{\vep}u_{k\vep} + f(y^{w_{k\vep}}) - f(x)\right)\,dx}\\ &\in \big(\exp(-\lVert \mathcal{J}_{k\vep}\rVert_{\infty}),\ \exp(\lVert \mathcal{J}_{k\vep}\rVert_{\infty})\big).
\end{align*}
By \cref{lem:prelimestim}, we have that
$$\bigg|\log\frac{\sqrt{\mathrm{det}(\nabla^2 u_{k\vep}(y^{w_{k\vep}})}}{(2\pi\vep)^{d/2}}\int \exp\left(\frac{1}{\vep}\langle x,y\rangle - \frac{1}{\vep} w_{k\vep}(y)-\frac{1}{\vep}u_{k\vep} + f(y^{w_{k\vep}}) - f(x)\right)\,dx\bigg|\le \vep\eta_t,$$
where $\eta_t$ is a constant depending on $\xi_t$ and $\xi_t(1)$. 

By combining the above observations and taking logarithms on both sides, we have:
$$\lVert\mathcal{I}_{k\vep}\rVert_{\infty} \le \vep \eta_t + \lVert\mathcal{J}_{k\vep}\rVert.$$
By repeating the same argument with $u_{k+1}^{\vep}$, we get:
$$\lVert \mathcal{J}_{(k+1)\vep}\rVert_{\infty}\le \eta_t \vep + \lVert \mathcal{I}_{k\vep}\rVert_{\infty}.$$

The above recursive relations imply that 
$$\lVert\mathcal{I}_{k\vep}\rVert_{\infty}=O(1),\qquad \mbox{and}\qquad \lVert\mathcal{J}_{k\vep}\rVert_{\infty}=O(1).$$
\end{proof}

\begin{prop}\label{prop:pmasec}
    Suppose that \cref{asn:solcon} holds. Observe that for $t\ge 0$, we have:
    \begin{align}\label{eq:matid1}
        \Ddot{u}_{t}(x)&=\frac{\partial\hfill}{\partial t^2}u_t(x)\nonumber \\ &=-g'(x^{u_{k\vep}})f'(x)+(g'(x^{u_{k\vep}}))^2 u_{k\vep}''(x)-2 g'(x^{u_{k\vep}})u_{k\vep}'''(x) w_{k\vep}''(x^{u_{k\vep}}) + f''(x) w_{k\vep}''(x^{u_{k\vep}})\nonumber \\ &\qquad -g''(x^{u_{k\vep}})u''_{k\vep}(x)+u^{(4)}_{k\vep}(x) (w''_{k\vep}(x^{u_{k\vep}}))^2+w_{k\vep}'''(x^{u_{k\vep}})u_{k\vep}'''(x).
    \end{align}
\end{prop}

\begin{proof}
   Hello.
\end{proof}

We are now in position to prove \cref{lem:matid}. To avoid notational clutter, we will drop the $k\vep$ notation in \cref{lem:matid} without loss of generality. This can be done as \cref{lem:matid} is a claim for every fixed $k\vep$ and there are no asymptotics involved in it. Therefore, in the sequel, we will write $u\equiv u_{k\vep}$, $w\equiv w_{k\vep}$ and following \eqref{eq:initdef1} and \eqref{eq:initdef2}, 
$$G(x)\equiv G_{k\vep}(x)=\frac{3f}{2}(x)-g(x^{u_{k\vep}})+\ldet\left(\frac{\partial x^{u_{k\vep}}}{\partial x\hfill}\right)\equiv \frac{3f}{2}(x)-g(x^u)+\ldet\left(\frac{\partial x^{u}}{\partial x\hfill}\right),$$
and
$$\bar{G}(y)\equiv \bar{G}_{k\vep}(y)=-G_{k\vep}(y^{w_{k\vep}})+g(y)+\frac{1}{2}\ldet\left(\frac{\partial y^{w_{k\vep}}}{\partial y\hfill}\right)\equiv -G(y^w)+g(y)+\frac{1}{2}\ldet\left(\frac{\partial y^{w}}{\partial y\hfill}\right).$$

 We will also need the following expressions for the derivatives of $\bar{G}_{k\vep}$. In particular,

    \begin{equation}\label{eq:matt4}
        \bar{G}_{k\vep}'(y)=-G_{k\vep}'(y^{w_{k\vep}}) w_{k\vep}''(y)+g'(y)+\frac{1}{2}\frac{w_{k\vep}'''(y)}{w_{k\vep}''(y)},
    \end{equation}
    and 
    \begin{equation}\label{eq:matt5}
        \bar{G}_{k\vep}''(y)=-G_{k\vep}'(y^{w_{k\vep}}) w_{k\vep}'''(y)-G_{k\vep}''(y^{w_{k\vep}})(w_{k\vep}''(y))^2+g''(y)+\frac{1}{2}\frac{w_{k\vep}^{(4)}(y)}{w_{k\vep}''(y)}-\frac{1}{2}\frac{(w_{k\vep}'''(y))^2}{(w_{k\vep}''(y))^2}.
    \end{equation}

\begin{proof}[Proof of \cref{lem:matid}]
    
    
    We will use $Z_{w,y}\sim  N(0,\nabla^2 w(y))$ in the sequel.
    
    We now move on to simplifying the difference between the $M[\cdot,\cdot]$ functional. In particular, by \eqref{eq:gaussdef}, we have:
    \begin{align}\label{eq:matid2}
        &\;\;\;\; M[w_{k\vep},\bar{G}_{k\vep}](x)-M[u_{k\vep},G_{k\vep}](x^{u_{k\vep}})=\sum_{i=1}^5 T_i(x),
    \end{align}
    where
    \begin{align*}
        T_1(x):=\frac{1}{2}\E\big(T[w_{k\vep}:3](x^{u_{k\vep}}+Z_{u_{k\vep},x}|x^{u_{k\vep}})\big)^2-\frac{1}{2}\E\big(T[u_{k\vep}:3](x+Z_{w_{k\vep},x^{u_{k\vep}}}|x)\big)^2,
    \end{align*}
    \begin{align*}
        T_2(x):=\E T[u_{k\vep}:4](x+Z_{w_{k\vep},x^{u_{k\vep}}}|x)- \E T[w_{k\vep}:4](x^{u_{k\vep}}+Z_{u_{k\vep},x}|x^{u_{k\vep}}),
    \end{align*}
    \begin{align*}
        T_3(x):= \E T[G_{k\vep}:2](x+Z_{w_{k\vep},x^{u_{k\vep}}}|x)- \E T[\bar{G}_{k\vep}:2](x^{u_{k\vep}}+Z_{u_{k\vep},x}|x^{u_{k\vep}}),
    \end{align*}
    \begin{align*}
        T_4(x):= \frac{1}{2}\E(T[\bar{G}_{k\vep}:1](x^{u_{k\vep}}+Z_{u_{k\vep},x}|x^{u_{k\vep}})^2-\frac{1}{2}\E(T[G_{k\vep}:1](x+Z_{w_{k\vep},x^{u_{k\vep}}}|x))^2,
    \end{align*}
    and
    \begin{align*}
        T_5(x)&:= \E T[w_{k\vep}:3](x^{u_{k\vep}}+Z_{u_{k\vep},x}|x^{u_{k\vep}}) T[\bar{G}_{k\vep}:1](x^{u_{k\vep}}+Z_{u_{k\vep},x}|x^{u_{k\vep}})\\ & \qquad -\E T[u_{k\vep}:3](x+Z_{w_{k\vep},x^{u_{k\vep}}}|x) T[G_{k\vep}:1](x+Z_{w_{k\vep},x^{u_{k\vep}}}|x).
    \end{align*}

    We will simplify each of these terms individually.

    \emph{Simplifying $T_1$.} We will show $T_1\equiv 0$ ({\color{red} This part I can show for general $d$}). Observe that 
    \begin{align*}
    \E\big(T[w_{k\vep}:3](x^{u_{k\vep}}+Z_{u_{k\vep},x}|x^{u_{k\vep}})\big)^2=\frac{15}{36} (w'''_{k\vep}(x^{u_{k\vep}}))^2 (u''_{k\vep}(x))^3=-\frac{15}{36} w_{k\vep}'''(x^{u_{k\vep}})u_{k\vep}'''(x),
    \end{align*}
    where the last conclusion follows from \eqref{eq:matt1}. Similarly, 
    \begin{align*}
        \E\big(T[u_{k\vep}:3](x+Z_{w_{k\vep},x^{u_{k\vep}}}|x)\big)^2=\frac{15}{36}(u_{k\vep}'''(x))^2 (w_{k\vep}''(x^{u_{k\vep}}))^3=-\frac{15}{36}w_{k\vep}'''(x^{u_{k\vep}})u_{k\vep}'''(x).
    \end{align*}
    Therefore $T_1\equiv 0$.

    \vspace{0.1in} 

    \emph{Simplifying $T_2$.} Observe that 
    \begin{align*}
    \E T[u_{k\vep}:4](x+Z_{w_{k\vep},x^{u_{k\vep}}}|x)=\frac{1}{8}u_{k\vep}^{(4)}(x)(w_{k\vep}''(x^{u_{k\vep}}))^2.
    \end{align*}
    By a similar computation for the other term in the definition of $T_2$, we get:
    \begin{align*}
        T_2(x)=\frac{1}{8}\left(u_{k\vep}^{(4)}(x)(w_{k\vep}''(x^{u_{k\vep}}))^2-w_{k\vep}^{(4)}(x^{u_{k\vep}})(u_{k\vep}''(x))^2\right).
    \end{align*}

    \vspace{0.1in}

    \emph{Simplifying $T_3$.} Observe that, by \eqref{eq:matt5}, we get: 
    \begin{align*}
        &\;\;\;\E T[\bar{G}_{k\vep}:2](x^{u_{k\vep}}+Z_{u_{k\vep},x}|x^{u_{k\vep}}) \\ &=\frac{1}{2}\bar{G}_{k\vep}''(x^{u_{k\vep}})u_{k\vep}''(x)\\ &= \frac{1}{2}\bigg(-G_{k\vep}'(x) w_{k\vep}'''(x^{u_{k\vep}})u_{k\vep}''(x)-G_{k\vep}''(x)(w_{k\vep}''(x^{u_{k\vep}}))+g''(x^{u_{k\vep}})u_{k\vep}''(x)\\ &+\frac{1}{2}w_{k\vep}^{(4)}(x^{u_{k\vep}})(u_{k\vep}''(x))^2-\frac{1}{2}(w_{k\vep}'''(x^{u_{k\vep}}))^2(u_{k\vep}''(x))^3\bigg).
    \end{align*}
    By invoking \eqref{eq:matt1}, we get:
    \begin{align*}
        T_3(x)&:= \frac{1}{2}\left(G_{k\vep}''(x)(w_{k\vep}''(x^{u_{k\vep}})-\bar{G}_{k\vep}''(x^{u_{k\vep}})u_{k\vep}''(x)\right)\\ &=\frac{1}{2}\bigg(2G_{k\vep}''(x)(w_{k\vep}''(x^{u_{k\vep}}))+G_{k\vep}'(x) w_{k\vep}'''(x^{u_{k\vep}})u_{k\vep}''(x)-g''(x^{u_{k\vep}})u_{k\vep}''(x)\\ & \quad - \frac{1}{2}w_{k\vep}^{(4)}(x^{u_{k\vep}})(u_{k\vep}''(x))^2-\frac{1}{2} w_{k\vep}'''(x^{u_{k\vep}})u_{k\vep}'''(x)\bigg).
    \end{align*}

    \vspace{0.1in}

    \emph{Simplifying $T_4$.} Observe that, by \eqref{eq:matt2}, we have: 
    \begin{align*}
        &\;\;\;\E (T[\bar{G}_{k\vep}:1](x^{u_{k\vep}}+Z_{u_{k\vep},x}|x^{u_{k\vep}}))^2\\ &=(\bar{G}_{k\vep}'(x^{u_{k\vep}}))^2 u_{k\vep}''(x) \\ &=\bigg((G_{k\vep}'(x))^2  w_{k\vep}''(x^{u_{k\vep}})+(g'(x^{u_{k\vep}}))^2 u_{k\vep}''(x)+\frac{1}{4} (w_{k\vep}'''(x^{u_{k\vep}}))^2 (u_{k\vep}''(x))^3 \\ &\qquad -2 G_{k\vep}'(x) g'(x^{u_{k\vep}}) - G_{k\vep}'(x) w_{k\vep}'''(x^{u_{k\vep}})u_{k\vep}''(x)+ g'(x^{u_{k\vep}})w_{k\vep}'''(x^{u_{k\vep}}) (u_{k\vep}''(x))^2\bigg).
    \end{align*}
    Therefore,
    \begin{align*}
        T_4(x)&=\frac{1}{2}(g'(x^{u_{k\vep}}))^2 u_{k\vep}''(x)+\frac{1}{8} (w_{k\vep}'''(x^{u_{k\vep}}))^2 (u_{k\vep}''(x))^3 - G_{k\vep}'(x) g'(x^{u_{k\vep}}) \\ &\qquad - \frac{1}{2} G_{k\vep}'(x) w_{k\vep}'''(x^{u_{k\vep}})u_{k\vep}''(x)+ \frac{1}{2} g'(x^{u_{k\vep}})w_{k\vep}'''(x^{u_{k\vep}}) (u_{k\vep}''(x))^2.
    \end{align*}

    \vspace{0.1in}

    \emph{Simplifying $T_5$.} Observe that by \eqref{eq:matt1}, we get:
    \begin{align*}
        &\;\;\;\;\E T[w_{k\vep}:3](x^{u_{k\vep}}+Z_{u_{k\vep},x}|x^{u_{k\vep}}) T[\bar{G}_{k\vep}:1](x^{u_{k\vep}}+Z_{u_{k\vep},x}|x^{u_{k\vep}})\\ &=-\frac{1}{2} \bar{G}_{k\vep}'(x^{u_{k\vep}})\frac{u_{k\vep}'''(x)}{u_k''(x)}\\ &=\frac{1}{2}\bigg(G_{k\vep}'(x)(w_{k\vep}''(x^{u_{k\vep}}))^2 u_{k\vep}'''(x)-g'(x^{u_{k\vep}})\frac{u_{k\vep}'''(x)}{u_{k\vep}''(x)}-\frac{1}{2}w_{k\vep}'''(x^{u_{k\vep}})u_{k\vep}'''(x)\bigg).
    \end{align*}

    Also observe that 
    \begin{align*}
        &\;\;\;\E T[u_{k\vep}:3](x+Z_{w_{k\vep},x^{u_{k\vep}}}|x) T[G_{k\vep}:1](x+Z_{w_{k\vep},x^{u_{k\vep}}}|x)\\ &=\frac{1}{2}G_{k\vep}'(x)u_{k\vep}'''(x) (w_{k\vep}''(x^{u_{k\vep}}))^2.
    \end{align*}

    This implies 
    $$T_5(x)=-\frac{1}{2}\bigg(g'(x^{u_{k\vep}})\frac{u_{k\vep}'''(x)}{u_{k\vep}''(x)}+\frac{1}{2}w_{k\vep}'''(x^{u_{k\vep}})u_{k\vep}'''(x)\bigg).$$

    By combining all the above observations with \eqref{eq:matid2}, we get:

    \begin{align}\label{eq:matid11}
        &\;\;\; 2M[w_{k\vep},\bar{G}_{k\vep}](x)-2M[u_{k\vep},G_{k\vep}](x^{u_{k\vep}}) \nonumber \\ &=\frac{1}{4}u_{k\vep}^{(4)}(x)(w_{k\vep}''(x^{u_{k\vep}}))^2-\frac{1}{4}w_{k\vep}^{(4)}(x^{u_{k\vep}})(u_{k\vep}''(x))^2 + 2G_{k\vep}''(x)(w_{k\vep}''(x^{u_{k\vep}}))\nonumber \\ &\quad +G_{k\vep}'(x) w_{k\vep}'''(x^{u_{k\vep}})u_{k\vep}''(x)-g''(x^{u_{k\vep}})u_{k\vep}''(x) - \frac{1}{2}w_{k\vep}^{(4)}(x^{u_{k\vep}})(u_{k\vep}''(x))^2\nonumber\\ & \quad -\frac{1}{2} w_{k\vep}'''(x^{u_{k\vep}})u_{k\vep}'''(x)+(g'(x^{u_{k\vep}}))^2 u_{k\vep}''(x)+\frac{1}{4} (w_{k\vep}'''(x^{u_{k\vep}}))^2 (u_{k\vep}''(x))^3\nonumber \\ &\quad - 2G_{k\vep}'(x) g'(x^{u_{k\vep}}) -  G_{k\vep}'(x) w_{k\vep}'''(x^{u_{k\vep}})u_{k\vep}''(x)+ g'(x^{u_{k\vep}})w_{k\vep}'''(x^{u_{k\vep}}) (u_{k\vep}''(x))^2\nonumber \\ &\quad -g'(x^{u_{k\vep}})\frac{u_{k\vep}'''(x)}{u_{k\vep}''(x)}-\frac{1}{2}w_{k\vep}'''(x^{u_{k\vep}})u_{k\vep}'''(x)\nonumber\\ &=\frac{1}{4}u_{k\vep}^{(4)}(x)(w_{k\vep}''(x^{u_{k\vep}}))^2-\frac{3}{4}w_{k\vep}^{(4)}(x^{u_{k\vep}})(u_{k\vep}''(x))^2-\frac{5}{4} w_{k\vep}'''(x^{u_{k\vep}})u_{k\vep}'''(x)\nonumber\\ &\quad +2 G_{k\vep}''(x)w_{k\vep}''(x^{u_{k\vep}})-g''(x^{u_{k\vep}})u_{k\vep}''(x)+(g'(x^{u_{k\vep}}))^2 u_{k\vep}''(x)-2 G_{k\vep}'(x)g'(x^{u_{k\vep}})\nonumber\\ &\quad -2 g'(x^{u_{k\vep}})\frac{u_{k\vep}'''(x)}{u_{k\vep}''(x)}.
    \end{align}

    Recall that $G_{k\vep}(x)=f(x)+v_{k\vep}(x)$. Also observe that 
    \begin{align*}
        v_{k\vep}'(x)=\frac{1}{2}f'(x)-g'(x^{u_{k\vep}})u_{k\vep}''(x)+\frac{u_{k\vep}'''(x)}{u_{k\vep}''(x)},
    \end{align*}
    and by using \eqref{eq:matt2}, we also have:
    \begin{align*}
        v_{k\vep}''(x)=\frac{1}{2}f''(x)-g''(x^{u_{k\vep}})(u_{k\vep}''(x))^2-g'(x^{u_{k\vep}})u_{k\vep}'''(x)+\frac{u_{k\vep}^{(4)}(x)}{u_{k\vep}''(x)}+w_{k\vep}'''(x^{u_{k\vep}})u_{k\vep}'''(x).
    \end{align*}

    Plugging the two observations above into \eqref{eq:matid11}, we get:
    \begin{align}\label{eq:matid12}
     &\;\;\; 2M[w_{k\vep},\bar{G}_{k\vep}](x)-2M[u_{k\vep},G_{k\vep}](x^{u_{k\vep}})\\ &=\frac{9}{4} u_{k\vep}^{(4)}(x)(w_{k\vep}(x^{u_{k\vep}}))^2-\frac{3}{4}w_{k\vep}^{(4)}(x^{u_{k\vep}})(u_{k\vep}''(x))^2+\frac{3}{4}w_{k\vep}'''(x^{u_{k\vep}})u_{k\vep}'''(x)+3f''(x)w_{k\vep}''(x^{u_{k\vep}})\\ &\quad -3g''(x^{u_{k\vep}})u_{k\vep}''(x)-6 g'(x^{u_{k\vep}})\frac{u_{k\vep}'''(x)}{u_{k\vep}''(x)}+3(g'(x^{u_{k\vep}}))^2 u_{k\vep}''(x)-3f'(x)g'(x^{u_{k\vep}}).
    \end{align}
    Scaling and then subtracting \eqref{eq:matid1} from \eqref{eq:matid12}, we get:
    \begin{align*}
     &\;\;\;2M[w_{k\vep},\bar{G}_{k\vep}](x)-2M[u_{k\vep},G_{k\vep}](x^{u_{k\vep}})-3\Ddot{u}_{k\vep}(x)\\ &=-\frac{3}{4}\left(u_{k\vep}^{(4)}(x)(w_{k\vep}(x^{u_{k\vep}}))^2+w_{k\vep}^{(4)}(x^{u_{k\vep}})(u_{k\vep}''(x))^2+3w_{k\vep}'''(x^{u_{k\vep}})u_{k\vep}'''(x)\right)=0,
    \end{align*}
    where the last equality follows from \eqref{eq:matt3}.
\end{proof}

\begin{proof}[Proof of \cref{thm:convergence}] 
Note that 

\begin{equation}\label{eq:step11}
\KL{\exp(-h_{k\vep})}{p_X \gvp_k}=\int \left(f(x)-h_{k\vep}(x)-\frac{1}{\vep}(u_{k+1}^{\vep}(x)-u_k^{\vep}(x))\right)\exp(-h_{k\vep}(x))\,dx.
\end{equation}

\noindent We focus on the integrand in \eqref{eq:step11}. Let us define some notation first. 
Fix $x\in\R^d$ and 
In order to prove \eqref{eq:step1show}, it suffices to prove the following:

\begin{equation}\label{eq:step16}
    \sup_{k\vep\le T}\bigg|\int \left[\frac{1}{\vep}\left(\opS[u_{k\vep}](x)-u_{k\vep}(x)\right)-f(x)+h_{k\vep}(x)\right]\exp(-h_{k\vep}(x))\,dx\bigg|=o_{\vep}(1),
\end{equation}
and 
\begin{equation}\label{eq:step17}
    \sup_{k\vep\le T}\E_{X\sim \exp(-h_{k\vep})}\bigg|\log{\E_{Y\sim \opQ[\opV[u_{k\vep}]](\cdot|X)}\left[\frac{1}{\E_{Z\sim \opP[u_{k\vep}](\cdot|Y)} \exp\left(\mfR{\xi_{k,\vep}^{(2)}}{Z}{x}\right)}\right]}\bigg|=o_{\vep}(1).
\end{equation}

\emph{Proof of \eqref{eq:step16}.} 
Fix $T>0$ and consider $k,\vep$ such that $k\vep\le T$. 
Recall that $\rho_t=\exp(-h_t)$. We first claim that 
\begin{align}\label{eq:toshow1}
\limsup_{M\to\infty}\sup_{\ k\vep\le T} \E_{X\sim \rho_{k\vep}}\left[\bigg|\frac{1}{\vep}(\opS[u_{k\vep}](X)-u_{k\vep}(X)\bigg|+|f(X)-h_{k\vep}(X)|\right]\bm{1}(\lVert X\rVert\ge M)=0.
\end{align}
Establishing \eqref{eq:toshow1} would reduce proving \eqref{eq:step16} to showing that
\begin{align}\label{eq:toshow2}
\limsup_{\vep\to 0,\ k\vep\le T}\sup_{\lVert x\rVert\le M} \bigg|\frac{1}{\vep}(\opS[u_{k\vep}](x)-u_{k\vep}(x))-f(x)+h_{k\vep}(x)\bigg|=0,
\end{align}
for all fixed but large (free of $\vep$) $M>0$. 

\emph{Proof of \eqref{eq:toshow1}.}
Consider $Y\sim e^{-g}$. Note that by \cref{asn:solcon}, we have:
\begin{align}\label{eq:tailest1}
&\;\;\;\;\limsup_{M\to\infty}\sup_{t\in [0,T]}\E_{X\sim \rho_t} |f(X)-h_t(X)|\bm{1}(\lVert X\rVert\ge M)\nonumber \\&\le \sqrt{\sup_{t\in [0,T]} \E\left(\frac{\partial u_t}{\partial t\hfill}(Y^{u_t^*})\right)^2}\limsup_{M\to\infty}\sup_{t\in [0,T]}\sqrt{P(\lVert Y^{u_t^*}\rVert\ge M)}=0.
\end{align}
In the final conclusion above, we use the fact that 
\begin{equation}\label{eq:callback}
\lVert Y^{u_t^*}\rVert \le c_T^{-1}Y+\sup_{t\in [0,T]} \lVert \nabla u_t^*(0)\rVert.
\end{equation}
Next, set
$$c_T:=\inf_{t\in [0,T],\ x\in \R^d} \lmn\left(\frac{\partial x^{u_t}}{\partial x\hfill}\right)>0,\quad C_T:=\sup_{t\in [0,T],\ x\in \R^d}\lmx\left(\frac{\partial x^{u_t}}{\partial x\hfill}\right)<\infty.$$
The following inequalities, which follow from straightforward Taylor expansions, will be used multiple times in this proof:
\begin{equation}\label{eq:maindisp1}
\frac{1}{2C_T}\lVert y-x^{u_{k\vep}}\rVert^2\le u_{k\vep}^*(y)+u_{k\vep}(x)-\langle x,y\rangle \le \frac{1}{2c_T}\lVert y-x^{u_{k\vep}}\rVert^2,
\end{equation}
and 
\begin{equation}\label{eq:maindisp2}
\frac{c_T}{2}\lVert x-y^{u_{k\vep}^*}\rVert^2\le u_{k\vep}^*(y)+u_{k\vep}(x)-\langle x,y\rangle \le \frac{C_T}{2}\lVert x-y^{u_{k\vep}^*}\rVert^2.
\end{equation}

Let $Z_{\vep,C}\sim  N(0,\sqrt{\vep C} I_d)$ for $C>0$. 
For fixed $y$, by using \eqref{eq:maindisp2}, we have:
Consequently,
\begin{align*}
\exp\left(\frac{1}{\vep}\opV[u_{k\vep}](y)-\frac{1}{\vep}u_{k\vep}^*(y)\right)&\ge \int \exp\left(-\frac{C_T}{2\vep}\lVert z-y^{u_{k\vep}^*}\rVert^2-f(z)\right)\,dz\\ &=\left(\frac{2\pi\vep}{C_T}\right)^{\frac{d}{2}}\E\exp(-f(y^{u_{k\vep}^*}+Z_{\vep,(C_T)^{-1}})).
\end{align*}
Consequently,
\begin{align}\label{eq:upbd1}
    &\;\;\;\;\exp\left(\frac{1}{\vep}(\opS[u_{k\vep}](x)-u_{k\vep}(x))\right)\nonumber \\ &\le \left(\frac{C_T}{2\pi\vep}\right)^{\frac{d}{2}}\int \left[E\exp(-f(y^{u_{k\vep}^*}+Z_{\vep,(C_T)^{-1}}))\right]^{-1}\exp\left(\frac{1}{\vep}\langle x,y\rangle - \frac{1}{\vep}u_{k\vep}^*(y)-\frac{1}{\vep}u_{k\vep}(x)-g(y)\right)\,dy\nonumber \\ &\le (C_T)^d \E_{Z_{\vep,C_T}}\left[\left(\E_{|Z_{\vep,C_T}}\exp\left(-f((x^{u_{k\vep}}+Z_{\vep,C_T})^{u_{k\vep}^*}+Z_{\vep,(C_T)^{-1}})\right)\right)^{-1}\exp(-g(x^{u_{k\vep}}+Z_{\vep,C_T}))\right]\nonumber \\ &=(C_T)^d \Theta_{C_T,\vep}(x^{u_{k\vep}}).
\end{align}
A similar computation as above also gives a lower bound for $\exp\left(\frac{1}{\vep}(\opS[u_{k\vep}](x)-u_{k\vep}(x))\right)$ with $C_T$ replaced by $c_T$, i.e.,
\begin{align}\label{eq:lbd1}
\exp\left(\frac{1}{\vep}(\opS[u_{k\vep}](x)-u_{k\vep}(x))\right)\ge (c_T)^d \Theta_{c_T,\vep}(x^{u_{k\vep}}).
\end{align}
We omit the details for brevity. Combining these observations with the Cauchy Schwartz inequality, we get:
\begin{align}\label{eq:zerocon}
    &\;\;\;\;\limsup_{M\to\infty}\sup_{\ k\vep\le T} \E_{X\sim \rho_{k\vep}}\left[\bigg|\frac{1}{\vep}(\opS[u_{k\vep}](X)-u_{k\vep}(X))\bigg|\bm{1}(\lVert X\rVert\ge M)\right]\nonumber \\ &\le\sqrt{4d((\log{C_T})^2+(\log{c_T})^2)+4(\tilde{\theta}_{C_T}+\tilde{\theta}_{c_T})}\limsup_{M\to\infty}\sup_{k\vep\le T}\sqrt{P(\lVert Y^{u_{k\vep}^*}\rVert\ge M)}=0.
\end{align}
The last equality again uses \eqref{eq:callback}. Combining the above display with \eqref{eq:tailest1} establishes \eqref{eq:toshow1}.

\vspace{0.1in}

\emph{Proof of \eqref{eq:toshow2}.} Fix $M_1,M>0$ and define 
$$R_{M,\vep}(x):=
\log\int_{\lVert y\rVert\le M} \exp\left(\frac{1}{\vep}\langle x,y\rangle - \frac{1}{\vep}u_{k\vep}(x)-\frac{1}{\vep}\opV[u_{k\vep}](y)-g(y)\right)\,dy.$$ 
We first claim that 
\begin{equation}\label{eq:showagain1}
\limsup_{M_2\to 0}\sup_{|x|\le M_1,\ k\vep\le T} \bigg|\frac{1}{\vep}(\opS[u_{k\vep}](x)-u_{k\vep}(x))-R_{M_2,\vep}(x)\bigg|=0.
\end{equation}

\emph{Proof of \eqref{eq:showagain1}.} 
Note that, by definition, 
$$R_{\infty,\vep}(x)=\frac{1}{\vep}(\opS[u_{k\vep}](x)-u_{k\vep}(x)).$$
Consequently, we can drop the absolute value in \eqref{eq:showagain1}. Therefore, to establish \eqref{eq:showagain1}, in view of \eqref{eq:zerocon}. it would then suffice to show that 

\begin{align}\label{eq:showagain12}
    \limsup\limits_{M_2\to 0}\sup_{k\vep\le T}\sup_{\lVert x\rVert\le M_1}(\exp(R_{\infty,\vep}(x))-\exp(R_{M_2,\vep}(x)))=0.
\end{align}

\begin{align*}
        v_{k\vep}'(x)=\frac{1}{2}f'(x)-g'(x^{u_{k\vep}})u_{k\vep}''(x)+\frac{u_{k\vep}'''(x)}{u_{k\vep}''(x)},
    \end{align*}
    and by using \eqref{eq:matt2}, we also have:
    \begin{align*}
        v_{k\vep}''(x)=\frac{1}{2}f''(x)-g''(x^{u_{k\vep}})(u_{k\vep}''(x))^2-g'(x^{u_{k\vep}})u_{k\vep}'''(x)+\frac{u_{k\vep}^{(4)}(x)}{u_{k\vep}''(x)}+w_{k\vep}'''(x^{u_{k\vep}})u_{k\vep}'''(x).
    \end{align*}

    Plugging the two observations above into \eqref{eq:matid11}, we get:
    
Towards this direction, note that 
$$\sup_{k\vep\le T}\sup_{\lVert x\rVert \le M_1} \lVert x^{u_{k\vep}}\rVert \le M_1 C_T+\sup_{t\in [0,T]}\lVert \nabla u_{t}(0)\rVert=:\ell_T.$$
by the same computation as in \eqref{eq:upbd1}, we get that:
\begin{align}\label{eq:keepshow}
    &\;\;\;\;\exp(R_{\infty,\vep}(x))-\exp(R_{M_2,\vep}(x))\nonumber \\&\le (C_T)^d \E_{Z_{\vep,C_T}}\bigg[\left(\E_{|Z_{\vep,C_T}}\exp\left(-f((x^{u_{k\vep}}+Z_{\vep,C_T})^{u_{k\vep}^*}+Z_{\vep,(C_T)^{-1}})\right)\right)^{-1}\nonumber \\ &\qquad \qquad \exp(-g(x^{u_{k\vep}}+Z_{\vep,C_T}))\bigg]\bm{1}(\lVert Z_{\vep,C_T}\rVert\ge M_2-\ell_T).
\end{align}
By taking $M_2\to\infty$ and using \cref{asn:solcon}, the conclusion in \eqref{eq:showagain12} and consequently \eqref{eq:showagain1} follows.

\vspace{0.1in} 

\noindent Now that we have reduced all relevant integrals to compact sets, we can use the standard Laplace approximation \cite{erdelyi1956asymptotic} to prove \eqref{eq:toshow2}. Towards this direction, we observe that for any $M>0$, 
$$\sup_{k\vep\le t}\sup_{\lVert y\rVert \le M}\bigg|\frac{1}{(2\pi\vep)^{\frac{d}{2}}}\exp\left(\frac{1}{\vep}\opV[u_{k\vep}](y)-\frac{1}{\vep}u_{k\vep}^*(y)+f(y^{u^*})+\frac{1}{2}\ldet\left(\frac{\partial y\hfill}{\partial y^{u_{k\vep}^*}}\right)\right)-1\bigg|\to 0,$$
as $\vep\to 0$. Applying the above coupled with \eqref{eq:showagain1} yields:
$$\sup_{k\vep\le t}\sup_{\lVert x\rVert \le M}\bigg|\exp\left(\frac{1}{\vep}\opS[u_{k\vep}](x)-\frac{1}{\vep}u_{k\vep}(x)-f(x)+g(x^{u_{k\vep}})-\frac{1}{2}\ldet\left(\frac{\partial x^{u_{k\vep}}}{\partial x\hfill}\right)\right)-1\bigg|\to 0,$$
as $\vep\to 0$. This completes the proof of \eqref{eq:step16}.

\vspace{0.1in}

\emph{Proof of \eqref{eq:step17}.} We will assume $\vep\le \vep_0$ from \cref{asn:Berman}. Define
$$\mathcal{G}_{\vep}(x):=\int \frac{\exp\left(\frac{1}{\vep}\langle x,y\rangle - \frac{1}{\vep}u_{k\vep}(x)-\frac{1}{\vep}u_{k\vep}^*(y)-g(y)-f(x)+h_{k\vep}(x)\right)}{\int \exp\left(\mfR{\xi_{k,\vep}^{(2)}}{z}{x}+\frac{1}{\vep}\langle z,y\rangle -\frac{1}{\vep}u_{k\vep}(x)-\frac{1}{\vep}u^*_{k\vep}(y)-f(z)\right)\,dz}\,dy.$$
By using \eqref{eq:step16}, it suffices to show that 
\begin{align}\label{eq:17show}
\int \big|\log{\mathcal{G}_{\vep}(x)}\big|\exp(-h_{k\vep}(x))\,dx\to 0,
\end{align}
as $\vep\to 0$. By \cref{asn:Berman}, it follows that:
\begin{align}\label{eq:17step1}
&\;\;\;\;\big|\mfR{\xi_{k,\vep}^{(2)}}{z}{x}\big|\nonumber\\ &\le (a_T+b_T\lVert x\rVert + b_T\lVert z-x\rVert)\lVert z-x\rVert\nonumber \\ &\le \big(a_T+b_T\lVert x\rVert\big) \big(\lVert z-y^{u_{k\vep}^*}\rVert+\lVert x-y^{u_{k\vep}^*}\rVert\big)+2b_T\big(\lVert z-y^{u_{k\vep}^*}\rVert^2+\lVert x-y^{u_{k\vep}^*}\rVert^2\big)\nonumber \\ &\le \big(a_T+b_T\lVert x\rVert\big) \big(\lVert z-y^{u_{k\vep}^*}\rVert+(c_T)^{-1}\lVert x^{u_{k\vep}}-y\rVert\big)+2b_T\big(\lVert z-y^{u_{k\vep}^*}\rVert^2+(c_T)^{-2}\lVert x^{u_{k\vep}}-y\rVert^2\big).
\end{align}
By the same computation as in \eqref{eq:upbd1} and \eqref{eq:lbd1}, we have:
\begin{align}\label{eq:17step2}
\mathcal{G}_{\vep}(x)&\le \E_{Z_{\vep,C_T}}\bigg[\bigg(\E_{|Z_{\vep,C_T}}\exp\bigg(-(a_T+b_T\lVert x\rVert)\lVert Z_{\vep,(C_T)^{-1}}\rVert-2b_T \lVert Z_{\vep,(C_T)^{-1}}\rVert^2\nonumber \\&-f((x^{u_{k\vep}}+Z_{\vep,C})^{u_{k\vep}^*}+Z_{\vep,(C_T)^{-1}})\bigg)\bigg)^{-1}\exp\big(-g(x^{u_{k\vep}}+Z_{\vep,C_T})\nonumber \\ &+(c_T)^{-1}(a_T+b_T\lVert x\rVert)\lVert Z_{\vep,C_T}\rVert+2b_T(c_T)^{-2}\lVert Z_{\vep,C_T}\rVert^2\big)\bigg]=\omega_{a_T,b_T,2b_T,c_T,C_T,\vep}.
\end{align}
A lower bound also holds for $\mathcal{G}_{\vep}(x)$ with $C_T$ replaced by $c_T$. By \cref{asn:solcon} and the same computation as in \eqref{eq:zerocon}, it suffices to show that, for any fixed $M>0$,
\begin{equation}\label{eq:17step3}
    \sup_{\lVert x\rVert\le M}\log{\mathcal{G}_{\vep}(x)}\to 0,\quad \mbox{as}\ \vep\to 0.
\end{equation}
We can then repeat the same computation as in \eqref{eq:keepshow} which will imply, after invoking \eqref{eq:17step1}, that it suffices to show (by symmetry) that 
\begin{align}\label{eq:17step3}
\sup_{\lVert x\rVert\le M} \log{\int_{\lVert y\rVert\le M} \frac{\exp\left(R^{(2,\vep)}(x,y)+\frac{1}{\vep}\langle x,y\rangle - \frac{1}{\vep}u_{k\vep}(x)-\frac{1}{\vep}u_{k\vep}^*(y)-g(y)-f(x)+h_{k\vep}(x)\right)}{\int \exp\left(-R_x^{(1,\vep)}(y,z)+\frac{1}{\vep}\langle z,y\rangle -\frac{1}{\vep}u_{k\vep}(x)-\frac{1}{\vep}u^*_{k\vep}(y)-f(z)\right)\,dz}\,dy}\to 0,
\end{align}
as $\vep\to 0$ for every fixed $M$, where 
$$R_x^{(1,\vep)}(y,z):=(a_T+b_T\lVert x\rVert)\lVert z-y^{u_{k\vep}^*}\rVert+2b_T\lVert z-y^{u_{k\vep}^*}\rVert^2,$$
and 
$$R^{(2,\vep)}(x,y):=(a_T+b_T\lVert x\rVert)\lVert z-y^{u_{k\vep}^*}\rVert+2b_T\lVert z-y^{u_{k\vep}^*}\rVert^2.$$
The conclusion, i.e., \eqref{eq:17step3} now follows by applying the Laplace approximation \cite{erdelyi1956asymptotic} as before, to the numerator and the denominator separately.
\end{proof}

Consider an SDE of the form
\begin{equation}\label{eq:sdegen}
dY_t = b(t,Y_t)dt + \sigma(t, Y_t) dB_t,
\end{equation}
where $B$ is a standard multidimensional Brownian motion and with an initial condition $Y_0=y_0$ ({\color{blue} In our case $Y_0$ is non-degenerate. Does that change anything?}). 


Stroock and Varadhan proved that (see \cite[Theorem 5.11]{klebaner}) that if $(y,t)\mapsto \sigma(y,t)$ is continuous, positive and if, for each $T>0$, there is a constant $K_T>0$ such that 
\begin{equation}\label{eq:weakexist}
\abs{b(y,t)} + \abs{\sigma(y,t)} \le K_T(1+ \abs{y}),
\end{equation}
for all $(y,t)\in \R^d \times [0,T]$, then the SDE \eqref{eq:sdegen} admits a unique weak solution for any $y_0$ which is also strong Markov. By stopping the process when it hits a ball of radius $R$, we can replace \eqref{eq:weakexist} by a local linear growth criterion: weak existence and uniqueness holds if for every $T>0$ and every $R>0$, there is a constant $K(T,R)$ such that   
\begin{equation}\label{eq:weakexistnew}
\sup_{\abs{y}\le R,\; 0\le t\le T }\left[ \abs{b(y,t)} + \abs{\sigma(y,t)}\right] \le K(T,R)(1+ \abs{y}),
\end{equation}

In the case of \eqref{eq:dualdiffSDE}, 
\[
b(y,t)= -\frac{\partial h_t\hfill}{\partial y^{w_t}}(y^{w_t}),\quad \sigma^2(y,t)= 2\frac{\partial y\hfill}{\partial y^{w_t}}. 
\]
As $(\nabla u_t)_{\#}\rho_t=\exp(-g)$ and $\rho_t=\exp(-h_t)$, \eqref{eq:pma} implies $$h_t(x)=\frac{\partial}{\partial t}u_t(x)-f(x).$$
Then writing both $b(y,t)$ and $\sigma(y,t)$ in terms of $u_t$, we get:
\[
b(y,t)= -\nabla^2 u_t(y^{w_t})\left(\frac{\partial}{\partial y}\frac{\partial}{\partial t}u_t(y^{w_t})-\frac{\partial}{\partial y}f(y^{w_t})\right),\quad \sigma^2(y,t)= 2\nabla^2 u_t(y^{w_t}). 
\]
By Assumptions \ref{asn:smoothfg} and \ref{asn:solcon}, both $b(\cdot,\cdot)$ and $\sigma(\cdot,\cdot)$ are continuous functions on $\{(y,t): |y|\le R,\ 0\le t\le T\}$, which in turn, yields \eqref{eq:weakexistnew}.



\end{comment}


\begin{comment}
\section{System Architecture}
\label{appendix:architecture}
\system has a novel modularized system architecture with three key components: 
\emph{StreamManager}, 
\emph{TxnManager} and \emph{TxnScheduler}. 
These components are instantiated in each thread locally.
The execution outline of \system is presented in Algorithm~\ref{alg:algo}.
Transactional stream processing is continuous and potentially never ends (Line 1$\sim$8).
The dependency resolution and execution of state transactions are separated into two non-overlapping phases by punctuations~\cite{Tucker:2003:EPS:776752.776780} (Line 2 and 5), which guarantees that no subsequent input event will have a smaller timestamp. 
Effectively, a batch of state transactions is collected during the first phase, and processed during the second phase.

In the first phase (i.e., stream processing phase), 
the \emph{StreamManager} conducts preprocessing for every input event ($e$). Similar to some prior works~\cite{tstream}, state transactions may be issued but not immediately processed during preprocessing (Line 3).
The \emph{pre\_processing} and \emph{post\_processing} functions are exposed as APIs to users.
The \emph{TxnManager} handles dependency resolution (Line 4) among state transactions and insert decomposed operations to construct a \tpg. We discuss the detailed two-phase \tpg construction process in Section~\ref{subsec:construction}.

In the second phase  (i.e., transaction processing phase), 
the \emph{TxnManager} is first involved again to refine (Line 6) the constructed \tpg with further dependency resolution.
The \emph{TxnScheduler} 
schedules operations for concurrent execution based on the constructed \tpg according to the three dimensions of scheduling decisions (Line 7). 
In particular, a scheduling decision model $M$ is instantiated based on the constructed \tpg (Line 14).
\textbf{\circled{1}} Guided by $M$, execution threads adopt an exploration strategy (Section~\ref{subsec:explore}) to explore the constructed \tpg for operations available to be scheduled constrained by dependencies. 
\textbf{\circled{2}} 
During exploration, one or multiple operations may be treated as the 
% basic 
unit of scheduling (Section~\ref{subsec:granularity}). 
Subsequently, \textbf{\circled{3}} every thread executes operation(s) in the unit of scheduling with various abort handling mechanisms (Section~\ref{subsec:abort_handling}).
Only when state transactions are processed (i.e., committed or aborted) can the associated input events be postprocessed (Line 8) by the \emph{StreamManager} based on transaction processing results.
\end{comment}

\begin{comment}
\begin{algorithm}
\footnotesize
    \KwData{$e$ \tcp{Input event}}
    \KwData{$txn_{ts}$ \tcp{State transaction}}
    \KwData{$G$ \tcp{The currently constructed TPG}}
    \While{!finish processing of input streams}{
        \eIf(\tcp*[h]{Phase 1}){\text{$e$ is not a $punctuation$}}{
                $txn_{ts}$ $\gets$ PRE\_Processing($e$)\;
                \textbf{TPG\_Construction}($G$, $txn_{ts}$)\; 
          }(\tcp*[h]{Phase 2}){
                \textbf{TPG\_Refinement}($G$)\; 
                \textbf{TXN\_Scheduling}($G$)\; 
                POST\_Processing()\;
          }
    }
    
    \SetKwFunction{FMain}{TPG\_Construction}
    \SetKwProg{Fn}{Function}{:}{}
    \Fn{\FMain{$G$, $txn_{ts}$}}{
        $O_{1..k}$ $\gets$ \textbf{Partition} $txn_{ts}$\;
        \ForEach{\text{operation $O_{i}$ $\in$ $O_{1..k}$}}{
            \textbf{Identify} its \ld\;
            $G$ $\gets$ $G$ + $O_{i}$ \;
        }
    }
    \SetKwFunction{FMain}{TPG\_Refinement}
    \SetKwProg{Fn}{Function}{:}{}
    \Fn{\FMain{$G$}}{
        \ForEach{\text{vertex $e_{i}$ $\in$ $G$}}{
            \textbf{Identify} its \td, \pd\;
        }
    }
    
    \SetKwFunction{FMain}{TXN\_Scheduling}
    \SetKwProg{Fn}{Function}{:}{}
    \Fn{\FMain{$G$}}{
        $M$ $\gets$ Instantiated with $G$;\tcp{A decision model}
        \While{!finish scheduling of $G$
        }{
          \textbf{\circled{2}} $Scheduling Unit$ $\gets$ \textbf{\circled{1}} \emph{Explore}($G$, $M$)\; 
            \textbf{\circled{3}} \emph{Execute with Abort Handling} ($Scheduling Unit$)\; 
        }
    }
  \caption{Execution Outline of \system}
  \label{alg:algo}
\end{algorithm}
\end{comment}

%\input{SinkhornSections/Bermantype}
\bibliographystyle{amsalpha}
\bibliography{references}
\end{document}



%Leftover stuff

{\color{blue}
\begin{proof}
    Given a measure $\mu$ with positive density and functions $\psi_1$, $\psi_2$, we define:
    $$\mgf{\mu}{\psi_1}{\psi_2}:=\int (\psi_2-\psi_1)(x)\,d\mu(x)+\int (\psi_2^*-\psi_1^*)(y) e^{-g(y)}\,dy.$$
    Given $\xi,\rho$ from the question, set $\mu=\xi$, $\psi_1$ to be the Brenier potential from $\xi$ to $e^{-g}$ and $\psi_2$ to be the Brenier potential from $\rho$ to $e^{-g}$. Note that 
    $$\frac{1}{2}\wass_2^2(\xi,e^{-g})=\int \left(\frac{1}{2}\lVert x\rVert^2-\psi_1(x)\right)\,d\xi(x)+\int \left(\frac{1}{2}\lVert y\rVert^2-\psi_1^*(y)\right)e^{-g(y)}\,dy.$$
    We can write a similar relation with $\rho$ and $\psi_2$. Some simple algebra then yields:
    $$D_U(\xi|\rho)=\mgf{\xi}{\psi_1}{\psi_2}\ge 0.$$
    Given any $x,y\in\R^d$, note that by assumption
    $$\psi_2^*(y)\le \psi_2^*(x)+\iprod{y-x,\nabla\psi_2^*(x)}+L\lVert y-x\rVert^2.$$
    Therefore, 
    \begin{align*}
        \psi_2(z)&\ge \sup_{y\in\R^d} \left(\iprod{y,z}-\psi_2^*(x)-\iprod{y-x,\nabla\psi_2^*(x)}-L\lVert y-x\rVert^2\right)\\ &=-\psi_2^*(x)+\iprod{x,z}+\frac{1}{4L}\lVert z-\nabla\psi_2^*(x)\rVert^2.
    \end{align*}
    By putting $z=\nabla\psi_1^*(x)$ and integrating over the measure $e^{-g}$, we get:
    \begin{align*}
        &\;\;\;\;\;\int \left(\psi_2(\nabla\psi_1^*(y))+\psi_2^*(y)\right)e^{-g(y)}\,dy-\int \iprod{y,\nabla\psi_1^*(y)}e^{-g(y)}\,dy\\ &\ge \frac{1}{4L}\int \lVert \nabla\psi_1^*(y)-\nabla\psi_2^*(y)\rVert^2 e^{-g(y)}\,dy=\lot{e^{-g}}(\xi,\rho).
    \end{align*}
    By using a change of variable formula, the first term in the above display simplifies to
    $$\int \left(\psi_2(\nabla\psi_1^*(y))+\psi_2^*(y)\right)e^{-g(y)}\,dy=\int \psi_2(x)\,d\xi(x)+\int \psi_2^*(y)e^{-g(y)}\,dy.$$ 
    For the next term, we note that $\iprod{y,\psi_1^*(y)}=\psi_1^*(y)+\psi_1(\nabla\psi_1^*(y))$, to see that it equals 
    $$\int \iprod{y,\nabla\psi_1^*(y)}e^{-g(y)}\,dy=\int \psi_1(x)\,d\xi(x)+\int \psi_1^*(y)e^{-g(y)}\,dy.$$
    This completes the proof.
\end{proof}}

\SP{What follows is a non-rigorous argument that shows \eqref{eq:newgradflow} satisfies \eqref{eq:mirrorgradflow}. Assuming enough smoothness it should be possible to be rigorously argued. If \eqref{eq:mirrorgradflow} has a unique solution, it also proves the converse.} 
\bigskip

\paragraph{\bf Interpretation using the Wasserstein distance.}

Let $(\rho_t, v_t\; t\ge 0)$ be the solution of the continuity equation $\dot{\rho_t}+ \div(v_t \rho_t)=0$. Assume that $v_t \in \tanspace_{\rho_t}$. The following formula is a special case of \cite[Theorem 5.24]{santambrogio2015optimal}:
\begin{equation}\label{eq:derivwass}
    \frac{d}{dt} \wass_2^2\left(\rho_t, e^{-g} \right) = 2\int \rho^U_t(x) \cdot v_t(x) \rho_t(x) dx, 
\end{equation}
where $\rho^U_t=\mathbf{id} - u_t$ is the gradient of the Kantorovich potential transporting $\rho_t$ to $e^{-g}$.

Consider the pair of infinitesimally close time points $t$ and $t+dt$. In the nonrigorous argument below we will drop every term $o(dt)$. This should be true by boundedness assumption on terms and the absolute continuity of all curves. 

By \eqref{eq:derivwass},
\[
\begin{split}
\wass_2^2\left(\rho_{t+dt}, e^{-g} \right)&= \wass_2^2\left(\rho_t, e^{-g} \right) + \left(2\int \rho_t^U(x) \cdot v_t(x) \rho_t(x) dx \right)dt\\
&=\int \norm{\rho_t^U(x)}^2\rho_t(x) dx +  \left(2\int \rho_t^U(x)  \cdot v_t(x) \rho_t(x) dx \right)dt\\
&=\int \norm{\rho_t^U(x)  + v_t(x) dt}^2 \rho_t(x)dx. 
\end{split}
\]
The second equality above follows from definition of $\rho^U_t$ and the third follows by dropping the $(dt)^2$ term.

Now, by the continuity equation, $\rho_{t+dt}(x)= \rho_t(x) - \div(v_t(x)\rho_t(x)) dt$. Substituting in the last integral and one integration by parts, one gets
\[
\begin{split}
&\wass_2^2\left(\rho_{t+dt}, e^{-g} \right)=\int \norm{\rho_t^U(x) + v_t(x) dt}^2 \rho_{t+dt}(x)dx\\
&+ \left(\int \nabla \norm{\rho_t^U(x) + v_t(x) dt}^2 v_t(x)\rho_t(x)dx\right) dt\\
&=\int\left( \norm{\rho_t^U(x) + v_t(x) dt}^2 + \nabla \norm{\rho_t^U(x) + v_t(x) dt}^2 v_t(x) dt\right)\rho_{t+dt}(x)dx.
\end{split}
\]
The final equality is allowed by ignoring $O(dt)$ terms that arise by substituting $\rho_{t+dt}$ for $\rho_t$ in the second line which then gets multiplied by $dt$ to give us a $(dt)^2$ term.

Now, let us look at the integrand. Ignoring repeatedly terms that are $(dt)^2$,
\[
\begin{split}
   & \norm{\rho_t^U(x) + v_t(x) dt}^2 + \nabla \norm{\rho_t^U(x) + v_t(x) dt}^2 v_t(x) dt\\
   &= \norm{\rho_t^U(x) + v_t(x) dt}^2 +  2\left(\rho_t^U(x) + v_t(x) dt\right)\left( \nabla \rho_t^U(x) \right) v_t(x) dt\\
   &=\norm{\rho_t^U(x)  + v_t(x) dt}^2 + 2\left(\rho_t^U(x)  \right)\left(  \frac{\partial x^{u_t}}{\partial x\hfill} - I \right) v_t(x) dt\\
   &= \norm{\rho_t^U(x)  + \frac{\partial x^{u_t}}{\partial x\hfill} v_t(x) dt }^2.
\end{split}
\]
Thus, 
\[
\wass_2^2\left(\rho_{t+dt}, e^{-g} \right)= \int \norm{\rho_t^U(x) + \frac{\partial x^{u_t}}{\partial x\hfill} v_t(x) dt }^2\rho_{t+dt}(x)dx. 
\]
But this tells us that 
\[
\rho_{t+dt}^U = \rho_t^U(x) + \frac{\partial x^{u_t}}{\partial x\hfill} v_t(x) dt.
\]
In other words, 
\[
\frac{d}{dt} \rho_t^U = \frac{\partial x^{u_t}}{\partial x\hfill} v_t(x).
\]

Now, consider \eqref{eq:newgradflow} for which $v_t=-\nabla_{x^{u_t}}\left(\frac{\delta F}{\delta \rho_t} \right)$ by \eqref{eq:expgrad}. Substituting this expression in the above display and using chain rule for derivatives give us \eqref{eq:mirrorgradflow}. By a change fo sign it gives \eqref{eq:mirrorgradflow2} which gives the PMA \eqref{eq:pma}.

%The italicized phrase above is implied by showing that the solution to \eqref{eq:expgrad}, once started from a convex gradient, remains a convex gradient. 

\bigskip


{\color{blue}
\paragraph{\bf Interpretation using the graph of the mappings.}
As another view point, 
we will derive \eqref{eq:newgradflow} from \eqref{eq:mirrorgradflow} using the geometry of the graph of the mappings. 

The graph of the mapping $T_t : \Omega \to \Lambda$ is a set in the product space $\Omega \times \Lambda$ of the source and the target domain. At each point $(x, T_t (x))$ in the graph, we can consider the tangent vector  $(v, w)$ consisting of the horizontal vector $v$ and the vertical vector $w$; here the horizontal vector $v$ is nothing but a tangent vector of the source domain at $x$ and the vertical vector $w$ is a tangent vector of the target domain at $T_t (x)$.

The graph of $T_t$ as a set $\Sigma_t$  in the product space of the source and the target, can also be seen as the graph of the inverse mapping $(T_t)^{-1}$. Then, the derivative $\frac{\partial}{\partial t} T_t (x)$ corresponds to the change of the graph $\Sigma_t$  following the vertical vector field  $\frac{\partial}{\partial t} T_t (x)$ at $(x, T_t(x))$,  while 
the derivative $\left[\frac{\partial}{\partial t} (T_t)^{-1}\right] (T_t(x))$ corresponds to the change of the graph $\Sigma_t$ following the horizontal vector field  $\left[\frac{\partial}{\partial t} (T_t)^{-1}\right] ( T_t(x))$ at $(x, T_t(x))$. 
These horizontal and vertical vectors at $(x, T_t(x))$ are related in the following way (from chain rule)



 Moreover, push-forward of $\rho_t$ by $T_t$ is $e^{-g}$, in other words, $\rho_t$ is the result of the push-forward of $e^{-g}$ by $(T_t)^{-1}=(\nabla u_t)^{-1}$. One can view the change of $\rho_t$ in the time $t$  as the result of the moving mass on source domain by the horizontal vector field $\left[\frac{\partial}{\partial t} (T_t)^{-1}\right] ( T_t(x))$.  Therefore, $\rho_t$ follows the continuity equation:
\begin{align*}
    \frac{\partial}{\partial t} \rho_t(x)  + \div \left(\rho_t (x) \left[\frac{\partial}{\partial t} (T_t)^{-1} \right] (T_t(x))\right) =0.
\end{align*}
%To see this last equation, notice that $\rho_t (x) = e^{-g (T_t(x)) } \det [\nabla T_t (x)]$. If we differentiate this in $t$ we get
%\begin{align*}
%  \frac{\partial}{\partial t}  \rho_t (x) & = 
 %   \left( \frac{\partial}{\partial t}  e^{-g (T_t(x)) }\right) \det [\nabla T_t (x)] + 

%    e^{-g (T_t(x)) } \frac{\partial}{\partial t}  \det [\nabla T_t (x)].
%\end{align*}
%We use the formula $\frac{\partial}{\partial t} \det A(t) = (\det A(t)) \tr A(t)^{-1}\frac{\partial}{\partial t} A(t)  $.


Now since 
we have that  $$\frac{\partial}{\partial t}\rho_t^U  = - \frac{\partial}{\partial t} \nabla u_t =  - \frac{\partial}{\partial t} T_t.$$  Therefore, when 
$$\frac{\partial}{\partial t}\rho_t^U =-\nabla_{\mathbb{W}} F (\rho_t) $$ as in   \eqref{eq:mirrorgradflow} we see from \eqref{eq:inverse_derivative} that 
\begin{align*}
    \frac{\partial}{\partial t} \rho_t(x)  - \div \left(\rho_t (x) [\nabla^2 u_t ]^{-1}(x) \left[\nabla_{\mathbb{W}} F (\rho_t)\right] (x)  \right) =0
\end{align*}
as in \eqref{eq:newgradflow}.



}
------------------------------------------


%On the JKO scheme

One may also ask about the corresponding implicit mirror scheme, which is the JKO scheme for gradient flow. For the choice of $U$ given above and a $\rho \in \probspace$ one can define the following Bregman divergence at $\rho$:
\[
D_U( \xi \mid \rho )= \frac{1}{2}\wass_2^2(\xi, e^{-g}) - \frac{1}{2}\wass_2^2(\rho, e^{-g}) - \iprod{ \nabla_{\wass}U(\rho), \nabla \gamma_{\rho}^{\rho'}}_{\rho},
\]
where $\xi\in \probspace$ is absolutely continuous with density also denoted by $\xi$, and $\gamma_{\rho}^{\xi}$ is  the direction of the generalized geodesic joining $\rho$ to $\xi$ with base at $e^{-g}$ (\cite[Definition 9.2.2]{ambrosio2005gradient}). That is 
\[
\gamma_{\rho}^{\xi}= \mathbf{t}^{\xi} - \mathbf{\rho},
\]
where $\mathbf{t}^\xi$ is the gradient of the Kantorovich potential transporting $e^{-g}$ to $\xi$ and $\mathbf{t}^\rho$ is defined similarly. 

Since $U$ is generalized geodesically convex $D_U(\cdot \mid \cdot)\ge 0$. Although we do not prove this in this article, we conjecture that a JKO scheme with this divergence will convergence to mirror gradient flow \eqref{eq:velmirror}. 

{\color{blue} ND (based on discussion with YH). An alternate way of approaching the JKO scheme would be to consider the following Bregman divergence in continuum:
\[
D_U( \xi \mid \rho )= \frac{1}{2}\wass_2^2(\xi, e^{-g}) - \frac{1}{2}\wass_2^2(\rho, e^{-g}) - \iprod{ \nabla_{\fv}U(\rho), \xi-\rho}.
\]
Here $\nabla_{\fv}$ denotes the first variation. We can argue $D_U(\cdot|\cdot)\ge 0$ as follows: note that $\nabla_{\fv}U(\rho)=\phi$, the $c$-concave (with $c(x,y)=\lVert x-y\rVert^2/2$) Kantorovich potential from $\rho$ to $e^{-g}$. Therefore,
$$\frac{1}{2}\wass_2^2(\rho,e^{-g})=\int \phi(x)\,d\rho(x)+\int \phi^c(y)e^{-g(y)}\,dy.$$
By duality and the $c$-concavity of $\phi$, we also get:
\begin{equation}\label{eq:ineqdual}
\frac{1}{2}\wass_2^2(\xi,e^{-g})\ge \int \phi(x)\,d\xi(x)+\int \phi^c(y)e^{-g(y)}\,dy.
\end{equation}
By combining the two displays above, we get:
$$D_U(\xi|\rho)\ge \int \phi(x)\,d(\xi-\rho)(x)-\iprod{ \nabla_{\fv}U(\rho), \xi-\rho}=0.$$
Further, equality in the above holds if and only if equality holds in \eqref{eq:ineqdual}. If, for instance, $\xi$ has a positive density, then by uniqueness of Kantorovich potentials, equality holds if and only if $\xi=\rho$. This I believe is a reformulation of the fact that $\wass_2^2(\cdot,e^{-g})$ is strictly convex on the line (see \cite[Proposition 7.19]{santambrogio2015optimal}). With this in view, one potential way to construct an implicit JKO type scheme is to interatively minimize the following:
$$\argmin_{\rho} \left[\KL{\rho^{(k)}}{e^{-f}}+\iprod{\log{\rho^{(k)}}+f,\rho-\rho^{(k)}}+\tau^{-1}D_U(\rho|\rho^{(k)})\right],$$
for some small $\tau>0$. This is the implicit mirror descent scheme with an appropriate choice of the mirror function (see \cite{leger2021gradient,nemirovskii83,beck2003mirror}). Algebraically this seems to be harder to handle even with the observation $D_U(\xi|\rho)\ge 0$ (I believe we would need $D_{\KL{\cdot}{e^{-f}}}(\xi|\rho)\le D_U(\xi|\rho)$ to make it work rigorously; this is achievable in the fixed $\vep$ case; see \cite[Lemma 4]{leger2021gradient}).\par

Instead we can look at the explicit mirror descent scheme which is more stable, and given by:
$$\rho^{(k+1)}:=\argmin_{\rho}\left[\KL{\rho}{e^{-f}}+\tau^{-1}D_U(\rho|\rho^{(k)})\right].$$
The objective function is lower bounded, lower semi-continuous and convex on lines (see \cite[Proposition 7.19]{santambrogio2015optimal}). To guarantee existence of a minimizer, we can proceed as in the proof of \cite[Proposition 8.5 and Lemma 8.6]{santambrogio2015optimal} which does not seem to require any other property. The stationary conditions would then be given by
$$\phi^{(k+1)}-\phi^{(k)}=-\tau\left(\log{\rho^{(k+1)}}+f\right),$$
where $\phi^{(k+1)}$ and $\phi^{(k)}$ are the Kantorovich potentials from $\rho^{(k+1)}$ and $\rho^{(k)}$ to $e^{-g}$. The Brenier Potentials are then given by $u^{(k)}:=\lVert \cdot\rVert^2/2-\phi^{(k)}$ and consequently,
$$u^{(k+1)}-u^{(k)}=\tau\log{\rho^{(k+1)}}=\tau\log{\rho^{(k)}}+o(\tau)=\tau\left(f(x)-g(x^{\phi^{(k)}})+\ldet\frac{\partial x^{\phi^{(k)}}}{\partial x\hfill}\right)+o(\tau).$$
Therefore, in the scaling limit, we should get back the PMA \eqref{eq:pma}. Next note that by optimality,
$$\KL{\rho^{(k+1)}}{e^{-f}}+\tau^{-1}D_U(\rho^{(k+1)}|\rho^{(k)})\le \KL{\rho^{(k)}}{e^{-f}}.$$
By summing over $k$, we get:
$$\sum_{k=0}^T D_U(\rho^{(k+1)}|\rho^{(k)})\le \tau \KL{\rho^{(T+1)}}{e^{-f}}.$$
Under some assumptions $D_U(\cdot|\cdot)$ seems to have a nice connection with the linearized OT distance; see  \cite{Wang2013,moosmuller2020linear,merigot2020quantitative} for the relevant definition an applications.

---------------------------------------

%\subsection{Connections to Linearized optimal transport}

\begin{lmm}
Suppose $\xi$, $\rho$ have positive densities on $\R^d$. Also suppose that the Brenier potential from $e^{-g}$ to $\rho$ has a Hessian with operator norm upper bounded uniformly by $2L>0$. Then we have
$$D_U(\xi|\rho)\ge \frac{1}{4L}\lot{e^{-g}}(\xi,\rho).$$
\end{lmm}

\begin{remark}
    In fact, if we put both upper and lower bounds (say $2L$ and $2/L$) on the operator norm of the Hessian of the Brenier potential from $e^{-g}$ to $\rho$, then we get the following bound:
    $$4L\lot{e^{-g}}(\xi,\rho)\ge D_U(\xi|\rho)\ge \frac{1}{4L}\lot{e^{-g}}(\xi,\rho).$$
\end{remark}
{\color{red} Does this help understand the Hessian corresponding to $U$ in some way? If we take $\xi=\rho+d\rho$ with $d\rho$ an appropriately chosen small perturbation? Can't formalize it yet.}
}


%%% Earlier attempts to show process convergence

\section{Convergence to Markov process}\label{sec:convproc}


Suppose $(X_k^{(\vep)},\; k =0,1,2,\ldots)$ refer to the Sinkhorn Markov chain with an initial condition $X_0^\vep=x_0$ that does not depend on $\vep$. Turn this to a continuous time process by a piecewise constant interpolation:
\[
X_t^{\vep} := X_k^{\vep}, \quad \text{if}\; k\vep \le t < (k+1)\vep. 
\]
Let $(u_t)$ be the corresponding solution of the PMA. Our objective will be to show the following. Let $P^\vep$ denote the law of the process $(X_t^{\vep},\; t\ge 0)$ on the Skorokhod space $D^d[0, \infty)$ of RCLL paths from $[0,\infty)$ to $\R^d$, then 
\begin{enumerate}
    \item $\left( P^\vep,\; \vep >0 \right)$ is a tight family of probability distributions and any limit point has full measure over the set of continuous paths $C^d[0, \infty)$.
    \item Any limit point is a solution of the \textit{martingale problem} (MP) \cite[Chapter 5, Definition 4.5]{karatzas1991brownian} associated with the SDE \eqref{eq:diffSDE}. 
\end{enumerate}
%An explanation of what MP means follows. Given that we have shown SDE \eqref{eq:diffSDE} admits a unique weak solution, the two steps outlined above shows weak convergence of $(X_t^\vep,\; t\ge 0)$ to a solution of \eqref{eq:diffSDE}.



%Reversing the diffeomorphism, one can now show that $(X_t^\vep,\; t\ge 0)$, as $\vep \rightarrow 0+$, converges weakly to a solution of \eqref{eq:diffSDE}. 

%A definition of the martingale problem associated with an SDE can be found in \cite[Chapter 5, Definition 4.5]{karatzas1991brownian}. However, we will utilize \cite[Chapter 5, Proposition 4.6]{karatzas1991brownian} to simplify our requirement. Let $P$ denote any limit point of $P^\vep$, as $\vep \rightarrow 0+$. By the first step, it is a probability measure on $C^d[0, \infty)$ with the usual filtration generated by the coordinate projections. 

%Let $Y\sim Q$. To show that $Q$ satisfies the MP, we need to show that the following processes are (local) martingales,
%\begin{equation}\label{eq:MPreq}
%\begin{split}
%M_t(i)&=X_t(i) - \int_0^t (\text{drift at time $u$}) du, \quad i \in [d], \quad \text{and},\\
%M_t(i)&M_t(j) - 2 \int_0^t \frac{\partial X_u(i)\hfill}{\partial X_u^{w_u}(j)}du, \quad i,j\in [d]^2.
%\end{split}
%\end{equation}
%\SP{What is the proper placement of coordinates of vectors?}

%In other words, we need to verify It\^o's rule for the functions $f(y)=y(i)$ and $f(y)=y(i)y(j)$. The fact that the processes \eqref{eq:MPreq} are indeed martingales follow from our calculations on conditional expectations and considitonal variances of the Sinkhorn Markov chain. 

%To prove tightness and limiting continuity of paths as required by step (1) above we need to verify some moment estimates. See \cite[Chapter 5, problem 3.15]{karatzas1991brownian}.

%\SP{What do we need from the Markov chain?} 

Let $\left(\fil_t,\; t\ge 0\right)$ denote the usual right continuous filtration generated by the coordinate projections in $D^d[0, \infty)$. For $k\vep \le t< (k+1)\vep$, and $x\in \R^d$, define
\begin{enumerate}[(i)]
    \item $b_t^\vep(x):=\E(X_{k+1}^\vep - X^\vep_k \mid X^\vep_k=x)$. We know
    \[
    b_{k\vep}^\vep(x) = \vep \left(-\frac{\partial f\hfill}{\partial x^{u_{k\vep}}}(x)-\frac{\partial g\hfill}{\partial x^{u_{k\vep}}}\left(x^{u_{k\vep}}\right)+\frac{\partial h_{k\vep}\hfill}{\partial x^{u_{k\vep}}}(x)\right)  + o(\vep),
    \]
    where $h_{k\vep}$ is defined by the relation $e^{-h_{k\vep}}=(\nabla u_{k\vep})_{\# e^{-g}}$. 
    \item Also define the conditional covariance matrix $\Sigma^\vep_t(x)$ where, for $(i,j)\in [d]^2$,
    \[
    \Sigma^\vep_t(x)(i,j):=\mathrm{Cov}\left(X_{k+1}^\vep(i), X_{k+1}^\vep(j)  \mid X_k=x \right)\approx 2\vep\frac{\partial x(i)\hfill}{\partial x^{u_{k\vep}}(j)} + o(\vep).
    \]
\end{enumerate}

Consider the discrete time vector-valued martingale $(M_{k\vep}^\vep,\; k=0,1,2,\ldots)$ given by $M_0^\vep=0$ and, inductively,
\begin{equation}\label{eq:definemgle}
M_{(k+1)\vep}^\vep - M_{k\vep}^\vep = \sqrt{\vep}\left(\Sigma^\vep_{k\vep}(X_k^\vep) \right)^{-1/2} \left( X_{k+1}^\vep - X^\vep_k - b_k^\vep(X^\vep_k) \right).
\end{equation}
Extend the process to $\left(M_t^\vep,\; t\ge 0 \right)$ for piecewise interpolation as above. By construction, this discrete time martingale $\left( M_{k\vep}^\vep, \; k=0,1,\ldots \right)$, with respect to the filtration $\left(\fil_{k\vep},\; k=0,1,2,\ldots \right)$, satisfies
\begin{equation}\label{eq:condcovmglediff}
\mathrm{Cov}\left( M_{(k+1)\vep}^\vep - M_{k\vep}^\vep \mid \fil_{k\vep}\right)= \vep I_{d\times d},
\end{equation}
where $I_{d\times d}$ is the $d$-dimensional identity matrix.

The original process $\left( X_{k}^\vep\right)$ can be recovered as a discrete time stochastic integral with respect to this martingale. To wit, 
\begin{equation}\label{eq:discstocinteg}
X_{k+1}^\vep = x_0 + \sum_{i=0}^k b_{i\vep}^\vep(X^\vep_i) + \vep^{-1/2}\sum_{i=0}^k  \left(\Sigma^\vep_{i\vep}(X_i^\vep) \right)^{1/2} \left( M_{(i+1)\vep}^\vep - M_{i\vep}^\vep\right),
\end{equation}
where the integrand is predictable with respect to the filtration $\left(\fil_{k\vep},\; k=0,1,\ldots \right)$. 

Our proof proceeds by showing two lemmas. 

\begin{lmm}\label{lem:mglecnv}
    As $\vep\downarrow 0$, $(M_t^\vep, \; t\ge 0)$ converges weakly to a standard $d$-dimensional Brownian motion in $D^d[0, \infty)$ equipped with the locally uniform metric.
\end{lmm}

\begin{proof}
    This is a consequence of martingale CLT \cite[Theorem 8.2.4]{durrettprob}. To apply this theorem, consider the sequence of discrete martingale $(M_{k\vep}^\vep,\; k=0,1,\ldots)$ with their corresponding filtration $(\fil_{k\vep},\; k=0,1,\ldots)$. Take any arbitrary sequence $\vep_n\downarrow 0$. For $i\in [d]$, consider the real-valued martingale sequence $(M_{k\vep_n}^{\vep_n}(i),\; k=0,1,\ldots)$
    By definition, using the notation in \cite[Theorem 8.2.4]{durrettprob}, for any $t >0$,
    \[
    V_{n, [t/\vep_n]}(i):= \sum_{k=1}^{[t/\vep_n]} \E\left( \left(M_{(k+1)\vep_n}^{\vep_n}(i) - M_{k\vep_n}^{\vep_n}(i) \right)^2 \mid \fil_{k\vep}\right)= \vep_n [t/\vep_n] \rightarrow t, 
    \]
    as $n\rightarrow \infty$. 

    Condition (ii) in \cite[Theorem 8.2.4]{durrettprob} follows from the bound
    \begin{equation}\label{eq:4momentbnd}
    \E\left( \left(M_{(k+1)\vep_n}^{\vep_n}(i) - M_{k\vep_n}^{\vep_n}(i) \right)^4 \mid \fil_{k\vep}\right)= o\left( \vep_n\right).
    \end{equation}
    \SP{This is the moment bound we need to argue.}
    Since, then, for any $\delta >0$, by Markov's inequality
    \[
    \begin{split}
    \E&\left( \left(M_{(k+1)\vep_n}^{\vep_n}(i) - M_{k\vep_n}^{\vep_n}(i) \right)^2 1\left\{ \abs{M_{(k+1)\vep_n}^{\vep_n}(i) - M_{k\vep_n}^{\vep_n}(i)} > \delta \right\} \mid \fil_{k\vep}\right)\\
    &\le \delta^{-2} \E\left( \left(M_{(k+1)\vep_n}^{\vep_n}(i) - M_{k\vep_n}^{\vep_n}(i) \right)^4 \mid \fil_{k\vep}\right).
    \end{split}
    \]
    Hence, 
    \[
    \begin{split}
    \sum_{k=0}^{1/\vep_n}& \E\left( \left(M_{(k+1)\vep_n}^{\vep_n}(i) - M_{k\vep_n}^{\vep_n}(i) \right)^2 1\left\{ \abs{M_{(k+1)\vep_n}^{\vep_n}(i) - M_{k\vep_n}^{\vep_n}(i)} > \delta \right\} \mid \fil_{k\vep}\right)\\
    &\le \frac{1}{\delta\vep_n}o\left( \vep_n\right),\quad \text{ which goes to zero as $n\rightarrow \infty$}.
    \end{split}
    \]
   Thus, by the cited Theorem, every $(M_t^{\vep_n},\; t\ge 0)$ converges weakly, in the locally uniform metric, to a standard linear Brownian motion. Since this is true for every $i$ and every sequence of $\vep_n\downarrow 0$, it follows that the vector-valued process $\left(M^{\vep}_{t},\; t\ge 0\right)$ converges jointly, weakly in the locally uniform metric on $D^d[0, \infty)$, to a process $(B_t,\; t\ge 0)$, whose every coordinate process is a standard Brownian motion. 

    The fact $(B_t,\; t\ge 0)$ is a $d$-dimensional standard Brownian motion follows from Knight's Theorem \cite[Chapter 3, Theorem 4.13]{karatzas1991brownian}. Essentially, it suffices to show that the mutual variation between the coordinate processes of $B$ is zero throughout, which, in turn, follows from \eqref{eq:condcovmglediff} that the off-diagonal elements in the conditional covariance matrix of the martingale increments is zero by construction. By our established weak convergence and Skorokhod's Theorem, on a certain probability space, for every fix $t$, and every $i\neq j$, the following convergence holds in probability, 
    \[
    \lim_{n\rightarrow \infty} \sum_{k=1}^{[t/\vep_n]} \left(M_{(k+1)\vep_n}^{\vep_n}(i) - M_{k\vep_n}^{\vep_n}(i) \right)\left(M_{(k+1)\vep_n}^{\vep_n}(j) - M_{k\vep_n}^{\vep_n}(j) \right) = \iprod{B(i), B(j)}_t
    \]
    The fact that the limit on the left side of above is zero follows by taking expectation and use \eqref{eq:4momentbnd} to show convergence in probability. We skip the standard argument. 
\end{proof}


\begin{lmm}  
    The pair of processes $(X_t^\vep,\; M_t^\vep\; t\ge 0)$ converges weakly, in the locally uniform metric on $D^d[0, \infty) \times D^d[0, \infty)$ to a solution $(X,B)$ of the SDE \eqref{eq:diffSDE}. 
\end{lmm}


\begin{proof}
    As in the proof of the last lemma, it suffices to take a sequence $\vep_n\rightarrow 0$. For simplicity, we will take $\vep_n=1/n$, but the argument below does not depend on this particular choice. 

    By Lemma \ref{lem:mglecnv} and Skorokhod's theorem, on some filtered probability space, we may assume that, for each $T>0$,  
    \[
    \lim_{n\rightarrow \infty}\sup_{0\le t \le T} \norm{M_t^{\vep_n} - B_t}_2=0
    \]
    On this filtered probability space, by \eqref{eq:discstocinteg}, one can construct a copy of each $(X_k^{\vep_n},\; k =1,2,\ldots)$. 

    Our first claim is that the pair $(X_t^{\vep_n}, M_t^{\vep_n},\; t\ge 0)$ is jointly tight. Since the second coordinate has an almost sure limit, it suffices to argue the tightness of the first coordinate process itself.  

    In order to do this, we will first assume that the drift and the diffusion coefficients of \eqref{eq:diffSDE} are bounded and uniformly continuous. 
    
    First we compare $X_t^{\vep_n}$ with $\tilde{X}_t^{\vep_n}$, the piecewise constant interpolation of the discrete process 
    \begin{equation}\label{eq:discstocinteg2}
        \tilde{X}_{(k+1)\vep_n}^{\vep_n} = x_0 + \sum_{i=0}^k \tilde{b}_{i\vep_n}^{\vep_n}(\tilde{X}^{\vep_n}_{i\vep_n}) + \vep^{-1/2}\sum_{i=0}^k  \left(\tilde{\Sigma}^{\vep_n}_{i\vep_n}(X_{i\vep_n}^{\vep_n}) \right)^{1/2} \left( B_{(i+1)\vep_n}^{\vep_n} - B_{i\vep_n}^{\vep_n}\right),
    \end{equation}
    where $\tilde{b}_{i{\vep_n}}^{\vep_n}, \tilde{\Sigma}_{i\vep_n}^{\vep_n}$ are ${b}_{i\vep_n}^{\vep_n}, {\Sigma}_{i\vep_n}^{\vep_n}$, respectively, without the $o(\vep_n)$ error terms.  

    It can be shown that, for any $T>0$, 
    \[
    \lim_{n\rightarrow \infty}\sup_{0\le t \le T} \norm{X_t^{\vep_n} - \tilde{X}_t^{\vep_n}}=0.
    \]

    One the other hand, pathwise,  
    \[
    \tilde{X}_t^{\vep_n} = x_0 + \int_0^t \tilde{b}_{s}^{\vep_n}(\tilde{X}_u^{\vep_n})du + \int_0^t {\Sigma}_{u}^{\vep_n}(\tilde{X}_u^{\vep_n})dB_u. 
    \]
    Now, as $n\rightarrow \infty$, the convergence of $\tilde{X}^{\vep_n}$ to $X$, as in the solution of \eqref{eq:diffSDE} follows from the Dominated Convergence Theorem for stochastic integrals. 

    One can now relax the assumption of bounded, uniformly continuous coefficients via localization by stopping the process $X$ as it exits a sequence of balls around the origin of larger and larger radii. This completes the proof.  
\end{proof}


-----------------------------------------

% Some formal calculations

Some likely examples we can write  can be found in \cite[Section 8.4.2]{santambrogio2015optimal}.


A consequence of this Riemannian formalism is the following chain rule identity 
\begin{equation}\label{eq:rateofdecay}
\frac{d}{dt} F(\rho_t)= -\norm{\nabla_{\wass}^U F(\rho_t)}^2_U
\end{equation}
which shows that $t\mapsto F(\rho_t)$ decreases strictly along the mirror gradient flow until the first order condition is satisfied. For the explicit formula \eqref{eq:expgrad}, by chain rule we obtain
\begin{equation}\label{eq:rateofdecayexp}
\begin{split}
    \frac{d}{dt} F(\rho_t)&=- \int \iprod{\nabla_{x^u}\left(\frac{\delta F}{\delta \rho}\right), \nabla_{x}\left(\frac{\delta F}{\delta \rho}\right)} \rho(dx)\\
    &=- \int \iprod{\nabla_{x}\left(\frac{\delta F}{\delta \rho}\right), \nabla_x^2 u_t(x) \nabla_{x}\left(\frac{\delta F}{\delta \rho}\right)} \rho(dx).
 \end{split}   
\end{equation}
Thus one can compare the rate of decay of $F$ along this gradient flow to the usual gradient flow by checking how aligned the two vector fields $\nabla_{x^u}\left(\frac{\delta F}{\delta \rho}\right)$ and $\nabla_{x}\left(\frac{\delta F}{\delta \rho}\right)$ are in $\ltwo(\rho)$. 
\begin{comment}
\item (Integral bounds) Let $Z_{\vep,C}\sim N(0,\sqrt{\vep C} I_d)$ for $C>0$. Define the functions $\Theta_{C,\vep}:\R^d\to\R$ for $k\ge 1$ and $\vep>0$ as follows:
\begin{equation}\label{eq:tdef}
\Theta_{C,\vep}(y):=\E_{Z_{\vep,C}}\left[\left(\E_{|Z_{\vep,C}}\exp\left(-f((y+Z_{\vep,C})^{u_{k\vep}^*}+Z_{\vep,\frac{1}{C}})\right)\right)^{-1}\exp(-g(y+Z_{\vep,C}))\right].
\end{equation}
Assume that 
$$\tilde{\theta}_C:=\sup_{k\vep\le T}\E\left(\log{\Theta_{C,\vep}(Y)}\right)^2<\infty,$$
for some $\vep$ small enough.
\item (Moment generating function type integral bounds) For $\bm{j}:=(j_1,j_2,j_3,j_4,j_5)$ a $5$-tuple of reals, define the functions $\omega_{\bm{j},\vep}$ for $k\ge 1$ and $\vep>0$ as follows:
\begin{align}\label{eq:omdef}
\omega_{\bm{j},\vep}(x)&:=\E_{Z_{\vep,j_5}}\bigg[\bigg(\E_{|Z_{\vep,j_5}}\exp\bigg(-(j_1+j_2\lVert x\rVert)\lVert Z_{\vep,(j_5)^{-1}}\rVert-j_3 \lVert Z_{\vep,(j_5)^{-1}}\rVert^2\nonumber \\&-f((x^{u_{k\vep}}+Z_{\vep,j_5})^{u_{k\vep}^*}+Z_{\vep,(j_5)^{-1}})\bigg)\bigg)^{-1}\exp\big(-g(x^{u_{k\vep}}+Z_{\vep,j_5})\nonumber \\ &+(j_4)^{-1}(j_1+j_2\lVert x\rVert)\lVert Z_{\vep,j_5}\rVert+j_3(j_4)^{-2}\lVert Z_{\vep,j_5}\rVert^2\big)\bigg]
\end{align}
Assume that 
$$\tilde{\omega}_{\bm{j}}:=\sup_{k\vep\le T}\E\left(\log{\omega_{\bm{j},\vep}(Y^{u_{k\vep}^*})}\right)^2<\infty,$$
for some $\vep$ small enough.
\end{comment}

-------------------------

As a second application of our gradient flow formulation, we have the following perturbation result for Sinkhorn PDE.

\begin{prop}\label{prop:perturb}
    Let $(\rho_t,\; t\ge0)$ and $(\rho_t',\; t\ge 0)$ denote two solutions of the Sinkhorn PDE. Suppose $e^{-f}$ is log-concave with $\nabla^2f \ge \lambda I$, uniformly for some $\lambda >0$ and 
    \[
    \inf_{x\in\R^d}\lmn\left(\frac{\partial x\hfill}{\partial x^{u_t}}\right), \inf_{x\in\R^d}\lmn\left(\frac{\partial x\hfill}{\partial x^{u'_t}}\right)\ge h>0.
    \]
    Then 
    \[
    \wass_2(\rho_t, \rho_t') \le \dlt{}{e^{-g}}{\rho_t}{\rho_t'} \le \dlt{}{e^{-g}}{\rho_0}{\rho'_0}e^{-\lambda h t}, \quad \forall \; t\ge 0.
    \]
\end{prop}

\begin{proof}[Proof of Proposition \ref{prop:perturb}]
Recall the concept of generalized geodesic semiconvexity \cite[Definition 9.2.4]{ambrosio2005gradient}. To put everything in our context, let $\sigma, \sigma'$ be any two probability measures in the Wasserstein space. A generalized geodesic from $\sigma$ to $\sigma'$ is given by the curve 
\begin{equation}\label{eq:gengeosigma}
\sigma_t^{\sigma \rightarrow \sigma'}:=\left( (1-t)y^w + t y^{w'}\right)_{\#e^{-g}},
\end{equation}
where $w$ and $w'$ are the convex Brenier potentials transporting $e^{-g}$ to $\sigma$ and $\sigma'$, respectively. 


For $\lambda \in \rr$, a function $\phi$ is said to be $\lambda$-semiconvex on the generalized geodesics if, for all $\sigma, \sigma'$, 
\begin{equation}\label{eq:gengeocon}
\phi(\sigma_t^{\sigma \rightarrow \sigma'})\le (1-t) \phi(\sigma) + t \phi(\sigma') - \lambda t(1-t) \dlt{2}{e^{-g}}{\sigma}{\sigma'}.
\end{equation}
It is known, see \cite[Section 9.3]{ambrosio2005gradient} that if $\nabla^2 f\ge \lambda I$, uniformly, for some $\lambda >0$, then $\phi(\sigma):=\KL{\sigma}{e^{-f}}$ is $\lambda$-semiconvex on generalized geodesics. Since $\lambda>0$, this implies uniform convexity over generalized geodesics.

The proof follows closely the argument in \cite[Theorem 11.1.4]{ambrosio2005gradient} with the role of $\wass_2$ replaced by $\dlt{}{e^{-g}}{}{}$. 

For any $\sigma$ in the Wasserstein space, consider the function 
\[
\gamma_t(\rho, \sigma)= \frac{d}{dt}\dlt{2}{e^{-g}}{\rho_t}{\sigma},
\]
assuming it exists $t$ a.e.. Similarly, consider $\gamma_t(\rho', \sigma)$. 

Let $\omega(t):=\dlt{2}{e^{-g}}{\rho_t}{\rho_t'}$. Then, following the above cited proof, by \cite[Lemma 4.3.4]{ambrosio2005gradient}, a.e.,  
\begin{equation}\label{eq:dlotderiv}
\dot{\omega}(t)= \gamma_t(\rho, \rho'_t)+ \gamma_t(\rho', \rho_t). 
\end{equation}
In order to evaluate the above, note
\[
\gamma_t(\rho, \sigma)= \frac{d}{dt} \int \norm{y^{w_t} - y^v}^2 e^{-g(y)}dy,
\]
where $w_t$ and $v$ are the convex Brenier potentials transporting $e^{-g}$ to $\rho_t$ and $\sigma$, respectively. By our assumption $T_t:=\nabla u_t=(\nabla w_t)^{-1}$ satisfies the PMA $\dot{T}_t= - \nabla_{\wass}\phi(\rho_t)$.

Thus
\[
\begin{split}
\gamma_t(\rho, \sigma) &= 2\int (y^{w_t}- y^v) \cdot \frac{d}{dt} (T_t)^{-1}(y) e^{-g(y)}dy\\
&= 2\int \left(x - \nabla v \circ T_t(x)\right) \cdot \frac{d}{dt} (T_t)^{-1}(T_t(x)) \rho_t(x)dx, 
\end{split}
\]
by a change of variable $y=T_t(x)$. By the chain rule for inverse functions \eqref{eq:inverse_derivative}, $\frac{d}{dt} (T_t)^{-1}(T_t)=-\nabla^U_{\wass}\phi(\rho_t)$. Hence, 
\[
\gamma_t(\rho, \sigma)=-2\int \left(x - \nabla v \circ T_t(x)\right) \cdot \nabla^U_{\wass}\phi(\rho_t)(x) \rho_t(x)dx.
\]

Thus, if $u_t', T_t'$ denote the corresponding quantities for $(\rho_t',\; t\ge 0)$, then, continuing from \eqref{eq:dlotderiv}, 
\begin{equation}\label{eq:omegaprime}
\begin{split}    
\dot{\omega}(t)&= -2\int \left(x - \left( T'_t\right)^{-1} \circ T_t(x)\right) \cdot \nabla^U_{\wass}\phi(\rho_t)(x) \rho_t(x)dx\\
&-2\int \left(x - \left( T_t\right)^{-1} \circ T'_t(x)\right) \cdot \nabla^U_{\wass}\phi(\rho'_t)(x) \rho'_t(x)dx.
\end{split}
\end{equation}

On the other hand, consider the AC curve $(\sigma^{\sigma \rightarrow \sigma'}_t,\; 0\le t \le 1)$ from \eqref{eq:gengeosigma}. For a compactly supported smooth test function $\xi$, 
\[
\begin{split}
\frac{d}{dt}& \int \xi(x) \sigma_t(dx)= \frac{d}{dt}\int \xi\left( (1-t)y^w + t y^{w'}\right) e^{-g(y)}dy\\
&=\int \nabla \xi((1-t)y^w + t y^{w'}) \cdot (y^{w'} - y^w) e^{-g(y)}dy.
\end{split}
\]
At $t=0$, the above expression becomes
\[
\int \nabla \xi(y^w) \cdot (y^{w'} - y^w) e^{-g(y)}dy= \int \nabla \xi(x) \cdot ( \nabla w' \circ (\nabla w)^{-1}(x) -x) \sigma(dx).
\]
Hence the velocity of this AC curve at $t=0$ is precisely $\nabla w' \circ (\nabla w)^{-1}(x) -x$.

Apply the subdifferential calculus for the function $\phi$ \cite[Sections 10.3, 10.4]{ambrosio2005gradient} to \eqref{eq:gengeocon}. By taking $t\rightarrow 0+$, 
\[
\begin{split}
\int \left( (\nabla w') \circ (\nabla w)^{-1}(x) - x\right) \cdot \nabla_{\wass}\phi(\sigma)(x) d\sigma &=\lim_{t \rightarrow 0+}\frac{\phi(\sigma_t^{\sigma \rightarrow \sigma'})- \phi(\sigma)}{t}\\
&\le \phi(\sigma') - \phi(\sigma) - \lambda \dlt{2}{e^{-g}}{\sigma}{\sigma'}.
\end{split}
\]
 

Taking $(\sigma, \sigma')$ to be $(\rho_t, \rho_t')$ and again $(\rho_t', \rho_t)$, we get the pair of inequalities
\[
\begin{split}
\phi(\rho'_t) - \phi(\rho_t) - \lambda \dlt{2}{e^{-g}}{\rho_t}{\rho_t'} &\ge -\int \left( x - (T_t')^{-1}\circ T_t(x)\right)\cdot \nabla_{\wass} \phi(\rho_t) \rho_t(x)dx\\
\phi(\rho_t) - \phi(\rho'_t) - \lambda \dlt{2}{e^{-g}}{\rho_t}{\rho_t'} &\ge -\int \left( x - (T_t)^{-1}\circ T'_t(x)\right)\cdot \nabla_{\wass} \phi(\rho_t) \rho'_t(x)dx.
\end{split}
\]
Adding the two,
\[
\begin{split}
-2\lambda \dlt{2}{e^{-g}}{\rho_t}{\rho_t'} &\ge -\int \left( x - (T_t')^{-1}\circ T_t(x)\right)\cdot \nabla_{\wass} \phi(\rho_t) \rho_t(x)dx\\
&- \int \left( x - (T_t)^{-1}\circ T'_t(x)\right)\cdot \nabla_{\wass} \phi(\rho'_t) \rho'_t(x)dx.
\end{split}
\]
To compare the above with \eqref{eq:omegaprime} we need to use the lower bound on the eigenvalues. 

\end{proof}

Also,
    \begin{align*}
        \mathcal{R}_{k\vep}(y)&:=w_{k\vep}(y)+\frac{\vep d}{2}\log{(2\pi\vep)}-\vep f(y^{w_{k\vep}})+\frac{\vep}{2}\ldet(\nabla^2 w_{k\vep}(y))\\ &\qquad -\vep^2 M[u_{k\vep},f](y)-\vep^2 \sum_{j=0}^{k-1} \mathcal{M}[u_{j\vep}](y^{w_{k\vep}}),
    \end{align*}
    
    \begin{align*}
        \mathcal{R}_{k\vep}(y;x)&:=w_{k\vep}(y)+\frac{\vep d}{2}\log{(2\pi\vep)}-\vep f(y^{w_{k\vep}})+\frac{\vep}{2}\ldet(\nabla^2 w_{k\vep}(y))\\ &\qquad -\vep^2 M[u_{k\vep},f](x^{u_{k\vep}})-\vep^2 \sum_{j=0}^{k-1} \mathcal{M}[u_{j\vep}](x).
    \end{align*}
    We define the remainder terms:
    \begin{align*}
        a_{k\vep}:=\sup_{y}\frac{1}{\vep}\bigg|\opV[u_k^{\vep}](y)-\mathcal{R}_{k\vep}(y)\bigg|,
    \end{align*}
    and 
    \begin{align*}
        \tilde{a}_{k\vep}:=\sup_{(y,x):\ \lVert y-x^{u_{k\vep}}\rVert \le r_{\vep}}\frac{1}{\vep}\bigg|\opV[u_k^{\vep}](y)-\mathcal{R}_{k\vep}(y;x)\bigg|.
    \end{align*}
    The corresponding terms for approximating the $\opU$ operator are as follows:
    \begin{align*}
        \tilde{\mathcal{R}}_{k\vep}(x)&:=u_{k\vep}(x)+\vep^2 \sum_{j=0}^{k-1} \mathcal{M}[u_{j\vep}](x),
    \end{align*}

    %In the final display, we have used the change of measure formula \cref{lem:jacobian} with $\phi=w_t$, $b=h_t$ , and $a=g$, to note that 
%    $$h_t(y^{w_t})=g(y)+\ldet\left(\frac{\partial y\hfill}{\partial y^{w_t}}\right).$$
%    By carrying out a further change of measure according to \cref{lem:jacobian} with $y=x^{u_t}$ in the final integral above, we further observe:
   

   % \begin{align*}
   % \int \bigg\lVert \frac{\partial\hfill}{\partial y} (f(y^{w_t})-h_t(y^{w_t}))\bigg\rVert^2 e^{-g(y)}\,dy&=\int \bigg\lVert \frac{\partial\hfill}{\partial x^{u_t}}(f(x)-h_t(x))\bigg\rVert^2 e^{-h_t(x)}\,dx\\ &=\int \lVert v_t(x)\rVert^2 \rho_t(x)\,dx.
   % \end{align*}
    %The last equality uses \eqref{eq:velocity}. This completes the proof of \eqref{eq:linot}.
    
    \begin{align*}
        \tilde{\mathcal{R}}_{k\vep}(x;y)&:=u_{k\vep}(x)+\vep^2 \sum_{j=0}^{k-1} \mathcal{M}[u_{j\vep}](y^{w_{k\vep}}).
    \end{align*}
    We define the remainder terms:
    \begin{align*}
        b_{k\vep}:=\sup_{x}\frac{1}{\vep}\bigg|u_k^{\vep}(x)-\tilde{\mathcal{R}}_{k\vep}(x)\bigg|,
    \end{align*}
    and 
    \begin{align*}
        \tilde{b}_{k\vep}:=\sup_{(x,y):\ \lVert x-y^{w_{k\vep}}\rVert \le r_{\vep}}\frac{1}{\vep}\bigg|u_k^{\vep}(x)-\tilde{\mathcal{R}}_{k\vep}(x;y)\bigg|.
    \end{align*}

 Note that for $k\ge 0$, 

\begin{align*}
    \Lambda & :=\exp\left(\frac{1}{\vep}\opV[u_k^{\vep}](y)-\frac{1}{\vep}\mathcal{R}_{k\vep}(y)\right)\\ &=\int \exp\left(\frac{1}{\vep}\langle x,y\rangle-\frac{1}{\vep}u_k^{\vep}(x)-\frac{1}{\vep}\mathcal{R}_{k\vep}(y)-f(x)\right)\,dx \\& = \int_{B_{r_{\vep}}(y^{w_{k\vep}})} \exp\left(\frac{1}{\vep}\langle x,y\rangle-\frac{1}{\vep}\tilde{\mathcal{R}}_{k\vep}(x;y)-\frac{1}{\vep}\mathcal{R}_{k\vep}(y)-f(x)-\frac{1}{\vep}u_k^{\vep}(x)+\frac{1}{\vep}\tilde{\mathcal{R}}_{k\vep}(x;y)\right)\,dx\\ &+\int_{B^c_{r_{\vep}}(y^{w_{k\vep}})} \exp\left(\frac{1}{\vep}\langle x,y\rangle-\frac{1}{\vep}\tilde{\mathcal{R}}_{k\vep}(x)-\frac{1}{\vep}\mathcal{R}_{k\vep}(y)-f(x)-\frac{1}{\vep}u_k^{\vep}(x)+\frac{1}{\vep}\tilde{\mathcal{R}}_{k\vep}(x)\right)\\ &=: \Lambda_1+\Lambda_2.
\end{align*}

We begin with $\Lambda_1$. Observe that 
\begin{align*}
&\;\;\;\;\frac{1}{\vep}\langle x,y\rangle-\frac{1}{\vep}\tilde{\mathcal{R}}_{k\vep}(x)-\frac{1}{\vep}\mathcal{R}_{k\vep}(y)-f(x)-\frac{1}{\vep}u_k^{\vep}(x)+\frac{1}{\vep}\tilde{\mathcal{R}}_{k\vep}(x;y)\\ &=\frac{1}{\vep}\mathcal{D}[u_{k\vep}](x|y)-f(x)+f(y^{w_{k\vep}})-\frac{1}{2}\ldet(\nabla^2 w_{k\vep}(y))+\vep M[u_{k\vep},f](y)-\frac{d}{2}\log{(2\pi\vep)}\\ &\qquad -\frac{1}{\vep}u_k^{\vep}(x)+\frac{1}{\vep}\tilde{\mathcal{R}}_{k\vep}(x;y).
\end{align*}
Therefore, with 
$$\Lambda_1^{(1)}:=\exp\left(f(y^{w_{k\vep}})+\vep M[u_{k\vep},f](y)\right)\frac{\sqrt{\mathrm{det}(\nabla^2 u_{k\vep}(y^{w_{k\vep}}))}}{(2\pi\vep)^{d/2}} \int\limits_{B_{r_{\vep}}(y^{w_{k\vep}})} \exp\left(\frac{1}{\vep}\mathcal{D}[u_{k\vep}](x|y)-f(x)\right)\,dx,$$
we have:
$$\exp(-\tilde{b}_{k\vep})\le \frac{\Lambda_1}{\Lambda_1^{(1)}}\le \exp(\tilde{b}_{k\vep}).$$
By \cref{lem:prelimestim}, we have:
\begin{align*}
    \big|\Lambda_1^{(1)}-1\big|\le \eta_t \left(\vep \xi_t(r_{\vep})+\vep^{3/2}(\log{(1/\vep)})^{9/2}\right).
\end{align*}
We now move to $\Lambda_2$. By invoking \cref{lem:prelimestim}, we get:
$$\Lambda_2\le \exp(b_{k\vep})\eta_t \vep^{10}.$$
Combining the above observations, we get:
\begin{align*}
    \exp\left(\frac{1}{\vep}\opV[u_k^{\vep}](y)-\frac{1}{\vep}\mathcal{R}_{k\vep}(y)\right)\ge \exp(-\tilde{b}_{k\vep})\left(1-\eta_t \left(\vep \xi_t(r_{\vep})+\vep^{3/2}(\log{(1/\vep)})^{9/2}\right)\right).
\end{align*}
Similarly, we have:
\begin{align*}
    &\;\;\;\;\exp\left(\frac{1}{\vep}\opV[u_k^{\vep}](y)-\frac{1}{\vep}\mathcal{R}_{k\vep}(y)\right)\\ &\le \exp(\tilde{b}_{k\vep})\left(1+\eta_t \left(\vep \xi_t(r_{\vep})+\vep^{3/2}(\log{(1/\vep)})^{9/2}\right)\right)+\eta_t\vep^{10}\exp(b_{k\vep}).
\end{align*}
By combining the two observations above, for $\vep>0$ small enough (depending only on $\eta_t$, we get: 
$$a_{k\vep}\le \tilde{b}_{k\vep}+\eta_t \left(\vep \xi_t(r_{\vep})+\vep^{3/2}(\log{(1/\vep)})^{9/2}\right)+\eta_t \vep^{10}\exp(b_{k\vep}).$$

We now fix $y,z$ such that $\lVert y-z^{u_{k\vep}}\rVert\le r_{\vep}$ and note that:

\vspace{0.1in}

    \emph{$k=1$ case}: 
     and $C_{D,1}$ depends on the first $4$ derivatives of $u_0$, the first $2$ derivatives of $f$ and their  modulus of continuities. Also define

    $$\omega[u_0,f](\delta):=\sup_{\lVert z_1-z_2\rVert \le \delta} \big|M[u_0,f](z_1)-M[u_0,f](z_2)\big|.$$
    Using this observation, we will invoke \cref{lem:prelimestim} for each iteration of the Sinkhorn algorithm. By applying \cref{lem:prelimestim} with $u\equiv u_0$, $G\equiv f$, and $\tilde{G}\equiv 0$, we get:
    \begin{align*}
        \lVert\mathcal{R}^{\vep}_0-\opV[u_0]\rVert_{\infty}\le C_{0} \vep^2 \left(\omega[u_0,f](\sqrt{\vep \log{(1/\vep)}})+\sqrt{\vep}(\log{(1/\vep)})^{9/2}\right)=:E_{\vep,0},
    \end{align*}
    
    Note that 
    \begin{align*}
        &\;\;\;\;\exp\left(\frac{1}{\vep}u_1^{\vep}(x)-\frac{1}{\vep}\tilde{u}_{\vep}(x)-\vep M[u_0,f](x^{u_0})\right)\\ &=\frac{\sqrt{\mathrm{det}(\nabla^2 w_0(x^{u_0}))}}{(2\pi\vep)^{d/2}}\int_{B_{r_{\vep}}^c(x^{u_0})}\exp\bigg(\frac{1}{\vep}\mcD[w_0](y|x)+G_0(y)-G_0(x^{u_0})\nonumber \\ &\qquad \qquad +\vep(M[u_0,f](y)-M[u_0,f](x^{u_0}))-\frac{1}{\vep}(\opV[u_0](y)-\mathcal{R}^{\vep}_0(y))\bigg)\,dy\\ &+\frac{\sqrt{\mathrm{det}(\nabla^2 w_0(x^{u_0}))}}{(2\pi\vep)^{d/2}}\int_{B_{r_{\vep}}(x^{u_0})}\exp\bigg(\frac{1}{\vep}\mcD[w_0](y|x)+G_0(y)-G_0(x^{u_0})\nonumber \\ &\qquad \qquad -\frac{1}{\vep}(\opV[u_0](y)-\mathcal{R}^{\vep}_0(y;x))\bigg)\,dy
    \end{align*}
    By using \eqref{eq:gradbound4}, the first term above is bounded by 
    $$C_1 \exp\left(\vep C_1+\vep E_{0,\vep}+2\vep \lVert M[u_0,f]\rVert_{\infty}\right)\vep^{10}.$$
    For the second term, note that 
    \begin{align*}
        &\;\;\;\;\log\frac{\int_{B_{r_{\vep}}(x^{u_0})}\exp\bigg(\frac{1}{\vep}\mcD[w_0](y|x)+G_0(y)-G_0(x^{u_0})-\frac{1}{\vep}(\opV[u_0](y)-\mathcal{R}^{\vep}_0(y;x))\bigg)\,dy}{\int_{B_{r_{\vep}}(x^{u_0})}\exp\bigg(\frac{1}{\vep}\mcD[w_0](y|x)+G_0(y)-G_0(x^{u_0})\bigg)\,dy}\\ &\in \left(-\sup_{y\in B_{r_{\vep}}(x^{u_0})} \frac{1}{\vep}\big|\opV[u_0](y)-\mathcal{R}^{\vep}_0(y;x)\big|,\ \sup_{y\in B_{r_{\vep}}(x^{u_0})} \frac{1}{\vep}\big|\opV[u_0](y)-\mathcal{R}^{\vep}_0(y;x)\big|\right).
    \end{align*}
    By \cref{lem:prelimestim}, we then have:
    \begin{align*}
        &\;\;\;\;\bigg|\int_{B_{r_{\vep}}(x^{u_0})}\exp\bigg(\frac{1}{\vep}\mcD[w_0](y|x)+G_0(y)-G_0(x^{u_0})\bigg)\,dy-1+\vep M[w_0,G_0](x)\bigg|\\ &\le C_2\left(\vep^{3/2}(\log{(1/\vep)})^{9/2}+\vep \omega_2(r_{\vep})\right).
    \end{align*}
    Consequently, we get:
    $$\big|u_1^{\vep}(x)-\tilde{u}_{\vep}(x)-\vep^2 M[u_0,f](x^{u_0})+\vep^2 M[w_0,G_0](x)\big|\le .$$

Next recall the notation $A_T$ and $B_T$ from \cref{asn:solcon}. By \cref{lem:prelimestim}, we further note that there exists $c_{T,2}>0$ such that 
\begin{align*}
&\;\;\;\xi_{k+1}^{\vep}(x,y)\\ &\le c_{T,2}\le c_{T,2}\frac{B_T}{(2\pi\vep)^{d/2}}\exp\left(-\frac{1}{2\vep A_T}\lVert x-y^{w_{k\vep}}\rVert^2+\lVert x-y^{w_{k\vep}}\rVert_1 \lVert \nabla f\rVert_{\infty}-g(y)\right).
\end{align*}
Let us define $\lambda_T:=\exp(2c_{T,1})c_{T,2}B_T$. By combining the above observation with \eqref{eq:jointcon2}, we get:
\begin{align}\label{eq:jointcon3}
    &\;\;\;\wass_2^2(\gamma_{k+1}^{\vep},\tilde{\xi}_{k\vep})\nonumber \\ &\le \frac{\lambda_T}{(2\pi\vep)^{d/2}} \int \exp(-g(y))\left(\int \exp\left(-\frac{1}{2\vep A_T}\lVert x-y^{w_{k\vep}}\rVert^2+\lVert x-y^{w_{k\vep}}\rVert_1 \lVert \nabla f\rVert_{\infty}\right)\,dx\right)\,dy\nonumber \\ &\le \lambda_T A_T^{d/2}\int \exp(-g(y)) \E[\lVert Z_{\vep}\rVert^2\exp(\lVert Z_{\vep}\rVert_1 \lVert \nabla f\rVert_{\infty})]\,dy\nonumber \\ &=\lambda_T A_T^{d/2}\E[\lVert Z_{\vep}\rVert^2\exp(\lVert Z_{\vep}\rVert_1 \lVert \nabla f\rVert_{\infty})],
\end{align}
We write $Z_{\vep}=(Z_{\vep,1},\ldots , Z_{\vep,d})$. Observe that 

\begin{align}\label{eq:secbdcall}
    \E[\lVert Z_{\vep}\rVert^2\exp(\lVert Z_{\vep}\rVert_1 \lVert \nabla f\rVert_{\infty})]&=d \E[Z_{\vep,1}^2 \exp(|Z_{\vep,1}|\lVert \nabla f\rVert_{\infty})]\prod_{j\ge 2}\E[\exp(|Z_{\vep,j}|\lVert \nabla f\rVert_{\infty})]\nonumber\\ &\le 2^{d-1}d\E[Z_{\vep,1}^2 \exp(|Z_{\vep,1}|\lVert \nabla f\rVert_{\infty})]\exp((d-1)\lVert \nabla f\rVert_{\infty}^2 \vep A_T/2)\nonumber \\ &\le 2^{d+1}d \vep A_T \exp(d\lVert \nabla f\rVert_{\infty}^2 \vep A_T).
\end{align}
 
 %\begin{align*}
%&=\bigg\langle \frac{\partial}{\partial x^{u_t}}\frac{\partial u_t}{\partial t\hfill}(x),\frac{\partial h_t}{\partial x\hfill}(x)\bigg\rangle-\sum_{i,j=1}^d \frac{\partial}{\partial x_i}\left(\left(\frac{\partial x\hfill}{\partial x^{u_t}}\right)_{i,j}\left(\frac{\partial x^{u_t}}{\partial t\hfill}\right)_j\right)\nonumber \\ &=\bigg\langle \frac{\partial}{\partial x^{u_t}}\frac{\partial u_t}{\partial t\hfill}(x),\frac{\partial h_t}{\partial x\hfill}(x)\bigg\rangle-\sum_{i,j} \left(\frac{\partial x\hfill}{\partial x^{u_t}}\right)_{i,j}\frac{\partial}{\partial t}\left(\frac{\partial x^{u_t}}{\partial x\hfill}\right)_{i,j}-\sum_{j=1}^d \left(\sum_{i=1}^d\frac{\partial^2 x_j}{\partial x_i\partial x^{u_t}_i}\right)\left(\frac{\partial x^{u_t}}{\partial t\hfill}\right)_j\nonumber \\
%&=\bigg\langle \frac{\partial x^{u_t}}{\partial t\hfill}(x), \frac{\partial h_t}{\partial x^{u_t}\hfill}(x)+\frac{\partial}{\partial x^{u_t}}\ldet\left(\frac{\partial x^{u_t}}{\partial x\hfill}\right)\bigg\rangle-\sum_{i,j} \left(\frac{\partial x\hfill}{\partial x^{u_t}}\right)_{i,j}\frac{\partial}{\partial t}\left(\frac{\partial x^{u_t}}{\partial x\hfill}\right)_{i,j}\nonumber \\
%&=\bigg\langle \frac{\partial x^{u_t}}{\partial t\hfill}(x), \frac{\partial g}{\partial x}(x^{u_t})\bigg\rangle-\sum_{i,j} \left(\frac{\partial x\hfill}{\partial x^{u_t}}\right)_{i,j}\frac{\partial}{\partial t}\left(\frac{\partial x^{u_t}}{\partial x\hfill}\right)_{i,j}.
%\end{align*}
%Here, we have used \cref{lem:tensorel} in the penultimate display. The final display follows by invoking \cref{lem:jacobian} with $b=g$, $a=h_t$ and $\phi=u_t$. 

 Let $Y\sim\exp(-g)$ and set $X_t=Y^{w_t}$ (recall $w_t=u_t^*$). Observe that $\frac{\partial u_t}{\partial t\hfill}=f-h_t$. Then note that 
\begin{align*}
    \int_{\R^d} \lVert v_t(x)\rVert^2\exp(-h_t(x))\,dx&=\E\left[\bigg\lVert \left(\frac{\partial X_t\hfill}{\partial X_t^{u_t}}\right)\nabla \frac{\partial u_t}{\partial t\hfill}(X_t)\bigg\rVert^2\right]\\ &\le \E\left[\bigg\lVert \nabla\frac{\partial u_t}{\partial t\hfill}(Y^{u_t^*})\bigg\rVert^2\lmx^2\left(\frac{\partial Y^{u_t^*}}{\partial Y\hfill}\right)\right]<\infty.
\end{align*}
By combining the above observation with \eqref{eq:jointcon3}, we get: 
$$\wass_2^2(\gamma_{k+1}^{\vep},\tilde{\xi}_{k\vep})\le \left[\lambda_T A_T^{d/2} 2^{d+1}d \exp(d\lVert \nabla f\rVert_{\infty}^2 \vep A_T)\right]\vep A_T .$$
This completes the proof.
