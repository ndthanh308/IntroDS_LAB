\section{Proof of technical results}\label{sec:pfres}
In this section we will prove the auxiliary results from the main paper whose proofs we deferred, except for Lemmas \ref{lem:erboundmain} and \ref{lem:pmaboundmain} (which will be proved in the next section). 
\begin{proof}[Proof of~\cref{lem:tensorel}]
By~\eqref{eq:conjrel}, the following matrix relationship holds:
\begin{align*}
\left(\secphx\right)\left(\secph\right)=I_d.
\end{align*}
Therefore, by expanding the matrix product, for any $i,j\in [d]^2$, we get:
\begin{align}\label{eq:tensorel1}
\sum_{\ell} \secphxil \cdot \secphlj=\sum_{\ell} \left(\secphx\right)_{i,\ell}\left(\secph\right)_{\ell,j}=\delta_{i,j},
\end{align}
where $\delta_{i,j}=0$ if $i\neq j$ and $\delta_{i,i}=1$. 

\noindent Fix $k\in [d]$. By differentiating the LHS  of~\eqref{eq:tensorel1} with respect to $\xsph_k$, i.e., applying the operator $\frac{\partial}{\partial \xsph_{k}}$, we get:
\begin{align}\label{eq:tensorel2}
\frac{\partial}{\partial \xsph_{k}}\left(\sum_{\ell} \secphxil \cdot \secphlj\right)&=\sum_{\ell} \frac{\partial}{\partial \xsph_{k}} \left(\secphxil\right)\cdot \secphlj+\sum_{\ell} \secphxil \cdot \frac{\partial}{\partial \xsph_k} \left(\secphlj\right)\nonumber \\
&=\sum_{\ell,m} \frac{\partial}{\partial x_{m}}\left(\secphxil\right)\cdot\secphmk \cdot \secphlj+\sum_{\ell} \secphxil \cdot \frac{\partial}{\partial \xsph_k} \left(\secphlj\right)\nonumber \\&=\sum_{\ell,m} \frac{\partial}{\partial x_{\ell}}\left(\secphxim\right)\cdot\secphmk \cdot \secphlj+\sum_{\ell} \secphxil \cdot \frac{\partial}{\partial \xsph_k} \left(\secphjl\right).
\end{align}
Here the first equality uses the product rule of derivatives. The second equality uses the chain rule of derivatives on the first term (the second term is intact here). Finally the third display follows by noting that $\frac{\partial}{\partial x_{m}}\left(\secphxil\right)=\frac{\partial}{\partial x_{\ell}}\left(\secphxim\right)$ and $\frac{\partial x}{\partial x^{\phi}}$ is a symmetric matrix (both of which follow from the $\diffcont^2$ diffeomorphism assumption on $\nabla \phi$). Next note that the derivative of the right hand side of \eqref{eq:tensorel1} with respect to $x_k^{\phi}$ is $0$. With this observation, by  combining~\eqref{eq:tensorel1}~and~\eqref{eq:tensorel2}, we get:
\begin{align*}
\sum_{\ell} \secphxil \cdot \frac{\partial}{\partial \xsph_k} \left(\secphjl\right)=-\sum_{\ell,m} \frac{\partial}{\partial x_{\ell}}\left(\secphxim\right)\cdot\secphmk \cdot \secphlj.
\end{align*}
By choosing $k=i$, summing up over $i$, we get:
\begin{align}\label{eq:tenso1}
\sum_{\ell} \frac{\partial}{\partial x_{\ell}}\left(\secphjl\right)=-\sum_{i,\ell,m} \frac{\partial}{\partial x_{\ell}}\left(\secphxim\right)\cdot\secphmi \cdot \secphlj.
\end{align}

\noindent Note that the left hand side of \eqref{eq:tenso1} is same as the right hand side of \eqref{eq:tensorelpf}. Let us now show that the right hand side of \eqref{eq:tenso1} matches the left hand side of \eqref{eq:tensorelpf}.

\noindent To achieve this, note that for a symmetric positive definite matrix $A$, by \cite[Section A.4.1]{boyd2004convex}, we have $\nabla_A \ldet(A)= A^{-1}$. Here the gradient is computed entry-wise. By using the above observation with $A=\frac{\partial x^{\phi}}{\partial x\hfill}$ (so $A^{-1}=\frac{\partial x\hfill}{\partial x^{\phi}}$), we get
\begin{align*}
\frac{\partial}{\partial \xsph_j} \ldet \left(\secphx\right)&= \sum_{i,m} \frac{\partial}{\partial \xsph_j} \left( \secphxim \right) \secphmi &= \sum_{i,\ell,m} \frac{\partial}{\partial x_{\ell}}\left(\secphxim\right)\cdot\secphlj\cdot\secphmi .
\end{align*}
The last equality above follows using the chain rule of derivatives. Coupling the above observation with \eqref{eq:tenso1}  completes the proof.
\end{proof}

\begin{proof}[Proof of \cref{lem:dualPMA}]
As $w_t=u_t^*$ with $u_t$ being the solution from \eqref{eq:pma}, given $y\in\R^d$, the following holds:
$$w_t(y)+u_t(y^{w_t})-\langle y, y^{w_t}\rangle=0.$$
By taking partial derivatives with respect to $t$ on both sides above, we further have:
\begin{align}\label{eq:dualPM1}
    \frac{\partial w_t}{\partial t\hfill}(y)+\frac{\partial}{\partial t}(u_t(y^{w_t}))-\iprod{y,\frac{\partial y^{w_t}}{\partial t\hfill}}=0
\end{align}
The second term in the left hand side of \eqref{eq:dualPM1} needs further simplification. To wit, by using the chain rule of derivatives, we get:
$$\frac{\partial}{\partial t}(u_t(y^{w_t}))=\frac{\partial}{\partial t}u_t(y^{w_t})+\iprod{(y^{w_t})^{u_t},\frac{\partial y^{w_t}}{\partial t\hfill}}=\frac{\partial}{\partial t}u_t(y^{w_t})+\iprod{y,\frac{\partial y^{w_t}}{\partial t\hfill}}.$$
In the last equality above, we have used the observation that $\nabla u_t(\nabla w_t(y))=y$. By coupling the above observation with \eqref{eq:dualPM1} we get
$$\frac{\partial w_t}{\partial t\hfill}(y)=-\frac{\partial}{\partial t}u_t(y^{w_t})=g((y^{w_t})^{u_t})-f(y^{w_t})-\ldet\left(\frac{\partial x^{u_t}}{\partial x\hfill}\right)\bigg|_{x=y^{w_t}}.$$
In the last equality we have used the PMA \eqref{eq:pma}. The conclusion of the lemma now follows by combining the above display with the following observations:
$$(y^{w_t})^{u_t}=y,\quad\quad \ldet\left(\frac{\partial x^{u_t}}{\partial x\hfill}\right)\bigg|_{x=y^{w_t}}=-\ldet\left(\frac{\partial y^{w_t}}{\partial y\hfill}\right).$$
\end{proof}

\begin{proof}[Proof of  \cref{lem:convexcall}]
By \eqref{eq:labelgrad}, $\nabla w_t(y)+\delta v_t(y^{w_t})$ is the gradient of the following function with respect to $y$
$$\Lambda_t(y):=w_t(y)+\delta \left(g(y)-f(y^{w_t})+\ldet\left(\frac{\partial y^{w_t}}{\partial y\hfill}\right)\right).$$
It remains to prove that $y\mapsto \Lambda_t(y)$ is convex. Towards this direction, we first assume the following claim and complete the proof.
\begin{align}\label{eq:eigbound}
    \sup_y \lVert \nabla^2 \Lambda_t(y)\rVert_{\mathrm{op}}<\infty,
\end{align}
where $\lVert \cdot \rVert_{\mathrm{op}}$ denotes the $L^2$ operator norm of a matrix (see~\cref{tab:table3}). Recall from \cref{asn:solcon}, part (i) that $\inf_y \lmn\left(\frac{\partial y^{w_t}}{\partial y\hfill}\right)>0$. Therefore there exists $\delta>0$ such that $$\delta < \ \frac{\inf_y \lmn\left(\frac{\partial y^{w_t}}{\partial y\hfill}\right)}{2\sup_y \lVert \nabla^2 \Lambda_t(y)\rVert_{\mathrm{op}}}.$$ 
By using Weyl's inequality (see \cite[Theorem 3.3.16]{horn1994topics}), we then get:
$$\inf_y \lmn(\Lambda_t(y))\ge \inf_y \lmn\left(\frac{\partial y^{w_t}}{\partial y\hfill}\right)-\delta \sup_y\lVert \Lambda_t(y)\rVert_{\mathrm{op}}>\frac{1}{2}\inf_y \lmn\left(\frac{\partial y^{w_t}}{\partial y\hfill}\right)>0.$$

This establishes the convexity of $y\mapsto \Lambda_t(y)$ and  completes the proof of \cref{lem:convexcall}. Therefore, it only remains to prove \eqref{eq:eigbound}. 

\emph{Proof of \eqref{eq:eigbound}.} By \cref{asn:solcon}, parts (i) and (iii), $\sup_y \lVert w_t(y)\rVert_{\mathrm{op}}<\infty$ and $\sup_y \lVert \nabla^2 g(y)\rVert_{\mathrm{op}}<\infty$. Similarly for $i,j\in [d]^2$, we get 
$$\frac{\partial^2}{\partial y_i\partial y_j}(f(y^{w_t}))=\sum_{\ell}\frac{\partial^3}{\partial y_i \partial y_j \partial y_{\ell}} w_t(y) \frac{\partial}{\partial y_{\ell}} f(y^{w_t})+\sum_{\ell,m}\left(\frac{\partial y_{\ell}^{w_t}}{\partial y_j\hfill}\right)\left(\frac{\partial y_m^{w_t}}{\partial y_i\hfill}\right)\frac{\partial^2}{\partial y_m\partial y_{\ell}}f(y^{w_t}).$$
Here we have used a combination of the product rule and the chain rule. Similar calculations were done in the proofs of \cref{thm:existlin} in the main paper, and so we skip the details for brevity. By the uniform boundedness of the first two derivatives of $f$ and the first $3$ derivatives of $w_t$ from \cref{asn:solcon}, part (iii) (also see \cref{rem:dualasn}), we immediately get:
$$\max_{i,j\in [d]^2}\sup_{y}\bigg|\frac{\partial^2}{\partial y_i\partial y_j}(f(y^{w_t}))\bigg|<\infty.$$
A similar computation also yields
\begin{align*}
    \frac{\partial^2}{\partial y_i \partial y_j}\left(\ldet\left(\frac{\partial y^{w_t}}{\partial y\hfill}\right)\right)&=\sum_{k,\ell}\left(\frac{\partial y_k\hfill}{\partial y_{\ell}^{w_t}}\right)\frac{\partial^4}{\partial y_i\partial y_j\partial y_k\partial y_{\ell}} w_t(y)\\ &\;\;\;+\sum_{k,\ell,m}\frac{\partial^3}{\partial y_k\partial y_{\ell}\partial y_m}u_t(y^{w_t})\frac{\partial^3}{\partial y_k\partial y_{\ell}\partial y_j}w_t(y)\left(\frac{\partial y_m^{w_t}}{\partial y_i}\right).
\end{align*}
By the uniform boundedness of the first two derivatives of $f$ and the first $4$ derivatives of $w_t$ from \cref{asn:solcon}, part (iii) (also see \cref{rem:dualasn}), we immediately get:
$$\max_{i,j\in [d]^2}\sup_y \Bigg|\frac{\partial^2}{\partial y_i \partial y_j}\left(\ldet\left(\frac{\partial y^{w_t}}{\partial y\hfill}\right)\right)\Bigg|<\infty.$$
Combining these observations, \eqref{eq:eigbound} follows.
\end{proof}

%\begin{proof}[Proof of \cref{cl:normalize}]
%Note that 
%\begin{align}\label{eq:claim11}
%\rho[u](x)&=\int \exp\left(-\frac{1}{\vep}u(x)-\frac{1}{2\vep}\norm{ x-y}^2\right)\exp\left(-\frac{1}{\vep}\opV[u](y)\right)d\nu(dy)\nonumber \\ 
%&=\int_y \frac{\exp(-(2\vep)^{-1}\norm{ x-y}^2 -(\vep)^{-1}u(x))}{\int_z \exp(-(2\vep)^{-1}\lVert y-z\rVert^2-(\vep)^{-1}u(z))\, d\mu(z)}\,d\nu(y).
%\end{align}
%By Tonelli's Theorem, interchanging the order of integration, we get:
%\begin{align*}
%\int \rho[u](x)\, d\mu(x)&=\int_y \frac{\int_x\exp(-(2\vep)^{-1}\lVert x-y\rVert^2-u(x))\, d\mu(x)}{\int_z \exp(-(2\vep)^{-1}\lVert y-z\rVert^2/2-u(z))\, d\mu(z)} d\nu(y)=\int_y  d\nu(y)=1.
%\end{align*}
%This completes the proof of the Proposition.
%\end{proof}

\section{Proof of Lemmas \ref{lem:erboundmain} and \ref{lem:pmaboundmain}}\label{sec:mainresultlems}
Recall the definitions of $\opU$ and $\opV$ from \eqref{eq:basedef}. 
\cref{lem:pmaboundmain} is a Laplace approximation applied to the integral operator $\opV$. A similar estimate can be found in \cite[Lemma 4.2]{berman2020} under different assumptions. On the other hand, \cref{lem:erboundmain} is a triangular approximation argument which is motivated from the proof of \cite[Lemma 4.4]{berman2020}. We will actually prove Lemmas \ref{lem:erboundmain} and \ref{lem:pmaboundmain} in the reverse order. We need to set up some notation first. For $k\ge 0$, consider the Bregman divergence from \eqref{eq:bregdiv} for the convex function $u_{k\vep}(\cdot)$ and $x,y\in\R^d$, 
\[
\mcD[u_{k\vep}](x|y) = u_{k\vep}(x) + u_{k\vep}^*(y) - \langle x,y\rangle.
\]
Note that $\mcD[u_{k\vep}^*](y|x)=\mcD[u_{k\vep}](x|y)$. Also given a sufficiently smooth function $h:\R^d\to\R$ and any $r\in \mathbb{N}$, define 
    \begin{align}\label{eq:taylornot}
        T[h:r](x|y):=\sum_{|\alpha|=r}\frac{D^{\alpha} h(y)}{\alpha!}(x-y)^{\alpha},
    \end{align}
    and 
    $$R[h:r](x|y):=\sum_{|\beta|=r}\frac{r}{\beta!}\left(\int_0^1 (1-t)^{r-1} D^{\beta} h(y+t(x-y))\,dt\right)(x-y)^{\beta}-T[h:r](x|y).$$
    Here, the $D^{\alpha}$ (or $D^{\beta}$) operators denote the standard multivariate differential operators given a nonnegative multi-index $\alpha=(\alpha_1,\ldots ,\alpha_d)$ (respectively $\beta=(\beta_1,\ldots ,\beta_d)$). Also $|\alpha|=\sum_{i=1}^d \alpha_i$, $|\beta|=\sum_{i=1}^d \beta_i$ and, as usual, $(x-y)^\alpha=\prod_{i=1}^d (x_i-y_i)^{\alpha_i}$ and so on.
    
    In other words, $T[h:r](x|y)$ denotes the $r$-th polynomial in the Taylor series expansion of $h$ at the point $x$ around the point $y$. On the other hand, $R[h:r](x|y)$ denotes the corresponding remainder term after a Taylor expansion of the function $h$ up to the $r$-th order term (at the point $x$  around the point $y$). 

    
    Finally, we note the following canonical estimates:
    \begin{equation}\label{eq:gradbound3}
    |T[h:r](x|y)|\le  C_r\lVert \nabla^r h\rVert_{\infty}\lVert x-y\rVert^{r},
    \end{equation}
    and
    \begin{equation}\label{eq:gbd4}
    |R[h:r](x|y)|\le C_r \lVert x-y\rVert^r \sup_{\lVert z_1-z_2\lVert \le \lVert x-y\rVert}\lVert \nabla^r h(z_1)-\nabla^r h(z_2)\rVert_{\infty},
    \end{equation}
    where $C_r>1$ is a universal constant (depending on $d$) that is free of $h$, $x$, and $y$. Here $\lVert \nabla^r h\rVert_{\infty}$ denotes the maximum of the supremum norms of every component function in the $r$-th order multiderivative. 


\begin{proof}[Proof of \cref{lem:pmaboundmain}]
Throughout the proof we always assume $k\vep\le T$. Further, $C$ will denote constants (possibly different in various steps) depending only on $d$ and all the other constants implicit in \cref{asn:solcon}. 

Let $x,y\in\R^d$. Recall from \eqref{eq:estimpf1} that 
$\mathcal{D}[u_{k\vep}](x|y)\ge \frac{A_T}{2}\lVert x-y^{w_{k\vep}}\rVert^2$, for some positive constant $A_T$. We are now in a position to simplify $\opV$. 
    
    \vspace{0.1in}
    
    By an algebraic identity using the definition of $\opV$ from \eqref{eq:basedef}, we observe that 
    \begin{align}\label{eq:simpl1}
       &\;\;\;\;\frac{\sqrt{\mathrm{det}(\nabla^2 u_{k\vep}(y^{w_{k\vep}}))}}{(2\pi\vep)^{\frac{d}{2}}}\exp\left(\frac{1}{\vep}\opV[u_{k\vep}](y)-\frac{1}{\vep}w_{k\vep}(y)+f(y^{w_{k\vep}})\right)\nonumber \\ &=\frac{\sqrt{\mathrm{det}(\nabla^2 u_{k\vep}(y^{w_{k\vep}}))}}{(2\pi\vep)^{\frac{d}{2}}}\int \exp\bigg(-\frac{1}{\vep}\mcD[u_{k\vep}](x|y)-f(x)+f(y^{w_{k\vep}})\bigg)\,dx.
    \end{align}
    Define 
    $$r_{\vep}:=\sqrt{-40d A_T^{-1} \vep\log{\vep}},\qquad \mbox{for}\ \vep\in (0,1/2).$$
    We now split the integral in \eqref{eq:simpl1} into two complementary domains: the first one is an integral over $B_{r_{\vep}}(y^{w_{k\vep}})$ and the second one is over $B^c_{r_{\vep}}(y^{w_{k\vep}})$. We will show that 
    \begin{equation}\label{eq:showsmall}
        \sup_y \frac{\sqrt{\mathrm{det}(\nabla^2 u_{k\vep}(y^{w_{k\vep}}))}}{(2\pi\vep)^{\frac{d}{2}}}\int\limits_{B^c_{r_{\vep}}(y^{w_{k\vep}})} \exp\bigg(-\frac{1}{\vep}\mcD[u_{k\vep}](x|y)-f(x)+f(y^{w_{k\vep}})\bigg)\,dx \le C \vep^{10},
    \end{equation}
    and 
    \begin{equation}\label{eq:showlarge}
        \sup_y \bigg|\frac{\sqrt{\mathrm{det}(\nabla^2 u_{k\vep}(y^{w_{k\vep}}))}}{(2\pi\vep)^{\frac{d}{2}}}\int\limits_{B_{r_{\vep}}(y^{w_{k\vep}})} \exp\bigg(-\frac{1}{\vep}\mcD[u_{k\vep}](x|y)-f(x)+f(y^{w_{k\vep}})\bigg)\,dx - 1\bigg|\le C \vep.
    \end{equation}
    Let us complete the proof by assuming \eqref{eq:showsmall} and \eqref{eq:showlarge} first. To wit, by combining \eqref{eq:showsmall} and \eqref{eq:showlarge}, with \eqref{eq:simpl1}, we get:
    $$\sup_y \bigg|\frac{\sqrt{\mathrm{det}(\nabla^2 u_{k\vep}(y^{w_{k\vep}}))}}{(2\pi\vep)^{\frac{d}{2}}}\exp\left(\frac{1}{\vep}\opV[u_{k\vep}](y)-\frac{1}{\vep}w_{k\vep}(y)+f(y^{w_{k\vep}})\right)-1\bigg|\le C \vep.$$
    The conclusion of \cref{lem:pmaboundmain} then follows by using the elementary inequality $|\log{(1+x)}|\le 2|x|$ for $|x|\le 1/2$. It only remains to show \eqref{eq:showsmall} and \eqref{eq:showlarge}.
    
    \vspace{0.1in}
    
    \emph{Proof of \eqref{eq:showsmall}.} We note the following sequence of displays with line-by-line explanations to follow:
    \begin{align}\label{eq:gradbound4}
        &\;\;\;\;\frac{\sqrt{\mathrm{det}(\nabla^2 u_{k\vep}(y^{w_{k\vep}}))}}{(2\pi\vep)^{\frac{d}{2}}}\int\limits_{B^c_{r_{\vep}}(y^{w_{k\vep}})} \exp\bigg(-\frac{1}{\vep}\mcD[u](x|y)-f(x)+f(y^{w_{k\vep}})\bigg)\,dx\nonumber \\ & \le \frac{C}{(2\pi\vep)^{\frac{d}{2}}}\int\limits_{B^c_{r_{\vep}}(y^{w_{k\vep}})} \exp\bigg(-\frac{A_T}{2\vep }\lVert x-y^{w_{k\vep}}\rVert^2-f(x)+f(y^{w_{k\vep}})\bigg)\,dx\nonumber \\ &= C\E\left[\exp\left(f(y^{w_{k\vep}})-f\left(y^{w_{k\vep}}+\sqrt{\vep A_T^{-1}} Z\right)\right)\mathbf{1}\left(\sqrt{\vep A_T^{-1}}Z\in B_{r_{\vep}}^c(0)\right)\right]\nonumber \\ &\le C  \E\left[\exp\left(\sqrt{\vep A_T^{-1}}\lVert \nabla f\rVert_{\infty}\sum_{i=1}^d |Z_i|\right)\mathbf{1}\left(\sqrt{\vep A_T^{-1}}Z\in B_{r_{\vep}}^c(0)\right)\right]
    \end{align}
    Above, in the first inequality, we have used  \eqref{eq:estimpf1} and \cref{asn:solcon}, part (i) to bound $\sqrt{\mathrm{det}(\nabla^2 u_{k\vep}(y^{w_{k\vep}}))}$ from above. In the following equality, we have adjusted some constants to rewrite the integral in terms of an expectation of a standard multivariate Gaussian random variable $Z$. The next inequality follows from the elementary observation that 
    \begin{align*}
    \left|f\left(y^{w_{k\vep}}+\sqrt{\vep A_T^{-1}}Z\right)-f(y^{w_{k\vep}})\right|&=\int_0^1 \left\langle \nabla f\left(y^{w_{k\vep}}+t\sqrt{\vep A_T^{-1}}Z\right),\sqrt{\vep A_T^{-1}}Z\right\rangle\,dt\\ &\le \sqrt{\vep A_T^{-1}}\lVert \nabla f\rVert_{\infty}\sum_{i=1}^d |Z_i|.
    \end{align*}
    Next, we apply the Cauchy-Schwartz inequality in \eqref{eq:gradbound4} to get:
    \begin{align*}
    &\;\;\;\;\frac{\sqrt{\mathrm{det}(\nabla^2 u_{k\vep}(y^{w_{k\vep}}))}}{(2\pi\vep)^{\frac{d}{2}}}\int\limits_{B^c_{r_{\vep}}(y^{w_{k\vep}})} \exp\bigg(-\frac{1}{\vep}\mcD[u](x|y)-f(x)+f(y^{w_{k\vep}})\bigg)\,dx\\ &\le
    C \sqrt{\E\exp\left(\sqrt{\vep A_T^{-1}}\lVert \nabla f\rVert_{\infty} \sum_{i=1}^d |Z_i|\right)}\sqrt{\Pr(\sqrt{\vep A_T^{-1}}\lVert Z\rVert\ge r_{\vep})}\\ &\le C \sqrt{\left(\E\exp\left(\sqrt{\vep A_T^{-1}}\lVert \nabla f\rVert_{\infty}  |Z_1|\right)\right)^d}\sqrt{d\Pr\left(|Z_1|\ge r_{\vep}/\sqrt{\vep d A_T^{-1}}\right)}\le C \vep^{10}.
    \end{align*}
    To understand the inequalities in the last line above, note that by using standard Gaussian tail bounds, we have 
    $$\E\exp\left(\sqrt{\vep A_T^{-1}}\lVert \nabla f\rVert_{\infty}  |Z_1|\right)\le \E\exp\left(\sqrt{A_T^{-1}}\lVert \nabla f\rVert_{\infty}|Z_1|\right)\le C,$$
    as $\vep\le 1$. Moreover, by the union bound,
     \begin{align*}
        \Pr&(\sqrt{\vep A_T^{-1}}\lVert Z\rVert\ge r_{\vep})\le \sum_{i=1}^d \Pr\left(|Z_i|\ge r_{\vep}/\sqrt{\vep d A_T^{-1}}\right)=d \Pr\left(|Z_1|\ge r_{\vep}/\sqrt{\vep d A_T^{-1}}\right)\\ &\le 2d \exp(-r_{\vep}^2/(2\vep d A_T^{-1}))=2d \exp(20 \log{\vep})=2d \vep^{20}.
    \end{align*}
\vspace{0.1in}

   \emph{Proof of \eqref{eq:showlarge}.} 
    Set $$\tilde{Z}^{(\vep)}_{w_{k\vep},y}\sim y^{w_{k\vep}}+\sqrt{\vep}Z_{w_{k\vep},y}, \quad \mbox{where} \quad Z_{w_{k\vep},y}\sim N(0,\nabla^2 w_{k\vep}(y)).$$
    By a third order Taylor series expansion, for any $k\ge 0$:
    \begin{small}
    \begin{align}\label{eq:taylor1}
        &\mcD[u_{k\vep}](x|y)=\frac{1}{2}(x-y^{w_{k\vep}})^{\top}\nabla^2 u_{k\vep}(y^{w_{k\vep}})(x-y^{w_{k\vep}}) + T[u_{k\vep}:3](x|y^{w_{k\vep}}) + R[u_{k\vep}:3](x|y^{w_{k\vep}}).
    \end{align}
    \end{small}
    Also, by a first order Taylor expansion to the function $f$:
    \begin{align}\label{eq:taylor2}
       f(x)&=f(y^{w_{k\vep}})+T[f:1](x|y^{w_{k\vep}})+R[f:1](x|y^{w_{k\vep}}).
    \end{align}
    By \eqref{eq:taylor1} and \eqref{eq:taylor2},
    \begin{align}\label{eq:largge}
        &\;\;\;\;\frac{\sqrt{\mathrm{det}(\nabla^2 u_{k\vep}(y^{w_{k\vep}}))}}{(2\pi\vep)^{\frac{d}{2}}}\int\limits_{B_{r_{\vep}}(y^{w_{k\vep}})} \exp\left(-\frac{1}{\vep}\mcD[u_{k\vep}](x|y)-f(x)+f(y^{w_{k\vep}})\right)\,dx \nonumber \\ &=\frac{\sqrt{\mathrm{det}(\nabla^2 u_{k\vep}(y^{w_{k\vep}}))}}{(2\pi\vep)^{\frac{d}{2}}}\int\limits_{B_{r_{\vep}}(y^{w_{k\vep}})} \exp\bigg(-\frac{1}{2\vep}(x-y^{w_{k\vep}})^{\top}\nabla^2 u_{k\vep}(y^{w_{k\vep}})(x-y^{w_{k\vep}})\bigg)\times \nonumber \\ 
        &  \exp\left(-\frac{1}{\vep}T[u_{k\vep}:3](x|y^{w_{k\vep}})- \frac{1}{\vep} R[u_{k\vep}:3](x|y^{w_{k\vep}})-T[f:1](x|y^{w_{k\vep}})-R[f:1](x|y^{w_{k\vep}})\right)\,dx.
    \end{align}

    Now in order to prove \eqref{eq:showlarge}, it suffices to show that the final equality above is $1+O(\vep)$. Let us sketch the rest of the argument first. Note that in the second line above, we have the density of a $N(y^{w_{k\vep}},\vep\nabla^2 w_{k\vep}(y))$ random variable, i.e., $\tilde{Z}^{(\vep)}_{w_{k\vep},y}$ declared above. This enables us to rewrite the above integral as a Gaussian integral as follows:

    \begin{align}\label{eq:large1}
    &\E\bigg[\exp\bigg(-\frac{1}{\vep}T[u_{k\vep}:3](\tilde{Z}^{(\vep)}_{w_{k\vep},y}|y^{w_{k\vep}})- \frac{1}{\vep} R[u_{k\vep}:3](\tilde{Z}^{(\vep)}_{w_{k\vep},y}|y^{w_{k\vep}})-T[f:1](\tilde{Z}^{(\vep)}_{w_{k\vep},y}|y^{w_{k\vep}})\nonumber \\ &-R[f:1](\tilde{Z}^{(\vep)}_{w_{k\vep},y}|y^{w_{k\vep}})\bigg)\mathbf{1}(\sqrt{\vep}Z_{w_{k\vep},y}\in B_{r_{\vep}}(0))\bigg].
    \end{align}

    Next we will use two simple approximations of the exponential function as follows: for $|z|\le M$, 
    \begin{align}\label{eq:taylorapp}
        \bigg|\exp(z)-1-z\bigg|\le \frac{z^2}{2}\exp(M),\qquad \mbox{and} \qquad \bigg|\exp(z)-1\bigg|\le |z|\exp(M).
    \end{align}
    The idea is to use \eqref{eq:taylorapp} to approximate the exponential terms in \eqref{eq:large1}. In particular, the terms in \eqref{eq:large1} featuring the Taylor polynomials in the exponent, namely $\exp\big(-\frac{1}{\vep}T[u_{k\vep}:3](\tilde{Z}^{(\vep)}_{w_{k\vep},y}|y^{w_{k\vep}})\big)$ and $\exp\big(-T[f:1](\tilde{Z}^{(\vep)}_{w_{k\vep},y}|y^{w_{k\vep}})\big)$, will be approximated using the first inequality in \eqref{eq:taylorapp}. The remainder terms $\exp\big(\frac{1}{\vep} R[u_{k\vep}:3](\tilde{Z}^{(\vep)}_{w_{k\vep},y}|y^{w_{k\vep}})\big)$ and $\exp\big(-R[f:1](\tilde{Z}^{(\vep)}_{w_{k\vep},y}|y^{w_{k\vep}})\big)$ on the other hand, will be approximated using the second inequality in \eqref{eq:taylorapp}. The role of $M$ in \eqref{eq:taylorapp} will be played by an appropriate function of $r_{\vep}$ thanks to the indicator term in \eqref{eq:large1}. Let us illustrate the above description concretely using one of the aforementioned terms. To wit, note that by \eqref{eq:gradbound3}, we have:
    \begin{align}\label{eq:large2}
    &\;\;\;\;\bigg|\frac{1}{\vep}T[u_{k\vep}:3](\tilde{Z}^{(\vep)}_{w_{k\vep},y}|y^{w_{k\vep}})\mathbf{1}(\sqrt{\vep}Z_{w_{k\vep},y}\in B_{r_{\vep}}(0))\bigg|\nonumber \\ &\le \frac{C}{\vep} \lVert \sqrt{\vep}Z_{w_{k\vep},y}\rVert^3 \mathbf{1}(\sqrt{\vep}Z_{w_{k\vep},y}\in B_{r_{\vep}}(0))\le \frac{C}{\vep} r_{\vep}^3 = C\sqrt{\vep}\left(\log{\left(\frac{1}{\vep}\right)}\right)^{3/2}.
    \end{align}
    Define
    \begin{align*}
    \vartheta^{(1)}_{\vep}(\tilde{Z}^{(\vep)}_{w_{k\vep},y}|y^{w_{k\vep}})&:=\exp\left(-\frac{1}{\vep}T[u_{k\vep}:3](\tilde{Z}^{(\vep)}_{w_{k\vep},y}|y^{w_{k\vep}})\right)-1+\frac{1}{\vep}T[u_{k\vep}:3](\tilde{Z}^{(\vep)}_{w_{k\vep},y}|y^{w_{k\vep}}).
    \end{align*}
    By invoking the first inequality in \eqref{eq:taylorapp} with $M=C\sqrt{\vep}\left(\log{\left(\frac{1}{\vep}\right)}\right)^{3/2}$ as in \eqref{eq:large2}, we get:
    \begin{align}\label{eq:large11}
    &\;\;\;\;\sup_y \big|\vartheta^{(1)}_{\vep}(\tilde{Z}^{(\vep)}_{w_{k\vep},y}|y^{w_{k\vep}})\big|\mathbf{1}(\sqrt{\vep}Z_{w_{k\vep},y}\in B_{r_{\vep}}(0))\nonumber \\ &\le \sup_y \left(\frac{1}{\vep}T[u_{k\vep}:3](\tilde{Z}^{(\vep)}_{w_{k\vep},y}|y^{w_{k\vep}})\right)^2 \mathbf{1}(\sqrt{\vep}Z_{w_{k\vep},y}\in B_{r_{\vep}}(0)) \exp\left(C\sqrt{\vep}\left(\log{\left(\frac{1}{\vep}\right)}\right)^{3/2}\right)\nonumber \\&\le C \vep \left(\log{\left(\frac{1}{\vep}\right)}\right)^3.
    \end{align}
    In the last inequality we have bounded the term $\exp(C\sqrt{\vep}(\log{(1/\vep)})^{3/2})$ by a constant using $\vep\in (0,1)$. In the same vein, 
    \begin{align}\label{eq:large12}
    &\;\;\;\;\sup_y \E\left[\big|\vartheta^{(1)}_{\vep}(\tilde{Z}^{(\vep)}_{w_{k\vep},y}|y^{w_{k\vep}})\big|\mathbf{1}(\sqrt{\vep}Z_{w_{k\vep},y}\in B_{r_{\vep}}(0))\right]\nonumber \\ &\le \sup_y \E\left[\left(\frac{1}{\vep}T[u_{k\vep}:3](\tilde{Z}^{(\vep)}_{w_{k\vep},y}|y^{w_{k\vep}})\right)^2 \mathbf{1}(\sqrt{\vep}Z_{w_{k\vep},y}\in B_{r_{\vep}}(0))\right] \exp\left(C\sqrt{\vep}\left(\log{\left(\frac{1}{\vep}\right)}\right)^{3/2}\right)\nonumber \\&\le  C \sup_y \E\left[\left(\frac{1}{\vep}T[u_{k\vep}:3](\tilde{Z}^{(\vep)}_{w_{k\vep},y}|y^{w_{k\vep}})\right)^2 \right]\le C \vep .
    \end{align}
    In the last display, we have additionally used standard Gaussian tail bounds.

    \vspace{0.1in}

    We can carry out the same line of argument for the other terms in the exponential as seen in \eqref{eq:large1}. We simply produce the corresponding definitions and bounds noting that they can be obtained similarly as the bounds for $\vartheta^{(1)}_{\vep}(\tilde{Z}^{(\vep)}_{w_{k\vep},y}|y^{w_{k\vep}})$ above.

    Define 
    \begin{align*}
    \vartheta^{(2)}_{\vep}(\tilde{Z}^{(\vep)}_{w_{k\vep},y}|y^{w_{k\vep}})&:=\exp\left(- \frac{1}{\vep} R[u_{k\vep}:3](\tilde{Z}^{(\vep)}_{w_{k\vep},y}|y^{w_{k\vep}})\right)-1.
    \end{align*}
    It satisfies 
    \begin{equation}\label{eq:large21}
        \sup_y \big|\vartheta^{(2)}_{\vep}(\tilde{Z}^{(\vep)}_{w_{k\vep},y}|y^{w_{k\vep}})\big|\mathbf{1}(\sqrt{\vep}Z_{w_{k\vep},y}\in B_{r_{\vep}}(0))\le C \vep \left(\log{\left(\frac{1}{\vep}\right)}\right)^2,
    \end{equation}
    and
    \begin{equation}\label{eq:large22}
        \sup_y \E\left[\big|\vartheta^{(2)}_{\vep}(\tilde{Z}^{(\vep)}_{w_{k\vep},y}|y^{w_{k\vep}})\big|\mathbf{1}(\sqrt{\vep}Z_{w_{k\vep},y}\in B_{r_{\vep}}(0))\right]\le C \vep.
    \end{equation}
    Define 
    \begin{align*}
    \vartheta^{(3)}_{\vep}(\tilde{Z}^{(\vep)}_{w_{k\vep},y}|y^{w_{k\vep}})&:=\exp\left(- T[f:1](\tilde{Z}^{(\vep)}_{w_{k\vep},y}|y^{w_{k\vep}})\right)-1+T[f:1](\tilde{Z}^{(\vep)}_{w_{k\vep},y}|y^{w_{k\vep}}).
    \end{align*}
    It satisfies 
    \begin{equation}\label{eq:large31}
        \sup_y \big|\vartheta^{(3)}_{\vep}(\tilde{Z}^{(\vep)}_{w_{k\vep},y}|y^{w_{k\vep}})\big|\mathbf{1}(\sqrt{\vep}Z_{w_{k\vep},y}\in B_{r_{\vep}}(0))\le C \vep \log{\left(\frac{1}{\vep}\right)},
    \end{equation}
    and
    \begin{equation}\label{eq:large32}
        \sup_y \E\left[\big|\vartheta^{(3)}_{\vep}(\tilde{Z}^{(\vep)}_{w_{k\vep},y}|y^{w_{k\vep}})\big|\mathbf{1}(\sqrt{\vep}Z_{w_{k\vep},y}\in B_{r_{\vep}}(0))\right]\le C \vep.
    \end{equation}
    Finally, define 
    \begin{align*}
    \vartheta^{(4)}_{\vep}(\tilde{Z}^{(\vep)}_{w_{k\vep},y}|y^{w_{k\vep}})&:=\exp\left(-  R[f:1](\tilde{Z}^{(\vep)}_{w_{k\vep},y}|y^{w_{k\vep}})\right)-1.
    \end{align*}
    It satisfies 
    \begin{equation}\label{eq:large41}
        \sup_y \big|\vartheta^{(4)}_{\vep}(\tilde{Z}^{(\vep)}_{w_{k\vep},y}|y^{w_{k\vep}})\big|\mathbf{1}(\sqrt{\vep}Z_{w_{k\vep},y}\in B_{r_{\vep}}(0))\le C \vep \log{\left(\frac{1}{\vep}\right)},
    \end{equation}
    and
    \begin{equation}\label{eq:large42}
        \sup_y \E\left[\big|\vartheta^{(4)}_{\vep}(\tilde{Z}^{(\vep)}_{w_{k\vep},y}|y^{w_{k\vep}})\big|\mathbf{1}(\sqrt{\vep}Z_{w_{k\vep},y}\in B_{r_{\vep}}(0))\right]\le C \vep.
    \end{equation}
    It is worth noting that the bounds \eqref{eq:large31} and \eqref{eq:large32} (which correspond to approximating terms involving Taylor polynomials) will require \eqref{eq:gradbound3} in the same way as the bounds \eqref{eq:large11} and \eqref{eq:large12} (which also correspond to approximating terms involving Taylor polynomials). On the other hand, the same step will require \eqref{eq:gbd4} while obtaining the bounds \eqref{eq:large21}, \eqref{eq:large22}, \eqref{eq:large41}, and \eqref{eq:large42} (which correspond to approximating remainder terms after suitable Taylor series approximations). 


    We will now use the above bounds in conjunction with \eqref{eq:largge} and \eqref{eq:large1}. First we use the definitions of $\vartheta^{(i)}_{\vep}(\tilde{Z}^{(\vep)}_{w_{k\vep},y}|y^{w_{k\vep}})$, $i=1,2,3,4$, in \eqref{eq:largge} and \eqref{eq:large1} to note that:

    \begin{align}\label{eq:gaussbreak}
        &\;\;\;\;\frac{\sqrt{\mathrm{det}(\nabla^2 u_{k\vep}(y^{w_{k\vep}}))}}{(2\pi\vep)^{\frac{d}{2}}}\int\limits_{B_{r_{\vep}}(y^{w_{k\vep}})} \exp\left(-\frac{1}{\vep}\mcD[u_{k\vep}](x|y)-f(x)+f(y^{w_{k\vep}})\right)\,dx\nonumber \\ 
        &=\E\bigg[\left(1-\frac{1}{\vep}T[u_{k\vep}:3](\tilde{Z}^{(\vep)}_{w_{k\vep},y}|y^{w_{k\vep}})+\vartheta^{(1)}_{\vep}(\tilde{Z}^{(\vep)}_{w_{k\vep},y}|y^{w_{k\vep}})\right)\left(1+\vartheta^{(2)}_{\vep}(\tilde{Z}^{(\vep)}_{w_{k\vep},y}|y^{w_{k\vep}})\right)\nonumber \\ &\qquad \left(1-T[f:1](\tilde{Z}^{(\vep)}_{w_{k\vep},y}|y^{w_{k\vep}})+\vartheta^{(3)}_{\vep}(\tilde{Z}^{(\vep)}_{w_{k\vep},y}|y^{w_{k\vep}})\right)\left(1+\vartheta^{(4)}_{\vep}(\tilde{Z}^{(\vep)}_{w_{k\vep},y}|y^{w_{k\vep}})\right)\nonumber \\ &\qquad \qquad \mathbf{1}(\sqrt{\vep}Z_{w_{k\vep},y}\in B_{r_{\vep}}(0))\bigg].
    \end{align}

We will now expand out all the brackets above. Let us first try to isolate the terms which are $o(\vep)$. To wit, note that if a term has a product of at least two of $\vartheta_{\vep}^{(i)}(\tilde{Z}^{(\vep)}_{w_{k\vep},y}|y^{w_{k\vep}})$ and $\vartheta^{(j)}_{\vep}(\tilde{Z}^{(\vep)}_{w_{k\vep},y}|y^{w_{k\vep}})$, then it is $o(\vep)$. 
This is because by \eqref{eq:large11}, \eqref{eq:large21}, \eqref{eq:large31}, and \eqref{eq:large41}, for $i\neq j$, we have \begin{align*}&\;\;\;\;\sup_y \big|\vartheta_{\vep}^{(i)}(\tilde{Z}^{(\vep)}_{w_{k\vep},y}|y^{w_{k\vep}})\vartheta^{(j)}_{\vep}(\tilde{Z}^{(\vep)}_{w_{k\vep},y}|y^{w_{k\vep}})\mathbf{1}(\sqrt{\vep}Z_{w_{k\vep},y}\in B_{r_{\vep}}(0))\big|\\ &\le C\vep^2\left(\log{\left(\frac{1}{\vep}\right)}\right)^{6}=o(\vep).
\end{align*}
Same goes for terms having product of some $\vartheta_{\vep}^{(i)}(\tilde{Z}^{(\vep)}_{w_{k\vep},y}|y^{w_{k\vep}})$ with either of $\frac{1}{\vep}T[u_{k\vep}:3](\tilde{Z}^{(\vep)}_{w_{k\vep},y}|y^{w_{k\vep}})$ or $T[f:1](\tilde{Z}^{(\vep)}_{w_{k\vep},y}|y^{w_{k\vep}})$. For example, by \eqref{eq:large2} and \eqref{eq:large21}, we have:
\begin{align*}
&\;\;\;\;\sup_y \bigg|\vartheta_{\vep}^{(i)}(\tilde{Z}^{(\vep)}_{w_{k\vep},y}|y^{w_{k\vep}})\frac{1}{\vep}T[u_{k\vep}:3](\tilde{Z}^{(\vep)}_{w_{k\vep},y}|y^{w_{k\vep}})\bigg|\mathbf{1}(\sqrt{\vep}Z_{w_{k\vep},y}\in B_{r_{\vep}}(0))\\ &\le C\vep^{3/2}\left(\log{\left(\frac{1}{\vep}\right)}\right)^{7/2}=o(\vep).
\end{align*}
Therefore, we can easily isolate the terms that potentially contribute $O(\vep)$ or higher. By doing this in \eqref{eq:gaussbreak}, we get that:
    
    \begin{align*}
        &\;\;\;\;\frac{\sqrt{\mathrm{det}(\nabla^2 u_{k\vep}(y^{w_{k\vep}}))}}{(2\pi\vep)^{\frac{d}{2}}}\int\limits_{B_{r_{\vep}}(y^{w_{k\vep}})} \exp\left(-\frac{1}{\vep}\mcD[u_{k\vep}](x|y)-f(x)+f(y^{w_{k\vep}})\right)\,dx\nonumber \\ 
        &=1-\E\bigg[\bigg(\frac{1}{\vep}T[u_{k\vep}:3](\tilde{Z}^{(\vep)}_{w_{k\vep},y}|y^{w_{k\vep}})+\frac{1}{\vep}T[u_{k\vep}:3](\tilde{Z}^{(\vep)}_{w_{k\vep},y}|y^{w_{k\vep}})T[f:1](\tilde{Z}^{(\vep)}_{w_{k\vep},y}|y^{w_{k\vep}})\\ &-T[f:1](\tilde{Z}^{(\vep)}_{w_{k\vep},y}|y^{w_{k\vep}})+\sum_{i=1}^4 \vartheta_{\vep}^{(i)}(\tilde{Z}^{(\vep)}_{w_{k\vep},y}|y^{w_{k\vep}})\bigg)\mathbf{1}(\sqrt{\vep}Z_{w_{k\vep},y}\in B_{r_{\vep}}(0))\bigg]+o(\vep).
    \end{align*}
    Here the $o(\vep)$ term, as argued above, is uniform in $y$. Next, we use the symmetry of Gaussians to note that 
    $$\E\left[\frac{1}{\vep}T[u_{k\vep}:3](\tilde{Z}^{(\vep)}_{w_{k\vep},y}|y^{w_{k\vep}})\mathbf{1}(\sqrt{\vep}Z_{w_{k\vep},y}\in B_{r_{\vep}}(0))\right]=0,$$
    and 
    $$\E\left[T[f:1](\tilde{Z}^{(\vep)}_{w_{k\vep},y}|y^{w_{k\vep}})\mathbf{1}(\sqrt{\vep}Z_{w_{k\vep},y}\in B_{r_{\vep}}(0))\right]=0.$$
    Therefore, the second and the fourth term above are both $0$. For the third term, note that by using \cref{asn:solcon}, part (iii), we get:
    \begin{align*}
        &\;\;\;\sup_y\E\left[\frac{1}{\vep}T[u_{k\vep}:3](\tilde{Z}^{(\vep)}_{w_{k\vep},y}|y^{w_{k\vep}})T[f:1](\tilde{Z}^{(\vep)}_{w_{k\vep},y}|y^{w_{k\vep}})\mathbf{1}(\sqrt{\vep}Z_{w_{k\vep},y}\in B_{r_{\vep}}(0))\right]\\ &\le \sup_y C\frac{1}{\vep}\E\lVert \tilde{Z}^{(\vep)}_{w_{k\vep},y}-y^{w_{k\vep}}\rVert^4\le C \vep.
    \end{align*}
    Finally, by \eqref{eq:large12}, \eqref{eq:large22}, \eqref{eq:large32}, and \eqref{eq:large42}, we get:
    $$\sup_y \sum_{i=1}^4 \E\left[\bigg|\vartheta_{\vep}^{(i)}(\tilde{Z}^{(\vep)}_{w_{k\vep},y}|y^{w_{k\vep}})\mathbf{1}(\sqrt{\vep}Z_{w_{k\vep},y}\in B_{r_{\vep}}(0))\bigg|\right]\le C\vep.$$
    Therefore, all the requisite terms are $O(\vep)$ or of a lower order. This establishes \eqref{eq:showlarge}, thereby completing the proof of \cref{lem:pmaboundmain}.
\end{proof}

In order to prove \cref{lem:erboundmain}, we consider the following proposition which will be used multiple times in the sequel. 

\begin{prop}\label{prop:contra}
Given $\phi_1, \phi_2\in \diffcont(\R^d)$, we have
$$\lVert\opV[\phi_1]-\opV[\phi_2]\rVert_{\infty}\le \lVert \phi_1-\phi_2\rVert_{\infty},\qquad \lVert \opU[\phi_1]-\opU[\phi_2]\rVert_{\infty}\le \lVert \phi_1-\phi_2\rVert_{\infty}.$$
\end{prop}

\begin{proof}
    By the variational representation of KL divergence, we have:
\begin{equation}\label{eq:dualRE}
\opV[\phi_1](y) = \vep\sup_{\nu:\ \KL{\nu}{e^{-f}}<\infty}\left[ \int \left(\frac{1}{\vep}\langle x,y\rangle - \frac{1}{\vep}\phi_1(x)\right) d\nu(x) - \KL{\nu}{e^{-f}}\right].
\end{equation}
Clearly the same representation as in \eqref{eq:dualRE} also holds with $\phi_1$ replacing $\phi_2$. Now, for any probability measure $\nu$ satisfying $\KL{\nu}{e^{-f}}<\infty$, we have:
\begin{align*}
\bigg| &\;\;\;\; \left[\int \left(\frac{1}{\vep}\langle x,y\rangle - \frac{1}{\vep}\phi_1(x)\right) d\nu(x) - \KL{\nu}{\mu}\right] \\ & -  \int \left[\int \left(\frac{1}{\vep}\langle x,y\rangle - \frac{1}{\vep}\phi_1(x)\right) d\nu(x) - \KL{\nu}{\mu}\right]\bigg| \le \norm{\phi_1 - \phi_2}_\infty. 
\end{align*}
Invoking \eqref{eq:dualRE} completes the proof for $\opV$s. The same strategy also works for $\opU$'s.
\end{proof}

We are now in position to prove the key lemma.
%\SP{Not sure you want to claim this is a "main" result. Perhaps a "key lemma"?}{\color{blue} Edited.}

\vspace{0.1in} 

\emph{Proof of \cref{lem:erboundmain}.} By the uniform Lipschitz property of $\opV$s as established in \cref{prop:contra}, we have: 
$$\lVert b_k^{\vep}\rVert_{\infty}=\frac{1}{\vep}\lVert \opV[u_k^{\vep}]-\opV[u_{k\vep}]\rVert_{\infty}\le \frac{1}{\vep}\lVert u_k^{\vep}-u_{k\vep}\rVert_{\infty}=\lVert a_k^{\vep}\rVert_{\infty}.$$
Therefore, it suffices to only work with $a_k^{\vep}$s as defined in \cref{lem:erboundmain}. 
%\SP{What does the above senetence mean? Are you saying that the proof is similar for b?} {\color{blue} Edited.}
Recall that $u_{k\vep}$'s are the solutions of the PMA \eqref{eq:pma} restricted to time points $t=\vep,2\vep,\ldots$. Consider the following surrogate sequence of functions 
$\bar{u}_{k+1}^{\vep}:=\opS[u_{k\vep}],$
and define,
\begin{equation}\label{eq:approx}
R_T(\vep):=\sup_{k\vep\le T}\lVert \bar{u}_{k+1}^{\vep}-u_{(k+1)\vep}\rVert_{\infty}.
\end{equation}

With the above notation, the proof of \cref{lem:erboundmain} will proceed in two steps. Once again throughout this proof, we will use $C$ to denote a generic constant (which might change from one line to the next) free of $\vep$. 

\emph{Step (a).} We will show that 
$$R_T(\vep)\le C \vep^2.$$

\emph{Step (b).} We then show that 
$$\lVert a_k^{\vep}\rVert_{\infty}\le (k/\vep) R_T(\vep),$$
for all $k$ such that $k\vep\le T$.

By combining steps (a) and (b), we get:
$$\sup_{k:\ k\vep\le T} \lVert a_k^{\vep}\rVert_{\infty}\le \frac{T}{\vep^2}R_T(\vep)\le CT.$$

This proves \cref{lem:erboundmain}. We now focus our attention on proving steps (a) and (b). 

\emph{Proof of step (a).} Recall the time derivatives of the $u_{k\vep}$s as in the PMA~\eqref{eq:pma}. By \cref{asn:solcon}, part (ii), we have:
\begin{align}\label{eq:pmareg}
    &\;\;\;\;\sup_{k:\ k\vep\le T}\sup_x \Bigg|u_{(k+1)\vep}(x)-u_{k\vep}(x)-\vep\left(f(x)-g(x^{u_{k\vep}})+\ldet\left(\frac{\partial x^{u_{k\vep}}}{\partial x\hfill}\right)\right)\Bigg|\nonumber \\ &=\sup_{k:\ k\vep\le T}\sup_x \Bigg|u_{(k+1)\vep}(x)-u_{k\vep}(x)-\vep\frac{\partial}{\partial t}u_t(x)\big|_{t=k\vep}\Bigg|\le C\vep.
\end{align}

%\SP{Unsure. Assumption 2.1 (ii) is about continuity. How does that lead to (6.25)?} {\color{blue} Edited. Earlier Assumption 2.1 (ii) had a uniformity missing as I figured the notation $\diffcont^{k,\ell}$ already incorporates that. I believe I have corrected it now.}


Next, we will show the following:
\begin{equation}\label{eq:stepa1}
    \sup_{k: \ k\vep\le T}\sup_x \bigg|\exp\left(\frac{\bar{u}_{k+1}^{\vep}(x)-u_{k\vep}(x)}{\vep}-\left(f(x)-g(x^{u_{k\vep}})+\ldet\left(\frac{\partial x^{u_{k\vep}}}{\partial x\hfill}\right)\right)\right)-1\bigg|\le C\vep,
\end{equation}
By taking logarithms in \eqref{eq:stepa1} and combining it with \eqref{eq:pmareg}, we easily get Step (a). Therefore, it only remains to prove \eqref{eq:stepa1}.

\emph{Proof of \eqref{eq:stepa1}.} First let us define 
$$\vartheta_{k\vep}(y):=\vep^{-1}\left(\opV[u_{k\vep}](y)-w_{k\vep}(y)-\frac{\vep d}{2}\log{(2\pi\vep)}+\vep f(y^{w_{k\vep}})-\frac{\vep}{2}\ldet\left(\frac{\partial y^{w_{k\vep}}}{\partial y\hfill}\right)\right).$$
The following estimate is an immediate consequence of \cref{lem:pmaboundmain}.
\begin{align}\label{eq:estcon}
    \sup_y \bigg|\exp\left(\frac{\vartheta_{k\vep}(y)}{\vep}\right)-1\bigg|\le C\vep.
\end{align}
We also note the following algebraic identity:
\begin{align*}
    &\;\;\;\;\exp\left(\frac{\bar{u}_{k+1}^{\vep}(x)-u_{k\vep}(x)}{\vep}-\left(f(x)-g(x^{u_{k\vep}})+\ldet\left(\frac{\partial x^{u_{k\vep}}}{\partial x\hfill}\right)\right)\right)\\ &=\frac{\sqrt{\mathrm{det}\left(\frac{\partial x\hfill}{\partial x^{u_{k\vep}}\hfill}\right)}}{(2\pi\vep)^{\frac{d}{2}}}\int \exp\Bigg(\frac{1}{\vep}\langle x,y\rangle-\frac{1}{\vep}w_{k\vep}(y)-\frac{1}{\vep}u_{k\vep}(x)+f(y^{w_{k\vep}})-f(x)-g(y)+g(x^{u_{k\vep}})\\ &\qquad -\frac{1}{2}\ldet\left(\frac{\partial y^{w_{k\vep}}}{\partial y\hfill}\right) + \frac{1}{2}\ldet\left(\frac{\partial x\hfill}{\partial x^{u_{k\vep}}\hfill}\right)\Bigg)\exp\left(-\frac{\vartheta_{k\vep}(y)}{\vep}\right)\,dy
\end{align*}
By combining the above display with \eqref{eq:estcon}, we have:
\begin{align*}
    &\;\;\;\;\sup_{k:\ k\vep\le T}\sup_x \Bigg|\exp\left(\frac{\bar{u}_{k+1}^{\vep}(x)-u_{k\vep}(x)}{\vep}-\left(f(x)-g(x^{u_{k\vep}})+\ldet\left(\frac{\partial x^{u_{k\vep}}}{\partial x\hfill}\right)\right)\right)\\ &\qquad\qquad - \frac{\sqrt{\mathrm{det}\left(\frac{\partial x\hfill}{\partial x^{u_{k\vep}}}\right)}}{(2\pi\vep)^{\frac{d}{2}}}\int \exp\Bigg(\frac{1}{\vep}\langle x,y\rangle-\frac{1}{\vep}w_{k\vep}(y)-\frac{1}{\vep}u_{k\vep}(x)+f(y^{w_{k\vep}})-f(x)\\ &\qquad\qquad -g(y)+g(x^{u_{k\vep}})-\frac{1}{2}\ldet\left(\frac{\partial y^{w_{k\vep}}}{\partial y\hfill}\right) + \frac{1}{2}\ldet\left(\frac{\partial x\hfill}{\partial x^{u_{k\vep}}\hfill}\right)\Bigg)\,dy\Bigg|\\ &\le (C\vep) \sup_{k:\ k\vep\le T}\sup_x \frac{\sqrt{\mathrm{det}\left(\frac{\partial x\hfill}{\partial x^{u_{k\vep}}}\right)}}{(2\pi\vep)^{\frac{d}{2}}}\int \exp\Bigg(\frac{1}{\vep}\langle x,y\rangle-\frac{1}{\vep}w_{k\vep}(y)-\frac{1}{\vep}u_{k\vep}(x)+f(y^{w_{k\vep}})-f(x)\\ &\qquad -g(y)+g(x^{u_{k\vep}})-\frac{1}{2}\ldet\left(\frac{\partial y^{w_{k\vep}}}{\partial y\hfill}\right) + \frac{1}{2}\ldet\left(\frac{\partial x\hfill}{\partial x^{u_{k\vep}}\hfill}\right)\Bigg)\,dy.
\end{align*}
Therefore in order to prove \eqref{eq:stepa1}, it suffices to prove: 
\begin{align*}
    &\;\;\;\;\sup_{k:\ k\vep\le T}\sup_x\bigg|\frac{\sqrt{\mathrm{det}\left(\frac{\partial x\hfill}{\partial x^{u_{k\vep}}}\right)}}{(2\pi\vep)^{\frac{d}{2}}}\int \exp\Bigg(\frac{1}{\vep}\langle x,y\rangle-\frac{1}{\vep}w_{k\vep}(y)-\frac{1}{\vep}u_{k\vep}(x)+f(y^{w_{k\vep}})-f(x)\nonumber\\ &\qquad -g(y)+g(x^{u_{k\vep}})-\frac{1}{2}\ldet\left(\frac{\partial y^{w_{k\vep}}}{\partial y\hfill}\right) + \frac{1}{2}\ldet\left(\frac{\partial x\hfill}{\partial x^{u_{k\vep}}\hfill}\right)\Bigg)\,dy-1\bigg|\le C\vep.
\end{align*}
Once again by taking logarithms on both sides above, it suffices to show:
\begin{align}\label{eq:stepa2}
&\sup_{k:\ k\vep\le T} \sup_x \bigg|\log\int \exp\left(\frac{1}{\vep}\langle x,y\rangle-\frac{1}{\vep}w_{k\vep}(y)+f(y^{w_{k\vep}})-g(y)-\frac{1}{2}\ldet\left(\frac{\partial y^{w_{k\vep}}}{\partial y\hfill}\right)\right)\,dy\nonumber \\ &\qquad - \frac{\vep d}{2}\log{(2\pi\vep)} - f(x) + g(x^{u_{k\vep}})-\ldet\left(\frac{\partial x^{u_{k\vep}}}{\partial x\hfill}\right)\bigg|\le C\vep. 
\end{align}
Now $$\widetilde{\opV}[w_{k\vep}](x):=\vep\log\int \exp\left(\frac{1}{\vep}\langle x,y\rangle-\frac{1}{\vep}w_{k\vep}(y)+f(y^{w_{k\vep}})-g(y)-\frac{1}{2}\ldet\left(\frac{\partial y^{w_{k\vep}}}{\partial y\hfill}\right)\right)\,dy$$
has the same form as $\opV[u_{k\vep}]$ with $u_{k\vep}$ replaced by $w_{k\vep}$ and $f$ replaced by the function $\tilde{f}$ given by 
$$\tilde{f}(y)=-f(y^{w_{k\vep}})+g(y)+\frac{1}{2}\ldet\left(\frac{\partial y^{w_{k\vep}}}{\partial y\hfill}\right).$$
By \cref{asn:solcon}, part (iii), $\tilde{f}$ satisfies the same assumptions as $f$ required in \cref{lem:pmaboundmain}. 
Therefore, reworking the same proof as \cref{lem:pmaboundmain} with $\tilde{f}$, we get:
%\SP{Are you invoking Lemma 4.5 or reworking the proof on a different function $\tilde{f}$ that satisfies the same assumption as $f$?} {\color{blue} Edited. I believe it is the latter. I was viewing these as being equivalent. }
$$\sup_{k:\ k\vep\le T}\sup_x \bigg| \widetilde{\opV}[w_{k\vep}](x)-\frac{\vep d}{2}\log{(2\pi\vep)}+\vep \tilde{f}(x^{u_{k\vep}})-\frac{\vep}{2}\ldet\left(\frac{\partial x^{u_{k\vep}}}{\partial x\hfill}\right)\bigg|\le C\vep^2.$$
As $\tilde{f}(x^{u_{k\vep}})=-f(x)+g(x^{u_{k\vep}})-\frac{1}{2}\ldet\left(\frac{\partial x^{u_{k\vep}}}{\partial x\hfill}\right)$, the above display establishes \eqref{eq:stepa2}. This establishes \eqref{eq:stepa1} and consequently, also establishes Step (a).

\vspace{0.1in}

\emph{Proof of Step (b).} To establish step (b), the crucial tool will be \cref{prop:contra}. To wit, note that for all $k$ such that $k\vep\le T$, we have
%\SP{small t or large T?} {\color{blue} Edited}
\begin{align*}
    \lVert a_k^{\vep}\rVert_{\infty}&\le \frac{1}{\vep}\lVert u_{k}^{\vep}-\bar{u}_{k}^{\vep}\rVert_{\infty}+\frac{1}{\vep} \lVert \bar{u}_{k}^{\vep}-u_{k\vep}\rVert_{\infty}\\ &\le \frac{1}{\vep}\lVert \opS[u_{k-1}^{\vep}]-\opS[u_{(k-1)\vep}]\rVert_{\infty}+\frac{1}{\vep}R_t(\vep)\\ &\le \lVert a_{k-1}^{\vep}\rVert_{\infty}+\frac{1}{\vep}R_t(\vep).
\end{align*}
The last inequality follows by noting that $\opS$ is $1$-Lipschitz in the uniform norm which in turn follows from the fact that $\opS=\opU\circ\opV$ and both $\opU$, $\opV$ are $1$-Lipschitz in the uniform norm by \cref{prop:contra}. Now, in order to prove the conclusion in Step (b), we will use the above inequality recursively, i.e., 
$$\lVert a_k^{\vep}\rVert_{\infty}\le \lVert a_{k-1}^{\vep}\rVert_{\infty}+\frac{1}{\vep}R_t(\vep)\le \lVert a_{k-2}^{\vep}\rVert_{\infty}+\frac{2}{\vep}R_t(\vep)\le \ldots \le \lVert a_0^{\vep}\rVert_{\infty}+\frac{k}{\vep}R_T(\vep).$$
Now, by definition, $a_0^{\vep}=\vep^{-1}(u_0^{\vep}-u_0)=0$ as we use the same initializer for the Sinkhorn algorithm \eqref{eq:sinkupdt} and the PMA \eqref{eq:pma}. This implies, using the above display coupled with step (a), 
$$\sup_{k:\ k\vep\le T}\lVert a_k^{\vep}\rVert_{\infty}\le \sup_{k:\ k\vep\le T} \frac{k}{\vep}R_T(\vep)\le \sup_{k:\ k\vep\le T} \frac{k}{\vep}\cdot C\vep^2\le CT.$$
This establishes step (b).
%\SP{Can you expand this last line? Write down a two-line argument using math symbols.} {\color{blue} Edited.}
\begin{comment}
\begin{lmm}\label{lem:stab2}
Consider the Sinkhorn algorithm initialized according to \eqref{eq:init}. Then for $t\ge 0$, and $k\vep \le t$, the following holds: 
$$\lVert u^{\vep}_{k}-\tilde{u}_{k\vep,\vep}\rVert_{\infty}\leq k R_t(\vep).$$
\end{lmm}

\begin{proof}
Once again, the proof proceeds by induction. The result is trivial for $k=0$ according to \eqref{eq:init}. Assume it holds upto some $k\equiv k_0$. To establish the bound at $k+1$, note that
$$\lVert u^{\vep}_{k+1}-\tilde{u}_{(k+1)\vep,\vep}\rVert_{\infty}\leq \lVert u^{\vep}_{k+1}-\bar{u}_{(k+1)\vep,\vep}\rVert_{\infty}+R_t(\vep)\leq \lVert u^{\vep}_{k}-\tilde{u}_{k\vep,\vep}\rVert_{\infty}+R_t(\vep).$$
Here the last inequality follows from~\cref{lem:stab1}. By the induction hypothesis $\lVert u^{\vep}_{k}-\tilde{u}_{k\vep,\vep}\rVert_{\infty}\leq k R_t(\vep)$. Combining this observation with the above display completes the induction. 
\end{proof}

We are now in the position to state the main lemma of this section which should lead to the desired conclusion. 

\begin{lmm}\label{lem:erbd}
    Under \cref{asn:solcon}, we have
    $$R_t(\vep)=O(\vep^3).$$
\end{lmm}

\begin{lmm}\label{lem:prelimestim}
 Given a convex function $u(\cdot)$, set $w:=u^*$. For any $\vep\in (0,1/2)$, consider a function $G(\cdot)$ on $\R^d$ which is smooth. Assume that 
 $$A_1:=\big(\inf_{x\in\R^d}\lmn(\nabla^2 u(x))\big)^{-1}\vee \big(\sup_{x\in\R^d} \lmx(\nabla^2 u(x))\big) \in (0,\infty).$$
 Also 
 \begin{align*}
 A_2&:=\max_{r=3}^6 \big(\lVert \nabla^r u\rVert_{\infty}\vee \lVert \nabla^r w\rVert_{\infty}\big)+\sum_{r=1}^4 \lVert \nabla^r G\rVert_{\infty} <\infty,
 \end{align*}
 and define 
 \begin{align*}
 \omega(\delta)&:=\sup_{\lVert x-y\rVert\le\delta} \left(\lVert \nabla^4 G(x)-\nabla^4 G(y)\rVert_{\infty}+\lVert \nabla^6 u(x)-\nabla^6 u(y)\rVert_{\infty}\right) <\infty,
 \end{align*}
 for all $\delta>0$. Assume $\omega(\delta)\to 0$ as $\delta\to 0$. Set $A:=A_1\vee A_2$. 
For any $\vep\in (0,1/2)$ and any $y\in\R^d$, define a probability measure with density
 $$ \exp\left(\frac{1}{\vep} \langle x,y\rangle -\frac{1}{\vep}u(x)-G(x)-\frac{1}{\vep}\mc(y)\right)$$
 where $\mc(y)$ is the (scaled) log-normalizing constant  for all $\vep>0$.  Then there exists a constant $C>0$ (depending on $A$, $d$, and $\omega(\cdot)$ such that the following estimates hold:
 \begin{small}
 \begin{align}\label{eq:prestim1}
     &\bigg|\mc(y)-w(y)+\vep \big(G(y^w)-\frac{d}{2}\log{(2\pi\vep)}-(1/2)\ldet(\nabla^2 w(y))\big)-\vep^2M[u,G](y) \bigg| \le C \vep^3
 \end{align}
 \end{small}
 where 
 \begin{align}\label{eq:gaussdef}
     M[u,G](y) &:=  \frac{1}{2}\E\big(T[u:3](y^w+Z_{w,y}|y^w)\big)^2-\E T[u:4](y^w+Z_{w,y}|y^w)\nonumber \\ &\qquad - \E T[G:2](y^w+Z_{w,y}|y^w) + \frac{1}{2}\E(T[G:1](y^w+Z_{w,y}|y^w))^2\nonumber \\ &\qquad + \E T[u:3](y^w+Z_{w,y}|y^w) T[G:1](y^w+Z_{w,y}|y^w).
 \end{align}
\end{lmm}
 %\begin{align}\label{eq:prestim2}
  %   &\;\;\;\;\big\lVert \E_{\rho_y}(X)-y^{w}-\frac{\vep}{2}\nabla(\ldet(\nabla^2 w))(y)+\vep\nabla(G(\nabla w))(y)\big\rVert \nonumber \\ &\le C_A \vep^{3/2}\big((-\log{\vep})^{3/2}+(\omega_1\vee\omega_2)(\sqrt{\vep\log{(1/\vep)}})\big).
 %\end{align}
 %and 
 %\begin{align}\label{eq:prestim3}
  %   &\;\;\;\;\big\lVert \mbox{Var}_{\rho_y}(X)-\vep \nabla^2 w(y)-\frac{\vep^2}{2}\nabla^2(\ldet(\nabla^2 w))(y)+\vep^2 \nabla^2 (G(\nabla w))(y) \big\rVert_{\infty}\nonumber \\ &\le C_A \left(\vep^2 (\omega_1\vee \omega_2)\big(\sqrt{\vep\log{(1/\vep)}}\big)+\vep^{5/2}\big(-\log{\vep})^{9/2}\big)\right).
 %\end{align}
 %\end{comment}


%\begin{remark}\label{rem:prest}
%We note here that $\nabla \mc(y)=\E_{\rho_y}(X)$ and $\vep\nabla^2 \mc(y)=\mbox{Var}_{\rho_y}(X)$.    
%\end{remark}

\begin{proof}
    
    By a multivariate Taylor series expansion, we have for any $k\ge 0$:
    \begin{align}\label{eq:taylor1}
        &\;\;\;\;\mcD[u](x|y)\nonumber \\ &=-\frac{1}{2}(x-y^{w})^{\top}\nabla^2 u(y^{w})(x-y^{w}) - \sum_{r=3}^4 T[u:r](x|y^{w})-R[u:4](x|y^{w}).
    \end{align}
    We obtain equivalent expansions for the functions  $G(\cdot)$ and $\tilde{G}_{\vep}(\cdot)$, to get:
    \begin{align}\label{eq:taylor2}
       G(x)&=G(y^{w})+T[G:1](x|y^{w})+T[G:2](x|y^w)+R[G:2](x|y^{w}).
    \end{align}
    Another observation we will use in this proof is as follows:
    \begin{equation}\label{eq:estimpf1}
\mcD[u](x|y) \le -\frac{1}{2A_1}\lVert x-y^{w}\rVert^2.
\end{equation}
    
    \noindent We are now in a position to simplify $\mc(\cdot)$. 
    
    \vspace{0.1in}
    
    \emph{Proof of \eqref{eq:prestim1}.} Observe that 
    \begin{align}\label{eq:simpl1}
       &\;\;\;\;\frac{\sqrt{\mathrm{det}(\nabla^2 u(y^{w}))}}{(2\pi\vep)^{\frac{d}{2}}}\exp\left(\frac{1}{\vep}\mc(y)-\frac{1}{\vep}w(y)+G(y^w)\right)\nonumber \\ &=\frac{\sqrt{\mathrm{det}(\nabla^2 u(y^{w}))}}{(2\pi\vep)^{\frac{d}{2}}}\int \exp\bigg(\frac{1}{\vep}\mcD[u](x|y)-G(x)+G(y^w)-\vep\tilde{G}_{\vep}(x)+\vep\tilde{G}_{\vep}(y^w)\bigg)\,dx.
    \end{align}
    Define 
    $$r_{\vep}:=\sqrt{-40d A_1 \vep\log{\vep}},\qquad \mbox{for}\ \vep\in (0,1/2).$$
    For the rest of the proof, $C$ will denote changing constants depending only on $d$, $A$, and $\omega(\cdot)$. Observe that by \eqref{eq:estimpf1}, we have:
    \begin{align}\label{eq:gradbound4}
        &\;\;\;\;\frac{\sqrt{\mathrm{det}(\nabla^2 u(y^{w}))}}{(2\pi\vep)^{\frac{d}{2}}}\int\limits_{B^c_{r_{\vep}}(y^{w})} \exp\bigg(\frac{1}{\vep}\mcD[u](x|y)-G(x)+G(y^{w})\bigg)\,dx\nonumber \\ &\le C\exp(G(y^{w}))\E\left[\exp\left(-G(y^{w}+\sqrt{\vep A_1} Z)\right)\mathbf{1}(\sqrt{\vep A_1}Z\in B_{r_{\vep}}^c(0))\right]\nonumber \\ &\le C \sqrt{\prod_{i=1}^d \E\exp\left(\sqrt{\vep A_1}\lVert \nabla G\rVert_{\infty} |Z_i|\right)}\sqrt{d\Pr(|Z_1|\ge r_{\vep}/\sqrt{\vep d A_1})}\le C \vep^{10}.
    \end{align}
    In the last step, we have used standard Gaussian tail bounds. Next, we set $$\tilde{Z}^{(\vep)}_{w,y}\sim y^{w}+\sqrt{\vep}Z_{w,y}.$$ 
    We now focus on the integral inside $B_{r_{\vep}}(y^{w})$. 
    \begin{align}\label{eq:gaussbreak}
        &\;\;\;\;\frac{\sqrt{\mathrm{det}(\nabla^2 u(y^{w}))}}{(2\pi\vep)^{\frac{d}{2}}}\int\limits_{B_{r_{\vep}}(y^{w})} \exp\left(\frac{1}{\vep}\mcD[u](x|y)-G(x)+G(y^{w})\right)\,dx\nonumber \\ &=\E\bigg[\left(\sum_{m=0}^{\infty}\frac{1}{\vep^m m!}(-T[u:3](\tilde{Z}^{(\vep)}_{w,y}|y^{w}))^m\right)\left(\sum_{m=0}^{\infty}\frac{1}{\vep^m m!}(- T[u:4](\tilde{Z}^{(\vep)}_{w,y}|y^{w}))^m\right)\nonumber \\ &\qquad \left(\sum_{m=0}^{\infty}\frac{1}{\vep^m m!}(-R[u:4](\tilde{Z}^{(\vep)}_{w,y}|y^{w}))^m\right)\left(\sum_{m=0}^{\infty}\frac{1}{m!}(-T[G:1](\tilde{Z}^{(\vep)}_{w,y}|y^{w}))^m\right)\nonumber \\  &\qquad \left(\sum_{m=0}^{\infty}\frac{1}{m!}(-T[G:2](\tilde{Z}^{(\vep)}_{w,y}|y^{w}))^m\right)\left(\sum_{m=0}^{\infty}\frac{1}{m!}(-R[G:2](\tilde{Z}^{(\vep)}_{w,y}|y^{w}))^m\right)\nonumber \\  &\qquad \mathbf{1}(\sqrt{\vep}Z_{w,y}\in B_{r_{\vep}}(0))\bigg].
    \end{align}
    We will now simplify the right hand side above. By invoking \eqref{eq:gradbound3} on the set $\{\sqrt{\vep}Z_{w,y}\in B_{r_{\vep}}(0)\}$, we get:
    $$\bigg|\sum_{m=3}^{\infty}\frac{1}{\vep^m m!}(-T[u:3](\tilde{Z}^{(\vep)}_{w,y}|y^{w}))^m\bigg|\le C \vep^{3/2}(-\log{\vep})^{9/2}.$$
    Similarly, we also have:
    $$\bigg|\sum_{m=2}^{\infty}\frac{1}{\vep^m m!}(- T[u:4](\tilde{Z}^{(\vep)}_{w,y}|y^{w}))^m\bigg|\le  C \vep^2 (-\log{\vep})^{4},$$
    $$\bigg|\sum_{m=2}^{\infty}\frac{1}{\vep^m m!}(-R[u:4](\tilde{Z}^{(\vep)}_{w,y}|y^{w}))^m\bigg|\le C \vep^2 (-\log{\vep})^{4} ,$$
    $$\bigg|\sum_{m=3}^{\infty}\frac{1}{m!}(-T[G:1](\tilde{Z}^{(\vep)}_{w,y}|y^{w}))^m\bigg|\le C\vep^{3/2}(-\log{\vep})^{3/2} ,$$
    $$\bigg|\sum_{m=2}^{\infty}\frac{1}{m!}(-T[G:2](\tilde{Z}^{(\vep)}_{w,y}|y^{w}))^m\bigg|\le  C \vep^2 (\log{\vep})^2,$$
    $$\bigg|\sum_{m=2}^{\infty}\frac{1}{m!}(-R[G:2](\tilde{Z}^{(\vep)}_{w,y}|y^{w}))^m\bigg|\le  C \vep^2 (\log{\vep})^2,$$
    on the set $\{\sqrt{\vep}Z_{w,y}\in B_{r_{\vep}}(0)\}$. 
    To handle the other summands in the right hand side of \eqref{eq:gaussbreak}, we observe the following:
    $$\E(T[u:3](\tilde{Z}^{(\vep)}_{w,y}|y^w))=\E(T[G:1](\tilde{Z}^{(\vep)}_{w,y}|y^w))=0,$$
    $$\E\left[\frac{1}{\vep}|R[u:4](\tilde{Z}^{(\vep)}_{w,y}|y^{w})|\mathbf{1}(\sqrt{\vep}Z_{w,y}\in B_{r_{\vep}}(0))\right]\le C\vep \omega(r_{\vep}),$$
    $$\E\left[|R[G:2](\tilde{Z}^{(\vep)}_{w,y}|y^{w})|\mathbf{1}(\sqrt{\vep}Z_{w,y}\in B_{r_{\vep}}(0))\right]\le C\vep \omega(r_{\vep}).$$
    Also,
    \begin{align*}
        &\;\;\;\;\E\bigg[\left(1-\frac{1}{\vep}T[u:3](\tilde{Z}^{(\vep)}_{w,y}|y^{w})+\frac{1}{2\vep^2}\big(T[u:3](\tilde{Z}^{(\vep)}_{w,y}|y^{w})\big)^2\right)\left(1-\frac{1}{\vep}T[u:4](\tilde{Z}^{(\vep)}_{w,y}|y^{w})\right)\\ & \quad \left(1-T[G:1](\tilde{Z}^{(\vep)}_{w,y}|y^{w})+\frac{1}{2}\big(T[G:1](\tilde{Z}^{(\vep)}_{w,y}|y^{w})\big)^2\right)\left(1-T[G:2](\tilde{Z}^{(\vep)}_{w,y}|y^{w})\right)\\ & \quad \mathbf{1}(\sqrt{\vep}Z_{w,y}\in B_{r_{\vep}}(0))\bigg]\\ &=1 + \frac{1}{2}\E\big(T[u:3](y^w+Z_{w,y}|y^w)\big)^2-\E T[u:4](y^w+Z_{w,y}|y^w) \nonumber \\ &\qquad + \frac{1}{2}\E(T[G:1](y^w+Z_{w,y}|y^w))^2 - \E T[G:2](y^w+Z_{w,y}|y^w)  \nonumber \\ & \qquad + \E T[u:3](y^w+Z_{w,y}|y^w) T[G:1](y^w+Z_{w,y}|y^w) \\ &= 1+\vep M[u,G](y) + R_{\vep}(y),
    \end{align*}
    where $\lVert R_{\vep}\rVert_{\infty}\le C\vep^{3/2}(\log{(1/\vep)})^{9/2}$. Combining the above observations and using them in  \eqref{eq:gaussbreak}, we get:
    \begin{align*}
        &\;\;\;\;\bigg|\frac{\sqrt{\mathrm{det}(\nabla^2 u(y^{w}))}}{(2\pi\vep)^{\frac{d}{2}}}\int\limits_{B_{r_{\vep}}(y^{w})} \exp\bigg(\frac{1}{\vep}D[u](x|y)-G(x)+G(y^{w})\bigg)\,dx-1 -\vep M[u,G](y)\bigg| \\ &\quad \le C\left(\vep^{3/2}(\log{(1/\vep)})^{9/2}+ \vep \omega(r_{\vep})\right).
    \end{align*}
    Combining the above observation with \eqref{eq:simpl1} and \eqref{eq:gradbound4}, establishes \eqref{eq:prestim1}.
\end{proof}

Another related result will be useful to establish convergence of the Markov chain. The proof of this new result is simpler than the proof of \cref{lem:prelimestim}. Hence, we skip the details of the proof for brevity.

\begin{lmm}\label{lem:prelimestim2}
Consider the same setting as in \cref{lem:prelimestim}. Then 
\begin{align*}
&\;\;\;\;\sqrt{\mathrm{det}\left(\frac{\partial x\hfill}{\partial x^u}\right)}\frac{1}{(2\pi\vep)^{\frac{d}{2}}}\int (y-x^u)\exp\left(\frac{1}{\vep}\langle x,y\rangle-\frac{1}{\vep}u(x)-\frac{1}{\vep}w(y)-G(y)+G(x^u)\right)\,dy\\ &=\vep \frac{\partial}{\partial x}\left(-G(x^u)+\frac{1}{2}\ldet\left(\frac{\partial x^u}{\partial x\hfill}\right)\right)+o(\vep),
\end{align*}
and 
\begin{align*}
&\;\;\;\;\sqrt{\mathrm{det}\left(\frac{\partial x\hfill}{\partial x^u}\right)}\frac{1}{(2\pi\vep)^{\frac{d}{2}}}\int (y-x^u)(y-x^u)^{\top}\exp\bigg(\frac{1}{\vep}\langle x,y\rangle-\frac{1}{\vep}u(x)-\frac{1}{\vep}w(y)\\ &-G(y)+G(x^u)\bigg)\,dy=\vep \left(\frac{\partial x^u}{\partial x\hfill}\right)+o(\vep),
\end{align*}
where the $o(\cdot)$ term is in the uniform norm.
\end{lmm}
\end{comment}

\begin{longtable}{|p{2cm}|p{9.8cm}|}
\caption{Notation chart}
\label{tab:table3}\\
\hline
\textbf{Notation} & \textbf{Meaning} \\
\hline
$\mathbb{N}$ & Set of natural numbers \\
\hline
$[n]$, $n\in\mathbb{N}$ & The set $\{1,2,\ldots ,n\}$ \\
\hline
$\lmn(A)$ & The minimum eigenvalue of a square matrix $A$ \\
\hline 
$\lmx(A)$ & The maximum eigenvalue of a square matrix $A$ \\
\hline 
$\trc(A)$ & The trace of a square matrix $A$ \\
\hline 
$\lVert A\rVert_{\hs}$ & The Frobenius norm of a square matrix $A$ \\ 
\hline 
$\lVert A\rVert_{\mathrm{op}}$ & The $L^2$ operator norm of a square matrix $A$\\
\hline
$\sqrt{A}$ & Cholesky square root of a symmetric and positive definite matrix $A$, i.e., $\sqrt{A}=S$ if and only if $A=S^2$
\\ 
\hline 
$A\preceq B$ & $B-A$ is non-negative definite \\ 
\hline
$I_d$ & The $d\times d$ identity matrix. We will drop the $d$ from the subscript when the dimension is obvious\\ 
\hline 
$\mathrm{Id}$ & The identity function on $\R^d$\\ \hline 
$|x|$ & The Euclidean norm of a vector $x$ \\ 
\hline 
$T_{\#}\mu$ & Given a probability measures $\mu$ on $\R^d$ and a function $T:\R^d \rightarrow \R^d$, this is the push-forward of $\mu$ by $T$, i.e., the probability distribution of $T(X)$ where $X\sim\mu$ \\ \hline 
$\px \gamma$ & The $X$ marginal density of the joint density $\gamma$ on $\R^d\times \R^d$\\ \hline 
$\py \gamma$ & The $Y$ marginal density of the joint density $\gamma$ on $\R^d\times \R^d$
\\ \hline 
$p_{X|Y}\gamma(\cdot|\cdot)$ & The conditional density of $X$ given $Y$ under the joint density $\gamma$\\ \hline 
$p_{Y|X}\gamma(\cdot|\cdot)$ & The conditional density of $Y$ given $X$ under the joint density $\gamma$\\ \hline 
$\diffcont(\R^d)$ & The space of continuous functions on $\R^d$\\ \hline 
$\diffcont^k(\R^d)$ & The space of $k$ times continuously differentiable functions on $\R^d$ \\ \hline 
$\diffcont^{k,\ell}(\R^d)$ & The space of functions on $[0,\infty)\times \R^d$ (time $\times$ space) with uniformly continuous mixed derivatives up to order $k$ in time and $\ell$ in space \\ \hline 
$B_r(x)$ & The Euclidean ball centered at $x$ with radius $r$ \\ \hline
$\div$ & The Divergence operator\\ \hline 
$\bmd$ & The Laplacian operator\\ \hline
$\nabla$ or $\frac{\partial}{\partial x}$ & The gradient or partial derivative with respect to the usual coordinate chart $x$\\ \hline 
$\nabla^r$ & The $r$-th order multi-derivative with respect to the coordinate chart $x$\\ \hline 
$\int$ & The integral notation without any domain specified will always imply that the integral is over $\R^d$\\ \hline
$\nabla^{-2}$ & The inverse of the Hessian matrix with respect to the coordinate chart $x$\\ \hline 
$\frac{\delta}{\delta \rho}$ & The first variation of a function with respect to a probability measure $\rho$\\ \hline
$\lVert \cdot\rVert_{L^2(\rho)}$ & The $L^2$ norm of a function computed with respect to a probability measure $\rho$\\ \hline
\end{longtable}

%\begin{proof}[Proof of~\cref{thm:conteq}]
%\end{proof}

\begin{comment}

Let 
$\bar{I}(\mu,\Sigma;A):=P( N(\mu,\Sigma)\in A^c).$
Then the above observation yields a further lower bound of 
\begin{align*}
\frac{(2\pi\vep)^{\frac{d}{2}}}{\sqrt{\det(\nabla^2 u(y^{u^*})+\eta I_d)}}\exp\left(\frac{1}{\vep}u^*(y)-f(y^{u^*})+\omega^I_f(y^{u^*};r_y)\right)\left(1-\bar{I}(y^{u^*},\Sigma_{\vep}(y);B_{r_y}(y^{u^*}))\right).
\end{align*}
From elementary computations, it further holds that 
$$\bar{I}(y^{u^*},\vep (\nabla^2 u(y^{u^*})+\eta I_d)^{-1};B_{r_y}(y^{u^*}))\le P(\chi^2_d\ge \vep^{-1}r_y^2\eta).$$
Combining the above observations, we then get:
\begin{align}\label{eq:son1}
   \exp\left(\frac{1}{\vep}\opV[u](y)\right)\ge \frac{(2\pi\vep)^{\frac{d}{2}}}{\sqrt{\det(\nabla^2 u(y^{u^*})+\eta I_d)}}\exp\left(\frac{1}{\vep}u^*(y)-f(y^{u^*})\right)R^{(1)}_{\vep}(y),
\end{align}
where 
$$R^{(1)}_{\vep}(y):=\exp\left(\omega^I_f(y^{u^*};r_y)\right)\left(1-P(\chi^2_d\ge \vep^{-1}r_y^2\eta)\right).$$
Therefore, for $x\in\R^d$, we have:
\begin{align*}
    \exp\left(\frac{1}{\vep}\opS[u](x)\right)\le \frac{1}{(2\pi\vep)^{\frac{d}{2}}}\int \frac{\sqrt{\det(\nabla^2 u(y^{u^*})+\eta I_d)}}{R_{\vep}^{(1)}(y)}\exp\left(\frac{1}{\vep}\langle x,y\rangle-\frac{1}{\vep}u^*(y)-g(y)+f(y^{u^*})\right)\,dy
\end{align*}

$$-\eta I_d\preceq \nabla^2 u(\tilde{y})-\nabla^2 u(y^{u^*})\preceq \eta I_d.$$ 
{\color{blue} We can choose $r_y$ to be free of $y$ if we are on compact domains or on the torus.} 
Then the following holds for all $z\in B_{r_y}(y^{u^*})$:
\begin{align*}
    \langle z,y\rangle - u(z)\ge u^*(y) - \frac{1}{2}(z-y^{u^*})^{\top}\left(\nabla^2 u(y^{u^*})+\eta I_d\right)(z-y^{u^*}).
\end{align*}
For $\delta>0$ and $x,z\in\R^d$, also define 
$$\omega^I_f(z;\delta):=-\delta\sup_{x\in B_{\delta}(z)}\lVert \nabla f(x)\rVert\le \inf_{x\in B_{\delta}(z)} (f(z)-f(x)),$$
and 
$$\Sigma_{\vep}(y):=\vep\left(\nabla^2 u(y^{u^*})+\eta I_d\right)^{-1}.$$

Let $m_{\vep}:=\sqrt{-10\vep\log{\vep}}$. Consequently, 
\begin{align*}
    &\;\;\;\;\;\exp\left(\frac{1}{\vep}\opS[u](x)-\frac{1}{\vep}u(x)\right)\nonumber \\&\le \frac{(C_T)^{\frac{d}{2}}}{(\pi\vep)^{\frac{d}{2}}}\int_{B^c_{m_{\vep}}(x^u)} \frac{1}{R_{\vep}^{(1)}(y)}\exp\left(-\frac{1}{4\vep C_T}\lVert y-x^u\rVert^2-g(y)+f(y^{u^*})\right)\,dy\nonumber \\ &+\frac{1}{(2\pi\vep)^{\frac{d}{2}}}\int_{B_{m_{\vep}}(x^u)} \frac{\sqrt{\det(\nabla^2 u(y^{u^*})+\eta I_d)}}{R_{\vep}^{(1)}(y)}\exp\left(\frac{1}{\vep}(\langle x,y\rangle-u^*(y)-u(x))-g(y)+f(y^{u^*})\right)\,dy\\ &=: T_1+T_2.
\end{align*}
Let $Z_{\vep}\sim N(0,2\vep C_T I_d)$. We focus on the first term $T_1$ above to get:
\begin{align*}
    T_1=2^{\frac{d}{2}} (C_T)^d \E\left[\frac{1}{R_{\vep}^{(1)}(x^u+Z_{\vep})}\exp\left(-g(x^u+Z_{\vep})+f((x^u+Z_{\vep})^{u^*})\right)\bm{1}(\lVert Z_{\vep}\rVert \ge m_{\vep})\right].
\end{align*}
Assume that there exists $\vep_0$ (free of $x$) and $\theta_0:\R^d\to\R$ such that 
\begin{align*}
    \left\{\E\left[\left(\frac{1}{R_{\vep}^{(1)}(x^u+Z_{\vep})}\right)^2\exp\left(-2g(x^u+Z_{\vep})+2f((x^u+Z_{\vep})^{u^*})\right)\right]\right\}^{\frac{1}{2}}\le \theta_0(x). 
\end{align*}
Then
$$T_1\le 2^{\frac{d}{2}}(C_T)^d\sqrt{\theta_0(x)}P\left(\chi^2_d\ge \frac{m_{\vep}^2}{2\vep C_T}\right).$$
By \eqref{eq:basedef}, we have:
\begin{align}\label{eq:temp1}
    &\;\;\;\;\;\exp\left(\frac{1}{\vep}\left(\opV[u_k^{\vep}](y)-\opV[\opS[u_{(k-1)\vep}]](y)\right)\right)\nonumber\\ &=\int\exp\left(-\frac{1}{\vep}(u_k^{\vep}(x)-\opS[u_{(k-1)\vep](x)})\right)\opP[\opS[u_{(k-1)\vep}]](x|y)\,dx,
\end{align}
Next, ideally, we want to look at and simplify
\begin{align*}
&\;\;\;\;\;\int \exp\left(\frac{1}{\vep}\left(\opV[u_k^{\vep}](y)-\opV[\opS[u_{(k-1)\vep}]](y)\right)\right)\opQ[\opV[u_{(k-1)\vep}]](y|z)\,dy\\ &=\int_x \exp\left(-\frac{1}{\vep}(u_k^{\vep}(x)-\opS[u_{(k-1)\vep}](x))\right)\opR[u_{(k-1)\vep}](x|z)\,dx.
\end{align*}
In the last display, we have used \eqref{eq:temp1}. Instead of the term in the first display above, in Soumik da's notes, the term is 
$$\int\exp\left({\color{red}\frac{1}{\vep}\left(\opV[u_k^{\vep}](y)-\opV[u_{k\vep}](y)\right)}\right)\opQ[\opV[u_{k\vep}]](y|z)\,dy.$$
Note that in $\opQ[\opV[u_{k\vep}]](y|z)$, the term $(1/\vep)\opV[u_{k\vep}]$ comes with a negative sign and so, there is no immediate cancellation. Instead, if we had 
\begin{align*}
    &\;\;\;\;\;\int\exp\left({\color{red}\frac{1}{\vep}\left(\opV[u_{k\vep}](y)-\opV[u_k^{\vep}](y)\right)}\right)\opQ[\opV[u_{k\vep}]](y|z)\,dy\\ &=\exp\left(\frac{1}{\vep}\left(u_{k+1}^{\vep}(x)-\opS[u_{k\vep}](x)\right)\right).
\end{align*}
The main issue is not so much with the indexing but the signs going the other way in the exponential.

\medskip

\hrule

\medskip

\SP{SP: New calculations}

\bigskip

We assume nice properties of the two PMA solutions $(u_t,\; t\ge 0)$ and $(v_t=u_t^*,\; t\ge 0)$ and Gaussian approximations for the transition kernels obtained from these solutions. 

Consider the titling of log-Sinkhorn iterates and use the notation $v^\vep_{k+1}:=\opV[u_k^\vep]$.  

\begin{equation}\label{eq:vtilt}
   \frac{1}{\vep}\left( v^\vep_{k+1}(y) - \opV[u_{k\vep}](y) \right)=  \log\int \exp\left( -\frac{1}{\vep} \left( u_k^\vep(x) - u_{k\vep}(x)\right) \right) p^{\vep}[u_{k\vep}](x\mid y)dx.  
\end{equation}

Similarly, with $u_{k+1}^\vep:=\opU[v^\vep_{k+1}]$,  
\begin{equation}\label{eq:utilt}
     \frac{1}{\vep}\left( u_{k+1}^\vep(z) - \opU[v_{k\vep}](z) \right)= \log\int \exp\left( -\frac{1}{\vep} \left( v_{k+1}^\vep(y) - v_{k\vep}(y)\right) \right) q^{\vep}[v_{k\vep}](y\mid z)dy.
\end{equation}

We wish to compare $v^\vep_{k+1}$ with $v_{(k+1)\vep}$ and $u^\vep_{k+1}$ with $u_{(k+1)\vep}$. Rearranging the above identities slightly,
\begin{equation}\label{eq:uvtilt2}
\begin{split}
    \frac{1}{\vep}\left( v^\vep_{k+1}(y) - v_{(k+1)\vep}(y) \right)&=  \frac{1}{\vep}\left(\opV[u_{k\vep}](y)- v_{(k+1)\vep}(y)\right)\\
    + &\log \int \exp\left( -\frac{1}{\vep} \left( u_k^\vep(x) - u_{k\vep}(x)\right) \right) p^{\vep}[u_{k\vep}](x\mid y)dx\\
    \frac{1}{\vep}\left( u_{k+1}^\vep(z) - u_{(k+1)\vep} \right)&=  \frac{1}{\vep}\left(\opU[v_{k\vep}](z) - u_{(k+1)\vep}(z)\right)\\
    + &\log \int \exp\left( -\frac{1}{\vep} \left( v_{k+1}^\vep(y) - v_{k\vep}(y)\right) \right) q^{\vep}[v_{k\vep}](y\mid z)dy.
\end{split}
\end{equation}


Now assume that $u_0^\vep=u_0$. For the induction hypothesis, we will assume all four of these below. 
\begin{enumerate}[(i)]
\item A global growth estimate:
\[
\frac{1}{\vep}\left( u^{\vep}_{k}(x) - u_{k\vep}(x) \right) \sim o_\vep(1) \abs{x} + o(\vep), 
\]
This is a shorthand for the claim that there exist two pairs of functions $\gamma_1, \gamma_2$ and $\delta_1, \delta_2:(0, \infty) \rightarrow \R$ such that 
\begin{equation}\label{eq:ihypothesis}
\lim_{\vep \rightarrow 0+} \gamma_i(\vep)=0,\; \lim_{\vep \rightarrow 0+} \vep^{-1} \delta_i(\vep)=0,
\end{equation}
for $i=1,2$, and 
\[
\gamma_1(\vep) \abs{x} + \delta_1(\vep) \le \frac{1}{\vep}\left( u^\vep_{k}(x) - u_{k\vep}(x) \right) \le  \gamma_2(\vep) \abs{x} + \delta_2(\vep).
\]
In what follows below, the functions $\gamma_i, \delta_i$ will be assumed to be uniform in $k \in [T/\vep]$ for a given $T>0$. 

\item A local estimate:
\[
\int \frac{1}{\vep}\left( u^\vep_{k}(x) - u_{k\vep}(x) \right) p^\vep_{u_{k\vep}}(x \mid y)dx= o(\vep)
\]
\item A global estimate on PMA:
\[
\frac{1}{\vep}\left( \opV[u_{k\vep}](y) - v_{k\vep}(y) \right) \sim o_\vep(1) \abs{y} + o(\vep), 
\]
{\color{blue}We should probably have $v_{(k+1)\vep}$ above as we are comparing $\opV[u_0]$ and $v_{\vep}$ in the first step. So e.g.,
\begin{equation}\label{eq:centering}
\frac{1}{\vep}\left( \opV[u_{0}](y) - v_{\vep}(y) \right)=o_{\vep}(1)+\frac{1}{\vep}\left(\opV[u_0](y)-w_0-\vep {w_0}\right)=o_{\vep}(1)-g(y)-\frac{1}{2}\ldet\frac{\partial y^{w_0}}{\partial y\hfill}.
\end{equation}
}
\item A local estimate on PMA:
\[
\int \frac{1}{\vep}\left( \opV[u_{k\vep}](y) - v_{k\vep}(y) \right) q^\vep[v_{k\vep}](y\mid z) dy= o(\vep). 
\]
{\color{blue}Then \eqref{eq:centering} suggests that we should center the above integral with the term 
$$g(z^{u_0})+\frac{1}{2}\ldet\left(\frac{\partial y^{w_0}}{\partial y\hfill}\right)\bigg|_{y=z^{u_0}}.$$}
\end{enumerate}



By assumption \eqref{eq:ihypothesis} is true for $k=0$. To make the following calculations simpler, we will replace each integral with respect to $p^\vep[u_{k\vep}]$ and $q^\vep[v_{k\vep}]$ by their corresponding normal approximation. This will add an $e^{o(\vep)}$ correction term, which will be absorbed by the integrand. Moreover, we use the well known identity that if $Z$ is a multidimensional normal random variable with mean $O(\vep)$ and variance $O(\vep)$, then 
\begin{equation}\label{eq:normalmgf}
\log \E(e^{ o_\vep(1)\abs{Z} + o(\vep)})= o(\vep^2) + o(\vep)= o(\vep). 
\end{equation}  

Now, by the first identity in \eqref{eq:uvtilt2} for $k=0$, we get
\[
\frac{1}{\vep}\left( v^\vep_{1}(y) - v_\vep(y) \right)= \frac{1}{\vep}\left( \opV[u_0](y) - v_\vep(y) \right) \sim o_\vep(1)\abs{y}.
\]




Assume that the induction hypotheses hold for $(u_k^\vep, u_{k\vep})$ pair. Then, we prove that it holds for the pairs $(v_{k+1}^\vep, v_{k\vep})$ and then $(u_{k+1}^\vep, u_{(k+1)\vep})$. First by \eqref{eq:uvtilt2} and our assumption, 
\[
\begin{split}
 \frac{1}{\vep}\left( v^\vep_{k+1}(y) - v_{(k+1)\vep}(y) \right)&=  \frac{1}{\vep}\left(\opV[u_{k\vep}](y)- v_{(k+1)\vep}(y)\right)\\
    + &\log \int \exp\left( -\frac{1}{\vep} \left( u_k^\vep(x) - u_{k\vep}(x)\right) \right) p^{\vep}[u_{k\vep}](x\mid y)dx\\
    &\sim o_\vep(1) \abs{y} + o(\vep) + \log \int \exp\left( o_\vep(1) \abs{x} + o(\vep)\right) p^\vep_{u_{k\vep}}(x\mid y)dx\\
    &\sim o_\vep(1) \abs{y} + o(\vep).
\end{split}
\]
The final $o(\vep)$ comes from our established bound that under $p^\vep_{u_{k\vep}}$, the random variable (\SP{has to be properly centered}) has mean $O(\vep)$ and variance $O(\vep)$.

    \emph{Proof of \eqref{eq:prestim2}.} We proceed in the same way as in the proof of \eqref{eq:prestim1}. Observe that 
    \begin{align*}
    &\;\;\;\;\E_{\rho_y}(X)-y^w\\ &=\frac{\frac{\sqrt{\mathrm{det}(\nabla^2 u(y^w))}}{(2\pi\vep)^{d/2}}\int (x-y^w)\exp\left(\frac{1}{\vep}D[u](x|y) - G(x) + G(y^w)\right)\,dx}{\frac{\sqrt{\mathrm{det}(\nabla^2 u(y^w))}}{(2\pi\vep)^{d/2}}\int \exp\left(\frac{1}{\vep}D[u](x|y) - G(x) + G(y^w)\right)\,dx}.
    \end{align*}
    We begin with the numerator in the above expression. Note that the denominator has already been dealt with in the proof of \eqref{eq:prestim1}. Note that: 
    \begin{align}\label{eq:gaussbreak2}
        &\;\;\;\;\frac{\sqrt{\mathrm{det}(\nabla^2 u(y^{w}))}}{(2\pi\vep)^{\frac{d}{2}}}\int\limits_{B_{r_{\vep}}(y^{w})} (x-y^w)\exp\left(\frac{1}{\vep}D[u](x|y)-G(x)+G(y^{w})\right)\,dx\nonumber \\ &=\E\bigg[\sqrt{\vep}Z^{(1)}_y\left(\sum_{m=0}^{\infty}\frac{1}{\vep^m m!}(-T[u:3](\tilde{Z}^{(1)}_{\vep,y}|y^{w}))^m\right)\left(\sum_{m=0}^{\infty}\frac{1}{\vep^m m!}(- T[u:4](\tilde{Z}^{(1)}_{\vep,y}|y^{w}))^m\right)\nonumber \\ &\qquad \left(\sum_{m=0}^{\infty}\frac{1}{m!}(-R[h:4](\tilde{Z}^{(1)}_{\vep,y}|y^{w}))^m\right)\left(\sum_{m=0}^{\infty}\frac{1}{m!}(-T[G:1](\tilde{Z}^{(1)}_{\vep,y}|y^{w}))^m\right)\nonumber \\  &\qquad \left(\sum_{m=0}^{\infty}\frac{1}{m!}(-T[G:2](\tilde{Z}^{(1)}_{\vep,y}|y^{w}))^m\right)\left(\sum_{m=0}^{\infty}\frac{1}{m!}(-R[G:2](\tilde{Z}^{(1)}_{\vep,y}|y^{w}))^m\right)\nonumber \\ &\qquad\mathbf{1}(\sqrt{\vep}Z^{(1)}_{y}\in B_{r_{\vep}}(0))\bigg].
    \end{align}
    We focus on two terms in the above expectation as provided below:
    \begin{align*}
        &\;\;\;\;\E\left[\sqrt{\vep}Z_y^{(1)}\left(1-\frac{1}{\vep}T[u:3](\tilde{Z}_{\vep,y}^{(1)}|y^w)\right)\left(1-T[G:1]\right)\right]
        \\ &=-\vep \E\big(Z_y^{(1)}T[u:3](Z^{(1)}_y|0)\big)-\E \big(Z_y^{(1)} T[G:1](Z^{(1)}_y|0)\big)+O(\vep^{3/2}(-\log{\vep})^{3/2})\\ &=\frac{\vep}{2}\nabla(\ldet(\nabla^2 w))(y)-\vep\nabla(G(\nabla w))(y)+O(\vep^{3/2}(-\log{\vep})^{3/2}).
    \end{align*}
    We can repeat the same techniques as in the proof of \eqref{eq:prestim1} to handle the other terms in the aforementioned expansion, as well as the integral over $B_{r_{\vep}}(y^w)$. This implies:
     \begin{align*}
        &\;\;\;\;\frac{\sqrt{\mathrm{det}(\nabla^2 u(y^{w}))}}{(2\pi\vep)^{\frac{d}{2}}}\int\limits_{B_{r_{\vep}}(y^{w})} (x-y^w)\exp\left(\frac{1}{\vep}D[u](x|y)-G(x)+G(y^{w})\right)\,dx \\ &=\frac{\vep}{2}\nabla(\ldet(\nabla^2 w))(y)-\vep\nabla(G(\nabla w))(y)+O(\vep^{3/2}(-\log{\vep})^{3/2})+O(\vep^{3/2}(\omega_1\vee\omega_2)(r_{\vep})).
    \end{align*}
    Combining the above observation with \eqref{eq:prestim1} completes the proof of \eqref{eq:prestim2}.
Similarly, 
\[
\int \frac{1}{\vep}\left( v^\vep_{k+1}(y) - v_{(k+1)\vep}(y) \right) q^{\vep}_{v_{k\vep}}(y \mid z) dy = \int \frac{1}{\vep}\left(\opV[u_{k\vep}](y)- v_{(k+1)\vep}(y)\right) q^{\vep}_{v_{k\vep}}(y \mid z) dy + o(\vep).
\]
That the entire integral above is $o(\vep)$ now follows from the induction hypothesis. 

Repeating the same argument from $v_k^\vep$ to $v^\vep_{k+1}$ gives us the induction step. 

{\color{blue} KL and more.}

TBA. 
    
    Combining the above equality with \eqref{eq:gradbound4} and \cref{lem:proxim}, we get:
    $$\sup_y \bigg|\opV[u_0](y)-w_0(y)-\frac{\vep d}{2}\log{(2\pi\vep)}+\vep f(y^{w_0})-\frac{\vep}{2}\ldet(\nabla^2 w_0(y))+\vep^2 M^{(1)}_{0,y}\bigg|=O(\vep^{9/4}).$$
    We now move on to the gradient of $\opV[u_0](\cdot)$. We first observe that 
    \begin{align*}
    &\;\;\;\;\nabla \opV[u_0](y)-y^{w_0}\\ &=\int (x-y^{w_0})\exp\left(\frac{1}{\vep}\langle x,y\rangle-\frac{1}{\vep}u_0(x)-\frac{1}{\vep}\opV[u_0](y)-f(x)\right)\,dx \\ &=\exp(-\mcI_0(y))\frac{\sqrt{\mathrm{det}(\nabla^2 u_0(y^{w_0}))}}{(2\pi\vep)^{\frac{d}{2}}}\int (x-y^{w_0})\exp\bigg(\frac{1}{\vep}D[u_0](x|y)-f(x)+f(y^{w_0})\bigg)\,dx\\ &=\left(-\vep \nabla (f(\nabla w_0))(y)+\frac{\vep}{2}\nabla(\ldet(\nabla^2 w_0))(y)+O(\vep^2)\right)\exp(-\mcI_0(y)).
    \end{align*}
    Consequently, 
    $$\lVert \nabla \mcI_0(y)\rVert_{\infty}=O(\vep).$$
    Next we discuss the Hessian. Observe that:
    \begin{align*}
        &\;\;\;\;\nabla^2 \opV[u_0](y)\\ &=\frac{1}{\vep}\int (x-y^{w_0})(x-y^{w_0})^{\top}\exp\left(\frac{1}{\vep}\langle x,y\rangle-\frac{1}{\vep}u_0(x)-\frac{1}{\vep}\opV[u_0](y)-f(x)\right)\,dx\\ &\qquad -\frac{1}{\vep}(\nabla \opV[u_0](y)-y^{w_0})(\nabla \opV[u_0](y)-y^{w_0})^{\top}\\ &=\nabla^2 w_0(y)+O(\vep).
    \end{align*}
    This implies 
    $$\lVert \nabla^2 \mcI_0(y)\rVert_{\infty}=O(1).$$
    We now do similar computations for $\mcJ_{\vep}(x)$, $\nabla \mcJ_{\vep}(x)$, and $\nabla^2 \mcJ_{\vep}(x)$. Towards this direction, observe that 
    \begin{align*}
        &\;\;\;\;\;\exp\left(\mcJ_{\vep}(x)+\mcI_0(x^{u_0})\right)\\ &=\frac{1}{(2\pi\vep)^{\frac{d}{2}}\sqrt{\mathrm{det}(\nabla^2 u_{0}(x))}}\int \exp\bigg(\frac{1}{\vep}\langle x,y\rangle - \frac{1}{\vep}u_{0}(x)-\frac{1}{\vep} w_{0}(y)-\log\sqrt{\frac{\mathrm{det}(\nabla^2 w_{0}(y))}{\mathrm{det}(\nabla^2 w_{0}(x^{u_{0}}))}}\nonumber \\ &\qquad +f(y^{w_{0}})-f(x)-g(y)+g(x^{u_{0}})-\mcI_{0}(y)+\mcI_0(x^{u_0})\bigg)\,dy\\ &=1+O(\vep)(1+\lVert \nabla \mathcal{I}_0\rVert_{\infty}^2+\lVert \nabla^2\mathcal{I}_0\rVert_{\infty}).
    \end{align*}
    Consequently,
    $$\lVert \mathcal{J}_{\vep}(x)+\mathcal{I}_0(x^{u_0})\rVert_{\infty}\le O(\vep)(1+\lVert \nabla \mathcal{I}_0\rVert_{\infty}^2+\lVert \nabla^2\mathcal{I}_0\rVert_{\infty}).$$
    Next, we observe that
    \begin{align*}
        &\;\;\;\;\nabla u_1^{\vep}(x)-x^{u_0}\\ &=\int (y-x^{u_0})\exp\left(\frac{1}{\vep}\langle x,y\rangle-\frac{1}{\vep}u_1^{\vep}(x)-\frac{1}{\vep}\opV[u_0](y)-g(y)\right)\,dy\\ &=\exp\left(-\mcJ_{\vep}(x)-\mcI_0(x^{u_0})\right)\frac{\sqrt{\mathrm{det}(\nabla^2 w_0(x^{u_0}))}}{(2\pi\vep)^{\frac{d}{2}}}\int (y-x^{u_0})\exp\bigg(\frac{1}{\vep}\langle x,y\rangle - \frac{1}{\vep} u_0(x) - \frac{1}{\vep} w_0(y) \\ & \qquad - f(x) + f(y^{w_0})  -\log\sqrt{\frac{\ldet(\nabla^2 w_0(y))}{\ldet(\nabla^2 w_0(x^{u_0}))}}-g(y)+g(x^{u_0})-\mcI_0(y)+\mcI_0(x^{u_0})\bigg)\,dy\\ &=\left(1+O(\vep)(\lVert \nabla \mathcal{I}_0\rVert_{\infty}^2+\lVert \nabla^2\mathcal{I}_0\rVert_{\infty})\right)\big(\vep\nabla(f(x)-g(x^{u_0})+\ldet(\nabla^2 u_0(x)))+O(\vep)\lVert \nabla \mcI_0\rVert_{\infty}\big).
    \end{align*}
    In a similar vein, we also have:
    \begin{align*}
        &\;\;\;\;\nabla^2 u_1^{\vep}(x)\\ &=\frac{1}{\vep}\int (y-x^{u_0})(y-x^{u_0})^{\top}\exp\left(\frac{1}{\vep}\langle x,y\rangle-\frac{1}{\vep}u_1^{\vep}(x)-\frac{1}{\vep}\opV[u_0](y)-g(y)\right)\,dy\\ &\qquad -\frac{1}{\vep}\big(\nabla u_1^{\vep}(x)-\nabla u_0(x)\big)\big(\nabla u_1^{\vep}(x)-\nabla u_0(x)\big)^{\top}\\ &=\nabla^2 u_0(x)+O(\vep)(\lVert \nabla \mathcal{I}_0\rVert_{\infty}^2+\lVert \nabla^2\mathcal{I}_0\rVert_{\infty}).
    \end{align*}

\noindent Generate $X_0^{\vep}\sim (\nabla u_0^*)_{\# e^{-g}}$. Given $X_0^{\vep}=x$, sample $Y_1^{\vep}\sim \opQ[\opV[u_0^{\vep}]](\cdot|x)$ and given $Y_1^{\vep}=y$, sample $X_1^{\vep}\sim \opP[\opS[u_0]](\cdot|y)$. In generally, given $X^{\vep}_{i}=x$, sample $Y_{i+1}^{\vep}\sim \opQ[\opV[u^{\vep}_i]](\cdot|x)=:\opQ_i(\cdot|x)$ and given $Y_{i+1}^{\vep}=y$, sample $X_{i+1}^{\vep}\sim \opP[\opS[u_i^{\vep}]](\cdot|y)=:\opP_i(\cdot|y)$. We proceed similarly to construct the Markov chain $(X_0^{\vep},Y_1^{\vep},X_1^{\vep},\ldots ,X_{k}^{\vep})$.

\noindent We generate the corresponding Markov chain with the PMA (see \eqref{eq:pma}). Given $X_{k\vep}=x$, sample 
$$Y_{(k+1)\vep}\ |\ X_{k\vep}=x\sim  N\left(x^{u_{k\vep}}+\vep\left(\frac{\partial f}{\partial x}(x)-\frac{1}{2}\frac{\partial g}{\partial x}(x^{u_{k\vep}})-\frac{1}{2}\frac{\partial h_{k\vep}}{\partial x\hfill}(x)\right),\ \vep \left(\frac{\partial x^{u_{k\vep}}}{\partial x\hfill}\right)\right).$$
We write the corresponding transition kernel by $\tpQ{k}(\cdot|x)$. Recall that $w_t=u_t^*$. Given $Y_{(k+1)\vep}=y$, sample 
$$X_{(k+1)\vep}\ |\ Y_{(k+1)\vep}=y\sim  N\left(y^{w_{k\vep}}-\vep\left(2\frac{\partial f}{\partial y}(y^{w_{k\vep}})-\frac{\partial h_{k\vep}}{\partial y}(y^{w_{k\vep}})\right),\ \vep \left(\frac{\partial y^{w_{k\vep}}}{\partial y\hfill}\right)\right).$$
We write the corresponding transition kernel by $\tpP{k}(\cdot|y)$. We start with $X_0=X_0^{\vep}$ as in the preceding paragraph. This yields the Markov chain $(X_0,Y_{\vep},X_{\vep},\ldots , X_{k\vep})$. 


\noindent Throughout this part, we assume that there exists fixed $T\in (0,\infty)$ such that $0<k\vep\le T$.

\noindent The goal is to show that 
\begin{equation}\label{eq:kldgoal}
\KL{(X_0,Y_{\vep},X_{\vep},\ldots , X_{k\vep})}{(X_0^{\vep},Y_1^{\vep},X_1^{\vep},\ldots ,X_{k}^{\vep})}=o_{\vep}(1),
\end{equation}
where the above limit is to be understood as $\vep\to 0$ and $k\vep\le T$. Recall that $X_0$ and $X_0^{\vep}$ have the same distribution. We want to eventually proceed inductively to prove \eqref{eq:kldgoal}. 

\vspace{0.05in}

\noindent \emph{$k=1$ case}. We begin with 
\begin{align*}
&\;\;\;\;\;\KL{(X_0,Y_{\vep},X_{\vep})}{(X_0^{\vep},Y_{1}^{\vep},X_{1}^{\vep})}\\ &=\E_{Y\sim Y_{\vep}} \left[\KL{\tpP{ }(\cdot|Y)}{\opP_1(\cdot|Y)}\right]+\E_{X\sim X_0}\left[\KL{\tpQ{ }(\cdot|X)}{\opQ_1(\cdot|X)}\right]
\end{align*}
The following lemma provides a series of recursions that express the Sinkhorn potentials \eqref{eq:twostepit}  explicitly in terms of the PMA \eqref{eq:pma} and the dual PMA \eqref{eq:dualPMA}. We set up some notation first. Define 
$$\mcI_0(y):=\log\frac{1}{(2\pi\vep)^{\frac{d}{2}}\sqrt{\mathrm{det}(\nabla^2 w_0(y))}}\int \exp\left(\frac{1}{\vep}\langle x,y\rangle - \frac{1}{\vep}u_0(x)-\frac{1}{\vep}w_0(y)-f(x)+f(y^{w_0})\right)\,dx,$$
and $\mcJ_{0}(x)=0$. 
Finally for $k\ge 0$, define:
$$\tilde{u}_{(k+1)\vep}(x)=u_{k\vep}(x)+\vep(f(x)-g(x^{u_{k\vep}})+\ldet(\nabla^2 u_{k\vep}(x))),$$
\begin{align*}
    \mcJ_{(k+1)\vep}(x)&:=\log\frac{1}{(2\pi\vep)^{\frac{d}{2}}\sqrt{\mathrm{det}(\nabla^2 u_{k\vep}(x))}}\int \exp\bigg(\frac{1}{\vep}\langle x,y\rangle - \frac{1}{\vep}u_{k\vep}(x)-\frac{1}{\vep} w_{k\vep}(y)\nonumber \\ &\qquad -\log\sqrt{\frac{\mathrm{det}(\nabla^2 w_{k\vep}(y))}{\mathrm{det}(\nabla^2 w_{k\vep}(x^{u_{k\vep}}))}}+f(y^{w_{k\vep}})-f(x)-g(y)+g(x^{u_{k\vep}})-\mcI_{k\vep}(y)\bigg)\,dy,
\end{align*}
\begin{align*}
    \mcI_{(k+1)\vep}(y)&:=\log\frac{1}{(2\pi\vep)^{\frac{d}{2}}\sqrt{\mathrm{det}(\nabla^2 w_{(k+1)\vep}(y))}}\int \exp\bigg(\frac{1}{\vep}\langle x,y\rangle - \frac{1}{\vep}\tilde{u}_{(k+1)\vep}(x)-\frac{1}{\vep}w_{(k+1)\vep}(y)\nonumber \\ &\qquad -f(x)+f(y^{w_{(k+1)\vep}})-\mcJ_{(k+1)\vep}(x)\bigg)\,dy,
\end{align*}

\begin{remark}
Note that both the $\mcI_{k\vep}$s and $\mcJ_{k\vep}$s are all purely functions of the PMA \eqref{eq:pma} and the dual PMA \eqref{eq:dualPMA}.
\end{remark}

\begin{lmm}\label{lem:proxim}
Suppose $u_0(\cdot)$ is the initial potential for both the Sinkhorn algorithm and the PMA. Then the following relations hold for all $k\ge 0$ and $x,y\in\R^d$:
\begin{equation}\label{eq:system3}
\begin{split}
\opV[u_{k}^{\vep}](y)=w_{k\vep}(y)+\frac{\vep d}{2}\log{(2\pi\vep)}-\vep f(y^{w_{k\vep}})+\frac{\vep}{2}\ldet(\nabla^2 w_{k\vep}(y))+\vep \mcI_{k\vep}(y),
\end{split}
\end{equation}
and 
\begin{equation}\label{eq:system4}
    u_{k+1}^{\vep}(x)=\tilde{u}_{(k+1)\vep}(x)+\vep \mcJ_{(k+1)\vep}(x).
\end{equation}
\end{lmm}


\begin{proof}
    The proof of \eqref{eq:system3} and \eqref{eq:system4} proceeds via induction. \par 

    First consider the case when $k=0$. Observe that 
    \begin{align*}
        \opV[u_0](y)&=\vep\log\int \exp\left(\frac{1}{\vep}\langle x,y\rangle - \frac{1}{\vep}u_0(x)-f(x)\right)\,dx\\ &=w_0(y)+\frac{\vep d}{2}\log{(2\pi\vep)}-\vep f(y^{w_0})-\frac{\vep}{2}\ldet(\nabla^2 u_0(y^{w_0})) \\ & + \vep \log\frac{\sqrt{\mathrm{det}(\nabla^2 u_0(y^{w_0}))}}{(2\pi\vep)^{\frac{d}{2}}}\int \exp\left(\frac{1}{\vep}\langle x,y\rangle - \frac{1}{\vep}w_0(y)-\frac{1}{\vep}u_0(x)-f(x)+f(y^{w_0})\right)\,dx.
    \end{align*}
    Noting that  $\nabla^2 u_0(y^{w_0})=\nabla^{-2}w_0(y)$, establishes \eqref{eq:system3} for $k=1$. Next, 
    \begin{align*}
        &\;\;\;\;u_1^{\vep}(x)-\tilde{u}_{\vep}(x)\\ &=\vep\log\int \exp\left(\frac{1}{\vep}\langle x,y\rangle - \frac{1}{\vep}\opV[u_0](y)-\frac{1}{\vep}u_{0}(x)-g(y)-f(x)+g(x^{u_0})-\ldet(\nabla^2 u_0(x))\right)\,dy\\ &=\vep\log\frac{1}{(2\pi\vep)^{\frac{d}{2}}}\int \exp\bigg(\frac{1}{\vep}\langle x,y\rangle - \frac{1}{\vep}w_0(y)-\frac{1}{\vep}u_0(x)+f(y^{w_0})-\frac{1}{2}\ldet(\nabla^2 w_0(y))\\ &\qquad\qquad -g(y)-f(x)+g(x^{u_0})-\ldet(\nabla^2 u_0(x))-\mcI_0(y)\bigg)\,dy\\ &=\vep\log\frac{1}{(2\pi\vep)^{\frac{d}{2}}\sqrt{\mathrm{det}(\nabla^2 u_{0}(x))}}\int \exp\bigg(\frac{1}{\vep}\langle x,y\rangle - \frac{1}{\vep}u_{0}(x)-\frac{1}{\vep}w_0(y)-g(y)+g(x^{u_0})\\ &\qquad +f(y^{w_0})-f(x)-\log\sqrt{\frac{\mathrm{det}(\nabla^2 w_{0}(y))}{\mathrm{det}(\nabla^2 w_{0}(x^{u_{0}}))}}-\mcI_{0}(y)\bigg)\,dy\\ &=\vep \mcJ_{\vep}(x).
    \end{align*}
    where we have used $\nabla^2 w_{\vep}(x^{u_{\vep}})=\nabla^{-2} u_{\vep}(x)$. This establishes \eqref{eq:system4} for the case $k=1$.\par 

    We now assume \eqref{eq:system3} and \eqref{eq:system4} for all $k\le k_0$. For $k=k_0+1$, we then have from the induction hypothesis:
    \begin{align*}
        \opV[u_{k_0+1}^{\vep}](y)&=\vep\log\int\exp\left(\frac{1}{\vep}\langle x,y\rangle - \frac{1}{\vep}\tilde{u}_{k_0+1}^{\vep}(x)-f(x)\right)\,dx\\ &=\vep \log\int \exp\left(\frac{1}{\vep}\langle x,y\rangle - \frac{1}{\vep}\tilde{u}_{(k_0+1)\vep}(x)-f(x)-\mcJ_{(k_0+1)\vep}(x)\right)\,dx\\ &=w_{(k_0+1)\vep}(y)+\frac{\vep d}{2}\log{(2\pi\vep)}-\vep f(y^{w_{(k_0+1)\vep}})+\frac{\vep}{2}\ldet(\nabla^2 w_{(k_0+1)\vep}(y))\\ &+\vep \log\frac{1}{(2\pi\vep)^{\frac{d}{2}}\sqrt{\mathrm{det}(\nabla^2 w_{(k_0+1)\vep}(y))}}\int \exp\bigg(\frac{1}{\vep}\langle x,y\rangle - \frac{1}{\vep}\tilde{u}_{(k_0+1)\vep}(x)\nonumber \\ &-\frac{1}{\vep}w_{(k_0+1)\vep}(y) -f(x)+f(y^{w_{(k_0+1)\vep}(y)})-\mcJ_{(k_0+1)\vep}(x)\bigg)\,dy
    \end{align*}
    This establishes \eqref{eq:system3} for $k=k_0+1$. A similar computation shows \eqref{eq:system4} holds for $k=k_0+1$ as well, thereby completing the proof.
\end{proof}

\noindent The following technical result is a version of Laplace's method of approximating integrals. We skip the details of the proof for brevity.

\begin{lmm}
  Recall that $\mu(dx)=\exp(-f(x))$ is a probability measure, and assume that $\lVert \nabla f\rVert_{\infty}\vee \lVert \nabla^2 f\rVert_{\infty}<\infty$. Also assume that $u:\R^d\to\R^d$ is a smooth function with $\inf_{x}\lmn(\nabla^2(u(x)))>0$, and $\lVert \nabla^3 u\rVert_{\infty}\vee \lVert \nabla^4 u\rVert_{\infty}<\infty$. Define 
  $$h_{\vep}[u](y)=\int \exp\left(\frac{1}{\vep}\langle x,y\rangle - \frac{1}{\vep} u(x)-f(x)\right)\,dx.$$
  Then there exists a constant 
  $$C\equiv C\bigg(\lVert \nabla f\rVert_{\infty},\ \lVert \nabla^2 f\rVert_{\infty},\ \inf_{x}\lmn(\nabla^2(u(x))),\ \lVert \nabla^3 u\rVert_{\infty},\ \lVert \nabla^4 u\rVert_{\infty}\bigg),$$
  such that for all $0<\vep\le 1/2$, then,
  $$\sup_{y}\bigg|\exp\bigg(-\frac{u^*(y)}{\vep}+f(y^{u^*})\bigg)\frac{h_{\vep}[u](y)}{(2\pi\vep)^{\frac{d}{2}}}\sqrt{\det(\nabla^2 u(y^{u^*}))}-1\bigg|\le C\vep \left(\log{\left(\frac{1}{\vep}\right)}\right)^3.$$
\end{lmm}

\noindent The following technical lemma is a refinement of \cite[Proposition 1.11, page 287]{Revuzyor}

\begin{lmm}\label{lem:Revyor1}
Given a function $\phi:\R^d\to\R$ which is twice continuously differentiable and satisfies 
$$\sup_{x\in\R^d} \lVert \nabla^2\phi(x)\rVert_{\infty}<\infty.$$
Also assume that $\nabla^2\phi(\cdot)$ is a uniformly continuous function on $\R^d$. 
Then,
$$\lim_{t\to 0} \bigg\lVert\frac{e^{-4t\lVert \nabla\phi\rVert^2}}{1+t^{1/4}\lVert \nabla \phi\rVert^3}\bigg|\frac{e^{\phi}P_t(e^{-\phi})-1}{t}-\frac{1}{2}\left(-\Delta \phi+\lVert \nabla \phi\rVert^2\right)\bigg|\bigg\rVert_{\infty}=0.$$
Here $$P_t(e^{-\phi})(x)=\E e^{-\phi(x+\sqrt{t}Z)},$$
where $Z\sim N(0,\mathrm{I_d})$.
\end{lmm}

\begin{proof}
     Assume $C:=\sup_{x\in\R^d} \lVert \nabla^2\phi(x)\rVert_{\infty}<\infty$ and define 
    $g(\delta):=\sup_{\lVert x-y\rVert\le \delta}\lVert \nabla^2 \phi(x)-\nabla^2\phi(y)\rVert_{\mathrm{op}}$. By our assumption, $g(\delta)\to 0$ as $\delta\to 0$.  Throughout this proof, we will hide constants depending on $d$ by the generic notation $C_A$ which could change from one line to the next. The idea is to keep taking the maximums of the constants that arise in each step and keep redefining $C_A$ as this maximum. Note that $C_A$ does not depend on $\phi(\cdot)$.\par

    
    By a second order Taylor approximation, we have:
    \begin{align*}
        \phi(x+\sqrt{t}Z)=\phi(x)+\sqrt{t}\nabla \langle\phi(x),Z\rangle+\frac{t}{2}Z^{\top}\nabla^2\phi(x)Z+\frac{t}{2}Z^{\top}(D_t-\nabla^2\phi(x))Z,
    \end{align*}
    where
    $$D_t:=2\int_0^1 (1-\lambda)\nabla^2\phi(x+\lambda\sqrt{t}Z)\,d\lambda.$$
    The above integral is to be interpreted entrywise. We hide the dependence on $x,Z,\phi$ in our notation for simplicity. Note that $D_t$ is a measurable function in $Z$ under our assumptions. Further, given $R>0$ and on the set $\lVert Z\rVert\le R$, the following estimate holds: 
    \begin{align}\label{eq:calllate}
    \sup_{x\in\R^d}\lVert D_t-\nabla^2\phi(x)\rVert_{\mathrm{op}}\le C_A g(\sqrt{t}R).
    \end{align}
    Consequently, we can write:
    \begin{align*}
    e^{\phi(x)}P_t(e^{-\phi})(x)-1=\E\big[e^{-\sqrt{t}\langle Z,\nabla \phi(x)\rangle-\frac{t}{2}Z^{\top}\nabla^2\phi(x)Z-\frac{t}{2}Z^{\top}(D_t-\nabla^2\phi(x))Z}\big]-1.
    \end{align*}

    \SP{Don't follow the calculation below. Please expand. Before you take expectation, you need to make sure that $\xi$ is a measurable function of $Z$. Is that obvious?} \ND{Switching to integral form of remainder  where measurability seems to follow from continuity and boundedness of Hessian. More steps added.}
    
    By a full Taylor series expansion of the exponential function, we get:
    \begin{align*}
        &\;\;\;\;\;e^{\phi(x)}P_t(e^{-\phi})(x)-1\\ &=\E\bigg[e^{-\sqrt{t}\langle Z,\nabla \phi(x)\rangle-\frac{t}{2}Z^{\top}\nabla^2\phi(x)Z}\bigg(\sum_{k\ge 0} \frac{(-t)^k}{2^k k!}(Z^{\top}(D_t-\nabla^2\phi(x))Z)^k\bigg)\bigg]-1\\ &=\E\bigg[e^{-\sqrt{t}\langle Z,\nabla \phi(x)\rangle-\frac{t}{2}Z^{\top}\nabla^2\phi(x)Z}\bigg(\sum_{k\ge 1} \frac{(-t)^k}{2^k k!}(Z^{\top}(D_t-\nabla^2\phi(x))Z)^k\bigg)\bigg]\\ &\qquad\qquad +\sum_{k\ge 1} \E\bigg[\frac{\big(-\sqrt{t}\langle Z,\nabla \phi(x)\rangle-\frac{t}{2}Z^{\top}\nabla^2\phi(x)Z\big)^k}{k!}\bigg]\\ &=\E\bigg[e^{-\sqrt{t}\langle Z,\nabla \phi(x)\rangle-\frac{t}{2}Z^{\top}\nabla^2\phi(x)Z}\bigg(\sum_{k\ge 1} \frac{(-t)^k}{2^k k!}(Z^{\top}(D_t-\nabla^2\phi(x))Z)^k\bigg)\bigg]\\ &\qquad\qquad-\frac{t}{2}\Delta \phi(x)+\frac{t}{2}\lVert \nabla \phi(x)\rVert^2+\frac{t^2}{8}\big((\Delta \phi(x))^2+\lVert \nabla^2 \phi(x)\rVert_{\hs}^2\big)\\ &\qquad\qquad +\sum_{k\ge 3} \E\bigg[\frac{\big(-\sqrt{t}\langle Z,\nabla \phi(x)\rangle-\frac{t}{2}Z^{\top}\nabla^2\phi(x)Z\big)^k}{k!}\bigg].
    \end{align*}
    Therefore,
    \begin{align}\label{eq:boundreq}
    &\;\;\;\;\;\bigg|\frac{e^{\phi(x)}P_t(e^{-\phi})(x)-1}{t}-\frac{1}{2}\left(-\Delta \phi(x)+\lVert \nabla \phi(x)\rVert^2\right)\bigg|\nonumber \\&\le \bigg|\E\bigg[e^{-\sqrt{t}\langle Z,\nabla \phi(x)\rangle-\frac{t}{2}Z^{\top}\nabla^2\phi(x)Z}\bigg(\sum_{k\ge 1} \frac{(-1)^k t^{k-1}}{2^k k!}(Z^{\top}(D_t-\nabla^2\phi(x))Z)^k\bigg)\bigg]\bigg|+tC_A\nonumber \\&+\frac{1}{2t}\sum_{k\ge 3}\frac{(2\sqrt{t})^k}{k!}\E|\langle Z,\nabla \phi(x)\rangle|^k+\frac{1}{2}\sum_{k\ge 3}\frac{t^{k-1}}{k!}\E\big[\big(\lVert Z\rVert^2 Cd\big)^k\big].
    \end{align}
    We bound each of the terms on the right hand side of \eqref{eq:boundreq}. Straightforward computations using Gaussian moment generating functions imply, for $t\le (1\wedge (Cd^2)^{-1})/256$, that:
    \begin{align}\label{eq:boundreq1}
    \frac{1}{2t}\sum_{k\ge 3}\frac{(2\sqrt{t})^k}{k!}\E|\langle Z,\nabla \phi(x)\rangle|^k\le C_A\sqrt{t}\lVert \nabla \phi(x)\rVert^3\exp(4t\lVert \nabla \phi(x)\rVert^2),
    \end{align}
    and 
    \begin{align}\label{eq:boundreq2}
    \frac{1}{2}\sum_{k\ge 3}\frac{t^{k-1}}{k!}\big(\lVert Z\rVert^2 Cd\big)^k\le C_A t^2.
    \end{align}
    Next let $$Z_{x,t}=(I+t\nabla^2\phi(x))^{-1}\sqrt{t}\nabla \phi(x)+(I+t\nabla^2\phi(x))^{-1/2}Z.$$
    Then by a change of variable argument, we get:
    \begin{align}\label{eq:boundreq2}
        &\;\;\;\;\;\bigg|\E\bigg[e^{-\sqrt{t}\langle Z,\nabla \phi(x)\rangle-\frac{t}{2}Z^{\top}\nabla^2\phi(x)Z}\bigg(\sum_{k\ge 1} \frac{(-1)^k t^{k-1}}{2^k k!}(Z^{\top}(D_t-\nabla^2\phi(x))Z)^k\bigg)\bigg]\bigg|\nonumber \\ &=\frac{e^{\frac{t}{2}\lVert \nabla \phi(x)\rVert^2}}{2\sqrt{\mbox{det}(I+t\nabla^2\phi(x))}}\E\left[Z_{x,t}^{\top}(D_t-\nabla^2\phi(x))Z_{x,t}e^{\frac{t}{2}Z_{x,t}^{\top}(D_t-\nabla^2\phi(x))Z_{x,t}}\right]\nonumber \\ &\le C_A e^{\frac{t}{2}\lVert \nabla \phi(x)\rVert^2}\E\left[\big|Z_{x,t}^{\top}(D_t-\nabla^2\phi(x))Z_{x,t}\big|e^{tCd\lVert Z_{x,t}\rVert^2}\right]\nonumber \\ &\le C_Ae^{\frac{t}{2}\lVert \nabla \phi(x)\rVert^2}\E\left[\lVert D_t-\nabla^2\phi(x)\rVert_{\mathrm{op}}\lVert Z_{x,t}\rVert^2 e^{tCd\lVert Z_{x,t}\rVert^2}\right].
    \end{align}
We will break the above expectation into two parts with indicators $\lVert Z\rVert\le R$ and $\lVert Z\rVert> R$. On the set $\lVert Z\rVert\le R$, the following holds:
\begin{equation}\label{eq:calate}
\lVert Z_{x,t}\rVert^2\le 8t\lVert \nabla\phi(x)\rVert^2+4\lVert Z\rVert^2.
\end{equation}
Therefore, 
\begin{align}\label{eq:boundreq3}
    &\;\;\;\;e^{\frac{t}{2}\lVert \nabla \phi(x)\rVert^2}\E\left[\lVert D_t-\nabla^2\phi(x)\rVert_{\mathrm{op}}\lVert Z_{x,t}\rVert^2 e^{tCd\lVert Z_{x,t}\rVert^2}\mathbf{1}(\lVert Z\rVert\le R)\right]\nonumber \\&\le C_A\exp(t\lVert \nabla \phi(x)\rVert^2)g(\sqrt{t}R)\left(t\lVert \nabla \phi(x)\rVert^2\E[e^{4tCd\lVert Z\rVert^2}]+\E[\lVert Z\rVert^2e^{4tCd\lVert Z\rVert^2}]\right)\nonumber \\ &\le C_A(1+t\lVert\nabla\phi(x)\rVert^2)e^{t\lVert\nabla\phi(x)\rVert^2}g(\sqrt{t}R).
\end{align}
In the first bound above, we have used \eqref{eq:calllate} and \eqref{eq:calate}. Next, observe that, 
\begin{align}\label{eq:boundreq4}
    &\;\;\;\;e^{\frac{t}{2}\lVert \nabla \phi(x)\rVert^2}\E\left[\lVert D_t-\nabla^2\phi(x)\rVert_{\mathrm{op}}\lVert Z_{x,t}\rVert^2 e^{tCd\lVert Z_{x,t}\rVert^2}\mathbf{1}(\lVert Z\rVert> R)\right]\nonumber \\&\le C_A\exp(t\lVert \nabla \phi(x)\rVert^2)\left(t\lVert \nabla \phi(x)\rVert^2\E[e^{4tCd\lVert Z\rVert^2}\mathbf{1}(\lVert Z\rVert>R)]+\E[\lVert Z\rVert^2e^{4tCd\lVert Z\rVert^2}\mathbf{1}(\lVert Z\rVert>R)]\right)\nonumber \\ &\le C_A(1+t\lVert\nabla\phi(x)\rVert^2)e^{t\lVert\nabla\phi(x)\rVert^2}\exp(-R^2/4d).
\end{align}
In the first bound above, we have used \eqref{eq:calate} and the boundedness of the Hessian. 

We next note the elementary inequality $\lVert \nabla\phi(x)\rVert^2\le 1+\lVert \nabla\phi(x)\rVert^3$.  By combining \eqref{eq:boundreq}, \eqref{eq:boundreq1}, \eqref{eq:boundreq2}, \eqref{eq:boundreq3}, and \eqref{eq:boundreq4}, we have the following bound:
\begin{align*}
    &\;\;\;\;\frac{C_A^{-1} e^{-4t\lVert \nabla\phi(x)\rVert^2}}{1+t^{1/4}\lVert \nabla \phi(x)\rVert^3}\bigg|\frac{e^{\phi(x)}P_t(e^{-\phi})(x)-1}{t}-\frac{1}{2}\left(-\Delta \phi(x)+\lVert \nabla \phi(x)\rVert^2\right)\bigg|\\ &\le \frac{t(1+t)}{1+t^{1/4}\lVert \nabla\phi(x)\rVert^3}+\frac{\sqrt{t}\lVert \nabla \phi(x)\rVert^3}{1+t^{1/4}\lVert \nabla\phi(x)\rVert^3}+\frac{1+t(1+\lVert \nabla\phi(x)\rVert^3)}{1+t^{1/4}\lVert \nabla\phi(x)\rVert^3}\bigg(\exp\bigg(\frac{-R^2}{4d}\bigg)+g(\sqrt{t}R)\bigg)\\ &\le t(1+t)+t^{1/4}+2\bigg(\exp\bigg(\frac{-R^2}{4d}\bigg)+g(\sqrt{t}R)\bigg).
\end{align*}
We can now arrive at the conclusion by taking supremum over $x$ and limits as $t\to 0$ followed by $R\to\infty$.
\end{proof}

For the next result, we define some basic notation first. 
For $y,z\in\R^d$ and $v\in C(\R^d)$, let us define
\begin{equation*}\label{eq:con3}
    \opQ[v](y|z):=\exp\left(\frac{1}{\vep}\langle y,z\rangle-\frac{1}{\vep}v(y)-\frac{1}{\vep}\opU[v](z)-g(y)\right).
\end{equation*}
It is easy to see that, for every given $z$, $\opQ[v](\cdot|z)$ is a probability density. Hence, it is a Markov transition density. 

Similarly, for $x,y\in\R^d$ and $u\in C(\R^d)$, we define
\begin{equation*}
    \opP[u](x|y):=\exp\left(\frac{1}{\vep}\langle x,y\rangle-\frac{1}{\vep}u(x)-\frac{1}{\vep}\opV[u](y)-f(x)\right).
\end{equation*}
As before, $\opP[u](\cdot|y)$ is a Markov transition density for every given $y$. 

We then define the two-step Markov transition density by combining $\opP$ and $\opQ$. For $x,z\in\R^d$ and $u\in C(\R^d)$, as follows:
\begin{equation*}
    \opR[u](x|z):=\int \opQ[\opV[u]](y|z)\opP[\opS[u]](x|y)\,dy.
\end{equation*}
Note that by \eqref{eq:con1} and \eqref{eq:con2}, the following identities hold:
\begin{align*}
    p_{Y|X}\gvp_{k}(y|z)=\opQ[\opV[u_{k-1}^{\vep}]](y|z),\quad p_{X|Y}\gvp_{k+1}(x|y)=\opP[\opS[u_{k-1}^{\vep}]](x|y).
\end{align*}

\begin{lmm}\label{lem:recursion}
Define 
$$\xi_{k,\vep}^{(2)}(x):=\frac{1}{\vep}u_{k\vep}(x)-\frac{1}{\vep}u_{k}^{\vep}(x).$$
For $x_1,x_2\in\R^d$ and any function $\lambda:\R^d\to\R$, define 
$$\mfR{\lambda}{x_1}{x_2}:=\lambda(x_1)-\lambda(x_2).$$
Then the following holds for all $k\ge 0$, $\vep>0$, and $x\in\R^d$:
\begin{align}\label{eq:step15}
\xi_{k+1,\vep}^{(2)}(x)-\xi_{k,\vep}^{(2)}(x)&=\frac{1}{\vep}\left(u_{(k+1)\vep}(x)-\opS[u_{k\vep}](x)\right)\nonumber \\&-\log{\E_{Y\sim \opQ[\opV[u_{k\vep}]](\cdot|x)}\left[\frac{1}{\E_{Z\sim \opP[u_{k\vep}](\cdot|Y)} \exp\left(\mfR{\xi_{k,\vep}^{(2)}}{Z}{x}\right)}\right]}.
\end{align}
\end{lmm}

\begin{proof}
    Note that
\begin{align}\label{eq:step12}
    &\;\;\;\;\;\frac{1}{\vep}\opU[\opV[u_k^{\vep}]](x)-\frac{1}{\vep}\opU[\opV[u_{k\vep}]](x)\nonumber \\ &=\log\int \exp\left(\frac{1}{\vep}\langle x,y\rangle - \frac{1}{\vep}\opU[\opV[u_{k\vep}]](x)-\frac{1}{\vep}\opV[u_k^{\vep}](y)-g(y)\right)\,dy\nonumber \\ &=\log\E_{Y\sim \opQ[\opV[u_{k\vep}]](\cdot|x)} \exp\left(\frac{1}{\vep}\left(\opV[u_{k\vep}](Y)-\opV[u_k^{\vep}](Y)\right)\right)\nonumber \\ &=\xi_{k,\vep}^{(1)}(x^{u_{k\vep}})+\log\E_{Y\sim \opQ[\opV[u_{k\vep}]](\cdot|x)} \exp\left(\mfR{\xi_{k,\vep}^{(1)}}{Y}{x^{u_{k\vep}}}\right),
\end{align}
where for all $y\in\R^d$, we define
$$\xi_{k,\vep}^{(1)}(y):=\frac{1}{\vep}\opV[u_{k\vep}](y)-\frac{1}{\vep}\opV[u_k^{\vep}](y).$$
Next, let us fix $y\in\R^d$, and recall that $w_{k\vep}=u_{k\vep}^*$. Note that by a similar computation as above, we can write:
\begin{align}\label{eq:step13}
    -\xi_{k,\vep}^{(1)}(y)&=\frac{1}{\vep}\opV[u_k^{\vep}](y)-\frac{1}{\vep}\opV[u_{k\vep}](y)\nonumber \\ &=\xi_{k,\vep}^{(2)}(y^{w_{k\vep}})+\log\E_{Z\sim \opP[u_{k\vep}](\cdot|y)} \exp\left(\mfR{\xi_{k,\vep}^{(2)}}{Z}{y^{w_{k\vep}}}\right).
\end{align}

By combining \eqref{eq:step12} and \eqref{eq:step13}, we then get that for any $x\in\R^d$
\begin{align}\label{eq:step14}
    &\;\;\;\;\;\frac{1}{\vep}u_{k+1}^{\vep}(x)-\frac{1}{\vep}\opS[u_{k\vep}](x)\nonumber \\ &=-\xi_{k,\vep}^{(2)}(x)-\log\E_{Z\sim \opP[u_{k\vep}](\cdot|x^{u_{k\vep}})} \exp\left(\mfR{\xi_{k,\vep}^{(2)}}{Z}{x}\right)\\ \nonumber &\quad\quad+\log\E_{Y\sim \opQ[\opV[u_{k\vep}]](\cdot|x)} \exp\left(\mfR{\xi_{k,\vep}^{(1)}}{Y}{x^{u_{k\vep}}}\right).
\end{align}
To simplify the right hand side above further, fix any $y\in\R^d$ and use \eqref{eq:step13} to get that:
\begin{align*}
    \mfR{\xi_{k,\vep}^{(1)}}{y}{x^{u_{k\vep}}}&=\xi_{k,\vep}^{(2)}(x)-\xi_{k,\vep}^{(2)}(y^{w_{k\vep}})-\log\E_{Z\sim \opP[u_{k\vep}](\cdot|y)} \exp\left(\mfR{\xi_{k,\vep}^{(2)}}{Z}{y^{w_{k\vep}}}\right)\\ &\quad\quad+\log\E_{Z\sim \opP[u_{k\vep}](\cdot|x^{u_{k\vep}})} \exp\left(\mfR{\xi_{k,\vep}^{(2)}}{Z}{x}\right).
\end{align*}
Therefore,
\begin{align*}
&\;\;\;\;\log{\E_{Y\sim \opQ[\opV[u_{k\vep}]](\cdot|x)} \exp\left(\mfR{\xi_{k,\vep}^{(1)}}{Y}{x^{u_{k\vep}}}\right)}\\ &=\log{\E_{Y\sim \opQ[\opV[u_{k\vep}]](\cdot|x)}\left[\frac{1}{\E_{Z\sim \opP[u_{k\vep}](\cdot|Y)} \exp\left(\mfR{\xi_{k,\vep}^{(2)}}{Z}{x}\right)}\right]}\\ &\quad\quad+\log{\E_{Z\sim \opP[u_{k\vep}](\cdot|x^{u_{k\vep}})} \exp\left(\mfR{\xi_{k,\vep}^{(2)}}{Z}{x}\right)}.
\end{align*}
By combining the above observation with \eqref{eq:step14}, the conclusion follows.

\end{proof}

\begin{lmm}\label{lem:grabound}
    Under the assumptions of ... we have:
    
\end{lmm}


Also,
    \begin{align*}
        \mathcal{R}_{k\vep}(y)&:=w_{k\vep}(y)+\frac{\vep d}{2}\log{(2\pi\vep)}-\vep f(y^{w_{k\vep}})+\frac{\vep}{2}\ldet(\nabla^2 w_{k\vep}(y))\\ &\qquad -\vep^2 M[u_{k\vep},f](y)-\vep^2 \sum_{j=0}^{k-1} \mathcal{M}[u_{j\vep}](y^{w_{k\vep}}),
    \end{align*}
    
    \begin{align*}
        \mathcal{R}_{k\vep}(y;x)&:=w_{k\vep}(y)+\frac{\vep d}{2}\log{(2\pi\vep)}-\vep f(y^{w_{k\vep}})+\frac{\vep}{2}\ldet(\nabla^2 w_{k\vep}(y))\\ &\qquad -\vep^2 M[u_{k\vep},f](x^{u_{k\vep}})-\vep^2 \sum_{j=0}^{k-1} \mathcal{M}[u_{j\vep}](x).
    \end{align*}
    We define the remainder terms:
    \begin{align*}
        a_{k\vep}:=\sup_{y}\frac{1}{\vep}\bigg|\opV[u_k^{\vep}](y)-\mathcal{R}_{k\vep}(y)\bigg|,
    \end{align*}
    %and 
    %\begin{align*}
    %    \tilde{a}_{k\vep}:=\sup_{(y,x):\ \lVert y-x^{u_{k\vep}}\rVert \le r_{\vep}}\frac{1}{\vep}\bigg|\opV[u_k^{\vep}](y)-\mathcal{R}_{k\vep}(y;x)\bigg|.
    %\end{align*}
    The corresponding terms for approximating the $\opU$ operator are as follows:
    \begin{align*}
        \tilde{\mathcal{R}}_{k\vep}(x)&:=u_{k\vep}(x)+\vep^2 \sum_{j=0}^{k-1} \mathcal{M}[u_{j\vep}](x),
    \end{align*}
    
    %\begin{align*}
    %    \tilde{\mathcal{R}}_{k\vep}(x;y)&:=u_{k\vep}(x)+\vep^2 \sum_{j=0}^{k-1} \mathcal{M}[u_{j\vep}](y^{w_{k\vep}}).
    %\end{align*}
    We define the remainder terms:
    \begin{align*}
        b_{k\vep}:=\sup_{x}\frac{1}{\vep}\bigg|u_k^{\vep}(x)-\tilde{\mathcal{R}}_{k\vep}(x)\bigg|,
    \end{align*}
    %and 
    %\begin{align*}
    %    \tilde{b}_{k\vep}:=\sup_{(x,y):\ \lVert x-y^{w_{k\vep}}\rVert \le r_{\vep}}\frac{1}{\vep}\bigg|u_k^{\vep}(x)-\tilde{\mathcal{R}}_{k\vep}(x;y)\bigg|.
    %\end{align*}

 Note that for $k\ge 0$, 

\begin{align*}
    \Lambda & :=\exp\left(\frac{1}{\vep}\opV[u_k^{\vep}](y)-\frac{1}{\vep}\mathcal{R}_{k\vep}(y)\right)\\ &=\int \exp\left(\frac{1}{\vep}\langle x,y\rangle-\frac{1}{\vep}u_k^{\vep}(x)-\frac{1}{\vep}\mathcal{R}_{k\vep}(y)-f(x)\right)\,dx \\& = \int_{B_{r_{\vep}}(y^{w_{k\vep}})} \exp\left(\frac{1}{\vep}\langle x,y\rangle-\frac{1}{\vep}\tilde{\mathcal{R}}_{k\vep}(x;y)-\frac{1}{\vep}\mathcal{R}_{k\vep}(y)-f(x)-\frac{1}{\vep}u_k^{\vep}(x)+\frac{1}{\vep}\tilde{\mathcal{R}}_{k\vep}(x;y)\right)\,dx\\ &+\int_{B^c_{r_{\vep}}(y^{w_{k\vep}})} \exp\left(\frac{1}{\vep}\langle x,y\rangle-\frac{1}{\vep}\tilde{\mathcal{R}}_{k\vep}(x)-\frac{1}{\vep}\mathcal{R}_{k\vep}(y)-f(x)-\frac{1}{\vep}u_k^{\vep}(x)+\frac{1}{\vep}\tilde{\mathcal{R}}_{k\vep}(x)\right)\\ &=: \Lambda_1+\Lambda_2.
\end{align*}

We begin with $\Lambda_1$. Observe that 
\begin{align*}
&\;\;\;\;\frac{1}{\vep}\langle x,y\rangle-\frac{1}{\vep}\tilde{\mathcal{R}}_{k\vep}(x)-\frac{1}{\vep}\mathcal{R}_{k\vep}(y)-f(x)-\frac{1}{\vep}u_k^{\vep}(x)+\frac{1}{\vep}\tilde{\mathcal{R}}_{k\vep}(x;y)\\ &=\frac{1}{\vep}\mathcal{D}[u_{k\vep}](x|y)-f(x)+f(y^{w_{k\vep}})-\frac{1}{2}\ldet(\nabla^2 w_{k\vep}(y))+\vep M[u_{k\vep},f](y)-\frac{d}{2}\log{(2\pi\vep)}\\ &\qquad -\frac{1}{\vep}u_k^{\vep}(x)+\frac{1}{\vep}\tilde{\mathcal{R}}_{k\vep}(x;y).
\end{align*}
Therefore, with 
$$\Lambda_1^{(1)}:=\exp\left(f(y^{w_{k\vep}})+\vep M[u_{k\vep},f](y)\right)\frac{\sqrt{\mathrm{det}(\nabla^2 u_{k\vep}(y^{w_{k\vep}}))}}{(2\pi\vep)^{d/2}} \int\limits_{B_{r_{\vep}}(y^{w_{k\vep}})} \exp\left(\frac{1}{\vep}\mathcal{D}[u_{k\vep}](x|y)-f(x)\right)\,dx,$$
we have:
$$\exp(-\tilde{b}_{k\vep})\le \frac{\Lambda_1}{\Lambda_1^{(1)}}\le \exp(\tilde{b}_{k\vep}).$$
By \cref{lem:prelimestim}, we have:
\begin{align*}
    \big|\Lambda_1^{(1)}-1\big|\le \eta_t \left(\vep \xi_t(r_{\vep})+\vep^{3/2}(\log{(1/\vep)})^{9/2}\right).
\end{align*}
We now move to $\Lambda_2$. By invoking \cref{lem:prelimestim}, we get:
$$\Lambda_2\le \exp(b_{k\vep})\eta_t \vep^{10}.$$
Combining the above observations, we get:
\begin{align*}
    \exp\left(\frac{1}{\vep}\opV[u_k^{\vep}](y)-\frac{1}{\vep}\mathcal{R}_{k\vep}(y)\right)\ge \exp(-\tilde{b}_{k\vep})\left(1-\eta_t \left(\vep \xi_t(r_{\vep})+\vep^{3/2}(\log{(1/\vep)})^{9/2}\right)\right).
\end{align*}
Similarly, we have:
\begin{align*}
    &\;\;\;\;\exp\left(\frac{1}{\vep}\opV[u_k^{\vep}](y)-\frac{1}{\vep}\mathcal{R}_{k\vep}(y)\right)\\ &\le \exp(\tilde{b}_{k\vep})\left(1+\eta_t \left(\vep \xi_t(r_{\vep})+\vep^{3/2}(\log{(1/\vep)})^{9/2}\right)\right)+\eta_t\vep^{10}\exp(b_{k\vep}).
\end{align*}
By combining the two observations above, for $\vep>0$ small enough (depending only on $\eta_t$, we get: 
$$a_{k\vep}\le \tilde{b}_{k\vep}+\eta_t \left(\vep \xi_t(r_{\vep})+\vep^{3/2}(\log{(1/\vep)})^{9/2}\right)+\eta_t \vep^{10}\exp(b_{k\vep}).$$

We now fix $y,z$ such that $\lVert y-z^{u_{k\vep}}\rVert\le r_{\vep}$ and note that:


For our next lemma, we need to discuss further notation. 
Fix $0\le s\le t$. Recall \eqref{eq:gaussdef} and set
    \begin{align*}
    M[u_{s},f](y)&\equiv M[u_s,f,0](y)\nonumber \\ & =\frac{1}{2}\E\big(T[u_s:3](y^{w_s}+Z_{w_s,y}|y^{w_s})\big)^2 -\E T[u_s:4](y^{w_s}+Z_{w_s,y}|y^{w_s}) \\ & - \E T[f:2](y^{w_s}+Z_{w_s,y}|y^{w_s}) + \frac{1}{2}\E(T[f:1](y^{w_s}+Z_{w_s,y}|y^{w_s}))^2\nonumber \\ & + \E T[u_s:3](y^{w_s}+Z_{w_s,y}|y^{w_s}) T[f:1](y^{w_s}+Z_{w_s,y}|y^{w_s}),
    \end{align*}
    where we used the same notation from \cref{lem:prelimestim}. Also given any $\tilde{G}_{\vep}$ satisfying the conditions of \cref{lem:prelimestim}, define
    \begin{align}\label{eq:rem1}
        \tilde{M}[u_s,f,\vep \tilde{G}_{\vep}](y):=M[u_s,f,\vep \tilde{G}_{\vep}](y)-M[u_s,f](y).
    \end{align}
    Next up, we set 
    \begin{align}\label{eq:subfun}
    G_s(y):=-f(y^{w_s})+g(y)+\frac{1}{2}\ldet(\nabla^2 w_s(y)).
    \end{align}
    Note that as $g(\cdot)$ has bounded and uniformly continuous derivatives of the second order, and $u_s$, $w_s$ have bounded and uniformly continuous derivatives of the sixth order as per the assumptions of \cref{thm:inftheo}, $G_s(\cdot)$ as defined above has bounded and absolutely continuous derivatives of the fourth order. Once again, note that 
    \begin{align*}
     M[w_{s},G_s](x)&\equiv M[w_s,G_s,0](x)\nonumber \\ & =\frac{1}{2}\E\big(T[w_s:3](x^{u_s}+Z_{u_s,x}|x^{u_s})\big)^2 -\E T[w_s:4](x^{u_s}+Z_{u_s,x}|x^{u_s}) \\ & - \E T[G_s:2](x^{u_s}+Z_{u_s,x}|x^{u_s}) + \frac{1}{2}\E(T[G_s:1](x^{u_s}+Z_{u_s,x}|x^{u_s}))^2\nonumber \\ & + \E T[u_s:3](x^{u_s}+Z_{u_s,x}|x^{u_s}) T[G_s:1](x^{u_s}+Z_{u_s,x}|x^{u_s}).
    \end{align*}
    In a similar vein as in \eqref{eq:rem1}, we have:
    \begin{align}\label{eq:rem2}
       \tilde{M}[w_s,G_s,\vep \tilde{G}_{\vep}])(x):=M[w_s,G_s,\vep \tilde{G}_{\vep}](x)-M[w_s,G_s](x). 
    \end{align}
    Define further 
    \begin{align}\label{eq:tosee}\mathcal{M}[u_s](x):=M[w_{s},G_s](x)-M[u_s,f](x^{u_s})-\frac{1}{2}\frac{\partial^2}{\partial s^2}u_s(x),
    \end{align}
    and 
    \begin{align}
    \Theta_{k\vep}(x):=\sum_{r=0}^{k-1} \mathcal{M}[u_{r\vep}](x),\qquad \Theta_{0}\equiv 0,
    \end{align}
    for $k\ge 1$. The final two definitions are as follows: 
    \begin{align*}
        \Theta^*_{1,k\vep}(x):=\sum_{r=0}^{k-1} \tilde{M}[w_{r\vep},G_{r\vep},-\vep \Theta_{r\vep}(\nabla w_{r\vep})](x),
    \end{align*}
    and 
    \begin{align*}
        \Theta^*_{2,k\vep}(x):=-\sum_{r=1}^{k-1} \tilde{M}[u_{r\vep},f,-\vep \Theta_{r\vep}](x^{u_{r\vep}}),
    \end{align*}
    When $k=0$, the above summand above is set to $0$.
    The next lemma is elementary and accordingly, we skip the algebraic details of the proof.

    \begin{lmm}\label{eq:estimmt}
        Fix $t>0$. Under \cref{asn:solcon}, given $k\ge 1$, $\vep>0$, satisfying $k\vep \le t$, there exists a constant $C_t>0$ and a function $\omega_t(\cdot)$ with $\lim_{\delta\to 0} \omega_t(\delta)=0$, such that 
        \begin{align*}
            \sup_{k\le \lceil \frac{t}{\vep}\rceil} \vep\left(\lVert \Theta_{k\vep}\rVert_{\infty}\ +\ \lVert \nabla \Theta_{k\vep}\rVert_{\infty}\ + \ \lVert \nabla^2 \Theta_{k\vep}\rVert_{\infty}\right)\le C_t.
        \end{align*}
    \end{lmm}

    We are now in the position to state and prove our main result.
    
    \begin{lmm}
        Fix $t>0$. Under \cref{asn:solcon} with $k\ge 1$, $\vep>0$ such that $k\vep \le t$, we have:
        $$\lVert u_k^{\vep}-u_{k\vep}-\vep^2 \Theta_{(k-1)\vep}-\vep^2 \Theta^*_{1,(k-1)\vep}-\vep^2 \Theta^*_{2,(k-1)\vep}\rVert_{\infty}=o(\vep)$$
    \end{lmm}

    \begin{proof}
        The proof proceeds by induction.
        \vspace{0.1in}

        \noindent{Base case $k=1$}: 
    \end{proof}
    
\vspace{0.1in}

    \emph{$k=1$ case}: 
     and $C_{D,1}$ depends on the first $4$ derivatives of $u_0$, the first $2$ derivatives of $f$ and their  modulus of continuities. Also define

    $$\omega[u_0,f](\delta):=\sup_{\lVert z_1-z_2\rVert \le \delta} \big|M[u_0,f](z_1)-M[u_0,f](z_2)\big|.$$
    Using this observation, we will invoke \cref{lem:prelimestim} for each iteration of the Sinkhorn algorithm. By applying \cref{lem:prelimestim} with $u\equiv u_0$, $G\equiv f$, and $\tilde{G}\equiv 0$, we get:
    \begin{align*}
        \lVert\mathcal{R}^{\vep}_0-\opV[u_0]\rVert_{\infty}\le C_{0} \vep^2 \left(\omega[u_0,f](\sqrt{\vep \log{(1/\vep)}})+\sqrt{\vep}(\log{(1/\vep)})^{9/2}\right)=:E_{\vep,0},
    \end{align*}
    
    Note that 
    \begin{align*}
        &\;\;\;\;\exp\left(\frac{1}{\vep}u_1^{\vep}(x)-\frac{1}{\vep}\tilde{u}_{\vep}(x)-\vep M[u_0,f](x^{u_0})\right)\\ &=\frac{\sqrt{\mathrm{det}(\nabla^2 w_0(x^{u_0}))}}{(2\pi\vep)^{d/2}}\int_{B_{r_{\vep}}^c(x^{u_0})}\exp\bigg(\frac{1}{\vep}\mcD[w_0](y|x)+G_0(y)-G_0(x^{u_0})\nonumber \\ &\qquad \qquad +\vep(M[u_0,f](y)-M[u_0,f](x^{u_0}))-\frac{1}{\vep}(\opV[u_0](y)-\mathcal{R}^{\vep}_0(y))\bigg)\,dy\\ &+\frac{\sqrt{\mathrm{det}(\nabla^2 w_0(x^{u_0}))}}{(2\pi\vep)^{d/2}}\int_{B_{r_{\vep}}(x^{u_0})}\exp\bigg(\frac{1}{\vep}\mcD[w_0](y|x)+G_0(y)-G_0(x^{u_0})\nonumber \\ &\qquad \qquad -\frac{1}{\vep}(\opV[u_0](y)-\mathcal{R}^{\vep}_0(y;x))\bigg)\,dy
    \end{align*}
    By using \eqref{eq:gradbound4}, the first term above is bounded by 
    $$C_1 \exp\left(\vep C_1+\vep E_{0,\vep}+2\vep \lVert M[u_0,f]\rVert_{\infty}\right)\vep^{10}.$$
    For the second term, note that 
    \begin{align*}
        &\;\;\;\;\log\frac{\int_{B_{r_{\vep}}(x^{u_0})}\exp\bigg(\frac{1}{\vep}\mcD[w_0](y|x)+G_0(y)-G_0(x^{u_0})-\frac{1}{\vep}(\opV[u_0](y)-\mathcal{R}^{\vep}_0(y;x))\bigg)\,dy}{\int_{B_{r_{\vep}}(x^{u_0})}\exp\bigg(\frac{1}{\vep}\mcD[w_0](y|x)+G_0(y)-G_0(x^{u_0})\bigg)\,dy}\\ &\in \left(-\sup_{y\in B_{r_{\vep}}(x^{u_0})} \frac{1}{\vep}\big|\opV[u_0](y)-\mathcal{R}^{\vep}_0(y;x)\big|,\ \sup_{y\in B_{r_{\vep}}(x^{u_0})} \frac{1}{\vep}\big|\opV[u_0](y)-\mathcal{R}^{\vep}_0(y;x)\big|\right).
    \end{align*}
    By \cref{lem:prelimestim}, we then have:
    \begin{align*}
        &\;\;\;\;\bigg|\int_{B_{r_{\vep}}(x^{u_0})}\exp\bigg(\frac{1}{\vep}\mcD[w_0](y|x)+G_0(y)-G_0(x^{u_0})\bigg)\,dy-1+\vep M[w_0,G_0](x)\bigg|\\ &\le C_2\left(\vep^{3/2}(\log{(1/\vep)})^{9/2}+\vep \omega_2(r_{\vep})\right).
    \end{align*}
    Consequently, we get:
    $$\big|u_1^{\vep}(x)-\tilde{u}_{\vep}(x)-\vep^2 M[u_0,f](x^{u_0})+\vep^2 M[w_0,G_0](x)\big|\le .$$




\hrule 

\begin{proof}
    We begin the proof with some definitions. 
\begin{align*}
    \xi_t&:=\sup_{k\le \lceil t/\vep\rceil} \vep \sum_{j=0}^k \lVert \mathcal{M}[u_{j\vep}]\rVert_{\infty} + \sup_{s\le t} \, (\lVert \nabla G_s\rVert_{\infty}+\lVert \nabla^2 G_s\rVert_{\infty})\\ &\quad +\lVert \nabla f\rVert_{\infty}+\lVert \nabla^2 f\rVert + \sup_{s\le t} \, (\lVert \nabla^2 u_s\rVert_{\infty}+\lVert \nabla^3 u_s\rVert_{\infty}+\lVert \nabla^4 u_s\rVert_{\infty})\infty,
\end{align*}
and the following function:
\begin{align*}
    \xi_t(\delta)&:=\sup_{s\le t} \sup_{\lVert z_1-z_2\rVert\le \delta} \bigg(|\mathcal{M}[u_s](z_1)-\mathcal{M}[u_s](z_2)|+\lVert \nabla^2 G_s(z_1)-\nabla^2 G_s(z_2)\rVert_{\infty}\\ &\qquad +\lVert \nabla^4 u_s(z_1)-\nabla^4 u_s(z_2)\rVert_{\infty} +\lVert \nabla^2 f(z_1)-\nabla^2 f(z_2)\rVert_{\infty}\bigg)<\infty.
\end{align*}

Observe that 
\begin{align*}
    \exp(\mathcal{I}_{k\vep}(y))&=\exp\left(\frac{1}{\vep}\opV[u_k^{\vep}](y)-\frac{1}{\vep}w_{k\vep}(y)-\frac{d}{2}\log{(2\pi\vep)}+f(y^{w_{k\vep}})-\frac{1}{2}\ldet(\nabla^2 w_{k\vep}(y))\right)\\ &=\frac{\sqrt{\mathrm{det}(\nabla^2 u_{k\vep}(y^{w_{k\vep}})}}{(2\pi\vep)^{d/2}}\int \exp\left(\frac{1}{\vep}\langle x,y\rangle - \frac{1}{\vep} w_{k\vep}(y)-\frac{1}{\vep}u_k^{\vep} + f(y^{w_{k\vep}}) - f(x)\right)\,dx
\end{align*}
Therefore, 
\begin{align*}
    &\;\;\;\;\frac{\exp(\mathcal{I}_{k\vep}(y))}{\frac{\sqrt{\mathrm{det}(\nabla^2 u_{k\vep}(y^{w_{k\vep}})}}{(2\pi\vep)^{d/2}}\int \exp\left(\frac{1}{\vep}\langle x,y\rangle - \frac{1}{\vep} w_{k\vep}(y)-\frac{1}{\vep}u_{k\vep} + f(y^{w_{k\vep}}) - f(x)\right)\,dx}\\ &\in \big(\exp(-\lVert \mathcal{J}_{k\vep}\rVert_{\infty}),\ \exp(\lVert \mathcal{J}_{k\vep}\rVert_{\infty})\big).
\end{align*}
By \cref{lem:prelimestim}, we have that
$$\bigg|\log\frac{\sqrt{\mathrm{det}(\nabla^2 u_{k\vep}(y^{w_{k\vep}})}}{(2\pi\vep)^{d/2}}\int \exp\left(\frac{1}{\vep}\langle x,y\rangle - \frac{1}{\vep} w_{k\vep}(y)-\frac{1}{\vep}u_{k\vep} + f(y^{w_{k\vep}}) - f(x)\right)\,dx\bigg|\le \vep\eta_t,$$
where $\eta_t$ is a constant depending on $\xi_t$ and $\xi_t(1)$. 

By combining the above observations and taking logarithms on both sides, we have:
$$\lVert\mathcal{I}_{k\vep}\rVert_{\infty} \le \vep \eta_t + \lVert\mathcal{J}_{k\vep}\rVert.$$
By repeating the same argument with $u_{k+1}^{\vep}$, we get:
$$\lVert \mathcal{J}_{(k+1)\vep}\rVert_{\infty}\le \eta_t \vep + \lVert \mathcal{I}_{k\vep}\rVert_{\infty}.$$

The above recursive relations imply that 
$$\lVert\mathcal{I}_{k\vep}\rVert_{\infty}=O(1),\qquad \mbox{and}\qquad \lVert\mathcal{J}_{k\vep}\rVert_{\infty}=O(1).$$
\end{proof}

\begin{prop}\label{prop:pmasec}
    Suppose that \cref{asn:solcon} holds. Observe that for $t\ge 0$, we have:
    \begin{align}\label{eq:matid1}
        \Ddot{u}_{t}(x)&=\frac{\partial\hfill}{\partial t^2}u_t(x)\nonumber \\ &=-g'(x^{u_{k\vep}})f'(x)+(g'(x^{u_{k\vep}}))^2 u_{k\vep}''(x)-2 g'(x^{u_{k\vep}})u_{k\vep}'''(x) w_{k\vep}''(x^{u_{k\vep}}) + f''(x) w_{k\vep}''(x^{u_{k\vep}})\nonumber \\ &\qquad -g''(x^{u_{k\vep}})u''_{k\vep}(x)+u^{(4)}_{k\vep}(x) (w''_{k\vep}(x^{u_{k\vep}}))^2+w_{k\vep}'''(x^{u_{k\vep}})u_{k\vep}'''(x).
    \end{align}
\end{prop}

\begin{proof}
   Hello.
\end{proof}

We are now in position to prove \cref{lem:matid}. To avoid notational clutter, we will drop the $k\vep$ notation in \cref{lem:matid} without loss of generality. This can be done as \cref{lem:matid} is a claim for every fixed $k\vep$ and there are no asymptotics involved in it. Therefore, in the sequel, we will write $u\equiv u_{k\vep}$, $w\equiv w_{k\vep}$ and following \eqref{eq:initdef1} and \eqref{eq:initdef2}, 
$$G(x)\equiv G_{k\vep}(x)=\frac{3f}{2}(x)-g(x^{u_{k\vep}})+\ldet\left(\frac{\partial x^{u_{k\vep}}}{\partial x\hfill}\right)\equiv \frac{3f}{2}(x)-g(x^u)+\ldet\left(\frac{\partial x^{u}}{\partial x\hfill}\right),$$
and
$$\bar{G}(y)\equiv \bar{G}_{k\vep}(y)=-G_{k\vep}(y^{w_{k\vep}})+g(y)+\frac{1}{2}\ldet\left(\frac{\partial y^{w_{k\vep}}}{\partial y\hfill}\right)\equiv -G(y^w)+g(y)+\frac{1}{2}\ldet\left(\frac{\partial y^{w}}{\partial y\hfill}\right).$$

 We will also need the following expressions for the derivatives of $\bar{G}_{k\vep}$. In particular,

    \begin{equation}\label{eq:matt4}
        \bar{G}_{k\vep}'(y)=-G_{k\vep}'(y^{w_{k\vep}}) w_{k\vep}''(y)+g'(y)+\frac{1}{2}\frac{w_{k\vep}'''(y)}{w_{k\vep}''(y)},
    \end{equation}
    and 
    \begin{equation}\label{eq:matt5}
        \bar{G}_{k\vep}''(y)=-G_{k\vep}'(y^{w_{k\vep}}) w_{k\vep}'''(y)-G_{k\vep}''(y^{w_{k\vep}})(w_{k\vep}''(y))^2+g''(y)+\frac{1}{2}\frac{w_{k\vep}^{(4)}(y)}{w_{k\vep}''(y)}-\frac{1}{2}\frac{(w_{k\vep}'''(y))^2}{(w_{k\vep}''(y))^2}.
    \end{equation}

\begin{proof}[Proof of \cref{lem:matid}]
    
    
    We will use $Z_{w,y}\sim  N(0,\nabla^2 w(y))$ in the sequel.
    
    We now move on to simplifying the difference between the $M[\cdot,\cdot]$ functional. In particular, by \eqref{eq:gaussdef}, we have:
    \begin{align}\label{eq:matid2}
        &\;\;\;\; M[w_{k\vep},\bar{G}_{k\vep}](x)-M[u_{k\vep},G_{k\vep}](x^{u_{k\vep}})=\sum_{i=1}^5 T_i(x),
    \end{align}
    where
    \begin{align*}
        T_1(x):=\frac{1}{2}\E\big(T[w_{k\vep}:3](x^{u_{k\vep}}+Z_{u_{k\vep},x}|x^{u_{k\vep}})\big)^2-\frac{1}{2}\E\big(T[u_{k\vep}:3](x+Z_{w_{k\vep},x^{u_{k\vep}}}|x)\big)^2,
    \end{align*}
    \begin{align*}
        T_2(x):=\E T[u_{k\vep}:4](x+Z_{w_{k\vep},x^{u_{k\vep}}}|x)- \E T[w_{k\vep}:4](x^{u_{k\vep}}+Z_{u_{k\vep},x}|x^{u_{k\vep}}),
    \end{align*}
    \begin{align*}
        T_3(x):= \E T[G_{k\vep}:2](x+Z_{w_{k\vep},x^{u_{k\vep}}}|x)- \E T[\bar{G}_{k\vep}:2](x^{u_{k\vep}}+Z_{u_{k\vep},x}|x^{u_{k\vep}}),
    \end{align*}
    \begin{align*}
        T_4(x):= \frac{1}{2}\E(T[\bar{G}_{k\vep}:1](x^{u_{k\vep}}+Z_{u_{k\vep},x}|x^{u_{k\vep}})^2-\frac{1}{2}\E(T[G_{k\vep}:1](x+Z_{w_{k\vep},x^{u_{k\vep}}}|x))^2,
    \end{align*}
    and
    \begin{align*}
        T_5(x)&:= \E T[w_{k\vep}:3](x^{u_{k\vep}}+Z_{u_{k\vep},x}|x^{u_{k\vep}}) T[\bar{G}_{k\vep}:1](x^{u_{k\vep}}+Z_{u_{k\vep},x}|x^{u_{k\vep}})\\ & \qquad -\E T[u_{k\vep}:3](x+Z_{w_{k\vep},x^{u_{k\vep}}}|x) T[G_{k\vep}:1](x+Z_{w_{k\vep},x^{u_{k\vep}}}|x).
    \end{align*}

    We will simplify each of these terms individually.

    \emph{Simplifying $T_1$.} We will show $T_1\equiv 0$ ({\color{red} This part I can show for general $d$}). Observe that 
    \begin{align*}
    \E\big(T[w_{k\vep}:3](x^{u_{k\vep}}+Z_{u_{k\vep},x}|x^{u_{k\vep}})\big)^2=\frac{15}{36} (w'''_{k\vep}(x^{u_{k\vep}}))^2 (u''_{k\vep}(x))^3=-\frac{15}{36} w_{k\vep}'''(x^{u_{k\vep}})u_{k\vep}'''(x),
    \end{align*}
    where the last conclusion follows from \eqref{eq:matt1}. Similarly, 
    \begin{align*}
        \E\big(T[u_{k\vep}:3](x+Z_{w_{k\vep},x^{u_{k\vep}}}|x)\big)^2=\frac{15}{36}(u_{k\vep}'''(x))^2 (w_{k\vep}''(x^{u_{k\vep}}))^3=-\frac{15}{36}w_{k\vep}'''(x^{u_{k\vep}})u_{k\vep}'''(x).
    \end{align*}
    Therefore $T_1\equiv 0$.

    \vspace{0.1in} 

    \emph{Simplifying $T_2$.} Observe that 
    \begin{align*}
    \E T[u_{k\vep}:4](x+Z_{w_{k\vep},x^{u_{k\vep}}}|x)=\frac{1}{8}u_{k\vep}^{(4)}(x)(w_{k\vep}''(x^{u_{k\vep}}))^2.
    \end{align*}
    By a similar computation for the other term in the definition of $T_2$, we get:
    \begin{align*}
        T_2(x)=\frac{1}{8}\left(u_{k\vep}^{(4)}(x)(w_{k\vep}''(x^{u_{k\vep}}))^2-w_{k\vep}^{(4)}(x^{u_{k\vep}})(u_{k\vep}''(x))^2\right).
    \end{align*}

    \vspace{0.1in}

    \emph{Simplifying $T_3$.} Observe that, by \eqref{eq:matt5}, we get: 
    \begin{align*}
        &\;\;\;\E T[\bar{G}_{k\vep}:2](x^{u_{k\vep}}+Z_{u_{k\vep},x}|x^{u_{k\vep}}) \\ &=\frac{1}{2}\bar{G}_{k\vep}''(x^{u_{k\vep}})u_{k\vep}''(x)\\ &= \frac{1}{2}\bigg(-G_{k\vep}'(x) w_{k\vep}'''(x^{u_{k\vep}})u_{k\vep}''(x)-G_{k\vep}''(x)(w_{k\vep}''(x^{u_{k\vep}}))+g''(x^{u_{k\vep}})u_{k\vep}''(x)\\ &+\frac{1}{2}w_{k\vep}^{(4)}(x^{u_{k\vep}})(u_{k\vep}''(x))^2-\frac{1}{2}(w_{k\vep}'''(x^{u_{k\vep}}))^2(u_{k\vep}''(x))^3\bigg).
    \end{align*}
    By invoking \eqref{eq:matt1}, we get:
    \begin{align*}
        T_3(x)&:= \frac{1}{2}\left(G_{k\vep}''(x)(w_{k\vep}''(x^{u_{k\vep}})-\bar{G}_{k\vep}''(x^{u_{k\vep}})u_{k\vep}''(x)\right)\\ &=\frac{1}{2}\bigg(2G_{k\vep}''(x)(w_{k\vep}''(x^{u_{k\vep}}))+G_{k\vep}'(x) w_{k\vep}'''(x^{u_{k\vep}})u_{k\vep}''(x)-g''(x^{u_{k\vep}})u_{k\vep}''(x)\\ & \quad - \frac{1}{2}w_{k\vep}^{(4)}(x^{u_{k\vep}})(u_{k\vep}''(x))^2-\frac{1}{2} w_{k\vep}'''(x^{u_{k\vep}})u_{k\vep}'''(x)\bigg).
    \end{align*}

    \vspace{0.1in}

    \emph{Simplifying $T_4$.} Observe that, by \eqref{eq:matt2}, we have: 
    \begin{align*}
        &\;\;\;\E (T[\bar{G}_{k\vep}:1](x^{u_{k\vep}}+Z_{u_{k\vep},x}|x^{u_{k\vep}}))^2\\ &=(\bar{G}_{k\vep}'(x^{u_{k\vep}}))^2 u_{k\vep}''(x) \\ &=\bigg((G_{k\vep}'(x))^2  w_{k\vep}''(x^{u_{k\vep}})+(g'(x^{u_{k\vep}}))^2 u_{k\vep}''(x)+\frac{1}{4} (w_{k\vep}'''(x^{u_{k\vep}}))^2 (u_{k\vep}''(x))^3 \\ &\qquad -2 G_{k\vep}'(x) g'(x^{u_{k\vep}}) - G_{k\vep}'(x) w_{k\vep}'''(x^{u_{k\vep}})u_{k\vep}''(x)+ g'(x^{u_{k\vep}})w_{k\vep}'''(x^{u_{k\vep}}) (u_{k\vep}''(x))^2\bigg).
    \end{align*}
    Therefore,
    \begin{align*}
        T_4(x)&=\frac{1}{2}(g'(x^{u_{k\vep}}))^2 u_{k\vep}''(x)+\frac{1}{8} (w_{k\vep}'''(x^{u_{k\vep}}))^2 (u_{k\vep}''(x))^3 - G_{k\vep}'(x) g'(x^{u_{k\vep}}) \\ &\qquad - \frac{1}{2} G_{k\vep}'(x) w_{k\vep}'''(x^{u_{k\vep}})u_{k\vep}''(x)+ \frac{1}{2} g'(x^{u_{k\vep}})w_{k\vep}'''(x^{u_{k\vep}}) (u_{k\vep}''(x))^2.
    \end{align*}

    \vspace{0.1in}

    \emph{Simplifying $T_5$.} Observe that by \eqref{eq:matt1}, we get:
    \begin{align*}
        &\;\;\;\;\E T[w_{k\vep}:3](x^{u_{k\vep}}+Z_{u_{k\vep},x}|x^{u_{k\vep}}) T[\bar{G}_{k\vep}:1](x^{u_{k\vep}}+Z_{u_{k\vep},x}|x^{u_{k\vep}})\\ &=-\frac{1}{2} \bar{G}_{k\vep}'(x^{u_{k\vep}})\frac{u_{k\vep}'''(x)}{u_k''(x)}\\ &=\frac{1}{2}\bigg(G_{k\vep}'(x)(w_{k\vep}''(x^{u_{k\vep}}))^2 u_{k\vep}'''(x)-g'(x^{u_{k\vep}})\frac{u_{k\vep}'''(x)}{u_{k\vep}''(x)}-\frac{1}{2}w_{k\vep}'''(x^{u_{k\vep}})u_{k\vep}'''(x)\bigg).
    \end{align*}

    Also observe that 
    \begin{align*}
        &\;\;\;\E T[u_{k\vep}:3](x+Z_{w_{k\vep},x^{u_{k\vep}}}|x) T[G_{k\vep}:1](x+Z_{w_{k\vep},x^{u_{k\vep}}}|x)\\ &=\frac{1}{2}G_{k\vep}'(x)u_{k\vep}'''(x) (w_{k\vep}''(x^{u_{k\vep}}))^2.
    \end{align*}

    This implies 
    $$T_5(x)=-\frac{1}{2}\bigg(g'(x^{u_{k\vep}})\frac{u_{k\vep}'''(x)}{u_{k\vep}''(x)}+\frac{1}{2}w_{k\vep}'''(x^{u_{k\vep}})u_{k\vep}'''(x)\bigg).$$

    By combining all the above observations with \eqref{eq:matid2}, we get:

    \begin{align}\label{eq:matid11}
        &\;\;\; 2M[w_{k\vep},\bar{G}_{k\vep}](x)-2M[u_{k\vep},G_{k\vep}](x^{u_{k\vep}}) \nonumber \\ &=\frac{1}{4}u_{k\vep}^{(4)}(x)(w_{k\vep}''(x^{u_{k\vep}}))^2-\frac{1}{4}w_{k\vep}^{(4)}(x^{u_{k\vep}})(u_{k\vep}''(x))^2 + 2G_{k\vep}''(x)(w_{k\vep}''(x^{u_{k\vep}}))\nonumber \\ &\quad +G_{k\vep}'(x) w_{k\vep}'''(x^{u_{k\vep}})u_{k\vep}''(x)-g''(x^{u_{k\vep}})u_{k\vep}''(x) - \frac{1}{2}w_{k\vep}^{(4)}(x^{u_{k\vep}})(u_{k\vep}''(x))^2\nonumber\\ & \quad -\frac{1}{2} w_{k\vep}'''(x^{u_{k\vep}})u_{k\vep}'''(x)+(g'(x^{u_{k\vep}}))^2 u_{k\vep}''(x)+\frac{1}{4} (w_{k\vep}'''(x^{u_{k\vep}}))^2 (u_{k\vep}''(x))^3\nonumber \\ &\quad - 2G_{k\vep}'(x) g'(x^{u_{k\vep}}) -  G_{k\vep}'(x) w_{k\vep}'''(x^{u_{k\vep}})u_{k\vep}''(x)+ g'(x^{u_{k\vep}})w_{k\vep}'''(x^{u_{k\vep}}) (u_{k\vep}''(x))^2\nonumber \\ &\quad -g'(x^{u_{k\vep}})\frac{u_{k\vep}'''(x)}{u_{k\vep}''(x)}-\frac{1}{2}w_{k\vep}'''(x^{u_{k\vep}})u_{k\vep}'''(x)\nonumber\\ &=\frac{1}{4}u_{k\vep}^{(4)}(x)(w_{k\vep}''(x^{u_{k\vep}}))^2-\frac{3}{4}w_{k\vep}^{(4)}(x^{u_{k\vep}})(u_{k\vep}''(x))^2-\frac{5}{4} w_{k\vep}'''(x^{u_{k\vep}})u_{k\vep}'''(x)\nonumber\\ &\quad +2 G_{k\vep}''(x)w_{k\vep}''(x^{u_{k\vep}})-g''(x^{u_{k\vep}})u_{k\vep}''(x)+(g'(x^{u_{k\vep}}))^2 u_{k\vep}''(x)-2 G_{k\vep}'(x)g'(x^{u_{k\vep}})\nonumber\\ &\quad -2 g'(x^{u_{k\vep}})\frac{u_{k\vep}'''(x)}{u_{k\vep}''(x)}.
    \end{align}

    Recall that $G_{k\vep}(x)=f(x)+v_{k\vep}(x)$. Also observe that 
    \begin{align*}
        v_{k\vep}'(x)=\frac{1}{2}f'(x)-g'(x^{u_{k\vep}})u_{k\vep}''(x)+\frac{u_{k\vep}'''(x)}{u_{k\vep}''(x)},
    \end{align*}
    and by using \eqref{eq:matt2}, we also have:
    \begin{align*}
        v_{k\vep}''(x)=\frac{1}{2}f''(x)-g''(x^{u_{k\vep}})(u_{k\vep}''(x))^2-g'(x^{u_{k\vep}})u_{k\vep}'''(x)+\frac{u_{k\vep}^{(4)}(x)}{u_{k\vep}''(x)}+w_{k\vep}'''(x^{u_{k\vep}})u_{k\vep}'''(x).
    \end{align*}

    Plugging the two observations above into \eqref{eq:matid11}, we get:
    \begin{align}\label{eq:matid12}
     &\;\;\; 2M[w_{k\vep},\bar{G}_{k\vep}](x)-2M[u_{k\vep},G_{k\vep}](x^{u_{k\vep}})\\ &=\frac{9}{4} u_{k\vep}^{(4)}(x)(w_{k\vep}(x^{u_{k\vep}}))^2-\frac{3}{4}w_{k\vep}^{(4)}(x^{u_{k\vep}})(u_{k\vep}''(x))^2+\frac{3}{4}w_{k\vep}'''(x^{u_{k\vep}})u_{k\vep}'''(x)+3f''(x)w_{k\vep}''(x^{u_{k\vep}})\\ &\quad -3g''(x^{u_{k\vep}})u_{k\vep}''(x)-6 g'(x^{u_{k\vep}})\frac{u_{k\vep}'''(x)}{u_{k\vep}''(x)}+3(g'(x^{u_{k\vep}}))^2 u_{k\vep}''(x)-3f'(x)g'(x^{u_{k\vep}}).
    \end{align}
    Scaling and then subtracting \eqref{eq:matid1} from \eqref{eq:matid12}, we get:
    \begin{align*}
     &\;\;\;2M[w_{k\vep},\bar{G}_{k\vep}](x)-2M[u_{k\vep},G_{k\vep}](x^{u_{k\vep}})-3\Ddot{u}_{k\vep}(x)\\ &=-\frac{3}{4}\left(u_{k\vep}^{(4)}(x)(w_{k\vep}(x^{u_{k\vep}}))^2+w_{k\vep}^{(4)}(x^{u_{k\vep}})(u_{k\vep}''(x))^2+3w_{k\vep}'''(x^{u_{k\vep}})u_{k\vep}'''(x)\right)=0,
    \end{align*}
    where the last equality follows from \eqref{eq:matt3}.
\end{proof}

\begin{proof}[Proof of \cref{thm:convergence}] 
Note that 

\begin{equation}\label{eq:step11}
\KL{\exp(-h_{k\vep})}{p_X \gvp_k}=\int \left(f(x)-h_{k\vep}(x)-\frac{1}{\vep}(u_{k+1}^{\vep}(x)-u_k^{\vep}(x))\right)\exp(-h_{k\vep}(x))\,dx.
\end{equation}

\noindent We focus on the integrand in \eqref{eq:step11}. Let us define some notation first. 
Fix $x\in\R^d$ and 
In order to prove \eqref{eq:step1show}, it suffices to prove the following:

\begin{equation}\label{eq:step16}
    \sup_{k\vep\le T}\bigg|\int \left[\frac{1}{\vep}\left(\opS[u_{k\vep}](x)-u_{k\vep}(x)\right)-f(x)+h_{k\vep}(x)\right]\exp(-h_{k\vep}(x))\,dx\bigg|=o_{\vep}(1),
\end{equation}
and 
\begin{equation}\label{eq:step17}
    \sup_{k\vep\le T}\E_{X\sim \exp(-h_{k\vep})}\bigg|\log{\E_{Y\sim \opQ[\opV[u_{k\vep}]](\cdot|X)}\left[\frac{1}{\E_{Z\sim \opP[u_{k\vep}](\cdot|Y)} \exp\left(\mfR{\xi_{k,\vep}^{(2)}}{Z}{x}\right)}\right]}\bigg|=o_{\vep}(1).
\end{equation}

\emph{Proof of \eqref{eq:step16}.} 
Fix $T>0$ and consider $k,\vep$ such that $k\vep\le T$. 
Recall that $\rho_t=\exp(-h_t)$. We first claim that 
\begin{align}\label{eq:toshow1}
\limsup_{M\to\infty}\sup_{\ k\vep\le T} \E_{X\sim \rho_{k\vep}}\left[\bigg|\frac{1}{\vep}(\opS[u_{k\vep}](X)-u_{k\vep}(X)\bigg|+|f(X)-h_{k\vep}(X)|\right]\bm{1}(\lVert X\rVert\ge M)=0.
\end{align}
Establishing \eqref{eq:toshow1} would reduce proving \eqref{eq:step16} to showing that
\begin{align}\label{eq:toshow2}
\limsup_{\vep\to 0,\ k\vep\le T}\sup_{\lVert x\rVert\le M} \bigg|\frac{1}{\vep}(\opS[u_{k\vep}](x)-u_{k\vep}(x))-f(x)+h_{k\vep}(x)\bigg|=0,
\end{align}
for all fixed but large (free of $\vep$) $M>0$. 

\emph{Proof of \eqref{eq:toshow1}.}
Consider $Y\sim e^{-g}$. Note that by \cref{asn:solcon}, we have:
\begin{align}\label{eq:tailest1}
&\;\;\;\;\limsup_{M\to\infty}\sup_{t\in [0,T]}\E_{X\sim \rho_t} |f(X)-h_t(X)|\bm{1}(\lVert X\rVert\ge M)\nonumber \\&\le \sqrt{\sup_{t\in [0,T]} \E\left(\frac{\partial u_t}{\partial t\hfill}(Y^{u_t^*})\right)^2}\limsup_{M\to\infty}\sup_{t\in [0,T]}\sqrt{P(\lVert Y^{u_t^*}\rVert\ge M)}=0.
\end{align}
In the final conclusion above, we use the fact that 
\begin{equation}\label{eq:callback}
\lVert Y^{u_t^*}\rVert \le c_T^{-1}Y+\sup_{t\in [0,T]} \lVert \nabla u_t^*(0)\rVert.
\end{equation}
Next, set
$$c_T:=\inf_{t\in [0,T],\ x\in \R^d} \lmn\left(\frac{\partial x^{u_t}}{\partial x\hfill}\right)>0,\quad C_T:=\sup_{t\in [0,T],\ x\in \R^d}\lmx\left(\frac{\partial x^{u_t}}{\partial x\hfill}\right)<\infty.$$
The following inequalities, which follow from straightforward Taylor expansions, will be used multiple times in this proof:
\begin{equation}\label{eq:maindisp1}
\frac{1}{2C_T}\lVert y-x^{u_{k\vep}}\rVert^2\le u_{k\vep}^*(y)+u_{k\vep}(x)-\langle x,y\rangle \le \frac{1}{2c_T}\lVert y-x^{u_{k\vep}}\rVert^2,
\end{equation}
and 
\begin{equation}\label{eq:maindisp2}
\frac{c_T}{2}\lVert x-y^{u_{k\vep}^*}\rVert^2\le u_{k\vep}^*(y)+u_{k\vep}(x)-\langle x,y\rangle \le \frac{C_T}{2}\lVert x-y^{u_{k\vep}^*}\rVert^2.
\end{equation}

Let $Z_{\vep,C}\sim  N(0,\sqrt{\vep C} I_d)$ for $C>0$. 
For fixed $y$, by using \eqref{eq:maindisp2}, we have:
Consequently,
\begin{align*}
\exp\left(\frac{1}{\vep}\opV[u_{k\vep}](y)-\frac{1}{\vep}u_{k\vep}^*(y)\right)&\ge \int \exp\left(-\frac{C_T}{2\vep}\lVert z-y^{u_{k\vep}^*}\rVert^2-f(z)\right)\,dz\\ &=\left(\frac{2\pi\vep}{C_T}\right)^{\frac{d}{2}}\E\exp(-f(y^{u_{k\vep}^*}+Z_{\vep,(C_T)^{-1}})).
\end{align*}
Consequently,
\begin{align}\label{eq:upbd1}
    &\;\;\;\;\exp\left(\frac{1}{\vep}(\opS[u_{k\vep}](x)-u_{k\vep}(x))\right)\nonumber \\ &\le \left(\frac{C_T}{2\pi\vep}\right)^{\frac{d}{2}}\int \left[E\exp(-f(y^{u_{k\vep}^*}+Z_{\vep,(C_T)^{-1}}))\right]^{-1}\exp\left(\frac{1}{\vep}\langle x,y\rangle - \frac{1}{\vep}u_{k\vep}^*(y)-\frac{1}{\vep}u_{k\vep}(x)-g(y)\right)\,dy\nonumber \\ &\le (C_T)^d \E_{Z_{\vep,C_T}}\left[\left(\E_{|Z_{\vep,C_T}}\exp\left(-f((x^{u_{k\vep}}+Z_{\vep,C_T})^{u_{k\vep}^*}+Z_{\vep,(C_T)^{-1}})\right)\right)^{-1}\exp(-g(x^{u_{k\vep}}+Z_{\vep,C_T}))\right]\nonumber \\ &=(C_T)^d \Theta_{C_T,\vep}(x^{u_{k\vep}}).
\end{align}
A similar computation as above also gives a lower bound for $\exp\left(\frac{1}{\vep}(\opS[u_{k\vep}](x)-u_{k\vep}(x))\right)$ with $C_T$ replaced by $c_T$, i.e.,
\begin{align}\label{eq:lbd1}
\exp\left(\frac{1}{\vep}(\opS[u_{k\vep}](x)-u_{k\vep}(x))\right)\ge (c_T)^d \Theta_{c_T,\vep}(x^{u_{k\vep}}).
\end{align}
We omit the details for brevity. Combining these observations with the Cauchy Schwartz inequality, we get:
\begin{align}\label{eq:zerocon}
    &\;\;\;\;\limsup_{M\to\infty}\sup_{\ k\vep\le T} \E_{X\sim \rho_{k\vep}}\left[\bigg|\frac{1}{\vep}(\opS[u_{k\vep}](X)-u_{k\vep}(X))\bigg|\bm{1}(\lVert X\rVert\ge M)\right]\nonumber \\ &\le\sqrt{4d((\log{C_T})^2+(\log{c_T})^2)+4(\tilde{\theta}_{C_T}+\tilde{\theta}_{c_T})}\limsup_{M\to\infty}\sup_{k\vep\le T}\sqrt{P(\lVert Y^{u_{k\vep}^*}\rVert\ge M)}=0.
\end{align}
The last equality again uses \eqref{eq:callback}. Combining the above display with \eqref{eq:tailest1} establishes \eqref{eq:toshow1}.

\vspace{0.1in}

\emph{Proof of \eqref{eq:toshow2}.} Fix $M_1,M>0$ and define 
$$R_{M,\vep}(x):=
\log\int_{\lVert y\rVert\le M} \exp\left(\frac{1}{\vep}\langle x,y\rangle - \frac{1}{\vep}u_{k\vep}(x)-\frac{1}{\vep}\opV[u_{k\vep}](y)-g(y)\right)\,dy.$$ 
We first claim that 
\begin{equation}\label{eq:showagain1}
\limsup_{M_2\to 0}\sup_{|x|\le M_1,\ k\vep\le T} \bigg|\frac{1}{\vep}(\opS[u_{k\vep}](x)-u_{k\vep}(x))-R_{M_2,\vep}(x)\bigg|=0.
\end{equation}

\emph{Proof of \eqref{eq:showagain1}.} 
Note that, by definition, 
$$R_{\infty,\vep}(x)=\frac{1}{\vep}(\opS[u_{k\vep}](x)-u_{k\vep}(x)).$$
Consequently, we can drop the absolute value in \eqref{eq:showagain1}. Therefore, to establish \eqref{eq:showagain1}, in view of \eqref{eq:zerocon}. it would then suffice to show that 

\begin{align}\label{eq:showagain12}
    \limsup\limits_{M_2\to 0}\sup_{k\vep\le T}\sup_{\lVert x\rVert\le M_1}(\exp(R_{\infty,\vep}(x))-\exp(R_{M_2,\vep}(x)))=0.
\end{align}

\begin{align*}
        v_{k\vep}'(x)=\frac{1}{2}f'(x)-g'(x^{u_{k\vep}})u_{k\vep}''(x)+\frac{u_{k\vep}'''(x)}{u_{k\vep}''(x)},
    \end{align*}
    and by using \eqref{eq:matt2}, we also have:
    \begin{align*}
        v_{k\vep}''(x)=\frac{1}{2}f''(x)-g''(x^{u_{k\vep}})(u_{k\vep}''(x))^2-g'(x^{u_{k\vep}})u_{k\vep}'''(x)+\frac{u_{k\vep}^{(4)}(x)}{u_{k\vep}''(x)}+w_{k\vep}'''(x^{u_{k\vep}})u_{k\vep}'''(x).
    \end{align*}

    Plugging the two observations above into \eqref{eq:matid11}, we get:
    
Towards this direction, note that 
$$\sup_{k\vep\le T}\sup_{\lVert x\rVert \le M_1} \lVert x^{u_{k\vep}}\rVert \le M_1 C_T+\sup_{t\in [0,T]}\lVert \nabla u_{t}(0)\rVert=:\ell_T.$$
by the same computation as in \eqref{eq:upbd1}, we get that:
\begin{align}\label{eq:keepshow}
    &\;\;\;\;\exp(R_{\infty,\vep}(x))-\exp(R_{M_2,\vep}(x))\nonumber \\&\le (C_T)^d \E_{Z_{\vep,C_T}}\bigg[\left(\E_{|Z_{\vep,C_T}}\exp\left(-f((x^{u_{k\vep}}+Z_{\vep,C_T})^{u_{k\vep}^*}+Z_{\vep,(C_T)^{-1}})\right)\right)^{-1}\nonumber \\ &\qquad \qquad \exp(-g(x^{u_{k\vep}}+Z_{\vep,C_T}))\bigg]\bm{1}(\lVert Z_{\vep,C_T}\rVert\ge M_2-\ell_T).
\end{align}
By taking $M_2\to\infty$ and using \cref{asn:solcon}, the conclusion in \eqref{eq:showagain12} and consequently \eqref{eq:showagain1} follows.

\vspace{0.1in} 

\noindent Now that we have reduced all relevant integrals to compact sets, we can use the standard Laplace approximation \cite{erdelyi1956asymptotic} to prove \eqref{eq:toshow2}. Towards this direction, we observe that for any $M>0$, 
$$\sup_{k\vep\le t}\sup_{\lVert y\rVert \le M}\bigg|\frac{1}{(2\pi\vep)^{\frac{d}{2}}}\exp\left(\frac{1}{\vep}\opV[u_{k\vep}](y)-\frac{1}{\vep}u_{k\vep}^*(y)+f(y^{u^*})+\frac{1}{2}\ldet\left(\frac{\partial y\hfill}{\partial y^{u_{k\vep}^*}}\right)\right)-1\bigg|\to 0,$$
as $\vep\to 0$. Applying the above coupled with \eqref{eq:showagain1} yields:
$$\sup_{k\vep\le t}\sup_{\lVert x\rVert \le M}\bigg|\exp\left(\frac{1}{\vep}\opS[u_{k\vep}](x)-\frac{1}{\vep}u_{k\vep}(x)-f(x)+g(x^{u_{k\vep}})-\frac{1}{2}\ldet\left(\frac{\partial x^{u_{k\vep}}}{\partial x\hfill}\right)\right)-1\bigg|\to 0,$$
as $\vep\to 0$. This completes the proof of \eqref{eq:step16}.

\vspace{0.1in}

\emph{Proof of \eqref{eq:step17}.} We will assume $\vep\le \vep_0$ from \cref{asn:Berman}. Define
$$\mathcal{G}_{\vep}(x):=\int \frac{\exp\left(\frac{1}{\vep}\langle x,y\rangle - \frac{1}{\vep}u_{k\vep}(x)-\frac{1}{\vep}u_{k\vep}^*(y)-g(y)-f(x)+h_{k\vep}(x)\right)}{\int \exp\left(\mfR{\xi_{k,\vep}^{(2)}}{z}{x}+\frac{1}{\vep}\langle z,y\rangle -\frac{1}{\vep}u_{k\vep}(x)-\frac{1}{\vep}u^*_{k\vep}(y)-f(z)\right)\,dz}\,dy.$$
By using \eqref{eq:step16}, it suffices to show that 
\begin{align}\label{eq:17show}
\int \big|\log{\mathcal{G}_{\vep}(x)}\big|\exp(-h_{k\vep}(x))\,dx\to 0,
\end{align}
as $\vep\to 0$. By \cref{asn:Berman}, it follows that:
\begin{align}\label{eq:17step1}
&\;\;\;\;\big|\mfR{\xi_{k,\vep}^{(2)}}{z}{x}\big|\nonumber\\ &\le (a_T+b_T\lVert x\rVert + b_T\lVert z-x\rVert)\lVert z-x\rVert\nonumber \\ &\le \big(a_T+b_T\lVert x\rVert\big) \big(\lVert z-y^{u_{k\vep}^*}\rVert+\lVert x-y^{u_{k\vep}^*}\rVert\big)+2b_T\big(\lVert z-y^{u_{k\vep}^*}\rVert^2+\lVert x-y^{u_{k\vep}^*}\rVert^2\big)\nonumber \\ &\le \big(a_T+b_T\lVert x\rVert\big) \big(\lVert z-y^{u_{k\vep}^*}\rVert+(c_T)^{-1}\lVert x^{u_{k\vep}}-y\rVert\big)+2b_T\big(\lVert z-y^{u_{k\vep}^*}\rVert^2+(c_T)^{-2}\lVert x^{u_{k\vep}}-y\rVert^2\big).
\end{align}
By the same computation as in \eqref{eq:upbd1} and \eqref{eq:lbd1}, we have:
\begin{align}\label{eq:17step2}
\mathcal{G}_{\vep}(x)&\le \E_{Z_{\vep,C_T}}\bigg[\bigg(\E_{|Z_{\vep,C_T}}\exp\bigg(-(a_T+b_T\lVert x\rVert)\lVert Z_{\vep,(C_T)^{-1}}\rVert-2b_T \lVert Z_{\vep,(C_T)^{-1}}\rVert^2\nonumber \\&-f((x^{u_{k\vep}}+Z_{\vep,C})^{u_{k\vep}^*}+Z_{\vep,(C_T)^{-1}})\bigg)\bigg)^{-1}\exp\big(-g(x^{u_{k\vep}}+Z_{\vep,C_T})\nonumber \\ &+(c_T)^{-1}(a_T+b_T\lVert x\rVert)\lVert Z_{\vep,C_T}\rVert+2b_T(c_T)^{-2}\lVert Z_{\vep,C_T}\rVert^2\big)\bigg]=\omega_{a_T,b_T,2b_T,c_T,C_T,\vep}.
\end{align}
A lower bound also holds for $\mathcal{G}_{\vep}(x)$ with $C_T$ replaced by $c_T$. By \cref{asn:solcon} and the same computation as in \eqref{eq:zerocon}, it suffices to show that, for any fixed $M>0$,
\begin{equation}\label{eq:17step3}
    \sup_{\lVert x\rVert\le M}\log{\mathcal{G}_{\vep}(x)}\to 0,\quad \mbox{as}\ \vep\to 0.
\end{equation}
We can then repeat the same computation as in \eqref{eq:keepshow} which will imply, after invoking \eqref{eq:17step1}, that it suffices to show (by symmetry) that 
\begin{align}\label{eq:17step3}
\sup_{\lVert x\rVert\le M} \log{\int_{\lVert y\rVert\le M} \frac{\exp\left(R^{(2,\vep)}(x,y)+\frac{1}{\vep}\langle x,y\rangle - \frac{1}{\vep}u_{k\vep}(x)-\frac{1}{\vep}u_{k\vep}^*(y)-g(y)-f(x)+h_{k\vep}(x)\right)}{\int \exp\left(-R_x^{(1,\vep)}(y,z)+\frac{1}{\vep}\langle z,y\rangle -\frac{1}{\vep}u_{k\vep}(x)-\frac{1}{\vep}u^*_{k\vep}(y)-f(z)\right)\,dz}\,dy}\to 0,
\end{align}
as $\vep\to 0$ for every fixed $M$, where 
$$R_x^{(1,\vep)}(y,z):=(a_T+b_T\lVert x\rVert)\lVert z-y^{u_{k\vep}^*}\rVert+2b_T\lVert z-y^{u_{k\vep}^*}\rVert^2,$$
and 
$$R^{(2,\vep)}(x,y):=(a_T+b_T\lVert x\rVert)\lVert z-y^{u_{k\vep}^*}\rVert+2b_T\lVert z-y^{u_{k\vep}^*}\rVert^2.$$
The conclusion, i.e., \eqref{eq:17step3} now follows by applying the Laplace approximation \cite{erdelyi1956asymptotic} as before, to the numerator and the denominator separately.
\end{proof}

Consider an SDE of the form
\begin{equation}\label{eq:sdegen}
dY_t = b(t,Y_t)dt + \sigma(t, Y_t) dB_t,
\end{equation}
where $B$ is a standard multidimensional Brownian motion and with an initial condition $Y_0=y_0$ ({\color{blue} In our case $Y_0$ is non-degenerate. Does that change anything?}). 


Stroock and Varadhan proved that (see \cite[Theorem 5.11]{klebaner}) that if $(y,t)\mapsto \sigma(y,t)$ is continuous, positive and if, for each $T>0$, there is a constant $K_T>0$ such that 
\begin{equation}\label{eq:weakexist}
\abs{b(y,t)} + \abs{\sigma(y,t)} \le K_T(1+ \abs{y}),
\end{equation}
for all $(y,t)\in \R^d \times [0,T]$, then the SDE \eqref{eq:sdegen} admits a unique weak solution for any $y_0$ which is also strong Markov. By stopping the process when it hits a ball of radius $R$, we can replace \eqref{eq:weakexist} by a local linear growth criterion: weak existence and uniqueness holds if for every $T>0$ and every $R>0$, there is a constant $K(T,R)$ such that   
\begin{equation}\label{eq:weakexistnew}
\sup_{\abs{y}\le R,\; 0\le t\le T }\left[ \abs{b(y,t)} + \abs{\sigma(y,t)}\right] \le K(T,R)(1+ \abs{y}),
\end{equation}

In the case of \eqref{eq:dualdiffSDE}, 
\[
b(y,t)= -\frac{\partial h_t\hfill}{\partial y^{w_t}}(y^{w_t}),\quad \sigma^2(y,t)= 2\frac{\partial y\hfill}{\partial y^{w_t}}. 
\]
As $(\nabla u_t)_{\#}\rho_t=\exp(-g)$ and $\rho_t=\exp(-h_t)$, \eqref{eq:pma} implies $$h_t(x)=\frac{\partial}{\partial t}u_t(x)-f(x).$$
Then writing both $b(y,t)$ and $\sigma(y,t)$ in terms of $u_t$, we get:
\[
b(y,t)= -\nabla^2 u_t(y^{w_t})\left(\frac{\partial}{\partial y}\frac{\partial}{\partial t}u_t(y^{w_t})-\frac{\partial}{\partial y}f(y^{w_t})\right),\quad \sigma^2(y,t)= 2\nabla^2 u_t(y^{w_t}). 
\]
By Assumptions \ref{asn:smoothfg} and \ref{asn:solcon}, both $b(\cdot,\cdot)$ and $\sigma(\cdot,\cdot)$ are continuous functions on $\{(y,t): |y|\le R,\ 0\le t\le T\}$, which in turn, yields \eqref{eq:weakexistnew}.



\end{comment}
