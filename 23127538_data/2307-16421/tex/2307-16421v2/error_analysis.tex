\begin{comment}
\section{Deviation from the PMA}\label{sec:error}


In this section we generalize the proof of Berman to quantify deviations of Sinkhorn from the PMA.

For this analysis let $(u_t,\; t\ge 0)$ be the solution of the PMA. Also, consider the following two transition probabilities. 
\[
\begin{split}
P^u_\vep(x \mid y) &= \frac{1}{(2\pi \vep)^{d/2}}\exp\left(  -\frac{1}{\vep} u(x) - f(x) - \frac{1}{2\vep} \norm{x-y}^2 - \frac{1}{\vep}\opV[u](y) \right)\\
\end{split}
\]
And
\[
Q^v_{\vep}(y \mid z) = \frac{1}{(2\pi \vep)^{d/2}}\exp\left(  -\frac{1}{\vep} v(y) - g(y) - \frac{1}{2\vep} \norm{y-z}^2 - \frac{1}{\vep}\opU[v](z) \right).
\]


When $u=u_t$, for some $t$, under appropriate regularity assumptions on the PMA, we have shown the following Gaussian approximations. 
For example,
\begin{equation}\label{eq:gapproxp}
\begin{split}
P^{\opS[u]}_\vep(\cdot\mid y) &\approx 
N\left( y^*-\vep  \left(2 \frac{\partial f}{\partial y}(y^*) - \frac{\partial h}{\partial y}(y^*)\right), \; \vep \left(\frac{\partial y^*}{\partial y}\right) \right),
\end{split}
\end{equation}
Thus the Markov transition kernel 
\begin{equation}\label{eq:gapproxq}
Q^{\opV[u]}_\vep(\cdot \mid z) \approx N\left( z_* + \vep \frac{\partial}{\partial z} f(z)- \frac{\vep}{2} \frac{\partial}{\partial z} g(z_*)- \frac{\vep}{2} \frac{\partial}{\partial z} h(z), \vep \left(\frac{\partial z_*}{\partial z}\right)\right).
\end{equation}
The errors for both these approximations are $o(\epsilon)$.

\subsection{From Log Sinkhorn Iterations to Sinkhorn}

We will now perform a \textit{change of measure} tilting to the log-Sinkhorn iterates. We proceed by induction. 

Suppose after $k$ iterations, the joint density in Sinkhorn is given by 
\[
\gamma_k'(x,y)=\frac{1}{(2\pi \vep)^{d/2}}\exp\left(  - \frac{1}{2\vep} \norm{y-x}^2 - \frac{1}{\vep}u^\vep_k(x) - \frac{1}{\vep} v^\vep_{k+1}(y)-f(x) - g(y)\right),
\]
where $v^\vep_{k+1}=\opV[u^\vep_k]$. Thus, the $Y$ marginal is exactly $e^{-g}$, while we denote the $X$ marginal by $\rho_k$. We are going to compare the $X$ marginal of this joint density with another one computed from the PMA. 

Recursively define the following quantities which are functions of the PMA. 
\[
\hat{u}_{(k+1)\vep}:= \opS[u_{k\vep}], \quad \hat{v}_{(k+1)\vep}:= \opV[u_{k\vep}].
\]
Let $\xi_k'$ be the joint density 
\[
\xi_k'(x,y)= \frac{1}{(2\pi \vep)^{d/2}} \exp\left( -\frac{1}{2\vep}\norm{x-y}^2 - \frac{1}{\vep} {u}_{k\vep}(x) - \frac{1}{\vep} \hat{v}_{(k+1)\vep}(y) - f(x) - g(y) \right).
\]
It is easy to see that the above joint density has $Y$-marginal $e^{-g}$ and two conditional densities given by $q_\vep^{\opV[u_{k\vep}]}(\cdot \mid x)$ and $p^{u_{k\vep}}(\cdot \mid y)$. 
These are the densities of the probability measures \eqref{eq:gapproxq} and \eqref{eq:gapproxp}, respectively. The $X$-marginal of $\xi_k'$ is given by 
\[
\rho^*_k(x)=\exp\left( \frac{1}{\vep}\hat{u}_{(k+1)\vep} - \frac{1}{\vep} u_{k\vep} - f(x) \right),
\]
while we denote the $X$-marginal of $\gamma_k'$ by $\rho^\vep_k$. 

Now, there exists $(a^\vep_k, b^\vep_k)$ such that 
\[
\gamma_k'(x,y)= \xi_k'(x,y) e^{-a^\vep_k(x) - b^\vep_k(y)}. 
\]
This is because if we let $a^\vep_k(x):=\frac{1}{\vep}\left(u^\vep_k(x)- u_{k\vep}(x)\right)$, then
\[
\begin{split}
b^\vep_k(y):=\frac{1}{\vep}\left(v^\vep_{k+1}(y)-\hat{v}_{(k+1)\vep}\right)&= \log \int e^{-a^\vep_k(x)} p_\vep^{u_{k\vep}}(x\mid y)dx. 
\end{split}
\]


Thus, the $X$-marginal is given by 
\[
\begin{split}
\rho^\vep_k(x)&= e^{-a^\vep_k(x)} \int e^{-b^\vep_k(y)} \xi_k'(x,y)dy= e^{-a^\vep_k(x)} \rho_k^*(x) \int e^{-b^\vep_k(y)} q_\vep^{\hat{v}_{k+1}}(y\mid x)dy\\
&={\color{red}\rho_k^*(x) e^{-a^\vep_k(x)}\int e^{-a^\vep_k(x)}\left( \int p_\vep^{u_{k\vep}}(x\mid y) q_\vep^{\hat{v}_{k+1}}(y\mid x)dy\right) dx}\\
&= e^{\hat{a}^\vep_{k+1}(x) - a^\vep_k(x)} \rho_k^*(x),
\end{split}
\]
where
{\color{red}\[
\hat{a}^\vep_{k+1}(x)=\log\int e^{-a^\vep_k(x)} \int p_\vep^{u_{k\vep}}(x\mid y) q_\vep^{\hat{v}_{k+1}}(y\mid x)dydx.
\]}

{\color{red} The integral with $p_\vep^{u_{k\vep}}(x\mid y)$ should be in the denominator??} 

Now, the $X$ marginal density from the PMA is given by approximately 
\[
\rho_{k\vep}= \exp\left( \frac{1}{\vep}\left( u_{(k+1)\vep} - u_{k\vep} \right) - f(x) + o_\vep(1)\right).
\]
Thus 
\[
\rho^\vep_k(x)= \exp\left( \hat{a}^\vep_{k+1}(x) - a^\vep_k(x) + \frac{1}{\vep}\left( \hat{u}_{(k+1)\vep} - u_{(k+1)\vep}\right) +o_\vep(1) \right) \rho_{k\vep}
\]


\textbf{Goal:} Our goal is to show that the term $\rho_k(x) \rightarrow \rho_{k\vep}$ as $\vep \rightarrow 0+$ and $k=O(1/\vep)$.  



Towards this goal we compare between $a^\vep_{k+1}$ and $\hat{a}^\vep_{k+1}$. 
By definition
\[
\begin{split}
a^\vep_{k+1}(x)&= \frac{1}{\vep} \left( u^\vep_{k+1}(x) - u_{(k+1)\vep}(x)\right)=\frac{1}{\vep} \left( \opU[v^\vep_{k+1}](x) - u_{(k+1)\vep}(x)\right).
\end{split}
\]
By a similar calculation as above
\[
\begin{split}
\opU[v^\vep_{k+1}](x)&=\vep \log \int \frac{1}{(2\pi \vep)^{d/2}} \exp\left( -\frac{1}{2\vep}\norm{y-x}^2 - \frac{1}{\vep}v^\vep_{k+1}(y) - g(y)\right)dy\\
=&\vep \log \int \frac{1}{(2\pi \vep)^{d/2}} \exp\left( -\frac{1}{2\vep}\norm{y-x}^2 - \frac{1}{\vep}\hat{v}^\vep_{k+1}(y) - b^\vep_k(y) - g(y)\right)dy\\
=& \hat{u}_{(k+1)\vep}(x) + \vep \log \int e^{-b^\vep_k(y)} q_\vep^{\hat{v}^\vep_{k+1}}(y\mid x)dy\\
=& {\color{red}\hat{u}_{(k+1)\vep}(x) + \vep (\hat{a}_{k+1}(x) - a^\vep_k(x))}. 
\end{split}
\]
{\color{red} Should it just be $\hat{a}_{k+1}^{\vep}(x)$ instead of $(\hat{a}^{\vep}_{k+1}(x) - a^\vep_k(x))$??}
On the other hand, the terms below are purely functions of the PMA. 
\[
\frac{1}{\vep}\left( \hat{u}_{k+1}(x) - u_{(k+1)\vep}(x) \right)=\frac{1}{\vep}\left( \opS[u_{(k\vep)}] - u_{(k+1)\vep} \right).
\]
Thus, by adding the two terms above, we have derived
\[
a^\vep_{k+1}(x)= (\hat{a}^\vep_{k+1}(x) - a^\vep_k(x)) - \frac{1}{\vep}\left( u_{(k+1)\vep} - \opS[u_{(k\vep)}] \right).
\]
Thus
\[
\rho^\vep_k(x)= e^{a_{k+1}^\vep(x) + o(1)} \rho_{k\vep}(x).
\]

The strategy is now to replicate the proof of Berman with these lower order potentials $(a^\vep_k, b^\vep_k)$. 
Start with the initial condition $a^\vep_0=0$.
We show that this new sequence of potentials $(a^\vep_k,b^\vep_k)$ satisfies a new recursive formula that converges to zero as $\vep \rightarrow 0$.  We show below that both sequences are $o(\vep)$. 

Let 
\[
\delta_{(k+1)\vep}:= \frac{1}{\vep}\left( u_{(k+1)\vep} - \opS[u_{(k\vep)}] \right).
\]
We know that $\delta_{(k+1)\vep}=o(\vep)$. Since $a^\vep_0=0$, then $\hat{a}^\vep_1=0$. Thus 
\[
a^\vep_1=-\delta_{\vep}=o(\vep). 
\]
It follows by induction that $a^\vep_k=k o(\vep)$. Thus, for $k=O(1/\vep)$, $a^\vep_k\rightarrow 0$. 

It is also immediate that the argument works if $a^\vep_0=o(\vep)$, so it should also give us multidimensional convergence. 

\end{comment}










