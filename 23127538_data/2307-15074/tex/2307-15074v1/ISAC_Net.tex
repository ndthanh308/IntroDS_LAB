%\documentclass[12pt, draftclsnofoot, onecolumn]{IEEEtran}
%\documentclass[10pt]{IEEEtran}
\documentclass[journal, 12pt, draftcls,onecolumn]{IEEEtran}
\usepackage{graphicx}
\usepackage{subfigure}
\usepackage{float}
\usepackage{amsmath}
\usepackage{epstopdf}
\usepackage{cases}
\usepackage{color}
\usepackage{algorithm}  
\usepackage{booktabs}
\usepackage{algorithmic}
%\usepackage{algpseudocode} 
\usepackage{amsmath}  
\usepackage{verbatim}
\usepackage{cite}

\usepackage[textsize=small]{todonotes}


\renewcommand{\algorithmicrequire}{\textbf{Input:}}  % Use Input in the format of Algorithm  
\renewcommand{\algorithmicensure}{\textbf{Output:}} % Use Output in the 


\usepackage{lipsum}
\makeatletter
\newenvironment{breakablealgorithm}
{% \begin{breakablealgorithm}
	\begin{center}
		\refstepcounter{algorithm}% New algorithm
		\hrule height.8pt depth0pt \kern2pt% \@fs@pre for \@fs@ruled
		\renewcommand{\caption}[2][\relax]{% Make a new \caption
			{\raggedright\textbf{\ALG@name~\thealgorithm} ##2\par}%
			\ifx\relax##1\relax % #1 is \relax
			\addcontentsline{loa}{algorithm}{\protect\numberline{\thealgorithm}##2}%
			\else % #1 is not \relax
			\addcontentsline{loa}{algorithm}{\protect\numberline{\thealgorithm}##1}%
			\fi
			\kern2pt\hrule\kern2pt
		}
	}{% \end{breakablealgorithm}
	\kern2pt\hrule\relax% \@fs@post for \@fs@ruled
\end{center}
}
\makeatother




\ifCLASSINFOpdf
% \usepackage[pdftex]{graphicx}
% declare the path(s) where your graphic files are
% \graphicspath{{../pdf/}{../jpeg/}}
% and their extensions so you won't have to specify these with
% every instance of \includegraphics
% \DeclareGraphicsExtensions{.pdf,.jpeg,.png}
\else
% or other class option (dvipsone, dvipdf, if not using dvips). graphicx
% will default to the driver specified in the system graphics.cfg if no
% driver is specified.
% \usepackage[dvips]{graphicx}
% declare the path(s) where your graphic files are
% \graphicspath{{../eps/}}
% and their extensions so you won't have to specify these with
% every instance of \includegraphics
% \DeclareGraphicsExtensions{.eps}
\fi
% graphicx was written by David Carlisle and Sebastian Rahtz. It is
% required if you want graphics, photos, etc. graphicx.sty is already
% installed on most LaTeX systems. The latest version and documentation
% can be obtained at: 
% http://www.ctan.org/pkg/graphicx
% Another good source of documentation is "Using Imported Graphics in
% LaTeX2e" by Keith Reckdahl which can be found at:
% http://www.ctan.org/pkg/epslatex
%
% latex, and pdflatex in dvi mode, support graphics in encapsulated
% postscript (.eps) format. pdflatex in pdf mode supports graphics
% in .pdf, .jpeg, .png and .mps (metapost) formats. Users should ensure
% that all non-photo figures use a vector format (.eps, .pdf, .mps) and
% not a bitmapped formats (.jpeg, .png). The IEEE frowns on bitmapped formats
% which can result in "jaggedy"/blurry rendering of lines and letters as
% well as large increases in file sizes.
%
% You can find documentation about the pdfTeX application at:
% http://www.tug.org/applications/pdftex





% *** MATH PACKAGES ***
%
%\usepackage{amsmath}
% A popular package from the American Mathematical Society that provides
% many useful and powerful commands for dealing with mathematics.
%
% Note that the amsmath package sets \interdisplaylinepenalty to 10000
% thus preventing page breaks from occurring within multiline equations. Use:
%\interdisplaylinepenalty=2500
% after loading amsmath to restore such page breaks as IEEEtran.cls normally
% does. amsmath.sty is already installed on most LaTeX systems. The latest
% version and documentation can be obtained at:
% http://www.ctan.org/pkg/amsmath





% *** SPECIALIZED LIST PACKAGES ***
%
%\usepackage{algorithmic}
% algorithmic.sty was written by Peter Williams and Rogerio Brito.
% This package provides an algorithmic environment fo describing algorithms.
% You can use the algorithmic environment in-text or within a figure
% environment to provide for a floating algorithm. Do NOT use the algorithm
% floating environment provided by algorithm.sty (by the same authors) or
% algorithm2e.sty (by Christophe Fiorio) as the IEEE does not use dedicated
% algorithm float types and packages that provide these will not provide
% correct IEEE style captions. The latest version and documentation of
% algorithmic.sty can be obtained at:
% http://www.ctan.org/pkg/algorithms
% Also of interest may be the (relatively newer and more customizable)
% algorithmicx.sty package by Szasz Janos:
% http://www.ctan.org/pkg/algorithmicx




% *** ALIGNMENT PACKAGES ***
%
%\usepackage{array}
% Frank Mittelbach's and David Carlisle's array.sty patches and improves
% the standard LaTeX2e array and tabular environments to provide better
% appearance and additional user controls. As the default LaTeX2e table
% generation code is lacking to the point of almost being broken with
% respect to the quality of the end results, all users are strongly
% advised to use an enhanced (at the very least that provided by array.sty)
% set of table tools. array.sty is already installed on most systems. The
% latest version and documentation can be obtained at:
% http://www.ctan.org/pkg/array


% IEEEtran contains the IEEEeqnarray family of commands that can be used to
% generate multiline equations as well as matrices, tables, etc., of high
% quality.

% *** SUBFIGURE PACKAGES ***
%\ifCLASSOPTIONcompsoc
%  \usepackage[caption=false,font=normalsize,labelfont=sf,textfont=sf]{subfig}
%\else
%  \usepackage[caption=false,font=footnotesize]{subfig}
%\fi
% subfig.sty, written by Steven Douglas Cochran, is the modern replacement
% for subfigure.sty, the latter of which is no longer maintained and is
% incompatible with some LaTeX packages including fixltx2e. However,
% subfig.sty requires and automatically loads Axel Sommerfeldt's caption.sty
% which will override IEEEtran.cls' handling of captions and this will result
% in non-IEEE style figure/table captions. To prevent this problem, be sure
% and invoke subfig.sty's "caption=false" package option (available since
% subfig.sty version 1.3, 2005/06/28) as this is will preserve IEEEtran.cls
% handling of captions.
% Note that the Computer Society format requires a larger sans serif font
% than the serif footnote size font used in traditional IEEE formatting
% and thus the need to invoke different subfig.sty package options depending
% on whether compsoc mode has been enabled.
%
% The latest version and documentation of subfig.sty can be obtained at:
% http://www.ctan.org/pkg/subfig




% *** FLOAT PACKAGES ***
%
%\usepackage{fixltx2e}
% fixltx2e, the successor to the earlier fix2col.sty, was written by
% Frank Mittelbach and David Carlisle. This package corrects a few problems
% in the LaTeX2e kernel, the most notable of which is that in current
% LaTeX2e releases, the ordering of single and double column floats is not
% guaranteed to be preserved. Thus, an unpatched LaTeX2e can allow a
% single column figure to be placed prior to an earlier double column
% figure.
% Be aware that LaTeX2e kernels dated 2015 and later have fixltx2e.sty's
% corrections already built into the system in which case a warning will
% be issued if an attempt is made to load fixltx2e.sty as it is no longer
% needed.
% The latest version and documentation can be found at:
% http://www.ctan.org/pkg/fixltx2e


%\usepackage{stfloats}
% stfloats.sty was written by Sigitas Tolusis. This package gives LaTeX2e
% the ability to do double column floats at the bottom of the page as well
% as the top. (e.g., "% Figure environment removed

In the conventional ISAC system, the communication signal detection and passive sensing are carried out in serial. First, the pilot signal is adopted to realize the preliminary passive sensing and channel estimation, and then the signal detection is carried out on the data signal. Then further passive sensing is realized according to the detected data signal. 
This scheme leads to the interference between the error of signal detection and that of passive sensing, which reduces the communication and sensing performance of the system. 

In this subsection, we propose a model-driven ISAC-NET. As $\mathrm{Fig.\ \ref{fig:ISAC_NET}}$ 
\footnote{$\bf H$ represents the channel matrix set ${\bf H}^l$ of $L_p$ paths; $\hat {\bf H}$ represents the estimated channel matrix; $\hat {\bf X}_d$ represents the estimated transmitted signal matrix;} 
shows, there are two parts in the ISAC-NET architecture:
\begin{itemize}
	\color{black}
	\item Pilot signal processing: In this part, since both the transmitted pilot signal and the received pilot signal are known, the traditional channel estimation method can be used to obtain the pilot channel information, $\bf \hat H$. Based on the pilot channel information, the traditional passive sensing algorithm, 2D-DFT, can be used to complete the passive sensing and obtain the rough ranging and velocity measurement results, $\left[\bf r, \bf v\right]$.
	\item ISAC-NET: In this part, we adopt the ISAC signal processing method based on model-driven DL. Compared with the traditional receiver architecture where the signal detector and passive sensing are designed separately, the ISAC-NET considers the characteristics of passive sensing errors in addition to the channel statistics when carrying out signal detection, and uses the detected data signals for passive sensing.
\end{itemize}
In comparison to the data-driven DL-based joint channel estimation and signal detection architecture proposed in \cite{[Main_16],[COM_4]}, which uses a large number of data to train the black-box-based network, we construct the network architecture by employing model-driven DL.
%
Different with the model-driven DL-based joint channel estimation and signal detection architecture proposed in \cite{[Main]}, which only optimizes the performance of communication signal detection without taking into account the performance of passive sensing, the ISAC-NET can optimize both communication and sensing performance to achieve an optimal trade off between the two functions.
%
The structure of the ISAC-NET is illustrated in $\mathrm{Fig.\ \ref{fig:ISAC_NET_1}}$, which consists of $K$ cascade layers and each has the same architecture but different trainable parameters.
Each layer of ISAC-NET contains three main modules, the signal detection module, the passive sensing module and the channel reconstruction module, which will be introduced in the following subsection.

\begin{breakablealgorithm} 
	\caption{ISAC signal processing algorithm}  
	\label{alg:CI_ISAC}  
	\begin{algorithmic} 
		\REQUIRE 
		Received data signal ${\bf Y}_d$, 
		estimated channel matrix ${{{\bf H}^l}^{k}}$
		and
		estimated data  signal $ {\bf X}_d^{k}$.
		\STATE \quad 1). Initialize: {\color{black} ${\bf Z}_{d}^1 = {\bf 0}$, ${\bf X}_{d}^1 = {\bf 0}$}.
		\STATE \quad 2). Update ${\bf X}_{d}^{k+1}$: 
		\begin{equation}\label{equ:ISAC_NET_layer_1}
			{\bf X}_{d}^{k+1} = \eta_1^k {\bf X}_{d}^{k} + \sum_{l=0}^{L_p -1} \left({\hat {{{\bf H}^l_a}^k}}\right)^T {\bf Z}_{d}^{k} \left({\hat {{ {\bf H}^l_{\varphi}}^k}}\right)^T.
		\end{equation} \\
		\STATE \quad 3). Update ${\bf Z}_{d}^{k+1}$: 
		\begin{equation}\label{equ:ISAC_NET_layer_2}
			{\bf Z}_{d}^{k+1} =  \eta_2^k {\bf Z}_{d}^{k} + {\bf Y}_d - \sum_{l=0}^{L_p -1} \hat {{\bf H}^l_a}^k {\bf X}_{d}^{k} \hat {{\bf H}^l_{\varphi}}^k.
		\end{equation} \\
		\WHILE{TRUE}
		\STATE {\color{black} $\left[{\bf r}^{k+1}, {\bf v}^{k+1} \right]  = \left[ \left[\quad\right],\left[\quad\right]\right]$}.
		\STATE ${\bf H}^l_{new}$ = ${{\bf H}^l}^k = \left \{ {{\bf H}^l}_a^k , {{\bf H}^l}_{\varphi}^k \right \}$.
		\STATE 4). Passive sensing by 2D-DFT: 
		\begin{equation}\label{equ:ISAC_NET_layer_3}
			\left[M_v, i_v \right] = {\mathsf {DFT}} \left({\bf X}_{d}^{k+1}, {\bf Y}_{d} \right),
		\end{equation} 
		\begin{equation}\label{equ:ISAC_NET_layer_3}
			\left[M_r, i_r \right] = {\mathsf {IDFT}} \left({\bf X}_{d}^{k+1}, {\bf Y}_{d} \right),
		\end{equation} 
		\STATE where $M_v$ and $M_r$ are the maximum DFT and IDFT outputs, {\color{black} $i_v$ and $i_r$ are the index of the maximum DFT and IDFT outputs}.
		\IF {$M_v \ge \eta_3^k$ or $M_r \ge \eta_3^k$}
		\STATE Obtain $v_{tmp}$ and $r_{tmp}$ based on \eqref{equ:Doppler} and \eqref{equ:distance}.
		%		\STATE ${\bf r}^{k+1}$.append($r_{tmp}$) 
		%		\STATE ${\bf v}^{k+1} $.append($v_{tmp}$)
		\STATE ${\bf r}^{k+1} = \left[{\bf r}^{k+1}, r_{tmp} \right]$.
		\STATE ${\bf v}^{k+1} = \left[{\bf v}^{k+1}, v_{tmp} \right]$.
		\STATE Reconstruct ISAC channel: 
		\begin{equation}\label{equ:ISAC_NET_layer_4}
			{\bf H}^l_{tmp} = {\mathsf {CHA}} \left(r_{tmp}, v_{tmp} \right),
		\end{equation} 
		\begin{equation}\label{equ:ISAC_NET_layer_4}
			{\bf H}^l_{new} = {{\bf H}^l}^k - {\bf H}^l_{tmp}.
		\end{equation} \\
		\ELSE
		\STATE break.
		\ENDIF
		\ENDWHILE
		\STATE 5). Update $ \left[{\bf r}^{k+1}, {\bf v}^{k+1} \right]$: 
		\begin{equation} \left[{\bf r}^{k+1}, {\bf v}^{k+1} \right] = \eta_4^k \left[{\bf r}^{k}, {\bf v}^{k} \right] + \left[{\bf r}^{k+1}, {\bf v}^{k+1} \right]
		\end{equation} \\
		\STATE 6). Reconstruct ISAC channel: 
		\begin{equation}\label{equ:ISAC_NET_layer_5}
			{{\bf H}^l}^{k+1} = {\mathsf {CHA}} \left([{\bf r}^{k+1}, {\bf v}^{k+1} ] \right).
		\end{equation} \\
		\STATE 7). Update ${{\bf H}^l}^{k+1}$: 
		\begin{equation}\label{equ:ISAC_NET_layer_6}
			{{\bf H}^l}^{k+1} = \eta_5^k {{\bf H}^l}^{k} + {{\bf H}^l}^{k+1}.
		\end{equation} \\
		\ENSURE Recovered data signal ${\bf X}_{d}^{K+1}$, estimated range and velocity of targets $\left[{\bf r}^{K+1}, {\bf v}^{K+1} \right]$.
	\end{algorithmic}  
\end{breakablealgorithm}

\subsection{Each Layer of ISAC-NET}\label{sec:ISAC-NET-2-1}

% Figure environment removed

As $\mathrm{Fig.\ \ref{fig:ISAC_NET_layer}}$ 
shows, for the $k$-th layer of the ISAC-NET, 
{\color{black}
\begin{itemize}
	\item the input is the estimated data signal $ {\bf X}_d^{k}$, the received data signal ${\bf Y}_d$, the estimated channel matrix ${\bf H}^k$ and the noise matrix ${\bf Z}^{k}$ from the $(k-1)$-th layer. 
	\item The ISAC signal processing module of each layer can be performed as in $\mathrm{Algorithm\ \ref{alg:CI_ISAC}}$, which is developed by unfolding the improved D-AMP algorithm, 2D-DFT algorithm and ISAC channel reconstruction algorithm.  
	\item The blue part of $\mathrm{Fig.\ \ref{fig:ISAC_NET_layer}}$ denotes the improved D-AMP algorithm, the green part denotes the 2D-DFT passive sensing algorithm, the yellow part denotes the ISAC channel reconstruction algorithm.
\end{itemize}
}
Compared with the conventional 2D-DFT passive sensing algorithm, the ISAC signal processing algorithm realizes the multi-target sensing by alternating between the estimation of target parameters and channel reconstruction.
This method has certain improvement in the performance of sensing to multiple targets, which is verified in the subsequent simulation analysis.
Moreover, there are five main learnable variables $\Omega^k = \left \{ \eta_1^k , \eta_2^k, \eta_3^k, \eta_4^k, \eta_5^k   \right \}$ in each layer, which is much less than the data-driven DL, greatly reducing the complexity of network training.

\subsection{Practical Implementation}\label{sec:ISAC-NET-2-2}

The developed ISAC-NET can be divided into the following stages. 
\begin{itemize}
	{\color{black}
	\item \textbf{Data generation:} 
	Inputs of the ISAC-NET is the received signal, Outputs of the ISAC-NET is the demodulated transmitted signal and estimated target velocity and target distance.
	We obtained training and testing data sets through MATLAB simulation, in which $95 \%$ data is used for training and $5 \%$ data is used for testing. The data set includes the transmitted signal set and the received signal set under different signal-to-noise ratio (SNR) conditions, from $-20 \sim 60$ dB. 
	The transmitted signal is obtained by modulating randomly generated bit data.
	The transmitted signal generates the received signal through the ISAC channel, which includes target parameters such as target velocity and target distance.
	%
	\item \textbf{Training settings:} 
	Each batch of training data is set up as an OFDM signal matrix containing 1024 subcarriers and 256 OFDM symbols. The batch size of each training epoch is 128, and the number of training epochs is 500. The learning rate is set to 0.001.
	}
	\item \textbf{Offline training stage:} 
	In the offline training stage, we obtain the optimized parameters, ${\bf \Omega} = \left \{\Omega^k \right \}_{k=1}^K$ for different SNRs based on the tensorflow platform. 
	\item \textbf{Deployment stage:} 
	The optimized parameters are stored to detect the
	modulated symbols in the deployment stage. 
	%
	The ISAC-NET can be interpreted as a new iterative detector after training. 
	The incorporated learnable parameters can adapt to practical channels, compensate for signal detection errors and passive sensing errors, and improve communication and sensing performance.
\end{itemize}
Although aforementioned implementation process is in an offline manner, the proposed ISAC-NET can also be implemented by online training to adapt to the fluctuations in the channel conditions owing to the superiority of its low demand for training data and computational resources.

\section{Simulation Results}\label{sec:Simulation}
{\color{black}
To demonstrate the relationship between the communication demodulation error and sensing accuracy in the ISAC system,
we first analyze the communication performance with sensing errors and the sensing performance in the condition of communication demodulation errors. 
To verify the improvement of ISAC-NET in the communication and sensing performance, the communication and sensing performance of the proposed ISAC-NET is simulated and compared with the typical communication and sensing signal processing algorithms.
It should be noted that, the SNR adopted in the simulation is denoted the SNR of the received signal, and each simulation in this paper is calculated over 500 Monte Carlo trials.
Simulation parameters used in this section are shown in $\mathrm{TABLE\ \ref{Parameter:simulation}}$ \cite{[5G_signal],[Main]}.
}

\begin{table}[h]
	\caption{Simulation parameters adopted in this paper.}
	\centering
	\label{Parameter:simulation}
	\begin{tabular}{c|c|l|c|c|l}
		\hline
		\hline
		Items & Value & Meaning of the parameter & Items & Value & Meaning of the parameter \\ \hline
		$f_c$ & 4 GHz & Carrier frequency & $\Delta f$ & $120$ kHz & Carrier frequency \\ \hline
		 $M$ & 256 & Number of OFDM symbols & $N$ & 1024 & Number of subcarriers \\ \hline
		\color{black} $M_I$ & \color{black} 2560 & \color{black} Number of DFT points & \color{black} $N$ & \color{black} 10240 &\color{black}  Number of IDFT points \\ \hline
		$T_p$ & 8.3 $\mu$s & OFDM symbol period & $T_c$ & 2.08 $\mu$s & CP period \\ \hline
		$T$ & 10.38 $\mu$s & The whole OFDM period & $B$ & 123 MHz & Frequency bandwidth \\ \hline
		$N_t$ & 8  & Number of transmitting antenna array & $N_r$ & 8  & Number of receiving antenna array \\ \hline
		$L$ & 1 $\sim$ 3  & Number of targets & $\sigma$ & 1 m$^2$ & RCS of targets \\ \hline
		16 QAM &   & Modulation type & SNR & $-20 \sim 60$ dB  & SNR of received signals \\ \hline
		$w_p$ & 1/14  & Pilot signal length ratio & &   &  \\ \hline
	\end{tabular}
\end{table}

\subsection{Communication and Sensing Performance Under Error Conditions}\label{sec:Simulation-1}

% Figure environment removed

% Figure environment removed

In this section, we analyze and simulate the communication and sensing performance in the condition of communication demodulation errors. 
{\color{black}
$\mathrm{Fig.\ \ref{fig:RMSE_BER}}$ presents the normalized mean squared errors (NMSEs) of range and velocity with different SNRs, which can be expressed as 
\begin{equation} \label{equ:Simulation_1_1}
	\begin{aligned}
		\Gamma_r &= \frac{1}{L} \sum_{l=0}^{L-1} \frac{\left |\hat r_{l} - r_l \right|^2}{r_l^2}, \\
		\Gamma_v &= \frac{1}{L} \sum_{l=0}^{L-1} \frac{\left|\hat v_{l} - v_l\right|^2}{v_l^2}
	\end{aligned},
\end{equation}
where $L$ is the number of targets, $r_l$ and $v_l$ are the actual range and velocity of the $l$-th target, $\hat r_{l}$ and $\hat v_{l}$ are the estimated range and velocity of the $l$-th target.
}
%
$\mathrm{Fig.\ \ref{fig:SER_BER_RMSE}}$ presents the bit error rate (BER) and symbol error rate (SER).

{\color{black}
As $\mathrm{Fig.\ \ref{fig:RMSE_R_BER}}$ and $\mathrm{Fig.\ \ref{fig:RMSE_V_BER}}$ show that, under the same BER condition, the NMSEs of range and velocity decrease with the increase of SNR, which means that the passive sensing accuracy is improved.
%
In addition, under the same SNR condition, the NMSEs of range and velocity increase with the increase of BER, which means that the passive sensing accuracy is decreased.
%
When the BER reaches 0.5, it indicates that the recovered communication signal is completely wrong. At this time, the NMSEs of range and velocity do not decrease with the increase of SNR, and the sensing accuracy can no longer decrease.
%
Therefore, we conclude that the accurate demodulation of the communication symbols is one of the preconditions for obtaining high-accurate passive sensing performance.
}

As $\mathrm{Fig.\ \ref{fig:BER_RMSE}}$ and $\mathrm{Fig.\ \ref{fig:SER_RMSE}}$ show that, under the same NMSE condition, with the increase of SNR, the BER and SER decrease, which means that the accuracy of communication demodulation is improved.
%
In addition, under the same SNR condition, the BER and SER increase with the increase of NMSE, which means that the accuracy of communication demodulation is decreased.
%
{\color{black}
When the NMSE reaches 100, the BER is close to 0.5, which indicates that the recovered communication signal is completely wrong. %, the SER is close to 1
%
Therefore, we conclude that accurate passive sensing is one of the preconditions for obtaining high-accurate communication performance.

The simulation results provide a theoretical basis for the proposed ISAC-NET based on communication and sensing iterative optimization.
}

% Figure environment removed

% Figure environment removed


% Figure environment removed

\subsection{Communication and Sensing Performance of ISAC-NET}\label{sec:Simulation-3}
{\color{black}
In this subsection, we analyze and simulate the communication and passive sensing performance of ISAC-NET. 
Since convergence is a necessary condition of the iterative algorithm, we first simulate and verify the convergence of ISAC-NET in sensing and communication.
After that, the communication and sensing performance of the proposed ISAC-NET is simulated and compared with the typical communication and sensing signal processing algorithms.
}
\subsubsection{Convergence Analysis}\label{sec:Simulation-3-1}
First, we analyze the convergence performance of $\mathrm{Algorithm\ \ref{alg:CI_ISAC}}$ and ISAC-NET. 
$\mathrm{Fig.\ \ref{fig:Convergence}}$ illustrates the BER and NMSE versus the number of layers (iterations) under $\mathrm{SNR\ =\ 0\ dB}$ and $\mathrm{SNR\ =\ 10\ dB}$.
%
From $\mathrm{Fig.\ \ref{fig:Conver_BER}}$, both the $\mathrm{Algorithm\ \ref{alg:CI_ISAC}}$ and
ISAC-NET converge within five layers (iterations) for all
the cases.
From $\mathrm{Fig.\ \ref{fig:Conver_NMSE_R}}$, both the $\mathrm{Algorithm\ \ref{alg:CI_ISAC}}$ and ISAC-NET converge within five layers (iterations) for all
the cases when $\mathrm{SNR\ =\ 10\ dB}$. 
{\color {black}
Moreover, the estimation error of ISAC-NET is smaller than that of $\mathrm{Algorithm\ \ref{alg:CI_ISAC}}$. 
%
}
Since $\mathrm{SNR\ =\ 0\ dB}$ does not meet the requirement of passive sensing, it is difficult to achieve high-accurate target detection, both the $\mathrm{Algorithm\ \ref{alg:CI_ISAC}}$ and ISAC-NET cannot converge.

\begin{table*}[ht]
	\centering
	\caption{Communication and sensing algorithms for simulation analysis}
	\label{label:algorithm}
	\begin{tabular}{l|l|l}
		\hline \hline
		Method & Communication & Sensing \\ \hline
		Perfect & \begin{tabular}[c]{@{}l@{}}MAP without CSI error\end{tabular}
		& 2D-DFT \cite{[OFDM]} without communication demodulation error  \\ \hline
		ISAC-NET & ISAC-NET with CSI error & \begin{tabular}[c]{@{}l@{}}ISAC-NET with communication demodulation error\end{tabular}
		\\ \hline
		OAMP-Net2 & OAMP-Net2 proposed in \cite{[Main]} with CSI error &  \\ \hline
		Conventional method & MAP with CSI error &  \\ \hline
		2D-DFT &  & 2D-DFT \cite{[OFDM]} with communication demodulation error \\ \hline
	\end{tabular}
\end{table*}

\subsubsection{Performance Comparison}\label{sec:Simulation-3-2}

{\color {black}
Through simulations, the communication and sensing performance of the proposed ISAC-NET is evaluated and compared to the performance of typical communication and sensing signal processing algorithms, which are listed in $\mathrm{TABLE\ \ref{label:algorithm}}$.
For the multi-targets detection scenario, the number of targets need to detect is set as three. 
The actual range and velocity of targets are $10\ \rm{m}$, $50\ \rm{m}$, $100\ \rm{m}$ and $5\ \rm{m/s}$, $10\ \rm{m/s}$, $15\ \rm{m/s}$ respectively.
}

$\mathrm{Fig.\ \ref{fig:BER_SER_ISAC_NET_1}}$ compares the BER and SER of MAP \cite{[Main]}, OAMP-Net2 \cite{[Main]}, Algorithm \ref{alg:CI_ISAC}, and ISAC-NET. 
The perfect line denotes the BER and SER without CSI. 
The conventional method refers to the direct demodulation algorithm of the received signal, the MAP algorithm, with channel estimation error.
{\color{black}
As $\mathrm{Fig.\ \ref{fig:BER_SER_ISAC_NET_1}}$ shows, ISAC-NET and OAMP-Net2 have similar BER and SER, smaller than Algorithm \ref{alg:CI_ISAC}, which can verify the communication performance gain of deep unfolding over conventional algorithm without unfolding. %(Algorithm \ref{alg:CI_ISAC}). 
}

$\mathrm{Fig.\ \ref{fig:NMSE_R_V_ISAC_NET_1}}$ compares the NMSE of the 2D-DFT algorithm \cite{[OFDM]} and ISAC-NET. 
The perfect line denotes the NMSE without communication demodulation errors. 
As $\mathrm{Fig.\ \ref{fig:NMSE_R_V_ISAC_NET_1}}$ shows, under the same SNR condition, NMSE of range and velocity by ISAC-NET is smaller than 2D-DFT. 
It means that ISAC-NET can obtain high-accurate passive sensing performance.
{\color{black}
Moreover, compared with the ISAC signal processing algorithm without unfolding (Algorithm \ref{alg:CI_ISAC}), ISAC-NET can obtain smaller NMSE of range and velocity, which can verify the sensing performance gain of deep unfolding over conventional algorithm without unfolding. %(Algorithm \ref{alg:CI_ISAC}).
}

\section{Conclusion}\label{sec:Conclusion}

In this paper, we proposed an integrated sensing and communication network based on model-driven DL, called ISAC-NET, that combines passive sensing with communication demodulation signal processing using model-driven DL.
%
{\color {black}
Dissimilar to existing passive sensing algorithm that the communication signal detection and passive sensing are carried out in serial.
The ISAC-NET could obtain passive sensing results and communication demodulated symbols simultaneously, 
which adopted the block-by-block signal processing method. The ISAC-NET was divided into the passive sensing module, signal detection module and channel reconstruction module.
}
Moreover, the communication and sensing performance improvements of ISAC-NET in range and velocity estimation, BER and SER are analyzed and simulated.
The simulation results verified that the ISAC-NET could obtain better communication and passive sensing performance.

\begin{appendices}

\section{Doppler frequency shift for passive sensing} \label{app:A}

% Figure environment removed
As $\mathrm{Fig.\ \ref{fig:PS_vs_AS_2}}$, the $l$-th target is moving with velocity ${v_l}'$ in the direction $\alpha^l$. The frequency of the ISAC signal at $\rm Tx$ is $f_c$.
Based on the source reference frame of relativistic Doppler effects \cite{[DP]}, the frequency of the ISAC signal arriving at the $l$-th target can be expressed as 
\begin{equation}\label{equ:PS_vs_AS_2_3}
	\begin{aligned}
		f_1  &= f_c \frac{1-\beta {\rm cos} \left(\alpha^l - \theta_t^l \right)}{ \sqrt {1 - \beta^2}}
	\end{aligned},
\end{equation}
where $\beta = \frac {{v_l}'}{c} \ll 1$ with $c$ being the velocity of light.
Similarly, the frequency of the ISAC signal at $\rm Rx$ can be given by
\begin{equation}\label{equ:PS_vs_AS_2_4}
	\begin{aligned}
		f_2  &= f_1 \frac{ \sqrt {1 - \beta^2}}{1+\beta {\rm cos} \left(\alpha^l - \theta_r^l \right)}
	\end{aligned}.
\end{equation}
The doppler frequency shift of the $l$-th target can be expressed as
\begin{equation}\label{equ:PS_vs_AS_2_5}
	\begin{aligned}
		f_{d,l} &= f_2 - f_c \\
		  &= f_c \left(  { \frac{ \sqrt {1 - \beta^2}}{1+\beta {\rm cos} \left(\alpha^l - \theta_r^l \right )}  \cdot  \frac{1-\beta {\rm cos} \left(\alpha^l - \theta_t^l \right)}{ \sqrt {1 - \beta^2}} }           -1 \right) \\
		  &= f_c \left( \frac{1-\beta {\rm cos} \left(\alpha^l - \theta_t^l \right)}{1+\beta {\rm cos} \left(\alpha^l - \theta_r^l \right)}     -1 \right) \\
		  &= -\beta f_c \frac{ {\rm cos} \left(\alpha^l - \theta_t^l \right) +  {\rm cos} \left(\alpha^l - \theta_r^l \right) }{1+\beta {\rm cos} \left(\alpha^l - \theta_r^l \right)}	 
	\end{aligned}.
\end{equation}
Based on $\beta = \frac {{v_l}'}{c} \ll 1$, the doppler frequency shift of the $l$-th target $f_{d,l}$ can be approximated as
\begin{equation}\label{equ:PS_vs_AS_2_6}
	\begin{aligned}
		f_{d,l} &\approx -\frac {{v_l}'}{c} f_c \cdot {\rm cos} \left(\frac{\theta_r^l - \theta_t^l}{2} \right) \cdot {\rm cos} \left(\alpha^l - \frac{\theta_r^l + \theta_t^l}{2} \right)
	\end{aligned}.
\end{equation}


\end{appendices}


\bibliographystyle{IEEEtran} 
\bibliography{reference}	



\ifCLASSOPTIONcaptionsoff
  \newpage
\fi



% trigger a \newpage just before the given reference
% number - used to balance the columns on the last page
% adjust value as needed - may need to be readjusted if
% the document is modified later
%\IEEEtriggeratref{8}
% The "triggered" command can be changed if desired:
%\IEEEtriggercmd{\enlargethispage{-5in}}

% references section

% can use a bibliography generated by BibTeX as a .bbl file
% BibTeX documentation can be easily obtained at:
% http://mirror.ctan.org/biblio/bibtex/contrib/doc/
% The IEEEtran BibTeX style support page is at:
% http://www.michaelshell.org/tex/ieeetran/bibtex/
%\bibliographystyle{IEEEtran}
% argument is your BibTeX string definitions and bibliography database(s)
%\bibliography{IEEEabrv,../bib/paper}
%
% <OR> manually copy in the resultant .bbl file
% set second argument of \begin to the number of references
% (used to reserve space for the reference number labels box)




% biography section
% 
% If you have an EPS/PDF photo (graphicx package needed) extra braces are
% needed around the contents of the optional argument to biography to prevent
% the LaTeX parser from getting confused when it sees the complicated
% \includegraphics command within an optional argument. (You could create
% your own custom macro containing the \includegraphics command to make things
% simpler here.)
%\begin{IEEEbiography}[{% Figure removed}]{Michael Shell}
% or if you just want to reserve a space for a photo:



% if you will not have a photo at all:


% insert where needed to balance the two columns on the last page with
% biographies
%\newpage



% You can push biographies down or up by placing
% a \vfill before or after them. The appropriate
% use of \vfill depends on what kind of text is
% on the last page and whether or not the columns
% are being equalized.

%\vfill

% Can be used to pull up biographies so that the bottom of the last one
% is flush with the other column.
%\enlargethispage{-5in}



% that's all folks
\end{document}


