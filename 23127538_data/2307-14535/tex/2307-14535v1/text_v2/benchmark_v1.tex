\textbf{Benchmark}
\label{sec:benchmark}
contains 18 tasks across 5 domains (Fig.~\ref{fig:tasks_panel} Tab. \ref{tab:benchmark}) highlights the following properties:

\vspace{-2mm}
\begin{itemize}[leftmargin=3mm]
      \item \textbf{Geometry Generalization.} In our bin transport domain, the robot must generalize its bin transport skill to unseen object instances, with novel shapes, sizes, and colors.
      \item  \textbf{Articulated Manipulation.} Due to object's complex geometry and articulation, the robot needs to utilize 6DoF manipulation skills.
      \item
            \textbf{Intuitive physics.}
            Robots should understand the physical properties of the world and use them to their advantage.
            Typically, this means to achieve a specific state, the robot needs to perform an informed precised manipulation.
            In the bus balance domain, the robot needs to figure out the precise grasping and placement pose to balance a large bus toy on a small block.
            In the catapult domain, where the robot places the block along a catapult arm determines how far the block will be launched, and, thus, which bin (if any) the block will land in.

      \item \textbf{Common-sense reasoning \& Tool-use.}
            Natural language task description is user-friendly but often under-specifies the task.
            Common-sense can help to fill in the gaps.
            In the mailbox domain, given the task ``send the package for return'', the robot should understand that it not only needs put the package inside, but also raise the mailbox flag to indicate that the package is ready for pickup.
            In the catapult domain, the robot needs to understand that pressing the catapult's button will activate the catapult, and that the block needs to be placed on the catapult arm to be launched.

      \item \textbf{Multi-task conditioning.}
            Given the same visual observations but different task description, the robot should perform different and task-relevant actions. The catapult domain has 3 tasks for three target bins, and the drawer domain has 12 tasks.

      \item \textbf{Long horizon behaviour.}
            Our longest horizon domain, mailbox, takes at least 4 subtasks to complete (open the mailbox, put the package in the mailbox while its opened, close the mailbox, then raise the mailbox flag) which can require up to 400 control cycles.
            In the drawer domain, the robot needs to open the drawer, move the object into the drawer, then close it, which takes about 150 control cycles.
            \vspace{-2mm}
\end{itemize}

\begin{table*}[t]
    \small
    \centering
    %\setlength\tabcolsep{0.5pt}
    \begin{tabularx}{\linewidth}{c l >{\centering\arraybackslash}X >{\centering\arraybackslash}X >{\centering\arraybackslash}X >{\centering\arraybackslash}X
    >{\centering\arraybackslash}X >{\centering\arraybackslash}X
    }
    \toprule
    \multirow{2}{*}{\textbf{Supervision}} & 
    \multirow{2}{*}{\bf Method} & 
    \textbf{Set5} & 
    \textbf{Set14} &
    \textbf{BSD100} &
    \textbf{Urban100} &
    \textbf{Manga109} &
    \textbf{DIV2K}\\
    %\cline{4-7}
    & & $\times2$/$\times4$ & $\times2$/$\times4$ & $\times2$/$\times4$ & $\times2$/$\times4$ & $\times2$/$\times4$  & $\times2$/$\times4$ \\
    %\cline{4-7}
    
    \midrule
    %BM3D~\cite{sparse} & {25.65} & {0.685}& {34.51} & {0.850} & {25.65} & {0.685}& {34.51} & {0.850} \\
    %WNNM~\cite{6909762}& {25.78} & {0.809} & {34.67} & {0.864} & {25.65} & {0.685}& {34.51} & {0.850} \\
    %K-SVD~\cite{DBLP:journals/corr/abs-1909-13164} & {26.88} & {0.842}& \textbf{36.49} & \textbf{0.897}& {25.65} & {0.685}& {34.51} & {0.850} \\
    %EPLL~\cite{Hurault_2018} & \textbf{27.11} & \textbf{0.870}  & {33.51} & {0.824}& {25.65} & {0.685}& {34.51} & {0.850} \\
    %\hline
    & {\footnotesize Bicubic} & {33.66/28.42} & {30.24/26.00} & {29.56/25.96} & {26.88/23.14} & {30.80/24.89} & {31.01/26.66} \\
    \midrule
    \multirow{8}{*}{{\footnotesize Supervised}}
    & {\footnotesize VDSR~\cite{kim2016accurate}} & {37.53/31.35} & {33.03/28.01} & {31.90/27.29} & {30.76/25.18} & {37.22/28.83} & {33.66/28.17} \\
    & {\footnotesize EDSR~\cite{lim2017enhanced}} & {38.11/32.46} & {33.92/28.80} & {32.32/27.71} & {32.93/26.64} & {39.10/31.02} & \textbf{36.22}/{30.52} \\
    & {\footnotesize CARN~\cite{ahn2018fast}} & {37.76/32.13} & {33.52/28.60} & {32.09/27.58} & {31.92/26.07} & {38.36/30.47} & \hspace{9pt}{-\hspace{8pt}/30.10}  \\
    & {\footnotesize RCAN~\cite{zhang2018image}} & {38.27/32.63} & {34.12/28.87} & {32.41/27.77} & {33.34/26.82} & {39.44/31.19} & {36.13}/{30.52} \\
    & {\footnotesize RDN~\cite{zhang2018residual}} & {38.24/32.47} & {34.01/28.81} & {32.34/27.72} & {32.89/26.61} & {39.18/31.00} & {-\hspace{8pt}/\hspace{8pt}-} \\
    & {\footnotesize DRN-S~\cite{guo2020closed}} & {37.80/32.68} & {33.30/28.93} & {31.97/27.78} & {31.40/26.84} & {38.11/31.52} & {35.77}/\textbf{30.79} \\
    & {\footnotesize LIIF~\cite{chen2021learning}} & {38.17/32.50} & {33.97/28.80} & {32.32/27.74} & {32.87/26.68} & {-\hspace{8pt}/\hspace{8pt}-} & {34.99/29.27} \\
    & {\footnotesize ELAN~\cite{ELAN-light}} & \textbf{38.36}/\textbf{32.75} & \textbf{34.20}/\textbf{28.96} & \textbf{32.45}/\textbf{27.83} & \textbf{33.44}/\textbf{27.13} & \textbf{39.62}/\textbf{31.68} & {-\hspace{8pt}/\hspace{8pt}-} \\
    %&  \textbf{IMF-SRSR$^{\dagger}$} & {} & {-} & {-} & {-} & {-} & {-} \\
    %&  \textbf{IMF-SRSR$^{\ddagger}$} & {-} & {-} & {-} & {-} & {-} &  {-} \\
    \midrule
    \multirow{4}{*}{\footnotesize Unsupervised} 
    & {\footnotesize SelfExSR~\cite{huang2015single}}  & {36.49/30.31} & {32.22/27.40} & {31.18/26.84} & {29.54/24.82} & {35.78/27.82} & {-\hspace{8pt}/\hspace{8pt}-} \\
    & {\footnotesize ZSSR~\cite{shocher2018zero}}  & {37.37/31.13} & {33.00/28.01} & {31.65/27.12} & {29.34/24.12} & {35.57/27.04} & \textbf{34.45}/\textbf{29.08} \\
    &  {\footnotesize MZSR~\cite{soh2020meta}} &  {37.25/31.59} & {33.16/27.90} & \hspace{-9pt}{31.64/\hspace{8pt}-} & {30.41/25.52} & \textbf{36.70}/\textbf{29.58} & {-\hspace{8pt}/\hspace{8pt}-} \\
    &  {\footnotesize DASR~\cite{wang2021unsupervised}} &  \textbf{37.87}/\textbf{31.99} & \textbf{33.34}/\textbf{28.50} & {\textbf{32.03/27.52}} & \textbf{31.49}/\textbf{25.82} & {-\hspace{8pt}/\hspace{8pt}-} & {-\hspace{8pt}/\hspace{8pt}-} \\
    \midrule
    \multirow{2}{*}{\footnotesize Self-supervised}
    %&  \textbf{IMF-SRSR~(Test)} & {36.41/29.49} & {32.44/27.19} & {31.34/26.82} & {30.26/24.66} & {36.29/27.82} & {35.02/29.45} \\
    &  {\footnotesize \textbf{ICF-SRSR}~(Ours)} & {37.01/30.81} & {32.86/27.76} & {31.54/26.99} & {30.39/24.72} & {36.45/28.01} & {35.19/29.48} \\
    &  {\footnotesize \textbf{EDSR~(LLR,LR)}~(Ours)} & {\textbf{37.09/31.06}} & {\textbf{32.91/27.97}} & {\textbf{31.63/27.10}} & {\textbf{30.51/24.92}} & {\textbf{36.68/28.29}} & {\textbf{35.26/29.64}} \\ 
    %\multirow{3}{*}{(fully self-supervised)}
    \bottomrule
    \end{tabularx}
    \vspace{-2mm}
    \caption{
        \textbf{Quantitative comparisons on synthetic datasets.} 
        %
        We compare ICF-SRSR with several supervised/unsupervised methods on the benchmarks~\cite{bevilacqua2012low, zeyde2010single, martin2001database, huang2015single, Manga109} and DIV2K~\cite{agustsson2017ntire} validation set for scales $\times 2$ and $\times 4$ with PSNR metric. 
        %
        ICF-SRSR refers to our self-supervised method, while EDSR~(LLR,LR) is the model EDSR trained on our generated pairs (LLR,LR) of the DIV2K.
        %
    }
    \label{tab:benchmark}
    \vspace{-2mm}
\end{table*}


The benchmark is built on top of the Mujoco~\cite{todorov2012mujoco} simulator, using assets from the Google Scanned dataset~\cite{downs2022scannedobjects,zakka2022scannedobjectsmujoco}. We use a table-top manipulation set-up with a 6DoF robot arm.
The task success for evaluation are manually designed success conditions, instead of LLM generated function used for data collection.
Details on task design and randomization factors are described in the supplementary material.