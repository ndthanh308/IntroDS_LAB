\textbf{Our Benchmark} contains 18 tasks across 5 domains (Fig.~\ref{fig:tasks_panel} Tab. \ref{tab:benchmark}), with the following properties:


\vspace{-2mm}
\begin{itemize}[leftmargin=3mm]
      \item  \textbf{6DoF \& articulated manipulation}, for deadling with complex object geometry and articulation.
      \item \textbf{Geometry Generalization.} In our bin transport domain, the robot must generalize its bin transport skill to unseen object instances, with novel shapes, sizes, and colors.
      \item \textbf{Intuitive physics.}
            Robots should understand the physical properties of the world and use this knowledge to perform tasks.
            In the bus balance domain, the robot needs to learn the precise grasping and placement to balance a large bus toy on a small block.
            In the catapult domain, where the block is placed along a catapult arm determines how far the block will be launched, and, thus, which bin (if any) the block will land in.

      \item \textbf{Common-sense reasoning \& Tool-use.}
            Natural language task description is user-friendly but often under-specifies the task.
            Common-sense can help to fill in the gaps.
            In the mailbox domain, given the task ``send the package for return'', the robot should understand that it not only needs put the package inside, but also raise the mailbox flag to indicate that the package is ready for pickup.
            In the catapult domain, the robot needs to understand that pressing the catapult's button will activate the catapult, and that the block needs to be placed on the catapult arm to be launched.

      \item \textbf{Multi-task conditioning.}
            Given the same visual observations but different task description, the robot should perform different and task-relevant actions. The catapult domain has 3 tasks for three target bins, and the drawer domain has 12 tasks.

      \item \textbf{Long horizon behaviour.}
            Our longest horizon domain, mailbox, takes at least 4 subtasks to complete (open the mailbox, put the package in the mailbox while its opened, close the mailbox, then raise the mailbox flag) which can require up to 800 control cycles.
            In the drawer domain, the robot needs to open the drawer, move the object into the drawer, then close it, which takes about 300 control cycles.
            \vspace{-2mm}
\end{itemize}

The benchmark is built on top of the MuJoCo~\cite{todorov2012mujoco} simulator, using assets from the Google Scanned dataset~\cite{downs2022scannedobjects,zakka2022scannedobjectsmujoco}. We use a table-top manipulation set-up with a 6DoF robot arm.
The task success in evaluation is a manually designed function, instead of LLM generated function used for data collection.