%%All LINK
% OVERLEAF FILE SHARING: https://www.overleaf.com/5524318945ffjbmzycxmvr

% Github link for all python coding: 
% https://github.com/TurinShayla/Python-Graph-Brain

%Drive Link
%https://drive.google.com/drive/folders/1Yad1VQFrOwFumw8zH0-kiSzqmbPuw8DD?usp=drive_link

% Doc link
%https://docs.google.com/document/d/1UONoYIbZrLpldYcS6WOtBjKoV8iQImrQVi5uhYbrI_w/edit?usp=sharing

% Lucid chart link for drawing and editing
% https://lucid.app/lucidchart/d16237b4-6907-4775-8cbd-5cbc4022848d/edit?viewport_loc=-1644%2C-630%2C5429%2C2811%2CFMfCGzQ4mc1q&invitationId=inv_92e19eb1-5d24-4b75-a9f9-f8d09efc84ed
%% This is file `sample-manuscript.tex',
%% generated with the docstrip utility.
%%
%% The original source files were:
%%
%% samples.dtx  (with options: `manuscript')
%% 
%% IMPORTANT NOTICE:
%% 
%% For the copyright see the source file.
%% 
%% Any modified versions of this file must be renamed
%% with new filenames distinct from sample-manuscript.tex.
%% 
%% For distribution of the original source see the terms
%% for copying and modification in the file samples.dtx.
%% 
%% This generated file may be distributed as long as the
%% original source files, as listed above, are part of the
%% same distribution. (The sources need not necessarily be
%% in the same archive or directory.)
%%
%% Commands for TeXCount
%TC:macro \cite [option:text,text]
%TC:macro \citep [option:text,text]
%TC:macro \citet [option:text,text]
%TC:envir table 0 1
%TC:envir table* 0 1
%TC:envir tabular [ignore] word
%TC:envir displaymath 0 word
%TC:envir math 0 word
%TC:envir comment 0 0
%%
%%
%% The first command in your LaTeX source must be the \documentclass command.
%%%% Small single column format, used for CIE, CSUR, DTRAP, JACM, JDIQ, JEA, JERIC, JETC, PACMCGIT, TAAS, TACCESS, TACO, TALG, TALLIP (formerly TALIP), TCPS, TDSCI, TEAC, TECS, TELO, THRI, TIIS, TIOT, TISSEC, TIST, TKDD, TMIS, TOCE, TOCHI, TOCL, TOCS, TOCT, TODAES, TODS, TOIS, TOIT, TOMACS, TOMM (formerly TOMCCAP), TOMPECS, TOMS, TOPC, TOPLAS, TOPS, TOS, TOSEM, TOSN, TQC, TRETS, TSAS, TSC, TSLP, TWEB.
% \documentclass[acmsmall]{acmart}

%%%% Large single column format, used for IMWUT, JOCCH, PACMPL, POMACS, TAP, PACMHCI
% \documentclass[acmlarge,screen]{acmart}

%%%% Large double column format, used for TOG
% \documentclass[acmtog, authorversion]{acmart}

%%%% Generic manuscript mode, required for submission
%%%% and peer review
\documentclass[manuscript,screen]{acmart}

%\documentclass[manuscript,review,anonymous]{acmart}
%% Fonts used in the template cannot be substituted; margin 
%% adjustments are not allowed.
%%
%% \BibTeX command to typeset BibTeX logo in the docs
\AtBeginDocument{%
  \providecommand\BibTeX{{%
    \normalfont B\kern-0.5em{\scshape i\kern-0.25em b}\kern-0.8em\TeX}}}

%% Rights management information.  This information is sent to you
%% when you complete the rights form.  These commands have SAMPLE
%% values in them; it is your responsibility as an author to replace
%% the commands and values with those provided to you when you
%% complete the rights form.
\setcopyright{acmcopyright}
\copyrightyear{2018}
\acmYear{2018}
\acmDOI{XXXXXXX.XXXXXXX}

%% These commands are for a PROCEEDINGS abstract or paper.
% \acmConference[Conference acronym 'XX]{Make sure to enter the correct
%   conference title from your rights confirmation emai}{June 03--05,
%   2018}{Woodstock, NY}
%
%  Uncomment \acmBooktitle if th title of the proceedings is different
%  from ``Proceedings of ...''!
%
\acmBooktitle{Woodstock '18: ACM Symposium on Neural Gaze Detection,
 June 03--05, 2018, Woodstock, NY} 
\acmPrice{15.00}
\acmISBN{978-1-4503-XXXX-X/18/06}
\usepackage{fancyhdr}
\pagestyle{fancy}
\markboth{Game vs Video}{Shayla}
%\acmHeader{Short Paper Title}
\usepackage{subcaption}
% Please add the following required packages to your document preamble:
\usepackage{multirow}
% Please add the following required packages to your document preamble:
\usepackage{multirow}
%%
%% Submission ID.
%% Use this when submitting an article to a sponsored event. You'll
%% receive a unique submission ID from the organizers
%% of the event, and this ID should be used as the parameter to this command.
%%\acmSubmissionID{123-A56-BU3}

%%
%% For managing citations, it is recommended to use bibliography
%% files in BibTeX format.
%%
%% You can then either use BibTeX with the ACM-Reference-Format style,
%% or BibLaTeX with the acmnumeric or acmauthoryear sytles, that include
%% support for advanced citation of software artefact from the
%% biblatex-software package, also separately available on CTAN.
%%
%% Look at the sample-*-biblatex.tex files for templates showcasing
%% the biblatex styles.
%%

%%
%% The majority of ACM publications use numbered citations and
%% references.  The command \citestyle{authoryear} switches to the
%% "author year" style.
%%
%% If you are preparing content for an event
%% sponsored by ACM SIGGRAPH, you must use the "author year" style of
%% citations and references.
%% Uncommenting
%% the next command will enable that style.
%%\citestyle{acmauthoryear}

%%
%% end of the preamble, start of the body of the document source.
\begin{document}

%%
%% The "title" command has an optional parameter,
%% allowing the author to define a "short title" to be used in page headers.
%\title{Comparing the Effectiveness of Game-Based and Video-Based Learning: Results of a Pilot Study from Learning Outcomes and Cognitive Engagement}
%\title{Comparing the Effectiveness of Game-Based and Video-Based Learning: Insight from Learning Outcomes and Cognitive Engagement}

\title{Insights into Cognitive Engagement: Comparing the Effectiveness of Game-Based and Video-Based Learning}

%%
%% The "author" command and its associated commands are used to define
%% the authors and their affiliations.
%% Of note is the shared affiliation of the first two authors, and the
%% "authornote" and "authornotemark" commands
%% used to denote shared contribution to the research.

\author{Shayla Sharmin}
\authornotemark[1]
\email{shayla@udel.edu}
\orcid{0000-0001-5137-1301}

\affiliation{%
  \institution{University of Delaware}
  \streetaddress{South College Avenue}
  \city{Newark}
  \state{Delaware}
  \country{USA}
  \postcode{19716}
}

\author{Arpan Bhattacharjee}
\affiliation{%
  \institution{University of Delaware}
    \streetaddress{South College Avenue}
  \city{Newark}
  \state{Delaware}
  \country{USA}
  \postcode{19716}
  }
\email{arpan@udel.edu}
\orcid{0009-0003-7965-3820}
\author{Rifat Sadik}
\affiliation{%
  \institution{University of Delaware}
    \streetaddress{South College Avenue}
  \city{Newark}
  \state{Delaware}
  \country{USA}
  \postcode{19716}
  }
\email{rifat@udel.edu}
\orcid{0000-0002-0068-1817}

\author{Priyanka Raju Patre}
\email{parte@udel.edu}
\orcid{0009-0002-1625-0672}
\affiliation{%
  \institution{University of Delaware}
   \streetaddress{South College Avenue}
  \city{Newark}
  \state{Delaware}
  \country{USA}
  \postcode{19716}
}

\author{Reza Koiler}
\affiliation{%
   \institution{University of Delaware}
   \streetaddress{}
  \city{Newark}
  \state{Delaware}
  \country{USA}
  \postcode{19716}}
\email{radis@udel.edu}
\orcid{}

\author{Nancy Getchell}
\affiliation{%
   \institution{University of Delaware}
   \streetaddress{}
  \city{Newark}
  \state{Delaware}
  \country{USA}
  \postcode{19716}}
\email{getchell@udel.edu}
\orcid{}
\author{Roghayeh Leila Barmaki} 
\authornotemark[1]
\affiliation{%
   \institution{University of Delaware}
   \streetaddress{South College Avenue}
  \city{Newark}
  \state{Delaware}
  \country{USA}
  \postcode{19716}}
\email{rlb@udel.edu}
\orcid{0000-0002-7570-5270}

%%
%% By default, the full list of authors will be used in the page
%% headers. Often, this list is too long, and will overlap
%% other information printed in the page headers. This command allows
%% the author to define a more concise list
%% of authors' names for this purpose.
\renewcommand{\shortauthors}{Sharmin, et al.}

%%
%% The abstract is a short summary of the work to be presented in the
%% article.

\begin{abstract}
The analysis of brain signals holds considerable importance in enhancing our comprehension of diverse learning techniques and cognitive mechanisms. Game-based learning is increasingly being recognized for its interactive and engaging educational approach. A pilot study of twelve participants divided into experimental and control groups was conducted to understand its effects on cognitive processes. Both groups were provided with the same contents regarding the basic structure of the graph. The participants in the experimental group engaged in a quiz-based game, while those in the control group watched a pre-recorded video. Functional Near-Infrared Spectroscopy (fNIRS) was employed to acquire cerebral signals, and a series of pre and post-tests were administered. The findings of our study indicate that the group engaged in the game activity displayed elevated levels of oxygenated hemoglobin ($\Delta HbO$) compared to the group involved in watching videos. Conversely, the deoxygenated hemoglobin levels ($\Delta HbR$) remained relatively consistent across both groups throughout the learning process. The aforementioned findings suggest that the use of game-based learning has a substantial influence on cognitive processes.
Furthermore, it is evident that both the game and video groups exhibited higher neural activity in the Lateral Prefrontal cortex (PFC). The $\Delta HbO$ ratio demonstrates that the game group had 2.33 times more neural processing in the Lateral PFC than the video group. This data is further supported by the knowledge gain analysis, which indicates that the game-based approach resulted in a 47.74\% higher knowledge gain than the video group, as calculated from the difference in pre-and post-test scores.
\end{abstract}

%%
%% The code below is generated by the tool at http://dl.acm.org/ccs.cfm.
%% Please copy and paste the code instead of the example below.
%%
\begin{CCSXML}
<ccs2012>
<concept>
<concept_id>10003120</concept_id>
<concept_desc>Human-centered computing</concept_desc>
<concept_significance>500</concept_significance>
</concept>
<concept>
<concept_id>10003120.10003123.10010860.10010858</concept_id>
<concept_desc>Human-centered computing~User interface design</concept_desc>
<concept_significance>500</concept_significance>
</concept>
</ccs2012>
\end{CCSXML}

\ccsdesc[500]{Human-centered computing}
\ccsdesc[500]{Human-centered computing~User interface design}
%%
%% Keywords. The author(s) should pick words that accurately describe
%% the work being presented. Separate the keywords with commas.
\keywords{Brain activity, fNIRS, Multimodal learning, Active learning, Game-based learning, Feasibility, Usability}

%% A "teaser" image appears between the author and affiliation
%% information and the body of the document, and typically spans the
%% page.
\begin{teaserfigure}
 % % Figure removed
% Figure removed
 \caption{Experimental setup for functional near-infrared spectroscopy (fNIRS) data collection and analysis during a learning task. The participant wears an fNIRS headband while engaging with learning materials presented on Laptop 1 (L1). Laptop 2 (L2) is connected to the fNIRS device and runs Cobi software for data collection. Laptop 3 (L3) sends triggers to mark separate task blocks. The fNIRSoft software on L2 is used for analyzing the collected fNIRS data.}
  \label{fig:HwSw}
  %\caption{Seattle Mariners at Spring Training, 2010.}
  \Description{}
  %\label{fig:teaser}
\end{teaserfigure}


% \received{20 February 2007}
% \received[revised]{12 March 2009}
% \received[accepted]{5 June 2009}

%%
%% This command processes the author and affiliation and title
%% information and builds the first part of the formatted document.
\maketitle

\section{Introduction} \label{Introduction}
With the advancement of technology, education and learning outcomes have been improved significantly \cite{shu2023empirical,ronghuai2014three}. Today’s students spend considerable time engaging with technologies such as mobile, laptops, and tabs to watch videos and play games for recreation and education purposes \cite{haleem2022understanding,panjeti2023impact}. So, researchers and educators are trying to find ways to make video and game content more interactive and engaging. It has been found that motivation and engagement in learning are key factors that influence academic achievement
 \cite{collie2019motivation}. Student learning outcomes and performance are reported to be higher among motivated students \cite{chattopadhyay2021motivation}. Educators are trying to keep students motivated and engaged by using different learning techniques, such as
Video-based learning (VBL)  and game-based learning (GBL). VBL allows students to watch and learn from recorded video lectures at their feasible time. On the other hand, GBL is considered active learning, which can be digital (video games) or non-digital (board games) \cite{gordillo2022comparing}.



% Teachers are trying to keep students motivated and engaged in learning by implementing various learning techniques. 
% Such as\textbf{Video-based learning} (VBL) allows people to watch and learn from recorded video lectures according to their feasible time.


Analyzing brain activity makes it possible to identify the key facts about how the brain responds to various learning techniques that keep students cognitively engaged \cite{materna2007jump}. By measuring brain activity, researchers and educators can gain insights into how the brain processes information, what parts of the brain are active during different learning tasks, and how different learning strategies affect brain function. However, it's important to note that brain signal analysis is still an emerging research topic. More research is needed to fully understand how best to use these techniques for educational purposes. 


For brain signal analysis, research on the prefrontal cortex shows progress because it involves many higher cognitive functions, such as decision-making, working memory, and attention \cite{kober2020game,doherty2023interdisciplinary}. Functional Near-Infrared Spectroscopy (fNIRS) can be used to determine neuron activity in the brain from the prefrontal cortex \cite{skau2021exhaustion}. It is a non-invasive optical neuroimaging technique used to collect the oxygenation and deoxygenation changes in the brain's blood while doing a task from the human cerebral cortex to investigate cognitive load \cite{ninaus2014neurophysiological}. The fNIRS calculate $\Delta HbO$ that refers to changes in oxygenated hemoglobin (HbO) concentration and $\Delta HbR$ deoxygenated hemoglobin (HbR) in the prefrontal cortex \cite{doherty2023interdisciplinary,li2023current}.

%WRITE RESEARCH GAP

When the brain actively participates in any task, the active brain area causes an inflow of oxygenated blood \cite{koiler2022impact}. The active neuron consumes oxygen when high neural activity in a brain area initially increases $\Delta HbR$ \cite{shealy2023changes,koiler2022impact,cakir2015optical}. When the demand for oxygen rises, oxygen flows to that brain area, increasing  $\Delta HbO$ levels \cite{koiler2022impact,cakir2015optical}. 
 Higher $\Delta HbO$ and $\Delta HbR$ indicate more blood flow to the brain and thus stronger neuron activities while doing a task \cite{cakir2015optical,ccakir2016behavioral,koiler2022impact}.
%Research gap
However, most of the studies investigate the neural activity for a specific game \cite{cakir2015optical,ccakir2016behavioral,kober2020game,greipl2021brain,samah2018using} or video \cite{} interface. To our knowledge, no study investigates both aspects of learning to support the effectiveness of these learning techniques. The comparative analysis becomes more valid if we can study the neural activity for video and game-based interfaces for a similar topic. So, analyzing and comparing the brain signals of learners playing an educational game and watching videos will be important in cognitive neuroscience. It will also help educators to design study materials and make the learning process more interactive.

%However most of the recent works focused on 
%game base and video based learning engagement dtermination
%neural activity

This work investigates how engagement and knowledge gain change while learning computer science topics such as the basic structure of graphs using a game and video with similar contents. Using fNIRS from the prefrontal cortex, engagement can be determined by measuring the change of $\Delta HbO$ and $\Delta HbR$  to the brain. We observed the knowledge gain and compared game-based and video-based learning by measuring the difference between pre and post-test scores. Our research questions include:

\begin{itemize}
    \item \textbf{RQ1:} What are the differences in neural activities between game-based and video-based learning?
    \item \textbf{RQ2:} What are the differences regarding the subjective results on usability, task load, and knowledge gain between game-based  and video-based learning modules? 
\end{itemize}

We developed a quiz-based learning game and prepared a video with the same content to answer the above research questions. Then for a pilot study, we formed a group of participants and divided them into an experimental group to play the game and a control group to watch the video content. We then evaluated both game-based and video-based learning in a small study by analyzing both groups' brain signals. We used a fNIRS device to collect the brain signal of the participants during the intervention to analyze the neural activities. A pre and post-test was conducted to measure the knowledge gained. 
We used standardized questionnaires: System Usability Scale (SUS)  \cite{brooke1996sus} and task load (NASA-TLX) \cite{hart1988development} for subjective analysis.

The contributions of this work are summarized below-
\begin{itemize}
    \item Evaluation on the effectiveness of game-based over video-based learning by measuring the neural activity.
    \item Development of an interactive learning interface using video and game content.
    
    \item  Analysis of the result of statistical difference in usability, task load, and knowledge gain in GBL and VBL.
    % \item To measure attention in GBL
\end{itemize}


The remainder of this paper is organized as follows. Section \ref{Related Works} presents related works on the game and video-based learning and analysis of brain signals. Section \ref{Materials and Method} describes the proposed methodology and study design. Section \ref{Result} discusses the results of our experiments.  Finally, Section \ref{Conclusion} concludes the paper by highlighting the proposed system's limitations and the scope for future work.



\section{Related Works}\label{Related Works}

%%%%%   WRITE in SECTIONS

%%%%% TODO
%%% game vs video
%%% 1%%%%
Analyzing brain signals is of utmost importance as it enables researchers to examine the neural underpinnings of cognitive processes and the mental workload of the human cortex \cite{essa2021brain,shealy2023changes}. Functional Near Infrared Spectroscopy (fNIRS) \cite{ayaz2012optical},  electroencephalogram (EEG) \cite{samah2018using}, and functional Magnetic Resonance Imaging (fMRI) \cite{greipl2021brain} are scientific methodologies that offer significant contributions to the understanding of brain activities. These techniques are particularly useful in evaluating attention, engagement, and cognitive load. Functional near-infrared spectroscopy (fNIRS), as a brain imaging technique that is comparatively more cost-effective, lightweight, and portable, plays a significant role in facilitating the analysis of brain signals flexibly \cite{doherty2023interdisciplinary, koiler2022impact,wei2023reduced,wang2023interaction}.

 Functional near-infrared spectroscopy (fNIRS) is a highly adaptable and non-invasive modality used to quantify cerebral activity, exhibiting wide-ranging utility across various domains \cite{doherty2023interdisciplinary}. This technique is widespread across various disciplines, including neuroscience in real life \cite{doherty2023interdisciplinary}, cognitive psychology \cite{arenth2007applications}, clinical and medical research \cite{wei2023reduced,green2004approaching,ye2023hotspots}, rehabilitation and motor control\cite{wang2023interaction,zhang2023motor,zheng2023cognitive}, human-computer interaction \cite{kosch2023survey,solovey2009using,ayaz2012optical}, education and learning \cite{shi2023improving,desoto2023utilization,zhou2023infant}, as well as sports performance analysis \cite{gao2023brief,carius2023increased}. 

 By studying brain signals, educators can design learning materials that align with cognitive processes, optimize learning experiences, and improve retention and understanding \cite{chang2021neuroscience}. Brain signal analysis techniques  fNIRS and EEg have been used in education like math \cite{skau2022proactive,cakir2015optical,poikonen2023nonlinear}, geometry\cite{shi2023improving}, engineering\cite{shealy2023changes,grohs2017evaluating}, and science \cite{naimi2010investigating} to investigate neural activity and cognitive processes. In recent years, there has been a rise in the popularity of game and video-based learning as effective methods for evaluating various learning approaches. In game-based learning (GBL) and video-based learning (VBL), experts have used brain signal analysis to learn more about the cognitive processes and neural activity behind these teaching methods.

 \subsection{Learning and Performance: Cognitive and Neural Factors}
Proactive control, hands-on learning, thinking aloud, and mathematical performance are reviewed in this section. Besides, cognitive processes, brain activation, and mathematical proficiency are discussed.

Suko et al., \cite{skau2022proactive} examined mathematical cognition, general cognition, and brain bases in 8- to 9-year-olds. The study found that proactive control, which means the cognitive ability to anticipate and prepare for potential challenges, correlates more strongly with mathematical performance than other cognitive abilities using additive mathematics tests, cognitive assessments, and fNIRS brain imaging.
%%
Shi et al. \cite{shi2023improving} conducted a study where hands-on geometry education was tested on 40 Chinese middle school pupils with various academic abilities. Compared to video teaching, hands-on experience increased the oxygenated hemoglobin concentration measured by fNIRS, indicating higher neuron activation. This study found that hands-on learning improved students' geometry comprehension and engagement.
%%%%
Tripp et al., \cite{shealy2023changes} measured changes in oxygenated hemoglobin  to observe the neurocognitive changes in designers when thinking aloud while designing using fNIRS. The findings of their study demonstrate that the act of expressing thoughts, commonly referred to as thinking aloud, has a significant impact on both the cognitive processes involved in design and the neurological processes underlying cognition. 
%%%
Artemenko et al., \cite{artemenko2019individual} investigated the relationship between mathematical proficiency and the level of complexity involved in multiplication and division operations. A correlation was observed between lower mathematical proficiency and decreased calculation durations, specifically for complex arithmetic problems. Individuals with lower math skills also exhibited diminished neural activity in the left supramarginal, superior temporal, and inferior frontal gyri. They used both EEG and fNIRS to observe neural activity. 

These findings illuminate learning and performance neurocognition. fNIRS indicates that hands-on geometry education is more engaging than video teaching. Designing aloud influences cognitive and neurological processes, according to fNIRS. EEG and fNIRS studies demonstrate that lesser mathematical proficiency is associated with longer calculation times and decreased neural activity in arithmetic processing regions. 
%%%%%%%%%%%%%%% related work brain signal on game %%%%%%%%%%%%%%%%%%%%%

\subsection{Brain Signal Analysis in Game and Video Based Learning: Enhancing Neural Activity and Engagement}
There have been some studies related to brain signal analysis while playing games and watching videos.

Desoto et al., \cite{desoto2023utilization} observed the concept of neuronal alignment in the context of education and the exchange of knowledge in their study. The finding demonstrates that varying inputs cause unique neural reactions and that identifiable trends in processing identical information across different brains may exist. While viewing STEM educational videos, neural activity is monitored using EEG and fNIRS techniques to understand the cognitive processes involved in processing identical information by the human brain. 

Tang et al. \cite{tang2023mind} used EEG and machine learning algorithms to identify instances of 'mind wandering' during video-based learning experiences. The results showed an average area under the receiver operating characteristic curve (AUC) of 0.876 for classifying mind wandering within individual participants and 0.703 for classifying mind wandering across different educational lectures. The obtained high levels of detection accuracy suggest that the implementation of this method has the potential to improve the outcomes of distance learning greatly.
A study by  Cakir  et al. \cite{cakir2015optical} aimed to assess the efficacy of game-based learning in enhancing math fluency with 27 college students participated in this study where the game group played "MathDash," and the control group faced a drill and practice approach to evaluate the behavioral and neural effects. For brain signal collection, the fNIRS device was used to get the oxyhemoglobin $\Delta HbO$. The $\Delta HbO$ was equal in both control groups. Still, during the post-test, the game group had higher $\Delta HbR$ concentration, which means their game training optimizes their brain metabolism. 

%%%%% 2%%%

Samah et al.\cite{samah2018using} examined brain functional connectivity during game-based problem-solving tasks. The primary focus is to draw gender differences based on brain signals while playing the Tower of Hanoi. For this purpose, they chose a computer-based Tower of Hanoi (ToH) game. Electroencephalogram (EEG) signals were employed in this study's experimental study to record participant performance, and partial directed coherence (PDC) analysis was performed to analyze the data. PDC is a statistical method used to analyze the directional interactions between different time series data, such as EEG signals. PDC analysis illustrates the interactions between time direction and spectral properties of a signal in the brain. According to the study, male and female respondents exhibited no appreciable differences in brain activity patterns.

%%%%%% 3 %%%%
 Using near-infrared spectroscopy, Kober et al.\cite{kober2020game} investigated behavioral performance while learning by playing games on a neurofunctional level. They used a NIRS device to calculate the oxyhemoglobin and deoxyhemoglobin concentration of the frontal brain from 59 healthy adults and interest\cite{kober2020game}. They found that the game-based version was more engaging than the non-game version by observing stronger activation on the prefrontal cortex\cite{kober2020game}.  They also used the Flow Short Scale (FKS), the User Experience Questionnaire (UEQ), and the Positive and Negative Affect Schedule (PANAS). The participants' subjective ratings also indicate game version was more rewarding and engaging.
%%%%% 4 %%%
In another work, Greipl et al. \cite{greipl2021brain} used fMRI and MRI to compare the effects of game-based and non-game-based learning on the brain. Forty two participants played a number line estimation task while their brain activity was measured. The results showed that game-based learning led to stronger activation in ventral tegmental (VTA) \& the substantia nigra (SNr) region areas associated with reward and in  the amygdala (AMY) \& anterior insula (aINS), indicating emotional processing, which refers to that game-based learning may enhance learning through rewards. The subjective analysis showed that the game-based version was rated more attractive, novel, and stimulating than the non-game-based version. However, there were no significant differences in the number of correct answers or time taken between the game-based and non-game-based versions of the task.

The results suggest that the utilization of game-based learning has the potential to result in increased neural activity and increased levels of engagement. These studies have demonstrated enhanced activation in distinct cerebral areas linked to rewards, emotional processing, and cognitive involvement.  In general, the aforementioned findings emphasize the favorable influence of game-based learning on cerebral activity and involvement.
\subsection{Effectiveness of Game-Based Learning Approaches in Various Educational Contexts}
Hsu and Lin \cite{hsu2016impact} compared using a Web Digital Game-Based Learning System in the experimental group with an online video-based learning system in the control group on computer game programming education. The experimental group exhibited superior learning performance and motivation compared to the control group. A separate investigation conducted by Mohsen \cite{ali2016use} explored the effects of virtual surgical simulation on educational achievements. The findings revealed that individuals in the experimental group exhibited better results in language comprehension and vocabulary recognition assessments than those in the control group. In their study, Chen et al.\cite{chen2021learning} conducted a comparative analysis of various degrees of technological engagement in the context of learning. Their findings indicated that using a simulation video game resulted in more substantial advancements in learning outcomes compared to both video-based instruction and traditional instructional methods. In their study, Gordillo et al. \cite{gordillo2022comparing} studied the efficacy of game-based learning in software engineering. The researchers discovered that video games developed by teachers exhibited greater knowledge acquisition and motivation when compared to video-based learning approaches. 
These studies support the efficacy of game-based and simulation-based learning approaches.
These approaches have demonstrated enhanced learning performance, increased motivation, enhanced language comprehension, and improved knowledge acquisition compared to conventional instructional methods and video-based learning approaches.

There is growing literature comparing game-based and video-based learning methods, but brain signal analysis to examine neural pathways is lacking. This emphasizes the need for such evaluations to evaluate various approaches. Brain signals can reveal cognitive involvement, attention, and neural activation. This can help explain the cognitive mechanisms behind game and video-based learning. Brain signal analysis helps researchers identify neural correlates of learning outcomes and motivation, expanding knowledge of how different instructional approaches affect the brain and cognitive processes.



\section{Materials and Method} \label{Materials and Method}

This study intends to compare game-based learning methods and video-based learning to analyze the differences in engagement levels by measuring neural activity and knowledge gained from pre and post-test score differences while learning STEM and computer science subjects, specifically focusing on topics related to 'graph theory.' To achieve this goal, we collected data for the oxygenation and deoxygenation flow within the prefrontal cortex of the participants using functional near-infrared spectroscopy (fNIRS). Pre-tests and post-tests were taken to assess any potential improvement in test scores before and after implementing the learning methods. Furthermore, a survey was conducted to evaluate the usability and task load of the study. To demonstrate the comparison, we divided our participants into two groups: the experimental group, who played a quiz-based game, and the control group, who viewed a recorded video on the same topic.


\subsection{Participants}\label{label:participants}

During the process of participant recruitment, a pre-screening phase was conducted. 
The participants were engaged in face-to-face interactions and were chosen for inclusion in the study according to the following criteria: being adults aged 18 years or older, possessing good health, having proficiency in the English language, displaying no sensitivity to alcohol rub, and having limited or no familiarity with fundamental graph terminologies. Individuals who did not meet the aforementioned criteria were excluded during this stage. Subsequently, participants were provided with a concise explanation regarding the terms and conditions of the subsequent experiment as well as the procedural aspects of the experiment. Furthermore, the participants willingly provided informed consent by signing a formal document. Demographic information, including age, courses, year of study, gender, and level of prior knowledge of graph theorem, was collected.

Twelve  participated in this experiment, and their demographic information is listed in table \ref{tab:participants}. No participants were excluded. In the game group (experimental group), there were
six students (1 F), and also, in the video group (control group), there were six students (3 F). 
\begin{table}[t]
%\centering
\caption{Participant background and characteristics ($n=12$).}
\label{tab:participants}
% \scriptsize
%\resizebox{\columnwidth}{!}{
\begin{tabular}
{p{0.35\columnwidth}p{0.15\columnwidth}p{0.15\columnwidth}} 
\hline
Characteristics  & Value  & Mean \\
\hline
Age   & [24 -- 33] & 27.83\,$\pm$\,2.99 \\
Gender                    &    &    \\
\hspace*{0.3cm} Male      & 8 & (66.67\%) \\
\hspace*{0.3cm} Female    & 4 & (33.33\%) \\
Education                &   &      \\

\hspace*{0.3cm} Graduate program & 12 & (100\%) \\

Graph terminology knowledge           &    &     \\
\hspace*{0.3cm} No prior knowledge   & 9 & (75\%) \\
\hspace*{0.3cm} Remember few topics & 1 & (8.33\%) \\
\hspace*{0.3cm} Don't remember the topics & 2 & (16.67\%) \\
Use video as learning material          &    &     \\
\hspace*{0.3cm} Never & 4 & (33.33\%) \\
\hspace*{0.3cm} Several times in a month   & 3 & (25.00\%) \\
\hspace*{0.3cm} Several times in a week & 5 & (41.67\%) \\
\hspace*{0.3cm} Daily & 4 & (33.33\%) \\
Use game as learning material          &    &     \\
\hspace*{0.3cm} Never & 10 & (83.33\%) \\
\hspace*{0.3cm} Several times in a week   & 1 & (8.33\%) \\
\hspace*{0.3cm} Several times in a year & 1 & (8.33\%) \\



\hline
\end{tabular}
%}
\end{table}
\subsection{Apparatus}

For this study, we used some hardware and software. Figure \ref{fig:HwSw} shows the hardware and setups. We used three computers in our experiment. An Alienware laptop (L1) has been used to play the game and watch the video. We used a desktop (L2) [Intel(R) Core(TM) i7-10700T CPU @ 2.00 GHz] that is connected to an fNIRS device. L2 also has Cognitive Optical Brain Imaging (COBI) Studio Software, which collects fNIRS signals, and fNIRSoft Software (Version 4.9), which analyzes the brain signal data by representing the mean activation during all blocks for each condition. Another laptop (L3) [Intel(R) Core(TM) i5 CPU M 460 @ 2.53 GHz] has a serial port that is connected to L2. A custom PsychoPy code is used from L3 for stimulus presentation and triggering the fNIRS device. PsychoPy sends markers to each block, such as definition, rest, quiz, and feedback to L2, that help us to identify the different blocks of brain signals. 
For data analysis, we used Python. Also, for demographic data collection, pre and post-test, and user feedback questionnaires, we used Qualtrix XM.



In this study, a combination of hardware and software was employed. The hardware and setups are depicted in Figure \ref{fig:HwSw}. Three computers were utilized in the process of our experiment. An Alienware laptop (L1) has been used to run learning materials (both game and video).In this study, a desktop computer (L2) equipped with an Intel(R) Core(TM) i7-10700T CPU operating at a frequency of 2.00 GHz was utilized. The desktop computer was connected to a fNIRS device for data acquisition. L2 is equipped with two software programs, namely Cognitive Optical Brain Imaging (COBI) Studio Software (Cobi) and fNIRSoft Software (Version 4.9). COBI Studio Software collects functional near-infrared spectroscopy (fNIRS) signals, while fNIRSoft Software analyzes the brain signal data by representing the average activation across all blocks for each condition. Another laptop, referred to as L3, features an Intel(R) Core(TM) i5 CPU M 460 @ 2.53 GHz and is equipped with a serial port linked to L2. A custom PsychoPy \cite{peirce2019psychopy2} code is used from L3 for stimulus presentation and triggering the fNIRS device. PsychoPy sends markers to each block, such as definition, rest, quiz, and feedback to L2, that help us identify the different blocks of brain signals.
Python was utilized for the purpose of data analysis. In our study, we employed Qualtrix XM to gather demographic data, conduct pre-and post-tests, and user feedback questionnaires.


\subsection{Procedure}
The experiment has been divided into multiple phases, as depicted in Figure \ref{fig:study_design}.
% Figure environment removed
Initially, we asked participants for pre-screening questionnaires based on the inclusion and exclusion criteria outlined in section \ref{label:participants}. Afterward, the participants attended a pre-test consisting of ten questions as outlined in section \ref{pretest}.  The participants were not provided any answers or feedback after the pre-test.  The subsequent stage involves implementing the instructional approach outlined in section \ref{label_learningMethod}. There exist two distinct groups. The experimental group played a game we developed using unity, and the control group watched the video to learn the basic terminologies of graphs we recorded on a similar topic. They wear the fNIRS headband at the beginning of the learning period and remove it upon session completion.
After the designated learning period, the participants participate in the post-test, thoroughly explained in section \ref{label_posttest}. The participants will not possess prior knowledge regarding similar or dissimilar questions in the post-test compared to the pre-test. Next, the participants will be given survey questions about their learning experiences.

\subsection{Design}
 The allocation of participants into the experimental (game) and control (video) groups was conducted randomly. Participants are randomly selected to engage in gameplay or watch the video content. The participants in the experimental group engaged in gameplay facilitated by the Unity game development platform.
The control group watched a pre-recorded video presentation created using Microsoft PowerPoint software, accompanied by audio. To maintain coherence, we employed comparable materials for both the definition and quiz inquiries in the interactive game and the accompanying video.


\subsubsection{Pre-test} \label{pretest}
\hfill\\
During the initial assessment phase, a Qualtrics form was generated, consisting of ten multiple-choice questions with five available options for each question, each carrying the weight of one point. The Qualtrics platform itself recorded the test duration.
Following the completion of the test, no feedback was provided to the participants. The test assessed the participants' understanding of the subject matter. The questions were randomized to ensure a non-biased and impartial distribution.

\subsubsection{Learning Methods}\label{label_learningMethod}
\hfill\\
This study uses two instructional materials, a game, and a video, to learn basic graph terminologies. The selected topics include the definition of a graph, the complete graph, the determination of the number of edges in a complete graph, the concept of a loop, the distinction between directed and undirected graphs, the degree of a graph, and the final concept of in-degree and out-degree in a graph. The illustration presented in Figure \ref{fig:game} provides a visual representation of the game, showcasing the components of the definition, quiz, and feedback interface.



% Figure environment removed

\subsubsection*{Game-based Learning Interface:} \hfill\\
An interactive game was developed using the Unity platform. At present, this game is exclusively designed for the Windows operating system. It is anticipated that participants will engage in the process of reading the definitions of fundamental terms depicted in the graph. Players follow the instructions to engage with the game, employing actions such as playing with their graph through drag-and-drop functionality to incorporate various elements, including houses and roads. After each definition, participants are subsequently presented with inquiries based on the corresponding term.
The game comprises a total of seven trials, with each trial having a duration of 90 seconds. Each trial has different parts: a definition period lasting 30 seconds and a rest period of 10 seconds. Subsequently, there is a quiz/drag-and-drop period of 10 seconds, followed by a feedback period of 10 seconds. Finally, another rest period concludes each trial. The game lasts approximately 10.5 minutes (see Fig.\ref{fig:psycho_py}). The quiz has game elements such as a timer, sound effects, confetti, clicking, and dragging and dropping options. 
 
 
% Figure environment removed

\subsubsection*{Video-based Learning Interface:}\hfill\\
The instructional video lasts 10.5 minutes and provides a comprehensive overview of graph theory concepts similar to those covered in the game. The initial step involved creating a PowerPoint slide dedicated to each individual topic.
Afterward, animation and audio were incorporated into each individual slide. The individual videos are merged to create a single video. The experiment consists of seven trials, each lasting 90 seconds. Within each trial, there are specific components, including definitions (30 seconds), rest periods (10 seconds), quizzes (30 seconds), feedback (10 seconds), and additional rest periods (10 seconds). Consequently, the total duration of the experiment amounts to 10.5 minutes (see Fig. \ref{fig:psycho_py}).


The video demonstrates that the participant verbally responds to the question without engaging with the screen. Conversely, the game group can manipulate the answer options through the action of dragging and dropping or by making a selection through the act of clicking. Furthermore, the video screen does not display a timer. The participants can view the accurate answers to quiz questions, which are presented within the recorded video and are not influenced by the participants' own responses.
In contrast to the game group, individuals in this particular group do not receive a numerical score or any form of evaluative feedback, whether positive or negative.


\subsubsection{Post-test and Survey Questions} \label{label_posttest}
\hfill\\
After completing the learning methods, a post-test was carried out using a Qualtrics form. The post-test consisted of ten multiple-choice questions, each offering five options. Participants were assigned a score of 1 for each correct response. The order of the questions was randomized for each participant. No feedback was provided to the participants.

After completing the post-test, we gave questionnaires to our participants to assess the learning technique comprehensively described in the section \ref{sec:userExp}.

\subsection{Measures}\label{sec:measure} 
\subsubsection{Brain Signal Measure (Quantitative):}
In this study, brain signals are employed as a measurement method. In our study, we are utilizing functional near-infrared spectroscopy (fNIRS) as a method to quantify alterations in cerebral oxygenation ($\Delta HbO$) and deoxygenation ($\Delta$HbR). Functional near-infrared spectroscopy (fNIRS) enables the non-invasive monitoring of neural activity and the acquisition of real-time data throughout learning sessions.

The prefrontal cortex comprises four distinct regions, as depicted in figure \ref{fig:brain} (a)). The regions of interest in this study include the Left Dorsolateral Prefrontal Cortex (LDL), Left AnteroMedial Prefrontal Cortex (LMP), Right AnteroMedial Prefrontal Cortex (RMP), and Right Dorsolateral Prefrontal Cortex (RDL) \cite{koiler2022impact, milla2019does}. The four regions mentioned in the study are categorized into two groups: Ventromedial PFC (VMPFC), which comprises the left and right medial prefrontal cortex (LMP and RMP), respectively, and Lateral PFC (LPFC), which encompasses the left and right dorsolateral prefrontal cortex (LDL and RDL) respectively \cite{koiler2022impact, milla2019does}. 
The four regions of the prefrontal cortex were categorized as Channel 1 (LDL, optodes 1-4), Channel 2 (LMp, optodes 5-8), Channel 3 (RMP, optodes 9-12), and Channel 4 (RDL, optodes 13-16), based on the optode position of the fNIRS headband (refer to Figure \ref{fig:brain} (b)). The functional near-infrared spectroscopy (fNIRS) technique is utilized to quantify the changes in the concentrations of $\Delta HbO$ and $\Delta HbR$ in the prefrontal cortex of human subjects.

Initially, the calculation and comparison of the changes in oxyhemoglobin ($\Delta HbO$) and deoxyhemoglobin ($\Delta HbR$) were conducted across four distinct regions(LDL, LMP, RMP, and RDL) under both game and video conditions. Next, we examined the activation level in the lateral and ventromedial groups during the learning process, using measurements of $\Delta HbO$ and $\Delta HbR$. Finally, we determined the pre-frontal cortex's overall blood flow of oxygenated $\Delta HbO$ and deoxygenated $\Delta HbR$.

 % Figure environment removed

\subsubsection*{$\Delta HbO$ and $\Delta HbR$ four regions (LDL, LMP, RMP, and RDL) of the brain:}
The prefrontal cortex comprises four distinct regions \cite{koiler2022impact}. The calculations of $\Delta HbO$ and $\Delta HbR$ are performed for each region, and the mean, standard deviation, and standard error are determined for both the game and video conditions in each region. In this manner, comprehension can be attained regarding the functioning of neurons within each respective region during learning.
\subsubsection*{$\Delta HbO$ and $\Delta HbR$ in the Lateral PFC and Ventromedial PFC of the brain :}
As previously stated, the prefrontal cortex can be divided into two groups comprising four regions. The Ventromedial PFC (VMPFC) encompasses the left and right medial prefrontal cortex (LMP and RMP). In contrast, the Lateral PFC (LPFC) encompasses the left and right dorsolateral prefrontal cortex (LDL and RDL). Our objective is to ascertain the ratio of LPFC (lateral prefrontal cortex) and VMPFC (ventromedial prefrontal cortex) to compare the activity of neurons in the game and video conditions. The following equation (\ref{eq:RNE} is the equation to determine the ratio.
\begin{equation}
\small
\label{eq:RNE}
    \textrm{$\Delta HbO$ or    $\Delta HbR$   ratio} 
       =\frac{ \textrm{Experimental Group Region of Interest (ROI)}} {\textrm{Control Group ROI}},\textrm {where ROI refers LPFC or VMPFC}
\end{equation}
\subsubsection*{$\Delta HbO$ and $\Delta HbR$ in overall Prefrontal Cortex:}
Finally, the total value of $\Delta HbO$ and $\Delta HbR$ within the prefrontal cortex allows us to determine the neural activities under different learning conditions, specifically the game-based and video-based conditions. This analysis uses equation \ref{eq:averageHbO} to get the mean value of $\Delta HbO$ and $\Delta HbR$.

%The equation in equation (\ref{eq:percentagePrePost}) calculates the percentage difference between $\Delta HbO$ and $\Delta HbR$ across the entire prefrontal cortex region.

\begin{equation}
\label{eq:averageHbO}
    \textrm{$\Delta HbO$ or $\Delta HbR$ in PFC} =\frac{ \textrm{Sum  ($\Delta HbO$ or $\Delta HbR$  in LDL, RDL, LMP, RMP)}} {\textrm{4}}
\end{equation}
% no need
%\begin{equation}
%\small
%\label{eq:RNI}
 %   \textrm{Percentage difference}=
  %      \frac  {  \textrm{diff{($\Delta HbO$ or $\Delta HbR$  of control, experimental group)}}}{ \textrm{max({$\Delta HbO$ or $\Delta HbO$ of control, experimental group})}} * 100
%\end{equation}
\subsubsection{Knowledge Gain (Quantitative):}
To find the knowledge gain, we have calculated the difference between the score of the pre and post-test by following the equation \ref{eq:knowledge_gain} \cite{becker2000analysis}.
\begin{equation} 
\small
\label{eq:knowledge_gain}
    \textrm{Knowledge gain} = 
    \textrm{pre test score}- \textrm{pre test score}
\end{equation}

We also calculate the percentage difference of pre and post-test for both game and video conditions using the following equation (\ref{eq:percentagePrePost}):
\begin{equation}
\small
\label{eq:percentagePrePost}
    \textrm{Percentage difference}=
    \frac  {  \textrm{diff{(between two scores)}}}{ \textrm{max(between two scores)}} * 100
\end{equation}
\subsubsection{User Experience (Subjective):}\label{sec:userExp}
We asked the participants about the usability and mental workload of the study design, along with their experience in terms of motivation and engagement.
\subsubsection*{Usability:}
We used the ten questions with five response options for respondents, from Strongly agree to disagree Strongly, of the System Usability Scale (SUS), a reliable tool for measuring the usability of our learning methods \cite{brooke1996sus}. 
 The SUS score was then converted to a range between 0–100\% (0–50\%: not acceptable, 51–67\%: poor, 68\%: okay, 69–80\%:
good, 81–100\%: excellent) \cite{brooke1996sus}.

\subsubsection*{Task Load- Nasa-TLX:}
We used the NASA task load index (NASA TLX), a tool for measuring and conducting a subjective mental workload (MWL) assessment to determine the MWL of a participant while they are performing the learning methods with six questions to determine an overall workload rating \cite{hart1988development}.

\subsubsection*{Engagement and Motivation:} \label{sec:engagement_question}
We asked the following five questions to our participants to find out about the user experience based on motivation, interaction, fun, engagement, and brain storming. We asked them to rate their experience on five scales Likert scale from 1 (low) to high (5). We used Cronbach's Alpha to determine whether the collected answers were consistent. This measure is used to verify how closely related a set of items is as a group \cite{taber2018use}. A higher score refers to high internal consistency or reliability \cite{taber2018use}.  

\begin{itemize}
\item Motivating: The learning methodology was motivating			
\item Fun: The learning methodology was fun			
\item Interactive: I found this learning method interactive			
\item Engaging: I found this learning method was engaging me with the topic effectively			
\item Brainstorming: The learning method allows me to brainstorm while learning the terminologies	
\end{itemize}







\section{Results} \label{Result}

\subsection{Brain Signal Analysis $\Delta HbO$ and $\Delta HbR$}
 


\subsubsection{$\Delta HbO$ and $\Delta HbR$in four channels (LDL, LMP, RMP, and RDL) of the brain:}

From fnirSoft software, we get an Excel file that contains oxyhemoglobin ($\Delta HbO$) and deoxyhemoglobin ($\Delta HbR$) data for each optode. We calculate $\Delta HbO$ and $\Delta HbR$ channel-wise (LDL, LMP, RMP, and RDL) for both game and video conditions.

Table \ref{tab:four} shows the mean, standard deviation, and standard errors of $\Delta HbO$ and $\Delta HbR$ for both game and video conditions.  The table shows that those who played the game have higher $\Delta HbO$ values in all four channels compared to the video condition. On the other hand, in the case of $\Delta HbR$, the video condition has a higher value than the game condition in LMP and RDL. 

\begin{table}[h]
\caption{ Summary of descriptive results of $\Delta HbO$ and $\Delta HbR$ for four channels 1: LDL, 2: LMP, 3: RMP, 4: RDL}
\label{tab:four}
\begin{tabular}{l|ll|ll}\hline
                 & \multicolumn{2}{c|}{\textbf{$\Delta HbO$}}                              & \multicolumn{2}{c}{\textbf{$\Delta HbR$}}                             \\\hline 
\textbf{Channel} & \multicolumn{1}{c}{\textbf{Game}} & \multicolumn{1}{c|}{\textbf{Video}} & \multicolumn{1}{c}{\textbf{Game}} & \multicolumn{1}{c}{\textbf{Video}} \\\hline 
LDL              & 0.33 (0.61) {[}0.18{]}            & 0.21 (0.67) {[}0.19{]}             & 0.25 (0.57) {[}0.17{]}             & 0.1 (0.2) {[}0.06{]}               \\
LMP              & 0.16 (0.3) {[}0.09{]}            & -1.21 (2.59) {[}0.75{]}            & 0.16 (0.32) {[}0.09{]}            & 0.42 (1.07) {[}0.31{]}             \\
RMP              & 0.14 (0.3) {[}0.09{]}            & -1.21 (2.87) {[}0.83{]}            & 0.09 (0.27) {[}0.08{]}            & 0.18 (0.56) {[}0.16{]}             \\
RDL              & 0.23 (0.58) {[}0.17{]}            & 0.03 (0.25) {[}0.07{]}             & 0.17 (0.29) {[}0.08{]}            & 0.36 (0.71) {[}0.21{]}     \\\hline        
\end{tabular}\\
\raggedright
\hspace*{3cm} 
\textit{All entities are in the format: mean value (standard deviation) [standard error].}
\end{table}





\subsubsection{$\Delta HbO$ and $\Delta HbR$in the Lateral and Ventromedial PFC of the brain:}
We can combine the four brain channels into two groups: Lateral PFC (LPFC, combining LDL and RDL) and Ventromedial PFC (VMPFC, combining LMP and RMP). We calculate the ratio of $\Delta HbO$ of these two groups (see equation \ref{eq:RNE}).
After combining both RMP and LMP, we can see the $\Delta HbO$ is lower in Ventromedial PFC (VMPFC) than Lateral PFC (LPFC) (see table \ref{tab:LPFC VMPFC}) in both game and video condition. So, we can say that LPFC consumed more oxygen during the learning period in both conditions. Interestingly, $\Delta HbO$ in the game condition in LPFC is 2.33 times (Game:video) higher than in the video condition. On the other hand, in the case of  $\Delta HbR$, the video condition has a higher value in both VMPFC and LPFC. Also, here, VMPFC releases more carbon dioxide during the learning period. The ratio of $\Delta HbR$ of LPFC and VMPFC is 0.4. 



\begin{table}[h]
\caption{Summary of descriptive results of $\Delta HbO$ and $\Delta HbR$ value for LPFC and VMPFC}
\label{tab:LPFC VMPFC}
\begin{tabular}{l|cc|cc} \hline
      & \multicolumn{2}{c|}{\textbf{$\Delta HbO$}}                         & \multicolumn{2}{c}{\textbf{$\Delta HbR$}}               \\\hline
   \textbf{Condition}   & \textbf{LPFC}                  & \textbf{VMPFC}                   & \textbf{LPFC}                   & \textbf{VMPFC}                  \\\hline
\textbf{Game}  & 0.28 (0.56) {[}0.16{]} & 0.15 (0.3) {[}0.09{]}  & 0.21 (0.38) {[}0.11{]} & 0.12 (0.25) {[}0.07{]} \\
\textbf{Video} & 0.12 (0.4) {[}0.12{]} & -1.21 (2.72) {[}0.79{]} & 0.23 (0.43) {[}0.13{]} & 0.3 (0.81) {[}0.23{]} \\ \hline
\end{tabular}\\
\raggedright
\hspace*{3cm} 
\textit{All entities are in the format: mean value (standard deviation) [standard error].}
\end{table}


\subsubsection{$\Delta HbO$ and $\Delta HbR$ in the prefrontal cortex:}

The game condition has higher $\Delta HbO$ (M=0.21 (SD=0.42)[SE=0.12)]) compared to the video condition (M=-0.54 (SD=1.19, [SE=0.34]). The result shows that the participants who played the game consumed 357.14\% more oxygen than those who watched the video. On the other hand, the video condition (M=0.26) released 38.46\%  more carbon dioxide than the game condition (M=0.16). 
 
The boxplot (figure \ref{fig:delHbRHbOBox}) shows that the median value of $\Delta HbO$ in-game condition is higher than the video condition. Also, the variability of $\Delta HbO$ of the game and $\Delta HbR$ of the video is less. For $\Delta HbO$, there was no significant difference between the game and video condition (F(5, 5)= 2.09, p = 0.179). We further analyzed the data with the pairwise t-test, and the results show no statistically significant
difference between the two groups (t = 1.44, df = 11, p = 0.179). There are also no statistically significant differences for $\Delta HbR$ for both game and video conditions (F(5, 5) = 0.11, p = 0.742). The t-test result shows no statistically significant difference (t = -0.34, df=11,p = 0.742). 



% Figure environment removed
\subsection{Knowledge Gain}

Seven questions totaling 350 points (each 50) were asked during the learning time. The combined learning time was 210 seconds, and the average result for both groups was 312.5.
Figure \ref{fig:scoreDiff} shows the average pre and post-test score difference between video and game conditions. The result shows that the game condition (M=3.5 $\pm$ 1.52) has  higher knowledge gain from the video condition (M=1.83 $\pm$ 1.33). The game-based approach has 47.74\% (see equation \ref{eq:percentagePrePost}) higher knowledge gain than the video group.
%%%%%%%%%%%%%%
There was no statistical difference between the game and video conditions (t=2.02,  p=0.07).
Figure \ref{fig:percentagePreandPost} shows the percentage difference between the pre and post-test. We found the percentage difference of the game group is higher (50\%) than the video group (33.33\%). The difference is 16.67\%. 


% Figure environment removed
\subsection{Survey Questions}
\subsubsection{Usability:}
 The results of system usability using the SUS questionnaire from all participants show an average score of
(M = 68.33 $\pm$ 13.29) for the game and (M = 61.67$\pm$ 12.32) for the video Condition (see Fig. \ref{fig:NASA_score} (left)). The results show a small difference between Conditions, and the game has the higher SUS score. The game group exhibited a higher SUS score (68.33), indicating a comparatively higher level of perceived usability than the video group (61.67). The results of this study suggest that the participants perceived the game-based learning environment to be more user-friendly and intuitive than the video-based learning environment, as evidenced by the higher System Usability Scale (SUS) score.

The ANOVA analysis revealed that the differences observed were not statistically significant (p > 0.05). Specifically, the one-way ANOVA showed no significant effect (F(5, 5) = 0.81, p = 0.389). Similarly, the t-test results indicated no statistically significant differences between the groups (p > 0.05). The independent samples t-test yielded a non-significant result (t(10) = -0.90, p = 0.389).
% Figure environment removed



\subsubsection{Task Load:}
The subjective task load among Conditions was assessed using an unweighted (raw) NASA-TLX questionnaire. The results for overall scores show an average of (M=25) for the game and (M = 51.67) for the video Condition. The results align with the subjective assessment, suggesting that participants viewed the game environment as less mentally demanding than watching the video.
Moreover, descriptive results show that the average scores of mental demand,
physical demand, performance, and frustration in the video are averagely higher,  and temporal demand and effort
are slightly less than the game condition (see Fig. \ref{fig:NASA_score} (right)). There was no statistical difference between the game and video conditions for all the questions.

\subsubsection{Engagement and Motivation:}
We asked the five questions our participants to find out about the user experience based on motivation, interaction, fun, engagement, and brainstorming mentioned in section {\ref{sec:engagement_question}}. We used a 5-scale Likert scale to record their experience from strongly disagree(1) to (5) strongly agree. The result in fig \ref{fig:fun} shows that in terms of motivation, fun, engagement, interaction, and brainstorming, the game got a higher value than the video. There was no statistical difference between the game and video conditions for all the questions.
We applied Cronbach Alpha to find out the reliability of the responses. Table \ref{tab:cronbach_game2} shows higher Cronbach Alpha, indicating that the survey results are reliable.
\begin{table}[]
\caption{Internal Consistency Measures Using Cronbach's Alpha}
\label{tab:cronbach_game2}
\begin{tabular}{llll}
\cline{1-4}
\textbf{Questions}                                                                                                      & \textbf{Items}         & \textbf{Game} & \textbf{Video} \\ \cline{1-4} 
                                                                                                                        & All itmes              & 0.97                           & 0.90                            \\
The learning methodology was motivating                                                                                 & Motivating excluded    & 0.97                           & 0.89                            \\
The learning methodology was fun                                                                                        & Fun excluded           & 0.95                           & 0.85                            \\
I found this learning method interactive                                                                                & Interactive excluded   & 0.97                           & 0.82                            \\
\begin{tabular}[c]{@{}l@{}}I found this learning method was engaging \\ me with the topic effectively\end{tabular}      & Engaging excluded      & 0.95                           & 0.90                            \\
\begin{tabular}[c]{@{}l@{}}The learning method allows me to brainstorm \\ while learning the terminologies\end{tabular} & Brainstorming excluded & 0.95                           & 0.90                            \\ \cline{1-4} 
\end{tabular}
\end{table}

% Figure environment removed



\section{Discussion and Conclusions}\label{Conclusion}
This study compares video-based and game-based learning techniques by examining brain signals within the participant's prefrontal cortex. Both game-based and video-based learning approaches contain similar content based on basic terminologies of graph structure. During the learning procedure, the fNIRS device measured changes in oxygenated hemoglobin ($\Delta HbO$) and deoxygenated hemoglobin ($\Delta HbR$) levels in the prefrontal cortex. The objective was to evaluate the efficacy of both approaches.

Although there is no statistically significant difference in the change of $\Delta HbO$ and $\Delta HbR$ during the learning period involving a game and a video, based on the mean value, the game group (M = 0.21) tended to have a (357.14\%) higher concentration of $\Delta HbO$   compared to the video group (M = -0.54). The observation of a positive mean value for $\Delta HbO$ suggests that there may be increased oxygenation during gameplay, indicating higher neural activity compared to video watching. A negative change in the hemoglobin concentration in the oxygenated state ($\Delta HbO$) implies a reduction in oxygenated hemoglobin relative to the baseline level. This observation suggests that there may be a decrease in oxygen flow to the prefrontal cortex during the video condition. However, the change in the $\Delta HbR$ concentration during the learning period is almost similar in both the game group (mean = 0.16) and the video group (mean = 0.26). In summary, it can be inferred that participants exhibit increased neural activity during game conditions compared to the video condition. 
%According to existing theories, changes in HbO and HbR levels indicate greater neural activity when individuals are engaged in task completion \cite{cakir2015optical,ccakir2016behavioral,koiler2022impact}.
Another observation reveals that when comparing the change in oxygenated hemoglobin ($\Delta HbO$) levels between the lateral prefrontal cortex (LPFC) and the ventromedial prefrontal cortex (VMPFC), LPFC consistently exhibits higher values. This suggests that the LPFC demonstrates higher neural activity during the learning phase than other prefrontal cortex regions. The $\Delta HbO$ ratio between the game and the video in the lateral prefrontal cortex (LPFC) indicates that the LPFC is 2.33 times more active during the game condition. 


The utilization of gaming interfaces, including timer mechanisms, drag-and-drop functionality, and click-based interactions, alongside prompt feedback in the form of sounds, visual representations such as emojis, scoring systems, and celebratory animations like confetti, within the Unity game, could lead to an increased oxygenation flow.  This is also reflected in the knowledge gain we get from the result. The game-based approach has 47.74\% higher knowledge gain than the video group. The user experience questionnaires also support the game-based approach as more user-friendly, fun, and interactive than the video one.

Further research could be conducted to explore the potential benefits of different game design elements and their impact on educational outcomes in the future. Furthermore, our objective is to mitigate certain constraints present in the existing research by augmenting the sample size and enhancing the gaming interface. In subsequent investigations, we intend to expand the scope of this study by conducting an additional version of the experiment with a group of middle school students.
 Additionally, we intend to integrate this game into Augmented Reality (AR) and Virtual Reality (VR) platforms, enabling participants to immerse themselves in a realistic environment and enhance their level of engagement. In addition, we plan to assess the system quantitatively by identifying users' facial expressions and monitoring their gaze.




\bibliographystyle{ACM-Reference-Format}
\bibliography{main}

\end{document}
\endinput
%%
%% End of file `sample-authordraft.tex'.
