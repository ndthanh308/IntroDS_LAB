%\documentclass[12pt,onecolumn,draftcls]{IEEEtran} 
\documentclass[journal]{IEEEtran}
\usepackage{verbatim}
\usepackage{graphicx}
\usepackage{algorithm,algorithmic}
\usepackage{amsmath,amssymb,amsfonts,bm}
\usepackage{subfigure}
\usepackage{psfrag}
\usepackage[dvips]{epsfig}
\usepackage{amsthm}
\usepackage{color}
\usepackage[usenames,dvipsnames]{pstricks}
\usepackage[dvips]{epsfig}
\usepackage{pst-grad} % For gradients
\usepackage{pst-plot} % For axes
%\usepackage{appendix}
%\usepackage{setspace}
%\usepackage{slashbox}
\usepackage{tikz,pgf}
\usepackage{xpatch,letltxmacro}
\tikzset{font={\fontsize{10pt}{12}\selectfont}}
\newcommand\scalemath[2]{\scalebox{#1}{\mbox{\ensuremath{\displaystyle #2}}}}
\usepackage{amsmath,graphicx}
\usepackage{amsmath,graphicx}
\usepackage{enumerate}
\usepackage{amsbsy}
\usepackage{amssymb}
\usepackage{amsthm}
\usepackage{amscd}
\usepackage{subfigure}
\usepackage{color}
\usepackage{cite}
\usepackage{listings}
\usepackage{tikz}
\tikzset{font={\fontsize{10pt}{12}\selectfont}}
\usepackage{pgf}
\newtheorem{theorem}{Theorem}
\newtheorem{lemma}{Lemma}
\usepackage{amsthm}
\usepackage{stackrel}
%\doublespacing
\newtheorem{thm}{Theorem}
\newtheorem{lemm}{Lemma}
\newtheorem{cor}{Corollary}
\newtheorem{rema}{Remark}
\newtheorem{assmp}{Assumption}
\newtheorem{prop}{Proposition}
%\newtheorem{proof}{Proof}
%\DeclareMathOperator{\diag}{diag} \DeclareMathOperator{\blkDiag}{blkDiag} \DeclareMathOperator{\tr}{tr} \DeclareMathOperator{\E}{E} \DeclareMathOperator{\Prob}{Prob} \setcounter{page}{1}
%\def\blkDiag{\mathbf{blkDiag}}
\DeclareMathOperator{\blkDiag}{blkDiag}
\def\wB{{\bw}}
\DeclareMathOperator{\tr}{tr}
\def\bone{\bf 1}
\def\Ps{P_{\rm s}}
\def\Pn{P_{\rm n}}
\def\Pint{P_{\rm i}}
\DeclareMathOperator{\E}{E}
\def\bx{{\bf x}}
\def\bnu{{\boldsymbol \nu}}
\def\bxi{{\boldsymbol \xi}}
\def\bq{{\bf q}}
\def\bT{{\bf T}}
\def\bn{{\bf n}}
\def\bt{{\bf t}}
\def\bz{{\bf z}}
\def\bZ{{\bf Z}}
\def\bD{{\bf D}}
\def\by{{\bf y}}
\def\bY{{\bf Y}}
\def\bV{{\bf V}}
\def\bW{{\bf W}}
\def\bX{{\bf X}}
\def\bu{{\bf u}}
\def\bh{{\bf h}}
\def\bb{{\bf b}}
\def\bH{{\bf H}}
\def\bP{{\bf P}}
\def\bQ{{\bf Q}}
\def\bv{{\bf v}}
\def\bc{{\bf c}}
\def\bC{{\bf C}}
\def\bB{{\bf B}}
\def\bg{{\bf g}}
\def\bM{{\bf M}}
\def\bG{{\bf G}}
\def\bw{{\bf w}}
\def\bR{{\bf R}}
\def\br{{\bf r}}
\def\bI{{\bf I}}
\def\bE{{\bf E}}
\def\bd{{\bf d}}
\def\bv{{\bf v}}
\def\bA{{\bf A}}
\def\ba{{\bf a}}
\def\bU{{\bf U}}
\def\bJ{{\bf J}}
\def\bK{{\bf K}}
\def\bl{{\bf l}}
\def\bs{{\bf s}}
\def\bm{{\bf m}}
\def\bS{{\bf S}}
\def\bi{{\bf i}}
\def\complexC{{\mathbb{C}}}
\def\realR{{\mathbb{R}}}
\def\bx{{\bf x}}
\def\bp{{\bf p}}
\def\bof{{\bf f}}
\def\bF{{\bf F}}
\def\bQ{{\bf Q}}
\def\be{{\bf e}}
\def\bPsi{{\bf \Psi}}
\def\balpha{{\boldsymbol \alpha}}
\def\bTheta{
{\boldsymbol \Theta}}
\def\btheta{{\boldsymbol \theta}}
\def\bPhi{{\bf \Phi}}
\def\bLambda{{\bf \Lambda}}
\def\bOmega{{\bf \Omega}}
\def\bDelta{{\bf \Delta}}
\def\bOmega{{\bf \Omega}}
\def\bGamma{{\bf \Gamma}}
\def\bUpsilon{{\bf \Upsilon}}
\def\blambda{{\boldsymbol \lambda}}
\def\diag{{\rm diag}}
\def\bSigma{{\boldsymbol \Sigma}}
\def\argmax{{\rm argmax}}
\def\rank{{\rm rank}}
\def\bzero{{\boldsymbol 0}}
\def\bone{{\boldsymbol 1}}
\def\Fig{Fig. }
\def\wi{10}
\def\he{4}
\makeatletter
\newcommand*{\rom}[1]{\expandafter\@slowromancap\romannumeral #1@}
\begin{document}
\title{Multi-UAV Enabled Integrated Sensing and Wireless Powered Communication: A Robust Multi-Objective Approach}

\author{Omid Rezaei, Mohammad~Mahdi~Naghsh\IEEEauthorrefmark{1}, \emph{Senior~Member, IEEE}, Seyed Mohammad Karbasi, \emph{Senior~Member, IEEE}, and Mohammad Mahdi Nayebi, \emph{Senior~Member, IEEE}

\thanks{A limited part of this work is accepted for publication in IEEE International Conference on Acoustics, Speech and Signal Processing (ICASSP), Rhodes Island, Greece, June 2023 \cite{10094604}.
	
	O. Rezaei, S. M. Karbasi, and M. M. Nayebi are with the Department of Electrical Engineering, Sharif University of Technology, Tehran, 11155-4363, Iran. M. M. Naghsh is with the Department of Electrical and Computer Engineering, Isfahan University of Technology, Isfahan, 84156-83111, Iran. *Please address all the correspondence to M. M. Naghsh, Phone: (+98) 31-33912450; Fax: (+98) 31-33912451; Email: mm\_naghsh@iut.ac.ir} }

\maketitle



\begin{abstract}
In this paper, we consider an integrated sensing and communication (ISAC) system with wireless power transfer (WPT) where multiple unmanned aerial vehicle (UAV)-based radars serve multiple clusters of energy-limited communication users in addition to their sensing functionality. In this architecture, the radars sense the environment in phase 1 (namely sensing phase) and meanwhile, the communications users (nodes) harvest and store the energy from the radar transmit signals. The stored energy is then used for information transmission from the nodes to UAVs in phase 2, i.e., uplink phase. Performance of the radar systems depends on the transmit signals as well as the receive filters; the energy of the transmit signals also affects the communication network because it serves as the source of uplink powers. Therefore, we cast a multi-objective design problem addressing performance of both radar and communication systems via optimizing UAV trajectories, radar transmit waveforms, radar receive filters, time scheduling and uplink powers. The design problem is further formulated as a robust non-convex optimization problem taking into account the the user location uncertainty. Hence, we devise a method based on alternating optimization followed by concepts of fractional programming, S-procedure, and tricky majorization-minimization (MM) technique to tackle it. Numerical examples illustrate the effectiveness of the proposed method for different scenarios. 
\end{abstract}
\begin{keywords}
	Integrated sensing and communication (ISAC), multi-objective optimization, robust resource allocation, unmanned aerial vehicle (UAV), waveform design, wireless power transfer (WPT).
\end{keywords}

\section{Introduction}
In recent years, driven by enormous demands of concurrent sensing and communication ability, there is a lot of interest in developing a new paradigm referred to as integrated sensing and communication (ISAC) in both academia and industry \cite{hassanien2019dual}. Note that researches related to this topic have also been addressed with different terminologies like joint radar communication (JRC) \cite{gameiro2018research,feng2020joint,ref10}, joint communication and radar sensing (JCAS) \cite{zhang2018multibeam,mishra2019toward,ref14}, dual-functional radar communication (DFRC) \cite{wang2020constrained,liu2021dual}, and radar communication (RadCom) \cite{zhang2020joint,de2021joint}.

 
Thanks to its high mobility and flexibility, unmanned aerial vehicle (UAV) has emerged as a key technology in future ISAC systems \cite{meng2022uav}. Precisely, it is expected that UAVs will bring better coverage and improved sensing and communication services in modern ISAC systems \cite{meng2022uav1}. 


There are several works in the literature that have considered the UAV-enabled ISAC \cite{lyu2021joint,meng2022throughput,chen2020performance,ref8,wang2020constrained,ref9,ref12,ref13}. The authors in \cite{lyu2021joint} have considered a sum-rate maximization problem constrained to sensing requirements for given targets in a single UAV-enabled ISAC model. In contrast with the mentioned work that the UAV must provide the communication services and sensing tasks simultaneously, the authors in \cite{meng2022throughput} considered a periodic sensing and communication strategy for their UAV-enabled ISAC model to separate the practical requirements of sensing and communication over time. 



It is possible to realize more effective sensing and communication with multi-UAV cooperation compared to a single UAV with limited sensing coverage and communication capability. The work in \cite{chen2020performance} proposed the problem of UAV sensing range maximization based on mutual sensing interference and the communication capacity constraints in a cooperative multi-UAV network. In \cite{ref8}, the completion time minimization problem for multi-UAV ISAC systems is studied. By considering a given required localization accuracy for radar
sensing, the authors in \cite{wang2020constrained} studied maximization of both sum and minimum communication rates under the Cramer-Rao bound constraint of the target localization in a multi-UAV ISAC system. In \cite{ref9}, authors proposed a framework based on the extended Kalman filter to track the ground users in a multi-UAV ISAC network. Then, a UAV swarm-enabled ISAC model in \cite{ref12} considered a distributed cooperative framework for multi-target tracking. In \cite{ref13}, the resource allocation problem for a multi-UAV ISAC system is addressed via a method based on reinforcement learning.







%In the above works, we have to use a low-power information-bearing signal as a radar transmit waveform which reduces our degrees of freedom to design the radar waveform. As opposed to the ISAC literature, in this paper we consider a UAV-enabled integrated sensing and wireless powered communication (ISWPC) where sensing can be done in the wireless power transfer (WPT) phase of the communication and therefore, the radar/WPT waveform can be designed with more degrees of freedom.

The aforementioned works assume that UAVs employ information-bearing signals as the radar probing signals as well; this leads to limitation of radar capabilities due to availability of low sensing energy as well as special characteristics of communication signals. This observation is a motivation for the current work in which sensing is performed in wireless power transfer (WPT) phase. Precisely, the main contributions of this paper can be summarized as follows.
\begin{itemize}
	\item  In this paper, we consider a multi-UAV enabled integrated sensing and wireless powered communication (ISWPC), where sensing is done in the WPT phase of the communication. Consequently, the radar/WPT waveforms can be designed for sensing purpose with more power leading to radar performance improvement. Precisely, in the first phase of our ISWPC model, multiple UAV-based phase modulated continuous
	wave (PMCW) radars transmit sensing waveforms and then, targets (e.g., non-authorized UAVs \cite{guerra2022networks,guvenc2018detection}) can be detected by filtering the backscattered signals. The energy of these sensing waveforms can also be harvested by the energy-limited ground users\footnote{The interested reader may refer to \cite{nguyen2022ris,li2022joint,xie2021uav,xie2018throughput,xu2018uav} where ground users harvest energy from UAV signals.}. Then, in the wireless information transfer (WIT) phase, the users in each cluster can upload their information signals to their associated UAVs. 
	\item
	The aim is to jointly maximize the minimum radar signal-to-interference-plus-noise ratio (SINR) and minimum throughput of communication users by designing the transmit radar/WPT waveforms, radar receive filters, time scheduling as well as uplink power of users, and UAV trajectories under and user location uncertainty. 
	\item
	The cast multi-objective robust design problem is non-convex and hence, hard to solve. Therefore, we first adopt the scalarization technique to rewrite the objective as a specific weighted sum of sensing and communication metrics. Then, we devise a method exploiting the concepts of fractional programming, S-procedure, and tricky majorization-minimization (MM) techniques in order to deal with
	the problem efficiently. Our simulation results show the effectiveness of the multi-UAV enabled ISWPC. 
	\item
	In addition to proposing a wireless powered model for the UAV-enabled ISAC, which is our main contribution, the following items
	are also the other contributions of this work: (i) designing the radar
	receive filters to maximize the minimum sensing SINR in an ISAC model, and
	(ii) proposing a multi-objective design problem for a joint sensing-communication metric using scalarization technique in an ISAC model.
%	Note that a limited part of this work has been presented in [ref];
%	more precisely, the aforementioned conference article addressed only a single-UAV system (which leads to an interference-free communication) with perfect user location information.		
\end{itemize}



The rest of this paper is organized as follows. The signal and system models are explained in Section \ref{sys}. In Section \ref{sum}, a multi-objective problem is formulated, and an optimization framework is proposed for dealing with the design problem associated with the ISWPC model. Section \ref{num} presents numerical examples to illustrate the effectiveness of the proposed method. Finally, conclusions are drawn in Section \ref{con}.


\emph{Notation:} Bold lowercase (uppercase) letters are used for vectors (matrices).
The notations $\Re \{ \cdot \}$, $\mathbb{E} [\cdot ]$, $\vert \cdot \vert$, $\|\cdot\|_2$, ${(\cdot)^{{T}}}$, $(\cdot)^{{H}}$, $(\cdot)^{{*}}$, and $\mbox{tr} \{ \cdot \}$ indicate the real-part, statistical expectation, absolute value, $l_2$-norm of a vector, transpose, Hermitian, complex conjugate, and trace of a matrix, respectively. The symbols $[\mathbf{A}]_{i,j}$ and $[\mathbf{a}]_i$ are used for element-wise representation of the matrix $\mathbf{A}$ and vector $\mathbf{a}$, respectively. The notation $\nabla f(\cdot)$ indicates the gradient of the function $f$. We denote $\mathcal{CN}(\boldsymbol{\omega},\mathbf{\Sigma})$ as a circularly symmetric complex Gaussian (CSCG) distribution with mean $\boldsymbol{\omega}$ and covariance $\mathbf{\Sigma}$. The set $\mathbb{R}$ represents real numbers and $\mathbb{R}^{N}$ and $\complexC^{N}$ are the set of ${N \times 1}$ real and complex vectors, respectively. The set of ${N \times N}$ Hermitian and identity matrices are denoted by $\mathbb{H}^{N \times N}$ and $\bI_{N}$, respectively. The notation $\bA \succeq \bB$ means that $\bA-\bB$ is positive semi-definite.


\section{System Model} \label{sys}
We consider a multi-UAV enabled ISWPC where $M$ UAVs are employed to serve $KM$ single-antenna ground users denoted by $\mathrm{U}_{k,m}, \hspace{1pt}1\leq k \leq K,\hspace{1pt}1\leq m \leq M, $ in $M$ clusters and also act as surveillance radars (see Fig.~\ref{ht}). First, the UAVs transmit energy to users and also perform radar sensing. Then, users in each cluster transmit their information signals to their associated UAVs. Each user has an energy harvesting circuit and can store energy for its operation. 
In this paper, we assume that UAVs are equipped with directional single-antenna with half power beam-width of $\zeta$ (in degree) for radar/communication transmitters as well as communication receivers; the radar receivers can have several antennas (i.e., an uniform planar array (UPA) of antennas) to estimate target direction in addition to its range and Doppler. The
UAVs are assumed to fly at the time-varying coordinate $\mathbf{q}_m (t) = [x_m(t), y_m(t), z_m (t)]^T \in \realR^{3},$ at
time interval $0 \leq t \leq T$. The period $T$ is discretized into $N$ equal time slots where the elemental slot
length $\delta_t = T/N$ is chosen to be sufficiently small such that the location of UAVs are considered as approximately unchanged within each time slot $\delta_t$. As a result, the trajectory of $m$th UAV, denoted by $\mathrm{UAV}_m$, can be approximated by
the sequence $\mathbf{q}_m[n] = [x_m[n], y_m[n], z_m[n]]^T,\hspace{1pt} 1 \leq n \leq N$. The user location information at UAVs, provided e.g., by GPS, may be also imperfect due to radio signal interference \cite{cui2018robust}. Thus, in this paper, we take into account the user location uncertainty for robust resource allocation.
		% Figure environment removed
Then, the coordinates of $\mathrm{U}_{k,m}$ are modeled as 
\begin{eqnarray}
x_{k,m}=\bar{x}_{k,m} + \Delta x_{k,m},
~~~~
y_{k,m}=\bar{y}_{k,m} + \Delta y_{k,m},
\end{eqnarray}
respectively, where $\bar{x}_{k,m}$ and $\bar{y}_{k,m}$ are the user location estimates
available at UAVs, and $\Delta x_{k,m}$ as well as $\Delta y_{k,m}$ denote the associated location uncertainties. Furthermore, we assume that UAVs
know their own location perfectly \cite{xie2021uav,xie2018throughput,xu2018uav,cui2018robust}. Then, letting
$
\mathbf{r}_{k,m} = [x_{k,m} , y_{k,m}]^T,
\hspace{2pt}
\bar{\mathbf{r}}_{k,m} = [\bar{x}_{k,m} , \bar{y}_{k,m}]^T$, and $\Delta \mathbf{r}_{k,m} = [\Delta x_{k,m} , \Delta y_{k,m}]^T,
$ the distance from $\mathrm{UAV}_m$ to $\mathrm{U}_{k,i}$ in time slot $n$ can be expressed as 
% Figure environment removed
\begin{equation}
d_{k,i,m} [n]= \sqrt{ {\parallel \widetilde{\mathbf{q}}_m [n] - \mathbf{r}_{k,i} \parallel }^2_2 + z_m^2 [n]},
\end{equation}
where $\widetilde{\mathbf{q}}_m [n]= [x_m[n], y_m[n]]^T $. We assume that the communication links from
UAVs to the ground users are dominated by the line-of-sight (LoS) links. The channel power gain from $\mathrm{UAV}_m$ to $\mathrm{U}_{k,i}$ during
slot $n$ follows the free-space path loss model which can be written as $h_{k,i,m} [n]=\rho_0 d^{-2}_{k,i,m} [n]$, where $\rho_0$ denotes the channel power at the reference distance $d_0 = 1$ m. 

Fig.~\ref{ht1} shows the timing protocol of the proposed method. Each time slot of width $\delta_t$ is divided into $K+1$ subslots where the first subslot with duration $\tau_0 \delta_t$ is used for sensing/WPT and the other subslots of duration $\tau_{k,m} [n] \delta_t$ are allocated for uplink transmission from users to UAVs. 
\subsection{Downlink: Sensing/WPT Phase}
The sensing interval consists of $L$ successive transmission of PMCW radar sequences $\widetilde{\mathbf{x}}_m \in \complexC^{\widetilde{N}},~1\leq m \leq M,$ with total transmission power of $p_m^{dl}[n]$. Note that we consider $\widetilde{N}$ as a fixed parameter  determined as per radar range resolution and other practical considerations. The $l$th received signal of a target associated with a given $\widetilde{\mathbf{x}}_m$ in $\mathrm{UAV}_m$ at the cell under test can be modeled as\footnote{We do not consider the effect of Doppler shift for intra-pulse code $\widetilde{\mathbf{x}}_m$. That is, this effect is incorporated in complex coefficients  $\alpha_{m} [n]$ and $\widetilde{\alpha}_{m,k} [n]$ as a constant phase. However, the inter-pulse Doppler shift is non-negligible and will be taken into account in the following. Besides, we ignore the interference from other UAV radars, i.e., $\mathrm{UAV}_i,~ \forall i \neq m$, due to their weak echo signals.}
\begin{align}  \label{en1}
	\widetilde{\mathbf{y}}_{m,l} [n]=& \alpha_{m} [n] \widetilde{\mathbf{x}}_m  +\sum_{k=-\widetilde{N}+1 ,k\neq 0}^{\widetilde{N}-1} \widetilde{\alpha}_{m,k} [n] \mathbf{J}_k  \widetilde{\mathbf{x}}_m \\ \nonumber&+\mathbf{n}_{m,l} [n],~ \forall n,m, 1 \leq l \leq L,
\end{align}
where $\alpha_{m} [n]$ and $\widetilde{\alpha}_{m,k} [n]$ are the complex parameter corresponding to the propagation and backscattering effects from the desired and interfering targets, respectively, $\mathbf{n}_{m,l} [n]\sim \mathcal{CN} \left(\mathbf{0},{{\sigma}_{m,l}^2}{\mathbf{I}}_{\widetilde{N}} \right)$ is a noise vector, and $\mathbf{J}_k$ denotes the periodic shift matrix 
\begin{equation}
	\mathbf{J}_k=\begin{bmatrix}
		\mathbf{0}_{(\widetilde{N}-k)\times k} & \mathbf{I}_{\widetilde{N}-k} \\ \mathbf{I}_{k} & \mathbf{0}_{k \times (\widetilde{N}-k)}
	\end{bmatrix}, k\geq 1, ~ \mathbf{J}_{-k}=\mathbf{J}^T_k.
\end{equation}
Then, the received signal $\widetilde{\mathbf{y}}_{m,l} [n]$ is processed via $\widetilde{\mathbf{w}}_m \in \complexC^{\widetilde{N}}$, viz. range processing. The time delay at which each receive filter output signal (i.e., the correlation between $\widetilde{\mathbf{w}}_m$ and $\widetilde{\mathbf{y}}_m$) has its maximum value can be used to estimate the target range. Following the range processing, Doppler processing is applied on each range-cell to obtain target speed via frequency analysis (usually implemented by FFT) of $L$ samples associated with the range cell (see Fig.~\ref{hh1}). Therefore, in practice, it is better to choose $L$ as $2^{i}, \hspace{1pt}lb \leq i \leq ub$, where $lb$ and $ub$ are respectively determined according to the minimum required sensing SINR and the allowed system complexity. The sensing/WPT subslot duration can be obtained as $\tau_0 \delta_t=L \widetilde{\tau}_0\delta_t$ where $\widetilde{\tau}_0$ is a fix parameter that can be determined by $\widetilde{N}$ and the sampling frequency of radars (see Fig.~\ref{ttt}). Using \eqref{en1}, the sensing SINR of  $\mathrm{UAV}_m$ after range processing block, i.e., the signal $\bar{y}_{m,l}[n]={\widetilde{\mathbf{w}}}^H_m \widetilde{\mathbf{y}}_{m,l}[n]$ can be written as
% Figure environment removed
% Figure environment removed
\begin{equation} \label{kkl}
	{{\mathrm{SINR}}}_{m,l}\hspace{1pt}[n]= \frac{  {\vert \alpha_{m}[n] \vert}^2  {\vert \widetilde{\mathbf{w}}_m^H \widetilde{\mathbf{x}}_m \vert}^2} {\widetilde{\mathbf{w}}_m^H \hm{\Xi}_{m,l} \widetilde{\mathbf{w}}_m},
\end{equation}
where
\begin{align} \label{kkl21}
	\hm{\Xi}_{m,l}=& {{\sigma}_{m,l}^2}{\mathbf{I}}_{\widetilde{N}} + \sum_{k=-\widetilde{N}+1 ,k\neq 0}^{\widetilde{N}-1} \widetilde{\sigma}_{m,k}^2 \mathbf{J}_k \widetilde{\mathbf{x}}_m \widetilde{\mathbf{x}}_m^H  \mathbf{J}^H_k,
\end{align}
is a positive definite matrix and $\widetilde{\sigma}_{m,k}^2=\mathbb{E} [\vert \widetilde{\alpha}_{m,k} [n] \vert^2]$. Then, assuming a-priori known target Doppler shift\footnote{Several techniques assume a-priori known Doppler frequency (see e.g., \cite{de2011design,naghsh2013unified}); however, in practice, the Doppler shift can be estimated at the receiver, e.g., via a bank of filters matched to different Doppler frequencies \cite{stoica2008transmit}.}, the total sensing SINR of $\mathrm{UAV}_m$ after Doppler processing block can be obtained as
\begin{equation} \label{kkl0}
	\widetilde{{\mathrm{SINR}}}_{m}\hspace{1pt}[n]= \frac{ \beta_m {\vert \alpha_{m}[n] \vert}^2  {\vert \widetilde{\mathbf{w}}_m^H \widetilde{\mathbf{x}}_m \vert}^2} {\widetilde{\mathbf{w}}_m^H \widetilde{\hm{\Xi}}_{m} \widetilde{\mathbf{w}}_m},
\end{equation}
where
\begin{equation}
	\beta_m=L+ \sum_{i=1}^{L}\sum_{j=1, j \neq i}^{L}
	[\widetilde{\mathbf{a}}_m^{*}]_i \hspace{1pt} [\widetilde{\mathbf{a}}_m]_j,
\end{equation}
with $\widetilde{\mathbf{a}}_m \in \complexC^{L}$ is the Doppler processing filter\footnote{For instance, the filter $\widetilde{\mathbf{a}}_m$ can be obtained as $\widetilde{\mathbf{a}}_m=[1, e^{-j\nu_m},...,$ $e^{-j{\nu_m}(L-1)}]^T$ as a simple FFT-based Doppler processing filter with $\nu_m$ being the normalized Doppler frequency between the $\mathrm{UAV}_m$ and a possible target.}, and 
\begin{align} \label{kkl210}
	\widetilde{\hm{\Xi}}_{m}=& \sum_{l=1}^{L} {{\sigma}_{m,l}^2}{\mathbf{I}}_{\widetilde{N}} +\beta_m \sum_{k=-\widetilde{N}+1 ,k\neq 0}^{\widetilde{N}-1} \widetilde{\sigma}_{m,k}^2 \mathbf{J}_k \widetilde{\mathbf{x}}_m \widetilde{\mathbf{x}}_m^H  \mathbf{J}^H_k,
\end{align}
is a positive definite matrix.
Assuming Swerling \rom{1} target model, $\alpha_{m}[n]$ is assumed to be constant over time slots, i.e., $\alpha_{m}[n]=\alpha_{m}$, and so $\widetilde{{\mathrm{SINR}}}_{m}\hspace{1pt}[n]=\widetilde{{\mathrm{SINR}}}_{m}\hspace{1pt}$.


The harvested energy at $\mathrm{U}_{k,m}$ for linear energy harvesting model can be expressed as\footnote{In practice, the minimum required energy for typical energy harvesting circuits can be in the range of $1\sim10~\mu$W \cite{clerckx2018fundamentals}.}
\begin{equation}
E_{k,m} [n]=\tau_0 \delta_t \epsilon_{k,m} \sum_{i=1}^{M} h_{k,m,i} [n] p_i^{dl}[n],
\end{equation}
where $\epsilon_{k,m}$ denotes the energy conversion efficiency. Note that in practice, there is a non-linear characteristic between $E_{k,m} [n]$ and $p_i^{dl}[n]$. However, a linear energy harvesting circuit is considered here to make the ISWPC model more tractable\footnote{The interested reader may see \cite{rezaei2019throughput} for more details about non-linear energy harvesting models.}. 
\subsection{Uplink: WIT Phase}
In the uplink phase, the transmit power of $\mathrm{U}_{k,m}$ in time slot $n$ is denoted by $p^{{ul}}_{k,m} [n]$. The following energy harvesting constraint for $\mathrm{U}_{k,m}$ at time slot $n$ should be satisfied: 
\begin{align} \label{key}
	& \scalemath{.97}{\tau_{k,m}[n] \delta_t  p^{{ul}}_{k,m} [n]  \leq \sum^{n}_{j=1} E_{k,m} [j] + E^0_{k,m} - \sum^{n-1}_{j=1} \tau_{k,m}[j] \delta_t  p^{{ul}}_{k,m} [j],}
\end{align}
where $ E^0_{k,m}$ is the remaining
energy for $\mathrm{U}_{k,m}$ from previous time periods\footnote{It is assumed that $E^0_{k,m}$ is associated with sensing-only mode where the system works before employing the proposed protocol in $[0 , T]$. Moreover, some parts of $E^0_{k,m}$ can be obtained by solar energy where $\mathrm{U}_{k,m}$ has a hybrid solar-RF energy collector circuit \cite{tran2022hybrid}. Note that $E^0_{k,m}$ guarantees the reliable energy for uninterrupted communication in the WIT phase.}. Then, the achievable throughput of $\mathrm{U}_{k,m}$ in time slot $n$ is given by\footnote{Notice that UAVs employ coordinated multi-point (CoMP) reception techniques \cite{xie2020common,liu2019comp} and as a result, the inter-cluster interference is removed.}
\begin{equation}  \label{sinr}
	R_{k,m} [n]=\tau_{k,m} [n] \delta_t \textrm{log}_2 \left(1+ \frac{p^{{ul}}_{k,m} [n] h_{k,m,m}[n]}{ {\sigma_{c,m}^2}} \right),
\end{equation}
where ${\sigma_{c,m}^2}$ is the power of the additive white Gaussian noise at the  communication receiver of $\mathrm{UAV}_{m}$. Thus, the average throughput of $\mathrm{U}_{k,m}$ over $N$ time slots is given by $R_{k,m}=\frac{1}{N} \sum^{N}_{n=1} R_{k,m} [n]$.	
\section{Problem Formulation And The Proposed Method}\label{sum}
	In this section, we cast the optimization problem in which we aim to jointly maximize the minimum radar SINR, i.e., $\min_m \widetilde{\mathrm{SINR}}_m$, and minimum throughput of communication users, i.e., $\min_{k,i} R_{k,i}$, in our ISWPC model by designing the radar/WPT waveforms $\widetilde{\mathbf{X}}=\lbrace \widetilde{\mathbf{x}}_m ,\hspace{2pt} \forall m \rbrace$, radar receive filters $\widetilde{\mathbf{W}}= \lbrace \widetilde{\mathbf{w}}_m ,\hspace{2pt} \forall m \rbrace$, time scheduling parameters $\mathbf{A}= \lbrace \tau_{k,m} [n],\hspace{2pt} \forall k,m,n \rbrace$, integration parameter $L$, uplink power of users $\mathbf{P}^{ul}= \lbrace  p^{ul}_{k,m} [n],\hspace{2pt} \forall k,m,n \rbrace$, and UAV trajectories $\mathbf{Q}= \lbrace \mathbf{q}_m [n],\hspace{2pt} \forall m,n \rbrace$. To find the Pareto-optimal solutions of the mentioned multi-objective problem, we adopt the scalarization technique \cite{boyd2004convex} using the Pareto weight $\mu \in [0 ,1]$ as follows
\begin{align}\label{maxmin1}
	&\hspace{-10pt} \max_{\widetilde{\mathbf{X}}, \widetilde{\mathbf{W}},{\mathbf{A}}, L, \mathbf{P}^{ul}, \mathbf{Q} }  ~ (1-\mu) \displaystyle \min_{\substack{{1 \leq k \leq  {K}}\\ {1 \leq i \leq  {M}}}} ~ \min_{  \Delta \mathbf{r}^T_{k,i}  \Delta \mathbf{r}_{k,i} \leq \bar{d}^2_{k,i} } R_{k,i} \\ \nonumber  & \hspace{55pt}+\mu \displaystyle \min_{1 \leq m \leq  {M}  } \hspace{1pt} \widetilde{\mathrm{SINR}}_m
	\\ \nonumber 
	&\hspace{-7pt} \mbox{s.t.}\\ \nonumber  & \scalemath{.97}{\textrm{C}_{1}:0 \leq \tau_{k,m} [n] \leq 1-L\widetilde{\tau}_0, \hspace{3pt}  \sum_{k=1}^{K} \tau_{k,m} [n] \leq 1- L\widetilde{\tau}_0, \hspace{2pt} \forall k,m,n,} \\ \nonumber & 
	\scalemath{.97}{\textrm{C}_{2}: {\parallel \mathbf{q}_m [n] - \mathbf{q}_m [n-1] \parallel}_2 \leq \delta_t  {v}_{\textrm{max}}, \hspace{2pt} z_{\textrm{min}} \leq z_m [n] \leq z_{\textrm{max}} , } \\ \nonumber & \hspace{14pt} \scalemath{.95}{\frac{1}{N} \sum_{n=1}^{N} z_m [n] \geq z^{\mathrm{tr}}_m, \hspace{2pt} \widetilde{\mathbf{q}}_m [n] \in \mathrm{CR}_m
	, \hspace{2pt} \mathbf{q}_m [0] = \mathbf{q}_m[N] ,\hspace{2pt} \forall m,n,}
	\\ \nonumber & 
	\textrm{C}_{3}: {\parallel \widetilde{\mathbf{q}}_m [n] - \mathbf{q}^{\mathrm{NFZ}}_{m,j} \parallel}^2_2 \geq \left(r^{\mathrm{NFZ}}_{m,j}\right)^2, \hspace{2pt} \forall m,n, \hspace{2pt} 1 \leq j \leq N_m^{\mathrm{NFZ}} ,
	\\ \nonumber & 
	\scalemath{.97}{\textrm{C}_{4}:L \in \lbrace 2^{lb}, 2^{lb+1}, ..., 2^{ub} \rbrace,~ \textrm{C}_{5}: \vert \widetilde{{x}}_m(i) \vert^2 = p_m^{dl} [n], \hspace{2pt} \forall m,i,n,}  \\ \nonumber &   
	\textrm{C}_{6}: \sum^{n}_{j=1}\tau_{k,m}[j] \delta_t  p^{{ul}}_{k,m} [j]  \leq   \min_{  \Delta \mathbf{r}^T_{k,m}  \Delta \mathbf{r}_{k,m} \leq \bar{d}^2_{k,m} } \sum^{n}_{j=1} E_{k,m} [j]  \\ \nonumber & \hspace{15pt} + E^0_{k,m}, \hspace{2pt} \forall k,m,n,
	%		\\ \nonumber \;\;&& \hspace{-30pt}
	%		\textrm{C}_{5}: R_k \geq R_{QoS},\hspace{2pt} \forall k,	
\end{align}
where $\bar{d}_{k,i}$ denotes the radius of the circular uncertainty region \cite{boshkovska2017robust}, ${v}_{\textrm{max}}$ is the maximum speed of the ISWPC UAVs, and $\mathrm{CR}_m$ indicates the $m$th cluster region. Note that the design variables $\widetilde{\mathbf{X}}$, $L$, and  $\mathbf{Q}$ are the joint parameters between sensing and communication tasks. The constraint $\frac{1}{N}\sum_{n=1}^{N} z_m [n] \geq z^{\mathrm{tr}}_m$ in $\textrm{C}_2$ is considered to bring the desired coverage for a given radar field-of-view (FOV) during the period of $T$ seconds. Precisely, $z^{\mathrm{tr}}_m$ can be determined numerically in such a way that the possible targets in $\mathrm{CR}_m$ are seen by the $\mathrm{UAV}_m$. The constraint $\textrm{C}_{3}$ introduces the cylindrical no-fly zones (NFZ)s\footnote{NFZs are considered due to security, privacy, or safety reasons \cite{valavanis2015handbook,li2018joint}. Note that the NFZs can also be modeled as a polygonal \cite{lee2020uav}.} in each cluster with coordinate center $\mathbf{q}^{\mathrm{NFZ}}_{m,j}$ and radius $r^{\mathrm{NFZ}}_{m,j}$, and $\textrm{C}_{5}$ represents the unimodularity constraint of the transmit sequence. 


It can be seen that the objective function and constraints $\textrm{C}_3-\textrm{C}_6$ are non-convex and so the problem. Therefore, in the following, we devise a method based on alternating optimization to deal with the non-convex design problem. Tackling subproblems corresponding to aforementioned alternating optimization is associated with novel tricks; e.g., by employing fractional programming, S-procedure, and MM to be discussed shortly.
\subsection{Maximization over $\widetilde{\mathbf{W}}$ for fixed $[\widetilde{\mathbf{X}}, \mathbf{A}, L, \mathbf{P}^{ul},\mathbf{Q}]$}\label{w}
We first consider the problem in \eqref{maxmin1} with respect to (w.r.t.) $\widetilde{\mathbf{W}}$ which is a unconstrained problem. Using the Cauchy-Schwartz inequality, we can write the following expression for $\widetilde{\mathrm{SINR}}_m$ in the objective function 
\begin{equation*} \scalemath{.93}{
	{\vert \widetilde{\mathbf{w}}_m^H \widetilde{\mathbf{x}}_m \vert}^2= {\vert \widetilde{\mathbf{w}}_m^H \widetilde{{\hm{\Xi}}}_m^{1/2} \widetilde{{\hm{\Xi}}}_m^{-1/2} \widetilde{\mathbf{x}}_m \vert}^2 \leq \left( \widetilde{\mathbf{w}}_m^H \widetilde{{\hm{\Xi}}}_m \widetilde{\mathbf{w}}_m \right) \left( \widetilde{\mathbf{x}}_m^H \widetilde{{\hm{\Xi}}}_m^{-1} \widetilde{\mathbf{x}}_m \right),}
\end{equation*}
where the equality holds for
\begin{equation} \label{jjk}
\widetilde{\mathbf{w}}_m={\widetilde{\hm{\Xi}}}_m^{-1} \widetilde{\mathbf{x}}_m,
\end{equation}
(by neglecting a multiplicative constant) which yields a closed-form solution for $\widetilde{\mathbf{w}}_m$. 
%Then, $\widetilde{\mathrm{SINR}}_m$ can be written as
%\begin{equation}
%\beta_m\vert \alpha_m \vert^2  \widetilde{\mathbf{x}}_m^H {\hm{\Xi}}_m^{-1} \widetilde{\mathbf{x}}_m.
%\end{equation}
%Thus, by introducing an auxiliary variable $\theta_a$, the problem in \eqref{maxmin1} for fixed $[{\mathbf{X}}, L, \mathbf{P}^{ul}, \mathbf{Q}]$ can be written in its epigraphic form as follows
%\begin{align}\label{maxmin2}
%	&~~\max_{{\mathbf{A}}, \theta_a } ~ \theta_a\\ \nonumber 
%	&\mbox{s.t.}~ \textrm{C}_{1},~\textrm{C}_{5}, \\ \nonumber 
%	&\scalemath{.93}{ \textrm{C}_{6}: \mu L \vert \alpha_T \vert^2  \widetilde{\mathbf{x}}_m^H {\hm{\Xi}}_m^{-1} \widetilde{\mathbf{x}}_m + (1-\mu) \hspace{-8pt}\min_{  \Delta \mathbf{w}^T_{k,i}  \Delta \mathbf{w}_{k,i} \leq \bar{d}^2_{k,i} } \hspace{-8pt} R_{k,i} \geq \theta_a,  \forall m,k,i.}	
%\end{align}
\subsection{Maximization over $[\widetilde{\mathbf{X}}, \mathbf{A}, \mathbf{P}^{ul},\mathbf{Q}]$ for fixed $[\widetilde{\mathbf{W}}, L]$}\label{joint}
Then, we consider the problem in \eqref{maxmin1} w.r.t. $[\widetilde{\mathbf{X}}, \mathbf{A}, \mathbf{P}^{ul},\mathbf{Q}]$ leading to the associated subproblem. The objective function and the constraints $\textrm{C}_3$, $\textrm{C}_5$, and $\textrm{C}_6$ are non-convex. Fig.~\ref{hh11} summarizes the procedure for dealing with this subproblem. First, we aim to deal with the sensing SINR in the objective function which is a non-convex fractional term. We rewrite the second term in the objective function as $f_1(\widetilde{\mathbf{x}}_m)= g_1 (\widetilde{\mathbf{x}}_m)/g_2 (\widetilde{\mathbf{x}}_m)$ for all $m$ where 
$
g_1(\widetilde{\mathbf{x}}_m)= \widetilde{\mathbf{x}}_m^H \hm{{\Gamma}}_m \widetilde{\mathbf{x}}_m
$
and 
$
g_2(\widetilde{\mathbf{x}}_m)=  \widetilde{\mathbf{x}}_m^H \widetilde{\hm{{\Gamma}}}_m \widetilde{\mathbf{x}}_m  +\gamma_m
$
with 
\begin{equation}
	\hm{\Gamma}_m=\mu \beta_m \vert \alpha_m \vert^2 \widetilde{\mathbf{w}}_m \widetilde{\mathbf{w}}_m^H,
\end{equation}
%\begin{equation}
%\widetilde{\hm{\Gamma}}_m=\vert \alpha_T \vert^2 \widetilde{\mathbf{w}}_m \widetilde{\mathbf{w}}_m^H,
%\end{equation}
\begin{equation}
	\widetilde{\hm{\Gamma}}_m= \beta_m \sum_{k=-\widetilde{N}+1 ,k\neq 0}^{\widetilde{N}-1} \widetilde{{\sigma}}_{m,k}^2 \mathbf{J}^H_k \widetilde{\mathbf{w}}_m \widetilde{\mathbf{w}}_m^H \mathbf{J}_k,
\end{equation}
and 
$
\gamma_m=\sum_{l=1}^{L} {\sigma_{m,l}^2} \hspace{2pt} \widetilde{\mathbf{w}}_m^H \widetilde{\mathbf{w}}_m.
$

\begin{prop} \label{cor1}
Let the objective function of the subproblem in \ref{joint} be as follows
\begin{equation}\label{mmb}
	q(\mathbf{\widetilde{X}})= a + \displaystyle \mu \min_{1\leq m \leq M} f_1(\mathbf{\widetilde{x}}_m),
\end{equation}
where
\begin{equation}
	a= (1-\mu) \displaystyle \min_{\substack{{1 \leq k \leq  {K}}\\ {1 \leq i \leq  {M}}}} ~ \min_{  \Delta \mathbf{r}^T_{k,i}  \Delta \mathbf{r}_{k,i} \leq \bar{d}^2_{k,i} } R_{k,i},
\end{equation}
is a constant term w.r.t. $\mathbf{\widetilde{X}}$.
By exploiting the idea of fractional programming \cite{dinkelbach1967nonlinear} and assuming $g_2(\widetilde{\mathbf{x}}_m) >0$ (to ensure that $f_1(\widetilde{\mathbf{x}}_m)$ has a finite value), it is proved that the objective function in \eqref{mmb} can be dealt with via iterative maximization of the below function w.r.t. $\mathbf{\widetilde{X}}$:
\begin{equation}
	q(\mathbf{\widetilde{X}})= a + \displaystyle \mu \min_{1\leq m \leq M} f_2(\mathbf{\widetilde{x}}_m),
\end{equation}
where
\begin{equation}
 f_2(\widetilde{\mathbf{x}}_m)= g_1(\widetilde{\mathbf{x}}_m) - f_1(\widetilde{\mathbf{x}}_m^{(\kappa-1)}) g_2(\widetilde{\mathbf{x}}_m). 
\end{equation}
	\end{prop}
	\begin{proof}
		Please refer to Appendix~\ref{app11}.
	\end{proof}
	Then, considering ${\parallel \widetilde{\mathbf{x}}_m \parallel }^2_2 =\widetilde{N} p_m^{dl}[n]$ from the unimodularity constraint $\textrm{C}_5$ in \eqref{maxmin1}, one can write
\begin{equation} \label{nnm}
	f_2(\widetilde{\mathbf{x}}_m)=\widetilde{\mathbf{x}}_m^H \hm{\Upsilon}_{m}^{(\kappa-1)}\widetilde{\mathbf{x}}_m,
\end{equation}
where
\begin{equation*}
	\hm{\Upsilon}_{m}^{(\kappa-1)}=
	\hm{{\Gamma}}_m- f_1(\widetilde{\mathbf{x}}_m^{(\kappa-1)}) \left (\widetilde{\hm{\Gamma}}_m+\frac{\gamma_m}{ \widetilde{N}p_m^{dl}[n]}\mathbf{I}_{\widetilde{N}} \right).
\end{equation*}
% Figure environment removed
Now, the second term in the objective function can be written as 
\begin{equation} \label{mm1}
\min_{1 \leq m \leq  {M}  } \hspace{1pt} \widetilde{\mathbf{x}}_m^H \widetilde{\hm{\Upsilon}}_{m}^{(\kappa-1)}\widetilde{\mathbf{x}}_m,
\end{equation}	
where ${\widetilde{\hm{\Upsilon}}}_{m}=\lambda_{m} \mathbf{I}_{\widetilde{N}} + \hm{\Upsilon}_{m}$ and $\lambda_{m}$ must be selected in such a way that ${\widetilde{\hm{\Upsilon}}}_{m}$ be a positive definite matrix \cite{soltanalian2014designing}.
Next, by defining a rank-1 matrix ${\mathbf{X}}_m=\widetilde{\mathbf{x}}_m \widetilde{\mathbf{x}}_m^H$, the quadratic term in \eqref{mm1} can be written as a linear term $\textrm{tr}\big( {\widetilde{\hm{\Upsilon}}}_{m}^{(\kappa-1)} {\mathbf{X}}_m \big)$ w.r.t. ${\mathbf{X}}_m$, and the constraint $\textrm{C}_5$ can be expressed as follows
\begin{equation} \label{ctil}
\widetilde{\textrm{C}}_{5}:\left[ {\mathbf{X}}_m \right]_{i,i}=p_m^{dl}[n],~ \textrm{rank} \left( {\mathbf{X}}_m \right)=1,~ {\mathbf{X}}_m\succeq \mathbf{0}, \hspace{2pt} \forall m,n,i.
\end{equation}
By using \eqref{ctil} and the linearized version of \eqref{mm1} as well as introducing an auxiliary variable $\theta$, the problem in \eqref{maxmin1} for fixed $[\widetilde{\mathbf{W}},L]$ can be reformulated as
\begin{align}\label{maxmin3}
&\max_{\lbrace {\mathbf{X}}_m\rbrace_{m=1}^{M},\mathbf{A}, \mathbf{P}^{ul}, \mathbf{Q}, \theta } ~~ \theta\\ \nonumber 
	&\mbox{s.t.} ~~\textrm{C}_{1}-\textrm{C}_{3}, \widetilde{\textrm{C}}_{5}, \textrm{C}_{6},
	\\ \nonumber &\scalemath{.96}{   \textrm{C}_{7}:   (1-\mu)  \min_{  \Delta \mathbf{r}^T_{k,i}  \Delta \mathbf{r}_{k,i} \leq \bar{d}^2_{k,i} } \hspace{-12pt} R_{k,i} +  \textrm{tr}\left( {\widetilde{\hm{\Upsilon}}}_{m}^{(\kappa-1)} {\mathbf{X}}_m \right) \geq \theta , \hspace{2pt} \forall k,i,m.}
\end{align}
%\begin{align}\label{maxmin3}
%	&\max_{\lbrace\widetilde{\mathbf{X}}_m\rbrace_{m=1}^{M},\mathbf{A}, \mathbf{P}^{ul}, \mathbf{Q}, \theta } ~ \theta\\ \nonumber 
%	&\mbox{s.t.} \\ \nonumber &\textrm{C}_{1},\textrm{C}_{2},\textrm{C}_{5}, \\ \nonumber &\textrm{C}_{4}:\textrm{tr} \left( \widetilde{\mathbf{X}} \right)=\widetilde{N}p^{dl}[n],~ \textrm{rank} \left( \widetilde{\mathbf{X}} \right)=1,~ \widetilde{\mathbf{X}}\succeq \mathbf{0}, \hspace{2pt} \forall n,  \\ \nonumber & \textrm{C}_{5}: \sum^{n}_{i=1} \tau_k[i] \delta_t  p^{{ul}}_k [i]  \leq E^0_k + \tau_0 \delta_t \epsilon_k \rho_0 \sum^{n}_{i=1}  \frac{p^{dl}[i]}{\phi_k [i] }, \hspace{2pt} \forall k,n,
%	\\ \nonumber &  \textrm{C}_{6}:  \textrm{tr} \left( {\widetilde{\hm{\Upsilon}}}^{(\kappa-1)} \widetilde{\mathbf{X}} \right) +(1-\mu) \frac{\delta_t}{N} \sum_{n=1}^{N}\tau_k[n]  \\ \nonumber & \hspace{20pt} \times \textrm{log}_2 \left(1+\frac{\rho_0 p^{{ul}}_k [n] }{{\sigma_c^2} [n] \phi_k [n] } \right) \geq \theta_b, \hspace{2pt} \forall k,
%	\\ \nonumber &  \textrm{C}_{7}:	{\parallel \mathbf{q} [n] - \mathbf{w}_k  \parallel }^2_2 +H^2 \leq \phi_k [n], \hspace{2pt} \forall k,n,
%\end{align}

Now, to proceed further, we focus on the first term in constraint $\textrm{C}_7$ and the right-hand side of the constraint $\textrm{C}_6$ which are associated with the circular uncertainty regions. 
By introducing $\phi_{k,m,i}[n]$ and applying a change of variable ${\psi}_{k,m}[n]=\tau_{k,m} [n] p^{ul}_{k,m}[n],\hspace{3pt} {\psi}_{k,m}[n] \geq 0$, the problem in \eqref{maxmin3} can be recast as  
\begin{align}\label{maxmin4}
	&\max_{\lbrace {\mathbf{X}}_m\rbrace_{m=1}^{M},\mathbf{A}, \mathbf{Q}, \hm{\Psi}, \hm{\Phi}, \theta } ~~ \theta\\ \nonumber 
	&\mbox{s.t.}~~ \textrm{C}_{1}-\textrm{C}_{3},\widetilde{\textrm{C}}_{5}, \\ \nonumber &	\widetilde{\textrm{C}}_{6}: \sum^{n}_{j=1} \delta_t \psi_{k,m} [j]  \leq   \tau_0 \delta_t \epsilon_{k,m} \rho_0 \sum^{n}_{j=1} \sum^{M}_{i=1}  \frac{p_i^{dl}[j]}{\phi_{k,m,i} [j] }  \\ \nonumber & \hspace{17pt}+ E^0_{k,m}, \hspace{3pt} \forall k,m,n, 
	\\ \nonumber &   \widetilde{\textrm{C}}_{7}:   (1-\mu) \frac{\delta_t}{N} \sum_{n=1}^{N} \tau_k [n]  \textrm{log}_2 \left(1+ \frac{\rho_0 \psi_{k,i} [n] }{ {\sigma_{c,i}^2} \tau_k [n] \phi_{k,i,i}[n] } \right)  \\ \nonumber & \hspace{18pt}+  \textrm{tr}\left( {\widetilde{\hm{\Upsilon}}}_{m}^{(\kappa-1)} {\mathbf{X}}_m \right) \geq \theta , \hspace{2pt} \psi_{k,i} [n] \geq 0, \hspace{2pt} \forall k,i,m,
		\\ \nonumber &  \textrm{C}_{8}:	{\parallel \widetilde{\mathbf{q}}_i [n] - ( \bar{\mathbf{r}}_{k,m} + \Delta \mathbf{r}_{k,m} ) \parallel }^2_2 +z_m^2 [n] \leq \phi_{k,m,i} [n], \\ \nonumber & \hspace{20pt} \Delta \mathbf{r}^T_{k,m}  \Delta \mathbf{r}_{k,m} \leq \bar{d}^2_{k,m}, \hspace{2pt} \forall k,m,i,n,
\end{align}
where 
\begin{equation}
\hm{\Psi}=\lbrace \psi_{k,m}[n],\forall k,m,n \rbrace,
\end{equation}
and
\begin{equation}
	\hm{\Phi}=\lbrace \phi_{k,m,i}[n],\forall k,m,i,n \rbrace.
\end{equation}

\begin{prop}\label{cor2}
The optimal solution of the problem in \eqref{maxmin3} can be obtained from the solution to the problem in \eqref{maxmin4}.
\end{prop}
\begin{proof}
Please refer to Appendix~\ref{app12}.
\end{proof}


It can be seen that the constraints $\textrm{C}_3$ and $\widetilde{\textrm{C}}_5$-$\widetilde{\textrm{C}}_7$ are still non-convex and $\textrm{C}_8$ has an infinite number of constraints due to the continuity of the corresponding user location uncertainty sets. The non-convexity of $\widetilde{\textrm{C}}_5$ originates from the rank-1 constraint. By adopting the semi-definite relaxation, we can drop the rank-1 constraint and proceed to solve the problem.  



To proceed further, we can deal with the non-convexity of $\textrm{C}_3$ and $\widetilde{\textrm{C}}_6$ in the following. The left-hand side of $\textrm{C}_3$ and the first term in the right-hand side of $\widetilde{\textrm{C}}_6$ can be minorized using their supporting hyperplane (see \cite{rezaei2019throughput} for more details). Therefore, the left-hand side of $\textrm{C}_3$ as well as the first term in the right-hand side of $\widetilde{\textrm{C}}_6$ can be respectively obtained at the $\kappa$th iteration of the following expressions
\begin{align} \label{ttt1}
	{\parallel \widetilde{\mathbf{q}}^{(\kappa-1)}_m [n] - \mathbf{q}^{\mathrm{NFZ}}_{m,j} \parallel}^2_2 + 2 & \left( \widetilde{\mathbf{q}}^{(\kappa-1)}_m [n] - \mathbf{q}^{\mathrm{NFZ}}_{m,j} \right)^T \\ \nonumber & \times \left( \widetilde{\mathbf{q}}_m [n] - \widetilde{\mathbf{q}}^{(\kappa-1)}_m [n] \right),
\end{align}
\begin{align} \label{taumami}	
	\tau_0 \delta_t \epsilon_{k,m} \rho_0 \sum^{n}_{j=1}\sum^{M}_{i=1} \bigg \lbrace & \frac{p_i^{dl}[j]}{\phi^{(\kappa-1)}_{k,m,i} [j] } \\ \nonumber & - \frac{p_i^{dl}[j]\left( \phi_{k,m,i} [j] -\phi^{(\kappa-1)}_{k,m,i} [j] \right)}{\left(\phi^{(\kappa-1)}_{k,m,i} [j]\right)^2 }   \bigg \rbrace.   
\end{align}

Next, we consider the constraint $\textrm{C}_8$. Let us introduce a lemma which can be used to transform $\textrm{C}_8$ into a finite number of linear matrix inequalities (LMI)s.
\begin{lemma}[S-procedure]
Let a function $h_m (\mathbf{x}), m \in \lbrace 1,2 \rbrace, \mathbf{x} \in \complexC^{N}$, be defined as
\begin{equation}
	h_m (\mathbf{x})= \mathbf{x}^H   \mathbf{B}_m \mathbf{x} + 2 \Re \lbrace \mathbf{b}^{H}_m  \mathbf{x} \rbrace + b_m,
\end{equation}
where, $\mathbf{B}_m \in \mathbb{H}^{N \times N}, \mathbf{b}_m  \in \complexC^{N}$ and $b_m \in \realR$. Then, the implication $h_1 (\mathbf{x}) \leq 0  \Rightarrow h_2 (\mathbf{x}) \leq 0$ holds if
and only if there exists an $\omega \geq 0$ such that 
\begin{equation}
	\omega
	\begin{bmatrix}
		\mathbf{B}_1 & \mathbf{b}_1  \\  \mathbf{b}^{H}_1 &  {b}_1
	\end{bmatrix}
	- 
	\begin{bmatrix}
		\mathbf{B}_2 & \mathbf{b}_2  \\  \mathbf{b}^{H}_2 &  {b}_2
	\end{bmatrix} \succeq \mathbf{0},
\end{equation}
provided that there exists a point $\widehat{\mathbf{x}}$ such that $h_m(\widehat{\mathbf{x}}) < 0$. 
\end{lemma}
\begin{proof}Please see \cite{boyd2004convex12}.\end{proof}
Then, we can rewrite constraint ${\textrm{C}}_{8}$ as 
\begin{align*}
	&\textrm{C}_{8}:  \Delta \mathbf{r}^T_{k,m}  \Delta \mathbf{r}_{k,m}  +2 \Re \lbrace   
	(  \bar{\mathbf{r}}_{k,m} - \widetilde{\mathbf{q}}_i [n])^T
	\Delta \mathbf{r}_{k,m} \rbrace \\ &+  (\widetilde{\mathbf{q}}_i [n] - \bar{\mathbf{r}}_{k,m} )^T (\widetilde{\mathbf{q}}_i [n] - \bar{\mathbf{r}}_{k,m} ) + z_m^2 [n]-  {\phi}_{k,m,i} [n] \leq 0.
\end{align*}
Then, using Lemma~1 and considering user location uncertianty region $\Delta \mathbf{r}^T_{k,m}  \Delta \mathbf{r}_{k,m} \leq \bar{d}^2_{k,m}$, we can equivalently rewrite the constraint ${\textrm{C}}_8$ as:
\begin{align} \label{mj10}
&	\widetilde{\textrm{C}}_8 : \mathbf{S} (\mathbf{Q},\hm{\Phi},\omega_{k,m,i}[n] )=\\ \nonumber & 
	\begin{bmatrix}
	\scalemath{.98}{	(\omega_{k,m,i}[n] -1) \mathbf{I}_{2}} & \scalemath{.98}{\widetilde{\mathbf{q}}_i [n] -\bar{\mathbf{r}}_{k,m}}  \\ 
		\scalemath{.98}{( \widetilde{\mathbf{q}}_i [n] -\bar{\mathbf{r}}_{k,m} )^T} & \substack{ \scalemath{.98}{- \omega_{k,m,i}[n] \bar{d}^2_{k,m}  + {\phi}_{k,m,i} [n]}
			\\   \scalemath{.98}{-   
				{\parallel \widetilde{\mathbf{q}}_i [n]- \bar{\mathbf{r}}_{k,m}\parallel }^2_2-z_m^2 [n]}}   
	\end{bmatrix}
	\succeq \mathbf{0},
\end{align}
with the variable $\omega_{k,m,i}[n] \geq0$. Note that the constraint $\widetilde{\textrm{C}}_8$ is still non-convex due to the quadratic term ${\parallel \widetilde{\mathbf{q}}_i [n]- \bar{\mathbf{r}}_{k,m}\parallel }^2_2$. For handling this, in light of MM, we construct a global underestimator for the mentioned quadratic term to minorize it and rewrite the constraint $\widetilde{\textrm{C}}_8$ at the $\kappa$th iteration as
\begin{align} \label{mj1}
	&	\bar{\textrm{C}}_8 : \mathbf{S} (\mathbf{Q},\hm{\Phi},\omega_{k,m,i}[n] )=\\ \nonumber & 
	\begin{bmatrix}
		\scalemath{.98}{	(\omega_{k,m,i}[n] -1) \mathbf{I}_{2}} & \scalemath{.98}{\widetilde{\mathbf{q}}_i [n] -\bar{\mathbf{r}}_{k,m}}  \\ 
		\scalemath{.98}{( \widetilde{\mathbf{q}}_i [n] -\bar{\mathbf{r}}_{k,m} )^T} & \substack{ \scalemath{.98}{- \omega_{k,m,i}[n] \bar{d}^2_{k,m}  + {\phi}_{k,m,i} [n]}
			\\   \scalemath{.98}{- \chi^{(\kappa)}_{k,m,i}[n]  
				-z_m^2 [n]}}   
	\end{bmatrix}
	\succeq \mathbf{0},
\end{align}
where 
\begin{align}
&\chi^{(\kappa)}_{k,m,i}[n] =	{\parallel \widetilde{\mathbf{q}}^{(\kappa-1)}_i [n] - \bar{\mathbf{r}}_{k,m} \parallel}^2_2 \\ \nonumber & + 2  \left( \widetilde{\mathbf{q}}^{(\kappa-1)}_i [n] - \bar{\mathbf{r}}_{k,m} \right)^T  \left( \widetilde{\mathbf{q}}_i [n] - \widetilde{\mathbf{q}}^{(\kappa-1)}_i [n] \right), \hspace{3pt}\forall k,m,i,n.
\end{align}
Now, based on the expressions in \eqref{ttt1}, \eqref{taumami}, and \eqref{mj1}, the problem in \eqref{maxmin4} can be restated as the following
\begin{align}\label{maxmin5}
	&\max_{\lbrace {\mathbf{X}}_m\rbrace_{m=1}^{M},\mathbf{A}, \mathbf{Q}, \hm{\Psi}, \hm{\Phi}, \hm{\Omega}, \theta } ~~ \theta\\ \nonumber 
	&\mbox{s.t.}~~ \textrm{C}_{1},~\textrm{C}_{2},~\widetilde{\textrm{C}}_{7},~\widetilde{\textrm{C}}_{3}: \eqref{ttt1} \geq \left(r^{\mathrm{NFZ}}_{m,j}\right)^2, \hspace{2pt} \forall m,n,j,  \\ \nonumber&
	\bar{\textrm{C}}_{5}:\left[ {\mathbf{X}}_m \right]_{i,i}=p_m^{dl}[n],~ {\mathbf{X}}_m\succeq \mathbf{0}, \hspace{2pt} \forall m,n,i,
	\\ \nonumber
	&\scalemath{.96}{	\bar{\textrm{C}}_{6}: \sum^{n}_{j=1} \delta_t \psi_{k,m} [j]  \leq  \eqref{taumami}+ E^0_{k,m}, ~ \forall k,m,n,}  
	\\ \nonumber &  \scalemath{.97}{ \bar{\textrm{C}}_{8}:	\mathbf{S} (\mathbf{Q},\hm{\Phi},\hm{\Omega} ) \succeq \mathbf{0},~\omega_{k,m,i}[n] \geq0, ~\forall k,m,i,n,}
\end{align}
where 
\begin{eqnarray}
	\hm{\Omega}=\lbrace \omega_{k,m,i}[n],\forall k,m,i,n \rbrace.
\end{eqnarray}
The logarithmic function of the first term in left-hand side of $\widetilde{\textrm{C}}_7$ is a non-concave term and so $\widetilde{\textrm{C}}_7$ is non-convex. By introducing the auxiliary variables $\pi_{k,i}[n]$, the problem in \eqref{maxmin5} can be equivalently rewritten as
\begin{align}\label{maxmin9}
	&\max_{\lbrace {\mathbf{X}}_m\rbrace_{m=1}^{M},\mathbf{A}, \mathbf{Q}, \hm{\Psi}, \hm{\Phi}, \hm{\Omega},\hm{\Pi}, \theta } ~~ \theta\\ \nonumber 
	&\mbox{s.t.}~~ \textrm{C}_{1},~\textrm{C}_{2},~\widetilde{\textrm{C}}_{3},~
	\bar{\textrm{C}}_{5},~	\bar{\textrm{C}}_{6},~\bar{\textrm{C}}_{8},
	\\ \nonumber &   \bar{\textrm{C}}_{7}:   (1-\mu) \frac{\delta_t}{N} \sum_{n=1}^{N} \tau_k [n]  \textrm{log}_2 \left(1+ \frac{\rho_0 \pi_{k,i}[n]  }{ {\sigma_{c,i}^2}  \tau_k [n]  } \right)  \\ \nonumber & \hspace{18pt}+  \textrm{tr}\left( {\widetilde{\hm{\Upsilon}}}_{m}^{(\kappa-1)} {\mathbf{X}}_m \right) \geq \theta , \hspace{2pt} \psi_{k,i} [n] \geq 0, \hspace{2pt} \forall k,i,m,
	\\ \nonumber &
	\textrm{C}_{9}:	\pi_{k,i}[n] \leq \frac{\psi_{k,i}[n]}{\phi_{k,i,i}[n]},~\forall k,i,n,
\end{align}
where 
\begin{eqnarray}
\hm{\Pi}=\lbrace \pi_{k,i}[n],\forall k,i,n \rbrace.
\end{eqnarray}
\begin{prop}\label{cor3}
The optimal solution of the problem in \eqref{maxmin5} can be obtained by solving the problem in \eqref{maxmin9}.
\end{prop}
\begin{proof}
Please refer to Appendix~\ref{app12}.
\end{proof}
Then, the non-convexity of $\textrm{C}_9$ can be dealt with by  minorizing its right-hand side using the following minorizer:
\begin{align} \label{keyjoint}
&\frac{\psi^{(\kappa-1)}_{k,i}[n]}{\phi^{(\kappa-1)}_{k,i,i}[n]} +\frac{1}{\phi^{(\kappa-1)}_{k,i,i}[n]} \left( \psi_{k,i}[n]-\psi^{(\kappa-1)}_{k,i}[n] \right) \\ \nonumber &- \frac{\psi^{(\kappa-1)}_{k,i}[n]}{\left(\phi^{(\kappa-1)}_{k,i,i}[n]\right)^2} \left( \phi_{k,i,i}[n]-\phi^{(\kappa-1)}_{k,i,i}[n] \right),
\end{align}
and the problem in \eqref{maxmin9} can be reformulated as
\begin{align}\label{maxmin6}
	&\max_{\lbrace {\mathbf{X}}_m\rbrace_{m=1}^{M},\mathbf{A}, \mathbf{Q}, \hm{\Psi}, \hm{\Phi}, \hm{\Omega},\hm{\Pi}, \theta } ~~ \theta\\ \nonumber 
	&\mbox{s.t.}~~ \textrm{C}_{1},~\textrm{C}_{2},~\widetilde{\textrm{C}}_{3},~\bar{\textrm{C}}_{5}-\bar{\textrm{C}}_{8}, ~
	\widetilde{\textrm{C}}_{9}:	\pi_{k,i}[n] \leq \eqref{keyjoint},~\forall k,i,n.
\end{align}
Since the logarithmic term in $\bar{\textrm{C}}_7$ is jointly concave w.r.t. $\tau_k [n]$ and $\pi_{k,i} [n]$, the constraint $\bar{\textrm{C}}_7$ and therefore, the problem in \eqref{maxmin6} are convex and can be solved efficiently by e.g., interior point methods. Note that $\mathbf{P}^{ul}$ can be synthesized after convergence of \eqref{maxmin6} as follows:
\begin{equation} \label{key2}
p_{k,m}^{ul}[n]=
\begin{cases}
\frac{\psi_{k,m}[n]}{\tau_{k,m} [n]},~  \tau_{k,m} [n] \neq 0,
\\
0, \hspace{29pt}  \tau_{k,m} [n]=0.
\end{cases}
\end{equation}
\subsection{Maximization over $L$ for fixed $[\widetilde{\mathbf{X}},\widetilde{\mathbf{W}}, \mathbf{A}, \mathbf{P}^{ul},\mathbf{Q}]$}\label{L}
As a final step, the problem in \eqref{maxmin1} w.r.t. the scalar $L$ can be solved via one-dimensional search over its finite possible values in $\textrm{C}_4$.
\begin{algorithm}[t] \label{tt}
	\caption{The Proposed Method for Joint Maximization of Minimum Radar SINR and Minimum Communication Throughput in a Multi-UAV ISWPC System}
	\begin{algorithmic}[tb]
		\STATE{\!\!\!\!\!\!\!\!\!\!\!\!\!  Main-0:}  Initialize ${\mathbf{X}}^{(i)}$, $L^{(i)}$, and set $i\leftarrow0$.
		\REPEAT
		\STATE{A:} Compute $\widetilde{\mathbf{w}}^{(i)}_m,~\forall m$ via the closed-form solution in \eqref{jjk}.
		\STATE{B-0:} Initialize $\widetilde{\mathbf{W}}^{(\kappa)}$, $L^{(\kappa)}$, $\hm{\Phi}^{(\kappa)}$, $\hm{\Psi}^{(\kappa)}$, and set $\kappa \leftarrow 0$.
		\REPEAT
		\STATE{B-1:} Solve the convex problem in \eqref{maxmin6}.
		\STATE{B-2:} Update $\kappa \leftarrow \kappa+1$.
		\UNTIL convergence
		\STATE{B-3:} Synthesize $\mathbf{P}^{ul}$ via \eqref{key2}.
		\STATE{C:} Solve the problem in \eqref{maxmin1} w.r.t. the scalar $L$ via one-dimensional search over its finite possible values in $\textrm{C}_4$.
		\STATE{Main-1:} Update $i \leftarrow i+1$.
		\UNTIL convergence
		\STATE{\!\!\!\!\!\!\!\!\!\!\!\!\!  Main-2:} Synthesize $\widetilde{\mathbf{x}}_m$ form ${\mathbf{X}}_m$.
	\end{algorithmic}
\end{algorithm}
\subsection{Waveform Synthesis, Convergence, and Complexity Analysis}
Algorithm~1 summarizes the steps of the proposed method for jointly maximizing the minimum radar SINR and minimum communication throughput in a multi-UAV enabled ISWPC system. The proposed method consists of
outer iterations which are denoted by superscript $i$. At each outer iteration, we have 3 steps associated with the subproblems in \ref{w}, \ref{joint} (which is denoted by superscript $\kappa$), and \ref{L}. At the end of the algorithm, we may synthesize the waveform $\mathbf{\widetilde{x}}_m$ from matrix $\mathbf{{X}}_m$ using e.g., the rank-1 approximation methods based on randomization techniques (see \cite{de2011design} for details). 


Note that to ensure convergence to a stationary point, the sequence of objective values of the problem in \eqref{maxmin1} must be ascending in each subproblem. For the subproblems in \ref{w} and \ref{L}, the global maximum is obtained. Also, applying the proposed fractional programming and MM techniques to the design problem in \ref{joint} increases the associated objective function and, under mild conditions, provides stationary points of the problem. Therefore, due to boundedness of the objective function in \eqref{maxmin1}, the sequence of objective values in \eqref{maxmin1} obtained by the proposed method converges.


Next, the computational complexity of the proposed method is considered. For the subproblem in Subsection~\ref{w}, the closed-form expression in \eqref{jjk} must be calculated which needs matrix multiplication and inversion leading to the complexity of\footnote{This can be decreased to $\mathcal{O}(\widetilde{N}^{2.373})$ by using the optimized algorithms (see e.g., \cite{davie2013improved} for details).} $\mathcal{O}(\widetilde{N}^3)$ for  $\mathrm{UAV}_m$. 
At each inner iterations of the subproblem in \ref{joint}, the dominant computational burden is associated with the constraints $\bar{\textrm{C}}_5$ and $\bar{\textrm{C}}_8$ due to adopting the semi-definite relaxation. Hence, considering the problem in \eqref{maxmin6}, the computational complexity is $\mathcal{O}(\sqrt{n}\textrm{log}(1/\epsilon)(mn^3 +m^2 n^2 +m^3))$ where $\epsilon > 0$ indicates the solution accuracy, $m=M(1+MK)$ is the number of semi-definite relaxation-based constraints, and $n=\widetilde{N}$ is associated with the size of the related positive semi-definite matrix \cite[Theorem~3.12]{bomze2010interior}. Finally, the one-dimensional search in Subsection~\ref{L} can be performed via the complexity of $\mathcal{O}(M\widetilde{N}^2)$ which comes from the objective function calculations.
 % Figure environment removed
\begin{rema}
Note that the value of $\widetilde{{\mathrm{SINR}}}_m$ is greater than $R_{k,i}$ in the objective function of $\eqref{maxmin1}$ for a typical numerical setup (see Section~\ref{num}). Therefore, to preserve the controlling role of the Pareto weight $\mu$, we modify the objective function of \eqref{maxmin1} as follows
	\begin{align}\label{keyr}
	\scalemath{.96}{(1-\mu) \displaystyle \min_{\substack{{1 \leq k \leq  {K}}\\ {1 \leq i \leq  {M}}}} ~ \min_{  \Delta \mathbf{r}^T_{k,i}  \Delta \mathbf{r}_{k,i} \leq \bar{d}^2_{k,i} } R_{k,i}  + \mu \hspace{2pt}\mu_0 \displaystyle \min_{1 \leq m \leq  {M}  } \hspace{1pt} \widetilde{\mathrm{SINR}}_m,}
	\end{align}
where $\mu_0 \in (0,1]$	is a constant parameter which can be determined under the numerical supervision, without losing the optimality of the solution to the problem.
\end{rema}
\begin{rema}
It is worth pointing out that the proposed algorithm can be modified to address the sum throughput/sensing SINR maximization problem. The interested reader may follow the steps in Appendix~\ref{app1} for the sum utility problem.
\end{rema}
 	\section{Numerical Examples} \label{num}	
In this section, we evaluate the effectiveness of the proposed method by numerical examples. The convex problem associated with the devised method is solved by CVX \cite{cvx}. We consider $\mathrm{\widetilde{SIR}}_{m,k}=\frac{\vert \alpha_{m} \vert^2}{\widetilde{\sigma}^2_{m,k}}=-10$ dB, $\forall m,k$, and $\mathrm{\widetilde{SNR}}_{m,l}=\frac{\vert \alpha_{m} \vert^2}{{\sigma}^2_{m,l}}=-10$ dB, $\forall m,l$
%, and $\omega_d=\pi/4$ 
for radar receiver; $\widetilde{N}=350$, $\widetilde{\tau}_0= 700$ $\mu\hspace{1pt}$second, $lb=5$, $ub=10$, and $p_m^{dl}[n]=37$ dBm, $\forall m,n$ \cite{nguyen2022ris} for radar/WPT waveform; $\sigma_{c,m}^2=-134$ dBm, $\forall m$ \cite{nguyen2022ris} for communication receiver; $\nu_m=0.05$ radians (for implementing the FFT-based Doppler processing filter), $\forall m$  \cite{li2022joint}, $\zeta=30$ degrees, $v_{\textrm{max}}=20$ m/s \cite{li2022joint}, and $\delta_t=1$ second \cite{wei2022safeguarding} as UAV flight parameters; $\rho_0=-30$ dB \cite{li2022joint} for channel power gain; $\epsilon_{m,k}=0.5, \hspace{1pt} \forall m,k$ \cite{nguyen2022ris}, and $E^0_{m,k}=1$ mJ, $\forall m,k$ for energy harvesting circuit; and $\mathrm{CR}_m = \lbrace \mathrm{CR}^x_m= 300\hspace{1pt}\textrm{m} \times \mathrm{CR}^y_m=300\hspace{1pt}\textrm{m} \rbrace,$ $\forall m$ \cite{wei2022safeguarding} for cluster regions. Also, we assume $\mu=0.5$ unless otherwise specified. Moreover,
we define the normalized radius of user location uncertainty as
\begin{equation}
\widetilde{r}_{k,m}=\frac{\bar{d}_{k,m}}{\frac{\textrm{min}\hspace{2pt} (\mathrm{CR}^x_m, \mathrm{CR}^y_m)}{2}},~\forall k,m.
\end{equation}
\subsection{2D Flight scenarios} 
% Figure environment removed
% Figure environment removed
First, we consider a 2D flight setup, i.e., $z_{\textrm{max}}=z_{\textrm{min}}=z^{\mathrm{tr}}_m= 100$ m, $\forall m$ \cite{li2022joint}, with $K=4$ for ground users which are located in $M=4$ clusters. Moreover, we set $N_{m}^{\mathrm{NFZ}}=0$, $\forall m$, and $\widetilde{r}_{k,m}=0.04$, $\forall k,m,$ in this subsection. 
%	Note that the setup parameters are chosen in such a way that the required radar field-of-view (FOV) is provided. However, one can consider a radar FOV constraint in the optimization problem to provide a mathematical guarantee on radar FOV (which is left for future works).

Fig.~\ref{jhjh} illustrates the optimized UAV trajectories for different $T$. It can be seen that in order to increase the harvested energy during the sensing/WPT phase, the UAVs adjust their trajectory center to be close to the center of users for all cases. By increasing $T$, the UAVs try to move closer to each user for increasing the communication throughput during the uplink phase. Precisely, UAVs hover around their cluster users for the maximum possible duration to maintain the closest situation. 
For instance, the amount of hovering time for $\mathrm{UAV}_1$ can be seen from its speed diagram in Fig.~\ref{jhjh12} for the case of $T=75$ seconds, where we can observe that the speed of $\mathrm{UAV}_1$ reduces to zero when flies right above each user. 
%Also, it is assumed that the radars have a $250\hspace{2pt}\textrm{m} \times 250\hspace{2pt}\textrm{m}$ field-of-view (FOV) at $H=100\hspace{2pt}\textrm{m}$ and therefore, can cover the entire area of their clusters during the period of $T$ sec.


Moreover, as an example for time scheduling, we illustrate the optimized fraction of the uplink time resource allocation of $\mathrm{UAV}_1$ for the case of $T=75$ seconds in Fig.~\ref{jhjh1}, where the optimal value of $L$ is equal to $512$ and therefore, $\tau_0=L \widetilde{\tau}_0=0.3584$ second is obtained for the sensing/WPT phase. We can also observe that in the optimized uplink time scheduling, only one user in each cluster (which is the closest to its associated UAV) is supported at each subslot. 


In Fig.~\ref{jhjh2}, the Pareto curves along with the optimized value of $L$ are shown for different values of Pareto weight $\mu$ assuming $T=35$ seconds. It is observed that by increasing $\mu$ till $0.85$, the minimum sensing SINR is increasing; and minimum communication throughput is decreasing. This is due to the fact that more attention is given to the sensing SINR which is confirmed by looking at larger values for optimal $L$. For $\mu=0.85$, $L$ reaches its upper bound $1024$ and so, larger $\mu$ does not change time scheduling and Pareto curves. Note that since the maximum power budget
for uplink transmission are determined by the amount of harvested
energy (see \eqref{key}), the performance of recharging procedure directly affects
the communication throughput. Indeed, the throughput values indicate the performance of both WPT and WIT phases.
As a final note, we remark on the fact that the parameter $L$ plays a key role to manage the sensing-communication trade-off.
% Figure environment removed
\subsection{3D Flight scenarios}
% Figure environment removed	


In this subsection, we study the general 3D flight mode with $z_{\textrm{max}}=150$ m, $z_{\textrm{min}}= 50$ m, $K=5$, $M=2$, $N_1^{\mathrm{NFZ}}=1$, $N_2^{\mathrm{NFZ}}=2$,  $r_{m,j}^{\mathrm{NFZ}}=10$ m, $\forall m,j$, $z_m^{\mathrm{tr}}= 85$ m, $\forall m$, and $\widetilde{r}_{k,m}=0.03$, $\forall k,m$.


Fig.~\ref{jhjh125} shows the UAV trajectories in a 3D scenario. The ability of the proposed method to avoid collision with obstacles in the NFZ can be seen from this figure. 


In Fig.~\ref{jhjh20} we study the effect of user location uncertainty on the communication throughput by comparing the robust
and the non-robust schemes. The robust scheme
considers the location uncertainties in the resource allocation stage as opposed to the non-robust scheme. As expected, by increasing the normalized radius of uncertainty, i.e., $\widetilde{r}_{k,m}$, the minimum communication throughput decreases. It can be observed that the performance gain of the robust method over the non-robust one is significant for higher values of $\widetilde{r}_{k,m}$.
% Figure environment removed
\section{Conclusion} \label{con}
In this paper, we proposed a multi-UAV aided ISWPC framework where a dual use of radar/WPT waveforms enables the UAVs to efficiently detect targets and serve a group of energy-limited ground users. We designed the radar receive filters, radar/WPT waveforms, uplink power along with time scheduling of ground users, and UAV trajectories to maximize a joint radar and communication performance metric under user location uncertainty. Through simulations, we demonstrated that ISWPC improves the joint performance of the radar and wireless powered communication, while choosing a suitable value for the total length of radar/WPT sequences. The proposed ISWPC model in this paper can be extended to a class of ISAC optimization problems with other performance metrics under some general constraints, some of which are discussed as follows for
future work.
\begin{itemize}
	\item One can consider the mutual information as the radar performance metric instead of minimum SINR in the cost function of the proposed multi-objective design problem. 
%	\item Sum utility maximization, i.e., sum of sensing SINR and communication throughput can be considered as the objective function in the design problem. Note that to deal with the sum sensing SINR maximization, the  some of non-convex fractional terms must be maximized which is an interesting mathematical challenge.
	\item 
	Some practical concepts such as UAV jittering, non-linear energy harvesting circuit, and UPA of antennas for all sensing/communication transmitters and receivers can be taken into account in the design problem. 
\end{itemize}
\appendices 
\section{Proof of Proposition~\ref{cor1}} \label{app11}
%Let us rewrite the objective function of the subproblem in \ref{joint} w.r.t. $\mathbf{\widetilde{X}}$ as 
%\begin{equation}
%q(\mathbf{\widetilde{X}})= a + \displaystyle \mu \min_{1\leq m \leq M} f_1(\mathbf{\widetilde{x}}_m),
%\end{equation}
%where
%\begin{equation}
%	a= (1-\mu) \displaystyle \min_{\substack{{1 \leq k \leq  {K}}\\ {1 \leq i \leq  {M}}}} ~ \min_{  \Delta \mathbf{r}^T_{k,i}  \Delta \mathbf{r}_{k,i} \leq \bar{d}^2_{k,i} } R_{k,i},
%\end{equation}
%is a constant w.r.t. $\mathbf{\widetilde{X}}$ and $f_1(\mathbf{\widetilde{x}}_m)$ is defined in Subsection~\ref{joint}. 
Suppose that $f_2(\widetilde{\mathbf{x}}_m)= g_1(\widetilde{\mathbf{x}}_m) - f_1(\widetilde{\mathbf{x}}_m^{(0)}) g_2(\widetilde{\mathbf{x}}_m),\hspace{2pt}\forall m$ where $\widetilde{\mathbf{x}}_m^{(0)}$ indicates the current value of $\widetilde{\mathbf{x}}_m$. Now, let us define $\widetilde{\mathbf{x}}_m^{\star}= \textrm{arg} \max_{\widetilde{\mathbf{x}}_m} f_2(\widetilde{\mathbf{x}}_m), \hspace{2pt}\forall m$. It is observed that $f_2(\widetilde{\mathbf{x}}^{\star}_m) \geq f_2(\widetilde{\mathbf{x}}^{(0)}_m)=0, \hspace{2pt}\forall m$. As a result, since $g_2(\widetilde{\mathbf{x}}_m) >0$, $f_2(\widetilde{\mathbf{x}}^{\star}_m)= g_1(\widetilde{\mathbf{x}}^{\star}_m) - f_1(\widetilde{\mathbf{x}}_m^{(0)}) g_2(\widetilde{\mathbf{x}}^{\star}_m) \geq 0,\hspace{2pt}\forall m$ leads to $f_1(\widetilde{\mathbf{x}}^{\star}_m) \geq f_1(\widetilde{\mathbf{x}}^{(0)}_m),\hspace{2pt}\forall m $, and therefore, $q(\widetilde{\mathbf{X}}^{\star}) \geq q(\widetilde{\mathbf{X}}^{(0)})$, where $\widetilde{\mathbf{X}}^{\star}=\lbrace \widetilde{\mathbf{x}}^{\star}_m ,\hspace{2pt} \forall m \rbrace$ and $\widetilde{\mathbf{X}}^{(0)}=\lbrace \widetilde{\mathbf{x}}^{(0)}_m ,\hspace{2pt} \forall m \rbrace$. Consequently, $\widetilde{\mathbf{X}}^{\star}$ can be a new matrix $\widetilde{\mathbf{X}}$
which increases $q(\widetilde{\mathbf{X}})$.
\section{Proof of Proposition~\ref{cor2} and Proposition~\ref{cor3}} \label{app12}
For the case of Proposition~\ref{cor2}, we need to prove the following expressions:  
\begin{itemize}
	\item at the optimal solution of the problem in \eqref{maxmin4}, $\textrm{C}_8$ is active;
	\item the optimal value of the problem in \eqref{maxmin3}, denoted by $\theta_{\ref{maxmin3}}$, is always greater than the optimal value of the problem in \eqref{maxmin4}, denoted by $\theta_{\ref{maxmin4}}$, and the equality holds when $\textrm{C}_8$ is active.
\end{itemize}
Let us proceed by contradiction to prove the first item. To this end, let us assume
that at the optimal point of the problem in \eqref{maxmin4}, the equality in $\textrm{C}_8$ does not hold. In this case, the value of $\theta$ can be increased by reducing $\phi_{k,m,i}[n]$ in $\textrm{C}_8$, which is evidently in contradiction with the assumption of $\textrm{C}_8$ is not active.

As to the second item, considering $\textrm{C}_7$ in \eqref{maxmin3} and $\widetilde{\textrm{C}}_8$ as well as $\textrm{C}_7$ in \eqref{maxmin4}, straightforwardly leads to the fact that $\theta_{\ref{maxmin3}} \geq \theta_{\ref{maxmin4}}$ and the equality holds holds when $\textrm{C}_8$ is active. 

The Proposition~\ref{cor3} can be similarly proved.
\section{Sum Utility Maximization Problem} \label{app1}
The proposed max-min problem in \eqref{maxmin1} can be extended to the weighted sum utility maximization problem as follows
\begin{align}\label{maxmin1s}
	&\hspace{-10pt} \max_{\widetilde{\mathbf{X}}, \widetilde{\mathbf{W}},{\mathbf{A}}, L, \mathbf{P}^{ul}, \mathbf{Q} }  (1-\mu) \displaystyle \sum_{k=1}^{K}  \sum_{i=1}^{M} ~\eta^{c}_{k,i} \min_{  \Delta \mathbf{r}^T_{k,i}  \Delta \mathbf{r}_{k,i} \leq \bar{d}^2_{k,i} } R_{k,i} \\ \nonumber  & \hspace{55pt}+\mu \displaystyle  \sum_{m=1}^{M} \hspace{1pt} \eta^{s}_{m} \widetilde{\mathrm{SINR}}_m
	\\ \nonumber 
	&\hspace{-7pt} \mbox{s.t.}~~ \textrm{C}_{1}-\textrm{C}_{6},	
\end{align}
where $\eta^{c}_{k,i}, \forall k,i$, and $\eta^{s}_{m}, \forall m$, denote the communication weights for $\mathrm{U}_{k,i}$ and sensing weights for $\mathrm{UAV}_{m}$, respectively. 
This problem is not convex due to the non-convex objective function and non-convex constraints in $\textrm{C}_{3}-\textrm{C}_{6}$. The maximization procedures over $\widetilde{\mathbf{W}}$ and $L$ which are presented in Subsections \ref{w} and \ref{L} can be straightforwardly used here, however, the Subsection~\ref{joint} must be slightly modified to address the problem in \eqref{maxmin1s}. Hence, to deal with the sum of multiple fractional terms in the second term of the objective function in \eqref{maxmin1s}, we equivalently recast the problem in \eqref{maxmin1s} for fixed $[\widetilde{\mathbf{W}},L]$ as follows
\begin{align}\label{maxmin2s}
	&\hspace{-12pt} \scalemath{.98}{ \max_{\widetilde{\mathbf{X}},\mathbf{A}, \mathbf{P}^{ul}, \mathbf{Q},\hm{\widetilde{\theta}} }  (1-\mu) \displaystyle \sum_{k=1}^{K}  \sum_{i=1}^{M}  \eta^{c}_{k,i} \min_{  \Delta \mathbf{r}^T_{k,i}  \Delta \mathbf{r}_{k,i} \leq \bar{d}^2_{k,i} } \hspace{-6pt} R_{k,i} +\mu \displaystyle  \sum_{m=1}^{M}  \eta^{s}_{m} \widetilde{\theta}_m}
	\\ \nonumber 
	&\hspace{-7pt} \mbox{s.t.}~~ \textrm{C}_{1}-\textrm{C}_{3},~\textrm{C}_{5},~\textrm{C}_{6},~ \textrm{C}^{\mathrm{sum}}_7: \widetilde{\mathrm{SINR}}_m \geq \widetilde{\theta}_m,~ \forall m,
\end{align}
where $\hm{\widetilde{\theta}}=\lbrace \widetilde{\theta}_m \geq 0, \forall m \rbrace$. It can be seen that the problem in \eqref{maxmin2s} is separable w.r.t. $[\widetilde{\mathbf{X}},\hm{\widetilde{\theta}}]$ and the other design variables i.e., $[\mathbf{A}, \mathbf{P}^{ul}, \mathbf{Q}]$. Firstly, we write the Lagrange function of the problem in \eqref{maxmin2s} w.r.t. $[\widetilde{\mathbf{X}},\hm{\widetilde{\theta}}]$ as
\begin{align}\label{lag}
	&\hspace{-6pt} \mathcal{L}(\widetilde{\mathbf{X}},\widetilde{\hm{\theta}},\hm{\varpi}, \hm{\varsigma})=(1-\mu) \displaystyle \sum_{k=1}^{K}  \sum_{i=1}^{M}  \eta^{c}_{k,i} \min_{  \Delta \mathbf{r}^T_{k,i}  \Delta \mathbf{r}_{k,i} \leq \bar{d}^2_{k,i} } \hspace{-6pt} R_{k,i} \\ \nonumber &\hspace{-6pt}+\mu \displaystyle  \sum_{m=1}^{M}  \eta^{s}_{m} \widetilde{\theta}_m +  \sum_{m=1}^{M} \varpi_m \left( g_1(\widetilde{\mathbf{x}}_m) - \widetilde{\theta}_m g_2(\widetilde{\mathbf{x}}_m)  \right) \\ \nonumber &\hspace{-6pt}+ \sum_{m=1}^{M} \sum_{i=1}^{\widetilde{N}} \sum_{n=1}^{N} \varsigma_{m,i,n} \left( \vert \widetilde{{x}}_m(i) \vert^2 - p_m^{dl} [n] \right),
\end{align}
where $\hm{\varpi}=\lbrace \varpi_m, \forall m \rbrace$ and $\hm{\varsigma}=\lbrace \varsigma_{m,i,n}, \forall m,i,n \rbrace$ are the Lagrange multipliers. Then, using \eqref{lag}, the Karush-Kuhn-Tucker (KKT) conditions \cite{benson2002global} for the problem in \eqref{maxmin2s} w.r.t. $[\widetilde{\mathbf{X}},\hm{\widetilde{\theta}}]$ can be written as
\begin{align}\label{kkt}
	&\frac{\partial \mathcal{L}}{\partial \widetilde{\mathbf{X}}}=   \sum_{m=1}^{M} \varpi^*_m \left( {\nabla} g_1(\widetilde{\mathbf{x}}^{*}_m) - \widetilde{\theta}^{*}_m {\nabla} g_2(\widetilde{\mathbf{x}}^{*}_m)  \right) \\ \nonumber &+ \sum_{m=1}^{M} \sum_{i=1}^{\widetilde{N}} \sum_{n=1}^{N} \varsigma_{m,i,n} {\nabla} \left( \vert \widetilde{{x}}^*_m(i) \vert^2 - p_m^{dl} [n] \right)=0,
	\\ \label{kkt1} & \frac{\partial \mathcal{L}}{\partial \widetilde{\theta}_m}= \mu \eta^{s}_{m} - \varpi^*_m g_2(\widetilde{\mathbf{x}}^*_m) =0,~\forall m,
	\\ \label{kkt2}& \varpi_m \frac{\partial \mathcal{L}}{\partial \varpi_m}=\varpi^*_m \left( g_1(\widetilde{\mathbf{x}}^*_m) - \widetilde{\theta}^*_m g_2(\widetilde{\mathbf{x}}^*_m)  \right)=0,~ \forall m, \\ \label{kkt3}&
\textrm{C}_5,~ \textrm{C}_7^{\mathrm{sum}},~ \varpi_m \geq 0,~\forall m,
\end{align}
where the superscript $*$ indicates the optimal value of the parameters. Considering the fact that $g_2 (\widetilde{\mathbf{x}}_m) > 0, \forall m$ for any $\widetilde{\mathbf{x}}_m$, the following expressions can be respectively obtained from \eqref{kkt1} and \eqref{kkt2}:
\begin{align} \label{lag1}
\varpi^*_m=\frac{\mu \eta^{s}_{m}}{g_2(\widetilde{\mathbf{x}}^*_m)},~~\widetilde{\theta}^*_m= \frac{g_1(\widetilde{\mathbf{x}}^*_m)}{g_2(\widetilde{\mathbf{x}}^*_m)},~\forall m.
\end{align}
Note that considering $\varpi_m=\varpi^*_m$ and $\widetilde{\theta}_m=\widetilde{\theta}^*_m$, the expressions in \eqref{kkt} and \eqref{kkt3} can be obtained from the KKT conditions of the following optimization problem w.r.t. $\widetilde{\mathbf{X}}$:
\begin{align}\label{maxminapp}
	&\hspace{-10pt}  \max_{\widetilde{\mathbf{X}},\mathbf{A}, \mathbf{P}^{ul}, \mathbf{Q}}  (1-\mu) \displaystyle \sum_{k=1}^{K}  \sum_{i=1}^{M}  \eta^{c}_{k,i} \min_{  \Delta \mathbf{r}^T_{k,i}  \Delta \mathbf{r}_{k,i} \leq \bar{d}^2_{k,i} } \hspace{-6pt} R_{k,i} \\ \nonumber  & \hspace{35pt}+  \sum_{m=1}^{M} \varpi_m \left( g_1(\widetilde{\mathbf{x}}_m) - \widetilde{\theta}_m g_2(\widetilde{\mathbf{x}}_m) \right)
	\\ \nonumber	
	&\hspace{-10pt} \mbox{s.t.}~~ \textrm{C}_{1}-\textrm{C}_{3},~\textrm{C}_{5},~\textrm{C}_{6}.
\end{align}
%Similarly, it can be shown that this proof can be done from \eqref{maxminapp} to \eqref{maxmin2s}.
Theretofore, the optimal solution of \eqref{maxmin2s} can be equivalently obtained from \eqref{maxminapp} by satisfying the expressions in \eqref{lag1}.

Now, we first tackle the problem in \eqref{maxminapp} for fixed $[\hm{\widetilde{\theta}} , \hm{\varpi}]$. The proposed techniques in Subsection~\ref{joint} can be exactly adopted here to handle the non-convex objective function as well as non-convex constraints in $\textrm{C}_3$, $\textrm{C}_5$, and $\textrm{C}_6$. After convergence of the subproblem for fixed $[\hm{\widetilde{\theta}} , \hm{\varpi}]$, the expressions in \eqref{lag1} for $[\hm{\widetilde{\theta}} , \hm{\varpi}]$ must be checked; if they are not satisfied, a differential update rule must be employed along with restarting the algorithm (for fixed $[\hm{\widetilde{\theta}} , \hm{\varpi}]$) to ensure holding \eqref{lag1} (see \cite{vaezy2019energy} for details).
\bibliographystyle{IEEETran}
\bibliography{myreff}
\end{document}

