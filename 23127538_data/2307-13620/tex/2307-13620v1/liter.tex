\section{Literature}
Though few studies have investigated the impact of TNCs on the labor supply of the taxi industry, there exists a broad literature on the discussion of the labor supply behavior of taxi drivers. 

From the macro-management perspective, researchers~\cite{contreras2017effects,cetin2013economic,ling2018analyzing} primarily focused on productivity improvements and taxi medallion utilization analyses. Meanwhile, there is an ongoing debate on the underlying behavioral assumptions for modeling the taxi driver objectives in the research community~\cite{douglas1972price,beesley1983information,yang2010nonlinear,yang2005regulating}. In an early work, Camerer et al.~\cite{camerer1997labor} presented a regression of log-transformation of drivers' daily work hours on log-transformation of average hourly earnings and characterized taxi drivers as having RDP behavior with a daily income target and quit driving once the target is met. Then, K\H{o}szegi and Rabin~\cite{kHoszegi2006model} proposed a theory of preferred personal equilibrium addressing that the anticipated wage will impact drivers' stopping decision (the drivers decide whether or not to look for an additional trip after the current trip). Besides, they clarified the anticipated wage is not status quo based but rational-expectation based~\cite{KszegiUtility}. Based on K\H{o}szegi and Rabin's theory, \highlighttext{Crawford and Meng~\cite{crawford2011new}} discussed a simplified point-based income target as well as target work hours utility function to model drivers' stopping behavior. In the study, they pointed out that taxi drivers' stopping decision is significantly related to hours but income as mentioned by Farber~~\cite{farber2005tomorrow}. In this direction, studies have also found that the labor supply curve of taxi drivers appears to slope downward~\cite{doran2014long,crawford2011new}, and the consensus of these works is that taxi drivers are targeting earners. For the NS model supporters, Solow~\cite{solow1956contribution} first proposed a positive relationship between labor supply and aggregated income. Later, Farber~\cite{farber2005tomorrow,farber2008reference,farber2015you} initially presented evidence for the existence of reference-dependence, but lately revised his study by analyzing the taxi supply pattern in weather-related conditions (e.g., rain and snow). And he deduced that taxi drivers are utility maximizers where drivers work more hours when the wage is high and work fewer hours when the wage is low. \highlighttext{Besides, Crawford and Meng~\cite{crawford2011new}} agreed that changes in the anticipated transitory wage would also be neoclassical. Ashenfelter et al.~\cite{ashenfelter2010shred} found an elasticity of -0.2 in response to the fixed fare change on the labor supply of NYC taxi drivers. Thakral et al.~\cite{thakral2017daily} speculated that more recent income has a stronger impact on drivers ending their shifts rather than income earned earlier. Buchholz et al.~\cite{buchholz2016semiparametric} presented that income-targeting behavior is associated with frequent shorter shifts, and the NS assumption is more suitable for explaining longer shifts. Recently, Frechette~\cite{frechette2019frictions} presented evidence of the reduction of market density when street-hailing and TNCs co-exist in the market based on the NS assumption.

These studies have provided valuable insights into patterns and mechanisms of labor supply in the taxi market. However, the labor supply analyses in these studies assume that the taxi market is still monopolistic, and the labor supply behavior of taxi drivers remains as the stationary NS assumption. However, the current taxi industry has undergone tremendous changes due to the competition from TNCs. And there is an emerging need to examine how the labor supply of the taxi market is affected and\added{ how taxi drivers' behavior changes} due to the growth of TNCs. In this study, we apply the datasets for both TNCs and taxis from January 2013 to December 2018 in NYC to address these issues. The taxi data depict \added{how the labor supply of taxi market changes and drivers' labor supply behavior varies over time.} The TNCs data are then used to mine the most affected aspects of taxi labor supply due to the rise of TNC trips. 
