\section{Conclusion}

The taxi industry has undergone tremendous changes in recent years due to the rapid growth of TNCs. This results in losses of ridership, labor supply, and asset value in the taxi market. \added{To explore the impacts of TNCs on the taxi market performance metrics and taxi drivers' labor supply behavior, we propose three research questions and the corresponding hypotheses. The analyses for three research questions further confirm that the rise of TNCs not only decreases in the market labor supply and revenue but significantly changes the work behavior of taxi drivers}, and such change is likely to continue with increasing competition from TNCs. The major findings of our study can be summarized in three aspects.

First, we investigate the impact of TNCs on the overall revenue and labor supply of taxi market for the first research question. We find that while the rise of TNCs takes over the market share from the taxi industry in Manhattan, it also contributes to solving the under-supply issue in other boroughs of NYC. Besides, the investigation of the yellow taxi market performance shows that 1\% increase in TNC trips leads to 0.02\% loss in monthly market revenue and 0.01\% loss in the total market labor supply. Despite individual monthly work hours being barely affected, the increase in the number of TNC trips results in the reduction of total market labor supply and individual wage expectations. 


Second, \added{to investigate the contradictory observations under the NS assumption, we propose the second and third research questions. The second research question addresses whether drivers' expected wage has been decreased due to the rise of TNC trips. The results suggest that even taxi drivers' expected wage remained unaffected before June 2016 for both green taxis and yellow taxis. Their expected wage decreased when the TNC trips grow rapidly. For this reason, taxi drivers can still reach the expectation-based income despite they pay the same work hours. Moreover, this insight sheds light on the existence of RDP behavior. The third research question is proposed to explore whether RDP behavior is present among taxi drivers and how much of the drivers present RDP behavior over the years. This question directly implies the lack of the taxi drivers' confidence in earning opportunities in the current market and the potential to quit the taxi market.} 

Third, we identify that the income target is the main factor that determines taxi drivers' labor supply behavior in the competitive market. In this domain, we apply the wage decomposition method to investigate the third research question and verified by the wage elasticity estimation. The significant increase in unanticipated wage variation shifts the drivers' labor supply behavior from NS to RDP with increasing competition from TNCs---this change occurred between July 2015 to June 2018 for yellow taxi drivers and after June 2016 for the green taxi drivers. The fact of 20\% RDP behavior among yellow taxi drivers at the end of 2017 indicates that the drivers quit the market due to the loss of confidence in the taxi market as RDP behavior increases. \highlighttext{\added{Over 14\% of yellow taxi drivers and 20\% of green taxi drivers presenting RDP behavior (four times than when the taxi market is still monopolistic) imply the low productivity and quantify the potential loss of taxi drivers. Furthermore, the estimated results from both wage variation and wage elasticity favor the co-existence of the NS and RDP behavior among taxi drivers, and supplement the existing literature for either supporting NS assumption or RDP assumption for drivers' labor supply behavior. Besides, it points out that the NS behavior is dominated at the TNCs' beginning stage. More importantly, it highlights the transition of individual behavior from NS to RDP due to the impact of TNCs and an unsustainable competition between taxis and TNCs. This imposes the need to reconsider entry regulations and pricing policies for the ride-hailing industry. }}

Several future directions are worth exploring based on the findings of the study. The first direction is to quantify the changes in the taxi drivers' target income due to the impact of TNCs, which can be conducted by using medallion transaction data along with the taxi and TNC trips data to gain insights on the net loss of taxi drivers with increasing competitions. In addition, it is meaningful to extend the research framework in our study to investigate the change in the labor supply of the taxi industry in other major cities around the world. While different cities are experiencing different levels of competition from TNCs, the joint analyses will lead to universal and unbiased understandings of the impact of TNC trips on the traditional taxi industry. Furthermore, the market equilibrium between demand and supply of rider sharing market may need to be revisited in light of the competition of multiple service providers under RDP behavior among the drivers.


