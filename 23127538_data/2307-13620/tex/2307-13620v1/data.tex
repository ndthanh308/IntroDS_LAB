\section{Data}
The datasets used in this study are collected by NYCTLC\footnote{In 2009, NYCTLC initiated the Taxi Passenger Enhancement Project, which mandated the use of upgraded metering and information technology in all New York medallion cabs.}, which include both trip-level and aggregated shift-level information. The trip-level data contain the daily average yellow/green taxi trips, average trip duration, vendor ID, pickup and drop-off timestamps and locations (zonal level), trip distance, number of passengers, trip fare, tip, extra surcharges\footnote{These are miscellaneous extras and surcharges, and currently include  \$0.50 and \$1 rush hour and overnight charges.} in rush hour or overnight, improvement surcharge \footnote{The
improvement surcharge began being levied in 2015, and \$0.30 improvement surcharge assessed trips at the flag drop.}, tax, and tolls. Each valid trip is defined as at least one passenger in the car with the value of trip duration, trip distance, and trip fare to be a positive number.

\highlighttext{In the lack of individual shift-level data, we use the aggregated shift-level data in the study.} The aggregated shift-level data are monthly based and are processed from approximately 8.32 billion yellow taxi trips, 0.73 billion green taxi trips, and 6.55 billion TNC trips over six years (January 2013 to December 2018). The data are obtained at the average individual-level and market-level. \highlighttext{The average individual-level data cover daily average work hours per yellow/green taxi driver, daily average income per yellow/green driver, daily average yellow/green taxi medallions, monthly average active days per medallion. The average market-level data include the daily average fare of yellow/green taxis (include tips from credit card), monthly yellow/green taxi drivers, monthly yellow/green taxi medallions, total monthly work hours of all yellow/green taxi medallions and taxi drivers. We calculate the monthly income per driver by dividing the total fare per month by the monthly number of drivers. The data applied in our analysis can be seen in Table~\ref{tab:dataset}.}


\begin{table}[!h]
\centering
\caption{Data preparation}
\label{tab:dataset}
\begin{tabular}{p{3cm}p{6cm}p{6cm}}

\toprule
   & \multicolumn{1}{c}{Market-level} & \multicolumn{1}{c}{Average individual-level} \\
 \hline
 \multirow{12}{*}{Trip-level} & 
\begin{itemize}[leftmargin=*]
  \item {Daily average yellow/green taxi trips}
  \item Average trip duration
\end{itemize} & 
\begin{itemize}
  \item Vendor ID
  \item Pickup and drop-off timestamps and locations
  \item Distance
  \item Number of passengers
  \item Fare, tip, extra surcharge, improvement surcharge, tax, and toll
\end{itemize}
\\
 \multirow{14}{*}{Aggregated shift-level}& \begin{itemize}
  \item Daily average fare of yellow/green taxis
  \item Monthly number of yellow/green taxi drivers
  \item Monthly number of yellow/green taxi medallions
  \item Daily average yellow/green taxi medallions
  \item Total monthly work hours of all yellow/green taxi medallions and taxi drivers
\end{itemize}& \begin{itemize}
  \item Daily average work hours per yellow/green taxi driver
  \item Daily average income per yellow/green driver
  \item Monthly average active days per medallion
\end{itemize}
\\
 \toprule
\end{tabular}
\highlighttext{*Note: the individual shift-level data are not available.} 
\end{table}

%\multicolumn{2}{l}{\begin{tabular}[c]{@{}l@{}}Vendor ID, pickup and drop-off timestamps and locations, distance, number of passengers, fare, tip, \\ extra surcharges, improvement surcharge, tax, toll;\end{tabular}}