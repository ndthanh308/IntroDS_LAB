\section{RESULTS}
\subsection{TNCs impact on overall taxi market performances}
\added{To explore the first research question, \emph{\textbf{how much do TNCs impact overall labor supply and revenue of the taxi market}}, we mine the overall taxi market performance metrics from January 2014 to December 2018 as shown in Figure~\ref{overall} and test the first hypothesis}. There is a consistent decrease in ridership, market revenue, and market total labor supply (work hours for both taxi drivers and taxi medallions) over time for yellow taxis. Besides, we observe a  substantial reduction in the monthly income per yellow taxi driver since the beginning of 2016, followed by a small income rebound in early 2018. Meanwhile, the monthly work hours per yellow taxi driver roughly remained the same over these years. In contrast to trends for the yellow taxi market, we find the increases in green taxi ridership before June 2015 and decreases after that time. As for the individual driver, the monthly income and monthly work hours are observed to first increase before June 2016 and then decrease rapidly. These results provide strong evidence that the nature of the taxi market has changed significantly in recent years. Meanwhile, the TNCs have seen their most rapid growth since early 2014. This overlapping of time consequently evokes a question of whether the rise of TNCs should account for the changes in taxi markets.

% Figure environment removed


\highlighttext{\added{The hypothesis testing comprehensively measures the impacts of the increasing number of TNC trips on overall market performance metrics in the taxi industry.} The estimated results of the OLS regression in five years range are shown in Table~\ref{tab:olsresult_market} and Table~\ref{tab:olsresult_individual}. The OLS estimates take the monthly TNC trips as an indicator variable and regress on a set of taxi market responses, which include trips, fares, and labor supply from both the market-level and average individual-level for yellow and green taxis. }Several significant findings can be identified based on the results. First, the monthly yellow taxi trips are negatively related to the increase of the TNC trips (at 0.001 significance level). \highlighttext{Besides, 1\% increase in TNC trips would lead to 0.02\% reduction in yellow taxi trips (TNC trips increase 2.2\% from November to December 2018 and lead to about 46,000 yellow taxi trips reduced in a month). As for the market revenue and market total labor supply of the yellow taxis, both variables are observed to be significantly negative to the increase of TNC trips. We observe that 1\% increase in TNC trips would result in 0.02\% reduction in monthly fare and 0.01\% reduction in monthly work hours of the yellow taxi market (TNC trips increase 2.2\% from November to December at 2018 lead to about \$687,500 revenue reduction and about 11,550 work hours reduction for yellow taxi market in a month).} Moreover, the monthly yellow taxi drivers, medallions, as well as daily taxi medallions decreased rapidly. These findings suggest that the rise of TNCs takes a vast market share from the yellow taxi market. Contrary to the yellow taxis, we also note no significant impacts on the green taxi market in terms of ridership, market revenue, and total labor supply in five years.



\begin{table}[!h]
\centering
\caption{The impact of TNC trips on the market level performance of labor supply (OLS estimation)}
\label{tab:olsresult_market}
\begin{tabular}{lcccc}
\toprule
\multirow{2}{*}{Dependent variables}                                & \multicolumn{2}{c}{Yellow taxi} & \multicolumn{2}{c}{Green taxi} \\
\cmidrule(l){2-3} \cmidrule(l){4-5}
                                                                    & Coefficient     & Adj.$R^2$  & Coefficient     & 
                                                                    Adj.$R^2$  \\
                                                                    \hline
\multirow{2}{*}{Monthly taxi trips}        & -0.0219   & 0.338      & -0.0133   & 0.031      \\
                                                                    & (9.76E-06 ***)  &            & (0.095 .)
                                                &            \\
\multirow{2}{*}{Monthly taxi fares}    & -0.0166   & 0.245     & -0.0101    & 0.013      \\
                                                                    & (3.48E-05 ***)  &            & (0.185) &            \\
\multirow{2}{*}{Monthly work hours for all taxi drivers}                    & -0.0146   & 0.269      & -0.0078   & 0.004     \\
                                                                    & (1.32E-05 **)  &            & (0.269) &            \\

\multirow{2}{*}{Monthly taxi drivers}                 & -0.0123   & 0.276      & -0.0075    & 0.011      \\
                                                                    & (9.79E-06 ***)  &            & (0.206)  &            \\
\multirow{2}{*}{Monthly taxi medallions}               & -0.0035    & 0.156      & -0.0016   & -0.016      \\
                                                                    & (0.001 ***)  &            & (0.767) &            \\
\multirow{2}{*}{Average daily taxi medallions}      & -0.0056   & 0.198      & -6.27E-05   & -0.017     \\
                                                                    & (0.0002 ***)  &            & (0.991)  &           \\
                                                             \bottomrule
                                                                    
\end{tabular}\\
Note: the p-value of the estimated coefficient is shown in the bracket (. : $\leq$ 0.1; *: P $\leq$ 0.05; **: P $\leq$ 0.01; ***: P $\leq$ 0.001).
\end{table}

\begin{table}[!h]
\centering
\caption{The impact of TNC trips on the individual level performance of labor supply (OLS estimation)}
\label{tab:olsresult_individual}
\begin{tabular}{lcccc}
\toprule
\multirow{2}{*}{Dependent variables}                                & \multicolumn{2}{c}{Yellow taxi} & \multicolumn{2}{c}{Green taxi} \\
\cmidrule(l){2-3} \cmidrule(l){4-5}
                                                                    & Coefficient &    Adj.$R^2$  & Coefficient     & 
                                                                    Adj.$R^2$  \\
                                                                    \hline
\multirow{2}{*}{Monthly income per taxi driver} & -0.0044   & 0.09     & -0.0026  & 0.002      \\
                                                                    & (0.011 *)  &            & (0.290)   &            \\
\multirow{2}{*}{Monthly work hours per taxi driver}               & -0.0023   & 0.075     & -0.0004    & -0.016      \\
                                                                    & (0.020 *)  &            & (0.828)  &            \\
\multirow{2}{*}{Monthly work hours per taxi medallion}               & -0.011   & 0.300     & -0.0064    & 0.121      \\
                                                                    & (3.57E-06 ***)  &            & (0.004 **)  &            \\
                                                                    
\multirow{2}{*}{Average daily work hours per taxi medallion}         & -0.0091   & 0.300      &-0.0082    & -0.31     \\
                                                                    & (6.99E-07 ***)  &            & (2.23E-06 ***) &            \\
\multirow{2}{*}{Average daily work hours per taxi driver}      & -0.0009   & 0.065      & -0.0066   & 0.261     \\
                                                                    & (0.028 *)  &            & (1.82E-05 ***)  &           \\                                                                    
\multirow{2}{*}{Average minutes per taxi trip}                     & 0.0085   & 0.162     & 0.0043    &0.064      \\
                                                                    & (0.001 ***)  &            & (0.028 *)&            \\
                                                                     \bottomrule

\end{tabular}\\

Note: the p-value of the estimated coefficient is shown in the bracket (. : $\leq$ 0.1; *: P $\leq$ 0.05; **: P $\leq$ 0.01; ***: P $\leq$ 0.001).
\end{table}


From the individual perspective, the monthly income and work hours per yellow taxi driver are observed to decrease 0.04{\textperthousand}  and 0.02{\textperthousand} along with 1\% increase of the TNC trips. And from the five years range, the rise of TNC captures a small proportion of the variation of yellow taxi drivers' monthly income and work hours. Moreover, the impacts on the utilization (work hours) of per taxi medallion are much more significant (at 0.001 significance level) than it is on the monthly work hours per yellow taxi driver (at 0.05 significance level), and green taxis yield the same result. In combination with the observed drastic reduction in total labor supply of the yellow taxi market (0.01\% reduction along with
1\% increase of TNC trips), such evidence implies that there is only a minor change in the monthly work hours per individual yellow taxi driver (0.02{\textperthousand} reduction along with 1\% increase of TNC trips) while a significant reduction in the monthly driver (0.01\% reduction along with 1\% increase of TNC trips) is observed. The impacts on monthly labor supply and monthly income per driver are not statistically significant for green taxis. However, daily work hours per green taxi driver are found to decrease significantly by 0.07{\textperthousand} along with 1\% increase in monthly TNC trips. This suggests that individual drivers work more days in a month since there are no significant changes in the monthly work hours.

The above results are based on January 2014 and December 2018, where we have seen the most rapid growth of the TNC sector. \highlighttext{The well-known fact is that the taxi market has lost significant ridership, and we can also verify that the rise of TNCs has significantly decreased total market revenue and labor supply for yellow taxis from the data. Nevertheless, such impact is found to be non-significant for the green taxi market, which is a special class of taxi service for serving areas outside Manhattan. These observations lead to two important implications.} First, the taxi demand within Manhattan, which used to be served exclusively by yellow taxis, is close to its saturated level, and the TNCs, therefore, directly compete with yellow cabs for existing passengers. However, for areas outside Manhattan, we find empirically that the TNCs lead to induced ridership that is unsatisfied before and contribute to mitigating the under-supply issue in these areas. 
\begin{table}[!h]
\centering
\caption{Yellow taxi revenue and labor supply variation (OLS estimation)}
\label{tab:yellow taxi market time period variation}
\begin{tabular}{lcccc}
\toprule 

\begin{tabular}[c]{@{}c@{}}Month/Year\end{tabular} & \begin{tabular}[c]{@{}c@{}}Monthly fare\\all taxi drivers \end{tabular} & \begin{tabular}[c]{@{}c@{}}Monthly income\\per taxi driver\end{tabular} & \begin{tabular}[c]{@{}c@{}}Monthly work hours\\all taxi drivers\end{tabular} & \begin{tabular}[c]{@{}c@{}}Monthly work hours\\per taxi driver\end{tabular} \\
\hline

\multirow{2}{*}{01/13-06/15}& -0.0013& 0.0018& -0.0019 &0.0004\\
                            &(0.421)&(0.434)&(0.125)&(0.673)
\\
\multirow{2}{*}{07/13-12/15} & -0.003& 0.0006&-0.0035&0.0001 \\
                           & (0.078 .)&(0.608)&(0.01 *)&(0.867)
\\
\multirow{2}{*}{01/14-06/16} & -0.0042& -0.0003& -0.0045&-0.0005
\\
                           &(0.055 .) & (0.856) &
                           (0.01 *) &(0.597)\\
\multirow{2}{*}{07/14-12/16} & -0.0079 &-0.0032&-0.0073&-0.0025                                                                           \\
                            &(0.008 **)                                                      & (0.091 .)                                                         & (0.002 **)  &(0.036 *)                                                        \\

\multirow{2}{*}{01/15-06/17} & -0.1308&-0.0615&-0.0914&-0.0221                                                                         \\
                           &(1.033E-05 ***)                                                      & (0.002 **)                                                         & (3.26E-05 ***) &(0.066 . )                                                         \\

\multirow{2}{*}{07/15-12/17} & -0.1913&-0.0668&-0.1469&-0.0238                                                                         \\
                           &(6.77E-05 ***)                                                      & (0.02 *)                                                         & (4.36E-05 ***) &(0.261)                                                         \\
\multirow{2}{*}{01/16-06/18} & -0.2925                                                      & -0.0392                                                           & -0.2552  &  -0.0018                                                      \\
                           &(3.87E-06 ***)                                                      & (0.325)                                                         & (1.73E-07 ***) &(0.937)                                                         \\
\multirow{2}{*}{07/16-12/18} & -0.2785 &0.0398&-0.2896&0.0287                                                                  \\
                           &(2.08E-05 ***)                                                      & (0.323)                                                         & (1.94E-07 ***)  &(0.221)                                                        \\

                           \bottomrule
\end{tabular}

Note: the p-value of the estimation of monthly TNC trips is in the bracket(. : $\leq$ 0.1; *: P $\leq$ 0.05; **: P $\leq$ 0.01; ***: P $\leq$ 0.001).
\end{table}



\begin{table}[!h]
\centering
\caption{Green taxi revenue and labor supply variation (OLS estimation)}
\label{tab:green taxi market time period variation}
\begin{tabular}{lcccc}
\toprule
Month/Year          & \begin{tabular}[c]{@{}c@{}}Monthly fare\\ all taxi drivers \end{tabular} & \begin{tabular}[c]{@{}c@{}}Monthly income\\  per taxi driver\end{tabular} & \begin{tabular}[c]{@{}c@{}}Monthly work hours\\ all taxi drivers\end{tabular} &
\begin{tabular}[c]{@{}c@{}}Monthly work hours\\ per taxi driver\end{tabular} \\
\hline


\multirow{2}{*}{01/14-06/16} & 0.0156& 0.0042& 0.0151&0.0038
\\
                           &(0.002 **) & (0.064 .) &
                           (0.0012 **) &(0.046 *)\\
\multirow{2}{*}{07/14-12/16} & -0.0039 &-0.0048&-0.0022&-0.003                                                                           \\
                            &(0.414)                                                      & (0.076 .)                                                         & (0.585)  &(0.096 .)                                                        \\
\multirow{2}{*}{01/15-06/17} & -0.2753&-0.109&-0.2144&-0.0481                                                                         \\
                           &(1.30E-06 ***)                                                      & (0.00016 ***)                                                         & (6.46E-06 ***) &(0.011 *)                                                         \\
\multirow{2}{*}{07/15-12/17} & -0.5422                                                      & -0.1448                                                           & -0.4691 &  -0.0717                                                      \\
                           &(2.80E-09 ***)                                                      & (0.0004 ***)                                                         & (1.56E-09 ***) &(0.009 **)                                                         \\
\multirow{2}{*}{01/16-06/18} & -0.7848 &-0.1155&-0.7355&-0.0661                                                                   \\
                           &(2.05E-013 ***)                                                      & (0.042 *)                                                         & (9.1E-16 ***)  &(0.061.)                                                        \\
\multirow{2}{*}{07/16-12/18} & -0.681 &0.0717&-0.7508&0.0018                                                                   \\
                           &(1.9E-12 ***)                                                      & (0.1303)                                                         & (4.49E-15 ***)  &(0.953)                                                        \\
                           
                           \bottomrule
\end{tabular}

Note: the p-value of the estimation of monthly TNC trips is shown in the bracket(. : $\leq$ 0.1; *: P $\leq$ 0.05; **: P $\leq$ 0.01; ***: P $\leq$ 0.001).
\end{table}

\highlighttext{Figure~\ref{overall} indicates the long-term examination might not be appropriate to understand the labor supply and revenue in taxi market due to their non-linear trend from 2014 to 2018. Thus, we testify the temporal variation based on the shorter time periods via OLS}. The results are presented in Table~\ref{tab:yellow taxi market time period variation} and Table~\ref{tab:green taxi market time period variation}. For yellow taxi drivers, the losses in total market revenue and labor supply (at 0.001 significance level) are found to be more significant than the losses at the individual level (at 0.05 significance level). The trend of the changes in monthly work hours of all taxi drivers has been observed to best echo the increasing trend in the number of TNC trips (as early as July 2013), followed by the market revenue. However, the drivers' monthly work hours are barely affected over time. The performances of the green taxi market at the beginning stage (January 2014) are different from that of the yellow taxi market. We observe more positive impacts in the green taxi market as the market revenue and total labor supply increase with positive coefficients, primarily due to the induced ridership outside Manhattan (e.g., there is 0.02\% increase in total labor supply of the green taxi market along with 1\% increase in the number of TNC trips). 

\added{The statistical estimates for both long-term and short-term taxi market performance directly reject the first hypothesis that the rise of TNC trips does not significantly impact the labor supply and revenue of the taxi market}. Our findings of the monthly income and monthly work hours for both yellow and green taxi markets lead to a counter-intuitive observation: individuals' work hours are barely affected by the significant decrease in their income. That observation contradicts the NS assumption, where taxi drivers are assumed to be revenue maximizers. One possible explanation for this observation is that the increased average minutes per taxi trip (positive coefficient at 0.001 and 0.05 significance level for yellow and green taxis) makes up for the reduced number of trips. Yet another plausible explanation is that the taxi drivers may have a specific reference target, and their work hours are no longer strictly positively related to their daily revenue. Following this explanation, taxi drivers may choose to decrease their income target and reach this target with lower wage rates and the same work hours. %This behavior can be illustrated in Figure~\ref{refer}, where the income target changes from $T^1$ to $T^2$, and the works hour keeps the same $H^*$, contrary to the decrease in work hours under NS as shown in Figure~\ref{neo}).
If this is the case, such an observation is indicative of the existence of RDP behavior in the current taxi market. And we next test the validity of such an explanation through statistical analyses. 

% % Figure environment removed

% % Figure environment removed





\subsection{Taxi drivers' labor supply behavior and wage elasticity}
\added{While it is observed that the taxi market has been significantly affected by the rise of TNCs, we notice that taxi drivers' work hours do not change even though their income decreases. As taxi drivers are aware of the competition from the TNC sector, under the neo-classical scheme, it is reasonable to expect that taxi drivers will lower their work hours if their expected wage decreases, which obviously contradicts the aforementioned observation that their work hours are barely affected. With the increasing number of TNC trips, the underlying logic is that the taxi market share is expected to decrease and taxi drivers are likely to be more difficult to reach the same income level as before with the same amount of daily efforts (work hours) due to the competition. \highlighttext{In light of this counter-intuitive observation, we propose %the following research questions to investigate the reasons: \emph{\textbf{do the drivers decrease their expected wage along with the increasing number of TNC trips?} And 
the second hypothesis that \emph{\textbf{the increase of TNC trips does not decrease the taxi drivers' expected wage}}}. %Taxi drivers are likely to be more difficult to reach an income level as before with the same amount of daily efforts (work hours) due to the increase number of TNC trips.}
As discussed earlier, targeting behavior discourages the employee's motivation, lead to lower productivity, eventually conduct a vicious circle for the industry. This idea gives rise to our suspicion that the taxi labor supply may now be better explained by the RDP behavior instead of the neoclassical one. On this basis, our interest is to shed light on quantifying the RDP behavior in the taxi market to interpret the taxi drivers' non-intuitive response to the TNCs competition. This question has its roots in \highlighttext{Crawford and Meng's~\cite{crawford2011new}} study, where the taxi labor supply behavior is mainly driven by the income target. %To this end, the labor supply estimation under both the RDP and NS models can be analyzed from two aspects: wage proportion and wage elasticity. 
In this regard, we propose %the third research question: \emph{\textbf{is the RDP behavior present among taxi drivers with the increasing number of TNC trips? }} 
the third hypothesis that \emph{\textbf{taxi drivers do not present RDP behavior,}}}  and use the wage decomposition method to testify the labor supply behavior.


\highlighttext{We examine the second hypothesis using PLS to regress the drivers' average daily expected wage on the set of temporal indicators, the improvement surcharge indicators, and the log-transformation of monthly TNC trips.} The results are given in Table~\ref{tab:expected wage1} and Table~\ref{tab:expected wage2}. 

% Please add the following required packages to your document preamble:
% \usepackage{multirow}
\begin{table}[!h]
\centering
\caption{Result of log-transformation of TNC trips: experiment \uppercase\expandafter{\romannumeral1} (PLS estimation)}
\label{tab:expected wage1}
\begin{tabular}{lcccc}
\toprule
\multirow{2}{*}{Month/Year}                                & \multicolumn{2}{c}{Yellow taxi} & \multicolumn{2}{c}{Green taxi} \\
\cmidrule(l){2-3} \cmidrule(l){4-5}
                                                                    & Coefficient     & P value  & Coefficient     & P value  \\
                                                                    \hline
\multirow{1}{*}{01/13-06/15} & 0.0009 & 0.457 &-&- \\
\multirow{1}{*}{07/13-12/15} & 0.0006& 0.622&-&-\\
\multirow{1}{*}{01/14-06/16} & -0.0005 & 0.698&0.0044&0.046 *\\
\multirow{1}{*}{07/14-12/16} & -0.0031& 0.085 .&-0.0050&0.044 *\\
\multirow{1}{*}{01/15-06/17} & -0.0518& 0.006 **&-0.1167&4.16E-06 ***\\
\multirow{1}{*}{07/15-12/17} & -0.0816& 0.001 ***&-0.1537 & 2.85E-05 ***\\
\multirow{1}{*}{01/16-06/18} & -0.0656& 0.069.&-0.1113 &0.028 *\\
\multirow{1}{*}{07/16-12/18} & 0.0665& 0.068.&0.0556&0.188\\
                           \bottomrule
\end{tabular}

Note: *: P $\leq$ 0.05; **: P $\leq$ 0.01; ***: P $\leq$ 0.001.

\end{table}


\begin{table}[!h]
\centering
\caption{Result of log-transformation of monthly TNC trips: experiment \uppercase\expandafter{\romannumeral2} (PLS estimation)}
\label{tab:expected wage2}
\begin{tabular}{lcccc}
\toprule
\multirow{2}{*}{Month/Year}                                & \multicolumn{2}{c}{Yellow taxi} & \multicolumn{2}{c}{Green taxi} \\
\cmidrule(l){2-3} \cmidrule(l){4-5}
                                                                    & Coefficient    & P value  & Coefficient    & P value  \\
                                                                    \hline
\multirow{1}{*}{01/13-12/14} &0.0013 &0.469 & -&-\\
\multirow{1}{*}{07/13-06/15} &0.0014 &0.293&-&-\\
\multirow{1}{*}{01/14-12/15} &0.0005 &0.761& 0.0045&0.073 .\\
\multirow{1}{*}{07/14-06/16} &-0.0020 &0.141&-0.0020&0.211 \\
\multirow{1}{*}{01/15-12/16} &-0.0650 &0.0002 ***&-0.0779&0.004 ** \\
\multirow{1}{*}{07/15-06/17} &-0.0401&0.0022 ** &-0.1447 & 0.003 **\\
\multirow{1}{*}{01/16-12/17} &-0.1218 & 0.005 **&-0.2065 &0.001 ***\\
\multirow{1}{*}{07/16-06/18} & 0.0323&0.512&0.0597&0.293\\
\multirow{1}{*}{01/17-12/18} & 0.1279&0.027 *&0.1900&0.003 **\\
                           \bottomrule
\end{tabular}

Note: *: P $\leq$ 0.05; **: P $\leq$ 0.01; ***: P $\leq$ 0.001.
\end{table}


Although yellow taxi drivers' expected wage is non-significant related with month TNC trips from January 2013 to December 2014. Green taxi drivers are found to increase their expected wage due to the increase of TNC trips, which implies that the taxi market is still under-supplied with TNC trips in the beginning stage. Besides, the TNC trips are also found to negatively impact the expected wage from January 2015 to December 2017 for both green and yellow taxi drivers, which indicates the taxi market gradually shifted into the over-supplied state with increasing competition between taxis and TNCs (at 0.01 significance level). However, we observe that the increasing of TNC trips is found to be statistic significant and positively related to drivers' expected wage in 2018 and an individual-level income rebound presents during this period, as shown in Figure~\ref{fig:monthly fare  & TNC trips}. The income rebound is primarily due to the fact that the loss of total market supply (drivers quit the market) is faster than the reduction in taxi ridership and total market revenue, as shown in Figure ~\ref{fig:proportion}. As a consequence, the results from the experiment reject the second hypothesis that the increase of TNC trips does not decrease the taxi drivers' expected wage. Besides, Table~\ref{tab:expected wage1} and Table~\ref{tab:expected wage2} also indicate the consistency of our results under different data compositions. 

% Figure environment removed

% Figure environment removed


\highlighttext{To testify taxi driver's labor supply behavior in the third hypothesis, we measure the unanticipated and anticipated transitory wage variation based on the wage decomposition method}. Besides, we conduct the t-test to examine the change of the unanticipated transitory wage proportion in the current year group compared with its base year group. 
\begin{table}[!h]
\centering
\caption{Results of wage variation decomposition: experiment \uppercase\expandafter{\romannumeral1} (PLS estimation)}
\label{tab:wage decomposition1}
\begin{tabular}{lcccccc}
\toprule
\multirow{2}{*}{Month/Year}                                & \multicolumn{3}{c}{Yellow taxi} & \multicolumn{3}{c}{Green taxi} \\
\cmidrule(l){2-4} \cmidrule(l){5-7}
                                                                    & Fixed     & Anticipated  & Unanticipated   & Fixed     & Anticipated  & Unanticipated   \\
                                                                    \hline
\multirow{2}{*}{01/13-06/15}& 9.04E-06& 0.0021& 0.0002 &-&-&-\\
                           & (0.42\%)&(90.93\%)&(8.65\%)&-&-&-
\\

\multirow{2}{*}{07/13-12/15} & 1.46E-05& 0.0020& 0.0002&-&-&-
\\
                           & (0.65\%) & (90.05\%) &
                           (9.30\%) &-&-&-\\  
                           
\multirow{2}{*}{01/14-06/16}& 0.0001& 0.00198& 0.0002  &0.000412&0.0049&0.0004 \\
                           & (4.42\%)&(85.64\%)&(9.94\%)&(7.26\%)&(86.42\%)&(6.32\%)
\\
\multirow{2}{*}{07/14-12/16} & 0.0002& 0.0018& 0.0003& 0.0002&0.0040&0.0006 
\\
                           & (9.42\%) & (78.45\%) &
                           (12.13\%)& (3.22\%) & (84.36\%) &
                           (12.42\%)  \\  
\multirow{2}{*}{01/15-06/17} & 0.0002                                                       & 0.0029                                                          & 0.0004 &     3.16E-30&0.0070&0.0011                                                    \\
                           & (5.62\%)                                                      & (82.09\%)                                                         & (12.29\%)  & (0\%)                                                      & (85.98\%)                                                         & (14.02\%)                                                        \\
\multirow{2}{*}{07/15-12/17 } & 6.04E-05                                                       & 0.0025                                                          & 0.0005&7.89E-31& 0.0062&0.0091                                                    \\
                           & (1.95\%)                                                      & (80.65\%)                                                         & (17.40\%)  & (0\%)                                                      & (87.24\%)                                                         & (12.76\%)                                                         \\

\multirow{2}{*}{01/16-06/18} & 0.0002                                                      & 0.0025                                                           & 0.0005  &7.10E-30& 0.0055&0.0013                                                \\
                           & (6.76\%)                                                      & (77.28\%)                                                         & (15.96\%)  & (0\%)                                                      & (80.86\%)                                                         & (19.14\%)                                                         \\
\multirow{2}{*}{07/16-12/18} & 5.82E-05                                                       & 0.0021                                                          & 0.0004 &    3.16E-30&0.0028&0.0008                                                    \\
                           & (2.33\%)                                                      & (83.33\%)                                                         & (14.34\%)  & (0\%)                                                      & (78.03\%)                                                         & (21.97\%)                                                         \\
                           \bottomrule
\end{tabular}

Note: the proportion of variation of each part in the total variation is in the bracket.
\end{table}


\begin{table}[!h]
\centering
\caption{Results of wage variation decomposition: experiment \uppercase\expandafter{\romannumeral2} (PLS estimation)}
\label{tab:wage decomposition2}
\begin{tabular}{lcccccc}
\toprule
\multirow{2}{*}{Month/Year}                                & \multicolumn{3}{c}{Yellow taxi} & \multicolumn{3}{c}{Green taxi} \\
\cmidrule(l){2-4} \cmidrule(l){5-7}
                                                                    & Fixed     & Anticipated  & Unanticipated   & Fixed     & Anticipated  & Unanticipated   \\
                                                                    \hline

\multirow{2}{*}{01/13-12/14 }& 9.6E-07& 0.0022& 0.0002 &-&-&-\\
                           & (0.04\%)&(92.84\%)&(7.12\%)&-&-&-
\\
\multirow{2}{*}{07/13-06/15} & 4.03E-05& 0.0021& 0.0002 &-&-&- \\
                           & (1.76\%)&(91.50\%)&(6.74\%)&-&-&-
\\
\multirow{2}{*}{01/14-12/15} & 4.65E-05& 0.0020& 0.0002&0.0005&0.0056&0.0003
\\
                           & (2.05\%) & (89.80\%) &
                           (8.15\%)& (8.29\%) & (87.17\%) &
                           (4.54\%)\\

\multirow{2}{*}{07/14-06/16} & 5.08E-05                                                       & 0.0015                                                           & 0.0002 &   7.15E-07&0.0018&0.0002                                                        \\
                           & (3.16\%)                                                      & (85.10\%)                                                         & (11.98\%) & (0.04\%)                                                      & (91.54\%)                                                         & (8.42\%)                                                           \\
\multirow{2}{*}{01/15-12/16} & 9.34E-05                                                       & 0.0020                                                           & 0.0003  & 3.16E-30&0.0047&0.0006                                                          \\
                           & (3.93\%)                                                      & (82.97\%)                                                         & (13.10\%) & (0\%)                                                      & (88.06\%)                                                         & (11.94\%)                                                          \\
\multirow{2}{*}{07/15-06/17} & 1.21E-05                                                       & 0.0024                                                          & 0.0005    & 7.89E-31&0.0066&0.0007                                                   \\
                           & (0.41\%)                                                      & (83.33\%)                                                         & (16.26\%)    & (0\%)                                                      & (89.84\%)                                                         & (10.16\%)                                                        \\
\multirow{2}{*}{01/16-12/17} & 8.97E-05                                                      & 0.0022                                                          & 0.0006    & 7.10E-30&0.0054&0.0012                                                       \\
                           & (3.10\%)                                                      & (76.90\%)                                                         & (20\%)& (0\%)                                                      & (82.05\%)                                                         & (17.95\%)                                                           \\
\multirow{2}{*}{07/16-06/18} & 7.05E-05                                                       & 0.0022&0.0005   &3.16E-05&0.0031&0.0008                                                                                                      \\
                           & (2.52\%)                                                      & (79.30\%)                                                         & (18.18\%)  & (0.80\%)                                                      & (78.68\%)                                                         & (20.52\%)                                                         \\
\multirow{2}{*}{01/17-12/18} & 0.0001                                                      & 0.0024                                                         & 0.0004    & 3.16E-30&0.0030&0.0009                                                 \\
                           & (4.35\%)                                                      & (81.33\%)                                                         & (14.32\%) & (0\%)                                                      & (76.55\%)                                                         & (23.45\%)                                                          \\
                                                  
                           \bottomrule
\end{tabular}

Note: the proportion of variation of each part in the total variation is in the bracket.
\end{table}

The wage decomposition results are presented in Table~\ref{tab:wage decomposition1} and Table~\ref{tab:wage decomposition2}. The results quantify the fixed, anticipated, and unanticipated transitory wage variation of drivers' wage rate over time. For the yellow taxi drivers, we observe that the anticipated transitory variation dominates the total variation in both experiments. The proportion of unanticipated wage variation (8\%) from January 2013 to December 2014 is close to the findings in Farber's study~\cite{farber2015you}, where the unanticipated hourly transitory wage variation is reported to be 12.1\% when the taxi market is still monopolistic. \highlighttext{That is, when supply is not saturated, almost all of the taxi drivers' work behavior can be explained by the NS behavior.} However, the proportion of anticipated wage variation is observed to decrease as TNCs grow. The unanticipated transitory wage variation reaches its peak at the end of 2017 when 20 \% of yellow taxi drivers' behavior can be explained by RDP as shown in Figure~\ref{fig:relationship}. This result matches well with our previous findings when drivers decrease their expected wage as shown in Table~\ref{tab:expected wage1} and Table~\ref{tab:expected wage2}, as well as when the market revenue and labor supply significantly decreased (see Table~\ref{tab:yellow taxi market time period variation}). For green taxi drivers, only a small proportion (6\%) of green taxi drivers shows RDP behavior in the beginning stage (from January 2014 to June 2016). With the increasing competition from TNCs (see Figure~\ref{overall}), green taxi drivers' expected wage decreases, which results in the reduction of green taxi drivers in their work hours and income from July 2015 to June 2017 (see Table~\ref{tab:green taxi market time period variation}). Therefore, green taxi drivers exhibit revenue optimizing behavior as suggested by the NS (drivers having RDP behavior also slightly decrease during this period from 14\% to 12\% in experiment \uppercase\expandafter{\romannumeral1} and 12\% to 10\% in experiment \uppercase\expandafter{\romannumeral2}). After the second half of 2017, green taxi drivers show increasing RDP behavior due to the increasing competition from TNCs (see Figure~\ref{fig:relationship}). Finally, we observe that RDP behavior captures over 20\% of the green taxi drivers in the current taxi market.

The t-test in Table~\ref{tab:test hypothesis 2 for exp 1} and Table ~\ref{tab:test hypothesis 2 for exp 2} show a significant higher unanticipated transitory wage for both yellow and green taxis than their base year. The relationship between TNC trips and unanticipated transitory wage variation can be seen in Figure~\ref{fig:relationship}. The results indicate that drivers are facing more uncertainty in earning opportunities and clearly illustrate the change of drivers' labor supply behavior over time due to the increasing number of TNC trips. Therefore, \added{we reject the third hypothesis and conduct that RDP behavior presents among taxi drivers with the increasing number of TNC trips}. Based on the results, we observe that yellow taxi drivers face much more serious competition than green taxi drivers before 2017. Moreover, such competition leads to an unsustainable state in the ride-sharing market and results in 20\% yellow taxi drivers having RDP behavior at the end of 2017, which means a high proportion of taxi drivers lose confidence in the taxi industry. Finally, yellow taxi drivers quit the taxi market. Besides, the green taxi market benefits from the increase in demand at the beginning of TNCs' growth. Meanwhile, green taxi drivers present NS behavior before July 2017. However, the unanticipated transitory wage variation of green taxi drivers is found to increase after June 2017 and account for over 20\% of the total wage variation in both experiments at the end of 2018, which is over three times as compared to when the taxi market is still monopolistic. Consequently, the RDP behavior should not be ignored, and NS behavior is no longer suitable to interpret the total taxi drivers' work behavior in a competitive market. Instead, at least $20\%$ of green taxi drivers perform in a loss-aversion manner in the market rather than the revenue-maximizing behavior, which is widely used when the taxi market is still monopolistic. This finding is aligned with the second explanation for the question that we raised to the OLS model. That is, the driver has a specific reference target. Moreover, the fact that individual labor supply is found to be barely unaffected (see Figure~\ref{fig:hours}) while their monthly income is found to be significantly decreased can be explained by the co-existence of NS and the RDP behavior in the market. Furthermore, combining the examinations of the second and third hypotheses, we conclude that drivers decrease their income target and some of them even quit the market, so that the remaining drivers are observed to still serve the same amount of work hours. It points out the necessity to consider the regulation of TNCs for the sustainability of the taxi market. This issue requires further investigation with related data sources. Finally, Table~\ref{tab:wage decomposition1} and Table~\ref{tab:wage decomposition2} again confirm the consistency of our results under different data compositions.


\begin{table}[!h]
\centering
\caption{t-test for the change of unanticipated wage variation in experiment\uppercase\expandafter{\romannumeral 1} }
\label{tab:test hypothesis 2 for exp 1}
\begin{tabular}{lcc}
\toprule
Month/Year   & Yellow taxi& Green taxi \\  
                                                                    \hline

01/13-06/15&0.896&-
\\
07/13-12/15&0.853&-
\\
01/14-06/16&0.696&0.957
\\

07/14-12/16  &         0.228&0.032 *                                            \\
01/15-06/17           &0.118     &0.0007 ***   \\
07/15-12/17  &0.038*&0.0047 ** \\
01/16-06/18 & 0.023 * &0.002 **\\
07/16-12/18   &0.051 .&0.003 *  \\
\bottomrule

\end{tabular}
\end{table}


\begin{table}[!h]
\centering
\caption{t-test for the change of unanticipated wage variation in experiment \uppercase\expandafter{\romannumeral 2} }
\label{tab:test hypothesis 2 for exp 2}
\begin{tabular}{lcc}
\toprule
Month/Year   & Yellow taxi& Green taxi \\  
                                                                    \hline

07/13-06/15&0.6461&-
\\
01/14-12/15&0.922&-
\\

07/14-06/16 &         0.367&0.606                                         \\
01/15-12/16           &0.121     &0.019 *   \\
07/15-06/17   &0.0104*&0.015 *  \\
01/16-12/17 & 0.048 * &0.002 **\\
07/16-06/18   &0.138&0.046 *\\
01/17-12/18& 0.195 &0.006 **\\
\bottomrule

\end{tabular}

\end{table}

% Figure environment removed


% Figure environment removed


\begin{table}[!h]
\centering
\caption{Wage elasticity of taxi labor supply: experiment \uppercase\expandafter{\romannumeral 1} (PLS estimation)}
\label{tab:elast1}
\begin{tabular}{lcccc}
\toprule
\multirow{2}{*}{Month/Year}& \multicolumn{2}{c}{Yellow taxi}&\multicolumn{2}{c}{Green taxi}\\
\cmidrule(l){2-3} \cmidrule(l){4-5}
     & Coefficient & $Adj.R^2$ & Coefficient & $Adj.R^2$ \\
\hline
\multirow{2}{*}{01/13-06/15} & 0.6611      & 0.844& -&-\\
          & (5.00E-13 ***)     &   &-&        \\
\multirow{2}{*}{07/13-12/15} & 0.6526      & 0.844&-&-\\
          & (5.06E-13 ***)     &         &-&     \\
\multirow{2}{*}{01/14-06/16} & 0.6383      & 0.837&0.8006&0.92\\
          & (9.39E-13 ***)     &    &(4.55E-07 ***)&          \\
\multirow{2}{*}{07/14-12/16} & 0.5592      & 0.731&0.6228&0.953\\
          & (1.10E-09 ***)     &        &(5.96E-08 ***)&       \\
\multirow{2}{*}{01/15-06/17} & 0.5040      & 0.764&0.5822&0.903\\
          & (1.69E-10 ***)     &     &(7.51E-13 ***)&         \\
\multirow{2}{*}{07/15-12/17} & 0.4780      & 0.712&0.6103&0.909\\
          & (2.82E-09 ***)     &  &(2.48E-16 ***)&             \\
\multirow{2}{*}{01/16-06/18} & 0.4924      & 0.745&0.5770&0.863\\
          & (5.23E-10 ***)     &      &(8.34E-14 ***)&        \\
\multirow{2}{*}{07/16-12/18}& 0.4944      & 0.699&0.4776&0.541\\
          & (5.49E-09 ***)     &     &(2.24E-06 ***)&          \\
          \bottomrule
          
\end{tabular}

Note: the p-value of the estimate for log-transformation of monthly income per taxi driver is in the bracket (. : p $\leq$ 0.1; *: P $\leq$ 0.05; **: P $\leq$ 0.01; ***: P $\leq$ 0.001).
\end{table}

\begin{table}[!h]
\centering
\caption{Wage elasticity of taxi labor supply: experiment \uppercase\expandafter{\romannumeral 2} (PLS estimation)}
\label{tab:elast2}
\begin{tabular}{lcccc}
\toprule
\multirow{2}{*}{Month/Year}& \multicolumn{2}{c}{Yellow taxi}&\multicolumn{2}{c}{Green taxi}\\
\cmidrule(l){2-3} \cmidrule(l){4-5}
     & Coefficient & $Adj.R^2$ & Coefficient & 
     $Adj.R^2$ \\
\hline
\multirow{2}{*}{01/13-12/14}
& 0.6477      & 0.762&-&-\\
          & (5.68E-11 ***)     &  &-&          \\
\multirow{2}{*}{07/13-06/15} & 0.6582      & 0.854&-&-\\
          & (7.09E-11 ***)     &&-&            \\
\multirow{2}{*}{01/14-12/15} & 0.6436      & 0.827&0.8095&0.929\\
          & (4.76E-10 ***)     & &(2.35E-14 ***) &          \\
\multirow{2}{*}{07/14-06/16} & 0.6243      & 0.772&0.6834&0.762\\
          & (1.02E-08 ***)     & &(1.59E-08 ***) &            \\
\multirow{2}{*}{01/15-12/16}  & 0.5362      & 0.702&0.6087&0.862\\
          & (1.99E-07 ***)     & &(3.74E-11 ***)  &           \\
\multirow{2}{*}{07/15-06/17} & 0.5151      & 0.74&0.6379&0.933\\
          & (4.24E-08 ***)     &   &(1.30E-14 ***) &           \\
\multirow{2}{*}{01/16-12/17} & 0.4746      & 0.687&0.5960&0.893\\
          & (3.38E-07 ***)     &  &(2.33E-12 ***)            \\
\multirow{2}{*}{07/16-06/18} & 0.488      & 0.696 &0.5376&0.719\\
          & (2.45E-07 ***)     &&(1.02E-07 ***) &              \\
\multirow{2}{*}{01/17-12/18} &0.5243      & 0.798&0.4523&0.515\\
          & (2.59E-09 ***)     & &(4.81E-05 ***) &             \\
          \bottomrule
          
\end{tabular}

Note: the p-value of the estimate for log-transformation of monthly income per taxi driver is in the bracket (. : p $\leq$ 0.1; *: P $\leq$ 0.05; **: P $\leq$ 0.01; ***: P $\leq$ 0.001).

\end{table}

% Figure environment removed

\highlighttext{Finally, although the increase in unanticipated transitory wage variation suggests that drivers' behavior may shift from NS to RDP, it is not equivalent to assert that the RDP should explain drivers' behavior. We further conduct the wage elasticity analysis to provide a better understanding of this issue. The results are presented in Table~\ref{tab:elast1} and Table~\ref{tab:elast2}. We observe that wage elasticity yields a similar trend as the change of unanticipated transitory wage variation proportion and the change of RDP behavior for both yellow and green taxi drivers (see Figure~\ref{fig:relationship}). The wage elasticity for both yellow and green taxi drivers remains positive, which is inconsistent with the elasticity of RDP models being -1. Although the wage elasticity is positive, it implies the yellow taxi drivers reached the lowest wage elasticity at the period of January of 2016 to December 2017 (see Figure~\ref{fig:elasticity}), when 1\% increase in drivers' wage rate leads to 0.47\% increase in monthly work hours. Besides, the wage elasticity of yellow taxi stayed above 0.6 before June 2016 but dropped rapidly until December 2017. This finding corresponds to when the transitory wage variation proportion has changed, as shown in Figure~\ref{fig:relationship}. For the wage elasticity of green taxi drivers, there is a rebound at the period of July 2015 to June 2017 (see Figure~\ref{fig:elasticity}), which confirms our insights that they show revenue-maximizing behavior along with the increasing of TNC trips during this period. Since then, the wage elasticity has decreased. The similar results from the comparison of the two experiments again verify the consistency of our results.}

\added{In conclusion, the results from both the wage variation decomposition and wage elasticity of labor supply are sufficient to reject the third hypothesis and confirm that taxi drivers show RDP behavior.} The labor supply behavior of both yellow and green taxi drivers has changed since the increase of TNC trips at the beginning of 2015. At the end of study period, RDP explains over 14\% yellow taxi drivers and 20\% green taxi drivers' behavior. The RDP behavior among taxi drivers implies that an increasing number of drivers would quit the taxi market due to the loss of confidence~\cite{eliaz2014reference}. And the slightly weakened RDP behavior among yellow taxis after 2018 is likely to support this claim, where the taxi market has lost a number of active taxi drivers so that the remaining drivers are less pessimistic about the market with less competition from the same sector. The insights from both labor supply estimation and wage elasticity analyses suggest that taxi drivers show increasingly negative responses to the market over time. \highlighttext{Since pricing regulation and market entry regulation serve as one fundamental tool in reshaping drivers' labor supply behavior and maintain the balance between taxis and FHVs. Taking labor supply behavior into consideration is crucial to promote a sustainable and equitable environment in the competitive market of taxis and FHVs.}











