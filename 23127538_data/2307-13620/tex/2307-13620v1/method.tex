\section{Methodology}

\subsection{Hypotheses}
\added{The rapid growth of TNCs has significantly altered the landscape of the mobility-on-demand market and leads to a great loss in the taxi industry. To quantify how much do TNCs impact the overall labor supply and revenue of the taxi market, we propose the first hypothesis that \emph{\textbf{the rise of TNCs trips does not significantly impact the labor supply and revenue of the taxi market}}. The hypothesis is verified by a set of ordinary least square (OLS) regression and the fact that drivers' income has been significantly decreased while their work hours are barely affected contradicts the NS assumption. That is, drivers are not revenue optimizers, where their work hours are positively related to the wage rate. Because the same amount of work hours can not return the same amount of economic benefits as before. \highlighttext{Along this direction, we propose the second and third hypotheses to investigate drivers' labor supply behavior in the current market. Two hypotheses are tested based on the wage decomposition method via the partial least squares regression (PLS).} The second hypothesis is that \emph{\textbf{the increase of TNC trips does not decrease the taxi drivers' expected wage}}. The examination of the second hypothesis not only quantifies the effects of TNCs on drivers' expected wage but also provides a clue to RDP behavior among taxi drivers. \highlighttext{To inquiry whether the driver's labor supply behavior varies over time and how many drivers have RDP behavior in the current market, we propose the third hypothesis that \emph{\textbf{taxi drivers do not present RDP behavior}}.} The examination of the third hypothesis speculates taxi drivers' attitude toward the competitive ride-hailing market and implies the productivity of taxi drivers. } \highlighttext{The framework of the study is presented in Figure~\ref{fig:framework}}.
% Figure environment removed

\subsection{Ordinary least square}
\highlighttext{To understand the aggregated labor supply changes under the impact of TNCs in the first research question, we adopt OLS regression as an unbiased estimation to infer the impact of TNC trips on the overall taxi market revenue and labor supply from January 2014 to December 2018. The estimated results can be found in section 5.1.} The natural log-transformation is conducted for TNC trips and the dependent variables so that the estimated coefficient can directly measure the elasticity of corresponding explanatory variables. The following model is proposed:
\begin{equation}\quad
ln(y_i)=a_1ln(x_{i,1})+\epsilon
\end{equation}
Where $y_i$ represents the $i_{th}$ observation of the dependent variable, $x_{i,1}$ is the $i_{th}$ observation of monthly TNC trips, and $\epsilon$ is the error term that is assumed to follow a normal distribution. The dependent variables are the collection of market-level and aggregated individual-level metrics for both yellow and green taxis. At the market-level, the dependent variables include the monthly taxi fare (revenue), monthly work hours for all taxi drivers, monthly taxi trips, monthly taxi drivers, monthly taxi medallions, average minutes per taxi trip, and average daily taxi medallions. At the average individual-level, the dependent variables involve monthly income per taxi driver, monthly work hours per taxi driver, monthly work hours per taxi medallion, average daily work hours per taxi driver, and average daily work hours per taxi medallion.
\highlighttext{\subsection{Reference-dependent preference}}
\highlighttext{RDP theory and NS theory are extensively used to describe the drivers' stopping decision, which describes the driver's decision whether or not to look for an additional trip after they finish current trip.} RDP theory has its root in prospect theory and indicates loss-aversion behavior, where the loss is considered more painful than the pleasure from the same amount of gain. \highlighttext{RDP suggests that drivers are rational at setting an income target and target work hours to decide whether or not to look for an additional trip after current trip. The income target in RDP describes that the driver works more hours in low wage rate days and works fewer hours in high wage rate days.} In contrast, the NS theory relies on the revenue-maximizing behavior and is a widely-used assumption in modeling drivers' behavior. It assumes a positive correlation between the drivers' work hours and wage rate and does not consider the targeting behavior.

\highlighttext{According to Crawford and Meng~\cite{crawford2011new}}, drivers' utility $V$ comprises the consumption utility $U_1 (I)+U_2 (H)$ and the gain-loss utility $R(I,H| I^r,H^r)$  with weights of $(1-\eta)$ and $\eta$, respectively $(0\leq \eta\leq 1)$. $I$ and $H$ are daily income and daily work hours. $I^r$ and $H^r$ denote the income target and the target work hours. The complete utility function of RDP can be written as:
\begin{equation}
V(I,H|I^r, H^r)=(1-\eta)(U_1(I)+U_2(H))+\eta R(I,H|I^r, H^r)
\end{equation}

The utility function of NS is:
\begin{equation}
V(I,H)=U_1 (I)+U_2 (H)
\end{equation}

According to K\H{o}szegi and Rabin's assumption~\cite{kHoszegi2006model}, consumption utility is additively separable across income and hours, with $U_1 (I)$ increasing in $I$ and $U_2 (H)$ decreasing in $H$, and both are concave. The coefficient of loss-aversion has a constant weight of $ \gamma(\gamma\geq 1)$ relative to gains. In addition, a linear relationship between expected income and hours is $I=w^e H$, where $w^e$ is the expected wage. In Farber's study~\cite{farber2015you}, the disutility of work hours can be expressed as:
\begin{equation}
    U_2 (H)=\frac{\theta}{(1+v)} H^{(1+v)}
\end{equation}
 where $\theta $ is the coefficient of disutility of work hours and $\frac{1}{v}$ is the wage elasticity of labor supply. Given any days, the gain-loss utility $R(I, H| I^r, H^r)$ have four scenarios as shown in Table~\ref{scenario}.
\begin{table}[!h]
    \centering
    \caption{Gain-loss utility $R(I,H|I^r,H^r)$ in different scenarios}
    \begin{tabular}{lcc}
    \toprule
         &  Income gain($I>I^r$) & Income loss($I< I^r$)\\
         \hline
        Hours gain($H<H^r$) &   Scenario 1 &  Scenario 2  \\
        Hours loss($H> H^r$) &    Scenario 4  &  Scenario 3  \\
        \bottomrule
    \end{tabular}
    \label{scenario}
\end{table}

% Therefore, the function of gain-loss utility $R(I, H| I^r, H^r)$ in each scenario is presented as follows:
% \begin{equation}
%     R_1(I,H|I^r,H^r)=1*(U_1 (I)-U_1 (I^r ))+1*(U_1 (H)-U_1 (H^r ))
% \end{equation}
% \begin{equation}
%     R_2(I,H|I^r,H^r)=\gamma*(U_1 (I)-U_1 (I^r ))+1*(U_1 (H)-U_1 (H^r ))
% \end{equation}
% \begin{equation}
%     R_3(I,H|I^r,H^r)=\gamma*(U_1 (I)-U_1 (I^r ))+\gamma*(U_1 (H)-U_1 (H^r ))
% \end{equation}
% \begin{equation}
%     R_4(I,H|I^r,H^r)=1*(U_1 (I)-U_1 (I^r ))+\gamma*(U_1 (H)-U_1 (H^r ))
% \end{equation}

% The utility $V$ in four scenarios:

% \textbf{Scenario 1:} Income gain($I>I^r$) and hour gain( $H< H^r$) 
% \begin{equation}
%     V(H)=(1-\eta)\left(wH-\frac{\theta}{(1+v)} H^{(1+v)}\right)+\eta(wH-I^r )+\eta\left(-\frac{\theta}{(1+v)} H^{r^{(1+v)}}+\frac{\theta}{(1+v)} H^{(1+v)}\right )
% \end{equation}


% \textbf{Scenario 2:} Income loss($I < I^r$) and hour gain( $H< H^r$)
% \begin{equation}
%     V(H)=(1-\eta)\left(wH-\frac{\theta}{(1+v)} H^{(1+v)} \right)+\eta\gamma(wH-I^r )+\eta\left(-\frac{\theta}{(1+v)} H^{r^{(1+v)}}+\frac{\theta}{(1+v)} H^{(1+v)}\right)
% \end{equation}

% \textbf{Scenario 3:} Income loss($I <I^r$) and hour loss( $H> H^r$)
% \begin{equation}
%     V(H)=(1-\eta)\left(wH-\frac{\theta}{(1+v)}H^{(1+v)} \right)+\eta\gamma(wH-I^r )+\eta\gamma\left(-\frac{\theta}{(1+v)} H^{r^{(1+v)}}+\frac{\theta}{(1+v)} H^{(1+v)} \right)
% \end{equation}


% \textbf{Scenario 4:} Income gain($I >I^r$) and hour loss( $H>H^r$)
% \begin{equation}
%     V(H)=(1-\eta)\left(wH-\frac{\theta}{(1+v)} H^{(1+v)} \right)+\eta(wH-I^r )+\eta\gamma\left(-\frac{\theta}{(1+v)} H^{r^{(1+v)}}+\frac{\theta}{(1+v)} H^{(1+v)}\right)
% \end{equation}


The drivers’ optimal stopping decision is to maximize their utility $V$, which depends on the first-order condition of $V$ in each scenario. At the beginning of working day, if the driver works on a \enquote*{good day} with a realized wage rate higher than the expected wage ($w>w^e$), he passes through the income gain and hours gain domain (scenario 1), which is consistent with the NS model; if the driver works on a \enquote*{bad day} with a realized wage rate lower than the expected wage ($w<w^e$), the driver starts from the income loss and hour gain domain (scenario 2). Before reaching the target work hours $H<H^r$, the income loss motivates the driver to work. Thus, the driver's optimal work hours in scenario 2 is $ H=\left(\frac{(1-\eta+\eta\gamma)w}{\theta}\right)^{(1/v)}$ and wage elasticity of labor supply is $\frac{1}{v}$. Once the target work hours reaches ($H>H^r$) while the realized income is less than target income($I<I^r$), the labor supply curve of the RDP is the same as its NS curve, and the same as it is in the scenario 1 when the drivers are in \enquote*{good day}. Then, the optimal work hour is decided by $ H=\left(\frac{w}{\theta}\right)^{(1/v)}$ with wage elasticity being $\frac{1}{v}$. \highlighttext{When the income target is reached ($I=I^r$), and the realized hours are above the target work hours ($H>H^r$), the wage elasticity at this condition is -1. As K\H{o}szegi and Rabin mentioned, the RDP dominates when the elasticity is 0.} \highlighttext{When the income is sufficiently high enough ($I>I^r$) to reverse target work hours that reach the first time and work hours is greater than target work hours ($H>H^r$), which is scenario 4. The hour loss lowers the drivers' incentive to work in this condition. The optimal work hour is decided by:} $H=\left(\frac{w}{(1-\eta+\eta\gamma)\theta}\right)^{(1/v)}$ and the wage elasticity of labor supply is $\frac{1}{v}$.


% If the driver works on a \enquote*{bad day} with a realized wage rate lower than the expected wage, he then passes through the income loss and hours loss domain (scenario 3) and then income gain and hours loss domain (scenario 4). If the driver works on a \enquote*{good day} with a realized wage rate higher than the expected wage, he passes through the income gain and hours gain domain (scenario 1), which is consistent with the NS model. The drivers' optimal stopping decision is to maximize their utility $V$. Thus, drivers' decision for labor supply distribution depends on the first-order condition of $V$ in each scenario. 

% \highlighttext{For the driver in scenario 1 (\enquote*{good days}), the driver reaches their income target before the target work hours ($I>I^r, H<H^r$). The optimal work hour and wage elasticity in this scenario would be $ H=\left(\frac{w}{\theta}\right)^{(1/v)}$ and $\frac{1}{v}$. For the driver in scenario 2 (\enquote*{bad days}), the driver doesn't reach the target work hours $H<H^r$ and earns a sufficient low wage rate $w$ ($w<w^e$). The income loss motivates the driver to work. Thus, the optimal work is }$
% H=\left(\frac{(1-\eta+\eta\gamma)w}{\theta}\right)^{(1/v)}$ and wage elasticity of labor supply is $\frac{1}{v}$. 

% Once the target work hours reaches ($H>H^r$) while the realized income is less than target income($I<I^r$), the labor supply curve of the RDP is the same as its NS curve, and the same as it is in the scenario 1 when the drivers are in \enquote*{good day}. Then, the optimal work hour is decided by $ H=\left(\frac{w}{\theta}\right)^{(1/v)}$ with wage elasticity being $\frac{1}{v}$. \highlighttext{When the income target is reached ($I=I^r$), and the realized hours are above the target work hours ($H>H^r$), the wage elasticity at this condition is -1. As K\H{o}szegi and Rabin mentioned, the RDP dominates when the elasticity is 0.}



% \highlighttext{When the wage rate is sufficiently high enough ($w>w^e$) to reverse target work hours that reach the first time and work hours is greater than target work hours ($H>H^r$), which is scenario 4. The hour loss lowers the drivers' incentive to work in this condition. The optimal work hour is decided by:} $H=\left(\frac{w}{(1-\eta+\eta\gamma)\theta}\right)^{(1/v)}$ and the wage elasticity of labor supply is $\frac{1}{v}$. 
      






%\highlighttext{ The optimal stopping decision in each condition is presented below:} %The labor supply curve under RDP utility (dotted curve) and the labor supply curve under NS utility (solid curve) are shown in Figure~\ref{fig3}.

%% Figure environment removed

% \begin{enumerate}
%     \item \highlighttext{For a sufficiently low wage rate $w$ ($w<w^e$), the driver doesn't reach the target work hours $H<H^r$ yet (scenario 2). The income loss motivates the driver to work. The optimal work hour under this scenario is decided by:}
%         \begin{equation}
%             H=\left(\frac{(1-\eta+\eta\gamma)w}{\theta}\right)^{(1/v)}
%         \end{equation}

%     The wage elasticity of labor supply is $\frac{1}{v}$. 

%     \item \highlighttext{When the target work hours have been reached for the first time and $w<w^e$, the optimal stopping decision in this condition is decided by:}
%         \begin{equation}
%             H=H^r=\frac{I^r}{w^e}
%         \end{equation}
    
    
%     \item \highlighttext{Once the target work hours are reached ($H>H^r$) while the realized income is less than target income($I<I^r$), the labor supply curve of the RDP is the same as its NS curve, and the same as it is in the scenario 1 when the drivers are in \enquote*{good day}. Then, the optimal work hour is decided by:}
%         \begin{equation}
%             H=\left(\frac{w}{\theta}\right)^{(1/v)}
%         \end{equation}
        
%     \item \highlighttext{When the income target is reached ($I=I^r$), and the realized hours are above the target work hours ($H>H^r$), the wage elasticity at this condition is -1. As K\H{o}szegi and Rabin mentioned, the RDP dominates when the elasticity is 0. 
%     The optimal work hours at this condition are decided by:}
%         \begin{equation}
%             H=\frac{I^r}{w}
%         \end{equation}


%     \item \highlighttext{When the target work hours have been reached at the second time and the wage rate is high enough ($w>w^e$), the optimal stopping decision is defined by:}

%         \begin{equation}
%             H=H_r=\frac{I_r}{w_e}
%         \end{equation}


%     \item \highlighttext{When the wage rate is sufficiently high enough ($w>w^e$) to reverse target work hours that reach the first time and work hours is greater than target work hours ($H>H^r$), which is scenario 4. The wage elasticity of labor supply is $\frac{1}{v}$. The hour loss lowers the drivers' incentive to work in this condition. The optimal work hour is decided by:}
%         \begin{equation}
%             H=\left(\frac{w}{(1-\eta+\eta\gamma)\theta}\right)^{(1/v)}
%         \end{equation}

        
% \end{enumerate}

 According to the preferred personal equilibrium theory~\cite{KszegiUtility}, the driver is more likely to work if there is a high probability that the future wage rate will increase. While the labor supply of taxi drivers would be negatively related to the wage rate when they face uncertainty~\cite{crawford2011new}. Thus, the gain-loss utility in RDP is only related to unanticipated transitory changes in wage. With the rise of TNCs trips, the driver will go through the scenarios in \enquote*{bad day} due to the competition, where the stopping decision for whether or not to look for an additional trip after the current trip is mainly determined by the income target~\cite{crawford2011new}. In our case, the basic idea is to investigate whether the taxi drivers decrease their expected wage under the growth of TNC trips based on the fact that taxi drivers' monthly work hours have barely affected while their monthly income decreases. Moreover, we are interested in how taxi drivers may respond to the current market, which is how many taxi drivers' behavior can be explained by the RDP and the NS. We follow the similar wage decomposition approach as applied in Farber's study~\cite{farber2015you} to investigate the change proportion of anticipated and unanticipated transitory wage variation.
 
 %This indicates that taxi drivers' labor supply in a time unit will be positively related to anticipated transitory wage changes. 

%Later, Farber~\cite{farber2015you} proposed the wage decomposition method to estimate the drivers' stopping decision based on the taxi dataset from 2009 to 2013. 




\subsection{Wage decomposition and wage elasticity}
The wage decomposition method investigates the second and third hypotheses and quantifies drivers' anticipated and unanticipated wage variations. As indicated by previous studies~\cite{kHoszegi2006model,farber2015you,crawford2011new}, the anticipated wage variation reflects the NS behavior and the RDP behavior is related to the proportion of unanticipated wage variation. Intuitively, the drivers' anticipated wage is expectation-based, which comes from the knowledge of the taxi market. The TNC trips, the improvement surcharge, the extra surcharges, and time variation will be the main aspects that impact drivers' judgments on the wage. Therefore, \highlighttext{the wage rate (wage per day) is decomposed by a two-stage process in the study}. In the first stage, we regress the natural log transformation of the average wage rate on the improvement surcharge indicator and extra surcharge indicator by PLS. These two surcharge indicators capture the fixed anticipated wage variation. Note that the residual of the first stage contains both anticipated and unanticipated transitory wage variation. In the second stage, we conduct PLS on the residual with the year indicators, month indicators, and the natural log transformation of monthly average TNC trips. The TNC trips over time affect the taxi drivers' judgment on temporal earning opportunities and expected wage. The variance captured by the second stage describes the anticipated transitory wage variation, and the remaining residual therefore represents the unanticipated transitory wage variation. Thus, the wage decomposition approach contains three parts: fixed anticipated wage variation, anticipated transitory wage variation, unanticipated transitory wage variation. The fixed anticipated wage and anticipated transitory wage are involved by the expected wage proposed in the second research question. The proportions of anticipated transitory wage variation and unanticipated transitory wage variation will be used to estimate the proportion of the RDP behavior and the NS behavior among taxi drivers, \highlighttext{and the estimates can be found in section 5.2. }

As for the data preparation, we take the data from January 2013 to December 2014 as the base year group for the yellow taxi estimation and January 2014 to December 2015 for green taxi estimation due to the limited number of TNC trips during this period. We further introduce two experiments to ensure a sufficient number of data samples and testify the consistency of our results. The dataset is grouped as 30 consecutive months for yellow taxis and as 24 consecutive months for green taxis. Two experiments follow the same wage decomposition approach as mentioned above. That is, the tested performance metric is the natural log transformation of drivers' average wage rate, and explanatory variables are the set of time indicators, natural log transformation of TNC trips, and the fare surcharge indicators. Thus, in experiment \uppercase\expandafter{\romannumeral1}, we have eight data groups for yellow taxis and six data groups for green taxis, which include: $ 01/13-06/15, 07/13-12/15, 01/14-06/16, 07/14-12/16, 01/15-06/17, 07/15-12/17, 01/16-06/18$, and $07/16-12/18$. In experiment \uppercase\expandafter{\romannumeral2}, we get nine data groups for yellow taxis and seven data groups for green taxis, which include: $01/13-12/14, 07/13-06/15,01/14-12/15, 07/14-06/16, 01/15-12/16, 07/15-06/17, 01/16-12/17, 07/16-06/18$, and $01/17-12/18$. Finally, the wage decomposition method is conducted on each of the groups.

Besides, we measure taxi drivers' wage elasticity as the percentage changes of monthly work hours to the changes in monthly wage. Therefore, the estimated coefficient describes the wage elasticity of the taxi labor supply. The datasets and experiments included in the analyses of wage elasticity are the same as the wage decomposition approach.

\subsection{Partial least squares}
\highlighttext{From real-world observations, drivers' wage rate expectation will be affected among a set of temporal indicators, fare surcharge indicators, and the TNC trips factor. In order to avoid overfitting issues and multicollinearity among variables, we apply the PLS regression to decompose their wage rate to gain an accurate estimate of the interactions between drivers' expected wage and the set of explanatory variables, and the results can be obtained in section 5.2. PLS regression was developed in the 1960s~\cite{wold2004partial} as an econometric method and is particularly suited for analyzing the system of equations that contain a vast number of indicator variables, in which the maximum likelihood covariance-based tools reach their limits.} The basic idea of PLS is to extract factors that account for as much variance in the response variables as possible~\cite{rashid2015methodological}. PLS considers both factors space and response space and seeks the dominant direction for the successive pairs between two spaces. Mathematically, it establishes a linear combination of the columns of $X$ and $Y$ such that their covariance is maximum~\cite{Haenlein2004AB}:
\begin{eqnarray}
X=TP^T+E\\
Y=UQ^T+F \\
u_i=b_it_i
\end{eqnarray}
where $P$ and $Q$ are termed as the loading matrices for $X$ and $Y$. $T$ and $U$ are the projections(scores) for $X$ and $Y$, respectively. $E$ and $F$ are the error terms and are assumed to be independent and identically distributed random normal variables. $u_i$ and $t_i$ represents the $i_{th}$ component of $U$ and $T$, and $b_i$ is the regression coefficient. In our model, $Y$ is the natural log-transformation of average daily income. $X$ is the set of variables, including the natural log-transformation of TNC trips, temporal indicators for years and months, the improvement surcharge indicator, and the surcharge indicator (during rush hours and overnight hours).







