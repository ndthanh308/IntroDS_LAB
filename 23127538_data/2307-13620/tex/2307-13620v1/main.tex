
%
% Transportation Research Board conference paper template
% version 1.1
% 
% David R. Pritchard, http://davidpritchard.org
%   1.0 - Mar. 2009
%   1.1 - Sep. 2011, fixes for captions
% PAGE LAYOUT
%------------------------------------------

% Custom paper settings...
%\documentclass[review]{elsarticle}
\documentclass[12pt]{elsarticle}
%\renewcommand{\baselinestretch}{1.5} 
\usepackage{latexsym}
\usepackage{booktabs}
\usepackage{geometry}
\usepackage{amsthm}
\usepackage{stmaryrd}
\usepackage{caption}
\usepackage[hyphens]{url}



\usepackage{setspace}
%\onehalfspacing
\doublespacing



%\usepackage{subcaption}

\usepackage{subfig}
\usepackage{color}
\usepackage[export]{adjustbox}
\usepackage{listings}
\usepackage{epstopdf} 
\lstset{language=Matlab}
\lstset{breaklines}
\lstset{extendedchars=false}
\usepackage{todonotes}
\usepackage{mathtools}
\usepackage[ruled,linesnumbered]{algorithm2e}
\DeclarePairedDelimiter\ceil{\lceil}{\rceil}
\DeclarePairedDelimiter\floor{\lfloor}{\rfloor}
\usepackage{graphicx}
\usepackage{epstopdf}
\usepackage{multirow}
\usepackage{paralist}
\usepackage{amsmath,amsthm,amssymb}
\usepackage{setspace}
\usepackage{float}
\usepackage{morefloats}
\usepackage[]{algorithm2e}
\usepackage{mdwlist}
\usepackage{booktabs} % FvW: makes tables look nicer
\usepackage{amsfonts}
\usepackage{paralist}
\usepackage{soul}
\newcommand{\be}{\begin{equation}}
\newcommand{\ee}{\end{equation}}

\usepackage[final]{changes}
\newcommand{\RNum}[1]{\uppercase\expandafter{\romannumeral #1\relax}}
\theoremstyle{remark}
\newtheorem*{claim}{Claim}
\usepackage{lineno,hyperref}
\modulolinenumbers[1]
\usepackage{geometry}
\usepackage{subfig}
\usepackage{color}
\usepackage[export]{adjustbox}
\usepackage{listings}
\usepackage{epstopdf} 
\usepackage{csquotes}
\lstset{language=Matlab}
\lstset{breaklines}
\lstset{extendedchars=false}
\geometry{verbose,letterpaper,tmargin=1in,bmargin=1in,lmargin=1in,rmargin=1in}

% \newcommand{\highlighttext}[1]{\textcolor{red}{#1}}
\newcommand{\highlighttext}[1]{#1}


% HEADINGS
%------------------------------------------
\renewcommand*{\refname}{\uppercase{References}}

%------------------------------------------
% Adjust lists a little. Not quite perfectly fitting TRB style, but vaguely
% close at least.
\usepackage{enumitem}
\setlist[enumerate]{itemsep=0mm}
\setlist[1]{labelindent=0.5in,leftmargin=*}
\setlist[2]{labelindent=0in,leftmargin=*}

% CAPTIONS
%------------------------------------------
% Get the captions right. Authors must still be careful to use "Title Case"
% for table captions, and "Sentence case." for figure captions.
%\usepackage{enumerate}

% \usepackage{ccaption}
\usepackage{amsmath}
\usepackage{booktabs,amsfonts,dcolumn}
\makeatletter
\renewcommand{\fnum@figure}{\textbf{FIGURE~\thefigure} }
\renewcommand{\fnum@table}{\textbf{TABLE~\thetable} }
\makeatother

% \captiontitlefont{\bfseries \boldmath}
% \captiondelim{\;}
%\precaption{\boldmath}


% FONTS
%------------------------------------------
% Three options for fonts. I prefer Times for text and Computer Modern for
% math.

% Times for text, Computer Modern for math
\usepackage{times}
% Times for text and math
%\usepackage{pslatex}
% Times for text and math
%\usepackage{times,mathptmx} 

% Some pdf conversion tricks? Unsure.
\usepackage[T1]{fontenc}
\usepackage{textcomp}
\usepackage{todonotes}
\usepackage{multirow}
% CITATIONS
%------------------------------------------
% TRB uses an Author (num) citation style. I haven't found a way to make
% LaTeX/Bibtex do this automatically using the standard \cite macro, but
% this modified \trbcite macro does the trick.

% TODO: sort&compress option?
\usepackage{float}
\usepackage{lineno}


\usepackage{url}
%% Define a new 'leo' style for the package that will use a smaller font.
\makeatletter
\def\url@leostyle{%
  \@ifundefined{selectfont}{\def\UrlFont{\sf}}{\def\UrlFont{\small\ttfamily}}}
\makeatother
%% Now actually use the newly defined style.
\urlstyle{leo}




%\pagewiselinenumbers

\newcommand{\trbcite}[1]{\citeauthor{#1} ({\it \citenum{#1}})}
\setcitestyle{round}

\begin{document}

% Outcomment only when entries are known. Otherwise leave as is and 
%   default values will be used.
	\begin{frontmatter}
        \title{Impact of Transportation Network Companies on Labor Supply and Wages for Taxi Drivers}
         \author[pu_address]{Lu Ling}
 		\ead{ling58@purdue.edu}
 		\author[ua_address]{Xinwu Qian}
 		\ead{xinwu.qian@ua.edu}
 				%% or include affiliations in footnotes:
 		\author[pu_address]{Satish V. Ukkusuri}
 		\cortext[mycorrespondingauthor]{Satish V. Ukkusuri is the corresponding author.}
 		\ead{sukkusur@purdue.edu}
		\address[pu_address]{Lyles School of Civil Engineering, Purdue University, 
  West Lafayette, IN, 47907}
      \address[ua_address]{Department of Civil, Construction and Environmental Engineering, The University of Alabama, Tuscaloosa, AL, United States, 35487}
  
		\begin{abstract}
		    While the growth of TNCs took a substantial part of ridership and asset value away from the traditional taxi industry, existing taxi market policy regulations and planning models remain to be reexamined, which requires reliable estimates of the sensitivity of labor supply and income levels in the taxi industry. This study aims to investigate the impact of TNCs on the labor supply of the taxi industry, estimate the wage elasticity, and understand the changes in taxi drivers' work preferences. We introduce the wage decomposition method to quantify the effects of  TNC trips on taxi drivers' work hours over time, based on taxi and TNC trip record data from 2013 to 2018 in New York City. The data are analyzed to evaluate the changes in overall market performances and taxi drivers' work behavior through statistical analyses, and our results show that the increase in TNC trips not only decreases the income level for taxi drivers but also discourages their willingness to work. We find that 1\% increase in TNC trips leads to 0.28\% reduction of monthly revenue of the yellow taxi industry and 0.68\% decrease in monthly revenue of the green taxi industry in recent years. More importantly, we report that work behavior of taxi drivers shifts from the widely accepted neoclassical standard behavior to the reference-dependent preference (RDP) behavior, which signifies a persistent trend of loss in labor supply for the taxi market and hints at the collapse of taxi industry if the growth of TNCs continues. In addition, we observe that yellow and green taxi drivers present different work preferences over time. Consistently increasing RDP behavior is found among the yellow taxi drivers. Green taxi drivers were initially revenue maximizers but later turned into income targeting strategy.
		\end{abstract}
		
		\begin{keyword}
			Taxi labor supply; Transportation network companies; Wage elasticity; Reference-dependent preference
			%\MSC[2019] 00-01\sep  99-00
		\end{keyword}
		
    \end{frontmatter}

	% \linenumbers
	% \pagewiselinenumbers % comment out for final manuscript; comment out if no line numbers on title page
	\thispagestyle{empty}

% Figure environment removed

\section{Introduction}
Automatic 3D reconstruction of clothed humans using image inputs has gained increasing significance due to its potential applications in a wide array of AR/VR scenarios. High-fidelity reconstructions typically depend on sophisticated capture systems, which are developed with dense camera arrays~\cite{collet2015high,joo2015panoptic,joo2018total}, programmable light-stages~\cite{Vlasic2009, guo2019relightables}, and depth sensors~\cite{newcombe2011kinectfusion,DoubleFusion,BodyFusion,dou2016fusion4d,newcombe2015dynamicfusion}. However, stringent capture environments equipped with complex hardware pose significant challenges for consumer-level applications.


In this context, considerable research effort has been dedicated to developing methods that allow for more flexible capture configurations, such as utilizing a few RGB inputs. Among these works, learning implicit functions \cite{iccv2020PIFu, saito2020pifuhd, hong2021stereopifu} has proven effective in achieving highly detailed reconstructions by integrating the advancements of deep neural networks. These methods employ large multi-layer perceptrons (MLPs) to predict the occupancy probability or truncated signed distance function (TSDF) value of every queried 3D point based on its associated local feature, which is extracted from images. They can recover a continuous surface at arbitrary resolutions without topology restrictions.


However, in typical MLP-based implicit networks, the occupancy or TSDF value at each location is solved independently with planar image features, rendering them less capable of addressing challenging cases such as occlusions. Consequently, these methods suffer from generalization and robustness issues, particularly when tackling strong occlusions caused by large motion or multiple interacting humans. 
Some follow-up studies  \cite{zheng2021deepmulticap,zheng2021pamir,huang2020arch} utilize an extra geometric model, SMPL~\cite{Loper2015}, to improve robustness by introducing strong shape priors. 
Their success typically relies on the assumption of geometrical similarity \cite{huang2020arch} between the shape prior and target reconstruction, making them intractable for handling complex cases with loose clothes and sensitive to errors in SMPL model fitting.



%\ping{this paragraph sounds like `TSDF is better than MLP/SMPL, and we use TSDF to solve the problem'. But in Sec 3, we are telling a different story, saying `MLP needs a 3D convolutional encoder'. We need to make these two sections consistent.}\sicong{I think in this paragraph we claim that the TSDF}


%We opt for Trucated Signed Distance Funtion (TSDF) volumetric representations as they are naturally suitable for convolution operations, which have shown remarkable performance for learning hierarchical features on 2D visual perception tasks \cite{SunXLW19}. 
%Meanwhile, TSDF also describes the gradual geometry change around shape surface, which is not reflected by occupancy volume. 

We instead revisit the 3D volumetric representation and resort to 3D convolutional neural networks (CNNs) for feature learning, due to their impressive performance in feature learning and the ability to incorporate spatial context. However, volumetric methods and 3D convolution involve discretization, which might raise concerns regarding whether a discretized volume can preserve subtle geometric details as continuous representations learned in implicit functions. We investigate the relationship between volume resolution and quantization error on synthetic data by converting target mesh objects to TSDF volumes, as shown in Figure~\ref{fig:quantization_error}. We observe that the quantization errors are significantly reduced by increasing volume resolution and become nearly negligible when reaching a relatively high resolution (e.g., 512 or higher). In other words, achieving fine-detailed reconstruction is not supposed to be restricted by the use of volume representations as long as a proper volume resolution is utilized. Therefore, we present a method with high-resolution feature volumes, e.g., 256 and 512, while traditional volumetric methods \cite{varol18_bodynet,gilbert2018volumetric} are often limited to much lower resolutions, such as 32 or 128.



On the other hand, an increase in volume resolution may lead to a cubic growth of memory overhead \cite{8100085}. Reducing memory costs while guaranteeing the granularity of volumetric representations is necessary for pursuing high-quality reconstruction. Thus, we adopt a coarse-to-fine approach and cull away irrelevant voxels to build a sparse high-resolution feature volume. At the coarse level, the network computes an initial TSDF by applying a U-Net with sparse 3D CNN \cite{3DSemanticSegmentationWithSubmanifoldSparseConvNet} on the sparse feature volume, which is carved by a visual hull. Through our experiments, it turns out that more than 95\% of the volume grids are discarded by the visual hull culling, making the sparse 3D CNN efficient. At the fine level, the network focuses on a narrow band near the zero-level set of the initial TSDF and discretizes the narrow band with smaller voxels. By employing this narrow-band culling, we further shrink the sampling space, resulting in a relatively small range of grid numbers (usually 300K--500K in our experiments) even with a high volume resolution of 512. The remaining voxels in the narrow band are associated with features that fuse high-frequency information from the computed normal maps upon the low-frequency shape from the coarse level to compute the TSDF at high resolution. The final mesh is then extracted from the TSDF using the Marching-Cube algorithm ~\cite{Lorensen87marchingcubes}.
% Different from the u-net sturcture to preserve global topology context, we then apply a shallow 3dcnn to compute the final TSDF $D_{final}$ which contain more local geometry detail.




% \ping{this paragraph can be expanded. It is an important contribution and often ignored by other works. stress on the novel idea of regressing blending weights instead of colors}

In addition to geometry, high-quality mesh texture is also a crucial factor contributing to visual appearance. Directly computing a color field in 3D space, as in \cite{iccv2020PIFu}, struggles to capture high-frequency texture details, while the neural radiance field (NeRF) \cite{yu2020pixelnerf} or the DoubleField~\cite{shao2022doublefield} require expensive per-instance optimization and are often unstable for sparse input images. In contrast, we adopt an image-based rendering approach to compute a texture atlas map, which is efficient and widely supported in existing computer graphics tools. 
Specifically, we compute a blending weight at each 3D point on the mesh surface to determine its color as a weighted average of the colors at its image projections. The blending weights can be computed at a relatively coarse resolution, e.g., 512 volume resolution in our case, and leave texture details to the high-resolution images, such as 1K or 2K. Unlike previous methods that generate blurry texturing results under sparse input, our method generalizes well on both synthetic and real data with just a few input views. 
Figure~\ref{fig:teaser} shows two examples reconstructed by our method. Despite the challenging garment, pose, and occlusion, our method recovers faithful shape, normal, and texture on the right.

%with a wide variety of poses and clothing styles, and it is also adaptive to handle input image with arbitrary resolutions.
%\sicong{For this concern we claim that when the resolution of dicretized volume meets certain threshold (which is 256 in our experiment), the quantization error can be neglected.} 



In summary, the main contributions of this paper are as follows:
\begin{itemize}
\vspace{-0.1in}
  \item 
  We revisit the 3D volumetric representation and demonstrate that it can support clothed human reconstruction with equal or even better performance compared to implicit representation. 
  \item 
  We develop a memory and computation-efficient method for high-resolution volumetric reconstruction using sophisticated sparse 3D CNN, coarse-to-fine estimation, and voxel culling by visual hull and narrow bands. 
  \item 
  We introduce a novel method to compute a texture atlas map, which captures rich appearance details from high-resolution input images.
  \item 
  We achieve impressive results on standard benchmark datasets Twindom and MultiHuman, significantly reducing the point-2-surface (P2S) precision to approximately 0.2cm from just six input views, with more than $50\%$ error reduction compared to the state-of-the-art methods, including DoubleField~\cite{shao2022doublefield} and PIFuHD~\cite{saito2020pifuhd}.
\end{itemize}
\section{Literature}
Though few studies have investigated the impact of TNCs on the labor supply of the taxi industry, there exists a broad literature on the discussion of the labor supply behavior of taxi drivers. 

From the macro-management perspective, researchers~\cite{contreras2017effects,cetin2013economic,ling2018analyzing} primarily focused on productivity improvements and taxi medallion utilization analyses. Meanwhile, there is an ongoing debate on the underlying behavioral assumptions for modeling the taxi driver objectives in the research community~\cite{douglas1972price,beesley1983information,yang2010nonlinear,yang2005regulating}. In an early work, Camerer et al.~\cite{camerer1997labor} presented a regression of log-transformation of drivers' daily work hours on log-transformation of average hourly earnings and characterized taxi drivers as having RDP behavior with a daily income target and quit driving once the target is met. Then, K\H{o}szegi and Rabin~\cite{kHoszegi2006model} proposed a theory of preferred personal equilibrium addressing that the anticipated wage will impact drivers' stopping decision (the drivers decide whether or not to look for an additional trip after the current trip). Besides, they clarified the anticipated wage is not status quo based but rational-expectation based~\cite{KszegiUtility}. Based on K\H{o}szegi and Rabin's theory, \highlighttext{Crawford and Meng~\cite{crawford2011new}} discussed a simplified point-based income target as well as target work hours utility function to model drivers' stopping behavior. In the study, they pointed out that taxi drivers' stopping decision is significantly related to hours but income as mentioned by Farber~~\cite{farber2005tomorrow}. In this direction, studies have also found that the labor supply curve of taxi drivers appears to slope downward~\cite{doran2014long,crawford2011new}, and the consensus of these works is that taxi drivers are targeting earners. For the NS model supporters, Solow~\cite{solow1956contribution} first proposed a positive relationship between labor supply and aggregated income. Later, Farber~\cite{farber2005tomorrow,farber2008reference,farber2015you} initially presented evidence for the existence of reference-dependence, but lately revised his study by analyzing the taxi supply pattern in weather-related conditions (e.g., rain and snow). And he deduced that taxi drivers are utility maximizers where drivers work more hours when the wage is high and work fewer hours when the wage is low. \highlighttext{Besides, Crawford and Meng~\cite{crawford2011new}} agreed that changes in the anticipated transitory wage would also be neoclassical. Ashenfelter et al.~\cite{ashenfelter2010shred} found an elasticity of -0.2 in response to the fixed fare change on the labor supply of NYC taxi drivers. Thakral et al.~\cite{thakral2017daily} speculated that more recent income has a stronger impact on drivers ending their shifts rather than income earned earlier. Buchholz et al.~\cite{buchholz2016semiparametric} presented that income-targeting behavior is associated with frequent shorter shifts, and the NS assumption is more suitable for explaining longer shifts. Recently, Frechette~\cite{frechette2019frictions} presented evidence of the reduction of market density when street-hailing and TNCs co-exist in the market based on the NS assumption.

These studies have provided valuable insights into patterns and mechanisms of labor supply in the taxi market. However, the labor supply analyses in these studies assume that the taxi market is still monopolistic, and the labor supply behavior of taxi drivers remains as the stationary NS assumption. However, the current taxi industry has undergone tremendous changes due to the competition from TNCs. And there is an emerging need to examine how the labor supply of the taxi market is affected and\added{ how taxi drivers' behavior changes} due to the growth of TNCs. In this study, we apply the datasets for both TNCs and taxis from January 2013 to December 2018 in NYC to address these issues. The taxi data depict \added{how the labor supply of taxi market changes and drivers' labor supply behavior varies over time.} The TNCs data are then used to mine the most affected aspects of taxi labor supply due to the rise of TNC trips. 

\lstMakeShortInline[columns=fixed]@
% Figure environment removed
\lstDeleteShortInline@

In this section, we describe how we collect examples for learning repair strategies without any version-controlled data. Specifically, we first detect \safeprogs and corresponding witnesses using \sawitnessfull (witnesses are sanitizers and guards that protect from vulnerabilities)  in Section~\ref{subsec:sa-witness}. Using these witness annotations, we generate unsafe programs and \textit{edits} from the \safeprog using a \textbf{witness-removal} step (Section ~\ref{subsec:witness-removal}). In the following, we define terminology for the \astree  data-structure we operate on. 


\astree refers to the abstract syntax tree representation of programs, augmented with data flow edges and annotations for sources, sinks, sanitizers, guards, witnesses etc. 
An \astree is a five-tuple 
$\langle \mathcal{N},\mathcal{V},\mathcal{T},\mathcal{E}, \mathcal{A} \rangle$, where:
\begin{enumerate}
\item
$\mathcal{N}=\{\mathit{id}_0,\ldots\mathit{id}_n\}$  is a set of nodes, where  $\mathit{id_i}\in\mathbb{N}$ for 
$ 0 \leq i \leq n$.
\item
$\mathcal{V}$ is a map from nodes to program snippets
represented as strings. For a node $n$, we have that $\mathcal{V}(n)$ is a string representing the code snippet associated with $n$
\item
$\mathcal{T}$ is a map from nodes to their types defined by 
 \sa~\cite{codeqlast}. For example, \callexpr is the type of a node representing a function call, \indexexpr is the type of a node representing an array index, and \blockstmt is the type of a node representing a basic block of statements.
\item
$\mathcal{E}$ is a set of directed edges.
Each edge is of the form $(n_1,n_2,\edgetype,z)$, where
$n_1$ is a source node, $n_2$ is a target node, 
$\edgetype \in \{\T{SynParent}, \T{SynChild}, \T{SemParent},
\T{SemChild} \}$ denotes the relationship from 
$n_1$ to $n_2$, as one of syntactic parent, syntactic child, semantic parent or semantic child,
and $z\in\mathbb{Z}$ is the index of $n_2$ among $n_1's$ children if this edge is a child edge, and $-1$ if the edge is a parent edge. 
\item
$\mathcal{A}$ is a set of annotations associated with each node. The annotations are from the set $\{\T{source},
\T{sink},\T{sanitizer},\T{guard}$,\T{witness}\}. We also refer to annotations using predicates or relations. For instance, for a node $n$, if an annotation  $\T{source}$ is present, we say that
the predicate $\T{source}(n)$ is true.
\end{enumerate}

%\setlength{\grammarindent}{5em} % increase separation between LHS/RHS

% Figure environment removed



A {\em traversal} or a {\em path} in an \astree is a sequence of edges $e_0,\ldots,e_{i-1},e_i,\ldots ,e_k$ such that the target node of $e_{i-1}$ is also the source node of $e_i$, for all $i\in\{1,\ldots,k\}$. That is, $e_{i-1}$ is of the form $(\_,n,\_,\_)$ and $e_i$ is of the form $(n,\_,\_,\_,\_)$. The source node of $e_0$ is the source of this path and the target node of $e_k$ is the target of the path.


\lstMakeShortInline[columns=fixed]@
%Note that these additional edges can capture long-range dependencies in programs. E.g. edge 4 in Figure ~\ref{fig:unsafememberex} links two nodes across the function boundaries. 
Figure~\ref{fig:example1-pdg} depicts a partial \pdg corresponding to the unsafe program in Figure~\ref{fig:unsafememberex}. Each oval corresponds to an \astree-node containing a type $\tau$ and an associated value. The dark edges denote the syntactic child edges. For example, the oval with value @foo(data)@ is an \astree-node with type \callexpr and has two children -- @foo@ and @data@, both with the type \varexpr. 
%Similarly, the \blockstmt node on the top refers to the function body between Line~\ref{lst:line:handlers-run} and Line~\ref{lst:line:handlers-run-end} in Figure ~\ref{fig:unsafememberex}. As the body of a function block can contain a variable number of children, we link to @handlers[callerId](data);@ as the k-th child of the \blockstmt. 
The semantic child edges are at the bottom in cyan. These edges correspond to the ones depicted in cyan in Figure ~\ref{fig:unsafememberex}. 
\lstDeleteShortInline@

%TODO:FIX THIS

%With this simplification, 
If $\prog$ is an \pdg then
we use  $\prog.\mathtt{source}$ to denote the source node, $\prog.\mathtt{sink}$ to denote the sink node, and $\prog.\mathtt{witness}$ to denote the witness node.
If the program has several sources, sinks and sanitizers then we generate a separate \pdg for each $(\mathtt{source},\mathtt{witness},\mathtt{sink})$ triple.
For a node $n$, its syntactic parent is $n.\mathtt{parent}$, syntactic children are $n.\mathtt{children}$, semantic parent is $n.\mathtt{semparent}$, and semantic children are $n.\mathtt{semchildren}$.

%\input{ql.tex}

\subsection{Static Analysis Witnessing}
\label{subsec:sa-witness}

\newcommand{\DMethodjudge}[1]{\texttt{#1(}\checknextarga}

% Figure environment removed

%\naman{TODO - sell this more as technique to work with any \sa tool ; our master query is a general framework implemented in \codeql that can work for any vulnerability -- easily extendable to other languages }
In this section, we show how to repurpose \sa tools to generate witnesses.
\sa tools perform dataflow analysis to check for rule-violations in programs. They use pattern matching to identify known sources, sinks, sanitizers, and guards. For commercial tools, these patterns are implemented (and continuously updated) manually by developers and encode this domain knowledge. Next, 
%these patterns are used to detect sources, sinks, sanitizers, and guards in programs and
\sa checks if there exists a flow between a source and a sink that does not cross a sanitizer or guard. We capture this formally in Figure~\ref{fig:judgements} (top two rules), and explain the notation used in it below.

\sa tools encode domain knowledge about the vulnerability by annotating nodes as \T{Source}, \T{Sink}, \T{Sanitizer}, and \T{Guard}. %These relations operate on the set of dataflow nodes in the programs.
So \DMethod{Source}{\I{n}}\ is true iff the node \I{n} is a \textit{source} node for a vulnerability. Next, \sa tools perform dataflow analysis by defining the relation \DMethod{SemChild}{$n_1$}{$n_2$}\ which is true iff there is a \taintpropedge between $n_1$ and $n_2$. Then the \DMethod{Vulnerability}{$n_1$}{$n_2$}\ relation can be defined as:
\begin{enumerate}
    \item $n_1$ and $n_2$ are source and sink nodes (\DMethod{Source}{$n_1$}\ and \DMethod{Sink}{$n_2$}\ are true)
    \item There exists a \textit{path} between $n_1$ and $n_2$ which is free of sanitizers or guards (\DMethod{SanGuardFree*}{$n_1$}{$n_2$}\ is true). A path is free of sanitizers and guards iff every \textit{edge} in the \textit{path} is free of sanitizers and guards. An edge between $n_1$ and $n_2$ is considered free of sanitizers and guards (\DMethod{SanGuardFree}{$n_1$}{$n_2$}\ is true) iff $(n_1, n_2, \_, \T{SemChild}) \in \mathcal{E}$ and neither of $n_1$ or $n_2$ is a sanitizer or a guard
\end{enumerate}

Here, we make the following observation - \emph{this domain knowledge present in these annotations and relations is helpful beyond just detecting vulnerabilities}. For instance, simply using the sanitizer relation allows us to query the different kinds of sanitizers domain experts have specified. We use this observation to discover \emph{\safeprogs} i.e., programs having a source, sink, and a sanitizer or guard that \textit{blocks} the \taintprop or, in simpler terms, make the program safe. In addition, we also detect the corresponding sanitizers or guards in the programs and refer to them as \textit{witnesses} because they serve as the evidence of making the program safe. We call this procedure \sawitnessfull (abbreviated as \sawitness). 
We define this as the \T{Witness} relation in Figure~\ref{fig:judgements} (bottom two rules). Specifically, \DMethod{Witness}{$n_1$}{$n_3$}{$n_2$}\ is defined as:
\begin{enumerate}
    \item $n_1$ and $n_2$ are source and sink nodes (\DMethod{Source}{$n_1$}\ and \DMethod{Sink}{$n_2$}\ are true)
    \item There exists a node $n_3$ such that it satisfies \DMethod{SanGuardInMid}{$n_1$}{$n_3$}{$n_2$}. \DMethod{SanGuardInMid}{$n_1$}{$n_3$}{$n_2$}\ is true iff there exists a \T{SemChild}
    %\naga{notation for flow inconsistent with (2) above} 
    path between $n_1$, $n_3$, between $n_3$ and $n_2$, with the additional constraint of $n_3$ being a sanitizer or guard. 
\end{enumerate}

The difference between the \T{Vulnerability} relation (which \sa populates) and \T{Witness} relations (which we want to find) is highlighted in {\color{red} red} and {\color{ForestGreen} green}. Notice that while defining the \T{Witness} relation, we simply use the existing relations that define the \T{Vulnerability} relation. Thus, we argue that \sawitness can be implemented on top of \sa by using the intermediate relations that \sa is computing.
%for every pair of source and sink, they track taint through a taint-flow analysis. If there is a flow from a source to a sink that does not go through a sanitizer or guard, then the source-sink pair is reported as vulnerable.

%We make the following observation - \emph{the patterns defined by experts encodes domain knowledge which can be used for use cases beyond just detecting vulnerabilities}. For instance, we can use the sanitizer patterns to search for all sanitizers in source-code. In this work, we use this idea to detect \safeprogs, which we define as programs having a source, sink, and a sanitizer or guard that blocks the \unsure{flow} or in other words, makes the program safe.  \naman{highlighted part of Figure somethings shows the difference between semantics of witnessing vs traditional semantics}

%We realize the following -- the set of patterns of sources, sinks, and sanitizers are useful beyond detecting vulnerabilities. We override the existing static analysis query that detects unsafe programs and use these encoded sanitizers for detecting sanitizers and guards in programs. Specifically, in the existing query that detects unsafe programs, we modify the taint-propagation steps to propagate taints through sanitizers and guards and then use static analysis to then find these dataflows containing sanitizers and guards. Thus, we can directly find the safe programs containing these \textit{witnesses} of safety. 
%Once such a dataset is collected, we use these witnesses to convert safe  to unsafe  and thus obtain paired examples for learning repair strategies (Section~\ref{subsec:witness-removal}). 

\lstMakeShortInline[columns=fixed]@
%We instantiate our \sawitness technique using \codeql~\cite{a}. It is an open-source \sa tool that allows implementing custom static analysis as queries in a high-level object-oriented extension of datalog. These queries usually contain a \Verb|select from where| statement that allows querying the program database. \codeql maintains these patterns of sources, sinks, sanitizers, and guards using \Verb"Configuration" classes. Consider an example of a simplified \Verb"Configuration" for \xss vulnerability in Figure~\ref{fig:configuration}. It defines a set of predicates @isSource@, @isSink@, @isSanitizer@, and @isGuard@. These predicates are written manually by \codeql authors and improved through rich community support\footnote{\url{https://github.com/github/codeql}}. With this configuration, vulnerabilities are reported by selecting source-sink pairs such that the @cfg.hasFlow@ predicate is true for the source, and the sink. This predicate is internally defined by \codeql and uses the patterns defined in the configuration to check for the presence of vulnerability-causing dataflows. %\spsays{Showing corresponding programs will be useful}

%Now, we demonstrate the static-analysis-witnessing approach for collecting examples of \safeprog and witnesses in Figure~\ref{fig:safe-configuration}. Specifically, we inherit from the existing configuration, using the same @isSource@ and @isSink@ predicates while overriding the @isSanitizer@ and @isGuard@ predicates to @none()@. This ensures that all the source and sink pairs are detected independent of the presence of sanitizers/guards between them. Finally, to detect our witnesses, we define the @isWitness@ predicate which uses the @isSanitizer@ and @isGuard@ predicates from the original configuration. Specifically, witnesses are defined as sanitizers/guards that lie between a source-sink pair. Finally, to report \safeprog and witnesses, the @cfg.hasFlow@ predicate is used to select all valid source-sink pairs and the corresponding witnesses are detected via the @isWitness@ predicate. Note that Figure~\ref{fig:configuration-vs-safe-configuration} depicts the key idea behind our approach in a simplified view. In practice, additional measures need to block the taint propagation internally and we share the actual \codeql queries used as part of the Appendix~\ref{app:codeql-queries}.


\subsection{Witness Removal}
\label{subsec:witness-removal}

We obtain \safeprogs and witnesses by applying \sawitness to a snapshot of a codebase. Recall that the witnesses block the flow between a source and a sink and thus help make programs  \textit{safe}. Hence, removing these witnesses will make the programs unsafe. Recall also that the witnesses are either sanitizing functions of the form @sanitize(taintedVar)@ or guards of the form @if checkSafe(taintedVar) {executeSink(taintedVar)}@. %Usually, they are used only for ensuring the safety of programs and are not critical to the functionality of programs. Therefore, 
We implement witness-removal perturbations  that precisely remove the guard-checks and sanitizer-functions. Note that our goal here is to generate unsafe programs and corresponding edits that enable learning repair strategies that insert such witnesses. So, while we generate the unsafe programs by perturbation, they should look structurally similar to natural unsafe programs written by the developers, otherwise the repair strategies learned on this artificially generated data through perturbations would not generalize to code in the wild. 
%At the same time, minor syntactic-semantic issues in parts of unsafe programs not directly relevant to the vulnerability or repair do not impact learning.
\lstDeleteShortInline@

% Figure environment removed

\lstMakeShortInline[columns=fixed]@

\input{witnessremoval.tex}

We use \rmSan and \rmGuard functions to programmatically remove the witnesses. A high-level sketch of these functions is illustrated in Figure~\ref{fig:remove-functions}. The functions use the structure of the corresponding \astree (node types $\tau$) to decide how to remove witnesses. Consider the \rmGuard function. It first computes the parent (\witnesspar) and grand-parent (\witnessparpar) of the witness guard condition. Then if the type of \witnesspar is \ifstmt (i.e., program is of the form @if (witness) body@ then we modify the \astree edge from \witnessparpar and \witnesspar to instead point to the body of the \ifstmt (index 1 child is body of \ifstmt). Similarly, if the type of \witnesspar is \binaryexpr with operator @&&@ (i.e. of the form @if (otherCond && guard)@ or @if (guard && otherCond)@) then we again modify the edge from \witnessparpar and \witnesspar to instead point to the non-guard child of \binaryexpr (@otherCond@ in the example). Note that since \binaryexpr has 3 children, the index of non-guard child is index of guard-child subtracted from 2. 
Figure~\ref{fig:witness-removal} depicts this removal on the \astree level, where the syntactic edges in red are removed and the syntactic edges in green are inserted.
In the end, the functions returns a tuple of the \pdg of the unsafe program ($\prog_{unsafe}$), \pdg of the safe program ($\prog_{safe}$)
and an edit object (\edit) which stores


\begin{enumerate}
    \item \astree for the removed witness (referred to as \editprog)
    \item location in the \pdg where the witness is removed (referred to as editloc
    %\naga{shouldn't it be editloc to be consistent with (1)?} 
    or \editloc)
    %\item an enum (\insertsc or \replace) depending on whether \concedit is inserted or replaced 
\end{enumerate}

Since $\prog_{unsafe}$ and edit-object can generate the safe program, we only propagate the unsafe programs and edits as the output of this step. Applying \rmGuard function to the safe program in Figure~\ref{fig:safememberex} removes the \ifstmt on Line~\ref{lst:line:fix-start} while preserving the @handlers[callerId](data);@ statement and in fact produces the unsafe program in Figure~\ref{fig:unsafememberex}. Additionally, it  returns the removed witness guard  @if handlers.hasOwnProperty(data.id){ ... }@ as the \editprog and \blockstmt (blue oval in Figure~\ref{fig:example1-pdg}) as the edit location \edit.editloc. Figure~\ref{fig:example1-editprog} shows the \astree for the \editprog containing the \ifstmt. 
The dashed line and dark circle correspond to the \textit{removed} \astree edge between the \blockstmt and the \expr @handlers[callerId](data)@. 

Note that Figure~\ref{fig:remove-functions} provides a high-level sketch of witness-removal and elides over implementation details that are required to make it work for real \js programs. We discuss these issues in the implementation section (Section~\ref{subsec:impl:witness-removal}).% and include the full implementation as part of supplementing source code\naga{we should make sure we are doing these, else remove this sentence}. 
%. In practice, we need implement such decisions more carefully to cover other traditional cases in which guards occur and we document them in the supplementing source code.
\lstDeleteShortInline@

%\naman{add examples $\dots$ } \spsays{do we re-run codeql on this generated bad program? -- NO (naman)}


\section{Method} \label{method_hybridaugment}
In this section, we formally define the problem, motivate our work and then present our proposed techniques.


\subsection{Preliminaries}
Let $\mathcal{F}(x;W)$ be an image classification CNN trained on the training set $\mathcal{T}_\text{train} = (x_{i}, y_{i})^{N}_{i=1}$  with $N$ samples, where $x$ and $y$ correspond to images and labels. The clean accuracy (CA) of $\mathcal{F}(x;W)$ is formally defined as its accuracy over a clean test set $\mathcal{T}_\text{test} = (x_{j}, y_{j})^{M}_{j=1}$. Assume two operators ${A}(\cdot)$ and ${C}(c, s)$ that adversarially attacks or corrupts a given set of images with the corruption category $c$ and severity $s$, respectively.  Let $A\mathcal{T}_\text{test}$ and $C\mathcal{T}_\text{test}$ be the adversarially attacked and corrupted versions of $\mathcal{T}_\text{test}$, and let $\mathcal{F}(x;W)$ have a robust accuracy (RA) on $A\mathcal{T}_\text{test}$ and a corruption accuracy (CRA) on $C\mathcal{T}_\text{test}$. 
The aim is to fit $\mathcal{F}(x;W)$ such that the model gains robustness (\ie. increased RA and CRA compared its the baseline version), while retaining (or improving) the clean accuracy of its baseline version trained without robustness concerns.


\noindent \textbf{What we know.} Our work builds on the following crucial observations: i) CNNs favour high-frequency content \cite{wang2020high}, ii) adversaries and corruptions often reside in high-frequency \cite{wang2020towards}, iii) images are dominated by low-frequency \cite{Saikia_2021_ICCV} and iv) models relying on low-frequency components are more robust \cite{li2022robust,wang2020towards}. The robustness-accuracy trade-off is visible; low-frequency reliant models are more robust, but tend to miss out on clean accuracy brought by the high-frequency components. 

\subsection{HybridAugment}
We hypothesize that a \textit{sweet spot} in the robustness-accuracy trade-off can be found. Unlike the \textit{hard} approaches that completely rule out the reliance on high-frequency components (i.e. low-pass filters), we propose to \textit{reduce} the reliance on them. To this end, we adopt a data augmentation approach that aims to diversify $\mathcal{T}_\text{train}$ by an operation $\mathcal{HA(\cdot)}$. Keeping the strong relation intact between labels and low-frequency content (i.e. labels come from low-frequency-component image), we propose to swap high and low-frequency components of images in a batch on-the-fly. Unlike \cite{mukai2022improving}, we \textit{do not} restrict the images to belong to the same class; this diversifies the training distribution even further while preserving the image semantics. We call this basic version of our approach \textit{HybridAugment}, which corresponds to: 
%
\begin{equation} \label{hybrid_augment_paired}
    \mathcal{HA_{P}}(x_{i}, x_{j}) = \mathcal{LF}(x_{i}) + \mathcal{HF}(x_{j})
\end{equation}
%
where $x_{i}$ is the input image and $x_{j}$ is a randomly sampled image from the whole training set, which we simply sample from the mini batch at each training iteration in practice. $\mathcal{HF}$ and $\mathcal{LF}$ operators select the high and low-frequency components of an input image, for which we use:
%
\begin{equation} \label{eq:cutoff}
\begin{split}
    \mathcal{LF}(x) = GaussBlur(x) \\
    \mathcal{HF}(x) = x - \mathcal{LF}(x)
    \end{split}
\end{equation}
%
where $GaussBlur$ is used as a low-pass filter. Note that a similar outcome is possible by using Discrete Fourier Transforms (DFT), swapping the frequency bands and then applying Inverse DFT (IDFT). We find the gaussian blur operation to be faster and better in practice. 


Inspired from \cite{chen2021amplitude}, in addition to the image-pair scheme in Eq.~\ref{hybrid_augment_paired}, we propose a single image variant of \textit{HybridAugment}. In the single image variant, instead of combining two images, $x_i$ and $x_{j}$ are obtained by applying randomly sampled augmentations to a single image. The single image variant $\mathcal{HA_{S}}$ can therefore be defined as 
%
\begin{equation} \label{hybrid_augment_single}
    \mathcal{HA_{S}}(x_{i}) = \mathcal{LF}(Aug(x_{i})) + \mathcal{HF}(\hat{Aug}(x_{i}))
\end{equation}
%
where $Aug$ and $\hat{Aug}$ correspond to two sets of randomly sampled augmentation operations. Note that paired and single versions can work in tandem ($\mathcal{HA_{PS}}$), and actually outperform single or paired image versions. 


\subsection{HybridAugment++}


The frequency analysis is a vast literature, however, two core aspects often stand out; frequency-band analysis (i.e. low, high) and the decomposition of signals into amplitude and phase. \textit{HybridAugment} covers the former and shows competitive results in various benchmarks (see Section \ref{sec:exp_hybridaugment}). The latter is investigated in $\mathcal{APR}$ \cite{chen2021amplitude}, where phase is shown to be the more relevant component for correct classification, and training models based on their phase labels and swapping amplitude components of images randomly lead to more robust models. Note that frequency-band and phase/amplitude discussions are arguably orthogonal, since frequency, phase and amplitude provide distinct characterizations of a signal: intuitively speaking, frequency, phase and amplitude can be seen as the separation of visual patterns in terms of scale, location and significance. 


We hypothesize these two approaches can be complementary; a model reliant on low-frequency and spatial information (i.e. phase) can further improve robustness. Inspired by the successes of cascaded augmentation methods \cite{hendrycks2019augmix,wang2021augmax,calian2022defending}, we unify these two core aspects into a single, hierarchical augmentation method. We refer to this method as \textit{HybridAugment++} and define its paired version as:
%
\begin{equation}
  \mathcal{HA_{P}}^{++}(x_{i}, x_{j}, x_{z}) = \mathcal{APR_{P}}(\mathcal{LF}(x_{i}), x_{z}) + \mathcal{HF}(x_{j})
\end{equation}
%
where $x_{i}$, $x_{j}$ and $x_{z}$ are images sampled from the same batch. Here, $\mathcal{APR_{P}}$~\cite{chen2021amplitude} is defined as
\begin{equation}
    \mathcal{APR_{P}}(x_{i}, x_{z}) = \mathcal{IDFT}(A_{x_{z}} \otimes e^{i. P_{x_{i}}}) \\
\end{equation}
%
where $\otimes$ is element-wise multiplication, $A$ is the amplitude and $P$ is the phase component. Similar to $\mathcal{HA}$ and $\mathcal{APR}$, we also define a single-image version of \textit{HybridAugment++} as
%
\begin{equation}
 \mathcal{HA_{S}}^{++}(x_{i}) = \mathcal{APR_{S}}(\mathcal{LF}(Aug(x_{i}))) + \mathcal{HF}(\hat{Aug}(x_{i}))
\end{equation}
%
where $\mathcal{APR_{S}}$~\cite{chen2021amplitude} is defined as
%
\begin{equation}
\mathcal{APR_{S}}(x_{i}) = \mathcal{IDFT}\left(A_{\bar{Aug}(x_{i})} \otimes e^{i. P_{\overline{Aug}\left(x_{i}\right)}}\right)    
\end{equation}
%
where $Aug$, $\hat{Aug}$, $\bar{Aug}$ and $\overline{Aug}$ are different sets of randomly sampled augmentation operations. Note that we essentially propose a framework; one can use different single and paired image augmentations, either individually or together, and can still achieve competitive results (see ablations in Section \ref{sec:exp_hybridaugment}). There are also other alternatives, such as swapping phase/amplitude first and then performing $\mathcal{HA}$, but we observe poor performance in practice; dividing the phase component into frequency-bands is not interpretable as frequencies of the phase component are not well defined. The pseudo-code of our methods can be found in the supplementary material.





\section{RESULTS}
\subsection{TNCs impact on overall taxi market performances}
\added{To explore the first research question, \emph{\textbf{how much do TNCs impact overall labor supply and revenue of the taxi market}}, we mine the overall taxi market performance metrics from January 2014 to December 2018 as shown in Figure~\ref{overall} and test the first hypothesis}. There is a consistent decrease in ridership, market revenue, and market total labor supply (work hours for both taxi drivers and taxi medallions) over time for yellow taxis. Besides, we observe a  substantial reduction in the monthly income per yellow taxi driver since the beginning of 2016, followed by a small income rebound in early 2018. Meanwhile, the monthly work hours per yellow taxi driver roughly remained the same over these years. In contrast to trends for the yellow taxi market, we find the increases in green taxi ridership before June 2015 and decreases after that time. As for the individual driver, the monthly income and monthly work hours are observed to first increase before June 2016 and then decrease rapidly. These results provide strong evidence that the nature of the taxi market has changed significantly in recent years. Meanwhile, the TNCs have seen their most rapid growth since early 2014. This overlapping of time consequently evokes a question of whether the rise of TNCs should account for the changes in taxi markets.

% Figure environment removed


\highlighttext{\added{The hypothesis testing comprehensively measures the impacts of the increasing number of TNC trips on overall market performance metrics in the taxi industry.} The estimated results of the OLS regression in five years range are shown in Table~\ref{tab:olsresult_market} and Table~\ref{tab:olsresult_individual}. The OLS estimates take the monthly TNC trips as an indicator variable and regress on a set of taxi market responses, which include trips, fares, and labor supply from both the market-level and average individual-level for yellow and green taxis. }Several significant findings can be identified based on the results. First, the monthly yellow taxi trips are negatively related to the increase of the TNC trips (at 0.001 significance level). \highlighttext{Besides, 1\% increase in TNC trips would lead to 0.02\% reduction in yellow taxi trips (TNC trips increase 2.2\% from November to December 2018 and lead to about 46,000 yellow taxi trips reduced in a month). As for the market revenue and market total labor supply of the yellow taxis, both variables are observed to be significantly negative to the increase of TNC trips. We observe that 1\% increase in TNC trips would result in 0.02\% reduction in monthly fare and 0.01\% reduction in monthly work hours of the yellow taxi market (TNC trips increase 2.2\% from November to December at 2018 lead to about \$687,500 revenue reduction and about 11,550 work hours reduction for yellow taxi market in a month).} Moreover, the monthly yellow taxi drivers, medallions, as well as daily taxi medallions decreased rapidly. These findings suggest that the rise of TNCs takes a vast market share from the yellow taxi market. Contrary to the yellow taxis, we also note no significant impacts on the green taxi market in terms of ridership, market revenue, and total labor supply in five years.



\begin{table}[!h]
\centering
\caption{The impact of TNC trips on the market level performance of labor supply (OLS estimation)}
\label{tab:olsresult_market}
\begin{tabular}{lcccc}
\toprule
\multirow{2}{*}{Dependent variables}                                & \multicolumn{2}{c}{Yellow taxi} & \multicolumn{2}{c}{Green taxi} \\
\cmidrule(l){2-3} \cmidrule(l){4-5}
                                                                    & Coefficient     & Adj.$R^2$  & Coefficient     & 
                                                                    Adj.$R^2$  \\
                                                                    \hline
\multirow{2}{*}{Monthly taxi trips}        & -0.0219   & 0.338      & -0.0133   & 0.031      \\
                                                                    & (9.76E-06 ***)  &            & (0.095 .)
                                                &            \\
\multirow{2}{*}{Monthly taxi fares}    & -0.0166   & 0.245     & -0.0101    & 0.013      \\
                                                                    & (3.48E-05 ***)  &            & (0.185) &            \\
\multirow{2}{*}{Monthly work hours for all taxi drivers}                    & -0.0146   & 0.269      & -0.0078   & 0.004     \\
                                                                    & (1.32E-05 **)  &            & (0.269) &            \\

\multirow{2}{*}{Monthly taxi drivers}                 & -0.0123   & 0.276      & -0.0075    & 0.011      \\
                                                                    & (9.79E-06 ***)  &            & (0.206)  &            \\
\multirow{2}{*}{Monthly taxi medallions}               & -0.0035    & 0.156      & -0.0016   & -0.016      \\
                                                                    & (0.001 ***)  &            & (0.767) &            \\
\multirow{2}{*}{Average daily taxi medallions}      & -0.0056   & 0.198      & -6.27E-05   & -0.017     \\
                                                                    & (0.0002 ***)  &            & (0.991)  &           \\
                                                             \bottomrule
                                                                    
\end{tabular}\\
Note: the p-value of the estimated coefficient is shown in the bracket (. : $\leq$ 0.1; *: P $\leq$ 0.05; **: P $\leq$ 0.01; ***: P $\leq$ 0.001).
\end{table}

\begin{table}[!h]
\centering
\caption{The impact of TNC trips on the individual level performance of labor supply (OLS estimation)}
\label{tab:olsresult_individual}
\begin{tabular}{lcccc}
\toprule
\multirow{2}{*}{Dependent variables}                                & \multicolumn{2}{c}{Yellow taxi} & \multicolumn{2}{c}{Green taxi} \\
\cmidrule(l){2-3} \cmidrule(l){4-5}
                                                                    & Coefficient &    Adj.$R^2$  & Coefficient     & 
                                                                    Adj.$R^2$  \\
                                                                    \hline
\multirow{2}{*}{Monthly income per taxi driver} & -0.0044   & 0.09     & -0.0026  & 0.002      \\
                                                                    & (0.011 *)  &            & (0.290)   &            \\
\multirow{2}{*}{Monthly work hours per taxi driver}               & -0.0023   & 0.075     & -0.0004    & -0.016      \\
                                                                    & (0.020 *)  &            & (0.828)  &            \\
\multirow{2}{*}{Monthly work hours per taxi medallion}               & -0.011   & 0.300     & -0.0064    & 0.121      \\
                                                                    & (3.57E-06 ***)  &            & (0.004 **)  &            \\
                                                                    
\multirow{2}{*}{Average daily work hours per taxi medallion}         & -0.0091   & 0.300      &-0.0082    & -0.31     \\
                                                                    & (6.99E-07 ***)  &            & (2.23E-06 ***) &            \\
\multirow{2}{*}{Average daily work hours per taxi driver}      & -0.0009   & 0.065      & -0.0066   & 0.261     \\
                                                                    & (0.028 *)  &            & (1.82E-05 ***)  &           \\                                                                    
\multirow{2}{*}{Average minutes per taxi trip}                     & 0.0085   & 0.162     & 0.0043    &0.064      \\
                                                                    & (0.001 ***)  &            & (0.028 *)&            \\
                                                                     \bottomrule

\end{tabular}\\

Note: the p-value of the estimated coefficient is shown in the bracket (. : $\leq$ 0.1; *: P $\leq$ 0.05; **: P $\leq$ 0.01; ***: P $\leq$ 0.001).
\end{table}


From the individual perspective, the monthly income and work hours per yellow taxi driver are observed to decrease 0.04{\textperthousand}  and 0.02{\textperthousand} along with 1\% increase of the TNC trips. And from the five years range, the rise of TNC captures a small proportion of the variation of yellow taxi drivers' monthly income and work hours. Moreover, the impacts on the utilization (work hours) of per taxi medallion are much more significant (at 0.001 significance level) than it is on the monthly work hours per yellow taxi driver (at 0.05 significance level), and green taxis yield the same result. In combination with the observed drastic reduction in total labor supply of the yellow taxi market (0.01\% reduction along with
1\% increase of TNC trips), such evidence implies that there is only a minor change in the monthly work hours per individual yellow taxi driver (0.02{\textperthousand} reduction along with 1\% increase of TNC trips) while a significant reduction in the monthly driver (0.01\% reduction along with 1\% increase of TNC trips) is observed. The impacts on monthly labor supply and monthly income per driver are not statistically significant for green taxis. However, daily work hours per green taxi driver are found to decrease significantly by 0.07{\textperthousand} along with 1\% increase in monthly TNC trips. This suggests that individual drivers work more days in a month since there are no significant changes in the monthly work hours.

The above results are based on January 2014 and December 2018, where we have seen the most rapid growth of the TNC sector. \highlighttext{The well-known fact is that the taxi market has lost significant ridership, and we can also verify that the rise of TNCs has significantly decreased total market revenue and labor supply for yellow taxis from the data. Nevertheless, such impact is found to be non-significant for the green taxi market, which is a special class of taxi service for serving areas outside Manhattan. These observations lead to two important implications.} First, the taxi demand within Manhattan, which used to be served exclusively by yellow taxis, is close to its saturated level, and the TNCs, therefore, directly compete with yellow cabs for existing passengers. However, for areas outside Manhattan, we find empirically that the TNCs lead to induced ridership that is unsatisfied before and contribute to mitigating the under-supply issue in these areas. 
\begin{table}[!h]
\centering
\caption{Yellow taxi revenue and labor supply variation (OLS estimation)}
\label{tab:yellow taxi market time period variation}
\begin{tabular}{lcccc}
\toprule 

\begin{tabular}[c]{@{}c@{}}Month/Year\end{tabular} & \begin{tabular}[c]{@{}c@{}}Monthly fare\\all taxi drivers \end{tabular} & \begin{tabular}[c]{@{}c@{}}Monthly income\\per taxi driver\end{tabular} & \begin{tabular}[c]{@{}c@{}}Monthly work hours\\all taxi drivers\end{tabular} & \begin{tabular}[c]{@{}c@{}}Monthly work hours\\per taxi driver\end{tabular} \\
\hline

\multirow{2}{*}{01/13-06/15}& -0.0013& 0.0018& -0.0019 &0.0004\\
                            &(0.421)&(0.434)&(0.125)&(0.673)
\\
\multirow{2}{*}{07/13-12/15} & -0.003& 0.0006&-0.0035&0.0001 \\
                           & (0.078 .)&(0.608)&(0.01 *)&(0.867)
\\
\multirow{2}{*}{01/14-06/16} & -0.0042& -0.0003& -0.0045&-0.0005
\\
                           &(0.055 .) & (0.856) &
                           (0.01 *) &(0.597)\\
\multirow{2}{*}{07/14-12/16} & -0.0079 &-0.0032&-0.0073&-0.0025                                                                           \\
                            &(0.008 **)                                                      & (0.091 .)                                                         & (0.002 **)  &(0.036 *)                                                        \\

\multirow{2}{*}{01/15-06/17} & -0.1308&-0.0615&-0.0914&-0.0221                                                                         \\
                           &(1.033E-05 ***)                                                      & (0.002 **)                                                         & (3.26E-05 ***) &(0.066 . )                                                         \\

\multirow{2}{*}{07/15-12/17} & -0.1913&-0.0668&-0.1469&-0.0238                                                                         \\
                           &(6.77E-05 ***)                                                      & (0.02 *)                                                         & (4.36E-05 ***) &(0.261)                                                         \\
\multirow{2}{*}{01/16-06/18} & -0.2925                                                      & -0.0392                                                           & -0.2552  &  -0.0018                                                      \\
                           &(3.87E-06 ***)                                                      & (0.325)                                                         & (1.73E-07 ***) &(0.937)                                                         \\
\multirow{2}{*}{07/16-12/18} & -0.2785 &0.0398&-0.2896&0.0287                                                                  \\
                           &(2.08E-05 ***)                                                      & (0.323)                                                         & (1.94E-07 ***)  &(0.221)                                                        \\

                           \bottomrule
\end{tabular}

Note: the p-value of the estimation of monthly TNC trips is in the bracket(. : $\leq$ 0.1; *: P $\leq$ 0.05; **: P $\leq$ 0.01; ***: P $\leq$ 0.001).
\end{table}



\begin{table}[!h]
\centering
\caption{Green taxi revenue and labor supply variation (OLS estimation)}
\label{tab:green taxi market time period variation}
\begin{tabular}{lcccc}
\toprule
Month/Year          & \begin{tabular}[c]{@{}c@{}}Monthly fare\\ all taxi drivers \end{tabular} & \begin{tabular}[c]{@{}c@{}}Monthly income\\  per taxi driver\end{tabular} & \begin{tabular}[c]{@{}c@{}}Monthly work hours\\ all taxi drivers\end{tabular} &
\begin{tabular}[c]{@{}c@{}}Monthly work hours\\ per taxi driver\end{tabular} \\
\hline


\multirow{2}{*}{01/14-06/16} & 0.0156& 0.0042& 0.0151&0.0038
\\
                           &(0.002 **) & (0.064 .) &
                           (0.0012 **) &(0.046 *)\\
\multirow{2}{*}{07/14-12/16} & -0.0039 &-0.0048&-0.0022&-0.003                                                                           \\
                            &(0.414)                                                      & (0.076 .)                                                         & (0.585)  &(0.096 .)                                                        \\
\multirow{2}{*}{01/15-06/17} & -0.2753&-0.109&-0.2144&-0.0481                                                                         \\
                           &(1.30E-06 ***)                                                      & (0.00016 ***)                                                         & (6.46E-06 ***) &(0.011 *)                                                         \\
\multirow{2}{*}{07/15-12/17} & -0.5422                                                      & -0.1448                                                           & -0.4691 &  -0.0717                                                      \\
                           &(2.80E-09 ***)                                                      & (0.0004 ***)                                                         & (1.56E-09 ***) &(0.009 **)                                                         \\
\multirow{2}{*}{01/16-06/18} & -0.7848 &-0.1155&-0.7355&-0.0661                                                                   \\
                           &(2.05E-013 ***)                                                      & (0.042 *)                                                         & (9.1E-16 ***)  &(0.061.)                                                        \\
\multirow{2}{*}{07/16-12/18} & -0.681 &0.0717&-0.7508&0.0018                                                                   \\
                           &(1.9E-12 ***)                                                      & (0.1303)                                                         & (4.49E-15 ***)  &(0.953)                                                        \\
                           
                           \bottomrule
\end{tabular}

Note: the p-value of the estimation of monthly TNC trips is shown in the bracket(. : $\leq$ 0.1; *: P $\leq$ 0.05; **: P $\leq$ 0.01; ***: P $\leq$ 0.001).
\end{table}

\highlighttext{Figure~\ref{overall} indicates the long-term examination might not be appropriate to understand the labor supply and revenue in taxi market due to their non-linear trend from 2014 to 2018. Thus, we testify the temporal variation based on the shorter time periods via OLS}. The results are presented in Table~\ref{tab:yellow taxi market time period variation} and Table~\ref{tab:green taxi market time period variation}. For yellow taxi drivers, the losses in total market revenue and labor supply (at 0.001 significance level) are found to be more significant than the losses at the individual level (at 0.05 significance level). The trend of the changes in monthly work hours of all taxi drivers has been observed to best echo the increasing trend in the number of TNC trips (as early as July 2013), followed by the market revenue. However, the drivers' monthly work hours are barely affected over time. The performances of the green taxi market at the beginning stage (January 2014) are different from that of the yellow taxi market. We observe more positive impacts in the green taxi market as the market revenue and total labor supply increase with positive coefficients, primarily due to the induced ridership outside Manhattan (e.g., there is 0.02\% increase in total labor supply of the green taxi market along with 1\% increase in the number of TNC trips). 

\added{The statistical estimates for both long-term and short-term taxi market performance directly reject the first hypothesis that the rise of TNC trips does not significantly impact the labor supply and revenue of the taxi market}. Our findings of the monthly income and monthly work hours for both yellow and green taxi markets lead to a counter-intuitive observation: individuals' work hours are barely affected by the significant decrease in their income. That observation contradicts the NS assumption, where taxi drivers are assumed to be revenue maximizers. One possible explanation for this observation is that the increased average minutes per taxi trip (positive coefficient at 0.001 and 0.05 significance level for yellow and green taxis) makes up for the reduced number of trips. Yet another plausible explanation is that the taxi drivers may have a specific reference target, and their work hours are no longer strictly positively related to their daily revenue. Following this explanation, taxi drivers may choose to decrease their income target and reach this target with lower wage rates and the same work hours. %This behavior can be illustrated in Figure~\ref{refer}, where the income target changes from $T^1$ to $T^2$, and the works hour keeps the same $H^*$, contrary to the decrease in work hours under NS as shown in Figure~\ref{neo}).
If this is the case, such an observation is indicative of the existence of RDP behavior in the current taxi market. And we next test the validity of such an explanation through statistical analyses. 

% % Figure environment removed

% % Figure environment removed





\subsection{Taxi drivers' labor supply behavior and wage elasticity}
\added{While it is observed that the taxi market has been significantly affected by the rise of TNCs, we notice that taxi drivers' work hours do not change even though their income decreases. As taxi drivers are aware of the competition from the TNC sector, under the neo-classical scheme, it is reasonable to expect that taxi drivers will lower their work hours if their expected wage decreases, which obviously contradicts the aforementioned observation that their work hours are barely affected. With the increasing number of TNC trips, the underlying logic is that the taxi market share is expected to decrease and taxi drivers are likely to be more difficult to reach the same income level as before with the same amount of daily efforts (work hours) due to the competition. \highlighttext{In light of this counter-intuitive observation, we propose %the following research questions to investigate the reasons: \emph{\textbf{do the drivers decrease their expected wage along with the increasing number of TNC trips?} And 
the second hypothesis that \emph{\textbf{the increase of TNC trips does not decrease the taxi drivers' expected wage}}}. %Taxi drivers are likely to be more difficult to reach an income level as before with the same amount of daily efforts (work hours) due to the increase number of TNC trips.}
As discussed earlier, targeting behavior discourages the employee's motivation, lead to lower productivity, eventually conduct a vicious circle for the industry. This idea gives rise to our suspicion that the taxi labor supply may now be better explained by the RDP behavior instead of the neoclassical one. On this basis, our interest is to shed light on quantifying the RDP behavior in the taxi market to interpret the taxi drivers' non-intuitive response to the TNCs competition. This question has its roots in \highlighttext{Crawford and Meng's~\cite{crawford2011new}} study, where the taxi labor supply behavior is mainly driven by the income target. %To this end, the labor supply estimation under both the RDP and NS models can be analyzed from two aspects: wage proportion and wage elasticity. 
In this regard, we propose %the third research question: \emph{\textbf{is the RDP behavior present among taxi drivers with the increasing number of TNC trips? }} 
the third hypothesis that \emph{\textbf{taxi drivers do not present RDP behavior,}}}  and use the wage decomposition method to testify the labor supply behavior.


\highlighttext{We examine the second hypothesis using PLS to regress the drivers' average daily expected wage on the set of temporal indicators, the improvement surcharge indicators, and the log-transformation of monthly TNC trips.} The results are given in Table~\ref{tab:expected wage1} and Table~\ref{tab:expected wage2}. 

% Please add the following required packages to your document preamble:
% \usepackage{multirow}
\begin{table}[!h]
\centering
\caption{Result of log-transformation of TNC trips: experiment \uppercase\expandafter{\romannumeral1} (PLS estimation)}
\label{tab:expected wage1}
\begin{tabular}{lcccc}
\toprule
\multirow{2}{*}{Month/Year}                                & \multicolumn{2}{c}{Yellow taxi} & \multicolumn{2}{c}{Green taxi} \\
\cmidrule(l){2-3} \cmidrule(l){4-5}
                                                                    & Coefficient     & P value  & Coefficient     & P value  \\
                                                                    \hline
\multirow{1}{*}{01/13-06/15} & 0.0009 & 0.457 &-&- \\
\multirow{1}{*}{07/13-12/15} & 0.0006& 0.622&-&-\\
\multirow{1}{*}{01/14-06/16} & -0.0005 & 0.698&0.0044&0.046 *\\
\multirow{1}{*}{07/14-12/16} & -0.0031& 0.085 .&-0.0050&0.044 *\\
\multirow{1}{*}{01/15-06/17} & -0.0518& 0.006 **&-0.1167&4.16E-06 ***\\
\multirow{1}{*}{07/15-12/17} & -0.0816& 0.001 ***&-0.1537 & 2.85E-05 ***\\
\multirow{1}{*}{01/16-06/18} & -0.0656& 0.069.&-0.1113 &0.028 *\\
\multirow{1}{*}{07/16-12/18} & 0.0665& 0.068.&0.0556&0.188\\
                           \bottomrule
\end{tabular}

Note: *: P $\leq$ 0.05; **: P $\leq$ 0.01; ***: P $\leq$ 0.001.

\end{table}


\begin{table}[!h]
\centering
\caption{Result of log-transformation of monthly TNC trips: experiment \uppercase\expandafter{\romannumeral2} (PLS estimation)}
\label{tab:expected wage2}
\begin{tabular}{lcccc}
\toprule
\multirow{2}{*}{Month/Year}                                & \multicolumn{2}{c}{Yellow taxi} & \multicolumn{2}{c}{Green taxi} \\
\cmidrule(l){2-3} \cmidrule(l){4-5}
                                                                    & Coefficient    & P value  & Coefficient    & P value  \\
                                                                    \hline
\multirow{1}{*}{01/13-12/14} &0.0013 &0.469 & -&-\\
\multirow{1}{*}{07/13-06/15} &0.0014 &0.293&-&-\\
\multirow{1}{*}{01/14-12/15} &0.0005 &0.761& 0.0045&0.073 .\\
\multirow{1}{*}{07/14-06/16} &-0.0020 &0.141&-0.0020&0.211 \\
\multirow{1}{*}{01/15-12/16} &-0.0650 &0.0002 ***&-0.0779&0.004 ** \\
\multirow{1}{*}{07/15-06/17} &-0.0401&0.0022 ** &-0.1447 & 0.003 **\\
\multirow{1}{*}{01/16-12/17} &-0.1218 & 0.005 **&-0.2065 &0.001 ***\\
\multirow{1}{*}{07/16-06/18} & 0.0323&0.512&0.0597&0.293\\
\multirow{1}{*}{01/17-12/18} & 0.1279&0.027 *&0.1900&0.003 **\\
                           \bottomrule
\end{tabular}

Note: *: P $\leq$ 0.05; **: P $\leq$ 0.01; ***: P $\leq$ 0.001.
\end{table}


Although yellow taxi drivers' expected wage is non-significant related with month TNC trips from January 2013 to December 2014. Green taxi drivers are found to increase their expected wage due to the increase of TNC trips, which implies that the taxi market is still under-supplied with TNC trips in the beginning stage. Besides, the TNC trips are also found to negatively impact the expected wage from January 2015 to December 2017 for both green and yellow taxi drivers, which indicates the taxi market gradually shifted into the over-supplied state with increasing competition between taxis and TNCs (at 0.01 significance level). However, we observe that the increasing of TNC trips is found to be statistic significant and positively related to drivers' expected wage in 2018 and an individual-level income rebound presents during this period, as shown in Figure~\ref{fig:monthly fare  & TNC trips}. The income rebound is primarily due to the fact that the loss of total market supply (drivers quit the market) is faster than the reduction in taxi ridership and total market revenue, as shown in Figure ~\ref{fig:proportion}. As a consequence, the results from the experiment reject the second hypothesis that the increase of TNC trips does not decrease the taxi drivers' expected wage. Besides, Table~\ref{tab:expected wage1} and Table~\ref{tab:expected wage2} also indicate the consistency of our results under different data compositions. 

% Figure environment removed

% Figure environment removed


\highlighttext{To testify taxi driver's labor supply behavior in the third hypothesis, we measure the unanticipated and anticipated transitory wage variation based on the wage decomposition method}. Besides, we conduct the t-test to examine the change of the unanticipated transitory wage proportion in the current year group compared with its base year group. 
\begin{table}[!h]
\centering
\caption{Results of wage variation decomposition: experiment \uppercase\expandafter{\romannumeral1} (PLS estimation)}
\label{tab:wage decomposition1}
\begin{tabular}{lcccccc}
\toprule
\multirow{2}{*}{Month/Year}                                & \multicolumn{3}{c}{Yellow taxi} & \multicolumn{3}{c}{Green taxi} \\
\cmidrule(l){2-4} \cmidrule(l){5-7}
                                                                    & Fixed     & Anticipated  & Unanticipated   & Fixed     & Anticipated  & Unanticipated   \\
                                                                    \hline
\multirow{2}{*}{01/13-06/15}& 9.04E-06& 0.0021& 0.0002 &-&-&-\\
                           & (0.42\%)&(90.93\%)&(8.65\%)&-&-&-
\\

\multirow{2}{*}{07/13-12/15} & 1.46E-05& 0.0020& 0.0002&-&-&-
\\
                           & (0.65\%) & (90.05\%) &
                           (9.30\%) &-&-&-\\  
                           
\multirow{2}{*}{01/14-06/16}& 0.0001& 0.00198& 0.0002  &0.000412&0.0049&0.0004 \\
                           & (4.42\%)&(85.64\%)&(9.94\%)&(7.26\%)&(86.42\%)&(6.32\%)
\\
\multirow{2}{*}{07/14-12/16} & 0.0002& 0.0018& 0.0003& 0.0002&0.0040&0.0006 
\\
                           & (9.42\%) & (78.45\%) &
                           (12.13\%)& (3.22\%) & (84.36\%) &
                           (12.42\%)  \\  
\multirow{2}{*}{01/15-06/17} & 0.0002                                                       & 0.0029                                                          & 0.0004 &     3.16E-30&0.0070&0.0011                                                    \\
                           & (5.62\%)                                                      & (82.09\%)                                                         & (12.29\%)  & (0\%)                                                      & (85.98\%)                                                         & (14.02\%)                                                        \\
\multirow{2}{*}{07/15-12/17 } & 6.04E-05                                                       & 0.0025                                                          & 0.0005&7.89E-31& 0.0062&0.0091                                                    \\
                           & (1.95\%)                                                      & (80.65\%)                                                         & (17.40\%)  & (0\%)                                                      & (87.24\%)                                                         & (12.76\%)                                                         \\

\multirow{2}{*}{01/16-06/18} & 0.0002                                                      & 0.0025                                                           & 0.0005  &7.10E-30& 0.0055&0.0013                                                \\
                           & (6.76\%)                                                      & (77.28\%)                                                         & (15.96\%)  & (0\%)                                                      & (80.86\%)                                                         & (19.14\%)                                                         \\
\multirow{2}{*}{07/16-12/18} & 5.82E-05                                                       & 0.0021                                                          & 0.0004 &    3.16E-30&0.0028&0.0008                                                    \\
                           & (2.33\%)                                                      & (83.33\%)                                                         & (14.34\%)  & (0\%)                                                      & (78.03\%)                                                         & (21.97\%)                                                         \\
                           \bottomrule
\end{tabular}

Note: the proportion of variation of each part in the total variation is in the bracket.
\end{table}


\begin{table}[!h]
\centering
\caption{Results of wage variation decomposition: experiment \uppercase\expandafter{\romannumeral2} (PLS estimation)}
\label{tab:wage decomposition2}
\begin{tabular}{lcccccc}
\toprule
\multirow{2}{*}{Month/Year}                                & \multicolumn{3}{c}{Yellow taxi} & \multicolumn{3}{c}{Green taxi} \\
\cmidrule(l){2-4} \cmidrule(l){5-7}
                                                                    & Fixed     & Anticipated  & Unanticipated   & Fixed     & Anticipated  & Unanticipated   \\
                                                                    \hline

\multirow{2}{*}{01/13-12/14 }& 9.6E-07& 0.0022& 0.0002 &-&-&-\\
                           & (0.04\%)&(92.84\%)&(7.12\%)&-&-&-
\\
\multirow{2}{*}{07/13-06/15} & 4.03E-05& 0.0021& 0.0002 &-&-&- \\
                           & (1.76\%)&(91.50\%)&(6.74\%)&-&-&-
\\
\multirow{2}{*}{01/14-12/15} & 4.65E-05& 0.0020& 0.0002&0.0005&0.0056&0.0003
\\
                           & (2.05\%) & (89.80\%) &
                           (8.15\%)& (8.29\%) & (87.17\%) &
                           (4.54\%)\\

\multirow{2}{*}{07/14-06/16} & 5.08E-05                                                       & 0.0015                                                           & 0.0002 &   7.15E-07&0.0018&0.0002                                                        \\
                           & (3.16\%)                                                      & (85.10\%)                                                         & (11.98\%) & (0.04\%)                                                      & (91.54\%)                                                         & (8.42\%)                                                           \\
\multirow{2}{*}{01/15-12/16} & 9.34E-05                                                       & 0.0020                                                           & 0.0003  & 3.16E-30&0.0047&0.0006                                                          \\
                           & (3.93\%)                                                      & (82.97\%)                                                         & (13.10\%) & (0\%)                                                      & (88.06\%)                                                         & (11.94\%)                                                          \\
\multirow{2}{*}{07/15-06/17} & 1.21E-05                                                       & 0.0024                                                          & 0.0005    & 7.89E-31&0.0066&0.0007                                                   \\
                           & (0.41\%)                                                      & (83.33\%)                                                         & (16.26\%)    & (0\%)                                                      & (89.84\%)                                                         & (10.16\%)                                                        \\
\multirow{2}{*}{01/16-12/17} & 8.97E-05                                                      & 0.0022                                                          & 0.0006    & 7.10E-30&0.0054&0.0012                                                       \\
                           & (3.10\%)                                                      & (76.90\%)                                                         & (20\%)& (0\%)                                                      & (82.05\%)                                                         & (17.95\%)                                                           \\
\multirow{2}{*}{07/16-06/18} & 7.05E-05                                                       & 0.0022&0.0005   &3.16E-05&0.0031&0.0008                                                                                                      \\
                           & (2.52\%)                                                      & (79.30\%)                                                         & (18.18\%)  & (0.80\%)                                                      & (78.68\%)                                                         & (20.52\%)                                                         \\
\multirow{2}{*}{01/17-12/18} & 0.0001                                                      & 0.0024                                                         & 0.0004    & 3.16E-30&0.0030&0.0009                                                 \\
                           & (4.35\%)                                                      & (81.33\%)                                                         & (14.32\%) & (0\%)                                                      & (76.55\%)                                                         & (23.45\%)                                                          \\
                                                  
                           \bottomrule
\end{tabular}

Note: the proportion of variation of each part in the total variation is in the bracket.
\end{table}

The wage decomposition results are presented in Table~\ref{tab:wage decomposition1} and Table~\ref{tab:wage decomposition2}. The results quantify the fixed, anticipated, and unanticipated transitory wage variation of drivers' wage rate over time. For the yellow taxi drivers, we observe that the anticipated transitory variation dominates the total variation in both experiments. The proportion of unanticipated wage variation (8\%) from January 2013 to December 2014 is close to the findings in Farber's study~\cite{farber2015you}, where the unanticipated hourly transitory wage variation is reported to be 12.1\% when the taxi market is still monopolistic. \highlighttext{That is, when supply is not saturated, almost all of the taxi drivers' work behavior can be explained by the NS behavior.} However, the proportion of anticipated wage variation is observed to decrease as TNCs grow. The unanticipated transitory wage variation reaches its peak at the end of 2017 when 20 \% of yellow taxi drivers' behavior can be explained by RDP as shown in Figure~\ref{fig:relationship}. This result matches well with our previous findings when drivers decrease their expected wage as shown in Table~\ref{tab:expected wage1} and Table~\ref{tab:expected wage2}, as well as when the market revenue and labor supply significantly decreased (see Table~\ref{tab:yellow taxi market time period variation}). For green taxi drivers, only a small proportion (6\%) of green taxi drivers shows RDP behavior in the beginning stage (from January 2014 to June 2016). With the increasing competition from TNCs (see Figure~\ref{overall}), green taxi drivers' expected wage decreases, which results in the reduction of green taxi drivers in their work hours and income from July 2015 to June 2017 (see Table~\ref{tab:green taxi market time period variation}). Therefore, green taxi drivers exhibit revenue optimizing behavior as suggested by the NS (drivers having RDP behavior also slightly decrease during this period from 14\% to 12\% in experiment \uppercase\expandafter{\romannumeral1} and 12\% to 10\% in experiment \uppercase\expandafter{\romannumeral2}). After the second half of 2017, green taxi drivers show increasing RDP behavior due to the increasing competition from TNCs (see Figure~\ref{fig:relationship}). Finally, we observe that RDP behavior captures over 20\% of the green taxi drivers in the current taxi market.

The t-test in Table~\ref{tab:test hypothesis 2 for exp 1} and Table ~\ref{tab:test hypothesis 2 for exp 2} show a significant higher unanticipated transitory wage for both yellow and green taxis than their base year. The relationship between TNC trips and unanticipated transitory wage variation can be seen in Figure~\ref{fig:relationship}. The results indicate that drivers are facing more uncertainty in earning opportunities and clearly illustrate the change of drivers' labor supply behavior over time due to the increasing number of TNC trips. Therefore, \added{we reject the third hypothesis and conduct that RDP behavior presents among taxi drivers with the increasing number of TNC trips}. Based on the results, we observe that yellow taxi drivers face much more serious competition than green taxi drivers before 2017. Moreover, such competition leads to an unsustainable state in the ride-sharing market and results in 20\% yellow taxi drivers having RDP behavior at the end of 2017, which means a high proportion of taxi drivers lose confidence in the taxi industry. Finally, yellow taxi drivers quit the taxi market. Besides, the green taxi market benefits from the increase in demand at the beginning of TNCs' growth. Meanwhile, green taxi drivers present NS behavior before July 2017. However, the unanticipated transitory wage variation of green taxi drivers is found to increase after June 2017 and account for over 20\% of the total wage variation in both experiments at the end of 2018, which is over three times as compared to when the taxi market is still monopolistic. Consequently, the RDP behavior should not be ignored, and NS behavior is no longer suitable to interpret the total taxi drivers' work behavior in a competitive market. Instead, at least $20\%$ of green taxi drivers perform in a loss-aversion manner in the market rather than the revenue-maximizing behavior, which is widely used when the taxi market is still monopolistic. This finding is aligned with the second explanation for the question that we raised to the OLS model. That is, the driver has a specific reference target. Moreover, the fact that individual labor supply is found to be barely unaffected (see Figure~\ref{fig:hours}) while their monthly income is found to be significantly decreased can be explained by the co-existence of NS and the RDP behavior in the market. Furthermore, combining the examinations of the second and third hypotheses, we conclude that drivers decrease their income target and some of them even quit the market, so that the remaining drivers are observed to still serve the same amount of work hours. It points out the necessity to consider the regulation of TNCs for the sustainability of the taxi market. This issue requires further investigation with related data sources. Finally, Table~\ref{tab:wage decomposition1} and Table~\ref{tab:wage decomposition2} again confirm the consistency of our results under different data compositions.


\begin{table}[!h]
\centering
\caption{t-test for the change of unanticipated wage variation in experiment\uppercase\expandafter{\romannumeral 1} }
\label{tab:test hypothesis 2 for exp 1}
\begin{tabular}{lcc}
\toprule
Month/Year   & Yellow taxi& Green taxi \\  
                                                                    \hline

01/13-06/15&0.896&-
\\
07/13-12/15&0.853&-
\\
01/14-06/16&0.696&0.957
\\

07/14-12/16  &         0.228&0.032 *                                            \\
01/15-06/17           &0.118     &0.0007 ***   \\
07/15-12/17  &0.038*&0.0047 ** \\
01/16-06/18 & 0.023 * &0.002 **\\
07/16-12/18   &0.051 .&0.003 *  \\
\bottomrule

\end{tabular}
\end{table}


\begin{table}[!h]
\centering
\caption{t-test for the change of unanticipated wage variation in experiment \uppercase\expandafter{\romannumeral 2} }
\label{tab:test hypothesis 2 for exp 2}
\begin{tabular}{lcc}
\toprule
Month/Year   & Yellow taxi& Green taxi \\  
                                                                    \hline

07/13-06/15&0.6461&-
\\
01/14-12/15&0.922&-
\\

07/14-06/16 &         0.367&0.606                                         \\
01/15-12/16           &0.121     &0.019 *   \\
07/15-06/17   &0.0104*&0.015 *  \\
01/16-12/17 & 0.048 * &0.002 **\\
07/16-06/18   &0.138&0.046 *\\
01/17-12/18& 0.195 &0.006 **\\
\bottomrule

\end{tabular}

\end{table}

% Figure environment removed


% Figure environment removed


\begin{table}[!h]
\centering
\caption{Wage elasticity of taxi labor supply: experiment \uppercase\expandafter{\romannumeral 1} (PLS estimation)}
\label{tab:elast1}
\begin{tabular}{lcccc}
\toprule
\multirow{2}{*}{Month/Year}& \multicolumn{2}{c}{Yellow taxi}&\multicolumn{2}{c}{Green taxi}\\
\cmidrule(l){2-3} \cmidrule(l){4-5}
     & Coefficient & $Adj.R^2$ & Coefficient & $Adj.R^2$ \\
\hline
\multirow{2}{*}{01/13-06/15} & 0.6611      & 0.844& -&-\\
          & (5.00E-13 ***)     &   &-&        \\
\multirow{2}{*}{07/13-12/15} & 0.6526      & 0.844&-&-\\
          & (5.06E-13 ***)     &         &-&     \\
\multirow{2}{*}{01/14-06/16} & 0.6383      & 0.837&0.8006&0.92\\
          & (9.39E-13 ***)     &    &(4.55E-07 ***)&          \\
\multirow{2}{*}{07/14-12/16} & 0.5592      & 0.731&0.6228&0.953\\
          & (1.10E-09 ***)     &        &(5.96E-08 ***)&       \\
\multirow{2}{*}{01/15-06/17} & 0.5040      & 0.764&0.5822&0.903\\
          & (1.69E-10 ***)     &     &(7.51E-13 ***)&         \\
\multirow{2}{*}{07/15-12/17} & 0.4780      & 0.712&0.6103&0.909\\
          & (2.82E-09 ***)     &  &(2.48E-16 ***)&             \\
\multirow{2}{*}{01/16-06/18} & 0.4924      & 0.745&0.5770&0.863\\
          & (5.23E-10 ***)     &      &(8.34E-14 ***)&        \\
\multirow{2}{*}{07/16-12/18}& 0.4944      & 0.699&0.4776&0.541\\
          & (5.49E-09 ***)     &     &(2.24E-06 ***)&          \\
          \bottomrule
          
\end{tabular}

Note: the p-value of the estimate for log-transformation of monthly income per taxi driver is in the bracket (. : p $\leq$ 0.1; *: P $\leq$ 0.05; **: P $\leq$ 0.01; ***: P $\leq$ 0.001).
\end{table}

\begin{table}[!h]
\centering
\caption{Wage elasticity of taxi labor supply: experiment \uppercase\expandafter{\romannumeral 2} (PLS estimation)}
\label{tab:elast2}
\begin{tabular}{lcccc}
\toprule
\multirow{2}{*}{Month/Year}& \multicolumn{2}{c}{Yellow taxi}&\multicolumn{2}{c}{Green taxi}\\
\cmidrule(l){2-3} \cmidrule(l){4-5}
     & Coefficient & $Adj.R^2$ & Coefficient & 
     $Adj.R^2$ \\
\hline
\multirow{2}{*}{01/13-12/14}
& 0.6477      & 0.762&-&-\\
          & (5.68E-11 ***)     &  &-&          \\
\multirow{2}{*}{07/13-06/15} & 0.6582      & 0.854&-&-\\
          & (7.09E-11 ***)     &&-&            \\
\multirow{2}{*}{01/14-12/15} & 0.6436      & 0.827&0.8095&0.929\\
          & (4.76E-10 ***)     & &(2.35E-14 ***) &          \\
\multirow{2}{*}{07/14-06/16} & 0.6243      & 0.772&0.6834&0.762\\
          & (1.02E-08 ***)     & &(1.59E-08 ***) &            \\
\multirow{2}{*}{01/15-12/16}  & 0.5362      & 0.702&0.6087&0.862\\
          & (1.99E-07 ***)     & &(3.74E-11 ***)  &           \\
\multirow{2}{*}{07/15-06/17} & 0.5151      & 0.74&0.6379&0.933\\
          & (4.24E-08 ***)     &   &(1.30E-14 ***) &           \\
\multirow{2}{*}{01/16-12/17} & 0.4746      & 0.687&0.5960&0.893\\
          & (3.38E-07 ***)     &  &(2.33E-12 ***)            \\
\multirow{2}{*}{07/16-06/18} & 0.488      & 0.696 &0.5376&0.719\\
          & (2.45E-07 ***)     &&(1.02E-07 ***) &              \\
\multirow{2}{*}{01/17-12/18} &0.5243      & 0.798&0.4523&0.515\\
          & (2.59E-09 ***)     & &(4.81E-05 ***) &             \\
          \bottomrule
          
\end{tabular}

Note: the p-value of the estimate for log-transformation of monthly income per taxi driver is in the bracket (. : p $\leq$ 0.1; *: P $\leq$ 0.05; **: P $\leq$ 0.01; ***: P $\leq$ 0.001).

\end{table}

% Figure environment removed

\highlighttext{Finally, although the increase in unanticipated transitory wage variation suggests that drivers' behavior may shift from NS to RDP, it is not equivalent to assert that the RDP should explain drivers' behavior. We further conduct the wage elasticity analysis to provide a better understanding of this issue. The results are presented in Table~\ref{tab:elast1} and Table~\ref{tab:elast2}. We observe that wage elasticity yields a similar trend as the change of unanticipated transitory wage variation proportion and the change of RDP behavior for both yellow and green taxi drivers (see Figure~\ref{fig:relationship}). The wage elasticity for both yellow and green taxi drivers remains positive, which is inconsistent with the elasticity of RDP models being -1. Although the wage elasticity is positive, it implies the yellow taxi drivers reached the lowest wage elasticity at the period of January of 2016 to December 2017 (see Figure~\ref{fig:elasticity}), when 1\% increase in drivers' wage rate leads to 0.47\% increase in monthly work hours. Besides, the wage elasticity of yellow taxi stayed above 0.6 before June 2016 but dropped rapidly until December 2017. This finding corresponds to when the transitory wage variation proportion has changed, as shown in Figure~\ref{fig:relationship}. For the wage elasticity of green taxi drivers, there is a rebound at the period of July 2015 to June 2017 (see Figure~\ref{fig:elasticity}), which confirms our insights that they show revenue-maximizing behavior along with the increasing of TNC trips during this period. Since then, the wage elasticity has decreased. The similar results from the comparison of the two experiments again verify the consistency of our results.}

\added{In conclusion, the results from both the wage variation decomposition and wage elasticity of labor supply are sufficient to reject the third hypothesis and confirm that taxi drivers show RDP behavior.} The labor supply behavior of both yellow and green taxi drivers has changed since the increase of TNC trips at the beginning of 2015. At the end of study period, RDP explains over 14\% yellow taxi drivers and 20\% green taxi drivers' behavior. The RDP behavior among taxi drivers implies that an increasing number of drivers would quit the taxi market due to the loss of confidence~\cite{eliaz2014reference}. And the slightly weakened RDP behavior among yellow taxis after 2018 is likely to support this claim, where the taxi market has lost a number of active taxi drivers so that the remaining drivers are less pessimistic about the market with less competition from the same sector. The insights from both labor supply estimation and wage elasticity analyses suggest that taxi drivers show increasingly negative responses to the market over time. \highlighttext{Since pricing regulation and market entry regulation serve as one fundamental tool in reshaping drivers' labor supply behavior and maintain the balance between taxis and FHVs. Taking labor supply behavior into consideration is crucial to promote a sustainable and equitable environment in the competitive market of taxis and FHVs.}













\section{Conclusion and Future Work}
In this work, I design corruption-robust algorithms for the Lipschitz contextual search problem. I present the \emph{agnostic checking} technique and demonstrate its effectiveness in designing corruption-robust algorithms. There are several open problems for future research. First, in the algorithm I propose for pricing loss, the schedule for agnostic checks is fixed upfront. Can the learner design an adaptive checking schedule for the pricing loss? Second, this work assumes the learner has knowledge of the Lipschitz constant $L$. Can the learner design efficient no-regret algorithms without knowledge of $L$? 
%\section{AUTHOR CONTRIBUTION}
Lu Ling: Conceptualization, Methodology, Design, Data Collection, Analysis, Writing- Original draft preparation, Reviewing and Editing.
Xinwu Qian: Supervision, Conceptualization, Methodology, Design, Reviewing and Editing.
Satish V. Ukkusuri: Supervision, Conceptualization, Methodology, Reviewing and Editing.
All authors reviewed the results and approved the final version of the manuscript.


\section*{Reference}
\bibliographystyle{unsrt}
%% ***   Set the bibliography file.   ***
\bibliography{main}

\end{document}


%A substantial of these studies assume the driver's work behavior remains to be standard neoclassical.