\section{HYPOTHESIS}
To investigate three research questions, we propose three hypothesis that statistically test whether there is a significant impacts of TNCs on the labor supply of taxi market and labor supply behavior of taxi driver. First of all, to measure how much do TNCs impact the overall taxi market performances, we give 

While it is observed that the taxi market has been significantly affected by the rise of TNCs, we notice that individual driver's work hours do not change even though the income decreases. As taxi drivers are aware of the competition from the TNC sector, under the neo-classical scheme, it is reasonable to expect that taxi drivers will lower their expected income level, which obviously contradicts the aforementioned observation. The underlying logic is that, with the increasing number of TNC trips, taxi market share is expected to decrease due to the increasing TNC competition, which may result in a lower income level. In light of this issue, we propose two hypotheses to investigate the reasons behind the dilemma. For the first hypothesis, we suspect that:
\added{\emph{\textbf{The increase of TNC trips does not decrease the taxi drivers' expected wage.} }}

On the other hand, with more TNC trips, taxi drivers are likely to encounter worse daily experiences where it is more difficult to reach a similar income level as before with the same amount of daily efforts (work hours). As discussed earlier, target work hours (reference income) will discourage the employee's motivation thus leading to lower productivity and eventually conducting a vicious circle for the industry. This idea gives rise to our suspicion that the taxi labor supply may now be better explained by the RDP behavior instead of the neoclassical one. On this basis, our interest is to shed the light on quantifying the RDP behavior in the taxi market to interpret the taxi drivers' non-intuitive response to the competition from the TNC sector. This question has its root in Crawford and Meng's~\cite{crawford2011new} study, where the taxi labor supply behavior is mainly driven by the income target. To this end, the labor supply estimation under both the RDP model and NS model can be analyzed from two aspects: wage proportion and wage elasticity. In this regard, we propose the second hypothesis:
\added{\emph{\textbf{Reference-dependent preference behavior does not present among taxi drivers with more number of TNC trips.}}}

And we conduct statistical analyses to examine these two hypotheses in the following sections. 
