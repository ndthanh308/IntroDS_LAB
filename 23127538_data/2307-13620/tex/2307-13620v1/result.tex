\subsection{Taxi drivers' labor supply behavior of and wage elasticity}
\added{While it is observed that the taxi market has been significantly affected by the rise of TNCs, we notice that taxi drivers' work hours do not change even though the income decreases. As taxi drivers are aware of the competition from the TNC sector, under the neo-classical scheme, it is reasonable to expect that taxi drivers will lower their work hours if their expected income decreases, which obviously contradicts the aforementioned observation. The underlying logic is that, with the increasing number of TNC trips, taxi market share is expected to decrease due to the increasing TNC competition, which may result in a lower income level and a lower expected income level. In light of this issue, we propose the research question two and three to investigate the reasons behind the dilemma. The second research question is: \emph{\textbf{does the driver decrease his expected wage along with the increasing number of TNC trips?}} On the other hand, with more TNC trips, taxi drivers are likely to encounter worse daily experiences where it is more difficult to reach a similar income level as before with the same amount of daily efforts (work hours). As discussed earlier, target work hours (reference income) will discourage the employee's motivation thus leading to lower productivity and eventually conducting a vicious circle for the industry. This idea gives rise to our suspicion that the taxi labor supply may now be better explained by the RDP behavior instead of the neoclassical one. On this basis, our interest is to shed the light on quantifying the RDP behavior in the taxi market to interpret the taxi drivers' non-intuitive response to the competition from the TNC sector. This question has its root in Crawford and Meng's~\cite{crawford2011new} study, where the taxi labor supply behavior is mainly driven by the income target. To this end, the labor supply estimation under both the RDP model and NS model can be analyzed from two aspects: wage proportion and wage elasticity. In this regard, we propose the third research question: \emph{\textbf{does the reference-dependent preference behavior present among taxi drivers with the increasing number of TNC trips? }}}


To test the second research question, we use the wage decomposition method by regressing the drivers' average daily expected wage on the natural log-transformation of monthly TNC trips. The results are given in Table~\ref{tab:expected wage1} and Table~\ref{tab:expected wage2}. 

% Please add the following required packages to your document preamble:
% \usepackage{multirow}
\begin{table}[!h]
\centering
\caption{Result of log-transformation of TNC trips: experiment \uppercase\expandafter{\romannumeral1}}
\label{tab:expected wage1}
\begin{tabular}{lcccc}
\toprule
\multirow{2}{*}{Month/Year}                                & \multicolumn{2}{c}{Yellow taxi} & \multicolumn{2}{c}{Green taxi} \\
                                                                    & Coefficient     & P value  & Coefficient     & P value  \\
                                                                    \hline
\multirow{1}{*}{01/13-06/15} & 0.0009 & 0.457 &-&- \\
\multirow{1}{*}{07/13-12/15} & 0.0006& 0.622&-&-\\
\multirow{1}{*}{01/14-06/16} & -0.0005 & 0.698&0.0044&0.046 *\\
\multirow{1}{*}{07/14-12/16} & -0.0031& 0.085 .&-0.0050&0.044 *\\
\multirow{1}{*}{01/15-06/17} & -0.0518& 0.006 **&-0.1167&4.16E-06 ***\\
\multirow{1}{*}{07/15-12/17} & -0.0816& 0.001 ***&-0.1537 & 2.85E-05 ***\\
\multirow{1}{*}{01/16-06/18} & -0.0656& 0.069.&-0.1113 &0.028 *\\
\multirow{1}{*}{07/16-12/18} & 0.0665& 0.068.&0.0556&0.188\\
                           \bottomrule
\end{tabular}

Note: *: P $\leq$ 0.05; **: P $\leq$ 0.01; ***: P $\leq$ 0.001.

\end{table}


\begin{table}[!h]
\centering
\caption{Result of log-transformation of monthly TNC trips: experiment \uppercase\expandafter{\romannumeral2}}
\label{tab:expected wage2}
\begin{tabular}{lcccc}
\toprule
\multirow{2}{*}{Month/Year}                                & \multicolumn{2}{c}{Yellow taxi} & \multicolumn{2}{c}{Green taxi} \\
                                                                    & Coefficient    & P value  & Coefficient    & P value  \\
                                                                    \hline
\multirow{1}{*}{01/13-12/14} &0.0013 &0.469 & -&-\\
\multirow{1}{*}{07/13-06/15} &0.0014 &0.293&-&-\\
\multirow{1}{*}{01/14-12/15} &0.0005 &0.761& 0.0045&0.073 .\\
\multirow{1}{*}{07/14-06/16} &-0.0020 &0.141&-0.0020&0.211 \\
\multirow{1}{*}{01/15-12/16} &-0.0650 &0.0002 ***&-0.0779&0.004 ** \\
\multirow{1}{*}{07/15-06/17} &-0.0401&0.0022 ** &-0.1447 & 0.003 **\\
\multirow{1}{*}{01/16-12/17} &-0.1218 & 0.005 **&-0.2065 &0.001 ***\\
\multirow{1}{*}{07/16-06/18} & 0.0323&0.512&0.0597&0.293\\
\multirow{1}{*}{01/17-12/18} & 0.1279&0.027 *&0.1900&0.003 **\\
                           \bottomrule
\end{tabular}

Note: *: P $\leq$ 0.05; **: P $\leq$ 0.01; ***: P $\leq$ 0.001.
\end{table}


Although monthly TNC trips are observed to be positively related to the yellow taxi drivers' expected wage from January 2013 to December 2014, the effect is non-significant. Meanwhile, green taxi drivers are found to increase their expected wage due to the increase of TNC trips, which implies that the taxi market is still under-supplied with TNC trips in the beginning stage. Besides, the TNC trips are also found to negatively impact the expected wage from January 2015 to December 2017 for both green and yellow taxi drivers, which means the taxi market gradually shifted into the over-supplied state with increasing competition between taxis and TNCs (at 0.01 significance level). However, the trend of decreasing expected wage changes in 2018 when the increase of TNC trips is found to be positively related to drivers' expected wage (at 0.05 significance level for both yellow and green taxi drivers in experiment \uppercase\expandafter{\romannumeral2}) due to an income rebound at individual-level as shown in Figure~\ref{fig:monthly fare  & TNC trips}. The income rebound is primarily due to the loss of total market supply (drivers quit the market), which is faster than the reduction in taxi ridership and total market revenue, as shown in Figure ~\ref{fig:proportion}. As a consequence, the results from the experiment indicate that the increase of TNC trips significantly decreases the taxi drivers' expected wage at most of the time. Besides, Table~\ref{tab:expected wage1} and Table~\ref{tab:expected wage2} also indicate the consistency of our results under different data compositions. 

% Figure environment removed

% Figure environment removed


To test the third research question, we measure the unanticipated and anticipated transitory wage variation based on the wage decomposition method and conduct the t-test on comparing the proportion of the unanticipated transitory wage in each group with its base year group. The proportion of the unanticipated transitory wage quantifies how much of the wage is unexpected to taxi drivers. The results from the t-test verify if there is a significant change in drivers' unexpected wage variation over time. 
\begin{table}[!h]
\centering
\caption{Results of wage variation decomposition: experiment \uppercase\expandafter{\romannumeral1}}
\label{tab:wage decomposition1}
\begin{tabular}{lcccccc}
\toprule
\multirow{2}{*}{Month/Year}                                & \multicolumn{3}{c}{Yellow taxi} & \multicolumn{3}{c}{Green taxi} \\
                                                                    & Fixed     & Anticipated  & Unanticipated   & Fixed     & Anticipated  & Unanticipated   \\
                                                                    \hline
\multirow{2}{*}{01/13-06/15}& 9.04E-06& 0.0021& 0.0002 &-&-&-\\
                           & (0.42\%)&(90.93\%)&(8.65\%)&-&-&-
\\

\multirow{2}{*}{07/13-12/15} & 1.46E-05& 0.0020& 0.0002&-&-&-
\\
                           & (0.65\%) & (90.05\%) &
                           (9.30\%) &-&-&-\\  
                           
\multirow{2}{*}{01/14-06/16}& 0.0001& 0.00198& 0.0002  &0.000412&0.0049&0.0004 \\
                           & (4.42\%)&(85.64\%)&(9.94\%)&(7.26\%)&(86.42\%)&(6.32\%)
\\
\multirow{2}{*}{07/14-12/16} & 0.0002& 0.0018& 0.0003& 0.0002&0.0040&0.0006 
\\
                           & (9.42\%) & (78.45\%) &
                           (12.13\%)& (3.22\%) & (84.36\%) &
                           (12.42\%)  \\  
\multirow{2}{*}{01/15-06/17} & 0.0002                                                       & 0.0029                                                          & 0.0004 &     3.16E-30&0.0070&0.0011                                                    \\
                           & (5.62\%)                                                      & (82.09\%)                                                         & (12.29\%)  & (0\%)                                                      & (85.98\%)                                                         & (14.02\%)                                                        \\
\multirow{2}{*}{07/15-12/17 } & 6.04E-05                                                       & 0.0025                                                          & 0.0005&7.89E-31& 0.0062&0.0091                                                    \\
                           & (1.95\%)                                                      & (80.65\%)                                                         & (17.40\%)  & (0\%)                                                      & (87.24\%)                                                         & (12.76\%)                                                         \\

\multirow{2}{*}{01/16-06/18} & 0.0002                                                      & 0.0025                                                           & 0.0005  &7.10E-30& 0.0055&0.0013                                                \\
                           & (6.76\%)                                                      & (77.28\%)                                                         & (15.96\%)  & (0\%)                                                      & (80.86\%)                                                         & (19.14\%)                                                         \\
\multirow{2}{*}{07/16-12/18} & 5.82E-05                                                       & 0.0021                                                          & 0.0004 &    3.16E-30&0.0028&0.0008                                                    \\
                           & (2.33\%)                                                      & (83.33\%)                                                         & (14.34\%)  & (0\%)                                                      & (78.03\%)                                                         & (21.97\%)                                                         \\
                           \bottomrule
\end{tabular}

Note: the proportion of variation of each part in the total variation is in the bracket.
\end{table}


\begin{table}[!h]
\centering
\caption{Results of wage variation decomposition: experiment \uppercase\expandafter{\romannumeral2}}
\label{tab:wage decomposition2}
\begin{tabular}{lcccccc}
\toprule
\multirow{2}{*}{Month/Year}                                & \multicolumn{3}{c}{Yellow taxi} & \multicolumn{3}{c}{Green taxi} \\
                                                                    & Fixed     & Anticipated  & Unanticipated   & Fixed     & Anticipated  & Unanticipated   \\
                                                                    \hline

\multirow{2}{*}{01/13-12/14 }& 9.6E-07& 0.0022& 0.0002 &-&-&-\\
                           & (0.04\%)&(92.84\%)&(7.12\%)&-&-&-
\\
\multirow{2}{*}{07/13-06/15} & 4.03E-05& 0.0021& 0.0002 &-&-&- \\
                           & (1.76\%)&(91.50\%)&(6.74\%)&-&-&-
\\
\multirow{2}{*}{01/14-12/15} & 4.65E-05& 0.0020& 0.0002&0.0005&0.0056&0.0003
\\
                           & (2.05\%) & (89.80\%) &
                           (8.15\%)& (8.29\%) & (87.17\%) &
                           (4.54\%)\\

\multirow{2}{*}{07/14-06/16} & 5.08E-05                                                       & 0.0015                                                           & 0.0002 &   7.15E-07&0.0018&0.0002                                                        \\
                           & (3.16\%)                                                      & (85.10\%)                                                         & (11.98\%) & (0.04\%)                                                      & (91.54\%)                                                         & (8.42\%)                                                           \\
\multirow{2}{*}{01/15-12/16} & 9.34E-05                                                       & 0.0020                                                           & 0.0003  & 3.16E-30&0.0047&0.0006                                                          \\
                           & (3.93\%)                                                      & (82.97\%)                                                         & (13.10\%) & (0\%)                                                      & (88.06\%)                                                         & (11.94\%)                                                          \\
\multirow{2}{*}{07/15-06/17} & 1.21E-05                                                       & 0.0024                                                          & 0.0005    & 7.89E-31&0.0066&0.0007                                                   \\
                           & (0.41\%)                                                      & (83.33\%)                                                         & (16.26\%)    & (0\%)                                                      & (89.84\%)                                                         & (10.16\%)                                                        \\
\multirow{2}{*}{01/16-12/17} & 8.97E-05                                                      & 0.0022                                                          & 0.0006    & 7.10E-30&0.0054&0.0012                                                       \\
                           & (3.10\%)                                                      & (76.90\%)                                                         & (20\%)& (0\%)                                                      & (82.05\%)                                                         & (17.95\%)                                                           \\
\multirow{2}{*}{07/16-06/18} & 7.05E-05                                                       & 0.0022&0.0005   &3.16E-05&0.0031&0.0008                                                                                                      \\
                           & (2.52\%)                                                      & (79.30\%)                                                         & (18.18\%)  & (0.80\%)                                                      & (78.68\%)                                                         & (20.52\%)                                                         \\
\multirow{2}{*}{01/17-12/18} & 0.0001                                                      & 0.0024                                                         & 0.0004    & 3.16E-30&0.0030&0.0009                                                 \\
                           & (4.35\%)                                                      & (81.33\%)                                                         & (14.32\%) & (0\%)                                                      & (76.55\%)                                                         & (23.45\%)                                                          \\
                                                  
                           \bottomrule
\end{tabular}

Note: the proportion of variation of each part in the total variation is in the bracket.
\end{table}

The wage decomposition results are presented in Table~\ref{tab:wage decomposition1}  and Table~\ref{tab:wage decomposition2}. The results quantify the fixed, the anticipated, and the unanticipated transitory variation of drivers' wage over time. For the yellow taxi drivers, we observe that the anticipated transitory variation dominates the total variation in both experiments. The proportion of unanticipated wage variation from January 2013 to December 2014 precisely matches the findings in Farber's study~\cite{farber2015you} when taxi market is still monopolistic. That is, when supply is not saturated, almost all of the taxi drivers' work behavior can be explained by the NS behavior, and the labor supply is positively correlated to their wage. However, the proportion of anticipated wage variation is observed to decrease as TNCs grow. And the unanticipated transitory variation gradually increases and reaches its \added{peak} at the end of 2017 when 20 \% of yellow taxi drivers' behavior can be explained by RDP as shown in Figure~\ref{fig:relationship}. This result matches well with our previous findings when drivers decrease their expected wage as shown in Table~\ref{tab:expected wage1} and Table~\ref{tab:expected wage2}, as well as when the market revenue and labor supply significantly decreased (see Table~\ref{tab:yellow taxi market time period variation}). For green taxi drivers, only a small proportion (4\%) of green taxi drivers shows RDP behavior in the beginning stage (from January 2014 to December 2015). With the increasing competition from TNCs(see Figure~\ref{overall}), green taxi drivers' expected wage decreases, which results in the reduction of green taxi drivers in their work hours and income from July 2015 to June 2017 (see Table~\ref{tab:green taxi market time period variation}). Therefore, green taxi drivers exhibit revenue optimizing behavior as suggested by the NS (drivers having RDP behavior also slightly decrease during this period from 14\% to 12\% in experiment \uppercase\expandafter{\romannumeral1} and 12\% to 10\% in experiment \uppercase\expandafter{\romannumeral2}). After the second half of 2017, green taxi drivers show increasing RDP behavior due to the increasing competition from TNCs (see Figure~\ref{fig:relationship}). Finally, we observe that RDP behavior captures over 20\% of the green taxi drivers in the current taxi market.

The results of the t-test are given in Table~\ref{tab:test hypothesis 2 for exp 1} and Table ~\ref{tab:test hypothesis 2 for exp 2}. The relationship between TNC trips and unanticipated transitory wage variation can be seen in Figure~\ref{fig:relationship}. The unanticipated transitory wage of yellow and green taxi drivers (both at 0.05 significance level) is significantly higher than those of the base year group. The results indicate that drivers are facing more uncertainty regarding daily wage and clearly illustrate the change of drivers' work behavior over time due to the increasing number of TNC trips. Therefore, \added{we conduct that RDP behavior presents among taxi drivers with more number of TNC trips}. Based on the results, we observe that yellow taxi drivers face much more serious competition than green taxi drivers before 2017. Moreover, such competition leads to an unsustainable state in the ride-sharing market and 20\% yellow taxi drivers having RDP behavior at the end of 2017, which means a high proportion of taxi drivers lose confidence about the taxi industry. Finally, \added{yellow taxi drivers} quit the taxi market and this leads to the decrease of RDP behavior among taxi drivers after 2017. On the other hand, the green taxi market benefits from the increase in demand at the beginning of TNCs' growth. Meanwhile, green taxi drivers present NS behavior with an increasing number of TNC trips before July 2017, which indicates that the green taxi market is more flexible than the yellow taxi market during this period. However, more green taxi drivers have RDP behavior after June 2017. And the unanticipated transitory wage variation of green taxi drivers is found to account for over 20\% of the total wage variation in both experiments at the end of 2018, which is over three times as compared to when the taxi market is still monopolistic. Consequently, the RDP behavior should not be ignored, and NS behavior is no longer suitable to interpret the total taxi drivers' work behavior in a competitive market. Instead, at least $20\%$ of green taxi drivers performs in a loss-aversion manner in the market rather than the revenue maximizing behavior, which is widely used when the taxi market is still monopolistic. This finding is aligned with the second explanation for the question that we raised to the OLS model, \added{which is that the driver has a specific reference target}. Moreover, based on the results from the OLS model, the monthly income of taxi drivers is found to be significantly decreased due to TNC competition while the individual labor supply is found to be barely unaffected (see Figure~\ref{fig:hours}),
which can be explained by neither the NS nor the RDP behavior. However, the trend of drivers' behavior shifting towards RDP suggests the reason for this observation. Furthermore, combining the results of the second and third research questions, we conclude that drivers decrease their income target and some of them even quit the market, so that the remaining drivers are observed to still serve the same amount of work hours. Meanwhile, the drop in medallion prices helps to partially offset their losses from the lowered income target. Nevertheless, it also points out the necessity to consider the regulation of TNCs for the sustainability of the taxi market. \added{This issue requires further investigation with related data sources. Finally, Table~\ref{tab:wage decomposition1} and Table~\ref{tab:wage decomposition2} again confirm the consistency of our results under different data compositions.}


\begin{table}[!h]
\centering
\caption{t-test for the change of unanticipated wage variation in experiment\uppercase\expandafter{\romannumeral 1} }
\label{tab:test hypothesis 2 for exp 1}
\begin{tabular}{lcc}
\toprule
Month/Year   & Yellow taxi& Green taxi \\  
                                                                    \hline

01/13-06/15&0.896&-
\\
07/13-12/15&0.853&-
\\
01/14-06/16&0.696&0.957
\\

07/14-12/16  &         0.228&0.032 *                                            \\
01/15-06/17           &0.118     &0.0007 ***   \\
07/15-12/17  &0.038*&0.0047 ** \\
01/16-06/18 & 0.023 * &0.002 **\\
07/16-12/18   &0.051 .&0.003 *  \\
\bottomrule

\end{tabular}
\end{table}


\begin{table}[!h]
\centering
\caption{t-test for the change of unanticipated wage variation in experiment \uppercase\expandafter{\romannumeral 2} }
\label{tab:test hypothesis 2 for exp 2}
\begin{tabular}{lcc}
\toprule
Month/Year   & Yellow taxi& Green taxi \\  
                                                                    \hline

07/13-06/15&0.6461&-
\\
01/14-12/15&0.922&-
\\

07/14-06/16 &         0.367&0.606                                         \\
01/15-12/16           &0.121     &0.019 *   \\
07/15-06/17   &0.0104*&0.015 *  \\
01/16-12/17 & 0.048 * &0.002 **\\
07/16-06/18   &0.138&0.046 *\\
01/17-12/18& 0.195 &0.006 **\\
\bottomrule

\end{tabular}

\end{table}

% Figure environment removed


% Figure environment removed


\begin{table}[!h]
\centering
\caption{Wage elasticity of taxi labor supply: experiment \uppercase\expandafter{\romannumeral 1}}
\label{tab:elast1}
\begin{tabular}{lcccc}
\toprule
\multirow{2}{*}{Month/Year}& \multicolumn{2}{c}{Yellow taxi}&\multicolumn{2}{c}{Green taxi}\\
     & Coefficient & Adj.R^2& Coefficient & Adj.R^2 \\
\hline
\multirow{2}{*}{01/13-06/15} & 0.6611      & 0.844& -&-\\
          & (5.00E-13 ***)     &   &-&        \\
\multirow{2}{*}{07/13-12/15} & 0.6526      & 0.844&-&-\\
          & (5.06E-13 ***)     &         &-&     \\
\multirow{2}{*}{01/14-06/16} & 0.6383      & 0.837&0.8006&0.92\\
          & (9.39E-13 ***)     &    &(4.55E-07 ***)&          \\
\multirow{2}{*}{07/14-12/16} & 0.5592      & 0.731&0.6228&0.953\\
          & (1.10E-09 ***)     &        &(5.96E-08 ***)&       \\
\multirow{2}{*}{01/15-06/17} & 0.5040      & 0.764&0.5822&0.903\\
          & (1.69E-10 ***)     &     &(7.51E-13 ***)&         \\
\multirow{2}{*}{07/15-12/17} & 0.4780      & 0.712&0.6103&0.909\\
          & (2.82E-09 ***)     &  &(2.48E-16 ***)&             \\
\multirow{2}{*}{01/16-06/18} & 0.4924      & 0.745&0.5770&0.863\\
          & (5.23E-10 ***)     &      &(8.34E-14 ***)&        \\
\multirow{2}{*}{07/16-12/18}& 0.4944      & 0.699&0.4776&0.541\\
          & (5.49E-09 ***)     &     &(2.24E-06 ***)&          \\
          \bottomrule
          
\end{tabular}

Note: the p-value of the estimate for log-transformation of monthly income per taxi driver is in the bracket (. : p $\leq$ 0.1; *: P $\leq$ 0.05; **: P $\leq$ 0.01; ***: P $\leq$ 0.001).
\end{table}

\begin{table}[!h]
\centering
\caption{Wage elasticity of taxi labor supply: experiment \uppercase\expandafter{\romannumeral 2}}
\label{tab:elast2}
\begin{tabular}{lcccc}
\toprule
\multirow{2}{*}{Month/Year}& \multicolumn{2}{c}{Yellow taxi}&\multicolumn{2}{c}{Green taxi}\\
     & Coefficient & Adj.R^2& Coefficient & 
     Adj.R^2 \\
\hline
\multirow{2}{*}{01/13-12/14}
& 0.6477      & 0.762&-&-\\
          & (5.68E-11 ***)     &  &-&          \\
\multirow{2}{*}{07/13-06/15} & 0.6582      & 0.854&-&-\\
          & (7.09E-11 ***)     &&-&            \\
\multirow{2}{*}{01/14-12/15} & 0.6436      & 0.827&0.8095&0.929\\
          & (4.76E-10 ***)     & &(2.35E-14 ***) &          \\
\multirow{2}{*}{07/14-06/16} & 0.6243      & 0.772&0.6834&0.762\\
          & (1.02E-08 ***)     & &(1.59E-08 ***) &            \\
\multirow{2}{*}{01/15-12/16}  & 0.5362      & 0.702&0.6087&0.862\\
          & (1.99E-07 ***)     & &(3.74E-11 ***)  &           \\
\multirow{2}{*}{07/15-06/17} & 0.5151      & 0.74&0.6379&0.933\\
          & (4.24E-08 ***)     &   &(1.30E-14 ***) &           \\
\multirow{2}{*}{01/16-12/17} & 0.4746      & 0.687&0.5960&0.893\\
          & (3.38E-07 ***)     &  &(2.33E-12 ***)            \\
\multirow{2}{*}{07/16-06/18} & 0.488      & 0.696 &0.5376&0.719\\
          & (2.45E-07 ***)     &&(1.02E-07 ***) &              \\
\multirow{2}{*}{01/17-12/18} &0.5243      & 0.798&0.4523&0.515\\
          & (2.59E-09 ***)     & &(4.81E-05 ***) &             \\
          \bottomrule
          
\end{tabular}

Note: the p-value of the estimate for log-transformation of monthly income per taxi driver is in the bracket (. : p $\leq$ 0.1; *: P $\leq$ 0.05; **: P $\leq$ 0.01; ***: P $\leq$ 0.001).

\end{table}

% Figure environment removed

Finally, although the increase in unanticipated transitory variation suggests that drivers' behavior may shift from NS to RDP, it is not equivalent to assert that the RDP should explain drivers' behavior. We further conduct the wage elasticity analysis to provide a better understanding of this issue. The results \added{are presented} in Table~\ref{tab:elast1} and Table~\ref{tab:elast2}. We observe that wage elasticity yields a similar trend as the change of unanticipated transitory variation proportion and the change of RDP behavior for both yellow and green taxi drivers (see Figure~\ref{fig:relationship}). The wage elasticity for both yellow and green taxi drivers remains positive, which is inconsistent with the elasticity of RDP models being -1. Although the wage elasticity is positive, it implies the yellow taxi drivers reached the lowest wage elasticity at the period of January of 2016 to December 2017 (see Figure~\ref{fig:elasticity}), when 1\% increase in drivers' wage leads to 0.47\% increase in monthly work hours. Besides, the wage elasticity of yellow taxi stayed above 0.6 before June 2016 but dropped rapidly until December 2017. This finding corresponds to when the transitory wage variation proportion has changed, as shown in Figure~\ref{fig:relationship}. For the wage elasticity of green taxi drivers, there is a rebound at the period of July 2015 to June 2017 (see Figure~\ref{fig:elasticity}), which confirms our insights that they show revenue-maximizing behavior along with the increasing of TNC trips during this period. Since then, the wage elasticity decreased. The similar results from the comparison of two experiments again verify the consistency of our results.

In conclusion, the results from both the wage variation decomposition and wage elasticity of labor supply are verify the conclusion that taxi drivers show RDP behavior. And the labor supply behavior of both yellow and green taxi drivers have changed since the increase of TNC trips at the beginning of 2015. In the current market, over 14\% yellow taxi and 20\% green taxi drivers' behavior can be explained by RDP. The RDP behavior among taxi drivers implies that more number of drivers will quit the taxi market due to the loss of confidence~\cite{eliaz2014reference}. And the slightly weakened RDP behavior among yellow taxis after 2018 is likely to support this claim, where the taxi market has lost a number of active taxi drivers so that the remaining drivers are less pessimistic about the market with less competition from the same sector. Finally, 1\% increase in TNC trips in the current market (based on the estimation from July 2016 to December 2018) will lead to 0.28\% decrease in yellow taxi monthly revenue and 0.68\% decrease in green taxi monthly revenue. Furthermore, 1\% increase in TNC trips in the
current market will result in 0.29\% reduction in yellow taxi monthly work hours (including these who quit from the market) and 0.75\% reduction in green taxi monthly work hours. The insights from both labor supply estimation and wage elasticity analyses suggest that taxi drivers show increasingly negative responses to the market over time. That is, along with the increase of TNC trips, more drivers show RDP behavior and tend to quit from the market if there exists no appropriate policy that regulates the market of FHVs and taxis. \added{And pricing regulation will serve as one fundamental tool in guiding drivers' behavior and reshaping market labor supply. Meanwhile, market entry regulation is also important to maintain the balance between taxis and FHVs so that neither sector will contribute to the growth of the RDP behavior in the other sector. In conclusion, to promote a sustainable and equitable environment in the competitive market of taxis and FHVs, it is important to reconsider the pricing regulation and entry restrictions by taking the behavior of labor supply into consideration.}







