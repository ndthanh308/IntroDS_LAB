\section{Introduction}
The recent rise of Transportation Network Companies (TNCs), such as Uber and Lyft, offers mobility-on-demand services to the general public, which has greatly changed the way people travel and has resulted in numerous externalities.  \highlighttext{Aside from the widely perceived issues such as heavier road congestion and safety concerns~\cite{hall2018uber,qian2020impact,lai2020evaluating,lai2020resilient,ukkusuri2020performance,ling2023influencing,ling2017reliable}, a recent study~\cite{Dan2018} revealed the significant loss for the taxi industry introduced by the involvement of TNCs' competitions, and the lack of regulations.The taxi industry suffered from losses in market share, ridership, labor supply, and asset values.} For instance, the leading TNCs in San Francisco had taken over almost $2/3$ of the original taxi market share between 2012 and 2014, which had caused the bankruptcy of the largest taxi operator in the city~\cite{timmurphy.org}. The official trip-level data released by New York City Taxi and Limousine Commission (NYCTLC) also disclosed that the taxi ridership is experiencing a steady decline since January 2009, as shown in Figure~\ref{fig1}. Along with the drop in taxi ridership, the value of NYC taxi medallion has dropped by 80\% since May 2013, which is a sharp contrast to the upward trend from 1975 to 2013 with a price of taxi medallion increased by 27 times~\cite{Barry2019} (see Figure~\ref{fig5}). As a consequence, the medallion transactions declined sharply after 2013~\cite{Dan2018}. Other taxi operators have reported 50\% idle rate of their medallions due to the lack of drivers~\footnote{Source: https://www1.nyc.gov/site/tlc/businesses/medallion-transfers.page}, creating the so-called taxi graveyard in the city~\cite{WF2015}. The significant loss of businesses in the traditional taxi market has put the taxi drivers in debt, resulted in casualties~\cite{Nikita2018}, and exacerbated the conflicts between taxi and TNC drivers. \highlighttext{In light of these issues, several regulation policies, such as the \enquote*{Tolling Program} in 2019 and the NYC 2020 \enquote*{No Action} baseline, are underway to control the overgrowth of TNCs and to promote a more sustainable ride-hailing market with the co-existence of TNCs and traditional taxi services.}


%NYC Department of Transportation (NYCDOT) has enacted a series policies to regulate the service of TNCs, including NYC 2020 'No Action' baseline} \footnote{Source: Improving Efficiency and Managing Growth in New York's For-Hire Vehicle Sector, made by New York City Taxi and Limousine Commission and Department of Transportation at June 2019. A set of future actions is mentioned in this report. In particular, the regulations about issuing new licenses for For-Hire-Vehicle require to define the cap of both FHV and taxi. Besides, the 'Tolling Program' has enacted in April 2019, where the FHV and taxi have different congestion charges. However, the effect of this action is waiting for evaluation}. Therefore, understanding the impacts of TNCs on the labor supply of taxi market is urgently needed and important, which provide underlying evidence for the policy formulation and implementation such as cap regulation and fee management for TNCs and taxi.
 



% Figure environment removed

% Figure environment removed

%% Figure removed

%While much attention has been paid to the changes from the demand side of the taxi industry~\cite{contreras2017effects,cetin2013economic}, little effort is made in exploring the impact of TNCs on the labor supply of taxi market.




\highlighttext{Lying at the core of effective taxi market regulations is our understanding of the labor supply behavior, i.e., the number of active taxi drivers and taxi drivers' work hours.}
While much attention has been paid to the changes from the demand side of the taxi industry~\cite{contreras2017effects,cetin2013economic,ling2019forecasting}, few studies address the attention on the impact of TNCs on the labor supply of the taxi market. \highlighttext{Existing studies exaggerate the representative power of the historical data in predicting the future and modeling the labor supply behavior, such as dynamic pricing~\cite{qian2017time} and infrastructure planning~\cite{asamer2016optimizing}.} Other studies~\cite{frechette2019frictions,zha2016economic,qian2017taxi,farber2015you,yang1998network} introduce mathematical models or simulations to assist in framing taxi entry regulation and pricing policies based on the standard neoclassical model (NS), which assumes that taxi drivers follow revenue-maximizing behavior. They suggest that the labor supply function of taxi drivers is positively related to the trip revenue so that a transitory increase in the wage rate (wage per time unit) results in an increase in their work hours, and such behavior is supported by real-world observations when the taxi market was monopolistic~\cite{farber2008reference}. \highlighttext{These studies assume labor supply behavior being static, even though the competition from TNCs would potentially make their behavior rather dynamic. Nevertheless, due to the fact that taxi drivers are self-employed with freedoms in working hours management, evidence~\cite{crawford2011new} indicated that taxi drivers are likely to present reference-dependent-preference (RDP) behavior whenever their working environment changes. This evidence speculated that drivers' labor supply is negatively related to the wage rate as their wage approaches to a reference point. Considering the increasing number of TNC trips in recent years,} \highlighttext{\added{taxi drivers' labor supply behavior may have changed dramatically} and the NS behavior assumption based on the historical taxi record data may no longer be valid. As the executives decided not to cut the wage to a reference point during the Great Recession in order to prevent the discouragement of taxi drivers~\cite{eliaz2014reference,bewley2009wages}, we expect the history would repeat itself during the modern age. As Figure~\ref{fig5} indicates, however, the recent drastic drops in transfer price and the number of taxi medallion can discourage the willingness to work among taxi drivers. In this regard, the overall taxi market supply and individual work behavior may have changed due to the competition from TNCs. Meanwhile, drivers' labor supply behavior serves as a crucial input for modeling the competition between TNCs and the taxi industry, and our accurate interpretation of such behavior will directly contribute to effective regulations of the TNCs and taxi market. It is therefore of vital importance to investigate the labor supply behavior and the corresponding wage elasticity and explore how taxi drivers distribute their work hours under competition in the current taxi market.} \added{This motivates us to systematically quantify the changes of labor supply in the taxi market due to the impact of the TNCs, and we propose the following three research questions:}
\begin{itemize}
    \item How much do TNCs impact the overall labor supply and revenue of the taxi market?
    \item Do drivers decrease their expected wage along with the increasing number of TNC trips?
    \item Is the RDP behavior present among taxi drivers with the increasing number of TNC trips?
\end{itemize}



%For the data analytics \added{part A} model, it is assumed that the dataset is stationary, and the market in the past is representative of the market nature into the future. For the mathematical model, the widely regarded NS model~\cite{yang1998network,qian2017taxi} suggests that the labor supply function of taxi drivers is positively sloped so that a transitory increase in the wage results in an increase in work hours. Such behavior is supported by real-world observations when the taxi market was monopolistic~\cite{farber2015you}. However, studies~\cite{farber2008reference,crawford2011new} speculate the loss-aversion behavior of taxi drivers, where drivers are less inclined to work, and drivers' labor supply is even negatively related to the increase of wage rate when they reach the reference point (\added{beyond a targeted total work hours or an income target}). Such reference-dependent preferences (RDP) behavior is supported by the employee's low productivity during Great Depression~\cite{eliaz2014reference}. In recent years, the transfer price and volume of taxi medallions reflect the negative enthusiasm of the taxi market (see Figure~\ref{fig1}). Intuitively, the overall supply of the taxi market and individual work hours may have changed due to the competition from TNCs. Therefore, the real unknown is the dynamics in the change of medallion prices and labor supply due to the evolution of the TNC market in recent years. Indeed, drivers' labor supply behavior serves as a crucial input for modeling the competition between TNCs and the taxi industry, and it is closely associated with regulations of TNCs and taxi supply. The labor supply behavior and the corresponding wage elasticity are therefore vital to investigate how drivers distribute their work hours in a competitive market. These motivate us to systematically quantify the changes in labor supply for the taxi market due to the impact of the TNCs.


% \added{The barriers to entry are quite a few in this market. Therefore, one can regard the number of taxi drivers, medallions, and work hours as influenced by market conditions. The real unknown is the dynamics in the change of medallion prices and labor supply due to the evolution of the TNC market in recent years.} The labor supply of taxi market can, therefore, be regarded as the number of taxi drivers or medallions as well as the taxi drivers' work hours. Moreover, the taxi medallions transfer price and volume reflect the negative enthusiasm toward the taxi market in recent years (see Figure~\ref{fig1}). As a consequence, the overall supply of taxi market and individual work hours may have changed due to the competition from TNCs. The interaction between taxi drivers' labor supply and revenue has been investigated for a long time. This is, in fact, one of the critical debates of \added{'battle of models'} \cite{farber2015you} in the underlying behavioral principles between reference-dependent preferences (RDP) behavior and neo-classical standard (NS) behavior.The NS model, assuming revenue maximizing, is the most prevalent assumption in modeling the drivers' behavior~\cite{yang1998network,qian2017taxi}. It suggests that the labor supply function of taxi drivers is positively sloped so that a transitory increase in the wage results in an increase in work hours. Such behavior is also supported by real-world observations when the taxi market was monopolistic~\cite{farber2015you}. However, studies~\cite{farber2008reference,crawford2011new} speculate the loss-aversion behavior of taxi drivers, where drivers are less inclined to work and drivers' labor supply is even negatively related to the change of wage rate, thus, decreases the productivity when they reach the reference point (\added{beyond a certain hour or an income target})~\cite{RePEc:ags:isfiwp:275783}. \added{At the same time, a broad recognition is that wage elasticity has significant economic consequences. For instance, it is a critical part of the management of the efficiency cost of income taxation in the general equilibrium model}~\cite{ballard1985general}. Drivers' labor supply behavior serves as a crucial input for modeling the competition between TNCs and taxi industry, and it is closely associated with regulations of TNCs and taxi supply. The labor supply behavior and the corresponding wage elasticity are, therefore, vital to investigate how drivers distribute their work hours in a competitive market. These motivate us to systematically quantify the changes in labor supply for the taxi market due to the impact of the TNCs. 


\added{To answer these research questions, we choose the taxi industry in NYC as a case study to understand the impacts of the TNC trips on the labor supply of the taxi market. For the first research question, we investigate how the competition from the TNCs, measured by the number of monthly TNC trips, affects the total work hours and revenue of both the yellow and green taxi markets. These analyses reveal what aspects of the taxi labor supply are significantly affected and how the effects change over time. While the cap for the NYC yellow taxi does not change, we observe a decline in the daily number of active drivers. This motivates us to further investigate the second and third questions, which are the effects of the growth of TNC trips on the taxi drivers' labor supply behavior.} \added{These two questions are examined by analyzing anticipated wage variation and wage elasticity using the wage decomposition method}. To support our analyses, we mine the NYC taxi and TNC trip record between January 2013, when TNCs were in their beginning stage, and December 2018, by the time that TNCs have accounted for over 50\% of the total mobility-on-demand market share (see Figure~\ref{fig1}). \highlighttext{\added{Our results show that 1\% increase in TNC trips in the current market will lead to 0.28\% decrease of monthly revenue in the yellow taxi market and 0.68\% decrease of monthly revenue in the green taxi market. Besides, 1\% increase in TNC trips in the current market will result in 0.29\% reduction of monthly work hours in the yellow taxi industry and 0.75\% reduction of monthly work hours in the green taxi industry.} Furthermore, the labor supply behavior of yellow and green taxis is different, where the unanticipated wage variation has increased by three times for yellow taxi drivers and by five times for the green taxi industry from January 2013 to December 2018}. \highlighttext{\added{This finding favors the co-existence of the NS and RDP behavior among taxi drivers, which supplements the existing literature for either supporting NS assumption or RDP assumption~\cite{farber2015you,crawford2011new}. Besides, it points out that the different labor supply behavior presents between yellow and green taxi drivers under the competition from TNCs, and the NS behavior is dominated at the TNCs' beginning stage. More importantly, it highlights the transition of individual behavior from NS to RDP due to the impact of TNCs, which is yet addressed by previous literature and indicates the potential collapse of the taxi industry if the extreme competition between taxi and TNCs continues.}} The insights have significant implications for taxi regulators and city policymakers in developing market planning models and framing regulatory policies on the cap and pricing management for taxis and for-hire vehicles (FHVs).


The rest of the study is organized as follows. Section 2 briefly reviews related studies. Section 3 outlines the data used in our study. \added{Section 4 introduces the methodology and corresponding hypotheses for the research questions. Section 5 presents the results of the analyses of the impact of TNCs on the overall taxi market. Section 6 presents the examination result of taxi drivers' labor supply behavior and wage elasticity under the presence and growth of TNCs. Section 7 summarizes the significant findings with concluding remarks.}