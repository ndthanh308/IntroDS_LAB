\documentclass[10pt,twocolumn,letterpaper]{article}

\usepackage{iccv}
\usepackage{times}
\usepackage{epsfig}
\usepackage{graphicx}
\usepackage{amsmath}
\usepackage{amssymb}

% Include other packages here, before hyperref.
\usepackage{booktabs}
\usepackage{enumitem}
\usepackage{multirow}
\usepackage{soul}
\usepackage{pifont} % http://ctan.org/pkg/pifont
\usepackage[T1]{fontenc}
\usepackage{mathabx}
\usepackage{makecell}
\usepackage[dvipsnames]{xcolor}
\usepackage{color, colortbl}
\usepackage{caption}

\newcommand{\hao}[1]{\textcolor{brown}{\small{\bf [ #1 -- Hao ]}}}
\newcommand{\mbc}[1]{{\protect\color{blue}{[ #1 --Mohit ]}}}
\newcommand{\blue}[1]{{\textbf{\color{blue}#1}}}
\newcommand\Tstrut{\rule{0pt}{2.3ex}}         % = `top' strut
\newcommand\Bstrut{\rule[-0.9ex]{0pt}{0pt}}   % = `bottom' strut

\def\CircleArrowright{\ensuremath{%
  \rotatebox[origin=c]{310}{$\circlearrowright$}}}
\DeclareRobustCommand{\star}{%
  \begingroup\normalfont
  % Figure removed%
  \endgroup
}
\newcommand{\vlnbert}{VLN$\protect\CircleArrowright$BERT}
\newcommand{\ours}{ScaleVLN}

% If you comment hyperref and then uncomment it, you should delete
% egpaper.aux before re-running latex.  (Or just hit 'q' on the first latex
% run, let it finish, and you should be clear).
\usepackage[pagebackref=true,breaklinks=true,letterpaper=true,colorlinks,bookmarks=false]{hyperref}

\iccvfinalcopy % *** Uncomment this line for the final submission

\def\iccvPaperID{4534} % *** Enter the ICCV Paper ID here
\def\httilde{\mbox{\tt\raisebox{-.5ex}{\symbol{126}}}}

% Pages are numbered in submission mode, and unnumbered in camera-ready
\ificcvfinal\pagestyle{empty}\fi

\begin{document}

%%%%%%%%% TITLE
\title{Scaling Data Generation in Vision-and-Language Navigation}

\author{Zun Wang$^{*\spadesuit1,2}$ \quad Jialu Li$^{*3}$ 
\quad Yicong Hong$^{*\dag1}$ \\
Yi Wang$^2$ \quad Qi Wu$^4$ \quad Mohit Bansal$^3$ \quad Stephen Gould$^1$ \quad
Hao Tan$^5$ \quad Yu Qiao$^2$ \\
$^1$The Australian National University\quad$^2$OpenGVLab, Shanghai AI Laboratory\quad\\
$^3$UNC, Chapel Hill\quad$^4$University of Adelaide\quad
$^5$Adobe Research\\ 
{\tt\small wangzun@pjlab.org.cn, jialuli@cs.unc.edu, mr.yiconghong@gmail.com} \\ 
{\tt\small  Project URL: \url{https://github.com/wz0919/ScaleVLN}}
}

\maketitle
% Remove page # from the first page of camera-ready.
\ificcvfinal\thispagestyle{empty}\fi

%%%%%%%%% ABSTRACT
\begin{abstract}
Recent research in language-guided visual navigation has demonstrated a significant demand for the diversity of traversable environments and the quantity of supervision for training generalizable agents. To tackle the common data scarcity issue in existing vision-and-language navigation datasets, we propose an effective paradigm for generating large-scale data for learning, which applies 1200+ photo-realistic environments from HM3D and Gibson datasets and synthesizes 4.9 million instruction-trajectory pairs using fully-accessible resources on the web. Importantly, we investigate the influence of each component in this paradigm on the agent's performance and study how to adequately apply the augmented data to pre-train and fine-tune an agent. Thanks to our large-scale dataset, the performance of an existing agent can be pushed up (+11\% absolute with regard to previous SoTA) to a significantly new best of 80\% single-run success rate on the R2R test split by simple imitation learning. The long-lasting generalization gap between navigating in seen and unseen environments is also reduced to less than 1\% (versus 8\% in the previous best method). 
Moreover, our paradigm also facilitates different models to achieve new state-of-the-art navigation results on CVDN, REVERIE, and R2R in continuous environments.

{\let\thefootnote\relax\footnote{$^{*}$Equal contribution. $^{\dag}$Project lead.}}
{\let\thefootnote\relax\footnote{ $^{\spadesuit}$Research done during internship at Shanghai AI Lab.}}

\end{abstract}

%%%%%%%%% BODY TEXT

\section{Introduction}

% Figure environment removed

Reinforcement Learning from Human Feedback (RLHF) has recently been used to great effect to align pretrained large language models (LLMs) to human preferences, optimizing for desirable qualities like harmlessness and helpfulness~\citep{bai2022training} and achieving state-of-the-art results across a variety of natural language tasks~\citep{openai2023gpt4}. %RLHF approaches fundamentally rely on collecting pairs of LLM outputs $(o_1, o_2)$ from a shared prompt $p$, with a human indicating which output in each pair is better on a specified attribute.
% A fundamental component of RLHF is a preference model derived from human labels, typically formatted as pairs of LLM outputs $(o_1, o_2)$ generated from a shared prompt $p$.

A standard RLHF procedure fine-tunes an initial unaligned LLM using an RL algorithm such as PPO~\citep{schulman2017proximal}, optimizing the LLM to align with human preferences. %\violet{not sure whether we need to provide this detail in the intro, especially this has nothing to do with our contribution.} % i feel like this context is useful later when e.g. explaining that context distillation is SFT
RLHF is thus critically dependent on a reward model derived from human-labeled preferences, typically \textit{pairwise preferences} on LLM outputs $(o_1, o_2)$ generated from a shared prompt $p$. % and labeled by humans. 

However, collecting human pairwise preference data, especially high-quality data, may be expensive and time consuming at scale. To address this problem, approaches have been proposed to obtain labels without human annotation, such as Reinforcement Learning from AI Feedback (RLAIF) and context distillation. 

\iffalse
raising the question of whether we can generate high-quality data for RLHF without using human labeling. %accurately-labeled preference pairs $(o_1, o_2)$
%, motivating model alignment approaches that aim to generate accurately-labeled preference pairs $(o_1, o_2)$ without human involvement. 
Two major categories of such approaches are . 
\fi

RLAIF approaches (e.g.,~\citet{bai2022constitutional}) simulate human pairwise preferences by scoring $o_1$ and $o_2$ with an LLM (Figure \ref{fig:rlcd_differences} center); the scoring LLM is often the same as the one used to generate the original pairs $(o_1, o_2)$. Of course, the resulting LLM pairwise preferences will be somewhat noisier compared to human labels. However, this problem is exacerbated by using the same prompt $p$ to generate both $o_1$ and $o_2$, causing $o_1$ and $o_2$ to often be of very similar quality and thus hard to differentiate (e.g., Table~\ref{tab:rlaif_bad_example}). Consequently, training signal can be overwhelmed by label noise, yielding lower-quality preference data. 

% While it avoids human labeling efforts, it has weakness. First, LLM preference labels will naturally be somewhat noisier compared to human labels. Furthermore, since the same prompt $p$ is used to generate both $o_1$ and $o_2$, their quality is often very similar and hard to differentiate (See Table~\ref{tab:rlaif_bad_example}). As a result, training signals can be overwhelmed by label noise, yielding lower-quality preference data. 

Meanwhile, context distillation methods (e.g., \citet{sun2023principle}) create more training signal by modifying the initial prompt $p$. 
%to create more significant training signal. 
The modified prompt $p_+$ typically contains additional context encouraging a \textit{directional attribute change} in the output $o_+$ (Figure \ref{fig:rlcd_differences} right). However, context distillation methods only generate a single output $o_+$ per prompt $p_+$, which is then used for supervised fine-tuning, losing the pairwise preferences which help RLHF-style approaches to 
%rather than using a RLHF-style preference model to 
derive signal from the contrast between outputs. 
Multiple works have observed that RL approaches using preference models for pairwise preferences can substantially improve over supervised fine-tuning by itself when aligning LLMs~\citep{ouyang2022training,dubois2023alpacafarm}. 

% conduct alignment by running supervised fine-tuning on model outputs $o_+$ generated from a modified prompt $p_+$. $p_+$ typically contains additional context encouraging desirable attributes (Figure \ref{fig:rlcd_differences} right), such as in \citet{sun2023principle}. However, multiple works have observed that RLHF-style approaches can substantially improve over supervised fine-tuning by itself when aligning LLMs~\citep{ouyang2022training,dubois2023alpacafarm}. 

Therefore, while both RLAIF and context distillation approaches have already been successfully applied in practice to align language models, we posit that it may be even more effective to combine the key advantages of both. That is, we will use RL with \textit{pairwise preferences}, while also using modified prompts to encourage \textit{directional attribute change} in outputs. %In particular, we will adapt the RLAIF data generation process with two different prompts rather than a single $p$, modifying both prompts similarly to context distillation. %\violet{this motivation is a little unexciting. I think we can more specifically discuss the potential benefits of our approach, like the benefits from RL: exploration/data generation; benefits from contrast. I don't think we get too much benefits from context distillation since we switched to the RL framework.} 

Concretely, we propose \oursfull{} (\ours{}). 
\ours{} generates preference data as follows. Rather than producing two i.i.d.\ model outputs $(o_1, o_2)$ from the same prompt $p$ as in RLAIF, \ours{} creates two variations of $p$: a \textit{positive prompt} $p_+$ similar to context distillation which encourages directional change toward a desired attribute, and a \textit{negative prompt} $p_-$ which encourages directional change \textit{against} it (Figure \ref{fig:rlcd_differences} left). We then generate model outputs $(o_+, o_-)$ respectively, and automatically label $o_+$ as preferred---that is, \ours{} automatically ``generates'' pairwise preference labels by construction. %, without further post hoc labeling.\violet{should make it clearer that our approach `generates' labels by construction} 
We then follow the standard RL pipeline of training a preference model followed by PPO. 

Compared to RLAIF-generated preference pairs $(o_1, o_2)$ from the same input prompt $p$, there is typically a clearer difference in the quality of $o_+$ and $o_-$ generated using \ours{}'s directional prompts $p_+$ and $p_-$, which may result in less label noise. %which may result in better training signal for the preference model. 
That is, intuitively, \ours{} exchanges having examples be \textit{closer to the classification boundary} for much more \textit{accurate labels} on average. Compared to standard context distillation methods, on top of leveraging pairwise preferences for RL training, \ours{} can derive signal not only from the positive prompt $p_+$ which improves output quality, but also from the negative prompt $p_-$ which degrades it. %\ours{} is not learning to imitate $o_+$, but to distill the \textit{contrast} between $o_+$ and $o_-$. 
Positive outputs $o_+$ don't need to be perfect; they only need to contrast with $o_-$ on the desired attribute while otherwise following a similar style.

% \todo{discuss our method and why intuitively it may be better.}

We evaluate the practical effectiveness of \ours{} through both human and automatic evaluations on three tasks, aiming to improve the ability of LLaMA-7B~\citep{touvron2023llama} to generate harmless outputs, helpful outputs, and high-quality story outlines. %\ours{} outperforms both RLAIF and context distillation baselines in pairwise comparisons on 
As shown in Sec. \ref{sec:experiments}, \ours{} substantially outperforms both RLAIF and context distillation baselines in pairwise comparisons when simulating preference data with LLaMA-7B, while still performing equal or better when simulating with LLaMA-30B. 
%On all three tasks, \ours{} substantially outperforms both RLAIF and context distillation baselines in pairwise comparisons---by a margin of at least 9\% and often more than 30\%---validating our method's efficacy. 
We will release all code at a later date, although in any case \ours{} is fairly easy to implement by modifying any reference RLAIF codebase. %We release all code at \todo{github link}.
\section{Related Work}
\label{sec:related}

\begin{table}[t]
\small
\centering
\caption{Comparison of our method with related settings}
\begin{tabular}{cccc}
\toprule
Setting & Detect Novel OOD Data & Semi-Supervised & Learns from Novel OOD Data \\
\midrule
SSOD & \xmark & \cmark & \xmark \\ 
Open-World OD & \cmark & \xmark & \cmark \\
Open-Set SSOD & \cmark & \cmark & \xmark \\ \midrule
\textbf{Our Method} & \cmark & \cmark & \cmark \\
\bottomrule
\end{tabular}
\label{tab:comparison}
\end{table}

\paragraph{Semi-Supervised Object Detection.} Semi-supervised object detection (SSOD) approaches have become popular to reduce the need for labeling \cite{sohn2020detection, berthelot2019mixmatch, jeong2019consistency}. Pseudo-labeling based methods such as FlexMatch \cite{zhang2021flexmatch}, TSSDL \cite{shi2018transductive}, and others \cite{iscen2019label, luo2018smooth, yan2019semi, liu2021unbiased, xu2021end}, first train a teacher model using only labeled data and then use that model to create pseudo-labels for unlabeled images. The pseudo-labels are then used along with the original labeled data to train a student model. On the other hand, consistency regularization approaches such as \cite{sajjadi2016regularization, laine2017temporal, tarvainen2017mean, liu2021certainty, luo2018smooth, jeong2019consistency, iscen2019label, liu2021unbiased, xu2021end}, aim to minimize a consistency loss between differently augmented versions of an image. All of these semi-supervised learning approaches assume a ``closed-world'' setting with a fixed set of classes in both training and testing, which is not a valid assumption in real-world applications.

\paragraph{Open-World Object Detection.} Open-world object detection enables the detection of novel objects by incrementally adding novel object classes to the set of known classes. Previous work \cite{kim2022learning, kuo2015deepbox, o2015learning, wang2020leads, Maaz2022Multimodal} has studied different methods of object proposals for novel objects by attempting to remove the notion of class (all objects are regarded the same). ORE \cite{joseph2021towards} is the first to propose an open-world object detector that identifies novel classes as ‘unknown’ and proceeds to learn the unknown classes once the labels become available. \cite{han2022expanding} aims to identify unknown objects by separating high/low-density regions in the latent space. Both these approaches work in a fully-supervised setting. Our setup goes a step further and situates the open-world problem in the context of semi-supervised learning, with limited amounts of labeled ID data \textit{only}, that more closely resembles the real-world settings. 

\paragraph{Unsupervised Object Localization.} Recently proposed methods such as CutLER \cite{wang2023cut}, FreeSolo \cite{wang2022freesolo}, LOST \cite{LOST}, and MOST \cite{rambhatla2023most} propose to localize objects in an unsupervised manner, either by segmentation masks or bounding boxes. Some of these \cite{wang2023cut, LOST, rambhatla2023most} use features from self-supervised trained transformers to localize objects in the scene. In our work, we evaluate the capabilities of such methods for localizing OOD objects, as they present open-world capabilities. Based on our evaluation (\ref{sec:expts:ablation}), we use CutLER as part of the OOD Explorer to localize OOD classes. Section \ref{sec:expts} provides the details of our evaluation. 

\paragraph{Open-Set/Open-world Semi-Supervised Object Detection.}
The open-set semi-supervised object detection problem \cite{liuopen} addressed some of the limitation of the above mentioned work. Furthermore, they address like the performance of ID classes in the presence of OOD data, but they do not learn from it or improve OOD performance. They propose an offline OOD detector to filter out OOD data, thus limiting the risk of ID performance in the presence of OOD data. In contrast, our approach \textit{both} improves performance for ID classes \textit{as well as} OOD classes, i.e., our proposed framework solves a strictly stronger problem. Specifically speaking, \cite{liuopen} solves for identifying novel classes and filters it out, but does not re-introduce the classes back into the training pipeline in order to be able to learn its features. \cite{mullappilly2024semi} addresses some of the limitations of the previous mentioned methods by extending the problem to a semi-supervised setting. However, their problem setting is similar to an incremental learning setting, access to unknown class labels is provided in subsequent tasks. Our generalized setting, on the other hand, does not require access to any unknown class labels. 

\section{Methodology}
\label{sec:method}

\subsection{Overview}
\label{sec:method_fmwk}

As shown in~\cref{fig:method_fmwk}, the proposed unsupervised MOT framework is trained with the widely-used contrastive learning technique~\cite{chen2020simple,he2020momentum}. 
\lk{Specifically, for multi-object tracking}, objects within the tracklet ($\boldsymbol{k}_{+}$) should be pulled together and different tracklets ($\boldsymbol{k}_{-}$) should be separated. It can be mathematically formulated as:

\begin{equation}
% \begin{split}
    \mathcal{L}_{cl}( \boldsymbol{q}; \boldsymbol{k}_{+}; \boldsymbol{k}_{-} )= 
    - \log \frac{\exp(\boldsymbol{q} \cdot \boldsymbol{k}_{+} / \epsilon)}{\sum_{i}\exp(\boldsymbol{q} \cdot \boldsymbol{k}_{i} / \epsilon)}
  \label{eq:method_nce}
% \end{split}  
\end{equation}

\noindent where $\mathcal{L}_{cl}$ denotes the InfoNCE~\cite{oord2018representation} loss function, and $\epsilon$ is the temperature hyper-parameter~\cite{wu2018unsupervised}. 
Within a video, following the unsupervised tracking fashion~\cite{liu2022online,shuai2022id}, the positive and negative keys mainly come from two sources, \ie pseudo-labeled historical frame and self-augmented frame. 

\lk{However, two issues occur: (1) the uncertainty reduces the accuracy of pseudo-tracklets and (2) the randomly augmented samples fail to learn the inter-frame consistency.} 
We argue the above issues are not independent. 
\lk{By leveraging the uncertainty in turn,} the accurate pseudo-tracklets can guide the qualified positive and negative augmentations.

To address these two issues, we propose an uncertainty-aware pseudo-tracklet labeling strategy in \cref{sec:method_uoap}, which integrates a verification-and-rectification mechanism into the tracklet generation. Our method significantly improves the accuracy of pseudo-identities, especially in long-term interval. 
Then we propose a tracklet-guided augmentation strategy in \cref{sec:method_ada_aug}, which brings the temporary information into spatial augmentation. The augmented samples simulates the objects' motion. A hierarchical uncertainty-based sampling strategy is proposed for hard sample mining. More details are described in the following section.


\subsection{Uncertainty-aware Tracklet-Labeling}
\label{sec:method_uoap}

Accurate pseudo tracklet is critical in \liuk{learning feature consistency}. 
However, without manual annotation, \lk{the aggravated uncertainty makes} the tracklet-labeling a huge challenge due to various interference factors, including similar appearance among objects, frequent object cross and occlusions, \etc. 
\lk{In fact, the uncertainty can also be leveraged to improve the pseudo-accuracy in turn.}
In this section, we propose an \textbf{U}ncertainty-aware \textbf{T}racklet-\textbf{L}abeling (\textbf{UTL}) strategy for better pseudo-tracklets.

Given an input video sequence $V = \{I^{1}, I^{2}, \cdots, I^{N}\}$, each frame $I^{t}$ is annotated with the bounding boxes $B^{t} = \{b_{1}^{t}, b_{2}^{t}, \cdots, b_{M^{t}}^{t}\}$ of $M^{t}$ objects in $t_{th}$ frame, where $b_{i}^{t} = (cx_{i}^{t}, cy_{i}^{t}, w_{i}^{t}, h_{i}^{t})$ is the center coordinate and shape of the $i_{th}$ object $o_{i}^{t}$. As shown in~\cref{fig:method_fmwk}, \mywork~generates accurate pseudo-tracklets in four main steps:

1) \textbf{Association}. For a certain object $o_{i}^{t}$ in frame $I^{t}$, the $\ell_2$-normalized representation $\boldsymbol{f}_{i}^{t}$ can be expressed as $\boldsymbol{f}_{i}^{t} = {\phi}(I^{t}, b_{i}^{t})$, 
% \begin{equation}
%   \boldsymbol{f}_{i}^{t} = {\phi}(I^{t}, b_{i}^{t})
%   % / {\Vert {\phi}(I^{t}, b_{i}^{t}) \Vert}_{2}
%   \label{eq:method_feat}
% \end{equation}
where the embedding encoder is denoted as $\phi$.

To associate the objects in frame $I^{t}$ with the objects or trajectories in previous $I^{t \minus 1}$, a similarity matrix is constructed with their appearance embeddings:

\begin{equation}
  \boldsymbol{C} \in \mathbb{R}^{M^{t} \times M^{t \minus 1}}, \;
  c_{i,j} = {\boldsymbol{f}_{i}^{t}} \cdot  \boldsymbol{f}_{j}^{t \minus 1}
  \label{eq:method_matrix}
\end{equation}

\noindent where $c_{i,j}$ represents the cosine similarity between the $i_{th}$ object in frame $I^{t}$ and the $j_{th}$ object (or trajectory) in frame $I^{t \minus 1}$. Then the Hungarian algorithm~\cite{kuhn1955hungarian} is adopted to generate the identity association results.

2) \textbf{Verification}. However, the appearance representations are sometimes unreliable, especially in the unsupervised scenario. To solve this issue, an uncertainty metric is proposed to evaluate the association after the first stage.

% For an object $o_{i}^{t}$ in frame $I^{t}$, the similarities against the $M^{t \minus 1}$ objects in the previous frame can be expressed as:

% \begin{equation}
%   \boldsymbol{s}_{i} = \boldsymbol{C}_{i} = [c_{i,1}, c_{i,2}, \cdots, c_{i,M^{t \minus 1}}]^T
%   \label{eq:method_svec}
% \end{equation}

% Inspired by margin-based OOD detection~\cite{hendrycks2016baseline}, we assume that the assignment ($o_{i}^{t} \!\sim\! o_{j}^{t \minus 1}$) in the association stage is not convincing under the following circumstances:

% \begin{itemize}
%     \setlength{\itemsep}{0pt}
%     \item The assigned similarity between $o_{i}^{t}$ and $o_{j}^{t \minus 1}$ is relatively low (\ie, $c_{i,j} < m_1$).
%     \item The second-highest similarity with others ($c_{i,j_{2}}$) is close to the assigned $o_{j}^{t \minus 1}$ (\ie, $c_{i,j} - c_{i,j_{2}} < m_2$).
% \end{itemize}

% Based on these assumptions, we define an association-level uncertainty metric, which is formulated as:



Object association can be viewed as multi-category classification.
And confidence-score has been proved efficient and effective on detecting mis-classified examples~\cite{hendrycks2016baseline}.
Inspired by this, we propose to detect the mis-associated objects through the similarity-scores.


Given an object $o_{i}^{t}$ associated with $o_{j}^{t \minus 1}$ in the previous frame based on \cref{eq:method_matrix}, the association ($o_{i}^{t} \!\sim\! o_{j}^{t \minus 1}$) is unconvincing in two cases: 
1) the assigned similarity $c_{i,j}$ is relatively low (\eg, partial occlusion or motion blur) and 
2) there are other objects whose similarities are close to the assigned $c_{i,j}$ (\eg, similar appearance or indistinguishable embedding).
It can be formulated as:

\begin{equation}
  c_{i,j} < m_1; \quad c_{i,j_{2}} > c_{i,j} - m_2
  \label{eq:method_margin}
\end{equation}


\noindent 
where $m_1,m_2$ are constant margins. Only the second-highest similarity with others ($c_{i,j_{2}}$) is considered for simplicity.
In an ideal association, $c_{i,j}$ should be close to 1 and $c_{i,j_{2}}$ close to 0.
We thus proposed to estimate the association \lk{risk} as:

% \sigma_{i,j} = - \left( 
% \log c_{i,j} + \log \left( 1 - c_{i,j_{2}} \right)
% + \overline{\log \left( 1 - c_{i,l} \right) }
% \right)  
\begin{equation}
  \sigma_{i,j} = - \log c_{i,j} - \log \left( 1 - c_{i,j_{2}} \right)
  \label{eq:method_energy}
\end{equation}

Detailed derivation process refers to the supplementary materials.
Combining with \cref{eq:method_margin} and \cref{eq:method_energy} , an adaptive threshold is proposed:

\begin{equation}
  % \gamma_{i,j} = -\log \left( 1 + m_2 - c_{i,j} \right) -\log m_1 \left( 1 - m_3 \right)
  \gamma_{i,j} =  -\log m_1 - \log \left( 1 + m_2 - c_{i,j} \right)
  \label{eq:method_border}
\end{equation}

As shown in~\cref{fig:method_verify}, when the \lk{risk} $\sigma_{i,j}$ is higher than the threshold $\gamma_{i,j}$, the assignment ($o_{i}^{t} \!\sim\! o_{j}^{t \minus 1}$) should be re-considered. 
\lk{The \textbf{association uncertainty} is quantified as:}

\begin{equation}
  \delta_{i,j} = \sigma_{i,j} - \gamma_{i,j}
  \label{eq:method_uncertain}
\end{equation}

The results are not sensitive to the exact margins. We set $m_1 = 0.5$ and $m_2 = 0.05$ for a slightly better performance.
% More experimental details are shown in the supplementary materials.

The uncertain pairs after the verification stage and unmatched objects after the association stage are gathered as uncertain candidates for the rectification stage.


3) \textbf{Rectification}. 
The rectification stage is performed among the uncertain candidate. The similarities between two adjacent frames are no longer convincing.
% due to irregular motion, severe occlusion, and so on. 
More information should be taken into account, including motion \lk{estimation} and appearance \lk{variation} within a tracklet. 
% Specifically, intersection-over-union (IoU)~\cite{bewley2016simple} is the widely-used motion metric. At the same time, the tracklet embeddings can provide complementary appearance information.

For the uncertain candidates, \mywork~constructs another similarity matrix for the secondary rectification. 
First, \lk{the motion constraints should be relaxed}, so the association shares overlap \lk{higher than} $\beta$ 
% in adjacent frames 
\lk{are preserved}. 
Second, \lk{the appearance should not vary extremely fast}, so we adopt the averaged similarity between object $o_{i}^{t}$ and tracklet $trk_{j} = \{o_{j}^{t \minus K}, \cdots, o_{j}^{t \minus 1}\}$ within previous $K$ frames. 
In this stage, we solve the sub-problem of global identity assignments, which can be formulated as:

\begin{equation}
\begin{split}
  \boldsymbol{C}^\prime \in \mathbb{R}^{{M^{t}}^\prime \times {M^{t \minus 1}}^\prime} & \\
  c^\prime_{i,j} = \left( \frac{1}{K} \sum_{\hat{t} = t \minus K}^{t \minus 1} {\boldsymbol{f}_{i}^{t}} \cdot  \boldsymbol{f}_{j}^{\hat{t}} \right) 
            \times \mathbb{I} & \left( \text{IoU} \left( b_{i}^{t}, b_{j}^{t \minus 1} \right) > \beta \right) 
  \label{eq:method_recti}
\end{split}
\end{equation}

\noindent where $\mathbb{I}(*)$ is the indicator function. Then the match set is updated based on the Hungarian algorithm.

\lk{
\textit{Remark.} Our core contribution is the uncertainty-based verification mechanism, rather than the specific rectification, which shall be adjusted in practice. Empirically we set $\beta=0.1$ and $K=5$.
}

% Figure environment removed

4) \textbf{Propagation}. The pseudo-tracklets are propagated frame-by-frame. As shown in~\cref{fig:method_reidacc}, our strategy brings \lk{consistently} accurate pseudo-identities, \lk{\eg, reaching 97\% accuracy across 100 frames}.
% The pseudo-tracklets are progressively updated during the training process.
The long-term intra-tracklet consistency is successfully maintained.
% by the accurate pseudo-identities.

It is worth mentioning that the \lk{verification and rectification} stages can be naturally applied to the inference process to boost the performance, \lk{which does not conflict with existing association methods}.

\subsection{Tracklet-Guided Augmentation}
\label{sec:method_ada_aug}

The accurate pseudo-tracklets can guide the sample augmentation in the unsupervised MOT framework.
To learn the \liuk{inter-frame consistency}~\cite{chen2020simple,zhang2021fairmot}, good training samples should be diverse and \liuk{temporal-aware}. 
However, as illustrated in~\cref{fig:method_ada_aug}, existing methods usually treat augmentation and multi-object tracking as two isolated tasks, leading to ineffective augmentations. 
Instead, this paper utilizes the tracklet's spatial displacements to guide the augmentation process. 
According to a properly selected anchor pair, the proposed strategy makes the augmented frames aligned to the historical frames, simulating realistic tracklet movements. The proposed method concurrently focuses on the hard negative samples.
Details \lk{of the \textbf{T}racklet-\textbf{G}uided \textbf{A}ugmentation (TGA)} are given below.

% Figure environment removed

We introduce the temporal information into spatial transformation. 
First, given a current frame $I^{t}$ with $M^{t}$ objects, we select a source-anchor object $o_{a}^{t}$, whose bounding box is denoted as $b_{a}^{t} = (cx_{a}^{t}, cy_{a}^{t}, w_{a}^{t}, h_{a}^{t})$. Then, we choose a target-anchor $o_{a}^{t \minus \tau}$ in $(t \minus \tau)_{th}$  historical frame from the pseudo-tracklet $trk_{a} = \{o_{a}^{t_0}, o_{a}^{t_1}, \cdots, o_{a}^{t}\}$. 
Finally, to augment the current $I^{t}$ to align with historical $I^{t \minus \tau}$,  a tracklet-guided affine transformation can be expressed as:

\begin{equation}
  \begin{bmatrix}
      x^{t \minus \tau} \\ y^{t \minus \tau} \\ 1
  \end{bmatrix}
  =
  \boldsymbol{M}_{t}^{t \minus \tau}
  \begin{bmatrix}
      x^{t} \\ y^{t} \\ 1
  \end{bmatrix}
  =
  \begin{bmatrix}
      m_{11} & m_{12} & m_{13} \\
      m_{21} & m_{22} & m_{23} \\
      0      & 0      & 1
  \end{bmatrix}
  \begin{bmatrix}
      x^{t} \\ y^{t} \\ 1
  \end{bmatrix}
  \label{eq:method_affine}
\end{equation}

\noindent where $x^*,y^*$ are spatial coordinates, and $\boldsymbol{M}_{t}^{t \minus \tau}$ can be solved by direct linear transform (DLT) algorithm ~\cite{detone2016deep}. 
% with the corner locations of the anchor pair $(o_{a}^{t} \!\sim\! o_{a}^{t \minus \tau})$. 
Then an augmented frame $\tilde{I}^{t}$ is generated based on the tracklet-guided affine transformation with perspective jitter, which can be expressed as $\tilde{I}^{t} = \mathcal{T}\left(I^{t}, M_{t}^{t \minus \tau} \right)$.
% \begin{equation}
%   \tilde{I}^{t} = \mathcal{T}\left(I^{t}, M_{t}^{t \minus \tau} \right)
%   \label{eq:method_aug}
% \end{equation}

Intuitively, a proper anchor-selection is vitally important for our augmentation strategy. 

First, the identity accuracy of anchor pair $(o_{a}^{t} \!\sim\! o_{a}^{t \minus \tau})$ is important. In other words, the consistency of anchor tracklet $trk_{a}$ should be guaranteed. We thus design a tracklet-level uncertain metric based on the propagated association-level uncertainty defined in \cref{eq:method_uncertain}, which is formulated as:

\begin{equation}
  \Omega_{i} = \frac{1}{n} \sum_{s=t_0}^{t} \exp (\delta_{i}^{s})
  % \Omega_{i} = \sqrt[n]{\sigma_{i}^{t_0} \cdot \sigma_{i}^{t_1} \cdots \sigma_{i}^{t}}
  \label{eq:method_tenergy}
\end{equation}

\noindent where $\Omega_{i}$ represents the uncertainty of tracklet $trk_{i}$, \lk{and $n$ is the tracklet length}.
An uncertainty-based sampling strategy is designed to select the source anchor $o_{a}^{t}$ (along with the anchor $trk_{a}$) from the $M^{t}$ objects in frame $I^{t}$, which can be formulated as:

\begin{equation}
  p\left(a=i \mid t \right) 
  % = softmax\left(-\Omega_{i}\right)
  = \frac{\exp{\left(-\Omega_{i}\right)}}{\sum_{\hat{i}=1}^{M^{t}}\exp{\left(-\Omega_{\hat{i}}\right)}}
  \label{eq:method_sel_an_src}
\end{equation}

\noindent where $p\left(a=i \mid t \right)$ represents the probability to choose the $i_{th}$ tracklet $trk_{i}$ as the anchor $trk_{a}$.
The uncertain tracklet with high $\Omega$ is less likely to be selected, avoiding dramatic augmentations from erroneous pseudo-tracklets.

Second, hard negative samples matters in discriminablity learning. We tend to choose an indistinguishable (or, high uncertain) target anchor $o_{a}^{t \minus \tau}$ along the tracklet $trk_{i}$. The selection probability can be formulated as:

\begin{equation}
  p\left(\pi=t \minus \tau \mid a \right) 
  = \frac{\exp{\left(\delta_{a}^{t \minus \tau}\right)}}{\sum_{\hat{\tau}=t_0}^{t-1}\exp{\left(\delta_{a}^{t-\hat{\tau}}\right)}}
  \label{eq:method_sel_an_tgt}
\end{equation}

\lk{A visualization example are displayed in the supplementary material to illustrate the hierarchical sampling process.}

Compared with conventional random transformation, the proposed tracklet-guided augmentation is well-directed and tracking-related. 
\lk{Together with accurate pseudo-tracklets, \mywork~successfully improves the inter-frame consistency, as shown in \cref{fig:method_disc_vis}. }


% Figure environment removed

% \subsection{Momentum Memory Dictionary}
% \label{sec:method_md}


%To reuse the encoded samples from the intermediate mini-batches, we maintain a queue for each video in the memory dictionary by enqueueing the $M^{t}$ objects in the current frame and removing the oldest samples.
%In representation learning, high-quality negative samples play an essential role~\cite{chen2020simple,he2020momentum}. However, existing unsupervised trackers only take negative samples from adjacent frames, augmented frames, and the current frame itself. The lack of negative sample diversity prevents trackers from learning discriminative representations. \mywork~adopts a momentum dictionary mechanism to alleviate this problem.

%As shown in~\cref{fig:method_fmwk}, we build a memory dictionary for each \textit{independent} video input during training. Given an input image $I^{t}$ from video $V$, we randomly sample a number of negative object samples from other videos in the memory dictionary, so as to enrich the negative sample diversity. To reuse the encoded samples from the intermediate mini-batches, we maintain a queue for each video in the memory dictionary by enqueueing the $M^{t}$ objects in the current frame and removing the oldest samples.
% \vspace{-0.5em}
\section{Experiments}
\label{sec:exp}
% \vspace{-0.5em}
% This paper introduces a novel concept called temporally coupled attacks, which distinguishes itself from standard adversarial attacks by incorporating temporal coupling. Previous research has primarily focused on attackers with different functionalities, specifically targeting either the state space or the action space.
In our experiments, we investigate various types of attackers on different attack domains including state perturbations, action perturbations, model uncertainty and mixed perturbations. We will study a diverse set of attack and compare with state-of-the-art baselines.
% This evaluation sheds light on the effectiveness of \ours across a wide range of attack scenarios against different types of adversaries.

\noindent\textbf{Experiment setup.} \quad
Our experiments are conducted on five various and challenging MuJoCo environments: Hopper, Walker2d, Halfcheetah, Ant, and Humanoid, all using the v2 version of MuJoCo. We use the Proximal Policy Optimization (PPO) algorithm as the policy optimizer for \ours training. For attack constraint $\epsilon$, we use the commonly adopted values $\epsilon$ for each environment. We set the  temporally-coupled constraint $\bar{\epsilon} = \epsilon/5$ (with minor adjustments in some environments). Ablation experiments study the choice ofOther choices of $\bar{\epsilon}$ will be further discussed in the ablation studies.
Our experiments are conducted on five various and challenging MuJoCo environments: Hopper, Walker2d, Halfcheetah, Ant, and Humanoid, all using the v2 version of MuJoCo. We use the Proximal Policy Optimization (PPO) algorithm as the policy optimizer for \ours training. For attack constraint $\epsilon$, we use the commonly adopted values $\epsilon$ for each environment. We set the  temporally-coupled constraint $\bar{\epsilon} = \epsilon/5$ (with minor adjustments in some environments). Ablation experiments study the choice of $\bar{\epsilon}$.

We report the average test episodic rewards both under no attack and against the strongest adversarial attacks to reflect both the natural performance and robustness of trained agents, by training adversaries targeting the trained agents from scratch. For reproducibility, we train each agent configuration with 10 seeds and report the one with the median robust performance, rather than the best one. More implementation details are in Appendix~\ref{app:exp:imp}.
% Figure environment removed
% \vspace{-0.5em}
\paragraph{Case I: Robustness against state perturbations.}
In this experiment, our focus is on evaluating the robustness of our methods against state adversaries that perturb the states received by the agent. Among the alternating training~\citep{zhang2021robust, sun2021strongest} methods, PA-ATLA-PPO is the most robust, which trains with the standard strongest PA-AD attacker. As a modification, we train PA-ATLA-PPO* with a temporally-coupled PA-AD attacker. WocaR-PPO~\citep{liang2022efficient} is the state-of-the-art defense method against state adversaries. Our \ours method utilizes the temporally-coupled PA-AD attacker for training. Figure~\ref{fig:state_attacks} presents the performance of baseline and \ours under both non-temporally-coupled and temporally-coupled state perturbations. 

Despite being trained to handle temporally-coupled adversaries, our method also demonstrates strong performance in the non-robust (``natural'') setting, expecially on the high-dimensional Humanoid task.
% Even without training with a non-temporally-coupled state adversary, our method demonstrates better robustness under the non-temporally-coupled type of attack, particularly in the highest-dimensional and challenging environment, Humanoid, where it outperforms other methods by a large margin.
Under our temporally-coupled attacks, the average performance of \ours is 45\% higher than the strongest baseline.
% \ours shows the best robustness against all types of state adversarial attacks.
% Figure environment removed

% \vspace{-0.5em}
\paragraph{Case II: Robustness against action uncertainty.}
Beyond assessing the susceptibility of \ours to state attacks, we also investigate its robustness against action uncertainty, where the agent intends to execute an action but ultimately takes a different action than anticipated. We scrutinize two specific forms of action uncertainty, as outlined in prior work~\citep{tessler2019action}. The first one is action perturbations, introduced by an action adversary, which strategically adds noise to the agent's intended action. The second scenario revolves around model uncertainty, where, with a probability denoted as $\alpha$, an alternative action replaces the originally planned action output by the agent. These scenarios closely parallel real-world control situations, such as dealing with mass uncertainty (e.g., when a robot's weight changes) or facing sudden, substantial external forces (e.g., when an external force unexpectedly pushes a robot).

In our baseline comparisons, we include PR-MDP and NR-MDP~\citep{tessler2019action}, which are robust to action noise and model uncertainty. We also incorporate WocaR-PPO into our baseline evaluations. We train \ours using a temporally-coupled action adversary and evaluate its robustness in both action perturbation and model uncertainty scenarios.

\noindent\textbf{Action Perturbations.}\quad
To obtain a stronger evasion action perturbation rather than OU noise and parameter noise, we are the first to train an RL-based action adversary following the trajectory outlined in Algorithm~\ref{alg:ours}. This strategy aims to showcase the worst-case performance of our robust agents under action perturbations. For evaluation, we train both temporally-coupled and non-temporally-coupled action adversaries for each robust model. In Figure~\ref{fig:action_attacks}, we present the exceptional performance of \ours against standard and temporally-coupling action perturbations. \ours  demonstrates a high degree of robustness. For example, on the Humanoid task it outperforms the baselines by a 17\% margin for standard attacks and by a 40\% advantage against temporally-coupling action attacks. 
% Across other tasks, \ours  outperforms other methods, especially when confronted with temporally-coupling action attacks, where it exhibits a significant advantage.
These results provide evidence of \ours's defense mechanism against various types of adversarial attacks in the action space.

% Figure environment removed
\noindent\textbf{Model Uncertainty.}\quad
To evaluate robustness under model uncertainty, we consider a range of noise probabilities denoted as $\alpha$ in the range of [0, 0.05, 0.1, 0.15, 0.2]. These values represent the probability of a randomly generated noise replacing the action selected by the victim agent. As depicted in Figure~\ref{fig:uncertainty}, \ours exhibits superior robustness compared to action-robust baselines across a spectrum of $\alpha$ uncertainty value without explicit exposure to model uncertainty noises during training.
% Figure environment removed
\vspace{-0.5em}
\paragraph{Case III: Robustness against mixed adversaries.}
In prior works, adversarial attacks typically focused on perturbing either the agent's observations or introducing noise to the action space. However, in real-world scenarios, agents may encounter both types of attacks simultaneously. To address this challenge, we propose a mixed adversary, which allows the adversary to perturb the agent's state and action at each time step. We employ alternating training to create a baseline as Mixed-ATLA using this mixed adversary type. Our \ours model and Mixed-ATLA are trained with temporally-coupled mixed attackers. The detailed algorithm for the mixed adversary is provided in Appendix~\ref{alg:mixed-ad}.
% We believe that a well-trained mixed adversary not only holds practical significance but also provides a more comprehensive validation of the effectiveness of \ours in enhancing robustness.

Our results in Figure~\ref{fig:mixed_attacks} indicate that the combination of two different forms of attacks can  target robust agents in most scenarios, providing strong evidence of their robustness. \ours outperforms other methods in all five environments against non-temporally-coupled mixed adversaries, with a margin of over 20\% in the Humanoid environment. Moreover, when defending against temporally-coupled mixed attacks, \ours outperforms baselines by  30\% in multiple environments, with a minimum improvement of 10\%.
% These results clearly demonstrate the robustness of \ours against attackers that can target across domains.

\noindent\textbf{Natural Performance.}\quad
We also evaluate the natural performance of \ours and the baselines, as shown in Figure~\ref{fig:state_natural}, which compares natural rewards vs. rewards under the strongest temporally-coupled attacks. It is evident that while achieving robustness, \ours maintains a comparable natural performance with the baselines; the agent's performance does not degrade significantly in environments without adversaries. The natural performance comparing \ours with action-robust models can be found in Appendix~\ref{app:natural}.
% Figure environment removed

%\textbf{Summary.} We calculated the average normalized rewards for each evaluation metric and each robust agent in all the environments as in Figure~\ref{fig:exp}. This visualization vividly showcases that \ours demonstrates notably superior robustness under both standard and temporally-coupled attacks, in comparison to other approaches. Overall, these findings emphasize our empirical potential and contributions of \ours and provide intuitive insights into improving the robustness of agents through a novel and convincing evaluation framework for robust RL.

\begin{wrapfigure}{r}{0.35\textwidth}
% \vspace{-1em}
    \centering
    % Figure removed
    % \vspace{-0.5em}
    \caption{Ablated studies for $\bar{\epsilon}$.
    % $\pi_\theta$ is the acting policy being trained by $\loss_{\pi_\theta}$ defined in Equation~\eqref{loss:policy}, including the original loss of the base DRL algorithm $\lossrl$, a regularization term $\lossreg$, as well as $\lossworst$, a term for improving the \worstqname based on $\worstcritic$. Here $\worstcritic$ which estimates the \worstqname of $\pi_\theta$ is updated by $\loss_{\worstcritic}=\lossest$ depending on $\pi_\theta$.  
    }
    \label{fig:eps_}
% \vspace{-1em}
\end{wrapfigure}
\noindent\textbf{Ablation studies for temporally-coupled constraint $\bar{\epsilon}$.} \quad
As defined in our framework, the temporally-coupled constraint $\bar{\epsilon}$ limits the perturbations within a range that varies between timesteps. When $\bar{\epsilon}$ is set too large, the constraint becomes ineffective, resembling a standard attacker. 
Conversely, setting $\bar{\epsilon}$ close to zero overly restricts perturbations, leading to a decline in attack performance. An appropriate value for $\bar{\epsilon}$ is critical for effective temporally-coupled attacks. Figure~\ref{fig:eps_} illustrates the performance of robust models against temporally-coupled state attackers trained with different maximum $\bar{\epsilon}$. For WocaR-PPO, the temporally-coupled attacker achieves good performance when the values of $\bar{\epsilon}$ are set to 0.02. As the $\bar{\epsilon}$ values increase and the temporally-coupled constraint weakens, the agent's performance improves, indicating a decrease in the adversary's attack effectiveness. In the case of \ours agents, they consistently maintain robust performance as the $\bar{\epsilon}$ values become larger. This observation highlights the impact of temporal coupling on the vulnerability of robust baselines to such attacks. In contrast, \ours agents consistently demonstrate robustness against these attacks. \looseness=-1






We proposed a machine-learning based method to approximate diagonal as well as non-diagonal elements of the Hessian of a molecule. The representation used is specific for every internal coordinates, and takes explicitly into account the bond order, which is sensible because a single point DFT calculation is computationally considerably less expensive that the explicit calculation of the Hessian.
We trained our ML model on a relatively small dataset (subset of QM7) of less than 7000 molecules. The Hessian was computed at the B3LYP/cc-pVDZ level of theory. 
The agreement between ML and DFT was satisfactory. In particular, the calculated MAPE for bond stretching force constant was below 2\%, and was particularly small for bonds involving hydrogen atoms because they point outwards and are less affected by the chemical environment. The MAPE for bending and torsion was of 5\% and 10\%, respectively. 
From the ML model trained on QM7 we were also able to predict the Hessian of some molecules representative of the QM9 dataset. The Hessian predicted in internal coordinates was then transformed into the mass-weighted Cartesian Hessian, the diagonalization of which yields the harmonic vibrational frequencies and normal modes, that can be compared to the ones calculated  explicitly from DFT.

High frequency vibrations and normal modes were predicted accurately, while lower frequency ones were not. This behaviour is analogous to the IR spectroscopy theory, where stretchings and bendings can be identified accurately, while torsion and delocalized vibrations are more difficult to be interpreted.

The approximate Hessian obtained with ML is computational inexpensive and can be used as an initial Hessian guess for geometry optimization, or in the context of alchemical geometry relaxation \cite{Domenichini2020,domenichini2022alchemical, shiraogawa2022exploration,shiraogawa2023optimization}. 
A good starting Hessian may speed up the convergence of the geometrical optimization. An in detail assessment of the performance of the ML Hessian proposed is not yet provided, but should carefully take into account many parameters on which the optimization depends, \textit{e.g.} the type of molecule, the initial geometry, the optimization algorithm, and the Hessian update scheme.


\vspace{-8pt}
\section{Acknowlegement}
\vspace{-6pt}
We thank ICCV reviewers for their helpful suggestions. This work is partially supported by the National Key R\&D Program of China (NO. 2022ZD0160100), and in part by the Shanghai Committee of Science and Technology (Grant No. 21DZ1100100).
We warmly thank Taesung Park for helpful suggestions on recovering rendered images.

%% ----------------------------------------

{\small
\bibliographystyle{ieee_fullname}
\bibliography{egbib}
}

\clearpage
\appendix
% \clearpage
\section*{Appendices}

We first describe the implementation details of our experiments in Sec.~\ref{sec_1}, including pre-training objectives and details of REVERIE experiments. In Sec.~\ref{sec_2}, we provide additional experiments about the effects of visual encoders, model initialization, and adding depth features. We then discuss the impact of ScaleVLN on different VLN agents and on learning the long-horizon VLN task (R4R). Leaderboard results of R2R and object grounding results for REVERIE are also included. Sec.~\ref{sec_3} and Sec.~\ref{sec_4} visualize our navigability graphs and the recovered images from Co-Modulated GAN~\cite{zhao2021comodgan}.

\section{Implementation Details (\S4\protect\footnote{Link to Section 4 in Main Paper.})}

\label{sec_1}


\subsection{Pre-Training Objectives \texorpdfstring{($\boldsymbol{\S}$}~4.1)}


 We mainly employ three proxy tasks, MLM, MRM, and SAP, for pre-training the agent. Here we describe these proxy tasks in detail. The inputs for these tasks are instruction $\mathcal{W}$ and demonstration path $\mathcal{P}$. During training, we randomly sample one task for each iteration with equal probability.

\paragraph{Masked Language Modeling (MLM)} involves predicting masked words based on textual context and the full trajectory. A special \verb|[mask]| token is used to randomly mask out 15\% of the tokens in $\mathcal{W}$. We predict the masked word distribution $p (w_i|\mathcal{W}_{\backslash i}, \mathcal{P})=f_{\text{MLM}}(x'_i)$ through a two-layer fully-connected network, where $\mathcal{W}_{\backslash i}$ is the masked instruction and $x'_i$ is the output embedding of the masked word $w_i$. The objective is to minimize the negative log-likelihood of predicting the original words: $\mathcal{L}_{\text{MLM}} = - \mathrm{log}\ p (w_i|\mathcal{W}_{\backslash i}, \mathcal{P})$. 
% The task is helpful for learning grounded language representations.


\paragraph{Masked Region Modeling (MRM)} is to predict labels for masked regions in history observations based on instructions and neighboring regions.  To achieve this, we randomly remove view images in $\mathcal{P}$ with a 15\% probability. For view images, the target labels are determined by an image classification model~\cite{dosovitskiy2020vit} pre-trained on ImageNet. To predict semantic labels for each masked visual token, we use a two-layer fully-connected network. The objective is to minimize the KL-divergence between the predicted and target probability distribution. 
% Note that we only apply MRM in ablation studies in \texorpdfstring{$\boldsymbol{\S}$}~4.3 (Effect of Pre-training Tasks) as we found it doesn't help with large-scale training data.


\paragraph{Single Action Prediction (SAP)} aims to predict the next action based on the instruction and the given path. Following \cite{chen2022duet}, we predict the probability for each candidate action in the action space via a two-layer fully-connected network. The objective is to minimize the negative log probability of the target view action $\mathcal{L}_{\mathrm{SAP}} = -\mathrm{log}\ p_t(a^*_t | \mathcal{W}, \mathcal{P}_{<t})$. 
% This task helps the model to make action decision conditioning on instruction and contextual path history.


\subsection{Implementation Details of REVERIE \texorpdfstring{($\boldsymbol{\S}$}~4.1)}

% \paragraph{All Details about REVERIE}

REVERIE data contains trajectories that lead to target objects specified by high-level instructions. Following AutoVLN~\cite{chen2022hm3dlearning}, for every visible object at a viewpoint, we sample paths with an edge length between 4 and 9 that end at the viewpoint. We filter out objects that are more than 3 meters away from the central of the viewpoint, resulting in 518,233 paths from HM3D, and 311,976 paths from the Gibson environments.
To generate instructions in REVERIE-style, we modify the GPT-2 architecture used in AutoVLN~\cite{chen2022hm3dlearning} by only encoding the target object in the final viewpoint as the prompt to generate the instructions. Our large-scale data augmentation paradigm creates 830,209 instruction-trajectory pairs for training. This size is $\times$38 larger than the original REVERIE dataset, and $\times$3.81 larger than the augmented dataset in AutoVLN~\cite{chen2022hm3dlearning}.


We follow DUET and SIA~\cite{lin2021sia} to pre-train the model with an additional Object Grounding (OG) task, which requires selecting a target from object candidates based on high-level instruction and observations along the path. We use CLIP ViT-H/14~\cite{radford2021clip} to extract the image features, and ViT-B/16~\cite{dosovitskiy2020vit} pre-trained on ImageNet to extract the object features.
We pre-train DUET for 100k iterations with a batch size of 128 and a learning rate of $5\times 10^{-5}$ on both HM3D and Gibson environments. We compare three model checkpoints at 30k, 40k, and 50k and pick the one with the highest fine-tuning performance. Then we fine-tune DUET for 150k iterations, with batch size 32 and learning rate $2\times 10^{-5}$ on a single NVIDIA A100 GPU.


\section{Additional Experiments (\S4)}
\label{sec_2}

Here we provide additional experiments to investigate the effect of visual encoder, model initialization, and depth features. We also experiment with different model architectures (\textit{i.e.}, HAMT~\cite{chen2021hamt}) on R2R dataset, and show object grounding results for the REVERIE task.

\subsection{Effect of Visual Encoders (\texorpdfstring{$\boldsymbol{\S}$}~4.2)}

We study the effect of visual encoders in Table \ref{tab:visual_encoders.}. Here we adopt CLIP's ViT backbone with different model sizes and input patches (\textit{i.e.}, Base/16, Large/14, and Huge/14).
We can see that the vision encoder has a major influence on SPL, suggesting the agent can make fewer wrong steps and is capable of efficient navigation.




\begin{table}[h]
  \begin{center}
  \resizebox{\columnwidth}{!}{
  \begin{tabular}{l|cccc|rrrr}
    \hline \hline
     \multirow{2}{*}{Visual Encoders} & \multicolumn{4}{c|}{R2R Val-Seen} &\multicolumn{4}{c}{R2R Val-Unseen} \\
     \cline{2-9} & \multicolumn{1}{c}{TL}& \multicolumn{1}{c}{NE$\downarrow$} & \multicolumn{1}{c}{SR$\uparrow$} &
    \multicolumn{1}{c|}{SPL$\uparrow$} & \multicolumn{1}{c}{TL}& \multicolumn{1}{c}{NE$\downarrow$} & \multicolumn{1}{c}{SR$\uparrow$} &
    \multicolumn{1}{c}{SPL$\uparrow$} \Tstrut\\
    \hline \hline
    CLIP-ViT-B/16 &    12.41 & \textbf{2.02} & 80.51 & 74.88 &
      13.16 & 2.53 & 78.08 & 68.31 \\
    CLIP-ViT-L/14 & 12.62 & 2.16 & 80.04 & 74.06 & 13.13 & 2.50 & 78.08 & 68.97 \\
    CLIP-ViT-H/14 & 12.53 & 2.15 & \textbf{81.19} & \textbf{76.83} & 12.61 & \textbf{2.49} & \textbf{78.20} & \textbf{69.71} \\
    \hline \hline
  \end{tabular}}
\end{center}
\vspace{-5pt}
\caption{ Effect of visual encoders.}
\label{tab:visual_encoders.}
\end{table}

\subsection{Effect of Initialization (\texorpdfstring{$\boldsymbol{\S}$}~4.2)}

Table \ref{tab:language_init} presents the performance of initializing the navigation agent with different pre-trained models in pre-training. We discovered that utilizing BERT to initialize the language encoder does not enhance downstream performance, and even harms the performance on the validation unseen set. We attribute this to the vast domain gap between uni-modal BERT's language representations and CLIP's visual representation. Results could be improved by initializing the model with LXMERT's language encoder~\cite{tan2019lxmert}, and even more by utilizing both the language encoder and cross-modal encoder from LXMERT, indicating that incorporating pre-trained vision-and-language models could benefit agent performance.

\begin{table}[h]
  \begin{center}
  \resizebox{\columnwidth}{!}{
  \begin{tabular}{l|cccc|rrrr}
    \hline \hline
     \multirow{2}{*}{\makecell{Language Encoder \\ Initialization}} & \multicolumn{4}{c|}{R2R Val-Seen} &\multicolumn{4}{c}{R2R Val-Unseen} \\
     \cline{2-9} &  \multicolumn{1}{c}{TL}&\multicolumn{1}{c}{NE$\downarrow$} & \multicolumn{1}{c}{SR$\uparrow$} &
    \multicolumn{1}{c|}{SPL$\uparrow$} &  \multicolumn{1}{c}{TL}&\multicolumn{1}{c}{NE$\downarrow$} & \multicolumn{1}{c}{SR$\uparrow$} &
    \multicolumn{1}{c}{SPL$\uparrow$} \Tstrut\\
    \hline \hline
    Random & 12.87 & 2.29  & 78.75 & 72.61  & 12.69 & 2.72 & 75.65 &67.00  \\
    BERT & 12.43 &2.29 & 79.04 & 73.72 & 12.95 &2.76 & 75.01 & 66.57 \\
    LXMERT (lang.) & 11.73 & \textbf{2.07} & \textbf{80.22} & \textbf{75.65} & 13.17 & 2.67 & 75.86 & 67.36 \\
    LXMERT (lang.+cross.) & 12.63 & 2.27 & 79.24 & 73.34 &
    12.83 & \textbf{2.62} & \textbf{76.59} & \textbf{67.74} \\
    \hline \hline
  \end{tabular}}
\end{center}
\vspace{-5pt}
\caption{Effect of different initialization, where \textit{LXMERT (lang.)} means only initialize the language encoder with LXMERT, and \textit{LXMERT (lang.+cross.)} means initialize both the langauge encoder and cross modal encoder with LXMERT. }
\label{tab:language_init}
\end{table}

\subsection{Effect of Depth Modality (\texorpdfstring{$\boldsymbol{\S}$}~4.2)}

We also explored leveraging depth information to improve visual representations as described in Table \ref{tab:depth}. In line with previous methods such as \cite{krantz2020navgraph,krantz2021waypoint,hong2022bridging,an20221st}, we directly concatenate the depth features from DDPPO~\cite{wijmans2020ddppo} (a ResNet backbone pre-trained on PointGoal navigation with depth inputs) and the RGB features (from CLIP ViT-B/16) to create the visual representations. Our findings indicate that when not using HM3D as the augmented environment, the agent's SR is significantly better if learning from the additional depth input. However, this conclusion changes when HM3D environments are involved: the agent's SR with RGBD was slightly lower than with RGB-only. We suspect that as the data is scaled up with more visual observations and language instructions, the agent may not require additional depth information to assist decision-making.

\begin{table}[h]
  \begin{center}
  \resizebox{\columnwidth}{!}{
  \begin{tabular}{c|c|rrrr|rrrr}
    \hline \hline
     \multicolumn{1}{c|}{\multirow{2}{*}{HM3D Aug}} & \multicolumn{1}{c|}{\multirow{2}{*}{Sensor}} & \multicolumn{4}{c|}{R2R Val-Seen} &\multicolumn{4}{c}{R2R Val-Unseen} \\
     \cline{3-10} &  & \multicolumn{1}{c}{TL} & \multicolumn{1}{c}{NE$\downarrow$} & \multicolumn{1}{c}{SR$\uparrow$} &
    \multicolumn{1}{c|}{SPL$\uparrow$} & \multicolumn{1}{c}{TL} & \multicolumn{1}{c}{NE$\downarrow$} & \multicolumn{1}{c}{SR$\uparrow$} &
    \multicolumn{1}{c}{SPL$\uparrow$} \Tstrut\\
    \hline \hline
    \multirow{2}{*}{$\times$} & RGB & 13.28 & \textbf{2.51} & 76.89 & 69.71 &
     13.53 & 3.06 & 72.92 & \textbf{62.82} \\
     & RGBD & 14.16 & 2.54 & \textbf{77.18} & \textbf{69.76} &15.14 & \textbf{3.02} & \textbf{74.12} & 62.54 \\
    \hline
    \multirow{2}{*}{\checkmark} & RGB  & 12.63 & 2.27 & 79.24 & 73.34 &
     12.83 & \textbf{2.62} & \textbf{76.59} & 67.74 \\
     & RGBD & 11.24 & \textbf{2.12} & \textbf{79.73} & \textbf{75.45} & 12.93 & 2.63 & 76.46 & \textbf{68.52} \\
    \hline \hline
  \end{tabular}}
\end{center}
\vspace{-5pt}
\caption{ Effect of adding depth modality.}
\label{tab:depth}
\end{table}

\subsection{ScaleVLN with Different VLN Models (\texorpdfstring{$\boldsymbol{\S}$}~4.2)} 


To evaluate the generalization ability of our \ours{} dataset, we also apply the augmented data to train different VLN agents, including Seq2Seq~\cite{anderson2018r2r}, EnvDrop~\cite{tan2019envdrop}, and HAMT \cite{chen2021hamt}. 
The HAMT model is pre-trained and fine-tuned with the same data and configurations as we pre-trained the DUET model, while we follow similar configurations of Seq2Seq and Envdrop to the original papers. All three agents are trained with the CLIP ViT-B-16 feature. The results are shown in Table \ref{tab:scalevln+x_results}. Compared to using only PREVALENT~\cite{hao2020prevalent} for augmentation, All three models significantly benefit from incorporating the ScaleVLN dataset, with 12.2\%, 3.8\%, 5.5\% 
absolute increase in SR for Seq2Seq, EnvDrop, and HAMT, respectively. This shows that
ScaleVLN strengthens models’ generalization ability.
Note that Seq2Seq and Envdrop perform better on Val-Seen when using PREVALENT, mainly caused by overfitting the training environments.

\begin{table}[h]
  \begin{center}
  \resizebox{\columnwidth}{!}{
  \begin{tabular}{l|c|c|rrr|rrr}
    \hline \hline
     \multicolumn{1}{c|}{\multirow{2}{*}{Model}} &\multicolumn{1}{c|}{\multirow{2}{*}{Pre-Train Data}} & \multicolumn{1}{c|}{\multirow{2}{*}{\makecell{Fine-Tune Data}}} & \multicolumn{3}{c|}{R2R Val-Seen} &\multicolumn{3}{c}{R2R Val-Unseen} \\
     \cline{4-9} & & & \multicolumn{1}{c}{NE$\downarrow$} & \multicolumn{1}{c}{SR$\uparrow$} &
    \multicolumn{1}{c|}{SPL$\uparrow$} & \multicolumn{1}{c}{NE$\downarrow$} & \multicolumn{1}{c}{SR$\uparrow$} &
    \multicolumn{1}{c}{SPL$\uparrow$} \Tstrut\\
    \hline \hline
         \multicolumn{1}{l|}{\multirow{2}{*}{Seq2Seq~\cite{anderson2018r2r}}}   &   - &     R2R, PREV    & \textbf{3.89} & \textbf{58.18} & \textbf{38.49} & 6.32 & 37.34 & 23.21 \\
    & - &     R2R, ScaleVLN  & 4.78 & 49.85 & 36.32  & \textbf{5.20} & \textbf{47.51} & \textbf{34.81} \\
         \hline
         \multicolumn{1}{l|}{\multirow{2}{*}{Envdrop~\cite{tan2019envdrop}}}   &    - &     R2R, PREV   & \textbf{3.65} & \textbf{66.12} & \textbf{61.72} & 4.41 & 59.22 & 52.35 \\
    & - &     R2R, ScaleVLN & 3.70 & 65.23 & 59.06 & \textbf{3.99} & \textbf{63.01} & \textbf{54.93} \\
    \hline
    \multicolumn{1}{l|}{\multirow{3}{*}{HAMT~\cite{chen2021hamt}}}   &    R2R, PREV &     R2R, PREV   & 2.58 & 74.93 & 71.52 & 3.69 & 64.90 & 60.11\\
    & R2R, PREV, ScaleVLN &     R2R & \textbf{2.15} & \textbf{79.53} & \textbf{76.64} & 3.43 & 67.56 & 62.32\\
     & R2R, PREV, ScaleVLN & R2R, ScaleVLN &  2.43 & 76.40 & 73.30 
     & \textbf{3.07} & \textbf{70.46} & \textbf{65.12} \\
    \hline \hline
  \end{tabular}}
\end{center}
\vspace{-5pt}
\caption{Influence of ScaleVLN on different VLN models.}
\label{tab:scalevln+x_results}
\end{table}

\subsection{ScaleVLN for Long-Horizon VLN (\texorpdfstring{$\boldsymbol{\S}$}~4.2)} 

We evaluate the impact of our dataset on a long-horizon VLN dataset, R4R~\cite{jain2019stay}. R4R extends the R2R dataset by concatenating two adjacent trajectories in R2R, resulting in longer navigation trajectories not biased by the shortest path prior. We directly fine-tune our pre-trained HAMT models from Table \ref{tab:scalevln+x_results} on R4R. Compared to pre-training with only R2R and PREVALENT, adding our \ours{} dataset in the pre-training stage leads to a  consistent gain, yielding +2.7\% SR, +1.5\% nDTW and +2.7\% SDTW~\cite{ilharco2019ndtw}. As suggested by the large improvement in nDTW between the ground-truth path and the executed path, our \ours{} data not only facilitate the model to reach the target but also follow the path described by the given instruction.

\begin{table}[h]
  \begin{center}
  \resizebox{\columnwidth}{!}{
  \begin{tabular}{c|c|rrrrr}
    \hline \hline
     \multicolumn{1}{c|}{\multirow{2}{*}{Pre-Train Data}} & \multicolumn{1}{c|}{\multirow{2}{*}{\makecell{Fine-Tune Data}}} &\multicolumn{5}{c}{R4R Val-Unseen} \\
     \cline{3-7}  & & \multicolumn{1}{c}{NE$\downarrow$} & \multicolumn{1}{c}{SR$\uparrow$} &
    \multicolumn{1}{c}{CLS$\downarrow$} & \multicolumn{1}{c}{NDTW$\uparrow$} &
    \multicolumn{1}{c}{SDTW$\uparrow$} \Tstrut\\
    \hline \hline
          R2R, PREV &     R4R   & 6.19 & 41.52 & 57.89 & 51.21 & 30.00  \\
    R2R, PREV, ScaleVLN &     R4R  & \textbf{6.09} & \textbf{44.20} & \textbf{59.55} & \textbf{52.77} & \textbf{32.73}  \\ 
    \hline \hline
  \end{tabular}
  }
\end{center}
\vspace{-5pt}
\caption{Effect of \ours{} on learning R4R.}
\label{tab:hamt_results}
\end{table}


\subsection{Leaderboard Results of R2R (\texorpdfstring{$\boldsymbol{\S}$}~4.4)}

We report the top seven submissions on the test-unseen leaderboard of R2R\footnote{R2R test server: \url{https://eval.ai/web/challenges/challenge-page/97/leaderboard/270}.} (Table~\ref{tab:leaderboard}). When ranking with success rate, we can see that (a) most methods have extremely low SPL (1\%) due to using beam search to find the optimal paths. Even so, our single-run result (\textit{EarlyToBed}) outperforms them by a large margin. When ranking with SPL (b), some methods pre-explored the test environments but their results are still much worse than ours. Apart from human followers, we are currently ranked first on the leaderboard.

\begin{table}[h]
\centering
% \def\arraystretch{1.2}
\begin{minipage}{0.49\linewidth}
    \centering
    % \setlength\tabcolsep{4.0pt}
    \resizebox{\textwidth}{!}{
        \begin{tabular}{lrrr}
         \hline \hline
        \multicolumn{1}{c|}{Team} & \multicolumn{1}{c}{NE$\downarrow$} & \multicolumn{1}{c}{SR$\uparrow$} & \multicolumn{1}{c}{SPL$\uparrow$} \\
            \hline \hline
            \multicolumn{1}{l|}{human} & 1.61 & 86 & 76 \\
            \hline
            \multicolumn{1}{l|}{\textbf{EarlyToBed} (ours)} & 2.27 & 80 & 70 \\
            \multicolumn{1}{l|}{LILY$^{\circ}$} & 2.54 & 78 & 1 \\
            \multicolumn{1}{l|}{Airbert$^{\circ}$} & 2.50 & 78 & 1 \\
            \multicolumn{1}{l|}{Shortest-Path-Prior$^{\circ}$} & 3.55 & 74 & 1 \\
            \multicolumn{1}{l|}{UU\_77} & 3.00 & 74 & 63 \\            
            \multicolumn{1}{l|}{TAIIC$^{\circ}$} & 2.99 & 74 & 1 \\
            \hline \hline
        \end{tabular}   
    }
    \small (a) Top 7 in SR.
\end{minipage}
%
\begin{minipage}{0.49\linewidth}
    \centering
    % \setlength\tabcolsep{4.0pt}
    \resizebox{\textwidth}{!}{
        \begin{tabular}{lrrr}
         \hline \hline
        \multicolumn{1}{c|}{Team} & \multicolumn{1}{c}{NE$\downarrow$} & \multicolumn{1}{c}{SR$\uparrow$} & \multicolumn{1}{c}{SPL$\uparrow$} \\
            \hline \hline
            \multicolumn{1}{l|}{human} & 1.61 & 86 & 76 \\
            \hline
            \multicolumn{1}{l|}{\textbf{EarlyToBed} (ours)} & 2.27 & 80 & 70 \\
            \multicolumn{1}{l|}{TAIICX$\dag$} & 3.00 & 73 & 69 \\
            \multicolumn{1}{l|}{Active Exploration$\dag$} & 3.30 & 70 & 68 \\
            \multicolumn{1}{l|}{sponge} & 3.26 & 71 & 67 \\ 
            \multicolumn{1}{l|}{Auxiliary Reasoning$\dag$} & 3.96 & 68 & 65 \\
            \multicolumn{1}{l|}{SE-Mixed} & 3.52 & 70 & 65 \\       
            \hline \hline
        \end{tabular}
    }
    \small (b) Top 7 in SPL.
\end{minipage}
\vspace{-0.2cm}
\caption{R2R leaderboard results (28.JUL.2023). $^{\circ}$: Beam search. $\dag$: Pre-exploration.}
\label{tab:leaderboard}
\end{table}


\subsection{REVERIE Object Grounding Result (\texorpdfstring{$\boldsymbol{\S}$}~4.4)}

We report the success rate of remote object grounding (RGS) and its path length-weighted result (RGSPL). As shown in Table~\ref{tab:reverie_obj_grounding}, \ours{} achieves state-of-the-art performance on object grounding task on the test leaderboard, comparable to the previous best method AutoVLN~\cite{chen2022hm3dlearning}.

\begin{table}[h]
  \begin{center}
  \resizebox{\columnwidth}{!}{
  \begin{tabular}{l|rrrr|rrrr}
    \hline \hline
     \multicolumn{1}{c|}{\multirow{2}{*}{Models}} & \multicolumn{4}{c|}{REVERIE Val-Unseen} &\multicolumn{4}{c}{REVERIE Test-Unseen} \\
     \cline{2-9} & \multicolumn{1}{c}{SR$\uparrow$}& \multicolumn{1}{c}{SPL$\uparrow$} & \multicolumn{1}{c}{RGS$\uparrow$} &
    \multicolumn{1}{c|}{RGSPL$\uparrow$} & \multicolumn{1}{c}{SR$\uparrow$}& \multicolumn{1}{c}{SPL$\uparrow$} & \multicolumn{1}{c}{RGS$\uparrow$} &
    \multicolumn{1}{c}{RGSPL$\uparrow$} \Tstrut\\
    \hline \hline
     % \\
    % \hline
    SIA~\cite{lin2021sia} & 31.53 & 16.28 & 22.41 & 11.56 & 30.80 & 14.85 & 19.02 & 9.20 \\ 
    HAMT~\cite{chen2021hamt} & 32.95 & 30.20 & 18.92 & 17.28 & 30.40 & 26.67 & 14.88 & 13.08 \\ 
    DUET~\cite{chen2022duet} & 46.98 & 33.73 & 32.15 & 23.03 & 52.51 & 36.06 & 31.88 & 22.06 \\ 
    AutoVLN~\cite{chen2022hm3dlearning} & 55.89 & 40.85 & \textbf{36.58} & \textbf{26.76} & 55.17 & 38.88 & 32.23 & 22.68 \\ 
    \hline
    DUET+\ours{}(ours) & \textbf{56.97} & \textbf{41.84} & 35.76 & 26.05 & \textbf{56.13} & \textbf{39.52} & \textbf{32.53} & \textbf{22.78} \\ 
    \hline \hline
  \end{tabular}}
\end{center}
\vspace{-5pt}
\caption{Object grounding performance on REVERIE.}
\label{tab:reverie_obj_grounding}
\end{table}


\section{Comparison of Navigability Graphs (\S3.2)}
\label{sec_3}

We visualize the navigability graphs produced by AutoVLN~\cite{chen2022hm3dlearning} and our method for several HM3D environments in Figure~\ref{fig:graph_com}. We can see that our graphs are denser, covering more regions, have viewpoints away from obstacles, and are fully traversable in open space.



\section{Recover High Quality Images (\S3.2)}
\label{sec_4}


As introduced in Main Paper \S3.2, we apply the Co-Modulated GAN~\cite{zhao2021comodgan} to recover the corrupted images rendered from the HM3D and Gibson environments. Specifically, we first render a panorama of shape 512$\times$1024 from the 3D mesh at each viewpoint. Then, we crop four images of shape 512$\times$512 centered at 0$^{\circ}$, 90$^{\circ}$, 180$^{\circ}$ and 270$^{\circ}$ of the panorama (with overlapping), and recover them separately. Note that, in VLN, the panoramic observation at a viewpoint is represented by 36 single-view images at 12 viewing angles and three elevations~\cite{anderson2018r2r}. We directly extract their corresponding regions from the four recovered images to obtain these single-view images for pre-training an agent.

Table~\ref{tab:recovered_envs} visualizes the difference between the rendered images and our recovered images. 
First, we can see that our method can recover missing regions, including outdoor scenes such as sky and trees (Example 1 \& 4) and indoor scenes such as floor and walls (Example 6).
Besides, the recovered images usually have less blurry or distorted areas, and the object boundaries are much clearer and sharper. For instance, the ceiling light in Example 2, the chairs in Example 3, and the door frames in Example 5.
Even for the highly corrupted images from Gibson (Examples 4--6), we can see that the method can still recover the scene to a reasonable quality.



% Figure environment removed

\begin{table*}[t]
    \centering
    \resizebox{\textwidth}{!}{
    \begin{tabular}{cccc}
\hline \hline
Examples & 
Environments & 
\multicolumn{1}{c}{Rendered} &
\multicolumn{1}{c}{Recovered} \\ \hline \hline
 1 & HM3D &  \begin{minipage}{0.8\columnwidth}
   % Figure removed 
    \end{minipage} & \begin{minipage}{0.8\columnwidth}
      % Figure removed 
    \end{minipage}  \\
    \hline
2 &    HM3D &  \begin{minipage}{0.8\columnwidth}
   % Figure removed 
    \end{minipage} & \begin{minipage}{0.8\columnwidth}
      % Figure removed 
    \end{minipage}  \\
    \hline
 3 &   HM3D &  \begin{minipage}{0.8\columnwidth}
   % Figure removed 
    \end{minipage} & \begin{minipage}{0.8\columnwidth}
      % Figure removed 
    \end{minipage}  \\
    \hline
    4 &   Gibson &  \begin{minipage}{0.8\columnwidth}
   % Figure removed 
    \end{minipage} & \begin{minipage}{0.8\columnwidth}
      % Figure removed 
    \end{minipage}  \\
    \hline
    5 &   Gibson &  \begin{minipage}{0.8\columnwidth}
   % Figure removed 
    \end{minipage} & \begin{minipage}{0.8\columnwidth}
      % Figure removed 
    \end{minipage}  \\
    \hline
    6 &   Gibson &  \begin{minipage}{0.8\columnwidth}
   % Figure removed 
    \end{minipage} & \begin{minipage}{0.8\columnwidth}
      % Figure removed 
    \end{minipage}  \\
    \hline \hline
    \end{tabular}
    }
    \caption{Qualitative examples of our recovered images from HM3D and Gibson environments. The vertical line at the middle of panorama is caused by directly sticking two independently recovered images at 0$^{\circ}$ and 180$^{\circ}$, which will not appear in the resulting augmented data, as explained in Appendix \S\ref{sec_4}.}
    \label{tab:recovered_envs}
\end{table*}

\end{document}