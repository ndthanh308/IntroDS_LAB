% !TEX TS-program = pdflatex
% !TEX encoding = UTF-8 Unicode

% This is a simple template for a LaTeX document using the "article" class.
% See "book", "report", "letter" for other types of document.

\documentclass[11pt]{article} % use larger type; default would be 10pt

\usepackage[utf8]{inputenc} % set input encoding (not needed with XeLaTeX)

\usepackage{cite}

%%% Examples of Article customizations
% These packages are optional, depending whether you want the features they provide.
% See the LaTeX Companion or other references for full information.

\usepackage{hyperref}
\hypersetup{colorlinks=false}

%%% PAGE DIMENSIONS
\usepackage{geometry,amsmath} % to change the page dimensions
\geometry{a4paper} % or letterpaper (US) or a5paper or....
 \geometry{margin=0.8in} % for example, change the margins to 2 inches all round
% \geometry{landscape} % set up the page for landscape
%   read geometry.pdf for detailed page layout information

\usepackage{graphicx} % support the \includegraphics command and options

\usepackage{lipsum}

\numberwithin{equation}{section}

%\usepackage{scrextend,rotating}
%\makeatletter
%\newenvironment{rotatepage}
%        {%
%            \if@twoside%
%                \ifthispageodd{\pagebreak[4]\global\pdfpageattr\expandafter{\the\pdfpageattr/Rotate 90}}{%
%                \pagebreak[4]\global\pdfpageattr\expandafter{\the\pdfpageattr/Rotate 270}}%
%            \else%
%                \pagebreak[4]\global\pdfpageattr\expandafter{\the\pdfpageattr/Rotate 90}%
%            \fi%
%        }%
%        {\pagebreak[4]\global\pdfpageattr\expandafter{\the\pdfpageattr/Rotate 0}}%
%
%\makeatother


% \usepackage[parfill]{parskip} % Activate to begin paragraphs with an empty line rather than an indent

%%% PACKAGES
\usepackage{booktabs} % for much better looking tables
\usepackage{array} % for better arrays (eg matrices) in maths
\usepackage{paralist} % very flexible & customisable lists (eg. enumerate/itemize, etc.)
\usepackage{verbatim} % adds environment for commenting out blocks of text & for better verbatim
\usepackage{subfig} % make it possible to include more than one captioned figure/table in a single float
% These packages are all incorporated in the memoir class to one degree or another...

\usepackage{float}

\usepackage{todonotes}

\usepackage{amssymb}
\usepackage{amsmath}
\usepackage{amsthm}
\usepackage{bm}
\usepackage{mathrsfs}


\usepackage{xcolor,pict2e}
%\usepackage[notcite,notref]{showkeys}

\newcommand{\intpsym}{\,\, \setlength{\unitlength}{0.4mm}
	\begin{picture}(0, 0)(5, 5)%
	\put(0, 4){\line(1,0){5}} \put(5,4){\line(0,1){8}}
	\end{picture}\,\,}
\newcommand{\intp}{\,\,\intpsym}
\newcommand{\es}{\operatorname{S}}
\newcommand{\esJ}{\es_\mathbf{J}}
\newcommand{\esmJ}{\es_{-\mathbf{J}}}
\newcommand{\esK}{\es_\mathbf{K}}
\newcommand{\te}{\operatorname{T}}
\newcommand{\teJ}{\te_\mathbf{J}}
\newcommand{\teK}{\te_\mathbf{K}}
\newcommand{\id}{\operatorname{id}}
\newcommand{\prol}{\operatorname{pr}}
\newcommand{\volf}{\operatorname{vol}}
\newcommand{\upd}{\operatorname{d}}
\newcommand{\uph}{\operatorname{h}}
\newcommand{\upv}{\operatorname{v}}
\newcommand{\p}{\partial}
\newcommand{\dv}{\upd_{\upv}}
\newcommand{\diffi}{\mathrm{D}_{n^i}}
\newcommand{\diffn}{\mathrm{D}_{n}}
\newcommand{\pde}{P$\Delta$E}
\newcommand{\mbf}{\mathbf{f}}
\newcommand{\mbn}{\mathbf{n}}
\newcommand{\mbu}{\mathbf{u}}
\newcommand{\mbv}{\mathbf{v}}
\newcommand{\mbx}{\mathbf{x}}
\newcommand{\mbI}{\mathbf{I}}
\newcommand{\mbJ}{\mathbf{J}}
\newcommand{\mbK}{\mathbf{K}}
\newcommand{\mcE}{\mathcal{E}}
\newcommand{\mcI}{\mathcal{I}}
\newcommand{\mcH}{\mathcal{H}}
\newcommand{\mcL}{\mathcal{L}}
\newcommand{\msL}{\mathscr{L}}
\newcommand{\bD}{\bm{\Delta}}
\newcommand{\dD}{\upd^{\vartriangle}}
\newcommand{\dDh}{\upd^{\vartriangle}_{\uph}}


\theoremstyle{plain}
\newtheorem{thm}{Theorem}[section]
\newtheorem{cor}[thm]{Corollary}
\newtheorem{lem}[thm]{Lemma}
\newtheorem{prop}[thm]{Proposition}

\theoremstyle{definition}
\newtheorem{defn}{Definition}[section]
\newtheorem{exm}[defn]{Example}
\newtheorem{rem}[defn]{Remark}

%\theoremstyle{remark}
%\newtheorem*{rem}{Remark}
%\newtheorem*{note}{Note}
%\newtheorem{case}{Case}



%%% HEADERS & FOOTERS
\usepackage{fancyhdr} % This should be set AFTER setting up the page geometry
\pagestyle{fancy} % options: empty , plain , fancy
\renewcommand{\headrulewidth}{0pt} % customise the layout...
\lhead{}\chead{}\rhead{}
\lfoot{}\cfoot{\thepage}\rfoot{}

%%% SECTION TITLE APPEARANCE
\usepackage{sectsty}
\allsectionsfont{\sffamily\mdseries\upshape} % (See the fntguide.pdf for font help)
% (This matches ConTeXt defaults)

%%% ToC (table of contents) APPEARANCE
\usepackage[nottoc,notlof,notlot]{tocbibind} % Put the bibliography in the ToC
\usepackage[titles,subfigure]{tocloft} % Alter the style of the Table of Contents
\renewcommand{\cftsecfont}{\rmfamily\mdseries\upshape}
\renewcommand{\cftsecpagefont}{\rmfamily\mdseries\upshape} % No bold!


\title{The difference variational bicomplex and multisymplectic systems}
\author{Linyu Peng{$^{1}$\footnote{Corresponding author. Email: L.Peng@mech.keio.ac.jp} ~ and Peter E. Hydon{$^2$}\footnote{Email: P.E.Hydon@kent.ac.uk} }\vspace{0.4cm}
\\
{\it 1. Department of Mechanical Engineering, Keio University,} \\
{\it Yokohama 223-8522, Japan}\\
{\it 2. School of Mathematics, Statistics and Actuarial Science,} \\
{\it University of Kent,  Canterbury CT2 7FS, UK}\\ }




%\date{}

\begin{document}

\maketitle

\begin{abstract}
The difference variational bicomplex, which is the natural setting for systems of difference equations, is constructed and used to examine the geometric and algebraic properties of various systems. Exactness of the bicomplex gives a coordinate-free setting for finite difference variational problems, Euler--Lagrange equations and Noether's theorem. We also examine the connection between the condition for existence of a Hamiltonian and the multisymplecticity of systems of partial difference equations. Furthermore, we define difference multimomentum maps of multisymplectic systems, which yield their conservation laws. To conclude, we demonstrate  how multisymplectic integrators can be comprehended even on non-uniform meshes through a generalized difference variational bicomplex.\vspace{0.2cm}

{Keywords: difference variational bicomplex; multisymplectic
system; multisymplectic integrator; conservation law; multimomentum map}

\end{abstract}



%%%%%%%%%%%%%%%%%%%%%%%%%%%%%%%%%%%%%%%%%%%%%%%%%%%%%%%%%%%%%%%%%%%%%%%%%%%%%%%%%%%%%%%%%%%%%
%%%%%%%%%%%%%%%%%%%%%%%%%%%%%%%%%%%%%%%%%%%%%%%%%%%%%%%%%%%%%%%%%%%%%%%%%%%%%%%%%%%%%%%%%%%%%
\section{Introduction}
Symmetry methods provide powerful tools for obtaining solutions and conservation laws of a given system of partial differential equations (PDEs) and for understanding structural features such as integrability \cite{BCA2010,Hy2000b,Ol1993}. In the formal geometric approach, the variational complex is central to the study
of symmetries, scalar conservation laws and Euler--Lagrange equations \cite{KO2003,Ol1993}. 
This complex is contained in the (augmented) variational bicomplex, which is a natural geometric setting for all of the above and also for multisymplectic PDEs \cite{BrHyLa2010} and other PDEs with form-valued conservation laws, as well as the Lagrangian multiform (or pluri-Lagrangian) formalism of integrable systems (see, e.g., \cite{LN2009,SV2016,SNC2020}).

The variational bicomplex is constructed by splitting the exterior
derivative into horizontal and vertical parts (see \cite{Vi2001,KrVi1999,Vi1984,An1992,An1989}), which reflect the distinction between independent and dependent variables. It is augmented by a projection, the interior Euler operator, which is used to derive Euler--Lagrange equations from a given Lagrangian form. Independently, Anderson \cite{An1992,An1989}, Tsujishita \cite{Ts1982} and Vinogradov \cite{Vi2001,Vi1984} proved that the augmented variational bicomplex is exact.

Over the last three decades, symmetry analysis for differential equations has been extended to difference equations (see \cite{Do2001,Do2010,Hy2005,RaHy2007,PH2022,Xe2018,Hy2014,LeTrWi2000,Pe2013,Pe2017}).
Difference forms \cite{MaHy2008} and the difference variational complex
\cite{HyMa2004} have also been developed. These results are
fundamentally important for the geometric analysis of finite difference numerical schemes. For instance, a variational integrator yields Euler--Lagrange equations from a difference Lagrangian form.

This paper formally introduces the difference variational bicomplex, that was proposed in the thesis \cite{Pe2013},  and examines some of its applications. Section \ref{sec:vb} begins with a brief review of the differential variational bicomplex, which is a natural setting for multisymplectic systems of PDEs. Section \ref{sec:dvb} develops the main ideas and structures for the difference variational bicomplex. Exactness of the difference variational bicomplex plays an essential role in applications, giving coordinate-free versions of Noether's theorem  for finite difference variational problems and multisymplectic partial difference equations (P$\Delta$Es). (A proof of exactness is given in an Appendix.) Section \ref{sec:hasy} uses the bicomplex to develop a coordinate-free difference version of Hamilton's
principle. In Section \ref{sec:mupde}, the
conservation of multisymplectic structures is studied for P$\Delta$Es and we explain the link between multisymplectic systems and degenerate difference Lagrangian structures. By defining discrete
multimomentum maps, conservation laws of multisymplectic systems are
obtained. In Section \ref{sec:multisyin}, we describe a generalized difference variational bicomplex that is the natural setting for multisymplectic integrators on a non-uniform mesh.
%%%%%%%%%%%%%%%%%%%%%%%%%%%%%%%%%%%%%%%%%%%%%%%%%%%%%%%%%%%%%%%%%%%%%%%%%%%%%%%%%%%%%%%%%%


\section{The (differential) variational bicomplex}
\label{sec:vb}
The variational bicomplex is a double complex of differential forms that arises by regarding independent variables as coordinates on a base space, with dependent variables and their derivatives coordinatizing fibres over each point of the base space. The geometric setting is the infinite prolongation bundle. We review the differential variational bicomplex (following Anderson's presentation \cite{An1992,An1989}, see also \cite{Ts1982,KrVi1999,Vi1984}), and show that this is a natural setting for multisymplectic systems of PDEs. 

\subsection{An overview of the variational bicomplex}
For differential equations with independent variables $\mbx=(x^1,x^2,\ldots,x^p)\in X\subset \mathbb{R}^p$ and dependent variables $\mbu=(u^1,u^2,\ldots, u^q)\in U\subset \mathbb{R}^q$, a natural geometric structure is the trivial fibred manifold 
\begin{equation}
\begin{aligned}
\pi: X\times U&\rightarrow X,\\
(\mbx,\mbu)&\mapsto \mbx.
\end{aligned}
\end{equation}
A solution $\mbu=f(\mbx)$ can be regarded as a local section, $s(\mbx)=(\mbx,f(\mbx))$. Restricting attention to neighbourhoods in which $f$ is smooth, a section can be prolonged to the infinite jet bundle $J^{\infty}(X\times U)$ whose coordinates represent derivatives:
\begin{equation}
\begin{aligned}
\pi^{\infty}: J^{\infty}(X\times U)&\rightarrow X,\\
(x^i,u^{\alpha},u_{\mathbf{1}_i}^{\alpha},\ldots,u^{\alpha}_{\mbJ},\ldots)&\mapsto \mbx.
\end{aligned}
\end{equation}
The prolonged section $s$ has coordinates 
\begin{equation}
x^i,\quad u^\alpha=f^\alpha(\mbx),\quad u_{\mathbf{1}_i}^{\alpha}=\frac{\partial f^{\alpha}(\mbx)}{\partial x^i}\,,\quad \ldots,\quad u^{\alpha}_{\mbJ}=\frac{\partial^{|\mbJ|}f^{\alpha}(\mbx)}{(\partial x^1)^{j^1}(\partial x^2)^{j^2}\cdots  (\partial x^p)^{j^p}}\,, \quad \ldots,   
\end{equation}
where $\mathbf{1}_i$ is the $p$-tuple whose only nonzero entry is $1$ in its $i$th place, $\mbJ=(j^1,j^2,\ldots,j^p)$ with all entries being non-negative integers, and $|\mbJ|=j^1+j^2+\cdots+j^p$. In this setting, a differential equation defines a variety on the jet bundle.

The exterior derivative on $J^{\infty}(X\times U)$ can be written in terms of these local coordinates. Let $[\mbu]$ denote $\mbu$ and finitely many of its partial derivatives; then the exterior derivative of a locally smooth function $f(\mbx,[\mbu])$ is
\begin{equation}\label{eq:exde}
\upd\!f(\mbx,[\mbu])=\frac{\partial f(\mbx,[\mbu])}{\partial x^i}\upd\!x^i+\frac{\partial f(\mbx,[\mbu])}{\partial u^{\alpha}_{\mbJ}}\upd\! u^{\alpha}_{\mbJ}\,.
\end{equation}
The Einstein summation convention is used from \eqref{eq:exde} on. It is natural to split the exterior derivative using the contact forms $\upd\!u^{\alpha}_{\mbJ}-u^{\alpha}_{\mbJ+\mathbf{1}_i}\upd\!x^i$, because the pullback of each contact form  by any section $s$ is zero. This splitting gives
\begin{equation}\label{eq:mexde}
\upd\!f(\mbx,[\mbu])=\left(D_if(\mbx,[\mbu])\right)\upd\!x^i+\frac{\partial f(\mbx,[\mbu])}{\partial u^{\alpha}_{\mbJ}}\left(\upd\!u^{\alpha}_{\mbJ}-u^{\alpha}_{\mbJ+\mathbf{1}_i}\upd\!x^i\right),
\end{equation}
where
\begin{equation}
D_i=\frac{\partial}{\partial x^i}+u^{\alpha}_{\mathbf{1}_i}\frac{\partial}{\partial u^{\alpha}}+\cdots+u_{\mbJ+\mathbf{1}_i}^{\alpha}\frac{\partial}{\partial u_{\mbJ}^{\alpha}}+\cdots
\end{equation}
is the total derivative with respect to $x^i$.  This splitting gives a basis for the set of differential one-forms: 
\begin{equation}\label{eq:dxdu}
\upd\!x^i,\qquad \upd\!u^{\alpha}_{\mbJ}-u^{\alpha}_{\mbJ+\mathbf{1}_i}\upd\!x^i,
\end{equation}
which is extended to a basis for the set $\Omega$ of all differential forms by using the wedge product. 
The exterior derivative splits into the horizontal derivative $\upd_{\uph}$ and the vertical derivative $\dv $, as follows:
\begin{equation}
\upd=\upd_{\uph}+\dv ,
\end{equation}
where
\begin{equation}\label{eq:dhdiff}
\upd_{\uph}=\upd\!x^i\wedge D_i\,,\quad \dv =\left(\upd\!u^{\alpha}_{\mbJ}-u^{\alpha}_{\mbJ+\mathbf{1}_i}\!\upd\!x^i\right)\wedge\frac{\partial}{\partial u^{\alpha}_{\mbJ}}\,.
\end{equation}
Direct calculation yields the identities 
\begin{equation}\label{eq:diffdhdv}
\upd_{\uph}^2=0,\qquad \upd_{\uph}\!\dv =-\dv \!\upd_{\uph},\qquad \dv ^2=0.
\end{equation}
The contact forms $\dv\!u^{\alpha}_{\mbJ}=\upd\!u^{\alpha}_{\mbJ}-u^{\alpha}_{\mbJ+\mathbf{1}_i}\!\upd\!x^i$ form a basis for the set of vertical differential one-forms; the set of horizontal differential one-forms has a basis $\upd_{\uph}\!x^i=\upd\!x^i$. Furthermore,
\begin{equation}\label{vec1}
D_i\intp \upd_{\uph}\!x^j=\delta^j_i,\qquad D_i\intp \dv \!u^\beta_\mbK=0,\qquad \frac{\partial}{\partial u^{\alpha}_{\mbJ}}\intp \upd_{\uph}\!x^j=0,\qquad \frac{\partial}{\partial u^{\alpha}_{\mbJ}}\intp \dv \!u^\beta_\mbK=\delta^\beta_\alpha\delta^\mbJ_\mbK.
\end{equation}
A locally smooth vertical vector field $\mbv_0=Q^\alpha\,\p/\p u^\alpha$ on $X\times U$ can be prolonged to all orders to yield the vector field $\mbv=D_{\mbJ}Q^\alpha\,\p/\p u^\alpha_{\mbJ}$ on $J^{\infty}(X\times U)$. This can be generalized, with the same prolongation formula, by allowing each $Q^\alpha$ to depend on finitely many derivatives of $\mbu$, in which case $\mbv$ is a vertical generalized vector field on $J^{\infty}(X\times U)$. By the prolongation formula, $\mbv$ commutes with each $D_i$. 

A $(k+l)$-form  $\sigma$ on $J^{\infty}(X\times U)$ is said to be a $(k,l)$-\textit{form} if it can be written as 
\begin{equation}
\sigma=f_{i_1,\ldots,i_k;\alpha_1,\ldots,\alpha_l}^{\mbJ_1,\ldots,\mbJ_l}(\mbx,[\mbu])\upd_{\uph}\!x^{i_1}\wedge \cdots \wedge \upd_{\uph}\!x^{i_k}\wedge \dv \! u_{\mbJ_1}^{\alpha_1}\wedge\cdots\wedge \dv \!u_{\mbJ_l}^{\alpha_l},
\end{equation}
where each $f_{i_1,\ldots,i_k;\alpha_1,\ldots,\alpha_l}^{\mbJ_1,\ldots,\mbJ_l}$ is a locally smooth function.
The Lie derivative of $\sigma$ with respect to a vector field $\mbv$, denoted $\mcL_\mbv \sigma$, may be obtained from Cartan's formula:
\begin{equation}
\mcL_\mbv \sigma=\mbv\intp\upd\;\!\!\sigma+\upd(\mbv\intp\sigma).
\end{equation}
Routine calculations, adapted to the horizontal and vertical derivatives, give the following results.
\begin{lem}
	Let $\sigma$ be a differential form on $J^{\infty}(X\times U)$. If $\mbv$ is a vertical generalized vector field on $J^{\infty}(X\times U)$ then
	\[
	\mbv\intp\upd_{\uph}\;\!\!\sigma+\upd_{\uph}(\mbv\intp\sigma)=0,
	\]
	so
	\[
	\mcL_\mbv\sigma=\mbv\intp\dv \;\!\!\sigma+\dv (\mbv\intp\sigma).
	\]
	Similarly,
\[
D_i\intp\dv \;\!\!\sigma+\dv (D_i\intp\sigma)=0,
\]
so
\[
\mcL_{D_i} \sigma=D_i\intp\upd_{\uph}\;\!\!\sigma+\upd_{\uph}(D_i\intp\sigma).
\]	
\end{lem}

Let $\Omega^{k,l}$ be the set of all $(k,l)$-forms over $J^{\infty}(X\times U)$. Then
\begin{equation}
\upd_{\uph}:\Omega^{k,l}\rightarrow \Omega^{k+1,l},\quad \dv :\Omega^{k,l}\rightarrow \Omega^{k,l+1},
\end{equation}
and the identities \eqref{eq:diffdhdv} yield a double complex called the variational bicomplex (Fig. \ref{fig:vb}). A $(k,l)$-form $\sigma$ is horizontally closed if
$\upd_{\uph}\!\sigma=0$ and
horizontally exact if there exists a form $\tau\in\Omega^{k-1,l}$
such that
$\sigma=\upd_{\uph}\!\tau$.
Similarly, $\sigma$ is vertically closed if
$\dv \!\sigma=0$ and
vertically exact if there exists $\tau\in\Omega^{k,l-1}$
such that
$\sigma=\dv \!\tau$.

% Figure environment removed

The cohomology of the variational bicomplex in Fig. \ref{fig:vb} has been well-studied; for proofs, see  \cite{An1992,Ts1982,KrVi1999,Vi1984}.
Each column of the bicomplex is the analogue of the de Rham complex
for a topologically trivial space, so that any vertically
closed form $\sigma$ is also vertically exact. However, this is not true for the rows, where the Poincar\'{e} Lemma fails. Specifically, for any $l\geq1$, there
exist horizontally closed $(p,l)$-forms that are not horizontally
exact. To overcome this inconvenience, a projection $\mcI$ on
$\Omega^{p,l},~l\geq1$ is used to make the rows exact, yielding the augmented variational bicomplex in Fig. \ref{fig:mvb}; here $\mathscr{F}^l:=\mcI\left(\Omega^{p,l}\right)$. The cohomology groups of the augmented  variational bicomplex are all trivial, reflecting the topological triviality of $J^\infty(X\times U)$. If one pulls back the complex to the submanifold of (infinitely prolonged) solutions of a given system of PDEs, the vertical cohomology groups remain trivial, but some horizontal cohomology groups may be nontrivial.

% Figure environment removed

To define $\mcI$, which is called the \textit{interior Euler operator}, it is helpful to use the multi-index notation 
\begin{equation}
D_{\mbJ}=D_1^{j^1}D_2^{j^2}\cdots D_p^{j^p},\qquad (-D)_{\mbJ}=(-D_1)^{j^1}(-D_2)^{j^2}\cdots (-D_p)^{j^p}=(-1)^{|\mbJ|}D_{\mbJ}\,;
\end{equation}
formally, $(-D)_{\mbJ}$ is the adjoint operator to $D_{\mbJ}$. Then $\mcI:\Omega^{p,l}\rightarrow \Omega^{p,l}$ is defined by 
\begin{equation}
\mcI(\sigma)=\frac{1}{l}\dv \!u^{\alpha}\wedge (-D)_{\mbJ}\left(\frac{\partial}{\partial u_{\mbJ}^{\alpha}}\intp \sigma\right),\quad \sigma\in\Omega^{p,l},
\end{equation} 
which gives the same outcome (up to a divergence) as integration by parts. The rows containing $\Omega^{k,l}$ (for fixed $l\geq 1$) are exact; in particular, $\text{ker}(\mcI)=\text{im}(\upd_{\uph})$. Moreover, the interior Euler operator is a projection (so $\mcI^2=\mcI$); hence, for each $\sigma\in\Omega^{p,l},~l\geq 1$, there exists $\tau\in\Omega^{p-1,l}$ such that 
\begin{equation}\label{eq:ii}
\sigma=\mcI(\sigma)-\upd_{\uph}\!\tau.
\end{equation}

The \textit{Euler--Lagrange operator} $\mcE:\Omega^{p,0}\rightarrow\mathscr{F}^1$ is defined by $\mcE:=\mcI\dv $. Given a Lagrangian form, $\msL[\mbu]=L(\mbx,[\mbu])\text{vol}$, where $\text{vol}=\upd\!x^1\wedge\cdots\wedge\upd\!x^p$ is the volume form, 
\begin{equation}
\mcE(\msL)=(-D)_{\mbJ}\left(\frac{\partial L(\mbx,[\mbu])}{\partial u^{\alpha}_{\mbJ}}\right)\dv \!u^\alpha\wedge\text{vol},
\end{equation}
so the Euler--Lagrange equations are the coefficients of $\mcE(\msL)=0$.

Bearing in mind that $\mathscr{F}^l\subset \Omega^{p,l}$, the operators $\delta_{\upv}:=\mcI\dv $ give higher variations; in particular, $\delta_{\upv}:\mathscr{F}^1\rightarrow \mathscr{F}^2$ gives the Helmholtz conditions for the inverse problem of variational calculus. 
The variational complex is the edge of the augmented variational bicomplex, consisting of the bottom row, the Euler--Lagrange operator, and the column containing the spaces $\mathscr{F}^l$. The variational complex is exact, so
\[
\text{ker}(\mcE)=\text{im}(\upd_{\uph}),\qquad \text{im}(\mcE)=\text{ker}(\delta_{\upv}),
\]
and the column containing $\mathscr{F}^l$ is exact.

\subsection{Multisymplectic systems via the variational bicomplex}

The (augmented) variational bicomplex provides a natural framework for studying multisymplectic PDEs; this framework was introduced in \cite{BrHyLa2010} for first-order quasilinear systems on Riemannian manifolds, but it applies equally to other types of multisymplectic systems. For simplicity, we restrict attention to the base manifold $\mathbb{R}^p$, but the results are local and can be adapted to other base manifolds.

The starting-point is Zuckerman's discovery in \cite{Zuck} of a `universal conserved current' for any given Lagrangian form $\msL\in \Omega^{p,0}$. This is a vertically closed differential form $\omega\in\Omega^{p-1,2}$ that is conserved on solutions of the Euler--Lagrange equations. It is instructive to revisit Zuckerman's proof using the augmented variational bicomplex. From \eqref{eq:ii}, there exists $\eta\in\Omega^{p-1,1}$ such that
\begin{equation}\label{fund}
\mcE(\msL)=\dv\!\msL+\upd_{\uph}\!\eta.
\end{equation}
The $(p-1,2)$-form $\omega=\dv\!\eta$ satisfies
\begin{equation}\label{dom}
\upd_{\uph}\!\omega=-\dv\!\upd_{\uph}\!\eta=-\dv\!\mcE(\msL).
\end{equation}
Consequently $\upd_{\uph}\!\omega=0$ on the solution submanifold of the Euler--Lagrange equations.

Independently, Gotay \cite{Gotay} developed a covariant Hamiltonian formalism for field theories of arbitrary order, using a generalized Legendre transformation to identify additional phase space variables $p_\alpha^{\mbJ}$ from the Lagrangian. A key ingredient is the Poincar\'{e}--Cartan form $\Theta$, which is a Lepagean equivalent of the Lagrangian form. In the above notation,
\begin{equation}
\Theta=\msL+\eta,
\end{equation}
which is a $p$-form of mixed type (from the viewpoint of the bicomplex). From \eqref{fund}, we have
\begin{equation}
\upd\!\Theta=\mcE(\msL)+\omega,
\end{equation}
and \eqref{dom} arises from $\upd^2\!\Theta=0$.


Up to inessential terms, Gotay's covariant Hamiltonian formalism yields the quasilinear first-order system of Euler--Lagrange equations for the modified Lagrangian
\begin{equation}
\widehat{L}=L(\mbx,[\mbu])+p_\alpha^{\mbJ+\mathbf{1}_i}(D_iu^\alpha_\mbJ-u^\alpha_{\mbJ+\mathbf{1}_i}).
\end{equation}
Here $u^\alpha_\mbJ, p_\alpha^{\mbJ}$ are distinct dependent variables on the phase space, with the latter variables playing the role of Lagrange multipliers. So nothing is lost by restricting attention to first-order quasilinear systems, in which case $[\mbu]$ fully coordinatizes the phase space. This approach was introduced by Bridges \cite{Br1997b}.

One can partially reverse the above derivation, using the fact that the vertical cohomology groups are trivial, even for the restricted bicomplex. A system of PDEs is \textit{multisymplectic} if there exists a vertically closed $(p-1,2)$-form $\omega$ such that $\upd_{\uph}\!\omega$ is zero on the solution submanifold (but not identically zero). So every system of Euler--Lagrange equations is multisymplectic.
In coordinates, write
\begin{equation}\label{omkap}
\omega=\kappa^i\wedge(D_i\intp \text{vol}),
\end{equation}
where each $\kappa^i$ is a vertically closed $(0,2)$-form. Then on solutions of the system, $\upd_{\uph}\!\omega=0$ amounts to the form-valued conservation law
\[
D_i\kappa^i=0.
\]

As $\omega$ is vertically closed, exactness of the vertical columns implies the existence of $\eta\in\Omega^{p-1,1}$ such that $\omega = \dv\!\eta$. Moreover, on the solution submanifold,
\[
\dv\!\upd_{\uph}\!\eta=-\upd_{\uph}\!\dv\!\eta=0,
\]
so there exists $\msL\in\Omega^{p,0}$ (restricted to this submanifold) such that
\[
\upd_{\uph}\!\eta+\dv\! \msL=0.
\]
Khavkine \cite{Khavkine} used this observation to prove the existence of a Lagrangian $(p,0)$-form on the full jet space, whose Euler--Lagrange equations are solved by the given multisymplectic system of PDEs. (However, these equations may be weaker than the given system and so admit other solutions.)

It is convenient to restrict attention to multisymplectic systems that are first-order and quasilinear, so that both $\omega$ and $\eta$ are defined over each point $\mbx$ in terms of the phase space variables and their vertical derivatives. For difference equations, however, this turns out not to be possible, as differences are not defined pointwise, nor are they derivations. Nevertheless, we now show that there is a difference analogue of the variational bicomplex which has very similar features to the differential case, and this gives rise to a standard form for multisymplectic difference equations.

\section{Construction of the difference variational bicomplex}\label{sec:dvb}
The building-blocks for the difference variational bicomplex are difference prolongation spaces \cite{MaRoHyPe2019}, difference forms \cite{HyMa2004,MaHy2008} and the difference variational complex \cite{HyMa2004} over the base space $\mathbb{Z}^p$.

Consider a \pde\ with $p$ independent variables, $n^i\in\mathbb{Z}$, and $q$ dependent variables, $u^\alpha\in\mathbb{R}$. These variables can be regarded as coordinates on the \textit{total space}, $\mathbb{Z}^p\times\mathbb{R}^q$: the discrete base space $\mathbb{Z}^p$ and the connected fibres $\mathbb{R}^q$ are coordinatized respectively by $\mbn=(n^1,n^2,\ldots,n^p)$ and $\mbu=(u^1,u^2,\dots,u^q)$. For simplicity, we shall assume that this coordinate system applies everywhere, though the results below can be adapted if more than one coordinate patch is needed for a particular \pde.
The fibres are mapped to one another by the horizontal translations
\begin{equation}
\begin{aligned}
\teJ:\mathbb{Z}^p\times\mathbb{R}^q&\rightarrow\mathbb{Z}^p\times\mathbb{R}^q\\
(\mbn,\mbu)&\mapsto(\mbn+\mbJ,\mbu).
\end{aligned}
\end{equation}
Note that $\teJ\circ\teK=\te_{\mbJ+\mbK}$ for all $\mbJ,\mbK\in\mathbb{Z}^p$. (See \cite{Hy2014} for other transformations of total space.)

As the total space is disconnected, it is helpful to construct a connected representation of this space over each base point. To do this, each fibre is prolonged to include the values of the coordinates on all other fibres as coordinates in a Cartesian product, using the pullback of each $u^\alpha$ with respect to every $\teJ$. The (connected) total prolongation space over an arbitrary base point, denoted $P(\mathbb{R}^q)$ (or $P_\mbn(\mathbb{R}^q)$ if the base point, $\mbn$, is specified), has coordinates $(u^\alpha_{\mbJ})$, where
\[
u^\alpha_{\mbJ}=\teJ^*\!u^\alpha.
\]
In particular, $u^{\alpha}_{\mathbf{0}}=u^{\alpha}$. The total prolongation space provides a convenient setting for the study of  geometric properties of difference equations.

The composition rule for horizontal translations gives the identities
\[
u^\alpha_{\mbJ+\mbK}=\teK^*\!u^\alpha_{\mbJ}.
\]
More generally, let $f$ be a function on $\mathbb{Z}^p\times P(\mathbb{R}^q)$ and denote its restriction to each total prolongation space $P_{\mbn}(\mathbb{R}^q)$ by $f_\mbn((u^{\alpha}_{\mbJ}))=f(\mbn,(u^{\alpha}_{\mbJ}))$; for simplicity, we assume that every $f_\mbn$ is smooth. Then the pullback of $f_{\mbn+\mbK}((u^{\alpha}_{\mbJ}))$ with respect to $\te_{\mbK}$ is the function $\te_{\mbK}^*\!f_{\mbn+\mbK}$ on $P_{\mbn}(\mathbb{R}^q)$ whose values are $f(\mbn+\mbK,(u^{\alpha}_{\mbJ+\mbK}))$. So the action of each horizontal translation $\te_{\mbK}$ on the space of smooth functions on $P_{\mbn}(\mathbb{R}^q)$ can be represented by the \textit{shift operator}
\begin{equation}
\begin{aligned}
\es_{\mbK}:C^\infty(P_\mbn(\mathbb{R}^q))&\rightarrow C^\infty(P_\mbn(\mathbb{R}^q))\\
f(\mbn,(u^{\alpha}_{\mbJ}))&\mapsto f(\mbn+\mbK,(u^{\alpha}_{\mbJ+\mbK})),
\end{aligned}
\end{equation}
so that $\es_{\mbK}\!f_\mbn=\te_{\mbK}^*\!f_{\mbn+\mbK}$. Similarly, let $\sigma$ be a differential form on $\mathbb{Z}^p\times P(\mathbb{R}^q)$ whose restriction to each $P_{\mbn}(\mathbb{R}^q)$ is $\sigma_\mbn$. Then the action of $\te_{\mbK}$ on $\sigma_\mbn$ is represented by the shift
\begin{equation}
\es_{\mbK}\!\sigma_\mbn=\te_{\mbK}^*\!\sigma_{\mbn+\mbK}.
\end{equation}
By the standard properties of the pullback, $\es_\mbK$ commutes with the wedge product and with the exterior derivative on the fibre $P(\mathbb{R}^q)$, which we denote by $\dv $ (as it acts on dependent variables only):
\begin{equation}\label{sids}
\es_\mbK(\sigma_1\wedge\sigma_2)=(\es_\mbK\!\sigma_1)\wedge(\es_\mbK\!\sigma_2),\qquad\es_\mbK(\dv\!\sigma)=\dv(\es_\mbK\!\sigma).
\end{equation}

The difference structure is a consequence of the ordering of each independent variable. For any multi-index $\mbJ=(j^1,j^2,\dots,j^p)=j^i\mathbf{1}_i$, the corresponding shift operator is
$\es_{\mbJ}=\es_{1}^{j^1}\es_{2}^{j^2}\cdots \es_{p}^{j^p}$, where $\es_i:=\es_{\mathbf{1}_i}$ denotes the forward shift with respect to $n^i$. Then the forward difference in the $n^i$-direction is represented on each $P_{\mbn}(\mathbb{R}^q)$ by the operator
\begin{equation}\label{symdiffi}
\diffi=\es_i-\id,
\end{equation}
where $\id$ is the identity mapping. In \cite{HyMa2004}, Hydon \& Mansfield introduced difference forms on $\mathbb{Z}^p$. These have the same algebraic properties as differential forms on $\mathbb{R}^p$, with the exterior algebra on $p$ symbols, $\Delta^1,\Delta^2,\dots,\Delta^p$, replacing the exterior algebra on $\upd\! x^1,\upd\! x^2,\dots,\upd\! x^p$. The symbols $\Delta^i$ at any two different points are related by (horizontal) translation, so that
\begin{equation}\label{sdel}
\Delta^i\big|_\mbn=\te_{\mbK}^*(\Delta^i\big|_{\mbn+\mbK})=:\es_\mbK(\Delta^i\big|_{\mbn}).
\end{equation}
A difference $k$-form $\sigma$ on $\mathbb{Z}^p$ assigns a $k$-form,
\[
\sigma_\mbn=f_{i_1,\ldots,i_k}(\mbn)\,
\Delta^{i_1}\big|_\mbn\wedge\cdots\wedge\Delta^{i_k}\big|_\mbn,
\]
to each $\mbn\in\mathbb{Z}^p$. In view of the invariance of $\Delta^i$ under horizontal translations, we write
\begin{equation}
\sigma=f_{i_1,\ldots,i_k}(\mbn)\,
\Delta^{i_1}\wedge\cdots\wedge\Delta^{i_k}.
\end{equation}
The exterior difference operator $\bD$ maps difference $k$-forms to difference $(k+1)$-forms as follows:
\begin{equation}
\bD\sigma=\Delta^i\wedge \diffi\sigma.
\end{equation}
Unlike the exterior derivative $\dv $, the exterior difference $\bD$ is not a derivation; however, like $\dv $, it satisfies the important identity $\bD^2=0$. Note also that $\bD n^i=\Delta^i$. The exterior difference acts pointwise on difference forms over $\mathbb{Z}^p$ and extends immediately to difference forms over $\mathbb{Z}^p\times P(\mathbb{R}^q)$,
\[
\sigma=f_{i_1,\ldots,i_k}(\mbn,(u^{\alpha}_{\mbJ}))\,
\Delta^{i_1}\wedge\cdots\wedge\Delta^{i_k}.
\]
In particular, the exterior difference of a difference $(p-1)$-form,
\[
\sigma=\sum_{i=1}^p(-1)^{i-1}F^i(\mbn,(u^{\alpha}_{\mbJ}))\,
\Delta^{1}\wedge\cdots\wedge\widehat{\Delta^{i}}\wedge\cdots\wedge\Delta^{p}
\]
where $\widehat{\Delta^i}$ denotes the absence of $\Delta^i$, is
\[
\bD\sigma=\text{Div}\,\mathbf{F}\,\Delta^{1}\wedge\cdots\wedge\Delta^{p},\qquad\text{where}\quad \text{Div}\,\mathbf{F}:=\diffi\!\left\{ F^i(\mbn,(u^{\alpha}_{\mbJ}))\right\}.
\]
Any function of the form $\text{Div}\,\mathbf{F}$, as defined above, is called a (difference) \textit{divergence}.

To obtain the difference variational bicomplex, we combine the above exterior difference and differential structures, using the wedge product. From here on, we consider only forms whose restriction to any particular $P_\mbn(\mathbb{R}^q)$ have coefficients depending only on $\mbn$ and a finite subset, denoted $[\mbu]$, of the coordinates $(u^{\alpha}_{\mbJ})$. Under this condition, a $(k,l)$-\textit{form} on
$\mathbb{Z}^p\times P(\mathbb{R}^q)$ is a $(k+l)$-form $\sigma$ that can be written (without redundancies) as
\begin{equation}\label{eq:klform}
\sigma=f^{\mbJ_1,\ldots,\mbJ_l}_{i_1,\ldots,i_k;\alpha_1,\ldots,\alpha_l}(\mbn,[\mbu])\,
\Delta^{i_1}\wedge\cdots\wedge\Delta^{i_k}\wedge\dv \!
u^{\alpha_1}_{\mbJ_1}\wedge\cdots\wedge\dv \!
u^{\alpha_l}_{\mbJ_l};
\end{equation}
we denote the set of all such forms by $\Omega^{k,l}$. The exterior derivative is the mapping $\dv:\Omega^{k,l}\rightarrow\Omega^{k,l+1}$ whose action on \eqref{eq:klform} gives
\begin{equation}\label{eq:dvom}
\dv\!\sigma=\frac{\p}{\p u^\alpha_\mbJ}\left\{f^{\mbJ_1,\ldots,\mbJ_l}_{i_1,\ldots,i_k;\alpha_1,\ldots,\alpha_l}(\mbn,[\mbu])\right\}\,
\dv\! u^\alpha_\mbJ\wedge\Delta^{i_1}\wedge\cdots\wedge\Delta^{i_k}\wedge\dv \!
u^{\alpha_1}_{\mbJ_1}\wedge\cdots\wedge\dv \!
u^{\alpha_l}_{\mbJ_l}.
\end{equation}
Shifts of \eqref{eq:klform} are given by
\begin{equation}\label{eq:siom}
\es_\mbK\;\!\!\sigma=\es_\mbK\left\{f^{\mbJ_1,\ldots,\mbJ_l}_{i_1,\ldots,i_k;\alpha_1,\ldots,\alpha_l}(\mbn,[\mbu])\right\}\,
\Delta^{i_1}\wedge\cdots\wedge\Delta^{i_k}\wedge\dv \!
u^{\alpha_1}_{\mbJ_1+\mbK}\wedge\cdots\wedge\dv \!
u^{\alpha_l}_{\mbJ_l+\mbK},
\end{equation}
because \eqref{sdel} implies that $\es_\mbK\!\Delta^j=\Delta^j$. 
Bearing this in mind, the exterior difference is the mapping
\begin{equation}\label{exdiff}
\begin{aligned}
\dDh:\Omega^{k,l}&\rightarrow\Omega^{k+1,l}\\
\sigma&\mapsto \Delta^i\wedge \diffi\sigma.  %,\quad\text{where }\diffi=\es_i-\id.
\end{aligned}
\end{equation}
%where $\diffi$ is the forward difference given by \eqref{symdiffi}.

%\noindent\textbf{Notes}:\\
\begin{rem}
1. The operator $\diffi$ is the \textit{Lie difference} with respect to the horizontal translation $\te_{\mathbf{1}_i}$, because
\[
(\diffi \sigma)|_\mbn =\te_{\mathbf{1}_i}^*(\sigma_{\mbn+\mathbf{1}_i})-\sigma_{\mbn},
\]
the right-hand side of this expression being the standard definition of the Lie difference \cite{CramPir1987}.\\
2. We use $\dDh$ instead of $\bD$ (which was designed for pure difference forms), as it is helpful to mirror the standard notation used for the differential variational bicomplex. Both $\dDh$ and $\dv$ are invariant under all  allowable changes of the coordinates used to describe their respective spaces, namely $GL(p,\mathbb{Z})$ transformations of the base space $\mathbb{Z}^p$ and diffeomorphisms of $\mathbb{R}^q$ (prolonged to $\mathbb{Z}^p\times P(\mathbb{R}^q)$). 
\end{rem}


\begin{lem}\label{lemdD}
	The operators $\dDh$ and $\dv$ satisfy the identity
	\begin{equation}\label{anticom}
	\dDh\dv=-\dv\dDh.
	\end{equation}
	Consequently, the operator $\dD:=\dDh+\dv$ satisfies $(\dD)^2=0$.
\end{lem}
\begin{proof}
	To prove \eqref{anticom}, apply $\dDh\dv$ to an arbitrary $(k,l)$-form $\sigma$, then use the identities \eqref{eq:siom} and $\es_i(\p f/\p u^\alpha_{\mbJ})=\p (\es_i\! f)/\p (\es_i\! u^\alpha_{\mbJ})$. The identity for $\dD$ follows from $(\dDh)^2=0$ and $\dv^2=0$.
\end{proof}

The operator $\dD$, which we call the \textit{exterior difference-derivative}, is analogous to the exterior derivative $\upd$ on the infinite jet bundle. It splits into horizontal and vertical components, from which the difference variational bicomplex can be constructed in the same way as for the differential case (with $\dDh$ replacing $\upd_{\uph}$).


% Figure environment removed

For variational problems, a difference version of the interior Euler operator is needed to form the augmented difference variational bicomplex, which is shown in Fig. \ref{fig:mdvb}. Here, summation by parts replaces integration by parts, yielding the \textit{difference interior Euler operator} $\mcI^{\vartriangle}$ defined by
\begin{equation} %\label{eq:intEuler}
\mcI^{\vartriangle}(\sigma):=\frac{1}{l}\,\dv \!u^{\alpha}
\wedge{\es_{-\mbJ}\left(\frac{\partial}{\partial
		u^{\alpha}_{\mbJ}}\intp\sigma\right)}, \qquad
\sigma\in\Omega^{p,l}.
\end{equation}
Note that $\es_{-\mbJ}$ is the formal adjoint of $\es_{\mbJ}$. The \textit{difference Euler--Lagrange operator} $\mcE^{\vartriangle}:\Omega^{p,0}\rightarrow\mathscr{F}^1$ is defined by $\mcE^{\vartriangle}:=\mcI^{\vartriangle}\dv$. For a difference Lagrangian form, $\msL[\mbu]=L(\mbn,[\mbu])\Delta^1\wedge\cdots\wedge\Delta^p\in\Omega^{p,0}$,
\begin{equation}
\mcE^{\vartriangle}(\msL)=\es_{-\mbJ}\left(\frac{\partial L(\mbn,[\mbu])}{\partial u^{\alpha}_{\mbJ}}\right)\dv \!u^\alpha\wedge\Delta^1\wedge\cdots\wedge\Delta^p.
\end{equation}
Therefore the difference Euler--Lagrange equations,
\[
\es_{-\mbJ}\left(\frac{\partial L(\mbn,[\mbu])}{\partial u^{\alpha}_{\mbJ}}\right)=0,
\]
are the coefficients of $\mcE^{\vartriangle}(\msL)=0$.
The operators
$\delta_{\upv}^{\vartriangle}:\mathscr{F}^l\rightarrow\mathscr{F}^{l+1}$ are
defined by
$\delta_{\upv}^{\vartriangle}:=\mcI^{\vartriangle}\dv $. Direct computation shows that $\mcI^{\vartriangle}$ is a projection, that is,
\begin{equation}\label{iproj}
\left(\mcI^{\vartriangle}\right)^2=\mcI^{\vartriangle},
\end{equation} and that the conditions for a cochain complex are satisfied by the rows, columns and edge sequence:
\begin{equation}\label{coch}
\mcI^{\vartriangle}\upd_{\uph}^{\vartriangle}=0,\qquad \mcE^{\vartriangle}\upd_{\uph}^{\vartriangle}=0,\qquad \delta_{\upv}^{\vartriangle}\mcE^{\vartriangle}=0,\qquad \left(\delta_{\upv}^{\vartriangle}\right)^2=0.
\end{equation}
Indeed, the augmented difference variational bicomplex is exact, just as in the differential case. A proof of this is outlined in the Appendix. 


Let $(\p_{n^1},\p_{n^2},\ldots,\p_{n^p})$ be the duals to the difference one-forms $(\Delta^1,\Delta^2,\ldots,\Delta^p)$; the duals to the differential one-forms $\dv\!u^\alpha_{\mbJ}$ are $\p/\p{u^{\alpha}_{\mbJ}}$. These satisfy
\begin{equation}
\p_{n^i}\intp\Delta^j=\delta_i^j,\qquad
\p_{n^i}\intp\dv \!
u^{\alpha}_{\mbJ}=0,\qquad
\frac{\partial}{\partial u^{\alpha}_{\mbJ}}\intp \Delta^j=0,\qquad \frac{\partial}{\partial u^{\alpha}_{\mbJ}}\intp \dv \!u^\beta_\mbK=\delta^\beta_\alpha\delta^\mbJ_\mbK.
\end{equation}

For difference equations, the base space $\mathbb{Z}^p$ is discrete, so every (tangent) vector field is vertical. A locally smooth vector field $\mbv_0=Q^\alpha\,\p/\p u^\alpha$ on the total space, prolonged to all orders, yields the vector field $\mbv=\es_{\mbJ}\!Q^\alpha\,\p/\p u^\alpha_{\mbJ}$ on $\mathbb{Z}^p\times P(\mathbb{R}^q)$. In much the same way as for differential equations, this prolongation formula also applies when each $Q^\alpha$ depends on $\mbn$ and finitely many shifts of $\mbu$, in which case $\mbv$ is a generalized vector field on $\mathbb{Z}^p\times P(\mathbb{R}^q)$; moreover,  $\mbv$ commutes with each $\es_i$ \cite{Hy2014}. The $q$-tuple $(Q^1,Q^2,\dots,Q^q)$ that determines the generalized vector field $\mbv$ is called its \textit{characteristic}.

The Lie derivative of a $(k,l)$-form $\sigma$ with respect to a generalized vector field $\mbv$ on $\mathbb{Z}^p\times P(\mathbb{R}^q)$ is, by Cartan's formula,
\begin{equation}
\mcL_\mbv\sigma=\mbv\intp\dv\;\!\!\sigma+\dv(\mbv\intp\sigma).
\end{equation}
(See also the definition \eqref{lieddiscrete} through the corresponding transformation.)
This mirrors the differential case, as do the proofs of the following results.
\begin{prop}\label{prop:symmetry}
	Let $\sigma$ be a $(k,l)$-form on $\mathbb{Z}^p\times P(\mathbb{R}^q)$. If $\mbv$ is a generalized vector field on $\mathbb{Z}^p\times P(\mathbb{R}^q)$ then
	\[
	\mbv\intp\dDh\;\!\!\sigma+\dDh(\mbv\intp\sigma)=0,
	\]
	so
	\[
	\mcL_\mbv \sigma=\mbv\intp\dD\;\!\!\sigma+\dD(\mbv\intp\sigma).	
	\]
	Furthermore,
	\[
	\p_{n^i}\intp\dv \;\!\!\sigma+\dv (\p_{n^i}\intp\sigma)=0,
	\]
	and the Lie difference of $\sigma$ with respect to the horizontal translation $\te_{\mathbf{1}_i}$ satisfies the identity
	\[
	\diffi\sigma=\p_{n^i}\intp\dDh\;\!\!\sigma+\dDh(\p_{n^i}\intp\sigma).
	\]
	Therefore,
	\[
	\diffi\sigma=\p_{n^i}\intp\dD\;\!\!\sigma+\dD(\p_{n^i}\intp\sigma).
	\]
\end{prop}

Remarkably, both the (vertical) Lie derivative and (horizontal) Lie difference satisfy a formula that is similar to Cartan's, with the exterior difference-derivative on $\mathbb{Z}^p\times P(\mathbb{R}^q)$ replacing the exterior derivative on $J^{\infty}(X\times U)$.

Let
$\msL[\mbu]=L(\mbn,[\mbu])\Delta^1\wedge\cdots\wedge\Delta^p\in\Omega^{p,0}$ be a given Lagrangian difference form, with Lagrangian $L(\mbn,[\mbu])$.
A generalized vector field $\mbv$ on $\mathbb{Z}^p\times P(\mathbb{R}^q)$ is a
\textit{variational symmetry generator} if $\mbv(L)$ is a null Lagrangian, that is, if
there exist functions $F^i(\mbn,[\mbu])$ such that
\begin{equation}
\mbv(L)=\diffi F^i(\mbn,[\mbu]).
\end{equation}
Equivalently, $\mbv$ is a variational symmetry generator if there exists
$\sigma\in\Omega^{p-1,0}$ such that
\begin{equation}\label{eq:vs}
\mbv\intp\dv \!\msL
=\dDh\!\sigma;
\end{equation}
in coordinates,
\begin{equation}
\sigma=\sum_{i=1}^p(-1)^{i-1}F^i(\mbn,[\mbu])\Delta^1\wedge\cdots\wedge\widehat{\Delta^i}\wedge\cdots\wedge\Delta^p=F^i(\mbn,[\mbu])\,\p_{n^i}\intp\text{vol},
\end{equation}
where $\text{vol}=\Delta^1\wedge\cdots\wedge\Delta^p$ is the volume $p$-form.
As $\mcI^{\vartriangle}$ is a projection and the difference variational bicomplex is exact, a difference version of equality \eqref{eq:ii} holds. For each $\sigma\in\Omega^{p,1}$, there exists $\tau\in\Omega^{p-1,1}$ such that 
\begin{equation}
\sigma=\mcI^{\vartriangle}(\sigma)-\upd_{\uph}^{\vartriangle}\!\tau.
\end{equation}
In particular, for $\sigma=\dv \!\msL$ there exists $\tau\in\Omega^{p-1,1}$ such that
\begin{equation}\label{eq:dvLag}
\dv \!\msL=\mcE^{\vartriangle}(\msL)
-\dDh\!\tau.
\end{equation}
By using \eqref{eq:vs}, \eqref{eq:dvLag} and Proposition \ref{prop:symmetry}, we obtain
\begin{equation}\label{eq:ddcl}
\dDh\!\left(\sigma-\mbv\intp\tau\right)
=\mbv\intp\mcE^{\vartriangle}(\msL),
\end{equation}
which gives a conservation law \label{sectiondvb}
\begin{equation}
\label{eq:Noecl}
\dDh\!\left(\sigma-\mbv\intp\tau\right)=0
\text{~~~~on solutions of~~~~}
\mcE^{\vartriangle}(\msL)=0.
\end{equation}
The conservation law \eqref{eq:Noecl} is a coordinate-free version of the difference conservation law obtained by Noether's (First) Theorem; its differential counterpart was proved in \cite{BrHyLa2010}. (See \cite{Ko2011} for a comprehensive history of Noether's theorems on variational symmetries.)


%%%%%%%%%%%%%%%%%%%%%%%%%%%%%%%%%%%%%%%%%%%%%%%%%%%%%%%%%%%%%%%%%%%%%%%%%%%%%%%%%%%%%%%%%%
\section{Discrete mechanics via the difference variational bicomplex}
\label{sec:hasy}
In \cite{MaWe2001} and references therein, discrete mechanics is
developed using the standard approach in classical mechanics, that is,
by studying the discrete equations of motion on a manifold equipped with a
closed nondegenerate two-form. In \cite{BrHyLa2010}, Bridges \textit{et al}. used
the (differential) variational bicomplex to re-examine classical
mechanics. In this section, we apply the augmented difference  variational
bicomplex to discrete mechanics, with base space $\mathbb{Z}$ and the fibre (in total space) $\mathbb{R}^2$ (for simplicity). In the usual notation\footnote{Throughout this section only, $p$ and $q$ are real-valued variables, not dimensions (which are given).}, let $(n,q,p)$
be the standard coordinates on the total space $\mathbb{Z}\times \mathbb{R}^2$; let $\es$ be the forward shift in $n$ and the forward difference operator be $\diffn=\es-\id$.

Consider the following $(0,2)$-form, which is vertically closed and nondegenerate:
\begin{equation}
\omega =\dv\!p\wedge \dv\!q.
\end{equation}
This gives each fibre in the total space the structure of a symplectic manifold. Suppose that
%all solutions of a system of first-order difference equations for $(p,q)$ preserve $\omega$. Then (on the solution manifold)
the horizontal translation map $\te_1:(n,p,q)\mapsto(n+1,p,q)$ is a symplectomorphism, so that $\te_1^*\omega_{n+1}=\omega_n$. In the prolongation space $\mathbb{Z}\times P(\mathbb{R}^2)$, this condition amounts to $\diffn\omega=0$, that is,
\[
\dv\!p_1\wedge \dv\!q_1-\dv\!p_0\wedge \dv\!q_0=0.
\]
As the augmented difference  variational bicomplex is exact, there exists a Hamiltonian function $H$ on $\mathbb{Z}\times P(\mathbb{R}^2)$ that satisfies
\begin{equation}
(p_1-p_0) \dv\!q_0-(q_1-q_0) \dv\!p_1 =-\dv\!H.
\end{equation}
Consequently, $H$ is a function of $(n,q_0,p_1)$ only. In coordinates, the
symplectic map is
\begin{equation}
\label{eq:flow}
q_1-q_0=\frac{\partial H(n,q_0,p_1)}{\partial p_1},\qquad
p_1-p_0=-\,\frac{\partial H(n,q_0,p_1)}{\partial q_0}.
\end{equation}
With the step-length incorporated into $H$, this is the Euler-B discretization method for a continuous
Hamiltonian system; see \cite{LeRe2004}.

Reversing the above argument (with $p_{-1}$ replacing $p_0$), one can start with a Hamiltonian $H(n,p,q)$ defined on the total space and apply the map
\begin{equation}
\label{eq:flow1}
q_1-q_0=\frac{\partial H(n,q_0,p_0)}{\partial p_0},\qquad
p_0-p_{-1}=-\,\frac{\partial H(n,q_0,p_0)}{\partial q_0}.
\end{equation}
on each $P_n(\mathbb{R}^2)$. Then
\[
(p_0-p_{-1}) \dv\!q_0-(q_1-q_{0}) \dv\!p_0 =-\dv\!H,
\]
so the map preserves the symplectic $(0,2)$-form
\begin{equation}\label{om1std}
\omega =\dv\!p_{-1}\wedge \dv\!q_0
\end{equation}
on $\mathbb{Z}\times P(\mathbb{R}^2)$. This approach has the advantage that
(similarly to the corresponding continuous case) the symplectic map \eqref{eq:flow1} can be written in terms of a self-adjoint matrix operator:
\begin{equation}
    \left(
\begin{array}{cc}
  0&-(\id-\es^{-1})\\
  \es-\id&0
\end{array}
    \right)
      \left(
\begin{array}{c}
  q_0\\
  p_0
\end{array}
    \right)=
         \left(
    \begin{array}{c}
    \partial H(n,q_0,p_0)/\partial q_0\\
    \partial H(n,q_0,p_0)/\partial p_0
    \end{array}
    \right).
\end{equation}

The system \eqref{eq:flow1} amounts to the Euler--Lagrange equations for the Lagrangian $(1,0)$-form
\[
\msL=\{p_0(q_1-q_0)-H(n,q_0,p_0)\}\Delta_n.
\]
Specifically,
\[
\mcE^{\vartriangle}(\msL)=\dv\!\msL+\upd_{\uph}^{\vartriangle}\!\eta,\qquad \eta=p_{-1}\dv\!q_0.
\]
So $\omega=\dv\!\eta$.

For symplectic difference maps in general, at most one of the Hamiltonian function and the symplectic form is defined on the total space, so it is essential to work in an appropriate prolongation space.

So far, we have worked mainly in terms of the given coordinates. For an entirely coordinate-free formulation, the exterior difference operator $\dDh$ is used in place of the Lie difference $\diffn$.


This construction is easily extended to mechanical systems with higher-dimensional fibres. We now generalize it to higher-dimensional base spaces, with application to multisymplectic P$\Delta$Es.
%%%%%%%%%%%%%%%%%%%%%%%%%%%%%%%%%%%%%%%%%%%%%%%%%%%%%%%%%%%%%%%%%%%%%%%%%%%%%%%%%%%%%%%%%%
\section{Multisymplectic systems of P$\Delta$Es and the bicomplex}
\label{sec:mupde}

First introduced by Bridges \cite{Br1997b}, multisymplectic structure generalizes the classical Hamiltonian
structure for finite-dimensional systems to infinite-dimensional systems. The
multisymplectic formulation and multisymplectic geometry has been greatly studied and widely applied during the last decades, e.g. 
\cite{Br1997a,BrRe2001,CoHoHy2007,HuDe2008,SuQi2003,Wa2008,Br2017,CaIbDe1999,La2000,La2004,MaPaSh1998}.

Let $\mbx$ and $\mbu$ be multi-dimensional continuous independent and dependent variables, respectively. A system of PDEs is multisymplectic if (but not only if) it can be
represented as a variational problem with a Lagrangian 
\begin{equation}
L[\mbu]=L_{\alpha}^i(\mbx,\mbu)D_iu^{\alpha}-H(\mbx,\mbu).
\end{equation}
Hence the Euler--Lagrange equations are
\begin{equation}
K_{\alpha\beta}^i(\mbx,\mbu)D_iu^{\beta}-\frac{\partial L_{\alpha}^i}{\partial x^i}-\frac{\partial H}{\partial
u^{\alpha}}=0,
\end{equation}
where
\begin{equation}
K^i_{\alpha\beta}(\mbx,\mbu)=\frac{\partial L_{\beta}^i}{\partial
u^{\alpha}}-\frac{\partial L_{\alpha}^i}{\partial u^{\beta}}.
\end{equation}
Bridges \cite{Br1997b} showed that closed symplectic two-forms can then be defined as
\begin{equation}
\kappa^i=\sum_{\alpha<\beta}K_{\alpha\beta}^i(\mbx,\mbu)\upd\!u^{\alpha}\wedge\upd\!u^{\beta},
\end{equation}
such that the structural conservation law $D_{i}\kappa^{i}=0$ holds on solutions of the Euler--Lagrange equations.
Bridges \textit{et al}. \cite{BrHyLa2010} generalized this approach by using the
differential variational bicomplex; see also \cite{Br2017}. Instead of closed symplectic
forms, one has vertically closed symplectic two-forms:
\begin{equation}
\kappa^i=\sum_{\alpha<\beta}K_{\alpha\beta}^i(\mbx,\mbu)\dv \!u^{\alpha}\wedge\dv \!u^{\beta};
\end{equation}
the structural conservation law $D_i\kappa^i$ again vanishes on solutions of the Euler--Lagrange equations.

 Similarly, for a system of P$\Delta$Es, a prerequisite of
multisymplecticity is the existence of vertically closed {\it standard}
$(0,2)$-forms
\begin{equation}
\label{kap}
\kappa^{i}=\es_{i}^{-1}\left({K_{\alpha\beta}^{i}(\mbn,\mbu)}\right)\dv \!\left(\es_{i}^{-1}u^{\alpha}\right)
\wedge\dv \!u^{\beta};
\end{equation}
here $K_{\alpha\beta}^{i}$ are smooth functions with respect to the
dependent variables $\mbu$ and $\es_{i}^{-1}=\es_{-\mathbf{1}_i}$. The vertically closed $(0,2)$-forms given by
(\ref{kap}) are conserved if they satisfy the condition for conservation laws, i.e., the
divergence expression
\begin{equation}
\diffi \kappa^i =0
\end{equation}
holds on solutions of the given system of P$\Delta$Es. Recall the 
volume element $\volf$ on $\mathbb{Z}^p$, 
a $(p,0)$-form
\begin{equation}
\volf=\Delta^1\wedge\Delta^2\wedge\cdots\wedge\Delta^p.
\end{equation}
Let $\zeta=\kappa^i\partial_{n^i}$, where each $\kappa^i$ is a vertically closed
$(2,0)$-form, and define the $(p-1,2)$-form
\begin{equation}
\label{eq3} \omega=\zeta\intp\volf.
\end{equation}
Then, as
\begin{equation}\label{okok}
\dDh\!\omega=\left(\diffi \kappa^i\right)\wedge\volf,
\end{equation}
the conservation of multisymplecticity is equivalent to the
condition
$\dDh\!\omega=0$.

Conversely, let $\omega$ be a vertically closed $(p-1,2)$-form on
the space $\mathbb{Z}^p\times U$ with $U\subset \mathbb{R}^q$, and
suppose on all solutions of some system of first-order P$\Delta$Es,
one has
\begin{equation}
\label{eq4}
\dDh\!\omega=0.
\end{equation}
In coordinates, the $(p-1,2)$-form $\omega$ can be written
as
\begin{equation}\label{eq:multiform}
\omega=\sum_{i=1}^p\es_i^{-1}\left(f_{\alpha\beta}^i(\mbn,\mbu)\right)\upd_
{\upv}\!\left(\es_i^{-1}\!u^{\alpha}\right)\wedge\dv \!u^{\beta}\wedge(\partial_{n^i}\intp
\volf).
\end{equation}
From (\ref{eq3}), we see that
$\omega=\zeta\intp\volf=\kappa^i\wedge(\partial_{n^i}\intp\volf)$
and hence the vertically closed $(0,2)$-forms are obtained:
\begin{equation}
\kappa^i=\es_i^{-1}\left(f_{\alpha\beta}^i(\mbn,\mbu)\right)\upd_
{\upv}\!\left(\es_i^{-1}\!u^{\alpha}\right)\wedge\dv \!u^{\beta}.
\end{equation}
These forms $\kappa^i$ are conserved because
$\dDh\!\omega=\left(\diffi \kappa^i\right)\wedge\volf$
and $\dDh\!\omega=0$
 on solutions of the difference system. This implies the
multisymplecticity of the system.

%\begin{rem}
 Note that since the $(p-1,2)$-form \eqref{eq:multiform} is defined in this form, and hence \eqref{kap}, because similarly the continuous case multisymplectic P$\Delta$Es  are governed by first-order fully degenerate Lagrangians (see, e.g. \eqref{lp}); we shall call these forms {\it standard multisymplectic forms}. For a general variational problem, its multisymplectic  form may be defined on the total prolongation space.
% \end{rem}

Now we are going to propose a systematic method, with which one can
proceed from a multisymplectic structure to an equivalent difference
Lagrangian structure. As $\omega$ is a vertically closed
$(p-1,2)$-form, there exists an $\eta\in\Omega^{p-1,1}$ such that
\begin{equation}
\dv \!\eta=\omega.
\end{equation}
From (\ref{eq4}), we conclude that on all solutions of the given
system,
\begin{equation}
\dv \!\dDh\!\eta=
-\dDh\!\dv \!\eta=
-\dDh\!\omega=0.
\end{equation}
This implies the existence (locally) of a $(p,0)$-form
$\msL$, such that on all solutions
\begin{equation}
\dDh\!\eta=\dv \!\msL.
\end{equation}
Therefore, as
\begin{equation}
\mcE^{\vartriangle}(\msL)=\mcI^{\vartriangle}(\dv \!
\msL)=\mcI^{\vartriangle}(\dDh\!\eta)=0~~~~(\text{on
solutions}),
\end{equation}
the $(p,0)$-form $\msL$ is a Lagrangian form for the
system of P$\Delta$Es.

One can also use multimomentum  maps for discrete multisymplectic
systems to derive conservation laws. Let $G$ denote a Lie group of transformations preserving
the $(p-1,2)$-form $\omega$, and let $\mbv_{\xi}$ be the infinitesimal
generator with $\xi\in \mathfrak{g}$ satisfying
\begin{equation}
\mbv_{\xi}:=\frac{\upd}{\upd\!\varepsilon}\Big|_{\varepsilon=0}\exp(\varepsilon\xi)(\mbn,\mbu);
\end{equation}
here $\mathfrak{g}$ is the associated Lie algebra. Obviously the
characteristic for $\mbv_{\xi}$ is
$Q^{\alpha}=\mbv_{\xi}\intp\dv \!u^{\alpha}$.
Assume its prolongation (with the same notation $\mbv_{\xi}$) yields a vertically exact
form $\mbv_{\xi}\intp\omega$; then there exists some $\lambda_{\xi}\in\Omega^{p-1,0}$ such
that
\begin{equation}
\label{eq5}
\mbv_{\xi}\intp\omega=\dv \!\lambda_{\xi}.
\end{equation}
Let $\mathfrak{g}^{\ast}$ denote the dual space of the Lie algebra
$\mathfrak{g}$, with which we define the difference multimomentum map
$J:\mathbb{Z}^p\times P_\mbn(\mathbb{R}^q)\to\mathfrak{g}^{\ast}\otimes\Omega^{p-1,0}$
as
\begin{equation}
\label{mm} J(\mbn,[\mbu])(\xi)=\lambda_{\xi}(\mbn,[\mbu]).
\end{equation}
We finish this section by deducing the conditions under which the
$(p-1,0)$-form $\lambda_{\xi}$ is in the form of a conservation law
given by \eqref{eq:Noecl} for the multisymplectic system written as  difference Euler--Lagrange equations. Suppose we are given a
Lagrangian $(p,0)$-form
\begin{equation}
\label{lp}
\msL=L_{\beta}^i(\mbn,\mbu)\dDh\!u^{\beta}
\wedge(\partial_{n^i}\intp\volf)-H(\mbn,\mbu)
\volf,
\end{equation}
where $H(\mbn,\mbu)$ is smooth with respect to $\mbu$. Through a direct
calculation,
\begin{equation}
L_{\beta}^i(\mbn,\mbu)\dDh\!u^{\beta}\wedge(\partial_{n^i}\intp\volf)=
L_{\beta}^i(\mbn,\mbu)\left(\diffi u^{\beta}\right)\volf,
\end{equation}
with which we find the difference Euler--Lagrange equations
associated with the Lagrangian form (\ref{lp}) as
\begin{equation}
\label{el} \sum_{i}\frac{\partial L_{\beta}^i(\mbn,\mbu)}{\partial
u^{\alpha}}\diffi u^{\beta}+
\sum_i{(\es_i^{-1}-\id)L_{\alpha}^i(\mbn,\mbu)}-\frac{\partial
H(\mbn,\mbu)}{\partial u^{\alpha}}=0.
\end{equation}
The following $(p-1,1)$-form $\eta$ is verified to satisfy
\eqref{eq:dvLag} through direct calculation:
\begin{equation}
\eta=\sum_{i}\left(\es_i^{-1}L_{\alpha}^i(\mbn,\mbu)\right)\dv \!
u^{\alpha}\wedge(\partial_{n^i}\intp\volf).
\end{equation}
It leads to a multisymplectic $(p-1,2)$-form as follows:
\begin{equation}\label{olol}
\omega=\dv \!\eta
=\sum_{i}\left(\es_i^{-1}\frac{\partial
L_{\alpha}^i(\mbn,\mbu)}{\partial u^{\beta}}\right)
\dv (\es_i^{-1}\!u^{\beta})
\wedge\dv \!u^{\alpha}\wedge
(\partial_{n^i}\intp\volf).
\end{equation}
Conversely, for a linear system of P$\Delta$Es, if there is a
vertically closed $(p-1,2)$-form $\omega$ satisfying
$\dDh\!\omega=0$, then
there exists a Lagrangian of the form (\ref{lp}). Therefore, the
deduction of Noether's finite difference conservation laws on page \pageref{sectiondvb} (or \eqref{eq:Noecl})
leads to a difference multimomentum map with
\begin{equation}
\lambda_{\xi}=\sigma_{\xi}-\mbv_{\xi}\intp\eta.
\end{equation}
Therefore, the extra conditions for $\lambda_{\xi}$ to be a
conservation law are
\begin{equation}
\begin{aligned}
\label{con}
&\dv \!\lambda_{\xi}=\mbv_{\xi}\intp\omega,\\
&\dDh\!\lambda_{\xi}=\mbv_{\xi}\intp\mcE^{\vartriangle}(\msL).
\end{aligned}
\end{equation}
Here, the first condition follows directly from (\ref{eq5}), while
the second one is implied by \eqref{eq:ddcl}.
Locally, if we write
 $\lambda_{\xi}\in\Omega^{p-1,0}$ as
\begin{equation}
\lambda_{\xi}=\lambda_{\xi}^i(\mbn,[\mbu])\partial_{n^i}\intp\volf,
\end{equation}
the second condition in (\ref{con}) has a local representation
\begin{equation}
\label{cll} \sum_i\diffi \lambda_{\xi}^i-Q^{\alpha}
\left(\sum_{i}\frac{\partial L_{\beta}^i(\mbn,\mbu)}{\partial
u^{\alpha}}\diffi u^{\beta}
+\sum_i(\es_i^{-1}-\id)L_{\alpha}^i(\mbn,\mbu)-\frac{\partial
H(\mbn,\mbu)}{\partial u^{\alpha}}\right)=0.
\end{equation}
\begin{thm}
For the multisymplectic system given in (\ref{el}), any discrete
multimomentum map $J(\mbn,[\mbu])(\xi)=\lambda_{\xi}(\mbn,[\mbu])$ satisfying
(\ref{con}) gives rise to a conservation law
$\dDh\!\lambda_{\xi}=0$
on all solutions of the system.
\end{thm}

Next we study several illustrative examples. The first Example \ref{exm51} shows how one can obtain mutisymplectic forms from a given degenerate Lagrangian. In the second Example \ref{exm52}, conservation laws of
multisymplectic systems are obtained through multisymplectic maps. In the last
Example \ref{exm53}, we obtain some constraints, with which the infinitesimal
generators will contribute to difference multimomentum maps.

\begin{exm}\label{exm51} Let the local coordinates  of  $\mathbb{Z}^3$ be $\mbn=(n^1,n^2,n^3)$, and let $u\in\mathbb{R}$.
Consider a Lagrangian $(3,0)$-form
\begin{equation}
\msL=\sum_{i=1}^3{L^i(\mbn,u)\dDh\!u\wedge(\partial_{n^i}\intp\volf)}
-H(\mbn,u)\volf.
\end{equation}
This amounts to the following multisymplectic P$\Delta$E
\begin{equation}
\sum_{i=1}^3{\left(\frac{\partial L^i}{\partial
u}\diffi u+(\es_i^{-1}-\id)L^i\right)}=\frac{\partial
H}{\partial u}.
\end{equation}
The corresponding multisymplectic $(2,2)$-form is
\begin{equation}
\omega=\sum_{i=1}^3\left(\left(\es_i^{-1}\frac{\partial L^i}{\partial
u}\right)\dv 
(\es_i^{-1}\!u)\wedge\dv \!u\wedge(\partial_{n^i}\intp\volf)\right).
\end{equation}
Alternatively, we can write the $(2,2)$-form $\omega$  as three vertically closed $(0,2)$-forms
\begin{equation}
\kappa^i=\left(\es_i^{-1}\frac{\partial L^i}{\partial
u}\right)\dv 
(\es_i^{-1}\!u)\wedge\dv \!u.
\end{equation}
\end{exm}

\begin{exm}\label{exm52}
 Consider a difference Euler--Lagrange equation on  $\mathbb{Z}^2\times P(\mathbb{R})$:
 \begin{equation}
\label{eq6} u_{1,0}+u_{-1,0}+u_{0,1}+u_{0,-1}-4u_{0,0}=0;
\end{equation}
here $u_{0,0}=u_{n^1,n^2}$, $u_{1,0}=u_{n^1+1,n^2}$, and so forth. It is
easily verified that this equation is of the form (\ref{el}) with
\begin{equation}
\begin{aligned}
L^1=L^2=u_{0,0},~~H=0,
\end{aligned}
\end{equation}
and $\mbv_{\xi}=\frac{\partial}{\partial u_{0,0}}$ is an
infinitesimal generator of its symmetries. From (\ref{olol}), the
multisymplectic $(1,2)$-form is
\begin{equation}
\omega=\dv \!u_{-1,0}\wedge\dv \!
u_{0,0}\wedge\Delta^2-\dv \!u_{0,-1}\wedge\dv \!u_{0,0}
\wedge\Delta^1.
\end{equation}
The exterior form
$\mbv_{\xi}\intp\omega=\dv \!\lambda_{\xi}$
is vertically exact with
\begin{equation}
\lambda_{\xi}=(u_{0,-1}-u_{0,0})\Delta^1+(u_{0,0}-u_{-1,0})\Delta^2.
\end{equation}
Hence its components are $\lambda_{\xi}^1=u_{0,0}-u_{-1,0}$ and
$\lambda_{\xi}^2=-(u_{0,-1}-u_{0,0})$.
 Therefore, (\ref{cll})
amounts to the conservation law
\begin{equation}
\sum_{i=1}^2{\diffi \lambda_{\xi}^i}=0~~~
\text{on the solutions of (\ref{eq6})}.
\end{equation}
%\begin{equation}
%\sum_{i=1}^2{\diffi \left(\lambda_{\xi}^i-u_{0,0}+\es_i^{-1}\!u_{0,0}\right)}=0~~~
%\text{on the solutions of (\ref{eq6})}.
%\end{equation}
\end{exm}

\begin{exm}\label{exm53}
Consider the following discrete Lagrangian defined on $\mathbb{Z}^2\times P(\mathbb{R}^3)$:
\begin{equation}
\begin{aligned}
L=\left(u^1_{0,1}-u^1_{1,0}\right)u^3_{0,0}-H
\end{aligned}
\end{equation}
with the Hamiltonian
\begin{equation}
H=-\left(u^1_{0,0}+u^3_{0,0}\right)u^2_{0,0}-C\ln u^2_{0,0}.
\end{equation}
Here $C$ is a constant. It has an equivalent representation as a
Lagrangian form given in (\ref{lp}) with nonzero components
\begin{equation}
L_1^1=-u^3_{0,0},~~L^2_1=u^3_{0,0},
\end{equation}
which leads to the associated difference Euler--Lagrange equations
\begin{equation}
\begin{aligned}
\begin{cases}
u^3_{0,-1}-u^3_{-1,0}+u^2_{0,0}=0,\vspace{0.15cm}\\
u^1_{0,0}+u^3_{0,0}+\frac{C}{u^2_{0,0}}=0,\vspace{0.15cm}\\
u^1_{0,1}-u^1_{1,0}+u^2_{0,0}=0.
\end{cases}
\end{aligned}
\end{equation}
It is a multisymplectic system with the underlying multisymplectic
$(1,2)$-form as follows
\begin{equation}
\omega=-\dv \!u^3_{-1,0}\wedge\dv \!u^1_{0,0}
\wedge\Delta^2-\dv \!u^3_{0,-1}\wedge\dv \!u^1_{0,0}
\wedge\Delta^1.
\end{equation}
Denote the infinitesimal generators $\mbv_{\xi}$ of Lie point symmetries
and the $(1,0)$-form $\lambda_{\xi}$ respectively by
\begin{equation}
\mbv_{\xi}=Q^{\alpha}(\mbn,\mbu)\frac{\partial}{\partial u^{\alpha}_{0,0}}
\end{equation}
and
\begin{equation}
\begin{aligned}
\lambda_{\xi}&=\lambda_{\xi}^i\left(\partial_{n^i}\intp \volf\right)\\
&=\lambda_{\xi}^1\Delta^2-\lambda^2_{\xi}\Delta^1.
\end{aligned}
\end{equation}
Therefore, $\dv \!\lambda_{\xi}=
\mbv_{\xi}\intp\omega$ (for the prolonged vector field) leads to
\begin{equation}
\begin{aligned}
&Q^1=\frac{\partial\lambda^1_{\xi}}{\partial u^3_{-1,0}}=-\frac{\partial\lambda^2_{\xi}}{\partial u^3_{0,-1}},\\
&Q^3=-\es_1\left(\frac{\partial\lambda^1_{\xi}}{\partial
u^1_{0,0}}\right)=\es_{2}\left(
\frac{\partial\lambda^2_{\xi}}{\partial u^1_{0,0}}\right).
\end{aligned}
\end{equation}
This also implies that
$\lambda^1_{\xi}=\lambda^1_{\xi}(n^1,n^2,u^1_{0,0},u^3_{-1,0})$,
$\lambda^2_{\xi}=\lambda^2_{\xi}(n^1,n^2,u^1_{0,0},u^3_{0,-1})$, and both
$\lambda_{\xi}^i$ are linear functions with respect to each of their continuous
variables. Consequently, only special type of infinitesimal generators
whose characteristics $Q^1=Q^1(n^1,n^2,u_{0,0}^1)$ and
$Q^3=Q^3(n^1,n^2,u_{0,0}^3)$ are both linear functions about the continuous variables can possibly amount to difference multimomentum maps.

%$Q^1=a(n^1,n^2)u_{0,0}^1+b(n^1,n^2)$ and
%$Q^3=f(n^1,n^2)u_{0,0}^3+h(n^1,n^2)$ can possibly amount to difference multimomentum maps.
\end{exm}



\section{Multisymplectic integrator via the generalized difference variational bicomplex on non-uniform meshes}
%%%%%%%%%%%%%%%%%%%%%%%%%%%%%%%%%%%%%%%%%%%%%%%%%%%%%%%%%%%%%%%%%%%%%%%%%%%%%%%%%%%%%%%%%%
\label{sec:multisyin}
In this section, we generalize the difference variational bicomplex
on $\mathbb{Z}^p$ to non-uniform meshes. Multisymplectic integrators
are re-investigated using the generalized bicomplex structure.

For the difference variational bicomplex proposed above, we
considered only uniform discrete independent variables, i.e.,
$\mbn\in\mathbb{Z}^p$. However, with practical problems such as
discretization or numerical methods for a differential
system, we frequently encounter spaces of independent variables that
are not $\mathbb{Z}^p$. Suppose the difference system is built on a
space coordinatized by $(\mbx_\mbn,\mbu_\mbn)$ where $\mbx_\mbn$ and $\mbu_\mbn$ are the
independent and dependent variables. The dimensions of $\mbx_\mbn$ and $\mbn$
are both $p$, while we set the dimension of $\mbu_\mbn$ to be $q$. Denote
the step size or the distance of two mesh points in each direction as
\begin{equation}
\epsilon_\mbn^i:=d(\mbx_{\mbn+\mathbf{1}_i},\mbx_{\mbn}),~~i=1,2,\ldots,p,
\end{equation}
where $d(\cdot,\cdot)$ is the distance of two points with respect to a given
metric.

\begin{rem}
If the points $\{\mbx_\mbn\}$ are located in the Euclidean space
$\mathbb{R}^p$, then the distance is $\epsilon_\mbn^i=x_{\mbn+\mathbf{1}_i}-x_\mbn$.
Otherwise, it depends on the specific geometric structure of the space. For example, if the
points are on a Riemannian manifold, then obviously the distance
connecting two points will be the length of the local geodesic
connecting them.
\end{rem}

Analogously, the \textit{exterior difference-derivative}
$\upd^{\vartriangle_{\mbn}}$ from the mesh viewpoint is defined as a
summation of a horizontal operator $\upd_{\uph}^{\vartriangle_{\mbn}}$, i.e., the exterior difference, and a
vertical operator $\dv $, i.e., the exterior derivative, as
follows:
\begin{equation}
\upd^{\vartriangle_{\mbn}}=\upd_{\uph}^{\vartriangle_{\mbn}}+\dv ,
\end{equation}
where
\begin{equation}
\upd_{\uph}^{\vartriangle_{\mbn}}:=\sum_{i=1}^p\Delta^i_{\mbn}\wedge\frac{\diffi }{\epsilon_\mbn^i},\quad 
\dv :=
\dv \!u_{\mbJ}^{\alpha}\wedge
\frac{\partial}{\partial u_{\mbJ}^{\alpha}}.
\end{equation}
Here the difference one-forms are
$\Delta_{\mbn}^i=\epsilon_\mbn^i\Delta^i$. The exterior derivative $\dv $ is exactly the one given by \eqref{eq:dvom}.

\begin{rem}
When the limit $\epsilon_\mbn^i\rightarrow 0$ is taken, $\Delta_{\mbn}^i$ tends to
$\upd\!x^i$ and
$\frac{\diffi }{\epsilon_\mbn^i}$ tends to the total
derivative  $D_i$. Namely, the horizontal
operator $\upd_{\uph}^{\vartriangle_{\mbn}}$ is
an approximation of the horizontal derivative
$\upd_{\uph}\!=\upd\!x^i\wedge
D_i$ (see \eqref{eq:dhdiff}).
\end{rem}

Rewrite the horizontal operator as follows
\begin{equation}
\begin{aligned}
\upd_{\uph}^{\vartriangle_{\mbn}}=\sum_{i=1}^p\Delta^i_{\mbn}\wedge\frac{\diffi }{\epsilon_\mbn^i}
=\sum_{i=1}^p\epsilon_\mbn^i\Delta^i\wedge\frac{\diffi }{\epsilon_\mbn^i}=\sum_i\Delta^i\wedge\diffi .
\end{aligned}
\end{equation}
This is exactly the same as the exterior difference
$\upd_{\uph}^{\vartriangle}$ we introduced
for the uniform lattice $\mathbb{Z}^p$. Therefore, the two
exterior difference operators $\upd^{\vartriangle_{\mbn}}$
and $\dD$ are consistent for any step size
$\epsilon_\mbn$. Those properties of the exterior difference-derivative we studied above in
the uniform case still hold in the non-uniform case. In the following we study several numerical examples using the non-uniform difference variational bicomplex.


\begin{exm}
Consider the semilinear scalar PDE
\begin{equation}\label{semil}
  u_{tt}+\varepsilon u_{xx}+V'(u)=0,\qquad\varepsilon=\pm1;
\end{equation}
here $V(u)$ is a potential function. Bridges \& Reich \cite{BrRe2001} embedded \eqref{semil} in a first-order multisymplectic system of PDEs,
\begin{equation}
    -v_{t}-w_{x}=V'(u),\qquad u_{t}+p_{x}=v,\qquad
    u_{x}-\varepsilon p_{t}=\varepsilon w,\qquad\varepsilon
    w_{t}-v_{x}=0,
\end{equation}
which they discretized using the following St\"{o}rmer--Verlet scheme (staggered in both the $x$- and
the $t$-directions):
\begin{equation}\label{mi1}
  \begin{aligned}
    &-\frac{v_{i,j+\frac12}-v_{i,j-\frac12}}{h_t}-\frac{w_{i+\frac12,j}-w_{i-\frac12,j}}{h_x}=V'(u_{i,j}),\\
    &\frac{u_{i,j+1}-u_{i,j}}{h_t}+\frac{p_{i+\frac12,j+\frac12}-p_{i-\frac12,j+\frac12}}{h_x}=v_{i,j+\frac12},\\
    &\frac{u_{i+1,j}-u_{i,j}}{h_x}-\varepsilon\frac{p_{i+\frac12,j+\frac12}-p_{i+\frac12,j-\frac12}}{h_t}=\varepsilon w_{i+\frac12,j},\\
    &\varepsilon\frac{w_{i+\frac12,j+1}-w_{i+\frac12,j}}{h_t}-\frac{v_{i+1,j+\frac12}-v_{i,j+\frac12}}{h_x}=0.
  \end{aligned}
\end{equation}
Here $h_x=x_{i+1}-x_i$ and $h_t=t_{j+1}-t_j$ are the uniform step sizes.
Difference one-forms on the mesh are generated by
\begin{equation}
   \Delta^x=h_x\Delta^1,\qquad\Delta^t=h_t\Delta^2,
\end{equation}
where $\Delta^1$ and $\Delta^2$ are the standard difference one-forms on the
uniform space $\mathbb{Z}^2$. The multisymplectic $(1,2)$-form
(\ref{eq3}) is therefore obtained: 
\begin{equation}
\begin{aligned}
  \omega=&\left(\dv \!v_{i,j-\frac12}\wedge\dv \!u_{i,j}+\varepsilon
  \dv \!p_{i+\frac12,j-\frac12}\wedge\dv \!w_{i+\frac12,j}\right)\wedge\Delta^x\\
  &-\left(\dv \!w_{i-\frac12,j}\wedge\dv \!u_{i,j}
  -\dv \!p_{i-\frac12,j+\frac12}\wedge
  \dv \!v_{i,j+\frac12}\right)\wedge\Delta^t,
  \end{aligned}
\end{equation}
satisfying
\begin{equation}
  \upd_{\uph}^{\vartriangle}\!\omega=0~~\text{on solutions
  of (\ref{mi1})}.
\end{equation}
A Lagrangian governing (\ref{mi1}) reads
\begin{equation}
\begin{aligned}
 L=~&v_{i,j+\frac12}\frac{u_{i,j+1}-u_{i,j}}{h_t}+\varepsilon
 p_{i+\frac12,j+\frac12}\frac{w_{i+\frac12,j+1}-w_{i+\frac12,j}}{h_t}+w_{i+\frac12,j}\frac{u_{i+1,j}-u_{i,j}}{h_x}\\
 &-p_{i+\frac12,j+\frac12}\frac{v_{i+1,j+\frac12}-v_{i,j+\frac12}}{h_x}
 -\left(V(u_{i,j})+\frac{v_{i,j+\frac12}^2+\varepsilon w_{i+\frac12,j}^2}{2}\right).
 \end{aligned}
\end{equation}


\end{exm}

\begin{exm}
In \cite{Wa2008}, Wang studied the multisymplectic formulation and
structure-preserving integrator for the Zakharov system
\begin{equation}
  i\phi_{t}+\phi_{xx}+2\phi\psi=0,~~\psi_{tt}-\psi_{xx}+(|\phi|^2)_{xx}=0,
\end{equation}
where $\phi(x,t)$ is complex-valued and $\psi(x,t)$ is real-valued. It can be rewritten as a first-order system of PDEs by introducing several new variables:
\begin{equation}
  \begin{aligned}
    &-v_{t}+p_{x}=-2u\psi,~~u_{t}+q_{x}=-2v\psi,~~-u_{x}=-p,~~-v_{x}=-q,\\
    &w_{t}=\psi-(u^2+v^2),~~-\psi_{t}+\varphi_{x}=0,~~-w_{x}=-\varphi.
  \end{aligned}
\end{equation}
Wang proposed a particular Euler box scheme of this system \cite{Wa2008}, which is multisymplectic:
\begin{equation}\label{eulerscheme}
  \begin{aligned}
    &-\frac{v_{i,j+1}-v_{i,j}}{h_t}+\frac{p_{i+1,j}-p_{i,j}}{h_x}=-2u_{i,j}\psi_{i,j},~~
    \frac{u_{i,j}-u_{i,j-1}}{h_t}+\frac{q_{i+1,j}-q_{i,j}}{h_x}=-2v_{i,j}\psi_{i,j},\\
    &-\frac{u_{i,j}-u_{i-1,j}}{h_x}=-p_{i,j},~~-\frac{v_{i,j}-v_{i-1,j}}{h_x}=-q_{i,j},~~
    \frac{w_{i,j+1}-w_{i,j}}{h_t}=\psi_{i,j}-(u_{i,j}^2+v_{i,j}^2),\\
    &-\frac{\psi_{i,j}-\psi_{i,j-1}}{h_t}+\frac{\varphi_{i,j}-\varphi_{i-1,j}}{h_x}=0,~~-\frac{w_{i+1,j}-w_{i,j}}{h_x}=-\varphi_{i,j}.
  \end{aligned}
\end{equation}
Using the bicomplex theory, the associated multisymplectic $(1,2)$-form is obtained as follows
\begin{equation}
  \begin{aligned}
    \omega=&\left(-\dv \!u_{i,j-1}\wedge\dv \!v_{i,j}
    +\dv \!\psi_{i,j-1}\wedge\dv \!w_{i,j}\right)\wedge\Delta^x\\
    &-\left(\dv \!u_{i-1,j}\wedge\dv \!p_{i,j}+
    \dv \!v_{i-1,j}\wedge\dv \!q_{i,j}-
    \dv \!\varphi_{i-1,j}\wedge\dv \!w_{i,j}\right)\wedge\Delta^t,
  \end{aligned}
\end{equation}
such that
$\upd_{\uph}^{\vartriangle}\!\omega$
vanishes on solutions of the system (\ref{eulerscheme}). A Lagrangian is
\begin{equation}
  \begin{aligned}
    L=&-u_{i,j}\frac{v_{i,j+1}-v_{i,j}}{h_t}+\psi_{i,j}\frac{w_{i,j+1}-w_{i,j}}{h_t}+u_{i,j}\frac{p_{i+1,j}-p_{i,j}}{h_x}
    +v_{i,j}\frac{q_{i+1,j}-q_{i,j}}{h_x}\\
    &-\varphi_{i,j}\frac{w_{i+1,j}-w_{i,j}}{h_x}-\left(\frac{1}{2}\psi^2_{i,j}
    -\psi_{i,j}(u_{i,j}^2+v_{i,j}^2)-\frac{p_{i,j}^2+q_{i,j}^2}{2}-\frac{1}{2}\varphi_{i,j}^2\right).
  \end{aligned}
\end{equation}

\end{exm}

%%%%%%%%%%%%%%%%%%%%%%%%%%%%%%%%%%%%%%%%%%%%%%%%%%%%%%%%%%%%%%%%%%%%%%%%%%%%%%%%%%%%%%%%%%%
\section{Conclusions}
We established the theory of  difference variational bicomplex, standing as
 a geometric framework for the study of finite difference equations, particularly of
variational problems, and their symmetries, conservation laws, etc. Using the bicomplex, we re-examined the equations of motion arising from discrete mechanics, and in particular found an equivalent condition for the existence of Hamiltonians for such equations.  It is also a natural structure for
investigating multisymplectic systems of P$\Delta$Es and serves as a proper theoretic foundation for  multisymplectic integrators. Finite difference conservation laws of discrete multisymplectic systems could be obtained either through Noether's first theorem, which was proved in a coordinate-free manner on the bicomplex, or by using difference multimomentum maps corresponding to groups of vertical transformations. 
Furthermore, we were able to analyze multisymplectic integrators by generalizing the difference variational bicomplex from uniform lattices to non-uniform meshes, for instance, their discrete multisymplectic structures and variational structures. Various illustrative examples are provided.
%%%%%%%%%%%%%%%%%%%%%%%%%%%%%%%%%%%%%%%%%%%%%%%%%%%%%%%%%%%%%%%%%%%%%%%%%%%%%%%%%%%%%%%%%%%
\section*{Acknowledgement} 
LP would like to thank Timothy Grant, Robert Gray and Jing Ping Wang for valuable discussions.  The authors would like to thank the Isaac Newton Institute for Mathematical Sciences for support and hospitality during the programme {\it `Geometry, Compatibility and Structure Preservation in Computational Differential Equations'}, when some work on this paper was undertaken. This work was partially supported by JSPS KAKENHI grant number JP20K14365, JST-CREST grant number JPMJCR1914,  the Fukuzawa Fund and KLL of Keio University, and EPSRC grant number EP/R014604/1.
%%%%%%%%%%%%%%%%%%%%%%%%%%%%%%%%%%%%%%%%%%%%%%%%%%%%%%%%%%%%%%%%%%%%%%%%%%%%%%%%%%%%%%%%%%%
\begin{thebibliography}{9}


\bibitem{An1992} I. M. Anderson, Introduction to the Variational Bicomplex, In M. Gotay, J.E. Marsden, V. E. Moncrief (eds.), {\it Mathematical Aspects of Classical Field Theory}, Providence, RI: AMS Publications, 1992 (pp. 51--73).

\bibitem{An1989} I. M. Anderson, {\it The Variational Bicomplex}, book manuscript, Utah State University, 1989.
            
 \bibitem{BCA2010} G. Bluman, A. Cheviakov and S. Anco, {\it Applications of Symmetry Methods to Partial Differential Equations}, New York: Springer, 2010.
 
%\bibitem{BlKu1989} G. W. Bluman and S. Kumei, {\it Symmetries and Differential Equations}, New York: Springer-Verlag, 1989.

\bibitem{Br1997a} T. J. Bridges, A geometric formulation of the conservation of wave action and its implications for signature and the classification of instabilities, {\it Proc. Roy. Soc. Lond. A} {\bf 453} (1997), 1365--1395.

\bibitem{Br1997b} T. J. Bridges, Multi-symplectic structures and wave propagation, {\it Math. Proc. Camb. Phil. Soc.} {\bf 121} (1997), 147--190.

\bibitem{Br2017} T. J. Bridges, {\it Symmetry, Phase Modulation and Nonlinear Waves}, Cambridge: Cambridge University Press, 2017.

%\bibitem{BrDe2001} T. J. Bridges and G. Derks, The symplectic Evans matrix, and the instability of solitary waves and fronts, {\it Arch. Rat. Mech. Anal.} {\bf 156} (2001), 1--87.

\bibitem{BrHyLa2010} T. J. Bridges, P. E. Hydon and J. K. Lawson, Multisymplectic structures and the variational bicomplex, {\it Math. Proc. Camb. Phil. Soc.} {\bf 148} (2010), 159--178.

\bibitem{BrRe2001} T. J. Bridges and S. Reich, Multi-symplectic integrators: numerical schemes for Hamiltonian PDEs that conserve symplecticity, {\it Phys. Lett. A} {\bf 284} (2001), 184--193.

\bibitem{CaIbDe1999} F. Cantrijn, A. Ibort and M. de Le\'{o}n, On the geometry of multisymplectic manifolds, {\it J. Austral. Math. Soc. (Ser. A)} {\bf 66} (1999), 303--330.

\bibitem{CoHoHy2007} C. J. Cotter, D. D. Holm and P. E. Hydon, Multisymplectic formulation of fluid dynamics using the inverse map, {\it Proc. Roy. Soc. Lond. A} {\bf 463} (2007), 2671--2687.

\bibitem{CramPir1987}  M. Crampin and F. A. E. Pirani, \textit{Applicable Differential Geometry}, Cambridge: Cambridge University Press, 1987.

\bibitem{Do2001} V. Dorodnitsyn, Noether-type theorems for difference equations, {\it Appl. Numer. Math.} {\bf 39} (2001), 307--321.

\bibitem{Do2010} V. Dorodnitsyn, {\it Applications of Lie Groups to Difference Equations}, Chapman \& Hall/CRC, 2010.

\bibitem{Gotay} M. J. Gotay, A multisymplectic framework for classical field theory and the calculus of variations. I: Covariant Hamiltonian formalism. In M. Francaviglia (ed.), \textit{Mechanics, Analysis and Geometry: 200 Years After Lagrange}, North-Holland, 1991 (pp. 203--235).

\bibitem{Hy2005} P. E. Hydon, Multisymplectic conservation laws for differential and differential-difference equations, {\it Proc. Roy. Soc. Lond. A} {\bf 461} (2005), 1627--1637.

\bibitem{Hy2014} P. E. Hydon, {\it Difference Equations by Differential Equation Methods}, Cambridge: Cambridge University Press, 2014.

\bibitem{Hy2000b} P. E. Hydon, {\it Symmetry Methods for Differential Equations: A Beginner's Guide}, Cambridge: Cambridge University Press, 2000.

\bibitem{HyMa2004} P. E. Hydon and E. L. Mansfield, A variational complex for difference equations, {\it Found. Comput. Math.} {\bf 4} (2004), 187--217.
              
\bibitem{HuDe2008} W. P. Hu and Z. C. Deng, Multi-symplectic method for generalized fifth-order KdV equation, {\it Chin. Phys. B} {\bf 17} (2008), 3923--3929.

\bibitem{Khavkine} I. Khavkine, Presymplectic current and the inverse problem of the calculus of variations, \textit{J. Math. Phys.} {\bf 54} (2013), 111502.

\bibitem{KO2003} I. A. Kogan and P. J. Olver, Invariant Euler--Lagrange equations and the invariant variational bicomplex, {\it Acta Appl. Math.} {\bf 76} (2003), 137--193.

\bibitem{Ko2011} Y. Kosmann-Schwarzbach, {\it The Noether Theorems: Invariance and Conservation Laws in the Twentieth Century}, New York: Springer-Verlag, 2011.

\bibitem{KrVi1999} I. S. Krasil'shchik and A. M. Vinogradov (eds.), {\it Symmetries and Conservation Laws for Differential Equations of Mathematical Physics}, Providence, RI: AMS Publications, 1999.

\bibitem{Ku1985} B. Kupershmidt, {\it Discrete Lax Equations and Differential-Difference Calculus}, Paris: Ast\'{e}risque 123, SMF, 1985.

\bibitem{La2000} J. K. Lawson, A frame-bundle generalization of multisymplectic geometry, {\it Rep. Math. Phys.} {\bf 45} (2000), 183--205.
              
\bibitem{La2004} J. K. Lawson, A frame-bundle generalization of multisymplectic momentum mappings, {\it Rep. Math. Phys.} {\bf 53} (2004), 19--37.
              
\bibitem{LeRe2004} B. Leimkuhler and S. Reich, {\it Simulating Hamiltonian Dynamics}, Cambridge: Cambridge University Press, 2004.

\bibitem{LeTrWi2000} D. Levi, S. Tremblay and P. Winternitz, Lie point symmetries of difference equations and lattices, {\it J. Phys. A.} {\bf 33} (2000), 8507--8523.

\bibitem{LN2009} S. Lobb and F. Nijhoff, Lagrangian multiforms and multidimensional consistency, {\it J. Phys. A: Math. Theor.} {\bf 42}  (2009), 454013.

\bibitem{MaHy2008} E. L. Mansfield and P. E. Hydon, Difference forms, {\it Found. Comput. Math.} {\bf 8} (2008), 427--467.

\bibitem{MaRoHyPe2019} E. L. Mansfield, A. Rojo-Echebur\'ua, P. E. Hydon and L. Peng, Moving frames and Noether's finite difference conservation laws I, {\it Transactions of Mathematics and its Applications} \textbf{3} (2019), tnz004. %doi:10.1093/imatrm/tnz004.

\bibitem{MaPaSh1998} J. E. Marsden, G. W. Patrick and S. Shkoller, Multisymplectic geometry, variational integrators, and nonlinear PDEs, {\it Comm. Math. Phys.} {\bf 199} (1998), 351--395.
               
\bibitem{MaWe2001} J. E. Marsden and M. West, Discrete mechanics and variational integrators, {\it Acta Numer.} {\bf 10} (2001), 357--514.

%\bibitem{Ma2002} D. Martin, {\it Manifold Theory: An Introduction for Mathematical Physicists}, Chichester: Horwood Publishing, 2002.

\bibitem{Ol1993} P. J. Olver, {\it Applications of Lie Groups to Differential Equations, (2nd edn)}, New York: Springer-Verlag, 1993.

\bibitem{Pe2013} L. Peng, {\it From Differential to Difference: The Variational Bicomplex and Invariant Noether's Theorems}, PhD Thesis, University of Surrey, 2013.

%\bibitem{Pe2015} L. Peng, Self-adjointness and conservation laws of difference equations, {\it Commun. Nonlinear Sci. Numer. Simul.} {\bf 23} (2015), 209--219.

\bibitem{Pe2017} L. Peng, Symmetries, conservation laws, and Noether's theorem for differential-difference equations, {\it Stud. Appl. Math.} {\bf 139} (2017), 457--502.

%\bibitem{Peng2022} L. Peng, The difference variational bicomplex and discrete integrable systems, New Trends in Lagrangian and Hamiltonian Aspects of Integrable Systems, Leeds University, Leeds, UK, May 13--14, 2022

\bibitem{PH2022} L. Peng and P. E. Hydon, Transformations, symmetries and Noether theorems for differential-difference equations, {\it Proc. R. Soc. A} {\bf 478} (2022), 20210944.

\bibitem{RaHy2007} O. G. Rasin and P. E. Hydon, Conservation laws for integrable difference equations, {\it J. Phys. A: Math. Theor.} {\bf 40} (2007), 12763--12773.

%\bibitem{RaHy2005} O. G. Rasin and P. E, Hydon, Conservation laws of discrete Korteweg--de Vries equation, {\it SIGMA} {\bf 1} (2005), 026.

\bibitem{SNC2020} D. Sleigh, F. Nijhoff and V. Caudrelier, Variational symmetries and Lagrangian multiforms, {\it Lett. Math. Phys.} {\bf 110} (2020), 805--826.

\bibitem{SuQi2003} J. Q. Sun and M. Z. Qin, Multi-symplectic methods for the coupled 1D nonlinear Schr\"{o}dinger system, {\it Comput. Phys. Commun.} {\bf 155} (2003), 221--235.

\bibitem{SV2016} Yu. B. Suris and M. Vermeeren, On the Lagrangian structure of integrable hierarchies, In A. I. Bobenko (ed.), {\it Advances in Discrete Differential Geometry}, Berlin: Springer, 2016 (pp. 347--378).

\bibitem{Ts1982} T. Tsujishita, On variation bicomplexes associated to differential equations, {\it Osaka J. Math.} {\bf 19} (1982), 311--363.

\bibitem{Vi2001} A. M. Vinogradov, {\it  Cohomological Analysis of Partial Differential Equations and Secondary Calculus}, Providence, RI: AMS Publications, 2001.

\bibitem{Vi1984} A. M. Vinogradov, The $\mathcal{C}$-spectral sequence, Lagrangian formalism and conservation laws I and II, {\it J. Math. Anal. Appl.} {\bf 100} (1984), 1--129.

\bibitem{Wa2008} J. Wang, Multisymplectic integrator of the Zakharov system, {\it Chin. Phys. Lett.} {\bf 25} (2008), 3531--3534.
               
\bibitem{Xe2018} P.  Xenitidis, Determining the symmetries of difference equations, {\it Proc. R. Soc. A} {\bf 747} (2018),  20180340.

\bibitem{Zuck} G. J. Zuckerman, Action principles and global geometry, In S. T. Yau (ed.), {\it Mathematical Aspects of String Theory}, World Scientific, 1987 (pp. 259--284). 

\bibitem{Zh2010} V. V. Zharinov, A differential-difference bicomplex, {\it Theor. Math. Phys.} {\bf 165} (2010), 1401--1420.

\end{thebibliography}

\clearpage

\appendix


\section*{Appendix A: Exactness of the augmented difference variational bicomplex}\renewcommand{\thesection}{A} 

Exactness of the augmented  difference variational bicomplex has been partially  proved in
\cite{HyMa2004,Ku1985}; the following proofs are based on the thesis \cite{Pe2013} by  constructing homotopy operators.  See also \cite{Zh2010} for Zharinov's construction. Local exactness of the augmented difference variational bicomplex in Fig. \ref{fig:mdvb} on
$\mathbb{Z}^p\times P_{\mbn}(\mathbb{R}^q)$ is illustrated by the
following three theorems.

\begin{thm}
For each $k=0,1,2,\ldots,p$, the vertical complex
\begin{equation*}
\Omega^{k,0}\stackrel{\dv }{\longrightarrow}
\Omega^{k,1}\stackrel{\dv }{\longrightarrow}
\Omega^{k,2}\stackrel{\dv }{\longrightarrow}
\cdots
\end{equation*}
is exact.
\end{thm}

\begin{proof}
Denote the prolongation of the vertical vector field
$u^{\alpha}\frac{\partial}{\partial u^{\alpha}}$ as
\begin{equation}
\mbv={(\es_{\mbJ}u^{\alpha})\frac{\partial}{\partial
u^{\alpha}_{\mbJ}}} ={u^{\alpha}_{\mbJ}\frac{\partial}{\partial
u^{\alpha}_{\mbJ}}},
\end{equation}
the flow generated by which on $\mathbb{Z}^p\times P_{\mbn}(\mathbb{R}^q)$ is a
one-parameter family of diffeomorphisms
\begin{equation}
\exp(\varepsilon \mbv)(\mbn,\mbu)=(\mbn,\ldots,e^{\varepsilon}u^{\alpha}_{\mbJ},\ldots),
\end{equation}
satisfying that\footnote{ The mapping $\exp$ is  called the
exponential mapping, and we claim that it is a diffeomorphism here
as the discrete parts can be considered as fixed parameters.}
\begin{equation}
\frac{\upd}{\upd\!\varepsilon}\Big|_{\varepsilon=0}\exp(\varepsilon
\mbv)(\mbn,\mbu)=\mbv.
\end{equation}
 As a consequence, naturally there exist induced push-forward and
pull-back mappings. Take any form $\sigma\in\Omega^{k,l}$. As the
transformation only occurs in the continuous parts, we can define
a derivative of $\sigma$ as\footnote{ In the continuous setting, this derivative
is exactly the Lie derivative, and we also refer to its notation and
call it the Lie derivative accordingly. It satisfies all the
properties that the canonical Lie derivative owns.}
\begin{equation}\label{lieddiscrete}
\mcL_{\mbv}\sigma=\lim_{\varepsilon\rightarrow0}\frac{\exp(\varepsilon \mbv)^{\ast}(\sigma)-\sigma}{\varepsilon}=\frac{\upd}{\upd\!\varepsilon}\Big|_{\varepsilon=0}
{\exp(\varepsilon \mbv)^{\ast}(\sigma)},
\end{equation}
which implies that
\begin{equation}
\frac{\upd}{\upd\!\varepsilon}{\exp(\varepsilon \mbv)^{\ast}(\sigma)}=
\exp(\varepsilon \mbv)^{\ast}(\mcL_\mbv\sigma).
\end{equation}

The
vertical homotopy operators
$h_{\upv}^{k,l}:\Omega^{k,l}\rightarrow\Omega^{k,l-1},l\geq1$ are defined by
\begin{equation}\label{eq:vehomo}
h^{k,l}_{\upv}(\sigma)=\int_0^1{\frac{1}{\lambda}\exp(\ln\lambda
\mbv)^{\ast}(\mbv\intp\sigma) \upd\!\lambda},
\end{equation}
such that
\begin{equation}
\begin{aligned}
\sigma=\dv (h_{\upv}^{k,l}(\sigma))
+h_{\upv}^{k,l+1}(\dv \!\sigma).
\end{aligned}
\end{equation}
The integrand in \eqref{eq:vehomo} is a smooth function of
$\lambda$ at $\lambda=0$, as $l\geq1$. This finishes the proof.
\end{proof}

\begin{thm}
For each $l\geq1$, the augmented horizontal complex
\begin{equation}
0\rightarrow
\Omega^{0,l}\stackrel{\dDh}{\longrightarrow}
\Omega^{1,l}\stackrel{\dDh}{\longrightarrow}
\cdots\stackrel{\dDh}{\longrightarrow}
\Omega^{p-1,l}\stackrel{\dDh}{\longrightarrow}
\Omega^{p,l}\stackrel{\mcI^{\vartriangle}}{\longrightarrow}
\mathscr{F}^l\rightarrow0 
\end{equation}
is exact.
\end{thm}

\begin{proof}
First of all, notice that $\ker
(\mcI^{\vartriangle})=\operatorname{im}
(\dDh)$, which is an
immediate consequence of \eqref{eq:ii}. The exactness of the
last part is implied by
$\mathscr{F}^l=\mcI^{\vartriangle}(\Omega^{p,l})$.

For any $\sigma\in\Omega^{k,l}$, define the following operators 
\begin{equation}
F_{\alpha}^{\mbJ}(\sigma)=\sum_{\mbI\supset \mbJ}\begin{pmatrix}
\mbI\\
\mbJ
\end{pmatrix}\es_{-\mbI}\left(\partial_{u_{\mbI}^{\alpha}}\intp\sigma\right).
\end{equation}
Here if we denote $\mbI=(i^1,i^2,\ldots,i^p)$ and $\mbJ=(j^1,j^2,\ldots,j^p)$,
then $\mbI\supset \mbJ$ means that $i^k\geq j^k$ for each $k$, and
\begin{equation}
\begin{pmatrix}
\mbI\\
\mbJ
\end{pmatrix}=\frac{\mbI!}{\mbJ!(\mbI- \mbJ)!}
\end{equation}
 with $\mbI!=i^1!i^2!\ldots i^p!$. 
For any $\sigma\in\Omega^{k,l}$ with $1\leq k\leq p$, we define
the horizontal homotopy operators as
\begin{equation}
h_{\uph}^{k,l}(\sigma)=\frac{1}{l}\sum_{\alpha,m,\mbI}\frac{|i^m|+1}{p-k+|\mbI|+1}(S-\id)_{\mbI}
\left(\dv \!u^{\alpha}\wedge
F_{\alpha}^{\mbI+\mathbf{1}_m}(\partial_{n^m}\intp\sigma)\right),
\end{equation}
which satisfy that
\begin{equation}\label{eq:hohomo}
h_{\uph}^{k+1,l}(\dDh\!\sigma)
+\dDh\left(h_{\uph}^{k,l}(\sigma)\right)=\sigma.
\end{equation} 
When $k=0$, by defining $\Omega^{-1,l}=0$, i.e.,
$\partial_{n^m}\intp\sigma=0$ for each $m$,  the identity \eqref{eq:hohomo} still holds.
\end{proof}


\begin{thm}
The boundary complex\footnote{Note that Kupershmidt in \cite{Ku1985} proved the exactness around $\mcE^{\vartriangle}$. In \cite{HyMa2004}, Hydon \& Mansfield proved the exactness in the difference variational complex analogue. }
\begin{equation}
0\rightarrow\mathbb{R}\rightarrow
\Omega^{0,0}\stackrel{\dDh}{\longrightarrow}
\Omega^{1,0}\stackrel{\dDh}{\longrightarrow}
\cdots\stackrel{\dDh}{\longrightarrow}
\Omega^{p-1,0}\stackrel{\dDh}{\longrightarrow}
\Omega^{p,0}\stackrel{\mcE^{\vartriangle}}{\longrightarrow}
\mathscr{F}^1\stackrel{\delta^{\vartriangle}_{\upv}}{\longrightarrow}
\mathscr{F}^2\stackrel{\delta^{\vartriangle}_{\upv}}{\longrightarrow}
\cdots
\end{equation}
is exact.
\end{thm}



\begin{proof}

Let us break the complex into two pieces
\begin{equation}
0\rightarrow\mathbb{R}\rightarrow
\Omega^{0,0}\stackrel{\dDh}{\longrightarrow}
\Omega^{1,0}\stackrel{\dDh}{\longrightarrow}
\cdots\stackrel{\dDh}{\longrightarrow}
\Omega^{p-1,0}\stackrel{\dDh}{\longrightarrow}
\Omega^{p,0}\stackrel{\mcE^{\vartriangle}}{\longrightarrow}
\mathscr{F}^1 
\end{equation}
and
\begin{equation}
\Omega^{p,0}\stackrel{\mcE^{\vartriangle}}{\longrightarrow}
\mathscr{F}^1\stackrel{\delta^{\vartriangle}_{\upv}}{\longrightarrow}
\mathscr{F}^2\stackrel{\delta^{\vartriangle}_{\upv}}{\longrightarrow}
\cdots. 
\end{equation}

We only show the exactness of the second piece here. For any
$\sigma\in\mathscr{F}^l,l\geq1$, the vertical homotopy
operators give that 
\begin{equation}\label{eq:vehomoeq}
\sigma=\dv (h_{\upv}^{p,l}(\sigma))
+h_{\upv}^{p,l+1}(\dv\! \sigma).
\end{equation}
From \eqref{eq:ii}, we have
\begin{equation}
\dv \!\sigma=\delta^{\vartriangle}_{\upv}\sigma
+\dDh\!\tau_1
\end{equation}
and
\begin{equation}
\dv (h_{\upv}^{p,l}(\sigma))=\delta^{\vartriangle}_{\upv}
(h_{\upv}^{p,l}(\sigma))+\dDh\!\tau_2,
\end{equation}
for some $\tau_1\in\Omega^{p-1,l+1}$ and $\tau_2\in\Omega^{p-1,l}$.
Since $\mcI^{\vartriangle}(\sigma)=\sigma$, we apply the
difference interior Euler operator $\mcI^{\vartriangle}$ to
equality \eqref{eq:vehomoeq} and obtain 
\begin{equation}\label{eq:Euleromega}
\begin{aligned}
\sigma&=\mcI^{\vartriangle}\left(\delta^{\vartriangle}_{\upv}
(h_{\upv}^{p,l}(\sigma))+\dDh\!\tau_2\right)+
\mcI^{\vartriangle}\circ
h_{\upv}^{p,l+1}\left(\delta^{\vartriangle}_{\upv}\sigma
+\dDh\!\tau_1\right)\\
&=\mcI^{\vartriangle}\circ\delta^{\vartriangle}_{\upv}\circ
h_{\upv}^{p,l}(\sigma) +\mcI^{\vartriangle}\circ
h_{\upv}^{p,l+1}\circ\delta^{\vartriangle}_{\upv}(\sigma)
+\mcI^{\vartriangle}\circ h_{\upv}^{p,l+1}(\dDh\!\tau_2)\\
&=\delta^{\vartriangle}_{\upv}\circ
h_{\upv}^{p,l}(\sigma) +\mcI^{\vartriangle}\circ
h_{\upv}^{p,l+1}\circ\delta^{\vartriangle}_{\upv}(\sigma)
-\mcI^{\vartriangle}\left(\dDh\left( h_{\upv}^{p,l+1}(\tau_2)\right)\right)\\
&=\delta^{\vartriangle}_{\upv}\circ
h_{\upv}^{p,l}(\sigma) +\mcI^{\vartriangle}\circ
h_{\upv}^{p,l+1}\circ\delta^{\vartriangle}_{\upv}(\sigma),
\end{aligned}
\end{equation}
where the third equality holds as
$\dDh$ anti-commutes
with $h_{\upv}^{p,l}$ as consequence of Proposition
\ref{prop:symmetry}. Note that when the operator
$\delta^{\vartriangle}_{\upv}$ is applied to
$(p,0)$-forms, it should be replaced by the difference
Euler--Lagrange operator $\mcE^{\vartriangle}$. To continue
 our proof, the equality \eqref{eq:Euleromega} should be separately
considered through the following two
cases.\\
$i)$ Let $l=1$. Recall that $\delta^{\vartriangle}_{\upv}\mcE^{\vartriangle}=0$, and for any $(p,1)$-form $\sigma$ satisfying
$\delta^{\vartriangle}_{\upv}(\sigma)=0$, \eqref{eq:Euleromega}
leads to that
\begin{equation}
\sigma=\mcE^{\vartriangle}\circ
h_{\upv}^{p,1}(\sigma),
\end{equation}
which implies the exactness. Namely, there exists a $(p,0)$-form
$\tau=h_{\upv}^{p,1}(\sigma)$,
such that $\sigma=\mcE^{\vartriangle}(\tau)$.\\
$ii)$ Let $l\geq2$.  We define homotopy operators to verify the exactness. The
following property is needed that for any $(p,m)$-form $\sigma$,
with $m\geq1$,
\begin{equation}
\delta^{\vartriangle}_{\upv}(\sigma)=
\delta^{\vartriangle}_{\upv}\mcI^{\vartriangle}(\sigma),
\end{equation}
which is an immediate result by applying
$\delta^{\vartriangle}_{\upv}$ to the equality
$\sigma=\mcI^{\vartriangle}(\sigma)+\dDh\!\tau$,
for some $\tau\in\Omega^{p-1,m}$. Therefore, for any
$\sigma\in\mathscr{F}^l,l\geq2$,
$h_{\upv}^{p,l}(\sigma)$ is a $(p,m)$-form with
$m=l-1\geq1$. Now the equality \eqref{eq:Euleromega} becomes 
\begin{equation}
\begin{aligned}
\sigma&=\delta^{\vartriangle}_{\upv}\circ
h_{\upv}^{p,l}(\sigma)
+\mcI^{\vartriangle}\circ h_{\upv}^{p,l+1}\circ\delta^{\vartriangle}_{\upv}(\sigma)\\
&=\delta^{\vartriangle}_{\upv}\circ\mcI^{\vartriangle}\circ
h_{\upv}^{p,l}(\sigma)
+\mcI^{\vartriangle}\circ h_{\upv}^{p,l+1}\circ\delta^{\vartriangle}_{\upv}(\sigma)\\
&=\delta^{\vartriangle}_{\upv}\left(\mcH^l(\sigma)\right)
+\mcH^{l+1}\left(\delta^{\vartriangle}_{\upv}(\sigma)\right),
\end{aligned}
\end{equation}
where the homotopy operators
$\mcH^l:\mathscr{F}^{l}\rightarrow\mathscr{F}^{l-1},l\geq2$
are defined by $\mcH^l=\mcI^{\vartriangle}\circ
h_{\upv}^{p,l}$. This finishes the proof of exactness
for the second piece.
\end{proof}

%%%%%%%%%%%%%%%%%%%%%%%%%%%%%%%%%%%%%%%%%%%%%%%%%%%%%%%%%%%%%%%%%%%%%%%%%%%%%%%%

\end{document}
