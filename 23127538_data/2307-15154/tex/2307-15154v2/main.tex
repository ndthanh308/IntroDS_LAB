\documentclass[10pt,fullpage,letterpaper]{article}

\oddsidemargin 0in
\evensidemargin 0in
\textwidth 6.5in
\topmargin -0.5in
\textheight 9.0in

\usepackage[utf8]{inputenc} % allow utf-8 input
\usepackage[T1]{fontenc}    % use 8-bit T1 fonts
\usepackage{hyperref}       % hyperlinks
\usepackage{url}            % simple URL typesetting
\usepackage{booktabs}       % professional-quality tables
\usepackage{amsfonts}       % blackboard math symbols
\usepackage{nicefrac}       % compact symbols for 1/2, etc.
\usepackage{microtype}      % microtypography
\usepackage{xcolor}         % colors
% \usepackage[linesnumbered,ruled,vlined]{algorithm2e}
\usepackage{algorithm}
\usepackage{algorithmic}
\usepackage{enumerate}
\usepackage{amsmath, amssymb}
\usepackage{amsthm}
\usepackage{mathrsfs}
\usepackage{bm}
\usepackage{graphicx}
\usepackage{float}
\usepackage{array}
\usepackage{dsfont}
\usepackage{comment}
\usepackage{natbib}
\usepackage{anyfontsize}
\usepackage{todonotes}
\usepackage{wrapfig}
\usepackage{caption}
\usepackage{subcaption}
\usepackage{tikz}
\usepackage{pstricks}
\allowdisplaybreaks

\usepackage{thmtools}
\usepackage{thm-restate}

\hypersetup{
    colorlinks=true,
    linkcolor=blue,
    filecolor=magenta,      
    urlcolor=blue,
}

\DeclareMathOperator*{\argmin}{argmin}
\DeclareMathOperator*{\argmax}{argmax}
\DeclareMathOperator*{\arginf}{arginf}
\DeclareMathOperator*{\argsup}{argsup}

\def\1{\mathds{1}}
\def\E{\mathbb{E}}
\def\P{\mathbb{P}}
\def\R{\mathbb{R}}
\def\X{\mathcal{X}}
\newcommand{\htheta}{\widehat{\theta}}
\newcommand{\otheta}{\overline{\theta}}
\newcommand{\hatone}{\widehat{(1)}}
\newcommand{\hthetat}{\widehat{\theta}^{(t)}}
\newcommand{\hdeltat}{\widehat{\Delta}^{(t)}}
\newcommand{\alt}{\mathrm{alt}}
\newcommand{\allthetat}{\Bp{\theta_t}_{t=1}^{T}}

\newcommand{\kevin}[1]{{\color{red}{Kevin: #1}}}
\newcommand{\romain}[1]{\textcolor{blue}{Romain: #1}}
\newcommand{\maryam}[1]{\textcolor{magenta}{Maryam: #1}}

\newcommand{\mc}[1]{\mathcal{#1}}
\newcommand{\Sp}[1]{\left(#1\right)}
\newcommand{\Mp}[1]{\left[#1\right]}
\newcommand{\Bp}[1]{\left\{#1\right\}}
\newcommand{\abs}[1]{\left|#1\right|}
\newcommand{\Norm}[1]{\left\|#1\right\|}
\newcommand{\ve}[1]{\mathbf{#1}}
\newcommand{\floor}[1]{\left\lfloor#1\right\rfloor}
\newcommand{\ceil}[1]{\left\lceil#1\right\rceil}
\newcommand{\inner}[1]{\left\langle#1\right\rangle}
\newcommand{\matenv}[1]{\left[\begin{matrix}#1\end{matrix}\right]}
\newcommand{\zhihan}[1]{{\color{red} Zhihan: #1}}
% \newcommand{\romain}[1]{{\color{pink} Romain: #1}}
\newcommand{\todol}[2][]{\todo[color=red!20,size=\tiny, #1]{L: #2}} 



\newtheorem{lemma}{Lemma}
\newtheorem{corollary}{Corollary}
\newtheorem{theorem}{Theorem}
\newtheorem{proposition}{Proposition}
% \newtheorem{conjecture}{Conjecture}
% \newtheorem{assumption}{Assumption}
\theoremstyle{definition}
\newtheorem{definition}{Definition}

\theoremstyle{remark}
\newtheorem{remark}{Remark}

\makeatletter
\newcommand{\IfTwoColumnElse}[2]{%
    \if@twocolumn
        #1 % Code for two-column layout
    \else
        #2 % Code for one-column layout
    \fi
}
\makeatother


\title{A/B Testing and Best-arm Identification for Linear Bandits with Robustness to Non-stationarity}
\author{Zhihan Xiong\footnote{Equal contribution.}\\ \url{zhihanx@cs.washington.edu} \and Romain Camilleri\footnotemark[\value{footnote}] \\ \url{camilr@cs.washington.edu} \and Maryam Fazel \\ \url{mfazel@uw.edu} \and Lalit Jain \\\url{lalitj@uw.edu} \and Kevin Jamieson \\ \url{jamieson@cs.washington.edu}}
\date{}



\begin{document}
\maketitle

\renewcommand*{\thefootnote}{\arabic{footnote}}

\begin{abstract}
    \begin{abstract}

The Fast Reciprocal Square Root Algorithm is a well-established approximation technique consisting of two stages: first, a coarse approximation is obtained by manipulating the bit pattern of the floating point argument using integer instructions, and second, the coarse result is refined through one or more steps, traditionally using Newtonian iteration but alternatively using improved expressions with carefully chosen numerical constants found by other authors. The algorithm was widely used before microprocessors carried built-in hardware support for computing reciprocal square roots. At the time of writing, however, there is in general no hardware acceleration for computing other fixed fractional powers. This paper generalises the algorithm to cater to all rational powers, and to support any polynomial degree(s) in the refinement step(s), and under the assumption of unlimited floating point precision provides a procedure which automatically constructs provably optimal constants in all of these cases. It is also shown that, under certain assumptions, the use of monic refinement polynomials yields results which are much better placed with respect to the cost/accuracy tradeoff than those obtained using general polynomials. Further extensions are also analysed, and several new best approximations are given.

\end{abstract}

\end{abstract}

\setcounter{footnote}{0} 
\section{Introduction}
Current quantum hardware is unable to carry out universal quantum computations due to the buildup of errors that occur during the computation. 
The magnitude of the individual error is currently above the value that the Threshold Theorem requires in order to kick-start quantum error correction and fault-tolerant quantum computation~\cite[Section 10.6]{nielsen_chuang_2010}. 
Although the experimentally achieved fidelity rates are promising and the error bounds are inching closer to the required threshold, we will have to work for the foreseeable future with quantum hardware with errors that build-up during the computation.  This implies that we can only do a limited number of steps before the output of the computation has become completely uncorrelated with the intended one.

For fault-tolerant quantum computing, we repeat four steps: 
1) We apply a number of single and two-qubit quantum gates, in parallel whenever possible; 
2) We perform a syndrome measurement on a subset of the qubits; 
3) We perform fast classical computations to determine which errors have occurred and how to correct them; 
and, 4) We apply correction terms based on the classical computations.
We then repeat these four steps with a next sequence of gates. 
These four steps are essential to fault-tolerant quantum computing. 


The starting point of this work is to use the four steps outlined above, not to carry out error correction and fault-tolerant computation, but to enhance short, constant-depth, {\em uncorrected} quantum circuits that perform single qubit gates and {\em nearest-neighbor} two qubit gates. 
Since in the long run we will have to implement error-correction and fault-tolerant computation anyhow, and this is done by such a four-step process, why not make other use of this architecture? Moreover, on some of the quantum hardware platforms, these operations are already in place.
Embracing this idea we naturally arrive at the question: what is the computational power of \textit{low-depth} quantum-classical circuits organized as in the four steps outlined above? 
We thus investigate circuits that execute a small, ideally constant, number of stages, where at each stage we may apply, in parallel, single qubit gates and {\em nearest-neighbor} two qubit gates, followed by measurements, followed by low-depth classical computations of which the outcome can control quantum gates in later stages. 
It is not clear, at first, whether such circuits, especially with constant depth, can do anything remotely useful. 
But we will see that this is indeed the case: many quantum computations can be done by such circuits in constant depth. 
By parallelizing quantum computations in this way, we improve the overall computational capabilities of these circuits, as we do not incur errors on qubits that are idle, simply because qubits are not idle for a very long time. 
Furthermore, reducing the depth of quantum circuits, at the cost of increasing width, allows the circuit to be run faster even if errors occur.

The first usage of such a four-step layout, not to do error correction, but to perform computations, can be found in the paradigm of measurement-based quantum computing~\cite{gottesman1999demonstrating,raussendorf2001one,jozsa2006introduction,clark2007generalised}: 
A universal form of quantum computing where a quantum state is prepared and operations are performed by measuring qubits in different bases, depending on previous measurements and intermediate measurements.

\citeauthor{PhamSvore2013} were the first to formalize the four-step protocol for performing computations~\cite{PhamSvore2013}. They included specific hardware topologies by considering two-dimensional graphs for imposing constraints on qubit interactions. In their model, they develop circuits for particularly useful multi-qubit gates, including specifying costs in the width, number of qubits, depth, number of concurrent time steps, size, and total number of non-Identity operations.
As a result, they find an algorithm that factors integers in polylogarithmic depth.
\citeauthor{Browne:2011} showed that the main tool in the work by \citeauthor{PhamSvore2013}, the fan-out gate, can also be replaced by additional log-depth classical computations in the measurement-based quantum computing setting~\cite{Browne:2011}.

More recently, \citeauthor{Cirac:2021} introduced a scheme to implement unitary operations involving quantum circuits combined with Local Operations and Classical Communication ($\mathsf{LOCC}$) channels: $\mathsf{LOCC}$-assisted quantum circuits~\cite{Cirac:2021}. Similarly to the four-step scheme we just described, they allow for a short depth circuit to be run on the qubits, followed by one round of $\mathsf{LOCC}$, in which ancilla qubits are measured and local unitaries are applied based on the measurement outcomes. They show that in this model any 1D transitionally invariant matrix-product state (MPS) with fixed bond dimension is in the same phase of matter as the trivial state. Similar ideas can be found in~\cite{TVV_NonAbelianTopologicalOrder_2022, tantivasadakarn2021long}.

In this work, we introduce a new model, called \textit{Local Alternating Quantum-Classical Computations} ($\LAQCC$). In this model we alternate between running quantum circuits (constrained by locality), ending in the measurement of a subset of qubits, and fast classical computations based on the measurement results. The outcome of the classical computations are then used to control future quantum circuits. We allow for flexibility in this model, by giving different constraints to the power of both the quantum circuits and the classical circuits as well as the number of alternations between them. 
Most attention will be given to $\LAQCC$ containing quantum circuits of constant depth, classical circuits of logarithmic depth and at most a constant number of alternations between them. 
Any circuit constructed in this model is considered to be of constant depth. 
We restrict ourselves to logarithmic depth classical computations, as this is the first natural and non-trivial extension beyond constant-depth classical computations. 
Constant-depth classical computations do however also have an equivalent constant-depth quantum implementation.

The definition of $\LAQCC$ sharpens the original definition of \citeauthor{PhamSvore2013} by adding constraints to the intermediate classical computations. This allows us to bound the power of $\LAQCC$ from above. 

The main result of \citeauthor{Cirac:2021}, that 1D translational invariant MPS with fixed bond dimension can be prepared by $\mathsf{LOCC}$-assisted circuits, relies on local symmetries of the MPS. These symmetries allow them to prepare local states (on a constant number of qubits) and glue them together by doing one round of the appropriate entangling measurement and corrections, after which they run a round of local unitaries to get the desired result. This general scheme for preparing states that exhibit an MPS description with the appropriate local symmetries requires only geometrically local unitaries and one round of measurement and corrections an therefore is accessible in $\LAQCC$. Studying different local symmetries, known as Symmetry Protected Topological (SPT) phases of matter, to find measurement-based constant depth circuits for states is a broad ongoing field of research~\cite{TVV_NonAbelianTopologicalOrder_2022, tantivasadakarn2021long, smith2023deterministic}. 
All these schemes have a $\LAQCC$ implementation.

%$\LAQCC$-circuits also exist for general schemes of preparing local states, based on the local tensors, and gluing them together using one round of entangled measurement and corrections, based on the local symmetry. 
%The main result of \citeauthor{Cirac:2021}, that 1D translational invariant MPS with fixed bond dimension can be prepared by $\mathsf{LOCC}$-assisted circuits, relies heavily on local symmetries of the MPS and as a result also has an equivalent $\LAQCC$ implementation. 
%The corrections applied after the measurement round are local unitaries depending on the local symmetries of the MPS. 

 

%This general scheme of preparing local states, based on the local tensors, and gluing it together by doing one round of entangled measurement and corrections, based on the local symmetry, is accessible in $\LAQCC$.
Note however that \citeauthor{Cirac:2021} also suggest a circuit for the $W$-state.
This circuit uses sequentially and dependent measurement-based corrections of the ancilla qubits. 
These dependent measurements translate to sequential alternations between the quantum and classical circuits and therefore increase the total depth to linear depth, exceeding the constant-depth constraints imposed by $\LAQCC$-circuits. 

We study the power of the $\LAQCC$ model with respect to state preparation, showing that even with only constant quantum-depth and logarithmic classical depth it remains possible to prepare states with long-range entanglement.
Another surprising result is that it is unlikely that $\LAQCC$ circuits are classically simulatable. We show that any instantaneous quantum polynomial-time (IQP) circuit~\cite{Bremner2010,Shepherd2009} has an $\LAQCC$ implementation.
Classical simulation of IQP circuits implies the collapse of the polynomial hierarchy to the third level, which is not believed to be true~\cite{Bremner2017}. Therefore, we expect that $\LAQCC$ circuits are unlikely to be classically simulatable. We bound the power of $\LAQCC$ by showing that it is contained in $\QNC^1$, the class of polynomial-size, log-depth circuits.

Next, we also study the power that intermediate classical calculations can add to quantum computations, by considering a new model that alternates between polynomially many polynomial-depth quantum circuits and unbounded classical computations
We study this model by doing a complexity theoretical analysis, where we draw inspiration from the notions of complexity given by \citeauthor{RosenthalYuen:2022}, \citeauthor{MetgerYuen:2023}, and \citeauthor{Aaronson:2004}.
All three complexity notions are based on the notion of state preparation, instead of more traditional definition of complexity such as the decidability of a computational problem. 
The first two consider classes based on sequences of quantum states preparable by a polynomial-sized quantum circuit, where the circuits are uniformly generated by a computational class, for instance, the class $\mathsf{PSPACE}$, which results in the complexity class $\mathsf{StatePSPACE}$~\cite{RosenthalYuen:2022,MetgerYuen:2023}.
The third notion considers a relative complexity, where the complexity is measured between two given states, and is measured by the number of gates, from a given gate-set, required to transform one state in another state~\cite{Aaronson:2004}. 
For our definition of state preparation complexity, we drop the uniformity constraint from~\cite{RosenthalYuen:2022,MetgerYuen:2023} and define a class as $\mathsf{StateX}$, which refers to states preparable by circuits of type $\mathsf{X}$. 
As an example, if $\mathsf{X} = \QNC^0$, this results in the class $\mathsf{StateQNC^0}$, which is the set of states preparable from the $\ket{0}^n$ state by poly-size constant-depth circuits. 
This notion is similar to the relative complexity from~\cite{Aaronson:2004}, where one state is the  $\ket{0}^n$ state and instead of counting the number of gates we consider the set of states preparable by a fixed number of gates. Using this notion of complexity we show that any state preparable by an $\LAQCC^*$ circuit is also preparable by a $\mathsf{PostQPoly}$ circuit, the class of circuits of polynomial depth with an additional post-selection gate. 

All Clifford circuits have a constant-depth $\LAQCC$ implementation, implying that any stabilizer state can be implemented by a constant-depth $\LAQCC$ circuit, see Section~\ref{sec:clifford_circuits} for a proof of this statement. 
Efficient circuits for stabilizer states have been known already through measurement-based quantum computing. Therefore this paper focuses on the preparation of non-stabilizer states, and as a surprising result we find novel constant-depth protocols for four very natural classes of non-stabilizer states.
Despite the extensive research into these four classes of non-stabilizer states and the many applications of them, no efficient constant- or low-depth state preparation protocols are known yet. We specifically consider these four classes as they are all often used as initial states in other algorithms.

The first state is a uniform superposition over an arbitrary number of states. 
This state finds applications in many quantum algorithms, as they often start with a uniform superposition over multiple states. 
This superposition is often achieved by applying Hadamard gates to every qubit due to its simplicity to prepare. 
Yet, the analysis of many algorithms, such as Shor's algorithm~\cite{Shor:1997}, would benefit from a different initial superposition. 
The circuit to prepare the uniform superposition over an arbitrary number of states uses an exact version of Grover search as a subroutine, that turns a probabilistic circuit, with a known constant probability of success, into a deterministic circuit. 
We use the circuit for preparing a uniform superposition over an arbitrary number of states as a subroutine in the next two quantum state preparation protocols. 

The second state is the $W$-state, the uniform superposition over all computational basis states of Hamming-weight~$1$, a natural long-ranged entangled state that displays a fundamentally nonequivalent type of entanglement from the Greenberger–Horne–Zeilinger state~\cite{WState:2000}, for which $\LAQCC$-type constant-depth circuits were previously known~\cite{PhamSvore2013, Cirac:2021}. 
The $W$-state is often used as benchmark for new quantum hardware~\cite{Haffner2005,Neeley2010,GarciaPerez:2021}. 
A novel way to prepare the $W$-state therefore gives a new way to benchmark different quantum devices with each other. 
A circuit for preparing the $W$-state was given in~\cite{Cirac:2021}, but this implementation requires sequentially alternating measurements followed by local unitaries, which in the $\LAQCC$ model is not considered to be of constant depth. 
We improve this protocol by giving an $\LAQCC$ implementation of the $W$-state, based on a compress-uncompress method that links the one-hot and binary encoding of integers.

The third state considered is the Dicke state, a generalization of the $W$-state, a superposition over all computational basis states with Hamming-weight $k$~\cite{Dicke:1954}. 
Dicke states have relevance in various practical settings.
For instance, for quantum game theory~\cite{zdemir2007}, quantum storage~\cite{Bacon_Compress:2006,Plesch:2010}, quantum error correction~\cite{ouyang2014permutation}, quantum metrology~\cite{toth2012multipartite}, and quantum networking~\cite{prevedel2009experimental}. 
Dicke states have been used as a starting state for variational optimization algorithms, most notably Quantum Alternating Operator Ansatz (QAOA)~\cite{Hadfield2019}, to find solutions to problems such as Maximum k-vertex Cover~\cite{Brandhofer2022,cook2020quantum}.
The ground states of physical Hamiltonians describing one-dimensional chains tend to show a resemblance to Dicke states such as states resulting from the Bethe ansatz, making them an ideal starting state when investigating the ground state behavior of these Hamiltonians~\cite{TDL_BetheAnsatzDerivation:2010,B_ExcitedStateQuantumPhaseTransitions:2013,DickeTransitions:2021}. 
For instance, the algorithm by \citeauthor{van2021preparing}, who give an algorithm to prepare the Bethe ansatz eigenstates of the spin-1/2 XXZ spin chain, starts by first preparing a Dicke state~\cite{van2021preparing}. 
A Dicke-state preparation protocol based on the compress-uncompress methodology used in the $W$-state furthermore finds applications in entanglement distillation, where the entanglement of a large state is concentrated on only a few qubits. 
Efficient deterministic circuits for preparing Dicke states have been proposed by \citeauthor{bartschi2019deterministic}~\cite{bartschi2019deterministic, bartschi2022deterministic_short_depth}. 
They provide a quantum circuit of depth $\mathO(k \log(\frac{n}{k}))$, allowing arbitrary connectivity, to prepare a Dicke state, which they conjecture to be optimal when $k$ is constant. 
In this work, we provide a constant-depth $\LAQCC$ circuit below their conjectured bound already for constant $k$. 
However, this does not directly disprove their conjecture, as we allow for intermediate measurements and classical computations. 
More significantly, we even construct constant-depth $\LAQCC$ circuits for $k = \mathO(\sqrt{n})$ greatly improving their bound.
This construction extends the compress-uncompress method for the $W$-state combined with additional subroutines. 

We continue with a log-depth state preparation protocol for the Dicke-state for arbitrary $k$. 
This protocol implements an efficient transformation between the factoradic number representation and the combinatorial number representation of a positive integer. 
The combinatorial number representation relates directly to the Dicke state. 
The provided efficient transformation between number representation systems might be of independent interest. 

We conclude by modifying our protocol for preparing a Dicke-state to a protocol that prepares quantum many-body scar states in constant-depth. 
These states have low entanglement and longer coherence times than states with similar energy density.
These characteristics make many-body scar states interesting to analyze and relevant within physics.
Many-body scar states appear for instance in the AKLT model~\cite{AKLT:1987,MRBAR:2018,MRB:2018} and different spin models~\cite{SI:2019,MOBFR:2020}.
Known methods for preparing these states have polynomial-depth~\cite{Gustafson:2023}, whereas our circuit has constant depth. 

% We conclude by studying the power that intermediate classical calculations can add to quantum computations. 
% In this study, we define a new model that relaxes constant-depth quantum circuits to polynomial depth quantum circuits, log-depth classical calculations to unbounded classical computations and a constant number of alternations to a polynomial number of alternations. 
% We call this model $\LAQCC^*$. 
% We study this model by doing a complexity theoretical analysis, where we draw inspiration from the notions of complexity given by \citeauthor{RosenthalYuen:2022}, \citeauthor{MetgerYuen:2023}, and \citeauthor{Aaronson:2004}.
% All three complexity notions are based on the notion of state preparation, instead of more traditional definition of complexity such as the decidability of a computational problem. 
% The first two consider classes based on sequences of quantum states preparable by a polynomial-sized quantum circuit, where the circuits are uniformly generated by a computational class, for instance, the class $\mathsf{PSPACE}$, which results in the complexity class $\mathsf{StatePSPACE}$~\cite{RosenthalYuen:2022,MetgerYuen:2023}.
% The third notion considers a relative complexity, where the complexity is measured between two given states, and is measured by the number of gates, from a given gate-set, required to transform one state in another state~\cite{Aaronson:2004}. 
% For our definition of state preparation complexity, we drop the uniformity constraint from~\cite{RosenthalYuen:2022,MetgerYuen:2023} and define a class as $\mathsf{StateX}$, which refers to states preparable by circuits of type $\mathsf{X}$. 
% As an example, if $\mathsf{X} = \QNC^0$, this results in the class $\mathsf{StateQNC^0}$, which is the set of states preparable from the $\ket{0}^n$ state by poly-size constant-depth circuits. 
% This notion is similar to the relative complexity from~\cite{Aaronson:2004}, where one state is the  $\ket{0}^n$ state and instead of counting the number of gates we consider the set of states preparable by a fixed number of gates. Using this notion of complexity we show that any state preparable by an $\LAQCC^*$ circuit is also preparable by a $\mathsf{PostQPoly}$ circuit, the class of circuits of polynomial depth with an additional post-selection gate. 

\paragraph{Summary of results}
\begin{itemize}
    \item We give a new definition of a computational model that captures the power of the four step process: applying a constant number of layers of one- and two-qubit gates; performing a syndrome measurement; perform a fast classical computation determining corrections; apply corrections. We call this model \emph{Local Alternating Quantum Classical Computations}, or $\LAQCC$ for short. In this model we bound the allowed quantum operations, intermediate classical calculations, and number of rounds separately. In Section~\ref{sec:LAQCC_model} we define this model and give a list of operations based on results from literature contained in this computational model. In some of these operations we explicitly use that we allow for multiple, but at most constant, rounds  of corrections.
    \item  We show show that there exist $\LAQCC$ circuits that can not be weakly simulated in Section~\ref{sec:IQP_in_LAQCC}. We further show that for every $\LAQCC$ circuit there exists a $\QNC^1$ circuit simulating it perfectly, in Section~\ref{sec:LAQCC_in_QNC1}.
    \item We introduce a new type computational complexity for preparing states and show that the extension of $\LAQCC$ where we allow a polynomial number of rounds and unbounded classical computation, is contained in $\mathsf{PostQPoly}$, the class of polynomial circuits with post-selection, in Section~\ref{sec:Complexity results}.
    \item We show a protocol to prepare the uniform superposition state of size $q$ in $\LAQCC$ using $\mathO(\ceil{\log_2(q)}^2)$ qubits in Section~\ref{sec:superposition_modulo_q}. 
    \item We show a protocol to prepare the $W_n$ state in $\LAQCC$ using $\mathO(n\log(n))$ qubits in Section~\ref{sec:W_state_in_LAQCC}.
    \item We show two ways of preparing the Dicke-$(n,k)$ state. The first method is in $\LAQCC$, works up to $k = \mathO(\sqrt{n})$, uses $\mathO(n^2\log(n))$ qubits, and is found in Section~\ref{sec:dicke:small_k}. The second method is in $\LAQCC\text{-}\mathsf{LOG}$ (an extension of $\LAQCC$ allowing for logarithmic number of alterations instead of constant), works for any $k$, uses $\mathO(\text{poly}(n))$ qubits, and is found in Section~\ref{sec:Dicke_in_LAQCC_LOG}. 
    \item We extend on our $\LAQCC$ method of generating Dicke-$(n,k)$ states for $k = \mathO(\sqrt{n})$ and show a protocol to generate many-body scar states for a particular Hamiltonian in $\LAQCC$ (Section~\ref{sec:many_body_scar}). 
\end{itemize}
Summarized in a table, we provide the following state generation protocols:
\begin{table}[htb]
\centering
\begin{tabular}{l|l|l|l}
\textbf{State description} & \textbf{Width} & \textbf{Depth} & \textbf{Implementation}\\
\hline 
Uniform superposition mod $q$: $\frac{1}{\sqrt{q}} \sum_{i = 0}^{q-1}\ket{i}$ & $\mathO(\ceil{\log^2 q})$ & $\mathO(1)$ & Section~\ref{sec:superposition_modulo_q}\\

$W$-state: $\frac{1}{\sqrt{n}}\sum_{i = 0}^{n-1}\ket{e_i}$ & $\mathO(n \log n)$ & $\mathO(1)$ & Section~\ref{sec:W_state_in_LAQCC}\\

Dicke-$(n,k)$, $k = \mathO(\sqrt{n})$: $\binom{n}{k}^{-1/2}\sum_{x \in \{0,1\}^n: |x| = k} \ket{x}$ &  $\mathO(n^2\log n)$ & $\mathO(1)$ 
&Section~\ref{sec:dicke:small_k}\\

Dicke-$(n,k)$: $\binom{n}{k}^{-1/2}\sum_{x \in \{0,1\}^n: |x| = k} \ket{x}$ & $\mathO(\text{poly}(n))$ & $\mathO(\log n)$ &Section~\ref{sec:Dicke_in_LAQCC_LOG}\\

QMBS: $\ket{S_k} = \frac{1}{k! \sqrt{\mathcal N(n,k)}}(Q^\dagger)^k \ket{\Omega}$ &  $\mathO(n^2\log n)$ & $\mathO(1)$  &  Section~\ref{sec:many_body_scar}
\end{tabular}
\caption{Summary of state preparation protocols given in this paper.}
\label{tab:sate_prep}
\end{table}
In the entry for the quantum many-body scar state $Q$ denotes the raising operator and $\mathcal N(n,k)=\binom{n-k-1}{k}$. 
Section~\ref{sec:many_body_scar} will provide more details on the variables and the implementation. 

\paragraph{Organization of the paper}
\noindent We first introduce relevant preliminaries in Section~\ref{sec:preliminaries}. 
In Section~\ref{sec:LAQCC_model} we formally define the class of Local Alternating Quantum-Classical Computations ($\LAQCC$). We also show that any Clifford circuit can be implemented in constant depth $\LAQCC$ (a result based on a result from measurement-based quantum computing~\cite{jozsa2006introduction}). 
This result allows us to give many useful multi-qubit gates and routines in Section~\ref{sec:gates_created_in_LAQCC}. 
Beyond that we show that constant depth $\LAQCC$ circuits are contained in $\QNC^1$ and that any $\mathsf{IQP}$ circuit has an $\LAQCC$ implementation.
We conclude this section with an analysis of a more powerful instantiation of $\LAQCC$ and show an inclusion with respect to the class $\mathsf{PostQPoly}$, which is the class of circuits of polynomial depth with one additional post-selection gate. 
In Section~\ref{sec:state_prep_in_LAQCC} we give $\LAQCC$ circuit implementations for preparing the uniform superposition over an arbitrary number of states, the $W$-state and the Dicke state up to $k = \mathO(\sqrt{n})$. We furthermore give a log-depth circuit implementation for preparing the Dicke state for any $k$. We conclude by showing a $\LAQCC$ circuit for generating many body scar states of a particular type of Hamiltonian.


\section{Related Work}
%\subsection{Cost Volume based Deep Stereo Matching}
%Stereo matching is a typical problem that has been studied for decades and a well-known four-step pipeline \cite{scharstein2002taxonomy} has been established, where cost volume construction is an indispensable step. Current state-of-the-art stereo matching methods are all cost volume based methods and they can be categorized into two types. Typically, a cost volume is a 4D tensor of height, width, disparity, and features. The first category just uses a full correlation to generate a single-feature cost volume. Such methods are usually efficient but lose much information because of the decimation of feature channels. Many previous work, including Dispnet \cite{dispnet}, MADNet \cite{madnet}, IResNet \cite{iresnet} and AANet \cite{aanet}, belong to this category. The second category usually uses concatenation \cite{gcnet} or group-wise correlation \cite{gwcnet} to generate a multi-feature 4D cost volume. Such a method can achieve better performance while requiring higher computational complexity and memory consumption. Actually, a majority of the top-performing networks in public leaderboards belong to this category, such as GANet \cite{ganet}, CSPN \cite{cspn} and ACFNet \cite{acfnet}. These methods generally employ multiple 3D convolution layers to constantly regularize the 4D cost volume and then apply softmax over the disparity dimension to produce a discrete disparity probability distribution. The final predicted disparity is obtained by softly weighting indices according to their probability, which is also called soft argmin in GCNet \cite{gcnet}. However, soft argmin leaves the output susceptible to multi-modal disparity probability distributions. ACFNet \cite{acfnet} observes this problem and proposes to directly supervise the cost volume with unimodal ground truth distributions. In contrast, we define an uncertainty estimation to quantify the degree to which the cost volume tends to be multi-modal distribution, higher implies the higher possibility of estimation error.

\subsection{Multi-scale Cost Volume based Stereo Matching}
Cost volume construction is an indispensable step in the well-known four-step pipeline for stereo matching \cite{scharstein2002taxonomy, pamisurvey1, pamisurvey2}. Typically, current state-of-the-art stereo matching methods can be categorized into two types of cost volume-based methods, where the cost volume is a 4D tensor of height, width, disparity, and features. The first category usually uses the single-feature 3D cost volume generated by full correlation, which is efficient while losing much information due to the decimation of feature channels. Many real-time methods, such as Dispnet \cite{dispnet}, MADNet \cite{madnet, madnet_pami} and AANet \cite{aanet}, belongs to the category. Moreover, two-stage refinement \cite{mcvmfc} and pyramidal towers \cite{madnet} are commonly applied in the single-feature cost volume based network to construct multi-scale cost volume. The second category usually uses the multi-feature 4D cost volume generated by concatenation \cite{gcnet} or group-wise correlation \cite{gwcnet}, which can achieve better performance with higher computational complexity and memory consumption. Most top-performing networks, including GANet \cite{ganet}, CSPN \cite{cspn} and ACFNet \cite{acfnet} belong to this category. 
% In these methods, the 4D cost volume is constantly regularized by multiple 3D convolution layers and then a discrete disparity probability distribution can be produced by softmax. Next, the final predicted disparity can be obtained by softly weighting indices according to their probability \cite{gcnet}. However, such output is susceptible to multimodal disparity probability distributions and ACFNet \cite{acfnet} gives a solution by directly supervising the cost volume with unimodal ground truth distributions to alleviate this problem. 
Recently, to alleviate the high computational complexity and memory consumption when employing multi-feature 4D cost volumes, \cite{cvpmvsnet, cascade, uscnet} propose to use cascade cost volume representation in multi-view stereo. These methods usually first predict an initial disparity at the coarsest resolution of the image and then gradually refine the disparity by narrowing down the disparity search space. More closely related to our approach is Casstereo \cite{cascade}, which first extended such representation to stereo matching. It selected to uniform sample a pre-defined range to generate the next stage’s disparity search range. Instead, we employ pixel-level uncertainty estimation to adaptively adjust the next stage disparity searching range and generate pseudo-labels for subsequent domain adaptation. Our method also shares similarities with UCSNet \cite{uscnet}, which constructs uncertainty-aware cost volume in multi-view stereo while it doesn’t employ uncertainty estimation to generate pseudo-labels.

%\subsection{Multi-scale Cost Volume based Deep Stereo Matching} 
% \subsection{Multi-scale Cost Volume based Stereo Matching} 
%Multi-scale cost volume firstly was applied in the single-feature cost volume based network with the form of two-stage refinement \cite{mcvmfc} and pyramidal towers \cite{madnet}. Recently, cascade cost volume representation \cite{cvpmvsnet, cascade, uscnet} was proposed in multi-view stereo to alleviate the high computational complexity and memory consumption when employing multi-feature 4D cost volumes. These methods generally predict an initial disparity at the coarsest resolution of the image. Then, they will narrow down the disparity search space and gradually refine the disparity. More closely related to our approach is Casstereo \cite{cascade}, which first extended such representation to stereo matching. It selected to uniform sample a pre-defined range to generate the next stage’s disparity search range. Instead, we employ uncertainty estimation to adaptively adjust the next stage pixel-level disparity searching range and push the next stage's cost volume to be predominantly unimodal.

% The single-feature cost volume based network with the form of two-stage refinement \cite{mcvmfc} and pyramidal towers \cite{madnet} first employ multi-scale cost volume for stereo matching. Recently, to alleviate the high computational complexity and memory consumption when employing multi-feature 4D cost volumes, \cite{cvpmvsnet, cascade, uscnet} propose to use cascade cost volume representation in multi-view stereo, which generally predict an initial disparity at the coarsest resolution of the image. Then, the disparity search space is narrowed down and the disparity is gradually refined. More closely related to our approach is Casstereo \cite{cascade}, which first extended such representation to stereo matching. It selected to uniform sample a pre-defined range to generate the next stage’s disparity search range. Instead, we employ uncertainty estimation to adaptively adjust the next stage pixel-level disparity searching range and push the next stage's cost volume to be predominantly unimodal.

% Figure environment removed

\subsection{Robust Stereo Matching} 
There exist three categories of generalization definitions for robust stereo matching. 1) Cross-domain Generalization: the network’s ability to perform well on unseen scenes (cannot see the image pairs of the target domain in advance). Towards this end, Jia et al \cite{sungeneralizaiton} propose to incorporate scene geometry priors into an end-to-end network. Zhang et al \cite{dsmnet} introduce a domain normalization and a trainable non-local graph-based filter to construct a domain-invariant stereo matching network. 2) Adapt Generalization: the network’s ability to adapt pre-trained models to the new domain with unlabeled target data. Previous work usually pre-trains the models on synthetic data and then adapts it to new target domains with Graph Laplacian regularization \cite{zoom}, non-adversarial progressive color transfer \cite{adastereo}, and Knowledge Reverse Distillation \cite{aohnet}. More closely related to our approach are \cite{aohnet, unsuperviseddomainadaptation} in stereo matching and Monoresmatch \cite{monoresmatch} in monocular depth estimation, which also proposes to generate a pseudo-label for domain adaptation. However, these methods all select to employ classical stereo matching methods \cite{sgm} alongside with confidence estimators, e.g., left-right consistency check to generate pseudo-labels. That is all these methods need an independent method to generate corresponding pseudo-labels. Instead, the proposed method is an end-to-end network that can generate the predicted disparity map, corresponding uncertainty map and pseudo-labels jointly, which is a more simple, yet efficient way. 
% Instead, our proposed method can employ pixel-level and area-level uncertainty estimation to self-distill the predicted disparity maps of our pre-training model and generate sparse while reliable pseudo-labels to align the domain gap, which is a more simple, yet efficient way. 
3) Joint Generalization: the network’s ability to perform well on a variety of datasets with the same model parameters. MCV-MFC \cite{mcvmfc} introduces a two-stage finetuning scheme to achieve a good trade-off between generalization and fitting capability on multiple datasets. However, it doesn’t touch the inner difference between diverse datasets, e.g, the unbalanced disparity distribution. To further address this problem, we propose a cascade cost volume to adaptively the next stage disparity searching space, where the pixel-level uncertainty estimation is at the core.

% \subsection{Monocular Depth Estimation}
% Monocular depth estimation aims to estimate depth values from a single image, instead of stereo images or multiple frames in a video. This problem is ill-posed because of the ambiguity of object sizes. However, humans could estimate the depth from a single image with prior knowledge of the scenes. Recently, learning based methods were explored to learn depth values by supervised or unsupervised learning. Eigen et al. first employed Convolutional Neural Networks (CNN) to predict depth in a coarse-to-fine manner and further improved its performance by multi-task learning. Liu et al. presented deep convolutional neural fields model by combining deep model with continuous CRF. Li et al. [22] refined deep CNN outputs with a hierarchical CRF. Multi-scale continuous CRF was formulated into a deep sequential network by Xu et al. [45] to refine depth estimation. Unsupervised methods tried to train monocular depth estimation with stereo
% image pairs or image sequences and test on single images. Garg et al. [9] used novel image view synthesis loss to train a depth estimation network in an unsupervised way. Godard et al. [11] introduced left-right consistency regularization to improve the performance of view synthesis loss. Recently, some work also propose to use the stereo matching network as a proxy to learn depth from synthetic data or directly employ traditional stereo matching methods to distill proxies labels from the target domain, which proves the feasibility of distilling stereo matching networks to learn monocular depth estimation.



\section{Preliminaries}
In this section, we describe the necessary background for automated planning and the significance of the International Planning Competition. 

% \subsection{Ontology}
% A formal ontology is typically represented as a set of concepts, relations, and axioms. A concept represents a set of objects or entities that share common properties, while a relation represents a connection or association between two or more concepts. Axioms are statements that define the relationships between concepts and relations. It is a formal representation of knowledge that is designed to facilitate automated reasoning and information processing. It acts as a structured vocabulary that describes a domain and promotes interoperability, data integration, and communication between humans and machines. Formally, an ontology $O$ can be represented as a tuple $(C, R, A)$, where $C$ is the set of concepts, $R$ is the set of relations, and $A$ is the set of axioms. Each concept \textit{c} $\in$ $C$ can be represented as a set of attributes, denoted as $Att(c)$. Similarly, each relation \textit{r} $\in$ $R$ can be represented as a set of attributes, denoted as $Att(r)$.

% Ontology is a branch of philosophy that deals with the nature of existence and being. In the field of computer science, however, ontology refers to a formal representation of knowledge that is designed to facilitate automated reasoning and information processing. It is a structured vocabulary that describes a domain and promotes interoperability, data integration, and communication between humans and machines. Various tools and methodologies, including Protege and ontology editors, are available for ontology creation. Ontologies are increasingly important in artificial intelligence, knowledge engineering, and the semantic web, and researchers are exploring their potential in diverse domains and applications.

% Figure environment removed

\subsection{Automated Planning}

Automated planning, also known as AI planning, is the process of finding a sequence of actions that will transform an initial state of the world into a desired goal state \cite{ghallab2004automated}. It involves constructing a plan or a sequence of actions that will achieve a specified objective while respecting any constraints or limitations that may be present. Formally, automated planning can be defined as a tuple $(S, A, T, I, G)$, where:
\begin{itemize}
    \item $S$ is the set of possible states of the world
    \item $A$ is the set of possible actions that can be taken
    \item $T$ is the transition function that describes the effects of taking an action on the current state of the world
    \item $I$ is the initial state of the world
    \item $G$ is the desired goal state
\end{itemize}
Using this notation, the problem of automated planning can be framed as finding a sequence of actions $\prec a_1, a_2, ..., a_k\succ$ that will transform the initial state $I$ into the goal state $G$, while respecting any constraints or limitations on the actions. 
 % In automated planning, 
 A problem is defined in terms of a domain and a problem instance. The domain defines the possible actions that can be taken and the effects of each action, while the problem instance specifies the initial state of the world and the desired goal state. 
Various techniques can be used to solve the planning problem, such as search algorithms, constraint-based reasoning, and optimization methods. These techniques involve exploring the space of possible plans and selecting the one that satisfies the objective and any constraints. Figure \ref{fig:planning_bw} illustrates an automated planning scenario for the blocksworld domain, where an initial state can be transformed into a goal state by executing a sequence of actions.

% \noindent \textbf{Attributes modeled about a domain.}
%   %\noindent \textbf{Attributes modeled in a domain file}
%  \begin{enumerate}
%      \item \textbf{Requirements:} A list of requirements that the planner must satisfy in order to solve the domain. Requirements include durative actions, conditional effects, or negative preconditions. For example, in blocksworld domain with types involved, one of the requirements is \emph{typing}.
%     \item \textbf{Predicates:} Predicates are fundamental elements in the planning domain that define the properties of the world. They are used to describe the initial and goal states, as well as the preconditions and effects of actions. Predicates are usually defined as logical expressions over a set of variables, where each variable can take on a finite number of values. In the context of planning, predicates are typically used to represent facts about the world that can be true or false, such as the location of an object or the status of a machine. For example, in blocksworld domain, the predicate \verb|(on b1 b2)| could indicate that block 'b2' is on top of block 'b1'.
%      \item \textbf{Actions:} Actions are the basic units of change in the planning domain. They represent atomic operations that can be performed to transform the world from one state to another. Each action has a name, a set of parameters, preconditions that must be satisfied before the action can be executed, and effects that describe the changes that the action makes to the world. Actions can be used to model a wide variety of operations, ranging from simple movements or transformations to complex processes such as planning or decision-making. For example, in blocksworld domain, the action \verb|unstack b2 b1| can be used to unstack block 'b2' from block 'b1'. 
     
%      \item \textbf{Preconditions:} Preconditions are the conditions that must be true before an action can be executed. They are usually defined using predicates and can involve multiple variables. Preconditions can also be negative, which means that a certain condition must not be true for an action to be executed. In planning, preconditions ensure that actions are only executed when the necessary conditions have been met, such as ensuring that a machine is turned off before it is serviced. For example, in blocksworld domain, the action \verb|unstack b2 b1| has a precondition of \verb|(on b1 b2)|, meaning that for the action to be valid, the block 'b2' should be on top of block 'b1'.
     
%      \item \textbf{Effects:} Effects describe the changes that an action makes to the world. They are usually defined using predicates and can involve multiple variables. Effects can be positive, which means that a certain condition becomes true after the action is executed, or negative, which means that a certain condition becomes false after the action is executed. In the context of planning, effects are used to model the changes that result from executing an action, such as moving an object from one location to another or turning a machine on. For example, in blocksworld domain, when the action \verb|unstack b2 b1| is executed, one of its effect is \verb|(not (on b1 b2))|, indicating that block 'b2' is no longer on top of block 'b1'.
     
%      \item \textbf{Constants:} Constants are values that are fixed and do not change during the execution of the planning problem. They are used to represent objects or entities in the world that have a fixed value, such as the speed limit on a road. Constants can be used to simplify the planning problem by reducing the number of variables that need to be considered and by providing a fixed set of values that can be used in predicates and actions. For example, in blocksworld domain, the constant \emph{table} could represent the surface on which the blocks are initially placed.
     
%      \item \textbf{Types:} Types are used to classify objects or entities in the world based on their attributes or properties. They are used to define the domain of values that a variable can take on and can be used to constrain the values that are assigned to variables. In the context of planning, types are typically used to group related objects or entities together, such as cars or bicycles, and to specify the properties that are common to all members of a type, such as their color or size. For example, in blocksworld domain with types involved, one can represent the predicate as \verb|(on ?x - block ?y - block)| stating that the parameters in the predicate are of type \emph{block}.

%  \end{enumerate}


% ######### Shorter version for AI Planning preliminaries
% \subsection{Automated Planning}

% Automated planning, also known as AI planning, finds actions transforming an initial world state into a goal state \cite{ghallab2004automated}. It involves creating a plan, respecting constraints, defined as $(S, A, T, I, G)$ where $S$ is the world states set, $A$ is the actions set, $T$ is the state transition function, $I$ is the initial state, and $G$ is the goal state. The challenge is to find actions $\prec a_1, a_2, ..., a_k\succ$ converting $I$ to $G$ under constraints. 

% A problem has a domain (defining actions and effects) and an instance (specifying initial and goal states). Various techniques can be used to solve the planning problem, such as search algorithms, constraint-based reasoning, and optimization methods. These techniques involve exploring the space of possible plans and selecting the one that satisfies the objective and any constraints. Figure \ref{fig:planning_bw} illustrates an automated planning scenario for the blocksworld domain, where an initial state can be transformed into a goal state by executing a sequence of actions.

\noindent \textbf{Attributes modeled about a domain.}
 \begin{enumerate}
     \item \textbf{Requirements:} A list of requirements that the planner must satisfy to solve the given domain, e.g., \emph{typing} in blocksworld with types.
     \item \textbf{Predicates:} Define world properties, e.g., \verb|(on b1 b2)| in blocksworld.
     \item \textbf{Actions:} Units of change with preconditions and effects, e.g., \verb|unstack b2 b1| in blocksworld.
     \item \textbf{Preconditions:} Conditions for action execution, e.g., \verb|(on b1 b2)| for \\ \verb|unstack b2 b1|.
     \item \textbf{Effects:} Post-action world changes, e.g., \verb|(not (on b1 b2))| after \\ \verb|unstack b2 b1|.
     \item \textbf{Constants:} Fixed values, e.g., \emph{table} in blocksworld.
     \item \textbf{Types:} Classifications based on attributes, e.g., \\ \verb|(on ?x - block ?y - block)| in typed blocksworld.
 \end{enumerate}

\noindent \textbf{Attributes modeled about a problem instance from a domain.}
\begin{enumerate}
    \item \textbf{Name:} The name of the planning problem.
    \item \textbf{Domain:} The name of the planning domain that the problem belongs to.
    \item \textbf{Objects:} A list of objects that are present in the planning problem. Objects are typically defined in terms of their type and name. In the example shown in Figure \ref{fig:planning_bw}, objects are b1, b2, and b3.
    \item \textbf{Initial State:} A description of the initial state of the world, including the values of all relevant predicates. Figure \ref{fig:planning_bw} represents an example initial state.
    \item \textbf{Goal State:} A description of the desired goal state of the world, including the values of all relevant predicates. Figure \ref{fig:planning_bw} represents an example goal state.
\end{enumerate}

% \vspace{2cm}
\subsection{International Planning Competition (IPC)}

% IPC serves as a significant means of assessing and comparing various planning systems. By presenting new planners and benchmark problems each year, the competitions aim to stimulate the advancement of new planning methodologies and reflect current trends and challenges in the field. The competition comprises multiple tracks, each covering various planning problems such as classical, temporal, and probabilistic planning. These tracks include benchmark problems that evaluate the performance of planners concerning parameters such as plan quality, plan length, and run time. The results of these competitions provide insights into the current state-of-the-art in planning and help identify the strengths and weaknesses of different planning systems. IPC can serve as an excellent starting point for building a planning-related ontology as the benchmark problems used in these competitions can provide a comprehensive overview of the domain and the types of problems that planners need to solve. 

IPC is pivotal for evaluating and contrasting planning systems. Introducing new planners and benchmarks, it promotes innovative planning methodologies and reflects the field's evolving challenges. The competition has multiple tracks, such as classical and probabilistic planning, with benchmarks assessing plan quality, length, and run time. IPC results offer a glimpse into the latest in planning, highlighting system pros and cons. The benchmarks from IPC are ideal for crafting a planning-related ontology, encapsulating the domain's breadth and planners' challenges.

\section{BAI FOR LINEAR BANDITS IN GENERAL NON-STATIONARY ENVIRONMENTS}
\label{sec:nonstationary}

In this section, we present a simple algorithm \textsf{G-BAI} for the general non-stationary environment and analyze its theoretical guarantee. The algorithm is based on the G-optimal design, which refers to the distribution $\lambda^*\in\triangle_{\X}$ such that
\begin{equation}
    \label{equ:g_design}
    \lambda^*=\arginf_{\lambda\in\triangle_{\X}}\max_{x\in\X}\Norm{x}^2_{A(\lambda)^{-1}}.
\end{equation}
Intuitively, G-optimal design allows us to estimate unknown parameter $\theta_t$ uniformly well over all directions of the arms in $\X$ \citep{soare2014best}. which is suitable for addressing non-stationarity since $\theta_t$ may change arbitrarily and each $x\in\X$ may become the optimal at time $t$. Meanwhile, to make sure the estimation of $\theta_t$ is unbiased in a non-stationary environment, we use an IPS estimator. 

Therefore, briefly speaking, at each time $t$, \textsf{G-BAI} simply samples $x_t$ based on G-optimal design and estimate $\theta_t$ through an IPS estimator, whose details are summarized in Algorithm \ref{algo:gbai}.\footnote{We can see $\widehat{\theta}_T$ exactly becomes the more commonly-seen IPS estimator examined in Eq. \eqref{equ:usual_ips} if we apply it to the multi-armed bandits setting, in which we have $K=d$ arms and $\mc{X}=\Bp{\ve{1}_1, \dots, \ve{e}_d}$.}
% \begin{algorithm}[ht]
%     \caption{G-optimal Best-arm Identification (G-BAI)}
%     \label{algo:gbai}
%     \SetAlgoLined
%     \KwIn{budget, $T\in\mathbb{N}$; arm set $\mc{X}\subset\R^d$}
%     Compute G-optimal design $\lambda^*$ based on Eq. \eqref{equ:g_design}\\
%     \For{$t=1, 2, \dots, T$}{
%         Sample $x_t\sim\lambda^*$ and receive reward $r_t$\\
%     }
%     Estimate $\widehat{\theta}_T\leftarrow \frac{1}{T}\sum_{t=1}^T\E_{x\sim\lambda^*}\Mp{xx^\top}^{-1}x_t r_t$\\
%     \textbf{return} $\argmax_{x\in\X}x^\top\widehat{\theta}_T$
% \end{algorithm}

\begin{algorithm}[ht]
    \caption{G-optimal Best-arm Identification (G-BAI)}
    \label{algo:gbai}
    \begin{algorithmic}[1]
        \REQUIRE budget, $T\in\mathbb{N}$; arm set $\mc{X}\subset\R^d$
        \STATE Compute G-optimal design $\lambda^*$ based on Eq. \eqref{equ:g_design}
        \FOR{$t=1, 2, \dots, T$}
            \STATE Sample $x_t\sim\lambda^*$ and receive reward $r_t$
        \ENDFOR
        \STATE Estimate $\widehat{\theta}_T\leftarrow \frac{1}{T}\sum_{t=1}^T\E_{x\sim\lambda^*}\Mp{xx^\top}^{-1}x_t r_t$
        \RETURN $\argmax_{x\in\X}x^\top\widehat{\theta}_T$
    \end{algorithmic}
\end{algorithm}


By the famous Kiefer-Wolfowitz theorem, an important property of the G-optimal design is that $\max_{x\in\X}\Norm{x}^2_{A(\lambda^*)^{-1}}=d$ \citep{lattimore2020bandit}. With this property, the variance of estimator $\htheta_t$ can be easily controlled. We can then bound the error probability of \textsf{G-BAI} by this fact and the result is summarized in the following theorem.
% \begin{theorem}
%     \label{theo:adv_upper_bound}
%     Fix time horizon $T$, arm set $\X\subset\R^d$ with $\abs{\X}=K$ and arbitrary unknown parameters $\Bp{\theta_t}_{t=1}^T$. If we run Algorithm \ref{algo:gbai} and obtain $x_{J_T}$, then it holds that
%     $$\P\Sp{J_T\neq (1)}\leq K\exp\Sp{-\frac{3T\Delta_{(1)}^2}{32d}}.$$
% \end{theorem}

\begin{restatable}[Error probability of \textsf{G-BAI}]{theorem}{advupperbound}
    \label{theo:adv_upper_bound}
    Fix time horizon $T$, arm set $\X\subset\R^d$ with $\abs{\X}=K$ and arbitrary unknown parameters $\Bp{\theta_t}_{t=1}^T$. If we run Algorithm \ref{algo:gbai} in this non-stationary environment and obtain $x_{J_T}$, then it holds that
    \IfTwoColumnElse{
        \begin{align*}
            &\P_{\otheta_T}\Sp{J_T\neq (1)}\leq K\exp\Sp{-\frac{T}{12H_{\textsf{G-BAI}}\Sp{\otheta_T}}},\\
            &\text{where }H_{\textsf{G-BAI}}\left(\otheta_T\right)=\frac{d}{\Delta_{(1)}^2}.
        \end{align*}
    }{
        $$\P_{\otheta_T}\Sp{J_T\neq (1)}\leq K\exp\Sp{-\frac{T}{12H_{\textsf{G-BAI}}\Sp{\otheta_T}}},\quad\text{where }H_{\textsf{G-BAI}}\left(\otheta_T\right)=\frac{d}{\Delta_{(1)}^2}.$$
    }
\end{restatable}

The proof of Theorem \ref{theo:adv_upper_bound} is deferred to Appendix \ref{sec:g_design_proof}. Here, we briefly compare this result with the one in multi-armed bandits, which can be treated as a special case of linear bandits by taking $\X=\Bp{\ve{e}_1, \dots, \ve{e}_K}$ to be the canonical vectors (standard basis) with $K=d$. 

In particular, \citet{abbasi2018best} shows that in multi-armed bandits setting, a simple uniform sampling algorithm reaches complexity $H_{\mathrm{UNIF}}\Sp{\otheta_T}=\frac{K}{\Delta_{(1)}^2}$ and it is optimal in non-stationary environments. Meanwhile, based on Theorem \ref{theo:adv_upper_bound}, we can see the complexity of \textsf{G-BAI} is $H_{\textsf{G-BAI}}\Sp{\otheta_T}=\frac{d}{\Delta_{(1)}^2}$, which is exactly $H_{\mathsf{UNIF}}(\otheta_T)$ if we treat multi-armed bandits as a special case of linear bandits since $d=K$. Furthermore, if we directly apply \textsf{G-BAI} to multi-armed bandits, meaning to use $\X=\Bp{\ve{e}_1, \dots, \ve{e}_K}$, then $\lambda^*$ is exactly the uniform distribution over $\X$. That is, in multi-armed bandits, \textsf{G-BAI} exactly recovers the optimal complexity in non-stationary environments, which shows that \textsf{G-BAI} is minimax optimal for linear bandits.

% \begin{remark}[Comparison to \textsf{GSE} \citep{azizi2021fixed}]
%     \textsf{G-BAI} attains error probability in a same order of \textsf{GSE}, which also utilizes G-optimal design for arm allocation. However, since \textsf{GSE} is designed only for stationary environment, \textsf{G-BAI} can be viewed as a strict generalization of \textsf{GSE}.
% \end{remark}


\section{A ROBUST ALGORITHM FOR STATIONARY/NON-STATIONARY ENVIRONMENTS}
\label{sec:bobw}

%We should probably use  the words stochastic and adversarial (not just nonstationary) 

In this section, we present and analyze a new robust linear bandits BAI algorithm called \textsf{P1-RAGE}, which performs comparable to \textsf{G-BAI} in non-stationary environments but much better than it in stationary environments. We will show that it attains good error probability in both stationary and non-stationary environments simultaneously, without knowing a priori which environment it will encounter. We first discuss some intuitions behind the algorithm design.

\textbf{Stationary environments.} The development of our algorithm \textsf{P1-RAGE} is largely inspired by the high-level idea of the robust algorithm \textsf{P1}, proposed in \citet{abbasi2018best}, and the allocation strategy of \textsf{RAGE}, proposed in \citet{fiez2019sequential}. In particular, as discussed in \citet{abbasi2018best}, in multi-armed bandits, to minimize the error probability in stationary environment, we need to control the estimation variance of the optimal arm well enough. Therefore, at each time step, algorithm \textsf{P1} pulls the current estimated best arm with the highest probability (unnormalized ``probability one''), then subsequently the second best arm with second highest probability (unnormalized ``probability half'') and so on. 
% In particular, \textsf{P1} allocates the estimated best arm with unnormalized ``probability one'', estimated second best arm with ``probability half'' and so on. 
We can notice that it actually matches the allocation strategy of the successive halving algorithm in \citet{karnin2013almost}, which is proved to be near-optimal for BAI in stationary multi-armed bandits. Therefore, we design our probability allocation based on the allocation strategy of \textsf{RAGE}, which is proven to be near-optimal for fixed-confidence BAI in stationary linear bandits \citep{fiez2019sequential}. In particular, with the estimated parameter $\htheta_t$, we first find the estimated best arms $\hat{x}^*_t=\argmax_{x\in\X}x^\top\htheta_t$. Then, we use a subroutine to repeatedly and virtually eliminate arms with estimated gaps larger than certain threshold and compute $\mc{XY}$-allocation of the (virtually) remaining arms.\footnote{The elimination is virtual because no samples are collected during the elimination subroutine.} Then, we average over the allocation probabilities computed during each iteration.


% As discussed in \citet{soare2014best}, for linear bandits best-arm identification problem, a more appropriate allocation strategy is $\mc{XY}$-allocation. The reason is that to correctly identify the best arm, we will essentially estimate the gap $\Delta_x$ for all $x\in\X$ instead of just $x^\top\theta^*$. Therefore, it will be more important to reduce the variance along the directions of $x_{(1)}-x$ for all $x\in\X$ instead of simply the directions of all $x\in\X$.

\textbf{Non-stationary environments.} Finally, to address the potential non-stationarity in environments, we uniformly mix the allocation probability computed above with a G-optimal design. With such a mixture, the variance over all arms can be controlled well and thus the algorithm will be robust for both stationary and non-stationary environments. The details of \textsf{P1-RAGE} are summarized in Algorithm \ref{algo:p1_rage} and the subroutine to compute the allocation probability, called \textsf{RAGE-Elimination}, is summarized in Algorithm \ref{algo:rage_elimination}.
% \begin{algorithm}[ht]
%     \caption{P1-RAGE}
%     \label{algo:p1_rage}
%     \SetAlgoLined
%     \KwIn{budget, $T\in\mathbb{N}$; arm set $\mc{X}\subset\R^d$; maximum number of virtual phases, $m$}
%     Compute G-optimal design $\lambda^*$ based on Eq. \eqref{equ:g_design} and initialize $\lambda_1=\lambda^*$\\
%     \For{$t=1, 2, \dots, T$}{
%         Sample $x_t\sim\lambda_t$ and receive reward $r_t$\\
%         Estimate $\widehat{\theta}_t\leftarrow\frac{1}{t}\sum_{s=1}^t\E_{x\sim\lambda_s}\Mp{xx^\top}^{-1}x_s r_s$\\
%         Update $\lambda_{t+1}\leftarrow$\textsf{RAGE-Elimination}$(\htheta_t, m)$
%     }
%     \textbf{return} $\argmax_{x\in\X}x^\top\widehat{\theta}_T$\\
%     \SetKwFunction{proc}{\textsf{RAGE-Elimination}}
%     \SetKwProg{myproc}{Subroutine}{}{}
%     \myproc{\proc{$\htheta_t, m$}}{\label{algo:rage_elimination}
%         Find $\hat{x}^*_t\leftarrow\argmax_{x\in\X}x^\top\htheta_t$\\
%         Initialize $\X_t^{(0)}\leftarrow\X$ and $i\leftarrow 0$\\
%         \While{$|\X_{t}^{(i)}|> 1$ and $i\leq m$}{
%             Compute $\lambda^{(i)}_t\leftarrow\arginf_{\lambda\in\triangle_{\X}}\max_{x, x'\in\X_t^{(i)}}\Norm{x-x'}^2_{A(\lambda)^{-1}}$\\
%             Update $\mc{X}_{t}^{(i+1)} \leftarrow \Bp{ x \in \mc{X}_t^{(i)}\left|\  \htheta_t^\top(\hat{x}^*_t-x) \leq 2^{-i}\right. }$\\
%             $i\leftarrow i+1$\\
%         }
%         % Compute $\bar{\lambda}_t\leftarrow \frac{1}{i}\sum_{i'=0}^{i-1}\lambda_t^{(i')}$\\
%         \textbf{return} $(\bar{\lambda}_t+\lambda^*)/2$, where $\bar{\lambda}_t= \frac{1}{i}\sum_{i'=0}^{i-1}\lambda_t^{(i')}$\\
%     }
% \end{algorithm}


\begin{algorithm}[ht]
    \caption{P1-RAGE}
    \label{algo:p1_rage}
    \begin{algorithmic}[1]
        \STATE \textbf{Input:} budget, $T\in\mathbb{N}$; arm set $\mc{X}\subset\R^d$; maximum number of virtual phases, $m$
        \STATE Compute G-optimal design $\lambda^*$ based on Eq. \eqref{equ:g_design} and initialize $\lambda_1=\lambda^*$
        \FOR{$t=1, 2, \dots, T$}
            \STATE Sample $x_t\sim\lambda_t$ and receive reward $r_t$
            \STATE Estimate $\widehat{\theta}_t\leftarrow\frac{1}{t}\sum_{s=1}^t\E_{x\sim\lambda_s}\Mp{xx^\top}^{-1}x_s r_s$
            \STATE Update $\lambda_{t+1}\leftarrow$\textsf{RAGE-Elimination}$(\htheta_t, m)$
            \IfTwoColumnElse{
                 \\ 
            }{}
            \hfill // \COMMENT{Call Algorithm \ref{algo:rage_elimination}}
           
        \ENDFOR
        \RETURN $\argmax_{x\in\X}x^\top\widehat{\theta}_T$
    \end{algorithmic}
\end{algorithm}

\begin{algorithm}[ht]
    \caption{RAGE-Elimination}
    \label{algo:rage_elimination}
    \begin{algorithmic}[1]
        \STATE \textbf{Input:} arm set $\mc{X}\subset\R^d$; current estimate $\htheta_t$; maximum number of virtual phases, $m$
        \STATE Find $\hat{x}^*_t\leftarrow\argmax_{x\in\X}x^\top\htheta_t$
        \STATE Initialize $\X_t^{(0)}\leftarrow\X$ and $i\leftarrow 0$
        \WHILE{$|\X_{t}^{(i)}|> 1$ and $i\leq m$}
            \STATE $\lambda^{(i)}_t\leftarrow\arginf_{\lambda\in\triangle_{\X}}\max_{x, x'\in\X_t^{(i)}}\Norm{x-x'}^2_{A(\lambda)^{-1}}$
            \STATE $\mc{X}_{t}^{(i+1)} \leftarrow \Bp{ x \in \mc{X}_t^{(i)}\left|\  \htheta_t^\top(\hat{x}^*_t-x) \leq 2^{-i}\right. }$
            \STATE $i\leftarrow i+1$
        \ENDWHILE
        \RETURN $(\bar{\lambda}_t+\lambda^*)/2$, where $\bar{\lambda}_t= \frac{1}{i}\sum_{i'=0}^{i-1}\lambda_t^{(i')}$
    \end{algorithmic}
\end{algorithm}


We bound the error probability of \textsf{P1-RAGE} under both stationary and non-stationary settings in the following theorem and its proof is deferred to Appendix \ref{sec:bobw_proof}.
\begin{theorem}[Error Probability of \textsf{P1-RAGE}]
    \label{theo:bobw_upper_bound}
    Fix arm set $\X\subset\R^d$ with $\abs{\X}=K$ and budget $T$. For a stationary environment with unknown parameter $\theta$, if $m\geq i_0=\ceil{\log_2\Sp{1/\Delta_{(1)}}}+1$, then there exists absolute constant $c>0$ such that the error probability of \textsf{P1-RAGE} satisfies
    \IfTwoColumnElse{
        \fontsize{9.5}{9.5}
        $$\P_{\theta}\Sp{J_T\neq (1)}\leq 2i_0 KT\exp\Sp{-\frac{cT}{H_{\textsf{P1-RAGE}}(\theta)}},$$
        \begin{equation}
            \label{equ:H_bobw}
            \begin{split}
                &H_{\textsf{P1-RAGE}}(\theta)= \frac{mi_0}{\Delta_{(1)}}\inf_{\lambda\in\triangle_{\X}}\max_{x\neq x_{(1)}}\frac{\Norm{x-x_{(1)}}^2_{A(\lambda)^{-1}}}{\Delta_x} \\
                &\quad + \frac{m\sqrt{d}}{\Delta_{(1)}}\inf_{\lambda\in\triangle_{\X}}\max_{x\neq x_{(1)}}\Norm{x-x_{(1)}}_{A(\lambda)^{-1}}.
            \end{split}
        \end{equation}
        \normalsize
    }{
        \begin{align}
            \P_{\theta}\Sp{J_T\neq (1)}\leq & 2i_0 KT\exp\Sp{-\frac{cT}{H_{\textsf{P1-RAGE}}(\theta)}},\nonumber\\
            \text{where } H_{\textsf{P1-RAGE}}(\theta)=& \frac{mi_0}{\Delta_{(1)}}\inf_{\lambda\in\triangle_{\X}}\max_{x\neq x_{(1)}}\frac{\Norm{x-x_{(1)}}^2_{A(\lambda)^{-1}}}{\Delta_x} + \frac{m\sqrt{d}}{\Delta_{(1)}}\inf_{\lambda\in\triangle_{\X}}\max_{x\neq x_{(1)}}\Norm{x-x_{(1)}}_{A(\lambda)^{-1}}.\label{equ:H_bobw}
        \end{align}
    }
    
    For a non-stationary environment with unknown parameter $\Bp{\theta_t}_{t=1}^{T}$, there exists absolute constant $c'>0$ such that the error probability of \textsf{P1-RAGE} satisfies
    $$\P_{\otheta_T}\Sp{J_T\neq (1)}\leq K\exp\Sp{-\frac{c'T\Delta_{(1)}^2}{d}}.$$
\end{theorem}

We can immediately see that in non-stationary environments, the error probability of \textsf{P1-RAGE} matches (up to a constant) with \textsf{G-BAI}, showing that \textsf{P1-RAGE} is minimax optimal for linear bandits under non-stationarity. On the other hand, because of the $\frac{1}{\Delta_{(1)}}$ factor, we can see that in stationary environments, $H_{\textsf{P1-RAGE}}(\theta)\gtrsim H_{\textsf{LB}}(\theta)$ (defined in Eq. \eqref{equ:rho_star}), which implies that \textsf{P1-RAGE} is suboptimal in stationary settings. However, this should be expected since even for multi-armed bandits, as proved in \citet{abbasi2018best}, it is impossible for an algorithm to achieve $H_{\textsf{LB}}(\theta)$ while being robust to non-stationarity, let alone linear bandits. 

Nevertheless, when applying Theorem \ref{theo:bobw_upper_bound} to multi-armed bandits ($\X=\Bp{\ve{e}_1, \dots, \ve{e}_K}$), as long as we choose $m\approx i_0$, we can show that (Corollary \ref{coro:bobw_linear_to_mab} in Appendix \ref{sec:bobw_proof})
$$H_{\textsf{P1-RAGE}}(\theta)=\widetilde{O}\Sp{\frac{1}{\Delta_{(1)}}\max_{k\in[K]}\frac{k}{\Delta_{(k)}}}=\widetilde{O}\Sp{H_{\mathrm{BOB}}(\theta)},$$
where $H_{\mathrm{BOB}}(\theta)$ is the best-of-both-worlds complexity proposed in \citet{abbasi2018best}. In particular, \citet{abbasi2018best} proves that $H_{\mathrm{BOB}}(\theta)$ is the best complexity that any algorithm can possibly achieve if it is constrained to be robust to non-stationarity. 
% \footnote{A more detaild discussion of this lower bound is given in Appendix \ref{sec:lower_bound_diss}.} 
That is, again, our algorithm \textsf{P1-RAGE} retains the near-optimal complexity for stationary multi-armed bandits if it is constrained to be robust in non-stationary environments.

\begin{remark}
    Here, we do not elaborate the proof details of Theorem \ref{theo:bobw_upper_bound} mainly because we do not recognize them as widely applicable techniques. However, we do want to emphasize that this proof is by no means a simple extension of the analysis of the algorithm \textsf{P1} in \citet{abbasi2018best}. In particular, our proof uses a different set of virtual events based on the estimated gaps. Meanwhile, the analysis of subroutine \textsf{RAGE-Elimination} is intricately tailored to the unique characteristics of being a virtual elimination strategy, which is not presented in neither \textsf{RAGE} nor \textsf{P1} \citep{abbasi2018best, fiez2019sequential}.
\end{remark}

% \begin{remark}[Theoretical limitations of \textsf{P1-RAGE}]
\textbf{Theoretical limitations of \textsf{P1-RAGE}.} Despite being near-optimal in multi-armed bandits, $H_{\textsf{P1-RAGE}}(\theta)$ includes an extra low-order term $\frac{m\sqrt{d}}{\Delta_{(1)}}\inf_{\lambda\in\triangle_{\X}}\max_{x\neq x_{(1)}}\Norm{x-x_{(1)}}_{A(\lambda)^{-1}}$. This term appears because the Bernstein's inequality requires a bound of the estimator's magnitude, which can be removed if the concentration bound only scales with the estimator's variance. Although this can often be accomplished by using Catoni's robust mean estimator \citep{wei2020taking}, it requires a concrete confidence level to be specified before estimation, which is not feasible in our fixed budget setting. Finding an approach to circumvent this difficulty and remove this extra term, or alternatively, demonstrate that it is necessary, is an open question. 


\begin{remark}
    % The question of whether the extra term is removable naturally relates to the lower bound of this problem. Although proving a lower bound can indeed provide a more complete picture, we would like to emphasize that this is a too ambitious goal which reasonably lies beyond the scope of this paper. In particular, for fixed-budget best-arm identification problem in linear bandits, an instance-dependent lower bound remains open even in the pure stationary setting.\footnote{\citet{yang2022minimax} proves a minimax lower bound instead of an instance-dependent lower bound.} Therefore, it would require a major breakthrough to achieve a lower bound for our problem setting immediately since it requires constructing both stationary and non-stationary counterexamples, which is strictly harder than the pure stationary setting.
    The question of whether the extra term is removable naturally relates the instance-dependent lower bound of this problem. However, proving an instance-dependent lower bound for our setting requires constructing both stationary and non-stationary counterexamples. This task is thereby more challenging compared to proving an instance-dependent lower bound for the fixed-budget best-arm identification problem in linear bandits within a purely stationary setting, an open question that persists (see \citet{yang2022minimax} for a minimax lower bound).  We thus leave establishing such instance-dependent lower bounds for future work.
\end{remark}
% \romain{
% \begin{remark}
%     The question of whether the extra term is removable naturally relates the instance-dependent lower bound of this problem. Proving an instance-dependent lower bound for our bobw problem setting requires constructing both stationary and non-stationary counterexamples. This is thus strictly harder than proving an instance-dependent lower bound for fixed-budget best-arm identification in linear bandits which remains an open problem in the pure stationary setting.\footnote{\citet{yang2022minimax} proves a minimax lower bound instead of an instance-dependent lower bound.} We thus leave establishing such instance-dependent lower bounds for future work.
%     % Therefore, it would require a major breakthrough to 
% \end{remark}}

\textbf{Parameter choice of \textsf{P1-RAGE}.} Although \textsf{P1-RAGE} requires a user-specified parameter $m\geq \ceil{\log(1/\Delta_{(1)})}+1$ to bound the total number of virtual phases, it is not difficult to choose a reasonable value for this parameter in a practical implementation. On the one hand, since its dependence on $\Delta_{(1)}^{-1}$ is only logarithmic, taking some moderate value such as $m=25$ should safely satisfy $m\geq i_0$ for most practical scenarios; on the other hand, in most real-world applications, a sub-optimal arm should always be acceptable as long as its gap is small enough. Indeed, if we take $\epsilon$ to be the largest acceptable sub-optimality gap and take $m\geq \ceil{\log(1/\epsilon)}+1$, then \textsf{P1-RAGE} will output arm $x_{J_T}$ that satisfies $\Delta_{J_T}\leq \max\Bp{\epsilon, \Delta_{(1)}}$ with high probability in pure stationary environments (Corollary \ref{coro:epsilon_bai} in Appendix \ref{sec:bobw_proof}). That is, the output arm will either be an optimal arm if $\epsilon\leq\Delta_{(1)}$ or an arm with an acceptable suboptimality gap $\epsilon$ otherwise.
% even in some rare case where $\Delta_{(1)}$ is unreasonably small, a more common practical demand is to identify an arm $x_{J_T}$ such that $\Delta_{J_T}\leq \epsilon$ for some $\epsilon>\Delta_{(1)}$. That is, in this scenario, we can simply take $m\geq i_0(\epsilon)=\ceil{\log(1/\epsilon)}+1$ and \textsf{P1-RAGE} will guarantee that the output arm $x_{J_T}$ satisfies $\Delta_{J_T}\leq \epsilon$ (Corollary \ref{coro:epsilon_bai}). More details of the implementation can be found in Appendix \ref{sec:experiment_details}.
% Meanwhile, computing a full \textsf{RAGE-Elimination} subroutine at each time step can make the algorithm unacceptably slow. Therefore, to be computationally efficient, we update the sampling distributio $\lambda_t$ in a low frequency. More details of the implementation can be found in


% Another limitation of \textsf{P1-RAGE} is its strong reliance on G-optimal design to confront the non-stationarity. That is, there is no mechanism allowing it to be adaptive to different levels of non-stationarity. Finding a good measure of non-stationarity for BAI problem and developing a corresponding adaptive algorithm can serve as another promising direction.
% \end{remark}

% Finally, although \textsf{P1-RAGE} has favorable theoretical guarantees, it requires user-specified parameter $m\geq i_0=\ceil{\log(1/\Delta_{(1)})}+1$, which is practically not always available. \romain{In practice, the logarithmic dependence enables to set a fixed number (e.g. $m=25$).} Meanwhile, computing a full \textsf{RAGE-Elimination} subroutine at each time step can make the algorithm unacceptably slow.  Therefore, we also design a implementation-friendly modified algorithm, called \textsf{P1-Peace}, that are both parameter-free and computationally efficient. The key modifications are that its (virtual) elimination strategy is based on \textsf{Peace} in \citet{katz2020empirical} and it updates the sampling distribution $\lambda_t$ in a low frequency. The algorithm details are summarized in Algorithm \ref{algo:p1_peace} in Appendix \ref{sec:modified_algo}. 
% \zhihan{TODO: add more details of \textsf{P1-Peace} here.}
\section{Experiments}
% \haizhou{Follow the same way of introduction as we did in Section2.}
% \noindent In this section, we will introduce datasets and experimental setups that we used. Then we evaluate our method, other self-supervised methods, and supervised methods under different distribution shifts (\ie, concept shifts and covariate shifts) under common settings (\ie, transductive, inductive settings). It has to note that we focus on node-level tasks (\eg, node classification) in this work. As for graph-level tasks, we leave it as our future work and some simple experiments can be found in Appendix~\ref{app:graph_classification}. 
In this section, we first introduce the experimental setup including datasets, training, and evaluation protocol in Section~\ref{sec:dataset}~and~\ref{sec:unsupervised}. 
% Next, we present our experimental setup and conduct extensive experiments to evaluate our method in Section~\ref{sec:unsupervised}. 
We then perform an ablation study to demonstrate the effectiveness of each proposed component in Section~\ref{sec:ablation}. 
Additionally, we analyze the impact of important hyper-parameters in Section~\ref{sec:sensitivity}. 
Subsequently, we integrate our method with various encoding models, showcasing the model-agnostic nature of our recipe in Section~\ref{sec:other_models}. 
Finally, we provide some qualitative results such as feature visualization in Section~\ref{sec:vis}.
It is important to note that we focus on node-level tasks (\eg, node classification) in this work. As for graph-level tasks, we leave it as our future work, while some simple experiments are also provided in Appendix~\ref{app:graph_classification}.

\subsection{Datasets}\label{sec:dataset}
There exist some benchmarks for evaluating graph out-of-distribution generalization~\cite{good,ji2022drugood,gds}. 
Among them, GOOD~\cite{good} is the most representative and comprehensive benchmark that curates more diverse graph datasets with diverse tasks, including single/multi-task graph classification, graph regression, and node classification involving more distribution shifts (\ie, concept shifts and covariate shifts). Hence in this work, we follow the evaluation protocol proposed in \cite{good}. Furthermore, we validate the effectiveness of our method in the datasets (\ie, Amazon-Photo, Elliptic) that are used in EERM~\cite{eerm}. The statistics and detailed introduction to these datasets can be found in Table~\ref{tab:dataset} and Appendix~\ref{app:datasets}.

\begin{table*}[htp]
\caption{The descriptions of datasets. ``Domain-Level'' means splitting by graphs, ``Time-Aware'' denotes splitting according to chronological order.``Word'' and ``Degree'' represent splitting according to word diversity and node degree respectively. ``Language'' means splitting by user language, suggesting the prediction should not be impacted by the language the user use. ``University'' denotes splitting according to the domain university, implying that the prediction of webpages should be based on word contents and link connections rather than university features. ``Color'' means that nodes are split according to node differences in covariate shift and color-label correlations in concept shift.}
\label{tab:dataset}
\centering
\begin{tabular}{cccccccc}
\toprule
Datasets     & Network Type        & \#Nodes & \#Edges & \#Attributes &\#Classes& Train/Val/Test Split     & Metric   \\
% Cora         & Artificial Transformation & 2,703   &         &              &         &                      & Accuracy \\
Amazon-Photo\footnotemark
             & Co-purchasing network      & 7,650   & 119,081   & 755          & 10      & Domain-Level         & Accuracy \\
Elliptic\footnotemark  
             & Bitcoin transactions       & 203,769 & 234,355   & 165          & 2       & Time-Aware           & F1-Score \\
GOOD-Cora    & Scientific publications    & 19,793  & 126,842   & 8,710         & 70      & Word/Degree          & Accuracy \\
% GOOD-Arxiv   & arXiv papers               & 169,343 & 2,315,598 & 128          & 40      & Time/Degree          & Accuracy \\
GOOD-Twitch  & Gamer network              & 34,120  & 892,346   & 128          & 2       & Language             & ROC-AUC  \\
GOOD-CBAS    & A BA-house graph           & 700     & 3,962     & 4             & 4       & Color                & Accuracy \\
GOOD-WebKB   & Webpage network            & 617     & 1,138     & 1,703         & 5       & University           & Accuracy \\
\bottomrule
\end{tabular}
\end{table*}
\footnotetext[5]{This dataset is adopted from~\cite{yang2016revisiting}. \cite{eerm} constructs ten graphs with different environment id’s for each graph.} 
\footnotetext[6]{The original is available on \hyperlink{https://www.kaggle.com/ellipticco/elliptic-data-set}{https://www.kaggle.com/ellipticco/elliptic-data-set}}

\subsection{Unsupervised Representation Learning}\label{sec:unsupervised}
\subsubsection{Transductive Setting}~\label{sec:trans}
% \noindent\textbf{Baselines.}\quad We conduct experiments with 12 baselines which consist of three categories: supervised methods and self-supervised generative methods, self-supervised contrastive methods. Specifically, we compare with three supervised baselines: empirical risk minimization~(ERM)~\cite{erm}, invariant risk minimization (IRM)~\cite{irm}, and a recent proposed graph OOD method dubbed EERM~\cite{eerm}. We also compare various unsupervised node-level representation learning methods: three self-supervised generative methods including GAE~\cite{gae}, VGAE~\cite{gae}, GraphMAE~\cite{gmae} and seven self-supervised contrastive methods: DGI~\cite{dgi}, MVGRL~\cite{mvgrl}, GRACE~\cite{grace}, RoSA~\cite{rosa}, BGRL~\cite{bgrl}, COSTA~\cite{costa}, SwAV~\cite{swav}. The descriptions of these methods can be found in Appendix~\ref{app:baselines}.
In this subsection, we focus on validating our proposed algorithm under the transductive setting, where the test nodes will participate in message passing~\cite{gilmer2017neural} during training following~\cite{good}. 

\noindent\textbf{Baselines.} We conduct experiments with 12 baselines from three categories: (i)~supervised methods, including empirical risk minimization~(\textbf{ERM})~\cite{erm}, invariant risk minimization (\textbf{IRM})~\cite{irm}, and a recent proposed graph OOD method \textbf{EERM}~\cite{eerm}; (ii)~self-supervised generative methods including Graph Autoencoder (\textbf{GAE})~\cite{gae}, Variational Graph Autoencoder (\textbf{VGAE})~\cite{gae}, Self-Supervised Masked Graph Autoencoders (\textbf{GraphMAE})~\cite{gmae}; (iii)~self-supervised contrastive methods including Deep Graph Infomax (\textbf{DGI})~\cite{dgi}, Contrastive Multi-View Representation Learning on Graphs (\textbf{MVGRL})~\cite{mvgrl}, Deep Graph Contrastive Representation Learning (\textbf{GRACE})~\cite{grace}, A Robust Self-Aligned Framework for Node-Node Graph Contrastive Learning (\textbf{RoSA})~\cite{rosa}, Bootstrapped Representation Learning on Graphs (\textbf{BGRL})~\cite{bgrl}, Covariance-Preserving Feature Augmentation for Graph Contrastive Learning (\textbf{COSTA})~\cite{costa}, Unsupervised Learning of Visual Features by Contrasting Cluster Assignments (\textbf{SwAV})~\cite{swav}. The detailed descriptions of these baselines can be found in Appendix~\ref{app:baselines}.

\noindent\textbf{Experimental setup.} We use the same graph encoder across different datasets for a fair comparison following~\cite{good}. We use grid search to find other hyper-parameters (\eg, learning rate, epochs) for different methods. For all experiments, we select the best checkpoints for ID and OOD tests according to results on ID and OOD validation sets following~\cite{good}, respectively. Experimental details and hyper-parameter selections are provided in Appendix~\ref{app:hyper}. For evaluating unsupervised methods, a linear classifier will be built on the frozen trained encoder after finishing pre-training. The reported results are the mean performance with standard deviation after 10 runs following~\cite{good}.

\noindent\textbf{Analysis.}\quad Based on the experimental results listed in Table~\ref{tab:trans_concept} and \ref{tab:trans_covariate}, we can draw the following conclusions: firstly, we find strong self-supervised methods (\eg, GRACE, BGRL, COSTA) are more robust to distribution shifts (concept shift in Table~\ref{tab:trans_concept} and covariate shift in Table~\ref{tab:trans_covariate}) compared to supervised methods. For instance, on GOOD-CBAS and GOOD-WebKB datasets, GRACE surpasses the best supervised method by large margins (over 6\% absolute improvement). Interestingly, we find the methods designed for OOD generalization (\ie, IRM) and graph OOD generalization (\ie, EERM) do not attain superior performance than the standard ERM on most of the datasets. For example, EERM shows superior OOD performance compared to ERM in only one experiment, and IRM outperforms ERM in four out of ten experiments across the conducted evaluations. This phenomenon is also observed in \cite{good,ahuja2020empirical,rosenfeld2021risks}, showcasing the challenge of achieving invariant prediction in non-Euclidean graph settings. 

Furthermore, our method surpasses other SOTA self-supervised methods on the OOD test set of all datasets by a considerable margin while achieving comparable performance in the in-distribution test set. For instance, on small datasets such as GOOD-CBAS and GOOD-WebKB, our method outperforms GRACE\footnote{MARIO is built up on GRACE according to our recipe. So, we make a comparison with GRACE here.} by over 2\% absolute accuracy on the OOD test set. On larger datasets such as GOOD-Cora and GOOD-Twitch, our method still outperforms other methods which shows its superiority. For instance, under covariate shift, MARIO surpasses other methods by over 7\% absolute accuracy on the GOOD-Twitch OOD test set. These statistics prove the effectiveness of our design.


\begin{table*}[htp]
\caption{Experimental results of all methods under concept shift. The bold font means the top-1 performance and the underline represents the second performance across the unsupervised methods. 'ID' represents in-distribution test performance and 'OOD' means out-of-distribution test performance. (OOM: out-of-memory on a GPU with 24GB memory)}
\label{tab:trans_concept}
\centering
\scalebox{0.95}{
\begin{tabular}{l|cc|cc|cc|cc|cc}
\toprule
\toprule
\multirow{3}{*}{concept shift} & \multicolumn{4}{c|}{GOOD-Cora}                   & \multicolumn{2}{c|}{GOOD-CBAS} & \multicolumn{2}{c|}{GOOD-Twitch} & \multicolumn{2}{c}{GOOD-WebKB} \\
                           & \multicolumn{2}{c}{word} & \multicolumn{2}{c|}{degree}& \multicolumn{2}{c|}{color}    & \multicolumn{2}{c|}{language}   & \multicolumn{2}{c}{university} \\
                           & ID         & OOD         & ID          & OOD          & ID            & OOD           & ID             & OOD            & ID            & OOD            \\
\midrule
ERM                        & 66.38±0.45 & 64.44±0.18  & 68.60±0.40  & 60.76±0.34   & 89.79±1.39    & 83.43±1.19    & 80.80±1.00     & 56.92±0.92     & 62.67±1.53    & 26.33±1.09     \\
IRM                        & 66.42±0.41 & 64.29±0.31  & 68.57±0.35  & 61.45±0.24   & 89.64±1.21    & 82.29±1.14    & 78.87±1.04     & 59.30±1.79     & 62.67±1.10    & 26.88±1.42     \\
EERM                       & 65.10±0.44 & 62.45±0.19  & 66.95±0.44  & 56.58±0.25   & 79.07±2.12    & 64.50±1.01    & OOM            & OOM            & 62.50±2.01    & 28.07±3.23      \\
\midrule
% Random-Init                & 37.53±1.74 & 32.12±1.24  & 37.82±1.71  & 27.74±1.14   &               &               &                &                & 60.33±2.21    & 27.07±1.70     \\
GAE                        & 60.65±0.89 & 58.00±0.55  & 62.59±1.11  & 53.44±0.80   & 75.28±1.36    & 68.07±2.05    & 81.25±0.81     & 51.51±1.05     & 62.17±3.34    & 25.78±1.85     \\
VGAE                       & 63.19±0.53 & 60.35±0.47  & 61.65±0.66  & 54.28±0.28   & 76.50±0.50    & 59.07±0.56    & 80.46±0.53     & 55.56±4.53     & 62.50±2.38    & 24.40±2.57     \\
GraphMAE                   & \underline{66.44±0.46} & \underline{64.87±0.30}  & 67.95±0.46  & 59.41±0.39   & 89.14±0.89    & 82.93±0.93    & 80.05±0.64     & 59.38±1.49     & 61.83±3.37    & 29.27±2.15     \\
DGI                        & 63.33±0.56 & 60.71±0.49  & 65.93±1.02  & 55.83±0.53   & 91.22±1.47    & 85.00±1.66    & 80.05±0.87     & 59.16±1.88     & 61.83±2.83    & 28.63±1.92      \\
MVGRL                      & OOM        & OOM         & OOM         & OOM          & 88.57±1.15    & 76.50±1.17    & OOM            & OOM            & 62.00±3.79    & 28.26±4.20     \\
GRACE                      & 65.61±0.61 & 63.92±0.44  & \textbf{68.59±0.35}  & 60.15±0.45   & 92.00±1.39    & 88.64±0.67    & \textbf{83.43±0.63}     & \underline{60.45±1.46}     & 64.00±3.43    & \underline{34.86±3.43}  \\
RoSA                       & 64.06±0.67 & 62.44±0.39  & 67.07±0.65  & 57.68±0.44   & 90.78±2.27    & 85.93±2.14    & 82.39±0.42     & 57.45±2.16     & 64.17±4.10    & 32.20±2.15     \\
BGRL                       & 65.18±0.43 & 63.43±0.45  & 66.83±0.80  & 59.63±0.38   & 92.36±1.16    & 87.14±1.60    & 82.52±0.60     & 55.48±1.48     & 63.67±2.33    & 31.47±3.43     \\
COSTA                      & 65.05±0.80 & 62.37±0.45  & 66.76±0.87  & 55.73±0.36   & \underline{93.50±2.62}    & \underline{89.29±3.11}    & 83.15±0.30 & 55.03±3.22     & 61.66±2.58    & 32.39±2.13 \\
% ArCL                       &            &             & 67.64±0.57  & 59.71±0.44   &               &               &                &                & 65.00±3.94    & 35.41±1.97 \\      
SwAV                       & 62.22±0.53 & 59.79±0.53  & 64.65±0.94  & 55.06±0.39   & 89.00±0.79    & 81.72±0.66    & \underline{83.32±0.15}     & 59.69±1.97     & \underline{65.17±3.76}    & 29.36±2.01    \\
\midrule
MARIO                       & \textbf{67.11±0.46} & \textbf{65.28±0.34}  & \underline{68.46±0.40}  & \textbf{61.30±0.28}   & \textbf{94.36±1.21}    & \textbf{91.28±1.10}    & 82.31±0.54     & \textbf{63.33±1.72}     & \textbf{65.67±2.81}    & \textbf{37.15±2.37}     \\
\bottomrule
\end{tabular}}
\end{table*}

\begin{table*}[htp]
\caption{Experimental results of all methods under covariate shift. The bold font means the top-1 performance and the underline represents the second performance across the unsupervised methods. 'ID' represents in-distribution test performance and 'OOD' means out-of-distribution test performance. (OOM: out-of-memory on a GPU with 24GB memory)}
\label{tab:trans_covariate}
\centering
\scalebox{0.95}{
\begin{tabular}{l|cc|cc|cc|cc|cc}
\toprule
\toprule
\multirow{3}{*}{covariate shift} & \multicolumn{4}{c|}{GOOD-Cora}                                   & \multicolumn{2}{c|}{GOOD-CBAS} & \multicolumn{2}{c|}{GOOD-Twitch} & \multicolumn{2}{c}{GOOD-WebKB} \\
                           & \multicolumn{2}{c}{word} & \multicolumn{2}{c|}{degree}& \multicolumn{2}{c|}{color}    & \multicolumn{2}{c|}{language}   & \multicolumn{2}{c}{university} \\
                           & ID         & OOD         & ID          & OOD          & ID            & OOD           & ID             & OOD            & ID            & OOD            \\
\midrule
ERM                        & 70.50±0.41 & 64.69±0.33  & 72.46±0.49  & 55.53±0.50   & 92.00±3.08    & 77.57±1.29    & 70.98±0.41     & 49.35±5.09     & 39.34±1.79    & 14.52±3.14   \\
IRM                        & 70.48±0.26 & 64.53±0.57  & 71.98±0.34  & 53.72±0.46   & 90.86±2.41    & 78.86±1.67    & 69.81±0.95     & 49.11±2.82     & 38.52±3.30    & 13.97±2.80     \\
EERM                       & OOM        & OOM         & OOM         & OOM          & 65.00±2.57    & 57.43±3.60    & OOM            & OOM            & 46.07±4.55    & 27.40±7.65     \\
\midrule
GAE                        & 56.63±0.79 & 48.93±0.93  & 66.30±0.88  & 34.01±0.87   & 73.00±2.16    & 60.86±3.01    & 67.24±1.23     & 47.65±2.49     & 45.08±6.32    & 28.02±6.29    \\
VGAE                       & 62.02±0.66 & 54.12±0.86  & 69.41±0.57  & 44.20±1.29   & 62.29±2.04    & 63.29±1.11    & 66.99±1.43     & \underline{50.48±4.58}     & 48.85±4.68    & 20.87±6.69     \\
GraphMAE                   & 68.14±0.43 & 64.00±0.33  & \textbf{73.36±0.56}  & 53.75±0.55   & 67.28±3.03    & 67.28±1.49    & 68.84±1.20     & 48.02±2.79     & 48.03±4.34    & 30.00±8.09     \\
DGI                        & 60.85±0.75 & 57.03±0.67  & 68.97±0.41  & 41.75±0.88   & 69.57±4.09    & 59.71±3.43    & 68.43±1.05     & 44.83±1.61     & 48.52±5.04    & 21.11±7.50     \\
MVGRL                      & OOM        & OOM         & OOM         & OOM          & 65.00±1.94    & 64.15±0.77    & OOM            & OOM           & \textbf{54.10±5.39}    & 16.59±6.51     \\
GRACE                      & \underline{68.77±0.33} & \underline{64.21±0.41}  & 72.69±0.34  & \underline{56.10±0.63}   & \underline{93.57±1.83}    & \underline{89.29±3.40}    & \underline{71.12±0.87} & 46.21±1.54 & 49.67±5.82    & 28.10±4.68    \\
RoSA                       & 68.19±0.56 & 62.48±0.61  & 71.04±0.62  & 52.72±0.79   & 84.71±4.14    &79.14±3.51     & 70.58±0.36     & 45.83±1.72     & 52.30±4.24    & \underline{34.24±7.92}     \\
BGRL                       & 67.23±0.43 & 61.33±0.36  & 72.11±0.39  & 49.15±0.73   & 89.00±2.56    & 79.86±3.29    & \textbf{71.43±0.53}     & 43.86±0.94     & 51.80±5.55    & 30.32±7.61    \\
COSTA                      & 65.28±0.60 & 60.33±0.53  & 70.65±0.62  & 54.03±0.28   & 92.29±1.59    & 82.71±2.74    & 69.29±1.37     & 49.07±2.13     & 50.49±3.01    & 29.84±4.75   \\
SwAV                       & 63.29±1.01 & 56.98±0.94  & 70.27±0.73  & 43.00±0.52   & 89.57±1.12    & 81.43±1.69    & 69.19±0.93     & 49.37±2.96     & 49.84±4.82    & 30.55±6.72   \\
\midrule
MARIO                       & \textbf{69.99±0.54} & \textbf{65.06±0.34}  & \underline{72.73±0.43}  & \textbf{57.73±0.45}  & \textbf{94.57±2.46}    & \textbf{91.00±2.48}     & 68.31±0.78 & \textbf{57.37±1.37}     & \underline{53.94±3.23}    & \textbf{35.24±4.98}   \\
\bottomrule
\end{tabular}}

\end{table*}

\subsubsection{Inductive Setting}
In this subsection, we conduct experiments under the inductive settings, where the test nodes are kept unseen during training. This setting is more suitable for domain generalization.
% But we think it is more convincing that conduct experiments under inductive settings which means test nodes are unseen during training. This setting is more appropriate for domain generalization.

\noindent\textbf{Baselines:} For GOOD-WebKB and GOOD-CBAS datasets, we adopt ERM, IRM, GraphMAE, and GRACE as our baselines. And for Amazon-Photo and Elliptic datasets, we select ERM, EERM, and GRACE as our baselines.

\noindent\textbf{Experimental setup:} For GOOD-WebKB and GOOD-CBAS datasets, we use the same model configuration in Section~\ref{sec:trans}.
% Besides, we add experiments on Amazon-Photo dataset~\cite{yang2016revisiting} and Elliptic~\cite{elliptic} dataset in this subsection. 
For Amazon-Photo dataset~\cite{yang2016revisiting} and Elliptic~\cite{elliptic} dataset, they consist of many snapshots (training data and testing data use different snapshots) which are naturally inductive. For Amazon-Photo dataset, we use 2-layer GCN~\cite{gcn} as the encoder and for elliptic dataset, we use 5-layer GraphSAGE~\cite{sage} as encoder following~\cite{eerm}.

% Figure environment removed

\noindent\textbf{Analysis:}
According to Figure~\ref{fig:amazon},\ref{fig:elliptic},\ref{fig:ind_con},\ref{fig:ind_cov}, we can draw following conclusions:
firstly, based on Figure~\ref{fig:amazon}, it is evident that our method outperforms other representative supervised and self-supervised methods on all test graphs (T1$\sim$T8). This superiority is reflected in the larger median value of our method compared to others. For instance, MARIO achieves over a 3\% absolute improvement compared to ERM in terms of the mean value of eight median values. Additionally, our method demonstrates higher stability across different random initializations, as indicated by the closer proximity of the first and third quartile values to the median value~(\eg, the difference of first and third quartile values of ERM, EERM, GRACE and MARIO are 4.2, 3.3, 6.7 and 1.0 on T8 respectively which indicates MARIO is much more stable than other methods). Furthermore, our method exhibits consistent performance across different graphs (\eg, The standard deviation of median values on T1$\sim$T8 for ERM, EERM, GRACE, and MARIO are 0.4, 1.1, 1.2, and 0.3, respectively.), indicating its robustness to environmental variations and its ability to extract invariant features: $g(G^e) \approx g(G^{e'})$ for all $e, e' \in \mathcal{E}^\text{train}$. In summary, our method showcases enhanced OOD generalization capabilities.
% $g(G^e)g(G^e^\prime)$ where $any e, e^\prime in \mathcal{E}^{train}$

Secondly, from the results presented in Figure~\ref{fig:elliptic}, we can observe that our method averagely harvests 10.9\% absolute improvement over GRACE and 12.5\% absolute improvement over EERM in terms of F1 scores on Elliptic dataset. This demonstrates the effectiveness of our method in handling distribution shifts and improving performance compared to existing approaches. It is worth noting that GRACE's performance worsens over time, indicating its inability to handle distribution shifts effectively. In contrast, our method consistently achieves better F1 scores, except for T9, which is caused by the dark market shutdown occurred after T7~\cite{elliptic}. The emergence of such an event introduces significant variations in data distributions, which subsequently results in performance degradation for all methods. Indeed, this event serves as an unpredictable external factor that introduces significant challenges for models trained on limited training data. The results indicate that the performance heavily depends on available training data. Nonetheless, our approach outperforms other methods even in such an extreme case. This highlights the effectiveness of our method in addressing distribution shifts and improving generalization performance.

Finally, based on the observations from Figure~\ref{fig:ind_con} and Figure~\ref{fig:ind_cov} MARIO demonstrates the best performances on both ID and OOD test sets for GOOD-WebKB and GOOD-CBAS datasets, under both concept shift and covariate shift. Notably, MARIO outperforms other methods by more than 3\% and 10\% absolute improvement on GOOD-WebKB and GOOD-CBAS, respectively, under covariate shift. We can draw similar conclusions as discussed in Section~\ref{sec:trans}. Even under the inductive setting, our method continues to demonstrate excellent OOD generalization capabilities and achieves comparable or even improved in-distribution test performance. These statistical results further validate the effectiveness of our method in handling distribution shifts and enhancing generalization performance.

Overall, the observations we have made provide strong evidence of the great capacity of our method for handling distribution shifts, validating its effectiveness and potential for real-world applications.



% Figure environment removed

% Figure environment removed


% Figure environment removed


\subsection{Ablation Studies}\label{sec:ablation}
\noindent Table~\ref{tab:aba} provides a detailed analysis of the effect of each component according to our proposed recipe for improving OOD generalization in graph contrastive learning. Let's examine the different variants of our method and their impact on performance.
Specifically, MARIO~(w/o ad) represents MARIO without  adversarial augmentation. MARIO~(w/o cmi) denotes we only maximize the mutual information between positive pairs without considering conditional mutual information. MARIO~(w/o cmi, ad) means a vanilla graph contrastive method that is similar to GRACE. 

From Table~\ref{tab:aba}, we can find MARIO~(w/o cmi) lags far behind MARIO on OOD test set which demonstrates appropriately minimizing the redundant information (\ie, conditional mutual information) is essential to improve OOD generalization of GCL methods. And adversarial augmentation can also boost OOD generalization because it can approximately serve as a supermum operator to learn more invariant features  discussed in Section~\ref{sec:aug}. Based on the analysis of these variants, it is evident that the proposed improvements on data augmentation and contrastive loss in the recipe are both effective in enhancing graph OOD generalization. Each component contributes to the overall performance improvement, and their combination leads to a stronger self-supervised graph learner in terms of graph OOD generalization. 

In short, the findings from Table~\ref{tab:aba} support the rationale behind your proposed recipe and provide empirical evidence of the effectiveness of each proposed component. By incorporating these enhancements, our method achieves superior performance in handling distribution shifts and improving graph OOD generalization in graph contrastive learning.
\begin{table*}[htp]
\caption{Ablation studies for MARIO by masking each component.}
\label{tab:aba}
\centering
\scalebox{0.9}{
\begin{tabular}{l|cc|cc|cc|cc|cc}
\toprule
\toprule
\multirow{3}{*}{concept shift} & \multicolumn{4}{c|}{GOOD-Cora}                       & \multicolumn{2}{c|}{GOOD-CBAS} & \multicolumn{2}{c|}{GOOD-Twitch} & \multicolumn{2}{c}{GOOD-WebKB} \\
                           & \multicolumn{2}{c}{word} & \multicolumn{2}{c|}{degree}& \multicolumn{2}{c|}{color}    & \multicolumn{2}{c|}{language}   & \multicolumn{2}{c}{university} \\
                           & ID         & OOD         & ID          & OOD          & ID            & OOD           & ID             & OOD            & ID            & OOD            \\
\midrule
MARIO                      & \textbf{67.11±0.46} & \textbf{65.28±0.34}  & \textbf{68.46±0.40}  & \textbf{61.30±0.28}      & \textbf{94.36±1.21}  & \textbf{91.28±1.10}    & 82.31±0.54     & \textbf{63.33±1.72}     & \textbf{65.67±2.81}    & \textbf{37.15±2.37}     \\
MARIO(w/o ad)              & 66.23±0.53 & 64.02±0.18  & 67.88±0.38  & 60.46±0.29   & 93.21±1.25    & 90.29±0.91    & 82.42±0.73     & 60.50±1.02     & 64.83±2.83    & 36.51±3.25    \\
MARIO(w/o cmi)             & 65.32±0.60 & 63.51±0.32  & 68.14±0.32  & 61.19±0.34   & 94.15±1.23    & 90.57±1.96    & \textbf{82.51±0.56}     & 61.41±2.63     & 64.50±4.35    & 35.78±2.53     \\
MARIO(w/o cmi, ad)         & 64.67±0.55 & 63.11±0.32  & 67.95±0.65  & 60.01±0.57   & 93.36±1.66    & 89.64±1.73    & 81.90±0.75     & 60.12±1.60     & 64.17±3.67    & 34.13±2.38     \\
\bottomrule
\end{tabular}}
\end{table*}
% & 65.32±0.60 & 63.51±0.32 exchange 64.67±0.55 & 63.11±0.32
% 68.14±0.32       id ood test: 60.95±0.43       ood ood test: 61.19±0.34


\subsection{Sensitivity Analysis}\label{sec:sensitivity}
\noindent In this subsection, we will analyze some important hyper-parameters of our method. We conduct sensitivity analysis on GOOD-WebKB dataset with concept shift, we chose two sensitive hyper-parameters (\ie, the coefficient $\gamma$ of condition mutual information in Equation~\ref{equ:cmi} and the number of prototypes $|C|$ in Equation~\ref{equ:pq}). The coefficient of CMI range in $[0.001, 0.01, 0.1, 0.5, 1]$ and the number of prototypes $|C|$ ranges in $[10, 50, 100, 200, 300]$. From Figure~\ref{fig:sensitivity}, we can observe that $\gamma$ reaches 0.1 and $|C|$ reaches 100 or 200 can achieve the best OOD test accuracy. Both higher and lower values of $\gamma$ result in suboptimal performance. This finding aligns with previous research such as DIB~\cite{dib}, indicating that an appropriate compression level is crucial for achieving optimal performance. Extremely high or low compression values are not ideal. 

Regarding the number of prototypes $|C|$, based on the results shown in Figure~\ref{fig:sensitivity}, it is found that setting $|C|=100$ leads to the best performance in terms of OOD test accuracy. This choice provides a moderate number of pseudo labels, which is beneficial for the learning process. 

Based on the sensitivity analysis, we determined that setting $\gamma=0.1$ and $|C|=100$ on most datasets. These hyperparameter values strike a balance between compression level and the number of prototypes, resulting in improved graph OOD generalization.
% Figure environment removed


\subsection{Integrated with Other Models}\label{sec:other_models}
% Figure environment removed

\begin{table}[htp]
\caption{Results of different learning approaches with different encoding models (\ie, GCN, GraphSAGE, GAT).}
\label{tab:others}
\centering
\scalebox{0.9}{
\begin{tabular}{cc|cc|cc}
\toprule
\toprule
\multirow{3}{*}{Model}& \multirow{3}{*}{Method} & \multicolumn{2}{c|}{GOOD-CBAS} & \multicolumn{2}{c}{GOOD-WebKB} \\
                & & \multicolumn{2}{c|}{color}    & \multicolumn{2}{c}{university} \\
                &   & ID          & OOD         & ID          & OOD            \\
\midrule
\multirow{3}{*}{GCN} 
&ERM               & 89.79±1.39 & 83.43±1.19  &  62.67±1.53 & 26.33±1.09         \\
&GRACE             & 92.00±1.39 & 88.64±0.67  &  64.00±3.43 & 34.86±3.43        \\
&MARIO             & 94.36±1.21 & 91.28±1.10  &  65.67±2.81 & 37.15±2.37        \\ \bottomrule
\multirow{3}{*}{SAGE} 
&ERM               & 95.07±1.51 & 75.14±1.19  & 73.67±2.08  & 46.33±3.42       \\
&GRACE             & 95.29±1.11 & 74.43±2.36  & 70.50±5.06  & 49.54±3.83        \\
&MARIO             & 96.00±1.07 & 76.29±3.01  & 71.00±3.82  & 51.74±4.63        \\ \bottomrule
\multirow{3}{*}{GAT} 
&ERM               & 78.64±3.63 & 72.93±2.64  & 61.33±3.71  & 28.99±2.63        \\
&GRACE             & 84.57±1.79 & 78.36±1.60  & 59.50±2.36  & 35.78±3.26        \\
&MARIO             & 84.93±1.95 & 80.43±1.89  & 62.17±4.78  & 38.17±3.10        \\
\bottomrule
\end{tabular}}
\end{table}



\noindent In the subsection, we demonstrate the model-agnostic nature of the recipe by integrating it with various graph neural network (GNN) models, including GCN, GraphSAGE, and GAT.

From Table~\ref{tab:others}, it can be observed that regardless of the specific GNN model used as the encoder, our method consistently achieves the best performance on the OOD test set. This indicates the effectiveness and robustness of our method across different GNN models.
By achieving superior performance across different GNN models, MARIO demonstrates its versatility and ability to improve the OOD generalization of various graph neural models. This highlights the broad applicability and effectiveness of our recipe in enhancing the performance of different GNN encoders.

Furthermore, we integrate our recipe with other GCL methods in Appendix~\ref{app:other_methods}. The results demonstrate our recipe can boost the OOD generalization ability of various GCL methods which means our recipe can serve as a plug-in for many current classical GCL methods.

% Figure environment removed

\subsection{Visualization}\label{sec:vis}
\subsubsection{Metric Score Curves}
We present metric score curves for ERM and MARIO, including training, ID validation, ID testing, OOD validation, and OOD testing accuracy, in Figure~\ref{fig:curve2}. Notably, MARIO demonstrates superior convergence with approximately 10\% absolute improvement on the OOD test set compared to ERM. Furthermore, MARIO effectively narrows the performance gap between in-distribution and out-of-distribution performance, showcasing its efficacy in enhancing OOD generalization for graph data. More metric score curves can be found in Appendix~\ref{app:curves}.


\subsubsection{Feature Visualization}
In order to assess the quality of learned embeddings, we adopt t-SNE~\cite{tsne} to visualize the node embedding on GOOD-Cora dataset (concept shift in word domain) using random-init of GCN, EERM, GRACE, and MARIO, where different classes have different colors in Figure~\ref{fig:vis}. For clarity, we select eight classes with the largest number of nodes to enhance the informativeness and interpretability of the visualization. We can observe that the 2D projection of node embeddings learned by MARIO has a better separation of clusters, which indicates the model can help learn representative features for downstream tasks. It has to note that we depict both ID nodes and OOD nodes in the same figure. 

Besides, we also separately visualize ID nodes and OOD nodes in the different figures in the Appendix~\ref{app:feature}. And we can find MARIO performs a clearer separation of clusters whether on ID nodes or OOD nodes compared to other methods.



%% -*- mode: LaTeX; fill-column: 78; -*-

\section{Concluding Remarks}
\label{sec:conclusions}

In this paper, we presented a novel SMC algorithm, \EventDPOR, tailored to the
characteristics of event-driven multi-threaded programs running under the SC
semantics. The algorithm was proven correct and optimal for event-driven
programs in which the variable accesses of events do not depend on how their
execution is interleaved with other threads.

We have implemented \EventDPOR in the \Nidhugg tool, and we will open-source
our implementation.
%
With a wide range of event-driven programs, we have shown that \EventDPOR
incurs only a moderate constant overhead over its baseline implementation
(\OptimalDPOR), it is exponentially faster than existing state-of-the-art SMC
algorithms in time and number of traces examined on programs where events'
actions do not conflict, and does not suffer from performance degradation
caused by having to examine
% a significant number of
non-serializable executions.
%
%% \bjcom{Should we include:
%% Moreover, in our benchmarks, also those that are not non-branching,
%% \EventDPOR explores only the optimal number of executions, and never
%% had to resort to a potentially expensive decision procedure.}

\EventDPOR assumes that handlers can process their events in arbitrary order.
Directions for future work include to retarget \EventDPOR for event-driven
programs with other policies (e.g., FIFO), and for specific event-driven
execution models.

\subsection*{Acknowledgements}

\noindent
USA {\textendash} U.S. National Science Foundation-Office of Polar Programs,
U.S. National Science Foundation-Physics Division,
U.S. National Science Foundation-EPSCoR,
Wisconsin Alumni Research Foundation,
Center for High Throughput Computing (CHTC) at the University of Wisconsin{\textendash}Madison,
Open Science Grid (OSG),
Extreme Science and Engineering Discovery Environment (XSEDE),
Frontera computing project at the Texas Advanced Computing Center,
U.S. Department of Energy-National Energy Research Scientific Computing Center,
Particle astrophysics research computing center at the University of Maryland,
Institute for Cyber-Enabled Research at Michigan State University,
and Astroparticle physics computational facility at Marquette University;
Belgium {\textendash} Funds for Scientific Research (FRS-FNRS and FWO),
FWO Odysseus and Big Science programmes,
and Belgian Federal Science Policy Office (Belspo);
Germany {\textendash} Bundesministerium f{\"u}r Bildung und Forschung (BMBF),
Deutsche Forschungsgemeinschaft (DFG),
Helmholtz Alliance for Astroparticle Physics (HAP),
Initiative and Networking Fund of the Helmholtz Association,
Deutsches Elektronen Synchrotron (DESY),
and High Performance Computing cluster of the RWTH Aachen;
Sweden {\textendash} Swedish Research Council,
Swedish Polar Research Secretariat,
Swedish National Infrastructure for Computing (SNIC),
and Knut and Alice Wallenberg Foundation;
Australia {\textendash} Australian Research Council;
Canada {\textendash} Natural Sciences and Engineering Research Council of Canada,
Calcul Qu{\'e}bec, Compute Ontario, Canada Foundation for Innovation, WestGrid, and Compute Canada;
Denmark {\textendash} Villum Fonden and Carlsberg Foundation;
New Zealand {\textendash} Marsden Fund;
Japan {\textendash} Japan Society for Promotion of Science (JSPS)
and Institute for Global Prominent Research (IGPR) of Chiba University;
Korea {\textendash} National Research Foundation of Korea (NRF);
Switzerland {\textendash} Swiss National Science Foundation (SNSF);
United Kingdom {\textendash} Department of Physics, University of Oxford.


\bibliography{References}
\bibliographystyle{plainnat}

\newpage
\appendix

\tableofcontents
\newpage

\section{ADDITIONAL ALGORITHMS IN IMPLEMENTATION}

\subsection{A \textsf{Peace}-based Robust Algorithm}
\label{sec:modified_algo}

In this section, we briefly explain how we design \textsf{P1-Peace} based on intuition similar to \textsf{P1-RAGE} and make it computationally efficient. First, we propose another subroutine, called \textsf{Peace-Elimination}, based on the elimination strategy in \textsf{Peace} \citet{katz2020empirical}, which has the same spirit as \textsf{RAGE}. Similar to \textsf{RAGE-Elimination}, \textsf{Peace-Elimination} also repeatedly computes $\mc{XY}$-allocation, but (virtually) eliminate arms so that the value of the remaining arms' optimal $\mc{XY}$-design is halved. In addition, in \textsf{P1-Peace}, we only update the sampling distribution $\lambda_t$ after a period of time. The intuition is that if the environment is stationary, then we do not need to update our allocation probability frequently just like \textsf{RAGE} and \textsf{Peace}; if the environment is non-stationary, then the non-stationarity is handled by the mixed G-optimal design $\lambda^*$, which is fixed from the very beginning. Therefore, updating $\lambda_t$ in a low frequency should not severely harm the performance. The new algorithm and elimination subroutine are summarized in Algorithm \ref{algo:p1_peace} and \ref{algo:peace_elimination}.

For convenience of presentation, for arm set $\mc{Z}\subset\R^d$ and distribution $\lambda\in\triangle_{\X}$, we define
\begin{equation}
    \label{equ:rho_z}
    \rho(\mc{Z}, \lambda)=\max_{x, x'\in\mc{Z}}\Norm{x-x'}^2_{A(\lambda)^{-1}}.
\end{equation}

% \begin{algorithm}[ht]
%     \caption{P1-Peace}
%     \label{algo:p1_peace}
%     \SetAlgoLined
%     \KwIn{budget, $T\in\mathbb{N}$; arm set $\mc{X}\subset\R^d$}
%     Compute epoch length $R\leftarrow\floor{\frac{T}{\log_2(\inf_{\lambda\in\triangle_{\X}}\rho(\X, \lambda))}}$\\
%     Compute G-optimal design $\lambda^*$ based on equation \eqref{equ:g_design} and initialize $\lambda_1=\lambda^*$\\
%     \For{$t=1, 2, \dots, T$}{
%         Sample $x_t\sim\lambda_t$ and receive reward $r_t$\\
%         Estimate $\widehat{\theta}_t\leftarrow\frac{1}{t}\sum_{s=1}^t\E_{x\sim\lambda_s}\Mp{xx^\top}^{-1}x_s r_s$\\
%         $\lambda_{t+1}\leftarrow\lambda_t$\\
%         \If{$t-1=cR$ for some integer $c$}{
%             Update $\lambda_{t+1}\leftarrow$\textsf{Peace-Elimination}$(\htheta_t)$\\
%         }
%     }
%     \textbf{return} $\argmax_{x\in\X}x^\top\widehat{\theta}_T$\\
%     \SetKwFunction{proc}{\textsf{Peace-Elimination}}
%     \SetKwProg{myproc}{Subroutine}{}{}
%     \myproc{\proc{$\htheta_t$}}{\label{algo:peace_elimination}
%         Find index $\widehat{(k)}_t$ such that $x_{\widehat{(1)}_t}^\top\htheta_t\geq x_{\widehat{(2)}_t}^\top\htheta_t\geq\dots\geq x_{\widehat{(K)}_t}^\top\htheta_t$\\
%         Initialize $\X_t^{(0)}\leftarrow\X$ and $i\leftarrow 0$\\
%         \While{$|\X_{t}^{(i)}|> 1$}{
%             Compute $\lambda^{(i)}_t\leftarrow\arginf_{\lambda\in\triangle_{\X}}\rho(\X_t^{(i)}, \lambda)$\\
%             Find the largest index $k_i$ such that
%             $$\inf_{\lambda\in\triangle_{\X}}\rho\Sp{\{x_{\widehat{(1)}_t}, \dots, x_{\widehat{(k_i)}_t}\}}\leq \frac{1}{2}\cdot\inf_{\lambda\in\triangle_{\X}}\rho(\X_t^{(i)}, \lambda)$$\\
%             Update $\X_{t}^{(i+1)}\leftarrow\Bp{x_{\widehat{(1)}_t}, \dots, x_{\widehat{(k_i)}_t}}$\\
%             % Eliminate and obtain $\mc{X}_{i+1}^{(t)} \leftarrow \Bp{ x \in \mc{X}_i^{(t)}\mid  \htheta_t^\top(\hat{x}^*_t-x) \leq 2^{-i} }$\\
%             $i\leftarrow i+1$\\
%         }
%         % Compute $\bar{\lambda}_t\leftarrow \frac{1}{i}\sum_{i'=0}^{i-1}\lambda_t^{(i')}$\\
%         \textbf{return} $(\bar{\lambda}_t+\lambda^*)/2$, where $\bar{\lambda}_t= \frac{1}{i}\sum_{i'=0}^{i-1}\lambda_t^{(i')}$
%     }
% \end{algorithm}

\begin{algorithm}[ht]
    \caption{P1-Peace}
    \label{algo:p1_peace}
    \begin{algorithmic}[1]
        \STATE \textbf{Input:} budget, $T\in\mathbb{N}$; arm set $\mc{X}\subset\R^d$
        \STATE Compute epoch length $R\leftarrow\floor{\frac{T}{\log_2(\inf_{\lambda\in\triangle_{\X}}\rho(\X, \lambda))}}$
        \STATE Compute G-optimal design $\lambda^*$ based on equation \eqref{equ:g_design} and initialize $\lambda_1=\lambda^*$
        \FOR{$t=1, 2, \dots, T$}
            \STATE Sample $x_t\sim\lambda_t$ and receive reward $r_t$
            \STATE Estimate $\widehat{\theta}_t\leftarrow\frac{1}{t}\sum_{s=1}^t\E_{x\sim\lambda_s}\Mp{xx^\top}^{-1}x_s r_s$
            \STATE $\lambda_{t+1}\leftarrow\lambda_t$
            \IF{$t-1=cR$ for some integer $c$}
                \STATE Update $\lambda_{t+1}\leftarrow$\textsf{Peace-Elimination}$(\htheta_t)$
            \ENDIF
        \ENDFOR
        \RETURN $\argmax_{x\in\X}x^\top\widehat{\theta}_T$
    \end{algorithmic}
\end{algorithm}

\begin{algorithm}
    \caption{Peace-Elimination}
    \label{algo:peace_elimination}
    \begin{algorithmic}[1]
        \STATE \textbf{Input:} arm set $\mc{X}\subset\R^d$; current estimate $\htheta_t$
        \STATE Find index $\widehat{(k)}_t$ such that $x_{\widehat{(1)}_t}^\top\htheta_t\geq x_{\widehat{(2)}_t}^\top\htheta_t\geq\dots\geq x_{\widehat{(K)}_t}^\top\htheta_t$
        \STATE Initialize $\X_t^{(0)}\leftarrow\X$ and $i\leftarrow 0$
        \WHILE{$|\X_{t}^{(i)}|> 1$}
            \STATE Compute $\lambda^{(i)}_t\leftarrow\arginf_{\lambda\in\triangle_{\X}}\rho(\X_t^{(i)}, \lambda)$
            \STATE Find the largest index $k_i$ such that
            $$\inf_{\lambda\in\triangle_{\X}}\rho\Sp{\{x_{\widehat{(1)}_t}, \dots, x_{\widehat{(k_i)}_t}\}}\leq \frac{1}{2}\cdot\inf_{\lambda\in\triangle_{\X}}\rho(\X_t^{(i)}, \lambda)$$
            \STATE Update $\X_{t}^{(i+1)}\leftarrow\Bp{x_{\widehat{(1)}_t}, \dots, x_{\widehat{(k_i)}_t}}$
            \STATE $i\leftarrow i+1$
        \ENDWHILE
        \RETURN $(\bar{\lambda}_t+\lambda^*)/2$, where $\bar{\lambda}_t= \frac{1}{i}\sum_{i'=0}^{i-1}\lambda_t^{(i')}$
    \end{algorithmic}
\end{algorithm}

\subsection{A Naive Baseline Mixed Algorithm}
\label{sec:naive_algo}
In this section, we present a naive mixture of \textsf{Peace} and the G-optimal design, called \textsf{Mixed-Peace}, which eliminates arms and computes design $\lambda_k$ during each epoch exactly the same as \textsf{Peace}. The only differences are that \textsf{Mixed-Peace} uses IPS estimator and when pulling an arm, it will pull an arm by following $x_t\sim(\lambda_k+\lambda^*)/2$, where $\lambda^*$ is the G-optimal design defined in equation \eqref{equ:g_design}. Its details are summarized in Algorithm \ref{algo:mixed_peace}.

% \begin{algorithm}[ht]
%     \caption{Mixed-Peace}
%     \label{algo:mixed_peace}
%     \SetAlgoLined
%     \KwIn{budget, $T\in\mathbb{N}$; arm set $\mc{X}\subset\R^d$}
%     Initialize $R\leftarrow\ceil{\log_2\Sp{\inf_{\lambda\in\triangle_{\X}}\rho(\X, \lambda)}}$, $N\leftarrow\floor{\frac{T}{R}}$, $\mc{X}_0\leftarrow\X$, $\htheta_0\leftarrow\ve{0}$ and $t\leftarrow 1$\\
%     Compute G-optimal design $\lambda^*$ using equation \eqref{equ:g_design}\\
%     \For{$r=0, \dots, R$}{
%         Find $\lambda_r\leftarrow(\arginf_{\lambda\in\triangle_{\X}}\rho(\X_r, \lambda)+\lambda^*)/2$\\
%         \While{$t\leq\min\Bp{T, (r+1)N}$}{
%             Sample $x_t\sim \lambda_r$ and receive reward $r_t$\\
%             Estimate $\widehat{\theta}_t\leftarrow \frac{t-1}{t}\cdot \htheta_{t-1}+\frac{1}{t}\cdot\E_{x\sim\lambda_r}\Mp{xx^\top}^{-1}x_t r_t$\\
%             $t\leftarrow t+1$\\
%         }
%         \If{$\abs{\X_r}>1$}{
%             Reindex $\X_r$ such that $x_1^\top\htheta_t\geq x_2^\top\htheta_t\geq\dots\geq x_{n_r}^\top\htheta_t$, where $n_r=\abs{\X_r}$\\
%             Find the largest index $k_r$ such that 
%             $$\inf_{\lambda\in\triangle_{\X}}\rho(\Bp{x_1, \dots, x_{k_r}}, \lambda)\leq \frac{1}{2}\cdot\inf_{\lambda\in\triangle_{\X}}\rho(\X_r, \lambda)$$\\
%             Update $\X_{r+1}\leftarrow\Bp{x_1, \dots, x_{k_r}}$
%         }
%     }
%     \textbf{return} $\argmax_{x\in\X}x^\top\htheta_T$
% \end{algorithm}

\begin{algorithm}
    \caption{Mixed-Peace}
    \label{algo:mixed_peace}
    \begin{algorithmic}[1]
        \STATE \textbf{Input:} budget, $T\in\mathbb{N}$; arm set $\mc{X}\subset\R^d$
        \STATE Initialize $R\leftarrow\ceil{\log_2\Sp{\inf_{\lambda\in\triangle_{\X}}\rho(\X, \lambda)}}$, $N\leftarrow\floor{\frac{T}{R}}$, $\mc{X}_0\leftarrow\X$, $\htheta_0\leftarrow\ve{0}$ and $t\leftarrow 1$
        \STATE Compute G-optimal design $\lambda^*$ using equation \eqref{equ:g_design}
        \FOR{$r=0, \dots, R$}
            \STATE Find $\lambda_r\leftarrow(\arginf_{\lambda\in\triangle_{\X}}\rho(\X_r, \lambda)+\lambda^*)/2$
            \WHILE{$t\leq\min\Bp{T, (r+1)N}$}
                \STATE Sample $x_t\sim \lambda_r$ and receive reward $r_t$
                \STATE Estimate $\widehat{\theta}_t\leftarrow \frac{t-1}{t}\cdot \htheta_{t-1}+\frac{1}{t}\cdot\E_{x\sim\lambda_r}\Mp{xx^\top}^{-1}x_t r_t$
                \STATE $t\leftarrow t+1$
            \ENDWHILE
            \IF{$\abs{\X_r}>1$}
                \STATE Reindex $\X_r$ such that $x_1^\top\htheta_t\geq x_2^\top\htheta_t\geq\dots\geq x_{n_r}^\top\htheta_t$, where $n_r=\abs{\X_r}$
                \STATE Find the largest index $k_r$ such that 
                $$\inf_{\lambda\in\triangle_{\X}}\rho(\Bp{x_1, \dots, x_{k_r}}, \lambda)\leq \frac{1}{2}\cdot\inf_{\lambda\in\triangle_{\X}}\rho(\X_r, \lambda)$$
                \STATE Update $\X_{r+1}\leftarrow\Bp{x_1, \dots, x_{k_r}}$
            \ENDIF
        \ENDFOR
        \textbf{return} $\argmax_{x\in\X}x^\top\htheta_T$
    \end{algorithmic}
\end{algorithm}
\section{Error Probability of Algorithm \ref{algo:gbai} in Non-stationary Environments}
\label{sec:g_design_proof}

\advupperbound*

\begin{proof}
    % Define $\otheta_T=\frac{1}{T}\sum_{t=1}^{T}\theta_t$ and $\Delta_k=\frac{1}{T}\sum_{t=1}^{T}\theta_t^\top(x_{(1)}-x_{k})$ for $k\neq (1)$. For $k=(1)$, we have $\Delta_{(1)}=\Delta_{(2)}$. 
    Based on the recommendation rule $x_{J_T}=\argmax_{x\in\X}x^\top\htheta_T$, we have
    \begin{align}
        \P\Sp{J_T\neq (1)}=&\P\Sp{\exists k \in [2:K] \text{ s.t. } x_{(k)}^\top\htheta_T\geq x_{(1)}^\top\htheta_T}\nonumber\\
        \leq & \P\Sp{\exists k\in[2:K]\text{ s.t. } x_{(k)}^\top\htheta_T-x_{(k)}^\top\otheta_T\geq\frac{\Delta_{(k)}}{2} \text{ or } x_{(1)}^\top\htheta_T-x_{(1)}^\top\otheta_T\leq -\frac{\Delta_{(1)}}{2}}\nonumber\\
        \leq & \P\Sp{x_{(1)}^\top\htheta_T-x_{(1)}^\top\otheta_T\leq -\frac{\Delta_{(1)}}{2}} + \sum_{k=2}^{K}\P\Sp{x_{(k)}^\top\htheta_T-x_{(k)}^\top\otheta_T\geq\frac{\Delta_{(k)}}{2}}.\label{equ:g_bernstein}
    \end{align}
    The above terms can be bounded by Bernstein's inequality. In particular, for the first term, we have
    $$\P\Sp{x_{(1)}^\top\htheta_T-x_{(1)}^\top\otheta_T\leq -\frac{\Delta_{(1)}}{2}}=\P\Sp{\sum_{t=1}^{T}x_{(1)}^\top\Sp{A(\lambda^*)^{-1}x_tr_t-\theta_t}\leq-\frac{T\Delta_{(1)}}{2}}.$$
    Since IPS estimator is unbiased, $x_{(1)}^\top\Sp{A(\lambda^*)^{-1}x_tr_t-\theta_t}$ is a zero-mean random variable. Based on our bounded reward assumption, we have
    $$\abs{x_{(1)}^\top\Sp{A(\lambda^*)^{-1}x_tr_t-\theta_t}}\leq \abs{x_{(1)}^\top A(\lambda^*)^{-1}x_t} + 2\leq \Norm{x_{(1)}}_{A(\lambda^*)^{-1}}\Norm{x_t}_{A(\lambda^*)^{-1}}+2\leq d+2\leq 3d,$$
    where we use the property of G-optimal design $\max_{x\in\X}\Norm{x}^2_{A(\lambda^*)^{-1}}\leq d$. We can similarly bound its variance by 
    \begin{align*}
        \E\Mp{\Sp{x_{(1)}^\top\Sp{A(\lambda^*)^{-1}x_tr_t-\theta_t}}^2}\leq & \E\Mp{\Sp{x_{(1)}^\top A(\lambda^*)^{-1}x_t}^2}\\
        = & x_{(1)}^\top A(\lambda^*)^{-1}\E\Mp{x_tx_t^\top}A(\lambda^*)^{-1}x_{(1)}\\
        = & x_{(1)}^\top A(\lambda^*)^{-1}A(\lambda^*)A(\lambda^*)^{-1}x_{(1)}\tag{Since $x_t\sim\lambda^*$ by algorithm}\\
        = & \Norm{x_{(1)}}^2_{A(\lambda^*)^{-1}} \leq  d
    \end{align*}
    Thus, by Bernstein's inequality, we have
    $$\P\Sp{x_{(1)}^\top\htheta_T-x_{(1)}^\top\otheta_T\leq -\frac{\Delta_{(1)}}{2}}\leq \exp\Sp{-\frac{T^2\Delta^2_{(1)}/8}{Td+Td\Delta_{(1)}/2}}\leq\exp\Sp{-\frac{T\Delta_{(1)}^2}{12d}},$$
    where the last inequality uses the assumption that $\Delta_{(1)}\leq 1$. By similarly applying Bernstein's inequality to other terms in \eqref{equ:g_bernstein}, we can then have
    \begin{align*}
        \P\Sp{J_T\neq x_{(1)}} \leq &\P\Sp{x_{(1)}^\top\htheta_T-x_{(1)}^\top\otheta_T\leq -\frac{\Delta_{(1)}}{2}} + \sum_{k=2}^{K}\P\Sp{x_{(k)}^\top\htheta_T-x_{(k)}^\top\otheta_T\geq\frac{\Delta_{(k)}}{2}}\\
        \leq & \sum_{k=1}^{K}\exp\Sp{-\frac{T\Delta_{(k)}^2}{12d}}\\
        \leq & K\exp\Sp{-\frac{T\Delta_{(1)}^2}{12d}}.
    \end{align*}
\end{proof}
\section{ERROR PROBABILITY OF ALGORITHM \ref{algo:p1_rage}}
\label{sec:bobw_proof}
\subsection{Stationary Environments}
We first prove an error probability of Algorithm \ref{algo:p1_rage} in stationary environments that contains unspecified parameters from the virtual phases. Without loss of generality, assume that the arms $x_1, \ldots, x_K$ are ordered such that $\theta^\top x_1 > \theta^\top x_2 \geq \dots \geq \theta^\top x_K$ and $\Delta_1=\Delta_2\leq\Delta_3\leq\dots\leq\Delta_K$.

Throughout this section, we will the following definitions: $i_0 = \ceil{\log_2(1/\Delta_{1})} + 1$, $\mc{A}_i=\Bp{x\in\X\mid \Delta_x\leq 2\cdot 2^{-i}}$, $\bar{i}(k)=\max\Bp{i\in[i_0 - 1]\mid \Delta_k\leq 2^{-i}}$ and  
    $$f(\mc{A}_i)=\min_{\lambda\in\triangle_{\X}}\max_{x, x'\in\mc{A}_i}\Norm{x-x'}^2_{A(\lambda)^{-1}}.$$

\begin{theorem}
    \label{theo:bobw_error_prob_raw}
    Let $\mc{D}=\Bp{\ve{a}\in[0, 1]^{i_0+1}\mid 0=a_0<a_1\leq a_2 \leq \ldots \leq a_{i_0}=1}$. Then, if $m\geq i_0$, The error probability of Algorithm \ref{algo:p1_rage} in a stationary environment with parameter $\theta$ is bounded as 
    \begin{align}
        \P_{\theta}\Sp{J_T\neq 1} \leq& 2i_0 KT \exp\Sp{-\frac{T}{\overline{H}_{\textsf{P1-RAGE}}(\theta)}},\nonumber\\
        \overline{H}_{\textsf{P1-RAGE}}(\theta)=&\min_{\ve{a}\in\mc{D}} \max_{k\in[K]}\frac{48m\sum_{i'=1}^{\bar{i}(k)}(a_{i'}-a_{i'-1})f(\mc{A}_{i'-2}) + 8(m\sqrt{df(\mc{X})}+1)a_{\bar{i}(k)}\Delta_k}{3a_{\bar{i}(k)}^2\Delta_k^2}.\label{equ:H_p1rage}
    \end{align}
\end{theorem}

\begin{proof}
With $0=n_0<n_1\leq n_2 \leq\ldots \leq n_{i_0}=T$.\footnote{We do not specify the values of $n_1, \dots, n_{i_0-1}$ for now.} we define the event $\xi_i$ with $i \geq 1$ as follows: after $n_i$ samples all the arms with true gap smaller than $2\cdot2^{-i}$ are estimated with precision $2^{-i}/2$, which is
\begin{align*}
    \xi_i = \{\forall t \geq n_i, \forall k\in[K] \text{ s.t. } \Delta_k \leq 2 \cdot 2^{-i} \implies | \Delta_k - \widehat{\Delta}^{(t)}_k| < 2^{-i}/2\} ,
\end{align*}
where $\widehat{\Delta}^{(t)}_k = (x_1-x_k)^\top\widehat{\theta}^{(t)}$ for $k>1$ and $\widehat{\Delta}^{(t)}_1 = (x_1-x_2)^\top\widehat{\theta}^{(t)}$.
% Note that $\xi_0$ holds for $n_0=0$ with probability $1$ since we assume $\Delta_K\leq 1$. 
We first show how these events $\Bp{\xi_i}_{i=1}^{i_0}$ relate the correctness of Algorithm \ref{algo:p1_rage}.

\textbf{Correctness.} If $\bigcap_{i=1}^{i_0} \xi_i$ holds then the algorithm successfully identifies the best arm. Indeed, if we assume it does not, then there must exist non-optimal arm $k_0$ such that $\widehat{\Delta}^{(T)}_{k_0} < 0$. As $\bigcap_{i=1}^{i_0} \xi_i$ holds, for some $i'\leq i_0$, it holds that $2^{-i'} < \Delta_{k_0} \leq 2\cdot2^{-i'}$ and then $ | \Delta_{k_0} - \widehat{\Delta}^{(T)}_{k_0}| < 2^{-i'}/2$. Therefore, we have $2^{-i'} < \Delta_{k_0} \leq \Delta_{k_0} - \widehat{\Delta}^{(T)}_{k_0} \leq | \Delta_{k_0} - \widehat{\Delta}^{(T)}_{k_0}| \leq 2^{-i'}/2$, which is a contradiction.

Thus, the error probability is upper bounded by $\P\left(\bigcup_{i=1}^{i_0} \xi_i^c \right)$, which gives us

\begin{align*}
    \P\Sp{J_T\neq 1} \leq& \P\Sp{\bigcup_{i=1}^{i_0} \xi_i^c} = \P\Sp{\bigcup_{i=1}^{i_0}\Sp{\xi_i^c\setminus\bigcup_{j=1}^{i-1}\xi_j^c}} \leq  \sum_{i=1}^{i_0}\P\Sp{\xi_i^c\setminus \bigcup_{j=1}^{i-1}\xi_j^c} \\
    =& \sum_{i=1}^{i_0}\P\Sp{\xi_i^c\cap\Sp{\bigcup_{j=1}^{i-1}\xi_j^c}^c} = \sum_{i=1}^{i_0}\P\Sp{\xi_i^c\cap\Sp{\bigcap_{j=1}^{i-1}\xi_j}} \\
    \leq& \sum_{i=1}^{i_0}\P\Sp{ \xi_i^c\left| \bigcap_{j=1}^{i-1} \xi_j\right.}.
\end{align*}

\textbf{Bernstein's inequality.} Now, we just need to find an upper bound of $\P\left(\xi_i^c \left|\bigcap_{j=1}^{i-1} \xi_{j}\right.\right)$. Assume $\exists t \geq n_i, \exists k\in[K] \text{ s.t. } \Delta_k \leq 2 \cdot 2^{-i}$.\footnote{Otherwise, $\xi_i$ is vacuously true and $\P\Sp{\xi_i^c}=0$.} Then, we have
\begin{align}
    &\P(| \Delta_k - \widehat{\Delta}^{(t)}_k| \geq 2^{-i}/2)\nonumber\\
    =& \P( |(\theta - \widehat{\theta}_t)^\top (x_1-x_k)| \geq 2^{-i}/2 ) \label{equ:epsilon_val}\\
    =&\P\Sp{\abs{\sum_{s=1}^{t}\left(\theta-A(\lambda_s)^{-1}x_sr_s\right)^\top (x_1-x_k)}\geq 2^{-i}t/2}\nonumber\\
    \overset{\text{(a)}}{\leq}& 2\exp\Sp{-\frac{2^{-2i}t^2/8}{2\sum_{s=1}^{t}\Norm{x_1-x_k}^2_{A(\bar{\lambda}_s)^{-1}}+\Sp{\sqrt{d}\max_{s\in[1:t]}\Norm{x_1-x_k}_{A(\bar{\lambda}_s)^{-1}}+1}t2^{-i}/3}}\tag{By Bernstein's inequality for martingale differences \cite{freedman1975tail}}\\
    \leq& 2\exp\Sp{-\frac{2^{-2i}t^2/8}{\text{term I}}},\nonumber
\end{align}
\begin{align*}
    \text{where term I}=&2\sum_{i'=1}^{i}\sum_{s=n_{i'-1}+1}^{n_{i'}}\Norm{x_1-x_k}^2_{A(\bar{\lambda}_s)^{-1}}+2\sum_{s=n_i+1}^{t}\Norm{x_1-x_k}^2_{A(\bar{\lambda}_s)^{-1}}\\
    &\quad +\left(\sqrt{d}\max_{s\in[1:t]}\Norm{x_1-x_k}_{A(\bar{\lambda}_s)^{-1}}+1\right) \cdot\frac{t2^{-i}}{3}.
\end{align*}
Here, to use Bernstein's inequality for martingale differences in the inequality (a) above, we need to bound the variance and magnitude of $\left(\theta-A(\lambda_s)^{-1}x_sr_s\right)^\top (x_1-x_k)$ condition on $\lambda_s$.\footnote{Since IPS estimator is unbiased and $\lambda_s$ is determined by the history prior to time $s$, we have $\E\Mp{\left(\theta-A(\lambda_s)^{-1}x_sr_s\right)^\top (x_1-x_k)\mid \mc{H}_{s-1}}=0$, which implies that it is a martingale difference sequence.} In particular, we have
\begin{align*}
    \abs{\left(\theta-A(\lambda_s)^{-1}x_sr_s\right)^\top (x_1-x_k)} \leq& \abs{(x_1-x_k)^\top A(\lambda_s)^{-1}x_s} + \Delta_k\\
    \leq& \Norm{x_1-x_k}_{A(\lambda_s)^{-1}}\Norm{x_s}_{A(\lambda_s)^{-1}} + 2\\
    \leq& 2\sqrt{d}\Norm{x_1-x_k}_{A(\bar{\lambda}_s)^{-1}}+2.\tag{Since $\lambda_s=(\bar{\lambda}_s+\lambda^*)/2$ and $\lambda\mapsto\Norm{x_1-x_k}^2_{A(\lambda)^{-1}}$ is convex in $\lambda$}
\end{align*}
\begin{align*}
    &\E\Mp{\Sp{\left(\theta-A(\lambda_s)^{-1}x_sr_s\right)^\top (x_1-x_k)}^2\mid\lambda_s}\\
    \leq & \E\Mp{\Sp{(x_1-x_k)^\top A(\lambda_s)^{-1}x_s}^2\mid \lambda_s}\\
    =&(x_1-x_k)^\top A(\lambda_s)^{-1}\E\Mp{x_sx_s^\top\mid\lambda_s}A(\lambda_s)^{-1}(x_1-x_k)\\
    =&\Norm{x_1-x_k}^2_{A(\lambda_s)^{-1}}\\
    \leq& 2\Norm{x_1-x_k}^2_{A(\bar{\lambda}_s)^{-1}}.\tag{Since $\lambda_s=(\bar{\lambda}_s+\lambda^*)/2$}
\end{align*}

\textbf{Single-term error probability.} Now, we need to use the property of the subroutine \textsf{RAGE-Elimination} (Line \ref{algo:rage_elimination} of Algorithm \ref{algo:p1_rage}) that generates $\lambda_s$. That is, by Lemma~\ref{lmm:subroutine_guarantees}, since $x_k\in\mc{A}_i\subseteq\mc{A}_{i'}$ for $i'\leq i$ and $m\geq i_0$, for $s\in[n_{i'-1}+1, n_{i'}]$, we have $\Norm{x_1 - x_k}_{A(\bar{\lambda}_s)^{-1}}^2 \leq m\inf_{\lambda\in\triangle_{\X}} \max_{x, x'\in\mc{A}_{i'-2}}\Norm{x - x'}^2_{A(\lambda)^{-1}}\overset{\mathrm{def}}{=}m f(\mc{A}_{i'-2})$. Thus, we have
\begin{align*}
    &\P(| \Delta_k - \widehat{\Delta}^{(t)}_k| \geq 2^{-i}/2)\\
    \leq& 2\exp\Sp{-\frac{2^{-2i}t^2/8}{2m\sum_{i'=1}^{i}(n_{i'}-n_{i'-1})f(\mc{A}_{i'-2})+2m(t-n_i)f(\mc{A}_{i-1})+(m\sqrt{df(\mc{X})}+1)t2^{-i}/3}}\\
    \leq & 2\exp\Sp{-\frac{2^{-2i}n_i^2/8}{2m\sum_{i'=1}^{i}(n_{i'}-n_{i'-1})f(\mc{A}_{i'-2})+(m\sqrt{df(\mc{X})}+1)n_i2^{-i}/3}},
\end{align*}
where the last inequality above holds because of $t\geq n_i$ and a simple fact that $t\mapsto\frac{t^2}{at+b}$ is an increasing function when $t\geq 0$ if $a>0$ and $b>0$. 

\textbf{Final error probability.} Then, with the union bound over all $ t \geq n_i$ and $ k\in[K]$, it holds for any $0<n_1 \leq n_2 \ldots \leq n_i \leq T$ that
\begin{align*}
    &\P\Sp{\xi_i^c \left|\bigcap_{j=1}^{i-1} \xi_{j}\right.}
    \leq 2KT \exp\Sp{-\frac{2^{-2i}n_i^2/8}{2m\sum_{i'=1}^{i}(n_{i'}-n_{i'-1})f(\mc{A}_{i'-2})+(m\sqrt{d f(\mc{X})}+1)n_i2^{-i}/3}}\\
    &\qquad\leq 2KT\max_{k\in[K]}\exp\Sp{-\frac{3n_{\bar{i}(k)}^2\Delta_k^2}{48m\sum_{i'=1}^{\bar{i}(k)}(n_{i'}-n_{i'-1})f(\mc{A}_{i'-2})+8(m\sqrt{d f(\mc{X})}+1)n_{\bar{i}(k)}\Delta_k}},
\end{align*}
where $\bar{i}(k)=\max\Bp{i\in[i_0 - 1]\mid \Delta_k\leq 2^{-i}}$. Here, the last inequality use the same simple fact that $t\mapsto\frac{t^2}{at+b}$ is an increasing function when $t\geq 0$ if $a>0$ and $b>0$.

With values of $0=n_0 < n_1 \leq n_2 \leq \dots\leq n_{i_0}=T$, we can define $a_i=\frac{n_i}{T}$, which implies $0=a_0<a_1\leq a_2\leq\dots\leq a_{i_0}=1$. Since the choice of values $\ve{a}\in\mc{D}$ is arbitrary, the final error probability can be bounded as 
\begin{align*}
    &\P\Sp{J_T\neq 1} \leq \sum_{i=1}^{i_0}\P\Sp{ \xi_j^c\left| \bigcap_{j=1}^{i-1} \xi_j\right.}\\
    &\leq 2i_0 KT \min_{\ve{a}\in\mc{D}} \max_{k\in[K]}\exp\Sp{-\frac{3Ta_{i(k)}^2\Delta_k^2}{48m\sum_{i'=1}^{\bar{i}(k)}(a_{i'}-a_{i'-1})f(\mc{A}_{i'-2})+8(m\sqrt{d f(\mc{X})}+1)a_{i(k)}\Delta_k}},
\end{align*}
which completes the proof
\end{proof}

% \textbf{Guarantees of \texttt{subroutine}}\\
% Define $\mc{A}_i=\Bp{j: \Delta_j\leq 2\cdot 2^{-i}}$.


\subsubsection{Properties of \textsf{RAGE-Elimination}}
In this section, we prove some properties of the \textsf{RAGE-Elimination} algorithm that will be useful for proving Theorem \ref{theo:bobw_error_prob_raw}.

\begin{lemma}
\label{lem:X_in_A}
Assume $t \geq n_i$. Then, under $\bigcap_{j=1}^{i-1} \xi_{j}$, when running \textsf{RAGE-Elimination} (line \ref{algo:rage_elimination} in Algorithm \ref{algo:p1_rage}), it holds that 
$$\mc{X}_{t}^{(i+1)}\subseteq \Bp{x\in\X\mid \widehat{\Delta}^{(t)}_x \leq 2^{-i}} \subseteq \mc{A}_i.$$
\end{lemma}
\begin{proof}
To show $\mc{X}_{t}^{(i+1)}\subseteq \Bp{x\in\X\mid \widehat{\Delta}^{(t)}_x \leq 2^{-i}}$, let $x_{\widehat{(1)}_t}=\argmax_{x\in\mc{X}}\inner{\htheta_t, x}$. Then, for some arm $x$, if we have $\inner{\widehat{\theta}^{(t)}, x_{\hatone_t}-x}\leq 2^{-i}$, it holds that
$$\inner{\widehat{\theta}^{(t)}, x_1-x}=\underbrace{\inner{\widehat{\theta}^{(t)}, x_1-x_{\hatone_t}}}_{\leq 0}+\underbrace{\inner{\widehat{\theta}^{(t)}, x_{\hatone_t}-x}}_{\leq 2^{-i}}\leq 2^{-i},$$
which implies $x\in\{x\in\X\mid \widehat{\Delta}^{(t)}_x \leq 2^{-i}\}$.

To show $\Bp{x\in\X\mid \widehat{\Delta}^{(t)}_x \leq 2^{-i}} \subseteq \mc{A}_i$, let $\widehat{\Delta}^{(t)}_x \leq 2^{-i}$ for some $x$ and assume for the sake of a contradiction that $\Delta_x > 2 \cdot 2^{-i}$. As $\Delta_x > 2 \cdot 2^{-i}$, there must exist $\tilde{i} \leq i-1$ such that
$2^{-\tilde{i}} < \Delta_x \leq 2\cdot2^{-\tilde{i}}$. Then $| \Delta_x - \widehat{\Delta}^{(t)}_x| < 2^{-\tilde{i}}/2$ since event $\xi_{\tilde{i}}$ holds. Meanwhile, we have $\widehat{\Delta}^{(t)}_x \leq 2^{-i} \leq 2^{-\tilde{i}}/2$ since $\tilde{i} \leq i-1$. Now, this leads to the contradiction
$$2^{-\tilde{i}}/2 = 2^{-\tilde{i}} - 2^{-\tilde{i}}/2 \leq \Delta_x - \widehat{\Delta}^{(t)}_x \leq  | \Delta_x - \widehat{\Delta}^{(t)}_j| < 2^{-\tilde{i}}/2.$$
Thus, under $\bigcap_{j=1}^{i-1} \xi_{j}$, we have
$$\Bp{x\in\X\mid \widehat{\Delta}^{(t)}_x \leq 2^{-i}} \subseteq \Bp{x\in\X\mid \Delta_x \leq 2 \cdot 2^{-i}}=\mc{A}_i.$$

\end{proof}


\begin{lemma}
\label{lem:A_in_X}
Assume $t\geq n_i$. Then, under $\bigcap_{j=1}^{i-1} \xi_{j}$, when running \textsf{RAGE-Elimination}, if $x\in\mc{A}_{i}$, then $x\in\mc{X}_{t}^{(i-1)}$.
\end{lemma}
\begin{proof}
If $x\in\mc{A}_i$, then $\inner{\theta, x_1-x}\leq 2\cdot 2^{-i}$. Again, let $x_{\hatone_t}=\argmax_{x\in\mc{X}}\inner{\htheta_t, x}$ and we have
\begin{align*}
    \inner{\htheta_t, \hat{x}_1^{(t)}-x}=&\inner{\htheta_t, x_{\hatone_t}-x_1}+\inner{\htheta_t, x_1-x}\\
    =&\inner{\htheta_t, x_{\hatone_t}-x_1}+\inner{\htheta_t-\theta, x_1-x}+\underbrace{\inner{\theta, x_1-x}}_{\leq 2\cdot 2^{-i}}\\
    \leq & \inner{\htheta_t, x_{\hatone_t}-x_1} + |\hdeltat_x-\Delta_x| + 2\cdot 2^{-i}\\
    \leq & \inner{\htheta_t, x_{\hatone_t}-x_1} + 2^{-i} + 2\cdot 2^{-i}\tag{Since $\xi_{i-1}$ holds}\\
    = & -\hdeltat_{x_{\hatone_t}}+ 2^{-i} + 2\cdot 2^{-i}\\
    \leq & 2^{-i} + 2^{-i} + 2\cdot 2^{-i}\\
    =& 4\cdot 2^{-i}.
\end{align*}
The last inequality above holds because under $\bigcap_{j=1}^{i-1} \xi_{j}$, by Lemma \ref{lem:X_in_A}, we have $x_{\hatone_t}\in\mc{A}_i$, meaning that $|\hdeltat_{x_{\hatone_t}}-\Delta_{x_{\hatone_t}}|< 2^{-i}\implies\hdeltat_{x_{\hatone_t}}>\Delta_{x_{\hatone_t}} - 2^{-i}>-2^{-i}$.
\end{proof}

\begin{lemma}\label{lmm:subroutine_guarantees}
Assume $t\geq n_i$ and $\bigcap_{j=1}^{i-1} \xi_{j}$ holds. When running \textsf{RAGE-Elimination}, If $x_k\in\mc{A}_{i}$, then
$$\Norm{x_1-x_k}^2_{A(\bar{\lambda}_t)^{-1}}\leq m \min_{\lambda\in\triangle_{\mc{X}}}\max_{x, x'\in\mc{A}_{i-2}}\Norm{x-x'}^2_{A(\lambda)^{-1}}.$$
\end{lemma}
\begin{proof}
By Lemma \ref{lem:A_in_X}, we have $x_1, x_k\in\mc{A}_i\implies x_1, x_k\in\mc{X}_{t}^{(i-1)}$, which means that $\abs{\X_{t}^{(i-1)}}\geq 2$ and $\bar{\lambda}_t=\frac{1}{i_t}\sum_{i'=1}^{i_t}\lambda_{t}^{(i')}$ for some $i_t$ satisfying $i-1\leq i_t\leq m$. Thus, We have
\begin{align*}
    \Norm{x_1-x_k}^2_{A(\bar{\lambda}_t)^{-1}}\leq & m\Norm{x_1-x_k}^2_{A\left(\lambda_{t}^{(i-1)}\right)^{-1}}\\
    \leq& m \max_{x, x'\in\mc{X}_{t}^{(i-1)}}\Norm{x-x'}^2_{A\left(\lambda_{t}^{(i-1)}\right)^{-1}}\tag{Since $x_1, x_k\in\mc{X}_{i-1}^{(t)}$}\\
    \overset{\text{(i)}}{\leq}& m \min_{\lambda\in\triangle_{\mc{X}}}\max_{x, x'\in\mc{A}_{i-2}}\Norm{x-x'}^2_{A(\lambda)^{-1}}.
\end{align*}
Here, the above inequality (i) holds because by Lemma \ref{lem:X_in_A}, we have $\mc{X}_{t}^{(i-1)}\subseteq\mc{A}_{i-2}$ and by algorithm construction, we have $\lambda_{t}^{(i-1)}\in\argmin_{\lambda\in\triangle_{\mc{X}}}\max_{x, x'\in\mc{X}_{t}^{(i-1)}}\Norm{x-x'}^2_{A(\lambda)^{-1}}$.
\end{proof}

\subsubsection{Simplified Stationary Complexity and its Relation to Multi-armed Bandits}

In this section, we simplify the complexity of Algorithm \ref{algo:p1_rage} obtained in Theorem \ref{theo:bobw_error_prob_raw} by appropriately choosing values $\ve{a}\in\mc{D}$. In particular, we have the following theorem.
\begin{theorem}
    For $\overline{H}_{\textsf{P1-RAGE}}(\theta)$ defined in equation \eqref{equ:H_p1rage}, we have
    \fontsize{9.5}{9.5}
    $$\overline{H}_{\textsf{P1-RAGE}}(\theta)\leq \frac{1024mi_0}{\Delta_1}\inf_{\lambda\in\triangle_{\X}}\max_{x\neq x_1}\frac{\Norm{x-x_1}^2_{A(\lambda)^{-1}}}{\Delta_x} + \frac{16m\sqrt{d}}{3\Delta_1}\inf_{\lambda\in\triangle_{\X}}\max_{x\neq x_1}\Norm{x-x_1}_{A(\lambda)^{-1}}+\frac{1}{3\Delta_1}.$$
    \normalsize
\end{theorem}
\begin{proof}
    For $i\in\Bp{1, \dots, i_0-1}$, we take $a_i=\frac{\Delta_1}{\Delta_{\bar{k}(i)}}$, where $\bar{k}(i)=\min\Bp{k\in[K]\mid \Delta_k\geq\frac{2^{-i}}{2}}$. Then, since $\bar{i}(k)=\max\Bp{i\in[i_0-1]\mid \Delta_k\leq 2^{-i}}$, for any $k\in[K]$, we have $\frac{2^{-\bar{i}(k)}}{2}\leq\Delta_{\bar{k}(\bar{i}(k))}\leq\Delta_k$, which further implies 
    $$a_{\bar{i}(k)}\Delta_k=\frac{\Delta_1}{\Delta_{\bar{k}(\bar{i}(k))}}\cdot\Delta_k\geq\Delta_1.$$
    Then, for $\overline{H}_{\textsf{P1-RAGE}}(\theta)$ (defined in equation \eqref{equ:H_p1rage}), we have
    \begin{align*}
        &\overline{H}_{\textsf{P1-RAGE}}(\theta)\leq \max_{k\in[K]}\Bp{\frac{16m\sum_{i'=1}^{\bar{i}(k)}(a_{i'}-a_{i'-1})f(\mc{A}_{i'-2}) }{a_{\bar{i}(k)}^2\Delta_k^2} + \frac{8(m\sqrt{df(\mc{X})}+1)}{3a_{\bar{i}(k)}\Delta_k}}\\
        &\quad \leq  \frac{16m}{\Delta_1}\max_{k\in[K]}\Bp{\frac{f(\mc{A}_{-1})}{\Delta_{\bar{k}(1)}}+\sum_{i'=2}^{\bar{i}(k)}\Sp{\frac{1}{\Delta_{\bar{k}(i')}}-\frac{1}{\Delta_{\bar{k}(i'-1)}}}f(\mc{A}_{i'-2})} + \frac{8(m\sqrt{df(\mc{X})}+1)}{3\Delta_1}.\tag{Since $a_0=0$ by definition}
    \end{align*}
    For the second term, using the definition of $f(\X)$, we simply have
    \begin{align}
        \frac{8(m\sqrt{df(\mc{X})}+1)}{3\Delta_1}=&\frac{8m\sqrt{d}}{3\Delta_1}\inf_{\lambda\in\triangle_{\X}}\max_{x, x'\in\X}\Norm{x-x_1+x_1-x'}_{A(\lambda)^{-1}}+\frac{1}{3\Delta_1}\nonumber\\
        \leq & \frac{16m\sqrt{d}}{3\Delta_1}\inf_{\lambda\in\triangle_{\X}}\max_{x\neq x_1}\Norm{x-x_1}_{A(\lambda)^{-1}}+\frac{1}{3\Delta_1}.\label{equ:expand_norm_square}
    \end{align}
    For the first term, by fixing arm index $k\in[K]$ and defining $j\in\argmax_{\ell\in[\bar{i}(k)]}\frac{f(\mc{A}_{\ell-2})}{\Delta_{\bar{k}(\ell)}}$, we have
    \begin{align*}
        &\frac{f(\mc{A}_{-1})}{\Delta_{\bar{k}(1)}}+\sum_{i'=2}^{\bar{i}(k)}\Sp{\frac{1}{\Delta_{\bar{k}(i')}}-\frac{1}{\Delta_{\bar{k}(i'-1)}}}f(\mc{A}_{i'-2})\\
        =&\frac{f(\mc{A}_{\bar{i}(k)-2})}{\Delta_{\bar{k}(\bar{i}(k))}}+\sum_{i'=1}^{\bar{i}(k)-1}\frac{f(\mc{A}_{i'-2})-f(\mc{A}_{i'-1})}{\Delta_{\bar{k}(i')}}\\
        \overset{\text{(a)}}{\leq} & \frac{f(\mc{A}_{j-2})}{\Delta_{\bar{k}(j)}}\Sp{1+\sum_{i'=1}^{\bar{i}(k)-1}\frac{f(\mc{A}_{i'-2})-f(\mc{A}_{i'-1})}{f(\mc{A}_{i'-2})}}\\
        \leq & \bar{i}(k)\frac{f(\mc{A}_{j-2})}{\Delta_{\bar{k}(j)}} \tag{Since $f(\mc{A}_{i'-2})\geq f(\mc{A}_{i'-1})$}\\
        \leq & i_0\max_{\ell\in[\bar{i}(k)]}\frac{f(\mc{A}_{\ell-2})}{\Delta_{\bar{k}(\ell)}} \tag{Since $\bar{i}(k)\leq i_0$ for any $k\in[K]$}\\
        = & i_0 \max_{\ell\in[\bar{i}(k)]} \inf_{\lambda\in\triangle_{\X}}\max_{x, x'\in\mc{A}_{\ell-2}}\frac{\Norm{x-x'}^2_{A(\lambda)^{-1}}}{\Delta_{\bar{k}(\ell)}}\\
        \leq & i_0\inf_{\lambda\in\triangle_{\X}} \max_{\ell\in[\bar{i}(k)]}\max_{x, x'\in\mc{A}_{\ell-2}}\frac{\Norm{x-x'}^2_{A(\lambda)^{-1}}}{\Delta_{\bar{k}(\ell)}} \tag{By the weak duality inequality}\\
        \leq &64i_0\inf_{\lambda\in\triangle_{\X}} \max_{\ell\in[\bar{i}(k)]}\max_{x\in\mc{A}_{\ell-2}, x\neq x_1}\frac{\Norm{x-x_1}^2_{A(\lambda)^{-1}}}{16\Delta_{\bar{k}(\ell)}} \tag{By reasoning similar to equation \eqref{equ:expand_norm_square}}\\
        \overset{\text{(b)}}{\leq} & 64i_0\inf_{\lambda\in\triangle_{\X}} \max_{\ell\in[\bar{i}(k)]}\max_{x\in\mc{A}_{\ell-2}, x\neq x_1}\frac{\Norm{x-x_1}^2_{A(\lambda)^{-1}}}{\Delta_x}\\
        \leq & 64i_0\inf_{\lambda\in\triangle_{\X}}\max_{x\neq x_1}\frac{\Norm{x-x_1}^2_{A(\lambda)^{-1}}}{\Delta_x}.
    \end{align*}
    Here, the inequality (a) above holds because $f(\mc{A}_{i'-2})\geq f(\mc{A}_{i'-1})$ and by definition of $j$, we have $\frac{f(\mc{A}_{\ell-2})}{\Delta_{\bar{k}(\ell)}}\leq\frac{f(\mc{A}_{j-2})}{\Delta_{\bar{k}(j)}}$. The inequality (b) above holds because by definitions of $\bar{k}(\ell)=\min\Bp{k\in[K]\mid \Delta_k\geq\frac{2^{-i}}{2}}$ and $\mc{A}_{\ell-2}=\Bp{x\in\X\mid \Delta_x\leq 2\cdot 2^{-(\ell-2)}}$, we have $16\Delta_{\bar{k}(\ell)}\geq\Delta_x$ for any $x\in\mc{A}_{\ell-2}.$

    Therefore, by plugging the bound of both terms back, we have
    \fontsize{9}{9}
    $$\overline{H}_{\textsf{P1-RAGE}}(\theta)\leq \frac{1024mi_0}{\Delta_1}\inf_{\lambda\in\triangle_{\X}}\max_{x\neq x_1}\frac{\Norm{x-x_1}^2_{A(\lambda)^{-1}}}{\Delta_x} + \frac{16m\sqrt{d}}{3\Delta_1}\inf_{\lambda\in\triangle_{\X}}\max_{x\neq x_1}\Norm{x-x_1}_{A(\lambda)^{-1}}+\frac{1}{3\Delta_1}.$$
    \normalsize
\end{proof}

In the following corollary, we show that the above simplified complexity is in a same order (up to logarithmic factors) of $H_{\mathrm{BOB}}$ proposed in \citet{abbasi2018best}.
\begin{corollary}
    \label{coro:bobw_linear_to_mab}
    In multi-armed bandits, meaning $d=K$ and $\X=\Bp{\ve{e}_1, \dots, \ve{e}_K}$, for $H_{\textsf{P1-RAGE}}(\theta)$ (defined in equation \eqref{equ:H_bobw}), if $m=i_0$, we then have
    $$H_{\textsf{P1-RAGE}}(\theta)\leq \frac{2i_0\Sp{i_0\log(2K)+1}}{\Delta_{(1)}}\max_{k\in[K]}\frac{k}{\Delta_{(k)}}=2i_0\Sp{i_0\log(2K)+1}H_{\mathrm{BOB}}(\theta).$$
\end{corollary}
\begin{proof}
    When in multi-armed bandits, for the first term in $H_{\textsf{P1-RAGE}}(\theta)$, we have
    $$\inf_{\lambda\in\triangle_{\X}}\max_{x\neq x_{(1)}}\frac{\Norm{x-x_{(1)}}^2_{A(\lambda)^{-1}}}{\Delta_x}\leq 2\sum_{k=1}^{K}\frac{1}{\Delta_k}\leq 2\log(2K)\max_{k\in[K]}\frac{k}{\Delta_{(k)}},$$
    where the first inequality above comes from \citet{soare2014best} and the second inequality comes from \citet{audibert2010best}. For the second term in $H_{\textsf{P1-RAGE}}(\theta)$, we have
    $$\inf_{\lambda\in\triangle_{\X}}\max_{x\neq x_{(1)}}\Norm{x-x_{(1)}}_{A(\lambda)^-1{}}=\inf_{\lambda\in\triangle_{\X}}\max_{k\neq (1)}\sqrt{\frac{1}{\lambda_{(1)}}+\frac{1}{\lambda_{k}}}=\sqrt{2K}, $$
    which then gives us $\frac{\sqrt{K}\cdot\sqrt{2K}}{\Delta_{(1)}}\leq\frac{2K}{\Delta_{(1)}\Delta_{(K)}}\leq\frac{2}{\Delta_{(1)}}\max_{k\in[K]}\frac{k}{\Delta_{(k)}}$. 

    Finally, by plugging these inequalities back into $H_{\textsf{P1-RAGE}}(\theta)$ (defined in equation \eqref{equ:H_bobw}), we have
    \begin{align*}
        H_{\textsf{P1-RAGE}}(\theta)=& \frac{mi_0}{\Delta_{(1)}}\inf_{\lambda\in\triangle_{\X}}\max_{x\neq x_{(1)}}\frac{\Norm{x-x_{(1)}}^2_{A(\lambda)^{-1}}}{\Delta_x} + \frac{m\sqrt{d}}{\Delta_{(1)}}\inf_{\lambda\in\triangle_{\X}}\max_{x\neq x_{(1)}}\Norm{x-x_{(1)}}_{A(\lambda)^{-1}}\\
        \leq & \frac{2i_0^2\log(2K)}{\Delta_{(1)}}\max_{k\in[K]}\frac{k}{\Delta_{(k)}} + \frac{2i_0}{\Delta_{(1)}}\max_{k\in[K]}\frac{k}{\Delta_{(k)}}\\
        = & \frac{2i_0\Sp{i_0\log(2K)+1}}{\Delta_{(1)}}\max_{k\in[K]}\frac{k}{\Delta_{(k)}}.
    \end{align*}
\end{proof}

\subsubsection{Approximate BAI of Algorithm \ref{algo:p1_rage}}
\begin{corollary}
    \label{coro:epsilon_bai}
    Fix arm set $\X\subset\R^d$ with $\abs{\X}=K$ and budget $T$. For a stationary environment with unknown parameter $\theta$, if $m\geq i_0(\epsilon)=\ceil{\log_2\Sp{1/\epsilon}}+1$ for some $\epsilon\geq \Delta_{1}$, then there exists absolute constant $c>0$ such that the error probability of \textsf{P1-RAGE} satisfies
    $$\P_{\theta}\Sp{J_T\notin\mc{A}(\epsilon) }\leq 2i_0(\epsilon) KT\exp\Sp{-\frac{cT}{H_{\textsf{P1-RAGE}}(\theta, \epsilon)}},$$
    where $\mc{A}(\epsilon)=\Bp{x\in\X\mid \Delta_x\leq\epsilon}$ and $H_{\textsf{P1-RAGE}}(\theta, \epsilon)$ is defined as replacing $i_0$ by $i_0(\epsilon)$ in $H_{\textsf{P1-RAGE}}(\theta)$ (defined in Eq. \eqref{equ:H_p1rage}).
\end{corollary}
\begin{proof}
    The proof is the same as Theorem \ref{theo:bobw_upper_bound} through simply replacing $i_0$ by $i_0(\epsilon)$.
\end{proof}

\subsection{Non-stationary Environments}
In this section, we prove the error probability of Algorithm \ref{algo:p1_rage} in general non-stationary environments.
\begin{theorem}
    Fix time horizon $T$, arm set $\X\subset\R^d$ with $\abs{\X}=K$ and arbitrary unknown parameters $\Bp{\theta_t}_{t=1}^T$. If we run Algorithm \ref{algo:p1_rage} in this non-stationary environment and obtain $x_{J_T}$, then it holds that
    $$\P_{\otheta_T}\Sp{J_T\neq (1)}\leq K\exp\Sp{-\frac{3T\Delta_{(1)}^2}{64d}}.$$
\end{theorem}
\begin{proof}
    The proof will basically resemble the one for Theorem \ref{theo:adv_upper_bound}. In particular, by the same reasoning to obtain equation \ref{equ:g_bernstein}, we have
    $$\P\Sp{J_T\neq (1)}\leq \P\Sp{x_{(1)}^\top\htheta_T-x_{(1)}^\top\otheta_T\leq -\frac{\Delta_{(1)}}{2}} + \sum_{k=2}^{K}\P\Sp{x_{(k)}^\top\htheta_T-x_{(k)}^\top\otheta_T\geq\frac{\Delta_{(k)}}{2}},$$
    $$\text{where }\P\Sp{x_{(1)}^\top\htheta_T-x_{(1)}^\top\otheta_T\leq -\frac{\Delta_{(1)}}{2}}=\P\Sp{\sum_{t=1}^{T}x_{(1)}^\top\Sp{A(\lambda_t)^{-1}x_tr_t-\theta_t}\leq-\frac{T\Delta_{(1)}}{2}}.$$
    Since $\lambda_t=\frac{\bar{\lambda}_t+\lambda^*}{2}$ and $\lambda\mapsto\Norm{x}^2_{A(\lambda)^{-1}}$ is convex in $\lambda$, to use the Berstein's inequality for martingale differences \citep{freedman1975tail}, we have
    $$\abs{x_{(1)}^\top\Sp{A(\lambda_t)^{-1}x_tr_t-\theta_t}}\leq 2\Norm{x_{(1)}}_{A(\lambda^*)^{-1}}\Norm{x_t}_{A(\lambda^*)^{-1}}+2\leq 2d+2\leq 4d,$$
    $$\E\Mp{\Sp{x_{(1)}^\top\Sp{A(\lambda_t)^{-1}x_tr_t-\theta_t}}^2\mid \lambda_t}=\Norm{x_{(1)}}^2_{A(\lambda_t)^{-1}} \leq  2\Norm{x_{(1)}}^2_{A(\lambda^*)^{-1}}\leq 2d.$$
    Therefore, we have
    $$\P\Sp{x_{(1)}^\top\htheta_T-x_{(1)}^\top\otheta_T\leq -\frac{\Delta_{(1)}}{2}}\leq\exp\Sp{-\frac{T\Delta_{(1)}^2/8}{2d+2d\Delta_{(1)}/3}}\leq \exp\Sp{-\frac{3T\Delta_{(1)}^2}{64d}}.$$
    By applying the same inequality to other terms, we have
    $$\P\Sp{J_T\neq (1)}\leq K\exp\Sp{-\frac{3T\Delta_{(1)}^2}{64d}}.$$
\end{proof}
% 
\section{A Brief Discussion of the Lower Bound}
\label{sec:lower_bound_diss}
% \zhihan{We can put this section into appendix if there is not enough space.}

Although it is not the main focus of this paper, in this section, we briefly discuss the known lower bound for multi-armed bandits from \citet{abbasi2018best} and explain why we believe one could seek a more refined lower bound, which can be another future direction. Here, we use $\theta\in[0, 1]^K$ to represent the parameter of a multi-armed bandit problem, in which $\theta_k$ is the mean reward of $k$-th arm. 
% Then, we raise an open problem of finding a truly instance-dependent lower bound, which we hope can enlighten future research. 

\paragraph{Multi-armed bandits lower bound} As mentioned before, \citet{abbasi2018best} proposes complexity measure $H_{\mathrm{BOB}}(\theta)=\frac{1}{\Delta_{(1)}}\max_{k\in[K]}\frac{k}{\Delta_{(k)}}$ and claims that this is the best complexity that any BAI algorithm can possibly achieve if it is constrained to be robust to non-stationarity.\footnote{Here, being robust to non-stationarity means, if using the terminology in \citet{abbasi2018best}, to achieve error probability in an order of $\exp\Sp{-T\Delta_{(1)}^2/K}$ under adversarial environments.} However, from a technical perspective, the previous statement is actually not precise. In particular, given some fixed $\theta^*\in[0, 1]^K$ and define $\mc{C}_4(\theta^*)=\Bp{\theta\in[0, 1]^K\mid H_{\mathrm{BOB}}(\theta^*)/4\leq H_{\mathrm{BOB}}(\theta)\leq 4H_{\mathrm{BOB}}(\theta^*)}$. Theorem 4 (the lower bound theorem) in \citet{abbasi2018best} states\footnote{This is the core message conveyed by its Theorem 4, which is although not exactly stated this way.} there exist absolute constants $c, c'>0$ such that for any algorithm, it holds that
% if $\sup_{\theta\in\mc{C}_4(\theta^*)}\P_{\theta}\Sp{J_T\neq (1)}\leq \exp\Sp{-\frac{cT}{H_{\mathrm{BOB}}(\theta^*)}}$, then there exists absolute constant $c'>0$ such that 
% $$\sup_{\allthetat \text{ with } \otheta_T\in\mc{C}_4(\theta^*)}\P_{\allthetat}\Sp{J_T\neq (1)}\geq c'.$$
\fontsize{8.5}{8.5}
\begin{equation}
    \label{equ:mab_lower_bound_state}
    \sup_{\theta\in\mc{C}_4(\theta^*)}\P_{\theta}\Sp{J_T\neq (1)}\leq \exp\Sp{-\frac{cT}{H_{\mathrm{BOB}}(\theta^*)}}\implies \sup_{\allthetat \text{ with } \otheta_T\in\mc{C}_4(\theta^*)}\P_{\allthetat}\Sp{J_T\neq (1)}\geq c'.
\end{equation}
\normalsize
That is, for some $\theta^*\in[0, 1]^K$, solely based on what this theorem conveys, a robust algorithm may achieve complexity strictly smaller than $H_{\mathrm{BOB}}(\theta^*)$, but it just cannot perform uniformly better than $H_{\mathrm{BOB}}(\theta)$ for any $\theta\in\mc{C}_4(\theta^*)$. 
% Thus, we can notice that this lower bound is not refined enough since for a single instance $\theta^*\in[0, 1]^K$, it fails to characterize what is the best error probability that any robust algorithm can possibly achieve.
Thus, we can notice for a single instance $\theta^*\in[0, 1]^K$ that this lower bound fails to characterize what is the best error probability that any robust algorithm can possibly achieve.


\paragraph{Open problem: a more refined lower bound} Therefore, it is natural to ask: can we characterize the best error probability of a single instance $\theta^*\in[0, 1]^K$? In words, this means to remove ``$\sup_{\theta\in\mc{C}_4(\theta^*)}$'' in equation \eqref{equ:mab_lower_bound_state}. Therefore, suppose $h:[0, 1]^K\mapsto\R$ is the complexity we hope to have and define similarly $\overline{\mc{C}}_4(\theta^*)=\Bp{\theta\in[0, 1]^K\mid h(\theta^*)/4\leq h(\theta)\leq 4h(\theta^*)}$. Then, an ideal theorem should state that for any fixed $\theta^*\in[0, 1]^K$ and any algorithm, if the algorithm satisfies $\P_{\theta^*}\Sp{J_T\neq (1)}\leq \widetilde{O}\Sp{\exp\Sp{-\frac{T}{h(\theta^*)}}}$, then we have absolute constant $c'>0$ such that
$$\sup_{\allthetat \text{ with } \otheta_T\in\overline{\mc{C}}_4(\theta^*)}\P_{\allthetat}\Sp{J_T\neq (1)}\geq c'.$$
Meanwhile, to be consistent with the lower bound in \citet{abbasi2018best}, for any $\theta^*\in[0, 1]^K$, there should exist some $\theta\in\mc{C}_4(\theta^*)$ such that $h(\theta)\approx H_{\mathrm{BOB}}(\theta^*)$, which explains why no robust algorithm can perform uniformly better than $H_{\mathrm{BOB}}(\theta)$ over $\mc{C}_4(\theta^*)$. Finally, we should be able to find some algorithm that nearly achieves complexity $h(\theta)$.

% \subsection{Difficulty in Linear Bandits' Lower Bound}
% \label{sec:linear_bobw_lower_bound}
\section{IMPLEMENTATION DETAILS AND ADDITIONAL EXPERIMENTS}
\label{sec:experiment_details}

In this section, we provide more implementation details and additional experiment results. Experiments are executed through Python 3.10 and paralleled by a Mac M1 Pro chip with 6 cores.

First, we notice that an algorithm for stationary environments usually determines a batch of arms to pull at once during each epoch, while in non-stationary environment, the order of pulling these arms will affect the rewards it will receive. Therefore, when applying stationary algorithms (\textsf{Peace} and \textsf{OD-LinBAI}) into a non-stationary environment, we use a random permutation to determine the order of pulling for each batch of arms. 

When implementing \textsf{P1-RAGE}, to be computationally efficient, we update $\lambda_t$ in the same frequency as \textsf{P1-Peace}, which is summarized in Algorithm \ref{algo:p1_peace}. We take $m=15$ for \textsf{P1-RAGE}, which, based on Theorem \ref{theo:bobw_upper_bound}, is valid as long as $\Delta_{(1)}\geq 2^{-13}\approx 1.22\times 10^{-4}$. Furthermore, when implementing \textsf{Peace}, for simplicity, we use $\inf_{\lambda\in\triangle_{\X}}\rho(\mc{Z}, \lambda)$, defined in equation \eqref{equ:rho_z}, to replace all $\gamma(\mc{Z})$ used in \citet{katz2020empirical}. Since the paper of \textsf{OD-LinBAI} does not provide code, we implement it based on the pseudocode in \citet{yang2022minimax}. Finally, we use Frank-Wolfe algorithm with stepsize $\frac{1}{2(i+2)}$ in $i$-th iteration to solve all optimization problems in a form of $\inf_{\lambda\in\triangle_{\X}}\max_{y\in\mc{Y}}\Norm{y}^2_{A(\lambda)^{-1}}$.

As for code snippets reference, we use part of the code from \citet{katz2020empirical} to implement the rounding procedure used in \textsf{Peace}\footnote{No license information.} and part of the code from \citet{fiez2019sequential} to generate the base stationary instance for the multivariate testing example.\footnote{Under MIT License.} We also use code from \citet{xu2018fully} to preprocess the Yahoo! Webscope dataset.\footnote{No license information.}

\subsection{Additional Experiments}
\label{sec:additional_experiments}

Here, we provide experiment results on some additional examples to corroborate our theoretical findings.

\paragraph{Malicious non-stationary example} Because of the nature of arm elimination, algorithms designed for stationary environment can fail easily in some malicious non-stationary environments. Here, we pick the same $\X$ as \citet{soare2014best}'s stationary benchmark example and set $\omega=0.5$. Then, we take
$$\theta_t=\begin{cases}
    \matenv{0 & 1 & 1 & \dots & 1}^\top & \text{for }t=1, \dots, \frac{T}{3},\\
    \matenv{2 & 0 & 0 & \dots & 0}^\top & \text{for }t=\frac{T}{3}+1, \dots, T.
\end{cases}$$
We can see that the overall best arm is still $x_{(1)}=\ve{e}_1$. However, because of the $\theta_t$ in the first $1/3$ rounds, algorithms like \textsf{Peace} and \textsf{OD-LinBAI} will eliminate $\ve{e}_1$ in its initial phase; on the other hand, our algorithms will be robust to this non-stationarity. Here, we take $T=10^4$ and the results are shown in right plot of Figure \ref{fig:error_malicious}.

% Figure environment removed

% \begin{wrapfigure}{r}{0.5\textwidth}
%     \begin{center}
%         % Figure removed
%     \end{center}
%     \caption{The error probabilities are estimated through 1000 repeated trials and the error bars represent $95\%$ confidence intervals.}
%     \label{fig:error_malicious}
% \end{wrapfigure}

% % Figure environment removed



\paragraph{Stationary multivariate testing example} We also test the performance of these algorithms in multivariate testing example when there is no non-stationarity, i.e. $\theta_t=\theta^*$ for all $t$. Here, we also take $T=10^4$ and the results are shown in Figure \ref{fig:error_multi}. We can see that our robust algorithm \textsf{P1-RAGE} again performs better than \textsf{G-BAI} and comparably with \textsf{Peace}.

% Figure environment removed

\paragraph{Non-stationary benchmark example} In this example, we add non-stationarity to \citet{soare2014best}'s stationary benchmark example in a more structured instead of malicious way. In particular, we keep the arm set $\X$ the same, take $\omega=0.5$ and set
$$\theta_t=\matenv{0.3 & 0 & 0 & \dots & -s\sin\Sp{\frac{2\pi t}{L}} + 0.5}^\top, $$
where $s$ is the oscillation scale and $L$ is the oscillation period, In the first series of instances, we fix $L=200$ and take values $m\in\Bp{0, 1, \dots, 9}$; in the second series of instances, we fix $m=1$ and take values $L\in\Bp{300, 600, \dots, 3000}$. All non-stationary instances have the same optimal arm as their stationary counterparts and we take $T=10^4$ for all of these instances. The results are shown in Figure \ref{fig:adv_soare}, from which we can see similar phenomenon as in Figure \ref{fig:adv_multi}. In particular, algorithms designed for stationary environments, \textsf{Peace} and \textsf{OD-LinBAI}, are very unstable in face of non-stationarity. Meanwhile, among the other four relatively robust algorithms, our algorithms \textsf{P1-RAGE} and \textsf{P1-Peace} consistently outperform the other two.

% Figure environment removed



\end{document}