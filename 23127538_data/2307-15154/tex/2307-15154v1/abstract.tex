We investigate the fixed-budget best-arm identification (BAI) problem for linear bandits in a potentially non-stationary environment. Given a finite arm set $\mathcal{X}\subset\mathbb{R}^d$, a fixed budget $T$, and an unpredictable sequence of parameters $\left\lbrace\theta_t\right\rbrace_{t=1}^{T}$, an algorithm will aim to correctly identify the best arm $x^* := \arg\max_{x\in\mathcal{X}}x^\top\sum_{t=1}^{T}\theta_t$ with probability as high as possible. Prior work has addressed the stationary setting where $\theta_t = \theta_1$ for all $t$ and demonstrated that the error probability decreases as $\exp(-T /\rho^*)$ for a problem-dependent constant $\rho^*$. But in many real-world $A/B/n$ multivariate testing scenarios that motivate our work, the environment is non-stationary and an algorithm expecting a stationary setting can easily fail. 
For robust identification, it is well-known that if arms are chosen randomly and non-adaptively from a G-optimal design over $\mathcal{X}$ at each time then the error probability decreases as $\exp(-T\Delta^2_{(1)}/d)$, where $\Delta_{(1)} = \min_{x \neq x^*} (x^* - x)^\top \frac{1}{T}\sum_{t=1}^T \theta_t$. As there exist environments where $\Delta_{(1)}^2/ d \ll 1/ \rho^*$, we are motivated to propose a novel algorithm \textsf{P1-RAGE} that aims to obtain the best of both worlds: robustness to non-stationarity and fast rates of identification in benign settings. We characterize the error probability of \textsf{P1-RAGE} and demonstrate empirically that the algorithm indeed never performs worse than G-optimal design but compares favorably to the best algorithms in the stationary setting.