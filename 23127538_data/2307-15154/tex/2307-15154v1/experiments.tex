\section{Experiments}
\label{sec:experiments}

% \begin{wrapfigure}{r}{0.5\textwidth}
% \vspace{-.33in}
%     \begin{center}
%         % Figure removed
%     \end{center}
%     \caption{Vertical axis %(error probability) 
%     is on log scale; shaded area represents the $95\%$ confidence interval. Error probability is estimated using 1000 repeated trials.}
%     \label{fig:sto_soare}
% \end{wrapfigure}
In this section, we present our experiment results on several stationary/non-stationary environments. Since to the best of our knowledge, we are the first to propose best-arm identification algorithms that tackle non-stationarity in linear bandits, the algorithms from other works that we compare with are all specifically designed for stationary environments. In particular, we will compare our algorithms with \textsf{Peace}, which is the first fixed-budget algorithm for linear bandits and also inspires our algorithmic design \citep{katz2020empirical}, and \textsf{OD-LinBAI}, which is the most recent algorithm of this kind and is claimed to be minimax optimal \citep{yang2022minimax}. We also examine the \textsf{Mixed-Peace}, which is a naive mixture of \textsf{Peace} and the G-optimal design and its details are summarized in Algorithm \ref{algo:mixed_peace} in Appendix \ref{sec:naive_algo}. More details and additional experiments can be found in Appendix \ref{sec:experiment_details}.
% Meanwhile, due to space constraint, we put implementation details and additional experiment results into Appendix \ref{sec:experiment_details}.

\textbf{Stationary benchmark example.} First, as a sanity check, we consider the famous stationary benchmark example proposed in \citet{soare2014best}. In particular, we have $\X=\Bp{\ve{e}_1, \dots, \ve{e}_d, x'}$, where $x'=\cos(\omega)\ve{e}_1+\sin(\omega)\ve{e}_2$ with some small $\omega>0$, and $\otheta_T=\theta^*=2\ve{e}_1$ so that $x_{(1)}=\ve{e}_1$. An efficient algorithm should pick $\ve{e}_2$ frequently to reduce the variance in the direction of $\ve{e}_1-x'$. In this example, we pick $d=10$ and $\omega=0.1$. 

The results are shown in Figure \ref{fig:sto_soare}. We can see that both our algorithms, \textsf{P1-RAGE} and \textsf{P1-Peace}, perform better than \textsf{G-BAI} and comparably with \textsf{Peace}, showing that our algorithms maintain good performance in stationary environments. Meanwhile, we also notice that \textsf{Mixed-Peace} has performance only comparable to \textsf{G-BAI}, showing that naively mixing the allocation strategy with the G-optimal design can downgrade the performance in stationary environments.

% Figure environment removed



\textbf{Non-stationary multivariate testing example.} We consider a multivariate testing example from \citet{fiez2019sequential}, which is also similar to the one discussed in Introduction. Considering a webpage with $D$ distinct slots and suppose each slot has two content choices, where we represent each layout as an element $w\in\mc{W}=\Bp{-1, 1}^D$. We hope to maximize the click-through rate and we assume it linearly depends on a layout-determined arm $x\in \X$ in a form of
$$x^\top\theta^*=\theta_0^*+\alpha_1\sum_{j=1}^{D}\theta_j^*w_j+\alpha_2\sum_{k=1}^{D-1}\sum_{\ell=k+1}^{D}\theta^*_{k, \ell}w_kw_{\ell}.$$
Here $\theta_0^*$ is the common bias, $\theta_j^*$ is the weight of $j$-th slot and $\theta^*_{k, \ell}$ is the weight of the interaction between $k$-th and $\ell$-th slots. Because of the periodic nature of people's life cycle, it is very likely that the real-world weights will periodically change. Therefore, to construct a non-stationary environment, we randomly oscillate the weights with scale $s$ and period $L$ to get
$$\theta_{t, i}=\theta^*_i+sI\Norm{\theta^*}_{\infty}\sin\Sp{\frac{2\pi t}{L}+\phi_i}, \quad\text{where }I\sim\mathsf{Unif}(\Bp{0, 1}), \phi_i\sim\mathsf{Unif}([0, 2\pi]).$$
Here, in the first series of instances, we fix $L=900$ and take values $s\in\Bp{0, 1, \dots, 9}$, and in the second series of instances, we fix $s=2$ and take values $L\in\Bp{300, 600, \dots, 3000}$. Finally, we take $\alpha_1=1$, $\alpha_2=0.5$, sample each component of $\theta^*$ uniformly in $[-0.1, 0.1]$ and guarantee that $\otheta_T$ has the same optimal arm as $\theta^*$. We take $T=10^4$ for all settings and the results are shown in Figure \ref{fig:adv_multi}.

% Figure environment removed

 From the plots, we can see that the error probabilities of \textsf{Peace} and \textsf{OD-LinBAI}, algorithms designed for stationary environments, can range from near 0 to 1 in different non-stationary environments, which is quite unstable. Meanwhile, we can see that the performance of the other four algorithms, which all in certain way contain a G-optimal design, is relatively much more stable.\footnote{All algorithms fluctuate in the upper right plot mainly because the minimum gaps also have large fluctuation.} Furthermore, among these four algorithms, we can see that our algorithms \textsf{P1-RAGE} and \textsf{P1-Peace} consistently outperform (never worse than) \textsf{G-BAI} and \textsf{Mixed-Peace}. 


% \begin{wrapfigure}{r}{0.5\textwidth}
%     \begin{center}
%         % Figure removed
%     \end{center}
%     \caption{The shaded area represents the $95\%$ confidence interval and each error probability is estimated through 1000 repeated trials.}
%     \label{fig:adv_yahoo}
% \end{wrapfigure}

\textbf{Non-stationary click-through example.} To create an instance using real-world data, we use the Yahoo! Webscope Dataset R6A.\footnote{\url{https://webscope.sandbox.yahoo.com/}} This dataset contains a fraction of user click log of Yahoo!'s news article from May 1st, 2009 to May 10th, 2009. For each click, we take the outer product between user and article features to get a vector in $\R^{36}$ and then we run a principle component analysis (PCA) to get arm set $\mc{Z}\subset\R^{24}$. To create a non-stationary example, we take data from May 1st to May 7th and for each day's data, we fit a ridge regression with regularization $0.01$, obtaining $\theta^*_1, \dots, \theta^*_7$, which can be used to simulate user's weekly periodic behavior. Suppose we receive $L$ visits each day, then, we can define a non-stationary environment via the following set of periodic parameters,
% $$\theta_t=\theta^*_{i(t)},\quad\text{where }i(t)=\ceil{\frac{t\mod (7L)}{L}}.$$
where each period consists of $\theta_1^*,\dots, \theta_1^*, \dots, \theta_7^*, \dots, \theta_7^*$ such that each $\theta_i^*$ repeats for $L$ times. Finally, we form our arm set $\X$ by picking the optimal arm from $\mc{Z}$ plus 23 randomly picked arms with gap at least $0.05$ so that $\mathrm{span}(\X)=\R^{24}$. We take $T=2.1\times 10^4$ and the results are shown in Figure \ref{fig:adv_yahoo}. Again, we can see that the performance of \textsf{Peace} and \textsf{OD-LinBAI} is very unstable and the performance of \textsf{P1-RAGE} and \textsf{P1-Peace} consistently outperforms the other two G-optimal-design-based algorithms.




