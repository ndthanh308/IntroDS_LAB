\documentclass[10pt]{amsart}

\usepackage{xy, graphicx, color, hyperref,array, mathtools}
 
% \usepackage[foot]{amsaddr}

\usepackage{amsmath}
\usepackage{amssymb}
\usepackage{amsfonts}
 
\usepackage{mathrsfs}

\usepackage{tikz,verbatim}
 \usepackage{dsfont}

\setlength{\textwidth}{5.1in} 
\setlength{\textheight}{7.8in} 

\makeatletter
\def\subsection{\@startsection{subsection}{3}%
  \z@{.9\linespacing\@plus.7\linespacing}{.1\linespacing}%
  {\normalfont\bfseries}}
\makeatother 
 
 \xyoption{all} 
 

\title[]{Central Extensions and Cohomology}
 
 \author{Rohit Joshi, Steven Spallone }
 
  

 
\renewcommand{\familydefault}{\rmdefault}
 \newtheorem{thm}{Theorem}[section]
\newtheorem{c.intro}[thm]{Corollary}
\newtheorem{lemma}[thm]{Lemma}
\newtheorem{prop}[thm]{Proposition}
\newtheorem{cor}{Corollary}[thm]
\newtheorem{problem}[thm]{Problem}
 
\theoremstyle{definition}
\newtheorem{remark}[thm]{Remark}
 
\newtheorem{defn}[thm]{Definition} 
\newtheorem{example}[thm]{Example} 
 
\newcommand{\nc}{\newcommand}

 
%\nc{\thm}{\theorem}
%\nc{\cor}{\corollary}
\nc{\mc}{\mathcal}
\nc{\mb}{\mathbb}
% \nc{\bf}{\mathbf}
\nc{\mf}{\mathfrak}

\nc{\ms}{\mathscr}

\nc{\ul}{\underline}
\nc{\ol}{\overline}
\nc{\N}{\mb N}

\nc{\R}{\mb R}
\nc{\Z}{\mb Z}
 
\nc{\C}{\mb C}


\nc{\bks}{\backslash}

\nc{\dmo}{\DeclareMathOperator}

\nc{\mat}[4]{
    \begin{pmatrix}
      #1 & #2 \\
      #3 & #4
    \end{pmatrix}
}


\dmo{\Ker}{Ker} 
\dmo{\Pin}{Pin}
 \dmo{\Aut}{Aut}
 \dmo{\Borel}{Borel}
\dmo{\Sq}{Sq}
\dmo{\Ext}{Ext}
\dmo{\odd}{odd}
\dmo{\sgn}{sgn}
\nc{\beq}{\begin{equation*}}
\nc{\eeq}{\end{equation*}}
\nc{\half}{\frac{1}{2}}
\dmo{\id}{id}
\dmo{\diag}{diag}
\dmo{\Mod}{mod}
\dmo{\pr}{pr}
\dmo{\GL}{GL}
\dmo{\res}{res}
\dmo{\lin}{lin}
 \dmo{\vol}{vol}
\dmo{\Sp}{Sp}
\dmo{\SO}{SO}
\dmo{\BSO}{BSO}
\dmo{\im}{im}
\dmo{\BO}{BO}

\dmo{\Or}{O}

\dmo{\SL}{SL}
\dmo{\ab}{ab}
 \dmo{\Spin}{Spin}
 
\nc{\la}{\lambda}
  \nc{\eps}{\varepsilon}
  
 \nc{\lip}{\langle}
 \nc{\rip}{\rangle}
\nc{\gm}{\gamma}

\dmo{\Perm}{Perm}
\dmo{\Res}{Res}
\dmo{\Ind}{Ind}
\dmo{\tr}{tr}
\dmo{\Sym}{Sym}
\dmo{\reg}{reg}
\dmo{\End}{End}
\dmo{\Hom}{Hom}
\dmo{\Int}{Int}
\dmo{\Bun}{{\bf Bun}}
\dmo{\bun}{Bun}
 \setcounter{tocdepth}{1} 





 \DeclareFontFamily{U}{cbgreek}{}
\DeclareFontShape{U}{cbgreek}{m}{n}{
        <-6>    grmn0500
        <6-7>   grmn0600
        <7-8>   grmn0700
        <8-9>   grmn0800
        <9-10>  grmn0900
        <10-12> grmn1000
        <12-17> grmn1200
        <17->   grmn1728
      }{}
  
 
\DeclareRobustCommand{\qoppa}{%
  \text{\usefont{U}{cbgreek}{\normalorbold}{n}\symbol{21}}%
}
 
\makeatletter
\newcommand{\normalorbold}{%
  \ifnum\pdf@strcmp{\math@version}{bold}=\z@ bx\else m\fi
}

\address{Indian Institute of Science Education and Research, Pune-411021, Maharashtra, India}
\email{rohitsj@students.iiserpune.ac.in}
\address{Indian Institute of Science Education and Research, Pune-411021, Maharashtra, India}
\email{sspallone@gmail.com}
%\date{\today}
\keywords{classifying spaces, central extensions, group cohomology}
\subjclass{Primary 57T10, Secondary 55R35}




\begin{document}
\maketitle
\begin{center} \today
\end{center}
 
 \tableofcontents
 
 \begin{abstract} 
 Let $G$ be a group which is topologically a CW-complex, $BG$ a classifying space for $G$, and $A$ a discrete abelian group. To a central extension of $G$ by $A$, we can associate a cohomology class in $H^2(BG,A)$. We prove this association is injective, and bijective in many cases.  A homomorphism of such groups lifts to a central extension iff the pullback of the associated cohomology class vanishes.
 %When $G$ is the semidirect product of its connected component with a discrete group, we prove that $\alpha_G$ is a bijection.
 \end{abstract}
 
 
 \section{Introduction}
 
Let $A$ be a discrete abelian group. When $G$ is a discrete group, there is a well-known natural correspondence between central extensions of $G$ by $A$ and the quadratic group cohomology $H^2(G,A)$.  
From this, one can systematically find cohomological criteria for lifting problems. In this paper we extend this correspondence to the topological group setting, when $G$ is topologically a CW-complex, for instance   a Lie group.
 
Let $\mb E(G,A)$ be the set of (equivalence classes of) central extensions of $G$ by $A$, and let $BG$ be a classifying space for $G$.  A construction of Milgram-Steenrod gives a natural map 
\beq
\alpha_G: \mb E(G,A) \to H^2(BG,A).
\eeq
When $G$ is discrete and $A=\Z/p\Z$ (with $p$ prime), the map $\alpha_G$ agrees with the classical bijection.
\begin{comment}

For $G$ connected, we prove $\alpha_G$ is bijective. More generally there is an exact sequence
 \beq
1 \to G^\circ \to G \to \pi_0 \to 1,
\eeq
where $G^\circ$ is the identity component, and $\pi_0=\pi_0(G)$ is the component group of $G$.
From the naturality of $\alpha_G$ comes the commutative diagram
 
 \begin{equation} \label{intro.diag}
\xymatrix{
0 \ar[r] & \mb E(\pi_0,A)\ar[d]^{\alpha_{\pi_0}} \ar[r] & \mb E(G,A) \ar[d]^{\alpha_G} \ar[r] & \mb E(G^\circ, A)  \ar[d]^{\alpha_{G^\circ}}  \\
0 \ar[r] & H^2(B\pi_0,A)\ar[r] & H^2(BG,A) \ar[r]  & H^2(BG^\circ, A)     }.
\end{equation}	
We demonstrate that the rows are (left) exact, and the injectivity of  $\alpha_G$ follows. 
\end{comment}
Here is our main result:

\begin{thm} The map $\alpha_G$ is an injection. It is an isomorphism when $G$ is connected. When $A=\Z/p\Z$ and $G$ is the semidirect product of a discrete group and a connected group, then $\alpha_G$ is an isomorphism.
\end{thm}

Note that orthogonal groups satisfy the latter hypothesis on $G$. From the injectivity of $\alpha_G$ we deduce the following lifting criterion.

\begin{thm} \label{mid.intro}
 Let $\varphi:G' \to G$ be a homomorphism, $A$ a discrete abelian group, and  $p: E \to G$ an extension of $G$ by $A$.  Then $\varphi$ lifts to $E$ iff 
\beq
\varphi^*(\alpha_G(E,i,p))=0.
\eeq
(Here $i: A \to E$ is the inclusion.)
\end{thm}
 


\begin{comment}
 When $A=\Z/p\Z$, we note in Remark \ref{our.remark} [??] that $\alpha_{G}$ is a bijection, when $G$ is the semidirect product of a discrete group and a connected group. This includes orthogonal groups. (One expects $\alpha_G$ to be bijective in general.)  
\end{comment}

For the orthogonal group $G=\Or(V)$ of a quadratic space, we describe its three inequivalent double covers, denoted by $\Pin^{\pm}(V)$ and $\widetilde \Or(V)$, and then describe $\alpha_G$ for each. 
 Finally, we give an application of our work to Stiefel-Whitney classes of orthogonal representations. %We deduce:
 
\begin{thm} \label{last.intro} Let $G$ be a topological group which is a CW-complex. An orthogonal representation $\pi: G \to \Or(V)$ lifts to
\begin{enumerate}
\item $\widetilde \Or(V)$ iff $w_1(\pi)^2=0$.
\item $\Pin^+(V)$ iff $w_2(\pi)=0$.
\item $\Pin^-(V)$ iff $w_2(\pi)+w_1(\pi)^2=0$.
\end{enumerate}
When $\pi$ lifts, the set of lifts of $\pi$ is acted on simply transitively by the group of continuous linear characters $\chi: G \to \{ \pm 1\}$. 
\end{thm}


 
% \subsection{Earlier works}
 
 
 The equivalence of $\mb E(G,A)$ and $H^2(BG,A)$ was observed in the '70s. A proof is given in \cite[Theorem 4]{Wigner} by comparing two spectral sequences in simplicial sheaf cohomology. We have nonetheless written this manuscript because  for a result of this importance, it is useful to have alternate proofs. When $G$ is connected, our correspondence essentially agrees with the obstruction class construction in \cite{NWW}, according to Remark 7.17 in that paper and Proposition \ref{alpha.beta} below. % Also,  our definition of $\alpha_G$ was taken from \cite[Section IV.1]{Milgram}.  
 
 The layout of the paper is as follows.  After reviewing definitions for topological groups in Section \ref{notprem}, we examine the category of their central extensions in Section \ref{exs.section}. %There we establish the  left exactness of the upper row of  \eqref{intro.diag}. An example is given to show it is not right exact in general, however we show it is right exact when  $G$ is the semidirect product of a discrete group and a connected group. 
 The theory of (principal) $G$-bundles is reviewed in Section \ref{G.bund.sec}.
  In Section \ref{class.space.section} we review the theory of classifying spaces, and in particular  the Milgram-Steenrod model and its functorial properties. %It is essential to note that this construction takes central extensions to principal bundles.
 In Section \ref{long.ex.seq.section} we define a correspondence $\beta_G$, rather than $\alpha_G$, when $G$ is connected by means of covering space theory. We also review basic properties of the Eilenberg-Maclane spaces $K(A,2)$. 
 A certain square arises from the connecting homomorphisms of these constructions, which we prove \emph{anticommutes} in Section \ref{Connecting Homomorphisms}.  This is perhaps the least straightforward part of this paper. For disconnected groups, we incorporate the exact sequence for the component group $\pi_0(G)$ in Section \ref{disc.section}. In Section \ref{kappa.section} we use the $K(A,2)$-spaces to define a natural map from the bundles of Section \ref{class.space.section} to the quadratic cohomology. By Section \ref{alpha.section} everything is in place for our definition of $\alpha_G$, and Theorem \ref{mid.intro} follows by gathering the earlier work. 

 Then we focus on the orthogonal groups. In Section \ref{SWC.section} we review Stiefel-Whitney classes, and in Section \ref{double.covs.ort.sec} we describe the double covers of the orthogonal group, culminating in Theorem \ref{last.intro}.
  
  \bigskip

\textbf{Acknowledgements.}  The first author was supported by a postdoctoral fellowship from NBHM (National Board of Higher Mathematics), India.  Both authors   publicly thank Karthik Vasisht for helpful discussions.
 \section{Notation and Preliminaries} \label{notprem}
 
 \subsection{Groups}
 When $G$ is a group, we write $1_G \in G$ for the identity, and $G^{\ab}$ for the abelianization of $G$. When $A$ is an abelian group, we may use additive notation, so the identity would be $0$.
If $x,g \in G$, put
 \beq
 \Int(x)(g)=x^{-1}gx.
 \eeq
  
 Let $\mu_n$ be the group of $n$th roots of unity in $\C^\times$. Let $C_n$ be the cyclic group of order $n$.
 Throughout this paper $A$ is a discrete abelian group.
 
  
 
 
 \subsection{Topological Groups}
 Throughout this paper we will consider various hypotheses on spaces and topological groups. For convenience we  discuss relevant definitions here.
 For us, topological groups will always be Hausdorff. For a topological group $G$, denote by $G^\circ$ the path component of $G$ containing the identity. It is a normal subgroup of $G$; write $\pi_0(G)=G^\circ \bks G$ for the quotient. Write $\Aut(G)$ for the (continuous) automorphisms of $G$.



\begin{defn} A space is \emph{locally path-connected} (LPC), provided it has a base of path-connected sets.
\end{defn}
  
   When $G$ is locally path-connected, $G^\circ$ is open in $G$, and so $\pi_0(G)$ is discrete \cite[Theorem 4.2, page 162]{Munkres}.
   Of course, Lie groups are LPC.   
   
    \begin{example} The additive group $\Z_p$ of $p$-adic integers is not LPC. The $p$-adic solenoid  $\mathbb T_p$ (see  \cite[page 20]{Hofmann}) is connected, but not LPC.
    \end{example}

  \begin{defn}
Let $B$ be a topological space.
We say that $B$ is \emph{semilocally simply connected}, provided for all $b \in B$, there is a neighborhood $U \ni b$ so that the induced homomorphism $\pi_1(U,b) \to \pi_1(B,b)$ is trivial. We say that $B$ is \emph{nice}, provided it is Hausdorff, locally path-connected, semilocally simply connected, and nonempty.
\end{defn}
   The significance for us is that when $B$ is nice and path-connected, it has a universal cover. If $p:E \to B$ is a cover with $B$ nice, then $E$ is automatically nice.
  
  
  \begin{defn} A topological group $G$ is a \emph{CW-group}, provided the underlying topological space is homeomorphic to a CW complex.
 \end{defn}
 
  For instance Lie groups are CW-groups. The additive group $\R^\infty$ of finite sequences is a CW-group.   A CW-group is necessarily LPC and nice. 
 
   
 
 \subsection{Topology on Products} \label{CG.section}
 
 Subtle topological issues arise  for products of spaces, when neither is locally compact. For example, if $X$ and $Y$ are infinite-dimensional CW-complexes, then the space $X \times Y$ with the usual product topology may not be a CW complex. Topologists prefer to work with ``compactly generated'' spaces, which include CW-complexes. 
 
 \begin{defn} A space $X$ is \emph{compactly generated} (CG), provided a subset of $X$ is closed iff its intersection with every compact subset of $X$ is closed.
 \end{defn}
 
Unfortunately, when $X$ and $Y$ are CG spaces, then $X \times Y$ with the  product topology may not be CG. 
 There is a standard refinement of the product topology, sometimes called the \emph{CG-topology} on $X \times Y$, which remedies this. We will write this as ``$X \times_k Y$''. When either $X$ or $Y$ is locally compact, then the topologies coincide, i.e., $X \times_k Y=X \times Y$. A standard reference is \cite[Chapter VII, Section 8]{Maclane}.   The continuity condition for a \emph{right $G$-action} on a space $X$ will be
 that it defines a continuous map $X \times_k G \to X$. On the other hand, when $G$ is a topological group,   the multiplication map is continuous when $G \times G$ is given the product topology.
 
  

  
 
 \section{Central Extensions} \label{exs.section}
 
 In this section we study the category of central extensions of topological groups by discrete groups. Throughout, $G$ is a topological group.
 
 %Our main concern is the exactness of the top sequence in \eqref{intro.diag}, which we observe is not a short exact sequence in general (Proposition \ref{weil.prop}).
 
% \subsection{Definition}
 
\begin{defn} A \emph{central extension of $G$ by $A$} is triplet $\ms E=(E,i,p)$, where $E$ is a topological group, $i: A \to E$ and  $p: E \to G$ are continuous homomorphisms. We require that $i$ is an injection onto a discrete central subgroup of $E$, that $p$ is open and surjective, and that $\ker p=\im i$.
  \end{defn}
Thus $\ms E$ can be identified with the exact sequence
\begin{equation} \label{first.ext}
\mathscr E: 0 \to A \overset{i}{\to} E \overset{p}{\to} G \to 1,
\end{equation}
of continuous homomorphisms of topological groups. Another central extension
\beq
\ms E': 0 \to A \overset{i'}{\to} E' \overset{p'}{\to} G \to 1
\eeq
is \emph{equivalent} to \eqref{first.ext} when there is a homomorphism $\varphi: E \to E'$ with $\varphi \circ i=i'$ and $p' \circ \varphi=p$. (It is necessarily an isomorphism.)
 In this case we write $\ms E \cong \ms E'$. Let ${\bf E}(G,A)$ be the category of central extensions of $G$ by $A$, and let $\mb E(G,A)$ be the set of central   extensions of $G$ by $A$, modulo equivalence.
 
 
 
 
 \subsection{Functorial Properties}
 
 A  morphism $\varphi: G' \to G$ gives rise to a ``pullback'' morphism  $ \varphi^*:\mb E(G,A) \to \mb E(G',A)$ as follows:
Given $\ms E=(E,i,p) \in {\bf E}(G,A)$, set
  \beq
 E' = \{(g',e) \mid \varphi(g') = p(e) \},
  \eeq
 viewed as a closed subgroup of the product $G' \times E$.
  Define $i':A \to E'$  by $i'(a)=(1,i(a))$, and let $p'$ be the first projection. Finally put $\varphi^* \ms E=(E',i',p')$.
  In particular, the group $\Aut(G)$  acts on $\mb E(G,A)$ by pullback. 
 
  
  Similarly, a homomorphism $\psi:A \to A'$  between discrete abelian groups gives rise to a ``pushout central extension'' as follows.  Given $\ms E=(E,i,p) \in {\bf E}(G,A)$, set %we can define a central extension of $G$ by $A'$ as follows. Set
  \beq
  A_\psi= \{(i(a),-\psi(a)) \mid a \in A\},
  \eeq
  a normal subgroup of $E \times A'$, and let
  \begin{equation} \label{e.prime}
 E'= (E \times A')/ A_{\psi}.
 \end{equation}
  Define $i': A' \to E'$ to be the evident inclusion, and $p':E' \to G$ by sending the coset of $(e,a') \mod A_{\psi}$ to $p(e)$. Then
  $\psi_* \ms E=(E',i',p')$ is a central extension of $G$ by $A'$, and this gives a homomorphism $\psi_*: \mb E(G,A) \to \mb E(G,A')$.
 
If we define $\psi_E: E \to E'$ by $\psi_E(e)=(e,1) \mod A_{\psi}$, then the diagram
\begin{equation} \label{diag.ek}  
	\xymatrix{
		A \ar[rr]^{\psi} \ar[d]^i &&A' \ar[d]^{i'}\\
		E \ar[rr]^{\psi_E}   \ar[rd]^p && E' \ar[ld]^{p'}  \\
		 & G & \\
	}
	\end{equation}
commutes.
 

  
\begin{lemma} \label{likely.commutes} If $\varphi:G' \to G$ and $\psi: A \to A'$ are group homomorphisms, then 
\beq
\psi_* \circ \varphi^*=\varphi^* \circ \psi_*: \mb E(G, A) \to \mb E(G',A').
\eeq
\end{lemma}

\begin{proof} Elementary; see for instance \cite[Exercise 5, page 114]{Maclane.Homology}.
\end{proof}
  
\begin{lemma} \label{liftingtheta} 
	Let $\theta \in \Aut(G)$ and $\ms E =(E,i,p) \in {\bf E}(G,A)$. Then $\theta^*(\ms E) \cong \ms E$ iff there exists    $\tilde{\theta} \in \Aut(E)$ so that
	\beq
	\xymatrix{
		E \ar[r]^{\tilde{\theta}} \ar[d]^p& E \ar[d]^p\\
		G \ar[r]^{\theta} & G\\
	}
	\eeq 
	commutes.
	
\end{lemma}
 
\begin{proof} Since $\theta^*(\ms E) \cong (E,i,\theta^{-1} \circ p)$, an equivalence between $\ms E$ and $\theta^*(\ms E)$ amounts to the existence of such a $\tilde \theta$.
\end{proof}

 
\begin{cor} \label{inner.triv.action} Given $x \in G$, the inner automorphism $\Int(x)$ acts trivially on $\mb E(G,A)$.
\end{cor}

\begin{proof} Let $\ms E =(E,i,p) \in {\bf E}(G,A)$. Pick $y \in E$ with $p(y)=x$. It is easy to see that $\Int(y)$ covers $\Int(x)$, so $\Int(x)^*(\ms E) \cong \ms E$ by Lemma \ref{liftingtheta}.
\end{proof}


\subsection{Baer Sums} \label{Baer.groups} 
Let $\ms E_1,\ms E_2 \in {\bf E}(G,A)$, with $\ms E_k=(E_k,i_k,p_k)$ for $k=1,2$. Write $E_3'$ for the pullback of $p_1$ and $p_2$:
\beq
	\xymatrix{
		E_3' \ar[r] \ar[d] &E_2 \ar[d]^{p_2}\\
		E_1 \ar[r]^{p_1}   & G  \\
	}
	\eeq 
Write $\triangledown: A \to E_3'$ for the antidiagonal map $ \triangledown(a)=(i_1(a),i_2(a)^{-1})$, and  let $E_3$ be the quotient of $E_3'$ by the image of $\triangledown$.
Define $i_3:A \to E_3$ by $i_3(a)=(a,1) \mod \triangledown(A)$, and $p_3:E_3 \to G$ by $p_3(e_1,e_2)=p_1(e_1)$. Then 
$\ms E_3=(E_3,i_3,p_3) \in {\bf E}(G,A)$, and is called the \emph{Baer sum} $\ms E_3=\ms E_1 + \ms E_2$
of $\ms E_1$ and $\ms E_2$. The Baer sum makes $\mb E(G,A)$ into an abelian group.  (See \cite[Definition 3.4.4]{weibel}.)  The neutral element is represented by the evident trivial extension
\beq
\mathscr E_0:  0 \to A \to G \times A \to G \to 1.
\eeq
 



 \subsection{Relation to Lifting Problems} \label{rel.lift.prob}
 The following is clear:
 \begin{prop} \label{ext.lift} Let $\varphi:G' \to G$ and $\ms E=(E,i,p) \in  {\bf E}(G,A)$ then $\varphi$ lifts to $E$ if and only if $\varphi^*(\ms E) =0 \in {\bf E}(G',A)$. 
 \end{prop}
\begin{comment}
 We can see this from the diagram:
 
 	\beq
 \xymatrix{ 
 	A \ar[d] & A \ar[d]\\
 	\varphi^*(E) \ar[r] \ar[d]&  E \ar[d]  \\
 	G' \ar[r]^{\varphi} \ar@/^/@{.>}[u]^{s} \ar@{.>}[ru]^{\hat{\varphi}}&   G\\
 }
 \eeq
 \end{comment}
The group $\Hom_c(G',A)$ of continuous homomorphisms from $G'$ to $A$ acts simply transitively on the set of lifts $\hat \varphi$ of $\varphi$ by the prescription $(\chi \odot \hat \varphi)(g)=\chi(g) \hat \varphi(g)$. Therefore, when $\varphi$ lifts to $E$, the set of lifts is in bijection with $\Hom_c(G',A)$.
  
 
  
  \subsection{An Exact Sequence}
Let $G$ be a topological group, and put $\pi_0=\pi_0(G)$. The inclusion map $\iota$ and quotient map $\qoppa$ (``qoppa'', the greek $q$) give an exact sequence
\begin{equation} \label{qoppa}
1 \to G^\circ \overset{\iota}{\to} G \overset{\qoppa}{\to} \pi_0 \to 1.
\end{equation}
 
  
  \begin{prop} \label{extexact} The following sequence is exact.  
  \beq
  0 \to \mb E(\pi_0,A) \overset{\qoppa^*}{\to} \mb E(G,A) \overset{\iota^*}{\to} \mb E(G^\circ,A).
  \eeq
 
  \end{prop}
  
  \begin{proof} 
 
 
 
First we prove $\qoppa^*: \mb E(\pi_0, A) \to \mb E(G,A)$ is injective.  Let $\ms E=(E,p,i) \in {\bf E}(\pi_0, A)$, and suppose $\qoppa^* \ms E$ is trivial. Let $s': G \to E'$ be a splitting of $p'$, and write $\tilde \qoppa: E' \to E$ for the projection. 

 \beq
  	\xymatrix{ & A \ar[d] & A \ar[d]\\
& E' \ar[r]^{\widetilde{\qoppa}} \ar[d]^{p'} & E \ar[d]^p\\
G^\circ \ar[r]^{\iota}& G \ar[r]^{\qoppa} \ar@/^/[u]^{s'}  & \pi_0\\ 	
}
\eeq

Since $p \circ \tilde \qoppa  \circ s'  \circ \iota=\qoppa \circ \iota= 1$,i.e., is trivial map, the image of   
\beq
  \tilde \qoppa   \circ s'  \circ \iota: G^\circ \to E
\eeq
 lies in $A$. Since $G^\circ$ is connected, it must be trivial. So the map $\tilde \qoppa  \circ s': G \to E$  descends to a map $s: \pi_0 \to E$ which is a splitting of $p$. (Since $p \circ \tilde \qoppa \circ s'=\qoppa$.) Hence $\qoppa^*$ is injective.

 Next, we show the sequence is exact at $\mb E(G,A)$. So let $\ms E=(E,p,i) \in {\bf E}(G,A)$ with $\iota^* \ms E=(E',p',i')$ trivial. 
 Let $s': G^\circ \to E'$ be a splitting. 
 Write $\tilde \iota: E' \to E$ for the projection, and let $E^0$ be the image of $G^\circ$ under $\tilde \iota \circ s$. 	Also put $E_0=E/E^0$,
 and note that $p$ descends to $p_0: E_0 \to \pi_0$. Moreover put $i_0=\tilde \qoppa \circ i: A \to E_0$. Then $\ms E_0=(E_0,p_0,i_0) \in \mb E(\pi_0,A)$ and $\qoppa^* \ms E_0=\ms E$.
 \begin{comment}
 \beq
      \xymatrix{A \ar[d] \ar[r] & A \ar[r]  \ar[d]^{i} &   A \ar[d]^{i_0} \\
E^0 \ar[d] \ar[r]^{\widetilde{\iota}} & E \ar[d]^{p} \ar[r]^{\tilde{\qoppa}} & E_0 \ar[d]^{p_0}\\
G^\circ \ar[r]^{\iota} \ar@/^/[u]^{s} & G \ar[r]^\qoppa & \pi_0 \\
}
\eeq
	\end{comment}
 
  
  \end{proof}
 
When $H  \unlhd G$ is a normal subgroup, the group $G$ acts on $H$ by conjugation, and hence  on $\mb E(H,A)$. This action descends to an action of $G/H$ on $\mb E(H,A)$ by Corollary \ref{inner.triv.action}. 

\begin{lemma} \label{rest.fixed.pts} The image of the restriction map $\mb E(G,A) \to \mb E(H,A)$ is fixed by the action of $G$ on $\mb E(H,A)$.
\end{lemma}

\begin{proof} 
Let $\ms E =(E,i,p) \in {\bf E}(G,A)$ and $g \in G$. Pick $y \in E$ with $p(y)=g$. Let $E_H$ be the preimage of $H$ under $p$; it is a normal subgroup of $E$ containing the image of $i$. The restriction map takes $\ms E$ to $\ms E_H=(E_H, p|_{E_H},i)$.
It is easy to see that $\Int(y)$ covers $\Int(g)$, so $\Int(g)^*(\ms E_H) \cong \ms E_H$ by Lemma \ref{liftingtheta}.
\end{proof}
 

 
Thus Proposition \ref{extexact} may be refined to the exact sequence
  \begin{equation} \label{left.ex.here}
  0 \to \mb E(\pi_0(G),A) \to \mb E(G,A) \to \mb E(G^\circ,A)^{\pi_0(G)}.
  \end{equation}
 


In the next section we give a counterexample to show that this sequence  is not right exact in general, however in the subsequent section we show right exactness under a certain hypothesis on $G$.

 
  \subsection{The Weil group} \label{Weil.subs}
  
  
\begin{example} A classic example of a nonsplit extension is $G=W_\R$, the \emph{Weil group} of $\R$. See for instance \cite[(1.4.3)]{Tate}. The Weil group fits into an exact sequence
\beq
1 \to \mb C^\times \to W_{\R} \to \Gamma \to 1,
\eeq
where $\Gamma$ is the Galois group of $\C$ over $\R$. Briefly, the group $W_{\R}$ is a union $\C^\times \cup j \C^\times$, with multiplication  rules $j^2=-1 \in \C^\times$, and $jzj^{-1}=\ol z$.

Let $Q$ be the subgroup of $W_{\R}$ generated by $i \in \C^\times$ and $j$; it is evidently quaternion of order $8$, and contains the subgroup $\mu_2=\{ \pm 1\}$.
\begin{comment}
 Here $G^\circ= \C^{\times}$ and $A = \Z/2\Z$. So we have the following commutative diagrams,

\beq
\xymatrix{G^0 =\C^{\times} \ar[r] & G\\
\{\pm 1\} \ar[r] \ar[u] & Q \ar[u]}
\quad \quad \quad\xymatrix{ \mb E(G^0, A) \ar[d]^{r_1}& \mb E(G, A) \ar[l]^{r_0} \ar[d]^{r_2} \\
	\mb E(\{\pm 1\}, A) & \mb E(Q, A) \ar[l]^{r_3}}
\eeq 

The maps $r_i$ are pullback maps.
Let $E^0=(\rho, \C^\times) \in \mb (G^0,A)$ where $\rho(z)=z^2$.(Does  $E^0 \in \mb E(G^\circ, A)^{\Gamma}$ ?? $\bigstar$) We will show that $E^0 \notin r_0(\mb E(G,A))$. Since the restriction map $res: H^2(Q, A) \to H^2(C_2, A)$ is the $0$ map, we have $r_3=0$.  
Let $C_4 \in \mb E(\{\pm 1\}, A)$ be the non-trivial extension. We have $r_1(E_0)= C_4$.

Suppose $\exists E \in \mb E(G,A)$ s.t. $r_0(E)= E^0$, then $r_1 \circ r_0(E) = r_1(E^\circ)= C_4 \neq 0$, on the other hand $r_3 \circ r_2(E) = 0$ which is a contradiction.

Hence we have in this case that the map $$r_0: \mb E(G,A) \to \mb E(G^\circ, A)^{\Gamma}$$ is not surjective.
\end{example}
\end{comment}

\begin{prop} \label{weil.prop} The restriction map $\mb E(W_\R,\Z/2\Z) \to \mb E(\C^\times,\Z/2\Z)^\Gamma$ is not surjective.
\end{prop}

\begin{proof} Let $\Sq: \C^\times \to \C^\times$ be the squaring map. Since $\Sq$ commutes with complex conjugation, this determines an extension  $\ms E_\bullet \in \mb E(\C^\times,\Z/2\Z)^\Gamma$ by Lemma \ref{liftingtheta}. Let $A = \Z/2\Z$. From the inclusions
\beq
\xymatrix{\mu_2  \ar[r] \ar[d] & G^\circ \ar[d]\\
Q \ar[r]   & G}
\eeq
we obtain the commutative diagram
\beq
\xymatrix{ \mb E(W_{\R}, A) \ar[d] \ar[r]& \mb E(\C^\times, A)  \ar[d]  \\
	\mb E(Q, A) \ar[r] & \mb E(\mu_2, A). }
	\eeq

Let us write $\ms E_2=(E_2,i_2,p_2)$ for the restriction of $\ms E_\bullet$ to $\mu_2$. Then $\ms E_2$ is nontrivial, since $E_2$ contains elements of order $4$.

It is easy to see that restriction  $H^1(Q,A) \to H^1(\mu_2,A)$ is the zero map, since each linear character of $Q$ is trivial on $\mu_2$, the derived group of $Q$.
From the description of $H^*(Q,A)$ in  \cite[Example 1.9, page 165]{Milgram}  or \cite[Section 3.2]{Malik.Spallone.SL}, one sees that the second cohomology group $H^2(Q,A)$ is contained in the subring of $H^*(Q,A)$ generated by $H^1(Q,A)$. Therefore the restriction $H^2(Q,A) \to H^2(\mu_2,A)$ is also the zero map, as claimed. In particular, the extension $\ms E_2$ is not in the image.

If there were an extension $\ms E \in \mb E(W_{\R},A)$ restricting to $\ms E_\bullet$, then its restriction to $Q$ would further restrict to $\ms E_2$, contradicting the above.
\end{proof}

\end{example}
  
   \subsection{Extensions of Semidirect Products}
   
   
   Let $H$ be a connected group, and $\Gamma$ a group acting discretely on $H$. 
We may form the semidirect product $G= H \rtimes \Gamma$.
   
   
 \begin{prop}\label{surj}  With the above notation, the restriction map
   \beq
   \mb E(G,A) \to \mb E (H,A)^{\Gamma}
  \eeq
  is a surjection.
  \end{prop}   
   
  \begin{proof} 
    Let  $\ms E_\bullet=(E_\bullet, i_\bullet,p_\bullet)  \in  \mb E (H,A)^{\Gamma}$. Suppose first that $E_\bullet$ is connected. 
    For each $\theta \in \Gamma$, there is a  lift $\tilde \theta$ of $\theta$ to $E_\bullet$ by Lemma \ref{liftingtheta}.
   In fact $\tilde \theta$ is unique:  Suppose $\tilde{\theta}_1$ and $\tilde{\theta}_2$ are lifts of $\theta$. Then we may define a continuous function $f: E_\bullet \to A$ by 
   \beq
   f(e^0)=\tilde{\theta_2}^{-1}(e^0)\tilde{\theta_2}(e^0).
   \eeq
    Since $E_\bullet$ is connected, $f$ is constant, and we deduce that $\tilde{\theta_1} = \tilde{\theta_2}$. The prescription $\theta \mapsto \tilde \theta$ is therefore a well-defined map $\Gamma \to \Aut(E_\bullet)$,
necessarily a homomorphism.  This corresponds to an action of $\Gamma$   on $E_\bullet$, and so we may form the  semidirect product $E=E_\bullet \rtimes \Gamma$. Clearly this gives an extension of $G$ by $A$  whose restriction to $H$ is  $E_\bullet$.
   
For $E_\bullet$ disconnected, write $E_\bullet^\circ$ for the connected component of $1$. Since $p_\bullet$ is an open map, the image of $E_\bullet^\circ$ is open in $H$, and hence closed. It follows that  $p_\bullet(E_\bullet^\circ)=H$. Let $A_1=A \cap E_\bullet^\circ$, and write $i_1: A_1 \hookrightarrow E_\bullet^\circ$ for the inclusion.
Then 
\beq
\ms E_\bullet':= (E_\bullet^\circ, i_1,p_\bullet|_{E_\bullet^\circ})
\eeq
 is a $\Gamma$-invariant $A_1$-extension of $G$, and $(E_\bullet^\circ)_{A} \cong E_\bullet$.
 
 
Recall that by Proposition \ref{likely.commutes}, the diagram
\beq
\xymatrix{
	\mb E(G,A_1) \ar[r]  \ar[d]& \mb E(H,A_1)^{\Gamma} \ar[d]\\
	 \mb E(G,A) \ar[r] & \mb E(H,A)^{\Gamma}\\
}
\eeq 
 commutes. Since $E_\bullet^\circ$ is connected, we have a lift $\ms E_1 \in \mb E(G,A_1)$ of $\ms E_\bullet'$, and this maps down to the required lift $\ms E$ of $\ms E_\bullet$.   \end{proof}
   
    


 \section{$G$-bundles} \label{G.bund.sec}

In this section we review the theory of (principal) $G$-bundles.  When $G$ is abelian, there is a Baer sum. When $G$ is discrete, $G$-bundles over a nice connected base are  classified by homomorphisms from the fundamental group into $G$.

\subsection{Definition} \label{g.bundle.defs}
 Let $G$ be a topological group, and $B$ a regular topological space. Suppose $X$ is CG with a right $G$-action, and $p:X \to B$ is a map. The pair $(X,p)$ is a \emph{$G$-bundle}, provided for all $b \in B$, there is a neighborhood $U \ni b$ and a homeomorphism $h_U: U \times G \to p^{-1}(U)$ so that for all $u \in U$ and $g,g' \in G$, we have $p(h_U(u,g))=u$ and $h_U(u,gg')=  h_U(u,g) \cdot g'$.

\begin{remark} Requiring that $X$ be CG and $B$ be regular sidesteps a technical issue. Since $B$ is regular, we may assume that $U$ is a \emph{regular} open set. This implies that the preimage $p^{-1}(U)$ is CG by \cite[Lemma A.1]{Uribe}, since $X$ is CG. Therefore if $h_U$ is a homeomorphism when we give $U \times G$ the product topology, it must be that $U \times G$ is CG, so $U \times G=U \times_k G$.  \end{remark}
 
 A \emph{morphism} of $G$-bundles over $B$ is a map $f: X \to X'$ with $f(x \cdot g)=f(x) \cdot g$. Write $\Bun_B(G)$ for the category of $G$-bundles with base space $B$. Write $\bun_B(G)$ for the set of $G$-bundles modulo isomorphism.
 
Let $p:X \to B$ be a $G$-bundle. When $B'$ is a regular space and $f: B' \to B$ is a map, write $f^*(X) \subseteq B' \times_k X$ for the pullback. Projection of the pullback to $B$ is a $G$-bundle, and projection to $X$  gives an \emph{overmap} $\tilde f: f^*(X) \to X$ which is $G$-equivariant.

\begin{lemma} \cite[Corollaire, page 94]{Bou.AT} \label{bou.pullback}  Let $(X,p)$ be a $G$-bundle over a space $B$ and $(X',p')$ be a $G$-bundle over a space $B'$. Suppose that $f: B' \to B$ and $f': X' \to X$ are continuous with $p \circ f'=f \circ p'$, and moreover that $f'$ is $G$-equivariant. Then 
\beq
	\xymatrix{
		X' \ar[r]^{f'} \ar[d]^{p'} &X \ar[d]^{p}\\
		B' \ar[r]^{f}   & B  \\
	}
	\eeq 
	is a pullback diagram.
  \end{lemma}

\begin{defn} \label{induced.bundle} Let $(X,p)$ be a $G$-bundle over $B$, let $G'$ be a CG topological group, and $\psi: G \to G'$ a homomorphism. Write $X[G']$ for the quotient of $X \times_k G'$ by the equivalence relation $(x,g') \sim (x \cdot g,\psi(g)^{-1}g')$,
with $x \in X$, $g \in G$ and $g' \in G'$. There is an evident right action of $G'$ on $X[G']$. 
The map taking the class of $(x,g')$ to $p(x)$ makes $X[G']$ into a $G'$-bundle over $B$, called the $G'$-bundle \emph{induced from} $(X,p)$ by $\psi$.
\end{defn}  
Write $f: X \to X[G']$ for the map over $B$, taking $x$ to the class of $(x,1_{G'})$, and satisfying
\begin{equation} \label{universal4}
f( x \cdot g)= f(x) \cdot \psi(g).
\end{equation} 
Then $f$ is universal in the sense that if $Y$ is a $G'$-bundle over $B$ and $f': X \to Y$ is a map over $B$ satisfying \eqref{universal4}, then there is a unique isomorphism $\theta: X[G'] \overset{\sim}{\to} Y$ as $G'$-bundles over $B$.
(See  \cite[Section 6.61]{Bourbaki.VD}.)


\subsection{Baer Sum of $A$-Bundles}

Let $A$ be an abelian topological group. Given $(X_1,p_1),(X_2,p_2) \in \Bun_B(A)$, let $X_3' \subseteq X_1 \times_k X_2$ be the fibre product:
\beq
	\xymatrix{
		X_3' \ar[r] \ar[d]^{p_3'} &X_2 \ar[d]^{p_2}\\
		X_1 \ar[r]^{p_1}   & B  \\
	}
	\eeq 
	
If we view $X_3'$ as an $A$-space under the action $(x_1,x_2) \cdot a=(x_1,x_2 \cdot a)$, then $(X_3',p_3')$ is an $A$-bundle.
	
We may also view $X_3'$ as an $A$-space under the anti-diagonal action $(x_1,x_2) \cdot a=(x_1 \cdot a,x_2 \cdot a^{-1})$. Let $X_3$ be the quotient of $X_3'$ by the anti-diagonal $A$-action. Then $X_3$ itself has an $A$-action by $(x_1,x_2) \cdot a=(x_1 \cdot a,x_2)$.
 
 If we define $p_3:X_3 \to B$ by $p_3(x_1,x_2)=p_1(x_1)$, then $(X_3,p_3) \in \Bun_B(A)$. (A common trivialization of $X_1$ and $X_2$ over $B$ will trivialize $X_3$ as well.) We call $(X_3,p_3)$ the \emph{Baer sum} of the two bundles.    

 
 \subsection{Associated Long Exact Sequence}  
  
    
  \begin{thm}  \label{homotopy.fibration}  Let $p:X \to B$ be a $G$-bundle, with $B$ path-connected. Let $x \in X$ with $p(x)=b$. Then there is a long exact sequence
  \beq
\cdots \to \pi_n(G,1_G)  \to  \pi_n(X,x) \to  \pi_n(B,b)  \to  \pi_{n-1}(G,1_G)  \to  \cdots,
\eeq
of groups, which ends with
\beq
\cdots \to \pi_1(B,b) \to \pi_0(G,1_G).
\eeq
When $X$ is path-connected, the last map is surjective. The image of $\pi_1(G)=\pi_1(G,1_G)$ in $\pi_1(X)=\pi_1(X,x)$ is contained in the center of $\pi_1(X)$.
 \end{thm}
 
 \begin{proof}
 See \cite[Section 17.11]{Steenrod.FB}, except for the last point. The group action of $G$ on $X$ induces an action of $\pi_1(G)$ on $\pi_1(X)$. One checks that for $a,b \in \pi_1(G)$ and $x,y \in \pi_1(X)$ we have $(a * b) \circ (x *y)=(a \circ x)*(b \circ y)$, where $*$ is the usual concatenation of paths, and $\circ$ is the action.
From this, it is easy to deduce the last point.
 \end{proof}

 
 

\subsection{$D$-covers}
 
 
When $D$ is a discrete group,  a $D$-bundle is necessarily a covering space, often simply called a \emph{$D$-cover}. 
\begin{comment}
Another approach to $D$-covers is through even actions, as follows.

\begin{defn} A right action of $D$ on a space $X$ is said to be \emph{even}, provided for all $x \in X$ there is a neighborhood $U \ni x$ so that for all $D \ni d \neq 1_D$, we have
\beq
U \cdot d \cap U=\emptyset.
\eeq
\end{defn}
(In the literature, many would say   ``proper discontinuous'' rather than ``even'', but we are following \cite{fulton.AT}.) For instance, the action of $D$ on a $D$-cover is even. Conversely, the quotient of a space by an even action is a $D$-cover (\cite[Lemma 11.17]{fulton.AT}).
 \end{comment}
 Let $B$ be path-connected, and $p:(X,x_0) \to (B,b)$ a pointed $D$-cover.  Given   $[\gm] \in \pi_1(B,b)$, let $\tilde \gm$ be the path in $X$ lifting $\gm$ with $\tilde \gm(0)=x_0$. Define $\partial_p([\gm])$ to be the unique $d \in D$ so that $x_0 \cdot d=\tilde \gm(1)$.
 
\begin{thm} \label{groth.}  \cite[Section 14a]{fulton.AT}
The above prescription defines a homomorphism $\partial_p: \pi_1(B,b) \to D$. If $B$ is nice and path-connected, then the correspondence $(X,p) \leadsto \partial_p$ gives a natural bijection
\beq
\{ \textrm{pointed $D$-covers of $B$} \}/ \text{isom} \overset{\sim}{\to}  \Hom(\pi_1(B),D).
\eeq
\end{thm}

Note that $\partial_p$ agrees with the connecting map for the LES of homotopy groups, as defined in \cite[page 453]{Bredon}.
	   
	
\begin{comment}  
\begin{lemma} \label{quotient.cover.lem} Suppose $G$ is an LPC group. A   $G$-bundle $p:X \to B$ descends to a $\pi_0(G)$-cover $X/ G^\circ \to B$.
\end{lemma}

\begin{proof}    For $x\in B$, take $x \ni U \subset B$  such that   $h: p^{-1}(U) \overset{\sim}{\to} U \times G$ is a trivialization. Choose $U^\bullet := h^{-1}(U \times G^\circ) \subseteq p^{-1}(U)$. This is an open set by our hypothesis on $G$. We can see that the action of $\pi_0(G)$ on $X/ G^\circ$ is even by taking $U^\bullet / G^\circ \subset X / G^\circ$, since $U^\bullet (g_1G^\circ) \cap U^\bullet (g_2G^\circ)= \emptyset$ whenever $g_1^{-1}g_2 \notin G^\circ$. Thus   $X/ G^\circ$ is a $\pi_0(G)$-cover over $B$.  
\end{proof}

In particular, since $\pi_0(G)$ is discrete, given a   $G$-bundle $p:X \to B$ we obtain a group homomorphism 
\begin{equation} \label{fulton's.hom}
\partial_p: \pi_1(B,b) \to \pi_0(G)
\end{equation}
 by Theorem \ref{groth.}, when $B$ is nice and connected.
 
\end{comment}


% \section{Extensions of Connected Groups}

%In this section we make precise the relationship between central extensions and covering spaces, when $G$ is connected. 

 
\subsection{Central Extensions  as Covers.}
Let $G$ be a path-connected CW-group, and $\ms E=(E,i,p) \in {\bf E}(G,A)$. 

\begin{lemma} $p:E \to G$ is an $A$-cover.
\end{lemma}

\begin{proof} 
It is enough to see that the action of $A$ on $E$ by left translation is even. Pick a neighborhood $U$ of $1_E$  so that $U \cap i(A)=\{ 1_E\}$, and a neighborhood $V$ of $1_E$  so that $V V^{-1} \subseteq U$.
It is clear that for $0_A \neq a \in A$, the intersection $V \cdot a \cap V =\emptyset$.
\end{proof}

Let $\ms E=(E,i,p) \in {\bf E}(G,A)$.   Since $G$ is locally path-connected, so too is $E$, hence $E^\circ$ is open in $E$.
\begin{lemma} \label{dec.30} We have $E=AE^\circ$.
\end{lemma}

\begin{proof}	
The subgroup $E^\circ$ and each of its cosets are open, hence $E'=AE^\circ$ is an open subgroup of $E$. Since $p$ is an open map, the image of each coset of $E'$ is open in $G$. Moreover the images of these cosets of $E'$ partition $G$.
Since $G$ is connected, we have $G=p(E')$, and therefore $E=E'$.
%Since $G$ is connected, in fact $G$ must equal $p(E^\circ )$, which implies that $E=A E^\circ$.
\end{proof}	

\begin{prop} \label{dont.forget} Let $\ms E_1,\ms E_2 \in {\bf E}(G,A)$. If the projections $p_1: E_1 \to G$ and $p_2:E_2 \to G$  are isomorphic as $A$-covers, then  $\ms E_1 \cong \ms E_2$.
\end{prop}

\begin{proof}
Let $f: E_1 \to E_2$ be an isomorphism in the category of $A$-covers of $G$. By translating by $A$ we may assume that $f(1_{E_1})=1_{E_2}$. Consider the map
\beq
E_1^\circ \times E_1^\circ \to A
\eeq
given by $(x,y) \mapsto f(xy)^{-1}f(x)f(y)$. Since the domain is connected and the codomain is discrete, this is in fact  the constant map $(x,y) \mapsto 0_A$. 
Thus the restriction of $f$ to $E_1^\circ$ is a group homomorphism. Using that $E_1=AE_1^\circ$ by Lemma \ref{dec.30}, we see that $f$ itself is a group homomorphism. The result follows.\
\end{proof}	
	
\begin{thm} \label{forget.group} \label{cmi.thurs} When $G$ is a path-connected CW-group, forgetting the group structure of an extension gives a  bijection
\beq
\mb E(G,A) \overset{\sim}{\to} \{ \text{pointed $A$-covers of $G$} \}/ \text{isom}.
\eeq
\end{thm}
	 
	
Combining  Theorem \ref{groth.} with Theorem \ref{forget.group} gives:
\begin{thm} For $G$ a path-connected CW-group, the map taking an extension $\ms E =(E,i,p) \in{\bf E}(G,A)$  to the connecting homomorphism $\partial_{\ms E}$ (from the $A$-bundle $p$) is a bijection
\beq
\mb E(G,A)   \overset{\sim}{\to} \Hom(\pi_1(G),A).
\eeq
\end{thm}

 

 


\begin{comment}
\subsection{Central Extensions  as Covers.}
%Let $G$ be a path-connected CW-group,   $A$ a discrete abelian group, and $\ms E=(E,i,p) \in {\bf E}(G,A)$. 

\begin{thm} \label{forget.group} When $G$ is a path-connected CW-group, forgetting the group structure of an extension gives a  bijection
\beq
\mb E(G,A) \overset{\sim}{\to} \{ \text{pointed $A$-covers of $G$} \}/ \text{isom}.
\eeq
\end{thm}

\begin{proof} (Sketch) For a central extension $\ms E=(E,i,p) \in {\bf E}(G,A)$, we have $E=AE^\circ$. By a connectivity argument, one proves that a morphism of $A$-covers gives rise to a group isomorphism.
\end{proof} 	 
	
Combining  Theorem \ref{groth.} with Theorem \ref{forget.group} gives:
\begin{thm} \label{cmi.thurs} For $G$ a path-connected CW-group, the map taking an extension $\ms E =(E,i,p) \in{\bf E}(G,A)$  to the connecting homomorphism $\partial_{\ms E}$ (from the $A$-bundle $p$) is a bijection $\mb E(G,A)   \overset{\sim}{\to} \Hom(\pi_1(G),A)$.
\end{thm}

 



 
\begin{lemma} $p:E \to G$ is an $A$-cover.
\end{lemma}
 
\begin{proof} 
It is enough to see that the action of $A$ on $E$ by left translation is even. Pick a neighborhood $U$ of $1_E$  so that $U \cap i(A)=\{ 1_E\}$, and a neighborhood $V$ of $1_E$  so that $V V^{-1} \subseteq U$.
It is clear that for $0_A \neq a \in A$, the intersection $V \cdot a \cap V =\emptyset$.
\end{proof}
 
Let $\ms E=(E,i,p) \in {\bf E}(G,A)$.   Since $G$ is locally path-connected, so too is $E$, hence $E^\circ$ is open in $E$.
\begin{lemma} \label{dec.30} We have $E=AE^\circ$.
\end{lemma}

\begin{proof}	
The subgroup $E^\circ$ and each of its cosets are open, hence $E'=AE^\circ$ is an open subgroup of $E$. Since $p$ is an open map, the image of each coset of $E'$ is open in $G$. Moreover the images of these cosets of $E'$ partition $G$.
Since $G$ is connected, we have $G=p(E')$, and therefore $E=E'$.
%Since $G$ is connected, in fact $G$ must equal $p(E^\circ )$, which implies that $E=A E^\circ$.
\end{proof}	

\begin{prop} \label{dont.forget} Let $\ms E_1,\ms E_2 \in {\bf E}(G,A)$. If the projections $p_1: E_1 \to G$ and $p_2:E_2 \to G$  are isomorphic as $A$-covers, then  $\ms E_1 \cong \ms E_2$.
\end{prop}

\begin{proof}
Let $f: E_1 \to E_2$ be an isomorphism in the category of $A$-covers of $G$. By translating by $A$ we may assume that $f(1_{E_1})=1_{E_2}$. Consider the map
\beq
E_1^\circ \times E_1^\circ \to A
\eeq
given by $(x,y) \mapsto f(xy)^{-1}f(x)f(y)$. Since the domain is connected and the codomain is discrete, this is in fact  the constant map $(x,y) \mapsto 0_A$. 
Thus the restriction of $f$ to $E_1^\circ$ is a group homomorphism. Using that $E_1=AE_1^\circ$ by Lemma \ref{dec.30}, we see that $f$ itself is a group homomorphism. The result follows.\
\end{proof}	
	
 

 \end{comment}


 \section{Classifying Spaces} \label{class.space.section}
 
 In this section we recall the general notion of a classifying space, and review in particular the Milgram-Steenrod  $BG$-functor and its properties. 
 Applying this functor to $\ms E \in {\bf E}(G,A)$ gives a $BA$-bundle over $BG$ which we will use heavily. We also exhibit  here a bijection from $\mb E(G,A)$ to $H^2(BG,A)$ when $G$ is connected, by using the connecting map of the LES of homotopy groups. Finally, we study the structure of the $K(A,2)$-space $B(BA)$.
 
 \subsection{Universal $G$-bundles} \label{BG.section}
 
 
 \begin{defn} A $G$-bundle   $p:E \to B$ is \emph{universal}, provided $E$ is contractible, and $B$ is paracompact. In this case, we call $B$ a \emph{classifying space} for $G$.
\end{defn}

\begin{example} If $G=\Z$, then $\exp: \C \to \C^\times$ is a universal $\Z$-bundle. So $\C^\times$ is a classifying space for $\Z$.
\end{example}

Let $E_u \to B_u$ be a universal $G$-bundle. When $B$ is paracompact,  the prescription
\beq
f \leadsto (f^*(E_u) \to B)
\eeq
gives a bijection 
 \begin{equation} \label{univ.class.propz}
[B,B_u] \overset{\sim}{\to} \bun_B(G).
\end{equation}
Here  $[X,Y]$ is the set of homotopy classes of maps from $X$ to $Y$. See, for instance \cite[page 45]{Milgram}. 

 \begin{lemma} \label{bu.he} Any two classifying spaces for $G$ are homotopically equivalent.
 \end{lemma}
 
 \begin{proof} Suppose $p: E \to B$ and $p': E' \to B'$ are both universal $G$-bundles. Let $f: B \to B'$ so that $E$ is the pullback of $E'$, and $f': B' \to B$ so that $E'$ is the pullback of $E$. Then $E$ is the pullback of $E$ under $f' \circ f$, and therefore $f' \circ f$ is homotopic to the identity map on $B$. Similarly, $f \circ f'$ is homotopic to the identity map on $B'$, and so $B$ and $B'$ are homotopy equivalent.
 \end{proof}
 
 
 \subsection{Milgram-Steenrod's Classifying Space}
  Given a CW-group $G$, Steenrod in \cite{Steenrod} recursively constructs a sequence
 \beq
 D_0 \subseteq E_0 \subseteq D_1 \subseteq E_1 \subseteq \cdots
 \eeq
 of $CW$-complexes, each a subcomplex of the next, so that each $E_i$ has a free right $G$-action, and each $D_n$ comes with a specific contraction.
 
 $D_0=\{e\}$, $E_0=G$, and
 $D_1$ is the reduced cone $G \wedge I$. The latter is the topological quotient of $G \times I$ by the subspace
 \beq
 (G \times \{ 0\}) \cup ( \{1_G\} \times I).
 \eeq
 Generally, $D_{n+1}$ is the ``enlargement to $E_{n}$ of the contraction of $D_n$'', meaning it is defined as the pushout 
 

 \beq
	\xymatrix{
		D_n \wedge I \ar[r] \ar[d]  &D_n \ar[d] \\
		E_n \wedge I \ar[r]   & D_{n+1}  \\
	}
	\eeq	
	
	and $E_{n+1}$ is the ``enlargement to $D_{n+1}$ of the $G$-action on $E_{n}$'', defined as the pushout
	 
 \beq
	\xymatrix{
		  E_n \times G \ar[r] \ar[d]  &E_n \ar[d] \\
		  D_{n+1} \times G \ar[r]   & E_{n+1}  \\
	}.
	\eeq 
	
 Then $EG$ is the union of the $E_n$'s, equivalently the union of the $D_n$'s. We give it the weak topology of the $D_n$'s. From being the union of the $E_n$'s, it inherits a free right $G$-action. From being the union of the $D_n$'s, it inherits a contraction.
 
 Finally, the quotient of $EG$ by $G$ defines a universal $G$-bundle $p:EG \to BG$. Then both $EG$ and $BG$ are CW-complexes.
 
 
 \begin{example} When $G$ is trivial, $D_n=E_n$ is a point for all $n$. So, $EG=BG$ is a point.
 \end{example}
  
  \begin{example} When $G=C_2$,  
  $D_n$ is the $n$-disc, and $E_n=S^n$ with antipodal action of $G$. So, $EG=S^\infty$ and $BG=\R \mb P^\infty$.
  \end{example} 
  
  \subsection{$EG$ as a group}
  
Steenrod defines in \cite{Steenrod} a group structure on $EG$, making it a CW-group. We record some of its properties here, partly just to give some flavor to the reader.
  
\begin{itemize}
\item $EG$ is generated as a group by $D_1$. 
\item The identity $G =E_0$ gives an injective homomorphism $i: G \to EG$.
\item The $G$-action on each $E_n$ agrees with the multiplication by $E_0$.
%\item The multiplication map takes $D_m \times D_n \to D_{m+n}$.
\end{itemize}
  
  Since $EG$ is generated by $D_1$, the map 
  \beq
  G \times I \twoheadrightarrow G \wedge I= D_1
  \eeq
  induces a surjection $F(G \times I) \twoheadrightarrow EG$,
  where $F(G \times I)$ is the free group on the set $G \times I$. From this surjection, we can identify the group $EG$   with the quotient of $F(G \times I)$ by the following relations
 \begin{itemize}
 \item $(g,0)=(1_G,t)=1$ for $g \in G$ and $t \in I$.
 \item $(g,t)(g',t)=(gg',t)$ for $g,g' \in G$ and $t \in I$.
 \item If $0<t'<t \leq 1$, then $(g,t)(g',t')=(gg'g^{-1},t')(g,t)$ for $g,g' \in G$.
 \end{itemize}
With this notation, the map $i: G \to EG$ is the homomorphism defined by $i(g)=(g,1)$. Write $BG$ for the quotient $EG/i(G)$. Let $e_{BG} \in BG$ be the image of $1_{EG}$ under the quotient. When $G$ is abelian, then $EG$ is abelian, and so $BG$ is a CW-group.
 
Given a homomorphism $\varphi: G \to G'$ of topological groups, the map $G \times I \to G' \times I$ defined by $(g,t) \mapsto (\varphi(g),t)$
induces a topological group homomorphism $E\varphi: EG \to EG'$
with the property that $E\varphi(x \cdot g)=E\varphi(x) \cdot \varphi(g)$. Hence it descends to a continuous map $B\varphi: BG \to BG'$.
  
  
  \begin{prop}  \label{fibration.normal}  \cite[Theorem 7.7]{Piccinini} Let  $G$ be a CW-group. If $\mathscr E=(E,i,p) \in  {\bf E}(G,A)$, then the induced map $Bp: BE \to BG$ is a $BA$-bundle. 
  \end{prop}
  
\begin{remark} The group $E$ is a CW-group, since it is an $A$-cover of the CW-group $G$. (See \cite[Theorem 8.10, page 198]{Bredon}.)  Therefore the hypotheses of \cite[Theorem 7.7]{Piccinini} are satisfied.
\end{remark}
 
 \subsection{From Extensions to Bundles} \label{rain.car}
 Let $G$ be a $CW$-group, and  $\ms E=(E,i,p) \in  {\bf E}(G,A)$. As noted above (Theorem \ref{fibration.normal}), $Bp: BE \to BG$ is a $BA$-bundle, and we write $\ms B \ms E=(BE,Bp)$. 
   It is easy to see that an isomorphism $\ms E_1 \cong \ms E_2$ induces a $BA$-bundle isomorphism from $\ms B \ms E_1$ to $\ms B \ms E_2$.
  
So we have a functor $ \ms B: {\bf E}(G,A) \to  \Bun_{BG}(BA)$ which induces a well-defined map $\ms B: \mb E(G,A) \to \bun_{BG}(BA)$.
Suppose $G,G'$ are CW-groups, $\ms E \in {\bf E}(G,A)$ and    $\ms E' \in {\bf E}(G',A)$, and  $f: E \to E'$ is a homomorphism with $f \circ i=i' $. Then one has a commutative diagram
 \beq 
	\xymatrix{
		A \ar[d]^{\id} \ar[r]^i & E  \ar[d]^{f}\ar[r]^p & G \ar[d] \\
		A \ar[r]^{i'}  & E' \ar[r]^{p'} & G'.}
	\eeq 
    
   Applying $B$ to the group action diagram 
   
    \beq 
	\xymatrix{
		E \times A \ar[r] \ar[d]_{f \times \id_A} & E  \ar[d]^{f}\\
		E' \times A \ar[r]^{ } & E',}
	\eeq  
 shows that $Bf$ is a $BA$-equivariant morphism from $BE$ to $BE'$.
  
    
    
\begin{prop} \label{Bp.funct}  If $\varphi: G' \to G$ is a homomorphism, and  $\ms E \in {\bf E}(G,A)$ then 
\beq
\ms B(\varphi^* \ms E) \cong (B \varphi)^* (\ms B \ms E).
\eeq
\end{prop}
 
 \begin{proof}
Applying  $\ms B$  to the pullback square for $\varphi^*E$ gives the commutative diagram 
 \beq 
	\xymatrix{
		B (\varphi^*E) \ar[r] \ar[d] & BE  \ar[d]^{Bp} \\
		BG' \ar[r]^{B \varphi} & BG}.
	\eeq  

We obtain the desired isomorphism by Lemma \ref{bou.pullback}.	
\begin{comment}
Thus we obtain a morphism 
\beq
\ms B(\varphi^* \ms E) \overset{\sim}{\to}(B \varphi)^* (\ms B \ms E)
\eeq
 of $BA$-bundles over $BG'$. 
 The proposition follows since all morphisms are isomorphisms. \cite[Theorem 3.2]{Husemoller}.
 \end{comment}
   
  \end{proof}
  
  \begin{prop} \label{Baer.2.Baer} The $\ms B$-functor takes Baer sums to Baer sums.
  \end{prop}
  
  
  \begin{proof} Let $\ms E_1,\ms E_2 \in{\bf E}(G,A)$ as in Section \ref{Baer.groups}. Note that the projection $E_3' \to E_2$ gives an extension  $\ms E_3' \in {\bf E}(E_2,A)$, and therefore $BE_3' \to BE_2$ is a $BA$-bundle.
  Both $BE_3'$ and $BE_1 \times_{BG} BE_2$ have two actions of $BA$. 
  The first action of $BA$ on $BE_3'$ comes from $Bi_3$, and the first action on the fibre product comes from the $BA$-action on $BE_1$. The second action on $BE_3'$ comes from $B\triangledown$, and the second action on the fibre product comes from the antidiagonal action of $BA$ on the product.
   
Consider the function
   \beq
 f=B\pr_1 \times B\pr_2: BE_3' \to BE_1 \times_{BG} BE_2.
 \eeq
 One checks that $f$ is $BA$-equivariant with respect to both actions. 
 From the first equivariance, it induces an isomorphism of $BA$-bundles over $BE_2$. 
 From the second equivariance, we see that $f$ descends to an isomorphism from $\ms B \ms E_3$ to the Baer sum
 $\ms B \ms E_1 + \ms B \ms E_2$.
  \end{proof}
  
 
 \begin{prop} $\ms B$ is natural in $A$.
 \end{prop}
 
 \begin{proof} Let $\psi: A \to A'$, and $\ms E \in {\bf E}(G,A)$. 
 Let $E'$ be as in \eqref{e.prime}. From Diagram \eqref{diag.ek}, we obtain a map $f': BE \to BE'$ over $BG$, satisfying $f'(ax)=B \psi(a) f'(x)$
 for all $x \in BE$ and $a \in BA$.  Thus as in Section \ref{g.bundle.defs}
 we deduce an isomorphism
 \beq
 \theta: BE[BA'] \overset{\sim}{\to} BE'
 \eeq
 of $BA'$-bundles over $BG$.
 \end{proof}
 
 
  

\section{Homotopy Theory of Classifying Spaces} \label{long.ex.seq.section}
 In this section let $G$ be a CW-group. 
\subsection{The Long Exact Sequence}
Applying Theorem \ref{homotopy.fibration} to the   $G$-bundle $EG \to BG$ gives a  long exact sequence
\beq
\cdots \to \pi_n(G)  \to  \pi_n(EG) \to  \pi_n(BG)  \to  \pi_{n-1}(G)  \to  \cdots,
\eeq
which is natural in $G$. Since $EG$ is contractible, this gives isomorphisms 
\begin{equation} \label{bottle}
\partial_n=\partial_n^G: \pi_n(BG) \overset{\sim}{\to} \pi_{n-1}(G),
\end{equation} 
for $n \geq 1$.   The set of path components $\pi_0(G,1)=G/G^\circ$ has a group structure, and 
\begin{equation}\label{pi1bg}
\partial_1: \pi_1(BG) \to \pi_0(G,1)
\end{equation}
is an isomorphism. By Hurewicz's theorem, this gives rise to natural isomorphisms
 \begin{equation} \label{stapler} 
 \partial_1: H_1(BG) \overset{\sim}{\to}  (G^\circ \bks G)^{\ab},
  \end{equation}
 and
\beq
\partial_1^*: \Hom(H_1(BG),A) \to \Hom_c(G,A).
\eeq
 Since $A$ is discrete, 
\beq
\Hom_c(G,A)=\Hom((G^\circ \bks G)^{\ab},A).
\eeq   
 By the UCT, we have
\beq
0 \to \Ext^1(H_0(BG),A) \to H^1(BG,A) \overset{q^1}{\to} \Hom(H_1(BG),A) \to 0.
\eeq
Since $H_0(BG)$ is free, the $\Ext$-term vanishes, and so $q^1$ gives an isomorphism from $H^1(BG,A)$ to $\Hom(H_1(BG),A)$. 
Combining this with   \eqref{stapler} gives:



\begin{prop} \label{m^1}
The composition 
\beq
 m^1= \partial_1^* \circ q^1:H^1(BG,A) \to \Hom_c(G,A)
 \eeq
  is a natural isomorphism.
\end{prop}
     
The Hurewicz map $h_2: \pi_2(BG) \to H_2(BG)$ dualizes to a homomorphism
\beq
h^2=h_2^*: \Hom(H_2(BG),A) \to \Hom(\pi_2(BG),A).
\eeq
Moreover the Universal Coefficient Theorem gives an exact sequence
\begin{equation} \label{UCT}
0 \to \Ext^1_{\Z}(H_1(BG), A) \to H^2(BG,A) \overset{q^2}{\to} \Hom_{\Z}(H_2(BG),A) \to 0.
\end{equation}
 
The map $m^2: H^2(BG,A) \to \Hom(\pi_1(G),A)$ defined by $m^2=(\partial_2^*)^{-1} \circ h^2 \circ q^2$ is the composition of natural maps and hence is natural itself.
 
 \subsection{Case of $G$ connected}
 \begin{prop} \label{Gconnect.m2} When $G$ is connected, $m^2$ is a bijection.
 \end{prop}
 
 \begin{proof} 
 By \eqref{bottle}, the group $\pi_1(BG)$ is trivial, and so necessarily $H_1(BG)$ is trivial. It then follows that  $h_2$ is an isomorphism. The $\Ext$-term in \eqref{UCT} vanishes, and so $m^2$ is the composition of two bijections.
   \end{proof}
   
 Combining $m^2$ with the bijection of Theorem \ref{cmi.thurs} gives:
  
 \begin{thm}\label{Gconnect.equiv} Let $G$ be a connected CW-group. The above maps give  an isomorphism
\beq
H^2(BG,A) \overset{\sim}{\to} \Hom(\pi_1(G),A) \overset{\sim}{\to} \mb E(G,A).
\eeq
\end{thm}
\begin{proof}
The first equivalence follows from Proposition \ref{Gconnect.m2}.
\end{proof}


\begin{defn} \label{beta.defn} For a  connected CW-group $G$, write
\beq
\beta_G: \mb E(G,A)  \to H^2(BG,A)  
\eeq
for the inverse of the above isomorphism.
\end{defn}

Note that $\beta_G(\ms E)$ is characterized by the property that
\begin{equation} \label{characterized} 
m^2(\beta_G(\ms E))=\partial_{\ms E},
\end{equation}
where $\partial_{\ms E}$ is the connecting homomorphism of Theorem \ref{cmi.thurs}.
 
 
 \subsection{Case of $G$ discrete}
 
Now let $G$ be a discrete group.  If $p: X \to BG$ is a $BA$-bundle, then applying Theorem \ref{homotopy.fibration} gives 
 a central extension
 \beq
 0 \to A \to \pi_1(X) \to G \to 1,
 \eeq
since $\pi_2(BG)=\pi_1(G)=0$ and $BA$ is connected.
 
This gives a map $\ms P: \bun_{BA}(BG) \to \mb E(G,A)$. 
If $(E,p,i) \in \mb E(G,A)$, then according to  \cite[1B.9, page 90]{hatcher}, we have $\pi_1(Bp)=p$ and $\pi_1(Bi)=i$.
 Therefore $\ms P \circ \ms B$ is the identity on $\mb E(G,A)$, and in particular:

 \begin{prop} \label{B.map.inj} When $G$ is discrete, the map $\ms B$ is injective.
 \end{prop}
   

 
\subsection{Normal Subgroups}


Let $G$ be a CW-group, and suppose $N \unlhd G$ is a closed normal subgroup, with $\ol G=N\backslash G$ also a CW-group.
In this section we give a variant of Proposition \ref{fibration.normal}, which is also asserted in \cite[page 48]{Milgram}, and \cite[Theorem 2.4.12]{Benson.II}. 
(We will show that $BG \to B \ol G$ is a fibration with homotopy fibre $BN$.) All products in this section are with the CG-topology as in Section \ref{CG.section}.

Define $B_GN=EG/N$. The group $G$ acts diagonally on the product $EG \times E\ol G$; let $B_uG$ be the quotient by this action. The second projection defines a map $p: B_uG \to B \ol G$.

\begin{lemma} \label{fibre.buns} The map $p$ is a fibre bundle with fibre $B_GN$.
\end{lemma}

\begin{proof} 
Since $p_{\ol G}: E \ol G \to B \ol G$ is a  $\ol G$-bundle, we can  choose an open $U \subseteq B \ol G$ such that $p_{\ol G}^{-1}(U) \cong U \times \ol G$. Moreover, 
\beq
\pr_2^{-1}(\ol p_{\ol G}^{-1}(U)) \cong EG \times U \times \ol G,
\eeq
 where the right hand side is understood as a $G$-space under $(x,u, \ol h) \cdot g=(x \cdot g,u,\ol {hg})$. The map $(x, \ol g) \mapsto x \cdot g$ induces a homeomorphism
 \beq
 (EG \times \ol G )/G \overset{\sim}{\to} EG/N=B_GN,
 \eeq
 with inverse induced by $x \mapsto (x,1_{\ol G})$. Thus 
\beq
\begin{split}
p^{-1}(U)  &\cong   (EG \times U \times \ol G)/G \\
		%&\cong U \times EG/N \\
		& \cong U \times B_GN. \\
		\end{split}.
		\eeq \end{proof}

Consider the map $i_G: B_GN \to B_uG$ induced from the inclusion $EG \to EG \times E \ol G$ into the first factor.

\begin{lemma} There are weak homotopy equivalences  $i_N:BN \to B_GN$ and $\ol{\Delta}:BG \to B_uG$ so that the following diagram commutes:
 \beq 
	\xymatrix{
		  BN\ar[r]^{Bi} \ar[d]^{i_N} & BG  \ar[r]  \ar[d]^{\ol \Delta} &B \ol G \\
		B_GN  \ar[r]^{i_G}& B_uG  \ar[ur]_p  }.
	\eeq 
	\end{lemma}

\begin{proof} 
The map $i_N: BN \to B_GN$ is induced from $EN \to EG$, and the map $\ol{\Delta}: BG \to B_uG$ is induced from the diagonal map $ \Delta: EG \to EG \times E \ol G$.
We have a morphism of $N$-bundles
 \beq 
	\xymatrix{
		  EN\ar[r]  \ar[d] & BN    \ar[d]   \\
		EG  \ar[r]  & B_GN  },
	\eeq 
which gives a commutative square
 \beq 
	\xymatrix{
		  \pi_i(BN) \ar[r]^\sim  \ar[d] & \pi_{i-1}(N)   \ar[d]^{||}   \\
		\pi_i(B_GN)  \ar[r]^\sim  & \pi_{i-1}(N)  }.
	\eeq 
Thus $i_N$ is a weak homotopy equivalence.
Similarly, $\Delta: EG \to EG \times E \ol G$ is a $G$-bundle morphism, and so the induced map $\ol{\Delta}:BG \to B_uG$ is a weak homotopy equivalence. 

\end{proof}

\begin{cor} \label{LES.N} We have a long exact sequence
\beq
\cdots \to \pi_{i+1}(B \ol G) \to \pi_i(BN) \to \pi_i(BG) \to \pi_i(B \ol G) \to \cdots,
\eeq
where  $\pi_i(BN) \to \pi_i(BG)$ and $\pi_i(BG) \to \pi_i(B \ol G)$ are the natural maps.
\end{cor}

\begin{proof}  There is a long exact sequence
\beq
\cdots \to \pi_{i+1}(B \ol G) \to \pi_i(B_GN) \to \pi_i(B_uG) \to \pi_i(B \ol G) \to \cdots
 \eeq
 attached to the fibre bundle $p$ of Lemma \ref{fibre.buns}, for instance by  \cite[Theorem 4.41 and Proposition 4.48]{hatcher}. Define the map  $\pi_{i+1}(B \ol G) \to \pi_i(BN)$ by composing the connecting map with the inverse map to $\pi_i(BN) \overset{\sim}{\to} \pi_i(B_GN)$. Clearly this gives the desired LES.
 \end{proof}
  

 
 Let $G$ be a CW-group, and put $\pi_0=\pi_0(G)$. The inclusion map $\iota$ and quotient map $\qoppa$ give an exact sequence
\beq
1 \to G^\circ \overset{\iota}{\to} G \overset{\qoppa}{\to} \pi_0 \to 1.
\eeq

 
 
 
\begin{cor} \label{isom.pi.bs} The map $\qoppa$ induces an isomorphism $B\qoppa_*: \pi_1(BG) \overset{\sim}{\to} \pi_1(B\pi_0)$.
 The map $\iota$ induces isomorphisms $B\iota_*: \pi_k(BG^\circ) \overset{\sim}{\to} \pi_k(BG)$ for $k \geq 2$. 
\end{cor}

\begin{proof} From Equation \eqref{pi1bg} we have $\pi_1(BG^\circ)=0$, moreover since $BG^\circ$ is connected we have $\pi_0(BG^\circ)=0$. So the first part follows from Corollary \ref{LES.N}. Since $\pi_0$ is discrete, we have $\pi_k(B\pi_0)=0$ for $k \geq 2$ and so the second part also follows.
\end{proof}
 
 
 
 
 \subsection{$K(A,2)$-Spaces} \label{ka2}
   When a path-connected topological space $K$ has $\pi_2(K) \cong A$ as its only nontrivial homotopy group, it is called a $K(A,2)$-space.   
 Fixing an isomorphism $\zeta: \pi_2(K)  \overset{\sim}{\to} A$, and composing with the Hurewicz map gives an isomorphism $h_2 \circ \zeta^{-1}: A \overset{\sim}{\to} H_2(K)$. The inverse of this map, we can write as  $\zeta \circ h_2^{-1} \in \Hom_{\Z}(H_2(K),A)$. The UCT gives an isomorphism
\beq
q^2: H^2(K,A)\overset{\sim}{\to} \Hom_{\Z}(H_2(K),A).
\eeq
Therefore there is a unique $\iota=\iota_K \in H^2(K,A)$ with $q^2(\iota)=\zeta \circ h_2^{-1}$.
 
Let $f : B \to K$, with $B$ path-connected. From $f$ we obtain $\pi_2(f): \pi_2(B) \to \pi_2(K)$, 
its transpose 
\beq
\pi_2(f)^*:  \Hom(\pi_2(B),A) \to \Hom(\pi_2(K),A),
\eeq
and also $f^*: H^2(K,A) \to H^2(B,A)$.
From the naturality of $h^2$ and $q^2$ we note:
	\begin{equation} \label{m2.f.iota}
	h^2 (q^2(f^*(\iota))) =\pi_2(f)^*(\zeta). % \circ f_*  
	\end{equation}
	 
	 

\begin{example} \label{bba} The classifying space $BA$ is an abelian CW-group, as noted in Section \ref{BG.section}. Let $K=B(BA)$; it is a $K(A,2)$-space. We may define $\zeta=\zeta_A: \pi_2(K)  \overset{\sim}{\to} A$ as the composition
\beq
\pi_2(K)  \overset{\sim}{\to}  \pi_1(BA)  \overset{\sim}{\to}  \pi_0(A)=A
\eeq
of the connecting maps from the long exact homotopy sequences associated to the $A$-bundle $EA \to BA$ and the $BA$-bundle $E(BA) \to K$.
\end{example}

Let $\psi: A \to A'$ be a homomorphism of abelian groups. It induces 
\beq
\psi_*: H^2(B^2A,A) \to H^2(B^2A,A')
\eeq
and
\beq
(B^2\psi)^*: H^2(B^2A',A') \to H^2(B^2A,A').
\eeq

\begin{lemma} \label{roti.sabzi} We have $\psi_*(\iota_A)=(B^2 \psi)^*(\iota_{A'})$.
\end{lemma} 
 
 \begin{proof}
 From the diagram
 \beq
	\xymatrix{
		\pi_2(B^2A) \ar[r] \ar[d]^{B^2 \psi_*} &  \pi_1(BA) \ar[d]^{B^2 \psi_*} \ar[r]& A \ar[d]^{\psi_*}  \\
		\pi_2(B^2A') \ar[r] & \pi_1(BA') \ar[r]  &  A'   \\
	}
	\eeq
 we see that $ \psi_* \circ \zeta_A= \zeta_{A'} \circ B^2 \psi_*$. The lemma then follows from the commutativity of the following diagram.
  \beq
	\xymatrix{
H^2(B^2A,A) \ar[r]^{\psi_*} \ar[d]^{h^2 \circ q^2} & H^2(B^2A,A')   \ar[d]^{h^2 \circ q^2}&H^2(B^2A',A') \ar[d]^{h^2 \circ q^2} \ar[l]_{(B^2 \psi)^*}\\
	 \Hom(\pi_2(B^2A),A) \ar[r]^{\psi_*}  &  \Hom(\pi_2(B^2A),A') &  \Hom(\pi_2(B^2A'),A') \ar[l]_{(B^2 \psi)^*} \\
	}
	\eeq
 \end{proof}
 

 
 Note that $K=BBA$ is an abelian group under 
 \beq
 \mu=B^2m: K \times K \to K,
 \eeq
 where $m$ is the addition map for $A$.
 Given $f_1,f_2: B \to K$, write
 \beq
 f_1+f_2=\mu \circ (f_1 \times f_2).
 \eeq
 
Denote by $i_1, i_2: K \to K\times K$   the inclusions $i_1(x)= (x,1_K)$ and $i_2(x)= (1_K,x)$.
 Let $\pr_1$ and $\pr_2$ be the first and second projections from $K \times K$ to $K$. 
 
 \begin{lemma}  \label{Hspace}  We have
 \beq
 (f_1+f_2)^*(\iota_A)=f_1^*(\iota_A)+f_2^*(\iota_A).
 \eeq
 \end{lemma}
  
 
 \begin{proof} 
 

%Since $\mu:K \times K \to K$ is a multiplication map, we have
Of course,
$\mu \circ i_1=\mu \circ i_2 = \id_K.$ Thus   $i_1^*(\mu^*(x))= x $ and $i_2^*(\mu^*(x))= x $  for every $x \in H^2(K,A)$.
%By Lemma \ref{inclka2} below we have
Since $K$ is simply connected, we have $\mu^*(\iota_A)= \pr_1^*(\iota_A) + \pr_2^*(\iota_A)$. Therefore
 \beq
 \begin{split}
(f_1+f_2)^*(\iota_A) &= (f_1 \times f_2)^*(\mu^*(\iota_A)) \\
	&=(f_1 \times f_2)^*(\pr_1^*(\iota_A) +\pr_2^*(\iota_A))\\
	&= f_1^*(\iota_A) + f_2^*(\iota_A). \\
\end{split}
\eeq
\end{proof}
 \begin{comment}
  \begin{lemma}\label{inclka2}  
 	Let $X$ be simply connected, and
  $\alpha \in H^2(X \times X, A)$. Then
   $$\alpha = \pr_1^*(i_1^*(\alpha)) + \pr_2^*(i_2^*(\alpha)).$$
 \end{lemma}
 \begin{proof}
 By elementary algebraic topology, the map
  \beq
H^2(X,A) \times H^2(X,A) \to H^2(X \times X, A)
\eeq
	 via $(x,y) \mapsto \pr_1^*(x) + \pr_2^*(y)$  is an isomorphism.  So we may write
 	$$ \alpha = \pr_1^*(x) + \pr_2^*(y),$$
 	for some $x,y \in H^2(X,A)$. Then $i_1^*(\alpha)=x$ and $i_2^*(\alpha) = y$. 
 \end{proof}	 
 \end{comment}

 \section{Connecting Homomorphisms} \label{Connecting Homomorphisms} 
In this section, we make a compatibility check between connecting homomorphisms arising from the various classifying spaces associated to an extension. We show a certain square \emph{anticommutes}, by an argument with loops.
 
 \subsection{Conventions on Loops}
 
 Let $(B,b)$ be a pointed space.  A path in $B$ is a map $\gm:[0,1] \to B$. The reverse of   $\gm$ is the map $\ol \gm: [0,1] \to B$ defined by $\ol \gm(t)=\gm(1-t)$. Write $c$ for the constant path $c(t)=b$. 
Write $\gm * \gm'$ for the usual concatenation of paths. Given a loop $\gm$ in $B$ with $\gm(0)=\gm(1)=b$, and $s \in (0,1]$, define a loop $\gm^{\leq s}$ in $B$ by the formula
  \beq
  \gm^{\leq s}(t)=\begin{cases} 
			\gm(\frac{t}{s}), &   t \in [0,s]\\
			b, & t \in [s,1]. \\
		\end{cases}
		\eeq
 Similarly, for $s \in [0,1)$, put
 \beq
		 \gm^{\geq s}(t) = \begin{cases} 
			b, &   t \in [0,s]\\
			\gm(\frac{t-s}{1-s}), & t \in [s,1]. \\
		\end{cases}
		\eeq
  %Of course, all $\gm^{\leq s}$ and $ \gm^{\geq s}$ are homotopic to $\gm$. 
  
  
Let $B'$ be another space. If $\gm$ is a path in $B$ and $\gm'$ is a path in $B'$, write
\beq
\gm \times \gm': [0,1] \to B \times B'
\eeq
for the product path $t \mapsto (\gm(t),\gm'(t))$.
 
 
 
 
 \subsection{An Anticommutative Square}
 
In this section let $A$ be a discrete group, $G$ a connected CW-group, and  $\ms E=(E,i,\rho) \in {\bf E}(G,A)$. Since $\rho$ is a fibration, we have a connecting homomorphism
\beq
\pi_1(G) \overset{\partial_{\ms E}}{\to} \pi_0(A)=A.
\eeq
We also have  connecting maps
\beq
\pi_1(BA)  \overset{\partial_1}{\to} \pi_0(A) \text{ and } \pi_2(BG)  \overset{\partial_2}{\to} \pi_1(G) 
\eeq
from \eqref{bottle}. (Here $\partial_1=\partial_1^A$ and  $\partial_2=\partial_2^G$.) 
Applying the functor
\beq
\ms B: {\bf E}(G,A) \to \Bun_{BG}(BA)
\eeq
gives a $BA$-bundle $BE \to BG$, and in particular another connecting homomorphism
\beq
\pi_2(BG) \overset{\partial_{BE}}{\to} \pi_1(BA)=A
\eeq


\begin{prop}\label{anticommute}
	
	The square  
	\beq
	\xymatrix{
		\pi_2(BG) \ar[r]^{\partial_2} \ar[d]_{\partial_{BE}} &  \pi_1(G) \ar[d]^{\partial_{\ms E}}  \\
		\pi_1(BA) \ar[r]^{\partial_1}  & A   \\
	}
	\eeq
	anticommutes, meaning that $\partial_1 \circ \partial_{BE}=- \partial_E \circ \partial_2$.
\end{prop}

 \begin{proof}  The $H$-bundles  $EH \to BH$ for $H=A,G,E$, the $BA$-bundle $BE \to BG$ and the $A$-bundle $\rho: E \to G$ combine with the composition $EA \to EE \to EG$  to form a diagram of pointed spaces:

\beq
	\xymatrix{
		(A,1_A) \ar[r] \ar[d]^{i}&  (EA, 1_{EA}) \ar[r]^{p_A} \ar[d]^{E i}&  (BA,1_{BA}) \ar[d]^{Bi} \\
		(E, 1_A) \ar[r] \ar[d]^{\rho}&  (EE, 1_{EE}) \ar[r]^{p_{E}} \ar[d]^{E \rho}&  (BE, e_{BE}) \ar[d]^{B \rho} \\
		(G, 1_G) \ar[r] &  (EG, 1_{EG}) \ar[r]^{p_G} & (BG, e_{BG}) \\
	},
	\eeq
 

%Here $(X, e_X)$ is a pointed space $X$ with a point $e_X \in X$ and the morphisms are of pointed spaces.

Since $EE \to BE$ is a   $E$-bundle, the action of $A$ on $EE$ is neat. Let $\sigma_{BG}: I^2 \to BG$ represent a member of $\pi_2(BG)$, and let $\sigma_{BE}: I^2 \to BE$ be a lift of $\sigma_{BG}$. Let $\gm_{BA}:S^1 \to BA$ be the restriction of $\sigma_{BE}$ to the boundary.
	Then
	\beq
	\partial_{BE} ([\sigma_{BG}])=[\gm_{BA}].
	\eeq
	
	
	Similarly lift $\sigma_{BG}$ to $\sigma_{EG}: I^2 \to EG$, and let $\gm_{G}:S^1 \to G$ be the restriction to the boundary. So
	\beq
	\partial_2([\sigma_{BG}])=[\gm_G].
	\eeq
  
		 
Lift $\gm_{BA}$ to a path $\gm_{EA}$ in $EA$ starting at $1_{EA}$, and $\gm_G$ and $\ol{\gm_G}$ to paths    $\gm_E$ and $\delta_E$ in $E$ starting at $1_E$. Note that $\delta_E(1)=-\gm_E(1)$ by Proposition \ref{groth.}.

 We consider $E$ and $EA$ as subsets of $EE$, and all these paths start at $1_{EE} \in EE$. 
 
\begin{lemma}  We have $\gm_{EA}(1)= \delta_E(1)$.
	\end{lemma}
\begin{proof}
 It is enough to prove that the path $\la= \ol{ \delta_E} * \gm_{EA}$ in $EE$ is a loop, since its endpoints are $\la(0)=  \delta_E(1)$ and $\la(1)=\gm_{EA}(1)$. Define	$B'=EG \times_{BG} BE$, we have a map
\beq
q=E \rho \times p_E: EE \to B'
	\eeq
It is easy to see that $q$ is surjective, by using the fact that $EE \to EG$ is an $EA$-bundle.	
Moreover the fibres of $q$ are the $A$-orbits of $EE$. It follows that $q$ is an $A$-cover.
	
	Now we have 
	\beq
	q_*(\lambda) =(\gm_{G} * c) \times (c * \gm_{BA}).
	\eeq
	For each $s \in [\half,1]$, the formula
		\beq
		H_s=\gm_G^{\leq s}\: \:  \times \: \:   \gm_{BE}^{\geq 1-s}
		\eeq
	defines a homotopy in $B'$ between $H_\half=q_*(\la)$ and $H_1=\gm_{G} \times \gm_{BA}$. Next, for $s \in I$, the function $K_s:\partial I^2  \to B'$ defined by 
	\beq	
	K_s(x)=(\sigma_{EG}(sx), \sigma_{BE}(sx))
	\eeq
	%has image in $B'$ since
	%\beq
		% (p_G \times B \rho)(H(x,t))= (\sigma_{BG}(tx), \sigma_{BG}(tx)).
		 %\eeq
		gives a homotopy in $B'$ between $K_0=c$ and $K_1=\gm_G \times \gm_{BA}$.		
	Therefore $q_*(\lambda)$ is nullhomotopic in $B'$. Since $q$ is a cover, $\la$ is a loop by \cite[Corollary 3.6, page 141]{Bredon}. \end{proof}
	
	 The proposition now follows since
	 \beq
	 \begin{split}
	 \partial_1 \partial_{BE}([\sigma_{BG}]) &= \partial_1 ([\gm_G]) \\
	 &= \gm_{EA}(1) \\
	 &=\delta_E(1)\\
	 &= -\gm_E(1)   \\
	 &= -\partial_E ([\gm_G]) \\
	 &= - \partial_E \partial_2 ([\sigma_{BG}]). \\
	\end{split}
	\eeq
	 
	 \end{proof}

 
 
 
 

  

 
\section{Disconnected Groups} \label{disc.section}

In this section we give a cohomology version of the exact sequence \eqref{left.ex.here}.  A more sophisticated approach would be using \cite[Theorem 5.9, page 147]{McCleary}.


\begin{comment}
Let $G$ be a CW-group, and put $\pi_0=\pi_0(G)$. The inclusion map $\iota$ and quotient map $\qoppa$ give an exact sequence
\beq
1 \to G^\circ \overset{\iota}{\to} G \overset{\qoppa}{\to} \pi_0 \to 1.
\eeq

\subsection{Homotopy}


 Let us apply Proposition \ref{normal.fibre}   to the normal subgroup $G^\circ$ of $G$: From Theorem \ref{homotopy.fibration} (and Lemma \ref{bu.he}), we obtain a long exact sequence
\beq
\cdots \to \pi_n(B_GG^\circ)  \to  \pi_n(B_uG) \to  \pi_n(B\pi_0)  \to  \pi_{n-1}(BG^\circ)  \to  \cdots \to \pi_1(B\pi_0) \to 1.
\eeq
 
\begin{lemma} \label{isom.pi.bs} The map $\qoppa$ induces an isomorphism $B\qoppa_*: \pi_1(BG) \overset{\sim}{\to} \pi_1(B\pi_0)$.
 The map $\iota$ induces isomorphisms $B\iota_*: \pi_k(BG^\circ) \overset{\sim}{\to} \pi_k(BG)$ for $k \geq 2$. 
\end{lemma}

\begin{proof} From Proposition \ref{stapler} we have $\pi_1(BG^\circ)=0$, moreover since $BG^\circ$ is connected we have $\pi_0(BG^\circ)=0$. Now use the above sequence.
\end{proof}
\end{comment}



\subsection{Cohomology}
 
 
 
\begin{prop} Let $G$ be a CW-group. The sequence
\beq \label{H_2exact} 
	 H_2(BG^\circ) \xrightarrow{B\iota_*} H_2(BG) \xrightarrow{B\qoppa_*} H_2(B\pi_0) \to 0
	 \eeq 
	obtained from \eqref{qoppa} is exact.
\end{prop}


\begin{proof}
Since $\qoppa \circ \iota$ is trivial, the induced map from $BG^\circ$ to $B \pi_0$ is constant, hence induces the zero map on cohomology. Next we show that the image of $B\iota_*$ is identical to the image of the Hurewicz map $h: \pi_2(BG) \to H_2(BG)$.
%We have $i: BG^\circ \to BG$ is an injection and we apply the long exact sequences. 
%Note that $\pi_k(BG^\circ) \to \pi_k(BG)$ is an isomorphism for $k \geq 2$. 
Since $BG^\circ$ is simply connected,  we may apply the Hurewicz theorem \cite[Corollary 10.8, page 478]{Bredon} to deduce that $h^\circ: \pi_2(BG^\circ) \to H_2(BG^\circ)$ is an isomorphism. By the commutative diagram
\beq 
\xymatrix{ \pi_2(BG^\circ) \ar[r]^{B\iota_*}_{\sim} \ar[d]^{h^\circ}_{\sim} & \pi_2(BG) \ar[d]^{h}\\
H_2(BG^\circ) \ar[r]^{B\iota_*} & H_2(BG)\\
},
\eeq 
 we deduce that $B\iota_*(H_2(BG^\circ)) = h(\pi_2(BG))$. By Corollary \ref{isom.pi.bs} the induced map 
$\pi_1(BG) \to \pi_1(B \pi_0)$ is an isomorphism. Moreover, $\pi_k(B \pi_0)=0$ for $k \geq 2$. The proposition is then a consequence of the following lemma. \end{proof}

\begin{lemma} Let $f:X \to Y$ be a continuous map between path-connected  spaces. Suppose that $f$ induces an isomorphism of fundamental groups, and that
$\pi_2(Y)=\pi_3(Y)=0$.
Then $f_*: H_2(X) \to H_2(Y)$ is surjective, and its kernel is  the image of the Hurewicz map $h: \pi_2(X) \to H_2(X)$.
\end{lemma}

\begin{proof} Let $M_f$ be the mapping cylinder associated to $f$. Then $f$ factors as
\beq
X \hookrightarrow M_f \to Y,
\eeq
and $M_f \to Y$ is a homotopy equivalence.  Note that the relative homotopy sets $\pi_1(M_f,X)$ and $\pi_2(M_f,X)$ are trivial.
By hypothesis, the induced map $H_1(X) \to H_1(Y)$ is an isomorphism, hence so too is $H_1(X) \to H_1(M_f)$. It follows from the LES of the pair $(M_f,X)$ that $H_1(M_f,X)=0$. 

Therefore by the Relative Hurewicz Theorem \cite[Theorem 10.7, page 478]{Bredon} we deduce that $H_2(M_f,X)=0$, and that the Hurewicz map is a surjection
\beq
  \pi_3(M_f,X) \twoheadrightarrow H_3(M_f,X).
\eeq
Furthermore, since $\pi_2(M_f)=\pi_3(M_f)=0$, the connecting map $\pi_3(M_f,X) \to \pi_2(X)$ is an isomorphism. The conclusion then follows from the diagram:
\beq
 \xymatrix{ 0 \ar[r] &
 \pi_3(M_f,X) \ar[d] \ar[r]^{\sim}& \pi_2(X) \ar[r] \ar[d] & 0 \\
H_3(M_f) \ar[r] & H_3(M_f,X)	\ar[r] & H_2(X) \ar[r] & H_2(M_f) \ar[r] & 0 \\
	}
\eeq 
 \end{proof}

 
\begin{thm}\label{cohomexact} For an abelian group $A$, there is an exact sequence
\beq
0 \to H^2(B \pi_0,A) \overset{B \qoppa^*}{\to} H^2(BG,A) \overset{B \iota^*}{\to} H^2(BG^\circ,A)^{\pi_0}.
\eeq
\end{thm}

\begin{proof}
By the naturality of the universal coefficient theorem (UCT) we have the following commutative diagram, with exact columns:

\beq \xymatrix{
& 0 \ar[d] & 0 \ar[d] &  \\
 & \Ext^1_{\Z}(H_1(B\pi_0),A) \ar[r]^{\sim} \ar[d]  & \Ext^1_{\Z}(H_1(BG),A) \ar[r]\ar[d]    & 0 \ar[d]   \\
& H^2(B \pi_0,A) \ar[r]^{B\qoppa^*} \ar[d]&  H^2(BG,A) \ar[r]^{B\iota^*}  \ar[d] & H^2(BG^\circ,A)  \ar[d]   \\
0  \ar[r] &   \Hom_{\Z}(H_2(B \pi_0),A)\ar[d]  \ar[r] & \Hom_{\Z}(H_2(BG),A) \ar[r]  \ar[d] & \Hom_{\Z}(H_2(BG^\circ),A) \ar[d] \\
&   0 & 0 & 0 \\
}
\eeq
 

The map between the two $\Ext$-groups is an isomorphism by Lemma \ref{isom.pi.bs}.
The $\Hom$-row is left exact by dualizing \eqref{H_2exact}. Therefore $B\qoppa^*$ is injective by the Five Lemma.
Note that $$B\iota^*(H^2(BG,A)) \subseteq H^2(BG^\circ, A)^{\pi_0}$$ by \cite[Lemma 3.1 Page 55]{Milgram}. 
It is clear that $B\iota^* \circ B\qoppa^* = 0$, and exactness at $H^2(BG,A)$ follows from a diagram chase. 
\end{proof} 
   

 \section{From $BA$-Bundles to $H^2$} \label{kappa.section}
     Let $B$ be a topological space. A $BA$-bundle $(X,p)$ over $B$  corresponds to a map $f_p: B \to BBA$, unique up to homotopy, with $(X,p)$ the pullback of $EBA \to BBA$ under $f_p$. Recall from Section \ref{ka2} that $B(BA)$ is a $K(A,2)$, and  a ``fundamental class'' $\iota \in H^2(BBA,A)$ was defined. 
Then we may set $\kappa(p)=f_{p}^*(\iota)$. This construction defines a map
\beq
\kappa=\kappa_{A,B}: \bun_B(BA) \to H^2(B,A).
\eeq

 
 
 
 
 \begin{prop}\label{alphabunnatural} The map $\kappa_{A,B}$ is additive, bijective, and natural in both $A$ and $B$.
 \end{prop}
  \begin{proof}
 
  Since $\kappa_{A,B}$ is the composition of the bijection $\bun_{B}(BA)  \overset{\sim}{\to} [B,BBA]$ of \eqref{univ.class.propz} with the bijection 
$  [B,BBA]  \overset{\sim}{\to} H^2(B,A)$ of  \cite[Theorem 4.57]{hatcher}, it is itself a bijection.
  \begin{comment}
 Bijectivity follows since  
 [$\bigstar$ Proof of bijectivity : by identifying $Bun_B(BA) \overset{\sim}{\to} [B,BBA]$ by the universal property of $BBA$ treating $BA$ as a group, and next-ly identifying $[B,BBA= K(A,2)] \overset{\sim}{\to} H^2(B,A)$ using Theorem 4.57 on page 393 of Hatcher.] 
  \end{comment}
For additivity, let $(X_1,p_1), (X_2,p_2) \in \Bun_B(BA)$, and write $(X_3, p_3)$ for their Baer sum. Let $f_k: B \to BBA$ be classifying maps for $(X_k,p_k)$, with $k=1,2$. Write $\tilde f_k: X_k \to EBA$ for the overmaps. Since $EBA$ is an abelian group, we may define a map
$\tilde f_3: X_1 \times_{B} X_2 \to EBA$ by 
 \beq
 \tilde f_3(x_1,x_2)=\tilde f_1(x_1)+\tilde f_2(x_2).
 \eeq
 Since $\tilde f_3$ is constant on (antidiagonal) $BA$-orbits, it descends to a morphism
 \beq
 \tilde f_3: X_3 \to EBA.
 \eeq
 
 Being $BA$-equivariant, $\tilde f_3$ descends to a map $f_3:B \to BBA$. 
 Now $f_3=f_1+f_2$, and  $f_3$ is a classifying map for $X_3$.
 Thus
 \beq
 \begin{split}
 \kappa(p_3) &= f_3^*(\iota) \\ 
 &=f_1^*(\iota)+f_2^*(\iota) \text{ (by Lemma \ref{Hspace})} \\
 &= \kappa(p_1) + \kappa(p_2) ,
  \end{split}
  \eeq
  as required. This shows additivity. It is easy to see that $\kappa$ is natural in $B$. 
  \begin{comment}
   Let $\varphi: B' \to B$ be a map, and $(X,p) \in  \Bun_B(BA)$. We have  a pullback diagram
 \beq
\xymatrix{  
X' \ar[r] \ar[d] & X \ar[r] \ar[d] & EBA \ar[d]\\
B' \ar[r]^{\varphi} & B \ar[r]^{f_{p} \: \: \: } & K(A, 2),\\
}
\eeq 
and  so we can take
\beq
f_{p'}=f_{p} \circ  \varphi.
\eeq
From this we deduce that $\kappa(p')=\varphi^*(\kappa(p))$, in other words $\kappa$ is natural in $B$.\end{comment}
 For naturality in $A$, one uses Lemma \ref{roti.sabzi}; we omit the details.
  
  \begin{comment}
  Now we show naturality in $A$.  Let $\psi: A \to A'$ be a homomorphism, and $(X,p) \in  \Bun_B(BA)$. 
  Let $f_p: B \to BBA$ be a classifying map, with overmap $\tilde f_p: X \to EBA$. Let $(X',p')$ be the $BA'$-bundle induced from $(X,p)$ by $B \psi$. (Thus $X'=X[BA']$.)
   
The map $\phi: X \times BA' \to EBA'$ given by
 \beq
 \phi(x,a')= (EB \psi \circ \tilde f_p)(x) \cdot a'
 \eeq
  descends to a map
 \beq
X' \to EBA'
 \eeq
making the following diagram commute:
  \beq
	\xymatrix{
		X' \ar[rr]^{\phi} \ar[d] &&EBA' \ar[d]\\
		B \ar[rr]^{B^2 \psi \circ   f_{p} }   && BBA'  \\
	}
	\eeq 
 Therefore 
 \begin{equation} \label{palakhi}
 f_{p'}=B^2 \psi \circ f_{p},
 \end{equation}
 which gives
 \beq
 \begin{split}
 \kappa_{A'}(\psi_*(p)) &= f_{p'}^*(\iota_{A'}) \\
 					&= f_{p}^*((B^2 \psi)^*(\iota_{A'})) \\
					&= f_{p}^*(\psi_*(\iota_A)) \text{ (by Proposition \ref{roti.sabzi}) }\\
					&=\psi_*(f_{p}^*(\iota_A)) \\
					&= \psi_*(\kappa_A(p)), \\
 \end{split}
 \eeq
 as desired.
  \end{comment}
 \end{proof}
  \section{From Extensions to $H^2$} \label{alpha.section}
  Finally we reach our main objective: the definition of the correspondence $\alpha$, its agreement with $-\beta$, and Theorem \ref{mid.intro}.
  \subsection{Definition of $\alpha_G$}
  Let   $G$ be a CW-group. Write $\alpha=\alpha_{G,A}$ for the 
  composition
  \beq
  \mb E(G,A) \overset{\ms B}{\to} \bun_{BG}(BA) \overset{\kappa}{\to} H^2(BG,A).
  \eeq
  
  By Propositions  \ref{B.map.inj} and \ref{alphabunnatural},  and the results of Section \ref{rain.car}, we deduce:
  \begin{prop}  \label{G.disc.alpha.inj} 
  The map $\alpha_{G,A}$ is a homomorphism of abelian groups, natural in $G$ and $A$. When $G$ is discrete, it is injective.
  \end{prop}
   
  \begin{comment}
  \begin{prop} \label{G.disc.alpha.inj} The map $\alpha_{G,A}$ is injective when $G$ is discrete.
  \end{prop}
  
  \begin{proof} This follows from Propositions  \ref{B.map.inj} and \ref{alphabunnatural}.
  \end{proof}
  \end{comment}
  
  \begin{remark}  \label{our.remark} The singular homology $H^*(BG,A)$ can be identified with the group cohomology $H^*(G,A)$ by
\cite[Remark 3.6, Page 58]{Milgram}. According to  \cite[Lemma 1.12, page 120]{Milgram}, the map $\alpha_{G,A}$ agrees with the usual bijection between $\mb E(G,A)$ and the group cohomology $H^2(G,A)$, when $A=\Z/p\Z$ for $p$ prime.
 \end{remark}
 
 Recall from Definition  \ref{beta.defn} the isomorphism $\beta_G:  \mb E(G,A) \overset{\sim}{\to} H^2(BG,A)$, valid for connected CW-groups $G$.

\begin{prop} \label{alpha.beta}
	When $G$ is connected, $\alpha_G=-\beta_G$.
\end{prop}
\begin{proof}
	
	Let $\ms E=(E,i,p) \in {\bf E}(G,A)$.
	With the above construction we have 
	\beq
	\alpha_G(\ms E)= f_{\ms E}^*(\iota) \in H^2(BG,A),
	\eeq
	where $\iota \in H^2(BBA,A)$ was introduced in Section \ref{ka2}.
Moreover the extension gives a map which we call $\pi_2(f_{\ms E}):\pi_2(BG) \to   \pi_2(K)= A$. 
\begin{comment}
We have a morphism of $BA$-bundles:

\begin{equation}
	\xymatrix{
BA \ar[r]^{=} \ar[d]& BA \ar[d]\\
BE \ar[r]^{\hat{f}_{\ms E}} \ar[d]	& E(BA)) \ar[d]\\
BG \ar[r]^{f_{\ms E}} & B(BA)\\
}
\end{equation}
\end{comment}
From the long exact sequences of homotopy groups we obtain:
\begin{equation} \label{ashish} 
      \xymatrix{
  \pi_2(BG) \ar[r]^{ \partial_{BE}} \ar[d]^{\pi_2(f_{\ms E})}& \pi_1(BA)  \ar[d]^{||} \\
 \pi_2(BBA) \ar[r]^{\partial_{BK}} & \pi_1(BA) \\
}
\end{equation}
 
Recall from \eqref{characterized} that $\beta_G$ is characterized by $m^2(\beta_G(\ms E))=\partial_{\ms E}$. Therefore we must prove that
\beq
h^2 q^2(f_{\ms E}^*(\iota_A))=-\partial_{2}^*( \partial_E).
\eeq

 
We have	
\beq
\begin{split}
h^2 q^2(f_{\ms E}^*(\iota_A)) &= \pi_2(f_{\ms E})^*(\zeta) \text{ by \eqref{m2.f.iota}} \\
					&=\zeta \circ \pi_2(f_{\ms E}) \\
			&= \partial_1 \circ \partial_{BK} \circ \pi_2(f_{\ms E})  \\
			&= \partial_1 \circ \partial_{BE} \text{ by \eqref{ashish} } \\
			&=  - \partial_E \circ \partial_2 \text{ (by Proposition \ref{anticommute})} \\
			&= -\partial_{2}^*( \partial_E), \\
\end{split}
\eeq
as required. \end{proof}
 
   \subsection{Injectivity when $G$ is a semidirect product}

We continue to assume that $G$ is a CW-group.
   
\begin{thm} \label{big.theorem}
The map $\alpha_G$ is injective. When $A=\Z/p\Z$ and $G$ is the semidirect product of a discrete group
and a connected group, then $\alpha_G$ is an isomorphism.
%When $G$ is a semidirect product of its connected component and a discrete subgroup, then $\alpha_G$ is an isomorphism.
	%Let $G = G^\circ \rtimes \pi_0(G)$. Then we have a natural isomorphism $H^2(BG,A) \cong \mb E(G, A)$
\end{thm}
\begin{proof}
 
  Put $\pi_0=\pi_0(G)$, and consider the diagram
\beq
\xymatrix{
0 \ar[r] & \mb E(\pi_0,A)\ar[d]^{\alpha_{\pi_0}} \ar[r]^{\qoppa^*} & \mb E(G,A) \ar[d]^{\alpha_G} \ar[r]^{\iota^*} & \mb E(G^\circ, A)^{\pi_0} \ar[d]^{\alpha_{G^\circ}}  \\
0 \ar[r] & H^2(B\pi_0,A)\ar[r]^{B\qoppa^*} & H^2(BG,A) \ar[r]^{B\iota^*} & H^2(BG^\circ, A)^{\pi_0}     }
\eeq	
 
 The top row is exact by Proposition \ref{extexact},  and the bottom row is exact by Theorem \ref{cohomexact}. The diagram is commutative by naturality of $\alpha$.
 The map $\alpha_{\pi_0}$ is injective by Proposition \ref{G.disc.alpha.inj}. The map $\alpha_{G^\circ}$ is a bijection by Theorem \ref{Gconnect.equiv} and Proposition \ref{alpha.beta}. The injectivity of $\alpha_G$ then follows by a diagram chase.  (See for example \cite[Exercise 1.3.3]{weibel}.)

Next, suppose  $G = G^\circ \rtimes \pi_0$. Then $\iota^*$ is surjective by Proposition  \ref{surj}, and $\alpha_{\pi_0}$ is a bijection by Remark \ref{our.remark}. From another diagram chase we deduce that $\alpha_G$ is an isomorphism. 
\end{proof}

 

  \subsection{Lifting and Cohomology}	
 
  
  
\begin{thm} \label{lifting.and.coh} Let $\varphi:G' \to G$ be a homomorphism between   CW-groups, and $\ms E=(E,i,p) \in {\bf E}(G,A)$. Then $\varphi$ lifts to $E$ iff 
\beq
\varphi^*(\alpha_G(\ms E))=0.
\eeq
\end{thm}
	
	
\begin{proof} 
From Proposition \ref{ext.lift}, we know $\varphi$ lifts to $E$ iff $\varphi^*(\ms E)=0$. But as $\alpha_{G'}$ is injective (Theorem \ref{big.theorem}), this is equivalent to the identity 
\beq
0= \alpha_{G'}(\varphi^* \ms E) =\varphi^* (\alpha_G(\ms E)).
\eeq \end{proof}

%Note from Section \ref{rel.lift.prob} and Proposition \ref{m^1}, there is a simply transitive action of $\Hom_c(G,A)$ on the set of lifts of $\varphi$.
 
 \section{Stiefel-Whitney Classes} \label{SWC.section} 
 
 In this section we define Stiefel-Whitney classes for orthogonal complex representations. This is equivalent to the definition in \cite[Section 2.6]{Benson.II} given for real representations.
 % [Refer to Either Toda or Peter May]
 
 
 Let $V$ be a complex vector space of dimension $n$, with a nondegenerate quadratic form $Q$. Write $\Or(V)$ for the corresponding orthogonal group. 
 Let $\mc B=\{e_1, \ldots, e_n\}$ be an orthonormal basis of $V$. Then
 \beq
\Gamma=\{ g \in \Or(V) \mid \forall i, ge_i= \pm  e_i \}
 \eeq
is evidently an $\mb F_2$-vector space with an obvious basis.  Write $v_1, \ldots, v_n$ for the dual basis, viewed in  $H^1(B\Gamma,\Z/2\Z)$. Then
 \beq
 H^*(B\Gamma,\Z/2\Z) =\Z/2\Z[v_1, \ldots, v_n],
 \eeq
 meaning the mod $2$ cohomology of $B \Gamma$ is a polynomial algebra in   the $v_i$. The restriction map
 \beq
 H^*(\BO(V), \Z/2\Z) \to  H^*(B\Gamma, \Z/2\Z)
 \eeq
is injective, and its image is the symmetric algebra   $\Z/2\Z[v_1, \ldots, v_n]^{S_n}$. Write $\mc E_1, \ldots, \mc E_n$ for the elementary symmetric polynomials in the $v_i$; by the Fundamental Theorem of Symmetric Polynomials, we can therefore identify
\beq
 H^*(\BO(V), \Z/2\Z) \cong \Z/2\Z[\mc E_1, \ldots, \mc E_n].
 \eeq
 For $1 \leq k \leq n$, write $w_k \in  H^k(\BO(V), \Z/2\Z)$ for the cohomology class corresponding to $\mc E_k$.  
 
 Now let $G$ be a CW-group, and $\pi: G \to \Or(V)$ an orthogonal (complex) representation. We define
 \beq
 w_k(\pi)=\pi^*(w_k) \in H^k(BG,\Z/2\Z);
 \eeq
 this is called the $k$th Stiefel-Whitney Class (SWC) of $\pi$.
 
Note that if  $\varphi: G_1 \to G_2$ is a homomorphism of CW-groups, and $\pi$ is an orthogonal representation of $G_2$, then  
  \begin{equation} \label{funct.w}
  \varphi^*(w(\pi))=w(\pi \circ \varphi).
  \end{equation}
 Recall the definition of $m^1$ from Proposition \ref{m^1}.
 
  \begin{thm} Let $\pi: G \to \Or(V)$ be an orthogonal representation of a CW-group $G$. Then the isomorphism $m^1: H^1(BG,\Z/2\Z) \to \Hom_c(G, \mu_2)$ takes $w_1(\pi)$ to $\det  \pi$. \end{thm}

\begin{proof}
Since $H^1(\BO(V), \Z/2\Z)=\{0,w_1\}$, the isomorphism 
\beq
m^1:  H^1(BG,\Z/2\Z) \to \Hom_c(\Or(n), \mu_2)
\eeq
 must take $w_1$ to $\det$.
 Now by the commutative square
   \beq
\xymatrix{ 
		 H^1(\BO(V),\Z/2\Z) \ar[r]^{m^1} \ar[d]^{\pi^*}&  \Hom_c(\Or(V),\Z/2\Z) \ar[d] \\
		 H^1(BG,\Z/2\Z) \ar[r]^{m^1} & \Hom_c(G,\Z/2\Z) \\
	}
\eeq
we find that
 \beq
 \begin{split}
 m^1(w_1(\pi)) &= m^1(\pi^* w_1) \\
 		&= \det \pi, \\
		\end{split}
		\eeq
		as required.
\end{proof}


 
   \section{Extensions of $\Or(V)$} \label{double.covs.ort.sec}
 
 Assume now that $\dim V \geq 2$. Since 
 \beq
 H^2(\BO(V),\Z/2\Z) =\{0,w_1^2, w_2, w_1^2+w_2 \},
 \eeq
 there are four inequivalent central extensions of $\Or(V)$ by $A=\Z/2\Z$. Let us undertake to describe these extensions.
 %To begin, here is a simple way to define an $A$-cover of $\Or(V)$. 
 
 The first one is easy to describe. Taking determinants gives a surjection from $\Or(V)$ to $\mu_2$. Of course, $\mu_2$ has an $A$-cover, by squaring the cyclic group $\mu_4 < \mb C^\times$ generated by the imaginary unit $i$. 
  Write $\widetilde \Or(V) \to \Or(V)$ for the pullback:
   \beq 
	\xymatrix{
		\widetilde \Or(V)   \ar[d]   \ar[r]  &  \mu_4  \ar[d]^{z \mapsto z^2}\\
		 \Or(V)  \ar[r]^{\det} & \mu_2}
	\eeq 
 Thus $\widetilde \Or(V)$ is the subgroup of  pairs $(g,z) \in \Or(V) \times \mu_4$ with $\det g=z^2$.
 
 
  \subsection{Pin Groups}
 
Let $V$ be a finite-dimensional (complex) vector space, with a nondegenerate quadratic form $Q$.  
Say $u \in V$ is a \emph{unit vector} provided $Q(u)=1$, and an \emph{antiunit vector}, provided $Q(u)=-1$. 
Note that if $u$ is a unit vector, then $iu$ is an antiunit vector.
 

The Clifford algebra $C(V)$ is the quotient of the tensor algebra $T(V)$ by the two-sided ideal generated by the set
\beq
 \{ v\otimes v -Q(v): v\in V\}.
 \eeq
 It contains $V$ as a subspace.  Write $C(V)^\times$ for the group of invertible members of $C(V)$. 
 
 \begin{defn} Write $\Pin^-(V)$ for the subgroup of $C(V)^\times$ generated by the antiunit vectors in $V \subset C(V)$, and
 $\Pin^+(V)$ for the subgroup generated by the unit vectors.
 \end{defn} 
  
 There are unique homomorphisms  $\rho^-: \Pin^-(V) \to \Or(V)$ and $\rho^+:  \Pin^+(V) \to \Or(V)$. Here $\rho^-$ (resp., $\rho^+$) sends an antiunit (resp., unit) vector $u$ to the reflection of $V$ determined by $u$. %, and $\tilde \rho$  
 The kernels of $\rho^+$ and $\rho^-$ are both equal to $\mu_2$.
 Since $\Or(V)$ is generated by reflections, $\rho^+$ and $\rho^-$ are surjective. Thus they are extensions of $\Or(V)$ by $\mu_2$.    See \cite[Chapter 20]{fulton} and \cite[Appendix 1]{frohlich} for details.   
 
 
 
 \subsection{Restriction to $\Gamma_2$}
 Now suppose $\dim V \geq 2$. Pick perpendicular unit vectors $e_1,e_2 \in V$, with corresponding reflections $r_1,r_2$.  Let $\Gamma_2< \Or(V)$ be the Klein $4$-subgroup generated by $r_1,r_2$. 
 We have:
    \beq 
	\xymatrix{
		\mb E(\Or(V),\Z/2\Z)   \ar[d]^{\alpha_G}   \ar[r]  &  \mb E(\Gamma_2,\Z/2\Z) \ar[d]^{\alpha_{\Gamma_2}}\\
		H^2(\BO(V),\Z/2\Z)  \ar[r] & H^2(B\Gamma_2,\Z/2\Z)       }
	\eeq
 
Since $\Gamma_2$ is discrete, the bijection $\alpha_{\Gamma_2}$ is well-understood; see for instance  \cite[page 830]{Dummit}. 
For each of the four extensions $\ms E$ above, one determines the restriction $\ms E_2$ of $\ms E$ to $\Gamma_2$, and computes
$\alpha_{\Gamma_2}(\ms E_2)$ explicitly with $2$-cocyles. Here is the result:

\begin{comment}
Let us give details for one example, the others being similar. 

The preimage of $\Gamma_2$ under $ \rho^-$ is 
\beq
 H^-=\{ \pm 1, \pm e_1, \pm e_2, \pm e_1e_2 \},
\eeq
a subgroup of $\Or(V)$ isomorphic to the quaternion group $Q$. A set-theoretic section $s: \Gamma_2 \to H^-$ is given by
\beq
s(0)=1, \: s(r_1)=e_1, \: s(r_2)=e_2, \: \text{ and } s(r_1r_2)=e_2e_1.
\eeq
To this corresponds the cocycle $f: \Gamma_2 \to \Z/2\Z$  with
\beq
f(r_1, r_1)= f(r_2,r_2)= f(r_1r_2, r_1r_2)= f(r_1,r_2)=f(r_1r_2,r_1)= f(r_2,r_1r_2)=1,
\eeq
and which vanishes at  other pairs. This is exactly the cocycle with cohomology class 
\beq
\mc E_2 + \mc E_1^2 \in H^2(B\Gamma_2,\Z/2\Z),
\eeq
whence
\beq
\alpha_{\Or(V)}(\Pin^-(V),i,\rho^-)=w_2+w_1^2.
\eeq
The preimage of $\Gamma_2$ under $\rho^+$ is
\beq
H^+=\{ \pm 1, \pm ie_1, \pm ie_2, \pm e_1e_2 \},
\eeq
a dihedral subgroup. Similar work to the previous case gives the cohomology class $\mc E_2$.

Meanwhile the pullback of $\Gamma_2$ under $\widetilde \Or(V) \to \Or(V)$ is the fibre product of $\Gamma_2$ and $\mu_4$ over $\mu_2$.
 This fibre product is isomorphic to the abelian group $\mu_4 \times \mu_2$, and the associated cohomology class is $\mc E_1^2$.
 Of course, the preimage of $\Gamma_2$ under the trivial extension $\Or(V) \times \mu_2 \to \Or(V)$ is an elementary abelian $2$-group, and corresponds to the trivial cohomology class.

 
 
 
\begin{prop} Suppose $\dim V \geq 2$. The set  $\mb E(\Or(V),\Z/2\Z)$  consists of  the four mutually inequivalent extensions
\beq
  \Pin^+(V),\:  \Pin^-(V),\:  \widetilde \Or(V), \text{  and  }\Or(V) \times \mu_2.
\eeq
 The correspondence with $H^2(\BO(V),\Z/2\Z)$ is given in the table below.
\end{prop} 
 
\begin{center}
\begin{tabular}{|c|c|c|}
	\hline
	Extension & Cohomology class &  preimage of $\Gamma_2$\\
	\hline \hline
	$\Pin^+(V)$   & $w_2$ & dihedral\\
	\hline 
	$\Pin^-(V)$  & $w_2 + w_1^2$ & quaternion\\
	\hline
	$\widetilde{\Or}(V)$   & $w_1^2$ & $C_4 \times C_2$ \\
	\hline 
	$\Or(V) \times \mu_2$  & $0$ & $C_2 \times C_2 \times C_2$\\
	\hline
\end{tabular}
 \end{center}
  
 %When  $\deg \pi=1$, let us define $w_2(\pi)=0$.
   \end{comment}
   
\begin{prop} Let $G$ be a CW-group. An orthogonal representation $\pi: G \to \Or(V)$ lifts to
\begin{enumerate}
\item $\widetilde \Or(V)$ iff $w_1(\pi)^2=0$.
\item $\Pin^+(V)$ iff $w_2(\pi)=0$.
\item $\Pin^-(V)$ iff $w_2(\pi)+w_1^2(\pi)=0$.
\end{enumerate}
The set of lifts of $\pi$ is acted on simply transitively by the group of continuous linear characters $\chi: G \to \mu_2$. 
\end{prop}
 


\begin{proof}
 This follows from the above and Theorem \ref{lifting.and.coh}.
 \end{proof}
 
 
   
   \bibliographystyle{alpha}
\bibliography{mybib}
 \end{document}
 
 
 \subsection{Associated Bundles}
 Let $E \to B$ be a numerable $G$-bundle, and $(\pi,V)$ a representation of $G$. From this we can form a vector bundle $E[V] \to B$. Since $EG \to BG$ is universal, there is a unique homotopy class of maps 
 \beq
 f: B \to BG
 \eeq
 so that $E[V]=f^*(EG[V])$.
 
 \begin{prop} We have
 \beq
 c(E[V])=f^*(c(\pi)).
 \eeq
 If $\pi$ is an orthogonal representation of $\pi$, then
 \beq
 w(E[V])=f^*(w(\pi)).
 \eeq
 \end{prop}
  
  

