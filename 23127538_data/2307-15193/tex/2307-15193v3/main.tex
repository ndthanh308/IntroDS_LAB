
% \newtheorem{definition}{Definition}
\newtheorem{observation}{Observation}
\newcommand{\prob}{\mathbb{P}}
\newcommand{\Nitem}{M}
%

\newcommand{\nitem}{m}
\newcommand{\Nround}{T}
\newcommand{\nround}{t}

\newcommand{\rigel}[1]{{\color{magenta}[\textsc{Rigel}: \emph{#1}]}}
\newcommand{\negin}[1]{{\color{red}[\textsc{NG}: \emph{#1}]}}
%\newcommand{\rigel}[1]{{[#1]}}
%\newcommand{\negin}[1]{{[#1]}}

%\usepackage{natbib}
% \bibpunct[, ]{(}{)}{,}{a}{}{,}%
% \def\bibfont{\small}%
 %\def\bibsep{\smallskipamount}%
 %\def\bibhang{10pt}%
 %\def\newblock{\ }%
 %\def\BIBand{and}%

%\usepackage[utf8]{inputenc} % allow utf-8 input
%\usepackage[T1]{fontenc}  
%\documentclass[format=acmsmall, review=false]{acmart}
\documentclass[msom,nonblindrev]{informs3}
%\usepackage[toc,page]{appendix}
\usepackage{hyperref}
\usepackage{url}
 \usepackage{natbib}
 \bibpunct[, ]{(}{)}{,}{a}{}{,}%
 \def\bibfont{\scriptsize}%
 \def\bibsep{\smallskipamount}%
 \def\bibhang{24pt}%
 \def\newblock{\ }
%\documentclass[12pt]{report}
\usepackage{bm}
%\usepackage{setspace}
%\renewcommand{\baselinestretch}{1.5} 
\usepackage{zref-totpages}
\usepackage{wrapfig} % This line includes the wrapfig package

\usepackage{pgfplots}
\usepgfplotslibrary{fillbetween}


\usepackage{booktabs} % For better table rules
\usepackage{booktabs} % For formal tables
\usepackage[ruled]{algorithm2e} % For algorithms
%\usepackage{graphicx}
%\usepackage{caption}
\usepackage{subcaption}
\renewcommand{\algorithmcfname}{ALGORITHM}

%\SetAlFnt{\small}
%\SetAlCapFnt{\small}
%\SetAlCapNameFnt{\small}
%\SetAlCapHSkip{0pt}
%\IncMargin{-\parindent}
%\pagenumbering{gobble}
% Choose a citation style by commenting/uncommenting the appropriate line:
%\setcitestyle{acmnumeric}
%\setcitestyle{authoryear}
\OneAndAHalfSpacedXI
%\SingleSpacedXI

\RequirePackage{amssymb,amsmath,ifthen,url,graphicx,color,array,theorem}
\TheoremsNumberedThrough     % Preferred (Theorem 1,

\begin{document}
% Title. Note the optional short title for running heads. In the interest of anonymization, please do not include any acknowledgements.

% Anonymized submission.
%\author{Submission XYZ}

% Abstract. Note that this must come before \maketitle.
\TITLE{Learning in Repeated Multi-Unit Pay-As-Bid Auctions}

\ARTICLEAUTHORS{
\AUTHOR{Rigel Galgana}
\AFF{Operations Research Center, Massachusetts Institute of Technology, \EMAIL{rgalgana@mit.edu} \URL{}}
\AUTHOR{Negin Golrezaei}
\AFF{Sloan School of Management, Massachusetts Institute of Technology,  \EMAIL{golrezaei@mit.edu} \URL{}}
} 



\ABSTRACT{

\textbf{Problem definition:} Motivated by Carbon Emissions Trading Schemes, Treasury Auctions, Procurement Auctions, and Wholesale Electricity Markets, which all involve the auctioning of homogeneous multiple units, we consider the problem of learning how to bid in repeated multi-unit pay-as-bid auctions. In each of these auctions, a large number of (identical) items are to be allocated to the largest submitted bids, where the price of each  of the winning bids is equal to the bid itself. In this work, we study the problem of optimizing bidding strategies from the perspective of a single bidder.

\textbf{Methodology/results:} Effective bidding in pay-as-bid (PAB) auctions is complex due to the combinatorial nature of the action space. We show that a utility decoupling trick enables a polynomial time algorithm to solve the offline problem where competing bids are known in advance. Leveraging this structure, we design efficient algorithms for the online problem under both full information and bandit feedback settings that achieve an upper bound on regret of $O(\Nitem \sqrt{\Nround \log \Nround})$ and $O(\Nitem \Nround^{\frac{2}{3}} \sqrt{\log \Nround})$ respectively, where $\Nitem$ is the number of units demanded by the bidder and  $\Nround$ is the total number of auctions. {\color{black}We accompany these results with a regret lower bound of $\Omega(M\sqrt{T})$ for the full information setting and $\Omega (M^{2/3}T^{2/3})$ for the bandit setting. We also present additional findings on the characterization of PAB equilibria.} 

\textbf{Managerial implications:} While the Nash equilibria of PAB auctions possess nice properties such as winning bid uniformity and high welfare \& revenue, they are not guaranteed under no regret learning dynamics. Nevertheless, our simulations suggest that these properties hold anyways, regardless of Nash equilibrium existence. Compared to its uniform price counterpart, the PAB dynamics converge faster and achieve higher revenue, making PAB appealing whenever revenue holds significant social value---e.g., ETS and Treasury Auctions.

\noindent

\textbf{Keywords.} 
Multi-unit pay-as-bid auctions, Bidding strategies, Regret analysis, Market dynamics.



}
\maketitle

% Title page for title and abstract only.
%\begin{titlepage}

%\maketitle

%\end{titlepage}

% Paper body
\section{Introduction}
Current quantum hardware is unable to carry out universal quantum computations due to the buildup of errors that occur during the computation. 
The magnitude of the individual error is currently above the value that the Threshold Theorem requires in order to kick-start quantum error correction and fault-tolerant quantum computation~\cite[Section 10.6]{nielsen_chuang_2010}. 
Although the experimentally achieved fidelity rates are promising and the error bounds are inching closer to the required threshold, we will have to work for the foreseeable future with quantum hardware with errors that build-up during the computation.  This implies that we can only do a limited number of steps before the output of the computation has become completely uncorrelated with the intended one.

For fault-tolerant quantum computing, we repeat four steps: 
1) We apply a number of single and two-qubit quantum gates, in parallel whenever possible; 
2) We perform a syndrome measurement on a subset of the qubits; 
3) We perform fast classical computations to determine which errors have occurred and how to correct them; 
and, 4) We apply correction terms based on the classical computations.
We then repeat these four steps with a next sequence of gates. 
These four steps are essential to fault-tolerant quantum computing. 


The starting point of this work is to use the four steps outlined above, not to carry out error correction and fault-tolerant computation, but to enhance short, constant-depth, {\em uncorrected} quantum circuits that perform single qubit gates and {\em nearest-neighbor} two qubit gates. 
Since in the long run we will have to implement error-correction and fault-tolerant computation anyhow, and this is done by such a four-step process, why not make other use of this architecture? Moreover, on some of the quantum hardware platforms, these operations are already in place.
Embracing this idea we naturally arrive at the question: what is the computational power of \textit{low-depth} quantum-classical circuits organized as in the four steps outlined above? 
We thus investigate circuits that execute a small, ideally constant, number of stages, where at each stage we may apply, in parallel, single qubit gates and {\em nearest-neighbor} two qubit gates, followed by measurements, followed by low-depth classical computations of which the outcome can control quantum gates in later stages. 
It is not clear, at first, whether such circuits, especially with constant depth, can do anything remotely useful. 
But we will see that this is indeed the case: many quantum computations can be done by such circuits in constant depth. 
By parallelizing quantum computations in this way, we improve the overall computational capabilities of these circuits, as we do not incur errors on qubits that are idle, simply because qubits are not idle for a very long time. 
Furthermore, reducing the depth of quantum circuits, at the cost of increasing width, allows the circuit to be run faster even if errors occur.

The first usage of such a four-step layout, not to do error correction, but to perform computations, can be found in the paradigm of measurement-based quantum computing~\cite{gottesman1999demonstrating,raussendorf2001one,jozsa2006introduction,clark2007generalised}: 
A universal form of quantum computing where a quantum state is prepared and operations are performed by measuring qubits in different bases, depending on previous measurements and intermediate measurements.

\citeauthor{PhamSvore2013} were the first to formalize the four-step protocol for performing computations~\cite{PhamSvore2013}. They included specific hardware topologies by considering two-dimensional graphs for imposing constraints on qubit interactions. In their model, they develop circuits for particularly useful multi-qubit gates, including specifying costs in the width, number of qubits, depth, number of concurrent time steps, size, and total number of non-Identity operations.
As a result, they find an algorithm that factors integers in polylogarithmic depth.
\citeauthor{Browne:2011} showed that the main tool in the work by \citeauthor{PhamSvore2013}, the fan-out gate, can also be replaced by additional log-depth classical computations in the measurement-based quantum computing setting~\cite{Browne:2011}.

More recently, \citeauthor{Cirac:2021} introduced a scheme to implement unitary operations involving quantum circuits combined with Local Operations and Classical Communication ($\mathsf{LOCC}$) channels: $\mathsf{LOCC}$-assisted quantum circuits~\cite{Cirac:2021}. Similarly to the four-step scheme we just described, they allow for a short depth circuit to be run on the qubits, followed by one round of $\mathsf{LOCC}$, in which ancilla qubits are measured and local unitaries are applied based on the measurement outcomes. They show that in this model any 1D transitionally invariant matrix-product state (MPS) with fixed bond dimension is in the same phase of matter as the trivial state. Similar ideas can be found in~\cite{TVV_NonAbelianTopologicalOrder_2022, tantivasadakarn2021long}.

In this work, we introduce a new model, called \textit{Local Alternating Quantum-Classical Computations} ($\LAQCC$). In this model we alternate between running quantum circuits (constrained by locality), ending in the measurement of a subset of qubits, and fast classical computations based on the measurement results. The outcome of the classical computations are then used to control future quantum circuits. We allow for flexibility in this model, by giving different constraints to the power of both the quantum circuits and the classical circuits as well as the number of alternations between them. 
Most attention will be given to $\LAQCC$ containing quantum circuits of constant depth, classical circuits of logarithmic depth and at most a constant number of alternations between them. 
Any circuit constructed in this model is considered to be of constant depth. 
We restrict ourselves to logarithmic depth classical computations, as this is the first natural and non-trivial extension beyond constant-depth classical computations. 
Constant-depth classical computations do however also have an equivalent constant-depth quantum implementation.

The definition of $\LAQCC$ sharpens the original definition of \citeauthor{PhamSvore2013} by adding constraints to the intermediate classical computations. This allows us to bound the power of $\LAQCC$ from above. 

The main result of \citeauthor{Cirac:2021}, that 1D translational invariant MPS with fixed bond dimension can be prepared by $\mathsf{LOCC}$-assisted circuits, relies on local symmetries of the MPS. These symmetries allow them to prepare local states (on a constant number of qubits) and glue them together by doing one round of the appropriate entangling measurement and corrections, after which they run a round of local unitaries to get the desired result. This general scheme for preparing states that exhibit an MPS description with the appropriate local symmetries requires only geometrically local unitaries and one round of measurement and corrections an therefore is accessible in $\LAQCC$. Studying different local symmetries, known as Symmetry Protected Topological (SPT) phases of matter, to find measurement-based constant depth circuits for states is a broad ongoing field of research~\cite{TVV_NonAbelianTopologicalOrder_2022, tantivasadakarn2021long, smith2023deterministic}. 
All these schemes have a $\LAQCC$ implementation.

%$\LAQCC$-circuits also exist for general schemes of preparing local states, based on the local tensors, and gluing them together using one round of entangled measurement and corrections, based on the local symmetry. 
%The main result of \citeauthor{Cirac:2021}, that 1D translational invariant MPS with fixed bond dimension can be prepared by $\mathsf{LOCC}$-assisted circuits, relies heavily on local symmetries of the MPS and as a result also has an equivalent $\LAQCC$ implementation. 
%The corrections applied after the measurement round are local unitaries depending on the local symmetries of the MPS. 

 

%This general scheme of preparing local states, based on the local tensors, and gluing it together by doing one round of entangled measurement and corrections, based on the local symmetry, is accessible in $\LAQCC$.
Note however that \citeauthor{Cirac:2021} also suggest a circuit for the $W$-state.
This circuit uses sequentially and dependent measurement-based corrections of the ancilla qubits. 
These dependent measurements translate to sequential alternations between the quantum and classical circuits and therefore increase the total depth to linear depth, exceeding the constant-depth constraints imposed by $\LAQCC$-circuits. 

We study the power of the $\LAQCC$ model with respect to state preparation, showing that even with only constant quantum-depth and logarithmic classical depth it remains possible to prepare states with long-range entanglement.
Another surprising result is that it is unlikely that $\LAQCC$ circuits are classically simulatable. We show that any instantaneous quantum polynomial-time (IQP) circuit~\cite{Bremner2010,Shepherd2009} has an $\LAQCC$ implementation.
Classical simulation of IQP circuits implies the collapse of the polynomial hierarchy to the third level, which is not believed to be true~\cite{Bremner2017}. Therefore, we expect that $\LAQCC$ circuits are unlikely to be classically simulatable. We bound the power of $\LAQCC$ by showing that it is contained in $\QNC^1$, the class of polynomial-size, log-depth circuits.

Next, we also study the power that intermediate classical calculations can add to quantum computations, by considering a new model that alternates between polynomially many polynomial-depth quantum circuits and unbounded classical computations
We study this model by doing a complexity theoretical analysis, where we draw inspiration from the notions of complexity given by \citeauthor{RosenthalYuen:2022}, \citeauthor{MetgerYuen:2023}, and \citeauthor{Aaronson:2004}.
All three complexity notions are based on the notion of state preparation, instead of more traditional definition of complexity such as the decidability of a computational problem. 
The first two consider classes based on sequences of quantum states preparable by a polynomial-sized quantum circuit, where the circuits are uniformly generated by a computational class, for instance, the class $\mathsf{PSPACE}$, which results in the complexity class $\mathsf{StatePSPACE}$~\cite{RosenthalYuen:2022,MetgerYuen:2023}.
The third notion considers a relative complexity, where the complexity is measured between two given states, and is measured by the number of gates, from a given gate-set, required to transform one state in another state~\cite{Aaronson:2004}. 
For our definition of state preparation complexity, we drop the uniformity constraint from~\cite{RosenthalYuen:2022,MetgerYuen:2023} and define a class as $\mathsf{StateX}$, which refers to states preparable by circuits of type $\mathsf{X}$. 
As an example, if $\mathsf{X} = \QNC^0$, this results in the class $\mathsf{StateQNC^0}$, which is the set of states preparable from the $\ket{0}^n$ state by poly-size constant-depth circuits. 
This notion is similar to the relative complexity from~\cite{Aaronson:2004}, where one state is the  $\ket{0}^n$ state and instead of counting the number of gates we consider the set of states preparable by a fixed number of gates. Using this notion of complexity we show that any state preparable by an $\LAQCC^*$ circuit is also preparable by a $\mathsf{PostQPoly}$ circuit, the class of circuits of polynomial depth with an additional post-selection gate. 

All Clifford circuits have a constant-depth $\LAQCC$ implementation, implying that any stabilizer state can be implemented by a constant-depth $\LAQCC$ circuit, see Section~\ref{sec:clifford_circuits} for a proof of this statement. 
Efficient circuits for stabilizer states have been known already through measurement-based quantum computing. Therefore this paper focuses on the preparation of non-stabilizer states, and as a surprising result we find novel constant-depth protocols for four very natural classes of non-stabilizer states.
Despite the extensive research into these four classes of non-stabilizer states and the many applications of them, no efficient constant- or low-depth state preparation protocols are known yet. We specifically consider these four classes as they are all often used as initial states in other algorithms.

The first state is a uniform superposition over an arbitrary number of states. 
This state finds applications in many quantum algorithms, as they often start with a uniform superposition over multiple states. 
This superposition is often achieved by applying Hadamard gates to every qubit due to its simplicity to prepare. 
Yet, the analysis of many algorithms, such as Shor's algorithm~\cite{Shor:1997}, would benefit from a different initial superposition. 
The circuit to prepare the uniform superposition over an arbitrary number of states uses an exact version of Grover search as a subroutine, that turns a probabilistic circuit, with a known constant probability of success, into a deterministic circuit. 
We use the circuit for preparing a uniform superposition over an arbitrary number of states as a subroutine in the next two quantum state preparation protocols. 

The second state is the $W$-state, the uniform superposition over all computational basis states of Hamming-weight~$1$, a natural long-ranged entangled state that displays a fundamentally nonequivalent type of entanglement from the Greenberger–Horne–Zeilinger state~\cite{WState:2000}, for which $\LAQCC$-type constant-depth circuits were previously known~\cite{PhamSvore2013, Cirac:2021}. 
The $W$-state is often used as benchmark for new quantum hardware~\cite{Haffner2005,Neeley2010,GarciaPerez:2021}. 
A novel way to prepare the $W$-state therefore gives a new way to benchmark different quantum devices with each other. 
A circuit for preparing the $W$-state was given in~\cite{Cirac:2021}, but this implementation requires sequentially alternating measurements followed by local unitaries, which in the $\LAQCC$ model is not considered to be of constant depth. 
We improve this protocol by giving an $\LAQCC$ implementation of the $W$-state, based on a compress-uncompress method that links the one-hot and binary encoding of integers.

The third state considered is the Dicke state, a generalization of the $W$-state, a superposition over all computational basis states with Hamming-weight $k$~\cite{Dicke:1954}. 
Dicke states have relevance in various practical settings.
For instance, for quantum game theory~\cite{zdemir2007}, quantum storage~\cite{Bacon_Compress:2006,Plesch:2010}, quantum error correction~\cite{ouyang2014permutation}, quantum metrology~\cite{toth2012multipartite}, and quantum networking~\cite{prevedel2009experimental}. 
Dicke states have been used as a starting state for variational optimization algorithms, most notably Quantum Alternating Operator Ansatz (QAOA)~\cite{Hadfield2019}, to find solutions to problems such as Maximum k-vertex Cover~\cite{Brandhofer2022,cook2020quantum}.
The ground states of physical Hamiltonians describing one-dimensional chains tend to show a resemblance to Dicke states such as states resulting from the Bethe ansatz, making them an ideal starting state when investigating the ground state behavior of these Hamiltonians~\cite{TDL_BetheAnsatzDerivation:2010,B_ExcitedStateQuantumPhaseTransitions:2013,DickeTransitions:2021}. 
For instance, the algorithm by \citeauthor{van2021preparing}, who give an algorithm to prepare the Bethe ansatz eigenstates of the spin-1/2 XXZ spin chain, starts by first preparing a Dicke state~\cite{van2021preparing}. 
A Dicke-state preparation protocol based on the compress-uncompress methodology used in the $W$-state furthermore finds applications in entanglement distillation, where the entanglement of a large state is concentrated on only a few qubits. 
Efficient deterministic circuits for preparing Dicke states have been proposed by \citeauthor{bartschi2019deterministic}~\cite{bartschi2019deterministic, bartschi2022deterministic_short_depth}. 
They provide a quantum circuit of depth $\mathO(k \log(\frac{n}{k}))$, allowing arbitrary connectivity, to prepare a Dicke state, which they conjecture to be optimal when $k$ is constant. 
In this work, we provide a constant-depth $\LAQCC$ circuit below their conjectured bound already for constant $k$. 
However, this does not directly disprove their conjecture, as we allow for intermediate measurements and classical computations. 
More significantly, we even construct constant-depth $\LAQCC$ circuits for $k = \mathO(\sqrt{n})$ greatly improving their bound.
This construction extends the compress-uncompress method for the $W$-state combined with additional subroutines. 

We continue with a log-depth state preparation protocol for the Dicke-state for arbitrary $k$. 
This protocol implements an efficient transformation between the factoradic number representation and the combinatorial number representation of a positive integer. 
The combinatorial number representation relates directly to the Dicke state. 
The provided efficient transformation between number representation systems might be of independent interest. 

We conclude by modifying our protocol for preparing a Dicke-state to a protocol that prepares quantum many-body scar states in constant-depth. 
These states have low entanglement and longer coherence times than states with similar energy density.
These characteristics make many-body scar states interesting to analyze and relevant within physics.
Many-body scar states appear for instance in the AKLT model~\cite{AKLT:1987,MRBAR:2018,MRB:2018} and different spin models~\cite{SI:2019,MOBFR:2020}.
Known methods for preparing these states have polynomial-depth~\cite{Gustafson:2023}, whereas our circuit has constant depth. 

% We conclude by studying the power that intermediate classical calculations can add to quantum computations. 
% In this study, we define a new model that relaxes constant-depth quantum circuits to polynomial depth quantum circuits, log-depth classical calculations to unbounded classical computations and a constant number of alternations to a polynomial number of alternations. 
% We call this model $\LAQCC^*$. 
% We study this model by doing a complexity theoretical analysis, where we draw inspiration from the notions of complexity given by \citeauthor{RosenthalYuen:2022}, \citeauthor{MetgerYuen:2023}, and \citeauthor{Aaronson:2004}.
% All three complexity notions are based on the notion of state preparation, instead of more traditional definition of complexity such as the decidability of a computational problem. 
% The first two consider classes based on sequences of quantum states preparable by a polynomial-sized quantum circuit, where the circuits are uniformly generated by a computational class, for instance, the class $\mathsf{PSPACE}$, which results in the complexity class $\mathsf{StatePSPACE}$~\cite{RosenthalYuen:2022,MetgerYuen:2023}.
% The third notion considers a relative complexity, where the complexity is measured between two given states, and is measured by the number of gates, from a given gate-set, required to transform one state in another state~\cite{Aaronson:2004}. 
% For our definition of state preparation complexity, we drop the uniformity constraint from~\cite{RosenthalYuen:2022,MetgerYuen:2023} and define a class as $\mathsf{StateX}$, which refers to states preparable by circuits of type $\mathsf{X}$. 
% As an example, if $\mathsf{X} = \QNC^0$, this results in the class $\mathsf{StateQNC^0}$, which is the set of states preparable from the $\ket{0}^n$ state by poly-size constant-depth circuits. 
% This notion is similar to the relative complexity from~\cite{Aaronson:2004}, where one state is the  $\ket{0}^n$ state and instead of counting the number of gates we consider the set of states preparable by a fixed number of gates. Using this notion of complexity we show that any state preparable by an $\LAQCC^*$ circuit is also preparable by a $\mathsf{PostQPoly}$ circuit, the class of circuits of polynomial depth with an additional post-selection gate. 

\paragraph{Summary of results}
\begin{itemize}
    \item We give a new definition of a computational model that captures the power of the four step process: applying a constant number of layers of one- and two-qubit gates; performing a syndrome measurement; perform a fast classical computation determining corrections; apply corrections. We call this model \emph{Local Alternating Quantum Classical Computations}, or $\LAQCC$ for short. In this model we bound the allowed quantum operations, intermediate classical calculations, and number of rounds separately. In Section~\ref{sec:LAQCC_model} we define this model and give a list of operations based on results from literature contained in this computational model. In some of these operations we explicitly use that we allow for multiple, but at most constant, rounds  of corrections.
    \item  We show show that there exist $\LAQCC$ circuits that can not be weakly simulated in Section~\ref{sec:IQP_in_LAQCC}. We further show that for every $\LAQCC$ circuit there exists a $\QNC^1$ circuit simulating it perfectly, in Section~\ref{sec:LAQCC_in_QNC1}.
    \item We introduce a new type computational complexity for preparing states and show that the extension of $\LAQCC$ where we allow a polynomial number of rounds and unbounded classical computation, is contained in $\mathsf{PostQPoly}$, the class of polynomial circuits with post-selection, in Section~\ref{sec:Complexity results}.
    \item We show a protocol to prepare the uniform superposition state of size $q$ in $\LAQCC$ using $\mathO(\ceil{\log_2(q)}^2)$ qubits in Section~\ref{sec:superposition_modulo_q}. 
    \item We show a protocol to prepare the $W_n$ state in $\LAQCC$ using $\mathO(n\log(n))$ qubits in Section~\ref{sec:W_state_in_LAQCC}.
    \item We show two ways of preparing the Dicke-$(n,k)$ state. The first method is in $\LAQCC$, works up to $k = \mathO(\sqrt{n})$, uses $\mathO(n^2\log(n))$ qubits, and is found in Section~\ref{sec:dicke:small_k}. The second method is in $\LAQCC\text{-}\mathsf{LOG}$ (an extension of $\LAQCC$ allowing for logarithmic number of alterations instead of constant), works for any $k$, uses $\mathO(\text{poly}(n))$ qubits, and is found in Section~\ref{sec:Dicke_in_LAQCC_LOG}. 
    \item We extend on our $\LAQCC$ method of generating Dicke-$(n,k)$ states for $k = \mathO(\sqrt{n})$ and show a protocol to generate many-body scar states for a particular Hamiltonian in $\LAQCC$ (Section~\ref{sec:many_body_scar}). 
\end{itemize}
Summarized in a table, we provide the following state generation protocols:
\begin{table}[htb]
\centering
\begin{tabular}{l|l|l|l}
\textbf{State description} & \textbf{Width} & \textbf{Depth} & \textbf{Implementation}\\
\hline 
Uniform superposition mod $q$: $\frac{1}{\sqrt{q}} \sum_{i = 0}^{q-1}\ket{i}$ & $\mathO(\ceil{\log^2 q})$ & $\mathO(1)$ & Section~\ref{sec:superposition_modulo_q}\\

$W$-state: $\frac{1}{\sqrt{n}}\sum_{i = 0}^{n-1}\ket{e_i}$ & $\mathO(n \log n)$ & $\mathO(1)$ & Section~\ref{sec:W_state_in_LAQCC}\\

Dicke-$(n,k)$, $k = \mathO(\sqrt{n})$: $\binom{n}{k}^{-1/2}\sum_{x \in \{0,1\}^n: |x| = k} \ket{x}$ &  $\mathO(n^2\log n)$ & $\mathO(1)$ 
&Section~\ref{sec:dicke:small_k}\\

Dicke-$(n,k)$: $\binom{n}{k}^{-1/2}\sum_{x \in \{0,1\}^n: |x| = k} \ket{x}$ & $\mathO(\text{poly}(n))$ & $\mathO(\log n)$ &Section~\ref{sec:Dicke_in_LAQCC_LOG}\\

QMBS: $\ket{S_k} = \frac{1}{k! \sqrt{\mathcal N(n,k)}}(Q^\dagger)^k \ket{\Omega}$ &  $\mathO(n^2\log n)$ & $\mathO(1)$  &  Section~\ref{sec:many_body_scar}
\end{tabular}
\caption{Summary of state preparation protocols given in this paper.}
\label{tab:sate_prep}
\end{table}
In the entry for the quantum many-body scar state $Q$ denotes the raising operator and $\mathcal N(n,k)=\binom{n-k-1}{k}$. 
Section~\ref{sec:many_body_scar} will provide more details on the variables and the implementation. 

\paragraph{Organization of the paper}
\noindent We first introduce relevant preliminaries in Section~\ref{sec:preliminaries}. 
In Section~\ref{sec:LAQCC_model} we formally define the class of Local Alternating Quantum-Classical Computations ($\LAQCC$). We also show that any Clifford circuit can be implemented in constant depth $\LAQCC$ (a result based on a result from measurement-based quantum computing~\cite{jozsa2006introduction}). 
This result allows us to give many useful multi-qubit gates and routines in Section~\ref{sec:gates_created_in_LAQCC}. 
Beyond that we show that constant depth $\LAQCC$ circuits are contained in $\QNC^1$ and that any $\mathsf{IQP}$ circuit has an $\LAQCC$ implementation.
We conclude this section with an analysis of a more powerful instantiation of $\LAQCC$ and show an inclusion with respect to the class $\mathsf{PostQPoly}$, which is the class of circuits of polynomial depth with one additional post-selection gate. 
In Section~\ref{sec:state_prep_in_LAQCC} we give $\LAQCC$ circuit implementations for preparing the uniform superposition over an arbitrary number of states, the $W$-state and the Dicke state up to $k = \mathO(\sqrt{n})$. We furthermore give a log-depth circuit implementation for preparing the Dicke state for any $k$. We conclude by showing a $\LAQCC$ circuit for generating many body scar states of a particular type of Hamiltonian.


\section{Preliminaries}

\subsection{Model}
\textbf{Auction format: Pay-as-bid.} Consider a setting with $N$ bidders and $M$ units to auction off in a pay-as-bid auction with reserve $r\ge 0$. Let ${\bf b}_n = (b_{n, m})_{n\in [N], m\in [M]}$  be the bids submitted by bidder $n$, where $b_{n, 1}\ge b_{n, 2}\ge b_{n, M}\ge 0$ and $b_{n, m}$ is the bid of bidder $n$ for the $m$-th unit. One can view $b_{n,m}$, $n\in [N]$ and $m\in [M]$, as a proxy for the bidder $n$'s (marginal) valuation for the $m$-th unit, denoted by $v_{n, m}$, where we assume that $v_{n, 1}\ge v_{n, 2}\ge \ldots \ge v_{n, M}$ to model the diminishing return of one additional unit.  
In a pay-as-bid auction with reserve $r$,  any  bid that is less than reserve price $r$ is first discarded. The remaining (cleared) bids are then sorted in decreasing order.  The $m$-th unit is allocated to the bidder with the $m$-th highest bid if the number of cleared bids is greater than or equal to $m$, charging him his bid.
 {\color{red}todo: briefly talk about the tie breaking rule.}  That is, if bidder $n$ is allocated $x_n(\mathbf b)$ units, he is charged $\sum_{m=1}^{x_n(\mathbf b)} b_{n,m}$ while his (total) valuation is $\sum_{m=1}^{x_n(\mathbf b)} v_{n,m}$, where $\mathbf b = ((\mathbf b_n)_{n\in [N]}; r)$ is the submitted bids by all the bidders and reserve price $r$ {\color{red} for now I combined the bids and reserve price}.  This leads to (quasi-linear) utility of 
\begin{align}\label{eq:mu}\mu_n(\mathbf b) = \sum_{m=1}^{x_n(\mathbf b)} \big(v_{n,m} -b_{n,m}\big)\,\end{align}
for bidder $n$.

%The auctioneer announces $\m$ units of a good for sale. Each player $i$ submits bids $\vec{b}_{i} = (b_{i,1}, \ldots, b_{i,\m})$, where $b_{i,j}$ 
 %is player $i$'s bid for $j$-th unit.
%	The auctioneer sorts  the bids in decreasing order. Then   for each $j = 1, \ldots, \m$,  the auctioneer allocates the $j$-th unit to the player that submitted the $j$-th highest bid, charging them a price equal to the $(\m+1)$-th highest bid.

\textbf{Repeated Setting.} In this work, we consider a repeated setting where the pay-as-bid auction is run over the course of $T$ rounds. We denote the bid of bidder $n$ in the $t$-th auction ($t\in [T]$) by ${\bf b}_n^t = (b_{n, m}^t)_{n\in [N], m\in [M]}$ and define $\mathbf b^t = ((\mathbf b_n^t)_{n\in [N]}; r^t)$, where $r^t$ is the reserve price in auction $t$. Similarly, we define $\mathbf b^t_{-n} = ((\mathbf b_i^t)_{i\in [N], i\ne n}; r^t)$ as the bids submitted by all the bidders expect bidder $n$ in auction $t$, and reserve prices $r_t$.



\textbf{Offline Setting.} In the offline setting, we aim to optimize the biding strategy of a bidder $n$ while having access to $H_T :=(\mathbf b^1_{-n}, \mathbf b^2_{-n}, \ldots , \mathbf b^T_{-n})$, which is the submitted bids by the other bidders and reserve prices in the past $T$ rounds. Mathematically speaking, we fix a bidder $n$, and we optimize  a bid vector $\mathbf b_n$ that maximizes the bidder $n$'s cumulative utility over $T$ rounds with competing bids and reserve prices are $H_T$:
\begin{align}\tag{Offline}\label{eq:offline}
    \max_{\mathbf b_n } \sum_{t=1}^T \mu_n ((\mathbf b_{-n}^t, \mathbf b_n)) \qquad  \text{s.t.} \qquad b_{n,1}\ge  \ldots \ge b_{n, M}\ge 0\,.\end{align}
{\color{red} todo: please motivate this setting and say what result we will have here and where}

\textbf{Online  Setting.} In the online setting, we again fix a bidder $n$, and we aim to 



{\color{red} discussion here should be written so that we can use it for both offline and online settings }
We consider $\Nround$ rounds of homogeneous multi-unit auctions, each with $\Nitem$ items to be allocated across $N$ agents. In round $\nround$, bidder $n$ is endowed with weakly decreasing marginal valuation profile $\bm{v}^{n, \nround} \in \{\bm{v}: 1 \geq v_1 \geq \ldots \geq v_\Nitem \geq 0\} \equiv [0, 1]^\Nitem$. {\color{red} we don't need discretization for the offline setting. So, I suggest we consider the continous setting here in the model section and when it comes to the learning section, we talk about discretization} Each bidder submits weakly decreasing bids $\bm{b}^{n, \nround} \in \{\bm{b}: 1 \geq b_1 \geq \ldots \geq b_\Nitem \geq 0, b_\nitem \in \mathcal{B} \forall \nitem \in [\Nitem]\}$, where the set of allowable bids $\mathcal{B} = \{B_1,\ldots,B_{|\mathcal{B}|}\}$ with $0 = B_1 < \ldots < B_{|\mathcal{B}|} = 1$ denotes some discretization of $[0, 1]$. We abuse notation and denote this set of non-decreasing $\mathcal{B}$-restricted bids as $\mathcal{B}^\Nitem$. The auctioneer selects anonymous reserve price $\pi^\nround \in \mathcal{B}$ which filters out all bids strictly below $\pi$ from consideration. Considering only bids at least $\pi^\nround$, agents are allocated some number of items $x(\bm{b}^{n, \nround}, \bm{b}^{-n, \nround}, \pi^\nround)$ equal to the number of bids within the $\Nitem$ largest bids. Here, $\bm{b}^{-n, \nround}$ denotes the largest $\Nitem$ bids among all other agents in decreasing order and we again abuse notation to let $\mathcal{B}^{-\Nitem}$ denote the set of possible $\bm{b}^{-n, \nround}$. We also assume tie-breaks are settled using a public knowledge, deterministic tie-breaking mechanism. Assuming agent $n$ wins $x^{n, \nround} = x(\bm{b}^{n, \nround}, \bm{b}^{-n, \nround}, \pi^\nround)$ items, the utility $\mu(\bm{v}^{n, \nround}, \bm{b}^{n, \nround}, \bm{b}^{-n, \nround}, \pi^\nround)$ is given by the difference in reward $\sum_{\nitem=1}^{x^{n, \nround}} v^{n, \nround}_\nitem$ and their payment $\sum_{\nitem=1}^{x^{n, \nround}}b^{n, \nround}_\nitem$. The welfare and revenue of the auction is equal to the sum over all agents' rewards and payments respectively. At the end of each auction $\nround$, the auctioneer reveals market clearing price $c^\nround = \max(b^{n, \nround}_{x^{n, \nround}}, b^{-n, \nround}_{x^{n, \nround}+1}, \pi^\nround)$, defined as the maximum of $\pi^\nround$ and the lowest winning bid round $\nround$. Agents additionally observe their own allocation $x^{n, \nround}$ for all $n \in [N]$. We define the set of information known to agent $n$ at round $\nround$ before submitting their bid to be $H^{n, \nround} = v^{n, \nround} \cup \{(v^{n, \tau}, x^{n, \tau}, c^\tau)\}_{\tau \in [\nround - 1]}$. The set of repeated auctions is described succinctly as follows:

% Algorithm
\begin{algorithm}[t]
	\KwIn{{\color{red} what is this? why do we have an algorithm  here?}Valuations $\{\bm{v}^{n, \nround}\}_{n \in [N], \nround \in [\Nround]}$ for $\bm{v}^{n, \nround} \in \{\bm{v}: 1 \geq v_1 \geq \ldots \geq v_\Nitem \geq 0\}$, Bids $\{\bm{b}^{n, \nround}\}_{n \in [N], \nround \in [\Nround]}$ for $\bm{b}^n \in \mathcal{B}^\Nitem,$ Reserves $\{\pi^\nround\}_{\nround \in [\Nround]}$}
	\KwOut{Aggregate utilities $\{\mu^n\}_{n \in [N]}$, aggregate welfare $\sum_{n = 1}^N \textsc{Reward}^n$, total revenue $\sum_{n = 1}^N \textsc{Payment}^n$.}
	$\mu^n, \textsc{Reward}^n, \textsc{Payment}^n \gets 0$ for all $n \in [N]$\;
        $H^{n, 1} = \{\bm{v}^{n, 1}\}$ for all $n \in [N]$
	\For{$\nround \in [\Nround]$:}{
            Bidders submit bids $\bm{b}^{n, \nround}$ for all $n \in [N]$\;
    	\For{$n \in [N]$:}{
                Observe allocation $x^{n, \nround}$ and clearing price $c^\nround = \max(b^{n, \nround}_{x^{n, \nround}}, b^{-n, \nround}_{x^{n, \nround}+1}, \pi^\nround)$\;
        		% $x^{n, \nround} \gets \sum_{\nitem=1}^\Nitem \textbf{1}_{b^{n, \nround}_\nitem \geq \max(\pi^\nround, b^{-n, \nround}_\nitem)}$\;
    		$\textsc{Reward}^n \gets \textsc{Reward}^n + \sum_{\nitem=1}^{x^{n, \nround}} v^{n, \nround}_\nitem$\;
    		$\textsc{Payment}^n \gets \textsc{Payment}^n + \sum_{\nitem=1}^{x^{n, \nround}} b^{n, \nround}_\nitem$\;
                $\mu^n \gets \mu^n + \textsc{Reward}^n - \textsc{Payment}^n$\;
                $H^{n, \nround + 1} \gets H^{n, \nround} \cup (\bm{v}^{n, \nround+1}, x^{n, \nround}, c^\nround)$
    	}
        }
        \textbf{Return} $\{\mu^n\}_{n \in [N]}, \sum_{n = 1}^N \textsc{Reward}^n, \sum_{n = 1}^N \textsc{Payment}^n$
	\caption{\textsc{MUA}$(\{\bm{v}^{n, \nround}, \bm{b}^{n, \nround}, \pi^\nround\}_{n \in [N], \nround \in [\Nround]})$}
	\label{alg:MUA}
\end{algorithm}

\rigel{Perhaps mention that our censored feedback setting is equivalent to Yanjun Han's} We explain two technicalities our model selection and description. First, the tiebreaks are implicitly handled within $x^{n, \nround}$ in a deterministic fashion known to each agent. To avoid dealing with indexing issues, we perturb each bidder $n$'s bid by $(N-n)\epsilon$ for some infinitesimal $\epsilon > 0$, which has the effect of prioritizing lower indexed bidders at a negligible increase in payment. For the remainder of the paper, we will implicitly assume as shorthand that the event $\{b^{n, \nround}_\nitem = B\}$ for $B \in \mathcal{B}$ actually means $\{b^{n, \nround}_\nitem = B + (N-n)\epsilon = B'\}$ for $B' \in \mathcal{B}_n$, where agent $n$'s bid set is defined as $\mathcal{B}_n \equiv \{B: B + (N-n)\epsilon\}_{B \in \mathcal{B}}$ as opposed to $\mathcal{B}$. Under this assumption, we can decompose $x^{n, \nround}$ as a function of $\bm{b}^{n, \nround}, \bm{b}^{-n, \nround}, \pi^\nround$ which greatly simplifies the offline and online learning algorithms and corresponding analyses. Second, we assume that information regarding the reserve price is only revealed after each auction via the clearing price. We will later consider variants of the auction where $H^{n, \nround} = (v^{n, \nround}, \pi^\nround) \cup \{(v^{n, \tau}, x^{n, \tau}, c^\tau)\}_{\tau \in [\nround - 1]}$ the reserve price is announced at the beginning of each auction, where the analysis is slightly more complicated. As reserve prices are selected adaptively by the auctioneer in many real world situations, from the point of view of the agents, we assume that reserve prices are be generated adversarially.
% We also discuss generalizations to discriminatory reserves $\{\pi^\nround_\nitem\}_{\nitem \in [\Nitem], \nround \in [\Nround]}$ or non-anonymous reserves $\{\pi^\nround_n\}_{n \in [N], \nround \in [\Nround]}$.

\subsection{Problem Statement}

The objective of each agent $n$ is to select a sequence of bid vectors $\{\bm{b}^{n, \nround}\}_{n \in [N], \nround \in [\Nround]}$ that maximizes their aggregate utility $\mu^n$. Optimizing $\mu^n$ is trivial in the offline setting where each of $\bm{b}^{-n, \nround}$ and $\pi^\nround$ are revealed to agent $n$ in advance and thus can correspondingly select the optimal $\bm{b}^{n, \nround}$ for each $\nround \in [\Nround]$. Rather than the optimal sequence of bid vectors, we can instead consider the optimal fixed bid vector which we show later can be computed using a dynamic program efficiently. Of course, $\bm{b}^{-n, \nround}$ and $\pi^\nround$ for all $\nround \in [\Nround]$ are not known beforehand and are instead revealed in an online fashion. As such, the agents must learn how to bid optimally during the sequence of auctions only given knowledge of historic auction results---which we will formalize shortly what this means---and their current valuation profile. The performance of an agent's learning strategy will be measured in terms of regret---the difference between their expected utility under their learning strategy and under the hindsight optimal fixed bid vector. In particular, agents will select a bid vector sampled from $F^{n, \nround}(H^{n, \nround}) = F^{n, \nround} \in \Delta(\mathcal{B}^\Nitem)$ where $\Delta(S)$ denotes the set of all valid probability measures over set $S$.
\begin{align*}
    \textsc{Regret}^{n, \Nround}(F^{n, \nround} \mid \{\bm{v}^{n, \nround}\}_{\nround \in [\Nround]} = \max_{\bm{b}^n \in \mathcal{B}^\Nitem} \sum_{\nround=1}^\Nround \mu(\bm{v}^{n, \nround}, \bm{b}^n, \bm{b}^{-n, \nround}, \pi^{\nround}) - \mathbb{E}_{\bm{b}^{n, \nround} \sim F^{n, \nround}} \sum_{\nround=1}^\Nround \mu (\bm{v}^{n, \nround}, \bm{b}^{n, \nround}, \bm{b}^{-n, \nround}, \pi^\nround)
\end{align*}
Where we assume that $\bm{v}^{n, \nround}$ (which can be thought of as a context as it is revealed at the start of each round), $\bm{b}^{-n, \nround}$ and $\pi^\nround$ can be selected by an adaptive adversary. In the bandit case, we assume that $\bm{v}^{n, \nround}$ are fixed to be $\bm{v}^\nround$. according to some known distribution. For simplicity, we omit the arguments when they are clear from context: $\textsc{Regret}^{n, \Nround} = \textsc{Regret}^{n, \Nround}(F^{n, \nround} \mid \{\bm{v}^{n, \nround}\}_{\nround \in [\Nround]}$. We wish to derive a learning algorithm---construct functions $F^{n, \nround}$---that achieves an upper bound on $\textsc{Regret}^{n, \Nround}$ that is polynomial (or better) in $\Nitem$ and sub-linear in $\Nround$. Assuming that all agents obey this learning algorithm, we derive bounds on the expected revenue and expected welfare for a sequence of reserves $\{\pi^{\nround}\}_{\nround \in [\Nround]}$ and compare these guarantees to the announced reserves setting.

\subsection{Contributions}

Our primary contribution is representing the allocation and utility functions in a form that enables maximal cross learning between bid vectors and valuations. In particular, we define $x^{n, \nround} = \sum_{\nitem=1}^\Nitem \textbf{1}_{b^{n, \nround}_\nitem \geq \max(\pi^\nround, b^{-n, \nround}_\nitem)}$ which decomposes the utility function as a sum over the the utilities per slot $\nitem$. 
We then construct an efficient $O(\Nitem |\mathcal{B}|^2)$ time and $O(\Nitem^2 |\mathcal{B}|)$ space complexity dynamic program that computes the hindsight optimal bid vector. In this dynamic program, we show a crucial (conditional) independence between agent $n$'s utility corresponding to their bid in the $\nitem$'th slot $b^{n, \nround}_\nitem$ and their bid in the $\nitem+1$'st slow $b^{n, \nround}_{\nitem + 1}$. With this, we decouple the aggregate utility of a bid vector $\bm{b}^{n}$ as a function of the per-slot aggregate utilities. This allows us to efficiently implement exponential weights in the full information setting, which achieves regret $O(\Nitem \sqrt{T \log |\mathcal{B}|}$, whilst allowing for adversarially generated valuation profiles $\{\bm{v}^{n, \nround}\}_{\nround \in [\Nround]}$. In the bandit setting, one may be hopeful to be able to apply the same decoupling argument to graph-feedback learning algorithms, such as $\textsc{Exp3.G}$ or $\textsc{Exp3.SET}$ \rigel{cite}, which would make use of side information in the form of $c^\nround$. Unfortunately, the learner never observes the entire feedback graph and therefore cannot construct meaningful, polynomially bounded variance utility estimates. Instead, we can implement Follow-the-Regularized-Leader (FTRL) style algorithms, such as $\textsc{O-REPS}$ or $\textsc{Component Hedge}$ \rigel{cite}, which achieves low regret $O(\Nitem |\mathcal{B}|\sqrt{T\log|\mathcal{B}|})$ and polynomial time and space complexity. However, these algorithms must assume a static valuation profile and require additional setup.

% We apply this idea again in the online learning setting. In particular, we derive a decoupled $\textsc{Exp3}$ algorithm that exploits cross learning (See algorithm $\textsc{Exp3.G}$ in \rigel{Cite paper}) which achieves small regret whilst retaining polynomial time and space complexities despite the combinatorially large bid space. We do this by upper bounding the independence number of any possible bid vector network by realizing that the utility function can be fully characterized by the $\Nitem$ largest bids amongst $\bm{b}^{-n, \nround}$. Lastly, we show how to convert the output of our decoupled $\textsc{Exp3.G}$ into a valid (weakly monotonic) bid vector.

\section{Bid Optimization: Bandit Feedback}

In this section, we now consider the more realistic feedback setting where agent $n$ only observes $H^{\nround}$ regarding each auction $\nround$'s allocation $x^\nround = x^\nround_n(\bm{b}^\nround)$. We apply ideas from solutions to the Stochastic Shortest Paths problem (SSP) to instead associate weights with each node given by a $(\nitem, B)$ pair, which can be interpreted as the action of bidding bid $B \in \mathcal{B}$ at slot $\nitem \in [\Nitem]$. \rigel{Fix the flow here} Using vanilla $\textsc{Exp3}$ algorithm will result in exponential regret and time complexity. Instead, one may be tempted to use generalizations of $\textsc{Exp3}$ that exploit additional information for cross-learning between bid vectors. Note that any bid above $c^\nround = b^{\nround}_{x^\nround}$ in the first $x^{\nround}$ slots is guaranteed to win the corresponding item, and any bid below $d^\nround = b^{\nround}_{x^\nround + 1}$ is guaranteed to lose the item. This implies that agent $n$ can determine the counterfactual utility of any bid vector $\bm{b}$ at round $\nround$ if $b_{x^{\nround} \geq c^\nround}$ and $b_{x^{\nround}+1} \leq d^\nround$. These generalized $\textsc{Exp3}$ algorithms will construct importance sampling estimators of the utility of bid vectors as follows: Let $\bm{b}$ denote the bid vector sampled at round $\nround$ according to distribution $F^{\nround}$, with corresponding $x^\nround, c^\nround, d^\nround$. Then, we define utility estimates $\hat{\mu}^{\nround}_n(\bm{B}') = \mu^{\nround}_n(\bm{B}') \textbf{1}(\text{Observe $\bm{B}'$ | $\bm{b}^{\nround} = \bm{B}$}) \prob(\text{Observe $\bm{B}'$ | $\bm{b}^{\nround} \sim F^{\nround}$})^{-1}$. However, this last term $\prob(\text{Observe $\bm{B}'$ | $\bm{b}^{\nround} \sim F^{\nround}$})$ is problematic. Recalling from above that the utility of $\bm{B}'$ is only observable under $\bm{b}^{\nround}$ if the following three conditions hold:
\begin{enumerate}
    \item The allocation $x^{\nround}_n(\bm{B}') = \sum_{\nitem=1}^\Nitem \textbf{1}_{B'_\nitem \geq \max(b^{\nround}_{-\nitem}, \pi^\nround}$ is equal to $x^\nround$.
    \item The clearing price $c^\nround = B'_{x^\nround}$ under $\bm{B}'$ is at least as large as $b^{\nround}_{x^\nround}$.
    \item The guaranteed losing price $d^\nround = '_{x^\nround+1}$ under $\bm{B}'$ is at most $b^{\nround}_{x^\nround+1}$.
\end{enumerate}
The set of $\bm{b}^{\nround}$ that satisfies the above conditions is never revealed to the bidder and hence, one of the terms required to define the utility estimates cannot be computed. As such, we resort to learning algorithms that construct utility estimates that do not depend on the entire set of observable bid vectors. Such algorithms include the FTRL based $\textsc{O-REPS}$ algorithm among other techniques to solve SSP problems. In these methods, the utility estimates depend only on the per-slot utilities of the selected bid vector. We first explain how to model our bid optimization problem as an SSP problem and then briefly describe an algorithm to solve it. Then, we show a variant of this algorithm that achieves better time and space complexity.

\subsection{Shortest Paths Formulation}

At a high level, the agent $n$ only updates their utility estimates corresponding to the slot-bid value pairs corresponding to their action $\{w^\nround_\nitem(b^\nround_\nitem)\}_{\nitem \in [\Nitem]}$ in addition to the aggregate utility $\mu^{\nround}_n(\bm{b}^\nround)$. This information feedback system is common among many shortest paths problems, where agents observe the utility derived from each selected edge in addition to the total path utility. Using this commonality, we frame our bid learning problem as an instance of episodic learning over Markov Decision Processes (MDPs) with adversarially changing costs.
\begin{enumerate}
    \item We define 2 types of states: source node $S_0$, followed by $\Nitem$ layers of states of width $|\mathcal{B}|$ between them which we denote by $\{(\nitem, B)\}_{\nitem \in [\Nitem], B \in \mathcal{B}}$. We let $\mathcal{S}$ denote the set of all states and let $s_\nitem$ denote which $B$ bid value the agent is at on layer $\nitem$.
    \item There are 2 types of actions available at each state which describe which state in the subsequent layer the agent will move to: actions from $S_0$ to the entire first layer of nodes $\{b_{1,B}\}_{B \in \mathcal{B}}$ and actions between layers $\nitem$ to layer $\nitem+1$. In particular, the latter set of actions requires non-decreasing bid value---i.e. $(\nitem, B)$ only has an action to $(\nitem+1, B')$ for $B' \leq B$. This reflects the bid monotonicity assumption. With probability 1, the agent will transition to the state prescribed by their action. We let $\mathcal{A} \equiv \mathcal{B}$ denote the set of possible actions with subsets $\mathcal{A}_B = \{B' \in \mathcal{B}: B' \leq B\}$ being the actions available at state $(\nitem, B)$ for $\nitem \in [\Nitem]$. Let $a_\nitem$ denote which action in $\mathcal{A}$ was taken at layer $\nitem$.
    \item There are 3 parameters: the valuation vector $\bm{v}$, adversarial (sorted) bid vector $\bm{b}^{\nround}_-$, and reserve price $\pi^\nround$. With these parameters, the weight of any incoming edge to state $(\nitem, B)$ is given by $w^{\nround}_{\nitem}(B) = 1_{B \geq \max(b^{\nround}_{-\nitem}, \pi^\nround)} (v^{\nround}_\nitem - B)$. This is precisely the same as the definition of $w^{\nround}_{\nitem}(B)$ as in offline bid optimization. The total reward of episode $\nround$ is equal to the sum of the edge weights traversed.
    \item We define MDP $\mathcal{M}^\nround = (\mathcal{S}, \mathcal{A}, \mathcal{P}, \{\bm{w}^\nround\})$ to have states and actions as prescribed above, deterministic transitions $\mathcal{P}$ such that $\prob(s_{\nitem+1} = B' \mid s_{\nitem} = B, a_\nitem = B') = 1$, and edge weights defined by $\bm{w}^\nround$.
\end{enumerate}
Let $\pi: \mathcal{S} \times \mathcal{A} \to [0, 1]$ denote a policy over MDP $\mathcal{M}^\nround$, where $\pi((\nitem, B), B')$ denotes the probability of selecting action $B' \leq B$ at state $(\nitem, B)$. We abuse notation and let $B \sim \pi$ denote a sequence of bid values $(s_1,\ldots,s_\Nitem)$ within a path sampled according to policy $\pi$. We then define a corresponding state-action occupancy measure $\rho^\pi((\nitem, B), B') = \prob_{\bm{B} \sim \pi}(s_\nitem = B, a_\nitem = B')$, with $\rho^\pi(S_0, B') = \prob_{\bm{B} \sim \pi}(s_0 = S_0, a_0 = B_1)$. \rigel{Look up Online network flow bandit problem} Since we are operating under a deterministic Markovian environment with (directed) edges between consecutive layers, we have:
\begin{align}
    \sum_{B' \leq B} \rho^\pi((\nitem, B), B') = \sum_{B" \geq B} \rho^\pi((\nitem-1, B"), B) \quad \text{and} \quad \sum_{B \in \mathcal{B}} \sum_{B' \leq B} \rho^\pi((\nitem, B), B') = 1
\end{align}
Let $\Pi$ denote the set of all policies on $\mathcal{M}_\nround$. With the above equalities and the obvious non-negativity constraints $\rho^\pi((\nitem, B), B') \geq 0$ for all $B' \leq B$, we can now define the set $\Delta$ of all possible state-action occupancy measures.
\begin{definition}
    Let $\Delta(\Pi)$ denote the set of all $\rho \in \mathcal{S} \times \mathcal{A} \to [0, 1]$ with the following properties:
    \begin{enumerate}
        \item $\sum_{B \in \mathcal{B}} \sum_{B' \leq B} \rho((\nitem, B), B') = 1$ for all $\nitem \in [\Nitem]$
        \item $\sum_{B \in \mathcal{B}} \rho(S_0, B) = 1$
        \item $\sum_{B' \leq B} \rho((\nitem, B), B') = \sum_{B" \geq B} \rho((\nitem-1, B"), B)$ for all $B \in \mathcal{B}, \nitem \in [\Nitem]$
        \item $\sum_{B' \leq B} \rho((1, B), B') = \rho(S_0, B)$
    \end{enumerate}
    This is equivalent to the condition that exists a policy $\pi \in \Pi$ such that $\rho((\nitem, B), B') = \prob_{\bm{B} \sim \pi}(s_\nitem = B_\nitem, a_\nitem = B_{\nitem+1})$ for all $B \geq B' \in \mathcal{B}$, $\nitem \in [\Nitem]$ and $\rho(S_0, B') = \prob_{\bm{B} \sim \pi}(s_0 = S_0, a_0 = B_1)$.
\end{definition}
We say that policy $\pi$ generates state-action occupancy measure $\rho$ if $\pi(s, a) = \frac{\rho(s, a)}{\sum_{b: \mathcal{A}(s)} \rho(s, b)}$ for all $s \in \mathcal{S}, a \in \mathcal{A}(s)$ where $\mathcal{A}(s)$ denotes the set of valid actions at state $s$. From the definition of $\Delta(\Pi)$, we can see that each $\rho$ uniquely determines $\pi$ and vice versa. We let $\pi^\rho$ denote the policy that generates the occupancy measure $\rho$. Hence, computing the regret minimizing policy $\pi^\rho \in \Pi$ is equivalent to minimizing the regret minimizing state-action occupancy measure $\rho$. Once computing $\rho$, we can compute the corresponding $\pi^\rho$ and then simulate a bid vector according to $\pi^\rho$. The key idea is that optimizing with respect to $\rho$, as opposed to $\pi$, can be framed as instance of online linear optimization. More specifically, we can compute the (expected) loss function at round $\nround$ as follows:
\begin{align}
    \mathbb{E}_{\bm{B} \sim \pi^\rho}\left[ \mu^{\nround}_n(\bm{B}) \right] = \mathbb{E}_{\bm{B} \sim \pi^\rho}\left[ \sum_{\nitem=1}^\Nitem w_\nitem^\nround(B_\nitem) \right] = \sum_{\nitem=1}^\Nitem \prob_{\bm{B} \sim \pi^\rho}(s_{\nitem-1} = B_{\nitem-1}, a_{\nitem-1} = B_{\nitem}) w_\nitem^\nround(B_\nitem)
\end{align}
Substituting in the definition of $\pi^\rho$ and the corresponding $\rho$, we have:
\begin{align}
    \mathbb{E}_{\bm{B} \sim \pi^\rho}\left[ \mu^{\nround}_n(\bm{B}) \right] = \sum_{\nitem=1}^\Nitem \sum_{B \in \mathcal{B}} \sum_{B' \leq B} \rho((\nitem-1, B), B') w_\nitem^\nround(B') = \langle \bm{\rho}, \bm{w}^\nround \rangle 
\end{align}
Where we assume $s_0 = S_0$ and define $\bm{w}^\nround \equiv \{w_\nitem^\nround(B')\}_{\nitem \in [\Nitem], B \geq B' \in \mathcal{B}}$. Assuming that the learner selects occupancy measure $\bm{\rho}^\nround$ at round $\nround$ with fixed valuation profile $\bm{v}$, the regret can then be written as:
\begin{align}
    \textsc{Regret}_\mathcal{B}^{\Nround} &= \max_{\bm{b} \in \mathcal{B}^{+\Nitem}} \sum_{\nround=1}^\Nround \mu^\nround_n(\bm{b}) - \mathbb{E}_{\bm{b}^\nround \sim \bm{\rho}^\nround} \sum_{\nround=1}^\Nround \mu^\nround_n(\bm{b}^\nround)\\
    &\leq \max_{\bm{\rho} \in \Delta(\Pi)} \mathbb{E}_{\bm{b} \sim \bm{\rho}} \sum_{\nround=1}^\Nround \mu(\bm{b}) - \mathbb{E}_{\bm{b}^\nround \sim \bm{\rho}^\nround} \sum_{\nround=1}^\Nround \mu^\nround_n(\bm{b}^\nround)\\
    &= \max_{\bm{\rho} \in \Delta(\Pi)} \sum_{\nround=1}^\Nround \langle \bm{\rho} - \bm{\rho}^\nround, \bm{w}^\nround \rangle = \max_{\bm{\rho} \in \Delta(\Pi)} \sum_{\nround=1}^\Nround \langle  \bm{\rho}^\nround - \bm{\rho}, -\bm{w}^\nround \rangle
\end{align}
\rigel{Add condition that $\bm{\rho}^\nround$ is in $\Delta(\Pi)$ for all $\nround$} Where the inequality follows as any deterministic bid can be represented as an expectation over a deterministic policy. Notice that to keep consistent with the SSP literature, we negate both terms in the dot product to represent the problem as loss minimization rather than utility maximization. Defining $D(\bm{\rho} || \bm{\rho}') = \sum_{s \in \mathcal{S}, a \in \mathcal{A}(s)} \rho(s, a)\frac{\log \rho(s, a)}{\rho'(s, a)} - (\rho(s, a) - \rho'(s, a))$, we are now ready to state the $\textsc{O-REPS}$ algorithm formally and give the corresponding performance guarantees.

\begin{algorithm}[t]
	\KwIn{Valuation $\bm{v} \in [0, 1]^\Nitem$, Learning rate $\eta > 0$, Adaptive Adversarial Environment $\textsc{Env}^\nround: \mathcal{H}^\nround \to \mathcal{B}^{-\Nitem} \times \mathcal{B}$ where $\mathcal{H}^\nround$ denotes the set of all possible historical auction results $H^\nround$ up to round $\nround$ for all $\nround \in [\Nround]$.}
	\KwOut{The aggregate utility $\sum_{\nround=1}^\Nround \mu^\nround_n(\bm{b}^\nround)$ corresponding to a sequence of bid vectors $\bm{b}^{1},\ldots,\bm{b}^{\Nround}$ sampled according to $\textsc{O-REPS}$.}
	$\pi^0(s, a) \gets \frac{1}{|\mathcal{A}(s)|}$ for all $s \in \mathcal{S}, a \in \mathcal{A}(s)$. Let $\bm{\rho}^0$ be the corresponding state-action occupancy measure \;
        $H^0 \gets \emptyset$ \;
	\For{$\nround \in [\Nround]$:}{
            $(\bm{b}^{\nround}_-, \pi^\nround) \gets \textsc{Env}^{\nround-1}(H^{\nround-1})$ and $\bm{b}^{\nround} \sim \bm{\pi}^{\nround-1}$\;
            Observe $\bm{b}^{\nround}_-, \pi^\nround$ and receive reward $\mu^\nround_n(\bm{b}^\nround))$\;
            $\hat{w}_\nitem^\nround(B, B') \gets \frac{w_\nitem^\nround(B')}{\rho^{\nround-1}((\nitem-1, B), B')} \textbf{1}_{B = b^{\nround}_{\nitem-1},  B' = b^{\nround}_\nitem}$ for all $\nitem \in [\Nitem]$ and $B \geq B' \in \mathcal{B}$\;
            $\bm{\rho}^\nround \gets \text{argmin}_{\bm{\rho} \in \Delta(\Pi)} \eta\langle \bm{\rho}, -\hat{\bm{w}}^\nround\rangle + D(\bm{\rho} || \bm{\rho}^{\nround-1})$ and $\pi^\nround(s, a) \gets \frac{\rho^\nround(s, a)}{\sum_{b: \mathcal{A}(s)} \rho^\nround(s, b)}$\;
        }
        \textbf{Return} $\sum_{\nround=1}^\Nround \mu^\nround_n(\bm{b}^\nround)$
	\caption{\textsc{O-REPS}}
	\label{alg: O-REPS}
\end{algorithm}

\begin{theorem}
    Under bandit feedback, Algorithm~\ref{alg: O-REPS} achieves regret rate $\textsc{Regret}_\mathcal{B}^{\Nround} \lesssim O(\Nitem |\mathcal{B}| \sqrt{T \log |\mathcal{B}|})$ using $\eta = |\mathcal{B}|^{-1}\sqrt{\frac{\log |\mathcal{B}|}{T}}$ with respect to the discretized benchmark.
\end{theorem}

\begin{proof}
    At a high level, we want to bound the regret of Follow the (Entropy) Regularized leader. We first do this by upper bounding the regret by the regret of the unconstrained Be the (Entropy) Regularized leader (see Lemma 13 of \rigel{Cite Rakhlin 2009}. In particular, we define the corresponding unconstrained optimization problem:
    \begin{align}
        \tilde{\bm{\rho}}^{\nround+1} = \text{argmin}_{\bm{\rho} \in \mathcal{S} \times \mathcal{A} \to \mathbb{R}^+} \left( \eta \langle \bm{\rho}, -\hat{\bm{w}}^{\nround} \rangle + D(\bm{\rho} || \bm{\rho}^{\nround}) \right)
    \end{align}
    Where we are optimizing over all non-negative functions over state-action pairs rather than $\Delta(\Pi)$. Using the unbiasedness of our estimators $\hat{\bm{w}}^\nround$, we can replace $\bm{w}^\nround$ with $\hat{\bm{w}}^\nround$ in the definition of regret. Now, as it is shown in Lemma 13 of \rigel{Cite Rakhlin 2009}, we can upper bound the expected estimated regret as a function of the unconstrained optimizer $\tilde{\bm{\rho}}^{\nround+1}$ and the unregularized relative entropy with respect to the initial state-action occupancy measure $\bm{\rho}^1$. 
    \begin{align}
        \textsc{Regret}_\mathcal{B}^{\Nround} = \max_{\bm{\rho} \in \Delta(\Pi)} \mathbb{E}\left[\sum_{\nround=1}^\Nround \langle \bm{\rho}^{\nround} - \bm{\rho}, -\hat{\bm{w}}^{\nround} \rangle \right] \leq \max_{\bm{\rho} \in \Delta(\Pi)}\mathbb{E}\left[\sum_{\nround=1}^\Nround \langle \bm{\rho}^{\nround} - \tilde{\bm{\rho}}^{ \nround+1}, -\hat{\bm{w}}^{\nround} \rangle + \eta^{-1}D(\bm{\rho} || \bm{\rho}^{1}) \right]
    \end{align}
    Furthermore, the unconstrained optimizer can be solved with $\tilde{\bm{\rho}}^{ \nround+1} = \bm{\rho}^{\nround} \exp(\eta \hat{\bm{w}}^{\nround})$. Applying $\exp(x) \geq 1 + x$ for $x = \exp(\eta \hat{\bm{w}}^{\nround})$, we obtain $\tilde{\bm{\rho}}^{ \nround+1} \exp(\eta \hat{\bm{w}}^{\nround}) \geq \bm{\rho}^{\nround} + \bm{\rho}^{\nround}\eta \hat{\bm{w}}^{\nround}$. Plugging this back in:
    \begin{align}
        \textsc{Regret}_\mathcal{B}^{\Nround} &\leq \mathbb{E}\left[\sum_{\nround=1}^\Nround \langle \bm{\rho}^{\nround} - \bm{\rho}^{\nround} \exp(\eta \hat{\bm{w}}^{\nround}), -\hat{\bm{w}}^{\nround} \rangle + \eta^{-1}D(\bm{\rho} || \bm{\rho}^{1}) \right]\\
        &= \mathbb{E}\left[\eta \sum_{\nround=1}^\Nround \sum_{\nitem = 1}^\Nitem \sum_{B \geq B' \in \mathcal{B}} \rho^{\nround}((\nitem-1, B), B') \hat{w}^{\nround}_\nitem(B, B')^2 + \eta^{-1}D(\bm{\rho} || \bm{\rho}^{1}) \right] \label{eq: node diff}
    \end{align}
    Note that $\hat{w}^{\nround}_\nitem(B, B') = \frac{w_\nitem^\nround(B')}{\rho^{\nround-1}((\nitem-1, B), B')} \textbf{1}_{B = b^{\nround}_{\nitem-1},  B' = b^{\nround}_\nitem}$ for all $\nitem \in [\Nitem]$ and $B \geq B' \in \mathcal{B}$ by definition. Since $w^{\nround}_\nitem(B) \leq 1$ and $\textbf{1}_{B = b^{\nround}_{\nitem-1},  B' = b^{\nround}_\nitem} \leq 1$ we have $\hat{w}^{\nround}_\nitem(B, B') \leq \frac{1}{\rho^{\nround}_\nitem((\nitem-1, B), B')}$ and we continue the above chain of inequalities with:
    \begin{align}
        \textsc{Regret}_\mathcal{B}^{\Nround} &\leq \mathbb{E}\left[\eta \sum_{\nround=1}^\Nround \sum_{\nitem = 1}^\Nitem \sum_{B \geq B' \in \mathcal{B}} \rho^{\nround}((\nitem-1, B), B') \hat{w}^{\nround}_\nitem(B, B') \frac{1}{\rho^{\nround}((\nitem-1, B), B')}  + \eta^{-1}D(\bm{\rho} || \rho^{1}) \right] \label{eq: full info difference}\\
        &= \mathbb{E}\left[\eta \sum_{\nround=1}^\Nround \sum_{\nitem = 1}^\Nitem \sum_{B \geq B' \in \mathcal{B}} \hat{w}^{\nround}_\nitem(B, B')  + \eta^{-1}D(\bm{\rho} || \bm{\rho}^{1}) \right]
    \end{align}
    Note that $D(\bm{\rho} || \bm{\rho}') \leq \sum_{\nitem=1}^\Nitem \sum_{B \geq B' \in \mathcal{B}} \left[-\rho((\nitem-1, B), B') \log \rho((\nitem-1, B), B') \right]$. Now, we use the fact that $\sum_{B \geq B' \in \mathcal{B}} \rho((\nitem-1, B), B') = 1$ for any $\rho \in \Delta(\Pi)$ and $\nitem \in [\Nitem]$, we have that $\rho((\nitem, \cdot), \cdot)$ is a valid probability mass function and has entropy upper bounded by $O(\log |\mathcal{B}|)$ which is the entropy of the uniform distribution over $|\mathcal{B}|^2$ items. Hence, $D(\bm{\rho} || \bm{\rho}') \lesssim \Nitem \log |\mathcal{B}|$. Plugging this back in:
    \begin{align}
        \textsc{Regret}_\mathcal{B}^{\Nround} &\lesssim \mathbb{E}\left[\eta \sum_{\nround=1}^\Nround \sum_{\nitem = 1}^\Nitem \sum_{B \geq B' \in \mathcal{B}} \hat{w}^{\nround}_\nitem(B, B')  + \eta^{-1}\Nitem \log |\mathcal{B}| \right]\\
        &\leq \mathbb{E}\left[\eta \sum_{\nround=1}^\Nround \sum_{\nitem=1} \sum_{B \geq B' \in \mathcal{B}} \hat{w}^{\nround}_\nitem(B, B') + \eta^{-1}\Nitem \log |\mathcal{B}| \right]\\
        &\leq \eta \sum_{\nround=1}^\Nround \sum_{\nitem=1} \sum_{B \geq B' \in \mathcal{B}} w^{\nround}_\nitem(B, B') + \eta^{-1}\Nitem \log |\mathcal{B}|\\
        &= \eta \Nround \Nitem |\mathcal{B}|^2 + \eta^{-1}\Nitem \log |\mathcal{B}|
    \end{align}
    Where in the last equality, we used the unbiasedness property of $\hat{\bm{w}}^{\nround}$. Setting $\eta = |\mathcal{B}|^{-1}\sqrt{\frac{\log |\mathcal{B}|}{T}}$, we obtain $\textsc{Regret}_\mathcal{B}^{\Nround} \leq \Nitem |\mathcal{B}| \sqrt{\Nround \log |\mathcal{B}|}$. To handle the full information case, we note that we can improve line~\ref{eq: full info difference} by instead replacing $\hat{\bm{w}}^{\nround}$ with $\bm{w}^{\nround}$ in the previous line to obtain: 
    \begin{align}
        \sum_{\nround=1}^\Nround \sum_{\nitem=1}^\Nitem \sum_{B \geq B' \in \mathcal{B}} \rho^{\nround}((\nitem-1, B), B') \hat{w}^{\nround}_\nitem(B)^2 &= \sum_{\nround=1}^\Nround \sum_{\nitem=1}^\Nitem \sum_{B \geq B' \in \mathcal{B}} \rho^{\nround}((\nitem-1, B), B') w^{\nround}_\nitem((\nitem-1, B), B')^2\\
        &\leq \sum_{\nround=1}^\Nround \sum_{\nitem=1}^\Nitem \sum_{B \geq B' \in \mathcal{B}} \rho^{\nround}_\nitem(B, B') = \sum_{\nround=1}^\Nround \sum_{\nitem=1}^\Nitem 1 = \Nround \Nitem
    \end{align}
    Setting $\eta = \sqrt{\frac{ \log |\mathcal{B}|}{T}}$, we obtain in the full information setting $\textsc{Regret}_\mathcal{B}^{\Nround} \leq \Nitem \sqrt{\Nround \log |\mathcal{B}|}$.
\end{proof}

One may wonder how to efficiently update the state-action occupancy measures by computing the minimizer of $\eta\langle \bm{\rho}, -\hat{\bm{w}}^\nround\rangle + D(\bm{\rho} || \bm{\rho}^{\nround-1})$. While we relegate the details to \rigel{Cite O-REPS paper here}, the idea is to first solve the unconstrained entropy regularized minimizer with $\tilde{\bm{\rho}}^{ \nround+1} = \bm{\rho}^{\nround} \exp(\eta \hat{\bm{w}}^{\nround})$. We then project this unconstrained minimizer to $\Delta(\Pi)$ with:
\begin{align}
    \bm{\rho}^{\nround + 1} = \text{argmin}_{\bm{\rho} \in \Delta(\Pi)} D(\bm{\rho}||\tilde{\bm{\rho}}^{\nround + 1})
\end{align}
As shown in the analysis of \rigel{Cite O-REPS again}, this can be solved efficiently as an unconstrained convex optimization problem in $\mathbb{R}^{|\mathcal{B}|^2}$. Unfortunately, this projection prevents the straightforward generalization of our method to handle time varying valuation profiles.










\subsection{Reduction to State-Occupancy Measures}

One may realize that as the rewards associated with each edge are independent of its initial location, we may hope to further simplify the problem by assuming proportional conditional probability mass functions. That is, letting $B' \leq B$, we enforce that $\{\pi((\nitem, B), B")\}_{B" \leq B'} \propto \{\pi((\nitem, B'), B")\}_{B" \leq B'}$ for all $\nitem \in [\Nitem]$. We let $\Pi'$ denote the set of all $\Pi$ that fulfill this condition. This condition can be written succinctly as $\frac{\pi((\nitem, B), B')}{\pi((\nitem, B), B")} = \frac{\pi((\nitem, B'), B')}{\pi((\nitem, B'), B")}$ for all $\nitem$ and $B" \leq B' \leq B$. Letting $B_0$ denote the maximal possible bid, these conditions can be written even more concisely as $\bigcap_{\nitem=1}^\Nitem \bigcap_{B' \in \mathcal{B}} \bigcap_{B" \leq B'} \{\frac{\pi((\nitem, B_0), B')}{\pi((\nitem, B_0), B")} = \frac{\pi((\nitem, B'), B')}{\pi((\nitem, B'), B")}\}$. Using the fact that $\sum_{B' \leq B} \pi((\nitem, B), B') = 1$, we have:
\begin{align}
    &\frac{\pi((\nitem, B_0), B')}{\pi((\nitem, B_0), B")} = \frac{\pi((\nitem, B'), B')}{\pi((\nitem, B'), B")} \\
    &\leftrightarrow \pi((\nitem, B_0), B')\pi((\nitem, B'), B") = \pi((\nitem, B'), B')\pi((\nitem, B_0), B")\\
    &\to \sum_{B" \leq B'} \pi((\nitem, B_0), B')\pi((\nitem, B'), B") = \sum_{B" \leq B'} \pi((\nitem, B'), B')\pi((\nitem, B_0), B") \\
    &\leftrightarrow \pi((\nitem, B_0), B') = \pi((\nitem, B'), B') \sum_{B" \leq B'} \pi((\nitem, B_0), B")\\
    &\leftrightarrow \pi((\nitem, B'), B') = \frac{\pi((\nitem, B_0), B')}{\sum_{B" \leq B'} \pi((\nitem, B_0), B")}\\
    &\to \pi((\nitem, B), B') = \frac{\pi((\nitem, B_0), B')}{\sum_{B" \leq B} \pi((\nitem, B_0), B")}
\end{align}
Where in the last equality, we plugged in the value of $\pi((\nitem, B'), B')$ back into the proportionality constraint. At a high level, this allows us to describe the entire policy with only $\{\pi(S_0, B')\}_{B' \in \mathcal{B}} \cup \{\pi((\nitem, B_0), B')\}_{\nitem \in [\Nitem], B' \in \mathcal{B}}$, which is a set of size $O(\Nitem |\mathcal{B}|)$. Letting $\psi_\nitem(B) = \pi((\nitem-1, B_0), B)$, we see that we can rewrite $\pi((\nitem, B), B') = \frac{\psi(\nitem+1, B')}{\sum_{B" \leq B} \psi(\nitem+1, B")}$. Once again abusing notation, let $\bm{B} \sim \psi$ denote a bid vector sampled according to the policy generated from $\psi$. Furthermore, letting $\Psi: [\Nitem] \times \mathcal{B} \to [0, 1]$ such that $\sum_{B \in \mathcal{B}} \psi_\nitem(B) = 1$ denote the set of possible condensed policies $\psi$, we can recursively define the state occupancy measure $q^\psi(\nitem, B) = \prob_{\bm{B} \sim \psi}(s_\nitem = B_\nitem)$. With base case $q^\psi(1, B') = \psi(S_0, B')$, we have:
\begin{align}
    q^\psi(\nitem, B') = \sum_{B \geq B'} q^\psi(\nitem - 1, B)\pi((\nitem-1, B), B') = \psi_\nitem(B') \sum_{B \geq B'}  \frac{q^\psi(\nitem - 1, B)}{\sum_{B" \leq B} \psi_\nitem(B")}
\end{align}
By strong induction, it is straightforward to show that indeed $q^\psi(\nitem, B) = \prob_{\bm{B} \sim \psi}(s_\nitem = B_\nitem)$. We define $\mathcal{Q}$ to be the set of all possible state occupancy measures $\bm{q}$ such that there exists a $\psi \in \Psi$ that generates $\bm{q}$. More formally,
\begin{align}
    \mathcal{Q} \equiv \{\bm{q} \in [\Nitem] \times \mathcal{B} \to [0, 1]: \exists \psi \in \Psi \text{ such that } q^\psi(\nitem, B') = \psi_\nitem(B') \sum_{B \geq B'}  \frac{q^\psi(\nitem - 1, B)}{\sum_{B" \leq B} \psi_\nitem(B")} \}
\end{align}
As such, the loss can be rewritten as:
\begin{align}
    \mathbb{E}_{\bm{B} \sim \psi}\left[ \mu^{\nround}_n(\bm{B}) \right] = \mathbb{E}_{\bm{B} \sim \psi}\left[ \sum_{\nitem=1}^\Nitem w_\nitem^\nround(B_\nitem) \right] = \sum_{\nitem=1}^\Nitem \prob_{\bm{B} \sim \psi}(s_\nitem = B_\nitem) w_\nitem^\nround(B_\nitem)
\end{align}
Substituting in our definition $q^\psi(\nitem, B) = \prob_{\bm{B} \sim \psi}(s_\nitem = B_\nitem)$, we obtain:
\begin{align}
    \mathbb{E}_{\bm{B} \sim \psi}\left[ \mu^{\nround}_n(\bm{B}) \right] = \sum_{\nitem=1}^\Nitem \sum_{B \in \mathcal{B}} q^\nround(\nitem, B_\nitem) w_\nitem^\nround(B_\nitem) = \langle \bm{q}^\nround, \bm{w}^\nround\rangle
\end{align}
Where $\bm{w}^\nround = \{w_\nitem^\nround(B)\}_{\nitem \in [\Nitem], B \in \mathcal{B}}$ which is different as in the previous section where it was defined additionally over $B' \leq B$. Following a similar argument, we obtain:
\begin{align}
    \textsc{Regret}_\mathcal{B}^{\Nround} \leq \max_{\bm{q} \in \mathcal{Q}} \sum_{\nround=1}^\Nround \langle  \bm{q}^\nround - \bm{q}, -\bm{w}^\nround \rangle
\end{align}
We can now repeat the same algorithm and analysis as in above, except using state occupancy measures $\bm{q}$ as opposed to state-action occupancy measures $\bm{\rho}$. The primary benefit is that in the regret analysis, we can replace the summation over $(\nitem, B, B')$ with a summation over $(\nitem, B)$. This allows us to obtain sharper regret bounds of $O(\Nitem \sqrt{|\mathcal{B} \Nround \log |\mathcal{B}|})$. Note that in order to sample a monotone bid vector, we must first reconstruct the policy $\bm{\pi}$ corresponding to the recovered $\bm{q}$. To do this, we first compute $\bm{\psi}$ recursively corresponding to $\bm{q}$, and from this, compute $\bm{\pi}$ using $\pi((\nitem, B), B') = \frac{\psi(\nitem+1, B')}{\sum_{B" \leq B} \psi(\nitem+1, B")}$. To compute $\bm{\psi}$ from $\bm{q}$, notice that for any $\nitem$:
\begin{align}
    q(\nitem, B_0) = \psi_\nitem(B_0) \sum_{B \geq B_0} \frac{q(\nitem-1, B)}{\sum_{B" \leq B} \psi_\nitem(B")} = \psi_\nitem(B_0) \frac{q(\nitem-1, B_0)}{\sum_{B" \leq B_0} \psi_\nitem(B")} = \psi_\nitem(B_0) q(\nitem-1, B_0)
\end{align}
Where the last equality follows as $\sum_{B" \leq B_0} \psi_\nitem(B") = \sum_{B" \in \mathcal{B}} \psi_\nitem(B") = 1$. In the recursive case, we have:
\begin{align}
    q(\nitem, B') = \psi_\nitem(B')\sum_{B \geq B'} \frac{q(\nitem-1, B)}{\sum_{B" \leq B} \psi_\nitem(B")} = \psi_\nitem(B')\sum_{B \geq B'} \frac{q(\nitem-1, B)}{1 - \sum_{B" > B} \psi_\nitem(B")}
\end{align}
Hence, we can solve for $\psi_\nitem(B') = q(\nitem, B')\left(\sum_{B \geq B'} \frac{q(\nitem-1, B)}{1 - \sum_{B" > B} \psi_\nitem(B")}\right)^{-1} $ in terms of $\bm{q}$ (which is known) and $\psi_\nitem(B")$ for $B" \geq B'$, which is known from induction. A more serious complication is that the projection step is now more involved, requiring the proportional policy constraints. What we can do is first convert $\bm{\psi}^\nround$ into an equivalent $\bm{\pi}^\nround$ and $\bm{\rho}^\nround$. We then compute the maximizer w.r.t. all state-action occupancy measures in $\bm{\rho} \in \Delta(\Pi)$ with the additional constraint that $\bm{\rho}$ must have a corresponding $\bm{\pi}^\rho \in \Pi'$. As there are only polynomially many $O(\Nitem |\mathcal{B}|)$ constraints on $\pi$ required for proportionality, and that there are polynomially many $O(\Nitem |\mathcal{B}|)$ constraints required for some arbitrary $\rho \in \mathcal{S} \times \mathcal{A} \to [0, 1]$ to have a corresponding generating policy, then the projection step can still be solved efficiently. \rigel{Big claim, can't prove easily} Once we have computed $\bm{\rho}^{\nround+1}$, we translate this back into $\bm{q}^{\nround+1}$ and the corresponding $\bm{\psi}^{\nround + 1}$. 


\begin{algorithm}[t]
	\KwIn{Valuation $\bm{v} \in [0, 1]^\Nitem$, Learning rate $\eta > 0$, Adaptive Adversarial Environment $\textsc{Env}^\nround: \mathcal{H}^\nround \to \mathcal{B}^{-\Nitem} \times \mathcal{B}$ where $\mathcal{H}^\nround$ denotes the set of all possible historical auction results $H^\nround$ up to round $\nround$ for all $\nround \in [\Nround]$.}
	\KwOut{The aggregate utility $\sum_{\nround=1}^\Nround \mu(\bm{b}^\nround)$ corresponding to a sequence of bid vectors $\bm{b}^{1},\ldots,\bm{b}^{\Nround}$ sampled according to $\textsc{Node O-REPS}$.}
	$\psi^0(\nitem, B) \gets \frac{1}{|\mathcal{B}|}$ for all $\nitem \in [\Nitem], B \in \mathcal{B}$. Let $\bm{q}^0$ be the corresponding state-action occupancy measure \;
        $H^0 \gets \emptyset$ \;
	\For{$\nround \in [\Nround]$:}{
            $(\bm{b}^{\nround}_-, \pi^\nround) \gets \textsc{Env}^{\nround-1}(H^{\nround-1})$ and $\bm{b}^{\nround} \sim \bm{\pi}^{\nround-1}$\;
            Observe $\bm{b}^{\nround}_-, \pi^\nround$ and receive reward $\mu^\nround_n(\bm{b}^\nround)$\;
            $\hat{w}_\nitem^\nround(B') \gets \frac{w_\nitem^\nround(B')}{q^{\nround-1}(\nitem, B')} \textbf{1}_{B' = b^{\nround}_\nitem}$ for all $\nitem \in [\Nitem]$ and $B' \in \mathcal{B}$\;
            $\bm{q}^\nround \gets \text{argmin}_{\bm{q} \in \mathcal{Q}} \eta\langle \bm{q}, -\hat{\bm{w}}^\nround\rangle + D(\bm{q} || \bm{q}^{\nround-1})$ \;
            Recursively compute for all $\nitem$, $B$ in decreasing order $\psi^\nround_\nitem(B') \gets q^\nround(\nitem, B')\left(\sum_{B \geq B'} \frac{q^\nround(\nitem-1, B)}{1 - \sum_{B" > B} \psi^\nround_\nitem(B")}\right)^{-1}$ \;
            Compute $\pi^\nround((\nitem, B), B') \gets \frac{\psi^\nround(\nitem+1, B')}{\sum_{B" \leq B} \psi^\nround(\nitem+1, B")}$ for all $\nitem \in [\Nitem], B \geq B' \in \mathcal{B}$\;
        }
        \textbf{Return} $\sum_{\nround=1}^\Nround \mu(\bm{b}^\nround)$
	\caption{\textsc{Node O-REPS}}
	\label{alg: Node O-REPS}
\end{algorithm}

\begin{theorem}
    Under bandit feedback, Algorithm~\ref{alg: Node O-REPS} achieves regret rate $\textsc{Regret}_\mathcal{B}^{\Nround} \lesssim O(\Nitem \sqrt{|\mathcal{B}| \Nround \log |\mathcal{B}|})$ using $\eta = \sqrt{\frac{\log |\mathcal{B}|}{|\mathcal{B}|\Nround}}$.
\end{theorem}

\begin{proof}
    The justification follows that of Algorithm~\ref{alg: O-REPS} and only changes at line~\ref{eq: node diff}, where we instead have:
    \begin{align}
        \textsc{Regret}_\mathcal{B}^{\Nround} \leq \mathbb{E}\left[\eta \sum_{\nround=1}^\Nround \sum_{\nitem = 1}^\Nitem \sum_{B \in \mathcal{B}} q^{\nround}(\nitem, B') \hat{w}^{\nround}_\nitem(B)^2 + \eta^{-1}D(\bm{q} || \bm{q}^{1}) \right]
    \end{align}
    Similarly, bounding the entropy term $D(\bm{q} || \bm{q}^1)$ only requires taking a sum over $(\nitem, B)$ pairs rather than $(\nitem, B, B')$. Nonetheless, since the entropy bound is logarithmic in the distribution size, we achieve the same bound of $O(\Nitem |\mathcal{B}|)$. Plugging this back in, we obtain:
    \begin{align}
        \textsc{Regret}_\mathcal{B}^{\Nround} \leq \mathbb{E}\left[\eta \sum_{\nround=1}^\Nround \sum_{\nitem = 1}^\Nitem \sum_{B \in \mathcal{B}} q^{\nround}(\nitem, B') \hat{w}^{\nround}_\nitem(B)^2 + \eta^{-1}\Nitem |\mathcal{B}| \right]
    \end{align}
    We can upper bound the first term with $\eta \Nround \Nitem |\mathcal{B}|$ in the bandit setting and $\eta \Nround \Nitem $ in the full information setting using the same arguments as in $\textsc{O-REPS}$. Setting $\eta = \sqrt{\frac{\log |\mathcal{B}|}{|\mathcal{B}|\Nround}}$ yields the desired regret bound in the bandit setting. Similarly, setting $\eta = \sqrt{\frac{\log |\mathcal{B}|}{\Nround}}$ in the full information setting yields regret rate $O(\Nitem\sqrt{\Nround \log|\mathcal{B}|})$.
\end{proof}



\begin{theorem}
    Under bandit feedback, Algorithm~\ref{alg: Node O-REPS} achieves regret rate $\textsc{Regret}_\mathcal{B}^{\Nround} \lesssim O(\Nitem \sqrt{|\mathcal{B}| \Nround \log |\mathcal{B}|})$ using $\eta = \sqrt{\frac{\log |\mathcal{B}|}{|\mathcal{B}|\Nround}}$.
\end{theorem}


\section{Experiments}
% \haizhou{Follow the same way of introduction as we did in Section2.}
% \noindent In this section, we will introduce datasets and experimental setups that we used. Then we evaluate our method, other self-supervised methods, and supervised methods under different distribution shifts (\ie, concept shifts and covariate shifts) under common settings (\ie, transductive, inductive settings). It has to note that we focus on node-level tasks (\eg, node classification) in this work. As for graph-level tasks, we leave it as our future work and some simple experiments can be found in Appendix~\ref{app:graph_classification}. 
In this section, we first introduce the experimental setup including datasets, training, and evaluation protocol in Section~\ref{sec:dataset}~and~\ref{sec:unsupervised}. 
% Next, we present our experimental setup and conduct extensive experiments to evaluate our method in Section~\ref{sec:unsupervised}. 
We then perform an ablation study to demonstrate the effectiveness of each proposed component in Section~\ref{sec:ablation}. 
Additionally, we analyze the impact of important hyper-parameters in Section~\ref{sec:sensitivity}. 
Subsequently, we integrate our method with various encoding models, showcasing the model-agnostic nature of our recipe in Section~\ref{sec:other_models}. 
Finally, we provide some qualitative results such as feature visualization in Section~\ref{sec:vis}.
It is important to note that we focus on node-level tasks (\eg, node classification) in this work. As for graph-level tasks, we leave it as our future work, while some simple experiments are also provided in Appendix~\ref{app:graph_classification}.

\subsection{Datasets}\label{sec:dataset}
There exist some benchmarks for evaluating graph out-of-distribution generalization~\cite{good,ji2022drugood,gds}. 
Among them, GOOD~\cite{good} is the most representative and comprehensive benchmark that curates more diverse graph datasets with diverse tasks, including single/multi-task graph classification, graph regression, and node classification involving more distribution shifts (\ie, concept shifts and covariate shifts). Hence in this work, we follow the evaluation protocol proposed in \cite{good}. Furthermore, we validate the effectiveness of our method in the datasets (\ie, Amazon-Photo, Elliptic) that are used in EERM~\cite{eerm}. The statistics and detailed introduction to these datasets can be found in Table~\ref{tab:dataset} and Appendix~\ref{app:datasets}.

\begin{table*}[htp]
\caption{The descriptions of datasets. ``Domain-Level'' means splitting by graphs, ``Time-Aware'' denotes splitting according to chronological order.``Word'' and ``Degree'' represent splitting according to word diversity and node degree respectively. ``Language'' means splitting by user language, suggesting the prediction should not be impacted by the language the user use. ``University'' denotes splitting according to the domain university, implying that the prediction of webpages should be based on word contents and link connections rather than university features. ``Color'' means that nodes are split according to node differences in covariate shift and color-label correlations in concept shift.}
\label{tab:dataset}
\centering
\begin{tabular}{cccccccc}
\toprule
Datasets     & Network Type        & \#Nodes & \#Edges & \#Attributes &\#Classes& Train/Val/Test Split     & Metric   \\
% Cora         & Artificial Transformation & 2,703   &         &              &         &                      & Accuracy \\
Amazon-Photo\footnotemark
             & Co-purchasing network      & 7,650   & 119,081   & 755          & 10      & Domain-Level         & Accuracy \\
Elliptic\footnotemark  
             & Bitcoin transactions       & 203,769 & 234,355   & 165          & 2       & Time-Aware           & F1-Score \\
GOOD-Cora    & Scientific publications    & 19,793  & 126,842   & 8,710         & 70      & Word/Degree          & Accuracy \\
% GOOD-Arxiv   & arXiv papers               & 169,343 & 2,315,598 & 128          & 40      & Time/Degree          & Accuracy \\
GOOD-Twitch  & Gamer network              & 34,120  & 892,346   & 128          & 2       & Language             & ROC-AUC  \\
GOOD-CBAS    & A BA-house graph           & 700     & 3,962     & 4             & 4       & Color                & Accuracy \\
GOOD-WebKB   & Webpage network            & 617     & 1,138     & 1,703         & 5       & University           & Accuracy \\
\bottomrule
\end{tabular}
\end{table*}
\footnotetext[5]{This dataset is adopted from~\cite{yang2016revisiting}. \cite{eerm} constructs ten graphs with different environment id’s for each graph.} 
\footnotetext[6]{The original is available on \hyperlink{https://www.kaggle.com/ellipticco/elliptic-data-set}{https://www.kaggle.com/ellipticco/elliptic-data-set}}

\subsection{Unsupervised Representation Learning}\label{sec:unsupervised}
\subsubsection{Transductive Setting}~\label{sec:trans}
% \noindent\textbf{Baselines.}\quad We conduct experiments with 12 baselines which consist of three categories: supervised methods and self-supervised generative methods, self-supervised contrastive methods. Specifically, we compare with three supervised baselines: empirical risk minimization~(ERM)~\cite{erm}, invariant risk minimization (IRM)~\cite{irm}, and a recent proposed graph OOD method dubbed EERM~\cite{eerm}. We also compare various unsupervised node-level representation learning methods: three self-supervised generative methods including GAE~\cite{gae}, VGAE~\cite{gae}, GraphMAE~\cite{gmae} and seven self-supervised contrastive methods: DGI~\cite{dgi}, MVGRL~\cite{mvgrl}, GRACE~\cite{grace}, RoSA~\cite{rosa}, BGRL~\cite{bgrl}, COSTA~\cite{costa}, SwAV~\cite{swav}. The descriptions of these methods can be found in Appendix~\ref{app:baselines}.
In this subsection, we focus on validating our proposed algorithm under the transductive setting, where the test nodes will participate in message passing~\cite{gilmer2017neural} during training following~\cite{good}. 

\noindent\textbf{Baselines.} We conduct experiments with 12 baselines from three categories: (i)~supervised methods, including empirical risk minimization~(\textbf{ERM})~\cite{erm}, invariant risk minimization (\textbf{IRM})~\cite{irm}, and a recent proposed graph OOD method \textbf{EERM}~\cite{eerm}; (ii)~self-supervised generative methods including Graph Autoencoder (\textbf{GAE})~\cite{gae}, Variational Graph Autoencoder (\textbf{VGAE})~\cite{gae}, Self-Supervised Masked Graph Autoencoders (\textbf{GraphMAE})~\cite{gmae}; (iii)~self-supervised contrastive methods including Deep Graph Infomax (\textbf{DGI})~\cite{dgi}, Contrastive Multi-View Representation Learning on Graphs (\textbf{MVGRL})~\cite{mvgrl}, Deep Graph Contrastive Representation Learning (\textbf{GRACE})~\cite{grace}, A Robust Self-Aligned Framework for Node-Node Graph Contrastive Learning (\textbf{RoSA})~\cite{rosa}, Bootstrapped Representation Learning on Graphs (\textbf{BGRL})~\cite{bgrl}, Covariance-Preserving Feature Augmentation for Graph Contrastive Learning (\textbf{COSTA})~\cite{costa}, Unsupervised Learning of Visual Features by Contrasting Cluster Assignments (\textbf{SwAV})~\cite{swav}. The detailed descriptions of these baselines can be found in Appendix~\ref{app:baselines}.

\noindent\textbf{Experimental setup.} We use the same graph encoder across different datasets for a fair comparison following~\cite{good}. We use grid search to find other hyper-parameters (\eg, learning rate, epochs) for different methods. For all experiments, we select the best checkpoints for ID and OOD tests according to results on ID and OOD validation sets following~\cite{good}, respectively. Experimental details and hyper-parameter selections are provided in Appendix~\ref{app:hyper}. For evaluating unsupervised methods, a linear classifier will be built on the frozen trained encoder after finishing pre-training. The reported results are the mean performance with standard deviation after 10 runs following~\cite{good}.

\noindent\textbf{Analysis.}\quad Based on the experimental results listed in Table~\ref{tab:trans_concept} and \ref{tab:trans_covariate}, we can draw the following conclusions: firstly, we find strong self-supervised methods (\eg, GRACE, BGRL, COSTA) are more robust to distribution shifts (concept shift in Table~\ref{tab:trans_concept} and covariate shift in Table~\ref{tab:trans_covariate}) compared to supervised methods. For instance, on GOOD-CBAS and GOOD-WebKB datasets, GRACE surpasses the best supervised method by large margins (over 6\% absolute improvement). Interestingly, we find the methods designed for OOD generalization (\ie, IRM) and graph OOD generalization (\ie, EERM) do not attain superior performance than the standard ERM on most of the datasets. For example, EERM shows superior OOD performance compared to ERM in only one experiment, and IRM outperforms ERM in four out of ten experiments across the conducted evaluations. This phenomenon is also observed in \cite{good,ahuja2020empirical,rosenfeld2021risks}, showcasing the challenge of achieving invariant prediction in non-Euclidean graph settings. 

Furthermore, our method surpasses other SOTA self-supervised methods on the OOD test set of all datasets by a considerable margin while achieving comparable performance in the in-distribution test set. For instance, on small datasets such as GOOD-CBAS and GOOD-WebKB, our method outperforms GRACE\footnote{MARIO is built up on GRACE according to our recipe. So, we make a comparison with GRACE here.} by over 2\% absolute accuracy on the OOD test set. On larger datasets such as GOOD-Cora and GOOD-Twitch, our method still outperforms other methods which shows its superiority. For instance, under covariate shift, MARIO surpasses other methods by over 7\% absolute accuracy on the GOOD-Twitch OOD test set. These statistics prove the effectiveness of our design.


\begin{table*}[htp]
\caption{Experimental results of all methods under concept shift. The bold font means the top-1 performance and the underline represents the second performance across the unsupervised methods. 'ID' represents in-distribution test performance and 'OOD' means out-of-distribution test performance. (OOM: out-of-memory on a GPU with 24GB memory)}
\label{tab:trans_concept}
\centering
\scalebox{0.95}{
\begin{tabular}{l|cc|cc|cc|cc|cc}
\toprule
\toprule
\multirow{3}{*}{concept shift} & \multicolumn{4}{c|}{GOOD-Cora}                   & \multicolumn{2}{c|}{GOOD-CBAS} & \multicolumn{2}{c|}{GOOD-Twitch} & \multicolumn{2}{c}{GOOD-WebKB} \\
                           & \multicolumn{2}{c}{word} & \multicolumn{2}{c|}{degree}& \multicolumn{2}{c|}{color}    & \multicolumn{2}{c|}{language}   & \multicolumn{2}{c}{university} \\
                           & ID         & OOD         & ID          & OOD          & ID            & OOD           & ID             & OOD            & ID            & OOD            \\
\midrule
ERM                        & 66.38±0.45 & 64.44±0.18  & 68.60±0.40  & 60.76±0.34   & 89.79±1.39    & 83.43±1.19    & 80.80±1.00     & 56.92±0.92     & 62.67±1.53    & 26.33±1.09     \\
IRM                        & 66.42±0.41 & 64.29±0.31  & 68.57±0.35  & 61.45±0.24   & 89.64±1.21    & 82.29±1.14    & 78.87±1.04     & 59.30±1.79     & 62.67±1.10    & 26.88±1.42     \\
EERM                       & 65.10±0.44 & 62.45±0.19  & 66.95±0.44  & 56.58±0.25   & 79.07±2.12    & 64.50±1.01    & OOM            & OOM            & 62.50±2.01    & 28.07±3.23      \\
\midrule
% Random-Init                & 37.53±1.74 & 32.12±1.24  & 37.82±1.71  & 27.74±1.14   &               &               &                &                & 60.33±2.21    & 27.07±1.70     \\
GAE                        & 60.65±0.89 & 58.00±0.55  & 62.59±1.11  & 53.44±0.80   & 75.28±1.36    & 68.07±2.05    & 81.25±0.81     & 51.51±1.05     & 62.17±3.34    & 25.78±1.85     \\
VGAE                       & 63.19±0.53 & 60.35±0.47  & 61.65±0.66  & 54.28±0.28   & 76.50±0.50    & 59.07±0.56    & 80.46±0.53     & 55.56±4.53     & 62.50±2.38    & 24.40±2.57     \\
GraphMAE                   & \underline{66.44±0.46} & \underline{64.87±0.30}  & 67.95±0.46  & 59.41±0.39   & 89.14±0.89    & 82.93±0.93    & 80.05±0.64     & 59.38±1.49     & 61.83±3.37    & 29.27±2.15     \\
DGI                        & 63.33±0.56 & 60.71±0.49  & 65.93±1.02  & 55.83±0.53   & 91.22±1.47    & 85.00±1.66    & 80.05±0.87     & 59.16±1.88     & 61.83±2.83    & 28.63±1.92      \\
MVGRL                      & OOM        & OOM         & OOM         & OOM          & 88.57±1.15    & 76.50±1.17    & OOM            & OOM            & 62.00±3.79    & 28.26±4.20     \\
GRACE                      & 65.61±0.61 & 63.92±0.44  & \textbf{68.59±0.35}  & 60.15±0.45   & 92.00±1.39    & 88.64±0.67    & \textbf{83.43±0.63}     & \underline{60.45±1.46}     & 64.00±3.43    & \underline{34.86±3.43}  \\
RoSA                       & 64.06±0.67 & 62.44±0.39  & 67.07±0.65  & 57.68±0.44   & 90.78±2.27    & 85.93±2.14    & 82.39±0.42     & 57.45±2.16     & 64.17±4.10    & 32.20±2.15     \\
BGRL                       & 65.18±0.43 & 63.43±0.45  & 66.83±0.80  & 59.63±0.38   & 92.36±1.16    & 87.14±1.60    & 82.52±0.60     & 55.48±1.48     & 63.67±2.33    & 31.47±3.43     \\
COSTA                      & 65.05±0.80 & 62.37±0.45  & 66.76±0.87  & 55.73±0.36   & \underline{93.50±2.62}    & \underline{89.29±3.11}    & 83.15±0.30 & 55.03±3.22     & 61.66±2.58    & 32.39±2.13 \\
% ArCL                       &            &             & 67.64±0.57  & 59.71±0.44   &               &               &                &                & 65.00±3.94    & 35.41±1.97 \\      
SwAV                       & 62.22±0.53 & 59.79±0.53  & 64.65±0.94  & 55.06±0.39   & 89.00±0.79    & 81.72±0.66    & \underline{83.32±0.15}     & 59.69±1.97     & \underline{65.17±3.76}    & 29.36±2.01    \\
\midrule
MARIO                       & \textbf{67.11±0.46} & \textbf{65.28±0.34}  & \underline{68.46±0.40}  & \textbf{61.30±0.28}   & \textbf{94.36±1.21}    & \textbf{91.28±1.10}    & 82.31±0.54     & \textbf{63.33±1.72}     & \textbf{65.67±2.81}    & \textbf{37.15±2.37}     \\
\bottomrule
\end{tabular}}
\end{table*}

\begin{table*}[htp]
\caption{Experimental results of all methods under covariate shift. The bold font means the top-1 performance and the underline represents the second performance across the unsupervised methods. 'ID' represents in-distribution test performance and 'OOD' means out-of-distribution test performance. (OOM: out-of-memory on a GPU with 24GB memory)}
\label{tab:trans_covariate}
\centering
\scalebox{0.95}{
\begin{tabular}{l|cc|cc|cc|cc|cc}
\toprule
\toprule
\multirow{3}{*}{covariate shift} & \multicolumn{4}{c|}{GOOD-Cora}                                   & \multicolumn{2}{c|}{GOOD-CBAS} & \multicolumn{2}{c|}{GOOD-Twitch} & \multicolumn{2}{c}{GOOD-WebKB} \\
                           & \multicolumn{2}{c}{word} & \multicolumn{2}{c|}{degree}& \multicolumn{2}{c|}{color}    & \multicolumn{2}{c|}{language}   & \multicolumn{2}{c}{university} \\
                           & ID         & OOD         & ID          & OOD          & ID            & OOD           & ID             & OOD            & ID            & OOD            \\
\midrule
ERM                        & 70.50±0.41 & 64.69±0.33  & 72.46±0.49  & 55.53±0.50   & 92.00±3.08    & 77.57±1.29    & 70.98±0.41     & 49.35±5.09     & 39.34±1.79    & 14.52±3.14   \\
IRM                        & 70.48±0.26 & 64.53±0.57  & 71.98±0.34  & 53.72±0.46   & 90.86±2.41    & 78.86±1.67    & 69.81±0.95     & 49.11±2.82     & 38.52±3.30    & 13.97±2.80     \\
EERM                       & OOM        & OOM         & OOM         & OOM          & 65.00±2.57    & 57.43±3.60    & OOM            & OOM            & 46.07±4.55    & 27.40±7.65     \\
\midrule
GAE                        & 56.63±0.79 & 48.93±0.93  & 66.30±0.88  & 34.01±0.87   & 73.00±2.16    & 60.86±3.01    & 67.24±1.23     & 47.65±2.49     & 45.08±6.32    & 28.02±6.29    \\
VGAE                       & 62.02±0.66 & 54.12±0.86  & 69.41±0.57  & 44.20±1.29   & 62.29±2.04    & 63.29±1.11    & 66.99±1.43     & \underline{50.48±4.58}     & 48.85±4.68    & 20.87±6.69     \\
GraphMAE                   & 68.14±0.43 & 64.00±0.33  & \textbf{73.36±0.56}  & 53.75±0.55   & 67.28±3.03    & 67.28±1.49    & 68.84±1.20     & 48.02±2.79     & 48.03±4.34    & 30.00±8.09     \\
DGI                        & 60.85±0.75 & 57.03±0.67  & 68.97±0.41  & 41.75±0.88   & 69.57±4.09    & 59.71±3.43    & 68.43±1.05     & 44.83±1.61     & 48.52±5.04    & 21.11±7.50     \\
MVGRL                      & OOM        & OOM         & OOM         & OOM          & 65.00±1.94    & 64.15±0.77    & OOM            & OOM           & \textbf{54.10±5.39}    & 16.59±6.51     \\
GRACE                      & \underline{68.77±0.33} & \underline{64.21±0.41}  & 72.69±0.34  & \underline{56.10±0.63}   & \underline{93.57±1.83}    & \underline{89.29±3.40}    & \underline{71.12±0.87} & 46.21±1.54 & 49.67±5.82    & 28.10±4.68    \\
RoSA                       & 68.19±0.56 & 62.48±0.61  & 71.04±0.62  & 52.72±0.79   & 84.71±4.14    &79.14±3.51     & 70.58±0.36     & 45.83±1.72     & 52.30±4.24    & \underline{34.24±7.92}     \\
BGRL                       & 67.23±0.43 & 61.33±0.36  & 72.11±0.39  & 49.15±0.73   & 89.00±2.56    & 79.86±3.29    & \textbf{71.43±0.53}     & 43.86±0.94     & 51.80±5.55    & 30.32±7.61    \\
COSTA                      & 65.28±0.60 & 60.33±0.53  & 70.65±0.62  & 54.03±0.28   & 92.29±1.59    & 82.71±2.74    & 69.29±1.37     & 49.07±2.13     & 50.49±3.01    & 29.84±4.75   \\
SwAV                       & 63.29±1.01 & 56.98±0.94  & 70.27±0.73  & 43.00±0.52   & 89.57±1.12    & 81.43±1.69    & 69.19±0.93     & 49.37±2.96     & 49.84±4.82    & 30.55±6.72   \\
\midrule
MARIO                       & \textbf{69.99±0.54} & \textbf{65.06±0.34}  & \underline{72.73±0.43}  & \textbf{57.73±0.45}  & \textbf{94.57±2.46}    & \textbf{91.00±2.48}     & 68.31±0.78 & \textbf{57.37±1.37}     & \underline{53.94±3.23}    & \textbf{35.24±4.98}   \\
\bottomrule
\end{tabular}}

\end{table*}

\subsubsection{Inductive Setting}
In this subsection, we conduct experiments under the inductive settings, where the test nodes are kept unseen during training. This setting is more suitable for domain generalization.
% But we think it is more convincing that conduct experiments under inductive settings which means test nodes are unseen during training. This setting is more appropriate for domain generalization.

\noindent\textbf{Baselines:} For GOOD-WebKB and GOOD-CBAS datasets, we adopt ERM, IRM, GraphMAE, and GRACE as our baselines. And for Amazon-Photo and Elliptic datasets, we select ERM, EERM, and GRACE as our baselines.

\noindent\textbf{Experimental setup:} For GOOD-WebKB and GOOD-CBAS datasets, we use the same model configuration in Section~\ref{sec:trans}.
% Besides, we add experiments on Amazon-Photo dataset~\cite{yang2016revisiting} and Elliptic~\cite{elliptic} dataset in this subsection. 
For Amazon-Photo dataset~\cite{yang2016revisiting} and Elliptic~\cite{elliptic} dataset, they consist of many snapshots (training data and testing data use different snapshots) which are naturally inductive. For Amazon-Photo dataset, we use 2-layer GCN~\cite{gcn} as the encoder and for elliptic dataset, we use 5-layer GraphSAGE~\cite{sage} as encoder following~\cite{eerm}.

% Figure environment removed

\noindent\textbf{Analysis:}
According to Figure~\ref{fig:amazon},\ref{fig:elliptic},\ref{fig:ind_con},\ref{fig:ind_cov}, we can draw following conclusions:
firstly, based on Figure~\ref{fig:amazon}, it is evident that our method outperforms other representative supervised and self-supervised methods on all test graphs (T1$\sim$T8). This superiority is reflected in the larger median value of our method compared to others. For instance, MARIO achieves over a 3\% absolute improvement compared to ERM in terms of the mean value of eight median values. Additionally, our method demonstrates higher stability across different random initializations, as indicated by the closer proximity of the first and third quartile values to the median value~(\eg, the difference of first and third quartile values of ERM, EERM, GRACE and MARIO are 4.2, 3.3, 6.7 and 1.0 on T8 respectively which indicates MARIO is much more stable than other methods). Furthermore, our method exhibits consistent performance across different graphs (\eg, The standard deviation of median values on T1$\sim$T8 for ERM, EERM, GRACE, and MARIO are 0.4, 1.1, 1.2, and 0.3, respectively.), indicating its robustness to environmental variations and its ability to extract invariant features: $g(G^e) \approx g(G^{e'})$ for all $e, e' \in \mathcal{E}^\text{train}$. In summary, our method showcases enhanced OOD generalization capabilities.
% $g(G^e)g(G^e^\prime)$ where $any e, e^\prime in \mathcal{E}^{train}$

Secondly, from the results presented in Figure~\ref{fig:elliptic}, we can observe that our method averagely harvests 10.9\% absolute improvement over GRACE and 12.5\% absolute improvement over EERM in terms of F1 scores on Elliptic dataset. This demonstrates the effectiveness of our method in handling distribution shifts and improving performance compared to existing approaches. It is worth noting that GRACE's performance worsens over time, indicating its inability to handle distribution shifts effectively. In contrast, our method consistently achieves better F1 scores, except for T9, which is caused by the dark market shutdown occurred after T7~\cite{elliptic}. The emergence of such an event introduces significant variations in data distributions, which subsequently results in performance degradation for all methods. Indeed, this event serves as an unpredictable external factor that introduces significant challenges for models trained on limited training data. The results indicate that the performance heavily depends on available training data. Nonetheless, our approach outperforms other methods even in such an extreme case. This highlights the effectiveness of our method in addressing distribution shifts and improving generalization performance.

Finally, based on the observations from Figure~\ref{fig:ind_con} and Figure~\ref{fig:ind_cov} MARIO demonstrates the best performances on both ID and OOD test sets for GOOD-WebKB and GOOD-CBAS datasets, under both concept shift and covariate shift. Notably, MARIO outperforms other methods by more than 3\% and 10\% absolute improvement on GOOD-WebKB and GOOD-CBAS, respectively, under covariate shift. We can draw similar conclusions as discussed in Section~\ref{sec:trans}. Even under the inductive setting, our method continues to demonstrate excellent OOD generalization capabilities and achieves comparable or even improved in-distribution test performance. These statistical results further validate the effectiveness of our method in handling distribution shifts and enhancing generalization performance.

Overall, the observations we have made provide strong evidence of the great capacity of our method for handling distribution shifts, validating its effectiveness and potential for real-world applications.



% Figure environment removed

% Figure environment removed


% Figure environment removed


\subsection{Ablation Studies}\label{sec:ablation}
\noindent Table~\ref{tab:aba} provides a detailed analysis of the effect of each component according to our proposed recipe for improving OOD generalization in graph contrastive learning. Let's examine the different variants of our method and their impact on performance.
Specifically, MARIO~(w/o ad) represents MARIO without  adversarial augmentation. MARIO~(w/o cmi) denotes we only maximize the mutual information between positive pairs without considering conditional mutual information. MARIO~(w/o cmi, ad) means a vanilla graph contrastive method that is similar to GRACE. 

From Table~\ref{tab:aba}, we can find MARIO~(w/o cmi) lags far behind MARIO on OOD test set which demonstrates appropriately minimizing the redundant information (\ie, conditional mutual information) is essential to improve OOD generalization of GCL methods. And adversarial augmentation can also boost OOD generalization because it can approximately serve as a supermum operator to learn more invariant features  discussed in Section~\ref{sec:aug}. Based on the analysis of these variants, it is evident that the proposed improvements on data augmentation and contrastive loss in the recipe are both effective in enhancing graph OOD generalization. Each component contributes to the overall performance improvement, and their combination leads to a stronger self-supervised graph learner in terms of graph OOD generalization. 

In short, the findings from Table~\ref{tab:aba} support the rationale behind your proposed recipe and provide empirical evidence of the effectiveness of each proposed component. By incorporating these enhancements, our method achieves superior performance in handling distribution shifts and improving graph OOD generalization in graph contrastive learning.
\begin{table*}[htp]
\caption{Ablation studies for MARIO by masking each component.}
\label{tab:aba}
\centering
\scalebox{0.9}{
\begin{tabular}{l|cc|cc|cc|cc|cc}
\toprule
\toprule
\multirow{3}{*}{concept shift} & \multicolumn{4}{c|}{GOOD-Cora}                       & \multicolumn{2}{c|}{GOOD-CBAS} & \multicolumn{2}{c|}{GOOD-Twitch} & \multicolumn{2}{c}{GOOD-WebKB} \\
                           & \multicolumn{2}{c}{word} & \multicolumn{2}{c|}{degree}& \multicolumn{2}{c|}{color}    & \multicolumn{2}{c|}{language}   & \multicolumn{2}{c}{university} \\
                           & ID         & OOD         & ID          & OOD          & ID            & OOD           & ID             & OOD            & ID            & OOD            \\
\midrule
MARIO                      & \textbf{67.11±0.46} & \textbf{65.28±0.34}  & \textbf{68.46±0.40}  & \textbf{61.30±0.28}      & \textbf{94.36±1.21}  & \textbf{91.28±1.10}    & 82.31±0.54     & \textbf{63.33±1.72}     & \textbf{65.67±2.81}    & \textbf{37.15±2.37}     \\
MARIO(w/o ad)              & 66.23±0.53 & 64.02±0.18  & 67.88±0.38  & 60.46±0.29   & 93.21±1.25    & 90.29±0.91    & 82.42±0.73     & 60.50±1.02     & 64.83±2.83    & 36.51±3.25    \\
MARIO(w/o cmi)             & 65.32±0.60 & 63.51±0.32  & 68.14±0.32  & 61.19±0.34   & 94.15±1.23    & 90.57±1.96    & \textbf{82.51±0.56}     & 61.41±2.63     & 64.50±4.35    & 35.78±2.53     \\
MARIO(w/o cmi, ad)         & 64.67±0.55 & 63.11±0.32  & 67.95±0.65  & 60.01±0.57   & 93.36±1.66    & 89.64±1.73    & 81.90±0.75     & 60.12±1.60     & 64.17±3.67    & 34.13±2.38     \\
\bottomrule
\end{tabular}}
\end{table*}
% & 65.32±0.60 & 63.51±0.32 exchange 64.67±0.55 & 63.11±0.32
% 68.14±0.32       id ood test: 60.95±0.43       ood ood test: 61.19±0.34


\subsection{Sensitivity Analysis}\label{sec:sensitivity}
\noindent In this subsection, we will analyze some important hyper-parameters of our method. We conduct sensitivity analysis on GOOD-WebKB dataset with concept shift, we chose two sensitive hyper-parameters (\ie, the coefficient $\gamma$ of condition mutual information in Equation~\ref{equ:cmi} and the number of prototypes $|C|$ in Equation~\ref{equ:pq}). The coefficient of CMI range in $[0.001, 0.01, 0.1, 0.5, 1]$ and the number of prototypes $|C|$ ranges in $[10, 50, 100, 200, 300]$. From Figure~\ref{fig:sensitivity}, we can observe that $\gamma$ reaches 0.1 and $|C|$ reaches 100 or 200 can achieve the best OOD test accuracy. Both higher and lower values of $\gamma$ result in suboptimal performance. This finding aligns with previous research such as DIB~\cite{dib}, indicating that an appropriate compression level is crucial for achieving optimal performance. Extremely high or low compression values are not ideal. 

Regarding the number of prototypes $|C|$, based on the results shown in Figure~\ref{fig:sensitivity}, it is found that setting $|C|=100$ leads to the best performance in terms of OOD test accuracy. This choice provides a moderate number of pseudo labels, which is beneficial for the learning process. 

Based on the sensitivity analysis, we determined that setting $\gamma=0.1$ and $|C|=100$ on most datasets. These hyperparameter values strike a balance between compression level and the number of prototypes, resulting in improved graph OOD generalization.
% Figure environment removed


\subsection{Integrated with Other Models}\label{sec:other_models}
% Figure environment removed

\begin{table}[htp]
\caption{Results of different learning approaches with different encoding models (\ie, GCN, GraphSAGE, GAT).}
\label{tab:others}
\centering
\scalebox{0.9}{
\begin{tabular}{cc|cc|cc}
\toprule
\toprule
\multirow{3}{*}{Model}& \multirow{3}{*}{Method} & \multicolumn{2}{c|}{GOOD-CBAS} & \multicolumn{2}{c}{GOOD-WebKB} \\
                & & \multicolumn{2}{c|}{color}    & \multicolumn{2}{c}{university} \\
                &   & ID          & OOD         & ID          & OOD            \\
\midrule
\multirow{3}{*}{GCN} 
&ERM               & 89.79±1.39 & 83.43±1.19  &  62.67±1.53 & 26.33±1.09         \\
&GRACE             & 92.00±1.39 & 88.64±0.67  &  64.00±3.43 & 34.86±3.43        \\
&MARIO             & 94.36±1.21 & 91.28±1.10  &  65.67±2.81 & 37.15±2.37        \\ \bottomrule
\multirow{3}{*}{SAGE} 
&ERM               & 95.07±1.51 & 75.14±1.19  & 73.67±2.08  & 46.33±3.42       \\
&GRACE             & 95.29±1.11 & 74.43±2.36  & 70.50±5.06  & 49.54±3.83        \\
&MARIO             & 96.00±1.07 & 76.29±3.01  & 71.00±3.82  & 51.74±4.63        \\ \bottomrule
\multirow{3}{*}{GAT} 
&ERM               & 78.64±3.63 & 72.93±2.64  & 61.33±3.71  & 28.99±2.63        \\
&GRACE             & 84.57±1.79 & 78.36±1.60  & 59.50±2.36  & 35.78±3.26        \\
&MARIO             & 84.93±1.95 & 80.43±1.89  & 62.17±4.78  & 38.17±3.10        \\
\bottomrule
\end{tabular}}
\end{table}



\noindent In the subsection, we demonstrate the model-agnostic nature of the recipe by integrating it with various graph neural network (GNN) models, including GCN, GraphSAGE, and GAT.

From Table~\ref{tab:others}, it can be observed that regardless of the specific GNN model used as the encoder, our method consistently achieves the best performance on the OOD test set. This indicates the effectiveness and robustness of our method across different GNN models.
By achieving superior performance across different GNN models, MARIO demonstrates its versatility and ability to improve the OOD generalization of various graph neural models. This highlights the broad applicability and effectiveness of our recipe in enhancing the performance of different GNN encoders.

Furthermore, we integrate our recipe with other GCL methods in Appendix~\ref{app:other_methods}. The results demonstrate our recipe can boost the OOD generalization ability of various GCL methods which means our recipe can serve as a plug-in for many current classical GCL methods.

% Figure environment removed

\subsection{Visualization}\label{sec:vis}
\subsubsection{Metric Score Curves}
We present metric score curves for ERM and MARIO, including training, ID validation, ID testing, OOD validation, and OOD testing accuracy, in Figure~\ref{fig:curve2}. Notably, MARIO demonstrates superior convergence with approximately 10\% absolute improvement on the OOD test set compared to ERM. Furthermore, MARIO effectively narrows the performance gap between in-distribution and out-of-distribution performance, showcasing its efficacy in enhancing OOD generalization for graph data. More metric score curves can be found in Appendix~\ref{app:curves}.


\subsubsection{Feature Visualization}
In order to assess the quality of learned embeddings, we adopt t-SNE~\cite{tsne} to visualize the node embedding on GOOD-Cora dataset (concept shift in word domain) using random-init of GCN, EERM, GRACE, and MARIO, where different classes have different colors in Figure~\ref{fig:vis}. For clarity, we select eight classes with the largest number of nodes to enhance the informativeness and interpretability of the visualization. We can observe that the 2D projection of node embeddings learned by MARIO has a better separation of clusters, which indicates the model can help learn representative features for downstream tasks. It has to note that we depict both ID nodes and OOD nodes in the same figure. 

Besides, we also separately visualize ID nodes and OOD nodes in the different figures in the Appendix~\ref{app:feature}. And we can find MARIO performs a clearer separation of clusters whether on ID nodes or OOD nodes compared to other methods.



\section{Extension: Time Varying Valuations}

\label{sec: time varying}
{\color{black}
We extend Algorithms \ref{alg: Decoupled Exponential Weights} and \ref{alg: Decoupled Exponential Weights - Path Kernels} to the time varying valuations setting. In particular, we assume that the valuations $\bm{v}$ are no longer fixed, and instead, in every round $t$, $\bm{v}^\nround$ is independently  drawn from some known distribution $F_{\bm{v}}$ with discrete, finite support $\mathcal{V}$. 
This contextual setting requires a  stronger  benchmark oracle in comparison to our original setup with a fixed valuation. The new benchmark oracle, which we will formalize shortly, possesses knowledge of the hindsight optimal bid vector for each context. That is, under this benchmark, we have the optimal mapping from any context (valuation vector) to an action (bid vector). 
Consequently, our current definitions of $\textsc{Regret}$ and $\textsc{Regret}_{\mathcal{B}}$ need to be updated to accommodate these contextual factors:
\begin{align}
\tag{Continuous Contextual Regret}
    \textsc{Regret}(F_{\bm{v}}) = \max_{\bm{b}: \mathcal{V} \to [0, 1]^{+\Nitem}} \sum_{\nround=1}^\Nround \mathbb{E}_{\bm{v} \sim F_{\bm{v}}}[\mu^\nround_n(\bm{b}(\bm{v}); \bm{v})] - \mathbb{E}\left[\sum_{\nround=1}^\Nround \mu^\nround_n(\bm{b}^\nround; \bm{v}^\nround)\right]\,.
\end{align}
Here, $\mu^\nround_n(\bm{b}; \bm{v})$ denotes the utility of bidder $n$ by submitting bid vector $\bm{b}$ with valuations $\bm{v}$ at round $t$ where the competing bids are $\bm{b}_{-}^{t}$. Observe that in the benchmark of $ \textsc{Regret}(F_{\bm{v}})$, i.e., $\max_{\bm{b}: \mathcal{V} \to [0, 1]^{+\Nitem}} \sum_{\nround=1}^\Nround \mathbb{E}_{\bm{v} \sim F_{\bm{v}}}[\mu^\nround_n(\bm{b}(\bm{v}); \bm{v})]$, we abuse notation and define valuation-to-bid vector mapping $\bm{b}: \mathcal{V} \to [0, 1]^{+M}$.  
We have an equivalent definition of discretized contextual regret:
\begin{align}
\tag{Discretized Contextual Regret}
    \textsc{Regret}_\mathcal{B}(F_{\bm{v}}) = \max_{\bm{b}:\mathcal {V}\to
    \mathcal{B}^{+\Nitem}} \sum_{\nround=1}^\Nround \mathbb{E}_{\bm{v} \sim F_{\bm{v}}}[\mu^\nround_n(\bm{b}(\bm{v}); \bm{v})] - \mathbb{E}\left[\sum_{\nround=1}^\Nround \mu^\nround_n(\bm{b}^\nround; \bm{v}^\nround)\right]\,.
\end{align}
An agent's goal is to minimize their contextual regret with respect to their valuation distribution $F_{\bm{v}}$. Using naive contextual bandit algorithms would lead to a large regret, as the regret of these algorithms  scales  with the square root of the number of contexts. However, we make an observation that we have \emph{complete cross-learning} over these contexts as in \cite{ContextBanditsCrossLearning2019}. That is,
 whenever the agents chooses bid $\bm b$ in round $t$ while having context/value $\bm v$ and receives reward $\mu_{n}^{t}({\bm b}; \mathbf v)$, they also learn the value of $\mu_{n}^{t}({\bm b}; \mathbf v')$ for any  contexts/values $\bm v'$
 This is because of the functional form of $\mu_n(\bm{b}; {\bm v}) = \sum_{\nitem=1}^{x_n(\bm{b}, {\bm b}_{-})} (v_{n, \nitem} - b_{n, \nitem})$.

As such, we borrow  from the results described in \cite{ContextBanditsCrossLearning2019}; specifically those explaining the cross-learning-across-contexts generalizations of the $\textsc{EXP3}$ algorithm in the stochastic contexts (valuations) and adversarial rewards setting (adversarial competing bids). 
We assume that the agent has access to their valuation distribution. Moreover,  as stated earlier, we assume that the support of this valuation distribution is finite; i.e., $|\mathcal{V}| < \infty$. This scenario occurs often in practice where bidders' valuations depend naturally on some natural events. For example, investors may prescribe a `low' or `high' value to certain assets depending on various market indices. 

We generalize the $\textsc{EXP3-CL}$ algorithm described in \cite{ContextBanditsCrossLearning2019} to our PAB setting, specifically Algorithm~\ref{alg: Decoupled Exponential Weights - Path Kernels}, and  achieve exactly the same regret rates as our non-contextual variants, albeit requiring an additional $O(|\mathcal{V}|)$ factor of memory and computation.

In order to make the generalization more clear, at a high level, the $\textsc{EXP3-CL}$ algorithm on a set of $K$ arms and $C$ contexts with full cross-learning constructs a reward estimator $\hat{r}(k; c) = \frac{r(k; c)}{\sum_{c}\prob(c)\prob(k^t = k | c^t = c)}\textbf{1}_{k^t = k}$ for each arm $k$ and context $c$ pair. Here, the term $\sum_{c}\prob(c)\prob(k^t = k | c^t = c)$ is the expected probability that arm $k^t = k$ was selected under context $c^t = c$, where in the summation we take expectation over the stochasticity of contexts $c$. This estimator mirrors that of standard $\textsc{EXP3}$ using the IPW estimator, except that the IPW is averaged over the context distribution. 

To apply this to our setting, we wish to mimic the behavior of the $\textsc{EXP3-CL}$ algorithm with our decoupled exponential weights algorithm. This can be done by running the $\textsc{EXP3-CL}$ estimator on all of the nodes $b \in \mathcal{B}$ within each layer $m \in [M]$. In particular, we use the following estimator  $\widehat{w}_m^t(b; \bm{v}) = 1 - \frac{1 - w_m^t(b; \bm{v})}{Q_m^t(b)} \textbf{1}_{b_m^t = b}$, where the  normalizer $Q_m^t(b) = \sum_{\bm{v} \in \mathcal{V}} \prob(\bm{v}^t = \bm{v}) q_m^t(b; \bm{v})$ in this estimator 
is the expected probability of selecting bid $b$, where the expectation is taken with respect to all valuation vectors $\bm{v} \in \mathcal{V}$. This procedure, formally described in Algorithm~\ref{alg: Decoupled Exponential Weights - Time Varying Known Finite} and analyzed in the online appendix (arXiv:2307.15193v3), yields the following regret upper bound:

\begin{theorem}[Time Varying Valuations - Decoupled Exponential Weights] \label{thm: time varying known finite}
    Under bandit feedback (resp. full information feedback), Algorithm~\ref{alg: Decoupled Exponential Weights - Time Varying Known Finite}, with appropriately chosen $\eta$, achieves contextual continuous regret $\textsc{Regret}(F_{\bm{v}})$ of order $O(\Nitem^\frac{4}{3} \Nround^{\frac{2}{3}} \sqrt{\log \Nround})$ (resp. $O(\Nitem^\frac{3}{2} \sqrt{\Nround \log \Nround})$ with total time time and space complexity polynomial in $M$, $|\mathcal{B}|$, $|\mathcal{V}|$, and $\Nround$.
\end{theorem}



\begin{algorithm}[t]
\footnotesize
	\KwIn{Learning rate $0 < \eta < \frac{1}{M}$, Valuation Distribution $F_{\bm{v}}$}
	\KwOut{The aggregate utility $\sum_{\nround=1}^\Nround \mu_n^\nround(\bm{b}^{\nround}; \bm{v}^\nround)$}
	$\widehat{W}_\nitem^0(b; \bm{v}) \gets 0$ for all $\nitem \in [\Nitem], b \in \mathcal{B}, \bm{v} \in \mathcal{V}$ such that $b \leq v_m$; else $\widehat{W}_\nitem^0(b; \bm{v}) \gets -\infty$.\; 
	\For{$\nround \in [1,\ldots,\Nround]$:}{
            \textbf{Observe Valuation Vector $\bm{v}^t \sim F_{\bm{v}}$}\;
            $b_{0}^t \gets \max \mathcal B$, and $\widehat{S}_{M+1}^t (\min \mathcal{B}; \bm{v}^t)=1$ for any $t\in[T]$\;
            \textbf{Recursively Computing Exponentially Weighted Partial Utilities $\bm{S}^t$}\;
            \textbf{for} $m \in [M,\ldots,1], b \in \mathcal{B}: \widehat{S}^t_\nitem(b; \bm{v}^t) \gets \exp(\eta \widehat{W}_\nitem^\nround(b; \bm{v}^t)) \sum_{b' \leq b} \widehat{S}_{\nitem + 1}^\nround(b'; \bm{v}^t)$ \hspace{0mm} $\backslash \backslash$ $\textsc{Compute}-\widehat{S}_\nitem$\;
        \textbf{Determining the Bid Vector $\bm{b}^\nround$ Recursively}\;
        \textbf{for} $m \in [1,\ldots,M], b \leq b_{m-1}^t: b_\nitem^\nround \gets b$ with probability $\frac{\widehat{S}^t_\nitem(b; \bm{v}^t)}{\sum_{b' \leq b_{\nitem-1}^t} \widehat{S}^t_{\nitem}(b'; \bm{v}^t)}; $ \hspace{1mm} $\backslash \backslash$ $\textsc{Sample}-\bm{b}$\;
        Observe $\bm{b}^{\nround}_-$ and receive reward $\mu_n^\nround(\bm{b}^{\nround}; \bm{v}^t)$\;
        $Q_m^t(b) \gets 0$ for all $m \in [M], b \in \mathcal{B}$\;
        \For{$\bm{v} \in \mathcal{V}$:}{
            \textbf{Recursively Computing Probability Measure $\bm{q}$ Under $\bm{v}\in {\mathcal V}$}\;
            $\widehat{S}^t_{M+1}(b; \bm{v}) \gets 1$ for all $m \in [M], b \in \mathcal{B}$\;
            \textbf{for} $m \in [M,\ldots,1], b \in \mathcal{B}: \widehat{S}^t_\nitem(b; \bm{v}) \gets \exp(\eta \widehat{W}_\nitem^\nround(b; \bm{v})) \sum_{b' \leq b} \widehat{S}_{\nitem + 1}^\nround(b'; \bm{v})$\;
            $q^t_1(b; \bm{v}) \gets \frac{\widehat{S}^\nround_m(b; \bm{v})}{\sum_{b' \in \mathcal{B}} \widehat{S}^\nround_m(b'; \bm{v})}$ for all $b \in \mathcal{B}$\;
            \textbf{for} $m \in [2,\ldots,M], b \in \mathcal{B}: q_\nitem^\nround(b; \bm{v}) \gets \sum_{b' \geq b} \frac{q_{\nitem-1}^t(b'; \bm{v})\widehat{S}^\nround_{\nitem}(b; \bm{v})}{\sum_{b" \geq b'} \widehat{S}^\nround_\nitem(b"; \bm{v})}$ for all $b \in \mathcal{B}$\;
            $Q_m^t(b) \gets Q_m^t(b) + \prob(\bm{v}^t = \bm{v})q_m^t(b; \bm{v})$
        }
        \textbf{Update Weight Estimates}\;
        \textbf{if} $\textsc{Bandit Feedback}$, \textbf{for} $m \in [M], b \in \mathcal{B}, \bm{v} \in \mathcal{V}$\; 
        $\widehat{W}^{\nround+1}_{\nitem}(b; \bm{v}) \gets \widehat{W}^{\nround}_{\nitem}(b; \bm{v}) + (1 - \frac{1 - (v - b)\textbf{1}_{b \geq b^t_m}}{Q_m^t(b)} \textbf{1}_{b^t_m = b})$ if $b \leq v$; else $\widehat{W}^{\nround+1}_{\nitem}(b; \bm{v}) \gets -\infty$\;
        \textbf{if} $\textsc{Full Information}$, \textbf{for} $m \in [M], b \in \mathcal{B}, \bm{v} \in \mathcal{V}$\;
        $\widehat{W}^{\nround+1}_{\nitem}(b; \bm{v}) \gets \widehat{W}^{\nround}_{\nitem}(b; \bm{v}) + (v - b)\textbf{1}_{b \geq b^t_m}$ if $b \leq v$; else $\widehat{W}^{\nround+1}_{\nitem}(b; \bm{v}) \gets -\infty$\;
        }
        \textbf{Return} $\sum_{\nround=1}^\Nround \mu_n^{\nround}(\bm{b}^{\nround}; \bm{v}^\nround)$
	\caption{\textsc{Decoupled $\textsc{EXP3-CL}$ - Time Varying Valuations}}
	\label{alg: Decoupled Exponential Weights - Time Varying Known Finite}
\end{algorithm}}
%% -*- mode: LaTeX; fill-column: 78; -*-

\section{Concluding Remarks}
\label{sec:conclusions}

In this paper, we presented a novel SMC algorithm, \EventDPOR, tailored to the
characteristics of event-driven multi-threaded programs running under the SC
semantics. The algorithm was proven correct and optimal for event-driven
programs in which the variable accesses of events do not depend on how their
execution is interleaved with other threads.

We have implemented \EventDPOR in the \Nidhugg tool, and we will open-source
our implementation.
%
With a wide range of event-driven programs, we have shown that \EventDPOR
incurs only a moderate constant overhead over its baseline implementation
(\OptimalDPOR), it is exponentially faster than existing state-of-the-art SMC
algorithms in time and number of traces examined on programs where events'
actions do not conflict, and does not suffer from performance degradation
caused by having to examine
% a significant number of
non-serializable executions.
%
%% \bjcom{Should we include:
%% Moreover, in our benchmarks, also those that are not non-branching,
%% \EventDPOR explores only the optimal number of executions, and never
%% had to resort to a potentially expensive decision procedure.}

\EventDPOR assumes that handlers can process their events in arbitrary order.
Directions for future work include to retarget \EventDPOR for event-driven
programs with other policies (e.g., FIFO), and for specific event-driven
execution models.

\bibliographystyle{ACM-Reference-Format}
\footnotesize{
\bibliography{ref.bib}
}
\newpage
% \begin{APPENDICES}
\begin{comment}
\section{System Architecture}
\label{appendix:architecture}
\system has a novel modularized system architecture with three key components: 
\emph{StreamManager}, 
\emph{TxnManager} and \emph{TxnScheduler}. 
These components are instantiated in each thread locally.
The execution outline of \system is presented in Algorithm~\ref{alg:algo}.
Transactional stream processing is continuous and potentially never ends (Line 1$\sim$8).
The dependency resolution and execution of state transactions are separated into two non-overlapping phases by punctuations~\cite{Tucker:2003:EPS:776752.776780} (Line 2 and 5), which guarantees that no subsequent input event will have a smaller timestamp. 
Effectively, a batch of state transactions is collected during the first phase, and processed during the second phase.

In the first phase (i.e., stream processing phase), 
the \emph{StreamManager} conducts preprocessing for every input event ($e$). Similar to some prior works~\cite{tstream}, state transactions may be issued but not immediately processed during preprocessing (Line 3).
The \emph{pre\_processing} and \emph{post\_processing} functions are exposed as APIs to users.
The \emph{TxnManager} handles dependency resolution (Line 4) among state transactions and insert decomposed operations to construct a \tpg. We discuss the detailed two-phase \tpg construction process in Section~\ref{subsec:construction}.

In the second phase  (i.e., transaction processing phase), 
the \emph{TxnManager} is first involved again to refine (Line 6) the constructed \tpg with further dependency resolution.
The \emph{TxnScheduler} 
schedules operations for concurrent execution based on the constructed \tpg according to the three dimensions of scheduling decisions (Line 7). 
In particular, a scheduling decision model $M$ is instantiated based on the constructed \tpg (Line 14).
\textbf{\circled{1}} Guided by $M$, execution threads adopt an exploration strategy (Section~\ref{subsec:explore}) to explore the constructed \tpg for operations available to be scheduled constrained by dependencies. 
\textbf{\circled{2}} 
During exploration, one or multiple operations may be treated as the 
% basic 
unit of scheduling (Section~\ref{subsec:granularity}). 
Subsequently, \textbf{\circled{3}} every thread executes operation(s) in the unit of scheduling with various abort handling mechanisms (Section~\ref{subsec:abort_handling}).
Only when state transactions are processed (i.e., committed or aborted) can the associated input events be postprocessed (Line 8) by the \emph{StreamManager} based on transaction processing results.
\end{comment}

\begin{comment}
\begin{algorithm}
\footnotesize
    \KwData{$e$ \tcp{Input event}}
    \KwData{$txn_{ts}$ \tcp{State transaction}}
    \KwData{$G$ \tcp{The currently constructed TPG}}
    \While{!finish processing of input streams}{
        \eIf(\tcp*[h]{Phase 1}){\text{$e$ is not a $punctuation$}}{
                $txn_{ts}$ $\gets$ PRE\_Processing($e$)\;
                \textbf{TPG\_Construction}($G$, $txn_{ts}$)\; 
          }(\tcp*[h]{Phase 2}){
                \textbf{TPG\_Refinement}($G$)\; 
                \textbf{TXN\_Scheduling}($G$)\; 
                POST\_Processing()\;
          }
    }
    
    \SetKwFunction{FMain}{TPG\_Construction}
    \SetKwProg{Fn}{Function}{:}{}
    \Fn{\FMain{$G$, $txn_{ts}$}}{
        $O_{1..k}$ $\gets$ \textbf{Partition} $txn_{ts}$\;
        \ForEach{\text{operation $O_{i}$ $\in$ $O_{1..k}$}}{
            \textbf{Identify} its \ld\;
            $G$ $\gets$ $G$ + $O_{i}$ \;
        }
    }
    \SetKwFunction{FMain}{TPG\_Refinement}
    \SetKwProg{Fn}{Function}{:}{}
    \Fn{\FMain{$G$}}{
        \ForEach{\text{vertex $e_{i}$ $\in$ $G$}}{
            \textbf{Identify} its \td, \pd\;
        }
    }
    
    \SetKwFunction{FMain}{TXN\_Scheduling}
    \SetKwProg{Fn}{Function}{:}{}
    \Fn{\FMain{$G$}}{
        $M$ $\gets$ Instantiated with $G$;\tcp{A decision model}
        \While{!finish scheduling of $G$
        }{
          \textbf{\circled{2}} $Scheduling Unit$ $\gets$ \textbf{\circled{1}} \emph{Explore}($G$, $M$)\; 
            \textbf{\circled{3}} \emph{Execute with Abort Handling} ($Scheduling Unit$)\; 
        }
    }
  \caption{Execution Outline of \system}
  \label{alg:algo}
\end{algorithm}
\end{comment}
\newpage

\TITLE{Learning in Repeated Multi-Unit Pay-As-Bid Auctions}

\ARTICLEAUTHORS{
\AUTHOR{Rigel Galgana}
\AFF{Operations Research Center, Massachusetts Institute of Technology, \EMAIL{rgalgana@mit.edu}, \URL{}}
\AUTHOR{Negin Golrezaei}
\AFF{Sloan School of Management, Massachusetts Institute of Technology,  \EMAIL{golrezaei@mit.edu}, \URL{}}
} 

\begin{center}
    {\LARGE{Learning in Repeated Multi-Unit Pay-As-Bid Auctions}}

    {\small{Rigel Galgana}}

    {\scriptsize{Operations Research Center, Massachusetts Institute of Technology, rgalgana@mit.edu}}

    {\small{Negin Golrezaei}}

    {\scriptsize{Sloan School of Management, Massachusetts Institute of Technology, golrezaei@mit.edu}}
\end{center}

\maketitle

In this online accompaniment, we include all of the omitted proofs of several of our key results as well as additional discussion and experiments regarding algorithm implementation and market dynamics. We conclude with a discussion regarding both the feasibility and practicality of the PAB versus Uniform Price auction formats.

\section{Online Appendix - Missing Proofs}

In this section, we include proofs of several of our key results. We first precisely characterize the PAB equilibrium under our $c = \lfloor v_{(M)}\rfloor_\delta = c_{-n}$ competitiveness assumption. Second, we flesh out the details of our OMD algorithm as well as its full proof, regret analysis, and complexity analysis. Lastly, we prove our time-varying valuations generalization of our decoupled exponential weights algorithm.


\subsection{Proof of Theorem \ref{thm: PNE existence}: Existence of an Approximately Efficient PNE}
\label{sec: equilibrium proof}

\textsc{Theorem 1} \textbf{(Existence of an Apprixxmately Efficient PNE)}

\emph{Define the clearing price $c = \lfloor v_{(M)} \rfloor_{\delta}$ to be the $M$'th largest valuation among all bidders, denoted by $v_{(M)}$,  rounded down to the nearest multiple of $\delta$, and similarly, define $c_{-n} = \lfloor v_{(-n, M)} \rfloor_{\delta}$ to be the rounded $M$'th largest valuation among all bidders except $n$. If $c = c_{-n}$ for all $n\in [N]$ and ties are broken in favor of higher indexed bidders, then there exists a PNE $(\bm{b}_1,\ldots,\bm{b}_N)$ where each bidder $n\in [N]$:  
    \begin{enumerate}
        \item submits bids of either all $c$ or all $c + \delta$ for units such that $v_{n,m} \geq c + \delta$,
        \item submit bids of $c$ for all units such that $v_{n,m} \in [c, c + \delta)$,
        \item submit bids smaller than $c$ for all other units.
    \end{enumerate}
    Moreover, this PNE is $M\delta$-approximately welfare optimal:
    \begin{align*}
        \sum_{m=1}^M v_{(m)} - \sum_{n=1}^N \sum_{m=1}^{x(\bm{b}_n, \bm{b}_{-n})} v_{n,m} \leq M\delta\,.
    \end{align*}}

{\color{black}
\begin{proof}{Proof of Theorem~\ref{thm: PNE existence}}

To prove Theorem~\ref{thm: PNE existence}, we proceed in two steps. First, we demonstrate that if all other bidders adhere to the three properties outlined in Theorem~\ref{thm: PNE existence}, then bidder $n$'s optimal bids must also satisfy these properties. Afterwards, using a monotonicity argument, we show that under our particular deterministic tie-breaking rule, there exists a specific configuration of bids—either $c$ or $c + \delta$—among each bidder’s winning bids that constitutes a PNE.

\textbf{First Part of the Proof.} {We first show the three properties. The first property implies that if for every bidder $i\ne n$, we have  bids of either all $c$ or all $c + \delta$ for units such that $v_{i,m} \geq c + \delta$, then bidder $n$ also submits bids of  either all $c$ or all $c + \delta$ for units such that $v_{n,m} \geq c + \delta$. This 
follows immediately from Lemma~\ref{lem: PNE uniform bidding} and recalling the definition of $c$.

 To demonstrate the second property, we argue that there is no incentive for bidder $n$ to decrease their winning bids below $c$, specifically for any unit $m$ with $v_{n, m} \in [c, c+\delta)$. Given that there are at least $M$ other valuations equal to or greater than $c = c_{-n}$, and considering the PNE characterization which states that for any bidder $i \neq n$, all units with a valuation at or above the clearing price (i.e., $v_{i, m} \geq c$, $i \in [N], i\ne n$) must be accompanied by a bid of either $c$ or $c + \delta$, it follows that there are at least $M$ bids of at least $c = c_{-n}$ submitted by other bidders. Therefore, reducing any of bidder's $n$ winning bids below $c$ would not result in winning a unit, making it sub-optimal.


Lastly, to show the final property, we show that there exists no incentive for bidder $n$ to increase their bid for the remaining items (i.e., any item $m$ with $v_{n, m}< c$) to at least $c$, we recall the definition of $c= c_{-n}$ being the $M$'th largest valuation among all bidders except bidder $n$. Thus, bidding above $c$ for these units violates the no-overbidding assumption.}

\textbf{Second Part of the Proof.} Now that we have shown the three properties, we fully characterize the PNE w.r.t. each bidders' largest bids. In particular, consider the perspective of bidder $n$, who has tie-break priority over bidders $1,\ldots,n-1$ but behind $n+1,\ldots,N$. Thus, any bids of $c+\delta$ submitted by the first $n-1$ bidders, and any bids of $c$ or $c + \delta$ submitted by bidders $n+1,\ldots,N$, take priority over any bids of $c$ submitted by bidder $n$. Let $M_n$ denote the number of bids of $c+\delta$ submitted by bidder $n$, which by our PNE characterization, is between $0$ and $\sum_{m=1}^M 1_{v_{n, m} \geq c + \delta}$. Fixing $M_{1:n-1}$ and $M_{n+1:N}$, we claim that the optimal $M_n$ is precisely either 0 or $\sum_{m=1}^M 1_{v_{n, m} \geq c + \delta}$---they either submit bids of all $c$ or all $c+\delta$ for all items they value at least $c + \delta$. 

To show this, notice that bidder $n$ can win all of the items they value at least $c + \delta$ by bidding at $c + \delta$, as there are fewer than $M$ values at least $c+\delta$ across all bidders. In addition to this, they can win some number of items by bidding at $c$ for all items with value at least $c$ via tie-break. To be more specific, 
there are $M$ items with $\sum_{n' \leq n} M_{n'}$ bids of $c + \delta$ and $\sum_{n' > n} \sum_{m=1}^M 1_{v_{n',m} \geq c}$ bids of at least $c$ (by the PNE characterization) that take priority over any bids of $c$ submitted by bidder $n$, where we note that the number of bids of $c$ submitted by bidder $n$ is at most $\sum_{m=1}^M 1_{v_{n,m} \geq c}$. Thus, bidder $n$'s allocation and utility as a function of $M_n$ for any fixed $\bm{M}_{-n} = (M_{-1},\ldots,M_{-(n-1)},M_{-(n+1)},\ldots,M_{-N})$ is given by:

\begin{align*}
    x_n(M_n | \bm{M}_{-n}) &= \min\left(\sum_{m=1}^M 1_{v_{n,m} \geq c}~,~ M_n + \max\big(0, \theta(\bm{M}_{-n}) - M_n\big)\right)\\
    \hspace{1mm} \mu_n(M_n | \bm{M}_{-n}) &= - M_n\delta +\sum_{m=1}^{x_n(M_n | \bm{M}_{-n})} (v_{n,m} - c)\,,
\end{align*}
where $\theta(\bm{M}_{-n}) = M - \sum_{n' < n} M_{n'} - \sum_{n' > n} \sum_{m=1}^M 1_{v_{n',m} \geq c}$. Here, $\min\left(\sum_{m=1}^M 1_{v_{n,m} \geq c}, \theta(\bm{M}_{-n})\right)$ denotes the number of units bidder $n$ would have won in tie-break by submitting $M_n = 0$ bids of $c + \delta$.
Now, let's consider the general case where $M_n\ge 0$. For  all $M_n \geq \theta(\bm{M}_{-n})$, the $\max(0, \theta(\bm{M}_{-n}) - M_n)$ term in the allocation function is 0, and thus, the allocation function  increases linearly in $M_n$ until $M_n = \sum_{m=1}^M 1_{v_{n,m} \geq c+\delta}$. In contrast, for $M_n <\theta(\bm{M}_{-n})$, the second term (i.e., $\max(0, \theta(\bm{M}_{-n}) - M_n)$) is non-zero, and the $M_n$ within the summation of the second term cancels out with the first term of $M_n$. Thus, the allocation function $x_n(M_n | \bm{M}_{-n}) = \theta(\bm{M}_{-n})$ is constant for all $M_n \leq \theta(\bm{M}_{-n})$, at which point it becomes precisely $x_n(M_n | \bm{M}_{-n}) = \theta(\bm{M}_{-n}) = M_n$ until $M_n = \sum_{m=1}^M 1_{v_{n, m} \geq c + \delta}$. This reflects the fact that each additional bid at $c + \delta$ submitted by bidder $n$ consumes an item that could have been won in tie-break at price $c$, which we illustrate in Figure~\ref{fig: allocation visualization PNE}. From Figure~\ref{fig: allocation utility vs mn}, we see that the optimal $M_{-n}$ is always achieved at $M_{-n} \in \{0, \sum_{m=1}^M 1_{v_{n,m} \geq c + \delta}\}$.

% Figure environment removed



% Figure environment removed

    
    Now that we have shown that the optimal number of bids to submit at $c + \delta$ of each bidder is either $M_n = 0$ or $M_n = \sum_{m=1}^M 1_{v_{n, m} \geq c + \delta}$, we finish the argument by claiming that the utility corresponding to $M_n = 0$ is weakly decreasing in $\bm{M}_{-n}$. This is because $x_n(\cdot | \bm{M}_{-n})$, and similarly $\mu_n(\cdot | \bm{M}_{-n})$, are weakly decreasing in these quantities. Because of this monotonicity, agents can run best response dynamics beginning with $(M_1,\ldots,M_N) = (0, \ldots, 0)$ and converge to a PNE w.r.t. $M_1,\ldots,M_N$. That is, each agent that switches from $M_n = 0$ to $M_n = \sum_{m=1}^M 1_{v_{n, m} \geq c + \delta}$ can only incentivize other bidders to also switch away from 0 and never towards 0.
    
\end{proof}

}

\subsection{Lemmas \ref{lem: QSpace Equivalence} and \ref{lem: Online Linear Optimization}, and their Proofs} \label{sec:QSpace Equivalence} 

Now, we complete the description and analysis of our OMD based algorithm. In Algorithm~\ref{alg: OMD}, we require that the space of all possible node probability measure $\bm{q}$ encompasses the set of node probability measures that correspond to any policy $\bm{\pi}$ over our DP graph.

\begin{lemma}[$\mathcal{Q}$-Space Equivalence]
    \label{lem: QSpace Equivalence}
    Let \[\Pi = \Big\{{\pi} \in [0,1]^{M\times |\mathcal B|\times |\mathcal B|}: \pi((m, b), b') = 0 ~~\forall b' > b, m\in [M], \sum_{b' \leq b} \pi((m, b), b') = 1, m\in [M]\Big\}\] denote the space of policies on our DP graph. With a slight abuse of notation, for any $\pi\in  \Pi$, define 
    \[q(\pi) = \{\mathbf{q} \in [0, 1]^{M\times |\mathcal B|}: \forall b \in \mathcal{B},  q_1(b) =\pi((0, b_0), b), q_{m+1}(b) = \sum_{b' \in \mathcal B} q_m(b')\pi((m, b'), b), m\in[M-1]\}\,\] as the node probabilities induced by $\pi$. Here, $b_0= \max \mathcal B$. Let $\mathcal{Q}_{\Pi} = \cup_{\pi \in \Pi} q(\pi)$. Then,   $\mathcal{Q}_{\Pi}$ is equivalent to the set $\mathcal{Q}$ where $\mathcal{Q}$ is defined in Equation \eqref{eq:Q}. 
\end{lemma}



Lemma \ref{lem: QSpace Equivalence} establishes that during the execution of Algorithm \ref{alg: OMD}, we can focus on the node probabilities in set $\mathcal{Q}$ without loss of generality. We recall that within $\mathcal{Q}$, the stochastic dominance conditions are enforced solely over node probabilities across layers. 
We now argue that we only need to consider optimizing over $\mathcal{Q}$ as opposed to the space of policies $\Pi$, as the regret can be rewritten strictly in terms of $\bm{q}$, independently of the corresponding $\bm{\pi}$.


\begin{lemma}
    \label{lem: Online Linear Optimization}
     Any sequence of policies $\bm{\pi}^1,\ldots,\bm{\pi}^\Nround$ over our DP graph with associated node probability measures $\bm{q}^1,\ldots,\bm{q}^\Nround$ has discretized regret $\textsc{Regret}_{\mathcal{B}} = \max_{\bm{q} \in \mathcal{Q}} \sum_{\nround=1}^\Nround \langle \bm{q} - \bm{q}^\nround, \bm{w}^\nround\rangle$. Here, $\bm{w}^\nround = \{w^\nround_m(b)\}_{m \in [M], b \in \mathcal{B}}$ represents vector of the round $\nround$ rewards for all possible $(m, b)$ unit-bid value pairs.
 \end{lemma}
      
\subsubsection{Proof of Lemma \ref{lem: QSpace Equivalence}}

In order to show equivalence, we show that (1) for any $\pi \in \Pi$, that $q(\pi) \in \mathcal{Q}$ and (2) for any $\bm{q} \in \mathcal{Q}$, there exists a $\pi \in \Pi$ such that $q(\pi) = \bm{q}$. We first prove (1). To do this, we simply need to check that for a given $\pi \in \Pi$, that $q^\pi = q(\pi)$ satisfies the constraints prescribed by $\mathcal{Q}$.

    The non-negativity constraint holds trivially as each $\pi((m, b), b')$ is non-negative. Since all $q^\pi_1(b) = \pi((0, \max\mathcal{B}), b) \geq 0$ for all $b \in \mathcal{B}$, by induction, $q^\pi_{m+1}(b) = \sum_{b" \geq b} q^\pi_m(b") \pi((m, b"), b)$ is also non-negative.
    
    Now we prove that each layer $m$ sums to 1, i.e., $\sum_{b \in \mathcal{B}} q^\pi_m(b) = 1$. Since $\sum_{b \in \mathcal{B}} q^\pi_1(b)$, the policy has total node probability 1 in the first layer, we can prove $\sum_{b \in \mathcal{B}} q^\pi_m(b) = 1$, that the policy has total node probability 1 in the $m$'th layer, via induction. This follows immediately from the fact that the DP graph is layered, i.e., edges exist only from nodes in layer $m$ to nodes in layer $m+1$, thus the only edges leading to layer $m+1$ are from layer $m$, in which there are no other edges. Hence, the total node probability in layer $m+1$ must be exactly that of layer $m$. More formally, we have:
    \begin{align*}
        \sum_{b \in \mathcal{B}} q^\pi_{m+1}(b) = \sum_{b \in \mathcal{B}} \sum_{b" \geq b} q^\pi_m(b") \pi((m, b"), b) = \sum_{b" \in \mathcal{B}} q^\pi_m(b") \sum_{b \leq b"} \pi((m, b"), b) = \sum_{b" \in \mathcal{B}} q^\pi_m(b")\,.
    \end{align*}

     To show the stochastic domination constraint $\sum_{b \leq b'} q^\pi_{m+1}(b) \geq \sum_{b \leq b'} q^\pi_m(b)$, we use the bid monotonicity constraint; i.e., the fact that the edges between layers are only from larger bids to (weakly) smaller bids. Recall that $\pi((m,b'), b")$ is the probability of transitioning from unit-bid value pair $(m, b')$ to $(m+1, b")$ and that the only edges leading to $(m+1, b")$ come from nodes $(m, b')$ for $b' \geq b"$. Then, we have:
     \begin{align*}
         \sum_{b \leq b'} q^\pi_{m+1}(b) &= \sum_{b \leq b'} \sum_{b" \geq b} q^\pi_m(b")\pi((m, b"), b) \\
         &=\sum_{b" > b'} q^\pi_m(b") \sum_{b \leq b'} \pi((m, b"), b) + \sum_{b" \leq b'} q^\pi_m(b") \sum_{b \leq b"} \pi((m, b"), b)\\
         &= \sum_{b" > b'} q^\pi_m(b") \sum_{b \leq b'} \pi((m, b"), b) + \sum_{b \leq b'} q^\pi_m(b)\\
         &\geq \sum_{b \leq b'} q^\pi_m(b)\,.
     \end{align*}
     Hence, we have shown that for any $\pi \in \Pi$, that the corresponding $q(\pi) \in \mathcal{Q}$.
     
     Now we show the other direction (2), that for any $\bm{q} \in \mathcal{Q}$, there exists a $\pi \in \Pi$ such that $q(\pi) = \bm{q}$. We proceed by showing that for all $m, b^*$, there exists $\{\pi((m, b), b')\}_{b, b' \in \mathcal{B}}$ such that the following conditions hold:
     \begin{enumerate}
         \item $\pi((m, b), b') \geq 0$ for all $b, b' \geq b^*$.
         \item $\pi((m, b), b') = 0$ for all $b' > b \geq b^*$.
         \item $\sum_{b' \leq b, b' \geq b^*} \pi((m, b), b') \leq 1$ for all $b^* \in \mathcal{B}$, with equality if and only if $b^* =  b_{\min}$ where $b_{\min} = \min \mathcal{B}$.
         \item $\sum_{b' \geq b^*} \sum_{b \geq b'} q_m(b)\pi((m, b), b') = \sum_{b' \geq b^*} q_{m+1}(b')$.
     \end{enumerate}
Let $\Pi(b^*; \mathbf{q})$, $b^*\in \mathcal B$, be the set of all policies under which the four conditions hold at $b^*$ and ${\mathbf q} \in \mathcal Q$. 
     
    These conditions trivially hold for $m = 0$, as we can set $\pi((0, \max \mathcal{B}), b) = q_1(b)$ and $\pi((0, b), b') = \textbf{1}_{b = b'}$. To solve for general $m$, we must show that there exists $\{\pi((m, b), b')\}_{b, b' \in \mathcal{B}}$ that satisfies the constraints prescribed by $\Pi$ and that $\sum_{b \geq b'} q_m(b)\pi((m, b), b') = q_{m+1}(b')$ for all $b' \in \mathcal{B}$. In order to do this, we show that conditions (1), (2), (3), and (4) for each $b^* \in \mathcal{B}$. In particular, if we show conditions (1) and (2) for $b^* = b_{\min}$, then we have already satisfied the first two conditions of $\Pi$. If we show that (3) holds for $b^* = b_{\min}$, then by condition (1), then (3) holds for all $b^* \in \mathcal{B}$ as well, as the summation only includes fewer terms as $b^*$ increases. Similarly, if we show condition (4) holds for two adjacent values of $b_-^* < b^*$, then we have that $\sum_{b \geq b' \geq b^*} q_m(b)\pi((m, b), b') = q_{m+1}(b^*)$. Thus, if condition (4) holds for all possible pairs of adjacent bid values, then we have that $\sum_{b \geq b'} q_m(b)\pi((m, b), b') = q_{m+1}(b')$ for all $b'$. These observations suggest use of induction over $b^*$, and indeed, we begin by showing that these conditions hold for $b^* = b_{\min}$. We then show that this implies that the conditions hold for the next smallest value of $b^*$, which would complete the induction proof.

    \textbf{Base Case}: Recall $b^* = b_{\min}$. We now show that there exists $\{\pi((m, b), b')\}_{b, b' \in \mathcal{B}}$ satisfying all four conditions. For any $m\in[M]$, let we set $\pi((m, b), b') = \textbf{1}_{b = b'}$. Then, 
 condition (4) is clearly satisfied: 
    \begin{align*}
        \sum_{b' \geq b^*} \sum_{b \geq b'} q_m(b)\pi((m, b), b') = \sum_{b' \geq b^*} q_{m+1}(b') \leftrightarrow \sum_{b \in \mathcal{B}} q_m(b) \sum_{b' \leq b} \pi((m, b), b') = \sum_{b \in \mathcal{B}} q_{m+1}(b) = 1\,.
    \end{align*}
    It is also easy to check that conditions (1)-(3) are also satisfied when we set $\pi((m, b), b') = \textbf{1}_{b = b'}$ for any $m$. This shows that $\Pi(b^*; {\mathbf{q}})$ is non-empty, as desired. 
   
    
    \textbf{Recursive Case}:
    For any $b\in \mathcal B$, let $b_-$ be the largest $b' \in \mathcal B$, which is strictly smaller than $b$. Here, we assume that 
    $\Pi(b^*_-; \bf{q})$ is not empty, and under this assumption, we show that set $\Pi(b^*; \bf{q})$ is not empty, where $\Pi(b^*; {\bf{q}}) \subseteq \Pi(b^*_-; \bf{q})$. 
    Let us start with condition (4). 
    We would like to show that there exists a $\bm{\pi}$ that satisfies condition (4) at $b^*$ along with the other three conditions. By the induction assumption, we have  
    \begin{align*}
        &\sum_{b' \geq b_-^*} \sum_{b \geq b'} q_m(b)\pi((m, b), b') = \sum_{b' \geq b_-^*} q_{m+1}(b') \to\\
        &\sum_{b' \geq b^*} \sum_{b \geq b'} q_m(b)\pi((m, b), b') + \sum_{b \geq b_-^*} q_m(b)\pi((m, b), b_-^*) = \sum_{b' \geq b^*} q_{m+1}(b') + q_{m+1}(b_-^*) \to\\
        &\sum_{b' \geq b^*} \sum_{b \geq b'} q_m(b)\pi((m, b), b') = \sum_{b' \geq b^*} q_{m+1}(b') + \left[q_{m+1}(b_-^*) - \sum_{b \geq b_-^*} q_m(b)\pi((m, b), b_-^*)\right] \to\\
        &\sum_{b \geq b^*} q_m(b) \sum_{b' \leq b; b' \geq b^*} \pi((m, b), b') = \sum_{b' \geq b^*} q_{m+1}(b') + \left[q_{m+1}(b_-^*) - \sum_{b \geq b_-^*} q_m(b)\pi((m, b), b_-^*)\right] 
    \end{align*}
    Thus, we can satisfy condition (4) if $q_{m+1}(b_-^*) = \sum_{b \geq b_-^*} q_m(b)\pi((m, b), b_-^*)$. We now observe that the latter summation depends linearly (and hence, continuously) in the values of $\pi((m, b), b_-^*)$. If we can show that there exists an assignment of these variables that satisfy $q_{m+1}(b_-^*) \geq \sum_{b \geq b_-^*} q_m(b)\pi((m, b), b_-^*)$ and also $q_{m+1}(b_-^*) \leq \sum_{b \geq b_-^*} q_m(b)\pi((m, b), b_-^*)$, then by the intermediate value theorem, there must be some assignment that achieves exact equality. 
    
    
    In order to show the first inequality, notice that if we set $\pi((m, b), b_-^*) = 1 - \sum_{b' < b_-^*} \pi((m, b), b')$ for all $b \geq b_-^*$ (this is required in order to guarantee conditions (1) and (3) are satisfied), then:
    \begin{align*}
        \sum_{b \geq b_-^*} q_m(b)\pi((m, b), b_-^*) &= \sum_{b \geq b_-^*} q_m(b) - \sum_{b \geq b_-^*} q_m(b)\sum_{b' < b_-^*} \pi((m, b), b') \\
        &= \sum_{b \geq b_-^*} q_{m}(b) - \sum_{b \in \mathcal{B}} q_m(b) \sum_{b' < b_-^*} \pi((m, b), b') + \sum_{b < b_-^*} q_m(b) \sum_{b' < b_-^*} \pi((m, b), b')\\
        &= \sum_{b \geq b_-^*} q_{m}(b) - \sum_{b' < b_-^*} q_{m+1}(b) + \sum_{b < b_-^*} q_m(b) \sum_{b' < b_-^*} \pi((m, b), b')\\
        &= \sum_{b \geq b_-^*} q_{m}(b) - \sum_{b' < b_-^*} q_{m+1}(b) + \sum_{b < b_-^*} q_m(b)\\
        &\geq \sum_{b \geq b_-^*} q_{m+1}(b) - \sum_{b' < b_-^*} q_{m+1}(b) + \sum_{b < b_-^*} q_{m+1}(b)\\
        &= \sum_{b \geq b_-^*} q_{m+1}(b)\\
        &\geq q_{m+1}(b_-^*)\,.
    \end{align*}
    Here, the third equality follows from the (strong) inductive hypothesis, and the first inequality is a result of the stochastic domination constraint in $\mathcal{Q}$. We also note that the values $\sum_{b' < b_-^*} \pi((m, b), b')$ have already been fixed
    as these were required to satisfy condition (4) in the previous iterates, and as condition (3) holds for $b^*_-$ by the inductive hypothesis, then $1 - \sum_{b' < b_-^*} \pi((m, b), b') \geq 0$. Conversely, if we set $\pi((m, b), b_-^*) = 0$ for all $b \geq b_-^*$, then:
    \begin{align*}
        \sum_{b \geq b_-^*} q_m(b)\pi((m, b), b_-^*) = 0 \leq q_{m+1}(b_-^*)\,.
    \end{align*}
    As the sum $\sum_{b \geq b_-^*} q_m(b)\pi((m, b), b_-^*)$ linearly (thus, continuously) depends on the values of $\pi((m, b), b_-^*)$, by the intermediate value theorem, there exists an assignment of $\{\pi((m, b), b_-^*)\}_{b \geq b_-^*}$ with each $\pi((m, b), b_-^*) \in [0, 1 - \sum_{b' < b_-^*} \pi((m, b), b')]$ such that the sum is precisely equal to $q_{m+1}(b_-^*) \in [0, 1]$. Now we observe that these values of $\pi((m, b), b_-^*) \in [0, 1 - \sum_{b' < b_-^*} \pi((m, b), b')]$ do not violate conditions (1), (2), or (3). Furthermore, note that any $\bm{\pi} \in \Pi$ also satisfied conditions (1), (2), and (3) under $b^*_-$ for $\{\pi((m, b), b')\}_{b \geq b_-^*, b' \leq b_-^*}$, then the assignment to $\{\pi((m, b), b')\}_{b \geq b_-^*, b' < b_-^*}$ will not violate these conditions as our new constraint on the variables $\{\pi((m, b), b_-^*)\}_{b \geq b_-^*}$ is independent of the values of $\{\pi((m, b), b')\}_{b \geq b_-^*, b' < b_-^*}$. Thus, the set $\Pi(b^*)$ is non-empty:
    \begin{align*}
        \Pi(b^*) = \{\{\pi((m, b), b')\}_{b, b' \in \mathcal{B}} \in \Pi(b_-^*): \sum_{b \geq b_-^*} q_m(b)\pi((m, b), b_-^*) = q_{m+1}(b_-^*)\} \neq \emptyset
    \end{align*}    
    With this, we have proven via induction that our four conditions hold for all $b^* \in \mathcal{B}$, implying that for a fixed $m$, every constraint in $\Pi$ pertaining to variables $\pi((m, b), b')$ is satisfied, as well as the node-measure constraints $\sum_{b \geq b'} q_m(b)\pi((m, b), b') = q_{m+1}(b')$ for all $b'$. By induction, this works for all $m \in [M]$, which concludes the proof.



\subsubsection{Proof of Lemma \ref{lem: Online Linear Optimization}}
We have by the definition of discretized regret:
\begin{align*}
    \textsc{Regret}_\mathcal{B} &= \max_{\bm{b} \in \mathcal{B}} \sum_{\nround=1}^\Nround \mu^\nround_n(\bm{b}) - \mathbb{E}\left[\sum_{\nround=1}^\Nround \mu^\nround_n(\bm{b}^\nround)\right] = \max_{\bm{q} \in \mathcal{Q}} \mathbb{E}\left[\sum_{\nround=1}^\Nround \langle \bm{q}, \bm{w}^\nround\rangle - \sum_{\nround=1}^\Nround \langle \bm{q}^\nround, \bm{w}^\nround\rangle\right]\,, 
\end{align*}
where in the first equality, we applied Equation~\eqref{eq: Loss of policy} which equated the dot product of utilities $\bm{w}^\nround$ and node probability weights $\bm{q}$ to the expected utility of bid vector $\bm{b} \sim \bm{\pi}$ with utilities $\{w_m^t(b)\}_{m \in [M], b \in \mathcal{B}} = \bm{w}^\nround$. Combining the two summations yields the desired result.

\subsection{Proof of Theorem~\ref{thm: OMD}: Online Mirror Descent Algorithm }

\label{sec: Proof of OMD}

\begin{proof}{Proof of Theorem~\ref{thm: OMD}: Online Mirror Descent Algorithm}

    The proof is divided into four parts, similar to the analysis of Algorithm~\ref{alg: Decoupled Exponential Weights}. In the first part, we rigorously show how our algorithm achieves the stated regret. In the second, we verify correctness of our procedure that recovers a policy $\bm{\pi}^\nround$ from $\bm{q}^\nround$. Then, we show the corresponding time and space complexity of our algorithm. Afterwards, we optimize over discretization error to obtain the continuous regret. 
    

    \textbf{Part 1: Regret of Online Linear Optimization.} Recall that from Lemma~\ref{lem: Online Linear Optimization}, we have
    \begin{align}
        \textsc{Regret}_\mathcal{B} = \max_{\bm{q} \in \mathcal{Q}} \mathbb{E}\left[ \sum_{\nround=1}^\Nround \langle \bm{q} - \bm{q}^\nround, \bm{w}^\nround \rangle\right] = \max_{\bm{q} \in \mathcal{Q}} \mathbb{E}\left[\sum_{\nround=1}^\Nround \langle  \bm{q}^\nround - \bm{q}, -\bm{w}^\nround \rangle\right]\, ,
    \end{align}
    where we negate the utility function into a loss function to be consistent with the OLO convention. We follow a standard analysis of OMD, which shows that the optimization step can be solved efficiently and the resulting iterates have bounded regret. For the former, we show that solution to the $\bm{q}$ optimization step in our algorithm $\bm{q}^{\nround} = \text{argmin}_{\bm{q} \in \mathcal{Q}} \eta\langle \bm{q}, -\bm{w}^\nround\rangle + D(\bm{q} || \bm{q}^{\nround-1})$ can be obtained as the projection of the unconstrained minimizer of \[\tilde{q}^{\nround}= \text{argmin}_{\bm{q} \in [0, 1]^{M \times |\mathcal{B}|}} \eta\langle \bm{q}, -\bm{w}^\nround\rangle + D(\bm{q} || \bm{q}^{\nround-1})\] to the space $\mathcal{Q}$ (See Projection Lemma, Lemma 8.6 of \cite{BartokLecNotes2011}). 
    Having characterized the exact form of the OMD iterates, all that remains is to upper bound the regret of OMD with the regret of Be-the-regularized-leader.
    \begin{lemma}[Lemma 9.2 of \cite{BartokLecNotes2011}]
        \label{lem: Be Regularized leader regret}
        Letting $D(\bm{q} || \bm{q}')$ denote the unnormalized KL divergence between $\bm{q}$ and $\bm{q}'$, we have:
        \begin{align*}
            \textsc{Regret}_\mathcal{B} \leq \max_{\bm{q} \in\mathcal{Q}} \mathbb{E}\Big[\eta^{-1} D(\bm{q} || \bm{q}^1) + \sum_{\nround=1}^\Nround \langle \bm{q}^\nround - \tilde{\bm{q}}^{\nround+1}, \bm{w}^\nround \rangle\Big]\,.
        \end{align*}
    \end{lemma} 
     The remainder of the regret analysis closely follows that of Theorem 1 in \cite{OREPS2013}. At a high level, we want to bound the regret of Online Mirror Descent by the regret of the unconstrained Be the
     (Negentropy) Regularized leader, via Lemma~\ref{lem: Be Regularized leader regret} (see Lemma 13 of \cite{LectureNotes2009} for the more general statement and proof of this lemma). 
     We then upper the contribution of the summation term by using the specific definition of the node weight estimators. Similarly, we upper bound the divergence term as a function of the dimension of the space $\mathcal{Q}$.
     
    
    
    To begin, note that our node utility estimators $\widehat{w}_\nitem^\nround(b)$ are unbiased:
    \begin{align}
        \mathbb{E}_{\bm{b} \sim \bm{\pi}^\nround}[\widehat{w}_\nitem^\nround(b)] = \mathbb{E}_{\bm{b} \sim \bm{\pi}^\nround}[\frac{w_\nitem^\nround(b)}{q^{\nround}_\nitem(b)} \textbf{1}_{b = b^{\nround}_\nitem}] = \frac{w_\nitem^\nround(b)}{q^{\nround}_\nitem(b)} \prob_{\bm{b} \sim \bm{\pi}^\nround} (b = b^\nround_\nitem) = \frac{w_\nitem^\nround(b)}{q^{\nround}_\nitem(b)} q^{\nround-1}_\nitem(b) = w_\nitem^\nround(b)\ .
        \label{proof: part1}
    \end{align}
    Now, consider the right hand side of the inequality in  Lemma \ref{lem: Be Regularized leader regret}. As the node utility estimators are unbiased, so we can replace $\bm{w}^\nround$ with $\widehat{\bm{w}}^\nround$. 
    Now, as per Lemma \ref{lem: Be Regularized leader regret}, we can upper bound the expected estimated regret as a function of the unconstrained optimizer $\tilde{\bm{q}}^{\nround+1}$ and the unregularized relative entropy  with respect to the initial state-edge occupancy measure $\bm{q}^1$. Applying the aforementioned lemma to Equation \eqref{proof: part1}, we obtain:
    \begin{align}
        \textsc{Regret}_\mathcal{B} = \max_{\bm{q} \in \mathcal{Q}} \mathbb{E}\left[\sum_{\nround=1}^\Nround \langle \bm{q}^{\nround} - \bm{q}, -\widehat{\bm{w}}^{\nround} \rangle \right] \leq \max_{\bm{q} \in \mathcal{Q}}\mathbb{E}\left[\sum_{\nround=1}^\Nround \langle \bm{q}^{\nround} - \tilde{\bm{q}}^{ \nround+1}, -\widehat{\bm{w}}^{\nround} \rangle + \eta^{-1}D(\bm{q} || \bm{q}^{1}) \right]
    \end{align}
    Applying $\exp(x) \geq 1 + x$ for $x = \exp(\eta \widehat{\bm{w}}^{\nround})$, we obtain $\tilde{\bm{q}}^{ \nround+1} = \bm{q}^\nround \exp(\eta \widehat{\bm{w}}^{\nround}) \geq \bm{q}^{\nround} + \eta \bm{q}^{\nround} \widehat{\bm{w}}^{\nround}$, which yields $\bm{q}^t - \bm{q}^t\exp(\eta\widehat{\bm{w}}^t) \ge  -\eta \bm{q}^t \widehat{\bm{w}}^t$. Plugging this back in:
    \begin{align}
        \textsc{Regret}_\mathcal{B} &\leq \max_{\bm{q} \in \mathcal{Q}} \mathbb{E}\left[\sum_{\nround=1}^\Nround \langle \bm{q}^{\nround} - \bm{q}^{\nround} \exp(\eta \widehat{\bm{w}}^{\nround}), -\widehat{\bm{w}}^{\nround} \rangle + \eta^{-1}D(\bm{q} || \bm{q}^{1}) \right]\\
        &\le  \max_{\bm{q} \in \mathcal{Q}} \mathbb{E}\left[\eta \sum_{\nround=1}^\Nround \sum_{\nitem = 1}^\Nitem \sum_{b \in \mathcal{B}} q^{\nround}_\nitem(b) \widehat{w}^{\nround}_\nitem(b)^2 + \eta^{-1}D(\bm{q} || \bm{q}^{1}) \right] \,.\label{eq: node diff}
    \end{align}
    Note that $\widehat{w}^{\nround}_\nitem(b) = \frac{w_\nitem^\nround(b)}{q^{\nround-1}_{\nitem}(b)} \textbf{1}_{b = b^{\nround}_{\nitem}}$ for all $\nitem \in [\Nitem]$ and $b \in \mathcal{B}$ by definition. Since $w^{\nround}_\nitem(b) \leq 1$ and $\textbf{1}_{b = b^{\nround}_{\nitem}} \leq 1$ we have $\widehat{w}^{\nround}_\nitem(b) \leq \frac{1}{q^{\nround}_\nitem(b)}$ and we continue the above chain of inequalities with:
    \begin{align}
        \textsc{Regret}_\mathcal{B} &\leq \max_{\bm{q} \in \mathcal{Q}} \mathbb{E}\left[\eta \sum_{\nround=1}^\Nround \sum_{\nitem = 1}^\Nitem \sum_{b \in \mathcal{B}} q^{\nround}_{\nitem}(b) \widehat{w}^{\nround}_\nitem(b) \frac{1}{q^{\nround}_{\nitem}(b)}  + \eta^{-1}D(\bm{q} || q^{1}) \right] \label{eq: full info difference appendix}\\
        &= \max_{\bm{q} \in \mathcal{Q}} \mathbb{E}\left[\eta \sum_{\nround=1}^\Nround \sum_{\nitem = 1}^\Nitem \sum_{b \in \mathcal{B}} \widehat{w}^{\nround}_\nitem(b)  + \eta^{-1}D(\bm{q} || \bm{q}^{1}) \right] \, .
    \end{align} 
    
    Recalling that $D(\bm{q} || \bm{q}^1) = \sum_{\nitem \in [\Nitem], b \in \mathcal{B}} q_\nitem(b)\log\frac{ q_\nitem(b)}{q^1_\nitem(b)} - (q_\nitem(b) - q^1_\nitem(b))$, we note that:
    
    \begin{align*}
        D(\bm{q} || \bm{q}^1) &= \sum_{m = 1}^M \sum_{b \in \mathcal{B}} q_m(b)\frac{\log q_m(b)}{\log q^1_m(b)} - q_m(b) + q^1_m(b) \\
        &= \sum_{\nitem=1}^\Nitem \sum_{b \in \mathcal{B}} q_{\nitem}(b)\log q_m(b) - q_m(b)\log q^1_m(b)\,,
    \end{align*} 
    where in the second equality, we used the fact that the elements both $\bm{q}$ and $\bm{q}^1$ all sum to $M$. Selecting $\bm{q}^1_m(\cdot)$ to be the uniform distribution over all $b \in \mathcal{B}$ and using the fact that the entropy of a discrete distribution over $|\mathcal{B}|$ items is $\log |\mathcal{B}|$, we obtain:
    \begin{align*}
        D(\bm{q} || \bm{q}^1) &= -\sum_{\nitem=1}^\Nitem H(\bm{q}_m) + \log |\mathcal{B}|\sum_{\nitem=1}^\Nitem \sum_{b \in \mathcal{B}} q_{\nitem}(b) \\
        &\leq \sum_{\nitem=1}^\Nitem \log |\mathcal{B}| + \log |\mathcal{B}|\sum_{\nitem=1}^\Nitem \sum_{b \in \mathcal{B}} q_{\nitem}(b) = \Theta(M\log|\mathcal{B}|)\,,
    \end{align*}
    where $H(\bm{x}) = -\sum_{x \in \bm{x}} x \log x $ denotes the discrete entropy function.  
    Plugging this back in:
    \begin{align*}
        \textsc{Regret}_\mathcal{B} \leq \mathbb{E}\left[\eta \sum_{\nround=1}^\Nround \sum_{\nitem = 1}^\Nitem \sum_{b \in \mathcal{B}} \widehat{w}^{\nround}_\nitem(b)  + \eta^{-1}\Nitem \log |\mathcal{B}| \right] \leq \eta \sum_{\nround=1}^\Nround \sum_{\nitem=1} \sum_{b \in \mathcal{B}} w^{\nround}_\nitem(b) + \eta^{-1}\Nitem \log |\mathcal{B}| = \eta \Nround \Nitem |\mathcal{B}| + \eta^{-1}\Nitem \log |\mathcal{B}|\,,
    \end{align*}
    where we used unbiasedness of $\widehat{\bm{w}}^{\nround}$. Setting $\eta = \sqrt{\frac{\log |\mathcal{B}|}{|\mathcal{B}|T}}$, we obtain $\textsc{Regret}_\mathcal{B}(\Nround) \leq \Nitem \sqrt{|\mathcal{B}| \Nround \log |\mathcal{B}|}$. 

    
    \textbf{Part 2: Determining Policy $\bm{\pi}$ from Node Probability Measures $\bm{q}$.} 
    Notice that in our regret analysis for both the bandit and full information setting, we do not require explicit knowledge of the policy $\bm{\pi}^t$, so long as it generates the desired node occupancy measure $\bm{q}^t$. In particular, we require a method of converting $\bm{q}^\nround$ to policy $\bm{\pi}^\nround$ which, in turn, is required in order to sample $\bm{b}^\nround$. Recall from Lemma~\ref{lem: QSpace Equivalence} that the mapping from the space of policies $\Pi$ to the space of node weight measures $\mathcal{Q}_\Pi = \mathcal{Q}$ is injective. Thus, for any $\bm{q} \in \mathcal{Q}$, there must exists a $\bm{\pi} \in \Pi$ such that $q(\bm{\pi}) = \bm{q}$. Moreover, the set $\Pi(\bm{q})$ of such $\bm{\pi}$ can be written as the intersection of two polyhedrons, and hence a polyhedron, from which a feasible solution can be computed efficiently (e.g., ellipsoid method), where  $ \Pi(\bm{q})$ is the set of policies $\pi \in [0,1]^{M\times |\mathcal B|\times |\mathcal B|}$ such that 
    \begin{itemize}
        \item $\pi((0, \max \mathcal{B}), b) = q_1(b)$, for any $b \in \mathcal{B}$;
        \item $\pi((0, b), b') = \textbf{1}_{b = b'}$ for any $b, b' < \max \mathcal{B}$;
        \item $q_{m+1}(b') = \sum_{b \in \mathcal{B}} q_m(b) \pi((m, b), b')\}$ for any $b'\in \mathcal B$ and $m \in [M-1]$.
    \end{itemize}



    \textbf{Part 3: Complexity analysis.} One may wonder how to efficiently update the state occupancy measures by computing the minimizer of $\eta\langle \bm{q}, -\widehat{\bm{w}}^\nround\rangle + D(\bm{q} || \bm{q}^{\nround-1})$. The idea is to first solve the unconstrained entropy regularized minimizer with $\tilde{\bm{q}}^{ \nround+1} = \bm{q}^{\nround} \exp(\eta \widehat{\bm{w}}^{\nround})$. We then project this unconstrained minimizer to $\mathcal{Q}$ with:
    \begin{align}
        \bm{q}^{\nround + 1} = \text{argmin}_{\bm{q} \in \mathcal{Q}} D(\bm{q}||\tilde{\bm{q}}^{\nround + 1})\,.
    \end{align}
    Relegating the details to \cite{OREPS2013}, the above constrained optimization problem can be solved as the minimizer of an equivalent unconstrained convex optimization problem with a polynomial (in $\Nitem$ and $|\mathcal{B}|$) number of variables, and therefore, can be computed efficiently. Combining with finding an initial feasible solution to $\Pi(\bm{q})$ as well as the optimization step, we achieve polynomial in $\Nitem, |\mathcal{B}|, \Nround$ total time complexity. For the space complexity, we only need store the values of $\bm{\pi}^\nround$, $\bm{q}^\nround$, and $\widehat{\bm{w}}^\nround$, for a total space complexity of $O(\Nitem |\mathcal{B}|^2)$.   
    
    \textbf{Part 4: Continuous Regret.} To obtain the continuous regret, recall that the discretization error is $O(\frac{\Nitem \Nround}{|\mathcal{B}|})$. As the discretized regret is $O\left(\Nitem \sqrt{|\mathcal{B}| \Nround \log |\mathcal{B}|}\right)$ in the bandit feedback setting, the optimal choice of $|\mathcal{B}|$ is $\Theta(\Nround^{\frac{1}{3}})$, which achieves continuous regret $\textsc{Regret} = O(\Nitem \Nround^{\frac{2}{3}} \sqrt{\log \Nround})$.

\end{proof}

\subsubsection{Proof of Corollary \ref{cor}}
\label{sec: Proof of cor}

We can straightforwardly extend Algorithm~\ref{alg: OMD} to the full information setting. To do this, we note that we can improve Equation~\eqref{eq: full info difference appendix} by instead replacing $\widehat{\bm{w}}^{\nround}$ with $\bm{w}^{\nround}$ in Equation~\eqref{eq: node diff} to obtain:
\begin{align*}
    \sum_{\nround=1}^\Nround \sum_{\nitem=1}^\Nitem \sum_{b \in \mathcal{B}} q^{\nround}_\nitem(b) \widehat{w}^{\nround}_\nitem(b)^2 = \sum_{\nround=1}^\Nround \sum_{\nitem=1}^\Nitem \sum_{b \in \mathcal{B}} q^{\nround}_\nitem(b) w^{\nround}_\nitem(b)^2 \leq \sum_{\nround=1}^\Nround \sum_{\nitem=1}^\Nitem \sum_{b \in \mathcal{B}} q^{\nround}_\nitem(b) = \sum_{\nround=1}^\Nround \sum_{\nitem=1}^\Nitem 1 = \Nround \Nitem\,.
\end{align*}
Setting $\eta = \sqrt{\frac{ \log |\mathcal{B}|}{T}}$, we obtain in the full information setting $\textsc{Regret}_\mathcal{B} = O(\Nitem \sqrt{\Nround \log |\mathcal{B}|})$. We can also compute the optimal choice of $|\mathcal{B}|$ to obtain optimal continuous regret. Using the optimal choice of $|\mathcal{B}|$ being $\Theta(\sqrt{T})$, we achieve continuous regret of $\textsc{Regret} = O(\Nitem \sqrt{\Nround \log \Nround})$. Note that due to the complexity of the optimization sub-routine in the projection step of OMD, for the full information setting, it is preferable to use Algorithm~\ref{alg: Decoupled Exponential Weights} instead.









{\color{black}
\subsection{Proof of Theorem \ref{thm: time varying known finite}}
We now prove the regret bounds of the contextualized version of our decoupled hedge algorithm (Algorithm~\ref{alg: Decoupled Exponential Weights - Time Varying Known Finite}) to handle time-varying valuations.
To begin, we can once again `decouple' the utility per unit-bid pair, but this time conditional on the valuation vector context. In particular, we have:
\begin{align*}
    \mu_n^t(\bm{b}; \bm{v}) =  \sum_{m=1}^M w_m^t(b_m; \bm{v}) = \sum_{m=1}^M (v_m - b_m)1_{b_m \geq b^t_{-m}} \quad \text{and} \quad \widehat{\mu}_n^t(\bm{b}; \bm{v}) =  \sum_{m=1}^M \widehat{w}_m^t(b_m; \bm{v})\,.
\end{align*}
As stated earlier, we define reward estimates based on Equation (6) of \cite{ContextBanditsCrossLearning2019} and our Algorithm~\ref{alg: Decoupled Exponential Weights - Path Kernels}: 
\begin{align*}
    \widehat{w}_m^t(b; \bm{v}) = 1 - \frac{1 - w_m^t(b; \bm{v})}{\sum_{\bm{v} \in \mathcal{V}} \prob(\bm{v}^t = \bm{v}) q_m^t(b; \bm{v})} \textbf{1}_{b_m^t = b} = 1 - \frac{1 - w_m^t(b; \bm{v})}{Q_m^t(b)} \textbf{1}_{b_m^t = b}\,.
\end{align*}
Here, $q_m^t(b; \bm{v}) = \prob(b^t_m = b | \bm{v}^t = \bm{v}) = \sum_{\bm{b}: b^t_m = b} \prob(\bm{b}^t = \bm{b} | \bm{v}^t = \bm{v})$ is the probability of selecting bid $b$ in slot $m$ with valuation $\bm v$. Similarly, $Q_m^t(b)$ is the probability of selecting bid $b$ for unit $m$, averaged across all possible valuations. One can verify unbiasedness of this estimator $\mathbb{E}[\hat{w}_m^t(b; \bm{v})] = w_m^t(b; \bm{v})$ for all $m \in [M], b \in \mathcal{B}, \bm{v} \in \mathcal{V}$. The second moment can similarly be computed as:
\begin{align*}
    \mathbb{E}[\widehat{w}_m^t(b; \bm{v})^2] = \mathbb{E}\left[\left( 1 - \frac{1-w_m^t(b; \bm{v})}{Q_m^t(b)} \textbf{1}_{b_m^t = b} \right)^2 \right] = 1 - 2\mathbb{E}\left[\frac{1-w_m^t(b; \bm{v})}{Q_m^t(b)}\textbf{1}_{b_m^t=b}\right] + \mathbb{E}\left[\left(\frac{1-w_m^t(b; \bm{v})}{Q_m^t(b)}\right)^2\textbf{1}_{b_m^t=b}\right]\,.
\end{align*}

Evaluating the expectations and recalling that $\mathbb{E}[\textbf{1}_{b_m^t = b}] = Q_m^t(b)$, we have:
\begin{align*}
    \mathbb{E}[\widehat{w}_m^t(b; \bm{v})^2] = 1 - \left[2 - 2w_m^t(b; \bm{v})\right] + \left[\frac{(1 - w_m^t(b; \bm{v}))^2}{Q_m^t(b)}\right] = 2w_m^t(b; \bm{v}) - 1 + \frac{1}{Q_m^t(b)} \leq 1 + \frac{1}{Q_m^t(b)} \leq \frac{2}{Q_m^t(b)}\,.
\end{align*}

Using this, the proof largely follows that of Algorithm~\ref{alg: Decoupled Exponential Weights - Path Kernels} up until Equation (\ref{eq: full info difference}). In particular, we have that the contextual regret can be written as:
\begin{align*}
    \textsc{Regret}_\mathcal{B}(F_{\bm{v}}) &= \mathbb{E}_{F_{\bm{v}}}\left[\sum_{t=1}^{T} \mu_n^t(\bm{b}'; \bm{v}^t) - \sum_{t=1}^{T} \mathbb{E}[\mu^t(\bm{b}^\nround; \bm{v}^t)]\right]\\
    &\lesssim \eta^{-1}M\log|\mathcal{B}| + \eta \mathbb{E}_{F_{\bm{v}}}\left[\sum_{\nround=1}^\Nround \sum_{\bm{b}} \prob(\bm{b}^\nround =\bm{b}| \bm{v}^t = \bm{v}) \mathbb{E}[(\sum_{m=1}^M \widehat{w}^\nround_m(b_m; \bm{v}))^2]\right]\\
    &= \eta^{-1}M\log|\mathcal{B}| + \eta \left[\sum_{\nround=1}^\Nround \sum_{\bm{v} \in \mathcal{V}} \prob(\bm{v}^t = \bm{v})\sum_{\bm{b}} \prob(\bm{b}^\nround =\bm{b}| \bm{v}^t =  \bm{v}) \mathbb{E}[(\sum_{m=1}^M \widehat{w}^\nround_m(b_m; \bm{v}))^2]\right]\\
    &= \eta^{-1}M\log|\mathcal{B}| + \eta M\left[\sum_{\nround=1}^\Nround \sum_{m=1}^M \sum_{\bm{v} \in \mathcal{V}} \prob(\bm{v}^t = \bm{v})\sum_{b \in \mathcal{B}} \mathbb{E}[\widehat{w}_m^t(b; \bm{v})^2]  \sum_{\bm{b}: b_m = b} \prob(\bm{b}^\nround =\bm{b}| \bm{v}^t = \bm{v})\right]\\
    &= \eta^{-1}M\log|\mathcal{B}| + \eta M\left[\sum_{\nround=1}^\Nround \sum_{m=1}^M \sum_{\bm{v} \in \mathcal{V}} \prob(\bm{v}^t = \bm{v})\sum_{b \in \mathcal{B}} \mathbb{E}[\widehat{w}_m^t(b; \bm{v})^2]  q_m^t(b; \bm{v})\right]\\
    &= \eta^{-1}M\log|\mathcal{B}| + 2\eta M\left[\sum_{\nround=1}^\Nround \sum_{m=1}^M \sum_{\bm{v} \in \mathcal{V}} \prob(\bm{v}^t = \bm{v})\sum_{b \in \mathcal{B}} \frac{1}{Q_m^t(b)}  q_m^t(b; \bm{v})\right]\\
    &= \eta^{-1}M\log|\mathcal{B}| + 2\eta M\left[\sum_{\nround=1}^\Nround \sum_{m=1}^M \sum_{b \in \mathcal{B}} \frac{1}{Q_m^t(b)} \sum_{\bm{v} \in \mathcal{V}} \prob(\bm{v}^t = \bm{v}) q_m^t(b; \bm{v})\right]\\
    &= \eta^{-1}M\log|\mathcal{B}| + 2\eta M\left[\sum_{\nround=1}^\Nround \sum_{m=1}^M \sum_{b \in \mathcal{B}} \frac{1}{Q_m^t(b)} Q_m^t(b)\right]\\
    &\leq \eta^{-1}M\log|\mathcal{B}| + \eta M^2|\mathcal{B}|T\,.
\end{align*}
(We will show the first inequality shortly.)
With $\eta = \Theta(\sqrt{\frac{\log |\mathcal{B}|}{M|\mathcal{B}|T}})$,  this yields  the  discretized contextual regret upper bounds of $O(M^{\frac{3}{2}}\sqrt{|\mathcal{B}| T \log |\mathcal{B}|})$ under the bandit setting. Accounting for the rounding error of order $O(\frac{MT}{|\mathcal{B}|})$, we obtain the stated continuous contextual regret upper bounds by optimizing with $|\mathcal{B}| = M^{-\frac{1}{3}}T^{\frac{1}{3}}$. 
To obtain the full information results, we simply replace $\widehat{w}_m^t(b_m; {\bm v}^t)$ with $w_m^t(b_m; {\bm v}^t)$ in the second line of the above equations, which leads to  the discretized contextual regret upper bounds of $O(M^{\frac{3}{2}}\sqrt{ T \log |\mathcal{B}|})$, as desired.


Next, following the proof of Algorithm \ref{alg: Decoupled Exponential Weights - Path Kernels}, we show   the first inequality. 
We define the potentials with respect to a fixed valuation vector $\bm{v}$: $\Phi^t(\bm{v}) = \sum_{\bm{b} \in \mathcal{B}^{+\Nitem}} \exp(\eta \sum_{\tau=1}^{t} \widehat{\mu}^\tau(\bm{b}; \bm{v}^\tau))$. Taking the ratio of adjacent terms, we obtain:
\begin{align*}
    \frac{\Phi^t(\bm{v})}{\Phi^{t-1}(\bm{v})} = \sum_{\bm{b} \in \mathcal{B}^{+M}} \frac{\exp(\eta \sum_{\tau=1}^{t-1} \widehat{\mu}^\tau(\bm{b}; \bm{v}^\tau))}{\Phi^{t-1}(\bm{v})} \exp(\eta \widehat{\mu}^t(\bm{b}; \bm{v}^t)) = \sum_{\bm{b} \in \mathcal{B}^{+M}} \prob(\bm{b}^\nround = \bm{b} | \bm{v}^\nround = \bm{v}) \exp(\eta \widehat{\mu}^t(\bm{b}; \bm{v}^\nround))\,,
\end{align*}
Where in the last equality, we used the condition that our algorithm samples bid vector $\bm{b}$ with probability proportional to $\exp(\eta \sum_{\tau=1}^{t-1} \widehat{\mu}^\nround(\bm{b}; \bm{v}))$ at round $\nround$ with valuations $\bm{v}^t = \bm{v}$. Combining this with inequalities $\exp(x) \leq 1 + x + x^2$ and $1 + x \leq \exp(x)$ for all $x \leq 1$, we obtain:
\begin{align*}
    \frac{\Phi^t(\bm{v})}{\Phi^{t-1}(\bm{v})} \leq \sum_{\bm{b} \in \mathcal{B}^{+M}} \prob(\bm{b}^\nround = \bm{b} | \bm{v}^\nround = \bm{v}) \exp(\eta \widehat{\mu}^t(\bm{b}; \bm{v})) \leq \exp(\sum_{\bm{b} \in \mathcal{B}^{+M}} \prob(\bm{b}^\nround = \bm{b} | \bm{v}^\nround = \bm{v}) \left[\eta\widehat{\mu}^t(\bm{b}; \bm{v}) + \eta^2 \widehat{\mu}^t(\bm{b}; \bm{v})^2 \right])\,.
\end{align*}
Combining this with Equations~\eqref{eq: Potentials} and the fact that $\Phi^0(\bm{v}) = M\log |\mathcal{B}|$, for any fixed bid vector $\bm{b}'$, we have:
\begin{align*}
    \sum_{t=1}^{T} \widehat{\mu}^t(\bm{b}'; \bm{v}) - \sum_{t=1}^{T} \sum_{\bm{b}} \prob(\bm{b}^\nround = \bm{b}| \bm{v}^t = \bm{v})  \widehat{\mu}^t(\bm{b}; \bm{v}) &\lesssim \eta^{-1} \Nitem \log |\mathcal{B}| + \eta \sum_{t=1}^T \sum_{\bm{b}}\prob(\bm{b}^\nround = \bm{b} | \bm{v}^\nround = \bm{v}) \widehat{\mu}^t(\bm{b}; \bm{v})^2\\
    &= \eta^{-1} \Nitem \log |\mathcal{B}| + \eta \sum_{t=1}^T \sum_{\bm{b}}\prob(\bm{b}^\nround = \bm{b} | \bm{v}^\nround = \bm{v}) (\sum_{m=1}^M \widehat{w}_m^t(b_m; \bm{v}))^2\,.
\end{align*}
Taking expectations over $\bm{b}$ and the supremum over all $\bm{b}'$ yields the desired first crucial regret inequality.

As for the time and space complexity, notice that the only algorithmic difference between Algorithm~\ref{alg: Decoupled Exponential Weights - Time Varying Known Finite} and Algorithm~\ref{alg: Decoupled Exponential Weights - Path Kernels} is precisely in computing the estimator, which in the former, requires having to compute the weights $Q_m^t(b)$ by iterating over all $\bm{v} \in \mathcal{V}$. As we also have to store reward estimates for each possible valuations, both the time complexity and space complexity of Algorithm~\ref{alg: Decoupled Exponential Weights - Time Varying Known Finite} are a factor $|\mathcal{V}|$ larger than in Algorithm~\ref{alg: Decoupled Exponential Weights - Path Kernels}, which are $O(M|\mathcal{B}| |\mathcal{V}| T)$ and $O(M|\mathcal{B}| |\mathcal{V}|)$ respectively.


}



\section{Online Appendix - Additional Discussion and Experiments}

In this section, we run several additional experiments and include further discussion of our previous results as well as these new experiments. First, we discuss the interaction between the underlying parameters $N, M, |\mathcal{B}|$ and the $c = c_{-n}$ condition required to guarantee existence of a PNE in Theorem~\ref{thm: PNE existence}. We then justify our use of the $\textsc{EXP3-IX}$ weight estimator rather than the unbiased estimator in the implementation of our algorithms in Section~\ref{sec: experiments}. We also discuss a method that achieves uniform exploration per item as per the $\textsc{EXP3.P}$ algorithm described in \cite{Lattimore2020}. Thirdly, we include several experiments omitted in the main body regarding faster convergence with a larger, more competitive market. We then conclude with a discussion of the practicality of the PAB vs. Uniform Price auction formats from the market design perspective.


\subsection{Nash Equilibrium Existence}

\label{sec: Nash equilibrium existence further discussion}

The assumption of competitiveness $c = c_{-n}$ used in Theorem~\ref{thm: PNE existence} can be relaxed. The key idea behind this assumption is that no individual bidder $n$ cannot lower their bids such that their decreased payment offsets their decreased allocation. To see this in effect without requiring the $c = c_{-n}$ condition, consider the following example. 
\begin{enumerate}
    \item Let $N=3, M=3$, and $\mathcal{B} = \{\frac{i}{10}\}_{i \in [10]}$.
    \item Let $v_1 = [1, 0, 0]$, $v_2 = [1, 0.7 - \epsilon, 0], v_3 = [1, 0.4, 0]$ for some small $\epsilon > 0$.
    \item Here, $c = 1$ but $c_{-1} = c_{-3} = 0.7 - \epsilon \neq c_{-2} = 0.4$. Despite $c \neq c_{-n}$, we have that a PNE exists in the form $[0.6, 0, 0], [0.5, 0.5, 0], [0.5, 0.4, 0]$. To verify that this is indeed a PNE, each of the bidders obtains an allocation of 1, with corresponding utilities of 0.4, 0.5, and 0.5 respectively. Considering only uniform winning bids, which contain the set of optimal responses as per Lemma~\ref{lem: near-uniform optimal bidding}, we see the bidders are bidding at Nash:
    \begin{enumerate}
        \item Bidder 1 only demands one item. Here is $b_{2,2} = 0.5$ is bidder 1's $M$'th largest competing bid. Since this belongs to a higher tie-break priority bidder, they must bid strictly higher, yielding a clearing price of 0.6 for bidder 1. Thus, bidder 1 cannot increase or decrease their bids and therefore must be playing their optimal response.
        \item Bidder 2 can win two items at a price of 0.6, which yields utility $0.5-\epsilon < 0.5$. Similarly, in order to win one item, their $M$'th highest competing bid is $b_{3,2} = 0.4$. Since this belongs to a higher tie-break priority bidder, they must bid strictly higher, yielding a clearing price of 0.5. Thus, bidder 2 cannot increase or decrease their bids and therefore must be playing their optimal response.
        \item Bidder 3 can win two items at a price of 0.6, which yields utility $0.2 < 0.5$. Similarly, in order to win one item, their $M$'th highest competing bid is $b_{2,2}=0.5$. Since this belongs to a lower tie-break priority bidder, this yields a clearing price of $0.5$ for bidder 3. Thus, bidder 3 cannot increase or decrease their bids and therefore must be playing their optimal response.
    \end{enumerate}
\end{enumerate}

We have shown that the $c = c_{-n}$ condition is not necessary for PNE existence. Unfortunately, giving a simple characterization of conditions that guarantee PNE existence is non-trivial. However, as mentioned at the beginning of this section, any equilibrium bids $(\bm{b}_1^*,\ldots,\bm{b}_N^*)$ requires that individual bidders cannot i) profitably sacrifice allocation in order to reduce costs and ii) profitably increase their bids to increase allocation. Let $\tilde{b}_{-n,m}$ be as in Lemma~\ref{lem: near-uniform optimal bidding}: $\tilde{b}_{-n,m}$ denotes bidder $n$'s $m$'th smallest competing bid rounded up to the next multiple of $\frac{1}{|\mathcal{B}|}$ if belonging to bidder $n' > n$ due to tie-breaking. In other words, submitting $\tilde{b}_{-n,m}$ for the first $m$ items minimizes the cost required to win $m$ items. As such, we can construct an efficient frontier $\bm{\mu}_n(\bm{b}_{-n}^*) \doteq \{\sum_{i=1}^m v_{n,i} - m\tilde{b}_{-n,m}\}_{m \in [M]}$ of bidder $n$'s utility by minimizing the cost required to win $m$ items. If this frontier is maximized at $\bm{b}_n^*$ for all $n$, then $(\bm{b}_1^*,\ldots,\bm{b}_N^*)$ constitutes a PNE. More formally, a PNE exists at $(\bm{b}_1^*,\ldots,\bm{b}_N^*)$ if:
\begin{align}
    b_{n,m}^* = \min\left(\tilde{b}_{-n,m^*}, \lfloor v_{n,m} \rfloor_{\delta}\right) \quad \text{where} \quad m^* = \text{argmax}_{m \in [M]} \bm{\mu}_n(\bm{b}_{-n}^*) \quad \forall m \in [M], n \in [N]\,.
\end{align}

With this, we can interpret the relaxed condition as saying that bidders prefer to win as many units as possible (subject to positive marginal per-unit utility) over winning fewer units at a reduced cost. This can equivalently interpreted as bidders being price-takers and having little manipulative power over the market price. While even this weaker assumption can be easily violated in artificial examples, e.g., the example in our bid convergence remark, we claim that it is a reasonable assumption in many of the PAB auction's real world applications and markets.
 
\begin{enumerate}
    \item Bidders only have high utility for a small number of units, i.e., the number of units $M$ each bidder $n$ demands is much smaller than the supply $\overline{M}$.
    This is reasonable in many relevant markets, e.g. electricity or emissions, as the total supply far exceeds any single firm's power usage or pollution capacity. 
    \item The total supply is smaller than the aggregate demand, i.e., $\sum_{n \in [N]} \overline{M}_n \gg M$.
    For example, electricity supply in the US in the early 2020's has been strained due to increasing power requirements across many industries, due in part to the rise of energy intensive AI technologies, cloud computing, and cryptocurrency mining. This is also the case in carbon markets, the supply $M$ is artificially limited so as to keep the prices from falling too low and becoming too weak of a disincentive (e.g. carbon markets) to pollute.
\end{enumerate}
In conjunction, these conditions imply that no individual bidder can sufficiently impact prices as their demand is too small compared to the aggregate demand and supply, thus, their reduced costs from underbidding is outweighed by their reduction in allocation. We do note that this result differs from existing characterizations of efficient Pure Nash Equilibria in PAB auctions \cite{inefficiency2013}. However, the Nash equilibria described in the latter require overbidding (bidding above one's marginal valuations) for units, so long as they are guaranteed not to win one of these units. In particular, if every bidder submits the same vector of $[b,\ldots,b]$ where $b$ is the $M$'th largest valuation among all bidders, and ties are broken in favor of bidders with the highest value, then this is also an efficient Pure Strategy Nash equilibrium. 

To illustrate this more quantitatively, we analyze the effects of changing $M, |\mathcal{B}|, N$ on the probability that a PNE (assuming no overbidding) exists in Figure~\ref{fig: PNE existence vs M B N}. Of course as $|\mathcal{B}|$ gets smaller and the number of agents $N$ becomes larger, the probability that the rounded-down $M$'th highest-other-valuation is equal for all $n$ increases. Curiously, the probability is also increasing, albeit slowly, as a function of $M$.

% Figure environment removed



\subsection{$\textsc{EXP3-IX}$ vs. Unbiased Reward Estimator}

\label{sec: IX}

In the experiments section, we ran a slightly modified version of our existing algorithms in the bandit feedback setting. We do this as the variance of the accumulated regret of our algorithms are high, as the node weight estimators normalize over vanishingly small probabilities $q^t_m(b)$. To mitigate the effect of such normalization, we use the $\textsc{EXP3-IX}$ estimator as described in \cite{Lattimore2020}. Under this estimator, rather than normalizing the probability of selecting bid $b^t_m$ for unit $m$ at time $t$ by $q^t_m(b_m^t)$, we instead normalize it by $q^t_m(b_m^t) + \gamma$ for some constant $\gamma > 0$. In the standard $K$-armed bandit setting, despite being a biased estimator, still achieves the same sublinear expected regret guarantee  with a smaller variance. This smaller variance indeed allows for stronger high probability guarantees on the magnitude of our regret; i.e., for $\delta > 0$ and $\gamma = \sqrt{\frac{\log(K) + \log(\frac{K+1}{\delta})}{4K\Nround}}$, the $\textsc{EXP3-IX}$ algorithm guarantees with probability at least $1 - \delta$ that the regret is upper bounded by $C\sqrt{KT\log K}$ for some absolute constant $c > 0$. We extend this algorithm to the multi-unit PAB setting algorithms, where for each node $(m, b)$, we set $\gamma = \sqrt{\frac{\log(K) + \log(\frac{K+1}{\delta})}{4K\Nround}}$ and $K = |\{b \in \mathcal{B}: b \leq v_m\}|$, for $\delta = 0.05$. Aside from the change in node weight estimators, the $\textsc{EXP3-IX}$ versions of Algorithms \ref{alg: Decoupled Exponential Weights - Path Kernels} and \ref{alg: OMD} are exactly the same.\\


\subsubsection{Empirical Performance of Original Algorithms vs. $\textsc{EXP3-IX}$ Variants}

In this section, we empirically analyze the modified variants of our algorithms which use the biased, but lower variance $\textsc{EXP3-IX}$ node-weight estimators (see Appendix~\ref{sec: IX}). We compare the distribution of the regret recovered by these modified algorithms versus the non-modified versions when the number of units is one. The bidder, endowed with valuation vector $\bm{v} = [1]$, will compete against a single adversary over the course of $T$ rounds for $\overline{M}=M=1$ item. This is the standard first price auction (FPA). Here, we compare performance when the adversary is stochastic (bids drawn uniformly random from $[0, 1]$) versus adaptive adversary (running the same algorithm, with a valuation drawn uniformly random from $[0, 1]$). 

We plot the regret of the bidder against the stochastic and adversarial competitors for moderate $T \in \{100, 500, 2000, 10000\}$. The stochastic adversary setting is shown in Figure~\ref{fig:eval_IX} (a) and the adversarial setting is shown in Figure~\ref{fig:eval_IX} (b). 
We observe that while the $\textsc{EXP3-IX}$ variants marginally worsens regret for small values of $T \in \{100, 500\}$ for both the stochastic and adaptive settings, it significantly mitigates the heavy tailed distribution of regret for large $T \in \{2000, 10000\}$, especially in the adversarial setting.



% Figure environment removed

\subsubsection{$\textsc{EXP3.P}$ vs. Unbiased Reward Estimator}

\label{sec: EXP3P}

One downside of the $\textsc{EXP3-IX}$ node weight estimator approach is that the added exploration aggregates over layers $m \in [M]$ in an uneven manner, as bid vectors must stay monotonic. As such, this per-node re-weighting does not guarantee convergence of the regret distribution in probability and only maintains the weaker guarantee over expectation convergence. To mitigate the effect of such uneven exploration aggregation, we instead modify the $\textsc{EXP3.P}$ algorithm as described in \cite{ Lattimore2020}. This algorithm explicitly mixes in uniform noise into the decisions made by the algorithm, and then normalizes accordingly. How does one sample uniformly from the exponentially large space of all monotone, individually rational bid vectors? We claim that the following procedure straightforwardly achieves such uniform mixing:
\begin{enumerate}
    \item Select $b_1$ uniformly at random from $\mathcal{B}$ subject to $b_1 < v_1$.
    \item If $b_m < v_{m+1}$, then set $b_{m+1} = b_m$. Otherwise, select $b_{m+1}$ uniformly at random from $\mathcal{B}$ subject to $b_{m+1} < v_{m+1}$.
\end{enumerate}
With a random exploration probability of $\gamma \in (0, 1)$, the $\textsc{EXP3.P}$ variant of our algorithm performs the above uniform exploration with probability $\gamma$ and follows the procedure in Algorithm~\ref{alg: Decoupled Exponential Weights - Path Kernels} otherwise. Now, we must also account for this in our node weight estimators. In particular, under the $\textsc{EXP3.P}$ variant, the probability of selecting bid $b^t_m$ for unit $m$ is given by $\hat{q}^t_m(b_m^t) = (1-\gamma)q^t_m(b_m^t) + \frac{\gamma}{|b \in \mathcal{B}: b < v_{m}|} = (1-\gamma)q^t_m(b_m^t) + \frac{\gamma \delta}{\lfloor v_{m}\rfloor_\delta}$. We then update the rewards of all bids $b < v_m$ in layer $m$ as:
\begin{align*}
    \widehat{W}^{\nround+1}_{\nitem}(b) \gets \widehat{W}^{\nround}_{\nitem}(b) + \frac{(v_m - b)\textbf{1}_{b \geq b^t_{-m}} + \beta_m}{\hat{q}^t_m(b)} \textbf{1}_{b^t_m = b})
\end{align*}
where $\beta_m = \Theta(\sqrt{\frac{\log(|\mathcal{B}| T/\delta)}{|\mathcal{B}|T}})$ for some high probability parameter $\delta \in (0, 1)$. In the standard $K$-armed bandits problem, the $\textsc{EXP3.P}$ algorithm guarantees that the regret is bounded above by $C\sqrt{KT\log(K/\delta)}$ with probability at least $1-\delta$ for some universal constant $C$. This high probability bound follows immediately from bounding the variance of the weight estimators. Of course, we cannot blindly apply this bound in our multi-unit setting, as there are an exponentially large number of possible bid vectors. Fortunately, we may apply the same utility decoupling trick as in the analyses of Algorithms~\ref{alg: Decoupled Exponential Weights - Path Kernels} and ~\ref{alg: OMD}. That is, we upper bound the variance of the reward estimate of a bid vector by the sum of the variances of each of its constituent bids, yielding a $1-\delta$ high probability bound of $CM^{\frac{3}{2}}\sqrt{|\mathcal{B}|T\log(|\mathcal{B}|/\delta)}$ in the bandit setting.


\subsection{Experiments with Larger $N$}

In this section, we empirically show the faster convergence of and superior welfare and revenue of more competitive PAB markets. More specifically, we evaluate the impact of competition by running the same revenue and welfare over time experiments for both the PAB and uniform price auctions with varying values of $N$. In Figure \ref{fig:comp}, we compare the distribution of welfare and revenue over time showing the 10th, 25th, 50th, 75th, and 90th percentiles for the bandit setting for a varying number of market participants $N \in \{2, 6, 12, 48\}$ with $T = 10^4$. Compared with the previous revenue and welfare plots with $N = 3$ bidders (Figure~\ref{fig:rev_wel_over_time}), we see that as $N$ grows larger, the welfare and revenue more smoothly, and with lower variance, increase towards their equilibrium values. In addition to the increased competition reducing the incentive for agents to strategically shade their bids, the final revenue is higher as expected.

% Figure environment removed


\subsection{Concluding Remarks}

In this work, we have shown that pay-as-bid (PAB) auctions have several advantages over its more commonly used uniform price counterpart. These advantages include simpler Nash bidding structure requiring only uniform bids (see Lemma~\ref{lem: PNE uniform bidding} and \ref{lem: near-uniform optimal bidding}) where this structure empirically extends to the dynamic setting. Additionally, we show that our utility decoupling insight enables online learning algorithms for PAB that are computationally and regret dominant compared to learning in uniform price auctions. Lastly, we showed that regardless of the parameterizations of $M, N, |\mathcal{B}|$, the PAB auction routinely out-performed uniform price in terms of revenue whilst remaining competitive in terms of welfare (see Table~\ref{table: learning dynamics full info} and ~\ref{table: learning dynamics full uniform}). This superior revenue and welfare trade-off can be seen more closely in Figure~\ref{fig: welfare_revenue_comparison_box_plot_intro}.

% Figure environment removed

Despite these computational and economic advantages of PAB over uniform price, we must still be careful in practice. For example, as PAB achieves higher revenue for the auctioneer, this conversely implies lower consumer surplus. Thus, switching from a uniform price auction to PAB may drive away participants from the platform and join a competing platform. While this is a non-issue for markets where the auctioneer holds a monopoly over a product---e.g., government issued pollution licenses or treasury bills--- this makes switching impractical in certain settings, such as electricity markets. As such, we leave the analysis of learning dynamics in such markets operating under uniform price or markets with multiple platforms for important future work.
% \section{Switching times}
% \input{response_letter}
% \section{Supplementary materials}
% \end{APPENDICES}
\end{document}
