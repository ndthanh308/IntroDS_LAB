\section{Concluding Remarks}


We have provided low-regret learning algorithms for PAB auctions in the full information and bandit settings with corresponding polynomial time and space complexities. In particular, we utilize our DP formulation and its equivalent graph representation to decouple the utility associated with bidding $b_\nitem = b$ for all $\nitem \in [\Nitem], b \in \mathcal{B}$. We derived two algorithms, one that mimics the exponential weights algorithm and another based on OMD, both of which allowed us to achieve polynomial (in $\Nitem$, $|\mathcal{B}|$, and $\Nround$) regret upper bounds, as well as time and space complexities, despite the combinatorially large bid space.
Furthermore, our experimental simulations highlight the convergence of winning bids in PAB auctions, coupled with higher revenue generation compared to uniform price auctions. However, we note a slight lag in welfare compared to uniform price auctions. Our findings provide actionable guidance for auction design and bidder strategy in multi-unit auctions, contributing to a deeper understanding of bidding dynamics and equilibria.


There are several intriguing avenues for future research that can be explored based on the current work. A promising direction is to leverage the structure induced by bid monotonicity in PAB auctions. 
Recent advancements in a simpler single-unit setting have demonstrated the efficacy of cross-learning between bids under certain feedback structures \citep{OptimalNoRegretFPA2020, LearningBidOptimallyAdversarialFPA2020}. It would be intriguing to investigate the potential benefits of applying cross-learning techniques  in our multi-unit setting. By incorporating such methods, we can explore whether they can enhance our regret bounds. Furthermore, inspired by our numerical results---where we show that the winning bids in PAB market dynamics converge to the same value---we can explore the design of online learning algorithms for the setting where bidders are restricted to a simplified bidding interface, wherein they are only allowed to submit a single  price and quantity for the units demanded rather than an entire vector of bids. 
Another potential research direction is to study conditions under which last iterate convergence for learning in PAB holds. As observed in Figure \ref{fig: bid cycling main body}, the learners can converge to a PNE, though proving this was only recently done by \cite{Deng_2022} for the class of \textit{mean-based} algorithms (defined in \cite{braverman2018noregretbuyer}) for the single unit, first price auction under certain conditions.
