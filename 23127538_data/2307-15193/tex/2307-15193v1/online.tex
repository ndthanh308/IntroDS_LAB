\section{Bid Optimization: Bandit Feedback}

In this section, we now consider the more realistic feedback setting where agent $n$ only observes $H^{n, \nround}$ regarding each auction $\nround$'s clearing price $c^\nround$ and allocation $x^{n, t} = x(\bm{b}^{n, \nround}, n, t)$. We first introduce a well known generalization of $\textsc{Exp3}$ that exploits cross-learning on feedback graphs, which we denote as $\textsc{Exp3.G}$. We then show how to represent the bandit bid optimization problem as an instance of graph-feedback online learning and derive corresponding regret bounds by bounding the independence number of all possible feedback graphs. Lastly, we show how to apply a similar decoupling trick as in the full information setting to obtain polynomial time and space complexity of bid vector sampling.

\subsection{Exp3.G Algorithm for Learning with Graph Feedback}

Consider the problem of online learning in repeated games where actions reveal the utility associated with other actions. To do this, we first define the notion of a feedback graph. Let $G = (\mathcal{V}, E)$ denote a directed graph with vertices $\mathcal{V}$ and edges $E = \{e_{i,j}\}_{i,j \in \mathcal{V}}$. Let $\mathcal{V}(i) = \{j \in \mathcal{V}: e_{i, j} \in E\}$ and $\mathcal{V}^{-1}(j) = \{i \in \mathcal{V}: e_{i, j} \in E\}$ denote the set of outgoing edges from $i$ and incoming edges to $j$ respectively. Then, we let $\mathcal{V}$ denote the set of actions where the environment chooses reward function $f: \mathcal{V} \to [0, 1]$. We let $\mathcal{F}$ denote the set of all possible reward functions and $\mathcal{E}$ denote the set of all possible edges. By selecting action $i \in \mathcal{V}$, a learner receives reward $f(i)$ and additionally observes the rewards of all outgoing neighboring actions $\{f(j)\}_{j \in \mathcal{V}(i)}$. Now, we state the algorithm \rigel{as described in...}:

\begin{algorithm}[t]
	\KwIn{Learning rates $\eta, \gamma > 0$, Action space $\mathcal{V}$, Adaptive Adversarial Environment $\textsc{Env}^\nround: \mathcal{H}^\nround \to \mathcal{E} \times \mathcal{F}$ where $\mathcal{H}^\nround$ denotes the set of all possible historical auction results $H^\nround$ up to round $\nround$ for all $\nround \in [\Nround]$.}
	\KwOut{The aggregate utility $\sum_{\nround=1}^\Nround f_\nround(i_\nround)$ corresponding to a sequence of actions $i_1,\ldots,i_\Nround$ sampled according to the exponentially weighted utility estimates.}
	$q_0(i) = \frac{1}{|\mathcal{V}|} \forall i \in \mathcal{V}$\;
        $H^0 \gets \emptyset$
	\For{$\nround \in [\Nround]$:}{
            $p_{\nround}(i) \gets (1 - \gamma)q_{\nround-1}(i) + \frac{\gamma}{|\mathcal{V}|}$ for all $i \in \mathcal{V}$\;
            $(E_\nround, f_\nround) \gets \textsc{Env}^\nround(\mathcal{H}^\nround)$, let $\mathcal{V}_\nround(i)$ and $\mathcal{V}^{-1}_\nround(j)$ be outgoing/incoming neighbors to $i$ and $j$ respectively\;
            Select action $i_\nround \sim \bm{p}_{\nround}$, receive reward $f_\nround(i_\nround)$, observe rewards $\{f_\nround(j)\}_{j \in \mathcal{V}_\nround(i)}$\;
            Compute utility estimates $\hat{f}_\nround(j) \gets \frac{f_\nround(j)}{\sum_{i \in \mathcal{V}^{-1}_\nround(j)} p_{\nround}(i)}\textbf{1}_{j \in \mathcal{V}_\nround(i_\nround)}$ for all $j \in \mathcal{V}$\;
            Update sampling distributions $q_\nround(i) \gets \frac{q_{\nround-1}(i)\exp(\eta \hat{f}_\nround(i))}{\sum_{j \in \mathcal{V}} q_{\nround-1}(j) \exp(\eta \hat{f}_\nround(j))}$ for all $i \in \mathcal{V}$\;
        }
        \textbf{Return} $\sum_{\nround=1}^\Nround f_\nround(i_\nround)$
	\caption{\textsc{Exp3.G}}
	\label{alg: Exp3.G}
\end{algorithm}

\begin{theorem}[\rigel{Cite here}]
    \label{thm: Exp3.G regret}
    Let $\alpha_\nround = \alpha_\nround(\mathcal{V}, E_\nround)$ denote the independence number of the directed graph $G_\nround = (\mathcal{V}, E_\nround)$. Assume bounded rewards $f_\nround(\cdot) \in [L, U]$ and that $E_\nround$ contains the entire set of self-loops $\{e_{i, i}\}_{i \in \mathcal{V}} \subseteq E_\nround$ for all $\nround \in [\Nround]$, i.e. the agent observes their reward each round, then we have that the expected regret of Algorithm \ref{alg: Exp3.G} can be upper bounded as follows:
    \begin{align}
        \textsc{Regret}_{\textsc{Exp3.G}}^\Nround = \max_{i \in \mathcal{V}} \sum_{\nround=1}^\Nround f_\nround(i) - \mathbb{E}_{i_\nround \sim p_\nround}\left[\sum_{\nround=1}^\Nround f_\nround(i_\nround)\right] \lesssim (U - L)\left(\sum_{\nround=1}^\Nround \alpha_\nround\right)^{\frac{1}{2}} \log(|\mathcal{V}|\Nround)
    \end{align}
    for $\gamma = \min(\frac{1}{2}, \left(\sum_{\nround=1}^\Nround \alpha_\nround\right)^{-\frac{1}{2}})$ and $\eta = 2\gamma$.
\end{theorem}

We note the independence number, which is normally defined over undirected graphs, is defined for directed graphs as the size of the maximum acyclic directed subgraph (excluding self loops).

\subsection{Repeated Multi-Unit Auctions as Instance of Learning with Graph Feedback}

We now show how our repeated multi-unit auction can be modeled as a learning problem with graph feedback. Let $\mathcal{V} = \mathcal{B}^\Nitem$ denote the set of all valid bid vectors. The crux of our model is representing the set of edges $E_\nround$ and neighbor functions $\mathcal{V}_\nround(i)$ and $\mathcal{V}^{-1}_\nround(j)$ as a function of bid vectors $i_\nround = \bm{b}^{n, \nround}$ and the observed feedback $c^\nround$. In particular, assuming that agent $n$ bids $\bm{b}^{n, \nround}$, receives allocation $x^{n, \nround}$, and observes the corresponding clearing price $c^\nround$, then the agent can precisely determine the utility associated with any bid vector $\bm{b}$ satisfying both $b_{x^{n, \nround}} > c^\nround$ and $b_{x^{n, \nround} + 1} \leq b^{n, \nround}_{x^{n, \nround}+1}$. As $c^\nround$ is a valid lower bound on $\max(b^{-n, \nround}_\nitem, \pi^\nround)$ for all $\nitem \in [x^{n, \nround}]$ and $b^{n, \nround}_{x^{n, \nround}+1}$ is a valid upper bound on $\max(b^{-n, \nround}_\nitem, \pi^\nround)$ for all $\nitem \in \{x^{n, \nround}+1,\ldots,\Nitem\}]$, we have four cases of reward observability for each $(\nitem, B)$ pair:
\begin{enumerate}
    \item $\nitem \leq x^{n, \nround}, B > c^\nround$: We know that $c^\nround = \max(b^{n, \nround}_{x^{n, \nround}}, b^{-n, \nround}_{x^{n, \nround}+1}, \pi^\nround) \geq \max(b^{-n, \nround}_{\nitem}, \pi^\nround)$ as $\bm{b}^{-n, \nround}$ is non-decreasing. Hence, any bid for slot $\nitem \leq x^{n, \nround}$ larger than the clearing price is guaranteed to win an item.
    \item $\nitem > x^{n, \nround}, B \leq b^{n, \nround}_{x^{n, \nround}+1}$: We have $B \leq b^{n, \nround}_{x^{n, \nround}+1} \leq \max(b^{-n, \nround}_{x^{n, \nround}+1}, \pi^\nround) \leq \max(b^{-n, \nround}_{\nitem}, \pi^\nround)$, once again, using monotonicity of $\bm{b}^{-n, \nround}$. Hence, any bid for slot $\nitem > x^{n, \nround}$ less than $b^{n, \nround}_{x^{n, \nround}+1}$ is guaranteed to lose an item.
    \item $\nitem \leq x^{n, \nround}, B \leq c^\nround$: We cannot say with certainty whether this bid would win or lose an item. For example, considering the case where $\Nitem = 1$ and $x^{n, \nround} = 1$ with bid $b^{n, \nround}_1$. Then, the clearing price is given by $c^\nround = b^{n, \nround}_1$ and the agent learns no information as to the minimum bid required to win the item. In other words, the agent does not observe reward function $w_1^{\nround}(B) = (v^{n, \nround}_1 - B)\textbf{1}_{B > \max(b^{-n, \nround}_1, \pi^\nround)}$ for $B < c^\nround$.
    \item $\nitem > x^{n, \nround}, B > b^{n, \nround}_{x^{n, \nround}+1}$: Again, we cannot say with certainty whether this bid would win or lose an item. Consider the case where $\Nitem = 2$, and assume $\bm{b}^{n, \nround} = [B^2, 0]$, $\bm{b}^{-n, \nround} = [0, B^1]$, and $\pi^\nround = 0$ with bids $B^1 > B^2 > 0$. Here, $x^{n, \nround} = 1$ and $c^\nround = B^2$, so the agent learns no information about the minimum bid $B$ required to win both items. In other words, the agent does not observe reward function $w_2^{\nround}(B) = (v^{n, \nround}_2 - B)\textbf{1}_{B > \max(b^{-n, \nround}_2, \pi^\nround)}$ for $B \geq b^{n, \nround}_{x^{n, \nround}+1}$.
\end{enumerate}

Note that we can potentially gain more information, as the case where $c^\nround$ does not belong to agent $n$'s bids $\bm{b}^{n, \nround}$ will yield additional information as 1) we can replace the bound of $b^{n, \nround}_{x^{n, \nround}+1}$ with $c^\nround$ in the second and fourth cases and 2) learn the minimum bid in slot $x^{n, \nround}+1$ required to win an additional item. Even disregarding this additional graph connectivity (which can only serve to decrease the independence number), we can upper bound the independence number by $O(\Nitem |\mathcal{B}|^2)$.
% We make these observations rigorous in the following proof of our upper bound on the independence number of these graphs. 
% Nonetheless, even without factoring in this additional graph connectivity (which can only increase the independence number), we only assume the outgoing edges from bid $\bm{b}^{n, \nround}$ to the set of bids $\bm{b}$ whose first $x^{n, \nround}$ bids are above the clearing price and the remaining bids below agent $n$'s first losing bid. We now show an upper bound on the independence number of any graph with edges constructed in the above fashion.
\begin{lemma}
    \label{lem: independence number}
    Consider a graph $G_\nround = (\mathcal{V}, E_\nround)$ with nodes $\mathcal{V} = \mathcal{B}^\Nitem$ and edges $E_\nround \equiv \{e_{\bm{b}, \bm{b}'}: \bm{b}, \bm{b}' \in \mathcal{V}, x(\bm{b}) = x(\bm{b}')$, $b'_{x(\bm{b})} \geq c^\nround$, and $b'_{x(\bm{b})+1} \leq b_{x(\bm{b})+1}\}$, where $x(\bm{b}) = x(\bm{b}, \bm{b}^{-n, \nround}, \pi^\nround)$ and $c^\nround = \max(b_{x(\bm{b})}, b^{-n, \nround}_{x(\bm{b})+1}, \pi^\nround)$ for $\bm{b},\bm{b}' \in \mathcal{B}^\Nitem$. Then, regardless of the choice of $\bm{b}^{-n, \nround}$ or $\pi^\nround$, the independence number of $G_\nround$ is $O(\Nitem|\mathcal{B}|^2)$.
\end{lemma}

\begin{proof}
    We want to show an upper bound on the size of the largest acyclic directed subgraph of the aforementioned graph. First, some preliminaries. For any bid vectors $\bm{b}$ and $\bm{b}'$ such that yield the same allocation $x(\bm{b}) = x(\bm{b}') = x$, we receive feedback in the form $c^\nround$ and $x$. The information received, however, is not necessarily symmetric; i.e. we may be able to deduce the exact utility of bid $\bm{b}'$ after playing $\bm{b}$ (exists an edge from $\bm{b}$ to $\bm{b}'$) but not vice versa. There are several possible sources of information asymmetry, mostly stemming from the event that the lowest winning bid corresponds to one of agent $n$'s bids. In this event, we know that the smallest guaranteed winning bid value (SGWB) for the first $x$ items is a function of $c^\nround = b_x$ and the highest guaranteed losing bid (HGLB) value for the remaining $\Nitem - x$ items is a function of $b_{x+1}$. Considering only the class of edges as described in the lemma, we have the following cases of information gain or asymmetry:
    \begin{enumerate}
        \item $b_x < \max(b^{-n, \nround}_{x+1}, \pi^\nround)$: With this information, the agent $n$ is revealed the entire reward function for slot $x+1$ since $w_{x+1}^\nround(B) = (v^{n, \nround}_{x+1} - B)\textbf{1}_{B > \max(b^{-n, \nround}_{x+1}, \pi^\nround)} = (v^{n, \nround}_{x+1} - B)\textbf{1}_{B > c^\nround}$. However, for slots $x+2,\ldots,\Nitem$, the HGLB is $b_{x+2}$. Hence, an edge exists from $\bm{b}$ to $\bm{b}'$ if and only if $b_{x+2} \geq b'_{x+2}$. 
        \item $b_x \geq \max(b^{-n, \nround}_{x+1}, \pi^\nround)$: Here, we have that the clearing price $c^\nround = b_x$ and only incomplete information regarding the SGWB for the first $x$ items. Hence, a necessary condition for there to be an edge from $\bm{b}$ to $\bm{b}'$ is $b_x \leq b'_x$. This condition alone is not sufficient as we also require the information regarding the HGLB for the remaining $\Nitem - x$ items. That is, we also require that $b_{x+1} \geq b'_{x+1}$.
    \end{enumerate}
    \rigel{Add color diagram here of incoming and outgoing edges over the $\Nitem$ by $|\mathcal{B}|$ network}
    We gain strictly more information in the former case, as the SGWB for the first $x+1$ items is given as $c^\nround$, as opposed to only the first $x$ items. Similarly, the HGLB for items $x+2,\ldots,\Nitem$ is $b_{x+2}$ as opposed to $b_{x+1}$. We also remark that there exist no edges from $\bm{b}$ to some $\bm{b}"$ such that $x(\bm{b}) = x > x" = x(\bm{b}")$, as we learn that $b_x$ is a guaranteed winning bid for the first $x" < x$ items, however we do not learn anything about the set of guaranteed losing bids for the remaining $\Nitem - x" > \Nitem - x$ items. Similarly for the case when $x < x"$. Hence, we claim that the size of the largest acyclic subgraph of the set of nodes $\bm{b}$ such that $x(\bm{b}) = x$ is of size $O(|\mathcal{B}|^2)$. More specifically, $b_x$ and $b_{x+1}$ are the only bids that determine $\mathcal{V}(\bm{b})$, the outgoing neighbor set of $\bm{b}$, hence there are a total of $O(|\mathcal{B}|^2)$ combinations of $(b_x, b_{x+1})$. Since there are $\Nitem+1$ possible values of $x$, as one could win anywhere from 0 to $\Nitem$ items, and there are no edges between nodes with differing allocations, the size of the largest acyclic subgraph is of size $O(\Nitem |\mathcal{B}|^2)$. \rigel{Perhaps we can do better? Maybe we can take advantage of one-sided cross learning between bids with different allocations or the first case where we learn more information; i.e. we can learn the entire loss function with $\Nitem$ bids, worst case $\Nitem |\mathcal{B}$ bids (one per layer). Might be an interesting research direction to pursue. Nonetheless, this section feels very repetitive, should go through again. Can make significantly more concise just by saying that the neighbor sets depend only on $c^\nround$ (which is a function of only $b_x$) and $b_{x+1}$. Since there are only $O(|\mathcal{B}|^2)$ combinations of $(b_x, b_{x+1})$ and there are no edges between bid vectors with different allocations, then the independence number if upper bounded by $O(\Nitem |\mathcal{B}|^2)$.}
\end{proof}

As this above bound on the independence number is independent of the choice of $\bm{b}^{-n, \nround}$ and $\pi^\nround$, this serves as an upper bound to all feedback graphs $G_1,\ldots,G_\Nround$. Applying this result to Theorem~\ref{thm: Exp3.G regret}, we obtain the following regret bound:
\begin{corollary}
    Consider an arbitrary sequence of graphs $\{G_\nround\}_{\nround \in [\Nround]}$ where $G_\nround = (\mathcal{V}, E_\nround)$ with $\mathcal{V} = \mathcal{B}^\Nitem$ and edges $E_\nround$ defined according to Lemma~\ref{lem: independence number}. Then assuming rewards are bounded between $[-\Nitem, \Nitem]$, Algorithm~\ref{alg: Exp3.G} guarantees the following regret:
    \begin{align}
        \textsc{Regret}_{\textsc{Exp3.G}}^\Nround \leq \Nitem^{\frac{5}{2}}|\mathcal{B}|\Nround^{\frac{1}{2}}\log(|\mathcal{B}|\Nround)
    \end{align}
    For $\gamma = \min(\frac{1}{2}, (\Nitem |\mathcal{B}|^2 \Nround)^{-\frac{1}{2}})$ and $\eta = 2\gamma$.
\end{corollary}

As in the full information online setting, we must still avoid the exponential space and time complexity of directly applying $\textsc{Exp3.G}$ on the set of all bid vectors. Fortunately, the decoupling argument from the previous section will still apply with minor modifications.

\subsection{Decoupled \textsc{Exp3.G} Algorithm}

\begin{enumerate}
    \item Main issue in directly applying Exp3.G is that we don't observe $\mathcal{V}^{-1}_\nround$ in its entirety so we can't compute the utility estimates $\hat{f}_\nround$ efficiently as the denominator requires knowledge of $\mathcal{V}^{-1}_\nround$. We only observe $\mathcal{V}_\nround$ which is a function of $c^\nround$ and $b^{n, \nround}_{x^\nround+1}$.
    \item Assume we observe allocation $x^\nround$, $c^\nround = \max(b^{-n, \nround}_{x^\nround}, \pi^\nround)$ and $d^\nround = \max(b^{-n, \nround}_{x^\nround+1}, \pi^\nround)$; i.e. we get to see the reserve as well as our adversary's highest losing bid and lowest winning bid. Now, we have that $\mathcal{V}_\nround$ contains the set of bid vectors $\bm{B}$ with allocation $x^\nround$, i.e. $B_{x^\nround} \geq c^\nround$ and $B_{x^\nround + 1} \leq d^\nround$ and we have access to the entirety of $\mathcal{V}^{-1}_\nround = \mathcal{V}_\nround$. With this, we can 1) correctly update aggregate weights, 2) reduce independence number to $\Nitem$.
    \item The latter follows since bidding $\bm{B}$ reveals the entire utility function for any bid vector with the same allocation as $\bm{B}$. Hence, any two bid vectors with the same allocation forms a cycle, and the largest acyclic subgraph must then be size $\Nitem$.
    \item Correctly updating the weights is more involved since we also have to decouple in order to reduce time and space complexity. The decoupling process cannot be done as in the full information case. That is, directly applying the decoupled exponential weights algorithm with per-slot utility estimates does not guarantee that the sampled bid vector may not itself be observable---even if we have unbiased estimates for the per-slot utility of slot-bid value pair $(\nitem, B)$, we must take into account the (un)observability of the entire sequence of bids. Assuming valuation vector $\bm{v}$, historical allocation $x^\nround = x^{n, \nround}$, SGWB $c^\nround = \max(b^{-n, \nround}_{x^{n, \nround}+1}, \pi^\nround)$ (slightly different from clearing price, since we're assuming we did not have the lowest winning bid), and HGLB $d^\nround = \max(b^{-n, \nround}_{x^\nround+1}, \pi^\nround)$ corresponding to bid $\bm{b}^{n, \nround}$, we can write the observable utility as follows:
    \begin{align*}
        \widebar{\mu}^{n, \nround}(\bm{B}) = \sum_{\nitem=1}^\Nitem (v_\nitem - B_\nitem) \sum_{\tau=1}^\nround \textbf{1}_{\nitem \leq x^\tau, B_{x^\tau} \geq c^\tau} \textbf{1}_{B_{x^\tau+1} \leq d^\tau}
    \end{align*}
    Now, the same trick in decomposing into $S_\nitem(B)$'s will not work as the indicator additionally depends on the bid value at indices $x^\tau$ and $x^\tau + 1$.
\end{enumerate}



\newpage

In the bandit setting, the source of uncertainty as to the utility associated with bidding $\bm{B}$ at round $\nround$ is due to not knowing the allocation. If the allocation $x(\bm{B})$ was known, then the utility is trivially given by $\sum_{\nitem=1}^\Nitem (v_\nitem - B\nitem) \textbf{1}_{\nitem \leq x(\bm{B})}$. One idea is to first condition on the allocation and only consider the utilities associated with the same historical allocation. In particular, we want to sample bids according to the exponentially weighted observable utility. To do this, we first define the observable utility of a bid vector $\bm{B}$. Assuming valuation vector $\bm{v}$, historical allocation $x^\nround = x^{n, \nround}$, SGWB $c^\nround = \max(b_{x^{n, \nround}}, b^{-n, \nround}_{x^{n, \nround}+1}, \pi^\nround)$ (which is the clearing price), and HGLB $d^\nround = b^{n, \nround}_{x^{n, \nround}+1}$ corresponding to bid $\bm{b}^{n, \nround}$, we define:
\begin{align*}
    \widebar{\mu}^{n, \nround}(\bm{B}) = \sum_{\tau=1}^\nround \textbf{1}_{B_{x^\tau} \geq c^\tau, B_{x^\tau + 1} \leq d^\tau} \mu(\bm{v}, \bm{B}, \bm{b}^{-n, \tau}, \pi^\tau) = \sum_{\tau=1}^\nround \textbf{1}_{B_{x^\tau} \geq c^\tau, B_{x^\tau + 1} \leq d^\tau} \sum_{\nitem=1}^\Nitem (v_\nitem - B_\nitem)\textbf{1}_{\nitem \leq x^\tau}
\end{align*}
Notice that we can reorder the summation and replace the condition $\{B_{x^\tau} \geq c^\tau\}$ with $\{B_\nitem \geq c^\tau\}$ using the non-increasing bid assumption:
\begin{align*}
    \widebar{\mu}^{n, \nround}(\bm{B}) = \sum_{\nitem=1}^\Nitem \sum_{\tau=1}^\nround (v_\nitem - B_\nitem) \textbf{1}_{\nitem \leq x^\tau, B_\nitem \geq c^\tau} \textbf{1}_{B_{x^\tau} \leq d^\tau}
\end{align*}
With this defined, we wish to sample bid vectors proportional to their observable utility. The sum of all observable exponentiated bid vectors can be written as follows:
\begin{align}
    \sum_{\bm{B} \in \mathcal{B}^\Nitem} \exp(\eta \widebar{\mu}^{n, \nround}(\bm{B})) &= \sum_{\bm{B} \in \mathcal{B}^\Nitem} \exp(\eta \sum_{\nitem=1}^\Nitem \sum_{\tau=1}^\nround (v_\nitem - B_\nitem) \textbf{1}_{\nitem \leq x^\tau, B_\nitem \geq c^\tau} \textbf{1}_{B_{x^\tau} \leq d^\tau})\\
    &= \sum_{\bm{B} \in \mathcal{B}^\Nitem} \prod_{\tau=1}^{\nround}  \exp(\eta \textbf{1}_{B_{x^\tau} \leq d^\tau} \sum_{\nitem=1}^\Nitem (v_\nitem - B_\nitem) \textbf{1}_{\nitem \leq x^\tau, B_\nitem \geq c^\tau})\\
    &= \sum_{\nitem'=0}^\Nitem \sum_{\bm{B} \in \mathcal{B}^\Nitem} \prod_{\tau=1}^{\nround}  \exp(\eta \textbf{1}_{B_{x^\tau} \leq d^\tau} \sum_{\nitem=1}^\Nitem (v_\nitem - B_\nitem) \textbf{1}_{\nitem \leq x^\tau, B_\nitem \geq c^\tau})\\
    &= 
\end{align}



\newpage

Notice that we can reorder the summation and replace the condition $\{B_{x^\tau} \geq c^\tau\}$ with $\{B_\nitem \geq c^\tau\}$ using the non-increasing bid assumption:
\begin{align*}
    \widebar{\mu}^{n, \nround}(\bm{B}) = \sum_{\nitem=1}^\Nitem \sum_{\tau=1}^\nround (v_\nitem - B_\nitem) \textbf{1}_{\nitem \leq x^\tau, B_\nitem \geq c^\tau} \textbf{1}_{B_{x^\tau} \leq d^\tau} = \sum_{\nitem=1}^\Nitem \sum_{\tau = 1}^\nround \widebar{w}_\nitem^\tau(B; d) = \sum_{\nitem=1}^\Nitem \widebar{W}_\nitem^\nround(B; d)
\end{align*}
Where $\widebar{w}_\nitem^tau(B; d) = $


This observability condition $\textbf{1}_{B_{x^\tau} \geq c^\tau, B_{x^\tau + 1} \leq d^\tau}$ forces an additional computational cost. To resolve this, we additionally condition on observability. In particular, we define 





To do this, we define slightly modified versions of the reward estimates corresponding to each slot. Assuming valuation vector $\bm{v}$, historical allocation $x^{n, \nround}$, and clearing price $c^\nround$ corresponding to bid $\bm{b}^{n, \nround}$, we define for each $\nround \in [\Nround]$:
\[
\widebar{w}^{\nround}_\nitem(B) = \textbf{1}_{B \geq c^\nround} \textbf{1}_{\nitem \leq x^{n, \nround}} (v_\nitem - B) \quad \text{and} \quad \widebar{W}^{\Nround}_\nitem(B) = \sum_{\nround=1}^\Nround \widebar{w}^{\nround}_\nitem(B)
\]
This $\widebar{w}^{\nround}_\nitem(B)$ has similar structure to that of $w_\nitem^\nround(B)$. The conditions $B \geq c^\nround$ and $\nitem \leq x^{n, \nround}$ simply guarantee that the reward corresponding to bid $B$ at slot $\nitem$ was observable. 


\newpage




The above decoupled exponential weights algorithm and justification does not work in the case of bandit feedback, as bidders no longer observe $b^{-n, \nround}_{\nitem}$ and hence cannot observe $w^{\nround}_{\nitem}(b)$ for all possible bids $b$. As such, we cannot run straightforward FTL style algorithms in the bandit setting. However, this setup allows us to make several useful observations. First, we can exploit cross learning by using the expression for $w^\nround_\nitem(B) = \textbf{1}_{B > \max(b^{-n, \nround}_{\nitem}, \pi^\nround)} (v_\nitem - b)$. Notice that for any $b \geq \max(b_\nitem^{-n, \nround}, \pi^\nround)$, the utility corresponding to bidding $b$ in the $\nitem$'th slot is exactly $v_\nitem - b$, and 0 otherwise. In other words, if we knew exactly the value of $\max(b_\nitem^{-n, \nround}, \pi^\nround)$, we can precisely compute the entire utility function corresponding to slot $\nitem$ in round $\nround$. Similarly, if we knew the value of all $\max(b_\nitem^{-n, \nround}, \pi^\nround)$ for $\nitem \in [\Nitem]$, then we can explicitly compute the utility function associated with any $\bm{b} \in \mathcal{B}^\Nitem$ in round $\nround$. Unfortunately, the agent can only learn about at most one of these values. In particular, assuming that agent $n$ bids $\bm{b}$, receives allocation $x^{n, \nround}$, and observes the clearing price $c^\nround$, then either:
\begin{enumerate}
    \item Agent $n$'s bid does not contain the clearing price $c^\nround$. In this case, the clearing price $c^\nround$ corresponds to another agent's bid or the reserve and hence, is exactly equal to $\max(b_{x^{n, \nround}}^{-n, \nround}, \pi^\nround)$. As $\max(b_\nitem^{-n, \nround}, \pi^\nround)$ is weakly increasing in $\nitem$, we have that $c^\nround$ lower bounds $\max(b_\nitem^{-n, \nround}, \pi^\nround)$ for $\nitem \leq x^{n, \nround}$. Consequently, any bids larger than $c^\nround$ in the first $x^{n, \nround}$ slots are guaranteed to win an item and agent $n$ can exactly determine $w^\nround_\nitem(b) = v_\nitem - b$ for all $\nitem \leq x^{n, \nround}, b > c^\nround$. Similarly, any bids smaller than $c^\nround$ in slots $x^{n, \nround}+1,\ldots,\Nitem$ are guaranteed not to win the item and thus, $w^\nround_\nitem(b) = 0$ for all $\nitem > x^{n, \nround}, b < c^\nround$.
    \item Agent $n$'s bid contains the clearing price $c^\nround$. In this case, agent $n$ knows that their bid $b_{x^{n, \nround}}$ must have been larger than $\max(b_{x^{n, \nround}}^{-n, \nround}, \pi^\nround)$. As in the previous case, $c^\nround$ lower bounds $\max(b_\nitem^{-n, \nround}, \pi^\nround)$ for $\nitem \leq x^{n, \nround}$, and consequently, any bids larger than $c^\nround$ in the first $\nitem$ slots are guaranteed to win an item and agent $n$ can determine $w^\nround_\nitem(b) = v_\nitem - b$ for all $\nitem \leq x^{n, \nround}, b \geq c^\nround$. However, as $c^\nround$ only serves as a lower bound, we cannot ascertain the exact value of $\max(b_{x^{n, \nround}}^{-n, \nround}, \pi^\nround)$. Thus, unlike the previous case, agent $n$'s utility functions $w^\nround_\nitem(b)$ for $m \geq x^{n, \nround}$ and $b \leq c^\nround$ are uncertain. However, we do know that conditioning on $b_{x^{n, \nround}} \geq c^\nround$, any bids in positions $\nitem \in \{x^{n, \nround},\ldots,\Nitem\}$ less than $c^\nround$ are guaranteed not to win an item. 
\end{enumerate}
Note that agent $n$ learns slightly more information in the former case as there is a sharp lower bound on the value of the bid that guarantees winning an item in the first $x^{n, \nround}$ slots. Similarly, we also know the minimum guaranteed winning bid for slot $x^{n, \nround} + 1$. This implies that agents with lower aggregate demand (and hence are less likely to possess the lowest winning bid) will receive more meaningful feedback each round. Regardless, we have that in either case, the agent can determine $w_\nitem^{\nround}(B)$ for any $\nitem \in [x^{n, \nround}]$ and $B \geq c^\nround$ or for any $\nitem \in \{x^{n, \nround},\ldots,\Nitem\}, B < c^\nround$. Using this information, we can construct reward estimates in the bandit setting that will motivate our choice of learning algorithm. Assuming historical allocation $x^{n, \nround}$ corresponding to bid $\bm{b}^{n, \nround}$, we define for each $\nround \in [\Nround]$:
\[
\widebar{w}^{\nround}_\nitem(B) = \textbf{1}_{B \geq c^\nround} \textbf{1}_{\nitem \leq x^{n, \nround}} (v_\nitem - B) \quad \text{and} \quad \widebar{W}^{\Nround}_\nitem(B) = \sum_{\nround=1}^\Nround \widebar{w}^{\nround}_\nitem(B)
\]
Note that these definitions take into account the cases where we are guaranteed to lose an item, as the utility in those cases would be 0, as well as when the reward is uncertain. Now, we let $c^\nround(\bm{B})$ denote the market clearing price at round $\nround$ given that agent $n$ bids $\bm{B}$. Similarly, $\phi^\nround(\bm{B}) = \{(\nitem, B): \nitem \leq x(\bm{B}, n, \nround), B \geq c^\nround(\bm{B})\}$ denote the set of slot-bid value pairs that are guaranteed to win an item; i.e. whose corresponding utility for winning are known with certainty. We define the inverse of $\phi^\nround$ as $\psi^\nround(\nitem, B) = \{\bm{B} \in \mathcal{B}^\Nitem: (\nitem, B) \in \phi^\nround(\bm{B})\}$. We show a variant of Algorithm \ref{alg: Decoupled Exponential Weights} achieves small regret with appropriate choice of exploration parameter $\gamma$ and the following unbiased utility estimates:
\[
\widehat{w}^{\nround}_\nitem(B) = \frac{v_\nitem - B}{\sum_{\bm{B} \in \psi^\nround(\nitem, B)} p^\nround(\bm{B})} \textbf{1}_{(\nitem, B) \in \phi^\nround(\bm{b}^{n, \nround})} \quad \text{and} \quad \widehat{W}^{\Nround}_\nitem(B) = \sum_{\nround=1}^\Nround \widehat{w}^{\nround}_\nitem(B)
\]
Where here, $p^\nround(\bm{B}) = (1 - \gamma)q^\nround(\bm{B}) + \frac{1}{|\mathcal{B}^\Nitem|}$ with $q^\nround(\bm{B})$ defined recursively with $q_1(\cdot) = \frac{1}{|\mathcal{B}^\Nitem|}$ and $q^\nround(\bm{B}) = \frac{q^{\nround-1}(\bm{B})\exp(\eta \mu(\bm{v}, \bm{B}, \bm{b}^{-n, \nround}, \pi^\nround))}{\sum_{\bm{B}' \in \mathcal{B}^\Nitem} q^{\nround-1}(\bm{B}')\exp(\eta \mu(\bm{v}, \bm{B}', \bm{b}^{-n, \nround}, \pi^\nround))}$. Note that while $\mu(\bm{v}, \bm{B}, \bm{b}^{-n, \nround}, \pi^\nround)$ may be unobservable for general $\bm{B}$, since we are considering the set of $\bm{B}$ such that \rigel{ok this is harder than expected, since these might not be observable. Need to be very careful in determining this, and also figuring out how to track the estimator denominator efficiently}

Note that these estimates are unbiased as $\sum_{\bm{B} \in \psi^\nround(\nitem, B)} p^\nround(\bm{B})$ is the probability of observing the reward corresponding to $(\nitem, B)$, which is exactly $\prob((\nitem, B) \in \phi^\nround(\bm{B}))$. As each of the per slot utility estimates $\widehat{w}^{\nround}_\nitem(B)$ are unbiased, the aggregated per slot utilities $\widehat{W}^{\nround}_\nitem(B)$ and aggregate utilities $\widehat{\mu}^{n, \Nround}(\bm{B}) = \sum_{\nitem=1}^\Nitem \widehat{w}^{\nround}_\nitem(B)$ will also be unbiased. As such, we can apply Algorithm \ref{alg: Decoupled Sampler} to sample bids according to the exponentially weighted estimated utility vectors $\exp(\eta \widehat{\mu}^{n, \Nround}(\bm{B}))$. Stating the algorithm in its entirety, we have:

\begin{algorithm}[t]
	\KwIn{Learning rates $\eta, \gamma > 0$, Valuations $\{\bm{v}^{n, \nround}\}$ for $\bm{v}^{n, \nround} \in \{\bm{v}': 1 \geq v'_1 \geq \ldots \geq v'_\Nitem \geq 0\}$, Adaptive Adversarial Environment $\textsc{Env}^\nround: \mathcal{H}^\nround) \to \mathcal{B}^{-\Nitem} \times \mathcal{B}$ where $\mathcal{H}^\nround$ denotes the set of all possible historical auction results $H^\nround$ up to round $\nround$ for all $\nround \in [\Nround]$.}
	\KwOut{The aggregate utility $\sum_{\nround=1}^\Nround \mu(\bm{v}^{n, \nround}, \bm{b}^{n, \nround}, \bm{b}^{-n, \nround}, \pi^\nround)$ corresponding to a sequence of bid vectors $\bm{b}^{n, 1},\ldots,\bm{b}^{n, \Nround}$ sampled according to the exponential weights algorithm under bandit feedback.}
	$q^0(\nitem, B) \gets \frac{1}{\Nitem|\mathcal{B}|}, \widehat{W}_\nitem^0(B) \gets 0, \textsc{NumberWon}_\nitem^0(B) \gets 0$ for all $\nitem \in [\Nitem], B \in \mathcal{B}$\;
        $H^0 \gets \emptyset$
	\For{$\nround \in [\Nround]$:}{
            Observe $\bm{v}^{n, \nround}$\;
            Update utilities estimates $\bm{\widehat{W}}^{\nround - 1} \gets \{\widehat{W}^{\nround-1}_{\nitem}(B) = \textsc{NumberWon}_\nitem^{\nround-1}(B) \frac{v^{n, \nround}_\nitem - B}{\sum_{\bm{B} \in \psi^\nround(\nitem, B)} p^\nround(\nitem, B)}\}_{\nitem \in [\Nitem], B \in \mathcal{B}}$\;
            $(\bm{b}^{-n, \nround}, \pi^\nround) \gets \textsc{Env}^{\nround-1}(H^{\nround-1})$ and $\bm{b}^{n, \nround} \gets \textsc{DecoupledSampler}(\bm{W}^{\nround-1}, \eta)$\;
            Observe $x^{n, t}, c^\nround$ and receive reward $\mu(\bm{v}^{n, \nround}, \bm{b}^{n, \nround}, \bm{b}^{-n, \nround}, \pi^\nround)$\;
            $\textsc{NumberWon}_\nitem^\nround(B) \gets \textsc{NumberWon}_\nitem^{\nround-1}(B) + \textbf{1}_{B \geq c^\nround}$ for all $\nitem \in [x^{n, t}], B \in \mathcal{B}$\;
        }
        \textbf{Return} $\sum_{\nround=1}^\Nround \mu(\bm{v}^{n, \nround}, \bm{b}^{n, \nround}, \bm{b}^{-n, \nround}, \pi^\nround)$
	\caption{\textsc{Decoupled Exponential Weights - Bandit Feedback}}
	\label{alg: Decoupled Exponential Weights - Bandit Feedback}
\end{algorithm}

\rigel{Hmm, will the non-stationary valuations change anything}