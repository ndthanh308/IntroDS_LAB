
% \newtheorem{definition}{Definition}
\newtheorem{observation}{Observation}
\newcommand{\prob}{\mathbb{P}}
\newcommand{\Nitem}{M}
%

\newcommand{\nitem}{m}
\newcommand{\Nround}{T}
\newcommand{\nround}{t}
% \newcommand{\rigel}[1]{{\color{blue}[\textsc{Rigel}: \emph{#1}]}}
% \newcommand{\negin}[1]{{\color{red}[\textsc{NG}: \emph{#1}]}}
\newcommand{\rigel}[1]{{[#1]}}
\newcommand{\negin}[1]{{[#1]}}
%\usepackage{natbib}
% \bibpunct[, ]{(}{)}{,}{a}{}{,}%
% \def\bibfont{\small}%
 %\def\bibsep{\smallskipamount}%
 %\def\bibhang{10pt}%
 %\def\newblock{\ }%
 %\def\BIBand{and}%

%\usepackage[utf8]{inputenc} % allow utf-8 input
%\usepackage[T1]{fontenc}  
%\documentclass[format=acmsmall, review=false]{acmart}
\documentclass[msom,nonblindrev]{informs3}
%\usepackage[toc,page]{appendix}
\usepackage{hyperref}
\usepackage{url}
 \usepackage{natbib}
 \bibpunct[, ]{(}{)}{,}{a}{}{,}%
 \def\bibfont{\small}%
 \def\bibsep{\smallskipamount}%
 \def\bibhang{24pt}%
 \def\newblock{\ }
%\documentclass[12pt]{report}
\usepackage{bm}
%\usepackage{setspace}
%\renewcommand{\baselinestretch}{1.5} 
\usepackage{zref-totpages}
\usepackage{booktabs} % For formal tables
\usepackage[ruled]{algorithm2e} % For algorithms
%\usepackage{graphicx}
%\usepackage{caption}
\usepackage{subcaption}
\renewcommand{\algorithmcfname}{ALGORITHM}
%\SetAlFnt{\small}
%\SetAlCapFnt{\small}
%\SetAlCapNameFnt{\small}
%\SetAlCapHSkip{0pt}
%\IncMargin{-\parindent}
%\pagenumbering{gobble}
% Choose a citation style by commenting/uncommenting the appropriate line:
%\setcitestyle{acmnumeric}
%\setcitestyle{authoryear}
\OneAndAHalfSpacedXI


\RequirePackage{amssymb,amsmath,ifthen,url,graphicx,color,array,theorem}
\TheoremsNumberedThrough     % Preferred (Theorem 1,
\ECRepeatTheorems

\begin{document}
% Title. Note the optional short title for running heads. In the interest of anonymization, please do not include any acknowledgements.

% Anonymized submission.
%\author{Submission XYZ}

% Abstract. Note that this must come before \maketitle.
\TITLE{Learning in Repeated Multi-Unit Pay-As-Bid Auctions}
\ARTICLEAUTHORS{%
\AUTHOR{Rigel Galgana}
\AFF{Operations Research Center, Massachusetts Institute of Technology, \EMAIL{rgalgana@mit.edu}, \URL{}}
\AUTHOR{Negin Golrezaei}
\AFF{Sloan School of Management, Massachusetts Institute of Technology,  \EMAIL{golrezaei@mit.edu}, \URL{}}
} 

\ABSTRACT{

Motivated by Carbon Emissions Trading Schemes, 
Treasury Auctions, Procurement
Auctions, and Wholesale Electricity Markets, which all involve the auctioning of homogeneous multiple units, 
we consider the problem of learning how to bid in repeated multi-unit pay-as-bid auctions. In each of these auctions, a large number of (identical) items are to be allocated to the largest submitted bids, where the price of each  of the winning bids is equal to the bid itself. In this work, we study the problem of optimizing bidding strategies from the perspective of a single bidder.


This problem is challenging due to the  combinatorial nature of the action space. We overcome this challenge by focusing on the offline setting, where the bidder optimizes their vector of bids while only having access to the past submitted bids by other bidders. We show that the optimal solution to  the offline problem can be obtained using a polynomial time dynamic programming (DP) scheme under which the bidder's utility is decoupled across units. We leverage the structure of the DP scheme to design online learning algorithms with polynomial time and space complexity under full information and bandit feedback settings. Under these two feedback structures, we achieve an upper bound on regret of $O(\Nitem \sqrt{\Nround \log |\mathcal{B}|})$ and $O(\Nitem\sqrt{|\mathcal{B}|\Nround\log |\mathcal{B}|})$ respectively, where $\Nitem$ is the number of units demanded by the bidder, $\Nround$ is the total number of auctions, and $|\mathcal{B}|$ is the size of the discretized bid space. We accompany these results with a regret lower bound, which match the linear dependency in $\Nitem$.

 Our numerical results suggest that when all agents behave according to our proposed no regret learning algorithms, the resulting market dynamics mainly converge to a welfare maximizing equilibrium where bidders submit uniform bids. We further show that added competition reduces the impact of strategization and bidders converge more rapidly to a
higher revenue and welfare steady state. Lastly,  our experiments  demonstrate that the pay-as-bid auction consistently generates significantly higher revenue compared to its popular alternative, the uniform price auction. This  advantage positions the pay-as-bid auction as an  appealing auction format in settings where earning high revenue holds significant social value, such as the Carbon Emissions Trading Scheme.
\noindent

\textbf{Keywords.} 
Multi-unit pay-as-bid auctions, Bidding strategies, Regret analysis, Market dynamics.



}

\maketitle

% Title page for title and abstract only.
%\begin{titlepage}

%\maketitle

%\end{titlepage}

% Paper body

\iffalse
TODO's

\begin{enumerate}
    \item Add section about CCE and Nash equilibrium, talk about properties of each of these and how they differ. Differences between discrete and continuous equilibria. See if we can discuss anything about CCE. See if we can say that all CCE that algorithms exhibit are also Nash equilibria, see if we have monotonicity, see what implications there are on revenue/welfare. See what aspects of the auction the auctioneer could change in order to improve revenue or reduce bid shading; i.e. reserve prices or randomize the supply
    \item Construct simulations using first synthetic data and then real world data afterwards
    \item Generalize active arm elimination as per Yanjun Han's paper
    \item Consider different feedback structure with the published clearing price
    \item Try different arm estimator, perhaps adding uniform exploration
\end{enumerate}
\fi
\section{Introduction}
Current quantum hardware is unable to carry out universal quantum computations due to the buildup of errors that occur during the computation. 
The magnitude of the individual error is currently above the value that the Threshold Theorem requires in order to kick-start quantum error correction and fault-tolerant quantum computation~\cite[Section 10.6]{nielsen_chuang_2010}. 
Although the experimentally achieved fidelity rates are promising and the error bounds are inching closer to the required threshold, we will have to work for the foreseeable future with quantum hardware with errors that build-up during the computation.  This implies that we can only do a limited number of steps before the output of the computation has become completely uncorrelated with the intended one.

For fault-tolerant quantum computing, we repeat four steps: 
1) We apply a number of single and two-qubit quantum gates, in parallel whenever possible; 
2) We perform a syndrome measurement on a subset of the qubits; 
3) We perform fast classical computations to determine which errors have occurred and how to correct them; 
and, 4) We apply correction terms based on the classical computations.
We then repeat these four steps with a next sequence of gates. 
These four steps are essential to fault-tolerant quantum computing. 


The starting point of this work is to use the four steps outlined above, not to carry out error correction and fault-tolerant computation, but to enhance short, constant-depth, {\em uncorrected} quantum circuits that perform single qubit gates and {\em nearest-neighbor} two qubit gates. 
Since in the long run we will have to implement error-correction and fault-tolerant computation anyhow, and this is done by such a four-step process, why not make other use of this architecture? Moreover, on some of the quantum hardware platforms, these operations are already in place.
Embracing this idea we naturally arrive at the question: what is the computational power of \textit{low-depth} quantum-classical circuits organized as in the four steps outlined above? 
We thus investigate circuits that execute a small, ideally constant, number of stages, where at each stage we may apply, in parallel, single qubit gates and {\em nearest-neighbor} two qubit gates, followed by measurements, followed by low-depth classical computations of which the outcome can control quantum gates in later stages. 
It is not clear, at first, whether such circuits, especially with constant depth, can do anything remotely useful. 
But we will see that this is indeed the case: many quantum computations can be done by such circuits in constant depth. 
By parallelizing quantum computations in this way, we improve the overall computational capabilities of these circuits, as we do not incur errors on qubits that are idle, simply because qubits are not idle for a very long time. 
Furthermore, reducing the depth of quantum circuits, at the cost of increasing width, allows the circuit to be run faster even if errors occur.

The first usage of such a four-step layout, not to do error correction, but to perform computations, can be found in the paradigm of measurement-based quantum computing~\cite{gottesman1999demonstrating,raussendorf2001one,jozsa2006introduction,clark2007generalised}: 
A universal form of quantum computing where a quantum state is prepared and operations are performed by measuring qubits in different bases, depending on previous measurements and intermediate measurements.

\citeauthor{PhamSvore2013} were the first to formalize the four-step protocol for performing computations~\cite{PhamSvore2013}. They included specific hardware topologies by considering two-dimensional graphs for imposing constraints on qubit interactions. In their model, they develop circuits for particularly useful multi-qubit gates, including specifying costs in the width, number of qubits, depth, number of concurrent time steps, size, and total number of non-Identity operations.
As a result, they find an algorithm that factors integers in polylogarithmic depth.
\citeauthor{Browne:2011} showed that the main tool in the work by \citeauthor{PhamSvore2013}, the fan-out gate, can also be replaced by additional log-depth classical computations in the measurement-based quantum computing setting~\cite{Browne:2011}.

More recently, \citeauthor{Cirac:2021} introduced a scheme to implement unitary operations involving quantum circuits combined with Local Operations and Classical Communication ($\mathsf{LOCC}$) channels: $\mathsf{LOCC}$-assisted quantum circuits~\cite{Cirac:2021}. Similarly to the four-step scheme we just described, they allow for a short depth circuit to be run on the qubits, followed by one round of $\mathsf{LOCC}$, in which ancilla qubits are measured and local unitaries are applied based on the measurement outcomes. They show that in this model any 1D transitionally invariant matrix-product state (MPS) with fixed bond dimension is in the same phase of matter as the trivial state. Similar ideas can be found in~\cite{TVV_NonAbelianTopologicalOrder_2022, tantivasadakarn2021long}.

In this work, we introduce a new model, called \textit{Local Alternating Quantum-Classical Computations} ($\LAQCC$). In this model we alternate between running quantum circuits (constrained by locality), ending in the measurement of a subset of qubits, and fast classical computations based on the measurement results. The outcome of the classical computations are then used to control future quantum circuits. We allow for flexibility in this model, by giving different constraints to the power of both the quantum circuits and the classical circuits as well as the number of alternations between them. 
Most attention will be given to $\LAQCC$ containing quantum circuits of constant depth, classical circuits of logarithmic depth and at most a constant number of alternations between them. 
Any circuit constructed in this model is considered to be of constant depth. 
We restrict ourselves to logarithmic depth classical computations, as this is the first natural and non-trivial extension beyond constant-depth classical computations. 
Constant-depth classical computations do however also have an equivalent constant-depth quantum implementation.

The definition of $\LAQCC$ sharpens the original definition of \citeauthor{PhamSvore2013} by adding constraints to the intermediate classical computations. This allows us to bound the power of $\LAQCC$ from above. 

The main result of \citeauthor{Cirac:2021}, that 1D translational invariant MPS with fixed bond dimension can be prepared by $\mathsf{LOCC}$-assisted circuits, relies on local symmetries of the MPS. These symmetries allow them to prepare local states (on a constant number of qubits) and glue them together by doing one round of the appropriate entangling measurement and corrections, after which they run a round of local unitaries to get the desired result. This general scheme for preparing states that exhibit an MPS description with the appropriate local symmetries requires only geometrically local unitaries and one round of measurement and corrections an therefore is accessible in $\LAQCC$. Studying different local symmetries, known as Symmetry Protected Topological (SPT) phases of matter, to find measurement-based constant depth circuits for states is a broad ongoing field of research~\cite{TVV_NonAbelianTopologicalOrder_2022, tantivasadakarn2021long, smith2023deterministic}. 
All these schemes have a $\LAQCC$ implementation.

%$\LAQCC$-circuits also exist for general schemes of preparing local states, based on the local tensors, and gluing them together using one round of entangled measurement and corrections, based on the local symmetry. 
%The main result of \citeauthor{Cirac:2021}, that 1D translational invariant MPS with fixed bond dimension can be prepared by $\mathsf{LOCC}$-assisted circuits, relies heavily on local symmetries of the MPS and as a result also has an equivalent $\LAQCC$ implementation. 
%The corrections applied after the measurement round are local unitaries depending on the local symmetries of the MPS. 

 

%This general scheme of preparing local states, based on the local tensors, and gluing it together by doing one round of entangled measurement and corrections, based on the local symmetry, is accessible in $\LAQCC$.
Note however that \citeauthor{Cirac:2021} also suggest a circuit for the $W$-state.
This circuit uses sequentially and dependent measurement-based corrections of the ancilla qubits. 
These dependent measurements translate to sequential alternations between the quantum and classical circuits and therefore increase the total depth to linear depth, exceeding the constant-depth constraints imposed by $\LAQCC$-circuits. 

We study the power of the $\LAQCC$ model with respect to state preparation, showing that even with only constant quantum-depth and logarithmic classical depth it remains possible to prepare states with long-range entanglement.
Another surprising result is that it is unlikely that $\LAQCC$ circuits are classically simulatable. We show that any instantaneous quantum polynomial-time (IQP) circuit~\cite{Bremner2010,Shepherd2009} has an $\LAQCC$ implementation.
Classical simulation of IQP circuits implies the collapse of the polynomial hierarchy to the third level, which is not believed to be true~\cite{Bremner2017}. Therefore, we expect that $\LAQCC$ circuits are unlikely to be classically simulatable. We bound the power of $\LAQCC$ by showing that it is contained in $\QNC^1$, the class of polynomial-size, log-depth circuits.

Next, we also study the power that intermediate classical calculations can add to quantum computations, by considering a new model that alternates between polynomially many polynomial-depth quantum circuits and unbounded classical computations
We study this model by doing a complexity theoretical analysis, where we draw inspiration from the notions of complexity given by \citeauthor{RosenthalYuen:2022}, \citeauthor{MetgerYuen:2023}, and \citeauthor{Aaronson:2004}.
All three complexity notions are based on the notion of state preparation, instead of more traditional definition of complexity such as the decidability of a computational problem. 
The first two consider classes based on sequences of quantum states preparable by a polynomial-sized quantum circuit, where the circuits are uniformly generated by a computational class, for instance, the class $\mathsf{PSPACE}$, which results in the complexity class $\mathsf{StatePSPACE}$~\cite{RosenthalYuen:2022,MetgerYuen:2023}.
The third notion considers a relative complexity, where the complexity is measured between two given states, and is measured by the number of gates, from a given gate-set, required to transform one state in another state~\cite{Aaronson:2004}. 
For our definition of state preparation complexity, we drop the uniformity constraint from~\cite{RosenthalYuen:2022,MetgerYuen:2023} and define a class as $\mathsf{StateX}$, which refers to states preparable by circuits of type $\mathsf{X}$. 
As an example, if $\mathsf{X} = \QNC^0$, this results in the class $\mathsf{StateQNC^0}$, which is the set of states preparable from the $\ket{0}^n$ state by poly-size constant-depth circuits. 
This notion is similar to the relative complexity from~\cite{Aaronson:2004}, where one state is the  $\ket{0}^n$ state and instead of counting the number of gates we consider the set of states preparable by a fixed number of gates. Using this notion of complexity we show that any state preparable by an $\LAQCC^*$ circuit is also preparable by a $\mathsf{PostQPoly}$ circuit, the class of circuits of polynomial depth with an additional post-selection gate. 

All Clifford circuits have a constant-depth $\LAQCC$ implementation, implying that any stabilizer state can be implemented by a constant-depth $\LAQCC$ circuit, see Section~\ref{sec:clifford_circuits} for a proof of this statement. 
Efficient circuits for stabilizer states have been known already through measurement-based quantum computing. Therefore this paper focuses on the preparation of non-stabilizer states, and as a surprising result we find novel constant-depth protocols for four very natural classes of non-stabilizer states.
Despite the extensive research into these four classes of non-stabilizer states and the many applications of them, no efficient constant- or low-depth state preparation protocols are known yet. We specifically consider these four classes as they are all often used as initial states in other algorithms.

The first state is a uniform superposition over an arbitrary number of states. 
This state finds applications in many quantum algorithms, as they often start with a uniform superposition over multiple states. 
This superposition is often achieved by applying Hadamard gates to every qubit due to its simplicity to prepare. 
Yet, the analysis of many algorithms, such as Shor's algorithm~\cite{Shor:1997}, would benefit from a different initial superposition. 
The circuit to prepare the uniform superposition over an arbitrary number of states uses an exact version of Grover search as a subroutine, that turns a probabilistic circuit, with a known constant probability of success, into a deterministic circuit. 
We use the circuit for preparing a uniform superposition over an arbitrary number of states as a subroutine in the next two quantum state preparation protocols. 

The second state is the $W$-state, the uniform superposition over all computational basis states of Hamming-weight~$1$, a natural long-ranged entangled state that displays a fundamentally nonequivalent type of entanglement from the Greenberger–Horne–Zeilinger state~\cite{WState:2000}, for which $\LAQCC$-type constant-depth circuits were previously known~\cite{PhamSvore2013, Cirac:2021}. 
The $W$-state is often used as benchmark for new quantum hardware~\cite{Haffner2005,Neeley2010,GarciaPerez:2021}. 
A novel way to prepare the $W$-state therefore gives a new way to benchmark different quantum devices with each other. 
A circuit for preparing the $W$-state was given in~\cite{Cirac:2021}, but this implementation requires sequentially alternating measurements followed by local unitaries, which in the $\LAQCC$ model is not considered to be of constant depth. 
We improve this protocol by giving an $\LAQCC$ implementation of the $W$-state, based on a compress-uncompress method that links the one-hot and binary encoding of integers.

The third state considered is the Dicke state, a generalization of the $W$-state, a superposition over all computational basis states with Hamming-weight $k$~\cite{Dicke:1954}. 
Dicke states have relevance in various practical settings.
For instance, for quantum game theory~\cite{zdemir2007}, quantum storage~\cite{Bacon_Compress:2006,Plesch:2010}, quantum error correction~\cite{ouyang2014permutation}, quantum metrology~\cite{toth2012multipartite}, and quantum networking~\cite{prevedel2009experimental}. 
Dicke states have been used as a starting state for variational optimization algorithms, most notably Quantum Alternating Operator Ansatz (QAOA)~\cite{Hadfield2019}, to find solutions to problems such as Maximum k-vertex Cover~\cite{Brandhofer2022,cook2020quantum}.
The ground states of physical Hamiltonians describing one-dimensional chains tend to show a resemblance to Dicke states such as states resulting from the Bethe ansatz, making them an ideal starting state when investigating the ground state behavior of these Hamiltonians~\cite{TDL_BetheAnsatzDerivation:2010,B_ExcitedStateQuantumPhaseTransitions:2013,DickeTransitions:2021}. 
For instance, the algorithm by \citeauthor{van2021preparing}, who give an algorithm to prepare the Bethe ansatz eigenstates of the spin-1/2 XXZ spin chain, starts by first preparing a Dicke state~\cite{van2021preparing}. 
A Dicke-state preparation protocol based on the compress-uncompress methodology used in the $W$-state furthermore finds applications in entanglement distillation, where the entanglement of a large state is concentrated on only a few qubits. 
Efficient deterministic circuits for preparing Dicke states have been proposed by \citeauthor{bartschi2019deterministic}~\cite{bartschi2019deterministic, bartschi2022deterministic_short_depth}. 
They provide a quantum circuit of depth $\mathO(k \log(\frac{n}{k}))$, allowing arbitrary connectivity, to prepare a Dicke state, which they conjecture to be optimal when $k$ is constant. 
In this work, we provide a constant-depth $\LAQCC$ circuit below their conjectured bound already for constant $k$. 
However, this does not directly disprove their conjecture, as we allow for intermediate measurements and classical computations. 
More significantly, we even construct constant-depth $\LAQCC$ circuits for $k = \mathO(\sqrt{n})$ greatly improving their bound.
This construction extends the compress-uncompress method for the $W$-state combined with additional subroutines. 

We continue with a log-depth state preparation protocol for the Dicke-state for arbitrary $k$. 
This protocol implements an efficient transformation between the factoradic number representation and the combinatorial number representation of a positive integer. 
The combinatorial number representation relates directly to the Dicke state. 
The provided efficient transformation between number representation systems might be of independent interest. 

We conclude by modifying our protocol for preparing a Dicke-state to a protocol that prepares quantum many-body scar states in constant-depth. 
These states have low entanglement and longer coherence times than states with similar energy density.
These characteristics make many-body scar states interesting to analyze and relevant within physics.
Many-body scar states appear for instance in the AKLT model~\cite{AKLT:1987,MRBAR:2018,MRB:2018} and different spin models~\cite{SI:2019,MOBFR:2020}.
Known methods for preparing these states have polynomial-depth~\cite{Gustafson:2023}, whereas our circuit has constant depth. 

% We conclude by studying the power that intermediate classical calculations can add to quantum computations. 
% In this study, we define a new model that relaxes constant-depth quantum circuits to polynomial depth quantum circuits, log-depth classical calculations to unbounded classical computations and a constant number of alternations to a polynomial number of alternations. 
% We call this model $\LAQCC^*$. 
% We study this model by doing a complexity theoretical analysis, where we draw inspiration from the notions of complexity given by \citeauthor{RosenthalYuen:2022}, \citeauthor{MetgerYuen:2023}, and \citeauthor{Aaronson:2004}.
% All three complexity notions are based on the notion of state preparation, instead of more traditional definition of complexity such as the decidability of a computational problem. 
% The first two consider classes based on sequences of quantum states preparable by a polynomial-sized quantum circuit, where the circuits are uniformly generated by a computational class, for instance, the class $\mathsf{PSPACE}$, which results in the complexity class $\mathsf{StatePSPACE}$~\cite{RosenthalYuen:2022,MetgerYuen:2023}.
% The third notion considers a relative complexity, where the complexity is measured between two given states, and is measured by the number of gates, from a given gate-set, required to transform one state in another state~\cite{Aaronson:2004}. 
% For our definition of state preparation complexity, we drop the uniformity constraint from~\cite{RosenthalYuen:2022,MetgerYuen:2023} and define a class as $\mathsf{StateX}$, which refers to states preparable by circuits of type $\mathsf{X}$. 
% As an example, if $\mathsf{X} = \QNC^0$, this results in the class $\mathsf{StateQNC^0}$, which is the set of states preparable from the $\ket{0}^n$ state by poly-size constant-depth circuits. 
% This notion is similar to the relative complexity from~\cite{Aaronson:2004}, where one state is the  $\ket{0}^n$ state and instead of counting the number of gates we consider the set of states preparable by a fixed number of gates. Using this notion of complexity we show that any state preparable by an $\LAQCC^*$ circuit is also preparable by a $\mathsf{PostQPoly}$ circuit, the class of circuits of polynomial depth with an additional post-selection gate. 

\paragraph{Summary of results}
\begin{itemize}
    \item We give a new definition of a computational model that captures the power of the four step process: applying a constant number of layers of one- and two-qubit gates; performing a syndrome measurement; perform a fast classical computation determining corrections; apply corrections. We call this model \emph{Local Alternating Quantum Classical Computations}, or $\LAQCC$ for short. In this model we bound the allowed quantum operations, intermediate classical calculations, and number of rounds separately. In Section~\ref{sec:LAQCC_model} we define this model and give a list of operations based on results from literature contained in this computational model. In some of these operations we explicitly use that we allow for multiple, but at most constant, rounds  of corrections.
    \item  We show show that there exist $\LAQCC$ circuits that can not be weakly simulated in Section~\ref{sec:IQP_in_LAQCC}. We further show that for every $\LAQCC$ circuit there exists a $\QNC^1$ circuit simulating it perfectly, in Section~\ref{sec:LAQCC_in_QNC1}.
    \item We introduce a new type computational complexity for preparing states and show that the extension of $\LAQCC$ where we allow a polynomial number of rounds and unbounded classical computation, is contained in $\mathsf{PostQPoly}$, the class of polynomial circuits with post-selection, in Section~\ref{sec:Complexity results}.
    \item We show a protocol to prepare the uniform superposition state of size $q$ in $\LAQCC$ using $\mathO(\ceil{\log_2(q)}^2)$ qubits in Section~\ref{sec:superposition_modulo_q}. 
    \item We show a protocol to prepare the $W_n$ state in $\LAQCC$ using $\mathO(n\log(n))$ qubits in Section~\ref{sec:W_state_in_LAQCC}.
    \item We show two ways of preparing the Dicke-$(n,k)$ state. The first method is in $\LAQCC$, works up to $k = \mathO(\sqrt{n})$, uses $\mathO(n^2\log(n))$ qubits, and is found in Section~\ref{sec:dicke:small_k}. The second method is in $\LAQCC\text{-}\mathsf{LOG}$ (an extension of $\LAQCC$ allowing for logarithmic number of alterations instead of constant), works for any $k$, uses $\mathO(\text{poly}(n))$ qubits, and is found in Section~\ref{sec:Dicke_in_LAQCC_LOG}. 
    \item We extend on our $\LAQCC$ method of generating Dicke-$(n,k)$ states for $k = \mathO(\sqrt{n})$ and show a protocol to generate many-body scar states for a particular Hamiltonian in $\LAQCC$ (Section~\ref{sec:many_body_scar}). 
\end{itemize}
Summarized in a table, we provide the following state generation protocols:
\begin{table}[htb]
\centering
\begin{tabular}{l|l|l|l}
\textbf{State description} & \textbf{Width} & \textbf{Depth} & \textbf{Implementation}\\
\hline 
Uniform superposition mod $q$: $\frac{1}{\sqrt{q}} \sum_{i = 0}^{q-1}\ket{i}$ & $\mathO(\ceil{\log^2 q})$ & $\mathO(1)$ & Section~\ref{sec:superposition_modulo_q}\\

$W$-state: $\frac{1}{\sqrt{n}}\sum_{i = 0}^{n-1}\ket{e_i}$ & $\mathO(n \log n)$ & $\mathO(1)$ & Section~\ref{sec:W_state_in_LAQCC}\\

Dicke-$(n,k)$, $k = \mathO(\sqrt{n})$: $\binom{n}{k}^{-1/2}\sum_{x \in \{0,1\}^n: |x| = k} \ket{x}$ &  $\mathO(n^2\log n)$ & $\mathO(1)$ 
&Section~\ref{sec:dicke:small_k}\\

Dicke-$(n,k)$: $\binom{n}{k}^{-1/2}\sum_{x \in \{0,1\}^n: |x| = k} \ket{x}$ & $\mathO(\text{poly}(n))$ & $\mathO(\log n)$ &Section~\ref{sec:Dicke_in_LAQCC_LOG}\\

QMBS: $\ket{S_k} = \frac{1}{k! \sqrt{\mathcal N(n,k)}}(Q^\dagger)^k \ket{\Omega}$ &  $\mathO(n^2\log n)$ & $\mathO(1)$  &  Section~\ref{sec:many_body_scar}
\end{tabular}
\caption{Summary of state preparation protocols given in this paper.}
\label{tab:sate_prep}
\end{table}
In the entry for the quantum many-body scar state $Q$ denotes the raising operator and $\mathcal N(n,k)=\binom{n-k-1}{k}$. 
Section~\ref{sec:many_body_scar} will provide more details on the variables and the implementation. 

\paragraph{Organization of the paper}
\noindent We first introduce relevant preliminaries in Section~\ref{sec:preliminaries}. 
In Section~\ref{sec:LAQCC_model} we formally define the class of Local Alternating Quantum-Classical Computations ($\LAQCC$). We also show that any Clifford circuit can be implemented in constant depth $\LAQCC$ (a result based on a result from measurement-based quantum computing~\cite{jozsa2006introduction}). 
This result allows us to give many useful multi-qubit gates and routines in Section~\ref{sec:gates_created_in_LAQCC}. 
Beyond that we show that constant depth $\LAQCC$ circuits are contained in $\QNC^1$ and that any $\mathsf{IQP}$ circuit has an $\LAQCC$ implementation.
We conclude this section with an analysis of a more powerful instantiation of $\LAQCC$ and show an inclusion with respect to the class $\mathsf{PostQPoly}$, which is the class of circuits of polynomial depth with one additional post-selection gate. 
In Section~\ref{sec:state_prep_in_LAQCC} we give $\LAQCC$ circuit implementations for preparing the uniform superposition over an arbitrary number of states, the $W$-state and the Dicke state up to $k = \mathO(\sqrt{n})$. We furthermore give a log-depth circuit implementation for preparing the Dicke state for any $k$. We conclude by showing a $\LAQCC$ circuit for generating many body scar states of a particular type of Hamiltonian.


%\section{Preliminaries}

\subsection{Model}
\textbf{Auction format: Pay-as-bid.} Consider a setting with $N$ bidders and $M$ units to auction off in a pay-as-bid auction with reserve $r\ge 0$. Let ${\bf b}_n = (b_{n, m})_{n\in [N], m\in [M]}$  be the bids submitted by bidder $n$, where $b_{n, 1}\ge b_{n, 2}\ge b_{n, M}\ge 0$ and $b_{n, m}$ is the bid of bidder $n$ for the $m$-th unit. One can view $b_{n,m}$, $n\in [N]$ and $m\in [M]$, as a proxy for the bidder $n$'s (marginal) valuation for the $m$-th unit, denoted by $v_{n, m}$, where we assume that $v_{n, 1}\ge v_{n, 2}\ge \ldots \ge v_{n, M}$ to model the diminishing return of one additional unit.  
In a pay-as-bid auction with reserve $r$,  any  bid that is less than reserve price $r$ is first discarded. The remaining (cleared) bids are then sorted in decreasing order.  The $m$-th unit is allocated to the bidder with the $m$-th highest bid if the number of cleared bids is greater than or equal to $m$, charging him his bid.
 {\color{red}todo: briefly talk about the tie breaking rule.}  That is, if bidder $n$ is allocated $x_n(\mathbf b)$ units, he is charged $\sum_{m=1}^{x_n(\mathbf b)} b_{n,m}$ while his (total) valuation is $\sum_{m=1}^{x_n(\mathbf b)} v_{n,m}$, where $\mathbf b = ((\mathbf b_n)_{n\in [N]}; r)$ is the submitted bids by all the bidders and reserve price $r$ {\color{red} for now I combined the bids and reserve price}.  This leads to (quasi-linear) utility of 
\begin{align}\label{eq:mu}\mu_n(\mathbf b) = \sum_{m=1}^{x_n(\mathbf b)} \big(v_{n,m} -b_{n,m}\big)\,\end{align}
for bidder $n$.

%The auctioneer announces $\m$ units of a good for sale. Each player $i$ submits bids $\vec{b}_{i} = (b_{i,1}, \ldots, b_{i,\m})$, where $b_{i,j}$ 
 %is player $i$'s bid for $j$-th unit.
%	The auctioneer sorts  the bids in decreasing order. Then   for each $j = 1, \ldots, \m$,  the auctioneer allocates the $j$-th unit to the player that submitted the $j$-th highest bid, charging them a price equal to the $(\m+1)$-th highest bid.

\textbf{Repeated Setting.} In this work, we consider a repeated setting where the pay-as-bid auction is run over the course of $T$ rounds. We denote the bid of bidder $n$ in the $t$-th auction ($t\in [T]$) by ${\bf b}_n^t = (b_{n, m}^t)_{n\in [N], m\in [M]}$ and define $\mathbf b^t = ((\mathbf b_n^t)_{n\in [N]}; r^t)$, where $r^t$ is the reserve price in auction $t$. Similarly, we define $\mathbf b^t_{-n} = ((\mathbf b_i^t)_{i\in [N], i\ne n}; r^t)$ as the bids submitted by all the bidders expect bidder $n$ in auction $t$, and reserve prices $r_t$.



\textbf{Offline Setting.} In the offline setting, we aim to optimize the biding strategy of a bidder $n$ while having access to $H_T :=(\mathbf b^1_{-n}, \mathbf b^2_{-n}, \ldots , \mathbf b^T_{-n})$, which is the submitted bids by the other bidders and reserve prices in the past $T$ rounds. Mathematically speaking, we fix a bidder $n$, and we optimize  a bid vector $\mathbf b_n$ that maximizes the bidder $n$'s cumulative utility over $T$ rounds with competing bids and reserve prices are $H_T$:
\begin{align}\tag{Offline}\label{eq:offline}
    \max_{\mathbf b_n } \sum_{t=1}^T \mu_n ((\mathbf b_{-n}^t, \mathbf b_n)) \qquad  \text{s.t.} \qquad b_{n,1}\ge  \ldots \ge b_{n, M}\ge 0\,.\end{align}
{\color{red} todo: please motivate this setting and say what result we will have here and where}

\textbf{Online  Setting.} In the online setting, we again fix a bidder $n$, and we aim to 



{\color{red} discussion here should be written so that we can use it for both offline and online settings }
We consider $\Nround$ rounds of homogeneous multi-unit auctions, each with $\Nitem$ items to be allocated across $N$ agents. In round $\nround$, bidder $n$ is endowed with weakly decreasing marginal valuation profile $\bm{v}^{n, \nround} \in \{\bm{v}: 1 \geq v_1 \geq \ldots \geq v_\Nitem \geq 0\} \equiv [0, 1]^\Nitem$. {\color{red} we don't need discretization for the offline setting. So, I suggest we consider the continous setting here in the model section and when it comes to the learning section, we talk about discretization} Each bidder submits weakly decreasing bids $\bm{b}^{n, \nround} \in \{\bm{b}: 1 \geq b_1 \geq \ldots \geq b_\Nitem \geq 0, b_\nitem \in \mathcal{B} \forall \nitem \in [\Nitem]\}$, where the set of allowable bids $\mathcal{B} = \{B_1,\ldots,B_{|\mathcal{B}|}\}$ with $0 = B_1 < \ldots < B_{|\mathcal{B}|} = 1$ denotes some discretization of $[0, 1]$. We abuse notation and denote this set of non-decreasing $\mathcal{B}$-restricted bids as $\mathcal{B}^\Nitem$. The auctioneer selects anonymous reserve price $\pi^\nround \in \mathcal{B}$ which filters out all bids strictly below $\pi$ from consideration. Considering only bids at least $\pi^\nround$, agents are allocated some number of items $x(\bm{b}^{n, \nround}, \bm{b}^{-n, \nround}, \pi^\nround)$ equal to the number of bids within the $\Nitem$ largest bids. Here, $\bm{b}^{-n, \nround}$ denotes the largest $\Nitem$ bids among all other agents in decreasing order and we again abuse notation to let $\mathcal{B}^{-\Nitem}$ denote the set of possible $\bm{b}^{-n, \nround}$. We also assume tie-breaks are settled using a public knowledge, deterministic tie-breaking mechanism. Assuming agent $n$ wins $x^{n, \nround} = x(\bm{b}^{n, \nround}, \bm{b}^{-n, \nround}, \pi^\nround)$ items, the utility $\mu(\bm{v}^{n, \nround}, \bm{b}^{n, \nround}, \bm{b}^{-n, \nround}, \pi^\nround)$ is given by the difference in reward $\sum_{\nitem=1}^{x^{n, \nround}} v^{n, \nround}_\nitem$ and their payment $\sum_{\nitem=1}^{x^{n, \nround}}b^{n, \nround}_\nitem$. The welfare and revenue of the auction is equal to the sum over all agents' rewards and payments respectively. At the end of each auction $\nround$, the auctioneer reveals market clearing price $c^\nround = \max(b^{n, \nround}_{x^{n, \nround}}, b^{-n, \nround}_{x^{n, \nround}+1}, \pi^\nround)$, defined as the maximum of $\pi^\nround$ and the lowest winning bid round $\nround$. Agents additionally observe their own allocation $x^{n, \nround}$ for all $n \in [N]$. We define the set of information known to agent $n$ at round $\nround$ before submitting their bid to be $H^{n, \nround} = v^{n, \nround} \cup \{(v^{n, \tau}, x^{n, \tau}, c^\tau)\}_{\tau \in [\nround - 1]}$. The set of repeated auctions is described succinctly as follows:

% Algorithm
\begin{algorithm}[t]
	\KwIn{{\color{red} what is this? why do we have an algorithm  here?}Valuations $\{\bm{v}^{n, \nround}\}_{n \in [N], \nround \in [\Nround]}$ for $\bm{v}^{n, \nround} \in \{\bm{v}: 1 \geq v_1 \geq \ldots \geq v_\Nitem \geq 0\}$, Bids $\{\bm{b}^{n, \nround}\}_{n \in [N], \nround \in [\Nround]}$ for $\bm{b}^n \in \mathcal{B}^\Nitem,$ Reserves $\{\pi^\nround\}_{\nround \in [\Nround]}$}
	\KwOut{Aggregate utilities $\{\mu^n\}_{n \in [N]}$, aggregate welfare $\sum_{n = 1}^N \textsc{Reward}^n$, total revenue $\sum_{n = 1}^N \textsc{Payment}^n$.}
	$\mu^n, \textsc{Reward}^n, \textsc{Payment}^n \gets 0$ for all $n \in [N]$\;
        $H^{n, 1} = \{\bm{v}^{n, 1}\}$ for all $n \in [N]$
	\For{$\nround \in [\Nround]$:}{
            Bidders submit bids $\bm{b}^{n, \nround}$ for all $n \in [N]$\;
    	\For{$n \in [N]$:}{
                Observe allocation $x^{n, \nround}$ and clearing price $c^\nround = \max(b^{n, \nround}_{x^{n, \nround}}, b^{-n, \nround}_{x^{n, \nround}+1}, \pi^\nround)$\;
        		% $x^{n, \nround} \gets \sum_{\nitem=1}^\Nitem \textbf{1}_{b^{n, \nround}_\nitem \geq \max(\pi^\nround, b^{-n, \nround}_\nitem)}$\;
    		$\textsc{Reward}^n \gets \textsc{Reward}^n + \sum_{\nitem=1}^{x^{n, \nround}} v^{n, \nround}_\nitem$\;
    		$\textsc{Payment}^n \gets \textsc{Payment}^n + \sum_{\nitem=1}^{x^{n, \nround}} b^{n, \nround}_\nitem$\;
                $\mu^n \gets \mu^n + \textsc{Reward}^n - \textsc{Payment}^n$\;
                $H^{n, \nround + 1} \gets H^{n, \nround} \cup (\bm{v}^{n, \nround+1}, x^{n, \nround}, c^\nround)$
    	}
        }
        \textbf{Return} $\{\mu^n\}_{n \in [N]}, \sum_{n = 1}^N \textsc{Reward}^n, \sum_{n = 1}^N \textsc{Payment}^n$
	\caption{\textsc{MUA}$(\{\bm{v}^{n, \nround}, \bm{b}^{n, \nround}, \pi^\nround\}_{n \in [N], \nround \in [\Nround]})$}
	\label{alg:MUA}
\end{algorithm}

\rigel{Perhaps mention that our censored feedback setting is equivalent to Yanjun Han's} We explain two technicalities our model selection and description. First, the tiebreaks are implicitly handled within $x^{n, \nround}$ in a deterministic fashion known to each agent. To avoid dealing with indexing issues, we perturb each bidder $n$'s bid by $(N-n)\epsilon$ for some infinitesimal $\epsilon > 0$, which has the effect of prioritizing lower indexed bidders at a negligible increase in payment. For the remainder of the paper, we will implicitly assume as shorthand that the event $\{b^{n, \nround}_\nitem = B\}$ for $B \in \mathcal{B}$ actually means $\{b^{n, \nround}_\nitem = B + (N-n)\epsilon = B'\}$ for $B' \in \mathcal{B}_n$, where agent $n$'s bid set is defined as $\mathcal{B}_n \equiv \{B: B + (N-n)\epsilon\}_{B \in \mathcal{B}}$ as opposed to $\mathcal{B}$. Under this assumption, we can decompose $x^{n, \nround}$ as a function of $\bm{b}^{n, \nround}, \bm{b}^{-n, \nround}, \pi^\nround$ which greatly simplifies the offline and online learning algorithms and corresponding analyses. Second, we assume that information regarding the reserve price is only revealed after each auction via the clearing price. We will later consider variants of the auction where $H^{n, \nround} = (v^{n, \nround}, \pi^\nround) \cup \{(v^{n, \tau}, x^{n, \tau}, c^\tau)\}_{\tau \in [\nround - 1]}$ the reserve price is announced at the beginning of each auction, where the analysis is slightly more complicated. As reserve prices are selected adaptively by the auctioneer in many real world situations, from the point of view of the agents, we assume that reserve prices are be generated adversarially.
% We also discuss generalizations to discriminatory reserves $\{\pi^\nround_\nitem\}_{\nitem \in [\Nitem], \nround \in [\Nround]}$ or non-anonymous reserves $\{\pi^\nround_n\}_{n \in [N], \nround \in [\Nround]}$.

\subsection{Problem Statement}

The objective of each agent $n$ is to select a sequence of bid vectors $\{\bm{b}^{n, \nround}\}_{n \in [N], \nround \in [\Nround]}$ that maximizes their aggregate utility $\mu^n$. Optimizing $\mu^n$ is trivial in the offline setting where each of $\bm{b}^{-n, \nround}$ and $\pi^\nround$ are revealed to agent $n$ in advance and thus can correspondingly select the optimal $\bm{b}^{n, \nround}$ for each $\nround \in [\Nround]$. Rather than the optimal sequence of bid vectors, we can instead consider the optimal fixed bid vector which we show later can be computed using a dynamic program efficiently. Of course, $\bm{b}^{-n, \nround}$ and $\pi^\nround$ for all $\nround \in [\Nround]$ are not known beforehand and are instead revealed in an online fashion. As such, the agents must learn how to bid optimally during the sequence of auctions only given knowledge of historic auction results---which we will formalize shortly what this means---and their current valuation profile. The performance of an agent's learning strategy will be measured in terms of regret---the difference between their expected utility under their learning strategy and under the hindsight optimal fixed bid vector. In particular, agents will select a bid vector sampled from $F^{n, \nround}(H^{n, \nround}) = F^{n, \nround} \in \Delta(\mathcal{B}^\Nitem)$ where $\Delta(S)$ denotes the set of all valid probability measures over set $S$.
\begin{align*}
    \textsc{Regret}^{n, \Nround}(F^{n, \nround} \mid \{\bm{v}^{n, \nround}\}_{\nround \in [\Nround]} = \max_{\bm{b}^n \in \mathcal{B}^\Nitem} \sum_{\nround=1}^\Nround \mu(\bm{v}^{n, \nround}, \bm{b}^n, \bm{b}^{-n, \nround}, \pi^{\nround}) - \mathbb{E}_{\bm{b}^{n, \nround} \sim F^{n, \nround}} \sum_{\nround=1}^\Nround \mu (\bm{v}^{n, \nround}, \bm{b}^{n, \nround}, \bm{b}^{-n, \nround}, \pi^\nround)
\end{align*}
Where we assume that $\bm{v}^{n, \nround}$ (which can be thought of as a context as it is revealed at the start of each round), $\bm{b}^{-n, \nround}$ and $\pi^\nround$ can be selected by an adaptive adversary. In the bandit case, we assume that $\bm{v}^{n, \nround}$ are fixed to be $\bm{v}^\nround$. according to some known distribution. For simplicity, we omit the arguments when they are clear from context: $\textsc{Regret}^{n, \Nround} = \textsc{Regret}^{n, \Nround}(F^{n, \nround} \mid \{\bm{v}^{n, \nround}\}_{\nround \in [\Nround]}$. We wish to derive a learning algorithm---construct functions $F^{n, \nround}$---that achieves an upper bound on $\textsc{Regret}^{n, \Nround}$ that is polynomial (or better) in $\Nitem$ and sub-linear in $\Nround$. Assuming that all agents obey this learning algorithm, we derive bounds on the expected revenue and expected welfare for a sequence of reserves $\{\pi^{\nround}\}_{\nround \in [\Nround]}$ and compare these guarantees to the announced reserves setting.

\subsection{Contributions}

Our primary contribution is representing the allocation and utility functions in a form that enables maximal cross learning between bid vectors and valuations. In particular, we define $x^{n, \nround} = \sum_{\nitem=1}^\Nitem \textbf{1}_{b^{n, \nround}_\nitem \geq \max(\pi^\nround, b^{-n, \nround}_\nitem)}$ which decomposes the utility function as a sum over the the utilities per slot $\nitem$. 
We then construct an efficient $O(\Nitem |\mathcal{B}|^2)$ time and $O(\Nitem^2 |\mathcal{B}|)$ space complexity dynamic program that computes the hindsight optimal bid vector. In this dynamic program, we show a crucial (conditional) independence between agent $n$'s utility corresponding to their bid in the $\nitem$'th slot $b^{n, \nround}_\nitem$ and their bid in the $\nitem+1$'st slow $b^{n, \nround}_{\nitem + 1}$. With this, we decouple the aggregate utility of a bid vector $\bm{b}^{n}$ as a function of the per-slot aggregate utilities. This allows us to efficiently implement exponential weights in the full information setting, which achieves regret $O(\Nitem \sqrt{T \log |\mathcal{B}|}$, whilst allowing for adversarially generated valuation profiles $\{\bm{v}^{n, \nround}\}_{\nround \in [\Nround]}$. In the bandit setting, one may be hopeful to be able to apply the same decoupling argument to graph-feedback learning algorithms, such as $\textsc{Exp3.G}$ or $\textsc{Exp3.SET}$ \rigel{cite}, which would make use of side information in the form of $c^\nround$. Unfortunately, the learner never observes the entire feedback graph and therefore cannot construct meaningful, polynomially bounded variance utility estimates. Instead, we can implement Follow-the-Regularized-Leader (FTRL) style algorithms, such as $\textsc{O-REPS}$ or $\textsc{Component Hedge}$ \rigel{cite}, which achieves low regret $O(\Nitem |\mathcal{B}|\sqrt{T\log|\mathcal{B}|})$ and polynomial time and space complexity. However, these algorithms must assume a static valuation profile and require additional setup.

% We apply this idea again in the online learning setting. In particular, we derive a decoupled $\textsc{Exp3}$ algorithm that exploits cross learning (See algorithm $\textsc{Exp3.G}$ in \rigel{Cite paper}) which achieves small regret whilst retaining polynomial time and space complexities despite the combinatorially large bid space. We do this by upper bounding the independence number of any possible bid vector network by realizing that the utility function can be fully characterized by the $\Nitem$ largest bids amongst $\bm{b}^{-n, \nround}$. Lastly, we show how to convert the output of our decoupled $\textsc{Exp3.G}$ into a valid (weakly monotonic) bid vector.

\section{Preliminaries}\label{sec:model}

\textbf{Notation.} We let $|\mathcal{S}|$ denote the size of set $\mathcal{S}$, and define $[k] = \{1,\ldots,k\}$ to be the set of the first $k$ positive integers. We define $\mathcal{S}^{+k}$ to be the set of non-increasing $k$-vectors of elements from set $\mathcal{S}$. Similarly, $\mathcal{S}^{-k}$ denotes the set of non-decreasing $k$-vectors of elements from set $\mathcal{S}$. We let $\Delta(\mathcal{S})$ denote the set of valid probability measures over set $\mathcal{S}$. We also say that the quantity $x$ is $\lesssim$, $\propto$, or $\gtrsim$ than $f(a_1,\ldots,a_k)$ some function of $k$ algorithm parameters if $x \in \mathcal{O}(f(a_1,\ldots,a_k))$, $x \in \Theta(f(a_1,\ldots,a_k))$, or $x \in \Omega(f(a_1,\ldots,a_k))$,  respectively.



\textbf{Auction format: Pay-as-bid.} 
 Consider a scenario in which there are $N$ bidders and $\overline{M}$ identical units available for auction in a PAB  format. In this context, we assume that each agent $n$ within the set $[N]$ desires a maximum of $M$ units. (It should be noted that our results can be extended to a situation where each agent $n$ may demand a different maximum number of units, denoted by $M_n$, which are not necessarily identical.)
 
 Let $\bm{v}_n = (v_{n, \nitem})_{\nitem \in [M]} \in [0, 1]^{+M}$ represent agent $n$'s non-increasing marginal valuation profile. This implies that for any given $n$ in the set $[N]$, the following conditions hold: (i) {valuation monotonicity} $v_{n,1}\ge v_{n, 2}\ge \ldots \ge v_{n, \Nitem}$ and (ii) the total valuation of agent $n$ after receiving $\nitem$ units is given by $\sum_{k=1}^{m}v_{n, k}$.

 Let $\bm{b}_n = \{b_{n, \nitem}\}_{\nitem \in [\Nitem]} \in [0, 1]^{+\Nitem}$  represent the non-increasing bids submitted by bidder $n$. Here, $b_{n, \nitem}$ refers to the bid made by bidder $n$ for the $\nitem$-th slot or, equivalently, the $\nitem$-th unit. Similar to $\bm{v}_n$, we have the following conditions for any $n \in [N]$: (i) {bid monotonicity}
$b_{n,1}\ge b_{n, 2}\ge \ldots \ge b_{n, \Nitem}$ and  (ii) individual rationality (IR) $b_{n, \nitem} \leq v_{n, \nitem}$ for all $m \in [M]$. 
It is important to note that the bid monotonicity condition is not an assumptions; it is implied by the auction rule that will be stated shortly. Consequently, the total payment made by bidder $n$ after receiving $\nitem$ units is given by $\sum_{k=1}^{m} b_{n, k}$. We define $\bm{b}_{-n} = (b_{-n,\nitem})_{\nitem \in [\overline{\Nitem}]} \in [0, 1]^{-\overline{M}}$  as the set of the $\overline{\Nitem}$ largest bids not belonging to agent $n$, arranged in increasing order (i.e., $b_{-n, 1} \leq \ldots \leq b_{-n, \overline{\Nitem}}$).



 The auction operates according to the following rules:
 In a PAB auction, all bids submitted across the $N$ bidders are arranged in descending order. The $\nitem$-th unit is assigned to the bidder with the $\nitem$-th highest bid, and they are charged the amount of their bid. We denote the allocation to agent $n$ as $x_n(\bm{b}_n) = x(\bm{b}_{n}, \bm{b}_{-n})$, and the (quasi-linear) utility as $\mu_n(\bm{b}_n) = \mu(\bm{b}_{n}, \bm{b}_{-n})$, where
\begin{align}
    \label{eq: allocation and utility definition}
    x(\bm{b}_{n}, \bm{b}_{-n}) = \sum_{\nitem=1}^\Nitem \textbf{1}_{b_{n, \nitem} \geq b_{-n, \nitem}} \quad \text{and} \quad \mu(\bm{b}_{n}, \bm{b}_{-n}) = \sum_{\nitem=1}^{x_n(\bm{b}_n)} (v_{n, \nitem} - b_{n, \nitem})\,.
\end{align}
respectively. Here, $b_{-n, \nitem}$ represents the $\nitem$-th smallest bid among the $\overline{M}$ largest bids of all other bidders except bidder $n$.
It should be noted that $x(\bm{b}_{n}, \bm{b}_{-n})$ denotes the number of units that agent $n$ receives in the auction. In the case of tied bids, we assume the use of an arbitrary, publicly known deterministic tie-breaking rule, denoted as $\textsc{TieBreak}_n: \mathcal{R}^{+\Nitem} \times \mathcal{R}^{-\overline{\Nitem}} \to [\Nitem]$, to determine the allocation for agent $n$. This tie-breaking rule is incorporated into the allocation as $x(\bm{b}_n, \bm{b}_{-n}) = \sum_{\nitem=1}^\Nitem \textbf{1}_{b_{n, \nitem} > b_{-n, \nitem}} + \textsc{TieBreak}_n(\bm{b}_n, \bm{b}_{-n})$, and we use this shorthand notation in Equation \eqref{eq: allocation and utility definition}.\medskip




\textbf{Online/Repeated Setting.} Consider a repeated setting where the PAB auction is conducted over $\Nround$ rounds. In this repeated setting, we will focus on the perspective of agent $n \in [N]$ and remove additional indexing when it is evident from the context. For each auction round, agent $n$ has a fixed valuation profile represented by 
$\bm{v} = (v)_{\nitem \in [\Nitem]} \in [0, 1]^{+\Nitem}$, and their
 bid vector in the $\nround$-th auction is denoted by $\bm{b}^\nround = (b^\nround_\nitem)_{\nitem \in [\Nitem]} \in [0, 1]^{+\Nitem}$.\footnote{In Section \ref{sec: time varying}, we extend our results to the case of time-varying valuations.} 
Similarly, $\bm{b}^\nround_{-} = (b^\nround_{-m})_{m \in [\overline{\Nitem}]} \in [0, 1]^{-\overline{\Nitem}}$ represents the competing bids in round $\nround$. In each round, agent $n$ receives $x^\nround_n(\bm{b}^\nround)$ units and earns a utility of $\mu^\nround_n(\bm{b}^\nround)$, where:
\begin{align}
    \label{eq: allocation and utility definition repeated}
    x^\nround_n(\bm{b}^\nround) = x(\bm{b}^\nround, \bm{b}^\nround_{-}) \quad \text{and} \quad \mu^\nround_n(\bm{b}^\nround) = \mu(\bm{b}^\nround, \bm{b}^\nround_{-})\,.
\end{align}
Recall that functions $x$ and $\mu$ are defined in Equation \eqref{eq: allocation and utility definition}. 
The goal of agent $n$ is to choose a sequence of bid vectors $(\bm{b}^\nround)_{\nround \in [\Nround]}$ that maximizes their total utility, given by $\sum_{\nround=1}^\Nround \mu^\nround_n(\bm{b}^\nround)$. However, the main challenge is that the vectors $\bm{b}^{\nround}_{-}$, representing the competing bids, are not known in advance. Instead, they are revealed in an online manner. Consequently, agents must learn how to bid optimally throughout the sequence of auctions, taking into account their previous allocations $H^{\nround-1} = (x^\tau(\bm{b}^\tau))_{\tau \in [\nround-1]}$ and their valuation profile $\bm{v}$. The performance of an agent's learning strategy is evaluated in terms of regret, which quantifies the difference between their expected utility using their learning strategy and the optimal utility achievable with perfect knowledge of the competing  bidding vectors in hindsight:
\begin{align}
\tag{Continuous Regret}
    \textsc{Regret} = \max_{\bm{b} \in [0, 1]^{+\Nitem}} \sum_{\nround=1}^\Nround \mu^\nround_n(\bm{b}) - \mathbb{E}\left[\sum_{\nround=1}^\Nround \mu^\nround_n(\bm{b}^\nround)\right]\,.
\end{align}
Here, $\bm{b}^{\nround}_-$ can be selected by an adaptive adversary; i.e. an adversary who can select $\bm{b}^\nround_-$ as a function of the entire auction history, which includes $\bm{b}^1,\ldots,\bm{b}^{\nround-1}$, but does not have access to the possible randomness when selecting $\bm{b}^\nround$. One example is when the other competitors are also behaving according to no-regret learning algorithms. 
We note that it is known that the time averaged iterates in the game dynamics induced by agents running no-regret learning algorithms converges to a coarse correlated equilibrium (CCE), which have desirable revenue and welfare guarantees via smooth-auction PoA analysis \citep{inefficiency2013, feldman2017correlated}. As few theoretical results are known regarding the structure of CCE's in PAB auctions, we hope our proposed algorithms and experiments will provide useful insights in this direction. We will discuss this further in Section~\ref{sec: experiments}.



The benchmark, $\max_{\bm{b} \in [0, 1]^{\Nitem}} \sum_{\nround=1}^\Nround \mu^\nround_n(\bm{b})$ used in the definition of continuous regret, is constructed considering all possible $\bm{b}^{\nround}_{-}$ for every round $\nround \in [\Nround]$, where in the hindsight optimal solution, the bid vector can be chosen from any vector $\bm{b} \in [0, 1]^{+\Nitem}$. However, in practice, bid vectors are often restricted to a discretization of $[0, 1]$ denoted by $\mathcal{B} = (B_1,\ldots,B_{|\mathcal{B}|})$, where $0 = B_1 < \ldots < B_{|\mathcal{B}|} = 1$. For such cases, we define an analogous version of regret:
\begin{align}
\tag{Discretized Regret}
    \textsc{Regret}_\mathcal{B} = \max_{\bm{b} \in \mathcal{B}^{+\Nitem}} \sum_{\nround=1}^\Nround \mu^\nround_n(\bm{b}) - \mathbb{E}\left[\sum_{\nround=1}^\Nround \mu^\nround_n(\bm{b}^\nround)\right]\,.
\end{align}
In both the definitions of continuous and discretized regret, we henceforth implicitly assume that $b_m\le v_m$ for any $m\in [M]$; that is, we have  the bid vector $\bm{b}$ is subject to individual rationality and overbidding is not allowed. Observe that the benchmark in defining $\textsc{Regret}_\mathcal{B}$ is weaker than that in $\textsc{Regret}$. Nevertheless, they are not too far from each other as we discuss in the following sections.  We wish to derive a learning algorithm that achieves an upper bound on $\textsc{Regret}$ that is polynomial  in $\Nitem$ and sub-linear in $\Nround$. To do so,  we consider the discretized setting, and bound $\textsc{Regret}_\mathcal{B}$; an upper bound on $\textsc{Regret}$ will be obtained by accounting for the discretization errors. 

We consider two feedback structures: (i) full information and (ii) bandit. In the full information setting, the agent's allocation and the values of $\bm{b}^\nround_-$ are revealed after the end of each round, whereas in the bandit setting, only the agent's allocation is revealed. 

\section{Hindsight Optimal Offline Solution}\label{sec:offline}


In the offline setting, our goal is to determine agent $n$'s optimal fixed bidding strategy for the $\Nround$ rounds of PAB auctions. Recall the following optimization problem with bid space $\mathcal{B}$:
\begin{align}
\tag{Offline}
\label{eq:offline}
\max_{\bm{b} \in \mathcal{B}^{+\Nitem}} \sum_{\nround=1}^\Nround \mu^\nround_n(\bm{b})\,,
\end{align}
where $\mu^\nround_n(\bm{b})$ (defined in Equation \eqref{eq: allocation and utility definition repeated}) represents the utility of agent $n$ in round $\nround$ given bid vector $\bm{b}$ and competing bids $\bm{b}_-^t$. As mentioned earlier, the solution to this optimization problem serves as a benchmark for evaluating the performance of online learning algorithms in the repeated setting. Furthermore, it provides valuable insights for designing algorithms with polynomial time and space complexity for the repeated setting.

  

To solve  Problem \eqref{eq:offline}, we take advantage of the following decomposition: 
\begin{align}
    \label{eq: decomposition}
    \sum_{\nround=1}^\Nround \mu^\nround_n(\bm{b}) &= \sum_{\nround=1}^\Nround \sum_{\nitem=1}^\Nitem (v_\nitem - b_\nitem)\textbf{1}_{b_\nitem \geq b_{-\nitem}^t} = \sum_{\nitem=1}^\Nitem \sum_{\nround=1}^\Nround (v_\nitem - b_\nitem)\textbf{1}_{b_\nitem \geq b_{-\nitem}^t}\\ &:= \sum_{\nitem=1}^\Nitem \sum_{\nround=1}^\Nround w_\nitem^\nround(b_\nitem) := \sum_{\nitem=1}^\Nitem W^{\Nround+1}_\nitem(b_\nitem)\,,
    \label{def: def mu w W}
\end{align}
where $w^\nround_\nitem(b) = \textbf{1}_{b \geq b^{\nround}_{-\nitem}} (v_\nitem - b)$ represents the utility in the $\nround$-th auction for winning the $\nitem$-th item with bid $b$, and $W^{\Nround+1}_\nitem(b) = \sum_{\nround=1}^\Nround w^{\nround}_\nitem(b)$ represents the cumulative utility gained from winning the $\nitem$-th item with bid $b$ across the $\Nround$ auctions. {(Here, in $w^\nround_\nitem(b)$, the same tie-breaking rule in Equation \eqref{eq: allocation and utility definition} is applied)}. To solve Problem \eqref{eq:offline}, we develop a polynomial-time DP scheme utilizing these $w^\nround_\nitem(b)$ and $W^{\Nround+1}_\nitem(b)$. In particular, for any $\nitem \in [\Nitem]$ and any bid $b \in \mathcal B$, let $U_\nitem(b)$ be the optimal cumulative utility of the agents from units $\nitem, \nitem+1, \ldots, \Nitem$ over $\Nround$ auctions assuming that bids for unit $\nitem$ is less than or equal to $b$.


We then have
\begin{align}
    U_\nitem(b) = \max_{b'\le b, b'\in \mathcal B}\left\{ W_{\nitem}^{\Nround+1} (b') + U_{\nitem+1}(b')\right\} \quad b\in \mathcal B, m\in [M] \quad \text{and} \quad U_{\Nitem+1}(b) = 0 \quad b \in \mathcal{B}\,.
\end{align}
Algorithm \ref{alg: Offline Full} uses the aforementioned  DP scheme to devise an optimal solution to Problem \eqref{eq:offline}. The following theorem, proven in Section~\ref{sec: offline proof}, shows the optimality of Algorithm \ref{alg: Offline Full}. 

\begin{theorem}\label{thm:offline}
Algorithm~\ref{alg: Offline Full} returns the optimal solution to problem \eqref{eq:offline} {with time and space complexity of $O(\Nitem |\mathcal{B}|^2)$ and $O(\Nitem|\mathcal{B}|)$ respectively.}
\end{theorem}






\begin{algorithm}[t]
\footnotesize
	\KwIn{Valuation $\bm{v}$ for $\bm{v} \in [0, 1]^{+\Nitem}$, Other bids $\{\bm{b}^{\nround}_-\}_{\nround \in [\Nround]}$ for $\bm{b}^{\nround}_- \in \mathcal{B}^{-\overline{\Nitem}}$.}
	\KwOut{Optimal bid vector $\bm{b}^* = \text{argmax}_{\bm{b} \in \mathcal{B}^\Nitem} \mu^{ \Nround}_n(\bm{b})$ and its corresponding utility.}
	Let $W^{\Nround+1}_\nitem(b) \gets \sum_{\nround=1}^\Nround \textbf{1}_{b \geq b^{\nround}_{-\nitem}} (v_\nitem - b)$,  $b \in \mathcal{B}, \nitem \in [\Nitem]$,   define 
        $U_{\Nitem+1}(b) \gets 0$, $b \in \mathcal{B}$, and set $b^*_{0} = \max(\mathcal B)$\;
        \textbf{for} $m \in [M,\ldots,1], b \in \mathcal{B}: U_\nitem(b) \gets \text{max}_{b' \in \mathcal{B}; b' \leq b}W^{\Nround+1}_\nitem(b') + U_{\nitem+1}(b')$\;
        \textbf{for} $m \in [1,\ldots,M]: b_m^\star \gets \arg\max_{b \le b_{m-1}^*} U_\nitem(b)$\;
        \textbf{Return} $U_1(\max(\mathcal{B}))$ and $\bm{b}^{*} =(b_{m}^*)_{m\in [M]}$.
	\caption{\textsc{Offline}$(\bm{v}, \{\bm{b}^{\nround}_-\}_{\nround \in [\Nround]})$ \label{alg: Offline Full}}	
\end{algorithm}


This algorithm to solve the offline bid optimization problem enables us to compute the hindsight optimal utility, which serves as a benchmark for evaluating the effectiveness of our online learning algorithms. It is worth mentioning that we can represent our DP algorithm as an equivalent graph with $\Nitem$ layers, with $|\mathcal{B}|$ nodes in each. More precisely, we define the (offline) DP graph as follows:
\begin{enumerate}
    \item \textbf{DP nodes/states.} There are $\Nitem$ layers, each with    $|\mathcal{B}|$ nodes in each, denoted by $\{(\nitem, b)\}_{\nitem \in [\Nitem], b \in \mathcal{B}}$. % We let $\mathcal{S}$ denote the set of all nodes/states.
    \item \textbf{DP edges.} In this graph, there are only (directed) edges between two consecutive layers, i.e., from layer $m$ to layer $m+1$ for any $m \in [M-1]$. 
    In particular, 
    node $(\nitem, b)$ only has an edge to node $(\nitem+1, b')$ for $b' \leq b$ and if $b' \leq v_{m+1}, b \leq v_m$. 
    \item \textbf{DP weights.} We define the weight of node/state $(\nitem, b)$ to be $W_{\nitem}^{T+1}(b) = \sum_{\tau=1}^{\Nround} \textbf{1}_{b \geq b^{\tau}_{-\nitem}} (v_\nitem - b)$.
\end{enumerate}

% Figure environment removed




For the online setting, we also note that we can define the DP graph at time $t$, as opposed to $T$, by setting $W_{\nitem}^t(b) = \sum_{\tau=1}^{\nround-1} \textbf{1}_{b \geq b^{\tau}_{-\nitem}} (v_\nitem - b)$. This allows us to construct algorithms for the full information and bandit settings by taking advantage of the structure of the DP graph to enhance efficiency and optimize storage of necessary computations. 








\section{Decoupled Exponential Weights Algorithms}

\label{sec: decoupled exp weights section}

In this section, we present our first algorithm for learning in the online setting. In particular, we construct a decoupled version of the Exponential Weights algorithm which circumvent the large space and time complexity of maintaining and updating the sampling distributions of all possible bid vectors. Our algorithms instead sequentially sample a singular bid value from each layer of our DP graph such that the probability of sampling a particular vector of bids $\bm{b}^\nround$ is precisely equal to the probability of the exponential weights algorithm selecting $\bm{b}^\nround$.\footnote{In Section~\ref{sec: time varying}, we present a generalization of our decoupled exponential weights algorithm to the setting with time varying valuations, where the valuations are drawn from some known, finite support distribution $F_{\bm{v}}$.} 
 

In the following sections, we provide a description of our algorithm in both the full information  and the bandit settings. It is important to note that our decoupled exponential weights algorithm achieves regret that is sub-optimal by a factor of $O(\sqrt{M})$.  Nonetheless, we present an alternative regret optimal algorithm based on OMD in a subsequent section. Despite this, our decoupled exponential weights algorithm remains practical as it does not necessitate solving a convex optimization problem at each time step $t \in [T]$.


\subsection{Full Information Setting}\label{sec:full}



Now let us focus on learning optimal bidding in an online fashion with full information feedback. One straightforward approach in this context is to apply the exponential weights algorithm \citep{DBLP:journals/iandc/LittlestoneW94} to the entire set of bid vectors. This algorithm guarantees per-round rewards within the range of $[-\Nitem, \Nitem]$. However, the challenge lies in the exponentially large bid space $\mathcal{B}^{+\Nitem}$. Tracking and updating weights for all possible bid vectors naively would lead to a non-polynomial time and space complexity. Although this approach achieves a small regret of $O(\Nitem^\frac{3}{2} \sqrt{T \log |\mathcal{B}|})$, we need a more efficient solution with a polynomial time and space complexity.

To do so, we leverage the DP scheme developed in Section \ref{sec:offline}. By utilizing the DP graph and the information it provides about bid vector utilities, we can effectively mimic the exponential weights algorithm without explicitly tracking weights for every bid vector. 
In Algorithm~\ref{alg: Decoupled Exponential Weights}, instead of 
 associating weights to each  possible bid vector, we  associate weights with each $(\nitem, b)$ pair for any $m\in [M]$ and $b\in \mathcal B$. These weights are then updated via variables $S_m^t(b)$ for $b\in \mathcal B, m\in[M]$, which are inspired by the DP scheme. For any round $t$, we define 
 \begin{align*} S^t_\nitem(b) &= \exp(\eta W_\nitem^\nround(b)) \sum_{\bm{b}'_{\nitem+1:\Nitem} \in \mathcal{B}^{+(\Nitem-\nitem)}, b'_{\nitem+1} \leq b'_\nitem = b} \exp(\eta \sum_{\nitem'=\nitem+1}^\Nitem W_{\nitem'}^\nround(b'_{\nitem'}))\\
 &= \exp(\eta W_\nitem^\nround(b)) \sum_{b' \in \mathcal{B}; b' \leq b} S_{\nitem+1}(b')\,.\end{align*}
 Here, $\eta> 0$ is the learning rate of the algorithm and we  recall that  $W_m^t(b)= \sum_{\tau=1}^{t-1} w^{\tau}_\nitem(b)$ is  cumulative utility gained across the first $t-1$ auctions from the winning the $\nitem$'th item with bid $b$, respectively. Computing $S_m^t(\cdot)$ is done in step $\textsc{Compute}-S_\nitem$ of the algorithm.  In step $\textsc{Sample}-\bm{b}$, the bid vector is then sampled  according to $S_m^t$'s subject to bid monotonicity.  To disallow overbidding, we initialize weights  $W^0_m(b) = -\infty$ for all $m \in [M], b \in \mathcal{B}$ such that $b > v_m$.
 
 

The concept of utility decoupling shares similarities with solutions used in combinatorial bandits, tabular reinforcement learning \citep{CMAB2013, OREPS2013}, and problems such as shortest path algorithms involving weight pushing or path kernels \citep{PathKernel2003, Hedging2010}. These methods are employed to solve variants of the shortest path or maximum weight path problems, where costs or weights are associated with edges rather than nodes. By exploiting the graph structure and the linearity of utilities with respect to the weights of each edge, these algorithms efficiently compute path weights based on edge weights, similar to how our algorithm computes path weights based on node weights. In addition to investigating a fundamentally different problem,   the key distinction is that our approach  considers weights associated with nodes instead of edges. In our setting, the reward associated with selecting bid $b'$ in slot $\nitem+1$ is independent of selecting bid $b \geq b'$ in slot $\nitem$. This allows us to get an improved regret bound and save a factor of $|\mathcal{B}|$ in terms of time and space complexity. Instead of storing and updating weights for $O(\Nitem |\mathcal{B}|^2)$ possible $(\nitem, b, b')$ slot-value-next value triplets, we only need to handle $O(\Nitem |\mathcal{B}|)$ possible $(\nitem, b)$ unit-bid pairs.

The following statement is the main result of this section. 

\begin{theorem}[Decoupled Exponential Weights: Full Information] \label{thm:full}
    With $\eta = \Theta(\sqrt{\frac{\log |\mathcal{B}|}{MT}})$, Algorithm \ref{alg: Decoupled Exponential Weights} achieves (discretized) regret $O(\Nitem^\frac{3}{2} \sqrt{ \Nround \log |\mathcal{B}|})$, with total time and space complexity polynomial in $\Nitem$, $|\mathcal{B}|$, and $\Nround$. Optimizing for discretization error from restricting the bid space to $\mathcal{B}$, we obtain a continuous regret of $O(\Nitem^\frac{3}{2} \sqrt{\Nround \log \Nround})$.
\end{theorem}
%\proof{}
%See  Section \ref{sec: path kernels regret}.
%\endproof

%It is worth noting that in both the discrete and continuous worst-case regret, the scaling with respect to $\Nitem$ is linear. This implies that the per-unit regret remains constant regardless of the number of items. 
 It is worth noting that both the time and space complexity exhibit polynomial scaling with $\Nitem$ and $|\mathcal{B}|$. Given that $\Nitem$ can be large in practical scenarios, such as carbon emissions license auctions or electricity markets, it becomes crucial to minimize the dependence on $\Nitem$.  




 \begin{algorithm}[t]
 \footnotesize
	\KwIn{Learning rate $0<\eta < \frac{1}{M}$, $\bm{v} \in [0, 1]^{+\Nitem}$.}
	\KwOut{The aggregate utility $\sum_{\nround=1}^\Nround \mu_n^\nround(\bm{b}^{\nround})$}
	$W_\nitem^0(b) \gets 0$ for all $\nitem \in [\Nitem], b \in \mathcal{B}$ such that $b \leq v_m$; else $W_\nitem^0(b) \gets -\infty$\;
        $b_{0}^t \gets \max \mathcal B$, and $S_{M+1}^t (\min \mathcal{B})=1$ for any $t\in[T]$\;
 
	\For{$\nround \in [1,\ldots,\Nround]$:}{
            %Adversary selects $\bm{b}^{\nround}_-$.
            \textbf{Recursively Computing Exponentially Weighted Partial Utilities $\bm{S}^t$}\;
            \textbf{for} $m \in [M,\ldots,1], b \in \mathcal{B}: S^t_\nitem(b) \gets \exp(\eta W_\nitem^\nround(b)) \sum_{b' \leq b} S_{\nitem + 1}^\nround(b')$. \label{eq: compute s} \hspace{0mm} $\backslash \backslash$ $\textsc{Compute}-S_\nitem$\;
        \textbf{Determining the Bid Vector $\bm{b}^\nround$ Recursively}\;
        \textbf{for} $m \in [1,\ldots,M], b \leq b_{m-1}^t: b_\nitem^\nround \gets b$ with probability $ \frac{S^t_\nitem(b)}{\sum_{b' \leq b_{\nitem-1}^t} S^t_{\nitem}(b')};$ \hspace{1mm} $\backslash \backslash$ $\textsc{Sample}-\bm{b}$\;
            %$\bm{b}^{\nround} \gets \textsc{Sampler}(\bm{W}^{\nround-1}, \eta)$\;
            Observe $\bm{b}^{\nround}_-$ and receive reward $\mu_n^\nround(\bm{b}^{\nround})$\;
        \textbf{Update Weight Estimates} \;
        \textbf{for} $\nitem \in [\Nitem], b \in \mathcal{B}: W^{\nround+1}_{\nitem}(b) \gets W^{\nround}_{\nitem}(b) +  (v_\nitem - b)\textbf{1}_{b \geq b^t_{-m}}$ if $b \leq v_m$; else $W^{\nround+1}_{\nitem}(b) \gets -\infty$\;
        }
        
        
        \textbf{Return} $\sum_{\nround=1}^\Nround \mu_n^{\nround}(\bm{b}^{\nround})$
	\caption{\textsc{Decoupled Exponential Weights - Full Information}}
	\label{alg: Decoupled Exponential Weights}
\end{algorithm}



\subsection{Bandit Feedback Setting}

\label{sec: decoupled exp weights, bandit feedback}


We extend Algorithm~\ref{alg: Decoupled Exponential Weights} for the bandit feedback setting. In the bandit feedback setting, the bidder's allocation and utility are not available for all possible bid vectors, unlike in the full information setting. Instead, the agent only observes their utility for the submitted bid vector. To handle this, we use inverse probability weighted (IPW) node weight estimates $\widehat{w}^t_m(b)$ instead of the node weights $w^t_m(b)$ in Algorithm~\ref{alg: Decoupled Exponential Weights}. This adaptation results in a regret of $O(M^{\frac{3}{2}}\sqrt{|\mathcal{B}| T \log |\mathcal{B}|})$, as shown in Theorem~\ref{thm:decoupled exp - bandit feedback}. This regret includes an additional factor of $\sqrt{|\mathcal{B}|}$ compared to the full information setting. 


The structure of Algorithm \ref{alg: Decoupled Exponential Weights - Path Kernels} is similar to that of Algorithm~\ref{alg: Decoupled Exponential Weights}. Both algorithms maintain node weight estimates, 
%, set $\widehat{W}^t_m(b) = -\infty$ for all $m \in [M], b \in \mathcal{B}$ such that $b > v_m$ to prevent overbidding,
compute the sum of exponentiated partial bid vector estimated utilities recursively, and sample bids for each unit recursively proportional to these summed exponentiated utilities. %The second and third steps are identical between the two algorithms. 
Specifically, Algorithm \ref{alg: Decoupled Exponential Weights - Path Kernels} samples bid vectors with probabilities proportional to the sum of the cumulative \textit{estimated} utility $\widehat{W}_{\nitem}^\nround(b) = \sum_{\tau=1}^{t-1} \widehat{w}_{\nitem}^\tau(b)$ over each unit-bid value pair, where   $\widehat{w}_{m}^\nround(b) = 1 - \frac{1 - (v_m - b)\textbf{1}_{b \geq b_{-m}^{\nround}}}{q_m^\nround(b)}\textbf{1}_{b_m^\nround = b}$ and $q_m^\nround(b)$ is the (unconditional) probability that bid $b$ is chosen for unit $m$ at time $t$.

Note that this bid vector utility estimator is a slightly different estimator {than the one} used in the standard $\textsc{Exp3}$ algorithm (See Chapter 11 of \cite{Lattimore2020}). In particular, the standard IPW estimator $\widehat{w}_m^\nround(b) = \frac{v_m-b}{q_m^\nround(b)} \textbf{1}_{b = b_m^t > b_{-m}^t}$, while unbiased, can be unboundedly large when $q_m^\nround(b)$ approaches 0, whereas our proposed estimator is bounded above by 1. As a consequence, we have that $\widehat{\mu}_n^\nround(\bm{b})$ is upper bounded by $M$, and therefore $ =\eta \widehat{\mu}^\nround(\bm{b}) = \eta \sum_{m=1}^M \widehat{w}_m^\nround(b)$ is upper bounded by 1 for $\eta < \frac{1}{M}$, which we crucially use in the proof. 

 \begin{algorithm}[t]
 \footnotesize
	\KwIn{Learning rate $0 < \eta < \frac{1}{M}$, $\bm{v} \in [0, 1]^{+\Nitem}$}
	\KwOut{The aggregate utility $\sum_{\nround=1}^\Nround \mu_n^\nround(\bm{b}^{\nround})$}
	$\widehat{W}_\nitem^0(b) \gets 0$ for all $\nitem \in [\Nitem], b \in \mathcal{B}$ such that $b \leq v_\nitem$; else $\widehat{W}_\nitem^0(b) \gets -\infty$.\;
        $b_{0}^t \gets \max \mathcal B$, and $\widehat{S}_{M+1}^t (\min \mathcal{B})=1$ for any $t\in[T]$\;
 
	\For{$\nround \in [1,\ldots,\Nround]$:}{
            \textbf{Recursively Computing Exponentially Weighted Partial Utilities $\bm{S}^t$}\;
            \textbf{for} $m \in [M,\ldots,1], b \in \mathcal{B}: \widehat{S}^t_\nitem(b) \gets \exp(\eta \widehat{W}_\nitem^\nround(b)) \sum_{b' \leq b} \widehat{S}_{\nitem + 1}^\nround(b')$ \hspace{0mm} $\backslash \backslash$ $\textsc{Compute}-\widehat{S}_\nitem$\;
        \textbf{Determining the Bid Vector $\bm{b}^\nround$ Recursively}\;
        \textbf{for} $m \in [1,\ldots,M], b \leq b_{m-1}^t: b_\nitem^\nround \gets b$ with probability $\frac{\widehat{S}^t_\nitem(b)}{\sum_{b' \leq b_{\nitem-1}^t} \widehat{S}^t_{\nitem}(b')}; $ \hspace{1mm} $\backslash \backslash$ $\textsc{Sample}-\bm{b}$\;
        Observe $\bm{b}^{\nround}_-$ and receive reward $\mu_n^\nround(\bm{b}^{\nround})$\;
        \textbf{Recursively Computing Probability Measure $\bm{q}$}\;
        $q^t_1(b) \gets \frac{\widehat{S}^\nround_m(b)}{\sum_{b' \in \mathcal{B}} \widehat{S}^\nround_m(b')}$ for all $b \in \mathcal{B}$\;
        \textbf{for} $m \in [2,\ldots,M], b \in \mathcal{B}: q_\nitem^\nround(b) \gets \sum_{b' \geq b} \frac{q_{\nitem-1}^t(b')\widehat{S}^\nround_{\nitem}(b)}{\sum_{b" \geq b'} \widehat{S}^\nround_\nitem(b")}$ for all $b \in \mathcal{B}$\;
        \textbf{Update Weight Estimates}\;
        \textbf{for} $m \in [M], b \in \mathcal{B}: \widehat{W}^{\nround+1}_{\nitem}(b) \gets \widehat{W}^{\nround}_{\nitem}(b) + (1 - \frac{1 - (v_m - b)\textbf{1}_{b \geq b^t_m}}{q^t_m(b)} \textbf{1}_{b^t_m = b})$ if $b \leq v_m$; else $\widehat{W}^{\nround+1}_{\nitem}(b) \gets -\infty$\;
        }
        \textbf{Return} $\sum_{\nround=1}^\Nround \mu_n^{\nround}(\bm{b}^{\nround})$
	\caption{\textsc{Decoupled Exponential Weights - Bandit Feedback}}
	\label{alg: Decoupled Exponential Weights - Path Kernels}
\end{algorithm}


   
    
    The primary difference in the implementation of Algorithm~\ref{alg: Decoupled Exponential Weights - Path Kernels} as compared to Algorithm~\ref{alg: Decoupled Exponential Weights} is that we require additional steps in order to obtain unbiased node weight estimates $\widehat{W}^{\nround+1}_{\nitem}(b) = \sum_{\tau=1}^{t} \widehat{w}_m^\tau(b)$ which we compute using an IPW estimator. In order to do this, we must compute $q_m^t(b)$---the  probabilities of selecting bid $b$ at slot $m$. 
    
   

\begin{theorem}[Decoupled Exponential Weights: Bandit Feedback] \label{thm:decoupled exp - bandit feedback}
    With $\eta = \Theta(\sqrt{\frac{\log |\mathcal{B}|}{M|\mathcal{B}|T}})$ such that $\eta < \frac{1}{M}$, Algorithm \ref{alg: Decoupled Exponential Weights - Path Kernels} achieves (discretized) regret $O(\Nitem^{\frac{3}{2}} \sqrt{ |\mathcal{{B}|}\Nround \log |\mathcal{B}|})$, with total time and space complexity polynomial in $\Nitem$, $|\mathcal{B}|$, and $\Nround$. Optimizing for discretization error from restricting the bid space to $\mathcal{B}$, we obtain a continuous regret of $O(M^{\frac{4}{3}}T^{\frac{2}{3}} \sqrt{\log \Nround})$.
\end{theorem}




%\section{Local surrogate models using proper generalized decomposition}
\label{sec:offline}

In this section, the procedure to construct the PGD local surrogate models in the offline phase is presented.
%
The local parametric problem to be solved in the generic subdomain $\Omega_i$ is: for all $\bmu \in \mathcal{P}$, find $u_i(\bmu)$ such that 
%
\begin{equation}
	\label{eq:localProb}
	\begin{array}{rcll}
		L_{}({u}_{i}(\bmu); \bmu) &=& s_i(\bmu) & \quad \text{in } \Omega_i\,,\\
		u_i(\bmu) &=& g^D_i(\bmu) & \quad \text{on } \Gamma_i^D,\\
		\Neum{u_i(\bmu)}{} &=& g^N_i(\bmu) & \quad \text{on } \Gamma_i^N,\\
		u_i(\bmu) &=& \lambda_i & \quad \text{on } \Gamma_i \, ,
	\end{array}
\end{equation}
%
for arbitrary Dirichlet data $\lambda_i$ at the interface $\Gamma_i$, where $\lambda_i=\lambda_i(\bx)$ is a space-dependent function. 

The arbitrariness of the boundary function $\lambda_i$ is dealt with by an appropriate parametrization.
%
Considering that problem \eqref{eq:localProb} is solved in a finite dimensional context, e.g., by the finite element method, the boundary function $\lambda_i$ can be expressed as a linear combination of suitable basis functions on $\Gamma_i$, say, $\eta^q_i(\bx), \ q=1,\ldots,N_{\Gamma_i}$, with coefficients $\Lambda^q_i$:
%
\begin{equation}\label{eq:lambdaRep}
 \lambda_i = \lambda_i(\bx) = \sum_{q = 1}^{N_{\Gamma_i}} \Lambda^q_i \, \eta^q_i(\bx) \, .
\end{equation}
%
For example, upon introducing a finite element space of continuous piecewise polynomial functions in $\Omega_i$, if $\varphi^q_i(\bx)$ are the finite element basis functions with non-null support at $\Gamma_i$, one can choose $\eta^q_i(\bx)$ to be the restriction of $\varphi^q_i(\bx)$ to $\Gamma_i$, i.e., $\eta^q_i(\bx) = \varphi^q_i(\bx)|_{\Gamma_i}$. Note that, while this is the approach used in the present work, other suitable bases can be considered on $\Gamma_i$. The dependence of the basis functions upon space is henceforth omitted, unless in the case of ambiguity.

The arbitrary coefficients $\bLambda_i = (\Lambda^1_i, \ldots, \Lambda^{N_{\Gamma_i}}_i) \in \mathcal{Q}_i$ thus become additional parameters of the local problem \eqref{eq:localProb}, with values in $ \mathcal{Q}_i = \mathcal{J}_i^1 \times \dots \times \mathcal{J}_i^{N_{\Gamma_i}}$,  where each $\mathcal{J}_i^q \subset \mathbb{R}$, $q = 1, \dots , N_{\Gamma_i}$ is a compact set.

\smallskip  

\begin{rem}
The sets of admissible boundary values introduced above need to be appropriately selected to ensure that the linear combination~\eqref{eq:lambdaRep} can approximate the trace $\lambda_i$ of the solution for all parameters $\bmu \in \mathcal{P}$. Therefore, the choice of the minimum and maximum values of $\mathcal{J}_i^q$ depends on the parameters $\bmu$.
\end{rem}

\smallskip

Although the introduction of the subdomains $\Omega_i$ in the DD procedure in Algorithm~\ref{alg:Schwarz} allows to work locally with a reduced number of spatial and parametric degrees of freedom, the parametrization of the boundary condition along the interface $\Gamma_i$ in the local problem~\eqref{eq:localProb} leads to a growth of the dimensionality of the local parametric problem, namely by introducing $N_{\Gamma_i}$ new dimensions, each associated with a coefficient $\Lambda^q_i, \ q = 1, \dots , N_{\Gamma_i}$. Unfortunately, it is well known that if $N_{\Gamma_i} \gg 1$ the solution of the local problem~\eqref{eq:localProb} with parametrized data~\eqref{eq:lambdaRep} might become unfeasible.

To overcome this difficulty, the linearity of the operator $L$ is exploited and the local problem~\eqref{eq:localProb} is split into a family of $N_i$ subproblems, each involving a \emph{sufficiently small} set of parameters $\mathcal{N}_i^j$, gathering the so-called \emph{active boundary parameters} (see Fig.~\ref{fig:activeBdryNodes}). More precisely,  $\{\mathcal{N}_i^j\}_{j=1,\ldots,N_i}$ denotes a disjoint partition of the set of indices $1,\ldots,N_{\Gamma_i}$ such that $\text{card}(\mathcal{N}_i^j) \ll N_{\Gamma_i}$, for all $j$.  Hence, the coefficients $\bLambda_i$ employed to characterize the trace functions can be split into subsets $\bLambda_i^j = (\Lambda_i^q)_{q \in \mathcal{N}_i^j} \in \mathcal{Q}_i^j$, with $\mathcal{Q}_i^j = \bigtimes_{q \in \mathcal{N}_i^j} \mathcal{J}_i^q \subset \mathcal{Q}_i$, and equation~\eqref{eq:lambdaRep} is rewritten as
%
\begin{equation}\label{eq:splittingBoundaryParameters}
\lambda_i = 
\sum_{q \in \mathcal{N}^1_i} \Lambda^q_i \, \eta^q_i + 
\sum_{q \in \mathcal{N}^2_i} \Lambda^q_i \, \eta^q_i + \ldots +
\sum_{q \in \mathcal{N}^{N_i}_i} \Lambda^q_i \, \eta^q_i .
\end{equation}

%
%
% Figure environment removed
%
%

The solution of the local problem~\eqref{eq:localProb} is thus expressed in terms of both the problem parameters $\bmu \in \mathcal{P}$ and the active boundary parameters $\bLambda_i^j \in \mathcal{Q}_i^j, \ j=1,\ldots,N_i$. By linearity, for all $\bmu \in \mathcal{P}$, the resulting solution is given by
%
\begin{equation}\label{eq:solutionSplit}
u_i(\bmu, \bLambda_i) = u_{i,0} (\bmu) + \sum_{j=1}^{N_i} u_{i,j}(\bmu, \bLambda_i^j) ,
\end{equation}
%
where $u_{i,0} (\bmu)$ satisfies the equation
%
\begin{subequations}\label{eq:subProb}
\begin{equation}\label{eq:sourceProb}
\begin{array}{rcll}
L({u}_{i, 0}(\bmu); \bmu) &=& s_i(\bmu) & \quad \text{in } \Omega_i,\\
u_{i, 0}(\bmu) &=& g^D_i(\bmu) & \quad \text{on } \Gamma_i^D,\\
\Neum{u_{i, 0}(\bmu)}{} &=& g^N_i(\bmu) &\quad \text{on } \Gamma_i^N,\\
u_{i, 0}(\bmu) &=& 0 & \quad \text{on } \Gamma_i ,
\end{array}
\end{equation}
%
whereas each $u_{i, j}(\bmu, \bLambda_i^j)$, with $j=1,\ldots,N_i$, is solution of
%
\begin{equation}\label{eq:boundaryProbs}
\begin{array}{rcll}
L(u_{i, j}(\bmu, \bLambda_i^j); \, \bmu) &=& 0 &\quad \text{in } \Omega_i,\\
u_{i, j}(\bmu, \bLambda_i^j) &=& 0 &\quad \text{on } \Gamma_i^D,\\
\Neum{u_{i, j}(\bmu, \bLambda_i^j)}{} &=& 0 & \quad \text{on } \Gamma_i^N,\\
u_{i, j}(\bmu, \bLambda_i^j) &=& \displaystyle\sum\limits_{q \in \mathcal{N}_i^j} \Lambda^q_i\,\eta^q_i & \quad \text{on } \Gamma_i ,
\end{array}
\end{equation}
%
for all $\bLambda_i^j \in \mathcal{Q}_i^j$.
\end{subequations}


%==========================================================================
\subsection{Separated representation of data and local solutions}
%==========================================================================

For the sake of readability and without any loss of generality, in this section the subindex $i$ is omitted in the description of problem data in each subdomain: for instance, the Dirichlet datum $g^D(\bmu)$ is employed to seamlessly describe $g^D_i(\bmu)$ on $\Gamma_i^D$ for any subdomain $\Omega_i$.

\smallskip

In order to construct a PGD approximation of problems~\eqref{eq:subProb}, data are assumed to be given in separated form, that is,
%
\begin{equation}\label{eq:separatedData}
\begin{array}{c}
 \nu = \displaystyle\sum_{\ell=1}^{n_{\nu}} \xi_{\nu}^\ell(\bmu)b_{\nu}^\ell(\bx) \,, \qquad
 \balpha = \displaystyle\sum_{\ell=1}^{n_{\alpha}} \xi_{\alpha}^\ell(\bmu) \bb_{\alpha}^\ell(\bx) \,, \qquad
 \gamma = \displaystyle\sum_{\ell=1}^{n_{\gamma}} \xi_{\gamma}^\ell(\bmu) b_{\gamma}^\ell(\bx) \,, \\
 s = \displaystyle\sum_{\ell=1}^{n_s} \xi_s^\ell(\bmu) b_s^\ell(\bx) \,, \qquad
  g^N = \displaystyle\sum_{\ell=1}^{n_N}  \xi_N^\ell(\bmu) b_N^\ell(\bx) ,
\end{array}
\end{equation}
%
where each term of the expressions~\eqref{eq:separatedData} is the product of a function depending on the spatial coordinate $\bx$ and a function of the parameters $\bmu$. Moreover, the parametric modes are assumed to be the product of one-dimensional functions of the parameters $\mu^1,\ldots,\mu^P$, e.g.,
%
\begin{equation}\label{eq:vectParam}
\xi_{\nu}^\ell(\bmu) = \prod_{p=1}^P \xi_{\nu,p}^\ell(\mu^p) .
\end{equation}
%
Although data are not directly given in the form~\eqref{eq:separatedData}, it is possible to numerically construct a good approximation in a separated form, see~\cite{DM-MZH:15}.
In addition, the PGD rationale assumes that the solutions $u_{i, 0}$ and $u_{i, j}$ of the local subproblems can be written in separated form. 

Consider the Hilbert space
\begin{equation*}
    \mathcal{V}_i = \{w_{i} \in {H}^1(\Omega_i) \, : \, w_{i} = 0 \text{ on }\partial \Omega_i \setminus \Gamma_i^N\}\,.
\end{equation*}
The solution of the local subproblem~\eqref{eq:sourceProb} depends only on space and on the parameters $\bmu$, and it can be written as
%
\begin{equation}\label{eq:solProbA}
 u_{i, 0}(\bmu) = v_{i, 0}(\bmu) + G^D(\bmu) ,
\end{equation}
%
where $G^D(\bmu) \in H^1(\Omega_i)$ is a suitable extension of the boundary datum $g^D(\bmu)$ at $\Gamma_i^D$ such that $G^D(\bmu) = g^D(\bmu)$ at $\Gamma_i^D$ and $G^D(\bmu) = 0$ at $\Gamma_i$. Similarly to~\eqref{eq:separatedData}, also the function $G^D(\bmu)$ can be written in separated form as
\begin{equation}\label{eq:separatedDataGD}
G^D = \displaystyle\sum_{\ell=1}^{n_D} \xi_D^\ell(\bmu) b_D^\ell(\bx) \, .
\end{equation}

By construction, it follows that $v_{i, 0}(\bmu) \in \mathcal{V}_i$, for all $\bmu \in \mathcal{P}$.  Similarly, the solution of each subproblem~\eqref{eq:boundaryProbs} for $j=1,\ldots,N_i$ depends both on the parameters $\bmu$ and on the active boundary parameters $\bLambda^j_i$ at the interface, and it can be expressed as
%
\begin{equation}\label{eq:solProbB}
u_{i, j}(\bmu, \bLambda_i^j) = v_{i, j}(\bmu, \bLambda_i^j) + \sum_{q \in \mathcal{N}_i^j} \Lambda^q_i\,\varphi^q_i ,
\end{equation}
%
with $v_{i, j}(\bmu, \bLambda_i^j) \in \mathcal{V}_i$,  for all $\bmu \in \mathcal{P}$ and for all $\bLambda_i^j \in \mathcal{Q}_i^j$. 

Following the standard procedure in PGD~\cite{Chinesta:2014}, the contributions of Dirichlet boundary conditions are handled by introducing \emph{ad-hoc}, sufficiently smooth modes.  The remaining terms $v_{i, 0}(\bmu)$ and $v_{i, j}(\bmu, \bLambda_i^j)$ are computed with homogeneous Dirichlet data, under the assumption of a separated representation of all the variables, that is, $\bx$ and $\bmu$ for $v_{i,0}$, and $\bx$, $\bmu$ and $\bLambda_i^j$ for $v_{i,j}$. This yields the PGD expansions
%
\begin{subequations}\label{eq:nonNormalisedPGD}
\begin{align}
    v_{i,0} \approx \vpgd_{i,0} &= \sum_{m=1}^{M_0} {V}_{i,0}^m(\bx) {\phi}_{i,0}^m(\bmu) \,  ,  \label{eq:nonNormalisedPGDA} \\
    v_{i,j} \approx \vpgd_{i,j}  &= \sum_{m=1}^{M_j} {V}_{i,j}^m(\bx) {\phi}_{i,j}^m(\bmu) {\psi}_{i,j}^m(\bLambda_i^j) \, , \label{eq:nonNormalisedPGDB}
\end{align}
\end{subequations}
%
where ${V}_{i,0}^m$ and ${V}_{i,j}^m$ are the $m$-th spatial modes,  whereas ${\phi}_{i,0}^m$, ${\phi}_{i,j}^m$ and ${\psi}_{i,j}^m$ denote the corresponding parametric modes. It is worth noticing that the numbers of modes $M_0$ and $M_j$ are \emph{a priori} unknown and are automatically determined by a greedy procedure, see~\cite{Diez:2020:ACME}.

In the following Sects.~\ref{sect:localPGD} and \ref{sec:algebraic}, the strategy to compute \eqref{eq:nonNormalisedPGD} is presented. The result is then employed to construct $\upgd_{i, 0}(\bmu)$ and $\upgd_{i, j}(\bmu,\bLambda_i^j)$ according to~\eqref{eq:solProbA} and \eqref{eq:solProbB}. The former is the surrogate model of the data-dependent parametric problem~\eqref{eq:sourceProb}, whereas the latter are employed to define the surrogate model $\upgd_{i,\Lambda}(\bmu, \bLambda_i)$ associated with the boundary parameters, namely,
%
\begin{equation}\label{eq:surrogateBdry}
\upgd_{i,\Lambda}(\bmu, \bLambda_i) = \sum_{j=1}^{N_i} \upgd_{i,j}(\bmu, \bLambda_i^j) \, .
\end{equation}
%
Finally, the complete surrogate model for subdomain $\Omega_i$ is obtained from~\eqref{eq:solutionSplit} as
%
\begin{equation}\label{eq:pgdFINALsol}
\upgd_i(\bmu, \bLambda_i) = \upgd_{i,0} (\bmu) + \upgd_{i,\Lambda}(\bmu, \bLambda_i) \, .
\end{equation}

\begin{rem}
The surrogate models $\upgd_{i,j}(\bmu, \bLambda_i^j)$ feature different supports $\mathcal{Q}_i^j$ in the space of parameters $\bLambda_i^j$. 
%
Hence, for the summation on the right-hand side of equation~\eqref{eq:surrogateBdry} to be well-defined, each surrogate model associated with the active boundary parameters needs to be appropriately extended to have support on the entire parametric space $\mathcal{Q}_i$.
%
This can be straightforwardly achieved in the framework of PGD approximations by defining the modal functions for the \emph{inactive} boundary parameters $\Lambda_i^q$, $q \not\in \mathcal{N}_i^j$ to be constant and equal to $1$.
\end{rem}


%==========================================================================
\subsection{Parametric weak form of the local subproblems}
\label{sect:localPGD}
%==========================================================================
A continuous Galerkin finite element strategy is employed to construct the PGD approximations of the solutions of the local subproblems. To this end, the weak forms of the parametric problems~\eqref{eq:subProb} are first presented.

For all $v,\deV \in \mathcal{V}_i$ and for all $\bmu \in \mathcal{P}$, let $\mathcal{A}$ be the bilinear form
\begin{equation}\label{eq:bilinearA}
\mathcal{A}(v, \deV; \bmu) 
= \int_{\Omega_i} \nu(\bmu)\nabla v\cdot \nabla \deV\,d\bx 
+ \int_{\Omega_i} \balpha(\bmu) {\cdot} \nabla v \, \deV \, d\bx
+ \int_{\Omega_i} \gamma(\bmu) \, v\, \deV \,d\bx \, .
\end{equation}

The parametric weak form of problem~\eqref{eq:sourceProb} becomes: find $v_{i, 0}(\bmu) \in \mathcal{V}_i$ such that
%
\begin{subequations}\label{eq:pbA}
\begin{equation}\label{eq:varfA}
 \mathcal{A}(v_{i,0}(\bmu), \deV; \bmu) = \mathcal{F}_0(\deV; \bmu)\qquad \forall \deV \in \mathcal{V}_i\, \text{ and } \forall \bmu \in \mathcal{P} \, ,
\end{equation}
%
where
%
\begin{equation}\label{eq:linearA}
\mathcal{F}_0(\deV; \bmu)
= \int_{\Omega_i} s(\bmu) \deV\,d\bx
+ \int_{\Gamma_i^N} g^N(\bmu) \deV\, d\bx
- \mathcal{A}(G^D(\bmu), \deV; \bmu) 
\, .
\end{equation}
\end{subequations}

In a similar fashion,  for each problem~\eqref{eq:boundaryProbs} for $j=1,\ldots,N_i$, the parametric weak formulation is: find $v_{i, j}(\bmu, \bLambda_i^j) \in \mathcal{V}_i$ such that
%
\begin{subequations}\label{eq:pbB}
\begin{equation}\label{eq:varfB}
	\mathcal{A}(v_{i,j}(\bmu, \bLambda_i^j), \deV; \bmu) = \mathcal{F}_j(\deV; \bmu; \bLambda_i^j )\quad \forall \deV \in \mathcal{V}_i \, , \forall \bmu \in \mathcal{P} \, \text{ and } \forall \bLambda_i^j \in \mathcal{Q}_i^j \, ,
\end{equation}
%
with
%
\begin{equation}\label{eq:linearB}
\mathcal{F}_j(\deV; \bmu; \bLambda_i^j ) = 
- \mathcal{A} \left(\sum_{q \in \mathcal{N}_i^j} \! \Lambda^q_i\,\varphi^q_i, \deV; \bmu \right) \, .
\end{equation}
%
\end{subequations}


%==========================================================================
\subsection{Parametric linear systems}
\label{sec:algebraic}
%==========================================================================
Under the assumption of an affine parameter dependence of the bilinear form $\mathcal{A}$ and of the linear forms $\mathcal{F}_0$ and $\mathcal{F}_j$ (see, e.g.,~\cite{Rozza:14}), the separated approximations~\eqref{eq:nonNormalisedPGD} are constructed using a greedy approach~\cite{Chinesta:2014}. In particular,  the non-intrusive implementation provided by the encapsulated PGD solver~\cite{Diez:2020:ACME} is employed.

This approach relies on rewriting the local problems~\eqref{eq:pbA} and~\eqref{eq:pbB} in algebraic form, as parametric linear systems. To this end,  the separated representation of data~\eqref{eq:separatedData} is substituted in the parametric weak forms~\eqref{eq:varfA} and~\eqref{eq:varfB} and the unknown solutions $v_{i,0}(\bmu)$ and $v_{i,j}(\bmu, \bLambda_i^j)$ are replaced by the their corresponding PGD approximations $\vpgd_{i,0}$ and $\vpgd_{i,j}$, see~\eqref{eq:nonNormalisedPGD}.

For all $v,\deV \in \mathcal{V}_i$ and for all $\bmu \in \mathcal{P}$, let $\Apgd$ be the bilinear form
\begin{equation}\label{eq:bilinearApgd}
\begin{array}{rcl}
\Apgd(v, \deV; \bmu) 
&=& \displaystyle \sum_{\ell=1}^{n_{\nu}} \xi_{\nu}^\ell(\bmu) \int_{\Omega_i}  b_{\nu}^\ell(\bx) \nabla v\cdot \nabla \deV\,d\bx \\[3pt]
&& \displaystyle + \sum_{\ell=1}^{n_{\alpha}} \xi_{\alpha}^\ell(\bmu) \int_{\Omega_i} \bb_{\alpha}^\ell(\bx) {\cdot} \nabla v \,  \deV \, d\bx\\[3pt]
&& \displaystyle + \sum_{\ell=1}^{n_{\gamma}} \xi_{\gamma}^\ell(\bmu) \int_{\Omega_i} b_{\gamma}^\ell(\bx) v \,  \deV \,d\bx \, .
\end{array}
\end{equation}

The PGD solution $\vpgd_{i,0}$ of problem~\eqref{eq:pbA} is computed by solving the parametric equation
%
\begin{subequations}\label{eq:pbApgd}
\begin{equation}\label{eq:varfApgd}
 \Apgd(\vpgd_{i,0}, \deV; \bmu) = \Fpgd_0(\deV; \bmu)\qquad \forall \deV \in \mathcal{V}_i\, \text{ and } \forall \bmu \in \mathcal{P} \, ,
\end{equation}
%
with
%
\begin{equation}\label{eq:linearApgd}
\begin{array}{rcl}
\Fpgd_0(\deV; \bmu)
&=& \displaystyle \sum_{\ell=1}^{n_s} \xi_s^\ell(\bmu) \int_{\Omega_i} b_s^\ell(\bx) \deV\,d\bx \\[3pt]
&& \displaystyle + \sum_{\ell=1}^{n_N} \xi_N^\ell(\bmu) \int_{\Gamma_i^N} b_N^\ell(\bx) \deV\, d\bx \\[3pt]
&& \displaystyle - \Apgd \left(\,\sum_{\ell=1}^{n_D} \xi_D^\ell(\bmu) b_D^\ell(\bx), \deV; \bmu \right) \, .
\end{array}
\end{equation}
\end{subequations}

The PGD approximation~\eqref{eq:nonNormalisedPGDA} is constructed using a continuous Galerkin finite element discretization for each spatial mode $V_{i,0}^m(\bx)$ and a pointwise collocation approach for the parametric modes $\phi_{i,0}^m(\bmu)$. More precisely,  a finite element mesh is introduced in each subdomain $\Omega_i$ and a spatial polynomial approximation $\mathbb{Q}_r$ of degree $r \geq 1$ is selected, with basis functions $\varphi_i^n$, $n=1,\ldots,\Nfem_i$. It follows that each spatial mode can be written as
%
\begin{equation}\label{eq:spatialMode}
V_{i,0}^m(\bx) = \sum_{n=1}^{\Nfem_i} V_{i,0}^{m,n} \varphi_i^n(\bx) \, ,
\end{equation}
%
where the coefficients $V_{i,0}^{m,n}$, $n=1,\ldots,\Nfem_i$ determine the vector of spatial finite element unknowns $\mathbf{V}_{i,0}^m$. 
%
Therefore, the integrals appearing in the bilinear and linear forms~\eqref{eq:bilinearApgd} and~\eqref{eq:linearApgd} give rise to standard finite element matrices and vectors, appropriately weighted by means of parametric functions stemming from the separated form of data~\eqref{eq:separatedData} and~\eqref{eq:separatedDataGD}.

The resulting parametric linear system for problem~\eqref{eq:sourceProb} is
%
\begin{equation}\label{eq:algebraicA}
\begin{aligned}
\left(
\sum_{\ell=1}^{n_{\nu}} \xi_{\nu}^\ell(\bmu) \mat{K}_{\nu}^\ell
+ \sum_{\ell=1}^{n_{\alpha}} \xi_{\alpha}^\ell(\bmu) \mat{K}_{\alpha}^\ell
+ \sum_{\ell=1}^{n_{\gamma}} \xi_{\gamma}^\ell(\bmu) \mat{K}_{\gamma}^\ell
\right) &
\bvpgd_{i,0}(\bmu) \\
=
\sum_{\ell=1}^{n_s} \xi_s^\ell(\bmu) \mathbf{f}_s^\ell
+ \sum_{\ell=1}^{n_N} & \xi_N^\ell(\bmu) \mathbf{f}_N^\ell
+ \sum_{\ell=1}^{n_D} \xi_D^\ell(\bmu) \mathbf{f}_D^\ell
\qquad \forall \bmu \in \mathcal{P} \, ,
\end{aligned}
\end{equation}
%
where the PGD separated solution is defined as 
%
\begin{equation}\label{eq:algebraicSolA}
\bvpgd_{i,0}(\bmu) = \sum_{m=1}^{M_0} \mathbf{V}_{i,0}^m \, \phi_{i,0}^m(\bmu) \, ,
\end{equation}
%
whereas $ \mat{K}_{\nu}^\ell$, $ \mat{K}_{\alpha}^\ell$ and $ \mat{K}_{\gamma}^\ell$ are weighted finite element matrices stemming from the diffusion, convection and reaction term, respectively, and $\mathbf{f}_s^\ell$, $\mathbf{f}_N^\ell$ and $\mathbf{f}_D^\ell$ denote the finite element vectors accounting for the source,  Neumann and Dirichlet data, respectively.

\smallskip

The parametric linear system associated with problem~\eqref{eq:pbB} is derived with an analogous procedure.  More precisely, let $\vpgd_{i,j}$ be the solution of the parametric equation
%
\begin{subequations}\label{eq:pbBpgd}
\begin{equation}\label{eq:varfBpgd}
	\Apgd(\vpgd_{i,j}, \deV; \bmu) = \Fpgd_j(\deV; \bmu; \bLambda_i^j )\quad \forall \deV \in \mathcal{V}_i\, ,  \forall \bmu \in \mathcal{P} \, \text{ and } \forall \bLambda_i^j \in \mathcal{Q}_i^j \, ,
\end{equation}
%
with
%
\begin{equation}\label{eq:linearBpgd}
\Fpgd_j(\deV; \bmu;  \bLambda_i^j ) = 
- \Apgd \left( \, \sum_{q \in \mathcal{N}_i^j} \! \Lambda^q_i\,\varphi^q_i, \deV; \bmu \right) \, .
\end{equation}
\end{subequations}

The continuous Galerkin finite element discretization introduced in~\eqref{eq:spatialMode} is employed also for the spatial modes $V_{i,j}^m(\bx)$, leading to the vector of spatial unknowns $\mathbf{V}_{i,j}^m$, whereas pointwise collocation is used for the parametric modes $\phi_{i,j}^m(\bmu)$ and $\psi_{i,j}^m(\bLambda_i^j )$. Hence, the PGD approximation~\eqref{eq:nonNormalisedPGDB} is determined by computing 
%
\begin{equation}\label{eq:algebraicSolB}
\bvpgd_{i,j}(\bmu, \bLambda_i^j) = \sum_{m=1}^{M_j} \mathbf{V}_{i,j}^m \, \phi_{i,j}^m(\bmu) \, \psi_{i,j}^m(\bLambda_i^j )
\end{equation}
%
as the solution of the parametric linear system
%
\begin{equation}\label{eq:algebraicB}
\left(
\sum_{\ell=1}^{n_{\nu}} \xi_{\nu}^\ell(\bmu) \mat{K}_{\nu}^\ell
+ \sum_{\ell=1}^{n_{\alpha}} \xi_{\alpha}^\ell(\bmu) \mat{K}_{\alpha}^\ell
+ \sum_{\ell=1}^{n_{\gamma}} \xi_{\gamma}^\ell(\bmu) \mat{K}_{\gamma}^\ell
\right)
\bvpgd_{i,j}(\bmu,\bLambda_i^j) 
=
\sum_{q \in \mathcal{N}_i^j} \Lambda^q_i \mathbf{f}_{\Lambda}^q \, ,
\end{equation}
%
for any value of $\bmu \in \mathcal{P}$ and $\bLambda_i^j \in \mathcal{Q}_i^j$.  In equation~\eqref{eq:algebraicB}, the vector $\mathbf{f}_{\Lambda}^q$ stems from imposing the parametric Dirichlet boundary condition at the interface $\Gamma_i$ in equation~\eqref{eq:boundaryProbs} within the finite element setting.

\smallskip

The encapsulated PGD library~\cite{Diez:2020:ACME} is utilized to solve equations~\eqref{eq:algebraicA} and \eqref{eq:algebraicB}. Technical details on the setup of problems~\eqref{eq:algebraicA} and~\eqref{eq:algebraicB} in the encapsulated PGD framework for a sample test case are presented in Appendix~\ref{append:encapsulatedPGD}.

\smallskip

\begin{rem}
For the case of two subdomains, the cost of the offline phase stems from the computation of two surrogate models accounting for the data-dependent parametric problems~\eqref{eq:sourceProb} and $N_1+N_2$ surrogate models related to problems~\eqref{eq:boundaryProbs} with active boundary parameters.
%
It is worth noticing that all PGD approximations mentioned above are independent from one another and can be efficiently computed in parallel. 
%
Moreover, the computational effort during the offline phase can be further reduced, e.g.,  by identifying a reference subdomain where local surrogate models are computed before being suitably mapped to the physical subdomains of the problems under consideration, as demonstrated in the example in Sect.~\ref{sec:testPatera}.
\end{rem}
% \section{Offline Setting}

Agent $n$ can think must construct a utility maximizing bid vector as a function of the complete auction history, which includes all competing bids $(\bm{b}^{\nround}_-)_{\nround \in [\Nround]}$ and reserves $(\pi^\nround)_{\nround \in [\Nround]}$. We additionally assume fixed valuation profiles; i.e. $\bm{v}^\nround = \bm{v}$ for all $\nround \in [\Nround]$. Recalling that the aggregate utility by bidding in all rounds with bid $\bm{b} \in \mathcal{B}^{+\Nitem}$ is given by $\sum_{\nround=1}^\Nround \mu^\nround_n(\bm{b})$ \rigel{Expand $\mu$ as a function of the $w$'s and $W$'s here so that it feels more natural as a solution}, we construct a dynamic programming solution that recovers the optimal bid vector as a function of the previous rounds' outcomes. We define for each $\nround \in [\Nround]$:
\[
w^{\nround}_\nitem(b) = \textbf{1}_{b > \max(b^{\nround}_{-\nitem}, \pi^\nround)} (v_\nitem - b) \quad \text{and} \quad W^{\Nround}_\nitem(b) = \sum_{\nround=1}^\Nround w^{\nround}_\nitem(b)
\]
Here, $w^\nround_\nitem(b)$ is the marginal utility of bidding $b$ in slot $\nitem$ at round $\nround$ and $W^\Nround_\nitem(b)$ the aggregate utility gained across auctions $\nround \in [\Nround]$ from the winning the $\nitem$'th item with bid $b$ respectively. We can construct our dynamic programming table(s):
\[
V^{\Nround}_\nitem(b) = W^{\Nround}_\nitem(b^*_\nitem(b)) + V^{\Nround}_{\nitem+1}(b^*_\nitem(b)) \quad \text{and} \quad U^{\Nround}_\nitem(b) = [b^{*}_\nitem(b), U^{\Nround}_{\nitem+1}(b^{*}_\nitem(b))]
\] 
Here, \[b^{*}_\nitem(b) = \text{argmax}_{b' \in \mathcal{B}; b' \leq b} W^{\Nround}_\nitem(b') + V^{\Nround}_{\nitem+1}(b').\] $V_\nitem^\Nround(b)$ denotes the optimal time-aggregate utility gained across slots $\nitem$ to $\Nitem$ restricting to bids of at most $b$, with base case $V^{\Nround}_\Nitem(B) = W^{\Nround}_\Nitem(b)$. Similarly, $U_\nitem^\Nround(b)$ is the corresponding optimal partial bid vector. The optimal utility is given by $V^{\Nround}_1(\max(\mathcal{B}))$. By using the recursive form for $V^{\Nround}_{\nitem}(b)$ and $U^{\Nround}_{\nitem}(b)$, computing tables $\bm{V}^{\Nround}$ and $\bm{U}^{\Nround}$ has time complexity of $O(\Nitem |\mathcal{B}|^2)$. As there are $\Nitem|\mathcal{B}|$ table entries, and each entry of $U^{\Nround}_\nitem$ is of length $O(\Nitem)$, the space complexity is $O(\Nitem^2|\mathcal{B}|)$. % If we are computing these tables for each bidder, then the time and space complexities increase by a factor of $N$. 

\begin{algorithm}[t]
	\KwIn{Competing bids, $(\bm{b}^{\nround}_-)_{\nround \in [\Nround]}$, Reserves $(\pi^\nround)_{\nround \in [\Nround]}$, and Valuations $\{\bm{v}^\nround\}_{\nround \in [\Nround]} = \bm{v}$}
	\KwOut{Optimal bid vector $\text{argmax}_{\bm{b} \in \mathcal{B}^{+\Nitem}} \sum_{\nround=1}^\Nround \mu^{\nround}_n(\bm{b})$ and its corresponding utility.}
	$W^{\Nround}_\nitem(b) \gets \sum_{\nround=1}^\Nround \textbf{1}_{b \geq \max(b^{ \nround}_{-\nitem}, \pi^\nround)} (v_\nitem - b) \quad \forall b \in \mathcal{B}, \nitem \in [\Nitem]$\;
        $V^{\Nround}_{\Nitem+1}(b) \gets 0$ and $U^{\Nround}_{\Nitem+1}(b) \gets [ \hspace{2mm} ] \quad  \forall b \in \mathcal{B}$\;
	\For{$\nitem \in \{\Nitem,\ldots,1\}$:}{
    	\For{$b \in \mathcal{B}$:}{
                $b^* \gets \text{argmax}_{b' \in \mathcal{B}; b' \leq b}W^{\Nround}_\nitem(b') + V^{\Nround}_{\nitem+1}(b') \quad \forall b \in \mathcal{B}$\;
                $V^{\Nround}_\nitem(b) \gets W^{\Nround}_\nitem(b^*) + V^{\Nround}_{\nitem+1}(b^*)$\;
                $U^{\Nround}_\nitem(b) \gets [b^*, U^{\Nround}_{\nitem+1}(b^*)]$\;
    	}
        }
        \textbf{Return} $U^{\Nround}_1(\max(\mathcal{B}))$, $V^{\Nround}_1(\max(\mathcal{B}))$
	\caption{\textsc{OfflineFull}$(\bm{v}, \{\bm{b}^{\nround}_-\}_{\nround \in [\Nround]}, \{\pi^\nround\}_{\nround \in [\Nround]})$}
	\label{alg: Offline Full}
\end{algorithm}

\begin{theorem}
    In the full information feedback setting, Algorithm~\ref{alg: Offline Full} returns the hindsight optimal bid vector with respect to competing bids, $(\bm{b}^{\nround}_-)_{\nround \in [\Nround]}$, reserves $(\pi^\nround)_{\nround \in [\Nround]}$, and valuations $\{\bm{v}^\nround\}_{\nround \in [\Nround]}$.
\end{theorem}

\begin{proof}
    We proceed by noting that $w^{\nround}_{\nitem}(b)$, the utility obtained in auction $\nround$ from winning item $\nitem$, is independent of the utility gained from any other items, conditional on the weakly monotonic bids condition. As the aggregate utilities between auctions are also independent, the time-aggregate optimal utilities $V^{\nround}_{\nitem}(b)$ can be constructed as a recursively as a function of the sum of the per-slot time aggregate utilities. To show that we can obtain the optimal utilities, we have:
    \begin{align}
        V^{\Nround}_\nitem(b) &= \max_{b_\nitem\geq\ldots\geq b_\Nitem; b_j \leq b \forall j \in [\nitem,\ldots,\Nitem]} \sum_{\nitem' = \nitem}^\Nitem W^{\Nround}_{\nitem'}(b_{\nitem'})\\
        &= \max_{b_{\nitem}\geq\ldots\geq b_\Nitem; b_j \leq b \forall j \in [\nitem,\ldots,\Nitem]} W^{\Nround}_{\nitem}(b_\nitem ) +  \sum_{\nitem' = \nitem+1}^\Nitem W^{\Nround}_{\nitem'}(b_{\nitem'})\\
        &= \max_{b' \in \mathcal{B}; b' \leq b} W^{\Nround}_\nitem(b') + V^{\Nround}_{\nitem+1}(b')
    \end{align}
    Where the last equality follows from the conditional independence between utilities obtained per slot. Since we have that $V^{\nround}_{\Nitem}(b ) = W^{\Nround}_{\Nitem} (B )$ from the base case, the optimality of $V^{\Nround}_\nitem(b)$ follows from induction.
\end{proof}

We note that the above algorithm also translates to the non stationary valuation setting by requiring input $\{\bm{v}^\nround\}_{\nround \in [\Nround]}$ and changing the first line to include $v^\nround_\nitem$ instead of $v_\nitem$. With this algorithm, we have established a method to obtain the hindsight optimal utility for which to gauge the efficacy of our online algorithms empirically.

\section{Online Setting}

Now we consider the problem of optimally learning how to bid in an online fashion. One obvious solution is applying exponential weights over the entire set of bid vectors, which has per-round rewards bounded in $[-\Nitem, \Nitem]$ (as $\sum_{\nitem=1}^\Nitem (v_\nitem - \max_{B \in \mathcal{B}} B) \geq -\Nitem$ and $\sum_{\nitem=1}^\Nitem (v_\nitem - \min_{B \in \mathcal{B}} B) \leq \Nitem$). While this achieves small regret $O(\Nitem \sqrt{\Nround \log |\mathcal{B}|})$ in the full information setting, the primary challenge here is that $\mathcal{B}^{+\Nitem}$ is exponentially large and naively tracking and updating these weights is expensive. We show how to sequentially select $(\nitem+1, B')$ given $(\nitem, B)$ for $B' \geq B$ to recover the entire bid vector $\bm{b}^{\nround}$.  % One way to circumvent this issue is by modeling the space of bids $\mathcal{B}^\Nitem$ as a layered graph and then applying ideas from Stochastic Shortest Path (SSP) problem. We will afterwards show that we can save a factor of $|\mathcal{B}|$ time and memory by representing the policy over edges in a lower dimensional space. 
An additional challenge is that we must be able to generalize across valuation profiles. Fortunately, as the impact of the valuations and bids on the utility is both additive and separable, we may utility full cross learning across valuations with no additional computational or memory overhead.

\subsection{Full Information Setting}

In exponential weights, the learner selects at round $\nround+1$ some bid vector $\bm{b}$ proportional to its $\eta$-exponentially weighted historical utility $\mu^{\nround}_n(\bm{b})$. Using our representation of $\mu^{\nround}_n(\bm{b})$ as a function of $W^\nround_\nitem(B)$ from the previous section, we have:
\begin{align}
    \prob_{EW(\eta)}^{\nround}(\bm{b}) = \frac{\exp(\eta \mu^{\nround}_n(\bm{b}))}{\sum_{\bm{b}' \in \mathcal{B}^{+\Nitem} \exp(\eta \mu^{\nround}_n(\bm{b}'))}} = \frac{\exp(\eta \sum_{\nitem=1}^\Nitem W_\nitem^\nround(b_\nitem))}{\sum_{\bm{b}' \in \mathcal{B}^{+\Nitem} \exp(\eta \sum_{\nitem=1}^\Nitem W_\nitem^\nround(b'_\nitem))}}
\end{align}
Since the denominator sums over an exponentially large set $\mathcal{B}^{+\Nitem}$, computing and storing these weights is difficult. Instead, we will recursively sample the values of $b_1,\ldots,b_\Nitem$. Let $S^\nround_1(B) = \sum_{\{\bm{b}: \bm{b} \in \mathcal{B}^\Nitem, b_1 = B\}} \exp(\eta \sum_{\nitem=1}^\Nitem W_\nitem^\nround(b_\nitem))$ denote the sum of exponentially weighted utilities over all bid vectors $\bm{b}$ such that $b_1 = B$. Then, we have that:
\begin{align}
    \sum_{\{\bm{b}: \bm{b} \in \mathcal{B}^\Nitem, b_1 = B\}} \prob_{EW(\eta)}^{\nround}(\bm{b}) = \frac{\sum_{\{\bm{b}: \bm{b} \in \mathcal{B}^\Nitem, b_1 = B\}} \exp(\eta \sum_{\nitem=1}^\Nitem W_\nitem^\nround(b_\nitem))}{\sum_{\{\bm{b}: \bm{b} \in \mathcal{B}^\Nitem\}} \exp(\eta \sum_{\nitem=1}^\Nitem W_\nitem^\nround(b_\nitem))} = \frac{S^\nround_1(B)}{\sum_{B' \in \mathcal{B}} S^\nround_1(B')}
\end{align}
More generally, let $S^\nround_\nitem(B) = \sum_{b_\nitem \geq \ldots \geq b_\Nitem, b_\nitem = B} \exp(\eta \sum_{\nitem=\nitem}^\Nitem W_\nitem^\nround(b_\nitem))$ denote the sum of exponentially weighted utilities corresponding to slots $\nitem$ through $\Nitem$ over all partial bid vectors satisfying $b_\nitem = B$. Then we can write $S^\nround_\nitem(B)$ recursively as:
\begin{align*}
    S^\nround_\nitem(B) = \sum_{b_\nitem \geq \ldots \geq b_\Nitem, b_\nitem = B} \exp(\eta \sum_{h=\nitem}^\Nitem W_h^\nround(b_h)) = \exp(\eta W_\nitem^\nround(B)) \sum_{b_{\nitem + 1} \geq \ldots \geq b_\Nitem, b_{\nitem+1} \leq B} \exp(\eta \sum_{\nitem=\nitem}^\Nitem W_\nitem^\nround(b_\nitem))
\end{align*}
Now summing over all possible values of $B'$, we obtain the desired recursion:
\begin{align*}
    S^\nround_\nitem(B) = \exp(\eta W_\nitem^\nround(B)) \sum_{B' \leq B} \sum_{b_{\nitem + 1} \geq \ldots \geq b_\Nitem, b_{\nitem+1} = B'} \exp(\eta \sum_{\nitem=\nitem}^\Nitem W_\nitem^\nround(b_\nitem)) = \exp(\eta W_\nitem^\nround(B)) \sum_{B' \leq B} S^\nround_{\nitem+1}(B')\\
\end{align*}
With the base case $S^\nround_\Nitem(B) = \exp(\eta W_\Nitem^\nround(B))$, we can recover all of the exponentially weighted partial utilities $S^\nround_{\nitem}(B)$ given $\bm{W}^\nround$. Once we have computed $S^\nround_{\nitem}(B)$, we can sample $\bm{b}$ according to its exponentially weighted utility $\exp(\eta \mu{n, \nround}(\bm{b})$ by sequentially sampling each $b_1,\ldots,b_\Nitem$.

\begin{algorithm}[t]
	\KwIn{Learning rate $\eta > 0$, Aggregate per-slot utilities $\bm{W}^\nround \equiv \{W_\nitem^\nround(B)\}_{\nitem \in [\Nitem], B \in \mathcal{B}}$.}
	\KwOut{Bid vector $\bm{b}$ sampled with probability $\exp(\eta \mu^{n, \nround}(\bm{b})$}
	$S^\nround_\Nitem(B) \gets \exp{(\eta W^\nround_{\Nitem}(B))}$ for all $B \in \mathcal{B}$ and $b_0 \gets \max_{B \in \mathcal{B}} B$\;
        \textbf{For} $\nitem \in [\Nitem-1,\ldots,1], B \in \mathcal{B}:$ \textbf{do} $S^\nround_\nitem(B) \gets \exp(\eta W_\nitem^\nround(B)) \sum_{B' \leq B} S^\nround_{\nitem + 1}(B')$\;
        \textbf{For} $\nitem \in [\Nitem]:$ \textbf{do} $b_\nitem \sim [S^\nround_\nitem(B)]_{B \in \mathcal{B}; B \leq b_{\nitem-1}}$\;
        \textbf{Return} $\bm{b} = (b_1,\ldots,b_\Nitem)$
	\caption{\textsc{DecoupledSampler}$(\bm{W}^\nround, \eta)$}
	\label{alg: Decoupled Sampler}
\end{algorithm}

\begin{theorem}
    Algorithm \ref{alg: Decoupled Sampler} samples $\bm{b}$ with probabilities equal to the exponential weights distribution $\exp(\eta \mu^{n, \nround}(\bm{b})$.
\end{theorem}

\begin{proof}
    Let $b_0 = B_0 = \max_{B \in \mathcal{B}} B$. We have that $\prob^\nround_{\textsc{DS}(\eta)}(\bm{B})$, the probability that our decoupled sampling Algorithm \ref{alg: Decoupled Sampler} returns bid vector $\bm{B} \in \mathcal{B}^\Nitem$, is given by:
    \begin{align}
        \prob^\nround_{\textsc{DS}(\eta)}(\bm{B}) &= \prob^\nround_{\textsc{DS}(\eta)}(b_1 = B_1) \prob^\nround_{\textsc{DS}(\eta)}(\bm{b}_{2:\Nitem} = \bm{B}_{2:\Nitem} \mid b_1 = B_1)\\
        &= \frac{S^\nround_1(B_1)}{\sum_{B \leq B_0} S^\nround_1(B)} \prob^\nround_{\textsc{DS}(\eta)}(\bm{b}_{2:\Nitem} = \bm{B}_{2:\Nitem} \mid b_1 = B_1)\\
        &= \frac{S^\nround_1(B_1)}{\sum_{B \leq B_0} S^\nround_1(B)} \prob^\nround_{\textsc{DS}(\eta)}(b_2 = B_2 \mid b_1 = B_1) \prob^\nround_{\textsc{DS}(\eta)}(\bm{b}_{3:\Nitem} = \bm{B}_{3:\Nitem} \mid b_1 = B_1, b_2 = B_2) \\
        &= \frac{S^\nround_1(B_1)}{\sum_{B \leq B_0} S^\nround_1(B)} \frac{S^\nround_2(B_2)}{\sum_{B \leq B_1} S^\nround_2(B)} \prob^\nround_{\textsc{DS}(\eta)}(\bm{b}_{3:\Nitem} = \bm{B}_{3:\Nitem} \mid b_1 = B_1, b_2 = B_2) \\
        &= \frac{S^\nround_1(B_1)}{\sum_{B \leq B_0} S^\nround_1(B)} \frac{S^\nround_2(B_2)}{\sum_{B \leq B_1} S^\nround_2(B)} \prob(\bm{b}_{3:\Nitem} = \bm{B}_{3:\Nitem} \mid b_2 = B_2) \ldots
    \end{align}
    Continuing the above expansion and then applying the definition of $S^\nround_\nitem(B)$, we obtain:
    \begin{align}
        \prob^\nround_{\textsc{DS}(\eta)}(\bm{B}) = \prod_{\nitem=1}^\Nitem \frac{S^\nround_\nitem(B_\nitem)}{\sum_{B \leq B_{\nitem-1}} S^\nround_\nitem(B)} = \prod_{\nitem=1}^\Nitem \frac{\exp(\eta W_\nitem^\nround(B_\nitem)) \sum_{B \leq B_{\nitem}} S^\nround_{\nitem+1}(B)}{\sum_{B \leq B_{\nitem-1}} S^\nround_\nitem(B)} = \prod_{\nitem=1}^\Nitem \exp(\eta W_\nitem^\nround(B_\nitem))
    \end{align}
    Here, the final term simplifies to $\prob_{EW(\eta)}^{\nround}(\bm{B}) = \exp(\eta \mu^{ \nround}_n(\bm{B}))$ which is precisely the exponentially weighted utility associated with bid vector $\bm{B}$.
\end{proof}

Note that this decomposition is similar to the loop-free path kernels method for SSP as detailed in \rigel{Cite Takimoto Path kernels/multiplicative weights paper}. The primary difference is that we consider weights over nodes rather than over edges, as in our setting, the reward associated with selecting bid $B'$ in slot $\nitem+1$ is independent of selecting bid $B \geq B'$ in slot $\nitem$. By doing so, we save a factor of $|\mathcal{B}|$ time and space as we store and update weights corresponding to $O(\Nitem |\mathcal{B}|)$ possible $(\nitem, B)$ slot-bid pairs rather than $O(\Nitem |\mathcal{B}|^2)$ possible $(\nitem, B, B')$ slot-bid-next bid triplets. Now, we state the full decoupled exponential weights algorithm and its associated regret, run-time, and space complexity.

\begin{algorithm}[t]
	\KwIn{Learning rate $\eta > 0$, Adaptive Adversarial Environment $\textsc{Env}^\nround: \mathcal{H}^\nround \to [0, 1]^\Nitem \times \mathcal{B}^{-\Nitem} \times \mathcal{B}$ where $\mathcal{H}^\nround$ denotes the set of all possible historical auction results $H^\nround$ up to round $\nround$ for all $\nround \in [\Nround]$.}
	\KwOut{The aggregate utility $\sum_{\nround=1}^\Nround \mu^\nround_n(\bm{b}^{\nround})$ corresponding to a sequence of bid vectors $\bm{b}^{1},\ldots,\bm{b}^{\Nround}$ sampled according to the exponential weights algorithm.}
	$W_\nitem^0(B) \gets 0, \textsc{NumberWon}_\nitem^0(B) \gets 0$ for all $\nitem \in [\Nitem], B \in \mathcal{B}$\;
        $H^0 \gets \emptyset$\;
	\For{$\nround \in [\Nround]$:}{
            $(\bm{v}^{n, \nround}, \bm{b}^{\nround}_-, \pi^\nround) \gets \textsc{Env}^{\nround-1}(H^{\nround-1})$ and $\bm{b}^{n, \nround} \gets \textsc{DecoupledSampler}(\bm{W}^{\nround-1}, \eta)$\;
            Observe $\bm{v}^{n, \nround}$ and update utilities $\bm{W}^{\nround - 1} \gets \{W^{\nround-1}_{\nitem}(B) = \textsc{NumberWon}_\nitem^{\nround-1}(B) (v^{n, \nround}_\nitem - B)\}_{\nitem \in [\Nitem], B \in \mathcal{B}}$\;
            Observe $\bm{b}^{\nround}_-, \pi^\nround$ and receive reward $\mu(\bm{v}^{n, \nround}, \bm{b}^{n, \nround}, \bm{b}^{\nround}_-, \pi^\nround)$\;
            $\textsc{NumberWon}_\nitem^\nround(B) \gets \textsc{NumberWon}_\nitem^{\nround-1}(B) + \textbf{1}_{B > \max(b^{-n, \nround}_\nitem, \pi^\nround)}$ for all $\nitem \in [\Nitem], B \in \mathcal{B}$\;
        }
        \textbf{Return} $\sum_{\nround=1}^\Nround \mu(\bm{v}^{n, \nround}, \bm{b}^{n, \nround}, \bm{b}^{\nround}_-, \pi^\nround)$
	\caption{\textsc{Decoupled Exponential Weights - Full Information}}
	\label{alg: Decoupled Exponential Weights}
\end{algorithm}

\begin{theorem}
    With $\eta \propto \sqrt{\frac{\log |\mathcal{B}|}{T}}$, Algorithm \ref{alg: Decoupled Exponential Weights} achieves regret $O(\Nitem \sqrt{ \Nround \log |\mathcal{B}|})$, with total time complexity $O(\Nitem |\mathcal{B}| \Nround)$ and space complexity $O(\Nitem |\mathcal{B}|)$.
\end{theorem}

\begin{proof}
    +As the rewards are bounded between $-\Nitem$ and $\Nitem$ and the state space is of size $O(|\mathcal{B}|^\Nitem)$, the exponential weights algorithm guarantees a regret upper bound $O(\eta \Nitem \Nround + \frac{\Nitem \log |\mathcal{B}|}{\eta})$ which achieves the desired regret bound with the state choice of $\eta$. Note our procedure works for non-stationary, potentially adversarially selected valuations as we have full cross-learning across all possible valuation profiles $\bm{v}^{\nround}$. More specifically, given $\bm{b}^\nround_-$ and $\pi^\nround$, the utility associated with any bid vector and valuation profile pair can be computed exactly. To do this efficiently, we decouple the utility per slot-bid value pair by tracking the number of items that would have been won across $\nround$ rounds by bidding bid $b$ at slot $\nitem$:
    \begin{align}
        W_\nitem^\nround(b) = \sum_{\tau=1}^\nround w_\nitem^\tau(b) = \sum_{\tau=1}^\nround \textbf{1}_{b \geq \max{b_{-\nitem}^{\tau}, \pi^\tau}} (v_\nitem^{\nround} - b) = \textsc{NumberWon}_\nitem^\nround(b)(v_\nitem^{\nround} - b) 
    \end{align}
    For the complexity analysis, updating and storing $\bm{W}^\nround$ and $\{\textsc{NumberWon}_\nitem^\nround(B)\}_{\nitem \in [\Nitem], B \in \mathcal{B}}$ at each $\nround$ requires $O(\Nitem |\mathcal{B}|)$ time and space. Similarly, each call to $\textsc{DecoupledSampler}$ requires computing $\{S^\nround_\nitem(B)\}_{\nitem \in [\Nitem], B \in \mathcal{B}}$, which can be done recursively in $O(\Nitem |\mathcal{B}|)$ time and space complexity (we save an additional factor of $|\mathcal{B}|$ by intermittently storing the values of $\sum_{B' \leq B} S^\nround_{\nitem+1}(B')$). Discarding old tables, the total time and space complexities of Algorithm \ref{alg: Decoupled Exponential Weights} are $O(\Nitem |\mathcal{B}| \Nround)$ and $O(\Nitem |\mathcal{B}|)$ respectively. 
\end{proof}

We remark that this algorithm works for any adversarially selected valuation profiles as the full information setting allows for perfect cross learning between valuations. While this is also true in the bandit setting, we will see that Hedge style algorithms break down.
% \section{Surrogate-based overlapping Schwarz method}
\label{sec:online}

In this section,  an efficient strategy to construct the global solution of problem~\eqref{eq:globalProb} for a fixed set of parametric values $\bar{\bmu} \in \mathcal{P}$ is presented.
%
The goal is to devise a procedure, suitable for real time execution, to appropriately glue the parametric solutions of the local subproblems, thus drastically reducing the cost of the overall DD algorithm.
%
To this end, the overlapping Schwarz method presented in Sect.~\ref{sec:schwarzAlg} is adapted to exploit the local PGD surrogate models constructed in Sect.~\ref{sec:offline}.

\begin{rem}
To easily perform the coupling between subdomains $\Omega_1$ and $\Omega_2$ in the online phase and to avoid expensive interpolation procedures among different grids, the meshes used for the spatial discretization of the local subdomains are assumed to be conforming with the interfaces (i.e., the interfaces do not cut through any elements of the meshes) and to coincide in the overlapping region $\Omega_{12}$.
\end{rem}

First, note that by construction (see~\eqref{eq:lambdaRep}),  the vector $\mathbf{u}_{\Gamma_i}\!(\bar{\bmu})$ of the nodal values of the solution at $\Gamma_i$ corresponds to the vector of parameters $\bLambda_i$. Hence, for any $\bar{\bmu} \in \mathcal{P}$, equation~\eqref{eq:SchwarzAlg} can be rewritten as
%
\begin{equation}\label{eq:SchwarzAlgLambda}
\mat{I}_{\Gamma_j}\bLambda_j
=
\mat{R}_{\Omega_i\to\Gamma_j} \mat{A}_{\Omega_i}^{-1} \mathbf{f}_{\Omega_i}\!(\bar{\bmu}) 
+ \mat{R}_{\Omega_i\to\Gamma_j} \mat{A}_{\Omega_i}^{-1} (-\mat{A}_{\Gamma_i}\bLambda_i )\, ,
\end{equation}
%
where $\mat{A}_{\Omega_i}^{-1} \mathbf{f}_{\Omega_i}\!(\bar{\bmu})$ corresponds to the solution of problem~\eqref{eq:sourceProb} and $\mat{A}_{\Omega_i}^{-1} (-\mat{A}_{\Gamma_i}\bLambda_i)$ denotes the solution of problem~\eqref{eq:boundaryProbs}.
% 
Exploiting the local surrogate models constructed in the offline phase, equation~\eqref{eq:SchwarzAlgLambda} reduces to
%
\begin{equation}\label{eq:SchwarzAlgPGD}
\mat{I}_{\Gamma_j}\bLambda_j
=
\mat{R}_{\Omega_i\to\Gamma_j} \bupgd_{i,0}\!(\bar{\bmu}) 
+ \mat{R}_{\Omega_i\to\Gamma_j} \bupgd_{i,\Lambda}\!(\bar{\bmu},\bLambda_i)  \, ,
\end{equation}
%
where $\bupgd_{i,0}\!(\bar{\bmu})$ and $\bupgd_{i,\Lambda}\!(\bar{\bmu},\bLambda_i)$ denote the vectors of the nodal values of the PGD solutions $\upgd_{i,0}$ and $\upgd_{i,\Lambda}$, respectively, evaluated for the target values $\bar{\bmu} \in \mathcal{P}$ and $\bLambda_i \in \mathcal{Q}_i$.

Let $\mat{A}_{i,\Lambda_i}^\texttt{PGD}$ be the local PGD operator 
%
\begin{equation}\label{eq:PGDlambdaOper}
\mat{A}_{i,\Lambda_i}^\texttt{PGD} : \bLambda_i \to \bupgd_{i,\Lambda}(\bar{\bmu},\bLambda_i) \, 
\end{equation}
%
that, given a set of boundary parameters $ \bLambda_i$, returns the nodal values of the PGD surrogate model $\upgd_{i,\Lambda}$ of problem~\eqref{eq:boundaryProbs} for the set of parameters $\bar{\bmu}$. Hence, equation \eqref{eq:SchwarzAlgPGD} can be rewritten as
\begin{equation}\label{eq:SchwarzAlgPGD_Op}
\mat{I}_{\Gamma_j}\bLambda_j
=
\mat{R}_{\Omega_i\to\Gamma_j} \bupgd_{i,0}\!(\bar{\bmu}) 
+ \mat{R}_{\Omega_i\to\Gamma_j} \mat{A}_{i, \Lambda_i}^\texttt{PGD} \bLambda_i  \, ,    
\end{equation}
%
and the surrogate-based overlapping Schwarz method is finally obtained by rewriting the interface system~\eqref{eq:systemSchwarzInterface} as
%
\begin{equation}\label{eq:systemSchwarzInterfacePGD}
\begin{pmatrix}
\mat{I}_{\Gamma_1} & -\mat{R}_{\Omega_2\to\Gamma_1} \mat{A}_{2,\Lambda_2}^\texttt{PGD} \\[3pt]
-\mat{R}_{\Omega_1\to\Gamma_2} \mat{A}_{1,\Lambda_1}^\texttt{PGD} & \mat{I}_{\Gamma_2}
\end{pmatrix}
\begin{pmatrix}
\bLambda_1 \\[3pt]
\bLambda_2
\end{pmatrix}
=
\begin{pmatrix}
\mat{R}_{\Omega_2\to\Gamma_1} \bupgd_{2,0}\!(\bar{\bmu}) \\[3pt]
\mat{R}_{\Omega_1\to\Gamma_2} \bupgd_{1,0}\!(\bar{\bmu})
\end{pmatrix} \, .
\end{equation}

Therefore, the online phase of the method consists of an iterative strategy to solve equation~\eqref{eq:systemSchwarzInterfacePGD}, e.g., by GMRES.  At convergence, say, at iteration $k=k^*$,  the approximation of the solution of the global problem~\eqref{eq:globalProb} for $\bar{\bmu} \in \mathcal{P}$ is thus given by
%
\begin{equation}\label{eq:globalSolnA}
	\bupgd(\bar{\bmu}) = 
	\begin{cases}
		\bupgd_{1,0}(\bar{\bmu}) + 
		\bupgd_{1,\Lambda}(\bar{\bmu},\bLambda_1^{(k^*)} ) \quad \text{in } \Omega_1,\\
		\noalign{\vskip5pt}
		\bupgd_{2,0}(\bar{\bmu}) + 
		\bupgd_{2,\Lambda}(\bar{\bmu},\bLambda_2^{(k^*)} ) \quad \text{in } \Omega_2 \setminus \Omega_{12}.
	\end{cases}
\end{equation}

It is worth noticing that the values $\bLambda_i^{(k)}$ computed by GMRES iterations to solve problem~\eqref{eq:systemSchwarzInterfacePGD} may not coincide with any of the values obtained from the discretization of the parametric domain $\mathcal{Q}_i$.  
%
If this is the case, the solution $\bupgd_{i,\Lambda}(\bar{\bmu},\bLambda_i^{(k)})$ provided by the operator $\mat{A}_{i,\Lambda}^\texttt{PGD}$ is obtained by performing a linear interpolation of the parametric modes depending on $\bLambda_i$ and associated with the available values closest to $\bLambda_i^{(k)}$.

\begin{rem}
Following Algorithm~\ref{alg:Schwarz}, at the beginning of the online phase,  an instance $\bar{\bmu}$ is selected in the set of parameters $\mathcal{P}$.
%
This is not strictly necessary and the described algorithm can be adapted to handle arbitrary parameters $\bmu$. 
%
In the latter case, at the end of the online phase, one would obtain a global surrogate model that represents a family of solutions of the global problem~\eqref{eq:globalProb} depending on $\bmu \in \mathcal{P}$, instead of an instance of such model for $\bmu = \bar{\bmu}$. 
\end{rem}
% \section{Bid Optimization: Bandit Feedback}


{\color{red}
\begin{itemize}
\item explain what we did in the full info setting and why that idea does not work here when we have bandit setting
\item at the high level, we want to convert the EXP3 algorithm we designed with the help of the DP formulation to the bandit setting.
\item on natural idea that has been presented in the prior work is the path kernel method that uses the idea of weight pushing. See also XXX (such as combinatorial bandits paper)
\item if we apply these algorithms, we get sub-optimal regret in terms of $M$.
\item this is because our problem has additional structure that we can use. 
\item explain the structure and may want to  present a lemma (remark or observation works too)
\item our algorithm presented in XXX uses this structure 
\item Explain the algorithm. 
\item Tell how/where it uses the structure/the DP formulation 
\item talk about the optimization problem
\item at a high level, this alg bears some resemblance to Component hedge (weight pushing), combinatorial m-sets, O-REPS. But, that algorithm (as it was the case for the kernel method) fails to use the structure and leads to a sub-optimal regret.  
\end{itemize}
}


\begin{algorithm}[t]
	\KwIn{Family of probability measures over states $\bm{\psi}: [\Nitem] \times \mathcal{B} \to [0, 1]$. }
	\KwOut{Bid vector $\bm{b}$.}
        \textbf{Recover Policy $\bm{\pi}$ from Condensed Policy $\bm{\psi}$}: $\pi(S_0, b) \gets \psi_1(b)$ and $\pi((m, b), b') \gets \frac{\psi_{m+1}(b')}{\sum_{b" \leq b} \psi_{m+1}(b)}$ for all $m \in [\Nitem], b \geq b', b, b' \in \mathcal{B}$.\;
        \textbf{Sampling $\bm{b}$ Recursively}: Set $b_1$ to $b \in \mathcal{B}$ with probability  $\pi(S_0, b)$.\;
        \For{$\nitem = 1,\ldots,\Nitem-1$:}{
            Set $b_{\nitem+1}$ to $b \in \mathcal{B}$ with probability $\pi((\nitem, b_\nitem), b)$
        }
        \textbf{Return} $\bm{b}$
	\caption{\textsc{Monotone Bid Sampling in Multi-Unit Pay-as-Bid Auctions}}
	\label{alg: Sample Node O-REPS}
\end{algorithm}

\begin{algorithm}[t]
	\KwIn{Valuation $\bm{v} \in [0, 1]^\Nitem$, and Learning rate $\eta > 0$. } %Adaptive Adversarial Environment $\textsc{Env}^\nround: \mathcal{H}^\nround \to \mathcal{B}^{-\Nitem} \times \mathcal{B}$ where $\mathcal{H}^\nround$ denotes the set of all possible historical auction results $H^\nround$ up to round $\nround$ for all $\nround \in [\Nround]$.}
	\KwOut{The aggregate utility $\sum_{\nround=1}^\Nround \mu(\bm{b}^\nround)$.}
	$\psi^0(\nitem, b) \gets \frac{1}{|\mathcal{B}|}$ for all $\nitem \in [\Nitem], b \in \mathcal{B}$. Let $\bm{q}^0 \in [\Nitem] \times \mathcal{B} \to [0, 1]$ be the corresponding state-action {\color{red} let's call this ``unit-bid" pair or sth like that} occupancy measure.%, where $q^\nround_\nitem(b)$ is the probability of selecting bid $b$ at state $\nitem$ \;
	\For{$\nround \in [\Nround]$:}{
            %Adversary selects $\bm{b}^{\nround}_-$ and 
            \textbf{Determining the Bid Vector $\bm {b}^t$ recursively.} \rigel{Write this as its own algorithm} Without observing  $\bm{b}^{\nround}_-$, choose $\bm{b}^{\nround} \sim \bm{\psi}^{\nround-1}$ is sampled according to the policy generated by $\bm{\psi}^{\nround-1}$ {\color{red}The same way we did in the previous section, we need to explain how this sampling is done. I think we agreed to do this in the previous meeting}\;
            Receive reward {\color{red} let's be consistent: $\mu^\nround_n$ versus $\mu^\nround$. We have both in this section.} $\mu^\nround_n(\bm{b}^\nround) = \sum_{\nitem=1}^\Nitem w_\nitem^\nround(b^\nround_\nitem)$ and observe $w_\nitem^\nround(b^\nround_\nitem)$ where $w_\nitem^\nround(b) = (v_\nitem - b)\textbf{1}_{b \geq b^\nround_{-\nitem}}$\;
            $\hat{w}_\nitem^\nround(b) \gets \frac{w_\nitem^\nround(b)}{q^{\nround-1}_\nitem(b)} \textbf{1}_{b = b^{\nround}_\nitem}$ for all $\nitem \in [\Nitem]$ and $b \in \mathcal{B}$\;
            \textbf{Determining Probability Measure $\bm{q}^t$ over any pair of $(m, b)$}
            $\bm{q}^\nround \gets \text{argmin}_{\bm{q} \in \mathcal{Q}} \eta\langle \bm{q}, -\hat{\bm{w}}^\nround\rangle + D(\bm{q} || \bm{q}^{\nround-1})$ where $\mathcal{Q}$ is as in Definition~\ref{def: Qspace} and \[D(\bm{q} || \bm{q}') = \sum_{\nitem \in [\Nitem], b \in \mathcal{B}} q_\nitem(b)\frac{\log q_\nitem(b)}{q'_\nitem(b)} - (q_\nitem(b) - q'_\nitem(b))\,.\]
            \textbf{Convert $\bm{q}^t$ to Policy $\bm{\psi}^t$.}
            {\color{red} write a for loop for that}
            Recursively compute for all $\nitem$, $b$ in decreasing order $\psi^\nround_\nitem(b) \gets q^\nround_\nitem(b)\left(\sum_{b' \geq b} \frac{q^\nround_{\nitem-1}(b')}{1 - \sum_{b" > b'} \psi^\nround_\nitem(b")}\right)^{-1}$\;
            % Compute $\pi^\nround((\nitem, b), b') \gets \frac{\psi^\nround(\nitem+1, b')}{\sum_{b" \leq b} \psi^\nround_{\nitem+1}(b")}$ for all $\nitem \in [\Nitem], b \geq b' \in \mathcal{B}$ \rigel{Get rid of $\pi$ here and then just explicitly write $\frac{\psi^\nround(\nitem+1, b')}{\sum_{b" \leq b} \psi^\nround(\nitem+1, b")}$ in the sampling $\bm{b}$ step}\;
        }
        \textbf{Return} $\sum_{\nround=1}^\Nround \mu(\bm{b}^\nround)$
	\caption{\textsc{Bandit Bid Optimization in Multi-Unit Pay as Bid Auctions}}
	\label{alg: Node O-REPS}
\end{algorithm}

In this section, we now consider the more realistic feedback setting where agent $n$ only observes each auction $\nround$'s allocation $x^\nround = x^\nround_n(\bm{b}^\nround)$. \negin{ why are talking about cross-learning between actions that we are not even using. That would raise questions and could hurt us} In the full information setting, we were able to run a decoupled version of exponential weights which allowed us to obtain the counterfactual utility for every possible bid vector. However, in the bandit feedback setting, $\bm{b}^\nround_-$ is not revealed to the learner and thus, the bidder's allocation (and hence utility) would not be available to the learner for all possible bid vectors and instead only observes their utility for their submitted bid vector. Vanilla $\textsc{Exp3}$ here would result in exponential in $\Nitem$ regret as the bid space is exponentially large. 

Rather than running a variant of an $\textsc{Exp3}$ algorithm, our algorithm, which is presented in Algorithm \ref{alg: Node O-REPS}, leverages the DP formulation in Section 
\ref{sec:offline}. Recall that in the DP formulation, we have $M$ layers, where in each layer there are $|\mathcal B|$ nodes and {\color{red}XXX edges}. Given the structure of the DP, one idea is maintain some policy $\pi$ which induces a family of probability measures $\rho$ over the edges in the DP graph. {\color{red} add some figure for the DP section. make sure the notion of "DP graph" is well-defined. Formally define  } More precisely, {\color{red} explain $\pi$ and $\rho$.} 

Following this idea would lead to an algorithm with regret of XXX.  In this section, as our main contribution, instead of maintaining probability measures over the edges in the DP graph, we maintain probability measures over nodes. This idea is based on an important observation we made. In particular, in the DP formulation, we know that regardless of the value of $b_\nitem$, transitioning to state $b_{\nitem+1}$ always yields the same utility $w_\nitem^\nround(b_{\nitem+1})$ {\color{red} explain the previous sentence using nodes and edges and is the the (immediate) reward we obtain when we go from some $(m, b)$ to another node}. {\color{red} this translates this property on $\pi$ and maybe $\rho$: todo add that property}\rigel{Add something here about the connection to $\pi$ and then relate back to the algorithm} 
{\color{red} given that, define $\psi$ and maybe $q$ based on $\rho$ and $\pi$.}

Now, we are ready to explain our alg in more details. The algorithm has three steps. {\color{red} todo: complete.}



\newpage 
It may be tempting to apply $\textsc{Exp3}$ variants \rigel{Cite Exp3.G paper, Exp3.Set} that  by constructing importance sampling estimators using the observation that any bid above $c^\nround = b^{\nround}_{x^\nround}$ in the first $x^{\nround}$ slots is guaranteed to win the corresponding item, and any bid below $d^\nround = b^{\nround}_{x^\nround + 1}$ is guaranteed to lose the item. However, as the feedback graph is not revealed to the learner, such methods do not apply to our setting. As such, we resort to learning algorithms that construct utility estimates that do not depend on the entire set of rewards of cross-learnable bid vectors. Such algorithms include weight pushing \rigel{cite Mehryar Mohri (1998)}, combinatorial bandit algorithms (\rigel{Cite Chen 2014}), or FTRL (Follow the Regularized Leader) or OMD (Online Mirror Descent) based stochastic shortest path (SSP) solvers \rigel{Cite O-REPS, Component Hedge}. In these methods, the utility estimates depend only on the per-slot  utilities of the selected bid vector \negin{What do you mean by `` per-slot  utilities of the selected bid vector".I found this discussion confusing}. However, applying these methods directly results in sub-optimal regret as they fail to exploit additional structure within the bid optimization problem. Instead, we restate our bid optimization problem as a node-weight-maximization problem over a layered graph. We then describe an algorithm to solve this problem and construct upper and lower bounds on regret.

\subsection{Graphical Formulation}

Similar to our dynamic program in the offline setting, the learner maintains a table of utility estimates corresponding to $\{w^\nround_\nitem(b^\nround_\nitem)\}_{\nitem \in [\Nitem]}$ instead of explicitly storing the utility estimates corresponding to entire bid vectors. In order to recover the utility estimate of a bid vector, the learner sums the utility estimates of each of the components of the bid vector. The learner wishes to maximize their cumulative utility across $\Nround$ rounds. This interpretation of the bid optimization problem lends itself naturally to a graphical interpretation, where we construct a graph $\mathcal{G} = (\mathcal{S}, \mathcal{A})$ where we associate with each node in $s \in \mathcal{S}$ a weight and we seek to find the maximum weight path in $\mathcal{G}$. More concretely, we define $\mathcal{G}$ as follows:
\begin{enumerate}
    \item We define 2 types of states: source node $S_0$, followed by $\Nitem$ layers of states of width $|\mathcal{B}|$ between them which we denote by $\{(\nitem, b)\}_{\nitem \in [\Nitem], b \in \mathcal{B}}$. We let $\mathcal{S}$ denote the set of all states.
    \item There are 2 types of actions available at each state which describe which state in the subsequent layer the agent will move to: actions $a_0$ from $S_0$ to the entire first layer of nodes and actions $a_\nitem$ between layers $\nitem$ to layer $\nitem+1$. In particular, the latter set of actions requires non-decreasing bid value---i.e. $(\nitem, b)$ only has an action to $(\nitem+1, b')$ for $b' \leq b$. This reflects the bid monotonicity assumption. With probability 1, the agent will transition to the state prescribed by their action, where we say that the agent took (valid) action $a_\nitem = b$ if they selected action $(\nitem, b_\nitem) \to (\nitem+1, b)$. We let $\mathcal{A} = \mathcal{B}$ denote the set of all possible actions.
    \item There are 2 parameters: the valuation vector $\bm{v}$ and adversarial bid vector $\bm{b}^{\nround}_-$. With these parameters, the weight of state $(\nitem, B)$ is given by $w^{\nround}_{\nitem}(B) = 1_{b \geq b^{\nround}_{-\nitem}} (v_\nitem - b)$. This is precisely the same as the definition of $w^{\nround}_{\nitem}(b)$ as in offline bid optimization. The total reward of episode $\nround$ is equal to the sum of the node weights traversed. We show an example graph in figure~\ref{fig: valid_bids_graph}.
\end{enumerate}

% Figure environment removed

\subsection{Reduction to Online Linear Optimization with Semi-Bandit Feedback}

The agent's objective is to maximize their cumulative reward across $\Nround$ rounds using only feedback in each round given as $\{w_\nitem^\nround(b^\nround_\nitem)\}_{\nitem \in [\Nitem]}$. In order to balance exploration and exploitation, the agent stores, updates, and samples bid vectors according to policies $\pi^\nround: \mathcal{S} \times \mathcal{A} \to [0, 1]$, where $\pi^\nround((\nitem, b), b')$ denotes the probability of selecting action $b' \leq b$ at state $(\nitem, b)$ and round $\nround \in [\Nround]$. Abusing notation, let $\bm{b} \sim \pi$ denote a sequence of bid values $(b_1,\ldots,b_\Nitem)$ within a path sampled according to policy $\pi$. A common idea in the literature regarding episodic learning in MDP's is to construct a corresponding state-action occupancy measure $\rho^\pi((\nitem, b), b') = \prob_{\bm{b} \sim \pi}(b_\nitem = b, a_\nitem = b')$, with $\rho^\pi(S_0, b') = \prob_{\bm{b} \sim \pi}(s_0 = S_0, a_0 = b_1)$. Since we are operating under a deterministic Markovian environment with (directed) edges between consecutive layers, we have:{\color{red} not obvious why we have this. Please explain why this holds }
\begin{align}
    \sum_{b' \leq b} \rho^\pi((\nitem, b), b') = \sum_{b" \geq b} \rho^\pi((\nitem-1, b"), b) \quad \text{and} \quad \sum_{b \in \mathcal{B}} \sum_{b' \leq b} \rho^\pi((\nitem, b), b') = 1
\end{align}{\color{red} it is not a good practice to explain things using certain variables ($\rho$ here) that do nor even show up in the algorithm }
The key idea of selecting state-action pairs with marginal probabilities given by $\bm{\rho}$ is that we can rephrase the bid optimization problem as an instance of online linear optimization with semi-bandit feedback. More specifically, we can compute the (expected) loss function at round $\nround$ as follows: {\color{red} maybe her we oversimplify our method. We had a DP with value to goes. We need to also explain how we took care of it. To me, this is hidden in the sampling procedure that we did not explain well   }
\begin{align}
    \mathbb{E}_{\bm{b} \sim \pi^\rho}\left[ \mu^{\nround}_n(\bm{b}) \right] = \mathbb{E}_{\bm{b} \sim \pi^\rho}\left[ \sum_{\nitem=1}^\Nitem w_\nitem^\nround(b_\nitem) \right] = \sum_{\nitem=1}^\Nitem \prob_{\bm{b} \sim \pi^\rho}(s_{\nitem-1} = b_{\nitem-1}, a_{\nitem-1} = b_{\nitem}) w_\nitem^\nround(b_\nitem)
\end{align}
Substituting in the definition of $\pi^\rho$ and the corresponding $\rho$, we have:
\begin{align}
    \mathbb{E}_{\bm{b} \sim \pi^\rho}\left[ \mu^{\nround}_n(\bm{b}) \right] = \sum_{\nitem=1}^\Nitem \sum_{b \in \mathcal{B}} \sum_{b' \leq b} \rho((\nitem-1, b), b') w_\nitem^\nround(b') = \langle \bm{\rho}, \bm{w}^\nround \rangle 
\end{align}
Where we assume $s_0 = S_0$ and define $\bm{w}^\nround \equiv \{w_\nitem^\nround(b')\}_{\nitem \in [\Nitem], b \geq b' \in \mathcal{B}}$. Assuming that the learner selects occupancy measure $\bm{\rho}^\nround$ at round $\nround$, the regret can then be written as:
\begin{align}
    \textsc{Regret}_\mathcal{B}(\Nround) &= \max_{\bm{b} \in \mathcal{B}^{+\Nitem}} \sum_{\nround=1}^\Nround \mu^\nround_n(\bm{b}) - \mathbb{E}_{\bm{b}^\nround \sim \bm{\rho}^\nround} \sum_{\nround=1}^\Nround \mu^\nround_n(\bm{b}^\nround)\\
    &\leq \max_{\bm{\rho} \in \Delta(\Pi)} \mathbb{E}_{\bm{b} \sim \bm{\rho}} \sum_{\nround=1}^\Nround \mu(\bm{b}) - \mathbb{E}_{\bm{b}^\nround \sim \bm{\rho}^\nround} \sum_{\nround=1}^\Nround \mu^\nround_n(\bm{b}^\nround)\\
    &= \max_{\bm{\rho} \in \Delta(\Pi)} \sum_{\nround=1}^\Nround \langle \bm{\rho} - \bm{\rho}^\nround, \bm{w}^\nround \rangle
\end{align}
{\color{red}I think that in your writing, you try to bring some parts of the proofs to the main text. This is good to do for the main idea. But, perhaps you are over doing it. Here is how I would write it. I would say this is our  algorithm. The algorithm has 3 main steps. Step 1- bidding recursively according to $\boldsymbol{\psi}^t$. In this step, we fist compute $b_t^1$, and then XXX. This step  returns a monotone bid vector. Step 2- computing $\textbf{q}^t$, which is XXX. While $\textbf{q}^t$ is useful, it cannot determine the bid vectors. This will be handled in Step 3. Step 3- we convert $\textbf{q}^t$, we convert $\textbf{q}^t$ to a policy  $\boldsymbol{\psi}^t$ that will be used to determine the bid vector for the next round. Then, you can talk about the novelty/intuitions/ideas in the design of the algorithm.   You can say sth like this: In the design of this algorithm, we used two/three main ideas: (I would then try to relate these ideas to the three steps you mentioned above) \\
Idea 1: maybe your first idea is the sampling method that helps you not be worried about value to goes in the DP. The sampling lets you determine bids sequentially while maintaining monotonicity.  Idea 2: maybe idea 2 is using the properties of the DP formulation that can be viewed as a deterministic MDP to XXXX.  Maybe you used this idea in computing $\bm{q}^t$.}
Where the inequality follows as any deterministic bid can be represented as an expectation over a deterministic policy, where the space of all measures on the set of all policies is given by $\Delta(\Pi)$. Existing algorithms for online linear optimization \rigel{Cite Lattimore 2020, Audibert 2013, O-REPS paper} over dimension $O(\Nitem |\mathcal{B}|^2)$ guarantee a regret upper bound of $O(\Nitem |\mathcal{B}| \sqrt{T \log \mathcal{B}})$. However, in our bid optimization setting, these algorithms achieve sub-optimal regret as the state-action occupancy measure does not fully exploit the structure of the problem. In particular, we know that regardless of the value of $b_\nitem$, transitioning to state $b_{\nitem+1}$ always yields the same utility $w_\nitem^\nround(b_{\nitem+1})$. \rigel{Add something here about the connection to $\pi$ and then relate back to the algorithm} Consequently, it is natural to remove the dependence on the value of $b_\nitem$ in the policy, subject to bid monotonicity. We do this by assuming a proportionality constraint on $\Pi$, the space of all possible policies, in the sense that the probability of transitioning to $(\nitem+1, b)$ versus $(\nitem, b')$ proportional under $b_\nitem$ and $b'_\nitem$. That is,
\[
\frac{\pi((\nitem, b), b')}{\pi((\nitem, b), b")} = \frac{\pi((\nitem, b'), b')}{\pi((\nitem, b'), b")} \text{ for all } \nitem \in [\Nitem], b" \leq b' \leq b, b", b', b \in \mathcal{B}
\]
Using the fact that $\sum_{b' \leq b} \pi((\nitem, b), b') = 1$, we have:
\begin{align}
    &\frac{\pi((\nitem, b_0), b')}{\pi((\nitem, b_0), b")} = \frac{\pi((\nitem, b'), b')}{\pi((\nitem, b'), b")} \\
    &\leftrightarrow \pi((\nitem, b_0), b')\pi((\nitem, b'), b") = \pi((\nitem, b'), b')\pi((\nitem, b_0), b")\\
    &\to \sum_{b" \leq b'} \pi((\nitem, b_0), b')\pi((\nitem, b'), b") = \sum_{b" \leq b'} \pi((\nitem, b'), b')\pi((\nitem, b_0), b") \\
    &\leftrightarrow \pi((\nitem, b_0), b') = \pi((\nitem, b'), b') \sum_{b" \leq b'} \pi((\nitem, b_0), b")\\
    &\leftrightarrow \pi((\nitem, b'), b') = \frac{\pi((\nitem, b_0), b')}{\sum_{b" \leq b'} \pi((\nitem, b_0), b")}\\
    &\to \pi((\nitem, b), b') = \frac{\pi((\nitem, b_0), b')}{\sum_{b" \leq b} \pi((\nitem, b_0), b")}
\end{align}
Where in the last equality, we plugged in the value of $\pi((\nitem, b'), b')$ back into the proportionality constraint. At a high level, this allows us to describe the entire policy with only $\{\pi(S_0, b')\}_{b' \in \mathcal{B}} \cup \{\pi((\nitem, b_0), b')\}_{\nitem \in [\Nitem], b' \in \mathcal{B}}$, which is a set of size $O(\Nitem |\mathcal{B}|)$. Letting $\psi_\nitem(b) = \pi((\nitem-1, b_0), b)$ denote the probability of transitioning from the maximum possible bid to $b$ at slot $\nitem$, we see that we can rewrite $\pi((\nitem, b), b') = \frac{\psi(\nitem+1, b')}{\sum_{b" \leq b} \psi(\nitem+1, b")}$. Furthermore, letting $\Psi: [\Nitem] \times \mathcal{B} \to [0, 1]$ such that $\sum_{b \in \mathcal{B}} \psi_\nitem(b) = 1$ denote the set of possible condensed policies $\psi$, we can recursively define the corresponding state occupancy measure $\bm{q}$ where $q_\nitem(b) = \prob_{\bm{b} \sim \psi}(b_\nitem = b)$ denotes the marginal probability that $b_\nitem = b$ when following the policy $\pi^\psi$ generated by $\psi$. With base case $q_1(b) = \psi(S_0, b)$, we have:
\begin{align}
    q_\nitem(b) = \sum_{b' \geq b} q_{\nitem-1}(b')\pi((\nitem-1, b'), b) = \psi_\nitem(b) \sum_{b' \geq b}  \frac{q_{\nitem - 1}(b')}{\sum_{b" \leq b'} \psi_\nitem(b")}
\end{align}
By strong induction, we have $q_\nitem(b) = \prob_{\bm{b} \sim \psi}(b_\nitem = b)$. We define $\mathcal{Q}$ to be the set of all possible state occupancy measures $\bm{q}$ such that there exists a $\psi \in \Psi$ that generates $\bm{q}$. More formally:
\begin{definition}
    \label{def: QSpace}
    Let $\mathcal{Q}$ be the set of all state occupancy measures $\bm{q} \in [\Nitem] \times \mathcal{B} \to [0, 1]$ that satisfy the following properties:
    \begin{enumerate}
        \item \textit{Probability measure validity}: $\sum_{b \in \mathcal{B}} q_\nitem(b) = 1$ for all $\nitem \in [\Nitem]$.
        \item \textit{Detailed balance and proportionality constraint}: There exists a function $\bm{\psi}: [\Nitem] \times \mathcal{B} \to [0, 1]$ with $\sum_{b \in \mathcal{B}} \psi_\nitem(b) = 1$ such that $q_1(b) = \psi(S_0, b)$ and $q_\nitem(b) = \psi_\nitem(b) \sum_{b' \geq b}  \frac{q_{\nitem - 1}(b')}{\sum_{b" \leq b'} \psi_\nitem(b")}$ for all $\nitem \in [\Nitem], b \in \mathcal{B}$.
    \end{enumerate}
\end{definition}

{\color{red} this should be a part of the proof, not the main text.}All that remains to show is how to recover $\bm{\psi}$ (from which we can recover $\pi$) given $q$. To do this, we first compute $\bm{\psi}$ recursively corresponding to $\bm{q}$, and from this, compute $\bm{\pi}$ using $\pi((\nitem, b), b') = \frac{\psi_{\nitem+1}(b')}{\sum_{b" \leq b} \psi_{\nitem+1}(b")}$. To compute $\bm{\psi}$ from $\bm{q}$, notice that for any $\nitem$:{\color{red}determine the range of $b_0$}
\begin{align}
    q_\nitem(b_0) = \psi_\nitem(b_0) \sum_{b \geq b_0} \frac{q_{\nitem-1}(b)}{\sum_{b" \leq b} \psi_\nitem(b")} = \psi_\nitem(b_0) \frac{q_{\nitem-1}(b_0)}{\sum_{b" \leq b_0} \psi_\nitem(b")} = \psi_\nitem(b_0) q_{\nitem-1}(b_0)
\end{align}
Where the last equality follows as $\sum_{b" \leq b_0} \psi_\nitem(b") = \sum_{b" \in \mathcal{B}} \psi_\nitem(b") = 1$. In the recursive case, we have:
\begin{align}
    q(\nitem, b') = \psi_\nitem(b')\sum_{b \geq b'} \frac{q_{\nitem-1}(b)}{\sum_{b" \leq b} \psi_\nitem(b")} = \psi_\nitem(b')\sum_{b \geq b'} \frac{q_{\nitem-1}(b)}{1 - \sum_{b" > b} \psi_\nitem(b")}
\end{align}
Hence, we can solve for $\psi_\nitem(b') = q_\nitem(b')\left(\sum_{b \geq b'} \frac{q_{\nitem-1}(b)}{1 - \sum_{b" > b} \psi_\nitem(b")}\right)^{-1} $ in terms of $\bm{q}$ (which is known) and $\psi_\nitem(b")$ for $b" \geq b'$, which is known from induction. Now that we have obtained concise description of the marginal probabilities of the event that $(\nitem, b)$ is observed in a bid vector sampled according to $\pi^\psi$, we can rewrite the regret as a function of the state occupancy measure $\bm{q}$. Rewriting the loss, we have:
\begin{align}
    \mathbb{E}_{\bm{b} \sim \psi}\left[ \mu^{\nround}_n(\bm{b}) \right] = \mathbb{E}_{\bm{b} \sim \psi}\left[ \sum_{\nitem=1}^\Nitem w_\nitem^\nround(b_\nitem) \right] = \sum_{\nitem=1}^\Nitem \prob_{\bm{b} \sim \psi}(b_\nitem = b) w_\nitem^\nround(b_\nitem)
\end{align}
Substituting in our definition $q_\nitem(b) = \prob_{\bm{b} \sim \psi}(b_\nitem = b)$, we obtain:
\begin{align}
    \mathbb{E}_{\bm{b} \sim \psi}\left[ \mu^{\nround}_n(\bm{b}) \right] = \sum_{\nitem=1}^\Nitem \sum_{b \in \mathcal{B}} q^\nround_\nitem(b_\nitem) w_\nitem^\nround(b_\nitem) = \langle \bm{q}^\nround, \bm{w}^\nround\rangle
\end{align}
Where $\bm{w}^\nround = \{w_\nitem^\nround(b)\}_{\nitem \in [\Nitem], b \in \mathcal{B}}$ which is different as in the previous section where it was defined additionally over $b' \leq b$. Following a similar argument, we obtain:
\begin{align}
    \textsc{Regret}_\mathcal{B}(\Nround) \leq \max_{\bm{q} \in \mathcal{Q}} \sum_{\nround=1}^\Nround \langle  \bm{q}^\nround - \bm{q}, -\bm{w}^\nround \rangle.
\end{align}
Here, the underlying dimension is of size $O(\Nitem |\mathcal{B}|)$ as opposed to $O(\Nitem |\mathcal{B}|^2)$ as in the previous section. As such, the regret corresponding to semi-bandit feedback negative-entropy regularized FTRL or OMD algorithms decrease by a factor of $\sqrt{|\mathcal{B}|}$ and we obtain:
\begin{align}
    \textsc{Regret}_\mathcal{B}(\Nround) \lesssim \Nitem \sqrt{|\mathcal{B}| \Nround \log |\mathcal{B}|}.
\end{align}

\subsection{ Regret Analysis}



\begin{theorem}
    Under bandit feedback, Algorithm~\ref{alg: Node O-REPS} achieves regret rate $\textsc{Regret}_\mathcal{B}(\Nround) \lesssim O(\Nitem \sqrt{|\mathcal{B}| \Nround \log |\mathcal{B}|})$ using $\eta = \sqrt{\frac{\log |\mathcal{B}|}{|\mathcal{B}|\Nround}}$. \rigel{Include time and space complexity here}
\end{theorem}

\begin{proof}
    The proof closely follows that of Theorem 1 in \rigel{Cite O-REPS paper}. At a high level, we want to bound the regret of Follow the (Entropy) Regularized leader. We first do this by upper bounding the regret by the regret of the unconstrained Be the (Entropy) Regularized leader (see Lemma 13 of \rigel{Cite Rakhlin 2009}. In particular, we define the corresponding unconstrained optimization problem:
    \begin{align}
        \tilde{\bm{q}}^{\nround+1} = \text{argmin}_{\bm{q} \in [\Nitem] \times \mathcal{B} \to \mathbb{R}^+} \left( \eta \langle \bm{q}, -\hat{\bm{w}}^{\nround} \rangle + D(\bm{q} || \bm{q}^{\nround}) \right)
    \end{align}
    Where $D(\bm{q} || \bm{q}') = \sum_{\nitem \in [\Nitem], b \in \mathcal{B}} q_\nitem(b)\frac{\log q_\nitem(b)}{q'_\nitem(b)} - (q_\nitem(b) - q'_\nitem(b))$. Here, notice that we are optimizing over all non-negative functions over states pairs rather than the space of state occupancy measures $\mathcal{Q}$. We continue by observing that the estimators $\hat{w}_\nitem^\nround(b) \gets \frac{w_\nitem^\nround(b)}{q^{\nround-1}_\nitem(b)} \textbf{1}_{b = b^{\nround}_\nitem}$ are unbiased:
    \begin{align}
        \mathbb{E}_{\bm{b} \sim \bm{\psi}^\nround}[\hat{w}_\nitem^\nround(b)] = \mathbb{E}_{\bm{b} \sim \bm{\psi}^\nround}[\frac{w_\nitem^\nround(b)}{q^{\nround-1}_\nitem(b)} \textbf{1}_{b = b^{\nround}_\nitem}] = \frac{w_\nitem^\nround(b)}{q^{\nround-1}_\nitem(b)} q^{\nround-1}_\nitem(b) = w_\nitem^\nround(b)\ .
    \end{align}
    Using the unbiasedness of our estimators $\hat{\bm{w}}^\nround$, we can replace $\bm{w}^\nround$ with $\hat{\bm{w}}^\nround$ in the definition of regret. Now, as it is shown in Lemma 13 of \rigel{Cite Rakhlin 2009}, we can upper bound the expected estimated regret as a function of the unconstrained optimizer $\tilde{\bm{q}}^{\nround+1}$ and the unregularized relative entropy with respect to the initial state-action occupancy measure $\bm{q}^0$. 
    \begin{align}
        \textsc{Regret}_\mathcal{B}(\Nround) = \max_{\bm{q} \in \mathcal{Q}} \mathbb{E}\left[\sum_{\nround=1}^\Nround \langle \bm{q}^{\nround} - \bm{q}, -\hat{\bm{w}}^{\nround} \rangle \right] \leq \max_{\bm{q} \in \mathcal{Q}}\mathbb{E}\left[\sum_{\nround=1}^\Nround \langle \bm{q}^{\nround} - \tilde{\bm{q}}^{ \nround+1}, -\hat{\bm{w}}^{\nround} \rangle + \eta^{-1}D(\bm{q} || \bm{q}^{1}) \right]
    \end{align}
    Furthermore, the unconstrained optimizer can be solved with $\tilde{\bm{q}}^{ \nround+1} = \bm{q}^{\nround} \exp(\eta \hat{\bm{w}}^{\nround})$. Applying $\exp(x) \geq 1 + x$ for $x = \exp(\eta \hat{\bm{w}}^{\nround})$, we obtain $\tilde{\bm{q}}^{ \nround+1} \exp(\eta \hat{\bm{w}}^{\nround}) \geq \bm{q}^{\nround} + \eta \bm{q}^{\nround} \hat{\bm{w}}^{\nround}$. Plugging this back in:
    \begin{align}
        \textsc{Regret}_\mathcal{B}(\Nround) &\leq \mathbb{E}\left[\sum_{\nround=1}^\Nround \langle \bm{q}^{\nround} - \bm{q}^{\nround} \exp(\eta \hat{\bm{w}}^{\nround}), -\hat{\bm{w}}^{\nround} \rangle + \eta^{-1}D(\bm{q} || \bm{q}^{1}) \right]\\
        &= \mathbb{E}\left[\eta \sum_{\nround=1}^\Nround \sum_{\nitem = 1}^\Nitem \sum_{b \in \mathcal{B}} q^{\nround}_\nitem(b) \hat{w}^{\nround}_\nitem(b)^2 + \eta^{-1}D(\bm{q} || \bm{q}^{1}) \right] \label{eq: node diff}
    \end{align}
    Note that $\hat{w}^{\nround}_\nitem(b) = \frac{w_\nitem^\nround(b)}{q^{\nround-1}_{\nitem}(b)} \textbf{1}_{b = b^{\nround}_{\nitem}}$ for all $\nitem \in [\Nitem]$ and $b \in \mathcal{B}$ by definition. Since $w^{\nround}_\nitem(b) \leq 1$ and $\textbf{1}_{b = b^{\nround}_{\nitem}} \leq 1$ we have $\hat{w}^{\nround}_\nitem(b) \leq \frac{1}{q^{\nround}_\nitem(b)}$ and we continue the above chain of inequalities with:
    \begin{align}
        \textsc{Regret}_\mathcal{B}(\Nround) &\leq \mathbb{E}\left[\eta \sum_{\nround=1}^\Nround \sum_{\nitem = 1}^\Nitem \sum_{b \in \mathcal{B}} q^{\nround}_{\nitem-1}(b) \hat{w}^{\nround}_\nitem(b) \frac{1}{q^{\nround}_{\nitem}(b)}  + \eta^{-1}D(\bm{q} || q^{1}) \right] \label{eq: full info difference}\\
        &= \mathbb{E}\left[\eta \sum_{\nround=1}^\Nround \sum_{\nitem = 1}^\Nitem \sum_{b \in \mathcal{B}} \hat{w}^{\nround}_\nitem(b)  + \eta^{-1}D(\bm{q} || \bm{q}^{1}) \right]
    \end{align}
    Noting that $D(\bm{q} || \bm{q}') \leq \sum_{\nitem=1}^\Nitem \sum_{b \in \mathcal{B}} \left[-q_{\nitem}(b) \log q_\nitem(b) \right] = \sum_{\nitem=1}^\Nitem H(\bm{q}_\nitem(\cdot))$ and $\sum_{b \in \mathcal{B}} q_\nitem(b) = 1$ for any $q \in \mathcal{Q}$ and $\nitem \in [\Nitem]$, we have that $\bm{q}_\nitem(\cdot)$ is a valid probability mass function. Thus, its entropy $H(\bm{q}_\nitem(\cdot))$ is upper bounded by $O(\log |\mathcal{B}|)$ which is the entropy of the uniform distribution over $|\mathcal{B}|^2$ items. Hence, $D(\bm{q} || \bm{q}') \lesssim \Nitem \log |\mathcal{B}|$. Plugging this back in:
    \begin{align}
        \textsc{Regret}_\mathcal{B}(\Nround) &\lesssim \mathbb{E}\left[\eta \sum_{\nround=1}^\Nround \sum_{\nitem = 1}^\Nitem \sum_{b \in \mathcal{B}} \hat{w}^{\nround}_\nitem(b)  + \eta^{-1}\Nitem \log |\mathcal{B}| \right]\\
        &\leq \mathbb{E}\left[\eta \sum_{\nround=1}^\Nround \sum_{\nitem=1} \sum_{b \in \mathcal{B}} \hat{w}^{\nround}_\nitem(b) + \eta^{-1}\Nitem \log |\mathcal{B}| \right]\\
        &\leq \eta \sum_{\nround=1}^\Nround \sum_{\nitem=1} \sum_{b \in \mathcal{B}} w^{\nround}_\nitem(b) + \eta^{-1}\Nitem \log |\mathcal{B}|\\
        &= \eta \Nround \Nitem |\mathcal{B}| + \eta^{-1}\Nitem \log |\mathcal{B}|
    \end{align}
    Where in the last equality, we used the unbiasedness property of $\hat{\bm{w}}^{\nround}$. Setting $\eta = \sqrt{\frac{\log |\mathcal{B}|}{|\mathcal{B}|T}}$, we obtain $\textsc{Regret}_\mathcal{B}(\Nround) \leq \Nitem \sqrt{|\mathcal{B}| \Nround \log |\mathcal{B}|}$. We can also extend this algorithm to the full information setting. To handle the full information case, we note that we can improve line~\ref{eq: full info difference} by instead replacing $\hat{\bm{w}}^{\nround}$ with $\bm{w}^{\nround}$ in the previous line to obtain: 
    \begin{align}
        \sum_{\nround=1}^\Nround \sum_{\nitem=1}^\Nitem \sum_{b \in \mathcal{B}} q^{\nround}_\nitem(b) \hat{w}^{\nround}_\nitem(b)^2 &= \sum_{\nround=1}^\Nround \sum_{\nitem=1}^\Nitem \sum_{b \in \mathcal{B}} q^{\nround}_\nitem(b) w^{\nround}_\nitem(b)^2\\
        &\leq \sum_{\nround=1}^\Nround \sum_{\nitem=1}^\Nitem \sum_{b \in \mathcal{B}} q^{\nround}_\nitem(b) = \sum_{\nround=1}^\Nround \sum_{\nitem=1}^\Nitem 1 = \Nround \Nitem
    \end{align}
    Setting $\eta = \sqrt{\frac{ \log |\mathcal{B}|}{T}}$, we obtain in the full information setting $\textsc{Regret}_\mathcal{B}(\Nround) \leq \Nitem \sqrt{\Nround \log |\mathcal{B}|}$.
\end{proof}


One may wonder how to efficiently update the state occupancy measures by computing the minimizer of $\eta\langle \bm{q}, -\hat{\bm{w}}^\nround\rangle + D(\bm{q} || \bm{q}^{\nround-1})$. While we relegate the details to \rigel{Cite O-REPS paper here}, the idea is to first solve the unconstrained entropy regularized minimizer with $\tilde{\bm{q}}^{ \nround+1} = \bm{q}^{\nround} \exp(\eta \hat{\bm{w}}^{\nround})$. We then project this unconstrained minimizer to $\mathcal{Q}$ with:
\begin{align}
    \bm{q}^{\nround + 1} = \text{argmin}_{\bm{q} \in \mathcal{Q}} D(\bm{\rho}||\tilde{\bm{q}}^{\nround + 1})
\end{align}
Similar to how \rigel{Cite O-REPS} shows that this projection step (for state-action occupancy measures) can be solved as the solution to an unconstrained convex optimization problem in $\mathbb{R}^{|\mathcal{B}|^2}$, we have a slightly more complicated convex optimization problem over $\mathbb{R}^{|\mathcal{B}|}$ with a polynomial number of constraints as prescribed in Definition~\ref{def: Qspace}. Perhaps more simply, we can instead solve a modified bid optimization shortest path problem instance. In particular, we can set existing edge costs to 0, duplicate each layer with a single edge between same-bid pairs $b$ in row $\nitem$ and the duplicated row $\nitem$ with reward $w^\nround_\nitem(b)$ (see Figure~\ref{fig: modified_o_reps}). We can then invoke existing stochastic shortest path solvers such as Component Hedge or $\textsc{O-REPS}$ on this modified problem in order to obtain the state-action occupancy measure $\bm{\rho}^{\nround+1}$. Once we have computed $\bm{\rho}^{\nround+1}$, we can translate this into $\bm{\pi}^{\nround+1}$ or back into $\bm{q}^{\nround+1}$ and the corresponding $\bm{\psi}^{\nround + 1}$. Unfortunately, regardless of the optimization procedure used to update the policies, this projection step prevents the straightforward generalization of our method to handle non-stationary valuation profiles. 

Additionally, we comment on a regret lower bound. We know that the full information setting had a regret lower bound of $\Omega(\Nitem \sqrt{\Nround \log |\mathcal{B}|})$, which is linear in $\Nitem$ and a factor of $\sqrt{|\mathcal{B}|}$ off from the bandit setting upper bound. We note that it is difficult to construct

% Figure environment removed







\if 0

\newpage







Let $\Pi$ denote the set of all possible policies. With the above equalities and the obvious non-negativity constraints $\rho^\pi((\nitem, b), b') \geq 0$ for all $b' \leq b$, we can now define the set $\Delta$ of all possible state-action occupancy measures.
\begin{definition}
    Let $\Delta(\Pi)$ denote the set of all $\rho \in \mathcal{S} \times \mathcal{A} \to [0, 1]$ with the following properties:
    \begin{enumerate}
        \item $\sum_{B \in \mathcal{B}} \sum_{B' \leq B} \rho((\nitem, B), B') = 1$ for all $\nitem \in [\Nitem]$
        \item $\sum_{B \in \mathcal{B}} \rho(S_0, B) = 1$
        \item $\sum_{B' \leq B} \rho((\nitem, B), B') = \sum_{B" \geq B} \rho((\nitem-1, B"), B)$ for all $B \in \mathcal{B}, \nitem \in [\Nitem]$
        \item $\sum_{B' \leq B} \rho((1, B), B') = \rho(S_0, B)$
    \end{enumerate}
    This is equivalent to the condition that exists a policy $\pi \in \Pi$ such that $\rho((\nitem, B), B') = \prob_{\bm{B} \sim \pi}(s_\nitem = B_\nitem, a_\nitem = B_{\nitem+1})$ for all $B \geq B' \in \mathcal{B}$, $\nitem \in [\Nitem]$ and $\rho(S_0, B') = \prob_{\bm{B} \sim \pi}(s_0 = S_0, a_0 = B_1)$.
\end{definition}





We say that policy $\pi$ generates state-action occupancy measure $\rho$ if $\pi(s, a) = \frac{\rho(s, a)}{\sum_{b: \mathcal{A}(s)} \rho(s, b)}$ for all $s \in \mathcal{S}, a \in \mathcal{A}(s)$ where $\mathcal{A}(s)$ denotes the set of valid actions at state $s$. From the definition of $\Delta(\Pi)$, we can see that each $\rho$ uniquely determines $\pi$ and vice versa. We let $\pi^\rho$ denote the policy that generates the occupancy measure $\rho$. Hence, computing the regret minimizing policy $\pi^\rho \in \Pi$ is equivalent to minimizing the regret minimizing state-action occupancy measure $\rho$. Once computing $\rho$, we can compute the corresponding $\pi^\rho$ and then simulate a bid vector according to $\pi^\rho$. The key idea is that optimizing with respect to $\rho$, as opposed to $\pi$, can be framed as instance of online linear optimization. More specifically, we can compute the (expected) loss function at round $\nround$ as follows:
\begin{align}
    \mathbb{E}_{\bm{b} \sim \pi^\rho}\left[ \mu^{\nround}_n(\bm{b}) \right] = \mathbb{E}_{\bm{b} \sim \pi^\rho}\left[ \sum_{\nitem=1}^\Nitem w_\nitem^\nround(b_\nitem) \right] = \sum_{\nitem=1}^\Nitem \prob_{\bm{b} \sim \pi^\rho}(s_{\nitem-1} = b_{\nitem-1}, a_{\nitem-1} = b_{\nitem}) w_\nitem^\nround(b_\nitem)
\end{align}
Substituting in the definition of $\pi^\rho$ and the corresponding $\rho$, we have:
\begin{align}
    \mathbb{E}_{\bm{b} \sim \pi^\rho}\left[ \mu^{\nround}_n(\bm{b}) \right] = \sum_{\nitem=1}^\Nitem \sum_{b \in \mathcal{B}} \sum_{b' \leq b} \rho((\nitem-1, b), b') w_\nitem^\nround(b') = \langle \bm{\rho}, \bm{w}^\nround \rangle 
\end{align}
Where we assume $s_0 = S_0$ and define $\bm{w}^\nround \equiv \{w_\nitem^\nround(b')\}_{\nitem \in [\Nitem], b \geq b' \in \mathcal{B}}$. Assuming that the learner selects occupancy measure $\bm{\rho}^\nround$ at round $\nround$, the regret can then be written as:
\begin{align}
    \textsc{Regret}_\mathcal{B}(\Nround) &= \max_{\bm{b} \in \mathcal{B}^{+\Nitem}} \sum_{\nround=1}^\Nround \mu^\nround_n(\bm{b}) - \mathbb{E}_{\bm{b}^\nround \sim \bm{\rho}^\nround} \sum_{\nround=1}^\Nround \mu^\nround_n(\bm{b}^\nround)\\
    &\leq \max_{\bm{\rho} \in \Delta(\Pi)} \mathbb{E}_{\bm{b} \sim \bm{\rho}} \sum_{\nround=1}^\Nround \mu(\bm{b}) - \mathbb{E}_{\bm{b}^\nround \sim \bm{\rho}^\nround} \sum_{\nround=1}^\Nround \mu^\nround_n(\bm{b}^\nround)\\
    &= \max_{\bm{\rho} \in \Delta(\Pi)} \sum_{\nround=1}^\Nround \langle \bm{\rho} - \bm{\rho}^\nround, \bm{w}^\nround \rangle = \max_{\bm{\rho} \in \Delta(\Pi)} \sum_{\nround=1}^\Nround \langle  \bm{\rho}^\nround - \bm{\rho}, -\bm{w}^\nround \rangle
\end{align}
Where the inequality follows as any deterministic bid can be represented as an expectation over a deterministic policy. We additionally assume that that $\bm{\rho}^\nround$ is in $\Delta(\Pi)$ for all $\nround$. Notice that to keep consistent with the SSP literature, we negate both terms in the dot product to represent the problem as loss minimization rather than utility maximization. Defining $D(\bm{\rho} || \bm{\rho}') = \sum_{s \in \mathcal{S}, a \in \mathcal{A}(s)} \rho(s, a)\frac{\log \rho(s, a)}{\rho'(s, a)} - (\rho(s, a) - \rho'(s, a))$, we are now ready to state the $\textsc{O-REPS}$ algorithm formally and give the corresponding performance guarantees.

\begin{algorithm}[t]
	\KwIn{Valuation $\bm{v} \in [0, 1]^\Nitem$, Learning rate $\eta > 0$, Adaptive Adversarial Environment $\textsc{Env}^\nround: \mathcal{H}^\nround \to \mathcal{B}^{-\Nitem} \times \mathcal{B}$ where $\mathcal{H}^\nround$ denotes the set of all possible historical auction results $H^\nround$ up to round $\nround$ for all $\nround \in [\Nround]$.}
	\KwOut{The aggregate utility $\sum_{\nround=1}^\Nround \mu^\nround_n(\bm{b}^\nround)$.}
	$\pi^0(s, a) \gets \frac{1}{|\mathcal{A}(s)|}$ for all $s \in \mathcal{S}, a \in \mathcal{A}(s)$. Let $\bm{\rho}^0$ be the corresponding state-action occupancy measure \;
        $H^0 \gets \emptyset$ \;
	\For{$\nround \in [\Nround]$:}{
            $\bm{b}^{\nround}_- \gets \textsc{Env}^{\nround-1}(H^{\nround-1})$ and $\bm{b}^{\nround} \sim \bm{\pi}^{\nround-1}$\;
            Receive reward $\mu^\nround_n(\bm{b}^\nround) = \sum_{\nitem=1}^\Nitem w_\nitem^\nround(b^\nround_\nitem)$ and observe $w_\nitem^\nround(b^\nround_\nitem)$ where $w_\nitem^\nround(b)$ is as defined in Equation~\ref{def: def mu w W}\;            $\hat{w}_\nitem^\nround(B, B') \gets \frac{w_\nitem^\nround(B')}{\rho^{\nround-1}((\nitem-1, B), B')} \textbf{1}_{B = b^{\nround}_{\nitem-1},  B' = b^{\nround}_\nitem}$ for all $\nitem \in [\Nitem]$ and $B \geq B' \in \mathcal{B}$\;
            $\bm{\rho}^\nround \gets \text{argmin}_{\bm{\rho} \in \Delta(\Pi)} \eta\langle \bm{\rho}, -\hat{\bm{w}}^\nround\rangle + D(\bm{\rho} || \bm{\rho}^{\nround-1})$ and $\pi^\nround(s, a) \gets \frac{\rho^\nround(s, a)}{\sum_{b: \mathcal{A}(s)} \rho^\nround(s, b)}$\;
        }
        \textbf{Return} $\sum_{\nround=1}^\Nround \mu^\nround_n(\bm{b}^\nround)$
	\caption{\textsc{O-REPS}}
	\label{alg: O-REPS}
\end{algorithm}

\begin{theorem}
    Under bandit feedback, Algorithm~\ref{alg: O-REPS} achieves regret rate $\textsc{Regret}_\mathcal{B}(\Nround) \lesssim O(\Nitem |\mathcal{B}| \sqrt{T \log |\mathcal{B}|})$ using $\eta = |\mathcal{B}|^{-1}\sqrt{\frac{\log |\mathcal{B}|}{T}}$ with respect to the discretized benchmark.
\end{theorem}

\begin{proof}
    The proof follows directly from \rigel{Cite O-REPS paper}. At a high level, we want to bound the regret of Follow the (Entropy) Regularized leader. We first do this by upper bounding the regret by the regret of the unconstrained Be the (Entropy) Regularized leader (see Lemma 13 of \rigel{Cite Rakhlin 2009}. In particular, we define the corresponding unconstrained optimization problem:
    \begin{align}
        \tilde{\bm{\rho}}^{\nround+1} = \text{argmin}_{\bm{\rho} \in \mathcal{S} \times \mathcal{A} \to \mathbb{R}^+} \left( \eta \langle \bm{\rho}, -\hat{\bm{w}}^{\nround} \rangle + D(\bm{\rho} || \bm{\rho}^{\nround}) \right)
    \end{align}
    Where we are optimizing over all non-negative functions over state-action pairs rather than $\Delta(\Pi)$. Using the unbiasedness of our estimators $\hat{\bm{w}}^\nround$, we can replace $\bm{w}^\nround$ with $\hat{\bm{w}}^\nround$ in the definition of regret. Now, as it is shown in Lemma 13 of \rigel{Cite Rakhlin 2009}, we can upper bound the expected estimated regret as a function of the unconstrained optimizer $\tilde{\bm{\rho}}^{\nround+1}$ and the unregularized relative entropy with respect to the initial state-action occupancy measure $\bm{\rho}^1$. 
    \begin{align}
        \textsc{Regret}_\mathcal{B}(\Nround) = \max_{\bm{\rho} \in \Delta(\Pi)} \mathbb{E}\left[\sum_{\nround=1}^\Nround \langle \bm{\rho}^{\nround} - \bm{\rho}, -\hat{\bm{w}}^{\nround} \rangle \right] \leq \max_{\bm{\rho} \in \Delta(\Pi)}\mathbb{E}\left[\sum_{\nround=1}^\Nround \langle \bm{\rho}^{\nround} - \tilde{\bm{\rho}}^{ \nround+1}, -\hat{\bm{w}}^{\nround} \rangle + \eta^{-1}D(\bm{\rho} || \bm{\rho}^{1}) \right]
    \end{align}
    Furthermore, the unconstrained optimizer can be solved with $\tilde{\bm{\rho}}^{ \nround+1} = \bm{\rho}^{\nround} \exp(\eta \hat{\bm{w}}^{\nround})$. Applying $\exp(x) \geq 1 + x$ for $x = \exp(\eta \hat{\bm{w}}^{\nround})$, we obtain $\tilde{\bm{\rho}}^{ \nround+1} \exp(\eta \hat{\bm{w}}^{\nround}) \geq \bm{\rho}^{\nround} + \bm{\rho}^{\nround}\eta \hat{\bm{w}}^{\nround}$. Plugging this back in:
    \begin{align}
        \textsc{Regret}_\mathcal{B}(\Nround) &\leq \mathbb{E}\left[\sum_{\nround=1}^\Nround \langle \bm{\rho}^{\nround} - \bm{\rho}^{\nround} \exp(\eta \hat{\bm{w}}^{\nround}), -\hat{\bm{w}}^{\nround} \rangle + \eta^{-1}D(\bm{\rho} || \bm{\rho}^{1}) \right]\\
        &= \mathbb{E}\left[\eta \sum_{\nround=1}^\Nround \sum_{\nitem = 1}^\Nitem \sum_{b \geq b' \in \mathcal{B}} \rho^{\nround}((\nitem-1, b), b') \hat{w}^{\nround}_\nitem(b, b')^2 + \eta^{-1}D(\bm{\rho} || \bm{\rho}^{1}) \right] \label{eq: node diff}
    \end{align}
    Note that $\hat{w}^{\nround}_\nitem(b, b') = \frac{w_\nitem^\nround(b')}{\rho^{\nround-1}((\nitem-1, b), b')} \textbf{1}_{b = b^{\nround}_{\nitem-1},  b' = b^{\nround}_\nitem}$ for all $\nitem \in [\Nitem]$ and $b \geq b' \in \mathcal{B}$ by definition. Since $w^{\nround}_\nitem(b) \leq 1$ and $\textbf{1}_{b = b^{\nround}_{\nitem-1},  b' = b^{\nround}_\nitem} \leq 1$ we have $\hat{w}^{\nround}_\nitem(b, b') \leq \frac{1}{\rho^{\nround}_\nitem((\nitem-1, b), b')}$ and we continue the above chain of inequalities with:
    \begin{align}
        \textsc{Regret}_\mathcal{B}(\Nround) &\leq \mathbb{E}\left[\eta \sum_{\nround=1}^\Nround \sum_{\nitem = 1}^\Nitem \sum_{b \geq b' \in \mathcal{B}} \rho^{\nround}((\nitem-1, b), b') \hat{w}^{\nround}_\nitem(b, b') \frac{1}{\rho^{\nround}((\nitem-1, b), b')}  + \eta^{-1}D(\bm{\rho} || \rho^{1}) \right] \label{eq: full info difference}\\
        &= \mathbb{E}\left[\eta \sum_{\nround=1}^\Nround \sum_{\nitem = 1}^\Nitem \sum_{b \geq b' \in \mathcal{B}} \hat{w}^{\nround}_\nitem(b, b')  + \eta^{-1}D(\bm{\rho} || \bm{\rho}^{1}) \right]
    \end{align}
    Note that $D(\bm{\rho} || \bm{\rho}') \leq \sum_{\nitem=1}^\Nitem \sum_{b \geq b' \in \mathcal{B}} \left[-\rho((\nitem-1, b), b') \log \rho((\nitem-1, b), b') \right]$. Now, we use the fact that $\sum_{b \geq b' \in \mathcal{B}} \rho((\nitem-1, b), b') = 1$ for any $\rho \in \Delta(\Pi)$ and $\nitem \in [\Nitem]$, we have that $\rho((\nitem, \cdot), \cdot)$ is a valid probability mass function and has entropy upper bounded by $O(\log |\mathcal{B}|)$ which is the entropy of the uniform distribution over $|\mathcal{B}|^2$ items. Hence, $D(\bm{\rho} || \bm{\rho}') \lesssim \Nitem \log |\mathcal{B}|$. Plugging this back in:
    \begin{align}
        \textsc{Regret}_\mathcal{B}(\Nround) &\lesssim \mathbb{E}\left[\eta \sum_{\nround=1}^\Nround \sum_{\nitem = 1}^\Nitem \sum_{b \geq b' \in \mathcal{B}} \hat{w}^{\nround}_\nitem(b, b')  + \eta^{-1}\Nitem \log |\mathcal{B}| \right]\\
        &\leq \mathbb{E}\left[\eta \sum_{\nround=1}^\Nround \sum_{\nitem=1} \sum_{b \geq b' \in \mathcal{B}} \hat{w}^{\nround}_\nitem(b, b') + \eta^{-1}\Nitem \log |\mathcal{B}| \right]\\
        &\leq \eta \sum_{\nround=1}^\Nround \sum_{\nitem=1} \sum_{b \geq b' \in \mathcal{B}} w^{\nround}_\nitem(b, b') + \eta^{-1}\Nitem \log |\mathcal{B}|\\
        &= \eta \Nround \Nitem |\mathcal{B}|^2 + \eta^{-1}\Nitem \log |\mathcal{B}|
    \end{align}
    Where in the last equality, we used the unbiasedness property of $\hat{\bm{w}}^{\nround}$. Setting $\eta = |\mathcal{B}|^{-1}\sqrt{\frac{\log |\mathcal{B}|}{T}}$, we obtain $\textsc{Regret}_\mathcal{B}(\Nround) \leq \Nitem |\mathcal{B}| \sqrt{\Nround \log |\mathcal{B}|}$. To handle the full information case, we note that we can improve line~\ref{eq: full info difference} by instead replacing $\hat{\bm{w}}^{\nround}$ with $\bm{w}^{\nround}$ in the previous line to obtain: 
    \begin{align}
        \sum_{\nround=1}^\Nround \sum_{\nitem=1}^\Nitem \sum_{B \geq B' \in \mathcal{B}} \rho^{\nround}((\nitem-1, B), B') \hat{w}^{\nround}_\nitem(B)^2 &= \sum_{\nround=1}^\Nround \sum_{\nitem=1}^\Nitem \sum_{B \geq B' \in \mathcal{B}} \rho^{\nround}((\nitem-1, B), B') w^{\nround}_\nitem((\nitem-1, B), B')^2\\
        &\leq \sum_{\nround=1}^\Nround \sum_{\nitem=1}^\Nitem \sum_{B \geq B' \in \mathcal{B}} \rho^{\nround}_\nitem(B, B') = \sum_{\nround=1}^\Nround \sum_{\nitem=1}^\Nitem 1 = \Nround \Nitem
    \end{align}
    Setting $\eta = \sqrt{\frac{ \log |\mathcal{B}|}{T}}$, we obtain in the full information setting $\textsc{Regret}_\mathcal{B}(\Nround) \leq \Nitem \sqrt{\Nround \log |\mathcal{B}|}$.
\end{proof}

One may wonder how to efficiently update the state-action occupancy measures by computing the minimizer of $\eta\langle \bm{\rho}, -\hat{\bm{w}}^\nround\rangle + D(\bm{\rho} || \bm{\rho}^{\nround-1})$. While we relegate the details to \rigel{Cite O-REPS paper here}, the idea is to first solve the unconstrained entropy regularized minimizer with $\tilde{\bm{\rho}}^{ \nround+1} = \bm{\rho}^{\nround} \exp(\eta \hat{\bm{w}}^{\nround})$. We then project this unconstrained minimizer to $\Delta(\Pi)$ with:
\begin{align}
    \bm{\rho}^{\nround + 1} = \text{argmin}_{\bm{\rho} \in \Delta(\Pi)} D(\bm{\rho}||\tilde{\bm{\rho}}^{\nround + 1})
\end{align}
As shown in the analysis of \rigel{Cite O-REPS again}, this can be solved efficiently as an unconstrained convex optimization problem in $\mathbb{R}^{|\mathcal{B}|^2}$. Unfortunately, this projection prevents the straightforward generalization of our method to handle time varying valuation profiles.










\subsection{Reduction to State-Occupancy Measures}

One may realize that as the rewards associated with each edge are independent of its initial location, we may hope to further simplify the problem by assuming proportional conditional probability mass functions. That is, letting $b' \leq b$, we enforce that $\{\pi((\nitem, b), b")\}_{b" \leq b'} \propto \{\pi((\nitem, b'), b")\}_{b" \leq b'}$ for all $\nitem \in [\Nitem]$. We let $\Pi'$ denote the set of all $\Pi$ that fulfill this condition. This condition can be written succinctly as $\frac{\pi((\nitem, b), b')}{\pi((\nitem, b), b")} = \frac{\pi((\nitem, b'), b')}{\pi((\nitem, b'), b")}$ for all $\nitem$ and $b" \leq b' \leq b$. Letting $b_0$ denote the maximal possible bid, these conditions can be written even more concisely as $\bigcap_{\nitem=1}^\Nitem \bigcap_{b' \in \mathcal{B}} \bigcap_{b" \leq b'} \{\frac{\pi((\nitem, b_0), b')}{\pi((\nitem, b_0), b")} = \frac{\pi((\nitem, b'), b')}{\pi((\nitem, b'), b")}\}$. Using the fact that $\sum_{b' \leq b} \pi((\nitem, b), b') = 1$, we have:
\begin{align}
    &\frac{\pi((\nitem, b_0), b')}{\pi((\nitem, b_0), b")} = \frac{\pi((\nitem, b'), b')}{\pi((\nitem, b'), b")} \\
    &\leftrightarrow \pi((\nitem, b_0), b')\pi((\nitem, b'), b") = \pi((\nitem, b'), b')\pi((\nitem, b_0), b")\\
    &\to \sum_{b" \leq b'} \pi((\nitem, b_0), b')\pi((\nitem, b'), b") = \sum_{b" \leq b'} \pi((\nitem, b'), b')\pi((\nitem, b_0), b") \\
    &\leftrightarrow \pi((\nitem, b_0), b') = \pi((\nitem, b'), b') \sum_{b" \leq b'} \pi((\nitem, b_0), b")\\
    &\leftrightarrow \pi((\nitem, b'), b') = \frac{\pi((\nitem, b_0), b')}{\sum_{b" \leq b'} \pi((\nitem, b_0), b")}\\
    &\to \pi((\nitem, b), b') = \frac{\pi((\nitem, b_0), b')}{\sum_{b" \leq b} \pi((\nitem, b_0), b")}
\end{align}
Where in the last equality, we plugged in the value of $\pi((\nitem, b'), b')$ back into the proportionality constraint. At a high level, this allows us to describe the entire policy with only $\{\pi(S_0, b')\}_{b' \in \mathcal{B}} \cup \{\pi((\nitem, b_0), b')\}_{\nitem \in [\Nitem], b' \in \mathcal{B}}$, which is a set of size $O(\Nitem |\mathcal{B}|)$. Letting $\psi_\nitem(b) = \pi((\nitem-1, b_0), b)$, we see that we can rewrite $\pi((\nitem, b), b') = \frac{\psi(\nitem+1, b')}{\sum_{b" \leq b} \psi(\nitem+1, b")}$. Once again abusing notation, let $\bm{b} \sim \psi$ denote a bid vector sampled according to the policy generated from $\psi$. Furthermore, letting $\Psi: [\Nitem] \times \mathcal{B} \to [0, 1]$ such that $\sum_{b \in \mathcal{B}} \psi_\nitem(b) = 1$ denote the set of possible condensed policies $\psi$, we can recursively define the state occupancy measure $q^\psi(\nitem, b) = \prob_{\bm{b} \sim \psi}(s_\nitem = b_\nitem)$. With base case $q^\psi(1, b') = \psi(S_0, b')$, we have:
\begin{align}
    q^\psi(\nitem, b') = \sum_{b \geq b'} q^\psi(\nitem - 1, b)\pi((\nitem-1, b), b') = \psi_\nitem(b') \sum_{b \geq b'}  \frac{q^\psi(\nitem - 1, b)}{\sum_{b" \leq b} \psi_\nitem(b")}
\end{align}
By strong induction, it is straightforward to show that indeed $q^\psi(\nitem, b) = \prob_{\bm{b} \sim \psi}(s_\nitem = b_\nitem)$. We define $\mathcal{Q}$ to be the set of all possible state occupancy measures $\bm{q}$ such that there exists a $\psi \in \Psi$ that generates $\bm{q}$. More formally,
\begin{align}
    \mathcal{Q} \equiv \{\bm{q} \in [\Nitem] \times \mathcal{B} \to [0, 1]: \exists \psi \in \Psi \text{ such that } q^\psi(\nitem, b') = \psi_\nitem(b') \sum_{b \geq b'}  \frac{q^\psi(\nitem - 1, b)}{\sum_{b" \leq b} \psi_\nitem(b")} \}
    \label{def: Qspace}
\end{align}
As such, the loss can be rewritten as:
\begin{align}
    \mathbb{E}_{\bm{b} \sim \psi}\left[ \mu^{\nround}_n(\bm{b}) \right] = \mathbb{E}_{\bm{b} \sim \psi}\left[ \sum_{\nitem=1}^\Nitem w_\nitem^\nround(b_\nitem) \right] = \sum_{\nitem=1}^\Nitem \prob_{\bm{b} \sim \psi}(s_\nitem = b_\nitem) w_\nitem^\nround(b_\nitem)
\end{align}
Substituting in our definition $q^\psi(\nitem, b) = \prob_{\bm{b} \sim \psi}(s_\nitem = b_\nitem)$, we obtain:
\begin{align}
    \mathbb{E}_{\bm{b} \sim \psi}\left[ \mu^{\nround}_n(\bm{b}) \right] = \sum_{\nitem=1}^\Nitem \sum_{b \in \mathcal{B}} q^\nround(\nitem, b_\nitem) w_\nitem^\nround(b_\nitem) = \langle \bm{q}^\nround, \bm{w}^\nround\rangle
\end{align}
Where $\bm{w}^\nround = \{w_\nitem^\nround(b)\}_{\nitem \in [\Nitem], b \in \mathcal{B}}$ which is different as in the previous section where it was defined additionally over $b' \leq b$. Following a similar argument, we obtain:
\begin{align}
    \textsc{Regret}_\mathcal{b}(\Nround) \leq \max_{\bm{q} \in \mathcal{Q}} \sum_{\nround=1}^\Nround \langle  \bm{q}^\nround - \bm{q}, -\bm{w}^\nround \rangle
\end{align}
We can now repeat the same algorithm and analysis as in above, except using state occupancy measures $\bm{q}$ as opposed to state-action occupancy measures $\bm{\rho}$. The primary benefit is that in the regret analysis, we can replace the summation over $(\nitem, b, b')$ with a summation over $(\nitem, b)$. This allows us to obtain sharper regret bounds of $O(\Nitem \sqrt{|\mathcal{B}| \Nround \log |\mathcal{B}|})$. Note that in order to sample a monotone bid vector, we must first reconstruct the policy $\bm{\pi}$ corresponding to the recovered $\bm{q}$. To do this, we first compute $\bm{\psi}$ recursively corresponding to $\bm{q}$, and from this, compute $\bm{\pi}$ using $\pi((\nitem, b), b') = \frac{\psi(\nitem+1, b')}{\sum_{b" \leq b} \psi(\nitem+1, b")}$. To compute $\bm{\psi}$ from $\bm{q}$, notice that for any $\nitem$:
\begin{align}
    q(\nitem, b_0) = \psi_\nitem(b_0) \sum_{b \geq b_0} \frac{q(\nitem-1, b)}{\sum_{b" \leq b} \psi_\nitem(b")} = \psi_\nitem(b_0) \frac{q(\nitem-1, b_0)}{\sum_{b" \leq b_0} \psi_\nitem(b")} = \psi_\nitem(b_0) q(\nitem-1, b_0)
\end{align}
Where the last equality follows as $\sum_{b" \leq b_0} \psi_\nitem(b") = \sum_{b" \in \mathcal{B}} \psi_\nitem(b") = 1$. In the recursive case, we have:
\begin{align}
    q(\nitem, b') = \psi_\nitem(b')\sum_{b \geq b'} \frac{q(\nitem-1, b)}{\sum_{b" \leq b} \psi_\nitem(b")} = \psi_\nitem(b')\sum_{b \geq b'} \frac{q(\nitem-1, b)}{1 - \sum_{b" > b} \psi_\nitem(b")}
\end{align}
Hence, we can solve for $\psi_\nitem(b') = q(\nitem, b')\left(\sum_{b \geq b'} \frac{q(\nitem-1, b)}{1 - \sum_{b" > b} \psi_\nitem(b")}\right)^{-1} $ in terms of $\bm{q}$ (which is known) and $\psi_\nitem(b")$ for $b" \geq b'$, which is known from induction. A more serious complication is that the projection step is now more involved, requiring the proportional policy constraints. What we can do is first convert $\bm{\psi}^\nround$ into an equivalent $\bm{\pi}^\nround$ and $\bm{\rho}^\nround$. We then compute the maximizer w.r.t. all state-action occupancy measures in $\bm{\rho} \in \Delta(\Pi)$ with the additional constraint that $\bm{\rho}$ must have a corresponding $\bm{\pi}^\rho \in \Pi'$. As there are only polynomially many $O(\Nitem |\mathcal{B}|)$ constraints on $\pi$ required for proportionality, and that there are polynomially many $O(\Nitem |\mathcal{B}|)$ constraints required for some arbitrary $\rho \in \mathcal{S} \times \mathcal{A} \to [0, 1]$ to have a corresponding generating policy, then the projection step can still be solved efficiently. \rigel{This next part of the paragraph makes it seem like our approach is too straightforward} Perhaps more simply, we can instead solve a modified bid optimization shortest path problem instance. In particular, we can set each existing edge cost to 0, duplicate each layer with a single edge between same-bid pairs $b$ in row $\nitem$ and the duplicated row $\nitem$ with reward $w^\nround_\nitem(b)$ (see Figure. We can then invoke standard $\textsc{O-REPS}$ on this modified problem in order to obtain the policy $\bm{\rho}^{\nround+1}$ whilst maintaining the lower regret of $\textsc{Node O-REPS}$. Once we have computed $\bm{\rho}^{\nround+1}$, we translate this back into $\bm{q}^{\nround+1}$ and the corresponding $\bm{\psi}^{\nround + 1}$.

% Figure environment removed


\begin{algorithm}[t]
	\KwIn{Valuation $\bm{v} \in [0, 1]^\Nitem$, Learning rate $\eta > 0$, Adaptive Adversarial Environment $\textsc{Env}^\nround: \mathcal{H}^\nround \to \mathcal{B}^{-\Nitem} \times \mathcal{B}$ where $\mathcal{H}^\nround$ denotes the set of all possible historical auction results $H^\nround$ up to round $\nround$ for all $\nround \in [\Nround]$.}
	\KwOut{The aggregate utility $\sum_{\nround=1}^\Nround \mu(\bm{b}^\nround)$.}
	$\psi^0(\nitem, b) \gets \frac{1}{|\mathcal{B}|}$ for all $\nitem \in [\Nitem], b \in \mathcal{B}$. Let $\bm{q}^0$ be the corresponding state-action occupancy measure, where $q^\nround(\nitem, b)$ is the probability of selecting bid $b$ at state $\nitem$ \;
        $H^0 \gets \emptyset$ \;
	\For{$\nround \in [\Nround]$:}{
            $\bm{b}^{\nround}_-\gets \textsc{Env}^{\nround-1}(H^{\nround-1})$ and $\bm{b}^{\nround} \sim \bm{\pi}^{\nround-1}$ \rigel{Explain that we need to be more explicit about sampling from $\psi$ since it needs to have monotonicity, get rid of $\psi$}\;
            Receive reward $\mu^\nround_n(\bm{b}^\nround) = \sum_{\nitem=1}^\Nitem w_\nitem^\nround(b^\nround_\nitem)$ and observe $w_\nitem^\nround(b^\nround_\nitem)$ where $w_\nitem^\nround(b)$ is as defined in Equation~\ref{def: def mu w W}\;
            $\hat{w}_\nitem^\nround(b') \gets \frac{w_\nitem^\nround(b')}{q^{\nround-1}(\nitem, b')} \textbf{1}_{b' = b^{\nround}_\nitem}$ for all $\nitem \in [\Nitem]$ and $b' \in \mathcal{B}$\;
            $\bm{q}^\nround \gets \text{argmin}_{\bm{q} \in \mathcal{Q}} \eta\langle \bm{q}, -\hat{\bm{w}}^\nround\rangle + D(\bm{q} || \bm{q}^{\nround-1})$ where $\mathcal{Q}$ is as in Equation~\ref{def: Qspace} \;
            Recursively compute for all $\nitem$, $b$ in decreasing order $\psi^\nround_\nitem(b') \gets q^\nround(\nitem, b')\left(\sum_{b \geq b'} \frac{q^\nround(\nitem-1, b)}{1 - \sum_{b" > b} \psi^\nround_\nitem(b")}\right)^{-1}$ \rigel{Move the $\nitem$ in the subscript} \;
            Compute $\pi^\nround((\nitem, b), b') \gets \frac{\psi^\nround(\nitem+1, b')}{\sum_{b" \leq b} \psi^\nround(\nitem+1, b")}$ for all $\nitem \in [\Nitem], b \geq b' \in \mathcal{B}$ \rigel{Get rid of $\pi$ here and then just explicitly write $\frac{\psi^\nround(\nitem+1, b')}{\sum_{b" \leq b} \psi^\nround(\nitem+1, b")}$ in the sampling $\bm{b}$ step}\;
        }
        \textbf{Return} $\sum_{\nround=1}^\Nround \mu(\bm{b}^\nround)$
	\caption{\textsc{Node O-REPS}}
	\label{alg: Node O-REPS}
\end{algorithm}

\begin{theorem}
    Under bandit feedback, Algorithm~\ref{alg: Node O-REPS} achieves regret rate $\textsc{Regret}_\mathcal{B}(\Nround) \lesssim O(\Nitem \sqrt{|\mathcal{B}| \Nround \log |\mathcal{B}|})$ using $\eta = \sqrt{\frac{\log |\mathcal{B}|}{|\mathcal{B}|\Nround}}$.
\end{theorem}

\begin{proof}
    The justification follows that of Algorithm~\ref{alg: O-REPS} and only changes at line~\ref{eq: node diff}, where we instead have:
    \begin{align}
        \textsc{Regret}_\mathcal{B}(\Nround) \leq \mathbb{E}\left[\eta \sum_{\nround=1}^\Nround \sum_{\nitem = 1}^\Nitem \sum_{b \in \mathcal{B}} q^{\nround}(\nitem, b') \hat{w}^{\nround}_\nitem(b)^2 + \eta^{-1}D(\bm{q} || \bm{q}^{1}) \right]
    \end{align}
    Similarly, bounding the entropy term $D(\bm{q} || \bm{q}^1)$ only requires taking a sum over $(\nitem, b)$ pairs rather than $(\nitem, b, b')$. Nonetheless, since the entropy bound is logarithmic in the distribution size, we achieve the same bound of $O(\Nitem |\mathcal{B}|)$. Plugging this back in, we obtain:
    \begin{align}
        \textsc{Regret}_\mathcal{B}(\Nround) \leq \mathbb{E}\left[\eta \sum_{\nround=1}^\Nround \sum_{\nitem = 1}^\Nitem \sum_{b \in \mathcal{B}} q^{\nround}(\nitem, b') \hat{w}^{\nround}_\nitem(b)^2 + \eta^{-1}\Nitem |\mathcal{b}| \right]
    \end{align}
    We can upper bound the first term with $\eta \Nround \Nitem |\mathcal{B}|$ in the bandit setting and $\eta \Nround \Nitem $ in the full information setting using the same arguments as in $\textsc{O-REPS}$. Setting $\eta = \sqrt{\frac{\log |\mathcal{B}|}{|\mathcal{B}|\Nround}}$ yields the desired regret bound in the bandit setting. Similarly, setting $\eta = \sqrt{\frac{\log |\mathcal{B}|}{\Nround}}$ in the full information setting yields regret rate $O(\Nitem\sqrt{\Nround \log|\mathcal{B}|})$.
\end{proof}



\begin{theorem}
    Under bandit feedback, Algorithm~\ref{alg: Node O-REPS} achieves regret rate $\textsc{Regret}_\mathcal{B}(\Nround) \lesssim O(\Nitem \sqrt{|\mathcal{B}| \Nround \log |\mathcal{B}|})$ using $\eta = \sqrt{\frac{\log |\mathcal{B}|}{|\mathcal{B}|\Nround}}$.
\end{theorem}

\fi
\section{Online Learning Algorithms: Mirror Descent} 
\label{sec:bandit}


In this section, we propose our second online learning algorithm. Instead of mimicking the exponential weights algorithm, we reformulate the problem as online linear optimization over node probabilities in our DP graph. We solve this using OMD and construct a policy that sequentially samples bids based on these probabilities. We first present the regret analysis for the bandit setting, followed by the full information setting. We provide a single algorithm for both settings, with only one line changing based on the feedback structure. Our OMD algorithm is regret optimal in both feedback structures (up to a factor of $\sqrt{|\mathcal{B}|\log|\mathcal{B}|}$). However, it requires solving a convex optimization problem at each iteration, which may slow it down in practice. Nonetheless, this algorithm is preferred when prioritizing regret optimality over computational complexity.
\subsection{Algorithm Statement}



Recall that in the DP graph, we have $\Nitem$ layers, where in each layer there are $|\mathcal B|$ nodes and $|\mathcal B|^2$ edges. Given the structure of the DP graph, one idea is to  maintain some policy $\bm{\pi}: [\Nitem] \times \mathcal{B} \times \mathcal{B} \to [0, 1]$ which induces a family of probability measures $\bm{\rho}: [\Nitem] \times \mathcal{B} \times \mathcal{B} \to [0, 1]$ over the edges in the DP graph. In particular, let $\pi((\nitem, b), b') = \prob(b_{\nitem+1} = b' \mid b_\nitem = b)$ be the probability that the agent selects bid $b'$ for slot $\nitem+1$ conditional on having already selected bid $b$ for slot $\nitem$. Further, define $\rho((\nitem, b), b') = \prob(b_\nitem = b, b_{\nitem+1} = b')$ as  the unconditional probability that agent selects bids $b$ and $b'$ for slots $\nitem$ and $\nitem+1$, respectively. Following this idea, one can transform the bid optimization problem as an equivalent online linear optimization (OLO) problem over the space of possible $\bm{\rho}$, which we will show in the following section. However, this approach would lead to an algorithm with sub-optimal regret of $O(\Nitem |\mathcal{B}| \sqrt{\Nround \log |\mathcal{B}|})$ as it fails to capture the additional structure within the DP graph; cf. \citep{CMAB2013, PathKernel2003, OREPS2013}.
We show later that it is possible to improve this regret by a factor of $\sqrt{|\mathcal{B}|}$.

In this section, as our main contribution, instead of maintaining probability measures $\bm{\rho}$ over the edges in the DP graph, we maintain probability measures $\bm{q}$ over nodes. This idea is based on an important observation that, in the DP formulation, the weight of a path depends only on the nodes traversed and not the edges. In other words, regardless of the value of $b_\nitem$, selecting the edge from $(\nitem, b_\nitem)$ to $(\nitem+1, b_{\nitem+1})$ at round $t$ always yields the same utility $w_{\nitem+1}^\nround(b_{\nitem+1})$. We then construct some policy $\bm{\pi}$ that generates the desired node probability measures $\bm{q}$, where there may be many such choices of $\bm{\pi}$, though we argue the specific choice will not affect the regret. 
Consequently, we can reduce the higher dimensional problem of regret minimization over policies to the simpler one of regret minimization over node measures.



\textbf{Algorithm Summary.} Our algorithm (Algorithm \ref{alg: OMD}) consists of four steps. First, we recursively sample $\bm{b}^\nround$ according to the policy $\bm{\pi}^\nround$. Second, we compute node utility estimates $\widehat{w}_m^t(b)$, either as the true loss $w_m^t(b)$ in the full information setting or the inverse probability weighted version $\frac{w_m^t(b)}{q_m^{t-1}(b)}\textbf{1}_{b = b_{m}^t}$ in the bandit setting. Third, we optimize the negentropy-regularized expected estimated utility with respect to the probability measure over states $\bm{q}^\nround$, using OMD or Follow-the-Negentropy-Regularized-Leader updates. We show how this update can be efficiently computed by projecting the unconstrained optimizer of the regularized utility to the feasible space of $\bm{q}$, denoted by $\mathcal Q$, where the set $\mathcal Q$
    \begin{align}        \mathcal{Q} = \Big\{\bm{q} \in [0, 1]^{M \times |\mathcal{B}|}:  \sum_{b \in \mathcal{B}} q_m(b) = 1, \sum_{b \leq b'} q_{m+1}(b) \geq \sum_{b \leq b'} q_m(b) \forall b, b' \in \mathcal{B}, m \in [M]\Big\}\,. \label{eq:Q}
    \end{align}
consists of probability distributions over bids satisfying certain stochastic dominance constraints {which reflect the bid monotonicity constraints}. That is, under $\mathcal Q$, $\bm{q}_{\nitem+1}$ stochastically dominates $\bm{q}_\nitem$ for all $\nitem \in [\Nitem-1]$. 
Fourth, we convert $\bm{q}^\nround$ to a corresponding policy representation $\bm{\pi}^\nround$, ensuring that a feasible solution $\bm{\pi}^\nround$ exists as long as $\bm{q}^\nround \in \mathcal{Q}$.  
    

    
    We next discuss the main ideas and provide insights regarding our algorithm design.

 \begin{algorithm}[t]
 \footnotesize
	\KwIn{Learning rate $\eta > 0$, Valuation $\bm{v} \in [0, 1]^{+\Nitem}$ } 
	\KwOut{The aggregate utility $\sum_{\nround=1}^\Nround \mu_n^\nround(\bm{b}^\nround)$.}
	$\pi_0((m, b), b') \gets \frac{1}{|\{b" \in \mathcal{B}: b" \leq b, b" \leq v_{m+1}\}|}$ for all $\nitem \in [\Nitem], b \geq b' \in \mathcal{B}, b' \leq v_{m+1}$. Let $\bm{q}^0 \in [\Nitem] \times \mathcal{B} \to [0, 1]$ be the corresponding unit-bid value pair occupancy measure\;
	\For{$\nround \in [\Nround]$:}{
            \textbf{Determining the Bid Vector $\bm {b}^t$ recursively.} Set $b_1$ to $b \in \mathcal{B}$ with probability  $q^t_1(b)$\;
            \textbf{for} $m \in [1,\ldots,M-1], b \in \mathcal{B}: b_{m+1} \gets b$ with probability $\pi^t((m, b_\nitem), b)$\;
            Receive reward $\mu^\nround_n(\bm{b}^\nround) = \sum_{\nitem=1}^\Nitem w_\nitem^\nround(b^\nround_\nitem)$ and observe $w_\nitem^\nround(b^\nround_\nitem)$ where $w_\nitem^\nround(b) = (v_\nitem - b)\textbf{1}_{b \geq b^\nround_{-\nitem}}$\;
            \textbf{Update Reward Estimates}\;
            \textbf{for} $\nitem \in [\Nitem], b \in \mathcal{B}: \widehat{w}_\nitem^\nround(b) \gets \frac{w_\nitem^\nround(b)}{q^{\nround-1}_\nitem(b)} \textbf{1}_{b = b^{\nround}_\nitem}$ if \emph{Bandit Feedback}, $\widehat{w}_\nitem^\nround(b) \gets w_\nitem^\nround(b)$ if \emph{Full Information}\;
            \textbf{Determining Probability Measure $\bm{q}^t$ over any unit-bid value pair $(m, b)$} Set
            \[\bm{q}^\nround \gets \text{argmin}_{\bm{q} \in \mathcal{Q}} \eta\langle \bm{q}, -\widehat{\bm{w}}^\nround\rangle + D(\bm{q} || \bm{q}^{\nround-1})\,,\] where $\mathcal{Q}$ is as in Equation ~\eqref{eq:Q} and $D(\bm{q} || \bm{q}') = \sum_{\nitem \in [\Nitem], b \in \mathcal{B}} q_\nitem(b)\log \frac{q_\nitem(b)}{q'_\nitem(b)} - (q_\nitem(b) - q'_\nitem(b))$.
            
            \textbf{Convert $\bm{q}^t$ to Policy $\bm{\pi}^t$}\;
            Compute any feasible solution $\bm{\pi}^t$ to constraints $q^t_m(b) = \sum_{b' \geq b} q^t_m(b') \pi^t((m-1, b'), b)$ and $\sum_{b" \leq b} \pi^t((m, b), b") = 1$ for all $m \in [M], b \in \mathcal{B}$.
        }
        \textbf{Return $\sum_{\nround=1}^\Nround \mu_n^\nround(\bm{b}^\nround)$.} 
	\caption{\textsc{OMD - Bid Optimization in Multi-Unit Pay as Bid Auctions}}
	\label{alg: OMD}
\end{algorithm}

\textbf{Main Idea: Using the DP formulation to reduce our problem to online linear optimization.}  To design our algorithm, we observe that our DP formulation  allows us to reduce the bidding problem to OLO over the space of possible node probability measures $\bm{q}$. Of course, we must justify why it is reasonable for our optimization procedure to only consider node probability measures instead of the larger space of possible policies $\bm{\pi}$. Recall that the reward at round $\nround$ for bidding $\bm{b}$ is given by the sum of utilities of unit-bid values $\mu_n^\nround(\bm{b}) = \sum_{\nitem=1}^\Nitem w_\nitem^\nround(b_\nitem)$. We then take expectations over the bid vector $\bm{b}$, sampled from the following policy $\bm{\pi}$ which induces probability measures $\bm{q}_1,\ldots,\bm{q}_{\Nitem}$ over bid values; i.e., $q_m(b) = \prob_{\bm{b} \sim \bm{\pi}}(b_m = b)$: 
\begin{align}
    \mathbb{E}_{\bm{b} \sim \bm{\pi}}\left[ \mu^{\nround}_n(\bm{b}) \right] = \sum_{\nitem=1}^\Nitem \mathbb{E}_{\bm{b} \sim \bm{\pi}}\left[ w_\nitem^\nround(b_\nitem) \right] = \sum_{\nitem=1}^\Nitem \mathbb{E}_{b_\nitem \sim \bm{q}_\nitem}\left[ w_\nitem^\nround(b_\nitem) \right] = \sum_{\nitem=1}^\Nitem  \sum_{b \in \mathcal{B}} q_\nitem(b) w_\nitem^\nround(b) = \langle \bm{q}, \bm{w}^\nround \rangle\,.
    \label{eq: Loss of policy}
\end{align}
Here, the last term is an inner product over the space $[\Nitem] \times \mathcal{B}$, and in the first equation, we invoke the linearity of bid vector utilities on its unit-bid value utilities. The second equality is justified because we are taking an expectation over possible bid vectors $\bm{b}$, as the $\bm{q}_m$'s are by definition the  probabilities of selecting bid $b$ and unit $m$. This addresses the question of why we concern ourselves only with the node probability measures $\bm{q}$ when optimizing, as the regret depends only on $\bm{q}$, rather than the associated policy. In other words, for a fixed $\bm{q}$, any policy $\bm{\pi}$ that induces node probability measures $\bm{q}$ will yield the same expected utility. Intuitively, this reflects the fact that the utilities are associated with nodes and not edges in our DP graph. 

Letting $\bm{q}^\nround$ denote the probability measures induced by the (condensed) policy at round $\nround$, the instantaneous utility at round $\nround$ is given by $\langle \bm{q}, \bm{w}^\nround \rangle$. Seeing this inner product begs use of OLO algorithms. However, most OLO algorithms require convexity of the action space which is, in our setting, the space of possible $\bm{q}$. To show that this space is convex, we invoke the following lemma.

\begin{lemma}[$\mathcal{Q}$-Space Equivalence]
    \label{lem: QSpace Equivalence}
    Let \[\Pi = \Big\{{\pi} \in [0,1]^{M\times |\mathcal B|\times |\mathcal B|}: \pi((m, b), b') = 0 ~~\forall b' > b, m\in [M], \sum_{b' \leq b} \pi((m, b), b') = 1, m\in [M]\Big\}\] denote the space of policies on our DP graph. With a slight abuse of notation, for any $\pi\in  \Pi$, define 
    \[q(\pi) = \{\mathbf{q} \in [0, 1]^{M\times |\mathcal B|}: \forall b \in \mathcal{B},  q_1(b) =\pi((0, b_0), b), q_{m+1}(b) = \sum_{b' \in \mathcal B} q_m(b')\pi((m, b'), b), m\in[M-1]\}\,\] as the node probabilities induced by $\pi$. Here, $b_0= \max \mathcal B$. Let $\mathcal{Q}_{\Pi} = \cup_{\pi \in \Pi} q(\pi)$. Then,   $\mathcal{Q}_{\Pi}$ is equivalent to the set $\mathcal{Q}$ where $\mathcal{Q}$ is defined in Equation \eqref{eq:Q}. 
\end{lemma}

At a high level, the proof requires constructing a bijection between elements of $\mathcal{Q}_\Pi$ and $\mathcal{Q}$. Showing that $\bm{q} \in \mathcal{Q}_\Pi$ implies $\bm{q} \in \mathcal{Q}$ follows straightforwardly by applying the linear transform $q$ to the $\bm{\pi}$ associated with $\bm{q}$. The reverse direction requires a careful construction of a sequence of nested, non-empty subsets of $\Pi$ that satisfy the $q_{m+1}(b) = \sum_{b' \in \mathcal B} q_m(b')\pi((m, b'), b)$ constraints. 


Lemma \ref{lem: QSpace Equivalence} establishes that during the execution of Algorithm \ref{alg: OMD}, we can focus on the node probabilities in set $\mathcal{Q}$ without loss of generality. We recall that within $\mathcal{Q}$, the stochastic dominance conditions are enforced solely over node probabilities across layers. In other words, when determining $\mathbf{q}^t$ in Algorithm \ref{alg: OMD}, it is sufficient to consider the feasible set restricted to $\mathcal{Q}$, which is a convex set as  $\mathcal{Q}$ is a polyhedron.


Now, we argue that we only need to consider optimizing over $\mathcal{Q}$ as opposed to $\Pi$, as the regret can be rewritten strictly in terms of $\bm{q}$, independently of the corresponding $\bm{\pi}$.


 
 \begin{lemma}
    \label{lem: Online Linear Optimization}
     Any sequence of policies $\bm{\pi}^1,\ldots,\bm{\pi}^\Nround$ over our DP graph with associated node probability measures $\bm{q}^1,\ldots,\bm{q}^\Nround$ has discretized regret $\textsc{Regret}_{\mathcal{B}} = \max_{\bm{q} \in \mathcal{Q}} \sum_{\nround=1}^\Nround \langle \bm{q} - \bm{q}^\nround, \bm{w}^\nround\rangle$. Here, $\bm{w}^\nround = \{w^\nround_m(b)\}_{m \in [M], b \in \mathcal{B}}$ represents vector of the round $\nround$ rewards for all possible $(m, b)$ unit-bid value pairs.
 \end{lemma}

 

Having shown convexity of our action space and the mapping to an equivalent OLO problem, we are ready to state the key result for Algorithm~\ref{alg: OMD}, for the bandit setting.

\begin{theorem}[Online Mirror Descent: Bandit Feedback] \label{thm: OMD}    With $\eta = \Theta(\sqrt{\frac{\log |\mathcal{B}|}{|\mathcal{B}|\Nround}})$, Algorithm~\ref{alg: OMD} achieves (discretized) regret $O(\Nitem \sqrt{|\mathcal{B}| \Nround \log |\mathcal{B}|})$, with total time and space complexity polynomial in $\Nitem$, $|\mathcal{B}|$, and $\Nround$. Optimizing for discretization error from restricting the bid space to $\mathcal{B}$, we obtain a continuous regret of $O(\Nitem \Nround^{\frac{2}{3}})$. 
\end{theorem}

Under full information, we recover the regret bound of Algorithm \ref{alg: Decoupled Exponential Weights} by replacing the node weight estimates with the true weights.

\begin{corollary}[Online Mirror Descent: Full Information] \label{cor}
    With $\eta = \Theta(\sqrt{\frac{\log |\mathcal{B}|}{T}})$, Algorithm \ref{alg: OMD} achieves (discretized) regret $O(\Nitem \sqrt{ \Nround \log |\mathcal{B}|})$, with total time and space complexity polynomial in $\Nitem$, $|\mathcal{B}|$, and $\Nround$. Optimizing for discretization error from restricting the bid space to $\mathcal{B}$, we obtain a continuous regret of $O(\Nitem \sqrt{\Nround \log \Nround})$.
\end{corollary}





\section{Regret Lower Bound}

\label{sec: lower bound}

We remark that our OMD algorithms, under both full information and bandit settings, were designed to be robust to adversarial environments and incur discretized regret linear in $M$. In this section, we show that this is the best one can do, even in the stochastic setting. More specifically, we construct a corresponding (discretized) regret lower bound for our online bid optimization problem. At a high level, we will construct two bid vectors with nearly optimal expected utility under stochastic highest other bids. We derive the precise distribution of highest other bids and, using Le Cam's method, show that no algorithm in the full information or bandit feedback setting can learn the optimal bid vector quickly enough to avoid incurring $O(\Nitem \sqrt{\Nround})$ regret. 
\begin{theorem}\label{thm:lower}
    Under the full information setting, the discretized regret is lower bounded with $\textsc{Regret}_{\mathcal{B}} \in \Omega(\Nitem \sqrt{\Nround})$. This implies an equivalent regret lower bound in the bandit feedback setting.
\end{theorem}


We remark that our regret lower bound matches our upper bound for the OMD algorithm in the full information setting (up to a $\sqrt{\log |\mathcal{B}|}$ factor), as well as in the bandit setting up to a factor of $\sqrt{|\mathcal{B}| \log |\mathcal{B}|}$ factor.


                                    
   








\section{Experiments}
% \haizhou{Follow the same way of introduction as we did in Section2.}
% \noindent In this section, we will introduce datasets and experimental setups that we used. Then we evaluate our method, other self-supervised methods, and supervised methods under different distribution shifts (\ie, concept shifts and covariate shifts) under common settings (\ie, transductive, inductive settings). It has to note that we focus on node-level tasks (\eg, node classification) in this work. As for graph-level tasks, we leave it as our future work and some simple experiments can be found in Appendix~\ref{app:graph_classification}. 
In this section, we first introduce the experimental setup including datasets, training, and evaluation protocol in Section~\ref{sec:dataset}~and~\ref{sec:unsupervised}. 
% Next, we present our experimental setup and conduct extensive experiments to evaluate our method in Section~\ref{sec:unsupervised}. 
We then perform an ablation study to demonstrate the effectiveness of each proposed component in Section~\ref{sec:ablation}. 
Additionally, we analyze the impact of important hyper-parameters in Section~\ref{sec:sensitivity}. 
Subsequently, we integrate our method with various encoding models, showcasing the model-agnostic nature of our recipe in Section~\ref{sec:other_models}. 
Finally, we provide some qualitative results such as feature visualization in Section~\ref{sec:vis}.
It is important to note that we focus on node-level tasks (\eg, node classification) in this work. As for graph-level tasks, we leave it as our future work, while some simple experiments are also provided in Appendix~\ref{app:graph_classification}.

\subsection{Datasets}\label{sec:dataset}
There exist some benchmarks for evaluating graph out-of-distribution generalization~\cite{good,ji2022drugood,gds}. 
Among them, GOOD~\cite{good} is the most representative and comprehensive benchmark that curates more diverse graph datasets with diverse tasks, including single/multi-task graph classification, graph regression, and node classification involving more distribution shifts (\ie, concept shifts and covariate shifts). Hence in this work, we follow the evaluation protocol proposed in \cite{good}. Furthermore, we validate the effectiveness of our method in the datasets (\ie, Amazon-Photo, Elliptic) that are used in EERM~\cite{eerm}. The statistics and detailed introduction to these datasets can be found in Table~\ref{tab:dataset} and Appendix~\ref{app:datasets}.

\begin{table*}[htp]
\caption{The descriptions of datasets. ``Domain-Level'' means splitting by graphs, ``Time-Aware'' denotes splitting according to chronological order.``Word'' and ``Degree'' represent splitting according to word diversity and node degree respectively. ``Language'' means splitting by user language, suggesting the prediction should not be impacted by the language the user use. ``University'' denotes splitting according to the domain university, implying that the prediction of webpages should be based on word contents and link connections rather than university features. ``Color'' means that nodes are split according to node differences in covariate shift and color-label correlations in concept shift.}
\label{tab:dataset}
\centering
\begin{tabular}{cccccccc}
\toprule
Datasets     & Network Type        & \#Nodes & \#Edges & \#Attributes &\#Classes& Train/Val/Test Split     & Metric   \\
% Cora         & Artificial Transformation & 2,703   &         &              &         &                      & Accuracy \\
Amazon-Photo\footnotemark
             & Co-purchasing network      & 7,650   & 119,081   & 755          & 10      & Domain-Level         & Accuracy \\
Elliptic\footnotemark  
             & Bitcoin transactions       & 203,769 & 234,355   & 165          & 2       & Time-Aware           & F1-Score \\
GOOD-Cora    & Scientific publications    & 19,793  & 126,842   & 8,710         & 70      & Word/Degree          & Accuracy \\
% GOOD-Arxiv   & arXiv papers               & 169,343 & 2,315,598 & 128          & 40      & Time/Degree          & Accuracy \\
GOOD-Twitch  & Gamer network              & 34,120  & 892,346   & 128          & 2       & Language             & ROC-AUC  \\
GOOD-CBAS    & A BA-house graph           & 700     & 3,962     & 4             & 4       & Color                & Accuracy \\
GOOD-WebKB   & Webpage network            & 617     & 1,138     & 1,703         & 5       & University           & Accuracy \\
\bottomrule
\end{tabular}
\end{table*}
\footnotetext[5]{This dataset is adopted from~\cite{yang2016revisiting}. \cite{eerm} constructs ten graphs with different environment id’s for each graph.} 
\footnotetext[6]{The original is available on \hyperlink{https://www.kaggle.com/ellipticco/elliptic-data-set}{https://www.kaggle.com/ellipticco/elliptic-data-set}}

\subsection{Unsupervised Representation Learning}\label{sec:unsupervised}
\subsubsection{Transductive Setting}~\label{sec:trans}
% \noindent\textbf{Baselines.}\quad We conduct experiments with 12 baselines which consist of three categories: supervised methods and self-supervised generative methods, self-supervised contrastive methods. Specifically, we compare with three supervised baselines: empirical risk minimization~(ERM)~\cite{erm}, invariant risk minimization (IRM)~\cite{irm}, and a recent proposed graph OOD method dubbed EERM~\cite{eerm}. We also compare various unsupervised node-level representation learning methods: three self-supervised generative methods including GAE~\cite{gae}, VGAE~\cite{gae}, GraphMAE~\cite{gmae} and seven self-supervised contrastive methods: DGI~\cite{dgi}, MVGRL~\cite{mvgrl}, GRACE~\cite{grace}, RoSA~\cite{rosa}, BGRL~\cite{bgrl}, COSTA~\cite{costa}, SwAV~\cite{swav}. The descriptions of these methods can be found in Appendix~\ref{app:baselines}.
In this subsection, we focus on validating our proposed algorithm under the transductive setting, where the test nodes will participate in message passing~\cite{gilmer2017neural} during training following~\cite{good}. 

\noindent\textbf{Baselines.} We conduct experiments with 12 baselines from three categories: (i)~supervised methods, including empirical risk minimization~(\textbf{ERM})~\cite{erm}, invariant risk minimization (\textbf{IRM})~\cite{irm}, and a recent proposed graph OOD method \textbf{EERM}~\cite{eerm}; (ii)~self-supervised generative methods including Graph Autoencoder (\textbf{GAE})~\cite{gae}, Variational Graph Autoencoder (\textbf{VGAE})~\cite{gae}, Self-Supervised Masked Graph Autoencoders (\textbf{GraphMAE})~\cite{gmae}; (iii)~self-supervised contrastive methods including Deep Graph Infomax (\textbf{DGI})~\cite{dgi}, Contrastive Multi-View Representation Learning on Graphs (\textbf{MVGRL})~\cite{mvgrl}, Deep Graph Contrastive Representation Learning (\textbf{GRACE})~\cite{grace}, A Robust Self-Aligned Framework for Node-Node Graph Contrastive Learning (\textbf{RoSA})~\cite{rosa}, Bootstrapped Representation Learning on Graphs (\textbf{BGRL})~\cite{bgrl}, Covariance-Preserving Feature Augmentation for Graph Contrastive Learning (\textbf{COSTA})~\cite{costa}, Unsupervised Learning of Visual Features by Contrasting Cluster Assignments (\textbf{SwAV})~\cite{swav}. The detailed descriptions of these baselines can be found in Appendix~\ref{app:baselines}.

\noindent\textbf{Experimental setup.} We use the same graph encoder across different datasets for a fair comparison following~\cite{good}. We use grid search to find other hyper-parameters (\eg, learning rate, epochs) for different methods. For all experiments, we select the best checkpoints for ID and OOD tests according to results on ID and OOD validation sets following~\cite{good}, respectively. Experimental details and hyper-parameter selections are provided in Appendix~\ref{app:hyper}. For evaluating unsupervised methods, a linear classifier will be built on the frozen trained encoder after finishing pre-training. The reported results are the mean performance with standard deviation after 10 runs following~\cite{good}.

\noindent\textbf{Analysis.}\quad Based on the experimental results listed in Table~\ref{tab:trans_concept} and \ref{tab:trans_covariate}, we can draw the following conclusions: firstly, we find strong self-supervised methods (\eg, GRACE, BGRL, COSTA) are more robust to distribution shifts (concept shift in Table~\ref{tab:trans_concept} and covariate shift in Table~\ref{tab:trans_covariate}) compared to supervised methods. For instance, on GOOD-CBAS and GOOD-WebKB datasets, GRACE surpasses the best supervised method by large margins (over 6\% absolute improvement). Interestingly, we find the methods designed for OOD generalization (\ie, IRM) and graph OOD generalization (\ie, EERM) do not attain superior performance than the standard ERM on most of the datasets. For example, EERM shows superior OOD performance compared to ERM in only one experiment, and IRM outperforms ERM in four out of ten experiments across the conducted evaluations. This phenomenon is also observed in \cite{good,ahuja2020empirical,rosenfeld2021risks}, showcasing the challenge of achieving invariant prediction in non-Euclidean graph settings. 

Furthermore, our method surpasses other SOTA self-supervised methods on the OOD test set of all datasets by a considerable margin while achieving comparable performance in the in-distribution test set. For instance, on small datasets such as GOOD-CBAS and GOOD-WebKB, our method outperforms GRACE\footnote{MARIO is built up on GRACE according to our recipe. So, we make a comparison with GRACE here.} by over 2\% absolute accuracy on the OOD test set. On larger datasets such as GOOD-Cora and GOOD-Twitch, our method still outperforms other methods which shows its superiority. For instance, under covariate shift, MARIO surpasses other methods by over 7\% absolute accuracy on the GOOD-Twitch OOD test set. These statistics prove the effectiveness of our design.


\begin{table*}[htp]
\caption{Experimental results of all methods under concept shift. The bold font means the top-1 performance and the underline represents the second performance across the unsupervised methods. 'ID' represents in-distribution test performance and 'OOD' means out-of-distribution test performance. (OOM: out-of-memory on a GPU with 24GB memory)}
\label{tab:trans_concept}
\centering
\scalebox{0.95}{
\begin{tabular}{l|cc|cc|cc|cc|cc}
\toprule
\toprule
\multirow{3}{*}{concept shift} & \multicolumn{4}{c|}{GOOD-Cora}                   & \multicolumn{2}{c|}{GOOD-CBAS} & \multicolumn{2}{c|}{GOOD-Twitch} & \multicolumn{2}{c}{GOOD-WebKB} \\
                           & \multicolumn{2}{c}{word} & \multicolumn{2}{c|}{degree}& \multicolumn{2}{c|}{color}    & \multicolumn{2}{c|}{language}   & \multicolumn{2}{c}{university} \\
                           & ID         & OOD         & ID          & OOD          & ID            & OOD           & ID             & OOD            & ID            & OOD            \\
\midrule
ERM                        & 66.38±0.45 & 64.44±0.18  & 68.60±0.40  & 60.76±0.34   & 89.79±1.39    & 83.43±1.19    & 80.80±1.00     & 56.92±0.92     & 62.67±1.53    & 26.33±1.09     \\
IRM                        & 66.42±0.41 & 64.29±0.31  & 68.57±0.35  & 61.45±0.24   & 89.64±1.21    & 82.29±1.14    & 78.87±1.04     & 59.30±1.79     & 62.67±1.10    & 26.88±1.42     \\
EERM                       & 65.10±0.44 & 62.45±0.19  & 66.95±0.44  & 56.58±0.25   & 79.07±2.12    & 64.50±1.01    & OOM            & OOM            & 62.50±2.01    & 28.07±3.23      \\
\midrule
% Random-Init                & 37.53±1.74 & 32.12±1.24  & 37.82±1.71  & 27.74±1.14   &               &               &                &                & 60.33±2.21    & 27.07±1.70     \\
GAE                        & 60.65±0.89 & 58.00±0.55  & 62.59±1.11  & 53.44±0.80   & 75.28±1.36    & 68.07±2.05    & 81.25±0.81     & 51.51±1.05     & 62.17±3.34    & 25.78±1.85     \\
VGAE                       & 63.19±0.53 & 60.35±0.47  & 61.65±0.66  & 54.28±0.28   & 76.50±0.50    & 59.07±0.56    & 80.46±0.53     & 55.56±4.53     & 62.50±2.38    & 24.40±2.57     \\
GraphMAE                   & \underline{66.44±0.46} & \underline{64.87±0.30}  & 67.95±0.46  & 59.41±0.39   & 89.14±0.89    & 82.93±0.93    & 80.05±0.64     & 59.38±1.49     & 61.83±3.37    & 29.27±2.15     \\
DGI                        & 63.33±0.56 & 60.71±0.49  & 65.93±1.02  & 55.83±0.53   & 91.22±1.47    & 85.00±1.66    & 80.05±0.87     & 59.16±1.88     & 61.83±2.83    & 28.63±1.92      \\
MVGRL                      & OOM        & OOM         & OOM         & OOM          & 88.57±1.15    & 76.50±1.17    & OOM            & OOM            & 62.00±3.79    & 28.26±4.20     \\
GRACE                      & 65.61±0.61 & 63.92±0.44  & \textbf{68.59±0.35}  & 60.15±0.45   & 92.00±1.39    & 88.64±0.67    & \textbf{83.43±0.63}     & \underline{60.45±1.46}     & 64.00±3.43    & \underline{34.86±3.43}  \\
RoSA                       & 64.06±0.67 & 62.44±0.39  & 67.07±0.65  & 57.68±0.44   & 90.78±2.27    & 85.93±2.14    & 82.39±0.42     & 57.45±2.16     & 64.17±4.10    & 32.20±2.15     \\
BGRL                       & 65.18±0.43 & 63.43±0.45  & 66.83±0.80  & 59.63±0.38   & 92.36±1.16    & 87.14±1.60    & 82.52±0.60     & 55.48±1.48     & 63.67±2.33    & 31.47±3.43     \\
COSTA                      & 65.05±0.80 & 62.37±0.45  & 66.76±0.87  & 55.73±0.36   & \underline{93.50±2.62}    & \underline{89.29±3.11}    & 83.15±0.30 & 55.03±3.22     & 61.66±2.58    & 32.39±2.13 \\
% ArCL                       &            &             & 67.64±0.57  & 59.71±0.44   &               &               &                &                & 65.00±3.94    & 35.41±1.97 \\      
SwAV                       & 62.22±0.53 & 59.79±0.53  & 64.65±0.94  & 55.06±0.39   & 89.00±0.79    & 81.72±0.66    & \underline{83.32±0.15}     & 59.69±1.97     & \underline{65.17±3.76}    & 29.36±2.01    \\
\midrule
MARIO                       & \textbf{67.11±0.46} & \textbf{65.28±0.34}  & \underline{68.46±0.40}  & \textbf{61.30±0.28}   & \textbf{94.36±1.21}    & \textbf{91.28±1.10}    & 82.31±0.54     & \textbf{63.33±1.72}     & \textbf{65.67±2.81}    & \textbf{37.15±2.37}     \\
\bottomrule
\end{tabular}}
\end{table*}

\begin{table*}[htp]
\caption{Experimental results of all methods under covariate shift. The bold font means the top-1 performance and the underline represents the second performance across the unsupervised methods. 'ID' represents in-distribution test performance and 'OOD' means out-of-distribution test performance. (OOM: out-of-memory on a GPU with 24GB memory)}
\label{tab:trans_covariate}
\centering
\scalebox{0.95}{
\begin{tabular}{l|cc|cc|cc|cc|cc}
\toprule
\toprule
\multirow{3}{*}{covariate shift} & \multicolumn{4}{c|}{GOOD-Cora}                                   & \multicolumn{2}{c|}{GOOD-CBAS} & \multicolumn{2}{c|}{GOOD-Twitch} & \multicolumn{2}{c}{GOOD-WebKB} \\
                           & \multicolumn{2}{c}{word} & \multicolumn{2}{c|}{degree}& \multicolumn{2}{c|}{color}    & \multicolumn{2}{c|}{language}   & \multicolumn{2}{c}{university} \\
                           & ID         & OOD         & ID          & OOD          & ID            & OOD           & ID             & OOD            & ID            & OOD            \\
\midrule
ERM                        & 70.50±0.41 & 64.69±0.33  & 72.46±0.49  & 55.53±0.50   & 92.00±3.08    & 77.57±1.29    & 70.98±0.41     & 49.35±5.09     & 39.34±1.79    & 14.52±3.14   \\
IRM                        & 70.48±0.26 & 64.53±0.57  & 71.98±0.34  & 53.72±0.46   & 90.86±2.41    & 78.86±1.67    & 69.81±0.95     & 49.11±2.82     & 38.52±3.30    & 13.97±2.80     \\
EERM                       & OOM        & OOM         & OOM         & OOM          & 65.00±2.57    & 57.43±3.60    & OOM            & OOM            & 46.07±4.55    & 27.40±7.65     \\
\midrule
GAE                        & 56.63±0.79 & 48.93±0.93  & 66.30±0.88  & 34.01±0.87   & 73.00±2.16    & 60.86±3.01    & 67.24±1.23     & 47.65±2.49     & 45.08±6.32    & 28.02±6.29    \\
VGAE                       & 62.02±0.66 & 54.12±0.86  & 69.41±0.57  & 44.20±1.29   & 62.29±2.04    & 63.29±1.11    & 66.99±1.43     & \underline{50.48±4.58}     & 48.85±4.68    & 20.87±6.69     \\
GraphMAE                   & 68.14±0.43 & 64.00±0.33  & \textbf{73.36±0.56}  & 53.75±0.55   & 67.28±3.03    & 67.28±1.49    & 68.84±1.20     & 48.02±2.79     & 48.03±4.34    & 30.00±8.09     \\
DGI                        & 60.85±0.75 & 57.03±0.67  & 68.97±0.41  & 41.75±0.88   & 69.57±4.09    & 59.71±3.43    & 68.43±1.05     & 44.83±1.61     & 48.52±5.04    & 21.11±7.50     \\
MVGRL                      & OOM        & OOM         & OOM         & OOM          & 65.00±1.94    & 64.15±0.77    & OOM            & OOM           & \textbf{54.10±5.39}    & 16.59±6.51     \\
GRACE                      & \underline{68.77±0.33} & \underline{64.21±0.41}  & 72.69±0.34  & \underline{56.10±0.63}   & \underline{93.57±1.83}    & \underline{89.29±3.40}    & \underline{71.12±0.87} & 46.21±1.54 & 49.67±5.82    & 28.10±4.68    \\
RoSA                       & 68.19±0.56 & 62.48±0.61  & 71.04±0.62  & 52.72±0.79   & 84.71±4.14    &79.14±3.51     & 70.58±0.36     & 45.83±1.72     & 52.30±4.24    & \underline{34.24±7.92}     \\
BGRL                       & 67.23±0.43 & 61.33±0.36  & 72.11±0.39  & 49.15±0.73   & 89.00±2.56    & 79.86±3.29    & \textbf{71.43±0.53}     & 43.86±0.94     & 51.80±5.55    & 30.32±7.61    \\
COSTA                      & 65.28±0.60 & 60.33±0.53  & 70.65±0.62  & 54.03±0.28   & 92.29±1.59    & 82.71±2.74    & 69.29±1.37     & 49.07±2.13     & 50.49±3.01    & 29.84±4.75   \\
SwAV                       & 63.29±1.01 & 56.98±0.94  & 70.27±0.73  & 43.00±0.52   & 89.57±1.12    & 81.43±1.69    & 69.19±0.93     & 49.37±2.96     & 49.84±4.82    & 30.55±6.72   \\
\midrule
MARIO                       & \textbf{69.99±0.54} & \textbf{65.06±0.34}  & \underline{72.73±0.43}  & \textbf{57.73±0.45}  & \textbf{94.57±2.46}    & \textbf{91.00±2.48}     & 68.31±0.78 & \textbf{57.37±1.37}     & \underline{53.94±3.23}    & \textbf{35.24±4.98}   \\
\bottomrule
\end{tabular}}

\end{table*}

\subsubsection{Inductive Setting}
In this subsection, we conduct experiments under the inductive settings, where the test nodes are kept unseen during training. This setting is more suitable for domain generalization.
% But we think it is more convincing that conduct experiments under inductive settings which means test nodes are unseen during training. This setting is more appropriate for domain generalization.

\noindent\textbf{Baselines:} For GOOD-WebKB and GOOD-CBAS datasets, we adopt ERM, IRM, GraphMAE, and GRACE as our baselines. And for Amazon-Photo and Elliptic datasets, we select ERM, EERM, and GRACE as our baselines.

\noindent\textbf{Experimental setup:} For GOOD-WebKB and GOOD-CBAS datasets, we use the same model configuration in Section~\ref{sec:trans}.
% Besides, we add experiments on Amazon-Photo dataset~\cite{yang2016revisiting} and Elliptic~\cite{elliptic} dataset in this subsection. 
For Amazon-Photo dataset~\cite{yang2016revisiting} and Elliptic~\cite{elliptic} dataset, they consist of many snapshots (training data and testing data use different snapshots) which are naturally inductive. For Amazon-Photo dataset, we use 2-layer GCN~\cite{gcn} as the encoder and for elliptic dataset, we use 5-layer GraphSAGE~\cite{sage} as encoder following~\cite{eerm}.

% Figure environment removed

\noindent\textbf{Analysis:}
According to Figure~\ref{fig:amazon},\ref{fig:elliptic},\ref{fig:ind_con},\ref{fig:ind_cov}, we can draw following conclusions:
firstly, based on Figure~\ref{fig:amazon}, it is evident that our method outperforms other representative supervised and self-supervised methods on all test graphs (T1$\sim$T8). This superiority is reflected in the larger median value of our method compared to others. For instance, MARIO achieves over a 3\% absolute improvement compared to ERM in terms of the mean value of eight median values. Additionally, our method demonstrates higher stability across different random initializations, as indicated by the closer proximity of the first and third quartile values to the median value~(\eg, the difference of first and third quartile values of ERM, EERM, GRACE and MARIO are 4.2, 3.3, 6.7 and 1.0 on T8 respectively which indicates MARIO is much more stable than other methods). Furthermore, our method exhibits consistent performance across different graphs (\eg, The standard deviation of median values on T1$\sim$T8 for ERM, EERM, GRACE, and MARIO are 0.4, 1.1, 1.2, and 0.3, respectively.), indicating its robustness to environmental variations and its ability to extract invariant features: $g(G^e) \approx g(G^{e'})$ for all $e, e' \in \mathcal{E}^\text{train}$. In summary, our method showcases enhanced OOD generalization capabilities.
% $g(G^e)g(G^e^\prime)$ where $any e, e^\prime in \mathcal{E}^{train}$

Secondly, from the results presented in Figure~\ref{fig:elliptic}, we can observe that our method averagely harvests 10.9\% absolute improvement over GRACE and 12.5\% absolute improvement over EERM in terms of F1 scores on Elliptic dataset. This demonstrates the effectiveness of our method in handling distribution shifts and improving performance compared to existing approaches. It is worth noting that GRACE's performance worsens over time, indicating its inability to handle distribution shifts effectively. In contrast, our method consistently achieves better F1 scores, except for T9, which is caused by the dark market shutdown occurred after T7~\cite{elliptic}. The emergence of such an event introduces significant variations in data distributions, which subsequently results in performance degradation for all methods. Indeed, this event serves as an unpredictable external factor that introduces significant challenges for models trained on limited training data. The results indicate that the performance heavily depends on available training data. Nonetheless, our approach outperforms other methods even in such an extreme case. This highlights the effectiveness of our method in addressing distribution shifts and improving generalization performance.

Finally, based on the observations from Figure~\ref{fig:ind_con} and Figure~\ref{fig:ind_cov} MARIO demonstrates the best performances on both ID and OOD test sets for GOOD-WebKB and GOOD-CBAS datasets, under both concept shift and covariate shift. Notably, MARIO outperforms other methods by more than 3\% and 10\% absolute improvement on GOOD-WebKB and GOOD-CBAS, respectively, under covariate shift. We can draw similar conclusions as discussed in Section~\ref{sec:trans}. Even under the inductive setting, our method continues to demonstrate excellent OOD generalization capabilities and achieves comparable or even improved in-distribution test performance. These statistical results further validate the effectiveness of our method in handling distribution shifts and enhancing generalization performance.

Overall, the observations we have made provide strong evidence of the great capacity of our method for handling distribution shifts, validating its effectiveness and potential for real-world applications.



% Figure environment removed

% Figure environment removed


% Figure environment removed


\subsection{Ablation Studies}\label{sec:ablation}
\noindent Table~\ref{tab:aba} provides a detailed analysis of the effect of each component according to our proposed recipe for improving OOD generalization in graph contrastive learning. Let's examine the different variants of our method and their impact on performance.
Specifically, MARIO~(w/o ad) represents MARIO without  adversarial augmentation. MARIO~(w/o cmi) denotes we only maximize the mutual information between positive pairs without considering conditional mutual information. MARIO~(w/o cmi, ad) means a vanilla graph contrastive method that is similar to GRACE. 

From Table~\ref{tab:aba}, we can find MARIO~(w/o cmi) lags far behind MARIO on OOD test set which demonstrates appropriately minimizing the redundant information (\ie, conditional mutual information) is essential to improve OOD generalization of GCL methods. And adversarial augmentation can also boost OOD generalization because it can approximately serve as a supermum operator to learn more invariant features  discussed in Section~\ref{sec:aug}. Based on the analysis of these variants, it is evident that the proposed improvements on data augmentation and contrastive loss in the recipe are both effective in enhancing graph OOD generalization. Each component contributes to the overall performance improvement, and their combination leads to a stronger self-supervised graph learner in terms of graph OOD generalization. 

In short, the findings from Table~\ref{tab:aba} support the rationale behind your proposed recipe and provide empirical evidence of the effectiveness of each proposed component. By incorporating these enhancements, our method achieves superior performance in handling distribution shifts and improving graph OOD generalization in graph contrastive learning.
\begin{table*}[htp]
\caption{Ablation studies for MARIO by masking each component.}
\label{tab:aba}
\centering
\scalebox{0.9}{
\begin{tabular}{l|cc|cc|cc|cc|cc}
\toprule
\toprule
\multirow{3}{*}{concept shift} & \multicolumn{4}{c|}{GOOD-Cora}                       & \multicolumn{2}{c|}{GOOD-CBAS} & \multicolumn{2}{c|}{GOOD-Twitch} & \multicolumn{2}{c}{GOOD-WebKB} \\
                           & \multicolumn{2}{c}{word} & \multicolumn{2}{c|}{degree}& \multicolumn{2}{c|}{color}    & \multicolumn{2}{c|}{language}   & \multicolumn{2}{c}{university} \\
                           & ID         & OOD         & ID          & OOD          & ID            & OOD           & ID             & OOD            & ID            & OOD            \\
\midrule
MARIO                      & \textbf{67.11±0.46} & \textbf{65.28±0.34}  & \textbf{68.46±0.40}  & \textbf{61.30±0.28}      & \textbf{94.36±1.21}  & \textbf{91.28±1.10}    & 82.31±0.54     & \textbf{63.33±1.72}     & \textbf{65.67±2.81}    & \textbf{37.15±2.37}     \\
MARIO(w/o ad)              & 66.23±0.53 & 64.02±0.18  & 67.88±0.38  & 60.46±0.29   & 93.21±1.25    & 90.29±0.91    & 82.42±0.73     & 60.50±1.02     & 64.83±2.83    & 36.51±3.25    \\
MARIO(w/o cmi)             & 65.32±0.60 & 63.51±0.32  & 68.14±0.32  & 61.19±0.34   & 94.15±1.23    & 90.57±1.96    & \textbf{82.51±0.56}     & 61.41±2.63     & 64.50±4.35    & 35.78±2.53     \\
MARIO(w/o cmi, ad)         & 64.67±0.55 & 63.11±0.32  & 67.95±0.65  & 60.01±0.57   & 93.36±1.66    & 89.64±1.73    & 81.90±0.75     & 60.12±1.60     & 64.17±3.67    & 34.13±2.38     \\
\bottomrule
\end{tabular}}
\end{table*}
% & 65.32±0.60 & 63.51±0.32 exchange 64.67±0.55 & 63.11±0.32
% 68.14±0.32       id ood test: 60.95±0.43       ood ood test: 61.19±0.34


\subsection{Sensitivity Analysis}\label{sec:sensitivity}
\noindent In this subsection, we will analyze some important hyper-parameters of our method. We conduct sensitivity analysis on GOOD-WebKB dataset with concept shift, we chose two sensitive hyper-parameters (\ie, the coefficient $\gamma$ of condition mutual information in Equation~\ref{equ:cmi} and the number of prototypes $|C|$ in Equation~\ref{equ:pq}). The coefficient of CMI range in $[0.001, 0.01, 0.1, 0.5, 1]$ and the number of prototypes $|C|$ ranges in $[10, 50, 100, 200, 300]$. From Figure~\ref{fig:sensitivity}, we can observe that $\gamma$ reaches 0.1 and $|C|$ reaches 100 or 200 can achieve the best OOD test accuracy. Both higher and lower values of $\gamma$ result in suboptimal performance. This finding aligns with previous research such as DIB~\cite{dib}, indicating that an appropriate compression level is crucial for achieving optimal performance. Extremely high or low compression values are not ideal. 

Regarding the number of prototypes $|C|$, based on the results shown in Figure~\ref{fig:sensitivity}, it is found that setting $|C|=100$ leads to the best performance in terms of OOD test accuracy. This choice provides a moderate number of pseudo labels, which is beneficial for the learning process. 

Based on the sensitivity analysis, we determined that setting $\gamma=0.1$ and $|C|=100$ on most datasets. These hyperparameter values strike a balance between compression level and the number of prototypes, resulting in improved graph OOD generalization.
% Figure environment removed


\subsection{Integrated with Other Models}\label{sec:other_models}
% Figure environment removed

\begin{table}[htp]
\caption{Results of different learning approaches with different encoding models (\ie, GCN, GraphSAGE, GAT).}
\label{tab:others}
\centering
\scalebox{0.9}{
\begin{tabular}{cc|cc|cc}
\toprule
\toprule
\multirow{3}{*}{Model}& \multirow{3}{*}{Method} & \multicolumn{2}{c|}{GOOD-CBAS} & \multicolumn{2}{c}{GOOD-WebKB} \\
                & & \multicolumn{2}{c|}{color}    & \multicolumn{2}{c}{university} \\
                &   & ID          & OOD         & ID          & OOD            \\
\midrule
\multirow{3}{*}{GCN} 
&ERM               & 89.79±1.39 & 83.43±1.19  &  62.67±1.53 & 26.33±1.09         \\
&GRACE             & 92.00±1.39 & 88.64±0.67  &  64.00±3.43 & 34.86±3.43        \\
&MARIO             & 94.36±1.21 & 91.28±1.10  &  65.67±2.81 & 37.15±2.37        \\ \bottomrule
\multirow{3}{*}{SAGE} 
&ERM               & 95.07±1.51 & 75.14±1.19  & 73.67±2.08  & 46.33±3.42       \\
&GRACE             & 95.29±1.11 & 74.43±2.36  & 70.50±5.06  & 49.54±3.83        \\
&MARIO             & 96.00±1.07 & 76.29±3.01  & 71.00±3.82  & 51.74±4.63        \\ \bottomrule
\multirow{3}{*}{GAT} 
&ERM               & 78.64±3.63 & 72.93±2.64  & 61.33±3.71  & 28.99±2.63        \\
&GRACE             & 84.57±1.79 & 78.36±1.60  & 59.50±2.36  & 35.78±3.26        \\
&MARIO             & 84.93±1.95 & 80.43±1.89  & 62.17±4.78  & 38.17±3.10        \\
\bottomrule
\end{tabular}}
\end{table}



\noindent In the subsection, we demonstrate the model-agnostic nature of the recipe by integrating it with various graph neural network (GNN) models, including GCN, GraphSAGE, and GAT.

From Table~\ref{tab:others}, it can be observed that regardless of the specific GNN model used as the encoder, our method consistently achieves the best performance on the OOD test set. This indicates the effectiveness and robustness of our method across different GNN models.
By achieving superior performance across different GNN models, MARIO demonstrates its versatility and ability to improve the OOD generalization of various graph neural models. This highlights the broad applicability and effectiveness of our recipe in enhancing the performance of different GNN encoders.

Furthermore, we integrate our recipe with other GCL methods in Appendix~\ref{app:other_methods}. The results demonstrate our recipe can boost the OOD generalization ability of various GCL methods which means our recipe can serve as a plug-in for many current classical GCL methods.

% Figure environment removed

\subsection{Visualization}\label{sec:vis}
\subsubsection{Metric Score Curves}
We present metric score curves for ERM and MARIO, including training, ID validation, ID testing, OOD validation, and OOD testing accuracy, in Figure~\ref{fig:curve2}. Notably, MARIO demonstrates superior convergence with approximately 10\% absolute improvement on the OOD test set compared to ERM. Furthermore, MARIO effectively narrows the performance gap between in-distribution and out-of-distribution performance, showcasing its efficacy in enhancing OOD generalization for graph data. More metric score curves can be found in Appendix~\ref{app:curves}.


\subsubsection{Feature Visualization}
In order to assess the quality of learned embeddings, we adopt t-SNE~\cite{tsne} to visualize the node embedding on GOOD-Cora dataset (concept shift in word domain) using random-init of GCN, EERM, GRACE, and MARIO, where different classes have different colors in Figure~\ref{fig:vis}. For clarity, we select eight classes with the largest number of nodes to enhance the informativeness and interpretability of the visualization. We can observe that the 2D projection of node embeddings learned by MARIO has a better separation of clusters, which indicates the model can help learn representative features for downstream tasks. It has to note that we depict both ID nodes and OOD nodes in the same figure. 

Besides, we also separately visualize ID nodes and OOD nodes in the different figures in the Appendix~\ref{app:feature}. And we can find MARIO performs a clearer separation of clusters whether on ID nodes or OOD nodes compared to other methods.



%\section{Discussion}
\label{sec: discussion}
\kmsdelete{In this work} We study \kmsreplace{Fairness-Aware PAC learning}{Fair-ERM} in the malicious noise model, and  in some cases allow 
the learner to maintain optimal overall accuracy despite the signal in Group $B$ being almost entirely washed out.
%when we allow learners to use the
%$\PQ$ randomized expansion of the hypothesis class $\mathcal{H}$
In particular we show that different fairness constraints have fundamentally different behavior in the presence of Malicious Noise, in terms of the amount of accuracy loss that a given level of Malicious Noise could cause a fairness-constrained learner to incur. 
The key to achieving our results, which are more optimistic than those in \cite{lampert}, is allowing for improper learners using the (P,Q)-randomized expansions of the given class $\mathcal{H}$.
%We \kmsreplace{present a picture of the}{prove upper and lower bounds on}
%accuracy loss for a range of fairness notions, given \kmsreplace{this simple randomization step.}{learning over $\PQ$.
%In general our results indicate Fair-ERM (given learning over $\PQ$) is more robust than claimed in \cite{lampert}.
The type of smoothness we create by using $\PQ$ seems to be a natural property that is likely shared by many natural hypothesis classes.

Fairness notions are motivated as a response to learned disparities when there is \kmsdelete{data corruption or} systemic error affecting \kmsdelete{the data for}
one group. 
Fairness notions are supposed to mitigate this by ruling out classifiers that have worse performance on a sub-group. 
This can peg both classifiers at a lower level of performance \kmsdelete{(e.g that the lower subgroup)} in order to \emph{motivate} \cite{hardt16} improving the data collection or labelling process to obtain more reliable performance. 
%So in \kmsreplace{some}{a} sense, sensitivity of the fairness notion to poor sub-group performance caused by malicious noise is the \textit{point} of fairness constraints! 
However, it also desirable that fairness constraints perform gracefully when subject to Malicious Noise because fairness constraints will be used in contexts where the data is unreliable and noisy and this might not be known to the learner.
This tension, exposed by our work, motivates 
%a revisiting of fairness notions from first principles approach and trying to axiomatize the 
%desired properties of a fairness intervention a la cryptography and privacy. \footnote{Work in multi-calibration \cite{multicalib} is a viable direction for this problem but it is unclear how 
%that and related notions behave with unreliable data. }
on going work studying the sensitivity level of fairness constraints. 
%If we we are to take a view, if a classifier is deployed 

% %% -*- mode: LaTeX; fill-column: 78; -*-

\section{Concluding Remarks}
\label{sec:conclusions}

In this paper, we presented a novel SMC algorithm, \EventDPOR, tailored to the
characteristics of event-driven multi-threaded programs running under the SC
semantics. The algorithm was proven correct and optimal for event-driven
programs in which the variable accesses of events do not depend on how their
execution is interleaved with other threads.

We have implemented \EventDPOR in the \Nidhugg tool, and we will open-source
our implementation.
%
With a wide range of event-driven programs, we have shown that \EventDPOR
incurs only a moderate constant overhead over its baseline implementation
(\OptimalDPOR), it is exponentially faster than existing state-of-the-art SMC
algorithms in time and number of traces examined on programs where events'
actions do not conflict, and does not suffer from performance degradation
caused by having to examine
% a significant number of
non-serializable executions.
%
%% \bjcom{Should we include:
%% Moreover, in our benchmarks, also those that are not non-branching,
%% \EventDPOR explores only the optimal number of executions, and never
%% had to resort to a potentially expensive decision procedure.}

\EventDPOR assumes that handlers can process their events in arbitrary order.
Directions for future work include to retarget \EventDPOR for event-driven
programs with other policies (e.g., FIFO), and for specific event-driven
execution models.

\section{Concluding Remarks}

We have provided low-regret learning algorithms for PAB auctions in the full information and bandit settings with corresponding polynomial time and space complexities. In particular, we utilize our DP formulation and its equivalent graph representation to decouple the utility associated with bidding $b_\nitem = b$ for all $\nitem \in [\Nitem], b \in \mathcal{B}$. We derived two algorithms, one that mimics the exponential weights algorithm and another based on OMD, both of which allowed us to achieve polynomial (in $\Nitem$, $|\mathcal{B}|$, and $\Nround$) regret upper bounds, as well as time and space complexities, despite the combinatorially large bid space.

There are several intriguing avenues for future research that can be explored based on the current work. A promising direction is to leverage the structure induced by bid monotonicity in PAB auctions. 
Recent advancements in a simpler single-unit setting have demonstrated the efficacy of cross-learning between bids under certain feedback structures \citep{OptimalNoRegretFPA2020, LearningBidOptimallyAdversarialFPA2020}. It would be intriguing to investigate the potential benefits of applying cross-learning techniques  in our multi-unit setting. By incorporating such methods, we can explore whether they can enhance our regret bounds. Furthermore, inspired by our numerical results---where we show that the winning bids in PAB market dynamics converge to the same value---we can explore the design of online learning algorithms for the setting where bidders are restricted to a simplified bidding interface, wherein they are only allowed to submit a single  price and quantity for the units demanded rather than an entire vector of bids. 

%we do not utilize this additional side information and assume that agents only observe the utility corresponding to their submitted bid vector. It is unclear how to utilize the side information to construct meaningfully lower variance, unbiased utility estimates of either entire bid vectors or of individual slot-bid pairs. Under  the decoupled sampler algorithm for the full information setting, we have perfect cross learning between bids \cite{ContextBanditsCrossLearning2019}, the bandit setting makes Hedge style algorithms break down and it is difficult to make use of side information. In particular, our bandit setting algorithm did not utilize any cross learning over bids. Cross learning over bids by applying algorithms such as $\textsc{Exp3.G}$ or $\textsc{Exp3.SET}$ is difficult as the feedback graph is never revealed, i.e. the agent never observes the reserve nor the adversary's lowest winning bid and highest losing bid. Without this knowledge, the agent cannot construct unbiased utility estimates of bid vectors, let alone per-slot utility estimates. Conversely, cross learning over valuations is difficult as FTRL algorithms have an updating step which is difficult to decouple due to the projection step. This is one interesting future avenue of investigation.

%Regarding the difficulty of cross learning assuming both non-stationary valuations and adversarial adversary actions, we note that \cite{LearningBidOptimallyAdversarialFPA2020, OptimalNoRegretFPA2020} proved a related result regarding the impossibility of achieving sublinear (in $\Nround$) regret when the adversary's actions $\bm{b}^{\nround}_-$ are dependent on $\bm{v}$ in the bandit setting, regardless if these valuations were generated adversarially or stochastically. They give a simple example---that extends straightforwardly to the multi-unit setting---with guaranteed linear regret under any learning algorithm. They provide an efficient algorithm that utilizes graph feedback for an arm elimination algorithm in the case of $\Nitem = 1$. It is unknown whether their result can be extended to the case of higher dimensional $\bm{b}^{\nround}_-$ as in our setting. Interestingly, their results showed that it is possible to utilize cross learning without always observing the entire graph feedback structure, which many existing similar cross-learning algorithms require. In contrast to their adversarial valuation and i.i.d. environment assumption, we assumed adversarial environment and fixed valuation profile in the bandit setting. As such, an interesting open question is how to utilize cross learning to allow our algorithms to operate under the adversarial or non-stationary valuations.
\footnotesize{
\bibliographystyle{ACM-Reference-Format}
\bibliography{ref.bib}
}
\newpage
% \begin{APPENDICES}
\begin{comment}
\section{System Architecture}
\label{appendix:architecture}
\system has a novel modularized system architecture with three key components: 
\emph{StreamManager}, 
\emph{TxnManager} and \emph{TxnScheduler}. 
These components are instantiated in each thread locally.
The execution outline of \system is presented in Algorithm~\ref{alg:algo}.
Transactional stream processing is continuous and potentially never ends (Line 1$\sim$8).
The dependency resolution and execution of state transactions are separated into two non-overlapping phases by punctuations~\cite{Tucker:2003:EPS:776752.776780} (Line 2 and 5), which guarantees that no subsequent input event will have a smaller timestamp. 
Effectively, a batch of state transactions is collected during the first phase, and processed during the second phase.

In the first phase (i.e., stream processing phase), 
the \emph{StreamManager} conducts preprocessing for every input event ($e$). Similar to some prior works~\cite{tstream}, state transactions may be issued but not immediately processed during preprocessing (Line 3).
The \emph{pre\_processing} and \emph{post\_processing} functions are exposed as APIs to users.
The \emph{TxnManager} handles dependency resolution (Line 4) among state transactions and insert decomposed operations to construct a \tpg. We discuss the detailed two-phase \tpg construction process in Section~\ref{subsec:construction}.

In the second phase  (i.e., transaction processing phase), 
the \emph{TxnManager} is first involved again to refine (Line 6) the constructed \tpg with further dependency resolution.
The \emph{TxnScheduler} 
schedules operations for concurrent execution based on the constructed \tpg according to the three dimensions of scheduling decisions (Line 7). 
In particular, a scheduling decision model $M$ is instantiated based on the constructed \tpg (Line 14).
\textbf{\circled{1}} Guided by $M$, execution threads adopt an exploration strategy (Section~\ref{subsec:explore}) to explore the constructed \tpg for operations available to be scheduled constrained by dependencies. 
\textbf{\circled{2}} 
During exploration, one or multiple operations may be treated as the 
% basic 
unit of scheduling (Section~\ref{subsec:granularity}). 
Subsequently, \textbf{\circled{3}} every thread executes operation(s) in the unit of scheduling with various abort handling mechanisms (Section~\ref{subsec:abort_handling}).
Only when state transactions are processed (i.e., committed or aborted) can the associated input events be postprocessed (Line 8) by the \emph{StreamManager} based on transaction processing results.
\end{comment}

\begin{comment}
\begin{algorithm}
\footnotesize
    \KwData{$e$ \tcp{Input event}}
    \KwData{$txn_{ts}$ \tcp{State transaction}}
    \KwData{$G$ \tcp{The currently constructed TPG}}
    \While{!finish processing of input streams}{
        \eIf(\tcp*[h]{Phase 1}){\text{$e$ is not a $punctuation$}}{
                $txn_{ts}$ $\gets$ PRE\_Processing($e$)\;
                \textbf{TPG\_Construction}($G$, $txn_{ts}$)\; 
          }(\tcp*[h]{Phase 2}){
                \textbf{TPG\_Refinement}($G$)\; 
                \textbf{TXN\_Scheduling}($G$)\; 
                POST\_Processing()\;
          }
    }
    
    \SetKwFunction{FMain}{TPG\_Construction}
    \SetKwProg{Fn}{Function}{:}{}
    \Fn{\FMain{$G$, $txn_{ts}$}}{
        $O_{1..k}$ $\gets$ \textbf{Partition} $txn_{ts}$\;
        \ForEach{\text{operation $O_{i}$ $\in$ $O_{1..k}$}}{
            \textbf{Identify} its \ld\;
            $G$ $\gets$ $G$ + $O_{i}$ \;
        }
    }
    \SetKwFunction{FMain}{TPG\_Refinement}
    \SetKwProg{Fn}{Function}{:}{}
    \Fn{\FMain{$G$}}{
        \ForEach{\text{vertex $e_{i}$ $\in$ $G$}}{
            \textbf{Identify} its \td, \pd\;
        }
    }
    
    \SetKwFunction{FMain}{TXN\_Scheduling}
    \SetKwProg{Fn}{Function}{:}{}
    \Fn{\FMain{$G$}}{
        $M$ $\gets$ Instantiated with $G$;\tcp{A decision model}
        \While{!finish scheduling of $G$
        }{
          \textbf{\circled{2}} $Scheduling Unit$ $\gets$ \textbf{\circled{1}} \emph{Explore}($G$, $M$)\; 
            \textbf{\circled{3}} \emph{Execute with Abort Handling} ($Scheduling Unit$)\; 
        }
    }
  \caption{Execution Outline of \system}
  \label{alg:algo}
\end{algorithm}
\end{comment}
% \section{Switching times}

% \section{Supplementary materials}
% \end{APPENDICES}
\end{document}
