\section{Offline Setting}

Agent $n$ can think must construct a utility maximizing bid vector as a function of the complete auction history, which includes all competing bids $(\bm{b}^{\nround}_-)_{\nround \in [\Nround]}$ and reserves $(\pi^\nround)_{\nround \in [\Nround]}$. We additionally assume fixed valuation profiles; i.e. $\bm{v}^\nround = \bm{v}$ for all $\nround \in [\Nround]$. Recalling that the aggregate utility by bidding in all rounds with bid $\bm{b} \in \mathcal{B}^{+\Nitem}$ is given by $\sum_{\nround=1}^\Nround \mu^\nround_n(\bm{b})$ \rigel{Expand $\mu$ as a function of the $w$'s and $W$'s here so that it feels more natural as a solution}, we construct a dynamic programming solution that recovers the optimal bid vector as a function of the previous rounds' outcomes. We define for each $\nround \in [\Nround]$:
\[
w^{\nround}_\nitem(b) = \textbf{1}_{b > \max(b^{\nround}_{-\nitem}, \pi^\nround)} (v_\nitem - b) \quad \text{and} \quad W^{\Nround}_\nitem(b) = \sum_{\nround=1}^\Nround w^{\nround}_\nitem(b)
\]
Here, $w^\nround_\nitem(b)$ is the marginal utility of bidding $b$ in slot $\nitem$ at round $\nround$ and $W^\Nround_\nitem(b)$ the aggregate utility gained across auctions $\nround \in [\Nround]$ from the winning the $\nitem$'th item with bid $b$ respectively. We can construct our dynamic programming table(s):
\[
V^{\Nround}_\nitem(b) = W^{\Nround}_\nitem(b^*_\nitem(b)) + V^{\Nround}_{\nitem+1}(b^*_\nitem(b)) \quad \text{and} \quad U^{\Nround}_\nitem(b) = [b^{*}_\nitem(b), U^{\Nround}_{\nitem+1}(b^{*}_\nitem(b))]
\] 
Here, \[b^{*}_\nitem(b) = \text{argmax}_{b' \in \mathcal{B}; b' \leq b} W^{\Nround}_\nitem(b') + V^{\Nround}_{\nitem+1}(b').\] $V_\nitem^\Nround(b)$ denotes the optimal time-aggregate utility gained across slots $\nitem$ to $\Nitem$ restricting to bids of at most $b$, with base case $V^{\Nround}_\Nitem(B) = W^{\Nround}_\Nitem(b)$. Similarly, $U_\nitem^\Nround(b)$ is the corresponding optimal partial bid vector. The optimal utility is given by $V^{\Nround}_1(\max(\mathcal{B}))$. By using the recursive form for $V^{\Nround}_{\nitem}(b)$ and $U^{\Nround}_{\nitem}(b)$, computing tables $\bm{V}^{\Nround}$ and $\bm{U}^{\Nround}$ has time complexity of $O(\Nitem |\mathcal{B}|^2)$. As there are $\Nitem|\mathcal{B}|$ table entries, and each entry of $U^{\Nround}_\nitem$ is of length $O(\Nitem)$, the space complexity is $O(\Nitem^2|\mathcal{B}|)$. % If we are computing these tables for each bidder, then the time and space complexities increase by a factor of $N$. 

\begin{algorithm}[t]
	\KwIn{Competing bids, $(\bm{b}^{\nround}_-)_{\nround \in [\Nround]}$, Reserves $(\pi^\nround)_{\nround \in [\Nround]}$, and Valuations $\{\bm{v}^\nround\}_{\nround \in [\Nround]} = \bm{v}$}
	\KwOut{Optimal bid vector $\text{argmax}_{\bm{b} \in \mathcal{B}^{+\Nitem}} \sum_{\nround=1}^\Nround \mu^{\nround}_n(\bm{b})$ and its corresponding utility.}
	$W^{\Nround}_\nitem(b) \gets \sum_{\nround=1}^\Nround \textbf{1}_{b \geq \max(b^{ \nround}_{-\nitem}, \pi^\nround)} (v_\nitem - b) \quad \forall b \in \mathcal{B}, \nitem \in [\Nitem]$\;
        $V^{\Nround}_{\Nitem+1}(b) \gets 0$ and $U^{\Nround}_{\Nitem+1}(b) \gets [ \hspace{2mm} ] \quad  \forall b \in \mathcal{B}$\;
	\For{$\nitem \in \{\Nitem,\ldots,1\}$:}{
    	\For{$b \in \mathcal{B}$:}{
                $b^* \gets \text{argmax}_{b' \in \mathcal{B}; b' \leq b}W^{\Nround}_\nitem(b') + V^{\Nround}_{\nitem+1}(b') \quad \forall b \in \mathcal{B}$\;
                $V^{\Nround}_\nitem(b) \gets W^{\Nround}_\nitem(b^*) + V^{\Nround}_{\nitem+1}(b^*)$\;
                $U^{\Nround}_\nitem(b) \gets [b^*, U^{\Nround}_{\nitem+1}(b^*)]$\;
    	}
        }
        \textbf{Return} $U^{\Nround}_1(\max(\mathcal{B}))$, $V^{\Nround}_1(\max(\mathcal{B}))$
	\caption{\textsc{OfflineFull}$(\bm{v}, \{\bm{b}^{\nround}_-\}_{\nround \in [\Nround]}, \{\pi^\nround\}_{\nround \in [\Nround]})$}
	\label{alg: Offline Full}
\end{algorithm}

\begin{theorem}
    In the full information feedback setting, Algorithm~\ref{alg: Offline Full} returns the hindsight optimal bid vector with respect to competing bids, $(\bm{b}^{\nround}_-)_{\nround \in [\Nround]}$, reserves $(\pi^\nround)_{\nround \in [\Nround]}$, and valuations $\{\bm{v}^\nround\}_{\nround \in [\Nround]}$.
\end{theorem}

\begin{proof}
    We proceed by noting that $w^{\nround}_{\nitem}(b)$, the utility obtained in auction $\nround$ from winning item $\nitem$, is independent of the utility gained from any other items, conditional on the weakly monotonic bids condition. As the aggregate utilities between auctions are also independent, the time-aggregate optimal utilities $V^{\nround}_{\nitem}(b)$ can be constructed as a recursively as a function of the sum of the per-slot time aggregate utilities. To show that we can obtain the optimal utilities, we have:
    \begin{align}
        V^{\Nround}_\nitem(b) &= \max_{b_\nitem\geq\ldots\geq b_\Nitem; b_j \leq b \forall j \in [\nitem,\ldots,\Nitem]} \sum_{\nitem' = \nitem}^\Nitem W^{\Nround}_{\nitem'}(b_{\nitem'})\\
        &= \max_{b_{\nitem}\geq\ldots\geq b_\Nitem; b_j \leq b \forall j \in [\nitem,\ldots,\Nitem]} W^{\Nround}_{\nitem}(b_\nitem ) +  \sum_{\nitem' = \nitem+1}^\Nitem W^{\Nround}_{\nitem'}(b_{\nitem'})\\
        &= \max_{b' \in \mathcal{B}; b' \leq b} W^{\Nround}_\nitem(b') + V^{\Nround}_{\nitem+1}(b')
    \end{align}
    Where the last equality follows from the conditional independence between utilities obtained per slot. Since we have that $V^{\nround}_{\Nitem}(b ) = W^{\Nround}_{\Nitem} (B )$ from the base case, the optimality of $V^{\Nround}_\nitem(b)$ follows from induction.
\end{proof}

We note that the above algorithm also translates to the non stationary valuation setting by requiring input $\{\bm{v}^\nround\}_{\nround \in [\Nround]}$ and changing the first line to include $v^\nround_\nitem$ instead of $v_\nitem$. With this algorithm, we have established a method to obtain the hindsight optimal utility for which to gauge the efficacy of our online algorithms empirically.

\section{Online Setting}

Now we consider the problem of optimally learning how to bid in an online fashion. One obvious solution is applying exponential weights over the entire set of bid vectors, which has per-round rewards bounded in $[-\Nitem, \Nitem]$ (as $\sum_{\nitem=1}^\Nitem (v_\nitem - \max_{B \in \mathcal{B}} B) \geq -\Nitem$ and $\sum_{\nitem=1}^\Nitem (v_\nitem - \min_{B \in \mathcal{B}} B) \leq \Nitem$). While this achieves small regret $O(\Nitem \sqrt{\Nround \log |\mathcal{B}|})$ in the full information setting, the primary challenge here is that $\mathcal{B}^{+\Nitem}$ is exponentially large and naively tracking and updating these weights is expensive. We show how to sequentially select $(\nitem+1, B')$ given $(\nitem, B)$ for $B' \geq B$ to recover the entire bid vector $\bm{b}^{\nround}$.  % One way to circumvent this issue is by modeling the space of bids $\mathcal{B}^\Nitem$ as a layered graph and then applying ideas from Stochastic Shortest Path (SSP) problem. We will afterwards show that we can save a factor of $|\mathcal{B}|$ time and memory by representing the policy over edges in a lower dimensional space. 
An additional challenge is that we must be able to generalize across valuation profiles. Fortunately, as the impact of the valuations and bids on the utility is both additive and separable, we may utility full cross learning across valuations with no additional computational or memory overhead.

\subsection{Full Information Setting}

In exponential weights, the learner selects at round $\nround+1$ some bid vector $\bm{b}$ proportional to its $\eta$-exponentially weighted historical utility $\mu^{\nround}_n(\bm{b})$. Using our representation of $\mu^{\nround}_n(\bm{b})$ as a function of $W^\nround_\nitem(B)$ from the previous section, we have:
\begin{align}
    \prob_{EW(\eta)}^{\nround}(\bm{b}) = \frac{\exp(\eta \mu^{\nround}_n(\bm{b}))}{\sum_{\bm{b}' \in \mathcal{B}^{+\Nitem} \exp(\eta \mu^{\nround}_n(\bm{b}'))}} = \frac{\exp(\eta \sum_{\nitem=1}^\Nitem W_\nitem^\nround(b_\nitem))}{\sum_{\bm{b}' \in \mathcal{B}^{+\Nitem} \exp(\eta \sum_{\nitem=1}^\Nitem W_\nitem^\nround(b'_\nitem))}}
\end{align}
Since the denominator sums over an exponentially large set $\mathcal{B}^{+\Nitem}$, computing and storing these weights is difficult. Instead, we will recursively sample the values of $b_1,\ldots,b_\Nitem$. Let $S^\nround_1(B) = \sum_{\{\bm{b}: \bm{b} \in \mathcal{B}^\Nitem, b_1 = B\}} \exp(\eta \sum_{\nitem=1}^\Nitem W_\nitem^\nround(b_\nitem))$ denote the sum of exponentially weighted utilities over all bid vectors $\bm{b}$ such that $b_1 = B$. Then, we have that:
\begin{align}
    \sum_{\{\bm{b}: \bm{b} \in \mathcal{B}^\Nitem, b_1 = B\}} \prob_{EW(\eta)}^{\nround}(\bm{b}) = \frac{\sum_{\{\bm{b}: \bm{b} \in \mathcal{B}^\Nitem, b_1 = B\}} \exp(\eta \sum_{\nitem=1}^\Nitem W_\nitem^\nround(b_\nitem))}{\sum_{\{\bm{b}: \bm{b} \in \mathcal{B}^\Nitem\}} \exp(\eta \sum_{\nitem=1}^\Nitem W_\nitem^\nround(b_\nitem))} = \frac{S^\nround_1(B)}{\sum_{B' \in \mathcal{B}} S^\nround_1(B')}
\end{align}
More generally, let $S^\nround_\nitem(B) = \sum_{b_\nitem \geq \ldots \geq b_\Nitem, b_\nitem = B} \exp(\eta \sum_{\nitem=\nitem}^\Nitem W_\nitem^\nround(b_\nitem))$ denote the sum of exponentially weighted utilities corresponding to slots $\nitem$ through $\Nitem$ over all partial bid vectors satisfying $b_\nitem = B$. Then we can write $S^\nround_\nitem(B)$ recursively as:
\begin{align*}
    S^\nround_\nitem(B) = \sum_{b_\nitem \geq \ldots \geq b_\Nitem, b_\nitem = B} \exp(\eta \sum_{h=\nitem}^\Nitem W_h^\nround(b_h)) = \exp(\eta W_\nitem^\nround(B)) \sum_{b_{\nitem + 1} \geq \ldots \geq b_\Nitem, b_{\nitem+1} \leq B} \exp(\eta \sum_{\nitem=\nitem}^\Nitem W_\nitem^\nround(b_\nitem))
\end{align*}
Now summing over all possible values of $B'$, we obtain the desired recursion:
\begin{align*}
    S^\nround_\nitem(B) = \exp(\eta W_\nitem^\nround(B)) \sum_{B' \leq B} \sum_{b_{\nitem + 1} \geq \ldots \geq b_\Nitem, b_{\nitem+1} = B'} \exp(\eta \sum_{\nitem=\nitem}^\Nitem W_\nitem^\nround(b_\nitem)) = \exp(\eta W_\nitem^\nround(B)) \sum_{B' \leq B} S^\nround_{\nitem+1}(B')\\
\end{align*}
With the base case $S^\nround_\Nitem(B) = \exp(\eta W_\Nitem^\nround(B))$, we can recover all of the exponentially weighted partial utilities $S^\nround_{\nitem}(B)$ given $\bm{W}^\nround$. Once we have computed $S^\nround_{\nitem}(B)$, we can sample $\bm{b}$ according to its exponentially weighted utility $\exp(\eta \mu{n, \nround}(\bm{b})$ by sequentially sampling each $b_1,\ldots,b_\Nitem$.

\begin{algorithm}[t]
	\KwIn{Learning rate $\eta > 0$, Aggregate per-slot utilities $\bm{W}^\nround \equiv \{W_\nitem^\nround(B)\}_{\nitem \in [\Nitem], B \in \mathcal{B}}$.}
	\KwOut{Bid vector $\bm{b}$ sampled with probability $\exp(\eta \mu^{n, \nround}(\bm{b})$}
	$S^\nround_\Nitem(B) \gets \exp{(\eta W^\nround_{\Nitem}(B))}$ for all $B \in \mathcal{B}$ and $b_0 \gets \max_{B \in \mathcal{B}} B$\;
        \textbf{For} $\nitem \in [\Nitem-1,\ldots,1], B \in \mathcal{B}:$ \textbf{do} $S^\nround_\nitem(B) \gets \exp(\eta W_\nitem^\nround(B)) \sum_{B' \leq B} S^\nround_{\nitem + 1}(B')$\;
        \textbf{For} $\nitem \in [\Nitem]:$ \textbf{do} $b_\nitem \sim [S^\nround_\nitem(B)]_{B \in \mathcal{B}; B \leq b_{\nitem-1}}$\;
        \textbf{Return} $\bm{b} = (b_1,\ldots,b_\Nitem)$
	\caption{\textsc{DecoupledSampler}$(\bm{W}^\nround, \eta)$}
	\label{alg: Decoupled Sampler}
\end{algorithm}

\begin{theorem}
    Algorithm \ref{alg: Decoupled Sampler} samples $\bm{b}$ with probabilities equal to the exponential weights distribution $\exp(\eta \mu^{n, \nround}(\bm{b})$.
\end{theorem}

\begin{proof}
    Let $b_0 = B_0 = \max_{B \in \mathcal{B}} B$. We have that $\prob^\nround_{\textsc{DS}(\eta)}(\bm{B})$, the probability that our decoupled sampling Algorithm \ref{alg: Decoupled Sampler} returns bid vector $\bm{B} \in \mathcal{B}^\Nitem$, is given by:
    \begin{align}
        \prob^\nround_{\textsc{DS}(\eta)}(\bm{B}) &= \prob^\nround_{\textsc{DS}(\eta)}(b_1 = B_1) \prob^\nround_{\textsc{DS}(\eta)}(\bm{b}_{2:\Nitem} = \bm{B}_{2:\Nitem} \mid b_1 = B_1)\\
        &= \frac{S^\nround_1(B_1)}{\sum_{B \leq B_0} S^\nround_1(B)} \prob^\nround_{\textsc{DS}(\eta)}(\bm{b}_{2:\Nitem} = \bm{B}_{2:\Nitem} \mid b_1 = B_1)\\
        &= \frac{S^\nround_1(B_1)}{\sum_{B \leq B_0} S^\nround_1(B)} \prob^\nround_{\textsc{DS}(\eta)}(b_2 = B_2 \mid b_1 = B_1) \prob^\nround_{\textsc{DS}(\eta)}(\bm{b}_{3:\Nitem} = \bm{B}_{3:\Nitem} \mid b_1 = B_1, b_2 = B_2) \\
        &= \frac{S^\nround_1(B_1)}{\sum_{B \leq B_0} S^\nround_1(B)} \frac{S^\nround_2(B_2)}{\sum_{B \leq B_1} S^\nround_2(B)} \prob^\nround_{\textsc{DS}(\eta)}(\bm{b}_{3:\Nitem} = \bm{B}_{3:\Nitem} \mid b_1 = B_1, b_2 = B_2) \\
        &= \frac{S^\nround_1(B_1)}{\sum_{B \leq B_0} S^\nround_1(B)} \frac{S^\nround_2(B_2)}{\sum_{B \leq B_1} S^\nround_2(B)} \prob(\bm{b}_{3:\Nitem} = \bm{B}_{3:\Nitem} \mid b_2 = B_2) \ldots
    \end{align}
    Continuing the above expansion and then applying the definition of $S^\nround_\nitem(B)$, we obtain:
    \begin{align}
        \prob^\nround_{\textsc{DS}(\eta)}(\bm{B}) = \prod_{\nitem=1}^\Nitem \frac{S^\nround_\nitem(B_\nitem)}{\sum_{B \leq B_{\nitem-1}} S^\nround_\nitem(B)} = \prod_{\nitem=1}^\Nitem \frac{\exp(\eta W_\nitem^\nround(B_\nitem)) \sum_{B \leq B_{\nitem}} S^\nround_{\nitem+1}(B)}{\sum_{B \leq B_{\nitem-1}} S^\nround_\nitem(B)} = \prod_{\nitem=1}^\Nitem \exp(\eta W_\nitem^\nround(B_\nitem))
    \end{align}
    Here, the final term simplifies to $\prob_{EW(\eta)}^{\nround}(\bm{B}) = \exp(\eta \mu^{ \nround}_n(\bm{B}))$ which is precisely the exponentially weighted utility associated with bid vector $\bm{B}$.
\end{proof}

Note that this decomposition is similar to the loop-free path kernels method for SSP as detailed in \rigel{Cite Takimoto Path kernels/multiplicative weights paper}. The primary difference is that we consider weights over nodes rather than over edges, as in our setting, the reward associated with selecting bid $B'$ in slot $\nitem+1$ is independent of selecting bid $B \geq B'$ in slot $\nitem$. By doing so, we save a factor of $|\mathcal{B}|$ time and space as we store and update weights corresponding to $O(\Nitem |\mathcal{B}|)$ possible $(\nitem, B)$ slot-bid pairs rather than $O(\Nitem |\mathcal{B}|^2)$ possible $(\nitem, B, B')$ slot-bid-next bid triplets. Now, we state the full decoupled exponential weights algorithm and its associated regret, run-time, and space complexity.

\begin{algorithm}[t]
	\KwIn{Learning rate $\eta > 0$, Adaptive Adversarial Environment $\textsc{Env}^\nround: \mathcal{H}^\nround \to [0, 1]^\Nitem \times \mathcal{B}^{-\Nitem} \times \mathcal{B}$ where $\mathcal{H}^\nround$ denotes the set of all possible historical auction results $H^\nround$ up to round $\nround$ for all $\nround \in [\Nround]$.}
	\KwOut{The aggregate utility $\sum_{\nround=1}^\Nround \mu^\nround_n(\bm{b}^{\nround})$ corresponding to a sequence of bid vectors $\bm{b}^{1},\ldots,\bm{b}^{\Nround}$ sampled according to the exponential weights algorithm.}
	$W_\nitem^0(B) \gets 0, \textsc{NumberWon}_\nitem^0(B) \gets 0$ for all $\nitem \in [\Nitem], B \in \mathcal{B}$\;
        $H^0 \gets \emptyset$\;
	\For{$\nround \in [\Nround]$:}{
            $(\bm{v}^{n, \nround}, \bm{b}^{\nround}_-, \pi^\nround) \gets \textsc{Env}^{\nround-1}(H^{\nround-1})$ and $\bm{b}^{n, \nround} \gets \textsc{DecoupledSampler}(\bm{W}^{\nround-1}, \eta)$\;
            Observe $\bm{v}^{n, \nround}$ and update utilities $\bm{W}^{\nround - 1} \gets \{W^{\nround-1}_{\nitem}(B) = \textsc{NumberWon}_\nitem^{\nround-1}(B) (v^{n, \nround}_\nitem - B)\}_{\nitem \in [\Nitem], B \in \mathcal{B}}$\;
            Observe $\bm{b}^{\nround}_-, \pi^\nround$ and receive reward $\mu(\bm{v}^{n, \nround}, \bm{b}^{n, \nround}, \bm{b}^{\nround}_-, \pi^\nround)$\;
            $\textsc{NumberWon}_\nitem^\nround(B) \gets \textsc{NumberWon}_\nitem^{\nround-1}(B) + \textbf{1}_{B > \max(b^{-n, \nround}_\nitem, \pi^\nround)}$ for all $\nitem \in [\Nitem], B \in \mathcal{B}$\;
        }
        \textbf{Return} $\sum_{\nround=1}^\Nround \mu(\bm{v}^{n, \nround}, \bm{b}^{n, \nround}, \bm{b}^{\nround}_-, \pi^\nround)$
	\caption{\textsc{Decoupled Exponential Weights - Full Information}}
	\label{alg: Decoupled Exponential Weights}
\end{algorithm}

\begin{theorem}
    With $\eta \propto \sqrt{\frac{\log |\mathcal{B}|}{T}}$, Algorithm \ref{alg: Decoupled Exponential Weights} achieves regret $O(\Nitem \sqrt{ \Nround \log |\mathcal{B}|})$, with total time complexity $O(\Nitem |\mathcal{B}| \Nround)$ and space complexity $O(\Nitem |\mathcal{B}|)$.
\end{theorem}

\begin{proof}
    +As the rewards are bounded between $-\Nitem$ and $\Nitem$ and the state space is of size $O(|\mathcal{B}|^\Nitem)$, the exponential weights algorithm guarantees a regret upper bound $O(\eta \Nitem \Nround + \frac{\Nitem \log |\mathcal{B}|}{\eta})$ which achieves the desired regret bound with the state choice of $\eta$. Note our procedure works for non-stationary, potentially adversarially selected valuations as we have full cross-learning across all possible valuation profiles $\bm{v}^{\nround}$. More specifically, given $\bm{b}^\nround_-$ and $\pi^\nround$, the utility associated with any bid vector and valuation profile pair can be computed exactly. To do this efficiently, we decouple the utility per slot-bid value pair by tracking the number of items that would have been won across $\nround$ rounds by bidding bid $b$ at slot $\nitem$:
    \begin{align}
        W_\nitem^\nround(b) = \sum_{\tau=1}^\nround w_\nitem^\tau(b) = \sum_{\tau=1}^\nround \textbf{1}_{b \geq \max{b_{-\nitem}^{\tau}, \pi^\tau}} (v_\nitem^{\nround} - b) = \textsc{NumberWon}_\nitem^\nround(b)(v_\nitem^{\nround} - b) 
    \end{align}
    For the complexity analysis, updating and storing $\bm{W}^\nround$ and $\{\textsc{NumberWon}_\nitem^\nround(B)\}_{\nitem \in [\Nitem], B \in \mathcal{B}}$ at each $\nround$ requires $O(\Nitem |\mathcal{B}|)$ time and space. Similarly, each call to $\textsc{DecoupledSampler}$ requires computing $\{S^\nround_\nitem(B)\}_{\nitem \in [\Nitem], B \in \mathcal{B}}$, which can be done recursively in $O(\Nitem |\mathcal{B}|)$ time and space complexity (we save an additional factor of $|\mathcal{B}|$ by intermittently storing the values of $\sum_{B' \leq B} S^\nround_{\nitem+1}(B')$). Discarding old tables, the total time and space complexities of Algorithm \ref{alg: Decoupled Exponential Weights} are $O(\Nitem |\mathcal{B}| \Nround)$ and $O(\Nitem |\mathcal{B}|)$ respectively. 
\end{proof}

We remark that this algorithm works for any adversarially selected valuation profiles as the full information setting allows for perfect cross learning between valuations. While this is also true in the bandit setting, we will see that Hedge style algorithms break down.