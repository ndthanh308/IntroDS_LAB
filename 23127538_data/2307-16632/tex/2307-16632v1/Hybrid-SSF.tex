\documentclass[prb, twocolumn, longbibliography]{revtex4-2}
\usepackage{graphicx}
\usepackage{amsmath}
\usepackage{amssymb}
\usepackage{color}
\usepackage{appendix}
\usepackage{braket}
\usepackage{bbm}
\usepackage{subfigure}
\usepackage{mathrsfs}
\newcommand{\loc}{\mathrm{loc}}
\renewcommand{\k}{\boldsymbol{k}}
\renewcommand{\r}{\boldsymbol{r}}
\newcommand{\BS}{\mathrm{BS}}
\usepackage[colorlinks,urlcolor=blue,citecolor=blue,linkcolor=blue]{hyperref}
\usepackage{lipsum}
\usepackage{diagbox}




\begin{document}
\title{Hybrid skin-scale-free effect in non-Hermitian systems: A transfer matrix approach}
\author{Yongxu Fu}
\email{yongxufu@pku.edu.cn}
\affiliation{International Center for Quantum Materials, School of Physics, Peking University, Beijing 100871, China}
\author{Yi Zhang}
\email{frankzhangyi@gmail.com}
\affiliation{International Center for Quantum Materials, School of Physics, Peking University, Beijing 100871, China}

\begin{abstract}
Surpassing the individual buildings of the non-Hermitian skin effect (NHSE) and the scale-free localization (SFL) observed lately, we systematically exploit the exponential decay behavior of bulk eigenstates via the transfer matrix approach in non-Hermitian systems. We concentrate on the one-dimensional (1D) finite-size non-Hermitian systems with $2\times2$ transfer matrices in the participation of boundary impurity. We analytically unveil that the unidirectional pure scale-free (UPSF) effect emerges with the singular transfer matrices, while the hybrid skin-scale-free (HSSF) effect emerges with the nonsingular transfer matrices even though impose open boundary condition (OBC). The UPSF effect exceeds the scope of the SFL in previous works, while the HSSF effect is a charming interplay between the finite-size NHSE and SFL. Our results reveal that the NHSE under OBC prevails in the blend with the SFL as the system tends to the thermodynamic limit. Our approach paves an avenue to rigorously explore the finite-size display of the NHSE and SFL in both Hermitian and non-Hermitian systems with generic boundary conditions.
\end{abstract}
\maketitle

\section{Introduction}
\label{section1}
Exceeding the requirement of Hermitian operators for physical observables in quantum mechanics, the non-Hermitian physics has been widely broadened in the past few years \cite{gong2018,kawabataprx,ashida2020,bergholtzrev2021}, which contains the basic energy band theory \cite{lee2016skin,leykam2017edge,shen2018,kunst2018,yao2018,song2019,yokomizo2019,lee2019an,longhi2019,zhang2020,origin2020,yang2020,lin2023nhse,yao201802,slager2020,xue2021simple,edgeburst2022,longhi2022,fu2023ana}, the higher-order topological phases \cite{kawabata2019second,lee2019ho,edvardsson2019,kawabatahigher,okugawa2020,fu2021,yu2021ho,palacios2021,st2022}, the unique exceptional points (EPs) in non-Hermitian systems \cite{torres2018anomalous,kawabata2019,yokomizo2020,jones2020,zhang2020ep,yang2021,denner2021,fu2022,mandal2021ep,delplace2021ep,liu2021ep,marcus2021ep,ghorashi2021dirac,ghorashi2021weyl}, and other subjects in the scope of condensed matter physics. The exploration of the NHSE in 1D non-Hermitian systems, which supports the accumulation of extensive number of nominal bulk eigenstates under OBC upon the single-particle level, is a milestone in the avenue of the research on the non-Hermitian systems \cite{yao2018,song2019,yokomizo2019,lee2019an,longhi2019,zhang2020,origin2020,yang2020,lin2023nhse}. The non-Bloch band theory with the concept of generalized Brillouin zone (GBZ) is fully explanatory for the NHSE \cite{yao2018,yokomizo2019,yang2020}, and more recently, similar counterparts have paved our way towards higher dimensions \cite{zhang2022uni,wang2022amoeba,yokomizo2022nonbloch,hu2023nonhermitian}.

A typical characteristic of the NHSE-eigenstates is their exponential decay in the bulk region. The exponential factor, corresponding to the relevant point on the GBZ, alters the wave vector in the traditional Bloch band theory from real value to complex value, a core context of the non-Bloch band theory. Strictly speaking, the complexity of the non-Hermitian systems does not constrain the localized behavior of the bulk eigenstates in the NHSE. The SFL, which suggests the existence of the eigenstates with size-dependent localization length in 1D non-Hermitian systems, is explored lately \cite{li2020critical,yokomizo2021scaling,li2021impurity,guo2023scale,libo2023scale,molignini2023anomalous,wang2023scalefree}. Nevertheless, the scale-free behavior is model-dependent, which is a seemingly accidental phenomenon closely related to the impurities \cite{zhu2014pt,felix2018topo,liu2020diag,takato2020impurity,liu2020exact,koch2020bbc,guo2021exact,wu2023effective}, and lack a unified perspective in non-Hermitian systems. Furthermore, the emergence of the interplay between NHSE and SFL remains an unsurveyed subject in the field of non-Hermitian physics.

The transfer matrix approach is a powerful tool for elucidating the tight-binding models for decades \cite{transfer1,transfer2,transfer3,transfer4,transfer5,transfer6,vatsal2016transfer}, which has a more terse and compact form than the method of directly solving the eigenvalue problem, and also applies to the non-Hermitian systems in recent years \cite{kunst2019transfer,luo2021transfer,hamed2023transfer}. In this paper, we utilize the transfer matrix approach to establish a unified depiction of the scale-free effect as well as the interplay between NHSE and SFL. Without loss of generality, we concentrate on the 1D finite-size non-Hermitian systems with $2\times 2$ transfer matrices, such as the Hatano-Nelson (HN) model \cite{hatano1996,hatano1997,hatano1998} and the non-Hermitian Su-Schrieffer-Heeger (NH-SSH) model \cite{lee2016skin,yao2018} with boundary impurity. We unveil that the UPSF effect emerges with the singular transfer matrices, while the HSSF effect appears with the nonsingular transfer matrices, which shows the existence of the bulk eigenstates possessing both NHSE and scale-free exponential decay factors. The UPSF effect must accompany the boundary impurity, while the HSSF effect may exist even under OBC. The localization length of the UPSF mode is generally quasi-linearly dependent on the system size, which is the generalized definition of the SFL in this paper - the results in previous works \cite{li2021impurity,guo2023scale,libo2023scale,molignini2023anomalous} are equivalent to specific cases with linear dependence. The HSSF effect implies that the NHSE under OBC blends with the emergent SFL in the finite sizes yet prevails over the competition in the thermodynamic limit, revealing the NHSE's dominance. Once we turn on the boundary impurity, the HSSF effect displays a finite-size dependence.

The rest of this paper is organized as follows. In Sec.~\ref{section2}, we establish the formalism for probing UPSF and HSSF effects from the transfer matrix perspective, corresponding to the cases with singular and nonsingular transfer matrices, respectively. In Secs. \ref{section3a} and \ref{section4a}, we analytically solve the eigenstates as well as the energy spectrum with the emergence of the pure scale-free (PSF) effect for the HN model and NH-SSH model with boundary impurity. The HSSF effect is expounded for the HN model with boundary impurity in Sec. \ref{section3b} and the NH-SSH model under OBC in Sec. \ref{section4b}. We give a conclusion and further discussions in Sec. \ref{section5}.

\section{Probing UPSF and HSSF effect via transfer matrix approach}
\label{section2}

\subsection{Review of transfer matrix approach for non-Hermitian tight-binding models}
\label{section2a}
Commonly, the real-space tight-binding Hamiltonian of a 1D non-interacting non-Hermitian system reads
\begin{align}
    \label{eqgeneraltba}
    \mathcal{H}&=\sum_{n}\sum_{l=-R}^{R}\sum_{\mu,\nu=1}^{q}t_{l,\mu\nu}c_{n,\mu}^{\dagger}c_{n+l,\nu}\nonumber\\
    &=\sum_{n}\sum_{l=-R}^{R}\mathrm{c}_{n}^{\dagger}\mathrm{t}_{l}\mathrm{c}_{n+l},
\end{align}
where $c_{n,\mu}^{\dagger}$ ($c_{n,\mu}$) is a creation (annihilation) operator with the index of internal degrees of freedom $\mu$ in the $n$-th unit cell, $\mathrm{c}_{n}^{\dagger}$ ($\mathrm{c}_{n}$) is a row (column) vector containing $q$ $c_{n,\mu}^{\dagger}$ ($c_{n,\mu}$), and $t_{l,\mu\nu}$ ($\mathrm{t}_{l}$) is a hopping amplitude (matrix) to the $l$-th nearest unit cell.   

Following the formalism established in Ref.~\cite{transfer1,vatsal2016transfer,kunst2019transfer}, we bundle up at least $R$ adjacent unit cells into a supercell, such that Eq. (\ref{eqgeneraltba}) reduces to a nearest-neighbor tight-binding Hamiltonian
\begin{align}
    \label{eqreduceham}
    \mathcal{H}=\sum_{n=1}^{N-1}\left[\boldsymbol{c}_{n}^{\dagger}\boldsymbol{J}_{L}\boldsymbol{c}_{n+1}+\boldsymbol{c}_{n}^{\dagger}\boldsymbol{M}\boldsymbol{c}_{n}+\boldsymbol{c}_{n+1}^{\dagger}\boldsymbol{J}_{R}^{\dagger}\boldsymbol{c}_{n}\right],
\end{align}
with creation (annihilation) operator denoted as $\boldsymbol{c}_{n}^{\dagger}$ ($\boldsymbol{c}_{n}$) and totally $N$ supercells. There are $\mathcal{N}\geq qR$ internal degrees of freedom in each supercell, and $\boldsymbol{J}_{L,R}$ and $\boldsymbol{M}$ are hopping matrices and onsite matrix, respectively. Without loss of generality, we require $\boldsymbol{J}_{R}=\boldsymbol{J}_{L}\equiv \boldsymbol{J}$ with $\boldsymbol{J}^{2}=0$ and impose non-Hermiticity merely on $\boldsymbol{M}$, i.e., $\boldsymbol{M}\neq \boldsymbol{M}^{\dagger}$ \footnote{This setup is adequate for the core physics in the current paper; also, we can always ensure the nilpotence of $\boldsymbol{J}$ by bundling a sufficient number of unit cells.}. Consequently, the tight-binding Hamiltonian further reduces to
\begin{align}
    \label{eqsimplereduceham}
    \mathcal{H}=\sum_{n=1}^{N-1}\left[\boldsymbol{c}_{n}^{\dagger}\boldsymbol{J}\boldsymbol{c}_{n+1}+\boldsymbol{c}_{n}^{\dagger}\boldsymbol{M}\boldsymbol{c}_{n}+\boldsymbol{c}_{n+1}^{\dagger}\boldsymbol{J}^{\dagger}\boldsymbol{c}_{n}\right].
\end{align}  
Given an arbitrary single-particle state
\begin{align}
    \label{eqgeneralstateform}
    \ket{\Psi}=\sum_{n=1}^{N}\Psi_{n}\boldsymbol{c}_{n}^{\dagger}\ket{0},
\end{align}
with $\Psi_{n} \in \mathbb{C}^{\mathcal{N}}$, the single-particle Schr\"odinger equation $\mathcal{H}\ket{\Psi}=\varepsilon\ket{\Psi}$ reduces to the recursion relation
\begin{align}
    \label{eqrecursionrelation}
    \boldsymbol{J}\Psi_{n+1}+\boldsymbol{J}^{\dagger}\Psi_{n-1}=\left(\varepsilon\mathbbm{1}-\boldsymbol{M}\right)\Psi_{n}.
\end{align}

Next, we define $\mathcal{G}=\left(\varepsilon\mathbbm{1}-\boldsymbol{M}\right)^{-1}$ as the onsite Green's function, which is nonsingular except when $\varepsilon$ is an eigenvalue of $\boldsymbol{M}$. Performing the reduced singular value decomposition (SVD) \cite{lay2016,strang2023}, we obtain
\begin{align}
    \label{eqsvd}
    \boldsymbol{J}=V\Xi W^{\dagger},
\end{align}
where $\Xi=\text{diag}\left\{\xi_{1},\ldots,\xi_{r}\right\}$ is a diagonal matrix of singular value $\xi_{i}\in\mathbb{R}^{+}, i=1,2,\ldots,r$ with $r=\text{rank}\left(\boldsymbol{J}\right)$, and $V$~($W^{\dagger}$) is the matrix comprised of $r$ orthonormal bases $v_{i}$~($w_{i}^{\dagger}$) in the column~(row) space of $\boldsymbol{J}$, that is
\begin{align}
    \label{orthogonalrelation}
    V^{\dagger}V=W^{\dagger}W=\mathbbm{1},\qquad V^{\dagger}W=0.
\end{align}

We focus on the $r=1$ case for simplicity; as a result, $\Xi\equiv\xi\in\mathbb{R}^{+}$ and $\left\{V\equiv v, W\equiv w\right\}$ constitutes a set of orthonormal basis of $\mathbb{C}^{2}$. In this basis, we expand $\Psi_{n}$ and $\mathcal{G}$ as  
\begin{align}
    \label{eqwavebasis}
    \Psi_{n}=\alpha_{n}v+\beta_{n}w,\quad
    \alpha_{n}=v^{\dagger}\Psi_{n},\quad \beta_{n}=w^{\dagger}\Psi_{n},
\end{align}
and
\begin{align}
    \label{eqonsitgreenexp}
    \mathcal{G}_{AB}=B^{\dagger}\mathcal{G}A\in\mathbb{C},\qquad A,B\in \left\{v, w\right\},
\end{align}
respectively.
Combining Eqs. (\ref{eqrecursionrelation})-(\ref{eqonsitgreenexp}), we obtain the propagating relation in the bulk
\begin{align}
    \label{eqtransfereq}
    \Phi_{n+1}=T\Phi_{n},\qquad 
    \Phi_{n}\equiv\left(\begin{matrix}
        \beta_{n}\\
        \alpha_{n-1}
    \end{matrix}\right),
\end{align}
where $T$ is the $2\times 2$ transfer matrix \cite{kunst2019transfer}
\begin{align}
    \label{eqtransfermatrix}
    T=\frac{1}{\xi\mathcal{G}_{vw}}\left(\begin{matrix}
        1&-\xi\mathcal{G}_{ww}\\
        \xi\mathcal{G}_{vv}&\xi^{2}\left(\mathcal{G}_{vw}\mathcal{G}_{wv}-\mathcal{G}_{vv}\mathcal{G}_{ww}\right)
    \end{matrix}\right).
\end{align}

We define the trace and determinant of $T$ as 
\begin{align}
    \label{eqtracedet}
    \Delta=\text{tr}\left(T\right),\qquad \Gamma=\det{\left(T\right)}\equiv\frac{\mathcal{G}_{wv}}{\mathcal{G}_{vw}},
\end{align}
which are rational functions of energy $\varepsilon$ in general, and for simplicity, we suppress the explicit dependence on $\varepsilon$ hereafter. If $T$ is singular ($\Gamma=0$), $T^{n}=\Delta^{n-1}T$; if $T$ is nonsingular ($\Gamma\neq0$) \cite{kunst2019transfer}, 
\begin{align}
    \label{eqnonsingtn}  
    T^{n}=\Gamma^{n/2}\left[\frac{U_{n-1}(z)}{\sqrt{\Gamma}}T-U_{n-2}(z)\mathbbm{1}\right],
\end{align}
where
\begin{align}
    \label{eqchebyshev} 
    U_{n}(z)=\frac{\sin{\left(\left(n+1\right)\phi\right)}}{\sin{\phi}}, 
\end{align}  
is the Chebyshev polynomials of second kind \cite{chebyshevpoly1,chebyshevpoly2} and
\begin{align}
    \label{eqzequalcos}
    z\equiv z(\varepsilon)=\frac{\Delta}{2\sqrt{\Gamma}}=\cos{\phi}\in\mathbb{C}.
\end{align}

After reviewing the transfer matrix approach for non-Hermitian tight-binding models, we shall utilize it for probing UPSF and HSSF effect.      

% Figure environment removed

\subsection{Transfer matrix approach for 1D tight-binding models with boundary impurity}
\label{section2b}

Without loss of generality, we consider the boundary impurity  
\begin{align}
    \label{eqboundimp}   \boldsymbol{c}_{N}^{\dagger}\kappa_{L}\boldsymbol{c}_{1}+\boldsymbol{c}_{1}^{\dagger}\kappa_{R}\boldsymbol{c}_{N}, 
\end{align}
with $\kappa_{L}=\gamma_{L}\boldsymbol{J}$, $\kappa_{R}=\gamma_{R}\boldsymbol{J}^{\dagger}$, $\gamma_{L},\gamma_{R}\geq0$ on top of the Hamiltonian Eq.~(\ref{eqsimplereduceham}), connecting the first and the last supercells \footnote{A more general impurity is still manageable with increased complexity in the calculations.}. Note that the periodic boundary condition (PBC), as well as the Bloch band theory, is reproduced in the $\gamma_{L}=\gamma_{R}=1$ case, while the OBC in the $\gamma_{L}=\gamma_{R}=0$ case. After some algebra, we obtain the equations on the boundaries (Appendix \ref{appendixa})
\begin{align}
    \label{eqgeneralbc}
    \Phi_{1}=K_{L} T\Phi_{N},\qquad \Phi_{2}=TK_{R}\Phi_{1},
\end{align}
where $K_{L}=\text{diag}\left\{1/\gamma_{L},1\right\}$ and $K_{R}=\text{diag}\left\{1,\gamma_{R}\right\}$. Together with Eq. (\ref{eqtransfereq}), we illustrate the full propagating relation of $\Phi_{n}$ with the boundary impurity in Fig. \ref{Fig-propageting}(a). Since $\Phi_{N}=T^{N-2}\Phi_{2}$, Eq. (\ref{eqgeneralbc}) gives
\begin{align}
    \label{eqboundbc}
    \Phi_{1}=K_{L}T^{N}K_{R}\Phi_{1}.
\end{align}
Defining $\varphi=K_{R}\Phi_{1}$ and $K=K_{R}K_{L}=\text{diag}\left\{1/\gamma_{L},\gamma_{R}\right\}$, we final obtain the compact boundary equation
\begin{align}
    \label{eqgeneralbeq}
    \varphi=KT^{N}\varphi.
\end{align}
It implies that the legitimate $\varphi$ is the eigenvector of $KT^{N}$ with respect to the eigenvalue $1$, thus the physical solution is $\Phi_{1}=K_{R}^{-1}\varphi$ and $\Phi_{n}=T^{n-1}\varphi$ for $n=2,3,\ldots,N$.

\subsubsection{Pure scale-free effect}
\label{section2b1}
We first consider the cases with singular transfer matrices, which suggest the occurrence of real-space EPs under OBC \cite{kunst2019transfer}. As we turn on the boundary impurity, the determinant of $KT^{N}$ is zero due to $\Gamma=0$, which implies $\text{tr}\left(KT^{N}\right)=1$ for satisfying Eq. (\ref{eqgeneralbeq}). Consequently, we obtain $\Delta^{N-1}\text{tr}\left(KT\right)=1$, namely ($\text{tr}\left(KT\right)\neq0$)
\begin{align}
    \label{eqgeneralpuresf}
    \Delta^{N}=\frac{\Delta}{\text{tr}\left(KT\right)}.
\end{align}
Assuming $\Delta^{N}=c^{-1}$ with $c$ being an undetermined constant, we can obtain the expression of $\varepsilon$ for $c$ through
\begin{align}
    \label{eqgeneralpuresfcal}
    \Delta=c^{-\frac{1}{N}}e^{-i\frac{2\pi m}{N}}, 
\end{align}
for each $m\in\left\{1,2,\ldots,N\right\}$, denoted as $\varepsilon_{m}$ (regardless of the possible band index). Substituting $\varepsilon_{m}$ into Eq. (\ref{eqgeneralpuresf}), we figure out the constant $c$ as well as energies for each $m$. Although multiple or null solutions of $c$ are perhaps obtained for some $m$ in general, we formally denote the solution of $c$ for each $m$ as $c_{m}$, that is all of $c's$ generate the nominal bulk bands.     

Consider a fixed $c_{m}$, the solution of eigenstate is given by
\begin{align}
    \label{eqsoluscale}
    \Phi_{1}^{m}&=K_{R}^{-1}\varphi_{m}, \nonumber\\
    \Phi_{n}^{m}&=c_{m}^{-\frac{n-2}{N}}e^{-i\frac{2\pi m}{N}(n-2)}T\varphi_{m},
\end{align}
where $n=2,3,\ldots,N$ and $KT^{N}\varphi_{m}=\varphi_{m}$. According to Eq. (\ref{eqwavebasis}), we obtain the eigenstate (up to the normalization coefficients hereafter) concerning energy $\varepsilon_{m}$
\begin{align}
    \Psi_{1}^{m}&=\left(T\varphi_{m}\right)_{2}v+\left(K_{R}^{-1}\varphi_{m}\right)_{1}w,\label{eqeigenstatescale1}\\
    \Psi_{n}^{m}&=c_{m}^{-\frac{n-1}{N}}e^{-i\frac{2\pi m}{N}(n-1)}\psi_{m},\,\, n=2,3,\ldots,N-1,\label{eqeigenstatescalebulk}\\
    \Psi_{N}^{m}&=\left(K_{R}^{-1}\varphi_{m}\right)_{2}v+\left(c_{m}^{-\frac{N-2}{N}}e^{-i\frac{2\pi m}{N}(N-2)}T\varphi_{m}\right)_{1}w\label{eqeigenstatescalen},
\end{align}
where $\psi_{m}=\left(T\varphi_{m}\right)_{2}v+c_{m}^{\frac{1}{N}}e^{i\frac{2\pi m}{N}}\left(T\varphi_{m}\right)_{1}w$, and $\left(\mathscr{V}\right)_{1,2}$ denote the first and second components of the column $2$-vector $\mathscr{V}$.

We specify the bulk region of the supercell systems Eq.~(\ref{eqsimplereduceham}) with boundary impurity Eq.~(\ref{eqboundimp}) as $\mathscr{B}=\left\{2,3,\ldots,N-1\right\}$, as well as the boundary region $\mathscr{E}=\left\{1,N\right\}$. The bulk solution Eq. (\ref{eqeigenstatescalebulk}) visibly displays a exponential decay 
\begin{align}
    \label{eqscaleexp}
      \exp{\left\{-\frac{\text{Re}(\log{c_{m}})}{N}(n-1)\right\}},
\end{align}
with the phase factor
\begin{align}
    \label{eqphasrfac} 
    \exp{\left\{-i\frac{\text{Im}(\log{c_{m}})+2\pi m}{N}(n-1)\right\}},
\end{align} 
which is a correction of the wave vector $2\pi m/N$. We define the occurrence of such exponential-decay behavior of eigenstates in $\mathscr{B}$ as the UPSF effect, which is a rigorous generalization of the SFL in previous work \cite{li2020critical,yokomizo2021scaling,li2021impurity,guo2023scale,libo2023scale,molignini2023anomalous}. Further, we denote the eigenstates like Eq.~(\ref{eqeigenstatescalebulk}) in $\mathscr{B}$ with $\text{Re}(\log{c_{m}})>0$~($<0$) as the left (right)-accumulation UPSF modes, and the localization lengths of these modes are  
\begin{align}
    \label{eqlocalength}
    \xi_{m}=\frac{N}{|\text{Re}(\log{c_{m}})|}.
\end{align}
Noteworthily, $c_{m}$ is generally dependent on system size $N$, thus we define the localization length $\xi_{m}$ quasi-linear dependent on $N$, a typical character of the UPSF effect.

\subsubsection{Hybrid skin-scale-free effect}
\label{section2b2}
More imperceptible phenomena exceeding the NHSE emerge in cases of nonsingular transfer matrices~\footnote{Here, we assume all physical energies corresponding to $\Gamma\neq0$, that is, neglect the extremely special situations where exist some discrete energies leading to $\Gamma=0$.}. When the system size is finite, the NHSE, a rigorous terminology in the thermodynamic limit under OBC, shows not only a finite-size extent but also a possible interplay with the SFL. 

The propagating relation of $\Phi_{n}$ under OBC is illustrated in Fig. \ref{Fig-propageting}(b), and accordingly, the physical condition (the constrain on $\phi$) is (Appendix \ref{appendixb})
\begin{align}
    \label{eqcondition}
    \frac{\sin{\left(N\phi\right)}}{\sin{\left[(N-1)\phi\right]}}=q,
\end{align}
with $q=\xi\sqrt{\frac{\mathcal{G}_{wv}}{\mathcal{G}_{vw}}}\mathcal{G}_{vw}$. Here, we have not simplified $q$ to avoid the branch problems of the square root. Together with Eq. (\ref{eqzequalcos}), we can, in principle, obtain the physical energy $\varepsilon$ and $\phi$ simultaneously. If the energy $\varepsilon$ renders $q$ real, the left-hand side of Eq. (\ref{eqcondition}) must be real, thus leading to real $\phi$. However, if the energy $\varepsilon$ renders $q$ complex, we obtain complex $\phi=\phi_{R}+i\phi_{I}$ and $\phi_{I}\sim c/N$, which is quasi-linearly dependent on $1/N$ (Appendix \ref{appendixb}). The eigenstate concerning energy $\varepsilon$ reads
\begin{align}
    \label{eqobceigenstate}
    \Psi_{n}=\Gamma^{n/2}\left[\mathscr{A}_{L}(\phi)e^{in\phi_{R}}e^{-n\phi_{I}}+\mathscr{A}_{R}(\phi)e^{-in\phi_{R}}e^{n\phi_{I}}\right],
\end{align}
with $n=1,2,\ldots,N,$ and the coefficients $\mathscr{A}_{L}(\phi),\mathscr{A}_{R}(\phi)$ given in Appendix \ref{appendixb}. The overall exponential factor $\Gamma^{n/2}$ indicates the emergence of the NHSE for $|\Gamma|\neq1$, while it is the Hermitian cases or the non-Hermitian cases with some special symmetries (such as parity-time (PT) symmetry) for $|\Gamma|=1$. The exponential factors $e^{\pm n\phi_{I}}$ indicate the SFL with bidirectionally accumulation for a set of $(N,\phi_{R})$. Therefore, we define the occurrence of $|\Gamma|=1$ and $\phi_{I}\neq0$ as the superposition of bidirectional scale-free (SBSF) effect, while $|\Gamma|=1$ and $\phi_{I}=0$ suggests the extended eigenstate (ES). Noteworthily, the SBSF effect reduces to the UPSF effect if one of $\mathscr{A}_{L,R}(\phi)$ vanishes, thus the PSF effect is defined as a union of the UPSF and SBSF effects. Further, the occurrence of $|\Gamma|\neq1$ and $\phi_{I}\neq0$ is defined as the HSSF effect, while the pure NHSE emerges with $|\Gamma|\neq1$ and $\phi_{I}=0$. These correspondences are summarized in Table I. 
\begin{table}
    \centering
    \setlength{\tabcolsep}{5mm}{
    \begin{tabular}{|c|c|c|}
         \hline
         \diagbox{$|\Gamma|=1$}{$\phi_{I}=0$} & Yes & No  \\
         \hline
         Yes & ES & PSF \\
         \hline
         No & NHSE & HSSF \\
         \hline
    \end{tabular}}
    \label{table1}
    \caption{Summary of ES, NHSE, PSF, and HSSF concerning nonsingular transfer matrices.}
\end{table}

However, $n\phi_{I}\rightarrow 0$ with $n$ lying in the deep bulk as $N\rightarrow +\infty$ under OBC, which means the exponential factor dominated by $\Gamma^{n/2}$ in the deep bulk, namely, the exact context of NHSE in the thermodynamic limit \cite{fu2023ana}. Hence, the emergent SFL participates in the NHSE under finite size, and the NHSE gradually dominates with increasing system size. 

Next, we turn on the boundary impurity in Eq. (\ref{eqboundimp}), and the physical condition is given by Eq. (\ref{eqgeneralbeq}). Again, the complex $\phi=\phi_{R}+i\phi_{I}$ emerges under finite size but the forms $\phi_{I}\sim c/N$ do not always exist when $N\rightarrow+\infty$ (Appendix \ref{appendixc}). The general form of the eigenstate in $\mathscr{B}$ with respect to energy $\varepsilon$ and complex $\phi$ reads
\begin{align}
    \label{eqimpbulkeigenstate}
    \Psi_{n}^{imp}&=\Gamma^{n/2}\left[\mathcal{B}_{L}(\phi)e^{in\phi_{R}}e^{-n\phi_{I}}+\mathcal{B}_{R}(\phi)e^{-in\phi_{R}}e^{n\phi_{I}}\right],
\end{align}
where the coefficients $\mathcal{B}_{L}(\phi)$, $\mathcal{B}_{R}(\phi)$ are given in Appendix \ref{appendixc}. Similar to the OBC cases under finite sizes, the various possible behaviors of eigenstates are also depicted in Table I. Nevertheless, the thermodynamic limit may prevent both the NHSE and SFL effects with the boundary impurity, an abrupt change from the OBC. In other words, the HSSF effect in the presence of boundary impurity is sensitive to the system size. 

\section{Hatano-Nelson model with boundary impurity}
\label{section3}
As a typical 1D non-Hermitian model, the HN model with boundary impurity reads
\begin{align}
    \label{eqhaimp}
    \mathcal{H}_{hn}=\sum_{n=1}^{N-1}\left(t_{L}c_{n}^{\dagger}c_{n+1}+t_{R}c_{n+1}^{\dagger}c_{n}\right)+\gamma_{R}c_{1}^{\dagger}c_{N}+\gamma_{L}c_{N}^{\dagger}c_{1},
\end{align}
where we assume $t_{L},t_{R},\gamma_{L},\gamma_{R}\geq0$ for simplicity. The model manifests an emergence of the SFL except for the NHSE \cite{li2021impurity,molignini2023anomalous}. In this section, we apply the transfer matrix approach to comprehensively study the display and interplay of NHSE and scale-free effect in Eq. (\ref{eqhaimp}). Note the single-band nature of the NH model offers an expedient expression of the transfer matrix in the single-particle wave function space, denoted as $\ket{\Psi}=\sum_{n=1}^{N}\psi_{n}c_{n}^{\dagger}\ket{0}$, instead of the nominal $\left\{v,w\right\}$ space, after some algebra, we obtain the propagating relation in the bulk (Appendix \ref{appendixd})
\begin{align}
    \label{eqhnimpprop}
    \left(\begin{matrix}\psi_{n+1}\\ \psi_{n}
    \end{matrix}\right)=T\left(\begin{matrix}\psi_{n}\\ \psi_{n-1}
    \end{matrix}\right),
\end{align}
where the transfer matrix with respect to energy $\varepsilon$ is 
\begin{align}
    \label{eqhntransfermat}
    T=\left(\begin{matrix}\frac{\varepsilon}{t_{L}}&-\frac{t_{R}}{t_{L}}\\1& 0
    \end{matrix}\right),
\end{align}
and we label 
\begin{align}
    \label{eqhnimptrace}
    \Delta&=\text{tr}\left(T\right)=\frac{\varepsilon}{t_{L}},\nonumber\\
    \Gamma&=\det{\left(T\right)}=\frac{t_{R}}{t_{L}}.
\end{align}

\subsection{Exact solutions of the PSF effect}
\label{section3a}
We first consider the case with a singular transfer matrix, namely, $t_{R}=0$ leading to $\Gamma=0$. The eigenvectors and the corresponding energies are (Appendix \ref{appendixd1})
\begin{align}
    \label{eqhnsingvec} 
    \varepsilon_{m}&=t_{L}c_{m}^{-\frac{1}{N-2}}e^{-i\frac{2\pi m}{N-2}},\quad m=1,2,\ldots,N,\nonumber\\ 
    \psi_{n}^{m}&=c_{m}^{-\frac{n-2}{N-2}}e^{-i\frac{2\pi m}{N-2}(n-2)},\quad n=3,4,\ldots,N, \nonumber\\
    \psi_{1}^{m}&=\frac{t_{L}}{\gamma_{L}}c_{m}^{-\frac{1}{N-2}}e^{-i\frac{2\pi m}{N-2}}\psi_{N}^{m},    
\end{align}
with the physical condition
\begin{align}
    \label{eqhnimpphycond3}
    \frac{t_{L}^{2}}{\gamma_{L}}c_{m}^{-\frac{2}{N-2}}e^{-i\frac{4\pi m}{N-2}}=t_{L}c_{m}+\gamma_{R},
\end{align}
where we have set $\psi_{2}^{m}=1$. Apparently, the exponential factor of $\psi_{n}^{m}$ leads to the PSF effect, with the localization length 
\begin{align}
    \label{eqhnsingloclen}
    \xi_{m}=\frac{N-2}{|\text{Re}\left(\log{c_{m}}\right)|},
\end{align}
quasi-linearly dependent on the system size, and the left (right)-accumulation UPSF modes correspond to $\text{Re}\left(\log{c_{m}}\right)>0$ ($<0$). The phase diagrams of the UPSF modes in $\gamma_{L}$-$\gamma_{R}$ plane are plotted in Figs.~\ref{Fig-hnsingleft}(A)(B) with system size $N=10, 30$, where the green regions label the existence of the UPSF modes with left-accumulation (LA), the cyan regions label that with right-accumulation (RA), and the yellow regions label that with both left- and right-accumulation (LRA) \footnote{The LRA-region may contain the extreme modes with $\text{Re}\left(\log{c_{m}}\right)=0$, which are not our concerns in this paper.}. In Fig.~\ref{Fig-hnsingleft}, we also plot the full single-particle energy spectrum with the corresponding left-accumulation UPSF modes for typical parameters [red stars in Figs.~\ref{Fig-hnsingleft}(A)(B)], where the analytic results Eq. (\ref{eqhnsingvec}) (red dots and lines) are perfectly consistent with the numerical results (blue dots) for various system sizes \footnote{We uniformly label the coordinate axis of the numerical energy spectrum (eigenstates) as $(\text{Re}(\text{E}),\text{Im}(\text{E}))$ [($\text{site},|\psi|$)] for simplicity throughout this paper.}. In addition, the scale-free localized modes given in Refs. \cite{li2021impurity,molignini2023anomalous} are equivalent to the strong non-reciprocity or extremely special $\gamma_{R}\rightarrow0$ cases in our transfer matrix formalism~(Appendix \ref{appendixd1}).

% Figure environment removed

% Figure environment removed

% Figure environment removed

\subsection{The emergent HSSF effect with boundary impurity}
\label{section3b}
We next consider the case with nonsingular transfer matrix, whose energy expression is given by 
\begin{align}
    \label{eqhnhssfenformu}
    \varepsilon=2\sqrt{t_{L}t_{R}}\cos{\phi},\quad \phi=\phi_{R}+i\phi_{I}\in\mathbb{C},
\end{align}
through Eq. (\ref{eqzequalcos}). The solutions of $\phi$ lie in $\left[0,\pi\right]$ under OBC, while lead to physical real or complex values with boundary impurity in Eq. (\ref{eqhaimp}) (Appendix \ref{appendixd2}). In order to observe the HSSF effect unambiguously, we perform a generalized gauge transformation $\mathcal{H}_{it}=S^{-1}\mathcal{H}_{hn}S$ to cancel out the more dominant NHSE amplification factor (Appendix \ref{appendixd2}), resulting in
\begin{align}
    \label{eqtargetham}
    \mathcal{H}_{it}=\sum_{n=1}^{N-1}t\left(c_{n}^{\dagger}c_{n+1}+\text{h.c.}\right)+(\delta+\gamma)c_{1}^{\dagger}c_{N}+(\delta-\gamma)c_{N}^{\dagger}c_{1},
\end{align}
where $S=\text{diag}\left\{r,r^{2},\ldots,r^{N}\right\}$ is the transformation matrix, $t=\sqrt{t_{L}t_{R}}$, $r=\sqrt{t_{R}/t_{L}}$, and we have set $r^{N-1}\gamma_{R}=\delta+\gamma$ and $r^{-(N-1)}\gamma_{L}=\delta-\gamma$. Subsequently, the eigenstate corresponding to energy $\varepsilon=2t\cos{\phi}$ of Eq. (\ref{eqtargetham}) reads 
\begin{align}
    \label{eqhnitvecs}
    \left(\begin{matrix}
        \psi_{n+1}\\\psi_{n}
    \end{matrix}\right)=\left[A_{L}(\phi)e^{i(n-1)\phi}+A_{R}(\phi)e^{-i(n-1)\phi}\right]\left(\begin{matrix}
        \psi_{2}\\\psi_{1}
    \end{matrix}\right), 
\end{align}
where the coefficients $A_{L,R}(\phi)$ are given in Appendix \ref{appendixd2}. In Figs.~\ref{Fig-hnhssf}(A)(B), we illustrate the energy spectrum for two typical points in the PT-broken region $\gamma\in\left[|\delta-t|,\delta+t\right]$ (Appendix \ref{appendixd2}). The real (complex) energy (blue dots) in Fig.~\ref{Fig-hnhssf}(A) corresponds to the ES (SBSF mode), of which the match-perfectly analytic (red lines) and numerical (blue dots) results are plotted in Figs. \ref{Fig-hnhssf}(a1)(a2). Further, the scale-free distributions of the LA [$A_{R}(\phi)e^{-i(n-1)\phi}$] and RA [$A_{L}(\phi)e^{i(n-1)\phi}$] components of the SBSF mode [Fig. \ref{Fig-hnhssf}(a2) with $\phi_{I}<0$] are illustrated in Fig. \ref{Fig-hnhssfcomp}. The two complex energies (blue dots) in Fig.~\ref{Fig-hnhssf}(B) correspond to the UPSF modes (vanishing RA term), of which the match-perfectly analytic (red lines) and numerical (blue dots) results are plotted in Figs. \ref{Fig-hnhssf}(b1)(b2). We remark that all solutions of $\phi$ for Fig. \ref{Fig-hnhssf}(B) are complex-valued, whose imaginary part is $\log{(\mu^{-1})}/N$ with $\mu=\delta/t+\sqrt{(\delta/t)^{2}-1}$ (Appendix \ref{appendixd2}). Therefore, we observe the pure NHSE (HSSF effect) of $\mathcal{H}_{hn}$ ($t_{L}\neq t_{R}$) corresponding to ESs (PSF modes) of $\mathcal{H}_{it}$ after the inverse gauge transformation $S^{-1}$. We emphasize that the HSSF effect with boundary impurity may be a finite-size phenomenon, since the thermodynamic limit may prevent simultaneous NHSE and SFL.

\section{The emergent UPSF and HSSF effects in the NH-SSH model}
\label{section4}
The well-known NH-SSH model is a typical $r=1$ Hamiltonian ($\mathcal{H}_{nssh}$) of Eq. (\ref{eqsimplereduceham}) with 
\begin{align}
    \label{eqnsshham}
    \boldsymbol{M}=\left(\begin{matrix}
        0&t_{1}+\gamma\\
        t_{1}-\gamma&0
    \end{matrix}\right),\quad
    \boldsymbol{J}=\left(\begin{matrix}
        0&0\\
        t_{2}&0
    \end{matrix}\right),
\end{align}
where we assume $t_{1},t_{2},\gamma\geq 0$ for simplicity. The transfer matrix reads
\begin{align}
    \label{eqnsshtransfmat}
    T=\frac{1}{t_{2}(t_{1}+\gamma)}\left(\begin{matrix}
        \varepsilon^{2}-t_{1}^{2}+\gamma^{2}&-\varepsilon t_{2}\\
        \varepsilon t_{2}&-t_{2}^{2}
    \end{matrix}\right),
\end{align}
with
\begin{align}
    \label{eqnsshimptrace}
    \Delta&=\text{tr}\left(T\right)=\frac{\varepsilon^{2}-t_{1}^{2}+\gamma^{2}-t_{2}^{2}}{t_{2}(t_{1}+\gamma)},\nonumber\\
    \Gamma&=\det{\left(T\right)}=\frac{t_{1}-\gamma}{t_{1}+\gamma}.
\end{align}
following the reduced SVD $\boldsymbol{J}=v\xi w^{\dagger}$ with $\xi=t_{2}$, $v=(0,1)^{T}$, and $w=(1,0)^{T}$. We introduce the boundary impurity in Eq. (\ref{eqboundimp}) into such an NH-SSH model and analyze its PSF or HSSF effect. 

% Figure environment removed

\subsection{The PSF effect with boundary impurity}
\label{section4a}
The transfer matrix at the critical point $t_{1}=\gamma$ of PT-symmetry breaking is singular with $\Gamma=0$, where we can directly apply the analytic results in Sec. \ref{section2b1}. The single-particle eigenstates with respect to the two energy bands ($\pm$) are
\begin{align}
    \label{eqnsshsingvecs}
    \varepsilon_{m}^{\pm}&=\pm\sqrt{t_{2}^{2}+2t_{2}\gamma c_{m}^{-\frac{1}{N}}e^{-i\frac{2\pi m}{N}}},\nonumber\\
    \Psi_{n}^{\pm,m}&=\alpha_{n}^{\pm,m}v+\beta_{n}^{\pm,m}w,
\end{align}
where $m=1,2,\ldots,N$, 
\begin{align}
    \label{eqnsshsingsolu}
    \alpha_{n}^{\pm,m}&=c_{m}^{-\frac{n}{N}}e^{-i\frac{2\pi m}{N}n}\frac{c_{m}\gamma_{L} t_{2}}{\varepsilon_{m}^{\pm}},\quad n=1,2,\ldots,N-1,\nonumber\\
    \beta_{n}^{\pm,m}&=c_{m}^{-\frac{n-1}{N}}e^{-i\frac{2\pi m}{N}(n-1)}c_{m}\gamma_{L},\quad n=2,3,\ldots,N,\nonumber\\
    \alpha_{N}^{\pm,m}&=\frac{\varepsilon_{m}^{\pm\,2}-c_{m}\gamma_{L}(\varepsilon_{m}^{\pm\,2}-t_{2}^{2})}{\gamma_{R}\varepsilon_{m}^{\pm}t_{2}},\nonumber\\
    \beta_{1}^{\pm,m}&=1,
\end{align}
and $c_{m}$ satisfies the physical condition
\begin{align}
    \label{eqnsshsingcond}
    t_{2}(1-\gamma_{L}\gamma_{R})=2\gamma c_{m}^{-\frac{1}{N}}e^{-i\frac{2\pi m}{N}}(\gamma_{L}c_{m}-1).
\end{align}
These results indicate the emergence of the UPSF effect with the localization length $\xi_{m}=N/|\text{Re}(\log{c_{m}})|$, and the phase diagram of the UPSF modes is illustrated in Fig.~\ref{Fig-nsshsing}(A). Applying Eq.~(\ref{eqnsshsingvecs}) to the three selected points (red stars) in Fig.~\ref{Fig-nsshsing}(A), which correspond to the UPSF modes with LA, RA, and LRA, respectively, we obtain full energy spectrum [red dots in Figs.~\ref{Fig-nsshsing}(a1)-(c1)] and eigenstates [red lines in Figs.~\ref{Fig-nsshsing}(a2)-(c2)] perfectly matching the numerical results (blue dots).

% Figure environment removed


\subsection{The HSSF effect under OBC}
\label{section4b}
The most counter-intuitive phenomenon with the emergent HSSF effect appears in the case with nonsingular transfer matrix under OBC. According to Appendix \ref{appendixb}, the physical condition is given by
\begin{align}
    \label{eqnsshhssfcond}
    \frac{\sin{\left(N\phi\right)}}{\sin{\left((N-1)\phi\right)}}=\frac{t_{2}\sqrt{(t_{1}+\gamma)(t_{1}-\gamma)}}{\varepsilon^{2}-t_{1}^{2}+\gamma^{2}}\equiv q,
\end{align}
which figures out the solutions of $\phi=\phi_{R}+i\phi_{I}$
\begin{align}
    \phi_{I}&=\frac{\log{|\mathscr{F}|}}{2N},\label{eqnsshhsshphiim}\\
    \phi_{R}&=\frac{2\pi m-\text{arg}(\mathscr{F})}{2N},\label{eqnsshhsshphire}\\
    \mathscr{F}&=\frac{1-qe^{-i\phi}}{1-qe^{i\phi}},\label{eqnsshhssfphisolu}
\end{align}
with $m=1,2,\ldots,N$. In the region $t_{1}>\gamma$, real $\phi$ and OBC energy spectrum exist simultaneously. However, in the region $t_{1}<\gamma$, complex energies lead to the presence of imaginary part of $\phi$, namely the quasi-$1/N$-proportion Eq. (\ref{eqnsshhsshphiim}). Hence, the HSSF effect emerges according to Eq. (\ref{eqobceigenstate}), which has never been uncovered in previous works.

As in Sec. \ref{section3b}, we perform a generalized gauge transformation for $\mathcal{H}_{nssh}$ to conceal the NHSE factor with $S=\text{diag}\left\{1,r,r,\ldots,r^{N-1},r^{N-1},r^{N}\right\}$ and $r=\sqrt{\Gamma}$, resulting in the Hamiltonian $\bar{\mathcal{H}}$ with
\begin{align}
    \label{eqnsshhambar}
    \bar{M}=\left(\begin{matrix}
        0&\bar{t}_{1}\\
        \bar{t}_{1}&0
    \end{matrix}\right),\quad
    \bar{J}=\boldsymbol{J}, \quad \bar{t}_{1}=\sqrt{(t_{1}+\gamma)(t_{1}-\gamma)}.
\end{align}

Subsequently, we obtain $\bar{\Delta}=(\varepsilon^{2}-\bar{t}_{1}^{2}-t_{2}^{2})/\bar{t}_{1}t_{2}$ and $\bar{\Gamma}=1$, which result the same physical condition for $\mathcal{H}_{nssh}$ in Eq.~(\ref{eqnsshhssfcond}), and according to Eq. (\ref{eqzequalcos}), we obtain the expression of the energy 
\begin{align}
    \label{eqnsshhssfenexp}
    \varepsilon^{2}=t_{1}^{2}-\gamma^{2}+t_{2}^{2}+2t_{2}\bar{t}_{1}\cos{\phi},
\end{align}
whose related eigenstates read 
\begin{align}
    \label{eqnsshbarvecs}
    \bar{\Psi}_{n}=\bar{\mathscr{A}}_{L}(\phi)e^{in\phi_{R}}e^{-n\phi_{I}}+\bar{\mathscr{A}}_{R}(\phi)e^{-in\phi_{R}}e^{n\phi_{I}},
\end{align}
where the coefficients are given by
\begin{align}
    \label{eqnsshbarcoof}
    \bar{\mathscr{A}}_{L}(\phi)&=\frac{1}{2iq\sin{\phi}}\left[t_{2}\bar{\mathcal{G}}_{vv}v+(e^{-i\phi}-qe^{-2i\phi})w\right],\nonumber\\
    \bar{\mathscr{A}}_{R}(\phi)&=\frac{1}{2iq\sin{\phi}}\left[-t_{2}\bar{\mathcal{G}}_{vv}v+(-e^{i\phi}+qe^{2i\phi})w\right],
\end{align}
with $\bar{\mathcal{G}}_{vv}=\varepsilon/(\varepsilon^{2}-t_{1}^{2}+\gamma^{2})$. Utilizing the full energy spectrum in Fig.~\ref{Fig-nsshhssf}(a), we figure out the solutions of $\phi$ for each energy labeled as $j$ in Figs.~\ref{Fig-nsshhssf}(b)(c), which are perfectly consistent with the analytic formulas Eqs.~(\ref{eqnsshhsshphiim})-(\ref{eqnsshhssfphisolu}). The analytic (red lines) and numerical (blue dots) eigenstates concerning the three selected points (blue dots) in Fig.~\ref{Fig-nsshhssf}(a) are plotted in Figs.~\ref{Fig-nsshhssf}(d)-(f), which match perfectly. Further, the scale-free distributions of the LA [$\bar{\mathscr{A}}_{L}(\phi)e^{in\phi_{R}}e^{-n\phi_{I}}$] and RA [$\bar{\mathscr{A}}_{R}(\phi)e^{-in\phi_{R}}e^{n\phi_{I}}$] components of these SBSF modes in Figs. \ref{Fig-nsshhssf}(d)-(f) are illustrated in Figs. \ref{Fig-nsshhssf}(g)-(i), respectively. Therefore, the SBSF modes $\bar{\Psi}_{n}$ of $\bar{\mathcal{H}}$, which resemble the ESs in appearance, indicating the emergent HSSF modes of the original NH-SSH model under OBC after the inverse gauge transformation. 

As we turn on the boundary impurity, the physical condition is given by 
\begin{align}
    \label{eqnsshhssfimpcond}
    &\left(\gamma_{L}\Gamma^{-\frac{N}{2}}+\gamma_{R}\Gamma^{\frac{N}{2}}\right)\sin{\phi}=\sin{\left((N+1)\phi\right)}\nonumber\\
    &+\frac{t_{2}(1-\gamma_{L}\gamma_{R})}{\sqrt{(t_{1}+\gamma)(t_{1}-\gamma)}}\sin{(N\phi)}-\gamma_{L}\gamma_{R}\sin{\left((N-1)\phi\right)},
\end{align}
according to Eq. (\ref{eqnsshhssfenexp}) and Appendix \ref{appendixc}, which is equivalent to the result in Ref. \cite{guo2021exact}. The HSSF modes in [Eq. (\ref{eqimpbulkeigenstate})] appear with possible emergent complex solutions of $\phi$, which match with the numerical results in Appendix \ref{appendixe}. However, the HSSF effect with boundary impurity is usually limited to finite sizes due to the restraints on the NHSE and SFL in large systems. 

\section{Conclusions and discussions}
\label{section5}
We have developed a transfer-matrix perspective for a unified understanding of the NHSE and the SFL in non-Hermitian systems with boundary impurity. We derive the analytic expressions for the single-particle bulk eigenstates with respect to the energy spectrum upon the 1D non-Hermitian systems with $2\times 2$ transfer matrices, which exhibit excellent consistency with numerical results in the HN model and NH-SSH model. We find that the UPSF effect accompanies singular transfer matrices, while the HSSF effect emerges with nonsingular transfer matrices, even under OBC. The UPSF effect portrays a localization length of the eigenstates quasi-linear with the system size, while the HSSF effect shows an interplay between the finite-size NHSE and SFL. We further reveal that the NHSE prevails over the SFL under OBC as the system size increases, thus $\phi_{I}=0$ in the thermodynamic limit. Interestingly, we may then generate the energy spectrum from $\Delta=2\sqrt{\Gamma}\cos{\phi_{R}}$ [Eq. (\ref{eqzequalcos})]; correspondingly, the norm $\sqrt{|\Gamma|}$ with corresponding $(\text{arg}(\Gamma),\phi_{R})$ outlines the GBZ and re-invents the non-Bloch band theory through the transfer matrix approach, open to further generalizations for future studies. 

The constraints over the assumption $\boldsymbol{J}_{R}=\boldsymbol{J}_{L}$, rank-$1$ of $\boldsymbol{J}$, namely $2\times 2$ transfer matrices, and the boundary impurity in Eq. (\ref{eqboundimp}), are sufficiently general capture the core mechanism and behaviors of the PSF and HSSF effects here. Nevertheless, there exist more questions to explore in the future, such as global impurity and disorder \cite{molignini2023anomalous,longhi2021disorder,okuma2021disorder,luo2021transfer,yuce202disorder,sarkar2022disorder}, the role of various symmetries \cite{gong2018,kawabataprx}, realistic experiments and applications, and other subjects of the NHSE and SFL in condensed matter physics. 


\section*{Acknowledgements}
We acknowledge support from the National Key R\&D Program of China (No.2022YFA1403700) and the National Natural Science Foundation of China (No.12174008 \& No.92270102). 

\appendix

\begin{widetext}
\section{The boundary equations with boundary impurity}
\label{appendixa}
Modifying Eq.~(\ref{eqrecursionrelation}) at the boundaries with the setting $\Psi_{0}\equiv\Psi_{N}$, we obtain
\begin{align}
    \label{eqappbounds}
    \Psi_{N}&=\mathcal{G}\boldsymbol{J}^{\dagger}\Psi_{N-1}+\gamma_{L}\mathcal{G}\boldsymbol{J}\Psi_{1}\nonumber\\
    \Psi_{1}&=\gamma_{R}\mathcal{G}\boldsymbol{J}^{\dagger}\Psi_{N}+\mathcal{G}\boldsymbol{J}\Psi_{2}.
\end{align}
Combining with Eqs. (\ref{eqsvd})-(\ref{eqonsitgreenexp}), we obtain 
\begin{align}
    \label{eqappboundsdetail}
    \alpha_{N}&=\mathcal{G}_{wv}\xi\alpha_{N-1}+\gamma_{L}\mathcal{G}_{vv}\xi\beta_{1},\nonumber\\
    \beta_{N}&=\mathcal{G}_{ww}\xi\alpha_{N-1}+\gamma_{L}\mathcal{G}_{vw}\xi\beta_{1},\nonumber\\
    \alpha_{1}&=\gamma_{R}\mathcal{G}_{wv}\xi\alpha_{N}+\mathcal{G}_{vv}\xi\beta_{2},\nonumber\\
    \beta_{1}&=\gamma_{R}\mathcal{G}_{ww}\xi\alpha_{N}+\mathcal{G}_{vw}\xi\beta_{2}.
\end{align}   
Further derivation gives
\begin{align}
    \label{eqappboundsfinal}
    \left(\begin{matrix}
        \beta_{1}\\
        \alpha_{N}
    \end{matrix}
        \right)
    &=\left(\begin{matrix}
        \frac{1}{\gamma_{L}}\xi^{-1}\mathcal{G}_{vw}^{-1}&-\frac{1}{\gamma_{L}}\xi^{-1}\mathcal{G}_{vw}^{-1}\mathcal{G}_{ww}\xi\\
        \mathcal{G}_{vv}\mathcal{G}_{vw}^{-1}&\left(\mathcal{G}_{wv}-\mathcal{G}_{vv}\mathcal{G}_{vw}^{-1}\mathcal{G}_{ww}\right)\xi
    \end{matrix}
    \right)
    \left(\begin{matrix}
        \beta_{N}\\
        \alpha_{N-1}
    \end{matrix}
        \right),\nonumber\\
    \left(\begin{matrix}
        \beta_{2}\\
        \alpha_{1}
    \end{matrix}
        \right)
    &=\left(\begin{matrix}
        \xi^{-1}\mathcal{G}_{vw}^{-1}&-\gamma_{R}\xi^{-1}\mathcal{G}_{vw}^{-1}\mathcal{G}_{ww}\xi\\
        \mathcal{G}_{vv}\mathcal{G}_{vw}^{-1}&\gamma_{R}\left(\mathcal{G}_{wv}-\mathcal{G}_{vv}\mathcal{G}_{vw}^{-1}\mathcal{G}_{ww}\right)\xi
    \end{matrix}\right)
    \left(\begin{matrix}
        \beta_{1}\\
        \alpha_{N}
    \end{matrix}
        \right),
\end{align}
which are exactly the equations on the boundaries
\begin{align}
    \label{eqappboundsresult}
    \Phi_{1}&=\left(\begin{matrix}
        \frac{1}{\gamma_{L}}&0\\
        0&1
    \end{matrix}\right)
    T\Phi_{N}=K_{L}T\Phi_{N},\nonumber\\
    \Phi_{2}&=T\left(\begin{matrix}
        1&0\\
        0&\gamma_{R}
    \end{matrix}\right)\Phi_{1}=TK_{R}\Phi_{1}.
\end{align}

\section{The solutions of nonsingular cases under OBC}
\label{appendixb}

The OBC refers to the hard or Dirichlet boundary condition $\Psi_{0}=\Psi_{N+1}=0$, implying $\alpha_{0}=\beta_{N+1}=0$ such that
\begin{align}
    \label{eqappobccond}
    \Phi_{1}=\left(\begin{matrix}\beta_{1}\\0
    \end{matrix}\right),\quad
    \Phi_{N+1}=\left(\begin{matrix}0\\\alpha_{N}
    \end{matrix}\right).
\end{align}
After normalization, we obtain the OBC equation
\begin{align}
    \label{eqappobceq}
    T^{N}\left(\begin{matrix}1\\0
    \end{matrix}\right)=\left(\begin{matrix}0\\\tau
    \end{matrix}\right),
\end{align}
with $\tau\in\mathbb{C}$. Accordingly, the propagating relation of $\Phi_{n}$ is illustrated in Fig. \ref{Fig-propageting}(b). Utilizing Eq. (\ref{eqnonsingtn}), we obtain the physical condition under OBC \cite{kunst2019transfer}
\begin{align}
    \label{eqappcondition}
    \frac{\sin{\left(N\phi\right)}}{\sin{\left((N-1)\phi\right)}}=q,
\end{align}
with $q=\xi\sqrt{\frac{\mathcal{G}_{wv}}{\mathcal{G}_{vw}}}\mathcal{G}_{vw}$. Further simplifying Eq. (\ref{eqappcondition}) with general complex $\phi=\phi_{R}+i\phi_{I}$ and $\phi_{R}\in[0,2\pi],\phi_{I}\in\mathbb{R}$, we obtain
\begin{align}
    \label{eqappphieq}
    e^{2N\phi_{I}}=\frac{1-qe^{-i\phi}}{1-qe^{i\phi}}e^{2iN\phi_{R}}.
\end{align}
Hence, the solutions of $\phi$ reads
\begin{align}
    \label{eqappobcphisolu}
    \phi_{I}&=\frac{\log{|\mathscr{F}|}}{2N},\nonumber\\
    \phi_{R}&=\frac{2\pi m-\text{arg}(\mathscr{F})}{2N},
\end{align}
where $\mathscr{F}$ labels $\left(1-qe^{-i\phi}\right)/\left(1-qe^{i\phi}\right)$ and $m=1,2,\ldots,N$. Note that $|\mathscr{F}|=1$ leads to $\phi_{I}=0$ if $q$ is real. In general complex-$q$ cases, the explicit $\phi_{I}$ and energy $\varepsilon$ are dependent on $N$ and $\phi_{R}$ through the self-consistent transcendental equations (\ref{eqzequalcos}) and (\ref{eqappobcphisolu}), that is we can denote $\phi_{I}=c/N$ with $c$ being a finite-value function of $N$ and $\phi_{R}$, and thus $\phi_{I}$ is quasi-linearly dependent on $1/N$.

The solution of an arbitrary eigenstate is given by $\Phi_{n+1}=T^{n}\left(\begin{matrix}1\\0\end{matrix}\right)$, resulting in
\begin{align}
    \label{eqappobceigensolu}
    \beta_{n}&=\frac{\Gamma^{(n-1)/2}}{q\sin{\phi}}\left[\sin{\left((n-1)\phi\right)}-q\sin{\left((n-2)\phi\right)}\right],\nonumber\\
    \alpha_{n}&=\frac{\Gamma^{n/2}}{q\sin{\phi}}\xi\mathcal{G}_{vv}\sin{(n\phi)},
\end{align}
with $n=1,2,\ldots,N$. Note that $\beta_{N+1}=0$ is exactly the physical condition Eq. (\ref{eqappcondition}) and $\alpha_{N}=\tau$. Consequently, an eigenstate concerning energy $\varepsilon$ reads
\begin{align}
    \label{eqappobceigenstate}
    \Psi_{n}=\Gamma^{n/2}\left[\mathscr{A}_{L}(\phi)e^{in\phi_{R}}e^{-n\phi_{I}}+\mathscr{A}_{R}(\phi)e^{-in\phi_{R}}e^{n\phi_{I}}\right],
\end{align}
where
\begin{align}
    \label{eqappobcstatecoff}
    \mathscr{A}_{L}(\phi)&=\frac{1}{2iq\sin{\phi}}\left[\xi\mathcal{G}_{vv}v+\left(e^{-i\phi}-qe^{-2i\phi}\right)\frac{w}{\sqrt{\Gamma}}\right],\nonumber\\
    \mathscr{A}_{R}(\phi)&=\frac{1}{2iq\sin{\phi}}\left[-\xi\mathcal{G}_{vv}v+\left(-e^{i\phi}+qe^{2i\phi}\right)\frac{w}{\sqrt{\Gamma}}\right].
\end{align}


\section{The solutions of nonsingular cases with boundary impurity}
\label{appendixc}
According to the boundary equation (\ref{eqgeneralbeq}) with boundary impurity, the eigenvalues of the $2\times2$ matrix $KT^{N}$ must be $1$ and 
\begin{align}
    \label{eqappdetktn}
    \det{\left(KT^{N}\right)}=\det{\left(K\right)}\left(\det{T}\right)^{N}=\frac{\gamma_{R}}{\gamma_{L}}\Gamma^{N},
\end{align} 
such that,
\begin{align}
    \label{eqapptrktn1}
    \text{tr}\left(KT^{N}\right)=1+\frac{\gamma_{R}}{\gamma_{L}}\Gamma^{N}=\Gamma^{N/2}\left(\frac{\gamma_{R}}{{\gamma_{L}}}\right)^{1/2}\left[\left(\frac{\gamma_{R}}{{\gamma_{L}}}\Gamma^{N}\right)^{-1/2}+\left(\frac{\gamma_{R}}{{\gamma_{L}}}\Gamma^{N}\right)^{1/2}\right].
\end{align}
On the other hand, utilizing the formula Eq.~(\ref{eqnonsingtn}) of nonsingular $T$, we obtain
\begin{align}
    \label{eqapptrktn2}
    \text{tr}\left(KT^{N}\right)=\Gamma^{N/2}\left[\frac{U_{N-1}(z)}{\sqrt{\Gamma}}\text{tr}\left(KT\right)-U_{N-2}(z)\text{tr}\left(K\right)\right]=\Gamma^{N/2}\left[\frac{\text{tr}\left(KT\right)}{\sqrt{\Gamma}}\frac{\sin{\left(N\phi\right)}}{\sin{\phi}}-\text{tr}\left(K\right)\frac{\sin{\left((N-1)\phi\right)}}{\sin{\phi}}\right].
\end{align}
Thus, we get the condition
\begin{align}
    \label{eqappcond}
    \left(\frac{\gamma_{R}}{{\gamma_{L}}}\right)^{1/2}\left[\left(\frac{\gamma_{R}}{{\gamma_{L}}}\Gamma^{N}\right)^{-1/2}+\left(\frac{\gamma_{R}}{{\gamma_{L}}}\Gamma^{N}\right)^{1/2}\right]\sin{\phi}=\frac{\text{tr}\left(KT\right)}{\sqrt{\Gamma}}\sin{\left(N\phi\right)}-\text{tr}\left(K\right)\sin{\left((N-1)\phi\right)}.
\end{align}
Further simplifying with general complex $\phi=\phi_{R}+i\phi_{I}$ and $\phi_{R}\in[0,2\pi],\phi_{I}\in\mathbb{R}$, we obtain
\begin{align}
    \label{eqappphiequ}
    a(\phi)\left(e^{N\phi_{I}}\right)^{2}+b(\phi)e^{N\phi_{I}}+c(\phi)=0,
\end{align}
where 
\begin{align}
    \label{eqappcoff}
    a(\phi)&=\frac{e^{-iN\phi_{R}}}{2i}\left[-\frac{\text{tr}\left(KT\right)}{\sqrt{\Gamma}}+\text{tr}\left(K\right)e^{i\phi}\right],\nonumber\\
    b(\phi)&=-\left[\left(\frac{\gamma_{R}}{{\gamma_{L}}}\Gamma^{N}\right)^{-1/2}+\left(\frac{\gamma_{R}}{{\gamma_{L}}}\Gamma^{N}\right)^{1/2}\right]\left(\frac{\gamma_{R}}{{\gamma_{L}}}\right)^{1/2}\sin{\phi},\nonumber\\
    c(\phi)&=\frac{e^{iN\phi_{R}}}{2i}\left[\frac{\text{tr}\left(KT\right)}{\sqrt{\Gamma}}-\text{tr}\left(K\right)e^{-i\phi}\right].
\end{align}
The solution of $\phi_{I}$ is
\begin{align}
    \label{eqappphisolu}
    e^{N\phi_{I}}=\frac{1}{2a(\phi)}\left[-b(\phi)\pm\sqrt{b^{2}(\phi)-4a(\phi)c(\phi)}\right],
\end{align}
leading to physical real or complex $\phi$ in general. However, since $a(\phi),b(\phi)$ are always finite and $b(\phi)$ contains the terms $\Gamma^{\pm N/2}$, the emergent forms $\phi_{I}\sim c/N$ with finite size are prevented by the large $N$.


The solution of an arbitrary eigenstate is given by $\Phi_{n+1}=T^{n}\varphi,\, n=1,2,\ldots,N-1$ with $KT^{N}\varphi=\varphi$, resulting in
\begin{align}
    \label{eqappimpsoluofeigen}
    \Phi_{n+1}=\Gamma^{n/2}\left[\mathcal{A}_{L}(\phi)e^{in\phi_{R}}e^{-n\phi_{I}}+\mathcal{A}_{R}(\phi)e^{-in\phi_{R}}e^{n\phi_{I}}\right],
\end{align}
where 
\begin{align}
    \label{eqappimpsolucoff}
    \mathcal{A}_{L}(\phi)&=\frac{1}{2i\sin{\phi}}\left(\frac{T}{\sqrt{\Gamma}}-e^{-i\phi}\mathbbm{1}\right)\varphi,\nonumber\\
    \mathcal{A}_{R}(\phi)&=\frac{-1}{2i\sin{\phi}}\left(\frac{T}{\sqrt{\Gamma}}-e^{i\phi}\mathbbm{1}\right)\varphi.
\end{align}
Further,
\begin{align}
    \label{eqappimpsolualphabeta}
    \beta_{1}&=\left(K_{R}^{-1}\varphi\right)_{1},\nonumber\\
    \alpha_{n}&=\Gamma^{n/2}\left[\mathcal{A}_{L,2}(\phi)e^{in\phi_{R}}e^{-n\phi_{I}}+\mathcal{A}_{R,2}(\phi)e^{-in\phi_{R}}e^{n\phi_{I}}\right],\quad n=1,2,\ldots,N-1,\nonumber\\
    \beta_{n}&=\Gamma^{(n-1)/2}\left[\mathcal{A}_{L,1}(\phi)e^{i(n-1)\phi_{R}}e^{-(n-1)\phi_{I}}+\mathcal{A}_{R,1}(\phi)e^{-i(n-1)\phi_{R}}e^{(n-1)\phi_{I}}\right],\quad n=2,3,\ldots,N,\nonumber\\
    \alpha_{N}&=\left(K_{R}^{-1}\varphi\right)_{2},  
\end{align}
where $\mathcal{A}_{L/R,i}(\phi),i=1,2$ denotes the $i$-th component of the column vector $\mathcal{A}_{L/R}(\phi)$. 
Consequently, an eigenstate concerning energy $\varepsilon$ reads
\begin{align}
    \label{eqappimpeigenstate}
    \Psi_{1}^{imp}&=\Gamma^{1/2}\left[\mathcal{A}_{L,2}(\phi)e^{i\phi_{R}}e^{-\phi_{I}}+\mathcal{A}_{R,2}(\phi)e^{-i\phi_{R}}e^{\phi_{I}}\right]v+\left(K_{R}^{-1}\varphi\right)_{1}w,\nonumber\\
    \Psi_{n}^{imp}&=\Gamma^{n/2}\left[\mathcal{B}_{L}(\phi)e^{in\phi_{R}}e^{-n\phi_{I}}+\mathcal{B}_{R}(\phi)e^{-in\phi_{R}}e^{n\phi_{I}}\right],\quad n=2,3,\ldots,N-1,\nonumber\\
    \Psi_{N}^{imp}&=\left(K_{R}^{-1}\varphi\right)_{2}v+\Gamma^{(N-1)/2}\left[\mathcal{A}_{L,1}(\phi)e^{i(N-1)\phi_{R}}e^{-(N-1)\phi_{I}}+\mathcal{A}_{R,1}(\phi)e^{-i(N-1)\phi_{R}}e^{(N-1)\phi_{I}}\right]w,
\end{align}
where 
\begin{align}
    \label{eqappimpeigencoffe}
    \mathcal{B}_{L}(\phi)&=\mathcal{A}_{L,2}(\phi)v+e^{-i\phi}\mathcal{A}_{L,1}(\phi)\frac{w}{\sqrt{\Gamma}},\nonumber\\
    \mathcal{B}_{R}(\phi)&=\mathcal{A}_{R,2}(\phi)v+e^{i\phi}\mathcal{A}_{R,1}(\phi)\frac{w}{\sqrt{\Gamma}}.
\end{align}


\section{Details of the HN model with boundary impurity}
\label{appendixd}
Utilizing the single-particle Schr\"odinger equation of Eq. (\ref{eqhaimp}) with the setting $\psi_{0}\equiv\psi_{N}$ , we obtain the bulk and boundary equations
\begin{align}
    t_{R}\psi_{n-1}+t_{L}\psi_{n+1}&=\varepsilon\psi_{n},\label{eqapphnbulk}\\
    \gamma_{R}\psi_{N}+t_{L}\psi_{2}&=\varepsilon\psi_{1},\label{eqapphnboun1}\\
    t_{R}\psi_{N-1}+\gamma_{L}\psi_{1}&=\varepsilon\psi_{N}\label{eqapphnboun2},
\end{align}
where $n\in\mathscr{B}$. The bulk equation (\ref{eqapphnbulk}) gives ($t_{L}\neq0$)
\begin{align}
    \label{eqapphnimpbulkequ}
    \psi_{n+1}=\frac{\varepsilon}{t_{L}}\psi_{n}-\frac{t_{R}}{t_{L}}\psi_{n-1},
\end{align}
thus leading the propagating relation Eq. (\ref{eqhnimpprop}).

\subsection{The case with singular transfer matrix}
\label{appendixd1}
The real-space EP, or infernal point, emerges under OBC in the case with singular transfer matrix ($t_{R}=0$), where all energies are degenerate at zero but with only one eigenvector \cite{,kunst2019transfer,denner2021,fu2022}. Combining the boundary equations (\ref{eqapphnboun1}) and (\ref{eqapphnboun2}) for $t_{R}=0$, we can deduce the physical condition with boundary impurity,
\begin{align}
    \label{eqapphnsingcond}
    \left(\begin{matrix}\psi_{2}\\\psi_{1}\end{matrix}\right)
    &=\left(\begin{matrix}\frac{\varepsilon}{t_{L}}&-\frac{\gamma_{R}}{t_{L}}\\
    1&0\end{matrix}\right)\left(\begin{matrix}\psi_{1}\\\psi_{N}\end{matrix}\right)\nonumber\\
    &=\left(\begin{matrix}\frac{\varepsilon}{t_{L}}&-\frac{\gamma_{R}}{t_{L}}\\
        1&0\end{matrix}\right)\left(\begin{matrix}\frac{\varepsilon}{\gamma_{L}}&0\\
            1&0\end{matrix}\right)\left(\begin{matrix}\psi_{N}\\\psi_{N-1}\end{matrix}\right)\nonumber\\
    &=\left(\begin{matrix}\frac{\varepsilon}{t_{L}}&-\frac{\gamma_{R}}{t_{L}}\\
        1&0\end{matrix}\right)\left(\begin{matrix}\frac{t_{L}}{\gamma_{L}}&0\\
            0&1\end{matrix}\right)\left(\begin{matrix}\frac{\varepsilon}{t_{L}}&0\\
                1&0\end{matrix}\right)\left(\begin{matrix}\psi_{N}\\\psi_{N-1}\end{matrix}\right)\nonumber\\
    &=\left(\begin{matrix}\frac{\varepsilon}{t_{L}}&-\frac{\gamma_{R}}{t_{L}}\\
        1&0\end{matrix}\right)\left(\begin{matrix}\frac{t_{L}}{\gamma_{L}}&0\\
            0&1\end{matrix}\right)\left(\begin{matrix}\frac{\varepsilon}{t_{L}}&0\\
                1&0\end{matrix}\right)T^{N-2}\left(\begin{matrix}\psi_{2}\\\psi_{1}\end{matrix}\right)\nonumber\\
    &=\left(\begin{matrix}\frac{\varepsilon}{\gamma_{L}}&-\frac{\gamma_{R}}{t_{L}}\\
        \frac{t_{L}}{\gamma_{L}}&0\end{matrix}\right)T^{N-1}\left(\begin{matrix}\psi_{2}\\\psi_{1}\end{matrix}\right)\nonumber\\
    &\equiv K_{d}T^{N-1}\left(\begin{matrix}\psi_{2}\\\psi_{1}\end{matrix}\right),            
\end{align}
where 
\begin{align}
    T=\left(\begin{matrix}\frac{\varepsilon}{t_{L}}&0\\
        1&0\end{matrix}\right),
\end{align}
is the singular transfer matrix with $\Delta=\text{tr}\left(T\right)=\varepsilon/t_{L}$. It implies that the physical eigenvector of $K_{d}T^{N-1}$ concerns the eigenvalue $1$. Due to $\det{\left(K_{d}T^{N-1}\right)}=\det{\left(K_{d}\right)}\Gamma^{N-1}=0$, we obtain $\text{tr}\left(K_{d}T^{N-1}\right)=1$, thus
\begin{align}
    \label{eqapphnimpphycond1}
    \text{tr}\left(K_{d}\Delta^{N-2}T\right)=\Delta^{N-2}\text{tr}\left(K_{d}T\right)=\left(\frac{\varepsilon}{t_{L}}\right)^{N-2}\frac{\varepsilon^{2}-\gamma_{L}\gamma_{R}}{t_{L}\gamma_{L}}=1,
\end{align}
which gives 
\begin{align}
    \label{eqapphnimpphycond2}
    \frac{\varepsilon^{2}}{\gamma_{L}}-\gamma_{R}=t_{L}\left(\frac{t_{L}}{\varepsilon}\right)^{N-2}.
\end{align}
Let $\left(t_{L}/\varepsilon\right)^{N-2}=c$ with $c$ being an undetermined coefficient, and thus 
\begin{align}
    \label{eqapphnsingenergy}
    \varepsilon=t_{L}c^{-\frac{1}{N-2}}e^{-i\frac{2\pi m}{N-2}},\quad m=1,2,\ldots,N-2.
\end{align}
Substituting Eq. (\ref{eqapphnsingenergy}) into Eq. (\ref{eqapphnimpphycond2}), we obtain the physical condition about $c$
\begin{align}
    \label{eqapphnimpphycond3}
    \frac{t_{L}^{2}}{\gamma_{L}}c^{-\frac{2}{N-2}}e^{-i\frac{4\pi m}{N-2}}=t_{L}c+\gamma_{R},
\end{align}
where we will label $c_{m}$ as the solution concerning $m$. Applying the propagating relation, we obtain
\begin{align}
    \left(\begin{matrix}\psi_{n+1}\\\psi_{n}\end{matrix}\right)&=T^{n-1}\left(\begin{matrix}\psi_{2}\\\psi_{1}\end{matrix}\right)=\Delta^{n-2}T\left(\begin{matrix}\psi_{2}\\\psi_{1}\end{matrix}\right)=\left(\begin{matrix}\left(\frac{\varepsilon}{t_{L}}\right)^{n-1}\psi_{2}\\\left(\frac{\varepsilon}{t_{L}}\right)^{n-2}\psi_{2}\end{matrix}\right),
\end{align}
with $n\in\mathscr{B}$. Together with the boundary equations (\ref{eqapphnboun1}) and (\ref{eqapphnboun2}) for $t_{R}=0$, we finally obtain the eigenvectors with respect to energies
\begin{align}
    \label{eqapphnsingvec}
    \varepsilon_{m}&=t_{L}c_{m}^{-\frac{1}{N-2}}e^{-i\frac{2\pi m}{N-2}},\quad m=1,2,\ldots,N,\nonumber\\
    \psi_{n}^{m}&=c_{m}^{-\frac{n-2}{N-2}}e^{-i\frac{2\pi m}{N-2}(n-2)}\psi_{2}^{m},\quad n=3,4,\ldots,N, \nonumber\\
    \psi_{1}^{m}&=\frac{t_{L}}{\gamma_{L}}c_{m}^{-\frac{1}{N-2}}e^{-i\frac{2\pi m}{N-2}}\psi_{N}^{m}.   
\end{align}

Furthermore, set $t_{L}=e^{\alpha}, t_{R}=e^{-\alpha}, \gamma_{L}=\mu e^{\alpha}, \gamma_{R}=\mu e^{-\alpha}$, then the strong non-reciprocity $e^{\alpha}\gg e^{-\alpha}$ corresponds to the singular case ($t_{R}\rightarrow0$). In the strong impurity case ($\mu\gg e^{\pm\alpha}$), the physical condition Eq. (\ref{eqapphnimpphycond3}) reduces to 
\begin{align}
    \label{eqapplimitstrongcond}
    -\mu e^{-2\alpha}=c,
\end{align}
which leading to the solution according to Eq. (\ref{eqapphnsingvec}) 
\begin{align}
    \label{eqapplimitstrongvec}
    \varepsilon_{m}&=e^{\alpha}e^{-\left(\kappa_{L}+ik_{m}\right)},\nonumber\\
    \psi_{n}^{(m)}&=e^{-\left(\kappa_{L}+ik_{m}\right)(n-2)}\psi_{2}^{(m)},\nonumber\\
    |\psi_{1}^{(m)}&|\sim|\frac{e^{2\alpha}}{\mu^{2}}|\ll1,
\end{align}
where $n=3,4,\ldots,N$, $\kappa_{L}=(\log{\mu}-2\alpha)/(N-2)$, and $k_{m}=\left((2m+1)\pi\right)/(N-2)$. We next consider the limit case $\gamma_{R}\rightarrow0$, and the condition Eq. (\ref{eqapphnimpphycond2}) reduces to 
\begin{align}
    \label{eqapplimitcond}
    \left(\frac{\varepsilon}{t_{L}}\right)^{N}=\frac{\gamma_{L}}{t_{L}}.
\end{align}
Then, the eigenvectors with corresponding energies read
\begin{align}
    \label{eqapplimitvec}
    \varepsilon_{m}&=t_{L}^{\frac{1}{N}(N-1)}\gamma_{L}^{\frac{1}{N}}e^{i\frac{2\pi m}{N}},\nonumber\\
    \psi_{n}^{(m)}&=\left(\frac{\gamma_{L}}{t_{L}}\right)^{\frac{n-1}{N}}e^{i\frac{2\pi m}{N}(n-1)}\psi_{1}^{(m)},   
\end{align}
with $m=1,2,\ldots,N$ and $n=2,3,\ldots,N$, which are similar to the results of the $t_{L}=\gamma_{L}=0$ case in Ref. \cite{molignini2023anomalous}. The weak impurity case ($\mu\ll 1$) corresponds the $\gamma_{R}\rightarrow0$ limit, the solution Eq. ~(\ref{eqapplimitvec}) becomes
\begin{align}
    \label{eqapplimitweaksolu}
    \varepsilon_{m}&=e^{\alpha}e^{i(k_{m}'+i\kappa_{L}')},\nonumber\\
    \psi_{n}^{(m)}&=e^{i(k_{m}'+i\kappa_{L}')(n-1)}\psi_{1}^{(m)},
\end{align}
where $\kappa_{L}'=-\log{\mu}/N$ and $k_{m}'=2\pi m/N$. Noteworthily, Eqs. (\ref{eqapplimitstrongvec}) and (\ref{eqapplimitweaksolu}) are exactly equivalent to the SFL solutions in Ref. \cite{li2021impurity}.


\subsection{The case with nonsingular transfer matrix}
\label{appendixd2}

Through the boundary equations (\ref{eqapphnboun1})(\ref{eqapphnboun2}), we can imitate Eq. (\ref{eqapphnsingcond}) to derive the physical condition in current case
\begin{align}
    \label{eqapphnhsshcond1}
    \left(\begin{matrix}\psi_{2}\\\psi_{1}\end{matrix}\right)
    =KT^{N-1}\left(\begin{matrix}\psi_{2}\\\psi_{1}\end{matrix}\right),    
\end{align}
where 
\begin{align}
    \label{eqapphnhsshkmat}
    K=\left(\begin{matrix}
        \frac{\varepsilon}{\gamma_{L}}&-\frac{\gamma_{R}}{t_{L}}\\
        \frac{t_{L}}{\gamma_{L}}&0
    \end{matrix}\right).
\end{align}
Consequently,
\begin{align}
    \label{eqapphnhsshcond2}
    \text{tr}\left(KT^{N-1}\right)&=1+\det{\left(KT^{N-1}\right)}=1+\frac{\gamma_{R}}{\gamma_{L}}\Gamma^{N-1}.  
\end{align}
On the other hand, utilizing the formula Eq.~(\ref{eqnonsingtn}) of nonsingular $T$, we obtain ($z=\cos{\phi}$)
\begin{align}
    \label{eqapphnhsshcond3}
    \text{tr}\left(KT^{N-1}\right)=\Gamma^{\frac{N-1}{2}}\left[\frac{U_{N-2}(z)}{\sqrt{\Gamma}}\text{tr}\left(KT\right)-U_{N-3}(z)\text{tr}\left(K\right)\right].
\end{align}
Together with Eq. (\ref{eqhnhssfenformu}) and after some algebra, we obtain the final form of physical condition
\begin{align}
    \label{eqapphnhssfcond4}
    \left(\frac{\gamma_{L}}{t_{L}}\Gamma^{-\frac{N}{2}}+\frac{\gamma_{R}}{t_{R}}\Gamma^{\frac{N}{2}}\right)\sin{\phi}=\sin{\left((N+1)\phi\right)}-\frac{\gamma_{L}\gamma_{R}}{t_{L}t_{R}}\sin{\left((N-1)\phi\right)},
\end{align}
which is exactly the result in Ref. \cite{guo2021exact}. The solutions of $\phi$ in Eq. (\ref{eqapphnhssfcond4}) are real or complex values for different parameters according to Ref. \cite{guo2021exact}, and the imaginary part of the complex solution $\phi=\phi_{R}+i\phi_{I}$ can possibly take the form $\phi_{I}=c/N$ with $c$ dependent on $N$ and $\phi_{R}$ as before.

The OBC condition corresponding to $\gamma_{L}=\gamma_{R}=0$ reduces Eq. (\ref{eqapphnhssfcond4}) to $\sin{\left((N+1)\phi\right)}=0$, which figures out the solutions $\phi=l\pi/(N+1)\in\left[0,\pi\right]$ with $l=1,2,\ldots,N$. The eigenstate concerning energy $\varepsilon=2\sqrt{t_{L}t_{R}}\cos{\phi}$ is deduced as
\begin{align}
    \label{eqapphnobcvec}
    \left(\begin{matrix}\psi_{n+1}\\\psi_{n}
    \end{matrix}\right)=T^{n}\left(\begin{matrix}1\\0
    \end{matrix}\right)=\frac{1}{\sin{\phi}}\left(\begin{matrix}\Gamma^{n/2}\sin{\left((n+1)\phi\right)}\\\Gamma^{(n-1)/2}\sin{\left(n\phi\right)}
    \end{matrix}\right),
\end{align}
and normalized as $\psi_{n}=\mathcal{N} \Gamma^{n/2}\sin{\left(n\phi\right)}$ with $\mathcal{N}$ being the normalization coefficient, which is exact the bulk eigenstate of pure finite-size NHSE.

We can conceal the NHSE factor $\Gamma^{n/2}$ through a generalized gauge transformation and obtain the target Hamiltonian 
\begin{align}
    \label{eqapptargetham}
    \mathcal{H}_{it}=S^{-1}\mathcal{H}_{hn}S=\sum_{n=1}^{N-1}t\left(c_{n}^{\dagger}c_{n+1}+c_{n+1}^{\dagger}c_{n}\right)+r^{N-1}\gamma_{R}c_{1}^{\dagger}c_{N}+\frac{\gamma_{L}}{r^{N-1}}c_{N}^{\dagger}c_{1},
\end{align}
where $S=\text{diag}\left\{r,r^{2},\ldots,r^{N}\right\}$ is the transformation matrix, $t=\sqrt{t_{L}t_{R}}$, and $r=\sqrt{t_{R}/t_{L}}$. We identify that $r^{N-1}\gamma_{R}=\delta+\gamma$ and $r^{-(N-1)}\gamma_{L}=\delta-\gamma$ for fixed $t_{L},t_{R}$, then Eq. (\ref{eqapptargetham}) is exactly the focused Hamiltonian in Ref. \cite{guo2023scale}, which is a original Hermitian Hamiltonian with non-Hermitian boundary impurity $(\delta+\gamma)c_{1}^{\dagger}c_{N}+(\delta-\gamma)c_{N}^{\dagger}c_{1}$. The transfer matrix of Eq. (\ref{eqapptargetham}) becomes 
\begin{align}
    \label{eqapphnittransfermat}
    T_{it}=\left(\begin{matrix}
        \frac{\varepsilon}{t}&-1\\
        1&0
    \end{matrix}\right),
\end{align}
with $\Gamma_{it}=1$ and $\Delta_{it}=\varepsilon/t$, and the physical condition of $\mathcal{H}_{it}$ is also Eq. (\ref{eqapphnhssfcond4}). In the PT-broken region $\gamma\in\left[|\delta-t|,\delta+t\right]$, the solutions of Eq. (\ref{eqapphnhssfcond4}) are the coexistence of real and complex values, and specially at $\gamma=\gamma_{a}=\sqrt{\delta^{2}-t^{2}}$, the complex solutions are $\phi=2\pi m/N-i\log{\mu}/N$ with $m=1,2\ldots,N$ and $\mu=\delta/t+\sqrt{(\delta/t)^{2}-1}$ \cite{guo2023scale}. The eigenstate concerning energy $\varepsilon=2t\cos{\phi}$ of Eq. (\ref{eqapptargetham}) is derived as 
\begin{align}
    \label{eqapphnitvecs}
    \left(\begin{matrix}
        \psi_{n+1}\\\psi_{n}
    \end{matrix}\right)=\left[A_{L}(\phi)e^{i(n-1)\phi_{R}}e^{-(n-1)\phi_{I}}+A_{R}(\phi)e^{-i(n-1)\phi_{R}}e^{(n-1)\phi_{I}}\right]\left(\begin{matrix}
        \psi_{2}\\\psi_{1}
    \end{matrix}\right), \quad n=1,2,\ldots,N-1,
\end{align}
where 
\begin{align}
    \label{eqapphnitvecoff}
    A_{L}(\phi)&=\frac{T_{it}-e^{-i\phi}\mathbbm{1}}{2i\sin{\phi}},\nonumber\\
    A_{R}(\phi)&=\frac{-T_{it}+e^{i\phi}\mathbbm{1}}{2i\sin{\phi}}.
\end{align}
The real (complex) solutions of $\phi$ correspond to the ESs (PSF effect) of Eq. (\ref{eqapptargetham}), and accordingly correspond to the pure NHSE (HSSF effect) of $\mathcal{H}_{hn}$ after the inverse of gauge transformation.

\section{The emergent HSSF modes of NH-SSH model with boundary impurity}
\label{appendixe}
% Figure environment removed

We again perform the generalized gauge transformation same as that in Sec. \ref{section4b} on $\mathcal{H}_{nssh}$ together with boundary impurity, and obtain the target Hamiltonian $\bar{\mathcal{H}}_{imp}$, the Hamiltonian $\bar{\mathcal{H}}$ in Sec. \ref{section4b} together with boundary impurity strength $\bar{\gamma}_{L}=r^{-N}\gamma_{L},\bar{\gamma}_{R}=r^{N}\gamma_{R}$. The solutions of the emergent PSF modes are Eq. (\ref{eqappimpeigenstate}) with $\bar{\Gamma}=1$, $\varphi=\bar{K}_{R}\Phi_{1}$, where
\begin{align}
    \Phi_{1}=\left(\begin{matrix}
        \psi_{1}\\\psi_{2N}
    \end{matrix}\right),\quad
    \bar{K}_{R}=\left(\begin{matrix}
        1&0\\
        0&\bar{\gamma}_{R}
    \end{matrix}\right),
\end{align}  
and $\psi$ is the numerical $1\times 2N$ eigenvector. The energy spectrum and the comparison between analytic and numerical results of the PSF modes are illustrated in Fig. \ref{Fig-nsshimphssf} for selected parameters, which expectedly match with each other. Therefore, the PSF effect of $\bar{\mathcal{H}}_{imp}$ implies the finite-size HSSF effect of $\mathcal{H}_{nssh}$ with boundary impurity.

\end{widetext}

\bibliography{reference}
\end{document}