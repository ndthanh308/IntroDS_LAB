\documentclass[twocolumn,showpacs,
amsmath,amssymb,superscriptaddress,prl]{revtex4-2}

\usepackage{graphicx}% Include figure files
\usepackage{dcolumn}% Align table columns on decimal point
\usepackage{bm}% bold math
%\usepackage{hyperref}% add hypertext capabilities
%\usepackage[mathlines]{lineno}% Enable numbering of text and display math
%\linenumbers\relax % Commence numbering lines
\usepackage[version=3]{mhchem} % Formula subscripts using \ce{}
\usepackage{amsmath}
\usepackage{esvect}

\usepackage{array}
\usepackage{graphicx}
\usepackage{graphicx}% Include figure files
\usepackage{dcolumn}% Align table columns on decimal point
\usepackage{bm}% math
%\usepackage{mathenv}% bold math
\usepackage{color}% color
\usepackage{pifont}
\usepackage{natbib}
\usepackage{hyperref}
\hypersetup{
colorlinks = true,
urlcolor   = blue,
linkcolor  = blue,
citecolor  = blue
}
\usepackage{nicefrac}

\newcommand*\mycommand[1]{\texttt{\emph{#1}}}

\usepackage{color}
\newcommand{\red}[1]{{\color{red} #1}}
\newcommand{\blue}[1]{{\color{black} #1}}
\newcommand{\green}[1]{{\color{green} #1}}
\newcommand{\del}[1]{{\color{red} \sout{#1}}}

%\bibliographystyle{apsrev4-2}

\begin{document}

\preprint{Preprint}

%\title{Pseudo-spin model of metallophilicity in silver bilayered honeycomb layered materials}
%\title{Pseudo-spin model of argentophilicity in bilayered honeycomb layered materials}
\title{Pseudo-spin model of argentophilicity in honeycomb bilayered materials}

\author{Godwill Mbiti Kanyolo}
\email{gmkanyolo@mail.uec.jp; gm.kanyolo@aist.go.jp}
\affiliation{Research Institute of Electrochemical Energy (RIECEN), National Institute of Advanced Industrial Science and Technology (AIST), 1-8-31 Midorigaoka, Ikeda, Osaka 563-8577, Japan}
\affiliation{The University of Electro-Communications, Department of Engineering Science,\\ 
1-5-1 Chofugaoka, Chofu, Tokyo 182-8585, Japan}

\author{Titus Masese}
\email{titus.masese@aist.go.jp}
\affiliation{Research Institute of Electrochemical Energy (RIECEN), National Institute of Advanced Industrial Science and Technology (AIST), 1-8-31 Midorigaoka, Ikeda, Osaka 563-8577, Japan}
\affiliation{AIST-Kyoto University Chemical Energy Materials Open Innovation Laboratory (ChEM-OIL), Yoshidahonmachi, Sakyo-ku, Kyoto-shi 606-8501, Japan}

%\date{\today}% It is always \today, today, but any date may be explicitly specified

\begin{abstract}
We introduce a pseudo-spin model of the argentophilic bond expected in silver-based bilayered materials arising from a spontaneous pseudo-magnetic field \blue{interacting with} %between 
pseudo-spins of two unconventional Ag ions, namely 
$\rm Ag^{2+}$ and $\rm Ag^{1-}$ electronically distinct from (\textit{albeit} energetically degenerate to) the conventional $\rm Ag^{1+}$ cation typically observed in monolayered materials. This model suggests the possibility of tuning the dimensionality and hence the conductor-semiconductor-insulator properties of honeycomb layered materials by application of external electric and magnetic fields, analogous to driving a superconducting system to the normal regime by critical electric and magnetic fields. 
\end{abstract}

\maketitle

%\tableofcontents

\textit{Introduction.}\textemdash%Due to the energy proximity of %the 
%\blue{silver} $4d^{10}$ and $5s^1$ orbitals %, of the silver atom,
%\blue{and} depending on environmental factors in materials such as crystal field splitting of the $4d^{10}$ orbitals and subsequent hybridisation of one of the orbitals with the $5s$ orbital ($sd$ hybridisation)
\blue{Due to energy proximity of silver $4d^{10}$ and $5s^1$ electrons and environmental factors in silver-based honeycomb bilayered materials such as crystal field splitting of the $4d$ orbitals encouraging $sd$ hybridisation}, the electronic configuration of silver atom ($\rm Ag$) can be in either one of two degenerate states, namely the expected \blue{[Kr]}$4d^{10}5s^1$ configuration \blue{responsible for the} %with 
oxidation state $\rm Ag^{1+}$ or the proposed \blue{[Kr]}$4d^95s^2$ configuration.\cite{masese2022honeycomb} %, suggesting
\blue{This suggests} two %new 
\blue{other} oxidation states exist, namely $\rm Ag^{2+}$ and $\rm Ag^{1-}$, %respectively 
obtained by designating the electrons in either the $5s^2$ or $4d^9$ orbitals %(or both) 
\blue{respectively} as the valence electrons \blue{(or both, in the case of $\rm Ag^{3+}$)}. While these degenerate states have been found to exist independently in various materials such as the anti-ferromagnetic material $\rm Ag^{2+}F_2^{1-}$ and silver clusters ${\rm Ag}^{1-}_N$ ($N = 1$)\cite{kurzydlowski2021fluorides, grzelak2017metal, schneider2005unusual, dixon1996photoelectron, ho1990photoelectron, minamikawa2022electron, *minamikawa2022correction}, by a theorem analogous to Peierls\cite{lee2011band, peierls1979surprises, garcia1992dimerization}, their co-existence %in a particular material 
requires their energy degeneracy to be spontaneously broken for the material to ultimately achieve %further 
stability.\cite{kanyolo2022advances, kanyolo2022advances2, masese2022honeycomb} 

For exemplar silver-based bilayered materials %such as but 
not limited to ${\rm Ag_6^{1/2+}}M_2\rm TeO_6$ ($M = \rm Ni, Co, Mg,Cu$ \textit{etc}) whose structure is shown in Figure \ref{Fig_1}, it has been proposed that the coexistence of the silver degenerate states naturally results in anomalous valence states (subvalent states) such as $\rm Ag_2^{1/2+} = Ag^{2+}Ag^{1-}$ or $\rm Ag_3^{2/3+} = Ag^{2+}Ag^{1-}Ag^{1+}$, which also requires the dimerisation of the two silver atoms in each primitive cell of the bipartite honeycomb lattice, %we refer to as
enabled by argentophilicity.\cite{masese2022honeycomb} It is this dimerisation that we propose to be the manifestation of \blue{pseudo-spin interactions with pseudo-magnetic fields} %argentophilicity 
in silver-based bilayered materials. In silver metal, %$sd$ hybridisation 
\blue{pseudo-spin} as the explanation for argentophilicity is superfluous since the concept of metallic bonds already suffices. Particularly, metallic bonds arise whenever the valence electrons are extremely delocalised, leading to their collective sharing amongst the metallic ions. Meanwhile, the argentophilic bonds in silver bilayers are not only shorter than silver metallic bonds ($\leq 2.83$\,\,\AA\,\,and $\leq 2.89$\,\,\AA\,\,in bilayered silver structures and elemental silver respectively)\cite{masese2022honeycomb}, but also require the valence electrons to be localised. This has led some researchers to consider electron localisation schemes in order to augment this discrepancy with the metallic bond picture.\cite{kovalevskiy2020uncommon, lobato2021comment, yin2021reply} 

% Figure environment removed

In the scheme involving silver degenerate states\cite{masese2022honeycomb}, localisation of silver valence electrons is already implied by the involvement of the electronic configuration \blue{[Kr]}$4d^95s^2$ (particularly, $4d_{z^2}^15s^2$) %via 
\blue{obtained by} $sd$ hybridisation instead of the conventional $4d^{10}5s^1$ electronic configuration with the usually itinerant $5s^1$ valence electrons localised by the closed-shell pairing in $5s^2$ state. In addition, like in graphene, the pseudo-spins on the bipartite honeycomb lattice are inherited from the behaviour of actual spins of the $4d_{z^2}^1$ orbital electrons\cite{kanyolo2022advances2, georgi2017tuning, mecklenburg2011spin, allen2010honeycomb, kvashnin2014phase} \textit{albeit} the pseudo-spin degree of freedom in silver is expected to clearly manifest for an isolated $4d_{z^2}^1$ orbital due to $4d^9$ crystal field splitting in linear or prismatic coordination to \blue{anions such as} $\rm F^{1-}$ or $\rm O^{2-}$.\cite{masese2022honeycomb}  %anions.\cite{masese2022honeycomb} 

Herein, it is prudent to employ units where reduced Planck's constant and speed of mass-less particles in the material are set to unity ($\hbar = \overline{c} = 1$). 

\textit{Pseudo-spin model.}\textemdash %Now, c
\blue{C}onsider $sd$ hybridisation as the interaction between localised spin impurities \blue{$S_{\alpha}^z$ and $S_{\beta}^z$ ($\alpha, \beta = i,j,k$) on a honeycomb lattice} comprising isolated $4d_z^1$ orbitals in the Ag configuration, \blue{[Kr]}$4d^95s^2$, mediated by conduction electrons comprising the $5s^2$ orbitals as illustrated in Figure \ref{Fig_2}. \blue{This leads} %leading 
to the exchange interaction terms $J(\vec{r}_{ij})$ \blue{between $i$ and $j$ atoms and possible linear terms $h(\vec{r}_{ij})$} in the Heisenberg Hamiltonian\cite{garcia1992dimerization, said1984nonempirical},
\begin{subequations}\label{Heisenberg_eq1}
\begin{multline}
    H = \sum_{i, j \in hc \in R^d \setminus i = j}\left [f(\vec{r}_{ij}) + 4J(\vec{r}_{ij})\left (\left\langle S_i^zS_j^z \right\rangle + 1/4 \right )\right ]\\
    = \sum_{i, j \in hc \in R^d \setminus i = j} \left [4J(\vec{r}_{ij})\left\langle S_i^zS_j^z \right\rangle + h(\vec{r}_{ij})\left \langle (S^z_i + S^z_j) \right \rangle\right ],
\end{multline}
under the constraint, 
\begin{align}
    f(\vec{r}_{ij}) = - J(\vec{r}_{ij}) + h(\vec{r}_{ij})\left \langle (S^z_i + S^z_j) \right \rangle,
\end{align}
\end{subequations}
where the sum is evaluated over silver atoms $i$ and $j$ excluding $i \neq j$ in a potentially bifurcated bipartite honeycomb hexagonal lattice ($hc$) comprising a pair of hexagonal lattices ($hx, hx^*$) displaced along the $z$ direction in Euclidean space $R^d$  \blue{(bifurcated honeycomb lattice)} when $d = 3$ \blue{dimensions (3D)}, and non-bifurcated for $d = 2$ \blue{dimensions (2D)} in the plane, $R^2$, as displayed in Figure \ref{Fig_3}. Here, $f(\vec{r}_{ij})$, $J(\vec{r}_{ij})$ and $h(\vec{r}_{ij})$ are yet unknown energy functions of the relative positions $\vec{r}_{ij} = \vec{r}_i - \vec{r}_j$ of next neighbour Ag atoms $i, j$ at positions $\vec{r}_i$ and $\vec{r}_j$ respectively, and $S^z_i$ or $S^z_j$ are the $z$ direction spin operators for pseudo-spin impurities at $i, j$ ($4d_{z^2}^1$ electron spins in each $4d^95s^2$ $\rm Ag$ atom) within a sea of conduction electrons. In this picture, the bond length is given by the silver separation distance, $r_{ij} = |\vec{r}_{ij}| \equiv r_{\rm bond}$ when $H = 0$. Thus, figuring out the functional form of the unknown energy functions on the honeycomb lattice which satisfy $H = 0$ either %experimentally 
\blue{analytically} or via \textit{ab initio} calculations amounts to solving the bifurcation problem\blue{, whose results can then be compared with experiment}. In this endeavour, next-neighbour summation approximation may be taken provided it is justified on physical grounds. While particularly unwieldy, similar computational challenges have been overcome for the analogous problem in $d = 1$ dimensions (1D), corresponding to the case of dimerisation of carbon atoms in polyacetylene and other similar conjugated hydrocarbons.\cite{garcia1992dimerization, said1984nonempirical} 

% Figure environment removed

Nonetheless, we are solely interested in pursuing the analytic behaviour of eq. (\ref{Heisenberg_eq1}), starting with the simplest case of the two-body problem. The next-neighbour pseudo-magnetisation and pseudo-spin correlation expectation values are given by, 
\begin{subequations}\label{pseudo_eq}
\begin{align}
    \tilde{m}^z = \left \langle (S^z_i + S^z_j)\right \rangle = 0, \pm 1,\\
    \tilde{c}^z = \left \langle S_i^zS_j^z \right \rangle = -1/4, +3/4,
\end{align}
respectively, which can be associated with the pseudo-spin wave functions as follows,
\begin{align}\label{pseudo_H_eq}
\begin{matrix}
\underline{{\rm wave\,\,function}, \blue{\Sigma_{ij}}} & \underline{\rm magnetisation} & \underline{\rm correlation} \\
\frac{1}{\sqrt{2}}\left (|\uparrow\,\downarrow\,\rangle - |\downarrow\,\uparrow\,\rangle \right ), & \tilde{m}^z = 0, & \tilde{c}^z = -1/4, \\ 
\frac{1}{\sqrt{2}}\left (|\uparrow\,\downarrow\,\rangle + |\downarrow\,\uparrow\,\rangle \right ), & \tilde{m}^z = 0, & \tilde{c}^z = + 3/4,\\
|\uparrow\,\uparrow\, \rangle, & \tilde{m}^z = 1, & \tilde{c}^z = + 3/4,\\ 
|\downarrow\,\downarrow\, \rangle, & \tilde{m}^z = -1, & \tilde{c}^z = + 3/4.
\end{matrix}    
\end{align}
\end{subequations}
It is prudent to also consider the actual spin magnetisation and correlations in the $z$ direction given by,
\begin{subequations}\label{actual_eq}
\begin{align}
    m^z = \left \langle (s^z_i + s^z_j)\right \rangle = 0, \pm 1,\\
    c^z = \left \langle s_i^zs_j^z \right \rangle = -1/4, +3/4,
\end{align}
respectively where $s_i, s_j$ are the actual spin operators, which can be associated with the actual spin wave functions as follows,
\begin{align}\label{actual_H_eq}
\begin{matrix}
\underline{{\rm wave\,\,function}, \blue{\sigma_{ij}}} & \underline{\rm magnetisation} & \underline{\rm correlation} \\
\frac{1}{\sqrt{2}}\left (|u\,d\,\rangle - |d\,u\,\rangle \right ), & m^z = 0, & c^z = -1/4, \\ 
\frac{1}{\sqrt{2}}\left (|u\,d\,\rangle + |d\,u\,\rangle \right ), & m^z = 0, & c^z = + 3/4,\\
|u\,u\, \rangle, & m^z = 1, & c^z = + 3/4,\\ 
|d\,d\, \rangle, & m^z = -1, & c^z = + 3/4.
\end{matrix}    
\end{align}
\end{subequations}
\blue{Here, $|u \rangle$ and $|d \rangle$ are respectively the spin up and down wave functions of itinerant $5s^2$ silver conduction electrons, to be distinguished from the pseudo-spin up and down wave functions, $|\uparrow \, \rangle$ and $|\downarrow \, \rangle$ respectively considered to originate from the spins of localised $4d_{z^2}^1$ silver electrons.} Thus, using the notation, $\Sigma_{ij} = (\tilde{m}^z, \tilde{c}^z)_{ij}$ to label the pseudo-spin wave functions and $\sigma_{ij} = (m^z, c^z)_{ij}$ to label the actual spin wave functions, we can consider the parity of the spatial wave functions in order to extract valuable information about the ground state of the pseudo-spin system given by eq. (\ref{Heisenberg_eq1}). 

In particular, the spatial %two-electron 
\blue{two-particle} wave function $\psi_{\pm}(\vec{r}_{ij})$ of %two 
\blue{adjacent silver} fermions in the absence of pseudo-spin degrees of freedom must satisfy,
\begin{multline}\label{wave function_eq}
    \psi_{\pm}(\vec{r}_{ij})\sigma^{\mp}_{ij} = -\psi_{\pm}(\vec{r}_{ji})\sigma^{\mp}_{ji}\\
    = \frac{1}{2}\left [\phi_i(\vec{r}_i)\phi_j(\vec{r}_j) \pm \phi_i(\vec{r}_j)\phi_j(\vec{r}_i)\right ]\sigma^{\mp}_{ij},
\end{multline}
implying $\sigma^{\mp}_{ij} = \mp \sigma^{\mp}_{ji}$, where $\phi_i(\vec{r}), \phi_j(\vec{r})$ is the spatial wave function of a conduction electron at $\vec{r}_i, \vec{r}_j$ hopping between Ag cations on adjacent lattice sites and $\vec{r}_{ij} = \vec{r}_i - \vec{r}_j$. However, assuming the honeycomb lattice introduces the pseudo-spin degree of freedom with wave functions, $\Sigma_{ij}$ the composite \blue{two-silver} wave function can be written in a generic form as, 
\begin{multline}\label{ground_state_eq}
    \psi^p(\vec{r}_{ij})\Sigma_{ij}^a\sigma_{ij}^b = -\psi^p(\vec{r}_{ji})\Sigma_{ji}^a\sigma_{ji}^b\\
    \frac{1}{2}\left [\phi_i(\vec{r}_i)\phi_j(\vec{r}_j) + p\phi_i(\vec{r}_j)\phi_j(\vec{r}_i)\right ]\Sigma_{ij}^a\sigma_{ij}^b,
\end{multline}
where $p = \pm$, $a = \pm$ and $b = \pm$ indicate the symmetry ($+$)/anti-symmetry ($-$) of the respective wave functions. \blue{For instance, for plane wave solutions, the symmetry/anti-symmetry of the spatial wave function corresponds to even/odd parity. This is evident for a model comprising left- and right-moving plane waves of conduction electrons hopping between $\rm Ag^{2+}$ and $\rm Ag^{1-}$ along a honeycomb edge,
\begin{subequations}
\begin{align}
    \phi_i(\vec{r}_i) = \frac{1}{\sqrt{\Omega}}\exp(\#\sqrt{-1}\, \vec{k}_{\rm F}\cdot\vec{r}_i),\\
    \phi_j(\vec{r}_j) = \frac{1}{\sqrt{\Omega}}\exp(\#\sqrt{-1}\, \vec{k}_{\rm F}\cdot\vec{r}_j),
\end{align}
\end{subequations}
where $\vec{k}_{\rm F}$ is the Fermi wave vector and their chirality is given by $\# = \pm$, where $\Omega$ is the wave function normalisation factor. Requiring next-neighbour interacting $\rm Ag$ states to have opposite chirality inherited from the conduction electrons (a requirement for the generation of a mass term between $\rm Ag^{2+}$ and $\rm Ag^{1-}$\cite{masese2022honeycomb}), we obtain plane wave solutions of the spatial wave function,
\begin{align}\label{wave function_eq2}
\phi^p(\vec{r}_{ij}) = 
   \begin{cases}
    \,\,\,\,\Omega^{-1}\cos(\vec{k}_{\rm F}\cdot\vec{r}_{ij}), \,\,\,\,p = +,\\
     \,\,\,\,\Omega^{-1}\sqrt{-1}\sin(\vec{k}_{\rm F}\cdot\vec{r}_{ij}), \,\,\,\,p = -,
\end{cases}
\end{align}
appearing in eq. (\ref{wave function_eq}).}

Proceeding, the only possible signs for a pair of \blue{silver} fermions are given by the following cases: 
\begin{align}\label{cases_eq}
\begin{matrix}
\underline{\rm cases}: & p & a & b \\
1. & + & + & - \\
2. & + & - & + \\
3. & - & - & - \\
4. & - & + & + .
\end{matrix} 
\end{align}
Moreover, \blue{assuming potentials $h(\vec{r}_{ij})$ and $J(\vec{r}_{ij})$ are symmetric in $\vec{r}_{ij}$ and the Hamiltonian $H$ commutes with the parity operator,} %since 
the odd parity states ($p = -$) in eq. (\ref{cases_eq}) are forbidden by %the requirement that symmetric potentials %should yield 
\blue{a known theorem that the ground state is} even parity\blue{.\cite{sakurai1995modern} Thus,} %ground states\cite{sakurai1995modern}, 
we should cross out all the $p = -$ states\blue{, which} %. This 
leaves the only viable %electron states at the Fermi energy as, 
\blue{ground state as}, 
\begin{align}\label{cases_eq2}
\begin{matrix}
\underline{\rm ground\,\,states}: & p & a & b \\
1. & + & + & - \\
2. & + & - & + .
\end{matrix} 
\end{align}
Note that, for vanishing respective magnetic fields, each of these ground states are 3-fold degenerate. However, degeneracy of the ground state is forbidden by a theorem by Peierls, which guarantees that the lattice must distort to lift the degeneracy.\cite{lee2011band, peierls1979surprises, garcia1992dimerization} %Since
\blue{Identifying} the distortion on the honeycomb lattice %corresponds to a
\blue{with the} finite pseudo-magnetic interaction, $h(\vec{r}_{ij}) \neq 0$, we find %that 
the unique ground state, 
\begin{align}\label{cases_eq3}
\begin{matrix}
\underline{\rm ground\,\,state}: & p & a & b \\
1. & + & + & - ,
\end{matrix} 
\end{align}
where the pseudo-magnetic interaction (%representing a
\blue{responsible for the} bifurcation of the monolayered silver honeycomb lattice into bilayers\cite{masese2022honeycomb}) must appear spontaneously thus breaking the degeneracy. On the other hand, a finite actual magnetic field selects instead the other ground state, 
\begin{align}\label{cases_eq4}
\begin{matrix}
\underline{\rm ground\,\,state}: & p & a & b \\
2. & + & - & + ,
\end{matrix} 
\end{align}
which is guaranteed to be non-degenerate (by the presence of the magnetic field). 

Consequently, due to the %parity 
wave function symmetry considerations given in eq. (\ref{cases_eq4}), such a ground state must be without strain or distortions ($h(\vec{r}_{ij}) = 0$), implying a finite external magnetic field is present, 
\begin{align}\label{B_eq}
    B \equiv (\vec{\nabla}\times\vec{A})_z \neq 0,
\end{align}
which tunes the argentophilic bond ($\vec{A}$ is the electromagnetic vector potential) and hence the monolayer-bilayer phase transition, %changing the dimensionality of the crystal at will. 
thus altering the dimensionality of the Ag lattice from 3D \blue{back} to 2D, corresponding to the displayed lattices in Figure \ref{Fig_3}. Intuitively, the pseudo-magnetic interaction is analogous to the order parameter in the Ginzburg-Landau or the energy gap in Bardeen–Cooper–Schrieffer (BCS) theories of superconductivity, which are related and can be tuned by an external magnetic field driving the superconducting state to normal.\cite{tinkham2004introduction} 

% Figure environment removed

\textit{Paired electrons.}\textemdash %In particular, t
\blue{T}he time component of the \blue{conduction electron} wave functions $\phi_{i,j}(\vec{r}_{i,j})$ \blue{or the overall silver ground state wave function} is the phase \blue{factors}, $\exp(-iE_{\pm}t)$ where the dispersion relation at any honeycomb vertex with a $\rm Ag^{2+}$ or $\rm Ag^{1-}$ %atom 
\blue{ion} satisfies the linear dispersion relation, $E_{\pm} = \pm |\vec{k}_{\rm F}|$ %where 
\blue{and} $\pm$ reflects the electron($+$)/hole($-$) energy dispersion relations. Thus, the Green's function of interest is the appropriately defined Fourier transform of the phase factor\cite{mattuck1992guide}, 
\begin{multline}
    \mathcal{G}_{\pm}(\omega) = \frac{1}{\omega - E_{\pm} + i\epsilon}\\
    = -i\int_{-\infty}^{\infty}dt\,\exp(i\omega t - \epsilon t)\theta(t)\exp(-iE_{\pm}t),
\end{multline}
where $i = \sqrt{-1}$, $\theta(t)$ is the Heaviside step function restricting the Fourier transform to the range $0 \leq t \leq \infty$ and $\epsilon \simeq 0$ is the infinitesimal with the properties $\epsilon \times \infty = \infty$ and $\epsilon\times 0  = 0$. In subsequent calculations \blue{using the Green's function}, we shall set $\epsilon = 0$. %in the Green's function. 

Now, departing from \blue{only considering adjacent Ag ions in} the two-body problem, %and 
\blue{but} assuming that the %next neighbour 
\blue{adjacent} pseudo-spins on the honeycomb lattice are \textit{anti-parallel} as shown in Figure \ref{Fig_3}(a), the modification of the electron dispersion relation by interactions with the pseudo-magnetisation component of $H$ in eq. (\ref{Heisenberg_eq1}) \blue{identically} vanishes since, 
\begin{multline}\label{R2_eq}
    U = \sum_{i, j \in hc \in R^2 \setminus i = j} h(\vec{r}_{ij})\left \langle (S^z_i + S^z_j) \right \rangle\\
    = \sum_{i, j \in hx \setminus i = j} h(\vec{r}_{ij})\left \langle (S^z_i + S^z_j) \right \rangle + \sum_{i, j \in hx^* \setminus i = j} h(\vec{r}_{ij})\left \langle (S^z_i + S^z_j) \right \rangle\\
    + \sum_{i, j \in hc \in R^2 \setminus i = j, \vec{r}_{ij} \in hx, hx^*} h(\vec{r}_{ij})\left \langle (S^z_i + S^z_j) \right \rangle\\
    = \sum_{i, j \in hx \setminus i = j} h(\vec{r}_{ij})\times 1 + \sum_{i, j \in hx^* \setminus i = j} h(\vec{r}_{ij})\times -1\\
    + \sum_{i, j \in hc \in R^2 \setminus i = j, \vec{r}_{ij} \in hx, hx^*} h(\vec{r}_{ij})\times 0 = 0.
\end{multline}
On the other hand, considering %next-neighbour 
\blue{adjacent} pseudo-spins to be \textit{parallel} instead (\textit{e.g.} pseudo-spin \textit{up} commensurate with the bifurcated lattice shown in Figure \ref{Fig_3}(b)), only the vectors not within the 2D hexagonal lattices $\vec{r}_{ij} \notin R^2$ (\textit{i.e.} $i,j \in hc \in R^3 \setminus i \neq j, \vec{r}_{ij} \in R^2$) will have a finite $z$ component,  
\begin{multline}\label{R3_eq}
     U = \sum_{i, j \in\,\,hc\,\,\in R^3 \setminus i = j} h(\vec{r}_{ij})\left \langle (S^z_i + S^z_j) \right \rangle\\
     = \sum_{i, j \in hx, hx^*\in R^2 \setminus i = j} h(\vec{r}_{ij})\times 1\\
     + \sum_{i,j \in hc \in R^3 \setminus i \neq j, \vec{r}_{ij} \in R^2} h(\vec{r}_{ij})\times 1\\
     = U_{R^2} + U_{R^3}. 
\end{multline}
Due to the finite $z$ component, the %first 
\blue{second} sum of terms \blue{given by $U_{R^3}$} in eq. (\ref{R3_eq}) together with the sum involving the functions $J(\vec{r}_{ij})$ in eq. (\ref{Heisenberg_eq1}) lead to the expression given in eq. (\ref{orbits_eq}) for the argentophilic bond. Meanwhile, the %second 
\blue{first} sum of terms in eq. (\ref{R3_eq}) \blue{given by $U_{R^2}$} leads to a mass gap for 2D electron \blue{(silver) dynamics} on the bifurcated lattice. %in reciprocal (momentum) space. 

To see this, we shall assume $|U_{R^2}|$ takes the same functional form in position as well as momentum space, which we \blue{can} justify using eq. (\ref{Theta_eq}). Intuitively, this arises from the fact that the reciprocal of the hexagonal lattice is another hexagonal \textit{albeit} with unit vectors rotated by $\pi/2$ and appropriately re-scaled momentum vectors. Now, we wish to calculate the effective Green's function due to interactions of $\mathcal{G}_+(\omega)$, $\mathcal{G}_-(\omega)$ and $U_{R^2}$ given by,
\begin{subequations}
\begin{multline}
    \mathcal{G}_+^{\rm eff}(\omega) \equiv \mathcal{G}_+ + \mathcal{G}_+U_{R^2}\mathcal{G}_-U_{R^2}^*\mathcal{G}_+\\
    + \mathcal{G}_+U_{R^2}\mathcal{G}_-U_{R^2}^*\mathcal{G}_+U_{R^2}\mathcal{G}_-U_{R^2}^*\mathcal{G}_+ + \cdots\\
    = \frac{1}{\mathcal{G}_+^{-1} - |U_{R^2}|^2\mathcal{G}_-} = \frac{\omega - E_-}{\omega^2 - E_{\pm}^2 - |U_{R^2}|^2},
\end{multline}
where we have used the generic expression $w(1 + z + z^2 + \cdots) = 1/(w^{-1} - z)$ under a regularisation scheme for values of $|z| > 1$ where $z, w$ can be arbitrary complex functions.\cite{mattuck1992guide} We can thus describe the %electron-hole 
\blue{electron (silver)} interactions alongside their chiral degrees of freedom as four-spinors by \blue{a} diagonalisation \blue{scheme} into quasi-particle interactions of the form, 
\begin{multline}
    \mathcal{G}_+^{\rm eff}(\omega) = \frac{\omega - E_-}{\omega^2 + \mathcal{E}_-\mathcal{E}_+} = \frac{\mu_+^2}{\omega - \mathcal{E}_+} + \frac{\mu_-^2}{\omega - \mathcal{E}_-}\\
    = \frac{\mu_+^2}{\omega - \vec{k}_{\rm F}\cdot\vec{\alpha} - |U_{R^2}|\gamma^0} + \frac{\mu_-^2}{\omega + \vec{k}_{\rm F}\cdot\vec{\alpha} + |U_{R^2}|\gamma^0},
\end{multline}
\end{subequations}
amounting to the result obtained by a Bogoliubov transformation of the particle/quasi-particle fermionic (creation, annihilation) operators ($c^{\dagger}, c$)/($d^{\dagger}, d$) respectively\cite{mattuck1992guide} with $\mathcal{E}_{\pm} = \pm \sqrt{E_{\pm}^2 + |U_{R^2}|^2} = \pm (\vec{k}_{\rm F}\cdot\vec{\alpha} + |U_{R^2}|\gamma^0)$ and, 
\begin{align}
    \mu_{\pm} = \sqrt{\frac{1}{2}\left (1 \pm \frac{E_{\pm}}{\mathcal{E}_{\pm}}\right )}, 
\end{align}
are the so-called BCS coherence factors in the Bogoliubov transformation, 
\begin{align}
    d^{\dagger} = \mu_+c^{\dagger} + \mu_- c,\\
    d = \mu_+ c + \mu_- c^{\dagger}.
\end{align}
Note that, $c^2 = (c^{\dagger})^2 = d^2 = (d^{\dagger})^2 = 0$, $d^{\dagger}d + dd^{\dagger} = c^{\dagger}c + cc^{\dagger} = \mu_-^2 + \mu_+^2 = 1$, and $\gamma^0$ and $\vec{\alpha}$ are $4\times 4$ matrices related to Dirac matrices $\gamma^{\mu} = \gamma^0(1 , \vec{\alpha})$ which satisfy the Clifford algebra $\gamma_{\mu}\gamma_{\nu} + \gamma_{\nu}\gamma_{\mu} = 2\eta_{\mu\nu}$ with \blue{$\gamma_{\mu} = \eta_{\mu\nu}\gamma^{\nu}$ and} $\eta_{\mu\nu}$ the Minkowski space-time metric tensor. Finally, it is clear that: 
\begin{enumerate}
    \item The Dirac mass corresponds to $m_{\Delta(d)} = |U_{R^2}|$ which must vanish at the critical point of the phase transition such as in eq. (\ref{R2_eq}) even for $h(\vec{r}_{ij}) \neq 0$;
    \item Moreover, \blue{from the conclusions using the parity argument above,} $U_{R^2}$ ought to depend on external magnetic fields %as argued before, 
    tuning it in eq. (\ref{R3_eq}) leading to $U_{R^2} = 0$;
    \item This tuning is expected to be possible even when the bifurcated lattice has $g - 1$ vacancies extracted by external electric fields, where $g$ is defined as the genus of an emergent manifold within the context of an idealised model\cite{kanyolo2022advances2, kanyolo2020idealised, kanyolo2022cationic};
    \item The expression expected for the mass term is $m_{\Delta(d)} \equiv 2m\Delta(d)$, with $m$ a constant with dimensions of mass/energy and $\Delta(d) = (d - 2)/2$ the conformal dimension for mass-less scalar fields.\cite{masese2022honeycomb, kanyolo2022advances2}
\end{enumerate}
Thus, determining an appropriate function $h(\vec{r}_{ij})$ in eq. (\ref{R3_eq}) manifesting these properties, alongside $J(\vec{r}_{ij})$ is tantamount to %arriving at 
\blue{finding} a solution for the functional form of the argentophilic bond in the bifurcated honeycomb lattice.


\textit{Proposed solution}\textemdash Introducing the lattice constant $a$, the vectors in the hexagonal lattices satisfy $r_{ij}^2/a^2 = 2n$, where $n \in \mathbb{N}$ is a finite positive integer. Moreover, we shall consider a trial function for the pseudo-magnetic field,
\begin{align}\label{pseudo_h_eq}
    h(\vec{r}_{ij}) = -\frac{m}{C}\left (\frac{\sqrt{2}\,a}{r_{ij}} \right )^{2(\hat{s} + \hat{s}^*)},
\end{align}
consistent with an emergent Liouville conformal field theory (CFT)\cite{kanyolo2022advances2}, 
\begin{multline}\label{define_h_Eq}
   -\frac{m}{C}\sum_{i, j \in hx, hx^*\in R^2 \setminus i = j} \exp(2(\hat{s} + \hat{s}^*)\Phi(\vec{r}_{ij}))\\
   = -\frac{m}{C}\sum_{i, j \in hx, hx^*\in R^2 \setminus i = j}\frac{(\sqrt{2}\,a)^{2(\hat{s} + \hat{s}^*)}}{r_{ij}^{2(\hat{s} + \hat{s}^*)}} \equiv U_{R^2}(\hat{s}, \hat{s}^*)\\
   = \sum_{i, j \in hx, hx^*\in R^2 \setminus i = j} h(\vec{r}_{ij}) = -\frac{2m}{C_{\rm eff}}\sum_{n = 1}^{\infty}\frac{C_n}{n^{\hat{s} + \hat{s}^*}},
\end{multline}
with $1/C$ a proportionality constant, where $\Phi(\vec{r}_{ij}) = -\frac{1}{2}\ln (r_{ij}^2/(\sqrt{2}\,a)^2)$ is the Liouville field defined on the hexagonal lattices\cite{kanyolo2022advances2}, the potential $U_{R^2}(\hat{s}, \hat{s}^*)$ is comprised solely of the sum over silver atoms $i,j$ connected only by 2D vectors $\vec{r}_{ij} \in R^2$ in the hexagonal lattices \textit{i.e.} $i,j \in hx, hx^* \in R^2\ i = j$ (eq. (\ref{R3_eq})) with the lattice now assumed to extend to infinity in $R^2$, $\hat{s}, \hat{s}^*$ are scaling parameters to be solved for, which shall determine the mass term proportional to $U_{R^2}$ and the coefficient $C_n$ is the number of vectors of norm $2n$ on each hexagonal lattice. %Notably, t
\blue{T}he effective proportionality constant $1/C_{\rm eff} \propto 1/C$ arises from translating the sum over the silver atoms $i,j$ to the sum over vectors $\vec{r}_{ij} \in hx, hx^* \in R^2$. 

% Figure environment removed

Notably, the coefficients $C_n$ can be generated by the theta function of the hexagonal lattice, 
\begin{subequations}\label{Theta_eq}
\begin{align}
    \Theta_{\Lambda}(\tau) = \sum_{\vec{r}_{ij} \in \Lambda} \exp\left (i\pi\tau\frac{r_{ij}^2}{a^2}\right )
    = \sum_{n = 0}^{\infty}C_nq^n(\tau),
\end{align}
where $\Lambda = hx \,\,{\rm or}\,\, hx^*$, $q(\tau) = \exp(2\pi i\tau)$, $\tau \in \mathbb{H}_+$ is complex-valued and restricted to the upper-half plane, $\mathbb{H}_+$ and $r_{ij} = |\vec{r}_{ij}|$ is displayed in Figure \ref{Fig_4}. Thus, we can use the Poisson summation formula\cite{pinsky2008introduction, kanyolo2022advances2} to yield the functional form of $\Theta_{\Lambda'}$ in momentum space ($\vec{k}_{ij} \in \Lambda'$), 
\begin{align}
     \Theta_{\Lambda}(-1/\tau) = -i\tau \Theta_{\Lambda'}(\tau),\\
     \Theta_{\Lambda'}(\tau) = \sum_{\vec{k}_{ij} \in \Lambda'} \exp\left (i\pi\tau\frac{k_{ij}^2}{k_0^2}\right )
    = \sum_{n = 0}^{\infty}C_nq^n(\tau),
\end{align}
\end{subequations}
with $|\vec{k}_{ij}| = k_{ij}$, $k_0$ the lattice constant of $\Lambda'$ %in momentum space
satisfying $k_{ij}^2/2k_0^2 = n \in \mathbb{N}$ and $\vec{k}_{ij} \in \Lambda' = {\rm dual}(hx) \,\,{\rm or}\,\, {\rm dual}(hx^*)) = hx' \,\,{\rm or}\,\, hx^{'*}$. \blue{Thus, due to the same functional form between the lattice theta functions $\Theta_{\Lambda}(\tau)$ and $\Theta_{\Lambda'}(\tau)$, their Mellin transform under a change of variable $\tau = i\beta/2\pi$, is proportional to the potential $U_{R^2}$, taking the same functional form in position and momentum spaces.\cite{kanyolo2022advances2}}  

Proceeding, the summation in eq. (\ref{define_h_Eq}) yields expressions where the first few coefficients are given by\cite{Sloane1964Theta},
\begin{align}\label{C_n_eq}
\begin{tabular}{c|ccccccccccc}
    $n$ & 0 & 1 & 2 & 3 & 4 & 5 & 6 & 7 & 8 & 9 & $\cdots$\\
    \hline
    $C_n$ & 1 & 6 & 0 & 6 & 6 & 0 & 0 & 12 & 0& 6& $\cdots$
\end{tabular}\,\,\,\,\,\,\,.
\end{align}
Nonetheless, we are interested in the continuum %limit, 
\blue{approximation}, $C_n \rightarrow C_{n + 1}$ and set $C_{\rm eff} = C_n$ to yield,
\begin{align}\label{continuum_h_Eq2}
   U_{R^2}(\hat{s}, \hat{s}^*) \rightarrow -2m\sum_{n = 1}^{\infty}\frac{1}{n^s} = -2m\zeta(s), 
\end{align}
where $\zeta(s)$ is the Riemann zeta function and $s = \hat{s} + \hat{s}^*$. Famously, $\zeta(s)$ has an analytic continuation,
\begin{align}\label{analytic_eq}
    \zeta(s) = 2^{s}\pi^{s-1}\sin\left( \frac{\pi s}{2} \right)\Gamma(1 - s)\zeta(1 - s), 
\end{align}
in the complex plane $s = \delta + i\gamma$ for real values of $\delta$ and $\gamma$, especially relevant in finding %the 
\blue{its} values %of $s$ 
for $\delta < 1$ with,
\begin{align}\label{Gamma_F_eq}
    \Gamma(s) = \int_0^{\infty} \frac{d\beta}{\beta}\beta^{s}\exp(-\beta),
\end{align}
the Gamma function. %Notably, 
\blue{Thus}, we find, 
\begin{multline}\label{zeta_eq}
-\zeta(s) = \Delta(d) = 
\begin{cases}
\,\,\,\,\,\,0 \,\,\,\,\,\,\,\,\, (s = -2g \neq 0)\\
\,\,\, 1/2 \,\,\,\,\,\,\,\,\,\,\,\,\,\,\,\,\,(s = 0)
\end{cases},
\end{multline}
where $\Delta(d) = (d - 2)/2$, $g$ is taken to be the genus of an emergent 2D manifold $\mathcal{A}_g$ in the Liouville CFT and $-2g \neq 0$ are the so-called trivial zeros of $\zeta(s)$, as shown in Figure \ref{Fig_5}. 

Transitions between these $g \in \mathbb{N}$ states labelling the number of vacancies $\nu = g - 1 \geq 0$ or equivalently the genus of the emerging manifold $\mathcal{A}_g$ with Gaussian curvature $K$ satisfying, 
\begin{align}
    2 - 2g = \frac{1}{2\pi}\int_{\mathcal{A}_g} d^{\,2}r_{ij}\,K\exp(2\Phi(\vec{r}_{ij})),
\end{align}
%are 
\blue{can be} induced by external electric fields,
\begin{align}\label{Liouville_eq}
    \vec{E} \propto \vec{\nabla}\Phi = -K\exp(2\Phi),
\end{align}
which create or annihilate vacancies from the lattices should de-intercalation or intercalation processes in the bilayered material be electrochemically permitted.\cite{kanyolo2022cationic, kanyolo2022advances2, liu2007} One can thus conclude that the non-trivial solutions for $\zeta(s) = 0$ corresponding to mass-less %electrons
\blue{particles} on the honeycomb lattice should correspond to actual magnetic field terms applied to the bilayered material. The Riemann hypothesis\cite{conrey2003riemann} asserts that $s = 1/2 + i\gamma$, where our framework requires the so-called essential zeros at $\gamma = \gamma_g$ correspond to the %values of fluxes,
\blue{flux values},
\begin{align}\label{gamma_B_eq}
    \gamma_g = \frac{1}{2\pi}\int_{\mathcal{A}_g}d^{\,2}r_{ij}\,B(\vec{r}_{ij}),
\end{align}
due to the \blue{$z$ component of the} external magnetic field $B$ \blue{given by eq. (\ref{B_eq})}, where the mass term is modulated to vanish accordingly
%, with $B$ the $z$ component of the external magnetic field, given in eq. (\ref{B_eq}) 
%and \blue{the first three positive and negative} 
\blue{for} $\gamma_g \neq 0$\blue{. %as shown 
The first three positive and negative values have been plotted} in Figure \ref{Fig_5}. 

To %encode 
\blue{reflect} these two pieces of actual and pseudo-magnetic information in a consistent mathematical framework, we propose a novel complex-valued tensor equation similar to the idealised model\cite{kanyolo2020idealised}, 
\begin{subequations}\label{Idealised_eq}
\begin{align}
    \partial_vK_{uv} = 4\pi\ell\Psi^*\partial_u\Psi,\\
    K_{uv} = \partial_u\partial_v\Phi - iq\epsilon_{uvw}A_w,
\end{align}
\end{subequations}
where $\ell$ is a cut-off scale for electromagnetic interactions along the $z$ direction, \blue{$\partial_u$ are partial derivatives,} $K_{uv} = K_{vu}^*$ is a complex-hermitian tensor, the repeated Euclidean indices $u, v$ and $w$ are summed over, $\Psi = \sqrt{\rho}\exp(iS)$, $A_w$, $\Phi$, $\epsilon_{uvw}$ are the vacancy wave function, the electromagnetic vector potential, the Liouville field and the completely anti-symmetric Levi-Civita symbol respectively. One can check that, the vacancy number density in 3D is given by $\rho = -4\pi \ell K\exp(2\Phi)$ and $S$ is the quantum phase.  

Using this equation, we can introduce new complex variables $s$, $\overline{s}$ simply as, 
\begin{subequations}\label{s_calculation_eq}
\begin{multline}
    \hat{s} = 2\ell\int_0^{\ell}\langle \Psi(z)|\partial_{z}|\Psi(z) \rangle_g dz\\
    = 2\ell\int_0^{\ell}\int_{\mathcal{A}_g}\Psi^*(\vec{r}_{ij}, z)\partial_z\Psi(\vec{r}_{ij}, z)d^{\,2}r_{ij}dz\\
    = \ell\int_{\mathcal{A}_g}\left (|\Psi(\vec{r}_{ij}, \ell)|^2 - |\Psi(\vec{r}_{ij}, 0)|^2\right )d^{\,2}r_{ij}\\
    + i2\ell\int_0^{\ell}\int_{\mathcal{A}_g}\partial_zS(\vec{r}_{ij}, z)|\Psi(\vec{r}_{ij}, z)|^2d^{\,2}r_{ij}dz\\
    = -g + i\ell \langle \partial_z S\rangle_g = -g + i\gamma_g,
\end{multline}
and, 
\begin{multline}
    \hat{s}^* = 2\ell\int_0^{\ell}\langle \Psi^*(z)|\partial_{z}|\Psi^*(z) \rangle_g dz\\
    = 2\ell\int_0^{\ell}\int_{\mathcal{A}_g}\Psi(\vec{r}_{ij}, z)\partial_z\Psi^*(\vec{r}_{ij}, z)d^{\,2}r_{ij}dz\\
    = \ell\int_{\mathcal{A}_g}\left (|\Psi(\vec{r}_{ij}, \ell)|^2 - |\Psi(\vec{r}_{ij}, 0)|^2\right )d^{\,2}r_{ij}\\
    - i2\ell\int_0^{\ell}\int_{\mathcal{A}_g}\partial_zS(\vec{r}_{ij}, z)|\Psi(\vec{r}_{ij}, z)|^2d^{\,2}r_{ij}dz\\
    = -g - i\ell \langle \partial_z S\rangle_g = -g - i\gamma_g,
\end{multline}
\end{subequations}
where $\vec{r}_{ij} \in R^2$ and we have defined $2\ell\int_{\mathcal{A}_g}|\Psi(\vec{r}_{ij}, 0)|^2d^{\,2}r_{ij} = 2(g - 1)$ for $g \neq 0$ and $2\ell\int_{\mathcal{A}_g}|\Psi(\vec{r}_{ij}, \ell)|^2d^{\,2}r_{ij} = -2$ corresponding to the negative values of the Euler characteristic, $-E(\mathcal{A}_g) = 2g - 2$ before ($g \neq 0$) and after ($g = 0$) bifurcation respectively, and $\langle \partial_zS \rangle_g  = \int_0^{\ell}\int_{\mathcal{A}_g} \partial_zS(\vec{r}_{ij}, z) |\Psi(\vec{r}_{ij}, z)|^2d^{\,2}r_{ij}$. 

% Figure environment removed

Thus, from eq. (\ref{Idealised_eq}), it is clear that \blue{a finite genus due to the external electric field in eq. (\ref{Liouville_eq}) corresponds to the %real part 
sum%of $\hat{s}$ and $\hat{s}^*$
},
\begin{align}
    s = \hat{s} + \hat{s}^* = -2g,
\end{align}
\blue{whereas} a finite magnetic flux due to the external magnetic field \blue{in eq. (\ref{gamma_B_eq})} corresponds to the imaginary part of $\hat{s}$ %and 
\blue{or} $\hat{s}^*$ \blue{depending on the direction of $B$} commensurate with eq. (\ref{zeta_eq}). Moreover, setting $S(\vec{r}_{ij}, z) + \Phi(\vec{r}_{ij}, z) = 0$ in eq. (\ref{s_calculation_eq}) guarantees flux is quantised $\gamma_g = g$, since there will be no distinctions between pseudo-magnetic and magnetic degrees of freedom, %\textit{i.e.},
\begin{align}
    |\Psi(\vec{r}_{ij}, z)|^2 = -K\exp(2\Phi(\vec{r}_{ij}, z))/4\pi\ell = \frac{B(\vec{r}_{ij})}{2\pi\ell}.
\end{align}
For instance, for 1D systems in a ring exhibiting Peierls instability, there are no distinctions between the pseudo-magnetic and actual magnetic degrees of freedom, which should lead to a magnetisation effect proportional to the order parameter\cite{liang2006peierls}. However, we are interested instead in a different scenario where the actual magnetic and pseudo-magnetic effects are distinguishable, \textit{i.e.} $S(\vec{r}_{ij}, z)$ and $\Phi(\vec{r}_{ij}, z)$ play complementary \textit{albeit} antagonistic role in the realisation of a stable bilayered structure. An important scheme that implements this is to introduce quantum operators/averaging on $g = \nu - 1$ treated as quantum harmonic oscillator operators.\cite{kanyolo2020idealised} Setting $g = aa^{\dagger}$, $\nu = a^{\dagger}a$ where $a^{\dagger}, a$ are the creation and annihilation operators satisfying the bosonic commutation relation $[a^{\dagger}, a] = 1$, we first take the quantum average of the genus via a weighted sum, 
\begin{subequations}
\begin{align}
    \langle g \rangle_{\mathcal{P}} = \sum_{\nu = 0}^{\nu = \infty}g(\nu)\mathcal{P}_{\nu}(\beta) = \frac{1}{1 - \exp(-\beta)},
\end{align}
where $H_{\nu} = \nu + 1/2 = g - 1/2$ is the appropriately normalised Hamiltonian,
\begin{align}
    \mathcal{P}_{\nu}(\beta) = \frac{\exp(-\beta H_{\nu})}{\sum_{\nu = 0}^{\nu = \infty}\exp(-\beta H_{\nu})},
\end{align}
is the probability for a certain genus to occur and $\beta$ plays the role of the inverse temperature. Proceeding, using the gamma function, $\Gamma(\hat{s} = -2g)$ as a distribution function in a subsequent averaging\cite{kanyolo2022advances2} yields the result we seek, 
\begin{multline}
     \langle\,\langle g \rangle_{\mathcal{P}}\,\rangle_{\Gamma (-2g)} = \frac{1}{\Gamma(-2g)}\int_0^{\infty} \frac{d\beta}{\beta}\beta^{-2g} \langle g \rangle_{\mathcal{P}}\exp(-\beta)\\
    = \frac{1}{\Gamma(-2g)}\int_0^{\infty} \frac{d\beta}{\beta}\beta^{-2g}\frac{1}{\exp(\beta) - 1} = \zeta(-2g), 
\end{multline}
\end{subequations}
%as required, 
subject to zeta function regularisation by eq. (\ref{analytic_eq}) for values ${\rm Re}(s) \blue{= \delta} < 1$. Evidently, 
\begin{subequations}\label{tilde_s_eqs}
\begin{align}
    \tilde{s} = \langle \langle \hat{s} \rangle_{\mathcal{P}} \rangle_{\Gamma(-2g)} = \Delta(d) + i\gamma_g,\\
    \tilde{s}^* = \langle \langle \hat{s}^* \rangle_{\mathcal{P}} \rangle_{\Gamma(-2g)} = \Delta(d) - i\gamma_g,
\end{align}
\end{subequations}
where the fluxes $\gamma_{g \neq 0} \neq 0$ %($g \neq 0$) 
must satisfy $\zeta(s) = 0$ in order to guarantee the vanishing of the interaction $U_{R^2}(\tilde{s}) = U_{R^2}(\tilde{s}^*) = 0$, whose functional form is defined in eq. (\ref{continuum_h_Eq2}). This means that, for tuning of the mass term to its critical value by an external field to be possible, the fluxes in each hexagonal lattice ($hx, hx^*$) ought to occur at the essential zeros of $\zeta(s)$ corresponding to ${\rm Re}(s) = \Delta(d = 3) = 1/2$.

% Figure environment removed

Finally, the argentophilic bonds described by eq. (\ref{Heisenberg_eq1}) and \blue{hence} eq. (\ref{R3_eq}) in the %two-body 
\blue{two-silver} approximation %correspond to,
\blue{ought to satisfy},
\begin{align}\label{Heisenberg_eq2}
    0 = 4J(\vec{r}_{ij})S_i^zS_j^z + \langle \langle h(\vec{r}_{ij}) \rangle \rangle (S_i^z + S_j^z),
\end{align}
where the %average of
\blue{double-averaging performed on $h(\vec{r}_{ij})$ given by} eq. (\ref{pseudo_h_eq}) %is given by,
\blue{corresponds to replacing $\hat{s}$ and $\hat{s}^*$ with their counterparts $\tilde{s}$ and $\tilde{s}^*$ respectively to yield}, 
\begin{align}
    \langle \langle h(\vec{r}_{ij}) \rangle \rangle = -\frac{m}{C}\left (\frac{\sqrt{2}\,a}{r_{ij}} \right )^{2(\tilde{s} + \tilde{s}^*)} = -\frac{Gm}{r_{ij}^2},
\end{align}
with $\vec{r}_{ij} \in R^3$ \blue{strictly in $R^3$}, $\tilde{s} + \tilde{s}^* = 1$ from eq. (\ref{tilde_s_eqs}), $G = 2a^2/C$ the square of the lattice constant with dimensions of [length]$^2$ and $J(\vec{r}_{ij}) = -A^2\chi_{\Delta(d)}(\vec{r}_{ij})$ the so-called Ruderman-Kittel-Kasuya-Yosida (RKKY) exchange interaction term in $d$ dimensions (Fourier transform of the Lindhard function in $d$ dimensions) characteristic of spin-orbit scattering of conduction electrons by the nonmagnetic %/pseudo-spin 
\blue{(herein, pseudo-spin)} impurities in calculations involving $sd$ hybridisation\cite{aristov1997indirect} %with\cite{aristov1997indirect},
\blue{and},
\begin{subequations}
\begin{align*}
    \chi_{\Delta}(\vec{r}_{ij})  = -\frac{1}{4\pi}\left (\frac{m_{\Delta}k_{\rm F}^2}{2\Delta + 1}\right )\left(\frac{k_{\rm F}}{2\pi r_{ij}}\right )^{2\Delta}f_{\Delta}(k_{\rm F}r_{ij}),\\
    f_{\Delta}(x) = \mathcal{J}_{\Delta(d)}(x)\mathcal{Y}_{\Delta(d)}(x) + \mathcal{J}_{\Delta(d) + 1}(x)\mathcal{Y}_{\Delta(d) + 1}(x).
\end{align*}
\end{subequations}
Here, $m_{\Delta(d)}$ is the generated electron mass on the honeycomb lattice due to bifurcation, $A^2$ a proportionality constant and $\mathcal{J}_{\alpha}(x), \mathcal{Y}_{\alpha}(x)$ are Bessel functions of the first and second kind respectively. Since $\chi_{\Delta}(\vec{r}_{ij})$ \blue{is} proportional to the generated mass, $m_{\Delta(d)} = 2m\Delta(d)$, evidently $J(\vec{r}_{ij})$ is finite only in $d = 3$, allowing its \blue{positive} value to counteract the $\vec{r}_{ij} \notin R^2$ pseudo-spin term in eq. (\ref{R3_eq}) thus obtaining $H = 3J(\vec{r}_{ij}) + h(\vec{r}_{ij}) = 0$ \blue{as the final result of the calculations that commenced from} %in 
eq. (\ref{Heisenberg_eq1}). In $d = 3$ \blue{dimensions (3D)}, where the RKKY terms are finite, %we obtain, 
\begin{multline}
    \chi_{\Delta(3)}(r_{ij}) = \frac{mk_{\rm F}}{8\pi^3r_{ij}^3}\left (\frac{\sin\left (2k_{\rm F}r_{ij}\right )}{2k_{\rm F}r_{ij}} - \cos\left (2k_{\rm F}r_{ij}\right ) \right )\\
    = -mk_{\rm F}/8\pi^3r_{ij}^3,
\end{multline}
where the last line is obtained by the constraints $\vec{r}_{ij}\cdot\vec{k}_{ij} = \pi N \neq 0$ and $\vec{r}_{ij}\times\vec{k}_{ij} = \vec{0}$. Here, $N \in \mathbb{N}$ is a non-vanishing positive integer, $\vec{r}_{ij} \neq 0, \vec{k}_{ij} \neq 0$ are both vectors in $R^3$, $|\vec{r}_{ij}| = r_{ij}$ and $|\vec{k}_{ij}| = k_{\rm F}$ defines the Fermi surface. 

The final expression for the argentophilic bond becomes, 
\begin{subequations}
\begin{align}\label{orbits_eq}
    0 = \frac{L^2}{m^2r_{ij}^3} - \frac{Gm}{r_{ij}^2} \equiv -\frac{\partial V(\vec{r}_{ij})}{\partial r_{ij}},
\end{align}
where,
\begin{align}
    V(\vec{r}_{ij}) = \frac{1}{2m^2}\frac{L^2}{r_{ij}^2} - \frac{Gm}{r_{ij}^2},
\end{align}
\end{subequations}
is the bonding potential and we have introduced $A^2 \equiv 8\pi^3L^2/3k_{\rm F}m^3$ in order to write the interaction in a familiar form, analogous to gravitational orbits under the potential $V(\vec{r}_{ij})$ representing the emergent attractive forces experienced by conduction electrons \blue{and diffusing silver cations in the material} %on the bifurcated lattice,
under a `Newton's inverse-square law' force with $G$ playing the role of a `gravitational constant' and $L$ the angular momentum\blue{,} %of the electrons, 
leading to a conductor-semiconductor-insulator phase transition \blue{for all charge transport on the bifurcated lattice}. %This allows one to calculate t
\blue{T}he argentophilic bond length \blue{corresponds to} $r_{\rm bond} = L^2/Gm^3 \leq 2.83$ \AA, %thus 
\blue{which} %displaying 
\blue{displays} the relation between the constants $L$, $m$ and $G$ in terms of the observed bond length. However, since these constants are presently undetermined, for illustration purposes we shall set $r_{\rm bond} \equiv 2.8$ \AA \,\, and plot in Figure \ref{Fig_6} the appropriately normalised potential,
\begin{align}\label{Bonding_eq}
    V_{\rm bond}^{(\tilde{m}^z, \tilde{c}^z)}(r_{ij}) = 2\tilde{c}^zr_{\rm bond}^2/3r_{ij}^2 - \tilde{m}^zr_{\rm bond}/r_{ij},
\end{align}
for various values of $\Sigma_{ij} = (\tilde{m}^z, \tilde{c})$ in eq. (\ref{pseudo_eq}), thus illustrating the presence of a minimum at $r_{\rm bond}$ only for the pseudo-spin wave function $\Sigma_{ij} = (+1, +3/4)_{ij}$ given in eq. (\ref{pseudo_eq}). Changing the sign of the pseudo-magnetic field $h(\vec{r}_{ij}) \blue{\rightarrow -h(\vec{r}_{ij})}$ %selects 
\blue{exchanges pseudo-spin up with pseudo-spin down in the calculation in eq. (\ref{R3_eq}), thus selecting} instead $\Sigma_{ij} = (-1, +3/4)_{ij}$ as the potential with the minimum at $r_{\rm bond}$. 

\textit{Conclusion.}\textemdash Exploiting the pseudo-spin degree of freedom %availed by the bipartite nature of the Ag honeycomb lattice of bilayered materials, 
in the bipartite honeycomb lattice of Ag cations inherited from the spin of their half-filled $4d_{z^2}$ orbitals, we introduced a pseudo-spin Heisenberg Hamiltonian to describe the argentophilic bond observed in silver-based bilayered materials, arriving at the same qualitative results as a previously proposed SU(2)$\times$ U(1) model.\cite{masese2022honeycomb} In the Heisenberg Hamiltonian, the monolayer-bilayer phase transition occurs due to the spontaneous emergence of a pseudo-magnetic interaction term, with the mechanism for Ag bilayers reminiscent of Zeeman splitting. The advantage conferred by the proposed approach is the novel possibility of engineering a crossover between 2D and 3D behaviour of %cations 
\blue{silver ions} and their valence electrons expected to be responsible for %paired electrons 
\blue{particle pairing/bonding} and a metal-semiconductor-insulator phase transition. Incidentally, since excellent conductors such as elemental silver are expected to be poor BCS superconductors due to fairly weak electron-phonon coupling\cite{tinkham2004introduction}, %our crossover mechanism explaining the origin of 
the electron pairing mechanism herein incidentally explains the low temperature superconductivity reported in $\rm Ag_2^{1/2+}F$, \textit{albeit} %expected to maintain some metallic properties\cite{wang1991anisotropic}
\blue{remains} a normal conductor instead of semi-conductor or insulator\cite{wang1991anisotropic} at temperatures well-above the reported transition temperature ($T_{\rm c} \simeq 66$ mK).\cite{andres1966superconductivity} Moreover, the silver bilayer in such materials is expected to be tuned not only by pseudo-magnetic fields in the form of stress and strain, but more readily by external magnetic fields whose flux values occur at the essential zeros of the so-called $L(s)$ functions such as the famous Riemann zeta function, corresponding to the Mellin transform of specific lattice theta functions.\cite{conrey2003riemann, kanyolo2022advances2} Such experiments should be performed on (near-)single crystals of pure silver-bilayered materials such as $\rm Ag_2^{1/2+}F$ or the more stable $\rm Ag_6^{1/2+}Mg_2TeO_6$ (or preferably any other silver bilayered material with non-magnetic slab cations in order to avoid %magnetic ordering 
effects such as magneto-resistance\cite{taniguchi2020butterfly} due to magnetic ordering of $M = \rm Ni, Co, Cu$ \textit{etc} cations in the slabs%skewing the interpretation of the experimental results
) prepared\blue{, for instance,} by mechanical exfoliation.\cite{taniguchi2020butterfly} %techniques. %suchlike those reported in \cite{taniguchi2020butterfly}.   %Cite max of 2 new citations 

\textit{Acknowledgements.}\textemdash This %research is 
\blue{work was} supported in part by the AIST Edge Runners Funding and Iketani Science and Technology Foundation. 

\bibliography{pseudo-spin}
%\bibliographystyle{apsrev4-2}

\end{document}

% ****** End of file apssamp.tex ******
