\documentclass{article}

% if you need to pass options to natbib, use, e.g.:
\PassOptionsToPackage{square,numbers}{natbib}
%\usepackage[square,numbers]{natbib}
% before loading neurips_2020

% ready for submission
\usepackage[preprint]{NEURIPS2023/neurips_2023}
%\usepackage[preprint]{NEURIPS2023/neurips_2023}

% to compile a preprint version, e.g., for submission to arXiv, add add the
% [preprint] option:
%     \usepackage[preprint]{neurips_2020}

% to compile a camera-ready version, add the [final] option, e.g.:
%     \usepackage[final]{neurips_2020}

% to avoid loading the natbib package, add option nonatbib:
% \usepackage[nonatbib]{neurips_2020}

\usepackage[utf8]{inputenc} % allow utf-8 input
\usepackage[T1]{fontenc}    % use 8-bit T1 fonts
\usepackage{hyperref}       % hyperlinks
\usepackage{url}            % simple URL typesetting
\usepackage{booktabs}       % professional-quality tables
\usepackage{amsfonts}       % blackboard math symbols
\usepackage{nicefrac}       % compact symbols for 1/2, etc.
\usepackage{microtype}      % microtypography
\usepackage{xcolor}         % colors

\usepackage{amsmath}
\usepackage{amsthm}
\usepackage{acronym}
\usepackage[ruled]{algorithm2e}
\usepackage{array}
\usepackage{wrapfig}
\usepackage{multirow}
\usepackage{tabu}

% Bibliography:
%\usepackage[backend=biber,style=alphabetic]{biblatex}
%\addbibresource{bibli.bib} %Imports bibliography file
\bibliographystyle{abbrvnat}

%--------------------------------------
%Paquet necessaire à l'insertion d'image :
%\usepackage{graphicx}
%\usepackage[pdftex]{graphics}
\usepackage[pdftex]{graphicx}
%\usepackage[final]{graphicx}
%\usepackage[caption = false]{subfig}
%--------------------------------------
%Paquet pour ajouter des titres au images non-flottantes :
\usepackage{caption}
\usepackage{float}
%\usepackage{subfigure}
\usepackage{subcaption}
\usepackage{comment}

\usepackage{listings}
\usepackage{color}
 
\definecolor{codegreen}{rgb}{0,0.6,0}
\definecolor{codegray}{rgb}{0.5,0.5,0.5}
\definecolor{codepurple}{rgb}{0.58,0,0.82}
\definecolor{backcolour}{rgb}{0.95,0.95,0.92}
 
\usepackage[colorinlistoftodos]{todonotes}


\title{
ETHER: Aligning Emergent Communication for Hindsight Experience Replay
}

% The \author macro works with any number of authors. There are two commands
% used to separate the names and addresses of multiple authors: \And and \AND.
%
% Using \And between authors leaves it to LaTeX to determine where to break the
% lines. Using \AND forces a line break at that point. So, if LaTeX puts 3 of 4
% authors names on the first line, and the last on the second line, try using
% \AND instead of \And before the third author name.

\author{%
  Kevin Denamganaï, \\
  University of York, UK \\
  \texttt{kevin.denamganai@york.ac.uk}, 
  \And 
  Daniel Hernandez, \\
  Sony AI \\
 \texttt{daniel.hernandez@sony.com} 
 \And 
  Ozan Vardal, \\
  University of York, UK \\
  \texttt{ozan.vardal@york.ac.uk}
 \And 
  Sondess Missaoui, \\ 
  University of York, UK \\
  \texttt{sondess.missaoui@york.ac.uk} 
 \And James Alfred Walker \\
  University of York, UK \\
  \texttt{james.walker@york.ac.uk}
%   Kevin DENAMGANAÏ\\
%   Department of Computer Science\\
%   University of York\\
%   York, UK \\
%   %Pittsburgh, PA 15213 \\
%   \texttt{kyd500@york.ac.uk} \\
%   % examples of more authors
%   \And
%   James Alfred WALKER \\
%   Department of Computer Science\\
%   University of York\\
%   York, UK \\
%   \texttt{james.walker@york.ac.uk} \\
%   % \AND
  % Coauthor \\
  % Affiliation \\
  % Address \\
  % \texttt{email} \\
}

\begin{document}

\maketitle

%\listoftodos

\begin{abstract}

The Fast Reciprocal Square Root Algorithm is a well-established approximation technique consisting of two stages: first, a coarse approximation is obtained by manipulating the bit pattern of the floating point argument using integer instructions, and second, the coarse result is refined through one or more steps, traditionally using Newtonian iteration but alternatively using improved expressions with carefully chosen numerical constants found by other authors. The algorithm was widely used before microprocessors carried built-in hardware support for computing reciprocal square roots. At the time of writing, however, there is in general no hardware acceleration for computing other fixed fractional powers. This paper generalises the algorithm to cater to all rational powers, and to support any polynomial degree(s) in the refinement step(s), and under the assumption of unlimited floating point precision provides a procedure which automatically constructs provably optimal constants in all of these cases. It is also shown that, under certain assumptions, the use of monic refinement polynomials yields results which are much better placed with respect to the cost/accuracy tradeoff than those obtained using general polynomials. Further extensions are also analysed, and several new best approximations are given.

\end{abstract}


\section{Introduction}
Current quantum hardware is unable to carry out universal quantum computations due to the buildup of errors that occur during the computation. 
The magnitude of the individual error is currently above the value that the Threshold Theorem requires in order to kick-start quantum error correction and fault-tolerant quantum computation~\cite[Section 10.6]{nielsen_chuang_2010}. 
Although the experimentally achieved fidelity rates are promising and the error bounds are inching closer to the required threshold, we will have to work for the foreseeable future with quantum hardware with errors that build-up during the computation.  This implies that we can only do a limited number of steps before the output of the computation has become completely uncorrelated with the intended one.

For fault-tolerant quantum computing, we repeat four steps: 
1) We apply a number of single and two-qubit quantum gates, in parallel whenever possible; 
2) We perform a syndrome measurement on a subset of the qubits; 
3) We perform fast classical computations to determine which errors have occurred and how to correct them; 
and, 4) We apply correction terms based on the classical computations.
We then repeat these four steps with a next sequence of gates. 
These four steps are essential to fault-tolerant quantum computing. 


The starting point of this work is to use the four steps outlined above, not to carry out error correction and fault-tolerant computation, but to enhance short, constant-depth, {\em uncorrected} quantum circuits that perform single qubit gates and {\em nearest-neighbor} two qubit gates. 
Since in the long run we will have to implement error-correction and fault-tolerant computation anyhow, and this is done by such a four-step process, why not make other use of this architecture? Moreover, on some of the quantum hardware platforms, these operations are already in place.
Embracing this idea we naturally arrive at the question: what is the computational power of \textit{low-depth} quantum-classical circuits organized as in the four steps outlined above? 
We thus investigate circuits that execute a small, ideally constant, number of stages, where at each stage we may apply, in parallel, single qubit gates and {\em nearest-neighbor} two qubit gates, followed by measurements, followed by low-depth classical computations of which the outcome can control quantum gates in later stages. 
It is not clear, at first, whether such circuits, especially with constant depth, can do anything remotely useful. 
But we will see that this is indeed the case: many quantum computations can be done by such circuits in constant depth. 
By parallelizing quantum computations in this way, we improve the overall computational capabilities of these circuits, as we do not incur errors on qubits that are idle, simply because qubits are not idle for a very long time. 
Furthermore, reducing the depth of quantum circuits, at the cost of increasing width, allows the circuit to be run faster even if errors occur.

The first usage of such a four-step layout, not to do error correction, but to perform computations, can be found in the paradigm of measurement-based quantum computing~\cite{gottesman1999demonstrating,raussendorf2001one,jozsa2006introduction,clark2007generalised}: 
A universal form of quantum computing where a quantum state is prepared and operations are performed by measuring qubits in different bases, depending on previous measurements and intermediate measurements.

\citeauthor{PhamSvore2013} were the first to formalize the four-step protocol for performing computations~\cite{PhamSvore2013}. They included specific hardware topologies by considering two-dimensional graphs for imposing constraints on qubit interactions. In their model, they develop circuits for particularly useful multi-qubit gates, including specifying costs in the width, number of qubits, depth, number of concurrent time steps, size, and total number of non-Identity operations.
As a result, they find an algorithm that factors integers in polylogarithmic depth.
\citeauthor{Browne:2011} showed that the main tool in the work by \citeauthor{PhamSvore2013}, the fan-out gate, can also be replaced by additional log-depth classical computations in the measurement-based quantum computing setting~\cite{Browne:2011}.

More recently, \citeauthor{Cirac:2021} introduced a scheme to implement unitary operations involving quantum circuits combined with Local Operations and Classical Communication ($\mathsf{LOCC}$) channels: $\mathsf{LOCC}$-assisted quantum circuits~\cite{Cirac:2021}. Similarly to the four-step scheme we just described, they allow for a short depth circuit to be run on the qubits, followed by one round of $\mathsf{LOCC}$, in which ancilla qubits are measured and local unitaries are applied based on the measurement outcomes. They show that in this model any 1D transitionally invariant matrix-product state (MPS) with fixed bond dimension is in the same phase of matter as the trivial state. Similar ideas can be found in~\cite{TVV_NonAbelianTopologicalOrder_2022, tantivasadakarn2021long}.

In this work, we introduce a new model, called \textit{Local Alternating Quantum-Classical Computations} ($\LAQCC$). In this model we alternate between running quantum circuits (constrained by locality), ending in the measurement of a subset of qubits, and fast classical computations based on the measurement results. The outcome of the classical computations are then used to control future quantum circuits. We allow for flexibility in this model, by giving different constraints to the power of both the quantum circuits and the classical circuits as well as the number of alternations between them. 
Most attention will be given to $\LAQCC$ containing quantum circuits of constant depth, classical circuits of logarithmic depth and at most a constant number of alternations between them. 
Any circuit constructed in this model is considered to be of constant depth. 
We restrict ourselves to logarithmic depth classical computations, as this is the first natural and non-trivial extension beyond constant-depth classical computations. 
Constant-depth classical computations do however also have an equivalent constant-depth quantum implementation.

The definition of $\LAQCC$ sharpens the original definition of \citeauthor{PhamSvore2013} by adding constraints to the intermediate classical computations. This allows us to bound the power of $\LAQCC$ from above. 

The main result of \citeauthor{Cirac:2021}, that 1D translational invariant MPS with fixed bond dimension can be prepared by $\mathsf{LOCC}$-assisted circuits, relies on local symmetries of the MPS. These symmetries allow them to prepare local states (on a constant number of qubits) and glue them together by doing one round of the appropriate entangling measurement and corrections, after which they run a round of local unitaries to get the desired result. This general scheme for preparing states that exhibit an MPS description with the appropriate local symmetries requires only geometrically local unitaries and one round of measurement and corrections an therefore is accessible in $\LAQCC$. Studying different local symmetries, known as Symmetry Protected Topological (SPT) phases of matter, to find measurement-based constant depth circuits for states is a broad ongoing field of research~\cite{TVV_NonAbelianTopologicalOrder_2022, tantivasadakarn2021long, smith2023deterministic}. 
All these schemes have a $\LAQCC$ implementation.

%$\LAQCC$-circuits also exist for general schemes of preparing local states, based on the local tensors, and gluing them together using one round of entangled measurement and corrections, based on the local symmetry. 
%The main result of \citeauthor{Cirac:2021}, that 1D translational invariant MPS with fixed bond dimension can be prepared by $\mathsf{LOCC}$-assisted circuits, relies heavily on local symmetries of the MPS and as a result also has an equivalent $\LAQCC$ implementation. 
%The corrections applied after the measurement round are local unitaries depending on the local symmetries of the MPS. 

 

%This general scheme of preparing local states, based on the local tensors, and gluing it together by doing one round of entangled measurement and corrections, based on the local symmetry, is accessible in $\LAQCC$.
Note however that \citeauthor{Cirac:2021} also suggest a circuit for the $W$-state.
This circuit uses sequentially and dependent measurement-based corrections of the ancilla qubits. 
These dependent measurements translate to sequential alternations between the quantum and classical circuits and therefore increase the total depth to linear depth, exceeding the constant-depth constraints imposed by $\LAQCC$-circuits. 

We study the power of the $\LAQCC$ model with respect to state preparation, showing that even with only constant quantum-depth and logarithmic classical depth it remains possible to prepare states with long-range entanglement.
Another surprising result is that it is unlikely that $\LAQCC$ circuits are classically simulatable. We show that any instantaneous quantum polynomial-time (IQP) circuit~\cite{Bremner2010,Shepherd2009} has an $\LAQCC$ implementation.
Classical simulation of IQP circuits implies the collapse of the polynomial hierarchy to the third level, which is not believed to be true~\cite{Bremner2017}. Therefore, we expect that $\LAQCC$ circuits are unlikely to be classically simulatable. We bound the power of $\LAQCC$ by showing that it is contained in $\QNC^1$, the class of polynomial-size, log-depth circuits.

Next, we also study the power that intermediate classical calculations can add to quantum computations, by considering a new model that alternates between polynomially many polynomial-depth quantum circuits and unbounded classical computations
We study this model by doing a complexity theoretical analysis, where we draw inspiration from the notions of complexity given by \citeauthor{RosenthalYuen:2022}, \citeauthor{MetgerYuen:2023}, and \citeauthor{Aaronson:2004}.
All three complexity notions are based on the notion of state preparation, instead of more traditional definition of complexity such as the decidability of a computational problem. 
The first two consider classes based on sequences of quantum states preparable by a polynomial-sized quantum circuit, where the circuits are uniformly generated by a computational class, for instance, the class $\mathsf{PSPACE}$, which results in the complexity class $\mathsf{StatePSPACE}$~\cite{RosenthalYuen:2022,MetgerYuen:2023}.
The third notion considers a relative complexity, where the complexity is measured between two given states, and is measured by the number of gates, from a given gate-set, required to transform one state in another state~\cite{Aaronson:2004}. 
For our definition of state preparation complexity, we drop the uniformity constraint from~\cite{RosenthalYuen:2022,MetgerYuen:2023} and define a class as $\mathsf{StateX}$, which refers to states preparable by circuits of type $\mathsf{X}$. 
As an example, if $\mathsf{X} = \QNC^0$, this results in the class $\mathsf{StateQNC^0}$, which is the set of states preparable from the $\ket{0}^n$ state by poly-size constant-depth circuits. 
This notion is similar to the relative complexity from~\cite{Aaronson:2004}, where one state is the  $\ket{0}^n$ state and instead of counting the number of gates we consider the set of states preparable by a fixed number of gates. Using this notion of complexity we show that any state preparable by an $\LAQCC^*$ circuit is also preparable by a $\mathsf{PostQPoly}$ circuit, the class of circuits of polynomial depth with an additional post-selection gate. 

All Clifford circuits have a constant-depth $\LAQCC$ implementation, implying that any stabilizer state can be implemented by a constant-depth $\LAQCC$ circuit, see Section~\ref{sec:clifford_circuits} for a proof of this statement. 
Efficient circuits for stabilizer states have been known already through measurement-based quantum computing. Therefore this paper focuses on the preparation of non-stabilizer states, and as a surprising result we find novel constant-depth protocols for four very natural classes of non-stabilizer states.
Despite the extensive research into these four classes of non-stabilizer states and the many applications of them, no efficient constant- or low-depth state preparation protocols are known yet. We specifically consider these four classes as they are all often used as initial states in other algorithms.

The first state is a uniform superposition over an arbitrary number of states. 
This state finds applications in many quantum algorithms, as they often start with a uniform superposition over multiple states. 
This superposition is often achieved by applying Hadamard gates to every qubit due to its simplicity to prepare. 
Yet, the analysis of many algorithms, such as Shor's algorithm~\cite{Shor:1997}, would benefit from a different initial superposition. 
The circuit to prepare the uniform superposition over an arbitrary number of states uses an exact version of Grover search as a subroutine, that turns a probabilistic circuit, with a known constant probability of success, into a deterministic circuit. 
We use the circuit for preparing a uniform superposition over an arbitrary number of states as a subroutine in the next two quantum state preparation protocols. 

The second state is the $W$-state, the uniform superposition over all computational basis states of Hamming-weight~$1$, a natural long-ranged entangled state that displays a fundamentally nonequivalent type of entanglement from the Greenberger–Horne–Zeilinger state~\cite{WState:2000}, for which $\LAQCC$-type constant-depth circuits were previously known~\cite{PhamSvore2013, Cirac:2021}. 
The $W$-state is often used as benchmark for new quantum hardware~\cite{Haffner2005,Neeley2010,GarciaPerez:2021}. 
A novel way to prepare the $W$-state therefore gives a new way to benchmark different quantum devices with each other. 
A circuit for preparing the $W$-state was given in~\cite{Cirac:2021}, but this implementation requires sequentially alternating measurements followed by local unitaries, which in the $\LAQCC$ model is not considered to be of constant depth. 
We improve this protocol by giving an $\LAQCC$ implementation of the $W$-state, based on a compress-uncompress method that links the one-hot and binary encoding of integers.

The third state considered is the Dicke state, a generalization of the $W$-state, a superposition over all computational basis states with Hamming-weight $k$~\cite{Dicke:1954}. 
Dicke states have relevance in various practical settings.
For instance, for quantum game theory~\cite{zdemir2007}, quantum storage~\cite{Bacon_Compress:2006,Plesch:2010}, quantum error correction~\cite{ouyang2014permutation}, quantum metrology~\cite{toth2012multipartite}, and quantum networking~\cite{prevedel2009experimental}. 
Dicke states have been used as a starting state for variational optimization algorithms, most notably Quantum Alternating Operator Ansatz (QAOA)~\cite{Hadfield2019}, to find solutions to problems such as Maximum k-vertex Cover~\cite{Brandhofer2022,cook2020quantum}.
The ground states of physical Hamiltonians describing one-dimensional chains tend to show a resemblance to Dicke states such as states resulting from the Bethe ansatz, making them an ideal starting state when investigating the ground state behavior of these Hamiltonians~\cite{TDL_BetheAnsatzDerivation:2010,B_ExcitedStateQuantumPhaseTransitions:2013,DickeTransitions:2021}. 
For instance, the algorithm by \citeauthor{van2021preparing}, who give an algorithm to prepare the Bethe ansatz eigenstates of the spin-1/2 XXZ spin chain, starts by first preparing a Dicke state~\cite{van2021preparing}. 
A Dicke-state preparation protocol based on the compress-uncompress methodology used in the $W$-state furthermore finds applications in entanglement distillation, where the entanglement of a large state is concentrated on only a few qubits. 
Efficient deterministic circuits for preparing Dicke states have been proposed by \citeauthor{bartschi2019deterministic}~\cite{bartschi2019deterministic, bartschi2022deterministic_short_depth}. 
They provide a quantum circuit of depth $\mathO(k \log(\frac{n}{k}))$, allowing arbitrary connectivity, to prepare a Dicke state, which they conjecture to be optimal when $k$ is constant. 
In this work, we provide a constant-depth $\LAQCC$ circuit below their conjectured bound already for constant $k$. 
However, this does not directly disprove their conjecture, as we allow for intermediate measurements and classical computations. 
More significantly, we even construct constant-depth $\LAQCC$ circuits for $k = \mathO(\sqrt{n})$ greatly improving their bound.
This construction extends the compress-uncompress method for the $W$-state combined with additional subroutines. 

We continue with a log-depth state preparation protocol for the Dicke-state for arbitrary $k$. 
This protocol implements an efficient transformation between the factoradic number representation and the combinatorial number representation of a positive integer. 
The combinatorial number representation relates directly to the Dicke state. 
The provided efficient transformation between number representation systems might be of independent interest. 

We conclude by modifying our protocol for preparing a Dicke-state to a protocol that prepares quantum many-body scar states in constant-depth. 
These states have low entanglement and longer coherence times than states with similar energy density.
These characteristics make many-body scar states interesting to analyze and relevant within physics.
Many-body scar states appear for instance in the AKLT model~\cite{AKLT:1987,MRBAR:2018,MRB:2018} and different spin models~\cite{SI:2019,MOBFR:2020}.
Known methods for preparing these states have polynomial-depth~\cite{Gustafson:2023}, whereas our circuit has constant depth. 

% We conclude by studying the power that intermediate classical calculations can add to quantum computations. 
% In this study, we define a new model that relaxes constant-depth quantum circuits to polynomial depth quantum circuits, log-depth classical calculations to unbounded classical computations and a constant number of alternations to a polynomial number of alternations. 
% We call this model $\LAQCC^*$. 
% We study this model by doing a complexity theoretical analysis, where we draw inspiration from the notions of complexity given by \citeauthor{RosenthalYuen:2022}, \citeauthor{MetgerYuen:2023}, and \citeauthor{Aaronson:2004}.
% All three complexity notions are based on the notion of state preparation, instead of more traditional definition of complexity such as the decidability of a computational problem. 
% The first two consider classes based on sequences of quantum states preparable by a polynomial-sized quantum circuit, where the circuits are uniformly generated by a computational class, for instance, the class $\mathsf{PSPACE}$, which results in the complexity class $\mathsf{StatePSPACE}$~\cite{RosenthalYuen:2022,MetgerYuen:2023}.
% The third notion considers a relative complexity, where the complexity is measured between two given states, and is measured by the number of gates, from a given gate-set, required to transform one state in another state~\cite{Aaronson:2004}. 
% For our definition of state preparation complexity, we drop the uniformity constraint from~\cite{RosenthalYuen:2022,MetgerYuen:2023} and define a class as $\mathsf{StateX}$, which refers to states preparable by circuits of type $\mathsf{X}$. 
% As an example, if $\mathsf{X} = \QNC^0$, this results in the class $\mathsf{StateQNC^0}$, which is the set of states preparable from the $\ket{0}^n$ state by poly-size constant-depth circuits. 
% This notion is similar to the relative complexity from~\cite{Aaronson:2004}, where one state is the  $\ket{0}^n$ state and instead of counting the number of gates we consider the set of states preparable by a fixed number of gates. Using this notion of complexity we show that any state preparable by an $\LAQCC^*$ circuit is also preparable by a $\mathsf{PostQPoly}$ circuit, the class of circuits of polynomial depth with an additional post-selection gate. 

\paragraph{Summary of results}
\begin{itemize}
    \item We give a new definition of a computational model that captures the power of the four step process: applying a constant number of layers of one- and two-qubit gates; performing a syndrome measurement; perform a fast classical computation determining corrections; apply corrections. We call this model \emph{Local Alternating Quantum Classical Computations}, or $\LAQCC$ for short. In this model we bound the allowed quantum operations, intermediate classical calculations, and number of rounds separately. In Section~\ref{sec:LAQCC_model} we define this model and give a list of operations based on results from literature contained in this computational model. In some of these operations we explicitly use that we allow for multiple, but at most constant, rounds  of corrections.
    \item  We show show that there exist $\LAQCC$ circuits that can not be weakly simulated in Section~\ref{sec:IQP_in_LAQCC}. We further show that for every $\LAQCC$ circuit there exists a $\QNC^1$ circuit simulating it perfectly, in Section~\ref{sec:LAQCC_in_QNC1}.
    \item We introduce a new type computational complexity for preparing states and show that the extension of $\LAQCC$ where we allow a polynomial number of rounds and unbounded classical computation, is contained in $\mathsf{PostQPoly}$, the class of polynomial circuits with post-selection, in Section~\ref{sec:Complexity results}.
    \item We show a protocol to prepare the uniform superposition state of size $q$ in $\LAQCC$ using $\mathO(\ceil{\log_2(q)}^2)$ qubits in Section~\ref{sec:superposition_modulo_q}. 
    \item We show a protocol to prepare the $W_n$ state in $\LAQCC$ using $\mathO(n\log(n))$ qubits in Section~\ref{sec:W_state_in_LAQCC}.
    \item We show two ways of preparing the Dicke-$(n,k)$ state. The first method is in $\LAQCC$, works up to $k = \mathO(\sqrt{n})$, uses $\mathO(n^2\log(n))$ qubits, and is found in Section~\ref{sec:dicke:small_k}. The second method is in $\LAQCC\text{-}\mathsf{LOG}$ (an extension of $\LAQCC$ allowing for logarithmic number of alterations instead of constant), works for any $k$, uses $\mathO(\text{poly}(n))$ qubits, and is found in Section~\ref{sec:Dicke_in_LAQCC_LOG}. 
    \item We extend on our $\LAQCC$ method of generating Dicke-$(n,k)$ states for $k = \mathO(\sqrt{n})$ and show a protocol to generate many-body scar states for a particular Hamiltonian in $\LAQCC$ (Section~\ref{sec:many_body_scar}). 
\end{itemize}
Summarized in a table, we provide the following state generation protocols:
\begin{table}[htb]
\centering
\begin{tabular}{l|l|l|l}
\textbf{State description} & \textbf{Width} & \textbf{Depth} & \textbf{Implementation}\\
\hline 
Uniform superposition mod $q$: $\frac{1}{\sqrt{q}} \sum_{i = 0}^{q-1}\ket{i}$ & $\mathO(\ceil{\log^2 q})$ & $\mathO(1)$ & Section~\ref{sec:superposition_modulo_q}\\

$W$-state: $\frac{1}{\sqrt{n}}\sum_{i = 0}^{n-1}\ket{e_i}$ & $\mathO(n \log n)$ & $\mathO(1)$ & Section~\ref{sec:W_state_in_LAQCC}\\

Dicke-$(n,k)$, $k = \mathO(\sqrt{n})$: $\binom{n}{k}^{-1/2}\sum_{x \in \{0,1\}^n: |x| = k} \ket{x}$ &  $\mathO(n^2\log n)$ & $\mathO(1)$ 
&Section~\ref{sec:dicke:small_k}\\

Dicke-$(n,k)$: $\binom{n}{k}^{-1/2}\sum_{x \in \{0,1\}^n: |x| = k} \ket{x}$ & $\mathO(\text{poly}(n))$ & $\mathO(\log n)$ &Section~\ref{sec:Dicke_in_LAQCC_LOG}\\

QMBS: $\ket{S_k} = \frac{1}{k! \sqrt{\mathcal N(n,k)}}(Q^\dagger)^k \ket{\Omega}$ &  $\mathO(n^2\log n)$ & $\mathO(1)$  &  Section~\ref{sec:many_body_scar}
\end{tabular}
\caption{Summary of state preparation protocols given in this paper.}
\label{tab:sate_prep}
\end{table}
In the entry for the quantum many-body scar state $Q$ denotes the raising operator and $\mathcal N(n,k)=\binom{n-k-1}{k}$. 
Section~\ref{sec:many_body_scar} will provide more details on the variables and the implementation. 

\paragraph{Organization of the paper}
\noindent We first introduce relevant preliminaries in Section~\ref{sec:preliminaries}. 
In Section~\ref{sec:LAQCC_model} we formally define the class of Local Alternating Quantum-Classical Computations ($\LAQCC$). We also show that any Clifford circuit can be implemented in constant depth $\LAQCC$ (a result based on a result from measurement-based quantum computing~\cite{jozsa2006introduction}). 
This result allows us to give many useful multi-qubit gates and routines in Section~\ref{sec:gates_created_in_LAQCC}. 
Beyond that we show that constant depth $\LAQCC$ circuits are contained in $\QNC^1$ and that any $\mathsf{IQP}$ circuit has an $\LAQCC$ implementation.
We conclude this section with an analysis of a more powerful instantiation of $\LAQCC$ and show an inclusion with respect to the class $\mathsf{PostQPoly}$, which is the class of circuits of polynomial depth with one additional post-selection gate. 
In Section~\ref{sec:state_prep_in_LAQCC} we give $\LAQCC$ circuit implementations for preparing the uniform superposition over an arbitrary number of states, the $W$-state and the Dicke state up to $k = \mathO(\sqrt{n})$. We furthermore give a log-depth circuit implementation for preparing the Dicke state for any $k$. We conclude by showing a $\LAQCC$ circuit for generating many body scar states of a particular type of Hamiltonian.



% % Figure environment removed

% % Figure environment removed

\vspacebeforesection
\section{Background}
\label{sec:background}

In this section, we provide the necessary background information to ensure a comprehensive understanding of the attack described in this paper. We start with a description of the Distributed Hash Table (DHT) used by IPFS, followed by its content resolution mechanisms. We also detail techniques for network size estimation, necessary for our attack detection and mitigation mechanisms.

\vspacebeforesection
\subsection{IPFS DHT}
\label{sec:kad_dht}

We review the features of the Kademlia DHT~\cite{maymounkov2002kademlia} and its \texttt{libp2p} implementation~\cite{libp2p_github} that are the most relevant to our attack.
To participate in the DHT, each peer generates a public/private key pair and derives an identity $\peerid \in \{0,1\}^{256}$ as the hash of its public key.
Ideally, each peer generates a random key pair and, therefore, peer IDs are distributed uniformly and independently over the space $\{0,1\}^{256}$.
While honest nodes follow this rule, malicious nodes may generate and choose from an arbitrary number of key pairs.
Each peer maintains a routing table consisting of $m=256$ buckets.
The $i$-th bucket contains the addresses of up to $k=20$ peers whose peer IDs share a common prefix of exactly $i$ bits with the peer's own peer ID. 

%
A new participant node joins the IPFS network by contacting one of the hardcoded bootstrap nodes. This bootstrap node provides the new node with some initial peers allowing it to join the DHT. The new node uses this information to perform a walk through the DHT towards its own peer ID.
The walk allows to: \textit{(i)}~make sure that there is no other node in the network with the same ID; \textit{(ii)}~discover new peers and fill the newcomer's DHT routing table. At the same time, the newcomer establishes \bitswap~\cite{de2021accelerating} connections to a subset of encountered peers (usually around 300 of them). The core role of the \bitswap protocol is to enable bilateral content transfer and to play the role of a cache for recently-accessed content.

The main DHT operation $\Call{GetClosestPeers}{\key}$ returns the $k=20$ closest peers to $\key$. 
%
In Kademlia, the distance between two keys $x$ and $y$ in the key space is given by $x \oplus y \in \{0,...,2^{256}-1\}$, where $\oplus$ denotes the bitwise XOR operation on the keys; the resulting binary string is interpreted as an integer.
%
When a client wants to find the peers with IDs closest to $\key$, it sends a request to the $\alpha=3$ peers in its routing table whose peer IDs are closest to $\key$. Each of these peers returns the $k$ closest peers to $\key$ in its own routing table and the addresses of these peers. 
%
The client again sends a request to the $\alpha$ peers closest to $\key$, among peers in its routing table and those whose addresses it just received. This process repeats until the client does not find any more peers closer to $\key$.
Due to network churn and imperfect routing tables, we observed in our experiments that successive calls to $\Call{GetClosestPeers}{\key}$ do not always return the same set of $k=20$ peers (we provide more details in \Cref{sec:evaluation}, \Cref{fig:20closest}). This is an important limitation affecting our attack.

\vspacebeforesection
\subsection{Content Resolution in IPFS}
\label{sec:ipfs}

IPFS is a content-centric network.
It allows its participant to request files without specifying their location. 
%
Content is indexed by content IDs $\cid \in \{0,1\}^{256}$ that are derived from a hash of that content.
Both peer IDs and CIDs are used as keys in the DHT.
Each node can play the role of a \provider, \downloader, or \resolver. 
The process of content advertisement and resolution is illustrated in \Cref{fig:add_get_provider}.

%
When a \provider wishes to publish content with a given $\cid$ on IPFS, it creates a \emph{provider record} that contains $cid$ and the \provider's address.
During a $\Call{Provide}{\cid}$ operation, the \provider first uses $\Call{GetClosestPeers}{\cid}$ to locate the $k=20$ peers with their peer IDs closest to $\cid$, 
%
and then sends them a $\mathsf{PutProvider}$ message including the provider record (\Cref{fig:add_get_provider}(a)).
We call the peers that hold provider records for $\cid$ the \emph{resolvers} for $\cid$.

Each CID can have several \providers. In fact, by default, each IPFS client becomes a provider for each piece of content it downloads for a fixed amount of time (12h, 24h, or 48h depending on the client version or custom configuration). As a result, the system provides an auto-scaling feature with supply automatically rising with demand.

%
When a \downloader wishes to fetch a piece of content, it first sends a request to all its \bitswap peers. If none of them has the content, the \downloader uses the DHT-based resolution system. We stress that the \bitswap protocol plays the supporting role of a cache in the dissemination of popular files. However, the mechanism does not provide reliable content resolution, in particular for new or less popular content. %

When \bitswap unstructured search fails, the \downloader resolves $\cid$ using $\Call{FindProviders}{\cid}$. This operation uses a DHT walk identical to that of $\Call{GetClosestPeers}{\cid}$ to find $k$ \resolvers but also queries encountered nodes for a provider record for $\cid$ (\Cref{fig:add_get_provider}(b)). The process terminates when either 20 \providers have been found, or all \resolvers have been asked. Querying all encountered nodes (\ie, not only the designated \resolvers) is useful because some of the encountered nodes may have a provider record in their cache.
%

Upon receiving a provider record, the client connects to the address specified in the provider record to retrieve the actual content (\Cref{fig:add_get_provider}(c)).
Provider records are not authenticated, and therefore malicious \providers may respond with incorrect provider records (or may not respond at all). However, the integrity of the content is preserved because the hash of the retrieved content can be verified against its $\cid$.
%


%

\input{img/add_get_provider.tex}

\vspacebeforesection
\subsection{Network Size Estimator}
\label{sec:netsize}

The number of nodes in a decentralized system is generally unknown due to the avoidance of centralized membership management.
This number is nonetheless useful for optimizations, deciding on individual node configurations, or security mechanisms.
Various methods were proposed for the decentralized estimation of unstructured and structured networks~\cite{eli-sohl-dht-size-estimation,kostoulas2005decentralized, manku2003symphony}.
We use in this work a mechanism developed initially by Protocol Labs as part of a mechanism for decreasing the latency of publishing content in IPFS~\cite{network-size-estimation-notion,network-size-estimation-github-pr}.

%
%
%
%
%
%
%
%
%
%

Each node in the DHT refreshes its routing table periodically (every $10$ minutes in \texttt{libp2p}). 
For this, the node samples $m$ random keys (one for each bucket of its routing table)
%
and queries the DHT to obtain the $k=20$ closest peer IDs to each key.
Using these, the node then computes the average distance between each one of these keys $\key_j$ for $j=1,\dots,m$ and their $i$-th closest peer ID for $i=1,...,k$ (with $m=256$ and $k=20$).
\begin{equation}
    \label{equ:avg-dist}
    \overline{D}_i = \frac{1}{m} \sum_{j=1}^m \operatorname{dist}(\key_j, \peerid_{j}^{(i)})
\end{equation}
where $\peerid_{j}^{(i)}$ is the $i$-th closest peer ID to $\key_j$.
With $N$ peers in the DHT and peer IDs uniformly distributed in the hash space, the expected distance between a $\key$ and its $i$-th closest peer ID is $\frac{2^{256}i}{N+1}$. The node then runs a least square regression to compute the value of $N$ for which the expected distances best fit the empirical average distances, \ie,
\begin{equation}
    \label{equ:netsize-least-squares}
    \hat{N} = \arg\min_{N} \sum_{i=1}^k \left(\overline{D}_i - \frac{2^{256}i}{N+1}\right)^2.
\end{equation}
The resulting estimate $\hat{N}$ can be computed in closed form.
%

When a node starts running, it must perform DHT queries for a few random keys to initialize its network size estimate. 
Since a larger number of queries will result in higher accuracy, making more queries than what is needed to initialize one's routing table is recommended.
Thereafter, keeping the estimate up-to-date does not require any excess DHT queries beyond what is already used for refreshing the routing table as this is done frequently (every 10 minutes).

While the network size estimate has a stochastic variance resulting from the probability distribution of the honest peer IDs, it is hard for an attacker to bias the estimate significantly. Since the estimator uses the density of peer IDs around keys chosen uniformly at random, the adversary would require numerous Sybil nodes (on the order of the whole network size) to significantly affect the peer ID density around those keys.


\section{Method} \label{method_hybridaugment}
In this section, we formally define the problem, motivate our work and then present our proposed techniques.


\subsection{Preliminaries}
Let $\mathcal{F}(x;W)$ be an image classification CNN trained on the training set $\mathcal{T}_\text{train} = (x_{i}, y_{i})^{N}_{i=1}$  with $N$ samples, where $x$ and $y$ correspond to images and labels. The clean accuracy (CA) of $\mathcal{F}(x;W)$ is formally defined as its accuracy over a clean test set $\mathcal{T}_\text{test} = (x_{j}, y_{j})^{M}_{j=1}$. Assume two operators ${A}(\cdot)$ and ${C}(c, s)$ that adversarially attacks or corrupts a given set of images with the corruption category $c$ and severity $s$, respectively.  Let $A\mathcal{T}_\text{test}$ and $C\mathcal{T}_\text{test}$ be the adversarially attacked and corrupted versions of $\mathcal{T}_\text{test}$, and let $\mathcal{F}(x;W)$ have a robust accuracy (RA) on $A\mathcal{T}_\text{test}$ and a corruption accuracy (CRA) on $C\mathcal{T}_\text{test}$. 
The aim is to fit $\mathcal{F}(x;W)$ such that the model gains robustness (\ie. increased RA and CRA compared its the baseline version), while retaining (or improving) the clean accuracy of its baseline version trained without robustness concerns.


\noindent \textbf{What we know.} Our work builds on the following crucial observations: i) CNNs favour high-frequency content \cite{wang2020high}, ii) adversaries and corruptions often reside in high-frequency \cite{wang2020towards}, iii) images are dominated by low-frequency \cite{Saikia_2021_ICCV} and iv) models relying on low-frequency components are more robust \cite{li2022robust,wang2020towards}. The robustness-accuracy trade-off is visible; low-frequency reliant models are more robust, but tend to miss out on clean accuracy brought by the high-frequency components. 

\subsection{HybridAugment}
We hypothesize that a \textit{sweet spot} in the robustness-accuracy trade-off can be found. Unlike the \textit{hard} approaches that completely rule out the reliance on high-frequency components (i.e. low-pass filters), we propose to \textit{reduce} the reliance on them. To this end, we adopt a data augmentation approach that aims to diversify $\mathcal{T}_\text{train}$ by an operation $\mathcal{HA(\cdot)}$. Keeping the strong relation intact between labels and low-frequency content (i.e. labels come from low-frequency-component image), we propose to swap high and low-frequency components of images in a batch on-the-fly. Unlike \cite{mukai2022improving}, we \textit{do not} restrict the images to belong to the same class; this diversifies the training distribution even further while preserving the image semantics. We call this basic version of our approach \textit{HybridAugment}, which corresponds to: 
%
\begin{equation} \label{hybrid_augment_paired}
    \mathcal{HA_{P}}(x_{i}, x_{j}) = \mathcal{LF}(x_{i}) + \mathcal{HF}(x_{j})
\end{equation}
%
where $x_{i}$ is the input image and $x_{j}$ is a randomly sampled image from the whole training set, which we simply sample from the mini batch at each training iteration in practice. $\mathcal{HF}$ and $\mathcal{LF}$ operators select the high and low-frequency components of an input image, for which we use:
%
\begin{equation} \label{eq:cutoff}
\begin{split}
    \mathcal{LF}(x) = GaussBlur(x) \\
    \mathcal{HF}(x) = x - \mathcal{LF}(x)
    \end{split}
\end{equation}
%
where $GaussBlur$ is used as a low-pass filter. Note that a similar outcome is possible by using Discrete Fourier Transforms (DFT), swapping the frequency bands and then applying Inverse DFT (IDFT). We find the gaussian blur operation to be faster and better in practice. 


Inspired from \cite{chen2021amplitude}, in addition to the image-pair scheme in Eq.~\ref{hybrid_augment_paired}, we propose a single image variant of \textit{HybridAugment}. In the single image variant, instead of combining two images, $x_i$ and $x_{j}$ are obtained by applying randomly sampled augmentations to a single image. The single image variant $\mathcal{HA_{S}}$ can therefore be defined as 
%
\begin{equation} \label{hybrid_augment_single}
    \mathcal{HA_{S}}(x_{i}) = \mathcal{LF}(Aug(x_{i})) + \mathcal{HF}(\hat{Aug}(x_{i}))
\end{equation}
%
where $Aug$ and $\hat{Aug}$ correspond to two sets of randomly sampled augmentation operations. Note that paired and single versions can work in tandem ($\mathcal{HA_{PS}}$), and actually outperform single or paired image versions. 


\subsection{HybridAugment++}


The frequency analysis is a vast literature, however, two core aspects often stand out; frequency-band analysis (i.e. low, high) and the decomposition of signals into amplitude and phase. \textit{HybridAugment} covers the former and shows competitive results in various benchmarks (see Section \ref{sec:exp_hybridaugment}). The latter is investigated in $\mathcal{APR}$ \cite{chen2021amplitude}, where phase is shown to be the more relevant component for correct classification, and training models based on their phase labels and swapping amplitude components of images randomly lead to more robust models. Note that frequency-band and phase/amplitude discussions are arguably orthogonal, since frequency, phase and amplitude provide distinct characterizations of a signal: intuitively speaking, frequency, phase and amplitude can be seen as the separation of visual patterns in terms of scale, location and significance. 


We hypothesize these two approaches can be complementary; a model reliant on low-frequency and spatial information (i.e. phase) can further improve robustness. Inspired by the successes of cascaded augmentation methods \cite{hendrycks2019augmix,wang2021augmax,calian2022defending}, we unify these two core aspects into a single, hierarchical augmentation method. We refer to this method as \textit{HybridAugment++} and define its paired version as:
%
\begin{equation}
  \mathcal{HA_{P}}^{++}(x_{i}, x_{j}, x_{z}) = \mathcal{APR_{P}}(\mathcal{LF}(x_{i}), x_{z}) + \mathcal{HF}(x_{j})
\end{equation}
%
where $x_{i}$, $x_{j}$ and $x_{z}$ are images sampled from the same batch. Here, $\mathcal{APR_{P}}$~\cite{chen2021amplitude} is defined as
\begin{equation}
    \mathcal{APR_{P}}(x_{i}, x_{z}) = \mathcal{IDFT}(A_{x_{z}} \otimes e^{i. P_{x_{i}}}) \\
\end{equation}
%
where $\otimes$ is element-wise multiplication, $A$ is the amplitude and $P$ is the phase component. Similar to $\mathcal{HA}$ and $\mathcal{APR}$, we also define a single-image version of \textit{HybridAugment++} as
%
\begin{equation}
 \mathcal{HA_{S}}^{++}(x_{i}) = \mathcal{APR_{S}}(\mathcal{LF}(Aug(x_{i}))) + \mathcal{HF}(\hat{Aug}(x_{i}))
\end{equation}
%
where $\mathcal{APR_{S}}$~\cite{chen2021amplitude} is defined as
%
\begin{equation}
\mathcal{APR_{S}}(x_{i}) = \mathcal{IDFT}\left(A_{\bar{Aug}(x_{i})} \otimes e^{i. P_{\overline{Aug}\left(x_{i}\right)}}\right)    
\end{equation}
%
where $Aug$, $\hat{Aug}$, $\bar{Aug}$ and $\overline{Aug}$ are different sets of randomly sampled augmentation operations. Note that we essentially propose a framework; one can use different single and paired image augmentations, either individually or together, and can still achieve competitive results (see ablations in Section \ref{sec:exp_hybridaugment}). There are also other alternatives, such as swapping phase/amplitude first and then performing $\mathcal{HA}$, but we observe poor performance in practice; dividing the phase component into frequency-bands is not interpretable as frequencies of the phase component are not well defined. The pseudo-code of our methods can be found in the supplementary material.





%\begin{table*}[t!]
    \centering
    \small
    \begin{tabular}{rcl}
    \toprule
        & & The fraction of... \\
        \textbf{Rate of Revision} & $R/N$ & \hspace{1cm} time steps in which the model revises \\
        \textbf{Rate of Recomputation} & $R'/N$ & \hspace{1cm}  time steps in which the model recomputes \\
        \textbf{Rate of Active Recomputation} & $(R'\cap R)/R'$ & \hspace{1cm} recomputations that actually causes a revision \\
        \midrule
        \textbf{R-Pertinence} & $(R \cap I) / R$ & \hspace{1cm}  revisions that edit incorrect prefixes (adapted precision)  \\
        \textbf{R-Appropriateness} & $(R \cap I) / I$ & \hspace{1cm} incorrect prefixes that are revised (adapted recall) \\
        \textbf{A-Pertinence} & $(A \cap C) / A$ & \hspace{1cm}  additions upon correct prefixes (adapted precision) \\
        \textbf{A-Appropriateness} & $(A \cap C) / C$ & \hspace{1cm}  correct prefixes that are not revised (adapted recall) \\
        \midrule
        \textbf{R$_e$-Pertinence} & $(R_e \cap I) / R$ & \hspace{1cm}  revisions that effectively edit incorrect prefixes   \\
        \textbf{R$_e$-Appropriateness} & $(R_e \cap I) / I$ & \hspace{1cm} incorrect prefixes that are revised effectively \\
    \bottomrule
    \end{tabular}

    \caption{Proposed metrics for evaluating recomputation and revision policies. $N$ is the total number of time steps.}
    \label{table:metrics}
\end{table*}

Traditional sequence labelling evaluation metrics like accuracy or F1 can be computed on label, sequence or dataset level. The incremental dimension requires its own metrics, some of which we discussed in $\S$\ref{sec:litreview}. Here, we propose specific metrics to evaluate revision and/or recomputation policies. For each time step $t$ in a sequence, either a revision ($R$) occurred, which is sometimes effective ($R_{e}$), or only an addition ($A$). Assuming we have established a metric for prefix correctness,\footnote{A binary variable or a continuous variable, like accuracy, with a defined threshold for tolerated incorrectness.} we know whether the prefix at $t-1$ was correct ($C$) or incorrect ($I$). That results in a distribution of $N$ actions in $\{R, A\} \times \{C, I\}$. From these counts, we derive the metrics in Table \ref{table:metrics}, computed either per sequence or over the whole dataset. Models that have the option to \textit{recompute} ($R'$) can also be evaluated in $\{R', \neg R'\} \times \{C, I\}$ with two additional metrics. 

Since only \textit{effective} revisions are actually desired, the $R$ in the numerators can be replaced by $R_{e}$ for a more focused evaluation. Revisions can be further weighted by how often and how far in the sentence processing they happen. Similarly, edits can be assessed by their correction time and survival time \citep{baumann2013:phd}.


\section{Experiments}
% \haizhou{Follow the same way of introduction as we did in Section2.}
% \noindent In this section, we will introduce datasets and experimental setups that we used. Then we evaluate our method, other self-supervised methods, and supervised methods under different distribution shifts (\ie, concept shifts and covariate shifts) under common settings (\ie, transductive, inductive settings). It has to note that we focus on node-level tasks (\eg, node classification) in this work. As for graph-level tasks, we leave it as our future work and some simple experiments can be found in Appendix~\ref{app:graph_classification}. 
In this section, we first introduce the experimental setup including datasets, training, and evaluation protocol in Section~\ref{sec:dataset}~and~\ref{sec:unsupervised}. 
% Next, we present our experimental setup and conduct extensive experiments to evaluate our method in Section~\ref{sec:unsupervised}. 
We then perform an ablation study to demonstrate the effectiveness of each proposed component in Section~\ref{sec:ablation}. 
Additionally, we analyze the impact of important hyper-parameters in Section~\ref{sec:sensitivity}. 
Subsequently, we integrate our method with various encoding models, showcasing the model-agnostic nature of our recipe in Section~\ref{sec:other_models}. 
Finally, we provide some qualitative results such as feature visualization in Section~\ref{sec:vis}.
It is important to note that we focus on node-level tasks (\eg, node classification) in this work. As for graph-level tasks, we leave it as our future work, while some simple experiments are also provided in Appendix~\ref{app:graph_classification}.

\subsection{Datasets}\label{sec:dataset}
There exist some benchmarks for evaluating graph out-of-distribution generalization~\cite{good,ji2022drugood,gds}. 
Among them, GOOD~\cite{good} is the most representative and comprehensive benchmark that curates more diverse graph datasets with diverse tasks, including single/multi-task graph classification, graph regression, and node classification involving more distribution shifts (\ie, concept shifts and covariate shifts). Hence in this work, we follow the evaluation protocol proposed in \cite{good}. Furthermore, we validate the effectiveness of our method in the datasets (\ie, Amazon-Photo, Elliptic) that are used in EERM~\cite{eerm}. The statistics and detailed introduction to these datasets can be found in Table~\ref{tab:dataset} and Appendix~\ref{app:datasets}.

\begin{table*}[htp]
\caption{The descriptions of datasets. ``Domain-Level'' means splitting by graphs, ``Time-Aware'' denotes splitting according to chronological order.``Word'' and ``Degree'' represent splitting according to word diversity and node degree respectively. ``Language'' means splitting by user language, suggesting the prediction should not be impacted by the language the user use. ``University'' denotes splitting according to the domain university, implying that the prediction of webpages should be based on word contents and link connections rather than university features. ``Color'' means that nodes are split according to node differences in covariate shift and color-label correlations in concept shift.}
\label{tab:dataset}
\centering
\begin{tabular}{cccccccc}
\toprule
Datasets     & Network Type        & \#Nodes & \#Edges & \#Attributes &\#Classes& Train/Val/Test Split     & Metric   \\
% Cora         & Artificial Transformation & 2,703   &         &              &         &                      & Accuracy \\
Amazon-Photo\footnotemark
             & Co-purchasing network      & 7,650   & 119,081   & 755          & 10      & Domain-Level         & Accuracy \\
Elliptic\footnotemark  
             & Bitcoin transactions       & 203,769 & 234,355   & 165          & 2       & Time-Aware           & F1-Score \\
GOOD-Cora    & Scientific publications    & 19,793  & 126,842   & 8,710         & 70      & Word/Degree          & Accuracy \\
% GOOD-Arxiv   & arXiv papers               & 169,343 & 2,315,598 & 128          & 40      & Time/Degree          & Accuracy \\
GOOD-Twitch  & Gamer network              & 34,120  & 892,346   & 128          & 2       & Language             & ROC-AUC  \\
GOOD-CBAS    & A BA-house graph           & 700     & 3,962     & 4             & 4       & Color                & Accuracy \\
GOOD-WebKB   & Webpage network            & 617     & 1,138     & 1,703         & 5       & University           & Accuracy \\
\bottomrule
\end{tabular}
\end{table*}
\footnotetext[5]{This dataset is adopted from~\cite{yang2016revisiting}. \cite{eerm} constructs ten graphs with different environment id’s for each graph.} 
\footnotetext[6]{The original is available on \hyperlink{https://www.kaggle.com/ellipticco/elliptic-data-set}{https://www.kaggle.com/ellipticco/elliptic-data-set}}

\subsection{Unsupervised Representation Learning}\label{sec:unsupervised}
\subsubsection{Transductive Setting}~\label{sec:trans}
% \noindent\textbf{Baselines.}\quad We conduct experiments with 12 baselines which consist of three categories: supervised methods and self-supervised generative methods, self-supervised contrastive methods. Specifically, we compare with three supervised baselines: empirical risk minimization~(ERM)~\cite{erm}, invariant risk minimization (IRM)~\cite{irm}, and a recent proposed graph OOD method dubbed EERM~\cite{eerm}. We also compare various unsupervised node-level representation learning methods: three self-supervised generative methods including GAE~\cite{gae}, VGAE~\cite{gae}, GraphMAE~\cite{gmae} and seven self-supervised contrastive methods: DGI~\cite{dgi}, MVGRL~\cite{mvgrl}, GRACE~\cite{grace}, RoSA~\cite{rosa}, BGRL~\cite{bgrl}, COSTA~\cite{costa}, SwAV~\cite{swav}. The descriptions of these methods can be found in Appendix~\ref{app:baselines}.
In this subsection, we focus on validating our proposed algorithm under the transductive setting, where the test nodes will participate in message passing~\cite{gilmer2017neural} during training following~\cite{good}. 

\noindent\textbf{Baselines.} We conduct experiments with 12 baselines from three categories: (i)~supervised methods, including empirical risk minimization~(\textbf{ERM})~\cite{erm}, invariant risk minimization (\textbf{IRM})~\cite{irm}, and a recent proposed graph OOD method \textbf{EERM}~\cite{eerm}; (ii)~self-supervised generative methods including Graph Autoencoder (\textbf{GAE})~\cite{gae}, Variational Graph Autoencoder (\textbf{VGAE})~\cite{gae}, Self-Supervised Masked Graph Autoencoders (\textbf{GraphMAE})~\cite{gmae}; (iii)~self-supervised contrastive methods including Deep Graph Infomax (\textbf{DGI})~\cite{dgi}, Contrastive Multi-View Representation Learning on Graphs (\textbf{MVGRL})~\cite{mvgrl}, Deep Graph Contrastive Representation Learning (\textbf{GRACE})~\cite{grace}, A Robust Self-Aligned Framework for Node-Node Graph Contrastive Learning (\textbf{RoSA})~\cite{rosa}, Bootstrapped Representation Learning on Graphs (\textbf{BGRL})~\cite{bgrl}, Covariance-Preserving Feature Augmentation for Graph Contrastive Learning (\textbf{COSTA})~\cite{costa}, Unsupervised Learning of Visual Features by Contrasting Cluster Assignments (\textbf{SwAV})~\cite{swav}. The detailed descriptions of these baselines can be found in Appendix~\ref{app:baselines}.

\noindent\textbf{Experimental setup.} We use the same graph encoder across different datasets for a fair comparison following~\cite{good}. We use grid search to find other hyper-parameters (\eg, learning rate, epochs) for different methods. For all experiments, we select the best checkpoints for ID and OOD tests according to results on ID and OOD validation sets following~\cite{good}, respectively. Experimental details and hyper-parameter selections are provided in Appendix~\ref{app:hyper}. For evaluating unsupervised methods, a linear classifier will be built on the frozen trained encoder after finishing pre-training. The reported results are the mean performance with standard deviation after 10 runs following~\cite{good}.

\noindent\textbf{Analysis.}\quad Based on the experimental results listed in Table~\ref{tab:trans_concept} and \ref{tab:trans_covariate}, we can draw the following conclusions: firstly, we find strong self-supervised methods (\eg, GRACE, BGRL, COSTA) are more robust to distribution shifts (concept shift in Table~\ref{tab:trans_concept} and covariate shift in Table~\ref{tab:trans_covariate}) compared to supervised methods. For instance, on GOOD-CBAS and GOOD-WebKB datasets, GRACE surpasses the best supervised method by large margins (over 6\% absolute improvement). Interestingly, we find the methods designed for OOD generalization (\ie, IRM) and graph OOD generalization (\ie, EERM) do not attain superior performance than the standard ERM on most of the datasets. For example, EERM shows superior OOD performance compared to ERM in only one experiment, and IRM outperforms ERM in four out of ten experiments across the conducted evaluations. This phenomenon is also observed in \cite{good,ahuja2020empirical,rosenfeld2021risks}, showcasing the challenge of achieving invariant prediction in non-Euclidean graph settings. 

Furthermore, our method surpasses other SOTA self-supervised methods on the OOD test set of all datasets by a considerable margin while achieving comparable performance in the in-distribution test set. For instance, on small datasets such as GOOD-CBAS and GOOD-WebKB, our method outperforms GRACE\footnote{MARIO is built up on GRACE according to our recipe. So, we make a comparison with GRACE here.} by over 2\% absolute accuracy on the OOD test set. On larger datasets such as GOOD-Cora and GOOD-Twitch, our method still outperforms other methods which shows its superiority. For instance, under covariate shift, MARIO surpasses other methods by over 7\% absolute accuracy on the GOOD-Twitch OOD test set. These statistics prove the effectiveness of our design.


\begin{table*}[htp]
\caption{Experimental results of all methods under concept shift. The bold font means the top-1 performance and the underline represents the second performance across the unsupervised methods. 'ID' represents in-distribution test performance and 'OOD' means out-of-distribution test performance. (OOM: out-of-memory on a GPU with 24GB memory)}
\label{tab:trans_concept}
\centering
\scalebox{0.95}{
\begin{tabular}{l|cc|cc|cc|cc|cc}
\toprule
\toprule
\multirow{3}{*}{concept shift} & \multicolumn{4}{c|}{GOOD-Cora}                   & \multicolumn{2}{c|}{GOOD-CBAS} & \multicolumn{2}{c|}{GOOD-Twitch} & \multicolumn{2}{c}{GOOD-WebKB} \\
                           & \multicolumn{2}{c}{word} & \multicolumn{2}{c|}{degree}& \multicolumn{2}{c|}{color}    & \multicolumn{2}{c|}{language}   & \multicolumn{2}{c}{university} \\
                           & ID         & OOD         & ID          & OOD          & ID            & OOD           & ID             & OOD            & ID            & OOD            \\
\midrule
ERM                        & 66.38±0.45 & 64.44±0.18  & 68.60±0.40  & 60.76±0.34   & 89.79±1.39    & 83.43±1.19    & 80.80±1.00     & 56.92±0.92     & 62.67±1.53    & 26.33±1.09     \\
IRM                        & 66.42±0.41 & 64.29±0.31  & 68.57±0.35  & 61.45±0.24   & 89.64±1.21    & 82.29±1.14    & 78.87±1.04     & 59.30±1.79     & 62.67±1.10    & 26.88±1.42     \\
EERM                       & 65.10±0.44 & 62.45±0.19  & 66.95±0.44  & 56.58±0.25   & 79.07±2.12    & 64.50±1.01    & OOM            & OOM            & 62.50±2.01    & 28.07±3.23      \\
\midrule
% Random-Init                & 37.53±1.74 & 32.12±1.24  & 37.82±1.71  & 27.74±1.14   &               &               &                &                & 60.33±2.21    & 27.07±1.70     \\
GAE                        & 60.65±0.89 & 58.00±0.55  & 62.59±1.11  & 53.44±0.80   & 75.28±1.36    & 68.07±2.05    & 81.25±0.81     & 51.51±1.05     & 62.17±3.34    & 25.78±1.85     \\
VGAE                       & 63.19±0.53 & 60.35±0.47  & 61.65±0.66  & 54.28±0.28   & 76.50±0.50    & 59.07±0.56    & 80.46±0.53     & 55.56±4.53     & 62.50±2.38    & 24.40±2.57     \\
GraphMAE                   & \underline{66.44±0.46} & \underline{64.87±0.30}  & 67.95±0.46  & 59.41±0.39   & 89.14±0.89    & 82.93±0.93    & 80.05±0.64     & 59.38±1.49     & 61.83±3.37    & 29.27±2.15     \\
DGI                        & 63.33±0.56 & 60.71±0.49  & 65.93±1.02  & 55.83±0.53   & 91.22±1.47    & 85.00±1.66    & 80.05±0.87     & 59.16±1.88     & 61.83±2.83    & 28.63±1.92      \\
MVGRL                      & OOM        & OOM         & OOM         & OOM          & 88.57±1.15    & 76.50±1.17    & OOM            & OOM            & 62.00±3.79    & 28.26±4.20     \\
GRACE                      & 65.61±0.61 & 63.92±0.44  & \textbf{68.59±0.35}  & 60.15±0.45   & 92.00±1.39    & 88.64±0.67    & \textbf{83.43±0.63}     & \underline{60.45±1.46}     & 64.00±3.43    & \underline{34.86±3.43}  \\
RoSA                       & 64.06±0.67 & 62.44±0.39  & 67.07±0.65  & 57.68±0.44   & 90.78±2.27    & 85.93±2.14    & 82.39±0.42     & 57.45±2.16     & 64.17±4.10    & 32.20±2.15     \\
BGRL                       & 65.18±0.43 & 63.43±0.45  & 66.83±0.80  & 59.63±0.38   & 92.36±1.16    & 87.14±1.60    & 82.52±0.60     & 55.48±1.48     & 63.67±2.33    & 31.47±3.43     \\
COSTA                      & 65.05±0.80 & 62.37±0.45  & 66.76±0.87  & 55.73±0.36   & \underline{93.50±2.62}    & \underline{89.29±3.11}    & 83.15±0.30 & 55.03±3.22     & 61.66±2.58    & 32.39±2.13 \\
% ArCL                       &            &             & 67.64±0.57  & 59.71±0.44   &               &               &                &                & 65.00±3.94    & 35.41±1.97 \\      
SwAV                       & 62.22±0.53 & 59.79±0.53  & 64.65±0.94  & 55.06±0.39   & 89.00±0.79    & 81.72±0.66    & \underline{83.32±0.15}     & 59.69±1.97     & \underline{65.17±3.76}    & 29.36±2.01    \\
\midrule
MARIO                       & \textbf{67.11±0.46} & \textbf{65.28±0.34}  & \underline{68.46±0.40}  & \textbf{61.30±0.28}   & \textbf{94.36±1.21}    & \textbf{91.28±1.10}    & 82.31±0.54     & \textbf{63.33±1.72}     & \textbf{65.67±2.81}    & \textbf{37.15±2.37}     \\
\bottomrule
\end{tabular}}
\end{table*}

\begin{table*}[htp]
\caption{Experimental results of all methods under covariate shift. The bold font means the top-1 performance and the underline represents the second performance across the unsupervised methods. 'ID' represents in-distribution test performance and 'OOD' means out-of-distribution test performance. (OOM: out-of-memory on a GPU with 24GB memory)}
\label{tab:trans_covariate}
\centering
\scalebox{0.95}{
\begin{tabular}{l|cc|cc|cc|cc|cc}
\toprule
\toprule
\multirow{3}{*}{covariate shift} & \multicolumn{4}{c|}{GOOD-Cora}                                   & \multicolumn{2}{c|}{GOOD-CBAS} & \multicolumn{2}{c|}{GOOD-Twitch} & \multicolumn{2}{c}{GOOD-WebKB} \\
                           & \multicolumn{2}{c}{word} & \multicolumn{2}{c|}{degree}& \multicolumn{2}{c|}{color}    & \multicolumn{2}{c|}{language}   & \multicolumn{2}{c}{university} \\
                           & ID         & OOD         & ID          & OOD          & ID            & OOD           & ID             & OOD            & ID            & OOD            \\
\midrule
ERM                        & 70.50±0.41 & 64.69±0.33  & 72.46±0.49  & 55.53±0.50   & 92.00±3.08    & 77.57±1.29    & 70.98±0.41     & 49.35±5.09     & 39.34±1.79    & 14.52±3.14   \\
IRM                        & 70.48±0.26 & 64.53±0.57  & 71.98±0.34  & 53.72±0.46   & 90.86±2.41    & 78.86±1.67    & 69.81±0.95     & 49.11±2.82     & 38.52±3.30    & 13.97±2.80     \\
EERM                       & OOM        & OOM         & OOM         & OOM          & 65.00±2.57    & 57.43±3.60    & OOM            & OOM            & 46.07±4.55    & 27.40±7.65     \\
\midrule
GAE                        & 56.63±0.79 & 48.93±0.93  & 66.30±0.88  & 34.01±0.87   & 73.00±2.16    & 60.86±3.01    & 67.24±1.23     & 47.65±2.49     & 45.08±6.32    & 28.02±6.29    \\
VGAE                       & 62.02±0.66 & 54.12±0.86  & 69.41±0.57  & 44.20±1.29   & 62.29±2.04    & 63.29±1.11    & 66.99±1.43     & \underline{50.48±4.58}     & 48.85±4.68    & 20.87±6.69     \\
GraphMAE                   & 68.14±0.43 & 64.00±0.33  & \textbf{73.36±0.56}  & 53.75±0.55   & 67.28±3.03    & 67.28±1.49    & 68.84±1.20     & 48.02±2.79     & 48.03±4.34    & 30.00±8.09     \\
DGI                        & 60.85±0.75 & 57.03±0.67  & 68.97±0.41  & 41.75±0.88   & 69.57±4.09    & 59.71±3.43    & 68.43±1.05     & 44.83±1.61     & 48.52±5.04    & 21.11±7.50     \\
MVGRL                      & OOM        & OOM         & OOM         & OOM          & 65.00±1.94    & 64.15±0.77    & OOM            & OOM           & \textbf{54.10±5.39}    & 16.59±6.51     \\
GRACE                      & \underline{68.77±0.33} & \underline{64.21±0.41}  & 72.69±0.34  & \underline{56.10±0.63}   & \underline{93.57±1.83}    & \underline{89.29±3.40}    & \underline{71.12±0.87} & 46.21±1.54 & 49.67±5.82    & 28.10±4.68    \\
RoSA                       & 68.19±0.56 & 62.48±0.61  & 71.04±0.62  & 52.72±0.79   & 84.71±4.14    &79.14±3.51     & 70.58±0.36     & 45.83±1.72     & 52.30±4.24    & \underline{34.24±7.92}     \\
BGRL                       & 67.23±0.43 & 61.33±0.36  & 72.11±0.39  & 49.15±0.73   & 89.00±2.56    & 79.86±3.29    & \textbf{71.43±0.53}     & 43.86±0.94     & 51.80±5.55    & 30.32±7.61    \\
COSTA                      & 65.28±0.60 & 60.33±0.53  & 70.65±0.62  & 54.03±0.28   & 92.29±1.59    & 82.71±2.74    & 69.29±1.37     & 49.07±2.13     & 50.49±3.01    & 29.84±4.75   \\
SwAV                       & 63.29±1.01 & 56.98±0.94  & 70.27±0.73  & 43.00±0.52   & 89.57±1.12    & 81.43±1.69    & 69.19±0.93     & 49.37±2.96     & 49.84±4.82    & 30.55±6.72   \\
\midrule
MARIO                       & \textbf{69.99±0.54} & \textbf{65.06±0.34}  & \underline{72.73±0.43}  & \textbf{57.73±0.45}  & \textbf{94.57±2.46}    & \textbf{91.00±2.48}     & 68.31±0.78 & \textbf{57.37±1.37}     & \underline{53.94±3.23}    & \textbf{35.24±4.98}   \\
\bottomrule
\end{tabular}}

\end{table*}

\subsubsection{Inductive Setting}
In this subsection, we conduct experiments under the inductive settings, where the test nodes are kept unseen during training. This setting is more suitable for domain generalization.
% But we think it is more convincing that conduct experiments under inductive settings which means test nodes are unseen during training. This setting is more appropriate for domain generalization.

\noindent\textbf{Baselines:} For GOOD-WebKB and GOOD-CBAS datasets, we adopt ERM, IRM, GraphMAE, and GRACE as our baselines. And for Amazon-Photo and Elliptic datasets, we select ERM, EERM, and GRACE as our baselines.

\noindent\textbf{Experimental setup:} For GOOD-WebKB and GOOD-CBAS datasets, we use the same model configuration in Section~\ref{sec:trans}.
% Besides, we add experiments on Amazon-Photo dataset~\cite{yang2016revisiting} and Elliptic~\cite{elliptic} dataset in this subsection. 
For Amazon-Photo dataset~\cite{yang2016revisiting} and Elliptic~\cite{elliptic} dataset, they consist of many snapshots (training data and testing data use different snapshots) which are naturally inductive. For Amazon-Photo dataset, we use 2-layer GCN~\cite{gcn} as the encoder and for elliptic dataset, we use 5-layer GraphSAGE~\cite{sage} as encoder following~\cite{eerm}.

% Figure environment removed

\noindent\textbf{Analysis:}
According to Figure~\ref{fig:amazon},\ref{fig:elliptic},\ref{fig:ind_con},\ref{fig:ind_cov}, we can draw following conclusions:
firstly, based on Figure~\ref{fig:amazon}, it is evident that our method outperforms other representative supervised and self-supervised methods on all test graphs (T1$\sim$T8). This superiority is reflected in the larger median value of our method compared to others. For instance, MARIO achieves over a 3\% absolute improvement compared to ERM in terms of the mean value of eight median values. Additionally, our method demonstrates higher stability across different random initializations, as indicated by the closer proximity of the first and third quartile values to the median value~(\eg, the difference of first and third quartile values of ERM, EERM, GRACE and MARIO are 4.2, 3.3, 6.7 and 1.0 on T8 respectively which indicates MARIO is much more stable than other methods). Furthermore, our method exhibits consistent performance across different graphs (\eg, The standard deviation of median values on T1$\sim$T8 for ERM, EERM, GRACE, and MARIO are 0.4, 1.1, 1.2, and 0.3, respectively.), indicating its robustness to environmental variations and its ability to extract invariant features: $g(G^e) \approx g(G^{e'})$ for all $e, e' \in \mathcal{E}^\text{train}$. In summary, our method showcases enhanced OOD generalization capabilities.
% $g(G^e)g(G^e^\prime)$ where $any e, e^\prime in \mathcal{E}^{train}$

Secondly, from the results presented in Figure~\ref{fig:elliptic}, we can observe that our method averagely harvests 10.9\% absolute improvement over GRACE and 12.5\% absolute improvement over EERM in terms of F1 scores on Elliptic dataset. This demonstrates the effectiveness of our method in handling distribution shifts and improving performance compared to existing approaches. It is worth noting that GRACE's performance worsens over time, indicating its inability to handle distribution shifts effectively. In contrast, our method consistently achieves better F1 scores, except for T9, which is caused by the dark market shutdown occurred after T7~\cite{elliptic}. The emergence of such an event introduces significant variations in data distributions, which subsequently results in performance degradation for all methods. Indeed, this event serves as an unpredictable external factor that introduces significant challenges for models trained on limited training data. The results indicate that the performance heavily depends on available training data. Nonetheless, our approach outperforms other methods even in such an extreme case. This highlights the effectiveness of our method in addressing distribution shifts and improving generalization performance.

Finally, based on the observations from Figure~\ref{fig:ind_con} and Figure~\ref{fig:ind_cov} MARIO demonstrates the best performances on both ID and OOD test sets for GOOD-WebKB and GOOD-CBAS datasets, under both concept shift and covariate shift. Notably, MARIO outperforms other methods by more than 3\% and 10\% absolute improvement on GOOD-WebKB and GOOD-CBAS, respectively, under covariate shift. We can draw similar conclusions as discussed in Section~\ref{sec:trans}. Even under the inductive setting, our method continues to demonstrate excellent OOD generalization capabilities and achieves comparable or even improved in-distribution test performance. These statistical results further validate the effectiveness of our method in handling distribution shifts and enhancing generalization performance.

Overall, the observations we have made provide strong evidence of the great capacity of our method for handling distribution shifts, validating its effectiveness and potential for real-world applications.



% Figure environment removed

% Figure environment removed


% Figure environment removed


\subsection{Ablation Studies}\label{sec:ablation}
\noindent Table~\ref{tab:aba} provides a detailed analysis of the effect of each component according to our proposed recipe for improving OOD generalization in graph contrastive learning. Let's examine the different variants of our method and their impact on performance.
Specifically, MARIO~(w/o ad) represents MARIO without  adversarial augmentation. MARIO~(w/o cmi) denotes we only maximize the mutual information between positive pairs without considering conditional mutual information. MARIO~(w/o cmi, ad) means a vanilla graph contrastive method that is similar to GRACE. 

From Table~\ref{tab:aba}, we can find MARIO~(w/o cmi) lags far behind MARIO on OOD test set which demonstrates appropriately minimizing the redundant information (\ie, conditional mutual information) is essential to improve OOD generalization of GCL methods. And adversarial augmentation can also boost OOD generalization because it can approximately serve as a supermum operator to learn more invariant features  discussed in Section~\ref{sec:aug}. Based on the analysis of these variants, it is evident that the proposed improvements on data augmentation and contrastive loss in the recipe are both effective in enhancing graph OOD generalization. Each component contributes to the overall performance improvement, and their combination leads to a stronger self-supervised graph learner in terms of graph OOD generalization. 

In short, the findings from Table~\ref{tab:aba} support the rationale behind your proposed recipe and provide empirical evidence of the effectiveness of each proposed component. By incorporating these enhancements, our method achieves superior performance in handling distribution shifts and improving graph OOD generalization in graph contrastive learning.
\begin{table*}[htp]
\caption{Ablation studies for MARIO by masking each component.}
\label{tab:aba}
\centering
\scalebox{0.9}{
\begin{tabular}{l|cc|cc|cc|cc|cc}
\toprule
\toprule
\multirow{3}{*}{concept shift} & \multicolumn{4}{c|}{GOOD-Cora}                       & \multicolumn{2}{c|}{GOOD-CBAS} & \multicolumn{2}{c|}{GOOD-Twitch} & \multicolumn{2}{c}{GOOD-WebKB} \\
                           & \multicolumn{2}{c}{word} & \multicolumn{2}{c|}{degree}& \multicolumn{2}{c|}{color}    & \multicolumn{2}{c|}{language}   & \multicolumn{2}{c}{university} \\
                           & ID         & OOD         & ID          & OOD          & ID            & OOD           & ID             & OOD            & ID            & OOD            \\
\midrule
MARIO                      & \textbf{67.11±0.46} & \textbf{65.28±0.34}  & \textbf{68.46±0.40}  & \textbf{61.30±0.28}      & \textbf{94.36±1.21}  & \textbf{91.28±1.10}    & 82.31±0.54     & \textbf{63.33±1.72}     & \textbf{65.67±2.81}    & \textbf{37.15±2.37}     \\
MARIO(w/o ad)              & 66.23±0.53 & 64.02±0.18  & 67.88±0.38  & 60.46±0.29   & 93.21±1.25    & 90.29±0.91    & 82.42±0.73     & 60.50±1.02     & 64.83±2.83    & 36.51±3.25    \\
MARIO(w/o cmi)             & 65.32±0.60 & 63.51±0.32  & 68.14±0.32  & 61.19±0.34   & 94.15±1.23    & 90.57±1.96    & \textbf{82.51±0.56}     & 61.41±2.63     & 64.50±4.35    & 35.78±2.53     \\
MARIO(w/o cmi, ad)         & 64.67±0.55 & 63.11±0.32  & 67.95±0.65  & 60.01±0.57   & 93.36±1.66    & 89.64±1.73    & 81.90±0.75     & 60.12±1.60     & 64.17±3.67    & 34.13±2.38     \\
\bottomrule
\end{tabular}}
\end{table*}
% & 65.32±0.60 & 63.51±0.32 exchange 64.67±0.55 & 63.11±0.32
% 68.14±0.32       id ood test: 60.95±0.43       ood ood test: 61.19±0.34


\subsection{Sensitivity Analysis}\label{sec:sensitivity}
\noindent In this subsection, we will analyze some important hyper-parameters of our method. We conduct sensitivity analysis on GOOD-WebKB dataset with concept shift, we chose two sensitive hyper-parameters (\ie, the coefficient $\gamma$ of condition mutual information in Equation~\ref{equ:cmi} and the number of prototypes $|C|$ in Equation~\ref{equ:pq}). The coefficient of CMI range in $[0.001, 0.01, 0.1, 0.5, 1]$ and the number of prototypes $|C|$ ranges in $[10, 50, 100, 200, 300]$. From Figure~\ref{fig:sensitivity}, we can observe that $\gamma$ reaches 0.1 and $|C|$ reaches 100 or 200 can achieve the best OOD test accuracy. Both higher and lower values of $\gamma$ result in suboptimal performance. This finding aligns with previous research such as DIB~\cite{dib}, indicating that an appropriate compression level is crucial for achieving optimal performance. Extremely high or low compression values are not ideal. 

Regarding the number of prototypes $|C|$, based on the results shown in Figure~\ref{fig:sensitivity}, it is found that setting $|C|=100$ leads to the best performance in terms of OOD test accuracy. This choice provides a moderate number of pseudo labels, which is beneficial for the learning process. 

Based on the sensitivity analysis, we determined that setting $\gamma=0.1$ and $|C|=100$ on most datasets. These hyperparameter values strike a balance between compression level and the number of prototypes, resulting in improved graph OOD generalization.
% Figure environment removed


\subsection{Integrated with Other Models}\label{sec:other_models}
% Figure environment removed

\begin{table}[htp]
\caption{Results of different learning approaches with different encoding models (\ie, GCN, GraphSAGE, GAT).}
\label{tab:others}
\centering
\scalebox{0.9}{
\begin{tabular}{cc|cc|cc}
\toprule
\toprule
\multirow{3}{*}{Model}& \multirow{3}{*}{Method} & \multicolumn{2}{c|}{GOOD-CBAS} & \multicolumn{2}{c}{GOOD-WebKB} \\
                & & \multicolumn{2}{c|}{color}    & \multicolumn{2}{c}{university} \\
                &   & ID          & OOD         & ID          & OOD            \\
\midrule
\multirow{3}{*}{GCN} 
&ERM               & 89.79±1.39 & 83.43±1.19  &  62.67±1.53 & 26.33±1.09         \\
&GRACE             & 92.00±1.39 & 88.64±0.67  &  64.00±3.43 & 34.86±3.43        \\
&MARIO             & 94.36±1.21 & 91.28±1.10  &  65.67±2.81 & 37.15±2.37        \\ \bottomrule
\multirow{3}{*}{SAGE} 
&ERM               & 95.07±1.51 & 75.14±1.19  & 73.67±2.08  & 46.33±3.42       \\
&GRACE             & 95.29±1.11 & 74.43±2.36  & 70.50±5.06  & 49.54±3.83        \\
&MARIO             & 96.00±1.07 & 76.29±3.01  & 71.00±3.82  & 51.74±4.63        \\ \bottomrule
\multirow{3}{*}{GAT} 
&ERM               & 78.64±3.63 & 72.93±2.64  & 61.33±3.71  & 28.99±2.63        \\
&GRACE             & 84.57±1.79 & 78.36±1.60  & 59.50±2.36  & 35.78±3.26        \\
&MARIO             & 84.93±1.95 & 80.43±1.89  & 62.17±4.78  & 38.17±3.10        \\
\bottomrule
\end{tabular}}
\end{table}



\noindent In the subsection, we demonstrate the model-agnostic nature of the recipe by integrating it with various graph neural network (GNN) models, including GCN, GraphSAGE, and GAT.

From Table~\ref{tab:others}, it can be observed that regardless of the specific GNN model used as the encoder, our method consistently achieves the best performance on the OOD test set. This indicates the effectiveness and robustness of our method across different GNN models.
By achieving superior performance across different GNN models, MARIO demonstrates its versatility and ability to improve the OOD generalization of various graph neural models. This highlights the broad applicability and effectiveness of our recipe in enhancing the performance of different GNN encoders.

Furthermore, we integrate our recipe with other GCL methods in Appendix~\ref{app:other_methods}. The results demonstrate our recipe can boost the OOD generalization ability of various GCL methods which means our recipe can serve as a plug-in for many current classical GCL methods.

% Figure environment removed

\subsection{Visualization}\label{sec:vis}
\subsubsection{Metric Score Curves}
We present metric score curves for ERM and MARIO, including training, ID validation, ID testing, OOD validation, and OOD testing accuracy, in Figure~\ref{fig:curve2}. Notably, MARIO demonstrates superior convergence with approximately 10\% absolute improvement on the OOD test set compared to ERM. Furthermore, MARIO effectively narrows the performance gap between in-distribution and out-of-distribution performance, showcasing its efficacy in enhancing OOD generalization for graph data. More metric score curves can be found in Appendix~\ref{app:curves}.


\subsubsection{Feature Visualization}
In order to assess the quality of learned embeddings, we adopt t-SNE~\cite{tsne} to visualize the node embedding on GOOD-Cora dataset (concept shift in word domain) using random-init of GCN, EERM, GRACE, and MARIO, where different classes have different colors in Figure~\ref{fig:vis}. For clarity, we select eight classes with the largest number of nodes to enhance the informativeness and interpretability of the visualization. We can observe that the 2D projection of node embeddings learned by MARIO has a better separation of clusters, which indicates the model can help learn representative features for downstream tasks. It has to note that we depict both ID nodes and OOD nodes in the same figure. 

Besides, we also separately visualize ID nodes and OOD nodes in the different figures in the Appendix~\ref{app:feature}. And we can find MARIO performs a clearer separation of clusters whether on ID nodes or OOD nodes compared to other methods.





\section{Discussion}
\label{sec: discussion}
\kmsdelete{In this work} We study \kmsreplace{Fairness-Aware PAC learning}{Fair-ERM} in the malicious noise model, and  in some cases allow 
the learner to maintain optimal overall accuracy despite the signal in Group $B$ being almost entirely washed out.
%when we allow learners to use the
%$\PQ$ randomized expansion of the hypothesis class $\mathcal{H}$
In particular we show that different fairness constraints have fundamentally different behavior in the presence of Malicious Noise, in terms of the amount of accuracy loss that a given level of Malicious Noise could cause a fairness-constrained learner to incur. 
The key to achieving our results, which are more optimistic than those in \cite{lampert}, is allowing for improper learners using the (P,Q)-randomized expansions of the given class $\mathcal{H}$.
%We \kmsreplace{present a picture of the}{prove upper and lower bounds on}
%accuracy loss for a range of fairness notions, given \kmsreplace{this simple randomization step.}{learning over $\PQ$.
%In general our results indicate Fair-ERM (given learning over $\PQ$) is more robust than claimed in \cite{lampert}.
The type of smoothness we create by using $\PQ$ seems to be a natural property that is likely shared by many natural hypothesis classes.

Fairness notions are motivated as a response to learned disparities when there is \kmsdelete{data corruption or} systemic error affecting \kmsdelete{the data for}
one group. 
Fairness notions are supposed to mitigate this by ruling out classifiers that have worse performance on a sub-group. 
This can peg both classifiers at a lower level of performance \kmsdelete{(e.g that the lower subgroup)} in order to \emph{motivate} \cite{hardt16} improving the data collection or labelling process to obtain more reliable performance. 
%So in \kmsreplace{some}{a} sense, sensitivity of the fairness notion to poor sub-group performance caused by malicious noise is the \textit{point} of fairness constraints! 
However, it also desirable that fairness constraints perform gracefully when subject to Malicious Noise because fairness constraints will be used in contexts where the data is unreliable and noisy and this might not be known to the learner.
This tension, exposed by our work, motivates 
%a revisiting of fairness notions from first principles approach and trying to axiomatize the 
%desired properties of a fairness intervention a la cryptography and privacy. \footnote{Work in multi-calibration \cite{multicalib} is a viable direction for this problem but it is unclear how 
%that and related notions behave with unreliable data. }
on going work studying the sensitivity level of fairness constraints. 
%If we we are to take a view, if a classifier is deployed 


\section{Conclusion and Future Work}
In this work, I design corruption-robust algorithms for the Lipschitz contextual search problem. I present the \emph{agnostic checking} technique and demonstrate its effectiveness in designing corruption-robust algorithms. There are several open problems for future research. First, in the algorithm I propose for pricing loss, the schedule for agnostic checks is fixed upfront. Can the learner design an adaptive checking schedule for the pricing loss? Second, this work assumes the learner has knowledge of the Lipschitz constant $L$. Can the learner design efficient no-regret algorithms without knowledge of $L$? 

\begin{comment}
\section*{Broader Impact}

This work consists solely of simulations, thus alleviating some of the ethical concerns, as well as concerns regarding any consequences emerging due to the failure of the system presented. With regards to the ethical aspects related to its inclusion in the field of Artificial Intelligence, we argue that our work aims to have positive outcomes on the development of human-machine interfaces, albeit being not yet mature enough to aim for this goal. The current state of our work does not allow us to extrapolate towards negative outcomes.

This work should benefit the research community of language emergence and grounding, in its current state.
\end{comment}

\begin{ack}
%\section*{Acknowledgements}
This work was supported by the EPSRC Centre for Doctoral Training in Intelligent Games \& Games Intelligence (IGGI) [EP/L015846/1]. 

We gratefully acknowledge the use of Python\cite{python-2009}, IPython\cite{ipython-perez-2007}, SciPy\cite{SciPy-NMeth2020}, Scikit-learn\cite{Scikit-learn:JMLR:v12:pedregosa11a}, Scikit-image\cite{scikit-image-van2014}, NumPy\cite{NumPy-Array2020}, Pandas\cite{pandas1-mckinney-proc-scipy-2010,pandas2-reback2020}, OpenCV\cite{opencv_library}, PyTorch\cite{pytorch-paszke-NEURIPS2019_9015}, TensorboardX\cite{huang2018tensorboardx}, Tensorboard from the Tensorflow ecosystem\cite{tensorflow2015-whitepaper}, without which this work would not be possible.
\end{ack}


% Bibliography:
% biblatex:
%\printbibliography
% natbib:
\bibliography{bibli}

\newpage
\appendix
%

\section{Broader impact}

No technology is safe from being used for malicious purposes, which equally applies to our research. However, we view many of the ethical concerns surrounding research to be mitigated in the present case. These include data-related concerns such as fair use or issues surrounding use of human subjects, given that our data consists solely of simulations.

With regards to the ethical aspects related to its inclusion in the field of Artificial Intelligence, we argue that our work aims to have positive outcomes on the development of human-machine interfaces since we investigate, among other things, alignment of emergent languages with natural-like languages.

The current state of our work does not allow extrapolation towards negative outcomes.
We believe that this work is of benefit to the research community of reinforcement learning, language emergence and grounding, in their current state.

%However, aiming to develop artificial agents that relies on the same symbolic behaviours and the same social assumptions (e.g. using CLBs) than human beings is aiming to reduce misunderstanding between human and machines.
%Thus, the current work is targeting benevolent applications.
%Subsequent works around the benchmark that we propose are prompted to focus on emerging protocols in general (not just posdis-compositional languages), while still aiming to provide a better understanding of artificial agent's symbolic behaviour biases and differences, especially when compared to human beings, thus aiming to guard against possible misunderstandings and misaligned behaviours. 

\section{Implementation Details}
\label{sec:implementation-details}

\begin{wraptable}{R}{0.45\linewidth}%[h]
%\centering
\vspace{-50pt}
\caption{Hyper-parameter values relevant to R2D2 in the different architectures presented. All missing parameters follow the ones in Ape-X~\citep{horgan2018apex}.}
\label{table:hyperparams}
\begin{tabular}{@{} ll @{}} 
\toprule
\multicolumn{2}{c}{R2D2} \\ 
\midrule
\multicolumn{1}{l}{Number of actors} & \multicolumn{1}{c}{32} \\
\multicolumn{1}{l}{Actor update interval} & \multicolumn{1}{c}{1 env. step} \\
\multicolumn{1}{l}{Sequence unroll length} & \multicolumn{1}{c}{20} \\
\multicolumn{1}{l}{Sequence length overlap} & \multicolumn{1}{c}{10} \\
\multicolumn{1}{l}{Sequence burn-in length} & \multicolumn{1}{c}{10} \\
\multicolumn{1}{l}{N-steps return} & \multicolumn{1}{c}{3} \\
\multicolumn{1}{l}{Replay buffer size} & \multicolumn{1}{c}{$1\times10^4$ obs.} \\
\multicolumn{1}{l}{Priority exponent} & \multicolumn{1}{c}{0.9} \\
\multirow{2}{0.45\linewidth}{Importance sampling exponent} & \multicolumn{1}{c}{0.6} \\
& \\
\multicolumn{1}{l}{Discount $\gamma$} & \multicolumn{1}{c}{$0.98$} \\
\multicolumn{1}{l}{Minibatch size} & \multicolumn{1}{c}{64} \\
\multicolumn{1}{l}{Optimizer} & \multicolumn{1}{c}{Adam~\citep{kingma2014adam}} \\
\multicolumn{1}{l}{Learning rate} & \multicolumn{1}{c}{$6.25\times10^{-5}$} \\
\multicolumn{1}{l}{Adam $\epsilon$} & \multicolumn{1}{c}{$10^{-12}$} \\
\multirow{2}{0.35\linewidth}{Target network update interval} & \multicolumn{1}{c}{2500} \\
&  \multicolumn{1}{c}{updates} \\
%\multicolumn{2}{l}{Target network update interval} & \multicolumn{2}{c}{2500 updates} \\
\multicolumn{1}{l}{Value function rescaling} & \multicolumn{1}{c}{None} \\
%h(x) = sign(x)(√|x|+ 1−1) +x, = 10−3
\bottomrule
\end{tabular}
\vspace{-45pt}
\end{wraptable}

Table~\ref{table:hyperparams} highlights the hyperparameters used for the off-policy RL algorithm, R2D2\citep{kapturowski2018recurrent}.
More details can be found, for reproducibility purposes, in our open-source implementation at HIDDEN-FOR-REVIEW-PURPOSES.

%Training was performed for each run on 1 NVIDIA GTX1080 Ti, and the average amount of training time for a run is 18 hours, on an observation budget of 5 million samples, with 32 actors.
Training was performed for each run on 1 NVIDIA GTX1080 Ti, and the amount of training time for a run is between 8 and 24 hours depending on the architecture. 
%, on an observation budget of 5 million samples, with 32 actors.


\section{On HIGhER's Ablation Study}
\label{sec:higher-ablation-study}

Prior to the architectures described in the main part of the paper, we iterated over many designs and induction biases.
%Firstly, we experimented with R2D2's features such as the burn-in feature.
Notably we experimented with R2D2's burn-in feature.

Table~\ref{tab:PickUp-higher-no-burnin} shows the success ratios of HIGhER agents without burn-in feature against baseline R2D2 without burn-in feature on the modified (one-pick-up) PickUpDist-v0 task from the BabyAI benchmark at the end of the $200k$ observation budget.
The results show that the contrastive learning scheme for the predicate function is rather hurting performance compared to HIGhER+, while still being above baseline.
The burn-in feature provides the RL agent better sample-efficiency by stabilising the training of the recurrent network in the architecture.
While the instruction generator/speaker agent is being trained, the resulting goal re-labelled experiences that enters the replay buffer are presumably non-stationary.
Thus, we attribute the lower performance of the above architectures to the fact that they struggle to deal with the non-stationarity of the goal re-labelled experiences in the absence of the stabilising burn-in feature. 

\begin{table}[b]%{l}{0.6\textwidth}
\caption{Success ratios (mean and standard deviation) for agents without the burn-in feature of R2D2 after 200k steps in a modified version of the BabyAI PickUpDist-v0 task. 3 random seeds for each agent.}
\label{tab:PickUp-higher-no-burnin}
\centering
%    \renewcommand{\arraystretch}{1.5}
\begin{tabular}{@{}lccc@{}} 
%\begin{tabular}{c|c|c}
\toprule
    \textbf{Agent} & \textbf{Mean} & \\ \hline
    R2D2 (w/o Burn-In) & 13.02 $\pm$ 1.26 & \\ \hline
    \midrule
    HIGhER+ (w/o Burn-In) & 16.02 $\pm$ 1.79 & \\ \hline
    HIGhER++ (n=1) (w/o Burn-In) & 14.97 $\pm$ 1.19 & \\ \hline
    HIGhER++ (n=2) (w/o Burn-In) & 15.89 $\pm$ 0.60 & \\ \hline
    HIGhER++ (n=4) (w/o Burn-In) & 13.93 $\pm$ 2.29 & \\
    \bottomrule
    \end{tabular}
\end{table}

\begin{comment} 
We also experimented with features related to our incremental improvements.
Most notably, we experimented with a version of HIGhER that does not share the observation encoder in the visual module.
We will refer to it as Agnostic-HIGhER. %\todo[]{replace values in the below table for agnostic higher with or withou burn-in, need to specify}

\begin{table}[t] %{L}{0.53\textwidth}
\caption{Success ratios (mean and standard deviation) for Agnostic agent variants with the burn-in feature of R2D2 after 200k steps in a modified version of the BabyAI PickUpDist-v0 task.}
\label{tab:PickUp-agnostic-higher}
\centering
%    \renewcommand{\arraystretch}{1.5}
\begin{tabular}{@{}lccc@{}} 
%\begin{tabular}{c|c|c}
\toprule
    \textbf{Agent} & \textbf{Mean} & \\ \hline
    R2D2 & 16.54 $\pm$ 1.37 & \\ \hline
    \midrule
    Agnostic-HIGhER+ & 0.1602 $\pm$ 0.0179 & \\ \hline
    Agnostic-HIGhER++ (n=1) & 0.1497 $\pm$ 0.01193 & \\ \hline
    Agnostic-HIGhER++ (n=2) & 0.1589 $\pm$ 0.00597 & \\ \hline
    Agnostic-HIGhER++ (n=4) & 0.1393 $\pm$ 0.02289 & \\
    \bottomrule
    \end{tabular}
\end{table}

%Table~\ref{tab:PickUp-agnostic-higher} shows the success ratios of HIGhER agents without burn-in feature against baseline R2D2 without burn-in feature either on the modified (one-pick-up) PickUpDist-v0 task from the BabyAI benchmark at the end of the $200k$ observation budget.
\todo[inline]{Analyse the results...}
%We interpret this as follows: without a shared observation encoder, the contrastive learning scheme does not provide any feedback to the RL agent, thus the performance should not increase in this context, but the fact that the performance are hurt could be solely due to \todo[inline]{the fact that the quality of the predicate function is not good enough yet to be used in the hindsight relabelling scheme.}
\end{comment}



\section{On algorithmic details of ETHER}
\label{sec:ether-details}

In this section, we detail how ETHER is built over HIGhER from an algorithmic point of view.
We start by presenting in Algorithm~\ref{alg:HIGhER} an extended version of the pseudo-code for the HIGhER algorithm from \citet{cideron2020higher} with the following additions:

\begin{enumerate}
    \item (i) contrasting further the \textbf{learning} vs \textbf{behavioural} reward function concerns that we highlighted in Section~\ref{subsec:her-limitations}, 
    \item (ii) flagging the reliance on the \textbf{learning} reward function that depends on the predicate function, which is provided as an oracle in both HER and HIGhER.
\end{enumerate}

\begin{algorithm}[H]
\caption{Hindsight Generation for Experience Replay (HIGhER)}
\label{alg:HIGhER}
\SetKwInOut{Given}{Given}
\SetKwInOut{Initialize}{Initialize}
\SetKwRepeat{Do}{repeat}{until}
\Given{
\begin{itemize}
    \item an off-policy RL algorithm (e.g., DQN, R2D2) and its replay buffer $R$, 
    \item a behavioural policy $\pi_{behaviour}: \mathcal{S} \times \mathcal{G} \rightarrow \mathcal{A}$
    \item a \textbf{learning} reward function $r: \mathcal{S} \times \mathcal{A} \times \mathcal{S} \times \mathcal{G} \rightarrow \mathbb{R}$ {\color{red}(oracle or learned - relying on the predicate function $f: \mathcal{S} \times \mathcal{G} \rightarrow \{0,1\}$)}, 
    \item a \textbf{behavioural} reward function $r: \mathcal{S} \times \mathcal{A} \times \mathcal{S} \times \mathcal{G} \rightarrow \mathbb{R}$ (provided by the environment), 
    \item a language scoring function (e.g., parser accuracy, BLEU, etc.).
\end{itemize}
}
\Initialize{ 
\begin{itemize}
    %\item the behavioural policy $\pi_{behaviour}$, 
    %\item the replay buffer $R$, 
    \item the dataset $\mathcal{D}_{sup}$ of $(state, goal)$ pairs and a train-test split strategy to yield $\mathcal{D}_{sup/train}$ and $\mathcal{D}_{sup/val}$, 
    \item the Instruction Generator $m_{HIGhER}$.
\end{itemize}
}
\For{$episode = 1, M$}{
    Sample a goal $g$ and an initial state $s_0$\ from the environment\;
    $t = -1$\;
    \Do{(episode ends)}{
        $t = t + 1$\;
        Execute an action $a_t$ chosen from the behavioural policy $\pi_{behavioural}$\;
        Observe a new state $s_{t+1}$ and a {\color{purple}\textbf{behavioural} reward $r_t = r_{behavioural}(s_t, a_t, g)$}\;
        Store the transition $(s_t, a_t, r_t, s_{t+1}, g)$ in $R$\;
        Update Q-network parameters using the policy $\pi_{behavioural}$ and sampled minibatches from $R$\;
    }
    \If{{\color{blue} \textbf{learning} reward $r_{learning}(s_t, a_t, s_{t+1}, g) = f(s_{t+1}, g) = = 1$}}{
        Store the pair $(s_{t+1}, g)$ in $\mathcal{D}_{sup/train}$ or $\mathcal{D}_{sup/val}$\;
        Update $m_{HIGhER}$ parameters by sampling minibatches from $\mathcal{D}_{sup/train}$\;
    }
    \ElseIf{$m_{HIGhER}$ validation score is high enough \& $\mathcal{D}_{sup/val}$ is big enough}{
        %Use the \textbf{final} re-labelling strategy as follows...\;
        Duplicate the previous episode's transitions in $R$\;
        Sample $\hat{g}^0 = m_{HIGhER}(s_{t+1})$\;
        %Replace $g$ by $\hat{g}^0$ in {\color{red} \textbf{ all the duplicated transitions} of the last episode}\;
        Compute the {\color{blue}\textbf{learning} rewards $\forall t, \hat{r}^0_t = r_{learning}(s_t, a_t, s_{t+1}, \hat{g}^0) = f(s_{t+1}, \hat{g}^0)$}\;
        Replace $g$ by $\hat{g}^0$ and $r_t$ by $\hat{r}^0_t$ in {\color{red}\textbf{all the duplicated transitions} of the last episode}\;
    }
}
\end{algorithm}


Following the added nuances to the HIGhER algorithm, we can now show in greater and contrastive details the ETHER algorithm in Algorithm~\ref{alg:ETHER}, where we highlight the following:

\begin{enumerate}
    \item (i) the RG training can be done in parallel at any time, thus we present it in the most-inner loop of the algorithm,
    \item (ii) since ETHER trains its RG speaker and listener agents on the whole state space $\mathcal{S}$, the ability to perform either \textbf{final} or \textbf{future-k} re-labelling strategy is recovered. We present the case of the \textbf{future-k} re-labelling strategy below.
\end{enumerate}


\begin{algorithm}[H]
\caption{Emergent Textual Hindsight Experience Replay (ETHER)}
\label{alg:ETHER}
\SetKwInOut{Given}{Given}
\SetKwInOut{Initialize}{Initialize}
\SetKwRepeat{Do}{repeat}{until}
\Given{
\begin{itemize}
    \item an off-policy RL algorithm (e.g., DQN, R2D2) and its replay buffer $R$, 
    \item a behavioural policy $\pi_{behaviour}: \mathcal{S} \times \mathcal{G} \rightarrow \mathcal{A}$
    \item a descriptive, discriminative RG algorithm, with its dataset buffer $\mathcal{D}_{RG}$ and its listener and speaker agents\;
    \item a \textbf{learning} reward function $r: \mathcal{S} \times \mathcal{A} \times \mathcal{S} \times \mathcal{G} \rightarrow \mathbb{R}$ {\color{green}( relying on the predicate function $f: \mathcal{S} \times \mathcal{G} \rightarrow \{0,1\}$ which is implemented via the RG's listener agent)}, 
    \item a \textbf{behavioural} reward function $r: \mathcal{S} \times \mathcal{A} \times \mathcal{S} \times \mathcal{G} \rightarrow \mathbb{R}$ (provided by the environment), 
    \item a language scoring function {\color{blue}(implemented via the RG's accuracy on the validation set)}.
\end{itemize}
}
\Initialize{ 
\begin{itemize}
    %\item the behavioural policy $\pi_{behaviour}$, 
    %\item the replay buffer $R$, 
    \item the dataset $\mathcal{D}_{sup}$ of $(state, goal)$ pairs and a train-test split strategy to yield $\mathcal{D}_{sup/train}$ and $\mathcal{D}_{sup/val}$, 
    \item the RG dataset $\mathcal{D}_{RG}$ of stimuli $state$ and a train-test split strategy to yield $\mathcal{D}_{RG/train}$ and $\mathcal{D}_{RG/val}$, 
    \item the Instruction Generator {\color{blue} $m_{ETHER}(\cdot)$, in the form of the RG's speaker agent}.
    \item {\color{blue} the learned predicate function $f_{ETHER}(\cdot)$, in the form of the RG's listener agent},
    \item {\color{blue} $K_{HER}\in\mathbb{N}$ specifying which re-labelling strategy to use (if $K_{HER}=0$ then \textbf{final}, otherwise \textbf{future-$K_{HER}$}).}
\end{itemize} 
}
\For{$episode = 1, M$}{
    Sample a goal $g$ and an initial state $s_0$\ from the environment\;
    $t = -1$\;
    \Do{episode ends}{
        $t = t + 1$\;
        Execute an action $a_t$ chosen from the behavioural policy $\pi_{behavioural}$\;
        Observe a new state $s_{t+1}$ and a {\color{purple}\textbf{behavioural} reward $r_t = r_{behavioural}(s_t, a_t, g)$}\;
        Store the transition $(s_t, a_t, r_t, s_{t+1}, g)$ in $R$\;
        Update Q-network parameters using the policy $\pi_{behavioural}$ and sampled minibatches from $R$\;
        {\color{blue} Store the stimulus $s_t$ in $\mathcal{D}_{RG/train}$ or $\mathcal{D}_{RG/val}$}\;
        {\color{blue} Update the RG's speaker and listener agents by playing $N_{RG}$ epochs of the RG, training on $\mathcal{D}_{RG/train}$ and performing evaluation on $\mathcal{D}_{RG/val}$}\;
    }
    \If{{\color{blue} \textbf{learning} reward $r_{learning}(s_t, a_t, s_{t+1}, g) = f_{ETHER}(s_{t+1}, g) = = 1$}}{
        Store the pair $(s_{t+1}, g)$ in $\mathcal{D}_{sup/train}$ or $\mathcal{D}_{sup/val}$\;
        Update {\color{blue}the RG's speaker agent parameters (ETHER) with supervised learning} by sampling minibatches from $\mathcal{D}_{sup/train}$\;
    }
    \ElseIf{\color{blue} RG validation accuracy on $\mathcal{D}_{RG/val}$ is high enough}{
        {\color{blue} Use the \textbf{future-$K_{HER}$} re-labelling strategy as follows...}\;
        $k = 0$, $T =$ last episode's length\;
        \Do{$k = = K_{HER}$}{
            Sample $T_{k}$ uniformly from $[1,T]$\; 
            Duplicate the previous episode's transitions in $R$, until sampled timestep $T_{k}$\;
            Sample $\hat{g}^0 = m_{ETHER}(s_{T_k})$\;
            %Replace $g$ by $\hat{g}^0$ in {\color{red} \textbf{ all the duplicated transitions} of the last episode}\;
            Compute the {\color{blue}\textbf{learning} rewards $\forall t, \hat{r}^0_t = r_{learning}(s_t, a_t, s_{t+1}, \hat{g}^0) = f_{ETHER}(s_{t+1}, \hat{g}^0)$}\;
            Replace $g$ by $\hat{g}^0$ and $r_t$ by $\hat{r}^0_t$ in {\color{red}\textbf{all the duplicated transitions} of the last episode}\;
            $k = k + 1$\;
        }
    }
}
\end{algorithm}
\section{On the Referential Game in ETHER}
\label{sec:ether-rg}

In the following, we detail further the referential game (RG) used in the ETHER architectures.

As highlighted in Section~\ref{subsec:emecom}, we follow the nomenclature proposed in \citet{DenamganaiAndWalker2020a} and focus on a \textit{descriptive object-centric (partially-observable) $2$-players/$L=10$-signal/$N=0$-round/$K=31$-distractor} RG variant, as illustrated in Figure~\ref{fig:ether-rg}.

The descriptiveness implies that the target stimulus may not be passed to the listener agent, but instead replaced with a descriptive distractor.
In effect, the listener agent's decision module therefore outpus a $K+2$-logit distribution where the $K+2$-th logit represents the meaning/prediction that none of the $K+1$ stimuli is the target stimulus that the speaker agent was `talking' about. 
The addition is made following \citet{Denamganai2023visual-COMPODIS} as a learnable logit value, $logit_{no-target}$, it is an extra parameter of the model. 
In this case the decision module output is no longer as specified in Equation~\ref{eq:discr-stgs}, but rather as follows:
\begin{equation}
\label{eq:descr-discr-stgs}
    p((d_i)_{i\in[0,K+1]} | (s_i)_{i\in[0,K]}; m) 
    = 
    Softmax \Bigl( ( h^l_L \cdot f(s_i)^T )_{i\in[0,K]} \cup \{\textcolor{red}{logit_{no-target}}\} 
    \Bigr).
\end{equation}

The descriptiveneness is ideal but not necessary in order to employ the listener agent as a predicate function for the hindsight experience replay scheme.
Thus, in the main results of the paper, we present the version without descriptiveness.

In the remainder of this section, we detail the STGS-LazImpa loss that we employed in our referential game, as illustrated in Figure~\ref{fig:ether-rg}.

%\todo[inline]{Linguistic Functions. Discuss \citet{Wu2021-entropy-decomposition}'s entropy decomposition trick to align with \citet{jakobson1960linguistics}'s functions of language, and how \citet{Lowe2019}'s concepts of positive signalling and positive listening factors in, in order to emphasise how our ETHER builds over each of those.}

%\todo[inline]{Discuss that ETHER allows using the future strategy from HER, whereas THER does not ; thanks to meaningful rewards along the way of multi-objective instructions, e.g. opening doors before picking up an object.}


\subsection{STGS-LazImpa Loss}
\label{subsec:stgs-lazimpa}

Emergent languages rarely bears the core properties of natural languages \citep{Kottur2017,Bouchacourt2018, Lazaridou2018, Chaabouni2020}, such as Zipf’s law of Abbreviation (ZLA). 
In the context of natural languages, this is an empirical law which states that the more frequent a word is, the shorter it tends to be~\citep{zipf2016human-zla, strauss2007word-zla}.
\citet{rita2020lazimpa} proposed LazImpa in order to make emergent languages follow ZLA.

To do so, Lazimpa adds to the speaker and listener agents some constraints to make the speaker lazy and the listener impatient.
Thus, denoting those constraints as $\mathcal{L}_{STGS-lazy}$ and $\mathcal{L}_{impatient}$, we obtain the STGS-LazImpa loss as follows:

\begin{equation}
\label{eq:stgs-lazimpa-loss}
    \mathcal{L}_{STGS-LazImpa} (m, (s_i)_{i\in[0,K]}) 
    =
    \mathcal{L}_{STGS-lazy}(m) 
    +
    \mathcal{L}_{impatient}(m, (s_i)_{i\in[0,K]}) .
\end{equation}

In the following, we detail those two constraints.

\textbf{Lazy Speaker.} The Lazy Speaker agent has the same architecture as common speakers. The ‘Laziness’ is originally implemented as a cost on the length of the message $m$ directly applied to the loss, of the following form:

\begin{equation}
\label{eq:lazy-loss}
\mathcal{L}_{lazy}( m ) = \alpha(acc)|m|
\end{equation}
where $acc$ represents the current accuracy estimates of the referential games being played, and $\alpha$ is a scheduling function, which is not differentiable. 
This is aimed to adaptively penalize depending on the message length.
Since the lazyness loss is not differentiable, they ought to employ a REINFORCE-based algorithm for the purpose of credit assignement of the speaker agent.

In this work, we use the STGS communication channel, which has been shown to be more sample-efficient than REINFORCE-based algorithms~\citep{Havrylov2017}, but it requires the loss functions to be differentiable.
Therefore, we modify the lazyness loss by taking inspiration from the variational autoencoders (VAE) literature~\citep{Kingma2013-VAE}.

The length of the speaker's message is controlled by the appearance of the EoS token, wherever it appears during the message generation process that is where the message is complete and its length is fixed.
Symbols of the message at each position are sampled from a distribution over all the tokens in the vocabulary that the listener agent outputs.
Let $(W_l)$ be this distribution over all tokens $w\in V$ at position $l\in [1,L]$, such that $\forall l\in[1,L],\, m_l \sim (W_l)$. 
We devise the lazyness loss as a Kullbach-Leibler divergence $D_{KL}(\cdot | \cdot)$ between these distribution and the distribution $(W_{EoS})$ which attributes all its weight on the EoS token.
Thus, we dissuade the listener agent from outputting distributions over tokens that deviate too much from the EoS-focused distribution $(W_{EoS})$, at each position $l$ with varying coefficients $\beta(l)$.
The coefficient function $\beta: [1,L] \rightarrow \mathbb{R}$ must be monotically increasing.
We obtain our STGS-lazyness loss as follows:
\begin{equation}
\label{eq:stgs-lazyness-loss}
    \mathcal{L}_{STGS-lazy}(m) 
    =
    \sum_{l\in[1,L]} 
    \beta(l)
    D_{KL} \Bigr( 
    (W_{EoS}) |
    (W_l)
    \Bigl)
\end{equation}

\textbf{Impatient Listener.} Our implementation of the Impatient
Listener agent follows the original work of \citet{rita2020lazimpa}: it is designed to guess the target stimulus as soon as possible, rather than solely upon reading the EoS token at the end of the speaker's message $m$. 
Thus, following Equation~\ref{eq:discr-stgs}, the Impatient Listener agent outputs a probability distribution over a set of $K+1$ stimuli $(s_0, ..., s_K)$ for all sub-parts/prefixes of the message $m=(m_1,...,m_l)_{l\in[1,L]}=(m_{\leq l})_{l\in[1,L]}$ :
\begin{equation}
\label{eq:impatient-discr-stgs}
    \forall l \in [1,L], \;\;
    p( \mathbf{(d^{\leq l}_i)_{i\in[0,K]}} | (s_i)_{i\in[0,K]} ; \mathbf{m^{\leq l}}) = 
    Softmax \Bigr( 
    ( \mathbf{h_{\leq l}} \cdot f(s_i)^T )_{i\in[0,K]} 
    \Bigl),
\end{equation}
where $\mathbf{h_{\leq l}}$ is the hidden state/output of the recurrent network in the language module (cf. Section~\ref{app:model-architecture}) after consuming tokens of the message from position $1$ to position $l$ included.

Thus, we obtain a sequence of $L$ probability distributions, which can each be contrasted, using the loss of the user's choice, against the target distribution $(D_{target})$ attributing all its weights on the decision $d_{target}$  where the target stimulus was presented to the listener agent.
Here, we employ \citet{Havrylov2017}'s Hinge loss.
Denoting it as $\mathbb{L}(\cdot)$, we obtain the impatient loss as follows:
\begin{equation}
\label{eq:impatient-loss}
    \mathcal{L}_{impatient/\mathbb{L}}( m, (s_i)_{i\in[0,K]}) 
    =
    \frac{1}{L}
    \sum_{l\in[1,L]}
    \mathbb{L}( (d^{\leq l}_{i\in[0,K]}, (D_{target}) ).
\end{equation}

\section{On the Semantic Co-Occurrence Grounding Loss}
\label{sec:co-occurrence-grounding}

In the following, we describe further how the semantic co-occurrence grounding loss is implemented in the ETHER+ architecture.
In Section~\ref{subsec:co-occurrence}, we introduced the \textbf{semantic co-occurrence grounding loss}, which aims to enhance an agent's language grounding ability during RG training. 
To do so, only the words/tokens present in the linguistic goal description provided are used as labels. 
Formally, let us define a linguistic goal description as a series of tokens, $g=(g_i)_{i\in [1,L]} \in \mathcal{G}$, where $L$ is the maximum sentence length hyperparameter, as defined in the RG setup.

Thus, for each of those token present in the goal $g$ of a given episode (out of all the tokens available in the vocabulary $V$, as defined in the RG setup), the semantic co-occurrence grounding loss will aim to bring a prior semantic-only embedding of the tokens closer to the visual embeddings of all the observations during the given episode.
We will denote by $(\lambda_w)_{w\in V}$ all the prior semantic-only embeddings for the vocabulary V.
And, on the other hand, it will also bring further away from the visual embeddings of all the observations during the episode the prior semantic-only embeddings of \textbf{all the tokens of the vocabulary that are not present in the current goal $g$}.

The semantic co-occurrence grounding loss is contrastive and inspired by \citet{radford2021CLIP}. 
More formally, as we defined $f(\cdot)$ as the visual module in Section~\ref{app:model-architecture}, we write the semantic co-occurrence grounding loss as follows:

\begin{equation}
\label{eq:semantic-co-occurrence-grounding-loss}
    \mathcal{L}^{sem.}_{co-occ.\, ground} ( g | (\lambda_w)_{w\in V} ) = \mathbb{E}_{s \sim \rho^\pi} \Biggl[ \sum_{w\in V} \mathcal{H}(w) \sum_{g_i \in g} \biggl( \mathbf{1}_{w}(g_i) - \frac{\lambda_w \cdot f(s)^T}{||\lambda_w ||_2 \cdot ||f(s)||_2} \biggr)^2 \Biggr],
\end{equation}

where $||\cdot||_2$ corresponds to the $L2$ norm, $\rho^\pi$ is the distribution over states in $s\in\mathcal{S}$ that is induced by using the policy $\pi$ to harvest the observations/stimuli, and ${1}_{w}(\cdot)$ is a noisy indicator function defined as follows:
\begin{equation}
\label{eq:indicator-function}
    \mathbf {1}_{w}(w^\prime):= (1-\epsilon_{noise}) \times \begin{cases}1~&{\text{ if }}~w^\prime = = w~,\\-1~&{\text{ if }}~w^\prime \neq w~.\end{cases}
\end{equation}
where $\epsilon_{noise}$ is some random noise uniformly sampled from $[0,0.2]$, following the noisy labels idea proposed in \citet{salimans2016improved-techniques-for-training-gans}.

As the loss is implemented over mini-batches of sampled stimuli, we also performing masking to reject tokens with null entropy over the mini-batch. For instance, in the proposed experiments performed on BabyAI~\citep{Chevalier-Boisvert2018}'s PickupDist-v0 task, the linguistic goal description always contains the prefix `pick up', therefore, when considering a mini-batch of stimuli (however they may come from different episodes), the likelihood of the tokens `pick' and `up' is maximal over the mini-batch and therefore their associated appearance distribution over the sampled stimuli will have null entropy. In Equation~\ref{eq:semantic-co-occurrence-grounding-loss}, $\mathcal{H}(w)$ denote the entropy of the appearance distribution of token $w\in V$.

\begin{comment}
\todo[inline]{address bootstrappig concerns of learning both f and lambdas at the same time}

\todo[inline]{talk about BYOL and self-supervised inspiration}
\end{comment}
\section{On the Semantic Co-Occurrence Hypothesis}
\label{sec:co-occurrence}

% Blue
\begin{wrapfigure}{R}{0.65\linewidth}
    %\centering
    \vspace{-20pt}
    \begin{subfigure}{0.32\textwidth}
        \centering
        % Figure removed
    \end{subfigure}
    \begin{subfigure}{0.32\textwidth}
        \centering
        % Figure removed
    \end{subfigure}
    \begin{subfigure}{0.32\textwidth}
        \centering
        % Figure removed
    \end{subfigure}
    \begin{subfigure}{0.32\textwidth}
        \centering
        % Figure removed
    \end{subfigure}
    \begin{subfigure}{0.32\textwidth}
        \centering
        % Figure removed
    \end{subfigure}
    \begin{subfigure}{0.32\textwidth}
        \centering
        % Figure removed
    \end{subfigure}
    \begin{subfigure}{0.32\textwidth}
        \centering
        % Figure removed
        \caption{Expert agent}
    \end{subfigure}
    \begin{subfigure}{0.32\textwidth}
        \centering
        % Figure removed
        \caption{Random agent}
    \end{subfigure}
    \caption{\textbf{Left:} Trajectories for the blue color goal from BabyAI's built-in expert agent which always reaches the goal. \textbf{Right:} Random agent trajectories. In both cases the semantics of the goal are among the most observed semantic features for any given trajectory. This effect is less pronounced in the random agent.}
    \label{fig:appendix_co_occurrence_blue}
\end{wrapfigure}    

In Section~\ref{subsec:co-occurrence}, we hypothesised that, upon specifiying a goal, agent observations would be biased to contain semantic components present in said goal. 
We tested this hypothesis in the BabyAI environment and provided some examples in Figure~\ref{fig:co_occurrence}, when the linguistic goal description was ``Pick up the green key'', showing that the semantics of the goal (colour ``green'' and shape ``key'') are some of the most salient observed semantics in the environment's observations of both expert trajectories and random walks.

Here, we provide further evidence that the linguistic goal description aligns with the observed semantics across different permutations of color goal and shape goal. As described in the main body text, this is consistent across both agents, but more visible in the expert agent. Figures \ref{fig:appendix_co_occurrence_none}, \ref{fig:appendix_co_occurrence_green}, \ref{fig:appendix_co_occurrence_blue}, \ref{fig:appendix_co_occurrence_grey}, \ref{fig:appendix_co_occurrence_purple}, \ref{fig:appendix_co_occurrence_red}, and \ref{fig:appendix_co_occurrence_yellow} present histograms for each combination of color and shape goal for both the expert and random agent. We note that semantic co-occurrence, while prevalent, is not always perfectly the case. For instance, Figure \ref{fig:appendix_co_occurrence_blue}, the most commonly observed semantic in the expert agent trajectories for the blue color and ball shape was "box", as opposed to the expected "ball" semantic.




% None
% Figure environment removed


% Green
% Figure environment removed


% Grey
% Figure environment removed

% Purple
% Figure environment removed

% Red
% Figure environment removed

% Yellow
% Figure environment removed


\end{document}
