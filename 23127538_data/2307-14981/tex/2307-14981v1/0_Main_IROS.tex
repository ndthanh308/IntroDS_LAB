%%%%%%%%%%%%%%%%%%%%%%%%%%%%%%%%%%%%%%%%%%%%%%%%%%%%%%%%%%%%%%%%%%%%%%%%%%%%%%%%
%2345678901234567890123456789012345678901234567890123456789012345678901234567890
%        1         2         3         4         5         6         7         8

\documentclass[letterpaper, 10 pt, conference]{ieeeconf}  % Comment this line out if you need a4paper

%\documentclass[a4paper, 10pt, conference]{ieeeconf}      % Use this line for a4 paper

\IEEEoverridecommandlockouts                              % This command is only needed if 
  % you want to use the \thanks command

\overrideIEEEmargins                                      % Needed to meet printer requirements.

%In case you encounter the following error:
%Error 1010 The PDF file may be corrupt (unable to open PDF file) OR
%Error 1000 An error occurred while parsing a contents stream. Unable to analyze the PDF file.
%This is a known problem with pdfLaTeX conversion filter. The file cannot be opened with acrobat reader
%Please use one of the alternatives below to circumvent this error by uncommenting one or the other
%\pdfobjcompresslevel=0
%\pdfminorversion=4

% See the \addtolength command later in the file to balance the column lengths
% on the last page of the document

% The following packages can be found on http:\\www.ctan.org
%\usepackage{graphics} % for pdf, bitmapped graphics files
%\usepackage{epsfig} % for postscript graphics files
%\usepackage{mathptmx} % assumes new font selection scheme installed
%\usepackage{times} % assumes new font selection scheme installed
%\usepackage{amsmath} % assumes amsmath package installed
%\usepackage{amssymb}  % assumes amsmath package installed


% add by jiadai
% https://tex.stackexchange.com/questions/247104/hyperref-doesnt-link-cite-command
\makeatletter
\let\NAT@parse\undefined
\makeatother
\usepackage[pagebackref=false,breaklinks=true,colorlinks=true,bookmarks=false,linkcolor={red!50!black},urlcolor={magenta!80!black},citecolor={green!60!black}]{hyperref} 
% \usepackage[pagebackref=false,breaklinks=true,colorlinks,bookmarks=false]{hyperref}
\usepackage{url}
\usepackage{pifont}
\usepackage{booktabs} % To thicken table lines

% 将模板中Table caption大写+换行 改为小写+不换行
% https://tex.stackexchange.com/questions/166814/table-caption-in-uppercase-i-dont-know-why
\usepackage{etoolbox}
\makeatletter
\patchcmd{\@makecaption}
  {\scshape}
  {}
  {}
  {}
\makeatletter
\patchcmd{\@makecaption}
  {\\}
  {.\ }
  {}
  {}
\makeatother
% \def\tablename{Table}
% \usepackage[
% singlelinecheck=false % <-- important
% ]{caption}


%\usepackage[numbers, sort]{natbib} %Ensure that references are not out of order
\def\jiadai#1{{\color{cyan}{\bf [Jiadai:} {{#1}}{\bf ]}}}


\usepackage{graphicx}
\usepackage{subfigure}
\usepackage{lipsum}
\usepackage{cite}
\usepackage{bm}
\usepackage{esvect}
\usepackage{amsmath}
\usepackage[ruled,vlined]{algorithm2e}
\usepackage{color}
\usepackage{amsmath,amssymb}
\usepackage{tabularx}
\usepackage{multirow}
\usepackage{arydshln}
% \usepackage[pagebackref,breaklinks,colorlinks]{hyperref}
\usepackage[dvipsnames]{xcolor}
%\usepackage[subtle]{savetrees}
%\usepackage[margin=2cm]{geometry}
\usepackage{tikz,amsmath, amssymb,bm,color, amsthm,amsfonts}
\usetikzlibrary{positioning, calc,chains,fit,shapes}
%\usetikzlibrary{circuits.logic.US,circuits.logic.IEC,fit}
\usepackage{enumerate}
\usepackage{comment}
\usepackage{tikz}
\usepackage{graphics}
%\usepackage[cm]{fullpage}
\usepackage{longtable}
\usepackage{mdframed}
\usepackage{caption}
\usepackage{subcaption}
\usepackage{slashbox}
\usepackage{url}
\usepackage{framed}
\usepackage{array}
\usepackage{tabu}
\usepackage{lscape}
\usepackage{multirow}
\usepackage{ulem}
\usepackage{multicol}
\usepackage{placeins}
\usepackage{cite}
\usepackage{enumitem}
\usepackage{mathtools}
%\usepackage[numbers]{natbib}
%\usepackage{mathtools}
%\usepackage{authblk}

\mdfsetup{skipabove=2pt,skipbelow=2pt}
%\setlenght {\marginparwidth }{2cm}
%\usepackage{todonotes}

%\usepackage{floatrow}
%\usepackage{adjustbox}
%\setlength{\extrarowheight}{.05ex}
%\renewcommand\thesubfigure{\roman{subfigure}}


%\newtheorem{theorem}{Theorem}[section]
%\newtheorem{lemma}[theorem]{Lemma}
%\newtheorem{observation}[theorem]{Observation}
%\newtheorem{corollary}[theorem]{Corollary}
%\newtheorem{proposition}[theorem]{Proposition}
%\newtheorem{definition}[theorem]{Definition}
\newtheorem{construction}{Construction}
%\newtheorem{conjecture}{Conjecture}
%\newtheorem{remark}[theorem]{Remark}

\newcommand{\pname}[1]{\textnormal{\textsc{#1}}}
\newcommand{\cclass}[1]{\textnormal{\textsf{#1}}}
\newcommand{\nog}{nine} % no of members in the gang!
\newcommand{\nogd}{nineteen} % no of members in the gang - for deletion/completion
\newcommand{\nogl}{eighteen} % no of members in the larger gang - for editing
\newcommand{\nogld}{thirty eight} % no of members in the larger gang - for deletion/completion
\newcommand{\diffnog}{ten} %
%\newcommand{\dominatedby}{dominated by} %
%\newcommand{\dominatingset}{dominating set} %
%\newcommand{\dominates}{dominates} %
\newcommand{\simulates}{simulates} %
\newcommand{\baseset}{base} %
\newcommand{\issimulatedby}{is simulated by} %

\newcommand{\StarSAT}{\pname{8-SAT$_{\geq 6}$}}
\newcommand{\FSAT}{\pname{4-SAT$_{\geq 2}$}}
\newcommand{\FISAT}{\pname{5-SAT$_{\geq 3}$}}
\newcommand{\SIXSAT}{\pname{6-SAT$_{\geq 4}$}}
\newcommand{\ESAT}{\pname{8-SAT$_{\geq 6}$}}
\newcommand{\KSAT}{\pname{$k$-SAT$_{\geq {k-2}}$}}
\newcommand{\KSATO}{\pname{$k$-SAT}}
\newcommand{\ESATO}{\pname{8-SAT}}
\newcommand{\FSATO}{\pname{4-SAT}}
\newcommand{\FISATO}{\pname{5-SAT}}
\newcommand{\TSAT}{\pname{3-SAT}}
\newcommand{\HED}{\pname{${H}$-free Edge Deletion}}
\newcommand{\AEE}{\pname{${A}$-free Edge Editing}}
\newcommand{\AED}{\pname{${A}$-free Edge Deletion}}
\newcommand{\TSED}{\pname{$t$-star-free Edge Deletion}}
\newcommand{\ATSED}{\pname{Annotated $t$-star-free Edge Deletion}}
\newcommand{\AFSED}{\pname{Annotated $4$-star-free Edge Deletion}}
\newcommand{\FSED}{\pname{$4$-star-free Edge Deletion}}
\newcommand{\FVSED}{\pname{$5$-star-free Edge Deletion}}
\newcommand{\HEE}{\pname{${H}$-free Edge Editing}}
\newcommand{\HEC}{\pname{${H}$-free Edge Completion}}
\newcommand{\HDEE}{\pname{${H'}$-free Edge Editing}}
\newcommand{\HDDEE}{\pname{${H''}$-free Edge Editing}}
\newcommand{\HDED}{\pname{${H'}$-free Edge Deletion}}
\newcommand{\HDEC}{\pname{${H'}$-free Edge Completion}}
\newcommand{\HBEE}{\pname{${\overline{H}}$-free Edge Editing}}
\newcommand{\HBED}{\pname{${\overline{H}}$-free Edge Deletion}}
\newcommand{\HBEC}{\pname{${\overline{H}}$-free Edge Completion}}
\newcommand{\HOEDCE}{\pname{${H_1}$-free Edge Deletion(Completion/Editing)}}
\newcommand{\HEDCE}{\pname{${H}$-free Edge Deletion(Completion/Editing)}}
\newcommand{\HEEDC}{\pname{${H}$-free Edge Editing(Deletion/Completion)}}
\newcommand{\HDEEDC}{\pname{${H'}$-free Edge Editing(Deletion/Completion)}}
\newcommand{\BFED}{\pname{Bow-free Edge Deletion}}
\newcommand{\ABFED}{\pname{Annotated Bow-free Edge Deletion}}
\newcommand{\DTIS}{\pname{Distance-3 Independent Set}}
\newcommand{\SVC}{\pname{Strong Vertex Cover}}
\newcommand{\CLIQUE}{\pname{Clique}}
\newcommand{\IS}{\pname{Independent Set}}
\newcommand{\PFS}{\pname{Propagational-$f$ Satisfiability}}
\newcommand{\RHED}{\pname{Restricted ${H}$-free Edge Deletion}}
\newcommand{\RHEC}{\pname{Restricted ${H}$-free Edge Completion}}
\newcommand{\RHDED}{\pname{Restricted ${H'}$-free Edge Deletion}}
\newcommand{\RHDEC}{\pname{Restricted ${H'}$-free Edge Completion}}
\newcommand{\RHEE}{\pname{Restricted ${H}$-free Edge Editing}}
\newcommand{\PH}{$\cclass{NP} \subseteq \cclass{coNP/poly}$}
\newcommand{\NOPH}{$\cclass{NP} \not\subseteq \cclass{coNP/poly}$}
\newcommand{\LG}{\mathcal{W}}
\newcommand{\LGD}{\mathcal{W}'}
\newcommand{\LGDD}{\mathcal{W}''}


%\let\oldvee\vee
\renewcommand\vee{\boxtimes}

\newcommand\addvmargin[1]{
  \node[fit=(current bounding box),inner ysep=#1,inner xsep=0]{};
}
\setlength{\fboxrule}{0pt}

\newcommand{\defstage}[2]{% PGD Version
  \hfill\\\smallskip\noindent%
  \begin{tabularx}{\textwidth}{|l X|}%
    \hline%
    \multicolumn{2}{|l|}{\textbf{#1}}\\%
    &#2\\\hline%
  \end{tabularx}%
%  \smallskip%
}
\setlength\extrarowheight{15pt}

\newcounter{rowcntr}[table]
\renewcommand{\therowcntr}{\thetable.\arabic{rowcntr}}

% A new columntype to apply automatic stepping
\newcolumntype{N}{>{\refstepcounter{rowcntr}\therowcntr}c}

% Reset the rowcntr counter at each new tabular
\AtBeginEnvironment{longtabu}{\setcounter{rowcntr}{0}}

\newcounter{rowcntra}[table]
\renewcommand{\therowcntra}{\arabic{rowcntra}}

% A new columntype to apply automatic stepping
\newcolumntype{M}{>{\refstepcounter{rowcntra}\therowcntra}c}

% Reset the rowcntr counter at each new tabular
\AtBeginEnvironment{tabular}{\setcounter{rowcntra}{0}}

\newcommand{\NPC}{NP-Complete}


\newcommand{\highlight}[1]{\textcolor{blue}{#1}}
\newcommand{\dhanya}[1]{\textcolor{blue}{dhanya: #1}}


%\newcommand{\XCD1}[1]{\pname{$\chi_{cd}$\ensuremath{(#1)}}}
\newcommand{\XCD}{\pname{$\chi_{cd}$}}
\newcommand{\SC}{\pname{$\omega_{s}$}}

\newcommand{\CDC}{\textsc{CD-coloring}}
\newcommand{\SCP}{\textsc{Separated-Cluster}}
\newcommand{\TD}{\textsc{Total Domination}}
\newcommand{\ISP}{\textsc{Independent Set}}
\newcommand{\CC}{\textsc{Clique Cover}}
\newcommand{\TETHS}{Further, the problem cannot be solved in time \ensuremath{2^{o(|V(G)|)}}, unless the ETH fails}
%\usetikzlibrary{positioning,chains,shapes,calc}
\usetikzlibrary{fit}
\thispagestyle{empty}
\usetikzlibrary{
  graphs,
  graphs.standard
}
% \newcommand{\zhelun}[1]{\textcolor{red}{#1}}

\newcommand{\Tref}[1]{Table~\ref{#1}}
\newcommand{\Eref}[1]{Equation~\eqref{#1}}
\newcommand{\Fref}[1]{Figure~\ref{#1}}
\newcommand{\Sref}[1]{Section~\ref{#1}}
\newcommand{\tabref}[1]{Tab.~\ref{#1}}
\newcommand{\equref}[1]{Eq.~\eqref{#1}}
\newcommand{\figref}[1]{Fig.~\ref{#1}}
\newcommand{\secref}[1]{Sec.~\ref{#1}}

\newcommand{\cmark}{\ding{51}}%
\newcommand{\xmark}{\ding{55}}%

\makeatletter
\usepackage{xspace}
%% the \onedot macro is producing only one dot at line ends.
%% thus \etal. will not produce et al..
\DeclareRobustCommand\onedot{\futurelet\@let@token\@onedot}
\def\@onedot{\ifx\@let@token.\else.\null\fi\xspace}
\def\eg{e.g\onedot} \def\Eg{E.g\onedot}
\def\ie{i.e\onedot} \def\Ie{I.e\onedot}
\def\cf{cf\onedot} \def\Cf{Cf\onedot}
\def\etc{etc\onedot} \def\vs{vs\onedot}
\def\wrt{w.r.t\onedot} \def\dof{d.o.f\onedot}
\def\etal{\emph{et al}\onedot}
% \def\etal{{et al}\onedot}
\makeatother


\newcommand{\chenming}[1]{\textbf{\textcolor[rgb]{0.80,0.00,0.0}{[Chenming::~#1~]}}}
\newcommand{\zhelun}[1]{\textbf{\textcolor[rgb]{0.00,0.00,0.80}{[Zhelun::~#1~]}}}
% \newcommand{\jiadai}[1]{\textbf{\textcolor[rgb]{0.00,0.00,0.80}{[Jiadai::~#1~]}}}
\title{\LARGE \bf
%Out-of-Trajectory Novel Driving Views Synthesis via \\Neural Radiance Field with HD-Map Fusion
MapNeRF: Incorporating Map Priors into Neural Radiance Fields \\ for Driving View Simulation
}


\author{Chenming Wu, Jiadai Sun, Zhelun Shen and Liangjun Zhang % <-this % stops a space
\thanks{Preprint version, accepted by IEEE/RSJ International Conference on Intelligent Robots and Systems (IROS) 2023.}
\thanks{
All authors are with Robotics and Autonomous Driving Lab (RAL), Baidu Research Institute, P.R. China. Z. Shen is the corresponding author.
{\tt\small \{wuchenming, sunjiadai, shenzhelun, liangjunzhang\}@baidu.com}}%
}


\begin{document}

\maketitle
\thispagestyle{empty}
\pagestyle{empty}


%%%%%%%%%%%%%%%%%%%%%%%%%%%%%%%%%%%%%%%%%%%%%%%%%%%%%%%%%%%%%%%%%%%%%%%%%%%%%%%%
\begin{abstract}
Simulating camera sensors is a crucial task in autonomous driving. Although neural radiance fields are exceptional at synthesizing photorealistic views in driving simulations, they still fail in generating extrapolated views. This paper proposes to incorporate map priors into neural radiance fields to synthesize out-of-trajectory driving views with semantic road consistency. The key insight is that map information can be utilized as a prior to guide the training of the radiance fields with uncertainty. Specifically, we utilize the coarse ground surface as uncertain information to supervise the density field and warp depth with uncertainty from unknown camera poses to ensure multi-view consistency. Experimental results demonstrate that our approach can produce semantic consistency in deviated views for vehicle camera simulation.

\end{abstract}

%%%%%%%%%%%%%%%%%%%%%%%%%%%%%%%%%%%%%%%%%%%%%%%%%%%%%%%%%%%%%%%%%%%%%%%%%%%%%%%%
\section{Introduction}

Autonomous driving (AD) %with deep learning networks 
has shown promising achievements and is considered an important technological breakthrough that could revolutionize the future of transportation. Currently, ensuring the safety of autonomous driving systems has become a topic of extensive development.
% There has been much discussion on how to verify the safety of autonomous driving systems.
One traditional solution for safety tests is to exhaustively enumerate real scenarios for validation. Nevertheless, this process is not only labor-intensive and costly but also dangerous. Simulation has emerged as a robust, safe, and efficient alternative for training and evaluating AD software and algorithms~\cite{li2019aads, amini2020learning, amini2022vista}.

% Figure environment removed

Recently, neural radiance field (NeRF)~\cite{mildenhall2020nerf} has gained significant attention in AD simulation~\cite{drivesim}. This approach leverages multi-view images to construct a 3D scene and enable novel view synthesis for both indoor and outdoor applications. When it comes to constructing NeRF models in AD simulation, there are two options available: 1) collecting a large amount of data to cover as many viewpoints as possible, and constructing a fine-grained scene offline; 2) directly using log data from road tests to quickly create an environment and dynamically simulate driving scenarios. The first choice can deliver high-quality simulation~\cite{tancik2022block} by transforming the problem of view extrapolation into view interpolation through the use of large amounts of data. However, it is time- and cost-intensive, which makes it challenging to generalize. As for the second choice, the collected images from log data are usually similar to each other along the running trajectory, which may result in unsatisfactory outcomes, particularly when the camera pose is placed out-of-trajectory (see \figref{figSupportComp} as an example), semantic consistency cannot be guaranteed when synthesizing images from deviated views. We observe this problem under this data condition in all neural radiance approaches, and to the best of our knowledge, none of the existing work has solved this issue.
In our opinion, semantic consistency is crucial for AD simulation, and synthesizing on deviated views is unavoidable for scalability.

AD simulation usually involves map data for planning and control, which can be obtained from a prebuilt High-Definition Map (HD Map) or an online mapping module. While the map data may not be pixel-perfect, it can provide semantic-level information that is useful for enhancing the semantic consistency of the trained neural radiance field.
In this paper, we propose incorporating map priors into neural radiance fields to enhance the semantic consistency and rendering quality of deviated driving view synthesis. Firstly, we employ ground information from maps to supervise the density field of NeRF, providing a more reliable road base for semantic entities. Next, we propose sampling rays to simulate unseen views. Unlike most NeRF augmentation methods~\cite{zhang2022ray, chen2022geoaug}, we utilize ground and lane information in sampling computations to guide the radiance field. More importantly, we model the above two supervision methods as weak supervision by using an uncertainty parameter and propose an uncertainty tempering scheme to increase the uncertainty. This ensures that map priors only guide the training process rather than enforce it towards their absolute values. As a result, our proposed method not only improves the rendering quality of interpolated novel view synthesis quantitatively but also enhances the semantic consistency of deviated novel view synthesis. 
Our contributions can be summarized as follows:
% We summarize the contributions of this paper as follows.



% To overcome the limitations of the collected data, this paper proposes a novel approach that leverages map information to enhance the semantic consistency of the synthesized driving views. 

% Autonomous driving (AD) vehicles are being trained with the help of deep learning networks and have shown promising achievements. This technology is considered to be a breakthrough that could change the way of transportation in the near future. However, there are many discussions on how to verify or judge the safety of autonomous driving systems.
% A straightforward solution towards the safety tests is to exhaustively enumerate real scenarios for validation as many as possible. However, the process of implementing different real scenarios is not only labor-intensive and costly, but also dangerous. Simulation has been proved to be an alternative, which is robust, safe, efficient in training, and evaluating AD software and algorithms.
% Now, the emerging technology of neural radiance field (NeRF)~\cite{} leverages multi-view images to construct a 3D scene and enable novel view synthesis for many indoor and outdoor applications. For AD simulation, there are two choices for constructing NeRF models: 1) collect a large amount of data, such as LiDAR and camera data, similar to mapping, to construct a fine-grained scene offline; or 2) directly use the log file (typically in the format of ROS bag) to rapidly create an environment and then dynamically simulate the driving scenarios.
% The first choice can achieve high-quality simulation, but it is time-consuming and expensive, making it difficult to generalize to very large scales. On the other hand, the second option is fast but can lead to low-quality simulation due to the data being sparse and similar to each other in log data. This paper tackles the problem raised by choosing the latter option and attempts to improve the quality of out-of-trajectory driving view synthesis by incorporating map information. This approach is practical for many autonomous driving tests.
% In conclusion, the use of NeRF technology for AD simulation is a promising avenue for training and evaluating AD software and algorithms. While both options for constructing NeRF models have their pros and cons, this paper addresses the challenges of the second option and proposes a potential solution to improve the quality of simulation.

%There exist a few attempts to facilitate training a NeRF model for synthesizing out-of-trajectory (or called as extrapo trajectory) views.


\begin{itemize}
    \item We propose a novel method to incorporate commonly used map priors in AD scenes into neural radiance fields to improve the out-of-trajectory driving view synthesis.
    \item We explicitly model the uncertainty in map priors as a parameter and propose an uncertainty tempering scheme to guide the training process of the neural radiance field.
    \item Experiments demonstrated that the proposed method can improve the semantic consistency of out-of-trajectory views and the rendering quality of novel view trajectory interpolation.
\end{itemize}

Our proposed method is easy to implement, can be easily plugged into existing NeRF algorithms, and has the capability of extending to other formats of priors.
\section{Related Work}
\label{sec:related}
We surveyed various lines of research that our work draws upon, and grouped them into two main areas: \emph{i)} AI regulation and governance, and \emph{ii)} responsible AI practices and toolkits. 

\subsection{AI Regulation and Governance}
The landscape of AI regulation and governance is constantly evolving~\cite{jobin2019global, mittelstadt2016ethics}. At the time of writing, the European Union (EU) has endorsed new transparency and risk-management rules for AI systems known as the EU AI Act~\cite{eu_ai_act}, which is expected to become law in 2023. Similarly, the United States (US) has recently passed a blueprint of the AI Bill of Rights in late 2022~\cite{us_ai_bill}. This bill comprises \emph{``five principles and associated practices to help guide the design, use, and deployment of automated systems to protect the rights of the American public in the age of AI.''} While both the EU and US share a conceptual alignment on key principles of responsible AI, such as fairness and explainability, as well as the importance of international standards (e.g., ISO), the specific AI risk management regimes they are developing are potentially diverging, creating an ``artificial divide''~\cite{ecfr}. The EU aims to become the leading regulator for AI globally, while the US takes the view that excessive regulation may impede innovation.

Notable predecessors to AI regulations include the EU GDPR law on data protection and privacy~\cite{eu_gdpr}, the US Anti-discrimination Act~\cite{us_anti_discrimination}, and the UK Equality Act 2010~\cite{uk_equality}. GDPR's Article 25 mandates that data controllers must implement appropriate technical and organizational measures during the design and implementation stages of data processing to safeguard the rights of data subjects. The Anti-discrimination Act prohibits employment decisions based on an individual's race, color, religion, sex (including gender identity, sexual orientation, and pregnancy), national origin, age (40 or older), disability, or genetic information. This legislation ensures fairness in AI-assisted hiring systems. Similarly, the UK Equality Act provides legal protection against discrimination in the workplace and wider society.

The National Institute of Standards and Technology (NIST), a renowned organization for developing frameworks and standards, recently published an AI risk management framework~\cite{nist2023aiRisk}. According to the NIST framework, an AI system is defined as \emph{``an engineered or machine-based system capable of generating outputs such as predictions, recommendations, or decisions that influence real or virtual environments, based on a given set of objectives. These systems are designed to operate with varying levels of autonomy.''} Similarly, the Principled Artificial Intelligence white paper from the Berkman Klein Center~\cite{fjeld2020principled} highlights eight key thematic trends that represent a growing consensus on responsible AI. These themes include privacy, accountability, safety and security, transparency and explainability, fairness and non-discrimination, human control of technology, professional responsibility, and the promotion of human values.

As AI regulation and governance continue to evolve, AI practitioners are faced with the challenge of staying updated with the changing guidelines and regulations, requiring significant time and effort. Therefore, the focus of this work is to develop an adaptable methodology for generating responsible AI guidelines.

\subsection{Responsible AI Practices and Toolkits}
\subsubsection{Responsible AI Practices.} 
\label{sec:sub-raipractices}
A growing body of research, typically discussed in conferences with a long-standing commitment to human-centered design, such as the Conference on Human Factors in Computing Systems (CHI) and the Conference on Computer-Supported Cooperative Work and Social Computing (CSCW), as well as in newer conferences like the Conference on AI, Ethics, and Society (AIES) and the Conference on Fairness, Accountability, and Transparency (FAccT), focuses on the work practices of AI practitioners in addressing responsible AI issues. This strand of research encompasses various aspects of responsible AI, including fairness, explainability, sustainability, and best practices for data and model documentation and evaluation.

Fairness is a fundamental value in responsible AI, but its definition is complex and multifaceted~\cite{narayanan21fairness}. To assess bias in classification outputs, various research efforts have introduced quantitative metrics such as disparate impact and equalized odds, as discussed by Dixon et al.~\cite{dixon2018measuring}. Another concept explored in the literature is ``equality of opportunity,'' advocated by Hardt et al.~\cite{hardt2016equality}, which ensures that predictive models are equally accurate across different groups defined by protected attributes like race or gender.

Explainable AI (XAI) is another aspect of responsible AI. XAI involves tools and frameworks that assist end users and stakeholders in understanding and interpreting predictions made by machine learning models~\cite{arrieta2020explainable, kulesza2015principles, gunning2019xai,liao2021human, ehsan2020human, ibm2019fairness}. Furthermore, the environmental impact of training AI models should also be considered. Numerous reports have highlighted the significant carbon footprint associated with deep learning and large language models~\cite{sharir2020cost, hao2019training, strubell2019energy}. 

Best practices for data documentation and model evaluation have also been developed to promote fairness in AI systems. Gebru et al.~\cite{gebru2021datasheets} proposed ``Datasheets for Datasets'' as a comprehensive means of providing information about a dataset, including data provenance, key characteristics, relevant regulations, test results, and potential biases. Similarly, Bender et al.\cite{bender2018data} introduced ``data statements'' as qualitative summaries that offer crucial context about a dataset's population, aiding in identifying biases and understanding generalizability. For model evaluation, Mitchell et al.~\cite{mitchell2019model} suggested the use of model cards, which provide standardized information about machine learning models, including their intended use, performance metrics, potential biases, and data limitations. Transparent reporting practices, such as the TRIPOD statement by Collins et al.~\cite{collins2015transparent} in the medical domain, emphasize standardized and comprehensive reporting to enhance credibility and reproducibility of AI prediction models.\\

\subsubsection{Responsible AI Toolkits} Translating these practices into practical responsible AI is another area of growing research. New tools and frameworks are being proposed to assist developers in mitigating biases~\cite{bird2020fairlearn, gebru2021datasheets}, explaining algorithmic decisions~\cite{arya2019one}, and ensuring privacy-preserving AI systems~\cite{fjeld2020principled}.

Fairness auditing tools typically offer a set of metrics to test for potential biases, and algorithms to mitigate biases that may arise in AI models~\cite{saleiro2018aequitas, baeza2018bias}. For instance, Google's fairness-indicators toolkit~\cite{google2022fairness} enables developers to evaluate the distribution of datasets, performance of models across user-defined groups, and delve into individual slices to identify root causes and areas for improvement. IBM's AI Fairness 360~\cite{ibm2022ai} implements metrics for comparing subgroups of datasets (e.g., differential fairness and bias amplification~\cite{foulds2020intersectional}) and algorithms for mitigating biases (e.g., learning fair representations~\cite{zemel2013learning}, adversarial debiasing~\cite{zhang2018mitigating}). Microsoft's Fairlearn provides metrics to assess the negative impact on specific groups by a model and compare multiple models in terms of fairness and accuracy metrics. It also offers algorithms to mitigate unfairness across various AI tasks and definitions of fairness~\cite{fairlearn2022}.

Explainable AI systems are typically achieved through interpretable models or model-agnostic methods. Interpretable models employ simpler models like linear or logistic regression to explain the outputs of black-box models. On the other hand, model-agnostic methods (e.g., LIME~\cite{ribeiro2016should} or SHAP~\cite{lundberg2017unified}) have shown effectiveness with any model. IBM's AI Explainability 360 provides metrics that serve as quantitative proxies for the quality of explanations and offers guidance to developers and practitioners on ensuring AI explainability~\cite{ibm2022ai}. Another research direction introduced new genres of AI-related visualizations for explainability, drawing inspiration from domains such as visual storytelling, uncertainty visualizations, and visual analytics. Examples include Google's explorables, which are interactive visual explanations of the internal workings of AI techniques~\cite{google2022pair}; model and data cards that support model transparency and accountability (e.g., NVIDIA's Model Card++)\cite{nvidia2022}; computational notebook additions for data validations like AIF360\cite{ibm2022ai}, Fairlearn~\cite{fairlearn2022}, and Aequitas~\cite{saleiro2018aequitas}; and data exploration dashboards such as Google's Know Your Data~\cite{google2022know} and Microsoft's Responsible AI dashboard~\cite{microsoft2022aiLab}.

Ensuring privacy-preserving AI systems is commonly attributed to the practice of ``Privacy by Design''~\cite{cavoukian2009privacy, cavoukian2010privacy}, which involves integrating data privacy considerations throughout the AI lifecycle, particularly during the design stage to ensure compliance with laws, regulations, and standards~\cite{fjeld2020principled} such as the European General Data Protection Regulation (GDPR)~\cite{eu_gdpr}. IBM's AI Privacy 360 is an example of a toolkit that assesses privacy risks and helps mitigate potential privacy concerns. It includes modules for data anonymization (training a model on anonymized data) and data minimization (collecting only relevant and necessary data for model training) to evaluate privacy risks and ensure compliance with privacy regulations.

While many toolkits and frameworks emphasize the importance of involving stakeholders from diverse roles and backgrounds, they often lack sufficient support for collaborative action. Wong et al.~\cite{wong2023seeing} have also highlighted the ``mismatch between the promise of toolkits and their current design'' in terms of supporting collaboration. Collaboration is key to enhance creativity by allowing AI practitioners to share knowledge with other stakeholders. To address this gap, we aim to develop a set of actionable guidelines that will facilitate the engagement of a diverse range of stakeholders in AI ethics. By doing so, we hope to take a significant step forward in fostering collaboration and inclusivity within the field.
\section{Preliminary and Problem Definition}

NeRF uses a differentiable model of volume rendering to represent a scene as a volumetric field. 
It can be built upon multilayer perceptrons (MLPs), neural graphic primitive (NGP) or voxel grid, \etc, to encode the scene, which can be represented as a 5D function that takes a 3D location ${\rm x}=(x, y, z)$ and 2D viewing direction ${\rm d} = (\theta, \Phi)$ as inputs:
\begin{equation}
    \sigma, {\rm \bf c} = F({\rm d}, {\rm x}).
\end{equation} 

% \noindent 
Given a set of images $\{{I}_i\}$ and corresponding camera poses $\{P_i\}$, NeRF casts each pixel from $I_i$ as a ray defined by the camera intrinsics, and sample particles to describe how much it blocks or emits lights along the ray. The color $\widehat{\mathcal{C}}({\rm{\bf r}})$ and depth $\widehat{\mathcal{D}}({\rm{\bf r}})$ of a ray $\rm \bf r$ can be approximated by integrating the sampled particles along the ray as follows,
\begin{equation}
    \widehat{\mathcal{C}}({\rm{\bf r}}) = \sum_{i=1}^N T_i (1-{\rm exp}(-\sigma_i\delta_i)) \rm {\bf c}_i ,
    \label{nerf:color}
\end{equation}
\begin{equation}
    \widehat{\mathcal{D}}({\rm{\bf r}}) = \sum_{i=1}^N T_i (1-{\rm exp}(-\sigma_i\delta_i)) \sum_{j=1}^i\delta_j,
    \label{nerf:depth}
\end{equation}
\noindent where $T_i\!=\!{\rm exp}(-\sum_{j=1}^{i-1}\sigma_i\delta_i)$ denotes the accumulated transmittance along the ray from the first sample to $i$-th sample, which is short for transmittance. $(1-{\rm exp}(-\sigma_i \delta_i))$ denotes the alpha value of the current sample contributed to the rendered color and depth, and $\sigma_i$ is the density of sample $i$, $c_i$ is the predicted color of sample $i$, and $\delta_i$ is the distance from sample $i$ to its next sample $i+1$. We denote the probability of ray termination as $h_i = T_i (1-{\rm exp}(-\sigma_i\delta_i))$. To supervise the training of $F$, an L2 photometric reconstruction loss is used:
\begin{equation}
    \mathcal{L}_{rgb}= \sum_i {\mathop{\mathbb{E}}_{{\rm \bf r}\in I_i} {|| \widehat{\mathcal{C}}({\rm{\bf r}}) - \mathcal{C}^{gt}_i({\rm{\bf r}}) ||_2^2}},
\end{equation}
\noindent where $\mathcal{C}^{gt}_i({\rm{\bf r}})$ is the ground truth color of $\rm \bf r$ from image $I_i$.
%This paper targets on designing a large-scale novel view synthesis model that supports faster rendering and training. To this end, we build our model on top of Plenoxels~\cite{yu2021plenoxels}, which is a view-dependent plenoptic volume rendering method that encodes sparse voxel grid with density and spherical harmonic coefficients. The essential difference between it and conventional NeRF models is that Plenoxels explicitly stores data for rendering on voxel grids instead of MLPs. Sampling along a ray becomes marching along rays on plenoptic voxel grids, and $\sigma_i$ and spherical harmonic coefficients can be trilinearly interpolated by its nearest eight voxel grids. The view-dependence of $c_i$ can be simulated by the sum of harmonic basis functions for each color channel.

% To achieve the goal of synthesizing 
%To synthesize novel views of scalable street scenes, we need to optimize a very large voxel grid using a large number of posed images. However, it is technically difficult to load the whole grid into GPU due to its memory limitation, unless we adopt a very coarse resolution to build the grid. In our approach, we first partition the target space into a set of smaller spherical grids that balance the quality of NVS and memory footprint. Then we train each spherical grid following the protocol of optimizing sparse strucutre encoded with spherical harmonic coefficients, and propose novel training strategies on street scenes, such as LiDAR initialization, spherical grid fusion and depth supervision on sparse grids. 

% An effective AD simulator is expected to rapidly simulate sensor outputs given inputs different from the original ones in log files. Since this paper aims to tackle the problem of rapidly simulating environment using visual log data and map information, the input data is limited compared to the one used in other work such as~\cite{tancik2022block}.
\vspace{4pt} \noindent \textbf{Problem Definition.} In our problem of using the neural radiance field $F$ for driving view synthesis, we are given $\{{I}_i\}$ and $\{P_i\}$, along with a semantic map $\mathcal{M}$. We define view interpolation as the use of a continuous function $\mathcal{F}_{in}$ to interpolate the orientations and translations of $\{P_i\}$, and synthesize novel views at any point on $\mathcal{F}_{in}$. Similarly, view extrapolation is defined as synthesizing a novel view at a point outside of $\mathcal{F}_{in}$, but near its nearest point on $\mathcal{F}_{in}$, typically to simulate a lane change in driving scenarios. In this setting, most existing NeRF methods exhibit decent performance on view interpolation but fail on view extrapolation.
To be more specific, we constrain $\mathcal{M}$ only to contain ground height and lane vectors, which are the fundamental entities that exist in most map formats. Our goal is to improve the quality of view extrapolation by incorporating $\mathcal{M}$ into the training of $F$ while ensuring that the performance of view interpolation is still maintained.


\section{Method}
% This section first introduces two supervision methods, and then elaborates how to incorporate map priors by using them to supervise the neural radiance field.
In this section, we first introduce two supervision methods, \ie, ground density and multi-view consistency supervision. Then, we elaborate on how to incorporate map priors into them. After that, an uncertainty term with an uncertainty tempering strategy is explicitly modeled to weakly supervise the training and avoid introducing errors from the semantic-level map that might harm the neural radiance field.

% \subsection{Overview}
% In AD simulation and test, map information usually contains high-level information such as ground, lane information, etc., to provide a base for planning and control. For example, Argoverse2~\cite{wilson2023argoverse} provides HD maps consist of lane graph, driveable area, ground surface height and area of local maps. Our method incorporates those high-level but not pixelwise

\subsection{Ground Density Supervision}
\label{sec:ground_density}

The coarse geometry of HD map, in the format of the ground height field, motivates us to take it as an uncertain signal to guide the reconstruction process of the density field. We first analyze the formulation of ray termination (\ie \equref{nerf:depth}), which is a continuous probability distribution over the sampling region. As shown in~\figref{figRayDist}, we partition the ray termination distribution into three regions:
\begin{itemize}
    \item \textbf{Uncertain region}, where we are not confident if the ray should terminate in this region or not, due to the possible existence of obstacles or constructions on the road ground, which are not encoded in the map;
    \item \textbf{Ground region}, where the ground surface exists within an error tolerance (a typical average error is around $30cm$);
    \item  \textbf{Certain region}, or called as unconcern region, where the existence of any object beneath the ground is not related to our task of view synthesis.
\end{itemize}

Because a ray cast from images might terminate at the surface of obstacles or constructions on the road surface, we must not apply any supervision on the uncertain region. Therefore, we define the ideal distribution of ray termination as a multi-modal distribution instead of the unimodal distribution used in~\cite{deng2022depth}. Given the ground height map of $\mathcal{M}$, we first generate a dense triangle mesh $\widehat{\mathcal{M}}$ using Delaunay triangulation. Then, we use the camera intrinsic and extrinsic parameters to render $\widehat{\mathcal{M}}$ to the dense depth map $\mathcal{D}^{pgt}$ for each camera pose. Note that the term `\textit{pgt}' means pseudo ground truth. We use the KL divergence~\cite{deng2022depth} and the distant line-of-sight priors~\cite{rematas2022urban} to formulate our ground density supervision function:
\begin{equation}
\begin{aligned}
    \mathcal{L}_{gd} = &\mathop{\mathbb{E}}_{{\rm \bf r}\in I_i} \int_{\mathcal{D}^{pgt}_{ij}-\epsilon} \log h(t) \exp\left(-\frac{(t - \mathcal{D}^{pgt}_{ij})^2}{2\epsilon^2}\right) dt \\
    & + {\mathop{\mathbb{E}}_{{\rm \bf r}\in I_i} \int_{{\mathcal{D}^{pgt}_{ij}+\epsilon}} h(t)^2 dt},
    \label{eq:ground}
\end{aligned}
\end{equation}
where $\epsilon$ is the uncertainty that measures the upper bound of error between the map and the ground truth data. This integral can be easily approximated using a discrete set of samples.
To improve the smoothness of the learned depth images, we adopt the depth smooth regularization by sampling $2\times2$ patches rather than pixels to compute the gradients. 


% Figure environment removed


\subsection{Multi-view Consistency Supervision}
The ground density supervision enables the model to improve further the multi-view semantic consistency based on ray terminations. The key insight is ``\textit{a lane on the road should be a lane no matter where we look at it}''. We use a similar random ray casting method in~\cite{zhang2022ray} to generate rays from unseen views. However, the online pseudo labels generated by~\cite{zhang2022ray} are not suitable for our problem: the ray termination is not precise enough to be directly used for generating pseudo labels, though it has been explicitly supervised with uncertainty. In contrast, we define an uncertainty function to measure the confidence that a ray is terminated on the surface of the ground as:
\begin{equation}
    \Gamma({\rm \bf r_i}, \mathcal{D}^{pgt}_i) = \exp\left(\frac{-||\widehat{\mathcal{D}}({\rm \bf r_i}) - \mathcal{D}^{pgt}_i||_1}{2\epsilon}\right),
\end{equation}
\noindent where $\epsilon$ is the same uncertainty used in~\secref{sec:ground_density}. To stabilize the training process, we use ground depth $\mathcal{D}^{gt}_i$ provided by map priors to generate pseudo labels and use $\Gamma({\rm \bf r_i}, \mathcal{D}_i)$ to weight the gradients of rays. Concretely, for a ray cast ${\rm \bf r_i} = {\rm \bf o_i} + t \cdot {\rm \bf d_i}$ from $I$ and a sampled position ${\rm \bf o'_i} = {\rm \bf o_i} + \delta$, we use $\mathcal{D}^{pgt}_i$ to obtain a 3D point $p_i$ in global coordinate, and connect $p_i$ to $\rm \bf o'$ to obtain a sampled ray ${\rm \bf r'_i} = {\rm \bf o'_i} + t\cdot(({p_i-\rm \bf o'_i})/||{p_i-\rm \bf o'_i}||_2)$, where $\delta$ is a randomly sampled 3D vector with a norm of $0.1$ in our experiments. Optionally, we can use DPT~\cite{ranftl2021vision} to filter occluded regions on $\mathcal{D}^{pgt}_i$ out using least square fitting similar to~\cite{yu2022monosdf}. As a result, the ground truth of color $\mathcal{C}^{gt}_i$ can be used for supervising $\rm \bf r'_i$ with a weighting term $\Gamma({\rm \bf r_i}, \mathcal{D}^{pgt}_i)$. Additionally, if ${\rm \bf r_i}$ is passing through a pixel that is near rendered vector lanes, it can be enlarged according to the minimal distance. The loss of multi-view consistency can be written as:
\begin{equation}
    \mathcal{L}_{v} = \sum_i {\mathop{\mathbb{E}}_{{\rm \bf r'}\in I_i} { \Gamma({\rm \bf r_i}, \mathcal{D}^{pgt}_i)  || \widehat{\mathcal{C}}({\rm{\bf r'}}) - \mathcal{C}^{gt}_i({\rm{\bf r}})  ||_2^2}}.
    \label{eq:view}
\end{equation}


\subsection{Uncertainty Tempering}
The goal of our method is to use map priors to guide the training of neural radiance fields, formulating a weak supervision paradigm. The term $\epsilon$ used in the definitions of $\mathcal{L}_{gd}$ and $\mathcal{L}_{v}$ describes the uncertainty that we leave the radiance fields to explore, mainly using $\mathcal{L}_{rgb}$. In this section, we propose a strategy to enlarge uncertainty $\epsilon$ gradually, opposite to the simulated annealing of the learning rate frequently used in deep learning. This strategy provides more freedom for the radiance field as the training proceeds. We use the exponential tempering strategy to define the update equation for uncertainty tempering as follows.
\begin{equation}
    \epsilon' = \gamma \epsilon,
    \label{eq:ut}
\end{equation}
\noindent where $\epsilon'$ is the updated uncertainty, and $\gamma$ is the exponential growth rate (opposite to the decay rate), we set $\gamma =1.0005$ in our experiments.
The overall loss for training our semantic-consistency neural radiance field is:
\begin{equation}
    \mathcal{L} = \mathcal{L}_{rgb} + \lambda_d \mathcal{L}_{gd} + \lambda_v \mathcal{L}_{v}.
\end{equation}
where $\lambda$ is used to balance each loss, $\lambda_d = 0.2$ and $\lambda_v = 0.5$.

%
\label{sec:Experiments}

We now explore the implications of the best-$k$ and good-$k$ problems through Monte Carlo experiments for both the algorithmic $h_a$ and human-like $h_h$ screeners.
Moreover, this section shows how our framework can explore different screening scenarios involving the initial screening order (ISO).
The algorithms~\ref{algo:Examination}--\ref{algo:HumanCascade} and simulation procedures are developed in R \cite{Rlang}.\footnote{See the GitHub repository for the code; \url{https://github.com/cc-jalvarez/initial-screening-order-problem/tree/main}.}
%
% The source code is provided in an anonymous repository: \url{https://anonymous.4open.science/r/initial-screening-order-problem-D078}.

\subsection{Experimental Setup}
\label{sec:Experiments.Setup}

% We consider one possible setup; naturally, these inputs can be changed to account for other screening settings.

\subsubsection{Generating the sample.}
We assume a sample consisting of $n$ triplets $\{ (\theta({c_i}), s(\mathbf{X}_{c_i}), W_{c_i}) \}_{i=1}^n$ 
% for $i = 1, \ldots, n$, 
drawn from a probability distribution with domain 
$\mathcal{G}_n \times \mathbb{R}^n \times \{0, 1\}^n$ where $\mathcal{G}_n$ is the set of all permutations of $\{1, \ldots, n\}$. 
Each sample represents a specific candidate pool $\candidatesset$ sorted according to an ISO $\theta$.
 
Regarding $s(\mathbf{X}_{c_i})$, we consider three possible distributions of candidate scores. 
All are based on the truncated normal distribution family $tN(\mu, \sigma)$ \cite{Botev2017} with values bounded in the interval $[0, 1]$.
Here, we wish to model scenarios in which very good candidates, as in those with top scores, occur with different probabilities.
All three scenarios are shown in Figure~\ref{fig:1} (left).
% %
% \begin{itemize}
    % \item 
    \textit{A symmetric distribution of scores} (in red) defined by $tN(0.5, 0.02)$ with mean/median of $0.5$ implies that top candidates occur with a very low probability. 
    % Hence, setting a large minimum basic requirement $\psi$ is highly selective of top candidates.
    %
    % \item 
    \textit{An asymmetric distribution of scores} (in blue) defined by $tN(0.8, 0.05)$ implies that top candidates occur with a higher probability and median value ($\approx 0.75$) compared to the previous scenario. 
    % Hence, setting a large $\psi$ is less selective of top candidates.
    %
    % \item 
    \textit{An increasing distribution of scores} (in green) defined by $tN(1, 0.05)$ implies that top candidates occur with an even higher probability and median value ($\approx 0.85$). 
    % In this scenario, there are many good candidates, and thus $\psi$ is not selective.
% \end{itemize}
% %

The score scenarios have implications, in particular, for the good-k problem where we must set the minimum score $\psi$ and the screener is not required to explore all of $\candidatesset$ under $\theta$ (recall Remark~\ref{remark:ISOandPS}).
Setting a large $\psi$ makes the screening process highly selective in the first setting, less selective in the second setting, and not selective at all in the third setting.
These settings represent a range of candidate pools with different candidate quality on average.
% For instance, in the ideal setting of having to choose from mostly top candidates (the third setting), the screener is not worried of choosing a high $\psi$.

Regarding $\theta$, we consider two possible variants for the ISO.
%
First: \textit{the ISO is randomly and independently generated from the candidate scores.} 
This setting models the case in which $\theta$ brings no information regarding candidate quality, and the screener prefers the alphabetical order or performs a random shuffle of candidates. 
We denote such a scenario as $\theta \ci s$.
%
Second: \textit{The ISO is randomly generated generated with a given Spearman's rank correlation $\rho$ with the candidate scores.}
Formally, $\rho$ is the Spearman's rank correlation of the pairs $\{ (\theta(c_i), s(\mathbf{X}_{c_i})) \}_{i=1}^k$. 
% for $i=1, \ldots, n$. 
It assesses how well $\theta$ monotonically relates to the scores of candidates.\footnote{To generate correlated initial ordering and scores, we rely on copulas -- see e.g., \cite[Section 3.4]{EMBRECHTS2003329}.}
We denote such scenario as $\theta \not\!\perp\!\!\!\perp s$.
In particular, for $\rho=-1$, it means that $\theta$ ranks candidates by descending scores similar to giving a ranked list of candidates to the screener.
%
With these two variants, we can consider a $\theta$ chosen by or provided to the screener, respectively.
% %
% \begin{itemize}
%     \item  $\theta$ is randomly generated independently from the scores. This setting models the case where the screener prefers the alphabetical order, or performs a random shuffle of candidates. We denote such a scenario as $\theta \ci s$.
% %
%     \item $\theta$ is randomly generated with a given Spearman's rank correlation $\rho$ with the scores.\footnote{Formally, $\rho$ is the Spearman's rank correlation of the pairs $(\theta(c_i), s(\mathbf{X}_{c_i})$ for $i=1, \ldots, n$. It assesses how well the initial order $\theta$ monotonically relates to the scores of candidates. To generate correlated initial ordering and scores, we rely on copulas -- see e.g., \cite[Section 3.4]{EMBRECHTS2003329}.} 
%     In particular, for $\rho=-1$, it means that $\theta$ ranks candidates by descending scores. %Correlation close to $-1$ models scenarios where the initial order is the result of some pre-evaluation of the candidates, including automatic ranking by ML models or manual ordering based on job-required skills. 
% \end{itemize}
% %

Regarding $W$, we initially consider 
% one variant for the protected attribute where 
the sample of candidates drawn from $Ber(\mathit{pr})$ such that $\mathit{pr}=0.2$ is the fraction of protected group candidates in the population.
The sample is independently drawn from both the scores and the ISO, according to the assumptions \textit{A1} and \textit{A2} from Section~\ref{sec:PositionBias}.
Here, we are determining the diversity of $\candidatesset$.
Later on, we increase $\mathit{pr}$ to study a more diverse $\candidatesset$ and its effect on the screener reaching the representational quota $q$.
% By setting a low $\mathit{pr}$, we picture a $\candidatesset$ where the protected individuals are under-presented, forcing the screener to search it longer to meet the representational quota $q$.
%For the former, it means that scores are not unfairly assigned by the screener. 
%For the the latter, it means that the initial order provided by the screener or by a ML ranking model is fair. Clearly, these two cases are also interesting to experiment with, but this is out of the scope of the paper.

Beyond the triplet $\big( \theta({c_i}), s(\mathbf{X}_{c_i}), W_{c_i} \big)$, 
% Finally, 
for the fatigued scores of $h_h$, we fix $\lambda=1$, hence $\Phi(t) = t$, and define:
% %
% \begin{itemize}
    % \item 
    $\epsilon_1 \sim \mathcal{N}(0, \, (0.005 \cdot (t-1))^2)$, hence with constant expectation and with standard deviation of $0.005 \cdot (t-1)$; and
    %
    % \item 
    $\epsilon_2 \sim \mathcal{N}(-0.005 \cdot (t-1), \, (0.001 \cdot (t-1))^2)$, hence with a decreasing expectation and with a smaller standard deviation than $\epsilon_1$.
% \end{itemize}
% %
Recall Section~\ref{sec:HumanScreener.BiasedScores} for details.

\subsubsection{Evaluation metrics.}
% For concreteness, 
We consider the solution $S^k_{\besttext}$ of the best-$k$ problem (\ref{eq:fair_objective_all_screener}) for $h_a$ (Algorithm~\ref{algo:Examination}) as the baseline solution. 
We compare it with the analysis of the solutions for the good-$k$ problem (\ref{eq:fair_objective_U_psi}) under $h_a$ (Algorithm~\ref{algo:Cascade}) and of the solutions for the best-$k$ and good-$k$ problems under $h_h$ (respectively, Algorithms \ref{algo:HumanExamination} and \ref{algo:HumanCascade}).
We introduce two comparison metrics that capture how close is the compared solution to the baseline solution.
   
The \textbf{\textit{ratio to baseline (RtB)}} is defined as the ratio of $U^k_{\addtext}$ between the compared solution and the baseline solution. 
For the solution $S_{\goodtext}^k$ under $h_a$, e.g., it is $U^k_{\addtext}(S_{\goodtext}^k) / U^k_{\addtext}(S^k_{\besttext})$. 
The closer the ratio is to $1$, the better the compared solution approximates the best-$k$ solution under $h_a$ in terms of $U^k_{\addtext}$ utility.
Here, when calculating the utility of $h_h$, we use the truthful scores, not the fatigued scores, to be able to compare w.r.t. the baseline.
%
The \textbf{\textit{Jaccard similarity (JdS)}} is defined as the proportion of candidates in both the compared and baseline solutions over those in at least one of the two. 
For the solution $S_{\goodtext}^k$ under $h_a$, e.g., it is $|S_{\goodtext}^k \cap S^k_{\besttext}|\, / \, |S_{\goodtext}^k \cup S^k_{\besttext}|$. 
Such a metric quantifies the share of candidates between the two solutions.
Essentially, the RtB metric captures whether the compared solution achieves the same utility as the baseline solution as measured by $U^k_{\addtext}$, while the JdS metric captures the overlap in candidates between the compared solution and the baseline solution.

\subsubsection{Simulations.}
For each setting of the parameters ($n$, $k$, $q$, $\rho$, $\psi$ or others), we run 10,000 times the experiments by randomly generating $n$ triplets at each run. 
The runs for which a solution of the problem does not exist are discarded. This
mainly occurs in the good-$k$ problem when there are not enough $k$ candidates with scores greater or equal than $\psi$.
% \footnote{This mainly occurs in the best-$k$ problem when there are not enough candidates with scores greater or equal than $\psi$.}
The plots report the mean output based on the evaluation metrics over all the runs.

\subsection{Experiments without Fatigue}
\label{sec:Experiments.Metrics.outFatigue}

%
% Figure environment removed
%
% %
% % Figure environment removed
% %\vspace{-3ex}
% %

We start by exploring the settings without fatigue, meaning we consider $h_a$, which allows for clarifying the relation between the best-$k$ (Algorithm~\ref{algo:Examination}) and good-$k$ (Algorithm~\ref{algo:Cascade}) solutions.
Given these two problem formulations, here we are mainly interested if their solutions differ in practice due to the ISO, especially since the best-$k$ requires a full search of $\candidatesset$ while the good-$k$ allows for a partial search of $\candidatesset$.

% First, w
We study the impact of the score distributions on the metrics at the variation of $\psi$.
We consider the case of $n=120$, $k=6$, $q=0.5$, and $\theta \ci s$. 
Note that these parameters are shared by both best-$k$ and good-$k$ problems. 
Instead, $\psi$ is specific to the good-$k$, which is why we focus on it.
We find that, as $\psi$ increases and screening becomes more selective, the good-$k$ approximates the best-$k$ when there is a low probability of having good candidates in $\candidatesset$.
%
Figure~\ref{fig:1}, under the RtB (center) and the JdS (right) metrics, illustrates this point for the three score distribution scenarios.
In particular, the symmetric distribution (in red) allows the good-$k$ to better approximate the best-$k$ for medium-to-high $\psi$ values.
This result is expected. 
Having few good candidates forces $h_a$ to explore more of $\candidatesset$ under $\theta$, especially as $\psi$ increases and the $k$ first good-enough candidates essentially become the $k$ top candidates.

The opposite holds for the other two distributions (in green and blue), which are more resilient to $\psi$ as each represents a higher concentration of good candidates.
It follows that having many good candidates makes it difficult for $h_a$ to select the $k$ top candidates under a partial search.
As the RtB and JdS metrics show in Figure~\ref{fig:1} (center, right), $h_a$ still achieves significant utilities under the other two distributions but is unlikely to derive the same selected set of candidates under a partial search w.r.t. a full search of $\candidatesset$.
These results are also expected, but worth emphasizing. 
Having many good candidates means that $h_a$ can still partially search $\candidatesset$ despite having a highly selective $\psi$: i.e., $h_a$ finds $k$ good-enough candidates with high-enough scores as the $k$ top candidates but not the same ones.

Figure~\ref{fig:1} illustrates how the two problems materialize differently when implemented due to the ISO $\theta$.
Clearly, as noted back in Remark~\ref{remark:ISOandPS}, where the $k$ top candidates appear in $\theta$ can determine if they are selected or not by $h_a$ under a partial search.
The position bias in the ISO becomes more prevalent under many good candidates as, e.g., even the $\candidatesset$'s best candidate may never be selected by $h_a$ under a partial search if it lies at the bottom of $\theta$.

We also study what occurs when $\theta \not\!\perp\!\!\!\perp s$. 
We present these results in Appendix~\ref{Appendix:Experiments.Metrics.outFatigue} for  $n=120$, $k=6$, and $q=0.5$.
Here, we briefly discuss these results as they further illustrate the role of $\theta$.
Recall that $\rho$ represents the correlation between $\theta$ and the scores, with a negative $\rho$ implying a descending order.
We find that the good-$k$ solution approximates quite well the best-$k$ solution already for $\rho=-0.5$; for $\rho=-1$, the two solutions are the same.
These results are expected as $\theta$ essentially represents the best-$k$ solution or an approximation of it depending on $\rho$'s strength. 
Under $\rho=-1$, e.g., the $\theta$ searched by $h_a$ is already sorted by the candidate scores and, in turn, the $k$ first good-enough candidates are also the $k$ best candidates in the candidate pool. 
See Figure~\ref{fig:2} (center, right) for details.

In Appendix~\ref{Appendix:Experiments.Metrics.outFatigue} we also study the impact of the number $n$ candidates in $\candidatesset$ and the number of $k$ candidates to be selected.
We find that under $\theta \ci s$, the ratio $k/n$ is positively correlated with the ability of best-$k$ to approximate good-$k$. Intuitively and unsurprisingly, it means that the more candidates we can select from $\candidatesset$, the better the chance to include top ones under a partial search.
Clearly, the influence of $\theta$ diminishes as $k/n$ increases.
See Figure~\ref{fig:2} (left) for details.
Similarly, we study the role of changing the representational quota $q$. Here, results are expected given our setup and underlying assumptions (\textit{A1} and \textit{A2} from Section~\ref{sec:PositionBias}), finding that under $\theta \ci s$, $q$ does not affect the relative strengths of best-$k$ and good-$k$ solutions.
See Figure~\ref{fig:3} (all) for details.

%%%
%%% Moved to Appendix
% Second, we consider the impact of the number $n$ candidates in $\candidatesset$ and the number of $k$ candidates to be selected. 
% We focus only on the symmetric distribution and the ratio to baseline, but the results are similar for the other two distributions and the Jaccard similarity. 
% Figure~\ref{fig:2} (left) compares the case $n=120, k=6$ considered earlier to two other scenarios. 
% The first scenario increases $k=20$ but leaves the ratio of selected $k/n = 0.05$ the same by also increasing $n=400$. 
% The second scenario, instead, leaves $k=6$ the same, but it increases $k/n = 0.2$ by decreasing $n=30$. 
% The plot shows that changes in the ratio $k/n$ affect the metric, in particular a larger ratio ($n=30$, $k=6$) leads good-$k$ to better approximate best-$k$ for a same $\psi$. The more candidates we can select from the pool, the better the chance to include top ones. 
% \textit{In summary, under $\theta \ci s$, the ratio $k/n$ is positively correlated with the ability of best-$k$ to approximate good-$k$}.

% Third, we now consider the impact of the correlation $\rho$ between the initial order $\theta$ and the scores. 
% Recall that $\rho = -1$ means that the candidates are ordered by descending scores. Under such a condition, the good-$k$ and best-$k$ procedures return the same solution. 
% This result is apparent in Figure~\ref{fig:2} (center, right) where we report the ratio to baseline for the symmetric (left) and the increasing (right) score distributions. 
% The plots show that even a moderate correlation of $\rho = -0.5$ leads the good-$k$ solution to approximate the best-$k$ one quite well. 
% For the increasing distribution (right plot), the ratio to baseline is around 95\%. \textit{In summary, initial orders that negatively correlate to the score greatly reduce the difference in utility between the good-$k$ and best-$k$ solutions}.

% %
% % Figure environment removed
% %\vspace{-3ex}
% %

% Let us now consider the quota parameter $q$, thus far set to $q=0.5$ over a population with a fraction of protected candidates set to $\mathit{pr}=0.2$. 
% Since we assumed that $W$ is independent from both scores and the ISO, the fraction of protected group in the solutions of best-$k$ and good-$k$ is, on average, $\mathit{min}\{q, \mathit{pr}\}$. 
% Figure~\ref{fig:3} (left) shows this result in the solution for good-$k$. 
% A less trivial question is whether $q$ is also not affecting the evaluation metrics: e.g., whether the quota $q$ changes the ratio to baseline? 
% Figure~\ref{fig:3} (center, right) show that this is not the case in two experimental settings. 
% Again, this result is theoretically implied by the independence of $W$ with scores and initial order. 
% In summary, under $\theta \ci s$, the quota parameter in the best-$k$ and good-$k$ problem does not affect the relative strengths of their solutions.
%%%
%%%

\subsection{Experiments with Fatigue}
\label{sec:Experiments.Metrics.withFatigue}

% We now consider the settings with fatigue, meaning we consider $h_h$.
We now focus on $h_h$.
First, we consider whether fatigue impacts utility w.r.t.~the baseline solution, namely the solution of best-$k$ without fatigue (Algorithm~\ref{algo:Examination}). 
We compare to such a baseline both the best-$k$ with fatigue (Algorithm.~\ref{algo:HumanExamination}) and good-$k$ with fatigue (Algorithm.~\ref{algo:HumanCascade}).
%See Appendix~\ref{Appendix.HumanAlgorithms} for implementation details on both of these algorithms.
%In the latter case, we readily extend the metric of ratio to baseline as the ratio of the utility for the solution of best-$k$ with fatigue over the one of best-$k$ without fatigue. 

%
% Figure environment removed
%
%
% Figure environment removed
% \vspace{-3ex}
%

Figure~\ref{fig:4} (left) shows the RtB metric for the three score distributions for the good-$k$ 
% (i.e., $h_h$ can perform a partial search) 
solution with fatigue based on $\epsilon_1$ for the fatigued scores. 
Based on Figure~\ref{fig:1} (center), for the asymmetric (in blue) and increasing (in green) distributions, 
which are the scenarios with many good candidates in $\candidatesset$,
there is not much difference w.r.t.~the case without fatigue.
For the symmetric distribution (in red), instead, there is a considerable decrease for high $\psi$ values. 
This can be attributed to the low number of top scores, for which the perturbation due the $\epsilon_1$'s has a large effect. 
For the other two distributions, instead, there are sufficiently many top scores, for which perturbation does not dramatically change the score distribution for top scores.
Intuitively, since the RtB metric captures achieving the utility of the baseline model, under a partial search $h_h$ is able to reach high utility solutions when $\candidatesset$ has many good candidates because $h_h$ can avoid evaluating all candidates.
This is unlikely to be the case when few good candidates are in $\candidatesset$. 
As $\psi$ increases and $h_h$ becomes more selective, it also becomes more tired under $\theta$ as it needs to evaluate more and more candidates to achieve $k$.

Figure~\ref{fig:4} (center) is analogous to (left), but considers the fatigued scores based on $\epsilon_2$. 
The effect for the symmetric distribution (in red) is not present in such a case, due to the lower standard deviation of $\epsilon_2$. 
The bias of $\epsilon_2$ does not impact too much, apart from high values of $\psi$ where it causes the problem not to have a solution as fatigued scores are smaller than scores of an already low number of top candidates. 
In summary, under $\theta \ci s$, variance appears more relevant than bias in the case of low probability of top scores.
This result illustrates the importance of how we define fatigue. It can also inform how the a human screener should behave in practice to diminish the role of position bias within $\theta$. Given these results, e.g., we would be interested in exploring under what settings would the human screener experience $\epsilon_1$ over $\epsilon_2$.

Figure~\ref{fig:4} (right) shows the RtB for the best-$k$ solution with fatigue at the variation of the quota $q$. There is a considerable and constant loss in utility under fatigue, which is more consistent for the symmetric distribution (in red). 
Interestingly enough, the RtB is lower than in the case of the good-$k$ with fatigue (see left) for $\psi \geq 0.5$. 
This result means that, for the symmetric distribution, the good-$k$ solution with fatigue has better utility than the best-$k$ solution with fatigue. 
This noteworthy result can inform the screening practice.

Finally, we consider the impact of the correlation $\rho$ on the ISO $\theta$ for the good-$k$ solution with fatigue.  
Figure~\ref{fig:5} (left) considers the symmetric distribution (in red). 
For the lower half of $\psi$'s, lines are similar to the analogous case without fatigue shown in Figure~\ref{fig:2} (center). 
For the higher half of $\psi$'s, instead, there is a decrease in the metric. 
This is, again, due to the low probability of top scores, for which the effects of the variability of $\epsilon_1$ is not counter-balanced by correlation of the scores and $\theta$.
Such an effect does not appear for $\epsilon_2$ nor for $\epsilon_1$ with the increasing distribution. 
In fact, the plots in Figure~\ref{fig:5} (center) and (right) closely resemble those in Figure~\ref{fig:2} (center) and (right) respectively. 
This is an interesting result on its own. 
It means that, in the presence of correlation, fatigue does not have an impact on utility of the good-$k$ solution if there are sufficiently many top scores or a sufficiently small variability of the fatigue.

Moreover, we believe this last result points at the importance in practice of providing a $\theta$ to the human screener that has some information about candidate quality. 
Intuitively, under a partial search procedure and the threat of position bias materializing through $\theta$, we would like to decrease $h_h$'s fatigue by minimizing its need to search more of $\candidatesset$.
A way to do is to already provide to $h_h$ a sorted $\theta$. 
Note that this point excludes the difficulty behind deriving such a sorted $\theta$ in the first place, which is the main goal of the fair set selection literature. 
This point, however, hints at an interesting line of future work focused on human screeners and their interactions with an algorithmic aid.

%
% EOS
%

% \subsection{Evaluation Metrics}
% \label{sec:Experiments.Metrics}

% The solution $S^k_{\besttext}$ of the fair best-$k$ problem (\ref{eq:fair_objective_all_screener}) for the algorithmic screener represents a baseline to compare with in the analysis of the solutions of the fair good-$k$ problem (\ref{eq:fair_objective_U_psi}) for the algorithmic screener, and of the solution of both fair best-$k$ and fair good-$k$ for the human-like screener. 
% We introduce two comparison metrics.
% %
% %\begin{itemize}
%     %\item 
    
%     The \textit{ratio to baseline} is defined as the ratio of $U^k_{\addtext}$-utility between the compared solution and the baseline solution. E.g.,~for the solution $S_{\goodtext}^k$ of the algorithmic screener, it is $U^k_{\addtext}(S_{\goodtext}^k) / U^k_{\addtext}(S^k_{\besttext})$. 
%     The closer the ratio is to $1$, the better the compared solution approximates the best-$k$ solution of the algorithmic screener in terms of $U^k_{\addtext}$ utility. 
    
%     %\item 
    
%     The \textit{Jaccard similarity} is defined as the proportion of candidates in both the compared and baseline solutions over those in at least one of the two. E.g., for the solution $S_{\goodtext}^k$ of the algorithmic screener, it is $|S_{\goodtext}^k \cap S^k_{\besttext}|/|S_{\goodtext}^k \cup S^k_{\besttext}|$. 
%     Such a metric quantifies the share of candidates between the two solutions. 
    %The closer the metric is to $1$, the more equivalent is the set of selected candidates between the problems.
%\end{itemize}
%
% SR
%We will experiment with the values of these two metrics at the variation of $\psi$ and other parameter settings.


%The proposed method is trained on a set of public datasets available in the ultrasound toolbox [10]. The proposed approach has been accepted for presentation during the Challenge on Ultrasound Beamforming with Deep Learning (CUBDL) at the 2020 IEEE International Ultrasonics Symposium (IUS) [11], [12]. synthetic / PICMUS difference

%ALSO DISCUSS ABOUT TYPE OF PRETRAINED VS TRAINED
Regarding the computing time, \yz{our approaches need 3-4 minutes to form one image, which is slower than DAS1, PCF~\cite{PCF} and MNV2~\cite{MNV2}, but faster \ji{than EMV ~\cite{asl_eigenspace-based_2010} and RED~\cite{RED_USIPB}, which need 8 and 20 minutes, respectively.} RED is slow because each iteration contains an inner iteration while \ji{EMV spends time on covariance matrix evaluation and decomposition.} Our iteration restoration approaches require multiple multiplication operations with the singular vector matrix, which currently hinders real-time imaging. }Accelerating this process is one of our key focuses for future work.

\begin{comment}
the computationally expensive SVD for DDRM actually does not affect imaging time since the SVD results can be precomputed, but the multiplication operation with the singular vector matrix during the image reconstruction process currently hinders real-time imaging. On our machine equipped with the GPU NVIDIA Quadro RTX 3000, each iteration takes approximately 4.5 seconds.
\end{comment}

\YZ{In conclusion, for the first time, we achieve the reconstruction of ultrasound images with} 
\DM{ two adapted diffusion models, DRUS and WDRUS. }
\yz{Different from previous model-based deep learning methods which are task-specific and require a large amount of data pairs for supervised training, our approach requires none or just a small fine-tuning dataset composed of high-quality (e.g., DAS101) images only (there is no need for paired data). Furthermore, the fine-tuned diffusion model can be used}
%applied to} 
\dm{for other US related inverse problems.}
%diverse inverse problems, e.g., DRUS and WDRUS, as long as the same prior knowledge.}
\YZ{Finally, our method demonstrated competitive performance compared to DAS75, and other state-of-the-art approaches on the PICMUS dataset.}



\begin{comment}
\YZ{Our approach has demonstrated superior performance compared to DAS, both in terms of visual quality and evaluation metrics, on both synthetic and PICMUS datasets}, \DM{even though we fine-tuned the \texttt{f-number} to fit DDRM while kept the default values from the open-source code for DAS, as 
%While we used slightly different parameters, e.g. \texttt{f-number} for DAS and our methods when testing on the PICMUS dataset, such as using default values from the PICMUS open-source code for DAS while fine-tuning these parameters for our methods to fit DDRM, 
fundamentally, the number of plane waves affects the image quality of DAS.} 
\YZ{Our method is able to compete with 75 plane waves, which is sufficient evidence of its effectiveness. As for the distortion observed in the WDRUS results, it may be due to the amplification of errors in the ultrasound model by the whitening operator, but the specific reason requires further investigation.
}

\YZ{
Our method provides an important insight for the medical imaging field by addressing the challenge of training} \DM{model-based deep learning methods
%neural networks 
}\YZ{when access to datasets is restricted due to privacy concerns. The model we used was trained on ImageNet only, without any ultrasound data.
}
\DM{However, it} 
\YZ{
%It 
should be noted that  %images in the 
ImageNet data %dataset 
are significantly different from ultrasound images. For example, pixel values in natural images are always positive, while} 
%in ultrasound image reconstruction, 
 \DM{the reconstructed ultrasound 
$\xv$ contains both positive and negative values, and 
%ultrasound images 
are typically displayed after log compression.}
\YZ{Therefore, fine-tuning existing models with a small amount of ultrasound data may lead to better results.
}

\YZ{
Although DDRM relies on the computationally expensive SVD, it does not affect imaging time since} 
%its results can be saved and repeatedly used. 
\DM{ the SVD results can be precomputed.}
\YZ{However, the multiplication operation} \DM{ with the singular vector matrix 
%between the singular matrix and other vectors 
}
\YZ{during the image reconstruction process currently hinders real-time imaging. On our machine equipped with the GPU NVIDIA Quadro RTX 3000, each iteration takes approximately 4.5 seconds. Accelerating this process is one of our key focuses for future work.}

\YZ{
In conclusion, for the first time, we achieve the reconstruction of ultrasound images with} 
%a diffusion model and test two ultrasound models, DRUS and WDRUS. 
\DM{ two adapted diffusion models, DRUS and WDRUS. }
\YZ{Our method with single plane wave is even comparable to DAS with 75 plane waves,}
%which are often used to produce target images, in the case where the generative model has never been trained on ultrasound data.
\DM{which is often used as reference to train generative models, whereas our diffusion model was never trained on ultrasound data}
\end{comment}

\footnotesize
\bibliographystyle{IEEEtran} 
% \bibliography{iros}
\bibliography{iros_abrv}


\end{document}
