\documentclass[12pt]{article}
\usepackage{amsmath}
\usepackage{graphicx}
\usepackage{enumerate}
\usepackage{natbib}
\usepackage{array}
%\usepackage{url} % not crucial - just used below for the URL 

%\pdfminorversion=4
% NOTE: To produce blinded version, replace "0" with "1" below.
\newcommand{\blind}{1}
\usepackage{mathtools}
\usepackage{etoolbox}
% DON'T change margins - should be 1 inch all around.
\addtolength{\oddsidemargin}{-.5in}%
\addtolength{\evensidemargin}{-1in}%
\addtolength{\textwidth}{1in}%
\addtolength{\textheight}{1.7in}%
\addtolength{\topmargin}{-1in}%

\usepackage{xcolor}

\DeclareMathOperator*{\argmin}{arg\,min}%\usepackage[latin 1]{inputenc}
\usepackage{caption}\newcommand\independent{\protect\mathpalette{\protect\independenT}{\perp}}
\def\independenT#1#2{\mathrel{\rlap{$#1#2$}\mkern2mu{#1#2}}}\usepackage{graphicx,latexsym,amssymb,amsmath}
\usepackage{subcaption}
\usepackage{bbm, bm}
\def\pp{\vskip 10pt}
\def\ppn{\vskip 10pt \noindent }
\def\d{{\mathrm{d}}}
\def\R{{\mathbb{R}}}
\def\Z{{\mathbb{Z}}}
\def\C{{\mathbb{C}}}
\def\N{{\mathbb{T}}}
\def\T{{\mathbb{T}}}
\def\P{{\mathbb{P}}}
\def\E{{\mathbb{E}}}
\def\indic{\hbox{\it 1\hskip -3.5pt I}} % indicator function
\def\convLaw{{\ \buildrel {\Ls} \over \longrightarrow \ }}
\newcommand{{\convp}}{{\buildrel p\over\longrightarrow}}



\newcommand{\red}[1]{{\color{black} #1}}

\newcommand{{\Vs}}{{\cal V}}
\newcommand{{\Ps}}{{\cal P}}
\newcommand{{\Ss}}{{\cal S}}
\newcommand{{\Xs}}{{\cal X}}
\newcommand{{\Ls}}{{\cal L}}
\newcommand{{\Ns}}{{\cal N}}
\newcommand{{\Zs}}{{\cal Z}}
\newcommand{{\Fs}}{{\cal F}}
\newcommand{\tbc}{{\bf ***to be completed***$\quad$}}
\newcommand{\tbchk}{{\bf ***to be checked!!!***$\quad$}}
\newcommand{\chk}{{\bf ***checked!!!***$\quad$}}
\newcommand{\tbm}{{\bf ***to be modified***$\quad$}}
\newcommand{\cuh}{{\medskip\bf ***correct until here***$\quad$ \medskip}}

\newcommand{\cvp}[1]{\stackrel{P}{#1}}
%\newcommand{\cvl}[1]{\stackrel{{d}}{#1}}
\newcommand{\cvl}[1]{\stackrel{{\Ls}}{#1}}
\newcommand{\cps}[1]{\stackrel{a.s.}{#1}}
\newcommand{\pco}[1]{\stackrel{co.}{#1}}
\allowdisplaybreaks
%%%%%%%%%%%%%%%%%%%%%%%%%%%%%%%%%%%%%%%%%%%%%%%%%%%%%%%%
\newtheorem{Lemma}{Lemma}[section] %numbering by section
\newtheorem{Assumption}{Assumption}
\newtheorem{Theorem}{Theorem}[section] %numbering by section

\newtheorem{Remark}{Remark}[section]
\newtheorem{Corollary}{Corollary}
\newcommand{\proofend}{$\hfill\Box{~}$}
\newenvironment{Proof}{\noindent {\em{\bf Proof.}}}{\proofend\\}
%\usepackage{latexsym}
\newtheorem{definition}{Definition}

\newcommand{\tcr}{\textcolor{red}} % track changes in RED
\newcommand{\bs}{\boldsymbol} 

\newcommand{\tb}{|||} 

\makeatother
\usepackage{xr-hyper} \externaldocument{paper}
\usepackage[colorlinks = true,
linkcolor = red,
urlcolor  = blue,
filecolor={red},
citecolor = blue,
anchorcolor = blue]{hyperref}
\usepackage{newfloat}
\DeclareFloatingEnvironment[
    fileext=los,
    listname={List of Schemes},
    name=Algorithm,
    within=section,
    placement=tbhp,
]{algorithm}
\begin{document}

%\bibliographystyle{natbib}

\def\spacingset#1{\renewcommand{\baselinestretch}%
{#1}\small\normalsize} \spacingset{0}


%%%%%%%%%%%%%%%%%%%%%%%%%%%%%%%%%%%%%%%%%%%%%%%%%%%%%%%%%%%%%%%%%%%%%%%%%%%%%%

\if1\blind
{
  \title{ Online appendix of ``Tuning-free testing of factor regression against factor-augmented sparse alternatives''}
  \author{Jad Beyhum\\\vspace{0.5em}
    CREST, ENSAI, France  \& Department of Economics, KU Leuven, Belgium\\\vspace{0.5em}
    Jonas Striaukas\hspace{.2cm}\\
    Department of Finance, Copenhagen Business School, Denmark.}
  \maketitle
} \fi

\if0\blind
{
  \bigskip
  \bigskip
  \bigskip
  \begin{center}
    {\LARGE\bfPenalty }
\end{center}
  \medskip
} \fi



\spacingset{1.5} 

\renewcommand*{\thesection}{\Alph{section}}



{\hypersetup{linkbordercolor=black,linkcolor = black,
urlcolor  = blue,
filecolor=black}
% or \hypersetup{linkcolor=black}, if the colorlinks=true option of hyperref is used
\tableofcontents
}
\clearpage


\section{Additional simulation results for $T=p=200$}
%In table \ref{tab.n200} we report simulation results under the same data generating processes as in the main article but with $T=p=200$ instead.

\begin{table}[h!]\setlength\extrarowheight{-4pt}

	\begin{center}
		\begin{tabular}{lccccccc}
		            \hline
			\multicolumn{8}{c}{{\it Design 1}: $\rho_f=\rho_{u}=\rho_e=0$}\\
			&\multicolumn{3}{c}{\text{Our test}}&&\multicolumn{3}{c}{\text{FabTest}}\\
			
			$m$&   $\alpha =0.1$ &$\alpha = 0.05$ &$\alpha =0.01$ && $\alpha =0.1$ &$\alpha =0.05$ &$\alpha =0.01$ \\
			\hline 
            0.0 &	0.0900 &	0.0435&	0.0110 &&	0.1010 &	0.0565&	0.0170\\
            0.1	&0.1435&	0.0770 &	0.0300&&	0.1170 &	0.0590 &	0.0140\\
            0.2&	0.6025&	0.4910&	0.2995&&	0.2865	&0.1825&	0.0700\\
            0.3 &	0.9625&	0.9335&	0.8370&&	0.7665	&0.6570	&0.4645\\
            0.4&	0.9995&	0.9990 &	0.9940&&	0.9850	&0.9685	&0.9020\\[0.5cm]
			\multicolumn{8}{c}{{\it Design 2}: $\rho_f=0.6$, $\rho_{u}=0.1$ and $\rho_e=0$}\\
			&\multicolumn{3}{c}{\text{Our test}}&&\multicolumn{3}{c}{\text{FabTest}}\\
			$m$&   $\alpha =0.1$ &$\alpha = 0.05$ &$\alpha =0.01$ && $\alpha =0.1$ &$\alpha =0.05$ &$\alpha =0.01$ \\
			\hline 
            0&	0.0920 &	0.0365	&0.0055	&&0.1050&	0.0515&	0.0120\\
            0.1&	0.1400&	0.0780&	0.0245&&	0.1210&	0.0625&	0.0245\\
            0.2&	0.5965	&0.4835	&0.3080&&	0.2770&	0.1855&	0.0735\\
            0.3&	0.9585&	0.9365	&0.8365&&	0.7680&	0.6580&	0.4555\\
            0.4&	1.0000&	0.9995&	0.9930&&	0.9900&	0.97650&	0.9035\\[0.5cm]

			\multicolumn{8}{c}{{\it Design 3}: $\rho_f=0.6$ and $\rho_{u}=\rho_e=0.1$}\\

			&\multicolumn{3}{c}{\text{Our test}}&&\multicolumn{3}{c}{\text{FabTest}}\\
			
			$m$&   $\alpha =0.1$ &$\alpha = 0.05$ &$\alpha =0.01$ && $\alpha =0.1$ &$\alpha =0.05$ &$\alpha =0.01$ \\
			\hline 
            0&	0.1015	&0.0435	&0.0090&&	0.1120&	0.055&	0.0130\\
            0.1&	0.1520	&0.0860 &	0.0270 &&	0.1290&	0.0675&	0.0290\\
            0.2 &	0.6105&	0.4975&	0.3180&&	0.2940&	0.1990&	0.0820\\
            0.3 &	0.9610 &	0.9375&	0.8460&&	0.7725&	0.6625&	0.4675\\
            0.4 &	1.0000&	0.9990&	0.9920&&	0.9905&	0.9765&	0.9050\\
			\hline\hline 
		\end{tabular}
	\end{center}
	\caption{Rejection probabilities with $T=200$ and $p=200$ for the three designs we consider.}
	\label{tab.n200}
\end{table}

\clearpage
\newpage





\section{On Theorem \ref{th}} This section contains material allowing to prove Theorem \ref{th}. In Section \ref{subsec.prem}, we define some useful mathematical objects. The proof of Theorem \ref{th} is given in Section \ref{subsec.proof} and makes use of results proved in later sections. Section \ref{subsec.distr} contains some auxiliary lemmas on distribution functions of random variables used in the proof of Theorem \ref{th}. Then, in Section \ref{subsec.prob}, we state and prove some lemmas on the probability of some events. Section \ref{subsec.seq}, contains results on some sequences introduced in the proof of Theorem \ref{th}.  Furthermore, Section \ref{subsec.fac} introduces results on the factors, the loadings and their estimators. Finally, Section \ref{subsec.pre} recalls pre-existing results on strong mixing sequences and high-dimensional Gaussian vectors. Our proofs borrow ideas and results from \cite{chernozukhov2013gaussian}, \cite{chernozhukov2015comparison}, \cite{lederer2021estimating}, \cite{fan2023latent} and \cite{fan2021bridging}. 
\subsection{Preliminaries}\label{subsec.prem}
We introduce some concepts which are latter useful in proving Theorem \ref{th}. First, as  \cite{lederer2021estimating}, we re-scale some quantities by multiplying them with $\sqrt{T}/2$. This re-scaling is convenient to apply some probabilistic results. For instance, we let $\widehat{\Pi}(\mu,e)=\left\|\widehat{W}(\mu,e)\right\|_\infty,$
where $$\widehat{W}(\mu,e) =\left(\widehat{W}_1(\mu,e), \dots, \widehat{W}_p(\mu,e)\right)^\top,\ \text{with } \widehat{W}_j(\mu,e)= \frac{1}{\sqrt{T}}\sum_{t=1}^T\widehat{u}_{tj} \widehat{\varepsilon}_{\frac{2}{\sqrt{T}}\mu,t} e_t .$$
Note that $\widehat{\Pi}(\mu,e)=\frac{\sqrt{T}}{2}\widehat{Q}(\lambda,e),$ for $\lambda =\frac{2}{\sqrt{T}}\mu.$
Similarly, for $\alpha\in(0,1)$, we define
\begin{align*}\widehat{\pi}_\alpha(\mu)&=\inf\{q:\P_e(\widehat{\Pi}(\mu,e)\le q)\ge 1-\alpha\};\\
\widehat{\mu}_\alpha &=\inf\{\mu>0\ :\  \widehat{\pi}_\alpha(\mu')\le \mu' \text{ for all } \mu'\ge \mu\},
\end{align*}
where $\widehat{\mu}_\alpha =\frac{\sqrt{T}}{2}\widehat{\lambda}_\alpha$. 

Next, to be able to compare $\widehat{\Pi}(e)$ with population analogs, we define several additional quantities. Let
 $\Pi(e)=\left\|W(e)\right\|_\infty$, where 
$$W(e)=\left(W_1,\dots,W_p(e)\right)^\top,\ \text{with } W_j(e)=\frac{1}{\sqrt{T}}\sum_{t=1}^Tu_{tj} \varepsilon_te_t$$
and let $\mu_\alpha$ be the $(1-\alpha)-$quantile of $\Pi(e)$ conditionally on $(F,U,\mathcal{E})$.  Formally, $\mu_\alpha=\inf\{q:\P^*_e(\Pi(e)\le q)\ge 1-\alpha\}$, where $\P_e^*(\cdot)=\P(\cdot|F,U,\mathcal{E})$.

Moreover, we define $\Pi^*= \left\|W^*\right\|_\infty$, where 
$$W^*=\left(W_1^*,\dots,W_p^*\right)^\top,\ \text{with } W_j^*=\frac{1}{\sqrt{T}}\sum_{t=1}^Tu_{tj} \varepsilon_t,$$
where $\mu_\alpha^*$ is the $(1-\alpha)$ quantile of $\Pi^*$. Finally, we also set $\Pi^G=\left\|G\right\|_\infty$ with $G$ a Gaussian vector with same covariance structure as $W^*$ and let $\mu_{\alpha}^G$ be the $(1-\alpha)$-quantile of $\Pi^G$. Auxiliary lemmas concerning the distributions of $\Pi(e)$, $\Pi^*$ and $\Pi^G$ can be found in Section \ref{subsec.distr}.

We also introduce the following useful quantities
$$\Delta= \left\|\frac1T \sum_{t=1}^T u_tu_t^\top\varepsilon_t^2-\E\left[u_tu_t^\top\varepsilon_t^2\right]  \right\|_\infty\text{, }R(\mu,e)=\frac{1}{\sqrt{T}}\left\|\widehat{W}(\mu,e)-W(e)\right\|_\infty,$$
for $\mu>0$, the event $\mathcal{S}_{\mu}=\left\{\frac{2}{T}\left\|\widehat{U}^\top (\widetilde{Y}-\widehat{U}\beta^*)\right\|_\infty\le {\frac{2}{\sqrt{T}}\mu}\right\}.$ 
The above terms and events are controlled in Section \ref{subsec.prob}.

The following sequences allow to bound some important terms in the proofs.
\begin{align*}
s_T^{(1)} &=\sqrt{\log(T\vee p)}\sqrt{\frac{\log(T)\log(p)}{T}};\\
s_T^{(2)} &=\sqrt{\log(T\vee p)}\left(\frac1p+\frac{\log(p)}{T} +\sqrt{\frac{\log(p)}{Tp}}\right)(\|\varphi^*\|_2\vee 1);\\
s_T^{(3)} &= \sqrt{\log(T\vee p)}\left(\frac{\log(p)}{T} +\frac1p+1\right);\\
s_T^{(4)}&= \sqrt{\log(T\vee p)} \left(\frac{\log(p)}{T} +\frac1p\right)\left(\log(Tp)^{2/\theta_1}\vee \left\|\varphi^*\right\|_2^2\right);\\
s_T^{(5)} &=\sqrt{\log(T\vee p)} \sqrt{\frac{\log(p)}{T}};\\
s_T^{(6)}&=\frac{2}{T^{1/4}}\sqrt{ \log(Tp)2\|\beta^*\|_1s_T^{(3)}};\\
s_T^{(7)}&=\sqrt{\log(Tp)\frac{s_T^{(4)}}{T}};\\
s_T^{(8)}&=\left(s_T^{(1)}\right)^{1/3}\left(1\vee 2\log(2p)\vee \log\left(1/s_T^{(1)}\right)\right)^{1/3}\log(2p)^{1/3};\\
s_T^{(9)}&=\bar C\Biggl(\frac{(\log(T)^{\theta_2+1}\log(p)+(\log(Tp))^{2/\theta}(\log(p))^2\log(T)}{\sqrt{T}\sigma_*^2}\\
&\quad + \frac{\log(p)^2+\log(p)^{3/2}\log(T)+\log(p)(\log(T))^{\theta_2+1}\log(Tp)}{T^{1/4}\sigma_*^2}\Biggl);\\
s_T^{(10)}&=\frac{1}{T\vee p}+s_T^{(9)};\\
s_T^{(11)}&= \bar K\left(\sqrt{2\log(2p)}+\sqrt{2\log(T\vee p)}\right);\\
s_T^{(12)}&= s_T^{(6)}\left(1+s_T^{(11)}\right)+\frac{\sqrt{T}}{2}s_T^{(2)}+s_T^{(6)}\left(1+(1+s_T^{(6)})s_T^{(11)}+\frac{\sqrt{T}}{2}s_T^{(2)}\right)+s_T^{(7)};\\
s_T^{(13)}&=s_T^{(6)}\left(1+s_T^{(11)}\right)+s_T^{(7)};\\
s_T^{(14)}&=s_T^{(9)}+\bar Ks_T^{(12)}\sqrt{1\vee \log\left(2p/s_T^{(12)}\right)} +s_T^{(8)}+\frac3T,
 \end{align*}
 where $\bar K=\kappa_2\E[\varepsilon_t^2]$, $\sigma_*^2=\kappa_1\E\left[\varepsilon_t^2\right]$, $\theta=2\theta_1^{-1}+\theta_2^{-1}$ and $\bar C$ is a constant introduced in Lemma \ref{prop.LV1}. The constants $\kappa_1,\kappa_2,\theta_1$ are defined in Assumption \ref{as.tail} and $\theta_2$ is introduced in Assumption \ref{as.mixing}. In Lemma \ref{lm.seq}, we show that these sequences all go to $0$ under Assumption \ref{as.Rates}. Note that the sequences involving $\varphi^*=\gamma^*-B^\top \beta^*$ are random.
 
Finally, we introduce the following events
\begin{align*}
\mathcal{S}_T^{(1)}&=\left\{\Delta \le s_T^{(1)}\right\};\\
\mathcal{S}_T^{(2)} &=\left\{\left\|\frac{\widehat{U}^\top (\widetilde{Y}-\widehat{U}\beta^*)}{T}- \frac{U^\top\mathcal{E}}{T}\right\|_\infty\le s_T^{(2)}\right\};\\
\mathcal{S}_T^{(3)}&=\left\{\max_{j\in[p]}\frac{1}{T}\sum_{t=1}^T \widehat{u}_{tj}^2\le s_T^{(3)}\right\};\\
\mathcal{S}_T^{(4)}&=\left\{\max_{j\in[p]}\frac{1}{T} \sum_{t=1}^T\left(\widehat{u}_{tj}\widetilde{\varepsilon}_t+\widetilde{f}_t^\top\varphi^*- u_{tj}\varepsilon_t\right)^2\le s_T^{(4)}\right\};\\
\mathcal{S}_T^{(5)}&= \left\{\left\|\frac{U^\top \mathcal{E}}{T}\right\|_\infty\le s_T^{(5)}\right\},
 \end{align*}
 where $\widetilde{\varepsilon}_t$ denotes the $t^{th}$ element of  $\left(I_T-\widehat{P}\right)\mathcal{E}$ and $\widetilde{f}_t$ is the $K\times 1$ vector corresponding to the $t^{th}$ row of $\left(I_T-\widehat{P}\right)F$.
 We show that the probabilities of these events go to $1$ with $T$ in Lemma \ref{lm.bound}. 

 \subsection{Proof of Theorem \ref{th}}\label{subsec.proof}
\textit{Proof of \eqref{thi}.}
We want to show that when $\beta^*=0$, we have 
\begin{equation}\label{toshow206i1}\P\left(\left\|\frac{\widehat{U}^\top \widetilde{Y}}{T}\right\|_\infty> \widehat{\lambda}_\alpha\right)\le \alpha+o(1).\end{equation}
Remark that
\begin{align}
\notag \P\left(\left\|\frac{\widehat{U}^\top \widetilde{Y}}{T}\right\|_\infty>\widehat{\lambda}_\alpha\right)&\le \P\left( \left\|\frac{U^\top \mathcal{E}}{T}\right\|_\infty+\left\|\frac{\widehat{U}^\top \widetilde{Y}}{T}-\frac{U^\top \mathcal{E}}{T}\right\|_\infty> \widehat{\lambda}_\alpha\right)\\
\notag &\le \P\left(\left\{\left\|\frac{U^\top \mathcal{E}}{T}\right\|_\infty>  \widehat{\lambda}_\alpha -s_T^{(2)}\right\}\cap\mathcal{S}_T^{(2)} \right)+ \P\left((\mathcal{S}_T^{(2)})^c \right)\\
\label{toshow206i2} &\le \P\left(\left\|\frac{U^\top \mathcal{E}}{T}\right\|_\infty> \widehat{\lambda}_\alpha - s_T^{(2)}\right)+o(1),
\end{align}
where in the last line we used Lemma \ref{lm.bound} (ii).
%Next, by Lemma \ref{prop.LV1}, we have 
%$$\P\left(\frac1T \|U^\top \mathcal{E}\|_\infty\ge \widehat{\lambda}_\alpha - \ets_T^{(1)}\right)\le \P\left(\Pi^G\ge  \widehat{\lambda}_\alpha-\ets_T^{(1)}\right)+C_2n^{-C_1}.$$

Let us define
$$\mathcal{T}_1=\mathcal{S}_{\mu^*_{\alpha+s_T^{(14)}}}\cap \mathcal{S}_T^{(1)}\cap \mathcal{S}_T^{(2)}\cap \mathcal{S}_T^{(3)}\cap\mathcal{S}_T^{(4)}.$$
Note that, by Lemmas \ref{lm.bound} and \ref{lm.ps}, and the fact that $s_T^{(14)}\to 0$ by Lemma \ref{lm.seq} \eqref{lsiv}, \eqref{lsv} and \eqref{lsvi},  the event $\mathcal{T}_1$ has probability going to $1-\alpha$. 

Hence, by \eqref{toshow206i2}, to show \eqref{toshow206i1}, it suffices to prove that,  on $\mathcal{T}_1$, we have \begin{equation}\label{toshow206i31}\widehat{\lambda}_\alpha\ge \frac{2}{\sqrt{T}}\mu_{\alpha+s_T^{(14)}}^*+s_T^{(2)}.\end{equation}
Indeed, in this case, we would have 
\begin{align*}\P\left(\left\|\frac{\widehat{U}^\top \widetilde{Y}}{T}\right\|_\infty> \widehat{\lambda}_\alpha\right)&\le \P\left(\left\|\frac{U^\top \mathcal{E}}{T}\right\|_\infty>  \widehat{\lambda}_\alpha - s_T^{(2)}\right)+o(1)\\
&\le \P\left(\left\{\left\|\frac{U^\top \mathcal{E}}{T}\right\|_\infty>  \widehat{\lambda}_\alpha - s_T^{(2)}\right\}\cap \mathcal{T}_1\right)+\P(\mathcal{T}_1^c)+o(1)\\
&\le \P\left(\left\{\left\|\frac{U^\top \mathcal{E}}{T}\right\|_\infty>  \frac{2}{\sqrt{T}}\mu_{\alpha+s_T^{(14)}}^*\right\}\cap \mathcal{T}_1\right) +\P(\mathcal{T}_1^c)+o(1)\\
&=0+\P(\mathcal{T}_1^c)+o(1) = \alpha+o(1),\end{align*}
where, on the last line, we used $\mathcal{S}_{\mu^*_{\alpha+s_T^{(14)}}}\subset \mathcal{T}_1$. 


Let us therefore prove that, on $\mathcal{T}_1$, \eqref{toshow206i31} holds. To do so, we show that, on $\mathcal{T}_1$,
\begin{equation}\label{toshow206i4}\P_e\left(\widehat{\Pi}(\mu,e)> \mu\right) >\alpha \end{equation}
for $\mu=(1+s_T^{(6)})\mu_{\alpha+s_T^{(14)}}^*+\frac{\sqrt{T}}{2}s_T^{(2)}>\mu_{\alpha+s_T^{(14)}}^*+\frac{\sqrt{T}}{2}s_T^{(2)}$, which implies that \eqref{toshow206i31} is true by definition of $\widehat{\lambda}_\alpha$ and the fact that  $\widehat{\mu}_\alpha =\frac{\sqrt{T}}{2}\widehat{\lambda}_\alpha$.
We have 
\begin{align*}
\P_e\left(\widehat{\Pi}(\mu,e)> \mu\right)&\ge \P_e\left(\Pi(e)-R(\mu,e)> \mu\right)\\
&\ge \P_e\left(\Pi(e)-R(\mu,e)> \mu, R(\mu,e)\le s_T^{(6)}\sqrt{\mu}+s_T^{(7)}\right)\\
&\ge  \P_e\left(\Pi(e)> \mu+s_T^{(6)}\sqrt{\mu}+s_T^{(7)}\right)-\P_e\left( R(\mu,e)>s_T^{(6)}\sqrt{\mu}+s_T^{(7)}\right)\\
&\ge \P_e\left(\Pi(e)> \mu+s_T^{(6)}\sqrt{\mu}+s_T^{(7)}\right)-\frac2T,
\end{align*}
where, on the last line, we used Lemma \ref{lm.lassodiff} and the facts that $ \mu\ge \mu_{\alpha+s_T^{(14)}}^*$ and we work on $\mathcal{S}^{(3)}_T\cap\mathcal{S}^{(4)}_T\cap \mathcal{S}_{\mu^*_{\alpha+s_T^{(14)}}}\subset \mathcal{T}_1$ to obtain that $\P_e\left( R(\mu,e)>s_T^{(6)}\sqrt{\mu}+s_T^{(7)}\right)\le \frac2T$.
By Lemma \ref{prop.LV2}, we obtain
\begin{equation}\label{eq2061}\P_e(\Pi(e)> \mu+s_T^{(6)}\sqrt{\mu}+s_T^{(7)})\ge \P(\Pi^G\ge\mu+s_T^{(6)}\sqrt{\mu}+s_T^{(7)}) -s_T^{(8)}-\frac2T.\end{equation}
Since $\sqrt{\mu}\le (1+\mu)$, for $T$ large enough, it holds that
\begin{align}
\notag &\P\left((\Pi^G> \mu+s_T^{(6)}\sqrt{\mu}+s_T^{(7)}\right)\\
\notag &\ge \P\left(\Pi^G> \mu+s_T^{(6)}(1+\mu)+s_T^{(7)}\right)\\
\notag &\ge \P\left(\Pi^G> (1+s_T^{(6)})\mu_{\alpha+s_T^{(14)}}^*+\frac{\sqrt{T}}{2}s_T^{(2)}+s_T^{(6)}\left(1+(1+s_T^{(6)})\mu_{\alpha+s_T^{(14)}}^*+\frac{\sqrt{T}}{2}s_T^{(2)}\right)+s_T^{(7)}\right)\\
\label{B4147}&\ge  \P\left(\Pi^G> \mu_{\alpha+s_T^{(14)}}^*+s_T^{(12)}\right)\\
\notag&\ge \P\left(\Pi^G>\mu_{\alpha+s_T^{(14)}}^*)- \P(|\Pi^G-\mu_{\alpha+s_T^{(14)}}^*|\le s_T^{(12)}\right)\\
\label{B3147}&\ge \P\left(\Pi^G>\mu_{\alpha+s_T^{(14)}}^*\right)-\bar Ks_T^{(12)}\sqrt{1\vee \log\left(2p/s_T^{(12)}\right)}\\
\label{B1147}&\ge  \P\left(\Pi^*>\mu_{\alpha+s_T^{(14)}}^*\right)-s_T^{(9)}-\bar Ks_T^{(12)}\sqrt{1\vee \log\left(2p/s_T^{(12)}\right)}\\
\notag &= \alpha+s_T^{(14)} -s_T^{(9)}-\bar Ks_T^{(12)}\sqrt{1\vee \log\left(2p/s_T^{(12)}\right)},
\end{align}
where, in \eqref{B4147}, we used  Lemma \ref{lm.G} and the fact that $s_T^{(14)}\to 0$ by Lemma \ref{lm.seq} \eqref{lsiv}, \eqref{lsv} and \eqref{lsvi}, to obtain that $\mu_{\alpha+s_T^{(14)}}^*\le s_T^{(11)}$ for $T$ large enough, in \eqref{B3147}, we leveraged Lemma \ref{prop.LV3} and \eqref{B1147} follows from Lemma \ref{prop.LV1}.
This and \eqref{eq2061}, therefore yield
$$\P_e\left(\Pi(e)> \mu\right) \ge  \alpha+s_T^{(14)} -s_T^{(9)}-\bar Ks_T^{(12)}\sqrt{1\vee \log\left(2p/s_T^{(12)}\right)} -s_T^{(8)}-\frac2T=\alpha+\frac1T>\alpha,$$
by definition of $s_T^{(14)}$. This shows \eqref{toshow206i4} and therefore concludes the proof of \eqref{thi}. \\

\noindent \textit{Proof of (ii).}
We want to show that if $\sqrt{\frac{\log(T\vee p)}{T\wedge p}}=o_P\left(\left\|\frac{U^\top U\beta^*}{T}\right\|_\infty\right)$, we have 
\begin{equation}\label{toshow206ii1}\P\left(\left\|\frac{\widehat{U}^\top \widetilde{Y}}{T}\right\|_\infty> \widehat{\lambda}_\alpha\right)\to 1.\end{equation}
It holds that
\begin{align}
\notag \P\left(\left\|\frac{\widehat{U}^\top \widetilde{Y}}{T}\right\|_\infty> \widehat{\lambda}_\alpha\right)
&\ge \P\left(\left\|\frac{U^\top U\beta^*}{T}\right\|_\infty-  \left\|\frac{U^\top \mathcal{E}}{T}\right\|_\infty-\left\|\frac{\widehat{U}^\top (\widetilde{Y}-\widehat{U}\beta^*)}{T}-\frac{U^\top \mathcal{E}}{T}\right\|_\infty> \widehat{\lambda}_\alpha\right)\\
\notag&\ge \P\left(\left\{\left\|\frac{U^\top U\beta^*}{T}\right\|_\infty>  \widehat{\lambda}_\alpha + s_T^{(2)}+s_T^{(5)}\right\}\cap\mathcal{S}_T^{(2)}\cap\mathcal{S}_T^{(5)} \right)\\
\notag& \ge \P\left(\left\|\frac{U^\top U\beta^*}{T}\right\|_\infty>  \widehat{\lambda}_\alpha + s_T^{(2)}+s_T^{(5)}\right)- \P\left(\left(\mathcal{S}_T^{(2)}\cap\mathcal{S}_T^{(5)}\right)^c\right)\\
\label{toshow206ii2} &= \P\left(\left\|\frac{U^\top U\beta^*}{T}\right\|_\infty>  \widehat{\lambda}_\alpha + s_T^{(2)}+s_T^{(5)}\right)+o(1),
\end{align}
where, in the last line, we used Lemma \ref{lm.bound}. 

Let us define 
$$\mathcal{T}_2=\mathcal{S}_{\mu^*_{2s_T^{(10)}}}\cap \mathcal{S}_T^{(1)}\cap \mathcal{S}_T^{(2)}\cap \mathcal{S}_T^{(3)}\cap\mathcal{S}_T^{(4)}\cap \mathcal{S}_T^{(5)}.$$Note that, by Lemmas \ref{lm.bound} and \ref{lm.ps}, and the fact that $s_T^{(10)}=\frac{1}{T\vee p}+s_T^{(9)}\to 0$ by Lemma \ref{lm.seq} \eqref{lsv},  the event $\mathcal{T}_2$ has probability going to $1$. 

Hence, by \eqref{toshow206ii2}, to show \eqref{toshow206ii1}, it suffices to prove that,  on $\mathcal{T}_2$, we have \begin{equation}\label{toshow206ii3}\widehat{\lambda}_\alpha\le \frac{2}{\sqrt{T}}\mu_{2s_T^{(10)}}^*,\end{equation}
for $T$ large enough.
Indeed, in this case, we would have 
\begin{align*}\P\left(\left\|\frac{\widehat{U}^\top \widetilde{Y}}{T}\right\|_\infty\ge \widehat{\lambda}_\alpha\right)&\ge  \P\left(\left\|\frac{U^\top U\beta^*}{T}\right\|_\infty>  \widehat{\lambda}_\alpha + s_T^{(2)}+s_T^{(5)}\right)+o(1),\\
&\ge \P\left(\left\{\left\|\frac{U^\top U\beta^*}{T}\right\|_\infty> \frac{2}{\sqrt{T}}\mu_{2s_T^{(10)}}^* + s_T^{(2)}+s_T^{(5)}\right\}\cap \mathcal{T}_2\right)+o(1)\\
&\ge \P\left(\left\{\left\|\frac{U^\top U\beta^*}{T}\right\|_\infty>  \frac{2}{\sqrt{T}}s_T^{(11)} + s_T^{(2)}+s_T^{(5)}\right\}\cap \mathcal{T}_2\right)+o(1)\\
&\ge \P\left(\left\|\frac{U^\top U\beta^*}{T}\right\|_\infty>  \frac{2}{\sqrt{T}}s_T^{(11)} + s_T^{(2)}+s_T^{(5)}\right)-\P(\mathcal{T}_2^c) +o(1)
\to 1 ,\end{align*}
where, in the third line, we used $\mu_{2s_T^{(10)}}^*\le s_T^{(11)} $ by Lemma \ref{lm.G} and, in the last line, we leveraged the facts $\frac{2}{\sqrt{T}}s_T^{(11)} + s_T^{(2)}+s_T^{(5)}=O_P\left(\sqrt{\frac{\log(T\vee p)}{T\wedge p}}\right)$ by Lemma \ref{lm.seq} \eqref{lsii} and that $\sqrt{\frac{\log(T\vee p)}{T\wedge p}}=o_P\left(\left\|\frac{U^\top U\beta^*}{T}\right\|_\infty\right)$ to obtain that $ \P\left(\left\|\frac{U^\top U\beta^*}{T}\right\|_\infty>  \frac{2}{\sqrt{T}}s_T^{(11)} + s_T^{(2)}+s_T^{(5)}\right)\to 1$.

Let us therefore prove that, on $\mathcal{T}_2$, \eqref{toshow206ii3} holds for $T$ large enough. To do so, we show that, on $\mathcal{T}_2$, for $T$ large enough,
\begin{equation}\label{toshow206ii4}\P_e\left(\widehat{\Pi}(\mu_{2s_T^{(10)}}^*,e)> \mu_{2s_T^{(10)}}^*\right) \le \alpha, \end{equation}
which implies \eqref{toshow206ii3} by definition of $\widehat{\lambda}_\alpha$. 
On $\mathcal{T}_2$, we have \begin{align*}
\P_e\left(\widehat{\Pi}(\mu_{2s_T^{(10)}}^*,e)> \mu_{2s_T^{(10)}}^*\right)&\le \P_e\left(\Pi(e)+R(\mu_{2s_T^{(10)}}^*,e)> \mu\right)\\
&\le \P_e\left(\Pi(e)> \mu_{2s_T^{(10)}}^* -R(\mu_{2s_T^{(10)}}^*,e), R(\mu_{2s_T^{(10)}}^*,e)\le s_T^{(6)}\sqrt{\mu_{2s_T^{(10)}}^*}+s_T^{(7)}\right)\\
&\quad +\P_e\left( R(\mu_{2s_T^{(10)}}^*,e)> s_T^{(6)}\sqrt{\mu_{2s_T^{(10)}}^*}+s_T^{(7)}\right)\\
&\le  \P_e\left(\Pi(e)> \mu_{2s_T^{(10)}}^*-s_T^{(6)}\sqrt{\mu_{2s_T^{(10)}}^*}-s_T^{(7)}\right)+\frac2T,
\end{align*}
where, in the last line, we used Lemma \ref{lm.lassodiff}.
By Lemma \ref{prop.LV2}, we obtain
\begin{equation}\label{eq2061ii}\P_e\left(\Pi(e)>\mu_{2s_T^{(10)}}^*\right)\le \P\left(\Pi^G\ge \mu_{2s_T^{(10)}}^*-s_T^{(6)}\sqrt{\mu_{2s_T^{(10)}}^*}-s_T^{(7)}\right) +s_T^{(8)}+\frac2T.\end{equation}
Since $\sqrt{\mu_{2s_T^{(10)}}^*}\le 1+\mu_{2s_T^{(10)}}^*$, it holds that\begin{align}
\notag &\P\left(\Pi^G> \mu_{2s_T^{(10)}}^*-s_T^{(6)}\sqrt{\mu_{2s_T^{(10)}}^*}-s_T^{(7)}\right)\\
\notag &\le \P\left(\Pi^G> \mu_{2s_T^{(10)}}^*-s_T^{(6)}(1+\mu_{2s_T^{(10)}}^*)-s_T^{(7)}\right)\\
\label{B42147} &\le \P\left(\Pi^G>  \mu_{2s_T^{(10)}}^*-s_T^{(6)}(1+s_T^{(11)})-s_T^{(7)}\right)\\
\notag &=  \P\left(\Pi^G>\mu_{2s_T^{(10)}}^*-s_T^{(13)}\right)\\
\notag &\le  \P\left(\Pi^G>\mu_{2s_T^{(10)}}^*\right)+\P\left(|\Pi^G-\mu_{2s_T^{(10)}}^*|\le s_T^{(13)}\right)\\
\label{B32147}&\le  \P\left(\Pi^G>\mu_{2s_T^{(10)}}^*\right)+\bar Ks_T^{(13)}\sqrt{1\vee \log\left(2p/s_T^{(13)}\right)}\\
\label{B12147}&\le  \P\left(\Pi^*>\mu_{2s_T^{(10)}}^*\right)+s_T^{(9)}+\bar Ks_T^{(13)}\sqrt{1\vee \log\left(2p/s_T^{(13)}\right)}\\
\notag&=2s_T^{(10)}+s_T^{(9)}+\bar Ks_T^{(13)}\sqrt{1\vee \log\left(2p/s_T^{(13)}\right)},
\end{align}
where, in \eqref{B42147},  we used Lemma \ref{lm.G} to obtain that $\mu_{2s_T^{(10)}}^*\le s_T^{(11)} $, in \eqref{B32147}, we leveraged Lemma \ref{prop.LV1} and \eqref{B12147} follows from Lemma \ref{prop.LV3}.
This and \eqref{eq2061ii}, therefore yield
$$\P_e\left(\Pi(e)> \mu_{2s_T^{(10)}}^*\right)\le 2s_T^{(10)}+s_T^{(9)}+\bar Ks_T^{(13)}\sqrt{1\vee \log\left(2p/s_T^{(13)}\right)} +s_T^{(8)}+\frac2T\le \alpha,$$
for $T$ large enough by Lemma \ref{lm.seq} \eqref{lsiv}, \eqref{lsv}, \eqref{lsvi}.
This shows that \eqref{toshow206ii4} holds and therefore concludes the proof of \eqref{thii}.

\subsection{Auxiliary lemmas on distributions} \label{subsec.distr}

\begin{Lemma}\label{prop.LV1} Under the assumptions of Theorem \ref{th}, it holds that
$$ \sup_{z\in\R}\left|\P(\Pi^*\le z)-\P\left(\Pi^G\le z\right)\right|<s_T^{(9)} .$$
\end{Lemma}
\begin{Proof} The result is a direct consequence of Lemma \ref{lm.highdimclt} applied to $Z_t = u_t\varepsilon_t$ (and the constant $\bar C$ used in the definition of $s_T^{(9)}$ is introduced in Lemma \ref{lm.highdimclt}). Condition \eqref{lhdi} of Lemma \ref{lm.highdimclt} is satisfied with $\zeta_1=\theta_1/2$ by Lemma \ref{lm.2exp}. Assumption \ref{as.mixing} implies that condition \eqref{lhdii} holds with $\zeta_2=\theta_2$ Condition \eqref{lhdiii} holds with $\zeta=\theta=(2\theta_1^{-1}+\theta_2^{-1})^{-1}$, since, by Assumption \ref{as.mixing}, $2\theta_1^{-1}+\theta_2^{-1}>1$. Concerning condition \eqref{lhdiv}, note that, by Assumption \ref{as.tail} \eqref{tailiv}, we have 
\begin{align*}\E\left[\left(\frac{1}{\sqrt{T}} \sum_{t=1}^T u_t\varepsilon_t\right)\left(\frac{1}{\sqrt{T}} \sum_{t=1}^T u_t\varepsilon_t\right) ^\top \right]&=\frac{1}{T} \sum_{t=1}^T\sum_{s=1}^T\E\left[ u_t\varepsilon_tu_{s}^\top\varepsilon_{s} \right]= \E\left[ u_tu_t^\top\varepsilon_t^2\right].\end{align*}
By Assumption \ref{as.tail} \eqref{tailii}, this implies that $$\sigma_{p}\left(\E\left[\left(\frac{1}{\sqrt{T}} \sum_{t=1}^T u_t\varepsilon_t\right)\left(\frac{1}{\sqrt{T}} \sum_{t=1}^T u_t\varepsilon_t\right) ^\top \right]\right)= \sigma_{p}\left(E\left[ u_tu_t^\top\varepsilon_t^2\right]\right)\ge \kappa_1>0,$$
and therefore that condition \eqref{lhdiv} holds. Finally, condition \eqref{lhdv} is satisfied by Assumption \ref{as.Rates} \eqref{rateii}.
\end{Proof}
\begin{Lemma}\label{prop.LV2} Let the assumptions of Theorem \ref{th} hold. On the event $\mathcal{S}_T^{(1)}$,
$$\sup_{z\in\R}\left|\P_e(\Pi(e)\le z)-\P\left(\Pi^G\le z\right)\right|\le s_T^{(8)}.$$
\end{Lemma}
\begin{Proof} Conditionally on $U,\mathcal{E}$, $W(e)$ is a centered Gaussian vector with covariance matrix $T^{-1}\sum_{t=1}^T u_tu_t^\top\varepsilon_t^2 $. Moreover, $G$ is a centered Gaussian vector with covariance matrix $$\E\left[\left(\frac{1}{\sqrt{T}} \sum_{t=1}^T u_t\varepsilon_t\right)\left(\frac{1}{\sqrt{T}} \sum_{t=1}^T u_t\varepsilon_t\right) ^\top \right]=\E\left[\varepsilon_t^2u_tu_t^\top\right],$$
see the proof of Lemma \ref{prop.LV1} for a justification of this equality. %Notice also that, by Assumption \ref{as.tail}, there exists constants $c,C>0$ such that $c<\E[u_{tj}u_{tj}^\top\varepsilon_t^2]<C$ for all $j\in[p].$ 
Remark that, by Assumption \ref{as.tail} \eqref{tailii} , $$\kappa_2 \E\left[\varepsilon_t^2\right]> \E\left[u_{tj}u_{tj}^\top\varepsilon_t^2\right]\ge \kappa_1,$$
for all $j\in[p]$.
We can therefore apply Lemma \ref{lm.glv2} to get 
$$\sup_{z\in\R}\left|\P_e(\Pi(e)\le z)-\P\left(\Pi^G\le z\right)\right|\le \pi(\Delta),$$
where $\pi(\Delta)=K_4\Delta^{1/3}(1\vee \log(2p)\vee \log(1/\Delta))^{1/3}\log(2p)^{1/3}.$
 This yields that, on the event $\mathcal{S}_T^{(1)}$, we have 
$$\sup_{z\in\R}\left|\P_e(\Pi(e)\le z)-\P\left(\Pi^G\le z\right)\right|\le s_T^{(8)}.$$

\end{Proof}
\begin{Lemma}\label{prop.LV3}Under the assumptions of Theorem \ref{th}, there exists a constant $K_4>0$ such that, for all $z_1,z_2>0$, we have 
$$\P\left(\left|\Pi^G-z_1\right|\le z_2\right)\le K_4z_2\sqrt{1\vee \log(2p/z_2)}.$$
\end{Lemma}
\begin{Proof}
This is a direct consequence of Lemma \ref{lm.glv1} of which the conditions are satisfied by Assumption \ref{as.tail} (see the proofs of Lemmas \ref{prop.LV1} and \ref{prop.LV2} for more details).
\end{Proof}

\begin{Lemma}\label{lm.G} There exists a constant $\bar K>0$ such that, for every $\alpha>s_T^{(10)}$, we have
$$\mu^{*}_\alpha\le s_T^{(11)}.$$
\end{Lemma}
\begin{Proof}
Notice that, by Assumption \ref{as.tail} \eqref{tailiv},
\begin{align*}\E\left[(W_j^*)^2\right]=\E\left[\left(\frac{1}{\sqrt{T}}\sum_{t=1}^Tu_{tj} \varepsilon_t\right)^2\right]= \E\left[u_{tj}^2 \varepsilon_t^2\right],
\end{align*}
which, by Assumption \ref{as.tail} \eqref{tailii} and \eqref{tailiv}, is bounded uniformly in $j$ and $t$ by $\bar K=\kappa_2\E[\varepsilon_t^2]>0$.
 Using  Lemma 7 in \cite{chernozhukov2015comparison} and remark A.8 in \cite{lederer2021estimating}, we have, for every $r>0$, 
$$\P\left(\|G\|_\infty \ge \E\left[\|G\|_\infty\right]+r\right)\le\exp\left(-\frac{r^2}{2\bar K}\right).$$
Taking $r=\bar K\sqrt{2\log(T\vee p)}$, we get
$$\P\left(\|G/\bar K\|_\infty \ge \E\left[\|G/\bar K\|_\infty\right]+\sqrt{2\log(T\vee p)}\right)\le\frac{1}{T\vee p}.$$
By the Gaussian maximal inequality (see e.g. Exercise 2.17 in \cite{boucheron2013concentration}), it holds that $\E\left[\|G/\bar K\|_\infty\right]\le \sqrt{2\log(2p)}$, which yields
$$\P\left(\|G\|_\infty \ge \bar K\left(\sqrt{2\log(2p)}+\sqrt{2\log(T\vee p)}\right)\right)\le\frac{1}{T\vee p},$$
so that $\mu^{G}_\alpha\le \bar K \left(\sqrt{2\log(2p)}+\sqrt{2\log(T\vee p)}\right)$ for $\alpha>1/(T\vee p)$ by definition of $\mu^G_\alpha$.
Now, for $\alpha>s_T^{(10)}=(T\vee p)^{-1}+s_T^{(9)}$, by Lemma \ref{prop.LV1}, we have 
$$\P\left(\Pi^*\ge\mu^G_{\alpha -s_T^{(9)}}\right)\le \P\left(\Pi^G\ge \mu^G_{\alpha -s_T^{(9)}}\right)+s_T^{(9)}\le \alpha- s_T^{(9)} +s_T^{(9)}=\alpha .$$
Hence, we obtain $\mu^*_{\alpha}\le \mu^G_{\alpha -s_T^{(9)}}\le \bar K \left(\sqrt{2\log(2p)}+\sqrt{2\log(T\vee p)}\right).$
\end{Proof}





\subsection{Auxiliary lemmas on probabilistic events} \label{subsec.prob}

\begin{Lemma}\label{lm.bound} Under the assumptions of Theorem \ref{th}, it holds that \begin{enumerate}[\textup{(}i\textup{)}]  
\item\label{lm.boundi}  $\P\left(\mathcal{S}_T^{(1)}\right)\to 1$;
\item\label{lm.boundii}  $\P\left(\mathcal{S}_T^{(2)}\right)\to 1$;
\item\label{lm.boundiii}  $\P\left(\mathcal{S}_T^{(3)}\right)\to 1$;
\item\label{lm.boundiv}   $\P\left(\mathcal{S}_T^{(4)}\right)\to 1$;
\item\label{lm.boundv}   $\P\left(\mathcal{S}_T^{(5)}\right)\to 1$.
\end{enumerate}
\end{Lemma}
\begin{Proof} \newline
Result \eqref{lm.boundi} follows directly from Lemma \ref{lm.lf2} \eqref{lf2v}; \eqref{lm.boundii} is a consequence of  Lemma \ref{fancomplex}; \eqref{lm.boundiii} comes from Lemmas \ref{lm.lf} \eqref{lfii} and Lemma \ref{lm.lf2} \eqref{lf2i} and the triangle inequality, \eqref{lm.boundiv} follows from Lemma \ref{20723} and \eqref{lm.boundv} is a direct consequence of Lemma \ref{lm.lf2} \eqref{lf2iii}. 
\end{Proof}

\begin{Lemma}\label{lm.lassodiff}Let the assumptions of Theorem \ref{th} hold. On the event $\mathcal{S}_T^{(3)}\cap \mathcal{S}_T^{(4)}\cap \mathcal{S}_{\mu}$,
we have, for all $\mu'\ge \mu$,
$$\P_e\left(R(\mu',e)\ge s_T^{(6)}\sqrt{ \mu'}+s_T^{(7)}\right) \le \frac2T.$$
\end{Lemma}
\begin{Proof} Take $\mu'\ge \mu$. Remember that $\widetilde{Y}=\left(I_T-\widehat{P}\right)\left(X\beta+F\varphi^*+ \mathcal{E}\right)$. This yields that
$$\widehat{\varepsilon}_{\frac{2}{\sqrt{T}}\mu',t}=\widetilde{y}_t-\widehat{u}_t^\top\widehat{\beta}_{\frac{2}{\sqrt{T}}\mu'}=\widehat{u}_t\left(\beta^*-\widehat{\beta}_{\frac{2}{\sqrt{T}}\mu'}\right)+\widetilde{f}_t^\top\varphi^*+\widetilde{\varepsilon}_t,$$
where we recall that $\widetilde{\varepsilon}_t$ is the $t^{th}$ element of  $\left(I_T-\widehat{P}\right)\mathcal{E}$ and $\widetilde{f}_t$ is the $K\times 1$ vector corresponding to the $t^{th}$ row of $\left(I_T-\widehat{P}\right)F$. This yields \begin{align}
\notag &R(\mu',e)\\
\notag &=\frac{1}{\sqrt{T}}\left\|\widehat{W}(\mu',e)-W(e)\right\|_\infty\\
\notag  &=\frac{1}{T}\max_{j\in[p]}\left|\sum_{t=1}^T\widehat{u}_{tj} \widehat{\varepsilon}_{\frac{2}{\sqrt{T}}\mu',t} e_t-  \sum_{t=1}^Tu_{tj} \varepsilon_te_t\right|\\
\label{decomp6163}&\le \frac{1}{T}\max_{j\in[p]}\left|\sum_{t=1}^T\widehat{u}_{tj}\widehat{u}_t^\top\left(\beta^*-\widehat{\beta}_{\frac{2}{\sqrt{T}}\mu'}\right) e_t \right|+\frac{1}{T}\max_{j\in[p]}\left| \sum_{t=1}^T\left(\widehat{u}_{tj}\widetilde{\varepsilon}_t+\widetilde{f}_t^\top\varphi^*- u_{tj}\varepsilon_t\right) e_t\right| .
\end{align}

	
			Now, we bound the two terms in \eqref{decomp6163}. We start with $\max_{j\in[p]}\left|\sum_{t=1}^T\widehat{u}_{tj}\widehat{u}_t^\top\left(\beta^*-\widehat{\beta}_{\frac{2}{\sqrt{T}}\mu'}\right) e_t \right|$.  Remark that given $(F,U,\mathcal{E})$, we have
		\begin{align*}
		\frac{1}{T}\sum_{t=1}^T\widehat{u}_{tj}\widehat{u}_t^\top\left(\widehat{\beta}_\lambda-\beta^*\right)e_t \sim\mathcal{N}\left(0, \frac{1}{T^2}\sum_{t=1}^T\left(\widehat{u}_{tj}\widehat{u}_{t}^\top\left(\widehat{\beta}_\lambda-\beta^*\right)\right)^2\right)  
		\end{align*}					
By the Gaussian tail bound (equation (2.10) in \cite{vershynin2018high}), for $z>0$, we obtain, for all $j\in[p]$ and $z>0$,
		\begin{equation}\label{0707232}\P_e^*\left(\left| \frac{1}{T}\sum_{t=1}^T\widehat{u}_{tj}\widehat{u}_t^\top\left(\widehat{\beta}_\lambda-\beta^*\right)e_t \right|>z\right) \le 2\exp\left(-\frac{z^2}{\frac{1}{T^2}\sum_{t=1}^T\left(\widehat{u}_{tj}\widehat{u}^\top_t\left(\widehat{\beta}_\lambda-\beta^*\right)\right)^2}\right).\end{equation}
Next, let $\lambda= {\frac{2}{\sqrt{T}}\mu}$ and $\lambda'= {\frac{2}{\sqrt{T}}\mu'}$. By definition of $\widehat{\beta}_{\lambda'}$, it holds that 
	%$$\widehat{\beta}_{\lambda'}\in\argmin_{\beta\in\R^{K}}\frac{1}{T}\left\|\widetilde{Y}-\widehat{U}\beta\right\|^2_2+ \lambda' \left\|\beta\right\|_1.$$
	%This yields
		$$\frac{1}{T}\left\|\widetilde{Y}-\widehat{U}\widehat{\beta}_{\lambda'}\right\|^2_2+ \lambda' \left\|\widehat{\beta}_\lambda\right\|_1\le\frac{1}{T}\left\|\widetilde{Y}-\widehat{U}\beta^*\right\|^2_2+ \lambda \|\beta^*\|_1.$$
		%$$\frac{1}{T}\left\|\widetilde{Y}-\widehat{U}\beta^*+\widehat{U}(\beta^*-\widehat{\beta}_{\lambda'})\right\|^2_2+ \lambda \|\widehat{\beta}_\lambda\|_1\le\frac{1}{T}\left\|\widetilde{Y}-\widehat{U}\beta^*\right\|^2_2+ \lambda \|\beta^*\|_1.$$
		%That is 
		This yields
					\begin{align}\notag &\frac{1}{T}\left\|\widehat{U}(\beta^*-\widehat{\beta}_{\lambda'})\right\|^2_2\\
					\notag& \le \frac2T \left(\widetilde{Y}-\widehat{U}\beta^*\right)^\top\widehat{U}\left(\widehat{\beta}_{\lambda'}-\beta^*\right)+ \lambda' \left(\|\beta^*\|_1-\left\|\widehat{\beta}_{\lambda'}\right\|_1\right)\\
					\notag &\le \frac2T \left\|\widehat{U}^\top \left(\widetilde{Y}-\widehat{U}\beta^*\right)\right\|_\infty\left\|\widehat{\beta}_{\lambda'}-\beta^*\right\|_1+ \lambda'\left(\|\beta^*\|_1- \left\|\widehat{\beta}_{\lambda'}\right\|_1\right)\\
					\notag &\le \lambda' \|\widehat{\beta}_{\lambda'}-\beta^*\|_1+ \lambda' \left(\|\beta^*\|_1-\left\|\widehat{\beta}_{\lambda'}\right\|_1\right)\\
					\label{ineqlasso} &\le 2  \lambda'\|\beta^*\|_1.
					\end{align}
					where we used Hölder's inequality and the fact that we work on $\mathcal{S}_\mu$.
		Moreover, we have 
		\begin{equation}\label{0707231}
		\begin{aligned}
		\frac{1}{T^2}\sum_{t=1}^T(\widehat{u}_{tj}\widehat{u}^\top_t(\widehat{\beta}_\lambda-\beta^*))^2&\le \frac{1}{T}\sum_{t=1}^T \widehat{u}_{tj}^2 \frac{1}{T}\left\|\widehat{U}\left(\beta^*-\widehat{\beta}_\lambda\right)\right\|^2_2\\
		&\le  s_T^{(3)}2\lambda'\|\beta^*\|_1,
		\end{aligned}
		\end{equation}
		by \eqref{ineqlasso} and because we work on $\mathcal{S}_T^{(3)}$. Recall that $s_T^{(6)}= 2\sqrt{ \log(Tp)\|\beta^*\|_1s_T^{(3)}T^{-1/2}}$. Using \eqref{0707232}, \eqref{0707231} and the union bound, we get
		 \begin{align}\notag &\P_e^*\left(\frac{1}{T}\max_{j\in[p]}\left|\sum_{t=1}^T\widehat{u}_{tj}\widehat{u}_t^\top\left(\beta^*-\widehat{\beta}_{\frac{2}{\sqrt{T}}\mu'}\right) e_t \right|_\infty>s_T^{(6)}\sqrt{\mu'}\right)\\
		 \notag &\le p \max_{j\in[p]}\P_e^*\left(\left| \frac{1}{T}\sum_{t=1}^T\widehat{u}_{tj}\widehat{u}_t^\top\left(\widehat{\beta}_\lambda-\beta^*\right)e_t \right|>s_T^{(6)}\sqrt{\mu'}\right)\\
		 \label{boundr1}&\le  \exp\left(-\frac{(s_T^{(6)})^2\mu'}{2\lambda'\|\beta^*\|_1s_T^{(3)}} +\log(p)\right)=T^{-1}.
		 \end{align}
		
		Let us now bound the term
		$\max_{j\in[p]}\left| \sum_{t=1}^T\left(\widehat{u}_{tj}\widetilde{\varepsilon}_t+\widetilde{f}_t^\top\varphi^*- u_{tj}\varepsilon_t\right) e_t\right|$. Conditional on $(F,U,\mathcal{E})$, we have
		\begin{align*}
		\frac1T\sum_{t=1}^T\left(\widehat{u}_{tj}\widetilde{\varepsilon}_t+\widetilde{f}_t^\top\varphi^*- u_{tj}\varepsilon_t\right)  e_t &\sim \mathcal{N}\left(0, \frac{1}{T^2}\sum_{t=1}^T\left(\widehat{u}_{tj}\widetilde{\varepsilon}_t+\widetilde{f}_t^\top\varphi^*- u_{tj}\varepsilon_t\right)^2\right).
		\end{align*}
		Since we work on $\mathcal{S}_T^{(4)}$, by the Gaussian tail bound, this yields, for all $j\in[p]$ and $z>0$,
		$$\P_e^*\left(\left|\frac1T\sum_{t=1}^T\left(\widehat{u}_{tj}\widetilde{\varepsilon}_t+\widetilde{f}_t^\top\varphi^*- u_{tj}\varepsilon_t\right)  e_t\right|>z \right)\le \exp\left(-\frac{Tz^2}{s_T^{(4)}}\right).$$
Recall that $s_T^{(7)}=\sqrt{\log(Tp) T^{-1}s_T^{(4)}}$. Using the union bound, we get
\begin{align}\notag &\P_e^*\left(\max_{j\in[p]}\left|\frac1T\sum_{t=1}^T\left(\widehat{u}_{tj}\widetilde{\varepsilon}_t+\widetilde{f}_t^\top\varphi^*- u_{tj}\varepsilon_t\right) e_t\right|>s_T^{(7)}\right)\\
\notag &\le p \max_{j\in[p]}\P_e\left(\left|\frac1T\sum_{t=1}^T\left(\widehat{u}_{tj}\widetilde{\varepsilon}_t+\widetilde{f}_t^\top\varphi^*- u_{tj}\varepsilon_t\right) e_t\right|>s_T^{(7)}\right)\\
\label{boundr2}&\le p\exp\left(-\frac{\left(s_T^{(7)}\right)^2}{T^{-1}s_T^{(4)}}\right)=T^{-1}.
\end{align}
Using the pigeonhole principle, \eqref{decomp6163}, \eqref{boundr1} and \eqref{boundr2}, we get 
$\P_e^*\left(R(\mu',e)\ge s_T^{(6)}\sqrt{ \mu'}+s_T^{(7)}\right) \le 2T^{-1},$
which yields $\P_e\left(R(\mu',e)\ge s_T^{(6)}\sqrt{ \mu'}+s_T^{(7)}\right) \le 2T^{-1},$ integrating over the distribution of $(F,U,\mathcal{E})$.

\end{Proof}

\begin{Lemma}\label{lm.ps}Under the assumptions of Theorem \ref{th}, we have 
$$\sup_{\alpha'\in(0,1)}\left|\P\left(\mathcal{S}_{\mu_{\alpha'}^*} \right)-(1-\alpha')\right|=o(1).$$
\end{Lemma}
\begin{Proof} Let us first bound $\P\left(\mathcal{S}_{\mu_{\alpha'}^*}\right)$ from above. For $\alpha'\in(0,1)$, we have \begin{align}
\notag \P\left(\mathcal{S}_{\mu_{\alpha'}^*}\right)&=\P\left(2\left\|\frac{\widehat{U}^\top (\widetilde{Y}-\widehat{U}\beta^*)}{T}\right\|_\infty\le \frac{2}{\sqrt{T}}\mu_{\alpha'}^*\right)\\
\notag &\le \P\left(\left\|\frac{U^\top \mathcal{E}}{T}\right\|_\infty-\left\|\frac{\widehat{U}^\top (\widetilde{Y}-\widehat{U}\beta^*)}{T}-\frac{U^\top \mathcal{E}}{T}\right\|_\infty\le  \frac{1}{\sqrt{T}}\mu_{\alpha'}^*\right) \\
\notag &\le \P\left(\left\{\left\|\frac{U^\top \mathcal{E}}{T}\right\|_\infty-\left\|\frac{\widehat{U}^\top (\widetilde{Y}-\widehat{U}\beta^*)}{T}-\frac{U^\top \mathcal{E}}{T}\right\|_\infty\le  \frac{1}{\sqrt{T}}\mu_{\alpha'}^*\right\}\cap\mathcal{S}_T^{(2)}\right) + \P\left( \left(\mathcal{S}_T^{(2)}\right)^c\right)
\\
\label{ps1} &\le \P\left(\left\|\frac{U^\top \mathcal{E}}{T}\right\|_\infty\le  \frac{1}{\sqrt{T}}\mu_{\alpha'}^*+s_T^{(2)}\right) +\P\left( \left(\mathcal{S}_T^{(2)}\right)^c\right).
\end{align}


Now,  we have 
\begin{align}\notag \P\left(\left\|\frac{U^\top \mathcal{E}}{T}\right\|_\infty\le  \frac{1}{\sqrt{T}}\mu_{\alpha'}^*+s_T^{(2)}\right)&=\P\left(\Pi^*\le  \mu_{\alpha'}^*+\sqrt{T}s_T^{(2)}\right)
\\
\notag & \le \P\left(\Pi^G\le  \mu_{\alpha'}^*+\sqrt{T}s_T^{(2)}\right)+s_T^{(9)}\\
\notag &\le \P\left(\Pi^G\le  \mu_{\alpha'}^*\right)+\P\left(\left|\Pi^G-  \mu_{\alpha'}^*\right|\le \sqrt{T}s_T^{(2)}\right)+s_T^{(9)}\\
\notag &\le \P\left(\Pi^*\le  \mu_{\alpha'}^*\right)+\P\left(\left|\Pi^G-  \mu_{\alpha'}^*\right|\le \sqrt{T}s_T^{(2)}\right)+2s_T^{(9)}\\
\label{ps2}&\le 1-\alpha'+\P\left(\left|\Pi^G-  \mu_{\alpha'}^*\right|\le \sqrt{T}s_T^{(2)}\right)+2s_T^{(9)},
\end{align}
where we used Lemma \ref{prop.LV1} in the second and fourth lines.
By Lemma \ref{prop.LV3}, we have $$\P\left(\left|\Pi^G-  \mu_{\alpha'}^*\right|\le s_T^{(2)}\right)\le K_4\sqrt{T}s_T^{(2)}\sqrt{1\vee \log\left(\frac{2p}{\sqrt{T}s_T^{(2)}}\right)}.$$
Combining this, \eqref{ps1} and \eqref{ps2}, we get \begin{equation}\label{ps3}\P\left(\mathcal{S}_{\mu_{\alpha'}^*}\right)\le 1-\alpha'+ \P\left( \left(\mathcal{S}_T^{(2)}\right)^c\right)+K_4\sqrt{T}s_T^{(2)}\sqrt{1\vee \log\left(\frac{2p}{\sqrt{T}s_T^{(2)}}\right)}+2s_T^{(9)}.\end{equation}

By a similar reasoning, we can show that \begin{equation}\label{ps4}\P\left(\mathcal{S}_{\mu_{\alpha'}^*}\right)\ge 1-\alpha'- \P\left( \left(\mathcal{S}_T^{(2)}\right)^c\right)-K_4\sqrt{T}s_T^{(2)}\sqrt{1\vee \log\left(\frac{2p}{\sqrt{T}s_T^{(2)}}\right)}-2s_T^{(9)}.\end{equation}
 Since $\sqrt{T}s_T^{(2)}\sqrt{1\vee \log\left(\frac{2p}{\sqrt{T}s_T^{(2)}}\right)}\to 0$, $s_T^{(8)}\to 0$, $s_T^{(9)}\to 0$ by Lemma \ref{lm.seq} \eqref{lsiii}, \eqref{lsiv}, \eqref{lsv} and $\P\left( \left(\mathcal{S}_T^{(2)}\right)^c\right)\to0 $ by  Lemma \ref{lm.bound} \eqref{lm.boundii}, \eqref{ps3} and \eqref{ps4} yield the result
\end{Proof}


\subsection{Auxiliary lemma on sequences}\label{subsec.seq}
\begin{Lemma} \label{lm.seq} Under Assumption \ref{as.Rates}, we have \begin{enumerate}[\textup{(}i\textup{)}]  \item\label{lsi} $s_T^{(12)}\sqrt{1\vee \log\left(2p/s_T^{(12)}\right)}=o_P(1)$;  \item\label{lsii} $2T^{-1/2}s_T^{(11)}+s_T^{(2)}+s_T^{(5)}=O_P\left(\sqrt{\log(T\vee p)/(T\wedge p)}\right)$;  \item\label{lsiii} $\sqrt{T}s_T^{(2)}\sqrt{1\vee \log\left(\frac{2p}{\sqrt{T}s_T^{(2)}}\right)}=o_P(1)$;  \item\label{lsiv} $s_T^{(8)}=o(1)$;\item\label{lsv} $s_T^{(9)}=o(1)$; \item\label{lsvi}$s_T^{(13)}\sqrt{1\vee \log\left(2p/s_T^{(13)}\right)}=o_P(1).$
\end{enumerate}
\end{Lemma}
\begin{Proof}
\newline \noindent \textit{Proof of \eqref{lsi}.} By Assumption \ref{as.Rates} \eqref{rateii}, we have $s_T^{(3)}=O\left(\sqrt{\log(T\vee p)}\right)$, so that \begin{equation}\label{seq1}s_T^{(6)}=O\left(\left(\frac{\log(T\vee p)^{4}}{T}\|\beta^*\|_1\right)^{1/2}\right).\end{equation} Since $s_T^{(11)}=O\left(\sqrt{\log(T\vee p)}\right)$, this yields
\begin{equation}\label{seq2} s_T^{(6)}s_T^{(11)}=O\left(\left(\frac{\log(T\vee p)^{6}}{T}\|\beta^*\|_1\right)^{1/2}\right).\end{equation}
We also have \begin{equation}\label{seq3} \left(s_T^{(6)}\right)^2s_T^{(11)}=O\left(\left(\frac{\log(T\vee p)^{6}}{T}\|\beta^*\|_1\right)^{1/2}\right),\end{equation}
 because $s_T^{(6)}=o(1)$ by \eqref{seq1} and Assumption \ref{as.Rates} \eqref{rateii}. Next, it holds that 
 \begin{equation}\label{seq4}s_T^{(2)}=O_P\left(\frac{\log(T\vee p)^{3/2}}{T\wedge p}(\|\varphi^*\|_2\vee1)\right),\end{equation}
  so that \begin{equation}\label{seq5}s_T^{(6)}s_T^{(2)}=o_P\left(s_T^{(2)}\right),\end{equation}
  since $s_T^{(6)}=o(1)$ by \eqref{seq1} and Assumption \ref{as.Rates} \eqref{rateii}. Moreover, it holds that $$s_T^{(4)}=O_P\left(\frac{\log(T\vee p)^{\frac32+\frac2{\theta_1}}}{T\wedge p}\left(\|\varphi^*\|_2^2\vee 1\right)\right)$$
 and, therefore, \begin{equation}\label{seq6} s_T^{(7)}=O_P\left(\sqrt{\frac{\log(T\vee p)^{\frac52+\frac2{\theta_1}}}{T(T\wedge p)}}\left(\|\varphi^*\|_2\vee 1\right)\right).\end{equation} Recall that
$$ s_T^{(12)}= 2s_T^{(6)}+2s_T^{(6)}s_T^{(11)}+\left(s_T^{(6)}\right)^2s_T^{(11)}+\frac{\sqrt{T}}{2}s_T^{(2)}+\frac{\sqrt{T}}{2}s_T^{(6)}s_T^{(2)}+s_T^{(7)}.$$
By \eqref{seq1}, \eqref{seq2}, \eqref{seq3}, \eqref{seq4}, \eqref{seq5}, \eqref{seq6}, we obtain 
\begin{align}\notag  s_T^{(12)}&=O_P\left(\left( \sqrt{\frac{\log(T\vee p)^{6}}{T}}\|\beta^*\|_1\right)^{1/2}+ \left(\frac{\log(T\vee p)^{3/2}\sqrt{T}}{(T\wedge p)}+\sqrt{\frac{\log(T\vee p)^{\frac52+\frac2{\theta_1}}}{T(T\wedge p)}}\right)(\|\varphi^*\|_2\vee1)\right)\\
\label{seq7}&= O_P\left(\left( \sqrt{\frac{\log(T\vee p)^{6}}{T}}\|\beta^*\|_1\right)^{1/2}+ \frac{\log(T\vee p)^{3/2}\sqrt{T}}{(T\wedge p)}\left(\sqrt{\frac{\log(T\vee p)^{\frac2{\theta_1}}}{T}}+1\right)\left(\|\varphi^*\|_2\vee1\right)\right).\end{align}
Additionally, we have $(T(T\wedge p))^{-1/2}=o_P\left(s_T^{(7)}\right)=O_P\left(s_T^{(12)}\right)$ so that $\log\left(2p/s_T^{(12)}\right)=O_P\left(\log(2p)+\log\left(\sqrt{T(T\wedge p)}\right)\right)=O_P(\log(T\vee p))$. This and \eqref{seq7} imply
\begin{align*}&s_T^{(12)}\sqrt{1\vee \log\left(2p/s_T^{(12)}\right)}\\
&=O_P\left(\left(\sqrt{\frac{\log(T\vee p)^{8}}{T}}\|\beta^*\|_1\right)^{1/2}+ \frac{\log(T\vee p)^{2}\sqrt{T}}{(T\wedge p)}\left(\sqrt{\frac{\log(T\vee p)^{\frac2{\theta_1}}}{T}}+1\right)\left(\|\varphi^*\|_2\vee1\right)\right)=o_P(1),\end{align*}
by Assumption \ref{as.Rates}.\\

\noindent \textit{Proof of \eqref{lsii}.} The result follows directly from Assumption \ref{as.Rates} and \eqref{seq4}.\\

\noindent \textit{Proof of \eqref{lsiii}.} We have $(T\wedge p)^{-1}=o_P\left(\sqrt{T}s_T^{(2)}\right)$, hence $\left(\frac{2p}{\sqrt{T}s_T^{(2)}}\right)=O_P(\log(T\vee p))$, so that 
$$\sqrt{T}s_T^{(2)}\sqrt{1\vee \left(\frac{2p}{\sqrt{T}s_T^{(2)}}\right)}=O_P\left(\frac{\log(T\vee p)^{5/2}\sqrt{T}}{T\wedge p}(\|\varphi^*\|_2\vee1) \right)=o_P(1),$$
by \eqref{seq4} and Assumption \ref{as.Rates} \eqref{rateii}.

\noindent \textit{Proof of \eqref{lsiv}.} It holds that $T^{-1/2}=o(s_T^{(1)})$, so that $\log\left(1/s_T^{(1)}\right)=O(\log(T))$. This yields 
\begin{align*}s_T^{(8)}&=O\left(\left(\sqrt{\log(T\vee p)}\sqrt{\frac{\log(T)\log(p)}{T}}\log(T\vee p) \log(p)\right)^{1/3}\right)\\
&=O\left(\left(\sqrt{\frac{\log(T\vee p)^7}{T}}\right)^{1/3}\right)=o(1),\end{align*}
by Assumption \ref{as.Rates} \eqref{rateii}.\\

\noindent \textit{Proof of \eqref{lsv}.} We have 
$$s_T^{(9)}=O\left(\sqrt{\frac{\log(T\vee p)^{4+2\theta_2}+ \log(T\vee p)^{6+\frac{4}{\theta}}}{T}}+\left(\frac{\log(T\vee p)^{10}+ \log(T\vee p)^{12+4\theta_2}}{T}\right)^{1/4}\right)=o(1),$$
by Assumption \ref{as.Rates}.\\

\noindent \textit{Proof of \eqref{lsvi}.} The proof is similar to that of (i) and therefore omitted.
 \end{Proof}



\subsection{Auxiliary lemmas on factors and loadings} \label{subsec.fac}
In this Section, we prove useful results on the factors, the factor loadings and their estimators. Let $H=T^{-1}V\widehat{F}^\top FB^\top B$, where $V$ is the $K\times K$ matrix corresponding the $K$ largest eigenvalues of $T^{-1}XX^\top$. Recall that the estimated loadings are $\widehat{B}=\left(\widehat{F}^\top \widehat{F}\right)^{-1}\widehat{F}^\top X=T^{-1}\widehat{F}^\top X$. Let $\widehat{b}_j$ and $b_j$ be the $K\times 1$ vectors corresponding to the $j^{th}$ row of $\widehat{B}$ and $B$, respectively. 
\begin{Lemma}\label{lm.lf}  Under the assumptions of Theorem \ref{th}, the following holds:
\begin{enumerate}[\textup{(}i\textup{)}]  
			\item\label{lfi} $\left\|\widehat{F}-FH^\top\right\|_2^2=O_P\left(\frac{T}{p}+1\right)$;
			\item\label{lfii} $\max_{j\in[p]}\sum_{t=1}^T |\widehat{u}_{tj}-u_{tj}|^2=O_P\left(\log(p)+\frac{T}{p}\right)$;
			\item\label{lfiii} $\left\|H^\top H-I_K\right\|_2^2 =O_P\left(\frac1T+\frac1p\right)$;
			\item\label{lfiv} $\max_{j\in[p]}\left\|\widehat{b}_j -Hb_j\right\|_2=O_P\left(\frac{1}{\sqrt{p}}+\sqrt{\frac{\log(p)}{T}}\right)$;
		 \item\label{lfv} $\left\|V^{-1}\right\|_2=O_P\left(\frac1p\right)$;
		  \item\label{lfvi} $\left\|\widehat{U}-U\right\|_\infty=o_P\left(1\right)$.
	\end{enumerate}
\end{Lemma}
\begin{Proof}
The results follow from Lemmas 5, 10, 11, 12 and Theorem 4 in \cite{fan2013large}, the conditions of these results being satisfied under Assumptions \ref{as.load}, \ref{as.tail}, \ref{as.mixing} and \ref{as.moments}. Indeed, Assumption 1 in \cite{fan2013large} corresponds to our Assumption \ref{as.load}, Assumptions 2 and 3 in \cite{fan2013large} are implied by our Assumptions \ref{as.tail} and \ref{as.mixing}, Assumption 4 (a) and (b) in \cite{fan2013large} corresponds exactly to our Assumption \ref{as.moments} and Assumption 4 (c) in \cite{fan2013large} is implied by our Assumption \ref{as.tail} \eqref{tailiii}.
\end{Proof}



\begin{Lemma}\label{lm.lf2} Under the assumptions of Theorem \ref{th}, the following holds:
\begin{enumerate}[\textup{(}i\textup{)}]  
			\item\label{lf2i}$\max_{j\in[p]}\left|\sum_{t=1}^T u_{tj}^2\right|=O_P(T)$;
			\item\label{lf2ii}$\max_{j\in[p],k\in[K]}\left|\sum_{t=1}^T u_{tj}f_{tk}\right|=O_P\left(\sqrt{T\log(p)}\right)$;
				\item\label{lf2iii}$\left\|U^\top \mathcal{E}\right\|_\infty=O_P\left(\sqrt{T\log(p)}\right)$;
		\item\label{lf2iv}$\max_{j\in[p],k\in[K]}\left|\sum_{t=1}^T  u_{tj}\left(\sum_{\ell=1}^pu_{t\ell}b_{\ell k}\right)\right|=O_P\left(T+\sqrt{Tp\log(p)}\right)$;
\item\label{lf2v}  $\left\|\frac1T \sum_{t=1}^T u_tu_t^\top\varepsilon_t^2-\E\left[u_tu_t^\top\varepsilon_t^2\right]  \right\|_\infty=O_P\left(\sqrt{\frac{\log(p)}{T}}\right);$
						\item\label{lf2vi}$\| \mathcal{E}\|_2=O_P\left(\sqrt{T}\right)$;
						\item\label{lf2vibis}$\| F\|_2=O_P\left(\sqrt{T}\right)$;
						\item\label{lf2vibisii}$\left\| \frac1T F^\top F-I_K\right\|_2=O_P\left(\frac{1}{\sqrt{T}}\right)$;
						\item\label{lf2vii}$\| U\|_2=O_P\left(\sqrt{Tp}\right)$;
				\item\label{lf2viii}$\left\|F^\top \mathcal{E}\right\|_2=O_P\left(\sqrt{T}\right)$;
					\item\label{lf2ix}$\left\|F^\top U\right\|_2=O_P\left(\sqrt{Tp\log(p)}\right)$;
					\item\label{lf2ixbis}$\left\|\mathcal{E}^\top U\right\|_2=O_P\left(\sqrt{Tp\log(p)}\right)$;
	 \item\label{lf2x} $\|UB\|_2^2=O_P\left(Tp\right)$;
			 \item\label{lf2xi} $\left\|F^\top UB\right\|_2^2=O_P\left(Tp\right)$;
			 \item\label{lf2xii} $\left\|\mathcal{E}^\top UB\right\|_2^2=O_P\left(Tp\right)$.
	\end{enumerate}
\end{Lemma}
\begin{Proof} In this proof, we will often apply Lemmas \ref{lm.nagaevl2} and \ref{lm.nagaevsup} to some specific processes. Following the arguments of the proof of Lemma \ref{prop.LV1}, it can be checked that the conditions of Lemmas \ref{lm.nagaevl2} and \ref{lm.nagaevsup} hold for these processes under the Assumptions of Theorem \ref{th}.\\

\noindent\textit{Proof of \eqref{lfi}.} We apply Lemma \ref{lm.nagaevsup} to $Z_t=\left(u_{tj}^2-\E\left[u_{tj}^2\right]\right)_{j=1}^p$
\begin{equation}\label{187}\max_{j\in[p]}\left|\frac1T\sum_{t=1}^T u_{tj}^2-\E\left[u_{tj}^2\right]\right|=O_P\left(\sqrt{\frac{\log(p)}{T}}\right).\end{equation} By the triangle inequality, we obtain
\begin{align*}\max_{j\in[p]}\left|\sum_{t=1}^T u_{tj}^2\right|&\le T \max_{j\in[p]}\left|\frac1T\sum_{t=1}^T u_{tj}^2-\E[u_{tj}^2]\right| + T \max_{j\in[p]}\E\left[u_{tj}^2\right]\\
&=O_P\left(T+\sqrt{T\log(p)}\right)=O_P(T),\end{align*}
where we used $\max_{j\in[p]}\E[u_{tj}^2]\le \|\Sigma\|_\infty \le \max_{j\in[p]}\sum_{\ell=1}^p|\Sigma_{j\ell}|  =O(1)$ by Assumption \ref{as.tail} \eqref{tailii}.\\

\noindent\textit{Proof of \eqref{lf2ii}, \eqref{lf2iii}, \eqref{lf2iv}.} We apply Lemma \ref{lm.nagaevsup} to 
\begin{align*}Z_t&=((u_{tj}f_{tk})_{j=1}^p)_{k=1}^K;\\
Z_t&=(u_{tj}\varepsilon_t)_{j=1}^p,
\end{align*}
 and obtain \eqref{lf2ii}, \eqref{lf2iii}.  \\
 
 \noindent\textit{Proof of \eqref{lf2iv}.} We apply Lemma \ref{lm.nagaevsup} to 
\begin{align*}
Z_t&=\left(\left(u_{tj}\left(p^{-1/2}\sum_{\ell=1}^pu_{t\ell}b_{\ell k}\right)-\E\left[u_{tj}\left(p^{-1/2}\sum_{\ell=1}^pu_{t\ell}b_{\ell k}\right)\right]\right)_{j=1}^p\right)_{k=1}^K,
\end{align*}
 and obtain 
\begin{equation}\label{2407soir}\max_{j\in[p],k\in[K]}\left| \sum_{t=1}^T\left(u_{tj}\left(p^{-1/2}\sum_{\ell=1}^pu_{t\ell}b_{\ell k}\right)-\E\left[u_{tj}\left(p^{-1/2}\sum_{\ell=1}^pu_{t\ell}b_{\ell k}\right)\right]\right)\right| =O_P\left(\sqrt{T\log(p)}\right).\end{equation}
Next, by Assumption \ref{as.moments} \eqref{f4iii}, we have
\begin{align}
\label{2407soir2} \max_{j\in[p],k\in[K]}\left| \sum_{t=1}^T\E\left[u_{tj}\left(\sum_{\ell=1}^pu_{t\ell}b_{\ell k}\right)\right]\right|      \le  TM.
\end{align}
By the triangle inequality and equations \eqref{2407soir} and \eqref{2407soir2}, we obtain
\begin{align*}
&\max_{j\in[p],k\in[K]}\left|\sum_{t=1}^T  u_{tj}\left(\sum_{\ell=1}^pu_{t\ell}b_{\ell k}\right)\right|\\
&\le \sqrt{p} \max_{j\in[p],k\in[K]}\left| \sum_{t=1}^T\left(u_{tj}\left(p^{-1/2}\sum_{\ell=1}^pu_{t\ell}b_{\ell k}\right)-\E\left[u_{tj}\left(p^{-1/2}\sum_{\ell=1}^pu_{t\ell}b_{\ell k}\right)\right]\right)\right|\\
&\quad + \max_{j\in[p],k\in[K]}\left| \sum_{t=1}^T\E\left[u_{tj}\left(\sum_{\ell=1}^pu_{t\ell}b_{\ell k}\right)\right]\right|   =O_P\left(\sqrt{Tp\log(p)}+T\right).\\
\end{align*}

\noindent\textit{Proof of \eqref{lf2v}.} The result directly follows from the application of Lemma \ref{lm.nagaevsup} to  $Z_t= u_tu_t^\top\varepsilon_t^2 -\E\left[u_tu_t^\top\varepsilon_t^2\right]$.\\

\noindent\textit{Proof of \eqref{lf2vi}.} The result follows from applying Lemma \ref{lm.nagaevl2} to $Z_t=\varepsilon_t^2-\E\left[\varepsilon_t^2\right] $ and using the triangle inequality.\\

\noindent\textit{Proof of \eqref{lf2vibis}.}  To obtain this statement, we apply Lemma \ref{lm.nagaevl2} to $Z_t=f_{tk}^2-\E[f_{tk}^2]$, sum over $k$ and use the triangle inequality, noticing that $\E[f_{tk}^2]=1$ by \eqref{id} from the main text. \\

\noindent\textit{Proof of \eqref{lf2vibisii}.}  Statement \eqref{lf2vibisii} follows from the application of Lemma \ref{lm.nagaevl2} to $Z_t=f_{tk}f_{t\ell}-\E[f_{tk}f_{t\ell}]$, summing over $k,\ell$ and using the fact that $\E[f_{t}f_t^\top]=I_K$ by \eqref{id} from the main text and Assumption \ref{as.tail} \eqref{taili}. \\

\noindent\textit{Proof of \eqref{lf2vii}.} This is a direct consequence of \eqref{lf2i}.\\

\noindent\textit{Proof of \eqref{lf2viii}.} We apply Lemma \ref{lm.nagaevl2} to $Z_t=\varepsilon_tf_{tk}$ and obtain $\sum_{t=1}^T\varepsilon_tf_{tk}=O_P\left(\sqrt{T}\right)$. This yields \eqref{lf2viii}, by $\left\|F^\top \mathcal{E}\right\|_2=\sqrt{\sum_{k=1}^K\left(\sum_{t=1}^T\varepsilon_tf_{tk}\right)^2}=O_P\left(\sqrt{T}\right).$ \\

\noindent\textit{Proof of \eqref{lf2ix} and \eqref{lf2ixbis}.} Statement \eqref{lf2ix} follows from \begin{align*}\left\|F^\top U \right\|_2&=\sqrt{\sum_{k=1}^K\sum_{j=1}^p \left(\sum_{t=1}^Tu_{tj}f_{tk}\right)^2}\\
&\le \sqrt{Kp} \max_{j\in[p],k\in[K]}\left|\sum_{t=1}^T u_{tj}f_{tk}\right|=O_P\left(\sqrt{Tp\log(p)}\right),
\end{align*}
by \eqref{lf2ii}. The proof of \eqref{lf2ixbis} leverages similarly \eqref{lf2iii}.\\

\noindent\textit{Proof of \eqref{lf2x}.} We apply Lemma \ref{lm.nagaevl2} to 
\begin{align*}Z_t&=\left(p^{-1/2}\sum_{\ell=1}^pu_{t\ell}b_{\ell k}\right)^2-\E\left[\left(p^{-1/2}\sum_{\ell=1}^pu_{t\ell}b_{\ell k}\right)^2\right]
\end{align*}
and obtain 
\begin{align}
 \label{2507}\max_{k\in[K]}\left|\sum_{t=1}^T \left(\left(p^{-1/2}\sum_{\ell=1}^pu_{t\ell}b_{\ell k}\right)^2-\E\left[\left(p^{-1/2}\sum_{\ell=1}^pu_{t\ell}b_{\ell k}\right)^2\right]\right)\right| &=O_P\left(\sqrt{T}\right).
\end{align}
Note that, by Assumption \ref{as.tail} \eqref{tailiii}, 
\begin{align}
\label{24072}\max_{k\in[K]}\E\left[\left(p^{-1/2}\sum_{\ell=1}^pu_{t\ell}b_{\ell k}\right)^2\right]=O(1).
\end{align}
Then, we obtain the result using the triangle inequality and equations \eqref{2407} and \eqref{24072}:
\begin{align*}
\|UB\|_2^2&= \sum_{t=1}^T \sum_{k=1}^K \left(p^{-1/2}\sum_{\ell=1}^pu_{t\ell}b_{\ell k}\right)^2\\
&\le K p \max_{k\in[K]}\left|\sum_{t=1}^T\left( \left(p^{-1/2}\sum_{\ell=1}^pu_{t\ell}b_{\ell k}\right)^2-\E\left[\left(p^{-1/2}\sum_{\ell=1}^pu_{t\ell}b_{\ell k}\right)^2\right)\right]\right|\\
&\quad + KTp\max_{k\in[K]}\E\left[\left(p^{-1/2}\sum_{\ell=1}^pu_{t\ell}b_{\ell k}\right)^2\right]=O_P\left(Tp\right).
\end{align*}

\noindent\textit{Proof of \eqref{lf2xi}, \eqref{lf2xii}.} We apply Lemma \ref{lm.nagaevl2} to 
\begin{align*}
Z_t&=f_{tk}\left(p^{-1/2}\sum_{\ell=1}^pu_{t\ell}b_{\ell h}\right);\\
Z_t&=\varepsilon_{t}\left(p^{-1/2}\sum_{\ell=1}^pu_{t\ell}b_{\ell k}\right)
\end{align*}
and obtain the result by summing over $k,h$.
\end{Proof}



\begin{Lemma}\label{lm.bai}Under the assumptions of Theorem \ref{th}, it holds that $$\widehat{F}-FH^\top = \frac1T FB^\top U^\top \widehat{F}V^{-1}+\frac1T  UBF^\top \widehat{F}V^{-1}+\frac1T UU^\top  \widehat{F}V^{-1}.$$
\end{Lemma}
\begin{Proof}
Recall that $H=\frac1T \widehat{V}\widehat{F}^\top FB^\top B$ and $\widehat{F}V=\frac1T XX^\top\widehat{F}$. As a result, we have
\begin{align*}
\widehat{F}V& = T^{-1}XX^\top\widehat{F}\\
&=  \frac1T (FB^\top  + U)(FB^\top  + U)^\top\widehat{F}\\
&= \frac1T FB^\top B F^\top\widehat{F}+\frac1T  FB^\top U^\top \widehat{F}+ \frac1T UBF^\top \widehat{F}+T^{-1}UU^\top  \widehat{F}.
\end{align*}
Multiplying both sides by $V^{-1}$, we get the result.
\end{Proof}
\begin{Lemma}\label{lm.complex} 
Under the assumptions of Theorem \ref{th}, we have 
$$\left\|\left(\widehat{F}-FH^\top\right)^\top \mathcal{E}\right\|_2=O_P\left(\sqrt{\frac{T\log(p)}{p}}+\sqrt{\frac{\log(p)}{p}}+\log(p) \right).$$
\end{Lemma}
\begin{Proof}
By Lemma \ref{lm.bai}, we have \begin{equation}\label{bai1}\left\|(\widehat{F}-FH^\top)^\top \mathcal{E}\right\|_2 \le J_1+J_2+J_3,\end{equation}
where 
\begin{align*}
J_1&=  \frac1T \left\|\mathcal{E}^\top FB^\top U^\top \widehat{F}V^{-1}\right\|_2;\\
J_2&=  \frac1T \left\|\mathcal{E}^\top UBF^\top \widehat{F}V^{-1}\right\|_2;\\
J_3&=  \frac1T \left\|\mathcal{E}^\top UU^\top  \widehat{F}V^{-1}\right\|_2.
\end{align*}
We have 
\begin{align}
\notag J_1&\le \frac1T \left\|\mathcal{E}^\top F \right\|_2\left(\|UB\|_2 \left\|\widehat{F}-FH^\top\right\|_2+\|H\|_2\left\|B^\top U^\top F\right\|_2\right)\|V^{-1}\|_2\\
\label{bai2}&=O_P\left(\frac{1}{T} \sqrt{T}\left(\sqrt{Tp}\sqrt{\frac{T}{p} +1}+\sqrt{Tp}\right)\frac1p\right)=O_P\left(\frac{1}{\sqrt{p}}+ \frac{\sqrt{T}}{p}\right),
\end{align}
by Lemmas \ref{lm.lf} \eqref{lfi}, \eqref{lfiii}, \eqref{lfv} and \ref{lm.lf2} \eqref{lf2viii}, \eqref{lf2x}, \eqref{lf2xi}.
Moreover, it holds that
\begin{align}
\notag J_2&\le \frac1T \|\mathcal{E}^\top UB\|_2\left\| F\right\|_2\left\|\widehat{F}\right\|_2\|V^{-1}\|_2\\
\label{bai3}&=O_P\left(\frac{1}{T}\sqrt{T} \sqrt{Tp}\sqrt{T}\frac1p\right)=O_P\left(\sqrt{\frac{T}{p}}\right),
\end{align}
by Lemmas \ref{lm.lf} \eqref{lfv} and \ref{lm.lf2} \eqref{lf2vibis}, \eqref{lf2xii} and the fact that $\left\|\widehat{F}\right\|_2=\sqrt{T}$.
We also have 
\begin{align}
\notag J_3&\le \frac1T \left\|\mathcal{E}^\top U\right\|_2\left(\|U\|_2 \left\|\widehat{F}-FH^\top\right\|_2+\left\|U^\top F\right\|_2\right)\|V^{-1}\|_2\\
\notag &=O_P\left(\frac{1}{T}\sqrt{Tp\log(p)} \left(\sqrt{Tp}\sqrt{\frac{T}{p}+1}+ \sqrt{Tp\log(p)}\right)\frac1p\right)\\
\label{bai4}&=O_P\left(\sqrt{\frac{T\log(p)}{p}}+\sqrt{\frac{\log(p)}{p}}+\log(p)\right),
\end{align}
where we used Lemmas \ref{lm.lf} \eqref{lfi}, \eqref{lfv} and \ref{lm.lf2} \eqref{lf2vii}, \eqref{lf2ix}, \eqref{lf2ixbis}.
We obtain the result by \eqref{bai1}, \eqref{bai2}, \eqref{bai3} and \eqref{bai4}.
\end{Proof}
\begin{Lemma}\label{20723} Under the assumptions of Theorem \ref{th}, we have 
$$\max_{j\in[p]} \sum_{t=1}^T\left(\widehat{u}_{tj}\widetilde{\varepsilon}_t+\widetilde{f}_t^\top\varphi^*- u_{tj}\varepsilon_t\right)^2=\left(\log(p) +\frac{T}{p}\right)\left(\log(Tp)^{2/\theta_1}\vee \left\|\varphi^*\right\|_2^2\right).$$
\end{Lemma}
\begin{Proof}
First, notice that, by the triangle inequality,
\begin{align}
\notag &\sqrt{ \sum_{t=1}^T\left(\widehat{u}_{tj}\widetilde{\varepsilon}_t+\widetilde{f}_t^\top\varphi^*- u_{tj}\varepsilon_t\right)^2}\\
\notag &=\sqrt{ \sum_{t=1}^T\left(\widehat{u}_{tj}\left(\widetilde{\varepsilon}_t-\varepsilon_t\right)+\widetilde{f}_t^\top\varphi^*+ \left(\widehat{u}_{tj}- u_{tj}\right)\varepsilon_t\right)^2}\\
\label{2071}&\le \sqrt{ \sum_{t=1}^T\left(\widehat{u}_{tj}\left(\widetilde{\varepsilon}_t-\varepsilon_t\right)\right)^2}+
\sqrt{ \sum_{t=1}^T\left(\widetilde{f}_t^\top\varphi^*\right)^2} +\sqrt{ \sum_{t=1}^T\left( \left(\widehat{u}_{tj}- u_{tj}\right)\varepsilon_t\right)^2}.
\end{align}
We first bound the term 
$\sum_{t=1}^T\left(\widehat{u}_{tj}\left(\widetilde{\varepsilon}_t-\varepsilon_t\right)\right)^2$. Remark that 
\begin{equation}\label{2072}\sum_{t=1}^T\left(\widehat{u}_{tj}\left(\widetilde{\varepsilon}_t-\varepsilon_t\right)\right)^2\le \left\|\widehat{U}\right\|_\infty^2 \left\| \left(I_T-\widehat{P}\right)\mathcal{E} - \mathcal{E}\right\|_2^2=\left\|\widehat{U}\right\|_\infty^2 \left\| \widehat{P}\mathcal{E}\right\|_2^2.\end{equation}
Now, using the tail bound in Assumption \ref{as.tail} \eqref{tailiii} and the union bound, we obtain $\left\|U\right\|_\infty=O_P\left(\log(Tp)^{1/\theta_1}\right)$. Combining this with Lemma \ref{lm.lf} \eqref{lfvi} and $\left\|\widehat{U}\right\|_\infty\le \left\|\widehat{U}-U\right\|_\infty+\left\|U\right\|_\infty$, we get \begin{equation}\label{2073}\left\|\widehat{U}\right\|_\infty^2=O_P\left(\log(Tp)^{2/\theta_1}\right).\end{equation} Next, recall that $\widehat{P}=T^{-1} \widehat{F}\widehat{F}^\top\mathcal{E}$ and $\left\|\widehat{F}\right\|_2=\sqrt{T}$. This yields
\begin{align}\notag \left\| \widehat{P}\mathcal{E}\right\|_2&\le \frac1T\left\|\widehat{F}\right\|_2\left\| \left(\widehat{F}-FH^\top\right)^\top\mathcal{E}\right\|_2+\frac1T\left\|\widehat{F}\right\|_2\left\| H\right\|_2\left\|F^\top\mathcal{E}\right\|_2\\
\label{2074} &=\frac{1}{\sqrt{T}} O_P\left(\sqrt{\frac{T\log(p)}{p}}+\sqrt{\frac{\log(p)}{p}}+\log(p)+\sqrt{T} \right)= O_P\left(1\right),
\end{align}
by Lemmas \ref{lm.lf} \eqref{lfiii}, \ref{lm.lf2} \eqref{lf2viii} and \ref{lm.complex} and the fact that $\log(p)/\sqrt{T}=o(1)$ by Assumption \ref{as.Rates} \eqref{rateii}. Thanks to \eqref{2072}, \eqref{2073} and \eqref{2074}, we obtain 
\begin{equation}\label{2075} \sum_{t=1}^T\left(\widehat{u}_{tj}\left(\widetilde{\varepsilon}_t-\varepsilon_t\right)\right)^2=O_P\left(\log(Tp)^{2/\theta_1}\right).
\end{equation}
Let us now bound the term $ \sum_{t=1}^T\left(\widetilde{f}_t^\top\varphi^*\right)^2$. We have 
\begin{equation}\label{2076}
 \sum_{t=1}^T\left(\widetilde{f}_t^\top\varphi^*\right)^2= \left\|\left(I_T- \widehat{P}\right) F\varphi^*\right\|_2^2
\le \left\|\left(I_T- \widehat{P}\right) F\right\|_2^2\left\|\varphi^*\right\|_2^2.
\end{equation}
Next, notice that 
\begin{align}
\notag \left\|\left(I_T- \widehat{P}\right) F\right\|_2&= \left\|\left(I_T-\frac1T\widehat{F}\widehat{F}^\top\right) F\right\|_2\\
\notag &\le \left\|\frac1T\left(\widehat{F}-FH^\top\right)\left(FH^\top\right)^\top F\right\|_2+ \left\| \frac1TFH^\top \left(\widehat{F}-FH^\top\right)^\top F\right\|_2\\
\label{2077}&\quad + \left\|\left(I_T- \frac1TFH^\top\left(FH^\top\right)^\top \right)F\right\|_2
\end{align}
Then, notice that 
\begin{align}
\notag &\left\|\frac1T\left(\widehat{F}-FH^\top\right)\left(FH^\top\right)^\top F\right\|_2+ \left\| \frac1TFH^\top \left(\widehat{F}-FH^\top\right)^\top F\right\|_2\\
\label{2078} &\le \frac2T \left\|\widehat{F}-FH^\top\right\|_2 \left\|F\right\|_2^2 \left\| H\right\|_2=O_P\left(\sqrt{\frac{T}{p}}+1\right),
\end{align}
by Lemmas \ref{lm.lf} \eqref{lfi}, \eqref{lfiii} and \ref{lm.lf2} \eqref{lf2vibis}.
Moreover, we have
\begin{align}
\notag &  \left\|\left(I_T- \frac1TFH^\top\left(FH^\top\right)^\top \right)F\right\|_2\\
\notag &\le \left\|\left(I_T- \frac1TFF^\top \right)F\right\|_2+  \left\| \frac1TF\left(H^\top H-I_K\right)F^\top F\right\|_2\\
  \label{2079}&\le \left\|F\right\|_2\left\|I_K- \frac1T F^\top F \right\|_2+  \frac1T\left\| F\right\|_2\left\|H^\top H-I_K\right\|_2\left\|F\right\|_2^2= O_P\left(1+\sqrt{\frac{T}{p}}\right),
\end{align}
by Lemmas \ref{lm.lf} \eqref{lfiii}  and \ref{lm.lf} \eqref{lf2vibis}, \eqref{lf2vibisii}. Combining \eqref{2076}, \eqref{2077}, \eqref{2078} and \eqref{2079}, we get 
\begin{equation}\label{20710} \sum_{t=1}^T\left(\widetilde{f}_t^\top\varphi^*\right)^2= O_P\left(1+\frac{T}{p}\right)\left\|\varphi^*\right\|_2^2. \end{equation}
Finally, we bound $\sum_{t=1}^T\left(\left(\widehat{u}_{tj}- u_{tj}\right)\varepsilon_t\right)^2$. Notice that 
\begin{equation}\label{decomp6162}
\max_{j\in[p]}\sum_{t=1}^T\left((\widehat{u}_{tj}-u_{tj})\varepsilon_t\right)^2 \le  \|\mathcal{E}\|_\infty^2\max_{j\in[p]} \sum_{t=1}^T(\widehat{u}_{tj}-u_{tj})^2
\end{equation}
Next, by the tail bound in Assumption \ref{as.tail} \eqref{tailiii} and the union bound, we have $ \|\mathcal{E}\|_\infty^2=O_P\left(\log(T)^{2/\theta_1}\right).$ This, Lemma \ref{lm.lf} \eqref{lfii} and equation
\eqref{decomp6162} yield that 
\begin{equation}\label{20711}
\max_{j\in[p]}\sum_{t=1}^T\left(\left(\widehat{u}_{tj}- u_{tj}\right)\varepsilon_t\right)^2=O_P\left( \left(\log(p) +\frac{T}{p}\right)\log(T)^{2/\theta_1}\right).
\end{equation}
Combining \eqref{2071}, \eqref{2075}, \eqref{20710} and \eqref{20711}, we obtain the result.

\end{Proof}
\begin{Lemma}\label{fancomplex} Under the assumptions of Theorem \ref{th}, we have 
$$\left\|\widehat{U}^\top \left(\widetilde{Y}-\widehat{U}\beta^*\right)-U^\top \mathcal{E}\right\|_2=(\|\varphi^*\|_2\vee 1)O_P\left(\frac{T}{p}+\log(p) +\sqrt{\frac{T\log(p)}{p}}\right).$$
\end{Lemma}
\begin{Proof}
In all this proof, we work on the event $\mathcal{E}_{\sigma}=\{\sigma_{p}\left(H^\top H\right)\ge 1/2\}$ which has probability going to $1$ by Lemma \ref{lm.lf} \eqref{lfiii}. Note that, on $\mathcal{E}_\sigma$, we have \begin{equation}\label{2307}\left\|\left(H^{\top}\right)^{-1}\right\|_2\le \sqrt{K}\left\|\left(H^{\top}\right)^{-1}\right\|_{op} \le \sqrt{K}\sigma_{p}\left(H^\top H\right)^{-1/2}\le \sqrt{2K}.\end{equation} Recall that $\widetilde{Y}=\left(I_T-\widehat{P}\right)(X\beta^* +F\varphi^* +\mathcal{E})$. This yields
\begin{equation}\label{key1}\begin{aligned}\left\|\widehat{U}^\top \left(\widetilde{Y}-\widehat{U}\beta^*\right)- U^\top\mathcal{E}\right\|_\infty &\le \left\|\widehat{U}^\top (F\varphi^*+\mathcal{E})- U^\top\mathcal{E}\right\|_\infty\\
  &\le \left\|\widehat{U}^\top F\varphi^*\right\|_\infty +\left\|\left(\widehat{U}- U\right)^\top\mathcal{E}\right\|_\infty.
\end{aligned}
\end{equation}
Let us first bound $ \left\|\widehat{U}^\top F\varphi^*\right\|_\infty$. Since $\widehat{U}^\top\widehat{F}=0$ and $H^\top$ is invertible on the event $\mathcal{E}_{\sigma}$, it holds that
\begin{equation}\label{key2}\left\|\widehat{U}^\top F\varphi^*\right\|_\infty\le \left\|\left(\widehat{U}-U\right)^\top \left(FH^\top -\widehat{F}\right)\left(H^{\top}\right)^{-1}\varphi^*\right\|_\infty +\left\|U^\top \left(FH^\top -\widehat{F}\right)\left(H^{\top}\right)^{-1}\varphi^*\right\|_\infty.
\end{equation}
We now bound the first term on the right-hand side of \eqref{key2}. By the inequality of Cauchy-Schwartz, we have 
\begin{align}
\notag &\left\|\left(\widehat{U}-U\right)^\top \left(FH^\top -\widehat{F}\right)\left(H^{\top}\right)^{-1}\varphi\right\|_\infty\\
\notag &=\max_{j\in[p]}\left|\left(\left(\widehat{U}-U\right)^\top \left(FH^\top -\widehat{F}\right)\left(H^{\top}\right)^{-1}\varphi^*\right)_j\right|\\
\notag &\le \left(\max_{j\in[p]}\sum_{t=1}^n \left|\widehat{u}_{tj}-u_{tj}\right|^2\right)^{1/2}\left\|\widehat{F}-FH^\top\right\|_2\left\|\left(H^{\top}\right)^{-1}\right\|_2\|\varphi^*\|_2\\
\label{key3} &=\|\varphi^*\|_2O_P\left(\sqrt{\log(p)+\frac{T}{p}}\sqrt{\frac{T}{p}+1}\right)=\|\varphi^*\|_2O_P\left(\frac{T}{p} + \sqrt{\log(p)}+\sqrt{\frac{\log(p)T}{p}}\right),
\end{align}
where we used Lemma \ref{lm.lf} \eqref{lfi}, \eqref{lfii}, \eqref{lfiii} and equation \eqref{2307}.
Next, we control the second term on the right-hand side of \eqref{key2}. By Lemma \ref{lm.bai}, it holds that 

\begin{equation}\label{key4}\left\|U^\top \left(FH^\top -\widehat{F}\right)\left(H^{\top}\right)^{-1}\varphi^*\right\|_\infty\le J_{1}+J_{2}+ J_{3},\end{equation}
where 
\begin{align*}J_{1}&= \frac1T\left\| U^\top FB^\top U^\top \widehat{F}V^{-1}\left(H^{\top}\right)^{-1}\varphi^*\right\|_\infty;\\
J_2&= \frac1T\left\| U^\top UBF^\top \widehat{F}V^{-1}\left(H^{\top}\right)^{-1}\varphi^*\right\|_\infty;\\
J_3&= \frac1T\left\| U^\top UU^\top  \widehat{F}\left(H^{\top}\right)^{-1}\varphi^*\right\|_\infty.
\end{align*}
Remark that
\begin{align}
\notag \left\|B^\top U^\top \widehat{F}\right\|_2&\le\left\|B^\top U^\top\right\|_2 \left\|\widehat{F}-FH^\top\right\|_2+\|H\|_2\left\|B^\top U^\top F\right\|_2\\
\label{160723}&=O_P\left(\sqrt{\frac{T}{p}+1}\sqrt{Tp} +\sqrt{Tp} \right)=O_P(T+\sqrt{Tp}),
\end{align}
by Lemmas \ref{lm.lf} \eqref{lfi}, \eqref{lfiii} and \ref{lm.lf2} \eqref{lf2x}, \eqref{lf2xi}.
By the inequality of Cauchy-Schwartz, this yields
\begin{align}
\notag J_{1}
\notag &=\frac1T \max_{j\in[p]}\left|\left(U^\top FB^\top U^\top \widehat{F}V^{-1}\left(H^{\top}\right)^{-1}\varphi^*\right)_j\right|\\
\notag &=\frac1T \max_{j\in[p]}\left|\sum_{k=1}^K\left(\sum_{t=1}^T u_{tj} f_{tk} \right) \left(B^\top U^\top \widehat{F}V^{-1}\left(H^{\top}\right)^{-1}\varphi^*\right)_k\right|\\
\notag &\le \frac1T \left(\max_{j\in[p]}\left|\sum_{k=1}^K\left(\sum_{t=1}^T u_{tj} f_{tk} \right)^2\right|\right)^{1/2} \left\|B^\top U^\top \widehat{F}\right\|_2 \left\|V^{-1}\right\|_2\left\|(H^{-1})^\top\right\|_2\|\varphi^*\|_2\\
\notag &\le \frac1T  \sqrt{K} \max_{j\in[p]}\left|\sum_{t=1}^T u_{tj} f_{tk} \right| \left\|B^\top U^\top \widehat{F}\right\|_2 \left\|V^{-1}\right\|_2\left\|(H^{-1})^\top\right\|_2\|\varphi^*\|_2\\
\label{key5} &=O_P\left(\frac{1}{Tp}\left(T+\sqrt{Tp} \right)\sqrt{T\log(p)} \right)\|\varphi^*\|_2=O_P\left(\sqrt{\frac{T\log(p)}{p}}\right)\|\varphi^*\|_2,
\end{align}
where we used Lemmas \ref{lm.lf} \eqref{lfiii}, \eqref{lfv} and \ref{lm.lf2} \eqref{lf2ii} and equations \eqref{160723} and \eqref{2307}.
Then, notice that, by Lemma \ref{lm.lf} \eqref{lfi} and \eqref{lfiii}, we have 
\begin{equation}\label{2407}\left\|F^\top \widehat{F}\right\|_2\le \|F\|_2 \left\|\widehat{F}-FH^\top\right\|_2+  \|F\|_2^2 \|H\|_2=O_P(T).\end{equation}
This allows to bound $J_{2}$. Indeed, by the inequality of Cauchy-Schwartz, it holds that 
\begin{align}
J_{2}\notag &=\frac1T \max_{j\in[p]}\left|\left(U^\top UBF^\top \widehat{F}V^{-1}\left(H^{\top}\right)^{-1}\varphi^*\right)_j\right|\\
\notag &= \frac1T \max_{j\in[p]}\left|\sum_{k=1}^K\sum_{t=1}^T  u_{tj}\left(\sum_{\ell=1}^pu_{t\ell}b_{\ell k}\right) \left(F^\top \widehat{F}V^{-1}\left(H^{\top}\right)^{-1}\varphi^*\right)_k\right|\\
\notag &\le \frac1T \max_{j\in[p]} \sqrt{\sum_{k=1}^K\left(\sum_{t=1}^T  u_{tj}\left(\sum_{\ell=1}^pu_{t\ell}b_{\ell k}\right)\right)^2 } \left\| F^\top \widehat{F}V^{-1}\left(H^{\top}\right)^{-1}\varphi^*\right\|_2\\
\notag &\le \frac{1}{T}\sqrt{K} \max_{j\in[p],k\in[K]}\left|\sum_{t=1}^T  u_{tj}\left(\sum_{\ell=1}^pu_{t\ell}b_{\ell k}\right)\right| \left\|F^\top \widehat{F}\right\|_2   \left\|V^{-1}\right\|_2\left\|(H^{-1})^\top\right\|_2\|\varphi^*\|_2\\
\label{key6}&=O_P\left(\frac{1}{Tp}T\left(T+\sqrt{Tp\log(p)} \right)\right)\|\varphi^*\|_2=O_P\left(\frac{T}{p}+\sqrt{\frac{T\log(p)}{p}}\right)\|\varphi^*\|_2,
\end{align}
by Lemmas \ref{lm.lf} \eqref{lfiii}, \eqref{lfv} and \ref{lm.lf2} \eqref{lf2iv}, \eqref{lf2vibis} and equations \eqref{2307} and \eqref{2407}.
Finally note that 
\begin{align}\notag \left\|U^\top \widehat{F}\right\|_2 &\le \left\|U^\top F\right\|_2 \left\|H^\top\right\|_2+ \|U\|_2 \left\|\widehat{F}-FH^\top\right\|_2\\
\label{24072}& =O_P\left(\sqrt{T}+\sqrt{T}\sqrt{\frac{T}{p}+1}\right)=O_P\left(\sqrt{T}+ \frac{T}{\sqrt{p}}\right),\end{align}
by Lemmas \ref{lm.lf} \eqref{lfi}, \eqref{lfiii} and \ref{lm.lf2} \eqref{lf2vii}, \eqref{lf2ix}. Thanks to this, we can bound $J_3$. Indeed, by the inequality of Cauchy-Schwartz, we have 
\begin{align}
\notag J_{3}
\notag &=\frac1T \max_{j\in[p]}\left|\left(U^\top UU^\top  \widehat{F}\left(H^{\top}\right)^{-1}\varphi^*\right)_j\right|\\
\notag &= \frac1T \max_{j\in[p]}\left|\sum_{\ell=1}^p\left(\sum_{t=1}^T  u_{tj}u_{t\ell}\right) \left(U^\top  \widehat{F}\left(H^{\top}\right)^{-1}\varphi^*\right)_{\ell}\right|\\
&\le \frac{1}{T}  \max_{j\in[p]}\sqrt{\sum_{\ell=1}^p\left(\sum_{t=1}^T  u_{tj}u_{t\ell}\right)^2} \left\|U^\top \widehat{F}\right\|_2   \left\|V^{-1}\right\|_2\left\|(H^{-1})^\top\right\|_2\|\varphi^*\|_2 \\
\notag &\le \frac{1}{T} \sqrt{p} \max_{j\in[p]}\left|\sum_{t=1}^T  u_{tj}^2\right|\left\|U^\top \widehat{F}\right\|_2   \left\|V^{-1}\right\|_2\left\|(H^{-1})^\top\right\|_2\|\varphi^*\|_2 \\
\label{key7}&=O_P\left(\frac{1}{Tp}T\sqrt{p} \left( \sqrt{T} +\frac{T}{\sqrt{p}}   \right)\right)\|\varphi^*\|_2=O_P\left(\sqrt{\frac{T}{p}}+ \frac{T}{p}\right)\|\varphi^*\|_2,
\end{align}
where we used Lemmas \ref{lm.lf} \eqref{lfv} and \ref{lm.lf2} \eqref{lf2i} and equations \eqref{2307} and \eqref{24072}.
Then, \eqref{key2}, \eqref{key3}, \eqref{key4}, \eqref{key5}, \eqref{key6}, \eqref{key7} imply that 
\begin{equation}\label{key8}\left\|\widehat{U}^\top F\varphi^*\right\|_\infty=O_P\left(\frac{T}{p}+\sqrt{\log(p)} +\sqrt{\frac{T\log(p)}{p}}\right)\|\varphi^*\|_2.\end{equation}

Let us now bound the second term on the right-hand side of \eqref{key1}, that is $\left\|\left(\widehat{U}- U\right)^\top\mathcal{E}\right\|_\infty.$
Note that \begin{align*} \widehat{U}^\top-U^\top&=X^\top -\widehat{B}\widehat{F}^\top-U^\top\\
&= BF^\top -\widehat{B}\widehat{F}^\top\\
&= B\left(I_K-H^\top H\right)F^\top -\left(\widehat{B}-BH^\top \right)\widehat{F}^\top - BH^\top \left(\widehat{F}-FH\right)^\top.
\end{align*}
This yields
\begin{equation}\label{key9}
\left\|\left(\widehat{U}- U\right)^\top\mathcal{E}\right\|_\infty \le K_1+K_2+K_3,\end{equation} 
where
\begin{align*}K_1&=\left\|B\left(I_K-H^\top H\right)F^\top\mathcal{E} \right\|_\infty;\\
K_2&= \left\|\left(\widehat{B}-BH^\top \right)\widehat{F}^\top\mathcal{E}  \right\|_\infty;\\
K_3&= \left\|BH^\top \left(\widehat{F}-FH\right)^\top\mathcal{E}\right\|_\infty.
\end{align*}
By the inequality of Cauchy-Schwartz, Lemmas \ref{lm.lf} \eqref{lfiii} and \ref{lm.lf2} \eqref{lf2viii} and Assumption \ref{as.moments} \eqref{f4i}, it holds that
\begin{align}\notag K_1 &=\max_{j\in[p]} \left|\sum_{k=1}^K b_{jk}\left(\left(I_K-H^\top H\right)F^\top\mathcal{E}\right)_{k} \right|\\
\notag &\le  \sqrt{K}\|B\|_\infty \left\|I_K-H^\top H\right\|_2\left\|F^\top \mathcal{E}\right\|_2\\
\label{key10}&=O_P\left(\sqrt{\frac{1}{T}+\frac{1}{p}} \sqrt{T}\right)= O_P\left(1+\sqrt{\frac{T}{p}}\right).
\end{align}
Next, we have
\begin{align}
\notag K_2&=\max_{j\in[p]} \left|\sum_{k=1}^K \left(\widehat{b}_{j}-Hb_j\right)_k\left(\widehat{F}^\top\mathcal{E}\right)_{k} \right|\\
\notag &\le  \max_{j\in[p]}\left\|\widehat{b}_j -Hb_j\right\|_2 \left\|\widehat{F}^\top\mathcal{E}\right\|_2\\
\notag &\le\max_{j\in[p]}\left\|\widehat{b}_j -Hb_j\right\|_2 \left(\left\|\left(\widehat{F}-FH^\top\right)^\top\mathcal{E}\right\|_2+\|H\|_2\left\| F^\top \mathcal{E}\right\|_2\right)\\
\notag &= O_P\left(\left(\frac{1}{\sqrt{p}}+\sqrt{\frac{\log(p)}{T}}\right) \left(\sqrt{T} +\sqrt{\frac{T\log(p)}{p}}+\sqrt{\frac{\log(p)}{p}}+\log(p)\right)\right)\\
\label{key11} &=O_P\left(\sqrt{\frac{T}{p}} +\sqrt{\log(p)}\right).
\end{align}
where we used the inequality of Cauchy-Schwartz, Lemmas \ref{lm.lf} \eqref{lfiii}, \eqref{lfiv}, \ref{lm.lf2} \eqref{lf2viii} and \ref{lm.complex} and the fact that $T^{-1/2}\log(p)\to 0$ by Assumption \ref{as.Rates} \eqref{rateii}.
Finally, by the inequality of Cauchy-Schwartz, Lemmas \ref{lm.lf2} \eqref{lfiii} and \ref{lm.complex} and Assumption \ref{as.moments} \eqref{f4i}, it holds that 
\begin{align}\notag K_3&=\max_{j\in[p]} \left|\sum_{k=1}^Kb_{jk}\left(H^\top \left(\widehat{F}-FH\right)^\top\mathcal{E}\right)_{k} \right|\\
\notag &\le  \sqrt{K}\left\|B\right\|_\infty\left\|\left(\widehat{F}-FH^\top\right)^\top\mathcal{E}\right\|_2 \left\|H\right\|_2\\
\label{key12}&=O_P\left(\sqrt{\frac{T\log(p)}{p}}+\sqrt{\frac{\log(p)}{p}}+\log(p)\right).
\end{align}
Combining \eqref{key9}, \eqref{key10}, \eqref{key11} and \eqref{key12} yields
\begin{align}
\label{key13}\frac1T\left\|\left(\widehat{U}- U\right)^\top\mathcal{E}\right\|_\infty=O_P\left(\sqrt{\frac{T\log(p)}{p}}+\sqrt{\frac{\log(p)}{p}}+\log(p)\right).
\end{align}
We obtain the result of the lemma by \eqref{key1}, \eqref{key8} and \eqref{key13}. 
\end{Proof}

\subsection{Pre-existing results on strong mixing sequences and high-dimensional Gaussian vectors}\label{subsec.pre}
In this section, we reformulate some results of \cite{fan2021bridging} and \cite{lederer2021estimating} that we use to prove Theorem \ref{th}.
\subsubsection{Results on strong mixing sequences}
The following result is a direct consequence of Lemmas S.20 and S.21 in \cite{fan2021bridging}. This lemma allows to show that products of variables in $u_{tj},f_{tk},\varepsilon_t$, $p^{-1/2} \sum_{j=1}^pb_ju_{tj}$ have exponential tails.
\begin{Lemma}\label{lm.2exp}
Let $Z_1$ and $Z_2$ be random variables such that, for all $z\ge 0$, we have
\begin{align*}
	\P\left(|Z_1|>z\right)&\le \exp\left(-\left(\frac{z}{K}\right)^{\zeta}\right)\\
		\P\left(|Z_2|>z\right)&\le \exp\left(-\left(\frac{z}{K}\right)^{\zeta}\right)
			\end{align*}
			for some constants $K,\zeta>0$. Then, there exists constants $K_1,K_2>0$ depending only on $K,\zeta$ such that, for all $z\ge0$, we have
			$$\P\left(|Z_1Z_2|>z\right)\le K_1 \exp\left(-\left(\frac{z}{K_2}\right)^{\zeta/2}\right).$$
\end{Lemma}
The next lemma is a tail bound on sums of strong mixing sequences following directly from Lemmas S.3 and S.20 in \cite{fan2021bridging}.
\begin{Lemma}\label{lm.nagaev}
Let $S_T=\sum_{t=1}^T Z_t$, where $\{Z_t\}_{t}$ is a sequence of mean-zero real-valued random variables such that 
\begin{enumerate}[\textup{(}i\textup{)}]  
\item\label{lnagi}  There exist constants $K_{11},K_{12},\zeta_1>0$ such that, for all $t\in[T]$ and $z>0$, we have 
$$\P\left(|Z_t|>z\right)\le K_{11} \exp\left(-\left(\frac{z}{K_{12}}\right)^{\zeta_1}\right);$$
\item\label{lnagii} Theres exist constants $K_2,\zeta_2>0$ such that the strong mixing coefficients of the sequence $\{Z_t\}_{t}$ satisfy $\alpha(t)\le \exp(-K_2n^{\zeta_2})$ for all $t\ge 2$;
\item\label{lnagiii} $\zeta <1$, where $\zeta^{-1}=\zeta_1^{-1}+\zeta_2^{-1}$. 
\end{enumerate}
Then, there exist constants $C_1,C_2,C_3,V>0$ depending only on $K_{11}, K_{12},K_2, \zeta_1,\zeta_2$ such that, for all $z>1$, we have 
$$\P\left(|S_T|\ge z\right)\le T \exp\left(-\frac{z^\zeta}{C_1}\right) +\exp\left(\frac{z^2}{C_2(1+TV)}\right)+ \exp\left(-\frac{z^2}{C_3T}\right). $$ 

\end{Lemma}
The next result is a direct consequence of Lemma \ref{lm.nagaev}, taking $z \propto \sqrt{T}$.
\begin{Lemma}\label{lm.nagaevl2}
Let $S_T=\sum_{t=1}^T Z_t$ satisfy the conditions of Lemma \ref{lm.nagaev} and assume that $\log(T)^{2/\zeta}/T=o(1)$, then we have 
$$|S_T|=O_P\left(\sqrt{T}\right). $$ 

\end{Lemma}
Then, we provide a result on the sup-norm of sums of strong mixing sequences. It is a direct consequence of Lemmas S.5 and S.20 in \cite{fan2021bridging}.
\begin{Lemma}\label{lm.nagaevsup}
Let $S_T=\sum_{t=1}^T Z_t$, where $\{Z_t\}_{t}$ is a sequence of mean-zero $p$-dimensional random vectors, such that 
\begin{enumerate}[\textup{(}i\textup{)}]  
\item\label{lnagsi}  There exist constants $K_{11},K_{12},\zeta_1>0$ such that, for all $t\in[T]$, $j\in[p]$ and $z>0$, we have 
$$\P\left(|Z_{tj}|>z\right)\le K_{11} \exp\left(-\left(\frac{z}{K_{12}}\right)^{\zeta_1}\right);$$
\item\label{lnagsii} Theres exist constants $K_2,\zeta_2>0$ such that the strong mixing coefficients of the sequence $\{Z_{tj}\}_{t}$ satisfy $\alpha(t)\le \exp(-K_2t^{\zeta_2})$ for all $j\in[p]$ and $t\ge 2$;
\item\label{lnagsiii}$\zeta <1$, where $\zeta^{-1}=\zeta_1^{-1}+\zeta_2^{-1}$;
\item\label{lnagsiv} $\log(p)^{(2/\zeta)-1}/T=o(1)$.
\end{enumerate}Then $\left\|S_T\right\|_\infty=O_P\left(\sqrt{T\log(p)}\right).$

\end{Lemma}
The last result of this subsection is a high-dimensional central limit theorem for strong mixing sequences due to Theorem S.13 and Lemma S.20 in \cite{fan2021bridging}.
\begin{Lemma}\label{lm.highdimclt}
Let $S_T=n^{-1/2}\sum_{t=1}^T Z_t$, where $\{Z_t\}_{t}$ is a sequence of mean-zero $p$-dimensional random vectors, such that 
\begin{enumerate}[\textup{(}i\textup{)}] 
\item\label{lhdi} There exist constants $K_{11},K_{12},\zeta_1>0$ such that, for all $t\in[T]$, $j\in[p]$ and $z>0$, we have 
$$\P\left(|Z_{tj}|>z\right)\le K_{11} \exp\left(-\left(\frac{z}{K_{12}}\right)^{\zeta_1}\right);$$
\item\label{lhdii}  There exist constants $K_2,\theta_2>0$ such that the strong mixing coefficients of the sequence $\{Z_{tj}\}_{t}$ satisfy $\alpha(t)\le \exp(-K_2t^{\zeta_2})$ for all $j\in[p]$ and $t\ge 2$;
\item\label{lhdiii} $\zeta <1$, where $\zeta^{-1}=\zeta_1^{-1}+\zeta_2^{-1}$;
\item\label{lhdiv}  There exists $\sigma_*>0$ such that $\sigma_{p}(\Sigma) \ge \sigma_*^2$, where $\Sigma=\E\left[S_TS_T^\top\right]$;
\item\label{lhdv}  $\log(p)^{(1/\zeta)-(1/2)}/T=o(1)$.
\end{enumerate} Let also $G\sim\mathcal{N}(0,\Sigma)$. Then, there exists a constant $\bar C$ such that, for $T$ (and therefore $d$) large enough, for all $z\ge0$, we have
\begin{align*}
&\sup_{z\in\R_+}\left|\P\left(\left\|S_T\right\|_\infty\le z\right)-\P(|G|_\infty \le z) \right|\\
&\le \bar C\Biggl(\frac{(\log(T)^{\zeta_2+1}\log(p)+(\log(Tp))^{2/\zeta}(\log(p))^2\log(T)}{\sqrt{T}\sigma_*^2}\\
&\quad + \frac{\log(p)^2+\log(p)^{3/2}\log(T)+\log(p)(\log(T))^{\zeta_2+1}\log(Tp)}{T^{1/4}\sigma_*^2}\Biggl).
\end{align*}
\end{Lemma}
\subsubsection{Results on high-dimensional Gaussian vectors}
The following two lemmas are direct consequences of Lemmas A.4 and A.5 and Remark A.8 in \cite{lederer2021estimating}. (Note that the lemmas in \cite{lederer2021estimating} themselves follow from results in \cite{chernozukhov2013gaussian} and \cite{chernozhukov2015comparison}).
\begin{Lemma}\label{lm.glv1}
Let $G:=(G_1,\dots,G_p)^\top$ be a mean zero $p$-dimensional Gaussian vector. Suppose that there exist constants $c_3,C_3$ such that $c_3\le \E[G_j^2]\le C_3$ for all $j\in[p]$, then, for every $z,\delta>0$, we have 
$$\P\left(\left|\left\|G\right\|_\infty -z\right|\le\Delta\right)\le C\delta\sqrt{1\vee \log(2p/\delta)},$$
where $C>0$ depends only on $c_3,C_3$. 
\end{Lemma}

\begin{Lemma}\label{lm.glv2}
Let $G:=(G_1,\dots,G_p)^\top$ and $G':=(G'_1,\dots,G_p')^\top$ be two mean zero $p$-dimensional Gaussian vectors with respective covariance matrices $\Sigma^{G}$ and $\Sigma^{G'}$. Define $\Delta= \left\|\Sigma^{G}-\Sigma^{G'}\right\|_\infty$. 
Suppose that there exist constants $c_3,C_3$ such that $c_3\le \E[G_j^2]\le C_3$ for all $j\in[p]$. Then, there exists a constant $C>0$ depending only on $c_3,C_3$ such that 
$$\sup_{z\in\R}\left|\P\left(\left\|G\right\|_\infty\le z\right)-\P\left(\left\|G'\right\|_\infty\le z\right)\right|\le C\delta^{1/3}(1\vee 2\log(2p)\vee \log(1/\delta)^{1/3} (\log(2p))^{1/3}.$$
\end{Lemma}



\bibliographystyle{agsm}
\bibliography{bibliography}




\end{document}





\end{document}


Let us discuss exponential Orlicz norms. For $\theta>0$, we let $$\normiii{Z}_{e^{\theta}}=\inf\left\{C>0: \ \E\left[\psi_\theta\left(\frac{|Z|}{C}\right)\right]\le 1\right\},$$ 
where $\psi_\theta$ is the convex hull of the mapping $x\in\R_+\mapsto \exp(x^\theta)-1$.
We have the following useful Lemmas, which are direct consequences of Lemmas S.20 and S.21 in \cite{fan2021bridging}.
\begin{Lemma} If there exists constants $C_1,C_2>0$  such that, for all $z>0$, we have
$$\P(|Z|>z)\le C_1\exp\left(\left(\frac{z}{C_2}\right)^\theta\right),$$
then $\normiii{Z}_{e^{\theta}}<C$ for some constant $C>0$ depending only on $C_1,C_2$.
\end{Lemma}
\begin{Lemma} If there exists constants $C_1,C_2>0$  such that, for all $z>0$, we have
$$\P(|Z|>z)\le C_1\exp\left(\left(\frac{z}{C_2}\right)^\theta\right),$$
then $\normiii{Z}_{e^{\theta}}<C$ for some constant $C>0$ depending only on $C_1,C_2$.
\end{Lemma}




