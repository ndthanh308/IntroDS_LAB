\documentclass[12pt]{article}
\usepackage{amsmath}
\usepackage{graphicx}
\usepackage{enumerate}
\usepackage{natbib}
\usepackage{array}
\usepackage{algorithmic}
\usepackage[plain]{algorithm}
\newcommand{\alglinelabel}{%
	\addtocounter{ALC@line}{-1}% Reduce line counter by 1
	\refstepcounter{ALC@line}% Increment line counter with reference capability
	\label% Regular \label
}
\newcommand*{\doublerule}{\hrule width \hsize height 0.5pt \kern 0.5mm \hrule width \hsize height 0.5pt}
\newcommand*{\singlerule}{\hrule width \hsize height 0.5pt}
%\usepackage{url} % not crucial - just used below for the URL 

%\pdfminorversion=4
% NOTE: To produce blinded version, replace "0" with "1" below.
\newcommand{\blind}{1}
\usepackage{mathtools}
\usepackage{etoolbox}
% DON'T change margins - should be 1 inch all around.
\addtolength{\oddsidemargin}{-.5in}%
\addtolength{\evensidemargin}{-1in}%
\addtolength{\textwidth}{1in}%
\addtolength{\textheight}{1.7in}%
\addtolength{\topmargin}{-1in}%
\newcommand{\vertiii}[1]{{\left\vert\kern-0.4ex\left\vert\kern-0.4ex\left\vert #1 
		\right\vert\kern-0.4ex\right\vert\kern-0.4ex\right\vert}}
\usepackage{xcolor}

\DeclareMathOperator*{\argmin}{arg\,min}%\usepackage[latin 1]{inputenc}
\usepackage{caption}\newcommand\independent{\protect\mathpalette{\protect\independenT}{\perp}}
\def\independenT#1#2{\mathrel{\rlap{$#1#2$}\mkern2mu{#1#2}}}\usepackage{graphicx,latexsym,amssymb,amsmath}
\usepackage{subcaption}
\usepackage{bbm, bm}
\def\pp{\vskip 10pt}
\def\ppn{\vskip 10pt \noindent }
\def\d{{\mathrm{d}}}
\def\R{{\mathbb{R}}}
\def\Z{{\mathbb{Z}}}
\def\C{{\mathbb{C}}}
\def\N{{\mathbb{T}}}
\def\T{{\mathbb{T}}}
\def\P{{\mathbb{P}}}
\def\E{{\mathbb{E}}}
\def\indic{\hbox{\it 1\hskip -3.5pt I}} % indicator function
\def\convLaw{{\ \buildrel {\Ls} \over \longrightarrow \ }}
\newcommand{{\convp}}{{\buildrel p\over\longrightarrow}}

\DeclareUnicodeCharacter{2212}{\ensuremath{-}}
\DeclareFontFamily{U}{mathx}{}
\DeclareFontShape{U}{mathx}{m}{n}{<-> mathx10}{}
\DeclareSymbolFont{mathx}{U}{mathx}{m}{n}
\DeclareMathAccent{\widehat}{0}{mathx}{"70}
\DeclareMathAccent{\widecheck}{0}{mathx}{"71}
\renewcommand{\overline}{\widecheck}


\newcommand{\red}[1]{{\color{black} #1}}

\newcommand{{\Vs}}{{\cal V}}
\newcommand{{\Ps}}{{\cal P}}
\newcommand{{\Ss}}{{\cal S}}
\newcommand{{\Xs}}{{\cal X}}
\newcommand{{\Ls}}{{\cal L}}
\newcommand{{\Ns}}{{\cal N}}
\newcommand{{\Zs}}{{\cal Z}}
\newcommand{{\Fs}}{{\cal F}}
\newcommand{\tbc}{{\bf ***to be completed***$\quad$}}
\newcommand{\tbchk}{{\bf ***to be checked!!!***$\quad$}}
\newcommand{\chk}{{\bf ***checked!!!***$\quad$}}
\newcommand{\tbm}{{\bf ***to be modified***$\quad$}}
\newcommand{\cuh}{{\medskip\bf ***correct until here***$\quad$ \medskip}}

\newcommand{\cvp}[1]{\stackrel{P}{#1}}
%\newcommand{\cvl}[1]{\stackrel{{d}}{#1}}
\newcommand{\cvl}[1]{\stackrel{{\Ls}}{#1}}
\newcommand{\cps}[1]{\stackrel{a.s.}{#1}}
\newcommand{\pco}[1]{\stackrel{co.}{#1}}
\allowdisplaybreaks
%%%%%%%%%%%%%%%%%%%%%%%%%%%%%%%%%%%%%%%%%%%%%%%%%%%%%%%%
\newtheorem{Lemma}{Lemma}[section] %numbering by section
\newtheorem{Assumption}{Assumption}
\newtheorem{Theorem}{Theorem}[section] %numbering by section

\newtheorem{Remark}{Remark}[section]
\newtheorem{Corollary}{Corollary}
\newcommand{\proofend}{$\hfill\Box{~}$}
\newenvironment{Proof}{\noindent {\em{\bf Proof.}}}{\proofend\\}
%\usepackage{latexsym}
\newtheorem{definition}{Definition}

\newcommand{\tcr}{\textcolor{red}} % track changes in RED
\newcommand{\bs}{\boldsymbol} 

\newcommand{\tb}{|||} 

\makeatother
\usepackage{xr-hyper} \externaldocument{paper}
\usepackage[colorlinks = true,
linkcolor = red,
urlcolor  = blue,
filecolor={red},
citecolor = blue,
anchorcolor = blue]{hyperref}
\usepackage{newfloat}
%\DeclareFloatingEnvironment[
%Xfileext=los,
%listname={List of Schemes},
%name=Algorithm,
%within=section,
%placement=tbhp,
%]{algorithm}


\usepackage{longtable}
\usepackage{xr}
\externaldocument[P-]{paper}

\begin{document}
	
	%\bibliographystyle{natbib}
	
	\def\spacingset#1{\renewcommand{\baselinestretch}%
		{#1}\small\normalsize} \spacingset{0}
	
	
	%%%%%%%%%%%%%%%%%%%%%%%%%%%%%%%%%%%%%%%%%%%%%%%%%%%%%%%%%%%%%%%%%%%%%%%%%%%%%%
	
	\if1\blind
	{
		\title{ Online appendix of ``Testing for sparse idiosyncratic components in factor-augmented regression models''}
		\author{Jad Beyhum\\\vspace{0.5em}
			Department of Economics, KU Leuven, Belgium\\\vspace{0.5em}
			Jonas Striaukas\hspace{.2cm}\\
			Department of Finance, Copenhagen Business School, Denmark.}
		\maketitle
	} \fi
	
	\if0\blind
	{
		\bigskip
		\bigskip
		\bigskip
		\begin{center}
			{\LARGE\bfPenalty }
		\end{center}
		\medskip
	} \fi
	
	
	
	\spacingset{1.5} 
	
	\renewcommand*{\thesection}{\Alph{section}}
	
	
	
	{\hypersetup{linkbordercolor=black,linkcolor = black,
			urlcolor  = blue,
			filecolor=black}
		% or \hypersetup{linkcolor=black}, if the colorlinks=true option of hyperref is used
		\tableofcontents
	}
	\clearpage
	\setcounter{page}{1}
	\setcounter{section}{0}
	\setcounter{equation}{0}
	\setcounter{table}{0}
	\setcounter{figure}{0}
	\renewcommand{\theequation}{OA.\arabic{equation}}
	\renewcommand\thetable{OA.\arabic{table}}
	\renewcommand\thefigure{OA.\arabic{figure}}
	\renewcommand\thepage{OA. - \arabic{page}}
	\renewcommand\thealgorithm{OA.\arabic{algorithm}}

\section{Testing procedure of the model with lagged idiosyncratic terms}\label{subsec.lag}
Algorithm \ref{algo-lags} presents the test for the model with lagged idiosyncratic elements, see \eqref{P-laggedu}. This algorithm differs from the main procedure in Algorithm \ref{P-algo} by augmenting the lagged idiosyncratic terms into the sparse component. The procedure could be extended to accommodate observed factors, similar to Algorithm \ref{P-algo-alternative}. The algorithm with lagged idiosyncratic terms is outlined as follows and can be easily adapted in cases where there is more than one lag. 
	
\begin{algorithm}
	\singlerule\vspace{0.1cm}
	\begin{algorithmic}
		\STATE \textbf{1.} Estimate $\widehat{K}$ by one of the available estimators of the number of factors.
		\STATE \textbf{2.} Let the columns of $\widehat{F}/\sqrt{T}$ be the eigenvectors corresponding to the leading $\widehat{K}$ eigenvalues of $XX^\top$.
		\STATE \textbf{3.} Compute $\widehat{U}=\left(I_T-\widehat{P}\right) X$ and $\widetilde{Y}=\left(I_T-\widehat{P}\right)Y$, where $\widehat{P}=T^{-1} \widehat{F}\widehat{F}^\top$.
		\STATE \begin{itemize}
			\item[\textbf{3a.}] Denote by $\widehat{u}_t$ the $T\times 1$ vector corresponding to transpose of the $t^{th}$ row of $\widehat{U}$. %Let $\widetilde{U}$ be the $\widetilde{T}\times (2p)$ matrix such that 
			For all $t\in\{2,\dots T\}$, let $\widetilde{u}_t=\left(\widehat{u}_t^\top, \widehat{u}_{t-1}^\top\right)$.
		\end{itemize}
		\STATE \textbf{4.} Calculate an approximation $\widehat{\lambda}_{\alpha,emp}$ of $\widehat{\lambda}_\alpha$ as follows:
		\STATE \begin{itemize}
			\item[\textbf{4a.}] Specify a grid $0<\lambda_1<\dots<\lambda_M<\bar{\lambda}$, with $\bar{\lambda}=2(T-1)^{-1}\left\|\sum_{t=2}^T \widetilde{u}_t\widetilde{y}_t\right\|_\infty$.
			\item[\textbf{4b.}]  For $\lambda>0$, and $e\in\R^{\widetilde{T}}$, let $\widehat{Q}\left(\lambda,e\right)=\left\|\frac{2}{T-1} \sum_{t=2}^{T}\widetilde{u}_t\widehat{\varepsilon}_{\lambda,t} e_t,\right\|_\infty$, where $\widehat{\varepsilon}_{\lambda,t}=\widetilde{y}_t -\widetilde{u}_t^\top\widehat{\beta}_{\lambda},\ t\in\{2,\dots,T\}$, for $\widehat{\beta}_\lambda = \argmin_{\beta\in\R^p}\frac{1}{T-1}\sum_{t=2}^T
			\left(\widetilde{y}_t -\widetilde{u}_t^\top\widehat{\beta}_{\lambda}\right)^2+\lambda\|\beta\|_1$. For $m\in[M]$, compute $\left\{\widehat{Q}\left(\lambda_m,e^{(\ell)}\right):\ \ell\in[L]\right\}$ for $L$ draws of $e\sim\mathcal{N}(0,I_{T-1})$ and the corresponding empirical $(1-\alpha)$-quantile $\widehat{q}_{\alpha,emp}(\lambda_m)$ from them.
			\item[\textbf{4c.}] Let $\widehat{\lambda}_{\alpha,emp}=\widehat{q}_{\alpha,emp}(\lambda_{\widehat{m}})$,  with $\widehat{m}=\min\{m\in[M]:\ \widehat{q}_{\alpha,emp}(\lambda_{m'})\le \lambda_{m'}\text{ for all }m'\ge m\}.$
		\end{itemize}
		\STATE \textbf{5.} Reject $H_0$ when $2(T-1)^{-1}\left\|\sum_{t=2}^T \widetilde{u}_t\widetilde{y}_t\right\|_\infty>\widehat{\lambda}_{\alpha,emp}$.
	\end{algorithmic}
	\doublerule
	\caption{Conducting a test of level $\alpha\in(0,1)$ with lagged idiosyncratic terms.\label{algo-lags}}
\end{algorithm}

\section{Additional simulation results}\label{subsec.add.sim}
	%In table \ref{tab.n200} we report simulation results under the same data generating processes as in the main article but with $T=p=200$ instead.
	
	\begin{table}[h!]\setlength\extrarowheight{-4pt}
		\begin{center}
			\begin{tabular}{lccccccc}
				& & p/T = 400/400 & & & & p/T = 2000/400 & \\
				\hline
				$m$&   $\alpha =0.1$ &$\alpha = 0.05$ &$\alpha =0.01$ && $\alpha =0.1$ &$\alpha =0.05$ &$\alpha =0.01$ \\
				\hline\\[-0.3cm]
				\multicolumn{8}{c}{{\it Design 1}: $s=\rho_f=\rho_{u}=\rho_e=0$}\\
				0.0 & 0.068 & 0.024 & 0.002 & & 0.072 & 0.026 & 0.005  \\
				0.1 & 0.085 & 0.038 & 0.005  & & 0.075 & 0.034 & 0.005 \\
				0.2 & 0.460 & 0.367 & 0.210 & & 0.170 & 0.101 & 0.034 \\
				0.3 & 0.949 & 0.926 & 0.847 & & 0.563 & 0.456 & 0.272  \\
				0.4 & 1.000 & 1.000 & 0.999 & & 0.898 & 0.856 & 0.734  \\
				\multicolumn{8}{c}{{\it Design 2}: $s=0.1$, $\rho_f=0.6$, $\rho_{u}=0.1$ and $\rho_e=0$}\\
				0.0 & 0.078 & 0.028 & 0.006 & & 0.070 & 0.027 & 0.005  \\
				0.1 & 0.098 & 0.046 & 0.011 & & 0.074 & 0.030 & 0.004  \\
				0.2 & 0.473 & 0.378 & 0.230 & & 0.174 & 0.101 & 0.034  \\
				0.3 & 0.961 & 0.942 & 0.873 & & 0.551 & 0.448 & 0.272  \\
				0.4 & 1.000 & 0.999 & 0.997 & & 0.900 & 0.850 & 0.720  \\
				\multicolumn{8}{c}{{\it Design 3}: $s=0.1$, $\rho_f=0.6$ and $\rho_{u}=\rho_e=0.1$}\\
				0.0 & 0.088 & 0.035 & 0.007  & & 0.084 & 0.032 & 0.005 \\
				0.1  & 0.112 & 0.052 & 0.014 & & 0.086 & 0.037 & 0.007  \\
				0.2 & 0.470 & 0.373 & 0.226 & & 0.185 & 0.102 & 0.035 \\
				0.3 & 0.961 & 0.935 & 0.870 & & 0.555 & 0.439 & 0.264  \\
				0.4 & 1.000 & 0.999 & 0.996 & & 0.895 & 0.843 & 0.721 \\ 
				\hline\hline
			\end{tabular}
		\end{center}
		\caption{Rejection probabilities for the three dependence designs we consider and sparse $\beta^*$. The data are generated with Gaussian variables. The sample size is $T=400$ while the number of regressors is $p\in\{400, 2000\}$.}
		\label{app.tab.rates_sparse}
	\end{table}


\begin{table}[h!]\setlength\extrarowheight{-4pt}
	\begin{center}
		\begin{tabular}{lccccccc}
			& & p/T = 400/400 & & & & p/T = 2000/400 & \\
			\hline
			$m$&   $\alpha =0.1$ &$\alpha = 0.05$ &$\alpha =0.01$ && $\alpha =0.1$ &$\alpha =0.05$ &$\alpha =0.01$ \\
			\hline\\[-0.3cm]
			\multicolumn{8}{c}{{\it Design 1}: $s=\rho_f=\rho_{u}=\rho_e=0$}\\
			0.0 & 0.068 & 0.024 & 0.002 & & 0.072 & 0.026 & 0.005  \\
			0.1  & 0.074 & 0.028 & 0.003 & & 0.070 & 0.030 & 0.005  \\
			0.2 & 0.086 & 0.032 & 0.004 & & 0.074 & 0.029 & 0.005  \\
			0.3 & 0.094 & 0.041 & 0.005 & & 0.084 & 0.034 & 0.004  \\
			0.4  & 0.104 & 0.052 & 0.006 & & 0.088 & 0.037 & 0.004  \\
			\multicolumn{8}{c}{{\it Design 2}: $s=0.1$, $\rho_f=0.6$, $\rho_{u}=0.1$ and $\rho_e=0$}\\
			0.0 & 0.073 & 0.035 & 0.005 & & 0.070 & 0.031 & 0.006 \\
			0.1 & 0.082 & 0.037 & 0.007 & & 0.070 & 0.032 & 0.008   \\
			0.2 & 0.092 & 0.041 & 0.009 & & 0.078 & 0.036 & 0.007   \\
			0.3  & 0.101 & 0.041 & 0.006 & & 0.080 & 0.036 & 0.008 \\
			0.4  & 0.112 & 0.048 & 0.007 & & 0.092 & 0.037 & 0.007 \\
			\multicolumn{8}{c}{{\it Design 3}: $s=0.1$, $\rho_f=0.6$ and $\rho_{u}=\rho_e=0.1$}\\
			0.0 & 0.090 & 0.043 & 0.006 & & 0.078 & 0.035 & 0.007  \\
			0.1 & 0.094 & 0.044 & 0.009  & & 0.083 & 0.038 & 0.008 \\
			0.2 & 0.110 & 0.046 & 0.008 & & 0.087 & 0.041 & 0.009  \\
			0.3 & 0.116 & 0.052 & 0.007 & & 0.094 & 0.046 & 0.009  \\
			0.4  & 0.131 & 0.056 & 0.007 & & 0.103 & 0.044 & 0.010 \\ 
			\hline\hline
		\end{tabular}
	\end{center}
	\caption{Rejection probabilities for the three dependence designs we consider and dense $\beta^*$. The data are generated  with Gaussian variables. The sample size is $T=400$ while the number of regressors is $p\in\{400, 2000\}$.}
	\label{app.tab.rates_dense}
\end{table}

\begin{table}[h!]\setlength\extrarowheight{-4pt}
	\begin{center}
		\begin{tabular}{lccccccc}
			& & p/T = 200/200 & & & & p/T = 1000/200 & \\
			\hline
			$m$&   $\alpha =0.1$ &$\alpha = 0.05$ &$\alpha =0.01$ && $\alpha =0.1$ &$\alpha =0.05$ &$\alpha =0.01$ \\
			\hline\\[-0.3cm]
			\multicolumn{8}{c}{{\it Design 1}: $s=\rho_f=\rho_{u}=\rho_e=0$}\\
			0.0 & 0.062 & 0.020 & 0.002 & & 0.054 & 0.018 & 0.002 \\
			0.1 & 0.082 & 0.034 & 0.005 & & 0.060 & 0.021 & 0.002  \\
			0.2 & 0.529 & 0.422 & 0.230  & & 0.200 & 0.116 & 0.035 \\
			0.3 & 0.936 & 0.886 & 0.766 & & 0.579 & 0.445 & 0.235  \\
			0.4 & 0.994 & 0.987 & 0.950  & & 0.872 & 0.785 & 0.585 \\
			\multicolumn{8}{c}{{\it Design 2}: $s=0.1$, $\rho_f=0.6$, $\rho_{u}=0.1$ and $\rho_e=0$}\\
			0.0 & 0.053 & 0.023 & 0.004 & & 0.050 & 0.014 & 0.002  \\
			0.1 & 0.090 & 0.040 & 0.006 & & 0.056 & 0.018 & 0.003  \\
			0.2 & 0.537 & 0.414 & 0.220 & & 0.196 & 0.114 & 0.040  \\
			0.3 & 0.942 & 0.894 & 0.761  & & 0.599 & 0.469 & 0.246 \\
			0.4 & 0.992 & 0.987 & 0.950 & & 0.878 & 0.801 & 0.594  \\
			\multicolumn{8}{c}{{\it Design 3}: $s=0.1$, $\rho_f=0.6$ and $\rho_{u}=\rho_e=0.1$}\\
			0.0 & 0.065 & 0.025 & 0.003  & & 0.060 & 0.018 & 0.002 \\
			0.1  & 0.096 & 0.048 & 0.006 & & 0.070 & 0.023 & 0.004  \\
			0.2  & 0.530 & 0.410 & 0.218 & & 0.210 & 0.114 & 0.040 \\
			0.3 & 0.936 & 0.892 & 0.755  & & 0.593 & 0.467 & 0.242 \\
			0.4 & 0.993 & 0.986 & 0.950 & & 0.876 & 0.799 & 0.590  \\ 
			\hline\hline
		\end{tabular}
	\end{center}
	\caption{Rejection probabilities for the three dependence designs we consider, sparse $\beta^*$ and with data generated with student-$t(5)$ variables. The sample size is $T=200$ while the number of regressors is $p\in\{200, 1000\}$.}
	\label{app.tab.rates_heavy}
\end{table}

\begin{table}[h!]\setlength\extrarowheight{-4pt}
	\begin{center}
		\begin{tabular}{lccccccc}
			& & p/T = 400/400 & & & & p/T = 2000/400 & \\
			\hline
			$m$&   $\alpha =0.1$ &$\alpha = 0.05$ &$\alpha =0.01$ && $\alpha =0.1$ &$\alpha =0.05$ &$\alpha =0.01$ \\
			\hline\\[-0.3cm]
			\multicolumn{8}{c}{{\it Design 1}: $s=\rho_f=\rho_{u}=\rho_e=0$}\\
			0.0 & 0.044 & 0.016 & 0.001 & & 0.037 & 0.010 & 0.001 \\
			0.1  & 0.056 & 0.024 & 0.004 & & 0.040 & 0.012 & 0.002  \\
			0.2 & 0.364 & 0.259 & 0.121 & & 0.118 & 0.055 & 0.015  \\
			0.3 & 0.880 & 0.822 & 0.632  & & 0.416 & 0.301 & 0.136 \\
			0.4 & 0.991 & 0.981 & 0.922 & & 0.761 & 0.643 & 0.418  \\
			\multicolumn{8}{c}{{\it Design 2}: $s=0.1$, $\rho_f=0.6$, $\rho_{u}=0.1$ and $\rho_e=0$}\\
			0.0 & 0.042 & 0.015 & 0.001 & & 0.031 & 0.009 & 0.002  \\
			0.1 & 0.054 & 0.021 & 0.004 & & 0.034 & 0.010 & 0.002 \\
			0.2 & 0.372 & 0.272 & 0.139 & & 0.104 & 0.049 & 0.014  \\
			0.3 & 0.890 & 0.829 & 0.641  & & 0.420 & 0.305 & 0.127 \\
			0.4 & 0.989 & 0.974 & 0.921  & & 0.760 & 0.657 & 0.410 \\
			\multicolumn{8}{c}{{\it Design 3}: $s=0.1$, $\rho_f=0.6$ and $\rho_{u}=\rho_e=0.1$}\\
			0.0 & 0.050 & 0.013 & 0.001  & & 0.039 & 0.011 & 0.002 \\
			0.1 & 0.056 & 0.021 & 0.004 & & 0.040 & 0.012 & 0.001  \\
			0.2 & 0.374 & 0.268 & 0.142  & & 0.114 & 0.054 & 0.014 \\
			0.3 & 0.883 & 0.820 & 0.633  & & 0.414 & 0.302 & 0.128 \\
			0.4 & 0.989 & 0.974 & 0.919 & & 0.753 & 0.656 & 0.404  \\ 
			\hline\hline
		\end{tabular}
	\end{center}
	\caption{Rejection probabilities for the three dependence designs we consider, sparse $\beta^*$ and with data generated with student-$t(5)$ variables. The sample size is $T=400$ while the number of regressors is $p\in\{400, 2000\}$.}
	\label{app.tab.rates_heavy2}
\end{table}


\begin{table}[h!]\setlength\extrarowheight{-4pt}
	\begin{center}
		\begin{tabular}{lccccccc}
			& & p/T = 200/200 & & & & p/T = 1000/200 & \\
			\hline
			$m$&   $\alpha =0.1$ &$\alpha = 0.05$ &$\alpha =0.01$ && $\alpha =0.1$ &$\alpha =0.05$ &$\alpha =0.01$ \\
			\hline\\[-0.3cm]
			\multicolumn{8}{c}{{\it Design 1}: $s=\rho_f=\rho_{u}=\rho_e=0$}\\
			0.0 & 0.084 & 0.036 & 0.006 & & 0.094 & 0.040 & 0.007  \\
			0.1  & 0.126 & 0.063 & 0.016  & & 0.106 & 0.052 & 0.011 \\
			0.2 & 0.594 & 0.494 & 0.322  & & 0.270 & 0.176 & 0.080 \\
			0.3 & 0.982 & 0.961 & 0.907  & & 0.669 & 0.580 & 0.391 \\
			0.4 & 1.000 & 1.000 & 0.999 & & 0.944 & 0.908 & 0.799  \\
			\multicolumn{8}{c}{{\it Design 2}: $s=0.1$, $\rho_f=0.6$, $\rho_{u}=0.1$ and $\rho_e=0$}\\
			0.0 & 0.092 & 0.046 & 0.008 & & 0.091 & 0.044 & 0.005  \\
			0.1 & 0.120 & 0.066 & 0.016 & & 0.111 & 0.055 & 0.009 \\
			0.2 & 0.609 & 0.518 & 0.348 & & 0.268 & 0.180 & 0.072  \\
			0.3 & 0.983 & 0.968 & 0.920 & & 0.704 & 0.613 & 0.408  \\
			0.4 & 1.000 & 1.000 & 0.999 & & 0.950 & 0.922 & 0.838  \\
			\multicolumn{8}{c}{{\it Design 3}: $s=0.1$, $\rho_f=0.6$ and $\rho_{u}=\rho_e=0.1$}\\
			0.0 & 0.091 & 0.041 & 0.006  & & 0.091 & 0.039 & 0.001 \\
			0.1 & 0.115 & 0.063 & 0.014 & & 0.109 & 0.052 & 0.005  \\
			0.2 & 0.607 & 0.517 & 0.347 & & 0.267 & 0.178 & 0.067  \\
			0.3 & 0.980 & 0.964 & 0.918  & & 0.700 & 0.611 & 0.405 \\
			0.4 & 0.997 & 0.999 & 0.997 & & 0.948 & 0.922 & 0.835  \\ 
			\hline\hline
		\end{tabular}
	\end{center}
	\caption{Rejection probabilities for the three dependence designs we consider, sparse $\beta^*$ and using $K = 5$ as the number of factors while the true number of factors is $K=2$. The data are generated with Gaussian variables. The sample size is $T=200$ while the number of regressors is $p\in\{200, 1000\}$.}
	\label{app.tab.rates_k5}
\end{table}

\begin{table}[h!]\setlength\extrarowheight{-4pt}
	\begin{center}
		\begin{tabular}{lccccccc}
			& & p/T = 400/400 & & & & p/T = 2000/400 & \\
			\hline
			$m$&   $\alpha =0.1$ &$\alpha = 0.05$ &$\alpha =0.01$ && $\alpha =0.1$ &$\alpha =0.05$ &$\alpha =0.01$ \\
			\hline\\[-0.3cm]
			\multicolumn{8}{c}{{\it Design 1}: $s=\rho_f=\rho_{u}=\rho_e=0$}\\
			0.0 & 0.079 & 0.028 & 0.002 & & 0.083 & 0.031 & 0.005  \\
			0.1 & 0.094 & 0.042 & 0.004  & & 0.083 & 0.034 & 0.005 \\
			0.2 & 0.451 & 0.362 & 0.212  & & 0.184 & 0.106 & 0.030 \\
			0.3 & 0.948 & 0.923 & 0.845 & & 0.548 & 0.456 & 0.268 \\
			0.4  & 1.000 & 1.000 & 0.998 & & 0.895 & 0.847 & 0.723  \\
			\multicolumn{8}{c}{{\it Design 2}: $s=0.1$, $\rho_f=0.6$, $\rho_{u}=0.1$ and $\rho_e=0$}\\
			0.0 & 0.080 & 0.035 & 0.008 & & 0.084 & 0.029 & 0.006  \\
			0.1 & 0.102 & 0.044 & 0.013 & & 0.088 & 0.032 & 0.005  \\
			0.2 & 0.467 & 0.374 & 0.233 & & 0.180 & 0.106 & 0.030  \\
			0.3 & 0.956 & 0.936 & 0.872  & & 0.545 & 0.442 & 0.262 \\
			0.4 & 1.000 & 0.999 & 0.998 & & 0.898 & 0.846 & 0.711  \\
			\multicolumn{8}{c}{{\it Design 3}: $s=0.1$, $\rho_f=0.6$ and $\rho_{u}=\rho_e=0.1$}\\
			0.0 & 0.077 & 0.034 & 0.007 & & 0.084 & 0.029 & 0.004  \\
			0.1 & 0.102 & 0.040 & 0.011 & & 0.088 & 0.032 & 0.003  \\
			0.2 & 0.465 & 0.371 & 0.231 & & 0.176 & 0.103 & 0.029 \\
			0.3 & 0.956 & 0.933 & 0.870  & & 0.540 & 0.438 & 0.259 \\
			0.4 & 0.995 & 0.996 & 0.996 & & 0.896 & 0.842 & 0.711  \\ 
			\hline\hline
		\end{tabular}
	\end{center}
	\caption{Rejection probabilities for the three dependence designs we consider, sparse $\beta^*$ and using $K = 5$ as the number of factors while the true number of factors is $K=2$. The data are generated with Gaussian variables. The sample size is $T=400$ while the number of regressors is $p\in\{400, 2000\}$.}
	\label{app.tab.rates2_k5}
\end{table}


\begin{table}[h!]\setlength\extrarowheight{-4pt}
	\begin{center}
		\begin{tabular}{lccccccc}
			& & p/T = 200/200 & & & & p/T = 1000/200 & \\
			\hline
			$m$&   $\alpha =0.1$ &$\alpha = 0.05$ &$\alpha =0.01$ && $\alpha =0.1$ &$\alpha =0.05$ &$\alpha =0.01$ \\
			\hline\\[-0.3cm]
			\multicolumn{8}{c}{{\it Design 1}: $s=\rho_f=\rho_{u}=\rho_e=0$}\\
			0.0 & 0.893 & 0.875 & 0.837  & & 0.843 & 0.812 & 0.743 \\
			0.1 & 0.895 & 0.876 & 0.837 & & 0.840 & 0.813 & 0.739  \\
			0.2 & 0.929 & 0.906 & 0.858 & & 0.853 & 0.824 & 0.740  \\
			0.3 & 0.987 & 0.973 & 0.931  & & 0.907 & 0.876 & 0.780 \\
			0.4  & 0.999 & 0.998 & 0.984 & & 0.969 & 0.934 & 0.847 \\
			\multicolumn{8}{c}{{\it Design 2}: $s=0.1$, $\rho_f=0.6$, $\rho_{u}=0.1$ and $\rho_e=0$}\\
			0.0 & 0.929 & 0.910 & 0.881  & & 0.877 & 0.856 & 0.800 \\
			0.1  & 0.928 & 0.913 & 0.879 & & 0.877 & 0.851 & 0.799  \\
			0.2 & 0.952 & 0.930 & 0.891  & & 0.882 & 0.858 & 0.810 \\
			0.3 & 0.986 & 0.969 & 0.928  & & 0.923 & 0.890 & 0.826 \\
			0.4 & 0.997 & 0.994 & 0.974  & & 0.964 & 0.936 & 0.858 \\
			\multicolumn{8}{c}{{\it Design 3}: $s=0.1$, $\rho_f=0.6$ and $\rho_{u}=\rho_e=0.1$}\\
			0.0 & 0.925 & 0.912 & 0.879 && 0.884 & 0.858 & 0.804   \\
			0.1 & 0.929 & 0.915 & 0.885 & & 0.879 & 0.850 & 0.804  \\
			0.2 & 0.951 & 0.925 & 0.892 & & 0.888 & 0.855 & 0.805  \\
			0.3 & 0.984 & 0.971 & 0.930 & & 0.918 & 0.885 & 0.821 \\
			0.4 & 0.997 & 0.994 & 0.976 & & 0.964 & 0.934 & 0.856   \\ 
			\hline\hline
		\end{tabular}
	\end{center}
	\caption{Rejection probabilities for the three dependence designs we consider, sparse $\beta^*$ and using $K = 1$ as the number of factors while the true number of factors is $K=2$. The data are generated  with Gaussian variables. The sample size is $T=200$ while the number of regressors is $p\in\{200, 1000\}$.}
	\label{app.tab.rates_k1}
\end{table}

\begin{table}[h!]\setlength\extrarowheight{-4pt}
	\begin{center}
		\begin{tabular}{lccccccc}
			& & p/T = 400/400 & & & & p/T = 2000/400 & \\
			\hline
			$m$&   $\alpha =0.1$ &$\alpha = 0.05$ &$\alpha =0.01$ && $\alpha =0.1$ &$\alpha =0.05$ &$\alpha =0.01$ \\
			\hline\\[-0.3cm]
			\multicolumn{8}{c}{{\it Design 1}: $s=\rho_f=\rho_{u}=\rho_e=0$}\\
			0.0 & 0.914 & 0.898 & 0.859 & & 0.867 & 0.839 & 0.777  \\
			0.1 & 0.917 & 0.895 & 0.859  & & 0.866 & 0.838 & 0.780 \\
			0.2 & 0.934 & 0.911 & 0.863  & & 0.868 & 0.843 & 0.780 \\
			0.3 & 0.973 & 0.959 & 0.906 & & 0.894 & 0.859 & 0.789  \\
			0.4 & 0.998 & 0.994 & 0.971 & & 0.943 & 0.909 & 0.824  \\
			\multicolumn{8}{c}{{\it Design 2}: $s=0.1$, $\rho_f=0.6$, $\rho_{u}=0.1$ and $\rho_e=0$}\\
			0.0 & 0.931 & 0.919 & 0.892 & & 0.890 & 0.869 & 0.829  \\
			0.1 & 0.935 & 0.921 & 0.892  & & 0.889 & 0.871 & 0.818 \\
			0.2 & 0.936 & 0.924 & 0.892 & & 0.889 & 0.869 & 0.820  \\
			0.3 & 0.978 & 0.961 & 0.919  & & 0.914 & 0.884 & 0.820 \\
			0.4 & 0.997 & 0.994 & 0.972 & & 0.946 & 0.914 & 0.840  \\
			\multicolumn{8}{c}{{\it Design 3}: $s=0.1$, $\rho_f=0.6$ and $\rho_{u}=\rho_e=0.1$}\\
			0.0 & 0.930 & 0.919 & 0.892 & & 0.890 & 0.866 & 0.829  \\
			0.1 & 0.933 & 0.922 & 0.891 & & 0.889 & 0.865 & 0.819  \\
			0.2 & 0.940 & 0.923 & 0.892 & & 0.891 & 0.869 & 0.815 \\
			0.3 & 0.977 & 0.961 & 0.920 & & 0.914 & 0.884 & 0.821  \\
			0.4 & 0.998 & 0.993 & 0.972 & & 0.945 & 0.914 & 0.838  \\ 
			\hline\hline
		\end{tabular}
	\end{center}
	\caption{Rejection probabilities for the three dependence designs we consider, sparse $\beta^*$ and using $K = 1$ as the number of factors while the true number of factors is $K=2$. The data are generated  with Gaussian variables. The sample size is $T=400$ while the number of regressors is $p\in\{400, 2000\}$.}
	\label{app.tab.rates2_k1}
\end{table}	
	
\begin{table}[h!]\setlength\extrarowheight{-4pt}
	\begin{center}
		\begin{tabular}{lccccccc}
			& & p/T = 200/200 & & & & p/T = 1000/200 & \\
			\hline
			$m$&   $\alpha =0.1$ &$\alpha = 0.05$ &$\alpha =0.01$ && $\alpha =0.1$ &$\alpha =0.05$ &$\alpha =0.01$ \\
			\hline\\[-0.3cm]
			\multicolumn{8}{c}{{\it Design 1}: $s=\rho_f=\rho_{u}=\rho_e=0$}\\
			0.0 &  0.084 & 0.038 & 0.007 &  &  0.073 & 0.032 & 0.003  \\
			0.1 &  0.115 & 0.063 & 0.015 &  &   0.078 & 0.035 & 0.004  \\
			0.2 & 0.537 & 0.442 & 0.290 &  &  0.216 & 0.135 & 0.052  \\
			0.3 &  0.973 & 0.958 & 0.896  &  &   0.633 & 0.530 & 0.346 \\
			0.4 &  1.000 & 1.000 & 0.999  &  &  0.933 & 0.902 & 0.794  \\
			\multicolumn{8}{c}{{\it Design 2}: $s=0.1$, $\rho_f=0.6$, $\rho_{u}=0.1$ and $\rho_e=0$}\\
			0.0 &  0.074 & 0.036 & 0.006 &  &  0.070 & 0.027 & 0.004  \\
			0.1  &  0.097 & 0.052 & 0.010 &  &  0.084 & 0.034 & 0.006 \\
			0.2 &  0.565 & 0.468 & 0.288 &  &  0.222 & 0.138 & 0.052  \\
			0.3 &  0.968 & 0.953 & 0.898 &  &  0.654 & 0.540 & 0.360  \\
			0.4 &  0.999 & 0.999 & 0.999 &  &  0.943 & 0.907 & 0.794  \\
			\multicolumn{8}{c}{{\it Design 3}: $s=0.1$, $\rho_f=0.6$ and $\rho_{u}=\rho_e=0.1$}\\
			0.0 &  0.089 & 0.042 & 0.010   &  &  0.084 & 0.030 & 0.004  \\
			0.1 &  0.112 & 0.059 & 0.014  &  &  0.093 & 0.038 & 0.008  \\
			0.2 &  0.571 & 0.460 & 0.282 &  &  0.226 & 0.135 & 0.055 \\
			0.3 &  0.969 & 0.951 & 0.895  &  &  0.647 & 0.535 & 0.355 \\
			0.4 &  0.999 & 0.999 & 0.999   &  &  0.935 & 0.897 & 0.792  \\
			\hline\hline
		\end{tabular}
	\end{center}
	\caption{Rejection probabilities for the three dependence designs we consider, sparse $\beta^*$ and lagged idiosyncratic terms. The data are generated with Gaussian variables. The sample size is $T=200$ while the number of regressors is $p\in\{200, 1000\}$.}
	\label{app.tab.rates_laggedu}
\end{table}

\begin{table}[h!]\setlength\extrarowheight{-4pt}
	\begin{center}
		\begin{tabular}{lccccccc}
			& & p/T = 400/400 & & & & p/T = 2000/400 & \\
			\hline
			$m$&   $\alpha =0.1$ &$\alpha = 0.05$ &$\alpha =0.01$ && $\alpha =0.1$ &$\alpha =0.05$ &$\alpha =0.01$ \\
			\hline\\[-0.3cm]
			\multicolumn{8}{c}{{\it Design 1}: $s=\rho_f=\rho_{u}=\rho_e=0$}\\
			0.0 &  0.070 & 0.028 & 0.005  &  &  0.061 & 0.027 & 0.005  \\
			0.1 &  0.081 & 0.038 & 0.011 &  & 0.064 & 0.028 & 0.004 \\
			0.2 & 0.400 & 0.320 & 0.184  &  &  0.127 & 0.066 & 0.016 \\
			0.3 &  0.932 & 0.903 & 0.823  &  &  0.467 & 0.373 & 0.206 \\
			0.4 &  1.000 & 0.999 & 0.995  &  &  0.862 & 0.812 & 0.664 \\
			\multicolumn{8}{c}{{\it Design 2}: $s=0.1$, $\rho_f=0.6$, $\rho_{u}=0.1$ and $\rho_e=0$}\\
			0.0 &  0.078 & 0.036 & 0.006  & &  0.059 & 0.020 & 0.001 \\
			0.1  &  0.088 & 0.042 & 0.008  & &  0.062 & 0.024 & 0.002 \\
			0.2 &  0.418 & 0.320 & 0.184  & & 0.130 & 0.070 & 0.021 \\
			0.3 &  0.936 & 0.905 & 0.823 & & 0.491 & 0.388 & 0.236   \\
			0.4 &  0.999 & 0.998 & 0.995  & & 0.878 & 0.826 & 0.693   \\
			\multicolumn{8}{c}{{\it Design 3}: $s=0.1$, $\rho_f=0.6$ and $\rho_{u}=\rho_e=0.1$}\\
			0.0 &  0.091 & 0.043 & 0.006 & &  0.068 & 0.028 & 0.002  \\
			0.1 &  0.106 & 0.048 & 0.007  & &  0.067 & 0.029 & 0.003  \\
			0.2 &  0.420 & 0.322 & 0.173 & & 0.137 & 0.077 & 0.024 \\
			0.3 &  0.931 & 0.904 & 0.812  & &  0.490 & 0.387 & 0.230 \\
			0.4 &  1.000 & 0.998 & 0.995  & &  0.875 & 0.823 & 0.682  \\
			\hline\hline
		\end{tabular}
	\end{center}
	\caption{Rejection probabilities for the three dependence designs we consider, sparse $\beta^*$, and lagged idiosyncratic terms. The data are generated with Gaussian variables. The sample size is $T=400$ while the number of regressors is $p\in\{400, 2000\}$.}
	\label{app.tab.rates2_laggedu}
\end{table}

\clearpage
	
\section{Additional empirical results}
		
\subsection{Macro application — FRED-QD}

\begin{table}[ht]
	\centering
	\begin{tabular}{rccc}
		Category & 10\% & 5\% & 1\% \\ 
		\hline
		NIPA (23)  & 0.478 & 0.435 & 0.261 \\
		Industrial Production (16) & 0.625 & 0.500 & 0.125 \\
		Employment and Unemployment (48)  & 0.688 & 0.562 & 0.375 \\
		Housing (10) & 0.400 & 0.200 & 0.200 \\
		Inventories, Orders, and Sales (7) & 0.286 & 0.286 & 0.286 \\
		Prices (36) & 0.417 & 0.306 & 0.194 \\
		Earnings and Productivity (13)  & 0.692 & 0.692 & 0.692 \\
		Interest Rates (15) & 0.400 & 0.400 & 0.133 \\
		Money and Credit (14) & 0.571 & 0.571 & 0.500 \\
		Household Balance Sheets (8)  & 0.250 & 0.125 &	0.125 \\
		Stock Markets (1) & 0.000 & 0.000 & 0.000 \\
		Exchange Rates (6) & 0.833 & 0.833 & 0.333 \\
		Other (2) & 0.000 & 0.000 & 0.000 \\
		Non-Household Balance Sheets (2) & 0.500 & 0.500 & 0.000 \\
		\hline\hline
	\end{tabular}
	\caption{Rejection ratios for each category of the FRED-QD dataset over the sample period 1959 Q3 to 2019 Q4. The sample size is $T=241$ while the number of regressors is $p=202$. In parentheses, we report the number of series per category. Data source: \cite{mccracken2021fred}. \label{app.tab.fredqdreject}}
\end{table}


\subsection{Finance I: additional details on data and results}\label{app.sec.fin1}

\begin{longtable}[ht]{rlll}
		& Variable name & Full name & Source \\ 
		\hline
		1 & Market & Market & \cite{dong2022anomalies} \\ 
		2 & Agric & Agriculture & \cite{french2024kenneth} \\ 
		3 & Food & Food Products & \cite{french2024kenneth} \\ 
		4 & Soda & Candy \& Soda & \cite{french2024kenneth} \\ 
		5 & Beer & Beer \& Liquor & \cite{french2024kenneth} \\ 
		6 & Smoke & Tobacco Products & \cite{french2024kenneth} \\ 
		7 & Toys & Recreation & \cite{french2024kenneth} \\ 
		8 & Fun & Entertainment & \cite{french2024kenneth} \\ 
		9 & Books & Printing and Publishing & \cite{french2024kenneth} \\ 
		10 & Hshld & Consumer Goods & \cite{french2024kenneth} \\ 
		11 & Clths & Apparel & \cite{french2024kenneth} \\ 
		12 & Hlth & Healthcare & \cite{french2024kenneth} \\ 
		13 & MedEq & Medical Equipment & \cite{french2024kenneth} \\ 
		14 & Drugs & Pharmaceutical Products & \cite{french2024kenneth} \\ 
		15 & Chems & Chemicals & \cite{french2024kenneth} \\ 
		16 & Rubbr & Rubber and Plastic Products & \cite{french2024kenneth} \\ 
		17 & Txtls & Textiles & \cite{french2024kenneth} \\ 
		18 & BldMt & Construction Materials & \cite{french2024kenneth} \\ 
		19 & Cnstr & Construction & \cite{french2024kenneth} \\ 
		20 & Steel & Steel Works Etc & \cite{french2024kenneth} \\ 
		21 & FabPr & Fabricated Products & \cite{french2024kenneth} \\ 
		22 & Mach & Machinery & \cite{french2024kenneth} \\ 
		23 & ElcEq & Electrical Equipment & \cite{french2024kenneth} \\ 
		24 & Autos & Automobiles and Trucks & \cite{french2024kenneth} \\ 
		25 & Aero & Aircraft & \cite{french2024kenneth} \\ 
		26 & Ships & Shipbuilding, Railroad Equipment & \cite{french2024kenneth} \\ 
		27 & Guns & Defense & \cite{french2024kenneth} \\ 
		28 & Gold & Precious Metals & \cite{french2024kenneth} \\ 
		29 & Mines & Non-Metallic and Industrial Metal Mining & \cite{french2024kenneth} \\ 
		30 & Coal & Coal & \cite{french2024kenneth} \\ 
		31 & Oil & Petroleum and Natural Gas & \cite{french2024kenneth} \\ 
		32 & Util & Utilities & \cite{french2024kenneth} \\ 
		33 & Telcm & Communication & \cite{french2024kenneth} \\ 
		34 & PerSv & Personal Services & \cite{french2024kenneth} \\ 
		35 & BusSv & Business Services & \cite{french2024kenneth} \\ 
		36 & Hardw & Computers & \cite{french2024kenneth} \\ 
		37 & Softw & Computer Software & \cite{french2024kenneth} \\ 
		38 & Chips & Electronic Equipment & \cite{french2024kenneth} \\ 
		39 & LabEq & Measuring and Control Equipment & \cite{french2024kenneth} \\ 
		40 & Paper & Business Supplies & \cite{french2024kenneth} \\ 
		41 & Boxes & Shipping Containers & \cite{french2024kenneth} \\ 
		42 & Trans & Transportation & \cite{french2024kenneth} \\ 
		43 & Whlsl & Wholesale & \cite{french2024kenneth} \\ 
		44 & Rtail & Retail & \cite{french2024kenneth} \\ 
		45 & Meals & Restaurants, Hotels, Motels & \cite{french2024kenneth} \\ 
		46 & Banks & Banking & \cite{french2024kenneth} \\ 
		47 & Insur & Insurance & \cite{french2024kenneth} \\ 
		48 & RlEst & Real Estate & \cite{french2024kenneth} \\ 
		49 & Fin & Trading & \cite{french2024kenneth} \\ 
		50 & Other & Other& \cite{french2024kenneth} \\ 
		\hline\hline
		\caption{A list of return portfolios for the results in reported in Table \ref{P-tab:finapp1}. We use average value weighted return. \label{app.tab:finapp1.target}}
\end{longtable}

\begin{table}[ht!]
	\centering\footnotesize
	\begin{tabular}{cc|cc|cc|cc|cc}
		\textbf{NoDur}$^{**}$ & 0.013 & \textbf{Durbl}$^{***}$ & 0.004 & \textbf{Manuf}$^{***}$ & 0.001 & \textbf{Enrgy}$^{**}$ & 0.016 & HiTec &
		0.533 \\
		Telcm & 0.625 & \textbf{Shops}$^{**}$ & 0.049 & Hlth & 0.265 & Utils & 0.252 & \textbf{Other}$^{***}$ &
		0.004 \\
		\hline\hline
	\end{tabular}
	\caption{P-values of industry excess returns (10 industry classification, see Table \ref{tab:finapp1.target2}) regressed on 100 characteristics. Bold entries with $*$, $**$, and $***$ indicate significance at the 10\%, 5\% and 1\% levels, respectively. \label{app.tab:finapp1}}
\end{table}

\begin{longtable}[ht]{rlll}
	& Variable name & Full name & Source \\ 
	\hline
	1 & NoDur & Consumer Nondurables & \cite{french2024kenneth} \\ 
	2 & Durbl & Consumer Durables & \cite{french2024kenneth} \\ 
	3 & Manuf & Manufacturing & \cite{french2024kenneth} \\ 
	4 &  Enrgy & Oil, Gas, and Coal & \cite{french2024kenneth} \\ 
	5 &  HiTec & Computers, Software, and Business/Electronic Equipment& \cite{french2024kenneth} \\ 
	6 &  Telcm & Telephone and Television Transmission & \cite{french2024kenneth} \\ 
	7 &  Shops & Wholesale, Retail, and Some Services  & \cite{french2024kenneth} \\ 
	8 &  Hlth & Healthcare, Medical Equipment, and Drugs & \cite{french2024kenneth} \\ 
	9 &  Utils & Utilities & \cite{french2024kenneth} \\ 
	10 & Other  & Other & \cite{french2024kenneth} \\ 
	\hline\hline
	\caption{A list of return portfolios for the results in reported in Table \ref{app.tab:finapp1}. We use average value weighted return. \label{tab:finapp1.target2}}
\end{longtable}



\subsection{Finance II: additional details on data}

 %monthly close-to-close excess returns from a cross-section of firms available for a period  November 1991 and runs until December 2022
 
 \begin{longtable}[ht]{rlll}
 	& Variable name & Factor & Source \\ 
 	\hline
 	1 & MKT  & Market & \cite{french2024kenneth} \\ 
	2 & SMB  & Small-minus-Big & \cite{french2024kenneth} \\ 
	3 & HML  & High-minus-Low & \cite{french2024kenneth} \\ 
	4 & CMA  & Conservative-minus-Aggressive & \cite{french2024kenneth} \\ 
	5 & RMW  & Robust-minus-Weak & \cite{french2024kenneth} \\ 
	6 & ni\_me& Earnings/price ratio & \cite{jensen2023there}\\
	7 & fcf\_me& Cash-flow/price ratio & \cite{jensen2023there}\\
	8 & div12m\_me& Dividend/price ratio & \cite{jensen2023there}\\
	9 & taccruals\_at & Accruals & \cite{jensen2023there}\\
	10 & beta\_60m & 60 Month CAPM Beta & \cite{jensen2023there}\\
	11 & turnover\_126d & Share turnover & \cite{jensen2023there}\\
	12 & rmax5\_rvol\_21d & Max Return to Volatility  & \cite{jensen2023there}\\
	13 & ivol\_ff3\_21d & Idiosync. vol. from the Fama-French 3-factor model  & \cite{jensen2023there}\\
	14 & ret\_3\_1 & Momentum 1-3 Months & \cite{jensen2023there}\\
	15 & ret\_60\_12 & Momentum 12-60 Months & \cite{jensen2023there}\\
 	\hline\hline
 	\caption{A list of observable factors used in Section \ref{P-finance-application-ii}.  \label{tab:finapp2.obs}}
 \end{longtable}


 \begin{longtable}[ht]{llllc}
		& 10 \% & 5 \% & 1 \% & Number of firms \\ 
		\hline
		Agric & 0.000 & 0.000 & 0.000 & 2 \\
		Food & 0.000 & 0.000 & 0.000 & 19 \\
		Soda & 0.000 & 0.000 & 0.000 & 4 \\
		Beer & 0.333 & 0.167 & 0.000 & 6 \\
		Smoke & 0.000 & 0.000 & 0.000 & 1 \\
		Toys & 0.000 & 0.000 & 0.000 & 5 \\
		Fun & 0.000 & 0.000 & 0.000 & 2 \\
		Books & 0.375 & 0.125 & 0.125 & 8 \\
		Hshld & 0.105 & 0.053 & 0.000 & 19 \\
		Clths & 0.000 & 0.000 & 0.000 & 9 \\
		Hlth & 0.100 & 0.100 & 0.100 & 10 \\
		MedEq & 0.053 & 0.000 & 0.000 & 19 \\
		Drugs & 0.053 & 0.053 & 0.000 & 19 \\
		Chems & 0.133 & 0.067 & 0.000 & 15 \\
		Rubbr & 0.000 & 0.000 & 0.000 & 3 \\
		Txtls & 0.000 & 0.000 & 0.000 & 5 \\
		BldMt & 0.000 & 0.000 & 0.000 & 15 \\
		Cnstr & 0.250 & 0.167 & 0.083 & 12 \\
		Steel & 0.222 & 0.111 & 0.111 & 9 \\
		FabPr & 0.000 & 0.000 & 0.000 & 1 \\
		Mach & 0.163 & 0.116 & 0.047 & 43 \\
		ElcEq & 0.167 & 0.167 & 0.083 & 12 \\
		Autos & 0.000 & 0.000 & 0.000 & 14 \\
		Aero & 0.083 & 0.083 & 0.000 & 12 \\
		Ships & 0.000 & 0.000 & 0.000 & 1 \\
		Guns & 0.000 & 0.000 & 0.000 & 4 \\
		Gold & 1.000 & 1.000 & 0.833 & 6 \\
		Mines & 0.000 & 0.000 & 0.000 & 2 \\ 
		Oil & 0.286 & 0.286 & 0.036 & 28 \\ 
		Util & 0.154 & 0.077 & 0.019 & 52 \\ 
		Telcm & 0.455 & 0.455 & 0.364 & 11 \\ 
		PerSv & 0.000 & 0.000 & 0.000 & 5 \\ 
		BusSv & 0.065 & 0.032 & 0.000 & 31 \\ 
		Hardw & 0.000 & 0.000 & 0.000 & 12 \\ 
		Softw & 0.100 & 0.050 & 0.000 & 20 \\
		Chips & 0.206 & 0.059 & 0.029 & 34 \\
		LabEq & 0.211 & 0.211 & 0.105 & 19 \\
		Paper & 0.000 & 0.000 & 0.000 & 11 \\
		Boxes & 0.000 & 0.000 & 0.000 & 3 \\
		Trans & 0.050 & 0.000 & 0.000 & 20 \\
		Whlsl & 0.179 & 0.143 & 0.071 & 28 \\
		Rtail & 0.138 & 0.069 & 0.069 & 29 \\
		Meals & 0.000 & 0.000 & 0.000 & 8 \\
		Banks & 0.096 & 0.027 & 0.014 & 73 \\
		Insur & 0.175 & 0.050 & 0.050 & 40 \\
		RlEst & 0.000 & 0.000 & 0.000 & 5 \\
		Fin & 0.077 & 0.077 & 0.000 & 13 \\
		Other & 0.000 & 0.000 & 0.000 & 2 \\
		\hline\hline
 	\caption{Rejection ratios for each industry of the returns of firms in the dataset of the application in Section  \ref{P-finance-application-ii} over the sample period January 1991 to December 2022, thus $T=384$ and $p=727$. In the Column \textit{Number of firms }, we report the number of firms per industry in our sample. Industry classification is based on 49 industry portfolios of \cite{french2024kenneth}, see Table \ref{app.tab:finapp1.target} for a full list of industries. Data source: \cite{jensen2023there}.  \label{tab:finapp2.res.ind}}
\end{longtable}

\section{On Theorem \ref{P-th}}\label{app.proof}This section contains material related to Theorem \ref{P-th}. In Section \ref{subsec.prem}, we define some useful mathematical objects. The proof of Theorem \ref{P-th} is given in Section \ref{subsec.proof} and makes use of results proved in later sections. Section \ref{subsec.distr} contains some auxiliary lemmas on distribution functions of random variables used in the proof of Theorem \ref{P-th}. Then, in Section \ref{subsec.prob}, we state and prove some lemmas on the probability of some events. Section \ref{subsec.seq}, contains results on some sequences introduced in the proof of Theorem \ref{P-th}.  Furthermore, Section \ref{subsec.fac} introduces results on the factors, the loadings and their estimators. Finally, Section \ref{subsec.pre} recalls pre-existing results on strong mixing sequences and high-dimensional Gaussian vectors. Finally, Section \ref{sec.rate-cond} discusses the rate condition in statement \eqref{P-thii} of Theorem \ref{P-th}. Our proofs borrow ideas and results from \cite{chernozukhov2013gaussian}, \cite{chernozhukov2015comparison}, \cite{lederer2021estimating}, \cite{fan2023latent} and \cite{fan2021bridging}. 

\subsection{Preliminaries}\label{subsec.prem}
We introduce some concepts which are latter useful in proving Theorem \ref{P-th}. 
	
First, we define (infeasible) estimators using the true number of factors.  These are all denoted using a ``check'' symbol. We let the columns of $\overline{F}/\sqrt{T}$ be the eigenvectors corresponding to the leading $K$ eigenvalues of $XX^\top$ and $\overline{B}=X^\top\overline{F} (\overline{F}^\top \overline{F})^{-1}=T^{-1} X^\top\overline{F}.$ Let also $\overline{P}=T^{-1} \overline{F}\left(\overline{F}^\top \overline{F}/T\right)^{-1}\overline{F}^\top=T^{-1} \overline{F}\overline{F}^\top$ be the projector on the vector space generated by the columns of $\overline{F}$. The estimator of $U$ is $\overline{U}=X-\overline{F}\overline{B}^\top =\left(I_T-\overline{P}\right) X$.  Similarly, we let $\overline{Y}=\left(I_T-\overline{P}\right)Y$ be the estimator of $Y-F\gamma^*$. The estimated loading $\overline{b}_{jk}$ corresponds to the $j^{th}$ element of the $k^{th}$ column of $\overline{B}$. Let also $\overline{b}_j=(\overline{b}_{j1},\dots,\overline{b}_{jK})^\top$. The second-step LASSO estimator using the true number of factors is then given by $$
\overline{\beta}_\lambda = \argmin_{\beta\in\R^p}\frac{1}{T}\left\|\overline{Y}-\overline{U}\beta\right\|_2^2+\lambda\|\beta\|_1,$$

For $t\in [T]$, we denote by $\overline{y}_t$ the $t^{th}$ element of $\overline{Y}$ and $\overline{u}_t$ as the $T\times 1$ vector corresponding to the $t^{th}$ row of $\overline{U}$.
For a given $\lambda>0$, let $\overline{\varepsilon}_{\lambda,t}=\overline{y}_t -\overline{u}_t^\top\overline{\beta}_{\lambda},\ t\in[T]$. The equivalent of $\widehat{Q}(\lambda,e)$ is then 
$$\overline{Q}(\lambda,e)=\left\|\frac2T \sum_{t=1}^T\overline{u}_t\overline{\varepsilon}_{\lambda,t} e_t,\right\|_\infty.$$
The estimator $\overline{q}_\alpha(\lambda)$ of $q_\alpha$ is the $(1-\alpha)$-quantile of the distribution of $\overline{Q}(\lambda,e)$ given $X$ and $Y$. Formally, $\overline{q}_\alpha(\lambda)=\inf\left\{q:\P_e(\overline{Q}(\lambda,e)\le q)\ge 1-\alpha\right\}$ (recall that $\P_e(\cdot)=\P(\cdot|X,Y)$). The analog of $\widehat{\lambda}_\alpha$ is given by
$$\label{choicelambda} \overline{\lambda}_\alpha =\inf\{\lambda>0\ :\  \overline{q}_\alpha(\lambda')\le \lambda' \text{ for all } \lambda'\ge \lambda\}.$$
Remark that, on the event $\{\widehat{K}=K\}$, we have $\widehat{F}=\overline{F}$, $\widehat{B}=\overline{B}$, $\widehat{U}=\overline{U}$, $\widetilde{Y}=\overline{Y}$, $\widehat{\beta}_\lambda=\overline{\beta}_\lambda$ and $\widehat{\lambda}_\alpha=\overline{\lambda}_\alpha$.
	
	
	
As in  \cite{lederer2021estimating}, we re-scale some quantities by multiplying them with $\sqrt{T}/2$. This re-scaling is convenient to apply some probabilistic results. For instance, we let $\overline{\Pi}(\mu,e)=\left\|\overline{W}(\mu,e)\right\|_\infty,$
where $$\overline{W}(\mu,e) =\left(\overline{W}_1(\mu,e), \dots, \overline{W}_p(\mu,e)\right)^\top,\ \text{with } \overline{W}_j(\mu,e)= \frac{1}{\sqrt{T}}\sum_{t=1}^T\overline{u}_{tj} \overline{\varepsilon}_{\frac{2}{\sqrt{T}}\mu,t} e_t .$$
Note that $\overline{\Pi}(\mu,e)=\frac{\sqrt{T}}{2}\overline{Q}(\lambda,e),$ for $\lambda =\frac{2}{\sqrt{T}}\mu.$
Similarly, for $\alpha\in(0,1)$, we define
\begin{align*}\overline{\pi}_\alpha(\mu)&=\inf\{q:\P_e(\overline{\Pi}(\mu,e)\le q)\ge 1-\alpha\};\\
	\overline{\mu}_\alpha &=\inf\{\mu>0\ :\  \overline{\pi}_\alpha(\mu')\le \mu' \text{ for all } \mu'\ge \mu\},	
\end{align*}
where $\overline{\mu}_\alpha =\frac{\sqrt{T}}{2}\overline{\lambda}_\alpha$. 
	
Next, to be able to compare $\overline{\Pi}(e)$ with population analogs, we define several additional quantities. Let $\Pi(e)=\left\|W(e)\right\|_\infty$, where 
$$W(e)=\left(W_1(e),\dots,W_p(e)\right)^\top,\ \text{with } W_j(e)=\frac{1}{\sqrt{T}}\sum_{t=1}^Tu_{tj} \varepsilon_te_t$$
and let $\mu_\alpha$ be the $(1-\alpha)-$quantile of $\Pi(e)$ conditionally on $(F,U,\mathcal{E})$.  Formally, $\mu_\alpha=\inf\{q:\P^*_e(\Pi(e)\le q)\ge 1-\alpha\}$, where $\P_e^*(\cdot)=\P(\cdot|F,U,\mathcal{E})$.
	
Moreover, we define $\Pi^*= \left\|W^*\right\|_\infty$, where $$W^*=\left(W_1^*,\dots,W_p^*\right)^\top,\ \text{with } W_j^*=\frac{1}{\sqrt{T}}\sum_{t=1}^Tu_{tj} \varepsilon_t,$$where $\mu_\alpha^*$ is the $(1-\alpha)$ quantile of $\Pi^*$. Finally, we also set $\Pi^G=\left\|G\right\|_\infty$ with $G$ a Gaussian vector with same covariance structure as $W^*$ and let $\mu_{\alpha}^G$ be the $(1-\alpha)$-quantile of $\Pi^G$. Auxiliary lemmas concerning the distributions of $\Pi(e)$, $\Pi^*$ and $\Pi^G$ can be found in Section \ref{subsec.distr}.
	
We also introduce the following useful quantities $$\Delta= \left\|\frac1T \sum_{t=1}^T u_tu_t^\top\varepsilon_t^2-\E\left[u_tu_t^\top\varepsilon_t^2\right]  \right\|_\infty\text{, }R(\mu,e)=\frac{1}{\sqrt{T}}\left\|\overline{W}(\mu,e)-W(e)\right\|_\infty,$$ for $\mu>0$ and the event $\mathcal{S}_{\mu}=\left\{\frac{2}{T}\left\|\overline{U}^\top (\overline{Y}-\overline{U}\beta^*)\right\|_\infty\le {\frac{2}{\sqrt{T}}\mu}\right\}.$ The above terms and events are controlled in Section \ref{subsec.prob}.
	
	The following sequences allow to bound some important terms in the proofs.
	\begin{align*}
		s_T^{(1)} &= \sqrt{\log(T)}\frac{p^{4/q}}{\sqrt{T}};\\
		s_T^{(2)} &= \sqrt{\log(T)}\left(\frac{1}{p}+\frac{p^{\frac2q}}{T}  +\frac{p^{\frac2q-\frac12}}{\sqrt{T}}\right)(\|\varphi^*\|_2\vee 1);\\
		s_T^{(3)} &= \sqrt{\log(T)}\left(1+\frac{p^{\frac{4}{q}}}{T} +\frac1p\right);\\
		s_T^{(4)}&= \sqrt{\log(T)} \left(\left(p^{\frac{4}{q}}T^{\frac{2}{q}-1}+\frac{T^{\frac{2}{q}}}{p}\right)+ \left(\frac1T+\frac{1}{p}\right)\left\|\varphi^*\right\|_2^2\right);\\
		s_T^{(5)} &=\sqrt{\log(T)} \frac{p^{\frac2q}}{\sqrt{T}};\\
		s_T^{(6)}&=\frac{2}{T^{1/4}}\sqrt{ \log(Tp)2\|\beta^*\|_1s_T^{(3)}};\\
		s_T^{(7)}&=\sqrt{\log(Tp)\frac{s_T^{(4)}}{T}};\\
		s_T^{(8)}&=K_1\left(\Delta\right)^{1/3}\left(1\vee 2\log(2p)\vee \log\left(1/\Delta\right)\right)^{1/3}\log(2p)^{1/3};\\
		s_T^{(9)}&= K_2\Big[\left(T^{\kappa-1/2}+T^{1-\frac{c}{2}(1-\frac{2}{h})}\right)\log(T)\log(p)+T^{-1/4}\log (p)^{3/2} \log(T)+ p^{\frac{2}{h}} T^{\frac{1}{h}-\frac12} \log(p)^2 \log(T)\\
		&\quad +\left( p\log(p)^{\frac{3}{2}h-4} \log(T)\log(Tp)\right)^{\frac{1}{h-2}}T^{-\frac14} +T^{\frac12-c\kappa} \left(p^{\frac{1}{h+1}} \sqrt{1\vee \log(p)}\right)\Big];\\
		s_T^{(10)}&=\frac{1}{T\vee p}+s_T^{(9)};\\
		s_T^{(11)}&= \kappa_2\left(\sqrt{2\log(2p)}+\sqrt{2\log(T\vee p)}\right);\\
		s_T^{(12)}&= s_T^{(6)}\left(1+s_T^{(11)}\right)+\frac{\sqrt{T}}{2}s_T^{(2)}+s_T^{(6)}\left(1+(1+s_T^{(6)})s_T^{(11)}+\frac{\sqrt{T}}{2}s_T^{(2)}\right)+s_T^{(7)};\\
		s_T^{(13)}&=s_T^{(6)}\left(1+s_T^{(11)}\right)+s_T^{(7)};\\
		s_T^{(14)}&=s_T^{(9)}+K_3s_T^{(12)}\sqrt{1\vee \log\left(2p/s_T^{(12)}\right)} +s_T^{(8)}+\frac3T,
	\end{align*}
	where $K_1,K_2$ and $K_3$ are constants introduced in Lemmas \ref{prop.LV2}, \ref{prop.LV1} and \ref{prop.LV3}, respectively. The constant $\kappa_2$ is defined in Assumption \ref{P-as.tail} and $h$ and $q$ are introduced in Assumptions \ref{P-as.tail} \eqref{P-tailiii} and \ref{P-as.mixing}. In Lemma \ref{lm.seq}, we show that these sequences all go to $0$ under Assumption \ref{P-as.Rates}. 
	
	Finally, we introduce the following events
	\begin{align*}
		\mathcal{S}_T^{(1)}&=\left\{\Delta \le s_T^{(1)}\right\};\\
		\mathcal{S}_T^{(2)} &=\left\{\left\|\frac{\overline{U}^\top (\overline{Y}-\overline{U}\beta^*)}{T}- \frac{U^\top\mathcal{E}}{T}\right\|_\infty\le s_T^{(2)}\right\};\\
		\mathcal{S}_T^{(3)}&=\left\{\max_{j\in[p]}\frac{1}{T}\sum_{t=1}^T \overline{u}_{tj}^2\le s_T^{(3)}\right\};\\
		\mathcal{S}_T^{(4)}&=\left\{\max_{j\in[p]}\frac{1}{T} \sum_{t=1}^T\left(\overline{u}_{tj}\widetilde{\varepsilon}_t+\widetilde{f}_t^\top\varphi^*- u_{tj}\varepsilon_t\right)^2\le s_T^{(4)}\right\};\\
		\mathcal{S}_T^{(5)}&= \left\{\left\|\frac{U^\top \mathcal{E}}{T}\right\|_\infty\le s_T^{(5)}\right\},
	\end{align*}
	where $\widetilde{\varepsilon}_t$ denotes the $t^{th}$ element of  $\left(I_T-\overline{P}\right)\mathcal{E}$ and $\widetilde{f}_t$ is the $K\times 1$ vector corresponding to the $t^{th}$ row of $\left(I_T-\overline{P}\right)F$.
	We show that the probabilities of these events go to $1$ with $T$ in Lemma \ref{lm.bound}. 
	
	\subsection{Proof of Theorem \ref{P-th}}\label{subsec.proof}
	\textit{Proof of \eqref{P-thi}.}
	We want to show that when $\beta^*=0$, we have 
	\begin{equation}\label{toshow206i1}\P\left(\left\|\frac{\widehat{U}^\top \widetilde{Y}}{T}\right\|_\infty> \widehat{\lambda}_\alpha\right)\le \alpha+o(1).\end{equation}
	First, we move from quantities depending on $\widehat{K}$ to quantities only depending on $K$. Notice that
	\begin{align*}
		&\P\left(\left\|\frac{\widehat{U}^\top \widetilde{Y}}{T}\right\|_\infty> \widehat{\lambda}_\alpha\right)\\&=\P\left(\left\{\left\|\frac{\widehat{U}^\top \widetilde{Y}}{T}\right\|_\infty> \widehat{\lambda}_\alpha\right\}\cap \left\{\widehat{K}=K\right\}\right)+  \P\left(\left\{\left\|\frac{\widehat{U}^\top \widetilde{Y}}{T}\right\|_\infty> \widehat{\lambda}_\alpha\right\}\cap \left\{\widehat{K}\ne K\right\}\right)\\
		&\le \P\left(\left\{\left\|\frac{\overline{U}^\top \overline{Y}}{T}\right\|_\infty> \overline{\lambda}_\alpha\right\}\cap \left\{\widehat{K}=K\right\}\right)+\P\left(\widehat{K}\ne K\right),
	\end{align*}
	where, we used the fact that, on the event $\{\widehat{K}=K\}$, we have $\widehat{U}=\overline{U}$, $\widetilde{Y}=\overline{Y}$ and $\widehat{\lambda}_\alpha=\overline{\lambda}_\alpha$.
	Now, using the formula $\P(C\cap D)=\P(C)+\P(D)-\P(C\cup D)$ (where $C$ and $D$ are two probabilistic events), we obtain
	\begin{align}
		\notag \P\left(\left\|\frac{\widehat{U}^\top \widetilde{Y}}{T}\right\|_\infty> \widehat{\lambda}_\alpha\right)&\le \P\left(\left\|\frac{\overline{U}^\top \overline{Y}}{T}\right\|_\infty> \overline{\lambda}_\alpha\right)+ \P(\widehat{K}=K) \\
		\notag &\quad - \P\left(\left\{\left\|\frac{\overline{U}^\top \overline{Y}}{T}\right\|_\infty> \overline{\lambda}_\alpha\right\}\cup \left\{\widehat{K}=K\right\}\right)+\P\left(\widehat{K}\ne K\right)\\
		\label{050324} &\le \P\left(\left\|\frac{\overline{U}^\top \overline{Y}}{T}\right\|_\infty> \overline{\lambda}_\alpha\right)+ \P\left(\widehat{K}\ne K\right)\\
		\label{0503242} &\le \P\left(\left\|\frac{\overline{U}^\top \overline{Y}}{T}\right\|_\infty> \overline{\lambda}_\alpha\right)+ o(1),
	\end{align}
	where, in equation \eqref{050324}, we used $\P(C\cup D)\ge \P(C)$ and, in equation \eqref{0503242}, we used Assumption \ref{P-as.nb}. This yields 
	\begin{align}
		\P\left(\left\|\frac{\widehat{U}^\top \widetilde{Y}}{T}\right\|_\infty> \widehat{\lambda}_\alpha\right)  &\le \P\left(\left\|\frac{\overline{U}^\top \overline{Y}}{T}\right\|_\infty>\overline{\lambda}_\alpha\right)+o(1)\\
		\notag &\le \P\left( \left\|\frac{U^\top \mathcal{E}}{T}\right\|_\infty+\left\|\frac{\overline{U}^\top \overline{Y}}{T}-\frac{U^\top \mathcal{E}}{T}\right\|_\infty> \overline{\lambda}_\alpha\right)+o(1)\\
		\notag &\le \P\left(\left\{\left\|\frac{U^\top \mathcal{E}}{T}\right\|_\infty>  \overline{\lambda}_\alpha -s_T^{(2)}\right\}\cap\mathcal{S}_T^{(2)} \right)+ \P\left((\mathcal{S}_T^{(2)})^c \right)+o(1)\\
		\label{toshow206i2} &\le \P\left(\left\|\frac{U^\top \mathcal{E}}{T}\right\|_\infty> \overline{\lambda}_\alpha - s_T^{(2)}\right)+o(1),
	\end{align}
	where in the last line we used Lemma \ref{lm.bound} (ii).
	%Next, by Lemma \ref{prop.LV1}, we have 
	%$$\P\left(\frac1T \|U^\top \mathcal{E}\|_\infty\ge \overline{\lambda}_\alpha - \ets_T^{(1)}\right)\le \P\left(\Pi^G\ge  \overline{\lambda}_\alpha-\ets_T^{(1)}\right)+C_2n^{-C_1}.$$
	
	Let us define
	$$\mathcal{T}_1=\mathcal{S}_{\mu^*_{\alpha+s_T^{(14)}}}\cap \mathcal{S}_T^{(1)}\cap \mathcal{S}_T^{(2)}\cap \mathcal{S}_T^{(3)}\cap\mathcal{S}_T^{(4)}.$$
	Note that, by Lemmas \ref{lm.bound} and \ref{lm.ps}, and the fact that $s_T^{(14)}\to 0$ by Lemma \ref{lm.seq} \eqref{lsiv}, \eqref{lsv} and \eqref{lsvi},  the event $\mathcal{T}_1$ has probability going to $1-\alpha$. 
	
	Hence, by \eqref{toshow206i2}, to show \eqref{toshow206i1}, it suffices to prove that,  on $\mathcal{T}_1$, we have \begin{equation}\label{toshow206i31}\overline{\lambda}_\alpha\ge \frac{2}{\sqrt{T}}\mu_{\alpha+s_T^{(14)}}^*+s_T^{(2)}.\end{equation}
	Indeed, in this case, we would have 
	\begin{align*}\P\left(\left\|\frac{\overline{U}^\top \overline{Y}}{T}\right\|_\infty> \overline{\lambda}_\alpha\right)&\le \P\left(\left\|\frac{U^\top \mathcal{E}}{T}\right\|_\infty>  \overline{\lambda}_\alpha - s_T^{(2)}\right)+o(1)\\
		&\le \P\left(\left\{\left\|\frac{U^\top \mathcal{E}}{T}\right\|_\infty>  \overline{\lambda}_\alpha - s_T^{(2)}\right\}\cap \mathcal{T}_1\right)+\P(\mathcal{T}_1^c)+o(1)\\
		&\le \P\left(\left\{\left\|\frac{U^\top \mathcal{E}}{T}\right\|_\infty>  \frac{2}{\sqrt{T}}\mu_{\alpha+s_T^{(14)}}^*\right\}\cap \mathcal{T}_1\right) +\P(\mathcal{T}_1^c)+o(1)\\
		&=0+\P(\mathcal{T}_1^c)+o(1) = \alpha+o(1),\end{align*}
	where, on the last line, we used $\mathcal{S}_{\mu^*_{\alpha+s_T^{(14)}}}\subset \mathcal{T}_1$. 
	
	
	Let us therefore prove that, on $\mathcal{T}_1$, \eqref{toshow206i31} holds. To do so, we show that, on $\mathcal{T}_1$,
	\begin{equation}\label{toshow206i4}\P_e\left(\overline{\Pi}(\mu,e)> \mu\right) >\alpha \end{equation}
	for $\mu=(1+s_T^{(6)})\mu_{\alpha+s_T^{(14)}}^*+\frac{\sqrt{T}}{2}s_T^{(2)}>\mu_{\alpha+s_T^{(14)}}^*+\frac{\sqrt{T}}{2}s_T^{(2)}$, which implies that \eqref{toshow206i31} is true by definition of $\overline{\lambda}_\alpha$ and the fact that  $\overline{\mu}_\alpha =\frac{\sqrt{T}}{2}\overline{\lambda}_\alpha$.
	We have 
	\begin{align*}
		\P_e\left(\overline{\Pi}(\mu,e)> \mu\right)&\ge \P_e\left(\Pi(e)-R(\mu,e)> \mu\right)\\
		&\ge \P_e\left(\Pi(e)-R(\mu,e)> \mu, R(\mu,e)\le s_T^{(6)}\sqrt{\mu}+s_T^{(7)}\right)\\
		&\ge  \P_e\left(\Pi(e)> \mu+s_T^{(6)}\sqrt{\mu}+s_T^{(7)}\right)-\P_e\left( R(\mu,e)>s_T^{(6)}\sqrt{\mu}+s_T^{(7)}\right)\\
		&\ge \P_e\left(\Pi(e)> \mu+s_T^{(6)}\sqrt{\mu}+s_T^{(7)}\right)-\frac2T,
	\end{align*}
	where, on the last line, we used Lemma \ref{lm.lassodiff} and the facts that $ \mu\ge \mu_{\alpha+s_T^{(14)}}^*$ and we work on $\mathcal{S}^{(3)}_T\cap\mathcal{S}^{(4)}_T\cap \mathcal{S}_{\mu^*_{\alpha+s_T^{(14)}}}\subset \mathcal{T}_1$ to obtain that $\P_e\left( R(\mu,e)>s_T^{(6)}\sqrt{\mu}+s_T^{(7)}\right)\le \frac2T$.
	By Lemma \ref{prop.LV2}, we obtain
	\begin{equation}\label{eq2061}\P_e(\Pi(e)> \mu+s_T^{(6)}\sqrt{\mu}+s_T^{(7)})\ge \P(\Pi^G\ge\mu+s_T^{(6)}\sqrt{\mu}+s_T^{(7)}) -s_T^{(8)}-\frac2T.\end{equation}
	Since $\sqrt{\mu}\le (1+\mu)$, for $T$ large enough, it holds that
	\begin{align}
		\notag &\P\left((\Pi^G> \mu+s_T^{(6)}\sqrt{\mu}+s_T^{(7)}\right)\\
		\notag &\ge \P\left(\Pi^G> \mu+s_T^{(6)}(1+\mu)+s_T^{(7)}\right)\\
		\notag &\ge \P\left(\Pi^G> (1+s_T^{(6)})\mu_{\alpha+s_T^{(14)}}^*+\frac{\sqrt{T}}{2}s_T^{(2)}+s_T^{(6)}\left(1+(1+s_T^{(6)})\mu_{\alpha+s_T^{(14)}}^*+\frac{\sqrt{T}}{2}s_T^{(2)}\right)+s_T^{(7)}\right)\\
		\label{B4147}&\ge  \P\left(\Pi^G> \mu_{\alpha+s_T^{(14)}}^*+s_T^{(12)}\right)\\
		\notag&\ge \P\left(\Pi^G>\mu_{\alpha+s_T^{(14)}}^*)- \P(|\Pi^G-\mu_{\alpha+s_T^{(14)}}^*|\le s_T^{(12)}\right)\\
		\label{B3147}&\ge \P\left(\Pi^G>\mu_{\alpha+s_T^{(14)}}^*\right)-K_3s_T^{(12)}\sqrt{1\vee \log\left(2p/s_T^{(12)}\right)}\\
		\label{B1147}&\ge  \P\left(\Pi^*>\mu_{\alpha+s_T^{(14)}}^*\right)-s_T^{(9)}-K_3s_T^{(12)}\sqrt{1\vee \log\left(2p/s_T^{(12)}\right)}\\
		\notag &= \alpha+s_T^{(14)} -s_T^{(9)}-K_3s_T^{(12)}\sqrt{1\vee \log\left(2p/s_T^{(12)}\right)},
	\end{align}
	where, in \eqref{B4147}, we used  Lemma \ref{lm.G} and the fact that $s_T^{(14)}\to 0$ by Lemma \ref{lm.seq} \eqref{lsiv}, \eqref{lsv} and \eqref{lsvi}, to obtain that $\mu_{\alpha+s_T^{(14)}}^*\le s_T^{(11)}$ for $T$ large enough, in \eqref{B3147}, we leveraged Lemma \ref{prop.LV3} and \eqref{B1147} follows from Lemma \ref{prop.LV1}.
	This and \eqref{eq2061}, therefore yield
	$$\P_e\left(\Pi(e)> \mu\right) \ge  \alpha+s_T^{(14)} -s_T^{(9)}-K_3s_T^{(12)}\sqrt{1\vee \log\left(2p/s_T^{(12)}\right)} -s_T^{(8)}-\frac2T=\alpha+\frac1T>\alpha,$$
	by definition of $s_T^{(14)}$. This shows \eqref{toshow206i4} and therefore concludes the proof of \eqref{P-thi}. \\
	
	\noindent \textit{Proof of (ii).}
	We want to show that if $\sqrt{\frac{\log(T\vee p)}{T\wedge p}}=o_P\left(\left\|\frac{U^\top U\beta^*}{T}\right\|_\infty\right)$, we have 
	\begin{equation}\label{toshow206ii1}\P\left(\left\|\frac{\widehat{U}^\top \widehat{Y}}{T}\right\|_\infty> \widehat{\lambda}_\alpha\right)\to 1.\end{equation}
	As in the proof of (i), we first move from quantities depending on $\widehat{K}$ to quantities only depending on $K$. Notice that
	\begin{align}
		\notag &\P\left(\left\|\frac{\widehat{U}^\top \widetilde{Y}}{T}\right\|_\infty> \widehat{\lambda}_\alpha\right)\\
		\notag &\ge \P\left(\left\{\left\|\frac{\widehat{U}^\top \widehat{Y}}{T}\right\|_\infty> \widehat{\lambda}_\alpha\right\}\cap \left\{\widehat{K}=K\right\}\right)\\
		\notag &= \P\left(\left\{\left\|\frac{\overline{U}^\top \overline{Y}}{T}\right\|_\infty> \overline{\lambda}_\alpha\right\}\cap \left\{\widehat{K}=K\right\}\right)\\
		\label{0503243} &= \P\left(\left\|\frac{\overline{U}^\top \overline{Y}}{T}\right\|_\infty> \overline{\lambda}_\alpha\right) -  \P\left(\left\{\left\|\frac{\overline{U}^\top \overline{Y}}{T}\right\|_\infty> \overline{\lambda}_\alpha\right\}\cap \left\{\widehat{K}\ne K\right\}\right)\\
		\notag &\ge \P\left(\left\|\frac{\overline{U}^\top \overline{Y}}{T}\right\|_\infty> \overline{\lambda}_\alpha\right) -  \P\left(\widehat{K}\ne K\right)\\
		\label{0503245}&\ge \P\left(\left\|\frac{\overline{U}^\top \overline{Y}}{T}\right\|_\infty> \overline{\lambda}_\alpha\right)+o(1),
	\end{align}
	where, in \eqref{0503243}, we used the formula $\P(C)=\P(C\cap D)+\P(C\cap D^c)$, and \eqref{0503245} follows from Assumption \ref{P-as.nb}.
	Hence, it holds that
	\begin{align}
		\notag &\P\left(\left\|\frac{\widehat{U}^\top \widetilde{Y}}{T}\right\|_\infty> \widehat{\lambda}_\alpha\right)\\
		\notag &\ge  \P\left(\left\|\frac{\overline{U}^\top \overline{Y}}{T}\right\|_\infty> \overline{\lambda}_\alpha\right)\\
		\notag &\ge \P\left(\left\|\frac{U^\top U\beta^*}{T}\right\|_\infty-  \left\|\frac{U^\top \mathcal{E}}{T}\right\|_\infty-\left\|\frac{\overline{U}^\top (\overline{Y}-\overline{U}\beta^*)}{T}-\frac{U^\top \mathcal{E}}{T}\right\|_\infty> \overline{\lambda}_\alpha\right)\\
		\notag&\ge \P\left(\left\{\left\|\frac{U^\top U\beta^*}{T}\right\|_\infty>  \overline{\lambda}_\alpha + s_T^{(2)}+s_T^{(5)}\right\}\cap\mathcal{S}_T^{(2)}\cap\mathcal{S}_T^{(5)} \right)\\
		\notag& \ge \P\left(\left\|\frac{U^\top U\beta^*}{T}\right\|_\infty>  \overline{\lambda}_\alpha + s_T^{(2)}+s_T^{(5)}\right)- \P\left(\left(\mathcal{S}_T^{(2)}\cap\mathcal{S}_T^{(5)}\right)^c\right)\\
		\label{toshow206ii2} &= \P\left(\left\|\frac{U^\top U\beta^*}{T}\right\|_\infty>  \overline{\lambda}_\alpha + s_T^{(2)}+s_T^{(5)}\right)+o(1),
	\end{align}
	where, in the last line, we used Lemma \ref{lm.bound}. 
	
	Let us define 
	$$\mathcal{T}_2=\mathcal{S}_{\mu^*_{2s_T^{(10)}}}\cap \mathcal{S}_T^{(1)}\cap \mathcal{S}_T^{(2)}\cap \mathcal{S}_T^{(3)}\cap\mathcal{S}_T^{(4)}\cap \mathcal{S}_T^{(5)}.$$Note that, by Lemmas \ref{lm.bound} and \ref{lm.ps}, and the fact that $s_T^{(10)}=\frac{1}{T\vee p}+s_T^{(9)}\to 0$ by Lemma \ref{lm.seq} \eqref{lsv},  the event $\mathcal{T}_2$ has probability going to $1$. 
	
	Hence, by \eqref{toshow206ii2}, to show \eqref{toshow206ii1}, it suffices to prove that,  on $\mathcal{T}_2$, we have \begin{equation}\label{toshow206ii3}\overline{\lambda}_\alpha\le \frac{2}{\sqrt{T}}\mu_{2s_T^{(10)}}^*,\end{equation}
	for $T$ large enough.
	Indeed, in this case, we would have 
	\begin{align*}\P\left(\left\|\frac{\overline{U}^\top \overline{Y}}{T}\right\|_\infty\ge \overline{\lambda}_\alpha\right)&\ge  \P\left(\left\|\frac{U^\top U\beta^*}{T}\right\|_\infty>  \overline{\lambda}_\alpha + s_T^{(2)}+s_T^{(5)}\right)+o(1),\\
		&\ge \P\left(\left\{\left\|\frac{U^\top U\beta^*}{T}\right\|_\infty> \frac{2}{\sqrt{T}}\mu_{2s_T^{(10)}}^* + s_T^{(2)}+s_T^{(5)}\right\}\cap \mathcal{T}_2\right)+o(1)\\
		&\ge \P\left(\left\{\left\|\frac{U^\top U\beta^*}{T}\right\|_\infty>  \frac{2}{\sqrt{T}}s_T^{(11)} + s_T^{(2)}+s_T^{(5)}\right\}\cap \mathcal{T}_2\right)+o(1)\\
		&\ge \P\left(\left\|\frac{U^\top U\beta^*}{T}\right\|_\infty>  \frac{2}{\sqrt{T}}s_T^{(11)} + s_T^{(2)}+s_T^{(5)}\right)-\P(\mathcal{T}_2^c) +o(1)
		\to 1 ,\end{align*}
	where, in the third line, we used $\mu_{2s_T^{(10)}}^*\le s_T^{(11)} $ by Lemma \ref{lm.G} and, in the last line, we leveraged the facts $\frac{2}{\sqrt{T}}s_T^{(11)} + s_T^{(2)}+s_T^{(5)}=O_P\left(\sqrt{\frac{\log(T\vee p)}{T\wedge p}}\right)$ by Lemma \ref{lm.seq} \eqref{lsii} and that $\sqrt{\frac{\log(T\vee p)}{T\wedge p}}=o_P\left(\left\|\frac{U^\top U\beta^*}{T}\right\|_\infty\right)$ to obtain that $ \P\left(\left\|\frac{U^\top U\beta^*}{T}\right\|_\infty>  \frac{2}{\sqrt{T}}s_T^{(11)} + s_T^{(2)}+s_T^{(5)}\right)\to 1$.
	
	Let us therefore prove that, on $\mathcal{T}_2$, \eqref{toshow206ii3} holds for $T$ large enough. To do so, we show that, on $\mathcal{T}_2$, for $T$ large enough,
	\begin{equation}\label{toshow206ii4}\P_e\left(\overline{\Pi}(\mu_{2s_T^{(10)}}^*,e)> \mu_{2s_T^{(10)}}^*\right) \le \alpha, \end{equation}
	which implies \eqref{toshow206ii3} by definition of $\overline{\lambda}_\alpha$. 
	On $\mathcal{T}_2$, we have \begin{align*}
		\P_e\left(\overline{\Pi}(\mu_{2s_T^{(10)}}^*,e)> \mu_{2s_T^{(10)}}^*\right)&\le \P_e\left(\Pi(e)+R(\mu_{2s_T^{(10)}}^*,e)> \mu\right)\\
		&\le \P_e\left(\Pi(e)> \mu_{2s_T^{(10)}}^* -R(\mu_{2s_T^{(10)}}^*,e), R(\mu_{2s_T^{(10)}}^*,e)\le s_T^{(6)}\sqrt{\mu_{2s_T^{(10)}}^*}+s_T^{(7)}\right)\\
		&\quad +\P_e\left( R(\mu_{2s_T^{(10)}}^*,e)> s_T^{(6)}\sqrt{\mu_{2s_T^{(10)}}^*}+s_T^{(7)}\right)\\
		&\le  \P_e\left(\Pi(e)> \mu_{2s_T^{(10)}}^*-s_T^{(6)}\sqrt{\mu_{2s_T^{(10)}}^*}-s_T^{(7)}\right)+\frac2T,
	\end{align*}
	where, in the last line, we used Lemma \ref{lm.lassodiff}.
	By Lemma \ref{prop.LV2}, we obtain
	\begin{equation}\label{eq2061ii}\P_e\left(\Pi(e)>\mu_{2s_T^{(10)}}^*\right)\le \P\left(\Pi^G\ge \mu_{2s_T^{(10)}}^*-s_T^{(6)}\sqrt{\mu_{2s_T^{(10)}}^*}-s_T^{(7)}\right) +s_T^{(8)}+\frac2T.\end{equation}
	Since $\sqrt{\mu_{2s_T^{(10)}}^*}\le 1+\mu_{2s_T^{(10)}}^*$, it holds that\begin{align}
		\notag &\P\left(\Pi^G> \mu_{2s_T^{(10)}}^*-s_T^{(6)}\sqrt{\mu_{2s_T^{(10)}}^*}-s_T^{(7)}\right)\\
		\notag &\le \P\left(\Pi^G> \mu_{2s_T^{(10)}}^*-s_T^{(6)}(1+\mu_{2s_T^{(10)}}^*)-s_T^{(7)}\right)\\
		\label{B42147} &\le \P\left(\Pi^G>  \mu_{2s_T^{(10)}}^*-s_T^{(6)}(1+s_T^{(11)})-s_T^{(7)}\right)\\
		\notag &=  \P\left(\Pi^G>\mu_{2s_T^{(10)}}^*-s_T^{(13)}\right)\\
		\notag &\le  \P\left(\Pi^G>\mu_{2s_T^{(10)}}^*\right)+\P\left(|\Pi^G-\mu_{2s_T^{(10)}}^*|\le s_T^{(13)}\right)\\
		\label{B32147}&\le  \P\left(\Pi^G>\mu_{2s_T^{(10)}}^*\right)+K_3s_T^{(13)}\sqrt{1\vee \log\left(2p/s_T^{(13)}\right)}\\
		\label{B12147}&\le  \P\left(\Pi^*>\mu_{2s_T^{(10)}}^*\right)+s_T^{(9)}+K_3s_T^{(13)}\sqrt{1\vee \log\left(2p/s_T^{(13)}\right)}\\
		\notag&=2s_T^{(10)}+s_T^{(9)}+K_3s_T^{(13)}\sqrt{1\vee \log\left(2p/s_T^{(13)}\right)},
	\end{align}
	where, in \eqref{B42147},  we used Lemma \ref{lm.G} to obtain that $\mu_{2s_T^{(10)}}^*\le s_T^{(11)} $, in \eqref{B32147}, we leveraged Lemma \ref{prop.LV3} and \eqref{B12147} follows from Lemma \ref{prop.LV1}.
	This and \eqref{eq2061ii}, therefore yield
	$$\P_e\left(\Pi(e)> \mu_{2s_T^{(10)}}^*\right)\le 2s_T^{(10)}+s_T^{(9)}+K_3s_T^{(13)}\sqrt{1\vee \log\left(2p/s_T^{(13)}\right)} +s_T^{(8)}+\frac2T\le \alpha,$$
	for $T$ large enough by Lemma \ref{lm.seq} \eqref{lsiv}, \eqref{lsv}, \eqref{lsvi}.
	This shows that \eqref{toshow206ii4} holds and therefore concludes the proof of \eqref{P-thii}.
	
	\subsection{Auxiliary lemmas on distributions} \label{subsec.distr}
	
	\begin{Lemma}\label{prop.LV1} Under the assumptions of Theorem \ref{P-th}, it holds that
		$$ \sup_{z\in\R}\left|\P(\Pi^*\le z)-\P\left(\Pi^G\le z\right)\right|<s_T^{(9)} .$$
	\end{Lemma}
	\begin{Proof} The result is a direct consequence of Lemma \ref{lm.highdimclt} applied to $Z_t = u_t\varepsilon_t$ (and the constant $K_1$ used in the definition of $s_T^{(9)}$ is introduced in Lemma \ref{lm.highdimclt}). Condition \eqref{lhdi} of Lemma \ref{lm.highdimclt} is satisfied by Lemma \ref{lm.2exp} and Assumption \ref{P-as.tail} \eqref{P-tailiii}. Assumption \ref{P-as.mixing} implies that conditions \eqref{lhdii} and \eqref{lhdiii} hold. Concerning condition \eqref{lhdiv}, note that, by Assumption \ref{P-as.tail} \eqref{P-tailiv}, we have 
		\begin{align*}\E\left[\left(\frac{1}{\sqrt{T}} \sum_{t=1}^T u_t\varepsilon_t\right)\left(\frac{1}{\sqrt{T}} \sum_{t=1}^T u_t\varepsilon_t\right) ^\top \right]&=\frac{1}{T} \sum_{t=1}^T\sum_{s=1}^T\E\left[ u_t\varepsilon_tu_{s}^\top\varepsilon_{s} \right]= \E\left[ u_tu_t^\top\varepsilon_t^2\right].\end{align*}
		By Assumption \ref{P-as.tail} \eqref{P-tailii}, this implies that $$\sigma_{p}\left(\E\left[\left(\frac{1}{\sqrt{T}} \sum_{t=1}^T u_t\varepsilon_t\right)\left(\frac{1}{\sqrt{T}} \sum_{t=1}^T u_t\varepsilon_t\right) ^\top \right]\right)= \sigma_{p}\left(E\left[ u_tu_t^\top\varepsilon_t^2\right]\right)\ge \kappa_1>0,$$
		and therefore that condition \eqref{lhdiv} holds. 
	\end{Proof}
	\begin{Lemma}\label{prop.LV2} Let the assumptions of Theorem \ref{P-th} hold. On the event $\mathcal{S}_T^{(1)}$,
		$$\sup_{z\in\R}\left|\P_e(\Pi(e)\le z)-\P\left(\Pi^G\le z\right)\right|\le s_T^{(8)}.$$
	\end{Lemma}
	\begin{Proof} Conditionally on $U,\mathcal{E}$, $W(e)$ is a centered Gaussian vector with covariance matrix $T^{-1}\sum_{t=1}^T u_tu_t^\top\varepsilon_t^2 $. Moreover, $G$ is a centered Gaussian vector with covariance matrix $$\E\left[\left(\frac{1}{\sqrt{T}} \sum_{t=1}^T u_t\varepsilon_t\right)\left(\frac{1}{\sqrt{T}} \sum_{t=1}^T u_t\varepsilon_t\right) ^\top \right]=\E\left[\varepsilon_t^2u_tu_t^\top\right],$$
		see the proof of Lemma \ref{prop.LV1} for a justification of this equality. %Notice also that, by Assumption \ref{as.tail}, there exists constants $c,C>0$ such that $c<\E[u_{tj}u_{tj}^\top\varepsilon_t^2]<C$ for all $j\in[p].$ 
		Remark that, by Assumption \ref{P-as.tail} \eqref{P-tailii}, $$\kappa_2 > \E\left[u_{tj}u_{tj}^\top\varepsilon_t^2\right]> \kappa_1,$$
		for all $j\in[p]$.
		We can therefore apply Lemma \ref{lm.glv2} to get 
		$$\sup_{z\in\R}\left|\P_e(\Pi(e)\le z)-\P\left(\Pi^G\le z\right)\right|\le \pi(\Delta),$$
		where $\pi(\Delta)=K_1\Delta^{1/3}(1\vee \log(2p)\vee \log(1/\Delta))^{1/3}\log(2p)^{1/3}.$
		
		
	\end{Proof}
	\begin{Lemma}\label{prop.LV3}Under the assumptions of Theorem \ref{P-th}, there exists a constant $K_2>0$ such that, for all $z_1,z_2>0$, we have 
		$$\P\left(\left|\Pi^G-z_1\right|\le z_2\right)\le K_2z_2\sqrt{1\vee \log(2p/z_2)}.$$
	\end{Lemma}
	\begin{Proof}
		This is a direct consequence of Lemma \ref{lm.glv1} of which the conditions are satisfied by Assumption \ref{P-as.tail} (see the proofs of Lemmas \ref{prop.LV1} and \ref{prop.LV2} for more details).
	\end{Proof}
	
	\begin{Lemma}\label{lm.G}For every $\alpha>s_T^{(10)}$, we have
		$$\mu^{*}_\alpha\le s_T^{(11)}.$$
	\end{Lemma}
	\begin{Proof}
		Notice that, by Assumption \ref{P-as.tail} \eqref{P-tailiv},
		\begin{align*}\E\left[(W_j^*)^2\right]=\E\left[\left(\frac{1}{\sqrt{T}}\sum_{t=1}^Tu_{tj} \varepsilon_t\right)^2\right]= \E\left[u_{tj}^2 \varepsilon_t^2\right],
		\end{align*}
		which, by Assumption \ref{P-as.tail} \eqref{P-tailii} and \eqref{P-tailiv}, is bounded uniformly in $j$ and $t$ by $\kappa_2>0$.
		Using  Lemma 7 in \cite{chernozhukov2015comparison} and remark A.8 in \cite{lederer2021estimating}, we have, for every $r>0$, 
		$$\P\left(\|G\|_\infty \ge \E\left[\|G\|_\infty\right]+r\right)\le\exp\left(-\frac{r^2}{2\kappa_2}\right).$$
		Taking $r=\kappa_2\sqrt{2\log(T\vee p)}$, we get
		$$\P\left(\|G/\kappa_2\|_\infty \ge \E\left[\|G/\kappa_2\|_\infty\right]+\sqrt{2\log(T\vee p)}\right)\le\frac{1}{T\vee p}.$$
		By the Gaussian maximal inequality (see e.g., Exercise 2.17 in \cite{boucheron2013concentration}), it holds that $\E\left[\|G/\kappa_2\|_\infty\right]\le \sqrt{2\log(2p)}$, which yields
		$$\P\left(\|G\|_\infty \ge \kappa_2\left(\sqrt{2\log(2p)}+\sqrt{2\log(T\vee p)}\right)\right)\le\frac{1}{T\vee p},$$
		so that $\mu^{G}_\alpha\le \kappa_2 \left(\sqrt{2\log(2p)}+\sqrt{2\log(T\vee p)}\right)$ for $\alpha>1/(T\vee p)$ by definition of $\mu^G_\alpha$.
		Now, for $\alpha>s_T^{(10)}=(T\vee p)^{-1}+s_T^{(9)}$, by Lemma \ref{prop.LV1}, we have 
		$$\P\left(\Pi^*\ge\mu^G_{\alpha -s_T^{(9)}}\right)\le \P\left(\Pi^G\ge \mu^G_{\alpha -s_T^{(9)}}\right)+s_T^{(9)}\le \alpha- s_T^{(9)} +s_T^{(9)}=\alpha .$$
		Hence, we obtain $\mu^*_{\alpha}\le \mu^G_{\alpha -s_T^{(9)}}\le \kappa_2 \left(\sqrt{2\log(2p)}+\sqrt{2\log(T\vee p)}\right).$
	\end{Proof}
	
	
	
	
	
	\subsection{Auxiliary lemmas on probabilistic events} \label{subsec.prob}
	
	\begin{Lemma}\label{lm.bound} Under the assumptions of Theorem \ref{P-th}, it holds that \begin{enumerate}[\textup{(}i\textup{)}]  
			\item\label{lm.boundi}  $\P\left(\mathcal{S}_T^{(1)}\right)\to 1$;
			\item\label{lm.boundii}  $\P\left(\mathcal{S}_T^{(2)}\right)\to 1$;
			\item\label{lm.boundiii}  $\P\left(\mathcal{S}_T^{(3)}\right)\to 1$;
			\item\label{lm.boundiv}   $\P\left(\mathcal{S}_T^{(4)}\right)\to 1$;
			\item\label{lm.boundv}   $\P\left(\mathcal{S}_T^{(5)}\right)\to 1$.
		\end{enumerate}
	\end{Lemma}
	\begin{Proof} \newline
		Result \eqref{lm.boundi} follows directly from Lemma \ref{lm.lf2} \eqref{lf2v}; \eqref{lm.boundii} is a consequence of  Lemma \ref{fancomplex}; \eqref{lm.boundiii} comes from Lemmas \ref{lm.lf} \eqref{lfii} and Lemma \ref{lm.lf2} \eqref{lf2i} and the triangle inequality, \eqref{lm.boundiv} follows from Lemma \ref{20723} and \eqref{lm.boundv} is a direct consequence of Lemma \ref{lm.lf2} \eqref{lf2iii}. 
	\end{Proof}
	
	\begin{Lemma}\label{lm.lassodiff}Let the assumptions of Theorem \ref{P-th} hold. On the event $\mathcal{S}_T^{(3)}\cap \mathcal{S}_T^{(4)}\cap \mathcal{S}_{\mu}$,
		we have, for all $\mu'\ge \mu$,
		$$\P_e\left(R(\mu',e)\ge s_T^{(6)}\sqrt{ \mu'}+s_T^{(7)}\right) \le \frac2T.$$
	\end{Lemma}
	\begin{Proof} Take $\mu'\ge \mu$. Remember that $\overline{Y}=\left(I_T-\overline{P}\right)\left(X\beta+F\varphi^*+ \mathcal{E}\right)$. This yields that
		$$\overline{\varepsilon}_{\frac{2}{\sqrt{T}}\mu',t}=\widetilde{y}_t-\overline{u}_t^\top\overline{\beta}_{\frac{2}{\sqrt{T}}\mu'}=\overline{u}_t\left(\beta^*-\overline{\beta}_{\frac{2}{\sqrt{T}}\mu'}\right)+\widetilde{f}_t^\top\varphi^*+\widetilde{\varepsilon}_t,$$
		where we recall that $\widetilde{\varepsilon}_t$ is the $t^{th}$ element of  $\left(I_T-\overline{P}\right)\mathcal{E}$ and $\widetilde{f}_t$ is the $K\times 1$ vector corresponding to the $t^{th}$ row of $\left(I_T-\overline{P}\right)F$. This yields \begin{align}
			\notag &R(\mu',e)\\
			\notag &=\frac{1}{\sqrt{T}}\left\|\overline{W}(\mu',e)-W(e)\right\|_\infty\\
			\notag  &=\frac{1}{T}\max_{j\in[p]}\left|\sum_{t=1}^T\overline{u}_{tj} \overline{\varepsilon}_{\frac{2}{\sqrt{T}}\mu',t} e_t-  \sum_{t=1}^Tu_{tj} \varepsilon_te_t\right|\\
			\label{decomp6163}&\le \frac{1}{T}\max_{j\in[p]}\left|\sum_{t=1}^T\overline{u}_{tj}\overline{u}_t^\top\left(\beta^*-\overline{\beta}_{\frac{2}{\sqrt{T}}\mu'}\right) e_t \right|+\frac{1}{T}\max_{j\in[p]}\left| \sum_{t=1}^T\left(\overline{u}_{tj}\widetilde{\varepsilon}_t+\widetilde{f}_t^\top\varphi^*- u_{tj}\varepsilon_t\right) e_t\right| .
		\end{align}
		
		
		Now, we bound the two terms in \eqref{decomp6163}. We start with $\max_{j\in[p]}\left|\sum_{t=1}^T\overline{u}_{tj}\overline{u}_t^\top\left(\beta^*-\overline{\beta}_{\frac{2}{\sqrt{T}}\mu'}\right) e_t \right|$.  Remark that given $(F,U,\mathcal{E})$, we have
		\begin{align*}
			\frac{1}{T}\sum_{t=1}^T\overline{u}_{tj}\overline{u}_t^\top\left(\overline{\beta}_\lambda-\beta^*\right)e_t \sim\mathcal{N}\left(0, \frac{1}{T^2}\sum_{t=1}^T\left(\overline{u}_{tj}\overline{u}_{t}^\top\left(\overline{\beta}_\lambda-\beta^*\right)\right)^2\right)  
		\end{align*}					
		By the Gaussian tail bound (equation (2.10) in \cite{vershynin2018high}), for $z>0$, we obtain, for all $j\in[p]$ and $z>0$,
		\begin{equation}\label{0707232}\P_e^*\left(\left| \frac{1}{T}\sum_{t=1}^T\overline{u}_{tj}\overline{u}_t^\top\left(\overline{\beta}_\lambda-\beta^*\right)e_t \right|>z\right) \le 2\exp\left(-\frac{z^2}{\frac{1}{T^2}\sum_{t=1}^T\left(\overline{u}_{tj}\overline{u}^\top_t\left(\overline{\beta}_\lambda-\beta^*\right)\right)^2}\right).\end{equation}
		Next, let $\lambda= {\frac{2}{\sqrt{T}}\mu}$ and $\lambda'= {\frac{2}{\sqrt{T}}\mu'}$. By definition of $\overline{\beta}_{\lambda'}$, it holds that 
		%$$\overline{\beta}_{\lambda'}\in\argmin_{\beta\in\R^{K}}\frac{1}{T}\left\|\overline{Y}-\overline{U}\beta\right\|^2_2+ \lambda' \left\|\beta\right\|_1.$$
		%This yields
		$$\frac{1}{T}\left\|\overline{Y}-\overline{U}\overline{\beta}_{\lambda'}\right\|^2_2+ \lambda' \left\|\overline{\beta}_\lambda\right\|_1\le\frac{1}{T}\left\|\overline{Y}-\overline{U}\beta^*\right\|^2_2+ \lambda \|\beta^*\|_1.$$
		%$$\frac{1}{T}\left\|\overline{Y}-\overline{U}\beta^*+\overline{U}(\beta^*-\overline{\beta}_{\lambda'})\right\|^2_2+ \lambda \|\overline{\beta}_\lambda\|_1\le\frac{1}{T}\left\|\overline{Y}-\overline{U}\beta^*\right\|^2_2+ \lambda \|\beta^*\|_1.$$
		%That is 
		This yields
		\begin{align}\notag &\frac{1}{T}\left\|\overline{U}(\beta^*-\overline{\beta}_{\lambda'})\right\|^2_2\\
			\notag& \le \frac2T \left(\overline{Y}-\overline{U}\beta^*\right)^\top\overline{U}\left(\overline{\beta}_{\lambda'}-\beta^*\right)+ \lambda' \left(\|\beta^*\|_1-\left\|\overline{\beta}_{\lambda'}\right\|_1\right)\\
			\notag &\le \frac2T \left\|\overline{U}^\top \left(\overline{Y}-\overline{U}\beta^*\right)\right\|_\infty\left\|\overline{\beta}_{\lambda'}-\beta^*\right\|_1+ \lambda'\left(\|\beta^*\|_1- \left\|\overline{\beta}_{\lambda'}\right\|_1\right)\\
			\notag &\le \lambda' \|\overline{\beta}_{\lambda'}-\beta^*\|_1+ \lambda' \left(\|\beta^*\|_1-\left\|\overline{\beta}_{\lambda'}\right\|_1\right)\\
			\label{ineqlasso} &\le 2  \lambda'\|\beta^*\|_1.
		\end{align}
		where we used Hölder's inequality and the fact that we work on $\mathcal{S}_\mu$.
		Moreover, we have 
		\begin{equation}\label{0707231}
			\begin{aligned}
				\frac{1}{T^2}\sum_{t=1}^T(\overline{u}_{tj}\overline{u}^\top_t(\overline{\beta}_\lambda-\beta^*))^2&\le \frac{1}{T}\sum_{t=1}^T \overline{u}_{tj}^2 \frac{1}{T}\left\|\overline{U}\left(\beta^*-\overline{\beta}_\lambda\right)\right\|^2_2\\
				&\le  s_T^{(3)}2\lambda'\|\beta^*\|_1,
			\end{aligned}
		\end{equation}
		by \eqref{ineqlasso} and because we work on $\mathcal{S}_T^{(3)}$. Recall that $s_T^{(6)}= 2\sqrt{ \log(Tp)\|\beta^*\|_1s_T^{(3)}T^{-1/2}}$. Using \eqref{0707232}, \eqref{0707231} and the union bound, we get
		\begin{align}\notag &\P_e^*\left(\frac{1}{T}\max_{j\in[p]}\left|\sum_{t=1}^T\overline{u}_{tj}\overline{u}_t^\top\left(\beta^*-\overline{\beta}_{\frac{2}{\sqrt{T}}\mu'}\right) e_t \right|_\infty>s_T^{(6)}\sqrt{\mu'}\right)\\
			\notag &\le p \max_{j\in[p]}\P_e^*\left(\left| \frac{1}{T}\sum_{t=1}^T\overline{u}_{tj}\overline{u}_t^\top\left(\overline{\beta}_\lambda-\beta^*\right)e_t \right|>s_T^{(6)}\sqrt{\mu'}\right)\\
			\label{boundr1}&\le  \exp\left(-\frac{(s_T^{(6)})^2\mu'}{2\lambda'\|\beta^*\|_1s_T^{(3)}} +\log(p)\right)=T^{-1}.
		\end{align}
		
		Let us now bound the term
		$\max_{j\in[p]}\left| \sum_{t=1}^T\left(\overline{u}_{tj}\widetilde{\varepsilon}_t+\widetilde{f}_t^\top\varphi^*- u_{tj}\varepsilon_t\right) e_t\right|$. Conditional on $(F,U,\mathcal{E})$, we have
		\begin{align*}
			\frac1T\sum_{t=1}^T\left(\overline{u}_{tj}\widetilde{\varepsilon}_t+\widetilde{f}_t^\top\varphi^*- u_{tj}\varepsilon_t\right)  e_t &\sim \mathcal{N}\left(0, \frac{1}{T^2}\sum_{t=1}^T\left(\overline{u}_{tj}\widetilde{\varepsilon}_t+\widetilde{f}_t^\top\varphi^*- u_{tj}\varepsilon_t\right)^2\right).
		\end{align*}
		Since we work on $\mathcal{S}_T^{(4)}$, by the Gaussian tail bound, this yields, for all $j\in[p]$ and $z>0$,
		$$\P_e^*\left(\left|\frac1T\sum_{t=1}^T\left(\overline{u}_{tj}\widetilde{\varepsilon}_t+\widetilde{f}_t^\top\varphi^*- u_{tj}\varepsilon_t\right)  e_t\right|>z \right)\le \exp\left(-\frac{Tz^2}{s_T^{(4)}}\right).$$
		Recall that $s_T^{(7)}=\sqrt{\log(Tp) T^{-1}s_T^{(4)}}$. Using the union bound, we get
		\begin{align}\notag &\P_e^*\left(\max_{j\in[p]}\left|\frac1T\sum_{t=1}^T\left(\overline{u}_{tj}\widetilde{\varepsilon}_t+\widetilde{f}_t^\top\varphi^*- u_{tj}\varepsilon_t\right) e_t\right|>s_T^{(7)}\right)\\
			\notag &\le p \max_{j\in[p]}\P_e\left(\left|\frac1T\sum_{t=1}^T\left(\overline{u}_{tj}\widetilde{\varepsilon}_t+\widetilde{f}_t^\top\varphi^*- u_{tj}\varepsilon_t\right) e_t\right|>s_T^{(7)}\right)\\
			\label{boundr2}&\le p\exp\left(-\frac{\left(s_T^{(7)}\right)^2}{T^{-1}s_T^{(4)}}\right)=T^{-1}.
		\end{align}
		Using the pigeonhole principle, \eqref{decomp6163}, \eqref{boundr1} and \eqref{boundr2}, we get 
		$\P_e^*\left(R(\mu',e)\ge s_T^{(6)}\sqrt{ \mu'}+s_T^{(7)}\right) \le 2T^{-1},$
		which yields $\P_e\left(R(\mu',e)\ge s_T^{(6)}\sqrt{ \mu'}+s_T^{(7)}\right) \le 2T^{-1},$ integrating over the distribution of $(F,U,\mathcal{E})$.
		
	\end{Proof}
	
	\begin{Lemma}\label{lm.ps}Under the assumptions of Theorem \ref{P-th}, we have 
		$$\sup_{\alpha'\in(0,1)}\left|\P\left(\mathcal{S}_{\mu_{\alpha'}^*} \right)-(1-\alpha')\right|=o(1).$$
	\end{Lemma}
	\begin{Proof} Let us first bound $\P\left(\mathcal{S}_{\mu_{\alpha'}^*}\right)$ from above. For $\alpha'\in(0,1)$, we have \begin{align}
			\notag \P\left(\mathcal{S}_{\mu_{\alpha'}^*}\right)&=\P\left(2\left\|\frac{\overline{U}^\top (\overline{Y}-\overline{U}\beta^*)}{T}\right\|_\infty\le \frac{2}{\sqrt{T}}\mu_{\alpha'}^*\right)\\
			\notag &\le \P\left(\left\|\frac{U^\top \mathcal{E}}{T}\right\|_\infty-\left\|\frac{\overline{U}^\top (\overline{Y}-\overline{U}\beta^*)}{T}-\frac{U^\top \mathcal{E}}{T}\right\|_\infty\le  \frac{1}{\sqrt{T}}\mu_{\alpha'}^*\right) \\
			\notag &\le \P\left(\left\{\left\|\frac{U^\top \mathcal{E}}{T}\right\|_\infty-\left\|\frac{\overline{U}^\top (\overline{Y}-\overline{U}\beta^*)}{T}-\frac{U^\top \mathcal{E}}{T}\right\|_\infty\le  \frac{1}{\sqrt{T}}\mu_{\alpha'}^*\right\}\cap\mathcal{S}_T^{(2)}\right) + \P\left( \left(\mathcal{S}_T^{(2)}\right)^c\right)
			\\
			\label{ps1} &\le \P\left(\left\|\frac{U^\top \mathcal{E}}{T}\right\|_\infty\le  \frac{1}{\sqrt{T}}\mu_{\alpha'}^*+s_T^{(2)}\right) +\P\left( \left(\mathcal{S}_T^{(2)}\right)^c\right).
		\end{align}
		
		
		Now,  we have 
		\begin{align}\notag \P\left(\left\|\frac{U^\top \mathcal{E}}{T}\right\|_\infty\le  \frac{1}{\sqrt{T}}\mu_{\alpha'}^*+s_T^{(2)}\right)&=\P\left(\Pi^*\le  \mu_{\alpha'}^*+\sqrt{T}s_T^{(2)}\right)
			\\
			\notag & \le \P\left(\Pi^G\le  \mu_{\alpha'}^*+\sqrt{T}s_T^{(2)}\right)+s_T^{(9)}\\
			\notag &\le \P\left(\Pi^G\le  \mu_{\alpha'}^*\right)+\P\left(\left|\Pi^G-  \mu_{\alpha'}^*\right|\le \sqrt{T}s_T^{(2)}\right)+s_T^{(9)}\\
			\notag &\le \P\left(\Pi^*\le  \mu_{\alpha'}^*\right)+\P\left(\left|\Pi^G-  \mu_{\alpha'}^*\right|\le \sqrt{T}s_T^{(2)}\right)+2s_T^{(9)}\\
			\label{ps2}&\le 1-\alpha'+\P\left(\left|\Pi^G-  \mu_{\alpha'}^*\right|\le \sqrt{T}s_T^{(2)}\right)+2s_T^{(9)},
		\end{align}
		where we used Lemma \ref{prop.LV1} in the second and fourth lines.
		By Lemma \ref{prop.LV3}, we have $$\P\left(\left|\Pi^G-  \mu_{\alpha'}^*\right|\le s_T^{(2)}\right)\le K_4\sqrt{T}s_T^{(2)}\sqrt{1\vee \log\left(\frac{2p}{\sqrt{T}s_T^{(2)}}\right)}.$$
		Combining this, \eqref{ps1} and \eqref{ps2}, we get \begin{equation}\label{ps3}\P\left(\mathcal{S}_{\mu_{\alpha'}^*}\right)\le 1-\alpha'+ \P\left( \left(\mathcal{S}_T^{(2)}\right)^c\right)+K_4\sqrt{T}s_T^{(2)}\sqrt{1\vee \log\left(\frac{2p}{\sqrt{T}s_T^{(2)}}\right)}+2s_T^{(9)}.\end{equation}
		
		By a similar reasoning, we can show that \begin{equation}\label{ps4}\P\left(\mathcal{S}_{\mu_{\alpha'}^*}\right)\ge 1-\alpha'- \P\left( \left(\mathcal{S}_T^{(2)}\right)^c\right)-K_4\sqrt{T}s_T^{(2)}\sqrt{1\vee \log\left(\frac{2p}{\sqrt{T}s_T^{(2)}}\right)}-2s_T^{(9)}.\end{equation}
		Since $\sqrt{T}s_T^{(2)}\sqrt{1\vee \log\left(\frac{2p}{\sqrt{T}s_T^{(2)}}\right)}\to 0$, $s_T^{(8)}\to 0$, $s_T^{(9)}\to 0$ by Lemma \ref{lm.seq} \eqref{lsiii}, \eqref{lsiv}, \eqref{lsv} and $\P\left( \left(\mathcal{S}_T^{(2)}\right)^c\right)\to0 $ by  Lemma \ref{lm.bound} \eqref{lm.boundii}, \eqref{ps3} and \eqref{ps4} yield the result
	\end{Proof}
	
	
	\subsection{Auxiliary lemma on sequences}\label{subsec.seq}
	\begin{Lemma} \label{lm.seq} Under Assumption \ref{P-as.Rates}, we have \begin{enumerate}[\textup{(}i\textup{)}]  \item\label{lsi} $s_T^{(12)}\sqrt{1\vee \log\left(2p/s_T^{(12)}\right)}=o(1)$;  \item\label{lsii} $2T^{-1/2}s_T^{(11)}+s_T^{(2)}+s_T^{(5)}=O\left(\sqrt{\frac{\log(T\vee p)}{(T\wedge p)}}p^\frac2q\right)\left(\|\varphi^*\|_2\vee 1\right)$;  \item\label{lsiii} $\sqrt{T}s_T^{(2)}\sqrt{1\vee \log\left(\frac{2p}{\sqrt{T}s_T^{(2)}}\right)}=o(1)$;  \item\label{lsiv} $s_T^{(8)}=o_P(1)$;\item\label{lsv} $s_T^{(9)}=o(1)$; \item\label{lsvi}$s_T^{(13)}\sqrt{1\vee \log\left(2p/s_T^{(13)}\right)}=o(1).$
		\end{enumerate}
	\end{Lemma}
	\begin{Proof}
		\newline \noindent \textit{Proof of \eqref{lsi}.} By Assumption \ref{P-as.Rates} \eqref{P-ratei}, we have $s_T^{(3)}=O\left(\sqrt{\log(T)}\right)$, so that \begin{equation}\label{seq1}s_T^{(6)}=O\left(\left(\frac{\log(T\vee p)^{2}}{\sqrt{T}}\|\beta^*\|_1\right)^{1/2}\right).\end{equation} Since $s_T^{(11)}=O\left(\sqrt{\log(T\vee p)}\right)$, this yields
		\begin{equation}\label{seq2} s_T^{(6)}s_T^{(11)}=O\left(\left(\frac{\log(T\vee p)^{3}}{\sqrt{T}}\|\beta^*\|_1\right)^{1/2}\right).\end{equation}
		We also have \begin{equation}\label{seq3} \left(s_T^{(6)}\right)^2s_T^{(11)}=O\left(\left(\frac{\log(T\vee p)^{3}}{\sqrt{T}}\|\beta^*\|_1\right)^{1/2}\right),\end{equation}
		because $s_T^{(6)}=o(1)$ by \eqref{seq1} and Assumption \ref{P-as.Rates} \eqref{P-rateii}. Next, it holds that 
		\begin{equation}\label{seq4}s_T^{(2)}=O\left(\frac{\sqrt{\log(T)}p^{\frac2q}}{T\wedge p}(\|\varphi^*\|_2\vee1)\right),\end{equation}
		so that \begin{equation}\label{seq5}s_T^{(6)}s_T^{(2)}=o\left(s_T^{(2)}\right),\end{equation}
		since $s_T^{(6)}=o(1)$ by \eqref{seq1} and Assumption \ref{P-as.Rates} \eqref{P-rateii}. Moreover, it holds that $$s_T^{(4)}=O\left(\sqrt{\log(T)}\frac{p^\frac4q T^\frac2q}{T\wedge p}\left(\|\varphi^*\|_2^2\vee 1\right)\right)$$
		and, therefore, \begin{equation}\label{seq6} s_T^{(7)}=O\left(\sqrt{\frac{\log(T\vee p)^2p^\frac4q T^\frac2q}{T(T\wedge p)}}\left(\|\varphi^*\|_2\vee 1\right)\right).\end{equation} Recall that
		$$ s_T^{(12)}= 2s_T^{(6)}+2s_T^{(6)}s_T^{(11)}+\left(s_T^{(6)}\right)^2s_T^{(11)}+\frac{\sqrt{T}}{2}s_T^{(2)}+\frac{\sqrt{T}}{2}s_T^{(6)}s_T^{(2)}+s_T^{(7)}.$$
		By \eqref{seq1}, \eqref{seq2}, \eqref{seq3}, \eqref{seq4}, \eqref{seq5}, \eqref{seq6}, we obtain 
		\begin{align}\notag  s_T^{(12)}&=O\left(\left(\frac{\log(T\vee p)^{3}}{\sqrt{T}}\|\beta^*\|_1\right)^{1/2}+ \left(\frac{\log(T\vee p)^{3/2}\sqrt{T}}{(T\wedge p)}+\sqrt{\frac{\log(T\vee p)^2p^\frac4q T^\frac2q}{T(T\wedge p)}}\right)(\|\varphi^*\|_2\vee1)\right)\\
			\label{seq7}&=O\left(\left(\frac{\log(T\vee p)^{3}}{\sqrt{T}}\|\beta^*\|_1\right)^{1/2}+ \frac{\log(T\vee p)^{3/2}\sqrt{T}}{(T\wedge p)}\left(1+\sqrt{\frac{p^\frac4q T^\frac2q}{T}}\right)(\|\varphi^*\|_2\vee1)\right)  .\end{align}
		Additionally, we have $(T(T\wedge p))^{-1/2}=o_P\left(s_T^{(7)}\right)=O\left(s_T^{(12)}\right)$ so that $\log\left(2p/s_T^{(12)}\right)=O\left(\log(2p)+\log\left(\sqrt{T(T\wedge p)}\right)\right)=O(\log(T\vee p))$. This and \eqref{seq7} imply
		\begin{align*}&s_T^{(12)}\sqrt{1\vee \log\left(2p/s_T^{(12)}\right)}\\
			&=O\left(\left(\frac{\log(T\vee p)^{5}}{\sqrt{T}}\|\beta^*\|_1\right)^{1/2}+\frac{\log(T\vee p)^{5/2}\sqrt{T}}{(T\wedge p)}\left(1+\sqrt{\frac{p^\frac4q T^\frac2q}{T}}\right)(\|\varphi^*\|_2\vee1)\right)=o(1),\end{align*}
		by Assumption \ref{P-as.Rates} and the fact that $q\ge 8$.\\
		
		\noindent \textit{Proof of \eqref{lsii}.} The result follows directly from Assumption \ref{P-as.Rates} and \eqref{seq4}.\\
		
		\noindent \textit{Proof of \eqref{lsiii}.} We have $(T\wedge p)^{-1}=o_P\left(\sqrt{T}s_T^{(2)}\right)$, hence $\log\left(\frac{2p}{\sqrt{T}s_T^{(2)}}\right)=O(\log(T\vee p))$, so that 
		\begin{align*}&\sqrt{T}s_T^{(2)}\sqrt{1\vee \log\left(\frac{2p}{\sqrt{T}s_T^{(2)}}\right)}\\
			&=O\left(\log(T\vee p)^{\frac32}\sqrt{T}\left(\frac{1}{p}+\frac{p^{\frac2q}}{T}  +\frac{p^{\frac2q-\frac12}}{\sqrt{T}}\right)(\|\varphi^*\|_2\vee 1) \right)=o(1)\end{align*}
		by the definition of $s_T^{(2)}$ and Assumption \ref{P-as.Rates}.
		
		\noindent \textit{Proof of \eqref{lsiv}.} We have $\Delta=o_P(1)$ by the facts that $\P(\mathcal{S}_T^{(1)})\to 1$ and that $s_T^{(1)}=o(1)$ by Assumption \ref{P-as.Rates}. This yields the result using that $\lim\limits_{x\to 0^+} x\log(x)\to 0$.\\
		
		\noindent \textit{Proof of \eqref{lsv}.} Recall that
		\begin{align*}
			s_T^{(9)} &=  K_1\Big[\left(T^{\kappa-1/2}+T^{1-\frac{c}{2}(1-\frac{2}{h})}\right)\log(T)\log(p)+T^{-1/4}\log (p)^{3/2} \log(T)\\
			& \quad + p^{\frac{2}{h}} T^{\frac{1}{h}-\frac12} \log(p)^2 \log(T)\\
			&\quad +\left( p\log(p)^{\frac{3}{2}h-4} \log(T)\log(Tp)\right)^{\frac{1}{h-2}}T^{-\frac14} +T^{\frac12-c\kappa} \left(p^{\frac{1}{h+1}} \sqrt{1\vee \log(p)}\right)\Big]\\
			&=O_P\Big( \left(T^{\kappa-1/2}+T^{1-\frac{c}{2}(1-\frac{2}{h})}\right)\log(T)\log(T^{r})+T^{-1/4}\log (T^{r})^{3/2} \log(T)\\
			&\quad+T^{\frac{2r}{h}+\frac{1}{h}-\frac12}  \log(T^r)^2 \log(T)\\
			&\quad +\left(\log(T^r)^{\frac{3}{2}h-4} \log(T)\log(T^{r+1})\right)^{\frac{1}{h-2}}T^{\frac{r}{h-2}-\frac14} +T^{\frac{r}{h+1}+\frac12-c\kappa}  \sqrt{1\vee \log(T^r)}\Big).
		\end{align*}
		By Assumption \ref{P-as.mixing}, we have $\kappa-1/2<0$ and $1-\frac{c}{2}(1-\frac{2}{h})<0$, so that $$\left(T^{\kappa-1/2}+T^{1-\frac{c}{2}(1-\frac{2}{h})}\right)\log(T)\log(T^{r})\to 0.$$ As $r$ is finite, $T^{-1/4}\log (T^{r})^{3/2} \log(T)\to 0$.  Moreover, since $r <\frac{h}{4} -\frac12$, we have $$T^{\frac{2r}{h}+\frac{1}{h}-\frac12}  \log(T^r)^2 \log(T)\to 0$$ and $$\left(\log(T^r)^{\frac{3}{2}h-4} \log(T)\log(T^{r+1})\right)^{\frac{1}{h-2}}T^{\frac{r}{h-2}-\frac14}\to 0$$ Additionally, it holds that $T^{\frac{r}{h+1}+\frac12-c\kappa}  \sqrt{1\vee \log(T^r)}\to 0$ because $r< (h+1)(c\kappa -\frac12)$. All of this yields \eqref{lsv}.\\
		
		\noindent \textit{Proof of \eqref{lsvi}.} The proof is similar to that of \eqref{lsi} and therefore omitted.
	\end{Proof}
	
	
	
	\subsection{Auxiliary lemmas on factors and loadings} \label{subsec.fac}
	In this Section, we prove useful results on the factors, the factor loadings and their estimators. Let $H=T^{-1}V\overline{F}^\top FB^\top B$, where $V$ is the $K\times K$ matrix corresponding the $K$ largest eigenvalues of $T^{-1}XX^\top$. Recall that the estimated loadings are $\overline{B}= X^\top \overline{F}\left(\overline{F}^\top \overline{F}\right)^{-1}=T^{-1}X^\top \overline{F} $ and $\overline{b}_j$ and $b_j$ are the $K\times 1$ vectors corresponding to the $j^{th}$ rows of $\overline{B}$ and $B$, respectively. 
	
	
	
	
	\begin{Lemma}\label{lm.lf2} Under the assumptions of Theorem \ref{P-th}, the following holds:
		\begin{enumerate}[\textup{(}i\textup{)}]  
			\item\label{lf2i}$\max_{j\in[p]}\left|\sum_{t=1}^T u_{tj}^2\right|=O_P(T)$;
			\item\label{lf2ii}$\max_{j\in[p],k\in[K]}\left|\sum_{t=1}^T u_{tj}f_{tk}\right|=O_P\left(p^{\frac{2}{q}}\sqrt{T}\right)$;
			\item\label{lf2iii}$\left\|U^\top \mathcal{E}\right\|_\infty=O_P\left(p^{\frac{2}{q}}\sqrt{T}\right)$;
			\item\label{lf2iv}$\max_{j\in[p],k\in[K]}\left|\sum_{t=1}^T  u_{tj}\left(\sum_{\ell=1}^pu_{t\ell}b_{\ell k}\right)\right|=O_P\left(\sqrt{T}p^{\frac{2}{q}+\frac12}+T\right)$;
			\item\label{lf2v}  $\left\|\frac1T \sum_{t=1}^T u_tu_t^\top\varepsilon_t^2-\E\left[u_tu_t^\top\varepsilon_t^2\right]  \right\|_\infty=O_P\left(\frac{p^{4/q}}{\sqrt{T}}\right);$
			\item\label{lf2vi}$\| \mathcal{E}\|_2=O_P\left(\sqrt{T}\right)$;
			\item\label{lf2vibis}$\| F\|_2=O_P\left(\sqrt{T}\right)$;
			\item\label{lf2vibisii}$\left\| \frac1T F^\top F-I_K\right\|_2=O_P\left(\frac{1}{\sqrt{T}}\right)$;
			\item\label{lf2vii}$\| U\|_2=O_P\left(\sqrt{Tp}\right)$;
			\item\label{lf2viii}$\left\|F^\top \mathcal{E}\right\|_2=O_P\left(\sqrt{T}\right)$;
			\item\label{lf2x} $\|UB\|_2=O_P\left(\sqrt{Tp}\right)$;
			\item\label{lf2ix}$\left\|F^\top U\right\|_2=O_P\left(\sqrt{Tp}\right)$;
			\item\label{lf2ixbis}$\left\|\mathcal{E}^\top U\right\|_2=O_P\left(\sqrt{Tp}\right)$;
			\item\label{lf2xi} $\left\|F^\top UB\right\|_2^2=O_P\left(Tp\right)$;
			\item\label{lf2xii} $\left\|\mathcal{E}^\top UB\right\|_2^2=O_P\left(Tp\right)$.
		\end{enumerate}
	\end{Lemma}
	\begin{Proof} In this proof, we will often apply Lemmas \ref{lm.2exp}, \ref{lm.nagaev}  and \ref{lm.nagaevsup} to some specific processes. Following the arguments of the proof of Lemma \ref{prop.LV1}, it can be checked that the conditions of these Lemmas hold for these processes under the assumptions of Theorem \ref{P-th}.\\
		
		\noindent\textit{Proof of \eqref{lfi}.} We apply Lemmas \ref{lm.2exp} and \ref{lm.nagaevsup} to $Z_t=\left(u_{tj}^2-\E\left[u_{tj}^2\right]\right)_{j=1}^p$ and get
		\begin{equation}\label{187}\max_{j\in[p]}\left|\frac1T\sum_{t=1}^T u_{tj}^2-\E\left[u_{tj}^2\right]\right|=O_P\left(\frac{p^{\frac{2}{q}}}{\sqrt{T}}\right).\end{equation} By the triangle inequality, we obtain
		\begin{align*}\max_{j\in[p]}\left|\sum_{t=1}^T u_{tj}^2\right|&\le T \max_{j\in[p]}\left|\frac1T\sum_{t=1}^T u_{tj}^2-\E[u_{tj}^2]\right| + T \max_{j\in[p]}\E\left[u_{tj}^2\right]\\
			&=O_P\left(T+\sqrt{T}p^{\frac{2}{q}}\right)=O_P(T),\end{align*}
		where we used $\max_{j\in[p]}\E[u_{tj}^2]\le \|\Sigma\|_\infty =O(1)$ by Assumption \ref{P-as.tail} \eqref{P-tailii} and the fact that $p^{\frac{2}{q}}/\sqrt{T}\to 0$ by Assumption \ref{P-as.Rates} \eqref{P-ratei}.\\
		
		\noindent\textit{Proof of \eqref{lf2ii}, \eqref{lf2iii}.} We apply Lemmas \ref{lm.2exp} and \ref{lm.nagaevsup} to 
		\begin{align*}Z_t&=((u_{tj}f_{tk})_{j=1}^p)_{k=1}^K;\\
			Z_t&=(u_{tj}\varepsilon_t)_{j=1}^p,
		\end{align*}
		and obtain \eqref{lf2ii}, \eqref{lf2iii}.  \\
		
		\noindent\textit{Proof of \eqref{lf2iv}.} We apply Lemmas \ref{lm.2exp} and  \ref{lm.nagaevsup} to 
		\begin{align*}
			Z_t&=\left(\left(u_{tj}\left(p^{-1/2}\sum_{\ell=1}^pu_{t\ell}b_{\ell k}\right)-\E\left[u_{tj}\left(p^{-1/2}\sum_{\ell=1}^pu_{t\ell}b_{\ell k}\right)\right]\right)_{j=1}^p\right)_{k=1}^K,
		\end{align*}
		and obtain 
		\begin{equation}\label{2407soir}\max_{j\in[p],k\in[K]}\left| \sum_{t=1}^T\left(u_{tj}\left(p^{-1/2}\sum_{\ell=1}^pu_{t\ell}b_{\ell k}\right)-\E\left[u_{tj}\left(p^{-1/2}\sum_{\ell=1}^pu_{t\ell}b_{\ell k}\right)\right]\right)\right| =O_P\left(p^{\frac{2}{q}}\sqrt{T}\right).\end{equation}
		Next, by Assumption \ref{P-as.tail} \eqref{P-tailv}, we have
		\begin{align}
			\label{2407soir2} \max_{j\in[p],k\in[K]}\left| \sum_{t=1}^T\E\left[u_{tj}\left(\sum_{\ell=1}^pu_{t\ell}b_{\ell k}\right)\right]\right|   =O(T).
		\end{align}
		By the triangle inequality and equations \eqref{2407soir} and \eqref{2407soir2}, we obtain
		\begin{align*}
			&\max_{j\in[p],k\in[K]}\left|\sum_{t=1}^T  u_{tj}\left(\sum_{\ell=1}^pu_{t\ell}b_{\ell k}\right)\right|\\
			&\le \sqrt{p} \max_{j\in[p],k\in[K]}\left| \sum_{t=1}^T\left(u_{tj}\left(p^{-1/2}\sum_{\ell=1}^pu_{t\ell}b_{\ell k}\right)-\E\left[u_{tj}\left(p^{-1/2}\sum_{\ell=1}^pu_{t\ell}b_{\ell k}\right)\right]\right)\right|\\
			&\quad + \max_{j\in[p],k\in[K]}\left| \sum_{t=1}^T\E\left[u_{tj}\left(\sum_{\ell=1}^pu_{t\ell}b_{\ell k}\right)\right]\right|   =O_P\left(\sqrt{T}p^{\frac{2}{q}+\frac12}+T\right).\\
		\end{align*}
		
		\noindent\textit{Proof of \eqref{lf2v}.} The result directly follows from the application of Lemmas \ref{lm.2exp} and \ref{lm.nagaevsup} to  $Z_t= u_tu_t^\top\varepsilon_t^2 -\E\left[u_tu_t^\top\varepsilon_t^2\right]$.\\
		
		\noindent\textit{Proof of \eqref{lf2vi}.} The result follows from applying Lemmas \ref{lm.2exp} and \ref{lm.nagaevsup} to $Z_t=\varepsilon_t^2-\E\left[\varepsilon_t^2\right] $ and using the triangle inequality.\\
		
		\noindent\textit{Proof of \eqref{lf2vibis}.}  To obtain this statement, we apply Lemmas \ref{lm.2exp} and \ref{lm.nagaevsup} to $Z_t=f_{tk}^2-\E[f_{tk}^2]$, sum over $k$ and use the triangle inequality, noticing that $\E[f_{tk}^2]=1$ by Assumption \ref{P-as.moments} \eqref{P-f4i}. \\
		
		\noindent\textit{Proof of \eqref{lf2vibisii}.}  Statement \eqref{lf2vibisii} follows from the application of Lemmas \ref{lm.2exp} and \ref{lm.nagaevsup} to $Z_t=f_{tk}f_{t\ell}-\E[f_{tk}f_{t\ell}]$, summing over $k,\ell$ and using the fact that $\E[f_{t}f_t^\top]=I_K$ by Assumption \ref{P-as.moments} \eqref{P-f4ii} from the main text. \\
		
		\noindent\textit{Proof of \eqref{lf2vii}.} We use the fact that $\E\left[\|U\|_2^2\right] =\E\left[\sum_{t=1}^T\sum_{j=1}^p u_{tj}^2\right]=O(Tp)$ by Assumption \ref{P-as.tail} \eqref{P-tailii} and Markov's inequality. \\
		
		\noindent\textit{Proof of \eqref{lf2viii}.} We apply Lemmas \ref{lm.2exp} and \ref{lm.nagaevsup} to $Z_t=\varepsilon_tf_{tk}$ and obtain $\sum_{t=1}^T\varepsilon_tf_{tk}=O_P\left(\sqrt{T}\right)$. This yields \eqref{lf2viii}, by $\left\|F^\top \mathcal{E}\right\|_2=\sqrt{\sum_{k=1}^K\left(\sum_{t=1}^T\varepsilon_tf_{tk}\right)^2}=O_P\left(\sqrt{T}\right).$ \\
		
		\noindent\textit{Proof of \eqref{lf2x}, \eqref{lf2ix} and \eqref{lf2ixbis}.} We apply Lemmas \ref{lm.2exp} and \ref{lm.nagaevsup} to 
		\begin{align*}Z_t&=\left(p^{-1/2}\sum_{\ell=1}^pu_{t\ell}b_{\ell k}\right)^2-\E\left[\left(p^{-1/2}\sum_{\ell=1}^pu_{t\ell}b_{\ell k}\right)^2\right]
		\end{align*}
		and obtain 
		\begin{align}
			\label{2507}\max_{k\in[K]}\left|\sum_{t=1}^T \left(\left(p^{-1/2}\sum_{\ell=1}^pu_{t\ell}b_{\ell k}\right)^2-\E\left[\left(p^{-1/2}\sum_{\ell=1}^pu_{t\ell}b_{\ell k}\right)^2\right]\right)\right| &=O_P\left(\sqrt{T}\right).
		\end{align}
		Note that, by Assumption \ref{P-as.tail} \eqref{P-tailiii}, 
		\begin{align}
			\label{24072}\max_{k\in[K]}\E\left[\left(p^{-1/2}\sum_{\ell=1}^pu_{t\ell}b_{\ell k}\right)^2\right]=O(1).
		\end{align}
		Then, we obtain the result using the triangle inequality and equations \eqref{2407} and \eqref{24072}:
		\begin{align*}
			\|UB\|_2^2&= p\sum_{t=1}^T \sum_{k=1}^K \left(p^{-1/2}\sum_{\ell=1}^pu_{t\ell}b_{\ell k}\right)^2\\
			&\le K p \max_{k\in[K]}\left|\sum_{t=1}^T\left( \left(p^{-1/2}\sum_{\ell=1}^pu_{t\ell}b_{\ell k}\right)^2-\E\left[\left(p^{-1/2}\sum_{\ell=1}^pu_{t\ell}b_{\ell k}\right)^2\right)\right]\right|\\
			&\quad + KTp\max_{k\in[K]}\E\left[\left(p^{-1/2}\sum_{\ell=1}^pu_{t\ell}b_{\ell k}\right)^2\right]=O_P\left(Tp\right).
		\end{align*}
		The proofs of  \eqref{lf2ix} and \eqref{lf2ixbis} are similar and therefore omitted.\\
		
		\noindent\textit{Proof of \eqref{lf2xi}, \eqref{lf2xii}.} We apply Lemmas \ref{lm.2exp} and \ref{lm.nagaevsup} to 
		\begin{align*}
			Z_t&=f_{t}\left(p^{-1/2}\sum_{\ell=1}^pu_{t\ell}b_{\ell k}\right);\\
			Z_t&=\varepsilon_{t}\left(p^{-1/2}\sum_{\ell=1}^pu_{t\ell}b_{\ell k}\right)
		\end{align*}
		and obtain the result by summing over $k$.
	\end{Proof}
	
	\begin{Lemma}\label{lm.lf}  Under the assumptions of Theorem \ref{P-th}, the following holds:
		\begin{enumerate}[\textup{(}i\textup{)}]  
			\item\label{lfi} $\left\|\overline{F}-FH^\top\right\|_2^2=O_P\left(\frac{T}{p}+1\right)$;
			\item\label{lfiii} $\left\|H^\top H-I_K\right\|_2^2 =O_P\left(\frac1T+\frac1p\right)$;
			\item\label{lfiv} $\max_{j\in[p]}\left\|\overline{b}_j -Hb_j\right\|_2=O_P\left(\frac{1}{\sqrt{p}}+\frac{p^{\frac{2}{q}}}{\sqrt{T}}\right)$;
			\item\label{lfv} $\left\|V^{-1}\right\|_2=O_P\left(\frac1p\right)$;
			\item\label{lfvi} $\left\|\overline{U}-U\right\|_\infty=o_P\left(\frac{p^{\frac{2}{q}}}{T^{1/2-1/q}} + \frac{T^{1/q}}{\sqrt{p}}\right)$;
			\item\label{lfii} $\max_{j\in[p]}\sum_{t=1}^T |\overline{u}_{tj}-u_{tj}|^2=O_P\left(p^{4/q} +\frac{T}{p}\right)$;
		\end{enumerate}
	\end{Lemma}
	\begin{Proof}
		The results \eqref{lfi} to \eqref{lfvi} follow from Lemmas S.9 and S.11 and Theorem 2 in \cite{fan2021bridging}. Assumption 2 in \cite{fan2021bridging} is satisfied by our Assumption \ref{P-as.tail}. Assumption 3 in \cite{fan2021bridging} corresponds to our Assumption \ref{P-as.moments}. 
		
		To show \eqref{lfii}, note that, for all $t\in[T]$ and $j\in[p]$, it holds that
		\begin{align*} \overline{u}_{tj}-u_{tj}&=x_{tj}^\top -\overline{b}_j^\top \overline{f}_t-u_{tj}\\
			&= b_j^\top f_t -\overline{b}_j^\top \overline{f}_t\\
			&=b_j^\top \left(I_K-H^\top H\right)f_t -\left(\overline{b}_j -Hb_j \right)^\top\overline{f}_t- b_j^\top H^\top \left(\overline{f}_t-Hf_t\right).
		\end{align*}
		This yields 
		\begin{align*}\max_{j\in[p]}\sum_{t=1}^T |\overline{u}_{tj}-u_{tj}|^2&\le K\|B\|_\infty^2\left\|H^\top H-I_K\right\|_2^2\|F\|_2^2\\
			&\quad +\max_{j\in[p]}\left\|\overline{b}_j -Hb_j\right\|_2^2T \\
			&\quad +  K\|B\|_\infty^2\|H\|_2^2\left\|\overline{F}-FH^\top\right\|_2^2=O_P\left(p^{4/q} +\frac{T}{p}\right),
		\end{align*}
		where we used \eqref{lfi}, \eqref{lfiii}, \eqref{lfiv}, Lemma \ref{lm.lf2} \eqref{lf2vibis} and Assumption \ref{P-as.moments} \eqref{P-f4iv}.
	\end{Proof}
	
	\begin{Lemma}\label{lm.bai}Under the assumptions of Theorem \ref{P-th}, it holds that $$\overline{F}-FH^\top = \frac1T FB^\top U^\top \overline{F}V^{-1}+\frac1T  UBF^\top \overline{F}V^{-1}+\frac1T UU^\top  \overline{F}V^{-1}.$$
	\end{Lemma}
	\begin{Proof}
		Recall that $H=\frac1T \overline{V}\overline{F}^\top FB^\top B$ and $\overline{F}V=\frac1T XX^\top\overline{F}$. As a result, we have
		\begin{align*}
			\overline{F}V& = T^{-1}XX^\top\overline{F}\\
			&=  \frac1T (FB^\top  + U)(FB^\top  + U)^\top\overline{F}\\
			&= \frac1T FB^\top B F^\top\overline{F}+\frac1T  FB^\top U^\top \overline{F}+ \frac1T UBF^\top \overline{F}+T^{-1}UU^\top  \overline{F}.
		\end{align*}
		Multiplying both sides by $V^{-1}$, we get the result.
	\end{Proof}
	\begin{Lemma}\label{lm.complex} 
		Under the assumptions of Theorem \ref{P-th}, we have 
		$$\left\|\left(\overline{F}-FH^\top\right)^\top \mathcal{E}\right\|_2=O_P\left(\sqrt{\frac{T}{p}} +1 \right).$$
	\end{Lemma}
	\begin{Proof}
		By Lemma \ref{lm.bai}, we have \begin{equation}\label{bai1}\left\|(\overline{F}-FH^\top)^\top \mathcal{E}\right\|_2 \le J_1+J_2+J_3,\end{equation}
		where 
		\begin{align*}
			J_1&=  \frac1T \left\|\mathcal{E}^\top FB^\top U^\top \overline{F}V^{-1}\right\|_2;\\
			J_2&=  \frac1T \left\|\mathcal{E}^\top UBF^\top \overline{F}V^{-1}\right\|_2;\\
			J_3&=  \frac1T \left\|\mathcal{E}^\top UU^\top  \overline{F}V^{-1}\right\|_2.
		\end{align*}
		We have 
		\begin{align}
			\notag J_1&\le \frac1T \left\|\mathcal{E}^\top F \right\|_2\left(\|UB\|_2 \left\|\overline{F}-FH^\top\right\|_2+\|H\|_2\left\|B^\top U^\top F\right\|_2\right)\|V^{-1}\|_2\\
			\label{bai2}&=O_P\left(\frac{1}{T} \sqrt{T}\left(\sqrt{Tp}\sqrt{\frac{T}{p} +1}+\sqrt{Tp}\right)\frac1p\right)=O_P\left( \frac{\sqrt{T}}{p}\frac{1}{\sqrt{p}}\right),
		\end{align}
		by Lemmas \ref{lm.lf} \eqref{lfi}, \eqref{lfiii}, \eqref{lfv} and \ref{lm.lf2} \eqref{lf2viii}, \eqref{lf2x}, \eqref{lf2xi}.
		Moreover, it holds that
		\begin{align}
			\notag J_2&\le \frac1T \|\mathcal{E}^\top UB\|_2\left\| F\right\|_2\left\|\overline{F}\right\|_2\|V^{-1}\|_2\\
			\label{bai3}&=O_P\left(\frac{1}{T}\sqrt{T} \sqrt{Tp}\sqrt{T}\frac1p\right)=O_P\left(\sqrt{\frac{T}{p}}\right),
		\end{align}
		by Lemmas \ref{lm.lf} \eqref{lfv} and \ref{lm.lf2} \eqref{lf2vibis}, \eqref{lf2xii} and the fact that $\left\|\overline{F}\right\|_2=\sqrt{KT}$.
		We also have 
		\begin{align}
			\notag J_3&\le \frac1T \left\|\mathcal{E}^\top U\right\|_2\left(\|U\|_2 \left\|\overline{F}-FH^\top\right\|_2+\left\|U^\top F\right\|_2\right)\|V^{-1}\|_2\\
			\notag &=O_P\left(\frac{1}{T}\sqrt{Tp}\left(\sqrt{Tp}\sqrt{\frac{T}{p}+1}+ \sqrt{Tp}\right)\frac1p\right)\\
			\label{bai4}&=O_P\left(\sqrt{\frac{T}{p}} + \frac{1}{\sqrt{p}} +1\right),
		\end{align}
		where we used Lemmas \ref{lm.lf} \eqref{lfi}, \eqref{lfv} and \ref{lm.lf2} \eqref{lf2vii}, \eqref{lf2ix}, \eqref{lf2ixbis}.
		We obtain the result by \eqref{bai1}, \eqref{bai2}, \eqref{bai3} and \eqref{bai4}.
	\end{Proof}
	\begin{Lemma}\label{20723} Under the assumptions of Theorem \ref{P-th}, we have 
		$$\max_{j\in[p]} \sum_{t=1}^T\left(\overline{u}_{tj}\widetilde{\varepsilon}_t+\widetilde{f}_t^\top\varphi^*- u_{tj}\varepsilon_t\right)^2=O_P\left(\left(p^{\frac{4}{q}}T^{\frac{2}{q}}+\frac{T^{\frac{2}{q}+1}}{p}\right)+ \left(1+\frac{T}{p}\right)\left\|\varphi^*\right\|_2^2\right).$$
	\end{Lemma}
	\begin{Proof}
		First, notice that, by the triangle inequality,
		\begin{align}
			\notag &\sqrt{ \sum_{t=1}^T\left(\overline{u}_{tj}\widetilde{\varepsilon}_t+\widetilde{f}_t^\top\varphi^*- u_{tj}\varepsilon_t\right)^2}\\
			\notag &=\sqrt{ \sum_{t=1}^T\left(\overline{u}_{tj}\left(\widetilde{\varepsilon}_t-\varepsilon_t\right)+\widetilde{f}_t^\top\varphi^*+ \left(\overline{u}_{tj}- u_{tj}\right)\varepsilon_t\right)^2}\\
			\label{2071}&\le \sqrt{ \sum_{t=1}^T\left(\overline{u}_{tj}\left(\widetilde{\varepsilon}_t-\varepsilon_t\right)\right)^2}+
			\sqrt{ \sum_{t=1}^T\left(\widetilde{f}_t^\top\varphi^*\right)^2} +\sqrt{ \sum_{t=1}^T\left( \left(\overline{u}_{tj}- u_{tj}\right)\varepsilon_t\right)^2}.
		\end{align}
		We first bound the term 
		$\sum_{t=1}^T\left(\overline{u}_{tj}\left(\widetilde{\varepsilon}_t-\varepsilon_t\right)\right)^2$. Remark that 
		\begin{equation}\label{2072}\sum_{t=1}^T\left(\overline{u}_{tj}\left(\widetilde{\varepsilon}_t-\varepsilon_t\right)\right)^2\le \left\|\overline{U}\right\|_\infty^2 \left\| \left(I_T-\overline{P}\right)\mathcal{E} - \mathcal{E}\right\|_2^2=\left\|\overline{U}\right\|_\infty^2 \left\| \overline{P}\mathcal{E}\right\|_2^2.\end{equation}
		Now, using Lemma \ref{lm.nagaev} and the tail bound in Assumption \ref{P-as.tail} \eqref{P-tailiii}, we obtain $\left\|U\right\|_\infty=O_P\left((Tp)^{1/q}\right)$. Combining this with Lemma \ref{lm.lf} \eqref{lfvi} and $\left\|\overline{U}\right\|_\infty\le \left\|\overline{U}-U\right\|_\infty+\left\|U\right\|_\infty$, we get \begin{equation}\label{2073}\left\|\overline{U}\right\|_\infty^2=O_P\left((Tp)^{2/q}\right).\end{equation} Next, recall that $\overline{P}=T^{-1} \overline{F}\overline{F}^\top\mathcal{E}$ and $\left\|\overline{F}\right\|_2=\sqrt{KT}$. This yields
		\begin{align}\notag \left\| \overline{P}\mathcal{E}\right\|_2&\le \frac1T\left\|\overline{F}\right\|_2\left\| \left(\overline{F}-FH^\top\right)^\top\mathcal{E}\right\|_2+\frac1T\left\|\overline{F}\right\|_2\left\| H\right\|_2\left\|F^\top\mathcal{E}\right\|_2\\
			\label{2074} &=\frac{1}{\sqrt{T}} O_P\left(\sqrt{\frac{T}{p}} +1\right)= O_P\left(\frac{1}{\sqrt{T}}+\frac{1}{\sqrt{p}}\right),
		\end{align}
		by Lemmas \ref{lm.lf} \eqref{lfiii}, \ref{lm.lf2} \eqref{lf2viii} and \ref{lm.complex}. Thanks to \eqref{2072}, \eqref{2073} and \eqref{2074}, we obtain 
		\begin{equation}\label{2075} \sum_{t=1}^T\left(\overline{u}_{tj}\left(\widetilde{\varepsilon}_t-\varepsilon_t\right)\right)^2=O_P\left((Tp)^{\frac{2}{q}}\left(\frac{1}{T}+\frac{1}{p}\right)\right).
		\end{equation}
		Let us now bound the term $ \sum_{t=1}^T\left(\widetilde{f}_t^\top\varphi^*\right)^2$. We have 
		\begin{equation}\label{2076}
			\sum_{t=1}^T\left(\widetilde{f}_t^\top\varphi^*\right)^2= \left\|\left(I_T- \overline{P}\right) F\varphi^*\right\|_2^2
			\le \left\|\left(I_T- \overline{P}\right) F\right\|_2^2\left\|\varphi^*\right\|_2^2.
		\end{equation}
		Next, notice that 
		\begin{align}
			\notag \left\|\left(I_T- \overline{P}\right) F\right\|_2&= \left\|\left(I_T-\frac1T\overline{F}\overline{F}^\top\right) F\right\|_2\\
			\notag &\le \left\|\frac1T\left(\overline{F}-FH^\top\right)\left(FH^\top\right)^\top F\right\|_2+ \left\| \frac1TFH^\top \left(\overline{F}-FH^\top\right)^\top F\right\|_2\\
			\label{2077}&\quad + \left\|\left(I_T- \frac1TFH^\top\left(FH^\top\right)^\top \right)F\right\|_2.
		\end{align}
		Then, notice that 
		\begin{align}
			\notag &\left\|\frac1T\left(\overline{F}-FH^\top\right)\left(FH^\top\right)^\top F\right\|_2+ \left\| \frac1TFH^\top \left(\overline{F}-FH^\top\right)^\top F\right\|_2\\
			\label{2078} &\le \frac2T \left\|\overline{F}-FH^\top\right\|_2 \left\|F\right\|_2^2 \left\| H\right\|_2=O_P\left(\sqrt{\frac{T}{p}}+1\right),
		\end{align}
		by Lemmas \ref{lm.lf} \eqref{lfi}, \eqref{lfiii} and \ref{lm.lf2} \eqref{lf2vibis}.
		Moreover, we have
		\begin{align}
			\notag &  \left\|\left(I_T- \frac1TFH^\top\left(FH^\top\right)^\top \right)F\right\|_2\\
			\notag &\le \left\|\left(I_T- \frac1TFF^\top \right)F\right\|_2+  \left\| \frac1TF\left(H^\top H-I_K\right)F^\top F\right\|_2\\
			\label{2079}&\le \left\|F\right\|_2\left\|I_K- \frac1T F^\top F \right\|_2+  \frac1T\left\| F\right\|_2\left\|H^\top H-I_K\right\|_2\left\|F\right\|_2^2= O_P\left(1+\sqrt{\frac{T}{p}}\right),
		\end{align}
		by Lemmas \ref{lm.lf2} \eqref{lf2vibis}, \eqref{lf2vibisii} and \ref{lm.lf} \eqref{lfiii}. Combining \eqref{2076}, \eqref{2077}, \eqref{2078} and \eqref{2079}, we get 
		\begin{equation}\label{20710} \sum_{t=1}^T\left(\widetilde{f}_t^\top\varphi^*\right)^2= O_P\left(1+\frac{T}{p}\right)\left\|\varphi^*\right\|_2^2. \end{equation}
		Finally, we bound $\sum_{t=1}^T\left(\left(\overline{u}_{tj}- u_{tj}\right)\varepsilon_t\right)^2$. Notice that 
		\begin{equation}\label{decomp6162}
			\max_{j\in[p]}\sum_{t=1}^T\left((\overline{u}_{tj}-u_{tj})\varepsilon_t\right)^2 \le  \|\mathcal{E}\|_\infty^2\max_{j\in[p]} \sum_{t=1}^T(\overline{u}_{tj}-u_{tj})^2
		\end{equation}
		Next, using Lemma \ref{lm.nagaev} and the tail bound in Assumption \ref{P-as.tail} \eqref{P-tailiii}, we have $ \|\mathcal{E}\|_\infty^2=O_P\left(T^{\frac{2}{q}}\right).$ This, Lemma \ref{lm.lf} \eqref{lfii} and equation
		\eqref{decomp6162} yield that 
		\begin{equation}\label{20711}
			\max_{j\in[p]}\sum_{t=1}^T\left(\left(\overline{u}_{tj}- u_{tj}\right)\varepsilon_t\right)^2=O_P\left(T^{\frac{2}{q}}p^{\frac{4}{q}} +\frac{T^{1+\frac{2}{q}}}{p}\right).
		\end{equation}
		Combining \eqref{2071}, \eqref{2075}, \eqref{20710} and \eqref{20711}, we obtain the result.
	\end{Proof}
	\begin{Lemma}\label{fancomplex} Under the assumptions of Theorem \ref{P-th}, we have 
		$$\left\|\overline{U}^\top \left(\overline{Y}-\overline{U}\beta^*\right)-U^\top \mathcal{E}\right\|_{\infty}=(\|\varphi^*\|_2\vee 1)O_P\left(\frac{T}{p}+p^{\frac2q}  +\sqrt{T}p^{\frac2q-\frac12}\right).$$
	\end{Lemma}
	\begin{Proof}
		In all this proof, we work on the event $\mathcal{E}_{\sigma}=\{\sigma_{p}\left(H^\top H\right)\ge 1/2\}$ which has probability going to $1$ by Lemma \ref{lm.lf} \eqref{lfiii}. Note that, on $\mathcal{E}_\sigma$, we have \begin{equation}\label{2307}\left\|\left(H^{\top}\right)^{-1}\right\|_2\le \sqrt{K}\left\|\left(H^{\top}\right)^{-1}\right\|_{op} \le \sqrt{K}\sigma_{p}\left(H^\top H\right)^{-1/2}\le \sqrt{2K}.\end{equation} Recall that $\overline{Y}=\left(I_T-\overline{P}\right)(X\beta^* +F\varphi^* +\mathcal{E})$. This yields
		\begin{equation}\label{key1}\begin{aligned}\left\|\overline{U}^\top \left(\overline{Y}-\overline{U}\beta^*\right)- U^\top\mathcal{E}\right\|_\infty &\le \left\|\overline{U}^\top (F\varphi^*+\mathcal{E})- U^\top\mathcal{E}\right\|_\infty\\
				&\le \left\|\overline{U}^\top F\varphi^*\right\|_\infty +\left\|\left(\overline{U}- U\right)^\top\mathcal{E}\right\|_\infty.
			\end{aligned}
		\end{equation}
		Let us first bound $ \left\|\overline{U}^\top F\varphi^*\right\|_\infty$. Since $\overline{U}^\top\overline{F}=0$ and $H^\top$ is invertible on the event $\mathcal{E}_{\sigma}$, it holds that
		\begin{equation}\label{key2}\left\|\overline{U}^\top F\varphi^*\right\|_\infty\le \left\|\left(\overline{U}-U\right)^\top \left(FH^\top -\overline{F}\right)\left(H^{\top}\right)^{-1}\varphi^*\right\|_\infty +\left\|U^\top \left(FH^\top -\overline{F}\right)\left(H^{\top}\right)^{-1}\varphi^*\right\|_\infty.
		\end{equation}
		We now bound the first term on the right-hand side of \eqref{key2}. By the inequality of Cauchy-Schwartz, we have 
		\begin{align}
			\notag &\left\|\left(\overline{U}-U\right)^\top \left(FH^\top -\overline{F}\right)\left(H^{\top}\right)^{-1}\varphi\right\|_\infty\\
			\notag &=\max_{j\in[p]}\left|\left(\left(\overline{U}-U\right)^\top \left(FH^\top -\overline{F}\right)\left(H^{\top}\right)^{-1}\varphi^*\right)_j\right|\\
			\notag &\le \left(\max_{j\in[p]}\sum_{t=1}^T \left|\overline{u}_{tj}-u_{tj}\right|^2\right)^{1/2}\left\|\overline{F}-FH^\top\right\|_2\left\|\left(H^{\top}\right)^{-1}\right\|_2\|\varphi^*\|_2\\
			\label{key3} &=\|\varphi^*\|_2O_P\left(\sqrt{p^{\frac{4}{q}}+\frac{T}{p}}\sqrt{\frac{T}{p}+1}\right)=\|\varphi^*\|_2O_P\left(\frac{T}{p} + p^{\frac{2}{q}}+p^{\frac{2}{q}-\frac12}\sqrt{T}\right),
		\end{align}
		where we used Lemma \ref{lm.lf} \eqref{lfi}, \eqref{lfiii}, \eqref{lfii} and equation \eqref{2307}.
		Next, we control the second term on the right-hand side of \eqref{key2}. By Lemma \ref{lm.bai}, it holds that 
		
		\begin{equation}\label{key4}\left\|U^\top \left(FH^\top -\overline{F}\right)\left(H^{\top}\right)^{-1}\varphi^*\right\|_\infty\le J_{1}+J_{2}+ J_{3},\end{equation}
		where 
		\begin{align*}J_{1}&= \frac1T\left\| U^\top FB^\top U^\top \overline{F}V^{-1}\left(H^{\top}\right)^{-1}\varphi^*\right\|_\infty;\\
			J_2&= \frac1T\left\| U^\top UBF^\top \overline{F}V^{-1}\left(H^{\top}\right)^{-1}\varphi^*\right\|_\infty;\\
			J_3&= \frac1T\left\| U^\top UU^\top  \overline{F}\left(H^{\top}\right)^{-1}\varphi^*\right\|_\infty.
		\end{align*}
		Remark that
		\begin{align}
			\notag \left\|B^\top U^\top \overline{F}\right\|_2&\le\left\|B^\top U^\top\right\|_2 \left\|\overline{F}-FH^\top\right\|_2+\|H\|_2\left\|B^\top U^\top F\right\|_2\\
			\label{160723}&=O_P\left(\sqrt{Tp}\sqrt{\frac{T}{p}+1} +\sqrt{Tp} \right)=O_P(T+\sqrt{Tp}),
		\end{align}
		by Lemmas \ref{lm.lf2} \eqref{lf2x}, \eqref{lf2xi} and \ref{lm.lf} \eqref{lfi}, \eqref{lfiii}.
		By the inequality of Cauchy-Schwartz, this yields
		\begin{align}
			\notag J_{1}
			\notag &=\frac1T \max_{j\in[p]}\left|\left(U^\top FB^\top U^\top \overline{F}V^{-1}\left(H^{\top}\right)^{-1}\varphi^*\right)_j\right|\\
			\notag &=\frac1T \max_{j\in[p]}\left|\sum_{k=1}^K\left(\sum_{t=1}^T u_{tj} f_{tk} \right) \left(B^\top U^\top \overline{F}V^{-1}\left(H^{\top}\right)^{-1}\varphi^*\right)_k\right|\\
			\notag &\le \frac1T \left(\max_{j\in[p]}\left|\sum_{k=1}^K\left(\sum_{t=1}^T u_{tj} f_{tk} \right)^2\right|\right)^{1/2} \left\|B^\top U^\top \overline{F}\right\|_2 \left\|V^{-1}\right\|_2\left\|(H^{-1})^\top\right\|_2\|\varphi^*\|_2\\
			\notag &\le \frac1T  \sqrt{K} \max_{j\in[p]}\left|\sum_{t=1}^T u_{tj} f_{tk} \right| \left\|B^\top U^\top \overline{F}\right\|_2 \left\|V^{-1}\right\|_2\left\|(H^{-1})^\top\right\|_2\|\varphi^*\|_2\\
			\label{key5} &=O_P\left(\frac{1}{Tp}\left(T+\sqrt{Tp} \right)\sqrt{T}p^{\frac{2}{q}} \right)\|\varphi^*\|_2=O_P\left(\sqrt{T}p^{\frac{2}{q}-1}+p^{\frac{2}{q}-\frac12}\right)\|\varphi^*\|_2,
		\end{align}
		where we used Lemmas  \ref{lm.lf2} \eqref{lf2ii} and \ref{lm.lf} \eqref{lfiii}, \eqref{lfv} and equations \eqref{160723} and \eqref{2307}.
		Then, notice that, by Lemma \ref{lm.lf} \eqref{lfi} and \eqref{lfiii}, we have 
		\begin{equation}\label{2407}\left\|F^\top \overline{F}\right\|_2\le \|F\|_2 \left\|\overline{F}-FH^\top\right\|_2+  \|F\|_2^2 \|H\|_2=O_P(T).\end{equation}
		This allows to bound $J_{2}$. Indeed, by the inequality of Cauchy-Schwartz, it holds that 
		\begin{align}
			J_{2}\notag &=\frac1T \max_{j\in[p]}\left|\left(U^\top UBF^\top \overline{F}V^{-1}\left(H^{\top}\right)^{-1}\varphi^*\right)_j\right|\\
			\notag &= \frac1T \max_{j\in[p]}\left|\sum_{k=1}^K\sum_{t=1}^T  u_{tj}\left(\sum_{\ell=1}^pu_{t\ell}b_{\ell k}\right) \left(F^\top \overline{F}V^{-1}\left(H^{\top}\right)^{-1}\varphi^*\right)_k\right|\\
			\notag &\le \frac1T \max_{j\in[p]} \sqrt{\sum_{k=1}^K\left(\sum_{t=1}^T  u_{tj}\left(\sum_{\ell=1}^pu_{t\ell}b_{\ell k}\right)\right)^2 } \left\| F^\top \overline{F}V^{-1}\left(H^{\top}\right)^{-1}\varphi^*\right\|_2\\
			\notag &\le \frac{1}{T}\sqrt{K} \max_{j\in[p],k\in[K]}\left|\sum_{t=1}^T  u_{tj}\left(\sum_{\ell=1}^pu_{t\ell}b_{\ell k}\right)\right| \left\|F^\top \overline{F}\right\|_2   \left\|V^{-1}\right\|_2\left\|(H^{-1})^\top\right\|_2\|\varphi^*\|_2\\
			\label{key6}&=O_P\left(\frac{1}{Tp}T\left(T+p^{\frac{2}{q}+\frac12}\sqrt{T} \right)\right)\|\varphi^*\|_2=O_P\left(\frac{T}{p}+\sqrt{T} p^{\frac{2}{q}-\frac12}\right)\|\varphi^*\|_2,
		\end{align}
		by Lemmas \ref{lm.lf2} \eqref{lf2iv}, \eqref{lf2vibis} and \ref{lm.lf} \eqref{lfiii}, \eqref{lfv} and equations \eqref{2307} and \eqref{2407}.
		Finally note that 
		\begin{align}\notag \left\|U^\top \overline{F}\right\|_2 &\le \left\|U^\top F\right\|_2 \left\|H^\top\right\|_2+ \|U\|_2 \left\|\overline{F}-FH^\top\right\|_2\\
			\label{24072}& =O_P\left(\sqrt{T}+\sqrt{T}\sqrt{\frac{T}{p}+1}\right)=O_P\left(\sqrt{T}+ \frac{T}{\sqrt{p}}\right),\end{align}
		by Lemmas \ref{lm.lf} \eqref{lfi}, \eqref{lfiii} and \ref{lm.lf2} \eqref{lf2vii}, \eqref{lf2ix}. Thanks to this, we can bound $J_3$. Indeed, by the inequality of Cauchy-Schwartz, we have 
		\begin{align}
			\notag J_{3}
			\notag &=\frac1T \max_{j\in[p]}\left|\left(U^\top UU^\top  \overline{F}\left(H^{\top}\right)^{-1}\varphi^*\right)_j\right|\\
			\notag &= \frac1T \max_{j\in[p]}\left|\sum_{\ell=1}^p\left(\sum_{t=1}^T  u_{tj}u_{t\ell}\right) \left(U^\top  \overline{F}\left(H^{\top}\right)^{-1}\varphi^*\right)_{\ell}\right|\\
			&\le \frac{1}{T}  \max_{j\in[p]}\sqrt{\sum_{\ell=1}^p\left(\sum_{t=1}^T  u_{tj}u_{t\ell}\right)^2} \left\|U^\top \overline{F}\right\|_2   \left\|V^{-1}\right\|_2\left\|(H^{-1})^\top\right\|_2\|\varphi^*\|_2 \\
			\notag &\le \frac{1}{T} \sqrt{p} \max_{j\in[p]}\left|\sum_{t=1}^T  u_{tj}^2\right|\left\|U^\top \overline{F}\right\|_2   \left\|V^{-1}\right\|_2\left\|(H^{-1})^\top\right\|_2\|\varphi^*\|_2 \\
			\label{key7}&=O_P\left(\frac{1}{Tp}T\sqrt{p} \left( \sqrt{T} +\frac{T}{\sqrt{p}}   \right)\right)\|\varphi^*\|_2=O_P\left(\sqrt{\frac{T}{p}}+ \frac{T}{p}\right)\|\varphi^*\|_2,
		\end{align}
		where we used Lemmas \ref{lm.lf2} \eqref{lf2i} and \ref{lm.lf} \eqref{lfv} and equations \eqref{2307} and \eqref{24072}.
		Then, \eqref{key2}, \eqref{key3}, \eqref{key4}, \eqref{key5}, \eqref{key6}, \eqref{key7} imply that 
		\begin{equation}\label{key8}\left\|\overline{U}^\top F\varphi^*\right\|_\infty=O_P\left(\frac{T}{p}+p^{\frac{2}{q}} +\sqrt{T}p^{\frac{2}{q}-\frac12} \right)\|\varphi^*\|_2.\end{equation}
		
		Let us now bound the second term on the right-hand side of \eqref{key1}, that is $\left\|\left(\overline{U}- U\right)^\top\mathcal{E}\right\|_\infty.$
		Note that \begin{align*} \overline{U}^\top-U^\top&=X^\top -\overline{B}\overline{F}^\top-U^\top\\
			&= BF^\top -\overline{B}\overline{F}^\top\\
			&= B\left(I_K-H^\top H\right)F^\top -\left(\overline{B}-BH^\top \right)\overline{F}^\top - BH^\top \left(\overline{F}-FH\right)^\top.
		\end{align*}
		This yields
		\begin{equation}\label{key9}
			\left\|\left(\overline{U}- U\right)^\top\mathcal{E}\right\|_\infty \le K_1+K_2+K_3,\end{equation} 
		where
		\begin{align*}K_1&=\left\|B\left(I_K-H^\top H\right)F^\top\mathcal{E} \right\|_\infty;\\
			K_2&= \left\|\left(\overline{B}-BH^\top \right)\overline{F}^\top\mathcal{E}  \right\|_\infty;\\
			K_3&= \left\|BH^\top \left(\overline{F}-FH\right)^\top\mathcal{E}\right\|_\infty.
		\end{align*}
		By the inequality of Cauchy-Schwartz, Lemmas \ref{lm.lf2} \eqref{lf2viii} and \ref{lm.lf} \eqref{lfiii} and Assumption \ref{P-as.moments} \eqref{P-f4iv}, it holds that
		\begin{align}\notag K_1 &=\max_{j\in[p]} \left|\sum_{k=1}^K b_{jk}\left(\left(I_K-H^\top H\right)F^\top\mathcal{E}\right)_{k} \right|\\
			\notag &\le  \sqrt{K}\|B\|_\infty \left\|I_K-H^\top H\right\|_2\left\|F^\top \mathcal{E}\right\|_2\\
			\label{key10}&=O_P\left(\sqrt{\frac{1}{T}+\frac{1}{p}} \sqrt{T}\right)= O_P\left(1+\sqrt{\frac{T}{p}}\right).
		\end{align}
		Next, we have
		\begin{align}
			\notag K_2&=\max_{j\in[p]} \left|\sum_{k=1}^K \left(\overline{b}_{j}-Hb_j\right)_k\left(\overline{F}^\top\mathcal{E}\right)_{k} \right|\\
			\notag &\le  \max_{j\in[p]}\left\|\overline{b}_j -Hb_j\right\|_2 \left\|\overline{F}^\top\mathcal{E}\right\|_2\\
			\notag &\le\max_{j\in[p]}\left\|\overline{b}_j -Hb_j\right\|_2 \left(\left\|\left(\overline{F}-FH^\top\right)^\top\mathcal{E}\right\|_2+\|H\|_2\left\| F^\top \mathcal{E}\right\|_2\right)\\
			\notag &= O_P\left(\left(\frac{1}{\sqrt{p}}+\frac{p^\frac{2}{q}}{\sqrt{T}}\right) \left(\sqrt{T} +\sqrt{\frac{T}{p}}\right)\right)\\
			\label{key11} &=O_P\left(\sqrt{\frac{T}{p}} +p^\frac{2}{q}\right).
		\end{align}
		where we used the inequality of Cauchy-Schwarz, Lemmas \ref{lm.lf2} \eqref{lf2viii}, \ref{lm.lf} \eqref{lfiii}, \eqref{lfiv}, and \ref{lm.complex}.
		Finally, by the inequality of Cauchy-Schwartz, Lemmas \ref{lm.lf2} \eqref{lfiii} and \ref{lm.complex} and Assumption \ref{P-as.moments} \eqref{P-f4iv}, it holds that 
		\begin{align}\notag K_3&=\max_{j\in[p]} \left|\sum_{k=1}^Kb_{jk}\left(H^\top \left(\overline{F}-FH\right)^\top\mathcal{E}\right)_{k} \right|\\
			\notag &\le  \sqrt{K}\left\|B\right\|_\infty\left\|\left(\overline{F}-FH^\top\right)^\top\mathcal{E}\right\|_2 \left\|H\right\|_2\\
			\label{key12}&=O_P\left(\sqrt{\frac{T}{p}}+1\right).
		\end{align}
		Combining \eqref{key9}, \eqref{key10}, \eqref{key11} and \eqref{key12} yields
		\begin{align}
			\label{key13}\frac1T\left\|\left(\overline{U}- U\right)^\top\mathcal{E}\right\|_\infty=O_P\left(\sqrt{\frac{T}{p}}+p^{\frac2q}\right).
		\end{align}
		We obtain the result of the lemma by \eqref{key1}, \eqref{key8} and \eqref{key13}. 
	\end{Proof}
	
	\subsection{Pre-existing results on strong mixing sequences and high-dimensional Gaussian vectors}\label{subsec.pre}
	In this section, we state some useful lemmas and reformulate some results of \cite{fan2021bridging} and \cite{lederer2021estimating} that we use to prove Theorem \ref{P-th}.
	\subsubsection{Results on variables with polynomial tails}
	The following result is a direct consequence of the inequality of Cauchy-Schwarz. This lemma allows to show that products of variables in $u_{tj},f_{tk},\varepsilon_t$, $p^{-1/2} \sum_{j=1}^pb_ju_{tj}$ have polynomial tails.
	\begin{Lemma}\label{lm.2exp}
		Let $Z_1$ and $Z_2$ be random variables such that $\vertiii{Z_1}_q \le C$ and  $\vertiii{Z_2}_q\le C$,
		for some constants $C,q>0$. Then, we have $\vertiii{Z_1Z_2}_{q/2}\le C^2$.
	\end{Lemma}
	The next lemma serves to bound the sup norm of some variables.
	\begin{Lemma}\label{lm.nagaev}
		Let $Z$ be a mean-zero $p$-dimensional random vector. Assume that there exist constants $C,h>0$ such that we have 
		$\vertiii{Z}_{h}\le C.$ Then, it holds that $\|Z\|_\infty=O_P\left(p^{1/h} \right).$
		
	\end{Lemma}
	
	\begin{Proof} For all $j\in[p]$ and $z>0$, we have $\P(|Z_j|\le z)= \P(|Z_j|^h\le z^h)\le \E[|Z_j|^h]/z^h$ by Markov's inequality. By the union bound, we obtain $\P(|Z|_\infty\ge z)\le p \max_{j\in[p]} \P(|Z_j|_\infty\ge z)\le p \left( \E[|Z_j|^h]/z^h\right)$. Taking $z\propto p^{1/h} $ we obtain the result.
	\end{Proof}
	
	\subsubsection{Results on strong mixing sequences}
	
	The next Lemma is a direct consequence of Lemma S.4 and Remark 4 in \cite{fan2021bridging}.
	\begin{Lemma}\label{lm.nagaevsup}
		Let $S_T=\sum_{t=1}^T Z_t$, where $\{Z_t\}_{t}$ is a sequence of mean-zero $p$-dimensional random vectors such that 
		\begin{enumerate}[\textup{(}i\textup{)}]  
			\item\label{lnagi}  There exist constants $C_1,\xi>0$ and $h\ge 2$ such that, for all $t\in[T]$, we have 
			$$\vertiii{Z_t}_{h+\xi}\le C_1;$$
			\item\label{lnagii} Theres exist constants $C_2,c>0$ such that the strong mixing coefficients of the sequence $\{Z_t\}_{t}$ satisfy $\tilde\alpha(t)\le C_2 t^{-c}$ for all $t\in\mathbb{Z}_+$;
			\item\label{lnagiii} $c>\frac{(h-1)(h+\xi)}{\xi}$.
		\end{enumerate}
		Then, it holds that $\|S_T\|_\infty=O_P\left(p^{1/h} \sqrt{T}\right).$
		
	\end{Lemma}
	
	
	The last result of this subsection is a direct consequence of the high-dimensional central limit theorem for strong mixing sequences due to Theorem S.7 (a) in \cite{fan2021bridging}.
	\begin{Lemma}\label{lm.highdimclt}
		Let $S_T=n^{-1/2}\sum_{t=1}^T Z_t$, where $\{Z_t\}_{t}$ is a sequence of mean-zero $p$-dimensional random vectors, such that 
		\begin{enumerate}[\textup{(}i\textup{)}] 
			\item\label{lhdi} There exist $C_1,\xi>0$ and $h\ge 4$ such that, for all $t\in[T]$, $j\in[p]$ and $z>0$, we have 
			$\vertiii{Z_{t}}_{h+\xi}\le C_1 ;$
			\item\label{lhdii}  There exist constants $C_2,c>0$ such that the strong mixing coefficients of the sequence $\{Z_{t}\}_{t}$ satisfy $\tilde \alpha(t)\le C_2t^{-c}$ for all $t\ge 2$;
			\item\label{lhdiii} $c >\left[ \left(\frac{h+\xi}{\xi}\right)\left(\frac{h}{2}-1\right)\right]\vee \left(\frac{2}{1-\frac{2}{h}}\right)$ and $\kappa= \left(\frac{\frac{1}{2} +\frac{h}{4(h+1)}}{c+\frac{h}{2(h+1)}}\right)<\frac12$
			\item\label{lhdiv}  There exists $\sigma_*>0$ such that $\sigma_{p}(\Sigma) \ge \sigma_*^2$, where $\Sigma=\E\left[S_TS_T^\top\right]$;
			%\item\label{lhdv}  $\log(p)^{(1/\zeta)-(1/2)}/T=o(1)$.
		\end{enumerate} Let also $G\sim\mathcal{N}(0,\Sigma)$. Then, there exists a constant $K_2>0$ such that, for all $z\ge0$, we have
		\begin{align*}
			&\sup_{z\in\R_+}\left|\P\left(\left\|S_T\right\|_\infty\le z\right)-\P(|G|_\infty \le z) \right|\\
			&\le  K_2\Big[\left(T^{\kappa-1/2}+T^{1-\frac{c}{2}(1-\frac{2}{h})}\right)\log(T)\log(p)+T^{-1/4}\log (p)^{3/2} \log(T)+ p^{\frac{2}{h}} T^{\frac{1}{h}-\frac12} \log(p)^2 \log(T)\\
			&\quad +\left( p\log(p)^{\frac{3}{2}h-4} \log(T)\log(Tp)\right)^{\frac{1}{h-2}}T^{-\frac14} +T^{\frac12-c\kappa} \left(p^{\frac{1}{h+1}} \sqrt{1\vee \log(p)}\right)\Big].
		\end{align*}
		
	\end{Lemma}
	\subsubsection{Results on high-dimensional Gaussian vectors}
	The following two lemmas are direct consequences of Lemmas A.4 and A.5 and Remark A.8 in \cite{lederer2021estimating}. (Note that the lemmas in \cite{lederer2021estimating} themselves follow from results in \cite{chernozukhov2013gaussian} and \cite{chernozhukov2015comparison}.)
	\begin{Lemma}\label{lm.glv1}
		Let $G:=(G_1,\dots,G_p)^\top$ be a mean zero $p$-dimensional Gaussian vector. Suppose that there exist constants $c_3,C_3$ such that $c_3\le \E[G_j^2]\le C_3$ for all $j\in[p]$, then, for every $z,\delta>0$, we have 
		$$\P\left(\left|\left\|G\right\|_\infty -z\right|\le\delta\right)\le C\delta\sqrt{1\vee \log(2p/\delta)},$$
		where $C>0$ depends only on $c_3,C_3$. 
	\end{Lemma}
	
	\begin{Lemma}\label{lm.glv2}
		Let $G:=(G_1,\dots,G_p)^\top$ and $G':=(G'_1,\dots,G_p')^\top$ be two mean zero $p$-dimensional Gaussian vectors with respective covariance matrices $\Sigma^{G}$ and $\Sigma^{G'}$. Define $\delta= \left\|\Sigma^{G}-\Sigma^{G'}\right\|_\infty$. 
		Suppose that there exist constants $c_3,C_3$ such that $c_3\le \E[G_j^2]\le C_3$ for all $j\in[p]$. Then, there exists a constant $C>0$ depending only on $c_3,C_3$ such that 
		$$\sup_{z\in\R}\left|\P\left(\left\|G\right\|_\infty\le z\right)-\P\left(\left\|G'\right\|_\infty\le z\right)\right|\le C\delta^{1/3}(1\vee 2\log(2p)\vee \log(1/\delta))^{1/3} (\log(2p))^{1/3}.$$
	\end{Lemma}
	
	\subsection{On the rate condition in statement \eqref{P-thii} of Theorem \ref{P-th}.}\label{sec.rate-cond}
	\subsubsection{Discussion}
	The power analysis in Theorem \ref{P-th} contains the rate conditon \begin{equation}\label{rate.th.suff}\sqrt{\frac{\log(T\vee p)}{T\wedge p}}=o_P\left(\frac1T\left\|U^\top U\beta^*\right\|_\infty\right),\end{equation} which we analyze in this subsection. First, we state the following Lemma, which contains sufficient conditions in terms of nonrandom quantities for \eqref{rate.th.suff} to hold (concerning \eqref{suff.thi}) and not hold (concerning \eqref{suff.thii}).
			\begin{Lemma}\label{lm.suff.th}Let the assumptions of Theorem \ref{P-th} hold. We have 
		\begin{enumerate}[\textup{(}i\textup{)}] 
					\item\label{suff.thi} If $\sqrt{\frac{\log(T\vee p)}{T\wedge p}}+ \frac{p^{\frac{4}{q}}}{\sqrt{T}}\|\beta^*\|_1=o\left(\|\Sigma \beta^*\|_\infty\right)$, then \eqref{rate.th.suff} holds. 
					\item\label{suff.thii} If $\|\Sigma \beta^*\|_\infty+\frac{p^{\frac{4}{q}}}{\sqrt{T}}\|\beta^*\|_1=o\left(\sqrt{\frac{\log(T\vee p)}{T\wedge p}}\right)$, then \eqref{rate.th.suff} does not hold. 
		\end{enumerate}

		\end{Lemma}
		Lemma  \ref{lm.suff.th} is proved in Section \ref{sec.proof.lm.suff.th}. This allows us to give examples of sequences of $\beta^*$ and $\Sigma$ such that \eqref{rate.th.suff} holds or does not hold under our assumptions.\\
		
		
		\noindent \textbf{Example where \eqref{rate.th.suff} holds.} Let $\beta^*=(b,0,\dots,0)$ and $\Sigma_{11}\ge\underline{\sigma}$, where $b\ne 0$ and $\underline{\sigma}>0$ are constants that do not depend on $T$. Assume also that $p=o(T^{\frac{q}{8}})$. Then, we have $\|\Sigma \beta^*\|_\infty\ge \Sigma_{11} |b|\ge \underline{\sigma}|b|$ and, therefore, by the fact that $p=o(T^{\frac{q}{8}})$, we obtain
		$$\sqrt{\frac{\log(T\vee p)}{T\wedge p}}+ \frac{p^{\frac{4}{q}}}{\sqrt{T}}\|\beta^*\|_1=\sqrt{\frac{\log(T\vee p)}{T\wedge p}}+ \frac{p^{\frac{4}{q}}}{\sqrt{T}}|b|=o(\underline{\sigma}|b|)= o\left(\|\Sigma \beta^*\|_\infty\right).$$
		By Lemma \ref{lm.suff.th} \eqref{suff.thi}, this shows that \eqref{rate.th.suff} holds. \\
		
\noindent \textbf{Example where \eqref{rate.th.suff} does not hold.} Let $\beta^*=\left(\left(\frac{p^{\frac{4}{q}}}{\sqrt{T}}\right)^{-1}\frac{1}{T\wedge p},0,\dots,0\right)$ and $\max_{j\in[p]} |\Sigma_{1j}|=O(1)$. Assume also that $p=o(T^{\frac{q}{8}})$. Then, we have $$\|\Sigma \beta^*\|_\infty\le \left(\max_{j\in[p]} |\Sigma_{1j}|\right)\left(\frac{p^{\frac{4}{q}}}{\sqrt{T}}\right)^{-1}\frac{1}{T\wedge p}$$ and, therefore,  we obtain
		$$\|\Sigma \beta^*\|_\infty+\frac{p^{\frac{4}{q}}}{\sqrt{T}}\|\beta^*\|_1\le  \left(\max_{j\in[p]} |\Sigma_{1j}|\right)\left(\frac{p^{\frac{4}{q}}}{\sqrt{T}}\right)^{-1}\frac{1}{T\wedge p}+\frac{1}{T\wedge p} = o\left(\sqrt{\frac{\log(T\vee p)}{T\wedge p}}\right).$$
		By Lemma \ref{lm.suff.th} \eqref{suff.thii}, this shows that \eqref{rate.th.suff} does not hold. 	

		\subsubsection{Proof of Lemma \ref{lm.suff.th}} \label{sec.proof.lm.suff.th}	
		
		First, we apply Lemmas \ref{lm.2exp} and \ref{lm.nagaevsup} to $Z_t=\left(\left(u_{tj}u_{tk}-\E\left[u_{tj}u_{tk}\right]\right)_{j=1}^p\right)_{k=1}^p$ (it can be shown that the conditions of these lemmas hold following the arguments of the proof of Lemma \ref{prop.LV1}). These Lemmas yield
		\begin{equation}\label{70624}\left\|\frac{U^\top U}{T}-\Sigma\right\|_\infty\le \max_{j,k\in[p]}\left|\frac1T\sum_{t=1}^T u_{tj}u_{tk}-\E\left[u_{tj}u_{tk}\right]\right|=O_P\left(\frac{p^{\frac{4}{q}}}{\sqrt{T}}\right).\end{equation}
		
		Next, we show \eqref{suff.thi}. By the triangle inequality, we have 
	$$\left\|\Sigma\beta^*\right\|_\infty\le  \left\|\left(\frac{U^\top U}{T}-\Sigma\right)\beta^*\right\|_\infty +\left\|\frac{U^\top U}{T}\beta^*\right\|_\infty .$$
	By Hölder's inequality applied to each component of the vector $\left(\frac{U^\top U}{T}-\Sigma\right)\beta^*$, it holds that 
	$$ \left\|\left(\frac{U^\top U}{T}-\Sigma\right)\beta^*\right\|_\infty\le \left\|\frac{U^\top U}{T}-\Sigma\right\|_\infty\|\beta^*\|_1.$$
	This leads to $$\left\|\frac{U^\top U}{T}\beta^*\right\|_\infty \ge  \|\Sigma \beta^*\|_\infty -\left\|\frac{U^\top U}{T}-\Sigma\right\|_\infty\|\beta^*\|_1.$$
		Hence, by \eqref{70624} and since, under \eqref{suff.thi}, it holds that
		 $\sqrt{\frac{\log(T\vee p)}{T\wedge p}}+ \frac{p^{\frac{4}{q}}}{\sqrt{T}}\|\beta^*\|_1=o\left(\|\Sigma \beta^*\|_\infty\right)$, we have 
		\begin{align*}\sqrt{\frac{\log(T\vee p)}{T\wedge p}}&=o\left(\|\Sigma \beta^*\|_\infty- \frac{p^{\frac{4}{q}}}{\sqrt{T}}\|\beta^*\|_1\right) \\
		&=o_P\left(\|\Sigma \beta^*\|_\infty -\left\|\frac{U^\top U}{T}-\Sigma\right\|_\infty\|\beta^*\|_1\right)\\
		&=o_P\left(T^{-1}\left\|U^\top U\beta^*\right\|_\infty\right),\end{align*}
		which proves \eqref{suff.thi}.

Finally, we show \eqref{suff.thii}. By a reasoning similar to that of the proof of \eqref{suff.thi}, we have 
	\begin{align*}\left\|\frac{U^\top U}{T}\beta^*\right\|_\infty &\le  \left\|\left(\frac{U^\top U}{T}-\Sigma\right)\beta^*\right\|_\infty +\left\|\Sigma\beta^*\right\|_\infty\\
	& \le  \left\|\frac{U^\top U}{T}-\Sigma\right\|_\infty\|\beta^*\|_1 +\left\|\Sigma\beta^*\right\|_\infty.
	\end{align*}
Hence, by \eqref{70624} and since, under \eqref{suff.thii}, it holds that
		 $\|\Sigma \beta^*\|_\infty+\frac{p^{\frac{4}{q}}}{\sqrt{T}}\|\beta^*\|_1=o\left(\sqrt{\frac{\log(T\vee p)}{T\wedge p}}\right)$, we have
		\begin{align*}\left\|\frac{U^\top U}{T}\beta^*\right\|_\infty&\le  \left\|\frac{U^\top U}{T}-\Sigma\right\|_\infty\|\beta^*\|_1 +\left\|\Sigma\beta^*\right\|_\infty \\
		&=O_P\left(\|\Sigma \beta^*\|_\infty+\frac{p^{\frac{4}{q}}}{\sqrt{T}}\|\beta^*\|_1\right)\\
		&=o_P\left(\sqrt{\frac{\log(T\vee p)}{T\wedge p}}\right),\end{align*}
		which proves \eqref{suff.thii} (i.e., that \eqref{rate.th.suff} cannot hold).

	\bibliographystyle{agsm}
	\bibliography{bibliography}
	
	
	
	
\end{document}





Let us discuss exponential Orlicz norms. For $\theta>0$, we let $$\normiii{Z}_{e^{\theta}}=\inf\left\{C>0: \ \E\left[\psi_\theta\left(\frac{|Z|}{C}\right)\right]\le 1\right\},$$ 
where $\psi_\theta$ is the convex hull of the mapping $x\in\R_+\mapsto \exp(x^\theta)-1$.
We have the following useful Lemmas, which are direct consequences of Lemmas S.20 and S.21 in \cite{fan2021bridging}.
\begin{Lemma} If there exists constants $C_1,C_2>0$  such that, for all $z>0$, we have
$$\P(|Z|>z)\le C_1\exp\left(\left(\frac{z}{C_2}\right)^\theta\right),$$
then $\normiii{Z}_{e^{\theta}}<C$ for some constant $C>0$ depending only on $C_1,C_2$.
\end{Lemma}
\begin{Lemma} If there exists constants $C_1,C_2>0$  such that, for all $z>0$, we have
$$\P(|Z|>z)\le C_1\exp\left(\left(\frac{z}{C_2}\right)^\theta\right),$$
then $\normiii{Z}_{e^{\theta}}<C$ for some constant $C>0$ depending only on $C_1,C_2$.
\end{Lemma}




