% convert this file using pdflatex
\documentclass{article}
\RequirePackage[a4paper]{geometry}
\pdfoutput=1 
\geometry{top=25mm,bottom=25mm,left=25mm,right=25mm,nohead,nofoot,includeheadfoot}
\pagestyle{empty}
\usepackage{mathptmx,graphicx}
\usepackage{siunitx}
\usepackage{amsmath}
\usepackage{amssymb}
\usepackage{amsthm}
\usepackage{abstract}
\usepackage{lineno}

\usepackage[
style=nature,
]{biblatex}
\addbibresource{ref.bib} %Import the bibliography file

\begin{document}

\begin{center}
{\Large\bfseries
Time-domain Compressed Sensing \par}
\vspace{3ex}
{\bfseries
Kilian Scheffter$^{1,2}$, Jonathan Will$^{1,2}$, Claudius Riek$^{3}$, Herve Jousselin$^{4}$, Sébastien Coudreau$^{4}$, Nicolas Forget$^{4}$, Hanieh Fattahi$^{1,2}$\par}
{\footnotesize\itshape
1. Max Planck Institute for the Science of Light, Staudtstr. 2, 91058 Erlangen, Germany\\
2. Friedrich-Alexander University Erlangen-Nürnberg, Staudtrstr. 7, 91085 Erlangen, Germany\\
3. Zurich Instruments Germany, Mühldorfstraße 15, 81671 Munich, Germany\\
4. Fastlite, rue des Cistes 165, 06600 Antibes, France
\par}
\vspace{3ex}
\end{center}

\begin{abstract}
Ultrashort time-domain spectroscopy, particularly field-resolved spectroscopy, are established methods for identifying the constituents and internal dynamics of samples. However, these techniques are often encumbered by the Nyquist criterion, leading to prolonged data acquisition and processing times as well as sizable data volumes. To mitigate these issues, we have successfully implemented the first instance of time-domain compressed sensing, enabling us to pinpoint the primary absorption peaks of atmospheric water vapor in response to terahertz light transients that exceed the Nyquist limit. Our method demonstrates successful identification of water absorption peaks up to 2.5 THz, even for sampling rates where the Nyquist frequency is as low as 0.75 THz, with a mean squared error of $12\times 10^{-4}$. Time-domain sparse sampling achieves considerable data compression while also expediting both the measurement and data processing time, representing a significant stride towards the realm of real-time spectroscopy.
		
		\noindent\textbf{Keywords:} sparse sampling, field-resolved spectroscopy, time-domain spectroscopy, compressed sensing, real-time detection
\end{abstract}



\section*{Introduction}

Detailed description of the constituent and internal dynamics of matter is mirrored in its transient response to an external field. Resolving and monitoring the encoded information in the electric field of the ultrashort excitation field or in the time-dependent changes of the optical properties of the sample, provides deep understanding and insights of matter \cite{wang2012molecular,coutaz2018principles}. The availability of few-cycle pulses has not only advanced various spectroscopic methods such as pump-probe spectroscopy and Fourier transform spectroscopy, but has also led to the emergence of innovative techniques such as dual-comb spectroscopy and field-resolved spectroscopy \cite{Zewail2000, Keilmann:04, Schiller:02, Picqué2019, Lanin2014, doi:10.1063/1.114909, timmers2018molecular, RevModPhys.50.607, doi:10.1080/05704920701829043, Pupeza2020, doi:10.1126/sciadv.aaw8794, Herbst_2022}.

Employing ultrashort pulses offers two distinct advantages for spectroscopic applications (Fig. \ref{Fig.1}). Firstly, their broad spectral bandwidth enables simultaneous data acquisition of the sample, eliminating the need for repeated measurements or laser tuning. With high-bandwidth acquisition, prior knowledge of the sample is not required, as all available information can be extracted from the measurement during post-processing. Secondly, their extreme temporal confinement allows for temporal gating of the sample's response from the excitation pulses. This response, which is enriched with comprehensive spectroscopic information, lasts from tens of femtoseconds to nanoseconds and is commonly probed by a shorter pulse at various time intervals. When combined with additional temporal or spatial dimensions, such as multi-dimensional coherent spectroscopy \cite{smallwood}, four-dimensional imaging \cite{Cocker2016}, and hyperspectral imaging \cite{Vicentini2021}, ultrashort spectroscopic techniques render quantitative, multivariate characterization of the sample under scrutiny and facilitate the identification of unknown constituents. However, real-time measurements are prevented due to the need to record the high bandwidth spectrum at each pixel image and time delay, which results in a prohibitively long acquisition time to attain sufficient signal-to-noise ratio. The measurement speed is limited to i) the required number of sample points dictated by the Nyquist-Shannon criteria, ii) the speed of spatio-temporal scanning, and iii) transportation and storage speed of the measured data. Although the use of short, high-bandwidth excitation pulses offers significant benefits, such as simultaneous data acquisition, the response of the sample is often a linear combination of $K$ basis vectors leading to a $K$-sparse frequency domain. To minimize data redundancy and obtain the desired spectral content with the minimum amount of data necessary, it is necessary to apply algorithms to the acquisition process.
%
% Figure environment removed
%
According to the Nyquist-Shannon sampling theorem, a signal having a frequency $f$ needs to be sampled at a minimum sample rate $f_\text{samp}$ of twice the frequency of the signal ($f_\text{samp} \geq 2 \cdot f$). This criterion sets a minimum requirement for the amount of measured data necessary for a successful sampling. Compressed Sensing can circumvent this fundamental barrier by exploiting sparsity and incoherence of the signal, enabling the recovery of the original signal from fewer samples or measurements while preserving its quality \cite{Edgar2019,Rani2018,Candes2008,Candes2006,Donoho2006,Candes2006_2}. The prior knowledge of a signal's sparsity enables the formulation of an optimization problem that allows for the reconstruction of the signal using a reduced number of sampling points below the Nyquist-Shannon criteria. The sparsity of the measured spectroscopic information of various materials has been discussed by analytical treatment on over-sampled measurements \cite{Ostic2021,TAKIZAWA2020103042,PhysRevApplied.15.024032,Katz2010,Kaestner2018} or analytical compression prior to storage or transmission \cite{Kawai2021,PhysRevApplied.18.024025}. However, the requirements of incoherence and random sampling have made real-time, time-domain sparse sampling challenging. In this work we demonstrate real-time, field-resolved compressed sensing of water vapour molecules beyond the Nyquist criteria. The reconstruction of the absorption spectrum of water vapor sets clear boundaries on the required sparsity of a sample for the effectiveness of compressed sensing, as the absorption spectrum includes both high cross-section peaks, and low amplitude adjacent peaks. Our approach is enabled by randomly sampling, rapidly scanning delay line, enabling sensitive, real-time sample analysis.

\section*{Results}
\subsection*{Experimental concept}
%
Fig. \ref{Fig.2}a shows the experimental setup for terahertz (THz) field-resolved compressed sensing. $\SI{2.1}{mJ}$, $\SI{54}{fs}$ pulses of a Ti:Sapphire amplifier operating at $\SI{1}{kHz}$ are used to generate THz transients via optical rectification in a $\SI{1}{mm}$-thick ZnTe crystal (see Fig. \ref{Fig.2}b and c). The THz pulses propagated through a box filled and sealed with water vapour molecules at 50\% concentration. To characterize the electric field of the THz pulses, an electro-optic sampling (EOS) setup incorporating
a 0.1 mm-thick ZnTe was developed. 
%i calculated the 40uW based on the ND filters I used. I never measured it...
$\SI{40}{\micro\watt}$ of the amplifier's output power is used to probe the THz pulses at the EOS stage, where the polarization changes of the probe pulses due to interaction with THz electric field strength is detected in an ellipsometer incorporating balanced photo-diodes. Afterwards, the detected signal is fed into a lock-in amplifier. A mechanical chopper is used to modulate the repetition frequency of the THz pulses at $\SI{500}{Hz}$. The modulated signal in combination with a boxcar filter is used to eliminate systematic drifts of the measurement. Moreover, the box-car filter of the lock-in amplifier ensures the high bandwidth detection of the signal during the random scanning. For real-time, random scanning of the probe pulses over the water vapour's molecular response, an acousto-optical delay line with kHz scanning rates was integrated into the probe's beam path \cite{Kaplan2002, schubert2013rapid}. The acousto-optic delay line allows for arbitrary relative time delay between the THz pulse and probe pulses and shot-to-shot random scanning of the electric field. Data acquisition is performed by the lock-in amplifier triggered by the radio frequency control signals from the acousto-optic delay line. Data acquisition trigger (green) and synchronization paths (red) are indicated in Fig. \ref{Fig.2}a by dashed lines. The time-delay module has the refresh time of $\SI{2}{ms}$ at an arbitrary temporal position within its scanning range. In order to capture the complete molecular response encoded in the THz field, which lasts for tens of picoseconds, a mechanical delay-line was added to the acousto-optic delay line, extending the scanning range of $\SI{6400}{fs}$ for a single acousto-optical delay line to $>\SI{40}{ps}$. Alternatively, the scanning range can be extended by coupling multiple acousto-optic modulators.  
%
% Figure environment removed
%
\subsection*{Reconstruction strategy}
Two categories of compressed sensing algorithms were investigated for the reconstruction of the electric field of the randomly sampled molecular response of water: convex compressed sensing and greedy algorithm (see Fig. \ref{Fig.3}). Basis Pursuit Denoising (BPD) and Lasso algorithms \cite{BergFriedlander:2008,spgl1site} from the family of convex compressed sensing algorithms were used for reconstruction, due to their noise robustness and low reconstruction error for signals with moderate sparsity \cite{7868430,Rani2018}. When comparing the mean squared error of the reconstruction for both algorithms, it was noted that the Lasso algorithm has a unique global minimum at a specific threshold value in the optimization problem, while the BPD algorithm demonstrates a negligible mean squared error across a range of threshold parameters ($\tau$ and $\sigma$ in supplementary information). BPD is also recognized for its ability to function as a filter by eliminating noise from oversampled signals \cite{doi:10.1137/S003614450037906X}. As finding a reference for arbitrary measurements to optimize the reconstruction threshold value is not always feasible, BPD was selected for further analysis. From the second category, the Stagewise Orthogonal Matching Pursuit (StOMP) greedy algorithm was chosen and developed owing to its fast computation and robustness to noise \cite{Rani2018}. To address the issue of significant amplitude reconstruction error in StOMP, a Nonlinear Least Square (NLS) algorithm has been incorporated into the StOMP reconstruction. Additionally, StOMP necessitates a distinct reconstruction threshold value at every iteration, which is inefficient and impractical. Consequently, an approach based on the Interquartile Range method \cite{dekking2005modern} was developed to calculate the threshold parameter automatically (see supplementary information for more detail). 
%
% Figure environment removed
%
\subsection*{Real-time field-resolved compressed sensing}

As a proof of principle experiment for time domain compressed sensing, we chose to resolve the spectroscopic information of atmospheric water vapor in response the THz excitation pulses centered at 1 THz. Among the most abundant molecules in the atmosphere, only water possesses a permanent dipole in this spectral range \cite{cox2015allen}. As a result, the spectral coverage of THz excitation pulses serves as a filter, isolating the study to only water vapor and its isotopes \cite{vanExter89}. The ambient air's absorption spectrum in this spectral range is characterized by a high density of absorption peaks \cite{CUI20153533}, which makes it an ideal platform for assessing the efficacy of compressed sensing in reconstructing moderately sparse spectra while also highlighting its limitations in reconstructing prominent absorption peaks and adjunct frequencies. Moreover, such realization holds promise for real-time gas detection in open-air environments \cite{Sitnikov_2019, 4337845}. For a comparison between conventional sampling and Compressed sensing, two categories of measurements with different number of sampling points were performed on the water vapor response: i) linear scans with sampling points at the time intervals of $t_{n+1} = t_{n} +\Delta t$, and ii) randomly distributed sampling points. Each measurement was repeated for ten times. Hereby, the location of the randomly located sample points of the second category varies for every measurement. To generate a reference, ten oversampled, linearly scanned measurements with $N = 2034$ sample points are averaged. This reference trace is shown by the blue curve in Fig. \ref{Fig.4}a. For the spectroscopic analysis, only the sparse molecular response in the decaying tail of the THz light transient, ranging from $5$ to $\SI{35}{ps}$ was considered. The upper limit of this range was determined by the internal reflection of the THz light transient in the sampling crystal, occurring at $\SI{36}{ps}$. The sampling points were reduced from 2023 to 46 data points in eighteen discrete steps and the reconstruction was performed by using both BPD and StOMP algorithms.
%
% Figure environment removed
%

Fig. \ref{Fig.4}a displays a measurement at an extreme limit, with the minimum random number of sample points at $N=46$, indicated by red dots, along with the reconstructed fields obtained using the BPD algorithm (in orange) and the StOMP algorithm (in green). The averaged Fourier transformations of ten reconstructions, as well as the Fourier transformation of a measurement with uniformly distributed $N=46$ sampling point are shown in Fig. \ref{Fig.4}b. The uniformly sampled waveform with $N=46$ has temporal steps of $\Delta t=\SI{666}{fs}$, corresponding to a Nyquist frequency of $\SI{0.75}{THz}$. The resulting spectrum clearly displays aliasing and incorporate fake frequencies at below $\SI{0.5}{THz}$. However, the spectrum obtained from the compressed sensing reconstructed waveform successfully retrieves water absorption peaks beyond the Nyquist limit, up to $\SI{2.2}{THz}$. To achieve a similar outcome using conventional sampling, the number of sampling points would need to be increased by at least three times. 

To quantitatively evaluate the performance of compressed sensing, we calculated the mean squared error between the oversampled reference and the reconstructed waveforms via BPD and StOMP, respectively. Fig. \ref{Fig.4}c shows the average and standard deviation of the mean squared error as a function of $N$ for each algorithms, while the purple curve represents the corresponding Nyquist frequency. We determined the value of the measurement noise by averaging the mean squared error of the conventionally sampled waveforms with respect to the oversampled reference for each $N$. As this value exhibits only slight variations across different $N$, it is indicated by a horizontal dotted line  (see supplementary information for more details). Not only does the BPD algorithm outperform the StOMP method, but for larger values of $N$, the waveforms reconstructed using BPD show a mean squared error that is lower than the measurement noise. This is due to the tendency of BPD to converge to a solution with maximum sparsity, which acts as a filter rejecting signal amplitudes other than absorption frequencies. For lower values of $N$, the mean squared error increases non-linearly, reaching $12\times 10^{-4}$ at $N = 46$. The low standard deviation between the ten different measurements for each value of $N$ demonstrates the stability of the reconstructions. Fig. \ref{Fig.4}d and e summarize the performance of the BPD and StOMP reconstructions for measurements at various $N$, where the averaged Fourier transformation of the ten compressed sensing reconstructed waveforms at different $N$ are presented. The amplitude at each $N$ is normalized to one, and the area bordered by the purple line denotes the recovered frequencies beyond the Nyquist limit. The spectral region ranging from 1 THz to 1.5 THz is crucial for this investigation, due to the presence of a large number of absorption lines with high cross-section and the high spectral density of the THz excitation pulses. Remarkably, the primary absorption peaks in this region are exceptionally well-recovered beyond the Nyquist limit with excellent quality. When dealing with lower values of $N$, the reconstruction of absorption lines with lower amplitude and the adjunct frequencies are more prone to noise, even for those below the Nyquist limit. A comparison of panel d) and e) reveals that while the less complex StOMP algorithm has the potential of a fast reconstruction speed compared to BPD, it results in noisier reconstructions of both the low amplitude frequencies below the Nyquist limit and the absorption lines beyond the Nyquist limit.

\section*{Discussion}

Ultrashort time-domain spectroscopy and in particular field-resolved spectroscopy have been gold standards for accurately identifying the constituents and dynamics of matter without the need for labels. %\cite{doi:10.1126/science.aac9788,cite visible range, RevModPhys.81.163, pump-probe}.
Despite numerous attempts to increase measurement speed in ultrafast spectroscopy \cite{Weigel2021-kh,schubert2013rapid, Mohler:17}, real-time measurements remain challenging due to a prolonged acquisition time, significant data volume, and processing time. For the first time, this study showcases the use of Compressed Sensing with rapidly sampled field resolved spectroscopy, providing a solution to overcome these limitations. In particular, when short excitation laser pulses interacting with matter, the signal carrying information on this interaction are temporally separated from the main pulse turning the characterization problem to a compressed sensing problem. Crucial for compressed sensing is the possibility for random sampling which has been achieved by employing an acousto-optic delay line and a box-car filter for broadband data acquisition with a high dynamic range. The relatively high density of absorption peaks of ambient air water in terahertz spectral range provides an ideal platform for assessing the efficacy of compressed sensing in reconstructing moderately sparse spectra while also highlighting its limitations. We report on resolving the absorption frequencies three times higher than the Nyquist limit. The reconstruction success rate for the most prominent absorption peak is $\SI{100}{\percent}$ showing the potential of successfully reconstructing the absorption for even lower number of sampling points. However, the success rate decreases below $\SI{90}{\percent}$ for adjacent frequencies, emphasizing the importance of sparsity of the signal for full reconstruction (see supplementary information). Additionally, we demonstrate that compressed sensing below the Nyquist criteria can suppress measurement noise, making it valuable not only for speeding up measurement time but also for denoising sensitive measurements.

Determining when a sufficient number of sample points have been acquired for measuring an unknown spectrum is a major challenge in compressed sensing. However, due to the ability of random sampling, the number of sampling points increases uniformly across the entire sampling range, allowing for the observation of the convergence of the reconstructed waveform or spectrum. With our compressed sensing algorithm possessing processing times below $\SI{3}{ms}$, real time analysis of the convergence is possible. 
% FPGA would improve but I have no number. Probably Jonathan knows more about the speed of algorithms 
%Moreover, by choosing faster algorithms and by implementing these onto fast hardware,??? we expect orders of magnitude faster algorithms making real time analysis for higher repetition rate systems possible. (can you put the last line in some numbers? which hardware can be improved? How you ended up with a value of order of magnitude?)
As the refresh time of the acousto-optic delay line is at $\SI{30}{kHz}$, individual random shot to shot sampling can be performed for laser pulses below this repetition rates. For higher repetition rates, partial random scanning can be performed. Here each launched acoustic wave packet inside the acousto optical delay line delays an incoming pulse train with equal spaced delays \cite{schubert2013rapid}, while the relative delay between different scans are randomized showing promise to introduce compressed sensing. 

Real-time ultrafast spectroscopy is of crucial importance in various fields. Our innovative technique can greatly accelerate data acquisition in ultrafast spectroscopy, particularly in higher-dimensional analyses, by data volume minimization, signal acquisition time reduction, and a contraction in the required number of measurements in each dimension. These advances alleviate the requirements for specialized measurement instruments, offering a range of benefits that extend well beyond traditional spectroscopy applications. For instance simplifying the handling of fragile specimens, enabling real-time environmental monitoring of short-lived pollutants, and real-time, open-air diagnostics of toxic and hazardous gases.
%we should add some references here over time. We should also cite less FK and check other people :)
\section*{Methods}
The laser delivers $\SI{54}{fs}$ pulses centered at $\SI{810}{nm}$ with $\SI{2.1}{mJ}$ energy at $\SI{1}{kHz}$ repetition rate. In the setup, the diameter of the beam is first reduced from $\SI{4.1}{mm}$ to $\SI{2.8}{mm}$ at ${1/e^2}$ via a Gallilean beam expander with the focal length of $f = \SI{-100}{mm}$ and $f = \SI{150}{mm}$. The beam is then split into two paths via a 90:10 unpolarized beamsplitter. The reflected beam containing $\SI{90}{\percent}$ of energy is used for THz-generation and is chopped at a frequency of $\SI{500}{\hertz}$ via a mechanical chopper. Therefore, every second pulse is blocked by the chopper. This modulation frequency in combination with a lock-in amplifier is used to reduce the systematic noise in the measurement. Afterwards, the laser pulses are sent through a THz-crystal to generate THz-pulses via optical rectification. After the THz generation, a $\SI{1.6}{mm}$ thick Silicon plate is used to filter the $\SI{800}{nm}$ pump-beam, while transmitting the THz-beam. The entire THz-beam path is placed inside a purge-box, filled with nitrogen to control the humidity level. The THz-pulses are focused on the detection crystal with a pierced parabola with the focal length of $f = \SI{50.3}{mm}$, while spatially overlapped with the probe beam.

The $\SI{10}{\percent}$ transmitted $\SI{800}{nm}$ beam is used to probe the electric field of the THz-pulses. An acousto optical delay-line (Dazzler) is implemented in the probe beam path for fast and random delay scanning. In front of the Dazzler, the polarization of the probe is flipped via a half-waveplate to provide the required input polarization. The diffracted output beam from the Dazzler has an orthogonal polarization relative to the input beam. Eventually the probe beam is focused through a pierced parabola to the detection crystal with a plano-convex lens ($f = \SI{130}{mm}$). In the detection crystal, the THz-beam co-propagates with the probe beam with altered polarization depending on the instantaneous THz field strength via the Pockels-effect. After collimation of the probe beam with an off-axis parabola ($f = \SI{100}{mm}$), its polarization status is analyzed with an ellipsometric detection scheme consisting of a quarter-waveplate, a Wollaston prism, and a balanced photodetector. 
The signal of the balanced photodetector is fed into a lock-in amplifier using a boxcar filter to eliminate systematic drifts. Data acquisition is performed by the lock-in amplifier triggered by control signals leaving the Dazzler radio frequency module. For more details we refer the readers to the supplementary information. 

\section*{Acknowledgment}

The authors thank Philip J. Russell, Francesco Tani, Martin Butryn, Stefan Malzer and Heidi Potts for their support. 

\section*{Author Contributions}
H.F. envisioned the experiment. K.S. performed the measurements. H.J, S.C., N.F., K.S., H.F. designed the random scanning. C.R. contributed in devising the strategy for data acquisition. K.S., J.W., H.F. performed the data reconstruction and analysis. All authors reviewed and contributed to the manuscript text.

\section*{Funding}
This work was supported by the research funding from Max Planck Society. K.S. is part of the Max Planck School of Photonics supported by the German Federal Ministry of Education and Research (BMBF), the Max Planck Society and the Fraunhofer Society.  

\section*{Disclosures}
The authors declare no conflicts of interest. 

%\setlength\parindent{0pt}\vspace{2ex}

%\textbf{References}
%\footnotesize
%[1] A. Herbst {et.al.}, ``Recent advances in petahertz electric field sampling,'' J. Phys. B: At. Mol. Opt. Phys. \textbf{55}, 172001 (2022).

%[2] O. Schubert {et.al.}, ``Rapid-scan acousto-optical delay line with 34 kHz scan rate and 15 as precision,'' Optics Letters \textbf{38}, 15, 2907-2910 (2013).

%[3] A. Weigel {et.al.}, ``Ultra-rapid electro-optic sampling of octave-spanning mid-infrared waveforms,'' Optics Express \textbf{29}, 13, 20747-20764 (2021).

%[4] Sushovit Adhikari {et. al.}, ``Accelerating Ultrafast Spectroscopy with Compressive Sensing,'' Phys. Rev. Applied \textbf{15}, 024032 (2021)
%shows time-domain spectroscopy but on resampled data (only data analysis) and not on the molecular response

%[5] Rachel Ostic and Jean-Michel Ménard, ``Speeding Up Ultrafast Spectroscopy,'' Physics \textbf{14}, 23 (2021) %just a letter

%[6] Silberberg ''Compressive Fourier Transform Spectroscopy'' 2010
% conference contribution, show cs on resampled data of single-pulse CARS, argues of spectral-superresolution (smaller time span but still same spectral resolution)

%[7] Takuro Ideguchi ''Compressive dual-comb spectroscopy'' 2021
% report on data compression to solve data size problem in dual-comb spectroscopy

%[8] Bernd Kästner ''Compressed sensing FTIR nano-spectroscopy and nono-imaging''
% resampled data

%[9] Keisuke Goda ''Compressed time-domain coherent Raman spectroscopy with real-time random sampling''
% they say 'real-time' sampling but they also report on resampled data. Also, the delay line is not really random, and it doesn't increase the number of sampling points across the entire scanning range. So you can't have a termination before it is scanned one time entirely
\printbibliography
\newpage
%

%\bibliographystyle{dinat}


\end{document}