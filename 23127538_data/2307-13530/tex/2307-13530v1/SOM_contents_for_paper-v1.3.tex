% This version does not duplicate figures that are in the body of the text.

Detailed occultation prediction tables and figures are available as Supplementary Online Material (SOM) in NASA's Planetary Data System Ring-Moon Systems node at \dataset[DOI: 10.17189/m9sk-g963]{https://doi.org/10.17189/m9sk-g963}. Users should download the entire repository to a local storage device, using the command {\tt wget}, freely available from \url{https://www.gnu.org/software/wget/}. To download the entire SOM contents, enter the following commands from the command line of a terminal:

{\tt cd destdir } (where {\tt destdir} is the local directory within which the SOM directory will reside)

{\tt wget -c -r -nH --cut-dirs=2 https://pds-rings.seti.org/rms-annex/french23\_occult\_pred/SOM/}

The top directory name is {\tt SOM/}. The approximate data volume is 28 GB.

The abbreviated directory structure of the {\tt SOM/} directory is shown below.
Each directory has its own {\tt aareadme.txt}, with its filename including its home directory. The entire SOM can be navigated by opening the {\tt index.html} file in a web browser. The {\tt SOM/docs/} directory contains two sample python programs to illustrate simple occultation searches of the SOM. They can be easily modified for more complex searches.

\input{SOM_directorystruct.txt.vb}
%
\subsection{\tt SOM/}
Contents of the {\tt SOM/aareadme.txt} file:
\input{README_SOM.txt.vb}
%
\subsection{\tt SOM/doc/}
Contents of the {\tt SOM/doc/aareadme.txt} file:
\input{README_doc.txt.vb}
%
\subsection{\tt SOM/events/}
Contents of the {\tt SOM/events/aareadme.txt} file:
\input{README_events.txt.vb}      
%
\subsection{\tt SOM/tables/}
Contents of the {\tt SOM/tables/aareadme.txt} file:
\input{README_tables.txt.vb}
%
\section{Examples of SOM files}
%[The following is reproduced in modified form from the published paper and is included here to provide self-contained documentation of the SOM.]

For each predicted occultation, the SOM includes visual  and tabular overviews of the observing circumstances. We illustrate these using the 2028-12-19 Uranus occultation. 

%Figure \ref{fig:examples}
%shows the first page of the summary file for the event: {\tt Uranus_2028-12-29T17_39_23.210_20230528a.pdf}.
%%, including the labeled observing sites and the anti-solar point as an open circle. In this instance, the mid-occultation time occurred just after sunrise at SAAO and just before sunrise at PIC, while Uranus was just rising as viewed from North and South America. Additional details of the event are shown in the figure labels: the closest approach time (C/A) truncated to 2022 Sep 13 05:09:54 UTC, the apparent geocentric coordinates of the target star in J2000 coordinates, the closest approach separation in arcsec (C/A 1.342$''$), the position angle on the sky of the closest approach point (PA), measured clockwise from north ($PA=163.50^\circ$), the apparent sky plane velocity of the target at the C/A time, relative to the geocenter (vsky), and the target distance from Earth ($D$=19.14 AU). For this occultation, we applied a correction to the nominal Uranus ephemeris in the form of an offset of the sky plane position of Uranus of $f_0=-688$ km and $g_0=-659$ km ($f_0$ increases eastward and $g_0$ increases northward). 
%%% Figure environment removed
%For Titan and Triton occultations, we include the projected path of the shadow across the Earth, as shown in Fig. \ref{fig:satelliteglobeexample}. %Observing sites are labeled using their three-letter codes.The solid black lines shows the shadow midline, marked by dots at 60 sec time intervals. The direction of the shadow motion is marked by the red arrow. The outline of the target shadow is shown as a red circle, and the projected shadow boundaries of the northern/southern limbs are shown as blue lines when they intersect the Earth and as red lines elsewhere. The antisolar point is marked by an open circle. This occultation earned an event type of {\it Pt} because it was predicted to be observable from at least one of our 13 topocentric sites (PMO, in this case), but not from the geocenter. Additional geometric details are included in the figure labels, as in Fig.~\ref{fig:globeexample}.
%%
%% % Figure environment removed
%For Jupiter, Saturn, Uranus, and Neptune occultations, we provide a figure showing the Earth view of the target and its ring system in the sky plane, along with occultation chords for any sites during the interval when the target and sun meet the altitude requirements enumerated above. Figure \ref{fig:examples} shows the Uranus sky plane for the 2022 Sept 13 event.
%% Each occultation chord is coded by the color of the observing site label. The solid dot on each chord marks the earliest point at which the occultation star is within the window of the figure and meets the sun and target altitude observability requirements. The geocentric chord (marked as Earth) is shown as a black dashed line. In this example, the SAAO chord intersects the ingress rings, but terminates before the planet occultation because of sunrise; the PAL chord begins just prior to atmospheric egress because prior to this time Uranus was too low to be observable. 
%% % Figure environment removed
%Additionally, for each occultation, we include a figure showing the altitude (elevation angle) of the target and sun above/below the horizon over the course of the event, as viewed from each topocentric site that satisfies the event observability constraints sometime within the vicinity of the closest approach time, as illustrated in Fig.~\ref{fig:altexample} for the 2022 Sep 13 Uranus occultation.
%% For each labeled site, the altitude of Uranus is shown as a solid line, and the altitude of the sun is shown as a dashed line. The time interval when the planet itself occults the star is shown as a thick solid line, and the times of individual ring occultation events are shown as dots. The lines are restricted to the times that meet the simultaneous requirements that the target be at least 5$^\circ$ above the horizon and the sun be at least 5$^\circ$ below the horizon. 
%The color coding of the observing sites matches that of the corresponding skyplane figure (Fig.~\ref{fig:skyplaneexample}).
%
%%% Figure environment removed

A single PDF file for each predicted occultation  includes key observational data and plotted figures showing the event geometry. Figure \ref{fig:pdfsummary} shows the summary page for this event.\footnote{\tt SOM/events/Uranus/2028/Uranus\_2028-12-29T17\_39\_23.210\_20230528a.pdf}
Both the view of Earth and the sky plane plots are included in the same SOM subdirectory. 
The summary page includes information about the target object and occultation star, along with other geometrical information defined in more detail below. This text is included in a separate plain text file on the SOM.\footnote{\tt SOM/events/Uranus/2028/Uranus\_2028-12-29T17\_39\_23.210\_20230528a.txt} At lower right, a finder chart image (created using the {\tt plot\_finder\_image} from the Python {\tt astronplan.plots} package) is shown, with the event star marked by crosshairs. 
%
At the bottom of the page, a convenient summary of the observability of the occultation from all 13 observing sites is included:
\begin{itemize}
\item The observing site code, name, and topocentric Earth location are shown. 
\item The observability of each individual ring event and the planet limb occultations is summarized. In time order, each of the ten Uranus rings is marked during ingress with a $+$ if the ring was observable, given the usual altitude constraints. 
\item The ingress and egress planet occultations are marked, followed in time order by the egress rings.\footnote{For some Saturn and Neptune ring occultations, the ingress and egress ring events all precede or follow the planet occultation, rather than being interrupted by the planet event, but for simplicity we retain the same format for all ringed planets.} For example, from PIC (Pic du Midi), all ingress and egress ring events were observable, but the grazing occultation chord missed the atmosphere. 
\item The next column lists the complete interval over which any marked events were predicted to be observable. 
\item We include a summary observed event code (OEcode) for each site, in the following format: {\tt PXYRxy}, where {\tt X} is set to {\tt i} if the target planet/satellite (denoted by {\tt P}) ingress limb event is observable and  {\tt Y} is set to {\it e}
if the egress limb event is observable. Similarly,  {\tt x} is set to {\tt i} if any ingress ring events (denoted by {\tt R }) are observable from the given site (unblocked by the planet and meeting the standard altitude criteria), and  {\tt y} is set to {\tt e}
if any egress ring events are observable. If the given events are not observable, the appropriate letters are set to {\tt n}. In this example, the OEcodes for the PIC and KAV observations are {\tt PnnRie} and {\tt PnnRne}, respectively; from KAV, only the outer five rings are observable during egress. An OEcode of {\tt PnnRnn} indicates that neither the planet nor any ring occultations were observable for the site in question, such as PAL for this example.
\end{itemize}
% Figure environment removed
For each site that has an OEcode indicating that a planet/target limb and/or ring occultation is observable, we include a separate page in the SOM PDF file that provides additional detailed information about the geometric circumstances of these events. Figure~\ref{fig:pdfsummaryPIC} shows this page for the predicted Pic du Midi observations of the 2028-12-29 Uranus event. Inset figures showing the Earth from Uranus and the altitude of the target and sun over time are included, available at full resolution in the SOM. The text shown includes details of the occultation event and included as a separate text file in the SOM.\footnote{{\tt SOM/events/Uranus/2022/Uranus\_2028-12-29T17\_39\_23.210\_PIC\_20230528a.txt}, where the event time in the filename is the geocentric C/A time.}

At the bottom of the page, we include detailed predictions for each ring event and planet limb that intersects the sky plane chord for the occultation as observed from this site:
\begin{itemize}
\item For each listed ring, we computed the ingress (I) and egress (E) predicted event times. (For Uranus, we use the full eccentric and inclined ring orbital elements for the rings, taken from \cite{French2023b}; for the other planets, we assume circular and equatorial ring orbits). 
\item The UTC time of each predicted event is given, along with the altitudes of the target object and sun at the event time, the ring plane radius probed (taking into account the orientation of the possibly inclined, eccentric ring at the observed time and accounting for general relativistic bending by the oblate planet), and the ring plane radial velocity, labeled as {\tt r-dot}, negative for ingress and positive for egress. 
\item If the planet occultation is observable, we include the predicted occultation time assuming the planetary shape/oblateness as specified in the kernel file {\tt pck00010.tpc}. Our atmospheric event times do not take into account the refractive bending of the atmospheric half-light ray that typically amounts to one atmospheric scale height. 
\item Ring events that are blocked by the planet are marked with a {\tt b} in cases where the rings are viewed nearly edge-on and the occultation sky plane chord intersects the rings only when in the shadow of the planet (not applicable in this instance). 
\item Events for which the target altitude is less than 5$^\circ$  or the sun's altitude is above $-5^\circ$ are marked by an {\tt x} to indicate that they are not observable. In this example, none of the  events are so marked.
\end{itemize}

% Figure environment removed

\section{Machine readable tables, LaTeX source and typeset files}
The complete prediction list for each target is contained in both machine-readable and typeset form. The {\tt SOM/tables/} directory contains subdirectories for each target, within which is a single machine-readable file in the form support by the American Astronomical Society journals and described at \url{https://journals.aas.org/mrt-overview/}.
Also included are the LaTeX source files used to typeset the tables for each target, and a PDF file containing the typeset tables. These make use of the document class {\tt aastex631.cls}, provided in each target subdirectory.

\section{Example occultation searches}
The body of this paper contains only a small subset of the full list of predicted events contained in the SOM. As part of the SOM documentation, we provide two example programs written in Python3 that perform searches of the entire database for occultations that match requested criteria. To run these example codes, users should first download the entire SOM repository to their local machines and then navigate to the {\tt SOM/doc/} directory. Both programs are provided as Python source files {\tt *.py} and as Jupyter notebooks {\tt *.ipynb}, with sample output files {\tt *.out} produced by running the codes in their default configurations. The codes are intended to be illustrative only, and can be modified to conduct more sophisticated searches.

\subsection{Example 1 - find selected occultations by geographical region}
In the first example, the user specifies the following search criteria:
\begin{itemize}
\item{The list of targets to search}
\item{The corresponding upper limits on the K magnitude for each target}
\item{The range of dates for the search}
\item{The geographical regions to search (see Table \ref{tbl:obslocs} in the main body of the paper)}
\end{itemize}
Additional options are included that control the output of the search. In its default mode, the output file produce by the program includes:
\begin{itemize}
\item{A list of the available quantities in the Python table read from the Machine Readable (MR) file for each target}
\item{A summary of the requested search criteria}
\item{For each occultation found, a listing of the summary text file of observing circumstances for all sites}
\item{The SOM pathnames for the summary PDF and text files for each event}
\item{The summary PDF file is opened for user viewing for each identified occultation}
\end{itemize}

In its tersest mode, the default search results in the following output:
%\scriptsize
\begin{verbatim}
********** Contents of  EventSearchExample1.out  ****************
Results of  EventSearchExample1.py
------------------------------------------------------------------------------
Results of search for occultations by Jupiter between 2024-01-01 and 2030-01-01
K magnitude limit 5
Visible from  N. Am., Eur.&N.Afr., S. Afr., S. Am., Oceania, E. Asia
 
  2027-02-22 11:12:09.02 Pgt     K= 4.926 is visible from N. Am., Oceania, E. Asia
 
------------------------------------------------------------------------------
Results of search for occultations by Saturn between 2024-01-01 and 2030-01-01
K magnitude limit 8
Visible from  N. Am., Eur.&N.Afr., S. Afr., S. Am., Oceania, E. Asia
 
  2024-08-06 17:12:21.22 PgtRgt  K= 7.716 is visible from N. Am., Oceania, E. Asia
  2029-07-19 00:57:19.09 PgtRgt  K= 7.241 is visible from Eur.&N.Afr., S. Afr., E. Asia
 
------------------------------------------------------------------------------
Results of search for occultations by Uranus between 2024-01-01 and 2030-01-01
K magnitude limit 10.9
Visible from  N. Am., Eur.&N.Afr., S. Afr., S. Am., Oceania, E. Asia
 
  2026-07-21 02:13:50.22 PgtRgt  K= 10.802 is visible from Eur.&N.Afr., S. Afr.
 
------------------------------------------------------------------------------
Results of search for occultations by Neptune between 2024-01-01 and 2030-01-01
K magnitude limit 11
Visible from  N. Am., Eur.&N.Afr., S. Afr., S. Am., Oceania, E. Asia
 
  2024-10-09 00:36:23.19 PgtRgt  K= 9.068 is visible from N. Am., Eur.&N.Afr., S. Afr., S. Am.
 
------------------------------------------------------------------------------
Results of search for occultations by Titan between 2024-01-01 and 2030-01-01
K magnitude limit 7
Visible from  N. Am., Eur.&N.Afr., S. Afr., S. Am., Oceania, E. Asia
 
  2029-03-04 02:37:38.25 Pt      K= 6.151 is visible from N. Am.
 
------------------------------------------------------------------------------
Results of search for occultations by Triton between 2024-01-01 and 2030-01-01
K magnitude limit 15
Visible from  N. Am., Eur.&N.Afr., S. Afr., S. Am., Oceania, E. Asia
 
  2029-06-11 05:23:26.00 Pt      K= 10.259 is visible from Eur.&N.Afr.
\end{verbatim}
%\normalsize
\subsection{Example 2 - find selected occultations by observing site}
In the second example, the user specifies the following search criteria:
\begin{itemize}
\item{The list of targets to search}
\item{The corresponding upper limits on the K magnitude for each target}
\item{The range of dates for the search}
\item{Specific observing sites for the search (see Table \ref{tbl:obslocs} in the main body of the paper)}
\end{itemize}
Additional options are included that control the output of the search. In its default mode, the output file produce by the program includes:
\begin{itemize}
\item{A list of the available quantities in the Python table read from the Machine Readable (MR) file for each target}
\item{A summary of the requested search criteria}
\item{For each occultation found, a listing of the summary text file of observing circumstances for all sites}
\item{The SOM pathnames for the summary PDF and text files for each event}
\item{The SOM pathnames for the individual event summary text files for each observing site}
\item{The summary PDF file is opened for user viewing for each identified occultation}
\end{itemize}

In its tersest mode,  the default search results in the following output:
%\scriptsize
\begin{verbatim}
********** Contents of  EventSearchExample2.out  ****************
Results of  EventSearchExample2.py
------------------------------------------------------------------------------
Results of search for occultations by Jupiter between 2024-01-01 and 2030-01-01
K magnitude limit 5
Visible from  IRTF, TEN, KPNO
 
  2027-02-22 11:12:09.02 Pgt     K= 4.926 is visible from IRTF, KPNO
 
------------------------------------------------------------------------------
Results of search for occultations by Saturn between 2024-01-01 and 2030-01-01
K magnitude limit 8
Visible from  IRTF, TEN, KPNO
 
  2024-08-06 17:12:21.22 PgtRgt  K= 7.716 is visible from IRTF
  2029-07-19 00:57:19.09 PgtRgt  K= 7.241 is visible from TEN
 
------------------------------------------------------------------------------
Results of search for occultations by Uranus between 2024-01-01 and 2030-01-01
K magnitude limit 10.9
Visible from  IRTF, TEN, KPNO
 
No events found with requested conditions
 
------------------------------------------------------------------------------
Results of search for occultations by Neptune between 2024-01-01 and 2030-01-01
K magnitude limit 11
Visible from  IRTF, TEN, KPNO
 
  2024-10-09 00:36:23.19 PgtRgt  K= 9.068 is visible from KPNO, TEN
 
------------------------------------------------------------------------------
Results of search for occultations by Titan between 2024-01-01 and 2030-01-01
K magnitude limit 7
Visible from  IRTF, TEN, KPNO
 
  2029-03-04 02:37:38.25 Pt      K= 6.151 is visible from KPNO
 
------------------------------------------------------------------------------
Results of search for occultations by Triton between 2024-01-01 and 2030-01-01
K magnitude limit 15
Visible from  IRTF, TEN, KPNO
 
  2029-06-11 05:23:26.00 Pt      K= 10.259 is visible from TEN
\end{verbatim}
%\normalsize


