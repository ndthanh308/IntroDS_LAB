\documentclass{amsart}
\usepackage{graphicx}
\usepackage{amssymb,amscd,amsthm,amsxtra}
\usepackage{latexsym}
\usepackage{epsfig}
\usepackage{esint}
\usepackage[colorlinks]{hyperref}
\usepackage[colorlinks]{hyperref}
\AtBeginDocument{
   \hypersetup{
    linkcolor=blue,
    citecolor=blue,
 }
}
\usepackage{cleveref}

\numberwithin{equation}{section}

\newtheorem{theorem}{Theorem}[section]
\newtheorem{lemma}{Lemma}[section]
\newtheorem{proposition}{Proposition}[section]
\newtheorem{corollary}{Corollary}[section]
\newtheorem{definition}[theorem]{Definition}
\newtheorem{problem}{Problem}
\newtheorem{remark}{Remark}[section]

\def\ds{\displaystyle}






\begin{document}

\title[Boundary Harnack in a slit domain]{The $C^{1,\alpha}$ boundary Harnack principle in a slit domain and its application to the Signorini problem}
\author{Chilin Zhang}
\address{School of Mathematical Sciences, Fudan University, Shanghai 200433, China}\email{zhangchilin@fudan.edu.cn}
\begin{abstract}
We prove the $C^{2,\alpha}$ regularity of the free boundary in the Signorini problem with variable coefficients. We use a $C^{1,\alpha}$ boundary Harnack inequality in slit domains. The key method is to study a non-standard degenerate elliptic equation and obtain a $C^{1,\alpha}$ Schauder estimate.
\end{abstract}
\maketitle

\section{Introduction}
\subsection{Basic theory in the Signorini problem}
The thin obstacle problem, or the Signorini problem, consists in minimizing the energy
\begin{equation*}
    J(U)=\int_{D}(\nabla U)^{t}A(\nabla U)+U(x)F(x)dx,\quad U\Big|_{\partial D}=g
\end{equation*}
in a domain $D\subseteq\mathbb{R}^{n+1}$ under the further constraint that $U\geq\psi$ on a $n$-dimensional hypersurface $M\subseteq D$ dividing $D$ into $D_{1}$ and $D_{2}$. One can imagine a scenario of a carpet hung over a string. The minimizing solution satisfies
\begin{equation}\label{general Signorini}
    \mathrm{div}(A\nabla U)=F(x)\quad\mbox{when}\ U>\psi\ \mbox{or}\ x\notin M.
\end{equation}
The free boundary is a codimension-$2$ surface defined as follows:
\begin{definition}
    Let $U\geq\psi$ on $M$ and assume that $U$ satisfies \eqref{general Signorini}. The free boundary $\Gamma\subseteq M$ is defined as $\Gamma:=\partial\Big|_{M}\{U=\psi\}$.
\end{definition}

Without loss of generality, we can assume $\psi=0$, by replacing $U$ with $U-\psi$ and also replacing $F(x)$ with $F(x)-\mathrm{div}(A\nabla\psi)$.

The Signorini problem has applications in semipermeable membranes and perpetual American options. See \cite{PSU12} for more information.

The constant coefficient model is called the ``model problem S'' in \cite{PSU12}. It consists of studying the special case where $A=\delta^{ij}$, $F,\psi=0$, $M=\mathbb{R}^{n}=\{x_{n+1}=0\}$. We further assume that $U$ is an even function in the $x_{n+1}$-direction. In this case, \eqref{general Signorini} becomes
\begin{equation*}
    \Delta U=0\quad\mbox{when}\ U>0\ \mbox{or}\ x_{n+1}\neq0.
\end{equation*}
In fact, the $x_{n+1}$-even function $U$ can also be understood as the Caffarelli-Silvestre extension of a $\frac{1}{2}$-harmonic function $U\Big|_{\mathbb{R}^{n}}$ corresponding to the nonlocal operator $(-\Delta)^{1/2}$, see \cite{CS07}.

In \cite{AC04}, Athanasopoulos and Caffarelli showed that the solution $U(x)$ of the constant coefficient model is $C^{1,1/2}$ on both sides of $\mathbb{R}^{n}$, and this is the optimal regularity of solutions.

Free boundary points are classified according to their blow-up and frequency. Let $x_{0}\in\Gamma$ be a boundary point, the Almgren's frequency function is defined as
\begin{equation*}
    N_{x_{0}}(U,r):=r\frac{\int_{B_{r}(x_{0})}|\nabla U|^{2}}{\int_{\partial B_{r}(x_{0})}U^{2}}.
\end{equation*}
This is an increasing function. If $\ds N=\lim_{r\to0}N_{x_{0}}(U,r)\in(1,2)$, then the only possibility is $N=3/2$. In this situation, up to a subsequence, the rescaled function
\begin{equation*}
    U_{r}(x):=\frac{U(rx+x_{0})}{(\fint_{\partial B_{r}(x_{0})}U^{2})^{1/2}}
\end{equation*}
converges in $C^{1,\alpha}$ sense to a $3/2$-homogeneous function $C_{n}\cdot Re((x_{n}+ix_{n+1})^{3/2})$ near $0$, after a coordinate rotation. We say such a free boundary point is a regular point.

Assume that the origin is a regular point, the free boundary $\Gamma$ is locally a $C^{1,\alpha}$ graph (a $(n-1)$-dimensional submanifold) for some small $\alpha$, as shown in \cite{ACS08}. In \cite{CSS08}, the result is extended to the fractional Laplace case.

In \cite{ACS08}, the authors first showed that the free boundary is Lipschitz near a regular point. If the rescaling $U_{r}$ converges to $C_{n}\cdot Re((x_{n}+ix_{n+1})^{3/2})$, then for every unit vector $\vec{e}$ with
\begin{equation*}
    \vec{e}\cdot\vec{e}_{n+1}=0,\quad \vec{e}\cdot\vec{e}_{n}>0,
\end{equation*}
and for any $\epsilon$, there is a sufficiently small $r$ (for simplicity, let $r=1$), so that
\begin{equation*}
    \partial_{\vec{e}}U_{1}(x)\geq C_{n}\partial_{\vec{e}}Re((x_{n}+ix_{n+1})^{3/2})-\epsilon\mbox{ in }B_{1}.
\end{equation*}
Notice that $\partial_{\vec{e}}U_{1}(x)$ vanishes on $\{x_{n+1}=0,U(x)=0\}$ and is harmonic elsewhere. The authors then showed that for vectors $\vec{e}$ satisfying
\begin{equation*}
    \vec{e}\cdot\vec{e}_{n+1}=0,\quad \vec{e}\cdot\vec{e}_{n}\geq\delta>0
\end{equation*}
for some sufficiently small $\delta$, it holds that
\begin{equation*}
    \partial_{\vec{e}}U_{1}(x)\geq0\mbox{ in }B_{1/2}.
\end{equation*}
Therefore, the free boundary is a Lipschitz graph
\begin{equation*}
    \Gamma=\{(x_{1},\cdots,x_{n},0):x_{n}=\gamma(x_{1},\cdots,x_{n-1})\}
\end{equation*}
near the origin.

After that, the authors applied the boundary Harnack principle to the ratios $\partial_{i}U/\partial_{n}U(i<n)$ to show the $C^{1,\alpha}$-regularity of the free boundary. More precisely, they first straightened the free boundary using the coordinate change
\begin{equation*}
    (x_{1},\cdots,x_{n+1})\rightarrow(x_{1},\cdots,x_{n-1},x_{n}-\gamma(x_{1},\cdots,x_{n-1}),x_{n+1}),
\end{equation*}
and then applied the following Lipschitz coordinate change to open the ``slit''
\begin{equation*}
    (x_{1},\cdots,x_{n-1},\rho\cos{\theta},\rho\sin{\theta})\rightarrow(x_{1},\cdots,x_{n-1},\rho\cos{\frac{\theta}{2}},\rho\sin{\frac{\theta}{2}}).
\end{equation*}
The original partial derivatives $u_{i}=\partial_{i}U$ in the original coordinate satisfy the uniformly elliptic equation:
\begin{equation*}
    \mathrm{div}(A\nabla u_{i})=0,\quad\lambda I\leq A\leq\Lambda I
\end{equation*}
in the half space $\mathbb{R}^{n+1}_{+}$ and vanish at the boundary. By a version of the boundary Harnack principle in \cite{CFMS81}, the authors proved that the ratios $\partial_{i}U/\partial_{n}U(i<n)$ are $C^{\alpha}$.

For more general Signorini problems \eqref{general Signorini} with variable coefficients, it was shown in \cite{G09} that when $M,\psi$ are $C^{1,\beta}$ for some $\beta>1/2$, $A$ is $C^{1,\gamma}$ for all $\gamma>0$ and $F$ is just H\"older continuous, then $U$ is $C^{1,1/2}$ on both sides of $M$. Later, in \cite{GSVG14,KRS16,RS17}, the optimal regularity of $U$ was improved subsequently to Lipschitz, $W^{1,p}$ and $C^{\alpha}$ coefficients. 

The regular free boundary points are also defined using the Almgren' frequency.
\begin{definition}
    Let $U\geq\psi=0$ on $M$ be a solution of \eqref{general Signorini}. We say that a free boundary point $x_{0}\in\Gamma$ is regular, if the Almgren's frequency function
    \begin{equation*}
        N_{x_{0}}(U,r):=r\frac{\int_{B_{r}(x_{0})}|\nabla U|^{2}}{\int_{\partial B_{r}(x_{0})}U^{2}}
    \end{equation*}
    converges to $3/2$ when $r\to0$.
\end{definition}

In the variable coefficient case, the $C^{1,\alpha}$ regularity of the free boundary $\Gamma$ near a regular point was obtained in \cite{GPSVG16,KRS17a,RS17} under the regularity assumptions above.

\subsection{Higher regularity of the free boundary.}
Once we know that $\Gamma\in C^{1,\alpha}$, we can study the higher regularity of $\Gamma$. The best result is obtained in \cite{KPS15} for the constant coefficients model and in \cite{KRS17b} for analytic coefficients, both showing that $\Gamma\in C^{\omega}$ (an analytic codimension-$2$ surface) near a regular point using the partial hodograph-Legendre transformation.

One can also use the method of higher order boundary Harnack principle to study the regularity of the free boundary. For example, it is shown in \cite{DS14} that $\Gamma\in C^{\infty}$ near a regular point in the ``model problem S''.

In this paper, we allow variable coefficients and use a $C^{1,\alpha}$ boundary Harnack principle to show that $\Gamma\in C^{2,\alpha}$ near a regular point. The main difficulty is that the blow-up limit differs along the free boundary $\Gamma$ due to the changing coefficient matrix. This will be discussed soon in the next subsection.

We assume $D=B_{1}$, $M=\mathbb{R}^{n}=\{x_{n+1}=0\}$ and $\psi=0$. Besides, $U$ is assumed to be $x_{n+1}$-even, meaning
\begin{equation*}
U(x_{1},\cdots,x_{n},x_{n+1})=U(x_{1},\cdots,x_{n},-x_{n+1}),
\end{equation*}
and the symmetric matrix $A$, accordingly, needs to be an $x_{n+1}$-odd function in entries $A^{ij},A^{ji}$ for $i\leq n,j=n+1$, and be an $x_{n+1}$-even function in all other entries.

In the discussion in the remaining part of this paper, we assume that $A$ satisfies the following properties:
\begin{itemize}
    \item[(a)] $\lambda I\leq A\leq\Lambda I$ and $[A]_{C^{\alpha}}$ (for some $\alpha<1/2$) is bounded,
    \item[(b)] $A^{ij}=A^{ji}=0$ for $i\leq n,j=(n+1)$,
    \item[(c)] $\partial_{i}A$'s are $C^{\alpha}(\mathbb{R}^{n+1})$ for all $i\leq n$.
    \item[(d)] $A$ is even in the $x_{n+1}$ direction.
\end{itemize}
\begin{remark}
    By assuming $\alpha<1/2$, the regularity of the free boundary $\Gamma$ in Theorem~\ref{application} is $C^{2,\alpha}$ for $\alpha<1/2$. The author believe that if $F(x)\equiv0$ (or vanishes on $\{x_{n}=0\}$ with some sufficiently large vanishing order) in Theorem~\ref{application}, then the assumptions $\alpha<1/2$ can be removed.
\end{remark}
\begin{remark}
    If the assumptions (a)-(d) hold, then we can assume that the solution $U$ is an even function in the $x_{n+1}$-variable. Otherwise, we replace $U$ with
\begin{equation*}
    \bar{U}(x_{1},\cdots,x_{n+1}):=\frac{1}{2}U(x_{1},\cdots,x_{n},x_{n+1})+\frac{1}{2}U(x_{1},\cdots,x_{n},-x_{n+1}).
\end{equation*}
Then $\bar{U}$ still satisfies the same equation and the free boundary is unchanged.
\end{remark}

Our description of $C^{2,\alpha}$ regularity of the free boundary is the following.
\begin{theorem}\label{application}
    Assume that the matrix $A$ satisfies the even conditions (a)-(d), $F(x)\in W^{1,\infty}$, and an $x_{n+1}$-even solution $U$ solves the thin obstacle problem
\begin{equation}\label{U in y abuse}
    \mathrm{div}(A\nabla U)=F(x),\quad\mbox{when}\ U>0\ \mbox{or}\ x_{n}\neq0.
\end{equation}
If $0$ is a regular boundary point, then the free boundary $\Gamma\subseteq\mathbb{R}^{n}$ is a $C^{2,\alpha}$ graph near $0$.
\end{theorem}

We can perform a scaling, replacing $U(x)$ by $R^{-3/2}U(Rx)$ near the origin. This keeps the blow-up of $U(x)$ at the regular boundary invariant, but we can assume $\|D^{i}A\|_{C^{\alpha}}(i\leq n)$ and $[F]_{W^{1,\infty}}$ are small by sending $R\to0$.

Up to a rotation, let us assume that the free boundary $\Gamma$ is a graph in $(x_{1},\cdots,x_{n-1})$ variables near the origin, meaning that there exists $\gamma(x_{1},\cdots,x_{n-1})$ so that
\begin{equation*}
    \Gamma=\{(x_{1},\cdots,x_{n-1},\gamma(x_{1},\cdots,x_{n-1}),0):(x_{1},\cdots,x_{n-1})\in\mathbb{R}^{n-1}\cap B_{1}\}.
\end{equation*}

The strategy is to show $w_{i}=U_{x_{i}}/U_{x_{n}}(i<n)$ is $C^{1,\alpha}$ on $\Gamma$. This is because $w_{i}$'s represent the gradient of the level set of $U$. If $w_{i}\Big|_{\Gamma}\in C^{1,\alpha}$, then so do $\partial_{i}\gamma$, meaning that $\Gamma$ is a $C^{2,\alpha}$ graph.

We perform a coordinate change
\begin{equation}\label{ytox abuse}
    x\rightsquigarrow x-\gamma(x_{1},\cdots,x_{n-1})\vec{e}_{n}
\end{equation}
which straightens the boundary $\Gamma$ to $\mathbb{R}^{n-1}$ (it will be made more clear in \eqref{ytox}). Then, we establish a $C^{1,\alpha}$ boundary Harnack principle on the straightened boundary.

We denote the ``straightened-slit'' as $S$, more precisely,
\begin{equation}
    S=\{x=(x_{1},...,x_{n+1}): x_{n}<0,x_{n+1}=0\}.
\end{equation}
It corresponds to the region $\{x_{n}\leq\gamma(x_{1},...,x_{n-1})\}\cap\mathbb{R}^{n}=\{U=0\}\cap\mathbb{R}^{n}$ before the coordinate change \eqref{ytox abuse}.

For each point $(x_{1},\cdots,x_{n+1})$ after the coordinate change, we abbreviate its first $(n-1)$ indices as $x^{T}$, and last two indices $(x_{n},x_{n+1})=x^{\perp}$.

As the blow-up scaling
\begin{equation*}
    U_{r}(x):=\frac{U(rx+x_{0})}{(\fint_{\partial B_{r}(x_{0})}U^{2})^{1/2}}
\end{equation*}
converges as $r\to0$ in $C^{1,\alpha}$ sense to a $3/2$-homogeneous solution like the previously mentioned
\begin{equation*}
    Re((x_{n}+i x_{n+1})^{3/2}),
\end{equation*}
we see that for $i\leq n$, $U_{x_{i}}$ converges to a multiple of a $1/2$-homogeneous solution like
\begin{equation*}
    Re((x_{n}+i x_{n+1})^{1/2})
\end{equation*}
in $C^{\alpha}$ sense near the regular point. In fact, the blow-up limit of $U_{x_{i}}$ is more complicated, in the sense that the expression depends on the coefficient matrix $A$. The dependence is presented below.

\subsection{Homogeneous solutions} Assume that after straightening the boundary $\Gamma$, which is turned into $\mathbb{R}^{n-1}$, we have a corresponding matrix $A$ (will be explicit in \eqref{equationu}). Let $\bar{A}=A(0)$ so that $\bar{A}^{i,n+1}=\bar{A}^{n+1,i}=0$ for all $i\leq n$, then the positive $1/2$-homogeneous solution of
\begin{equation}
    \mathrm{div}(\bar{A}\nabla\bar{u})=0\ \mbox{in}\ \mathbb{R}^{n+1}\setminus S,\quad\bar{u}\Big|_{S}=0
\end{equation}
is given up to a constant multiple by
\begin{equation}\label{diffenent blow up}
\bar{u}_{\bar{A}}=\sqrt{\frac{x_{n}+\sqrt{x_{n}^{2}+\kappa^{2}x_{n+1}^{2}}}{2}},\quad\kappa=\kappa(\bar{A})=(\frac{\bar{A}^{nn}}{\bar{A}^{n+1,n+1}})^{1/2}.
\end{equation}
In particular, when $\bar{A}=\delta^{ij}$, then $\kappa=1$, and we set
\begin{equation}\label{rho and xi}
\rho:=\sqrt{x_{n}^{2}+x_{n+1}^{2}}=|x^{\perp}|,\quad\xi:=\bar{u}_{I}=\sqrt{\frac{x_{n}+\rho}{2}}=Re\Big((x_{n}+i x_{n+1})^{1/2}\Big).
\end{equation}
Simple computation yields that $0\leq\xi\leq\sqrt{\rho}$ and $\ds|\nabla\xi|=\frac{1}{2\sqrt{\rho}}$. We will use these new coordinates quite often.

It suffices to study the special case $\bar{A}=\delta^{ij}$ since homogeneous solutions differ only by a stretching:
\begin{equation*}
    \bar{u}_{\bar{A}}(x_{n},x_{n+1})=\xi(x_{n},\kappa x_{n+1}).
\end{equation*}
It's worthwhile to mention that when $A$ is uniformly elliptic, then $\kappa(A)$ is bounded from above and below when we move the center point elsewhere, and there is a uniform constant $C(\lambda,\Lambda)$ so that for any $x$,
\begin{equation*}
    C(\lambda,\Lambda)^{-1}\leq\frac{\bar{u}_{\bar{A}_{1}}}{\bar{u}_{\bar{A}_{2}}}(x)\leq C(\lambda,\Lambda).
\end{equation*}
However, the ratio $\ds\frac{\bar{u}_{\bar{A}_{1}}}{\bar{u}_{\bar{A}_{2}}}(x)$ is not continuous when $\kappa(A_{1})\neq\kappa(A_{2})$. It depends only on the angle $arg(x_{n}+i x_{n+1})$, so when $|x^{\perp}|\to0$, the oscillation of $\ds\frac{\bar{u}_{\bar{A}_{1}}}{\bar{u}_{\bar{A}_{2}}}(x)$ remains large. Besides, one can check that the ratio $\ds\frac{\bar{u}_{\bar{A}_{1}}}{\bar{u}_{\bar{A}_{2}}}(x)$ is Lipschitz with respect to the angle $arg(x_{n}+i x_{n+1})$.

\subsection{\texorpdfstring{$C^{1,\alpha}$}{Lg} boundary Harnack principle} In this paper, we will sometimes use a path distance to describe the metric space in $\mathbb{R}^{n+1}\setminus S$, that is
\begin{equation}
    dist(x,y):=\inf\{|\gamma|:\gamma(0)=x,\gamma(1)=y,\gamma(t)\in\mathbb{R}^{n+1}\setminus S\}.
\end{equation}
We describe a kind of H\"older continuity which will be used quite often in this paper. We say a scalar/vector valued function $f(x)$ satisfies property $(\mathcal{F}_{A})$ in $B_{R}\setminus S$ if
\begin{itemize}
    \item[$(\mathcal{F}_{A})$] There exists a constant $C$ and $h(x)$ so that $f(x)=\bar{u}_{A(x^{T})}(x)h(x)$ and
    \begin{equation*}
        \|h(x)\|_{C^{\alpha}(B_{R}\setminus S)}\leq C,\quad\mbox{w.r.t. path distance}\ dist(\cdot,\cdot).
    \end{equation*}
\end{itemize}
We can also simply write $(\mathcal{F}_{A})$ as $(\mathcal{F})$, provided that the matrix $A$ is known on $\mathbb{R}^{n-1}$. Besides, if $f(x)$ is $x_{n+1}$-even, then there's no distinction between using the path distance $dist(\cdot,\cdot)$ and using $|x-y|$ to describe regularity.

The $C^{1,\alpha}$ boundary Harnack principle is the following.
\begin{theorem}\label{HOBH straight}
    Assume that $\lambda I\leq A\leq\Lambda I$ and $A\in C^{\alpha}$($\alpha<1/2$). $\phi_{1},\phi_{2}$ are $L^{\infty}$ functions. $u_{1},u_{2}$ are defined on $B_{1}$, and they satisfy
    \begin{equation}\label{standardx}
        \mathrm{div}(A\cdot\nabla u_{i})=\mathrm{div}(\vec{f}_{i})+\phi_{i}\ \mbox{in}\ B_{1}\setminus S,\quad u_{i}\Big|_{B_{1}'\cap S}=0.
    \end{equation}
    We also assume $u_{2}/\xi\geq c_{0}>0$ in $B_{1}$. If $\vec{f}_{1}$ and $\vec{f}_{2}$ both have property $(\mathcal{F}_{A})$ in $B_{1}$ with $\vec{f}_{i}\cdot\vec{e}_{n+1}\equiv0$, then the ratio $\ds w=\frac{u_{1}}{u_{2}}$, when restricted on $\mathbb{R}^{n-1}$, is $C^{1,\alpha}$. Besides, $u_{1}$ and $u_{2}$ both have property $(\mathcal{F}_{A})$ in $B_{1/2}$.
\end{theorem}

In the next subsection, in order to prove the $C^{1,\alpha}$ boundary Harnack principle, we follow a similar method as in \cite{TTV22}. We derive the equation satisfied by the ratio $w$, which can be turned into a degenerate equation. The $C^{1,\alpha}$ boundary Harnack principle is then a consequence of a Schauder estimate of the ratio $w$.
\subsection{A degenerate equation}
By direct computation, we obtain the equation satisfied by the ratio.
\begin{lemma}\label{ratio}
    If $u_{1}$ and $u_{2}$ satisfy $\mathrm{div}(A\cdot\nabla u_{i})=\mathrm{div}(\vec{f}_{i})+\phi_{i}$, then the ratio $\ds w=\frac{u_{1}}{u_{2}}$ satisfies
    \begin{equation}\label{eq. ratio equation}
        \mathrm{div}(u_{2}^{2}A\nabla w)=\mathrm{div}(u_{2}\vec{f}_{1}-u_{1}\vec{f_{2}})+(\vec{f}_{2}\cdot\nabla u_{1}-\vec{f}_{1}\cdot\nabla u_{2})+(u_{2}\phi_{1}-u_{1}\phi_{2})
    \end{equation}
    as long as $A$ is symmetric.
\end{lemma}
\begin{proof}
    As $u_{1}=u_{2}w$, by expanding $\ds u_{2}\mathrm{div}\Big(A\nabla(u_{2}w)\Big)=u_{2}\mathrm{div}(\vec{f}_{1})+u_{2}\phi_{1}$, we have
    \begin{equation*}
        u_{2}^{2}\mathrm{div}(A\nabla w)+2u_{2}A(\nabla u_{2},\nabla w)=u_{2}[\mathrm{div}(\vec{f}_{1})+\phi_{1}]-u_{2}w\cdot \mathrm{div}(A\nabla u_{2}).
    \end{equation*}
    Its left-hand side is
    \begin{equation*}
        u_{2}^{2}\mathrm{div}(A\nabla w)+A(\nabla u_{2}^{2},\nabla w)=\mathrm{div}(u_{2}^{2}A\nabla w),
    \end{equation*}
    and right-hand side is
    \begin{align*}
        &u_{2}[\mathrm{div}(\vec{f}_{1})+\phi_{1}]-u_{1}[\mathrm{div}(\vec{f}_{2})+\phi_{2}]\\
        =&\mathrm{div}(u_{2}\vec{f}_{1}-u_{1}\vec{f_{2}})+(\vec{f}_{2}\cdot\nabla u_{1}-\vec{f}_{1}\cdot\nabla u_{2})+u_{2}\phi_{1}-u_{1}\phi_{2}.
    \end{align*}
\end{proof}
In the context of $C^{1,\alpha}$ boundary Harnack principle, we can also absorb the term $(\vec{f}_{2}\cdot\nabla u_{1}-\vec{f}_{1}\cdot\nabla u_{2})$ into the divergence term, see Remark~\ref{absorb remark} at the end of section~\ref{only average}.

Assume that $A(0)=\delta^{ij}$ for simplicity, meaning that the blow-up of $u_{2}$ at the origin is $\xi$ (defined in \eqref{rho and xi}). We consider a model degenerate equation:
\begin{equation}\label{main}
    \mathrm{div}(\xi^{2}A\cdot\nabla w)=\mathrm{div}(\xi^{2}\vec{f})+\xi^{2}g\quad\mbox{in}\ B_{1}\setminus S.
\end{equation}
Other terms can also appear on the right-hand side, and we give some explanation in Remark~\ref{absorb remark} and Remark~\ref{xiphi on the RHS}.

The assumption on $w$ is that $w\in L^{2}(B_{1}\setminus S,\rho^{-1}dx)\cap H^{1}_{loc}$, and later we will obtain a Schauder estimate for \eqref{main}. To be precise about the notation, $w\in L^{2}(B_{1}\setminus S,\rho^{-1}dx)$ means that for $\rho$ defined in \eqref{rho and xi},
\begin{equation*}
    \int_{B_{1}\setminus S}\frac{w^{2}}{\rho}dx<\infty.
\end{equation*}
Besides, we say $w\in H^{1}_{loc}$ if for any point $x\in B_{1}\setminus S$, there exists a radius $\epsilon$ such that $B_{\epsilon}(x)\subseteq B_{1}\setminus S$ and $w\in H^{1}(B_{\epsilon}(x))$.

To describe its solution, we realize that if $w$ is an $x_{n+1}$-even, the linearization of $w$ is a non-smooth ``polynomial'' in the variable $(x_{1},...,x_{n},\rho)$. We make the following definition.
\begin{definition}
We refer to ``polynomials'' as elements in the ring $\mathbb{R}[x_{1},...,x_{n},\rho]$. A ``polynomial'' is called linear, if it is in the form
\begin{equation*}
    L=c_{0}+c_{1}x_{1}+\cdots+c_{n}x_{n}+c_{\rho}\rho.
\end{equation*}
\end{definition}
We also define ``pointwise'' $C^{\alpha}$ vector fields, which are the gradients of some linear ``polynomial'' with $O(|x|^{\alpha})$ error.
\begin{definition}\label{def. 1.6}
    A vector field $\vec{f}$ is called ``pointwise'' $C^{\alpha}$ at the origin, if it can be written as $\vec{f}=\vec{f}_{0}+\vec{f}_{\alpha}$, where
\begin{equation*}
\vec{f}_{0}=c_{1}e_{1}+,\cdots+c_{n}\vec{e}_{n}+c_{\rho}\nabla\rho,\quad|\vec{f}_{\alpha}|\leq c_{\alpha}|x|^{\alpha},
\end{equation*}
and we denote its norm as $\|\vec{f}\|_{C^{\alpha}(B_{1},0)}:=|c_{1}|+\cdots+|c_{n}|+|c_{\rho}|+c_{\alpha}$.
\end{definition}
We prove the following pointwise Schauder estimate:
\begin{proposition}\label{gammaschauder}
Assume that $A^{ij}(0)=\delta^{ij}$, $[A^{ij}]_{C^{\alpha}(B_{1}^{+})}\leq\varepsilon_{0}$ and $w\in L^{2}(B_{1}\setminus S,\rho^{-1}dx)\cap H^{1}_{loc}$ is an $x_{n+1}$-even solution of \eqref{main}. Let $\ds p=\frac{n+3}{1-\alpha}$. Given that $\varepsilon_{0}$ is small, there exist a linear ``polynomial'' $L$ and $C=C(n,\alpha)$ such that
\begin{align}\label{average C1alpha}
    &\|L\|_{C^{0,1}(B_{1}\setminus S)}^{2}+\frac{1}{r^{n+2+2\alpha}}\int_{B_{r}\setminus S}\frac{|w-L|^{2}}{\rho}dx\notag\\
    \leq&C\Big\{\|\vec{f}\|_{C^{\alpha}(B_{1},0)}^{2}+\|g\|_{L^{p}(B_{1}\setminus S,\rho\xi^{2}dx)}^{2}+\|w\|_{L^{2}(B_{1}\setminus S,\rho^{-1}dx)}^{2}\Big\}.
\end{align}
\end{proposition}

We can call such a description of $w$ as $C^{1,\alpha}$ in average sense. If $w$ is $C^{1,\alpha}$ in average sense everywhere on $\mathbb{R}^{n-1}$, then its restriction on $\mathbb{R}^{n-1}$ is $C^{1,\alpha}$ in the classical sense.

The $L^{p}$ assumption on $g$ can be weakened, since it will be used only in \eqref{g actual} during the proof, and we just need to assume
\begin{equation}\label{g weaken}
    \frac{1}{r^{(n+2+2\alpha)}}\int_{B_{r}\setminus S}\rho\xi^{4}g^{2}dx\leq C,\quad\forall r\leq1.
\end{equation}

The Schauder estimate for \eqref{main} would be trivial, if the weight $\lambda=\xi^{2}$ were an $A_{2}$-Muckenhoupt weight (see \cite{Fabes2,Fabes3,Fabes1}), which satisfies the condition that for any ball $B$,
\begin{equation*}
    (\frac{1}{|B|}\int_{B}\lambda)\cdot(\frac{1}{|B|}\int_{B}\lambda^{-1})\leq C.
\end{equation*}
Unfortunately, one can verify that the weight $\lambda=\xi^{2}$ is not $A_{2}$-Muckenhoupt. Nevertheless, inspired by \cite{DP23a,DP23b}, we derive the Schauder estimate by working with a weighted Sobolev space.

In order to apply Proposition~\ref{gammaschauder} to the equation in Lemma~\ref{ratio}, we need to show that $u_{2}/\xi$ is $C^{\alpha}$ at the origin, and has a positive lower bound. Therefore, we use a similar method to study a uniform equation
\begin{equation}\label{uniform}
    \mathrm{div}(A\nabla u)=\mathrm{div}(\frac{\vec{f}}{\sqrt{\rho}})\ \mbox{in}\ B_{1}\setminus S,\quad u\Big|_{S}=0
\end{equation}
in trace sense and $\vec{f}=0$ on $\mathbb{R}^{n-1}$. In Theorem~\ref{Holder estimate} we will show that $u$ will have property $(\mathcal{F}_{A})$ if $\vec{f}\in C^{\alpha}$. We also use it to prove a Hopf-type result in Proposition~\ref{hopf}.
\subsection{More comments}
We see from \eqref{diffenent blow up} that different matrices $A$ correspond to different homogeneous solutions $\bar{u}_{A}$, the vertical coordinate change \eqref{ytox abuse} is not suitable in proving $\Gamma\in C^{3,\alpha},\cdots,C^{\omega}$.

If there is a coordinate change different from \eqref{ytox abuse}, so that $\kappa(A)$ is constant along $\mathbb{R}^{n-1}$, then we can take tangential derivatives of $w_{i}$ and obtain higher order regularity of $w_{i}$, perhaps also obtain the analyticity of $\Gamma$ like in \cite{KPS15}\cite{KRS17b} by a bootstrap argument.

We consider that if $A=A_{D}$ or equivalently $A_{O}=0$, and if $A^{n+1,n+1}$ is a constant, then a geodesic flow that starts from $\Gamma$ and its normal vector $\nu\in\mathbb{R}^{n}$ will be the desired coordinate change. To be precise, at each point $x\in\Gamma$, we solve the ODE that
\begin{equation*}
    \nabla_{c'}c'=0,\quad c(0)=x,\quad c'(0)=\nu(x).
\end{equation*}
This locally parameterizes the hyperplane $x_{n+1}=0$ as $\Gamma\times(-\epsilon,\epsilon)$. We then write the equation \eqref{general Signorini} in this new coordinate.

Besides, in order to prove $\Gamma\in C^{\omega}$, we also need to improve Proposition~\ref{gammaschauder}, so that $C^{1,\alpha}$ estimate is in the classical sense rather than in the average sense.

The coordinate change obtained by geodesic flow will be $C^{1,\alpha}$ only when we assume $\Gamma\in C^{2,\alpha}$, so in this paper we still have to use the vertical coordinate change. It seems to be a common phenomenon (like \cite{DS14} and \cite{DS15}) that the proof of $C^{2,\alpha}$ regularity of the free boundary is different from the proof of higher regularity.


\section{A weighted Sobolev space}
In this section we define a weighted $H^{1}$ space, and establish the corresponding weighed energy estimate.
\subsection{Poincar\'e inequality} In this sub-section, we establish a suitable Poincar\'e inequality in Proposition~\ref{thin poincare}. In this paper, we need to introduce a complex coordinate. For each $x=(x^{T},x^{\perp})$ with $x^{\perp}=(x_{n},x_{n+1})$, we write
\begin{equation}\label{complex cordinate}
    \xi+i\eta=(x_{n}+ix_{n+1})^{1/2},
\end{equation}
where $\xi$ here is the same as \eqref{rho and xi}. In the $(x_{1},\cdots,x_{n-1},\xi,\eta)$-coordinate, we first fix the first $(n-1)$ coordinates $(x_{1},\cdots,x_{n-1})$ and obtain a Poincar\'e estimate for the $(\xi,\eta)$-coordinates.
\begin{lemma}
    Assume that $w(\xi,\eta)\in C^{1}_{loc}$ in the two-dimensional half space $\{(\xi,\eta):\xi>0\}$. If $w\Big|_{|x|>r}=0$ for some $r$, then
    \begin{equation}\label{regular poincare}
\int_{\{\xi>0\}}w^{2}d\xi d\eta\leq4\int_{\{\xi>0\}}\xi^{2}|\nabla w|^{2}d\xi d\eta.
\end{equation}
\end{lemma}
\begin{proof}
First, we fix the coordinate $\eta$, then by changing the order of integration,
    \begin{equation*}
\int_{0}^{\infty}|w(\xi,\eta)|d\xi\leq\int_{0}^{\infty}\int_{\xi}^{\infty}|\partial_{\xi}w(h,\eta)|dhd\xi=\int_{0}^{\infty}h|\partial_{\xi}w(h,\eta)|dh.
    \end{equation*}
We now set $\eta$ free and integrate in $\eta$, then
    \begin{equation*}
        \int_{\{\xi>0\}}|w(\xi,\eta)|d\xi d\eta\leq\int_{\{\xi>0\}}\xi|\partial_{\xi}w(\xi,\eta)|d\xi d\eta\leq\int_{\{\xi>0\}}\xi|\nabla w(\xi,\eta)|d\xi d\eta.
    \end{equation*}
We replace $w$ with $w^{2}$, then by Cauchy-Schwartz inequality,
    \begin{equation*}
        4\int_{\{\xi>0\}}w^{2}\int_{\{\xi>0\}}\xi^{2}|\nabla w|^{2}\geq(\int_{\{\xi>0\}}\xi|\nabla w^{2}|)^{2}\geq(\int_{\{\xi>0\}}w^{2})^{2}.
    \end{equation*}
\end{proof}

Consequently, we have the following Poincar\'e inequality for a slit domain.
\begin{proposition}\label{thin poincare}
If a function $w\in C^{1}_{loc}(\mathbb{R}^{n+1}\setminus S)$ is compactly supported, i.e. $w\Big|_{|x|>r}=0$, then
\begin{equation}
    \int_{B_{r}\setminus S}\frac{w^{2}}{\rho}dx\leq4\int_{B_{r}\setminus S}\xi^{2}|\nabla w|^{2}dx.
\end{equation}
\end{proposition}
\begin{proof}
It suffices to show that for each fixed $x^{T}$,
\begin{equation*}
\int_{\mathbb{C}\setminus\mathbb{R}_{-}}\frac{w^{2}}{\rho}dx_{n}dx_{n+1}\leq 4\int_{\mathbb{C}\setminus\mathbb{R}_{-}}\xi^{2}|\nabla^{\perp}w|^{2}dx_{n}dx_{n+1}.
\end{equation*}
Between two coordinates $(x_{n},x_{n+1})$ and $(\xi,\eta)$, we have
\begin{equation*}
    d\xi d\eta=\frac{1}{4\rho}dx_{n}dx_{n+1},\quad|\nabla_{\xi,\eta}w|^{2}d\xi d\eta=|\nabla^{\perp}w|^{2}dx_{n}dx_{n+1}.
\end{equation*}
Now, by writing \eqref{regular poincare} in the $(x_{n},x_{n+1})$-coordinate, we have
\begin{align*}
&\int_{\mathbb{C}\setminus\mathbb{R}_{-}}\frac{w^{2}}{\rho}dx_{n}dx_{n+1}=4\int_{Re(\alpha)>0}w^{2}d\xi d\eta\\
\leq&4\int_{Re(\alpha)>0}\xi^{2}|\nabla_{\xi,\eta}w|^{2}d\xi d\eta=4\int_{\mathbb{C}\setminus\mathbb{R}_{-}}\xi^{2}|\nabla^{\perp}w|^{2}dx_{n}dx_{n+1}.
\end{align*}
\end{proof}
It is then natural for us to define a weighted Sobolev space $H^{1}(B_{r}\setminus S,\xi^{2}dx)$. It is a subspace of $H^{1}_{loc}(B_{r}\setminus S)$ functions which locally have weak derivatives in $B_{r}\setminus S$, such that its weighted energy is bounded.
\begin{definition}
    $H^{1}(B_{r}\setminus S,\xi^{2}dx)$ is the space of $H^{1}_{loc}$ functions $w$ such that
\begin{equation}
    \|w\|_{H^{1}(B_{r}\setminus S,\xi^{2}dx)}^{2}:=\int_{B_{r}\setminus S}\xi^{2}|\nabla w|^{2}dx+\int_{B_{r}\setminus S}\frac{w^{2}}{\rho}dx<\infty.
\end{equation}
Its subspace $H^{1}_{0}(B_{r}\setminus S,\xi^{2}dx)$ is the set of all $w\in H^{1}(B_{r}\setminus S,\xi^{2}dx)$, such that $w=0$ on $(\partial B_{r})\setminus S$ in the trace sense.
\end{definition}


\begin{remark}
    The space $H^{1}_{0}(B_{r}\setminus S,\xi^{2}dx)$ can also be understood as the completion of smooth functions vanishing outside $B_{r}$ with respect to the $H^{1}(B_{r}\setminus S,\xi^{2}dx)$ norm. See Remark~\ref{rmk. density} below.
\end{remark}
\begin{remark}\label{rmk. density}
    For each $w\in H^{1}(B_{r}\setminus S,\xi^{2}dx)$, there exists $w_{i}\in H^{1}(B_{r}\setminus S,\xi^{2}dx)$, so that $w_{k}\in C^{\infty}(B_{r}\setminus S)\cap C^{0}(\overline{B_{r}\setminus S})$ as well, and
\begin{equation*}
    \int_{B_{r}\setminus S}\xi^{2}|\nabla(w-w_{i})|^{2}dx+\int_{B_{r}\setminus S}\frac{(w-w_{i})^{2}}{\rho}dx\to0.
\end{equation*}

To see this, it suffices to just argue with a specific $r=1$. We can use the complex coordinate $(x^{T},\xi,\eta)$ mentioned in \eqref{complex cordinate}. It turns $B_{1}\setminus S$ to $\Omega_{1}=\{(x^{T},\xi,\eta):\xi>0,|x^{T}|^{2}+(\xi^{2}+\eta^{2})^{2}\leq1\}$, and the slit $S$ becomes the plane $\{\xi=0\}$, which is now a $C^{1}$ boundary. The corresponding weighted Sobolev semi-norm $\ds\int_{B_{1}\setminus S}\xi^{2}|\nabla w|^{2}dx$ becomes
\begin{equation*}
    \int_{\Omega_{1}}\Big\{\xi^{2}[(\partial_{\xi}w)^{2}+(\partial_{\eta}w)^{2}]+4\xi^{2}\rho\sum_{i}^{n-1}(\partial_{i}w)^{2}\Big\}dx_{1}\cdots dx_{n-1}d\xi d\eta.
\end{equation*}
In the $(x^{T},\xi,\eta)$ coordinate, let
\begin{equation*}
    L_{h}(x^{T},\xi,\eta)=((1-4h)x^{T},(1-4h)\xi+2h,(1-4h)\eta),\quad h>0
\end{equation*}
be an affine map that maps $\Omega_{1}$ into $\Omega_{1}$. Notice that
\begin{equation*}
    2h\leq dist(x,L(\Omega_{1}))\leq8h,\quad\forall x\in\partial\Omega_{1}.
\end{equation*}
We can then follow the standard method of smooth approximation, that we let $\phi_{h}$ be a smooth mollifier supported in $B_{h}$, and approximate $w$ by the smooth function
\begin{equation*}
    w_{h}(x^{T},\xi,\eta)=(w*\phi_{h})\circ L_{h}(x^{T},\xi,\eta),\quad h\to0.
\end{equation*}
\end{remark}
\begin{remark}
    As a simple consequence of the density of smooth functions inside $H^{1}(B_{r}\setminus S,\xi^{2}dx)$, we see that Proposition~\ref{thin poincare} not only holds for $C^{1}_{loc}$ functions, but also for functions in $H^{1}(\mathbb{R}^{n+1}\setminus S,\xi^{2}dx)$ with compact support.
\end{remark}



\subsection{Caccioppoli inequalities}In this subsection, we develop two Caccioppoli inequalities for the degenerate model \eqref{main}. We write it down again.
\begin{equation}\label{main again}
    \mathrm{div}(\xi^{2}A\cdot\nabla w)=\mathrm{div}(\xi^{2}\vec{f})+\xi^{2}g\quad\mbox{in}\ B_{1}\setminus S.
\end{equation}

First we establish an interior estimate.

\begin{proposition}\label{caccioppoli}
Assume that $\lambda I\leq A\leq\Lambda I$. If $w\in H^{1}_{loc}$ is a solution of \eqref{main again}, then there exists $C=C(\lambda,\Lambda)$ so that
\begin{equation}
    \int_{B_{r/2}\setminus S}\xi^{2}|\nabla w|^{2}dx\leq C\Big\{\int_{B_{r}\setminus S}\frac{w^{2}}{\rho}dx+\int_{B_{r}\setminus S}(\xi^{2}|\vec{f}|^{2}+\rho\xi^{4}g^{2})dx\Big\}.
\end{equation}
\end{proposition}
\begin{proof}
Let $\varphi(x),\eta_{h}(x)$ be two positive smooth functions satisfying
\begin{align*}
    &\varphi\Big|_{|x|\leq r/2}=1,\quad\varphi\Big|_{|x|\geq r}=0,\quad|\nabla\varphi|\leq\frac{C}{r},\\
    &\eta_{h}\Big|_{\xi\geq h}=1,\quad\eta_{h}\Big|_{\xi=0}=0,\quad|\nabla\eta_{h}|\leq\frac{C}{h\sqrt{\rho}}.
\end{align*}
We also denote $\varphi_{h}=\varphi\cdot\eta_{h}$ and it's not hard to show that
\begin{equation*}
    \xi^{2}|\nabla\varphi_{h}|^{2}\leq C\Big(\frac{\xi^{2}}{r^{2}}+\frac{\chi_{\{\xi\leq h\}}}{\rho}\Big)\leq\frac{C}{\rho}.
\end{equation*}
Now multiplying $\varphi_{h}^{2}w$ on both sides of \eqref{main again}, integration by parts implies
\begin{align*}
&\int_{B_{r}\setminus S}\xi^{2}A\Big(\nabla(\varphi_{h}w),\nabla(\varphi_{h}w)\Big)-\int_{B_{r}\setminus S}\xi^{2}w^{2}A(\nabla\varphi_{h},\nabla\varphi_{h})\\
=&\int_{B_{r}\setminus S}\xi^{2}\vec{f}\cdot\nabla(\varphi_{h}^{2}w)-\int_{B_{r}\setminus S}\xi^{2}g\varphi_{h}^{2}w.
\end{align*}
As $\lambda I\leq A\leq\Lambda I$ and $\ds\xi^{2}|\nabla\varphi_{h}|^{2}\leq\frac{C}{\rho}$, we have
\begin{align*}
&\int_{B_{r}\setminus S}\xi^{2}A\Big(\nabla(\varphi_{h}w),\nabla(\varphi_{h}w)\Big)\geq\lambda\int_{B_{r}\setminus S}\xi^{2}|\nabla(\varphi_{h}w)|^{2},\\
&\int_{B_{r}\setminus S}\xi^{2}w^{2}A(\nabla\varphi_{h},\nabla\varphi_{h})\leq C\cdot\Lambda\int_{B_{r}\setminus S}\frac{w^{2}}{\rho}.
\end{align*}
Besides, as $0\leq\varphi_{h}\leq1$,
\begin{align*}
|\int_{B_{r}\setminus S}\xi^{2}\vec{f}\cdot\nabla(\varphi_{h}^{2}w)|\leq&\varepsilon\int_{B_{r}\setminus S}\xi^{2}|\nabla(\varphi_{h}w)|^{2}+\frac{1}{\varepsilon}\int_{B_{r}\setminus S}\xi^{2}|\vec{f}|^{2}+C\int_{B_{r}\setminus S}\frac{w^{2}}{\rho},\\
|\int_{B_{r}\setminus S}\xi^{2}g\varphi_{h}^{2}w|\leq&\frac{1}{2}\int_{B_{r}\setminus S}\frac{w^{2}}{\rho}+\frac{1}{2}\int_{B_{r}\setminus S}\rho\xi^{4}g^{2}.
\end{align*}
Putting those estimates together yields that when $\varepsilon=\varepsilon(\lambda,\Lambda)$,
\begin{align*}
&\int_{B_{r/2}\cap\{\xi\geq h\}}\xi^{2}|\nabla w|^{2}\leq\int_{B_{r}\setminus S}\xi^{2}|\nabla(\varphi_{h}w)|^{2}\\
\leq&C\Big\{\int_{B_{r}\setminus S}(\xi^{2}|\vec{f}|^{2}+\rho\xi^{4}g^{2})+\int_{B_{r}\setminus S}\frac{w^{2}}{\rho}\Big\}.
\end{align*}
Finally, since in pointwise sense we have
\begin{equation*}
    \lim_{h\to0}\xi^{2}|\nabla w|^{2}\chi_{\{\xi\geq h\}}=\xi^{2}|\nabla w|^{2},
\end{equation*}
we can send $h\to0$, and apply Fatou lemma to finish the proof.
\end{proof}


Next, here is a global inequality for solutions of \eqref{main again} with zero boundary data.

\begin{proposition}\label{dirichlet}
Assume that $\lambda I\leq A\leq\Lambda I$. If $w\in H^{1}_{loc}$ is a solution of \eqref{main again} so that $w\Big|_{|x|\geq r}=0$, then there exists $C=C(\lambda,\Lambda)$ so that
\begin{equation}
    \int_{B_{r}\setminus S}\xi^{2}|\nabla w|^{2}dx\leq C\int_{B_{r}\setminus S}(\xi^{2}|\vec{f}|^{2}+\rho\xi^{4}g^{2})dx.
\end{equation}
\end{proposition}

\begin{proof}
The proof is similar to Proposition~\ref{caccioppoli}.
\end{proof}

\subsection{Weak solution}
Assume that $\lambda I\leq A\leq\Lambda I$. To find a solution of \eqref{main again} satisfying some boundary condition, we can use the Lax-Milgram lemma. Let the bi-linear form and linear functional
\begin{equation*}
B(w,\varphi):=\int_{B_{r}\setminus S}\xi^{2}\nabla\varphi^{T}A\nabla w dx,\quad F(\varphi)=\int_{B_{r}\setminus S}(\xi^{2}\vec{f}\cdot\nabla\varphi-\xi^{2}g\varphi)dx
\end{equation*}
be defined for $w,\varphi\in H^{1}_{0}(B_{r}\setminus S,\xi^{2}dx)$. If $w\in H^{1}_{0}(B_{r}\setminus S,\xi^{2}dx)$, i.e., $w=0$ when $|x|>r$, then by Proposition~\ref{thin poincare},
\begin{equation*}
    B(w,w)\geq\frac{\lambda}{5}\|w\|^{2}_{H^{1}_{0}(B_{r}\setminus S,\xi^{2}dx)},\quad B(w,v)\leq\Lambda\|w\|_{H^{1}_{0}(B_{r}\setminus S,\xi^{2}dx)}\|v\|_{H^{1}_{0}(B_{r}\setminus S,\xi^{2}dx)}.
\end{equation*}
Besides, the norm of $F(\varphi)$ is
\begin{equation*}
    |F(\varphi)|\leq(\|\vec{f}\|_{L^{2}(B_{r}\setminus S,\xi^{2}dx)}+\|g\|_{L^{2}(B_{r}\setminus S,\rho\xi^{4}dx)})\|\varphi\|_{H^{1}_{0}(B_{r}\setminus S,\xi^{2}dx)}.
\end{equation*}

In the special case $\vec{f}=0$ and $g=0$, we have the following existence result:
\begin{proposition}\label{exist}
    Assume that $\lambda I\leq A\leq\Lambda I$. If $w_{0}\in H^{1}(B_{r}\setminus S,\xi^{2}dx)$, then there exists a unique weak solution $w\in H^{1}(B_{r}\setminus S,\xi^{2}dx)$ of
    \begin{equation*}
        \mathrm{div}(\xi^{2}A\nabla w)=0
    \end{equation*}
    with boundary data $w_{0}$ (i.e., $w-w_{0}\in H^{1}_{0}(B_{r}\setminus S,\xi^{2}dx)$). Besides, there exists some $C=C(\lambda,\Lambda)$ such that
    \begin{equation}
        \int_{B_{r}\setminus S}\xi^{2}|\nabla w|^{2}dx\leq C\int_{B_{r}\setminus S}\xi^{2}|\nabla w_{0}|^{2}dx.
    \end{equation}
\end{proposition}
\begin{proof}
    It suffices to find a solution $w_{1}\in H^{1}_{0}(B_{r}\setminus S,\xi^{2}dx)$ so that $\ds w_{1}\Big|_{|x|>r}=0$ and
    \begin{equation*}
        \mathrm{div}(\xi^{2}A\nabla w_{1})=-\mathrm{div}(\xi^{2}A\nabla w_{0}).
    \end{equation*}
    Let $\vec{f}=-A\nabla w_{0}$, then as $w_{0}\in H^{1}(B_{r}\setminus S,\xi^{2}dx)$, the linear functional $F(\varphi)$ is bounded. By Lax-Milgram lemma, $w_{1}$ exists and
    \begin{equation*}
        \int_{B_{r}\setminus S}\xi^{2}|\nabla w_{1}|^{2}\leq C\int_{B_{r}\setminus S}\xi^{2}|\vec{f}|^{2}\leq C\int_{B_{r}\setminus S}\xi^{2}|\nabla w_{0}|^{2}.
    \end{equation*}
    If $w=w_{0}+w_{1}$, then $\mathrm{div}(\xi^{2}A\nabla w)=0$ with boundary data $w_{0}$.
\end{proof}




\section{Schauder estimate in average sense}\label{only average}
In this section, we should give a $C^{1,\alpha}$ estimate in average sense of model equation \eqref{main again} at the origin. Before we move into the proof, we should mention that in Theorem~\ref{gammaschauder}, the assumption $A(0)=\delta^{ij}$ can be weakened, so that $\lambda I\leq A(0)\leq\Lambda I$, $A^{i,n+1}(0)=A^{n+1,i}(0)=0$ for $i\leq n$, and $A^{n,n}=A^{n+1,n+1}$. This can be reduced to the case $A(0)=\delta^{ij}$ by taking a linear coordinate change.

We first give two approximation lemmas.
\begin{lemma}\label{constant coefficient}
If $h\in H^{1}(B_{1}\setminus S,\xi^{2}dx)$ is an $x_{n+1}$-even function solving the equation
\begin{equation*}
    \mathrm{div}(\xi^{2}\nabla h)=0,
\end{equation*}
then there exists a linear ``polynomial'' $l$ such that for all $r\leq1/2$,
\begin{equation}\label{eq. 3.1}
    \|l\|_{C^{0,1}(B_{1}\setminus S)}^{2}+\frac{1}{r^{n+4}}\int_{B_{r}\setminus S}\frac{|h-l|^{2}}{\rho}dx\leq C(\lambda,\Lambda)\int_{B_{1}\setminus S}\frac{h^{2}}{\rho}dx.
\end{equation}
\end{lemma}
\begin{proof}
Let $h_{k}\in H^{1}(B_{1}\setminus S,\xi^{2}dx)\cap C^{\infty}(B_{1}\setminus S)\cap C^{0}(\overline{B_{1}\setminus S})$ be a $H^{1}(B_{1}\setminus S,\xi^{2}dx)$ approximation of $h$, then $H_{k}=\xi h_{k}$ is a smooth $H^{1}(B_{1}\setminus S,dx)$ approximation of $H=\xi h$ and each $H_{k}$ vanishes at $S$. Therefore, $H\Big|_{B_{1}\cap S}=0$ in trace sense. Notice that $\Delta H=0$, and the desired inequality follows from the $C^{6}$ boundary estimate of $H$ in the $(\xi,\eta)$-coordinate, see Lemma~\ref{harmonic functions} and Remark~\ref{rmk. harmonic remark} below.
\end{proof}

\begin{lemma}\label{harmonic functions}
    Assume that $\Delta H(x)=0$ in $B_{1}\setminus S$ and $H\Big|_{S}=0$ in trace sense. If $H\in H^{1}(B_{1},dx)$, then $H=H(x^{T},\xi,\eta)\in C^{6}(x^{T},\xi,\eta)$ near the origin, when written in the $(x^{T},\xi,\eta)$ coordinate, here $\xi\geq0$. See the definition of normal coordinates $(\xi,\eta)$ in \eqref{complex cordinate}.
\end{lemma}
\begin{remark}\label{rmk. harmonic remark}
    Moreover, we can confirm that
    \begin{equation*}
        H/\xi\in span\{1,x_{1},\cdots,x_{n-1},x_{n},\rho\}\oplus O(|x|^{2})
    \end{equation*}
    by analyzing all polynomials in $(x^{T},\xi,\eta)$ whose degree is at most $6$. In particular, we can write
    \begin{equation*}
        h=H/\xi=l+O(|x|^{2})
    \end{equation*}
    for some linear ``polynomial'', so $|h-l|^{2}=O(|x|^{4})$, and then the estimate \eqref{eq. 3.1} follows immediately.
\end{remark}
The proof can be seen in \cite{DS15} (Theorem 4.5). We briefly sketch the main idea below.
\begin{proof}
    We can apply the weak Harnack principle to $H_{+}$ and $H_{-}$ on the slit $S$, showing that $H$ is $C^{\delta}$ at $B_{3/4}\cap S$, hence $H\in C^{\delta}(B_{1/2})$ by an interior estimate. Now we can inductively show that $H$ is differentiable in $(\xi,\eta)$.
    \begin{itemize}
        \item[Step 1] We take discrete quotients in tangential directions to show that tangential derivatives $(D^{T})^{k}H$ are all $C^{\delta}$ for arbitrarily large $k$. This implies that $\Delta^{T}H+H_{\xi\xi}+H_{\eta\eta}=(1-4\xi^{2}-4\eta^{2})f$, where $f=\Delta^{T}H\in C^{\delta}$.
        \item[Step 2] As $H\Big|_{S}=0$, it also vanishes on $\{\xi=0\}$, so the boundary Schauder estimate applied to $H$ in $(\xi,\eta)$-coordinate implies that $H_{\xi},H_{\eta}\in C^{\delta}$.
        \item[Step 3] Let $\tilde{H}=\Delta^{T}H$, then it also satisfies $\Delta^{T}H+\tilde{H}_{\xi\xi}+\tilde{H}_{\eta\eta}=(1-4\xi^{2}-4\eta^{2})\tilde{f}$ for some $\tilde{f}$, we repeat Step 2 and obtain that $\tilde{H}_{\xi},\tilde{H}_{\eta}\in C^{\delta}$.
        \item[Step 4] Since $f_{\xi}=\tilde{H}_{\xi},f_{\eta}=\tilde{H}_{\eta}\in C^{\delta}$, we get that $H_{\xi\xi},H_{\xi\eta},H_{\eta\eta}\in C^{\delta}$ in $(\xi,\eta)$-coordinate.
        \item[Step 5] We repeat Step 2-4 infinitely many times, and finally obtain that $H$ is smooth in $(x^{T},\xi,\eta)$-coordinate.
    \end{itemize}
\end{proof}

\begin{lemma}\label{harmonic replacement}
Assume that $A^{ij}(0)=\delta^{ij}$ and $[A^{ij}]_{C^{\alpha}(B_{r}\setminus S)}\leq\varepsilon_{0}$. Let $w\in H^{1}_{loc}(B_{r}\setminus S,\xi^{2}dx)$ be a solution of \eqref{main again}, then there is a weak solution
\begin{equation}
    \mathrm{div}\Big(\xi^{2}\nabla h\Big)=0,\quad h\Big|_{(\partial B_{r/2})\setminus S}=w
\end{equation}
in the space $H^{1}(B_{r/2}\setminus S,\xi^{2}dx)$, and there exists $C$ so that
\begin{align}
    \int_{B_{r/2}\setminus S}\frac{h^{2}}{\rho}dx\leq&C\Big\{\int_{B_{r}\setminus S}(\xi^{2}|\vec{f}|^{2}+\rho\xi^{4}g^{2})dx+\int_{B_{r}\setminus S}\frac{w^{2}}{\rho}dx\Big\},\label{h1}\\
    \int_{B_{r/2}\setminus S}\frac{|h-w|^{2}}{\rho}dx\leq&C\Big\{\int_{B_{r}\setminus S}(\xi^{2}|\vec{f}|^{2}+\rho\xi^{4}g^{2})dx+\varepsilon_{0}^{2}r^{2\alpha}\int_{B_{r}\setminus S}\frac{w^{2}}{\rho}dx\Big\}.\label{h2}
\end{align}
\end{lemma}
\begin{proof}
We first use Proposition~\ref{caccioppoli} and~\ref{exist} to get the existence of $h$ with
\begin{equation*}
    \int_{B_{r/2}\setminus S}\xi^{2}|\nabla h|^{2}\leq C\int_{B_{r/2}\setminus S}\xi^{2}|\nabla w|^{2}\leq C\Big\{\int_{B_{r}\setminus S}(\xi^{2}|\vec{f}|^{2}+\rho\xi^{4}g^{2})+\int_{B_{r}\setminus S}\frac{w^{2}}{\rho}\Big\}.
\end{equation*}
If we denote $\ds\vec{f}'=\Big(A^{ij}(x)-\delta^{ij}\Big)\partial_{j}h\cdot \vec{e}_{i}$, then
\begin{align}\label{fprime}
    \int_{B_{r/2}\setminus S}\xi^{2}|\vec{f}'|^{2}\leq&C\varepsilon_{0}^{2}r^{2\alpha}\int_{B_{r/2}\setminus S}\xi^{2}|\nabla h|^{2}\notag\\
    \leq&C\varepsilon_{0}^{2}r^{2\alpha}\Big\{\int_{B_{r}\setminus S}(\xi^{2}|\vec{f}|^{2}+\rho\xi^{4}g^{2})+\int_{B_{r}\setminus S}\frac{w^{2}}{\rho}\Big\}
\end{align}
and $(h-w)$ satisfies $\ds \mathrm{div}\Big(\xi^{2}A(x)\cdot\nabla (h-w)\Big)=\mathrm{div}\Big(\xi^{2}(\vec{f}'-\vec{f})\Big)-\xi^{2}g$ with zero boundary data at $(\partial B_{r/2})\setminus S$. Proposition~\ref{thin poincare} and~\ref{dirichlet}, applied to $(h-w)$, give
\begin{equation*}
    \int_{B_{r/2}\setminus S}\frac{|h-w|^{2}}{\rho}\leq C\int_{B_{r/2}\setminus S}\xi^{2}|\nabla(h-w)|^{2}\leq C\int_{B_{r/2}\setminus S}\big(\xi^{2}|\vec{f}|^{2}+\xi^{2}|\vec{f}'|^{2}+\rho\xi^{4}g^{2}\big).
\end{equation*}
By using \eqref{fprime} we prove \eqref{h2}. Now \eqref{h1} follows from the triangle inequality.
\end{proof}
Now let us prove the Schauder estimate of $w$ in an average sense.

\begin{proof}[Proof of Proposition~\ref{gammaschauder}]Without loss of generality, we can assume $\ds\lim_{x\to0}\vec{f}=0$. Otherwise, if $\vec{f}_{0}=c_{1}e_{1}+,\cdots+c_{n}\vec{e}_{n}+c_{\rho}\nabla\rho\neq0$, then we study the equation of
\begin{equation*}
    w'=w-c_{1}x_{1}-\cdots-c_{n}x_{n}-c_{\rho}\rho,
\end{equation*}
which is reduced to the case $\ds\lim_{x\to0}\vec{f}=0$. We inductively define
\begin{equation}
    w_{0}=w,\quad w_{k+1}=w_{k}-l_{k}(k\geq0),\quad L_{k}=w-w_{k}=\sum_{i=0}^{k}l_{i},
\end{equation}
where linear ``polynomials'' satisfying $\mathrm{div}(\xi^{2}l_{k})=0$ will be chosen in \eqref{latereplace}. We also define
\begin{equation}
\vec{f}_{0}=\vec{f},\quad\vec{f}_{k+1}=\vec{f}_{k}+\Big(\delta^{ij}-A^{ij}(x)\Big)\Big(\partial_{j}l_{k}\Big)\vec{e}_{i},(k\geq0).
\end{equation}
By induction, we know $\ds\lim_{x\to0}\vec{f}_{k}=0$ and $\ds \mathrm{div}(\xi^{2}A\cdot\nabla w_{k})=\mathrm{div}(\xi^{2}\vec{f}_{k})+\xi^{2}g$ for all $k\geq0$. Let $\ds\varepsilon_{0}^{\frac{2}{n+4}}\leq\lambda\leq\frac{1}{4}$ be a shrinking rate, and we define three quantities
\begin{align}
    \sigma_{k}^{2}:=&\frac{1}{\lambda^{k(n+2+2\alpha)}}\int_{B_{\lambda^{k}}\setminus S}\frac{w_{k}^{2}}{\rho}dx,\\
    \phi_{k}:=&[\vec{f}_{k}]_{C^{\alpha}(B_{\lambda^{k}}\setminus S)},\\
    \gamma_{k}^{2}:=&\frac{1}{\lambda^{k(n+2+2\alpha)}}\int_{B_{\lambda^{k}}\setminus S}\rho\xi^{4}g^{2}dx\leq C(n,\alpha)\|g\|_{L^{p}(B_{1}\setminus S,\rho\xi^{2}dx)}^{2}.\label{g actual}
\end{align}
Here, $\sigma_{k}$ measures how $w$ and $L_{k}$ differ in $C^{1,\alpha}$ sense. Let $h_{k}\in H^{1}(B_{\lambda^{k}/2}\setminus S,\xi^{2}dx)$ be a replacement of $w_{k}$ in $B_{\lambda^{k}/2}\setminus S$, and $l_{k}$ be the linearization of $h_{k}$, i.e.
\begin{equation}\label{latereplace}
    \left\{
    \begin{aligned}
    &\mathrm{div}\Big(\xi^{2}\nabla h_{k}\Big)=0\\
    &h\Big|_{(\partial B_{\lambda^{k}/2})^{+}}=w_{k}
    \end{aligned}
    \right.,\quad l_{k}(x)=linearize(h_{k}).
\end{equation}
By applying Lemma~\ref{harmonic replacement} and Lemma~\ref{constant coefficient} to $w_{k+1}=(w_{k}-h_{k})+(h_{k}-l_{k})$, we have
\begin{equation}\label{iteration inequality}
\left\{
    \begin{aligned}
    &\phi_{k+1}\leq C(n)(\phi_{k}+\varepsilon_{0}\lambda^{k\alpha}\sigma_{k}+\varepsilon_{0}\lambda^{k\alpha}\gamma_{k}),\\
    &\sigma_{k+1}^{2}\leq C(n)\lambda^{2-2\alpha}\sigma_{k}^{2}+\frac{C(n)}{\lambda^{n+2+2\alpha}}(\phi_{k}^{2}+\gamma_{k}^{2}),\\
    &|l_{k}(0)|+|\nabla l_{k}(0)|\leq C(n)\lambda^{k\alpha}(\phi_{k}+\gamma_{k}+\sigma_{k}).
    \end{aligned}
    \right.
\end{equation}
Clearly, when $\varepsilon_{0}$ is very small, then there is room to choose a $\lambda=\lambda(n,\alpha)$ satisfying
\begin{equation*}
    \varepsilon_{0}^{\frac{2}{n+4}}\leq\lambda\leq\frac{1}{4},\quad C(n)\lambda^{2-2\alpha}\leq\frac{1}{2}.
\end{equation*}
With this the iteration inequality \eqref{iteration inequality} implies
\begin{align*}
    \sigma_{k}+\|L_{k}\|_{C^{0,1}(B_{1}\setminus S)}\leq&C(n,\alpha)(\phi_{0}+\sigma_{0}+\|g\|_{L^{p}(B_{1}\setminus S,\rho\xi^{2}dx)})\\
    \leq&C(\|\vec{f}\|_{C^{\alpha}(B_{1}\setminus S)}+\|g\|_{L^{p}(B_{1}\setminus S,\rho\xi^{2}dx)})+\|w\|_{L^{2}(B_{1}\setminus S,\xi^{2}dx)}).
\end{align*}
Besides, it follows that the sequence of linear ``polynomials'' $L_{k}$ converges to $L$ with
\begin{equation*}
    \|L\|_{C^{0,1}(B_{1}\setminus S)}\leq C(\|\vec{f}\|_{C^{\alpha}(B_{1}\setminus S)}+\|g\|_{L^{p}(B_{1}\setminus S,\rho\xi^{2}dx)})+\|w\|_{L^{2}(B_{1}\setminus S,\xi^{2}dx)}).
\end{equation*}
$L$ is a $C^{1,\alpha}$ approximation of $w$ because $\sigma_{k}$'s are bounded.

Finally, if $\ds\lim_{x\to0}\vec{f}\neq0$, then we can subtract a linear function from $w$ so that the remainder $w'$ satisfies $\mathrm{div}(\xi^{2}A\cdot\nabla w')=\mathrm{div}(\xi^{2}\vec{f}')$ with $\vec{f}'(0)=0$. Now we are reduced to the first case.
\end{proof}
\begin{remark}\label{absorb remark}
    If the right-hand side of \eqref{main again} has a term $\ds\frac{\xi}{\sqrt{\rho}}h$, where
    \begin{equation*}
        (h-c\sqrt{\rho}\frac{\partial\xi}{\partial x_{n}})=(h-c\frac{\xi}{2\sqrt{\rho}})\in C^{\alpha}(0)\cap O(|x|^{\alpha})
    \end{equation*}
    for some constant $c$, then it could be absorbed into $\mathrm{div}(\xi^{2}\vec{f})$ by setting
    \begin{equation}\label{eq. absorb construction}
        \vec{f}(x):=g(x^{T},\xi,\eta)\frac{\partial}{\partial\xi}=g(x^{T},\xi,\eta)(2\xi \vec{e}_{n}+2\eta \vec{e}_{n+1}),
    \end{equation}
    where $g(x^{T},\xi,\eta)$ is expressed as the following integral:
    \begin{equation*}
        g(x^{T},\xi,\eta)=\frac{1}{(\xi^{2}+\eta^{2})\xi^{2}}\int_{0}^{\xi}t\sqrt{t^{2}+\eta^{2}}h(x^{T},t,\eta)dt.
    \end{equation*}
    In fact, we have
    \begin{align*}
        \mathrm{div}(\xi^{2}g\frac{\partial}{\partial\xi})=&\mathrm{div}(2\xi^{3}g\vec{e}_{n}+2\xi^{2}\eta g\vec{e}_{n+1})=\frac{\partial}{\partial x_{n}}(2\xi^{3}g)+\frac{\partial}{\partial x_{n+1}}(2\xi^{2}\eta g)\\
        =&\Big(\frac{\xi}{2\rho}\frac{\partial}{\partial\xi}(2\xi^{3}g)-\frac{\eta}{2\rho}\frac{\partial}{\partial\xi}(2\xi^{3}g)\Big)+\Big(\frac{\eta}{2\rho}\frac{\partial}{\partial\xi}(2\xi^{2}\eta g)+\frac{\xi}{2\rho}\frac{\partial}{\partial\xi}(2\xi^{2}\eta g)\Big)\\
        =&\xi^{2}\frac{\partial g}{\partial\xi}+\frac{4\xi^{3}+2\xi\eta^{2}}{\rho}g=\frac{1}{\rho}\frac{\partial}{\partial\xi}(\rho\xi^{2}g)=\frac{\xi}{\sqrt{\rho}}h(x^{T},\xi,\eta).
    \end{align*}
    Moreover, it could be verified that $\ds\int_{0}^{\xi}t\sqrt{t^{2}+\eta^{2}}dt$ is of order $\xi^{2}\sqrt{\rho}$, so if $h(x)$ is $L^{\infty}$ in the $x$-variable, then $\vec{f}(x)$ constructed in \eqref{eq. absorb construction} is $L^{\infty}$. Moreover, if $(h-c\frac{\xi}{2\sqrt{\rho}})\in C^{\alpha}(0)\cap O(|x|^{\alpha})$, by decomposing $h$ into $(h-c\frac{\xi}{2\sqrt{\rho}})$ and $c\frac{\xi}{2\sqrt{\rho}}$, we see that
    \begin{equation*}
        \vec{f}=c_{1}(\frac{\xi^{2}}{\rho}\vec{e}_{n}+\frac{\xi\eta}{\rho}\vec{e}_{n+1})+\vec{f}_{2},\quad\mbox{where }\vec{f}_{2}\in C^{\alpha}(0)\cap O(|x|^{\alpha}).
    \end{equation*}
    Notice that $\frac{\xi^{2}}{\rho}\vec{e}_{n}+\frac{\xi\eta}{\rho}\vec{e}_{n+1}$ is a linear combination of $\vec{e}_{n}$ and $\nabla\rho$, then we have that $\vec{f}\in C^{\alpha}(B_{1},0)$ as defined in Definition~\ref{def. 1.6}.
\end{remark}

\begin{remark}\label{xiphi on the RHS}
If $\alpha<1/2$, then the right-hand side of \eqref{main again} can also have a term $\xi\phi$, where $\phi\in L^{\infty}$. This is because when $g=\phi/\xi$, then $\xi\phi=\xi^{2}g$, and $g$ satisfies the weakened assumption \eqref{g weaken}.
\end{remark}



\section{Estimate of ratio \texorpdfstring{$v=u/\xi$}{Lg}}
\subsection{Equivalent descriptions of property \texorpdfstring{$(\mathcal{F})$}{Lg}}
The property $(\mathcal{F}_{A})$, or $(\mathcal{F})$ if the matrix $A$ is known, given in the introduction is easier for the readers to understand, but with it only it's hard to do estimate, so we provide a few parallel properties.

In $\mathbb{R}^{n+1}\setminus S$, we define a cone of opening $1$, radius $r>0$, centered at $(x^{T},0)$ as $Cone_{r}(x^{T},0)$, or simply $Cone_{r}(x^{T})$. More explicitly,
\begin{equation}
    Cone_{r}(x^{T}):=\{y=(y^{T},y^{\perp}): |y^{\perp}|\leq r, |y^{T}-x^{T}|\leq|y^{\perp}|\}\setminus S.
\end{equation}
We say $f(x)$ defined in $B_{R}\setminus S$ satisfies properties $(\mathcal{F}_{1})$, $(\mathcal{F}_{2})$ or $(\mathcal{F}_{3})$ if:
\begin{itemize}
    \item[$(\mathcal{F}_{1})$] There exists a constant $C$, so that for every $Cone_{r}(x^{T})\in B_{R}\setminus S$,
    \begin{equation*}
        \Big\|\frac{f(y)}{\bar{u}_{A(x^{T})}(y)}\Big\|_{C^{\alpha}(Cone_{r}(x^{T})\setminus Cone_{r/2}(x^{T}))}\leq C,\quad\mbox{w.r.t. path distance}\ dist(\cdot,\cdot).
    \end{equation*}
    \item[$(\mathcal{F}_{2})$] There exists a constant $C$, so that for every $Cone_{r}(x^{T})\in B_{R}\setminus S$,
    \begin{equation*}
        \Big\|\frac{f(y)}{\bar{u}_{A(x^{T})}(y)}\Big\|_{C^{\alpha}(Cone_{r}(x^{T}))}\leq C,\quad\mbox{w.r.t. path distance}\ dist(\cdot,\cdot).
    \end{equation*}
    \item[$(\mathcal{F}_{3})$] There exists a constant $C$, so that for every $B_{2r}(x^{T})\subseteq B_{R}$, $\ds\Big|\frac{f(y)}{\bar{u}_{A(x^{T})}(y)}\Big|\leq C$ in $B_{r}(x^{T})$, and for every pair $y\in B_{r}(x^{T})$, $z\in Cone_{r}(x^{T})$, we have
    \begin{equation*}
        \Big|\frac{f(y)}{\bar{u}_{A(x^{T})}(y)}-\frac{f(z)}{\bar{u}_{A(x^{T})}(z)}\Big|\leq C\cdot dist(y,z)^{\alpha}.
    \end{equation*}
\end{itemize}
Again, $\bar{u}_{x^{T}}$ is an abbreviation of $\bar{u}_{A(x^{T})}$. Just like $(\mathcal{F})$, these three properties are also defined when assuming $A$ is known on $\mathbb{R}^{n-1}$.

If $\lambda I\leq A\leq\Lambda I$ and $[A]_{C^{\alpha}}$ is small, then the properties $(\mathcal{F})$, $(\mathcal{F}_{1})$, $(\mathcal{F}_{2})$, $(\mathcal{F}_{3})$ are equivalent up to a shrinking of radius, meaning that for example, if $f(x)$ has property $(\mathcal{F})$ in $B_{R}$, then it also has property $(\mathcal{F}_{1})$ in $B_{R/100}$. The proof uses some similar technique like in \cite{JV23} and is postponed to section~\ref{equivalence}.
\subsection{H\"older estimate} In this section, we study the H\"older continuity of \eqref{uniform}. We state the equation again,
\begin{equation}\label{uniform again}
    \mathrm{div}(A\nabla u)=\mathrm{div}(\frac{\vec{f}}{\sqrt{\rho}})\ \mbox{in}\ B_{1}\setminus S,\quad u\Big|_{S}=0.
\end{equation}
\begin{lemma}\label{Holder average}
Assume that $A^{ij}(0)=\delta^{ij}$, $[A^{ij}]_{C^{\alpha}(B_{1}^{+})}\leq\varepsilon_{0}$, and $u\in L^{2}(B_{1}\setminus S,dx)\cap H^{1}_{loc}$ vanishing at $S$ is an $x_{n+1}$-even solution of \eqref{uniform again} with $\vec{f}(0)=0$. Given that $\varepsilon_{0}$ is small, there exist two constants $\bar{c}$ and $C=C(n,\alpha)$ (here, $C(n,\alpha)$ is universal) such that
\begin{equation}\label{average holder formula}
    |\bar{c}|^{2}+\frac{1}{r^{n+2\alpha}}\int_{B_{r}\setminus S}\frac{|u/\xi-\bar{c}|^{2}}{\rho}dx\leq C\Big\{\|\vec{f}\|_{C^{\alpha}(B_{1},0)}^{2}+\|u\|_{L^{2}(B_{1}\setminus S,dx)}^{2}\Big\}.
\end{equation}
Here, the norm $\|\vec{f}\|_{C^{\alpha}(B_{1},0)}$ is the same as that in Proposition~\ref{gammaschauder}.
\end{lemma}
\begin{proof}
    We write $v=u/\xi$, and also denote sequences $u_{k}=u-c_{k}\xi$ and $v_{k}=u_{k}/\xi$, where $c_{k}$'s are constants yet to be decided. It follows that $u_{k}$ satisfies the equation
    \begin{equation*}
        \mathrm{div}(A\nabla u_{k})=\mathrm{div}(\frac{\vec{f}_{k}}{\sqrt{\rho}}),\quad\vec{f}_{k}=\vec{f}-c_{k}\sqrt{\rho}(A-\delta^{ij})\nabla\xi.
    \end{equation*}
    As $\ds|\nabla\xi|=\frac{1}{2\sqrt{\rho}}$, we see that $|f_{k}(x)|\leq\phi_{k}|x|^{\alpha}$, where
    \begin{equation}\label{holder induction 1}
        \phi_{k}\leq C(\|\vec{f}\|_{C^{\alpha}(B_{1},0)}+\varepsilon_{0}|c_{k}|).
    \end{equation}

    As $u_{k}$ vanishes at $S$, we have the following estimate (will be given later in Lemma~\ref{lem. hardy}):
\begin{equation}\label{hardy}
    \int_{B_{r}\setminus S}\frac{v_{k}^{2}}{\rho}dx=\int_{B_{r}\setminus S}\frac{u_{k}^{2}}{\xi^{2}\rho}dx\leq C\int_{B_{r}\setminus S}|\nabla u_{k}|^{2}dx.
\end{equation}
Besides, the Caccioppoli inequality for uniform elliptic equation implies that
\begin{equation}\label{standard caccioppoli}
    \int_{B_{r}\setminus S}|\nabla u_{k}|^{2}dx\leq C\int_{B_{2r}\setminus S}(\frac{u_{k}^{2}}{r^{2}}+\frac{|\vec{f}_{k}|^{2}}{\rho})dx.
\end{equation}
    
    We make the following claim:
    \begin{itemize}
        \item Claim: there exists a converging sequence $c_{k}$ and some $\lambda<1$, so that
    \begin{equation*}
        \int_{B_{\lambda^{k}}\setminus S}u_{k}^{2}dx\leq C\lambda^{k(n+2+2\alpha)},
    \end{equation*}
    \end{itemize}
    If the claim in correct, then we can use \eqref{hardy} and \eqref{standard caccioppoli} to get \eqref{average holder formula} for $\ds\bar{c}=\lim_{k\to\infty}c_{k}$.

    The remaining part is to prove the claim above. We follow a similar Campanato iteration like in Proposition~\ref{gammaschauder}. Without loss of generality, we assume that $\|\vec{f}\|_{C^{\alpha}(B_{1},0)}=\|u\|_{L^{2}(B_{1}\setminus S,dx)}=1$. We let $c_{0}=0$, $\ds\varepsilon_{0}^{\frac{2}{n+4}}\leq\lambda\leq1/4$, and define
\begin{equation*}
    \sigma_{k}^{2}=\frac{1}{\lambda^{k(n+2+2\alpha)}}\int_{B_{\lambda^{k}}\setminus S}u_{k}^{2}dx.
\end{equation*}
Clearly we have $\sigma_{0}=1$. By \eqref{standard caccioppoli} we have that
\begin{equation*}
    \int_{B_{\lambda^{k}/2}\setminus S}|\nabla u_{k}|^{2}\leq C\lambda^{k(n+2\alpha)}(\sigma_{k}^{2}+\phi_{k}^{2}),
\end{equation*}
where $\phi_{k}\leq C(1+\varepsilon_{0}|c_{k}|)$. Let $H_{k}$ be a harmonic replacement obtained using the Lax-Milgram lemma or energy minimizing method, so that it vanishes on $S$ in trace sense and
\begin{equation*}
    \Delta H_{k}=0\ \mbox{in}\ B_{\lambda^{k}/2}\setminus S,\quad H_{k}\Big|_{\partial B_{\lambda^{k}/2}\setminus S}=u_{k}.
\end{equation*}
By Lemma~\ref{harmonic functions} plus that $H_{k}$ vanishes on $S$, we can set $\ds c_{k+1}=c_{k}+\lim_{x\to0}\frac{H_{k}}{\xi}$, and
\begin{equation*}
    |c_{k}-c_{k+1}|^{2}+\frac{1}{\lambda^{n+4}r^{n+4}}\int_{B_{\lambda r}\setminus S}|H_{k}+c_{k}\xi-c_{k+1}\xi|^{2}\leq\frac{C}{r^{n+2}}\int_{B_{r}\setminus S}|H_{k}|^{2}
\end{equation*}
for all $\lambda<1$ if $H_{k}$ is $x_{n+1}$-even. It follows that
\begin{equation}\label{holder induction 2}
    |c_{k+1}|\leq|c_{k}|+C\lambda^{k\alpha}(\sigma_{k}+\phi_{k})
\end{equation}
As $u_{k+1}=(u_{k}-H_{k})+(H_{k}-c_{k+1}\xi+c_{k}\xi)$, we have
\begin{equation}\label{holder induction 3}
    \sigma_{k+1}^{2}\leq C(n)\lambda^{2-2\alpha}\sigma_{k}^{2}+\frac{C(n)}{\lambda^{n+2+2\alpha}}\phi_{k}^{2}.
\end{equation}
The iterative system \eqref{holder induction 1}\eqref{holder induction 2}\eqref{holder induction 3}, if we further assume $C(n)\lambda^{2-2\alpha}\leq1/2$, will imply the boundedness of $\sigma_{k}$ and convergence of $c_{k}$. Such a $\lambda$ which simultaneously satisfies $\ds\varepsilon_{0}^{\frac{2}{n+4}}\leq\lambda\leq1/4$ mentioned before exists, if $\varepsilon_{0}$ is small enough. This proves the claim.
\end{proof}

Now we give the proof of \eqref{hardy}. It is a Hardy-type inequality, and we state it again below:
\begin{lemma}\label{lem. hardy}
    If $u\in C^{\infty}_{loc}(B_{r}/S)$ vanishes on the slit $S$, then for $v=u/\xi$, we have
\begin{equation*}
    \int_{B_{r}\setminus S}\frac{v^{2}}{\rho}dx=\int_{B_{r}\setminus S}\frac{u^{2}}{\xi^{2}\rho}dx\leq C\int_{B_{r}\setminus S}|\nabla u|^{2}dx.
\end{equation*}
\end{lemma}
\begin{proof}
Instead of using the $x$-coordinate, we use the $(x^{T},\xi,\eta)$-coordinate mentioned in \eqref{complex cordinate}. It follows that
\begin{align*}
    \int_{B_{r}\setminus S}|\nabla u|^{2}dx\geq&\int_{B_{r}\setminus S}|\nabla^{\perp}u|^{2}dx\geq\int_{\Omega}|\partial_{\xi}u|^{2}dx^{T}d\xi d\eta,\\
    \int_{B_{r}\setminus S}\frac{u^{2}}{\xi^{2}\rho}dx=&\int_{\Omega}\frac{u^{2}}{\xi^{2}}dx^{T}d\xi d\eta,
\end{align*}
where $\Omega=\{(x^{T},\xi,\eta):\xi>0,|x^{T}|^{2}+(\xi^{2}+\eta^{2})^{2}\leq r^{2}\}$ corresponds to the $B_{r}\setminus S$ region in $(x^{T},\xi,\eta)$-coordinate. We fix $x^{T}$ and $\eta$, and write $u(\xi)=u(x^{T},\xi,\eta)$, then it suffices to find a constant $C$ independent of $(x^{T},\eta)$ and the bound $A$, so that for all $u(\xi)\in H^{1}_{loc}(\mathbb{R}_{+})$ vanishing at $0$, we have
\begin{equation*}
C\int_{0}^{A}|u'|^{2}d\xi\geq\int_{0}^{A}\frac{u^{2}}{\xi^{2}}d\xi.
\end{equation*}
The strategy is to show a $W^{1,1}$ Hardy inequality $\ds\int_{0}^{A}\frac{|u'|}{\xi}d\xi\geq\int_{0}^{A}\frac{|u|}{\xi^{2}}d\xi$ and then replace $u$ with $u^{2}$, just like the method in \eqref{regular poincare}. For simplicity, we assume $w\in C^{\infty}$, then the $W^{1,1}$ Hardy inequality follows from changing the order of integration:
\begin{align*}
    \int_{0}^{A}\frac{|u(\xi)|}{\xi^{2}}d\xi\leq\int_{0}^{A}\frac{1}{\xi^{2}}\int_{0}^{\xi}|u'(h)|dhd\xi=\int_{0}^{A}\int_{h}^{A}\frac{1}{\xi^{2}}|u'(h)|d\xi dh\leq\int_{0}^{A}\frac{|u'(h)|}{h}dh.
\end{align*}
\end{proof}

As a consequence of Lemma~\ref{Holder average}, we have the following interior regularity of $u/\xi$.
\begin{theorem}\label{Holder estimate}
Under the same assumption of Lemma~\ref{Holder average}, and further assume that $\vec{f}$ is $C^{\alpha}$ inside $B_{1}$ and $\vec{f}(x)=0$ for all $x\in\mathbb{R}^{n-1}$, then $u(x)$ has the property $(\mathcal{F})$ in $B_{1/2}$, and $\sqrt{\rho}\nabla(u-c\xi)$ belongs to $C^{\alpha}(0)\cap O(|x|^{\alpha})$ in $B_{1/2}\setminus S$ for some constant $c$.
\end{theorem}
\begin{proof}
In each conic annulus $Cone_{r}\setminus Cone_{r/2}$, whose closure is contained in $B_{2r}\setminus B_{r/4}$, \eqref{uniform} is a non-degenerate equation, and we infer from \eqref{average holder formula} that
\begin{equation}\label{eq. 4.9}
    r^{-\frac{n+1}{2}}\|\tilde{u}\|_{L^{2}(Cone_{r}\setminus Cone_{r/2})}\leq r^{\frac{1}{2}+\alpha},\quad \tilde{u}=u-c\xi
\end{equation}
for some coefficient $c$. Then, by the standard boundary Schauder estimate, we have that under the distance function $dist(\cdot,\cdot)$,
    \begin{equation*}
        [u]_{C^{1,\alpha}(Cone_{r}\setminus Cone_{r/2})}\leq C r^{-1/2}
    \end{equation*}
    for some $C$ independent of $r$. Then, we see
    \begin{equation*}
        [\sqrt{\rho}\nabla\tilde{u}]_{C^{\alpha}(Cone_{r}\setminus Cone_{r/2})}\leq C,
    \end{equation*}
    so $[\sqrt{\rho}\nabla\tilde{u}]_{C^{\alpha}(Cone_{3/4})}\leq C$. Moreover, by \eqref{eq. 4.9}, we see that the limit of $\sqrt{\rho(x)}\nabla\tilde{u}(x)$ when $x$ tends to $0$ inside the cone $Cone_{1}$ is equal to $0$. For every $y\in B_{1/4}$, we construct
    \begin{equation*}
        z=y^{T}+\frac{|y^{T}|}{|y^{\perp}|}y^{\perp},
    \end{equation*}
    then similar to the computation in Lemma~\ref{sheaf},
    \begin{align}\label{sqrt rho nabla u}
        &\sqrt{\rho(y)}\nabla\tilde{u}(y)-\sqrt{\rho(z)}\nabla\tilde{u}(z)\\\notag
        =&\Big(\sqrt{\rho_{\kappa_{y^{T}}}(y)}\nabla\tilde{u}(y)-\sqrt{\rho_{\kappa_{y^{T}}}(z)}\nabla\tilde{u}(z)\Big)+\Big(\sqrt{\rho_{\kappa_{y^{T}}}(z)}-\sqrt{\rho(z)}\Big)\nabla\tilde{u}(z),
    \end{align}
    where
    \begin{equation*}
        \rho_{\kappa_{y^{T}}}=\sqrt{x_{n}^{2}+\kappa_{y^{T}}^{2}x_{n+1}^{2}},\quad\kappa_{y^{T}}=(\frac{\bar{a}(y^{T})^{nn}}{\bar{a}(y^{T})^{n+1,n+1}})^{1/2}.
    \end{equation*}
    In fact, since $z^{\perp}$ is parallel to $y^{\perp}$, the first term in \eqref{sqrt rho nabla u} is invariant if we add a $\frac{1}{2}$-order homogeneous function (for example, $c\xi$ or $c\xi_{\kappa_{y^{T}}}$) to $\tilde{u}$.
    
    Consequently, by combining \eqref{sqrt rho nabla u} with $[\sqrt{\rho}\nabla\tilde{u}]_{C^{\alpha}(Cone_{3/4})}\leq C$, we conclude that $\sqrt{\rho}\nabla\tilde{u}\in C^{\alpha}(0)\cap O(|x|^{\alpha})$.
    
    Besides, as $u\Big|_{S}=0$ and
    \begin{equation*}
        \xi\sim\frac{|x_{n+1}|}{\sqrt{\rho}}\quad\mbox{in }(Cone_{r}\setminus Cone_{r/2})\cap\{x_{n}<0\},
    \end{equation*}
    we conclude that $[u/\xi]_{C^{\alpha}(Cone_{r}\setminus Cone_{r/2})}\leq C$ with respect to $dist(\cdot,\cdot)$, which means $u$ has property $(\mathcal{F}_{1})$. As $(\mathcal{F}_{1})$ is ``equivalent'' to $(\mathcal{F})$ up to a shrinking of radius, see Appendix~\ref{equivalence}, we have finished the proof of Theorem~\ref{Holder estimate}.
\end{proof}

\subsection{Non-degeneracy}
We can use a similar method to prove that the growth rate of $u$ is exactly proportional to $\xi$ as long as $u$ is close to $\xi$ in $L^{2}$ sense.
\begin{proposition}\label{hopf}
    Assume that $A(0)=\delta^{ij}$ and $[A]_{C^{\alpha}(B_{1}\setminus S)}\leq\varepsilon_{0}$. $\vec{f}\Big|_{\mathbb{R}^{n-1}}=0$ and $[\vec{f}]_{C^{\alpha}(B_{1}\setminus S)}\leq\varepsilon_{0}$. $u$ is an $x_{n+1}$-even solution of \eqref{uniform again} with zero trace at $S$ and satisfies $\|u-\xi\|_{L^{2}(B_{1}\setminus S)}\leq\varepsilon_{0}$. Given that $\varepsilon_{0}$ is small enough, then we have
    \begin{equation}
        \inf_{B_{1/8}\setminus S}\frac{u}{\xi}\geq\frac{1}{2}.
    \end{equation}
\end{proposition}
\begin{proof}
    For every point $x^{T}\in\mathbb{R}^{n-1}\cap B_{1/2}$, as $|A(x^{T})-\delta^{ij}|\leq\varepsilon_{0}$, the corresponding homogeneous solution $\bar{u}_{x^{T}}=\bar{u}_{A(x^{T})}$ is close to $\xi$ in $L^{2}$ sense, so
    \begin{equation*}
        \|u-\bar{u}_{x^{T}}\|_{L^{2}(B_{1/2}(x^{T})\setminus S)}\leq\|u-\xi\|_{L^{2}(B_{1/2}(x^{T})\setminus S)}+\|\xi-\bar{u}_{x^{T}}\|_{L^{2}(B_{1/2}(x^{T})\setminus S)}\leq C\varepsilon_{0}.
    \end{equation*}
    At each center $x^{T}$, the function $\tilde{u}=(u-\bar{u}_{x^{T}})$ satisfies
    \begin{equation*}
        \mathrm{div}(A\nabla\tilde{u})=\mathrm{div}\Big(\frac{\vec{f}}{\sqrt{\rho}}+(\bar{A}-A)\nabla\bar{u}_{\kappa}\Big)=:\mathrm{div}(\frac{\vec{f}'}{\sqrt{\rho}}),
    \end{equation*}
    where $\vec{f}'$ vanishes at $x^{T}$ with $[\vec{f}']_{C^{\alpha}(B_{1/2}(x^{T})\setminus S)}\leq C\varepsilon_{0}$. We repeat the method in Theorem~\ref{Holder estimate} and get that
    \begin{equation*}
        \|\frac{u-\bar{u}_{x^{T}}}{\bar{u}_{x^{T}}}\|_{C^{\alpha}(Cone_{1/4}(x^{T})\setminus S)}\leq C\varepsilon_{0}.
    \end{equation*}
    For every $y\in B_{1/8}$, we construct
    \begin{equation*}
        z=y^{T}+\frac{|y^{T}|}{|y^{\perp}|}y^{\perp}
    \end{equation*}
    and get
    \begin{align*}
        |\frac{u-\xi}{\xi}(y)|\leq&|\frac{u-\xi}{\xi}(z)|+|\frac{u}{\xi}(z)-\frac{u}{\xi}(y)|\\
        =&|\frac{u-\xi}{\xi}(z)|+\frac{\bar{u}_{x^{T}}}{\xi}(y)|\frac{u}{\bar{u}_{x^{T}}}(z)-\frac{u}{\bar{u}_{x^{T}}}(y)|\\
        \leq&\|\frac{u-\xi}{\xi}\|_{C^{\alpha}(Cone_{1/4}\setminus S)}+C(A(x^{T}))\|\frac{u-\bar{u}_{x^{T}}}{\bar{u}_{x^{T}}}\|_{C^{\alpha}(Cone_{1/4}(x^{T})\setminus S)}\\
        \leq&C\varepsilon_{0}.
    \end{align*}
    When $\varepsilon_{0}$ is small, then $\ds|\frac{u-\xi}{\xi}(y)|\leq\frac{1}{2}$ for all $y\in B_{1/8}$, which proves the non-degeneracy assertion.
\end{proof}

\section{\texorpdfstring{$C^{1,\alpha}$}{Lg} boundary Harnack for straight boundary}
With Lemma~\ref{ratio}, plus the Theorem~\ref{Holder estimate} previously obtained, we can prove the $C^{1,\alpha}$ boundary Harnack principle for straight boundary.

\begin{proof}[Proof of Theorem~\ref{HOBH straight}]After a linear coordinate change, we can assume that $A(0)=\delta^{ij}$ and $\bar{u}_{A(0)}=\xi$. In fact, to say $u_{2}/u_{A(0)}\geq c_{0}>0$, this is independent of the homogeneous solution $\bar{u}_{A(0)}$, since the ratio of $\bar{u}_{A(0)}$'s between different $A(0)$'s is bounded from above and below.
\begin{itemize}
    \item Step 1: We first absorb $\phi_{i}$ into the divergence term by
    \begin{equation}\label{phi also absorb}
        \mathrm{div}(\vec{f}_{i})+\phi_{i}(x)=\mathrm{div}\Big(\vec{f}_{i}+\int_{0}^{x_{n+1}}\phi_{i}(x_{1},\cdots,x_{n},h)dh\cdot \vec{e}_{n+1}\Big)=:\mathrm{div}(\vec{f}_{i}').
    \end{equation}
    We write $\ds\vec{f}_{i}'=\frac{\sqrt{\rho}\vec{f}_{i}'}{\sqrt{\rho}}$ so that $\sqrt{\rho}\vec{f}_{i}'\in C^{\alpha}$ and vanishes at $\mathbb{R}^{n-1}$, then by Theorem~\ref{Holder estimate} we have that $u_{i}$'s have the property $(\mathcal{F})$ in $B_{3/4}$, and  $\sqrt{\rho}\nabla(u_{i}-c_{i}\xi)$ are $C^{\alpha}(0)\cap O(|x|^{\alpha})$ in $B_{3/4}$ for some constants $c_{i}$.
    \item Step 2: By Lemma~\ref{ratio}, we know $\ds w=\frac{u_{1}}{u_{2}}$ satisfies \eqref{eq. ratio equation}. Now we write
    \begin{equation*}
        \tilde{A}=(\frac{u_{2}}{\xi})^{2}A,\quad\vec{f}=\frac{u_{2}}{\xi}\frac{\vec{f}_{1}}{\xi}-\frac{u_{1}}{\xi}\frac{\vec{f}_{2}}{\xi},
    \end{equation*}
    and add two additional term mentioned in Remark~\ref{absorb remark} and Remark~\ref{xiphi on the RHS}:
    \begin{equation*}
        h=(\frac{\vec{f}_{2}}{\xi})\cdot(\sqrt{\rho}\nabla u_{1})-(\frac{\vec{f}_{1}}{\xi})\cdot(\sqrt{\rho}\nabla u_{2}),\quad \phi=\frac{u_{2}}{\xi}\phi_{1}-\frac{u_{1}}{\xi}\phi_{2}.
    \end{equation*}
    We then have that
    \begin{equation*}
        \mathrm{div}(\xi^{2}\tilde{A}\nabla w)=\mathrm{div}(\xi^{2}\vec{f})+\frac{\xi}{\sqrt{\rho}}h+\xi\phi.
    \end{equation*}
    Notice that as $\vec{f}_{i}\cdot\vec{e}_{n+1}=0$, we have that $(h-c\frac{\xi}{2\sqrt{\rho}})\in C^{\alpha}(0)\cap O(|x|^{\alpha})$ for some constant $c$, so by absorbing $\frac{\xi}{\sqrt{\rho}}h$ into $\mathrm{div}(\xi^{2}\vec{f})$ (see Remark~\ref{absorb remark}), we see that $w$ satisfies an equation in the form \eqref{main}.
    
    By the assumption of $\vec{f}_{i}$ and by the regularity of $u_{i}$, we know
    \begin{equation*}
        \tilde{A},\vec{f},h\in C^{\alpha}(0),\quad \phi\in L^{\infty}\mbox{ near the origin}.
    \end{equation*}
    Besides, by the assumption $u_{2}/\xi\geq c_{0}>0$, the matrix $\tilde{A}$ also satisfies the uniformly elliptic condition
    \begin{equation*}
        \tilde{\lambda}I\leq\tilde{A}\leq\tilde{\Lambda}I.
    \end{equation*}
    Therefore, $w$ at the origin is $C^{1,\alpha}$ in average sense by applying Proposition~\ref{gammaschauder}.
    \item Step 3: Such an argument works not only at the origin, but also for all points on $\mathbb{R}^{n-1}$. More precisely, assume that at $x^{T}\in\mathbb{R}^{n-1}$, $\ds\kappa=(\frac{\bar{a}^{nn}}{\bar{a}^{n+1,n+1}}(x^{T}))^{1/2}$, then we set
\begin{equation}
\rho_{\kappa}=\sqrt{x_{n}^{2}+\kappa^{2}x_{n+1}^{2}}.
\end{equation}
It turns out that $w$ is approximated by a linear ``polynomial'' $L$ in the variables $(x_{1},\cdots,x_{n},\rho_{\kappa})$, such that
\begin{equation}\label{eq. 5.3}
    \|L\|_{C^{0,1}(B_{1}(x^{T})\setminus S)}^{2}+\frac{1}{r^{n+2+2\alpha}}\int_{B_{r}(x^{T})\setminus S}\frac{|w-L|^{2}}{\rho}dx\leq C.
\end{equation}
By overlapping $B_{r}(x^{T})$ between different $x^{T}\in\mathbb{R}^{n-1}$, we obtain that $D^{T}w\in C^{\alpha}(B_{1/2}\cap\mathbb{R}^{n-1})$, so restriction of $w$ on $\mathbb{R}^{n-1}$ is $C^{1,\alpha}$ in the classical sense. Precisely speaking, let $L_{0}$ and $L_{y^{T}}$ be two linear ``polynomials'' in the variables $(x_{1},\cdots,x_{n},\rho)$ and $(x_{1},\cdots,x_{n},\rho_{\kappa_{y^{T}}})$, respectively. Let $r=|y^{T}|$, then by \eqref{eq. 5.3}, we have:
\begin{align*}
    &\frac{1}{r^{n+2+2\alpha}}\int_{B_{r}\cap B_{r}(y^{T})\setminus S}\frac{|w-L_{0}|^{2}}{\rho}dx\leq C,\\
    &\frac{1}{r^{n+2+2\alpha}}\int_{B_{r}\cap B_{r}(y^{T})\setminus S}\frac{|w-L_{y^{T}}|^{2}}{\rho}dx\leq C.
\end{align*}
These two estimates, together with the triangle equality, imply the $C^{0,1}$ smallness of $|L_{y^{T}}-L_{0}|$. In particular,
\begin{equation*}
    |D^{T}w(y^{T})-D^{T}w(0)|=|D^{T}L_{y^{T}}-D^{T}L_{0}|\leq C r^{\alpha}=C|y^{T}|^{\alpha}.
\end{equation*}
\end{itemize}
\end{proof}


\section{Proof of Theorem~\ref{application}}
In this section, to avoid ambiguity, we use $y$-coordinate to represent the coordinate before straightening the boundary, and $x$-coordinate to be the new coordinate. We re-write \eqref{ytox abuse} in a more clear way as
\begin{equation}\label{ytox}
    x_{i}=y_{i},\forall i\leq n-1,\quad x_{n}=y_{n}-\gamma(y^{T}),\quad x_{n+1}=y_{n+1}.
\end{equation}
The equation \eqref{U in y abuse} will be better written as
\begin{equation}\label{U in y}
    \partial_{y_{p}}(B^{pq}\partial_{y_{q}}U)=F(y).
\end{equation}
Here, we use the matrix $B(y)$ as the original matrix in $y$-coordinate satisfying the assumptions (a)-(d) mentioned in the introduction.

Assume that $U(y)\geq0$ solves the thin obstacle problem \eqref{U in y} whose blow-up is $C^{1,\alpha}$-converging to a $3/2$-homogeneous solution near the regular point $0$. After the coordinate change \eqref{ytox}, we have that $u_{m}=U_{y_{m}}$ with $1\leq m\leq n$ satisfy the equations:
\begin{equation}\label{equationu}
    \partial_{x_{i}}(a^{ij}\partial_{x_{j}}u_{m})+\partial_{x_{i}}(\frac{\partial x_{i}}{\partial y_{p}}\partial_{y_{m}}b^{pq}\cdot u_{q})=\partial_{y_{m}}F(y),\quad a^{ij}=b^{pq}\frac{\partial x_{i}}{\partial y_{p}}\frac{\partial x_{j}}{\partial y_{q}}.
\end{equation}
One can check that matrix $A$ also satisfies assumptions (a)-(d). We write
\begin{equation}
    \vec{f}_{m}=-\frac{\partial x_{i}}{\partial y_{p}}\partial_{y_{m}}b^{pq}\cdot u_{q}\vec{e}_{i},\quad\phi_{m}=\partial_{y_{m}}F(y(x)),
\end{equation}
then
\begin{equation}
    \mathrm{div}(A\nabla u_{m})=\mathrm{div}(\vec{f}_{m})+\phi_{m}.
\end{equation}
We also notice that the assumption $b^{n+1,i}=b^{i,n+1}=0$ for $i\leq n$ implies that $\vec{f}_{m}\cdot\vec{e}_{n+1}=0$ for all $m\leq n$.

Since the coordinate change \eqref{ytox} is $C^{1,\alpha}$ and $F(y)\in W^{1,\infty}$, we know $\phi_{m}\in L^{\infty}$ in the $x$-coordinate. Besides, once we know all $u_{m}$'s for $m\leq n$ have property $(\mathcal{F})$ (to be shown in Step 1 below), then so do $\vec{f}_{m}$'s.

Like in \eqref{phi also absorb}, we again absorb $\phi_{m}$ into the divergence term in Step 1\&2.
\begin{itemize}
    \item Step 1: By Theorem~\ref{Holder estimate}, as $\ds\sqrt{\rho}\vec{f}_{m}'$ are $C^{\alpha}$ and vanish at $\mathbb{R}^{n-1}$, we have that $u_{m}$'s for $m\leq n$ are all $(\mathcal{F})$ in $B_{7/8}$.
    \item Step 2: We now show the non-degeneracy of $u_{m}$. Without loss of generality, we assume that $B(0)=\delta^{ij}$ and $\nabla\Gamma(0)=0$, then $a^{ij}(0)=\delta^{ij}$. As the blow-up of $U$ converges in $C^{1,\alpha}$ to the positive $3/2$ harmonic homogeneous solution $\ds Re\Big((x_{n}+ix_{n+1})^{3/2}\Big)$, we have that the blow-up of $u_{n}=\partial_{y_{n}}U$ converges to $\ds\xi=Re\Big((x_{n}+ix_{n+1})^{1/2}\Big)$ in $C^{\alpha}$ sense. When $\alpha<1/2$, then by zooming in at the origin, we have that $\sqrt{\rho}\vec{f}_{n}'$ vanish at $S$ and
\begin{equation*}
    [A]_{C^{\alpha}},\|u_{n}-\xi\|_{L^{2}},[\sqrt{\rho}\vec{f}_{n}']_{C^{\alpha}}\leq\varepsilon_{0}
\end{equation*}
as long as $\|\phi_{m}\|_{L^{\infty}},\|D^{i}A\|_{C^{\alpha}}\leq\varepsilon_{0}(i\leq n)$. By Proposition~\ref{hopf}, we see that $u_{n}/\xi\geq c_{0}>0$ in $B_{3/4}$.
\item Step 3: Finally, as discussed before, $u_{m}$'s being $(\mathcal{F})$ implies that $\vec{f}_{m}$'s for $m\leq n$ all have property $(\mathcal{F})$, so by Theorem~\ref{HOBH straight}, we see that $\ds w_{m}=\frac{u_{m}}{u_{n}}$ is $C^{1,\alpha}$ in average sense at the origin for $m\leq n$, and in particular, $w_{m}$ is $C^{1,\alpha}$ when restricted to $\mathbb{R}^{n-1}$ or $\Gamma$.
\end{itemize}

\appendix
\section{``Equivalence'' of regularity}\label{equivalence}
In the Appendix, we prove the ``equivalence'' of properties $(\mathcal{F})$, $(\mathcal{F}_{1})$, $(\mathcal{F}_{2})$, $(\mathcal{F}_{3})$. The most obvious relation is clearly $(\mathcal{F}_{3})\Rightarrow(\mathcal{F}_{2})\Leftrightarrow(\mathcal{F}_{1})$ up to a shrinking of radius.

The following lemmas prove the rest of ``equivalence''. In this section, when $\bar{u}_{\bar{A}}$ corresponds to the matrix $A(x^{T})$ where $x^{T}\in\mathbb{R}^{n-1}$, we can also simply write the homogeneous solution $\bar{u}_{A(x^{T})}$ as $\bar{u}_{x^{T}}$.
\begin{lemma}\label{sheaf}
    If $\lambda I\leq A\leq\Lambda I$, then $(\mathcal{F}_{2})\Rightarrow(\mathcal{F}_{3})$ up to a shrinking of radius.
\end{lemma}
\begin{proof}
    For simplicity, we assume that $x^{T}=0$, $A(0)=\delta^{ij}$, meaning that $\bar{u}_{0}=\xi$. If $y\in B_{r}(0)\setminus(S\cup Cone_{r})$, $z\in Cone_{r}(0)$, then $|y^{\perp}|\leq|y^{T}|$ and we let
    \begin{equation}\label{project to cone}
        w=y^{T}+\frac{|y^{T}|}{|y^{\perp}|}y^{\perp}\in Cone_{r}(0)\cap Cone_{r}(y^{T}).
    \end{equation}
    It's not hard to verify that
    \begin{equation*}
        |y-w|\leq2|y-z|,\quad|z-w|\leq3|y-z|.
    \end{equation*}
    In the cone $Cone_{r}(y^{T})$ we also have for $\bar{u}_{y^{T}}=\bar{u}_{A(y^{T})}$,
    \begin{equation*}
        C(\lambda,\Lambda)^{-1}\leq\bar{u}_{y^{T}}/\xi\leq C^{-1}(\lambda,\Lambda),\quad[f/\bar{u}_{y^{T}}]_{C^{\alpha}(Cone_{r})(y^{T})}\leq C.
    \end{equation*}
    Since $0,y^{T},y,w$ are co-planar, we have $\ds\frac{\bar{u}_{y^{T}}}{\xi}(w)=\frac{\bar{u}_{y^{T}}}{\xi}(y)$. Therefore,
    \begin{align*}
        |\frac{f(y)}{\xi(y)}-\frac{f(z)}{\xi(z)}|\leq&|\frac{f(y)}{\bar{u}_{y^{T}}(y)}\frac{\bar{u}_{y^{T}}(y)}{\xi(y)}-\frac{f(w)}{\bar{u}_{y^{T}}(w)}\frac{\bar{u}_{y^{T}}(w)}{\xi(w)}|+|\frac{f(w)}{\xi(w)}-\frac{f(z)}{\xi(z)}|\\
        \leq&\frac{\bar{u}_{y^{T}}}{\xi}(y)|\frac{f}{\bar{u}_{y^{T}}}(y)-\frac{f}{\bar{u}_{y^{T}}}(w)|+|\frac{f}{\xi}(w)-\frac{f}{\xi}(z)|\\
        \leq&C(|y-w|^{\alpha}+|w-z|^{\alpha})\leq(2^{\alpha}+3^{\alpha})C|y-z|^{\alpha}.
    \end{align*}
\end{proof}
\begin{lemma}
    If $\lambda I\leq A\leq\Lambda I$ and $[A]_{C^{\alpha}}$ is small, then $(\mathcal{F})\Rightarrow(\mathcal{F}_{1})$ up to a shrinking of radius.
\end{lemma}
\begin{proof}
    For simplicity, we assume that $x^{T}=0$, $A(0)=\delta^{ij}$, meaning that $\bar{u}_{0}=\xi$. It suffices to show that for every $y,z\in Cone_{r}(0)\setminus Cone_{r/2}(0)$, we have
    \begin{equation*}
        |\frac{\bar{u}_{y^{T}}(y)}{\xi(y)}h(y)-\frac{\bar{u}_{z^{T}}(z)}{\xi(z)}h(z)|\leq C\cdot dist(y,z)^{\alpha}.
    \end{equation*}
    The left-hand-side is a sum of $\ds D_{1}=\frac{\bar{u}_{z^{T}}(z)}{\xi(z)}\cdot|h(y)-h(z)|$ and
    \begin{equation*}
        D_{2}=|h(y)|\cdot|\frac{\bar{u}_{y^{T}}(z)}{\xi(z)}-\frac{\bar{u}_{z^{T}}(z)}{\xi(z)}|,\quad D_{3}=|h(y)|\cdot|\frac{\bar{u}_{y^{T}}(y)}{\xi(y)}-\frac{\bar{u}_{y^{T}}(z)}{\xi(z)}|.
    \end{equation*}
    As $h(x)$ is $C^{\alpha}$, we have $D_{1}\leq C\cdot dist(y,z)^{\alpha}$. When $z$ is fixed, $\ds\frac{\bar{u}_{\bar{A}}(z)}{\xi(z)}$ is Lipschitz about $\bar{A}$, which is a $C^{\alpha}$ function as $[A]_{C^{\alpha}}$ is small. Therefore, we also have
    \begin{equation*}
        D_{2}\leq C\cdot dist(y^{T},z^{T})^{\alpha}\leq C\cdot dist(y,z)^{\alpha}.
    \end{equation*}
    To estimate $D_{3}$, we first use complex coordinate to write
    \begin{equation*}
        y_{n}+iy_{n+1}=r_{y}e^{i\theta_{y}},\quad z_{n}+iz_{n+1}=r_{z}e^{i\theta_{z}},\quad\theta_{y},\theta_{z}\in(-\pi,\pi).
    \end{equation*}
    It follows that $\bar{u}_{y^{T}}/\xi$ is determined and Lipschitz by the argument angle $\theta_{y},\theta_{z}$. Besides, as $A\in C^{\alpha}$, the Lipschitz norm about $\theta$ is in fact bounded by
    \begin{equation*}
        \Big|\partial_{\theta}\frac{\bar{u}_{y^{T}}(r,\theta)}{\xi(r,\theta)}\Big|\leq C|y^{T}|^{\alpha}
    \end{equation*}
    The angle between $y^{\perp},z^{\perp}$ without crossing the slit $S$, say $\angle{(y^{\perp},z^{\perp})}=|\theta_{y}-\theta_{z}|$, which takes value in $(0,2\pi)$, satisfies
    \begin{equation*}
        \angle{(y^{\perp},z^{\perp})}\leq\left\{\begin{aligned}
            \sin^{-1}\frac{dist(y^{\perp},z^{\perp})}{|y^{\perp}|}&,\ \mbox{if}\ dist(y^{\perp},z^{\perp})<|y^{\perp}|\\
            2\pi&,\ \mbox{if}\ dist(y^{\perp},z^{\perp})\geq|y^{\perp}|
        \end{aligned}\right.\leq C\frac{dist(y^{\perp},z^{\perp})}{|y^{\perp}|}.
    \end{equation*}
    Notice that when $y\in Cone_{r}$, $|y^{\perp}|\geq|y^{T}|$, so
    \begin{align*}
        D_{3}\leq&C\cdot\Big\|\partial_{\theta}\frac{\bar{u}_{y^{T}}(r,\theta)}{\xi(r,\theta)}\Big\|_{L^{\infty}}\cdot|\theta_{y}-\theta_{z}|\leq C|y^{T}|^{\alpha}\cdot\frac{dist(y^{\perp},z^{\perp})}{|y^{\perp}|}\\
        \leq&C\Big(\frac{dist(y^{\perp},z^{\perp})}{|y^{\perp}|}\Big)^{1-\alpha}dist(y^{\perp},z^{\perp})^{\alpha}\leq C3^{1-\alpha}dist(y,z)^{\alpha}.
    \end{align*}
    Here we used $2|y^{\perp}|\geq|z^{\perp}|$ in the last step, since $y,z\in Cone_{r}(0)\setminus Cone_{r/2}(0)$.
\end{proof}
\begin{lemma}
    If $\lambda I\leq A\leq\Lambda I$ and $[A]_{C^{\alpha}}$ is small, then $(\mathcal{F}_{3})\Rightarrow(\mathcal{F})$ up to a shrinking of radius.
\end{lemma}
\begin{proof}
    We write $\ds h(x)=\frac{f(x)}{\bar{u}_{x^{T}}(x)}$ in $B_{R}$, then it suffices to show $h(x)\in C^{\alpha}(B_{R/100})$. Let $x,y\in B_{R/100}$, then as $x\in Cone_{R/10}(x^{T})$, we have that
    \begin{equation*}
        |h(x)-\frac{f(y)}{\bar{u}_{x^{T}}(y)}|=|\frac{f(x)}{\bar{u}_{x^{T}}(x)}-\frac{f(y)}{\bar{u}_{x^{T}}(y)}|\leq C|x-y|^{\alpha}.
    \end{equation*}
    Therefore, as it is obvious that $h\in L^{\infty}(B_{R/100})$, we have
    \begin{align*}
        |h(x)-h(y)|\leq&C|x-y|^{\alpha}+|h(y)|\cdot|\frac{\bar{u}_{y^{T}}(y)}{\bar{u}_{x^{T}}(y)}-1|\\
        \leq&C|x-y|^{\alpha}+C|x^{T}-y^{T}|^{\alpha}.
    \end{align*}
\end{proof}


\documentclass[a4paper,11pt]{article}
\pdfoutput=1 % if your are submitting a pdflatex (i.e. if you have
             % images in pdf, png or jpg format)

%\usepackage[utf8]{inputenc}
%\usepackage{mathrsfs, amssymb, amsmath}  
%\usepackage{comment}
%\usepackage{dcolumn}
%\usepackage{multirow}
%\usepackage{color}
%\usepackage{amsfonts,amssymb,amsmath, txfonts}
%\usepackage{float}

\usepackage{jcappub} % for details on the use of the package, please
                     % see the JCAP-author-manual

\usepackage[T1]{fontenc} % if needed

\hypersetup{ linktoc=all,
    colorlinks=true, linkcolor={blue},  
       citecolor={red}, urlcolor={darkred}
}
\definecolor{Redgreen}{RGB}{153,76,0}
\definecolor{vividviolet}{rgb}{0.62, 0.0, 1.0}
\definecolor{green}{RGB}{11,98,17}
\definecolor{darkgreen}{RGB}{40,150,65}
\definecolor{darkblue}{rgb}{0,0,0.3}
\definecolor{darkred}{rgb}{0.7,0,0}

\def\blue{\textcolor{blue}}
\def\red{\textcolor{red}}
\def\be{\begin{equation}}
\def\ee{\end{equation}}
\def\bea{\begin{eqnarray}}
\def\eea{\end{eqnarray}}


\title{MCMC Marginalisation Bias and $\Lambda$CDM tensions}
%\title{Overcoming bias in MCMC marginalisation to elucidate $\Lambda$CDM tensions}
%\title{Temp}

%%Markov Chain Monte Carlo

%% %simple case: 2 authors, same institution
%% \author{A. Uthor}
%% \author{and A. Nother Author}
%% \affiliation{Institution,\\Address, Country}

% more complex case: 4 authors, 3 institutions, 2 
\author[a]{Eoin \'O Colg\'ain}
\author[b]{Saeed Pourojaghi}
\author[b, c]{M. M. Sheikh-Jabbari}
\author[a]{Darragh Sherwin}

% The "\note" macro will give a warning: "Ignoring empty anchor..."
% you can safely ignore it.

\affiliation[a]{Atlantic Technological University, Ash Lane, Sligo, Ireland}
\affiliation[b]{School of Physics, Institute for Research in Fundamental Sciences (IPM), P.O.Box 19395-5531, Tehran, Iran}
\affiliation[c]{The Abdus Salam ICTP, Strada Costiera 11, I-34014 Trieste, Italy}

% e-mail addresses: one for each author, in the same order as the authors
\emailAdd{eoin.ocolgain@atu.ie}
\emailAdd{pourojaghi@ipm.ir}
\emailAdd{jabbari@theory.ipm.ac.ir}
\emailAdd{darragh.sherwin@research.atu.ie}




\abstract{Probability distributions become non-Gaussian when the flat $\Lambda$CDM model is fitted to redshift binned data in the late Universe. We explain mathematically why this non-Gaussianity arises and confirm that Markov Chain Monte Carlo (MCMC) marginalisation leads to biased inferences in observational Hubble data (OHD). In particular, in high redshift bins we find that $\chi^2$ minima, as identified from both least squares fitting and the MCMC chain, fall outside of the $1 \sigma$ confidence intervals. We resort to profile distributions to correct this bias. Doing so, we observe that $z \gtrsim 1$ cosmic chronometer (CC) data currently prefers a non-evolving (constant) Hubble parameter over a Planck-$\Lambda$CDM cosmology at $\sim 2 \sigma$. We confirm that both mock simulations and profile distributions agree on this significance. Moreover, on the assumption that the Planck-$\Lambda$CDM cosmological model is correct, using profile distributions we confirm  a $> 2 \sigma$ discrepancy with Planck-$\Lambda$CDM in a combination of  CC and baryon acoustic oscillations (BAO) data beyond $ z \sim 1.5$ that was noted earlier through comparison of least square fits of observed and mock data.}



\begin{document}
\maketitle
\flushbottom

\section{Introduction}
\label{sec:intro}
The flat $\Lambda$CDM model is the minimal model that fits Cosmic Microwave Background (CMB) data. Remarkably, CMB data from the Planck satellite \cite{Planck:2018vyg} constrains the $\Lambda$CDM model to sub-percent errors, thereby not only providing the strongest constraints, but also a concrete prediction for cosmological probes in the late Universe. The unmitigated success of the $\Lambda$CDM model is that CMB, Type Ia supernovae (SN) \cite{Riess:1998cb, Perlmutter:1998np} and baryon acoustic oscillations (BAO) \cite{Eisenstein:2005su} agree on a $\Lambda$CDM Universe that is approximately $30 \%$ matter. Thus, one key prediction of the Planck-$\Lambda$CDM model agrees across early and late Universe cosmological probes. Given this non-trivial agreement, any discrepancies that arise elsewhere constitute challenging puzzles. 

Nevertheless, one cannot define any \textit{model} for a dynamical system, especially a complicated system like the Universe, using data from a cosmic snapshot.\footnote{Here, we mean CMB data with an effective redshift $z \sim 1100$.} At best, one has a \textit{prediction} and not a model. In recent years, key predictions of Planck data have been challenged by late Universe determinations of the Hubble constant $H_0$ \cite{Riess:2021jrx, Freedman:2021ahq, Pesce:2020xfe, Blakeslee:2021rqi, Kourkchi:2020iyz} and the $S_8:= \sigma_8 \sqrt{\Omega_m/0.3}$ parameter \cite{HSC:2018mrq, KiDS:2020suj, DES:2021wwk, Boruah:2019icj, Said:2020epb}. Given the diversity of the late Universe probes (see reviews \cite{Perivolaropoulos:2021jda, Abdalla:2022yfr}), it is highly unlikely that any single systematic can be found to explain the discrepancies. That being said, in astrophysics one can never preclude systematics; 3 decades after Phillips' seminal paper \cite{Phillips:1993ng}, we are still debating an ad hoc correction for the mass of the host galaxy in Type Ia SN \cite{NearbySupernovaFactory:2018qkd, Kang:2019azh, Brout:2020msh, Lee:2021txi}. Bearing in mind that Type Ia SN are one of our best understood cosmological probes, one quickly understands that any systematics debate may be endless. 

Thus, it is far more expedient to assume that the $\Lambda$CDM model is breaking down and to look for tell-tale signatures of model breakdown. If signatures cannot be found, one arrives at a contradiction, and revisits the assumption that the model is breaking down. For physicists, \textit{model breakdown comes about when model fitting parameters return discrepant values at different time slices or epochs}. Translated into astronomy, this equates to discrepant cosmological parameters in different redshift ranges. The usual $H_0, S_8$ tensions  may also be viewed in the same light: a discrepancy between high and low redshift inferences/measurements of the parameters \cite{Perivolaropoulos:2021jda, Abdalla:2022yfr}. Nevertheless, early and late Universe observables are typically not the same, so one is confronted with a rich set of potential systematics. 

Within the context of $\Lambda$CDM tensions, it was recently observed that the integration constant from the Friedmann equations, aka the Hubble constant $H_0$, picks up redshift dependence whenever our model assumption - required to close the Friedmann equations - disagrees with the Hubble parameter $H(z)$ extracted from observations \cite{Krishnan:2020vaf, Krishnan:2022fzz}. %\footnote{One is free to speculate about the nature of the missing physics \cite{Liao:2020zko, Montani:2023xpd}.} 
Similarly, $\rho_{m0}=H_0^2\Omega_m$, an integration constant of the matter continuity equation, implies matter density $\Omega_m$ is a mathematically constant quantity. 
These are irrefutable predictions from mathematics, i. e. a prediction that is \textit{robust to systematics}. However, observationally $H_0$ and $\Omega_{m}$ are model fitting parameters and nothing precludes them picking up redshift dependence (except of course if one assumes they do not!), and providing a signature of model breakdown. If this happens in the late Universe within the $\Lambda$CDM model, $H_0$ is correlated with matter density $\Omega_m$, 
while $\Omega_m$ is correlated with $S_8 \propto \sigma_8 \sqrt{\Omega_m}$. Thus, there is at least one simple scenario, namely redshift evolution of cosmological parameters in the late Universe, where ``$H_0$ tension'' and ``$S_8$ tension'' are not independent and simply symptoms of $\Lambda$CDM model breakdown. 

The next relevant question is, where is the evidence for evolving cosmological parameters in the late Universe? Starting with strong lensing time delay \cite{Wong:2019kwg, Millon:2019slk},\footnote{Systematics are explored in \cite{Millon:2019slk} and the descending trend is not an obvious systematic. The lensed system RXJ1131-1231 \cite{Sluse:2003iy}, which partly drives the trend, has recently been re-analysed using spatially resolved stellar kinematics of the host galaxy \cite{Shajib:2023uig}, and the higher $H_0$ value remains robust, admittedly with inflated errors. As TDCOSMO project to analyse 40 lenses, the prospect of a discovery of a descending $H_0$ trend assuming the $\Lambda$CDM model remain strong.} descending trends of $H_0$ with redshift have been reported in Type Ia SN \cite{Dainotti:2021pqg, Colgain:2022nlb, Colgain:2022rxy,  Malekjani:2023dky, Hu:2022kes, Jia:2022ycc} and combinations of data sets \cite{Krishnan:2020obg, Dainotti:2022bzg}. On the other hand, larger values of $\Omega_m$ have been noted in high redshift observables, primarily quasars (QSOs) \cite{Risaliti:2015zla, Risaliti:2018reu, Lusso:2020pdb, Yang:2019vgk, Khadka:2020vlh, Khadka:2020tlm, Khadka:2021xcc, Pourojaghi:2022zrh},\footnote{Just as with Type Ia SN, the systematics of QSOs are being investigated \cite{Zajacek:2023qjm}.} but also Type Ia SN \cite{Colgain:2022nlb, Colgain:2022rxy, Malekjani:2023dky, Pasten:2023rpc} (see also \cite{Wagner:2022etu, Sakr:2023hrl}). Note, as emphasised earlier, if $H_0$ evolves at the background level, correlated fitting parameters are expected to also evolve. Moreover, mock analysis within the $\Lambda$CDM setting reveals that evolution of best fit $(H_0, \Omega_m)$ parameters cannot be precluded, and conversely possesses a finite likelihood, in either observational Hubble data (OHD) \textit{or} angular diameter distance data \textit{or} luminosity distance data \cite{Colgain:2022tql}. We stress that this result \textit{rests on mock analysis}; it represents a purely mathematical statement about the $\Lambda$CDM model that is independent of systematics. 

Separately, at the perturbative level, redshift evolution of $S_8$ or $\sigma_8$ has been reported in galaxy cluster number counts and Lyman-$\alpha$ spectra \cite{Esposito:2022plo}, $f \sigma_8$ constraints from peculiar velocities and redshift space distortions (RSD) 
 \cite{Adil:2023jtu}, comparison between weak \cite{HSC:2018mrq, KiDS:2020suj, DES:2021wwk} and CMB lensing \cite{ACT:2023dou, ACT:2023kun}. What is important here is that these observations appear to restrict the evolution in $S_8$ to the late Universe. In \cite{ACT:2023ipp} the possibility was raised that \textit{``tracers at higher redshift and probing larger scales prefer higher $S_8$''}.\footnote{There are also conflicting observations of high redshift $\sigma_8$ or $S_8$ values that are lower than Planck in the late Universe \cite{Miyatake:2021qjr, Alonso:2023guh}, so either this trend is not universal, or systematics are at play.} Nevertheless, one can argue against evolution with scale on the grounds that cosmic shear \cite{HSC:2018mrq, KiDS:2020suj, DES:2021wwk}, which is sensitive to smaller scales (larger $k$), and peculiar velocity constraints \cite{Boruah:2019icj, Said:2020epb}, which are sensitive to larger scales (smaller $k$), both prefer lower values of $S_8$. Moreover, both galaxy clusters and Lyman-$\alpha$ spectra are expected to probe similar scales.\footnote{We thank Matteo Viel for correspondence on this point.} Thus, if systematics are not impacting results, then redshift evolution is the only point of agreement in the observations \cite{Esposito:2022plo, Adil:2023jtu, HSC:2018mrq, KiDS:2020suj, DES:2021wwk, ACT:2023dou, ACT:2023kun, ACT:2023ipp}. Note also that redshift is more fundamental than scale in FLRW cosmology; one must solve the Friedmann equations in either time or redshift before one contemplates any discussion of scale.  

 The purpose of this letter is to revisit the analysis presented in \cite{Colgain:2022rxy,Colgain:2022tql}, where the evidence for evolution was quantified on the basis of mock simulations and not Markov Chain Monte Carlo (MCMC), the technique most familiar in cosmology. The fundamental problem is that once one bins low redshift data and studies evolution of cosmological parameters with bin redshift, one quickly encounters projection effects in MCMC analyses. These effects are not just the preserve of exotic models \cite{Herold:2021ksg, Gomez-Valent:2022hkb, Meiers:2023gft}, such as Early Dark Energy (EDE) \cite{Poulin:2018cxd, Niedermann:2019olb}, and happen in the simplest model when one bins data. The most striking demonstration of the resulting bias is that the peaks of MCMC posteriors no longer coincide with the minimum of the likelihood (see \cite{Gomez-Valent:2022hkb}). Ultimately, this bias is expected  because one is working in a regime of the $\Lambda$CDM model with non-Gaussian probability distributions   \cite{Colgain:2022tql}  (see also \cite{Colgain:2022rxy}).

 The structure of this paper is as follows. In section \ref{sec:MCMC_bias} we confirm the bias in MCMC marginalisation. In section \ref{sec:PD} we introduce profile distributions (PDs) \cite{Gomez-Valent:2022hkb} as a means of addressing the bias and confirm that the statistical significance of discrepancies from mock simulations agree well with PD analysis. In section \ref{sec:tension}, we revisit and confirm the high redshift OHD tensions reported in \cite{Colgain:2022rxy}. We end in section \ref{sec:discussion} with concluding remarks. 
 %A short appendix is also added on Fisher matrix for $\Lambda$CDM mdoel. 

\section{A bias in MCMC marginalisation}
\label{sec:MCMC_bias}
In this section we illustrate a bias in MCMC marginalisation that arises in the (flat) $\Lambda$CDM model when data is binned by redshift. This bias can be traced to a regime of the $\Lambda$CDM model with non-Gaussian distributions and is independent of systematics  \cite{Colgain:2022rxy, Colgain:2022tql}. 

\subsection{Mathematical Foundations}
\label{sec:math}
Consider an exercise where one bins OHD and confronts it to the $\Lambda$CDM Hubble Parameter $H(z)$ in the late Universe, a setting where the radiation sector can be safely decoupled. In high redshift bins ($z \gg 0$) in the matter-dominated regime, the Hubble parameter becomes insensitive to the dark energy (DE) sector: 
\be
\label{eq:lcdm}
H(z) = H_0 \sqrt{1-\Omega_m + \Omega_m (1+z)^3} \xrightarrow[z \gg 0]{} H_0 \sqrt{\Omega_m} (1+z)^{\frac{3}{2}}.  
\ee
More concretely, taking $z \rightarrow \infty$ we see that data can only constrain the combination $\rho_{m0}=H_0^2{\Omega_m}$. For \textit{hypothetical} data in a redshift bin with effective redshift $z = \infty$, this means that one can only constrain the combination $\Omega_m h^2$ ($h:= H_0/100)$, but $H_0$ and $\Omega_m$ remain unconstrained. Alternatively put, for any given $\Omega_m h^2$ constraint, there is an infinite number of corresponding $(H_0, \Omega_m)$ pairs. Translated into a probability density function (PDF), this is simply the statement that in a very high redshift bin at $z = \infty$, one expects uniform or flat distributions for $H_0$ and $\Omega_m$ with the model (\ref{eq:lcdm}).  

Of course, observed data resides at finite $z$ and not $z = \infty$. As a result, one does not encounter \textit{exactly} flat PDFs in $H_0$ and $\Omega_m$ at high redshift, but \textit{almost} flat PDFs. More important to us is the observation that these PDFs must flatten in a non-Gaussian manner. To appreciate this fact, we observe that high redshift OHD only constrains $\Omega_m h^2$ well.\footnote{Note that observables like SN or QSO that measure $D_L(z)=c (1+z)\int_0^z \textrm{d} z'/H(z')$ are mainly sensitive to the low redshift part of $H(z)$, i. e. the combination $H_0^2 (1-\Omega_m)$, and in this sense they are complementary to the OHD data which is more sensitive to high redshift part of $H(z)$, $H_0^2\Omega_m$. The complementarity can be demonstrated by combining $H(z)$ and $D_{L}(z)$ constraints and checking that one recovers mock data input parameters in all redshift bins \cite{Colgain:2022tql}. } For this reason, best fit parameters are constrained to a $\Omega_m h^2 = \textrm{constant}$ curve in the $(H_0, \Omega_m)$-plane. The almost flat $H_0$ and $\Omega_m$ PDFs can only arise if this curve stretches in the $(H_0, \Omega_m)$-plane. As a result of this stretching, one ends up with a relatively uniform distribution on a curve. At the extremes of the curve, one finds a distribution of large $H_0$ values, which do not differ greatly in $\Omega_m$, and they get projected to a peak at small values on the $\Omega_m$ axis. Conversely, at the other end of the curve, one finds a distribution of small $\Omega_m$ values, which do not differ greatly in $H_0$, and they get projected onto a peak at large values on the $H_0$ axis.  This is a ``projection effect'' in common cosmology parlance.  It is driven by the irrelevance of the DE sector at high redshift and the constraint $\Omega_m h^2 = \textrm{constant}$ from the $\Lambda$CDM model (\ref{eq:lcdm}). Together these features distort the distribution away from a Gaussian configuration. 

Thus, simply by binning and fitting OHD to the $\Lambda$CDM model one enters a non-Gaussian regime as the effective redshift of the bin increases. This effect, which is expected from the purely mathematical arguments above, has been confirmed in mock data \cite{Colgain:2022rxy, Colgain:2022tql}, and in line with expectations, we demonstrate that it impacts MCMC inferences with observed data in the next subsection.  

% Figure environment removed

\subsection{Cosmic Chronometer (CC) Data}
\label{sec:CCbias}
Here we work with OHD from the cosmic chronometer (CC) program \cite{Jimenez:2001gg}. Concretely, we work with 34 $H(z)$ constraints spanning the redshift range $0.07 \leq z \leq 1.965$ \cite{Stern:2009ep, Moresco:2012jh, Zhang:2012mp, Moresco:2016mzx, Ratsimbazafy:2017vga, Borghi:2021rft, Jiao:2022aep, Tomasetti:2023kek}. We illustrate the data in Fig.~\ref{fig:CC}, where it is consistent with Fig. 9 of \cite{Tomasetti:2023kek} {modulo the fact that we have an additional data point at $z = 0.8$, which is not independent. See Table 1.1 of \cite{Moresco:2023zys}. While CC data may eventually be good enough to arbitrate on Hubble tension \cite{Moresco:2023zys}, the data is not good enough on its own to do cosmology. To put this comment in context, we observe that the errors in Fig.~\ref{fig:CC} do not include systematic errors (see \cite{Moresco:2020fbm} for an account of the systematics). As a result the constraints we get on cosmological parameters will be underestimated. Thus, from our perspective the data in Fig.~\ref{fig:CC} is simply some representative cosmological data in the OHD class.}

\paragraph{Methodology:} We impose a low redshift cut-off on the OHD $z_{\textrm{min}}$, removing all data points with redshifts $z_i < z_{\textrm{min}}$, and then extremising the $\chi^2$ likelihood, 
\be
\label{eq:chi2}
\chi^2 = Q^{T} \cdot C^{-1} \cdot Q, 
\ee
where $C$ is the covariance matrix, which is simply the square of the $H_i$ errors on the diagonal, and $Q$ is the vector, 
\be
\label{eq:Q}
Q_i = H_i - H_{\textrm{model}}(z_i), 
\ee
where $H_i:=H(z_i)$ denotes OHD and $H_{\textrm{model}}(z)$ is the model (\ref{eq:lcdm}) without the high redshift limit. The best fit $(H_0, \Omega_m)$ parameters correspond to the minumum of the $\chi^2$, while on the assumption of Gaussian errors, we estimate the errors from a Fisher matrix (appendix \ref{sec:fisher}). In parallel, we perform MCMC marginalisation through \textit{emcee} \cite{Foreman-Mackey:2012any}. More concretely, subject to the priors $H_0 \in [0, 200 ]$ and $\Omega_m \in [ 0, 1]$, the latter restricting us to a physical regime, we record $16^{\textrm{th}}$, $50^{\textrm{th}}$ and $84^{\textrm{th}}$ percentiles for MCMC posteriors, as is common practice with Gaussian distributions. Thus, both techniques are tailored to Gaussian posteriors, yet non-Gaussianities will be evident in MCMC posteriors. By comparing the output from these two techniques in Table \ref{tab:LCDM_CC} for different values of $z_{\textrm{min}}$ we observe that error estimates from Fisher matrix and MCMC quickly disagree as $z_{\textrm{min}}$ increases. 

From Table \ref{tab:LCDM_CC}, we see that MCMC inferences lead to non-Gaussian $1 \sigma$ confidence intervals, where in line with the expectations from \cite{Colgain:2022tql}, $H_0$ errors are larger for smaller values, and $\Omega_m$ errors are larger for larger values, respectively. This is expected if the $H_0$ and $\Omega_m$ posteriors are peaked at larger and smaller values, respectively, in line with our earlier mathematical argument. Only for the full data set with $z_{\textrm{min}} = 0$  do we find reasonable agreement between the Fisher matrix and MCMC $1 \sigma$ confidence intervals. As can be seen from the lopsided MCMC confidence intervals, the non-Gaussianity becomes more pronounced with increasing $z_{\textrm{min}}$. Interestingly, beyond $z_{\textrm{min}} = 1$, the minimum of the $\chi^2$ falls outside of the MCMC $1 \sigma$ confidence intervals. Nevertheless, by evaluating the MCMC chains on the $\chi^2$ likelihood (\ref{eq:chi2}), we confirm that the parameters corresponding to the minimum $\chi^2$ value are tracking the best fit. Note, the peak of the MCMC posterior is no longer a measure of goodness of fit and inferences have become biased in a regime of model parameter space where distributions are expected to be inherently non-Gaussian. Our analysis here underscores potential problems with a blind MCMC analysis with the traditional $16^{\textrm{th}}$, $50^{\textrm{th}}$ and $84^{\textrm{th}}$ percentiles.       



\begin{table}[htb]
    \centering
    \begin{tabular}{c|c|c|c|c|c}
    \rule{0pt}{3ex} $z_{\textrm{min}}$ & \# CC & \multicolumn{2}{c}{Fisher Matrix}  & \multicolumn{2}{|c}{MCMC} \\
    \hline
    \rule{0pt}{3ex} & & $H_0$ (km/s/Mpc) & $\Omega_m$ & $H_0$ (km/s/Mpc) & $\Omega_m$ \\
    \hline
    \rule{0pt}{3ex} $0$ & $34$ & $68.14 \pm 3.07$ & $0.320 \pm 0.059$ & $67.76^{+3.03}_{-3.09}$  ($68.12$) & $0.328^{+0.065}_{-0.055}$ ($0.321$) \\
    \hline 
    \rule{0pt}{3ex} $0.2$ & $27$ & $65.03 \pm 6.65$ & $0.368 \pm 0.118$ & $63.05^{+6.64}_{-7.23}$ ($64.98$) & $0.405^{+0.170}_{-0.111}$ ($0.369$) \\
    \hline 
    \rule{0pt}{3ex} $0.4$ & $22$ & $62.42 \pm 8.38$ & $0.411 \pm 0.161$ & $59.54^{+8.30}_{-8.22}$ ($62.39$) & $0.470^{+0.229}_{-0.151}$ ($0.411$)\\
    \hline 
    \rule{0pt}{3ex} $0.6$ & $15$ & $59.83 \pm 17.21$ & $0.454 \pm 0.338$ & $56.45^{+13.16}_{-9.33}$ ($59.86$) & $0.526^{+0.288}_{-0.225}$ ($0.453$) \\
    \hline 
    \rule{0pt}{3ex} $0.7$ & $14$ & $79.11 \pm 19.40$ & $0.222 \pm 0.162$ & $67.59^{+19.19}_{-16.57}$ ($79.18$) & $0.344^{+0.344}_{-0.178}$ ($0.222$) \\
    \hline 
    \rule{0pt}{3ex} $0.8$ & $11$ & $103.97 \pm 24.94$ & $0.097 \pm 0.088$ & $82.43^{+28.33}_{-27.03}$ ($104.02$) & $0.206^{+0.357}_{-0.131}$ ($0.096$) \\
    \hline 
    \rule{0pt}{3ex} $1$ & $8$ & $150.37 \pm 31.21$ & $0.010 \pm 0.035$ & $108.92^{+33.94}_{-44.47}$ ($150.38$) & $0.087^{+0.304}_{-0.068}$ ($0.010$) \\
    \hline 
    \rule{0pt}{3ex} $1.2$ & $7$ & $154.35 \pm 42.95$ & $0.006 \pm 0.042$ & $83.07^{+48.52}_{-32.19}$ ($154.47$) & $0.194^{+0.439}_{-0.159}$ ($0.006$) \\
    \hline 
    \rule{0pt}{3ex} $1.4$ & $4$ & $125.41 \pm 79.55$ & $0.039 \pm 0.132$ & $65.32^{+44.88}_{-20.30}$ ($125.44$) & $0.320^{+0.423}_{-0.250}$ ($0.039$) \\
    \hline 
    \rule{0pt}{3ex} $1.5$ & $3$ & $36.12 \pm 72.69$ & $1.000 \pm 4.269$ & $55.19^{+34.64}_{-14.73}$ ($36.16$) & $0.393^{+0.387}_{-0.283}$ ($0.999$)
    \end{tabular}
    \caption{Comparison between Fisher matrix and MCMC analysis for CC data with a low redshift cut-off $z_{\textrm{min}}$. We record the number of data points, the extremum of the $\chi^2$ and $1 \sigma$ confidence interval estimated from the Fisher matrix,  $16^{\textrm{th}}$, $50^{\textrm{th}}$ and $84^{\textrm{th}}$ percentiles from MCMC posteriors corresponding to $1 \sigma$ confidence intervals, and the minimum $\chi^2$ from the MCMC chain in brackets. MCMC marginalisation exhibits non-Gaussian $1 \sigma$ confidence intervals, and for $z_{\textrm{min}} > 1$, the minimum value of the $\chi^2$ from the MCMC chain falls outside of this interval. The latter tracks the best fit up to small numbers in line with expectations. }
    \label{tab:LCDM_CC}
\end{table}

\subsection{Features in CC Data}
\label{sec:features}
Once one accounts for biases, it is clear from Table \ref{tab:LCDM_CC} that there are trends in CC data when it is binned. Starting from $z_{\textrm{min}} = 0$ through to $z_{\textrm{min}} = 0.6$ we see a decreasing trend in best fit values of $H_0$ (also central $H_0$ values from MCMC), which is compensated by a increasing trend in $\Omega_m$ best fit values. From Fig.~\ref{fig:CC} it is difficult to visibly discern any trend from the raw data. From $z_{\textrm{min}} = 0.7$ through to $z_{\textrm{min}} = 1.4$, there is in contrast a preference for larger $H_0$ and smaller $\Omega_m$ values. This trend is evident from the raw data, where at higher redshifts one sees large scatter and large fractional errors in the data. For $z_{\textrm{min}} = 1$, it is clear that the best fit line in magenta corresponding to $(H_0, \Omega_m) = (150.4, 0.01)$ (Table \ref{tab:LCDM_CC}) is closer to horizontal line than the Planck-$\Lambda$CDM cosmology in red. To be more explicit, for $z_{\textrm{min}} = 0$, $\rho_{m0}:=H_0^2\Omega_m\simeq 1500$ which is close to the Planck value, whereas for $z_{\textrm{min}} = 1$, $\rho_{m0}\simeq 225$. The sharp drop in $\rho_{m0}$ means the magenta line should be almost horizontal. For $z_{\textrm{min}} = 1.5$, we switch to an opposite regime of parameter space with unexpectedly low and high values of $H_0$ and $\Omega_m$, respectively, a trend which is evident in the data, but there are only three data points. Despite, the small number of data points, the tendency for smaller $H_0$ and larger $\Omega_m$ inferences within $\Lambda$CDM cosmology at high redshifts has been documented across three independent observables \cite{Colgain:2022rxy}. We will come back to this claim in section \ref{sec:tension}. Finally, it is worth noting that for large $z_{\textrm{min}}$ and samples with few data points, one expects broad MCMC posteriors. These posteriors are severely impacted by the prior on $\Omega_m$, as is evident from Table \ref{tab:LCDM_CC}. 

For the moment we leave physical speculations to the discussion and return to the trend in CC data above $z=1$ favouring less evolution in the Hubble parameter than the Planck-$\Lambda$CDM model. We would like to quantify the significance of this trend, but since we are working in a non-Gaussian regime of the model, we can expect both Fisher matrix and MCMC to give biased results. In Fig.~\ref{fig:CCsplit1} we show MCMC posteriors for $z>1$ CC data in blue alongside posteriors for low redshift ($z < 1$) CC data, which is simply added to aid comparison and also highlight the Gaussianity of the low redshift posteriors. One notes that the peaks of the $z > 1$ distributions are a little displaced from to the values minimising the $\chi^2$. However, the emergence of the lower peak in the $H_0$ posterior at $H_0 \sim 50$ km/s/Mpc has the hallmarks of a projection effect. To appreciate this, note that the configurations in the blue curve in the top left corner of the 2D posterior are projected onto the lower $H_0$ peak. Moreover, if one shifts the $H_0$ peak from $H_0 \sim 150$ to $H_0 \sim 50$ km/s/Mpc while maintaining $\Omega_m \sim 0$, this shifts the magenta curve in Fig. \ref{fig:CC} outside of all the data points, so the lower $H_0$ peak is a phantom artefact unrelated to the goodness of fit. We also observe a shift in the higher $H_0$ peak away from the minimum of the $\chi^2$.

Ignoring these features, one could attempt to interpret the overlap in the 2D posteriors in Fig. \ref{fig:CCsplit1}. Doing so, one may conclude that low and high redshift CC data are consistent within $1 \sigma$. However, since Hubble tension is a 1D problem (local $H_0$ determinations are insensitive to other parameters), to compare with locally observed values of $H_0$ one needs to project onto the $H_0$ axis. Alternatively put, Hubble tension is a problem in 1D posteriors. Projecting onto the $H_0$ axis by determining $16^{\textrm{th}}$, $50^{\textrm{th}}$ and $84^{\textrm{th}}$ percentiles, one sees from Table \ref{tab:LCDM_CC} that the $z_{\textrm{min}} = 1$ MCMC confidence interval encloses the $z_{\textrm{min}} = 0$ central values within $1 \sigma$,\footnote{Note, removing the eight high redshift data points from the $z_{\textrm{min}} = 0$ sample will not shift the central values much.} but not the point in parameter space that best fits the data!


% Figure environment removed



Evidently, given the non-Gaussian posteriors, care is required when interpreting the significance of the trend towards a non-evolving (horizontal) $H(z)$ at higher redshifts in Fig.~\ref{fig:CC}. We cannot use the errors from the Fisher matrix as we are clearly in a non-Gaussian regime, whereas MCMC inferences are impacted by projection effects to the extent that the minimum of the $\chi^2$ (confirmed from the MCMC chain) falls outside of the $1 \sigma$ confidence interval. For this reason, we resort to mock simulations. While this may seem a little redundant if we are going to employ profile distributions in section \ref{sec:PD}, there is motivation for this exercise. In \cite{Colgain:2022rxy} the significance of a descending $H_0$/increasing $\Omega_m$ trend with effective redshift in OHD, Type Ia SN and QSOs was estimated to be a $\sim 3 \sigma$ effect on the basis of combining $\sim 2 \sigma$ effects in each of the \textit{independent} data sets using Fisher's method. Here, working with the same data throughout, we can directly compare the significance of a discrepancy estimated through mock simulations from the significance of a discrepancy estimated through profile distributions. In particular, we will address the question: how significant is a constant $H(z)$ with $z_{\textrm{min}}=1$ (8 data points) against the Planck consistent cosmology favoured by the full data set ($z_{\textrm{min}}=0$ entry in Table \ref{tab:LCDM_CC})? Note, the significance will be overestimated due to missing systematic uncertainties (see \cite{Moresco:2020fbm}), but we can still make comparison between the two techniques.

\paragraph{{Mock simulations:}} To address this question using mock simulations, we begin with the MCMC chains for the full sample. For each entry in the MCMC chain (approximately 15,000 entries in total), we generate a new realisation of the 8 high redshift data points $(z > 1)$ that are by construction statistically consistent with both the best fits from the full sample and also the Planck-$\Lambda$CDM values \cite{Planck:2018vyg}. More concretely, for each $(H_0, \Omega_m)$ entry in our MCMC chain, we displace the data points to the corresponding $\Lambda$CDM Hubble parameter before generating new data points in a normal distribution where the errors serve as standard deviations. We then fit back the $\Lambda$CDM model to each realisation of the mock data and record the best fit $(H_0, \Omega_m)$ values, which give us a distribution of expected $(H_0, \Omega_m)$ best fits. The distributions are presented in Fig.~\ref{fig:CCsims} alongside the best fits from observed data. Throughout, we assume canonical values $(H_0, \Omega_m) = (70, 0.3)$ for the initial guess of the fitting algorithm. Best fits can saturate our bounds, i. e. $\Omega_m = 0$ and $\Omega_m = 1$, and this leads to an unsightly pile up of best fits at $\Omega_m = 0$ and $\Omega_m = 1$ in Fig.~\ref{fig:CCsims} \cite{Colgain:2022rxy}. It is important to retain all the configurations, otherwise one is not accounting for the probability that a best fit falls outside our priors. As a consistency check, we see that the median or 50$^{\textrm{th}}$ percentile, $(H_0, \Omega_m) = (68.32, 0.321)$ agrees well with the mock input parameters, thereby demonstrating that there are an equal number of best fits with values above and below the injected parameters in the mocks. We find that probability of a more extreme (larger) $H_0$ value to be $p = 0.022$, while the probability of a more extreme (smaller) $\Omega_m$ value to be $p = 0.035$, respectively. Converted into a Gaussian statistic, these correspond to $2 \sigma$ and $1.8 \sigma$, respectively, for a one-sided normal distribution. Thus, on the basis of mock simulations, we estimate the non-evolving constant $H(z)$ with $z_{\textrm{min}} = 1$ as a $\sim 2 \sigma$ effect. In the next section we will recover this number more or less from the profile distribution analysis. 

% Figure environment removed


\section{Profile Distributions}
\label{sec:PD}
Having explained the mathematics behind the bias, which gives rise to a projection effect, in subsection \ref{sec:math}, and having illustrated how it affects MCMC inferences in subsection \ref{sec:CCbias} - the minimum of the $\chi^2$ may fall outside of $1 \sigma$ confidence intervals - we turn to profile distributions (PDs) \cite{Gomez-Valent:2022hkb}, an extension of the profile likelihood, e. g. \cite{Trotta:2017wnx}, in order to address the bias. Consider two sets of parameters $\theta_1$ and $\theta_2$ and a normalised distribution $\mathcal{P}(\theta_1, \theta_2)$. The basic idea \cite{Gomez-Valent:2022hkb} is to study the ratio 
\be
\label{R}
R(\theta_1) = \frac{\tilde{\mathcal{P}}(\theta_1)}{\max_{\theta_1} \tilde{\mathcal{P}}(\theta_1) } = \frac{\tilde{\mathcal{P}}(\theta_1)}{\max_{\theta_1, \theta_2} \mathcal{P}(\theta_1, \theta_2) },  
\ee
where $\tilde{\mathcal{P}}(\theta_1)$ is the PD, defined to be the maximum of $\mathcal{P}$ for each $\theta_1$ along the $\theta_2$ direction: 
\be
\label{PD}
\tilde{\mathcal{P}} (\theta_1) = \max_{\theta_2} \mathcal{P}(\theta_1, \theta_2). 
\ee
The advantage of this approach is that $R(\theta_1)$ can serve as a probability distribution function (up to an overall normalization), however we do not need to perform any integration, so $R(\theta_1)$ is not prone to volume or projection effects. At this juncture, given the simplicity of our setup with only two parameters $(H_0, \Omega_m)$, we can be more explicit. Consider the probability distribution,   
\be
\mathcal{P}(\theta_1, \theta_2) = \exp \left( - \frac{1}{2} \chi^2(\theta_1, \theta_2) \right), 
\ee
where $\theta_i \in \{H_0, \Omega_m \}$  and $\chi^2(H_0, \Omega_m)$ is our earlier likelihood (\ref{eq:chi2}). The maximum value of $\mathcal{P}$ occurs for the minimum value of $\chi^2$ from the MCMC chain, $\mathcal{P}_{\textrm{max}} = e^{-\frac{1}{2} \chi^2_{\textrm{min}}}$. In this concrete setting, the PD becomes 
\be
\tilde{\mathcal{P}}(\theta_1) = e^{-\frac{1}{2} \chi^2_{\textrm{min}}(\theta_1)}, 
\ee
where $\chi^2_{\textrm{min}}(\theta_1)$ denotes the minimum value of the $\chi^2$ along the $\theta_2$ direction for a fixed $\theta_1$ value. It should not be confused with the overall minimum $\chi^2_{\textrm{min}}$, which can be extracted easily from the MCMC chain. In practice, one can also determine $\chi^2_{\textrm{min}}(\theta_1)$ from the MCMC chain by breaking the $\theta_1$ direction up into bins and finding the minimum of the $\chi^2$ for each bin. Having done so, we are in a position to define a PDF \cite{Gomez-Valent:2022hkb}: 
\be
\label{eq:w}
w(\theta_1) = \frac{e^{-\frac{1}{2} \chi^2_{\textrm{min}}(\theta_1)}}{\int e^{-\frac{1}{2} \chi^2_{\textrm{min}}(\theta_1)} \, \textrm{d} \theta_1} = \frac{R(\theta_1)}{\int R(\theta_1) \, \textrm{d} \theta_1}, 
\ee
where in the second equality we have divided top and bottom by $\mathcal{P}_{\textrm{max}} = e^{-\frac{1}{2} \chi^2_{\textrm{min}}}$. As a result, $R(\theta_1) = e^{-\frac{1}{2} \Delta \chi_{\textrm{min}}^2}$, where $\Delta \chi^2_{\textrm{min}} := \chi_{\textrm{min}}^2(\theta_1) - \chi^2_{\textrm{min}}$, so that $R(\theta_1)$ peaks at $R(\theta_1) = 1$. Note that $\int_{-\infty}^{+\infty} w(\theta_1) \, \textrm{d} \theta_1 = 1$ by construction, so $w(\theta_1)$ describes a properly normalised PDF. Thus we can identify the $1 \sigma, 2 \sigma$ and $3 \sigma$ confidence intervals corresponding to the 68\%, 95\% and 99.7\% confidence level, respectively, by simply identifying $\theta_1^{(1)}$ and $\theta_1^{(2)}$ such that \cite{Gomez-Valent:2022hkb}
\be
\label{eq:wsigma}
\int_{\theta_1^{(1)}}^{\theta_1^{(2)}} w(\theta_1) \, \textrm{d} \theta_1 = I, \quad w(\theta_1) = w(\theta_2), \quad I \in \{0.68, 0.95, 0.997\}. 
\ee
We will outline how these conditions can most easily be satisfied when we turn to explicit examples. 

Our first port of call is making sure that the PD methodology gives sensible results. This can be best judged by applying it to the CC data with $z_{\textrm{min}} = 0$, since this is where we expect a distribution closest to a Gaussian distribution, as is evident from the agreement between Fisher matrix and MCMC results in Table \ref{tab:LCDM_CC}. In particular, we will be interested in a comparison between $1 \sigma$ confidence intervals to make sure that (\ref{eq:wsigma}) is not underestimating or overestimating the $1 \sigma$ confidence interval. 

% Figure environment removed

We start by running a long MCMC chain (100,000 iterations) in order to ensure bins are well populated, and begin by analysing $\theta_1 = H_0$ with $\theta_2 = \Omega_m$. From the MCMC chain we identify the smallest and largest value of $H_0$ in the chain and break up this range into approximately 200 uniform bins, which we label using the $H_0$ value at the centre of the bin. We omit any empty bins. One can increase the number of bins by simply running a longer MCMC chain. In each $H_0$ bin we identify the minimum value of the $\chi^2$, $\chi^2_{\textrm{min}}(H_0)$, and calculate $R(H_0)$. One then repeats the steps for $\Omega_m$. In Fig.~\ref{fig:R_zmin0} we plot $R(H_0)$ against $H_0$ and $R(\Omega_m)$ against $\Omega_m$, noting that the distributions are Gaussian to first approximation. 

Since the distributions from the MCMC chain are sparse in the tails, empty bins are evident in Fig.~\ref{fig:R_zmin0}. Nevertheless, with 200 bins, modulo any empty bins, we have sufficient density of points to calculate the total area under the $R(H_0)$ and $R(\Omega_m)$ curve using Simpson's rule. Any concern about precision can simply be mitigated by running a longer MCMC chain and increasing the number of bins. 
One may directly use $R(H_0)\leq 1$ and $R(\Omega_m)\leq 1$   to find $68$, $95$ and $99.7$ percentiles,  respectively corresponding to $1 \sigma, 2 \sigma$ and $3 \sigma$ confidence intervals. Consider $F_\kappa:= \int_{R\geq \kappa} R (\theta_1) \, \textrm{d} \theta_1$, where $\kappa \leq 1$. Observe that $F_{\kappa=1}=0$ and $F_{\kappa=0}:=F_0=\int R(\theta_1) \textrm{d} \, \theta_1$. Then move $\kappa$ through and terminate the process when $F_\kappa/F_0$ is equal to $0.68$, $0.95$ and $0.997$. This gives the corresponding range for $\theta_1$ that defines the confidence interval.
Working with the precision afforded to us by approximately 200 bins, the $H_0$ and $\Omega_m$ $1 \sigma$ confidence intervals are presented in Fig.~\ref{fig:R_zmin0} and the first entry in Table \ref{tab:LCDM_CC_PD}. The outcome is in excellent agreement with both Fisher matrix and MCMC analysis. In particular, a mild non-Gaussianity in $\Omega_m$ is evident in both Fig.~\ref{fig:R_zmin0} and the errors. 
Thus, we have succeeded in recovering results in the (almost) Gaussian regime that are consistent with Fisher matrix and MCMC analysis and this provides an important check of the methodology.  

% Figure environment removed

We now apply the same PD methodology to the non-Gaussian regime where MCMC marginalisation leads to biased results. To be concrete, we focus on the eight data points in the range $1 < z < 2$ where a non-evolving $H(z)$ trend is evident in the raw data in Fig.~\ref{fig:CC}. Our goal here is to quantify the disagreement with the full data set, where one infers $H_0 \sim 68$ km/s/Mpc and $\Omega_m \sim 0.32$. A similar exercise was performed in subsection \ref{sec:features} with mock simulations and the disagreement was estimated to be approximately $2 \sigma$. Repeating the steps outlined above for the CC data with $z_{\textrm{min}} = 1$ we find the distributions in Fig.~\ref{fig:R_zmin1}. The first observation is that the distributions are non-Gaussian, but a comparison to the MCMC posteriors from the same data in blue in Fig.~\ref{fig:CCsplit1} reveals that there is no secondary $H_0$ peak at $H_0 \sim 50$ km/s/Mpc. Thus, we confirm the secondary peak to be a projection effect. That being said, the primary $H_0$ peak from Fig.~\ref{fig:CCsplit1} has shifted to the dashed line corresponding to the minimum of the $\chi^2$, since the peak of the distribution and $\chi^2$ minimum agree by construction. Comparing the blue $\Omega_m$ distribution from Fig.~\ref{fig:CCsplit1} to the $R(\Omega_m)$ distribution in Fig.~\ref{fig:R_zmin1}, we see that the peak is close to $\Omega_m = 0$ and that the tails continue to $\Omega_m = 1$. In both plots we see that there is a non-zero probability of inferring $\Omega_m = 1$. In some sense, this is not so surprising, the reason being that one is free to adopt generous priors for $H_0$, so that probability of large and small $H_0$ values is zero, but the priors on $\Omega_m$ in the flat $\Lambda$CDM model are restricted. For this reason, as a distribution spreads one invariably finds that distributions are impacted by the $\Omega_m$ priors.\footnote{Note, this is a problem for the flat $\Lambda$CDM model. In particular, one may easily find that the peak of the $\Omega_m$ distribution is larger than $\Omega_m=1$, as is the case with Hubble Space Telescope SN with redshifts $z > 1$ in the Pantheon+ sample \cite{Malekjani:2023dky}.}

It is evident from Fig.~\ref{fig:R_zmin1} that any tension that exists is confined to the $H_0$ parameter. Moreover, since there may be only one binned value of $\Omega_m$ below the $R(\Omega_m)$ peak, at the precision afforded to us by 200 bins, the $R(\Omega_m)$ distribution in Fig.~\ref{fig:R_zmin1} is essentially one-sided and the $1 \sigma$ confidence interval stretches beyond $\Omega_m \sim 0.32$, so there is no disagreement in the $\Omega_m$ parameter. Nevertheless, in the $H_0$ parameter we see that $H_0 \sim 68$ km/s/Mpc, the value favoured by the full data set is just under $2 \sigma$ removed from the peak. The main point here is that, as is obvious from the raw data, current CC data with $z > 1$ has a preference for a non-evolving Hubble parameter $H(z)$ with a large constant $H_0 \sim 150$ km/s/Mpc. The disagreement is just under $2 \sigma$, more accurately $1.9 \sigma$ from $R(H_0)$, and only $0.9 \sigma$ from $R(\Omega_m)$. Although this may not be a serious discrepancy, essentially because of the poor data quality (8 data points), this disagreement supports the $\sim 2 \sigma$ discrepancy seen in the mock simulations. It should be borne in mind that systematic uncertainties have been omitted and these will reduce this discrepancy once properly propagated. Given the agreement between the PD and mock simulation analysis, there is nothing to suggest that the three independent trends highlighted in \cite{Colgain:2022rxy} across OHD, Type Ia SN and QSOs are not \textit{bona fide} disagreements and that redshift evolution is present in the sample. The task remains to combine them at the level of a $\chi^2$ likelihood instead of combining them using Fisher's method on the basis that they are independent probabilities. We leave this exercise for a forthcoming paper, but revisit the tension in OHD data in the following section.  %\ref{sec:tension}. 
For completeness, in Table \ref{tab:LCDM_CC_PD} we perform a reanalysis of CC data subsets with the PD approach and record the $1 \sigma$ intervals.  

\begin{table}[htb]
    \centering
    \begin{tabular}{c|c|c|c}
    \rule{0pt}{3ex} $z_{\textrm{min}}$ & \# CC & \multicolumn{2}{c}{PD}  \\
    \hline
    \rule{0pt}{3ex} & & $H_0$ (km/s/Mpc) & $\Omega_m$ \\
    \hline
    \rule{0pt}{3ex} $0$ & $34$ & $68.15^{+3.04}_{-3.11}$ & $0.320^{+0.065}_{-0.055}$ \\
    \hline 
    \rule{0pt}{3ex} $0.2$ & $27$ & $65.03^{+6.52}_{-7.03}$ & $0.368^{+0.167}_{-0.110}$ \\
    \hline 
    \rule{0pt}{3ex} $0.4$ & $22$ & $62.42^{+7.78}_{-8.74}$ & $0.411^{+0.236}_{-0.113}$ \\
    \hline
    \rule{0pt}{3ex} $0.6$ & $15$ & $59.75^{+11.73}_{-13.97}$ & $0.455^{+0.355}_{-0.160}$ \\
    \hline
    \rule{0pt}{3ex} $0.7$ & $14$ & $79.10^{+16.42}_{-20.56}$ & $0.222^{+0.386}_{-0.117}$ \\
    \hline
    \rule{0pt}{3ex} $0.8$ & $11$ & $103.94^{+22.88}_{-28.54}$ & $0.097^{+0.378}_{-0.074}$ \\
    \hline
    \rule{0pt}{3ex} $1$ & $8$ & $150.35^{+17.12}_{-35.95}$ & $ < 0.339$ \\
    \hline
    \rule{0pt}{3ex} $1.2$ & $7$ & $154.26^{+14.88}_{-54.82}$ & $ < 0.570$ \\
    \hline
    \rule{0pt}{3ex} $1.4$ & $4$ & $124.81^{+35.38}_{-52.60}$ & $ < 0.661$ \\
    \hline
    \rule{0pt}{3ex} $1.5$ & $3$ & $36.11^{+72.87}_{-2.43}$ & $ > 0.354$
    \end{tabular}
    \caption{Same as Table \ref{tab:LCDM_CC} but with the PD methodology in lieu of Fisher matrix and MCMC analysis. The high redshift $R(\Omega_m)$ distributions are typically one-sided, so one encounters $1 \sigma$ upper and lower bounds.}
    \label{tab:LCDM_CC_PD}
\end{table}




\section{A tension with Planck}
\label{sec:tension}
A $2 \sigma$ ($p = 0.021$) tension with Planck has been reported in OHD through best fits and mock simulations in \cite{Colgain:2022rxy}. In particular, it was noted that a combination of 7 CC and BAO data points above $z = 1.45$ resulted in a $(H_0, \Omega_m) = (37.8, 1)$ best fit, where in line with analysis here, an $\Omega_m \in [0, 1]$ uniform prior was assumed. Based on mock simulations, the probability of such a best fit configuration arising by chance in mocks assuming input parameters consistent with Planck was estimated to be $p = 0.021$ \cite{Colgain:2022rxy}. A similar best fit appears in the last entry of Table \ref{tab:LCDM_CC} and Table \ref{tab:LCDM_CC_PD}, but there is no tension with Planck within the errors, even with our PD analysis, because CC data is inherently of poorer quality than BAO data. One further difference between the analysis is that \cite{Colgain:2022rxy} imposes a Gaussian Planck prior $\Omega_m h^2 = 0.1430 \pm 0.0011$ \cite{Planck:2018vyg} \footnote{This prior essentially prevents high redshift CC data from tracking a non-evolving $H(z)$.} to fix the high redshift behaviour of $H(z)$, whereas our analysis here so far has not introduced a prior. 

% Figure environment removed

Nevertheless, armed with a new PD methodology, we are in a position to revisit the earlier result and see if we can recover the $2 \sigma$ tension with Planck. Since \cite{Colgain:2022rxy} made use of older BAO data, here we replace QSO and Lyman-$\alpha$ BAO with the latest eBOSS results \cite{Hou:2020rse, Neveux:2020voa, duMasdesBourboux:2020pck}. Moreover, we work directly with the $D_{H}/r_d$ constraints and do not invert them. This entails assuming a value for the radius of the sound horizon, which we take to be the Planck value, $r_d = 147.09 \pm 0.26$ Mpc \cite{Planck:2018vyg}. In addition, we reinstate the prior $\Omega_m h^2 = 0.1430 \pm 0.0011$, so that the only difference with \cite{Colgain:2022rxy} is simply to update OHD BAO to the latest constraints. We stress that the priors we introduce are consistent with the Planck cosmology, so \textit{they cannot be driving any disagreement}. Moreover, the $\Omega_m h^2$ prior restricts one to a curve in the $(H_0, \Omega_m)$, but it cannot dictate where one is on the curve, this is done by the remaining 3 CC and 3 BAO data points.  

We again marginalise over the free parameters $(H_0, \Omega_m, r_d)$ with MCMC. In Fig.~\ref{fig:CC_BAO_MCMC} we present the posteriors. While $r_d$ is Gaussian and peaked on our Planck prior, as expected, the $\Omega_m$ posterior is peaked at $\Omega_m \sim 0.6$ and the fact that the fall off in the distribution is gradual beyond the peak leads to a pile up of configurations in the top left corner of the $(H_0, \Omega_m)$-plane. This fall off continues beyond $\Omega_m = 1$ and if the prior is relaxed, the $H_0$ peak shifts to smaller values. So,  once again all the hallmarks of projection effects are present. That being said, given the sharp fall off in the $\Omega_m$ distribution to smaller $\Omega_m$ values, some tension appears to be evident with the Planck values (dashed lines). 

% Figure environment removed

We now run the MCMC chain through our PD methodology. From Fig.~\ref{fig:CC_BAO}, we can see that the $R(H_0)$ and $R(\Omega_m)$ distributions prefer smaller values of $H_0$ and larger values of $\Omega_m$. The peak of the distributions occurs at $H_0 = 42.40$ km/s/Mpc and $\Omega_m = 0.795$.  The lone dot in the $R(H_0)$ distribution at low values of $H_0$ tells us that the distribution falls off sharply below $H_0 = 40$ km/s/Mpc. Note, since we employed generous uniform priors $H_0 \in [0, 200]$, the priors are not impacting the $R(H_0)$ distribution, so it is expected that the distribution falls off to zero on both sides. In contrast, the $R(\Omega_m)$ distribution is one-sided and fails to fall off in the direction of larger values within the uniform priors $\Omega_m \in [0, 1]$. The tension with Planck falls between $2 \sigma$ and $3 \sigma$. By integrating the PDF as far as the black lines corresponding to the Planck values in Fig.~\ref{fig:CC_BAO}, we estimate that the Planck $H_0$ is located at $2.1 \sigma$ from the peak, while the Planck $\Omega_m$ value is $2.5 \sigma$ from the peak.

The main take-away from this section is that OHD data comprising CC and BAO data points beyond $z=1.45$ is inconsistent with the Planck cosmology at in excess of $2 \sigma$. We have employed Planck priors to arrive at this result, but these priors cannot drive the disagreement. Moreover, independent analysis based on least squares fitting and mock simulations presented in \cite{Colgain:2022rxy} also points to a $2 \sigma$ tension, albeit with less up-to-date high redshift BAO data. In summary, different methodologies agree on a $2 \sigma$ discrepancy with Planck, which is robust to interchanging older and newer BAO data. 

\section{Concluding remarks}
\label{sec:discussion}
A $\chi^2$ likelihood is a metric or measure of how well a model fits data. The point in model parameter space that fits the data the best possesses the lowest $\chi^2$. Once one has identified this point, the problem remains to establish $1 \sigma$, $2 \sigma$, etc, confidence intervals in parameter space. In cosmology and astrophysics, MCMC is the prevailing technique for estimating confidence intervals. Its great advantage is that it allows one to i) globally sample the parameter space and ii) arrive at posteriors that serve as an estimate of the errors even with non-Gaussian distributions. In contrast, if one minimises the $\chi^2$ by gradient descent, there is always a risk that one ends up in a local minimum, i. e. the global minimum is missed, while error estimation through Fisher matrix assumes any distribution is Gaussian. The appeal of MCMC marginalisation is that it is widely applicable. However, the point of this paper is that limitations exist, even in the simplest model. 

Indeed, what happens when the MCMC posterior no longer tracks points in parameter space that fit the data better? Traditionally, volume effects are seen as the preserve of higher-dimensional models, e. g. \cite{Herold:2021ksg, Gomez-Valent:2022hkb, Meiers:2023gft}, but projection effects also occur in the minimal $\Lambda$CDM model when one fits the model to data binned by redshift in the late Universe \cite{Colgain:2022tql}. As explained in \cite{Colgain:2022tql}, this ``projection effect'' is driven by OHD, $H(z_i)$, and angular diameter or luminosity distance data, $D_{A}(z_i)$ or $D_{L}(z_i)$, {respectively} only constraining the combinations $\Omega_m h^2$ and $ (1-\Omega_m) h^2$ well, with high redshift data $z_i \gg 0$. In practice, this restricts MCMC configurations to constant $\Omega_m h^2$ and constant $(1-\Omega_m) h^2$ curves in the $(H_0, \Omega_m)$ plane, and as the curves stretch due to DE or matter being less well constrained in high redshift bins, projection effects lead to shifts in the peaks of MCMC posteriors and the emergence of non-Gaussian tails \cite{Colgain:2022tql}. We stress that one sees the same effect in PDFs of best fit $(H_0, \Omega_m)$ parameters in a large number of mock data realisations \cite{Colgain:2022tql}, so the problem is more general than MCMC; there is an inherent bias in the $\Lambda$CDM model when one fits it to redshift binned $H(z)$ \textit{or} $D_{A}(z)$ \textit{or} $D_{L}(z)$ data. Within MCMC, one sees this effect in the errors, but also in the drift of the parameters corresponding to the $\chi^2$ minimum outside of the $1 \sigma$ confidence intervals. Highlighting this (expected) bias in MCMC using OHD is the opening salvo (result) of this paper.     

Why should one care? This is evidently only a problem if one bins data and confronts the $\Lambda$CDM model. First, note that some data sets are inherently binned. For example, effective redshifts are assigned to CC and BAO analysed in a given redshift bin, while each strongly lensed system constitutes its own bin. Working with binned data is unavoidable. Secondly, $\Lambda$CDM tensions point to a problem with the $\Lambda$CDM model once the tensions become widespread and persistent. As explained in \cite{Krishnan:2020vaf}, if the minimal $\Lambda$CDM model is too simple, one expects redshift evolution of $\Lambda$CDM cosmological parameters as it is confronted to redshift binned data. Hints of these trends are now evident in $H_0$ \cite{Wong:2019kwg, Millon:2019slk, Dainotti:2021pqg, Colgain:2022nlb, Colgain:2022rxy, Malekjani:2023dky, Hu:2022kes, Jia:2022ycc, Krishnan:2020obg, Dainotti:2022bzg}, $\Omega_m$ \cite{Risaliti:2015zla, Risaliti:2018reu, Lusso:2020pdb, Yang:2019vgk, Khadka:2020vlh, Khadka:2020tlm, Khadka:2021xcc, Pourojaghi:2022zrh, Colgain:2022nlb, Colgain:2022rxy, Malekjani:2023dky, Pasten:2023rpc, Sakr:2023hrl} and $S_8$/$\sigma_8$ \cite{Esposito:2022plo, Adil:2023jtu, ACT:2023dou, ACT:2023kun} (also \cite{Miyatake:2021qjr, Alonso:2023guh}) across a host of different observables. This evolution is an expected hallmark of model breakdown, which must happen at some redshift if systematics are not universally at play. 

The main problem with redshift dependent $\Lambda$CDM cosmological parameters\footnote{There is a separate interpretation problem as the cosmology literature works with  parameters ``defined today''. In more mathematical language, this is simply the statement that one solves an ordinary differential equation (ODE), namely the Friedmann equation or equivalent, by specifying an integration constant, e.g. $H_0 = H(z=0)$ or $\rho_m(z=0)=\rho_{m0}=H_0^2\Omega_{m}$. However, this is a mathematical statement and it still needs to be confirmed observationally that $H_0$ or $\rho_{m0}$ are \textit{bona fide} constants. This cannot be \textit{a priori} assumed, because it is mathematical prediction of the model. If the model is correct, a constant $H_0$ and $\Omega_m$  will be supported by the data. See \cite{Krishnan:2020vaf} for further discussion.} is one needs to assign a statistical significance to any trend. At a purely practical level, this entails constructing bins centered on different redshifts and identifying discrepancies in $\Lambda$CDM parameters between bins, \textit{ideally in the same observable}, so that the potential systematics are under greatest control. As demonstrated both mathematically and observationally with the CC data in section \ref{sec:MCMC_bias}, MCMC marginalisation leads to biased inferences when one bins the data. In this paper we have resorted to profile distributions \cite{Gomez-Valent:2022hkb} to overcome this bias and have applied the technique to a setting where $\Lambda$CDM distributions are expected to be non-Gaussian for the reasons outlined above and in section \ref{sec:MCMC_bias}. This new technique, provides a complementary perspective that confirms the least square fits of observed and mock data presented in \cite{Colgain:2022nlb, Colgain:2022rxy, Malekjani:2023dky}, where evidence for redshift evolution in $H_0$ and $\Omega_m$ was presented. Regardless of the methodology, the objective is to drill down on the prevailing \textit{assumption} that cosmological parameters are constants. \textit{In the era of tensions in cosmology, nothing can be assumed, especially noting that the tensions are in essence showing an example of evolution of these parameters with redshift.}

More concretely, in this paper with both mock simulations and profile distributions we have shown that high redshift CC data has a preference for a non-evolving $H(z)$ over Planck-$\Lambda$CDM at approximately $\sim 2 \sigma$. This trend, which constitutes the second result of the paper, is unquestionable, as it is visible in the data. Note, we have not propagated systematic uncertainties, so the significance will be less when these are properly propagate. Nevertheless, low and high redshift CC data currently have a preference for different $\Lambda$CDM cosmological parameters. This is important because if the CC program is claiming an 8\% constraint on the Hubble constant, $H_0 = 66.7 \pm 5.5$ km/s/Mpc \cite{Moresco:2023zys}, it is imperative that \textit{all subsets of the data are consistent with this result}. If they are not, then we are staring at either systematics or model breakdown. Admittedly, demanding self-consistency of subsets of a data set confronted to a model is a high bar, but it is important that data sets result in overlapping constraints on $\Lambda$CDM parameters, otherwise this makes cosmological inferences moot. Note, the $\Lambda$CDM model is largely only well tested in the DE dominated regime $z \lesssim 1$ and at very high redshifts $z \sim 1100$, which leaves a wide expanse of redshifts to be explored in order to confirm or refute the model. Given the existing $\Lambda$CDM tensions \cite{Perivolaropoulos:2021jda, Abdalla:2022yfr}, and the hints of evolution in $H_0$, $\Omega_m$ and $S_8$ across assorted probes in the late Universe $z \lesssim 5$, it would be surprising if all discrepancies could be explained away by systematics.\footnote{We are open to the possibility, we just consider it a bad bet at the moment. The odds can of course change as observations improve.}

As an aside, it is intriguing that CC data has a preference for larger best fit values of $H_0$ and smaller best fit values of $\Omega_m$ beyond $z_{\textrm{min}} = 0.7$, as this is traditionally the transition redshift between decelerated and accelerated expansion. % where $\ddot{a} = 0$. 
Moreover, at higher redshifts $z \sim 2.3$, there is not only a longstanding anomaly in Lyman-$\alpha$ BAO \cite{duMasdesBourboux:2020pck}, but QSOs also show a preference for a lower luminosity distance, $D_{L}(z)$, relative to Planck-$\Lambda$CDM \cite{Risaliti:2015zla, Risaliti:2018reu}. Translated into $\Lambda$CDM parameters, this corresponds to conversely larger $\Omega_m$ values, e. g.  \cite{Yang:2019vgk, Khadka:2020vlh, Khadka:2020tlm, Khadka:2021xcc, Pourojaghi:2022zrh}, and consequently smaller $H_0$ values. Thus, the emerging probes CC and QSOs  \cite{Moresco:2022phi} do not appear to be in sync on high redshift $\Lambda$CDM inferences. Nevertheless, neither may be inconsistent with the anomaly in Lyman-$\alpha$ BAO. Relative to Planck-$\Lambda$CDM, Lyman-$\alpha$ BAO prefers \textit{smaller} values of $D_{M}(z) := c \int_{0}^z 1/H(z^{\prime}) \, \textrm{d} z$ and \textit{smaller} values of $H(z)$ (larger values of $D_{H}(z) := c/H(z)$).\footnote{In this statement we assumed the Planck value $r_d \sim 147$ Mpc \cite{Planck:2018vyg} If we reinstate the radius of the sound horizon in these expressions, one recognises that changing the sound horizon, as advocated by early Universe resolutions to Hubble tension, cannot consistently address the Lyman-$\alpha$ BAO anomaly. In general, even for the Planck-$\Lambda$CDM sound horizon, one cannot get both a smaller $D_{M}(z)$ and smaller $H(z)$ from a strictly increasing function, such as the $\Lambda$CDM $H(z)$. As a result, deviations from the Planck-$\Lambda$CDM model that address this anomaly are expected to lead to wiggles in $H(z)$ \cite{Akarsu:2022lhx}, which are unsurprisingly seen in data reconstructions \cite{Zhao:2017cud, Wang:2018fng, Escamilla:2021uoj}. Finally, evolution in $H_0, \Omega_m$ discussed here cannot be explained or accommodated by early resolutions to Hubble tension relying on a change in the $r_d$ at very high $z$.}. If CC data prefer less evolution in $H(z)$ in the matter-dominated regime, then this is consistent with the preference for a smaller $H(z)$ from Lyman-$\alpha$ BAO. Furthermore, QSO data prefers smaller luminosity distances $D_{L}(z)$ relative to Planck, which are consistent with the smaller $D_{M}(z) \propto D_{L}(z)$ values preferred by Lyman-$\alpha$ BAO. Thus, even if CC and QSOs appear to be showing diverging behaviour in the cosmological parameters $(H_0, \Omega_m)$, this may still turn out to be consistent with Lyman-$\alpha$ BAO. We await future DESI \cite{DESI:2023ytc} data releases to ascertain if the non-evolving $H(z)$ trend in high redshift CC data is physical or not. 

Finally, we come to our third and main result outlined in section \ref{sec:tension}. We have revisited a $\sim 2 \sigma$ tension between high redshift CC and BAO data reported in \cite{Colgain:2022rxy}, where the significance was estimated through mock simulations. Here, we have upgraded the BAO data to the latest constraints and again  recover a $>2 \sigma$ discrepancy in $(H_0, \Omega_m)$ with different methodology. This provides a consistency check that there is evolution in OHD between low and high redshifts in the late Universe. Note, this evolution runs contrary to the non-evolving $H(z)$ seen in high redshift CC data because it assumes Planck has accurately constrained the high redshift behaviour of the Hubble parameter in (\ref{eq:lcdm}). Nevertheless, both with and without a Planck prior on $\Omega_m h^2$, evolution at $ \gtrsim 2 \sigma$ is evident in OHD data. It should be stressed that evolution is evident in PDFs of best fit $\Lambda$CDM parameters fitted to a large number of Planck-$\Lambda$CDM mocks \cite{Colgain:2022tql}, so evolution in observed data can be expected. It is imperative to revisit the remaining observations in \cite{Colgain:2022rxy, Malekjani:2023dky} in order to confirm the significance of $\sim 2 \sigma$ hints of evolution found separately in Type Ia SN and QSO data sets. 




\acknowledgments
We would like to thank Adri\`a G\'omez-Valent for discussions and comments on the draft. We thank Gabriela Marques, Mike Hudson and Matteo Viel for related discussions on late Universe evolution in $S_8$. E\'OC thanks Yonsei University and Asia Pacific Center for Theoretical Physics for hospitality. 
This article/publication is based upon work from COST Action CA21136 – “Addressing observational tensions in cosmology with systematics and fundamental physics (CosmoVerse)”, supported by COST (European Cooperation in Science and Technology). SP and MMShJ acknowledge SarAmadan grant No. ISEF/M/401332. MMShJ thanks the support from ICTP associates office (under Senior Associate program) and ICTP HECAP section for hospitality.  


\appendix
\section{Fisher Matrix}
\label{sec:fisher}
Consider the $\chi^2$ (\ref{eq:chi2}). 
Defining $H_{\textrm{model}}(z) = H_0 \sqrt{1-\Omega_m + \Omega_m (1+z)^3}$ and $Q_i$ as in \eqref{eq:Q}, we can now work out the derivatives
\begin{equation}
    \begin{split}
\partial_{H_0} Q_i &= -\sqrt{1-\Omega_m + \Omega_m (1+z_i)^3}, \\  \partial_{\Omega_m} Q_i &= - \frac{1}{2} H_0 (z_i^3 + 3 z_i^2 + 3 z_i)/\sqrt{1-\Omega_m + \Omega_m (1+z_i)^3}, \\
\partial^2_{H_0} Q_i &= 0, \\
\partial_{H_0} \partial_{\Omega_m} Q_i &= - \frac{1}{2} (z_i^3 + 3 z_i^2 + 3 z_i)/\sqrt{1-\Omega_m + \Omega_m (1+z_i)^3}, \\
\partial^2_{\Omega_m} Q_i =& \frac{1}{4} H_0 (z_i^3 + 3 z_i^2 + 3 z_i)^2/(1-\Omega_m + \Omega_m (1+z_i)^3)^{\frac{3}{2}}.      
    \end{split}
\end{equation}
We can then define the Fisher matrix 
\be
F_{ij} = \frac{1}{2} \frac{\partial^2 \chi^2(H_0, \Omega_m)}{\partial p_i \partial p_j}
\ee
where $p_i \in \{ H_0, \Omega_m \}$. Note that the Fisher matrix is evaluated on the best fit parameters. The result is a $2 \times 2$ matrix, which one inverts and the estimated errors are the square root of the diagonal entries. 








\begin{thebibliography}{99}

\bibitem{Planck:2018vyg}
N.~Aghanim \textit{et al.} [Planck],
``Planck 2018 results. VI. Cosmological parameters,''
Astron. Astrophys. \textbf{641} (2020), A6
% doi:10.1051/0004-6361/201833910
%[arXiv:1807.06209 [astro-ph.CO]].

\bibitem{Riess:1998cb}
A.~G.~Riess \textit{et al.} [Supernova Search Team],
``Observational evidence from supernovae for an accelerating universe and a cosmological constant,''
Astron. J. \textbf{116} (1998), 1009-1038
% doi:10.1086/300499
%[arXiv:astro-ph/9805201 [astro-ph]].
%13031 citations counted in INSPIRE as of 02 Feb 2021

\bibitem{Perlmutter:1998np}
S.~Perlmutter \textit{et al.} [Supernova Cosmology Project],
``Measurements of $\Omega$ and $\Lambda$ from 42 high redshift supernovae,''
Astrophys. J. \textbf{517} (1999), 565-586
% doi:10.1086/307221
%[arXiv:astro-ph/9812133 [astro-ph]].
%13057 citations counted in INSPIRE as of 02 Feb 2021

\bibitem{Eisenstein:2005su}
D.~J.~Eisenstein \textit{et al.} [SDSS],
``Detection of the Baryon Acoustic Peak in the Large-Scale Correlation Function of SDSS Luminous Red Galaxies,''
Astrophys. J. \textbf{633} (2005), 560-574
%doi:10.1086/466512
%[arXiv:astro-ph/0501171 [astro-ph]].
%3380 citations counted in INSPIRE as of 08 Oct 2020

\bibitem{Riess:2021jrx}
A.~G.~Riess, W.~Yuan, L.~M.~Macri, D.~Scolnic, D.~Brout, S.~Casertano, D.~O.~Jones, Y.~Murakami, L.~Breuval and T.~G.~Brink, \textit{et al.}
``A Comprehensive Measurement of the Local Value of the Hubble Constant with 1 km s$^{?1}$ Mpc$^{?1}$ Uncertainty from the Hubble Space Telescope and the SH0ES Team,''
Astrophys. J. Lett. \textbf{934} (2022) no.1, L7
%doi:10.3847/2041-8213/ac5c5b
%[arXiv:2112.04510 [astro-ph.CO]].
%370 citations counted in INSPIRE as of 09 Jan 2023

\bibitem{Freedman:2021ahq}
W.~L.~Freedman,
``Measurements of the Hubble Constant: Tensions in Perspective,''
Astrophys. J. \textbf{919} (2021) no.1, 16
%doi:10.3847/1538-4357/ac0e95
%[arXiv:2106.15656 [astro-ph.CO]].
%179 citations counted in INSPIRE as of 09 Jan 2023

\bibitem{Pesce:2020xfe}
D.~W.~Pesce, J.~A.~Braatz, M.~J.~Reid, A.~G.~Riess, D.~Scolnic, J.~J.~Condon, F.~Gao, C.~Henkel, C.~M.~V.~Impellizzeri and C.~Y.~Kuo, \textit{et al.}
%``The Megamaser Cosmology Project. XIII. Combined Hubble constant constraints,''
Astrophys. J. Lett. \textbf{891} (2020) no.1, L1
%doi:10.3847/2041-8213/ab75f0
%[arXiv:2001.09213 [astro-ph.CO]].
%96 citations counted in INSPIRE as of 12 Jul 2021

\bibitem{Blakeslee:2021rqi}
J.~P.~Blakeslee, J.~B.~Jensen, C.~P.~Ma, P.~A.~Milne and J.~E.~Greene,
%``The Hubble Constant from Infrared Surface Brightness Fluctuation Distances,''
Astrophys. J. \textbf{911} (2021) no.1, 65
%doi:10.3847/1538-4357/abe86a
%[arXiv:2101.02221 [astro-ph.CO]].
%11 citations counted in INSPIRE as of 12 Jul 2021

\bibitem{Kourkchi:2020iyz}
E.~Kourkchi, R.~B.~Tully, G.~S.~Anand, H.~M.~Courtois, A.~Dupuy, J.~D.~Neill, L.~Rizzi and M.~Seibert,
%``Cosmicflows-4: The Calibration of Optical and Infrared Tully\textendash{}Fisher Relations,''
Astrophys. J. \textbf{896} (2020) no.1, 3
%doi:10.3847/1538-4357/ab901c
%[arXiv:2004.14499 [astro-ph.GA]].
%15 citations counted in INSPIRE as of 12 Jul 2021

\bibitem{HSC:2018mrq}
C.~Hikage \textit{et al.} [HSC],
``Cosmology from cosmic shear power spectra with Subaru Hyper Suprime-Cam first-year data,''
Publ. Astron. Soc. Jap. \textbf{71}, 43  (2019).
%doi:10.1093/pasj/psz010

\bibitem{KiDS:2020suj}
M.~Asgari \textit{et al.} [KiDS],
``KiDS-1000 Cosmology: Cosmic shear constraints and comparison between two point statistics,''
Astron. Astrophys. \textbf{645} (2021), A104
%doi:10.1051/0004-6361/202039070
%[arXiv:2007.15633 [astro-ph.CO]].
%113 citations counted in INSPIRE as of 18 Aug 2021

\bibitem{DES:2021wwk}
T.~M.~C.~Abbott \textit{et al.} [DES],
``Dark Energy Survey Year 3 results: Cosmological constraints from galaxy clustering and weak lensing,''
Phys. Rev. D \textbf{105} (2022) no.2, 023520
%doi:10.1103/PhysRevD.105.023520
%[arXiv:2105.13549 [astro-ph.CO]].
%519 citations counted in INSPIRE as of 14 Jul 2023

\bibitem{Boruah:2019icj}
S.~S.~Boruah, M.~J.~Hudson and G.~Lavaux,
``Cosmic flows in the nearby Universe: new peculiar velocities from SNe and cosmological constraints,''
Mon. Not. Roy. Astron. Soc. \textbf{498} (2020) no.2, 2703-2718
%doi:10.1093/mnras/staa2485
%[arXiv:1912.09383 [astro-ph.CO]].
%54 citations counted in INSPIRE as of 14 Jul 2023

\bibitem{Said:2020epb}
K.~Said, M.~Colless, C.~Magoulas, J.~R.~Lucey and M.~J.~Hudson,
``Joint analysis of 6dFGS and SDSS peculiar velocities for the growth rate of cosmic structure and tests of gravity,''
Mon. Not. Roy. Astron. Soc. \textbf{497} (2020) no.1, 1275-1293
%doi:10.1093/mnras/staa2032
%[arXiv:2007.04993 [astro-ph.CO]].
%49 citations counted in INSPIRE as of 14 Jul 2023

\bibitem{Perivolaropoulos:2021jda}
L.~Perivolaropoulos and F.~Skara,
``Challenges for \ensuremath{\Lambda}CDM: An update,''
New Astron. Rev. \textbf{95}, 101659  (2022).
%doi:10.1016/j.newar.2022.101659
%\href{https://arxiv.org/abs/2105.05208}{2105.05208}

\bibitem{Abdalla:2022yfr}
E.~Abdalla, G.~Franco Abell\'an, A.~Aboubrahim, A.~Agnello, O.~Akarsu, Y.~Akrami, G.~Alestas, D.~Aloni, L.~Amendola and L.~A.~Anchordoqui, \textit{et al.}
``Cosmology intertwined: A review of the particle physics, astrophysics, and cosmology associated with the cosmological tensions and anomalies,''
JHEAp \textbf{34}, 49  (2022).
%doi:10.1016/j.jheap.2022.04.002
%\href{https://arxiv.org/abs/2203.06142}{2203.06142}

\bibitem{Phillips:1993ng}
M.~M.~Phillips,
``The absolute magnitudes of Type IA supernovae,''
Astrophys. J. Lett. \textbf{413} (1993), L105-L108
%doi:10.1086/186970
%1245 citations counted in INSPIRE as of 24 Aug 2021

\bibitem{NearbySupernovaFactory:2018qkd}
M.~Rigault \textit{et al.} [Nearby Supernova Factory],
``Strong Dependence of Type Ia Supernova Standardization on the Local Specific Star Formation Rate,''
Astron. Astrophys. \textbf{644} (2020), A176
%doi:10.1051/0004-6361/201730404
%[arXiv:1806.03849 [astro-ph.CO]].
%143 citations counted in INSPIRE as of 20 Jul 2023

\bibitem{Kang:2019azh}
Y.~Kang, Y.~W.~Lee, Y.~L.~Kim, C.~Chung and C.~H.~Ree,
``Early-type Host Galaxies of Type Ia Supernovae. II. Evidence for Luminosity Evolution in Supernova Cosmology,''
Astrophys. J. \textbf{889} (2020) no.1, 8
%doi:10.3847/1538-4357/ab5afc
%[arXiv:1912.04903 [astro-ph.GA]].
%56 citations counted in INSPIRE as of 20 Jul 2023

\bibitem{Brout:2020msh}
D.~Brout and D.~Scolnic,
``It\textquoteright{}s Dust: Solving the Mysteries of the Intrinsic Scatter and Host-galaxy Dependence of Standardized Type Ia Supernova Brightnesses,''
Astrophys. J. \textbf{909} (2021) no.1, 26
%doi:10.3847/1538-4357/abd69b
%[arXiv:2004.10206 [astro-ph.CO]].
%82 citations counted in INSPIRE as of 20 Jul 2023

\bibitem{Lee:2021txi}
Y.~W.~Lee, C.~Chung, P.~Demarque, S.~Park, J.~Son and Y.~Kang,
``Evidence for strong progenitor age dependence of type Ia supernova luminosity standardization process,''
Mon. Not. Roy. Astron. Soc. \textbf{517} (2022) no.2, 2697-2708
%doi:10.1093/mnras/stac2840
%[arXiv:2107.06288 [astro-ph.GA]].
%5 citations counted in INSPIRE as of 20 Jul 2023


\bibitem{Krishnan:2020vaf}
C.~Krishnan, E.~\'O~Colg\'ain, M.~M.~Sheikh-Jabbari and T.~Yang,
``Running Hubble Tension and a H0 Diagnostic,''
Phys. Rev. D \textbf{103} (2021) no.10, 103509
%doi:10.1103/PhysRevD.103.103509
%[arXiv:2011.02858 [astro-ph.CO]].
%65 citations counted in INSPIRE as of 14 Jul 2023 

\bibitem{Krishnan:2022fzz}
C.~Krishnan and R.~Mondol,
``$H_0$ as a Universal FLRW Diagnostic,''
[arXiv:2201.13384 [astro-ph.CO]].
%12 citations counted in INSPIRE as of 14 Jul 2023

%\bibitem{Liao:2020zko}
%K.~Liao, A.~Shafieloo, R.~E.~Keeley and E.~V.~Linder,
%``Determining Model-independent H 0 and Consistency Tests,''
%Astrophys. J. Lett. \textbf{895} (2020) no.2, L29
%doi:10.3847/2041-8213/ab8dbb
%[arXiv:2002.10605 [astro-ph.CO]].
%51 citations counted in INSPIRE as of 14 Jul 2023

%\bibitem{Montani:2023xpd}
%G.~Montani, M.~De Angelis, F.~Bombacigno and N.~Carlevaro,
%``Metric $f(R)$ gravity with dynamical dark energy as a paradigm for the Hubble Tension,''
%[arXiv:2306.11101 [gr-qc]].
%1 citations counted in INSPIRE as of 14 Jul 2023

\bibitem{Wong:2019kwg}
K.~C.~Wong, S.~H.~Suyu, G.~C.~F.~Chen, C.~E.~Rusu, M.~Millon, D.~Sluse, V.~Bonvin, C.~D.~Fassnacht, S.~Taubenberger and M.~W.~Auger, \textit{et al.}
``H0LiCOW \textendash{} XIII. A 2.4 per cent measurement of H0 from lensed quasars: 5.3\ensuremath{\sigma} tension between early- and late-Universe probes,''
Mon. Not. Roy. Astron. Soc. \textbf{498} (2020) no.1, 1420-1439
%doi:10.1093/mnras/stz3094
%[arXiv:1907.04869 [astro-ph.CO]].
%804 citations counted in INSPIRE as of 18 May 2023

\bibitem{Millon:2019slk}
M.~Millon, A.~Galan, F.~Courbin, T.~Treu, S.~H.~Suyu, X.~Ding, S.~Birrer, G.~C.~F.~Chen, A.~J.~Shajib and D.~Sluse, \textit{et al.}
``TDCOSMO. I. An exploration of systematic uncertainties in the inference of $H_0$ from time-delay cosmography,''
Astron. Astrophys. \textbf{639} (2020), A101
%doi:10.1051/0004-6361/201937351
%[arXiv:1912.08027 [astro-ph.CO]].
%114 citations counted in INSPIRE as of 18 May 2023

\bibitem{Sluse:2003iy}
D.~Sluse, J.~Surdej, J.~F.~Claeskens, D.~Hutsemekers, C.~Jean, F.~Courbin, T.~Nakos, M.~Billeres and S.~V.~Khmil,
``A Quadruply imaged quasar with an optical Einstein ring candidate: 1RXS J113155.4-123155,''
Astron. Astrophys. \textbf{406} (2003), L43-L46
%doi:10.1051/0004-6361:20030904
%[arXiv:astro-ph/0307345 [astro-ph]].
%83 citations counted in INSPIRE as of 14 Jul 2023

\bibitem{Shajib:2023uig}
A.~J.~Shajib, P.~Mozumdar, G.~C.~F.~Chen, T.~Treu, M.~Cappellari, S.~Knabel, S.~H.~Suyu, V.~N.~Bennert, J.~A.~Frieman and D.~Sluse, \textit{et al.}
``TDCOSMO. XIII. Improved Hubble constant measurement from lensing time delays using spatially resolved stellar kinematics of the lens galaxy,''
Astron. Astrophys. \textbf{673} (2023), A9
%doi:10.1051/0004-6361/202345878
%[arXiv:2301.02656 [astro-ph.CO]].
%3 citations counted in INSPIRE as of 18 May 2023

\bibitem{Dainotti:2021pqg}
M.~G.~Dainotti, B.~De Simone, T.~Schiavone, G.~Montani, E.~Rinaldi and G.~Lambiase,
``On the Hubble constant tension in the SNe Ia Pantheon sample,''
Astrophys. J. \textbf{912}, 150  (2021).
%doi:10.3847/1538-4357/abeb73


\bibitem{Colgain:2022nlb}
E.~\'O~Colg\'ain, M.~M.~Sheikh-Jabbari, R.~Solomon, G.~Bargiacchi, S.~Capozziello, M.~G.~Dainotti and D.~Stojkovic,
``Revealing intrinsic flat \ensuremath{\Lambda}CDM biases with standardizable candles,''
Phys. Rev. D \textbf{106}, L041301  (2022).
%doi:10.1103/PhysRevD.106.L041301

\bibitem{Colgain:2022rxy}
E.~\'O~Colg\'ain, M.~M.~Sheikh-Jabbari, R.~Solomon, M.~G.~Dainotti and D.~Stojkovic,
``Putting Flat $\Lambda$CDM In The (Redshift) Bin,''
[arXiv:2206.11447 [astro-ph.CO]].
%42 citations counted in INSPIRE as of 14 Jul 2023

%\cite{Colgain:2022tql}
%\bibitem{Colgain:2022tql}
%E.~\'O.~Colg\'ain, M.~M.~Sheikh-Jabbari and R.~Solomon,
%``High redshift \ensuremath{\Lambda}CDM cosmology: To bin or not to bin?,''
%Phys. Dark Univ. \textbf{40} (2023), 101216
%doi:10.1016/j.dark.2023.101216
%[arXiv:2211.02129 [astro-ph.CO]].
%10 citations counted in INSPIRE as of 25 Jul 2023


\bibitem{Malekjani:2023dky}
M.~Malekjani, R.~M.~Conville, E.~\'O.~Colg\'ain, S.~Pourojaghi and M.~M.~Sheikh-Jabbari,
``Negative Dark Energy Density from High Redshift Pantheon+ Supernovae,''
[arXiv:2301.12725 [astro-ph.CO]].
%13 citations counted in INSPIRE as of 17 Jul 2023

\bibitem{Hu:2022kes}
J.~P.~Hu and F.~Y.~Wang,
``Revealing the late-time transition of H0: relieve the Hubble crisis,''
Mon. Not. Roy. Astron. Soc. \textbf{517}, 576  (2022).

\bibitem{Jia:2022ycc}
X.~D.~Jia, J.~P.~Hu and F.~Y.~Wang,
``Evidence of a decreasing trend for the Hubble constant,''
Astron. Astrophys. \textbf{674} (2023), A45
%doi:10.1051/0004-6361/202346356
%[arXiv:2212.00238 [astro-ph.CO]].
%10 citations counted in INSPIRE as of 17 Jul 2023

\bibitem{Krishnan:2020obg}
C.~Krishnan, E.~\'O~Colg\'ain, Ruchika, A.~A.~Sen, M.~M.~Sheikh-Jabbari and T.~Yang,
``Is there an early Universe solution to Hubble tension?,''
Phys. Rev. D \textbf{102} (2020) no.10, 103525
%doi:10.1103/PhysRevD.102.103525
%[arXiv:2002.06044 [astro-ph.CO]].
%69 citations counted in INSPIRE as of 17 Jul 2023

\bibitem{Dainotti:2022bzg}
M.~G.~Dainotti, B.~De Simone, T.~Schiavone, G.~Montani, E.~Rinaldi, G.~Lambiase, M.~Bogdan and S.~Ugale,
``On the Evolution of the Hubble Constant with the SNe Ia Pantheon Sample and Baryon Acoustic Oscillations: A Feasibility Study for GRB-Cosmology in 2030,''
Galaxies \textbf{10}, 24  (2022).
%doi:10.3390/galaxies10010024

\bibitem{Risaliti:2015zla}
G.~Risaliti and E.~Lusso,
``A Hubble Diagram for Quasars,''
Astrophys. J. \textbf{815} (2015), 33
%doi:10.1088/0004-637X/815/1/33
%[arXiv:1505.07118 [astro-ph.CO]].
%146 citations counted in INSPIRE as of 16 Jun 2023

\bibitem{Risaliti:2018reu}
G.~Risaliti and E.~Lusso,
``Cosmological constraints from the Hubble diagram of quasars at high redshifts,''
Nature Astron. \textbf{3}, 272  (2019).

\bibitem{Lusso:2020pdb}
E.~Lusso, G.~Risaliti, E.~Nardini, G.~Bargiacchi, M.~Benetti, S.~Bisogni, S.~Capozziello, F.~Civano, L.~Eggleston and M.~Elvis, \textit{et al.}
``Quasars as standard candles III. Validation of a new sample for cosmological studies,''
Astron. Astrophys. \textbf{642}, A150  (2020).


\bibitem{Yang:2019vgk}
T.~Yang, A.~Banerjee and E.~\'O~Colg\'ain,
``Cosmography and flat $\Lambda$CDM tensions at high redshift,''
Phys. Rev. D \textbf{102}, 123532  (2020).

\bibitem{Khadka:2020vlh}
N.~Khadka and B.~Ratra,
``Using quasar X-ray and UV flux measurements to constrain cosmological model parameters,''
Mon. Not. Roy. Astron. Soc. \textbf{497}, 263  (2020).


\bibitem{Khadka:2020tlm}
N.~Khadka and B.~Ratra,
``Determining the range of validity of quasar X-ray and UV flux measurements for constraining cosmological model parameters,''
Mon. Not. Roy. Astron. Soc. \textbf{502}, 6140  (2021).


\bibitem{Khadka:2021xcc}
N.~Khadka and B.~Ratra,
``Do quasar X-ray and UV flux measurements provide a useful test of cosmological models?,''
Mon. Not. Roy. Astron. Soc. \textbf{510}, 2753  (2022).
%doi:10.1093/mnras/stab3678

\bibitem{Pourojaghi:2022zrh}
S.~Pourojaghi, N.~F.~Zabihi and M.~Malekjani,
``Can high-redshift Hubble diagrams rule out the standard model of cosmology in the context of cosmography?,''
Phys. Rev. D \textbf{106}, 123523  (2022).


\bibitem{Zajacek:2023qjm}
M.~Zaja\v{c}ek, B.~Czerny, N.~Khadka, R.~Prince, S.~Panda, M.~L.~Mart\'\i{}nez-Aldama and B.~Ratra,
``Extinction biases quasar luminosity distances determined from quasar UV and X-ray flux measurements,''
[arXiv:2305.08179 [astro-ph.GA]].
%0 citations counted in INSPIRE as of 17 Jul 2023

\bibitem{Pasten:2023rpc}
E.~Past\'en and V.~H.~C\'ardenas,
``Testing \ensuremath{\Lambda}CDM cosmology in a binned universe: Anomalies in the deceleration parameter,''
Phys. Dark Univ. \textbf{40} (2023), 101224
%doi:10.1016/j.dark.2023.101224
%[arXiv:2301.10740 [astro-ph.CO]].

\bibitem{Wagner:2022etu}
J.~Wagner,
``Casting the $H_0$ tension as a fitting problem of cosmologies,''
[arXiv:2203.11219 [astro-ph.CO]].
%5 citations counted in INSPIRE as of 28 Jul 2023

\bibitem{Sakr:2023hrl}
Z.~Sakr,
``One matter density discrepancy to alleviate them all or further trouble for $\Lambda$CDM model,''
[arXiv:2305.02846 [astro-ph.CO]].
%0 citations counted in INSPIRE as of 24 Jul 2023


\bibitem{Colgain:2022tql}
E.~\'O~Colg\'ain, M.~M.~Sheikh-Jabbari and R.~Solomon,
``High redshift \ensuremath{\Lambda}CDM cosmology: To bin or not to bin?,''
Phys. Dark Univ. \textbf{40} (2023), 101216
%doi:10.1016/j.dark.2023.101216
[arXiv:2211.02129 [astro-ph.CO]].
%10 citations counted in INSPIRE as of 28 Jun 2023

\bibitem{Esposito:2022plo}
M.~Esposito, V.~Ir\v{s}i\v{c}, M.~Costanzi, S.~Borgani, A.~Saro and M.~Viel,
``Weighing cosmic structures with clusters of galaxies and the intergalactic medium,''
Mon. Not. Roy. Astron. Soc. \textbf{515}, 857  (2022).
%doi:10.1093/mnras/stac1825
[arXiv:2202.00974 [astro-ph.CO]].

\bibitem{Adil:2023jtu}
S.~A.~Adil, \"O.~Akarsu, M.~Malekjani, E.~\'O~Colg\'ain, S.~Pourojaghi, A.~A.~Sen and M.~M.~Sheikh-Jabbari,
``$S_8$ increases with effective redshift in $\Lambda$CDM cosmology,''
[arXiv:2303.06928 [astro-ph.CO]].
%1 citations counted in INSPIRE as of 14 Jul 2023

\bibitem{ACT:2023dou}
F.~J.~Qu \textit{et al.} [ACT],
``The Atacama Cosmology Telescope: A Measurement of the DR6 CMB Lensing Power Spectrum and its Implications for Structure Growth,''
[arXiv:2304.05202 [astro-ph.CO]].
%10 citations counted in INSPIRE as of 14 Jul 2023

\bibitem{ACT:2023kun}
M.~S.~Madhavacheril \textit{et al.} [ACT],
``The Atacama Cosmology Telescope: DR6 Gravitational Lensing Map and Cosmological Parameters,''
[arXiv:2304.05203 [astro-ph.CO]].
%10 citations counted in INSPIRE as of 14 Jul 2023

\bibitem{ACT:2023ipp}
G.~A.~Marques \textit{et al.} [ACT and DES],
``Cosmological constraints from the tomography of DES-Y3 galaxies with CMB lensing from ACT DR4,''
[arXiv:2306.17268 [astro-ph.CO]].
%0 citations counted in INSPIRE as of 14 Jul 2023

\bibitem{Miyatake:2021qjr}
H.~Miyatake, Y.~Harikane, M.~Ouchi, Y.~Ono, N.~Yamamoto, A.~J.~Nishizawa, N.~Bahcall, S.~Miyazaki and A.~A.~Plazas Malag\'on,
``First Identification of a CMB Lensing Signal Produced by 1.5~Million Galaxies at z\ensuremath{\sim}4: Constraints on Matter Density Fluctuations at High Redshift,''
Phys. Rev. Lett. \textbf{129} (2022) no.6, 061301
%doi:10.1103/PhysRevLett.129.061301
[arXiv:2103.15862 [astro-ph.CO]].
%7 citations counted in INSPIRE as of 25 Jul 2023

\bibitem{Alonso:2023guh}
D.~Alonso, G.~Fabbian, K.~Storey-Fisher, A.~C.~Eilers, C.~Garc\'\i{}a-Garc\'\i{}a, D.~W.~Hogg and H.~W.~Rix,
``Constraining cosmology with the Gaia-unWISE Quasar Catalog and CMB lensing: structure growth,''
[arXiv:2306.17748 [astro-ph.CO]].
%0 citations counted in INSPIRE as of 25 Jul 2023


\bibitem{Herold:2021ksg}
L.~Herold, E.~G.~M.~Ferreira and E.~Komatsu,
``New Constraint on Early Dark Energy from Planck and BOSS Data Using the Profile Likelihood,''
Astrophys. J. Lett. \textbf{929} (2022) no.1, L16
%doi:10.3847/2041-8213/ac63a3
%[arXiv:2112.12140 [astro-ph.CO]].
%43 citations counted in INSPIRE as of 17 Jul 2023

\bibitem{Gomez-Valent:2022hkb}
A.~G\'omez-Valent,
``Fast test to assess the impact of marginalization in Monte~Carlo analyses and its application to cosmology,''
Phys. Rev. D \textbf{106} (2022) no.6, 063506
%doi:10.1103/PhysRevD.106.063506
%[arXiv:2203.16285 [astro-ph.CO]].
%20 citations counted in INSPIRE as of 11 Jul 2023

\bibitem{Meiers:2023gft}
M.~Meiers, L.~Knox and N.~Sch\"oneberg,
``Exploration of the Pre-recombination Universe with a High-Dimensional Model of an Additional Dark Fluid,''
[arXiv:2307.09522 [astro-ph.CO]].
%0 citations counted in INSPIRE as of 22 Jul 2023

\bibitem{Poulin:2018cxd}
V.~Poulin, T.~L.~Smith, T.~Karwal and M.~Kamionkowski,
``Early Dark Energy Can Resolve The Hubble Tension,''
Phys. Rev. Lett. \textbf{122} (2019) no.22, 221301
%doi:10.1103/PhysRevLett.122.221301
%[arXiv:1811.04083 [astro-ph.CO]].
%608 citations counted in INSPIRE as of 17 Jul 2023

\bibitem{Niedermann:2019olb}
F.~Niedermann and M.~S.~Sloth,
``New early dark energy,''
Phys. Rev. D \textbf{103} (2021) no.4, L041303
%doi:10.1103/PhysRevD.103.L041303
[arXiv:1910.10739 [astro-ph.CO]].
%140 citations counted in INSPIRE as of 24 Jul 2023

\bibitem{Jimenez:2001gg}
R.~Jimenez and A.~Loeb,
``Constraining cosmological parameters based on relative galaxy ages,''
Astrophys. J. \textbf{573} (2002), 37-42
%doi:10.1086/340549
%[arXiv:astro-ph/0106145 [astro-ph]].
%598 citations counted in INSPIRE as of 28 Jun 2023

\bibitem{Stern:2009ep}
D.~Stern, R.~Jimenez, L.~Verde, M.~Kamionkowski and S.~A.~Stanford,
``Cosmic Chronometers: Constraining the Equation of State of Dark Energy. I: H(z) Measurements,''
JCAP \textbf{02} (2010), 008
%doi:10.1088/1475-7516/2010/02/008
%[arXiv:0907.3149 [astro-ph.CO]].
%740 citations counted in INSPIRE as of 20 May 2022

\bibitem{Moresco:2012jh}
M.~Moresco, A.~Cimatti, R.~Jimenez, L.~Pozzetti, G.~Zamorani, M.~Bolzonella, J.~Dunlop, F.~Lamareille, M.~Mignoli and H.~Pearce, \textit{et al.}
``Improved constraints on the expansion rate of the Universe up to z\textasciitilde{}1.1 from the spectroscopic evolution of cosmic chronometers,''
JCAP \textbf{08} (2012), 006
%doi:10.1088/1475-7516/2012/08/006
%[arXiv:1201.3609 [astro-ph.CO]].
%508 citations counted in INSPIRE as of 20 May 2022

\bibitem{Zhang:2012mp}
C.~Zhang, H.~Zhang, S.~Yuan, T.~J.~Zhang and Y.~C.~Sun,
``Four new observational $H(z)$ data from luminous red galaxies in the Sloan Digital Sky Survey data release seven,''
Res. Astron. Astrophys. \textbf{14} (2014) no.10, 1221-1233
%doi:10.1088/1674-4527/14/10/002
%[arXiv:1207.4541 [astro-ph.CO]].
%425 citations counted in INSPIRE as of 20 May 2022

\bibitem{Moresco:2016mzx}
M.~Moresco, L.~Pozzetti, A.~Cimatti, R.~Jimenez, C.~Maraston, L.~Verde, D.~Thomas, A.~Citro, R.~Tojeiro and D.~Wilkinson,
``A 6\% measurement of the Hubble parameter at $z\sim0.45$: direct evidence of the epoch of cosmic re-acceleration,''
JCAP \textbf{05} (2016), 014
%doi:10.1088/1475-7516/2016/05/014
%[arXiv:1601.01701 [astro-ph.CO]].
%505 citations counted in INSPIRE as of 17 May 2022

\bibitem{Ratsimbazafy:2017vga}
A.~L.~Ratsimbazafy, S.~I.~Loubser, S.~M.~Crawford, C.~M.~Cress, B.~A.~Bassett, R.~C.~Nichol and P.~V\"ais\"anen,
``Age-dating Luminous Red Galaxies observed with the Southern African Large Telescope,''
Mon. Not. Roy. Astron. Soc. \textbf{467} (2017) no.3, 3239-3254
%doi:10.1093/mnras/stx301
%[arXiv:1702.00418 [astro-ph.CO]].
%162 citations counted in INSPIRE as of 17 May 2022

\bibitem{Borghi:2021rft}
N.~Borghi, M.~Moresco and A.~Cimatti,
``Toward a Better Understanding of Cosmic Chronometers: A New Measurement of H(z) at z \ensuremath{\sim} 0.7,''
Astrophys. J. Lett. \textbf{928} (2022) no.1, L4
%doi:10.3847/2041-8213/ac3fb2
%[arXiv:2110.04304 [astro-ph.CO]].
%10 citations counted in INSPIRE as of 17 May 2022

\bibitem{Jiao:2022aep}
K.~Jiao, N.~Borghi, M.~Moresco and T.~J.~Zhang,
``New Observational H(z) Data from Full-spectrum Fitting of Cosmic Chronometers in the LEGA-C Survey,''
Astrophys. J. Suppl. \textbf{265} (2023) no.2, 48
%doi:10.3847/1538-4365/acbc77
%[arXiv:2205.05701 [astro-ph.CO]].
%14 citations counted in INSPIRE as of 17 Jul 2023

\bibitem{Tomasetti:2023kek}
E.~Tomasetti, M.~Moresco, N.~Borghi, K.~Jiao, A.~Cimatti, L.~Pozzetti, A.~C.~Carnall, R.~J.~McLure and L.~Pentericci,
``A new measurement of the expansion history of the Universe at z=1.26 with cosmic chronometers in VANDELS,''
[arXiv:2305.16387 [astro-ph.CO]].
%1 citations counted in INSPIRE as of 28 Jun 2023

\bibitem{Moresco:2023zys}
M.~Moresco,
``Addressing the Hubble tension with cosmic chronometers,''
[arXiv:2307.09501 [astro-ph.CO]].
%0 citations counted in INSPIRE as of 24 Jul 2023

\bibitem{Moresco:2020fbm}
M.~Moresco, R.~Jimenez, L.~Verde, A.~Cimatti and L.~Pozzetti,
``Setting the Stage for Cosmic Chronometers. II. Impact of Stellar Population Synthesis Models Systematics and Full Covariance Matrix,''
Astrophys. J. \textbf{898} (2020) no.1, 82
%doi:10.3847/1538-4357/ab9eb0
[arXiv:2003.07362 [astro-ph.GA]].
%57 citations counted in INSPIRE as of 28 Jul 2023

\bibitem{Foreman-Mackey:2012any}
D.~Foreman-Mackey, D.~W.~Hogg, D.~Lang and J.~Goodman,
``emcee: The MCMC Hammer,''
Publ. Astron. Soc. Pac. \textbf{125} (2013), 306-312
%doi:10.1086/670067
%[arXiv:1202.3665 [astro-ph.IM]].
%3393 citations counted in INSPIRE as of 17 Jul 2023


\bibitem{Hou:2020rse}
J.~Hou, A.~G.~S\'anchez, A.~J.~Ross, A.~Smith, R.~Neveux, J.~Bautista, E.~Burtin, C.~Zhao, R.~Scoccimarro and K.~S.~Dawson, \textit{et al.}
``The Completed SDSS-IV extended Baryon Oscillation Spectroscopic Survey: BAO and RSD measurements from anisotropic clustering analysis of the Quasar Sample in configuration space between redshift 0.8 and 2.2,''
Mon. Not. Roy. Astron. Soc. \textbf{500} (2020) no.1, 1201-1221
%:10.1093/mnras/staa3234
%[arXiv:2007.08998 [astro-ph.CO]].
%135 citations counted in INSPIRE as of 28 Jun 2023

\bibitem{Neveux:2020voa}
R.~Neveux, E.~Burtin, A.~de Mattia, A.~Smith, A.~J.~Ross, J.~Hou, J.~Bautista, J.~Brinkmann, C.~H.~Chuang and K.~S.~Dawson, \textit{et al.}
``The completed SDSS-IV extended Baryon Oscillation Spectroscopic Survey: BAO and RSD measurements from the anisotropic power spectrum of the quasar sample between redshift 0.8 and 2.2,''
Mon. Not. Roy. Astron. Soc. \textbf{499} (2020) no.1, 210-229
%doi:10.1093/mnras/staa2780
%[arXiv:2007.08999 [astro-ph.CO]].
%133 citations counted in INSPIRE as of 28 Jun 2023

\bibitem{duMasdesBourboux:2020pck}
H.~du Mas des Bourboux, J.~Rich, A.~Font-Ribera, V.~de Sainte Agathe, J.~Farr, T.~Etourneau, J.~M.~Le Goff, A.~Cuceu, C.~Balland and J.~E.~Bautista, \textit{et al.}
``The Completed SDSS-IV Extended Baryon Oscillation Spectroscopic Survey: Baryon Acoustic Oscillations with Ly\ensuremath{\alpha} Forests,''
Astrophys. J. \textbf{901} (2020) no.2, 153
%doi:10.3847/1538-4357/abb085
%[arXiv:2007.08995 [astro-ph.CO]].
%172 citations counted in INSPIRE as of 28 Jun 2023

\bibitem{Trotta:2017wnx}
R.~Trotta,
``Bayesian Methods in Cosmology,''
[arXiv:1701.01467 [astro-ph.CO]].
%96 citations counted in INSPIRE as of 18 Jul 2023

\bibitem{Moresco:2022phi}
M.~Moresco, L.~Amati, L.~Amendola, S.~Birrer, J.~P.~Blakeslee, M.~Cantiello, A.~Cimatti, J.~Darling, M.~Della Valle and M.~Fishbach, \textit{et al.}
``Unveiling the Universe with emerging cosmological probes,''
Living Rev. Rel. \textbf{25} (2022) no.1, 6
%doi:10.1007/s41114-022-00040-z
%[arXiv:2201.07241 [astro-ph.CO]].
%71 citations counted in INSPIRE as of 16 Jun 2023

\bibitem{DESI:2023ytc}
G.~Adame \textit{et al.} [DESI],
``The Early Data Release of the Dark Energy Spectroscopic Instrument,''
%doi:10.5281/zenodo.7964161
[arXiv:2306.06308 [astro-ph.CO]].
%13 citations counted in INSPIRE as of 26 Jul 2023

\bibitem{Akarsu:2022lhx}
O.~Akarsu, E.~\'O~Colg\'ain, E.~\"Ozulker, S.~Thakur and L.~Yin,
``Inevitable manifestation of wiggles in the expansion of the late Universe,''
Phys. Rev. D \textbf{107} (2023) no.12, 123526
%doi:10.1103/PhysRevD.107.123526
%[arXiv:2207.10609 [astro-ph.CO]].
%6 citations counted in INSPIRE as of 17 Jul 2023

\bibitem{Zhao:2017cud}
G.~B.~Zhao, M.~Raveri, L.~Pogosian, Y.~Wang, R.~G.~Crittenden, W.~J.~Handley, W.~J.~Percival, F.~Beutler, J.~Brinkmann and C.~H.~Chuang, \textit{et al.}
``Dynamical dark energy in light of the latest observations,''
Nature Astron. \textbf{1} (2017) no.9, 627-632
%doi:10.1038/s41550-017-0216-z
%[arXiv:1701.08165 [astro-ph.CO]].
%356 citations counted in INSPIRE as of 17 Jul 2023

\bibitem{Wang:2018fng}
Y.~Wang, L.~Pogosian, G.~B.~Zhao and A.~Zucca,
``Evolution of dark energy reconstructed from the latest observations,''
Astrophys. J. Lett. \textbf{869} (2018), L8
%doi:10.3847/2041-8213/aaf238
%[arXiv:1807.03772 [astro-ph.CO]].
%92 citations counted in INSPIRE as of 17 Jul 2023

\bibitem{Escamilla:2021uoj}
L.~A.~Escamilla and J.~A.~Vazquez,
``Model selection applied to reconstructions of the Dark Energy,''
Eur. Phys. J. C \textbf{83} (2023) no.3, 251
%doi:10.1140/epjc/s10052-023-11404-2
%[arXiv:2111.10457 [astro-ph.CO]].
%13 citations counted in INSPIRE as of 17 Jul 2023

\end{thebibliography}
\end{document}

%\bibliographystyle{plain}
%\bibliography{main}

\end{document}
