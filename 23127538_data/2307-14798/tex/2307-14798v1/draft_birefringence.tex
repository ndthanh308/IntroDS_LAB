\documentclass[aps,reprint,groupedaddress,preprintnumbers,nofootinbib,nobibnotes,amsmath,amssymb]{revtex4-1}



\label{key}
\usepackage{graphicx}
\usepackage{dcolumn}
\usepackage{bm}
\usepackage[usenames]{color}
\usepackage{hyperref}
\usepackage{cleveref}
\usepackage{physics,amsmath}
\usepackage{amssymb}
\usepackage{graphicx}
\usepackage{hyperref}
\usepackage{xcolor}
\usepackage{mathtools}
\usepackage{cancel}
\usepackage{changes}
\usepackage{adjustbox}




%%%%%%%%%%%%%%%%%%%%%%%%%%%%%%%%%%%%%%%%%%%%%%%%%%
\def\lsim{\mathrel{\rlap{\lower4pt\hbox{\hskip1pt$\sim$}} \raise1pt\hbox{$<$}}}
\def\gsim{\mathrel{\rlap{\lower4pt\hbox{\hskip1pt$\sim$}} \raise1pt\hbox{$>$}}}
%%%%%%%%%%%%%%%%%%%%%%%%%%%%%%%%%%%%%%%%%%%%%%%%%%

\renewcommand{\refeq}[1]{Eq.~(\ref{eq:#1})}
\newcommand{\refeqs}[2]{Eqs.~(\ref{eq:#1})-(\ref{eq:#2})}
\newcommand{\reffig}[1]{Fig.~\ref{fig:#1}}


% \newcommand{\sml}[1]{\textcolor{magenta}{ #1}}
% \newcommand{\csml}[1]{\textbf{\color{magenta}[SML: #1]}}
% \definecolor{RedWine}{rgb}{0.743,0,0}
% \renewcommand{\dj}[1]{\textcolor{RedWine}{\textbf{[DJ: #1]}}}
% \newcommand{\dwk}[1]{\textcolor{blue}{\textbf{[DWK: #1]}}}
% \newcommand{\scp}[1]{\textcolor{red}{\textbf{[SCP: #1]}}}



\begin{document}




\preprint{CERN-TH-2022-117}
\preprint{KIAS-P22049}
\preprint{APCTP-Pre2022-015}





\title{
Cosmological Consequences of Kinetic Mixing between Photon and Dark Photon
}


\author{Sung Mook Lee$^{1}$}
\email{sungmook.lee@yonsei.ac.kr}
%\thanks{the first author}
\author{Dong Woo Kang$^{2,3}$}
\email{dongwookang@kias.re.kr}
\author{Jinn-Ouk Gong$^{4,5}$}
\email{jgong@ewha.ac.kr}
\author{Donghui Jeong$^{2,6}$}
\email{djeong@psu.edu}
\author{Dong-Won Jung$^{1}$}
\email{dongwon.jung@yonsei.ac.kr}
\author{Seong Chan Park$^{1,2}$}
\email{sc.park@yonsei.ac.kr}


\affiliation{
$^{1}$Department of Physics \& IPAP \& Lab for Dark Universe, Yonsei University, Seoul 03722, Korea
\\
$^{2}$School of Physics, Korea Institute for Advanced Study, Seoul, 02455, Korea
\\
$^{3}$Theoretical Physics Department, CERN, CH-1211 Gen\`eve 23, Switzerland
\\
$^{4}$Department of Science Education, Ewha Womans University, Seoul 03760, Korea
\\
$^{5}$Asia Pacific Center for Theoretical Physics, Pohang 37673, Korea
\\
$^{6}$Department of Astronomy and Astrophysics and Institute for Gravitation and the Cosmos, The Pennsylvania State University, University Park, PA 16802, USA
}


\date{\today}


\begin{abstract}
%
We study the kinetic mixing between the cosmic microwave background (CMB) photon and the birefringent dark photon as a source of cosmic birefringence.
%
We show that indeed the birefringence of the dark photon  propagates to the CMB photon, but the resulting birefringence may not be uniform over the sky.
%
Moreover, our investigation sheds light on the essential role played by kinetic mixing in the generation of two fundamental characteristics of the CMB: circular polarization and spectral distortion. 
%
\end{abstract}



\maketitle





\section{Introduction}

Although Maxwell's theory of electrodynamics upholds parity as a fundamental symmetry, it can be disrupted by introducing a Chern-Simons coupling with a pseudoscalar field $\theta$~\cite{Carroll:1989vb,Carroll:1991zs,Harari:1992ea}:
%
\begin{align}
{\cal L}_{\rm CS} = g_{\theta} \theta F_{\mu\nu}\tilde{F}^{\mu\nu}\,,
\label{eq:Lcs}
\end{align}
%
where $g_{\theta}$ is a coupling constant, $F_{\mu\nu}\equiv\partial_\mu A_\nu-\partial_\nu A_\mu$ is the field strength tensor, and $\tilde{F}^{\mu\nu}\equiv2\epsilon^{\mu\nu\alpha\beta}F_{\alpha\beta}/\sqrt{-g}$ is the dual strength with the Levi-Civita symbol $\epsilon^{\mu\nu\alpha\beta}$.
%
A popular example for the pseudoscalar field is an axion \cite{Weinberg:1977ma,Wilczek:1977pj} or axion-like particles (ALPs) $a(x)$, for which one can identify $g_{\theta} \theta = g_{a} a/f_{a}$, where $g_{a}$ is a coupling constant of the order of fine-structure constant $\alpha=e^2/(4\pi)$, and $f_{a}$ is the axion decay constant. See e.g. \cite{Choi:2020rgn} for a recent review.  
%
With this interaction, the dispersion relations of two circular polarization modes of the electromagnetic waves differ from each other, i.e. parity is violated. 
%

%
The Thomson scattering on the last scattering surface of the cosmic microwave background (CMB) leads to the linear polarization of the CMB~\cite{Rees:1968,Kosowsky:1994cy}.
%
Thus, the interaction~\refeq{Lcs} yields rotations of the linear polarizations, called ``cosmic birefringence''.
%
When decomposing the angular distribution of the CMB polarization by the $E$-mode (even-parity) and the $B$-mode (odd-parity)~\cite{Zaldarriaga:1996xe,Kamionkowski:1996ks} the $EB$-cross correlation vanishes in the standard $\Lambda$CDM cosmological model~\cite{Lue:1998mq,Gluscevic:2010vv}.
%
Therefore, the detection of the $EB$-cross correlations will be a clear smoking gun of parity-violating new physics beyond the standard model of particle physics (BSM). 
%

%
Interestingly, recent analyses of the CMB have provided a tantalizing hint of the $EB$-cross correlation that is consistent with an isotropic birefringence signal~\cite{Minami:2020odp,Diego-Palazuelos:2022dsq,Eskilt:2022wav,Eskilt:2022cff} due to the interaction~\refeq{Lcs}.
%
Since ALPs may contribute to dark matter and/or dark energy, the observation of parity-violating physics in the polarization of the CMB could represent a significant step toward our understanding of the dark sector~\cite{Komatsu:2022nvu}.
%

%
In this article, motivated by the hints of the parity violation in our universe, we investigate the consequences of the parity violation in the dark sector from an alternative interaction that the photon can participate in: kinetic coupling to other massless U(1) gauge fields.
%
Especially, we assume the new gauge field is completely secluded from the standard model (SM) sector other than the kinetic coupling but is birefringent due to its interactions with dark sector particles.
%


%
The article is organized as follows. 
%
In Section~\ref{section:model}, we begin by reviewing the model of dark photons 
{with the kinetic mixing to SM and modification of the Maxwell equations.}
%
In Section~\ref{section:polarization}, we derive the relation of birefringence in SM and that in the dark photon by considering the polarization tensors of each sector.
%
In Section~\ref{section:implication}, we discuss the implications of our findings and the current and future constraints on the model.
%
Finally, we conclude in Section.~\ref{section:conclusion}.


%%%%%%%%%%%%%%%%%%%%%%%%%%%%%%%%%%%%%%%
\section{Maxwell Equations with Dark Photon Kinetic Mixing}
\label{section:model}
%%%%%%%%%%%%%%%%%%%%%%%%%%%%%%%%%%%%%%%

The model consists of the photon of ${\rm U}(1)_{\rm EM}$ denoted by $ \hat{A}^{\mu} $, and a massless dark photon of a dark ${\rm U(1)}_X$ gauge theory denoted by $\hat{A}_X^{\mu}$, whose Lagrangian density contains the following kinetic terms:
%
\begin{align}
\frac{\mathcal{L}_{\rm kin}}{\sqrt{-g}} 
	 = 
	- \frac{1}{4} \hat{F}^{\mu\nu} \hat{F}_{\mu\nu} - \frac{1}{4} \hat{F}_X^{\mu\nu} \hat{F}_{X \mu\nu} 
	- \frac{\varepsilon}{2}  \hat{F}_{\mu\nu} \hat{F}_X^{\mu\nu} 
    \, ,
\end{align}
%
where $\varepsilon$ is the kinetic mixing coefficient, $\hat{F}_{\mu\nu} \equiv \partial_{\mu} \hat{A}_{\nu} - \partial_{\nu} \hat{A}_{\nu}$, and $\hat{F}_{X\mu\nu} \equiv \partial_{\mu} \hat{A}_{X\nu} - \partial_{\nu} \hat{A}_{X\nu}$ are the field strength tensors.
%
The Lagrangian density is conveniently diagonalized with the following linear transformation:
\begin{align}
\binom{\hat{A}^{\mu}}{\hat{A}_X^{\mu}} 
=\begin{pmatrix}
\dfrac{1}{\sqrt{1-\varepsilon^{2}}} & 0 
\vspace{0.3em}\\
-\dfrac{\varepsilon}{\sqrt{1-\varepsilon^{2}}} & 1
\end{pmatrix}
\binom{A^{\mu}}{A_X^{\mu}} 
\, ,
\end{align}
%
where $ A^{\mu} $ and $A_X^{\mu} $ are, respectively, what we identify as the photon and the dark photon respectively.
%

%
The kinetic mixing changes the interaction Lagrangian, {modifying} the interactions of photon and dark photon with the electric and dark-electric currents:
%
\begin{align}
\frac{{\cal L}_{\rm int}}{\sqrt{-g}} \supset 	& ~e j_{\mu} \hat{A}^{\mu} + e_{X} j_{X \mu} \hat{A}_X^{\mu} \nonumber \\
	& \approx 
	 \left( e j_{\mu} - \varepsilon e_{X} j_{X \mu}	\right) A^{\mu}
	+e_{X} j_{X \mu} A_X^{\mu} \, ,
	\label{eq:massless interaction}
\end{align}
where we take $\varepsilon \ll 1$.
%
Note that the photon couples to the dark current $ j_{X \mu}$ with a coupling proportional to the kinetic mixing parameter $ \varepsilon $, but the dark photon is inert to the SM charged matters.
%
In literature, the coupling between the photon and the dark current is often parameterized as a {\it milli-charge} \cite{Huh:2007zw,Fabbrichesi:2020wbt}
\begin{align}
\epsilon  \equiv - \varepsilon \frac{e_X}{e} \, .
\end{align}
%

%
Indirect constraints to $\epsilon$ come from the milli-charged particle (MCP) searches~\cite{Fabbrichesi:2020wbt, Berlin:2022hmt}. 
%
The constraints from LEP and LHC allow $\epsilon\lesssim 0.1$ for a MCP mass $\in[1,300]~{\rm GeV}$.
%
The future experiments such as FerMINI~\cite{Kelly:2018brz} and milliQAN~\cite{Ball:2016zrp} will cover up to $ \epsilon \lesssim \mathcal{O}\left( 10^{-3} \right) $ {in this mass range}.
%
For a mass range $m_{\rm MCP} \gsim 1~{\rm TeV}$, there hardly exist constraints on the kinetic mixing coming from MCP.
%%
Also, the total energy density of dark photon is bounded at  the time of CMB (and BBN). Explicitly, we request $\rho_{\gamma_{X}}/\rho_\gamma \leq 0.065$ to be consistent with the CMB bounds on $ \Delta N_{\rm eff}$~\cite{Planck:2018vyg}.
%

%
Maxwell's equations for photon and dark photon along with the corresponding currents are
%
\begin{align}
	\nabla_{\mu}F^{\mu\nu} = 4 \pi \left(j^{\nu} + \epsilon j^{\nu}_{X} \right)
	\quad \text{and} \quad 
	\nabla_{\mu}F_X^{\mu\nu} = 4 \pi j_{X}^{\nu} \, .
 \label{eq:new maxwell}
\end{align}
%
Both photon and dark photon satisfy the Bianchi identity: $\partial_{\rho} \tilde{F}_{\mu\nu} =\partial_{\rho} \tilde{F}_{X\mu\nu}=0 $. 
%
%
%
In this article, we assume that the dark photon is birefringent which happens when the dark current $j_X$ is intrinsically parity-violating.
%
A concrete example includes, but is not limited to, the dark current induced from the axion and dark photon Chern-Simons coupling, i.e. $ j_{X}^{\mu} \propto {g_{aX}}(\partial_{\nu} a) \tilde{\hat{F}}_{X}^{\mu\nu}$ with some coupling constant $g_{aX}$.\footnote{
%
Having two photons, we can have three axion couplings,
%
\begin{align*}
 {\cal L}_{\rm int} \supset \frac{a}{f_a} \left[
 c_1 \hat{F}_{\mu\nu} \tilde{\hat{F}}^{\mu\nu} + c_2 \hat{F}_{X\mu\nu}\tilde{\hat{F}}_X^{\mu\nu}
 + c_3 \hat{F}_{\mu\nu}\tilde{\hat{F}}_X^{\mu\nu} \right] \, , 
\end{align*}
% \begin{align}
% {\cal L}\ni \frac{a}{f_a}\left[
% c_1 \hat{F}_{\mu\nu}\tilde{\hat{F}}^{\mu\nu} + c_2 \hat{F}_{X\mu\nu}\tilde{\hat{F}}_X^{\mu\nu}
% +\frac{c_3}{2} (\hat{F}_{\mu\nu}\tilde{\hat{F}}_X^{\mu\nu}+\hat{F}_{X\mu\nu}\tilde{\hat{F}}^{\mu\nu})\right], \nonumber
% \end{align}
where $c_1,c_2$ and $c_3$ are, in principle, independent parameters~\cite{Lee:2015zqz}.
%
In this article, we set $c_2\sim 1$ as the only non-zero parameter, then $c_3\sim \epsilon c_2\sim \epsilon$ and $c_1\sim \epsilon^2 c_2\sim \epsilon^2$ are effectively induced by kinetic mixings after the diagonalization.
%
We note that $c_1$ is responsible for the isotropic birefringence of photon, and $c_3$ affects other observables such as intensity, anisotropic birefringence, or circular polarization.
\label{footnote:axion}
}
%




%%%%%%%%%%%%%%%%%%%%%%%%%%%%%%%%%%%%%
\section{Polarization tensor with birefringent dark photon}
\label{section:polarization}
%%%%%%%%%%%%%%%%%%%%%%%%%%%%%%%%%%%%%

In the expanding universe described by a flat FLRW metric $d s^2=a(\eta)^2 (-d \eta^2 + \delta_{ij} dx^idx^j)$
with a conformal time $d\eta=dt/a(t)$, Eq.~\eqref{eq:new maxwell} implies that 
a linear combination, $ \tilde{A}^{\mu} \equiv A^{\mu} - \epsilon A_X^{\mu}$, propagates freely as a monochromatic wave in the SM vacuum $j^\nu=0$.
%
The photon component $A^{\mu}$ is determined by the monochromatic wave $\tilde{A}^\mu$ and {birefringent wave} $A_X^\mu$.
%
Therefore, the parity-violating effect appears in the visible component if the dark component is parity-violating, even though the $\epsilon$ factor suppresses the effect.
%

%
Without loss of generality, we set the direction of the propagation in the $z$-direction, and the initial amplitude of (partially) linearly polarized photon and dark photon in the $xy$-plane as, respectively,
%
\begin{align}
    \bm{E}_{\gamma, i}^{(p)} &= \sqrt{I_{0} P} \begin{pmatrix}
        1 \\ 0
    \end{pmatrix}\,, \\
    \bm{E}^{(p)}_{X, i} &= \sqrt{I_{X}P_{X}} e^{i\delta_{X}}\begin{pmatrix}
        \cos \alpha \\
        \sin \alpha
    \end{pmatrix}\,,
\end{align}
%
where we allow a phase factor $\delta_X$ and an angle $\alpha$ with respect to the photon.
%
Here, $P ~(P_X)\in [0,1]$ is the degree of polarization of photon (dark photon), and $I_{0}$ $(I_X)$ is the initial intensity of photon (dark photon).
%
In general, they depend on the {direction of the} line of sight $\hat{\bm{n}}$.
%
For a brief review of the polarization theory and related definitions, see Appendix~\ref{app:polarization}.
%

%
When the dark photon propagates through a birefringent medium (a time-varying axion medium, for instance), the polarization vector evolves into an emergent state: 
%
\begin{align}
\bm{E}_{X,i}^{(p)} \to  \bm{E}_{X}^{(p)} &= \hat{U}(\beta_X) \bm{E}_{X, i}^{(p)}
\nonumber\\
&=\sqrt{I_{X}P_{X}} e^{i \delta_{X}}\begin{pmatrix}
  \cos(\alpha + \beta_{X}) \\ \sin(\alpha + \beta_{X})
\end{pmatrix},
\end{align}
%
with the rotation matrix 
$\hat{U}(\beta_X)=\begin{pmatrix}
\cos\beta_X & -\sin\beta_X  \\
\sin\beta_X &  \cos\beta_X \\
\end{pmatrix}$ induced by dark birefringence.
%
A concrete example is the axion coupling to dark photon, that generates $\beta_{X}\propto g_{aX} \int_{\eta_{i}}^{\eta} d\eta^{\prime} \frac{da}{d\eta}(\eta^{\prime})$.
%
Schematic pictures of our setup and the birefringences of photon and dark photon are depicted in Figure~\ref{fig:schematic}.
%


%%%%%%%%%%%%%%%%%%%%%%%%%%%%%%%%%%%%%%%%%%%%%%%%%%%%%%%%%%%

% Figure environment removed

%%%%%%%%%%%%%%%%%%%%%%%%%%%%%%%%%%%%%%%%%%%%%%%%%%%%%%%%%%%


%
The polarization tensor is given for the partially polarized dark photon:
\begin{align}
    \rho_{X}= 
    \frac{1}{2} \begin{pmatrix}
    1 + P_{X} \cos (2\alpha + 2\beta_X)  & P_{X} \sin   (2\alpha + 2\beta_X)  \\
        P_{X} \sin  (2\alpha + 2\beta_X) &  1 - P_{X} \cos (2\alpha + 2\beta_X)
\end{pmatrix}
\end{align}
%
with the Stokes parameters being
$Q_{X} / I_{X} =\rho_{X11}-\rho_{X22}$, $U_{X} / I_{X} =\rho_{X12}+\rho_{X21}$ and $ V_{X} / I_{X} = i(\rho_{X12}-\rho_{X21})=0$.
%
We note that no circular polarization is generated from the birefringence. 
%
Because $\tilde{\bm{E}} = \bm{E} - \epsilon \bm{E}_X$ freely propagates, the Jones matrix constructed with this degree is time-independent, i.e. $\tilde{J}_{\alpha\beta}(t) \equiv \left\langle \tilde{E}_\alpha \tilde{E}^{*}_\beta \right\rangle_T =\tilde{J}_{\alpha\beta}(0)$.
%
Here, $\langle \, \cdots \, \rangle_{T}$ means taking an average over a time interval $T \gg \omega^{-1} $ where $\omega $ is the frequency of the oscillation.
%
As $\langle \tilde{E}\tilde{E}\rangle_T = \langle (E-\epsilon E_X)(E-\epsilon E_X)\rangle_T = \langle EE\rangle_T-\epsilon \langle EE_X +E_XE\rangle_T +\epsilon^2 \langle E_XE_X\rangle_T$, the photon polarization $\langle EE\rangle_T = \langle \tilde{E}\tilde{E}\rangle_T+ \epsilon \langle EE_X+E_XE\rangle_T +{\cal O}(\epsilon^2)$ evolves with $\epsilon$.
%
Explicitly, by subtracting the values at $t>0$ and $t=t_{\rm ini}=0$, we have
%
\begin{widetext}
    \begin{align}
    J(t) = J(0)+ \epsilon \sqrt{I_0 I_{X}} \sqrt{P P_{X}} \begin{pmatrix}
        - 4 \cos \delta_{X} \sin \left( \alpha + \dfrac{\beta_{X}}{2} \right) \sin \left( \dfrac{\beta_{X}}{2} \right) & 2 e^{-i \delta_{X}} \cos \left( \alpha + \dfrac{\beta_{X}}{2} \right)   \sin \left( \dfrac{\beta_{X}}{2} \right) 
        \vspace{0.3em}\\
        2 e^{i \delta_{X}} \cos \left( \alpha + \dfrac{\beta_{X}}{2} \right)   \sin \left( \dfrac{\beta_{X}}{2} \right) & 0 
    \end{pmatrix} + \mathcal{O}(\epsilon^{2})
    \, ,
\end{align}
\end{widetext}
where $J_{\alpha\beta}(t)=\left\langle E_\alpha E_\beta^*\right\rangle_{T}(t)$, and the initial tensor, given by the initial condition of the photon, is 
\begin{align}
    J_{\alpha\beta}(0)
=\dfrac{1}{2} I_0 \begin{pmatrix}
      1+P & 0 \\
        0 & 1-P
        \end{pmatrix} \equiv I_{0} \rho_{0} \, .
\end{align}   
%


%
As a direct consequence, we find that photon intensity changes inducing spectral distortion of ${\cal O}(\epsilon)$, the corresponding polarization tensor $\rho = J/I$ and the Stokes parameters for the photon are given, respectively, as
%
\begin{widetext}
    \begin{align}
      %\begin{aligned}
        \Delta I&=I-I_0= {\rm Tr}( J(t)-J(0) )   
        = - 4 \epsilon \sqrt{I_0 I_X P P_X}
         \cos \delta_{X} 
        \sin \left( \alpha + \frac{\beta_{X}}{2} \right) \sin \left( \frac{\beta_{X}}{2} \right) + \mathcal{O}(\epsilon^{2}) \, ,
   \label{eq:intensity}
    \end{align}
    \begin{equation}
    \begin{aligned}
    \rho 
    %& = \frac{J}{I} \\
    & = \rho_{0}
    - 2 \epsilon \sqrt{\frac{I_{X}}{I_{0}}} \sqrt{P P_{X}}
    \begin{pmatrix}
        (1-P) \cos \delta_{X} \sin \left( \alpha + \dfrac{\beta_{X}}{2} \right) \sin \left(\dfrac{\beta_{X}}{2}\right) & e^{-i \delta_{X}} \cos \left( \alpha + \dfrac{\beta_{X}}{2} \right)   \sin \left(\dfrac{\beta_{X}}{2}\right) 
        \vspace{0.3em}\\
         e^{i \delta_{X}} \cos \left( \alpha + \dfrac{\beta_{X}}{2} \right)   \sin \left(\dfrac{\beta_{X}}{2}\right) & - (1-P) \cos \delta_{X} \sin \left( \alpha + \dfrac{\beta_{X}}{2} \right) \sin \left(\dfrac{\beta_{X}}{2}\right)
    \end{pmatrix} +{\cal O}(\epsilon^2) \, ,
    \label{eq:polarization}
    \end{aligned}
    \end{equation}
%
and
%
\begin{equation}
\begin{split}
    \frac{Q}{I} & = \rho_{11} - \rho_{22} = P - 4 \epsilon (1-P) \sqrt{\frac{I_{X}}{I_{0}}} \sqrt{P P_{X}} \cos \delta_{X} \sin \left( \alpha + \frac{\beta_{X}}{2} \right) \sin \left( \frac{\beta_{X}}{2} \right)  + \mathcal{O}(\epsilon^{2})\,,  \\
    \frac{U}{I} & = \rho_{12} + \rho_{21} = 4 \epsilon  \sqrt{\frac{I_{X}}{I_{0}}} \sqrt{P P_{X}} \cos \delta_{X} \cos \left( \alpha + \frac{\beta_{X}}{2} \right) \sin \left( \frac{\beta_{X}}{2} \right)  + \mathcal{O}(\epsilon^{2})\,, \\
    \frac{V}{I} & = i (\rho_{12} - \rho_{21} ) = 4 \epsilon  \sqrt{\frac{I_{X}}{I_{0}}} \sqrt{P P_{X}} \sin \delta_{X} \cos \left( \alpha + \frac{\beta_{X}}{2} \right) \sin \left( \frac{\beta_{X}}{2} \right) + \mathcal{O}(\epsilon^{2})\, . 
\end{split}    
\label{eq:stokes}
\end{equation}
\end{widetext}
%


Change of intensity and polarization tensor are the key results of this article. 
%
Our results explicitly suggest that a birefringence effect in the photon could be induced by the polarization of the dark photon. 
%



%%%%%%%%%%%%%%%%%%%%%%%%%%%%%%%%%%
\section{Observational Implications}
\label{section:implication}
%%%%%%%%%%%%%%%%%%%%%%%%%%%%%%%%%%

%
In this section, we will discuss the possible observational implications. 
%
For a definite case study, let us assume that the phase space distribution of dark photons follows the thermal (Planck) equilibrium at the time of CMB decoupling time with a dark temperature $ T_{X} \equiv r T_{\gamma}$, which is different from the CMB temperature $T_\gamma$ in general by a factor $r$. To satisfy the $N_{\rm eff}$ constraint from Planck \cite{Planck:2018vyg}, $r$ must be smaller than 0.4 \cite{Gurian:2021qhk}.
%
This may be the case when the dark photons were in thermal equilibrium with dark matter at an earlier time, before the dark recombination and dark decoupling \cite{Kaplan:2009de,Fan:2013yva,Gurian:2021qhk}.
%
After the decoupling, the dark photon would freely stream while keeping in the phase space the Planck distribution function with reduced temperature.
%
Note that, to avoid the constraints coming from the lack of dark acoustic oscillation \cite{Cyr-Racine:2013fsa}, the dark decoupling should happen well ahead of the cosmic recombination at $z\simeq 1100$ \cite{Gurian:2021qhk}.
%

\subsection{Spectral distortion}

%
The spectral distortion is given in Eq.~\eqref{eq:intensity}.
%
While the spatial average of the random distortion vanishes, the statistical dispersion does not even though its size is bounded from above by the kinetic mixing angle $\epsilon$ as long as $I_X < I_0$:
%
\begin{align}
	\frac{\delta I}{I_{0}} \simeq 2 \epsilon \sqrt{\frac{I_{X}}{I_0}} \sqrt{\bar{P} \bar{P}_{X}} \left\vert \sin  \left( \frac{\beta_{X}}{2} \right) \right\vert \lesssim 2\epsilon \, ,
 \label{Eq:SpectralDistortion}
\end{align}
where $\bar{P}_{(X)} \equiv \sqrt{\left\langle P_{(X)}^{2} \right\rangle}$ with $\langle \, \cdots \, \rangle $ taking an ensemble average over the sky. 
Since 
$ I
\propto {k^{3}}/\big[{e^{k/(2 \pi T)}-1}\big]$ 
for blackbody photons, we find 
%
\begin{align}
	\frac{I_{X}}{I_{0}}
	=
	\begin{dcases}
		r & (k \ll T_{\rm X}) \\
		\exp\left( - \frac{1-r}{r} \frac{k}{2 \pi T_{\rm \gamma}} \right) & (k \gg T_{\rm X})
	\end{dcases}\,.
	\label{eq:kdep}
\end{align}
%
This characteristic frequency dependence can be a smoking gun signal of the kinetic mixing.
%
At high frequencies with $\hbar \omega\gg 3k_B T_{X}$ in the Wien tail, the intensity of dark photon is suppressed thus the effect on the CMB polarization is minuscule. 
%




\subsection{Birefringence}



Non-zero value of $U$ in Eq.~\eqref{eq:stokes} implies that there exists birefringence in CMB as 
\begin{align}
   % \begin{aligned}
          &\beta (\hat{\bm{n}}) = \frac{1}{2} \arctan \left(\frac{U}{Q}\right) %\simeq \frac{1}{2} \frac{U}{Q} 
    \\  &= 2 \epsilon \sqrt{ \frac{I_{X} P_{X}}{I_0 P}} \cos \delta_{X} \cos \left( \alpha + \frac{\beta_{X}}{2} \right) \sin \left( \frac{\beta_{X}}{2} \right) +{\cal O}(\epsilon^2)\, . \label{eq:birefringence}
    %\end{aligned}
    \nonumber
\end{align}
%
Note that, in general, the angle $\alpha$ can take any values. For example, if dark recombination \cite{Gurian:2021qhk} happened, as for the atomic dark matter model \cite{Kaplan:2009de}, the linear polarization of dark photons would be determined by local quadrupole at the dark recombination time, which must be much earlier than the cosmic recombination time \cite{Cyr-Racine:2013fsa}.
%
As a result, $\alpha$ is essentially random, and this sets the expectation value for the monopole, a constant isotropic birefringence angle $\beta_{\rm iso} \equiv \langle \beta( \hat{\bm{n}} ) \rangle $ over the sky, to appear at most $\mathcal{O}(\epsilon^2)$. Hence, for the birefringence, we are mostly interested in the anisotropic part of the birefringence, $\beta_{\rm aniso} \equiv \beta( \hat{\bm{n}} ) - \langle \beta \rangle $.
%
From the same reason, no sizable correlation is expected for the birefringence angle measured in two different directions, i.e $\langle \beta (\hat{\bm{n}}) \beta (\hat{\bm{n}}^{\prime}) \rangle \simeq 0$ for $\hat{\bm{n}} \neq \hat{\bm{n}}^{\prime} $.
%
However, we expect to have non-zero variance as much as
\begin{align}
\left\langle \beta^{2}_{\rm aniso} \right\rangle 
&  \simeq \epsilon^2 \frac{I_{X}}{I_{0}} \left\langle \frac{P_{X}}{P} \right\rangle  \sin^{2} \left( \frac{\beta_{X}}{2} \right)\, + \mathcal{O}(\epsilon^{2})\,.
\end{align}

In the case of $\beta_{\rm iso} = 0$, we have no $EB$-correlation since $C_\ell^{EB}\propto \sin(4\beta_{\rm iso})$. However, there still exist some observables which non-zero $\left\langle\beta_{\rm aniso}^2 \right\rangle$ affects.
%
For example, $TE$-correlation is corrected from the existence of the variance as \cite{Li:2008tma,Gluscevic_2012}
\begin{align}
    C_{\ell}^{TE} 
    \rightarrow
    %C_{\ell}^{TE} \langle \cos 2 \beta \rangle \simeq  
    C_{\ell}^{TE} \cos (2\beta_{\rm iso}) 
    \left( 1- 2 \langle \beta_{\rm aniso}^{2} \rangle \right) \, .
\end{align}
%
Similarly, $C^{EE}_\ell$- and $C^{BB}_\ell$-correlations, and higher order correlations  $C^{EBEB}_\ell$ are given as functions of $\beta_{\rm aniso}$ and provide measureable tests.
%
The current bound for the variance of birefringence is $ \langle \beta_{\rm aniso}^{2} \rangle \lesssim 3 \times 10^{-4}$~\cite{Pan:2016vai,Mei:2014iaa}.
%



\subsection{Circular polarization}

The non-vanishing circular polarization, called the CMB $V$-mode, is predicted in Eq.~\eqref{eq:stokes}:
%
\begin{align}
\langle V^{2} \rangle \simeq 4 \epsilon^{2} I_{0} I_{X}  \bar{P} \bar{P}_{X} \sin^{2} \left( \frac{\beta_{X}}{2} \right).
\end{align}
%
We emphasize that the non-vanishing $V$ is a characteristic feature of our model with kinetic mixing, which is not present in the case of the direct coupling of the photon and the axion background. We notice the current bound on circular polarization is obtained at the level of $\mathcal{O}(10 \mu \text{K})$~\cite{SPIDER:2017kwu}.
%


%%%%%%%%%%%%%%%%%%%%%%%%%%%%%%%%%%%%%%%%%
\section{Conclusions}
\label{section:conclusion}
%%%%%%%%%%%%%%%%%%%%%%%%%%%%%%%%%%%%%%%%%

Recent CMB observations have hinted at the cosmic birefringence in tantalizing 3.6-$\sigma$ level. In literature, the birefringence is usually attributed to the direct coupling between photons and a pseudoscalar field, like an axion. Here, we investigate the lowest-order alternative to that explanation by using the kinetic mixing between the photon and the dark photon.


%
Our main results are encapsulated in Eq.~\eqref{eq:intensity} and Eq. \eqref{eq:polarization}, which represent, respectively, the intensity and polarization tensor of the photon kinetically mixed to the birefringent dark photon.
%
We find that the kinetic mixing not only transfers the dark photon's birefringence to the CMB photon, but also yields the unique, distinctive features such as direction-dependent spectral distortion and circular polarization.
%
The birefringence from the kinetic mixing is anisotropic birefringence with non-zero variance. 


Finally, we acknowledge that exploring a more concrete model of parity violation in the dark sector would open up broader theoretical possibilities and potentially uncover new avenues for observations of BSM physics.
%
We leave these tasks to future work.


\vspace{1.0cm}




\acknowledgments

We thank Gongjun Choi, Johannes Eskilt, Kunio Kaneta, Suro Kim, Kazunori Kohri and Marko Simonovic for helpful discussions. 
%
This work is supported in part by the National Research Foundation grants 2019R1A2C2085023 (JG), 2021R1A2B5B02087078, 2021R1A2C2011003 and 
RS-2023-00246268 (DWJ), 2019R1A2C1089334 and 2021R1A4A2001897 (SCP).
%
SML was also supported in part by the Hyundai Motor Chung Mong-Koo Foundation Scholarship, and the Korea-CERN Theoretical Physics Collaboration and Developing Young High-Energy Theorists Fellowship Program (NRF-2012K1A3A2A0105178151).
%
DWK is supported in part by KIAS Individual Grant PG076202.
%
JG is further supported in part by the Korea-Japan Basic Scientific Cooperation Program supported by the National Research Foundation of Korea and the Japan Society for the Promotion of Science (NRF-2020K2A9A2A08000097), and
the Ewha Womans University Research Grant of 2022 (1-2022-0606-001-1) and 2023 (1-2023-0748-001-1).
%
DJ is supported by KIAS Individual Grant PG088301 at Korea Institute for Advanced Study.
%
DWJ was supported in part by the Yonsei University Research Fund(Post Doc. Researcher Supporting Program) of 2022 (project no.: 2022-12-0035).
%
SML and DWK are grateful to the CERN for hospitality while this work was under progress.
%
JG is also grateful to the Asia Pacific Center for Theoretical Physics for their hospitality while this work was in progress.
%
JG and DJ thank the Yukawa Institute for Theoretical Physics at Kyoto University, where this work was completed during the YITP-T-23-03 ``Revisiting cosmological non-linearities in the era of precision surveys.''



\appendix

%%%%%%%%%%%%%%%%%%%%%%%
\section{Review of (Partially) Polarized Light}
\label{app:polarization}
%%%%%%%%%%%%%%%%%%%%%%%

In this appendix, we briefly review the theory of partially polarized light. Especially, we will explicitly show how the polarization tensor is defined and set the notations. This part mainly relies on Ref.~\cite{Landau:1975pou}.
%

%
A electric field at a fixed position $\bm{x} = 0$ is given as 
$ \bm{E} (t) e^{- i \omega t} $ for a fixed $\bm{k}$ with $\vert \bm{k} \vert = \omega$.
%
Here, $\bm{E}$ can have a time dependence in general and is decomposed into polarized part $(p)$ and unpolarized (or natural) part $(n)$ as
\begin{align}
    \bm{E} = \bm{E}^{(p)} + \bm{E}^{(n)} \, . 
\end{align}
%
The unpolarized part satisfies
\begin{align}
    \left\langle E_{\alpha}^{(n)} E_{\beta}^{(n) *} \right\rangle_{T} = \frac{1}{2} I^{(n)} \delta_{\alpha\beta}
    \, ,
\end{align}
%
where $\langle \, \cdots \, \rangle_{T}$ is an average over a time interval $T$ much larger than $\omega^{-1}$.
%\footnote{In quantum mechanical language, the unpolarized part is the maximally mixed state.}
%
Here, $I^{(n)}$ is the intensity of the unpolarized part of the electric field. On the other hand, the polarized part is assumed to be nearly constant compared to the time scale of the average.
%
With this decomposition, we define the Jones matrix
\begin{align}
    J_{\alpha\beta} \equiv \left\langle E_{\alpha} E_{\beta}^{*} \right\rangle_{T}
    = E_{\alpha}^{(p)} E_{\beta}^{(p)*}  + \frac{1}{2} I^{(n)} \delta_{\alpha\beta}
    \,,
\end{align}
the intensity $ I \equiv \text{Tr} \, J $ and polarization tensor $\rho_{\alpha\beta} \equiv J_{\alpha\beta} / I $.
%
If a quantity is slowly varying in a much larger time scale than the time scale of averaging, there may be residual time dependence after averaging fast modes.
%
This also includes the observational effects we discuss in the main text arouse from the slow birefringence of the dark photon $\beta_{X}(t)$.
%

%
In terms of Stokes parameters, the polarization tensor is written as
\begin{align}
    \rho = \frac{1}{2 I} \begin{pmatrix}
        I + Q & U - i V \\
      U + i V  & I - Q
    \end{pmatrix}
    \, .
\end{align}
%
We also introduce the degree of the polarization $P \in [0,1]$ as $\det \rho \equiv (1-P^{2})/4$ where $P = 0$ corresponds to the unpolarized light, and $P=1$ is for the completely polarized one.
%
With these definitions, $I^{(n)} = I (1-P)$, and $I^{(p)} \equiv \vert \bm{E}^{(p)} \vert^{2} = I P $. Also, $P = \sqrt{Q^{2} + U^{2} + V^{2}} / I$.
%

\newpage

\bibliography{draft_birefringenceNotes.bib}





\end{document}

