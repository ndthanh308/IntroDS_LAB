%\naman{Informally dumping a bunch of papers right now....}
Automatic program repair is an active area of research. 
We refer the reader to~\cite{survey1, survey2} for broad survey. Below, we list and compare related works that use static analysis as well as other approaches.\\
\textbf{Static Analysis Based Repair.} 
Systems such as \othersysname{FootPatch}~\cite{FootPatch} and \othersysname{SenX}~\cite{senx} utilize static-analysis-information to create repair strategies. Specifically, \othersysname{FootPatch} reasons about semantic properties of programs and \othersysname{SenX} determines safety properties being violated to generate the patches. 
Unlike these approaches, \sysname  uses static-analysis information (namely witnesses) to generate a paired unsafe-safe dataset of programs and then uses program synthesis to learn repair strategies from the dataset.
Prior work uses static analysis to detect bugs, and then use cross-commit-data from manual fixes to create paired unsafe-safe dataset of programs, and learn repair strategies.
For instance,
\othersysname{SpongeBugs}~\cite{sonarcube} uses \othersysname{SonarCube}~\cite{sonarsa} static analysis to find bugs, and cross-commit data to create paired dataset.
Similarly,~\othersysname{Avatar}~\cite{Liu:mining,liu2019avatar} uses \othersysname{FindBugs}~\cite{findbugs} static analysis and cross-commit data.
\othersysname{GetAFix}~\cite{bader2019getafix} similarly mines general tree-edit-patterns from cross-commit data using anti-unification.
However, these fix-templates or edit patterns are purely syntactic whereas our repair strategies use semantic knowledge 
 (specifically data flow) of programs, which is necessary for fixing information flow vulnerabilities. 
 The \othersysname{Phoenix} tool~\cite{bavishi2019phoenix} makes more use of semantic information, and is closest to our approach in terms of learning repair strategies. However, we use \sawitness to learn repair strategies from a single snapshot of codebases whereas all previous approaches including \othersysname{Phoenix} run \sa across all historical commits of the repository to get paired cross-commit data, which is inherently noisy. Getting clean paired data from commits for bug fixes is a difficult problem, and we avoid this problem entirely. Additionally, our \dsl supports \kleeneedge based operators that allow learning general repairs across examples where data flow paths have variable lengths, which is  not supported by \othersysname{Phoenix}.  

\noindent \textbf{Other Automated Program Repair approaches.} 
\othersysname{Blade}~\cite{blade} and \othersysname{Lifty}~\cite{lifty} repair information-leaks in programs using type analysis. \othersysname{HyperGI}~\cite{hypergi} performs repair on information-leak bugs using test suites.
\othersysname{Refazer}~\cite{rolim2017learning,zhang2022overwatch} learns program transformations from developer edits using program synthesis. \othersysname{VurLe}~\cite{vurle} and \othersysname{Seader}~\cite{exampleBasedJava} both learn program repairs from examples. \othersysname{CDRep}~\cite{cdrep} proposes an approach to repair speculative leaks from cryptographic code. However, the approach requires users to manually write repair templates for cryptographic APIs. \othersysname{BovInspector}~\cite{bovinspector} implements guard templates for fixing buffer overflow in \othersysname{C} programs. Automated program repair techniques have used mutations of buggy programs to pass test cases in a suite ~\cite{GenProg:ICSE2012, ACS:ICSE2017, SearchRepair:ASE2015, ssFix:ASE2017, genesis, Prophet}. More recently, machine learning-based techniques have also started to gain attention for performing repair~\cite{graphbased, allamanis2021self, yasunaga2021break}. Our neural baselines emulate the recent advancements in neural large language models for code-generation, repair, etc. ~\cite{LLMrepair, Xia2022PracticalPR}.


% \noindent \url{https://escholarship.org/content/qt0t20j69d/qt0t20j69d_noSplash_0092dec083579a52a5ba8289bccccb31.pdf} (type based line of work) (its predecessors) ; \url{https://cseweb.ucsd.edu/~dstefan/pubs/vassena:2021:blade.pdf} (type based leak fixing ) ; \url{https://ieeexplore.ieee.org/stamp/stamp.jsp?tp=&arnumber=9678758}

%\noindent Finally, we use \pbe approach to learn repair strategies. We defer the readers to ~\cite{synthesis-now} for a summary of the  techniques involved. 