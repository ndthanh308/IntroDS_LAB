\input{limitationex.tex}
\lstMakeShortInline[columns=fixed]@

We propose a novel approach to use static analysis and a repository of correct programs that satisfy a property, to automatically learn  strategies to repair programs that violate the property.
We have implemented our approach in the \sysname system.
We evaluate our approach by performing repairs on two specific \js vulnerabilities (unvalidated dynamic call and cross-site scripting) and learn general repair strategies. These repair strategies are able to automatically repair over 90\% of the violations of these properties we found in over 1000 files collected from open-source repositories. 


%With some minor adaptations, we believe that we can also learn repair strategies that judiciously introduce "prepared" statements~\cite{sqlprevention} and  prevent SQL Injection. 

\out{
Our work makes two distinct contributions: (1) to use static analysis witnesses to learn paired datasets, (2) new program synthesis techniques (including novel domain specific language, strategy learning methods, etc) to learn non-local strategies from these datasets. We can potentially use the datasets collected from our approach to train or fine-tune better repair strategies using neural networks~\cite{Codex,Austin}, even though the synthesis approach outperforms neural baselines (see Section 6). 
}
Our approach has two known limitations that can be potentially addressed in future work. 
The first limitation is due to our current implementation architecture.
While our \astree implementation can trace data flows across method boundaries, our \astree is limited to a single file.
A better \astree builder would allow \sysname{} to repair flow vulnerabilities that cross file boundaries.
The second limitation is a conceptual one.
In addition to introducing sanitizers and guards judiciously, repairing information flow violations may also require application-specific side-effect handling, which is beyond the scope of this paper. For example, in Figure~\ref{fig:imperfect-fix1}, the guard blocks the dynamic execution of the function call to avoid the vulnerability. However, real-world fixes would also require appropriate error handling for the "else" branch, such as sending a suitable error message. The repair shown in Figure~\ref{fig:imperfect-fix2} suffers from the same issue, where it terminates function execution via a @return@ without any error message or returning an error value. We imagine a human-in-the-loop repair process where our \sysname suggests the repair witnesses and human reviewers judge the repairs and additionally handle context-specific side effects such as error handling. We also envision a neuro-symbolic program repair system where the broader application context is \textit{predicted} by a neural model like Codex~\cite{Codex} as a future direction.
% We need to use more flexible implementation architecture to be able to build and analyze \astree that crosses file boundaries, to be able 

Though we have evaluated our approach on two specific instances information flow properties, our approach has the potential to repair many classes of information-flow vulnerabilities such as null-dereferencing~\cite{nullowasp}, zip-slips~\cite{zipslipcodeql}, tainted-path~\cite{taintedpathcodeql}, SQL and Code Injection. 