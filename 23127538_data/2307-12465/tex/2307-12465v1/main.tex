\pdfoutput=1
%%
%% This is file `sample-acmsmall-conf.tex',
%% generated with the docstrip utility.
%%
%% The original source files were:
%%
%% samples.dtx  (with options: `acmsmall-conf')
%% 
%% IMPORTANT NOTICE:
%% 
%% For the copyright see the source file.
%% 
%% Any modified versions of this file must be renamed
%% with new filenames distinct from sample-acmsmall-conf.tex.
%% 
%% For distribution of the original source see the terms
%% for copying and modification in the file samples.dtx.
%% 
%% This generated file may be distributed as long as the
%% original source files, as listed above, are part of the
%% same distribution. (The sources need not necessarily be
%% in the same archive or directory.)
%%
%%
%% Commands for TeXCount
%TC:macro \cite [option:text,text]
%TC:macro \citep [option:text,text]
%TC:macro \citet [option:text,text]
%TC:envir table 0 1
%TC:envir table* 0 1
%TC:envir tabular [ignore] word
%TC:envir displaymath 0 word
%TC:envir math 0 word
%TC:envir comment 0 0
%%
%%
%% The first command in your LaTeX source must be the \documentclass
%% command.
%%
%% For submission and review of your manuscript please change the
%% command to \documentclass[manuscript, screen, review]{acmart}.
%%
%% When submitting camera ready or to TAPS, please change the command
%% to \documentclass[sigconf]{acmart} or whichever template is required
%% for your publication.
%%
%%
\documentclass[acmsmall,screen]{acmart}

%%
%% \BibTeX command to typeset BibTeX logo in the docs
\AtBeginDocument{%
  \providecommand\BibTeX{{%
    Bib\TeX}}}


%% Conference information
%% Supplied to authors by publisher for camera-ready submission;
%% use defaults for review submission.
% \acmConference[]{}{}{}
% \acmYear{}
% \acmISBN{} % \acmISBN{978-x-xxxx-xxxx-x/YY/MM}
% \acmDOI{} % \acmDOI{10.1145/nnnnnnn.nnnnnnn}
\startPage{1}

%% Copyright information
%% Supplied to authors (based on authors' rights management selection;
%% see authors.acm.org) by publisher for camera-ready submission;
%% use 'none' for review submission.
\setcopyright{none}
%\setcopyright{acmcopyright}
%\setcopyright{acmlicensed}
%\setcopyright{rightsretained}
%\copyrightyear{2018}           %% If different from \acmYear
\renewcommand\footnotetextcopyrightpermission[1]{} % removes footnote with conference information in first column
\pagestyle{plain} % removes running headers

\makeatletter
\renewcommand\@formatdoi[1]{\ignorespaces}
\makeatother

\def\firstfoot{\def\@firstfoot{}}
\def\firsthead{\def\@firsthead{}}

%% Bibliography style
\bibliographystyle{ACM-Reference-Format}
%% Citation style
%\citestyle{acmauthoryear}  %% For author/year citations
%\citestyle{acmnumeric}     %% For numeric citations
%\setcitestyle{nosort}      %% With 'acmnumeric', to disable automatic
                            %% sorting of references within a single citation;
                            %% e.g., \cite{Smith99,Carpenter05,Baker12}
                            %% rendered as [14,5,2] rather than [2,5,14].
%\setcitesyle{nocompress}   %% With 'acmnumeric', to disable automatic
                            %% compression of sequential references within a
                            %% single citation;
                            %% e.g., \cite{Baker12,Baker14,Baker16}
                            %% rendered as [2,3,4] rather than [2-4].


%%%%%%%%%%%%%%%%%%%%%%%%%%%%%%%%%%%%%%%%%%%%%%%%%%%%%%%%%%%%%%%%%%%%%%
%% Note: Authors migrating a paper from traditional SIGPLAN
%% proceedings format to PACMPL format must update the
%% '\documentclass' and topmatter commands above; see
%% 'acmart-pacmpl-template.tex'.
%%%%%%%%%%%%%%%%%%%%%%%%%%%%%%%%%%%%%%%%%%%%%%%%%%%%%%%%%%%%%%%%%%%%%%


%% Some recommended packages.
\usepackage{booktabs}   %% For formal tables:
                        %% http://ctan.org/pkg/booktabs
\usepackage{subcaption} %% For complex figures with subfigures/subcaptions
                        %% http://ctan.org/pkg/subcaption
\usepackage{}
\usepackage{listings}
\usepackage{xspace}
\usepackage[utf8]{inputenc}
\usepackage[T1]{fontenc}
\usepackage{xcolor}
\usepackage[normalem]{ulem}
\usepackage{footnote}
\usepackage{tablefootnote}
\usepackage{pdfpages}
\usepackage{multicol}
\usepackage{multirow}
\usepackage{enumitem}
\usepackage{lstautogobble}
\usepackage{tikz}
\usetikzlibrary{calc}
\usepackage[shortcuts]{extdash}
\usepackage{amsmath}
\usepackage{newtxmath}
\usepackage{adjustbox}
\usepackage{syntax} 
\usepackage{bold-extra}
\usepackage{ifmtarg}% http://ctan.org/pkg/ifmtarg


\usepackage{alltt}
\usepackage{algorithmicx}
\usepackage{algorithm} % http://ctan.org/pkg/algorithms
\usepackage[noend]{algpseudocode}
\usepackage{alltt}
%\usepackage[dvipsnames]{xcolor}

%%%%%%%%% my function definitions formatting
\algblockdefx[myFuncBlock]{myFuncStart}{myFuncEnd}%
[2]{\underline{\textsc{#1}} (#2)}%
[0]{}

\makeatletter
\ifthenelse{\equal{\ALG@noend}{t}}%
  {\algtext*{myFuncEnd}}
  {}%
\makeatother


\newcommand{\myFunctwo}[2]{\myFuncStart{#1}{#2}\myFuncEnd} 
\newcommand{\myFunc}[2]{\Statex \underline{\textsc{#1}} (#2)} 

%%%%%%%%%% 
\newcommand{\emptylist}{empty \textbf{list}}
\newcommand{\emptydict}{empty \textbf{dict}}
\newcommand{\algout}[1]{\textbf{out} #1}

%%%%%%%%%% my (attempt) at for loop
\algloopdefx{myFor}[2]{\textbf{for each} $#1 \in #2$:}

\algcblockdefx[myForBlock]{myFor}{Eeee}{myForEnd}
[1]{\textbf{Eeee} "#1"}
{}


%%%%%%%%%
\renewcommand\algorithmicthen{:}
\renewcommand\algorithmicdo{:}

\makeatletter
\newcommand\fs@ruled@notop{\def\@fs@cfont{\bfseries}\let\@fs@capt\floatc@ruled
  %\def\@fs@pre{\hrule height.8pt depth0pt \kern2pt}% <----removed
  \def\@fs@pre{}%
  \def\@fs@post{}%
  \def\@fs@mid{}%
  %\def\@fs@post{\kern2pt\hrule\relax}%
  %\def\@fs@mid{\kern2pt\hrule\kern2pt}%
  \let\@fs@iftopcapt\iftrue}
\renewcommand\fst@algorithm{\fs@ruled@notop}
\makeatother



\usepackage{hyperref}

\usepackage{xspace}
\usepackage{tikz}
\usetikzlibrary{arrows.meta}
\usepackage{adjustbox}
\usepackage{alltt}
\usepackage{fancyvrb}
%\usepackage{minted}
\usepackage{mathtools}
\usepackage{bold-extra}
\usepackage{lipsum}
%\usepackage{cite}
\usepackage{lineno}
\usepackage{proof}
\usepackage{mathpartir}

\usepackage{highlighter}
\usepackage[framemethod=TikZ]{mdframed}


\newcommand\calF{\mathcal{F}}
\newcommand\calG{\mathcal{G}}
\newcommand\calM{\mathcal{M}}
\newcommand\calV{\mathcal{V}}
\newcommand\calU{\mathcal{U}}
\newcommand\calW{\mathcal{W}}
\newcommand\calP{\mathcal{P}}
\newcommand\calD{\mathbb{D}}
%%%%%%%%%%%%%%%%%
%% macros introduced by Luke 
\newcommand\mydef[1]{{\bf\em #1}}
%%%%%%%%%%%%%%%%%

\newcommand{\numviparams}{{| \lambda |}}
\newcommand{\scoreaccvars}[1]{s_1^{#1}, \ldots, s_{\numviparams}^{#1}}
\newcommand{\scoreaccvar}[2]{s_{#1}^{#2}}
\newcommand{\isdeterm}[1]{\text{Deterministic}({#1})}


\newcommand{\expect}[1]{\mathbb{E}\left[{#1}\right]}
\newcommand{\var}[1]{\mathbb{V}\left[ {#1} \right]}
\newcommand{\expectdist}[2]{\mathbb{E}_{#1}\left[ {#2} \right]}
\newcommand{\vardist}[2]{\mathbb{V}_{#1}\left[ {#2} \right]}
\newcommand{\cov}[2]{\mathbb{C}\text{ov}[{#1}][{#2}]}
\newcommand{\covv}[1]{\mathbb{C}\text{ov}[{#1}]}
\newcommand{\corr}[1]{\mathbb{C}\text{orr}[{#1}]}

\newcommand{\fix}[1]{\mathit{fix}\left({#1}\right)}
\newcommand{\sbr}[1]{\left\llbracket {#1} \right\rrbracket}
\newcommand{\ctxtype}[3]{{#1} \cong_\text{ctx} {#2} : {#3}}
\newcommand{\bigstep}[3]{{#1} \Downarrow_{#2} {#3}}


% PCF types
\newcommand{\bool}{\mathit{bool}}
\newcommand{\nat}{\mathit{nat}}

\newcommand{\ctx}[1]{\mathcal{C}\left[ {#1}\right] }
\newcommand{\pcft}[1]{\text{PCF}_{#1}}

\newcommand{\nfl}{\mathbb{N}_\bot}
\newcommand{\bfl}{\mathbb{B}_\bot}

% PCF constructs
\newcommand{\succc}[1]{\mathbf{succ}({#1})}
\newcommand{\succcn}[2]{\mathbf{succ}^{#1}({#2})}
\newcommand{\zero}{\mathbf{0}}
\newcommand{\zerotest}[1]{\mathbf{zero}\left({#1}\right)}
\newcommand{\pred}[1]{\mathbf{pred}\left( {#1} \right)}
\newcommand{\predn}[2]{\mathbf{pred}^{#1}\left( {#2} \right)}
\def\solvable{\#}

\newcommand{\true}{\mathbf{true}}
\newcommand{\false}{\mathbf{false}}
\newcommand{\pcffix}[1]{\mathbf{fix}\left({#1}\right)}
\newcommand{\pcffn}[3]{\mathbf{fn}~{#1}:{#2}\mathpunct{.}{#3}}
\newcommand{\pairtype}[2]{{#1} * {#2}}
\newcommand{\pairexp}[2]{\mathbf{pair}({#1}, {#2})}
\newcommand{\leftexp}[1]{\mathbf{left}({#1})}
\newcommand{\rightexp}[1]{\mathbf{right}({#1})}

\newcommand{\RationalPos}{\mathbb{Q}^{+}}

\newcommand{\meas}[1]{\mathbb{M}\left( {#1} \right) }
\newcommand{\integ}[1]{\sbr{#1}_I}

\newcommand{\notbigstep}[2]{{#1}~\cancel{\Downarrow}_{#2}}
\newcommand{\subtrace}[3]{{#1}^{{#2} \ldots {#3}}}
\newcommand{\supp}[1]{\textsf{supp}\left({#1}\right)}
\newcommand{\dom}[1]{\textsf{Dom}\left({#1}\right)}
\newcommand{\suppk}[2]{\textsf{Supp}^{#1}\left({#2}\right)}
\newcommand{\tracespace}{\bigcup_{n \in \mathbb{N}}[0, 1]^n}
\newcommand{\generictracespace}{\mathbb{T}}
\newcommand{\nnreals}{\mathbb{R}_{\geq 0}}
\newcommand{\posreals}{\mathbb{R}_{> 0}}
\newcommand{\reals}{\mathbb{R}}

\newcommand{\unrollkM}[2]{\textsf{unroll}_{#1}\left({#2}\right)}
\newcommand{\nphmcint}[5]{\Psi_\textsf{NP}\left({#1}, {#2}, {#3}, {#4}, {#5}\right)}

%SPCF constructs
\newcommand{\spcfvalues}{\Lambda^0_v}

\newcommand{\prevalueM}[1]{\textsf{value}^{-1}_{#1}(\spcfvalues{})}
\newcommand{\num}[1]{\underline{#1}}

% \theoremstyle{definition}
% \newtheorem{thm}{Theorem}
% \newtheorem{lem}{Lemma}
% \newtheorem{defn}{Definition}
% \newtheorem{conj}{Conjecture}
% \newtheorem{prop}{Proposition}

%\theoremstyle{definition}
%\newtheorem{defn}{Definition}[section]
%\newtheorem{example}[defn]{Example}
%
%
%\theoremstyle{plain}
%\newtheorem{thm}{Theorem}[section]
%\newtheorem{lem}[thm]{Lemma}
%\newtheorem{cor}[thm]{Corollary}
%\newtheorem{conj}[thm]{Conjecture}
%\newtheorem{prop}[thm]{Proposition}
%\newtheorem{remark}[thm]{Remark}

%% Proofs
%\let\oldproof\proof
%\renewcommand{\proof}{\color{blue}\oldproof}


\definecolor{codegreen}{rgb}{0,0.6,0}
\definecolor{codegray}{rgb}{0.5,0.5,0.5}
\definecolor{codepurple}{rgb}{0.58,0,0.82}
\definecolor{backcolour}{rgb}{0.95,0.95,0.92}

\lstdefinestyle{myStyle}{
    belowcaptionskip=1\baselineskip,
    breaklines=true,
    frame=none,
    basicstyle=\footnotesize\ttfamily,
    keywordstyle=\bfseries\color{green!40!black},
    commentstyle=\itshape\color{purple!40!black},
    identifierstyle=\color{blue},
    backgroundcolor=\color{gray!10!white},
    %backgroundcolor=\color{backcolour}, 
    numberstyle=\tiny\color{codegray},
    stringstyle=\color{codepurple},
    breakatwhitespace=false,                          
    keepspaces=true,                 
    numbers=left,       
    numbersep=5pt,                  
    showspaces=false,                
    showstringspaces=false,
    showtabs=false,                  
    tabsize=2,
}

% argmin/argmax
\DeclareMathOperator*{\argmax}{arg\,max}
\DeclareMathOperator*{\argmin}{arg\,min}

% Concatenation of lists
\newcommand\doubleplus{+\kern-1.3ex+\kern0.8ex}

% Program configurations
\newcommand{\tuple}[1]{\ensuremath{\langle #1 \rangle}}
% Rule based definitions
\newcommand{\Rule}[4][]{\ensuremath{\inferrule*[lab={\hypertarget{#2}{(\TirName{#2})}},#1]{#3}{#4}}}

% Calligraphic symbols
\newcommand{\calI}{{\mathcal I}} 
\newcommand{\calT}{{\mathcal T}}

%  Macro for new Y operator.
\newcommand{\yBounded}[3]{\mu^{#1}_{#2}\rvert_{#3}}

%%%%%%%%%%%%%%%%%
 
%%%%%%%%%%%%%%%%%

\newcommand{\expv}{\mathbb{E}}

\newcommand{\combTr}[2]{\left[\begin{matrix}
		#1\\
		#2
	\end{matrix} \right]}

\newcommand{\exType}[2]{\left\{\begin{matrix}
		#1\\
		#2
	\end{matrix} \right\}}
\newcommand{\myint}[1]{ [#1]}
\newcommand{\Uniform}{\ensuremath{\mathrm{Uniform}}}
\newcommand{\Normal}{\ensuremath{\mathrm{normal}}}
\DeclareMathOperator{\abs}{abs}
\DeclareMathOperator{\pdf}{pdf}

\newcommand{\intConf}[1]{\lceil#1\rceil}
\newcommand{\tr}{\boldsymbol{t}}

\newcommand{\sample}{\tt{sample}}
%\newcommand{\fix}{\texttt{fix}}
%\newcommand{\num}[1]{\underline{#1}}
\newcommand{\myif}{\texttt{if}}
\newcommand{\mylet}{\texttt{let} \, }
\newcommand{\myin}{\, \texttt{in} \,}
\newcommand{\mythen}{\, \texttt{then} \,}
\newcommand{\myelse}{\, \texttt{else} \,}
\newcommand{\score}{\tt{score}}
\newcommand{\tick}{\tt{tick}}

\newcommand{\term}{\tt{term}}
\newcommand{\pv}{\mathbf{v}}
\newcommand{\rv}{\mathbf{r}}

\newcommand{\interval}{\mathfrak{I}}

\newcommand{\typeReal}{\textbf{\textsf{R}}}

\newcommand{\symbolInt}{\myint{\cdot}}

\newcommand{\LambdaInterval}{\Lambda_{\interval}}
\newcommand{\LambdaSymbolic}{\Lambda_{\text{sym}}}

\newcommand{\toIntervalTerm}[1]{#1^{2\interval}}

%Others
\newcommand{\Sset}{\mathbb{S}}
\newcommand{\Iset}{\mathbb{I}}
\newcommand{\Rset}{\mathbb{R}}
\newcommand{\Nset}{\mathbb{N}}
\newcommand{\Zset}{\mathbb{Z}}

\newcommand{\Term}{\mathbb{T}}
\newcommand{\prob}{\mathbb{P}}
\newcommand{\expt}{\mathbb{E}}


\newcommand{\Leb}{\tt{Leb}}
\newcommand{\Red}{\tt{Red}}
\newcommand{\cost}{\text{cost}}

%\newcommand{\intervalab}[2]{\underline{[#1,#2]}}
\newcommand{\intervalab}{\underline{[a,b]}}
\newcommand{\interI}{\mathcal{I}}
\newcommand{\trans}{\mathcal{T}}

\newcommand{\iv}{\mathbb{I}}

% Programming language constructs
\newcommand{\lit}[1]{\underline{#1}}
\newcommand{\letIn}[1]{\mathsf{let}\,{#1}\,\mathsf{in}\,}
\newcommand{\fixLam}[2]{\mu {#1} {#2}.}
\newcommand{\ifElse}[3]{\mathsf{if} (#1 \le \num{0}) \, {#2} \,\mathsf{else}\, {#3}}

%%Basic notions
\newcommand{\pspace}{(\Omega,\mathcal{F},\probm)}
\newcommand{\probm}{\mathbb{P}}
\newcommand{\condexpv}[2]{{\expt}{\left[{#1} \mid {#2}\right]}}

\newcommand{\stdConf}[1]{(#1)}
%\newcommand{\intConf}[1]{\lceil#1\rceil}
%\newcommand{\intConf}[1]{(#1)}
%\newcommand{\symConf}[1]{\langle\!\langle  #1 \rangle\!\rangle}
%\newcommand\symPath[1]{(#1)}
\newcommand{\symPath}[1]{\langle\!\langle  #1 \rangle\!\rangle}
\newcommand\symConf[1]{(#1)}

\newcommand{\ifSimple}[3]{\mathsf{if}(#1, #2, #3)}
%\newcommand{\ifElse}[3]{\mathsf{if} (#1 \le 0) \, \allowbreak {#2} \, \allowbreak \mathsf{else}\, {#3}}
%\newcommand{\ifElse}[3]{\ifSimple{#1}{#2}{#3}}

%\newcommand{\trace}{\mathsf{s}}
%
%\newcommand\defn[1]{{\bf \em #1}}
\newcommand{\traces}{\mathbb{T}}
%
%\newcommand{\stdConf}[1]{(#1)}
%%\newcommand{\intConf}[1]{\lceil#1\rceil}
%\newcommand{\intConf}[1]{(#1)}
%%\newcommand{\symConf}[1]{\langle\!\langle  #1 \rangle\!\rangle}
%%\newcommand\symPath[1]{(#1)}
%\newcommand{\symPath}[1]{\langle\!\langle  #1 \rangle\!\rangle}
%\newcommand\symConf[1]{(#1)}

\newcommand{\valueSem}[1]{\mathsf{val}_{#1}} % value (semantics)
\newcommand{\weightSem}[1]{\mathsf{wt}_{#1}} % weight (semantics)
\newcommand{\measureSem}[1]{\llbracket #1 \rrbracket}
\newcommand{\posterior}{\mathsf{posterior}}


%%%%%%%%%
% 
%%%%%%%%
\newcommand{\loc}{\ell}
\newcommand{\locs}{\mathit{L}}
\newcommand{\blocs}{\mathit{L}_{\mathrm{b}}}

\newcommand{\iflocs}{\mathit{L}_{\mathrm{if}}}
\newcommand{\looplocs}{\mathit{L}_{\mathrm{while}}}

\newcommand{\alocs}{\mathit{L}_{\mathrm{a}}}
\newcommand{\wlocs}{\mathit{L}_{\mathrm{w}}}
\newcommand{\rlocs}{\mathit{L}_{\mathrm{r}}}
\newcommand{\Alocs}[1]{\mathit{L}_{\mathrm{A}}^{\mathsf{#1}}}
\newcommand{\Dlocs}{\mathit{L}_{\mathrm{nd}}}
\newcommand{\transitions}{{\rightarrow}}

%%% 
\newcommand{\plocs}{\mathit{L}_{\mathrm{p}}}
\newcommand{\tlocs}{\mathit{L}_{\mathrm{t}}}

\newcommand{\lin}{\loc_\mathrm{init}}
\newcommand{\lout}{\loc_\mathrm{out}}
\newcommand{\val}[1]{\mbox{\sl Val}_{#1}}

\newcommand{\pvars}{V_\mathrm{p}}
\newcommand{\rvars}{V_{\mathrm{r}}}
\newcommand{\pre}{\mathrm{pre}}

\newcommand{\sle}{\sqsubseteq}
\newcommand{\sge}{\sqsupseteq}

\newcommand{\lfp}{\mathrm{lfp}}
\newcommand{\gfp}{\mathrm{gfp}}

\newcommand{\rdvarjdis}{\mathcal D}
\newcommand{\sampset}{\textit{supp}}

\newcommand{\upd}{\mbox{\sl upd}}
\newcommand{\wet}{\mbox{\sl wt}}
\newcommand{\transset}{\mathfrak T}
\newcommand{\valin}{\pv_{\mathrm{init}}}
\newcommand{\ret}{\mbox{\sl ret}}

\newcommand{\win}{w_{\mathrm{init}}}

\newcommand{\sampdpd}{\overline{\Upsilon}}

\newcommand{\outmap}{\text{O}}
\newcommand{\sat}[1]{\langle #1 \rangle}
\newcommand{\monoid}{\mbox{\sl Monoid}}
\newcommand{\handelmanformat}{(\dagger)}

\newcommand{\trunc}{\mathcal{B}}

\newcommand{\ewt}{\mbox{\sl ewt}}
\newcommand{\statemap}{\text{St}}

\newcommand{\valrd}{{\mathbf{r}}}
\newcommand{\frmloc}{\ell^{\mathrm{src}}}
\newcommand{\toloc}{\ell^{\mathrm{dst}}}

\newcommand{\monomials}{\mathbf{M}}

\sloppy
\begin{document}

%% Title information
\title{\sysname : From Static Analysis to Static Repair}%[\sysname]         %% [Short Title] is optional;
                                        %% when present, will be used in
                                        %% header instead of Full Title.
%\titlenote{}             %% \titlenote is optional;
                                        %% can be repeated if necessary;
                                        %% contents suppressed with 'anonymous'
%\subtitle{} %% \subtitle is optional
%\subtitlenote{} %% \subtitlenote is optional;
                                        %% can be repeated if necessary;
                                        %% contents suppressed with 'anonymous'


%% Author information
%% Contents and number of authors suppressed with 'anonymous'.
%% Each author should be introduced by \author, followed by
%% \authornote (optional), \orcid (optional), \affiliation, and
%% \email.
%% An author may have multiple affiliations and/or emails; repeat the
%% appropriate command.
%% Many elements are not rendered, but should be provided for metadata
%% extraction tools.

% %% Author with single affiliation.
% \author{Anonymous}
% %\authornote{with author1 note}          %% \authornote is optional;
%                                         %% can be repeated if necessary
% \orcid{nnnn-nnnn-nnnn-nnnn}             %% \orcid is optional
% \affiliation{
%   \position{Position1}
%   \department{Department1}              %% \department is recommended
%   \institution{Institution1}            %% \institution is required
%   \streetaddress{Street1 Address1}
%   \city{City1}
%   \state{State1}
%   \postcode{Post-Code1}
%   \country{Country1}                    %% \country is recommended
% }
% \email{first1.last1@inst1.edu}          %% \email is recommended

\author{Naman Jain}
\author{Shubham Gandhi}
\author{Atharv Sonwane}
\author{Aditya Kanade}
\author{Nagarajan Natarajan}
\author{Suresh Parthasarathy}
\author{Sriram Rajamani}
\author{Rahul Sharma}
\affiliation{
\institution{Microsoft Research India}
\country{India}}


%% Abstract
%% Note: \begin{abstract}...\end{abstract} environment must come
%% before \maketitle command
\begin{abstract}
Static analysis tools are traditionally used to detect and flag programs that violate properties. We show that static analysis tools
%can also analyze and learn from programs that satisfy the property being checked. Specifically, static analysis tools
can also be used to perturb programs that satisfy a property to construct variants that violate the property. Using this insight we can construct paired data sets of unsafe-safe program pairs, and learn strategies to automatically repair property violations. 
We present a system called \sysname, which automatically repairs information flow vulnerabilities using this approach.
Since information flow properties are non-local (both to check and repair), \sysname also introduces a novel domain specific language (DSL) and strategy learning algorithms for synthesizing non-local repairs.
We use \sysname to synthesize strategies for repairing two types of information flow vulnerabilities, unvalidated dynamic calls and cross-site scripting, and show that \sysname successfully repairs several hundred vulnerabilities from open source {\sc JavaScript} repositories, outperforming neural baselines built using {\sc CodeT5} and {\sc Codex}. Our datasets can be downloaded from \url{http://aka.ms/StaticFixer}.

\end{abstract}


%% 2012 ACM Computing Classification System (CSS) concepts
%% Generate at 'http://dl.acm.org/ccs/ccs.cfm'.


%% \maketitle
%% Note: \maketitle command must come after title commands, author
%% commands, abstract environment, Computing Classification System
%% environment and commands, and keywords command.
\settopmatter{printacmref=false}
\setcopyright{none}
% \renewcommand\footnotetextcopyrightpermission[1]{}
\settopmatter{printfolios=true}
\maketitle
%\thispagestyle{fancy}

\section{Introduction}
\label{sec:intro}
% Figure environment removed

\section{Introduction}
Automatic 3D reconstruction of clothed humans using image inputs has gained increasing significance due to its potential applications in a wide array of AR/VR scenarios. High-fidelity reconstructions typically depend on sophisticated capture systems, which are developed with dense camera arrays~\cite{collet2015high,joo2015panoptic,joo2018total}, programmable light-stages~\cite{Vlasic2009, guo2019relightables}, and depth sensors~\cite{newcombe2011kinectfusion,DoubleFusion,BodyFusion,dou2016fusion4d,newcombe2015dynamicfusion}. However, stringent capture environments equipped with complex hardware pose significant challenges for consumer-level applications.


In this context, considerable research effort has been dedicated to developing methods that allow for more flexible capture configurations, such as utilizing a few RGB inputs. Among these works, learning implicit functions \cite{iccv2020PIFu, saito2020pifuhd, hong2021stereopifu} has proven effective in achieving highly detailed reconstructions by integrating the advancements of deep neural networks. These methods employ large multi-layer perceptrons (MLPs) to predict the occupancy probability or truncated signed distance function (TSDF) value of every queried 3D point based on its associated local feature, which is extracted from images. They can recover a continuous surface at arbitrary resolutions without topology restrictions.


However, in typical MLP-based implicit networks, the occupancy or TSDF value at each location is solved independently with planar image features, rendering them less capable of addressing challenging cases such as occlusions. Consequently, these methods suffer from generalization and robustness issues, particularly when tackling strong occlusions caused by large motion or multiple interacting humans. 
Some follow-up studies  \cite{zheng2021deepmulticap,zheng2021pamir,huang2020arch} utilize an extra geometric model, SMPL~\cite{Loper2015}, to improve robustness by introducing strong shape priors. 
Their success typically relies on the assumption of geometrical similarity \cite{huang2020arch} between the shape prior and target reconstruction, making them intractable for handling complex cases with loose clothes and sensitive to errors in SMPL model fitting.



%\ping{this paragraph sounds like `TSDF is better than MLP/SMPL, and we use TSDF to solve the problem'. But in Sec 3, we are telling a different story, saying `MLP needs a 3D convolutional encoder'. We need to make these two sections consistent.}\sicong{I think in this paragraph we claim that the TSDF}


%We opt for Trucated Signed Distance Funtion (TSDF) volumetric representations as they are naturally suitable for convolution operations, which have shown remarkable performance for learning hierarchical features on 2D visual perception tasks \cite{SunXLW19}. 
%Meanwhile, TSDF also describes the gradual geometry change around shape surface, which is not reflected by occupancy volume. 

We instead revisit the 3D volumetric representation and resort to 3D convolutional neural networks (CNNs) for feature learning, due to their impressive performance in feature learning and the ability to incorporate spatial context. However, volumetric methods and 3D convolution involve discretization, which might raise concerns regarding whether a discretized volume can preserve subtle geometric details as continuous representations learned in implicit functions. We investigate the relationship between volume resolution and quantization error on synthetic data by converting target mesh objects to TSDF volumes, as shown in Figure~\ref{fig:quantization_error}. We observe that the quantization errors are significantly reduced by increasing volume resolution and become nearly negligible when reaching a relatively high resolution (e.g., 512 or higher). In other words, achieving fine-detailed reconstruction is not supposed to be restricted by the use of volume representations as long as a proper volume resolution is utilized. Therefore, we present a method with high-resolution feature volumes, e.g., 256 and 512, while traditional volumetric methods \cite{varol18_bodynet,gilbert2018volumetric} are often limited to much lower resolutions, such as 32 or 128.



On the other hand, an increase in volume resolution may lead to a cubic growth of memory overhead \cite{8100085}. Reducing memory costs while guaranteeing the granularity of volumetric representations is necessary for pursuing high-quality reconstruction. Thus, we adopt a coarse-to-fine approach and cull away irrelevant voxels to build a sparse high-resolution feature volume. At the coarse level, the network computes an initial TSDF by applying a U-Net with sparse 3D CNN \cite{3DSemanticSegmentationWithSubmanifoldSparseConvNet} on the sparse feature volume, which is carved by a visual hull. Through our experiments, it turns out that more than 95\% of the volume grids are discarded by the visual hull culling, making the sparse 3D CNN efficient. At the fine level, the network focuses on a narrow band near the zero-level set of the initial TSDF and discretizes the narrow band with smaller voxels. By employing this narrow-band culling, we further shrink the sampling space, resulting in a relatively small range of grid numbers (usually 300K--500K in our experiments) even with a high volume resolution of 512. The remaining voxels in the narrow band are associated with features that fuse high-frequency information from the computed normal maps upon the low-frequency shape from the coarse level to compute the TSDF at high resolution. The final mesh is then extracted from the TSDF using the Marching-Cube algorithm ~\cite{Lorensen87marchingcubes}.
% Different from the u-net sturcture to preserve global topology context, we then apply a shallow 3dcnn to compute the final TSDF $D_{final}$ which contain more local geometry detail.




% \ping{this paragraph can be expanded. It is an important contribution and often ignored by other works. stress on the novel idea of regressing blending weights instead of colors}

In addition to geometry, high-quality mesh texture is also a crucial factor contributing to visual appearance. Directly computing a color field in 3D space, as in \cite{iccv2020PIFu}, struggles to capture high-frequency texture details, while the neural radiance field (NeRF) \cite{yu2020pixelnerf} or the DoubleField~\cite{shao2022doublefield} require expensive per-instance optimization and are often unstable for sparse input images. In contrast, we adopt an image-based rendering approach to compute a texture atlas map, which is efficient and widely supported in existing computer graphics tools. 
Specifically, we compute a blending weight at each 3D point on the mesh surface to determine its color as a weighted average of the colors at its image projections. The blending weights can be computed at a relatively coarse resolution, e.g., 512 volume resolution in our case, and leave texture details to the high-resolution images, such as 1K or 2K. Unlike previous methods that generate blurry texturing results under sparse input, our method generalizes well on both synthetic and real data with just a few input views. 
Figure~\ref{fig:teaser} shows two examples reconstructed by our method. Despite the challenging garment, pose, and occlusion, our method recovers faithful shape, normal, and texture on the right.

%with a wide variety of poses and clothing styles, and it is also adaptive to handle input image with arbitrary resolutions.
%\sicong{For this concern we claim that when the resolution of dicretized volume meets certain threshold (which is 256 in our experiment), the quantization error can be neglected.} 



In summary, the main contributions of this paper are as follows:
\begin{itemize}
\vspace{-0.1in}
  \item 
  We revisit the 3D volumetric representation and demonstrate that it can support clothed human reconstruction with equal or even better performance compared to implicit representation. 
  \item 
  We develop a memory and computation-efficient method for high-resolution volumetric reconstruction using sophisticated sparse 3D CNN, coarse-to-fine estimation, and voxel culling by visual hull and narrow bands. 
  \item 
  We introduce a novel method to compute a texture atlas map, which captures rich appearance details from high-resolution input images.
  \item 
  We achieve impressive results on standard benchmark datasets Twindom and MultiHuman, significantly reducing the point-2-surface (P2S) precision to approximately 0.2cm from just six input views, with more than $50\%$ error reduction compared to the state-of-the-art methods, including DoubleField~\cite{shao2022doublefield} and PIFuHD~\cite{saito2020pifuhd}.
\end{itemize}

\section{Background And Overview}
\label{sec:background}
\vspacebeforesection
\section{Background}
\label{sec:background}

In this section, we provide the necessary background information to ensure a comprehensive understanding of the attack described in this paper. We start with a description of the Distributed Hash Table (DHT) used by IPFS, followed by its content resolution mechanisms. We also detail techniques for network size estimation, necessary for our attack detection and mitigation mechanisms.

\vspacebeforesection
\subsection{IPFS DHT}
\label{sec:kad_dht}

We review the features of the Kademlia DHT~\cite{maymounkov2002kademlia} and its \texttt{libp2p} implementation~\cite{libp2p_github} that are the most relevant to our attack.
To participate in the DHT, each peer generates a public/private key pair and derives an identity $\peerid \in \{0,1\}^{256}$ as the hash of its public key.
Ideally, each peer generates a random key pair and, therefore, peer IDs are distributed uniformly and independently over the space $\{0,1\}^{256}$.
While honest nodes follow this rule, malicious nodes may generate and choose from an arbitrary number of key pairs.
Each peer maintains a routing table consisting of $m=256$ buckets.
The $i$-th bucket contains the addresses of up to $k=20$ peers whose peer IDs share a common prefix of exactly $i$ bits with the peer's own peer ID. 

%
A new participant node joins the IPFS network by contacting one of the hardcoded bootstrap nodes. This bootstrap node provides the new node with some initial peers allowing it to join the DHT. The new node uses this information to perform a walk through the DHT towards its own peer ID.
The walk allows to: \textit{(i)}~make sure that there is no other node in the network with the same ID; \textit{(ii)}~discover new peers and fill the newcomer's DHT routing table. At the same time, the newcomer establishes \bitswap~\cite{de2021accelerating} connections to a subset of encountered peers (usually around 300 of them). The core role of the \bitswap protocol is to enable bilateral content transfer and to play the role of a cache for recently-accessed content.

The main DHT operation $\Call{GetClosestPeers}{\key}$ returns the $k=20$ closest peers to $\key$. 
%
In Kademlia, the distance between two keys $x$ and $y$ in the key space is given by $x \oplus y \in \{0,...,2^{256}-1\}$, where $\oplus$ denotes the bitwise XOR operation on the keys; the resulting binary string is interpreted as an integer.
%
When a client wants to find the peers with IDs closest to $\key$, it sends a request to the $\alpha=3$ peers in its routing table whose peer IDs are closest to $\key$. Each of these peers returns the $k$ closest peers to $\key$ in its own routing table and the addresses of these peers. 
%
The client again sends a request to the $\alpha$ peers closest to $\key$, among peers in its routing table and those whose addresses it just received. This process repeats until the client does not find any more peers closer to $\key$.
Due to network churn and imperfect routing tables, we observed in our experiments that successive calls to $\Call{GetClosestPeers}{\key}$ do not always return the same set of $k=20$ peers (we provide more details in \Cref{sec:evaluation}, \Cref{fig:20closest}). This is an important limitation affecting our attack.

\vspacebeforesection
\subsection{Content Resolution in IPFS}
\label{sec:ipfs}

IPFS is a content-centric network.
It allows its participant to request files without specifying their location. 
%
Content is indexed by content IDs $\cid \in \{0,1\}^{256}$ that are derived from a hash of that content.
Both peer IDs and CIDs are used as keys in the DHT.
Each node can play the role of a \provider, \downloader, or \resolver. 
The process of content advertisement and resolution is illustrated in \Cref{fig:add_get_provider}.

%
When a \provider wishes to publish content with a given $\cid$ on IPFS, it creates a \emph{provider record} that contains $cid$ and the \provider's address.
During a $\Call{Provide}{\cid}$ operation, the \provider first uses $\Call{GetClosestPeers}{\cid}$ to locate the $k=20$ peers with their peer IDs closest to $\cid$, 
%
and then sends them a $\mathsf{PutProvider}$ message including the provider record (\Cref{fig:add_get_provider}(a)).
We call the peers that hold provider records for $\cid$ the \emph{resolvers} for $\cid$.

Each CID can have several \providers. In fact, by default, each IPFS client becomes a provider for each piece of content it downloads for a fixed amount of time (12h, 24h, or 48h depending on the client version or custom configuration). As a result, the system provides an auto-scaling feature with supply automatically rising with demand.

%
When a \downloader wishes to fetch a piece of content, it first sends a request to all its \bitswap peers. If none of them has the content, the \downloader uses the DHT-based resolution system. We stress that the \bitswap protocol plays the supporting role of a cache in the dissemination of popular files. However, the mechanism does not provide reliable content resolution, in particular for new or less popular content. %

When \bitswap unstructured search fails, the \downloader resolves $\cid$ using $\Call{FindProviders}{\cid}$. This operation uses a DHT walk identical to that of $\Call{GetClosestPeers}{\cid}$ to find $k$ \resolvers but also queries encountered nodes for a provider record for $\cid$ (\Cref{fig:add_get_provider}(b)). The process terminates when either 20 \providers have been found, or all \resolvers have been asked. Querying all encountered nodes (\ie, not only the designated \resolvers) is useful because some of the encountered nodes may have a provider record in their cache.
%

Upon receiving a provider record, the client connects to the address specified in the provider record to retrieve the actual content (\Cref{fig:add_get_provider}(c)).
Provider records are not authenticated, and therefore malicious \providers may respond with incorrect provider records (or may not respond at all). However, the integrity of the content is preserved because the hash of the retrieved content can be verified against its $\cid$.
%


%

\input{img/add_get_provider.tex}

\vspacebeforesection
\subsection{Network Size Estimator}
\label{sec:netsize}

The number of nodes in a decentralized system is generally unknown due to the avoidance of centralized membership management.
This number is nonetheless useful for optimizations, deciding on individual node configurations, or security mechanisms.
Various methods were proposed for the decentralized estimation of unstructured and structured networks~\cite{eli-sohl-dht-size-estimation,kostoulas2005decentralized, manku2003symphony}.
We use in this work a mechanism developed initially by Protocol Labs as part of a mechanism for decreasing the latency of publishing content in IPFS~\cite{network-size-estimation-notion,network-size-estimation-github-pr}.

%
%
%
%
%
%
%
%
%
%

Each node in the DHT refreshes its routing table periodically (every $10$ minutes in \texttt{libp2p}). 
For this, the node samples $m$ random keys (one for each bucket of its routing table)
%
and queries the DHT to obtain the $k=20$ closest peer IDs to each key.
Using these, the node then computes the average distance between each one of these keys $\key_j$ for $j=1,\dots,m$ and their $i$-th closest peer ID for $i=1,...,k$ (with $m=256$ and $k=20$).
\begin{equation}
    \label{equ:avg-dist}
    \overline{D}_i = \frac{1}{m} \sum_{j=1}^m \operatorname{dist}(\key_j, \peerid_{j}^{(i)})
\end{equation}
where $\peerid_{j}^{(i)}$ is the $i$-th closest peer ID to $\key_j$.
With $N$ peers in the DHT and peer IDs uniformly distributed in the hash space, the expected distance between a $\key$ and its $i$-th closest peer ID is $\frac{2^{256}i}{N+1}$. The node then runs a least square regression to compute the value of $N$ for which the expected distances best fit the empirical average distances, \ie,
\begin{equation}
    \label{equ:netsize-least-squares}
    \hat{N} = \arg\min_{N} \sum_{i=1}^k \left(\overline{D}_i - \frac{2^{256}i}{N+1}\right)^2.
\end{equation}
The resulting estimate $\hat{N}$ can be computed in closed form.
%

When a node starts running, it must perform DHT queries for a few random keys to initialize its network size estimate. 
Since a larger number of queries will result in higher accuracy, making more queries than what is needed to initialize one's routing table is recommended.
Thereafter, keeping the estimate up-to-date does not require any excess DHT queries beyond what is already used for refreshing the routing table as this is done frequently (every 10 minutes).

While the network size estimate has a stochastic variance resulting from the probability distribution of the honest peer IDs, it is hard for an attacker to bias the estimate significantly. Since the estimator uses the density of peer IDs around keys chosen uniformly at random, the adversary would require numerous Sybil nodes (on the order of the whole network size) to significantly affect the peer ID density around those keys.


\section{Data Collection}
\label{sec:data}
\lstMakeShortInline[columns=fixed]@
% Figure environment removed
\lstDeleteShortInline@

In this section, we describe how we collect examples for learning repair strategies without any version-controlled data. Specifically, we first detect \safeprogs and corresponding witnesses using \sawitnessfull (witnesses are sanitizers and guards that protect from vulnerabilities)  in Section~\ref{subsec:sa-witness}. Using these witness annotations, we generate unsafe programs and \textit{edits} from the \safeprog using a \textbf{witness-removal} step (Section ~\ref{subsec:witness-removal}). In the following, we define terminology for the \astree  data-structure we operate on. 


\astree refers to the abstract syntax tree representation of programs, augmented with data flow edges and annotations for sources, sinks, sanitizers, guards, witnesses etc. 
An \astree is a five-tuple 
$\langle \mathcal{N},\mathcal{V},\mathcal{T},\mathcal{E}, \mathcal{A} \rangle$, where:
\begin{enumerate}
\item
$\mathcal{N}=\{\mathit{id}_0,\ldots\mathit{id}_n\}$  is a set of nodes, where  $\mathit{id_i}\in\mathbb{N}$ for 
$ 0 \leq i \leq n$.
\item
$\mathcal{V}$ is a map from nodes to program snippets
represented as strings. For a node $n$, we have that $\mathcal{V}(n)$ is a string representing the code snippet associated with $n$
\item
$\mathcal{T}$ is a map from nodes to their types defined by 
 \sa~\cite{codeqlast}. For example, \callexpr is the type of a node representing a function call, \indexexpr is the type of a node representing an array index, and \blockstmt is the type of a node representing a basic block of statements.
\item
$\mathcal{E}$ is a set of directed edges.
Each edge is of the form $(n_1,n_2,\edgetype,z)$, where
$n_1$ is a source node, $n_2$ is a target node, 
$\edgetype \in \{\T{SynParent}, \T{SynChild}, \T{SemParent},
\T{SemChild} \}$ denotes the relationship from 
$n_1$ to $n_2$, as one of syntactic parent, syntactic child, semantic parent or semantic child,
and $z\in\mathbb{Z}$ is the index of $n_2$ among $n_1's$ children if this edge is a child edge, and $-1$ if the edge is a parent edge. 
\item
$\mathcal{A}$ is a set of annotations associated with each node. The annotations are from the set $\{\T{source},
\T{sink},\T{sanitizer},\T{guard}$,\T{witness}\}. We also refer to annotations using predicates or relations. For instance, for a node $n$, if an annotation  $\T{source}$ is present, we say that
the predicate $\T{source}(n)$ is true.
\end{enumerate}

%\setlength{\grammarindent}{5em} % increase separation between LHS/RHS

% Figure environment removed



A {\em traversal} or a {\em path} in an \astree is a sequence of edges $e_0,\ldots,e_{i-1},e_i,\ldots ,e_k$ such that the target node of $e_{i-1}$ is also the source node of $e_i$, for all $i\in\{1,\ldots,k\}$. That is, $e_{i-1}$ is of the form $(\_,n,\_,\_)$ and $e_i$ is of the form $(n,\_,\_,\_,\_)$. The source node of $e_0$ is the source of this path and the target node of $e_k$ is the target of the path.


\lstMakeShortInline[columns=fixed]@
%Note that these additional edges can capture long-range dependencies in programs. E.g. edge 4 in Figure ~\ref{fig:unsafememberex} links two nodes across the function boundaries. 
Figure~\ref{fig:example1-pdg} depicts a partial \pdg corresponding to the unsafe program in Figure~\ref{fig:unsafememberex}. Each oval corresponds to an \astree-node containing a type $\tau$ and an associated value. The dark edges denote the syntactic child edges. For example, the oval with value @foo(data)@ is an \astree-node with type \callexpr and has two children -- @foo@ and @data@, both with the type \varexpr. 
%Similarly, the \blockstmt node on the top refers to the function body between Line~\ref{lst:line:handlers-run} and Line~\ref{lst:line:handlers-run-end} in Figure ~\ref{fig:unsafememberex}. As the body of a function block can contain a variable number of children, we link to @handlers[callerId](data);@ as the k-th child of the \blockstmt. 
The semantic child edges are at the bottom in cyan. These edges correspond to the ones depicted in cyan in Figure ~\ref{fig:unsafememberex}. 
\lstDeleteShortInline@

%TODO:FIX THIS

%With this simplification, 
If $\prog$ is an \pdg then
we use  $\prog.\mathtt{source}$ to denote the source node, $\prog.\mathtt{sink}$ to denote the sink node, and $\prog.\mathtt{witness}$ to denote the witness node.
If the program has several sources, sinks and sanitizers then we generate a separate \pdg for each $(\mathtt{source},\mathtt{witness},\mathtt{sink})$ triple.
For a node $n$, its syntactic parent is $n.\mathtt{parent}$, syntactic children are $n.\mathtt{children}$, semantic parent is $n.\mathtt{semparent}$, and semantic children are $n.\mathtt{semchildren}$.

%\input{ql.tex}

\subsection{Static Analysis Witnessing}
\label{subsec:sa-witness}

\newcommand{\DMethodjudge}[1]{\texttt{#1(}\checknextarga}

% Figure environment removed

%\naman{TODO - sell this more as technique to work with any \sa tool ; our master query is a general framework implemented in \codeql that can work for any vulnerability -- easily extendable to other languages }
In this section, we show how to repurpose \sa tools to generate witnesses.
\sa tools perform dataflow analysis to check for rule-violations in programs. They use pattern matching to identify known sources, sinks, sanitizers, and guards. For commercial tools, these patterns are implemented (and continuously updated) manually by developers and encode this domain knowledge. Next, 
%these patterns are used to detect sources, sinks, sanitizers, and guards in programs and
\sa checks if there exists a flow between a source and a sink that does not cross a sanitizer or guard. We capture this formally in Figure~\ref{fig:judgements} (top two rules), and explain the notation used in it below.

\sa tools encode domain knowledge about the vulnerability by annotating nodes as \T{Source}, \T{Sink}, \T{Sanitizer}, and \T{Guard}. %These relations operate on the set of dataflow nodes in the programs.
So \DMethod{Source}{\I{n}}\ is true iff the node \I{n} is a \textit{source} node for a vulnerability. Next, \sa tools perform dataflow analysis by defining the relation \DMethod{SemChild}{$n_1$}{$n_2$}\ which is true iff there is a \taintpropedge between $n_1$ and $n_2$. Then the \DMethod{Vulnerability}{$n_1$}{$n_2$}\ relation can be defined as:
\begin{enumerate}
    \item $n_1$ and $n_2$ are source and sink nodes (\DMethod{Source}{$n_1$}\ and \DMethod{Sink}{$n_2$}\ are true)
    \item There exists a \textit{path} between $n_1$ and $n_2$ which is free of sanitizers or guards (\DMethod{SanGuardFree*}{$n_1$}{$n_2$}\ is true). A path is free of sanitizers and guards iff every \textit{edge} in the \textit{path} is free of sanitizers and guards. An edge between $n_1$ and $n_2$ is considered free of sanitizers and guards (\DMethod{SanGuardFree}{$n_1$}{$n_2$}\ is true) iff $(n_1, n_2, \_, \T{SemChild}) \in \mathcal{E}$ and neither of $n_1$ or $n_2$ is a sanitizer or a guard
\end{enumerate}

Here, we make the following observation - \emph{this domain knowledge present in these annotations and relations is helpful beyond just detecting vulnerabilities}. For instance, simply using the sanitizer relation allows us to query the different kinds of sanitizers domain experts have specified. We use this observation to discover \emph{\safeprogs} i.e., programs having a source, sink, and a sanitizer or guard that \textit{blocks} the \taintprop or, in simpler terms, make the program safe. In addition, we also detect the corresponding sanitizers or guards in the programs and refer to them as \textit{witnesses} because they serve as the evidence of making the program safe. We call this procedure \sawitnessfull (abbreviated as \sawitness). 
We define this as the \T{Witness} relation in Figure~\ref{fig:judgements} (bottom two rules). Specifically, \DMethod{Witness}{$n_1$}{$n_3$}{$n_2$}\ is defined as:
\begin{enumerate}
    \item $n_1$ and $n_2$ are source and sink nodes (\DMethod{Source}{$n_1$}\ and \DMethod{Sink}{$n_2$}\ are true)
    \item There exists a node $n_3$ such that it satisfies \DMethod{SanGuardInMid}{$n_1$}{$n_3$}{$n_2$}. \DMethod{SanGuardInMid}{$n_1$}{$n_3$}{$n_2$}\ is true iff there exists a \T{SemChild}
    %\naga{notation for flow inconsistent with (2) above} 
    path between $n_1$, $n_3$, between $n_3$ and $n_2$, with the additional constraint of $n_3$ being a sanitizer or guard. 
\end{enumerate}

The difference between the \T{Vulnerability} relation (which \sa populates) and \T{Witness} relations (which we want to find) is highlighted in {\color{red} red} and {\color{ForestGreen} green}. Notice that while defining the \T{Witness} relation, we simply use the existing relations that define the \T{Vulnerability} relation. Thus, we argue that \sawitness can be implemented on top of \sa by using the intermediate relations that \sa is computing.
%for every pair of source and sink, they track taint through a taint-flow analysis. If there is a flow from a source to a sink that does not go through a sanitizer or guard, then the source-sink pair is reported as vulnerable.

%We make the following observation - \emph{the patterns defined by experts encodes domain knowledge which can be used for use cases beyond just detecting vulnerabilities}. For instance, we can use the sanitizer patterns to search for all sanitizers in source-code. In this work, we use this idea to detect \safeprogs, which we define as programs having a source, sink, and a sanitizer or guard that blocks the \unsure{flow} or in other words, makes the program safe.  \naman{highlighted part of Figure somethings shows the difference between semantics of witnessing vs traditional semantics}

%We realize the following -- the set of patterns of sources, sinks, and sanitizers are useful beyond detecting vulnerabilities. We override the existing static analysis query that detects unsafe programs and use these encoded sanitizers for detecting sanitizers and guards in programs. Specifically, in the existing query that detects unsafe programs, we modify the taint-propagation steps to propagate taints through sanitizers and guards and then use static analysis to then find these dataflows containing sanitizers and guards. Thus, we can directly find the safe programs containing these \textit{witnesses} of safety. 
%Once such a dataset is collected, we use these witnesses to convert safe  to unsafe  and thus obtain paired examples for learning repair strategies (Section~\ref{subsec:witness-removal}). 

\lstMakeShortInline[columns=fixed]@
%We instantiate our \sawitness technique using \codeql~\cite{a}. It is an open-source \sa tool that allows implementing custom static analysis as queries in a high-level object-oriented extension of datalog. These queries usually contain a \Verb|select from where| statement that allows querying the program database. \codeql maintains these patterns of sources, sinks, sanitizers, and guards using \Verb"Configuration" classes. Consider an example of a simplified \Verb"Configuration" for \xss vulnerability in Figure~\ref{fig:configuration}. It defines a set of predicates @isSource@, @isSink@, @isSanitizer@, and @isGuard@. These predicates are written manually by \codeql authors and improved through rich community support\footnote{\url{https://github.com/github/codeql}}. With this configuration, vulnerabilities are reported by selecting source-sink pairs such that the @cfg.hasFlow@ predicate is true for the source, and the sink. This predicate is internally defined by \codeql and uses the patterns defined in the configuration to check for the presence of vulnerability-causing dataflows. %\spsays{Showing corresponding programs will be useful}

%Now, we demonstrate the static-analysis-witnessing approach for collecting examples of \safeprog and witnesses in Figure~\ref{fig:safe-configuration}. Specifically, we inherit from the existing configuration, using the same @isSource@ and @isSink@ predicates while overriding the @isSanitizer@ and @isGuard@ predicates to @none()@. This ensures that all the source and sink pairs are detected independent of the presence of sanitizers/guards between them. Finally, to detect our witnesses, we define the @isWitness@ predicate which uses the @isSanitizer@ and @isGuard@ predicates from the original configuration. Specifically, witnesses are defined as sanitizers/guards that lie between a source-sink pair. Finally, to report \safeprog and witnesses, the @cfg.hasFlow@ predicate is used to select all valid source-sink pairs and the corresponding witnesses are detected via the @isWitness@ predicate. Note that Figure~\ref{fig:configuration-vs-safe-configuration} depicts the key idea behind our approach in a simplified view. In practice, additional measures need to block the taint propagation internally and we share the actual \codeql queries used as part of the Appendix~\ref{app:codeql-queries}.


\subsection{Witness Removal}
\label{subsec:witness-removal}

We obtain \safeprogs and witnesses by applying \sawitness to a snapshot of a codebase. Recall that the witnesses block the flow between a source and a sink and thus help make programs  \textit{safe}. Hence, removing these witnesses will make the programs unsafe. Recall also that the witnesses are either sanitizing functions of the form @sanitize(taintedVar)@ or guards of the form @if checkSafe(taintedVar) {executeSink(taintedVar)}@. %Usually, they are used only for ensuring the safety of programs and are not critical to the functionality of programs. Therefore, 
We implement witness-removal perturbations  that precisely remove the guard-checks and sanitizer-functions. Note that our goal here is to generate unsafe programs and corresponding edits that enable learning repair strategies that insert such witnesses. So, while we generate the unsafe programs by perturbation, they should look structurally similar to natural unsafe programs written by the developers, otherwise the repair strategies learned on this artificially generated data through perturbations would not generalize to code in the wild. 
%At the same time, minor syntactic-semantic issues in parts of unsafe programs not directly relevant to the vulnerability or repair do not impact learning.
\lstDeleteShortInline@

% Figure environment removed

\lstMakeShortInline[columns=fixed]@

\input{witnessremoval.tex}

We use \rmSan and \rmGuard functions to programmatically remove the witnesses. A high-level sketch of these functions is illustrated in Figure~\ref{fig:remove-functions}. The functions use the structure of the corresponding \astree (node types $\tau$) to decide how to remove witnesses. Consider the \rmGuard function. It first computes the parent (\witnesspar) and grand-parent (\witnessparpar) of the witness guard condition. Then if the type of \witnesspar is \ifstmt (i.e., program is of the form @if (witness) body@ then we modify the \astree edge from \witnessparpar and \witnesspar to instead point to the body of the \ifstmt (index 1 child is body of \ifstmt). Similarly, if the type of \witnesspar is \binaryexpr with operator @&&@ (i.e. of the form @if (otherCond && guard)@ or @if (guard && otherCond)@) then we again modify the edge from \witnessparpar and \witnesspar to instead point to the non-guard child of \binaryexpr (@otherCond@ in the example). Note that since \binaryexpr has 3 children, the index of non-guard child is index of guard-child subtracted from 2. 
Figure~\ref{fig:witness-removal} depicts this removal on the \astree level, where the syntactic edges in red are removed and the syntactic edges in green are inserted.
In the end, the functions returns a tuple of the \pdg of the unsafe program ($\prog_{unsafe}$), \pdg of the safe program ($\prog_{safe}$)
and an edit object (\edit) which stores


\begin{enumerate}
    \item \astree for the removed witness (referred to as \editprog)
    \item location in the \pdg where the witness is removed (referred to as editloc
    %\naga{shouldn't it be editloc to be consistent with (1)?} 
    or \editloc)
    %\item an enum (\insertsc or \replace) depending on whether \concedit is inserted or replaced 
\end{enumerate}

Since $\prog_{unsafe}$ and edit-object can generate the safe program, we only propagate the unsafe programs and edits as the output of this step. Applying \rmGuard function to the safe program in Figure~\ref{fig:safememberex} removes the \ifstmt on Line~\ref{lst:line:fix-start} while preserving the @handlers[callerId](data);@ statement and in fact produces the unsafe program in Figure~\ref{fig:unsafememberex}. Additionally, it  returns the removed witness guard  @if handlers.hasOwnProperty(data.id){ ... }@ as the \editprog and \blockstmt (blue oval in Figure~\ref{fig:example1-pdg}) as the edit location \edit.editloc. Figure~\ref{fig:example1-editprog} shows the \astree for the \editprog containing the \ifstmt. 
The dashed line and dark circle correspond to the \textit{removed} \astree edge between the \blockstmt and the \expr @handlers[callerId](data)@. 

Note that Figure~\ref{fig:remove-functions} provides a high-level sketch of witness-removal and elides over implementation details that are required to make it work for real \js programs. We discuss these issues in the implementation section (Section~\ref{subsec:impl:witness-removal}).% and include the full implementation as part of supplementing source code\naga{we should make sure we are doing these, else remove this sentence}. 
%. In practice, we need implement such decisions more carefully to cover other traditional cases in which guards occur and we document them in the supplementing source code.
\lstDeleteShortInline@

%\naman{add examples $\dots$ } \spsays{do we re-run codeql on this generated bad program? -- NO (naman)}



\section{Strategies and Learning Algorithm}
\label{sec:learning}
In this section, we describe how to learn repair strategies from the  unsafe programs and edits collected in Section~\ref{sec:data}. We define a \dsl (Section~\ref{subsec:dsl}) to express repair strategies that take an \pdg of an unsafe program  as input and generate a safe program as output. The DSL is expressive and can even express bad strategies that don't generalize well to programs in the wild. We provide examples of such bad strategies and good strategies that generalize well  (Section~\ref{subsec:examples}). We learn good repair strategies  in a data-driven manner using an example-based synthesis algorithm (Section~\ref{subsec:synthesis}). %Finally, given a new unsafe program and a set of learned repair strategies, we apply these strategies and generate  candidate repairs (Section~\ref{subsec:applying}).


%Our goal is to use the collected data to learn high-level general repair strategies. We learn these repair strategies over a joint representation of the \astree with the annotations inferred from the \sa tool (the representation referred to as \pdg ahead).  These inferred \sa tool annotations allow us to take the advantage of rich semantic information while performing \unsure{repairs}. Figure ~\ref{fig:example1-pdg} shows an example \pdg corresponding to the unsafe code shown in Figure ~\ref{fig:unsafememberex}. We develop a powerful \dsl that can utilize the annotations in the \pdg structure and learns repair strategies using a deductive synthesis algorithm. More specifically, strategies in this \dsl operate over the \pdg structure of unseen code-snippets and suggest appropriate edits correspondingly. \aksays{The following sentence can be removed if space becomes a constraint.} Section~\ref{subsec:dsl} describes the \dsl, Section~\ref{subsec:synthesis} talks about the synthesis algorithm, and Section~\ref{subsec:applying} demonstrates strategies in this \dsl can be applied. 


\subsection{\dsl for repair strategies}
\label{subsec:dsl}
We introduce a novel \dsl to express repair strategies in Figure~\ref{fig:fixing-dsl}.
%that use the knowledge of program semantics annotated on \pdg instead of just using the syntactic program structure and in-turn are more expressive and generalize better. These strategies take the an unsafe-program as input and return candidate repair programs by performing tree-edit-operations.
At a high level, the strategies define a three-step process where  they provide a computation to identify the edit-location node \editloc, a computation to identify the child index $\editindex$ of \editloc where repair happens, and a computation to generate the AST that must be placed at  index $\editindex$ of \editloc for the repair. The main part of these computations involve traversing paths of the input unsafe program \prog.
%The edit-operation can either be inserting a syntactic-child at \editloc (\insertsc) or replacing a syntactic-child with another tree at \editloc (\replace). The index at the \astree-node $\editloc$ where the insertion or replacement occurs is called the edit-index (\editindex). The tree that is inserted or replaces another existing tree at the \editloc is materialized hierarchically for the given example by defining abstract program structure using a combination of constant structure and references to \astree-nodes in the existing program \prog. These \astree-nodes are called reference-locations (\refloc). To find these locations (\editloc, \refloc) in a given program, the strategies abstractly store \textit{traversals} which materialize into a \textit{concrete} \astree-node in the given programs.  

%\naman{todo - talk about traversal in the introduction, background etc.}
\input{dsl}

%We present our \dsl in Figure ~\ref{fig:fixing-dsl}. 
%The DSL is a list of definitions for various non-terminals in the grammar. For each non-terminal, we define a corresponding type and a set of production rules. Each production rule is either a fixed expression, or an operator applied to other non-terminals or fixed-expressions in the grammar.  
The top-level production rule of the DSL defines strategies, \strategy, with type \newtextsc{Strategy}. 
%A \node is either the source node (\prog.source) or an application of \traversal on another \node. 
\gettraversal, \getclauses, and \getindex are all functions that take a \node $n$ as input and return a \node, \bool, and \integer as output respectively. The edit-AST, \eastree, is similar to a syntactic variant of \astree (i.e. no semantic edges) which we define in Section~\ref{sec:data} with one addition. It has reference nodes that, when applying the strategy to the input \pdg of \prog, are materialized from sub-trees of this \pdg, where the root nodes of these sub-trees are identified by traversing paths in the input. 
%Finally, edge-type (denoted by \edgetype) is an enumeration describing the type of edge, i.e. syntactic or semantic, and parent or child, as defined in Section~\ref{sec:data}. 

% Given these types, we now define the operators used in our \dsl. 
The strategy \strategy is of two types, \insertsc or \replace. \DMethod{Insert}{\I{L}}{\I{I}}{\I{O}}\ declaratively expresses the computation that computes the edit-location \editloc by traversing the path supplied in \I{L}, then computes \editindex, the index of edit-location,  by evaluating \I{I}(\editloc), and inserts the materialization of \I{O} as a syntactic-child \astree at index \editindex of the edit-location \editloc. \DMethod{Replace}{\I{L}}{\I{I}}{\I{O}}\ is similar and performs a replacement instead of an insertion.
%computes the \editloc and \editindex, and replaces syntactic-child of \editloc at \editindex with \I{O}.
%The insertion and replacement operations modify the nodes $\mathcal{N}$ and edges $\mathcal{E}$ of the \astree (Figure~\ref{fig:astsyntax}) appropriately. 

\node (\I{L}) is either the node corresponding to the source of vulnerability (\prog.source) or the target of the path corresponding to the traversal \DMethod{ApplyTraversal}{L}{\I{F}$_k\ o\ $\I{F}$_{k-1}\ o\ \cdots$\I{F}$_0$}. 
Here, each \I{F}$_i$ 
is a function that takes a node 
$n$ as input, performs a traversal from $n$, and returns the traversal's target node $n'$. 
Thus, \T{ApplyTraversal} can be recursively defined as \DMethod{ApplyTraversal}{\I{F}$_0$(L)}{\I{F}$_k\ o\ $\I{F}$_{k-1}\ o\ \cdots$\I{F}$_1$}\ if $k>0$ and \I{F}$_0$(L) otherwise. 

\newtextsc{GetTraversal} (\I{F}) defines a function that takes a node $n$ and returns a node $n'$ reachable from $n$ and can be of two types. Given $n$, the \DMethod{GetEdge}{\I{ET}}{\I{I}}\ operator first finds the possible single-edge traversals of type \I{ET} and indexes it using \I{I}. Specifically, if edge type \I{ET} is a parent then it returns the parent of $n$. Otherwise, 
it finds a set of $N$ of nodes that are connected with $n$ via the edge type \I{ET}, i.e., $N = \mathcal{E}(n, \I{ET})$, and returns the node $N[I(\I{n})]$ at the index given by $I$. In contrast, $\DMethod{GetKleeneStar}{\I{ET}}{\I{C}}(n)$  performs a \newtextsc{KleeneStarTraversal} that iteratively traverses edges of type \I{ET}, staring from input node $n$, until it reaches an edge whose target  node $n^{i}$ satisfies the condition defined by the clause \I{C}. Formally, \newtextsc{KleeneStarTraversal} can be defined recursively as $KE(n_1,ET,C) = \I{C}(n_1)? n_1 : \left(let\ t\in\mathcal{E}(n_1,ET)\ in\ KE(t,ET,C)\right)$. Here, the node $t$, which is target of an edge with source $n_1$ and type $ET$,  is chosen non-deterministically and our implementation resolves this non-determinism through a breadth-first search.
%A \traversal is a relation between nodes $n_1$ and $n_2$ such that there is an edge or a sequence of edges between them. 

\newtextsc{GetIndex} (\I{I}) defines a function that takes a node $n$ and returns a \integer. It is either a constant function that returns a fixed integer $z$ or a \DMethod{GetOffsetIndex}{\I{L}, \I{z}}. \DMethod{GetOffsetIndex}{\I{L}, \I{z}}\ takes a node $n$ as input and returns an integer $DO(n,L)+z$, where $DO(n_1,n_2)$ returns the index of syntactic child of $n_2$ who is a syntactic ancestor of $n_1$. 

\eastree (\I{O}) defines the edit \astree with reference nodes which, given an input program \prog, are materialized to a concrete \astree. The \eastree can either be a \T{ConstantAST} or a \T{ReferenceAST}. Specifically, \DMethod{ConstantAST}{$\tau$}{\I{value}}{\I{O}$_1$}{\I{O}$_2$}{$\cdots$}{{\I{O}$_k$}}\ returns an \eastree that has a type $\tau$, string representation \I{value}, and is recursively constructed with sub-trees \I{O}$_1 \cdots$ \I{O}$_k$ as syntactic children, each of which can either be a \T{ConstantAST} or a \T{ReferenceAST}. The \DMethod{ReferenceAST}{\I{L}}, when applying the strategy, finds a node $n$ in \prog by traversing the path described in \I{L} and returns a copy of the (syntactic) sub-tree of \prog rooted at $n$. %Next, we show examples of strategies written in this DSL and how to learn them automatically.

% Finally, note that the traversals can be composed by applying multiple \T{ApplyTraversal} operators sequentially. We use this key insight into developing our learning from examples setup. 
\newcommand{\newwrapbox}[2]{\adjustbox{margin=1pt 1.3pt, bgcolor=white, frame=1pt, cframe=#1, color=#1}{#2}}
% Figure environment removed


\lstMakeShortInline[columns=fixed]@
\subsection{Example of strategies in our \dsl}
\label{subsec:examples}
Figure~\ref{fig:repair-strategy-ex1} describes   two possible repair strategies that are sufficient to repair the motivating example in Figure~\ref{fig:vulnerabilty-example1}. We first describe the good strategy in Figure~\ref{fig:strat1}, referred to as \strategyone,  and then compare it with the bad strategy \strategytwo in Figure~\ref{fig:strat2}. 

Given the program \prog in Figure~\ref{fig:vulnerabilty-example1}(a) as input, the strategy \strategyone
performs a replacement at index \I{I} of edit-location $L_e$ with the materialization of \I{O} (line 20 of \strategyone).
This process requires first finding the "semantic location" node \semloc. %The semantic location for \prog is shown in red in Figure~\ref{fig:example1-pdg}.
To this end, the strategy 
first  traverses a path from the node annotated as \T{source} by \sa  using \DMethod{GetKleeneStar}\ in Line~\ref{lst:line:semkleene} of \strategyone.  This \newtextsc{KleeneStarTraversal} starts from \T{source}, traverses semantic dataflow edges, and stops at a node 
corresponding to an identifier being used as the function name in a function call. 
 For the input program $P$, the traversal takes the semantic-child-edges 1-7 (Figure~\ref{fig:example1-pdg}) and stops at @foo@ in Line~\ref{lst:line:callerId-sink} of Figure~\ref{fig:vulnerabilty-example1}(a). Next, to reach the edit-location $L_e$, the strategy uses a \newtextsc{KleeneStarTraversal} that starts from \semloc, traverses syntactic parent edges,  and stops when it reaches a \blockstmt. For $P$, this traversal sets $L_e$  as the node corresponding to the \blockstmt between Lines~\ref{lst:line:handlers-run} and ~\ref{lst:line:handlers-run-end} of Figure~\ref{fig:vulnerabilty-example1}(a). Next, in Line~\ref{lst:line:offseteditindex} of \strategyone, the index \I{I} is set to the index corresponding to the  syntactic child of the edit-location $L_e$ who is an ancestor of the semantic location $L_s$ . For $P$, this index  materializes into $13$; the edge  outgoing from blue \blockstmt in Figure~\ref{fig:example1-pdg} to an ancestor of semantic location (shown in red) has label \T{ch:13}. Next, we materialize the \eastree defined in Line~\ref{lst:line:eastree} of \strategyone by  materializing the  reference-nodes. The \eastree \I{O} serializes into @if (REF1.hasOwnProperty(REF2)) { REF3 } @ where @REF1@, @REF2@, and @REF3@ correspond to \T{ReferenceAST} operators with locations as \reflocone, \refloctwo, and \reflocthree. \refloctwo traverses semantic-parent edges  from \semloc (Line~\ref{lst:line:goodref}) and materialize into @data.id@. Similarly, \reflocone and \reflocthree traverse syntactic children edges and materialize into @handlers@ and @foo(data);@ respectively. Thus, the \eastree \I{O} materializes  into @if (handlers.hasOwnProperty(data.id)) { foo(data); }@, which is the required repair. 

%When \strategyone is given the program \prog in Figure~\ref{fig:vulnerabilty-example1}(a) as input, then it first  traverses a path from the source node to the "semantic location"  \semloc using \DMethod{GetKleeneStar}{"SemChild"}{\DMethod{GetClause}{"Expr"}}\ in Line~\ref{lst:line:semkleene}. This leads to a \newtextsc{KleeneStarTraversal} with the stopping condition $\lambda n.\mathcal{T}[n] = \text{"CallExpr"}$. For the input program, the traversal skips through the semantic-child-edges 1-7 and reaches @foo@ in Line~\ref{lst:line:callerId-sink}. Next, it applies another \newtextsc{KleeneStarTraversal} starting from \semloc to reach \editloc in Line~\ref{lst:line:synkleene}. This traversal skips over syntactic-parent-edges and reaches the \blockstmt between Lines~\ref{lst:line:handlers-run} and ~\ref{lst:line:handlers-run-end}. Next, in Line~\ref{lst:line:offseteditindex}, the index \I{I} is computed as \DMethod{GetOffsetIndex}{Ls, 0}\ which means to pick the child-index of \editloc that has \semloc as its descendent. For our example, this index would materialize into the statement number in the block statement containing @foo@, which turns out to be $13$. Next, we instantiate the \eastree in Line~\ref{lst:line:eastree} which hierarchically defines the children-nodes or reference-nodes. The \eastree \I{O} deserializes into @if (REF1.hasOwnProperty(REF2)) { REF3 } @ where @REF1@, @REF2@, and @REF3@ correspond to \T{ReferenceAST} operators with locations as \reflocone, \refloctwo, and \reflocthree. \reflocone and \refloctwo use the semantic-parent edge traversals from \semloc (Line~\ref{lst:line:goodref}) and materialize into @handlers@ and @data.id@. \reflocthree performs a syntactic-child edge traversal from \editloc and materializes into @foo(data);@ thus materializing the entire \eastree \I{O} into @if (handlers.hasOwnProperty(data.id)) { foo(data); }@, i.e. the required repair. 

Now consider the repair strategy \strategytwo in Figure~\ref{fig:strat2}. This strategy shares a similar structure with the earlier strategy but differs in the way traversals and the index $\I{I}$ are computed. There are four key differences
\begin{enumerate}
    \item In order to reach \semloc from \prog.source, \strategytwo performs the \T{EdgeTraversal} using semantic-child edge seven times in Line~\ref{lst:line:nosemkleene}. The number of semantic edges varies widely across programs and prevents generalization to other scenarios. \T{KleeneStarTraversal} operator instead uses \newtextsc{Clauses} over nodes to find the edit-location.
    \item To reach $L_e$ from \semloc, \strategytwo performs the \T{EdgeTraversal} using syntactic-parent edge seven times in Line~\ref{lst:line:nosynkleene}. Consider a program that instead assigns output of the function-call @let out = foo(data)@. \strategytwo will find \assignexpr as the edit-location and fail to generalize whereas \strategyone will appropriately adjust and take four parent steps.
    \item In order to compute the index at which replacement needs to occur, \strategytwo uses a \DMethod{ConstantIndex}{13}\ in Line~\ref{lst:line:consteditindex} of Figure~\ref{fig:strat2}, which effectively assumes that replacement should always occur at 13$^{th}$ child of $L_e$ and again doesn't generalize. \strategyone on the other hand uses of \T{GetOffsetIndex} operator to instead compute index dynamically for a given input program
    \item In order to materialize reference nodes, \strategytwo uses syntactic edge traversals (Line~\ref{lst:line:badref} of Figure~\ref{fig:strat2}) which assume definite structure about the structure of the program (@GetConstant(7)@ used as syntactic child index to solve a long-ranged-dependency). \strategyone instead uses semantic-parent edges to capture the semantics here and produces a better generalizing repair.
\end{enumerate} 

\lstDeleteShortInline@

\noindent These programs highlight that our \dsl is expressive enough to perform complicated non-local repairs in a generic manner. At the same time, while many strategies can repair a given program, all applicable strategies are not equally good. A key realization is that we \emph{prefer shorter traversal functions} (\newtextsc{KleeneStarTraversal}\ over a long sequence of \newtextsc{EdgeTraversal}). Similarly, we \textit{prefer the traversals with none or small constants}. For example, we prefer \DMethod{GetOffsetIndex}{\semloc}{0}\  over \DMethod{GetConstant}{13}\ and semantic-parent traversal over syntactic-parent traversal with index \DMethod{GetConstant}{7}. %Finally, we also \emph{prefer strategies that share traversals across localizing \editloc and \refloc}.
We use these insights to guide the search in our synthesis algorithm.

% The strategy (\strategy) is defined by 
% performs this localization using an edit path (\editpath). We define a path (\genpath) in the strategy as a sequence of edges in the \prog. An edge is either a syntactic \astree edge or a semantic \taintpropedge in either direction (i.e. towards parent or child). 
% The localized node in the \prog is called edit location (\editloc). Next, the strategy either inserts or replaces a child of the edit location with a new \astree. This new \astree can either be a constant node or reference a node in the original \prog using a reference path (\rfpath). Figure ~\ref{fig:approach-notations} summarizes the notations 

% This \dsl was created so we can use the \sa annotations seamlessly and is guided by how humans fix such vulnerabilities. A line of previous works~\cite{} manually write repair patterns for fixing code. Our \dsl-based approach is strictly more general as it can perform various kinds of repairs and the exact repair strategies are learned from data. Moreover, we make effective use of high-level patterns and domain insights, and annotations. So while these other approaches tend to be simplistic and \textbf{either do not generalize well or over-generalize (generate a large number of false positives)}, concrete instantiations of strategies in our \dsl are better at capturing the high-level repair intent better. Following we describe the terminologies used in the repair \dsl.

% \lstMakeShortInline[columns=fixed]@
% The top level rule in our \dsl defines the Repair Strategy (denoted by \strategy). It is parameterized by the type of vulnerability the strategy fixes and the edit \edit. We consider two kinds of edits, either an insert operation or a replace operation. This means that the edit \edit either inserts an \astree child or replaces an \astree child with another \astree. Since we are fixing taint-flow vulnerabilities, we found these two operations to be sufficient. However, our \dsl can be expanded to also handle delete operations \aksays{Why can't we say that delete is replacement with an empty tree?}. In Figure ~\ref{fig:vulnerabilty-example1}, the fix applied in replaces the \astree corresponding to @handlers[callerId](data)@ (line ~\ref{lst:line:callerId-sink}, Figure ~\ref{fig:unsafememberex}) with the if statement in lines ~\ref{lst:line:fix-start}-\ref{lst:line:fix-end}, (Figure ~\ref{fig:safememberex}) and depicts a replace edit. 

% %\paragraph{Edit (\edit)} Since we are solving source-sink-sanitizer vulnerabilities, our \edit either inserts child \astree at edit-locations (denoted by \editloc) or replaces a child with another \astree (at \editloc). %This \editloc is a node in the \pdg which is reachable from the vulnerability source (as provided by the \astree) by traversing syntactic (\astree) or semantic edges in \pdg. %Once the \editloc is found, the new \astree (either replacing the existing child being inserted as a child) can be   

% Notice that in the \pdg, @handlers[callerId](data)@ is a child of the \blockstmt (parenthesis block between lines ~\ref{lst:line:handlers-run}-\ref{lst:line:handlers-run-end} and marked in blue in Figure ~\ref{fig:example1-pdg}). So while applying the fix, we replace the \textit{$k^{th}$} child of \blockstmt with the \ifstmt. We call the node in the \pdg where the edit operation applies as the edit location (\editloc). When a strategy applies, it has to determine this edit location based on the \pdg structure. Our \dsl defines an edit path (denoted by \editpath) to find edit location. In Figure ~\ref{fig:example1-pdg}, starting from the source-node @event@ (marked in orange), we take 7 semantic edges (reaching @callerId@) and then after hopping four synactic parent edges we reach the edit location \aksays{The notion of semantic edges should be defined and explained before this.}. This sequence of edge traversal defines our edit path. More generally, our \dsl considers the \editpath to be a set of semantic edges followed by a set of syntactic edges. This constraint on the paths allows expressivity to learn general strategies while also keeping the search space small. The semantic edges in \editpath allows navigating to ``somewhere close'' to sink location. Next once semantic edges are traversed, \editloc is reached by traversing a set of syntactic edges. Additionally, since the number of semantic edges might vary across examples, our \dsl allows a powerful  operator that navigates an indefinite number of semantic edges. This formulation helps our strategies to generalize well across widely different sets of programs. 

% %\paragraph{Edit Location (\editloc)} Edit location is the node in the \pdg where the edit operation (i.e. insertion or replacement of a child node) applies. \editloc is reachable from the vulnerability source found by traversing the edit-path (\editpath) in the \pdg.  So, in our running example, "$\dots$ function (data){$\dots$}" node (marked in blue) is the edit root and is reachable from the source via first traversing the semantic edges followed by traversing to "syntactic parent" four times. 
% Our kleene-star operator navigates indefinite semantic child edges until a ``stopping node'' (parameterized by a stopping condition) is reached. This stopping-condition is defined by a set of predicates applied on a \aksays{an} \astree node. We find that simplistic predicates about \newtextsc{ASTType} or \newtextsc{ASTValue} of \astree node and its neighbours suffice in locating this stopping node. For our running example, the stopping condition is the conjunction of the predicates @ASTType(node.parent) = IndexExpr@, @ASTType(node.parent.parent) = MethodCallExpr@. The stopping node lies on the taint-flow path from source to sink and therefore is quite relevant to the insert or replace operations (being a proxy for the semantic information of the vulnerability). Therefore, we call the stopping node as the semantic location (\semloc). In our running example, @calledId@, the sink-node is also the \semloc.

% As described above, our \dsl either inserts a new \astree, or it replaces an existing \astree with another \astree. An \astree is defined by three properties -- type, value and an array of children \astree. One can construct such an \astree by concretely initializing it using specific types and values for the tree and its descendants. However, a constant \astree cannot generalize because the fix depends on existing variables in the source code. Therefore, in addition to a constant \astree, our \dsl also allows referring to any existing node in the \astree. This referral is computed by traversing a path from the semantic location (\semloc defined above) to the node to reference \aksays{What is node to reference?}. The corresponding path is known as reference path. For e.g. the condition @handers.hasOwnProperty(callerId)@ which is used in the fix refers to @handlers@ and @callerId@ nodes in the tree and combines them in a constant \callexpr \astree. 

% We described edit paths and reference paths above. More generally a path is a sequence of edges in the \pdg where the edge can be one of syntactic or semantic or ancestral. When selecting a child edge, we also need to store \textit{which child} to select and it is determined by an index. Figure ~\ref{fig:repair-strategy-ex1} shows the entire strategy that fixes the example in Figure ~\ref{fig:vulnerabilty-example1}.
% \lstDeleteShortInline@

% \paragraph{Semantic Location (\semloc)} The node at which Kleene-Star traversal of semantic edges stops is called semantic location. This node is a key component in the repair since this node is a proxy for the semantically important values that would be necessary for making the edit. The sink node, callerId, is also the \semloc in our example. %Additionally, the \editloc is near this node and 

% \paragraph{Paths (\dslpath)} A path is described as a sequence of edges in the \pdg. The edges can be syntactic parent or child edges, semantic child edges, and ancestor edges in the \pdg.

% \paragraph{Index (\dslindex)}
% While selecting a child edge or while determining where the \concinsertcode needs to be inserted or replaced with, we need some index of which child to follow. Generally, this index is a constant value however it can be computed as an offset from the \semloc descendent direction as computed with the \newtextsc{OffsetFrom} operator. \naman{explain the requirement of offset with example}

%\paragraph{\astree} \astree is the tree-representation of the editcode that will replace some existing child of \editloc or will be inserted at some child indices of \editloc. One possible way to construct this \astree is to concretely initialize it using specific values and types. However, a constant \astree cannot generalize because the inserted code almost always depends on specific variables and the structures in the code. Therefore, in addition to a constant structure, the \astree can also refer to existing elements in the \pdg. This referral is again found using a path (\dslpath) traversal from the \semloc, the intuition being that it is a good \textit{proxy for the semantics of the vulnerability} and necessary variables to refer would be close to it.  


\subsection{Synthesizing \dsl strategies from examples}
\label{subsec:synthesis}
%\naman{The discussion about anti-unification would go in related work I presume?}

Given this high-level \dsl, we will now describe our example-based synthesis algorithm. 
We take as input a set of unsafe programs and edits generated as output at the end of data collection step (Section~\ref{sec:data}). 
Let $\{(\prog_{1},\edit_1),(\prog_{2},\edit_2),\dots$ $,(\prog_{n},\edit_n)\}$. 
Here, $\prog_{i}$ is the $i^{th}$ unsafe program and $\edit_i$ is the corresponding edit. Edit ($\edit$) contains the \astree-node of the edit-location ($\edit$.loc), the \textit{concrete} \astree of the edit-program ($\edit$.editprog), and the type of edit i.e. \insertsc or \replace ($\edit$.type). We use these to learn high-level repair strategies in our \dsl. 
\lstMakeShortInline[columns=fixed]@
Our goal is to combine specific paths, learned over examples that share similar repairs in different semantic and syntactic contexts, to obtain general strategies. Our repair strategies abstractly learn the following:
% \begin{enumerate}
%     \item the traversal for localizing edit-locations (\editloc) and reference-locations (\refloc). 
%     \item template-repair-program-representations using the reference-traversals . 
% \end{enumerate}
\begin{enumerate}
    \item Traversals for localizing edit-locations (\editloc) and reference-locations (\refloc). For example, @Ls@ in Line~\ref{lst:line:semkleene} (Strategy \strategyone) depicts a \T{KleeneTraversal} abstraction we can learn from examples having a variable number of semantic-edges. Similarly, @I@ in Line~\ref{lst:line:offseteditindex} (of \strategyone) depicts a generalized index expression we can learn from examples.
    \item \eastree which use reference-traversals. For example, @O@ in Line~\ref{lst:line:goodstratO} demonstrates templated-program-structure that we can learn from examples (say by generalizing from the witnessed guards @handlers.has(data)@  and @events.storage.has(event.name)@).
\end{enumerate}
\lstDeleteShortInline@


% In particular, we wish to abstract over examples that share similar repairs in different semantic or syntactic contexts. Consider the example abstractions below: 
% \begin{enumerate}
%     \item  
%     \item Line~\ref{lst:line:eastree} in Figure~\ref{fig:strat1} depicts 
%     the guard condition in Figure~\ref{fig:safememberex} @handlers.hasOwnProperty(data.id)@ can be abstracted with another guard @eventHandlers._storage.hasOwnProperty(event.name)@ into an abstract template @REF1.hasOwnProperty(REF2)@ where @REF1@ and @REF2@ are \T{ReferenceAST} have use traversals
% \end{enumerate}
% For example, . 
%Consider the example in Figures~\ref{fig:static-witnessing-1},~\ref{fig:static-witnessing-2}, and ~\ref{fig:static-witnessing-3}. They describe three different kinds of syntactic repairs and are not candidates to merge. Instead, 

We depict our synthesis algorithm in Figure~\ref{fig:strategy-learning}. At a high-level, our synthesis algorithm, first pre-processes the inputs, storing the required \textit{concrete} traversals. Next, it performs ranked pair-wise merging over the processed edits to synthesize strategies.
%We merge non-terminals recursively by deductively choosing production rules to enumerate and merging the non-terminals appearing in the productions. %During this recursion, it learns the \textit{traversals} and program templates abstractly. 

\noindent \textbf{Pre-processing.} In this step, given the programs and edits, we store the concrete traversals required for learning \editloc and \refloc (Line~\ref{algo:line:preprocess}). Naively computing all such traversals is very expensive and also leads to \textit{bad strategies}. Here, based on the insights from Section~\ref{subsec:examples}, we only compute the traversals that lead to shorter  
traversals
%\textit{abstract traversals}
which generalize better. In addition, we also share traversals between between \editloc and \refloc. Pre-processing has following three key steps:
\begin{enumerate}
    \item \textbf{Edit Traversals.} We compute the traversals between \prog.source and \editloc (Line~\ref{algo:line:conceditloc} of Figure~\ref{fig:strategy-learning}) that have the form of a sequence of semantic-edges followed by a sequence of syntactic-edges. This allows abstracting variable-length sequences of semantic-edge traversals as a \kleeneedge (corresponding to an abstract \newtextsc{KleeneTraversal}). We implement this using \newtextsc{BiDirecBFS} method at Line ~\ref{algo:line:bidirecbfs}. For every edit-traversal ($\I{T}_e$), we define {\em semantic-location} (\semloc for brevity) as the last-node on the semantic (dataflow) traversal before traversing the syntactic-edges.
    \item \textbf{Compressing Edit Traversals.} We compress these edit-traversals using the \newtextsc{Compress} method in Line~\ref{algo:line:compress}. It takes in a sequence of (syntactic or semantic) edges as input, greedily combines the consecutive edges with the same edge-type (\edgetype) into a \kleeneedge. The \kleeneedge is constructed using the edge type \edgetype, and a set of clauses $\clause_i$ that satisfy the target node of \kleeneedge. These clauses are either $\lambda n. \mathcal{T}(n) = \tau$ that check the type  or $\lambda n. \mathcal{T}(F_i(n)) = \nu$ that check the type of a neighbor. \newtextsc{Compress} returns a sequence of edges or \kleeneedges as output. 
    \item \textbf{Reference Traversals.} For every node of the edit-program, we locate nodes in the \pdg with the same \textit{value} using a \newtextsc{LevelOrderBFS} until a max-depth (Line~\ref{algo:line:maxlevel}). We perform this traversal from \semloc (defined in (1) above). We thus share parts of traversals between locating \editloc and \refloc which helps in learning \textit{better strategies}. The motivation behind using \semloc is that the expressions necessary for repair will be close to \semloc as it lies on the information-flow path. 
\end{enumerate}
%Specifically, for \editloc, we find the traversals between \prog.source and \editloc (Line~\ref{algo:line:conceditloc}) that first navigate a set of semantic-edges followed by a set of syntactic-edges. We implement this using the \newtextsc{BiDirecBFS} function at Line ~\ref{algo:line:bidirecbfs}. %It traverses semantic-edges from the source, syntactic-edges from the edit-location, and returns the intersecting traversals. 
%For every edit-traversal ($\I{T}_e$), we define semantic-location (\semloc for brevity) as the last-node on the semantic (dataflow) traversal before navigating a syntactic-edge. 
%Next, we compress the traversals greedy by combining consecutive edges of the same edge-type (\edgetype) into a \kleeneedge using the \newtextsc{Compress} method in Line~\ref{algo:line:compress}. Every \kleeneedge stores the \edgetype, and a set of clauses $\clause_i : i \in {1,\dots,n}$ that satisfy the end-node of \kleeneedge. These clauses are either $\lambda n. \mathcal{T}(n) = \nu$, i.e. a clause on the type of \semloc or $\lambda n. \mathcal{T}(F_i(n)) = \nu$, i.e. a clause on the type of a syntactic-neighbour of \semloc. We then compute traversals for finding reference locations. Here, instead of computing traversals from the source-node, we instead compute traversals from the semantic-locations. The expressions referenced in repairs are usually close to the \semloc (as it lies on the information-flow path and thus is affiliated with variables likely necessary for building the repair). This traversal-sharing optimizes the search and generalization of our strategies.

\noindent \textbf{Strategy Synthesis.} Given the edits and the associated traversal meta-data, we synthesize the strategy by pair-wise merging  (Line~\ref{algo:line:callmerge}). \newtextsc{MergeEdits}, the top-level synthesis method, takes a pair of edits as inputs and returns a list of strategies satisfying the example edits. We synthesize the strategies recursively using a deductive search over the non-terminals of the DSL (Figure~\ref{fig:fixing-dsl}). Specifically, to synthesize an expression corresponding to a non-terminal, we deduce which production to use and recursively synthesize the non-terminals given by its production-rule. This has the following key components: 
\begin{enumerate}
    \item \newtextsc{MergeEdits}: It takes pairs of edits as inputs and returns the strategy. It recursively synthesizes the traversal (for \editloc), index, and \eastree. It combines and returns them using the edit-type. 
    \item \newtextsc{MergeTraversal}: It takes two concrete traversals (sequence of edges or \kleeneedges) as inputs and returns the abstracted traversal. by merging elements in the sequence.
    \item \newtextsc{MergeEdge}: It takes two edges or \kleeneedges as inputs and returns a \T{GetKleeneTraversal} or \T{GetEdgeTraversal}. We combine two \kleeneedges using their edge-types and intersecting the clauses stored during pre-processing. We combine two edges using their edge-types, and recursively combining their indices.
    \item \newtextsc{MergeIndex}: It takes two integer indices as inputs and returns an abstracted index. If the two input indices are equal, we return a \T{GetConstant} operator with the input index value. Otherwise, we compute offset as the difference between input-index and index of child of $n$ which has \semloc as descendent (computed by $DO(n, \semloc)$). We return this offset if they are equal and an empty-list otherwise. 
    \item \newtextsc{MergeProg}: It takes two programs as input and returns a list of \eastree, where each list element can materialize into the input programs. If the top-level node in the programs have equal values and types, we combine them as a \T{ConstantAST}. Otherwise, we recursively combine their children. Finally, we merge the reference-traversals corresponding to the input programs and combine them into \T{ReferenceAST}.
\end{enumerate}

Our synthesis procedure is inspired by anti-unification~\cite{anti-unification} and we abstract the paths and edit-programs across different examples. Specifically, our \T{KleeneTraversal} and \T{OffsetIndex} functions allow generalization across paths having different number of edges and different indices where naive abstractions fail. Similarly, \eastree also resemble anti-unification over tree-edits. However, again we use traversals over syntactic and longer-context semantic-edges, for better generalizations  and repairs. 

Finally, note that while we perform pair-wise merges over the edits, the strategy synthesis algorithm can be extended to merge bigger cluster of edits together as well. However, from our experience, we find that the pair-wise merging performs well and is sufficient for our experiments. 
%by recursively synthesizing values corresponding to the non-terminals in our \dsl. 
%Thus, to synthesize a strategy, we synthesize traversals for the \editloc. To synthesize a traversal, we check if the two traversals contain an equal number of edges. Next, we try to merge the corresponding pairs of edges on each traversal. To merge an edge, if it is simply a syntactic or semantic edge, we instantiate a \T{GetEdge} operator appropriately. However, if the two edges are \kleeneedge, then we instantiate a \T{GetKleeneEdge} operator where the clauses are constructed by intersecting the clauses computed in \newtextsc{Compress} step. It additionally ensures that the number of clauses after the intersection is more than $1$ to prevent over-generalizing strategies. In order to merge strategies, we next merge the edit-programs (\edit.editprog). We first combine the programs as a \T{ConstantAST} by checking whether the type and value match and then recursively merge the children and finally take the cartesian product of children. Next, based on the concrete reference traversals, we merge them and construct \T{ReferenceAST} \eastree. 



%Note that in order to learn repair strategies, we need to synthesize paths corresponding to all locations (\location) used in the strategy. We use \location in three production rules in our \dsl (\locationone, \locationtwo, \locationthree in Rules~\ref{dslrule:strategy},~\ref{dslrule:index},~\ref{dslrule:node}). Naively, trying to synthesize these paths is very expensive and will lead to non-generalizing strategies.
%Here, based on common fix-patterns for these vulnerabilities, we reduce the search space by enforcing structure over the paths we learn. Additionally, we also share paths between the three locations. 

%We find that performing these 

%At a high-level, our synthesis algorithm takes these unsafe-programs and edits as inputs, preprocesses them to store relevant meta-data, clusters them and then recursively enumerates over the non-terminals in the \dsl. How


% \aksays{This section is very dense. It would be good to take a small example and illustrate the key steps visually.}
\input{synthesisalgo2}

% Given this high-level \dsl, we will now describe our synthesis algorithm. We build a top-down synthesis algorithm that learns strategies in this \dsl through a \pbe approach. We receive a set of unsafe codes and concrete edits \unsure{the terminology of concrete edits can be confusing for the paper. Basically, anologous to all things in \dsl, we have concrete edits, paths, etc.}. Let $\{(\prog_{1},\concedit_1),(\prog_{2},\concedit_2),\dots$ $,(\prog_{n},\concedit_n)\}$ be the data we collect from our data collection step where $\prog_{i}$ is the ith unsafe code snippets and $\concedit_i$ is the corresponding concrete edit. $\concedit_i$ contains the concrete edit-location $\conceditloc_i$, the \astree of the editcode $\concinsertcode_i$, and the type of edit i.e. \insertsc or \replace.

% Given this data, we instantiate our algorithm to learn repair strategies. Our algorithm takes in these set of examples and learns a set of ($\{\strategy_1,\strategy_1,\dots,\strategy_k\}$) that are supposed to cover the training examples. Later, these repair strategies, when given an unsafe program \pdg will generate the edit that needs to be applied. The sketch of our synthesis and learning algorithm is presented in Figure~\ref{fig:strategy-learning}.

% %%O := \{\(\strategy\sb{1},\strategy\sb{2},\dots,\strategy\sb{k}\)\}
\HUGE DONT USE THIS - USE synthesisalgo2
% Figure environment removed

% \unsure{\textbf{Terminology for reference!!:} Letters with overlines are concrete elements ($\concedit$ is a concrete edit, $\conceditpath$ is a concrete edit path) while the letters without lines are abstract elements that can generalize over examples ($\edit$ is an abstract edit, $\editpath$ is an abstract edit path). Following, we again define the various abbreviations used in the algorithm
% \begin{itemize}
%     \item \prog is the \pdg containing \sa annotations
%     \item \concedit is the concrete edit which itself contains editcode, edit-location, edit-type, and indices
%     \item \concinsertcode is the concrete editcode that is either inserted or replaces some existing region in the unsafe code. \concinsertcode itself can be represented as an \astree
%     \item \conceditloc is the concrete edit location i.e. where the edit takes place in a \prog
%     \item \concpath is a concrete path as a sequence of edges
%     \item \conceditpath is a concrete edit path from source to \conceditloc
%     \item semLoc or semantic location is the stopping node of \conceditpath
%     \item \concrefpath is a concrete reference path from semantic location to a particular node matching value of a \astree node in \concinsertcode
%     \item edit-type refers to whether edit is \insertsc or \replace
% \end{itemize}}

% \spsays{Consider breaking these into subsections and have a running example. for instance, first section can be running bidirectional bfs, other could be pairwise merging, and so on..} 

% Our top-level \newtextsc{Learn} method receives the programs and concrete edits as inputs (Line~\ref{algo:line:learn}). This method first invokes the \newtextsc{PreProcessConcEdit} method which computes edit-paths and reference-paths in the $\concedit$. Next, \newtextsc{Learn} method ranks edits based on the similarity of their $\concinsertcode$ and in that order tries to combine edits together in a pairwise manner using the \newtextsc{MergeEdit} method.

% \newtextsc{PreProcessConcEdit} method (in Line~\ref{algo:line:preprocess}) stores the relevant paths in the $\concedit$ structure that will be useful during pairwise-merging. It first computes a set of edit paths from source to sink using a bi-directional breadth-first search (\newtextsc{BiDirectBFS}) (Line~\ref{algo:line:conceditloc}) and stores it in the edit. Note that during this \newtextsc{BiDirectBFS}, it traverses only semantic edges from the source and only syntactic edges from the edit location. This naturally leads to paths that follow the required pattern of a set of semantic edges followed by a set of syntactic edges. Note that edit-paths also contain the semantic-locations in the semLoc field. Then for every edit-node in the edit code and every semantic location in the edit path, it stores reference paths from semantic locations to nodes in the \astree having the same value as edit-node (Line~\ref{algo:line:concrefpath}. These paths are computed using a \newtextsc{MaxLevelBFS} until a certain depth and storing the satisfying nodes. Note that during implementation, we memoize the path-finding steps to avoid repeating computations.

% \newtextsc{PairSimilar} method computes a score for every pair of edits in the edit set. To compute the similarity for a given pair of edits it performs \newtextsc{ASTSimilarity} on their editcodes. 

% \newtextsc{MergeEdit} is the top-level method of our deductive top-down synthesis algorithm. The merging procedure is intuitive. For every element in the edit, it recursively calls \newtextsc{Merge} operation on the elements and then assembles an edit using their outputs. Specifically, this method first ensures that edits are of the same type (\insertsc or \replace. Then it obtains a set of candidate edit-paths by calling the \newtextsc{MergeEditPaths} method on the concrete edit paths stored in the input concrete edits. Next for every candidate edit-path, it finds a candidate editcode using the \newtextsc{MergeEditCode} method. Finally, for every editpath and editcode pair, it assembles the final edit (Line~\ref{algo:line:assembleedit})
    
% \newtextsc{MergeEditPath} method tries to merge two concrete editpaths. It first computes a set of intersecting predicates over the semantic-locations of the two paths. Using the predicates, it builds the \semkleene edge. Finally, over the remaining set of edges in the edit-path, it calls the \newtextsc{MergePath} method which inturn merges all the edges successively (Line~\ref{algo:line:mergepath})


% \subsection{Applying the learned strategies}
% \label{subsec:applying}
% Given an unsafe-program and a set of repair strategies, we apply each strategy to the program to generate candidate repairs. 
% To apply a single repair strategy, we use the definitions of operators described in Section~\ref{subsec:dsl} to generate candidate repair programs. We find that, in practice, we obtain a few distinct repairs and we return them as the output of our system. 

%Once these high level repair strategies are learnt, applying them is natural. For an unsafe program, strategy \strategy ingests the \pdg of the program. Then it tries to build an edit, by first locating the edit location (using the edit path \editpath) and building the \astree recursively depending on whether it is a constant or reference tree. If edit location and \astree are generated, the edit operation is applied appropriately based on whether it is an \insertsc or \replace edit. Otherwise, if either of location or \astree is not generated then no edit is applied. 
% \subsection{Scribblings}
% Each node $n$ of the AST has an identifier $\mathit{id}\in\mathbb{N}$. The AST is characterized by a set $\mathcal{N}$ of node $\mathit{id}$s, i.e., $\mathcal{N}=\{\mathit{id}_0,\ldots\mathit{id}_k\}$. We have a map $\mathcal{T}$ from nodes to their types, i.e., $\mathcal{T}(n)=\tau$, where the primitive types $\tau$ include {\sc MethodCallExpr}, {\sc IndexExpr}, etc. We also have a set $\mathcal{E}$ of edges, where each edge is $(n_1,n_2,ET,z)$. Here, $n_1$ is a source node, $n_2$ is a target node, $ET$ is the type of edge (syntactic parent, syntactic child, semantic parent, or semantic child), and $z$ is a child's index (set to $-1$ if the edge is a parent edge). 

% The strategy $S$ is of two types, insert an AST $O$ at index $I$ of location $L$, $\mathit{Insert}(L(\mathit{source}),I,O)$, and replace the AST at index $I$ of location $L$ with $O$, $\mathit{Replace}(L(\mathit{source}),I,O)$. Each $L(n)$ takes a node $n$ and traverses a path to reach a location, i.e.,  $L(n)=\mathit{ApplyPath}(n,F_k\circ\ldots\circ F_0)$, where the output node is $F_k(F_{k-1}(\ldots F_0(n)\ldots)$. Each $F$ instantiates the edge traversal function $TE[I,ET,C]$ with an index $I$, an edge type $ET$, and an optional clause $C$ (relevant for KleeneEdge). We define $TE[I,ET](n)$ as $let\ i=I(n)\ in\ let\ N=\mathcal{E}(n,ET)\ in\ N[i]$, which gets an integer index $i$ of children of $n$, gets a set $N$ of nodes by dereferencing edges of type $ET$ from $n$ and returns the $i^{th}$ child of $n$.
% $F$ can also be $TE[ET,C](n)\equiv KE(n,ET,C)$.
%  The KleeneEdge $KE$ keeps dereferencing edges of type $ET$ till it hits a node where a clause $C$ holds, i.e., $KE(n,ET,C)$ is defined as $C(n)? n : \left(let\ t=\mathcal{E}(n,ET)\ in\ KE(t,ET,C)\right)$. The node $t$ which is target of an edge with type $ET$ and $n$ as a source here is chosen non-deterministically and our implementation resolves this non-determinism through a breadth-first search. A clause is a conjunction of predicates of the form $\lambda n. \mathcal{T}(L(n))=\tau$. The index $I$ is either an integer $z$ or of the form $\lambda n.DO(n,L(\mathit{source}))+z$, where $DO(n_1,n_2)$ returns the index of syntactic child of $n_2$ who is a syntactic ancestor of $n_1$. 
 
%  A strategy can fail to apply if in $DO(n_1,n_2)$ there is no path from $n_2$ to $n_1$,

\section{Implementation}
\label{sec:impl}
%-------------------------------------------------------------------------------
\section{Implementation} \label{imple}
%-------------------------------------------------------------------------------

% Figure environment removed

\sys's implementation consists of (\romannumerber{1}) a fully-functional \sys switch prototype which implements the overall in-fabric logic; and (\romannumerber{2}) a set of software APIs exposed to applications. Our prototype is built upon a FPGA-assisted commodity switch while a P4 programmable switch implementation is also provided.

%\sys's implementation consists of (\romannumerber{1}) a FPGA-based fully-functional prototype which implements the self-defined switch logic and (\romannumerber{2}) a set of software APIs exposed to multicast applications. \sys's switch logic can also be implemented on the Tofino P4 switch, as described in $\S$\ref{dis}.

\parab{FPGA-based prototype.} We implement the group registration, data packet duplication, header modification, and feedback aggregation logics on an FPGA board. The board is equipped with a commodity FPGA chip~\cite{ultrascale} and four 100Gbps Ethernet interfaces. The FPGA resource utilization is shown in Table~\ref{tab:overhead}. We build our testbed with the FPGA board, a commodity Ethernet switch, and four servers, as illustrated in Fig.~\ref{fig:fpgaprototype}. Each server is equipped with a commodity RNIC. The FPGA board and four RNICs are connected to the commodity switch through 100Gbps Ethernet interfaces. 

The commodity switch is configured by Access Control List (ACL) to route the servers' multicast traffic to the FPGA board. The FPGA board identifies the multicast data (ACK\footnote{In Fig.~\ref{fig:fpgaprototype}, we use ACK to represent all types of feedback.}) packets through the specific packet header by \textit{Parser} and \textit{Arbiter}. The data (ACK) packets will be duplicated (aggregated) by \textit{Duplicator} (\textit{ACK Aggregator}). The resulting packets will be pushed in \textit{Queue System}, waiting for the \textit{Multiplexer} to schedule in case for queue competition. Finally, the duplicated data (aggregated ACK) packets are sent back to the commodity switch. During processing, the \textit{Multicast Forwarding Table} is accessed when needed. %\todo{describe Fig. 8} Note those resulting packets' destination IP would be unicast IPs, so the switch routes them as normal unicast packets.
%

\parab{P4-based implementation.}
\sys in-fabric logic can be implemented on the P4 switch as well with some special handling. For the one-to-many data forwarding, P4 switch duplicates packets at the Traffic Manager (TM). The extended table states in Fig.~\ref{fig:table} are stored in the egress pipeline, indexed by <GroupID, EgressPort>. Because the lookup key in P4 switch is at most 32bits, we can use the least significant 24bits of GroupIP plus the 8bits port number as the real index. 

For the many-to-one feedback aggregation, there are two challenges due to the limited capability of P4 switch. Commodity P4 switch switch contains many stages, each with minimal computation capability and independent memory. Firstly, a single stage cannot support the \textit{wrapped-around} PSN comparison. To handle it, we simplify the standard PSN comparison to match the stage's capability, resulting in a tighter PSN space reduced from $2^{23}$ to $2^{22}$. Secondly, a single stage cannot iterate the entire table entries and find the minimum PSN. To handle it, we leverage multiple stages, each responsible for partial entries. Thus, the maximum number of entries supported in each multicast group is limited by the total stage number.

Besides, the P4 switch lacks the computation capability to recalculate the Invariant Cyclic Redundancy Checksum (ICRC) for the modified (aggregated) data (ACK) packets. As a result, those packets would violate the ICRC validation and be discarded at the receiver. This is why we choose the FPGA-based prototype to evaluate \sys in this work. However, a recent work shows that some RNICs provide the ability to bypass ICRC validation~\cite{switchML}. 
%Although this can work, it's a compromised method having security risks. 
%So we select the more integrated FPGA-based prototype to evaluate in $\S$\ref{eva}. Other operators can choose their preferred implementation based on their requirements.

%\begin{algorithm}[t]
%	\caption{Update PSN record and find PSN minimum in P4}\label{alg:psncomp}
%	\begin{algorithmic}[1]
%		%\Function{Generation}{}
%		\State $ack.psn, ack.port\gets $ the PSN and port of ACK packet
%		\State $rec.psn, rec.port\gets$ the PSN and port recorded
%		\State $isTrigger\gets$ whether the packet is a trigger packet
%		%\State $last\_ack\_psn\gets$ last aggregated ACK's psn
%		%\State $min\_port\gets$ port with minimum $ack\_psn$ last time
%		\State min.psn = ack.psn;
%		\State \textcolor{purple}{// every stage compare the PSN}
%		\If{$ack.psn > rec.psn$ or $(ack.psn <= 24'b3fffff$ and $rec.psn >= 24'b600000)$}
%			\State min.psn = rec.psn;
%			\If{rec.port == ack.port}
%				\State rec.psn = ack.psn;
%			\EndIf
%		\EndIf
%		\State \textcolor{purple}{// last stage write the min PSN back}
%		\If{isTrigger} 
%			\State ack.psn = min.psn;
%			\State Forward ACK.
%		\EndIf
%	\end{algorithmic}
%\end{algorithm}

\parab{Software APIs.} We provide various communication libraries and middleboxes for \sys multicast support. Take the commonly-used OpenMPI as an example, we modify the OpenMPI (v4.1.1)~\cite{openmpi} and UCX (v2.3)~\cite{ucx} to adapt to \sys's design, as shown in Fig.~\ref{fig:fpgaprototype}. Specifically, we add a new implementation of $MPI\_Bcast$ and modify UCX for multicast QPs creation and data transmission. When the new $MPI\_Bcast$ is called, the MPI process calls the UCX to establish QPs for multicast. Multicast members exchange their QPs information, and the handshake starts, as described in Appendix \ref{apx:regis}. Once the multicast group is successfully established, the UCX finally calls the RDMA primitives defined in the well-known \textit{libibverbs}~\cite{libibverbs} to transmit data. The software modifications at the end-host are transparent to the upper-layer applications and don't require any RNIC or RDMA driver modification.
%\parab{Coalescence of unicast and multicast}
%When we design \sys, there is a question in our mind: \textit{which is better, maintaining unicast and multicast transports separately at end-host, or utilizing the in-network support to enabling them to match the same transport?} Because of the long-standing resource limit in RNICs and the emeging trend of shifting appropriate computation task to programmability network, we believe the latter is the correct selection.

\begin{table}[t]
	\small
    \centering
%	\begin{center}
%    \begin{tabular}{l|c|c|c}
    \begin{tabular}{|p{0.2\linewidth}|p{0.18\linewidth}|p{0.18\linewidth}|p{0.18\linewidth}|}
    \hline
    \textbf{Resource} & \hfil \textbf{LUT} & \hfil \textbf{Register} & \hfil \textbf{BRAM} \\
    \hline
   	\textbf{Usage} & \hfil 53169 & \hfil 15391 & \hfil 188 \\
    \hline
    \end{tabular}
%    \end{center}
    \caption{Resource usage of the \sys in-fabric logic.}
    \label{tab:overhead}
    \vspace{-0.25cm}
\end{table}

%\parab{Resource overhead}  Note that the size of multicast forwarding table is determined by the number of ports of the switch and doesn't scale up with the multicast group size. 2.7MB memory can support upto 1K multicast groups, which is satisfied in datacenter. We provide a detailed calculation of the maxmum group support in Appendix \ref{apx:cal}. 

\section{Experiments}
\label{sec:experiments}
\section{Experiments}
% \haizhou{Follow the same way of introduction as we did in Section2.}
% \noindent In this section, we will introduce datasets and experimental setups that we used. Then we evaluate our method, other self-supervised methods, and supervised methods under different distribution shifts (\ie, concept shifts and covariate shifts) under common settings (\ie, transductive, inductive settings). It has to note that we focus on node-level tasks (\eg, node classification) in this work. As for graph-level tasks, we leave it as our future work and some simple experiments can be found in Appendix~\ref{app:graph_classification}. 
In this section, we first introduce the experimental setup including datasets, training, and evaluation protocol in Section~\ref{sec:dataset}~and~\ref{sec:unsupervised}. 
% Next, we present our experimental setup and conduct extensive experiments to evaluate our method in Section~\ref{sec:unsupervised}. 
We then perform an ablation study to demonstrate the effectiveness of each proposed component in Section~\ref{sec:ablation}. 
Additionally, we analyze the impact of important hyper-parameters in Section~\ref{sec:sensitivity}. 
Subsequently, we integrate our method with various encoding models, showcasing the model-agnostic nature of our recipe in Section~\ref{sec:other_models}. 
Finally, we provide some qualitative results such as feature visualization in Section~\ref{sec:vis}.
It is important to note that we focus on node-level tasks (\eg, node classification) in this work. As for graph-level tasks, we leave it as our future work, while some simple experiments are also provided in Appendix~\ref{app:graph_classification}.

\subsection{Datasets}\label{sec:dataset}
There exist some benchmarks for evaluating graph out-of-distribution generalization~\cite{good,ji2022drugood,gds}. 
Among them, GOOD~\cite{good} is the most representative and comprehensive benchmark that curates more diverse graph datasets with diverse tasks, including single/multi-task graph classification, graph regression, and node classification involving more distribution shifts (\ie, concept shifts and covariate shifts). Hence in this work, we follow the evaluation protocol proposed in \cite{good}. Furthermore, we validate the effectiveness of our method in the datasets (\ie, Amazon-Photo, Elliptic) that are used in EERM~\cite{eerm}. The statistics and detailed introduction to these datasets can be found in Table~\ref{tab:dataset} and Appendix~\ref{app:datasets}.

\begin{table*}[htp]
\caption{The descriptions of datasets. ``Domain-Level'' means splitting by graphs, ``Time-Aware'' denotes splitting according to chronological order.``Word'' and ``Degree'' represent splitting according to word diversity and node degree respectively. ``Language'' means splitting by user language, suggesting the prediction should not be impacted by the language the user use. ``University'' denotes splitting according to the domain university, implying that the prediction of webpages should be based on word contents and link connections rather than university features. ``Color'' means that nodes are split according to node differences in covariate shift and color-label correlations in concept shift.}
\label{tab:dataset}
\centering
\begin{tabular}{cccccccc}
\toprule
Datasets     & Network Type        & \#Nodes & \#Edges & \#Attributes &\#Classes& Train/Val/Test Split     & Metric   \\
% Cora         & Artificial Transformation & 2,703   &         &              &         &                      & Accuracy \\
Amazon-Photo\footnotemark
             & Co-purchasing network      & 7,650   & 119,081   & 755          & 10      & Domain-Level         & Accuracy \\
Elliptic\footnotemark  
             & Bitcoin transactions       & 203,769 & 234,355   & 165          & 2       & Time-Aware           & F1-Score \\
GOOD-Cora    & Scientific publications    & 19,793  & 126,842   & 8,710         & 70      & Word/Degree          & Accuracy \\
% GOOD-Arxiv   & arXiv papers               & 169,343 & 2,315,598 & 128          & 40      & Time/Degree          & Accuracy \\
GOOD-Twitch  & Gamer network              & 34,120  & 892,346   & 128          & 2       & Language             & ROC-AUC  \\
GOOD-CBAS    & A BA-house graph           & 700     & 3,962     & 4             & 4       & Color                & Accuracy \\
GOOD-WebKB   & Webpage network            & 617     & 1,138     & 1,703         & 5       & University           & Accuracy \\
\bottomrule
\end{tabular}
\end{table*}
\footnotetext[5]{This dataset is adopted from~\cite{yang2016revisiting}. \cite{eerm} constructs ten graphs with different environment id’s for each graph.} 
\footnotetext[6]{The original is available on \hyperlink{https://www.kaggle.com/ellipticco/elliptic-data-set}{https://www.kaggle.com/ellipticco/elliptic-data-set}}

\subsection{Unsupervised Representation Learning}\label{sec:unsupervised}
\subsubsection{Transductive Setting}~\label{sec:trans}
% \noindent\textbf{Baselines.}\quad We conduct experiments with 12 baselines which consist of three categories: supervised methods and self-supervised generative methods, self-supervised contrastive methods. Specifically, we compare with three supervised baselines: empirical risk minimization~(ERM)~\cite{erm}, invariant risk minimization (IRM)~\cite{irm}, and a recent proposed graph OOD method dubbed EERM~\cite{eerm}. We also compare various unsupervised node-level representation learning methods: three self-supervised generative methods including GAE~\cite{gae}, VGAE~\cite{gae}, GraphMAE~\cite{gmae} and seven self-supervised contrastive methods: DGI~\cite{dgi}, MVGRL~\cite{mvgrl}, GRACE~\cite{grace}, RoSA~\cite{rosa}, BGRL~\cite{bgrl}, COSTA~\cite{costa}, SwAV~\cite{swav}. The descriptions of these methods can be found in Appendix~\ref{app:baselines}.
In this subsection, we focus on validating our proposed algorithm under the transductive setting, where the test nodes will participate in message passing~\cite{gilmer2017neural} during training following~\cite{good}. 

\noindent\textbf{Baselines.} We conduct experiments with 12 baselines from three categories: (i)~supervised methods, including empirical risk minimization~(\textbf{ERM})~\cite{erm}, invariant risk minimization (\textbf{IRM})~\cite{irm}, and a recent proposed graph OOD method \textbf{EERM}~\cite{eerm}; (ii)~self-supervised generative methods including Graph Autoencoder (\textbf{GAE})~\cite{gae}, Variational Graph Autoencoder (\textbf{VGAE})~\cite{gae}, Self-Supervised Masked Graph Autoencoders (\textbf{GraphMAE})~\cite{gmae}; (iii)~self-supervised contrastive methods including Deep Graph Infomax (\textbf{DGI})~\cite{dgi}, Contrastive Multi-View Representation Learning on Graphs (\textbf{MVGRL})~\cite{mvgrl}, Deep Graph Contrastive Representation Learning (\textbf{GRACE})~\cite{grace}, A Robust Self-Aligned Framework for Node-Node Graph Contrastive Learning (\textbf{RoSA})~\cite{rosa}, Bootstrapped Representation Learning on Graphs (\textbf{BGRL})~\cite{bgrl}, Covariance-Preserving Feature Augmentation for Graph Contrastive Learning (\textbf{COSTA})~\cite{costa}, Unsupervised Learning of Visual Features by Contrasting Cluster Assignments (\textbf{SwAV})~\cite{swav}. The detailed descriptions of these baselines can be found in Appendix~\ref{app:baselines}.

\noindent\textbf{Experimental setup.} We use the same graph encoder across different datasets for a fair comparison following~\cite{good}. We use grid search to find other hyper-parameters (\eg, learning rate, epochs) for different methods. For all experiments, we select the best checkpoints for ID and OOD tests according to results on ID and OOD validation sets following~\cite{good}, respectively. Experimental details and hyper-parameter selections are provided in Appendix~\ref{app:hyper}. For evaluating unsupervised methods, a linear classifier will be built on the frozen trained encoder after finishing pre-training. The reported results are the mean performance with standard deviation after 10 runs following~\cite{good}.

\noindent\textbf{Analysis.}\quad Based on the experimental results listed in Table~\ref{tab:trans_concept} and \ref{tab:trans_covariate}, we can draw the following conclusions: firstly, we find strong self-supervised methods (\eg, GRACE, BGRL, COSTA) are more robust to distribution shifts (concept shift in Table~\ref{tab:trans_concept} and covariate shift in Table~\ref{tab:trans_covariate}) compared to supervised methods. For instance, on GOOD-CBAS and GOOD-WebKB datasets, GRACE surpasses the best supervised method by large margins (over 6\% absolute improvement). Interestingly, we find the methods designed for OOD generalization (\ie, IRM) and graph OOD generalization (\ie, EERM) do not attain superior performance than the standard ERM on most of the datasets. For example, EERM shows superior OOD performance compared to ERM in only one experiment, and IRM outperforms ERM in four out of ten experiments across the conducted evaluations. This phenomenon is also observed in \cite{good,ahuja2020empirical,rosenfeld2021risks}, showcasing the challenge of achieving invariant prediction in non-Euclidean graph settings. 

Furthermore, our method surpasses other SOTA self-supervised methods on the OOD test set of all datasets by a considerable margin while achieving comparable performance in the in-distribution test set. For instance, on small datasets such as GOOD-CBAS and GOOD-WebKB, our method outperforms GRACE\footnote{MARIO is built up on GRACE according to our recipe. So, we make a comparison with GRACE here.} by over 2\% absolute accuracy on the OOD test set. On larger datasets such as GOOD-Cora and GOOD-Twitch, our method still outperforms other methods which shows its superiority. For instance, under covariate shift, MARIO surpasses other methods by over 7\% absolute accuracy on the GOOD-Twitch OOD test set. These statistics prove the effectiveness of our design.


\begin{table*}[htp]
\caption{Experimental results of all methods under concept shift. The bold font means the top-1 performance and the underline represents the second performance across the unsupervised methods. 'ID' represents in-distribution test performance and 'OOD' means out-of-distribution test performance. (OOM: out-of-memory on a GPU with 24GB memory)}
\label{tab:trans_concept}
\centering
\scalebox{0.95}{
\begin{tabular}{l|cc|cc|cc|cc|cc}
\toprule
\toprule
\multirow{3}{*}{concept shift} & \multicolumn{4}{c|}{GOOD-Cora}                   & \multicolumn{2}{c|}{GOOD-CBAS} & \multicolumn{2}{c|}{GOOD-Twitch} & \multicolumn{2}{c}{GOOD-WebKB} \\
                           & \multicolumn{2}{c}{word} & \multicolumn{2}{c|}{degree}& \multicolumn{2}{c|}{color}    & \multicolumn{2}{c|}{language}   & \multicolumn{2}{c}{university} \\
                           & ID         & OOD         & ID          & OOD          & ID            & OOD           & ID             & OOD            & ID            & OOD            \\
\midrule
ERM                        & 66.38±0.45 & 64.44±0.18  & 68.60±0.40  & 60.76±0.34   & 89.79±1.39    & 83.43±1.19    & 80.80±1.00     & 56.92±0.92     & 62.67±1.53    & 26.33±1.09     \\
IRM                        & 66.42±0.41 & 64.29±0.31  & 68.57±0.35  & 61.45±0.24   & 89.64±1.21    & 82.29±1.14    & 78.87±1.04     & 59.30±1.79     & 62.67±1.10    & 26.88±1.42     \\
EERM                       & 65.10±0.44 & 62.45±0.19  & 66.95±0.44  & 56.58±0.25   & 79.07±2.12    & 64.50±1.01    & OOM            & OOM            & 62.50±2.01    & 28.07±3.23      \\
\midrule
% Random-Init                & 37.53±1.74 & 32.12±1.24  & 37.82±1.71  & 27.74±1.14   &               &               &                &                & 60.33±2.21    & 27.07±1.70     \\
GAE                        & 60.65±0.89 & 58.00±0.55  & 62.59±1.11  & 53.44±0.80   & 75.28±1.36    & 68.07±2.05    & 81.25±0.81     & 51.51±1.05     & 62.17±3.34    & 25.78±1.85     \\
VGAE                       & 63.19±0.53 & 60.35±0.47  & 61.65±0.66  & 54.28±0.28   & 76.50±0.50    & 59.07±0.56    & 80.46±0.53     & 55.56±4.53     & 62.50±2.38    & 24.40±2.57     \\
GraphMAE                   & \underline{66.44±0.46} & \underline{64.87±0.30}  & 67.95±0.46  & 59.41±0.39   & 89.14±0.89    & 82.93±0.93    & 80.05±0.64     & 59.38±1.49     & 61.83±3.37    & 29.27±2.15     \\
DGI                        & 63.33±0.56 & 60.71±0.49  & 65.93±1.02  & 55.83±0.53   & 91.22±1.47    & 85.00±1.66    & 80.05±0.87     & 59.16±1.88     & 61.83±2.83    & 28.63±1.92      \\
MVGRL                      & OOM        & OOM         & OOM         & OOM          & 88.57±1.15    & 76.50±1.17    & OOM            & OOM            & 62.00±3.79    & 28.26±4.20     \\
GRACE                      & 65.61±0.61 & 63.92±0.44  & \textbf{68.59±0.35}  & 60.15±0.45   & 92.00±1.39    & 88.64±0.67    & \textbf{83.43±0.63}     & \underline{60.45±1.46}     & 64.00±3.43    & \underline{34.86±3.43}  \\
RoSA                       & 64.06±0.67 & 62.44±0.39  & 67.07±0.65  & 57.68±0.44   & 90.78±2.27    & 85.93±2.14    & 82.39±0.42     & 57.45±2.16     & 64.17±4.10    & 32.20±2.15     \\
BGRL                       & 65.18±0.43 & 63.43±0.45  & 66.83±0.80  & 59.63±0.38   & 92.36±1.16    & 87.14±1.60    & 82.52±0.60     & 55.48±1.48     & 63.67±2.33    & 31.47±3.43     \\
COSTA                      & 65.05±0.80 & 62.37±0.45  & 66.76±0.87  & 55.73±0.36   & \underline{93.50±2.62}    & \underline{89.29±3.11}    & 83.15±0.30 & 55.03±3.22     & 61.66±2.58    & 32.39±2.13 \\
% ArCL                       &            &             & 67.64±0.57  & 59.71±0.44   &               &               &                &                & 65.00±3.94    & 35.41±1.97 \\      
SwAV                       & 62.22±0.53 & 59.79±0.53  & 64.65±0.94  & 55.06±0.39   & 89.00±0.79    & 81.72±0.66    & \underline{83.32±0.15}     & 59.69±1.97     & \underline{65.17±3.76}    & 29.36±2.01    \\
\midrule
MARIO                       & \textbf{67.11±0.46} & \textbf{65.28±0.34}  & \underline{68.46±0.40}  & \textbf{61.30±0.28}   & \textbf{94.36±1.21}    & \textbf{91.28±1.10}    & 82.31±0.54     & \textbf{63.33±1.72}     & \textbf{65.67±2.81}    & \textbf{37.15±2.37}     \\
\bottomrule
\end{tabular}}
\end{table*}

\begin{table*}[htp]
\caption{Experimental results of all methods under covariate shift. The bold font means the top-1 performance and the underline represents the second performance across the unsupervised methods. 'ID' represents in-distribution test performance and 'OOD' means out-of-distribution test performance. (OOM: out-of-memory on a GPU with 24GB memory)}
\label{tab:trans_covariate}
\centering
\scalebox{0.95}{
\begin{tabular}{l|cc|cc|cc|cc|cc}
\toprule
\toprule
\multirow{3}{*}{covariate shift} & \multicolumn{4}{c|}{GOOD-Cora}                                   & \multicolumn{2}{c|}{GOOD-CBAS} & \multicolumn{2}{c|}{GOOD-Twitch} & \multicolumn{2}{c}{GOOD-WebKB} \\
                           & \multicolumn{2}{c}{word} & \multicolumn{2}{c|}{degree}& \multicolumn{2}{c|}{color}    & \multicolumn{2}{c|}{language}   & \multicolumn{2}{c}{university} \\
                           & ID         & OOD         & ID          & OOD          & ID            & OOD           & ID             & OOD            & ID            & OOD            \\
\midrule
ERM                        & 70.50±0.41 & 64.69±0.33  & 72.46±0.49  & 55.53±0.50   & 92.00±3.08    & 77.57±1.29    & 70.98±0.41     & 49.35±5.09     & 39.34±1.79    & 14.52±3.14   \\
IRM                        & 70.48±0.26 & 64.53±0.57  & 71.98±0.34  & 53.72±0.46   & 90.86±2.41    & 78.86±1.67    & 69.81±0.95     & 49.11±2.82     & 38.52±3.30    & 13.97±2.80     \\
EERM                       & OOM        & OOM         & OOM         & OOM          & 65.00±2.57    & 57.43±3.60    & OOM            & OOM            & 46.07±4.55    & 27.40±7.65     \\
\midrule
GAE                        & 56.63±0.79 & 48.93±0.93  & 66.30±0.88  & 34.01±0.87   & 73.00±2.16    & 60.86±3.01    & 67.24±1.23     & 47.65±2.49     & 45.08±6.32    & 28.02±6.29    \\
VGAE                       & 62.02±0.66 & 54.12±0.86  & 69.41±0.57  & 44.20±1.29   & 62.29±2.04    & 63.29±1.11    & 66.99±1.43     & \underline{50.48±4.58}     & 48.85±4.68    & 20.87±6.69     \\
GraphMAE                   & 68.14±0.43 & 64.00±0.33  & \textbf{73.36±0.56}  & 53.75±0.55   & 67.28±3.03    & 67.28±1.49    & 68.84±1.20     & 48.02±2.79     & 48.03±4.34    & 30.00±8.09     \\
DGI                        & 60.85±0.75 & 57.03±0.67  & 68.97±0.41  & 41.75±0.88   & 69.57±4.09    & 59.71±3.43    & 68.43±1.05     & 44.83±1.61     & 48.52±5.04    & 21.11±7.50     \\
MVGRL                      & OOM        & OOM         & OOM         & OOM          & 65.00±1.94    & 64.15±0.77    & OOM            & OOM           & \textbf{54.10±5.39}    & 16.59±6.51     \\
GRACE                      & \underline{68.77±0.33} & \underline{64.21±0.41}  & 72.69±0.34  & \underline{56.10±0.63}   & \underline{93.57±1.83}    & \underline{89.29±3.40}    & \underline{71.12±0.87} & 46.21±1.54 & 49.67±5.82    & 28.10±4.68    \\
RoSA                       & 68.19±0.56 & 62.48±0.61  & 71.04±0.62  & 52.72±0.79   & 84.71±4.14    &79.14±3.51     & 70.58±0.36     & 45.83±1.72     & 52.30±4.24    & \underline{34.24±7.92}     \\
BGRL                       & 67.23±0.43 & 61.33±0.36  & 72.11±0.39  & 49.15±0.73   & 89.00±2.56    & 79.86±3.29    & \textbf{71.43±0.53}     & 43.86±0.94     & 51.80±5.55    & 30.32±7.61    \\
COSTA                      & 65.28±0.60 & 60.33±0.53  & 70.65±0.62  & 54.03±0.28   & 92.29±1.59    & 82.71±2.74    & 69.29±1.37     & 49.07±2.13     & 50.49±3.01    & 29.84±4.75   \\
SwAV                       & 63.29±1.01 & 56.98±0.94  & 70.27±0.73  & 43.00±0.52   & 89.57±1.12    & 81.43±1.69    & 69.19±0.93     & 49.37±2.96     & 49.84±4.82    & 30.55±6.72   \\
\midrule
MARIO                       & \textbf{69.99±0.54} & \textbf{65.06±0.34}  & \underline{72.73±0.43}  & \textbf{57.73±0.45}  & \textbf{94.57±2.46}    & \textbf{91.00±2.48}     & 68.31±0.78 & \textbf{57.37±1.37}     & \underline{53.94±3.23}    & \textbf{35.24±4.98}   \\
\bottomrule
\end{tabular}}

\end{table*}

\subsubsection{Inductive Setting}
In this subsection, we conduct experiments under the inductive settings, where the test nodes are kept unseen during training. This setting is more suitable for domain generalization.
% But we think it is more convincing that conduct experiments under inductive settings which means test nodes are unseen during training. This setting is more appropriate for domain generalization.

\noindent\textbf{Baselines:} For GOOD-WebKB and GOOD-CBAS datasets, we adopt ERM, IRM, GraphMAE, and GRACE as our baselines. And for Amazon-Photo and Elliptic datasets, we select ERM, EERM, and GRACE as our baselines.

\noindent\textbf{Experimental setup:} For GOOD-WebKB and GOOD-CBAS datasets, we use the same model configuration in Section~\ref{sec:trans}.
% Besides, we add experiments on Amazon-Photo dataset~\cite{yang2016revisiting} and Elliptic~\cite{elliptic} dataset in this subsection. 
For Amazon-Photo dataset~\cite{yang2016revisiting} and Elliptic~\cite{elliptic} dataset, they consist of many snapshots (training data and testing data use different snapshots) which are naturally inductive. For Amazon-Photo dataset, we use 2-layer GCN~\cite{gcn} as the encoder and for elliptic dataset, we use 5-layer GraphSAGE~\cite{sage} as encoder following~\cite{eerm}.

% Figure environment removed

\noindent\textbf{Analysis:}
According to Figure~\ref{fig:amazon},\ref{fig:elliptic},\ref{fig:ind_con},\ref{fig:ind_cov}, we can draw following conclusions:
firstly, based on Figure~\ref{fig:amazon}, it is evident that our method outperforms other representative supervised and self-supervised methods on all test graphs (T1$\sim$T8). This superiority is reflected in the larger median value of our method compared to others. For instance, MARIO achieves over a 3\% absolute improvement compared to ERM in terms of the mean value of eight median values. Additionally, our method demonstrates higher stability across different random initializations, as indicated by the closer proximity of the first and third quartile values to the median value~(\eg, the difference of first and third quartile values of ERM, EERM, GRACE and MARIO are 4.2, 3.3, 6.7 and 1.0 on T8 respectively which indicates MARIO is much more stable than other methods). Furthermore, our method exhibits consistent performance across different graphs (\eg, The standard deviation of median values on T1$\sim$T8 for ERM, EERM, GRACE, and MARIO are 0.4, 1.1, 1.2, and 0.3, respectively.), indicating its robustness to environmental variations and its ability to extract invariant features: $g(G^e) \approx g(G^{e'})$ for all $e, e' \in \mathcal{E}^\text{train}$. In summary, our method showcases enhanced OOD generalization capabilities.
% $g(G^e)g(G^e^\prime)$ where $any e, e^\prime in \mathcal{E}^{train}$

Secondly, from the results presented in Figure~\ref{fig:elliptic}, we can observe that our method averagely harvests 10.9\% absolute improvement over GRACE and 12.5\% absolute improvement over EERM in terms of F1 scores on Elliptic dataset. This demonstrates the effectiveness of our method in handling distribution shifts and improving performance compared to existing approaches. It is worth noting that GRACE's performance worsens over time, indicating its inability to handle distribution shifts effectively. In contrast, our method consistently achieves better F1 scores, except for T9, which is caused by the dark market shutdown occurred after T7~\cite{elliptic}. The emergence of such an event introduces significant variations in data distributions, which subsequently results in performance degradation for all methods. Indeed, this event serves as an unpredictable external factor that introduces significant challenges for models trained on limited training data. The results indicate that the performance heavily depends on available training data. Nonetheless, our approach outperforms other methods even in such an extreme case. This highlights the effectiveness of our method in addressing distribution shifts and improving generalization performance.

Finally, based on the observations from Figure~\ref{fig:ind_con} and Figure~\ref{fig:ind_cov} MARIO demonstrates the best performances on both ID and OOD test sets for GOOD-WebKB and GOOD-CBAS datasets, under both concept shift and covariate shift. Notably, MARIO outperforms other methods by more than 3\% and 10\% absolute improvement on GOOD-WebKB and GOOD-CBAS, respectively, under covariate shift. We can draw similar conclusions as discussed in Section~\ref{sec:trans}. Even under the inductive setting, our method continues to demonstrate excellent OOD generalization capabilities and achieves comparable or even improved in-distribution test performance. These statistical results further validate the effectiveness of our method in handling distribution shifts and enhancing generalization performance.

Overall, the observations we have made provide strong evidence of the great capacity of our method for handling distribution shifts, validating its effectiveness and potential for real-world applications.



% Figure environment removed

% Figure environment removed


% Figure environment removed


\subsection{Ablation Studies}\label{sec:ablation}
\noindent Table~\ref{tab:aba} provides a detailed analysis of the effect of each component according to our proposed recipe for improving OOD generalization in graph contrastive learning. Let's examine the different variants of our method and their impact on performance.
Specifically, MARIO~(w/o ad) represents MARIO without  adversarial augmentation. MARIO~(w/o cmi) denotes we only maximize the mutual information between positive pairs without considering conditional mutual information. MARIO~(w/o cmi, ad) means a vanilla graph contrastive method that is similar to GRACE. 

From Table~\ref{tab:aba}, we can find MARIO~(w/o cmi) lags far behind MARIO on OOD test set which demonstrates appropriately minimizing the redundant information (\ie, conditional mutual information) is essential to improve OOD generalization of GCL methods. And adversarial augmentation can also boost OOD generalization because it can approximately serve as a supermum operator to learn more invariant features  discussed in Section~\ref{sec:aug}. Based on the analysis of these variants, it is evident that the proposed improvements on data augmentation and contrastive loss in the recipe are both effective in enhancing graph OOD generalization. Each component contributes to the overall performance improvement, and their combination leads to a stronger self-supervised graph learner in terms of graph OOD generalization. 

In short, the findings from Table~\ref{tab:aba} support the rationale behind your proposed recipe and provide empirical evidence of the effectiveness of each proposed component. By incorporating these enhancements, our method achieves superior performance in handling distribution shifts and improving graph OOD generalization in graph contrastive learning.
\begin{table*}[htp]
\caption{Ablation studies for MARIO by masking each component.}
\label{tab:aba}
\centering
\scalebox{0.9}{
\begin{tabular}{l|cc|cc|cc|cc|cc}
\toprule
\toprule
\multirow{3}{*}{concept shift} & \multicolumn{4}{c|}{GOOD-Cora}                       & \multicolumn{2}{c|}{GOOD-CBAS} & \multicolumn{2}{c|}{GOOD-Twitch} & \multicolumn{2}{c}{GOOD-WebKB} \\
                           & \multicolumn{2}{c}{word} & \multicolumn{2}{c|}{degree}& \multicolumn{2}{c|}{color}    & \multicolumn{2}{c|}{language}   & \multicolumn{2}{c}{university} \\
                           & ID         & OOD         & ID          & OOD          & ID            & OOD           & ID             & OOD            & ID            & OOD            \\
\midrule
MARIO                      & \textbf{67.11±0.46} & \textbf{65.28±0.34}  & \textbf{68.46±0.40}  & \textbf{61.30±0.28}      & \textbf{94.36±1.21}  & \textbf{91.28±1.10}    & 82.31±0.54     & \textbf{63.33±1.72}     & \textbf{65.67±2.81}    & \textbf{37.15±2.37}     \\
MARIO(w/o ad)              & 66.23±0.53 & 64.02±0.18  & 67.88±0.38  & 60.46±0.29   & 93.21±1.25    & 90.29±0.91    & 82.42±0.73     & 60.50±1.02     & 64.83±2.83    & 36.51±3.25    \\
MARIO(w/o cmi)             & 65.32±0.60 & 63.51±0.32  & 68.14±0.32  & 61.19±0.34   & 94.15±1.23    & 90.57±1.96    & \textbf{82.51±0.56}     & 61.41±2.63     & 64.50±4.35    & 35.78±2.53     \\
MARIO(w/o cmi, ad)         & 64.67±0.55 & 63.11±0.32  & 67.95±0.65  & 60.01±0.57   & 93.36±1.66    & 89.64±1.73    & 81.90±0.75     & 60.12±1.60     & 64.17±3.67    & 34.13±2.38     \\
\bottomrule
\end{tabular}}
\end{table*}
% & 65.32±0.60 & 63.51±0.32 exchange 64.67±0.55 & 63.11±0.32
% 68.14±0.32       id ood test: 60.95±0.43       ood ood test: 61.19±0.34


\subsection{Sensitivity Analysis}\label{sec:sensitivity}
\noindent In this subsection, we will analyze some important hyper-parameters of our method. We conduct sensitivity analysis on GOOD-WebKB dataset with concept shift, we chose two sensitive hyper-parameters (\ie, the coefficient $\gamma$ of condition mutual information in Equation~\ref{equ:cmi} and the number of prototypes $|C|$ in Equation~\ref{equ:pq}). The coefficient of CMI range in $[0.001, 0.01, 0.1, 0.5, 1]$ and the number of prototypes $|C|$ ranges in $[10, 50, 100, 200, 300]$. From Figure~\ref{fig:sensitivity}, we can observe that $\gamma$ reaches 0.1 and $|C|$ reaches 100 or 200 can achieve the best OOD test accuracy. Both higher and lower values of $\gamma$ result in suboptimal performance. This finding aligns with previous research such as DIB~\cite{dib}, indicating that an appropriate compression level is crucial for achieving optimal performance. Extremely high or low compression values are not ideal. 

Regarding the number of prototypes $|C|$, based on the results shown in Figure~\ref{fig:sensitivity}, it is found that setting $|C|=100$ leads to the best performance in terms of OOD test accuracy. This choice provides a moderate number of pseudo labels, which is beneficial for the learning process. 

Based on the sensitivity analysis, we determined that setting $\gamma=0.1$ and $|C|=100$ on most datasets. These hyperparameter values strike a balance between compression level and the number of prototypes, resulting in improved graph OOD generalization.
% Figure environment removed


\subsection{Integrated with Other Models}\label{sec:other_models}
% Figure environment removed

\begin{table}[htp]
\caption{Results of different learning approaches with different encoding models (\ie, GCN, GraphSAGE, GAT).}
\label{tab:others}
\centering
\scalebox{0.9}{
\begin{tabular}{cc|cc|cc}
\toprule
\toprule
\multirow{3}{*}{Model}& \multirow{3}{*}{Method} & \multicolumn{2}{c|}{GOOD-CBAS} & \multicolumn{2}{c}{GOOD-WebKB} \\
                & & \multicolumn{2}{c|}{color}    & \multicolumn{2}{c}{university} \\
                &   & ID          & OOD         & ID          & OOD            \\
\midrule
\multirow{3}{*}{GCN} 
&ERM               & 89.79±1.39 & 83.43±1.19  &  62.67±1.53 & 26.33±1.09         \\
&GRACE             & 92.00±1.39 & 88.64±0.67  &  64.00±3.43 & 34.86±3.43        \\
&MARIO             & 94.36±1.21 & 91.28±1.10  &  65.67±2.81 & 37.15±2.37        \\ \bottomrule
\multirow{3}{*}{SAGE} 
&ERM               & 95.07±1.51 & 75.14±1.19  & 73.67±2.08  & 46.33±3.42       \\
&GRACE             & 95.29±1.11 & 74.43±2.36  & 70.50±5.06  & 49.54±3.83        \\
&MARIO             & 96.00±1.07 & 76.29±3.01  & 71.00±3.82  & 51.74±4.63        \\ \bottomrule
\multirow{3}{*}{GAT} 
&ERM               & 78.64±3.63 & 72.93±2.64  & 61.33±3.71  & 28.99±2.63        \\
&GRACE             & 84.57±1.79 & 78.36±1.60  & 59.50±2.36  & 35.78±3.26        \\
&MARIO             & 84.93±1.95 & 80.43±1.89  & 62.17±4.78  & 38.17±3.10        \\
\bottomrule
\end{tabular}}
\end{table}



\noindent In the subsection, we demonstrate the model-agnostic nature of the recipe by integrating it with various graph neural network (GNN) models, including GCN, GraphSAGE, and GAT.

From Table~\ref{tab:others}, it can be observed that regardless of the specific GNN model used as the encoder, our method consistently achieves the best performance on the OOD test set. This indicates the effectiveness and robustness of our method across different GNN models.
By achieving superior performance across different GNN models, MARIO demonstrates its versatility and ability to improve the OOD generalization of various graph neural models. This highlights the broad applicability and effectiveness of our recipe in enhancing the performance of different GNN encoders.

Furthermore, we integrate our recipe with other GCL methods in Appendix~\ref{app:other_methods}. The results demonstrate our recipe can boost the OOD generalization ability of various GCL methods which means our recipe can serve as a plug-in for many current classical GCL methods.

% Figure environment removed

\subsection{Visualization}\label{sec:vis}
\subsubsection{Metric Score Curves}
We present metric score curves for ERM and MARIO, including training, ID validation, ID testing, OOD validation, and OOD testing accuracy, in Figure~\ref{fig:curve2}. Notably, MARIO demonstrates superior convergence with approximately 10\% absolute improvement on the OOD test set compared to ERM. Furthermore, MARIO effectively narrows the performance gap between in-distribution and out-of-distribution performance, showcasing its efficacy in enhancing OOD generalization for graph data. More metric score curves can be found in Appendix~\ref{app:curves}.


\subsubsection{Feature Visualization}
In order to assess the quality of learned embeddings, we adopt t-SNE~\cite{tsne} to visualize the node embedding on GOOD-Cora dataset (concept shift in word domain) using random-init of GCN, EERM, GRACE, and MARIO, where different classes have different colors in Figure~\ref{fig:vis}. For clarity, we select eight classes with the largest number of nodes to enhance the informativeness and interpretability of the visualization. We can observe that the 2D projection of node embeddings learned by MARIO has a better separation of clusters, which indicates the model can help learn representative features for downstream tasks. It has to note that we depict both ID nodes and OOD nodes in the same figure. 

Besides, we also separately visualize ID nodes and OOD nodes in the different figures in the Appendix~\ref{app:feature}. And we can find MARIO performs a clearer separation of clusters whether on ID nodes or OOD nodes compared to other methods.




\section{Related Work}
\label{sec:related}
\section{Related Work}
\label{appsec: related work}
Bayesian causal discovery literature has primarily focused on inference in linear models with closed-form posteriors or marginalized parameters. Early works considered sampling directed acyclic graphs (DAGs) for discrete~\cite{cooper1992bayesian, madigan1995bayesian, heckerman2006bayesian} and Gaussian random variables~\cite{friedman2003being, tong2001active} using Markov chain Monte Carlo (MCMC) in the DAG space. However, these approaches exhibit slow mixing and convergence~\cite{eaton2012bayesian,grzegorczyk2008improving}, often requiring restrictions on number of parents~\cite{kuipers2017partition}. %Alternative exact dynamic programming methods are limited to small settings~\cite{koivisto2012advances}. 

Recent advances in variational inference~\cite{zhang2018advances} have facilitated graph inference in DAG space, with gradient-based methods employing the NOTEARS DAG penalty \cite{zheng2018dags}.\cite{annadani2021variational} samples DAGs from autoregressive adjacency matrix distributions, while \cite{lorch2021dibs} utilizes Stein variational approach \cite{liu2016stein} for DAGs and causal model parameters. \cite{cundy2021bcd} proposed a variational inference framework on node orderings using the gumbel-sinkhorn gradient estimator \cite{mena2018learning}. \cite{deleu2022bayesian,nishikawa2022bayesian} employ the GFlowNet framework \cite{bengio2021gflownet} for inferring the DAG posterior. Most methods, except\cite{lorch2021dibs} are restricted to linear models, while \cite{lorch2021dibs} has high computational costs and lacks DAG generation guarantees compared to our method.
% at least quadratic scaling complexity, both with respect to the number of nodes (due to the DAG penalty) as well as number of posterior samples. Our proposed approach instead has linear complexity with respect to number of posterior samples and does not require any additional DAG penalty.     

In contrast, \emph{quasi-Bayesian} methods, such as DAG bootstrap \cite{friedman2013data}, demonstrate competitive performance. DAG bootstrap resamples data and estimates a single DAG using PC \cite{spirtes2000causation}, GES \cite{chickering2002optimal}, or similar algorithms, weighting the obtained DAGs by their unnormalized posterior probabilities. Recent neural network-based works employ variational inference to learn DAG distributions and point estimates for nonlinear model parameters \cite{charpentier2022differentiable,geffner2022deep}.

\section{Discussion}
\label{sec:discuss}
\section{Discussion}
\label{sec: discussion}
\kmsdelete{In this work} We study \kmsreplace{Fairness-Aware PAC learning}{Fair-ERM} in the malicious noise model, and  in some cases allow 
the learner to maintain optimal overall accuracy despite the signal in Group $B$ being almost entirely washed out.
%when we allow learners to use the
%$\PQ$ randomized expansion of the hypothesis class $\mathcal{H}$
In particular we show that different fairness constraints have fundamentally different behavior in the presence of Malicious Noise, in terms of the amount of accuracy loss that a given level of Malicious Noise could cause a fairness-constrained learner to incur. 
The key to achieving our results, which are more optimistic than those in \cite{lampert}, is allowing for improper learners using the (P,Q)-randomized expansions of the given class $\mathcal{H}$.
%We \kmsreplace{present a picture of the}{prove upper and lower bounds on}
%accuracy loss for a range of fairness notions, given \kmsreplace{this simple randomization step.}{learning over $\PQ$.
%In general our results indicate Fair-ERM (given learning over $\PQ$) is more robust than claimed in \cite{lampert}.
The type of smoothness we create by using $\PQ$ seems to be a natural property that is likely shared by many natural hypothesis classes.

Fairness notions are motivated as a response to learned disparities when there is \kmsdelete{data corruption or} systemic error affecting \kmsdelete{the data for}
one group. 
Fairness notions are supposed to mitigate this by ruling out classifiers that have worse performance on a sub-group. 
This can peg both classifiers at a lower level of performance \kmsdelete{(e.g that the lower subgroup)} in order to \emph{motivate} \cite{hardt16} improving the data collection or labelling process to obtain more reliable performance. 
%So in \kmsreplace{some}{a} sense, sensitivity of the fairness notion to poor sub-group performance caused by malicious noise is the \textit{point} of fairness constraints! 
However, it also desirable that fairness constraints perform gracefully when subject to Malicious Noise because fairness constraints will be used in contexts where the data is unreliable and noisy and this might not be known to the learner.
This tension, exposed by our work, motivates 
%a revisiting of fairness notions from first principles approach and trying to axiomatize the 
%desired properties of a fairness intervention a la cryptography and privacy. \footnote{Work in multi-calibration \cite{multicalib} is a viable direction for this problem but it is unclear how 
%that and related notions behave with unreliable data. }
on going work studying the sensitivity level of fairness constraints. 
%If we we are to take a view, if a classifier is deployed 



%% Bibliography
\bibliography{main}


\end{document}
