%\documentclass[wcp,gray]{jmlr} % test grayscale version
 %\documentclass[wcp]{jmlr}% former name JMLR W\&CP
\documentclass[pmlr]{jmlr}% new name PMLR (Proceedings of Machine Learning)

 % The following packages will be automatically loaded:
 % amsmath, amssymb, natbib, graphicx, url, algorithm2e

 %\usepackage{rotating}% for sideways figures and tables
\usepackage{longtable}% for long tables
\usepackage{makecell}
\usepackage{graphicx}
\usepackage{siunitx}
% \usepackage[load-configurations={abbreviations}]{siunitx}
% \usepackage{caption}
% \usepackage{subcaption}
% \usepackage[caption=true,font=footnotesize]{subfig}
\usepackage{float}
\usepackage{xcolor,color,soul,bm}
\usepackage{makecell}
 % The booktabs package is used by this sample document
 % (it provides \toprule, \midrule and \bottomrule).
 % Remove the next line if you don't require it.
\usepackage{booktabs}
\usepackage{amssymb}
\newcommand{\systemname}{UDAMA}
\newcommand{\imt}[1]{\textcolor{black}{#1}}
 % The booktabs package is used by this sample docume÷nt
 % (it provides \toprule, \midrule and \bottomrule).
 % 
 % book quality tables
\usepackage{booktabs}
\usepackage{cleveref}
\usepackage{hyperref}
\crefname{subfigure}{}{}
 
 % The siunitx package is used by this sample document
 % to align numbers in a column by their decimal point.
 % Remove the next line if you don't require it.
% \usepackage[load-configurations=version-1]{siunitx} % newer version
 %\usepackage{siunitx}
\newcommand{\hj}[1]{\sethlcolor{orange}\hl{[HJ: #1]}}

\newcommand{\yw}[1]{\sethlcolor{lime}\hl{[YW: #1]}}
\newcommand{\ds}[1]{\sethlcolor{pink}\hl{[DS: #1]}}
\newcommand{\cm}[1]{\sethlcolor{yellow}\hl{[CM: #1]}}
\newcommand{\mlhc}[1]{\textcolor{black}{#1}}

\makeatletter
\def\set@curr@file#1{\def\@curr@file{#1}} %temp workaround for 2019 latex release
\makeatother

 % The following command is just for this sample document:
\newcommand{\cs}[1]{\texttt{\char`\\#1}}

 % Define an unnumbered theorem just for this sample document:
\theorembodyfont{\upshape}
\theoremheaderfont{\scshape}
\theorempostheader{:}
\theoremsep{\newline}
\newtheorem*{note}{Note}

 % change the arguments, as appropriate, in the following:
% \jmlrvolume{}
\jmlrvolume{219}
\jmlryear{2023}
\jmlrworkshop{Machine Learning for Healthcare}

% H: Not sure if we need this:
% Short headings should be running head and authors last names
% \ShortHeadings{A Really Awesome MLHC Article}{Lastname, PhD and Lastname, MD}
% \firstpageno{1}

% \title[Short Title]{Title of Your MLHC Article}
% \title[UDAMA]{\textit{Turning Silver into Gold}: Domain Adaptation with Weak Labels for Wearable \mlhc{Cardio-fitness} Prediction}
\title[UDAMA]{\textit{UDAMA}: Unsupervised Domain Adaptation through Multi-discriminator Adversarial Training with Noisy Labels Improves Cardio-fitness Prediction}

\author{\Name{Yu Wu\textsuperscript{1}} \Email{yw573@cam.ac.uk}\\
  % \AND
  \Name{Dimitris Spathis\textsuperscript{1,2}} \Email{ds806@cam.ac.uk}\\
  % \AND
  \Name{Hong Jia\textsuperscript{1}} \Email{hj359@cam.ac.uk}\\
  % \AND
  \Name{Ignacio Perez-Pozuelo\textsuperscript{3}} \Email{ip325@cam.ac.uk}\\
  % \AND
  \Name{Tomas I. Gonzales\textsuperscript{3}} \Email{tomas.gonzales@mrc-epid.cam.ac.uk}\\
  % \AND
  \Name{Soren Brage\textsuperscript{3}} \Email{soren.brage@mrc-epid.cam.ac.uk}\\
  % \AND
  \Name{Nicholas Wareham\textsuperscript{3}} \Email{nick.wareham@mrc-epid.cam.ac.uk}\\
  % \AND
  \Name{Cecilia Mascolo\textsuperscript{1}} \Email{cm542@cam.ac.uk}\\
  \addr \textsuperscript{1}Department of Computer Science and Technology, University of Cambridge, UK \\
  \addr \textsuperscript{2}Nokia Bell Labs, Cambridge, UK \\
  \addr \textsuperscript{3}MRC Epidemiology Unit, School of Clinical Medicine, University of Cambridge, UK
 }


% \editor{Editor's name}

\begin{document}

\maketitle

\begin{abstract}
% \yw{Make sure co-author's affliations}


% \yw{Restore to last version focus on generalizable ML for healthcare?}
% Deep learning models are increasingly used to predict wearable-based cardio-respiratory fitness (CRF), with maximal oxygen consumption (VO$_{2}$max) being the benchmark measurement. 
Deep learning models have shown great promise in various healthcare \mlhc{monitoring} applications. 
% However, most of them are typically developed and validated on small-scale datasets\cm{is it true? xray image datasets are big. is it possible it is true for time series data only?},\yw{fixed} as directly collecting high-quality (gold-standard) data for health applications is often costly and time-consuming.
\mlhc{However, most healthcare datasets with high-quality (gold-standard) labels are small-scale, as directly collecting ground truth is often costly and time-consuming. As a result, models developed and validated on small-scale datasets often suffer from overfitting and do not generalize well to unseen scenarios.}
% As a result, these models often suffer from overfitting and do not generalize well to unseen datasets. 
At the same time, large amounts of imprecise (silver-standard) labeled data, 
% of sensor\cm{sensor comes out of the blue} \yw{fixed} data, 
annotated by approximate methods with the help of modern wearables and in the absence of ground truth validation, are starting to emerge. However, due to measurement differences, this data displays significant label distribution shifts, which motivates the use of domain adaptation. To this end, we introduce \textbf{\systemname{}}, a method with two key components: \textbf{Unsupervised Domain Adaptation} and \textbf{Multi-discriminator Adversarial Training}, where we pre-train on the silver-standard data and employ adversarial adaptation with the gold-standard data along with two domain discriminators.
% \hj{to improve the model performance.}
% We validate our framework on the maximal oxygen consumption (VO$_{2}$max) prediction task using two free-living cohort studies (N=11,059 and N=181) for pre-training and fine-tuning, respectively.
% We show that \systemname{} achieves the best performance of corr = 0.701 $\pm$ 0.032 compared to vanilla transfer learning and state-of-the-art domain adaptation models, paving the way for leveraging noisy labeled data towards accurate fitness estimation at scale.
\mlhc{In particular, we showcase the practical potential of \systemname{} by applying it to Cardio-respiratory fitness (CRF) prediction. CRF is a crucial determinant of metabolic disease and mortality, and it presents labels with various levels of noise (gold- and silver-standard), making it challenging to establish an accurate prediction model. Our results show promising performance by alleviating distribution shifts in various label shift settings. Additionally, by using data from two free-living cohort studies (Fenland and BBVS), we show that \systemname{} consistently outperforms up to 12\% 
% \ds{how much better compared to the SOTA?}\yw{fixed} 
compared to competitive transfer learning and state-of-the-art domain adaptation models, paving the way for leveraging noisy labeled data to improve fitness estimation at scale.}

% \yw{pre-pare github-repo link here...}


\end{abstract}


\input{01_new_introduction}
% !TEX root = ../AttackGraphBasedRiskAnalysis.tex
% !TEX spellcheck = en_US
% !TEX encoding = UTF-8 Unicode

\section{Related Work}\label{sec: related work}
%\todo{Discuss all frameworks regarding consequences.}

Kordy et al.~\cite{DAGpaper} categorize thirty-three frameworks for graphical analysis of attack and defense scenarios into (1) \emph{attack and/or defense modeling}, which focus on the formal aspects of attacks or defenses, and (2) \emph{static or sequential modeling}, which focus on the temporal aspects or dependencies between actions. 
Using the same categorization, this section provides an overview of all the frameworks, and it describes these frameworks that fulfill the majority of properties incorporated in the framework of this article.

By reviewing frameworks from current literature, we identify seven properties for graphically modeling and managing an entire risk landscape.
The first property is \emph{attack vectors}, which enables the relations (shown as edges) between attack steps (shown as nodes) and, therefore, the formation of attack paths (i.e., attack vectors). 
The second property is the \emph{directed acyclic graph (DAG) structure}, thereby enabling linear (i.e., directed) and finite (i.e., acyclic) series of attack steps towards multiple potential attack goals (i.e., graph). 
The third property is \emph{node attributes}, which enables the quantification and, therefore, the evaluation of attack steps. 
The fourth property is \emph{dynamic connectors}, thereby enabling extensive attack refinements (besides the basic AND-OR refinements). 
The fifth property is \emph{edge attributes}, which enables the quantification and, therefore, the evaluation of relations between attack steps. 
The sixth property is \emph{countermeasure nodes}, thereby enabling actions to reduce the negative consequences of attacks.
The final property is \emph{consequence nodes}, enabling the presentation of consequences of successful attacks, which is also necessary to constitute the impact.

\begin{table*}[h]
\rowcolors{2}{gray!10}{gray!40}
\renewcommand{\arraystretch}{1.2}
\caption{Static attack modeling frameworks compared to the seven defined properties.}
\label{tab: static attack modeling}
\noindent\makebox[\textwidth]{%
\begin{tabular}[t]{>{\raggedright}p{0.15\textwidth}>{\raggedright}p{0.06\textwidth}>{\raggedright}p{0.08\textwidth}>{\raggedright}p{0.1\textwidth}>{\raggedright}p{0.09\textwidth}>{\raggedright\arraybackslash}p{0.08\textwidth}>{\raggedright\arraybackslash}p{0.15\textwidth}>{\raggedright\arraybackslash}p{0.1\textwidth}}
\toprule
 & Attack Vectors & DAG Structure & Node Attributes & Dynamic Connectors & Edge Attributes & Countermeasure Nodes & Consequence Nodes
\tabularnewline
\midrule
% \textbf{Static Attack Modeling} & & & & &
% \tabularnewline
Attack Trees & \checkmark & - & (\checkmark) & - & - & - & -
\tabularnewline
Augmented Vulnerability Trees & \checkmark & - & (\checkmark) & - & - & -  & -
\tabularnewline
Augmented Attack Trees & \checkmark & - & (\checkmark) & - & - & -  & -
\tabularnewline
OWA Trees & \checkmark & - & - & (\checkmark)  & (\checkmark) & -  & -
\tabularnewline
Parallel Model for Multi-Parameter Attack Trees & \checkmark & - & (\checkmark) & - & - & -  & -
\tabularnewline
Extended Fault Trees & \checkmark & - & (\checkmark) & - & - & -  & -
\tabularnewline
\bottomrule
\end{tabular}}
\end{table*}

Each one of the thirty-three frameworks presented in this section considers only subsets of the seven identified properties. 
None of these frameworks are suitable, as all seven properties are necessary to perform a full risk assessment.
To overcome this limitation, this article incorporates all seven identified properties into a framework for a graphical solution for performing risk analysis and examines its applicability to different risk analysis standards.

\subsection{Static Attack Modeling}\label{sec: static attack modeling}




Six frameworks for \emph{static attack modeling}, namely \emph{Attack Trees}~\cite{weiss1991}, \emph{Augmented Vulnerability Trees}~\cite{AugmentedVulnerabilityTrees}, \emph{Augmented Attack Trees}~\cite{AugmentedAttackTrees}, \emph{OWA Trees}~\cite{Yager2006OWATA}, \emph{Parallel Model for Multi-Parameter Attack Trees}~\cite{ParallelModelForMultiParameterAttackTrees}, and \emph{Extended Fault Trees}~\cite{ExtendedFaultTrees}, are summarised in Table~\ref{tab: static attack modeling}.
All frameworks fulfill the attack vectors property, but none of them supports the DAG structure, countermeasure nodes, and consequence nodes properties. 
Of the six frameworks, OWA trees stand out as they at least partially fulfill the dynamic connectors and edge attributes properties, despite being the only framework that does not fulfill the node attributes property. 
This section describes attack trees, which was the first graphical security modeling framework, and OWA trees, which is the framework that at least partially fulfills most of the seven identified properties.

\subsubsection{Attack Trees}\label{sec: attack trees}

The first \emph{tree-based approach}, shown as an AND-OR tree structure for graphical security modeling, was the \emph{threat logic trees}, which was introduced by Weiss in 1991~\cite{weiss1991}.
Today, all AND-OR tree structures are referred to as \emph{attack trees}, a term first introduced by Salter et al. in 1998~\cite{Salter1998}.

In attack trees, the root node (i.e., the tree's root) indicates the attack's main goal. 
The main goal is then conjunctively (AND) or disjunctively (OR) refined into sub-goals until they represent basic actions corresponding to atomic components that can be easily understood and quantified. 
Conjunctive refinements indicate that \emph{all} sub-goals need to be fulfilled in order to achieve the main goal, whereas disjunctive refinements indicate that \emph{at least one} sub-goal needs to be fulfilled for achieving the main goal~\cite{weiss1991}.

\subsubsection{OWA Trees}\label{sec: owa trees}

\emph{Ordered weighted averaging (OWA) trees} were proposed by Yager in 2005 to include the concept of \emph{uncertainty} into attack trees~\cite{Yager2006OWATA}. 
This was made possible by replacing the AND-OR nodes with OWA nodes (i.e., quantifiers, such as \emph{most}, \emph{some}, \emph{half of}, etc.) and therefore taking into consideration situations where the number of sub-goals that need to be fulfilled in order to achieve the main goal remains unknown. 
Finally, OWA trees allow for the evaluation of success probability and cost attributes, which can be jointly used to calculate the cheapest and most probable attack.

\subsection{Sequential Attack Modeling}\label{sec: sequential attack modeling}

\begin{table*}[h]
\rowcolors{2}{gray!10}{gray!40}
\renewcommand{\arraystretch}{1.2}
\caption{Sequential attack modeling frameworks compared to the seven defined properties.}
\label{tab: sequential attack modeling}
\noindent\makebox[\textwidth]{%
\begin{tabular}[t]{>{\raggedright}p{0.15\textwidth}>{\raggedright}p{0.06\textwidth}>{\raggedright}p{0.08\textwidth}>{\raggedright}p{0.1\textwidth}>{\raggedright}p{0.09\textwidth}>{\raggedright\arraybackslash}p{0.08\textwidth}>{\raggedright\arraybackslash}p{0.15\textwidth}>{\raggedright\arraybackslash}p{0.1\textwidth}}
\toprule
 & Attack Vectors & DAG Structure & Node Attributes & Dynamic Connectors & Edge Attributes & Countermeasure Nodes & Consequence Nodes
\tabularnewline
\midrule
% \textbf{Sequential Attack Modeling} & & & & &
% \tabularnewline
Cryptographic DAGs & \checkmark & \checkmark & - & - & - & - & - 
\tabularnewline
Fault Trees for Security & \checkmark & - & \checkmark & (\checkmark) & - & -  & -
\tabularnewline
Bayesian Networks for Security & \checkmark & \checkmark & \checkmark & - & \checkmark & - & -
\tabularnewline
Bayesian Attack Graphs & \checkmark & \checkmark & \checkmark & - & \checkmark & - & -
\tabularnewline
Compromise Graphs & \checkmark & \checkmark & - & - & (\checkmark) & - & -
\tabularnewline
Enhanced Attack Trees & \checkmark & - & \checkmark & - & (\checkmark) & - & -
\tabularnewline
Vulnerability Cause Graphs & (\checkmark) & \checkmark & - & - & - & - & -
\tabularnewline
Dynamic Fault Trees for Security & \checkmark & - & (\checkmark) & - & - & - & -
\tabularnewline
Serial Model for Multi-Parameter Attack Trees & \checkmark & - & (\checkmark) & - & - & - & -
\tabularnewline
Improved Attack Trees & \checkmark & - & (\checkmark) & - & - & - & -
\tabularnewline
Time-dependent Attack Trees & \checkmark & \checkmark & (\checkmark) & - & - & - & -
\tabularnewline
\bottomrule
\end{tabular}}
\end{table*}

Eleven frameworks for \emph{sequential attack modeling}, namely \emph{Cryptographic DAGs}~\cite{Meadows1996ARO}, \emph{Fault Trees for Security}~\cite{FaultTreesForSecurity}, \emph{Bayesian Networks for Security}~\cite{BayesianNetworksForSecurity}, \emph{Bayesian Attack Graphs}~\cite{BayesianAttackGraphs}, \emph{Compromise Graphs}~\cite{CompromiseGraphs}, \emph{Enhanced Attack Trees}~\cite{EnhancedAttackTrees}, \emph{Vulnerability Cause Graphs}~\cite{VulnerabilityCauseGraphs}, \emph{Dynamic Fault Trees for Security}~\cite{DynamicFaultTreesForSecurity}, \emph{Serial Model for Multi-Parameter Attack Trees}~\cite{SerilModelForMultiParameterAttackTrees}, \emph{Improved Attack Trees}~\cite{ImprovedAttackTrees}, and \emph{Time-dependent Attack Trees}~\cite{TimeDependentAttackTrees}, are summarised in Table~\ref{tab: sequential attack modeling}. 
Again, none of the frameworks fulfills the countermeasure and consequence nodes property. In addition, only Fault Trees for Security offer a wide range of dynamic connectors, and only Bayesian-based models fulfill the edge attributes property. 
Finally, Compromise Graphs and Enhanced Attack Trees are two frameworks that at least partially fulfill the edge attributes property, and Vulnerability Cause Graphs are the only framework that only partially fulfills the attack vectors property. 
This section describes Cryptographic DAGs, which was the first graph-based approach for security modeling, and Bayesian Attack Graphs, which combine attack trees and Bayesian networks and also fulfill four of the seven identified properties.

\subsubsection{Cryptographic DAGs}\label{sec: cryptographic dags}

\emph{Cryptographic directed acyclic graphs} were proposed by Meadows~\cite{Meadows1996ARO} in 1996 to provide a \emph{novel} simple representation of sequences and dependencies of attack steps towards the main goal of the attack. 
Instead of a tree-based approach, Cryptographic DAGs introduced a \emph{graph-based approach} for security modeling. 
However, they eventually do not offer the possibility to perform risk assessment as other properties are still not fulfilled.

\subsubsection{Bayesian Networks and Bayesian Attack Graphs}\label{sec: bayesian Networks and Attack Graphs}

For the last couple of decades, researchers have been focusing on \emph{Bayesian networks} for the purposes of security modeling.
The origin of Bayesian networks, which are also known as \emph{belief} or \emph{causal networks}, lies in artificial intelligence.
In Bayesian networks, nodes represent events or objects and are associated with probabilistic variables. 
Hence, analyzing the uncertainty of events is also possible. 
Bayesian networks follow a DAG structure, where the directed edges represent the causal dependencies between the nodes~\cite{BayesianAttackGraphs}.

\emph{Bayesian attack graphs} are a fusion of (general) attack trees and (computational procedures) of Bayesian networks, and they were first introduced by Liu and Man in 2005 to analyze network vulnerability scenarios~\cite{BayesianAttackGraphs}. 
Subsequently, calculating general security metrics regarding information system networks~\cite{Frigault2008, Noel2010} and capturing dynamic behavior~\cite{Frigault2008Dyn} was also made possible.

Finally, although Bayesian attack graphs allow for assigning values to nodes and for performing computations using the graphs, they do not allow for a dynamic selection of connectors and for including countermeasures. 
As a result, Bayesian attack graphs cannot be used to perform risk assessment.

\subsection{Static Attack and Defense Modeling}\label{sec: static attack and defense modeling}

\begin{table*}[h]
\rowcolors{2}{gray!10}{gray!40}
\renewcommand{\arraystretch}{1.2}
\caption{Static attack and defense modeling frameworks compared to the seven defined properties.}
\label{tab: static attack and defense modeling}
\noindent\makebox[\textwidth]{%
\begin{tabular}[t]{>{\raggedright}p{0.15\textwidth}>{\raggedright}p{0.06\textwidth}>{\raggedright}p{0.08\textwidth}>{\raggedright}p{0.1\textwidth}>{\raggedright}p{0.09\textwidth}>{\raggedright\arraybackslash}p{0.08\textwidth}>{\raggedright\arraybackslash}p{0.15\textwidth}>{\raggedright\arraybackslash}p{0.1\textwidth}}
\toprule
 & Attack Vectors & DAG Structure & Node Attributes & Dynamic Connectors & Edge Attributes & Countermeasure Nodes & Consequence Nodes
\tabularnewline
\midrule
% \textbf{Static Attack and Defense Modeling}
% \tabularnewline
Anti-Models & \checkmark & - & - & - & - & \checkmark & -
\tabularnewline
Defense Trees & \checkmark & - & (\checkmark) & - & - & \checkmark & -
\tabularnewline
Protection Trees & - & - & \checkmark & - & - & \checkmark & -
\tabularnewline
Security Activity Graphs & \checkmark & \checkmark & (\checkmark) & - & - & \checkmark & -
\tabularnewline
Attack Countermeasure Trees & \checkmark & - & \checkmark & - & - & \checkmark & -
\tabularnewline
Attack-Defense Trees & \checkmark & - & \checkmark & - & - & \checkmark & -
\tabularnewline
Countermeasure Graphs & \checkmark & \checkmark & \checkmark & - & - & \checkmark & -
\tabularnewline
\bottomrule
\end{tabular}}
\end{table*}

Seven frameworks for \emph{static attack and defense modeling}, namely \emph{Anti-Models}~\cite{AntiModels}, \emph{Defense Trees}~\cite{DefenseTrees}, \emph{Protection Trees}~\cite{ProtectionTrees}, \emph{Security Activity Graphs}~\cite{SecurityActivityGraphs}, \emph{Attack Countermeasure Trees}~\cite{AttackCountermeasureTrees}, \emph{Attack-Defense Trees}~\cite{AttackDefenseTrees}, and \emph{Countermeasure Graphs}~\cite{CountermeasureGraphs}, are summarised in Table~\ref{tab: static attack and defense modeling}. 
All frameworks fulfill the countermeasure nodes property, and only Protection Trees do not fulfill the attack vectors property. 
In addition, Anti-Models is the only framework that does not at least partially fulfill the node attributes property.
However, none of these frameworks considers consequence nodes in their design.
This section describes Security Activity Graphs and Countermeasure Graphs, which are the two frameworks that fulfill four of the seven identified properties.

\subsubsection{Security Activity Graphs}\label{sec: security activity graphs}

\emph{Security activity graphs (SAGs)} were developed by Ardi et al.~\cite{SecurityActivityGraphs} in 2006 to improve security throughout the software development process. 
SAGs are loosely based on fault trees, and the root of a SAG is associated with a vulnerability. 
Vulnerability mitigations are modeled using activities (i.e., leaf nodes), which are assigned boolean variables to indicate whether an activity \enquote{is implemented perfectly during software development} (true) or not (false). 
Finally, besides AND-OR gates, which follow a strictly Boolean logic, SAGs also include \emph{split gates}, which allow one activity to be used in several parent activities, thus creating a DAG structure.

However, SAGs lack the ability to represent the consequences of threats and edge attributes, and both are necessary to calculate a risk value.
Furthermore, there is only a limited option for connectors and node attributes.
Therefore, rendering SAGs impractical for risk assessment.

\subsubsection{Countermeasure Graphs}\label{sec: countermeasure graphs}

\emph{Countermeasure graphs} were introduced by Baca and Petersen~\cite{CountermeasureGraphs} in 2010 to simplify countermeasure selection through cumulative voting. 
Countermeasure graphs are created by identifying actors, goals, attacks, and countermeasures. Related events are connected with edges. 
That is, actors are connected to pursued goals and likely executable attacks, and countermeasures are connected to preventable attacks. 
Finally, actors, goals, attacks, and countermeasures are assigned priorities according to the rules of hierarchical cumulative voting. 
Higher assigned priorities imply higher threat levels of the corresponding events and vice versa. 
Using hierarchical cumulative voting, the most effective countermeasures can be identified.

Countermeasure Graphs provide a useful system overview, but the computational rules focus on finding the most effective countermeasure instead of the most likely and severe attack. 
This limitation could at least partially be addressed with the threat level. 
However, the threat level value is determined by the subjective assessment of the graph creator rather than by calculations over meaningful attributes, thereby raising issues of validity.

\subsection{Sequential Attack and Defense Modeling}\label{sec: sequential attack and defense modeling}

\begin{table*}[h]
\rowcolors{2}{gray!10}{gray!40}
\renewcommand{\arraystretch}{1.2}
\caption{Sequential attack and defense modeling frameworks compared to the seven defined properties.}
\label{tab: sequential attack and defense modeling}
\noindent\makebox[\textwidth]{%
\begin{tabular}[t]{>{\raggedright}p{0.15\textwidth}>{\raggedright}p{0.06\textwidth}>{\raggedright}p{0.08\textwidth}>{\raggedright}p{0.1\textwidth}>{\raggedright}p{0.09\textwidth}>{\raggedright\arraybackslash}p{0.08\textwidth}>{\raggedright\arraybackslash}p{0.15\textwidth}>{\raggedright\arraybackslash}p{0.1\textwidth}}
\toprule
 & Attack Vectors & DAG Structure & Node Attributes & Dynamic Connectors & Edge Attributes & Countermeasure Nodes & Consequence Nodes
\tabularnewline
\midrule
% \textbf{Sequential Attack and Defense Modeling} & & & & &
% \tabularnewline
Insecurity Flows & \checkmark & \checkmark & \checkmark & - & - & \checkmark & -
\tabularnewline
Intrusion DAGs & \checkmark & \checkmark & - & - & - & \checkmark & -
\tabularnewline
Bayesian Defense Graphs & \checkmark & \checkmark & (\checkmark) & - & - & \checkmark & -
\tabularnewline
Security Goal Indicator Trees & - & - & - & - & - & \checkmark & -
\tabularnewline
Attack Response Trees & \checkmark & -  & \checkmark & - & - & \checkmark & -
\tabularnewline
Boolean Logic Driven Markov Processes & \checkmark & \checkmark & (\checkmark) & (\checkmark) & - & \checkmark & -
\tabularnewline
Cyber Security Modeling Language & \checkmark & \checkmark & (\checkmark) & - & (\checkmark) & \checkmark & -
\tabularnewline
Security Goal Models & \checkmark & \checkmark & - & - & - & \checkmark & -
\tabularnewline
Unified Parameterizable Attack Trees & \checkmark & - & \checkmark & - & (\checkmark) & \checkmark & -
\tabularnewline
\bottomrule
\end{tabular}}
\end{table*}

Finally, nine frameworks for \emph{sequential attack and defense modeling}, namely \emph{Insecurity Flows}~\cite{InsecurityFlows}, \emph{Intrusion DAGs}~\cite{IntrusionDAGs}, \emph{Bayesian Defense Graphs}~\cite{BayesianDefenseGraphs}, \emph{Security Goal Indicator Trees}~\cite{SecurityGoalIndicatorTrees}, \emph{Attack Response Trees}~\cite{AttackResponseTrees}, \emph{Boolean Logic Driven Markov Process}~\cite{BooleanLogicDrivenMarkovProcess}, \emph{Cyber Security Modeling Language}~\cite{CyberSecurityModelingLanguage2010}, \emph{Security Goal Models}~\cite{SecurityGoalModels}, and \emph{Unified Parameterizable Attack Trees}~\cite{UnifiedParameterizableAttackTrees}, are summarized in Table~\ref{tab: sequential attack and defense modeling}. 
All frameworks fulfill the countermeasure nodes property, and only Security Goal Indicator Trees do not fulfill the attack vectors property. 
In addition, Boolean Logic Driven Markov Processes (BDMPs) is the only framework that offers a wide range of connectors and, therefore, at least partially fulfills the dynamic connectors property. 
This section describes BDMPs and Cyber Security Modeling Language (CySeMoL), which are the two frameworks that at least partially fulfill five of the seven identified properties.

\subsubsection{Boolean Logic Driven Markov Processes}\label{sec: boolean logic driven markov processes}

\emph{Boolean logic driven Markov processes (BDMPs)} are a security modeling framework, which can also be used to perform risk assessment~\cite{BooleanLogicDrivenMarkovProcess}. 
It was invented by Bouissou and Bon~\cite{BooleanLogicDrivenMarkovProcess} in 2003 for the safety and reliability domains, and it was later adapted to security modeling by Piètre-Cambacédès and Bouissou in 2010. 
BDMPs combine the readability of attack trees with the modeling power of Markov chains. 
The root (top event) of a BDMP represents the main goal of the attack, and the leaves represent the attack steps or security events.
BDMPs offer a wide range of node attributes, including time-domain metrics, such as mean-time to success, attack tree-related metrics, such as costs of attacks, boolean indicators, such as specific requirements, and risk assessment tools, such as sensibility graphs.

However, the lack of edge attributes, in addition to issues of usability with respect to leaf nodes and connectors~\cite{BDMPCritic}, render BDMPs impractical for risk assessment.

\subsubsection{Cyber Security Modeling Language}\label{sec: cyber security modeling language}

\emph{Cyber security modeling language (CySeMoL)} was developed by Sommestad et al. in 2010 to assess the cyber security of \emph{supervisory control and data acquisition (SCADA)} system architectures~\cite{CyberSecurityModelingLanguage2010, CyberSecurityModelingLanguage2013}.
Simply modeling the system architecture and the characteristics of the involved assets is sufficient, as CySeMoL already includes information about how attacks and defenses are quantitatively related. 
The attacker is assumed to be a professional penetration tester with a fixed time of one week to perform an attack.
CySeMoL was extended by Holm in 2014 and renamed to \emph{predictive, probabilistic cyber security modeling language ((P$^2$)CySeMoL)}, introducing more flexible and useful computations, the possibility to model assets, attacks, and defenses that are not necessarily SCADA-related, and the option to specify the time needed to perform an attack~\cite{PredictiveProbabilisticCyberSecurityModelingLanguage}.
Computations can be conducted automatically (i.e., without personalized inputs) as (P$^2$)CySeMoL already includes qualitative information gathered from literature reviews, empirical studies, as well as surveys involving domain experts~\cite{CyberSecurityModelingLanguage2010, CyberSecurityModelingLanguage2013, PredictiveProbabilisticCyberSecurityModelingLanguage}.

The results of the computations show the likelihood of an attack. 
However, the severity of an attack is not considered, and therefore the risk of an attack cannot be properly assessed. 
Furthermore, (P$^2$)CySeMoL does not include connectors, and therefore it seems an inconvenient tool for graphical risk assessment.

\subsection{Summary of Remarks}\label{sec2: summary of remarks}

This section provides an overview of thirty-three frameworks for analysis of attack and defense scenarios, and it describes eight of these frameworks in more detail. 
Thirty frameworks fulfill the attack vectors property, sixteen frameworks fulfill the countermeasure nodes property, and only thirteen frameworks fulfill the DAG structure property.
In addition, node/edge attributes and connectors are, in most cases, fixed and limited, thereby reducing the usability and usefulness of the frameworks with respect to the purposes of risk assessment. 
The complex nature and rapid development of (information) systems, attacks, and defenses motivate the need for proper risk management.
Existing methods are mainly consisting of tables with graphical solutions mostly utilized for support, if at all.
As shown in this section, current graphical solutions support threat or vulnerability management and sometimes even calculations to determine which attack vector might be the easiest to execute or, in other terms, which is most probable to occur.
The risk value cannot be equated with probability, though, and is usually determined using the probability of an event and its impact.
However, none of the methods described in this section can represent an event's consequences and impact, rendering them incapable of performing risk assessment.


\section{Cardio-respiratory Fitness Prediction}\label{4}
\mlhc{CRF is one of the strongest predictors of CVD compared with other risk factors like hypertension and type 2 diabetes~\citep{predictor}. Routinely assessing CRF through VO$_2$max, which is considered the benchmark measurement, provides valuable insights into a person's overall fitness. However, obtaining gold-standard VO$_2$max measurements, as shown in Figure~\ref{fig:workflow}, is time-consuming and thus rarely performed in clinical settings. In particular, it requires participants to undergo a maximal exercise test to reach exhaustion on a treadmill, while wearing a face mask with a computerized gas analysis system to monitor ventilation and expired gas fractions.}

\mlhc{Recently, less-accurate measurement schemes such as  sub-maximal exercise tests (silver-standard) utilizing modern wearables embedded with accelerometers and ECG sensors have started to provide opportunities for population-level fitness prediction. However, this alternative measurement method has been shown to demonstrate a measurement bias ranging from -3.0 to -1.6 ml O2/\textit{min}/\textit{kg} and a Pearson’s r ranging from 0.57 to 0.79~\citep{gonzales2020submaximal} compared to gold-standard. Apart from producing less accurate VO$_2$max values, these measurements also exhibit distribution mismatches, making it difficult to integrate into clinical practice.}

\mlhc{Similar to other healthcare applications, the distribution shift between silver- and gold-standard labels in the CRF prediction task is often ill-defined. 
% Consequently, addressing the distribution shift is crucial to improve the performance and reliability of CRF prediction models.
To tackle this issue, this paper aims to adapt the source domain, characterized by noisy yet large-scale silver-standard labels, to the target domain with small-scale gold-standard datasets. Specifically, we introduce a novel adversarial-based unsupervised domain adaptation framework with multiple domain discriminators, i.e., \systemname{} to learn domain-invariant features and improve model validation on gold-standard VO$_2$max prediction. }

% However, obtaining gold-standard VO2max measurements routinely is rarely performed in clinical settings due to its time-consuming nature \hj{obtaining gold-standard VO$_2$max measurements, as shown in Figure [FIGURE1], is time-consuming and thus rarely performed in clinical settings.}\hj{cite} Recently, modern wearables such as Apple Watch, can capture dynamic biosignals and \hj{provide opportunities for population-level fitness prediction through less accurate measurement schemes} follow less accurate measurement schemes providing the opportunity to improve population-level fitness prediction. However, these measurements result in silver-standard VO$_2$max with lower accuracy and exhibit distribution mismatches, making it difficult to integrate into clinical practice.}

% \mlhc{Specifically, in CRF and other healthcare applications, the issue of label distribution shift \hj{between xxx? from silver-standard labelling?} is prevalent and often ill-defined. \yw{How to make here more coherent and logical.} \hj{To solve this, }This paper aims to addresses the mismatch of label distribution shift caused by measurement error where two datasets are collected using two different measurements. Our study focuses on \hj{a realistic scenario where} two cohorts annotated with small-scale gold-standard measurement and large-scale silver-standard measurement that exhibit prediction bias and demographic difference (\S\ref{4.1}). These mismatches motivate domain adaptation tasks, where the goal is to transfer knowledge from a source domain with abundant data but noisy labels to a target domain with high-quality labels but limited data. }

% Therefore, the setting requires the proposed methods to learn domain-invariant features without overfitting to the large noisy distribution. Towards this goal, we introduce a novel adversarial-based unsupervised domain adaptation framework with multiple domain discriminators, i.e., \systemname{}.








\section{Methods}
This work introduces a novel unsupervised domain adaptation framework that utilizes multi-discriminator during adversarial training. The overall model architecture and multi-discriminator training scheme are shown in Figure~\ref{fig:workflow}. 
Herein, in this section, we formally discuss the problem formulation (\S\ref{3.1}) and details of our framework, which includes the first-step pre-training and second-step multi-discriminator domain adaptation training (\S\ref{3.3}). 

\subsection{Problem Formulation and Notation}\label{3.1}
Here, we denote \(\bm{D_s}\) as the source domain containing silver-standard labels and \(\bm{D_t}\) as the target domain with gold-standard labels, as shown in Figure~\ref{fig:workflow}. 
For each domain, we assume the data as \bm{$X = (x_1, ..., x_n) \in \mathbb{R}^{N\times T \times F}$} corresponds to the accelerometer and Electrocardiogram (ECG) data from a chest ECG device \imt{
and a target regression VO$_2$max labels \bm{$y = (y_1, ..., y_n)\in \mathbb{R}^{N}$}. 
}
%\bm{$x[n]$} corresponds to a specific subject with \bm{$T$} length and \bm{$F$} features. 
Additionally, we take into account contextual information such as the height or weight as metadata \(\bm{M = (m_1, ..., m_n)\in \mathbb{R}^{N \times F} }\). \imt{For the input data $\bm{X}$ and $\bm{M}$, \bm{$N$} represents the number of samples/subjects, \bm{$T$} represents the length of input sequences, and \bm{$F$} represents the number of input features.   }
Besides, \imt{
We use coarse- and fine-grained domain labels to train our model during adaptation and utilize multiple discriminators to differentiate between them. In particular, \bm{$y_c = (y_c[1], ..., (y_c[n])$} is the categorical value representing the coarse-grained binary domain label. \bm{$y_d = (y_d[1], ..., (y_d[n])$} is the numerical value that denotes the fine-grained domain distribution label. 
}
The coarse-grained domain discriminator is \(\bm{D_c}\) and the fine-grained domain discriminator is \(\bm{D_f}\). Also, for the training process, we denote the feature encoder with \(\bm{E}\) and the regression predictor with \(\bm{G_y}\). 
The overall networks thus can be represented as \(\bm{\hat{y_c}= D_c \cdot E }\), \(\bm{\hat{y_d} = D_f \cdot E }\) and \(\bm{\hat{y} = G_y \cdot E }\). The full table is shown in Table~\ref{tab:notation}.

\imt{\subsection{Unsupervised Domain Adaptation and Multi-discriminator Adversarial Training} \label{3.3}}
Domain adaptation is a method for learning a mapping between domains with distinct distributions, including data distribution shifts such as covariate shift, conditional shift, and label distribution ~\citep{da_survey}. In this paper, we propose \systemname{}, the unsupervised domain adversarial training, to address the label distribution shift problem,
particularly when the source domain contains numerous noisy labels. 
% To achieve this goal, we develop multi-discriminators to distinguish the sample domain information, force the feature extractor to generate more domain-invariant features, and boost the predictor's learning performance. In particular, \systemname{} contains a coarse-grained discriminator and a fine-grained discriminator to discriminate domain labels.


\imt{As shown in Figure~\ref{fig:workflow}, after pre-training on source domain with large-scale silver-standard labels, we first incorporate part of prior knowledge from the \(\bm{D_s}\) to create the adversarial training environment. Then we use the mixed silver-standard and gold-standard data to train the predictors and discriminator during the adaptation phase.}

\imt{
In particular, the adversarial training process consists of an encoder ($\bm{E}$), a VO2max label predictor($\bm{G_y}$), and two domain classifiers/discriminators ($\bm{D}$) designed for label shift problems by distinguishing both the domain and domain distribution information. During training, the fine- ($\bm{D_f}$)and coarse-grained ($\bm{D_c}$) discriminators are first optimized to identify the domain of each sample (i.e., max{$\bm{D_f,D_c}$}). In adversarial, the label predictor and encoder are then optimized to predict continuous fitness values from encoding(i.e., min{$\bm{E, G_y}$}).  The above-mentioned adversarial process will finally achieve the trade-off (i.e., the best prediction result in the most difficult-to-distinguish domain). 
}




\subsubsection{Coarse-grained discriminator}
The coarse-grained discriminator (\(\bm{D_c}\)) is similar to other DAs~\citep{Mathelin-2020,Zhao-2018} and follows the DANN~\citep{Ganin-2015} style. In other words, \(\bm{D_c}\) aims to discriminate the source of each data point \imt{in the mixture of pre-training and target labeled data} as a binary classification task, where 0 represents the data comes from the \(\bm{D_s}\) and 1 from \(\bm{D_t}\). In specific, After getting the representation matrix by fine-tuned feature extractor, two fully connected layers with the corresponding activation in \(\bm{D_c}\) are used to discriminate the rough binary domains labels and predict a probability vector \(\bm{\hat{y_c}}\). Let \(\bm{\hat{Y_c} = {\hat{y_c}[n]}}\) denote the predicted probability vectors for all the data points in (\(\bm{D_t}\)). The classification loss of the coarse-grained discriminator is defined as:
\begin{equation}
L_{CSE} = \sum_{N}l_c(y_c[n], \hat{{y_c}}[n])
\end{equation}
where \(l_c\) is the cross entropy loss of a single data point, and by optimizing \(\bm{L_{CSE}}\) for \(\bm{D_c}\), we can force the extractor to learn a general feature by maximizing such divergence.

\subsubsection{Fine-grained discriminator}

However, a simple binary classification task cannot properly represent the domain label distribution. Therefore, we augment adversarial training with a fine-grained discriminator (\(\bm{D_f}\)) to discriminate distribution differences. Specifically, instead of generating the binary domain labels (0 or 1) for the source or target domain for each sample, we construct a more complex pseudo-label (\(\bm{y_d}\)) to represent its domain label distribution. Based on our observation of the health outcome labels, which conform to the Gaussian distribution, as shown in Figure~\ref{distribution}, we then assign the (\(\bm{y_d}\)) for each training sample \imt{during adaptation}.
\imt{Specifically, $\bm{y_d}$ represents the mean and variance of the regression label distribution. This label is based on whether the sample is from the pre-training or target domain.
}
Therefore, after generating the feature matrix using \(\bm{E}\), \(\bm{D_f}\) is designed to distinguish the mean and variance of the label distribution using two fully connected layers with the corresponding activation. Let \(\bm{\hat{Y_d} = {\hat{y_d}}[n]}\) denote the predicted probability vectors for all the data points in (\(\bm{D_t}\)). Then, the loss of the fine-grained discriminator is defined as:
\begin{equation}
L_{GLL} = \sum_{N}l_g(y_d[n], \hat{{y_d}}[n])
\end{equation}
where \(l_g\) is each data point's Gaussian Negative Log-Likelihood (GLL) loss. In particular, GLL optimizes the mean and variance of a distribution and thus further maximizes the nuance changes among the sample and updates the discriminator.


After that, \(D_c\) and \(D_f\) can maximize the difference between the source and target domains for the multiple-domain discriminator training scheme. Meanwhile, the encoder and the predictor try to maximize the correct prediction of \(y\). These two modules play two games and finally reach a balance during the training. As a result, the encoder and predictor can learn a representation that cannot tell the difference between the source and target domains after the training is converged.


\subsubsection{Objective functions and training}
For adversarial training, the shared encoder first leverages the pre-trained model and extract general feature. Then discriminators are trained simultaneously to differentiate the domain labels. The predictor is used for the regression healthcare outcomes prediction task, and a mean squared error loss \(\bm{L_{MSE}}\) is applied to optimize the \(\bm{G_y}\). The detailed training overflow and input for each module are shown in Figure~\ref{fig:train_overflow}. Finally, the whole framework can be optimized by the total loss 
\(\bm{L}\), which is defined as:
\begin{equation}
L = \alpha L_{MSE} - \lambda_1L_{CSE} -\lambda_2L_{GLL}
\end{equation}
% \hj{loss notation is non-identical with the description}\yw{fixed}
We optimize the overall loss \(\bm{L}\) to minimize the predictor loss while maximizing the loss of domain discriminators. In detail, $\alpha$ is used to scale down the predictor loss to the same level as predictors, $\lambda_1$ and $\lambda_2$ control the relative weight of the discriminator loss, and $\lambda_1$ + $\lambda_2$ = 1. Hyperparameter choice details are discussed in Section~\S\ref{experiment}.









\section{Experiments}
\paragraph{Implementation details} Our models are implemented in Pytorch~\cite{Paszke17nipsw} and trained from scratch with random weights on our synthetic dataset. 
For the experiments on our real dataset, we finetune our pre-trained network on the real training data. 
We use Adam optimizer\cite{Kingma15iclr} with a learning rate of $1\mathrm{e}{-3}$ and batch size of 4 and train for $1250$ epochs on the synthetic and real dataset.
We supervise our network with the $L1$ loss computed on the predicted and ground truth occlusion-free images.
To measure the quality of the reconstructed image, we use structural similarity index measure (SSIM) \cite{Wang04tip}, peak signal-to-noise ratio (PSNR), and mean absolute error (MAE).

\paragraph{Baselines}
We evaluate our approach against four state-of-the-art image-inpainting solutions: MAT \cite{li_mat22cvpr}, MISF \cite{li_misf22cvpr}, PUT \cite{liu22cvpr}, and ZITS \cite{dong22cvpr}.
These baselines are trained on challenging large-scale datasets with over eight million images. 
Thus, for our experiments, we use the weights provided by the authors. %
Since dynamic occlusion removal from a single viewpoint using an event camera is a novel task, there currently do not exist any event-based approaches that tackle this problem directly.
The closest baseline uses events and frames to reconstruct the background image from multiple viewpoints (EF-SAI) \cite{liao22cvpr}.
EF-SAI \cite{liao22cvpr} uses a refocus module with events to blur the foreground and focus on the target depth plane of the background. 
We adapt this baseline for our task by providing it with a single image and events from only a single viewpoint and train this network on our dataset from scratch.
Additionally, we also compare against events-to-image reconstruction methods \cite{Rebecq19cvpr}.
This method is adapted by combining the reconstructed event images with the occluded images using the groundtruth mask.
Note that all of the evaluated baselines were originally not designed for this particular task but are the closest related methods applicable to our task. %

\subsection{Results on Synthetic Dataset}
To compare our method against the baselines in controlled conditions, we evaluate all methods on our synthetic dataset.
\Tab \ref{tab:syb_main} summarizes the quantitative results.
Our method outperforms the best image inpainting baseline by \SI{3}{dB} in terms of PSNR.
\Fig \ref{fig:syn_qual} shows qualitative comparisons between different methods.
Image inpainting methods tend to hallucinate the background, resulting in visually more pleasing information rather than the true scene.
An example of this can be seen in \Fig \ref{fig:syn_qual} third row, where the inpainting method is unable to reconstruct the characters on the bus, whereas our method is able to better preserve this information.
We also outperform the event-based synthetic aperture imaging baseline EF-SAI \cite{liao22cvpr} and the event-to-image reconstruction baseline E2VID \cite{Rebecq19cvpr}.
Although the input to our method and EF-SAI consists of events and a single image, the EF-SAI baseline was designed for multi-view reconstruction.
The simple event accumulation baseline is one of the lowest performing baselines even with the knowledge of the correct contrast threshold, as the occlusions are too complex for the basic event generation model to capture the intensity changes as discussed in the \Sec \ref{sec:method:basic}.
We also analyze the effect of occlusion density on the performance of all the methods and summarize them in \Tab \ref{tab:syn_coverage}.
As expected, increasing the occlusion density decreases the performance of all the approaches.
However, at higher occlusion densities, the image inpainting methods drastically degrade in performance as they tend to hallucinate occluded areas.
In contrast, our method uses events that provide continuous intensity changes, which results in a better performance at higher occlusion densities.
We provide more qualitative results of other baselines in the supplementary material.

\begin{table}[!t]
    \centering
    \begin{adjustbox}{max width=\linewidth}
    \setlength{\tabcolsep}{4pt}
    {\small
    \begin{tabular}{lcccc}
        \toprule
         Method & Input  & PSNR $\uparrow$  &  SSIM $\uparrow$ & MAE $\downarrow$ \\
        \midrule
        MAT \cite{li_mat22cvpr} & I & 30.7620 & 0.9217 & 0.0107 \\
        MISF \cite{li_misf22cvpr}& I & 31.1884 & 0.9229 & 0.0101  \\
        PUT \cite{liu22cvpr} & I & 26.9858 & 0.8608 & 0.0187  \\
        ZITS \cite{dong22cvpr}& I & 31.2971 & 0.9328 & 0.0100  \\
        EF-SAI \cite{liao22cvpr}& I+E & 26.7557 & 0.8688 & 0.0257  \\
        E2VID \cite{Rebecq19cvpr}& E & 19.2278 & 0.6086 & 0.0508  \\
        Ours (Acc. Method)& E & 20.3444 & 0.6955 & 0.0373  \\
        Ours (Learning) & I+E & \textbf{34.6203} & \textbf{0.9536} & \textbf{0.0085}  \\
        \bottomrule
    \end{tabular}}
    \end{adjustbox}
    \caption{Reconstruction performance on our synthetic dataset. 'I' and 'E' stand for image and events, respectively.}
    \label{tab:syb_main}
\end{table}

\input{figures/fig_syn_qual}
\begin{table}[!t]
    \centering
    \begin{adjustbox}{max width=\linewidth}
    \setlength{\tabcolsep}{4pt}
    {\small
    \begin{tabular}{lccccccc}
        \toprule
        Method & Input  & \multicolumn{6}{c}{PSNR $\uparrow$} \\
        Coverage &   & 10\% &  20\% &  30\%&  40\%&  50\%&  60\%\\
        \midrule
        MAT \cite{li_mat22cvpr} & I & 35.4937 & 31.4344 & 31.895 & 29.8665 & 27.7057 & 28.1765\\
        MISF \cite{li_misf22cvpr}& I & 35.6868 & 31.8720 & 31.7689 & 30.5100 &  28.3776 & 28.9148 \\
        PUT \cite{liu22cvpr} & I & 32.7979 & 28.6378 & 27.0988 & 26.1156 & 23.7555 & 23.5093 \\
        ZITS \cite{dong22cvpr}& I & 35.6619 &  31.9159 & 32.3340 &  30.4923 & 28.3828 & 28.9959 \\
        EF-SAI \cite{liao22cvpr}& I+E & 29.2994 & 27.9049 & 27.7354 & 26.2447 & 24.7267 & 24.6231\\
        E2VID \cite{Rebecq19cvpr}& E & 22.0048 & 19.8868 & 20.2063 & 18.1452 & 17.5353 & 17.5884 \\
        Ours (Acc. Method)& E & 28.0390 & 22.4906 & 20.6897 & 17.9743 & 17.0200 & 15.8527\\
        Ours (Learning) & I+E & \textbf{40.1286} &\textbf{ 36.3622} & \textbf{35.8476} & \textbf{33.2527} & \textbf{31.0083} & \textbf{31.1224}\\
        \bottomrule
    \end{tabular}}
    \end{adjustbox}
    \caption{Reconstruction performance  on our synthetic dataset in terms of PSNR divided according to the occlusion density of the test samples.}
    \label{tab:syn_coverage}
\end{table}

\paragraph{Ablation Study}
To study the effect of the EAM and OFF modules in our network, we report the network performance by removing the specific modules.
For computational reasons, we performed the ablation of the OFF module without including the EAM module, which reduces significantly the training time.
As shown in \Tab \ref{tab:syn_ablation}, adding the OFF module to the network improves the performance with respect to all metrics. 
This improvement can be explained by the ability of the OFF module to better localize the occlusions and adaptively fuse image and event features throughout the network scales.
Including the EAM module leads to an even higher performance increase of 1.9 dB in PSNR.
One possible reason for the improvement over the network without recurrent event encoding is that the EAM module is better at handling multiple overlapping occlusions by selecting events relevant to the true background and ignoring the redundant event information.
For more ablation studies showing the effect of the event and image features on the textured and uniform image regions, we refer to the supplementary material.
\input{tables/tab_synthetic_ablation}

\subsection{Results on Real Dataset}
We also evaluate our method on our real dataset collected with a custom build beamsplitter setup.
Unlike the synthetic dataset, we do not have an occlusion mask available for the sequences.
We, therefore, approximate the mask by subtracting the occluded frame from the groundtruth frame and applying thresholding.
This mask is used as an input for the image inpainting methods and event image reconstruction baselines.
The quantitative results are shown in \Tab \ref{tab:real_gray}.
Our results on the real dataset follow a similar trend to the synthetic dataset.
In \Fig \ref{fig:real_qual}, we show the qualitative results for different sequences.
While the image inpainting method results in visually clean images, the hallucination artifacts can be clearly seen in the small image patches, e.g., the fingers around the camera are missing in the sample visualized in the first row or the watch in the sample shown in the second row. 
Our method, on the other hand, can reconstruct the background information accurately.

\global\long\def\figWidth{0.2\linewidth}
% Figure environment removed

\begin{table}[!t]
    \centering
    \begin{adjustbox}{max width=\linewidth}
    \setlength{\tabcolsep}{4pt}
    {\small
    \begin{tabular}{lcccc}
        \toprule
         Method & Input  & PSNR $\uparrow$  &  SSIM $\uparrow$ & MAE $\downarrow$ \\
        \midrule
        MAT \cite{li_mat22cvpr} & I & 26.7451 & 0.5670 & 0.0285 \\
        MISF \cite{li_misf22cvpr}& I & 29.0281 & 0.6951 & 0.0222  \\
        PUT \cite{liu22cvpr} & I & 18.9135 & 0.2726 & 0.0801  \\
        ZITS \cite{dong22cvpr}& I & 30.2675 & 0.7542 & 0.0190  \\
        EF-SAI \cite{liao22cvpr}& I+E & 30.2390 & 0.8037 & 0.0194  \\
        E2VID \cite{Rebecq19cvpr}& E & 16.5642 & 0.3203 & 0.1066  \\
        Ours ( (Acc. Method)& E & 19.5110 & 0.5685 & 0.0606  \\
        Ours (Learning) & I+E & \textbf{33.0701} & \textbf{0.8173} & \textbf{0.0166}  \\
        \bottomrule
    \end{tabular}}
    \end{adjustbox}
    \caption{Reconstruction performance on our real-world dataset.}
    \label{tab:real_gray}
\end{table}


\section{Results and Discussion}\label{6}
% \yw{should we emphasize the semi-synthetic data since we only have one datasets,}
In this section, we present the results of applying \systemname{} to the cardio-fitness prediction task and compare it with the baselines (\S\ref{6.1}). Additionally, we discuss the performance in addressing the domain shift problem (\S\ref{6.2}) and the impact of injected source domain knowledge (\S\ref{6.3}). Furthermore, we conduct an ablation study to verify the effectiveness of our framework's structure (\S\ref{6.4}). Finally, we discuss the model robustness in \S\ref{6.5} with semi-synthetic data. 

% \begin{table}
%   \caption{\textbf{Evaluation of different methods on CRF prediction task.} Each result displays the mean value with standard deviation from three-fold cross-validation. In particular, \textbf{In-domain} means the model is trained on the same \imt{domain, namely trained on \(\bm{D_t}\) and tested on $\bm{D_t}$}, while \textbf{Out-of-domain} corresponds to models trained on Fenland (\(\bm{D_s}\)) and  adapted/fine-tuned to BBVS (\(\bm{D_t}\)). All \textbf{Out-of-domain} models are evaluated on the BBVS test set.  
% %   \hj{add a seperator between ID and OoD. Also horizontally centralize OoD}\yw{fixed}
%   }
%   \resizebox{1\textwidth}{!}{%
%   \begin{tabular}
%   % {\textwidth}{p{0.15\linewidth}p{0.19\linewidth}p{0.14\linewidth}p{0.14\linewidth}p{0.14\linewidth}p{0.14\linewidth}}
%   {llllll}
%   \toprule
%     \textbf{In-domain} &\textbf{Training method} & \textbf{R$^2$} & \textbf{Corr} &  \textbf{MSE} & \textbf{MAE} \\ 
%     \midrule
%     $D_t$ \rightarrow  $D_t$ & Supervised & 0.123 $\pm$ 0.111  & 0.622 $\pm$ 0.036 & 43.778 $\pm$ 6.012  & 5.263 $\pm$ 0.277  \\
%     \toprule
%     \textbf{Out-of-domain} &\textbf{Training method} & \textbf{R$^2$} & \textbf{Corr} &  \textbf{MSE} & \textbf{MAE} \\
%     \midrule
%      $D_s$ \rightarrow  $D_t$ & \imt{Supervised} & -0.096 $\pm$ 0.100 & 0.007 $\pm$ 0.250 & 58.048 $\pm$ 10.061  & 6.336 $\pm$ 0.621  \\ 
%      &WDGRL ~\citep{wdgrl} & -0.100   $\pm$ 0.073 & 0.004 $\pm$ 0.161 & 55.611 $\pm$ 10.61 & 6.044 $\pm$ 0.615 \\
%      & Autoencoder~\citep{Srivastava_2015} & -0.067 $\pm$ 0.069 & 0.127 $\pm$ 0.222 & 53.254 $\pm$ 4.878 & 5.973 $\pm$ 0.194 \\
%     &Deep-Coral~\citep{coral} & 0.021 $\pm$ 0.073  & 0.360  $\pm$ 0.057 & 49.044 $\pm$ 6.553 &  5.638 $\pm$ 0.374\\
%     &Transfer learning (TF)  & 0.283 $\pm$ 0.037  & 0.621 $\pm$ 0.012 & 35.399 $\pm$  5.910 & 4.744 $\pm$ 0.433
%  \\ 
%   &DANN~\citep{Ganin-2015} & 0.288 $\pm$ 0.077   & 0.617 $\pm$ 0.037 & 35.458 $\pm$ 3.920 & 4.679 $\pm$ 0.382  \\
%   \hline
%     &\textbf{UDAMA (ours)} & \textbf{0.459} $\pm$ \textbf{0.063} & \textbf{0.701} $\pm$ \textbf{0.032} &  \textbf{27.469} $\pm$ \textbf{6.456} & \textbf{4.111} $\pm$ \textbf{0.353} \\
    
%     \bottomrule
%   \end{tabular}%
% }
%   \label{table-results}

% \end{table}

\begin{table}
  \caption{\textbf{Evaluation of different methods on CRF prediction task.} Each result displays the mean value with standard deviation from three-fold cross-validation. In particular, \textbf{In-domain} means the model is trained on the same domain, namely trained on \(D_t\) and tested on \(D_t\), while \textbf{Out-of-domain} corresponds to models trained on Fenland (\(D_s\)) and  adapted/fine-tuned to BBVS (\(D_t\)). All \textbf{Out-of-domain} models are evaluated on the BBVS test set.}
  \resizebox{1\textwidth}{!}{%
  \begin{tabular}
  {llllll}
  \toprule
    \textbf{In-domain} &\textbf{Training method} & \textbf{R$^2$} & \textbf{Corr} &  \textbf{MSE} & \textbf{MAE} \\ 
    \midrule
    \(D_t \rightarrow  D_t\) & \textit{Supervised} & \(0.123 \pm 0.111\)  & \(0.622 \pm 0.036\) & \(43.778 \pm 6.012\)   & \(5.263 \pm 0.277\)  \\
    \toprule
    \textbf{Out-of-domain} &\textbf{Training method} & \textbf{R$^2$} & \textbf{Corr} &  \textbf{MSE} & \textbf{MAE} \\
    \midrule
     \(D_s \rightarrow  D_t\) & \textit{Supervised} & \(-0.096 \pm 0.100\) & \(0.007 \pm 0.250\) & \(58.048 \pm 10.061\)  & \(6.336 \pm 0.621\)  \\ 
     &WDGRL ~\citep{wdgrl} & \(-0.100   \pm 0.073\) & \(0.004 \pm 0.161\) & \(55.611 \pm 10.61\) & \(6.044 \pm 0.615\) \\
     & Autoencoder~\citep{Srivastava_2015} & \(-0.067 \pm 0.069\) & \(0.127 \pm 0.222\) & \(53.254 \pm 4.878\) & \(5.973 \pm 0.194\) \\
    &Deep-Coral~\citep{coral} & \(0.021 \pm 0.073\)  & \(0.360  \pm 0.057\) & \(49.044 \pm 6.553\) &  \(5.638 \pm 0.374\)\\
    &Transfer learning (TF)  & \(0.283 \pm 0.037\)  & \(0.621 \pm 0.012\) & \(35.399 \pm  5.910\) & \(4.744 \pm 0.433\) \\ 
  &DANN~\citep{Ganin-2015} & \(0.288 \pm 0.077\)   & \(0.617 \pm 0.037\) & \(35.458 \pm 3.920\) & \(4.679 \pm 0.382\)  \\
  \hline
    &\textbf{UDAMA (ours)} & \textbf{0.459} \(\pm\) \textbf{0.063} & \textbf{0.701} \(\pm\) \textbf{0.032} &  \textbf{27.469} \(\pm\) \textbf{6.456} & \textbf{4.111} \(\pm\) \textbf{0.353} \\
    
    \bottomrule
  \end{tabular}%
}
  \label{table-results}
\end{table}



\subsection{Fitness prediction}\label{6.1}
We took 60 participants from the BBVS dataset as test samples to predict their CRF by predicting VO$_{2}$max values. The comparison between the proposed domain adaptation framework and baseline approaches is shown in Table~\ref{table-results}.

\imt{First, in the out-of-domain comparison, adversarial-based DA or transfer learning shows better performance than the discrepancy-based method under label distribution shift.} In particular, the discrepancy-based method, Deep-coral, increases the Corr and MAE to 0.36 and 5.638, respectively. Meanwhile, WDGRL aims to minimize the feature difference by employing Wasserstein distance, yielding results comparable to the out-of-domain supervised method, which directly applies the model trained on Fenland for testing BBVS. In contrast, the adversarial-based methods here display better results. DANN learns a representation that is predictive of the regression task but uninformative to the input domain and improves the Corr and MAE to 0.617 and 4.679, respectively. According to Corr and MAE, methods such as transfer learning with fine-tuning techniques also improve performance, achieving 0.621 and 4.744, compared to the discrepancy-based method. 

\imt{In general, high Corr and R2 values and low MSE and MAE demonstrate the model's ability to leverage noisy, large-scale labeled VO2max data for gold-standard VO2max prediction under label shift. However, the limited size of the test set might result in increased uncertainty during model evaluation. The results from both TF and DANN are similar as shown in Table~\ref{table-results}}. In contrast, \systemname{} utilizing both fine-tuning and adversarial-based domain adaptation methods outperforms all the abovementioned baselines. We observe that the correlation (Corr) outperforms the basic transfer learning methods by 12.9\%, the MSE increases by 22.4\%, and R$^2$ improves by 62.2\%. Moreover, compared with the in-domain supervised training on $\bm{D_t}$, \systemname{} shows a significant increase, improving R$^2$ from 0.123 to 0.459. 
% Therefore, \systemname{} has the capability to make use of large, weakly labeled VO2max data for accurate gold-standard VO2max predictions despite label shift }
Our method also achieves good performance when generalizing the model from the in-domain to out-of-domain BBVS setting, compared with the dramatic performance drop-down, as shown in Table~\ref{table-results}. Therefore, \systemname{} can leverage the large-scale noisy datasets information and alleviate the model performance degeneration performance compared with directly validating models on small-scale sensing datasets. 
% Figure environment removed


\subsection{Domain shift}\label{6.2}
%%%our method can catch the mean and the range here, while the others fail xxx 
To better compare whether \systemname{} or baseline DA methods can solve the label distribution shift problem effectively. Figure~\ref{distribution} presents the predicted label distribution of the BBVS test set from different methods. 
% \hj{move plots into 2 plots per line}\yw{fixed}
First, it shows that the \(\bm{D_s}\) dataset (i.e., Fenland) shares a different VO$_{2}$max underlying distribution compared to \(\bm{D_t}\) BBVS. Our results demonstrate that \systemname{} can learn the small dataset distribution during the adaptation phase and achieves promising results compared with other methods. Besides, we observe that both adversarial-based methods and transfer learning capture the mean and range of target domain distribution as shown in Figure~\cref{fig:distribution-shift_good}, whereas the discrepancy-based methods fail to learn the general distribution. We attribute this performance degeneration to the fact that discrepancy-based approaches, mainly designed to minimize the divergence between feature spaces, cannot alleviate the impact of noisy labeling.

Moreover, we use the Hellinger Distance (HD), which calculates the similarity of distributions between prediction and ground truth to examine the distance between two label distributions. Specifically, our framework's prediction of fitness level lies in the same range as the ground truth, while methods like Deep-coral or WDGRL fail at learning within this range. Besides, the distribution of \systemname{} ties close compared to baseline methods, where the normalized HD for \systemname{} is 0.179, and HD for TF is 0.264, for Deep-coral is 0.305. These results indicate that our framework effectively alleviates the distribution shift problem of the VO$_{2}$max prediction task and \systemname{} can leverage noisy silver-standard data to improve the performance on the gold-standard dataset. 

% Figure environment removed

% % Figure environment removed





% % Figure environment removed


\subsection{Impact of Injected Knowledge from Source Domain}\label{6.3}
% \yw{modifying the caption here...}\yw{fixed}
Instead of using a generator, we incorporated source domain knowledge to create the adversarial training environment. Figure~\cref{fig:scale} shows the average performance of adding the different scales of injected source domain samples to the target domain.
Each box plot shows the average MSE results of 15 runs with added random samples from the source domain. 
As shown in Figure~\cref{fig:scale}, our method performs better than the baseline methods with added samples in all cases, even after adding 100\% of noisy samples of $\bm{D_t}$ from $\bm{D_s}$.
Specifically, we observe that it achieves the best results and showcases the most positive transfer by only adding 0.4\% of the source domain, which equals 10\% of \(\bm{D_t}\). In contrast, if we continue to add more noisy information to the adaptation stage, the performance will gradually decrease as the source and target domain data reach the ratio of 1:1. After that, the adaptation tends to learn the noisy source domain representation instead of the target domain, yielding a negative transfer. Therefore, only injecting a few samples from the source domain to create the adversarial training environment might help to learn more domain-invariant features and achieve optimal results.

% % Figure environment removed






\subsection{Ablation Study} \label{6.4}
Our framework comprises two joint discriminators, so we perform an ablation study to understand each discriminator's effect. Based on the observation that incorporating a 10\% amount of BBVS from the \(\bm{D_t}\) Fenland dataset achieves the best transfer, we train different discriminators under this setting. We observe that a single discriminator (coarse-grained or fine-grained) exhibits more competitive performance than the baseline TF or DANN methods, as shown in Table~\ref{table:ablation study} and Figure~\cref{fig:scale}. Specifically, the fine-grained discriminator alone significantly outperforms the baseline results: +15.8\% TF (based on MSE). This indicates that utilizing a fine-grained distribution domain discriminator, which discriminates the domain label distribution, enhances the training framework. Therefore, with the combination of \(\bm{D_f}\) and \(\bm{D_c}\), \systemname{} can learn the cross-domain information without capturing the domain information of the input and alleviate the noisy labeling problem. 

\imt{Additionally, our method is unique in that it uses wearable-based CRF prediction task, unlike other methods that solely rely on anthropometric data~\citep{Nes_2011}. Our experiments have demonstrated that combining sensor data with anthropometric information leads to improved prediction accuracy. Although the sensor data alone is unreliable and insufficient for accurate VO${2}$max prediction, when combined with anthropometric data, the accuracy of \systemname{} increases from 0.679 (using only anthropometric data) to 0.701 (using anthropometric data and wearable sensor data such as acceleration, heart rate, and heart rate variability). As a result, using anthropometric and low-cost wearable devices together enables more accurate VO$_{2}$max prediction through \systemname{}.  }




\begin{table*}
  \centering 
  \small 
  \caption{\textbf{Ablation study by removing one of the discriminators.
%   \hj{Keep precisions identical}\yw{fixed}
  }}
  \begin{tabular*}{\textwidth}{l @{\extracolsep{\fill}} llll}
  \toprule
    \textbf{Discriminator} & \textbf{R$^2$} & \textbf{Corr} & \textbf{MSE} & \textbf{MAE} \\
    \midrule
    Coarse-grained & 0.339 $\pm$ 0.053 & 0.626 $\pm$ 0.032 & 33.485 $\pm$ 6.740  & 4.580 $\pm$ 0.337 \\ 
    Fine-grained  & 0.394 $\pm$ 0.030 & 0.666 $\pm$ 0.030 & 30.578 $\pm$ 5.300  & 4.498 $\pm$ 0.324 \\ 
    \textbf{UDAMA (ours)} & \textbf{0.459} $\pm$ \textbf{0.063} & \textbf{0.701} $\pm$ \textbf{0.032} &  \textbf{27.469} $\pm$ \textbf{6.456} & \textbf{4.111} $\pm$ \textbf{0.353} \\
    \bottomrule
  \end{tabular*}

  \label{table:ablation study} 
\end{table*}

%%% KL-divergency and simulated shifted dataset

\begin{table*}
  \centering 
  \small 
  \caption{\imt{\textbf{Simulated Label Distribution Shift}. Kullback–Leibler divergence (KL divergence) calculates the distribution difference between the shifted $D_s$ and fixed $D_t$.}
  }
  \begin{tabular*}{\textwidth}{l @{\extracolsep{\fill}} llll}
  \toprule
      Source dataset& \textbf{KL divergence} & \textbf{UDAMA Corr} & \textbf{DANN Corr }  \\
    \midrule
    Fenland  & 0.461  & 0.701 $\pm$ 0.032  &  0.617 $\pm$ 0.037 \\ 
     Left shifted Fenland & 1.607 & 0.656 $\pm$ 0.065 &  0.589 $\pm$ 0.027 \\ 
  Right shifted Fenland & 3.188  & 0.646 $\pm$ 0.035  &  0.608 $\pm$ 0.037  \\
    \bottomrule
  \end{tabular*}

  \label{table:simulation} 
  \vspace{-0.1in}
\end{table*}


% % Figure environment removed



% % Figure environment removed

\subsection{Robustness assessment with semi-synthetic data shift}\label{6.5}
% \yw{add a section to describe the synthetic label shift method, its impact of our method to fitness prediction using domain adaptation method}
% \imt{There are not many available datasets with similar label distribution shift problems in the literature, especially in healthcare domain. Therefore, we perform additional experiments with a semi-synthetic dataset to generate various label distribution shifts\cm{this is a killer sentence..you are essentially saying there is no application for your work. how about saying, while the problem is very real cite examples, there are very few open available healthcare datasets with label distribution shift on which we could try our technique }. 
\mlhc{Although the distribution shift among gold- and silver-standard labels are prevalent in healthcare applications, there are very few open available datasets with such a challenge where we can apply our method. Therefore, to further evaluate the efficacy of \systemname{}, we generate semi-synthetic datasets with various label distribution shifts for the cardio-fitness prediction.}

Motivated by the label distribution shift simulation for classification, as seen in ~\citep{Lipton_2018}, 
we shift the labels in the $\bm{D_s}$ by a fixed offset and Gaussian noise. By conducting experiments with varying degrees of label shifts, we gain a comprehensive understanding of the effect of label shifts on CRF prediction tasks. As the shift becomes greater from the target domain, the performance decreases. In particular, we show two extreme cases by pushing the source domain shift to left and right to stress-test \systemname{}, and the results are shown in Table ~\ref{table:simulation}. Despite the increasing KL divergence, which indicates a greater deviation from the ground truth data, our method still displays robust performance compared to the baseline DA method, especially in the case of low fitness on the left side. These stress tests highlight the versatility and robustness of our model in dealing with different input distributions.

% \subsection{Interpretability}\label{6.5}
% To understand the effectiveness of \systemname{}, we further examine the saliency map to interpret predictions on the data-point level. Finally, motivated by recent works~\citep{cam-timeseries,RAM,cam}, we employ the regression activation mapping (RAM) framework to visualize which timestamps were relevant to predict the respective VO$_{2}$max value.

% For the 60 participants in the testset, we plot all the RAM output for the same participants using \systemname{} and other baselines such as DeepCoral, DANN, and TF on their raw accelerometer data. The color of each  timestep in ~\ref{fig:udama_dann} represents the extracted salience weight.
% First, we demonstrate the general interpretability for each method, as shown in Figure~\ref{fig:ram_general}. 
% \imt{To provide a comprehensive view of the model interpretability, Figure~\ref{fig:ram_general} showcases the interpretability of the model for the entire test set. After performing RAM on each participant, we aggregate the mean of each timestep and display the density plot for 600 timesteps. The results in Figure~\ref{fig:ram_general} indicate that UDAMA can better differentiate the importance of each timestep compared to other methods, as evidenced by its wider range and lower density.
% }
% In contrast, other methods rate each timestep nearly equally for all the participants and manifest in low accuracy.


% %%%heatmap ; explain the weight? the value... 1/600 = 0.00167
% Notably, for the detailed salient map for a specific sample, we observe that in Figure~\ref{fig:udama_dann}, the regions in \systemname{} that contribute to estimations are distinct compared with other baseline methods. Furthermore, \systemname{} can capture the saliency of temporal features, which seems to help the model predict the fitness score more accurately. Specifically, we examine RAM plots of 4 participants to indicate that \systemname{} digest more timesteps during training. Therefore, these results demonstrate that \systemname{} has better interpretability and robustness than TF and other baseline methods. 




\vspacecaption
\section{Conclusion}
\vspacecaption
\looseness -1 We proposed \alexp, an algorithm for simultaneous online model selection and bandit optimization. 
As a first, our approach leads to anytime valid guarantees for model selection and bandit regret, and does not rely on a priori determined exploration schedule. 
Further, we showed how the Lasso can be used together with the exponential weights algorithm to construct a low-variance online learner. 
This new connection between high-dimensional statistics and online learning opens up avenues for future research on high-dimensional online learning.
We established empirically that \alexp has favorable exploration--exploitation dynamics, and outperforms existing baselines. 
We addressed the open problem of \citet{agarwal2017corralling}, and showed that $\log M$ dependency for regret is achievable for linearly parametrizable rewards. However, this problem remains open for more general, non-parametric reward classes.






% ACKNOWLEDGEMENTS ONLY GO IN THE CAMERA-READY, NOT THE SUBMISSION
% \yw{confirm acknowledgement with C}
\acks{This work was supported by ERC Project 833296 (EAR) and by Nokia Bell Labs.}

\bibliography{main}

\newpage
\appendix
\begin{comment}
\section{System Architecture}
\label{appendix:architecture}
\system has a novel modularized system architecture with three key components: 
\emph{StreamManager}, 
\emph{TxnManager} and \emph{TxnScheduler}. 
These components are instantiated in each thread locally.
The execution outline of \system is presented in Algorithm~\ref{alg:algo}.
Transactional stream processing is continuous and potentially never ends (Line 1$\sim$8).
The dependency resolution and execution of state transactions are separated into two non-overlapping phases by punctuations~\cite{Tucker:2003:EPS:776752.776780} (Line 2 and 5), which guarantees that no subsequent input event will have a smaller timestamp. 
Effectively, a batch of state transactions is collected during the first phase, and processed during the second phase.

In the first phase (i.e., stream processing phase), 
the \emph{StreamManager} conducts preprocessing for every input event ($e$). Similar to some prior works~\cite{tstream}, state transactions may be issued but not immediately processed during preprocessing (Line 3).
The \emph{pre\_processing} and \emph{post\_processing} functions are exposed as APIs to users.
The \emph{TxnManager} handles dependency resolution (Line 4) among state transactions and insert decomposed operations to construct a \tpg. We discuss the detailed two-phase \tpg construction process in Section~\ref{subsec:construction}.

In the second phase  (i.e., transaction processing phase), 
the \emph{TxnManager} is first involved again to refine (Line 6) the constructed \tpg with further dependency resolution.
The \emph{TxnScheduler} 
schedules operations for concurrent execution based on the constructed \tpg according to the three dimensions of scheduling decisions (Line 7). 
In particular, a scheduling decision model $M$ is instantiated based on the constructed \tpg (Line 14).
\textbf{\circled{1}} Guided by $M$, execution threads adopt an exploration strategy (Section~\ref{subsec:explore}) to explore the constructed \tpg for operations available to be scheduled constrained by dependencies. 
\textbf{\circled{2}} 
During exploration, one or multiple operations may be treated as the 
% basic 
unit of scheduling (Section~\ref{subsec:granularity}). 
Subsequently, \textbf{\circled{3}} every thread executes operation(s) in the unit of scheduling with various abort handling mechanisms (Section~\ref{subsec:abort_handling}).
Only when state transactions are processed (i.e., committed or aborted) can the associated input events be postprocessed (Line 8) by the \emph{StreamManager} based on transaction processing results.
\end{comment}

\begin{comment}
\begin{algorithm}
\footnotesize
    \KwData{$e$ \tcp{Input event}}
    \KwData{$txn_{ts}$ \tcp{State transaction}}
    \KwData{$G$ \tcp{The currently constructed TPG}}
    \While{!finish processing of input streams}{
        \eIf(\tcp*[h]{Phase 1}){\text{$e$ is not a $punctuation$}}{
                $txn_{ts}$ $\gets$ PRE\_Processing($e$)\;
                \textbf{TPG\_Construction}($G$, $txn_{ts}$)\; 
          }(\tcp*[h]{Phase 2}){
                \textbf{TPG\_Refinement}($G$)\; 
                \textbf{TXN\_Scheduling}($G$)\; 
                POST\_Processing()\;
          }
    }
    
    \SetKwFunction{FMain}{TPG\_Construction}
    \SetKwProg{Fn}{Function}{:}{}
    \Fn{\FMain{$G$, $txn_{ts}$}}{
        $O_{1..k}$ $\gets$ \textbf{Partition} $txn_{ts}$\;
        \ForEach{\text{operation $O_{i}$ $\in$ $O_{1..k}$}}{
            \textbf{Identify} its \ld\;
            $G$ $\gets$ $G$ + $O_{i}$ \;
        }
    }
    \SetKwFunction{FMain}{TPG\_Refinement}
    \SetKwProg{Fn}{Function}{:}{}
    \Fn{\FMain{$G$}}{
        \ForEach{\text{vertex $e_{i}$ $\in$ $G$}}{
            \textbf{Identify} its \td, \pd\;
        }
    }
    
    \SetKwFunction{FMain}{TXN\_Scheduling}
    \SetKwProg{Fn}{Function}{:}{}
    \Fn{\FMain{$G$}}{
        $M$ $\gets$ Instantiated with $G$;\tcp{A decision model}
        \While{!finish scheduling of $G$
        }{
          \textbf{\circled{2}} $Scheduling Unit$ $\gets$ \textbf{\circled{1}} \emph{Explore}($G$, $M$)\; 
            \textbf{\circled{3}} \emph{Execute with Abort Handling} ($Scheduling Unit$)\; 
        }
    }
  \caption{Execution Outline of \system}
  \label{alg:algo}
\end{algorithm}
\end{comment}

\end{document}
