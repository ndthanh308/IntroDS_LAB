\section{Cardio-respiratory Fitness Prediction}\label{4}
\mlhc{CRF is one of the strongest predictors of CVD compared with other risk factors like hypertension and type 2 diabetes~\citep{predictor}. Routinely assessing CRF through VO$_2$max, which is considered the benchmark measurement, provides valuable insights into a person's overall fitness. However, obtaining gold-standard VO$_2$max measurements, as shown in Figure~\ref{fig:workflow}, is time-consuming and thus rarely performed in clinical settings. In particular, it requires participants to undergo a maximal exercise test to reach exhaustion on a treadmill, while wearing a face mask with a computerized gas analysis system to monitor ventilation and expired gas fractions.}

\mlhc{Recently, less-accurate measurement schemes such as  sub-maximal exercise tests (silver-standard) utilizing modern wearables embedded with accelerometers and ECG sensors have started to provide opportunities for population-level fitness prediction. However, this alternative measurement method has been shown to demonstrate a measurement bias ranging from -3.0 to -1.6 ml O2/\textit{min}/\textit{kg} and a Pearson’s r ranging from 0.57 to 0.79~\citep{gonzales2020submaximal} compared to gold-standard. Apart from producing less accurate VO$_2$max values, these measurements also exhibit distribution mismatches, making it difficult to integrate into clinical practice.}

\mlhc{Similar to other healthcare applications, the distribution shift between silver- and gold-standard labels in the CRF prediction task is often ill-defined. 
% Consequently, addressing the distribution shift is crucial to improve the performance and reliability of CRF prediction models.
To tackle this issue, this paper aims to adapt the source domain, characterized by noisy yet large-scale silver-standard labels, to the target domain with small-scale gold-standard datasets. Specifically, we introduce a novel adversarial-based unsupervised domain adaptation framework with multiple domain discriminators, i.e., \systemname{} to learn domain-invariant features and improve model validation on gold-standard VO$_2$max prediction. }

% However, obtaining gold-standard VO2max measurements routinely is rarely performed in clinical settings due to its time-consuming nature \hj{obtaining gold-standard VO$_2$max measurements, as shown in Figure [FIGURE1], is time-consuming and thus rarely performed in clinical settings.}\hj{cite} Recently, modern wearables such as Apple Watch, can capture dynamic biosignals and \hj{provide opportunities for population-level fitness prediction through less accurate measurement schemes} follow less accurate measurement schemes providing the opportunity to improve population-level fitness prediction. However, these measurements result in silver-standard VO$_2$max with lower accuracy and exhibit distribution mismatches, making it difficult to integrate into clinical practice.}

% \mlhc{Specifically, in CRF and other healthcare applications, the issue of label distribution shift \hj{between xxx? from silver-standard labelling?} is prevalent and often ill-defined. \yw{How to make here more coherent and logical.} \hj{To solve this, }This paper aims to addresses the mismatch of label distribution shift caused by measurement error where two datasets are collected using two different measurements. Our study focuses on \hj{a realistic scenario where} two cohorts annotated with small-scale gold-standard measurement and large-scale silver-standard measurement that exhibit prediction bias and demographic difference (\S\ref{4.1}). These mismatches motivate domain adaptation tasks, where the goal is to transfer knowledge from a source domain with abundant data but noisy labels to a target domain with high-quality labels but limited data. }

% Therefore, the setting requires the proposed methods to learn domain-invariant features without overfitting to the large noisy distribution. Towards this goal, we introduce a novel adversarial-based unsupervised domain adaptation framework with multiple domain discriminators, i.e., \systemname{}.







