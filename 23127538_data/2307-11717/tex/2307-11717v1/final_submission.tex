%%%%%%%%%%%%%%%%%%%%%%%%%%%%%%%%%%%%%%%%%%%%%%%%%%%%%%%%%%%%%%%%%%%%%%%%%%%%%%%%
%2345678901234567890123456789012345678901234567890123456789012345678901234567890
%        1         2         3         4         5         6         7         8

\documentclass[letterpaper, 10 pt, conference]{ieeeconf}  % Comment this line out if you need a4paper

%\documentclass[a4paper, 10pt, conference]{ieeeconf}      % Use this line for a4 paper

\IEEEoverridecommandlockouts                              % This command is only needed if 
                                                          % you want to use the \thanks command

\overrideIEEEmargins                                      % Needed to meet printer requirements.

%In case you encounter the following error:
%Error 1010 The PDF file may be corrupt (unable to open PDF file) OR
%Error 1000 An error occurred while parsing a contents stream. Unable to analyze the PDF file.
%This is a known problem with pdfLaTeX conversion filter. The file cannot be opened with acrobat reader
%Please use one of the alternatives below to circumvent this error by uncommenting one or the other
%\pdfobjcompresslevel=0
%\pdfminorversion=4

% See the \addtolength command later in the file to balance the column lengths
% on the last page of the document

% The following packages can be found on http:\\www.ctan.org
% \usepackage{graphics} % for pdf, bitmapped graphics files
% \usepackage{epsfig} % for postscript graphics files
% \usepackage{mathptmx} % assumes new font selection scheme installed
% \usepackage{times} % assumes new font selection scheme installed
% \usepackage{amsmath} % assumes amsmath package installed
% \usepackage{amssymb}  % assumes amsmath package installed



% The following packages can be found on http:\\www.ctan.org
\usepackage{graphics} % for pdf, bitmapped graphics files
\usepackage{epsfig} % for postscript graphics files
\usepackage{mathptmx} % assumes new font selection scheme installed
\usepackage{times} % assumes new font selection scheme installed
\usepackage{amsmath} % assumes amsmath package installed
\usepackage{amssymb}  % assumes amsmath package installed
\usepackage{algorithm} 
\usepackage{cite}
\usepackage{algpseudocode} 
\usepackage{lipsum}% http://ctan.org/pkg/lipsum
\usepackage{graphicx}% http://ctan.org/pkg/graphicx
\usepackage[dvipsnames]{xcolor}
\usepackage{eucal}
%\usepackage{flush}
\usepackage{mathtools,leftidx}
\usepackage{subcaption}
\usepackage{wrapfig}

\usepackage{multirow}
% \usepackage[table]{xcolor}

\usepackage{color, colortbl}
\usepackage[bookmarks=true]{hyperref}


\usepackage[utf8]{inputenc}
\usepackage{fontenc}
\usepackage{siunitx}
\usepackage{tikz}
\usepackage{textcomp}
\pdfminorversion 4 

\newcommand{\MA}[1]{\textcolor{blue}{{\bf MA:} #1}}
\newcommand\footnoteref[1]{\protected@xdef\@thefnmark{\ref{#1}}\@footnotemark}




\makeatletter
\newcommand*\titleheader[1]{\gdef\@titleheader{#1}}
\AtBeginDocument{%
  \let\st@red@title\@title
  \def\@title{%
    \bgroup\normalfont\large\centering\@titleheader\par\egroup
    \vskip1.5em\st@red@title}
}
\makeatother



\title{\LARGE \bf
GP-Frontier for Local Mapless Navigation
}



\author{Mahmoud Ali and Lantao Liu% <-this % stops a space
%\thanks{*This work was not supported by any organization}% <-this % stops a space
\thanks{$^{1}$Mahmoud Ali and Lantao Liu are with the Luddy School of Informatics, Computing, and Engineering, Indiana University, Bloomington, IN 47408 USA, {\tt\small \{alimaa, lantao\}@iu.edu}}
}


\titleheader{ \textcolor{gray}{This paper has been accepted for publication at 2023 IEEE International Conference on Robotics and Automation} \textcolor{blue}{\href{https://www.icra2023.org/}{(ICRA 2023)}} }




\begin{document}



\maketitle
\thispagestyle{empty}
\pagestyle{empty}


%%%%%%%%%%%%%%%%%%%%%%%%%%%%%%%%%%%%%%%%%%%%%%%%%%%%%%%%%%%%%%%%%%%%%%%%%%%%%%%%
\begin{abstract}

We propose a new frontier concept called the Gaussian Process Frontier (GP-Frontier) that can be used to locally navigate a robot towards a goal without building a map. The GP-Frontier is built on the uncertainty assessment of an efficient variant of sparse Gaussian Process.  %called  Variational Sparse  Gaussian Process (VSGP). 
Based only on local ranging sensing measurement, the GP-Frontier can be used for navigation in both known and unknown environments. The proposed method is validated through intensive evaluations, and the results show that the GP-Frontier can navigate the robot in a safe and persistent way, i.e., the robot moves in the most open space (thus reducing the risk of collision) without relying on a map or a path planner. 
A supplementary video that demonstrates the robot navigation behavior is available at \url{https://youtu.be/ndpqTNYqGfw}. 
\end{abstract}

\vspace{-5pt}
%%%%%%%%%%%%%%%%%%%%%%%%%%%%%%%%%%%%%%%%%%%%%%%%%%%%%%%%%%%%%%%%%%%%%%%%%%%%%%%%
\section{Introduction}
% Mobile robots have recently been used in different areas, such as rescue or inspection missions in harsh or crowded environments. Researchers have developed algorithms that assist the robots in reaching their goals and performing the required task in real-world environments based on sensors' data. 
% Classical path planning methods typically depend on an environment representation; however, robots may need to perform navigation tasks in unknown environments. Consequently, the robotics community developed obstacle avoidance techniques that enable the robot to navigate towards its goal while avoiding static and dynamic obstacles.
% On the other hand, Gaussian Process (GP) is used in many areas, such as classification and regression, while explicitly identifying the certainty interval of the predictions. Titsias~\cite{titsias2009variational} introduced the Variational Sparse Gaussian Process (VSGP) to decrease GP complexity. In robotics, the VSGP can be used in building the occupancy map generated from the Lidars' point clouds.
% Furthermore, obstacle avoidance techniques generally use geometry and perception data to find gaps or sub-goals and build the local path of the robot until it reaches the desired goal. 
The concept of \textit{frontier-based} navigation %and exploration 
was firstly proposed by \cite{yamauchi1997frontier}. The frontiers are defined as the boundary grids between unexplored
and explored space, and they usually appear on the maximum range of sensing (or the ``edge of the sensing sweep" that does not return any obstacle detection). 
Since the frontiers are on the boundary between known free space and unknown space that has not yet been sensed, the frontiers are used to navigate the robot to further scan the unknown space, and repeatedly, the known (mapped) territory will continuously expand by pushing the boundary toward the unknown areas, leading to interesting exploration behavior.  
When there is no new frontier left, the unknown space exploration is deemed complete. 

%To navigate the robot, the frontier-based methods explicitly detect frontiers in the current map, then a path is planned to guide the robot to move toward the frontiers. 

There are some drawbacks for existing frontier-based navigation.  
The first important issue for the existing frontier concept lies in the inappropriate assumption that the frontiers (discrete boundary grids) are independent to each other. 
In the real world, space has continuity and correlation, and this property has been ignored. 
The second issue is the reliance between frontiers and a map. The conventional frontiers are defined, and thus dependent, on a map (data) structure and, usually, the mapping process. Consequently, existing work typically leverages frontiers for unknown space exploration and map construction where the spatial coverage is an important goal.  In many practical tasks, the robot does not need to explore or map the space, but simply needs to continuously navigate in the environment and might repetitively revisit the same locations that have been visited  many times before (e.g., patrolling,  surveillance).
% https://www.overleaf.com/project/63139b7d8b88bb9f637f6e57

To tackle the above two issues simultaneously, 
we propose a new frontier concept called GP-Frontier based on %the uncertainty associated with %the occupancy surface 
a novel compact form of perception model 
constructed with an onboard ranging sensor. 
The GP-Frontier can guide and navigate the robot in known or unknown space, with or without a goal. 
Different from existing work, our solution does not rely on any map and can navigate the robot continuously and safely.  
%
This is achieved by 
utilizing the variational sparse  Gaussian Process (VSGP) to build a local occupancy surface, where all the 3D occupied points observed by the ranging sensor are projected onto a 2D circular surface %modeled with spatial correlation and also model uncertainty. 
that considers the correlations of the observed points and the uncertainty of the regression model.
Frontiers selected thus are typically located in the most open space, which is important for safe local navigation.  
In other words, the GP-Frontier shows unique navigation capabilities because its foundation is based on spatial correlation and uncertainty assessment, which is very different from conventional frontier definitions.  
%, see Fig. \ref{fig_oc_srfc}. 
%Instead, our proposed local frontier using the GP framework considers the correlations of sensing points and is selected by leveraging the uncertainty of the regression model. 
%%%%%%%%%%%%%%%%%%%%%
% Figure environment removed
%%%%%%%%%%%%%%%%%%%%%
%%%%%%%%%%%%%%%%%%%%% introduce our alg in simple way 
% We propose a new frontier concept called GP-Frontier based on the uncertainty associated with the occupancy surface.
Specifically, in this paper we present the GP-Frontier and its local navigation method by using only an onboard ranging sensor -- we use LiDAR as an example. 
The local observation is represented as an occupancy surface, where all the 3D occupied points observed by the LiDAR are projected onto a 2D circular surface modeled with VSGP. 
%The proposed approach utilizes the VSGP to model the local occupancy surface, then, 
Then the uncertainty of the VSGP model is used to detect all potential GP-Frontiers (sub-goals) around the robot. Based on the distance and direction of each GP-Frontier relative to both the robot and the final goal, a cost function selects the most promising GP-Frontier that will drive the robot to the final goal. At the last step, a motion command is generated as a function of both the distance and direction of the selected GP-Frontier relative to the robot, see Fig.~\ref{fig_subgoal} for an illustration.
% ##################################################################
% ################################# Related Work ####################
% ##################################################################
\vspace{-10pt}
\section{Related Work} \label{related_work}
\vspace{-3pt}
%%%%%%%%%%%%%%%%%%%%%%%%%%% global and local navigation 
% %=== Geometric-frontier methods ===
The frontier based navigation and exploration has been studied~\cite{yamauchi1997frontier, yamauchi1998frontier, holz2010evaluating}.
Most existing frontier exploration is coupled with a mapping method. For example, A 3D frontier detection method, named Stochastic Differential Equation-based Exploration algorithm~(SDEE) was proposed in \cite{shen2012stochastic} %. %The proposed method determine regions for further exploration based on a stochastic differential equation. 
where the authors demonstrate the efficiency on quadrotors in a 3D map representation. 
The frontier-based exploration has also been extended to UAV-UGV collaborations\cite{butzke20153,wang2018collaborative}, as well as multi-agent cooperative exploration \cite{burgard2000collaborative}  which integrates both the cost of reaching a target and also the utility of target points. 
Recently, exploration at high speeds leveraged the frontiers where a goal frontier is locally selected from the robot's field of view \cite{cieslewski2017rapid} so that the change in velocity to reach the local goal is minimized.
 %, which proved prodigious potential in mobile robotics applications, 
Another framework in the same line
is the gap-based method, where a gap is a free space between two obstacles that the robot can pass by. The first method, Nearness Diagram, was introduced by~\cite{1266644}, then many variants were developed based on this approach. As stated in~\cite{mujahed2018admissible}, ND-based methods showed undesired oscillatory motion. To solve the gap-based method's limitations, researchers in robotics developed schemes based on the geometry of the gap. 
The Follow-the-Gap Method (FGM)~\cite{sezer2012novel}, selects one of the detected gaps based on the gap area and calculates the robot heading based on the direction of the gap center relative to both the robot, and the final goal. Sub-goal seeking approach~\cite{ye2009sub} defines a cost for each sub-goal as a function of both the sub-goal and goal heading errors with respect to the robot heading, then it selects the sub-goal with the minimum cost (error). 
Our work is most related to %paper compares GP-Frontier to 
the Admissible Gap (AG) approach as both aim to address  reactive collision avoidance~\cite{mujahed2018admissible}. AG, an iterative algorithm, considers the exact shape and kinematic constraints of the robot, finds the possible admissible gaps, and then chooses the nearest gap as the goal, thus decreasing motion oscillations, path length, and safety risks. 


% Another criterion proposed on \cite{yan2020mapless} defines the cost as $d_{sum} = d_{r} + d_{g}$, where $d_{r}$ is the distance between robot and sub-goal, and $d_{g}$ is the distance between the sub-goal and the final goal. The admissible gap AG approach~\cite{mujahed2018admissible} selects the gap which is the closest to the final goal. 

% %=== Information-theoretic methods (still frontier)) ===
%\MA{I think this paragraph is more towards global exploration, probably we can remove it if we exceed 6 pages }
% Proximity work also includes information-driven %or informative 
% exploration.
% Approaches in such framework typically select a path that maximizes certain information-theoretic metrics, e.g.,  mutual information to actively maximize the information gathered by the robot. 
% Although informative exploration and planning \cite{charrow2015information,zhang2019fsmi,francis_functional_2020,francis_occupancy_2019,bai_information-theoretic_2016,jadidi_mutual_2015}
% have demonstrated interesting fast exploration behaviors for covering unknown space, they might not be useful for the repetitive traversals of static already-explored environments (e.g the trash collection task).  

% Although Mutual Information has been successfully demonstrated to navigate robots to explore new areas~\cite{julian2014mutual}, it is computationally costly. To overcome the bottleneck, many variants of MI were proposed, such as CSQMI~\cite{charrow2015information}, FSMI~\cite{zhang2019fsmi}, and FCMI. Those proposed MI-based criteria could be computed quickly and are applicable for real-time applications.
%\LL{this last paragraph needs to be revised by leveraging rss.}

%%%%%%%%%%%%%%%%%%%%%%%%%% GPs variants
Distinct from AG, our work is based on a variant of GP. The standard form of GP is known to suffer from few limitations~\cite{rasmussen2002infinite}, the most significant one is the cubic computational complexity of a vanilla implementation. 
However, different methods -collectively known as Sparse Gaussian Process (SGP)- tackle the computation complexity of GP~\cite{snelson2006sparse,titsias2009variational,hoang2015unifying,bauer2016understanding}. For instance, online SGP~\cite{ma2017informative} has been proposed to reduce the computational demand associated with modeling large data sets using GP. Also, a Mixture of GPs was adopted by \cite{kim2012building, luo2018adaptive} to capture non-stationarity environmental attributes.
%%%%%%%%%%%%%%%%%%%%%%%%%% GPs ccupancy mapping and GMM
Naturally, many GP-based occupancy mapping methods were intensively explored in the literature \cite{GPOM, jadidi2018gaussian, kim2012building}. 
%A recent approach leverages the Gaussian Mixture Models (GMM) instead of GP to represent the occupancy map as probabilistic distribution \cite{realtime-expo, goel2020rapid}.
%\cite{realtime-expo, var-res-map-GMM, goel2020rapid, design2019communication}. 
%The GMM is used to encode the entire occupancy map as a two large sets of Gaussian distributions representing the free and the occupied space. % in the environment.  
%We will further discuss and compare this line of work in later sections. 
In this work, we choose the VSGP due to its efficiency in computation and sensing representation which is important for real-time robot navigation. 

% Robot motion planning methods can be classified into various categories: global path planning methods and local path planning methods. Generally speaking, global path planning methods, also referred to as map-based methods, need prior knowledge about the environment to perform the path planning assuming the obstacles are static~\cite{nguyen2016path}. In contrast, in local path planning methods, also called sensor-based or obstacle avoidance techniques, the path is planned online (onboard). The latter approaches are considered more reliable and can be deployed in dynamic environments~\cite{nguyen2016path}. This paper only tackles local path planning approaches.

% One of the earliest local path planning algorithms is the Bug algorithms' family. Although these algorithms generally ensure that the robot will reach its goal, they may generate long paths~\cite{sezer2012novel}, and researchers usually tested them in static environments~\cite{mujahed2018admissible}. Another approach is the Artificial Potential Fields (APF) introduced in~\cite{khatib1986real}, where repulsive and attractive potentials represent the obstacles and the goal, respectively. The main drawbacks of this method are summarized by the local minima problem and the possibility of having unstable motions~\cite{mujahed2018admissible}. Other methods based on APF succeeded in avoiding the local minima; however, they require prior knowledge about the environment~\cite{sezer2012novel}. Other approaches tried to solve the robot's motion constraints from an admissible velocity perspective, known as velocity space approaches, rather than a steering direction approach~\cite{mujahed2018admissible}. From these approaches, The Dynamic Window Approach (DWA) is presented first in~\cite{seder2007dynamic}, where the system checks the space of velocities and then chooses the most acceptable one. A velocity is considered admissible if the robot can stop before hitting an obstacle~\cite{mohanan2018survey}. However, the local minima can also occur with the DWA algorithm~\cite{mujahed2018admissible}. 

% Another approach, which proved prodigious potential in mobile robotics applications, is the gap-based method. A gap is a free space between two obstacles where the robot can pass by~\cite{mujahed2018admissible}. The first method, Nearness Diagram, was introduced by~\cite{minguez2004nearness}, then many variants were developed based on this approach. As stated in~\cite{mujahed2018admissible}, ND-based methods showed oscillatory motion generation. To solve these gap-based methods limitations, researchers in robotics developed methods based on the geometry of the gap. This paper compares GP-Frontier to the Admissible Gap (AG) for reactive collision avoidance~\cite{mujahed2018admissible}. AG, an iterative algorithm, considers the exact shape and kinematic constraints of the robot, finds the possible gaps, and then chooses the admissible gap, thus decreasing motion oscillations, path length, and safety risks.






%%%%%%%%%%%%%%%% paper organization  %%%%%%%%%%%%%%%% 
%The paper is organized as follow: ...
%%%%%%%%%%%%%%%%%%%%%%%%%%%%%% 
% % Figure environment removed


 

% \section{Background} \label{background}

% GP is a non-parametric model defined by a mean and a co-variance function (known as kernel), $m(\textbf{x})$ and   $k(\textbf{x}, \textbf{x}^{\prime})$ respectively~\cite{rasmussen2003gaussian}; where $\textbf{x}$ is the {GP} input
% \begin{equation}
%     f(\mathbf{x}) \sim \mathcal{G P}\left(m(\mathbf{x}), k\left(\mathbf{x}, \mathbf{x}^{\prime}\right)\right).
%     \label{eq_full_gp}
% \end{equation}
% % Assume we have a dataset $\mathcal{D} = \{(\textbf{X}_i, y_i)\}_{i=1}^n$ of $n$ realizations
% Assume $\mathcal D = \left\{\left(\mathbf{x}_{i}, y_{i}\right)\right\}_{i=1}^{n}$ is a data set consists of $n$ inputs $\mathbf{x} \in \mathbb{R}^d $ and their corresponding scalar outputs  $y\in \mathbb{R}$ (i.e., observations), where $\mathcal D\in \mathbb{R}^{nx{(d+1)}}$. 
% When {GP} is used for regression,
% we consider that $y_{i}=f_{i}+\epsilon_{i}$ where $f_{i}=f\left(\mathbf{x}_{i}\right)$ is the unknown latent function and $\epsilon_{i}$ is a noise drawn from a Gaussian distribution $ \mathcal{N}\left(0, \sigma^{2}\right)$.
% The {GP} posterior is represented by the posterior mean $m_{\mathbf{y}}(\mathbf{x})$ and posterior covariance $ k_{\mathbf{y}}\left(\mathbf{x}, \mathbf{x}^{\prime}\right)$~\cite{titsias2009variational} as
% \begin{equation}
%     \begin{gathered}
%     m_{\mathbf{y}}(\mathbf{x})=K_{\mathbf{x} n}\left(\sigma^{2} I+K_{n n}\right)^{-1} \mathbf{y}, \\
%     k_{\mathbf{y}}\left(\mathbf{x}, \mathbf{x}^{\prime}\right)=k\left(\mathbf{x}, \mathbf{x}^{\prime}\right)-K_{\mathbf{x} n}\left(\sigma^{2} I+K_{n n}\right)^{-1} K_{n \mathbf{x}^{\prime}},
%     \end{gathered}
%     \label{eq_posterior_mean_kernel_full_gp}
% \end{equation}
% where $K_{nn}$ is $n \times n$ co-variance matrix of the inputs, $K_{xn}$ is n-dimensional row vector of kernel function values between $\mathbf{x}$ and the inputs, and $K_{nx} = K_{xn}^T$. The GP posterior predicts the value $y_*$ for any query point $\textbf{x}_*$
% \begin{equation}
%     p(y_* | \textbf{y}) = \mathcal{N}(y_* | m_{\textbf{y}}(\textbf{x}_*), k_\textbf{y}(\textbf{x}_*,\textbf{x}_*) + \sigma^2). 
%     \label{eq_predictive_eq_full_gp}
% \end{equation} 

% For accurate GP prediction, the kernel parameters $\Theta$ and the noise variance $\sigma^2$ should be correctly tuned. The most common method to tune the hyperparameters $(\Theta, \sigma^2)$ is maximizing the log marginal likelihood:
% \begin{equation}
%     \log p(\mathbf{y})=\log \left[\mathcal{N}\left(\mathbf{y} \mid \mathbf{0}, \sigma^{2} I+K_{n n}\right)\right].
%     \label{eq_mlml_full_gp}
% \end{equation} 

% GP suffers from high computation complexity $\mathcal{O}(n^3)$. 
% Many approximation methods tackle GP's computation complexity by considering only $M$ input points ({\em inducing points}) to describe the entire training data \cite{rasmussen2003gaussian, titsias2009variational}. The values of the latent function $f(\mathbf{x})$ at the {\em inducing points} $X_m$ are called the {inducing variables} $f_m$. This approximation introduces the Sparse variant of GP, known as the {\em Sparse Gaussian Process (SGP)}, with a lower computational complexity of $\mathcal{O}(NM^2)$.

% In this work, we use VSGP~\cite{titsias2009variational} that jointly estimates the kernel hyperparameters $\Theta$ and selects the inducing inputs $X_m$ by approximating the GP posterior  $p(f | y, \theta)$ with a variational posterior distribution.
% We opt to employ VSGP due to its practical efficiency in terms of computation and sensor representation.  
% %which is important for real-time robot navigation. 
% More details of VSGP can be found in  Titsias's seminal work~\cite{titsias2009variational}. 




% ##################################################################
% ################################# Methodology ####################
% ##################################################################
\section{Methodology} \label{methodology}



% ####################################################
\subsection{{Occupancy Surface Construction with Sensor Data}}
% #####################################################
We define the LiDAR local observation (pointcloud) in the spherical coordinate system, where any point is represented by the tuple $(\theta, \alpha, r)$ which describes the azimuth, elevation, and distance (radius) values of the 3D point with respect to the sensor origin, respectively. 
The occupied points observed by LiDAR are projected onto the  occupancy surface, which is a circular surface around the sensor origin with a predefined radius $r_{oc}$~\cite{realtime-expo}.
% , we called this circular surface as the {\em occupancy surface}. 
Any point on the occupancy surface is defined by two attributes $\theta$ and $\alpha$, where $\theta$'s values range from $-\pi$ to $\pi$ and $\alpha$'s values depend on the LiDAR's  vertical Field of View (FoV)  (for VLP16 $\alpha$ range is  $-15^{\circ}$ to $15^{\circ}$). Each point  $\mathbf{x}_i = (\theta_i,  \alpha_i)$ on the occupancy surface is assigned an occupancy value $oc_i = r_{oc} - r_i$, where $r_i$ is the point radius. All points that have a radius $r$ shorter than the occupancy surface radius $r_{oc}$ ($r < r_{oc}$) form the {\em occupied space} of the occupancy surface; the rest of the points on the surface with radius $r$ greater than or equal to the occupancy radius $(r >= r_{oc})$ are considered as the {\em free space} of the occupancy surface ($oc=0$), see Fig.~\ref{fig_vsgp_mdl_a}.
\vspace{-2pt}
%########################################################
\subsection{{Occupancy Surface Representation with VSGP}} \label{vsgp_oc_mdl}
% #####################################################
% Figure environment removed
%################################################################
The occupied points on the occupancy surface are transformed to a training data set $\mathcal D = \left\{\left(\mathbf{x}_{i}, y_{i}\right)\right\}_{i=1}^{n}$ with an $n$ input points, $\mathbf{x}_i=(\theta_i, \alpha_i)$, and their corresponding occupancy values, $y_i=oc_i$. %, where $\mathcal D\in \mathbb{R}^{nx{(d+1)}}$. 
$\mathcal D$ is then used to train a 2D VSGP occupancy model $f(\mathbf{x}_i)$ which describes the probability of occupancy over the occupancy surface as follow:
\vspace{-5pt}
\begin{equation}
    \begin{gathered}
    f(\mathbf{x}) \sim {VSGP}\left(m(\mathbf{x}), k_{RQ}\left(\mathbf{x}, \mathbf{x}^{\prime}\right)\right),  \\
    k_{\mathrm{RQ}}\left(\mathbf{x}, \mathbf{x}^{\prime}\right)=\sigma^{2}\left(1+\frac{\left(\mathbf{x}-\mathbf{x}^{\prime}\right)^{2}}{2 \alpha \ell^{2}}\right)^{-\alpha},
    \end{gathered}
    \label{eq_mean_kernel_vsgp}    
\end{equation}
where $k_{RQ}\left(\mathbf{x}, \mathbf{x}^{\prime}\right)$ is the Rational Quadratics (RQ) co-variance function (known as a kernel) with signal variance $\sigma^{2}$, length-scale $l$, and  relative weighting factor $\alpha$. % sets the relative weighting of large and small scale variations.
A Gaussian noise  $\epsilon$ is added to the model to reflect the measurement's noise. Therefore, the probability of occupancy $y_i$ of any point $\mathbf{x}_i$ is defined as $y_i=f(\mathbf{x}_i)+\epsilon$, where $\epsilon$ is sampled from a Gaussian distribution $\mathcal{N}\left(0, \sigma_{n}^{2}\right)$ with a variance $\sigma_{n}^{2}$.
The {GP} posterior is represented by the posterior mean $m_{\mathbf{y}}(\mathbf{x})$ and posterior covariance $ k_{\mathbf{y}}\left(\mathbf{x}, \mathbf{x}^{\prime}\right)$~\cite{titsias2009variational} as
\vspace{-5pt}
\begin{equation}
    \begin{gathered}
    m_{\mathbf{y}}(\mathbf{x})=K_{\mathbf{x} n}\left(\sigma^{2} I+K_{n n}\right)^{-1} \mathbf{y}, \\
    k_{\mathbf{y}}\left(\mathbf{x}, \mathbf{x}^{\prime}\right)=k\left(\mathbf{x}, \mathbf{x}^{\prime}\right)-K_{\mathbf{x} n}\left(\sigma^{2} I+K_{n n}\right)^{-1} K_{n \mathbf{x}^{\prime}},
    \end{gathered}
    \label{eq_posterior_mean_kernel_full_gp}
\end{equation}
where $\mathbf{y}= \left\{y_{i}\right\}_{i=1}^{n}$, 
$K_{nn}$ is $n \times n$ co-variance matrix of the inputs, $K_{xn}$ is n-dimensional row vector of kernel function values between $\mathbf{x}$ and the inputs, and $K_{nx} = K_{xn}^T$. 
% The probability of occupancy $y^*$ for any query point $\mathbf{x}^*$ on the occupancy surface is estimated by the GP prediction equation as follows  
% % GP posterior predicts the value $y_*$ for any query point $\textbf{x}_*$
% \begin{equation}
%     p(y_* | \textbf{y}) = \mathcal{N}(y_* | m_{\textbf{y}}(\textbf{x}_*), k_\textbf{y}(\textbf{x}_*,\textbf{x}_*) + \sigma^2). 
%     \label{eq_predictive_eq_full_gp}
% \end{equation} 

For accurate GP prediction, the kernel parameters $\Theta$ and the noise variance $\sigma_n^2$ should be correctly tuned. 
% The most common method to tune the hyperparameters $(\Theta, \sigma^2)$ is maximizing the log marginal likelihood:
% \begin{equation}
%     \log p(\mathbf{y})=\log \left[\mathcal{N}\left(\mathbf{y} \mid \mathbf{0}, \sigma^{2} I+K_{n n}\right)\right].
%     \label{eq_mlml_full_gp}
% \end{equation} 
In this work, we use a sparse variant of GP where only $m$ input points ({\em inducing points} $X_m$) are considered to describe the entire training data. %\cite{rasmussen2003gaussian, titsias2009variational}. 
The values of the latent function $f(\mathbf{x})$ at $X_m$ are called the {inducing variables} $f_m$.
Specifically, we leverage the VSGP framework proposed in~\cite{titsias2009variational} that jointly estimates the kernel hyperparameters $\Theta$ and selects the inducing inputs $X_m$ by approximating the GP posterior $p(f | y, \theta)$ with a variational posterior distribution. 
The computation complexity of SGP, $\mathcal{O}(nm^2)$ is dependent on the number of inducing points $m$, however, it is always less than the full GP's computation complexity, $\mathcal{O}(n^3)$.  
More details of the VSGP can be found in  Titsias's seminal work~\cite{titsias2009variational}. 

% ------------- Mean ----------
\ A zero-mean function is used to represent a zero-occupancy prior over the surface, which means that before acquiring any observation, the occupancy of all points is set to zero. 
% ------------- Kernel ----------
The RQ kernel is selected to form a flexible GP prior with a set of functions that vary across different length scales. The resolution of the LiDAR along the azimuth and elevation axes is used to initiate different length scales along both axes.
% ------------- inducing points ----------
The VSGP optimizes both the variational parameters (inducing inputs $X_m$) and the hyperparameters $\Theta$ through a variational Expectation-Maximization (EM) algorithm. In our model, the limited input domain of the VSGP ( $-\pi<\theta<\pi$ and $\alpha_{min}<\alpha <\alpha_{max}$) is exploited to initialize the inducing inputs $X_m$ at evenly distributed points on the {\em occupied part} of the occupancy surface. At each E-step, a different set of points is chosen to maximize the variational objective function %$F_{V}$ and minimize the  $\mathbb{K} \mathbb{L}[q(f, f_m)||p(f|y,\theta)]$, 
while the hyperparameters are updated during the M-step~\cite{titsias2009variational}.
%(More details on inducing point selection can be found in \cite{titsias2009variational}).
% \footnote{For more information about the inducing point selection process, check \cite{titsias2009variational}}. 

% of the gazebo scene shown in Fig.~\ref{fig_subgoal_a}

\ After estimating the hyperparameters and the inducing points, the VGSP occupancy model is used to predict the probability of occupancy $y^*$ for any point $\mathbf{x}^*$ on the occupancy surface by employing the GP prediction equation
\begin{equation}
    p(y_* | \textbf{y}) = \mathcal{N}(y_* | m_{\textbf{y}}(\textbf{x}_*), k_\textbf{y}(\textbf{x}_*,\textbf{x}_*) + \sigma^2_n). 
    \label{eq_predictive_eq_full_gp}
\end{equation} 
Fig.~\ref{fig_vsgp_mdl_a} shows the \textit{original occupancy surface} (inner circular surface) and the \textit{reconstructed occupancy} surface predicted by the VSGP occupancy model (the outer circular surface). Both the original and the predicted occupancy surfaces in Fig.~\ref{fig_vsgp_mdl_a} are color-coded, where warm colors reflect low occupancy; low occupancy means higher radius values as $r_i = r_{oc} - oc_i$. 
Therefore, for any direction defined by the azimuth $\theta$ and elevation $\alpha$ angels, our VSGP occupancy model estimates the distance $r_i$ to the obstacle along that direction. 
% The accuracy of VSGP model has been investigated in our previous work~\cite{ali2023light}, where the findings showed that an SGP model with 400 inducing points results in a reconstructed point cloud with an average error of around \SI{12}{\centi\metre}.
Our previous study~\cite{ali2023light} investigated the accuracy of the VSGP model, where the results demonstrated that a VSGP with 400 inducing points results in an average error of around \SI{12}{\centi\metre} in the reconstructed point cloud.
% in GP-MPPI 
% The precision of the SGP occupancy model is intensively evaluated in our previous work [22], where the results showed that an SGP occupancy model comprising of 400 inducing points generates a reconstructed point cloud with an average error of approximately 12 cm





% #####################################################
\subsection{{GP-Frontiers for Local Navigation} }
% #####################################################
A key benefit of using the GP and its variations over other modeling methods is their ability to quantify the uncertainty, or variance, associated with the predicted value at any given query point. 
For each point on the reconstructed occupancy surface, the occupancy $\mu_{oc}$ predicted by the VSGP model is associated with a variance value $\sigma_{oc}$.
While the occupancy surface can be considered as a \textit{local} 3D map -projected on a 2D circular surface- of the robot locality (see Fig. \ref{fig_vsgp_mdl_a}), the variance $\sigma_{oc}$ associated with the reconstructed occupancy surface can be considered as a local uncertainty-map of the robot locality, see Fig.~\ref{fig_vsgp_mdl_b}.
The \textit{variance surface} (grey-coded surface in Fig.~\ref{fig_vsgp_mdl_b}) generated by the VSGP defines the certain and uncertain regions in the robot locality. Therefore, any region on the variance surface can be classified as known space (low uncertainty regions) shown as black regions on the variance surface, or unknown space (high uncertain regions) which we call a GP-Frontier candidate shown as white regions on the variance surface, see Fig.~\ref{fig_vsgp_mdl_b}.
% %=== frontier  ===
% Frontiers are defined as the boundary regions between known and unknown space, and they usually appear on the maximum range of sensing (or the "edge of the sensing sweep" that does not return any obstacle detection). Since the frontiers are on the boundary between known (free space) and unknown, the frontiers are used to guide the robot to further scan the unknown space to find a path to the goal.

The well-known frontier concept introduced in \cite{yamauchi1997frontier} is associated with exploration and occupancy map building, however, in this paper, we do not consider any kind of global map or map building techniques.
Instead, our proposed GP-Frontier exploits the correlations of the observed points and is selected by leveraging the uncertainty of the VSGP occupancy model. GP-Frontiers on the occupancy surface are typically located in either the most open space (non-occupied space) or the region in the occupied space that has a large discontinuity on the occupancy value. This large discontinuity is explained as the gaps between different obstacles in the occupied space, which are also considered as GP-Frontier candidates.
In this context, we only consider one full sensor scan/observation 
to represent the occupancy of the robot locality as a VSGP occupancy model and define local navigation points (i.e. GP-Frontiers) based on the VSGP uncertainty. Any region on the variance surface with a variance that is higher than a threshold $V_{th}$ is considered a GP-Frontier. %\MA{and size that can fit the robot}
Actually, The variance associated with the occupancy surface varies for each observation and is influenced by the quantity and the arrangement of the observed points on the surface.
Consequently, the variance threshold $V_{th}$ is determined as a variable that varies with the variance distribution across the surface, $V_{th}= K_m*v_m$ where $v_m$ is the variance distribution mean and $K_m$ is a tuning parameter.
% see Algorithm~\ref{pseudo_algorithm}.

Formally, a GP-Frontier $f_i=(\theta_{f_i}, \alpha_{f_i}, r_{f_i})$ is defined by its centroid point on the variance surface $\mathbf{x}_{f_i}= (\theta_{f_i}, \alpha_{f_i})$, and the distance $r_{f_i}$ between the GP-Frontier centroid and the occupancy surface origin. $r_{f_i}$ is estimated using the VSGP occupancy model, 
where $oc_{f_i}=vgsp((\theta_{f_i}, \alpha_{f_i}))$ and $r_{f_i}=r_{oc} - oc_{f_i}$. In practice, for 2D navigation, GP-Frontier can be defined as $f_i=(\theta_{f_i}, 0, r_{f_i})$ because $\alpha_{f_i}$ is a constant ($\alpha_{f_i} =\alpha_{xy} = 0$), where $\alpha_{xy}$ is the elevation angel of the 2D XY-plane. $\theta_{f_i}$ is used to define the GP-Frontier direction with respect to the robot heading, considering the transformation between the robot frame $\mathcal{R}$ and the LiDAR fame is known. 
The cartesian coordinates of GP-Frontier $f_i$ in a global world frame $\mathcal{W}$ are calculated as $(x^{w}_{f_i}, y^{w}_{f_i}) = \prescript{W}{}{\textbf{T}}_{R} (x^{R}_{f_i}, y^{R}_{f_i})$ where $\prescript{W}{}{\textbf{T}}_{R}$ is the transformation between $\mathcal{R}$ and $\mathcal{W}$ (robot localization) and  $(x^{R}_{f_i}, y^{R}_{f_i})$ are the cartesian coordinates of the GP-Frontier $f_i$ in $\mathcal{R}$ which correspond to the GP-Frontier spherical coordinates $(\theta_{f_i}, 0, r_{f_i})$.
% $\prescript{W}{}{\textbf{T}}_{R}$
%For 3D navigation, both $\alpha_{f_i}$ and $\theta_{f_i}$ are used for describing the GP-Frontier direction with respect to robot heading. 
% #####################################################
\subsection{{GP-Frontier for Goal-Oriented Navigation} }
% #####################################################
To navigate towards a given final goal $\mathbf{g} = (x_g, y_g)$ in $\mathcal{W}$, local navigation approaches use different criteria to select one sub-goal (i.e., an ideal GP-Frontier $f^*$, or a gap) from the local GP-Frontiers.  % sub-goals 
%(gaps). % detected from the local observation. 
% From the sub-goals detected by the VSGP model uncertainty, only one goal is selected to drive the robot to the final goal. 
% The Follow-the-Gap Method (FGM)~\cite{sezer2012novel}, selects one of the detected gaps based on the gap area and calculates the robot heading based on the direction of the gap center relative to both the robot, and the final goal. Sub-goal seeking approach~\cite{ye2009sub} defines a cost for each sub-goal as a function of both the sub-goal and goal heading errors with respect to the robot heading, then it selects the sub-goal with the minim cost (error). Another criterion proposed on \cite{yan2020mapless} defines the cost as $d_{sum} = d_{r} + d_{g}$, where $d_{r}$ is the distance between robot and sub-goal, and $d_{g}$ is the distance between the sub-goal and the final goal. The admissible gap AG approach~\cite{mujahed2018admissible} selects the gap which is the closest to the final goal. 
Our approach combines both distance and direction criteria. A cost function $C$ is used to select only one GP-Frontier $f^*$ to act as the next navigation sub-goal. The proposed cost function $C$ adopts the distance criteria proposed in~\cite{yan2020mapless} and the direction criteria proposed in~\cite{ye2009sub}: 
\vspace{-10pt}
\begin{equation}
 \label{eq_cost_fun} 
    \begin{aligned}
C\left(f_{i}\right) &=  k_{dst}  d_{sum} + k_{dir} \theta_{f_i}^2 ,  \\
d_{sum} &= r_{f_i} + \sqrt{(x_g-x^w_{f_i})^2+(y_g-y^w_{f_i})^2},\\
f^{*} &=\operatorname{arg} \min _{f_{i} \in \mathcal{F}}\left(C\left(f_{i}\right)\right), 
\end{aligned}     
\end{equation}
% \vspace{-5pt}
where $k_{dst}$, $k_{dir}$ are weighting factors.
% $\theta_{f_i}$ is the direction of the GP frontier $f_i$ with respect to the robot heading
% and $d_{fg}=\sqrt{(x_g-x_{f_i})^2+(y_g-y_{f_i})^2}$ is the euclidian distance between $\mathbf{g}$ and the cartesian coordinates of the GP frontier $f_i$, $(x^{w}_{f_i}, y^{w}_{f_i})$ in $\mathcal{W}$.
% and $k_{dst}$, $k_{dir}$ are weighting factors.
Integrating both distance and direction criteria decreases the chance of getting stuck in local minima. More discussion is provided in Sec.~\ref{sec_sim_results}.

 %%%%%%%%%%%%%%%%%%%%%%%%% add if there is a time 
 %%%%%%%%%%%%%%%%%%%%% alpha as a penality functionn 
 %$\alpha$ as a penalty term for non reachable frontiers (not in same plane as robot). 
%  The term of the utility function associated with the local frontier direction $\theta$ is a squared to give higher penality for the.  Provided that the robot has no sense of the global environment (no global map), the part of the utility function associated with the robot direction $\theta$ is given a square function to  
 %%%%%%%%%%%%%%%%%%%%% penality for high theta to avoid local minimium 
 
%\ 
% Once a GP-Frontier is selected, its corresponding spherical 3D location is converted to a 3D Cartesian local goal $(x_g, y_g, z_g)$ using the spherical-to-cartesian relation of the VLP16: %\ref{vlp_manual}:  
% \begin{equation}
%     \begin{gathered}
%     x_g = r_g \cos{\theta_{f^*}} \sin{\alpha_{f^*}}, \ 
%     y_g = r_g \sin{\theta_{f^*}} \sin{\alpha_{f^*}}, \ 
%     z_g = r_g \cos{\alpha_{f^*}}, \ 
%     \end{gathered}
%     \label{eq_sph_to_cart}    
% \end{equation}
% where $r_g$ is a constant specifying the Euclidean distance between the local goal and the robot; $(r_g \leq r_{oc})$. $\theta_{f^*}$ and  $\alpha_{f^*}$ are the azimuth and the elevation angles of the GP-Frontier with the highest utility ($\alpha_{f^*}$ is a constant for 2D navigation; in case of VLP16, $\alpha_{f^*}= \pi/2)$. 

% #####################################################
%\subsection{GP-Frontier based Motion Control}
% #####################################################
Finally, a motion command $(v, \omega)$ is generated to drive the robot towards the center of the selected GP-Frontier $f^*$. 
The linear velocity $v$ varies directly with the distance to the sub-goal ($r_{f^*}$) and inversely with the direction to the sub-goal $\theta_{f^*}$;  $v = k_a r_{f^*} - k_b \|\theta_{f^*}\|$. The angular velocity $\omega$ is proportional to the direction of the sub-goal; $\omega = k_c \theta_{f^*} $. $k_a, k_b$ and $k_c$ are tunable coefficients.
% they are used to limit the linear and the angular velocity according to the kinematic of the robot. 
If the final goal is inside the local FoV of the robot, then the motion command drives the robot directly towards the goal, otherwise, the motion command drives the robot towards the selected sub-goal $f^*$.


% The occupancy surface acts as an obstacle avoidance checking algorithm while the robot navigate towards the local goal. To sum up, in our approach the surface variance is used to detect the local frontiers, where the local navigation goal is calculated, then the occupancy surface checks for any obstacles collision, All without the need for any kind of occupancy map.


%%%%%%%%%%%%%%%%%%%%%%%%%%%%%%%%%%%%%%%%%%%%%%%%%%%%%%%%%%%% 
\begin{algorithm}
{\small
	\caption{
%	\textbf{GP-Frontier (GPF)}: 
 GP-Frontier Local Mapless Navigation \\
        \textbf{INPUT}  : LiDAR Observation (Pointcloud (PCL) ) \\
        \textbf{OUTPUT}: Motion Command
	}  
	\begin{algorithmic}[1]
 	    % \State Instantiate: VSGP(kernel$\gets RQ$, likelihood$\gets Gaussian$)
	    \State $X_m$: Inducing Points
	    \State $\mathbf{X^*} \gets (\theta, \alpha)$: 2D Variance Grid 
            \State $\mathbf{g} = (x_g, y_g)$: Navigation goal 
            \While { New Sensor Observation (PCL) }
	        \State $(\theta_i, \alpha_i, r_i) \gets$ \textit{Catersian2Spherical}(PCL)
	        \State $oc_i=r_{oc}-r_i$
	        \State $\mathbf{x}_i= (\theta_i, \alpha_i) $,  $y_i=oc_i$
                \State $\mathcal D = \left\{\left(\mathbf{x}_{i}, y_{i}\right)\right\}_{i=1}^{n} \gets \mathbf{x}_i, y_i$
                \State  $f(\mathbf{x}) \sim {VSGP}\left(m(\mathbf{x}), k_{RQ}\left(\mathbf{x}, \mathbf{x}^{\prime}\right)\right)$        
	        % \State Convert PCL to Occupancy Surface ($\theta, \alpha, oc=(r_{oc}-r)$)
	        % \State VSGP input$\gets\mathbf{x}=(\theta, \alpha)$, VSGP output $\gets y=oc$
		    % \State Train VSGP Hyperparameters $\Theta$ \& Optimize{X\_m}
		    \State \textit{Optimize}( $\Theta, X_m$) $\gets \mathcal{D}$
                \State $\mu_{oc}, \sigma_{oc} \gets \mathcal{N}(\textbf{y}^* | m_{\textbf{y}}(\textbf{X}^*), k_\textbf{y}(\textbf{X}^*,\textbf{X}^*) + \sigma^2_n)$
		    % \State $(Mean, Variance)\gets$ \textit{VSGP\_predict}($X^*$)
		    \State $v_m\gets$\textit{Mean}($\sigma_{oc}$) 
		    % \State $v_{std}\gets$standard deviation($Variance)$ 
		    \State $V_{th} \gets K_m v_m$ 
		    \State GP-Frontiers $\gets$  $\sigma_{oc}>V_{th}$
			\For { each GPF $fi$ in GPFs}
            % 	\If{$v_i<v_{th}$} 
                \State $\mathbf{x}_{f_i} = (\theta_{f_i}, \alpha_{f_i}) \gets$ \textit{CentroidOfGP}-Frontier    
                % \State $oc_{f_i} \gets  VSGP\_predict(\theta_{f_i}, \alpha_{f_i})$
                \State ${oc}_{f_i}, \sigma_{f_i} \gets \mathcal{N}(\textbf{y}^* | m_{\textbf{y}}(\textbf{x}_{f_i}), k_\textbf{y}(\textbf{x}_{f_i},\textbf{x}_{f_i}) + \sigma^2_n)$
                
                \State $r_{f_i} \gets r_{oc} - oc_{f_i}$   
                \State $(x^R_{f_i}, y^R_{f_i}, 0)$ $\gets$ \textit{Spherical2Cartesian}($\theta_{f_i}, 0, r_{f_i}$)
                \State $(x^w_{f_i}, y^w_{f_i}, 0)$ $\gets$ $\prescript{W}{}{\textbf{T}}_{R} (x^R_{f_i}, y^R_{f_i}, 0)$
                \State $d_{sum} = r_{f_i} + \sqrt{(x_g-x^w_{f_i})^2+(y_g-y^w_{f_i})^2}$
                \State $C\left(f_{i}\right) =  k_{dst}  d_{sum} + k_{dir} \theta_{f_i}^2$
			\EndFor
			\State $f^{*} \gets \operatorname{arg} \min _{f_{i} \in \mathcal{F}}\left(C\left(f_{i}\right)\right),$
			% \State sub-goal $\gets$ $(x^w_{f^*}, y^w_{f^*})$
			\State  $v = k_a r_{f^*} - k_b \|\theta_{f^*}\|$
			\State $\omega  \gets k_c \theta_{f^*} $
			% \State \textit{MotionCommand} $ \gets (v, \omega) $
		\EndWhile
		\end{algorithmic} 
	\label{pseudo_algorithm}
}
\end{algorithm}
% \vspace{-10pt}
%%%%%%%%%%%%%%%%%%%%%%%%%%%%%%%%%%%%%%%%%%%%%%%%%%%%%%%%%%%% 
% \MA{ discuss compactness of vsgp see Fig. \ref{vsgp_oc} }
% \LL{I think we can separate a section and discuss it later.}
% Figure environment removed
% % % ####################################################
% % Figure environment removed
% % % ####################################################
% % Figure environment removed
% % %%%%%%%%%%%%%%%%%%%%%%%%%%%%%%%%%%%%%%%%%%%%%%%%%%%%%%%%%%%%%%%%
% % \lipsum[1-2]
% % Figure environment removed
% % \lipsum[3-10]
% % ####################################################
% % ####################################################
% Figure environment removed
%%%%%%%%%%%%%%%%%%%%%%%%%%%%%%%%%%%%%%%%%%%%%%%%%%%%%%%%%%%%%%%%
%%%%%%%%%%%%%%%%%%%%%%%%%exp without lineear and anguler velocities %%%%%%%%%%%%%%%%%%%%%%%%%
% \lipsum[1-2]
% % Figure environment removed
% \lipsum[3-10]
% Figure environment removed
% ####################################################
%%%%%%%%%%%%%%%%%%%%%%%%%%%%%%%%%% bNARY1 
% ################################################################################
% ########################## Experiments and Results ############################
% ################################################################################
\vspace{10pt}
\section{Experimental Design and Results}
% ####################################################
\subsection{Simulation Setup} \label{simulation_experiment}
% ####################################################
The proposed GP-Frontier is built on top of GPFlow \cite{matthews2017gpflow} and executed in real-time. 
Real-time simulation in Gazebo and real-time demonstration were considered 
to evaluate the performance of our approach and to compare it to a baseline -- the AG method~\cite{mujahed2018admissible} that was published recently to address the same problem. 
During all of the simulation experiments, the maximum linear and angular velocities were set to \num{1.0} \si{\metre/\second} and \num{1.5} \si{\radian/\second} respectively.
LiDAR configurations were set to a $5 m$ maximum range (to have limited FoV compared to the environment size), a $5 Hz$ frequency, and a resolution of $(0.35^{\circ}, 2^\circ)$ along the azimuth and the elevation axis, respectively. 
% The occupancy surface was designed with a radius $r_{oc}$ of \num{5} meters, a full azimuth range of $-180^o$ \textit{to} $180^o$, and elevation height of $0^o$ \textit{to} $15^o$. During the experiments, tunning parameters and constants
The surface for occupancy was created with a radius $r_{oc}$ of \SI{5}{\metre} and a complete azimuth range from $-180^o$ to $180^o$, with an elevation height spanning from $0^o$ to $15^o$. Throughout the experiments, specific tuning parameters and constants were 
assigned as follows:  inducing points $X_m = 400$, variance threshold constant $K_m=0.4$, where the distance and direction  weighting factors $K_{dst}$ and $K_{dir}$ were set to \num{5} and \num{4}, respectively.
% For the occupancy prediction, the surface is represented by a 2D grid $\mathbf{X}^*$ which has same resolution as the LiDAR resolution along the azimuth and elevation angels. 
To predict occupancy, a 2D grid $\mathbf{X}^*$ is used to represent the surface, with a resolution identical to that of the LiDAR resolution along both the azimuth and elevation angles. 

Two environments \textbf{A} and \textbf{B}, each of them has an area of 20x20 meters, are used to evaluate the local navigation performance. Specifically, \textbf{A} is a cluttered environment with random obstacles, while \textbf{B} is a maze-like environment with 3 u-shaped rooms U1, U2, and U3, see Fig. \ref{fig_env}. 
Different experiments were designed to thoroughly evaluate the performance of the GP-Frontier and the AG methods. 
Specifically, 
i) MD: where the robot has to go along the main diagonal (MD) of environment \textbf{A}, see Fig.~\ref{fig_exp_md}.
ii) X: where the robot has to go parallel to the X-axis of environment \textbf{B}, see Fig.~\ref{fig_exp_x}.
iii) SU: where the starting pose is located inside U1, see Fig.~\ref{fig_exp_env2_b}.
iv) CU: where the robot has to cross U2 to reach the goal, see Fig.~\ref{fig_exp_env2_c}.
v) GU: where the goal is located inside U1, see Fig.~\ref{fig_exp_env2_d}. 
The starting pose and the final goal for each experiment are shown in Table~\ref{tab_performance}.
% We run each experiment 10 times. To determine the average performance evaluation for the GP-Frontier and AG methods, 
Five metrics are considered to evaluate the local navigation behavior~\cite{mujahed2018admissible}: \\
i) $T_{\text {tot }}$: total time to reach to the final goal $\textbf{g}$.\\
ii) $D_{\text {acc}}$: traveled distance to reach the final goal .\\
iii) $\mathrm{C}_{\text {chg }}$: trajectory curvature change measure that reflects the oscillations along the path.
\vspace{-3pt}
$$
\mathrm{C}_{\text {chg }}=\frac{1}{\mathrm{~T}_{\text {tot }}} \int_0^{\mathrm{T}_{\text {tot }}}\left|k^{\prime}(t)\right| d t, \quad k(t)=\left|\frac{\omega(t)}{v(t)}\right|.
$$
\\\vspace{-4pt}
iv) $\mathrm{J}_{\mathrm{acc}}$: accumulated jerk measure for  trajectory smoothness. \vspace{-4pt}
$$
\mathrm{J}_{\mathrm{acc}}=\frac{1}{\mathrm{~T}_{\text {tot }}} \int_0^{\mathrm{T}_{\text {tot }}}[\ddot{v}(t)]^2 d t. 
$$
\\\vspace{-4pt}
v) $\mathrm{R}_{\mathrm{obs}}$: risk measure that reflects  proximity to obstacles.
$$
\mathrm{R}_{\mathrm{obs}}=\int_0^{\mathrm{T}_{\mathrm{tot}}} \frac{1}{r_{\min }(t)} d t, 
$$
where $r_{\min }$ is the distance to the closest obstacle. For all these five metrics described above, a lower value indicates a better performance. 
% Additionally, we investigate the runtime of the VSGP model, including both the time required to train the VSGP model on the observed points sensed by LiDAR and the time it takes to predict the occupancy over the surface. We conducted 10 trials for each experiment to calculate the average performance evaluation.



% \lipsum[1-3]
\begin{table*}[t]
\vspace{5pt}
  \centering
  % \begin{tabular}{c|c|c|c|c|c|c|c} \vspace{2pt}
  \begin{tabular}{cccccccccc} 
  \hline \\[-2ex]
 \textbf{Env.} & \textbf{Exp.} & \textbf{Starting pose} & \textbf{Goal} &  \textbf{Method} & $\mathbf{T_{tot}}$[\si{\second}] &  $\mathbf{D_{acc}}$ [\si{\metre}]& $\mathrm{\mathbf{J}}_{\mathrm{\textbf{acc}}}$[\si{\metre\per\cubic\second}]  & $\mathrm{\textbf{C}}_{\text {\textbf{chg} }}$[\si{\radian/\metre}]& $\mathrm{\textbf{R}}_{\mathrm{\textbf{obs}}}$[\si{\per\metre}]\\ \hline \\[-2ex]
  
\multirow{2}{*}{\textbf{A}} &  \multirow{2}{*}{MD}  &  \multirow{2}{*}{$(-8.5, -8.5, 45^o)$} & \multirow{2}{*}{$(8.5, 8.5, 45^o)$} &  GP-Frontier  &\textbf{ 28.68$\pm$1.2}  &  \textbf{28.2$\pm$0.74}  & \textbf{30.2$\pm$10.2} &  \textbf{18.14$\pm$18.9}  &  \textbf{19.76$\pm$1.1}\\ \\[-2ex]
& &   &  &  AG   & 34.11$\pm$4.3  &  28.4$\pm$0.66  & 74.1$\pm$9.7  &  76.19$\pm$81.8  &  26.27$\pm$3.0\\ \hline \\[-2ex]


\multirow{2}{*}{\textbf{B}} &  \multirow{2}{*}{X}   & \multirow{2}{*}{$(-8, 1, -45^o)$} & \multirow{2}{*}{$(8, 1, 0^o)$} &  GP-Frontier  & \textbf{21.12$\pm$0.6}  &  20.8$\pm$0.52  & \textbf{51.70$\pm$8.9} &  \textbf{10.03$\pm$12.36 } &  \textbf{13.8$\pm$0.6}\\ \\[-2ex]
 &     &   &  &  AG   & 22.85$\pm$0.8  &  \textbf{20.5$\pm$0.26}  & 66.36$\pm$15.2 &  25.68$\pm$26.53  &  14.8$\pm$0.9\\ \hline \\[-2ex]


\multirow{2}{*}{\textbf{B}} & \multirow{2}{*}{SU}   & \multirow{2}{*}{$(4, 4, 90^o)$} & \multirow{2}{*}{$(-2, -8, 0^o)$ }&  GP-Frontier  & $\mathbf{40.1\pm0.6}$  & \textbf{ 39.50$\pm$0.6}  & \textbf{46.20$\pm$6.1} &  \textbf{19.5$\pm$13.6}  & \textbf{ 28.2$\pm$1.6}\\ \\[-2ex]
  &    &   &  &  AG   & Fail  &  Fail  & Fail &  Fail  &  Fail\\ \hline \\[-2ex]


\multirow{2}{*}{\textbf{B}} & \multirow{2}{*}{CU}   &\multirow{2}{*}{ $(-8, 8, -45^o)$ }& \multirow{2}{*}{$(8, -8, 0^o)$} &  GP-Frontier  & \textbf{43.1$\pm$0.7}  &  \textbf{40.2$\pm$0.45}  & \textbf{48.1$\pm$6.9} &  \textbf{21.17$\pm$15.2}  & \textbf{ 27.9$\pm$1.1}\\ \\[-2ex]
 &    &  &   &  AG   & Fail  &  Fail  & Fail &  Fail  &  Fail\\ \hline \\[-2ex]

  
\multirow{2}{*}{\textbf{B}} &  \multirow{2}{*}{GU}   & \multirow{2}{*}{$(-5, -8, 45^o)$} & \multirow{2}{*}{$(4, 4, -90^o)$} &  GP-Frontier  & \textbf{30.9$\pm$0.5 } & \textbf{ 30.6$\pm$0.3}  & \textbf{41.3$\pm$5.6} &  \textbf{19.08$\pm$8.4}  &  \textbf{20.9$\pm$0.8}\\ \\[-2ex]
 &     &   &   &  AG   & Fail(110)     &  Fail  & Fail &  Fail  &  Fail\\ \hline  \\[-2ex]

\multirow{2}{*}{\textbf{Cafeteria}} & \multirow{2}{*}{-}   & \multirow{2}{*}{$(0,0,0^o)$} & \multirow{2}{*}{$(16, -4, 1^o)$} &  GP-Frontier   &$\mathbf{55.9\pm13.8}$    & $\mathbf{20.5\pm2.1}$  & $\mathbf{4.7\pm2.9}$ & $\mathbf{0.3\pm0.05}$  & $\mathbf{36.4\pm10.3}$ \\   \\[-2ex]
&     & & &  AG   & $83.9\pm17.9$     &  $22.6\pm4.7$ & $38.4\pm11.8$ &  $7.1\pm8.6$  & $98.9\pm24.3$ \\ \hline \\[-2ex]
\vspace{-2pt}
\end{tabular} \vspace{-8pt}
  \caption{Local Navigation Performance, improved metrics over the baseline is highlighted in bold for each experiment.\vspace{-2pt}}
  \label{tab_performance}
\end{table*}



% ####################################################
\subsection{Simulation Results} \label{sec_sim_results}
% ####################################################
% In general, both the GP-Frontier and the AG methods succeeded in finding a path without any collision for both experiments MD and X. However, the AG method failed to reach the goal for the other 3 experiments: SU, CU, GU (considering 10 trials). 
Overall, the GP-Frontier and AG techniques were able to find a collision-free path for both the MD and X experiments, see Fig.~\ref{fig_env}. However, in the case of the SU, CU, and GU experiments, the AG approach failed to reach the goal (considering 10 trials).
Table \ref{tab_performance} shows that the average GP-Frontier and AG traveled distances for experiments MD and X are almost equal. Nevertheless, the GP-Frontier outperforms the AG in terms of total time, accumulated jerk, curvature change, and risk measure. 
The GP-Frontier generates smoother paths and decreases the risk value since the robot follows the center of the open space. However, these advantages (i.g. smoothness and distance from obstacles) may lead to slightly longer paths. For example, in experiment X, the AG's average traveled distance (20.5 m) is lower than that of the GP-Frontier (20.8 m), see Fig.~\ref{fig_exp_x} and Table~\ref{tab_performance}.
On the other hand, most of the paths generated by the AG methods include a point where the robot oscillates (either left and right or forward and backward) until it eventually selects a sub-goal, see the highlighted blue circle in Fig. \ref{fig_exp_md} and its corresponding blue square in Fig. \ref{fig_exp_md_w}. Fig.~\ref{fig_exp_md_time} shows the running time performance of VSGP model during experiment MD. The training time (green dots) required to train the VSGP model on the dataset $\mathcal{D}$ is below \num{20} milli-seconds for almost all observations, while the prediction time (blue dots) needed to estimate the probability of occupancy over the surface and its associated variance is around \num{60} milli-seconds for all observations. 


The AG method encountered a local minimum in the remaining experiments: SU, CU, and GU, because it only takes into account the distance metric when selecting a sub-goal; it selects the gap that is nearest to the goal. 
In Exp. SU, at the initial position (indicated by a red circle in Fig. \ref{fig_exp_env2_b}), the robot can observe two gaps (located in the upper left and upper right corners of U1). The AG method chooses the upper left gap since it is the closest to the goal. However, as the robot departs from U1 towards the upper left gap, the lower edge of U1 goes out of the field of view, leading to the emergence of a new gap (lower gap) inside U1 that is now closest to the goal. The AG method subsequently selects the lower gap as the current sub-goal. Once the lower edge of U1 reappears, the AG method switches back to selecting the upper left gap as the sub-goal. This pattern of actions repeats continuously, as seen in Figs. \ref{fig_exp_env2_b} and \ref{fig_exp_env2_e}.
The situation is different for the GP-Frontier method because it selects a sub-goal based on both the distance and direction of the GP-Frontiers (gaps). So, Even when the new lower gap inside U1 appears, it will have a high cost in terms of direction with respect to the robot's heading. As a result, GP-Frontier will keep the upper left gap as the current sub-goal.
In  Exp. CU, the AG method encountered a similar problem as in  Exp. SU, getting stuck in a local minimum where the robot oscillates left and right inside U2, see Fig.~\ref{fig_exp_env2_c}.

In Exp. GU, the situation is different from that in Exp. SU and CU, as the AG gets stuck outside the U-shaped obstacles, however, the AG is stuck because of the same reason. When the robot reaches the lower right corner of U1, it encounters three gaps (with directions: up, down, and left (inside U3)). Since the upper and left gaps have approximately the same distance to the goal (inside U1), the AG method selects the gap nearest to the goal (assume the upper gap). But, as the robot moves towards the selected gap (following a curve around the lower right corner of U1), the other gap (left gap) will become the nearest to the goal. Thus, the AG method switched to it once again. This same sequence is repeated continuously, as depicted in Fig.~\ref{fig_exp_env2_d}. This behavior can be seen in the supplementary video. %\footnote{\label{ft_video}https://youtu.be/4tUpmzIcFDY}
% \footnotemark%\footnote{https://youtu.be/4tUpmzIcFDY}.

We believe that the AG method encountered local minima in environment \textbf{B} due to the challenging navigation through U-shaped rooms and wide walls, resulting in two gaps that are equidistant from the final goal. In contrast, environment \textbf{A} is less challenging because the obstacles' size is smaller. Our approach, unlike the AG method, takes into account both the direction and distance of the GP-Frontiers (gaps) in the cost function $C$, which minimizes the probability of getting trapped in a local minimum.


%%%%%%%%%%%%%%%%%%%%%%%%%%%%%%%%%%% bNARY3 
% Figure environment removed


% ####################################################
\vspace{-5pt}
\subsection{Hardware Demonstration} \label{sec_sim_results}
% ####################################################
A Jackal mobile robot, equipped with a VLP-16 LiDAR 
%and an Intel® Core™ i7 {\em{ NUC11}} PC equipped with 64 GB RAM and 6 GB  Geforce $RTX2060$ GPU, 
is used to validate our GP-Frontier approach. Our algorithm runs in real-time with a frequency of 10 Hz. 
% to be faster than the VLP-16 frequency. 
Our GP-Frontier method is validated in both indoor and outdoor environments. The university cafeteria is chosen to represent a cluttered indoor environment similar to environment \textbf{A}, as shown in Fig.~\ref{fig_indoor}. In addition, the GP-Frontier is demonstrated in a harsh outdoor environment by testing it on a real forest trail, see Fig.~\ref{fig_outdoor}. 
% Compared to other local gap-based navigation approaches, which directly process the ranging observations, our approach first converts the ranging data to a compact representation of the environment and uses its variance surface which is more robust against noisy measurements, see 
%the noisy pointcloud generated in the forest
% Fig.~\ref{fig_outdoor_b}. 
% In contrast to other local gap-based navigation methods that directly operate on ranging measurements, our approach utilizes a compressed representation of the environment, VSGP occupancy model.
Compared to other local gap-based navigation approaches, GP-Frontiers relies on the variance surface of the VSGP representation, which is more resilient to noisy measurements. This property is particularly valuable in noisy environments, as demonstrated in the noisy pointcloud generated in the forest, see Fig.~\ref{fig_outdoor_b}. 
By smoothing out the raw pointcloud observation, the variance surface enables better detection of the GP-Frontier (gaps) around the robot. The surface also exhibits a smoothing property for indoor environments, as evidenced by Fig.~\ref{fig_indoor}, where it smooths out the occupied and free spaces.
Specifically, the variance surface represents each table and its chairs as a single "big" obstacle, black regions on the surface, instead of being a sparse set of points as seen in the raw pointcloud data.


The performance table (Table~\ref{tab_performance}) demonstrates that our proposed approach outperforms the AG method in all performance metrics. The accumulated jerk $\mathrm{\mathbf{J}}_{\mathrm{\textbf{acc}}}$ and the trajectory curvature $\mathrm{\textbf{C}}_{\text {\textbf{chg} }}$ values in the real-world experiments are lower than those in the simulation experiment due to the maximum linear velocity being limited to \num{0.5} \si{\metre/\second} during the hardware demonstration.
Robot navigation in these environments with real-time computed GP-Frontiers can be watched in the supplementary video.%\footnotemark[\ref{ft_video}].
% \footnoteref{ft_video}.
% \footnotemark[\ref{ft_video}].%: \hyperlink{https://youtu.be/4tUpmzIcFDY}{https://youtu.be/4tUpmzIcFDY}


% Figure environment removed
% ####################################################
% ################### CONCLUSION ####################
% ####################################################
\vspace{-5pt}
\section{Conclusion} \label{conclusion}
\vspace{-3pt}
We present the GP-Frontier and its navigation approach to navigate the robot safely towards a goal without the need of any global map or path planner. %Based on the simulation results, 
The proposed approach is built on the uncertainty assessment of the VSGP occupancy model of the robot surrounding. The VSGP model provides a high-level representation of the environment, which is a more efficient representation to detect the open (safe) gaps around the robot and is more robust again the noisy measurement. The intensive evaluation shows that our approach has salient advantages over the most recent baseline method.


\vspace{-5pt}
\section*{ACKNOWLEDGMENT}
\vspace{-5pt}
This work was supported by the National Science Foundation with grant numbers 2006886 and 2047169.
We thank Hassan Jardali for his help during the field experiments. 
%%%%%%%%%%%%%%%%%%%%%%%%%%%%%%%%%%%%%%%%%%%%%%%%%%%%%%%%%%%%%%%%%%%%%%%%%%%%%%%%



%\bibliographystyle{plainnat}
\bibliographystyle{unsrt}
\bibliography{references,ref2,ref3,ref-iros}






% %%%%%%%%%%%%%%%%%%%%%%%%%%%%%%%%%%%%%%%%%%%%%%%%%%%%%%%%%%%%%%%%%%%%%%%%%%%%%%%%
% \section*{APPENDIX}

% Appendixes should appear before the acknowledgment.


% The preferred spelling of the word ÒacknowledgmentÓ in America is without an ÒeÓ after the ÒgÓ. Avoid the stilted expression, ÒOne of us (R. B. G.) thanks . . .Ó  Instead, try ÒR. B. G. thanksÓ. Put sponsor acknowledgments in the unnumbered footnote on the first page.



% %%%%%%%%%%%%%%%%%%%%%%%%%%%%%%%%%%%%%%%%%%%%%%%%%%%%%%%%%%%%%%%%%%%%%%%%%%%%%%%%

% References are important to the reader; therefore, each citation must be complete and correct. If at all possible, references should be commonly available publications.



% \begin{thebibliography}{99}

% \bibitem{c1} G. O. Young, ÒSynthetic structure of industrial plastics (Book style with paper title and editor),Ó 	in Plastics, 2nd ed. vol. 3, J. Peters, Ed.  New York: McGraw-Hill, 1964, pp. 15Ð64.
% \bibitem{c2} W.-K. Chen, Linear Networks and Systems (Book style).	Belmont, CA: Wadsworth, 1993, pp. 123Ð135.
% \bibitem{c3} H. Poor, An Introduction to Signal Detection and Estimation.   New York: Springer-Verlag, 1985, ch. 4.
% \bibitem{c4} B. Smith, ÒAn approach to graphs of linear forms (Unpublished work style),Ó unpublished.
% \bibitem{c5} E. H. Miller, ÒA note on reflector arrays (Periodical styleÑAccepted for publication),Ó IEEE Trans. Antennas Propagat., to be publised.
% \bibitem{c6} J. Wang, ÒFundamentals of erbium-doped fiber amplifiers arrays (Periodical styleÑSubmitted for publication),Ó IEEE J. Quantum Electron., submitted for publication.
% \bibitem{c7} C. J. Kaufman, Rocky Mountain Research Lab., Boulder, CO, private communication, May 1995.
% \bibitem{c8} Y. Yorozu, M. Hirano, K. Oka, and Y. Tagawa, ÒElectron spectroscopy studies on magneto-optical media and plastic substrate interfaces(Translation Journals style),Ó IEEE Transl. J. Magn.Jpn., vol. 2, Aug. 1987, pp. 740Ð741 [Dig. 9th Annu. Conf. Magnetics Japan, 1982, p. 301].
% \bibitem{c9} M. Young, The Techincal Writers Handbook.  Mill Valley, CA: University Science, 1989.
% \bibitem{c10} J. U. Duncombe, ÒInfrared navigationÑPart I: An assessment of feasibility (Periodical style),Ó IEEE Trans. Electron Devices, vol. ED-11, pp. 34Ð39, Jan. 1959.
% \bibitem{c11} S. Chen, B. Mulgrew, and P. M. Grant, ÒA clustering technique for digital communications channel equalization using radial basis function networks,Ó IEEE Trans. Neural Networks, vol. 4, pp. 570Ð578, July 1993.
% \bibitem{c12} R. W. Lucky, ÒAutomatic equalization for digital communication,Ó Bell Syst. Tech. J., vol. 44, no. 4, pp. 547Ð588, Apr. 1965.
% \bibitem{c13} S. P. Bingulac, ÒOn the compatibility of adaptive controllers (Published Conference Proceedings style),Ó in Proc. 4th Annu. Allerton Conf. Circuits and Systems Theory, New York, 1994, pp. 8Ð16.
% \bibitem{c14} G. R. Faulhaber, ÒDesign of service systems with priority reservation,Ó in Conf. Rec. 1995 IEEE Int. Conf. Communications, pp. 3Ð8.
% \bibitem{c15} W. D. Doyle, ÒMagnetization reversal in films with biaxial anisotropy,Ó in 1987 Proc. INTERMAG Conf., pp. 2.2-1Ð2.2-6.
% \bibitem{c16} G. W. Juette and L. E. Zeffanella, ÒRadio noise currents n short sections on bundle conductors (Presented Conference Paper style),Ó presented at the IEEE Summer power Meeting, Dallas, TX, June 22Ð27, 1990, Paper 90 SM 690-0 PWRS.
% \bibitem{c17} J. G. Kreifeldt, ÒAn analysis of surface-detected EMG as an amplitude-modulated noise,Ó presented at the 1989 Int. Conf. Medicine and Biological Engineering, Chicago, IL.
% \bibitem{c18} J. Williams, ÒNarrow-band analyzer (Thesis or Dissertation style),Ó Ph.D. dissertation, Dept. Elect. Eng., Harvard Univ., Cambridge, MA, 1993. 
% \bibitem{c19} N. Kawasaki, ÒParametric study of thermal and chemical nonequilibrium nozzle flow,Ó M.S. thesis, Dept. Electron. Eng., Osaka Univ., Osaka, Japan, 1993.
% \bibitem{c20} J. P. Wilkinson, ÒNonlinear resonant circuit devices (Patent style),Ó U.S. Patent 3 624 12, July 16, 1990. 






% \end{thebibliography}




\end{document}
