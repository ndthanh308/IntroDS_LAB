\documentclass{article}
\usepackage{graphicx} % Required for inserting images
\usepackage[top=3cm,bottom=3cm,left=3cm,right=3cm]{geometry}

\usepackage{palatino}
\usepackage{latexsym}

\usepackage[mathscr]{eucal}
\usepackage{amsmath}
\usepackage{amsthm}
\usepackage{amsfonts}
\usepackage{amssymb}
\usepackage{amscd}
\usepackage{color}
\usepackage{graphicx}
\usepackage{graphics}
\usepackage{pifont}
\usepackage{subfigure}
\usepackage[makeroom]{cancel}
\usepackage[normalem]{ulem}
\usepackage[dvipsnames]{xcolor}
\usepackage{tikz}
\usepackage{cite}
\theoremstyle{plain}
\newtheorem{theorem}{Theorem }
\newtheorem*{theoremA}{Theorem A}
\newtheorem*{theoremB}{Theorem B}
\newtheorem{remark}[theorem]{Remark}
\newtheorem{lemma}[theorem]{Lemma}
\newtheorem{proposition}[theorem]{Proposition}
\newtheorem{definition}[theorem]{Definition}
\newtheorem{corollary}[theorem]{Corollary}
\newtheorem{example}[theorem]{Example}
\newtheorem{assumption}[theorem]{Assumption}

\newcommand{\RR}{\mathcal{R}}
\newcommand{\R}{{\mathbb R}}
\newcommand{\N}{{\mathbb N}}

\newcommand{\C}{{\mathbb C}}
\newcommand{\TT}{{\mathcal{T}}}
\newcommand{\la}{{\langle}}
\newcommand{\ra}{{\rangle}}
\newcommand{\grad}{{\nabla}}
\renewcommand{\div}{\operatorname{div}}


\title{Small-time controllability for the nonlinear Schrödinger equation on $\R^N$ via bilinear electromagnetic fields}
\author{Alessandro Duca,\footnote{Universit\'e de Lorraine, CNRS, INRIA, IECL, F-54000 Nancy, France;  e-mail: alessandro.duca@inria.fr}\and{Eugenio Pozzoli\footnote{Dipartimento di Matematica, Università di Bari, I-70125 Bari, Italy. (eugenio.pozzoli@uniba.it).}}}

\begin{document}%Gross-Pitaevskii equation

\maketitle

\begin{abstract}
We address the small-time controllability problem for a nonlinear Schr\"odinger equation (NLS) on $\R^N$ in presence of magnetic and electric external fields. We chose a particular framework where the equation becomes $i\partial_t \psi = [-\Delta+u_0(t)h_{\vec{0}}+\la u(t), P\ra +\kappa|\psi|^{2p}]\psi$. Here, the control operators are defined by the zeroth Hermite function $h_{\vec{0}}(x)$ and the momentum operator $P=i\grad$. %The time-dependent functions $u_0:\R^+\longrightarrow \R$ and $u:\R^+\longrightarrow \R^N$ are freely chosen among the piecewise constant functions, and play the role of control laws. 


In detail, we study when it is possible to control the dynamics of (NLS) as fast as desired via sufficiently large control signals $u_0$ and $u$. We firstly show the existence of a family of quantum states for which this property is verified. Secondly, by considering some specific states belonging to this family, as a main physical implication we show the possibility of controlling precisely any arbitrary change of energy in the quantum system, in time zero. 

Our results are proved by exploiting the idea that the nonlinear term in (NLS) is only a perturbation of the linear problem when the time is as small as desired. % The core of the proof, then, is the controllability of the linear equation which is tackled via a Lie algebraic approach of conjugated linear propagators.
The core of the proof, then, is the controllability of the bilinear equation which is tackled by using specific non-commutativity properties of infinite-dimensional propagators.
\end{abstract}

% REQUIRED
\textbf{Keywords}: Nonlinear Schr\"odinger equation, approximate controllability, small-time quantum control.\\

% REQUIRED
\textbf{MSCcodes}
35Q55, 81Q93, 93B05\\
\section{Introduction}
\subsection{The %physical 
model}

The dynamics of a quantum particle moving in the \\$N$-dimensional euclidean space subject to external electric and magnetic fields, and state nonlinearity, can be described via the following nonlinear Schr\"odinger equation 
%which play the role of controls.
%We deal with the case in which the dynamics of the particle is driven by the following nonlinear Schr\"odinger equation on $\R^N$
\begin{equation}\label{eq:schro_intro}
i \frac{\partial}{\partial t}\psi=\left[-\big(\div+iA\big)\circ\big(\grad+iA\big)  +E+\kappa|\psi|^{2p}\right]\psi,
\end{equation}
where $\psi=\psi(x,t), (x,t)\in\mathbb{R}^N\times[0,T]$, $\kappa\in\mathbb{R}$, $p\geq 1$ and $T>0$.  The multipolarized magnetic field is represented by the $\mathbb{R}^N$-valued function $A$ which, in our model, only depends on time $$A=(A_1,...,A_N):[0,T]\longrightarrow \R^N.$$ The scalar electric field, instead, is modeled by the $\mathbb{R}$-valued function $E$ depending on both time and space: $$E:[0,T]\times \R^N\longrightarrow \R.$$
The particle is represented by the quantum state $\psi$ evolving in the unit sphere of $L^2(\R^N,\C)$. An example of such evolution is the celebrated Gross-Pitaevskii equation, when $p=1$.
The aim of this paper is to investigate small-time controllability properties for \eqref{eq:schro_intro} via suitable electromagnetic fields. We consider suitable quantum states and we show that is possible to control them as fast as desired via sufficiently large electromagnetic fields $A$ and $E$. 

\smallskip

%\subsection{The setting}
In this paper, we consider the equation \eqref{eq:schro_intro} in a particular case where it can be rewritten as the following nonlinear Schr\"odinger equation on the Hilbert space $H^s(\mathbb{R}^N,\mathbb{C})$, $s>N/2$,
\begin{equation}\label{eq:schro}\tag{NLS}\begin{cases}
i \dfrac{\partial}{\partial t}\psi(x,t)=\left[-\Delta+u_0(t)h_{\vec{0}}(x)+\la u(t), P\ra +\kappa|\psi(x,t)|^{2p}\right]\psi(x,t),\quad & \\
\psi(\cdot,0)=\psi_0(\cdot)\in H^s(\mathbb{R}^N,\mathbb{C}), \quad (x,t)\in \mathbb{R}^N\times[0,T],&
\end{cases}
\end{equation}
where $u_0:[0,T]\longrightarrow \R$ and $u=(u_1,...,u_N):[0,T]\longrightarrow \R^N$ and
$$\Delta=\sum_{l=1}^N\frac{\partial^2}{\partial {x_l}^2},\quad  \quad P=i \grad=i\left(\frac{\partial}{\partial {x_1}},\dots,\frac{\partial}{\partial {x_N}}\right), \quad \quad h_{\vec{0}}(x)=\pi^{-\frac{N}{4}}e^{-\frac{|x|^2}{2}}.$$%=\pi^{-\frac{N}{4}}e^{-\frac{x_1^2+...+x_N^2}{2}}.$$
These are respectively the kinetic energy operator (that is, the Laplacian $\Delta$), the momentum operator $P$ (its components will also be denoted as $P_j=i\partial/\partial {x_j}$), and the zeroth $N$-dimensional Hermite function $h_{\vec{0}}$. %The state of the system is described by the complex-valued wavefunction $\psi$ evolving in the sphere (of radius $\|\psi(\cdot,0)\|_{L^2}$) of the space $L^2(\mathbb{R},\mathbb{C})$.
The equation \eqref{eq:schro} can be rewritten in the form of \eqref{eq:schro_intro} by choosing (see Section \ref{sec:preliminaries} for further details)
$$A(t)=%u(t)=
-\frac{1}{2}(u_1(t),...,u_N(t)),\quad \quad \quad E(t,x)=u_0(t)h_{\vec{0}}(x)-\frac{1}{4}|u|^2.$$%=u_0(t)h_{\vec{0}}(x)-\sum_{l=1}^N u_l^2(t)$$


\smallskip


In the framework of \eqref{eq:schro}, the particle is subjected to two external control fields: the first one is coupled to the momentum of the particle $P$, and the second one is coupled to a function of its position $x$, which is $h_{\vec{0}}(x)$. Notice that the choice of the control operator $h_{\vec{0}}$ is mathematically convenient (see Section \ref{sec:saturation} for more details) and is realistic from a physical point of view, as the dipolar interaction is concentrated on small values of the position $x$, and becomes weak for large $x$.
The time-dependent functions $u_0,...,u_N$ can be freely chosen among the piecewise constant functions, and play the role of control laws which steer the quantum dynamics towards desired targets. 


\subsection{The main results}
 The first main result of the work is a specific small-time approximate controllability property which is stated in the following theorem.
\begin{theorem}\label{thm:main-result}
Let $s\in\N^*$ be such that $s>N/2$. Consider any initial state $\psi_0\in H^s(\mathbb{R}^N,\mathbb{C})$ and any $\phi\in H^{2s}(\mathbb{R}^N,\mathbb{R})$. Then, for any positive error and time $\varepsilon,T>0$, there exist a smaller time $\tau\in[0,T)$ and piecewise constant controls $(u_0,u):[0,\tau]\to \mathbb{R}^{N+1}$ such that the solution $\psi(t;\psi_0)$ of \eqref{eq:schro} with initial condition $\psi_0$ and controls $(u_0,u)$ satisfies
$$\|\psi(\tau;\psi_0)-e^{i\phi}\psi_0\|_{H^s(\R^N)}<\varepsilon. $$
\end{theorem} 

Theorem \ref{thm:main-result} allows to steer, as fast as desired, any sufficiently smooth quantum state to any other state obtained by multiplying it with an exponential function. Notice that this operation is not just a change of phase since the function $\phi$ is not necessarily a constant. We consider the parameter $s>N/2$ so that the equation \eqref{eq:schro} is locally well-posed in the space $H^s(\R^N,\C)$, and a final phase $\phi\in H^{2s}(\R^N,\R)$ so that $e^{i\phi}\psi_0\in H^s(\R^N,\C)$.  

Notice that already in the linear case (i.e. when $\kappa=0$) Theorem \ref{thm:main-result} is new: in this case, the result is also valid in $ H^s(\R^N,\C)$ for $s\leq N/2$ since the system is well-posed even in this lower regular cases.

\smallskip

It is worth stressing that, as a main physical consequence, Theorem \ref{thm:main-result} implies the capability of producing precise and arbitrary changes of energy in the quantum system, in time zero. To better visualize this fact, we introduce the following quantum states:
% As a consequence of Theorem \ref{thm:main-result}, we get the approximate controllability in small times between the generalized complex eigenfunctions of the Laplacian $-\Delta$ on every subset $S$ of $\R^N$ of finite measure. The result is presented in the following theorem where we use the notation 
\begin{equation*}
\phi_{\xi,S}=e^{i\xi x}\chi^\epsilon_S(x),\quad \xi\in\mathbb{R},
\end{equation*} 
where $S\subset \mathbb{R}^N$ is a subset with finite Lebesgue measure ${\rm meas}(S)$, and
$$
C^\infty_c(\R^N,\R)\ni\chi^\epsilon_S(x)=
\begin{cases}
1, & x\in S\\
0, & x\in S+\epsilon
\end{cases}
$$
is a cut-off function. Such state has approximately (i.e. for $\epsilon>0$ small enough) energy $\xi^2{\rm meas}(S)$. We then have the following result.

 
\begin{corollary}\label{thm:eigenmodes}
Let $s\in\N^*$ be such that $s>N/2$. Let $\xi,\nu\in\mathbb{R}$ be two frequencies. Consider any subset $S\subset\mathbb{R}^N$ such that ${\rm meas}(S)<\infty$. Then, for any positive error and time $\varepsilon,T>0$, there exist a smaller time $\tau\in[0,T)$ and a piecewise constant control $(u_0,u):[0,\tau]\to \mathbb{R}^{N+1}$ such that the solution $\psi(t;\phi_{\xi, S})$ of \eqref{eq:schro} with initial condition $\phi_{\xi, S}$ and control $(u_0,u)$ satisfies
$$\|\psi(\tau;\phi_{\xi, S})-\phi_{\nu, S}\|_{H^s(\R^N)}<\varepsilon. $$
\end{corollary}
Corollary \ref{thm:eigenmodes} thus yields the capability of controlling instantaneous change from the energy $\xi^2{\rm meas}(S)$ to the energy $\mu^2{\rm meas}(S)$, for any frequencies $\xi,\mu\in \mathbb{R}$ and any finite support $S\subset \R^N$ where the quantum states are concentrated.
%Corollary \ref{thm:eigenmodes} yields the approximate controllability between any couples of generalized eigenfunctions of the Laplacian $-\Delta$ when we restrict them on any finite measure subsets of $\R^N$: these states are relevant since they are local (i.e., away from the boundary of $S$) stationary states of the free evolution in the linear case.% (i.e., when $\kappa=0$).

%Notice that when $\kappa=0$, we can define a well-posed flow in the Hilbert space $L^2(\R^N,\C)$ and, then, we can replace the smooth function $\chi_S$ with the characteristic function of the subset $S$. In any case, the choice of restricting the quantum states to finite measure subsets of $\R^N$ allows to define a locally well-posed dynamics in $H^s(\R^N,\C)$ (or $L^2(\R^N,\C)$ in the linear case) and obtain the controllability result. %sarebbe bello aggiungere qualcosa sulle generalized eigenfunctions




\smallskip

Similar results can be found in the work \cite{duca-nersesyan} by Nersesyan and the first author, and in \cite{small-time-molecule,small-time-wave} by Chambrion and the second author. Nevertheless, they all deal with different frameworks from the one considered here. %, and in particular with PDEs on compact manifolds. 
From this perspective, the main novelties of Theorem \ref{thm:main-result} are the following. %is in the capability of controlling the  Schr\"odinger equation on the whole space $\mathbb{R}^N$. Furthermore, two additional main features of our technique consist in the possibility of assessing controllability properties in small times, and also in the presence of state nonlinearity.
%\begin{itemize}
  

    
\begin{itemize}
  
    
    
    \item Bilinear controls, not only for the Schr\"odinger equation, but also for the heat or for the wave equations, are usually studied on compact manifolds. Indeed, most of the classical techniques rely upon spectral techniques needing a drift with point spectrum only. % is that the classical techniques are based on the existence of a base for $L^2$ made by eigenfunctions of the Laplacian.
    Our approach, inspired by \cite{duca-nersesyan,small-time-molecule,small-time-wave}, does not require such features and then can be also applied in $\R^N$.

\item Up to the recent works \cite{duca-nersesyan,small-time-molecule,small-time-wave}, classical approximately controllability results has been proved in large times. %Typically, the more we want to improve the error estimate, the more the controllability time growths. %aggiungere referenze UGo Mario ecc e magari citare le tecniche usate
In our main results, the controllability is as fast as desired and the obvious price to pay is that the corresponding control amplitudes are more and more large. 

\item The controllability properties of Theorem \ref{thm:main-result} and Corollary \ref{thm:eigenmodes} are ensured despite the non-linear behaviour of the dynamics. Up to \cite{duca-nersesyan}, most of the existing works on the subject can only deal with the linear case. Here, we improve the the techniques from \cite{small-time-molecule,small-time-wave} in order to treat also non-linear equations. 


\end{itemize}
    
%    \item The controllability via bilinear controls, not only for the Schr\"odinger equation, but also for the heat or for the wave equations, is usually studied on compact manifolds. The main reason is that the classical techniques require the existence of a complete orthonormal system of the space $L^2$ composed by eigenfunction of the Laplacian. Our approach, inspired by the techniques from \cite{small-time-molecule,small-time-wave}, does not require such property and then can be exploited also in $\R^N$.

%\item Up to the previously mentioned works \cite{duca-nersesyan,small-time-molecule,small-time-wave}, classical approximately controllability results has been proved in very large time. Typically, the more we want to improve the error estimate, the more the controllability time growths. On the subject, we refer to .......... %aggiungere referenze UGo Mario ecc e magari citare le tecniche usate
%In our main results, the controllability is as fast as desired and the obvious price to pay is that the corresponding controls are more and more large. 

%\item The controllability is ensured despite the non-linear behaviour of the dynamics. Up to \cite{duca-nersesyan}, the existing works on the subject study the bilinear approximate controllability only in the linear case. This is again due to the different choice of techniques adopted to ensure the result. In this work, we show that the techniques developed in \cite{small-time-molecule,small-time-wave} can be improved in order to deal with non-linear equations. 


%\end{itemize}



%infinite-dimensional quantum controllability property proved in Theorem \ref{thm:main-result} and the control between the truncated eigenfunctions proved in Corollary \ref{thm:eigenmodes} have been recently shown by the first author and Nersesyan \cite[Theorems A and B]{duca-nersesyan} on the torus by means of low modes, and hence Theorem \ref{thm:main-result} proposes a different unbounded setting where this property still holds.
\subsection{The technique}
The control strategy to show Theorem \ref{thm:main-result} is explicit, and consists in applying large controls in short time intervals (which is natural, since we want to control the system in small times). These kind of techniques are well-known in finite-dimensional geometric control, where large controls on short time intervals are usually considered to avoid the effect of the drift on the dynamics \cite{jurdje-kupka,glaser-brockett,domenico}. 

\smallskip

From a PDE perspective, this technique is inspired by the work of the first author and Nersesyan \cite{duca-nersesyan}, previously introduced, which treats the bilinear control of nonlinear Schr\"odinger equations on tori by means of low mode forcing. There, the authors leveraged the drift term to prove small-time approximate controllability among eigenstates. In this paper, beside considering a different unbounded model, we exploit different properties of non-commutativity of infinite-dimensional propagators: instead of leveraging the drift generated by the Laplacian $-\Delta$, we get rid of it and exploit the momentum operator P to control the quantum system.


\smallskip

More precisely, if one forgets about the nonlinearity, the main idea of the proof is to consider the following small-time limit of conjugated dynamics:%, holding w.r.t. the $L^2$-norm for any initial condition $\psi_0\in L^2(\mathbb{R},\mathbb{C})$ :
\begin{equation}\label{eq:limit-intro}
\lim_{\tau\to 0} e^{ih_{\vec{0}}/\tau}e^{-i\tau P_j}e^{-ih_{\vec{0}}/\tau}\psi_0=e^{-P_jh_{\vec{0}}}\psi_0.
\end{equation}
This allows to gain a new non-directly accessible direction where the system can be steered: in analogy to finite-dimensional systems, we may regard $h_{\vec{0}}$ and $P_j$ as two linearly independent and directly accessible directions for the control system; then 
$$-P_jh_{\vec{0}}=i\pi^{-N/4}x_je^{-|x|^2/2}$$
is the first Hermite function in the $x_j$-variable and the zeroth in the other variables. Hence we generate a linearly independent direction which was not directly accessible. This new direction is, in fact, defined by (minus) the commutator of the first two (an operation also referred to as Lie bracket in geometric control): 
$$[P_j,h_{\vec{0}}]\psi_0:=(P_jh_{\vec{0}}-h_{\vec{0}}P_j)\psi_0=-i\pi^{-N/4}x_je^{-|x|^2/2}\psi_0.$$
 We then iterate this strategy to generate any linear combination of Hermite functions (this procedure is usually called \emph{saturation}, and has been introduced by Agrachev and Sarychev for controlling Navier-Stokes equations \cite{navier-stokes}, and we refer also to their recent work \cite{agrachev-ensemble} where Hermite functions are specifically considered for saturating purposes): being this set dense, we obtain the small-time approximate controllability property stated in Theorem \ref{thm:main-result} in the linear case. The result in the nonlinear case then follows by showing that the nonlinearity is a perturbation that does not influence the small-time limit \eqref{eq:limit-intro} (we refer to Proposition \ref{lemma:limit-nonlinear}). %In the proofs of our main results, we treat the nonlinear term as a perturbation of the linear problem, which is tackled via a Lie algebraic approach of conjugated linear propagators (in the spirit of \cite{small-time-molecule,small-time-wave}). 
\subsection{%Some Main contributions w.r.t. the 
Other literature}

The study of the controllability properties of bilinear Schr\"odinger equations plays an important role in many applications, due to the relevance of quantum phenomena in physics, chemistry, engineering and information science. In particular, much attention has been devoted to these kind of systems due to their fundamental relevance in the development of quantum technologies \cite{Glaser2015}. 



\smallskip

From a mathematical point of view, %bilinear infinite-dimensional control systems are not exactly controllable in the natural functional space where the evolution is defined (in the case of \eqref{eq:schro}, in $L^2$) \cite{BMS,Chambrion-Caponigro-Boussaid-2020}. Despite this obstruction, 
many controllability results %in finer topologies (e.g., smaller functional spaces as target states, or approximate controllability) 
have been obtained in the last two decades. Local exact controllability in higher Sobolev norms was firstly proven by Beauchard in \cite{beauchard1} (see also \cite{laurent}). For the global approximate controllability, the first achievements were made by Adami, Boscain, Chambrion, Mason, and Sigalotti \cite{adami,BCMS} and by Nersesyan \cite{nersesyan}. All of those works rely on the discreteness of the drift spectrum. %Most of the control strategy for bilinear Schr\"odinger equations rely upon spectral techniques needing a drift with point spectrum only. 

\smallskip

Different quantum controllability properties (e.g., control between bound states) in the presence of a continuous spectrum in the drift as in the case of our work were studied by Mirrahimi \cite{mirra2}, and Chambrion \cite{chambrion}. 

\smallskip

The problem of minimizing the controllability time, and more in general proving small-time controllability properties, % Time-optimal control 
is of prime importance in quantum mechanics, with very few results available in the PDE setting. Boussaïd, Caponigro, and Chambrion \cite{minimal-time-thomas} showed that a particular bilinear conservative equation on the circle is globally approximately controllable in small times. Beauchard, Coron, and Teismann \cite{minimal-time-coron,minimal-time-approximate} proved an obstruction to small-time approximate controllability for bilinear Schr\"odinger equations with sub-quadratic uncontrolled potential (which is instead approximately controllable in large times). We also refer to the recent work \cite{coron-small-semiclassical} by Coron, Xiang, and Zhang, where the geometric technique of low mode forcing is used for small-time control of semiclassical
bilinear Schr\"odinger equations.

We conclude by noticing that the capability of controlling quantum evolutions by means of their momentum (instead of the more standard position operator) was speculated e.g. in the work of Boscain, Mason, Panati, and Sigalotti \cite[Section I.C]{panati} on spin-boson systems, and the present paper thus provides some insights also in this direction.
\smallskip

%Previous controllability results in the non-linear setting were obtained in the two papers \cite{laurent,duca-nersesyan}. 
%Recently, new geometric control techniques based on low modes forcing have been developed by the first author and Nersesyan \cite{duca-nersesyan} for showing small-time approximate controllability among the eigenstates of the drift on a torus of arbitrary dimension (we refer also to the work of Chambrion and the second author \cite{small-time-molecule} in which similar properties are studied on the sphere, and to the paper of Coron, Xiang, and Zhang \cite{coron-small-semiclassical} where low mode forcing is used for small-time control of semiclassical bilinear Schr\"odinger equations) and serve as a basis for our analysis. 

%\smallskip

%We conclude by mentioning another novelty of this paper: it is in the use of the momentum operator to control the system; the capability of controlling the quantum evolution by means of the momentum (instead of the more standard position operator) was speculated e.g. in the work of Boscain, Mason, Panati, and Sigalotti \cite[Section I.C]{panati} on spin-boson systems. 


\subsection{Structure of the paper} The paper is organised as follows. In Section \ref{sec:preliminaries}, we recall the notion of solution for \eqref{eq:schro}. In Section \ref{sec:small-time}, we prove the small-time limit of conjugated dynamics \eqref{eq:limit-intro}, and we treat also the nonlinear case. In Section \ref{sec:saturation}, we show a density property of the directions where the control system can be steered. We conclude in Section \ref{sec:proof} by proving Theorem \ref{thm:main-result}.



\subsection{Acknowledgments}
The authors would like to thank Ugo Boscain, Nabile Boussa\"id, Thomas Chambion, David Dos Santos Ferreira, and Vahagn Nersesyan for fruitful conversations.

E.P. acknowledges support by the PNRR MUR project PE0000023-NQSTI. This work
was also part of the project CONSTAT, supported by the Conseil Régional de Bourgogne Franche-Comté and the European Union through the PO FEDER Bourgogne 2014/2020 programs, by the French ANR through the grant QUACO (ANR-17-CE40-0007-01) and by EIPHI Graduate School (ANR-17-EURE-0002).



%\subsection{Notations}
% In what follows of the manuscript, we use the following notation.
%\begin{itemize}
%    \item 
%\end{itemize}


 
\section{Preliminaries and well-posedness}\label{sec:preliminaries}

We start by showing the computations allowing to rewrite the equation \eqref{eq:schro_intro} in the form of \eqref{eq:schro} when 
\begin{equation}\label{fields}A(t)=%u(t)=
-\frac{1}{2}(u_1(t),...,u_N(t)),\quad \quad \quad E(t,x)=u_0(t)h_{\vec{0}}(x)-\frac{1}{4}|u|^2,
\end{equation}
as presented in the introduction. We firstly rewrite the equation \eqref{eq:schro_intro} in the form
\begin{equation}\label{eq:preliminaries_1}\begin{split}
i \frac{\partial}{\partial t}\psi&= -\div(\grad \psi)  - i\div(A(t)\psi)     - i\la A(t), \grad\psi\ra  +|A(t)|^2\psi+E(x,t)\psi+\kappa|\psi|^{2p}\psi.\\
\end{split}\end{equation}
Notice that the magnetic field $A(t)$ only depends on time and, then, we have
$$\div(A(t)\psi)= \div(A(t))\psi + \la A(t), \grad \psi\ra =\la A(t), \grad \psi\ra.$$
The last identity allows to rewrite the equation \eqref{eq:preliminaries_1} as follows 
\begin{equation*}\begin{split}
i \frac{\partial}{\partial t}\psi&= -\Delta\psi  - 2i\la A(t), \grad \psi\ra  +|A(t)|^2\psi+E(x,t)\psi+\kappa|\psi|^{2p}\psi\\
&= -\Delta\psi +  \la - 2A(t), i\grad \psi\ra  +\big(|A(t)|^2+E(x,t)\big)\psi+\kappa|\psi|^{2p}\psi.\\
\end{split}\end{equation*}
Finally, we can write $i\grad = P$, $-2A(t)=u(t)$ and $|A(t)|^2+E(x,t)=u_0(t)h_{\vec{0}}(x)$
thanks to \eqref{fields}, which lead to the nonlinear Schr\"odinger equation \eqref{eq:schro}.

\smallskip

We state now a standard lemma, whose proof we give for completeness.
\begin{lemma}\label{lem:s-a}
Let $D(P_j)=H^1(\mathbb{R},\mathbb{C})\otimes L^2(\mathbb{R}^{N-1},\mathbb{C})$ and $D(\Delta)=H^2(\mathbb{R}^N,\mathbb{C})$ be the domains of the self-adjoint operators $P_j$ and $\Delta$ on $L^2(\mathbb{R}^N,\mathbb{C})$. Then $-\Delta+uP_j$ is self-adjoint on the domain $H^2(\mathbb{R}^N,\mathbb{C})$ for any $u\in\mathbb{R}$.
\end{lemma}
\begin{proof} 
For any $\psi\in H^2(\mathbb{R},\mathbb{C})$ and $\varepsilon>0$ we have that
\begin{align*}\|P_j\psi\|_{L^2}^2=\langle P_j\psi,P_j\psi\rangle_{L^2}=\langle P_j^2\psi,\psi\rangle_{L^2}\leq\langle -\Delta\psi,\psi\rangle_{L^2}\leq \frac{1}{2}\left(\varepsilon \|\Delta\psi\|^2_{L^2}+\frac{1}{\varepsilon}\|\psi\|^2_{L^2}\right), \end{align*}
having used the positivity of $-(\Delta-\partial^2/\partial x_j^2)$ and the Young inequality. This shows that $P_j$ is infinitesimally small w.r.t. $\Delta$. Then, a standard application of the Kato-Rellich Theorem \cite[Theorem X.12]{rs2} concludes the proof. 
\end{proof}
Being $h_{\vec{0}}$ a bounded self-adjoint operator on $L^2(\mathbb{R}^N,\mathbb{C})$ (seen as a multiplication), also $-\Delta+u_0h_{\vec{0}}+\la u,P\ra$ with domain $H^2(\mathbb{R}^N,\mathbb{C})$ for any $(u_0,u)\in\mathbb{R}^{N+1}$ is self-adjoint on $L^2(\mathbb{R}^N,\mathbb{C})$.

\smallskip

Lemma \ref{lem:s-a} and the spectral theorem hence allow one to define the notion of solution to \eqref{eq:schro} when $\kappa=0$. Indeed, given any piecewise constant control law $$(u_0,u):[0,T]\to \mathbb{R}^{N+1},\ \ \ \ \ \ \ (u_0(\tau),u(\tau))=(u^{(j)}_0,u^{(j)})\in\mathbb{R}^{N+1}$$ for $\sum_{l=1}^{j-1}t_l\leq \tau<\sum_{l=1}^j t_l\leq T$ and any initial condition $\psi_0\in L^2(\mathbb{R}^N,\mathbb{C})$, we have the existence of a unique mild solution  $$\psi\in C^0(\mathbb{R},L^2(\mathbb{R}^N,\mathbb{C}))$$ of \eqref{eq:schro} for $\kappa=0$ satisfying $\psi(t=0)=\psi_0$. Such solution can be written as a composition of time independent propagators applied to $\psi_0$ so that for $\sum_{l=1}^{j-1}\leq t<\sum_{l=1}^j t_l$:
\begin{equation}\label{eq:propagator}
\psi(t)=e^{(t-\sum_{l=1}^{j-1}t_l)(-\Delta+u_0^{(j)}h_{\vec{0}}+\la u^{(j)},P\ra)}\circ\dots\circ e^{t_1(-\Delta+u_0^{(1)}h_{\vec{0}}+\la u^{(1)},P\ra)}\psi_0.
\end{equation}
 

%Denote as $\psi(t)=\mathcal{R}(t,\psi_0,u)$ the solution of \eqref{eq:schro} at time $t$ associated to the initial condition $\psi_0$ and control law $u=(u_0,u)$. From the unitarity and the linearity of the propagators (in the case $\kappa=0$), we have
%\begin{equation}
%\|\mathcal{R}(t,\psi_0,u)-\mathcal{R}(t,\psi_1,u)\|_{L^2(\R^N)}=\|\psi_0-\psi_1\|_{L^2(\R^N)}.
%\end{equation}


The existence and the unicity of local solutions in the nonlinear case, {i.e} $\kappa\neq 0 $, is ensured by the following proposition. Here, we collect some well-known results concerning the local well-posedness and stability of the nonlinear Schr\"odinger equation in Sobolev spaces. In such results the parameters~$N\ge1$,~$p\ge1$, and~$\kappa\in \R $ are arbitrarily chosen. 


\begin{proposition}\label{prop:well}
Let $s\in\N^*$ be such that $s>N/2$. For any $\psi_0\in H^s(\mathbb{R}^N,\mathbb{C})$, and   $(u_0,u)\in L^1_{loc}(\R_+, \R^{N+1})$, there is a maximal time $\TT=\TT(\psi_0, u)>0$ and a unique unitary solution $\psi$ of the problem \eqref{eq:schro} with initial state $\psi_0$ and control $(u_0,u)$ such that its restriction to $[0,T]$ with $T<\TT$ belongs  to~$$C^0\Big(\big[0,T], H^s(\mathbb{R}^{N},\mathbb{C}\big)\Big).$$ When~$\TT<+\infty$, as soon as $t\to \TT^-$, we have $\|\psi(t)\|_{H^s(\R^N)}\to +\infty.$ For any~$T<\TT$, if we denote $$K= \|\psi_0\|_{H^s(\R^N)}+\| u_0\|_{L^1([0,T],\R)}+\| u\|_{L^1([0,T],\R^{N})},$$ then there exists a constant $C=   C(T,K)>0$ such that 
 \begin{align}\label{propreg}
&\|\RR(\cdot,\psi_0,(u_0,u))\|_{C([0,T],H^s(\R^N))}\!\!\leq\!\! C \left( \|\psi_0\|_{H^s(\R^N)} + \|u_0\|_{L^1((0,T), \R)} + \|u\|_{L^1((0,T), \R^{N}) } \right).
\end{align}
In addition, when we call $\tilde K= \|\psi\|_{C([0,T],H^s(\R^N))}+\| u_0\|_{L^1([0,T],\R)}+\| u\|_{L^1([0,T],\R^{N})},$ there exist two constants 


 $$\delta=\delta(T,\tilde K)>0,\ \ \ \ \ \ \ \tilde C=\tilde C(T,\tilde K)>0,$$
such that the following properties hold.


\begin{enumerate}

\item[(i)] For any $\hat\psi_0\in H^s(\R^N,\C) $   and $(\hat u_0,\hat u)\in L^1([0,T], \R^{N+1})$  satisfying  

\begin{equation}\label{stability}
\|\hat\psi_0- \psi_0\|_{H^s(\R^N)}+ \|\hat u_0- u_0\|_{L^1((0,T),\R)} + \|\hat u-  u\|_{L^1((0,T),\R^{N})}  < \delta,
\end{equation}

  the problem \eqref{eq:schro} admits a unique solution $\psi\in C\big([0,T],H^s(\mathbb{R}^{N},\mathbb{C})\big).$

\item[(ii)] Denote by $\RR$ the unitary propagator for the Schr\"odinger equation~\eqref{eq:schro} relating any initial state $\psi_0$ and any control $(u_1,u_2)$
 to the corresponding solution $R(t,\psi_0,u)=\psi(t)$. For any $\hat\psi_0\in H^s(\R^N,\C)$ and $(\hat u_0,\hat u)\in L^1([0,T], \R^{N+1})$ satisfying \eqref{stability}, we have     
 \begin{align*}
&\|\RR(\cdot,\hat \psi_0,(\hat u_0,\hat u)) -\RR(\cdot,\psi_0,(u_0,u))\|_{C([0,T],H^s(\R^N))}\\
&\leq \tilde C \left( \|\hat \psi_0-\psi_0\|_{H^s(\R^N)} + \|\hat u_0-u_0\|_{L^1((0,T), \R)} + \|\hat u-u\|_{L^1((0,T), \R^{N}) } \right).
\end{align*}

\end{enumerate}

\end{proposition}

The proof of Proposition \ref{prop:well} is quite standard: the crucial point is that the momentum operator $P_j$ are infinitesimally small w.r.t. $\Delta$ (see the proof of Lemma \ref{lem:s-a}) and the well-posedness of the Schr\"odinger equation in presence of polynomial nonlinearities is straightforward, when $s>N/2$. To this purpose, we omit its proof and we refer to some suitable references on the two aspects, which are \cite[Section 4.10]{C-2003} and \cite[Section 6.3]{P-1992}. For further results on the subject, we also refer to \cite[Section~3.3]{TT-2006}.




\medskip



Let us recall that the concatenation $v*u$ of two scalar control laws $u:[0,T_1]\to \mathbb{R}^{N+1},v:[0,T_2]\to \mathbb{R}^{N+1} $ is the scalar control law defined on $[0,T_1+T_2]$ as follows
$$(v*u)(t)=\begin{cases}
u(t), & t\in[0,T_1]\\
v(t-T_1), & t\in(T_1,T_1+T_2],
\end{cases} $$
and the definition extends to controls with values in $\mathbb{R}^{N+1}$ componentwise. Note also that
$$\mathcal{R}(T_1+t,\psi_0,v*u)=\mathcal{R}(t,\mathcal{R}(T_1,\psi_0,u),v),\quad t>0. $$
\section{Small-time limits}\label{sec:small-time}


The aim of this section is to prove some small-time limits, as the one presented in the introduction (cf. \eqref{eq:limit-intro}).

\begin{proposition}\label{lemma:main-tool}
Let $\psi_0\in L^2(\mathbb{R}^{N},\mathbb{C})$ and $\varphi\in C^{\infty}(\mathbb{R}^{N},\mathbb{R})$. For any $\tau\in\mathbb{R}$ and $\delta>0$, the following limits hold w.r.t. the $L^2$-norm:
\begin{itemize}
\item[(i)]$$\lim_{\delta\to 0^+}\exp\left(-i\delta\left(-\Delta\mp\frac{\varphi}{\delta\tau}\right)\right)\psi_0=\exp\left(\pm i \frac{\varphi}{\tau}\right)\psi_0;$$
\item[(ii)]$$\lim_{\delta\to 0^+}\exp\left(-i\delta\left(-\Delta+\frac{\tau}{\delta}P_j\right)\right)\psi_0=\exp\left(- i \tau P_j\right)\psi_0,\quad j=1,\dots,N;$$
\item[(iii)]$$\lim_{\tau\to 0^+}\exp\left(\mp i \frac{\varphi}{\tau}\right)\exp(-i\tau P_j)\exp\left(\pm i \frac{\varphi}{\tau}\right)\psi_0=\exp(\pm  P_j\varphi)\psi_0,\quad j=1,\dots,N. $$
%\item[(iii)]$$\text{if}\quad  \varphi\in W^{1,\infty}(\mathbb{R}^{N},\mathbb{R}), \quad\lim_{\tau\to 0}\exp\left(\mp i \frac{\varphi}{\tau}\right)\exp(-i\tau P)\exp\left(\pm i \frac{\varphi}{\tau}\right)\psi_0=\exp(\pm  P\varphi)\psi_0. $$
%\begin{align*}
%&\lim_{\tau\to 0}\lim_{\delta_3\to 0^+}\lim_{\delta_2\to 0^+}\lim_{\delta_1\to 0^+}\\
%&\exp\!\left(\!-i\delta_3\!\left(\!-\Delta\pm\frac{\varphi}{\delta_3\tau}\!\right)\!\right)\!\exp\!\left(\!-i\delta_2\!\left(-\Delta+\frac{\tau}{\delta_2}P\!\right)\!\right)\!\exp\!\left(\!-i\delta_1\!\left(\!-\Delta\mp\frac{\varphi}{\delta_1\tau}\!\right)\!\right)\!\psi_0 \\
%&=\lim_{\tau\to 0}\exp\left(\mp i \frac{\varphi}{\tau}\right)\exp(-i\tau P)\exp\left(\pm i \frac{\varphi}{\tau}\right)\psi_0\\
%&=\exp(\pm i P\varphi)\psi_0.
%\end{align*}
\end{itemize}
\end{proposition}
\begin{proof}
To prove (i) and (ii), we first notice that, for any $\psi_0\in H^2(\mathbb{R}^N,\mathbb{C})$, the following limits hold w.r.t. the $L^2$-norm:
\begin{align*}
&\lim_{\delta\to 0^+}-i\delta\left(-\Delta\mp\frac{\varphi}{\delta\tau}\right)\psi_0=\pm i \frac{\varphi}{\tau}\psi_0,\\
&\lim_{\delta\to 0^+}-i\delta\left(-\Delta+\frac{\tau}{\delta}P_j\right)\psi_0=- i \tau P_j\psi_0.
\end{align*}
Since $-i\delta\left(-\Delta\mp\frac{\varphi}{\delta\tau}\right)$ and $-i\delta\left(-\Delta+\frac{\tau}{\delta}P_j\right)$ are skew-adjoint operators with domain $H^2(\mathbb{R}^N,\mathbb{C})$ (independent of $\delta$), a standard application of the Trotter Theorem \cite[Theorem VIII.21]{rs1} concludes the proof.

In order to prove (iii), we start by claiming that
$$\exp\left(\mp i \frac{\varphi}{\tau}\right)\exp(-i\tau P_j)\exp\left(\pm i \frac{\varphi}{\tau}\right)\psi_0=\exp\left(\exp\left(\mp i \frac{\varphi}{\tau}\right)(-i\tau P_j)\exp\left(\pm i \frac{\varphi}{\tau}\right)\right)\psi_0. $$
%thanks to (i) and (ii), we are left to prove that for any $\psi_0\in L^2(\mathbb{R},\mathbb{C})$ the following limits hold w.r.t. the $L^2$-norm:
%$$\lim_{\tau\to 0}\exp\left(\mp i \frac{\varphi}{\tau}\right)\exp(-i\tau P)\exp\left(\pm i \frac{\varphi}{\tau}\right)\psi_0=\exp(\pm i P\varphi)\psi_0. $$
%In order to compute the latter limit, we first claim that
%$$\exp\left(\mp i \frac{\varphi}{\tau}\right)\exp(-i\tau P)\exp\left(\pm i \frac{\varphi}{\tau}\right)\psi_0=\exp\left(\exp\left(\mp i \frac{\varphi}{\tau}\right)(-i\tau P)\exp\left(\pm i \frac{\varphi}{\tau}\right)\right)\psi_0. $$
To prove this property, define the following self-adjoint operator with domain \\$H^1(\mathbb{R}^N,\mathbb{C})$
\begin{equation}\label{eq:claim}
L^\pm_\tau:= \exp\left(\mp i \frac{\varphi}{\tau}\right)P_j\exp\left(\pm i \frac{\varphi}{\tau}\right).
\end{equation}
Consider then its associated unitary group applied to the initial condition
\begin{equation}\label{eq:1}
\psi(t)=\exp(-itL^\pm_\tau)\psi_0,
\end{equation}
which solves 
$$i\frac{d}{dt}\psi(t)=L^\pm_\tau\psi(t),\quad \psi(0)=\psi_0. $$
Then, we make a unitary transformation
$$\Psi(t):=\exp\left(\pm i\frac{\varphi}{\tau}\right)\psi(t), $$
and the transformed state solves
$$i\frac{d}{dt}\Psi(t)=P_j\,\Psi(t),\quad  \Psi(0)=\exp\left(\pm i\frac{\varphi}{\tau}\right)\psi_0.$$
Then, necessarily,
$$\Psi(t)=\exp(-itP_j)\exp\left(\pm i\frac{\varphi}{\tau}\right)\psi_0,$$
which implies
\begin{equation}\label{eq:2}
 \psi(t)=\exp\left(\mp i \frac{\varphi}{\tau}\right)\exp(-it P_j)\exp\left(\pm i \frac{\varphi}{\tau}\right)\psi_0.
 \end{equation}
 By putting together \eqref{eq:1} and \eqref{eq:2} computed at time $t=\tau$, the claim \eqref{eq:claim} is proved. Now, from the computation
 $$\exp\left(\mp i \frac{\varphi}{\tau}\right)(-i\tau P_j)\exp\left(\pm i \frac{\varphi}{\tau}\right)\psi_0=\pm (P_j\varphi)\psi_0-i\tau P_j\psi_0, $$
 we obtain the following limit w.r.t. the $L^2$-norm for any $\psi_0\in H^1(\mathbb{R}^N,\mathbb{C})$:
 $$\lim_{\tau\to 0}\exp\left(\mp i \frac{\varphi}{\tau}\right)(-i\tau P_j)\exp\left(\pm i \frac{\varphi}{\tau}\right)\psi_0=\pm (P_j\varphi)\psi_0.   $$
 Since $\exp\left(\mp i \frac{\varphi}{\tau}\right)(-i\tau P_j)\exp\left(\pm i \frac{\varphi}{\tau}\right)$ is a skew-adjoint operator with domain \\$H^1(\mathbb{R}^N,\mathbb{C})$ (independent of $\tau$), a standard application of the Trotter Theorem \cite[Theorem VIII.21]{rs1} concludes the proof of (iii).
\end{proof}

\begin{remark}
One may also prove Proposition \ref{lemma:main-tool} (iii) using a version of the Baker-Campbell-Hausdorff formula in infinite dimensions. Indeed, being $\varphi$ bounded, for any $\psi_0\in H^1(\mathbb{R}^N,\mathbb{C})$ we can write
$$\exp\left(\mp i \frac{\varphi}{\tau}\right)(-i\tau P_j)\exp\left(\pm i \frac{\varphi}{\tau}\right)\psi_0=\left(\sum_{k=0}^\infty \frac{(\pm i)^k(-i)\tau^{1-k}}{k!}{\rm ad}^k_\varphi P_j\right)\psi_0, $$
where the adjoint action is defined as ${\rm ad}^0_{\varphi}(P_j)=P_j$ and ${\rm ad}^k_{\varphi}(P_j)=[{\rm ad}^{k-1}_{\varphi}(P_j),\varphi]$ for every $k\geq1$. Notice that ${\rm ad}^k_{\varphi}(P_j)=0$ for $k>1$, we get
$$\exp\left(\mp i \frac{\varphi}{\tau}\right)(-i\tau P_j)\exp\left(\pm i \frac{\varphi}{\tau}\right)\psi_0=-i\tau P_j\psi_0\pm[P_j,\varphi]\psi_0\to \pm[P_j,\varphi]\psi_0,\quad \tau\to 0, $$
w.r.t. the $L^2$-norm. Using now \eqref{eq:claim} and the Trotter Theorem \cite[Theorem VIII.21]{rs1}, we see that for any $\psi_0\in L^2(\mathbb{R}^N,\mathbb{C})$
$$\lim_{\tau\to 0}\exp\left(\mp i \frac{\varphi}{\tau}\right)\exp(-i\tau P_j)\exp\left(\pm i \frac{\varphi}{\tau}\right)\psi_0=\exp(\pm[P_j,\varphi])\psi_0=\exp(\pm P_j\varphi)\psi_0, $$
w.r.t. the $L^2$-norm.
\end{remark}

Now, we show that the limits (i) and (ii) of Proposition \ref{lemma:main-tool} can also be ensured w.r.t. the stronger norm of $H^s$ when we consider smoother initial states.

\begin{lemma}\label{lemma:main-tool-bis}
Let $\psi_0\in H^s(\mathbb{R}^{N},\mathbb{C})$ with $s\in\N^*$. For any $\tau\in\mathbb{R}$ and $\delta>0$, the following limits hold w.r.t. the $H^s$-norm:
\begin{itemize}
\item[(i)]$$\lim_{\delta\to 0^+}\exp\left(-i\delta\left(-\Delta\mp\frac{\varphi}{\delta\tau}\right)\right)\psi_0=\exp\left(\pm i \frac{\varphi}{\tau}\right)\psi_0;$$
\item[(ii)]$$\lim_{\delta\to 0^+}\exp\left(-i\delta\left(-\Delta+\frac{\tau}{\delta}P_j\right)\right)\psi_0=\exp\left(- i \tau P_j\right)\psi_0,\quad j=1,\dots,N.$$
\end{itemize}
\end{lemma}
\begin{proof}
Let $Q_0=h_{\vec{0}}$ and $Q_l=P_l$ for $l\geq 1$. We want to prove that the following limit:
\begin{align*}\lim_{\delta\to 0^+}\left\|\exp\left(-i {\delta } \left(-\Delta+\frac{u}{{ \delta}} Q_j\right)\right)\psi_0-\exp(-i u Q_j)\psi_0 \right\|_{H^s(\R^N)}=0,\end{align*}
where $u=\tau$ or $u={\mp 1}/{\tau}$ according to the situation. Notice that%, when $2s-s_1>0$,
%$$\|(-\Delta)^\frac{s}{2}\, \cdot\|_{L^2}^2=\la (-\Delta)^\frac{s_1}{2}\, \cdot,(-\Delta)^\frac{2s-s_1}{2}\, \cdot\ra\leq \|(-\Delta)^\frac{s_1}{2}\, \cdot\|_{L^2} \|(-\Delta)^\frac{2s-s_1}{2}\, \cdot\|_{L^2}$$
%and, if $4s-3s_1>0$, then we can repeat the same apporach and write
%$$\|(-\Delta)^\frac{2s-s_1}{2}\, \cdot\|^2=\la (-\Delta)^\frac{s_1}{2}\, \cdot,(-\Delta)^\frac{4s-3s_1}{2}\, \cdot\ra\leq \|(-\Delta)^\frac{s_1}{2}\, \cdot\|_{L^2} \|(-\Delta)^\frac{4s-3s_1}{2}\, \cdot\|_{L^2}.$$
%In general, when $4ns-(4n-1)s_1>0$ with $n\in\N^*$, we have
%$$\|(-\Delta)^\frac{2ns-(2n-1)s_1}{2}\, \cdot\|^2\leq \|(-\Delta)^\frac{s_1}{2}\, \cdot\|_{L^2} \|(-\Delta)^\frac{4ns-(4n-1)s_1}{2}\, %\cdot\|_{L^2}.$$

$$\|(-\Delta)^\frac{s}{2}\, \cdot\|_{L^2(\R^N)}^2=\la (-\Delta)^\frac{s}{2}\, \cdot,(-\Delta)^\frac{ s }{2}\, \cdot\ra_{L^2(\R^N)}\leq \|(-\Delta)^{s }\, \cdot\|_{L^2(\R^N)}\ \ \| \cdot\|_{L^2(\R^N)}.$$
%By using together the previous relations, we can show that there exist $r_1, r_2>0$ such that $$\|(-\Delta)^\frac{s}{2}\, \cdot\|_{L^2}\leq  \|(-\Delta)^\frac{s_1}{2}\, \cdot\|_{L^2}^{r_1}\ \ \|\cdot\|_{L^2}^{r_2}$$ and, t
Then, there exists $C>0$ such that
\begin{align}\label{interpolation:1}\|\cdot\|_{H^{s}(\R^N)}^2\leq C (\|\cdot\|_{H^{2s}(\R^N)}\|\cdot\|_{L^2(\R^N)}+ \|\cdot\|_{L^2(\R^N)}^2).\end{align}
Assume $\psi_0\in H^{2s}(\R^N,C)$. Thanks to Proposition \ref{lemma:main-tool}, we have 
$$\lim_{\delta\to 0^+} \left\|\exp\left(-i {\delta } \left(-\Delta+\frac{u}{{ \delta }} Q_j\right)\right)\psi_0-\exp(-i u Q_j)\psi_0\right\|_{L^2(\R^N)}=0.$$
If now we set $v^\delta= u/\delta$ and $T=\delta$, we notice that the $L^1-$norm of $v^\delta$ is uniformly bounded w.r.t. $\delta$. The relation \eqref{propreg} showed in Proposition \ref{prop:well} (with $\kappa=0$) implies the existence of $ C>0$ (not depending on $\delta$) such that
%  the  propagator $\exp\left(-i t \left(-\Delta+v Q_j\right)\right)$ and the operator $\exp(-i uQ_j)$ are unitary w.r.t. the $H^s-$norm. %$\|(-\Delta)^{s}\cdot\|_{L^2}+\|\cdot\|_{L^2}$ when $\kappa=0$.
%Then, there exists $C>0$ 
\begin{align*}&\left\|\exp\left(-i {\delta } \left(-\Delta+\frac{u}{{ \delta }} Q_j\right)\right)\psi_0-\exp(-i u Q_j)\psi_0\right\|_{H^{2s}(\R^N)}\leq %\left\|\exp\left(-i {\delta } \left(-\Delta+\frac{u}{{ \delta }} Q_j\right)\right)\psi_0\right\|_{H^{2s}}\\
%&+\|\exp(-i uQ_j)\psi_0\|_{H^{2s}(\R^N,\C)} \leq 
C\big(\|\psi_0\|_{H^{2s}(\R^N)}+1\big).\end{align*}
Finally, thanks to the interpolation properties \eqref{interpolation:1}, there exists $C>0$ such that
\begin{align*}&\lim_{\delta\to 0^+} \left\|\exp\left(-i {\delta } \left(-\Delta+\frac{u}{{ \delta }} Q_j\right)\right)\psi_0-\exp(-i u Q_j)\psi_0\right\|_{H^{2s}(\R^N)}\\
&\leq \lim_{\delta\to 0^+} C
%&\leq \lim_{\delta\to 0^+} \|\mathcal{R}(\delta,\psi_0,e_j u/\delta)-\exp(-i uQ_j)\psi_0\|_{H^{s_1}(\R^N,\C)}^{r_1} \|\mathcal{R}(\delta,\psi_0,e_j u/\delta)-\exp(-i uQ_j)\psi_0\|_{L^2(\R^N,\C)}^{r_2}\\
\left(\big(\|\psi_0\|_{H^{2s}(\R^N)}+1\big)\left\|\exp\left(-i {\delta } \left(-\Delta+\frac{u}{{ \delta }} Q_j\right)\right)\psi_0-\exp(-i u Q_j)\psi_0\right\|_{L^2(\R^N)}\right.\\
&+\left.\left\|\exp\left(-i {\delta } \left(-\Delta+\frac{u}{{ \delta }} Q_j\right)\right)\psi_0-\exp(-i u Q_j)\psi_0\right\|_{L^2(\R^N)}^2\right)=0.
\end{align*}
The last relation yields that the statement is proved for every $\psi_0\in H^{2s}(\R^N,\C)$. By density and again thanks to the unitarity of the linear propagator, we can conclude the result also for $\psi_0\in H^{s}(\R^N,\C)$.
\end{proof}
We shall also need a more general version of Lemma \ref{lemma:main-tool-bis}.

\begin{lemma}\label{coro:main-tool-bis}
Let $\psi_0\in H^s(\mathbb{R}^{N},\mathbb{C})$ with $s\in\N^*$. Let $(\delta_n)_{n\in\N^*}$ be a sequence of positive numbers such that $\delta_n\xrightarrow{n\rightarrow +\infty} 0 $ and $(u_n)_{n\in\N^*}\in \ell^\infty(\R)$. The following limit
holds \begin{align*}\lim_{n \to \infty}\left\|\exp\left(-i { \delta_n} \left(-\Delta+\frac{u_n}{{  \delta_n}} Q_j\right)\right)\psi_0-\exp(-i u_n Q_j)\psi_0 \right\|_{H^s(\mathbb{R}^N)}=0,\end{align*}
where $Q_0=h_{\vec{0}}$ and $Q_l=P_l$ for $l\geq 1$.
\end{lemma}

%\begin{corollary}\label{coro:main-tool-bis}
%Let $\psi_0\in H^s(\mathbb{R}^{N},\mathbb{C})$ with $s>0$ and $\varphi\in W^{s,\infty}(\mathbb{R}^{N},\mathbb{R})$. Let $(\delta_n)_{n\in\N^*}$ be a sequence of positive numbers such that $\delta_n\xrightarrow{n\rightarrow +\infty} 0 $ and $f$ be a uniformly bounded function. For any $\tau\in\mathbb{R}$, the following limits hold w.r.t. the $H^s$-norm:
%\begin{itemize}
%\item[(i)]$$\lim_{n\to +\infty}\exp\left(-i\delta_n f(\delta_n) \left(-\Delta\mp\frac{\varphi}{\delta_n\tau}\right)\right)\psi_0=\exp\left(\pm i \frac{\varphi}{\tau}\right)\psi_0;$$
%\item[(ii)]$$\lim_{n\to +\infty}\exp\left(-i\delta_n f(\delta_n) \left(-\Delta+\frac{\tau}{\delta_n}P\right)\right)\psi_0=\exp\left(- i \tau P\right)\psi_0;$$
%\end{itemize}
%\end{corollary}
\begin{proof}
%The result is proved w.r.t. the $L^2-$norm by using the techniques adopted in the proof of Lemma \ref{lemma:main-tool}. The arguments leading to Lemma \ref{lemma:main-tool-bis} ensure the property w.r.t. the $H^s-$norm. 
Let $\psi_0\in H^2(\R^N,\C)$, and for any $u\in\R$ denote 
$$\Psi(u):=\exp\left(iu\left(Q_j+\frac{\delta_n}{u_n}\Delta\right)\right)\psi_0\in H^2(\R^N,\C).$$
Using the Duhamel formula, we write
\begin{align*}
\exp\left(-i { \delta_n} \left(-\Delta+\frac{u}{{  \delta_n}} Q_j\right)\right)\psi_0\Big|_{u=u_n}&=\exp\left(iu\left(Q_j+\frac{\delta_n}{u_n}\Delta\right)\right)\psi_0\Big|_{u=u_n}\\
&=e^{iuQ_j}\psi_0+\int_0^ue^{i(u-v)Q_j}\frac{\delta_n}{u_n}\Delta\Psi(v)dv\Big|_{u=u_n}.
\end{align*}
Hence, using the Cauchy-Schwartz inequality, we have
\begin{align*}
&\left\|\exp\left(-i { \delta_n} \left(-\Delta+\frac{u_n}{{  \delta_n}} Q_j\right)\right)\psi_0-\exp(-i u_n Q_j)\psi_0\right\|^2_{L^2(\R^N)}\\
&\leq \delta_n\int_0^{u_n}\|\Psi(v)\|^2_{H^2(\R^N)}dv\to 0, \quad n\to \infty.
\end{align*}
By density, we can conclude the same result for $\psi_0\in L^2(\R^N)$. The result can then be extended to hold w.r.t. the $H^s$-norm exactly as it is done in the proof of Lemma \ref{lemma:main-tool-bis} using the interpolation inequality \eqref{interpolation:1}.
\end{proof}


We are finally ready to generalize the validity of the limits (i) and (ii) of Lemma \ref{lemma:main-tool-bis} in presence of state nonlinearity by treating it as a perturbation.

\begin{proposition}\label{lemma:limit-nonlinear}
Let $s\in\N^*$ be such that $s>N/2$, and $\psi_0\in H^s(\mathbb{R}^N)$. Then, for any $u\in\mathbb{R}$, there exists $\tau_0>0$ such that, for every $0<\tau<\tau_0$, the solution $\mathcal{R}\big(t,\psi_0,e_j \frac{u}{\tau}\big)$ is well-defined in $[0,\tau]$ and the following limit holds in $H^s(\mathbb{R}^N)$:
\begin{align}\label{nonlin}\lim_{\tau\to 0^+}\mathcal{R}\Big(\tau,\psi_0,e_j \frac{u}{\tau}\Big)=\exp(-i uQ_j)\psi_0,\end{align}
where $\{e_j\}_{j=0,.., N}$ is the standard basis of $\mathbb{R}^{N+1}$, $Q_0=h_{\vec{0}}$ and $Q_l=P_l$ for $l\geq 1$.
\end{proposition}
\begin{proof}
Let $\delta>0$ and $j\geq 1.$ Thanks to the well-posedness stated in Proposition \ref{prop:well}, the solution $\psi(t)=\mathcal{R}\big(t,\psi_0,e_j \frac{u}{\tau}Q_j\big)$ is well-defined in $[0,T^\delta)$ with $T^\delta>0$. We denote $$\varphi(t)=\exp(-i t uQ_j )\psi_0 \ \ \ \ \ \text{and}\ \ \  \ \ \phi(t)=\psi(\delta t)-\varphi(t),$$ which is well-defined for $t<\delta^{-1} T^\delta $. We want to prove the existence of $\delta_0>0$ such that, for every $0<\delta <\delta_0 $, we have $\delta^{-1} T^\delta>1$ and finally ensure the limit 
 $$\|\phi(t=1)\|^2_{H^s(\mathbb{R}^N)}\xrightarrow{\delta\rightarrow 0^+} 0.$$ By the Duhamel formula, we can write 
$$\phi(t) =\left(\exp\left(-i \delta t \left(-\Delta+\frac{u}{\delta} Q_j\right)\right)-\exp(-i t uQ_j)\right)\psi_0 -i\int_0^{\delta t} e^{i(\delta t-s)\Delta}|\psi|^{2p}\psi ds$$
for $j\geq 0$. 
Thanks to the Cauchy-Schwartz inequality and the unitarity of the group $e^{it\Delta}$, there exists $C>0$ such that
\begin{equation}\begin{split}\label{limit1}
    \left\|(-\Delta)^\frac{s}{2}\!\!\! \int_0^{\delta t} \!\!\!\!e^{i({\delta t}-s)\Delta}|\psi|^{2p}\psi ds \right\|^2_{L^2(\mathbb{R}^N)}\!\!\!\!\!\!\!\!&= 
    \int_{\R^N}\left|(-\Delta)^\frac{s}{2} \int_0^{\delta t} e^{i({\delta t}-s)\Delta}|\psi|^{2p}\psi ds\right|^2dx\\
    &\leq \int_{\R^N}{\delta t}\int_0^{\delta t}   \left|(-\Delta)^\frac{s}{2}e^{i({\delta t}-s)\Delta}|\psi|^{2p}\psi\right|^2 ds\,dx\\
    &= {\delta t}\int_0^{\delta t}   \left\|e^{i({\delta t}-s)\Delta}(-\Delta)^\frac{s}{2}|\psi|^{2p}\psi\right\|^2_{L^2(\mathbb{R}^N)}ds \\
    &= {\delta t} \int_0^{\delta t} \|(-\Delta)^\frac{s}{2} |\psi|^{2p}\psi\,\|^2_{L^2(\mathbb{R}^N)}ds\\
    &\leq {\delta t} \int_0^{\delta t} \|\psi^{2p+1}\|^{2}_{H^s(\mathbb{R}^N)}ds\leq C{\delta t} \int_0^{\delta t} \|\psi\|^{4p+2}_{H^s(\mathbb{R}^N)}ds.\\
        \end{split}
\end{equation}
Thanks to \eqref{limit1}, there exists $C>0$ such that
%using the continuity of the Sobolev embedding $H^s(\mathbb{R}^N)\subset L^q(\mathbb{R}^N)$ whenever $s\geq N/2$ and $2\leq q<\infty$, there exists a constant $C>0$ such that
\begin{equation*}\begin{split}\label{limit2}
    \left\|\int_0^{\delta t} e^{i({\delta t}-s)\Delta}|\psi|^{2p}\psi ds \right\|^2_{H^s(\mathbb{R}^N)}%&={\delta t}\int_0^{\delta t} \|\psi\|_{L^{4p+2}(\mathbb{R}^N)}^{4p+2}ds\leq C {\delta t}\int_0^{\delta t}  \|\psi\|_{H^s(\mathbb{R}^N)}^{4p+2}ds\\
 %   \end{split}
%\end{equation}
%Assume $\delta\leq\delta_0\leq 1$ and there exists $C>0$
%\begin{equation}\begin{split}\label{limit3}
   % \left\|\int_0^{\delta t} e^{i({\delta t}-s)\Delta}|\psi|^{2p}\psi ds \right\|^2_{L^2(\mathbb{R}^N)}
   &\leq C {\delta t}  \left(\int_0^{\delta t}\|\varphi\|_{H^s(\mathbb{R}^N)}^{4p+2}ds+\int_0^{\delta t} \|\phi\|_{H^s(\mathbb{R}^N)}^{4p+2}ds\right).
    \end{split}
\end{equation*}
Now, for $C>0$, we have
\begin{equation}\begin{split}\label{limit4}
\|\phi(t)\|_{H^s(\mathbb{R}^N)}^2&\leq C\left(\left\|\exp\left(-i \delta t \left(-\Delta+\frac{u}{\delta} Q_j\right)\right)\psi_0-\exp(-i t uQ_j)\psi_0 \right\|_{H^s(\mathbb{R}^N)}^2\right.\\
&\left.+  {\delta t}\int_0^{\delta t} \|\varphi\|_{H^s(\mathbb{R}^N)}^{4p+2}ds  +{\delta t}\int_0^{\delta t} \|\phi\|_{H^s(\mathbb{R}^N)}^{4p+2}ds\right).
 \end{split}
\end{equation}
Denote $\varepsilon^\delta:=\sup\{t<\delta^{-1} T^\delta:\ \|\phi(t)\|_{H^s(\R^N)}<1 \}$. To prove the existence of $\delta_0>0$ such that $\delta^{-1} T^\delta>1$ for every $0<\delta <\delta_0 $, we show that $\varepsilon^\delta>1$ for $0<\delta <\delta_0 $ and $\delta_0$ small. If $\|\phi(t)\|_{H^s(\R^N)}<1$ for every $t>0$, the results is obviously proved. Assume the contrary and proceed by contradiction. We assume the existence of at least a sequence of positive numbers $\delta_n\xrightarrow{n\rightarrow +\infty} 0 $ such that $\varepsilon^{\delta_n}\leq 1$ for every $n\in\N^*$. Thanks to \eqref{limit4} and since $\|\phi(t)\|_{H^s(\mathbb{R}^N)}<1$ in $[0,\varepsilon^{\delta_n})$, we have
\begin{equation}\begin{split}\label{limit5}
1\!\!=\!\! \|\phi(\varepsilon^{\delta_n})\!\!\!\|_{H^s(\mathbb{R}^N)}^2&\!\!<\!\!C\!\!\left(\left\|\exp\left(-i \delta_n \varepsilon^{\delta_n}\left(-\Delta+\frac{u}{{\delta_n}} Q_j\right)\right)\psi_0\!\!-\!\!\exp(-i {\varepsilon^{\delta_n}} uQ_j)\psi_0 \right\|_{H^s(\mathbb{R}^N)}^2\right.\\
&\left.+  {({\delta_n} {\varepsilon^{\delta_n}})^2 } (\|\varphi\|_{H^s(\mathbb{R}^N)}^{4p+2}  +1)\right).
 \end{split}
\end{equation}
However, if we set $\tilde \delta_n:={\delta_n} \varepsilon^{\delta_n}$ and $u_n=\varepsilon^{\delta_n} u$, then $\tilde\delta_n\xrightarrow{n\rightarrow +\infty} 0$ and $(u_n)_{n\in\N^*}\in\ell^\infty(\R)$. Thanks to Lemma \ref{coro:main-tool-bis}, we have that, for $n$ sufficiently large,

$$C\left\|\exp\left(-i {\tilde \delta_n} \left(-\Delta+\frac{u_n}{{\tilde \delta_n}} Q_j\right)\right)\psi_0-\exp(-i u_n Q_j)\psi_0 \right\|_{H^s(\mathbb{R}^N)}^2 < \frac{1}{2}$$
and also, 
\begin{align*}
&C\left(\left\|\exp\left(-i {\tilde \delta_n} \left(-\Delta+\frac{u_n}{\tilde \delta_n} Q_j\right)\right)\psi_0-\exp(-i u_nQ_j)\psi_0 \right\|_{H^s(\mathbb{R}^N)}^2 \!\!\!\!\!+ {\tilde \delta_n^2 } (\|\varphi\|_{H^s(\mathbb{R}^N)}^{4p+2}  +1)\right)\\
& < 1.
\end{align*}
The last identity contradicts \eqref{limit5} and then $\varepsilon^\delta>1$ for $0<\delta <\delta_0 $ and $\delta_0$ small enough.
Finally, the result is proved since $1\in [0,\delta^{-1}T^\delta)$ and, thanks to Lemma \ref{lemma:main-tool-bis}, the following limit is verified
\begin{equation*}\begin{split}
\|\phi(1)\|_{H^s(\mathbb{R}^N)}^2&\leq C\left(\left\|\exp\left(-i \delta \left(-\Delta+\frac{u}{\delta} Q_j\right)\right)\psi_0-\exp(-i uQ_j)\psi_0 \right\|_{H^s(\mathbb{R}^N)}^2\right.\\
&\left.+  {\delta^2 } (\|\varphi\|_{H^s(\mathbb{R}^N)}^{4p+2}  +1)\right)\xrightarrow{\delta\rightarrow 0^+} 0.
 \end{split}
\end{equation*}
\end{proof}









\section{Saturation}\label{sec:saturation} Recall that the 1D Hermite functions are defined for any $n\in\mathbb{N}$ as
\begin{equation}\label{def:hermite}
h_n(x)=(-1)^n(2^nn!\sqrt{\pi})^{-1/2}e^{x^2/2}\frac{d^n}{dx^n}e^{-x^2}.
\end{equation}
%They form an (orthonormal) Hilbert basis of $L^2(\mathbb{R},\mathbb{R})$:
%\begin{equation}\label{eq:density}
%\overline{\rm span}_\mathbb{R}\{h_n,n\in\mathbb{N}\}=L^2(\mathbb{R},\mathbb{R}), 
%\end{equation}
%where the closure is taken w.r.t. the $L^2$-norm.
In $N$ dimensions, we consider the tensor products of 1-D Hermite functions:
$$h_{n_1,\dots,n_N}(x_1,\dots,x_N)=h_{n_1}(x_1)\dots h_{n_N}(x_N).$$
It is well-known that the Hermite functions
form an (orthonormal) Hilbert basis of $L^2(\mathbb{R}^N,\mathbb{R})$. Moreover, we have the following.
\begin{lemma}\label{lem:density2}For any $s\geq 0$, one has that 
$$
\overline{\rm span}_\mathbb{R}\{h_{n_1,\dots,n_N},n_1,\dots,n_N\in\mathbb{N}\}=H^s(\mathbb{R}^N,\mathbb{R}),$$
 where the closure is taken w.r.t. the $H^s$-norm.
\end{lemma}
\begin{proof}
Let $f\in H^s(\R^N,\R)$ be such that 
$$\langle(1+(-\Delta)^{s/2})f,(1+(-\Delta)^{s/2})h_{\vec{n}} \rangle_{L^2}=0,\quad \forall n\in\N.$$ 
We prove the statement by showing that, necessarily, $f=0$. Denote by $\mathcal{F}$ the Fourier transform: being an isometry in $L^2$, we have that 
$$\langle \mathcal{F}[(1+\Delta^{s/2})f],\mathcal{F}[(1+\Delta^{s/2})h_{\vec{n}}]\rangle_{L^2}=0,$$
We then compute
\begin{align*} \langle \mathcal{F}[(1+(-\Delta)^{s/2})f],\mathcal{F}[(1+(-\Delta)^{s/2})h_{\vec{n}}]\rangle_{L^2}&=\langle(1+|\lambda|^{s/2})\mathcal{F}[f] ,(1+|\lambda|^{s/2})\mathcal{F}[h_{\vec{n}}] \rangle_{L^2} \\
&=\langle(1+|\lambda|^{s/2})\widehat{f} ,(1+|\lambda|^{s/2})h_{\vec{n}} \rangle_{L^2},
\end{align*}
where we used that $\mathcal{F}[h_{\vec{n}}]=h_{\vec{n}}$. We then have that $\mathcal{F}[f]$ is in the orthogonal complement of ${\rm span}_\mathbb{R}\{h_{\vec{n}},\vec{n}\in\mathbb{N}^N\}$ w.r.t. the $L^2$-scalar product associated with the measure $(1+|\lambda|^{s/2})^2d\lambda$. Since the set of linear combinations of Hermite functions is dense in the $L^2$-space (also w.r.t. the weight $(1+|\lambda|^{s/2})^2$), we have that $\mathcal{F}[f]=0$. Hence, we conclude that $f=0$.
\end{proof}


 We introduce now an increasing sequence of vector subspaces of $L^2(\mathbb{R}^N,i\mathbb{R})$: define
$$\mathcal{H}_0={\rm span}_\mathbb{R}\{ih_{0,\dots,0}\}, $$
and then iteratively, for $j\geq 1$, $\mathcal{H}_j$ as the largest vector space whose elements can be written as
$$\phi_0+i\sum_{k=1}^N P_k\phi_k,\quad \phi_0,\phi_k\in\mathcal{H}_{j-1}. $$
Finally, we define the saturation space as $\mathcal{H}_\infty=\cup_{j=0}^\infty \mathcal{H}_j$.
\begin{lemma}\label{lem:density}
The vector space $\mathcal{H}_\infty$ is dense in $H^s(\mathbb{R}^N,i\mathbb{R}), s\geq 0$.
\end{lemma}
\begin{proof}
We prove it for $N=1$ and the extension to higher dimensions is straightforward thanks to the tensor product structure. Thanks to Lemma \ref{lem:density2}, it is enough to show that 
\begin{equation}\label{eq:Hinfty}
\overline{\rm span}_{\mathbb{R}}\{ih_n,n\in\mathbb{N}\}\subset \mathcal{H}_\infty.
\end{equation}
From the recurrence relations
$$Ph_0=-\frac{i}{\sqrt{2}}h_1,\quad Ph_n=\sqrt{\frac{n}{2}}ih_{n-1}-\sqrt{\frac{n+1}{2}}ih_{n+1},\quad n\geq 1, $$
which can be derived straightforwardly by induction (using the definition \eqref{def:hermite}), we have
$$ih_1=iP(\sqrt{2}ih_0)\in\mathcal{H}_1,\quad ih_{n+1}=\sqrt{\frac{n}{n+1}}ih_{n-1}+iP\left(\sqrt{\frac{2}{n+1}}ih_n\right)\in\mathcal{H}_{n+1}, n\geq1, $$
and the conclusion follows.
\end{proof}

%\begin{lemma}\label{lem:density-bis}
%The vector space $\mathcal{H}_\infty$ is dense in $H^s(\mathbb{R}^N,i\mathbb{R})$ for every $s\in\N^*$.
%\end{lemma}
%\begin{proof}
%The proof is a simple consequence of the properties developed in the proof of Lemma \ref{lem:density}. Indeed, any derivative of each Hermite function is a linear combination of a finite number of Hermite functions which yields to
%$$\overline{\rm span}_\mathbb{R}\{h_{n_1,\dots,n_N},n_1,\dots,n_N\in\mathbb{N}\}=H^s(\mathbb{R}^N,\mathbb{R}),$$
%where the closure is taken w.r.t. the $H^s$-norm. Finally, the proof of Lemma \ref{lem:density} yields that $\mathcal{H}_{\infty}$ contains all the Hermite functions which concludes the proof.
%\end{proof}


\section{Proof of Theorem \ref{thm:main-result}}\label{sec:proof}
We follow and readapt the proofs of \cite[Theorem 2.2]{duca-nersesyan} and \cite[Lemma 7]{small-time-wave}. Let us start by proving the following property by induction on $n\in\mathbb{N}$.
\begin{itemize}

\smallskip

\item[($P_n$)] Let $\psi_0\in H^s(\mathbb{R}^N,\mathbb{C})$ with $s\in\N^*$ such that $s>N/2$ and $\phi\in\mathcal{H}_n$. There exists $C>0$ such that, for any $\varepsilon,T>0$, there exist $\tau\in[0,T)$ and $(u_0,u):[0,\tau]\to \mathbb{R}^{N+1}$ piecewise constant such that $$\|(u_0,u)\|_{L^1((0,\tau),\R^{N+1})}\leq C$$ and the solution $\psi(t;\psi_0)$ of \eqref{eq:schro} associated with the control $(u_0,u)$ and with the initial condition $\psi_0$ satisfies
\begin{align}\label{Pn}\|\psi(\tau;\psi_0)-e^{\phi}\psi_0\|_{L^2(\mathbb{R}^N)}< \varepsilon. \end{align}
\end{itemize}

\textbf{Basis of induction: validity of $(P_n)$ for $n=0$}\\
If $\phi\in\mathcal{H}_0$, there exists $\alpha\in\mathbb{R}$ such that $\phi(x)=i\alpha h_{0,\dots,0}(x)$. Consider then the solution $\mathcal{R}(t,\psi_0,(-\alpha/\delta,0))$ of \eqref{eq:schro} associated with the constant control $(u_0,u)^{\delta,\alpha}:=(-\alpha/\delta,0)\in\mathbb{R}^{N+1}$ and with the initial condition $\psi_0$.
%$$
%\mathcal{R}(t,\psi_0,(0,-\alpha/\delta))=\exp\left(-it \left(-\Delta-\frac{\alpha}{\delta}h_0\right)\right)\psi_0.$$
Applying Proposition \ref{lemma:limit-nonlinear}, we find $\delta\in[0,T)$ such that
$$\|\mathcal{R}(\delta,\psi_0,(-\alpha/\delta,0))-\exp(i\alpha h_{0,\dots,0})\psi_0\|_{L^2(\R^N)}<\varepsilon, $$ 
%the solution computed at time $\tau$ is arbitrarily close to the state 
%$$\begin{pmatrix}
%w_0\\
%\dot{w}_0+\phi w_0
%\end{pmatrix}=e^{\phi\mathcal{B}}\begin{pmatrix}
%w_0\\
%\dot{w}_0
%\end{pmatrix}$$ 
%if $\tau$ is small enough (in the above equality we used $\mathcal{B}^2=0$)
which proves \eqref{Pn}.
 The validity of the uniform bound w.r.t. the $L^1-$norm is ensured since the control is defined on the time interval $[0,\delta]$ and is of the order $\frac{1}{\delta}$.\\

\textbf{Inductive step: validity of $(P_n)$ for $n\Rightarrow n+1$}\\
Assuming that $(P_n)$ holds, we prove $(P_{n+1})$. If $\phi\in\mathcal{H}_{n+1}$, there exist $(\phi_j)_{j=0,..,N}\subset\mathcal{H}_n$ such that 
$$\phi=\phi_0+i\sum_{j=1}^N P_j\phi_j.$$ Let us start by considering the term $\phi_1$: thanks to Proposition \ref{lemma:main-tool} (iii), we can fix $\gamma\in[0,T/3)$ small enough such that
$$\left\|\exp\left(\frac{\phi_1}{\gamma}\right)\exp(-i\gamma P_1)\exp\left( -\frac{\phi_1}{\gamma}\right)\psi_0-\exp(iP_1\phi_1)\psi_0\right\|_{L^2(\R^N)}< \varepsilon/2.  $$

Thanks to the inductive hypothesis, for any $\epsilon,T,\gamma>0$, there exist $\delta_1\in[0,T/3)$ and a piecewise constant control $(u_0,u)^{\delta_1,\gamma}:[0,\delta_1]\to \mathbb{R}^{N+1}$ such that 
%the associated solution of \eqref{eq:wave} with initial condition $(w_0,\dot{w}_0)$ satisfies
\begin{equation}\label{eq:first-impulsion}
 \left\|
\mathcal{R}(\delta,\psi_0,(u_0,u)^{\delta_1,\gamma})-\exp\left(-\frac{\phi_1}{\gamma}\right)\psi_0\right\|_{L^2(\R^N)}< \epsilon. 
\end{equation}
% \left\|\begin{pmatrix}
%w(\cdot,\delta)\\
%\frac{\partial}{\partial t}w(\cdot,\delta)
%\end{pmatrix}-e^{\tau^{-1/2}\phi_1\mathcal{B}}\begin{pmatrix}
%w_0\\
%\dot{w}_0
%\end{pmatrix}\right\|_{H^1\times L^2(\mathbb{T}^d)}< \epsilon.  
We now consider a constant control $(u_0,u)^{\delta_2,\gamma}=(0,e_1\gamma/\delta_2):[0,\delta_2]\to\mathbb{R}^{N+1}$: thanks to Proposition \ref{lemma:limit-nonlinear}, we can find $\delta_2\in[0,T/3)$ such that
%Consider now a zero control $0|_{[0,\gamma]}=(0,\dots,0):[0,\gamma]\to\mathbb{R}^{2d+1}$ (that is, a free evolution), applied on a time interval of size $\gamma$: thanks to \eqref{eq:first-impulsion} and the fact that, for any $t\in\mathbb{R}$, $e^{t\mathcal{A}}$ is bounded, there exists $C=C(\gamma)$ such that
% the solution $e^{\tau\mathcal{A}}(w(\cdot,\delta)\frac{\partial}{\partial t}w(\cdot,\delta))$ associated with the concatenation of the two controls $p^{\delta,\tau}$ and $p^0$ satisfies
\begin{align*}
& \left\|
\mathcal{R}(\delta_1+\delta_2,\psi_0,(u_0,u)^{\delta_2,\gamma}*(u_0,u)^{\delta_1,\gamma})-\exp(-i\gamma P_1)\exp\left(-\frac{\phi_1}{\gamma}\right)\psi_0\right\|_{L^2(\R^N)}\\
%=&\left\|
%\exp\left(-i\delta_2\left(-\Delta+\frac{\gamma}{\delta_2}P\right)\right)\mathcal{R}(\delta_1,\psi_0,(u_1,u_2)^{\delta_1,\gamma})-\exp(-i\gamma P)\exp\left(-\frac{\phi_1}{\gamma}\right)\psi_0\right\|_{L^2}\\
\leq &\left\|
\mathcal{R}(\delta_2,\mathcal{R}(\delta_1,\psi_0,(u_0,u)^{\delta_1,\gamma}),(u_0,u)^{\delta_2,\gamma})\!-\!\exp(-i\gamma P_1)\mathcal{R}(\delta_1,\psi_0,(u_0,u)^{\delta_1,\gamma})\right\|_{L^2(\R^N)}\\
+ & \left\|
\exp(-i\gamma P_1)\mathcal{R}(\delta_1,\psi_0,(u_0,u)^{\delta_1,\gamma})-\exp(-i\gamma P_1)\exp\left(-\frac{\phi_1}{\gamma}\right)\psi_0\right\|_{L^2(\R^N)}<2\epsilon.
\end{align*}
%w(\cdot,\delta+\tau)\\
%\frac{\partial}{\partial t}w(\cdot,\delta+\tau)
%\end{pmatrix}-e^{\tau\mathcal{A}}e^{\tau^{-1/2}\phi_1\mathcal{B}}\begin{pmatrix}
%w_0\\
%\dot{w}_0
%\end{pmatrix}\right\|_{H^1\times L^2(\mathbb{T}^d)}< C\epsilon.  
Now, we use again the inductive hypothesis to deduce that there exist $\delta_3\in[0,T/3)$ and a piecewise constant control $(u_0,u)^{\delta_3,\gamma}:[0,\delta_3]\to \mathbb{R}^{2}$ such that
% the associated solution associated with $p^{\delta',\tau}$ and initial condition $e^{\tau\mathcal{A}}e^{\tau^{-1/2}\phi_1\mathcal{B}}\begin{pmatrix}
%w_0\\
%\dot{w}_0
%\end{pmatrix}$ 
%satisfies
\begin{align*}
&\Big\|\mathcal{R}\left(\delta_3,\exp(-i\gamma P_1)\!\exp\!\left(\!\!-\frac{\phi_1}{\gamma}\right)\!\psi_0,(u_0,u)^{\delta_3,\gamma}\right)\!\\
&-\!\exp\!\left(\frac{\phi_1}{\gamma}\right)\!\exp\!(-i\gamma P_1)\!\exp\!\left(-\frac{\phi_1}{\gamma}\right)\!\psi_0\Big\|_{L^2(\R^N)}
<\epsilon. 
\end{align*}
Then, thanks to Proposition \ref{prop:well} (ii), there exists $C>0$ such that
\begin{align*}
& \left\|
\mathcal{R}(\delta_1+\delta_2+\delta_3,\psi_0,(u_0,u)^{\delta_3,\gamma}*(u_0,u)^{\delta_2,\gamma}*(u_0,u)^{\delta_1,\gamma})-\exp(iP_1\phi_1)\psi_0\right\|_{L^2(\R^N)}\\
\leq & \Big\| \mathcal{R}(\delta_3,\mathcal{R}(\delta_1+\delta_2,\psi_0,(u_0,u)^{\delta_2,\gamma}*(u_0,u)^{\delta_1,\gamma}),(u_0,u)^{\delta_3,\gamma})\\
&-\mathcal{R}\left(\delta_3,\exp(-i\gamma P_1)\exp\left(-\frac{\phi_1}{\gamma}\right)\psi_0,(u_0,u)^{\delta_3,\gamma}\right) \Big\|_{L^2(\R^N)}\\
+&\Big\|\mathcal{R}\left(\delta_3,\exp(-i\gamma P_1)\exp\left(-\frac{\phi_1}{\gamma}\right)\psi_0,(u_0,u)^{\delta_3,\gamma}\right)\\
&-\exp\left(\frac{\phi_1}{\gamma}\right)\exp(-i\gamma P_1)\exp\left(-\frac{\phi_1}{\gamma}\right)\psi_0\Big\|_{L^2(\R^N)}\\
+&\left\|\exp\left(\frac{\phi_1}{\gamma}\right)\exp(-i\gamma P_1)\exp\left( -\frac{\phi_1}{\gamma}\right)\psi_0-\exp(iP_1\phi_1)\psi_0\right\|_{L^2(\R^N)}\\
< & C\epsilon+2\epsilon+\varepsilon/2.
\end{align*}



%\begin{equation}\label{eq:final-concatenation}
% \left\|\begin{pmatrix}
%w(\cdot,\delta+\tau+\delta')\\
%\frac{\partial}{\partial t}w(\cdot,\delta+\tau+\delta')
%\end{pmatrix}-e^{-\tau^{-1/2}\phi_1\mathcal{B}}e^{\tau\mathcal{A}}e^{\tau^{-1/2}\phi_1\mathcal{B}}\begin{pmatrix}
%w_0\\
%\dot{w}_0
%\end{pmatrix}\right\|_{H^1\times L^2(\mathbb{T}^d)}< \varepsilon.  
%\end{equation}
Choosing $\epsilon>0$ small enough such that $C\epsilon+2\epsilon<\varepsilon/2$, we have then proved that the piecewise constant control $$(u_0,u):=(u_0,u)^{\delta_3,\gamma}*(u_0,u)^{\delta_2,\gamma}*(u_0,u)^{\delta_1,\gamma}$$ steers $\psi_0$, the initial state, $\varepsilon$-close to the target $\exp(iP_1\phi_1)\psi_0$ in the time $\tau:=\delta_1+\delta_2+\delta_3<T$. 

\smallskip



Now, the $L^1$-norm of $(u_0,u)$ is uniformly bounded: the $L^1-$norms of $(u_0,u)^{\delta_1,\gamma}$ and $(u_0,u)^{\delta_1,\gamma}$ are uniformly bounded thanks to the inductive hypothesis, and the same is true for $(u_0,u)^{\delta_2,\gamma}$ since this control is defined on the time interval $[0,\delta_2]$ and is of the order of $\frac{1}{\delta_2}$. 


\smallskip


Finally, the argument for generating the other $\exp(iP_j\phi_j)\psi_0$ is completely identical. To conclude, by inductive hypothesis, there exists a piecewise contant control $(u_0,u)$ steering the state $\exp(i\sum_{j=1}^NP_j\phi_j)\psi_0$ arbitrarily close to the state 
$$\exp(\phi_0) \exp\left(i\sum_{j=1}^NP_j\phi_j\right)\psi_0=\exp\left(\phi_0+i\sum_{j=1}^NP_j\phi_j\right)\psi_0=\exp(\phi)\psi_0$$
 in arbitrarily small times. This fact concludes the proof of the property $(P_n)$.\\




\textbf{Conclusion of the proofs of Theorem \ref{thm:main-result} and Corollary \ref{thm:eigenmodes}}\\
We prove now the validity of the approximate controllability inequality \eqref{Pn} w.r.t. the $H^s-$norm by considering a similar propagation of regularity argument to the one adopted in the proof of Proposition \ref{lemma:main-tool-bis}. 

\smallskip


Let $\psi_0\in H^{2s}(\R^N,C)$ with $s\in\N^*$ such that $s>N/2$ and $\phi\in\mathcal{H}_n$ with $n\in\N^*$. First, the property $(P_n)$ yields the existence of $(u_0^\tau,u^\tau):[0,\tau]\to \mathbb{R}^{N+1}$ piecewise constant controls such that the solutions $\psi(t;\psi_0)$ of \eqref{eq:schro} associated with the controls $(u_0^\tau,u^\tau)$ and with the initial condition $\psi_0$ satisfies
\begin{align}\label{limit-L2}\lim_{\tau\rightarrow 0}\|\psi(\tau;\psi_0)-e^{\phi}\psi_0\|_{L^2(\mathbb{R}^N)}=0.\end{align}
%$$\|(-\Delta)^\frac{s}{2}\, \cdot\|_{L^2}^2=\la (-\Delta)^\frac{s}{2}\, \cdot,(-\Delta)^\frac{ s }{2}\, \cdot\ra\leq \|(-\Delta)^{s }\, \cdot\|_{L^2}\ \ \| \cdot\|_{L^2}.$$
%By using together the previous relations, we can show that there exist $r_1, r_2>0$ such that $$\|(-\Delta)^\frac{s}{2}\, \cdot\|_{L^2}\leq  \|(-\Delta)^\frac{s_1}{2}\, \cdot\|_{L^2}^{r_1}\ \ \|\cdot\|_{L^2}^{r_2}$$ and, t
%Then, there exists $C>0$ for $s_1=2s$ such that
%\begin{align}\label{interpolation}\|\cdot\|_{H^{s}(\R^N,\C)}^2\leq C (\|\cdot\|_{H^{s_1}(\R^N,\C)}\|\cdot\|_{L^2(\R^N,\C)}+ \|\cdot\|_{L^2(\R^N,\C)}^2).\end{align}
%Thanks to Proposition \ref{limit-L2}, we have 
%$$\lim_{\delta\to 0^+} \|\mathcal{R}(\delta,\psi_0,e_j u/\delta)-\exp(-i uQ_j)\psi_0\|_{L^2(\R^N,\C)}=0.$$ {interpolation}
In addition, there exists $C>0$, independent from $\tau$, such that $$\| u_0^\tau\|_{L^1((0,\tau),\R)}+\| u^\tau\|_{L^1((0,\tau),\R^{N})}\leq C.$$ This fact and the relation \eqref{propreg} showed in Proposition \ref{prop:well} imply the existence of $ C>0$ such that 
\begin{align*}\|\psi(\tau;\psi_0)-e^{\phi}\psi_0\|_{H^{2s}(\R^N)}&\leq \|\psi(\tau;\psi_0)\|_{H^{2s}(\R^N)}+\|e^{\phi}\psi_0\|_{H^{2s}(\R^N)}\\
& \leq C(\|\psi_0\|_{H^{2s}(\R^N)}+1).\end{align*}
Finally, thanks to the interpolation of the Sobolev norms \eqref{interpolation:1}, there exists $C>0$ such that
\begin{align*}&\lim_{\delta\to 0^+} \|\psi(\tau;\psi_0)-e^{\phi}\psi_0\|_{H^{s}(\R^N)}\leq \lim_{\delta\to 0^+} C\\
%&\leq \lim_{\delta\to 0^+} \|\mathcal{R}(\delta,\psi_0,e_j u/\delta)-\exp(-i uQ_j)\psi_0\|_{H^{s_1}(\R^N)}^{r_1} \|\mathcal{R}(\delta,\psi_0,e_j u/\delta)-\exp(-i uQ_j)\psi_0\|_{L^2(\R^N)}^{r_2}\\
&\times\left(\big(\|\psi_0\|_{H^{2s}(\R^N)}+1\big)\|\psi(\tau;\psi_0)-e^{\phi}\psi_0\|_{L^2(\R^N)}+\|\psi(\tau;\psi_0)-e^{\phi}\psi_0\|_{L^2(\R^N)}^2\right)=0.
\end{align*}
The last identity ensures the approximate controllability w.r.t. the $H^s$-norm when $\psi_0\in H^{2s}(\R^N,\C)$ and, by density, we can extend the result for $\psi_0\in H^{s}(\R^N,\C)$. The validity of the approximate controllability inequality \eqref{Pn} w.r.t. the $H^s-$norm is then guaranteed. The density property of $\mathcal{H}_\infty$ in $H^s(\R^N,\R)$ when $s\in\N^*$ proved in Lemma \ref{lem:density} yields the validity of Theorem \ref{thm:main-result}. Corollary \ref{thm:eigenmodes} is then a direct consequence of Theorem \ref{thm:main-result}: it suffices to consider (for $\epsilon>0$ small enough) $$\phi(x)=(\nu-\xi)x\chi^\epsilon_S(x).$$ 



\bibliographystyle{siamplain}
\bibliography{references}


\end{document}
