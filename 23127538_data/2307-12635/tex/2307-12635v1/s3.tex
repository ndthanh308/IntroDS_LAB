\section{Contact homology and symplectic cohomology}
\subsection{Contact homology}
We will first recall the definition of contact homology. Let $(Y,\xi)$ be a co-oriented contact manifold and $\alpha$ a contact form such that all Reeb orbits are non-degenerate. Let $V$ denote the $\Q$ vector space generated by formal variables $q_\gamma$ for each good orbit $\gamma$ of $(Y,\alpha)$. We grade $q_\gamma$ by $|q_{\gamma}|:=\mu_{CZ}(\gamma)+n-3$ (the SFT degree), which should be understood as a well-defined $\Z/2$ grading in general unless $\det_{\C}\xi$ is trivialized. The chain complex  $\CC(Y,\alpha)$ for the contact homology is the free symmetric algebra $SV$. The differential is defined as follows.
\begin{equation}\label{eqn:partial}
\partial_{\CH}(q_{\gamma}) = \sum_{[\Gamma]} \#\overline{\cM}_{Y}(\{\gamma \},\Gamma) \frac{1}{\mu_{\Gamma}\kappa_{\Gamma}}q^{\Gamma}.
\end{equation}
The sum is over all multisets $[\Gamma]$, i.e.\ sets with duplicates. And $\Gamma$ is an ordered representation of $[\Gamma]$, e.g.\ $$\Gamma=\{\underbrace{\gamma_1,\ldots,\gamma_1}_{i_1}, \ldots, \underbrace{\gamma_m,\ldots,\gamma_m}_{i_m}\}$$ is an ordered set of good orbits with $\gamma_i\ne \gamma_j$ for $i\ne j$. We write $\mu_{\Gamma}=i_1!\ldots i_m!$ and $\kappa_{\Gamma}=\kappa^{i_1}_{\gamma_1}\ldots \kappa^{i_m}_{\gamma_m}$ is the product of multiplicities of the Reeb orbits, and $q^{\Gamma}=q_{\gamma_1} \ldots  q_{\gamma_m}$. Here $\overline{\cM}_{Y}(\Gamma^+,\Gamma^-)$ is the SFT compactification \cite{zbMATH02062477} of the moduli space of rational holomorphic curves with asymptotic Reeb orbits $\Gamma^+=\{\gamma_1^+,\ldots,\gamma_{s^+}^+ \}$, $\Gamma^-=\{\gamma_1^-,\ldots,\gamma_{s^-}^- \}$ near positive and negative punctures respectively, modulo the $\R$-translation, in the symplectization $(\R_t\times Y, \rd(e^t\alpha))$ using a cylindrical $\R$-invariant almost complex structure $J$. The virtual dimension of $\overline{\cM}_{Y}(\Gamma^+,\Gamma^-)$ is given by
$$(n-3)(2-s^+-s^-)+\sum_{i=1}^{s^+}\mu_{CZ}(\gamma_i^+) - \sum_{i=1}^{s^-}\mu_{CZ}(\gamma_i^-)-1$$
where the Conley-Zehnder index is defined using a symplectic trivialization of $u^*\xi$ for $u\in \overline{\cM}_{Y}(\Gamma^+,\Gamma^-)$.


The orientation property of $\overline{\cM}_{Y}(\{\gamma\},\Gamma)$ implies that \eqref{eqn:partial} is independent of the representative $\Gamma$. \eqref{eqn:partial} is always a finite sum by the SFT compactness \cite[Theorem 10.1]{zbMATH02062477} as the Hofer energy is bounded by twice of the period of $\gamma$ \cite[Lemma 5.16]{zbMATH02062477}. The differential on $\CC(Y,\alpha)$ is defined by the Leibniz rule
$$\partial(q_{\gamma_1} \ldots  q_{\gamma_l})=\sum_{j=1}^l (-1)^{|q_{\gamma_1}|+\ldots + |q_{\gamma_{j-1}}|}q_{\gamma_1} \ldots  q_{\gamma_{j-1}} \partial(q_{\gamma_j}) q_{\gamma_{j+1}}\ldots  q_{\gamma_l}.$$
The relation $\partial^2=0$ follows from the boundary configuration of $\overline{\cM}_{Y}(\{\gamma\},\Gamma)$ with virtual dimension $1$.

Given an exact cobordism $(X,\lambda)$ from $Y_-$ to $Y_+$, then we have an algebra map $\phi$ from $\CC(Y_+,\lambda|_{Y_+})$ to $\CC(Y_-,\lambda|_{Y_-})$, which on generators is defined by
$$\phi(q_{\gamma})= \sum_{[\Gamma]}\# \overline{\cM}_{X}(\{\gamma\},\Gamma)\frac{1}{\mu_{\Gamma}\kappa_{\Gamma}}q^{\Gamma},$$
where $\Gamma$ is a collection of good orbits of $Y_-$. Here $\overline{\cM}_{X}(\Gamma^+,\Gamma^-)$ is the SFT compactification of the moduli space of rational holomorphic curves in the cobordism. The boundary configuration of $\overline{\cM}_{X}(\{\gamma\},\Gamma)$ with virtual dimension $1$ gives the relation $\partial\circ \phi = \phi \circ \partial$.

Even though contact homology is the simplest version of SFT, the analytical foundation to establish the counting of moduli spaces above as well as their algebraic relations is very involved and requires sophisticated virtual techniques going well beyond the classical approach by choosing $J$ carefully. The \emph{homology level} of contact homology was established by Pardon \cite{zbMATH07085531} using the implicit atlas and VFC \cite{zbMATH06578598} and Bao-Honda \cite{zbMATH07656377} by Kuranishi perturbation theory. It is expected that they give the same contact homology. In this paper, we will use Pardon's construction and virtual techniques.

\begin{theorem}[\cite{zbMATH07085531}]
	The homology $\CHA_*(Y):=H_*(\CC(Y,\alpha))$ above realized in VFC is well defined and gives a monoidal functor from the symplectic cobordism category (with objects contact manifolds and morphisms exact cobordisms up to homotopy) to the category of (super)commutative algebras.
\end{theorem}


\begin{theorem}[\cite{zbMATH05709738}]\label{thm:AO}
    If $Y$ is overtwisted, then $\CHA_*(Y)=0$.
\end{theorem}

\begin{remark}
    The vanishing of contact homology implies the vanishing of rational SFT as well as SFT on homology level \cite{zbMATH05658836}. Along with \Cref{thm:AO}, this led Bourgeois and Niederkr{\"u}ger to define contact manifolds with vanishing contact homology to be algebraically overtwisted manifolds. As overtwisted manifolds sitting in the bottom of the symplectic cobordism category by the work of Etnyre-Honda \cite{zbMATH01801587} and Eliashberg-Murphy \cite{zbMATH07600533}, algebraically overtwisted manifolds sit in the bottom of the symplectic cobordism category in an algebraic sense, after which there are hierarchies describing the increasing levels of tightness using SFT, see Latschev-Wendl's algebraic torsions \cite{zbMATH06000009} and Moreno and author's hierarchy functors \cite{moreno2020landscape}.
\end{remark}

\subsection{Symplectic cohomology}
Before reviewing symplectic cohomology, we first recall the following theorem from \cite{bowden2022making}, which serves as a motivation for \Cref{thm:main,thm:main'}.
\begin{theorem}[{\cite[Theorem 4.4, 4.5]{bowden2022making}}]\label{thm:flex}
    Assume $(M,\xi)$ is a strongly fillable contact manifold, such that $c_1(\xi)$ is torsion. Let $Y:=\partial(V\times \D)$, where $V$ is a Liouville domain. Then the following holds.
\begin{enumerate}
    \item If $W$ is a strong filling of $Y$, the composition $H_*(V\times \{pt\};\Q)\to H_*(Y;\Q)\to H_*(W;\Q)$ is injective.
    \item If $c_1(V)$ is torsion and $W$ is a strong filling of $Y\#M$, then  the following composition is injective 
    $$H_*(V\times \{pt\};\Q) \to H_*(Y;\Q)\to H_*(W;\Q).$$
\end{enumerate}
    Let $W_0$ be a $2n$-dimensional flexible Weinstein domain with $c_1(W_0)$ torsion. Then for any strong filling $W$ of the connect sum $Y:=\partial W_0 \# M$,  we have that $H^*(W;\Q)\to H^*(Y;\Q)$ is surjective onto the image of 
    $$H^*(W_0;\Q)\to H^*(\partial W_0;\Q)\subset H^*(Y;\Q)=H^*(\partial W_0;\Q)\oplus H^*(M;\Q),$$
    for $1\le * \le n$ when $n$ is odd, and for $2\le *\le n$ when $n$ is even. Equivalently, we have the dual statement that the kernel of $H_*(\partial W_0;\Q)\to H_*(W_0;\Q)$ contains the kernel of $H_*(\partial W_0;\Q) \hookrightarrow  H_*(Y;\Q)\to H_*(W;\Q)$ for the same $*$ above.
\end{theorem}

As a corollary of \Cref{thm:flex}, we have the following weaker form of \Cref{thm:main,thm:main'}.
\begin{corollary}\label{cor:nofilling}
    Let $Y_{\Lambda}$ be the contact manifold in \Cref{thm:main,thm:main'} from $(+1)$-surgeries. Assume $Y_2$ is strongly fillable with  $c_1(Y=Y_1\#Y_2)$ torsion. if then $Y_{\Lambda}$ has no strong filling.
\end{corollary}
\begin{proof}
    If $Y_{\Lambda}$ has a strong filling, we then obtain a strong filling of the contact manifold in \Cref{thm:flex} by gluing with the surgery cobordism. Then the homology assumption on $\Lambda$ yields a contradiction with \Cref{thm:flex}. 
\end{proof}
The proof of \Cref{thm:flex} was based on the study of symplectic cohomology of strong fillings and was previously done in \cite{zbMATH07673358} for exact fillings. The strategy for \Cref{thm:main,thm:main'} is to take out the holomorphic curves (perturbed by a Hamiltonian) obstructing the filling in \Cref{cor:nofilling} and put them in the surgery cobordism, whose degeneration will yield the vanishing of the contact homology of the concave boundary. 

\subsubsection{Basics of Symplectic cohomology}
Let $(W,\lambda)$ be an exact filling and $(\widehat{W},\widehat{\lambda})=W\cup (1,\infty)_r\times \partial W$ be the completion, where $\widehat{\lambda}=\lambda$ on $W$ and $\widehat{\lambda}=r\lambda|_{\partial W}$ on $(1,\infty)_r\times \partial W$. Note that $(0,\infty)_t\times \partial W = (1,\infty)_r\times \partial W$ for $r=e^t$. Let $H$ be a time-dependent Hamiltonian on the completion $(\widehat{W},\widehat{\lambda})$, then the symplectic action for an orbit $\gamma$ is 
\begin{equation}\label{eqn:action}
\cA_H(\gamma)=-\int \gamma^*\widehat{\lambda}+\int_{S^1} (H\circ \gamma)\rd t.
\end{equation}
We say an almost complex structure $J$ is cylindrically convex near $\{r_0\} \times \partial W $ iff near the hypersurface $r=r_0$ we have that $\widehat{\lambda}\circ J =\rd r$. We will consider a Hamiltonian $H$, which is a $C^2$ small perturbation (for the more precise meaning, see \eqref{ii} below) to the Hamiltonian that is $0$ on $W$ and linear with slope $a>0$ on $(1,\infty)_r\times \partial W$, such that $a$ is not a period of Reeb orbits of $R_{\lambda}$ for contact form $\lambda$ on $\partial W$. We may assume the Hamiltonian is non-degenerate, where the Hamiltonian vector field is defined by $\rd \widehat{\lambda}(\cdot, X_{H})=\rd H$, then the periodic orbits consist of the following.
\begin{enumerate}[(i)]
	\item\label{i} Constant orbits on $W$ with $\cA_H\approx 0$.
	\item\label{ii} Non-constant orbits near Reeb orbits of $R_{\lambda}$ on $\{1\} \times \partial W$, with action close to the negative period of the Reeb orbits. In particular, we have $\cA_H\in (-a,0)$.  To see this, note that our $H$ is an $S^1$-dependent $C^2$ small perturbation to an autonomous Hamiltonian $h(r)$ with $h'(r)=a$ for $r>1+\epsilon$ with $\epsilon$ small and $h(r)=0$ for $r\le 1$. The non-constant orbits of $h(r)$ are in $S^1$ families like $(\gamma(h'(r_0)t),r_0)$, where $\gamma$ is a Reeb orbit of $R_{\lambda}$ with the period of $\gamma$ is $h'(r_0)$ for $1<r_0<1+\epsilon$. Therefore the symplectic action of such an orbit is
	$$-h'(r_0)r_0+h(r_0).$$
	It is clear that when $\epsilon\ll 1$, we have that the symplectic action is approximately the negative period of $\gamma$, which, in particular, is in $(-a,0)$. Then the $C^2$-small perturbation to $h(r)$ in \cite[Lemma 3.3]{MR2475400} will break the $S^1$ family orbits into two non-degenerate orbits with symplectic action arbitrarily close to the original $S^1$ family. We use $\check{\gamma},\hat{\gamma}$ to denote the two Hamiltonian orbits from a Reeb orbit $\gamma$, their difference is characterized by $\mu_{CZ}(\check{\gamma})=\mu_{CZ}(\gamma)$ and $\mu_{CZ}(\hat{\gamma})=\mu_{CZ}(\gamma)+1$. 
\end{enumerate} 
After fixing an $S^1$-dependent compatible almost complex structure $J$ that is cylindrically convex near a slice (i.e.\ a hypersurface $r=r_0$) where the Hamiltonian is linear with slope $a$, we can consider the compactified moduli space of Floer cylinders, i.e.\ solutions to $\partial_s u+J(\partial_t u-X_{H})=0$ modulo the $\R$ translation and asymptotic to two Hamiltonian orbits
$$\overline{\cM}_{H}(x,y)=\overline{\left\{u:\R_s\times S^1_t\to \widehat{W}\left|\partial_s u+J(\partial_t u-X_{H})=0, \lim_{s\to \infty} u(s,\cdot)=x, \lim_{s\to -\infty} u(s,\cdot)=y \right.\right\}/\R}.$$
With a generic choice of $J$, the count of rigid Floer cylinders defines a cochain complex $C^*(H)$, which is a $\Q$-vector space generated by Hamiltonian orbits. The differential $\delta$ is defined as
$$\delta(x)=\sum_{y,\dim \cM_{x,y}=0} \# \overline{\cM}_H(x,y)\cdot y.$$
The orbits of type \eqref{i} form a subcomplex $C^*_0(H)$, whose cohomology is $H^*(W;\Q)$. The orbits of type \eqref{ii} form a quotient complex $C^*_+(H)$. The cochain complexes are graded by $n-\mu_{CZ}$, which is in general a $\Z/2$ grading unless we choose a trivialization of $\det_{\C}\oplus^N TW$ for some $N\in \N_+$. Given two Hamiltonians $H_a,H_b$ with slopes $a<b$, we can consider a non-increasing homotopy of Hamiltonians $H_s$ from $H_b$ to $H_a$, i.e.\ $H_s=H_b$ for $s\ll0$ and $H_s=H_a$ for $s\gg 0$. Then the count of rigid solutions to the parameterized Floer's equation $\partial_s u+J(\partial_t u-X_{H_s})=0$ defines a continuation map $C^*(H_a)\to C^*(H_b)$, which is compatible with splitting into zero and positive complexes. Then the (positive) symplectic cohomology of $W$ is defined as
$$SH^*(W):=\lim_{a\to \infty} H^*(C^*(H_a)),\quad SH_+^*(W):=\lim_{a\to \infty} H^*(C^*_+(H_a)),$$
which fit into a tautological exact sequence,
$$\ldots \to H^*(W)\to SH^*(W)\to SH^*_+(W)\to H^{*+1}(W)\to \ldots $$
We define the filtered symplectic cohomology $SH^*_{<a}(W)$ by $H^*(C^*(H_a))$ and $SH^*_{+,<a}(W)$ by $H^*(C^*_+(H_a))$, which are independent of $H_a$ as long as $a$ is not a period of Reeb orbits on $\partial W$, see \cite[Proposition 2.8]{zhou2020mathbb}. The filtered symplectic cohomology depends on the Liouville form on the boundary, but we will suppress the notation of dependence as the specific contact form will be clear in the proofs.

One can also use autonomous Hamiltonians for symplectic cohomology, where Hamiltonian orbits come in $S^1$ families. The setup for symplectic cohomology requires choosing auxiliary Morse functions on the image of the Hamiltonian orbits to build the cascades model \cite{MR2475400}, also see \cite[\S 2.2]{zbMATH07673358}. Each Reeb orbit $\gamma$ gives rise to two generators of the cochain complex again, also denoted by $\check{\gamma},\hat{\gamma}$ corresponding to the minimum and maximum of the auxiliary Morse function on $S^1$. We use $\overline{\gamma}$ to represent an unspecified (check or hat) generator from a Reeb orbit $\gamma$.

We define 
$$\delta_{\partial}:SH_+^{*}(W)\to H^{*+1}(W;\Q)\to H^{*+1}(\partial W;\Q).$$
For any closed submanifold $S\subset \partial W$, we can define a map $C_+(H)\to \Q$ by counting the compactified moduli space  $\overline{\cM}_H(x,S)$
$$\overline{\left\{ u:\C\to \widehat{W}\left|\partial_su+J(\partial_tu-X_H)=0, \lim_{s\to \infty} u = x, u(0)\in \{1-\eta\} \times S \subset W  \right.\right\}/\R}$$
for $0<\eta \ll 1$. Here $H$ is assumed to be zero below the level of $r=1-\eta$. Such a map is clearly a cochain map, and, on the cohomology level, it is the same as the map $\la \delta_{\partial}(\cdot),[S]\ra:SH_+^*(W)\to \Q$, where $\la \cdot, \cdot\ra$ is the pairing $H^*(W;\Q)\otimes H_*(W;\Q)\to \Q$. 

\subsubsection{Viterbo transfer}
Let $(V,\lambda_V)\subset (W,\lambda_W)$ be an exact subdomain, i.e.\ $\lambda_W|_V=\lambda_V$, then there exists $\epsilon>0$, such that $V_{\epsilon}:=V\cup  (1,1+\epsilon]_r\times \partial V \subset W$ with $\lambda_W|_{V_{\epsilon}}=\widehat{\lambda}_V$. Then we can consider a Hamiltonian $H_{VT}$ on $\widehat{W}$ as a $C^2$-small non-degenerate perturbation to the following.
\begin{enumerate}
	\item $H_{VT}$ is $0$ on $V$.
	\item $H_{VT}$ is linear with slope $B$ on $[1,1+\epsilon]_r\times \partial V$, such that $B$ is not the period of a Reeb orbit on $\partial V$.
	\item $H_{VT}$ is $B\epsilon$ on $W\backslash V_{\epsilon}$.
	\item $H_{VT}$ is linear with slope $A\le B\epsilon$ on $[1,\infty)_r\times \partial W$, such that $A$ is not the period of a Reeb orbit on $\partial W$.  
\end{enumerate}
Then there are five classes of periodic orbits of $X_{H_{VT}}$.
\begin{enumerate}[(I)]
	\item\label{I} Constant orbits on $V$ with $\cA_{H_{VT}}\approx 0$.
	\item\label{II} Non-constant periodic orbits near $\partial V$ with $\cA_{H_{VT}} \in (-B,0)$.
	\item\label{III} Non-constant periodic orbits near $\{1+\epsilon\}\times \partial V$ with action $\cA_{H_{VT}}\in (B\epsilon-(1+\epsilon)B, B\epsilon)=(-B,B\epsilon)$. To see the action region, it is similar to \eqref{ii} after \eqref{eqn:action}. Those orbits are close to Reeb orbits of $(\partial W, \lambda|_{\partial W})$, but placed near $r=1+\epsilon$. Therefore $-\int \gamma^*\widehat{\lambda}$ in \eqref{eqn:action} is close to $(1+\epsilon)$  times (as $\widehat{\lambda}|_{r=1+\epsilon}=(1+\epsilon)\lambda|_{\partial W}$) the period of the Reeb orbits and $\int_{S^1} (H_{VT}\circ \gamma) \rd t$ is close to $B\epsilon$. Hence the claim follows.
	\item\label{IV} Constant orbits on $W\backslash V_{\epsilon}$ with $\cA_{H_{VT}}\approx B\epsilon$.
	\item\label{V} Non-constant periodic orbits near $\partial W$ with $\cA_{H_{VT}}\in (B\epsilon-A, B\epsilon)$
\end{enumerate}
In particular, when $B\epsilon \ge A$, the quotient complex generated by orbits with non-positive action are generated by type \eqref{I}, \eqref{II} orbits along with some of the type \eqref{III} orbits. However, there is no Floer cylinder from a type \eqref{III} orbit to a type \eqref{I} or \eqref{II} orbit \cite[Figure 6]{cieliebak2018symplectic}. This follows from the asymptotic behavior lemma \cite[Lemma 2.3]{cieliebak2018symplectic} and the integrated maximum principle \cite[Lemma 2.2]{cieliebak2018symplectic}. Therefore orbits of the form \eqref{I}, \eqref{II} form a quotient complex when $B\epsilon\ge A$. Let $H_V$ denote the Hamiltonian on $\widehat{V}$ which is the linear extension of the truncation of $H_{VT}$ on $V_{\epsilon}$. Then by \cite[Lemma 2.2]{cieliebak2018symplectic}, the quotient complex is identified with $C^*(H_V)$. Those two applications of the integrated maximum principle are where the exactness of the cobordism $W\backslash V$ is crucial. Next, we consider a Hamiltonian $H_W$ on $\widehat{W}$ which is a $C^2$ small perturbation to the function that is zero on $W$ and linear with slope $A$ on $ (1,\infty)\times \partial W$. Then $H_W\le H_{VT}$ and we can find non-increasing homotopy from $H_{VT}$ to $H_W$, which defines a continuation map. Therefore we have a map 
$$C^*(H_W)\stackrel{\text{continuation}}{\longrightarrow} C^*(H_{VT}) \stackrel{\text{quotient}}{\longrightarrow} C^*(H_V).$$
Since the continuation map increases the symplectic action, the above map respects the splitting into $C_0,C_+$. Taking the direct limit for $B$ yields the Viterbo transfer map which is compatible with the tautological exact sequence,
$$
\xymatrix{\ldots \ar[r] & H^*(W;\Q) \ar[r] \ar[d] & SH^*_{<A}(W) \ar[r] \ar[d] & SH_{+,<A}^*(W) \ar[r]\ar[d] & H^{*+1}(W;\Q) \ar[r]\ar[d] & \ldots \\
	\ldots \ar[r] & H^*(V;\Q) \ar[r] & SH^*_{<B}(V) \ar[r] & SH_{+,<B}^*(V) \ar[r] & H^{*+1}(V;\Q) \ar[r] &  \ldots
 }
$$
If we also take the direct limit for $B$ first then for $A$, we get the Viterbo transfer map for the full symplectic cohomology.

\subsubsection{Symplectic cohomology of contact manifolds with DGA augmentations}\label{ss:aug}
Let $(W,\lambda)$ be an exact filling of $(Y,\alpha=\lambda|_Y)$, i.e.\ an exact cobordism from $\emptyset$ to $Y$, then the functorial package of contact homology gives rise to differential graded algebra (DGA) morphism
$$\epsilon_W:\CC(Y,\alpha)\to \Q,$$
i.e.\ a DGA augmentation. On the other hand, given a DGA augmentation $\epsilon$, one can define the positive symplectic cohomology of $\epsilon$ as follows. Let $H$ be a Hamiltonian on $\widehat{Y}=\R_t\times Y \simeq (\R_+)_r\times Y$, where $r=e^t$, such that $H$ is $0$ on $(0,1]_r\times Y$ and is same as the Hamiltonian on the cylindrical end (with slope $a$) in the definition of symplectic cohomology. Then we can consider the following compactified moduli space $\overline{\cM}_{Y,H}(x,y,\Gamma)$ for $\Gamma=\{\gamma_1,\ldots,\gamma_k\}$ a multiset of good Reeb orbits and $x,y$ \emph{non-constant} Hamiltonian orbits,
$$\overline{\left\{u:\R_s\times S^1_t\backslash \{p_1,\ldots,p_k\} \to \widehat{Y}\left|\partial_s u+J(\partial_t u-X_{H})=0, \lim_{s\to \infty} u(s,\cdot)=x, \lim_{s\to -\infty} u(s,\cdot)=y, \lim_{p_i} u= \gamma_i \right.\right\}/\R}.$$
Here $\lim_{p_i} u= \gamma_i$ is a short-hand for that $u$ is asymptotic to $\gamma_i$ at $p_i$ viewed as a negative puncture, where the Hamiltonian is zero and the equation is the usual Cauchy-Riemann equation. Such compactification will be a mixture of Floer-type breaking at non-constant Hamiltonian orbits (it can not break at a constant orbit of $H$ by symplectic action reasons) and SFT building breaking at the lower level. Then we define a differential $\delta_{\epsilon}$ on $C_+(H)$ as follows:
\begin{equation}\label{eqn:diff}
    \delta_{\epsilon}(x) = \sum_{[\Gamma]}  \frac{1}{\mu_{\Gamma}\kappa_{\Gamma}}\#\overline{\cM}_{Y,H}(x,y,\Gamma)\prod_{\gamma\in \Gamma}\epsilon(q_{\gamma}) \cdot y.
\end{equation}
The boundary configuration of $\overline{\cM}_{Y,H}(x,y,\Gamma)$ of virtual dimension $1$ gives that $\delta_{\epsilon}^2=0$. Similarly, by considering the compactified moduli space $\overline{\cM}_{Y,H}(x,C,\Gamma)$ for a singular chain $C\subset Y$ as follows, 
$$\overline{\left\{u:\C \backslash \{p_1,\ldots,p_k\} \to \widehat{Y}=\R_+\times Y\left|
\begin{array}{c}
\partial_s u+J(\partial_t u-X_{H})=0, p_i\ne 0,\\
\displaystyle \lim_{s\to \infty} u(s,\cdot)=x, u(0)\in \{ 1-\eta\}\times C, \lim_{p_i} u= \gamma_i
\end{array}\right.\right\}/\R},$$
for a fixed $0<\eta\ll 1$, we define $\delta_{\partial,\epsilon}:C^*_+(H)\to C^{*+1}(Y)$ by 
$$\delta_{\partial,\epsilon}(x)(C) =  \sum_{[\Gamma]} \frac{1}{\mu_{\Gamma}\kappa_{\Gamma}}\#\overline{\cM}_{Y,H}(x,C,\Gamma)\prod_{\gamma\in \Gamma}\epsilon(q_{\gamma}).$$
This is a cochain map. In actual constructions, to avoid working with the infinite-dimensional model $C^*(Y)$ (to avoid shrinking the space of admissible choices in the chosen virtual technique too much), one can use a finite-dimensional model for $Y$, for example, a simplicial complex from a triangulation or a Morse complex using an auxiliary Morse function. Moreover, if one only wants to understand the effect of $\delta_{\partial,\epsilon}$ on a homology class $Y$ under the paring $H^*(Y;\Q)\otimes H_*(Y;\Q)\to \Q$, one may take a singular chain or a submanifold representing the homology class. 

We use $SH^*_+(Y,\alpha,\epsilon)$ to denote the direct limit of the cohomology of $(C_+(H),\delta_{\epsilon})$ through the continuation map of increasing slopes (corrected by the augmentation by the same way as \eqref{eqn:diff}), and $\delta_{\partial,\epsilon}$ to denote the map $SH^*_+(Y,\alpha,\epsilon)\to H^{*+1}(Y;\Q)$. We summarize the properties of them in the following.
\begin{enumerate}
    \item The analytical foundation of such symplectic cohomology is established by Pardon's work on Hamiltonian-Floer cohomology \cite{zbMATH06578598} and contact homology \cite{zbMATH07085531}. But strictly speaking, as $\CC(Y,\alpha)$ as well as the counting of $\overline{\cM}_H(x,y,\Gamma)$ for $\Gamma=\{\gamma_1,\ldots,\gamma_k\}$ depend on auxiliary choices, i.e.\  choices in \cite[\S 4.8]{zbMATH07085531} (in addition to the choice of contact forms), the symplectic cohomology also depends on such choices a priori. However, it is expected that such symplectic cohomology is independent of contact forms as well as augmentations up to homotopies. In particular, the set of all such symplectic cohomology is a contact invariant. This requires upgrading the homotopy of the full contact homology algebra chain complex to a DGA homotopy in the contact homology package of \cite{zbMATH07085531}.
    \item When $\epsilon$ is $\epsilon_{W}$ from an exact filling (or a strong filling and we use the Novikov field coefficient), $SH^*_+(Y,\alpha,\epsilon_W)\to H^{*+1}(Y)$ is isomorphic to $SH^*_+(W)\to H^{*+1}(W)\to H^{*+1}(Y)$ for any choice of $\alpha$ and auxiliary choices in the VFC. The proof is no different from the functoriality of compositions for contact homology by neck-stretching in \cite[\S 1.5]{zbMATH07085531}. This perspective of symplectic cohomology was introduced by Bourgeois and Oancea \cite{MR2471597}, where they proved the equivalence between positive $S^1$-equivariant symplectic cohomology with linearized contact homology w.r.t.\ the augmentation from the filling. 
    \item Given an exact cobordism $(X,\lambda)$ from $Y_-$ to $Y_+$, we can consider a Hamiltonian $H_+$ on $\widehat{X}$ which is linear on the positive cylindrical end of $\widehat{X}$ and zero everywhere else. Given an augmentation $\epsilon$ of $Y_-$, one can define similarly the symplectic cohomology $SH_+^*(X,\lambda,\epsilon)$ (with additional choices in the VFC). We consider a Hamiltonian $H_{VT}$ on $\widehat{X}$ that is similar to the Hamiltonian in the construction of the Viterbo transfer map, following the same recipe of the Viterbo transfer map, along with the correction from the augmentation $\epsilon$, we get a Viterbo transfer map
    $$SH_+^{*}(X,\lambda,\epsilon)\to SH_+^*(Y_-,\lambda|_{Y_-},\epsilon)$$
    By neck-stretching along $Y_+\subset \widehat{X}$, we get an isomorphism 
    $$SH_+^*(X,\lambda,\epsilon)=SH_+^*(Y_+,\lambda|_{Y_+},\epsilon\circ \phi_X)$$
    where $\phi_X$ is the DGA morphism from the cobordism $X$, hence $\epsilon\circ \phi_X$ is a DGA augmentation of $\CC(Y_+,\lambda|_{Y_+})$. 
\end{enumerate}
We do not use the well-definedness (independence of various choices) of such invariants in this paper. The idea of the above Viterbo transfer maps with augmentations is used in the contact connected sum for \Cref{thm:main'}, but only in a very special form, where the augmentation is trivial.

\begin{remark}
One can modify the proof of \Cref{thm:flex} using symplectic cohomology of augmentations to prove that $Y_{\Lambda}$ in \Cref{thm:main,thm:main'} has no DGA augmentations in a similar way to \Cref{cor:nofilling}. 
\end{remark}
