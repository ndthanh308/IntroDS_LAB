\section{Vanishing of contact homology}
\subsection{Holomorphic curves hitting $\Lambda$}\label{ss:41}
\subsubsection{Case 1: $Y=\partial(V\times \D)$} \label{ss:411}
Let $f$ be a Morse function on $V$ such that $\partial_r f>0$ near the boundary where $r$ is the collar coordinate. Following \cite[\S 2.1]{zbMATH07673358}, we can endow $Y$ with a contact form such that:
\begin{enumerate}
	\item Each critical point $p$ of $f$ corresponds to a simple non-degenerate Reeb orbit $\gamma_p$, which is the circle over $p$ in the region $V\times S^1\subset \partial(V\times \D)$;
	\item Using the natural bounding disk from the $\D$-component, the Conley-Zehnder index of $\gamma_p$ is given by $2+\dim{V}/2-\ind(p)$, hence the SFT degree is positive;
	\item The period of $\gamma_p$ is approximately $1$ and $\gamma_p$ has longer period than $\gamma_q$ if and only if $f(p)<f(q)$. 
\end{enumerate}	
The orbits $\{\gamma_p\}$ from critical points are all the Reeb orbits with period at most $2$. If we use a Hamiltonian $H$ with slope $2$ which vanishes on the filling, then we have two non-constant Hamiltonian orbits $\check{\gamma}_p,\hat{\gamma}_p$ from each Reeb orbit $\gamma_p$, where $\mu_{CZ}(\check{\gamma}_p)=\mu_{CZ}(\gamma_p),\mu_{CZ}(\hat{\gamma}_p)=\mu_{CZ}(\gamma_p)+1$. Moreover, we have the following properties \cite[Proposition 2.6]{zbMATH07673358}.
\begin{enumerate}
\item We have $SH^{*}_{+,<2}(V\times \D) \simeq H^*(V;\Q)[1]\oplus H^*(V;\Q)[2]$, where the $H^*(V;\Q)[1]$ component is generated by $\check{\gamma}_p$ and the $H^*(V;\Q)[2]$ component is generated by $\hat{\gamma}_p$.
\item $\delta_{\partial}$ restricted to $H^*(V;\Q)[1]$ is injective and is isomorphism when restricted to $V\times \{\mathrm{pt}\}\subset V\times S^1\subset  \partial(V\times \D)$. $\delta_{\partial}$ is zero on the $H^*(V;\Q)[2]$ component. 
\end{enumerate}	
In particular, if $[\Lambda]$ is not in the kernel of $H_*(\partial(V\times \D);\Q)\to H_*(V\times \D;\Q)$, then we can find a linear combination of (check) orbits $x=\sum a_i\overline{\alpha}_i$, such that $x$ is closed and $\langle \delta_{\partial}(x), [\Lambda]\rangle\ne 0$, here $\la \cdot,\cdot\ra$ is the pairing $H^*(\partial W;\Q)\otimes H_*(\partial W;\Q)\to \Q$. For simplicity, we assume $x$ is a simple orbit $\overline{\alpha}$ with coefficient $1$, otherwise the argument in the general case only differs in notation.  As a consequence, for generic choices of almost complex structures, we have 
\begin{enumerate}
	\item $\#\overline{\cM}_H(\overline{\alpha},\overline{\beta})=0$, for any other non-constant Hamiltonian orbit $\overline{\beta}$;
	\item $\#\overline{\cM}_H(\overline{\alpha},\Lambda) \ne 0$,
\end{enumerate}	 
here by counting the number of points, we implicitly require that the virtual dimensions are zero. The Conley-Zehnder index of $\overline{\alpha}$ using the natural bounding disk is $2$. 


\subsubsection{Case 2: $Y=\partial W$ for a flexible $W$ with $c_1(W)$ torsion}
By the work of Lazarev \cite{zbMATH07184228}, there exists a contact form $\alpha$ of $\partial W$ and a positive real number $D$, such that all Reeb orbits of period smaller than $D$ have the following properties.
\begin{enumerate}
	\item They are non-degenerate and have Conley-Zehnder indices at least $1$, here the Conley-Zehnder indices are computed using a trivialization of $\det_{\C} \oplus^N\xi$ for some $N\in \N$ as $c_1(W)$ is torsion.
	\item $SH^*_{+,<D}(W)\to H^{*+1}(W;\Q)$ is surjective. 
\end{enumerate}	
We may rescale $\alpha$ such that $D=2$. Let $H$ be a Hamiltonian with slope $2$ that vanishes on the filling. If $[\Lambda]$ is not in the kernel of $H_*(\partial W;\Q)\to H_*(W;\Q)$, we can find a linear combination of Hamiltonian orbits (check and hat orbits from Reeb orbits), which we assume to be a single orbit $\overline{\alpha}$ as before, whose Conley-Zehnder index is $2$, such that $\overline{\alpha}$  represents a closed cochain in the positive cochain complex and $\la \delta_{\partial}(\overline{\alpha}),[\Lambda]\ra \ne 0$. Hence for generic choices of almost complex structures, we also have
\begin{enumerate}
	\item $\#\overline{\cM}_H(\overline{\alpha},\overline{\beta})=0$,  for any other non-constant Hamiltonian orbit $\overline{\beta}$;
	\item $\#\overline{\cM}_H(\overline{\alpha},\Lambda) \ne 0$.
\end{enumerate}	 
With the notions above, we consider those curves in the symplectization via neck stretching.
\begin{proposition}\label{prop:curve}
	For both $Y=\partial (V\times \D)$ and $Y$ flexibly fillable with $c_1(Y)$ torsion, $H$ is the Hamiltonian considered above but viewed as defined on $\widehat{Y}$, we have the following:
\begin{enumerate}
    \item\label{m2} $\#\overline{\cM}_{Y,H}(\overline{\alpha},\overline{\beta})=0$, for those moduli spaces with expected dimension $0$, where $\overline{\beta}$ is a non-constant Hamiltonian orbit. 
    \item\label{m1} $\#\overline{\cM}_{Y,H}(\overline{\alpha},\overline{\beta},\Gamma)=\emptyset$ for $\Gamma\ne \emptyset$. 
    \item\label{m3} $\overline{\cM}_{Y,H}(\overline{\alpha},\Lambda,\Gamma)=\emptyset$ for $\Gamma \ne \emptyset$ and $\#\overline{\cM}_{Y,H}(\overline{\alpha},\Lambda) \ne 0$.
\end{enumerate}
\end{proposition}
\begin{proof}
When $Y=\partial(V\times \D)$, this follows from the proof of \cite[Proposition 2.7, 3.2]{zbMATH07673358} by neck-stretching. For $Y=\partial W$ for flexible $W$ with $c_1(W)$ torsion, \eqref{m2} and \eqref{m3} follow from the proof of \cite[Theorem A]{zbMATH07367119}. For \eqref{m1}, since we can assume all Hamiltonian orbits have Conley-Zehnder indices at last $1$  and SFT degrees of all Reeb orbits of period smaller than $2$ are positive by \cite{zbMATH07184228}, the claim follows from that the expected dimension 
$$\mu_{CZ}(\overline{\alpha})-\mu_{CZ}(\overline{\beta})-1-\sum_{\gamma\in \Gamma}(\mu_{CZ}(\gamma)+n-3)$$
must be negative, since $\mu_{CZ}(\overline{\alpha})=2$.
\end{proof}

We consider $\overline{\cM}_{H,W_{\Lambda}}(\overline{\alpha},L,\Gamma)$, the compactified moduli space 
\begin{equation}\label{eqn:WH}
    \overline{\left\{u:\C \backslash \{p_1,\ldots,p_k\} \to \widehat{W_{\Lambda}}\left|
    \begin{array}{c}
    \partial_s u+J(\partial_t u-X_{H})=0, p_i\ne 0,\\
     \lim_{s\to \infty} u(s,\cdot)=\overline{\alpha}, u(0)\in L_{1-\eta}, \lim_{p_i} u= \gamma_i \end{array}\right.\right\}/\R},
\end{equation}
where $\Gamma=\{\gamma_i\}$ is a multiset of good Reeb orbits of $Y_{\Lambda}$, $p_i$ are negative punctures and $L_{1-\eta}$ the intersection of Lagrangian disk $L$ (the cocore of the Weinstein handle in the surgery cobordism) with the complement of $(1-\eta,1)_r\times Y$. $H$ is the Hamiltonian above but viewed as defined on $\widehat{W_{\Lambda}}$.

\begin{proposition}\label{prop:corbordism}
    We have
    $$\partial_{\CH}\left(\sum\frac{1}{\mu_{\Gamma}\kappa_{\Gamma}}\#\overline{\cM}_{H,W_{\Lambda}}(\overline{\alpha},L,\Gamma)q^{\Gamma}\right)\ne 0 \in \Q,$$hence $\CHA_*(Y_{\Lambda})=0$ and \Cref{thm:main} follows.
\end{proposition}
\begin{proof}
    We consider the boundary configuration of $\overline{\cM}_{H,W_{\Lambda}}(\overline{\alpha},L,\Gamma)$ with virtual dimension $1$. The boundary contains the following cases:
    \begin{enumerate}
        \item\label{t1} Floer breaking.  To see the curves with Floer breaking does not contribute to the boundary. First of all, \eqref{m1} of \Cref{prop:curve} implies that the top-level curve from Floer breaking has no negative punctures asymptotic to Reeb orbits by neck-stretching along $\{1-\eta\}\times Y \subset W_{\Lambda}$. When $Y=\partial W$ for a flexible $W$ with $c_1(W)$ torsion, we have $c_1(W_{\Lambda})$ is torsion. If a Floer cylinder without negative punctures in $\widehat{W}_{\Lambda}$ has virtual dimension $0$, the Floer cylinder in $\widehat{Y}$ with the same asymptotic orbits also has virtual dimension $0$ as the virtual dimension only depends on $\overline{\beta}$ in this case. Then by a neck-stretching along  $\{1-\eta\}\times Y \subset W_{\Lambda}$, we see that Floer cylinders with virtual dimension $0$ coincide with \eqref{m2} of \Cref{prop:curve}, as the SFT degrees of orbits are positive. When $Y=\partial(V\times \D)$, we can apply neck-stretching along  $\{1-\eta\}\times Y \subset W_{\Lambda}$, then by action reasons, the Floer cylinders in $\widehat{W}_{\Lambda}$ coincide with Floer cylinders in $\widehat{Y}$, which have no negative punctures and the rigid ones are counted as $0$ in \eqref{m2} of \Cref{prop:curve}.
            % Figure environment removed
        \item\label{t2} The point constraint goes to the boundary of $L_{1-\eta}$. This can be identified with $\cM_{Y,H}(\overline{\alpha},\Lambda)$ by \eqref{m3} of \Cref{prop:curve} using neck-stretching along $\{1-\eta\}\times Y \subset W_{\Lambda}$.
        \item\label{t3} SFT breaking at negative punctures, which corresponds to applying the contact homology differential. 
    \end{enumerate}
    In summary, that 
    $$0=\#\partial \overline{\cM}_{H,W_{\Lambda}}(\overline{\alpha},L,\Gamma), \text{ where}\virdim \overline{\cM}_{H,W_{\Lambda}}(\overline{\alpha},L,\Gamma)=1$$
    implies that
    $$\partial_{\CH}(\sum_{\virdim = 0}\frac{1}{\mu_{\Gamma}\kappa_{\gamma}}\#\cM_{H,W_{\Lambda}}(\overline{\alpha},L,\Gamma)q^{\Gamma})=\#\overline{\cM}_Y(\overline{\alpha},\Lambda) \ne 0.$$
    To make sense of the above computation, we need to apply Pardon's VFC. Type \eqref{t1}, \eqref{t2} of the boundary, as discussed above, is supplied by curves in \Cref{prop:curve}, which are cut out transversely for a generic choice of almost complex structures that is sufficiently stretched along $\{1-\eta\}\times Y$. Type \eqref{t3} of boundary is from the SFT breaking at the negative punctures, which are taken care of by the VFC without modifying the count of former transversely cut-off curves by \cite[Proposition 4.33]{zbMATH07085531}.
\end{proof}

%The following proposition is useful for future geometric applications. 
%\begin{proposition}\label{prop:intersection}
%	In the case of $V\times \D$, assume $\Lambda$ is disjoint from $\partial V \times \{0\} \subset \partial(V\times \D)$, then the curve kills contact homology of $Y_{\Lambda}$ will intersect the symplectization of $\partial  V \times \{0\}$ at most once. 
%\end{proposition}	
%\begin{proof}
%To make use of the intersection theory of holomorphic curves with complex hypersurfaces, we need to get rid of the Hamiltonian perturbation from the proof. We can use the linearized contact homology with action threshold $2$ for $V\times \D$, as \cite{} showed that $LCH_{<2}(V\times \D)\simeq SH^*_{+,S^1,<2}(V\times \D)$ because there is no transversality issue by simplicity of the Reeb orbits. Moreover, by counting holomorphic planes with a marked point, we get $LCH_{<2}(V\times \D)\to H^*(V\times \D) \to H^*(\partial(V\times \D))$ is same as $SH^*_{+,S^1,<2}(V\times \D)\to H_{S^1}^*(V\times \D) \to H^*(V\times \D)\to H^*(\partial(V\times \D))$. We can run the same argument as in the proof of \Cref{thm:main} but with moduli spaces without Hamiltonians, i.e. those for $LCH$. Since the winding number between the key orbit $\alpha$ and $\partial V \times \{0\}$ is $1$, we know the key curve in $LCH\to H^*(\partial(Y\times \D))$ intersect the sympletization of $\partial V\times \{0\}$ exactly once by the positivity of intersection (if we use an almost complex structure such that the sympletization of $\partial V\times \{0\}$ is holomorphic). In the cobordism argument, as $\Lambda$ does not intersect $\partial V\times \{0\}$, we can assume the Reeb flow is tangent to $\partial V\times \{0\}$ and has very long Reeb orbits there. As a consequence, the curves in the cobordism can not be asymptotics to Reeb orbits on $\partial V\times \{0\}$.  That  $\Lambda$ does not intersect $\partial V\times \{0\}$ also implies that we can view the sympletization of $\partial V\times \{0\}$ as contained in the completion of the surgery cobordism. Finally, by positivity of intersection, we have that the curves kill the contact homology have intersection number with $\partial(V)\times \{0\}$ at most one.
%\end{proof}	
%\begin{remark}
%	In the case of $V=T^*S^n$ and $\Lambda$ is the Legendrian lift of the zero section in $\partial(T^*S^n\times \D)$. 
%\end{remark}	

\subsection{Contact connected sum}
In the contact connected sum case, we need to establish the analog of \Cref{prop:curve}. We will use the Viterbo transfer map w.r.t.\ augmentations to transfer the curve in $Y_1$ to $Y_1\#Y_2$. 
% Figure environment removed
We view the above $1$-handle attachment as a degenerate exact cobordism $(W,\lambda)$ from $Y_1\sqcup Y_2$ to $Y_1\# Y_2$, i.e. the cobordism has width $0$ on $Y_1\sqcup Y_2$ away from the surgery region. We call such part the thin part of the cobordism. Let $\widehat{W}$ be the completion of $W$ w.r.t.\ the Liouville vector field. We use $\overline{\lambda}$ to denote $1$-form that is the Liouville form $\lambda$ on $W$ and is $\lambda|_{\partial_+W}$ on $\partial_+ W\times \R_+$ and $\lambda|_{\partial_-W}$ on $\partial_-W\times \R_-$, this form is not smooth along the boundary of the handle, but smooth everywhere else. We will be using almost complex structures $J$ (possibly depending on other parameters, like the $S^1$-coordinate of the cylinder), such that $\rd \overline{\lambda} (\cdot,J\cdot)\ge 0$ on where $\rd \overline{\lambda}$ is defined.  A neighborhood of $W$ in $\widehat{W}$ can be colored as follows:
% Figure environment removed
Here the concave boundary of the blue region and the convex boundary of the reg region are slices of the symplectization of $Y_1\cup Y_2$ and $Y_1\#Y_2$ respectively. The blue region has width $1$ measured w.r.t.\ the $r$-coordinate of the concave boundary of the blue region, i.e.\ the Liouville form on $\widehat{W}$ restricted to the concave boundary of the blue region is $\frac{1}{2}\lambda|_{\partial_-W}$. The width of the red region outside of $W$ is also $1$, hence the Liouville form on $\widehat{W}$ restricted to the convex boundary of the red region is $2\lambda|_{\partial_+W}$.


We consider an \emph{autonomous} Hamiltonian $H_{VT}$ in the Viterbo transfer for $\widehat{W}$ that is a perturbation to the following
\begin{enumerate}
    \item $H_{VT}$ is zero below the blue region.
    \item $H_{VT}$ is linear on the blue region of slope $4$;
    \item $H_{VT}$ is constant on the red region.
    \item $H_{VT}$ is linear on the green region of slope $4$.
\end{enumerate}
% Figure environment removed
As a consequence, $X_{H_{VT}}$ is zero in the red region and parallel to the Reeb vector field on $Y_1\sqcup Y_2$ and $Y_1\#Y_2$ in the blue and green regions respectively. As a consequence, we have $\int u^*\rd \overline{\lambda} \ge 0$, whenever $u$ solves the Floer equation for $H_{VT}$ on $\widehat{W}$ possibly with extra negative punctures. Under the conditions of \Cref{thm:main'}, we can assume the contact form on $Y_1$ enjoys the properties in \S \ref{ss:41} and $Y_2$ has a very large contact form such that all Reeb orbits have period $\gg 8$. After apply the thin $1$-handle $Y_1\sqcup Y_2$, Reeb orbits on $Y_1\#Y_2$ of period smaller than $2$ consists on 
\begin{enumerate}
    \item Reeb orbits with period smaller than $2$, i.e.\ those considered in  \S \ref{ss:41}, we use $\gamma_{\#}$ to stand for the orbit on $Y_1\#Y_2$ corresponding to $\gamma$ on $Y_1$. 
    \item Multiple covers of the Reeb orbit $\gamma_h$ from the $1$-handle in the co-core of the handle, the Conley-Zehnder index of $\gamma_h^k$ is given by $2k+(\dim Y_1+1)/2-2$ (using the obvious bounding disk in the handle), where $k$ is the multiplicity.
\end{enumerate}
Therefore $H_{VT}$ only sees those orbits in the green region due to the re-scaling. In the blue region, $H_{VT}$ sees Reeb orbits on $(Y_1\cup Y_2,\lambda|_{\partial_-W})$ with period at most $8$, which are always on $Y_1$ by our assumption. 

\subsubsection{The first case: $c_1(Y_1),c_1(Y_2)$ are torsion} We can assume SFT degrees of orbits, which are globally defined, with periods up to $8$ are strictly positive. So even though $\CC(Y_1\cup Y_2,\lambda|_{\partial_-W})$ may not have an augmentation, the part of $\CC(Y_1\cup Y_2,\frac{1}{2}\lambda|_{\partial_-W})$ with action at most $4$ has an augmentation, namely the trivial map $\epsilon_0$ sending every generator to zero by our degree assumptions, or equivalently the augmentation induced from the natural filling of $Y_1$ which is trivial on orbits of $Y_2$ (those are invisible with the action threshold above). We consider Hamiltonian $H$ on the completion of $(Y_1\#Y_2, 2\lambda|_{\partial_+W})$ with slope $4$ and Hamiltonian $G$ on the completion of $(Y_1\cup Y_2,\frac{1}{2}\lambda|_{\partial_-W})$ with slope $4$, we have the following Viterbo transfer map using $H_{VT}$ (and the cascades model to deal with the $S^1$ family of Hamiltonian orbits) above
\begin{equation}\label{eqn:diagram}
\xymatrix{
SH_{+,<4}^*(Y_1\#Y_2, 2\lambda|_{\partial_+W},\epsilon_0\circ \phi_{\overline{W}})\ar[d]^{\simeq}  & SH_{+,\le 4}(Y_1,\frac{1}{2}\lambda|_{\partial_-W},\epsilon_0)\\
SH_{+,<4}^*(\overline{W},\widehat{\lambda},\epsilon_0) \ar[r]^{\phi_{Viterbo}\qquad }\ar[d] & SH_{+,<4}^*(Y_1\cup Y_2, \frac{1}{2} \lambda|_{\partial_-W},  \epsilon_0)\ar[u]^{\simeq}\ar[d]^{\delta_{\partial}}\\
H^{*+1}(W;\Q) \ar[r] & H^{*+1}(Y_1\cup Y_2;\Q)
}
\end{equation}
where $\overline{W}=W\cup (1,2]\times \partial_+W \cup [1/2,1)\times \partial_-W$ is the cobordism composed from the blue and red regions. In the Viterbo transfer map, which is a continuation map from $H$ to $H_VT$ then to the quotient complex determined by $G$, since  $\int u^*\rd \overline{\lambda} \ge 0$ for all relevant Floer cylinders, we know that 
\begin{equation}\label{eqn:vit}
    \phi_{Viterbo}(\overline{\gamma}_{\#})=\overline{\gamma}+\sum c_i\overline{\beta}_i
\end{equation}
for $\overline{\gamma}$ not from the $1$-handle and $\beta_i$ has smaller period than $\gamma$ on $Y_1$. This follows from that $\int u^*\rd \overline{\lambda} =0$ implies that $u$ is contained in $\R\times \Ima \gamma$. The leading coefficient comes from the transversely cut-out trivial cylinder by our choice of Hamiltonian, i.e.\ only depends on the coordinate from the Liouville vector field in the thin region, where $\gamma$ lies. 


\begin{proposition}\label{prop:corb}
 In the case \eqref{thm1} of  \Cref{thm:main'}, we consider Hamiltonian $H$ on the completion of $(Y_1\#Y_2,\lambda|_{\partial_+W})$ with slope $2$. Using the the setup in the discussion above for $Y=Y_1\#Y_2$, there exists $x=\sum a_i\overline{\alpha_i}_\#$, such that 
\begin{enumerate}
    \item\label{c1} $\sum a_i\#\overline{\cM}_{Y,H}(\overline{\alpha_i}_{\#},\overline{\beta})=0$, where $\overline{\beta}$ is a non-constant Hamiltonian orbit; 
    \item\label{c2} $\overline{\cM}_{Y,H}(\overline{\alpha_i}_{\#},\overline{\beta},\Gamma)=\emptyset$ for all $i$ and $\overline{\beta}$ and $\Gamma \ne \emptyset$;
    \item\label{c3} $\overline{\cM}_{Y,H}(\overline{\alpha_i}_{\#},\Lambda,\Gamma)=\emptyset$ for $\Gamma \ne \emptyset$ and $\sum a_i\#\overline{\cM}_{Y,H}(\overline{\alpha_i}_{\#},\Lambda) \ne 0$.
\end{enumerate}
\end{proposition}	
\begin{proof} 
    Let $\overline{\alpha}$ be the orbit in \Cref{prop:curve}, then $\mu_{CZ}(\overline{\alpha})=2$.  We can assume $\dim Y\ge 5$ for otherwise $Y=\partial(V\times \D)$ for a punctured surface $V$,  which will be dealt with in \eqref{thm2} of \Cref{thm:main'}. Since $\mu_{CZ}(\check{\gamma}_h^k),\mu(\hat{\gamma}_{h}^k)\ge \frac{\dim Y+1}{2}\ge 3$, $\phi_{Viterbo}$ is an isomorphism in the $\mu_{CZ}=2$ piece by \eqref{eqn:vit}.  We can take $x$ to be $\phi^{-1}_{Viterbo}(\overline{\alpha})$. As the SFT degrees of orbits (with period up to $2$) on $Y$ are positive, the claims for the moduli spaces $\overline{\cM}_{Y,H}$ are equivalent to the same claims for moduli spaces $\overline{\cM}_{W,H}$ on the cobordism $W$ by neck-stretching, the latter appears in the Viterbo transfer map. That $\overline{\cM}_{W,H}(\overline{\alpha_i}_{\#},\Lambda,\Gamma)=\emptyset$ for $\Gamma \ne \emptyset$ follows from dimension counting as $\mu_{CZ}(\overline{\alpha_i}_\#)=2$ and SFT degrees of orbits in $\Gamma$ are positive. That $\overline{\cM}_{W,H}(\overline{\alpha_i}_{\#},\overline{\beta},\Gamma)=\emptyset$ follows from degree reasons when $Y$ is the boundary of a flexible domain as $\mu_{CZ}(\overline{\beta})\ge 1$ and SFT degree of orbits are positive, and by action reasons when $Y=\partial(V\times \D)$.  Since $\mu_{CZ}(\delta_{\epsilon_0}(x))=1$, where $\phi_{Viterbo}$ is again an isomorphism, then $x$ represents a closed cochain as $\overline{\alpha}$ does.  Finally, the lower bottom of \eqref{eqn:diagram} implies that  $\sum a_i\#\overline{\cM}_{W,H}(\overline{\alpha_i}_{\#},\Lambda) \ne 0$.
\end{proof}

\subsubsection{The second case: $Y_1=\partial(V\times \D)$ for a Weinstein domain $V$ without $c_1$ torsion conditions}In this case, we will use an argument motivated from \cite[Lemma 5.5]{zbMATH06562001} regarding the lower bound of energy from the binding region. Our situation is simpler as we are working with the trivial open book $\partial(V\times \D)$. The contact form used in this paper, i.e.\ from \cite[\S 2,1]{zbMATH07673358}, is from rounding the corner of 
$$\left(\partial(V\times \D) = V\times S^1\cup_{\partial V \times S^1} \partial V \times \D, \lambda_V+\frac{r^2}{2\pi} \rd \theta\right)$$
then put a perturbation on the page region $V\times S^1$. The key Reeb orbit $\alpha$ winds around the bounding $\partial V \times \{0\} \subset \partial V\times \D$ once. 

Now on the binding region $\partial V \times \D$, the contact structure is given by
$$\xi_{\partial V }\oplus \la \partial_x+yR_{\lambda_V},\partial_y-xR_{\lambda_V} \ra,$$
where $\xi_{\partial V}\subset T\partial V$ is the contact structure $\ker \lambda_V$, $R_{\lambda_V}$ is the Reeb vector of $(\partial V, \lambda_V)$ and $x,y$ are coordinates on $\D$. The Reeb vector field on the binding region is $R_{\lambda_V}$. We use the following almost complex structure on $\partial V \times \D \times \R_t$
$$J:\xi_{\partial V} \to \xi_{\partial V} \text{ compatible with } \rd \lambda_V, \quad J(\partial_x+yR_{\lambda_V})=\partial_y-xR_{\lambda_V}, \quad  J(R_{\lambda_V})=\partial_t.$$
We use $\pi_{\D}$ to denote the projection from $\partial V \times \D \times \R_t$ to $\D$. Then it is direct to check that $\pi_{\D}$ is $(J,i)$ holomorphic, where $i$ is the standard complex structure on $\D$. When the Hamiltonian only depends on the $t$-coordinate, the Hamiltonian vector is parallel to $R_{\lambda_V}$. Then if $u$ solves the Floer equation in $\partial V \times \D \times \R_t$
$$\partial_su+J(\partial_tu-X_H)=0,$$
$\pi_{\D} u$ is a holomorphic map to $\D$. Moreover, we have 
\begin{equation}\label{eqn:area}
    \int u^*\rd\lambda \ge \int (\pi_{\D}u)^*\rd (\frac{r^2}{2\pi}\rd \theta)\ge 0.
\end{equation}

Following \cite[\S 2.1]{zbMATH07673358}, we can choose a Morse function $f$ on $V$ such that $\partial_rf\to \infty,f\to 1$ when approaching the boundary of $V$, such that the contact form on $V\times S^1 = \partial(V\times \D)\backslash \partial V \times \D $ is given by 
$\lambda_V +\frac{1}{2\pi f} \rd \theta$, and on $\partial V\times \D$, the contact form is given by $\lambda_V+\frac{r^2}{2\pi}\rd \theta$. Then Reeb orbits with period at most $2$ are simple and correspond to critical points of $f$. Let $\gamma_p$ be the Reeb orbit corresponding to critical point $p$, then the period of $\gamma_p$ is $\frac{1}{f(p)}$. We attach the $1$-handle to the $V\times S^1$ region, such that the following holds:
\begin{equation}\label{eqn:condition}
    \frac{1}{f(p)}-1<\int \gamma_h^*\lambda
\end{equation}
whenever $p$ is not the minimum point of $f$. This can be arranged by first fixing $f$ in a Darboux chart that has a unique minimum, and we attach the $1$-handle in this region without creating orbits of period at most $2$ other than $\gamma_h^k$. Then we extend $f$ to $V$ and push all other critical points (with Morse index at least $1$) to the boundary of $V$ such that \eqref{eqn:condition} hold. We also require that $f$ is self-indexing in the sense that $f(p)\ge f(q)$ if and only if $\ind(p)\ge \ind (q)$ for critical points $p,q$. In the following, we use almost complex structures extending the above almost complex structure on $\partial V \times \D \times \R_t$.

\begin{proposition}\label{prop:area}
    With the setup above, for $Y=Y_1\#Y_2$ and $W$ is the connected sum cobordism, we have the following:
    \begin{enumerate}
        \item\label{a1} $\overline{\cM}_{Y,H}(\overline{\gamma}_p,\overline{\gamma}_h^k)=\emptyset$, $\overline{\cM}_{W,H}(\overline{\gamma}_p,\overline{\gamma}_h^k)=\emptyset$ for any $k$, where $p$ is not the minimum point of $f$.
        \item\label{a2}  The Viterbo transfer map in \eqref{eqn:vit} maps $\overline{\gamma}_h^k$ to zero.
    \end{enumerate}
\end{proposition}
\begin{proof}
     Since the linking number of $\gamma_p$ around the binding $\partial V$ is $1$ and the linking number of $\gamma_h^k$ around $\partial V$ is zero, for any curve $u$ in $\overline{\cM}_{Y,H}(\overline{\gamma}_p,\overline{\gamma}_h^k,\Gamma)$, we must have 
     $$\int \beta^*\lambda-\int(\gamma_h^k)^*\lambda\ge \int_{u^{-1}(\partial V \times \D \times \R_t)}u^*\rd \lambda\ge \int_{u^{-1}(\partial V \times \D \times \R_t)}(\pi_{\D}u)^*(\frac{r^2}{2\pi}\rd \theta)=1,$$
     where the second inequality is \eqref{eqn:area} and the last equality follows from the linking number. Hence we arrive at a contradiction with \eqref{eqn:condition}. 

    The intersection number of the curve $u$ in the Viterbo transfer map from  $\overline{\gamma}_h^k$  with $\partial V$ is negative, as orbits on $Y_1$ have positive linking with $\partial V$. However, this contradicts that $\pi_{\D}u$ is holomorphic, which implies that the intersection number is non-negative.  
\end{proof}

\begin{remark}
    \eqref{eqn:condition} can not be arranged for the minimum point $p$ of $f$, as $\overline{\cM}_{Y,H}(\overline{\gamma}_p,\overline{\gamma}_h)$ is expected to be non-empty in view of subcritical surgery formula for symplectic cohomology \cite{zbMATH01798901}. 
\end{remark}

\begin{proposition}\label{prop:corb'}
     \Cref{prop:corb} holds \eqref{thm2} of \Cref{thm:main'}.
\end{proposition}
\begin{proof}
    Let $\overline{\alpha}$ be the orbit in \Cref{prop:curve} for $Y_1=\partial(V\times \D)$, then $\alpha$ has the minimal period by the self-indexing property. \eqref{eqn:diagram} still holds for the trivial augmentation, as the differentials and continuation maps do not have action room for augmentations.  Then $\phi_{Viterbo}^{-1}(\overline{\alpha})=\overline{\alpha}_\#$  by \eqref{a2} of \Cref{prop:area} and \eqref{eqn:vit}. Then $\overline{\cM}_{Y,H}(\overline{\alpha}_\#,\overline{\beta},\Gamma)=\emptyset$ follows from that $\alpha$ has the minimal period and \Cref{prop:area}, where $\Gamma$ could be the empty set. $\overline{\cM}_{Y,H}(\overline{\alpha}_\#,\Lambda ,\Gamma)=\emptyset$ for $\Gamma \ne \emptyset$ follows from the same reason. We have $\# \overline{\cM}_{W,H}(\overline{\alpha}_\#,\Lambda)\ne 0$ from \eqref{eqn:diagram}. In the neck-stretching along $Y$, a similar argument to \Cref{prop:area} rules out the possibility to develop punctures asymptotics to $\gamma_h^k$ and other punctures are ruled out as $\alpha$ has the minimal period, we have $\# \overline{\cM}_{Y,H}(\overline{\alpha}_\#,\Lambda)\ne 0$. 
\end{proof}


\begin{proof}[Proof of \Cref{thm:main'}]
	It follows from the same argument of \Cref{prop:corbordism} using \Cref{prop:corb,prop:corb'}.
\end{proof}	

\begin{proof}[Proof of \Cref{rmk:twist}]
We consider curves in $W_{\Lambda}$ recording the homology classes. Proofs of \Cref{thm:main,thm:main'} imply the vanishing of contact homology with  twisted coefficient $\Q[H_2(Y_{\Lambda};\R)]$ if $H_2(Y_{\Lambda};\R)\to H_2(W_{\Lambda};\R)$ is injective. If $\dim Y\ne 5$, we have $H_3(W_{\Lambda},Y_{\Lambda};\R)=0$, the long exact sequence for pairs implies the injectivity of $H_2(Y_{\Lambda};\R)\to H_2(W_{\Lambda};\R)$. When $\dim Y=5$, since $\Lambda$ is homological non-trivial over $\Q$, we have $H_3(W_{\Lambda},Y;\R)=0$, hence by universal coefficient theorem and Lefschetz duality, we have $H_3(W_{\Lambda},Y_{\Lambda};\R)\simeq H^3(W_{\Lambda},Y_{\Lambda};\R)=H_3(W_{\Lambda},Y;\R)=0$.
\end{proof}	



\subsection{Functorial explanation}\label{ss:43}
The proof of \Cref{thm:main} has a functorial explanation as follows. Let $H$ be the Hamiltonian on $\widehat{Y}$ which is zero on $(0,1)_r\times Y$ and has slope $a$ on $(1,\infty)_r\times Y$. Then the counting of $\overline{\cM}_{Y,H}(x,y,\Gamma)$ makes $C^{-*}_+(H)\otimes \CC_*(Y)$ into a $\CC_*(Y)$-DGA module, where the differential on $C^{-*}_+(H)\otimes \CC_*(Y)$ is given by 
$$\delta(x\otimes w) = (-1)^{|x|}x\otimes \partial_{\CH}(w)+ \sum_{[\gamma],y}\frac{1}{\mu_{\Gamma}\kappa_{\Gamma}} \# \overline{\cM}_{Y,H}(x,y,\Gamma) y\otimes q^{\Gamma}\cdot w$$
for $x\in C_+^{-*}(H)$ and $w\in \CC_*(Y)$. Then by counting $\overline{\cM}_{Y,H}(x,C,\Gamma)$, we get a $CC_*(Y)$-DGA module map $C^{-*}_+(H)\otimes\CC_*(Y)\to C^{-*+1}(Y)\otimes \CC_*(Y)$, where $C^{-*+1}(Y)\otimes \CC_*(Y)$ is the trivial $CC_*(Y)$ DGA module and $C^*(Y)$ is the (Morse) cochain complex of $Y$. Symplectic cohomology w.r.t.\ an augmentation $\epsilon$ in \ref{ss:aug}, is the tensor product $(C_+^{-*}(H)\otimes CC_*(Y))\otimes_{\CC_*(Y)} \Q$, where $\Q$ is considered as $CC_*(Y)$ module using the augmentation $\epsilon$.

Now given an exact cobordism $W$ from $Y_-$ to $Y_+$, by counting $\overline{\cM}_{H,W}(x,C,\Gamma)$ similar to \eqref{eqn:WH}, we get a $CC_*(Y_+)$-DGA module map $C^{-*}_+(H)\otimes\CC_*(Y_+)\to C^{-*+1}(W)\otimes \CC_*(Y_-)$, where $\CC_*(Y_-)$ is viewed as a $CC_*(Y_+)$-DGA module map by the DGA morphism $\Phi_{W}:CC_*(Y_+)\to \CC_*(Y_-)$ from the cobordism $W$. Then we have the following diagram of $\CC_*(Y_+)$-DGA module maps, which on homology is commutative by a similar argument to \cite[Proposition 3.2]{zbMATH07367119}. 
$$
\xymatrix{
C_+^{-*}(H)\otimes \CC_*(Y_+)\ar[d] \ar[r] & C^{-*+1}(Y_+)\otimes \CC_*(Y_+)\ar[d]\\
C^{-*+1}(W) \otimes \CC_*(Y_-)\ar[r] & C^{-*+1}(Y_+)\otimes \CC_*(Y_-)}
$$
Then \Cref{prop:curve,prop:corb,prop:corb'} imply that $\overline{\alpha}\otimes 1$ is closed in $C_+^{-*}(H)\otimes \CC_*(Y_+)$, which is mapped to $\delta_{\partial}(\overline{\alpha})\otimes 1\in H_*(C^{-*+1}(Y_+)\otimes \CC_*(Y_-))$. If $H^*(W;\Q)\to H^*(Y_+;\Q)$ does not hit $\delta_{\partial}(\overline{\alpha})$, then we must have $1=0\in H_*(\CC(Y_-))$. This is the case of the surgery cobordism in \Cref{thm:main,thm:main'}.




\subsection{Infinite non-loose Legendrians}
By the work of Lazarev \cite{zbMATH07305775}, any Weinstein structure on $\C^n$ for $n\ge 3$ can be obtained as attaching a critical handle to a Legendrian sphere in $(\partial(T^*S^{n-1}\times \D),\xi_{\std})\simeq S^{n-1}\times S^n$, where the embedding of the Legendrian is in the same homotopy class as $S^{n-1}\stackrel{\simeq}{\to} S^{n-1}\times \{p\}\subset S^{n-1}\times S^n$. When the Weinstein structure is standard, the attaching Legendrian is the loose one $\Lambda_{loose}$, i.e.\ the Legendrian lift of the Lagrangian zero section in $T^*S^{n-1}$. 

A formal Legendrian submanifold of a contact manifold $(Y^{2n-1},\xi)$ consists of an embedding of a $n-1$ dimensional manifold $f:\Lambda\subset Y$ and a family of injective bundle maps $F_s:T\Lambda \to  f^*TY $ for $s\in [0,1]$, such that $F_0=\rd f$ and $F_1$'s images are Lagrangian subspace of $f^*\xi$ w.r.t.\ the natural conformal symplectic structure, or equivalently (up to homotopy) a totally real subspace for a fixed almost complex structure on $\xi$ compatible with the conformal symplectic structure. Formal Legendrian isotopy is an isotopy of those data. Given two Legendrian embeddings $f_0,f_1:\Lambda \to Y$, if they are smoothly isotopic by $f_s$, then the Legendrian embedding $f_1$ is formally Legendrian isotopic to $(f_0,F_s)$, where $F_s = P_s\rd f_s$, where $P_s$ is the parallel transportation from $f_s(p)$ back to $f_0(p)$ using an auxiliary complex connection on $\xi$. Therefore $F_1$ and $\rd f_0$ are differed by a section $s$ of the bundle $U(f_0^*\xi)$ of unitary automorphisms. If this section is homotopic to the identity section, then $(f_0,F_s)$, hence $(f_1,\rd f_1)$, is formally Legendrian isotopic to  $(f_0,\rd f_0)$. Now we can consider the situation of Legendrian spheres $f_0,f_1:S^{n-1}\to Y$, then $f_0^*\xi$ is isomorphic to $i^*TT^*S^{n-1}$ for the zero section $i:S^{n-1}\to T^*S^{n-1}$, i.e.\ the complexification $T_{\C}S^{n-1}$ of $TS^n$. Note that $T_{\C}S^{n-1}\oplus \underline{\C}\simeq \underline{\C}^{n}$, we get a inclusion of $U(T_{\C}S^{n-1})$ to a trivial $U(n)$ bundle. Since $U(n-1)\subset U(n)$ is isomorphic on the $2n-2$ skeleton, whether $s$ is homotopic to identity in  $U(T_{\C}S^{n-1})$ is the same as whether $s$ is homotopic to identity in the trivial $U(n)$ bundle whenever $n-1< 2n-2$, i.e.\ $n>1$. Finally, such a section is always homotopic to identity by Bott periodicity as $\pi_{n-1}(U(n))=0$ for $n>1$ odd.

\begin{proposition}
    Let $W_1,W_2$ be two different Weinstein structures on $\C^n$ for $n\ge 3$ that are different from the standard one. Then the corresponding attaching Legendrian spheres are non-loose, not Legendrian isotopic, smoothly isotopic, and formally Legendrian isotopic when $n$ is odd.
\end{proposition}
\begin{proof}
    If the attaching Legendrian is loose, then the resulting Weinstein structure must be the standard Weinstein structure by \cite{zbMATH06054083}. As they are different Weinstein structures, the attaching Legendrian can not be Legendrian isotopic. Finally, applying the smooth Whitney trick, one sees that the attaching Legendrian is smoothly isotopic to $\Lambda_{loose}$. And by the discussion above it is formally Legendrian isotopic to $\Lambda_{loose}$ if $n$ is odd. 
\end{proof}



In view of the proposition above, \Cref{prop:exotic} follows from the existence of infinitely many exotic Weinstein structures on $\C^n$ for $n\ge 3$. Such structures were first found by Seidel and Smith \cite{zbMATH02242665} for $n$ even, then infinitely many for $n$ even by McLean \cite{zbMATH05553983}. The most flexible and efficient way of constructing infinite such exotic structures for any $n\ge 3$ was due to Abouzaid and Seidel \cite{abouzaid2010altering}. Their exoticity and differences are both illustrated using symplectic cohomology.