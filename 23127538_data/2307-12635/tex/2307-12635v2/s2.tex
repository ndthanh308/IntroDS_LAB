\section{Contact $(+1)$-surgeries}
In this section, we review contact $(+1)$-surgeries following Conway-Etnyre \cite{zbMATH07206659}. The definition of a contact $(+1)$–surgery along a Legendrian sphere is implicit in
the theory of Weinstein handle attachments \cite{zbMATH06054083,zbMATH04147116,zbMATH00011093}. We first recall a model for the Weinstein $k$-handle for $k\le n$ as follows. Let $\omega = \sum_{i=1}^k\rd q_i\wedge \rd p_i + \sum_{i=1}^{n-k}\rd x_i\wedge \rd y_i$ be the standard symplectic structure on $\R^{2k}\times \R^{2n-2k}$, then we have the following Liouville vector field 
$$v=\sum_{i=1}^k(-p_i\partial_{p_i}+2q_i\partial_{q_i})+\frac{1}{2}\sum_{i=1}^{n-k}(x_i\partial_{x_i}+y_i\partial_{y_i}).$$
For $a,b>0$, let $D_a$ be the radius $a$ disk in the $p_i$ subspace and $D_b$ the radius $b$ disk in the $q_i,x_i,y_i$ subspace. Then $(H_{a,b}=D_a\times D_b,\omega)$ is a model for the  Weinstein $k$-handle. The Liouville vector points out along $\partial_+H_{a,b}:=D_a\times \partial D_b$ and points in along $\partial_-H_{a,b}:=\partial D_a \times D_b$, which in particular determines contact structures on the boundary.

Note that $S_a=\partial D_a \times \{0\}$ is isotropic sphere and $S_b=\{0\} \times \partial D_b$ is a coisotropic sphere. By Moser's trick, the germ of contact structure along $S_a\subset \partial_-H_{a,b}$ is contactomorphic to the germ of a contact structure along an isotropic sphere with a trivial conformal symplectic normal bundle (the quotient of the symplectic orthogonal of the tangent bundle by the tangent bundle) in any contact manifold, and the germ of contact structure along $S_b\subset \partial_+H_{a,b}$ is contactomorphic to the germ of a contact structure along a coisotropic sphere in any contact manifold \cite[Lemma 3.4]{zbMATH07206659}. As a consequence, given an isotropic sphere with a trivialization of the conformal symplectic normal bundle in a contact manifold $Y$, we can glue $H_{a,b}$ to a neighborhood of the isotropic sphere along $\partial_-H_{a,b}$ for $b\ll 1$ using the Liouville vector field. This yields a Weinstein cobordism with concave boundary $Y$ and a new convex boundary $Y'$, we call this procedure an isotropic surgery. On the other hand, given a coisotropic sphere in $Y$, we can glue $H_{a,b}$ to a neighborhood of the coisotropic sphere along $\partial_+H_{a,b}$ for $a\ll 1$. This yields a Weinstein cobordism with convex boundary $Y$ and a new concave boundary $Y''$. This procedure is referred to as a coisotropic surgery. The isotropic surgery and coisotropic surgery are reverse operations to each other, i.e.\ we can undo the isotropic/coisotropic surgery by applying a coisotropic/isotropic surgery to the coisotropic/isotropic sphere in the surgery handle \cite[Lemma 3.9, Proposition 3.10]{zbMATH07206659}. 

By the functoriality of contact homology, only coisotropic surgeries has a chance to kill the contact homology. However, unlike the situation for isotropic spheres $S^{k-1}$ for $k<n$, coisotropic spheres do not enjoy an $h$-principle \cite[Theorem 1.3 and Remark 6.1]{zbMATH06670705}, which makes the existence of such manifolds much harder to identify. The situation for Legendrian spheres, i.e.\ both isotropic and coisotropic spheres, is a mixture of flexibility and rigidity, i.e.\ we have an $h$-principle for loose Legendrians by Murphy \cite{MR4172336}, yet an enormous class of non-loose Legendrians exhibiting various forms of symplectic rigidity. 

\begin{definition}
A contact $(+1)$-surgery along a Legendrian sphere $\Lambda \in Y$ is a coisotropic surgery along the Legendrian. We use $Y_{\Lambda}$ to denote the resulted contact manifold, and $W_{\Lambda}$ to denote the surgery Weinstein cobordism from $Y_{\Lambda}$ to $Y$.  
\end{definition}

\begin{remark}
    The $(+1)$-surgery depends on a parametrization $S^{n-1}\simeq \Lambda$, the resulted contact manifold, even the underlying smooth manifold, depends on this parametrization in general. We suppress this choice in our theorems, as the vanishing of contact homology does not depend on the parametrization.
\end{remark}

In \cite{zbMATH07455583}, Avdek gave an alternative definition of $(+1)$ surgery on Legendrian spheres, based around a Dehn-Seidel twist. Such definition was used in \cite{zbMATH07010365}, where Casals, Murphy, and Presas show that the result of $(+1)$ surgery on a loose Legendrian knot is overtwisted.