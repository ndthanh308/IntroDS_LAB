% ****** Start of file apssamp.tex ******
%
%   This file is part of the APS files in the REVTeX 4.2 distribution.
%   Version 4.2a of REVTeX, December 2014
%
%   Copyright (c) 2014 The American Physical Society.
%
%   See the REVTeX 4 README file for restrictions and more information.
%
% TeX'ing this file requires that you have AMS-LaTeX 2.0 installed
% as well as the rest of the prerequisites for REVTeX 4.2
%
% See the REVTeX 4 README file
% It also requires running BibTeX. The commands are as follows:
%
%  1)  latex apssamp.tex
%  2)  bibtex apssamp
%  3)  latex apssamp.tex
%  4)  latex apssamp.tex
%
\documentclass[%
 preprint,
3p,times,onecolumn]{elsarticle}
\pdfoutput=1
\geometry{left=0.52in, top=0.52in, right=0.52in, bottom=1in} 
\usepackage{etoolbox}
\usepackage[utf8]{inputenc}
\usepackage[colorlinks]{hyperref}
%\usepackage{cite}
\usepackage{graphicx}
\usepackage{natbib}
\usepackage{tabularx}
\usepackage{lipsum}
\usepackage{wrapfig}
\usepackage{amsmath}
\usepackage{float}
\usepackage{subfigure}
\usepackage{gensymb}
\usepackage{hyperref}
\usepackage{cleveref}
\usepackage{chngcntr}
\usepackage{multirow}
\usepackage{makecell}
\usepackage{array}
\usepackage{booktabs}
\usepackage[none]{hyphenat} %% no words with hyphens
\usepackage{enumitem}
\usepackage{ragged2e}
\usepackage{amssymb}
\usepackage[section]{placeins}
\usepackage{siunitx}
\usepackage{epigraph}
\usepackage{lineno}
\usepackage{braket}
\usepackage{xparse,nameref}
\usepackage[usestackEOL]{stackengine}
\usepackage{xcolor}
\usepackage{mathtools}
\usepackage{multirow}
\usepackage{makecell}
\usepackage{array}
\usepackage{booktabs}
\usepackage{textcase}
\usepackage[toc,page]{appendix}
\usepackage{pdfpages}
\usepackage{titlesec}
\newcommand{\secref}[1]{Sec.~\ref{sec:#1}}
\newcommand{\figref}[1]{Figure~\ref{fig:#1}}
\newcommand{\tabref}[1]{Table~\ref{tab:#1}}
\newcommand{\eqnref}[1]{\eqref{eq:#1}}
\newcommand{\thmref}[1]{Theorem~\ref{#1}}
\newcommand{\prgref}[1]{Program~\ref{#1}}
\newcommand{\algref}[1]{Algorithm~\ref{#1}}
\newcommand{\clmref}[1]{Claim~\ref{#1}}
\newcommand{\lemref}[1]{Lemma~\ref{#1}}
\newcommand{\ptyref}[1]{Property~\ref{#1}}
\titleformat{\subsection} {\normalfont\bfseries}{\thesubsection}{1em}{}
%\titleformat{\paragraph} {\normalfont\bfseries\itshape}{\paragraph}{1em}{}
\titleformat{\paragraph} {\normalfont\itshape}{\paragraph}{1em}{}
\raggedbottom

\usepackage{dcolumn}   % needed for some tables
% Extra vertical space in tables
\newcommand\Tstrut{\rule{0pt}{2.4ex}}       % top strut
\newcommand\Bstrut{\rule[-1.3ex]{0pt}{0pt}} % bottom strut

\newcommand\TstrutLarge{\rule{0pt}{3.6ex}}       % top strut
\newcommand\BstrutLarge{\rule[-2.3ex]{0pt}{0pt}} % bottom strut

\newcolumntype{M}[1]{>{\centering\arraybackslash}m{#1}}

\usepackage[tablename=TABLE, labelsep=period, figurename=Figure, skip=0pt, font=small]{caption}
\renewcommand\thetable{\Roman{table}}

% Set proper line breaking
\setlength\emergencystretch{\hsize}\hbadness=10000

% Special footnote for the deceased colleagues
% \usepackage{etoolbox}
% \makeatletter
%     %replace first instance (first tnote)
%     \patchcmd{\tnotemark}{\ding{73}}{\dag}{}{\@latex@error{Failed to path \string\tnotemark\space for \string\ding{73}}}
%     %replace second instance (second tnote)
%     \patchcmd{\tnotemark}{\ding{73}\ding{73}}{\dag\dag}{}{\@latex@error{Failed to path \string\tnotemark\space for \string\ding{73}\string\ding{73}}}
%     %replace first instance (first tnote)
%     \patchcmd{\tnotetext}{\ding{73}}{\dag}{}{\@latex@error{Failed to path \string\tnotetext\space for \string\ding{73}}}
%     %replace second instance (second tnote)
%     \patchcmd{\tnotetext}{\ding{73}\ding{73}}{\dag\dag}{}{\@latex@error{Failed to path \string\tnotetext\space for \string\ding{73}\string\ding{73}}}
% \makeatother

%remove preprint submitted to elsevier
\makeatletter
\def\ps@pprintTitle{%
 \let\@oddhead\@empty
 \let\@evenhead\@empty
 \def\@oddfoot{}%
 \let\@evenfoot\@oddfoot}
\makeatother

\begin{document}
  %\begin{frontmatter}
       
   \title{\textbf{Final results of Borexino on CNO solar neutrinos}}

\author[Milano]{D.~Basilico}
\author[Milano]{G.~Bellini}
\author[PrincetonChemEng]{J.~Benziger}
\author[LNGS]{R.~Biondi\fnref{MPI}}
\author[Milano]{B.~Caccianiga}
\author[Princeton]{F.~Calaprice}
\author[Genova]{A.~Caminata}
\author[Lomonosov]{A.~Chepurnov}
\author[Milano]{D.~D'Angelo}
\author[Peters]{A.~Derbin}
\author[LNGS]{A.~Di Giacinto}
\author[LNGS]{V.~Di Marcello}
\author[Princeton]{X.F.~Ding\fnref{IHEP}}
\author[Princeton]{A.~Di Ludovico\fnref{LNGSG}} 
\author[Genova]{L.~Di Noto}
\author[Peters]{I.~Drachnev}
\author[APC]{D.~Franco}
\author[Princeton,GSSI]{C.~Galbiati}
\author[LNGS]{C.~Ghiano}
\author[Milano]{M.~Giammarchi}
\author[Princeton]{A.~Goretti\fnref{LNGSG}}
\author[Lomonosov,Dubna]{M.~Gromov}
\author[Mainz]{D.~Guffanti\fnref{Bicocca}}
\author[LNGS]{Aldo~Ianni}
\author[Princeton]{Andrea~Ianni}
\author[Krakow]{A.~Jany}
\author[Kiev]{V.~Kobychev}
\author[London,Atomki]{G.~Korga}
\author[Juelich,RWTH]{S.~Kumaran\fnref{CALI}}
\author[LNGS]{M.~Laubenstein}
\author[Kurchatov,Kurchatovb]{E.~Litvinovich}
\author[Milano]{P.~Lombardi}
\author[Peters]{I.~Lomskaya}
\author[Juelich,RWTH]{L.~Ludhova}
\author[Kurchatov,Kurchatovb]{I.~Machulin}
\author[Mainz]{J.~Martyn}
\author[Milano]{E.~Meroni}
\author[Milano]{L.~Miramonti}
\author[Krakow]{M.~Misiaszek}
\author[Peters]{V.~Muratova}
\author[Kurchatov]{R.~Nugmanov}
\author[Munchen]{L.~Oberauer}
\author[Mainz]{V.~Orekhov}
\author[Perugia]{F.~Ortica}
\author[Genova]{M.~Pallavicini}
\author[Juelich,RWTH]{L.~Pelicci}
\author[Juelich]{\"O.~Penek\fnref{GSI}}
\author[Princeton]{L.~Pietrofaccia\fnref{LNGSG}}
\author[Peters]{N.~Pilipenko}
\author[UMass]{A.~Pocar}
\author[Kurchatov]{G.~Raikov}
\author[LNGS]{M.T.~Ranalli}
\author[Milano]{G.~Ranucci}
\author[LNGS]{A.~Razeto}
\author[Milano]{A.~Re}
\author[LNGS]{N.~Rossi}
\author[Munchen]{S.~Sch\"onert}
\author[Peters]{D.~Semenov}
\author[Juelich]{G.~Settanta\fnref{ISPRA}}
\author[Kurchatov,Kurchatovb]{M.~Skorokhvatov}
\author[Juelich,RWTH]{A.~Singhal}
\author[Dubna]{O.~Smirnov}
\author[Dubna]{A.~Sotnikov}
\author[LNGS]{R.~Tartaglia}
\author[Genova]{G.~Testera}
\author[Peters]{E.~Unzhakov}
\author[LNGS,Aquila]{F.L.~Villante}
\author[Dubna]{A.~Vishneva}
\author[Virginia]{R.B.~Vogelaar}
\author[Munchen]{F.~von~Feilitzsch}
\author[Krakow]{M.~Wojcik}
\author[Mainz]{M.~Wurm}
\author[Genova]{S.~Zavatarelli}
\author[Dresda]{K.~Zuber}
\author[Krakow]{G.~Zuzel}

\fntext[LNGSG]{Present affiliation: INFN Laboratori Nazionali del Gran Sasso, 67010 Assergi (AQ), Italy}
\fntext[ISPRA]{Present affiliation: Istituto Superiore per la Protezione e la Ricerca Ambientale, 00144 Roma, Italy}
\fntext[Bicocca]{Present affiliation: Dipartimento di Fisica, Università degli Studi e INFN Milano-Bicocca, 20126 Milano, Italy}
\fntext[IHEP]{Present affiliation: IHEP Institute of High Energy Physics, 100049 Beijing, China}
\fntext[GSI]{Present affiliation: GSI Helmholtzzentrum für Schwerionenforschung GmbH, 64291 Darmstadt, Germany}
\fntext[CALI]{Present affiliation: Department of Physics and Astronomy, University of California, Irvine, California, USA}
\fntext[MPI]{Present affiliation: Max-Planck-Institut für Kernphysik, 69117 Heidelberg, Germany}
\address{\bf{The Borexino Collaboration}}

\address[APC]{AstroParticule et Cosmologie, Universit\'e Paris Diderot, CNRS/IN2P3, CEA/IRFU, Observatoire de Paris, Sorbonne Paris Cit\'e, 75205 Paris Cedex 13, France}
\address[Dubna]{Joint Institute for Nuclear Research, 141980 Dubna, Russia}
\address[Genova]{Dipartimento di Fisica, Universit\`a degli Studi e INFN, 16146 Genova, Italy}
\address[Krakow]{M.~Smoluchowski Institute of Physics, Jagiellonian University, 30348 Krakow, Poland}
\address[Kiev]{Institute for Nuclear Research of NASU, 03028 Kyiv, Ukraine}
\address[Kurchatov]{National Research Centre Kurchatov Institute, 123182 Moscow, Russia}
\address[Kurchatovb]{ National Research Nuclear University MEPhI (Moscow Engineering Physics Institute), 115409 Moscow, Russia}
\address[LNGS]{INFN Laboratori Nazionali del Gran Sasso, 67010 Assergi (AQ), Italy}
\address[Milano]{Dipartimento di Fisica, Universit\`a degli Studi e INFN, 20133 Milano, Italy}
\address[Perugia]{Dipartimento di Chimica, Biologia e Biotecnologie, Universit\`a degli Studi e INFN, 06123 Perugia, Italy}
\address[Peters]{St. Petersburg Nuclear Physics Institute NRC Kurchatov Institute, 188350 Gatchina, Russia}
\address[Princeton]{Physics Department, Princeton University, Princeton, NJ 08544, USA}
\address[PrincetonChemEng]{Chemical Engineering Department, Princeton University, Princeton, NJ 08544, USA}
\address[UMass]{Amherst Center for Fundamental Interactions and Physics Department, University of Massachusetts, Amherst, MA 01003, USA}
\address[Virginia]{Physics Department, Virginia Polytechnic Institute and State University, Blacksburg, VA 24061, USA}
\address[Munchen]{Physik-Department, Technische Universit\"at  M\"unchen, 85748 Garching, Germany}
\address[Lomonosov]{Lomonosov Moscow State University Skobeltsyn Institute of Nuclear Physics, 119234 Moscow, Russia}
\address[GSSI]{Gran Sasso Science Institute, 67100 L'Aquila, Italy}
\address[Dresda]{Department of Physics, Technische Universit\"at Dresden, 01062 Dresden, Germany}
\address[Mainz]{Institute of Physics and Excellence Cluster PRISMA+, Johannes Gutenberg-Universit\"at Mainz, 55099 Mainz, Germany}
\address[Juelich]{Institut f\"ur Kernphysik, Forschungszentrum J\"ulich, 52425 J\"ulich, Germany}
\address[RWTH]{III. Physikalisches Institut B, RWTH Aachen University, 52062 Aachen, Germany}
\address[London]{Department of Physics, Royal Holloway University of London, Egham, Surrey,TW20 0EX, UK}
\address[Atomki]{Institute of Nuclear Research (Atomki), Debrecen, Hungary}
\address[Aquila]{Dipartimento di Scienze Fisiche e Chimiche, Universit\`a dell'Aquila, 67100 L'Aquila, Italy}

% ABSTRACT LENGTH
%PRD 	About 5% of article length & < 500 words
% source: https://journals.aps.org/prd/authors

\begin{abstract}

In this paper, we report the first measurement of CNO solar neutrinos by Borexino that uses the Correlated Integrated Directionality (CID) method, exploiting the sub-dominant Cherenkov light in the liquid scintillator detector. The directional information of the solar origin of the neutrinos is preserved by the fast Cherenkov photons from the neutrino scattered electrons, and is used to discriminate between signal and background. %Cherenkov photons are emitted promptly and can be separated from the dominant but slower scintillation light. 
The directional information is independent from the spectral information on which the previous CNO solar neutrino measurements by Borexino were based. While the CNO spectral analysis could only be applied on the Phase-III dataset, the directional analysis can use the complete Borexino data taking period from 2007 to 2021. The absence of CNO neutrinos has been rejected with $>$5$\sigma$ credible level using the Bayesian statistics. The directional CNO measurement is obtained without an external constraint on the $^{210}$Bi contamination of the liquid scintillator, which was applied in the spectral analysis approach. The final and the most precise CNO measurement of Borexino is then obtained by combining the new CID-based CNO result with an improved spectral fit of the Phase-III dataset. Including the statistical and the systematic errors, the extracted CNO interaction rate is $R\mathrm{(CNO)}=6.7^{+1.2}_{-0.8} \, \mathrm{cpd/100\, tonnes}$. Taking into account the neutrino flavor conversion, the resulting CNO neutrino flux at Earth is $\Phi_\mathrm{CNO}=6.7 ^{+1.2}_{-0.8}  \times 10^8 \, \mathrm{cm^{-2} s^{-1}}$, which is found to be in agreement with the high metallicity Standard Solar Models.
The results described in this work reinforce the role of the event directional information in large-scale liquid scintillator detectors and open up new avenues for the next-generation liquid scintillator or hybrid neutrino experiments. A particular relevance is expected for the latter detectors, which aim to combine the advantages from both Cherenkov-based and scintillation-based detection techniques.


   \end{abstract}

  %\end{frontmatter}
  
\maketitle

    \twocolumn 
    \tableofcontents

\section{\label{sec:Intro}Introduction}

Solar neutrinos are produced in the core of the Sun by nuclear reactions in which hydrogen is transformed into helium. The dominant sequence of reactions is the so-called $pp$ chain~\cite{Bahcall:1989ks,Vinyoles_New_Gen_SSM} which is responsible for most of the solar luminosity, while approximately 1\,\% of the solar energy is produced by the so-called Carbon-Nitrogen-Oxygen (CNO) cycle. Even though the CNO cycle plays only a marginal role in the solar fusion mechanisms, it is expected to take over the luminosity budget for main sequence stars more massive, older, and hotter than the Sun~\cite{Salaris}.
Solar neutrinos have proven to be a powerful tool to study the solar core ~\cite{Bx_nature_CNO,Bx_Nature_2018,superK_solar_neutrino_IV,SNOPlus_B8} and, at the same time, have been of paramount importance in shedding light on the neutrino oscillation phenomenon ~\cite{Cleveland:1998nv,SAGE:2009eeu,Kaether:2010ag,KamLAND:2004mhv,PhysRevLett.89.011301,Bellini:2011rx}. 


One important open question concerning solar physics regards the metallicity of the Sun, that is, the abundance of elements with $Z>2$.
In fact, different analyses of spectroscopic data yield significantly different metallicity results, that can be grouped in two classes: the so-called High-Metallicity (HZ) \cite{HZ,HZ1} and Low-Metallicity (LZ) \cite{LZ,LZ1,LZ2} models.
The solar neutrino fluxes, in particular that from the CNO cycle reactions, can address this issue. Indeed, the SSM predictions of the CNO neutrino flux depend on the solar metallicity directly, via the abundances of C and N in the solar core, and indirectly, via its effect on the solar opacity and  temperature profile. 

Borexino delivered the first direct experimental proof of the existence of the CNO cycle in the Sun with a significance of $\sim$7\,$\sigma$, also providing a slight preference towards High-Metallicity models ~\cite{Bx_nature_CNO, Bx_improved_CNO}. 
This result was obtained with a multivariate analysis of the energy and radial distributions of selected events. To disentangle the CNO signal from the background, the multivariate fit requires an independent external constraint on the $pep$ neutrino rate and on the $^{210}$Bi rate; the latter is obtained by tagging $^{210}$Bi-$^{210}$Po coincidences in a temperature stabilized, layered scintillator fluid (see \cite{Bx_nature_CNO, Bx_improved_CNO} for more details).
For this reason, the CNO measurement has been performed only on approximately one third of the Borexino data, the so-called Phase-III.

In this paper, we present new results on CNO neutrinos obtained exploiting the "Correlated and Integrated Directionality" (CID) technique, which uses the directional information encoded in the Cherenkov light emitted alongside the scintillation, to separate the solar signal from non-solar backgrounds.
Borexino demonstrated the viability of this technique using $^7$Be solar neutrinos \cite{Bx_CID_long, Bx_CID_short}.
Here we apply the CID technique to the CNO analysis, obtaining two important results: we show that we can extract the evidence of solar CNO neutrinos on the entire Borexino dataset following an alternative approach with respect to the standard multivariate analysis and, consequently, without the help of the $^{210}$Bi constraint; we also show that by combining the information coming from the directionality with the standard multivariate analysis performed on Phase-III data we obtain an improved measurement of the CNO neutrino interaction rate.



%In this paper, we report the Correlated and Integrated Directionality (CID) ~\cite{Bx_CID_short,Bx_CID_long} analysis results for the measurement of the CNO neutrino signal. This approach is different from the one adopted for previous Borexino analyses for CNO measurement, which instead relied on the multivariate, spectral analysis with an independent $^{210}$Bi rate constraint~\cite{Bx_nature_CNO,Bx_improved_CNO}.
%The basic idea behind the CID technique is to exploit the different emission time profiles between the sub-dominant Cherenkov photons and the emission of scintillation photons, which are generated with slower characteristic times. The Cherenkov photons are produced almost instantaneously and retain the directional information of the recoil electron, while the scintillation photons are emitted isotropically thanks to molecular de-excitations, with decay times of the order of nanoseconds. For each of the events in a dataset, the hit pattern of the first few detected photons is correlated with the position of the Sun, leading to an angular distribution between the hit PMT and the well-known solar direction. 
%This distribution is expected to show a characteristic forward-peaked signature for neutrinos, corresponding to the Cherenkov cone angle in the liquid scintillator.
%The radioactive backgrounds which are uncorrelated to the Sun are expected to show a flat angular distribution. 
%In this way, the CNO neutrino signal can be efficiently disentangled from the residual background. It is worth to note that this technique does not require any a priori information on the $^{210}$Bi background rate. 

%In addition, the results obtained with the CID technique can be included into the improved multivariate analysis approach, leading to the most precise CNO neutrino measurement ever achieved by Borexino.

The paper is structured as follows.
Section~\ref{sec:bxdet} describes the Borexino detector and summarizes the event reconstruction techniques. The CID analysis for the CNO neutrino measurement is illustrated in Sec.~\ref{sec:CID}, outlining the methods, reporting the results, and detailing the main sources of systematic uncertainties.
Finally, in Sec.~\ref{sec:MV} we show our best result on CNO neutrinos obtained combining the CID and the standard multivariate analysis.

\section{The Borexino experiment}

\label{sec:bxdet}


% Figure environment removed

Borexino was a liquid scintillator (LS) neutrino detector~\cite{bx_det} that ran until October 2021 with unprecedented radiopurity levels~\cite{Bx_long_paper,Bx_Nature_2018}, a necessary feature of its solar neutrino measurements.
The detector was located deep underground at the Laboratori Nazionali del Gran Sasso (LNGS) in Italy, with about \SI{3800}{m} water equivalent rock shielding suppressing the cosmic muon flux by a factor of $\sim$$10^6$. 

The detector layout is schematically shown in Fig.~\ref{fig:bx_det}. The Stainless Steel Sphere (SSS) with a 6.85\,m radius supported 2212 8-inch photomultiplier tubes (PMTs) and contained 280\,tonnes of pseudocumene (1,2,4-trimethylbenzene, PC) doped with 1.5\% of PPO (2,5-diphenyloxazole) wavelenght shifter, confined in a nylon inner vessel of 4.25\,m radius. The density of the scintillator was (0.878 $\pm$ 0.004)\,g\,cm$^{-3}$ with the electron density of (3.307 $\pm$ 0.015) $\times$ $10^{31}$ $e^-/100$\,tonnes. 
The PC-based buffer liquid in the region between the SSS and IV shielded the LS from external $\gamma$ radiation and neutrons.
The nylon Outer Vessel, that separated the buffer in two sub-volumes, prevented the inward diffusion of $^{222}$Rn.  The SSS itself is submerged in a domed, cylindrical tank filled with $\sim$1\,kton of ultra-pure water, equipped with 208 PMTs. The water tank provided shielding against external backgrounds and also served as an active Cherenkov veto for residual cosmic muons passing through the detector. 

Borexino detected solar neutrinos via their elastic scattering on electrons of the LS, a process sensitive, with different probability, to all neutrino flavors. Electrons, and charged particles in general, deposit their energy in the LS, excite its molecules and the resulting scintillation light is emitted isotropically. Using $n $$\approx$1.55 as scintillator index of refraction at 400\,nm wavelength, sub-dominant but directional Cherenkov light is emitted when the electron kinetic energy exceeds 0.165\,MeV. Cherenkov light is emitted over picosecond timescale while the fastest scintillation light component from the LS has an emission time constant at the nanosecond level. The fraction of light emitted as Cherenkov photons in Borexino was less than 0.5\% for 1 MeV recoiling electrons.


The effective total light yield was $\sim$500 photoelectrons per MeV of electron equivalent deposited energy, normalized to 2000 PMTs~\cite{Bx_long_paper}. 
The energy scale is intrinsically non-linear due to ionization quenching and the emission of Cherenkov radiation. The \textsc{Geant4} based Monte Carlo (MC) software \cite{Bx_monte_carlo} simulates all relevant physics processes. It is tuned using the data obtained during calibration campaigns with radioactive sources~\cite{Bx_calibration}. 
Distinct energy estimators have been defined, based on different ways of counting the number of detected photons~\cite{Bx_long_paper, Bx_CID_long}. 
The position reconstruction of each event is performed by using the time-of-flight corrected detection time of photons on hit PMT ~\cite{Bx_long_paper}. Particle identification is also possible in Borexino~\cite{Bx_long_paper}, in particular $\alpha/\beta$ discrimination~\cite{Bx_nature_CNO}, by exploiting different scintillation light emission time profiles.

The Borexino data-taking period is divided into three phases: Phase-I (May 2007-May 2010), Phase-II (December 2011-May 2016), and Phase-III (July 2016-October 2021). 
Phase-II started after the detector calibration~\cite{Bx_calibration} and an additional purification of the LS, that enabled a comprehensive measurement of the $pp$ chain solar neutrinos~\cite{Bx_Nature_2018}.
Phase-III is characterized by a thermally stable detector with greatly suppressed seasonal convective currents. This condition has made it possible to extract an upper limit constraint on the $^{210}$Bi contamination in the LS, and thus, to provide the first direct observation of solar CNO neutrinos~\cite{Bx_nature_CNO}. 

\section{Correlated and Integrated Directionality for CNO}
\label{sec:CID}

Cherenkov photons emitted by the electrons scattered in neutrino interactions retain information about the original direction of the incident neutrino. Typically, in water Cherenkov neutrino detectors, this information is accessed through an event-by-event direction reconstruction, as demonstrated by the measurements of $^{8}$B neutrinos, at energies larger than $\SI{3.5}{\mega\electronvolt}$~\cite{super-kamiokande_IV, SNOPlus_B8}.
Instead, the Borexino experiment has provided a proof-of-principle for the use of this Cherenkov hit information in a LS detector and at neutrino energies below $\SI{1}{\mega\electronvolt}$ through the so-called "Correlated and Integrated Directionality" (CID) technique. A detailed explanation of the method can be found in \cite{Bx_CID_short, Bx_CID_long}.

The CID method discriminates the signal originating in the Sun - due to solar neutrinos - from the background. Cherenkov light is sub-dominant in Borexino, but it is emitted almost instantaneously with respect to the slower scintillation light. Consequently, directional information is contained in the first hits of an event (after correcting for the time-of-flight of each photon). 
%This parameter is calculated for each of an optimized-number-of-earliest hits for events from an energy region of interest (RoI), which maximizes the contribution of solar neutrino signal of interest. 
The CID analysis is based on the $\cos\alpha$ observable: for a given PMT hit in an event, $\alpha$ is the aperture angle between the Sun and the hit PMT at the reconstructed position of the event (see also Fig.~3 in ~\cite{Bx_CID_long}).
%for a given hit in a given event, $\alpha$ is the angle between the assumed neutrino direction, defined by the known position of the Sun and the reconstructed direction of the photon causing the hit, obtained joining the vertex of the event and the hit PMT position. 
For background events, the $\cos\alpha$ distribution is nearly uniform regardless which hit is considered. For solar neutrino events, the $\cos\alpha$ distribution is flat for scintillation photons which are emitted isotropically, but has a characteristic non flat distribution peaked at $\cos\alpha\sim0.7$ for Cherenkov hits correlated with the position of the Sun.
Since we cannot distinguish Cherenkov and scintillation photons, in our previous work \cite{Bx_CID_short} we have used only the $1^\text{st}$ and $2^\text{nd}$ hits of each event, which have the largest probability of being Cherenkov hits.
In the new analysis presented in this paper we fully exploit the directional information contained in the first several hits.
%For each of these photon hits we build a $\cos\alpha$ distribution which enters separately in the fit. 
This choice is supported by Monte Carlo simulations and sensitivity studies as discussed in Sec.~\ref{sec:Nth-hit}.

The solar neutrino signal is obtained by fitting the $\cos\alpha$ distributions of the selected first several hits, as a sum of signal and backgrounds contributions. The expected $\cos \alpha$ distributions for signal and background are obtained from Monte Carlo simulations.
As in Ref.~\cite{Bx_CID_long}, for each selected data event we simulate 200 MC events of solar neutrinos (represented by $^7$Be or $pep$ according to RoI, see below) and the same amount of the background events (represented by $^{210}$Bi). These events are simulated with the same astronomical time as the data event and with the position smeared around the reconstructed vertex.
From the fit we then obtain the total number of solar neutrinos $N_\nu$ detected in the RoI. The fit also includes two nuisance parameters. The effective Cherenkov group velocity correction $\text{gv}_\text{ch}$ nuisance parameter accounts for small differences in the relative hit time distribution between scintillation and Cherenkov hits in data relative to the MC. The second parameter is the event position mis-reconstruction in the initial electron direction $\Delta r_\text{dir}$, an indirect effect of the Cherenkov hits, where the reconstructed position is slightly biased towards early hit PMTs of the corresponding event. 
Here $\Delta r_\text{dir}$ is a free parameter of the fit, while $\text{gv}_\text{ch}$ is obtained independently and is constrained in the fit.
%In the previous publication, $\text{gv}_\text{ch}$ has been obtained using $^{40}$K gamma calibration data  (see \cite{Bx_CID_long}). In this work we calibrate $\text{gv}_\text{ch}$ exploiting the $^7$Be solar neutrino events which allows us to extend the analysis to the full Borexino dataset, as explained in the next Section.

% Figure environment removed



Compared to the previous proof-of-principle analysis~\cite{Bx_CID_short, Bx_CID_long}, the current CID analysis has been improved in a variety of ways. 
The full detector live time can be used now thanks to a novel $\text{gv}_\text{ch}$ correction calibration.
In the previous publication, $\text{gv}_\text{ch}$ has been obtained using $^{40}$K $\gamma$ calibration data  (see \cite{Bx_CID_long}). In this work instead we calibrate $\text{gv}_\text{ch}$ by exploiting the $^7$Be solar neutrino events which allows us to extend the analysis to the full Borexino dataset, as explained in Sec.~\ref{sec:strategy_and_data}.
%The full detector live time could be used thanks to a novel $\text{gv}_\text{ch}$ correction calibration without any use of gamma sources, the principle of which is discussed in Sec.~\ref{sec:strategy_and_data}. 
Additionally, indirect Cherenkov information from the systematic influence on the vertex reconstruction and consequently on the $\cos\alpha$ distribution was exploited by the inclusion in the analysis of later hits with negligible contribution of Cherenkov photons, see Sec.~\ref{sec:Nth-hit}. Technical details of the CID fitting procedure can be found in Sec.~\ref{sec: CID_fit}, while different systematic effects are discussed in Sec.~\ref{sec:cid_systematics}. The final CID results regarding the CNO measurement are reported in Sec.~\ref{subsec:CID_Results}.



\subsection{CID strategy for the CNO measurement with the full dataset}
\label{sec:strategy_and_data}

In the previous Borexino works~\cite{Bx_CID_short, Bx_CID_long}, the  calibration of the Cherenkov light group velocity $\text{gv}_\text{ch}$ has been performed using $\gamma$ sources deployed during the Borexino calibration campaign in 2009~\cite{Bx_CID_long}. The solar neutrino analysis was performed on the Phase-I dataset, that has been taken close in time to the source calibration of the detector ~\cite{Bx_calibration}. 
The $\text{gv}_\text{ch}$ found in this way was used to obtain the first measurement of $^7$Be solar neutrinos with the CID method \cite{Bx_CID_short}.
For the CID measurement of CNO in this paper, the entire Borexino dataset is used (from 2007 until 2021). Since the sub-nanosecond stability of the detector time response cannot be guaranteed for long periods, and no more calibrations have been performed after 2009, we developed a method to calibrate $\text{gv}_\text{ch}$ on the $^{7}$Be shoulder data.
This is done by using the same RoI as in ~\cite{Bx_CID_short, Bx_CID_long} (here called RoI$_{\text{gvc}}$ electron equivalent energy range of $\SI{0.5}{MeV} \lesssim T_e \lesssim \SI{0.8}{MeV}$) and performing the CID analysis where the $^7$Be is constrained to the Standard Model predictions ~\cite{Vinyoles_New_Gen_SSM}. 
The $\text{gv}_\text{ch}$ correction extracted in this way is then used in the CID analysis of the RoI$_{\text{CNO}}$, in which the CNO contribution is maximized, and which is fully independent from RoI$_{\text{gvc}}$. This step has been found to be justified according to MC studies, as the wavelength distribution of the detected Cherenkov photons produced by electrons from RoI$_{\text{gvc}}$ and RoI$_{\text{CNO}}$ is the same.
With this new strategy, the Cherenkov light $\text{gv}_\text{ch}$ can be calibrated on the same data-taking period as the one used for the CNO analysis. 
Two analyses have been performed in parallel for Phase-I (May 2007 to May 2010, 740.7\,days) and Phase-II+III (December 2011 to October 2021, 2888.0\,days).
The $\text{gv}_\text{ch}$ correction obtained for Phase-I can be compared to the one previously obtained from the $^{40}$K $\gamma$ source~\cite{Bx_CID_long}. Additionally, the analyses of the two independent data-sets allows for the investigation of any variation of the detector response over time.
The RoI$_{\text{gvc}}$ and the RoI$_{\text{CNO}}$ are shown for the Phase-II+III dataset in Fig.~\ref{fig:cid_ROI} for illustrative purposes.
The results on $\text{gv}_\text{ch}$ are provided in Sec.~\ref{subsub:gvcresults}.

In the final analysis, the Three-Fold-Coincidence algorithm~\cite{Bx_nature_CNO} is applied to the RoI$_{\text{CNO}}$ to suppress the cosmogenic $^{11}$C background, preserving the exposure with a signal survival fraction of $55.77\%\pm0.02\%$ for Phase-I and $63.97\%\pm0.02\%$ for Phase-II+III.
The radial ($R_\text{FV}$) and $T_e$ energy cuts of RoI$_{\text{CNO}}$ were optimized considering the expected number of solar neutrinos over the statistical uncertainty of the total number of events.
The optimized cuts are $R_\text{FV}<3.05 \,(2.95)$\,m and $0.85\,(0.85)\,\text{MeV} < T_e < 1.3\,(1.29)\,\text{MeV}$) for the Phase-I (Phase-II+III). In addition, all other cuts including the muon veto and data quality cuts have been applied as in Refs.~\cite{Bx_nature_CNO, Bx_improved_CNO}.
The overall exposures for the CID CNO analysis are $740.7\,\text{days} \times 104.3\,\text{tonnes}\times55.77\%$ for Phase-I and $2888.0\,\text{days} \times 94.4\,\text{tonnes} \times 63.97\%$ for Phase-II+III.
The total exposure of Phase-II+III (477.81\,years $\times$ tonnes) is about four times larger than that of Phase-I (118.04\,years $\times$ tonnes). 

\subsection{N$^\text{th}$-hit analysis approach}
\label{sec:Nth-hit}

As mentioned above, the CID analysis is performed on the first several early hits of ToF corrected hit times from each event in the RoI.
In this subsection we describe the optimization of the number of hits from each event to be used in the CID analysis. The procedure is based on the comparison of the MC-produced $\cos\alpha$ distributions of signal and background.

First, PMT hits of each individual event are sorted according to their ToF-corrected hit times and are labeled in this order as "N$^\text{th}$-hit", with N = 1, 2, ... up to the total number of hits. Second, the $\cos\alpha$ distributions are constructed for each N$^\text{th}$-hit for both the signal and background MC. 
%Third, from each N$^\text{th}$-hit $\cos\alpha$ distribution are extracted 10,000 toy datasets with the number of events as in real data. As the next step, the $\Delta\chi^2$ between the signal and background $\cos\alpha$ distributions is calculated for each N$^\text{th}$-hit. 
Third, for each N$^\text{th}$-hit $\cos\alpha$ distribution a number of 10,000 toy MC samples are simulated with the number of events observed as in the real data.
Next, we perform a direct signal to background comparison, based on a standard $\chi^2$-test.
Figure~\ref{fig:cid_nth_hit_selection} shows the resulting $\Delta\chi^2$ for Phase-II+III in the  RoI$_{\text{CNO}}$ averaged over the 10,000 toy datasets as a function of N$^\text{th}$-hit. 
A larger average $\Delta\chi^2$ corresponds to a greater difference between the MC signal and background and thus a larger expected sensitivity for the CID fit, independent of the true signal to background ratio.
While only the earliest $\sim$4 N$^\text{th}$-hits have a relevant, \textit{direct} contribution of Cherenkov hits to the $\cos\alpha$ distribution of the neutrino signal, later N$^\text{th}$-hits also contribute to the CID sensitivity due to the \textit{indirect} Cherenkov influence on $\Delta r_\text{dir}$.
A more in-depth explanation of the $\Delta r_\text{dir}$ effect is shown in Fig.~9 in ~\cite{Bx_CID_long}.

A possible impact of the $\text{gv}_\text{ch}$ and $\Delta r_\text{dir}$ nuisance parameters on the N$^\text{th}$-hit selection has been investigated and is presented in Fig.~\ref{fig:cid_nth_hit_selection}.
The first hits of the events provide the largest $\Delta\chi^2$ values thanks to the direct Cherenkov light. A decrease of $\text{gv}_\text{ch}$ is decreasing the group velocity of Cherenkov photons and thus their contribution at early hits. 
The impact of $\Delta r_\text{dir}$ can be seen for N$^\text{th}\text{-hit}>4$, where the contribution of direct Cherenkov hits becomes negligible relative to the scintillation hits, but the signal and background MC $\cos\alpha$ histograms are still different from each other ($\Delta\chi^2 > 0$).

In conclusion, the early hits selection for the CID analysis in both RoI$_{\text{gvc}}$ and RoI$_{\text{CNO}}$ is then performed from the first hit up to the $\text{N$^\text{th}$-hit(max)} = 15, 17$ for Phase-I and Phase-II+III, respectively. This is an optimization where all direct and indirect Cherenkov information is used, while at the same time this selection keeps the contribution from delayed scintillation photons, undergoing various optical process during the propagation through the detector, relatively small.
%while at the same time it keeps the contribution of later hits with larger fraction of indirect light relatively small. 

The Cherenkov-to-scintillation photon ratio as a function of N$^\text{th}$-hit has also been checked explicitly, as is shown in Fig.~\ref{fig:cheratio} for RoI$_{\text{CNO}}$. As expected, it can be seen that the early  N$^\text{th}$-hits benefit from the largest Cherenkov-to-scintillation ratio of $\sim 13\%$ for the first hit. The overall total Cherenkov-to-scintillation ratio is small and found to be $0.475\%$ in the MC.



% Figure environment removed


\subsection{CID fit procedure}
\label{sec: CID_fit}

The fitting strategy follows the procedure developed in our previous CID analysis~\cite{Bx_CID_long}.
The data $\cos\alpha$ distributions from the selected RoI, constructed for each N$^\text{th}$-hit from the first up to the N$^\text{th}$-hit(max),
are fitted simultaneously with the MC produced, expected $\cos\alpha$ distributions of the neutrino signal and background, where the signal $\cos\alpha$ distribution depends on $\text{gv}_\text{ch}$ and $\Delta r_\text{dir}$.
The nuisance parameter $\Delta r_\text{dir}$ cannot be calibrated in Borexino and is left free to vary without a dedicated pull term. The number of $\cos\alpha$ histogram bins used in the analyses is $i = 60$ for all energy regions and phases, as values of $i < 30$ reduce the expected CID sensitivity.

\subsubsection {Fit in the RoI$_{\text{gvc}}$}

The CID analysis in RoI$_{\text{gvc}}$ used for the $\text{gv}_\text{ch}$ calibration is based on the $\chi^{2}$-test:

\begin{align}
\begin{split}
    &\chi^{2}_{\text{gv}_\text{ch}}(N_{\nu}, \text{gv}_\text{ch}, \Delta r_\text{dir}) =\\
    &= \sum_{n=1}^\text{ N$^\text{th}$-hit(max)}\sum_{i=1}^{I} \left( \frac{ \left( \mathcal{N} \cdot M_{i}^{n} - D_{i}^{n} \right)^{2} }{ \mathcal{N} \cdot  M_{i}^{n} + \mathcal{N}^{2} \cdot M_{i}^{n} } \right) - 2 \ln\left(P(N_{\nu})\right),
\end{split}
\label{eq:cid_gvc_chi2_def}
\end{align}

\noindent where $D_{i}^{n}$ and  $M_{i}^{n}$ are the numbers of $\cos\alpha$ histogram entries at bin $i$ for a given N$^\text{th}$-hit $n$, for data and MC, respectively. The term $\mathcal{N}$ is the scaling factor between the MC and the data event statistics and the term "$\mathcal{N}^{2} \cdot M_{i}^{n}$" in the denominator takes into account the finite statistics of MC. The explicit dependence of the fit on $N_{\nu}$, $\text{gv}_\text{ch}$, and $\Delta r_\text{dir}$ can be expressed by decomposing the MC contribution to the one from the signal $S$ and the background $B$:
 
\begin{align}
M_{i}^{n} =\frac{N_\nu}{N_{\text{data}}} \cdot M_{\text{S},i}^{n}(\Delta r_{\text{dir}}, gv_{\text{ch}}) + (1 - \frac{N_\nu}{N_{\text{data}}}) \cdot M_{{\text{B}}, i}^{n}.
\label{eq:Mi}
\end{align}

\noindent The number of neutrino events $N_{\nu}$ and $\Delta r_\text{dir}$ are treated as nuisance parameters to produce the $\chi^{2}(\text{gv}_\text{ch})$ profile, where $N_{\nu}$ is constrained by the SSM expectation. For this, the neutrino prior probability distribution $P(N_{\nu})$ is given by the sum of the Gaussian probability distributions with mean and sigma from the high-metallicity~(HZ) SSM and low-metallicity~(LZ) SSM~\cite{Vinyoles_New_Gen_SSM} predictions on the number of $^{7}$Be+$pep$-$\nu$ in $^{7}$Be-$\nu$ shoulder region, which is then convoluted with a uniform distribution of CNO-$\nu$ between zero and the HZ-SSM CNO expectation + $5\sigma$. 
In this way, by leaving CNO reasonably free to vary, we avoid a potential correlation of the $\text{gv}_\text{ch}$ calibration and the subsequent measurement of the CNO-$\nu$ rate using this $\text{gv}_\text{ch}$ constraint.


\subsubsection {Fit in the RoI$_{\text{CNO}}$}

% Figure environment removed


The $\chi^{2}$-test for the measurement of number of solar neutrinos ($N_{\nu}$) in RoI$_{\text{CNO}}$ is:

\begin{align}
\begin{split}
    &\chi^{2}_{\nu}(N_\nu, \text{gv}_\text{ch}, \Delta r_\text{dir}) =\\
    &= \sum_{n=1}^\text{ N$^\text{th}$-hit(max)}\sum_{i=1}^{I} \left( \frac{ \left( \mathcal{N} \cdot M_{i}^{n} - D_{i}^{n} \right)^{2} }{ \mathcal{N} \cdot  M_{i}^{n} + \mathcal{N}^{2} \cdot M_{i}^{n} } \right) + \Delta\chi_{\text{gv}_\text{ch}}^{2}\left(\text{gv}_\text{ch}\right)
\end{split}
\label{eq:cid_stt_chi2_def}
\end{align}

\noindent using $\text{gv}_\text{ch}$ and $\Delta r_\text{dir}$ as nuisance parameters. The $\text{gv}_\text{ch}$ parameter is now constrained by the previous calibration in RoI$_\text{gvc}$ through the pull term $\Delta\chi_{\text{gv}_\text{ch}}^{2}\left(\text{gv}_\text{ch}\right)$.

\subsection{Systematic uncertainties}
\label{sec:cid_systematics}

In this work we have performed a detailed evaluation of the systematic uncertainties. The quantitative evaluation of the systematic uncertainties on the CID analysis results are given in Sec.~\ref{subsec:CID_Results}.

It has been found that the choice of N$^\text{th}$-hit(max) and the histogram binning do not introduce any systematic uncertainty. Backgrounds different than $^{210}$Bi also contribute in the analysed energy intervals, but have been found to be indistinguishable in the CID analysis and do not contribute to the systematic uncertainty budget. Even if the external $\gamma$ events are not uniformly distributed in the FV, due to their attenuation in the LS, the difference in the $\cos\alpha$ distribution between these events and uniform background events is found to be safely negligible, given the statistics of the data.

The following effects have an impact on the final results:

 {\it PMT selection}
 
Some PMTs feature an intrinsic misbehaviour of their hit time distribution, if compared to all other PMTs. 
They are identified using an enriched sample of $^{11}$C events. In addition, a small number of PMTs feature inconsistency between the data and MC in the relative contribution to the first hits. The systematic effect has been evaluated by varying the selection of usable PMT.

{\it Relative PMT hit time correction}

It has been found by analyzing an enriched sample of $^{11}$C events, that the data PMTs have a small relative time offset between each other with an average value of 0.3\,ns. This time offset has been measured with an uncertainty of up to $\pm$0.1\,ns.
It is reasonable to correct this relative offset in data, as it does not exists in MC.
This makes it also necessary to propagate the uncertainty of the PMT time correction through the entire analysis chain, which introduces a systematic uncertainty. 


{\it Influence of low number of signal events}

As described above, signal and background MC are produced on an event-by-event basis. This could  introduce an additional systematic uncertainty through the particular choice of the event positions and neutrino directions used for the production of the signal MC. This systematic uncertainty is estimated by producing a large number of the signal MC $\cos\alpha$ distributions corresponding to a random selection of the expected number of signal events from data in each phase. The data is then analyzed again with these reduced signal MC $\cos\alpha$ histograms. This effect contributes to the systematic uncertainty only in the RoI$_{\text{CNO}}$ for Phase-I.


{\it CNO-$\nu$ and $pep$-$\nu$ $\cos\alpha$ distributions}

The CNO-$\nu$ and $pep$-$\nu$ events show a significantly different energy distribution in the selected RoI. The expected Cherenkov to scintillation hits ratio for $pep$-$\nu$ events (0.475\%) is higher than for CNO events (0.469\%) due to their different energy distribution in RoI$_{\text{CNO}}$. Moreover, the angular distribution of recoiled electrons by CNO-$\nu$ and $pep$-$\nu$ is also different due to its dependence on energy distributions of the neutrino and recoiled electron. 
The final fit on the number of solar neutrinos is performed with the $pep$-$\nu$ MC and the systematic uncertainty is estimated by performing the CID analysis again with the CNO-$\nu$ MC.
The absolute difference between the two analyses is used conservatively as the systematic uncertainty. This systematic is negligible for the $\text{gv}_\text{ch}$ calibration, as CNO+$pep$ neutrinos are sub-dominant to the $^7$Be neutrinos.

{\it Constraint on non-CNO neutrinos}

The CID analysis is only sensitive to the measurement of the total number of solar neutrinos $N_\nu$ and cannot differentiate between different solar neutrino species.
Therefore, a measurement of the number of CNO neutrinos depends on the subtraction of the non-CNO neutrinos from $N_\nu$ obtained in the RoI$_{\text{CNO}}$. The number of $pep$ neutrinos is constrained by the SSM predictions~\cite{Vinyoles_New_Gen_SSM}, while $^{8}$B is constrained using the high precision flux measurement of Super-Kamiokande~\cite{superK_solar_neutrino_IV}. 

{\it Exposure }

In this category we cover systematic uncertainties on the determination of the fiducial mass and on the fraction of solar neutrinos in the RoI. These uncertainties are estimated using toy-MC studies, based on the results of the calibration campaign~\cite{Bx_calibration} on the performance of the position reconstruction and on the uncertainty on the energy scale described in~\cite{Bx_nature_CNO}, respectively. The uncertainty on the fiducial mass includes also the uncertainty on the scintillator density. Additionally, in the $\text{gv}_\text{ch}$ calibration using the constraint on the expected number of all solar neutrino events, we consider also the uncertainty on $\alpha / \beta$ discrimination applied to suppress $^{210}$Po $\alpha$ decays in the RoI$_{\text{gvc}}$.



\subsection{Results of the CID analysis}
\label{subsec:CID_Results}


\subsubsection {Effective $\text{gv}_\text{ch}$  calibration on the $^7$Be edge}
\label{subsub:gvcresults}

% Figure environment removed

The effective calibration of the Cherenkov light as a results of the CID analysis on the $^7$Be edge, using Eq.~\ref{eq:cid_gvc_chi2_def}, is $\text{gv}_\text{ch} = (0.140 \pm 0.029)\,\text{ns\,m}^{-1}$ for Phase-I and $\text{gv}_\text{ch} = (0.089 \pm 0.019)\,\text{ns\,m}^{-1}$ for Phase-II+III, including the systematic uncertainties summarized in Table~\ref{tab:cid_gvc_systeamtics}.
The compatibility between the data and the MC model, illustrated in Fig.~\ref{fig:bestgvc_cosalpha}, is good with $\chi^{2} / \text{ndf} = 874.9/897$, $p \text{ value} =0.70$ for Phase-I and $\chi^{2} / \text{ndf} = 1036.2/1017$, $p \text{ value} = 0.33$ for Phase-II+III. 
The $\chi^{2} / \text{ndf}$ and $p$ values have also been investigated for the individual N$^\text{th}$-hits $\cos\alpha$ histograms, with different binning choices to investigate the fit performance. For all cases, the best fit MC model $\cos\alpha$ distribution is always in agreement with the data.
Figure ~\ref{fig:bestgvc_cosalpha} shows an illustration of the best fit results (red) relative to a pure background hypothesis (blue).
For early hits, direct Cherenkov light causes the peak around $\cos\alpha\sim0.7$ and the influence of $\Delta r_\text{dir}$ induces the negative slope for $\cos\alpha < 0$.
For later N$^\text{th}$-hits the Cherenkov peak washes away, but the indirect impact of the Cherenkov hits on the position reconstruction bias $\Delta r_\text{dir}$ makes it still possible to distinguish between the neutrino signal and the background.
The non-flat background $\cos\alpha$ distribution originates from the live PMT non-isotropic distribution relative to the position distribution of the Sun around Borexino.
These $\text{gv}_\text{ch}$ values for Phase-I and Phase-II+III differ by less than $1.5\sigma$ and both are in agreement with the previous calibration performed at the end of Phase-I using a $^{40}$K $\gamma$ source: $\text{gv}_\text{ch} = (0.108 \pm 0.039)\,\text{ns\,m}^{-1}$  (Fig.\,13 in \cite{Bx_CID_long}).

\begin{table}[t]
\centering
\setlength\extrarowheight{2pt}
\begin{tabular}{c|cc}
\hline
\multicolumn{1}{c}{}                &  &  \\ [-13pt] \hline
Source of $\text{gv}_\text{ch}$ uncertainty & Phase-I & Phase-II+III \\ [2pt] \hline
PMT selection & 2.1\%     & 1.6\%       \\
PMT time corrections  & 3.7\%     & 2.1\%      \\
MLP event selection  & 1.0\%     & 1.0\%      \\
Fiducial mass & $\left(_{-1.2}^{+0.2}\right)\%$    & $\left(_{-1.2}^{+0.2}\right)\%$ \\
Fraction of neutrinos in RoI & 1.3\%    & 0.9\%      \\ \hline
\multicolumn{1}{c}{}                &  &  \\ [-13pt] \hline
\multicolumn{1}{c}{}                &  &  \\ [-10pt]
\end{tabular}
\caption{Systematic uncertainties of the $\text{gv}_\text{ch}$ measurement in the RoI$_{\text{gvc}}$, relative to the best fit value.}
\label{tab:cid_gvc_systeamtics}
\end{table}





\subsubsection {CNO measurement with CID} 
\label{subsub:CNOresults}

% Figure environment removed

This section describes the results of the measurement of CNO solar neutrinos with CID using the full Borexino live time from May 2007 to October 2021. 
The $\text{gv}_\text{ch}$ values presented in Sec.~\ref{subsub:gvcresults} are used as independent pull terms in the Eq.~\ref{eq:cid_stt_chi2_def} for the fit in RoI$_{\text{CNO}}$ of their respective phases.
This takes into account the  potential systematic differences of the detector response between Phase-I and Phase-II+III. The resulting number of solar neutrino events $N_{\nu}$ in the RoI$_{\text{CNO}}$ can be converted into the number of CNO neutrinos detected in the same energy region after constraining the contributions from $pep$ and $^8$B neutrinos, but without any a-priori knowledge of the backgrounds. This number of CNO events can be further transformed into the measurement of the CNO-$\nu$ interaction rate in Borexino and the CNO flux at Earth. 

%The best fit values for the number of solar neutrinos in RoI$_{\text{CNO}}$ are $N_\nu = 680_{-224}^{+235}\,\text{(stat)}$ for Phase-I and $N_\nu = 2805_{-494}^{+518}\,\text{(stat)}$ for Phase-II+III without inclusion of any systematic uncertainties or corrections (not even gvc syst).

The best fit values for the number of solar neutrinos in RoI$_{\text{CNO}}$ are $N_\nu = 691_{-224}^{+235}\,\text{(stat)}$ for Phase-I and $N_\nu = 2828_{-494}^{+518}\,\text{(stat)}$ for Phase-II+III without inclusion of any systematic uncertainties or corrections. The compatibility betwwen the data and the MC model is good with $\chi^{2} / \text{ndf} = 884.8 / 897$, $p \text{ value} = 0.61$ for Phase-I and $\chi^{2} / \text{ndf} = 1000.7 / 1017$, $p \text{ value} = 0.64$ for Phase-II+III.
The MC model is able to reproduce the data $\cos\alpha$ distribution, which has also been investigated for the individual N$^\text{th}$-hits $\cos\alpha$ histograms.

Figure~\ref{fig:bestfit_cosalpha} illustrates the best fit results (red) relative to a pure background hypothesis (blue), in which the CID $\cos\alpha$ histograms of data (black) are shown for the sum of Phase-I + Phase-II+III, as well as for the sum of the early first to fourth N$^\text{th}$-hits (top) and the sum of the later N$^\text{th}$-hits from the fifth to $\mathrm{N^{th}-hit(max)}$ (bottom). The actual fit is performed on Phase-I and Phase-II+III independently.
The same observations made for Fig.~\ref{fig:bestgvc_cosalpha} hold also true for Fig.~\ref{fig:bestfit_cosalpha}, where the early hits show the Cherenkov peak and the later hits show the impact of $\Delta r_\text{dir}$.
%For early hits, direct Cherenkov light causes the peak around $\cos\alpha\sim0.7$ and the influence of $\Delta r_\text{dir}$ induces the negative slope for $\cos\alpha < 0$.
%For later N$^\text{th}$-hits the Cherenkov peak washes away, but the indirect impact of the Cherenkov hits on the position reconstruction bias $\Delta r_\text{dir}$ makes it still possible to distinguish between the neutrino signal and the background.
%The non-flat background $\cos\alpha$ distribution originates from the live PMT distribution relative to the position distribution of the Sun around Borexino.

\paragraph{Fit response bias correction}
\label{par:fit_response_bias}

Toy-MC analyses found that the fit of the number of solar neutrinos in RoI$_{\text{CNO}}$ shows a small systematic shift between the injected number of neutrinos and the best fit number of neutrinos, due to the correlation between the nuisance parameters ($\text{gv}_\text{ch}$, $\Delta r_\text{dir}$) and the relatively low total number of neutrino events.
This fit response bias is induced by the two nuisance parameters as they only impact the shape of the neutrino signal MC $\cos\alpha$ distribution but not that of background. We note that this effect was found to be negligible in RoI$_{\text{gvc}}$ due to the relative large number of signal events and the large signal to background ratio.

\begin{table}[t]
\centering
\setlength\extrarowheight{2pt}
\begin{tabular}{c|cc}
\hline
\multicolumn{1}{c}{}                &  &  \\ [-13pt] \hline
Source of uncertainty   & Phase-I & Phase-II+III \\ [2pt] \hline
\multicolumn{3}{c}{For N$_{\nu}$}   \\
\hline
PMT selection & 1.3\%     & 0.6\%       \\
PMT time corrections  & 4.2\%     & 2.4\%      \\
Low number of signal events & 2.2\%     &  --       \\
CNO-$\nu$ vs. $pep$-$\nu$ MC & 2.2\%     & 2.0\%      \\
\hline
\hline
\multicolumn{3}{c}{For N$_\text{CNO}$}     \\
[2pt] \hline
 &  &  \\ [-12pt]
$pep$+$^{8}$B-$\nu$ constraint & 4.6\%    & 1.8\%       \\ \hline
 \hline
\multicolumn{3}{c}{For R$_\text{CNO}$} \\ \hline
Fiducial mass                 & $\left(_{-1.2}^{+0.2}\right)\%$    & $\left(_{-1.2}^{+0.2}\right)\%$ \\
Fraction of CNO-$\nu$ in RoI & 1.4\%    & 1.4\%      \\ \hline
\multicolumn{1}{c}{}                &  &  \\ [-13pt] \hline
\multicolumn{1}{c}{}                &  &  \\ [-10pt]
\end{tabular}
\caption{Systematic uncertainties on the number of solar neutrino events $N_{\nu}$ in RoI$_{\text{CNO}}$, relative to the best fit value. The uncertainty from $pep$+$^{8}$B-$\nu$ constraint is relevant only for N$_\text{CNO}$. The last two rows are relevant only for the CNO-$\nu$ rate (R$_\text{CNO}$) calculation.}
\label{tab:cid_cno_systeamtics}
\end{table}


The value of the fit response bias in RoI$_{\text{CNO}}$ is estimated using the Bayesian posterior distribution of $N_{\nu}$~\cite{dagostini_bayes}, which is produced through a toy-MC rejection sampling, described in summary below.
The prior distribution for the number of neutrino events is chosen to be uniform between zero and the number of selected data events (2990 for Phase-I and 5974 for Phase-II+III), the prior distribution of $\Delta r_\text{dir}$ is also uniform, and the prior distribution of $\text{gv}_\text{ch}$ is given by the measurement at the $^7$Be-$\nu$ edge RoI$_{\text{gvc}}$ $\left(P\left(\text{gv}_\text{ch}\right) \propto \exp\left(-\frac{1}{2}\Delta\chi^2(\text{gv}_\text{ch})\right)\right)$.
The pseudo-data inputs $\left( N_{\nu}^{\text{sim}}, \text{gv}_\text{ch}^{\text{sim}}, \Delta r_\text{dir}^{\text{sim}} \right)$ are sampled from the MC signal and background $\cos\alpha$ distributions following these model parameter prior distributions.
The analysis is then performed in the same way as for the real data and results in best fit values of the pseudo-data $\left( N_{\nu}^{\text{fit}}, \text{gv}_\text{ch}^{\text{fit}}, \Delta r_\text{dir}^{\text{fit}} \right)$.
The real data result now defines a multivariate Gaussian distribution $\text{P}_\text{accept}(N_{\nu}, \text{gv}_\text{ch}, \Delta r_\text{dir})$ with a mean value given by its best fit values and with a standard deviation given by the systematic uncertainty of the PMT time corrections.
The sampled true values of the triplet  $\left( N_{\nu}^{\text{sim}}, \text{gv}_\text{ch}^{\text{sim}}, \Delta r_\text{dir}^{\text{sim}} \right)$ are then saved only with a probability of $\text{P}_\text{accept}(N_{\nu}^{\text{fit}}, \text{gv}_\text{ch}^{\text{fit}}, \Delta r_\text{dir}^{\text{fit}})$, given by the best fit result of the pseudo-data, otherwise they are rejected.
The resulting distributions of the true values for $\left( N_{\nu}, \text{gv}_\text{ch}, \Delta r_\text{dir} \right)$ then correspond to their Bayesian posterior distributions. 



% Figure environment removed

The fit response bias is illustrated in Fig.~\ref{fig:fit_response_bias} for Phase II+III. The likelihood distribution $P(N_\nu) \propto \exp \left(-\frac{1}{2} \Delta\chi^2(N_{\nu}\right)$ given by the $\chi^2$ fit of data with Eq.~\ref{eq:cid_stt_chi2_def} and averaged over the 1000 fits with different PMT time offsets is shown in blue. The black distribution is given by the simulation of 20k pseudo-data analyses, selected through the rejection sampling MC described above. The red distribution is produced by shifting $P(N_\nu)$ by a value of $\Delta N_\nu = -109\pm4$ and this distribution is well in agreement with the black rejection sampled distribution. It is therefore used as the posterior distribution of the CID analyses. We note that for the Phase-I the situation is similar and the shift is found to be $-50\pm4$ events.


\paragraph{Inclusion of systematics}
\label{par:fit_systematics}

The final result of the CID analysis for the number of solar neutrinos is given by the Bayesian posterior distribution of $N_\nu$, marginalized over the nuisance parameters and convoluted with the systematic uncertainties. The relevant systematic uncertainties are shown in Table~\ref{tab:cid_cno_systeamtics} and assumed to be normally distributed. 

The posterior distributions $P\left(N_\nu\right)$ in Phase-I and Phase-II+III including these systematics are shown in Fig.~\ref{fig:cid_numcnopep}. The resulting number of solar neutrinos detected in the RoI$_{\text{CNO}}$ is $N_\nu = 643_{-224}^{+235}\,\text{(stat)}_{-30}^{+37}\,\text{(sys)}$ for Phase-I and $N_\nu = 2719_{-494}^{+518}\,\text{(stat)}_{-83}^{+85}\,\text{(sys)}$ for Phase-II+III, including all systematics and correcting for the fit response bias. The quoted uncertainties are calculated from the posterior distributions using an $68\%$ equal-tailed credible interval (CI).
The one-sided zero neutrino hypothesis can be excluded with $P(N_\nu=0) = 2.8\times10^{-5}$ ($\sim4.2\sigma$) for Phase-I and $P(N_\nu = 0) = 6.4\times10^{-11}$ ($\sim6.5\sigma$) for Phase-II+III.




\paragraph{CID results on CNO}
\label{par:CIDCNO}

The interpretation of the CID results requires the correct treatment of the physical boundaries of the analysis, i.e. $0 \leq N_\nu \leq 2990$ (5974) for Phase-I (Phase-II+III), respectively.
This is done in a Bayesian interpretation, based on the posterior distribution $P\left(N_\nu\right)$ shown in Fig.~\ref{fig:cid_numcnopep}.


% Figure environment removed


% Figure environment removed
% Figure environment removed

Next, the distribution of the number of CNO-$\nu$ events is estimated by constraining the expected number of $pep$ and $^8$B neutrino events ($N_{pep+^8\text{B}}$) where the constraint on the number of $pep$ neutrinos uses the SSM predictions~\cite{Vinyoles_New_Gen_SSM} and $^{8}$B is constrained using the high precision flux measurement of Super-Kamiokande~\cite{superK_solar_neutrino_IV} including model uncertainties, the difference between HZ-SSM and LZ-SSM predictions, as well as the Borexino FV and energy systematic uncertainties from Table~\ref{tab:cid_cno_systeamtics}.
This is done through the convolution of the $N_\nu$ posterior distributions from Fig.~\ref{fig:cid_numcnopep} with the predicted $P(N_{pep+^8\text{B}})$ probability distribution: $P(N_\text{CNO}) = P(N_\nu)\ast P(-N_{pep+^8\text{B}})$.
The resulting $P(N_\text{CNO})$ posterior distributions are shown in Fig.~\ref{fig:cid_numcno}.
The CID measurement for the number of CNO-$\nu$ events is then $N_\text{CNO} = 270_{-169}^{+218}\,\text{(stat)}_{-25}^{+33}\,\text{(sys)}$ for Phase-I and $N_\text{CNO} = 1146_{-486}^{+518}\,\text{(stat)}_{-89}^{+92}\,\text{(sys)}$ for Phase-II+III, where the uncertainty corresponds to the equal-tail $68\%$ CI within the physical boundaries, including all systematics.

It has been observed that Phase-I and Phase-II+III do not show prohibitively different behavior for the full CID analysis-chain and the MC model is well able to reproduce the data $\cos\alpha$ histograms for both phase selections and for each selected energy region.
It is then reasonable to combine the conditionally independent results of Phase-I and Phase-II+III, through the convolution of both posterior distributions $P\left(N_\text{CNO}\right)^{\text{I+II+III}} = P\left(N_\text{CNO}\right)^{\text{I}} \ast P\left(N_\text{CNO}\right)^{\text{II+III}}$.
The probability that exactly zero CNO-$\nu$ events contribute to the measured data CID $\cos\alpha$ distribution is $P(N_\text{CNO}=0) = 1.35\times10^{-3}$ for Phase-I, $P(N_\text{CNO}=0) = 5.87\times10^{-5}$ for Phase-II+III, and $P(N_\text{CNO}=0) = 7.93\times10^{-8}$ for the combined result. This corresponds to a one-sided exclusion of the zero-CNO hypothesis at about $5.3\sigma$ credible level for the combination of Phase-I and Phase-II+III.

The CNO-$\nu$ rate probability density function is calculated from the measured posterior distribution of CNO-$\nu$ events, using the exposure of the respective phases. The effective exposure is given by the product of the fiducial mass, the detector live time, the TFC-exposure, the trigger efficiency, and the fraction of CNO-$\nu$ events within the selected energy region. 
The final CID result for the CNO-$\nu$ rate, using the full dataset of Phase-I + Phase-II+III, is $R_\text{CNO}^\text{CID} = 7.2\pm2.5\,\left(\text{stat}\right)\pm0.4\,\left(\text{sys}\right)\,_{-0.8}^{+1.1} \,\left(\text{nuisance}\right) \text{ cpd/100\,tonnes} = 7.2_{-2.7}^{+2.8} \, \text{ cpd/100\,tonnes}$
%$R_\text{CNO}^\text{CID} = 7.2_{-2.5}^{+2.5}\text{\,(stat)}_{-0.9}^{+1.2}\text{\,(sys)} \, \frac{\text{cpd}}{100\,\text{tonnes}} = 7.2_{-2.7}^{+2.8} \, \frac{\text{cpd}}{100\,\text{tonnes}}$.
The quoted uncertainties now also show the systematic uncertainties from Table~\ref{tab:cid_cno_systeamtics} separately from the influence of the nuisance parameters $\text{gv}_\text{ch}$ and $\Delta r_\text{dir}$. The quoted statistical uncertainty corresponds to a hypothetical, perfect calibration of these CID nuisance parameters.
The results are summarized in Table~\ref{tab:cno_cno_final_result}.


\begin{table}[t]
\centering
\setlength\extrarowheight{4pt}
\begin{tabular}{c|cc}\hline
\multicolumn{1}{c}{}                &  &  \\ [-15pt] \hline
CID results        & $P(N_\text{CNO}=0)$ & $R_\text{CNO}$ $\left[ \frac{ \text{cpd} }{ \SI{100}{tonnes}} \right]$ \\[6pt] \hline
Phase-I   & $1.35\times10^{-3}$  & $6.4_{-4.1}^{+5.2}$\\[4pt]
Phase-II+III & $5.87\times10^{-5}$ & $7.3_{-3.2}^{+3.4}$ \\[4pt] \hline
Combined & $7.93\times10^{-8}$ & $7.2_{-2.7}^{+2.8}$ \\[4pt]  \hline
\multicolumn{1}{c}{}                &  &  \\ [-15pt] \hline
\multicolumn{1}{c}{}                &  &  \\ [-10pt]
\end{tabular}
\caption{CID CNO-$\nu$ results with systematic uncertainties.}
\label{tab:cno_cno_final_result}
\end{table}

These CID results are well in agreement with the HZ-SSM prediction of $(4.92\pm0.78) \text{ cpd/100\,tonnes}$ ($0.6\,\sigma$), while the LZ-SSM prediction $(3.52\pm0.52) \text{ cpd/100\,tonnes}$ ($1.1\,\sigma$) is 1.7 times less likely to be true, given the results of the Borexino CID analysis.



\section{Combined CID and multivariate analysis}
\label{sec:MV}



In this Section we combine the CID analysis with the standard multivariate fit of Phase-III to improve the result on CNO neutrinos.
This is done by including the posterior distributions of solar  neutrinos from the CID analysis, shown in Fig.~\ref{fig:cid_numcnopep}, in the multivariate analysis likelihood. By statistically subtracting the sub-dominant $^{8}$B neutrinos contribution and converting $N_\text{CNO+pep}$ to the corresponding interaction rate, it is possible to use these posterior distributions as external likelihood terms in the minimization routine. Following this procedure, two multiplicative pull terms  constraining the number of CNO and $pep$ neutrino events are used: the first one is related to the Phase-I ($\mathcal L^\mathrm{P-I}_\mathrm{CID}$), while the second one refers to Phase-II+III datasets ($\mathcal L^\mathrm{P-II+III}_\mathrm{CID}$).

The overall combined likelihood used for this analysis becomes:
\begin{equation}
     \mathcal L_\mathrm{MV +CID} = \mathcal L_\mathrm{MV} \cdot \mathcal L_\mathrm{pep} \cdot \mathcal L_\mathrm{^{210}Bi} \cdot  \mathcal L^\mathrm{P-I}_\mathrm{CID} \cdot  \mathcal L^\mathrm{P-II+III}_\mathrm{CID} 
     \label{eqn:MV_Likelihood_Full}
\end{equation}
%where the first three terms correspond to the standard multivariate analysis described in \cite{Bx_improved_CNO} with an optimized 2D version of the fitter.
%As usual, the analysis is performed on the energy and radial distributions of events.  
where the first three terms correspond to an improved version of the standard multivariate analysis described in \cite{Bx_improved_CNO}.  This approach couples the one-dimensional Poisson likelihood for the TFC-Tagged dataset with a two-dimensional one (energy and radius) for the TFC-subtracted, to enhance the separation between signal and backgrounds.  We have improved the binning optimization and used an updated version of Monte Carlo.

The $pep$ neutrinos interaction rate is constrained with 1.4\% precision to the $2.74 \pm 0.04$ cpd/100 tonnes value, by combining the Standard Solar Model predictions~\cite{Vinyoles_New_Gen_SSM}, the most current flavor oscillation parameters set~\cite{Capozzi:2021fjo} and the solar neutrino data~\cite{Vescovi:2020wyz,Bergstrom:2016cbh}. 
This constraint is applied with the Gaussian pull term $\mathcal L_\mathrm{pep}$.
An upper limit on the $^{210}$Bi rate of ($10.8 \pm 1.0$ cpd/100 tonnes) is applied with the half Gaussian term $\mathcal L_\mathrm{^{210}Bi}$. This upper limit is obtained from the rate of the $^{210}$Bi daughter $^{210}$Po (see~\cite{Bx_nature_CNO,Bx_improved_CNO} for more details).


\subsection{Results} 
\label{subsec:MV_Results}



As in \cite{Bx_improved_CNO}, the energy RoI for the multivariate analysis is $\SI{0.32}{MeV} < T_e < \SI{2.64}{MeV}$ for electron recoil kinetic energy. The reconstructed energy spectrum scale is quantified in the $N_h$ estimator, representing total number of detected hits for a given event (see Sec.~\ref{sec:bxdet}). The dataset is the same one analyzed in ~\cite{Bx_improved_CNO}, in which the exposure amounts to 1431.6 days $\times$ 71.3 tonnes.

Along with CNO solar neutrinos, the free parameters of the fit are divided into three categories: internal ($^{85}$Kr and $^{210}$Po) and external ($^{208}$Tl, $^{214}$Bi, and $^{40}$K) backgrounds, cosmogenic backgrounds ($^{11}$C, $^{6}$He, and $^{10}$C), and solar neutrinos ($^{7}$Be). Since $^{8}$B solar neutrinos exhibit a flat and marginal contribution, the corresponding interaction rate is fixed at high-metallicity expectations from Solar Standard Model. As discussed in Eq.~\ref{eqn:MV_Likelihood_Full}, the interaction rates of $pep$ neutrinos and $^{210}$Bi background are constrained with likelihood pull terms, and CID results reported in Sec.~\ref{subsec:CID_Results} are accounted for as additional external constraints. The result of the fit for the energy and radial projections is shown in Fig.~\ref{fig:mv_spectralfit}.

% Figure environment removed

The multivariate fit returns an interaction rate of CNO neutrinos of $6.7^{+1.2}_{-0.7} \, \mathrm{cpd/100 \, tonnes}$ (statistical error only). The agreement between the model and data is quantified with a $p$ value of 0.2. 

To account for sources of systematic uncertainty, the same Monte Carlo method described in~\cite{Bx_nature_CNO,Bx_improved_CNO} has been adopted. In a nutshell, hundred thousands Monte Carlo pseudo-experiments were generated, including relevant effects able to introduce a systematic error, such as the energy response non linearity and non uniformity, the time variation of the scintillator light yield and the different theoretical models for the $^{210}$Bi spectral shape. The analysis is performed on these pseudo datasets assuming the standard response to study how this impacts the final result on CNO, yielding a total systematic uncertainty of $^{+0.31}_{-0.24} \, \mathrm{cpd/100 \, tonnes}$. Other sources of systematic error are included in the estimation of the upper limit on $^{210}$Bi contamination, as discussed more in detail in ~\cite{Bx_nature_CNO,Bx_improved_CNO}. 

The negative log-likelihood profile as a function of the CNO rate is reported in Fig.~\ref{fig:results_dchi2}. The solid and dashed black lines show the results with and without systematic uncertainty, respectively. The result without CID constraint reported in ~\cite{Bx_improved_CNO} is included (blue line), for comparison. The improvement is clear especially for the upper value of the CNO rate.
The CNO interaction rate is  extracted from the 68\% quantile of the likelihood profile convoluted with the resulting systematic uncertainty, as $R^\mathrm{MV+CID}\mathrm{(CNO)}=6.7^{+1.2}_{-0.8} \, \mathrm{cpd/100 \, tonnes}$. The significance to the no-CNO hypothesis reaches about 8$\sigma$ C.L., while the resulting CNO flux at Earth is $\Phi(\mathrm{CNO})=6.7 ^{+1.2}_{-0.8} \times 10^8 \, \mathrm{cm^{-2} \, s^{-1}}$. 
Following the same procedure used in \cite{Bx_improved_CNO}, we use this result together with the $^8$B flux obtained from the global analysis of all solar data  to determine the abundance of C + N with respect to H in the Sun with an improved precision, for which we find $\rm{N_{\rm CN}} =5.81 ^{+1.22} _{-0.94} \times 10^{-4}$. This error includes both the statistical uncertainty due to the CNO measurement, and the systematic errors due to the additional contribution of the SSM inputs, to the ${^8}$B flux measurement, and to the $^{13}$N/$^{15}$O fluxes ratio. Similar to what was inferred from our previous publication, this result is in agreement with the High Metallicity measurements \cite{HZ, HZ1}, and features a 2$\sigma$ tension with Low Metallicity ones \cite{LZ,LZ1,LZ2}. 
Similarly, if we combine the new CNO result with the other Borexino results on $^7$Be and $^8$B in a frequentist hypothesis test based on a likelihood ratio test statistics we find that, assuming the HZ-SSM to be true, our data disfavours LZ-SSM at $3.2\sigma$ level.

% Figure environment removed

\section{Conclusions}
\label{sec:conclusion}
%\textcolor{blue}{Main writers: Davide and Luca}


In this work we have presented the results on CNO solar neutrinos obtained using the "Correlated and Integrated Directionality" (CID) technique.

We have shown that  the CID technique can be used to extract the CNO signal without any a priori assumptions on the backgrounds, in particular that of $^{210}$Bi. 
The Phase-I (May 2007 to May 2010, 740.7\,days) and Phase-II+III (December 2011 to October 2021, 2888.0\,days) datasets have been analyzed independently to investigate possible variations of the detector response over time. By adopting the Bayesian statistics, we have combined the conditionally independent results of Phase-I and Phase-II+III: the resulting CNO rate obtained with CID only is $7.2_{-2.7}^{+2.8} \, \text{cpd/100 \, tonnes}$. The no-CNO hypothesis including the $pep$ constraint only is rejected at 5.3$\sigma$ level. This result, albeit less precise than the one published by Borexino using the standard multivariate analysis, is the first obtained without the application of a $^{210}$Bi constraint.

We have also obtained an improved CNO solar neutrino result by combining the standard multivariate analysis with the CID technique.
The CID technique helps in separating the solar signal from non solar backgrounds, improving the significance and precision of the CNO measurement with respect to the result previously published by Borexino. 
The resulting CNO interaction rate is $6.7^{+1.2}_{-0.8} \, \mathrm{cpd/100\,tonnes}$ and the significance against the absence of a CNO signal, considered as the null hypothesis, is about 8$\sigma$.
The C+N abundance with respect to H is calculated from this result following the procedure adopted in \cite{Bx_improved_CNO} and is found to be $\rm{N_{\rm CN}} =5.81 ^{+1.22} _{-0.94} \times 10^{-4}$, compatible with the SSM-HZ metallicity measurements.

In conclusion, we have shown that the directional information of the Cherenkov radiation can be effectively combined with the spectral information coming from scintillation, for solar neutrino studies.
%providing an additional powerful tool for background separation. 
This combined detection approach provides a measurement that is more powerful than the individual methods on their own.
The sensitivity of the CID method could be significantly improved in future liquid scintillator-based detectors by optimizing the Cherenkov-to-scintillation ratio and by performing dedicated calibrations campaigns.

{\it Acknowledgments:}
We acknowledge the generous hospitality and support of the Laboratori Nazionali del Gran Sasso (Italy). The Borexino program is made possible by funding from Istituto Nazionale di Fisica Nucleare (INFN) (Italy), National Science Foundation (NSF) (USA), Deutsche Forschungsgemeinschaft (DFG), Cluster of Excellence PRISMA+ (Project ID 39083149), and recruitment initiative of Helmholtz-Gemeinschaft (HGF) (Germany), Russian Foundation for Basic Research (RFBR) (Grants No. 19-02-00097A), Russian Science Foundation (RSF) (Grant No. 21-12-00063) and Ministry of Science and Higher Education of the Russian Federation (Project FSWU-2023-0073) (Russia), and Narodowe Centrum Nauki (NCN) (Grant No. UMO 2017/26/M/ST2/00915) (Poland). We gratefully acknowledge the computing services of Bologna INFN-CNAF data centre and U-Lite Computing Center and Network Service at LNGS (Italy). 







%\bibliography{apssamp}% Produces the bibliography via BibTeX.
\bibliographystyle{apsrev4-1}
\bibliography{main.bib}
%\documentclass[a4paper,11pt]{article}
\pdfoutput=1 % if your are submitting a pdflatex (i.e. if you have
             % images in pdf, png or jpg format)

%\usepackage[utf8]{inputenc}
%\usepackage{mathrsfs, amssymb, amsmath}  
%\usepackage{comment}
%\usepackage{dcolumn}
%\usepackage{multirow}
%\usepackage{color}
%\usepackage{amsfonts,amssymb,amsmath, txfonts}
%\usepackage{float}

\usepackage{jcappub} % for details on the use of the package, please
                     % see the JCAP-author-manual

\usepackage[T1]{fontenc} % if needed

\hypersetup{ linktoc=all,
    colorlinks=true, linkcolor={blue},  
       citecolor={red}, urlcolor={darkred}
}
\definecolor{Redgreen}{RGB}{153,76,0}
\definecolor{vividviolet}{rgb}{0.62, 0.0, 1.0}
\definecolor{green}{RGB}{11,98,17}
\definecolor{darkgreen}{RGB}{40,150,65}
\definecolor{darkblue}{rgb}{0,0,0.3}
\definecolor{darkred}{rgb}{0.7,0,0}

\def\blue{\textcolor{blue}}
\def\red{\textcolor{red}}
\def\be{\begin{equation}}
\def\ee{\end{equation}}
\def\bea{\begin{eqnarray}}
\def\eea{\end{eqnarray}}


\title{MCMC Marginalisation Bias and $\Lambda$CDM tensions}
%\title{Overcoming bias in MCMC marginalisation to elucidate $\Lambda$CDM tensions}
%\title{Temp}

%%Markov Chain Monte Carlo

%% %simple case: 2 authors, same institution
%% \author{A. Uthor}
%% \author{and A. Nother Author}
%% \affiliation{Institution,\\Address, Country}

% more complex case: 4 authors, 3 institutions, 2 
\author[a]{Eoin \'O Colg\'ain}
\author[b]{Saeed Pourojaghi}
\author[b, c]{M. M. Sheikh-Jabbari}
\author[a]{Darragh Sherwin}

% The "\note" macro will give a warning: "Ignoring empty anchor..."
% you can safely ignore it.

\affiliation[a]{Atlantic Technological University, Ash Lane, Sligo, Ireland}
\affiliation[b]{School of Physics, Institute for Research in Fundamental Sciences (IPM), P.O.Box 19395-5531, Tehran, Iran}
\affiliation[c]{The Abdus Salam ICTP, Strada Costiera 11, I-34014 Trieste, Italy}

% e-mail addresses: one for each author, in the same order as the authors
\emailAdd{eoin.ocolgain@atu.ie}
\emailAdd{pourojaghi@ipm.ir}
\emailAdd{jabbari@theory.ipm.ac.ir}
\emailAdd{darragh.sherwin@research.atu.ie}




\abstract{Probability distributions become non-Gaussian when the flat $\Lambda$CDM model is fitted to redshift binned data in the late Universe. We explain mathematically why this non-Gaussianity arises and confirm that Markov Chain Monte Carlo (MCMC) marginalisation leads to biased inferences in observational Hubble data (OHD). In particular, in high redshift bins we find that $\chi^2$ minima, as identified from both least squares fitting and the MCMC chain, fall outside of the $1 \sigma$ confidence intervals. We resort to profile distributions to correct this bias. Doing so, we observe that $z \gtrsim 1$ cosmic chronometer (CC) data currently prefers a non-evolving (constant) Hubble parameter over a Planck-$\Lambda$CDM cosmology at $\sim 2 \sigma$. We confirm that both mock simulations and profile distributions agree on this significance. Moreover, on the assumption that the Planck-$\Lambda$CDM cosmological model is correct, using profile distributions we confirm  a $> 2 \sigma$ discrepancy with Planck-$\Lambda$CDM in a combination of  CC and baryon acoustic oscillations (BAO) data beyond $ z \sim 1.5$ that was noted earlier through comparison of least square fits of observed and mock data.}



\begin{document}
\maketitle
\flushbottom

\section{Introduction}
\label{sec:intro}
The flat $\Lambda$CDM model is the minimal model that fits Cosmic Microwave Background (CMB) data. Remarkably, CMB data from the Planck satellite \cite{Planck:2018vyg} constrains the $\Lambda$CDM model to sub-percent errors, thereby not only providing the strongest constraints, but also a concrete prediction for cosmological probes in the late Universe. The unmitigated success of the $\Lambda$CDM model is that CMB, Type Ia supernovae (SN) \cite{Riess:1998cb, Perlmutter:1998np} and baryon acoustic oscillations (BAO) \cite{Eisenstein:2005su} agree on a $\Lambda$CDM Universe that is approximately $30 \%$ matter. Thus, one key prediction of the Planck-$\Lambda$CDM model agrees across early and late Universe cosmological probes. Given this non-trivial agreement, any discrepancies that arise elsewhere constitute challenging puzzles. 

Nevertheless, one cannot define any \textit{model} for a dynamical system, especially a complicated system like the Universe, using data from a cosmic snapshot.\footnote{Here, we mean CMB data with an effective redshift $z \sim 1100$.} At best, one has a \textit{prediction} and not a model. In recent years, key predictions of Planck data have been challenged by late Universe determinations of the Hubble constant $H_0$ \cite{Riess:2021jrx, Freedman:2021ahq, Pesce:2020xfe, Blakeslee:2021rqi, Kourkchi:2020iyz} and the $S_8:= \sigma_8 \sqrt{\Omega_m/0.3}$ parameter \cite{HSC:2018mrq, KiDS:2020suj, DES:2021wwk, Boruah:2019icj, Said:2020epb}. Given the diversity of the late Universe probes (see reviews \cite{Perivolaropoulos:2021jda, Abdalla:2022yfr}), it is highly unlikely that any single systematic can be found to explain the discrepancies. That being said, in astrophysics one can never preclude systematics; 3 decades after Phillips' seminal paper \cite{Phillips:1993ng}, we are still debating an ad hoc correction for the mass of the host galaxy in Type Ia SN \cite{NearbySupernovaFactory:2018qkd, Kang:2019azh, Brout:2020msh, Lee:2021txi}. Bearing in mind that Type Ia SN are one of our best understood cosmological probes, one quickly understands that any systematics debate may be endless. 

Thus, it is far more expedient to assume that the $\Lambda$CDM model is breaking down and to look for tell-tale signatures of model breakdown. If signatures cannot be found, one arrives at a contradiction, and revisits the assumption that the model is breaking down. For physicists, \textit{model breakdown comes about when model fitting parameters return discrepant values at different time slices or epochs}. Translated into astronomy, this equates to discrepant cosmological parameters in different redshift ranges. The usual $H_0, S_8$ tensions  may also be viewed in the same light: a discrepancy between high and low redshift inferences/measurements of the parameters \cite{Perivolaropoulos:2021jda, Abdalla:2022yfr}. Nevertheless, early and late Universe observables are typically not the same, so one is confronted with a rich set of potential systematics. 

Within the context of $\Lambda$CDM tensions, it was recently observed that the integration constant from the Friedmann equations, aka the Hubble constant $H_0$, picks up redshift dependence whenever our model assumption - required to close the Friedmann equations - disagrees with the Hubble parameter $H(z)$ extracted from observations \cite{Krishnan:2020vaf, Krishnan:2022fzz}. %\footnote{One is free to speculate about the nature of the missing physics \cite{Liao:2020zko, Montani:2023xpd}.} 
Similarly, $\rho_{m0}=H_0^2\Omega_m$, an integration constant of the matter continuity equation, implies matter density $\Omega_m$ is a mathematically constant quantity. 
These are irrefutable predictions from mathematics, i. e. a prediction that is \textit{robust to systematics}. However, observationally $H_0$ and $\Omega_{m}$ are model fitting parameters and nothing precludes them picking up redshift dependence (except of course if one assumes they do not!), and providing a signature of model breakdown. If this happens in the late Universe within the $\Lambda$CDM model, $H_0$ is correlated with matter density $\Omega_m$, 
while $\Omega_m$ is correlated with $S_8 \propto \sigma_8 \sqrt{\Omega_m}$. Thus, there is at least one simple scenario, namely redshift evolution of cosmological parameters in the late Universe, where ``$H_0$ tension'' and ``$S_8$ tension'' are not independent and simply symptoms of $\Lambda$CDM model breakdown. 

The next relevant question is, where is the evidence for evolving cosmological parameters in the late Universe? Starting with strong lensing time delay \cite{Wong:2019kwg, Millon:2019slk},\footnote{Systematics are explored in \cite{Millon:2019slk} and the descending trend is not an obvious systematic. The lensed system RXJ1131-1231 \cite{Sluse:2003iy}, which partly drives the trend, has recently been re-analysed using spatially resolved stellar kinematics of the host galaxy \cite{Shajib:2023uig}, and the higher $H_0$ value remains robust, admittedly with inflated errors. As TDCOSMO project to analyse 40 lenses, the prospect of a discovery of a descending $H_0$ trend assuming the $\Lambda$CDM model remain strong.} descending trends of $H_0$ with redshift have been reported in Type Ia SN \cite{Dainotti:2021pqg, Colgain:2022nlb, Colgain:2022rxy,  Malekjani:2023dky, Hu:2022kes, Jia:2022ycc} and combinations of data sets \cite{Krishnan:2020obg, Dainotti:2022bzg}. On the other hand, larger values of $\Omega_m$ have been noted in high redshift observables, primarily quasars (QSOs) \cite{Risaliti:2015zla, Risaliti:2018reu, Lusso:2020pdb, Yang:2019vgk, Khadka:2020vlh, Khadka:2020tlm, Khadka:2021xcc, Pourojaghi:2022zrh},\footnote{Just as with Type Ia SN, the systematics of QSOs are being investigated \cite{Zajacek:2023qjm}.} but also Type Ia SN \cite{Colgain:2022nlb, Colgain:2022rxy, Malekjani:2023dky, Pasten:2023rpc} (see also \cite{Wagner:2022etu, Sakr:2023hrl}). Note, as emphasised earlier, if $H_0$ evolves at the background level, correlated fitting parameters are expected to also evolve. Moreover, mock analysis within the $\Lambda$CDM setting reveals that evolution of best fit $(H_0, \Omega_m)$ parameters cannot be precluded, and conversely possesses a finite likelihood, in either observational Hubble data (OHD) \textit{or} angular diameter distance data \textit{or} luminosity distance data \cite{Colgain:2022tql}. We stress that this result \textit{rests on mock analysis}; it represents a purely mathematical statement about the $\Lambda$CDM model that is independent of systematics. 

Separately, at the perturbative level, redshift evolution of $S_8$ or $\sigma_8$ has been reported in galaxy cluster number counts and Lyman-$\alpha$ spectra \cite{Esposito:2022plo}, $f \sigma_8$ constraints from peculiar velocities and redshift space distortions (RSD) 
 \cite{Adil:2023jtu}, comparison between weak \cite{HSC:2018mrq, KiDS:2020suj, DES:2021wwk} and CMB lensing \cite{ACT:2023dou, ACT:2023kun}. What is important here is that these observations appear to restrict the evolution in $S_8$ to the late Universe. In \cite{ACT:2023ipp} the possibility was raised that \textit{``tracers at higher redshift and probing larger scales prefer higher $S_8$''}.\footnote{There are also conflicting observations of high redshift $\sigma_8$ or $S_8$ values that are lower than Planck in the late Universe \cite{Miyatake:2021qjr, Alonso:2023guh}, so either this trend is not universal, or systematics are at play.} Nevertheless, one can argue against evolution with scale on the grounds that cosmic shear \cite{HSC:2018mrq, KiDS:2020suj, DES:2021wwk}, which is sensitive to smaller scales (larger $k$), and peculiar velocity constraints \cite{Boruah:2019icj, Said:2020epb}, which are sensitive to larger scales (smaller $k$), both prefer lower values of $S_8$. Moreover, both galaxy clusters and Lyman-$\alpha$ spectra are expected to probe similar scales.\footnote{We thank Matteo Viel for correspondence on this point.} Thus, if systematics are not impacting results, then redshift evolution is the only point of agreement in the observations \cite{Esposito:2022plo, Adil:2023jtu, HSC:2018mrq, KiDS:2020suj, DES:2021wwk, ACT:2023dou, ACT:2023kun, ACT:2023ipp}. Note also that redshift is more fundamental than scale in FLRW cosmology; one must solve the Friedmann equations in either time or redshift before one contemplates any discussion of scale.  

 The purpose of this letter is to revisit the analysis presented in \cite{Colgain:2022rxy,Colgain:2022tql}, where the evidence for evolution was quantified on the basis of mock simulations and not Markov Chain Monte Carlo (MCMC), the technique most familiar in cosmology. The fundamental problem is that once one bins low redshift data and studies evolution of cosmological parameters with bin redshift, one quickly encounters projection effects in MCMC analyses. These effects are not just the preserve of exotic models \cite{Herold:2021ksg, Gomez-Valent:2022hkb, Meiers:2023gft}, such as Early Dark Energy (EDE) \cite{Poulin:2018cxd, Niedermann:2019olb}, and happen in the simplest model when one bins data. The most striking demonstration of the resulting bias is that the peaks of MCMC posteriors no longer coincide with the minimum of the likelihood (see \cite{Gomez-Valent:2022hkb}). Ultimately, this bias is expected  because one is working in a regime of the $\Lambda$CDM model with non-Gaussian probability distributions   \cite{Colgain:2022tql}  (see also \cite{Colgain:2022rxy}).

 The structure of this paper is as follows. In section \ref{sec:MCMC_bias} we confirm the bias in MCMC marginalisation. In section \ref{sec:PD} we introduce profile distributions (PDs) \cite{Gomez-Valent:2022hkb} as a means of addressing the bias and confirm that the statistical significance of discrepancies from mock simulations agree well with PD analysis. In section \ref{sec:tension}, we revisit and confirm the high redshift OHD tensions reported in \cite{Colgain:2022rxy}. We end in section \ref{sec:discussion} with concluding remarks. 
 %A short appendix is also added on Fisher matrix for $\Lambda$CDM mdoel. 

\section{A bias in MCMC marginalisation}
\label{sec:MCMC_bias}
In this section we illustrate a bias in MCMC marginalisation that arises in the (flat) $\Lambda$CDM model when data is binned by redshift. This bias can be traced to a regime of the $\Lambda$CDM model with non-Gaussian distributions and is independent of systematics  \cite{Colgain:2022rxy, Colgain:2022tql}. 

\subsection{Mathematical Foundations}
\label{sec:math}
Consider an exercise where one bins OHD and confronts it to the $\Lambda$CDM Hubble Parameter $H(z)$ in the late Universe, a setting where the radiation sector can be safely decoupled. In high redshift bins ($z \gg 0$) in the matter-dominated regime, the Hubble parameter becomes insensitive to the dark energy (DE) sector: 
\be
\label{eq:lcdm}
H(z) = H_0 \sqrt{1-\Omega_m + \Omega_m (1+z)^3} \xrightarrow[z \gg 0]{} H_0 \sqrt{\Omega_m} (1+z)^{\frac{3}{2}}.  
\ee
More concretely, taking $z \rightarrow \infty$ we see that data can only constrain the combination $\rho_{m0}=H_0^2{\Omega_m}$. For \textit{hypothetical} data in a redshift bin with effective redshift $z = \infty$, this means that one can only constrain the combination $\Omega_m h^2$ ($h:= H_0/100)$, but $H_0$ and $\Omega_m$ remain unconstrained. Alternatively put, for any given $\Omega_m h^2$ constraint, there is an infinite number of corresponding $(H_0, \Omega_m)$ pairs. Translated into a probability density function (PDF), this is simply the statement that in a very high redshift bin at $z = \infty$, one expects uniform or flat distributions for $H_0$ and $\Omega_m$ with the model (\ref{eq:lcdm}).  

Of course, observed data resides at finite $z$ and not $z = \infty$. As a result, one does not encounter \textit{exactly} flat PDFs in $H_0$ and $\Omega_m$ at high redshift, but \textit{almost} flat PDFs. More important to us is the observation that these PDFs must flatten in a non-Gaussian manner. To appreciate this fact, we observe that high redshift OHD only constrains $\Omega_m h^2$ well.\footnote{Note that observables like SN or QSO that measure $D_L(z)=c (1+z)\int_0^z \textrm{d} z'/H(z')$ are mainly sensitive to the low redshift part of $H(z)$, i. e. the combination $H_0^2 (1-\Omega_m)$, and in this sense they are complementary to the OHD data which is more sensitive to high redshift part of $H(z)$, $H_0^2\Omega_m$. The complementarity can be demonstrated by combining $H(z)$ and $D_{L}(z)$ constraints and checking that one recovers mock data input parameters in all redshift bins \cite{Colgain:2022tql}. } For this reason, best fit parameters are constrained to a $\Omega_m h^2 = \textrm{constant}$ curve in the $(H_0, \Omega_m)$-plane. The almost flat $H_0$ and $\Omega_m$ PDFs can only arise if this curve stretches in the $(H_0, \Omega_m)$-plane. As a result of this stretching, one ends up with a relatively uniform distribution on a curve. At the extremes of the curve, one finds a distribution of large $H_0$ values, which do not differ greatly in $\Omega_m$, and they get projected to a peak at small values on the $\Omega_m$ axis. Conversely, at the other end of the curve, one finds a distribution of small $\Omega_m$ values, which do not differ greatly in $H_0$, and they get projected onto a peak at large values on the $H_0$ axis.  This is a ``projection effect'' in common cosmology parlance.  It is driven by the irrelevance of the DE sector at high redshift and the constraint $\Omega_m h^2 = \textrm{constant}$ from the $\Lambda$CDM model (\ref{eq:lcdm}). Together these features distort the distribution away from a Gaussian configuration. 

Thus, simply by binning and fitting OHD to the $\Lambda$CDM model one enters a non-Gaussian regime as the effective redshift of the bin increases. This effect, which is expected from the purely mathematical arguments above, has been confirmed in mock data \cite{Colgain:2022rxy, Colgain:2022tql}, and in line with expectations, we demonstrate that it impacts MCMC inferences with observed data in the next subsection.  

% Figure environment removed

\subsection{Cosmic Chronometer (CC) Data}
\label{sec:CCbias}
Here we work with OHD from the cosmic chronometer (CC) program \cite{Jimenez:2001gg}. Concretely, we work with 34 $H(z)$ constraints spanning the redshift range $0.07 \leq z \leq 1.965$ \cite{Stern:2009ep, Moresco:2012jh, Zhang:2012mp, Moresco:2016mzx, Ratsimbazafy:2017vga, Borghi:2021rft, Jiao:2022aep, Tomasetti:2023kek}. We illustrate the data in Fig.~\ref{fig:CC}, where it is consistent with Fig. 9 of \cite{Tomasetti:2023kek} {modulo the fact that we have an additional data point at $z = 0.8$, which is not independent. See Table 1.1 of \cite{Moresco:2023zys}. While CC data may eventually be good enough to arbitrate on Hubble tension \cite{Moresco:2023zys}, the data is not good enough on its own to do cosmology. To put this comment in context, we observe that the errors in Fig.~\ref{fig:CC} do not include systematic errors (see \cite{Moresco:2020fbm} for an account of the systematics). As a result the constraints we get on cosmological parameters will be underestimated. Thus, from our perspective the data in Fig.~\ref{fig:CC} is simply some representative cosmological data in the OHD class.}

\paragraph{Methodology:} We impose a low redshift cut-off on the OHD $z_{\textrm{min}}$, removing all data points with redshifts $z_i < z_{\textrm{min}}$, and then extremising the $\chi^2$ likelihood, 
\be
\label{eq:chi2}
\chi^2 = Q^{T} \cdot C^{-1} \cdot Q, 
\ee
where $C$ is the covariance matrix, which is simply the square of the $H_i$ errors on the diagonal, and $Q$ is the vector, 
\be
\label{eq:Q}
Q_i = H_i - H_{\textrm{model}}(z_i), 
\ee
where $H_i:=H(z_i)$ denotes OHD and $H_{\textrm{model}}(z)$ is the model (\ref{eq:lcdm}) without the high redshift limit. The best fit $(H_0, \Omega_m)$ parameters correspond to the minumum of the $\chi^2$, while on the assumption of Gaussian errors, we estimate the errors from a Fisher matrix (appendix \ref{sec:fisher}). In parallel, we perform MCMC marginalisation through \textit{emcee} \cite{Foreman-Mackey:2012any}. More concretely, subject to the priors $H_0 \in [0, 200 ]$ and $\Omega_m \in [ 0, 1]$, the latter restricting us to a physical regime, we record $16^{\textrm{th}}$, $50^{\textrm{th}}$ and $84^{\textrm{th}}$ percentiles for MCMC posteriors, as is common practice with Gaussian distributions. Thus, both techniques are tailored to Gaussian posteriors, yet non-Gaussianities will be evident in MCMC posteriors. By comparing the output from these two techniques in Table \ref{tab:LCDM_CC} for different values of $z_{\textrm{min}}$ we observe that error estimates from Fisher matrix and MCMC quickly disagree as $z_{\textrm{min}}$ increases. 

From Table \ref{tab:LCDM_CC}, we see that MCMC inferences lead to non-Gaussian $1 \sigma$ confidence intervals, where in line with the expectations from \cite{Colgain:2022tql}, $H_0$ errors are larger for smaller values, and $\Omega_m$ errors are larger for larger values, respectively. This is expected if the $H_0$ and $\Omega_m$ posteriors are peaked at larger and smaller values, respectively, in line with our earlier mathematical argument. Only for the full data set with $z_{\textrm{min}} = 0$  do we find reasonable agreement between the Fisher matrix and MCMC $1 \sigma$ confidence intervals. As can be seen from the lopsided MCMC confidence intervals, the non-Gaussianity becomes more pronounced with increasing $z_{\textrm{min}}$. Interestingly, beyond $z_{\textrm{min}} = 1$, the minimum of the $\chi^2$ falls outside of the MCMC $1 \sigma$ confidence intervals. Nevertheless, by evaluating the MCMC chains on the $\chi^2$ likelihood (\ref{eq:chi2}), we confirm that the parameters corresponding to the minimum $\chi^2$ value are tracking the best fit. Note, the peak of the MCMC posterior is no longer a measure of goodness of fit and inferences have become biased in a regime of model parameter space where distributions are expected to be inherently non-Gaussian. Our analysis here underscores potential problems with a blind MCMC analysis with the traditional $16^{\textrm{th}}$, $50^{\textrm{th}}$ and $84^{\textrm{th}}$ percentiles.       



\begin{table}[htb]
    \centering
    \begin{tabular}{c|c|c|c|c|c}
    \rule{0pt}{3ex} $z_{\textrm{min}}$ & \# CC & \multicolumn{2}{c}{Fisher Matrix}  & \multicolumn{2}{|c}{MCMC} \\
    \hline
    \rule{0pt}{3ex} & & $H_0$ (km/s/Mpc) & $\Omega_m$ & $H_0$ (km/s/Mpc) & $\Omega_m$ \\
    \hline
    \rule{0pt}{3ex} $0$ & $34$ & $68.14 \pm 3.07$ & $0.320 \pm 0.059$ & $67.76^{+3.03}_{-3.09}$  ($68.12$) & $0.328^{+0.065}_{-0.055}$ ($0.321$) \\
    \hline 
    \rule{0pt}{3ex} $0.2$ & $27$ & $65.03 \pm 6.65$ & $0.368 \pm 0.118$ & $63.05^{+6.64}_{-7.23}$ ($64.98$) & $0.405^{+0.170}_{-0.111}$ ($0.369$) \\
    \hline 
    \rule{0pt}{3ex} $0.4$ & $22$ & $62.42 \pm 8.38$ & $0.411 \pm 0.161$ & $59.54^{+8.30}_{-8.22}$ ($62.39$) & $0.470^{+0.229}_{-0.151}$ ($0.411$)\\
    \hline 
    \rule{0pt}{3ex} $0.6$ & $15$ & $59.83 \pm 17.21$ & $0.454 \pm 0.338$ & $56.45^{+13.16}_{-9.33}$ ($59.86$) & $0.526^{+0.288}_{-0.225}$ ($0.453$) \\
    \hline 
    \rule{0pt}{3ex} $0.7$ & $14$ & $79.11 \pm 19.40$ & $0.222 \pm 0.162$ & $67.59^{+19.19}_{-16.57}$ ($79.18$) & $0.344^{+0.344}_{-0.178}$ ($0.222$) \\
    \hline 
    \rule{0pt}{3ex} $0.8$ & $11$ & $103.97 \pm 24.94$ & $0.097 \pm 0.088$ & $82.43^{+28.33}_{-27.03}$ ($104.02$) & $0.206^{+0.357}_{-0.131}$ ($0.096$) \\
    \hline 
    \rule{0pt}{3ex} $1$ & $8$ & $150.37 \pm 31.21$ & $0.010 \pm 0.035$ & $108.92^{+33.94}_{-44.47}$ ($150.38$) & $0.087^{+0.304}_{-0.068}$ ($0.010$) \\
    \hline 
    \rule{0pt}{3ex} $1.2$ & $7$ & $154.35 \pm 42.95$ & $0.006 \pm 0.042$ & $83.07^{+48.52}_{-32.19}$ ($154.47$) & $0.194^{+0.439}_{-0.159}$ ($0.006$) \\
    \hline 
    \rule{0pt}{3ex} $1.4$ & $4$ & $125.41 \pm 79.55$ & $0.039 \pm 0.132$ & $65.32^{+44.88}_{-20.30}$ ($125.44$) & $0.320^{+0.423}_{-0.250}$ ($0.039$) \\
    \hline 
    \rule{0pt}{3ex} $1.5$ & $3$ & $36.12 \pm 72.69$ & $1.000 \pm 4.269$ & $55.19^{+34.64}_{-14.73}$ ($36.16$) & $0.393^{+0.387}_{-0.283}$ ($0.999$)
    \end{tabular}
    \caption{Comparison between Fisher matrix and MCMC analysis for CC data with a low redshift cut-off $z_{\textrm{min}}$. We record the number of data points, the extremum of the $\chi^2$ and $1 \sigma$ confidence interval estimated from the Fisher matrix,  $16^{\textrm{th}}$, $50^{\textrm{th}}$ and $84^{\textrm{th}}$ percentiles from MCMC posteriors corresponding to $1 \sigma$ confidence intervals, and the minimum $\chi^2$ from the MCMC chain in brackets. MCMC marginalisation exhibits non-Gaussian $1 \sigma$ confidence intervals, and for $z_{\textrm{min}} > 1$, the minimum value of the $\chi^2$ from the MCMC chain falls outside of this interval. The latter tracks the best fit up to small numbers in line with expectations. }
    \label{tab:LCDM_CC}
\end{table}

\subsection{Features in CC Data}
\label{sec:features}
Once one accounts for biases, it is clear from Table \ref{tab:LCDM_CC} that there are trends in CC data when it is binned. Starting from $z_{\textrm{min}} = 0$ through to $z_{\textrm{min}} = 0.6$ we see a decreasing trend in best fit values of $H_0$ (also central $H_0$ values from MCMC), which is compensated by a increasing trend in $\Omega_m$ best fit values. From Fig.~\ref{fig:CC} it is difficult to visibly discern any trend from the raw data. From $z_{\textrm{min}} = 0.7$ through to $z_{\textrm{min}} = 1.4$, there is in contrast a preference for larger $H_0$ and smaller $\Omega_m$ values. This trend is evident from the raw data, where at higher redshifts one sees large scatter and large fractional errors in the data. For $z_{\textrm{min}} = 1$, it is clear that the best fit line in magenta corresponding to $(H_0, \Omega_m) = (150.4, 0.01)$ (Table \ref{tab:LCDM_CC}) is closer to horizontal line than the Planck-$\Lambda$CDM cosmology in red. To be more explicit, for $z_{\textrm{min}} = 0$, $\rho_{m0}:=H_0^2\Omega_m\simeq 1500$ which is close to the Planck value, whereas for $z_{\textrm{min}} = 1$, $\rho_{m0}\simeq 225$. The sharp drop in $\rho_{m0}$ means the magenta line should be almost horizontal. For $z_{\textrm{min}} = 1.5$, we switch to an opposite regime of parameter space with unexpectedly low and high values of $H_0$ and $\Omega_m$, respectively, a trend which is evident in the data, but there are only three data points. Despite, the small number of data points, the tendency for smaller $H_0$ and larger $\Omega_m$ inferences within $\Lambda$CDM cosmology at high redshifts has been documented across three independent observables \cite{Colgain:2022rxy}. We will come back to this claim in section \ref{sec:tension}. Finally, it is worth noting that for large $z_{\textrm{min}}$ and samples with few data points, one expects broad MCMC posteriors. These posteriors are severely impacted by the prior on $\Omega_m$, as is evident from Table \ref{tab:LCDM_CC}. 

For the moment we leave physical speculations to the discussion and return to the trend in CC data above $z=1$ favouring less evolution in the Hubble parameter than the Planck-$\Lambda$CDM model. We would like to quantify the significance of this trend, but since we are working in a non-Gaussian regime of the model, we can expect both Fisher matrix and MCMC to give biased results. In Fig.~\ref{fig:CCsplit1} we show MCMC posteriors for $z>1$ CC data in blue alongside posteriors for low redshift ($z < 1$) CC data, which is simply added to aid comparison and also highlight the Gaussianity of the low redshift posteriors. One notes that the peaks of the $z > 1$ distributions are a little displaced from to the values minimising the $\chi^2$. However, the emergence of the lower peak in the $H_0$ posterior at $H_0 \sim 50$ km/s/Mpc has the hallmarks of a projection effect. To appreciate this, note that the configurations in the blue curve in the top left corner of the 2D posterior are projected onto the lower $H_0$ peak. Moreover, if one shifts the $H_0$ peak from $H_0 \sim 150$ to $H_0 \sim 50$ km/s/Mpc while maintaining $\Omega_m \sim 0$, this shifts the magenta curve in Fig. \ref{fig:CC} outside of all the data points, so the lower $H_0$ peak is a phantom artefact unrelated to the goodness of fit. We also observe a shift in the higher $H_0$ peak away from the minimum of the $\chi^2$.

Ignoring these features, one could attempt to interpret the overlap in the 2D posteriors in Fig. \ref{fig:CCsplit1}. Doing so, one may conclude that low and high redshift CC data are consistent within $1 \sigma$. However, since Hubble tension is a 1D problem (local $H_0$ determinations are insensitive to other parameters), to compare with locally observed values of $H_0$ one needs to project onto the $H_0$ axis. Alternatively put, Hubble tension is a problem in 1D posteriors. Projecting onto the $H_0$ axis by determining $16^{\textrm{th}}$, $50^{\textrm{th}}$ and $84^{\textrm{th}}$ percentiles, one sees from Table \ref{tab:LCDM_CC} that the $z_{\textrm{min}} = 1$ MCMC confidence interval encloses the $z_{\textrm{min}} = 0$ central values within $1 \sigma$,\footnote{Note, removing the eight high redshift data points from the $z_{\textrm{min}} = 0$ sample will not shift the central values much.} but not the point in parameter space that best fits the data!


% Figure environment removed



Evidently, given the non-Gaussian posteriors, care is required when interpreting the significance of the trend towards a non-evolving (horizontal) $H(z)$ at higher redshifts in Fig.~\ref{fig:CC}. We cannot use the errors from the Fisher matrix as we are clearly in a non-Gaussian regime, whereas MCMC inferences are impacted by projection effects to the extent that the minimum of the $\chi^2$ (confirmed from the MCMC chain) falls outside of the $1 \sigma$ confidence interval. For this reason, we resort to mock simulations. While this may seem a little redundant if we are going to employ profile distributions in section \ref{sec:PD}, there is motivation for this exercise. In \cite{Colgain:2022rxy} the significance of a descending $H_0$/increasing $\Omega_m$ trend with effective redshift in OHD, Type Ia SN and QSOs was estimated to be a $\sim 3 \sigma$ effect on the basis of combining $\sim 2 \sigma$ effects in each of the \textit{independent} data sets using Fisher's method. Here, working with the same data throughout, we can directly compare the significance of a discrepancy estimated through mock simulations from the significance of a discrepancy estimated through profile distributions. In particular, we will address the question: how significant is a constant $H(z)$ with $z_{\textrm{min}}=1$ (8 data points) against the Planck consistent cosmology favoured by the full data set ($z_{\textrm{min}}=0$ entry in Table \ref{tab:LCDM_CC})? Note, the significance will be overestimated due to missing systematic uncertainties (see \cite{Moresco:2020fbm}), but we can still make comparison between the two techniques.

\paragraph{{Mock simulations:}} To address this question using mock simulations, we begin with the MCMC chains for the full sample. For each entry in the MCMC chain (approximately 15,000 entries in total), we generate a new realisation of the 8 high redshift data points $(z > 1)$ that are by construction statistically consistent with both the best fits from the full sample and also the Planck-$\Lambda$CDM values \cite{Planck:2018vyg}. More concretely, for each $(H_0, \Omega_m)$ entry in our MCMC chain, we displace the data points to the corresponding $\Lambda$CDM Hubble parameter before generating new data points in a normal distribution where the errors serve as standard deviations. We then fit back the $\Lambda$CDM model to each realisation of the mock data and record the best fit $(H_0, \Omega_m)$ values, which give us a distribution of expected $(H_0, \Omega_m)$ best fits. The distributions are presented in Fig.~\ref{fig:CCsims} alongside the best fits from observed data. Throughout, we assume canonical values $(H_0, \Omega_m) = (70, 0.3)$ for the initial guess of the fitting algorithm. Best fits can saturate our bounds, i. e. $\Omega_m = 0$ and $\Omega_m = 1$, and this leads to an unsightly pile up of best fits at $\Omega_m = 0$ and $\Omega_m = 1$ in Fig.~\ref{fig:CCsims} \cite{Colgain:2022rxy}. It is important to retain all the configurations, otherwise one is not accounting for the probability that a best fit falls outside our priors. As a consistency check, we see that the median or 50$^{\textrm{th}}$ percentile, $(H_0, \Omega_m) = (68.32, 0.321)$ agrees well with the mock input parameters, thereby demonstrating that there are an equal number of best fits with values above and below the injected parameters in the mocks. We find that probability of a more extreme (larger) $H_0$ value to be $p = 0.022$, while the probability of a more extreme (smaller) $\Omega_m$ value to be $p = 0.035$, respectively. Converted into a Gaussian statistic, these correspond to $2 \sigma$ and $1.8 \sigma$, respectively, for a one-sided normal distribution. Thus, on the basis of mock simulations, we estimate the non-evolving constant $H(z)$ with $z_{\textrm{min}} = 1$ as a $\sim 2 \sigma$ effect. In the next section we will recover this number more or less from the profile distribution analysis. 

% Figure environment removed


\section{Profile Distributions}
\label{sec:PD}
Having explained the mathematics behind the bias, which gives rise to a projection effect, in subsection \ref{sec:math}, and having illustrated how it affects MCMC inferences in subsection \ref{sec:CCbias} - the minimum of the $\chi^2$ may fall outside of $1 \sigma$ confidence intervals - we turn to profile distributions (PDs) \cite{Gomez-Valent:2022hkb}, an extension of the profile likelihood, e. g. \cite{Trotta:2017wnx}, in order to address the bias. Consider two sets of parameters $\theta_1$ and $\theta_2$ and a normalised distribution $\mathcal{P}(\theta_1, \theta_2)$. The basic idea \cite{Gomez-Valent:2022hkb} is to study the ratio 
\be
\label{R}
R(\theta_1) = \frac{\tilde{\mathcal{P}}(\theta_1)}{\max_{\theta_1} \tilde{\mathcal{P}}(\theta_1) } = \frac{\tilde{\mathcal{P}}(\theta_1)}{\max_{\theta_1, \theta_2} \mathcal{P}(\theta_1, \theta_2) },  
\ee
where $\tilde{\mathcal{P}}(\theta_1)$ is the PD, defined to be the maximum of $\mathcal{P}$ for each $\theta_1$ along the $\theta_2$ direction: 
\be
\label{PD}
\tilde{\mathcal{P}} (\theta_1) = \max_{\theta_2} \mathcal{P}(\theta_1, \theta_2). 
\ee
The advantage of this approach is that $R(\theta_1)$ can serve as a probability distribution function (up to an overall normalization), however we do not need to perform any integration, so $R(\theta_1)$ is not prone to volume or projection effects. At this juncture, given the simplicity of our setup with only two parameters $(H_0, \Omega_m)$, we can be more explicit. Consider the probability distribution,   
\be
\mathcal{P}(\theta_1, \theta_2) = \exp \left( - \frac{1}{2} \chi^2(\theta_1, \theta_2) \right), 
\ee
where $\theta_i \in \{H_0, \Omega_m \}$  and $\chi^2(H_0, \Omega_m)$ is our earlier likelihood (\ref{eq:chi2}). The maximum value of $\mathcal{P}$ occurs for the minimum value of $\chi^2$ from the MCMC chain, $\mathcal{P}_{\textrm{max}} = e^{-\frac{1}{2} \chi^2_{\textrm{min}}}$. In this concrete setting, the PD becomes 
\be
\tilde{\mathcal{P}}(\theta_1) = e^{-\frac{1}{2} \chi^2_{\textrm{min}}(\theta_1)}, 
\ee
where $\chi^2_{\textrm{min}}(\theta_1)$ denotes the minimum value of the $\chi^2$ along the $\theta_2$ direction for a fixed $\theta_1$ value. It should not be confused with the overall minimum $\chi^2_{\textrm{min}}$, which can be extracted easily from the MCMC chain. In practice, one can also determine $\chi^2_{\textrm{min}}(\theta_1)$ from the MCMC chain by breaking the $\theta_1$ direction up into bins and finding the minimum of the $\chi^2$ for each bin. Having done so, we are in a position to define a PDF \cite{Gomez-Valent:2022hkb}: 
\be
\label{eq:w}
w(\theta_1) = \frac{e^{-\frac{1}{2} \chi^2_{\textrm{min}}(\theta_1)}}{\int e^{-\frac{1}{2} \chi^2_{\textrm{min}}(\theta_1)} \, \textrm{d} \theta_1} = \frac{R(\theta_1)}{\int R(\theta_1) \, \textrm{d} \theta_1}, 
\ee
where in the second equality we have divided top and bottom by $\mathcal{P}_{\textrm{max}} = e^{-\frac{1}{2} \chi^2_{\textrm{min}}}$. As a result, $R(\theta_1) = e^{-\frac{1}{2} \Delta \chi_{\textrm{min}}^2}$, where $\Delta \chi^2_{\textrm{min}} := \chi_{\textrm{min}}^2(\theta_1) - \chi^2_{\textrm{min}}$, so that $R(\theta_1)$ peaks at $R(\theta_1) = 1$. Note that $\int_{-\infty}^{+\infty} w(\theta_1) \, \textrm{d} \theta_1 = 1$ by construction, so $w(\theta_1)$ describes a properly normalised PDF. Thus we can identify the $1 \sigma, 2 \sigma$ and $3 \sigma$ confidence intervals corresponding to the 68\%, 95\% and 99.7\% confidence level, respectively, by simply identifying $\theta_1^{(1)}$ and $\theta_1^{(2)}$ such that \cite{Gomez-Valent:2022hkb}
\be
\label{eq:wsigma}
\int_{\theta_1^{(1)}}^{\theta_1^{(2)}} w(\theta_1) \, \textrm{d} \theta_1 = I, \quad w(\theta_1) = w(\theta_2), \quad I \in \{0.68, 0.95, 0.997\}. 
\ee
We will outline how these conditions can most easily be satisfied when we turn to explicit examples. 

Our first port of call is making sure that the PD methodology gives sensible results. This can be best judged by applying it to the CC data with $z_{\textrm{min}} = 0$, since this is where we expect a distribution closest to a Gaussian distribution, as is evident from the agreement between Fisher matrix and MCMC results in Table \ref{tab:LCDM_CC}. In particular, we will be interested in a comparison between $1 \sigma$ confidence intervals to make sure that (\ref{eq:wsigma}) is not underestimating or overestimating the $1 \sigma$ confidence interval. 

% Figure environment removed

We start by running a long MCMC chain (100,000 iterations) in order to ensure bins are well populated, and begin by analysing $\theta_1 = H_0$ with $\theta_2 = \Omega_m$. From the MCMC chain we identify the smallest and largest value of $H_0$ in the chain and break up this range into approximately 200 uniform bins, which we label using the $H_0$ value at the centre of the bin. We omit any empty bins. One can increase the number of bins by simply running a longer MCMC chain. In each $H_0$ bin we identify the minimum value of the $\chi^2$, $\chi^2_{\textrm{min}}(H_0)$, and calculate $R(H_0)$. One then repeats the steps for $\Omega_m$. In Fig.~\ref{fig:R_zmin0} we plot $R(H_0)$ against $H_0$ and $R(\Omega_m)$ against $\Omega_m$, noting that the distributions are Gaussian to first approximation. 

Since the distributions from the MCMC chain are sparse in the tails, empty bins are evident in Fig.~\ref{fig:R_zmin0}. Nevertheless, with 200 bins, modulo any empty bins, we have sufficient density of points to calculate the total area under the $R(H_0)$ and $R(\Omega_m)$ curve using Simpson's rule. Any concern about precision can simply be mitigated by running a longer MCMC chain and increasing the number of bins. 
One may directly use $R(H_0)\leq 1$ and $R(\Omega_m)\leq 1$   to find $68$, $95$ and $99.7$ percentiles,  respectively corresponding to $1 \sigma, 2 \sigma$ and $3 \sigma$ confidence intervals. Consider $F_\kappa:= \int_{R\geq \kappa} R (\theta_1) \, \textrm{d} \theta_1$, where $\kappa \leq 1$. Observe that $F_{\kappa=1}=0$ and $F_{\kappa=0}:=F_0=\int R(\theta_1) \textrm{d} \, \theta_1$. Then move $\kappa$ through and terminate the process when $F_\kappa/F_0$ is equal to $0.68$, $0.95$ and $0.997$. This gives the corresponding range for $\theta_1$ that defines the confidence interval.
Working with the precision afforded to us by approximately 200 bins, the $H_0$ and $\Omega_m$ $1 \sigma$ confidence intervals are presented in Fig.~\ref{fig:R_zmin0} and the first entry in Table \ref{tab:LCDM_CC_PD}. The outcome is in excellent agreement with both Fisher matrix and MCMC analysis. In particular, a mild non-Gaussianity in $\Omega_m$ is evident in both Fig.~\ref{fig:R_zmin0} and the errors. 
Thus, we have succeeded in recovering results in the (almost) Gaussian regime that are consistent with Fisher matrix and MCMC analysis and this provides an important check of the methodology.  

% Figure environment removed

We now apply the same PD methodology to the non-Gaussian regime where MCMC marginalisation leads to biased results. To be concrete, we focus on the eight data points in the range $1 < z < 2$ where a non-evolving $H(z)$ trend is evident in the raw data in Fig.~\ref{fig:CC}. Our goal here is to quantify the disagreement with the full data set, where one infers $H_0 \sim 68$ km/s/Mpc and $\Omega_m \sim 0.32$. A similar exercise was performed in subsection \ref{sec:features} with mock simulations and the disagreement was estimated to be approximately $2 \sigma$. Repeating the steps outlined above for the CC data with $z_{\textrm{min}} = 1$ we find the distributions in Fig.~\ref{fig:R_zmin1}. The first observation is that the distributions are non-Gaussian, but a comparison to the MCMC posteriors from the same data in blue in Fig.~\ref{fig:CCsplit1} reveals that there is no secondary $H_0$ peak at $H_0 \sim 50$ km/s/Mpc. Thus, we confirm the secondary peak to be a projection effect. That being said, the primary $H_0$ peak from Fig.~\ref{fig:CCsplit1} has shifted to the dashed line corresponding to the minimum of the $\chi^2$, since the peak of the distribution and $\chi^2$ minimum agree by construction. Comparing the blue $\Omega_m$ distribution from Fig.~\ref{fig:CCsplit1} to the $R(\Omega_m)$ distribution in Fig.~\ref{fig:R_zmin1}, we see that the peak is close to $\Omega_m = 0$ and that the tails continue to $\Omega_m = 1$. In both plots we see that there is a non-zero probability of inferring $\Omega_m = 1$. In some sense, this is not so surprising, the reason being that one is free to adopt generous priors for $H_0$, so that probability of large and small $H_0$ values is zero, but the priors on $\Omega_m$ in the flat $\Lambda$CDM model are restricted. For this reason, as a distribution spreads one invariably finds that distributions are impacted by the $\Omega_m$ priors.\footnote{Note, this is a problem for the flat $\Lambda$CDM model. In particular, one may easily find that the peak of the $\Omega_m$ distribution is larger than $\Omega_m=1$, as is the case with Hubble Space Telescope SN with redshifts $z > 1$ in the Pantheon+ sample \cite{Malekjani:2023dky}.}

It is evident from Fig.~\ref{fig:R_zmin1} that any tension that exists is confined to the $H_0$ parameter. Moreover, since there may be only one binned value of $\Omega_m$ below the $R(\Omega_m)$ peak, at the precision afforded to us by 200 bins, the $R(\Omega_m)$ distribution in Fig.~\ref{fig:R_zmin1} is essentially one-sided and the $1 \sigma$ confidence interval stretches beyond $\Omega_m \sim 0.32$, so there is no disagreement in the $\Omega_m$ parameter. Nevertheless, in the $H_0$ parameter we see that $H_0 \sim 68$ km/s/Mpc, the value favoured by the full data set is just under $2 \sigma$ removed from the peak. The main point here is that, as is obvious from the raw data, current CC data with $z > 1$ has a preference for a non-evolving Hubble parameter $H(z)$ with a large constant $H_0 \sim 150$ km/s/Mpc. The disagreement is just under $2 \sigma$, more accurately $1.9 \sigma$ from $R(H_0)$, and only $0.9 \sigma$ from $R(\Omega_m)$. Although this may not be a serious discrepancy, essentially because of the poor data quality (8 data points), this disagreement supports the $\sim 2 \sigma$ discrepancy seen in the mock simulations. It should be borne in mind that systematic uncertainties have been omitted and these will reduce this discrepancy once properly propagated. Given the agreement between the PD and mock simulation analysis, there is nothing to suggest that the three independent trends highlighted in \cite{Colgain:2022rxy} across OHD, Type Ia SN and QSOs are not \textit{bona fide} disagreements and that redshift evolution is present in the sample. The task remains to combine them at the level of a $\chi^2$ likelihood instead of combining them using Fisher's method on the basis that they are independent probabilities. We leave this exercise for a forthcoming paper, but revisit the tension in OHD data in the following section.  %\ref{sec:tension}. 
For completeness, in Table \ref{tab:LCDM_CC_PD} we perform a reanalysis of CC data subsets with the PD approach and record the $1 \sigma$ intervals.  

\begin{table}[htb]
    \centering
    \begin{tabular}{c|c|c|c}
    \rule{0pt}{3ex} $z_{\textrm{min}}$ & \# CC & \multicolumn{2}{c}{PD}  \\
    \hline
    \rule{0pt}{3ex} & & $H_0$ (km/s/Mpc) & $\Omega_m$ \\
    \hline
    \rule{0pt}{3ex} $0$ & $34$ & $68.15^{+3.04}_{-3.11}$ & $0.320^{+0.065}_{-0.055}$ \\
    \hline 
    \rule{0pt}{3ex} $0.2$ & $27$ & $65.03^{+6.52}_{-7.03}$ & $0.368^{+0.167}_{-0.110}$ \\
    \hline 
    \rule{0pt}{3ex} $0.4$ & $22$ & $62.42^{+7.78}_{-8.74}$ & $0.411^{+0.236}_{-0.113}$ \\
    \hline
    \rule{0pt}{3ex} $0.6$ & $15$ & $59.75^{+11.73}_{-13.97}$ & $0.455^{+0.355}_{-0.160}$ \\
    \hline
    \rule{0pt}{3ex} $0.7$ & $14$ & $79.10^{+16.42}_{-20.56}$ & $0.222^{+0.386}_{-0.117}$ \\
    \hline
    \rule{0pt}{3ex} $0.8$ & $11$ & $103.94^{+22.88}_{-28.54}$ & $0.097^{+0.378}_{-0.074}$ \\
    \hline
    \rule{0pt}{3ex} $1$ & $8$ & $150.35^{+17.12}_{-35.95}$ & $ < 0.339$ \\
    \hline
    \rule{0pt}{3ex} $1.2$ & $7$ & $154.26^{+14.88}_{-54.82}$ & $ < 0.570$ \\
    \hline
    \rule{0pt}{3ex} $1.4$ & $4$ & $124.81^{+35.38}_{-52.60}$ & $ < 0.661$ \\
    \hline
    \rule{0pt}{3ex} $1.5$ & $3$ & $36.11^{+72.87}_{-2.43}$ & $ > 0.354$
    \end{tabular}
    \caption{Same as Table \ref{tab:LCDM_CC} but with the PD methodology in lieu of Fisher matrix and MCMC analysis. The high redshift $R(\Omega_m)$ distributions are typically one-sided, so one encounters $1 \sigma$ upper and lower bounds.}
    \label{tab:LCDM_CC_PD}
\end{table}




\section{A tension with Planck}
\label{sec:tension}
A $2 \sigma$ ($p = 0.021$) tension with Planck has been reported in OHD through best fits and mock simulations in \cite{Colgain:2022rxy}. In particular, it was noted that a combination of 7 CC and BAO data points above $z = 1.45$ resulted in a $(H_0, \Omega_m) = (37.8, 1)$ best fit, where in line with analysis here, an $\Omega_m \in [0, 1]$ uniform prior was assumed. Based on mock simulations, the probability of such a best fit configuration arising by chance in mocks assuming input parameters consistent with Planck was estimated to be $p = 0.021$ \cite{Colgain:2022rxy}. A similar best fit appears in the last entry of Table \ref{tab:LCDM_CC} and Table \ref{tab:LCDM_CC_PD}, but there is no tension with Planck within the errors, even with our PD analysis, because CC data is inherently of poorer quality than BAO data. One further difference between the analysis is that \cite{Colgain:2022rxy} imposes a Gaussian Planck prior $\Omega_m h^2 = 0.1430 \pm 0.0011$ \cite{Planck:2018vyg} \footnote{This prior essentially prevents high redshift CC data from tracking a non-evolving $H(z)$.} to fix the high redshift behaviour of $H(z)$, whereas our analysis here so far has not introduced a prior. 

% Figure environment removed

Nevertheless, armed with a new PD methodology, we are in a position to revisit the earlier result and see if we can recover the $2 \sigma$ tension with Planck. Since \cite{Colgain:2022rxy} made use of older BAO data, here we replace QSO and Lyman-$\alpha$ BAO with the latest eBOSS results \cite{Hou:2020rse, Neveux:2020voa, duMasdesBourboux:2020pck}. Moreover, we work directly with the $D_{H}/r_d$ constraints and do not invert them. This entails assuming a value for the radius of the sound horizon, which we take to be the Planck value, $r_d = 147.09 \pm 0.26$ Mpc \cite{Planck:2018vyg}. In addition, we reinstate the prior $\Omega_m h^2 = 0.1430 \pm 0.0011$, so that the only difference with \cite{Colgain:2022rxy} is simply to update OHD BAO to the latest constraints. We stress that the priors we introduce are consistent with the Planck cosmology, so \textit{they cannot be driving any disagreement}. Moreover, the $\Omega_m h^2$ prior restricts one to a curve in the $(H_0, \Omega_m)$, but it cannot dictate where one is on the curve, this is done by the remaining 3 CC and 3 BAO data points.  

We again marginalise over the free parameters $(H_0, \Omega_m, r_d)$ with MCMC. In Fig.~\ref{fig:CC_BAO_MCMC} we present the posteriors. While $r_d$ is Gaussian and peaked on our Planck prior, as expected, the $\Omega_m$ posterior is peaked at $\Omega_m \sim 0.6$ and the fact that the fall off in the distribution is gradual beyond the peak leads to a pile up of configurations in the top left corner of the $(H_0, \Omega_m)$-plane. This fall off continues beyond $\Omega_m = 1$ and if the prior is relaxed, the $H_0$ peak shifts to smaller values. So,  once again all the hallmarks of projection effects are present. That being said, given the sharp fall off in the $\Omega_m$ distribution to smaller $\Omega_m$ values, some tension appears to be evident with the Planck values (dashed lines). 

% Figure environment removed

We now run the MCMC chain through our PD methodology. From Fig.~\ref{fig:CC_BAO}, we can see that the $R(H_0)$ and $R(\Omega_m)$ distributions prefer smaller values of $H_0$ and larger values of $\Omega_m$. The peak of the distributions occurs at $H_0 = 42.40$ km/s/Mpc and $\Omega_m = 0.795$.  The lone dot in the $R(H_0)$ distribution at low values of $H_0$ tells us that the distribution falls off sharply below $H_0 = 40$ km/s/Mpc. Note, since we employed generous uniform priors $H_0 \in [0, 200]$, the priors are not impacting the $R(H_0)$ distribution, so it is expected that the distribution falls off to zero on both sides. In contrast, the $R(\Omega_m)$ distribution is one-sided and fails to fall off in the direction of larger values within the uniform priors $\Omega_m \in [0, 1]$. The tension with Planck falls between $2 \sigma$ and $3 \sigma$. By integrating the PDF as far as the black lines corresponding to the Planck values in Fig.~\ref{fig:CC_BAO}, we estimate that the Planck $H_0$ is located at $2.1 \sigma$ from the peak, while the Planck $\Omega_m$ value is $2.5 \sigma$ from the peak.

The main take-away from this section is that OHD data comprising CC and BAO data points beyond $z=1.45$ is inconsistent with the Planck cosmology at in excess of $2 \sigma$. We have employed Planck priors to arrive at this result, but these priors cannot drive the disagreement. Moreover, independent analysis based on least squares fitting and mock simulations presented in \cite{Colgain:2022rxy} also points to a $2 \sigma$ tension, albeit with less up-to-date high redshift BAO data. In summary, different methodologies agree on a $2 \sigma$ discrepancy with Planck, which is robust to interchanging older and newer BAO data. 

\section{Concluding remarks}
\label{sec:discussion}
A $\chi^2$ likelihood is a metric or measure of how well a model fits data. The point in model parameter space that fits the data the best possesses the lowest $\chi^2$. Once one has identified this point, the problem remains to establish $1 \sigma$, $2 \sigma$, etc, confidence intervals in parameter space. In cosmology and astrophysics, MCMC is the prevailing technique for estimating confidence intervals. Its great advantage is that it allows one to i) globally sample the parameter space and ii) arrive at posteriors that serve as an estimate of the errors even with non-Gaussian distributions. In contrast, if one minimises the $\chi^2$ by gradient descent, there is always a risk that one ends up in a local minimum, i. e. the global minimum is missed, while error estimation through Fisher matrix assumes any distribution is Gaussian. The appeal of MCMC marginalisation is that it is widely applicable. However, the point of this paper is that limitations exist, even in the simplest model. 

Indeed, what happens when the MCMC posterior no longer tracks points in parameter space that fit the data better? Traditionally, volume effects are seen as the preserve of higher-dimensional models, e. g. \cite{Herold:2021ksg, Gomez-Valent:2022hkb, Meiers:2023gft}, but projection effects also occur in the minimal $\Lambda$CDM model when one fits the model to data binned by redshift in the late Universe \cite{Colgain:2022tql}. As explained in \cite{Colgain:2022tql}, this ``projection effect'' is driven by OHD, $H(z_i)$, and angular diameter or luminosity distance data, $D_{A}(z_i)$ or $D_{L}(z_i)$, {respectively} only constraining the combinations $\Omega_m h^2$ and $ (1-\Omega_m) h^2$ well, with high redshift data $z_i \gg 0$. In practice, this restricts MCMC configurations to constant $\Omega_m h^2$ and constant $(1-\Omega_m) h^2$ curves in the $(H_0, \Omega_m)$ plane, and as the curves stretch due to DE or matter being less well constrained in high redshift bins, projection effects lead to shifts in the peaks of MCMC posteriors and the emergence of non-Gaussian tails \cite{Colgain:2022tql}. We stress that one sees the same effect in PDFs of best fit $(H_0, \Omega_m)$ parameters in a large number of mock data realisations \cite{Colgain:2022tql}, so the problem is more general than MCMC; there is an inherent bias in the $\Lambda$CDM model when one fits it to redshift binned $H(z)$ \textit{or} $D_{A}(z)$ \textit{or} $D_{L}(z)$ data. Within MCMC, one sees this effect in the errors, but also in the drift of the parameters corresponding to the $\chi^2$ minimum outside of the $1 \sigma$ confidence intervals. Highlighting this (expected) bias in MCMC using OHD is the opening salvo (result) of this paper.     

Why should one care? This is evidently only a problem if one bins data and confronts the $\Lambda$CDM model. First, note that some data sets are inherently binned. For example, effective redshifts are assigned to CC and BAO analysed in a given redshift bin, while each strongly lensed system constitutes its own bin. Working with binned data is unavoidable. Secondly, $\Lambda$CDM tensions point to a problem with the $\Lambda$CDM model once the tensions become widespread and persistent. As explained in \cite{Krishnan:2020vaf}, if the minimal $\Lambda$CDM model is too simple, one expects redshift evolution of $\Lambda$CDM cosmological parameters as it is confronted to redshift binned data. Hints of these trends are now evident in $H_0$ \cite{Wong:2019kwg, Millon:2019slk, Dainotti:2021pqg, Colgain:2022nlb, Colgain:2022rxy, Malekjani:2023dky, Hu:2022kes, Jia:2022ycc, Krishnan:2020obg, Dainotti:2022bzg}, $\Omega_m$ \cite{Risaliti:2015zla, Risaliti:2018reu, Lusso:2020pdb, Yang:2019vgk, Khadka:2020vlh, Khadka:2020tlm, Khadka:2021xcc, Pourojaghi:2022zrh, Colgain:2022nlb, Colgain:2022rxy, Malekjani:2023dky, Pasten:2023rpc, Sakr:2023hrl} and $S_8$/$\sigma_8$ \cite{Esposito:2022plo, Adil:2023jtu, ACT:2023dou, ACT:2023kun} (also \cite{Miyatake:2021qjr, Alonso:2023guh}) across a host of different observables. This evolution is an expected hallmark of model breakdown, which must happen at some redshift if systematics are not universally at play. 

The main problem with redshift dependent $\Lambda$CDM cosmological parameters\footnote{There is a separate interpretation problem as the cosmology literature works with  parameters ``defined today''. In more mathematical language, this is simply the statement that one solves an ordinary differential equation (ODE), namely the Friedmann equation or equivalent, by specifying an integration constant, e.g. $H_0 = H(z=0)$ or $\rho_m(z=0)=\rho_{m0}=H_0^2\Omega_{m}$. However, this is a mathematical statement and it still needs to be confirmed observationally that $H_0$ or $\rho_{m0}$ are \textit{bona fide} constants. This cannot be \textit{a priori} assumed, because it is mathematical prediction of the model. If the model is correct, a constant $H_0$ and $\Omega_m$  will be supported by the data. See \cite{Krishnan:2020vaf} for further discussion.} is one needs to assign a statistical significance to any trend. At a purely practical level, this entails constructing bins centered on different redshifts and identifying discrepancies in $\Lambda$CDM parameters between bins, \textit{ideally in the same observable}, so that the potential systematics are under greatest control. As demonstrated both mathematically and observationally with the CC data in section \ref{sec:MCMC_bias}, MCMC marginalisation leads to biased inferences when one bins the data. In this paper we have resorted to profile distributions \cite{Gomez-Valent:2022hkb} to overcome this bias and have applied the technique to a setting where $\Lambda$CDM distributions are expected to be non-Gaussian for the reasons outlined above and in section \ref{sec:MCMC_bias}. This new technique, provides a complementary perspective that confirms the least square fits of observed and mock data presented in \cite{Colgain:2022nlb, Colgain:2022rxy, Malekjani:2023dky}, where evidence for redshift evolution in $H_0$ and $\Omega_m$ was presented. Regardless of the methodology, the objective is to drill down on the prevailing \textit{assumption} that cosmological parameters are constants. \textit{In the era of tensions in cosmology, nothing can be assumed, especially noting that the tensions are in essence showing an example of evolution of these parameters with redshift.}

More concretely, in this paper with both mock simulations and profile distributions we have shown that high redshift CC data has a preference for a non-evolving $H(z)$ over Planck-$\Lambda$CDM at approximately $\sim 2 \sigma$. This trend, which constitutes the second result of the paper, is unquestionable, as it is visible in the data. Note, we have not propagated systematic uncertainties, so the significance will be less when these are properly propagate. Nevertheless, low and high redshift CC data currently have a preference for different $\Lambda$CDM cosmological parameters. This is important because if the CC program is claiming an 8\% constraint on the Hubble constant, $H_0 = 66.7 \pm 5.5$ km/s/Mpc \cite{Moresco:2023zys}, it is imperative that \textit{all subsets of the data are consistent with this result}. If they are not, then we are staring at either systematics or model breakdown. Admittedly, demanding self-consistency of subsets of a data set confronted to a model is a high bar, but it is important that data sets result in overlapping constraints on $\Lambda$CDM parameters, otherwise this makes cosmological inferences moot. Note, the $\Lambda$CDM model is largely only well tested in the DE dominated regime $z \lesssim 1$ and at very high redshifts $z \sim 1100$, which leaves a wide expanse of redshifts to be explored in order to confirm or refute the model. Given the existing $\Lambda$CDM tensions \cite{Perivolaropoulos:2021jda, Abdalla:2022yfr}, and the hints of evolution in $H_0$, $\Omega_m$ and $S_8$ across assorted probes in the late Universe $z \lesssim 5$, it would be surprising if all discrepancies could be explained away by systematics.\footnote{We are open to the possibility, we just consider it a bad bet at the moment. The odds can of course change as observations improve.}

As an aside, it is intriguing that CC data has a preference for larger best fit values of $H_0$ and smaller best fit values of $\Omega_m$ beyond $z_{\textrm{min}} = 0.7$, as this is traditionally the transition redshift between decelerated and accelerated expansion. % where $\ddot{a} = 0$. 
Moreover, at higher redshifts $z \sim 2.3$, there is not only a longstanding anomaly in Lyman-$\alpha$ BAO \cite{duMasdesBourboux:2020pck}, but QSOs also show a preference for a lower luminosity distance, $D_{L}(z)$, relative to Planck-$\Lambda$CDM \cite{Risaliti:2015zla, Risaliti:2018reu}. Translated into $\Lambda$CDM parameters, this corresponds to conversely larger $\Omega_m$ values, e. g.  \cite{Yang:2019vgk, Khadka:2020vlh, Khadka:2020tlm, Khadka:2021xcc, Pourojaghi:2022zrh}, and consequently smaller $H_0$ values. Thus, the emerging probes CC and QSOs  \cite{Moresco:2022phi} do not appear to be in sync on high redshift $\Lambda$CDM inferences. Nevertheless, neither may be inconsistent with the anomaly in Lyman-$\alpha$ BAO. Relative to Planck-$\Lambda$CDM, Lyman-$\alpha$ BAO prefers \textit{smaller} values of $D_{M}(z) := c \int_{0}^z 1/H(z^{\prime}) \, \textrm{d} z$ and \textit{smaller} values of $H(z)$ (larger values of $D_{H}(z) := c/H(z)$).\footnote{In this statement we assumed the Planck value $r_d \sim 147$ Mpc \cite{Planck:2018vyg} If we reinstate the radius of the sound horizon in these expressions, one recognises that changing the sound horizon, as advocated by early Universe resolutions to Hubble tension, cannot consistently address the Lyman-$\alpha$ BAO anomaly. In general, even for the Planck-$\Lambda$CDM sound horizon, one cannot get both a smaller $D_{M}(z)$ and smaller $H(z)$ from a strictly increasing function, such as the $\Lambda$CDM $H(z)$. As a result, deviations from the Planck-$\Lambda$CDM model that address this anomaly are expected to lead to wiggles in $H(z)$ \cite{Akarsu:2022lhx}, which are unsurprisingly seen in data reconstructions \cite{Zhao:2017cud, Wang:2018fng, Escamilla:2021uoj}. Finally, evolution in $H_0, \Omega_m$ discussed here cannot be explained or accommodated by early resolutions to Hubble tension relying on a change in the $r_d$ at very high $z$.}. If CC data prefer less evolution in $H(z)$ in the matter-dominated regime, then this is consistent with the preference for a smaller $H(z)$ from Lyman-$\alpha$ BAO. Furthermore, QSO data prefers smaller luminosity distances $D_{L}(z)$ relative to Planck, which are consistent with the smaller $D_{M}(z) \propto D_{L}(z)$ values preferred by Lyman-$\alpha$ BAO. Thus, even if CC and QSOs appear to be showing diverging behaviour in the cosmological parameters $(H_0, \Omega_m)$, this may still turn out to be consistent with Lyman-$\alpha$ BAO. We await future DESI \cite{DESI:2023ytc} data releases to ascertain if the non-evolving $H(z)$ trend in high redshift CC data is physical or not. 

Finally, we come to our third and main result outlined in section \ref{sec:tension}. We have revisited a $\sim 2 \sigma$ tension between high redshift CC and BAO data reported in \cite{Colgain:2022rxy}, where the significance was estimated through mock simulations. Here, we have upgraded the BAO data to the latest constraints and again  recover a $>2 \sigma$ discrepancy in $(H_0, \Omega_m)$ with different methodology. This provides a consistency check that there is evolution in OHD between low and high redshifts in the late Universe. Note, this evolution runs contrary to the non-evolving $H(z)$ seen in high redshift CC data because it assumes Planck has accurately constrained the high redshift behaviour of the Hubble parameter in (\ref{eq:lcdm}). Nevertheless, both with and without a Planck prior on $\Omega_m h^2$, evolution at $ \gtrsim 2 \sigma$ is evident in OHD data. It should be stressed that evolution is evident in PDFs of best fit $\Lambda$CDM parameters fitted to a large number of Planck-$\Lambda$CDM mocks \cite{Colgain:2022tql}, so evolution in observed data can be expected. It is imperative to revisit the remaining observations in \cite{Colgain:2022rxy, Malekjani:2023dky} in order to confirm the significance of $\sim 2 \sigma$ hints of evolution found separately in Type Ia SN and QSO data sets. 




\acknowledgments
We would like to thank Adri\`a G\'omez-Valent for discussions and comments on the draft. We thank Gabriela Marques, Mike Hudson and Matteo Viel for related discussions on late Universe evolution in $S_8$. E\'OC thanks Yonsei University and Asia Pacific Center for Theoretical Physics for hospitality. 
This article/publication is based upon work from COST Action CA21136 – “Addressing observational tensions in cosmology with systematics and fundamental physics (CosmoVerse)”, supported by COST (European Cooperation in Science and Technology). SP and MMShJ acknowledge SarAmadan grant No. ISEF/M/401332. MMShJ thanks the support from ICTP associates office (under Senior Associate program) and ICTP HECAP section for hospitality.  


\appendix
\section{Fisher Matrix}
\label{sec:fisher}
Consider the $\chi^2$ (\ref{eq:chi2}). 
Defining $H_{\textrm{model}}(z) = H_0 \sqrt{1-\Omega_m + \Omega_m (1+z)^3}$ and $Q_i$ as in \eqref{eq:Q}, we can now work out the derivatives
\begin{equation}
    \begin{split}
\partial_{H_0} Q_i &= -\sqrt{1-\Omega_m + \Omega_m (1+z_i)^3}, \\  \partial_{\Omega_m} Q_i &= - \frac{1}{2} H_0 (z_i^3 + 3 z_i^2 + 3 z_i)/\sqrt{1-\Omega_m + \Omega_m (1+z_i)^3}, \\
\partial^2_{H_0} Q_i &= 0, \\
\partial_{H_0} \partial_{\Omega_m} Q_i &= - \frac{1}{2} (z_i^3 + 3 z_i^2 + 3 z_i)/\sqrt{1-\Omega_m + \Omega_m (1+z_i)^3}, \\
\partial^2_{\Omega_m} Q_i =& \frac{1}{4} H_0 (z_i^3 + 3 z_i^2 + 3 z_i)^2/(1-\Omega_m + \Omega_m (1+z_i)^3)^{\frac{3}{2}}.      
    \end{split}
\end{equation}
We can then define the Fisher matrix 
\be
F_{ij} = \frac{1}{2} \frac{\partial^2 \chi^2(H_0, \Omega_m)}{\partial p_i \partial p_j}
\ee
where $p_i \in \{ H_0, \Omega_m \}$. Note that the Fisher matrix is evaluated on the best fit parameters. The result is a $2 \times 2$ matrix, which one inverts and the estimated errors are the square root of the diagonal entries. 








\begin{thebibliography}{99}

\bibitem{Planck:2018vyg}
N.~Aghanim \textit{et al.} [Planck],
``Planck 2018 results. VI. Cosmological parameters,''
Astron. Astrophys. \textbf{641} (2020), A6
% doi:10.1051/0004-6361/201833910
%[arXiv:1807.06209 [astro-ph.CO]].

\bibitem{Riess:1998cb}
A.~G.~Riess \textit{et al.} [Supernova Search Team],
``Observational evidence from supernovae for an accelerating universe and a cosmological constant,''
Astron. J. \textbf{116} (1998), 1009-1038
% doi:10.1086/300499
%[arXiv:astro-ph/9805201 [astro-ph]].
%13031 citations counted in INSPIRE as of 02 Feb 2021

\bibitem{Perlmutter:1998np}
S.~Perlmutter \textit{et al.} [Supernova Cosmology Project],
``Measurements of $\Omega$ and $\Lambda$ from 42 high redshift supernovae,''
Astrophys. J. \textbf{517} (1999), 565-586
% doi:10.1086/307221
%[arXiv:astro-ph/9812133 [astro-ph]].
%13057 citations counted in INSPIRE as of 02 Feb 2021

\bibitem{Eisenstein:2005su}
D.~J.~Eisenstein \textit{et al.} [SDSS],
``Detection of the Baryon Acoustic Peak in the Large-Scale Correlation Function of SDSS Luminous Red Galaxies,''
Astrophys. J. \textbf{633} (2005), 560-574
%doi:10.1086/466512
%[arXiv:astro-ph/0501171 [astro-ph]].
%3380 citations counted in INSPIRE as of 08 Oct 2020

\bibitem{Riess:2021jrx}
A.~G.~Riess, W.~Yuan, L.~M.~Macri, D.~Scolnic, D.~Brout, S.~Casertano, D.~O.~Jones, Y.~Murakami, L.~Breuval and T.~G.~Brink, \textit{et al.}
``A Comprehensive Measurement of the Local Value of the Hubble Constant with 1 km s$^{?1}$ Mpc$^{?1}$ Uncertainty from the Hubble Space Telescope and the SH0ES Team,''
Astrophys. J. Lett. \textbf{934} (2022) no.1, L7
%doi:10.3847/2041-8213/ac5c5b
%[arXiv:2112.04510 [astro-ph.CO]].
%370 citations counted in INSPIRE as of 09 Jan 2023

\bibitem{Freedman:2021ahq}
W.~L.~Freedman,
``Measurements of the Hubble Constant: Tensions in Perspective,''
Astrophys. J. \textbf{919} (2021) no.1, 16
%doi:10.3847/1538-4357/ac0e95
%[arXiv:2106.15656 [astro-ph.CO]].
%179 citations counted in INSPIRE as of 09 Jan 2023

\bibitem{Pesce:2020xfe}
D.~W.~Pesce, J.~A.~Braatz, M.~J.~Reid, A.~G.~Riess, D.~Scolnic, J.~J.~Condon, F.~Gao, C.~Henkel, C.~M.~V.~Impellizzeri and C.~Y.~Kuo, \textit{et al.}
%``The Megamaser Cosmology Project. XIII. Combined Hubble constant constraints,''
Astrophys. J. Lett. \textbf{891} (2020) no.1, L1
%doi:10.3847/2041-8213/ab75f0
%[arXiv:2001.09213 [astro-ph.CO]].
%96 citations counted in INSPIRE as of 12 Jul 2021

\bibitem{Blakeslee:2021rqi}
J.~P.~Blakeslee, J.~B.~Jensen, C.~P.~Ma, P.~A.~Milne and J.~E.~Greene,
%``The Hubble Constant from Infrared Surface Brightness Fluctuation Distances,''
Astrophys. J. \textbf{911} (2021) no.1, 65
%doi:10.3847/1538-4357/abe86a
%[arXiv:2101.02221 [astro-ph.CO]].
%11 citations counted in INSPIRE as of 12 Jul 2021

\bibitem{Kourkchi:2020iyz}
E.~Kourkchi, R.~B.~Tully, G.~S.~Anand, H.~M.~Courtois, A.~Dupuy, J.~D.~Neill, L.~Rizzi and M.~Seibert,
%``Cosmicflows-4: The Calibration of Optical and Infrared Tully\textendash{}Fisher Relations,''
Astrophys. J. \textbf{896} (2020) no.1, 3
%doi:10.3847/1538-4357/ab901c
%[arXiv:2004.14499 [astro-ph.GA]].
%15 citations counted in INSPIRE as of 12 Jul 2021

\bibitem{HSC:2018mrq}
C.~Hikage \textit{et al.} [HSC],
``Cosmology from cosmic shear power spectra with Subaru Hyper Suprime-Cam first-year data,''
Publ. Astron. Soc. Jap. \textbf{71}, 43  (2019).
%doi:10.1093/pasj/psz010

\bibitem{KiDS:2020suj}
M.~Asgari \textit{et al.} [KiDS],
``KiDS-1000 Cosmology: Cosmic shear constraints and comparison between two point statistics,''
Astron. Astrophys. \textbf{645} (2021), A104
%doi:10.1051/0004-6361/202039070
%[arXiv:2007.15633 [astro-ph.CO]].
%113 citations counted in INSPIRE as of 18 Aug 2021

\bibitem{DES:2021wwk}
T.~M.~C.~Abbott \textit{et al.} [DES],
``Dark Energy Survey Year 3 results: Cosmological constraints from galaxy clustering and weak lensing,''
Phys. Rev. D \textbf{105} (2022) no.2, 023520
%doi:10.1103/PhysRevD.105.023520
%[arXiv:2105.13549 [astro-ph.CO]].
%519 citations counted in INSPIRE as of 14 Jul 2023

\bibitem{Boruah:2019icj}
S.~S.~Boruah, M.~J.~Hudson and G.~Lavaux,
``Cosmic flows in the nearby Universe: new peculiar velocities from SNe and cosmological constraints,''
Mon. Not. Roy. Astron. Soc. \textbf{498} (2020) no.2, 2703-2718
%doi:10.1093/mnras/staa2485
%[arXiv:1912.09383 [astro-ph.CO]].
%54 citations counted in INSPIRE as of 14 Jul 2023

\bibitem{Said:2020epb}
K.~Said, M.~Colless, C.~Magoulas, J.~R.~Lucey and M.~J.~Hudson,
``Joint analysis of 6dFGS and SDSS peculiar velocities for the growth rate of cosmic structure and tests of gravity,''
Mon. Not. Roy. Astron. Soc. \textbf{497} (2020) no.1, 1275-1293
%doi:10.1093/mnras/staa2032
%[arXiv:2007.04993 [astro-ph.CO]].
%49 citations counted in INSPIRE as of 14 Jul 2023

\bibitem{Perivolaropoulos:2021jda}
L.~Perivolaropoulos and F.~Skara,
``Challenges for \ensuremath{\Lambda}CDM: An update,''
New Astron. Rev. \textbf{95}, 101659  (2022).
%doi:10.1016/j.newar.2022.101659
%\href{https://arxiv.org/abs/2105.05208}{2105.05208}

\bibitem{Abdalla:2022yfr}
E.~Abdalla, G.~Franco Abell\'an, A.~Aboubrahim, A.~Agnello, O.~Akarsu, Y.~Akrami, G.~Alestas, D.~Aloni, L.~Amendola and L.~A.~Anchordoqui, \textit{et al.}
``Cosmology intertwined: A review of the particle physics, astrophysics, and cosmology associated with the cosmological tensions and anomalies,''
JHEAp \textbf{34}, 49  (2022).
%doi:10.1016/j.jheap.2022.04.002
%\href{https://arxiv.org/abs/2203.06142}{2203.06142}

\bibitem{Phillips:1993ng}
M.~M.~Phillips,
``The absolute magnitudes of Type IA supernovae,''
Astrophys. J. Lett. \textbf{413} (1993), L105-L108
%doi:10.1086/186970
%1245 citations counted in INSPIRE as of 24 Aug 2021

\bibitem{NearbySupernovaFactory:2018qkd}
M.~Rigault \textit{et al.} [Nearby Supernova Factory],
``Strong Dependence of Type Ia Supernova Standardization on the Local Specific Star Formation Rate,''
Astron. Astrophys. \textbf{644} (2020), A176
%doi:10.1051/0004-6361/201730404
%[arXiv:1806.03849 [astro-ph.CO]].
%143 citations counted in INSPIRE as of 20 Jul 2023

\bibitem{Kang:2019azh}
Y.~Kang, Y.~W.~Lee, Y.~L.~Kim, C.~Chung and C.~H.~Ree,
``Early-type Host Galaxies of Type Ia Supernovae. II. Evidence for Luminosity Evolution in Supernova Cosmology,''
Astrophys. J. \textbf{889} (2020) no.1, 8
%doi:10.3847/1538-4357/ab5afc
%[arXiv:1912.04903 [astro-ph.GA]].
%56 citations counted in INSPIRE as of 20 Jul 2023

\bibitem{Brout:2020msh}
D.~Brout and D.~Scolnic,
``It\textquoteright{}s Dust: Solving the Mysteries of the Intrinsic Scatter and Host-galaxy Dependence of Standardized Type Ia Supernova Brightnesses,''
Astrophys. J. \textbf{909} (2021) no.1, 26
%doi:10.3847/1538-4357/abd69b
%[arXiv:2004.10206 [astro-ph.CO]].
%82 citations counted in INSPIRE as of 20 Jul 2023

\bibitem{Lee:2021txi}
Y.~W.~Lee, C.~Chung, P.~Demarque, S.~Park, J.~Son and Y.~Kang,
``Evidence for strong progenitor age dependence of type Ia supernova luminosity standardization process,''
Mon. Not. Roy. Astron. Soc. \textbf{517} (2022) no.2, 2697-2708
%doi:10.1093/mnras/stac2840
%[arXiv:2107.06288 [astro-ph.GA]].
%5 citations counted in INSPIRE as of 20 Jul 2023


\bibitem{Krishnan:2020vaf}
C.~Krishnan, E.~\'O~Colg\'ain, M.~M.~Sheikh-Jabbari and T.~Yang,
``Running Hubble Tension and a H0 Diagnostic,''
Phys. Rev. D \textbf{103} (2021) no.10, 103509
%doi:10.1103/PhysRevD.103.103509
%[arXiv:2011.02858 [astro-ph.CO]].
%65 citations counted in INSPIRE as of 14 Jul 2023 

\bibitem{Krishnan:2022fzz}
C.~Krishnan and R.~Mondol,
``$H_0$ as a Universal FLRW Diagnostic,''
[arXiv:2201.13384 [astro-ph.CO]].
%12 citations counted in INSPIRE as of 14 Jul 2023

%\bibitem{Liao:2020zko}
%K.~Liao, A.~Shafieloo, R.~E.~Keeley and E.~V.~Linder,
%``Determining Model-independent H 0 and Consistency Tests,''
%Astrophys. J. Lett. \textbf{895} (2020) no.2, L29
%doi:10.3847/2041-8213/ab8dbb
%[arXiv:2002.10605 [astro-ph.CO]].
%51 citations counted in INSPIRE as of 14 Jul 2023

%\bibitem{Montani:2023xpd}
%G.~Montani, M.~De Angelis, F.~Bombacigno and N.~Carlevaro,
%``Metric $f(R)$ gravity with dynamical dark energy as a paradigm for the Hubble Tension,''
%[arXiv:2306.11101 [gr-qc]].
%1 citations counted in INSPIRE as of 14 Jul 2023

\bibitem{Wong:2019kwg}
K.~C.~Wong, S.~H.~Suyu, G.~C.~F.~Chen, C.~E.~Rusu, M.~Millon, D.~Sluse, V.~Bonvin, C.~D.~Fassnacht, S.~Taubenberger and M.~W.~Auger, \textit{et al.}
``H0LiCOW \textendash{} XIII. A 2.4 per cent measurement of H0 from lensed quasars: 5.3\ensuremath{\sigma} tension between early- and late-Universe probes,''
Mon. Not. Roy. Astron. Soc. \textbf{498} (2020) no.1, 1420-1439
%doi:10.1093/mnras/stz3094
%[arXiv:1907.04869 [astro-ph.CO]].
%804 citations counted in INSPIRE as of 18 May 2023

\bibitem{Millon:2019slk}
M.~Millon, A.~Galan, F.~Courbin, T.~Treu, S.~H.~Suyu, X.~Ding, S.~Birrer, G.~C.~F.~Chen, A.~J.~Shajib and D.~Sluse, \textit{et al.}
``TDCOSMO. I. An exploration of systematic uncertainties in the inference of $H_0$ from time-delay cosmography,''
Astron. Astrophys. \textbf{639} (2020), A101
%doi:10.1051/0004-6361/201937351
%[arXiv:1912.08027 [astro-ph.CO]].
%114 citations counted in INSPIRE as of 18 May 2023

\bibitem{Sluse:2003iy}
D.~Sluse, J.~Surdej, J.~F.~Claeskens, D.~Hutsemekers, C.~Jean, F.~Courbin, T.~Nakos, M.~Billeres and S.~V.~Khmil,
``A Quadruply imaged quasar with an optical Einstein ring candidate: 1RXS J113155.4-123155,''
Astron. Astrophys. \textbf{406} (2003), L43-L46
%doi:10.1051/0004-6361:20030904
%[arXiv:astro-ph/0307345 [astro-ph]].
%83 citations counted in INSPIRE as of 14 Jul 2023

\bibitem{Shajib:2023uig}
A.~J.~Shajib, P.~Mozumdar, G.~C.~F.~Chen, T.~Treu, M.~Cappellari, S.~Knabel, S.~H.~Suyu, V.~N.~Bennert, J.~A.~Frieman and D.~Sluse, \textit{et al.}
``TDCOSMO. XIII. Improved Hubble constant measurement from lensing time delays using spatially resolved stellar kinematics of the lens galaxy,''
Astron. Astrophys. \textbf{673} (2023), A9
%doi:10.1051/0004-6361/202345878
%[arXiv:2301.02656 [astro-ph.CO]].
%3 citations counted in INSPIRE as of 18 May 2023

\bibitem{Dainotti:2021pqg}
M.~G.~Dainotti, B.~De Simone, T.~Schiavone, G.~Montani, E.~Rinaldi and G.~Lambiase,
``On the Hubble constant tension in the SNe Ia Pantheon sample,''
Astrophys. J. \textbf{912}, 150  (2021).
%doi:10.3847/1538-4357/abeb73


\bibitem{Colgain:2022nlb}
E.~\'O~Colg\'ain, M.~M.~Sheikh-Jabbari, R.~Solomon, G.~Bargiacchi, S.~Capozziello, M.~G.~Dainotti and D.~Stojkovic,
``Revealing intrinsic flat \ensuremath{\Lambda}CDM biases with standardizable candles,''
Phys. Rev. D \textbf{106}, L041301  (2022).
%doi:10.1103/PhysRevD.106.L041301

\bibitem{Colgain:2022rxy}
E.~\'O~Colg\'ain, M.~M.~Sheikh-Jabbari, R.~Solomon, M.~G.~Dainotti and D.~Stojkovic,
``Putting Flat $\Lambda$CDM In The (Redshift) Bin,''
[arXiv:2206.11447 [astro-ph.CO]].
%42 citations counted in INSPIRE as of 14 Jul 2023

%\cite{Colgain:2022tql}
%\bibitem{Colgain:2022tql}
%E.~\'O.~Colg\'ain, M.~M.~Sheikh-Jabbari and R.~Solomon,
%``High redshift \ensuremath{\Lambda}CDM cosmology: To bin or not to bin?,''
%Phys. Dark Univ. \textbf{40} (2023), 101216
%doi:10.1016/j.dark.2023.101216
%[arXiv:2211.02129 [astro-ph.CO]].
%10 citations counted in INSPIRE as of 25 Jul 2023


\bibitem{Malekjani:2023dky}
M.~Malekjani, R.~M.~Conville, E.~\'O.~Colg\'ain, S.~Pourojaghi and M.~M.~Sheikh-Jabbari,
``Negative Dark Energy Density from High Redshift Pantheon+ Supernovae,''
[arXiv:2301.12725 [astro-ph.CO]].
%13 citations counted in INSPIRE as of 17 Jul 2023

\bibitem{Hu:2022kes}
J.~P.~Hu and F.~Y.~Wang,
``Revealing the late-time transition of H0: relieve the Hubble crisis,''
Mon. Not. Roy. Astron. Soc. \textbf{517}, 576  (2022).

\bibitem{Jia:2022ycc}
X.~D.~Jia, J.~P.~Hu and F.~Y.~Wang,
``Evidence of a decreasing trend for the Hubble constant,''
Astron. Astrophys. \textbf{674} (2023), A45
%doi:10.1051/0004-6361/202346356
%[arXiv:2212.00238 [astro-ph.CO]].
%10 citations counted in INSPIRE as of 17 Jul 2023

\bibitem{Krishnan:2020obg}
C.~Krishnan, E.~\'O~Colg\'ain, Ruchika, A.~A.~Sen, M.~M.~Sheikh-Jabbari and T.~Yang,
``Is there an early Universe solution to Hubble tension?,''
Phys. Rev. D \textbf{102} (2020) no.10, 103525
%doi:10.1103/PhysRevD.102.103525
%[arXiv:2002.06044 [astro-ph.CO]].
%69 citations counted in INSPIRE as of 17 Jul 2023

\bibitem{Dainotti:2022bzg}
M.~G.~Dainotti, B.~De Simone, T.~Schiavone, G.~Montani, E.~Rinaldi, G.~Lambiase, M.~Bogdan and S.~Ugale,
``On the Evolution of the Hubble Constant with the SNe Ia Pantheon Sample and Baryon Acoustic Oscillations: A Feasibility Study for GRB-Cosmology in 2030,''
Galaxies \textbf{10}, 24  (2022).
%doi:10.3390/galaxies10010024

\bibitem{Risaliti:2015zla}
G.~Risaliti and E.~Lusso,
``A Hubble Diagram for Quasars,''
Astrophys. J. \textbf{815} (2015), 33
%doi:10.1088/0004-637X/815/1/33
%[arXiv:1505.07118 [astro-ph.CO]].
%146 citations counted in INSPIRE as of 16 Jun 2023

\bibitem{Risaliti:2018reu}
G.~Risaliti and E.~Lusso,
``Cosmological constraints from the Hubble diagram of quasars at high redshifts,''
Nature Astron. \textbf{3}, 272  (2019).

\bibitem{Lusso:2020pdb}
E.~Lusso, G.~Risaliti, E.~Nardini, G.~Bargiacchi, M.~Benetti, S.~Bisogni, S.~Capozziello, F.~Civano, L.~Eggleston and M.~Elvis, \textit{et al.}
``Quasars as standard candles III. Validation of a new sample for cosmological studies,''
Astron. Astrophys. \textbf{642}, A150  (2020).


\bibitem{Yang:2019vgk}
T.~Yang, A.~Banerjee and E.~\'O~Colg\'ain,
``Cosmography and flat $\Lambda$CDM tensions at high redshift,''
Phys. Rev. D \textbf{102}, 123532  (2020).

\bibitem{Khadka:2020vlh}
N.~Khadka and B.~Ratra,
``Using quasar X-ray and UV flux measurements to constrain cosmological model parameters,''
Mon. Not. Roy. Astron. Soc. \textbf{497}, 263  (2020).


\bibitem{Khadka:2020tlm}
N.~Khadka and B.~Ratra,
``Determining the range of validity of quasar X-ray and UV flux measurements for constraining cosmological model parameters,''
Mon. Not. Roy. Astron. Soc. \textbf{502}, 6140  (2021).


\bibitem{Khadka:2021xcc}
N.~Khadka and B.~Ratra,
``Do quasar X-ray and UV flux measurements provide a useful test of cosmological models?,''
Mon. Not. Roy. Astron. Soc. \textbf{510}, 2753  (2022).
%doi:10.1093/mnras/stab3678

\bibitem{Pourojaghi:2022zrh}
S.~Pourojaghi, N.~F.~Zabihi and M.~Malekjani,
``Can high-redshift Hubble diagrams rule out the standard model of cosmology in the context of cosmography?,''
Phys. Rev. D \textbf{106}, 123523  (2022).


\bibitem{Zajacek:2023qjm}
M.~Zaja\v{c}ek, B.~Czerny, N.~Khadka, R.~Prince, S.~Panda, M.~L.~Mart\'\i{}nez-Aldama and B.~Ratra,
``Extinction biases quasar luminosity distances determined from quasar UV and X-ray flux measurements,''
[arXiv:2305.08179 [astro-ph.GA]].
%0 citations counted in INSPIRE as of 17 Jul 2023

\bibitem{Pasten:2023rpc}
E.~Past\'en and V.~H.~C\'ardenas,
``Testing \ensuremath{\Lambda}CDM cosmology in a binned universe: Anomalies in the deceleration parameter,''
Phys. Dark Univ. \textbf{40} (2023), 101224
%doi:10.1016/j.dark.2023.101224
%[arXiv:2301.10740 [astro-ph.CO]].

\bibitem{Wagner:2022etu}
J.~Wagner,
``Casting the $H_0$ tension as a fitting problem of cosmologies,''
[arXiv:2203.11219 [astro-ph.CO]].
%5 citations counted in INSPIRE as of 28 Jul 2023

\bibitem{Sakr:2023hrl}
Z.~Sakr,
``One matter density discrepancy to alleviate them all or further trouble for $\Lambda$CDM model,''
[arXiv:2305.02846 [astro-ph.CO]].
%0 citations counted in INSPIRE as of 24 Jul 2023


\bibitem{Colgain:2022tql}
E.~\'O~Colg\'ain, M.~M.~Sheikh-Jabbari and R.~Solomon,
``High redshift \ensuremath{\Lambda}CDM cosmology: To bin or not to bin?,''
Phys. Dark Univ. \textbf{40} (2023), 101216
%doi:10.1016/j.dark.2023.101216
[arXiv:2211.02129 [astro-ph.CO]].
%10 citations counted in INSPIRE as of 28 Jun 2023

\bibitem{Esposito:2022plo}
M.~Esposito, V.~Ir\v{s}i\v{c}, M.~Costanzi, S.~Borgani, A.~Saro and M.~Viel,
``Weighing cosmic structures with clusters of galaxies and the intergalactic medium,''
Mon. Not. Roy. Astron. Soc. \textbf{515}, 857  (2022).
%doi:10.1093/mnras/stac1825
[arXiv:2202.00974 [astro-ph.CO]].

\bibitem{Adil:2023jtu}
S.~A.~Adil, \"O.~Akarsu, M.~Malekjani, E.~\'O~Colg\'ain, S.~Pourojaghi, A.~A.~Sen and M.~M.~Sheikh-Jabbari,
``$S_8$ increases with effective redshift in $\Lambda$CDM cosmology,''
[arXiv:2303.06928 [astro-ph.CO]].
%1 citations counted in INSPIRE as of 14 Jul 2023

\bibitem{ACT:2023dou}
F.~J.~Qu \textit{et al.} [ACT],
``The Atacama Cosmology Telescope: A Measurement of the DR6 CMB Lensing Power Spectrum and its Implications for Structure Growth,''
[arXiv:2304.05202 [astro-ph.CO]].
%10 citations counted in INSPIRE as of 14 Jul 2023

\bibitem{ACT:2023kun}
M.~S.~Madhavacheril \textit{et al.} [ACT],
``The Atacama Cosmology Telescope: DR6 Gravitational Lensing Map and Cosmological Parameters,''
[arXiv:2304.05203 [astro-ph.CO]].
%10 citations counted in INSPIRE as of 14 Jul 2023

\bibitem{ACT:2023ipp}
G.~A.~Marques \textit{et al.} [ACT and DES],
``Cosmological constraints from the tomography of DES-Y3 galaxies with CMB lensing from ACT DR4,''
[arXiv:2306.17268 [astro-ph.CO]].
%0 citations counted in INSPIRE as of 14 Jul 2023

\bibitem{Miyatake:2021qjr}
H.~Miyatake, Y.~Harikane, M.~Ouchi, Y.~Ono, N.~Yamamoto, A.~J.~Nishizawa, N.~Bahcall, S.~Miyazaki and A.~A.~Plazas Malag\'on,
``First Identification of a CMB Lensing Signal Produced by 1.5~Million Galaxies at z\ensuremath{\sim}4: Constraints on Matter Density Fluctuations at High Redshift,''
Phys. Rev. Lett. \textbf{129} (2022) no.6, 061301
%doi:10.1103/PhysRevLett.129.061301
[arXiv:2103.15862 [astro-ph.CO]].
%7 citations counted in INSPIRE as of 25 Jul 2023

\bibitem{Alonso:2023guh}
D.~Alonso, G.~Fabbian, K.~Storey-Fisher, A.~C.~Eilers, C.~Garc\'\i{}a-Garc\'\i{}a, D.~W.~Hogg and H.~W.~Rix,
``Constraining cosmology with the Gaia-unWISE Quasar Catalog and CMB lensing: structure growth,''
[arXiv:2306.17748 [astro-ph.CO]].
%0 citations counted in INSPIRE as of 25 Jul 2023


\bibitem{Herold:2021ksg}
L.~Herold, E.~G.~M.~Ferreira and E.~Komatsu,
``New Constraint on Early Dark Energy from Planck and BOSS Data Using the Profile Likelihood,''
Astrophys. J. Lett. \textbf{929} (2022) no.1, L16
%doi:10.3847/2041-8213/ac63a3
%[arXiv:2112.12140 [astro-ph.CO]].
%43 citations counted in INSPIRE as of 17 Jul 2023

\bibitem{Gomez-Valent:2022hkb}
A.~G\'omez-Valent,
``Fast test to assess the impact of marginalization in Monte~Carlo analyses and its application to cosmology,''
Phys. Rev. D \textbf{106} (2022) no.6, 063506
%doi:10.1103/PhysRevD.106.063506
%[arXiv:2203.16285 [astro-ph.CO]].
%20 citations counted in INSPIRE as of 11 Jul 2023

\bibitem{Meiers:2023gft}
M.~Meiers, L.~Knox and N.~Sch\"oneberg,
``Exploration of the Pre-recombination Universe with a High-Dimensional Model of an Additional Dark Fluid,''
[arXiv:2307.09522 [astro-ph.CO]].
%0 citations counted in INSPIRE as of 22 Jul 2023

\bibitem{Poulin:2018cxd}
V.~Poulin, T.~L.~Smith, T.~Karwal and M.~Kamionkowski,
``Early Dark Energy Can Resolve The Hubble Tension,''
Phys. Rev. Lett. \textbf{122} (2019) no.22, 221301
%doi:10.1103/PhysRevLett.122.221301
%[arXiv:1811.04083 [astro-ph.CO]].
%608 citations counted in INSPIRE as of 17 Jul 2023

\bibitem{Niedermann:2019olb}
F.~Niedermann and M.~S.~Sloth,
``New early dark energy,''
Phys. Rev. D \textbf{103} (2021) no.4, L041303
%doi:10.1103/PhysRevD.103.L041303
[arXiv:1910.10739 [astro-ph.CO]].
%140 citations counted in INSPIRE as of 24 Jul 2023

\bibitem{Jimenez:2001gg}
R.~Jimenez and A.~Loeb,
``Constraining cosmological parameters based on relative galaxy ages,''
Astrophys. J. \textbf{573} (2002), 37-42
%doi:10.1086/340549
%[arXiv:astro-ph/0106145 [astro-ph]].
%598 citations counted in INSPIRE as of 28 Jun 2023

\bibitem{Stern:2009ep}
D.~Stern, R.~Jimenez, L.~Verde, M.~Kamionkowski and S.~A.~Stanford,
``Cosmic Chronometers: Constraining the Equation of State of Dark Energy. I: H(z) Measurements,''
JCAP \textbf{02} (2010), 008
%doi:10.1088/1475-7516/2010/02/008
%[arXiv:0907.3149 [astro-ph.CO]].
%740 citations counted in INSPIRE as of 20 May 2022

\bibitem{Moresco:2012jh}
M.~Moresco, A.~Cimatti, R.~Jimenez, L.~Pozzetti, G.~Zamorani, M.~Bolzonella, J.~Dunlop, F.~Lamareille, M.~Mignoli and H.~Pearce, \textit{et al.}
``Improved constraints on the expansion rate of the Universe up to z\textasciitilde{}1.1 from the spectroscopic evolution of cosmic chronometers,''
JCAP \textbf{08} (2012), 006
%doi:10.1088/1475-7516/2012/08/006
%[arXiv:1201.3609 [astro-ph.CO]].
%508 citations counted in INSPIRE as of 20 May 2022

\bibitem{Zhang:2012mp}
C.~Zhang, H.~Zhang, S.~Yuan, T.~J.~Zhang and Y.~C.~Sun,
``Four new observational $H(z)$ data from luminous red galaxies in the Sloan Digital Sky Survey data release seven,''
Res. Astron. Astrophys. \textbf{14} (2014) no.10, 1221-1233
%doi:10.1088/1674-4527/14/10/002
%[arXiv:1207.4541 [astro-ph.CO]].
%425 citations counted in INSPIRE as of 20 May 2022

\bibitem{Moresco:2016mzx}
M.~Moresco, L.~Pozzetti, A.~Cimatti, R.~Jimenez, C.~Maraston, L.~Verde, D.~Thomas, A.~Citro, R.~Tojeiro and D.~Wilkinson,
``A 6\% measurement of the Hubble parameter at $z\sim0.45$: direct evidence of the epoch of cosmic re-acceleration,''
JCAP \textbf{05} (2016), 014
%doi:10.1088/1475-7516/2016/05/014
%[arXiv:1601.01701 [astro-ph.CO]].
%505 citations counted in INSPIRE as of 17 May 2022

\bibitem{Ratsimbazafy:2017vga}
A.~L.~Ratsimbazafy, S.~I.~Loubser, S.~M.~Crawford, C.~M.~Cress, B.~A.~Bassett, R.~C.~Nichol and P.~V\"ais\"anen,
``Age-dating Luminous Red Galaxies observed with the Southern African Large Telescope,''
Mon. Not. Roy. Astron. Soc. \textbf{467} (2017) no.3, 3239-3254
%doi:10.1093/mnras/stx301
%[arXiv:1702.00418 [astro-ph.CO]].
%162 citations counted in INSPIRE as of 17 May 2022

\bibitem{Borghi:2021rft}
N.~Borghi, M.~Moresco and A.~Cimatti,
``Toward a Better Understanding of Cosmic Chronometers: A New Measurement of H(z) at z \ensuremath{\sim} 0.7,''
Astrophys. J. Lett. \textbf{928} (2022) no.1, L4
%doi:10.3847/2041-8213/ac3fb2
%[arXiv:2110.04304 [astro-ph.CO]].
%10 citations counted in INSPIRE as of 17 May 2022

\bibitem{Jiao:2022aep}
K.~Jiao, N.~Borghi, M.~Moresco and T.~J.~Zhang,
``New Observational H(z) Data from Full-spectrum Fitting of Cosmic Chronometers in the LEGA-C Survey,''
Astrophys. J. Suppl. \textbf{265} (2023) no.2, 48
%doi:10.3847/1538-4365/acbc77
%[arXiv:2205.05701 [astro-ph.CO]].
%14 citations counted in INSPIRE as of 17 Jul 2023

\bibitem{Tomasetti:2023kek}
E.~Tomasetti, M.~Moresco, N.~Borghi, K.~Jiao, A.~Cimatti, L.~Pozzetti, A.~C.~Carnall, R.~J.~McLure and L.~Pentericci,
``A new measurement of the expansion history of the Universe at z=1.26 with cosmic chronometers in VANDELS,''
[arXiv:2305.16387 [astro-ph.CO]].
%1 citations counted in INSPIRE as of 28 Jun 2023

\bibitem{Moresco:2023zys}
M.~Moresco,
``Addressing the Hubble tension with cosmic chronometers,''
[arXiv:2307.09501 [astro-ph.CO]].
%0 citations counted in INSPIRE as of 24 Jul 2023

\bibitem{Moresco:2020fbm}
M.~Moresco, R.~Jimenez, L.~Verde, A.~Cimatti and L.~Pozzetti,
``Setting the Stage for Cosmic Chronometers. II. Impact of Stellar Population Synthesis Models Systematics and Full Covariance Matrix,''
Astrophys. J. \textbf{898} (2020) no.1, 82
%doi:10.3847/1538-4357/ab9eb0
[arXiv:2003.07362 [astro-ph.GA]].
%57 citations counted in INSPIRE as of 28 Jul 2023

\bibitem{Foreman-Mackey:2012any}
D.~Foreman-Mackey, D.~W.~Hogg, D.~Lang and J.~Goodman,
``emcee: The MCMC Hammer,''
Publ. Astron. Soc. Pac. \textbf{125} (2013), 306-312
%doi:10.1086/670067
%[arXiv:1202.3665 [astro-ph.IM]].
%3393 citations counted in INSPIRE as of 17 Jul 2023


\bibitem{Hou:2020rse}
J.~Hou, A.~G.~S\'anchez, A.~J.~Ross, A.~Smith, R.~Neveux, J.~Bautista, E.~Burtin, C.~Zhao, R.~Scoccimarro and K.~S.~Dawson, \textit{et al.}
``The Completed SDSS-IV extended Baryon Oscillation Spectroscopic Survey: BAO and RSD measurements from anisotropic clustering analysis of the Quasar Sample in configuration space between redshift 0.8 and 2.2,''
Mon. Not. Roy. Astron. Soc. \textbf{500} (2020) no.1, 1201-1221
%:10.1093/mnras/staa3234
%[arXiv:2007.08998 [astro-ph.CO]].
%135 citations counted in INSPIRE as of 28 Jun 2023

\bibitem{Neveux:2020voa}
R.~Neveux, E.~Burtin, A.~de Mattia, A.~Smith, A.~J.~Ross, J.~Hou, J.~Bautista, J.~Brinkmann, C.~H.~Chuang and K.~S.~Dawson, \textit{et al.}
``The completed SDSS-IV extended Baryon Oscillation Spectroscopic Survey: BAO and RSD measurements from the anisotropic power spectrum of the quasar sample between redshift 0.8 and 2.2,''
Mon. Not. Roy. Astron. Soc. \textbf{499} (2020) no.1, 210-229
%doi:10.1093/mnras/staa2780
%[arXiv:2007.08999 [astro-ph.CO]].
%133 citations counted in INSPIRE as of 28 Jun 2023

\bibitem{duMasdesBourboux:2020pck}
H.~du Mas des Bourboux, J.~Rich, A.~Font-Ribera, V.~de Sainte Agathe, J.~Farr, T.~Etourneau, J.~M.~Le Goff, A.~Cuceu, C.~Balland and J.~E.~Bautista, \textit{et al.}
``The Completed SDSS-IV Extended Baryon Oscillation Spectroscopic Survey: Baryon Acoustic Oscillations with Ly\ensuremath{\alpha} Forests,''
Astrophys. J. \textbf{901} (2020) no.2, 153
%doi:10.3847/1538-4357/abb085
%[arXiv:2007.08995 [astro-ph.CO]].
%172 citations counted in INSPIRE as of 28 Jun 2023

\bibitem{Trotta:2017wnx}
R.~Trotta,
``Bayesian Methods in Cosmology,''
[arXiv:1701.01467 [astro-ph.CO]].
%96 citations counted in INSPIRE as of 18 Jul 2023

\bibitem{Moresco:2022phi}
M.~Moresco, L.~Amati, L.~Amendola, S.~Birrer, J.~P.~Blakeslee, M.~Cantiello, A.~Cimatti, J.~Darling, M.~Della Valle and M.~Fishbach, \textit{et al.}
``Unveiling the Universe with emerging cosmological probes,''
Living Rev. Rel. \textbf{25} (2022) no.1, 6
%doi:10.1007/s41114-022-00040-z
%[arXiv:2201.07241 [astro-ph.CO]].
%71 citations counted in INSPIRE as of 16 Jun 2023

\bibitem{DESI:2023ytc}
G.~Adame \textit{et al.} [DESI],
``The Early Data Release of the Dark Energy Spectroscopic Instrument,''
%doi:10.5281/zenodo.7964161
[arXiv:2306.06308 [astro-ph.CO]].
%13 citations counted in INSPIRE as of 26 Jul 2023

\bibitem{Akarsu:2022lhx}
O.~Akarsu, E.~\'O~Colg\'ain, E.~\"Ozulker, S.~Thakur and L.~Yin,
``Inevitable manifestation of wiggles in the expansion of the late Universe,''
Phys. Rev. D \textbf{107} (2023) no.12, 123526
%doi:10.1103/PhysRevD.107.123526
%[arXiv:2207.10609 [astro-ph.CO]].
%6 citations counted in INSPIRE as of 17 Jul 2023

\bibitem{Zhao:2017cud}
G.~B.~Zhao, M.~Raveri, L.~Pogosian, Y.~Wang, R.~G.~Crittenden, W.~J.~Handley, W.~J.~Percival, F.~Beutler, J.~Brinkmann and C.~H.~Chuang, \textit{et al.}
``Dynamical dark energy in light of the latest observations,''
Nature Astron. \textbf{1} (2017) no.9, 627-632
%doi:10.1038/s41550-017-0216-z
%[arXiv:1701.08165 [astro-ph.CO]].
%356 citations counted in INSPIRE as of 17 Jul 2023

\bibitem{Wang:2018fng}
Y.~Wang, L.~Pogosian, G.~B.~Zhao and A.~Zucca,
``Evolution of dark energy reconstructed from the latest observations,''
Astrophys. J. Lett. \textbf{869} (2018), L8
%doi:10.3847/2041-8213/aaf238
%[arXiv:1807.03772 [astro-ph.CO]].
%92 citations counted in INSPIRE as of 17 Jul 2023

\bibitem{Escamilla:2021uoj}
L.~A.~Escamilla and J.~A.~Vazquez,
``Model selection applied to reconstructions of the Dark Energy,''
Eur. Phys. J. C \textbf{83} (2023) no.3, 251
%doi:10.1140/epjc/s10052-023-11404-2
%[arXiv:2111.10457 [astro-ph.CO]].
%13 citations counted in INSPIRE as of 17 Jul 2023

\end{thebibliography}
\end{document}


\end{document}
%
% ****** End of file apssamp.tex ******
