\section{Additional Results}
\label{sec:appendix_additional_results}

\subsection{Content Personalization}
\label{sec:appendix_content_personalization}

Fig.~\ref{fig:ablation_features} shows \manner{}'s performance on MIND for different inputs to the NE. Even the \texttt{CR-Module} exposed to titles only (i.e., no abstract or entity information) outperforms all of the baselines on content recommendation.
Fig. \ref{fig:ablation_training} illustrates \manner{}'s performance for alternative architecture designs and training objectives (cf. \$\ref{sec:content_personalization}).\footnote{For brevity, we report results on MIND; findings on Adressa exhibit identical trends.}
%
% Figure environment removed


% Figure environment removed



\subsection{Single-Aspect Customization}
\label{sec:appendix_single_apsect_customization}
%
% Figure environment removed
Figure \ref{fig:single_aspect_div_adressa} illustrates \manner{}'s performance in single-aspect diversification tasks for different values of $\lambda$\textsubscript{ctg} and $\lambda$\textsubscript{snt} on Adressa. Sentiment diversity reaches peak performance for $\lambda_{snt}=-0.6$, while category diversity continues to increase all the way to $\lambda_{ctg}=-1.0$. The best trade-off (i.e., maximal performance w.r.t. T\textsubscript{A\textsubscript{p}}@10), is achieved for $\lambda_{ctg}=-0.3$ for topical categories, and $\lambda_{snt}=-0.2$ for sentiment. Similarly, Fig. \ref{fig:single_aspect_pers_adressa} explores the trade-off between content and aspect personalization, for different positive values of $\lambda_{A_p}$ on the Adressa dataset. We obtain the best topical category personalization (PS\textsubscript{ctg}) for $\lambda_{ctg}>0.4$.

Fig. \ref{fig:tsne_sent_mind} shows the 2-dimensional t-SNE visualizations \cite{van2008visualizing} of the news embeddings produced with sentiment-specialized NE trained on MIND. 
%
Fig. \ref{fig:tsne_embeddings_adressa} shows analogous visualizations \cite{van2008visualizing} of the news embeddings produced with aspect-specialized encoders trained on Adressa for: (a) topical categories, and (b) sentiment. 
%
% Figure environment removed

\subsection{Multi-Aspect Customization}
\label{sec:appendix_multi_apsect_customization}

Fig. \ref{fig:multi_aspect_results} explores the trade-off between content personalization and multi-aspect diversification, for different values of the aspect coefficients $\lambda_{ctg}$ and $\lambda_{snt}$, on MIND and Adressa, respectively.

% Figure environment removed   

\subsection{Cross-Lingual Transfer}
\label{sec:appendix_xlt}

%
% Figure environment removed
Fig. \ref{fig:adressa_transfer_mind_pers} summarizes the cross-lingual transfer results for single-aspect personalization on the target-language dataset MIND.
%
Fig. \ref{fig:xlt_adressa_results} summarizes the cross-lingual transfer results for single-aspect diversification and personalization, respectively, on the target-language dataset Adressa.
%
% Figure environment removed
