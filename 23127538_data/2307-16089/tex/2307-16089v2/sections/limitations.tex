\section*{Limitations}
\label{sec:limitations}

\manner{} targets exclusively content-based neural news recommendation and leverages solely textual features. In practice, recommender systems may incorporate content features from various other modalities (e.g., image, video), as well as similarities between users in a collaborative filtering manner. While in this work we experimented only with textual inputs (e.g., titles, named entities, topical categories), we believe that \manner{} can easily be extended to handle multi-modal content (e.g., either as additional input to the news encoder or as a dedicated \texttt{A-Module}), as well as collaborative user relations (e.g., by training an \texttt{A-Module} to group together users who consume similar articles). 

Our framework fully fine-tunes a PLM per aspect-specific module (either for content-relevance in the \texttt{CR-Module} or for aspect similarity in the \texttt{A-Module}). As all modules share the same PLM as backbone, parameter efficient fine-tuning (PEFT), e.g.  LoRA \cite{hu2021lora}, would bypass the need to repeatedly load the entire PLM per module into memory. PEFT has been shown to closely meet the performance of full fine-tuning. This represents a key advantage for deploying \manner{} in real-world applications. We however fully fine-tuned models to avoid PEFT as a confounding factor in our experiments. We further chose base-sized PLMs as the backbone of the news encoder in all models due to computational constraints. While in fine-tuning they remain competitive to large language models (LLMs), the latter may capture richer semantics, which can prove particularly useful for cross-lingual transfer applications. With a PEFT approach, \manner{} could easily leverage LLMs without a corresponding increase in computational resources.

Lastly, there exist varied approaches for measuring both the descriptive \cite{castells2021novelty}, as well as the normative \cite{vrijenhoek2023radio} diversity of recommendations. While some of these metrics can be tailored to support arbitrary aspects (i.e., to measure the diversity of recommendations w.r.t. to a particular categorical feature), we opted to quantify aspect-based diversity as generally as possible, leveraging only the distribution of an aspect's values in the recommendation list. We leave exploration of further diversity metrics to future work.

