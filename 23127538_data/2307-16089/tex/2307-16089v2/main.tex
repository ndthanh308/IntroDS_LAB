% This must be in the first 5 lines to tell arXiv to use pdfLaTeX, which is strongly recommended.
\pdfoutput=1
% In particular, the hyperref package requires pdfLaTeX in order to break URLs across lines.

\documentclass[11pt]{article}

% Change "review" to "final" to generate the final (sometimes called camera-ready) version.
% Change to "preprint" to generate a non-anonymous version with page numbers.
\usepackage[]{acl}

% Standard package includes
\usepackage{times}
\usepackage{latexsym}

% For proper rendering and hyphenation of words containing Latin characters (including in bib files)
\usepackage[T1]{fontenc}
% For Vietnamese characters
% \usepackage[T5]{fontenc}
% See https://www.latex-project.org/help/documentation/encguide.pdf for other character sets

% This assumes your files are encoded as UTF8
\usepackage[utf8]{inputenc}

% This is not strictly necessary, and may be commented out,
% but it will improve the layout of the manuscript,
% and will typically save some space.
\usepackage{microtype}

% This is also not strictly necessary, and may be commented out.
% However, it will improve the aesthetics of text in
% the typewriter font.
\usepackage{inconsolata}

%Including images in your LaTeX document requires adding
%additional package(s)
\usepackage{graphicx}
\usepackage{url}
\usepackage{float}
\usepackage{multirow}
\usepackage{color, colortbl}
\usepackage{arydshln}
\usepackage{array}
\usepackage{booktabs}
\usepackage{caption}
\usepackage{subcaption}
\usepackage{tikz}
\usepackage{amsmath}

\newcommand*\circled[1]{\tikz[baseline=(char.base)]{
            \node[shape=circle,draw,inner sep=2pt] (char) {#1};}}

\newcommand{\manner}{MANNeR}
\newcommand{\mannertitle}{MANNeR\textsubscript{title}}

\definecolor{Gray}{gray}{0.9}
% If the title and author information does not fit in the area allocated, uncomment the following
%
%\setlength\titlebox{<dim>}
%
% and set <dim> to something 5cm or larger.

\title{Train Once, Use Flexibly: A Modular Framework for Multi-Aspect Neural News Recommendation}

% Author information can be set in various styles:
% For several authors from the same institution:
% \author{Author 1 \and ... \and Author n \\
%         Address line \\ ... \\ Address line}
% if the names do not fit well on one line use
%         Author 1 \\ {\bf Author 2} \\ ... \\ {\bf Author n} \\
% For authors from different institutions:
% \author{Author 1 \\ Address line \\  ... \\ Address line
%         \And  ... \And
%         Author n \\ Address line \\ ... \\ Address line}
% To start a separate ``row'' of authors use \AND, as in
% \author{Author 1 \\ Address line \\  ... \\ Address line
%         \AND
%         Author 2 \\ Address line \\ ... \\ Address line \And
%         Author 3 \\ Address line \\ ... \\ Address line}

% \author{First Author \\
%   Affiliation / Address line 1 \\
%   Affiliation / Address line 2 \\
%   Affiliation / Address line 3 \\
%   \texttt{email@domain} \\\And
%   Second Author \\
%   Affiliation / Address line 1 \\
%   Affiliation / Address line 2 \\
%   Affiliation / Address line 3 \\
%   \texttt{email@domain} \\}

\author{
 \textbf{Andreea Iana\textsuperscript{1}},
 \textbf{Goran Glavaš\textsuperscript{2}},
 \textbf{Heiko Paulheim\textsuperscript{1}},
\\
 \textsuperscript{1}Data and Web Science Group, University of Mannheim, Germany
\\
 \textsuperscript{2}CAIDAS, University of Würzburg, Germany
\\
 \small{
  \{andreea.iana, heiko.paulheim\}@uni-mannheim.de, goran.glavas@uni-wuerzburg.de
 }
}

\begin{document}
\maketitle
\begin{abstract}
Recent neural news recommenders (NNRs) extend content-based recommendation (1) by aligning additional \textit{aspects} (e.g., topic, sentiment) between candidate news and user history or (2) by diversifying recommendations w.r.t. these aspects. This customization is achieved by ``hardcoding`` additional constraints into the NNR's architecture and/or training objectives: any change in the desired recommendation behavior thus requires retraining the model with a modified objective. This impedes widespread adoption of multi-aspect news recommenders.  
%
In this work, we introduce \manner{}, a modular framework for \textit{multi-aspect} neural news recommendation that supports on-the-fly customization over individual aspects at inference time. With metric-based learning as its backbone, \manner{} learns aspect-specialized news encoders and then \textit{flexibly} and \textit{linearly} combines the resulting aspect-specific similarity scores into different ranking functions, alleviating the need for ranking function-specific retraining of the model. 
%
Extensive experimental results show that \manner{} consistently outperforms state-of-the-art NNRs on both standard content-based recommendation and single- and multi-aspect customization. 
%
Lastly, we validate that \manner{}'s aspect-customization module is robust to language and domain transfer.   
\end{abstract}


\section{Introduction}
Current quantum hardware is unable to carry out universal quantum computations due to the buildup of errors that occur during the computation. 
The magnitude of the individual error is currently above the value that the Threshold Theorem requires in order to kick-start quantum error correction and fault-tolerant quantum computation~\cite[Section 10.6]{nielsen_chuang_2010}. 
Although the experimentally achieved fidelity rates are promising and the error bounds are inching closer to the required threshold, we will have to work for the foreseeable future with quantum hardware with errors that build-up during the computation.  This implies that we can only do a limited number of steps before the output of the computation has become completely uncorrelated with the intended one.

For fault-tolerant quantum computing, we repeat four steps: 
1) We apply a number of single and two-qubit quantum gates, in parallel whenever possible; 
2) We perform a syndrome measurement on a subset of the qubits; 
3) We perform fast classical computations to determine which errors have occurred and how to correct them; 
and, 4) We apply correction terms based on the classical computations.
We then repeat these four steps with a next sequence of gates. 
These four steps are essential to fault-tolerant quantum computing. 


The starting point of this work is to use the four steps outlined above, not to carry out error correction and fault-tolerant computation, but to enhance short, constant-depth, {\em uncorrected} quantum circuits that perform single qubit gates and {\em nearest-neighbor} two qubit gates. 
Since in the long run we will have to implement error-correction and fault-tolerant computation anyhow, and this is done by such a four-step process, why not make other use of this architecture? Moreover, on some of the quantum hardware platforms, these operations are already in place.
Embracing this idea we naturally arrive at the question: what is the computational power of \textit{low-depth} quantum-classical circuits organized as in the four steps outlined above? 
We thus investigate circuits that execute a small, ideally constant, number of stages, where at each stage we may apply, in parallel, single qubit gates and {\em nearest-neighbor} two qubit gates, followed by measurements, followed by low-depth classical computations of which the outcome can control quantum gates in later stages. 
It is not clear, at first, whether such circuits, especially with constant depth, can do anything remotely useful. 
But we will see that this is indeed the case: many quantum computations can be done by such circuits in constant depth. 
By parallelizing quantum computations in this way, we improve the overall computational capabilities of these circuits, as we do not incur errors on qubits that are idle, simply because qubits are not idle for a very long time. 
Furthermore, reducing the depth of quantum circuits, at the cost of increasing width, allows the circuit to be run faster even if errors occur.

The first usage of such a four-step layout, not to do error correction, but to perform computations, can be found in the paradigm of measurement-based quantum computing~\cite{gottesman1999demonstrating,raussendorf2001one,jozsa2006introduction,clark2007generalised}: 
A universal form of quantum computing where a quantum state is prepared and operations are performed by measuring qubits in different bases, depending on previous measurements and intermediate measurements.

\citeauthor{PhamSvore2013} were the first to formalize the four-step protocol for performing computations~\cite{PhamSvore2013}. They included specific hardware topologies by considering two-dimensional graphs for imposing constraints on qubit interactions. In their model, they develop circuits for particularly useful multi-qubit gates, including specifying costs in the width, number of qubits, depth, number of concurrent time steps, size, and total number of non-Identity operations.
As a result, they find an algorithm that factors integers in polylogarithmic depth.
\citeauthor{Browne:2011} showed that the main tool in the work by \citeauthor{PhamSvore2013}, the fan-out gate, can also be replaced by additional log-depth classical computations in the measurement-based quantum computing setting~\cite{Browne:2011}.

More recently, \citeauthor{Cirac:2021} introduced a scheme to implement unitary operations involving quantum circuits combined with Local Operations and Classical Communication ($\mathsf{LOCC}$) channels: $\mathsf{LOCC}$-assisted quantum circuits~\cite{Cirac:2021}. Similarly to the four-step scheme we just described, they allow for a short depth circuit to be run on the qubits, followed by one round of $\mathsf{LOCC}$, in which ancilla qubits are measured and local unitaries are applied based on the measurement outcomes. They show that in this model any 1D transitionally invariant matrix-product state (MPS) with fixed bond dimension is in the same phase of matter as the trivial state. Similar ideas can be found in~\cite{TVV_NonAbelianTopologicalOrder_2022, tantivasadakarn2021long}.

In this work, we introduce a new model, called \textit{Local Alternating Quantum-Classical Computations} ($\LAQCC$). In this model we alternate between running quantum circuits (constrained by locality), ending in the measurement of a subset of qubits, and fast classical computations based on the measurement results. The outcome of the classical computations are then used to control future quantum circuits. We allow for flexibility in this model, by giving different constraints to the power of both the quantum circuits and the classical circuits as well as the number of alternations between them. 
Most attention will be given to $\LAQCC$ containing quantum circuits of constant depth, classical circuits of logarithmic depth and at most a constant number of alternations between them. 
Any circuit constructed in this model is considered to be of constant depth. 
We restrict ourselves to logarithmic depth classical computations, as this is the first natural and non-trivial extension beyond constant-depth classical computations. 
Constant-depth classical computations do however also have an equivalent constant-depth quantum implementation.

The definition of $\LAQCC$ sharpens the original definition of \citeauthor{PhamSvore2013} by adding constraints to the intermediate classical computations. This allows us to bound the power of $\LAQCC$ from above. 

The main result of \citeauthor{Cirac:2021}, that 1D translational invariant MPS with fixed bond dimension can be prepared by $\mathsf{LOCC}$-assisted circuits, relies on local symmetries of the MPS. These symmetries allow them to prepare local states (on a constant number of qubits) and glue them together by doing one round of the appropriate entangling measurement and corrections, after which they run a round of local unitaries to get the desired result. This general scheme for preparing states that exhibit an MPS description with the appropriate local symmetries requires only geometrically local unitaries and one round of measurement and corrections an therefore is accessible in $\LAQCC$. Studying different local symmetries, known as Symmetry Protected Topological (SPT) phases of matter, to find measurement-based constant depth circuits for states is a broad ongoing field of research~\cite{TVV_NonAbelianTopologicalOrder_2022, tantivasadakarn2021long, smith2023deterministic}. 
All these schemes have a $\LAQCC$ implementation.

%$\LAQCC$-circuits also exist for general schemes of preparing local states, based on the local tensors, and gluing them together using one round of entangled measurement and corrections, based on the local symmetry. 
%The main result of \citeauthor{Cirac:2021}, that 1D translational invariant MPS with fixed bond dimension can be prepared by $\mathsf{LOCC}$-assisted circuits, relies heavily on local symmetries of the MPS and as a result also has an equivalent $\LAQCC$ implementation. 
%The corrections applied after the measurement round are local unitaries depending on the local symmetries of the MPS. 

 

%This general scheme of preparing local states, based on the local tensors, and gluing it together by doing one round of entangled measurement and corrections, based on the local symmetry, is accessible in $\LAQCC$.
Note however that \citeauthor{Cirac:2021} also suggest a circuit for the $W$-state.
This circuit uses sequentially and dependent measurement-based corrections of the ancilla qubits. 
These dependent measurements translate to sequential alternations between the quantum and classical circuits and therefore increase the total depth to linear depth, exceeding the constant-depth constraints imposed by $\LAQCC$-circuits. 

We study the power of the $\LAQCC$ model with respect to state preparation, showing that even with only constant quantum-depth and logarithmic classical depth it remains possible to prepare states with long-range entanglement.
Another surprising result is that it is unlikely that $\LAQCC$ circuits are classically simulatable. We show that any instantaneous quantum polynomial-time (IQP) circuit~\cite{Bremner2010,Shepherd2009} has an $\LAQCC$ implementation.
Classical simulation of IQP circuits implies the collapse of the polynomial hierarchy to the third level, which is not believed to be true~\cite{Bremner2017}. Therefore, we expect that $\LAQCC$ circuits are unlikely to be classically simulatable. We bound the power of $\LAQCC$ by showing that it is contained in $\QNC^1$, the class of polynomial-size, log-depth circuits.

Next, we also study the power that intermediate classical calculations can add to quantum computations, by considering a new model that alternates between polynomially many polynomial-depth quantum circuits and unbounded classical computations
We study this model by doing a complexity theoretical analysis, where we draw inspiration from the notions of complexity given by \citeauthor{RosenthalYuen:2022}, \citeauthor{MetgerYuen:2023}, and \citeauthor{Aaronson:2004}.
All three complexity notions are based on the notion of state preparation, instead of more traditional definition of complexity such as the decidability of a computational problem. 
The first two consider classes based on sequences of quantum states preparable by a polynomial-sized quantum circuit, where the circuits are uniformly generated by a computational class, for instance, the class $\mathsf{PSPACE}$, which results in the complexity class $\mathsf{StatePSPACE}$~\cite{RosenthalYuen:2022,MetgerYuen:2023}.
The third notion considers a relative complexity, where the complexity is measured between two given states, and is measured by the number of gates, from a given gate-set, required to transform one state in another state~\cite{Aaronson:2004}. 
For our definition of state preparation complexity, we drop the uniformity constraint from~\cite{RosenthalYuen:2022,MetgerYuen:2023} and define a class as $\mathsf{StateX}$, which refers to states preparable by circuits of type $\mathsf{X}$. 
As an example, if $\mathsf{X} = \QNC^0$, this results in the class $\mathsf{StateQNC^0}$, which is the set of states preparable from the $\ket{0}^n$ state by poly-size constant-depth circuits. 
This notion is similar to the relative complexity from~\cite{Aaronson:2004}, where one state is the  $\ket{0}^n$ state and instead of counting the number of gates we consider the set of states preparable by a fixed number of gates. Using this notion of complexity we show that any state preparable by an $\LAQCC^*$ circuit is also preparable by a $\mathsf{PostQPoly}$ circuit, the class of circuits of polynomial depth with an additional post-selection gate. 

All Clifford circuits have a constant-depth $\LAQCC$ implementation, implying that any stabilizer state can be implemented by a constant-depth $\LAQCC$ circuit, see Section~\ref{sec:clifford_circuits} for a proof of this statement. 
Efficient circuits for stabilizer states have been known already through measurement-based quantum computing. Therefore this paper focuses on the preparation of non-stabilizer states, and as a surprising result we find novel constant-depth protocols for four very natural classes of non-stabilizer states.
Despite the extensive research into these four classes of non-stabilizer states and the many applications of them, no efficient constant- or low-depth state preparation protocols are known yet. We specifically consider these four classes as they are all often used as initial states in other algorithms.

The first state is a uniform superposition over an arbitrary number of states. 
This state finds applications in many quantum algorithms, as they often start with a uniform superposition over multiple states. 
This superposition is often achieved by applying Hadamard gates to every qubit due to its simplicity to prepare. 
Yet, the analysis of many algorithms, such as Shor's algorithm~\cite{Shor:1997}, would benefit from a different initial superposition. 
The circuit to prepare the uniform superposition over an arbitrary number of states uses an exact version of Grover search as a subroutine, that turns a probabilistic circuit, with a known constant probability of success, into a deterministic circuit. 
We use the circuit for preparing a uniform superposition over an arbitrary number of states as a subroutine in the next two quantum state preparation protocols. 

The second state is the $W$-state, the uniform superposition over all computational basis states of Hamming-weight~$1$, a natural long-ranged entangled state that displays a fundamentally nonequivalent type of entanglement from the Greenberger–Horne–Zeilinger state~\cite{WState:2000}, for which $\LAQCC$-type constant-depth circuits were previously known~\cite{PhamSvore2013, Cirac:2021}. 
The $W$-state is often used as benchmark for new quantum hardware~\cite{Haffner2005,Neeley2010,GarciaPerez:2021}. 
A novel way to prepare the $W$-state therefore gives a new way to benchmark different quantum devices with each other. 
A circuit for preparing the $W$-state was given in~\cite{Cirac:2021}, but this implementation requires sequentially alternating measurements followed by local unitaries, which in the $\LAQCC$ model is not considered to be of constant depth. 
We improve this protocol by giving an $\LAQCC$ implementation of the $W$-state, based on a compress-uncompress method that links the one-hot and binary encoding of integers.

The third state considered is the Dicke state, a generalization of the $W$-state, a superposition over all computational basis states with Hamming-weight $k$~\cite{Dicke:1954}. 
Dicke states have relevance in various practical settings.
For instance, for quantum game theory~\cite{zdemir2007}, quantum storage~\cite{Bacon_Compress:2006,Plesch:2010}, quantum error correction~\cite{ouyang2014permutation}, quantum metrology~\cite{toth2012multipartite}, and quantum networking~\cite{prevedel2009experimental}. 
Dicke states have been used as a starting state for variational optimization algorithms, most notably Quantum Alternating Operator Ansatz (QAOA)~\cite{Hadfield2019}, to find solutions to problems such as Maximum k-vertex Cover~\cite{Brandhofer2022,cook2020quantum}.
The ground states of physical Hamiltonians describing one-dimensional chains tend to show a resemblance to Dicke states such as states resulting from the Bethe ansatz, making them an ideal starting state when investigating the ground state behavior of these Hamiltonians~\cite{TDL_BetheAnsatzDerivation:2010,B_ExcitedStateQuantumPhaseTransitions:2013,DickeTransitions:2021}. 
For instance, the algorithm by \citeauthor{van2021preparing}, who give an algorithm to prepare the Bethe ansatz eigenstates of the spin-1/2 XXZ spin chain, starts by first preparing a Dicke state~\cite{van2021preparing}. 
A Dicke-state preparation protocol based on the compress-uncompress methodology used in the $W$-state furthermore finds applications in entanglement distillation, where the entanglement of a large state is concentrated on only a few qubits. 
Efficient deterministic circuits for preparing Dicke states have been proposed by \citeauthor{bartschi2019deterministic}~\cite{bartschi2019deterministic, bartschi2022deterministic_short_depth}. 
They provide a quantum circuit of depth $\mathO(k \log(\frac{n}{k}))$, allowing arbitrary connectivity, to prepare a Dicke state, which they conjecture to be optimal when $k$ is constant. 
In this work, we provide a constant-depth $\LAQCC$ circuit below their conjectured bound already for constant $k$. 
However, this does not directly disprove their conjecture, as we allow for intermediate measurements and classical computations. 
More significantly, we even construct constant-depth $\LAQCC$ circuits for $k = \mathO(\sqrt{n})$ greatly improving their bound.
This construction extends the compress-uncompress method for the $W$-state combined with additional subroutines. 

We continue with a log-depth state preparation protocol for the Dicke-state for arbitrary $k$. 
This protocol implements an efficient transformation between the factoradic number representation and the combinatorial number representation of a positive integer. 
The combinatorial number representation relates directly to the Dicke state. 
The provided efficient transformation between number representation systems might be of independent interest. 

We conclude by modifying our protocol for preparing a Dicke-state to a protocol that prepares quantum many-body scar states in constant-depth. 
These states have low entanglement and longer coherence times than states with similar energy density.
These characteristics make many-body scar states interesting to analyze and relevant within physics.
Many-body scar states appear for instance in the AKLT model~\cite{AKLT:1987,MRBAR:2018,MRB:2018} and different spin models~\cite{SI:2019,MOBFR:2020}.
Known methods for preparing these states have polynomial-depth~\cite{Gustafson:2023}, whereas our circuit has constant depth. 

% We conclude by studying the power that intermediate classical calculations can add to quantum computations. 
% In this study, we define a new model that relaxes constant-depth quantum circuits to polynomial depth quantum circuits, log-depth classical calculations to unbounded classical computations and a constant number of alternations to a polynomial number of alternations. 
% We call this model $\LAQCC^*$. 
% We study this model by doing a complexity theoretical analysis, where we draw inspiration from the notions of complexity given by \citeauthor{RosenthalYuen:2022}, \citeauthor{MetgerYuen:2023}, and \citeauthor{Aaronson:2004}.
% All three complexity notions are based on the notion of state preparation, instead of more traditional definition of complexity such as the decidability of a computational problem. 
% The first two consider classes based on sequences of quantum states preparable by a polynomial-sized quantum circuit, where the circuits are uniformly generated by a computational class, for instance, the class $\mathsf{PSPACE}$, which results in the complexity class $\mathsf{StatePSPACE}$~\cite{RosenthalYuen:2022,MetgerYuen:2023}.
% The third notion considers a relative complexity, where the complexity is measured between two given states, and is measured by the number of gates, from a given gate-set, required to transform one state in another state~\cite{Aaronson:2004}. 
% For our definition of state preparation complexity, we drop the uniformity constraint from~\cite{RosenthalYuen:2022,MetgerYuen:2023} and define a class as $\mathsf{StateX}$, which refers to states preparable by circuits of type $\mathsf{X}$. 
% As an example, if $\mathsf{X} = \QNC^0$, this results in the class $\mathsf{StateQNC^0}$, which is the set of states preparable from the $\ket{0}^n$ state by poly-size constant-depth circuits. 
% This notion is similar to the relative complexity from~\cite{Aaronson:2004}, where one state is the  $\ket{0}^n$ state and instead of counting the number of gates we consider the set of states preparable by a fixed number of gates. Using this notion of complexity we show that any state preparable by an $\LAQCC^*$ circuit is also preparable by a $\mathsf{PostQPoly}$ circuit, the class of circuits of polynomial depth with an additional post-selection gate. 

\paragraph{Summary of results}
\begin{itemize}
    \item We give a new definition of a computational model that captures the power of the four step process: applying a constant number of layers of one- and two-qubit gates; performing a syndrome measurement; perform a fast classical computation determining corrections; apply corrections. We call this model \emph{Local Alternating Quantum Classical Computations}, or $\LAQCC$ for short. In this model we bound the allowed quantum operations, intermediate classical calculations, and number of rounds separately. In Section~\ref{sec:LAQCC_model} we define this model and give a list of operations based on results from literature contained in this computational model. In some of these operations we explicitly use that we allow for multiple, but at most constant, rounds  of corrections.
    \item  We show show that there exist $\LAQCC$ circuits that can not be weakly simulated in Section~\ref{sec:IQP_in_LAQCC}. We further show that for every $\LAQCC$ circuit there exists a $\QNC^1$ circuit simulating it perfectly, in Section~\ref{sec:LAQCC_in_QNC1}.
    \item We introduce a new type computational complexity for preparing states and show that the extension of $\LAQCC$ where we allow a polynomial number of rounds and unbounded classical computation, is contained in $\mathsf{PostQPoly}$, the class of polynomial circuits with post-selection, in Section~\ref{sec:Complexity results}.
    \item We show a protocol to prepare the uniform superposition state of size $q$ in $\LAQCC$ using $\mathO(\ceil{\log_2(q)}^2)$ qubits in Section~\ref{sec:superposition_modulo_q}. 
    \item We show a protocol to prepare the $W_n$ state in $\LAQCC$ using $\mathO(n\log(n))$ qubits in Section~\ref{sec:W_state_in_LAQCC}.
    \item We show two ways of preparing the Dicke-$(n,k)$ state. The first method is in $\LAQCC$, works up to $k = \mathO(\sqrt{n})$, uses $\mathO(n^2\log(n))$ qubits, and is found in Section~\ref{sec:dicke:small_k}. The second method is in $\LAQCC\text{-}\mathsf{LOG}$ (an extension of $\LAQCC$ allowing for logarithmic number of alterations instead of constant), works for any $k$, uses $\mathO(\text{poly}(n))$ qubits, and is found in Section~\ref{sec:Dicke_in_LAQCC_LOG}. 
    \item We extend on our $\LAQCC$ method of generating Dicke-$(n,k)$ states for $k = \mathO(\sqrt{n})$ and show a protocol to generate many-body scar states for a particular Hamiltonian in $\LAQCC$ (Section~\ref{sec:many_body_scar}). 
\end{itemize}
Summarized in a table, we provide the following state generation protocols:
\begin{table}[htb]
\centering
\begin{tabular}{l|l|l|l}
\textbf{State description} & \textbf{Width} & \textbf{Depth} & \textbf{Implementation}\\
\hline 
Uniform superposition mod $q$: $\frac{1}{\sqrt{q}} \sum_{i = 0}^{q-1}\ket{i}$ & $\mathO(\ceil{\log^2 q})$ & $\mathO(1)$ & Section~\ref{sec:superposition_modulo_q}\\

$W$-state: $\frac{1}{\sqrt{n}}\sum_{i = 0}^{n-1}\ket{e_i}$ & $\mathO(n \log n)$ & $\mathO(1)$ & Section~\ref{sec:W_state_in_LAQCC}\\

Dicke-$(n,k)$, $k = \mathO(\sqrt{n})$: $\binom{n}{k}^{-1/2}\sum_{x \in \{0,1\}^n: |x| = k} \ket{x}$ &  $\mathO(n^2\log n)$ & $\mathO(1)$ 
&Section~\ref{sec:dicke:small_k}\\

Dicke-$(n,k)$: $\binom{n}{k}^{-1/2}\sum_{x \in \{0,1\}^n: |x| = k} \ket{x}$ & $\mathO(\text{poly}(n))$ & $\mathO(\log n)$ &Section~\ref{sec:Dicke_in_LAQCC_LOG}\\

QMBS: $\ket{S_k} = \frac{1}{k! \sqrt{\mathcal N(n,k)}}(Q^\dagger)^k \ket{\Omega}$ &  $\mathO(n^2\log n)$ & $\mathO(1)$  &  Section~\ref{sec:many_body_scar}
\end{tabular}
\caption{Summary of state preparation protocols given in this paper.}
\label{tab:sate_prep}
\end{table}
In the entry for the quantum many-body scar state $Q$ denotes the raising operator and $\mathcal N(n,k)=\binom{n-k-1}{k}$. 
Section~\ref{sec:many_body_scar} will provide more details on the variables and the implementation. 

\paragraph{Organization of the paper}
\noindent We first introduce relevant preliminaries in Section~\ref{sec:preliminaries}. 
In Section~\ref{sec:LAQCC_model} we formally define the class of Local Alternating Quantum-Classical Computations ($\LAQCC$). We also show that any Clifford circuit can be implemented in constant depth $\LAQCC$ (a result based on a result from measurement-based quantum computing~\cite{jozsa2006introduction}). 
This result allows us to give many useful multi-qubit gates and routines in Section~\ref{sec:gates_created_in_LAQCC}. 
Beyond that we show that constant depth $\LAQCC$ circuits are contained in $\QNC^1$ and that any $\mathsf{IQP}$ circuit has an $\LAQCC$ implementation.
We conclude this section with an analysis of a more powerful instantiation of $\LAQCC$ and show an inclusion with respect to the class $\mathsf{PostQPoly}$, which is the class of circuits of polynomial depth with one additional post-selection gate. 
In Section~\ref{sec:state_prep_in_LAQCC} we give $\LAQCC$ circuit implementations for preparing the uniform superposition over an arbitrary number of states, the $W$-state and the Dicke state up to $k = \mathO(\sqrt{n})$. We furthermore give a log-depth circuit implementation for preparing the Dicke state for any $k$. We conclude by showing a $\LAQCC$ circuit for generating many body scar states of a particular type of Hamiltonian.


\section{Related Work}
%\subsection{Cost Volume based Deep Stereo Matching}
%Stereo matching is a typical problem that has been studied for decades and a well-known four-step pipeline \cite{scharstein2002taxonomy} has been established, where cost volume construction is an indispensable step. Current state-of-the-art stereo matching methods are all cost volume based methods and they can be categorized into two types. Typically, a cost volume is a 4D tensor of height, width, disparity, and features. The first category just uses a full correlation to generate a single-feature cost volume. Such methods are usually efficient but lose much information because of the decimation of feature channels. Many previous work, including Dispnet \cite{dispnet}, MADNet \cite{madnet}, IResNet \cite{iresnet} and AANet \cite{aanet}, belong to this category. The second category usually uses concatenation \cite{gcnet} or group-wise correlation \cite{gwcnet} to generate a multi-feature 4D cost volume. Such a method can achieve better performance while requiring higher computational complexity and memory consumption. Actually, a majority of the top-performing networks in public leaderboards belong to this category, such as GANet \cite{ganet}, CSPN \cite{cspn} and ACFNet \cite{acfnet}. These methods generally employ multiple 3D convolution layers to constantly regularize the 4D cost volume and then apply softmax over the disparity dimension to produce a discrete disparity probability distribution. The final predicted disparity is obtained by softly weighting indices according to their probability, which is also called soft argmin in GCNet \cite{gcnet}. However, soft argmin leaves the output susceptible to multi-modal disparity probability distributions. ACFNet \cite{acfnet} observes this problem and proposes to directly supervise the cost volume with unimodal ground truth distributions. In contrast, we define an uncertainty estimation to quantify the degree to which the cost volume tends to be multi-modal distribution, higher implies the higher possibility of estimation error.

\subsection{Multi-scale Cost Volume based Stereo Matching}
Cost volume construction is an indispensable step in the well-known four-step pipeline for stereo matching \cite{scharstein2002taxonomy, pamisurvey1, pamisurvey2}. Typically, current state-of-the-art stereo matching methods can be categorized into two types of cost volume-based methods, where the cost volume is a 4D tensor of height, width, disparity, and features. The first category usually uses the single-feature 3D cost volume generated by full correlation, which is efficient while losing much information due to the decimation of feature channels. Many real-time methods, such as Dispnet \cite{dispnet}, MADNet \cite{madnet, madnet_pami} and AANet \cite{aanet}, belongs to the category. Moreover, two-stage refinement \cite{mcvmfc} and pyramidal towers \cite{madnet} are commonly applied in the single-feature cost volume based network to construct multi-scale cost volume. The second category usually uses the multi-feature 4D cost volume generated by concatenation \cite{gcnet} or group-wise correlation \cite{gwcnet}, which can achieve better performance with higher computational complexity and memory consumption. Most top-performing networks, including GANet \cite{ganet}, CSPN \cite{cspn} and ACFNet \cite{acfnet} belong to this category. 
% In these methods, the 4D cost volume is constantly regularized by multiple 3D convolution layers and then a discrete disparity probability distribution can be produced by softmax. Next, the final predicted disparity can be obtained by softly weighting indices according to their probability \cite{gcnet}. However, such output is susceptible to multimodal disparity probability distributions and ACFNet \cite{acfnet} gives a solution by directly supervising the cost volume with unimodal ground truth distributions to alleviate this problem. 
Recently, to alleviate the high computational complexity and memory consumption when employing multi-feature 4D cost volumes, \cite{cvpmvsnet, cascade, uscnet} propose to use cascade cost volume representation in multi-view stereo. These methods usually first predict an initial disparity at the coarsest resolution of the image and then gradually refine the disparity by narrowing down the disparity search space. More closely related to our approach is Casstereo \cite{cascade}, which first extended such representation to stereo matching. It selected to uniform sample a pre-defined range to generate the next stage’s disparity search range. Instead, we employ pixel-level uncertainty estimation to adaptively adjust the next stage disparity searching range and generate pseudo-labels for subsequent domain adaptation. Our method also shares similarities with UCSNet \cite{uscnet}, which constructs uncertainty-aware cost volume in multi-view stereo while it doesn’t employ uncertainty estimation to generate pseudo-labels.

%\subsection{Multi-scale Cost Volume based Deep Stereo Matching} 
% \subsection{Multi-scale Cost Volume based Stereo Matching} 
%Multi-scale cost volume firstly was applied in the single-feature cost volume based network with the form of two-stage refinement \cite{mcvmfc} and pyramidal towers \cite{madnet}. Recently, cascade cost volume representation \cite{cvpmvsnet, cascade, uscnet} was proposed in multi-view stereo to alleviate the high computational complexity and memory consumption when employing multi-feature 4D cost volumes. These methods generally predict an initial disparity at the coarsest resolution of the image. Then, they will narrow down the disparity search space and gradually refine the disparity. More closely related to our approach is Casstereo \cite{cascade}, which first extended such representation to stereo matching. It selected to uniform sample a pre-defined range to generate the next stage’s disparity search range. Instead, we employ uncertainty estimation to adaptively adjust the next stage pixel-level disparity searching range and push the next stage's cost volume to be predominantly unimodal.

% The single-feature cost volume based network with the form of two-stage refinement \cite{mcvmfc} and pyramidal towers \cite{madnet} first employ multi-scale cost volume for stereo matching. Recently, to alleviate the high computational complexity and memory consumption when employing multi-feature 4D cost volumes, \cite{cvpmvsnet, cascade, uscnet} propose to use cascade cost volume representation in multi-view stereo, which generally predict an initial disparity at the coarsest resolution of the image. Then, the disparity search space is narrowed down and the disparity is gradually refined. More closely related to our approach is Casstereo \cite{cascade}, which first extended such representation to stereo matching. It selected to uniform sample a pre-defined range to generate the next stage’s disparity search range. Instead, we employ uncertainty estimation to adaptively adjust the next stage pixel-level disparity searching range and push the next stage's cost volume to be predominantly unimodal.

% Figure environment removed

\subsection{Robust Stereo Matching} 
There exist three categories of generalization definitions for robust stereo matching. 1) Cross-domain Generalization: the network’s ability to perform well on unseen scenes (cannot see the image pairs of the target domain in advance). Towards this end, Jia et al \cite{sungeneralizaiton} propose to incorporate scene geometry priors into an end-to-end network. Zhang et al \cite{dsmnet} introduce a domain normalization and a trainable non-local graph-based filter to construct a domain-invariant stereo matching network. 2) Adapt Generalization: the network’s ability to adapt pre-trained models to the new domain with unlabeled target data. Previous work usually pre-trains the models on synthetic data and then adapts it to new target domains with Graph Laplacian regularization \cite{zoom}, non-adversarial progressive color transfer \cite{adastereo}, and Knowledge Reverse Distillation \cite{aohnet}. More closely related to our approach are \cite{aohnet, unsuperviseddomainadaptation} in stereo matching and Monoresmatch \cite{monoresmatch} in monocular depth estimation, which also proposes to generate a pseudo-label for domain adaptation. However, these methods all select to employ classical stereo matching methods \cite{sgm} alongside with confidence estimators, e.g., left-right consistency check to generate pseudo-labels. That is all these methods need an independent method to generate corresponding pseudo-labels. Instead, the proposed method is an end-to-end network that can generate the predicted disparity map, corresponding uncertainty map and pseudo-labels jointly, which is a more simple, yet efficient way. 
% Instead, our proposed method can employ pixel-level and area-level uncertainty estimation to self-distill the predicted disparity maps of our pre-training model and generate sparse while reliable pseudo-labels to align the domain gap, which is a more simple, yet efficient way. 
3) Joint Generalization: the network’s ability to perform well on a variety of datasets with the same model parameters. MCV-MFC \cite{mcvmfc} introduces a two-stage finetuning scheme to achieve a good trade-off between generalization and fitting capability on multiple datasets. However, it doesn’t touch the inner difference between diverse datasets, e.g, the unbalanced disparity distribution. To further address this problem, we propose a cascade cost volume to adaptively the next stage disparity searching space, where the pixel-level uncertainty estimation is at the core.

% \subsection{Monocular Depth Estimation}
% Monocular depth estimation aims to estimate depth values from a single image, instead of stereo images or multiple frames in a video. This problem is ill-posed because of the ambiguity of object sizes. However, humans could estimate the depth from a single image with prior knowledge of the scenes. Recently, learning based methods were explored to learn depth values by supervised or unsupervised learning. Eigen et al. first employed Convolutional Neural Networks (CNN) to predict depth in a coarse-to-fine manner and further improved its performance by multi-task learning. Liu et al. presented deep convolutional neural fields model by combining deep model with continuous CRF. Li et al. [22] refined deep CNN outputs with a hierarchical CRF. Multi-scale continuous CRF was formulated into a deep sequential network by Xu et al. [45] to refine depth estimation. Unsupervised methods tried to train monocular depth estimation with stereo
% image pairs or image sequences and test on single images. Garg et al. [9] used novel image view synthesis loss to train a depth estimation network in an unsupervised way. Godard et al. [11] introduced left-right consistency regularization to improve the performance of view synthesis loss. Recently, some work also propose to use the stereo matching network as a proxy to learn depth from synthetic data or directly employ traditional stereo matching methods to distill proxies labels from the target domain, which proves the feasibility of distilling stereo matching networks to learn monocular depth estimation.



\section{Cross-Lingual Diffusion Language Model}
\label{sec:XDLMusion}

% In this section, we present our proposed language modeling objectives designed specifically for diffusion and the diffusion model applied for cross-lingual translation. These objectives cater to both monolingual and multilingual data, and they are situated within the diffusion model framework for facilitating cross-lingual translation.

In this section, we present the Cross-lingual Diffusion Language Model (XDLM), which incorporates a pre-training phase on cross-lingual data, utilizing diffusion techniques for the purpose of non-autoregressive machine translation, and a fine-tuning phase generating corresponding text from one language to another language based on the pre-trained model.

% \subsection{Preliminary}
% \subsubsection{Cross-lingual translation}
% (\irene{combine 3.1.1 and 3.1.2 as NAR machine translation, and, there is no such term called \textit{Cross-lingual translation}, all translation is cross-lingual, it should be either \textit{machine translation} or \textit{cross-lingual language model}})

% Cross-lingual translation typically involves generating an output sequence $Y=\{y_1, y_2,…, y_{|Y|}\}$ from a given input sequence $X=\{x_1,x_2,…,x_{|X|}\}$, with each sequence being in a different language. Three common generative paradigms exist for cross-lingual translation: AutoRegressive (AR) generation, Non-AutoRegressive (NAR) generation, and semi-NAR generation. Ordinarily, diffusion models employ the NAR approach for generation tasks.

% \subsubsection{Non-AutoRegressive(NAR) generation}
% The NAR generation follows the conditional probality: 
% $$
% p_{\theta}(Y|X)=\prod_{i=1}^{|Y|} p_{\theta}(y_i|X)
% $$

% Unlike AutoRegressive (AR) generation, all tokens $y_i$$(0\leq i \leq |Y|)$ in the generated sequence Y are predicted concurrently. The generation solely depends on the input sequence X, without any dependency on preceding tokens. This attribute presents a challenge in determining the length of the generated sequence. To address this issue, the prediction of the output sequence is introduced as an auxiliary task \cite{gu2017non}.

\textbf{Non-AutoRegressive (NAR) Machine Translation}
In machine translation, given the input sequence from a source language $X=\{x_1,x_2,…,x_{|X|}\}$, the task is to generate the output sequence of the translation in the target language $Y=\{y_1, y_2,…, y_{|Y|}\}$. In this work, we focus on the Non-AutoRegressive (NAR) translation setting with the diffusion model. Typically, it has the following conditional probability:  
$$
p_{\theta}(Y|X)=\prod_{i=1}^{|Y|} p_{\theta}(y_i|X).
$$

Unlike AutoRegressive (AR) text generation, all tokens $y_i$$(0\leq i \leq |Y|)$ in the generated sequence $Y$ are predicted concurrently. The generation solely depends on the input sequence $X$, without any dependency on preceding tokens. This attribute presents a challenge in determining the length of the generated sequence. To address this issue, the length prediction of the output sequence is introduced as an auxiliary task \cite{gu2017non}. And the training loss is defined as a weighted sum between the translation loss and the length prediction loss.

\textbf{Diffusion Models}
The Denoising Diffusion Probabilistic Model (DDPM) \cite{ho2020denoising} is a parametrized Markov chain, and it is trained using variational inference to generate samples that match the original input data. 
% a diffusion process for generative tasks was introduced by \cite{ho2020denoising}, yielding impressive results.
The diffusion process comprises a noise-adding forward process and a noise-removing backward process, both of which can be viewed as discrete-time Markov processes. During the forward process, the model gradually introduces random noise with different scheduled variance $\beta_1,...,\beta_t$, with the aim of generating a standard Gaussian noise $x_t$ after $t$ turns. This can be formalized as follows:
$$
q(x_{t+1}|x_t)=\mathcal{N}(x_{t+1};\sqrt{1-\beta_{t+1}}x_t,\beta_{t}\mathbf{I}).
$$

The backward process, the reverse of the forward process, attempts to reconstruct the target sequence from the standard noise. Like the forward process, this procedure is also applied incrementally and can be formalized as follows:

$$
    p(x_{t-1}|x_t)=\mathcal{N}(x_{t-1};\mu_{\theta}^{t-1},\sigma_{\theta}^{t-1}),
$$
$$
    \mu_{\theta}^{t-1}=\frac{1}{\sqrt{\alpha_{t}}}(x_t-\frac{\beta_{t}}{\sqrt{1-\overline(\alpha_{t})}}z_{\theta}(x_{t},t)), 
$$
$$
    \sigma_{\theta}^{{t-1}^2}=\frac{1-\overline{\alpha_{t-1}}}{1-\overline{\alpha_{t}}}\dot \beta_{t},
$$

where $\alpha_t=1-\beta_t, \overline{\alpha_{t}}=\prod_{i=1}^t \alpha_{i}$ and $z_\theta$ comes from the prediction of model parameterized by $\theta$. 
In this work, we apply discrete diffusion for text generating and cross-lingual translation. Based on \citet{zheng2023reparameterized}, we follow the proposed discrete diffusion model with the following routing mechanism.

$
    x_{t-1}, v_{t-1} \sim q(x_{t-1},v_{t-1}|x_t,x_0) \\
    q(v_{t-1}|x_t,x_0)=q(v_{t-1})=Bernoulli(\lambda) \\
    q(x_{t-1}|v_{t-1},x_t,x_0)= \\
    v_{t-1}x_t+(1-v^{(1)}_{t-1})q_{noise}, \quad if \quad x_t = x_0 \\
    v_{t-1}x_0+(1-v_{t-1}^{(2)})q_{noise} (x_t), \quad if \quad x_t \neq x_0 \\
$


Which models the joint distribution over both $x$ and $v$. The sampling process here also takes the reparameterized method, which improves flexibility and expressiveness compared to the original process.

% Figure environment removed
\textbf{Translation Diffusion Language Modeling (TDLM)}
% Contrary to previous language modeling objectives for diffusion models, which primarily focus on monolingual data and neglect the potential to harness cross-lingual modeling capabilities from parallel datasets, we propose a pretraining process for parallel language pairs along with a corresponding modeling objective.
Unlike previous diffusion model objectives for language modeling that primarily concentrate on monolingual data, we target to exploit cross-lingual modeling capabilities from parallel datasets. Consequently, we propose a pretraining process named Translation Diffusion Language Modeling (TDLM), aiming at enhancing cross-lingual pretraining with diffusion models. As illustrated in Figure 1, we first concatenate both source and target sentences and generate the corresponding language and position embedding sequences, and then stack them as the input to a diffusion model. 
% we select both source and target sentences, generate their corresponding language and position embedding series, and concatenate them to form the input text stream. 
In a similar vein to \citet{lin2023text}, we random mask 15\% of the tokens to the input as \cite{lample2019cross} designed, tasking the model with predicting the noise and its surrounding text based on the cross-lingual context. This denoising setting assists the model in grasping the cross-lingual context.


\section{Experimental Setup}
\label{sec:experimental_setup}

We compare \manner{} against state-of-the-art NNRs on a range of single- and multi-aspect recommendation tasks. We experiment with two aspects: \textit{topical categories} (\textit{ctg}) and news \textit{sentiment} (\textit{snt}).

\vspace{1.4mm}
\noindent\textbf{Baselines.} 
We evaluate several NNRs trained on classification objectives. We follow \citet{wu2021empowering} and replace the original NEs of all baselines that do not use PLMs (instead, contextualizing word embeddings with convolution or self-attention layers) with the same PLM used in \manner{}.\footnote{The only exception is the final text embedding, where \citet{wu2021empowering} pool tokens with an attention network.} % instead of using the \texttt{[CLS]} embedding.}
%
We include two models optimized purely for content personalization: (1) NRMS \cite{wu2019nrms}, and (2) MINER \cite{li2022miner}. We further evaluate seven NNRs that inject aspect information. Thereof, five incorporate \textit{topical categories}, i.e., (3) NAML \cite{wu2019naml}, (4) LSTUR \cite{an2019lstur}, (5) MINS \cite{wang2022news}, (6) CAUM \cite{qi2022news}, (7) TANR \cite{wu2019tanr}, and two the news \textit{sentiment}: (8) SentiRec \cite{wu2020sentirec}, and (9) SentiDebias \cite{wu2022removing}. For more details, see Appendix \ref{sec:appendix_baselines}. 

\vspace{1.4mm}
\noindent\textbf{Data.}
%
We carry out the evaluation on two prominent monolingual news recommendation benchmarks: MINDlarge (denoted MIND) \cite{wu2020mind} with news in English and Adressa-1 week \cite{gulla2017adressa} (denoted Adressa) with Norwegian news. We provide further details about dataset usage and statistics in Appendix \ref{sec:appendix_datasets}. 
%
As Adressa contains no disambiguated named entities, we use only the news title as input to \manner{}' NE, while on MIND we use all news features as NE input.


\vspace{1.4mm}
\noindent\textbf{Evaluation Metrics.}
We report performance with AUC, MRR, nDCG@k ($k \hspace{-0.2em} = \hspace{-0.2em} \{5, 10\}$). We measure aspect-based diversity of recommendations at position $k$ as the normalized entropy of the distribution of aspect $A_p$'s values in the recommendation list: 
%
\begin{equation}
\small
    D_{A_p}@k \hspace{-0.2em} = \hspace{-0.2em} - \sum_{j \in A_p} \frac{p(j) \log p(j)}{\log(|A_p|)}
\end{equation}
where $A_p \hspace{-0.2em} \in \hspace{-0.2em} \{ctg, snt\}$, and $|A_p|$ is the number of $A_p$ classes.
%
If aspect-based personalization is successful, aspect $A_p$'s distribution in the recommendations should be similar to its distribution in the user history. We evaluate personalization with the generalized Jaccard similarity \cite{bonnici2020kullback}:
% , a robust measure of similarity between two probability distributions 
%
\begin{equation}
\small
    \mathit{PS}_{A_p}@k \hspace{-0.2em} = \hspace{-0.2em} \frac{\sum_{j=1}^{|A_p|} \min(\mathcal{R}_j, \mathcal{H}_j)}{\sum_{j=1}^{|A_p|} \max(\mathcal{R}_j, \mathcal{H}_j)},
\end{equation}
where $R_j$ and $H_j$ represent the probability of a news with class $j$ of $A_p$ to be contained in the recommendations list $R$, and, respectively, in the user history $H$.
%
As all metrics are bounded to $[0, 1]$, we measure the trade-off between content-based personalization (nDCG@$k$) and either aspect-based diversity $D_{A_p}@k$ or aspect-based personalization $\mathit{PS}_{A_p}@k$ with the harmonic mean. We denote this T\textsubscript{A\textsubscript{p}}@$k$ for single-aspect. For multi-aspect evaluation, i.e., when ranking for content-personalization by diversifying simultaneously over topics and sentiment, we adopt as evaluation metric the harmonic mean between nDCG@$k$, D\textsubscript{ctg}@$k$ (topical category), and D\textsubscript{snt}@$k$ (sentiment), denoted 
T\textsubscript{all}@$k$. 

\vspace{1.4mm}
\noindent\textbf{Training Details.}
%
We use RoBERTa Base \cite{liu2019roberta} and NB-BERT Base \cite{kummervold2021operationalizing,nielsen2023scandeval} in experiments on MIND and Adressa, respectively. 
%
We set the maximum history length to 50. We tune the main hyperparameters of all NNRs. We train all models with mixed precision, the Adam optimizer \cite{kingma2014adam}, the learning rate of 1e-5 on MIND, 1e-6 on Adressa, and 1e-6 for the sentiment\,\texttt{A-Module} on both datasets.
%%
In \texttt{A-Module} training, we sample 20 instances per class,\footnote{For $M$ class instances, we obtain $\frac{M^2-M}{2}$ positive pairs for that class for SCL. 
% Preliminary experiments with larger $M$ did not yield improvements; we thus kept $M=20$ to reduce the computational footprint of our experiments.
} while in \texttt{CR-Module} training we sample four negatives per positive example. We find the optimal temperature of 0.36 on MIND, and 0.14 on Adressa, for the \texttt{CR-Module}, and of 0.9 for all \texttt{A-Modules} on both datasets.
%
We train all baselines and the \texttt{CR-Module} for 5 epochs on MIND and 20 epochs on Adressa, with a batch size of 8. We train each \texttt{A-Module} for 100 epochs, with the batch size of 60 for sentiment and 360 for topics. We repeat runs five times with different seeds and report averages and standard deviations for all metrics. We refer to Appendices \ref{sec:appendix_model_parameters} - \ref{sec:appendix_hyperparameters} for further details about model sizes and hyperparameters. 


\section{Results and Discussion}
\label{sec:results_discussion}
%
\begin{table*}[ht]
\centering
\scriptsize
% \def\arraystretch{0.9}
%
\resizebox{\textwidth}{!}{%
    \begin{tabular}{l|cccc|cccc}
        \toprule
        \multicolumn{1}{c}{} & \multicolumn{4}{c}{\textbf{MIND}} & \multicolumn{4}{c}{\textbf{Adressa}} \\  
        \cmidrule(lr){2-5} \cmidrule(lr){6-9}
        
        \multicolumn{1}{l|}{Model} 
        & AUC & MRR & nDCG@5 & nDCG@10
        & AUC & MRR & nDCG@5 & nDCG@10
        \\
        \hline

       NRMS-PLM 
        % MIND
        & 63.0\textsubscript{$\pm$1.5} % AUC
        & 35.5\textsubscript{$\pm$0.6}  % MRR
        & 33.4\textsubscript{$\pm$0.7} % nDCG@5
        & 39.9\textsubscript{$\pm$0.6}  % nDCG@10

        % Adressa
        & 72.3\textsubscript{$\pm$3.3} % AUC
        & 43.0\textsubscript{$\pm$2.7} % MRR
        & 44.3\textsubscript{$\pm$2.8} % nDCG@5
        & 51.3\textsubscript{$\pm$2.3} % nDCG@10
        \\

        MINER
        % MIND
        & 63.1\textsubscript{$\pm$1.2} % AUC
        & 35.5\textsubscript{$\pm$1.1}  % MRR
        & 33.7\textsubscript{$\pm$1.1} % nDCG@5
        & 40.0\textsubscript{$\pm$1.0}  % nDCG@10

        % Adressa
        & 70.1\textsubscript{$\pm$4.9} % AUC
        & 37.3\textsubscript{$\pm$4.1} % MRR
        & 38.5\textsubscript{$\pm$5.1} % nDCG@5
        & 46.3\textsubscript{$\pm$4.1} % nDCG@10
        \\

        \hdashline

        NAML-PLM
        % MIND
        & 60.6\textsubscript{$\pm$3.4} % AUC
        &  \underline{37.6\textsubscript{$\pm$0.4}}  % MRR
        &  \underline{35.9\textsubscript{$\pm$0.4}} % nDCG@5
        &  \underline{42.2\textsubscript{$\pm$0.4}}  % nDCG@10

        % Adressa
        & 50.0\textsubscript{$\pm$0.0} % AUC
        & \underline{45.0\textsubscript{$\pm$5.0}} % MRR
        & 47.2\textsubscript{$\pm$5.5} % nDCG@5
        & 52.5\textsubscript{$\pm$4.1} % nDCG@10
        \\

        LSTUR-PLM
         % MIND
        & 54.6\textsubscript{$\pm$3.0} % AUC
        & 33.3\textsubscript{$\pm$1.5}  % MRR
        & 31.7\textsubscript{$\pm$1.8} % nDCG@5
        & 38.3\textsubscript{$\pm$1.7}  % nDCG@10

        % Adressa
        & 65.0\textsubscript{$\pm$7.2} % AUC
        & 43.1\textsubscript{$\pm$1.7} % MRR
        & 44.8\textsubscript{$\pm$2.6} % nDCG@5
        & 51.2\textsubscript{$\pm$2.0} % nDCG@10
        \\

        MINS-PLM 
        % MIND
        & 61.3\textsubscript{$\pm$2.7} % AUC
        & 36.2\textsubscript{$\pm$0.3}  % MRR
        & 34.5\textsubscript{$\pm$0.4} % nDCG@5
        & 40.8\textsubscript{$\pm$0.3}  % nDCG@10

        % Adressa
        & 74.3\textsubscript{$\pm$3.2} % AUC
        & 44.2\textsubscript{$\pm$2.9} % MRR
        & \underline{47.3\textsubscript{$\pm$3.3}} % nDCG@5
        & \underline{53.0\textsubscript{$\pm$3.4}} % nDCG@10      
        \\ 

        CAUM\textsubscript{no entities}-PLM
        % MIND
        &  \underline{66.2\textsubscript{$\pm$3.0}} % AUC
        & 36.6\textsubscript{$\pm$2.0}  % MRR
        & 34.6\textsubscript{$\pm$2.0} % nDCG@5
        & 41.0\textsubscript{$\pm$1.9}  % nDCG@10

        % Adressa
        & \underline{76.5\textsubscript{$\pm$1.2}} % AUC
        & 43.6\textsubscript{$\pm$1.3} % MRR
        & 46.9\textsubscript{$\pm$1.3} % nDCG@5
        & 52.0\textsubscript{$\pm$1.2} % nDCG@10
        \\

        CAUM-PLM 
        % MIND
        & 66.4\textsubscript{$\pm$3.1} % AUC
        & 36.2\textsubscript{$\pm$1.2}  % MRR
        & 34.3\textsubscript{$\pm$1.3} % nDCG@5
        & 40.8\textsubscript{$\pm$1.3}  % nDCG@10

        % Adressa
        & -- % AUC
        & -- % MRR
        & -- % nDCG@5
        & -- % nDCG@10
        \\

        TANR-PLM
         % MIND
        & 63.3\textsubscript{$\pm$1.1} % AUC
        & 37.0\textsubscript{$\pm$1.0}  % MRR
        & 35.2\textsubscript{$\pm$1.0} % nDCG@5
        & 41.6\textsubscript{$\pm$0.9}  % nDCG@10

        % Adressa
        & 50.0\textsubscript{$\pm$0.0} % AUC
        & 43.8\textsubscript{$\pm$1.0} % MRR
        & 45.6\textsubscript{$\pm$1.3} % nDCG@5
        & 51.4\textsubscript{$\pm$0.6} % nDCG@10
        \\
        
        \hdashline

        SentiRec-PLM
         % MIND
        & 62.2\textsubscript{$\pm$0.7} % AUC
        & 35.7\textsubscript{$\pm$0.4}  % MRR
        & 33.9\textsubscript{$\pm$0.4} % nDCG@5
        & 40.5\textsubscript{$\pm$0.4}  % nDCG@10

        % Adressa
        & 67.6\textsubscript{$\pm$2.7} % AUC
        & 33.1\textsubscript{$\pm$2.4} % MRR
        & 32.9\textsubscript{$\pm$3.8} % nDCG@5
        & 40.8\textsubscript{$\pm$2.4} % nDCG@10
        \\

        SentiDebias-PLM
        % MIND
        & 55.0\textsubscript{$\pm$2.5} % AUC
        & 27.8\textsubscript{$\pm$1.9}  % MRR
        & 25.5\textsubscript{$\pm$1.9} % nDCG@5
        & 32.2\textsubscript{$\pm$2.0}  % nDCG@10

        % Adressa
        & 67.4\textsubscript{$\pm$2.4} % AUC
        & 35.7\textsubscript{$\pm$3.4}  % MRR
        & 36.4\textsubscript{$\pm$4.2} % nDCG@5
        & 44.2\textsubscript{$\pm$2.9}  % nDCG@10 
        \\ 
        \hline
        
        \manner{} (\texttt{CR-Module})
        % MIND
        & \textbf{69.7\textsubscript{$\pm$0.9 }} % AUC
        & \textbf{38.6\textsubscript{$\pm$0.6}}  % MRR
        & \textbf{37.0\textsubscript{$\pm$0.6}} % nDCG@5
        & \textbf{43.2\textsubscript{$\pm$0.6}}  % nDCG@10

        % Adressa
        & \textbf{79.4\textsubscript{$\pm$1.7}} % AUC
        & \textbf{47.0\textsubscript{$\pm$2.4}} % MRR
        & \textbf{48.9\textsubscript{$\pm$2.8}} % nDCG@5
        & \textbf{54.3\textsubscript{$\pm$2.5}} % nDCG@10  
        \\
        \hline

        \rowcolor{Gray}
        Improvement (\%)
        % MIND
        & + 5.4% AUC
        & + 2.8% MRR
        & + 3.1% nDCG@5
        & + 2.3% nDCG@10

         % Adressa
        & + 3.7% AUC
        & + 4.6% MRR
        & + 3.3% nDCG@5
        & + 2.5% nDCG@10
        \\ 
        
        \bottomrule
        
    \end{tabular}%
    }
\caption{Content-based recommendation performance. We average results across five runs, and report the relative improvement over the best baseline. The best results per column are highlighted in bold, the second best underlined.}
\label{tab:results_recommendation}
%

\vspace{-0.5em}
\end{table*}

%

We first discuss \manner{}'s content personalization performance. We then analyze its capability for single- and multi-aspect (i) diversification and (ii) personalization. In the aspect customization setups, we treat \manner{}'s \texttt{CR-Module} as a baseline. Lastly, we evaluate its ability to re-use pretrained aspect-specific modules in cross-lingual transfer. 

\subsection{Content Personalization}
\label{sec:content_personalization}

Table \ref{tab:results_recommendation} summarizes the results on content personalization. Since the task does not require any aspect-based customization, we evaluate the \manner{} variant that uses only its CR-Module at inference time (i.e., $\lambda \hspace{-0.2em} = \hspace{-0.2em} 0$).
%%
\manner{} consistently outperforms all state-of-the-art NNRs in terms of both classification and ranking metrics on both datasets. Given that \manner{}'s \texttt{CR-Module} derives the user embedding by merely averaging clicked news embeddings, these results question the need for complex parameterized UEs, present in all the baselines, in line with the findings of \citet{iana2023simplifying}. 

% \vspace{1.4mm}
% \noindent\textbf{Ablations.}
% 
We ablate \texttt{CR-Module}'s content personalization performance for (i) different inputs to the NE and (ii) alternative architecture designs and training objectives.
%
We find that all groups of features (e.g., abstract, named entities) contribute to the overall performance (cf. Fig. \ref{fig:ablation_features}). 
% 
Moreover, we confirm the findings of \citet{iana2023simplifying} that (i) late fusion outperforms a parameterized UE (i.e., early fusion), and that (ii) SCL better separates classes than cross-entropy loss, in line with other similarity-oriented NLP tasks  \cite{reimers2019sentence}.

\subsection{Single-Aspect Customization}
%%
\newcolumntype{g}{>{\columncolor{Gray}}c}

\begin{table*}[ht]
\centering

\resizebox{\textwidth}{!}{%
    \begin{tabular}{l|cgcgcc|cgcgcc}
        \toprule
        \multicolumn{1}{c}{} & \multicolumn{6}{c}{\textbf{MIND}} & \multicolumn{6}{c}{\textbf{Adressa}} \\  
        \cmidrule(lr){2-7} \cmidrule(lr){8-13}
        
        \multicolumn{1}{l|}{Model} 
        & nDCG@10 
        & D\textsubscript{ctg}@10 & T\textsubscript{ctg}@10 
        & D\textsubscript{snt}@10 & T\textsubscript{snt}@10 
        & T\textsubscript{all}@10
        
        & nDCG@10 
        & D\textsubscript{ctg}@10 & T\textsubscript{ctg}@10 
        & D\textsubscript{snt}@10 & T\textsubscript{snt}@10 
        & T\textsubscript{all}@10
        \\
        \hline

        NAML-PLM
        % MIND
        & 41.5$\pm$0.2 % nDCG@10
        & 47.4$\pm$0.7  % div_c@10
        & 44.3$\pm$0.3  % tradeoff_ctg
        & 65.6$\pm$0.4  % div_s@10
        & 50.8$\pm$0.2  % tradeoff_snt
        & 49.7$\pm$0.2  % tradeoff_all

        % Adressa
        & 50.2$\pm$1.9  % nDCG@10
        & 31.9$\pm$2.3  % div_c@10
        & 39.0$\pm$2.1  % tradeoff_ctg
        & 61.5$\pm$0.8  % div_s@10
        & 55.3$\pm$1.5  % tradeoff_snt
        & 44.4$\pm$1.9  % tradeoff_all        
        \\
        
       NRMS-PLM 
       % MIND
        & 38.4$\pm$2.9 % nDCG@10
        & 50.2$\pm$2.4  % div_c@10
        & 43.4$\pm$1.2  % tradeoff_ctg
        & 66.6$\pm$0.9  % div_s@10
        & 48.7$\pm$2.2  % tradeoff_snt
        & 49.1$\pm$0.9  % tradeoff_all

        % Adressa
        & 51.2$\pm$1.6  % nDCG@10
        & 32.4$\pm$0.5  % div_c@10
        & 39.7$\pm$0.5  % tradeoff_ctg
        & 62.1$\pm$0.3  % div_s@10
        & 56.1$\pm$1.0  % tradeoff_snt
        & 45.1$\pm$0.5  % tradeoff_all     
        \\

        LSTUR\textsubscript{ini}-PLM
       % MIND
        & 39.0$\pm$2.4 % nDCG@10
        & 48.4$\pm$2.2  % div_c@10
        & 43.1$\pm$0.7  % tradeoff_ctg
        & 65.6$\pm$0.5  % div_s@10
        & 48.9$\pm$1.8  % tradeoff_snt
        & 48.7$\pm$0.6  % tradeoff_all

        % Adressa
        & 51.3$\pm$2.4  % nDCG@10
        & 27.6$\pm$5.7  % div_c@10
        & 35.6$\pm$5.3  % tradeoff_ctg
        & 60.8$\pm$0.8  % div_s@10
        & 55.6$\pm$1.5  % tradeoff_snt
        & 41.2$\pm$5.0  % tradeoff_all     
        \\

        LSTUR\textsubscript{con}-PLM
        % MIND
        & 31.8$\pm$1.2 % nDCG@10
        & 43.1$\pm$2.0  % div_c@10
        & 36.6$\pm$1.5  % tradeoff_ctg
        & 66.7$\pm$0.4  % div_s@10
        & 43.0$\pm$1.1  % tradeoff_snt
        & 43.0$\pm$1.4  % tradeoff_all

        % Adressa
        & 46.4$\pm$2.9  % nDCG@10
        & 29.0$\pm$1.3  % div_c@10
        & 35.7$\pm$1.8  % tradeoff_ctg
        & 61.1$\pm$0.9  % div_s@10
        & 52.7$\pm$2.2  % tradeoff_snt
        & 41.4$\pm$1.7  % tradeoff_all     
        \\

        MINS-PLM
        % MIND
        & 40.2$\pm$1.1 % nDCG@10
        & 48.7$\pm$1.0  % div_c@10
        & 44.0$\pm$0.4  % tradeoff_ctg
        & 66.7$\pm$0.7  % div_s@10
        & 50.1$\pm$0.8  % tradeoff_snt
        & 49.6$\pm$0.4  % tradeoff_all

        % Adressa
        & 52.6$\pm$1.7  % nDCG@10
        & 34.4$\pm$1.5  % div_c@10
        & \textbf{41.6$\pm$1.5}  % tradeoff_ctg
        & 62.0$\pm$0.3  % div_s@10
        & 56.9$\pm$1.1  % tradeoff_snt
        & \textbf{46.7$\pm$1.2}  % tradeoff_all     
        \\

        CAUM\textsubscript{no entities}-PLM
        % MIND
        & 42.1$\pm$0.9 % nDCG@10
        & 47.8$\pm$0.6  % div_c@10
        & 44.8$\pm$0.6  % tradeoff_ctg
        & 65.8$\pm$0.8  % div_s@10
        & 51.3$\pm$0.6  % tradeoff_snt
        & 50.1$\pm$0.5  % tradeoff_all

        % Adressa
        & 49.5$\pm$3.4  % nDCG@10
        & \underline{35.5$\pm$1.0}  % div_c@10
        & \underline{41.3$\pm$1.5}  % tradeoff_ctg
        & 62.0$\pm$0.2  % div_s@10
        & 55.0$\pm$2.2  % tradeoff_snt
        & \underline{46.5$\pm$1.3}  % tradeoff_all 
        \\

        CAUM-PLM
        % MIND
        & 41.7$\pm$1.5 % nDCG@10
        & 47.8$\pm$1.0  % div_c@10
        & 44.5$\pm$0.8  % tradeoff_ctg
        & 66.1$\pm$0.4  % div_s@10
        & 51.1$\pm$1.1  % tradeoff_snt
        & 50.0$\pm$0.7  % tradeoff_all

        % Adressa
        & --  % nDCG@10
        & --  % div_c@10
        & --  % tradeoff_ctg
        & --  % div_s@10
        & --  % tradeoff_snt
        & --  % tradeoff_all
        \\

        MINER\textsubscript{max}
        % MIND
        & 39.4$\pm$0.7 % nDCG@10
        & 47.5$\pm$0.9 % div_c@10
        & 43.1$\pm$0.7 % tradeoff_ctg
        & 63.7$\pm$1.3 % div_s@10
        & 48.7$\pm$0.8 % tradeoff_snt
        & 48.3$\pm$0.8 % tradeoff_all

        % Adressa
        & 51.5$\pm$2.6  % nDCG@10
        & 32.6$\pm$0.9  % div_c@10
        & 39.9$\pm$0.8  % tradeoff_ctg
        & 61.6$\pm$0.3  % div_s@10
        & 56.1$\pm$1.5  % tradeoff_snt
        & 45.2$\pm$0.7  % tradeoff_all     
        \\

        MINER\textsubscript{mean}
        % MIND
        & 40.0$\pm$1.0 % nDCG@10
        & 49.4$\pm$1.2 % div_c@10
        & 44.2$\pm$0.4 % tradeoff_ctg
        & 65.7$\pm$0.9 % div_s@10
        & 4.7$\pm$1.0 % tradeoff_snt
        & 49.6$\pm$0.5 % tradeoff_all

        % Adressa
        & 47.8$\pm$4.0  % nDCG@10
        & 31.0$\pm$1.1  % div_c@10
        & 37.6$\pm$1.8  % tradeoff_ctg
        & 60.9$\pm$0.3  % div_s@10
        & 53.5$\pm$2.5  % tradeoff_snt
        & 43.1$\pm$1.6  % tradeoff_all     
        \\

        MINER\textsubscript{weighted}
        % MIND
        & 34.0$\pm$3.1 % nDCG@10
        & 49.1$\pm$1.6  % div_c@10
        & 40.1$\pm$1.9  % tradeoff_ctg
        & 64.6$\pm$1.1  % div_s@10
        & 44.5$\pm$2.7  % tradeoff_snt
        & 45.9$\pm$1.7  % tradeoff_all

        % Adressa
        & 47.8$\pm$4.9  % nDCG@10
        & 32.5$\pm$0.6  % div_c@10
        & 38.7$\pm$1.8  % tradeoff_ctg
        & 61.0$\pm$0.7  % div_s@10
        & 53.5$\pm$3.3  % tradeoff_snt
        & 44.0$\pm$1.7  % tradeoff_all     
        \\
        \hdashline

        TANR-PLM 
        % MIND
        & 40.2$\pm$1.3 % nDCG@10
        & 49.9$\pm$1.6  % div_c@10
        & 44.5$\pm$0.4  % tradeoff_ctg
        & 66.6$\pm$1.0  % div_s@10
        & 50.2$\pm$0.7  % tradeoff_snt
        & 50.0$\pm$0.3  % tradeoff_all

        % Adressa
        & 50.3$\pm$4.4  % nDCG@10
        & 30.3$\pm$3.2  % div_c@10
        & 37.6$\pm$1.4  % tradeoff_ctg
        & 61.0$\pm$1.5  % div_s@10
        & 55.0$\pm$2.2  % tradeoff_snt
        & 43.1$\pm$1.5  % tradeoff_all     
        \\

        SentiRec-PLM
        % MIND
        & 39.8$\pm$0.7 % nDCG@10
        & 50.4$\pm$0.4  % div_c@10
        & 44.5$\pm$0.6  % tradeoff_ctg
        & 66.8$\pm$0.7  % div_s@10
        & 49.8$\pm$0.5  % tradeoff_snt
        & 50.0$\pm$0.5  % tradeoff_all

        % Adressa
        & 38.9$\pm$1.1  % nDCG@10
        & \textbf{35.7$\pm$1.0}  % div_c@10
        & 37.2$\pm$0.4  % tradeoff_ctg
        & \textbf{68.5$\pm$1.0}  % div_s@10
        & 49.6$\pm$1.1  % tradeoff_snt
        & 43.9$\pm$0.4  % tradeoff_all     
        \\ 

        SentiDebias-PLM
        % MIND
        & 34.7$\pm$2.2 % nDCG@10
        & \underline{50.5$\pm$1.8}  % div_c@10
        & 41.1$\pm$1.3  % tradeoff_ctg
        & \underline{68.0$\pm$0.9}  % div_s@10
        & 45.9$\pm$1.9  % tradeoff_snt
        & 47.3$\pm$1.1  % tradeoff_all

        % Adressa
        & 52.0$\pm$2.7 % nDCG@10
        & 32.7$\pm$0.6  % div_c@10
        & 40.2$\pm$1.1  % tradeoff_ctg
        & 62.4$\pm$0.4  % div_s@10
        & 56.7$\pm$1.1  % tradeoff_snt
        & 45.6$\pm$1.0  % tradeoff_all 
        \\
        \hline

       \mannertitle{} (\texttt{CR-Module})
         % MIND
        & 42.4$\pm$0.4 % nDCG@10
        & 48.3$\pm$0.3 % div_c@10
        & 45.2$\pm$0.3 % tradeoff_ctg
        & 65.1$\pm$0.5 % div_s@10
        & 51.4$\pm$0.4 % tradeoff_snt
        & 50.3$\pm$0.4 % tradeoff_all

        % Adressa
        & \textbf{54.3$\pm$2.5}  % nDCG@10
        & 31.7$\pm$0.2  % div_c@10
        & 40.0$\pm$0.7  % tradeoff_ctg
        & 61.4$\pm$0.3  % div_s@10
        & \underline{57.6$\pm$1.5}  % tradeoff_snt
        & 45.3$\pm$0.6  % tradeoff_all     
        \\ 

        \mannertitle{} ($\lambda_{ctg}=-0.3$, $\lambda_{snt}=0$)
        % MIND
        & 41.2$\pm$0.5 % nDCG@10
        & 49.8$\pm$0.5  % div_c@10
        & 45.1$\pm$0.4  % tradeoff_ctg
        & 65.4$\pm$0.6  % div_s@10
        & 50.6$\pm$0.5  % tradeoff_snt
        & 50.3$\pm$0.4  % tradeoff_all

        % Adressa
        & 50.9$\pm$2.5  % nDCG@10
        & 34.1$\pm$0.3  % div_c@10
        & 40.8$\pm$0.8  % tradeoff_ctg
        & 61.9$\pm$0.3  % div_s@10
        & 55.8$\pm$1.6  % tradeoff_snt
        & 46.0$\pm$0.7  % tradeoff_all     
        \\ 

        \mannertitle{} ($\lambda_{ctg}=0$, $\lambda_{snt}=-0.2$)
        % MIND
        & 41.8$\pm$0.3 % nDCG@10
        & 48.8$\pm$0.5  % div_c@10
        & 45.0$\pm$0.4  % tradeoff_ctg
        & 65.4$\pm$0.5  % div_s@10
        & 51.0$\pm$0.4  % tradeoff_snt
        & 50.3$\pm$0.4  % tradeoff_all

        % Adressa
        & \underline{53.8$\pm$2.5}  % nDCG@10
        & 32.4$\pm$0.2  % div_c@10
        & 40.4$\pm$0.7  % tradeoff_ctg
        & \underline{63.0$\pm$0.3}  % div_s@10
        & \textbf{58.0$\pm$1.5}  % tradeoff_snt
        & 45.9$\pm$0.6  % tradeoff_all     
        \\ 
        \hdashline

        \manner{} (\texttt{CR-Module})
         % MIND
        & \textbf{43.2$\pm$0.6} % nDCG@10
        & 49.3$\pm$0.3  % div_c@10
        & 46.0$\pm$0.3  % tradeoff_ctg
        & 65.4$\pm$0.6  % div_s@10
        & \underline{52.0$\pm$0.4}  % tradeoff_snt
        & 51.1$\pm$0.2  % tradeoff_all

        % Adressa
        & --  % nDCG@10
        & --  % div_c@10
        & --  % tradeoff_ctg
        & --  % div_s@10
        & --  % tradeoff_snt
        & --  % tradeoff_all     
        \\ 

        \manner{} ($\lambda_{ctg}=-0.2$, $\lambda_{snt}=0$)
        % MIND
        & 42.0$\pm$0.6 % nDCG@10
        & \textbf{51.5$\pm$0.3}  % div_c@10
        & \textbf{46.2$\pm$0.3}  % tradeoff_ctg
        & 65.6$\pm$0.6  % div_s@10
        & 51.2$\pm$0.4  % tradeoff_snt
        & \underline{51.3$\pm$0.3}  % tradeoff_all

        % Adressa
        & --  % nDCG@10
        & --  % div_c@10
        & --  % tradeoff_ctg
        & --  % div_s@10
        & --  % tradeoff_snt
        & --  % tradeoff_all     
        \\ 

        \manner{} ($\lambda_{ctg}=0$, $\lambda_{snt}=-0.2$)
        % MIND
        & \underline{43.1$\pm$0.6} % nDCG@10
        & 49.5$\pm$0.3  % div_c@10
        & \underline{46.1$\pm$0.3}  % tradeoff_ctg
        & \textbf{68.1$\pm$0.4}  % div_s@10
        & \textbf{52.8$\pm$0.5}  % tradeoff_snt
        & \textbf{51.6$\pm$0.2}  % tradeoff_all

        % Adressa
        & --  % nDCG@10
        & --  % div_c@10
        & --  % tradeoff_ctg
        & --  % div_s@10
        & --  % tradeoff_snt
        & --  % tradeoff_all      
        \\ 
        
        \bottomrule
        
    \end{tabular}%
    }
\caption{Diversity (and content personalization) performance on diversification tasks. For \manner{}, we list the best results (in terms of T\textsubscript{A\textsubscript{p}}) of single-aspect diversification ($\lambda_{ctg}=-0.2$ and $\lambda_{snt}=0$ for topical category diversification; $\lambda_{ctg}=0$ and $\lambda_{snt}=-0.2$ for sentiment diversification) on MIND. For \mannertitle{}, we list the best results (in terms of T\textsubscript{A\textsubscript{p}}) of single-aspect diversification ($\lambda_{ctg}=-0.3$ and $\lambda_{snt}=0$ for topical category diversification; $\lambda_{ctg}=0$ and $\lambda_{snt}=-0.2$ for sentiment diversification) on Adressa. The best results per column are highlighted in bold, the second best are underlined.
}
\label{tab:results_diversification}

\end{table*}


%
% Figure environment removed



\vspace{1.4mm}
\noindent\textbf{Diversification.}
Table~\ref{tab:results_diversification} summarizes the results on aspect diversification tasks.
%%%
Most baselines (including \manner{}'s \texttt{CR-Module} without aspect diversification) obtain similar diversification scores (D\textsubscript{ctg} and D\textsubscript{snt}). 
%
The sentiment-aware SentiRec-PLM, with an explicit auxiliary sentiment diversification objective, yields the highest sentiment diversity on Adressa; this comes at the expense of content personalization quality (lowest nDCG).
%
On MIND, the sentiment-specific SentiDebias-PLM achieves the highest sentiment diversity, but also exhibits lower content personalization performance. Overall, these results point to a trade-off between content personalization and aspectual diversity: models with higher D\textsubscript{A\textsubscript{p}} tend to have a lower nDCG. 

Unlike all other models, \manner{} can trade content personalization for diversity (and vice-versa) with different values of the aspect coefficients $\lambda_{A_p}$. Fig. \ref{fig:single_aspect_div_mind} illustrates its performance in single-aspect diversification tasks for different values of $\lambda$\textsubscript{ctg} and $\lambda$\textsubscript{snt} on MIND.
%%%
The steady drop in nDCG together with the steady increase in D\textsubscript{A\textsubscript{p}} indeed indicate the existence of a trade-off between content personalization and aspect diversification. For topical categories we observe a steeper decline in content personalization quality with improved diversification than for sentiment. Sentiment diversity reaches peak performance for $\lambda_{snt} \hspace{-0.2em} = \hspace{-0.2em} -0.4$, whereas category diversity continues to increase all the way to $\lambda_{ctg} \hspace{-0.2em} = \hspace{-0.2em} -0.9$. Intuitively, content-based recommendation is more aligned with the topical than with the sentiment consistency of recommendations. The best trade-off (i.e., maximal performance w.r.t. T\textsubscript{A\textsubscript{p}}@10) is achieved for $\lambda_{ctg}\hspace{-0.2em} = \hspace{-0.2em} -0.2$ for topics, and $\lambda_{snt} \hspace{-0.2em} = \hspace{-0.2em} -0.3$ for sentiment. We report analogous results on Adressa in Appendix \ref{sec:appendix_single_apsect_customization}.
%
We attribute these effects to the representation spaces of the \texttt{A-Modules}. Fig. \ref{fig:tsne_categ_mind} shows the 2-dimensional t-SNE visualizations \cite{van2008visualizing} of the news embeddings produced with category-specialized encoders trained on MIND (see Fig. \ref{fig:tsne_sent_mind} for sentiment). The results confirm that the latent representation space of the encoder was reshaped to group same-class instances. The separation of classes, however, is less prominent for representation spaces of the encoders trained on Adressa (cf. Fig. \ref{fig:tsne_embeddings_adressa}) than for those learned on MIND (e.g., the effect is stronger on the category-shaped embedding space).\footnote{We believe that this is because Adressa has 10 times fewer news than MIND (and contrastive learning, naturally, benefits from more news pairs), with over half of the topical categories in Adressa being represented with fewer than 100 examples.} 
%
\newcolumntype{g}{>{\columncolor{Gray}}c}

\begin{table*}[ht]
\centering

\resizebox{\textwidth}{!}{%
    \begin{tabular}{l|cgcgcc|cgcgcc}
        \toprule
        \multicolumn{1}{c}{} & \multicolumn{6}{c}{\textbf{MIND}} & \multicolumn{6}{c}{\textbf{Adreesa}} \\  
        \cmidrule(lr){2-7} \cmidrule(lr){8-13}
        
        \multicolumn{1}{l|}{Model} 
        & nDCG@10 
        & PS\textsubscript{ctg}@10 & T\textsubscript{ctg}@10 
        & PS\textsubscript{snt}@10 & T\textsubscript{snt}@10 
        & T\textsubscript{all}@10
        
        & nDCG@10 
        & PS\textsubscript{ctg}@10 & T\textsubscript{ctg}@10 
        & PS\textsubscript{snt}@10 & T\textsubscript{snt}@10 
        & T\textsubscript{all}@10
        \\
        \hline

        NAML-PLM
        % MIND
        & 41.5$\pm$0.2 % nDCG@10
        & \underline{25.6$\pm$0.3} % pers_c@10
        & \underline{31.7$\pm$0.3} % tradeoff_ctg
        & 35.0$\pm$0.2 % pers_s@10
        & 38.0$\pm$0.2 % tradeoff_snt
        & \underline{32.7$\pm$0.2} % tradeoff_all

        % Adreesa
        & 50.2$\pm$1.9 % nDCG@10
        & 35.6$\pm$0.9 % pers_c@10
        & 41.6$\pm$0.6 % tradeoff_ctg
        & 41.8$\pm$0.1 % pers_s@10
        & 45.6$\pm$0.8 % tradeoff_snt
        & 41.7$\pm$0.4 % tradeoff_all
        \\
        
       NRMS-PLM 
        % MIND
        & 38.4$\pm$2.9 % nDCG@10
        & 23.2$\pm$1.1 % pers_c@10
        & 28.9$\pm$1.7 % tradeoff_ctg
        & 35.0$\pm$0.1 % pers_s@10
        & 28.4$\pm$1.5 % tradeoff_snt
        & 30.7$\pm$1.3 % tradeoff_all

        % Adreesa
        & 51.2$\pm$1.6 % nDCG@10
        & 34.2$\pm$03 % pers_c@10
        & 41.0$\pm$0.6 % tradeoff_ctg
        & 41.8$\pm$0.0 % pers_s@10
        & 46.0$\pm$0.6 % tradeoff_snt
        & 41.2$\pm$0.4 % tradeoff_all
        \\

        LSTUR\textsubscript{ini}-PLM
        % MIND
        & 39.0$\pm$2.4 % nDCG@10
        & 24.1$\pm$0.6 % pers_c@10
        & 29.8$\pm$1.1 % tradeoff_ctg
        & 34.9$\pm$0.4 % pers_s@10
        & 36.8$\pm$1.2 % tradeoff_snt
        & 31.3$\pm$0.9 % tradeoff_all

        % Adreesa
        & 51.3$\pm$2.4 % nDCG@10
        & \underline{36.2$\pm$2.3} % pers_c@10
        & \underline{42.4$\pm$1.8} % tradeoff_ctg
        & 41.9$\pm$0.1 % pers_s@10
        & 46.1$\pm$1.0 % tradeoff_snt
        & 42.2$\pm$1.2 % tradeoff_all
        \\

        LSTUR\textsubscript{con}-PLM
        % MIND
        & 31.8$\pm$1.2 % nDCG@10
        & 18.7$\pm$0.9 % pers_c@10
        & 23.5$\pm$1.1 % tradeoff_ctg
        & 33.9$\pm$0.0 % pers_s@10
        & 32.8$\pm$0.7 % tradeoff_snt
        & 26.2$\pm$0.9 % tradeoff_all

        % Adreesa
        & 46.4$\pm$2.9 % nDCG@10
        & 33.3$\pm$0.2 % pers_c@10
        & 38.8$\pm$1.0 % tradeoff_ctg
        & 41.8$\pm$0.1 % pers_s@10
        & 43.9$\pm$1.3 % tradeoff_snt
        & 39.7$\pm$0.7 % tradeoff_all
        \\

        MINS-PLM
        % MIND
        & 40.2$\pm$1.1 % nDCG@10
        & 24.9$\pm$0.8 % pers_c@10
        & 30.7$\pm$0.8 % tradeoff_ctg
        & 34.8$\pm$0.3 % pers_s@10
        & 37.3$\pm$0.5 % tradeoff_snt
        & 32.0$\pm$0.7 % tradeoff_all

        % Adreesa
        & 52.6$\pm$1.7 % nDCG@10
        & 33.4$\pm$0.7 % pers_c@10
        & 40.9$\pm$0.5 % tradeoff_ctg
        & 41.8$\pm$0.1 % pers_s@10
        & 46.5$\pm$0.6 % tradeoff_snt
        & 41.2$\pm$0.3 % tradeoff_all
        \\

        CAUM\textsubscript{no entities}-PLM
        % MIND
        & 42.1$\pm$0.9 % nDCG@10
        & 25.2$\pm$0.2 % pers_c@10
        & 31.5$\pm$0.4 % tradeoff_ctg
        & 35.1$\pm$0.2 % pers_s@10
        & 38.3$\pm$0.4 % tradeoff_snt
        & 32.6$\pm$0.3 % tradeoff_all

        % Adreesa
        & 49.5$\pm$3.4 % nDCG@10
        & 33.1$\pm$0.5 % pers_c@10
        & 39.6$\pm$1.1 % tradeoff_ctg
        & 41.8$\pm$0.1 % pers_s@10
        & 45.2$\pm$1.4 % tradeoff_snt
        & 40.3$\pm$0.8 % tradeoff_all
        \\

        CAUM-PLM 
        % MIND
        & 41.7$\pm$1.5 % nDCG@10
        & 25.1$\pm$0.5 % pers_c@10
        & 31.3$\pm$0.7 % tradeoff_ctg
        & 35.0$\pm$0.1 % pers_s@10
        & 38.1$\pm$0.7 % tradeoff_snt
        & 32.5$\pm$0.5 % tradeoff_all

        % Adreesa
        & -- % nDCG@10
        & -- % pers_c@10
        & -- % tradeoff_ctg
        & -- % pers_s@10
        & -- % tradeoff_snt
        & -- % tradeoff_all
        \\

        MINER\textsubscript{max}
        % MIND
        & 39.4$\pm$0.7 % nDCG@10
        & 24.3$\pm$0.4 % pers_c@10
        & 30.1$\pm$0.5 % tradeoff_ctg
        & 35.2$\pm$0.2 % pers_s@10
        & 37.2$\pm$0.4 % tradeoff_snt
        & 31.6$\pm$0.3 % tradeoff_all

        % Adreesa
        & 51.5$\pm$2.6 % nDCG@10
        & 34.0$\pm$0.3 % pers_c@10
        & 41.0$\pm$0.9 % tradeoff_ctg
        & 41.9$\pm$0.0 % pers_s@10
        & 46.2$\pm$1.0 % tradeoff_snt
        & 41.3$\pm$0.6 % tradeoff_all
        \\

        MINER\textsubscript{mean}
        % MIND
        & 40.0$\pm$1.0 % nDCG@10
        & 23.9$\pm$0.4 % pers_c@10
        & 29.9$\pm$0.5 % tradeoff_ctg
        & 35.0$\pm$0.2 % pers_s@10
        & 37.3$\pm$0.4 % tradeoff_snt
        & 31.5$\pm$0.4 % tradeoff_all

        % Adreesa
        & 47.8$\pm$4.0 % nDCG@10
        & 34.5$\pm$0.2 % pers_c@10
        & 40.0$\pm$1.3 % tradeoff_ctg
        & 42.0$\pm$0.0 % pers_s@10
        & 44.7$\pm$1.7 % tradeoff_snt
        & 40.6$\pm$0.9 % tradeoff_all
        \\

        MINER\textsubscript{weighted}
        % MIND
        & 34.0$\pm$3.1 % nDCG@10
        & 22.3$\pm$0.7 % pers_c@10
        & 26.9$\pm$1.4 % tradeoff_ctg
        & 35.0$\pm$0.3 % pers_s@10
        & 34.5$\pm$1.6 % tradeoff_snt
        & 29.2$\pm$1.2 % tradeoff_all

        % Adreesa
        & 47.8$\pm$4.9 % nDCG@10
        & 33.7$\pm$0.2 % pers_c@10
        & 39.4$\pm$1.8 % tradeoff_ctg
        & 41.9$\pm$0.1 % pers_s@10
        & 44.6$\pm$2.1 % tradeoff_snt
        & 40.2$\pm$1.2 % tradeoff_all
        \\
        \hdashline

        TANR-PLM 
        % MIND
        & 40.2$\pm$1.3 % nDCG@10
        & 24.6$\pm$0.4 % pers_c@10
        & 30.6$\pm$0.6 % tradeoff_ctg
        & 35.0$\pm$0.3 % pers_s@10
        & 37.5$\pm$0.6 % tradeoff_snt
        & 31.9$\pm$0.5 % tradeoff_all

        % Adreesa
        & 50.3$\pm$4.4 % nDCG@10
        & 35.0$\pm$1.4 % pers_c@10
        & 41.3$\pm$2.4 % tradeoff_ctg
        & 41.9$\pm$0.2 % pers_s@10
        & 45.6$\pm$1.9 % tradeoff_snt
        & 41.4$\pm$1.7 % tradeoff_all
        \\

        SentiRec-PLM
        % MIND
        & 39.8$\pm$0.7 % nDCG@10
        & 23.8$\pm$0.3 % pers_c@10
        & 29.7$\pm$0.4 % tradeoff_ctg
        & 33.6$\pm$0.5 % pers_s@10
        & 36.5$\pm$0.6 % tradeoff_snt
        & 30.9$\pm$0.4 % tradeoff_all

        % Adreesa
        & 38.9$\pm$1.1 % nDCG@10
        & 32.2$\pm$0.4 % pers_c@10
        & 35.2$\pm$0.7 % tradeoff_ctg
        & 38.8$\pm$0.2 % pers_s@10
        & 38.8$\pm$0.5 % tradeoff_snt
        & 36.3$\pm$0.4 % tradeoff_all
        \\ 
        
        SentiDebias-PLM
        % MIND
        & 34.7$\pm$2.2 % nDCG@10
        & 21.8$\pm$1.2 % pers_c@10
        & 26.8$\pm$1.6 % tradeoff_ctg
        & 34.4$\pm$0.2 % pers_s@10
        & 34.6$\pm$1.2 % tradeoff_snt
        & 28.9$\pm$1.2 % tradeoff_all

        % Adreesa
        & 52.0$\pm$2.7 % nDCG@10
        & 34.1$\pm$0.2 % pers_c@10
        & 41.2$\pm$0.8 % tradeoff_ctg
        & 41.7$\pm$0.1 % pers_s@10
        & 46.3$\pm$1.0 % tradeoff_snt
        & 41.4$\pm$0.5 % tradeoff_all
        \\
        \hline
        
        \mannertitle{} (\texttt{CR-Module})
        % MIND
        & 42.4$\pm$0.4 % nDCG@10
        & 24.8$\pm$0.1 % pers_c@10
        & 31.3$\pm$0.2 % tradeoff_ctg
        & 35.2$\pm$0.1 % pers_s@10
        & 38.5$\pm$0.1 % tradeoff_snt
        & 32.5$\pm$0.1 % tradeoff_all

        % Adreesa
        & \textbf{54.3$\pm$2.5} % nDCG@10
        & 34.5$\pm$0.1 % pers_c@10
        & 42.2$\pm$0.8 % tradeoff_ctg
        & 42.0$\pm$0.1 % pers_s@10
        & \underline{47.3$\pm$0.9} % tradeoff_snt
        & 42.1$\pm$0.5 % tradeoff_all
        \\ 

        
        \mannertitle{} ($\lambda_{ctg}=0.4$, $\lambda_{snt}=0$)
        % MIND
        & 41.9$\pm$0.4 % nDCG@10
        & 25.2$\pm$0.1 % pers_c@10
        & 31.5$\pm$0.1 % tradeoff_ctg
        & 35.1$\pm$0.2 % pers_s@10
        & 38.2$\pm$0.1 % tradeoff_snt
        & 32.6$\pm$0.1 % tradeoff_all

        % Adreesa
        & 53.6$\pm$1.9 % nDCG@10
        & \textbf{36.2$\pm$0.1} % pers_c@10
        & \textbf{43.2$\pm$0.7} % tradeoff_ctg
        & \underline{42.1$\pm$0.1} % pers_s@10
        & 47.2$\pm$0.7 % tradeoff_snt
        & \textbf{42.9$\pm$0.4} % tradeoff_all
        \\ 

        
        \mannertitle{} ($\lambda_{ctg}=0$, $\lambda_{snt}=0.1$)
        % MIND
        & 42.4$\pm$0.4 % nDCG@10
        & 24.9$\pm$0.1 % pers_c@10
        & 31.3$\pm$0.2 % tradeoff_ctg
        & 35.2$\pm$0.2 % pers_s@10
        & 38.5$\pm$0.1 % tradeoff_snt
        & 32.5$\pm$0.1 % tradeoff_all

        % Adreesa
        & \underline{54.1$\pm$2.4} % nDCG@10
        & 34.7$\pm$0.1 % pers_c@10
        & 42.2$\pm$0.8 % tradeoff_ctg
        & \textbf{42.2$\pm$0.1} % pers_s@10
        & \textbf{47.4$\pm$0.9} % tradeoff_snt
        & \underline{42.2$\pm$0.5} % tradeoff_all
        \\ 
        \hdashline
        
        \manner{} (\texttt{CR-Module})
        % MIND
        & \textbf{43.2$\pm$0.6}% nDCG@10
        & 24.7$\pm$0.1 % pers_c@10
        & 31.4$\pm$0.2 % tradeoff_ctg
        & 35.1$\pm$0.1 % pers_s@10
        & \underline{38.7$\pm$0.2} % tradeoff_snt
        & 32.6$\pm$0.2 % tradeoff_all

        % Adreesa
        & -- % nDCG@10
        & -- % pers_c@10
        & -- % tradeoff_ctg
        & -- % pers_s@10
        & -- % tradeoff_snt
        & -- % tradeoff_all
        \\ 

        
        \manner{} ($\lambda_{ctg}=0.7$, $\lambda_{snt}=0$)
        % MIND
        & \underline{42.9$\pm$0.3} % nDCG@10
        & \textbf{27.2$\pm$0.1} % pers_c@10
        & \textbf{33.3$\pm$0.1} % tradeoff_ctg
        & \underline{35.2$\pm$0.0} % pers_s@10
        & 38.7$\pm$0.1 % tradeoff_snt
        & \textbf{33.9$\pm$0.1} % tradeoff_all

        % Adreesa
        & -- % nDCG@10
        & -- % pers_c@10
        & -- % tradeoff_ctg
        & -- % pers_s@10
        & -- % tradeoff_snt
        & -- % tradeoff_all
        \\ 

        
        \manner{} ($\lambda_{ctg}=0$, $\lambda_{snt}=0.2$)
        % MIND
        & 42.8$\pm$0.5 % nDCG@10
        & 24.7$\pm$0.1 % pers_c@10
        & 31.3$\pm$0.2 % tradeoff_ctg
        & \textbf{35.8$\pm$0.1} % pers_s@10
        & \textbf{39.0$\pm$0.2} % tradeoff_snt
        & 32.7$\pm$0.1 % tradeoff_all

        % Adreesa
        & -- % nDCG@10
        & -- % pers_c@10
        & -- % tradeoff_ctg
        & -- % pers_s@10
        & -- % tradeoff_snt
        & -- % tradeoff_all
        \\ 

        \bottomrule
        
    \end{tabular}%
    }

\caption{Aspect (and content) personalization performance on personalization tasks. For \manner{}, we list the best results (in terms of T\textsubscript{A\textsubscript{p}}) of single-aspect customization ($\lambda_{ctg} = 0.7$ and $\lambda_{snt}=0$ for topical category personalization; $\lambda_{ctg} = 0$ and $\lambda_{snt}=0.2$ for sentiment personalization) on MIND. For \mannertitle{}, we list the best results (in terms of T\textsubscript{A\textsubscript{p}}) of single-aspect customization ($\lambda_{ctg} = 0.4$ and $\lambda_{snt}=0$ for topical category personalization; $\lambda_{ctg} = 0$ and $\lambda_{snt}=0.1$ for sentiment personalization) on Adressa. The best results per column are highlighted in bold, the second best are underlined.
}
\label{tab:results_personalization}
\vspace{-1em}
\end{table*}



\vspace{1.4mm}
\noindent\textbf{Personalization.}
Table \ref{tab:results_personalization} displays the results on aspect personalization tasks. TANR, trained with an auxiliary topic classification task, underperforms NAML, which uses topical categories as NE input features, in category personalization on both datasets . 
%
\manner{}'s \texttt{CR-Module} alone (i.e., without any aspect customization) yields competitive category personalization performance. We believe that this is because (i) the \texttt{CR-Module} is best in content personalization and (ii) category personalization is well-aligned with content personalization (i.e., news with similar content tend to belong to the same category). 
%
Fig. \ref{fig:single_aspect_pers_mind} explores the trade-off between content and aspect personalization, for different positive values of $\lambda_{A_p}$ on MIND (see Fig. \ref{fig:single_aspect_pers_adressa} for Adressa). The best topical category personalization (PS\textsubscript{ctg}), obtained for $\lambda_{ctg} \hspace{-0.2em} > \hspace{-0.2em} 0.7$, comes at the small expense of content personalization: too much weight on the category similarity of news dilutes the impact of content relevance. Increased sentiment personalization, however, is much more detrimental to content personalization. Intuitively, users do not choose articles based on sentiment. Tailoring recommendations according to the sentiment of previously clicked news thus leads to more content-irrelevant suggestions. 


\subsection{Multi-Aspect Customization} 
% \vspace{1.4mm}
% \noindent\textbf{Diversification.}
We further explore the trade-off between content personalization and multi-aspect diversification, i.e. diversifying simultaneously over both topical categories and sentiments, for different values of the aspect coefficients $\lambda_{ctg}$ and $\lambda_{snt}$. We achieve the highest T\textsubscript{all} for $\lambda_{ctg} \hspace{-0.2em} = \hspace{-0.2em} -0.2$ and $\lambda_{snt} \hspace{-0.2em} = \hspace{-0.2em} -0.25$ on MIND (cf. Fig. \ref{fig:multi_aspect_div_mind}). In line with results on single-aspect diversification, we observe that improving diversity in terms of topical categories rather than sentiments has a more negative effect on content personalization quality, i.e. steeper decline in T\textsubscript{all}. Overall, these results confirm that \manner{} can generalize to diversify for multiple aspects at once by weighting individual aspect relevance scores less than in the single-aspect task. This can be explained by the fact that weighting several aspects higher at the same time acts as a double discounting for content personalization, diluting content relevance disproportionately.
%
% \vspace{1.4mm}
% \noindent\textbf{Personalization.}
Similarly, for multi-aspect personalization, we achieve the best multi-aspect trade-off on MIND (cf. Fig. \ref{fig:multi_aspect_pers_mind}) for $\lambda_{ctg} \hspace{-0.2em} = \hspace{-0.2em} 0.45$ and $\lambda_{snt} \hspace{-0.2em} = \hspace{-0.2em} 0.25$. Stronger enforcing of alignment of candidate news with user's history is needed for topical categories than for sentiment (i.e., $\lambda_{ctg} \hspace{-0.2em} > \hspace{-0.2em} \lambda_{snt}$). This is because sentiment exhibits low variance within topical categories (e.g., \textit{politics} news are mostly negative) and enforcing categorical personalization thus partly also achieves sentiment personalization. 
%
% Figure environment removed
%                                        

\subsection{Cross-Lingual Transfer}
\label{sec:xlt}

Lastly, we analyze the transferability of \manner{} across datasets and languages in single-aspect customization experiments.\footnote{We evaluate only the title-based version of \manner{}, as the full version cannot be trained on Adressa.} 
%
Concretely, we train the \texttt{CR-Module} and \texttt{A-Modules} on both MIND (i.e., in English) and Adressa (i.e., in Norwegian), respectively. 
At inference, we evaluate all combinations of pretrained \texttt{CR-Module} and \texttt{A-Modules} on the test set of MIND. 
We replace the monolingual PLMs used in \manner{}'s NE with a multilingual DistilBERT Base \cite{sanh2019distilbert} to enable cross-lingual transfer (\texttt{XLT}).
%
Fig. \ref{fig:cross_lingual_transfer_results} summarizes the \texttt{XLT} results for single-aspect diversification. We refer to Appendix \ref{sec:appendix_xlt} for similar results on single-aspect personalization and on Adressa as target-language dataset.
%
As expected, \manner{} trained fully on Adressa suffers a large drop in content personalization performance, compared to the counterpart trained on MIND. 
%
In contrast, transferring only the \texttt{A-Module}, i.e., training the \texttt{CR-Module} on MIND and the \texttt{A-Module} (for topics and sentiment) on Adressa, yields performance comparable to that of complete in-language training (i.e., both \texttt{CR-Module} and \texttt{A-Module} trained on MIND). 
% 
This is particularly the case for the sentiment \texttt{A-Module}, since the sentiment labels between the datasets are more aligned than those for topical categories. 
%
These results indicate that the plug-and-play of \texttt{A-Modules} enables zero-shot \texttt{XLT}, as modules trained on the much smaller Norwegian Adressa transfer well to the large English MIND. 
%
This suggests that, coupled with multilingual PLMs, \manner{} can be used for effective news recommendation in lower-resource languages, where training data and aspectual labels are scarce. Furthermore, the results demonstrate that the \texttt{A-Modules} could be trained on general-purpose classification datasets  (e.g. topic or sentiment classification datasets), 
alleviating the need for aspect-specific annotation of news stories. 
% Figure environment removed
\section{Conclusion and Future Work}
In this work, I design corruption-robust algorithms for the Lipschitz contextual search problem. I present the \emph{agnostic checking} technique and demonstrate its effectiveness in designing corruption-robust algorithms. There are several open problems for future research. First, in the algorithm I propose for pricing loss, the schedule for agnostic checks is fixed upfront. Can the learner design an adaptive checking schedule for the pricing loss? Second, this work assumes the learner has knowledge of the Lipschitz constant $L$. Can the learner design efficient no-regret algorithms without knowledge of $L$? 
\section{Limitations}

Describe limitation of each of the components of the workflow

\paragraph{Text-To-Design}
What are the limits on design primitives?
What are the limits of constraints?
What are the limits on modularity/hierarchical design?

Evaluate validity of the design.
Evaluate correctness.

\paragraph{Text-To-Design-Space}
Limits of parametric design.
Limits of interpolation/extrapolation.
Limits of grammars.

\paragraph{Design-To-Manufacturing}
Limits of validity of the translation.
Limits of correctness.
Limits on the length of the output. Complexity of the output.
Limits on custom manufacturing processes.

\paragraph{Design-To-Performance}
Which performance metrics are understandable?
Which performance metrics need external simulation?

\paragraph{Performance-and-Design-Space To Design}
Which objective function are understandable?
What are the limits on specifying constraints?
What are the limits of the search process for the inverse?


\section*{Ethical Considerations}
\label{sec:ethical_consideations}

We consider several ethical considerations that arise when working with recommender systems and  open benchmark datasets. On the one hand, any biases or misinformation that might exist in the news and user data provided in the public datasets could be propagated through the recommendation pipeline. Similarly, the PLMs used as the recommenders' backbone could contain social biases captured from the training data. On the other hand, the \texttt{A-Modules} in \manner{} could be abused to reduce the diversity of recommendations by over-weighting the aspectual-similarity with the user's history, particularly for sensitive aspects such as news stance. This, in turn, could lead to reinforcing the users' existing worldviews and stances \cite{li2019survey}. Therefore, safeguards should be incorporated in the recommendation models to ensure not only that the outputs are accurate and truthful, but also that the systems are not misused to constrain access to diverse viewpoints. 
\subsection*{Acknowledgements}

\noindent
USA {\textendash} U.S. National Science Foundation-Office of Polar Programs,
U.S. National Science Foundation-Physics Division,
U.S. National Science Foundation-EPSCoR,
Wisconsin Alumni Research Foundation,
Center for High Throughput Computing (CHTC) at the University of Wisconsin{\textendash}Madison,
Open Science Grid (OSG),
Extreme Science and Engineering Discovery Environment (XSEDE),
Frontera computing project at the Texas Advanced Computing Center,
U.S. Department of Energy-National Energy Research Scientific Computing Center,
Particle astrophysics research computing center at the University of Maryland,
Institute for Cyber-Enabled Research at Michigan State University,
and Astroparticle physics computational facility at Marquette University;
Belgium {\textendash} Funds for Scientific Research (FRS-FNRS and FWO),
FWO Odysseus and Big Science programmes,
and Belgian Federal Science Policy Office (Belspo);
Germany {\textendash} Bundesministerium f{\"u}r Bildung und Forschung (BMBF),
Deutsche Forschungsgemeinschaft (DFG),
Helmholtz Alliance for Astroparticle Physics (HAP),
Initiative and Networking Fund of the Helmholtz Association,
Deutsches Elektronen Synchrotron (DESY),
and High Performance Computing cluster of the RWTH Aachen;
Sweden {\textendash} Swedish Research Council,
Swedish Polar Research Secretariat,
Swedish National Infrastructure for Computing (SNIC),
and Knut and Alice Wallenberg Foundation;
Australia {\textendash} Australian Research Council;
Canada {\textendash} Natural Sciences and Engineering Research Council of Canada,
Calcul Qu{\'e}bec, Compute Ontario, Canada Foundation for Innovation, WestGrid, and Compute Canada;
Denmark {\textendash} Villum Fonden and Carlsberg Foundation;
New Zealand {\textendash} Marsden Fund;
Japan {\textendash} Japan Society for Promotion of Science (JSPS)
and Institute for Global Prominent Research (IGPR) of Chiba University;
Korea {\textendash} National Research Foundation of Korea (NRF);
Switzerland {\textendash} Swiss National Science Foundation (SNSF);
United Kingdom {\textendash} Department of Physics, University of Oxford.

% Bibliography entries for the entire Anthology, followed by custom entries
%\bibliography{anthology,custom}
% Custom bibliography entries only
\bibliography{anthology,custom}

\appendix
\section{Baselines}
\label{sec:appendix_baselines}

We compare \manner{} against the following content-based NNRs, in which we replace the original NEs of all baselines that do not use PLMs with the same PLM employed in \manner{}:
\begin{enumerate}
    \item NRMS \cite{wu2019nrms} encodes only the news title, and learns user representations with an encoder combining multi-head self-attention and additive attention. 
    \item MINER \cite{li2022miner} employs a poly attention scheme to extract multiple user interest vectors for the users' representations using additive attention layers.
    \item NAML \cite{wu2019naml} leverages information about topical categories, in addition to textual content from the news title and abstract, as input to the NE. It learns user representations from the clicked news embeddings with a user encoder based on additive attention.
    \item LSTUR \cite{an2019lstur} also incorporates category information in the NE, next to title and abstract. It learns user representations via recurrent neural networks, and differentiates between short-term user preferences encoded with a GRU \cite{cho2014learning}, and long-term embeddings, consisting of a randomly initialized and fine-tuned component.
    \item MINS \cite{wang2022news} embeds both textual features (i.e, title, abstract), as well as categories. It employs a UE which combines multi-head self-attention, multi-channel GRU-based recurrent network, and additive attention to generate user embeddings.
    \item CAUM \cite{qi2022news} leverages not only titles and corresponding named entities, but also topical categories as input to the NE. In contrast to the other baselines, it combines a candidate-aware self-attention network with a candidate-aware convolutional neural network to learn candidate-aware user representations.
    \item TANR \cite{wu2019tanr} injects category information by jointly optimizing the NE for content-based personalization and topic classification. Its UE is analogous to that of NAML.
    \item SentiRec \cite{wu2020sentirec} adds regularization for sentiment diversity to its primary content personalization objective, and encodes users similarly to NRMS.
    \item SentiDebias \cite{wu2022removing} uses sentiment-debiasing based on adversarial learning to reduce the NNR's sentiment bias (originating from the user data) and generate sentiment-diverse recommendations.
\end{enumerate}

\section{Dataset Statistics}
\label{sec:appendix_datasets}
%
\begin{table*}[]
\vspace{1.5em}
\caption{Comparison of three dataset sources for CarPatch3D.} 
\label{tab:dataset}
\begin{center}
\begin{tabular}{ccccccc}
\toprule
Dataset &
  \begin{tabular}[c]{@{}c@{}}Background\end{tabular} &
  \begin{tabular}[c]{@{}c@{}}Unknown camera \\ intrinsics and poses\end{tabular} &
  \begin{tabular}[c]{@{}c@{}}Car orientaion\end{tabular} &
  \begin{tabular}[c]{@{}c@{}}Viewpoint diversity\end{tabular} &
  \begin{tabular}[c]{@{}c@{}}Filtered(details in \ref{sec:preprocessing}) \\ car patches number\end{tabular} &
  \begin{tabular}[c]{@{}c@{}}Filtered(details in \ref{sec:preprocessing}) \\ car instances number\end{tabular} \\
\midrule
KITTI-MOT & \checkmark   & \texttimes{} & Mostly front/back-view  & Single view + multi-view & 4481   & 383    \\
KITTI-DET & \checkmark   & \texttimes{} & Mostly front/back-view  & Single view only      & 4945   & 4945   \\
DVM-Cars  & \texttimes{} & \checkmark   & Many side-view & Single view + multi-view & 521275 & 201075 \\
\bottomrule
\end{tabular}
\end{center}
\vspace{-2.5em}
\end{table*}
%
\section{Reproducibility Details}
\label{sec:appendx_reproducibility}

\subsection{Model Parameters.}
\label{sec:appendix_model_parameters}
Table \ref{tab:model_parameters} lists the number of model parameters, in millions, for both datasets. 

\subsection{Hyperparameters and Implementation}
\label{sec:appendix_hyperparameters}

\vspace{1.4mm}
\noindent\textbf{Hyperparameter Optimization.}
We use RoBERTa Base \cite{liu2019roberta} and NB-BERT Base \cite{kummervold2021operationalizing,nielsen2023scandeval} as the backbone PLMs of all models, in experiments on MIND and Adressa, respectively. In both cases, we fine-tune only the PLM's last four layers.\footnote{In preliminary results, we did not see significant differences between full fine-tuning of all layers and fine-tuning only the last four layers. In the interest of computational efficiency, we thus froze the first eight layers of the transformer.}
In the cross-lingual transfer experiments from \$\ref{sec:xlt}, we fine-tune all of the 6 layers of DistilBERT.
%
We use 100-dimensional TransE embeddings \cite{bordes2013translating} pretrained on Wikidata as input to the entity encoder in the NE of the knowledge-aware NNRs. We perform hyperparameter tuning on the main hyperparameters of \manner{} and the baselines using grid search. Table \ref{tab:hyperparameters} lists the search spaces for all the optimized hyperparameters, as well as the best values. We repeat each experiment five times with the seeds ($\{42, 43, 44, 45, 46\}$) set with PyTorch Lightning's \texttt{seed\_everything}.
%
\newcolumntype{g}{>{\columncolor{Gray}}c}

\begin{table}[h]
\centering
    \resizebox{\columnwidth}{!}{%
    \begin{tabular}{lr|rg|rg}
        \toprule
         & \multicolumn{1}{c}{}
         & \multicolumn{2}{c}{\textbf{MIND}} 
         & \multicolumn{2}{c}{\textbf{Adressa}} 
         \\ \cmidrule(lr){3-4}  \cmidrule(lr){5-6}
         
        \textbf{Model} 
        & Non-trainable
        & Trainable & Total 
        & Trainable & Total
        \\ \hline
        
        NRMS-PLM 
        & 56.7 
        & 73 & 129 
        & 126 & 182 
        \\

        MINER 
        & 56.7 
        & 68.2 & 124 
        & 121 & 178 
        \\ 
        \hdashline
        
        NAML-PLM
        & 56.7
        & 70.8 & 127
        & 124  & 180
        \\
        
        LSTUR-PLM 
        & 56.7 
        & 633 & 690 
        & 200 & 257 
        \\
        
        MINS-PLM 
        & 56.7 
        & 73.3 & 130 
        & 126 & 183 
        \\ 
        
        CAUM\textsubscript{no entities}-PLM
        & 56.7 
        & 73.2 & 129 
        & 126 & 183 
        \\ 

        CAUM-PLM
        & 56.7 
        & 74.9 & 131 
        & -- & --
        \\ 

       
        TANR-PLM
        & 56.7 
        & 70.6 & 127 
        & 123 & 180 
        \\

        \hdashline
        
        SentiRec-PLM 
        & 56.7 
        & 73 & 129 
        & 126 & 182 
        \\ 

        SentiDebias-PLM 
        & 56.7 
        & 73.3 & 130 
        & 126 & 183 
        \\ 
        \hline 
        
        \manner{} (CR-Module\textsubscript{title} / A-Module\textsubscript{title}) -- monolingual 
        & 56.7 
        & 67.9 & 124
        & 121 & 177
        \\ 
        

        \manner{} (CR-Module / A-Module) -- monolingual 
        & 56.7 
        & 70.3 & 126 
        & -- & --
        \\ 

        \hdashline
         \manner{} (CR-Module\textsubscript{title} / A-Module\textsubscript{title}) -- multilingual 
        & 0
        & 134 & 134
        & 134 & 134
        \\ 
        
        \bottomrule
    \end{tabular}%
}
\caption{Number of model parameters (in millions). CR-Module\textsubscript{title} / A-Module\textsubscript{title} denote the \manner{} modules trained with only the news title as input to the NE.}
\label{tab:model_parameters}
\vspace{-0.5em}
\end{table}
%
\begin{table*}[ht]
\centering

\resizebox{\textwidth}{!}{%
    \begin{tabular}{lccccccccccc} 
    \toprule
     &  \multicolumn{1}{c}{\texttt{\texttt{lr}}} &  \multicolumn{1}{c}{\texttt{num\textsubscript{heads}}} &  \multicolumn{1}{c}{\texttt{query\textsubscript{dim}}} &  \multicolumn{1}{c}{\texttt{UE agg}} &  \multicolumn{1}{c}{$K$} &  \multicolumn{1}{c}{\texttt{score agg}} &  \multicolumn{1}{c}{$\lambda$} &  \multicolumn{1}{c}{$\mu$} &  \multicolumn{1}{c}{$\alpha$} &  \multicolumn{1}{c}{$\tau_{CR-Module}$} &  \multicolumn{1}{c}{$\tau_{A-Module}$} \\
        \bottomrule
        \textbf{Search Space} & [$1e^{-4}$, $1e^{-6}$] & \{8, 12, 16, 24, 32\} & [50, 200]  & \{ini, con\} & \{8, 16, 32, 48\} & \{mean, max, weighted\} & [0.1, 0.3] & [5, 15] & [0.05, 0.2] & [0.1, 0.5] & [0.1, 0.9] \\

        \textbf{Step} & $1e^{-1}$ & -- & 50 & -- & -- & -- & 0.05 & 5 & 0.05 & 0.02 & 0.05 \\

        \hline  
        NRMS-PLM & $1e^{-5}$ / $1e^{-6}$ & 32 / 8 &  150 / 200 & -- & -- & -- & -- & -- & -- & -- & -- \\
        MINER & $1e^{-5}$ / $1e^{-6}$ & -- & -- & -- & 32 / 48 & mean / mean & -- & -- & -- & -- & -- \\

        \hdashline 
        
        NAML-PLM & $1e^{-5}$ / $1e^{-6}$ & 16 / 8 & 200 / 200  & -- & -- & -- & -- & -- & -- & -- & -- \\
        LSTUR-PLM & $1e^{-5}$ / $1e^{-6}$ & 24 / 8 & 150 / 50 & ini / ini & -- & -- & -- & -- & -- & -- & -- \\
        MINS-PLM & $1e^{-5}$ / $1e^{-6}$ & 32 / 12 & 100 / 200 & -- & -- & -- & -- & -- & -- & -- & -- \\
        CAUM-PLM & $1e^{-5}$ / $1e^{-6}$ & 16 / 16 & 50 / 150 & -- & -- & -- & -- & -- & -- & -- & -- \\
        TANR-PLM & $1e^{-5}$ / $1e^{-6}$ & 32 / 8 & 150 / 50 & -- & -- & -- & 0.3 / 0.15 & -- & -- & -- & -- \\

        \hdashline
        SentiRec-PLM & $1e^{-5}$ / $1e^{-6}$ & 32 / 8 & 200 / 200 & -- & -- & -- & -- & 5 / 5 & -- & -- & -- \\
        SentiDebias-PLM & $1e^{-5}$ / $1e^{-6}$ & 8 / 12 & 100 / 150 & -- & -- & -- & -- & -- & 0.15 / 0.15 & -- & -- \\

        \hdashline
        
        MANNeR& $1e^{-5}$ / $1e^{-6}$ & -- & 200 / 200 & -- & -- & -- & -- & -- & -- & 0.36 / 0.14 & 0.9 / 0.9  \\
        \bottomrule
    \end{tabular}%
    }

\caption{Search spaces used for hyperparameter optimization and best values found for all models. We report the optimal values in the format \textit{value\textsubscript{MIND} / value\textsubscript{Adressa}}. We use the following abbreviations: \texttt{lr} = learning rate, \texttt{num\textsubscript{heads}} = number of attention heads, \texttt{query\textsubscript{dim}} = dimensionality of the query vector in additive attention, \texttt{UE agg} = aggregation method used to combine the long-term and the short-term user representations into a final user embedding in LSTUR \cite{an2019lstur}, $K$ = number of context codes in MINER \cite{li2022miner}, \texttt{score agg} = aggregation function for the final user click score calculation in MINER \cite{li2022miner}, $\lambda$ = weight of the topic classification task in TANR \cite{wu2019tanr}, $\mu$ = weight of the sentiment diversity regularization loss in SentiRec \cite{wu2020sentirec}, $\alpha$ = adversarial loss coefficient in SentiDebias \cite{wu2022removing}, $\tau$ = temperature parameter in SCL in \manner{}, \textit{ini} = initialize, \textit{con} = concatenate, \textit{categ} = category.}
\label{tab:hyperparameters}
\vspace{-0.5em}
\end{table*}

%

\vspace{1.4mm}
\noindent\textbf{Code.} We train \manner{}, as well as all the baselines, using the implementations provided in the NewsRecLib library \cite{iana-etal-2023-newsreclib}.\footnote{\href{https://github.com/andreeaiana/newsreclib}{https://github.com/andreeaiana/newsreclib}}

\vspace{1.4mm}
\noindent\textbf{Infrastructure and Compute.}
We conduct all experiments on a cluster with virtual machines. We train \manner{} on both datasets, as well as the baselines on MIND, on a single NVIDIA A100 40GB GPU. We train the baselines on Adressa on a single NVIDIA Tesla V100 32GB GPU. 

 
\section{Additional Results}
\label{sec:appendix_additional_results}

\subsection{Content Personalization}
\label{sec:appendix_content_personalization}

Fig.~\ref{fig:ablation_features} shows \manner{}'s performance on MIND for different inputs to the NE. Even the \texttt{CR-Module} exposed to titles only (i.e., no abstract or entity information) outperforms all of the baselines on content recommendation.
Fig. \ref{fig:ablation_training} illustrates \manner{}'s performance for alternative architecture designs and training objectives (cf. \$\ref{sec:content_personalization}).\footnote{For brevity, we report results on MIND; findings on Adressa exhibit identical trends.}
%
% Figure environment removed

\subsection{Single-Aspect Customization}
\label{sec:appendix_single_apsect_customization}

% Figure environment removed
%
Figure \ref{fig:single_aspect_results_addresa} explores the trade-off between content and aspect diversification, and respectively, personalization tasks for different values of $\lambda$\textsubscript{ctg} and $\lambda$\textsubscript{snt} on the Adressa dataset. 
%
Fig. \ref{fig:tsne_embeddings_adressa} shows the 2-dimensional t-SNE visualizations \cite{van2008visualizing} of the news embeddings produced with aspect-specialized NEs trained on Adressa. 

\subsection{Multi-Aspect Customization}
\label{sec:appendix_multi_apsect_customization}

Fig. \ref{fig:multi_aspect_results_addressa} explores the trade-off between content personalization and multi-aspect diversification on Adressa.

%
% Figure environment removed

% Figure environment removed   

\subsection{Cross-Lingual Transfer}
\label{sec:appendix_xlt}

% Figure environment removed

Fig. \ref{fig:xlt_adressa_mind_pers} summarizes the \texttt{XLT} results for single-aspect personalization on the target-language dataset MIND, whereas Fig. \ref{fig:xlt_mind_adressa} shows the analogous \texttt{XLT} results for single-aspect diversification and personalization, respectively, on the target-language dataset Adressa.

% Figure environment removed

\subsection{Time Complexity Analysis}
\label{sec:appendix_time}

Table \ref{tab:inference_time} shows the average inference time for the entire MIND (365,201 impressions), and respectively, Adressa (146,284 impressions) test sets. Note that runtimes heavily depend on the computing infrastructure used, as well as parallel usage of the infrastructure for other tasks, as experiments are conducted on a HPC cluster. We highlight that \manner{} achieves a much lower inference time than the other NNRs.

\newcolumntype{g}{>{\columncolor{Gray}}r}
\begin{table}[h!]
% \def\arraystretch{0.9}
\resizebox{\columnwidth}{!}{%  
  \begin{tabular}{lll}
    \toprule
     \textbf{Model} & \textbf{MIND} & \textbf{Adressa}\\ \midrule
     NRMS-PLM & 17.53\textsubscript{$\pm$0.48}  &  7.13\textsubscript{$\pm$0.27}\\
     MINER & 16.03\textsubscript{$\pm$1.66}  & 9.96\textsubscript{$\pm$0.73} \\
     NAML-PLM & 33.99\textsubscript{$\pm$0.51}  & 7.09\textsubscript{$\pm$0.14} \\
     MINS-PLM & 27.50\textsubscript{$\pm$10.87}  & 7.81\textsubscript{$\pm$0.21} \\
     CAUM\textsubscript{no entities}-PLM & 22.67\textsubscript{$\pm$2.46}  & 8.12\textsubscript{$\pm$0.13} \\
     CAUM-PLM & 25.22\textsubscript{$\pm$0.45} & -- \\
     TANR-PLM & 17.02\textsubscript{$\pm$1.07}  & 6.98\textsubscript{$\pm$0.08} \\
     SentiRec-PLM &  17.93\textsubscript{$\pm$0.34} & 7.02\textsubscript{$\pm$0.08} \\
     SentiDebias-PLM & 21.01\textsubscript{$\pm$3.03}  & 13.28\textsubscript{$\pm$0.83} \\
     \manner{} (CR-Module) & 1.34\textsubscript{$\pm$0.03}  & 2.09\textsubscript{$\pm$0.06} \\
     \manner{} (CR-Module + \textit{ctg} A-Module) &  1.68\textsubscript{$\pm$0.08} & 2.78\textsubscript{$\pm$0.10} \\
     \manner{} (CR-Module + \textit{snt} A-Module) & 1.65\textsubscript{$\pm$0.01}  & 2.73\textsubscript{$\pm$0.06}\\
     \manner{} (CR-Module + 2 A-Modules) & 2.13\textsubscript{$\pm$0.05}  & 3.17\textsubscript{$\pm$0.01}\\
    
  \bottomrule
\end{tabular}%
}
\caption{Inference time (in thousands of seconds) for the different NNRs on the test portions of the MIND and Adressa datasets, respectively.}
\label{tab:inference_time}
\vspace{-0.5em}
\end{table}

\end{document}
