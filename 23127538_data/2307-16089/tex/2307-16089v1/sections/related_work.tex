\section{Background and Related Work}
\label{sec:related_work}

\sparagraph{Personalized NNR.}
Neural content-based models have become the main vehicle of personalized news recommendation, replacing traditional recommendation systems relying on manual feature engineering \cite{wu2023personalized}. Most NNR models consist of a dedicated (i) news encoder (NE) and (ii) user encoder (UE) \cite{wu2023personalized}. The NE transforms input features into news embeddings \cite{wu2023personalized, wu2019nrms, wu2019npa}, whereas UEs create user-level representations by aggregating and contextualizing the embeddings of clicked news from the user's history \cite{okura2017embedding, an2019lstur, wu2022news}. The candidate's recommendation score is finally computed by comparing its embedding against the user embedding \cite{wang2018dkn, wu2019naml}. NNR models are primarily trained via standard point-wise classification objectives with negative sampling \cite{huang2013learning, wu2021empowering}. 

Exploiting users' past behavior as NNR supervision unsurprisingly leads to recommendations that are content-wise closest to previously consumed news, in contrast to methods based on non-personalized criteria, e.g., news popularity \cite{ludmann2017recommending, yang2016effects} or geographic information \cite{son2013location, chen2017location}.
%%
More recent NNRs seek to augment content-based personalization by considering other aspects, such as topical categories, sentiment, entities, outlets, popularity, or recency \cite{wu2023personalized}. These are incorporated in the NNR either as additional input to the NE \cite{wang2018dkn, gao2018fine, wu2019naml, liu2020kred, sheu2020context, lu2020beyond, qi2021personalized, xun2021we}, or in the form of an auxiliary training objective for the NE \cite{wu2019tanr, wu2020sentirec, qi2021pp}.



\rparagraph{Diversification.}
%%%%
Optimized for personalization, i.e., to favor matches in news content (and other aspects like topic or sentiment), personalized NNR reduces exposure to news dissimilar from those consumed in the past. Recommending ``more of the same'', constrains access to diverse viewpoints and information \cite{freedman1965selective, heitz2022benefits} and leads to homogeneous news diets and ``filter bubbles'' \cite{pariser2011filter}, in turn reinforcing users' initial stances \cite{li2019survey}.
%%%%
A significant body of work acknowledges this and attempts to diversify recommendations. One line of work \cite{li2011scene, rao2013taxonomy, kaminskas2016diversity, gharahighehi2023diversification} re-ranks the set of  personalized recommendations to increase some measure of diversity (e.g. intra-list distance \cite{zhang2008avoiding}). A second line of work resorts to multi-task training of NNRs \cite{gabriel2019contextual, wu2020sentirec, shi2022dcan, wu2022end, choi2022not}, coupling the primary content-based personalization objective with auxiliary objectives that force aspect-based diversification.
%%%%


\rparagraph{Limitations of Current NNRs.} 
Current NNR approaches have an obvious limitation: by ``hardcoding'' aspectual requirements (i.e., personalization or diversification for an aspect) into the NNR's architecture and/or training objectives, they cannot be easily adjusted for different (i.e., custom) recommendation goals. Because even a minor change in the recommendation objective requires retraining of the NNR model,  existing NNRs are ill-equipped for multi-aspect customization, and generally limited to fixed single-aspect recommendation scenarios (e.g., content-based personalization with topical diversification).  
%%
In this work, we rethink personalized news recommendation and propose a novel, modular multi-aspect recommendation framework that 
%leverages content-based as well as aspect-based relevance in a modular fashion, 
allows for ad-hoc creation of recommendation functions over aspects at inference time. 
%seamlessly enables custom multi-aspect recommendation. 
%(i.e., simultaneous personalization over some aspects and diversification over others) at inference time. 
This enables fundamentally different news recommendation: one that lets each user define their own custom recommendation function, choosing the extent of personalization or diversification for each aspect. 
%\manner{} enables this by default, without the need to train dedicated NNR models for different users. 
