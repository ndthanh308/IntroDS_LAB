\section{Related Work}
\label{sec:related_work}

%\vspace{1.4mm}
\noindent\textbf{Personalized NNR.}
Neural content-based models have become the main vehicle of personalized news recommendation, replacing traditional recommenders relying on manual feature engineering \cite{wu2023personalized}. Most NNRs consist of a dedicated (i) news encoder (NE) and (ii) user encoder (UE) \cite{wu2023personalized}. The NE transforms input features into news embeddings \cite{wu2023personalized, wu2019nrms, wu2019npa}, whereas UEs create user-level representations by aggregating and contextualizing the embeddings of clicked news from the user's history \cite{okura2017embedding, an2019lstur, wu2022news}. The candidate's recommendation score is computed by comparing its embedding against the user embedding \cite{wang2018dkn, wu2019naml}. NNRs are primarily trained via point-wise classification objectives with negative sampling \cite{huang2013learning, wu2021empowering}. 
%
Exploiting users' past behavior as NNR supervision leads to recommendations that are content-wise closest to previously consumed news, in contrast to methods based on non-personalized criteria \cite{son2013location, chen2017location, ludmann2017recommending}.
%%
More recent NNRs seek to augment content-based personalization by considering other aspects, such as categories, sentiment, emotions \cite{sertkan2022exploring}, entities \cite{iana2022survey}, outlets, or recency \cite{wu2023personalized}. These are incorporated in the NNR either as additional input to the NE \cite{wang2018dkn, gao2018fine, wu2019naml, liu2020kred, sheu2020context, lu2020beyond, qi2021personalized, xun2021we}, or in the form of an auxiliary training objective for the NE \cite{wu2019tanr, wu2020sentirec, qi2021pp}.


\vspace{1.4mm}
\noindent\textbf{Diversification.}
%%%%
Personalized NNR reduces exposure to news dissimilar from those consumed in the past. Recommending ``more of the same'' constrains access to diverse viewpoints and information \cite{freedman1965selective,heitz2022benefits} and leads to homogeneous news diets and ``filter bubbles'' \cite{pariser2011filter}, in turn reinforcing users' initial stances \cite{li2019survey}.
%%%%
Consequently, a significant body of work attempts to diversify recommendations, either by re-ranking them to increase some measure of diversity (e.g. intra-list distance \cite{zhang2008avoiding}) or by resorting to multi-task training \cite{gabriel2019contextual,wu2020sentirec,shi2022dcan,wu2022end,choi2022not,raza2023bias}, coupling the primary content-based personalization objective with auxiliary objectives that force aspect-based diversification.


\vspace{1.4mm}
\noindent\textbf{Current NNR Limitations.} 
Critically, existing approaches, by ``hardcoding'' aspectual requirements (i.e., personalization or diversification for an aspect) into the NNR's architecture and/or training objectives, cannot be easily adjusted for varying recommendation goals. Since even minor changes in the recommendation objective require retraining the NNR,  current models are generally limited to fixed single-aspect recommendation scenarios (e.g., content-based personalization with topical diversification), and ill-equipped for multi-aspect customization.  
%%
In this work, we rethink personalized news recommendation and propose a novel, modular multi-aspect recommendation framework that allows for ad-hoc creation of recommendation functions over aspects at inference time. 
This enables fundamentally different recommendation: one that lets each user define their own custom recommendation function, choosing the amount of personalization or diversification for each aspect. 

