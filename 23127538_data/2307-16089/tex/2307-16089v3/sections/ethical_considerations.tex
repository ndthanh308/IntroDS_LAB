\section*{Ethical Considerations}
\label{sec:ethical_consideations}

We consider several ethical considerations that arise when working with recommender systems and  open benchmark datasets. On the one hand, any biases or misinformation that might exist in the news and user data provided in the public datasets could be propagated through the recommendation pipeline. Similarly, the PLMs used as the recommenders' backbone could contain social biases captured from the training data. On the other hand, the \texttt{A-Modules} in \manner{} could be abused to reduce the diversity of recommendations by over-weighting the aspectual-similarity with the user's history, particularly for sensitive aspects such as news stance. This, in turn, could lead to reinforcing the users' existing worldviews and stances \cite{li2019survey}. Therefore, safeguards should be incorporated in the recommendation models to ensure not only that the outputs are accurate and truthful, but also that the systems are not misused to constrain access to diverse viewpoints. 