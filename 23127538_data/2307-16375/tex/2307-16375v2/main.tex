\documentclass{article}

\usepackage{microtype}
\usepackage{graphicx}
\usepackage{subcaption}
\usepackage{multirow}
\usepackage{makecell}
\usepackage{booktabs}

\usepackage{hyperref}


\newcommand{\theHalgorithm}{\arabic{algorithm}}

\usepackage[arxiv]{conference2024}

\usepackage{amsmath}
\usepackage{amssymb}
\usepackage{mathtools}
\usepackage{amsthm}
\allowdisplaybreaks[4]

\usepackage[capitalize,noabbrev]{cleveref}

\theoremstyle{plain}
\newtheorem{theorem}{Theorem}[section]
\newtheorem{proposition}[theorem]{Proposition}
\newtheorem{lemma}[theorem]{Lemma}
\newtheorem{corollary}[theorem]{Corollary}
\theoremstyle{definition}
\newtheorem{definition}[theorem]{Definition}
\newtheorem{assumption}[theorem]{Assumption}
\theoremstyle{remark}
\newtheorem{remark}[theorem]{Remark}

\usepackage[textsize=tiny]{todonotes}


\conferencetitlerunning{UniAP: Unifying Inter- and Intra-Layer Automatic Parallelism by Mixed Integer Quadratic Programming}

\begin{document}

\twocolumn[
\conferencetitle{\texorpdfstring{UniAP: Unifying Inter- and Intra-Layer Automatic Parallelism\\by Mixed Integer Quadratic Programming}{UniAP: Unifying Inter- and Intra-Layer Automatic Parallelism by Mixed Integer Quadratic Programming}}

\conferencesetsymbol{equal}{*}

\begin{conferenceauthorlist}
\conferenceauthor{Hao Lin}{equal,nju}
\conferenceauthor{Ke Wu}{equal,nju}
\conferenceauthor{Jie Li}{nju}
\conferenceauthor{Jun Li}{nju}
\conferenceauthor{Wu-Jun Li}{nju}
\end{conferenceauthorlist}

\conferenceaffiliation{nju}{National Key Laboratory for Novel Software Technology, Department of Computer Science and Technology, Nanjing University, Nanjing 210023, China}

\conferencecorrespondingauthor{Wu-Jun Li}{liwujun@nju.edu.cn}

\conferencekeywords{Automatic Parallelism, Mixed Integer Quadratic Programming}

\vskip 0.3in
]


\printAffiliationsAndNotice{\conferenceEqualContribution}

\begin{abstract}
Distributed learning is commonly used for training deep learning models, especially large models. In distributed learning, manual parallelism~(MP) methods demand considerable human effort and have limited flexibility. Hence, automatic parallelism~(AP) methods have recently been proposed for automating the parallel strategy optimization process. Existing AP methods suffer from sub-optimal solutions because they do not jointly optimize the two categories of parallel strategies~(i.e., inter-layer parallelism and intra-layer parallelism). In this paper, we propose a novel AP method called UniAP, which unifies inter- and intra-layer automatic parallelism by mixed integer quadratic programming. To the best of our knowledge, UniAP is the first parallel method that can jointly optimize the two categories of parallel strategies to find an optimal solution. Experimental results show that UniAP outperforms state-of-the-art methods by up to 1.71$\times$ in throughput and reduces strategy optimization time by up to 107$\times$ across five Transformer-based models.
\end{abstract}


\section{Introduction}
Deep learning models have been widely used in many applications.
For example, BERT~\citep{devlin_bert_2019}, GPT-3~\citep{brown_language_2020}, and T5~\citep{raffel_exploring_2020} achieved state-of-the-art~(SOTA) results on different natural language processing~(NLP) tasks. 
For computer vision~(CV), Transformer-like models such as ViT~\citep{dosovitskiy_image_2021} and Swin Transformer~\citep{liu_swin_2021} deliver excellent accuracy performance upon multiple tasks. 


At the same time, training deep learning models has been a critical problem troubling the community due to the long training time, especially for those large models with billions of parameters~\citep{brown_language_2020}. 
In order to enhance the training efficiency, researchers propose some manually designed parallel training strategies~\citep{narayanan_efficient_2021,shazeer_mesh-tensorflow_2018,xu_gspmd_2021}. 
However, selecting, tuning, and combining these strategies require extensive domain knowledge in deep learning models and hardware environments. With the increasing diversity of modern hardware architectures~\cite{flynn_very_1966,flynn_computer_1972} and the rapid development of deep learning models, these manually designed approaches are bringing heavier burdens to developers. 
Hence, \emph{automatic parallelism} is introduced to automate the parallel strategy searching for training models.


There are two main categories of parallelism in deep learning models: inter-layer parallelism~\citep{huang_gpipe_2019,narayanan_pipedream_2019,narayanan_memory-efficient_2021,fan_dapple_2021,li_chimera_2021,lepikhin_gshard_2021,du_glam_2022,fedus_switch_2022} and intra-layer parallelism~\citep{li_pytorch_2020,narayanan_efficient_2021,rasley_deepspeed_2020,fairscale_authors_fairscale_2021}. 
Inter-layer parallelism partitions the model into disjoint sets on different devices without slicing tensors. 
Alternatively, intra-layer parallelism partitions tensors in a layer along one or more axes and distributes them across different devices.


Current automatic parallelism techniques focus on optimizing strategies within these two categories. However, they treat these two categories separately. 
Some methods~\citep{zhao_vpipe_2022,jia_exploring_2018,cai_tensoropt_2022,wang_supporting_2019,jia_beyond_2019,schaarschmidt_automap_2021,liu_colossal-auto_2023} overlook potential opportunities for inter- or intra-layer parallelism, the others optimize inter- and intra-layer parallelism hierarchically and sequentially~\citep{narayanan_pipedream_2019,fan_dapple_2021,he_pipetransformer_2021,tarnawski_efficient_2020,tarnawski_piper_2021,zheng_alpa_2022}. 
As a result, current automatic parallelism techniques often fail to achieve the global optima and instead become trapped in local optima. 
Therefore, a unified inter- and intra-layer approach is needed to enhance the effectiveness of automatic parallelism.


This paper aims to find the optimal parallelism strategy while simultaneously considering inter- and intra-layer parallelism. 
It enables us to search in a more extensive strategy space where the globally optimal solution lurk. 
However, unifying inter- and intra-layer parallelism in automatic parallelism brings us two challenges. 
Firstly, to adopt a unified perspective on the inter- and intra-layer automatic parallelism, we should not formalize them with separate formulations as prior works. Therefore, how can we express these parallelism strategies in a unified formulation? 
Secondly, previous methods take a long time to obtain the solution with a limited strategy space. Therefore, how can we ensure that the best solution can be obtained in a reasonable time while expanding the strategy space?


To solve the above challenges, we propose UniAP. For the first challenge, UniAP adopts the mixed integer quadratic programming~(MIQP)~\citep{lazimy_mixed_1982} to search for the globally optimal parallel strategy automatically. 
It unifies the inter- and intra-layer automatic parallelism in a single MIQP formulation. 
For the second challenge, our complexity analysis and experimental results show that UniAP can obtain the globally optimal solution in a significantly shorter time.


The contributions of this paper are summarized as follows: 
\begin{itemize}
    \item We propose UniAP, the first framework to unify inter- and intra-layer automatic parallelism in model training.
    \item The optimal parallel strategies discovered by UniAP exhibit scalability on training throughput and strategy searching time.
    \item The experimental results show that UniAP speeds up model training on four Transformer-like models by up to 1.70$\times$ and reduces the strategy searching time by up to 16$\times$, compared with the SOTA method.
\end{itemize}



\section{Background}
\subsection{Parallel Strategy}
\label{subsec:background:parallel-strtegy}
\paragraph{Pipeline parallelism~(PP)} In PP, each worker~(machine or GPU) holds a subset of model layers. Adjacent layers on different workers need to transfer activations in the forward propagation~(FP) step and gradients in the backward propagation~(BP) step. 
\paragraph{Data parallelism~(DP)} In DP, each worker holds a replica of the whole model and partitions training samples. In each iteration, each worker computes gradients and synchronizes them with the other workers using all-reduce collective communication~(CC). All workers will have the same model parameters after the synchronization step.
\paragraph{Tensor parallelism~(TP)} In TP, each worker holds a replica of training samples and partitions within model layers. In each iteration, each worker computes its local outputs in FP and its local gradients in BP. To synchronize outputs and gradients, all workers will perform all-reduce CC in FP and BP steps according to the partition scheme.
\paragraph{Fully sharded data parallelism~(FSDP)} FSDP partitions optimizer states, parameters and gradients of the model into separate workers. During the FP and BP step of each iteration, FSDP performs an all-gather CC to obtain the complete parameters for the relevant layer, respectively. After computing the gradients, FSDP conducts a reduce-scatter CC to distribute the global gradients among the workers.

\subsection{Manual Parallelism}
MP refers to the parallel methods in which human experts design and optimize the parallel strategies. Representative MP methods include Megatron-LM~\citep{narayanan_efficient_2021}, Mesh-TensorFlow~\citep{shazeer_mesh-tensorflow_2018}, and GSPMD~\citep{xu_gspmd_2021}. Megatron-LM manually designs TP and PP strategies for training Transformer-based models and exhibits superior efficiency. Mesh-TensorFlow and GSPMD require human effort to designate and tune the intra-layer parallel strategy. These methods rely on expert design and have little flexibility, challenging their automatic application to other models.

\subsection{Automatic Parallelism}
\paragraph{Inter-layer-only AP or intra-layer-only AP} For inter-layer-only AP, GPipe~\citep{huang_gpipe_2019} and vPipe~\citep{zhao_vpipe_2022} employ a balanced partition algorithm and a dynamic layer partitioning middleware to partition pipelines, respectively. For intra-layer-only AP, OptCNN~\citep{jia_exploring_2018}, TensorOpt~\citep{cai_tensoropt_2022}, and Tofu~\citep{wang_supporting_2019} employ dynamic programming methods to optimize DP and TP strategies together. FlexFlow~\citep{jia_beyond_2019} and Automap~\citep{schaarschmidt_automap_2021} use the Monte Carlo method to find the optimal DP and TP strategy. Colossal-Auto~\citep{liu_colossal-auto_2023} utilizes integer programming techniques to generate intra-layer parallelism and activation checkpointing strategies without optimizing inter-layer parallelism. All these methods optimize only one category of parallel strategies.


\paragraph{Inter- and intra-layer AP} PipeDream~\citep{narayanan_pipedream_2019}, DAPPLE~\citep{fan_dapple_2021}, and PipeTransformer~\citep{he_pipetransformer_2021} use dynamic programming to determine optimal strategies for both DP and PP. DNN-partitioning~\citep{tarnawski_efficient_2020} adopts integer and dynamic programming to explore DP and PP strategies. Piper~\citep{tarnawski_piper_2021} and Alpa~\citep{zheng_alpa_2022} adopt a parallel method considering DP, TP, and PP.
Galvatron~\citep{miao_galvatron_2022} uses dynamic programming to determine DP, TP, and FSDP strategies in a single pipeline stage. As for PP, it partitions stages and determines micro-batch size using naive greedy algorithms. All these methods are hierarchical, which will result in sub-optimal solutions.




\section{Methodology}
\label{sec:method}

\subsection{Overview}
\label{sec:method_fmwk}

As shown in~\cref{fig:method_fmwk}, the proposed unsupervised MOT framework is trained with the widely-used contrastive learning technique~\cite{chen2020simple,he2020momentum}. 
\lk{Specifically, for multi-object tracking}, objects within the tracklet ($\boldsymbol{k}_{+}$) should be pulled together and different tracklets ($\boldsymbol{k}_{-}$) should be separated. It can be mathematically formulated as:

\begin{equation}
% \begin{split}
    \mathcal{L}_{cl}( \boldsymbol{q}; \boldsymbol{k}_{+}; \boldsymbol{k}_{-} )= 
    - \log \frac{\exp(\boldsymbol{q} \cdot \boldsymbol{k}_{+} / \epsilon)}{\sum_{i}\exp(\boldsymbol{q} \cdot \boldsymbol{k}_{i} / \epsilon)}
  \label{eq:method_nce}
% \end{split}  
\end{equation}

\noindent where $\mathcal{L}_{cl}$ denotes the InfoNCE~\cite{oord2018representation} loss function, and $\epsilon$ is the temperature hyper-parameter~\cite{wu2018unsupervised}. 
Within a video, following the unsupervised tracking fashion~\cite{liu2022online,shuai2022id}, the positive and negative keys mainly come from two sources, \ie pseudo-labeled historical frame and self-augmented frame. 

\lk{However, two issues occur: (1) the uncertainty reduces the accuracy of pseudo-tracklets and (2) the randomly augmented samples fail to learn the inter-frame consistency.} 
We argue the above issues are not independent. 
\lk{By leveraging the uncertainty in turn,} the accurate pseudo-tracklets can guide the qualified positive and negative augmentations.

To address these two issues, we propose an uncertainty-aware pseudo-tracklet labeling strategy in \cref{sec:method_uoap}, which integrates a verification-and-rectification mechanism into the tracklet generation. Our method significantly improves the accuracy of pseudo-identities, especially in long-term interval. 
Then we propose a tracklet-guided augmentation strategy in \cref{sec:method_ada_aug}, which brings the temporary information into spatial augmentation. The augmented samples simulates the objects' motion. A hierarchical uncertainty-based sampling strategy is proposed for hard sample mining. More details are described in the following section.


\subsection{Uncertainty-aware Tracklet-Labeling}
\label{sec:method_uoap}

Accurate pseudo tracklet is critical in \liuk{learning feature consistency}. 
However, without manual annotation, \lk{the aggravated uncertainty makes} the tracklet-labeling a huge challenge due to various interference factors, including similar appearance among objects, frequent object cross and occlusions, \etc. 
\lk{In fact, the uncertainty can also be leveraged to improve the pseudo-accuracy in turn.}
In this section, we propose an \textbf{U}ncertainty-aware \textbf{T}racklet-\textbf{L}abeling (\textbf{UTL}) strategy for better pseudo-tracklets.

Given an input video sequence $V = \{I^{1}, I^{2}, \cdots, I^{N}\}$, each frame $I^{t}$ is annotated with the bounding boxes $B^{t} = \{b_{1}^{t}, b_{2}^{t}, \cdots, b_{M^{t}}^{t}\}$ of $M^{t}$ objects in $t_{th}$ frame, where $b_{i}^{t} = (cx_{i}^{t}, cy_{i}^{t}, w_{i}^{t}, h_{i}^{t})$ is the center coordinate and shape of the $i_{th}$ object $o_{i}^{t}$. As shown in~\cref{fig:method_fmwk}, \mywork~generates accurate pseudo-tracklets in four main steps:

1) \textbf{Association}. For a certain object $o_{i}^{t}$ in frame $I^{t}$, the $\ell_2$-normalized representation $\boldsymbol{f}_{i}^{t}$ can be expressed as $\boldsymbol{f}_{i}^{t} = {\phi}(I^{t}, b_{i}^{t})$, 
% \begin{equation}
%   \boldsymbol{f}_{i}^{t} = {\phi}(I^{t}, b_{i}^{t})
%   % / {\Vert {\phi}(I^{t}, b_{i}^{t}) \Vert}_{2}
%   \label{eq:method_feat}
% \end{equation}
where the embedding encoder is denoted as $\phi$.

To associate the objects in frame $I^{t}$ with the objects or trajectories in previous $I^{t \minus 1}$, a similarity matrix is constructed with their appearance embeddings:

\begin{equation}
  \boldsymbol{C} \in \mathbb{R}^{M^{t} \times M^{t \minus 1}}, \;
  c_{i,j} = {\boldsymbol{f}_{i}^{t}} \cdot  \boldsymbol{f}_{j}^{t \minus 1}
  \label{eq:method_matrix}
\end{equation}

\noindent where $c_{i,j}$ represents the cosine similarity between the $i_{th}$ object in frame $I^{t}$ and the $j_{th}$ object (or trajectory) in frame $I^{t \minus 1}$. Then the Hungarian algorithm~\cite{kuhn1955hungarian} is adopted to generate the identity association results.

2) \textbf{Verification}. However, the appearance representations are sometimes unreliable, especially in the unsupervised scenario. To solve this issue, an uncertainty metric is proposed to evaluate the association after the first stage.

% For an object $o_{i}^{t}$ in frame $I^{t}$, the similarities against the $M^{t \minus 1}$ objects in the previous frame can be expressed as:

% \begin{equation}
%   \boldsymbol{s}_{i} = \boldsymbol{C}_{i} = [c_{i,1}, c_{i,2}, \cdots, c_{i,M^{t \minus 1}}]^T
%   \label{eq:method_svec}
% \end{equation}

% Inspired by margin-based OOD detection~\cite{hendrycks2016baseline}, we assume that the assignment ($o_{i}^{t} \!\sim\! o_{j}^{t \minus 1}$) in the association stage is not convincing under the following circumstances:

% \begin{itemize}
%     \setlength{\itemsep}{0pt}
%     \item The assigned similarity between $o_{i}^{t}$ and $o_{j}^{t \minus 1}$ is relatively low (\ie, $c_{i,j} < m_1$).
%     \item The second-highest similarity with others ($c_{i,j_{2}}$) is close to the assigned $o_{j}^{t \minus 1}$ (\ie, $c_{i,j} - c_{i,j_{2}} < m_2$).
% \end{itemize}

% Based on these assumptions, we define an association-level uncertainty metric, which is formulated as:



Object association can be viewed as multi-category classification.
And confidence-score has been proved efficient and effective on detecting mis-classified examples~\cite{hendrycks2016baseline}.
Inspired by this, we propose to detect the mis-associated objects through the similarity-scores.


Given an object $o_{i}^{t}$ associated with $o_{j}^{t \minus 1}$ in the previous frame based on \cref{eq:method_matrix}, the association ($o_{i}^{t} \!\sim\! o_{j}^{t \minus 1}$) is unconvincing in two cases: 
1) the assigned similarity $c_{i,j}$ is relatively low (\eg, partial occlusion or motion blur) and 
2) there are other objects whose similarities are close to the assigned $c_{i,j}$ (\eg, similar appearance or indistinguishable embedding).
It can be formulated as:

\begin{equation}
  c_{i,j} < m_1; \quad c_{i,j_{2}} > c_{i,j} - m_2
  \label{eq:method_margin}
\end{equation}


\noindent 
where $m_1,m_2$ are constant margins. Only the second-highest similarity with others ($c_{i,j_{2}}$) is considered for simplicity.
In an ideal association, $c_{i,j}$ should be close to 1 and $c_{i,j_{2}}$ close to 0.
We thus proposed to estimate the association \lk{risk} as:

% \sigma_{i,j} = - \left( 
% \log c_{i,j} + \log \left( 1 - c_{i,j_{2}} \right)
% + \overline{\log \left( 1 - c_{i,l} \right) }
% \right)  
\begin{equation}
  \sigma_{i,j} = - \log c_{i,j} - \log \left( 1 - c_{i,j_{2}} \right)
  \label{eq:method_energy}
\end{equation}

Detailed derivation process refers to the supplementary materials.
Combining with \cref{eq:method_margin} and \cref{eq:method_energy} , an adaptive threshold is proposed:

\begin{equation}
  % \gamma_{i,j} = -\log \left( 1 + m_2 - c_{i,j} \right) -\log m_1 \left( 1 - m_3 \right)
  \gamma_{i,j} =  -\log m_1 - \log \left( 1 + m_2 - c_{i,j} \right)
  \label{eq:method_border}
\end{equation}

As shown in~\cref{fig:method_verify}, when the \lk{risk} $\sigma_{i,j}$ is higher than the threshold $\gamma_{i,j}$, the assignment ($o_{i}^{t} \!\sim\! o_{j}^{t \minus 1}$) should be re-considered. 
\lk{The \textbf{association uncertainty} is quantified as:}

\begin{equation}
  \delta_{i,j} = \sigma_{i,j} - \gamma_{i,j}
  \label{eq:method_uncertain}
\end{equation}

The results are not sensitive to the exact margins. We set $m_1 = 0.5$ and $m_2 = 0.05$ for a slightly better performance.
% More experimental details are shown in the supplementary materials.

The uncertain pairs after the verification stage and unmatched objects after the association stage are gathered as uncertain candidates for the rectification stage.


3) \textbf{Rectification}. 
The rectification stage is performed among the uncertain candidate. The similarities between two adjacent frames are no longer convincing.
% due to irregular motion, severe occlusion, and so on. 
More information should be taken into account, including motion \lk{estimation} and appearance \lk{variation} within a tracklet. 
% Specifically, intersection-over-union (IoU)~\cite{bewley2016simple} is the widely-used motion metric. At the same time, the tracklet embeddings can provide complementary appearance information.

For the uncertain candidates, \mywork~constructs another similarity matrix for the secondary rectification. 
First, \lk{the motion constraints should be relaxed}, so the association shares overlap \lk{higher than} $\beta$ 
% in adjacent frames 
\lk{are preserved}. 
Second, \lk{the appearance should not vary extremely fast}, so we adopt the averaged similarity between object $o_{i}^{t}$ and tracklet $trk_{j} = \{o_{j}^{t \minus K}, \cdots, o_{j}^{t \minus 1}\}$ within previous $K$ frames. 
In this stage, we solve the sub-problem of global identity assignments, which can be formulated as:

\begin{equation}
\begin{split}
  \boldsymbol{C}^\prime \in \mathbb{R}^{{M^{t}}^\prime \times {M^{t \minus 1}}^\prime} & \\
  c^\prime_{i,j} = \left( \frac{1}{K} \sum_{\hat{t} = t \minus K}^{t \minus 1} {\boldsymbol{f}_{i}^{t}} \cdot  \boldsymbol{f}_{j}^{\hat{t}} \right) 
            \times \mathbb{I} & \left( \text{IoU} \left( b_{i}^{t}, b_{j}^{t \minus 1} \right) > \beta \right) 
  \label{eq:method_recti}
\end{split}
\end{equation}

\noindent where $\mathbb{I}(*)$ is the indicator function. Then the match set is updated based on the Hungarian algorithm.

\lk{
\textit{Remark.} Our core contribution is the uncertainty-based verification mechanism, rather than the specific rectification, which shall be adjusted in practice. Empirically we set $\beta=0.1$ and $K=5$.
}

% Figure environment removed

4) \textbf{Propagation}. The pseudo-tracklets are propagated frame-by-frame. As shown in~\cref{fig:method_reidacc}, our strategy brings \lk{consistently} accurate pseudo-identities, \lk{\eg, reaching 97\% accuracy across 100 frames}.
% The pseudo-tracklets are progressively updated during the training process.
The long-term intra-tracklet consistency is successfully maintained.
% by the accurate pseudo-identities.

It is worth mentioning that the \lk{verification and rectification} stages can be naturally applied to the inference process to boost the performance, \lk{which does not conflict with existing association methods}.

\subsection{Tracklet-Guided Augmentation}
\label{sec:method_ada_aug}

The accurate pseudo-tracklets can guide the sample augmentation in the unsupervised MOT framework.
To learn the \liuk{inter-frame consistency}~\cite{chen2020simple,zhang2021fairmot}, good training samples should be diverse and \liuk{temporal-aware}. 
However, as illustrated in~\cref{fig:method_ada_aug}, existing methods usually treat augmentation and multi-object tracking as two isolated tasks, leading to ineffective augmentations. 
Instead, this paper utilizes the tracklet's spatial displacements to guide the augmentation process. 
According to a properly selected anchor pair, the proposed strategy makes the augmented frames aligned to the historical frames, simulating realistic tracklet movements. The proposed method concurrently focuses on the hard negative samples.
Details \lk{of the \textbf{T}racklet-\textbf{G}uided \textbf{A}ugmentation (TGA)} are given below.

% Figure environment removed

We introduce the temporal information into spatial transformation. 
First, given a current frame $I^{t}$ with $M^{t}$ objects, we select a source-anchor object $o_{a}^{t}$, whose bounding box is denoted as $b_{a}^{t} = (cx_{a}^{t}, cy_{a}^{t}, w_{a}^{t}, h_{a}^{t})$. Then, we choose a target-anchor $o_{a}^{t \minus \tau}$ in $(t \minus \tau)_{th}$  historical frame from the pseudo-tracklet $trk_{a} = \{o_{a}^{t_0}, o_{a}^{t_1}, \cdots, o_{a}^{t}\}$. 
Finally, to augment the current $I^{t}$ to align with historical $I^{t \minus \tau}$,  a tracklet-guided affine transformation can be expressed as:

\begin{equation}
  \begin{bmatrix}
      x^{t \minus \tau} \\ y^{t \minus \tau} \\ 1
  \end{bmatrix}
  =
  \boldsymbol{M}_{t}^{t \minus \tau}
  \begin{bmatrix}
      x^{t} \\ y^{t} \\ 1
  \end{bmatrix}
  =
  \begin{bmatrix}
      m_{11} & m_{12} & m_{13} \\
      m_{21} & m_{22} & m_{23} \\
      0      & 0      & 1
  \end{bmatrix}
  \begin{bmatrix}
      x^{t} \\ y^{t} \\ 1
  \end{bmatrix}
  \label{eq:method_affine}
\end{equation}

\noindent where $x^*,y^*$ are spatial coordinates, and $\boldsymbol{M}_{t}^{t \minus \tau}$ can be solved by direct linear transform (DLT) algorithm ~\cite{detone2016deep}. 
% with the corner locations of the anchor pair $(o_{a}^{t} \!\sim\! o_{a}^{t \minus \tau})$. 
Then an augmented frame $\tilde{I}^{t}$ is generated based on the tracklet-guided affine transformation with perspective jitter, which can be expressed as $\tilde{I}^{t} = \mathcal{T}\left(I^{t}, M_{t}^{t \minus \tau} \right)$.
% \begin{equation}
%   \tilde{I}^{t} = \mathcal{T}\left(I^{t}, M_{t}^{t \minus \tau} \right)
%   \label{eq:method_aug}
% \end{equation}

Intuitively, a proper anchor-selection is vitally important for our augmentation strategy. 

First, the identity accuracy of anchor pair $(o_{a}^{t} \!\sim\! o_{a}^{t \minus \tau})$ is important. In other words, the consistency of anchor tracklet $trk_{a}$ should be guaranteed. We thus design a tracklet-level uncertain metric based on the propagated association-level uncertainty defined in \cref{eq:method_uncertain}, which is formulated as:

\begin{equation}
  \Omega_{i} = \frac{1}{n} \sum_{s=t_0}^{t} \exp (\delta_{i}^{s})
  % \Omega_{i} = \sqrt[n]{\sigma_{i}^{t_0} \cdot \sigma_{i}^{t_1} \cdots \sigma_{i}^{t}}
  \label{eq:method_tenergy}
\end{equation}

\noindent where $\Omega_{i}$ represents the uncertainty of tracklet $trk_{i}$, \lk{and $n$ is the tracklet length}.
An uncertainty-based sampling strategy is designed to select the source anchor $o_{a}^{t}$ (along with the anchor $trk_{a}$) from the $M^{t}$ objects in frame $I^{t}$, which can be formulated as:

\begin{equation}
  p\left(a=i \mid t \right) 
  % = softmax\left(-\Omega_{i}\right)
  = \frac{\exp{\left(-\Omega_{i}\right)}}{\sum_{\hat{i}=1}^{M^{t}}\exp{\left(-\Omega_{\hat{i}}\right)}}
  \label{eq:method_sel_an_src}
\end{equation}

\noindent where $p\left(a=i \mid t \right)$ represents the probability to choose the $i_{th}$ tracklet $trk_{i}$ as the anchor $trk_{a}$.
The uncertain tracklet with high $\Omega$ is less likely to be selected, avoiding dramatic augmentations from erroneous pseudo-tracklets.

Second, hard negative samples matters in discriminablity learning. We tend to choose an indistinguishable (or, high uncertain) target anchor $o_{a}^{t \minus \tau}$ along the tracklet $trk_{i}$. The selection probability can be formulated as:

\begin{equation}
  p\left(\pi=t \minus \tau \mid a \right) 
  = \frac{\exp{\left(\delta_{a}^{t \minus \tau}\right)}}{\sum_{\hat{\tau}=t_0}^{t-1}\exp{\left(\delta_{a}^{t-\hat{\tau}}\right)}}
  \label{eq:method_sel_an_tgt}
\end{equation}

\lk{A visualization example are displayed in the supplementary material to illustrate the hierarchical sampling process.}

Compared with conventional random transformation, the proposed tracklet-guided augmentation is well-directed and tracking-related. 
\lk{Together with accurate pseudo-tracklets, \mywork~successfully improves the inter-frame consistency, as shown in \cref{fig:method_disc_vis}. }


% Figure environment removed

% \subsection{Momentum Memory Dictionary}
% \label{sec:method_md}


%To reuse the encoded samples from the intermediate mini-batches, we maintain a queue for each video in the memory dictionary by enqueueing the $M^{t}$ objects in the current frame and removing the oldest samples.
%In representation learning, high-quality negative samples play an essential role~\cite{chen2020simple,he2020momentum}. However, existing unsupervised trackers only take negative samples from adjacent frames, augmented frames, and the current frame itself. The lack of negative sample diversity prevents trackers from learning discriminative representations. \mywork~adopts a momentum dictionary mechanism to alleviate this problem.

%As shown in~\cref{fig:method_fmwk}, we build a memory dictionary for each \textit{independent} video input during training. Given an input image $I^{t}$ from video $V$, we randomly sample a number of negative object samples from other videos in the memory dictionary, so as to enrich the negative sample diversity. To reuse the encoded samples from the intermediate mini-batches, we maintain a queue for each video in the memory dictionary by enqueueing the $M^{t}$ objects in the current frame and removing the oldest samples.




% % Figure environment removed


% % Figure environment removed

% % Figure environment removed

% % Figure environment removed

% % Figure environment removed

% % Figure environment removed

% Figure environment removed


% Figure environment removed

% Figure environment removed


\subsection{Implementation Details}


\paragraph{Network.} In order to disentangle shape and color latent information within the hashgrids, we split the single hash table in the NeRF network architecture of Instant-NGP~\cite{mueller2022instant} into two: a density grid $\mathcal{G}^{\sigma}$ and a color grid $\mathcal{G}^c$, with the same settings as the original density grid in the open-source PyTorch implementation torch-ngp~\cite{torch-ngp}. We do this to make it possible to make fine-grained edits of one to one of the color or geometry properties without affecting the other. The rest of the network architecture remains the same, including a sigma MLP $f^\sigma$ and a color MLP $f^c$. For a spatial point $\mathbf{x}$ with view direction $\mathbf{d}$, the network predicts volume density $\sigma$ and color $c$ as follows:
\begin{align}
    \sigma, \mathbf{z} &= f^\sigma(\mathcal{G}^{\sigma}(\mathbf{x})) \\
    c &= f^c(\mathcal{G}^c(\mathbf{x}),\mathbf{z},\mathrm{SH}(\mathbf{d}))
\end{align}
where $\mathbf{z}$ is the intermediate geometry feature, and $\mathrm{SH}$ is the spherical harmonics directional encoder~\cite{mueller2022instant}. The same as Instant-NGP's settings, $f^\sigma$ has 2 layers with hidden channel 64, $f^c$ has 3 layers with hidden channel 64, and $\mathbf{z}$ is a 15-channel feature.

We compare our modified NeRF network with the vanilla architecture in the Lego scene of NeRF Blender Synthetic dataset\cite{mildenhall2020nerf}. We train our network and the vanilla network on the scene for 30,000 iterations. The result is as follows:
\begin{itemize}
    \item Ours: training time 441s, PSNR 35.08dB
    \item Vanilla: training time 408s, PSNR 34.44dB
\end{itemize}
We observe slightly slower runtime and higher quality for our modified architecture, indicating that this modification causes negligible changes.

\paragraph{Training.}
% We use Instant-NGP\fcite{NGP} as our editing framework backbone to achieve real-time editing preview. 
We select Instant-NGP~\cite{mueller2022instant} as the NeRF backbone of our editing framework.
Our implementations are based on the open-source PyTorch implementation torch-ngp~\cite{torch-ngp}. All experiments are run on a single NVIDIA RTX 3090 GPU. Note that we make a slight modification to the original network architecture. Please refer to the supplementary material for details.

During the pretraining stage, we set $\msymbol{weight_pretrain_color}=\msymbol{weight_pretrain_sigma}=1$ and the learning rate is fixed to $0.05$. During the finetuning stage, we set $\msymbol{weight_train_color} = \msymbol{weight_train_depth} = 1$ with an initial learning rate of 0.01. 
% The bit field mask of the editing space is filled so that the editing space can be fully sampled during training. 
Starting from a pretrained NeRF model, we perform 50-100 epochs of local pretraining (for about 0.5-1 seconds) and about 50 epochs of global finetuning (for about 40-60 seconds). The number of epochs and time consumption can be adjusted according to the editing type and the complexity of the scene. Note that we test our performance in the absence of tiny-cuda-nn~\cite{tiny-cuda-nn} which achieves superior speed to our backbone, which indicates that our performance has room for further optimization.
% Note that the training speed is evaluated when tiny-cuda-nn is not enabled.

\paragraph{Datasets.}
We evaluate our editing in the synthetic\Skip{lego, chair, and ship from} NeRF Blender Dataset~\cite{mildenhall2020nerf}, and the real-world captured \Skip{family and truck from}Tanks and Temples~\cite{Knapitsch2017} and \Skip{, and scan83 from} DTU~\cite{jensen2014large} datasets. We follow the official dataset split of the frames for the training and evaluation.


% Figure environment removed

% Figure environment removed

% Figure environment removed

\subsection{Experimental Results}
\label{sec-results}
% \paragraph{Comparisons of rendering quality between teacher and student network.} 


\paragraph{Qualitative NeRF editing results.} 
We provide extensive experimental results in all kinds of editing categories we design, including bounding shape (\cref{fig-bbox,fig-bbox-elf}), brushing (\cref{fig-brush}), anchor (\cref{fig-anchor}), and color (\cref{fig-teaser}). Our method not only achieves a huge performance boost, supporting instant preview at the second level but also produces more visually realistic editing appearances, such as shading effects on the lifted side in \cref{fig-brush} and shadows on the bumped surface in \cref{fig-neumesh}. Besides, results produced by the student network can even outperform the teacher labels, \eg in \cref{fig-bbox-elf} the $F^t$ output contains floating artifacts due to view inconsistency. As analyzed in \cref{sec-train}, the distillation process manages to eliminate this. We also provide an example of object transfer (\cref{fig-bbox-baby}): the bulb in the Lego scene (of Blender dataset) is transferred to the child's head in the family scene of Tanks and Temples dataset.
% \Skip{
% We evaluate our method on all the editing types we design, \ie bounding shape, brushing and anchor, respectively:
% \begin{itemize}
%     \item Bounding shape editing. As shown in \cref{fig-bbox}, we scale the warning light on the top of the Lego model, shorten the chair leg, \zjs{TBD}, and provides plausible results.
%     \item Brushing and color editing. As shown in \cref{fig-brush}, our method edits the scene according to the user's paintings (\ie a cross sign on the chair back, a heart shape on the car logo, and \zjs{TBD}). Note that our brushing method supports simultaneous geometry lifting, as shown in the ``cross'' example. Due to our shading preservation strategy in HSL space, the edited surface can contain realistic visual effects (see the shading effects of the lifted surface).
%     \item Anchor editing. As shown in \cref{fig-anchor}, our method edits the scene according to the anchor points (\ie ship's bow, bulldozer's shovel and \zjs{TBD}) and the stretching direction. The edited geometry has consistent appearance with the anchored area.
% \end{itemize}
% }

% Figure environment removed

% Figure environment removed

\paragraph{Comparisons to baselines.} Existing works have strong restrictions on editing types, which focus on either geometry editing or appearance editing, while ours is capable of doing both simultaneously. Our brushing and anchor tools can create user-guided out-of-proxy geometry structures, which no existing methods support. We make comparisons on color and texture painting supported by NeuMesh~\cite{neumesh} and Liu \etal~\cite{liu2021editing}. 

\cref{fig-neumesh} illustrates two comparisons between our method and NeuMesh~\cite{neumesh} in scribbling and a texture painting task. Our method significantly outperforms NeuMesh, which contains noticeable color bias and artifacts in the results. In contrast, our method even succeeds in rendering the shadow effects caused by geometric bumps.

\cref{fig-neumesh-mic} illustrates the results of the same non-rigid blending applied to the Mic from NeRF Blender\cite{mildenhall2020nerf}. It clearly shows that being mesh-free, We have more details than NeuMesh\cite{neumesh}, unlimited by mesh resolution.

\cref{fig-editnerf} shows an overview of the pixel-wise editing ability of existing NeRF editing methods and ours. Liu \etal~\cite{liu2021editing}'s method does not focus on the pixel-wise editing task and only supports textureless simple objects in their paper. Their method causes an overall color deterioration within the edited object, which is highly unfavorable. This is because their latent code only models the global color feature of the scene instead of fine-grained local features. Our method supports fine-grained local edits due to our local-aware embedding grids.

% \yq{describe the difference}

% \paragraph{Artistic applications (a comic on NeRF).} Based on the four example tools we implemented, we created a comic \textit{Bob the Bulb} (Fig. \ref{fig-comic}) to show the potential applications of our editing method. This might be the first artwork created with NeRF.

% \subsection{Experiments on Bounding Shape Editing}
% \subsection{Experiments on Brush Editing}
% \subsection{Experiments on Anchor Editing}
% \subsection{Experiments on Color Shape Editing}
% \subsection{Real-world Example: NeRF-Rendered Comic}


\subsection{Ablation Studies}
\label{sec-ablation}
\paragraph{Effect of the two-stage training strategy.} To validate the effectiveness of our pretraining and finetuning strategy, we make comparisons between our full strategy (3\textsuperscript{rd} row), finetuning-only (1\textsuperscript{st} row) and pretraining-only (2\textsuperscript{nd} row) in \cref{fig-ab_pre}. Our pretraining can produce a coarse result in only 1 second, while photometric finetuning can hardly change the appearance in such a short period. The pretraining stage also enhances the subsequent finetuning, in 30 seconds our full strategy produces a more complete result. However, pretraining has a side effect of local overfitting and global degradation. Therefore, our two-stage strategy makes a good balance between both and produces optimal results.

\paragraph{MLP fixing in the pretraining stage.} In \cref{fig-ab_fix}, we validate our design of fixing all MLP parameters in the pretraining stage. The result confirms our analysis that MLP mainly contains global information so it leads to global degeneration when MLP decoders are not fixed.




We proposed a machine-learning based method to approximate diagonal as well as non-diagonal elements of the Hessian of a molecule. The representation used is specific for every internal coordinates, and takes explicitly into account the bond order, which is sensible because a single point DFT calculation is computationally considerably less expensive that the explicit calculation of the Hessian.
We trained our ML model on a relatively small dataset (subset of QM7) of less than 7000 molecules. The Hessian was computed at the B3LYP/cc-pVDZ level of theory. 
The agreement between ML and DFT was satisfactory. In particular, the calculated MAPE for bond stretching force constant was below 2\%, and was particularly small for bonds involving hydrogen atoms because they point outwards and are less affected by the chemical environment. The MAPE for bending and torsion was of 5\% and 10\%, respectively. 
From the ML model trained on QM7 we were also able to predict the Hessian of some molecules representative of the QM9 dataset. The Hessian predicted in internal coordinates was then transformed into the mass-weighted Cartesian Hessian, the diagonalization of which yields the harmonic vibrational frequencies and normal modes, that can be compared to the ones calculated  explicitly from DFT.

High frequency vibrations and normal modes were predicted accurately, while lower frequency ones were not. This behaviour is analogous to the IR spectroscopy theory, where stretchings and bendings can be identified accurately, while torsion and delocalized vibrations are more difficult to be interpreted.

The approximate Hessian obtained with ML is computational inexpensive and can be used as an initial Hessian guess for geometry optimization, or in the context of alchemical geometry relaxation \cite{Domenichini2020,domenichini2022alchemical, shiraogawa2022exploration,shiraogawa2023optimization}. 
A good starting Hessian may speed up the convergence of the geometrical optimization. An in detail assessment of the performance of the ML Hessian proposed is not yet provided, but should carefully take into account many parameters on which the optimization depends, \textit{e.g.} the type of molecule, the initial geometry, the optimization algorithm, and the Hessian update scheme.



\newpage


%\section*{Impact Statement}\label{sec:impact-statement}
%This paper proposes UniAP to advance the field of deep learning infrastructure.
%We have not found immediate ethical concerns or negative societal consequences of our work.
%Therefore, none of the negative consequences must be specifically highlighted.

%However, training a Transformer-based model often requires a significant amount of energy.
%UniAP depicts the opportunity to greatly accelerate the training process, thereby minimizing energy consumption as much as possible.
%Hence, we feel the necessity to highlight its positive environmental impact here.
% Therefore, we do not feel the necessity to highlight them here.


\input{bibiliography.bbl}


\newpage
\appendix
\onecolumn

\section{Proof of the Linear Form for the Contiguous Set}\label{appendix:linear-form-of-contigous-constraint}
To facilitate our discussion, we adopt the linear form of the order-preserving constraint as presented in the main paper. We denote $P_{ui}$ as a 0-1 variable indicating whether layer $u$ is to be placed on the $i$-th computation stage, $deg$ as the number of computation stages in the pipeline. Besides, $\mathcal{G}(V, E)$ represents the computation graph for the model. Then, we formalize the theorem as follows:

\begin{theorem}
    A subgraph $V_i=\{\forall u\in V: P_{ui}=1\}$ is contiguous if and only if there exists $Z_{vi}$ such that \eqref{eqn:method:order-preserving:1}, \eqref{eqn:method:order-preserving:2}, and \eqref{eqn:method:order-preserving:3} are satisfied.
\end{theorem}

Previous work \citep{tarnawski_efficient_2020} has proven this theorem. Our proof draws on the process of this work. The details of the proof are as follows:

\begin{proof}
\vskip -0.1in
    "If": Assume that there exists nodes $u, w\in V_i$ and $v \notin V_i$ such that $v$ and $w$ are reachable from $u$ and $v$, respectively. Hence, $P_{ui} = 1$, $P_{wi} = 1$, and $P_{vi} = 0$. Without losing generality, we assume $\langle u, v\rangle\in E$. Thus, according to \eqref{eqn:method:order-preserving:3}, we have $Z_{vi}\leqslant P_{vi}-P_{ui}+1=0$. By applying \eqref{eqn:method:order-preserving:2} repeatedly following the path from $v$ to $w$, we have $Z_{wi}\leqslant Z_{vi}$. Thus, $Z_{wi}\leqslant 0$. However, we also have $Z_{wi}\geqslant P_{wi}=1$ according to \eqref{eqn:method:order-preserving:1}. A contradiction.

    "Only if": First, we define $Z_{vi}=1$ if a node $w\in S$ is reachable from $v$. Otherwise, $Z_{vi}=0$. Thus, \eqref{eqn:method:order-preserving:1} and \eqref{eqn:method:order-preserving:2} are satisfied according to this kind of definition. For \eqref{eqn:method:order-preserving:3}, if $P_{vi}=1$, the constraint will hold true regardless of whether $P_{ui}$ is $1$ or $0$. If $P_{vi}=0$ and $P_{ui}=0$, $Z_{vi}\leqslant P_{vi}-P_{ui}+1=1$ will also hold true because $Z_{vi}$ could be either $0$ or $1$. Finally, if $P_{vi}=0$ and $P_{ui}=1$, $Z_{vi}=0$ will hold true because $V_i$ is a contiguous set and we cannot find any $w\in V_i$, such that $w$ is reachable from $v$.
\end{proof}

\section{QIP Formulation for Intra-layer-only Parallelism}\label{appendix:miqp-for-intra-layer-parallelism}
Here we present the QIP formulation for intra-layer-only parallelism with explanations.

\paragraph{Objective function} In terms of intra-layer-only parallelism, there is only one computation stage involved. As a result, the objective function takes into account only the value of $p_1$. We hereby formalize the equation as
\begin{equation}
    \min\quad tpi_{gpipe}=p_1.\label{eqn:appendix:lp-no-pp}
\end{equation}

\paragraph{Computation-stage constraint} With only one computation stage in intra-layer-only parallelism, the communication-stage constraint can be omitted, and the computation and communication cost can be modeled for $p_1$. Thus, we could formalize the constraint as
\begin{equation}
\sum_{u\in V}S_{u}^\mathsf{T}A_{u}+\sum_{\langle u,v\rangle\in E}S_{u}^\mathsf{T}R_{uv}S_{v}=p_1.~\label{eqn:appendix:computation-for-no-pp}
\end{equation}
In the equation, the first summation term for any $u\in V$ represents the cost of choosing intra-layer strategies for all layers, while the second term represents the summation of resharding costs on all edges.


\paragraph{Memory constraint} Similar to the memory constraint in MIQP, it is necessary to ensure that the memory usage on a single device does not exceed its device memory bound $m$ in QIP. This restriction gives 
\begin{equation}
    \sum_{u\in V}S_u^\mathsf{T} M_u\leqslant m.~\label{eqn:appendix:memory-constraint-for-no-pp}
\end{equation}
It is worth noting that $m$ should be an identical constant across multiple devices if these devices are homogeneous. Otherwise, the value of $m$ varies.

\paragraph{Strategy-selection constraint} For intra-layer-only parallelism, the layer-placement constraint can be safely omitted because it is designed for PP. However, the strategy-selection constraint is necessary because each layer can only select one intra-layer strategy. Therefore, the strategy-selection constraint for QIP is identical to \eqref{eqn:method:strategy-selection:1} and \eqref{eqn:method:strategy-selection:2} for MIQP.

By combining objective function \eqref{eqn:appendix:lp-no-pp} and constraints \eqref{eqn:method:strategy-selection:1}, \eqref{eqn:method:strategy-selection:2}, \eqref{eqn:appendix:computation-for-no-pp}, and \eqref{eqn:appendix:memory-constraint-for-no-pp}, we have the QIP expression for optimizing the intra-layer-only AP. Like MIQP expression for optimizing the inter- and intra-layer AP, UniAP will eventually get the minimum TPI and corresponding parallel strategies by invoking the off-the-shelf solver.

\section{Visualization for the Candidate Solution}\label{appendix:visualization_for_p_and_s}
% Figure environment removed

In this section, we proceed to visually represent a potential solution for UOP. Given a deep learning model $\mathcal{G}$, pipeline degree $deg$, and number of micro-batches $c$, UniAP will determine layer placement $P$ for inter-layer parallelism and parallel strategy $S$ for intra-layer parallelism using an off-the-shelf solver. As Figure \ref{fig:ps_explanation} shows, the solver is optimizing a three-layer model with two pipeline stages, each assigned four GPUs. At this time, a candidate solution could be
\begin{equation}
    P=\begin{bmatrix}
        1 & 0 \\
        1 & 0 \\
        0 & 1
    \end{bmatrix},~
    S=
    \begin{bmatrix}
        1 & 0 & 0 \\
        0 & 1 & 1 \\
        0 & 0 & 0 \\ 
        \vdots & \vdots & \vdots \\
        0 & 0 & 0
    \end{bmatrix}.
\end{equation}

Here, the $u$-th row of matrix $P$ denotes the placement strategy for layer $u$, where $P_{ui}=1$ signifies the placement of layer $u$ on stage $i$, while $0$ indicates otherwise. For example, $P_{l_{0}}=\left[1,~0\right]$ denotes the placement of layer $l_0$ on pipeline stage 1. Additionally, the $u$-th column of matrix $S$ denotes the selected intra-layer parallel strategy for layer $u$, where $S_{uj}=1$ denotes the selection of the $j$-th strategy from the intra-layer parallel strategy set. For example, $S_{l_{0}}=\left[1,~0,~0,~\cdots,~0\right]^{\mathsf{T}}$ indicates that layer $l_0$ will adopt only the DP strategy, while $S_{l_{1}}=\left[0,~1,~0,~\cdots,~0\right]^{\mathsf{T}}$ indicates that layer $l_1$ will employ a strategy where DP is performed on GPU 0, 1 and GPU 2, 3, and TP is performed across these two GPU groups.

There exist numerous combinations of $P$ and $S$. The off-the-shelf solver will automatically search for the optimal solution given pipeline degree $deg$ and the number of micro-batches $c$. By solving the MIQP expression and enumerating every possible $deg$ and $c$ in the UOP process, UniAP will ultimately derive an optimal parallel strategy for the deep learning model within the current hardware environment.

\section{Experiment Detail}\label{appendix:experiment-settings}
\paragraph{Gurobi configuration} When tackling the MIQP problem, UniAP employs several configurations for the Gurobi Optimizer 10.1~\citep{gurobi_optimization_llc_gurobi_2023}. In particular, we set \textit{TimeLimit} to 60 seconds, \textit{MIPFocus} to 1, \textit{NumericFocus} to 1, and remain other configurations to default. For instance, we establish the \textit{MIPGap} parameter as the default value of 1e-4 to serve as a strict termination criterion. Furthermore, we have implemented an early stopping mechanism to terminate the optimization process as early as possible. There are two conditions that can activate the mechanism. Firstly, if the current runtime exceeds 15 seconds and the relative MIP optimality gap is less than 4\%, we will terminate the optimization. Secondly, if the current runtime exceeds 5 seconds and the best objective bound is worse than the optimal solution obtained in the previous optimization process, we will terminate the optimization.

\begin{table}[t]
\caption{Details for five Transformer-based models. L: Number of hidden layers; H: Hidden size; S: Sequence length.}
\label{tab:appendix-details-for-five-transformer-based-models}
\vskip 0.15in
\begin{center}
\begin{small}
\begin{sc}
    \begin{tabular}{ccccc}
    \toprule
    Model & L & H & S & \#params \\
    \midrule
    BERT-Huge & 32    & 1280  & 512   & 672M \\
    T5-Large & 16/16 & 1024  & 512   & 502M \\
    ViT-Huge & 32    & 1280  & 196 & 632M \\
    Swin-Huge & 2/2/42/2 & 320/640/1280/2560 & 49*64/49*16/49*4/49*1 & 1.02B \\
    LLaMA-7B & 32 & 4096 & 2048 & 7B \\
    \bottomrule
    \end{tabular}
\end{sc}
\end{small}
\end{center}
\vskip -0.1in
\end{table}

\paragraph{Model detail} Table~\ref{tab:appendix-details-for-five-transformer-based-models} presents the details of five Transformer-based models selected for our evaluations. Three of these models, namely BERT-Huge~\citep{devlin_bert_2019}, T5-Large~\citep{raffel_exploring_2020}, and LLaMA-7B~\citep{touvron_llama_2023,touvron_llama_2023-1}, belong to the domain of natural language processing (NLP). At the same time, the remaining two, ViT-Huge~\citep{dosovitskiy_image_2021} and Swin-Huge~\citep{liu_swin_2021}, are associated with computer vision (CV). It is noteworthy that BERT, ViT, and LLaMA maintain consistent types of hidden layers respectively, whereas T5 and Swin have different types of hidden layers. Numbers separated by slashes represent the statistical information for different layer types. For instance, Swin-Huge comprises four types of layers, each with 2, 2, 42, and 2 layers, respectively.

\paragraph{Training detail} UniAP is based on the PyTorch framework and integrates models from HuggingFace Transformers. It employs various types of parallelism, including Pipeline Parallelism~(PP), Data Parallelism~(DP), Tensor Parallelism~(TP), and Fully Sharded Data Parallelism~(FSDP), utilizing GPipe~\citep{huang_gpipe_2019}, PyTorch DDP~\citep{li_pytorch_2020}, Megatron-LM~\citep{narayanan_efficient_2021}, and FairScale~\citep{fairscale_authors_fairscale_2021}, respectively. For NLP models, we use the English Wikipedia dataset~\citep{wikidump}, while the ImageNet-1K dataset~\citep{imagenet15russakovsky} is used for CV models. We train these models using the Adam optimizer~\citep{kingma_adam_2015} and precision of FP32. We omit hyperparameters here such as learning rate and weight decay as these have minimal impact on training throughput. 
The model parameters in the HuggingFace Transformer are configured to align with the specifications of each individual model. For instance, we set \textit{hidden\_size} to 1280, \textit{num\_hidden\_layers} to 32, \textit{num\_attention\_heads} to 16, and \textit{seq\_length} to 512 for BERT-Huge. Regarding other hyperparameters in the HuggingFace configurations, we set \textit{hidden\_dropout\_prob} and \textit{attention\_probs\_dropout\_prob} to 0.0 for ViT-Huge. For Swin-Huge, we set \textit{drop\_path\_rate} to 0.2. We remain other configurations to default. It should be noted that the training batch sizes for each experiment are outlined in the main paper.

\section{Case study: BERT-Huge}\label{appendix:case-study}
% Figure environment removed
In this section, we present a visualization of the optimal parallel strategy discovered by UniAP. As represented in Figure \ref{fig:visualization-model}, the strategy pertains to training BERT-Huge with 32 hidden layers in a 2-node environment \textit{EnvB} with a mini-batch size of 16. Each node was equipped with 2 Xeon E5-2620 v4 CPUs, 4 TITAN Xp 12GB GPUs, and 125GB memory. These nodes are interconnected via a 10Gbps network. It should be noted that we only showcase the parallel strategy for the hidden layers here for simplicity but without losing generality.

Here, we provide further topological information for a node within \textit{EnvB}. As illustrated in Figure~\ref{fig:topo}, we categorize the GPUs numbered 0 and 1 in each node and refer to them collectively as \textit{GPUGroup0}. Similarly, we label the GPUs numbered 2 and 3 as \textit{GPUGroup1}. In \textit{EnvB}, the interconnects within each GPU group (i.e., PCIe) have superior bandwidth than that between different groups (i.e., QPI). We collectively designate these two connection bandwidths as intra-node bandwidth, which is higher than inter-node bandwidth.

% Figure environment removed

In this example, UniAP has identified a parallel strategy for inter-layer parallelism that involves a two-stage pipeline. This strategy utilizes parallelism in a manner that is both efficient and effective. Specifically, the communication cost of point-to-point~(P2P) between two nodes is less than that of all-reduce. Additionally, the inter-node bandwidth is lower than that of the intra-node. These factors make the two-stage PP a reasonable choice. Moreover, the pipeline has been designed such that each stage comprises an equal number of layers. This design leverages the homogeneity of the nodes and ensures load balancing across the cluster.

UniAP employs an intra-layer parallel strategy within each PP stage. It utilizes a 2-way DP for the initial 12 hidden layers in each stage between \textit{GPUGroup0} and \textit{GPUGroup1}. For the remaining four hidden layers, a 2-way FSDP is utilized between \textit{GPUGroup0} and \textit{GPUGroup1} to reduce memory footprint and meet memory constraints. Within each GPU group, UniAP employs a 2-way TP for each layer. In general, TP incurs more significant communication volumes than DP and FSDP. In order to achieve maximum training throughput on \textit{EnvB}, it is necessary to implement parallel strategies that prioritize higher communication volumes within each group and lower volumes between groups. Therefore, the strategy for BERT-Huge with 32 hidden layers combines the best elements of PP, DP, TP, and FSDP to maximize training throughput.

In addition, we have conducted calculations for the model FLOPs utilizatio~(MFU) for Galvatron, Alpa, and UniAP in this scenario to validate our analysis. MFU is a metric introduced by \citet{chowdhery_palm_2023}, which is independent of hardware, frameworks, or implementations. Therefore, it allows us to examine the performance of different parallel strategies solely from a strategic perspective. For BERT-Huge-32, the resulting MFUs for UniAP, Galvatron, and Alpa on \textit{EnvB} are 23.6\%, 13.7\% and 19.6\%
, respectively. Therefore, we conclude that UniAP does utilize its larger strategy space to find an optimal solution, rather than a sub-optimal one.



\end{document}
