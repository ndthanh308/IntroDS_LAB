
\PassOptionsToPackage{table,xcdraw}{xcolor}
% \usepackage[table,xcdraw]{xcolor}

% \documentclass[acmtog, sigconf]{acmart}
\documentclass[sigconf]{acmart}
\usepackage{textcomp}
\settopmatter{printacmref=false}
\let\oldftcp\footnotetextcopyrightpermission
\renewcommand\footnotetextcopyrightpermission[1]{\oldftcp{%
% \textcopyright{}
}}


%%
%% \BibTeX command to typeset BibTeX logo in the docs
\AtBeginDocument{%
  \providecommand\BibTeX{{%
    Bib\TeX}}}

%% Rights management information.  This information is sent to you
%% when you complete the rights form.  These commands have SAMPLE
%% values in them; it is your responsibility as an author to replace
%% the commands and values with those provided to you when you
%% complete the rights form.
% \setcopyright{acmcopyright}
% \copyrightyear{2018}
% \acmYear{2018}
% \acmDOI{XXXXXXX.XXXXXXX}

%% These commands are for a PROCEEDINGS abstract or paper.
\acmConference[ACM REP’23]{The ACM Conference on Reproducibility and Replicability}{June 27--29,2023}{Santa Cruz, California}

%%
%%  Uncomment \acmBooktitle if the title of the proceedings is different
%%  from ``Proceedings of ...''!
%%
%%\acmBooktitle{Woodstock '18: ACM Symposium on Neural Gaze Detection,
%%  June 03--05, 2018, Woodstock, NY}
% \acmPrice{15.00}
% \acmISBN{978-1-4503-XXXX-X/18/06}


%%
%% Submission ID.
%% Use this when submitting an article to a sponsored event. You'll
%% receive a unique submission ID from the organizers
%% of the event, and this ID should be used as the parameter to this command.
%%\acmSubmissionID{123-A56-BU3}

%%
%% For managing citations, it is recommended to use bibliography
%% files in BibTeX format.
%%
%% You can then either use BibTeX with the ACM-Reference-Format style,
%% or BibLaTeX with the acmnumeric or acmauthoryear sytles, that include
%% support for advanced citation of software artefact from the
%% biblatex-software package, also separately available on CTAN.
%%
%% Look at the sample-*-biblatex.tex files for templates showcasing
%% the biblatex styles.
%%

%%
%% The majority of ACM publications use numbered citations and
%% references.  The command \citestyle{authoryear} switches to the
%% "author year" style.
%%
%% If you are preparing content for an event
%% sponsored by ACM SIGGRAPH, you must use the "author year" style of
%% citations and references.
%% Uncommenting
%% the next command will enable that style.
%%\citestyle{acmauthoryear}



\usepackage{graphicx}

\usepackage{multirow}
\usepackage{caption}
\usepackage{subcaption}
\usepackage{lipsum} 
\usepackage{listings}
\usepackage{framed}
\usepackage{amsmath}
\usepackage{balance}

\usepackage{enumerate}
\usepackage[shortlabels]{enumitem}

\definecolor{codepurple}{rgb}{0.58,0,0.82}
\definecolor{backcolour}{RGB}{239, 239, 239}
\definecolor{codeorange}{RGB}{191, 94, 45}
\definecolor{codeblue}{RGB}{0, 0, 255}
\definecolor{codegreenl}{rgb}{0,0.6,0}
\definecolor{codegreend}{RGB}{101, 139, 111}
\definecolor{codegray}{rgb}{0.5,0.5,0.5}


\lstset{
  emph={prepare, launch, finalize, fetch, tasks, depends_on, hosts_selector, app_selector, conf_selector, grouping, prefix, hosts, copy, src, dest, shell, poll, loop, async, dst, rate, loss, delay, default_rate, default_loss, default_delay, def_rate, def_loss, def_delay, networks, environment, name, site, cluster, layers, services, quantity, roles, env, iotlab, g5k, archi, validate_checksum},
  emphstyle={\color{codeorange}}%
}

 
% https://www.overleaf.com/learn/latex/code_listing#Reference_guide
% http://texdoc.net/texmf-dist/doc/latex/listings/listings.pdf
\lstdefinestyle{mystyle}{
    backgroundcolor=\color{backcolour},   
    commentstyle=\color{codepurple},
    keywordstyle=\color{codeorange},
    numberstyle=\small\color{codegray},
    stringstyle=\color{codegreenl},
    % basicstyle=\ttfamily\footnotesize,
    basicstyle=\fontsize{8}{8}\selectfont\ttfamily,
    breakatwhitespace=false,     
    breaklines=true,                 
    captionpos=b,                    
    keepspaces=true,                 
    numbers=none,                    
    numbersep=2pt,        
    numbers=left,
    stepnumber=1,
    showspaces=false,                
    showstringspaces=false,
    showtabs=false,                  
    tabsize=2
}
\lstset{style=mystyle}


%%
%% end of the preamble, start of the body of the document source.


% \copyrightyear{2023}
% \acmYear{2023}
% \setcopyright{acmlicensed}\acmConference[ACM REP '23]{2023 ACM Conference on Reproducibility and Replicability}{June 27--29, 2023}{Santa Cruz, CA, USA}
% \acmBooktitle{2023 ACM Conference on Reproducibility and Replicability (ACM REP '23), June 27--29, 2023, Santa Cruz, CA, USA}
% \acmPrice{15.00}
% \acmDOI{10.1145/3589806.3600032}
% \acmISBN{979-8-4007-0176-4/23/06}



\begin{document}

% \title{KheOps: A Collaborative Environment for Cost-effective Reproducibility of Edge-to-Cloud Experiments}

\title{KheOps: Cost-effective Repeatability, Reproducibility, and Replicability of Edge-to-Cloud Experiments}



%%
%% The "author" command and its associated commands are used to define
%% the authors and their affiliations.
%% Of note is the shared affiliation of the first two authors, and the
%% "authornote" and "authornotemark" commands
%% used to denote shared contribution to the research.


\author{Daniel Rosendo}
\email{daniel.rosendo@inria.fr}
% \orcid{0000-0003-1175-8426}
\affiliation{
 \institution{Univ Rennes, Inria, CNRS, , IRISA}
 \city{Rennes}
 \country{France}}

\author{Kate Keahey}
\email{keahey@mcs.anl.gov}
% \orcid{0000-0002-5251-5466}
\affiliation{
 \institution{Argonne National Laboratory}
 \city{Chicago}
 \country{USA}}

\author{Alexandru Costan}
\email{alexandru.costan@inria.fr}
% \orcid{0000-0003-3111-6308}
\affiliation{
 \institution{Univ Rennes, Inria, CNRS, IRISA}
 \city{Rennes}
 \country{France}}

\author{Matthieu Simonin}
\email{matthieu.simonin@inria.fr}
% \orcid{0000-0002-9063-0334}
\affiliation{
\institution{Univ Rennes, Inria, CNRS, IRISA}
 \city{Rennes}
 \country{France}}

\author{Patrick Valduriez}
\email{patrick.valduriez@inria.fr}
% \orcid{0000-0001-6506-7538}
\affiliation{
 \institution{Univ Montpellier, Inria, CNRS, LIRMM}
 \city{Montpellier}
 \country{France}}

\author{Gabriel Antoniu}
\email{gabriel.antoniu@inria.fr}
% \orcid{0000-0001-6525-3736}
\affiliation{
 \institution{Univ Rennes, Inria, CNRS, IRISA}
 \city{Rennes}
 \country{France}}

 
%%
%% By default, the full list of authors will be used in the page
%% headers. Often, this list is too long, and will overlap
%% other information printed in the page headers. This command allows
%% the author to define a more concise list
%% of authors' names for this purpose.
% \renewcommand{\shortauthors}{Rosendo, et al.}


%%
%% The abstract is a short summary of the work to be presented in the
%% article.
\begin{abstract}
Distributed infrastructures for computation and analytics are now evolving towards an interconnected ecosystem allowing complex scientific workflows to be executed across hybrid systems spanning from IoT Edge devices to Clouds, and sometimes to supercomputers (the Computing Continuum). Understanding the performance trade-offs of large-scale workflows deployed on such complex Edge-to-Cloud Continuum is challenging. To achieve this, one needs to systematically perform experiments, to enable their reproducibility and allow other researchers to replicate the study and the obtained conclusions on different infrastructures. This breaks down to the tedious process of reconciling the numerous experimental requirements and constraints with low-level infrastructure design choices. 

To address the limitations of the main state-of-the-art approaches for distributed, collaborative experimentation, such as Google Colab, Kaggle, and Code Ocean, we propose KheOps, a collaborative environment specifically designed to enable cost-effective reproducibility and replicability of Edge-to-Cloud experiments. KheOps is composed of three core elements: (1) an experiment repository; (2) a notebook environment; and (3) a multi-platform experiment methodology.


% Based on the limitations of the main state-of-the-art approaches like Google Colab, Kaggle, and Code Ocean, we propose KheOps, a collaborative environment for the cost-effective reproducibility and replicability of Edge-to-Cloud experiments. KheOps is composed of three core elements to enable reproducible Computing Continuum research: (1) Trovi portal: for sharing experiment artifacts; (2) Jupyter environment: for packaging code, data, environment, and results; and (3) Multi-platform experiment methodology: for abstracting all the complexities to deploy workflows on large-scale scientific testbeds with heterogeneous resources, such as Grid5000, Chameleon, FIT IoT lab, and CHI@Edge.

We illustrate KheOps with a real-life Edge-to-Cloud application. The evaluations explore the point of view of the authors of an experiment described in an article (who aim to make their experiments reproducible) and the perspective of their readers (who aim to replicate the experiment). The results show how KheOps helps authors to systematically perform repeatable and reproducible experiments on the Grid5000 + FIT IoT LAB testbeds. Furthermore, KheOps helps readers to cost-effectively replicate authors experiments in different infrastructures such as Chameleon Cloud + CHI@Edge testbeds, and obtain the same conclusions with high accuracies ($>$88\% for all performance metrics).
\end{abstract}

%%
%% The code below is generated by the tool at http://dl.acm.org/ccs.cfm.
%% Please copy and paste the code instead of the example below.
%%
\begin{CCSXML}
<ccs2012>
   <concept>
       <concept_id>10010147</concept_id>
       <concept_desc>Computing methodologies</concept_desc>
       <concept_significance>500</concept_significance>
       </concept>
    <concept>
       <concept_id>10010147.10010919</concept_id>
       <concept_desc>Computing methodologies~Distributed computing methodologies</concept_desc>
       <concept_significance>500</concept_significance>
       </concept>
    <concept>
        <concept_id>10002944.10011123.10011131</concept_id>
        <concept_desc>General and reference~Experimentation</concept_desc>
        <concept_significance>500</concept_significance>
        </concept>
    <concept>
        <concept_id>10002944.10011123.10010916</concept_id>
        <concept_desc>General and reference~Measurement</concept_desc>
        <concept_significance>500</concept_significance>
    </concept>
 </ccs2012>
\end{CCSXML}

\ccsdesc[500]{Computing methodologies}
\ccsdesc[500]{Computing methodologies~Distributed computing methodologies}
\ccsdesc[500]{General and reference~Experimentation}
\ccsdesc[500]{General and reference~Measurement}

%%
%% Keywords. The author(s) should pick words that accurately describe
%% the work being presented. Separate the keywords with commas.
\keywords{Reproducibility, Replicability, Repeatability, Computing Continuum, Workflows, Edge Computing, Cloud Computing}


% \received{27 January 2023}
% \received[revised]{12 March 2009}
% \received[accepted]{29 March 2023}

%%
%% This command processes the author and affiliation and title
%% information and builds the first part of the formatted document.
\maketitle


\section{Introduction}
\label{sec:introduction}

\section{Introduction}
Deep learning models have been widely used in many applications.
For example, BERT~\citep{devlin_bert_2019}, GPT-3~\citep{brown_language_2020}, and T5~\citep{raffel_exploring_2020} achieved state-of-the-art~(SOTA) results on different natural language processing~(NLP) tasks. 
For computer vision~(CV), Transformer-like models such as ViT~\citep{dosovitskiy_image_2021} and Swin Transformer~\citep{liu_swin_2021} deliver excellent accuracy performance upon multiple tasks. 


At the same time, training deep learning models has been a critical problem troubling the community due to the long training time, especially for those large models with billions of parameters~\citep{brown_language_2020}. 
In order to enhance the training efficiency, researchers propose some manually designed parallel training strategies~\citep{narayanan_efficient_2021,shazeer_mesh-tensorflow_2018,xu_gspmd_2021}. 
However, selecting, tuning, and combining these strategies require extensive domain knowledge in deep learning models and hardware environments. With the increasing diversity of modern hardware architectures~\cite{flynn_very_1966,flynn_computer_1972} and the rapid development of deep learning models, these manually designed approaches are bringing heavier burdens to developers. 
Hence, \emph{automatic parallelism} is introduced to automate the parallel strategy searching for training models.


There are two main categories of parallelism in deep learning models: inter-layer parallelism~\citep{huang_gpipe_2019,narayanan_pipedream_2019,narayanan_memory-efficient_2021,fan_dapple_2021,li_chimera_2021,lepikhin_gshard_2021,du_glam_2022,fedus_switch_2022} and intra-layer parallelism~\citep{li_pytorch_2020,narayanan_efficient_2021,rasley_deepspeed_2020,fairscale_authors_fairscale_2021}. 
Inter-layer parallelism partitions the model into disjoint sets on different devices without slicing tensors. 
Alternatively, intra-layer parallelism partitions tensors in a layer along one or more axes and distributes them across different devices.


Current automatic parallelism techniques focus on optimizing strategies within these two categories. However, they treat these two categories separately. 
Some methods~\citep{zhao_vpipe_2022,jia_exploring_2018,cai_tensoropt_2022,wang_supporting_2019,jia_beyond_2019,schaarschmidt_automap_2021,liu_colossal-auto_2023} overlook potential opportunities for inter- or intra-layer parallelism, the others optimize inter- and intra-layer parallelism hierarchically and sequentially~\citep{narayanan_pipedream_2019,fan_dapple_2021,he_pipetransformer_2021,tarnawski_efficient_2020,tarnawski_piper_2021,zheng_alpa_2022}. 
As a result, current automatic parallelism techniques often fail to achieve the global optima and instead become trapped in local optima. 
Therefore, a unified inter- and intra-layer approach is needed to enhance the effectiveness of automatic parallelism.


This paper aims to find the optimal parallelism strategy while simultaneously considering inter- and intra-layer parallelism. 
It enables us to search in a more extensive strategy space where the globally optimal solution lurk. 
However, unifying inter- and intra-layer parallelism in automatic parallelism brings us two challenges. 
Firstly, to adopt a unified perspective on the inter- and intra-layer automatic parallelism, we should not formalize them with separate formulations as prior works. Therefore, how can we express these parallelism strategies in a unified formulation? 
Secondly, previous methods take a long time to obtain the solution with a limited strategy space. Therefore, how can we ensure that the best solution can be obtained in a reasonable time while expanding the strategy space?


To solve the above challenges, we propose UniAP. For the first challenge, UniAP adopts the mixed integer quadratic programming~(MIQP)~\citep{lazimy_mixed_1982} to search for the globally optimal parallel strategy automatically. 
It unifies the inter- and intra-layer automatic parallelism in a single MIQP formulation. 
For the second challenge, our complexity analysis and experimental results show that UniAP can obtain the globally optimal solution in a significantly shorter time.


The contributions of this paper are summarized as follows: 
\begin{itemize}
    \item We propose UniAP, the first framework to unify inter- and intra-layer automatic parallelism in model training.
    \item The optimal parallel strategies discovered by UniAP exhibit scalability on training throughput and strategy searching time.
    \item The experimental results show that UniAP speeds up model training on four Transformer-like models by up to 1.70$\times$ and reduces the strategy searching time by up to 16$\times$, compared with the SOTA method.
\end{itemize}


\section{Background}
\label{sec:background}
\section{Background}
\subsection{Parallel Strategy}
\label{subsec:background:parallel-strtegy}
\paragraph{Pipeline parallelism~(PP)} In PP, each worker~(machine or GPU) holds a subset of model layers. Adjacent layers on different workers need to transfer activations in the forward propagation~(FP) step and gradients in the backward propagation~(BP) step. 
\paragraph{Data parallelism~(DP)} In DP, each worker holds a replica of the whole model and partitions training samples. In each iteration, each worker computes gradients and synchronizes them with the other workers using all-reduce collective communication~(CC). All workers will have the same model parameters after the synchronization step.
\paragraph{Tensor parallelism~(TP)} In TP, each worker holds a replica of training samples and partitions within model layers. In each iteration, each worker computes its local outputs in FP and its local gradients in BP. To synchronize outputs and gradients, all workers will perform all-reduce CC in FP and BP steps according to the partition scheme.
\paragraph{Fully sharded data parallelism~(FSDP)} FSDP partitions optimizer states, parameters and gradients of the model into separate workers. During the FP and BP step of each iteration, FSDP performs an all-gather CC to obtain the complete parameters for the relevant layer, respectively. After computing the gradients, FSDP conducts a reduce-scatter CC to distribute the global gradients among the workers.

\subsection{Manual Parallelism}
MP refers to the parallel methods in which human experts design and optimize the parallel strategies. Representative MP methods include Megatron-LM~\citep{narayanan_efficient_2021}, Mesh-TensorFlow~\citep{shazeer_mesh-tensorflow_2018}, and GSPMD~\citep{xu_gspmd_2021}. Megatron-LM manually designs TP and PP strategies for training Transformer-based models and exhibits superior efficiency. Mesh-TensorFlow and GSPMD require human effort to designate and tune the intra-layer parallel strategy. These methods rely on expert design and have little flexibility, challenging their automatic application to other models.

\subsection{Automatic Parallelism}
\paragraph{Inter-layer-only AP or intra-layer-only AP} For inter-layer-only AP, GPipe~\citep{huang_gpipe_2019} and vPipe~\citep{zhao_vpipe_2022} employ a balanced partition algorithm and a dynamic layer partitioning middleware to partition pipelines, respectively. For intra-layer-only AP, OptCNN~\citep{jia_exploring_2018}, TensorOpt~\citep{cai_tensoropt_2022}, and Tofu~\citep{wang_supporting_2019} employ dynamic programming methods to optimize DP and TP strategies together. FlexFlow~\citep{jia_beyond_2019} and Automap~\citep{schaarschmidt_automap_2021} use the Monte Carlo method to find the optimal DP and TP strategy. Colossal-Auto~\citep{liu_colossal-auto_2023} utilizes integer programming techniques to generate intra-layer parallelism and activation checkpointing strategies without optimizing inter-layer parallelism. All these methods optimize only one category of parallel strategies.


\paragraph{Inter- and intra-layer AP} PipeDream~\citep{narayanan_pipedream_2019}, DAPPLE~\citep{fan_dapple_2021}, and PipeTransformer~\citep{he_pipetransformer_2021} use dynamic programming to determine optimal strategies for both DP and PP. DNN-partitioning~\citep{tarnawski_efficient_2020} adopts integer and dynamic programming to explore DP and PP strategies. Piper~\citep{tarnawski_piper_2021} and Alpa~\citep{zheng_alpa_2022} adopt a parallel method considering DP, TP, and PP.
Galvatron~\citep{miao_galvatron_2022} uses dynamic programming to determine DP, TP, and FSDP strategies in a single pipeline stage. As for PP, it partitions stages and determines micro-batch size using naive greedy algorithms. All these methods are hierarchical, which will result in sub-optimal solutions.




\section{Limitations of Existing Collaborative Environments}
\label{sec:limitations}


\begin{table*}[t]
\small
% \fontsize{8.5}{12}\selectfont
\centering
\caption{Limitations of Existing Collaborative Environments.}
\label{tbl:limitations}
\begin{tabular}{m{2.5cm}m{4.6cm}m{4.6cm}m{4.6cm}}
\hline
\textbf{\begin{tabular}[c]{@{}r@{}} \end{tabular}}
\textbf{Limitation} & \textbf{Google Colab} & \textbf{Code Ocean} & \textbf{Kaggle}                                         \\ \hline

\rowcolor[HTML]{D9D9D9} 
\textbf{\begin{tabular}[c]{@{}l@{}}Resource\\ heterogeneity \end{tabular}} 
& 
CPU, disk, and memory limits; GPU types available; no access to IoT/Edge devices; &
experiments run on AWS virtual machines; no access to IoT/Edge devices; &
limits CPU, GPU, and TPU access; does not support IoT/Edge devices;
\\

\textbf{\begin{tabular}[c]{@{}l@{}}Large-scale\\ experiments\end{tabular}} & 
limits sessions to 12 hours; paid access to multiple computing resources. &
limits access to 10 compute hours; paid access to multiple computing resources. &
limits execution time to 12 hours; paid access to Google Cloud Services.
\\
\rowcolor[HTML]{D9D9D9} 
\textbf{\begin{tabular}[c]{@{}l@{}}Repeatability, \\Reproducibility, \\Replicability \end{tabular}} & 
hard to repeat and reproduce experiments on the same hardware: resource availability varies over time and usage limits fluctuate. Replicability in different infrastructures (\emph{e.g.,} beyond Google machines) is not straightforward. 
&
lacks support for the reproducibility of distributed experiments. Computing and storage resources are available in AWS virtual machines in the clients virtual private cloud. Hard to replicate experiments in different infrastructures.
&
lacks support for the repeatability and reproducibility of distributed experiments. Computing resources vary over time and hence between accesses. Replicability in different infrastructures is not easy to set up.  
\\ \hline
\end{tabular}
\end{table*}


We briefly discuss the limitations of state-of-the-art collaborative environments, with a focus on the specific challenges of the Computing Continuum.



\paragraph{\textbf{Google Colab~\cite{colab}}} 
Mainly used by the AI community (more than 50K users), it is a ready-to-use Jupyter notebook service. Colab notebooks are stored in the \emph{.ipynb} open-source Jupyter notebook format~\cite{colab-faq}, and come with the most popular AI libraries and frameworks installed (\emph{e.g.,} Scikit-Learn~\cite{pedregosa2011scikit}, TensorFlow~\cite{abadi2016tensorflow}, PyTorch~\cite{paszke2019pytorch}, \emph{etc.}) and allow users to run python code through the browser. It is typically used for machine learning, data analysis and education. Colab is popular because it allows users to share Jupyter notebooks without having to download, install, or run anything. Besdides, it provides free access to very expensive computing resources such as GPUs and TPUs. Colab permits multiple users to collaborate on the same notebook. Sharing datasets, ML models, pipelines, and notebooks on AI Hub~\cite{aihub} is also possible (more than 167 notebooks). Its GitHub integration allows users to quickly open GitHub-hosted Jupyter notebooks in Google Colab.



\paragraph{\textbf{Kaggle~\cite{kaggle}}} 
This is a data science and AI platform that offers a customizable Jupyter notebook environment. Kaggle is a subsidiary of Google and, like Colab, it provides free access to GPUs as well as a repository of community-published (more than 10.3 million users) datasets (more than 50K public datasets) and code (\emph{e.g.,} machine learning code) with more than 400K public notebooks. Kaggle is integrated with AI Hub and is popular in the data science and machine learning communities. Kaggle is also well-known for promoting Community Competitions in machine learning at no cost. The main differences~\cite{kaggle-vs-colab} between Colab and Kaggle are: (1) Kaggle allows collaboration with other users on its Web site, while Colab allows collaboration with anyone using the notebook link; (2) Kaggle has a lot of data sets that users can use directly (\emph{e.g.,} notebooks already set up with Kaggle databases~\cite{kaggle-datasets}), while in Colab setting up notebooks with Google Drive~\cite{colab-drive} or managing files~\cite{gsutil-tool} (\emph{e.g.,} to load data sets, files, and images) requires extra work; and (3) Kaggle creates a history of notebook commits that we can  be reviewed.



\paragraph{\textbf{Code Ocean~\cite{clyburne2019computational}}} 
Designed according to FAIR~\cite{wilkinson2016fair} (\emph{i.e.,} Findable, Accessible, Interoperable, and Reusable), Code Ocean aims to make scientific work reproducible. It introduces the concept of Compute Capsule, which refers to Docker~\cite{docker} containers composed of code, data, environments, and results. Capsules provide ready-to-use tools such as Git, Jupyter, RStudio, among others. Its integration with Git allows users to save changes on capsules and then commit them with just one click. Furthermore, users can easily share the link of a capsule and grant permissions. Code Ocean provides scalable compute and storage resources hosted on Amazon Web Services. Resources used by capsules are scaled out when the demand exceeds the machine capacity. Finally, Code Ocean provides a public Capsule Repository~\cite{code-ocean} with more than 1K research capsules. It allows authors of an article to incorporate capsules into the submission process via a Hub publishing API.


\begin{framed}
\vspace{-0.1cm}

Despite these systems being widely used by the AI and data science communities, they present some limitations that hinder their adoption for Computing Continuum research. Table~\ref{tbl:limitations} summarizes these limitations in terms of: 

\begin{enumerate}
    \item access to heterogeneous computing resources, from the IoT/Edge to the Cloud/HPC;
    \item support for large-scale experimental evaluations;
    \item repeatability and reproducibility of experiments on the same hardware setup, and replicability on different infrastructures.
\end{enumerate}

In summary, collaborative environments lack support for providing access to heterogeneous resources (\emph{e.g.,} Edge-to-Cloud); performing experiments at large-scale; and achieving the repeatability, reproducibility, and the replicability of experiments in different testbeds. Hence, the need for novel approaches for reproducible evaluations of workflows targeting the characteristics of the Computing Continuum. 
%Novel systems must support  deployed on the complex Edge-to-Cloud Continuum.

\vspace{-0.1cm}
\end{framed}


% \section{Extending \textbf{E2C}\textit{lab} to Enable Provenance Capture of Edge-to-Cloud Workflows}
% \section{ProvLight: Efficient Provenance Capture of Complex Edge-to-Cloud Workflows}
\section{Kheops Design}
\label{sec:kheops}
% Figure environment removed




This section introduces KheOps, a collaborative environment for the cost-effective reproducibility and replicability of Edge-to-Cloud experiments. KheOps is designed to meet the experimental requirements of both authors and readers as presented in Section~\ref{sec:introduction}. 



\subsection{Architecture and implementation}
\label{subsec:kheops_arch_impl}




Figure~\ref{fig:kheops-archi} presents the architecture of KheOps, which consists of three main components: \emph{(i)} Trovi sharing portal; \emph{(ii)} Jupyter environment (JupyterHub service and JupyterLab server); and \emph{(iii)} E2Clab framework (multi-platform experiment methodology). Next, present the integration details of KheOps three components, and we briefly describe their main roles.

% Figure environment removed



% Figure environment removed








\subsubsection{Experiment repository} KheOps uses Trovi to share research artifacts such as packaged experiments. These artifacts may be publicly available to allow others to recreate and rerun experiments. Trovi provides a REST API to manage experiment artifacts and integrate them with other systems. The JupyterHub in KheOps uses the Trovi REST API to download artifacts and launch them in the JupyterLab server. 

Artifacts hosted in Trovi can also provide references to repositories like container registries (\emph{e.g.,} DockerHub~\cite{dockerhub}), multipurpose repositories (\emph{e.g.,} Zenodo~\cite{zenodo}), code repositories (\emph{e.g.,} Github~\cite{github}), and among others.

\subsubsection{Notebook environment} Following our previous work~\cite{anderson2019case} on integrating experiment workflows with Jupyter notebooks, we extend JupyterHub to authenticate users and to download (using the Trovi REST API) the experiment artifacts available at Trovi. We also extend JupyterLab to allow users to easily share their experiments in Trovi. Furthermore, JupyterLab is set up with the E2Clab framework as an experimental methodology.

The JupyterLab is packaged with code, data, environment configurations, and experiment results. Its notebooks (file extension \textit{.ipynb}) allow users to run experiments step-by-step by combining text (\emph{e.g.,} explaining the reasoning of the experiments: \emph{What} parameters? \emph{Why} these parameters? and \emph{How} it was set up?) with executable code. Such notebooks are ready to use (\emph{e.g.,} installed with required library/software), executed through a browser, and shared as a Trovi artifact.


\subsubsection{Multi-testbed experiment methodology} KheOps uses the E2Clab methodology to deploy experiments on large-scale scientific testbeds such as Grid'5000, Chameleon Cloud, CHI@Edge, and FIT IoT LAB. Notebooks come with three main template files (\emph{e.g.,} executable code cells in the notebook, presented in Listings~\ref{lis-e2c-layers} to~\ref{lis-e2c-workflow}) that users can benefit from to easily configure and adapt the deployment logic (\emph{e.g.,} computing resources, network, and application execution) according to their experimental needs. 

The first file, named \emph{layers\_services.yaml} and presented in Listing~\ref{lis-e2c-layers}, allows users to lease IoT/Edge and Cloud/HPC resources. Through this file, users may also set up their applications and services as presented in Listing~\ref{lis-e2c-svc}. Next, the \emph{network.yaml} file (Listing~\ref{lis-e2c-net}) allows users to define delay, loss, and bandwidth between computing resources. Finally, the \emph{workflow.yaml} file (Listing~\ref{lis-e2c-workflow}) guides users to define the experiment workflow through three main steps: \emph{prepare} (\emph{e.g.,} copy artifacts to remote nodes, install libraries, \emph{etc.}), \emph{launch} (\emph{e.g.,} execute the application parts), and \emph{finalize} (\emph{e.g.,} backup results from remote nodes to the JupyterLab server).

E2Clab abstracts all the complexities of deploying and executing experiments across various testbeds. To do so, users need to add the credential files of the respective testbeds to their notebooks. Setting up a VPN is also supported as this may be required to enable the communication between different geographically distributed tesbeds (\emph{e.g.,} Chameleon in the USA and Grid'5000 in France).







% Figure environment removed



\subsection{Experimental workflow}
\label{subsec:kheops_workflow}

In summary, the workflow for launching an experiment artifact on large-scale testbeds consists of 5 main steps. First, through a web interface, users can browse the list of experimental artifacts publicly available in Trovi (step 1). Selecting an artifact displays details such as the experiment description, the authors and contact information, and the artifact versions. 

A \emph{launch} button allows users to execute the artifact (step 2). This button redirects users to the JupyterHub service. After authentication, the request to launch the artifact is sent to the JupyterHub Spawner. Next, the Spawner spawns the JupyterLab server (step 3)  and then it downloads experimental artifacts such as notebooks, code, and datasets, among others (step 4). The JupyterLab service is set up with the E2Clab framework as the experimental methodology. Finally, users can execute the code cells from the notebook to lease IoT/Edge and Cloud/HPC computing resources available on the testbeds, deploy and execute the application, and gather the experiment results (step 5).


% Figure environment removed




Steps 2 to 4 are automatically executed. This is a one-click feature that allows users to have a ready-to-use environment for reproducing and replicating complex Edge-to-Cloud experiments in a cost-effective manner. Note that the whole workflow requires only three clicks: selecting the experiment artifact (step 1); then launching it (steps 2 to 4); and executing it on the testbeds (step 5).  




\section{Evaluation}
\label{sec:evaluation}
In this section, we provide details on our main experiments. First, we give an overview of the experimental setup and implementation details. Next, we present our findings along with the results.

\subsection{Experimental setup}

\paragraph{Large Language Model} In our main experiments, we employ \flan{}~\citep{flan}, particularly \flan{}-large, as the base LLM. 
The model has shown impressive abilities to perform zero-shot and few-shot learning.

\paragraph{Candidate LoRA Modules} Our methodology requires a compendium of LoRA modules trained on preceding tasks. For parity with FLAN, we adopt the tasks utilized to instruct \flan{}, thereby incorporating nearly $200$ distinct tasks and their corresponding instructions~\footnote{We released used the dataset at \href{https://huggingface.co/datasets/lorahub/flanv2}{\texttt{huggingface.co/datasets/lorahub/flanv2}}.}. Following this, we trained several LoRA modules as potential candidates.
During each experimental sequence, we randomly select $20$ LoRA modules from them as the candidate for our \lorahub{} learning. 

\paragraph{Dataset and evaluation}

Our method is evaluated using the Big-Bench Hard (BBH) benchmark, a well-established standard that consists of multiple-choice questions from a variety of domains.
The benchmark consists of $27$ different tasks, which are regarded to be challenging for language models.
For all tasks, we employ the exact match (EM) as our evaluation metric.

\begin{table}[t]
\centering
\small

\caption{Experimental results of zero-shot learning (Zero), few-shot in-context learning (ICL), IA3 fine-tuning (IA3), LoRA tuning (LoRA), full fine-tuning (FFT) and our proposed few-shot \lorahub{} learning (\lorahub{}) on the BBH benchmark with \flan{}-large as the base LLM. We denote algorithmic tasks with the superscript $\S$ following previous work~\citep{DBLP:journals/corr/abs-2303-17564}. Note that we employ three runs, each leveraging different $5$-shot examples per task, as demonstrations for all few-shot methods. The average performance of all methods is reported below, and the best performance of each few-shot method can be found in the Appendix~\ref{sec:maximum_appendix}.}
\label{tab:performance}
\begin{tabular}{lccccccc}
\toprule
Task & Zero & ICL$_{\rm avg}$ & IA3$_{\rm avg}$ & LoRA$_{\rm avg}$ & FFT$_{\rm avg}$ & \lorahub{}$_{\rm avg}$ \\
\midrule
Boolean Expressions & 54.0 & 59.6 & 56.2 & 56.0 & 62.2 & 55.5 \\
Causal Judgement & 57.5 & 59.4 & 60.2 & 55.6 & 57.5 & 54.3 \\
Date Understanding & 15.3 & 20.4 & 20.0 & 35.8 & 59.3 & 32.9 \\
Disambiguation & 0.0 & 69.1 & 0.0 & 68.0 & 68.2 & 45.2 \\
Dyck Languages & 1.3 & 0.9 & 4.2 & 22.2 & 19.5 & 1.0 \\
Formal Fallacies & 51.3 & 55.3 & 51.5 & 53.6 & 54.0 & 52.8 \\
Geometric Shapes & 6.7 & 19.6 & 14.7 & 24 & 31.1 & 7.4 \\
Hyperbaton & 6.7 & 71.8 & 49.3 & 55.3 & 77.3 & 62.8 \\
\begin{tabular}[c]{@{}l@{}}Logical Deduction$^\S$ \\ {\small ~~~~~~~~~~~~~(five objects)}\end{tabular}  & 21.3 & 39.1 & 32.7 & 40.0 & 42.2 & 36.1 \\
\begin{tabular}[c]{@{}l@{}}Logical Deduction$^\S$ \\ {\small ~~~~~~~~~~~~~(seven objects)}\end{tabular} & 12.7 & 40.7 & 33.8 & 37.3 & 44.9 & 36.8 \\
\begin{tabular}[c]{@{}l@{}}Logical Deduction$^\S$ \\ {\small ~~~~~~~~~~~~~(three objects)}\end{tabular} & 0.0 & 51.6 & 8.5 & 53.6 & 52.9 & 45.7 \\
Movie Recommendation & 62.7 & 55.8 & 61.8 & 51.5 & 66.0 & 55.3 \\
Multistep Arithmetic & 0.7 & 0.7 & 0.7 & 0.2 & 0.0 & 0.4 \\
Navigate & 47.3 & 45.3 & 46.2 & 48.0 & 48.0 & 47.1 \\
Object Counting & 34.7 & 32.4 & 35.1 & 38.7 & 35.6 & 33.7 \\
Penguins in a Table & 43.5 & 41.3 & 45.0 & 36.2 & 31.9 & 35.9 \\
Reasoning about Colored Objects & 32.0 & 40.2 & 40.7 & 39.6 & 37.6 & 40.0 \\
Ruin Names & 23.3 & 19.3 & 24.4 & 37.8 & 61.3 & 24.4 \\
Salient Translation Error Detection & 37.3 & 47.3 & 37.1 & 16.0 & 16.2 & 36.0 \\
Snarks & 50.0 & 54.2 & 53.9 & 55.6 & 66.7 & 56.9 \\
Sports Understanding & 56.0 & 54.7 & 55.1 & 56.5 & 54.0 & 56.7 \\
Temporal Sequences & 16.7 & 25.1 & 18.2 & 25.1 & 37.8 & 18.2 \\
\begin{tabular}[c]{@{}l@{}}Tracking Shuffled Objects$^\S$ \\ {\small ~~~~~~~~~~~~~\quad\quad\quad(five objects)}\end{tabular} & 12.0 & 12.0 & 12.0 & 13.8 & 16.9 & 12.3 \\
\begin{tabular}[c]{@{}l@{}}Tracking Shuffled Objects$^\S$ \\ {\small ~~~~~~~~~~~~~\quad\quad\quad(seven objects)}\end{tabular} & 6.7 & 6.7 & 6.7 & 10.0 & 9.8 & 7.7 \\
\begin{tabular}[c]{@{}l@{}}Tracking Shuffled Objects$^\S$ \\ {\small ~~~~~~~~~~~~~\quad\quad\quad(three objects)}\end{tabular} & 24.7 & 31.1 & 30.7 & 30.9 & 32.0 & 29.2 \\
Web of Lies & 54.0 & 53.8 & 54.2 & 52.7 & 48.2 & 50.1 \\
Word Sorting & 1.3 & 0.5 & 1.3 & 4.9 & 4.9 & 1.1 \\
\midrule
Avg Performance Per Task & 27.0 & 37.3 & 31.6 & 37.7 & 42.1 & 34.7 \\
Avg Tokens Per Example & 111.6 & 597.8 & 111.6 & 111.6 & 111.6 & 111.6 \\
Gradient-based Training & No & No & Yes & Yes & Yes & No \\
\bottomrule
\end{tabular}
\end{table}


\paragraph{Baseline Setup}

To enhance the demonstration of our method's performance, we expanded our comparisons beyond the zero-shot and in-context learning settings. We specifically chose three representative gradient-based methods for comparison: full fine-tuning (FFT), LoRA tuning (LoRA)~\citep{hu2022lora}, and IA3 fine-tuning (IA3)~\citep{Liu2022FewShotPF}.
For all gradient-based methods, for a fair comparsion, we train for $40$ epochs on the same three runs of $5$ examples employed in our methods.
In the case of FFT, a learning rate of 3e-5 is employed, whereas for IA3 and LoRA, we adopt a learning rate of 2e-4.
We report the performance of each method on the test set at the end of training (averaged over three runs) without any model selection to avoid potential selection bias.

\subsection{Main results} 

As shown in Table~\ref{tab:performance}, our experimental results demonstarte the superior efficacy of our method in comparison to zero-shot learning while closely resembling the performance of in-context learning (ICL) in few-shot scenarios. This observation is derived from an average performance of three runs, each leveraging different few-shot examples.
Importantly, our model utilizes an equivalent number of tokens as the zero-shot method, notably fewer than the count used by ICL. 
Although occasional performance fluctuations, our method consistently outperforms zero-shot learning in most tasks.
In the era of LLMs, the input length is directly proportional to the inference cost, and thus \lorahub's ability to economize on input tokens while approaching the peak performance grows increasingly significant.
Moreover, as shown in Appendix Table~\ref{tab:max_perf}, the upper bound performance of our method across these runs can surpass ICL on $18$ tasks, demonstrating its potential for future development.

Even when compared to certain gradient-based optimization methods, our approach consistently demonstrates competitive performance. For example, as depicted in Table~\ref{tab:performance}, our method exhibits a notable improvement of $3.1\%$ on average in contrast to the promising IA3 method. Nevertheless, we acknowledge that our approach still falls behind LoRA tuning and full fine-tuning, especially in tasks that exhibit significant deviation from the upstream task. Taking Dyck Languages as an example, both \lorahub and ICL achieve only an average performance of nearly $1.0\%$ on these tasks, while LoRA and FFT methods showcase impressive results with only $5$ examples.

\subsection{Discussion}

LoraHub addresses the challenge of reducing inference costs by eliminating the need for processing additional tokens, resulting in a noticeable reduction in overall inference expenses. However, it introduces an inherent cost during the \textsc{Adapt} stage, necessitating extra inference steps, such as the $40$ steps employed in our experiments. This introduces a trade-off between choosing the ICL approach and LoraHub, with the decision typically hinging on the nature of the situation.

For one-time ad-hoc tasks, the ICL approach should be more pragmatic due to LoraHub's additional inference step costs. In such scenarios, where immediate, single-use solutions are preferred, the simplicity and efficiency of ICL might outweigh the benefits of potential savings offered by LoraHub. Conversely, for recurring or similar tasks, LoraHub emerges as a compelling option. Despite the added inference step cost, LoraHub's ability to efficiently handle repetitive tasks, often occurring thousands of times, while concurrently reducing overall expenses, positions it as a viable option in such kind of situations.

In summary, our intention is not to replace ICL, but to present LoraHub as a complementary strategy with performance-efficiency trade-offs. Thus, we encourage a careful consideration of specific use cases and requirements when choosing between ICL and LoraHub, recognizing that the optimal solution may vary based on the nature and frequency of the tasks at hand.


\section{Discussion}
\label{sec:discussion}
In this work, we investigated functional mimicry as a smooth and effective telemanipulation paradigm. Functional mimicry allows a robot to exploit flexibility in task tolerances to generate more accurate, smooth, and feasible motions. Our user evaluation demonstrated that the autonomous adjustments within task tolerances led to equal or better performance without sacrificing perceived control of the robot. Moreover, the functional mimicry manipulator that exploited task tolerances was perceived to be more under control, predictable, fluent, and trustworthy than the manipulator that exactly mimicked its human operator. Below, we discuss additional findings, limitations of this work, and implications for future teleoperation systems.

\subsection{Subgroup Analysis}

Figure \ref{fig: subgroup_results} shows our subgroup analysis results. As mentioned in \cref{sec:perception_shared_control}, prior works have found that people with a high internal Locus of Control (LoC) have \replaced{trouble}{problems} giving up control to an autonomous system. We observed a similar phenomenon in our study. Although not statistically significant, the autonomous adjustments provided in functional mimicry led to less performance and perception improvement in high internal LoC participants than in participants with an average LoC. Moreover, we also classified participants' expertise according to their reported familiarity in operating robots, avatars in video games, and virtual objects in Computer-Aided Design software or virtual reality. We identified 9 participants with low expertise and 11 with high expertise. Although not statistically significant,  functional mimicry brought a larger performance and perception improvement in expert users than in non-expert users. We note that the result is not directly comparable with what is found in prior shared control work \cite{milliken2017modeling}, where non-expert users benefit more from autonomous assistance. As described in \cref{sec:shared_control}, in prior work the shared control system \textit{explicitly}  assists users in avoiding obstacles and directly contributes to task completion, so non-expert users \replaced{with insufficient}{of low} obstacle avoidance skills gain more benefits from the explicit assistance. Meanwhile, our system \textit{implicitly} assists them by generating high-quality robot motions. While both expert and non-expert users benefit from the high-quality motions, we speculate that expert users can take more advantage of the high-quality motions to complete the manipulation tasks. 

% Figure environment removed

\subsection{Limitations}
While our results demonstrate the potential for functional mimicry to provide smooth and effective telemanipulation, the limitations of the present work suggest directions for future research.
In our empirical study, participants and the robot were located in the same room, which gives the participants perfect situational awareness of the robot and its surroundings and allows the robot to be controlled with low latency. With limited situational awareness (\textit{e.g.}, viewing the robot's workspace through a camera) and high latency, human operators may be more sensitive and disturbed by autonomous robot adjustments, even if they are within task tolerances. Future work should evaluate functional mimicry when the human operator and the robot are in separate physical spaces. While this work demonstrates the benefits of exploiting task tolerances, the task tolerances in our experiment were manually specified. Future work should investigate algorithms to automatically detect task tolerances. 

\subsection{Implications}
Instead of forcing a robot manipulator to exactly mimic its human operator, a mimicry-based teleoperation system can allow some task-specific inaccuracy. Such flexibility in task tolerance can be exploited by a robot to generate accurate, smooth, and feasible motions, leading to user perception and performance improvements. We believe that \replaced{our interaction paradigm is beneficial to teleoperation of welding, sanding, painting, wiping, pouring, and many other tasks that allow some positional or rotational inaccuracy.}{many high-precision teleoperation scenarios would benefit from our interaction paradigm.} Additionally, our user study results suggest that autonomous robot adjustments do not impede perceived control of the robot as long as the end-effector poses are within task tolerances. This finding provides a robot more freedom to enhance motion generation. Aside from the improved accuracy, smoothness, and manipulability demonstrated in this work, we believe that other aspects of robot motions, such as legibility, predictability \cite{dragan2013legibility}, and expressiveness \cite{venture2019robot}, can be enhanced with the flexibility in task tolerances.

\vspace{-0.5mm}
\section{Acknowledgement}
We would like to thank Seth Peterson for 3D printing tool mounts for the robot used in the study\added{ and anonymous reviewers for their suggestions and comments}. 



\section{Related Work}
\label{sec:related-work}
\vspace{-.2cm}
\section{Related Work}
\vspace{-.2cm}
\model closely aligns with two research directions: (1) augmentation in text-based recommendation, and (2) LLM for recommendation. A comprehensive discussion is provided in Appendix~\ref{sec:detailed_related_work}. 

\noindent\textbf{Augmentation in Text-based Recommendation.} Text-based recommendation systems leverage natural language processing and machine learning techniques to provide personalized recommendations to users based on textual information~\citep{lops2019trends, qiang2020short}. However, the performance of such systems can be compromised when dealing with incomplete or insufficient textual information. To address this limitation, several studies have suggested strategies for enhancing textual information. For instance, \citet{li2010contextual} proposed to extract contextual cues from online reviews, leveraging these narratives to uncover users' preferences and underlying factors influencing their choices~\citep{sachdeva2020useful}. Other approaches infer linguistic attributes from diverse sources, including emotion, sentiment, and topic, to refine the modeling of both items and users~\citep{sun2015mining, sailunaz2019emotion, ramage2010characterizing,chen2010short}. Furthermore, some works explore the integration of external knowledge bases to enrich the contextual understanding of items~\citep{di2012linked, musto2018semantics}. In a more recent development, \citet{bai2022improving} introduced an approach that employs pre-trained language models to generate additional product attributes, such as product names, to augment item contextual information. Diverging from these prior approaches, our contribution is the \model framework, which employs large language models to enhance input text, providing a versatile solution for recommendations. A more detailed discussion on the distinctions between \model and these related work can be found in Section~\ref{sec:discussions_and_conclusions}.



\input{tab_fig/eval_framework}
\noindent\textbf{LLM for Recommendation.}
Due to LLMs' remarkable text generation ability, many studies have leveraged LLMs as a data augmentation tool~\citep{dai2023auggpt, li2022elevater}. \citet{liu2023llava} used an LLM to produce multimodal language-image instruction-following datasets. Through a process of instruction tuning using this generated data, their proposed framework demonstrated an impressive aptitude in advancing vision and language comprehension. There have also been efforts to use LLMs to augment the input side of personalized recommendation. For instance, \citet{chen2023palr} incorporated user history behaviors, such as clicks, purchases, and ratings, into LLMs to generate user profiles. These profiles were then combined with the history interaction sequence and candidate items to construct the final recommendation prompt. LLMs were subsequently employed to predict the likelihood of user-item interaction based on this prompt. \citet{xi2023towards} introduced a method that leverages the reasoning knowledge of LLMs regarding user preferences and the factual knowledge of LLMs about items. However, our study focuses specifically on using LLMs' knowledge and reasoning ability to generate augmented input text that better captures the characteristics and nuances of items, leading to improved recommendation performance. 





\section{Conclusion}
\label{sec:conclusions}
\section{Conclusion and Future Work}
\label{sec: Conclusion and Future Work}
This paper explores formal method-based reachability analysis of variable-length time series regression neural networks (NNs) using approximate Star methods in the context of predictive maintenance, which is crucial with the rise of Industry 4.0 and the Internet of Things. The analysis considers sensor noise introduced in the data. Evaluation is conducted on two datasets, employing a unified reachability analysis that handles varying features and variable time sequence lengths while analyzing the output with acceptable upper and lower bounds. Robustness and monotonicity properties are verified for the TEDS dataset. Real-world datasets are used, but further research is needed to establish stronger connections between practical industrial problems and performance metrics. The study opens new avenues for exploring perturbation contributions to the output and extending reachability analysis to 3-dimensional time series data like videos. Future work involves verifying global monotonicity properties as well, and including more predictive maintenance and anomaly detection applications as case studies. \newblue{The study focuses solely on offline data analysis and lacks considerations for real-time stream processing and memory constraints, which present fascinating avenues for future research.}
\paragraph{\textbf{Acknowledgements.}}
The material presented in this paper is based upon work supported by the National Science Foundation (NSF) through grant numbers 1910017, 2028001, 2220418, 2220426, and 2220401, and the Defense Advanced Research Projects Agency (DARPA) under contract number FA8750-18-C-0089 and FA8750-23-C-0518, and the Air Force Office of Scientific Research (AFOSR) under contract number FA9550-22-1-0019 and FA9550-23-1-0135. Any opinions, findings, conclusions, or recommendations expressed in this paper are those of the authors and do not necessarily reflect the views of AFOSR, DARPA, or NSF. We also want to thank our colleagues, Tianshu and Barnie for their valuable feedback.
 



\section*{Acknowledgments}
This work was funded by Inria through the HPC-BigData Inria Challenge (IPL) and through the UNIFY Associate Team joint in the framework of the JLESC international lab and the HPDaSc associate team with Brazil. It was co-funded by the French ANR OverFlow project (ANR-15- CE25-0003). Experiments presented in this paper were carried out using the Chameleon Cloud, CHI@Edge, Grid'5000, and FIT IoT LAB testbeds, supported by a scientific interest group hosted by several Universities. We also would like to thank Argonne National Laboratory for supporting this work. This material is based upon work supported by the U.S. Department of Energy, Office of Science, under contract number DE-AC02-06CH11357 as well as by the NSF award 2130889 and NIFA award 2021-67021-33775.

%%
%% The next two lines define the bibliography style to be used, and
%% the bibliography file.
\bibliographystyle{ACM-Reference-Format}
\balance
\bibliography{acmart.bib}


\end{document}

