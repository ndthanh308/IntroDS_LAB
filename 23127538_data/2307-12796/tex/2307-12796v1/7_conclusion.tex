
KheOps is, to the best of our knowledge, the first collaborative environment supporting the cost-effective reproducibility of applications on the Edge-to-Cloud Continuum. It provides simplified abstractions for systematically defining and explaining the experimental environment through Jupyter notebooks (\emph{e.g.,} infrastructures, services, network, and workflow execution); provides access to heterogeneous computing resources from the IoT/Edge to the Cloud/HPC; and allows researchers to easily find and share the experiment artifacts in the Trovi portal.

The experimental validation shows that KheOps helps authors to make their experiments repeatable and reproducible on the Grid5000 and FIT IoT LAB testbeds. Furthermore, KheOps helps readers to cost-effectively replicate authors experiments in different infrastructures such as Chameleon Cloud + CHI@Edge testbeds, and obtain the same conclusions with accuracies $>$88\% for all performance metrics. %Finally, we believe that the core idea behind KheOps, of a collaborative environment for advancing 3Rs in Computing Continuum research, may prove useful for realizing the Computing Continuum vision in practice.



