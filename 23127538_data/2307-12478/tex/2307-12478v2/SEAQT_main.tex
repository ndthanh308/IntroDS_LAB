\documentclass{article}
\usepackage{graphicx} % Required for inserting images
\graphicspath{ {./Images/} }
\usepackage[margin=1in]{geometry}
\usepackage{breqn}
\usepackage{array}
\usepackage{authblk}
\usepackage{xcolor}
\usepackage[utf8]{inputenc}
\usepackage{amssymb}
\DeclareUnicodeCharacter{2212}{-}
%%\DeclareUnicodeCharacter{U+2212}{{-}}
%\usepackage{unicode-math}
\usepackage[backend=biber,sorting=none,maxbibnames=200,style=phys]{biblatex}
\addbibresource{SEAQT_references.bib}

%\title{}
%\author{}
%\date{April 2023}
\title{\textbf{Predicting Coupled Electron and Phonon Transport Using Steepest-Entropy-Ascent Quantum Thermodynamics}} 
\author{Jarod Worden$^\dagger$, Micheal von Spakovsky$^\ddagger$, Celine Hin$^{\dagger}$}
\affil{$^\dagger$MSE Dept., Virginia Tech, Falls Church, VA 22043}
\affil{$^\ddagger$Center for Energy Systems Research, ME Dept., Virginia Tech, Blacksburg, VA 24061}
\date{}

\begin{document}

\maketitle
%
%a. ME department,
%Northern Virginia Center, Virginia Tech 
%7054 Haycock Rd, , United States
%
%b. MSE department,
%Northern Virginia Center, Virginia Tech
%7054 Haycock Rd, Falls Church, VA 22043, United States

\smallbreak
\textbf{Abstract}
\smallbreak

The current state of the art for determining thermoelectric properties is limited to the investigation of electrons or phonons without including the inherent electron-phonon coupling that is in all materials. This gives rise to limitations in accurately calculating base material properties that are in good agreement with experimental data. Steepest-entropy-ascent quantum thermodynamics is a general non-equilibrium thermodynamic ensemble framework that provides a general equation of motion for non-equilibrium system state evolution. This framework utilizes the electron and phonon density of states as input to compute material properties, while taking into account the electron-phonon coupling. It is able to span across multiple spatial and temporal scales in a single analysis. Any system's thermoelectric properties can, therefore, be attained provided the accurately determined density of states is available.
%\break
\newline

\textbf{Introduction}
\smallbreak

Nonequilibirum systems are notoriously difficult to model while comprising a majority of natural systems and mechanisms. Due to their prevalence, many methods of describing microscopic and mesoscopic models describing nonequilibrium phenomena have been developed  (such as nonequilibrium molecular dynamics (NEMD), Monte Carlo methods, continuum models, etc.), but are limited in their general application \cite{cercignani_theory_1976,newman_monte_1999,rapaport_art_1996,hoover_nonequilibrium_1992,jou_extended_2010,groot_non-equilibrium_1984,vogl_non-equilibrium_2010}. NEMD is limited due to the calculation of intensive properties that are only defined at equilibrium and is unable to predict nonquasiequilibrium processes producing nonlocal equilibrium states \cite{kondo_molecular-dynamics_2006}. Nonequilibrium Green’s function is a rigorous and general framework for modeling non-equilibrium processes, but requires a large number of approximations of momentum, charge, and energy conservation laws from a many-body standpoint that can fail badly even with plausible approximations \cite{vogl_non-equilibrium_2010}. Kinetic Monte Carlo uses transition rates to model stochastic trajectories that propagate in a system from state to state, but require high accuracy for the pre-defined, possibly incomplete list, of possible processes leading to incorrect or limited results \cite{andersen_practical_2019}. Boltzmann transport equations predict the evolution of individual energy levels through scattering processes based on Newton’s laws and quantum mechanics, but is limited based on its local equilibrium requirement \cite{chen_thermal_1998}. Continuum models with nonequilibrium effects follow a two step method of first approximating the nonequilibrium using a fine mesh of local systems, each to be assumed in equilibrium, and then to address the nonequilibrium effects of system dynamics via the inclusion of a set of phenomenological coefficients (diffusivity, conductivities, etc.). Both steps represent incomplete and inadequate descriptions where the mesh is limited in its ability to describe each actual local state and the coefficients are often based on uncoupled behavior. Such other limitations of these models come from inherent linear characteristics that constrain the systems of interest when they could be nonlinear. Another necessity of these models when obtaining material characteristics is having systems evaluated at equilibrium to obtain intensive properties when many systems are in a non-equilibrium state \cite{li_steepest_2018,li_steepest_2018-1}. 

In order to go beyond the equilibrium regime, an advanced nonequilibrium thermodynamics-ensemble based framework called steepest-entropy-ascent quantum thermodynamics (SEAQT) provides a general thermodynamically rigorous equation of motion for nonequilibrium system state evolution. This method focuses on the relaxation of the system state, the conservation laws of thermodynamics, and the principle of steepest entropy ascent (SEA) \cite{beretta_quantum_1984,beretta_nonlinear_2009,li_steepest-entropy-ascent_2016,beretta_steepest-entropy-ascent_2017}. Utilizing the development of hypoequilibrium state and the density of state method for electrons and phonons \cite{li_steepest-entropy-ascent_2016,li_generalized_2016,li_modeling_2016,li_steepest-entropy-ascent_2018,li_atomistic-level_2014}, SEAQT can be applied from a practical standpoint at all temporal and spatial scales, as shown in Fig. 1. The SEAQT framework provides an equation of motion that can be used to investigate the evolution of local nonequilibrium states from an entropy generation standpoint. The use of hypoequilibrium further simplifies this nonequilibrium description using nonequilibrium intensive properties (e.g., temperature and pressure) and extends the thermodynamic equilibrium description (e.g., the Onsager relations, the Gibbs relation, etc.) into the nonequilibrium realm. SEAQT addresses deficiencies in current methods to provide a general thermodynamically rigorous equation of motion for nonequilibrium system state evolution that crosses all time and length scales (Fig. 1) and i) inherently satisfies the laws of thermodynamics and quantum mechanics, ii) is applicable throughout the non-equilibrium realm, and iii) is able to easily model coupled phenomena. The qualities of SEAQT are unique in that it inherently satisfies both laws of quantum and thermodynamics (1st and 2nd laws), crosses both spatial and temporal scales in a single analysis, can be applied to quantum and classical systems, is able to generalize the equilibrium description (i.e. extensive and intensive properties) to non-equilibrium, and has significantly reduced computational time to conventional approaches. The framework can thus be used for studying thermoelectric transport properties of materials such as electrical conductivity, thermal conductivity, seebeck coefficient, temperature evolution over time, etc. 


The SEAQT framework using electron and phonon density of states (DOS) will be utilized in determining critical material thermoelectric properties which will be compared with experimental and computational benchmark data. Many methods have been developed for interpolating electron DOS to obtain electron transport properties. The current benchmark for such calculations is the BoltzTraP code that utilizes smooth Fourier interpolation of the band energies while maintaining group symmetries \cite{madsen_boltztrap_2006}. Other such electron transport calculation methods include Boltzwann \cite{pizzi_boltzwann_2014} and LanTraP \cite{wang_lantrap_2020} which use maximally-localized Wannier function basis and the Landauer formalism of transport theory, respectively. The primary limitation of these methods is the exclusion of the coupling aspect of the electron-phonon interactions that are apparent in all materials, but especially important in metals and heavily doped materials. Other limitations include the inability of calculating total thermal conductivity, the assumption of a constant relaxation time, and inability of analyzing polycrystalline materials nor materials with defects. Many codes such as LAMMPS \cite{thompson_lammps_2022} and phonopy \cite{togo_implementation_2023} are useful in determining strictly phonon transport properties, but suffer similar limitations such as the lack of electron-phonon coupling, only calculating the lattice thermal conductivity and not the total thermal conductivity, and large computational expense.


% Figure environment removed

\break 

\textbf{Methods}
\smallbreak

The Steepest Entropy-Ascent Framework is used to derive the equation of motion for moving through thermodynamic state space \cite{chen_nanoscale_2005}. In such a state space, there are \emph{m} single particle energy eigenlevels that can be occupied by particles. A system eigenstate is denoted by \textbar{}\(n^{a}n^{b}\ldots n^{m}\)〉 where \(n^{k}\) is the occupation number representation at the \emph{k}th single-particle eigenlevel \(\epsilon^{k}\). \(n^{k}\) represents fermions (value 0,1) and bosons \((0,1,\ldots,\infty)\). A system thermodynamic state is selected from the Hilbert space spanned by the system eigenstates and can be represented by a \emph{w}-dimensional vector \{\(p_{n^{k}}^{k}\)\}. \(p_{n^{k}}^{k}\) represents the probability that \(n^{k}\) particles are observed at the \emph{k}th single-particle eigenlevel. The system thermodynamic state can alternatively be denoted \((\gamma)\) by using the square root of the vector:

\begin{equation}
\gamma = \textrm{vect}(\gamma_{n^{k}}^{k}) = \textrm{vect}(\sqrt{p_{n^k}^{k}}) ,   k = a, \ldots{} , m 
\end{equation}

The \(\gamma\ \)space can then be defined as a manifold whose elements are all of the \emph{w} vectors of the real finite numbers \emph{X} = vect(\(x_{l}\)) and \emph{Y} = vect(\(y_{l}\)) with an inner product given by:

\begin{equation}
(X | Y) = \sum_{l=1}^{w}x_{l}y_{l}
\end{equation}
The time evolution of the thermodynamic state \(\gamma(t)\) follows the equation of motion in \(w\)-vector form given by

\begin{equation}
|d\gamma/dt)= |\Pi_{\gamma})
\end{equation}

The formalism of \textbar{}\(\Pi_{\gamma}\)) is derived from the SEA principle subject to a set of conservation laws denoted by \{\(\overline{C}(\gamma)\)\}. Such laws include \(\overline{H}(\gamma) =  \sum_{k} \sum_{n} n^{k}\epsilon_{k}p_{n^{k}}^{k}\ \)and of particle number \(\overline{N}(\gamma)\) as well as \(m\ \)probability normalization conditions \({\overline{I}}_{k}(\gamma)\). Using eq. 2, the time evolution of these conserved system properties \{\(\overline{C}(\gamma)\)\} = \{\(\overline{H}\),\(\overline{N}\),\({\overline{I}}_{a}\),\ldots,\({\overline{I}}_{m}\)\} and the system entropy \(\overline{S}(\gamma)\) obey the following equations of motion:

\begin{equation}
\Pi_{C_{i}} \equiv dC_{i}/dt = (\psi_{i} | \Pi_{\gamma})
= 0 , \textrm{with } | \psi_{i})
\equiv | \delta{\overline{C}}_{i}(\gamma)/ \delta\gamma)
\end{equation}
\begin{equation}
\Pi_{S} \equiv dS/ dt = ( \Phi | \Pi_{\gamma})
\geq 0, \textrm{with } | \phi)
\equiv | \delta\overline{S}(\gamma)/  \delta\gamma)
\end{equation}

The time evolution of the system corresponds to a trajectory in thermodynamic state space, which obeys the SEA variational principle. The corresponding variational problem consists of finding the instantaneous ``direction'' of \textbar{}\(\Pi_{\gamma}\)) which maximizes the entropy production rate \(\Pi_{S}\) subject to the conservation law constraint \(\Pi_{C_{i}} = 0\). To do this, the state space must have a metric field with which to compute the distance traveled during the evolution and the norm of \(\Pi_{\gamma}\). The differential of the distance traveled along the path in state space is then expressed as:

\begin{equation}
dl = \sqrt{(\Pi_{\gamma} | \overline{G}(\gamma) | \Pi_{\gamma})}dt 
\end{equation}

\(\overline{G}(\gamma)\) is a real, symmetric, positive-definite operator on the manifold which defines the thermodynamic state space \cite{li_steepest-entropy-ascent_2018-1}. The additional constraint (\(dl \)/\(dt \))\textsuperscript{2} is set equal to some small positive constant so that the norm of \(\Pi_{\gamma}\) is kept constant. The solution is found using the method of Lagrange multipliers where the Lagrangian is written as

\begin{equation}
\Gamma = \Pi_{S} - \sum_{i} \beta_{i}\Pi_{C_{i}} - \frac{\tau}{2} ( \Pi_{\gamma} | \overline{G}(\gamma) | \Pi_{(\gamma)} )
\end{equation}
\(\beta_{i}\) and \(\tau\)/2 are the Lagrange multipliers. The fermi distribution is expressed using the \(f\ \)term. \(\beta_{E}\) and \(\beta_{N}\) represent the Lagrange multipliers corresponding to the generators of motion \emph{Ĥ} and \emph{N} that corresponds to nonequilibrium properties of chemical potential and temperature, respectively. Taking the variational derivative with respect to \textbar{}\(\Pi_{\gamma}\)) and setting it equal to zero results in:

\begin{equation}
\frac{\delta\Gamma}{\delta\Pi_{\gamma}} = | \phi )
- \sum_{i} \beta_{i} | \psi_{i})
- \tau\overline{G} | \Pi_{\gamma}) = 0\  
\end{equation}

The SEA equation of motion takes the form:

\begin{equation}
|\Pi_{\gamma}) = L|\phi)
- L \sum_{i} \beta_{i} | \psi_{i}) 
\end{equation}

\(L \equiv {\overline{G}}^{- 1}\)/\(\tau\ \)is assumed to be diagonal. The diagonal terms of \(L\ \){[}\(\tau_{n^{k}}^{k},\, k = 1,\ldots,m;\, n^{k} = 0,1\) for fermion, and \(0,1,\ldots,\infty\ \)for bosons{]} are related to the relaxation times of the system single--particle eigenlevels so that:

\begin{equation}
L = \textrm{diag} \Bigl\{\frac{1}{\tau_{n^{k}}^{k}}\Bigl\}
\end{equation}

The values of the Lagrange multipliers are calculated by inserting eq. 9 into the conservation laws expressed by eq. 4 \& 5 resulting in:

\begin{equation}
\sum_{j=1}^{m+2} (\psi_{i}|L|\psi_{j}) \beta_{j} = ( \psi_{i}|L|\phi)
\end{equation}
The \(\beta_{j}\) can be used to define the measurements of nonequilibrium system intensive properties (e.g., temperature, pressure, and chemical potential). The equations of motion for the \(\gamma^{k}\) and the \(p_{n}^{k}\), of the probability distribution among the single-particle eigenlevels \(\epsilon^{k}\) can be written as:

\begin{equation}
\frac{d\gamma_{n^{k}}^{k}}{dt} = \frac{1}{\tau_{n^{k}}^{k}}( - \gamma_{n^{k}}^{k}ln(\rho_{n^{k}}^{k}) - n^{k}\epsilon^{k}\gamma_{n^{k}}^{k}\beta_{E} - n^{k}\gamma_{n^{k}}^{k}\beta_{N} - \gamma_{n^{k}}^{k}\beta_{l}^{k})
\end{equation}

\begin{equation}
\frac{d\rho_{n^{k}}^{k}}{dt} = \frac{1}{\tau_{n^{k}}^{k}}( - \rho_{n^{k}}^{k}ln(\rho_{n^{k}}^{k}) - n^{k}\epsilon^{k}\rho_{n^{k}}^{k}\beta_{E} - n^{k}\rho_{n^{k}}^{k}\beta_{N} - \rho_{n^{k}}^{k}\beta_{l}^{k})
\end{equation}
\(k\ \)stands for the single-particle eigenlevel index and \(n^{k}\) for the occupation number at this level, \(\beta_{E}\), \(\beta_{N}\), and \(\beta_{l}^{k}\) are, respectively, the Lagrange multipliers corresponding to the generators of motion \(\overline{H}\), \(\overline{N}\), and \({\overline{I}}_{k}\).

\smallbreak
\textbf{Hypoequilibrium state}
\smallbreak

The hypoequilibrium concept can simplify the expression for the equation of motion and facilitates the physical interpretation of the evolution in state \cite{li_generalized_2016,li_modeling_2016,li_atomistic-level_2014}. It is assumed of any significant generality that the particles occupying the same single-particle eigenlevel \(\epsilon\ \)are initially in mutual equilibrium with respect to the chemical potential \(\mu^{\epsilon}\) and temperature \(T^{\epsilon}\) so that the initial probability distribution \(p_{n^{\epsilon}}^{\epsilon}\) among the different occupation states is Maxwellian:

\begin{equation}
p_{n^{\epsilon}}^{\epsilon} = e^{- \beta_{l}^{\epsilon} - \beta_{N}^{\epsilon} - \beta_{E}^{\epsilon}n^{\epsilon}\epsilon}
\end{equation}

\(\beta_{E}^{\epsilon} \equiv \,\frac{1}{k_{B}T^{\epsilon}}\) is defined by the temperature \(T^{\epsilon}\), \(\beta_{N}^{\epsilon} \equiv \,\frac{\mu^{\epsilon}}{k_{B}T^{\epsilon}}\) by the chemical potential \(\mu^{\epsilon}\), and \(\beta_{l}^{\epsilon}\, \equiv \,\ln \,\Xi^{\epsilon}\) by the single-particle level partition function given by:

\begin{equation}
\Xi^{\epsilon}( \beta_{E}^{\epsilon},\beta_{N}^{\epsilon} ) = \sum_{n^\epsilon}^{\infty} e^{- \beta_{N}^{\epsilon}n^{k}}e^{- \beta_{E}^{\epsilon}n^{\epsilon}\epsilon}
\end{equation}

Such an initial state is called a hypoequilibirum state. It is assumed that the different occupation states of the same single-particle eigenlevel have the same relaxation time, namely:

\begin{equation}
\tau_{n^{\epsilon}}^{\epsilon} = \tau^{e} \textrm{ for all } n^{\epsilon} \textrm{ in
the same } \epsilon 
\end{equation}
This means that each relaxation time is a property of a given single-particle eigenlevel. Under these two conditions, it is proven that the system remains in a hypoequilibrium state throughout the entire time evolution \cite{li_steepest-entropy-ascent_2016,li_generalized_2016,li_steepest-entropy-ascent_2018}. As a consequence, the time evolution of the system can be determined via the motion of the of the state of a single-particle eigenlevel defined by

\begin{equation}
y^{\epsilon} = \beta_{N}^{\epsilon} + \beta_{E}^{\epsilon}\epsilon
\end{equation}
Taking the derivative with respect to time yields

\begin{equation}
\frac{dy^{\epsilon}}{dt} = - \frac{1}{\tau^{\epsilon}}( y^{e} - \beta_{E}\epsilon - \beta_{N})
\end{equation}

Using the previous equations, a systems extensive properties of particle number, energy, and entropy and integrating over \(n^{\epsilon}\), the contributions to these properties from a single-particle eigenlevel \(\epsilon\ \)provide the following evolutions:

\begin{equation}
\frac{d{\langle N\rangle}_{\epsilon}}{dt} = \frac{1}{\tau^{\epsilon}}A_{NN}^{\epsilon}( y^{\epsilon} - \beta_{E}\epsilon - \beta_{N})
\end{equation}

\begin{equation}
\frac{d{\langle e\rangle}_{\epsilon}}{dt} = \epsilon\frac{d{\langle N\rangle}_{\epsilon}}{dt}
\end{equation}

\begin{equation}
\frac{d{\langle s\rangle}_{\epsilon}}{dt} = y^{\epsilon}\frac{d{\langle N\rangle}_{\epsilon}}{dt}
\end{equation}
\({\langle N\rangle}_{\epsilon}\), \({\langle e\rangle}_{\epsilon}\), and \({\langle s\rangle}_{\epsilon}\) are the expectation values of the particle number, energy, and entropy of the single-particle eigenlevel. \(A_{NN}^{\epsilon}\) is the particle number fluctuation of the single-particle eigenlevel defined as:

\begin{equation}
A_{NN}^{\epsilon} = {\langle N^{2}\rangle}^{\epsilon} - ( {\langle N\rangle}^{\epsilon})^{2} = \frac{\partial^{2}}{\partial\beta_{N}^{\epsilon}}\textrm{ln}\Xi^{\epsilon} = - \frac{\partial{\langle N\rangle}^{\epsilon}}{\partial\beta_{N}^{\epsilon}} = \frac{1}{e^{y} \pm 1}\, \mp \,\frac{1}{( e^{y} \pm 1 )^{2}}
\end{equation}
Here, fermions are represented with the plus sign and the bosons are the negative sign.

\smallbreak
\textbf{Electron transport}
\smallbreak

The particle number evolution at a location \emph{A} is given by

\begin{equation}
\frac{d{\langle N\rangle}^{A}}{dt} = \int \frac{V}{\tau^{A,\epsilon}}A_{NN}^{A,\epsilon}( \beta_{N}^{A,\epsilon}\epsilon + \beta_{N}^{A,\epsilon} - \beta_{E}\epsilon - \beta_{N})D^{A}(\epsilon)d\epsilon
\end{equation}
\(D^{A}(\epsilon)\) is the DOS per unit volume at location \(A\ \)with volume \(V\). In the near equilibrium realm, it is assumed that systems \(A\ \)and \(B\ \)are both approximately in stable equilibrium so that:

\begin{equation}
\frac{d{\langle N\rangle}^{A}}{dt} = ( \beta_{N}^{A} - \beta_{N}) \int \frac{\tau^{A,\epsilon}}{V_{\epsilon}} A_{NN}^{A,\epsilon}D^{A}(\epsilon)d\epsilon\, + ( \beta_{E}^{A} - \beta_{E})\int \frac{V_{\epsilon}}{\tau^{A,\epsilon}}A_{NN}^{A,\epsilon}D^{A}(\epsilon)d\epsilon
\end{equation}

The near-equilibrium assumption permits retention of the zeroth-order approximation for the terms inside the integrals so that only the first-order approximation of \(d{\langle N\rangle}^{A}\)/\(dt\ \)is retained. It is assumed that system \(A\ \)and \(B\ \)have the same energy eigenstructure so that the relaxation times and DOS of \(A\ \)and \(B\ \)are the same. It is also assumed that the fluctuations of systems \(A\ \)and \(B\ \)are approximately equal to their mutual equilibrium value at \(\left( \beta_{N},\beta_{E} \right)\). The particle flow from \(B\ \)to \(A\ \)is then found that:

\begin{equation}
\frac{d{\langle N\rangle}^{A}}{dt} - \frac{d{\langle N\rangle}^{B}}{dt} = 2\frac{d{\langle N\rangle}^{A}}{dt} = ( \beta_{N}^{A} - \beta_{N}^{B}) \int \frac{V}{\tau^{\epsilon}}A_{NN}^{\epsilon}D(\epsilon)d\epsilon + ( \beta_{E}^{A} - \beta_{E}^{B}) \int \frac{V_{\epsilon}}{\tau^{\epsilon}}A_{NN}^{\epsilon}D(\epsilon)d\epsilon
\end{equation}
The second equal term is the result of particle conservation. The total particle flow to \(A\ \)can be written as:

\begin{equation}
AJ_{N} \equiv 2\frac{d{\langle N\rangle}^{A}}{dt} = \int \delta\lbrack \beta_{E}(\epsilon + \mu) \rbrack\frac{V}{\tau^{\epsilon}}A_{NN}^{\epsilon}D(\epsilon)d\epsilon
\end{equation}
\(A\) is the cross-sectional area of the interface separating systems \(A\) and \(B\). \(J_{N}\) is the particle flux across the interface between system \(A\) and \(B\). If system \(A\) is part of a series of local systems, the flow along a given direction for \(A\) is given by the contributions from two interfaces. By using the following variational relation for every energy level:

\begin{equation}
\delta\lbrack \beta_{E}(\epsilon + \mu) \rbrack = \left( (\epsilon + \mu)\frac{d\beta_{E}}{dx} + \beta_{E}\frac{d\mu}{dx}\right)\delta x
\end{equation}
\(\delta x\ \)Is the distance between locations \(A\ \)and \(B\ \). Eq. 27 can be rewritten as:

\begin{equation}
AJ_{N} = \delta x \int \frac{V}{\tau^{\epsilon}}\left( (\epsilon + \mu)\frac{d\beta_{E}}{dx} + \beta_{E}\frac{d\mu}{dx} \right)A_{NN}^{\epsilon}D(\epsilon)d\epsilon
\end{equation}

When the system is initially in a hypoequilibrium state, the relationship between the fluctuation \(A_{NN}^{\epsilon}\) and the fermi distribution \(f\ \)is expressed as:

\begin{equation}
A_{NN}^{\epsilon} = \beta_{E}^{- 1}\frac{\partial f}{\partial\epsilon},
\textrm{ with } f = \frac{1}{e^{\beta_{E}\epsilon + \beta_{N}} + 1} 
\end{equation}
The particle flux can then be rewritten as:

\begin{equation}
\begin{split}
J_{N} = \frac{V\delta x}{A}\frac{1}{\tau^{\epsilon}}\left( (\epsilon + \mu)\frac{d\beta_{E}}{dx} + \beta_{E}\frac{d\mu}{dx} \right)\beta_{E}^{- 1}\frac{\partial f}{\partial\epsilon}D(\epsilon)d\epsilon
\\ 
= - \frac{V\delta x}{A}\left( \frac{dE_{f}^{0}}{dx} + e\varepsilon \right)\frac{1}{\tau^{\epsilon}}\frac{\partial f}{\partial\epsilon}D(\epsilon)d\epsilon
\\ 
+ \frac{V\delta x}{A}\beta_{E}^{- 1}\frac{d\beta_{E}}{dx}\frac{1}{\tau^{\epsilon}}\left( \epsilon - E_{f} \right)\frac{\partial f}{\partial\epsilon}D(\epsilon)d\epsilon\
\end{split}
\end{equation}
where \(E_{f} = E_{f}^{0} + e\Phi = - \mu = - \frac{\beta_{N}}{\beta_{E}}\), \(E_{f}^{0}\) is the fermi level without an external field, \(- d\mu = dE_{f}^{0} + ed\Phi\) is the differential chemical potential, and \(\frac{d\Phi}{dx} = \varepsilon\) is the external field. Here, \(e\) and \(\Phi\) are the electric charge and electric field potential, respectively. As a comparison, the Boltzmann transport equation (BTE) in the low-field region results in the following particle flux expression \cite{wang_two-temperature_2012}:

\begin{equation}
J_{N} = - \frac{1}{3}\left( \frac{dE_{f}^{0}}{dx} + e\varepsilon \right)\int \tau^{\prime}v^{2}\frac{\partial f}{\partial\epsilon}D(\epsilon)d\epsilon\, - \,\frac{1}{3T}\frac{dT}{dx}\int \tau^{\prime}v^{2}\left( \epsilon - E_{f} \right)\frac{\partial f}{\partial\epsilon}D(\epsilon)d\epsilon
\end{equation}
where \(\tau^{\prime}\) is the relaxation time in the BTE given by:

\begin{equation}
\frac{\partial f}{\partial t} + \frac{\partial r}{\partial t} \bullet \nabla_{r}f + \frac{\partial p}{\partial t} \bullet \nabla_{p}f = \frac{f - f_{0}}{\tau^{\prime}(r,p)}
\end{equation}
where \(r\ \)and \(p\ \)are the particle position and momentum vectors, respectively. \(f\ \)is the particle probability density function and \(f_{0}\) is its local equilibrium value. The SEAQT and BTE relaxation time \(\tau^{\epsilon}\) and \(\tau^{\prime}\), respectively, are defined by different equations of motion. When the SEAQT relaxation times, the \(\tau^{\epsilon}\), are chosen via the relation:

\begin{equation}
\tau^{\epsilon} = \frac{\left( \frac{\delta x}{v_{x}} \right)^{2}}{\tau^{\prime}(\epsilon)} = \frac{3m(\delta x)^{2}}{2\epsilon\tau^{\prime}(\epsilon)}
\end{equation}
The SEAQT equation of motion recovers the BTE in the low-field region and its corresponding particle flux.

\smallbreak
\textbf{Phonon transport equation}
\smallbreak

The SEA transport equation for phonon transfer can be derived in a similar fashion to what was done for electron transport. However, there is no particle number conservation in phonon transport so a different set of conservation laws are used. Assuming as before initial hypoequilibirum states, the time evolution of energy at location \(A\ \)is expressed as:

\begin{equation}
\frac{d{\langle E\rangle}^{A}}{dt} = \int \frac{V}{\tau^{A,\epsilon}}\epsilon^{2}A_{NN}^{A,\epsilon}\left( \beta_{E}^{A,\epsilon} - \beta_{E} \right)D^{A}(\epsilon)d\epsilon
\end{equation}

In the near-equilibrium realm, it is assumed that systems \(A\ \)and \(B\ \)are both approximately in stable equilibrium. The three relations for phonons, which are the counterparts to the local equilibrium, energy eigenstructure, and the fluctuation assumptions made for electrons:

\begin{equation}
\beta_{E}^{A,\epsilon} = \beta_{E}^{A} 
 \textrm{ , } \beta_{E}^{B,\epsilon} = \beta_{E}^{B}
\end{equation}
\begin{equation}
\tau^{A,\epsilon} = \tau^{B,\epsilon} = \tau^{\epsilon} \textrm{ , } D^{A}(\epsilon) = D^{B}(\epsilon) = \beta(\epsilon)
\end{equation}
\begin{equation}
A_{NN}^{A,\epsilon} = A_{NN}^{B,\epsilon} = A_{NN}^{\epsilon}\left( \beta_{N},\beta_{E} \right)
\end{equation}

The energy flow from \(B\ \)to \(A\ \)is then given by subtracting from eq. 35 the corresponding one for system \(B\ \). Thus:

\begin{equation}
\frac{d{\langle E\rangle}^{A}}{dt} - \frac{d{\langle E\rangle}^{B}}{dt} = \frac{V}{A}\left( \beta_{E}^{A} - \beta_{E}^{B} \right)  \int \frac{\epsilon^{2}}{\tau^{\epsilon}}A_{NN}^{\epsilon}D(\epsilon)d\epsilon = \frac{V\delta x}{A}\frac{d\beta_{E}}{dx} \int \frac{\epsilon^{2}}{\tau^{\epsilon}}A_{NN}^{e}D(\epsilon)d\epsilon
\end{equation}
where the fluctuation \(A_{NN}^{\epsilon}\) can be written in terms of the boson distribution:

\begin{equation}
A_{NN}^{\epsilon} = - \frac{k_{B}T^{2}}{\epsilon}\frac{\partial f}{\partial T},
\textrm{ with } f = \frac{1}{e^{\beta_{E}\epsilon} - 1}
\end{equation}
assuming contributions from two interfaces, the energy flux can be written as:

\begin{equation}
J_{E} = \frac{V\delta x}{A}\frac{d\beta_{E}}{dx}\int \frac{\epsilon^{2}}{\tau^{\epsilon}}\left( - \frac{k_{B}T^{2}}{\epsilon}\frac{\partial f}{\partial T} \right)D(\epsilon)d\epsilon = \frac{V\delta x}{A}\frac{dT}{dx}\int \frac{\epsilon}{\tau^{\epsilon}}\frac{\partial f}{\partial T}D(\epsilon)d\epsilon
\end{equation}

\begin{equation}
= \frac{dT}{dx}\int \frac{(\delta x)^{2}}{\tau^{\epsilon}} \hbar \omega \frac{\partial f}{\partial T}D(\omega)d\omega
\end{equation}
The integral argument has been converted from the energy \(\epsilon\ \)into the frequency \(\omega\ \)after the last equal sign. The relaxation time is chosen as before via:

\begin{equation}
\tau^{\epsilon} = \frac{\left( \frac{\delta x}{v_{x}} \right)^{2}}{\tau^{\prime}(\epsilon)}
\end{equation}

\smallbreak
\textbf{Electron-Phonon Coupling}
\smallbreak

If phonons and electrons have temperatures \(\beta_{E}^{p}\) and \(\beta_{E}^{e}\) initially:

\begin{equation}
\frac{1}{\tau^{p}}A_{ES}^{p} + \frac{1}{\tau^{e}}A_{ES}^{e} = \beta_{E}\left( \frac{1}{\tau^{p}}A_{EE}^{p} + \frac{1}{\tau^{e}}A_{EE}^{e} \right) + \beta_{N}\frac{1}{\tau^{e}}A_{EN}^{e},
\end{equation}
\begin{equation}
\frac{1}{\tau^{e}}A_{NS}^{e} = \beta_{E}\frac{1}{\tau^{e}}A_{EN}^{e} + \beta_{N}\frac{1}{\tau^{e}}A_{NN}^{e}
\end{equation}
where \(A_{XY}^{e(p)}\) is the total fluctuation in a local electron (phonon) system. By solving for a system of equations found in the reference, the local temperature of electrons and phonons can be solved:

\begin{equation}
\beta_{E}^{p} = \frac{A_{ES}^{p}}{A_{EE}^{p}},\beta_{E}^{e} = \frac{A_{NN}^{e}A_{ES}^{e} - A_{EN}^{e}A_{NS}^{e}}{A_{NN}^{e}A_{EE}^{e} - A_{EN}^{e}A_{EN}^{e}}
\end{equation}
where \(\beta_{E}^{p}\) and \(\beta_{E}^{e}\) are the temperature of the phonon and electrons of the subsystem, respectively. The interactions with neighboring systems contributes to one term in the calculation of the relaxation time and temperature:

\begin{equation}
\frac{1}{{\overline{\tau}}^{e(p)}} = \frac{1}{\tau^{e(p)}} + \frac{1}{\tau^{e(p)}} + \frac{1}{\tau^{e(p)}},
\end{equation}
\begin{equation}
\frac{{\overline{\beta}}_{E}^{e}}{{\overline{\tau}}^{e}} = \frac{\beta_{E,ee}^{up}}{\tau^{e}} + \frac{\beta_{E,ee}^{down}}{\tau^{e}} + \frac{\beta_{E,ep}^{}}{\tau^{e}},
\end{equation}
\begin{equation}
\frac{{\overline{\beta}}_{E}^{p}}{{\overline{\tau}}^{p}} = \frac{\beta_{E,pp}^{up}}{\tau^{p}} + \frac{\beta_{E,pp}^{down}}{\tau^{p}} + \frac{\beta_{E,ep}^{}}{\tau^{p}}
\end{equation}

Subscripts ee, pp, and ep stand for electron diffusion, phonon diffusion, and electron-phonon coupling. \emph{Up} and \emph{Down} superscripts represent interaction between upstream and downstream local systems and can be seen in Fig. 2. Each electron and phonon subsystem is confined to a ``block'' which represents the physical location of the material relative to the other blocks in the system. The size of each block is represented by the SEAQT relaxation time parameter as defined in eq. 33 for electrons and eq. 42 for phonons. The equation for each electron takes the form:

\begin{equation}
\frac{d\beta_{E}^{e}}{dt} = \frac{{\langle\delta x\rangle}^{2}}{\tau^{e}}\frac{d^{2}\beta_{E}^{e}}{dx^{2}} - \frac{(1 - x)}{\tau^{e}}\left( \beta_{E}^{e} - \beta_{E}^{p} \right)
\end{equation}
Similarly, the temperature evolution of a phonon is given by:

\begin{equation}
\frac{d\beta_{E}^{p}}{dt} = \frac{(\delta x)^{2}}{\tau^{p}}\frac{d^{2}\beta_{E}^{p}}{dx^{2}} - \frac{x}{\tau^{p}}\left( \beta_{E}^{p} - \beta_{E}^{e} \right)
\end{equation}

Eqs. 49 \& 50 are equivalent to the two temperature model of electron-phonon coupling \cite{wang_two-temperature_2012}, where the first term to the right of the equal sign in both equations is the heat diffusion and the second term is the phonon-electron coupling.

% Figure environment removed

\smallbreak
\textbf{Transport properties}
\smallbreak

Eq. 51 describes the function \(X(\epsilon)\) that is characterized by electrical charge (e), velocity (\(\nu\ \)), relaxation time (\(\tau\ \)), and DOS per meter (\(D(\epsilon)\)). The \(\nu^{2}\tau(\epsilon)\,\)term can be changed by eq. 52 to the size of the subsystem (\(a\ \)) divided by \(\tau(\epsilon)\,\). Electrical conductivity can thus be calculated by following similar method as used in BoltzTraP code:

\begin{equation}
X(\epsilon) = e^{2}\,\nu^{2}\tau D(\epsilon)
\end{equation}
\begin{equation}
\nu^{2}\tau(\epsilon) = \frac{a^{2}}{\tau(\epsilon)} 
\end{equation}
\begin{equation}
\sigma = \int_{-\infty}^{\infty} \frac{\partial f}{\partial\epsilon}X(\epsilon)\, d\epsilon
\end{equation}
\(\frac{\partial f}{\partial\epsilon}\) represents the change in the fermi distribution based on the change in energy level. The thermal conductivity \(k\) is computed using the specific heat per unit frequency \(C_{\omega}\). The boson distribution is defined by \(f\ \). Here:

\begin{equation}
f = \frac{1}{e^{\beta_{E}\epsilon} - 1}
\end{equation}
\begin{equation}
C_{\omega} = \hbar \omega D(\omega)\frac{\partial f}{\partial T}
\end{equation}
\begin{equation}
k = \frac{1}{3}\int \tau^{\prime} v^{2} C_{\omega}d\omega
\end{equation}

Both electron and phonon play roles in the thermal conductivity with the main change between them being whether to use the fermi distribution for electrons and boson distribution for phonons. The calculation for the Seebeck coefficient follows a similar equation setup as follows in BoltzTraP code calculations. \(\mu\ \)in the equation is the chemical potential:

\begin{equation}
S = \frac{1}{eT}\frac{\int_{-\infty}^{\infty} (\epsilon - \mu)\frac{\partial f}{\partial\epsilon}X(\epsilon)d\epsilon}{\int_{-\infty}^{\infty} \frac{\partial f}{\partial\epsilon}X(\epsilon)d\epsilon}
\end{equation}

The figure of merit is a quantity used to define the performance of semiconductors for electrical applications where higher figure of merits are preferable. \emph{S} represents the seebeck coefficient, \(\sigma\ \)is electrical conductivity, \emph{T} is the temperature, and \(k\ \)is the thermal conductivity. Seebeck coefficient and conductivities are calculated at the given temperature:

\begin{equation}
zt = \frac{S^{2}\sigma T}{k}
\end{equation}

\smallbreak
\textbf{Energy Eigenstructures}
\smallbreak

Appropriate energy landscapes as inputs to the SEAQT framework for electrons and phonons are necessary for accurate calculations of nonequilibrium intensive properties. The most common formation of the electron energy landscape is through density functional theory (DFT) to obtain the electron DOS. Phonon DOS can be acquired through DFT calculations as well as experimental results. An additional parameter needed for the phonon input values in the SEAQT framework involves accurate calculations of the phonon relaxation time and phonon velocity based on the energy of the phonon energy landscape. Another method used to calculate the phonon DOS that will be used to study the phonon transport of Silicon and Germanium is a framework centered on elastic constants. The framework utilizes the continuum elastic wave equation to obtain particle displacements that can be used to obtain the energy landscape:

\begin{equation}
\rho\frac{\partial^{2}R}{\partial t^{2}} = \nabla \bullet \sigma
\end{equation}
\(\rho\) is the density of the material, R is the particle displacement, \(\sigma\) is the stress tensor. By knowing the elastic constants that go into forming the stress tensor, the DOS for a given frequency can be calculated using:

\begin{equation}
g_{n}(\omega) = \frac{1}{t}\left\lbrack \frac{k_{n}(\omega)}{2\pi V_{n}(\omega)} \right\rbrack
\end{equation}

Where \(t\ \)is the thickness of the material, \(k_{n}(\omega)\) is the x component of the wavevector for a given angular frequency, and \(V_{n}(\omega)\) is the group velocity given a specific angular
frequency.

\smallbreak
\textbf{Relaxation times and velocities}
\smallbreak

To appropriately model the systems that are investigated, electron/phonon relaxation times and phonon velocities need to be determined. While, typically, constant relaxation times are used to model systems, other more accurate models of relaxation time and velocity are necessary. These adjustments to the relaxation time and velocity come from experimental or computational methods that align within the given energy landscape of the electrons and/or phonons. This is important to provide the best results for thermal conductivity and figure of merit. However, constant relaxation time models for this work are sufficient in obtaining accurate thermal conductivity when considering phonon transport properties. The electron relaxation time, as shown in eq. 33, is also an important parameter for calculating the electrical conductivity and seebeck coefficient more accurately compared to traditional electron transport codes that only use constant relaxation time models that would not fully capture the proper electron velocity.

\smallbreak
\textbf{Results}
\smallbreak

In the following sections, the applied methods for calculating electrical and thermal conductivity, seebeck coefficient, and figure of merit will be used on studied semiconductors such as Silicon, doped Silicon, and Bi$_2$Te$_3$. Germanium is an additional case study that can be found in the supplementary materials. For all case studies, the systems analyzed will follow the setup shown in Fig. 2 where the simulation has 20 blocks, each with a length of 1e-7 meters, the same crystal structures, and the same orientation. In order for the SEAQT code to run appropriately, a small, induced temperature difference needs to be incorporated for the equation of motion to appropriately run. Therefore, the temperature of the first 10 blocks will be 298 K and the second 10 blocks will be 302 K. The electron DOS was calculated using DFT calculations as employed by VASP and using Local Density Approximation (LDA) pseudopotential for cell relaxation and static band structure calulations. 

\smallbreak
\textbf{Bi2Te3}
\smallbreak

Bi$_2$Te$_3$ takes on additional computational complexity compared to simpliar semiconductor systems due to the spin orbit coupling between the Bi and Te atoms. Bi$_2$Te$_3$ is a case study used to compare SEAQT code to the BoltzTraP code. The Phonon DOS was taken from other first principle calculations using the DFT method \cite{li_thermal_2015}. As is shown in the methods section, the relaxation time is important in determining the thermal conductivity of the material. This is especially true for an anisotropic material such as Bi$_2$Te$_3$ that changes depending on xx or zz directions \cite{hellman_phonon_2014}. The velocity for the phonons are dependent on the transverse and longitudinal directions and are taken from reference \cite{alnofiay_brillouin_2014} and are shown in Table 1. As for the relaxation times, they were taken from reference \cite{feng_prediction_2014}. The anisotropic change of the effective mass controls the electron relaxation time and is displayed in table 2 acquired from reference \cite{larson_electronic_2000}.

\begin{center}
Table 1: Phonon velocities for directions of the Bi$_2$Te$_3$ cell.
\smallbreak
\begin{tabular}{| c | c |} 
  \hline
  Direction & Phonon Velocity (m/s) \\ 
  \hline
  xx & 1800 \\ 
  \hline
  zz & 2600 \\ 
  \hline
\end{tabular}
\end{center}

Using the adjusted effective masses, the SEAQT Seebeck coefficient in either the xx or zz direction is in good agreement with BoltzTraP calculations. The thermoelectric properties were calculated using an averaging of the two directions of the anisotropic cell. Figures 5(a-d) show that the SEAQT thermoelectric properties match very well with the experimental data. The SEAQT thermal conductivity is shown to be higher compared to the experimental data. However, this discrepancy is due to the spark-plasma sintering method that is used to fabricate the experimental materials. The impacts of this fabrication method are explained in the discussion section. In order to compare single-crystal samples, the individual lattice thermal conductivity is found to be in good agreement with the literature which is displayed in fig. 6.

\begin{center}
Table 2: Bi$_2$Te$_3$ electron effective masses for different directions.
\smallbreak
\scalebox{0.89}{
\begin{tabular}{|c|c|c|}
  \hline
  Direction & Effective mass Conduction Band (free electron mass) & Effective mass Valance Band (free electron mass) \\
  \hline
  xx & 46.9 & 32.5 \\
  \hline
  zz & 9.5 & 9.02  \\
  \hline
\end{tabular}}
\end{center}

% Figure environment removed

% Figure environment removed

% Figure environment removed

% Figure environment removed

\break
\textbf{Silicon}
\smallbreak

Silicon is a highly studied semiconductor under both experimental considerations and DFT calculations \cite{dabbadie_enhancement_2013,fulkerson_thermal_1968,shanks_thermal_1963}.  Germanium was also studied and found to have good agreement with the experimental values. Germanium data can be found in the supplementary materials (Supplementary Table 1, Fig.2). The electron DOS was calculated using Density Functional Theory (DFT) as employed by VASP \cite{kresse_ab_1993,kresse_efficiency_1996,kresse_efficient_1996}. Local Density Approximation (LDA) pseudopotential is used along with hybrid HSE03 to correct for the inherent band gap problem in modeling semiconductors as seen in Fig. 6a. The obtained electron DOS is comparable to other computations \cite{noauthor_density_nodate}. As for the Phonon DOS, the heat conduction code that is described in the energy eigen structures section is used and is displayed in Fig.6b. Due to the material being isotropic, the direction does not impact the relaxation time of electrons. This also applies to the phonons where we consider the [1,0,0] direction and take the velocities and relaxation time from \cite{henry_spectral_2008}. The necessary input parameters of phonon group velocity and relaxation times are 6791 m/s and 10 ns, respectively \cite{lacroix_phonon_2009,ward_intrinsic_2010}. The thermoelectric properties are displayed in Fig. 7 and are compared to experimental data. 


% Figure environment removed

% Figure environment removed

\break
\textbf{Doped Silicon}
\smallbreak

The doping of Silicon follows similar methods that BoltzTraP assumes where the doping levels do not substantially impact the electron DOS, but primarily adjusts the fermi level depending on the type of doping \cite{walukiewicz_intrinsic_2001}. This change in fermi level is crucial in modeling the change in transport properties of the doped Si in the SEAQT framework. The change in fermi level is taken from the BoltzTraP code that determines the change in fermi level depending on the additional charge carriers of the doping material \cite{madsen_boltztrap2_2018}. The doping levels were chosen based on modeling electrical conductivity levels for experimental n and p-type doping \cite{pearson_electrical_1949}. The concentration of dopants studied here does not substantially adjust the phonon DOS. The primary change in phonon transport properties comes from the change in phonon relaxation time due to the additional scattering by impurities, phonon scattering on free electrons/holes, and bound electrons/holes where the calculated phonon relaxation time can be found in table 3  \cite{asheghi_thermal_2002}. As can be seen in Figs. 9a, 9b, and 9d, the SEAQT electrical conductivity and seebeck coefficient are in good agreement with the experimental results. Fig. 9c shows the thermal conductivity contribution of the electrons to the total thermal conductivity. The reduction of thermal conductivity in the SEAQT model for phosphorus doping follows much more closely to experimental values compared to the Boron doping. This could be due to the inherent higher dislocation concentration from the higher lattice mismatch between boron atoms and silicon atoms which would decrease thermal conductivity additionally. It can be seen in Fig. 10 that the electron thermal conductivity increases with increasing doping levels as expected with increasing electrical conductivity, however the contribution is lower than the phonon contribution.

\begin{center}
Table 3: Doped Silicon Relaxation Times.
\smallbreak
\begin{tabular}{| c | c |}
  \hline
  Doping and Concentration & Relaxation Time (ns) \\ 
  \hline
  P-type 2e16 & 9.522\\ 
  \hline
  P-type 1e20 & 4.634 \\ 
  \hline
  N-type 2e17 & 9.137 \\ 
  \hline
  N-type 1e18 & 6.083 \\ 
  \hline
  N-type 2e19 & 5.268 \\ 
  \hline
  N-type 1e20 & 3.942 \\ 
  \hline
\end{tabular}
\end{center}

% Figure environment removed

% Figure environment removed

\break
\textbf{Discussion}
\smallbreak
The electrical conductivity and seebeck coefficient are controlled by the mobility of electrons and electron holes that is controlled by the fermi level in semiconductors. At 0 K, electrons remain in the valance band and do not move toward the conduction band. With an increasing temperature, the fermi level increases, allowing for some electrons to move into the conduction band leaving behind holes in the valance band, contributing towards electrical conductivity. As for the seebeck coefficient, the induced voltage caused by a temperature gradient is due to the fact that electrons are electrical and thermal carriers. With an induced heat gradient, electrons move from a high temperature end to a low temperature end, causing an electric current \cite{le_comber_electrical_1979}. The magnitude and sign of the seebeck coefficient are related to an asymmetry of the electron distribution around the Fermi level. Fermi level changes with temperature which adjusts the electron and hole concentration, thus adjusting the seebeck magnitude (concentration increase/decrease and electron/hole mobility) as well as the sign (n-type for excess electrons and p-type for excess holes) \cite{jiang_thermoelectric_2008, mahan_seebeck_2014}.

It is important to show the validity and reliability of the SEAQT model by comparing the results to the standard methods. In the case of this paper, SEAQT is compared to BoltzTraP calculations based on VASP electron DOS which are then compared to the experimental data. For the first comparison of the Bi$_2$Te$_3$ system, SEAQT is compared to the experimental data which is shown in Fig. 5. The electrical conductivity, seebeck coefficient, and zt factor are in good agreement with the provided references where the SEAQT data is generally the average of the reference values. The disparities of the thermoelectric properties from the references between each other potentially come from the fabrication method that is explained later in the discussion section. This is also the case for the thermal conductivity where the SEAQT values are higher compared to experimental values, but the SEAQT lattice thermal conductivity compares well with experimental results as shown in Fig. 6. For Silicon, SEAQT and BoltzTraP are compared to experimental data which is shown in Fig.  8. For the electrical conductivity, BoltzTraP shows two curves with an upper and lower limits of potential constant electron relaxation times given to be either 10$^{-14}$ or 10$^{-15}$ seconds. The experimental data shows similar trends of increasing electrical conductivity with similar magnitudes as shown by SEAQT and in between expected electrical conductivity with the BoltzTraP data. The seebeck coefficient of the SEAQT data is in very close agreement with the references with the BoltzTraP also being close to the experimental data, but does not undergo as sharp a decrease in seebeck coefficient at higher temperatures. The SEAQT thermal conductivity is in very close agreement with the references and the SEAQT ZT factor is also close to the experimental data. For the doped Silicon system, SEAQT and BoltzTraP with a constant relaxation time of 10$^{-14}$ seconds are compared to experimental data in Fig. 9. The SEAQT data for the electrical conductivity compares well with the experimental data showing similar magnitudes and a decreasing trend of electrical conductivity with increasing temperature. The BoltzTraP for electrical conductivity does not compare as well to the experimental data where, depending on the doping levels, does not have correct magnitudes nor displays the decreasing electrical conductivity over increasing temperature trend. The seebeck coefficient also shows similar issues where the SEAQT data is closer to the expeirmental data than BoltzTraP calculations, but both SEAQT and BoltzTraP show correct n-type and p-type negativity and positivity, repectivily. The thermal conductivity of the SEAQT data decreases with increasing doping concentrations due to the adjusted relaxation times from the scattering mechanisms of the impurity atoms. There is slight discrepancy between the SEAQT and experimental data where the SEAQT values are greater. This could be potentially due to the adjustment of the relaxation time and density of states that do not include other defects that could be present in the samples such as dislocations that would impact the thermal conductivity. Even with the slight discrepancy with the thermal conductivity, the ZT factor of SEAQT matches very well with the provided references at the given doping levels. 

The most obvious limitation of the BoltzTraP code comes from using the constant relaxation time model that does not have the most accurate predictions of electrical conductivity when looking at the silicon electrical conductivity. The variability from the constant relaxation time model leads to ambiguity of the results showing either an overestimation or an underestimation that can be by a factor of 10. The largest disparity comes from analyzing the doped Silicon systems in Figs. 9a and 9b. At lower doping levels, such as n-doped 1.05e17 1/cm$^3$, the BoltzTraP electrical conductivity is close to the experimental value, but does not have the correct trend of decreasing electrical conductivity with increasing temperature. With higher doping, such as n-doped 2.6e20 1/cm$^3$, the BoltzTraP value has approximately a factor of 5 difference between the experimental and SEAQT values. This overestimation of electrical conductivity from BoltzTraP could come from the exclusion of the electron-phonon coupling that influences electron mobility. Generally, electron-phonon scattering results in lower electron mobility, directly reducing the electrical conductivity \cite{vaurette_evidence_2008}. When analyzing the seebeck coefficient, SEAQT analysis of doped Silicon (Fig. 9d) reproduces experimental values more accurately than BoltzTraP. This is due to the constant relaxation time approximation employed by BoltzTraP that is generally appropriate when the relaxation time is not energy dependent \cite{fedorova_anomalous_2022}. However, when the relaxation time is energy dependent, the seebeck coefficient can deviate significantly. This problem is already addressed in the SEAQT framework that inherently employs energy dependency through the electron relaxation time parameter in eq. 33 that has the energy dependency in the denominator of the expression. 


When analyzing the increase in the thermal conductivity for the doped Silicon, this is proportional following the Lorentz number that states the increase in electrical thermal conductivity is proportional to the electrical conductivity given by:

\begin{equation}
\frac{\kappa_{E}}{\sigma} = LK 
\end{equation}
where \(\kappa_{E}\) is the thermal conductivity due to electrons, \(K\ \)is the temperature, and \(L\ \)is the Lorentz number that is approximated to be \(2.4\, \times \, 10^{- 8}\) \(W\Omega K^{2}\) and \(1.5 \times 10^{- 8}\) \(W\Omega K^{2}\) for nondegenerate semiconductors \cite{thesberg_lorenz_2017,kim_characterization_2015}. The contributions of the doped silicon is shown to follow the Lorentz number by increasing the total thermal conductivity dependent on the equation:
\begin{equation}
\tau_{tot} = \tau_{l} + \tau_{e}
\end{equation}
where \(\tau_{l}\) is the thermal conductivity of the phonon and \(\tau_{e}\) is the thermal conductivity of the electrons. At lower electrical conductivities, the electron contribution remains low but increases with increasing doping levels that correspond to increasing electrical conductivity. In Fig. 10, it is shown that increasing doping has an increasing electrical thermal conductivity, but it is not high enough to substantially impact the total thermal conductivity of the doped system.

Another unique quality about the SEAQT framework compared to other computational models is the calculation of many different materials characteristics that allows for the calculation of the figure of merit. The SEAQT data provide data closer in agreement with experimental data than BoltzTraP for different systems, but can demonstrate high discrepancies with some experimental data. This is due to the fact that bulk materials (or any materials that would be analyzed) will have inherent defects that cause additional scattering of electrons and phonons \cite{zhao_defects_2020,bodnarchuk_structural_2011}. 1 dimensional (vacancies, interstitial atoms, substitutional atoms, line dislocations), 2 dimensional (stacking faults, grain boundaries, twin boundaries), and 3 dimensional (precipitates, intermetallic phases, pores, cracks) defects increase the scattering mechanisms with increasing defect dimension and concentration \cite{zhao_defects_2020,bodnarchuk_structural_2011}. Vacancy clusters are known to impact electrical conductivity due to dangling bonds, floating bonds, and high strain atoms that generate a localized state where electrons are confined to a certain region and causes it to behave like a metal as well as increasing lower frequency phonons and decreasing phonon velocities \cite{huang_effects_2014}. Dislocations interfere with the symmetry of perfect crystal which changes the band gap energy and have varying effects on electrical conductivity due to either behaving as an acceptor or donor \cite{basu_structure_2011}. The thermal conductivity decreases with dislocation density with the magnitude change of single dislocations being dependent on orientation in the lattice \cite{cheng_phonondislocation_2021}. These defects could potentially explain the higher thermal conductivity of the SEAQT Bi$_2$Te$_3$ as shown in Fig. 5b where the experimental samples would have higher inherent defects, such as line dislocations and grain boundaries, that reduce thermal conductivity in  the experimental samples. Separating SEAQT Bi$_2$Te$_3$ thermal conductivity into its lattice and electron contributions as done in eq. 62, the lattice conductivity matches well with experimental and molecular dynamics simulations as shown in the Fig. 10 that comes from very pure samples and perfect simulation conditions. The experimental thermal conductivity samples in Fig. 5b are much lower than the lattice thermal conductivity data and can be attributed to the reduction of the phonon contribution due to the phonon scattering on grain boundaries and dislocations of the samples \cite{ge_enhanced_2018}. These defects are inherent depending on the type of fabrication process where the Bi$_2$Te$_3$ samples were constructed using spark plasma sintering (SPS) method \cite{ge_enhanced_2018,li_thermoelectric_2012,ivanov_metal-ceramic_2022,yaprintsev_effects_2018}. Defect free materials produced from SPS is extremely difficult to construct due to the many separate mechanisms that need to be controlled to adjust the microstructure. These include temperature inhomogeneities due to complex current distribution, non-uniform stress distributions, die material, applied electric field, applied pressure, and heating rates \cite{anselmi-tamburini_spark_2021}. It is incredibly difficult to adjust these specific mechanisms for fabricating pure, bulk materials as understanding what mechanisms control which microstructures are still under-researched with the main theories being plasma generation, electroplastic effect, and joule heating \cite{anselmi-tamburini_spark_2021,suarez_challenges_2013}, leading to materials with inherent defects that change the thermoelectric properties compared to single-crystal materials. This issue primarily arises with small grain sizes that are inherent with the SPS method due to the micropowder that is sintered, leading to much lower thermal conductivities compared to single-crystal samples \cite{suarez_challenges_2013}. The SEAQT framework can be appropriately adjusted to account for these defects by changing the input parameters appropriately. This involves many methods such as obtaining DOS of the materials through experimental methods or using other computational methods such as employing DFT calculations with imputed defects to obtain the adjusted electron/phonon DOS, as well as adjusting relaxation time and group velocities as seen in doped silicon that decreased the relaxation time of phonons to produce appropriate thermal conductivities and other thermoelectric properties. 

\smallbreak
\textbf{Conclusion}
\smallbreak 

Utilization of the Steepest Entropy Ascent Quantum Thermodynamics framework is shown to accurately reproduce experimental results of semiconductor materials. SEAQT takes in electron and phonon energy landscapes to determine material transport properties that can not be determined by current computational/physical methods. SEAQT inherently satisfies the 1st and 2nd laws of quantum mechanics and thermodynamics that utilizes an equation of motion to determine the unique non-equilibrium thermodynamic path through Hilbert space that is able to cross all spatial and temporal scales in a single analysis. The SEAQT method is used to determine the transport properties of Bi2Te3, Silicon, and doped Silicon with good accuracy and provides similar results to the current computational and physical standards while taking into account the coupling of electron and phonon phenomenon that gives more accurate results. The robustness of the SEAQT framework is able to be applied to a multitude of systems and will be used to analyze the effect of defect structures and thermoelectric breakdown of materials in future work.

\smallbreak
\textbf{Declaration of Competing Interests}
\smallbreak

The authors declare no competing interest.
\smallbreak
Supplementary Information is available for this paper.

%\bibliographystyle{unsrt}
%\bibliography{SEAQT_references.bib}
\printbibliography
\end{document}

