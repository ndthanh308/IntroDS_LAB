\documentclass{article}
\usepackage{graphicx} % Required for inserting images
%\graphicspath{ {./Images/} }
\usepackage{caption}
\usepackage[margin=1in]{geometry}
\usepackage{breqn}
\usepackage{array}
\usepackage{authblk}
\usepackage{xcolor}
\usepackage[utf8]{inputenc}
\usepackage{amssymb}
\usepackage{eqnarray,amsmath}
\usepackage{mathrsfs}
\usepackage{tabularx}
\DeclareUnicodeCharacter{2212}{-}
%%\DeclareUnicodeCharacter{U+2212}{{-}}
%\usepackage{unicode-math}
% \usepackage[backend=biber,sorting=none,maxbibnames=200,style=phys]{biblatex}
\usepackage[style=nature]{biblatex}
\addbibresource{SEAQT_references.bib}

\title{\textbf{Supplementary Materials}} 

\author{\textbf{Jarod Worden$^a$, Michael von Spakovsky$^a$, Celine Hin$^{a,b}$}}


\affil{\normalsize{\textit{$^a$ME Dept., Center for Energy Systems Research, Virginia Tech, Blacksburg, VA 24061, United States}}}
\affil{\normalsize{\textit{$^b$MSE Dept., Northern Virginia Center, Virginia Tech, Falls Church, VA 22043, United States}}}


\begin{document}

\date{}

\maketitle




\section*{Appendix A: Si}
\smallbreak

Figs. S1 a) and b) depict the electron and phonon density of states, respectively. The electron relaxation parameters calculated using Eq. (34) from the main text are shown in Figure S2. To calculate the SEAQT phonon relaxation parameter using Eq. (34), the phonon velocity and relaxation time are required. The phonon velocity, obtained from DFPT calculations with Phonopy \cite{togo_implementation_2023}, is 4066 m/s. The phonon relaxation time, taken as 0.4 ns, is obtained from \cite{henry_spectral_2008}.
% Figure environment removed

% Figure environment removed

% Figure environment removed

% Figure environment removed

\section*{Appendix B: Bi$_2$Te$_3$}
Bi$_2$Te$_3$ serves as a case study for applying the SEAQT framework to predicting the transport properties of a narrow-bandgap semiconductor. The presence of spin-orbit coupling between Bi and Te atoms introduces additional computational intricacy, particularly concerning the electron density of states (DOS) compared to simpler semiconductor systems. VASP is used to determine the e-DOS as shown in Fig. S3. Conversely, the phonon DOS, depicted in Fig. S4, is extracted from first-principle DFPT calculations \cite{li_thermal_2015}. In addition, the phonon relaxation parameter ($\tau^\epsilon$) plays an important role in modulating the material's thermal conductivity, particularly in the context of anisotropic materials like Bi$_2$Te$_3$ where the thermal conductivity varies along the $xx$- or $zz$-directions (i.e., transverse and longitudinal directions) \cite{hellman_phonon_2014}. This anisotropy influences phonon group velocities, which are sourced from \cite{alnofiay_brillouin_2014} and presented in Table S1. The calculated phonon relaxation parameters, $\tau^\epsilon$, also appear in Table S1 and are based on the phonon velocities and a constant phonon relaxation parameter, $\tau^\prime$, of 2.210$^{-11}$s obtained from reference \cite{feng_prediction_2014}. Per Eq. (34) of the main text, the phonon $\tau^\epsilon$  does not vary with the eigenenergies since $\tau^\prime$ does not in this case. 

As to the electron $\tau^\epsilon$, these are affected by the anisotropic change of the effective mass as well as by the eigenenergies (see Eq. (34) of the main text again). These masses are given in Table S2 and are taken from \cite{larson_electronic_2000}. Averaging the effective masses in each direction, the electron $\tau^\epsilon$  in the $xx$- and $zz$-directions as a function of the eigenenergies are reported in Figs. S5 and S6, respectively.
To evaluate the comprehensive thermoelectric characteristics of Bi$_2$Te$_3$,  Matthiessen's rule \cite{chen_nanoscale_2005} is used to derive an aggregate relaxation parameter ($\tau^\epsilon$) as a function of the eigenenergies given in Figs. S5 and S6. 

% Figure environment removed

% Figure environment removed

Fig. S7 shows that the SEAQT transport property predictions for Bi$_2$Te$_3$ relative to those from various experimental studies \cite{ge_enhanced_2018,li_thermoelectric_2012,ivanov_metal-ceramic_2022,yaprintsev_effects_2018}, accurately capture the trends in the electrical and thermal conductivities, the Seebeck coefficient, and the figure of merit ($ZT$ factor). While the values align generally with the experimental data, disparities exist between the various experimental studies. For instance, the SEAQT thermal conductivity (Fig. S7b)) tends to be slightly higher compared to experimental data. However, this variance arises from predicted values based on a pristine crystal structure devoid of defects, whereas the experimental samples, fabricated via spark-plasma sintering, differ due to the presence of defects. The implications of this fabrication method are elaborated in Appendix C. Conversely, Fig. S8 compares the SEAQT outcomes for the phonon (lattice) thermal conductivity (i.e., excluding the electron contribution) with experimental \cite{satterthwaite_electrical_1957,goldsmid_thermal_1956} and molecular dynamic results \cite{qiu_molecular_2009}, showcasing significant agreement. Nevertheless, as evidenced in Fig. S8, excluding the electron contribution, notably underestimates total thermal conductivity for narrow-bandgap materials like Bi$_2$Te$_3$. Thus, simultaneous consideration of both electron and phonon contributions is imperative.

% Figure environment removed

% Figure environment removed

\section*{Appendix C: Bi$_2$Te$_3$ Thermal Conductivity Discussion}

Experimental Bi$_2$Te$_3$ samples typically exhibit higher levels of inherent defects, including line dislocations and grain boundaries, which reduce thermal conductivity. When separating the SEAQT Bi$_2$Te$_3$ thermal conductivity into lattice and electron contributions, the SEAQT phonon (lattice) thermal conductivity predictions closely match the experimental and molecular dynamic (MD) simulations as shown in Fig. S8, since the experimenal samples are relatively pure and the MD simulation conditions idealized. This is born out by comparing the experimental total thermal conductivity values in Fig. S7b), particularly at low temperatures, with those in Fig. S8. As can be seen, they are significantly lower than the experimental phonon thermal conductivity data in Fig. S8 due to phonon scattering on grain boundaries and dislocations \cite{ge_enhanced_2018}. These defects are intrinsic to the spark plasma sintering (SPS) fabrication method used for the Bi$_2$Te$_3$ samples in Fig. S7b) \cite{ge_enhanced_2018,li_thermoelectric_2012,ivanov_metal-ceramic_2022,yaprintsev_effects_2018}. Fabricating defect-free materials via SPS is challenging due to various mechanisms that need to be controlled, including temperature inhomogeneities, non-uniform stress distributions, die material, applied electric field, applied pressure, and heating rates \cite{anselmi-tamburini_spark_2021}. Despite ongoing research into controlling these mechanisms, primarily through plasma generation, the electroplastic effect, and Joule heating, SPS often results in materials with inherent defects that alter thermoelectric properties compared to single-crystal materials \cite{anselmi-tamburini_spark_2021,suarez_challenges_2013}. This issue is exacerbated by small grain sizes inherent to the SPS method due to the sintering of a micropowder, leading to significantly lower thermal conductivities compared to single-crystal samples \cite{suarez_challenges_2013}. While alternative methods have been shown to produce single-crystal Bi$_2$Te$_3$ samples with lower defect concentrations compared to SPS, there is a lack of thermoelectric property analysis within the temperature range of interest to compare with the SEAQT thermal conductivity data \cite{strieter_growth_1961,ainsworth_single_1956}.



\hfill \break









\hfill \break

\begin{table}[h!tbp]
\centering
\captionsetup{format=hang} \caption*{Table S1: Doped Si relaxation parameter, $\tau^\epsilon$.}
\smallbreak
\begin{tabular}{| c | c |} 
  \hline
  Doping Concentration (1/cm$^{-3}$) & Relaxation Parameter, $\tau^\epsilon$ (ps) \\ 
  \hline
  p-type 2 x 10$^{16}$ & 1.375 \\
  \hline
  p-type 1 x 10$^{20}$ & 2.827 \\
  \hline
  n-type 2 x 10$^{17}$ & 1.433 \\
  \hline
  n-type 1 x 10$^{18}$ & 2.153 \\
  \hline
  n-type 2 x 10$^{19}$ & 2.487 \\
  \hline
  n-type 1 x 10$^{20}$ & 3.323 \\
  \hline
\end{tabular}
\label{SupTable:1}
\end{table}

\begin{table}[h!tbp]
\centering
\captionsetup{format=hang} \caption*{Table S2: 
 Phonon velocity and calculated phonon relaxation parameter, $\tau^\epsilon$, for the $xx$- and $zz$-directions of the Bi$_2$Te$_3$ cell.}
\smallbreak
\begin{tabular}{| c | c | c |} 
  \hline
  Direction & Phonon Velocity (m/s) & Phonon Relaxation Parameter, $\tau^\epsilon (s)$ \\ 
  \hline
  $xx$ & 1800 & 1.4 x 10$^{-10}$ \\
  \hline
  $zz$ & 2600 & 6.7 x 10$^{-10}$\\
  \hline
\end{tabular}
\label{SupTable:2}
\end{table}

\begin{table}[h!tbp]
\centering
\captionsetup{format=hang} \caption*{Table S3: Bi$_2$Te$_3$ effective free electron masses for different directions.}
\smallbreak
\begin{tabularx}{16cm}{| X | X | X |} 
  \hline
  Direction & Effective Free Electron Mass for the Conduction  Band & Effective Free Electron Mass for the Valance Band \\ 
  \hline
  $xx$ & 46.9 & 32.5 \\
  \hline
  $zz$ & 9.5 & 9.02\\
  \hline
\end{tabularx}
\label{SupTable:3}
\end{table}

\hfill
\printbibliography
\emergencystretch 3em
\end{document}