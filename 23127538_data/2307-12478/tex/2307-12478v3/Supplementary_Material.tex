\documentclass{article}
\usepackage{graphicx} % Required for inserting images
\graphicspath{ {./Images/} }
\usepackage{caption}
\usepackage[margin=1in]{geometry}
\usepackage{breqn}
\usepackage{array}
\usepackage{authblk}
\usepackage{xcolor}
\usepackage[utf8]{inputenc}
\usepackage{amssymb}
\usepackage{eqnarray,amsmath}
\usepackage{mathrsfs}
\DeclareUnicodeCharacter{2212}{-}
%%\DeclareUnicodeCharacter{U+2212}{{-}}
%\usepackage{unicode-math}
\usepackage[backend=biber,sorting=none,maxbibnames=200,style=phys]{biblatex}
\addbibresource{SEAQT_references.bib}

\title{\textbf{Supplementary Materials}} 
\author{Jarod Worden$^\ddagger$, Micheal von Spakovsky$^\ddagger$, Celine Hin$^{\dagger}$}
\affil{$^\dagger$MSE Dept., Virginia Tech, Falls Church, VA 22043}
\affil{$^\ddagger$Center for Energy Systems Research, ME Dept., Virginia Tech, Blacksburg, VA 24061}
\date{}

\begin{document}

\maketitle

\section{Germanium}
\smallbreak

The SEAQT framework is applied to germanium to calculate the thermodynamic properties that are described in the main body work. The electron and phonon density of states are calculated following similar techniques used in acquiring the density of states for Silicon and are displayed in Fig. 1. The SEAQT relaxation time parameter is calculated from Eq. 34 and is displayed in Fig. 2. with Table 1 showing the phonon relaxation time parameter calculated from Ref \cite{sahasrabudhe_temperature_2023}. Germanium transport properties are calculated using the same structural parameters as shown in Fig. 2 of the main document. Fig. 3 shows the electrical conductivity, thermal conductivity, Seebeck coefficient, and ZT Factor for SEAQT, Boltztrap, and experimental data. The SEAQT electrical conductivity data follows the expected trends of the experimental data and Boltztrap curves. The larger Boltztrap curve uses a constant relaxation time of 1e-14 s and the smaller curve uses a constant relaxation time of 1e-15 s. These values correspond with the the BTE constant relaxation time approximation. The SEAQT Seebeck coefficient follows the experimental curves much more closely compared to the Boltztrap values. This can be attributed to the energy dependent relaxation time that SEAQT employs and is explained in the main text. The SEAQT thermal conductivity also follows expected trends by other experimental data. The SEAQT ZT factor follows expected trends with the literature, but is shown to be lower than the experimental data at higher temperatures. This discrepancy appears to primarily come from the lower Seebeck coefficient at higher temperatures. This could potentially be an indication that the electron DOS for germanium is not as accurate as some of the other materials DOS conducted in this work, but still displays the expected trends for the material, with the ZT factor approaching a maximum with increasing temperatures. 

\begin{center}
Table 1: Germanium phonon group velocity and SEAQT relaxation time parameter.
\smallbreak
\begin{tabular}{| c | c | c |} 
  \hline
  System & Velocity (m/s) & Relaxation Time (ps) \\ 
  \hline
  Germanium & 4777 & 1.46$\times10^{-11}$ \\ 
  \hline
\end{tabular}
\end{center}

% Figure environment removed

% Figure environment removed

% Figure environment removed

% \bibliographystyle{unsrt}
% \bibliography{SEAQT_references.bib}
\printbibliography
\end{document}