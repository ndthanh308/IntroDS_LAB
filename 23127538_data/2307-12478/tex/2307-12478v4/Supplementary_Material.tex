\documentclass{article}
\usepackage{graphicx} % Required for inserting images
\graphicspath{ {./Images/} }
\usepackage{caption}
\usepackage[margin=1in]{geometry}
\usepackage{breqn}
\usepackage{array}
\usepackage{authblk}
\usepackage{xcolor}
\usepackage[utf8]{inputenc}
\usepackage{amssymb}
\usepackage{eqnarray,amsmath}
\usepackage{mathrsfs}
\DeclareUnicodeCharacter{2212}{-}
%%\DeclareUnicodeCharacter{U+2212}{{-}}
%\usepackage{unicode-math}
\usepackage[backend=biber,sorting=none,maxbibnames=200,style=phys]{biblatex}
\addbibresource{SEAQT_references.bib}

\title{\textbf{Supplementary Materials}} 
\author{Jarod Worden$^b$, Michael von Spakovsky$^b$, Celine Hin$^a$}
\affil{$^a$MSE Dept., Virginia Tech, Falls Church, VA 22043}
\affil{$^b$Center for Energy Systems Research, ME Dept., Virginia Tech, Blacksburg, VA 24061}
\date{}

\begin{document}

\maketitle

\section{Ge}
\smallbreak

The SEAQT framework is applied to Ge to calculate the transport properties that are described in the main body of this work. The electron and phonon DOS for Ge are displayed in Fig \ref{FigDOS} and were determined using the same methods, i.e., VASP and the elastic
constants/elastic wave equation method, used in acquiring the DOS for Si. The SEAQT electron relaxation parameter, $\tau^\epsilon$, is based on Eq. (34) of the main body of this work and is displayed in Fig. \ref{FigTau} with Table \ref{TableVelocityTau} providing the phonon relaxation parameter, $\tau^\epsilon$, which is based  on Ref \cite{sahasrabudhe_temperature_2023} and Eq. (34). The Ge transport properties are determined using the same network of local non-equilibrium systems shown in Fig. 2 of the main body of this work. Fig. \ref{FigTransProp} shows comparisons of the SEAQT and BoltzTrap predicted values for the electrical conductivity, thermal conductivity, Seebeck coefficient, and $ZT$ factor to experimental data \cite{dabbadie_enhancement_2013,meddins_apparatus_1969,glassbrenner_thermal_1964}. The SEAQT predicted electrical conductivity (upper left figure) follows the expected trends of the experimental data and the two BoltzTrap curves. The upper BoltzTrap curve is based on the default BoltzTrap constant relaxation parameter of 1$\times10^{-14}$ s, while the lower curve corresponds to a constant relaxation time of 1$\times10^{-15}$ s. The quantitative differences between the SEAQT results and the experimental values at higher temperatures are due to the fact that the experimental samples are not defect free. In fact, as seen with the thermal conductivity (upper right figure), the presence of defects in the experimental samples also causes a spread in the experimental data. Nonetheless, the SEAQT results compare well with experiment. As to the Seebeck coefficient (lower left figure), the SEAQT results follow the experimental curves much more closely than those predicted by BoltzTrap where it is noted that the two BoltzTrap curves for the two different relaxation parameter values essentially fall on top of each other in this figure. The SEAQT $ZT$ factor (lower right figure) follows the trends of the experimental sample results  but is somewhat lower than those results at higher temperatures. This discrepancy is primarily due to the lower Seebeck coefficient predicted by the SEAQT framework at higher temperatures as a result of the SEAQT model being defect free.   

\begin{table}
\centering
\captionsetup{format=hang} \caption{Ge phonon velocity and relaxation parameter. $\tau^\epsilon$, taken from \cite{sahasrabudhe_temperature_2023}.}
\smallbreak
\begin{tabular}{| c | c | c |} 
  \hline
  System & Phonon Velocity (m/s) & Phonon Relaxation Parameter, $\tau^\epsilon$ (ps) \\ 
  \hline
  Germanium & 4777 & 4.86$\times10^{-14}$ \\ 
  \hline
\end{tabular}
\label{TableVelocityTau}
\end{table}

% Figure environment removed

% Figure environment removed

% Figure environment removed

% \bibliographystyle{unsrt}
% \bibliography{SEAQT_references.bib}
\hfill \break
\hfill \break
\hfill \break
\hfill \break
\hfill \break
\hfill \break

\printbibliography
\end{document}