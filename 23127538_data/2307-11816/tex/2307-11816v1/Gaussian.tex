\documentclass[aps,notitlepage, floatfix, prb, twocolumn]{revtex4-2}

\usepackage[colorlinks=true, allcolors=blue, bookmarks=false]{hyperref}
\usepackage{amsfonts, amsmath, amssymb}
\usepackage{graphicx}
\usepackage{bbm}

\DeclareMathOperator{\diag}{diag}


\begin{document}
\title{Gaussian state approximation of quantum many-body scars}
\author{Wouter Buijsman}
\email{buijsman@post.bgu.ac.il}

\affiliation{Department of Physics, Ben-Gurion University of the Negev, Beer-Sheva
84105, Israel}
\author{Yevgeny Bar~Lev}
\affiliation{Department of Physics, Ben-Gurion University of the Negev, Beer-Sheva
84105, Israel}
\date{\today}
\begin{abstract}
Quantum many-body scars are atypical, highly nonthermal eigenstates
of kinetically constrained systems embedded in a sea of thermal eigenstates.
These special eigenstates are characterized, for example, by a bipartite
entanglement entropy that scales as most logarithmically with subsystem
size. We use numerical optimization techniques to investigate if quantum
many-body scars of the experimentally relevant PXP model are well
approximated by Gaussian states. These states are described by a number
of parameters that scales quadratically with system size, thereby
having a much lower complexity than generic quantum many-body states.
We find that this is a good description for the quantum many-body
scars away from the center of the spectrum.
\end{abstract}
\maketitle

\section{Introduction}

Typical isolated quantum many-body systems thermalize under their
own internal dynamics \cite{Borgonovi16,DAlessio16,Deutsch18}. Under
time evolution, such systems loose information about their initial
condition, leading to the emergence of statistical mechanics. Recent
times show a keen interest in quantum many-body systems that fall
out of this paradigm. By now, several mechanisms leading to the breakdown
of thermalization have been identified, among them many-body localization
\cite{Basko07,Nandkishore15,Abanin19}, quantum many-body scarring
\cite{Serbyn21,Papic22,Moudgalya22}, and Hilbert space fragmentation
\cite{Sala20,Khemani20}.

Quantum many-body scarring is a form of ergodicity breaking that can
be observed in constrained quantum systems \cite{Bernien17}. In contrast
to many-body localized systems, quantum many-body scarred systems
do not thermalize only when being initialized in certain highly polarized
out-of-equilibrium states \cite{Turner18}. These systems show long-living
approximate periodic revivals to their initial state, which can be
related to a small number of special, highly nonthermal eigenstates
embedded in a sea of thermal eigenstates \cite{Turner18-2}. These
special eigenstates are known as quantum many-body scars, in loose
analogy to the single-body quantum scars first observed by Heller
in 1984 \cite{Heller84}. Quantum many-body scars have attracted tremendous
attention both theoretically \cite{Serbyn21,Papic22,Moudgalya22}
and experimentally \cite{Bernien17,Bluvstein21,Bluvstein22,Zhang23-2}
in recent years. Several  models capturing the phenomenon of quantum
many-body scarring have been introduced, with the so-called PXP model
for a chain of Rydberg atoms being arguably the most paradigmatic
example \cite{Lesanovsky12,Turner18,Turner18-2}.

For a number of models, quantum many-body scarred eigenstates can
be constructed analytically~\cite{Moudgalya22,Chandran23}. For quantum
many-body scars without a known exact form, approximate matrix product
states can be obtained \cite{Lin19,Ho19,Chattopadhyay20,Zhang23}.
Other works used for example mean-field like methods to approximate
quantum many-body scars \cite{Turner21,Omiya23,Hummel23}, and it
has been suggested that they occur due to proximity of the model to
to an integrable point \cite{Khemani19,Surace20}.

Quantum many body scars have an entanglement entropy which grows at
most logarithmically with subsystem size \cite{Turner18,Turner18-2}.
Since ground states of local quadratic Hamiltontians scale similarly
with system size \cite{Calabrese04}, in this work, we consider if
quantum many-body scars can be well approximated by ground states
of non-interacting systems. These states belong to the a family of
Gaussian states (also known as \emph{coherent states}) and are fully
described by a number of parameters that scales quadratically with
the system size, thereby having a much lower degree of complexity
than generic quantum many-body states \cite{Perelomov86}. Gaussian
states have been found to provide an effective description of many-body
states in a broad range of settings, for example in the context of
the mean-field theory of superconductivity \cite{Asano21}, or many-body
localization \cite{Bera15,Bera17,Lezama17,Villalonga18,Buijsman18,Hopjan20,Oritro21,Hopjan21}.

In this work, we numerically optimize the parameters of a general
non-interacting fermionic system towards a maximum overlap of a symmetrized
ground state with a given quantum many-body scar of the PXP model,
which is a paradigmatic toy model of quantum many-body scars. Related
optimization procedure for a complimentary problem have been recently
proposed  \cite{Turner17,Pachos22}.

The outline of this work is as follows. In Section~\ref{sec: model},
we introduce the model and the properties utilized in the remainder
of this work. Section~\ref{sec: Gaussian} outlines the approximation
procedure, and in Section~\ref{sec: results} we present our results.
In Section~\ref{sec: conclusions}, we conclude and outline some
possible directions for future investigations.

\section{Model}

\label{sec: model} We consider the PXP model with open boundary conditions,
\begin{equation}
\hat{H}_{\text{PXP}}=\sum_{i=1}^{L-2}\bigg(\hat{P}_{i}\,\hat{X}_{i+1}\,\hat{P}_{i+2}\bigg)+\hat{X}_{1}\hat{P}_{2}+\hat{P}_{L-1}\hat{X}_{L},\label{eq: H-PXP}
\end{equation}
where $\hat{X}_{i}=\hat{\sigma}_{i}^{x}$ and $\hat{P}_{i}=\tfrac{1}{2}\left(1-\hat{\sigma}_{i}^{z}\right)$
with $\hat{\sigma}_{i}^{x}$ and $\hat{\sigma}_{i}^{z}$ denoting
the Pauli $x$ and $z$ operators acting on site $i$, respectively.
Motivated by experiments for which this model results as an effective
description \cite{Bernien17}, up and down spin states are referred
to as ``ground'' and ``excited'' states, respectively. We consider
the experimentally relevant subspace of the Hilbert space that does
not contain two neighboring sites in the excited state. Due to the
projector terms $\hat{P}_{i}$, this subspace is decoupled from the
rest of the Hilbert space. The Hamiltonian is symmetric with respect
to spatial inversion, governed by the operator $\hat{\pi}$, which
maps site $i$ to site $L-i+1$. The Hamiltonian also anti-commutes
with the parity operator, $\hat{\mathcal{C}}=\prod_{i=1}^{L}\hat{\sigma}_{i}^{z}$,
such that if $\left|\psi_{E}\right\rangle $ is an eigenstate of $\hat{H}_{\text{PXP}}$
with eigenvalue $E$, then $\hat{\mathcal{C}}\left|\psi_{E}\right\rangle $
is an eigenstate with energy $-E$. The spectrum is, therefore, symmetric
around energy zero and contains an exponentially (in system size)
large number of zero-energy eigenstates \cite{Schecter18,Karle21,Buijsman22}.
Since the parity operator has eigenvalues $\pm1$, any eigenstate
of the Hamiltonian can be decomposed as $\left|\psi_{E}\right\rangle =(\hat{P}_{+}+\hat{P}_{-})\left|\psi_{E}\right\rangle $,
where $\hat{P}_{\pm}$ are the projectors on the corresponding subspaces
of $\hat{\mathcal{C}}$. Applying the operator $\mathcal{\hat{C}}$
to this state gives, $\mathcal{\hat{C}}\left|\psi_{E}\right\rangle =(\hat{P}_{+}-\hat{P}_{-})\left|\psi_{E}\right\rangle $,
which as mentioned above, corresponds to an eigenstate of energy $-E$.
Therefore, for $E\neq0$, $\langle \psi_{E}|\hat{\mathcal{C}}|\psi_{E}\rangle =\langle \psi_{E}|\hat{P}_{+}|\psi_{E}\rangle -\langle \psi_{E}|\hat{P}_{-}|\psi_{E}\rangle =0$,
and we see that $\langle \psi_{E}|\hat{P}_{+}|\psi_{E}\rangle =\langle \psi_{E}|\hat{P}_{-}|\psi_{E}\rangle =1/2$.

Quantum many-body scars are eigenstates characterized by an anomalously
high overlap with the $\mathbb{Z}_{2}$-ordered states $|\mathbb{Z}_{2}\rangle=|\bullet\circ\bullet\circ\dots\bullet\circ\rangle$
and $|\mathbb{Z}'_{2}\rangle=|\circ\bullet\circ\bullet\dots\circ\bullet\rangle$,
where $\circ$ and $\bullet$ are pictorial representations of a site
in the ground and excited state, respectively \cite{Turner18,Turner18-2}.
These special eigenstates display a bipartite entanglement entropy
that scales at most logarithmically with subsystem size. The null
space of the Hamiltonian is also known to host a quantum many-body
scar for certain system sizes \cite{Turner18,Lin19,Surace21}. As
motivated below, we do not consider these zero-energy scars in this
work. The quantum-many body scars have almost equal energy separations
of $\Omega\approx1.31$, which only weakly depends of the system size.
This results in the appearance of long-lived periodic revivals to
the initial state starting from a $\mathbb{Z}_{2}$-ordered state.

\section{Gaussian state approximation}

\label{sec: Gaussian} 

The most general quadratic Hamiltonian with $L$ fermionic modes is
given by 
\begin{equation}
\hat{H}=\sum_{i,j=1}^{L}\bigg[A_{ij}\hat{c}_{i}^{\dagger}\hat{c}_{j}+\tfrac{1}{2}\big(B_{ij}\hat{c}_{i}^{\dagger}\hat{c}_{j}^{\dagger}-B_{ij}^{*}\hat{c}_{i}\hat{c}_{j}\big)\bigg],\label{eq: H-quadratic}
\end{equation}
where $A$ is Hermitian and $B$ is antisymmetric and the operators
$\hat{c}_{i}$ and $\hat{c}_{j}^{\dagger}$ obey the standard fermionic
anticommutation relations $\{ \hat{c}_{i},\hat{c}_{j}\} =\{ \hat{c}_{i}^{\dagger},\hat{c}_{j}^{\dagger}\} =0$
and $\{ \hat{c}_{i},\hat{c}_{j}^{\dagger}\} =1$. Fermions
are created and annihilated in pairs, meaning that eigenstates can
have either an even or an odd number of fermions. Hamiltonian \eqref{eq: H-quadratic}
is diagonalized by a Bogoliubov transformation \cite{Bogoliubov47,Lieb61,Perelomov86},
\begin{align}
\hat{d}_{i}= & \sum_{j}\big(U_{ij}\hat{c}_{j}+V_{ij}\hat{c}_{j}^{\dagger}\big)\\
\hat{d}_{i}^{\dagger}= & \sum_{j}\big(V_{ij}^{*}\hat{c}_{j}^{\dagger}+U_{ij}^{*}\hat{c}_{j}\big),
\end{align}
where $U$ and $V$ are required to obey $UV^{T}+VU^{T}=0$ and $UU^{\dagger}+VV^{\dagger}=1$
in order for $\hat{d}_{i}$, $\hat{d}_{j}^{\dagger}$ to obey the
fermionic anti-commutation relations. The eigenstates of Hamiltonian
\eqref{eq: H-quadratic} are thus given by product states in the basis
of the quasi-particles created by $\hat{d}_{i}^{\dagger}$ on top of a quasi-particle
vacuum, $|\psi_{0}\rangle$.

In this work, we focus on the question whether quantum many-body scars
can be well approximated by symmetrized ground state of a non-interacting
Hamiltonian, 
\begin{equation}
|\psi_{\pm}\rangle=\mathcal{N}\left(|\psi_{0}\rangle\pm\hat{\pi}|\psi_{0}\rangle\right).\label{eq: psi-ansatz}
\end{equation}
Here $\mathcal{N}$ is a normalization factor, and $|\psi_{0}\rangle$
is the ground-state of \eqref{eq: H-quadratic}. The states are symmetrized
to follow the inversion symmetry of the scars, which results in a
better approximation (see the Appendix for details). For quantum many-body
scars which are symmetric with respect to inversion we take $|\psi_{+}\rangle$,
and for quantum many-body scars which are antisymmetric we take $|\psi_{-}\rangle$.

We look for matrices $A$ and $B$ characterizing the quadratic Hamiltonian
\eqref{eq: H-quadratic}, which give maximal overlap between $|\psi_{\pm}\rangle$
and a given quantum many-body scar. To compute the overlap, we take
the state $\left|\psi_{\pm}\right\rangle $ in the basis where $\hat{n}_{i}=\hat{c}_{i}^{\dagger}\hat{c}_{i}$
is diagonal, and the quantum many-body scar in the basis where $\hat{\sigma}_{i}^{z}$
is diagonal. The PXP model is time-reversal symmetric, therefore we
lower the computational costs by restricting $A$ and $B$ to be real.
Since the quantum many-body scars have considerable overlap with the
$\mathbb{Z}_{2}$ state, for the initial guess of the matrices $A$
and B we use
\begin{equation}
A=\text{diag}\left(1,1,-1,1,\dots,-1,1\right)\qquad B=0,\label{eq:initial_state}
\end{equation}
such that the initial ground state of \eqref{eq: H-quadratic} is
given by the $\mathbb{Z}_{2}$ state. The number of fermions in this
ground state corresponds to the number of $\left(-1\right)$'s on
the main diagonal of $A.$ Since in the fermionic language the parity
operator is given by $\hat{\mathcal{C}}=\left(-1\right)^{\hat{N}}$,
where $\hat{N}$ is the operator counting the number of fermions,
a state with an even (odd) number of fermions is an eigenstate of
the parity operator with eigenvalue $+1$ $\left(-1\right)$. By changing
the sign of the first element on the main diagonal of the initial
guess for $A$ we can control the evenness of the fermion number and
as such the parity of the ground state. We have found empirically
that the best optimized output is obtained by changing the sign of
the first (or last) diagonal element of $A$, instead of changing
the sign of other diagonal elements of $A$. Taking the diagonal elements
of $A$ as $\pm1$ in a random fashion leads to significantly lower
optimized overlaps.

For the optimization procedure we use the Limited-Memory Broyden-Fletcher-Goldfarb-Shanno
(also known as LM-BFGS) algorithm \cite{Byrd95}, which we terminate
when the gradient of the overlap with respect to the optimization
parameters is equal to zero up to numerical precision. We remark that
it is generically impossible to analytically find the optimal parameters
of a Gaussian state approximating a given many-body state \cite{Zhang16}.

We consider relatively modest system sizes due to the high computational
costs of the optimization procedure. Typically, optimization requires
several thousands, or with outliers, several tens of thousands evaluations
of the overlap. For each such overlap the \emph{many-body} ground-state
of the quadratic system has to be re-calculated. We note that, while
the single-particle states of a quadratic model can be computed in
time polynomial with the system size, the computation of the \emph{many-body}
ground state scales exponentially with the system size.

We note in passing, that the outlined procedure does \emph{not} typically
correspond to the calculation of the natural orbitals from the diagonalization
of the single-particle density matrix. In fact, it is known that a
ground state constructed in the basis of natural orbitals produces
optimal results only for states with two fermions \cite{Zhang16,Aoto20}.

\section{Results}

\label{sec: results} Since the quadratic Hamiltonian \eqref{eq: H-quadratic}
conserves the evenness of the number of fermions, we have to approximate
the projections of the scars onto different parity sectors, $\hat{P}_{\pm}|\psi_{\text{scar}}\rangle$,
separately. We note, that for system sizes dividable by $4$, the
$\mathbb{Z}_{2}$ state lies in the positive parity sector and in
the negative parity sector otherwise. On the other hand, as shown
in Section \ref{sec: model}, all eigenstates of the PXP model (with
nonzero eigenvalue), including the scars, have the same overlap with
both sectors. We denote by $|\psi_{\text{init}}\rangle$ the symmetrized
ground state of Hamiltonian \eqref{eq: H-quadratic} which corresponds
to the initial choice of matrices $A$ and $B$ according to \eqref{eq:initial_state}.
The resulting optimized symmetrized state will be denoted by $|\psi_{\text{opt}}\rangle$.
For convenience, in Fig.~\ref{fig: conv-Z2} we plot $4|\langle \psi_{\text{opt}}|\hat{P}_{\pm}|\psi_{\text{scar}}\rangle |^{2}$,
which is bounded from above by unity, since $\langle \psi_{\text{scar}}|\hat{P}_{\pm}|\psi_{\text{scar}}\rangle =1/2$
(see Section~\ref{sec: model}). We focus only on scars with positive
energies, $E_{\text{scar}}>0$, since the spectrum of the PXP model
is symmetric around zero. We do not consider quantum many-body scars
at zero energy, since scars are not uniquely defined due to numerous
degeneracy at zero energy.

Fig. \ref{fig: conv-Z2} shows the initial $4|\langle \psi_{\text{init}}|\hat{P}_{\pm}|\psi_{\text{scar}}\rangle|^{2}$
and optimized overlaps $4|\langle \psi_{\text{opt}}|\hat{P}_{\pm}|\psi_{\text{scar}}\rangle|^{2}$
for system sizes $L=8$ to $L=14$ as a function of the energies of
the scars, $E_{\text{scar}}$. We observe that the optimized overlap
is close to unity for the two quantum many-body scars closest to the
edge of the spectrum. In fact, the highest-energy scars are the highest
excited eigenstates (or, equivalently, the ground states). We also
note that the optimization leads to a significant improvement of the
overlap for quantum many-body scars away from the center of the spectrum,
as can be seen by comparing to the overlap with the initial guess,
$\left|\psi_{\text{init}}\right\rangle =\left(|\mathbb{Z}_{2}\rangle\pm|\mathbb{Z}'_{2}\rangle\right)/\sqrt{2}$.

% Figure environment removed

Quantum many-body scars distinguish themselves from other types of
non-ergodic many-body states by their anomalously high overlap with
the $|\mathbb{Z}_{2}\rangle$ and $|\mathbb{Z}'_{2}\rangle$ states.
This can be seen at the lower (red) set of lines in Fig.~\ref{fig: conv-Z2},
which shows the overlap, $4|\langle \psi_{\text{init}}|\hat{P}_{+}|\psi_{\text{scar}}\rangle|^{2}$,
where $\left|\psi_{\text{init}}\right\rangle =\left(|\mathbb{Z}_{2}\rangle\pm|\mathbb{Z}'_{2}\rangle\right)/\sqrt{2}$.
In Fig.~\ref{fig: Z2}, we see that also the optimized state has
a qualitatively similar overlap with the $|\mathbb{Z}_{2}\rangle$
and $|\mathbb{Z}'_{2}\rangle$ states, by plotting $\left|\left\langle \psi_{\text{init}}|\psi_{\text{opt}}\right\rangle \right|^{2}$
as a function of the energy of the quantum many-body scars for the
system sizes considered above. It is interesting to note that at the
edges of the spectrum where the approximation of the scars is the
best, the optimized and initial states are almost orthogonal to each
other.

% Figure environment removed

The structure of the optimized matrices $A$ and $B$ could provide
insight on the structure of the quantum many-body scars. Fig.~\ref{fig: matrixAB}
shows color plots of the optimized matrices for the scar with the
second-highest (the highest-energy scar is the ground state) energy
at system size $L=14$. The optimized Hamiltonian is not translationally
invariant (even in the bulk) and exhibits a notion of locality, reflected
in the band-like structure. This observation is presumably related
to the low-entanglement property of the quantum many-body scars.

% Figure environment removed


\section{Conclusions and outlook}

\label{sec: conclusions} We have studied to what extent quantum many-body
scars in the PXP model can be described by inversion symmetrized Gaussian
state, which corresponds to a ground-state of a quadratic Hamiltonian
with no particle number conservation. For this, we numerically optimized
a quadratic fermionic Hamiltonian whose ground-state has a maximal
overlap with the quantum many-body scar under consideration. We found
that quantum many-body scars away from the center of the spectrum
can be well described by states of this form. This holds in particular
for the highest (or equivalently, lowest) energy quantum many-body
scar. We also showed, that the optimal quadratic Hamiltonian is local,
has a non-negligible pairing and is \emph{not} translationally invariant.
In fact, enforcing translation invariance in the optimization procedure,
provides considerably lower overlaps (not shown). Since entanglement
entropy of ground-states of local quadratic Hamiltonians scales logarithmically
with the system size \cite{Calabrese04}, our result suggests that
similar scaling will hold also for quantum many-body scars, at least
not too close to the center of the spectrum.

In this work, we have used a distinct quadratic Hamiltonian for each
quantum many-body scar. In future studies, it would be interesting
to see if a single optimal quadratic Hamiltonian can be used to reasonably
capture the structure of each of the scars, as also to understand
the origin of such an effective single-particle description. It would
be also interesting to investigate further if similar results can
be obtained for other quantum many-body scarred models.

\begin{acknowledgments}
This research was supported by a grant from the United States-Israel
Binational Foundation (BSF, Grant No. $2019644$), Jerusalem, Israel,
and the United States National Science Foundation (NSF, Grant No.
DMR$-1936006$), and by the Israel Science Foundation (grants No.
527/19 and 218/19). WB acknowledges support
from the Kreitman School of Advanced Graduate Studies at Ben-Gurion
University.
\end{acknowledgments}


\appendix*

\section{Optimization results for Gaussian states}

Here, we study the overlap of quantum many-body scars with optimized
Gaussian states, here denoted by $|\psi_{\text{opt},0}\rangle$, instead
of the symmetrized version $|\psi_{\text{opt}}\rangle$ {[}see \eqref{eq: psi-ansatz}{]}.
As this investigation is only for illustrative purposes, we restrict
the analysis to optimization with respect to the parity sector containing
the $\mathbb{Z}_{2}$ state, which is arguably physically the most
interesting. Fig.~\ref{fig: appendix} shows the optimized overlaps
as a function of the energy of the quantum many-body scar for several
system sizes. Comparing the results with those shown in Fig.~\ref{fig: conv-Z2},
we observe a substantially lower overlap, which highlights the importance
of symmetrization.

% Figure environment removed

\bibliography{references}

\end{document}
