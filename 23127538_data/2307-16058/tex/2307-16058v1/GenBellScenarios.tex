%  LaTeX support: latex@mdpi.com 
%  For support, please attach all files needed for compiling as well as the log file, and specify your operating system, LaTeX version, and LaTeX editor.

%=================================================================
\documentclass[preprints,article,accept,pdftex,moreauthors]{Definitions/mdpi} 

% For posting an early version of this manuscript as a preprint, you may use "preprints" as the journal and change "submit" to "accept". The document class line would be, e.g., \documentclass[preprints,article,accept,moreauthors,pdftex]{mdpi}. This is especially recommended for submission to arXiv, where line numbers should be removed before posting. For preprints.org, the editorial staff will make this change immediately prior to posting.

%--------------------
% Class Options:
%--------------------
%----------
% journal
%----------
% Choose between the following MDPI journals:
% acoustics, actuators, addictions, admsci, adolescents, aerospace, agriculture, agriengineering, agronomy, ai, algorithms, allergies, alloys, analytica, animals, antibiotics, antibodies, antioxidants, applbiosci, appliedchem, appliedmath, applmech, applmicrobiol, applnano, applsci, aquacj, architecture, arts, asc, asi, astronomy, atmosphere, atoms, audiolres, automation, axioms, bacteria, batteries, bdcc, behavsci, beverages, biochem, bioengineering, biologics, biology, biomass, biomechanics, biomed, biomedicines, biomedinformatics, biomimetics, biomolecules, biophysica, biosensors, biotech, birds, bloods, blsf, brainsci, breath, buildings, businesses, cancers, carbon, cardiogenetics, catalysts, cells, ceramics, challenges, chemengineering, chemistry, chemosensors, chemproc, children, chips, cimb, civileng, cleantechnol, climate, clinpract, clockssleep, cmd, coasts, coatings, colloids, colorants, commodities, compounds, computation, computers, condensedmatter, conservation, constrmater, cosmetics, covid, crops, cryptography, crystals, csmf, ctn, curroncol, currophthalmol, cyber, dairy, data, dentistry, dermato, dermatopathology, designs, diabetology, diagnostics, dietetics, digital, disabilities, diseases, diversity, dna, drones, dynamics, earth, ebj, ecologies, econometrics, economies, education, ejihpe, electricity, electrochem, electronicmat, electronics, encyclopedia, endocrines, energies, eng, engproc, ent, entomology, entropy, environments, environsciproc, epidemiologia, epigenomes, est, fermentation, fibers, fintech, fire, fishes, fluids, foods, forecasting, forensicsci, forests, foundations, fractalfract, fuels, futureinternet, futureparasites, futurepharmacol, futurephys, futuretransp, galaxies, games, gases, gastroent, gastrointestdisord, gels, genealogy, genes, geographies, geohazards, geomatics, geosciences, geotechnics, geriatrics, hazardousmatters, healthcare, hearts, hemato, heritage, highthroughput, histories, horticulturae, humanities, humans, hydrobiology, hydrogen, hydrology, hygiene, idr, ijerph, ijfs, ijgi, ijms, ijns, ijtm, ijtpp, immuno, informatics, information, infrastructures, inorganics, insects, instruments, inventions, iot, j, jal, jcdd, jcm, jcp, jcs, jdb, jeta, jfb, jfmk, jimaging, jintelligence, jlpea, jmmp, jmp, jmse, jne, jnt, jof, joitmc, jor, journalmedia, jox, jpm, jrfm, jsan, jtaer, jzbg, kidney, kidneydial, knowledge, land, languages, laws, life, liquids, literature, livers, logics, logistics, lubricants, lymphatics, machines, macromol, magnetism, magnetochemistry, make, marinedrugs, materials, materproc, mathematics, mca, measurements, medicina, medicines, medsci, membranes, merits, metabolites, metals, meteorology, methane, metrology, micro, microarrays, microbiolres, micromachines, microorganisms, microplastics, minerals, mining, modelling, molbank, molecules, mps, msf, mti, muscles, nanoenergyadv, nanomanufacturing, nanomaterials, ncrna, network, neuroglia, neurolint, neurosci, nitrogen, notspecified, nri, nursrep, nutraceuticals, nutrients, obesities, oceans, ohbm, onco, oncopathology, optics, oral, organics, organoids, osteology, oxygen, parasites, parasitologia, particles, pathogens, pathophysiology, pediatrrep, pharmaceuticals, pharmaceutics, pharmacoepidemiology, pharmacy, philosophies, photochem, photonics, phycology, physchem, physics, physiologia, plants, plasma, pollutants, polymers, polysaccharides, poultry, powders, preprints, proceedings, processes, prosthesis, proteomes, psf, psych, psychiatryint, psychoactives, publications, quantumrep, quaternary, qubs, radiation, reactions, recycling, regeneration, religions, remotesensing, reports, reprodmed, resources, rheumato, risks, robotics, ruminants, safety, sci, scipharm, seeds, sensors, separations, sexes, signals, sinusitis, skins, smartcities, sna, societies, socsci, software, soilsystems, solar, solids, sports, standards, stats, stresses, surfaces, surgeries, suschem, sustainability, symmetry, synbio, systems, taxonomy, technologies, telecom, test, textiles, thalassrep, thermo, tomography, tourismhosp, toxics, toxins, transplantology, transportation, traumacare, traumas, tropicalmed, universe, urbansci, uro, vaccines, vehicles, venereology, vetsci, vibration, viruses, vision, waste, water, wem, wevj, wind, women, world, youth, zoonoticdis 

%---------
% article
%---------
% The default type of manuscript is "article", but can be replaced by: 
% abstract, addendum, article, book, bookreview, briefreport, casereport, comment, commentary, communication, conferenceproceedings, correction, conferencereport, entry, expressionofconcern, extendedabstract, datadescriptor, editorial, essay, erratum, hypothesis, interestingimage, obituary, opinion, projectreport, reply, retraction, review, perspective, protocol, shortnote, studyprotocol, systematicreview, supfile, technicalnote, viewpoint, guidelines, registeredreport, tutorial
% supfile = supplementary materials

%----------
% submit
%----------
% The class option "submit" will be changed to "accept" by the Editorial Office when the paper is accepted. This will only make changes to the frontpage (e.g., the logo of the journal will get visible), the headings, and the copyright information. Also, line numbering will be removed. Journal info and pagination for accepted papers will also be assigned by the Editorial Office.

%------------------
% moreauthors
%------------------
% If there is only one author the class option oneauthor should be used. Otherwise use the class option moreauthors.

%---------
% pdftex
%---------
% The option pdftex is for use with pdfLaTeX. If eps figures are used, remove the option pdftex and use LaTeX and dvi2pdf.

%=================================================================
% MDPI internal commands
\firstpage{1} 
\makeatletter 
\setcounter{page}{\@firstpage} 
\makeatother
\pubvolume{1}
\issuenum{1}
\articlenumber{0}
\pubyear{2023}
\copyrightyear{2023}
%\externaleditor{Academic Editor: Firstname Lastname} % For journal Automation, please change Academic Editor to "Communicated by"
\datereceived{} 
\dateaccepted{} 
\datepublished{} 
%\datecorrected{} % Corrected papers include a "Corrected: XXX" date in the original paper.
%\dateretracted{} % Corrected papers include a "Retracted: XXX" date in the original paper.
\hreflink{https://doi.org/} % If needed use \linebreak
%\doinum{}
%------------------------------------------------------------------
% The following line should be uncommented if the LaTeX file is uploaded to arXiv.org
\pdfoutput=1

%=================================================================
% Add packages and commands here. The following packages are loaded in our class file: fontenc, inputenc, calc, indentfirst, fancyhdr, graphicx, epstopdf, lastpage, ifthen, lineno, float, amsmath, setspace, enumitem, mathpazo, booktabs, titlesec, etoolbox, tabto, xcolor, soul, multirow, microtype, tikz, totcount, changepage, attrib, upgreek, cleveref, amsthm, hyphenat, natbib, hyperref, footmisc, url, geometry, newfloat, caption


\usepackage{sansmath}
\usepackage{amsmath}
\usepackage{amssymb}
\usepackage{xcolor}

\newcommand{\comment}[1]{}

%Nomenclature and abbreviations
\newcommand{\Local}{\mathcal{L}_{G}} %Local set
\newcommand{\Lnd}{\mathcal{L}_{ND}} %Local set with non-disturbing strategies
\newcommand{\Lnc}{\mathcal{L}_{NC}} %Local set with non-contextual strategies
\newcommand{\NC}{\mathcal{NC}} %Non-Contextual
\newcommand{\NCinA}{\mathcal{NC}_{A}} %Non-contextual for Alice
\newcommand{\NCinB}{\mathcal{NC}_{B}} %Non-contextual for Bob
\newcommand{\NS}{\mathcal{NS}} %Non-Signaling set
\newcommand{\ND}{\mathcal{ND}} %Non-disturbing set
\newcommand{\NDinA}{\mathcal{ND}_{A}} %Non-disturbing in Alice
\newcommand{\NDinB}{\mathcal{ND}_{B}} %Non-disturbing in Bob
\newcommand{\NSND}{\NS \cap \ND } %Non-Signaling and Non-Disturbing
\newcommand{\LND}{\Local \cap \ND } %Local and Non-disturbing
\newcommand{\NCrf}{NC} %Non-contextual response functions
\newcommand{\NDrf}{ND} %Non-disturbing response functions
\newcommand{\Grf}{G} %General response functions.
\newcommand{\QuantumSet}{\mathcal{Q}} %Quantum set.


\newcommand{\carlos}[1]{\textcolor{brown}{[CH: {#1}]}}
\newcommand{\ruffolo}[1]{\textcolor{purple}{[RF: {#1}]}}

%=================================================================
%% Please use the following mathematics environments: Theorem, Lemma, Corollary, Proposition, Characterization, Property, Problem, Example, ExamplesandDefinitions, Hypothesis, Remark, Definition, Notation, Assumption
%% For proofs, please use the proof environment (the amsthm package is loaded by the MDPI class).

%=================================================================
% Full title of the paper (Capitalized)
\Title{Generalized Bell scenarios: disturbing consequences on local-hidden-variable models}

% MDPI internal command: Title for citation in the left column
\TitleCitation{Title}

% % Author Orchid ID: enter ID or remove command
% \newcommand{\orcidauthorA}{0000-0002-2338-8162} % Add \orcidA{} behind the author's name
% \newcommand{\orcidauthorB}{0000-0001-5833-6365} % Add \orcidB{} behind the author's name
% \newcommand{\orcidauthorC}{0000-0001-5007-2991} % Add \orcidB{} behind the author's name
% \newcommand{\orcidauthorD}{0000-0002-3284-1221} % Add \orcidB{} behind the author's name
% \newcommand{\orcidauthorE}{0000-0001-6326-9072} % Add \orcidB{} behind the author's name
% \newcommand{\orcidauthorF}{0000-0002-8990-5329} % Add \orcidB{} behind the author's name


% Authors, for the paper (add full first names)
\Author{André Mazzari $^{1}$, Gabriel Ruffolo $^{1}$, Carlos Vieira $^{2,3}$, Tassius Temistocles $^{4,5}$, Rafael Rabelo $^{1}$, and Marcelo Terra Cunha $^{2}$*}

%\longauthorlist{yes}

% MDPI internal command: Authors, for metadata in PDF
\AuthorNames{André Mazzari, Gabriel Ruffolo, Carlos Vieira, Tassius Temistocles, Marcelo Terra Cunha, and Rafael Rabelo}

% MDPI internal command: Authors, for citation in the left column
\AuthorCitation{Mazzari, A.; Ruffolo, G.; Vieira, C; Temistocles, T.; Terra Cunha, M.; Rabelo, R.}
% If this is a Chicago style journal: Lastname, Firstname, Firstname Lastname, and Firstname Lastname.

% Affiliations / Addresses (Add [1] after \address if there is only one affiliation.)
\address{%
$^{1}$ \quad Instituto de Física Gleb Wataghin, Universidade Estadual de Campinas (Unicamp), 130830-859, Campinas, Brazil \\
$^{2}$ \quad Instituto de Matemática, Estatística e Computação Científica,
Universidade Estadual de Campinas (Unicamp), 130830-859, Campinas, Brazil \\
$^{3}$ \quad Department of Computer Science, The University of Hong Kong, Pokfulam Road, Hong Kong \\
$^{4}$ \quad Departamento de Física, Instituto de Ciências Exatas,
Universidade Federal de Minas Gerais, 30123-970, Belo Horizonte, Brazil \\
$^{5}$ \quad Instituto Federal de Alagoas - Campus Penedo, Rod. Eng. Joaquim Gonçalves - Dom Constantino, 57200-000, Penedo, Brazil
}

% Contact information of the corresponding author
\corres{Correspondence: tcunha@unicamp.br}

% Current address and/or shared authorship
\firstnote{T.T. on leave, visiting Instituto de Matemática, Estatística e Computação Científica and Instituto de Física Gleb Wataghin, Universidade Estadual de Campinas (Unicamp), 130830-859, Campinas, Brazil.} 
\secondnote{M.T.C. in collaboration with Departamento de Física, Instituto de Ciências Exatas, Universidade Federal de Minas Gerais, 30123-970, Belo Horizonte, Brazil.}
% The commands \thirdnote{} till \eighthnote{} are available for further notes

%\simplesumm{} % Simple summary

%\conference{} % An extended version of a conference paper

% Abstract (Do not insert blank lines, \textit{i.e.} \\) 
\abstract{Bell nonlocality and Kochen-Specker contextuality are among the main topics of foundations of quantum theory. Both of them are related to stronger-than-classical correlations, with the former usually referring to spatially separated systems while the latter considering a single system. In recent works, a unified framework for these phenomena was presented. This article reviews, expands and obtains new results regarding this framework. Contextual and disturbing features inside the local models are explored, which allows for the definition of different local sets with a non-trivial relation among them. The relations between the set of quantum correlations and these local sets are also considered, and post-quantum local behaviours are found. Moreover, examples of correlations that are both local and non-contextual but such that these two classical features cannot be expressed by the same hidden variable model are shown. Extensions of the Fine-Abramsky-Brandenburger theorem are also discussed.}

% Keywords
\keyword{Bell Nonlocality; Contextuality; Generalized Bell Scenarios} 

% The fields PACS, MSC, and JEL may be left empty or commented out if not applicable
%\PACS{J0101}
%\MSC{}
%\JEL{}

%%%%%%%%%%%%%%%%%%%%%%%%%%%%%%%%%%%%%%%%%%
% Only for the journal Diversity
%\LSID{\url{http://}}

%%%%%%%%%%%%%%%%%%%%%%%%%%%%%%%%%%%%%%%%%%
% Only for the journal Applied Sciences:
%\featuredapplication{Authors are encouraged to provide a concise description of the specific application or a potential application of the work. This section is not mandatory.}
%%%%%%%%%%%%%%%%%%%%%%%%%%%%%%%%%%%%%%%%%%

%%%%%%%%%%%%%%%%%%%%%%%%%%%%%%%%%%%%%%%%%%
% Only for the journal Data:
%\dataset{DOI number or link to the deposited data set in cases where the data set is published or set to be published separately. If the data set is submitted and will be published as a supplement to this paper in the journal Data, this field will be filled by the editors of the journal. In this case, please make sure to submit the data set as a supplement when entering your manuscript into our manuscript editorial system.}

%\datasetlicense{license under which the data set is made available (CC0, CC-BY, CC-BY-SA, CC-BY-NC, etc.)}

%%%%%%%%%%%%%%%%%%%%%%%%%%%%%%%%%%%%%%%%%%
% Only for the journal Toxins
%\keycontribution{The breakthroughs or highlights of the manuscript. Authors can write one or two sentences to describe the most important part of the paper.}

%%%%%%%%%%%%%%%%%%%%%%%%%%%%%%%%%%%%%%%%%%
% Only for the journal Encyclopedia
%\encyclopediadef{Instead of the abstract}
%\entrylink{The Link to this entry published on the encyclopedia platform.}
%%%%%%%%%%%%%%%%%%%%%%%%%%%%%%%%%%%%%%%%%%
\begin{document}

\section{Introduction}
Quantum theory is arguably the most successful scientific theory ever developed, due to its remarkable predictions in a wide range of phenomena. Nevertheless, many of its features are counter-intuitive and still demand a better understanding. Among these features, two nonclassical phenomena that are particularly interesting are \emph{Bell nonlocality} \cite{BCPSW_14} and \emph{Kochen-Specker (KS) contextuality} \cite{BCGKL_21}.
\par
Both Bell nonlocality and KS contextuality are concerned with the possibility of explaining the results of quantum theory by means of \emph{hidden variable models}, in an effort to retrieve more intuitive descriptions of the observed results. The main difference between such concepts is the fact that the former is related to experiments involving two or more spatially separated systems, while the latter bypasses such demand, at the expense of demanding \textit{a priori} compatibility relations between sets of measurements. In seminal works \cite{Bell_64, KS_67}, it was shown that, for both phenomena, the predictions of quantum mechanics cannot be explained by such classical-like mechanisms, a fact that has been experimentally verified in several works \cite{FC_72, ADR_82, Hensen_etal_15, Giustina_etal_15, Shalm_etal_15,Weinfurter_etal_16,KZG_2009, ARBC_2009, LLS_2011}.
%, and has also been the topic of the Physics Nobel Prize of 2022 \cite{Nobel_2022}.
\par
Despite the similar motivations, the studies on Bell nonlocality and KS contextuality were developed through independent paths. In recent years, however, there were different research programs that proposed unified theoretical frameworks for both phenomena \cite{CSW_14, AFLS_15, RDTCC_14, KCK_2014}. Among such proposals, the one in ref.~\cite{TRC_2019} deals with extensions of Bell scenarios that include the possibility of compatible measurements within each laboratory, allowing the investigation of Bell nonlocality and KS contextuality in well-defined experimental scenarios. In the same work, it was shown that this approach leads to new types of Bell inequalities, which were useful to witness the nonlocality of quantum states that could not be detected by standard Bell tests. Moreover, in ref.~\cite{XRMTCRP_2022}, the same framework was applied to show that Bell nonlocality and KS contextuality could be concomitantly observed in the same experiment, thus proving that there is no monogamic relation between such concepts. %The ideas of this framework were also used to activate nonlocality for some quantum states in the family of two-qubits isotropic states \cite{BHC_2021}.
\par
In this manuscript, we expand the ideas introduced in ref.~\cite{TRC_2019} and obtain several new results related to Bell nonlocality and KS contextuality. In particular, it is shown that this framework allows for the definition of different local sets, depending on the types of local strategies available for each party. The cornerstone of this generalization of Bell scenarios is that each separated party is allowed to measure local contexts; this opens possibilities to explore features related to contextuality and disturbance within the local model. One of our main results shows that there are correlations that are non-disturbing on the observational level but are local if and only if disturbance is allowed inside the local decomposition. 
\par
Moreover, different possibilities for sets of behaviours that are both local and non-contextual are also explored. Another interesting result is that there are correlations with a local decomposition for the joint behaviour and independent non-contextual decompositions for the marginal behaviours of each party, such that it is not possible to find a unified model that encloses these two classical features. Extensions of the Fine-Abramsky-Brandenburger \cite{F_1982, AB_2011} Theorem are also considered, and it is shown that, in the framework used in this work, this is not the most general way to characterize behaviours that are both local and non-contextual. Finally, the set of quantum behaviours is defined, and some local behaviours that do not belong to it are presented.
\par
We organized this work as follows. Section \ref{section:UsualScenarios} reviews the concepts for standard Bell and KS contextuality scenarios; in section \ref{section:ExtendedScenarios}, the concepts involving the framework of Bell scenarios with compatible measurements are reviewed and developed; section \ref{section:Polytopes} discusses the representation of the correlation sets as convex polytopes; the relation of quantum behaviours with some of the sets in this extension of Bell scenarios is explored in section \ref{section:QuantumSet}; finally, section \ref{section:FineTheorem} presents some extensions of the Fine-Abramsky-Brandenburger Theorem. The paper is, then, closed with a discussion in section \ref{section:Discussion}, and technical appendices.

\section{Bell nonlocality and Kochen-Specker contextuality}
\label{section:UsualScenarios}

A Bell scenario consists of an experimental setup where spatially separated laboratories receive physical systems from the same source and perform measurements on them. Each laboratory has a finite set of measurement settings to choose from, and it is assumed that there is no communication between different parties during the measurement processes. This procedure is iterated (in every round, each party receives a new system from the same source), and, after many runs, it is possible to estimate the probabilities for all the possible results of all possible measurements for each party. This work considers bipartite Bell scenarios, referred to as Alice and Bob, but the ideas presented here can be straightforwardly generalized to scenarios with more parties.
\par
In the standard notion of a Bell scenario, Alice and Bob each chooses only one measurement per round. Let $\mathcal{M}_{A}$ and $\mathcal{M}_{B}$ be the sets of possible measurements for Alice and Bob, respectively, and $\{a\}_{A}$ and $\{b\}_{B}$ be the sets of possible results given the measurements $A \in \mathcal{M}_{A}$ and $B \in \mathcal{M}_{B}$. Then, the Bell experiment is described by probabilities of the form:
\begin{equation}
    p(a,b|A,B), \quad \forall \, a,b,A,B.
    \label{StandardBellScenarioProbabilities}
\end{equation}
These probabilities can be organized as entries of a vector $\vec{\boldsymbol{p}} \in \mathbb{R}^{d}$, for an appropriate $d$, called \textit{behaviour vector}. The entries of $\vec{\boldsymbol{p}}$ must be all non-negative,
\begin{equation}
    p(a,b|A,B) \geq 0\text{, }\forall \, a,b,A,B,
    \label{UsualNonNegativity}
\end{equation}
and, given $A$ and $B$, the associate entries must be normalized:
\begin{equation}
    \sum_{a,b} p(a,b|A,B) = 1, \quad \forall \, A,B.
    \label{UsualNormalization}
\end{equation}
The first usual informational demands are the \textit{non-signaling} conditions:
\begin{subequations}
\begin{align}
        p(a|A) &= \sum_{b} p(a,b|A,B), \quad \forall \, a,A,B, \\
        p(b|B) &= \sum_{a} p(a,b|A,B), \quad \forall \, b,A,B.
        \label{UsualNonSignaling}
\end{align}
\end{subequations}
They represent, mathematically, the demand that probabilities that can be inferred by Alice (Bob) alone should not be affected by the measurement choice of Bob (Alice). The set of behaviours that satisfy non-signaling is denoted $\NS$.
\par
The standard notion of \textit{locality} is that a local hidden variable (LHV) model can reproduce the correlations between the laboratories. This is expressed mathematically as:
\begin{equation}
    p(a,b|A,B) = \sum_{\lambda}  p(\lambda) p_{\lambda}(a|A) p_{\lambda}(b|B), \quad \forall a,b,A,B,
    \label{LHVmodel}
\end{equation}
where $p(\lambda)$ are the coefficients of a convex combination, while the families of probability distributions $\{ p_{\lambda}(a|A) \}$ and $\{ p_{\lambda}(b|B) \}$, known as \textit{response functions}, depend only on the latent variable $\lambda$ and Alice and Bob measurements, respectively. If the components of the behaviour vector $\vec{\boldsymbol{p}}$ can be decomposed as equation \eqref{LHVmodel}, then it is said to be \textit{local}; otherwise, it is \textit{nonlocal}. The set of local behaviours is denoted $\mathcal{L}$. All local behaviours are also non-signaling, so $\mathcal{L} \subset \NS$.
\par
A phenomenon which is related to Bell nonlocality is the so-called Kochen-Specker contextuality. Here, suppose there is a single laboratory, run by Bob, that receives physical systems and performs measurements on them. Assume, though, that in each round of the experiment, Bob is able to perform \textit{compatible measurements} concomitantly. Each set of compatible measurements is referred to as a \textit{context}, and the set of maximal\footnote{A context $B_{m}$ is maximal if it not a subset of any context.} contexts is denoted $\mathcal{C}$. Let $\boldsymbol{\mathsf{B}}= \left( B_{1},..., B_{t} \right) \in \mathcal{C}$ be a context of Bob and $\boldsymbol{\mathsf{b}} =\left(b_{1},...,b_{t}\right)$ a set of results, where $b_j \in \{b\}_{B_{j}}$ is a possible result of measurement $B_{j}$. The probabilities that describe the experiment are of the form:
\begin{equation}
    p(\boldsymbol{\mathsf{b}}| \boldsymbol{\mathsf{B}}), \quad \forall \, \boldsymbol{\mathsf{b}}, \boldsymbol{\mathsf{B}}.
\end{equation}
These probabilities can also be organized in a behaviour vector $\vec{\boldsymbol{p}} \in \mathbb{R}^{d}$. The non-negativity conditions,
\begin{align}
    p(\boldsymbol{\mathsf{b}}| \boldsymbol{\mathsf{B}}) & \geq 0, \quad \forall \, \boldsymbol{\mathsf{b}}, \boldsymbol{\mathsf{B}},
    \label{ContextualityNonNegativity}
    \end{align}
and normalization conditions,
\begin{align}
    \sum_{\boldsymbol{\mathsf{b}}} p(\boldsymbol{\mathsf{b}}| \boldsymbol{\mathsf{B}}) & = 1, \quad \forall \,  \boldsymbol{\mathsf{B}},
    \label{ContextualityNormalization}
\end{align}
must be satisfied. Furthermore, it is usual to demand the \textit{non-disturbing} condition, which is analogous to non-signaling. Given two contexts $\boldsymbol{\mathsf{B}}_{1}$ and $\boldsymbol{\mathsf{B}}_{2}$ of Bob such that their intersection $\boldsymbol{\mathsf{B}}^{\prime} = \boldsymbol{\mathsf{B}}_{1} \cap \boldsymbol{\mathsf{B}}_{2}$ is non-empty, the following must hold:
\begin{equation}
p(\boldsymbol{\mathsf{b}}^{\prime}|\boldsymbol{\mathsf{B}}^{\prime}) = \sum_{\boldsymbol{\mathsf{b}}_{1} \setminus
\boldsymbol{\mathsf{b}}^{\prime}} p(\boldsymbol{\mathsf{b}}_{1}|\boldsymbol{\mathsf{B}}_{1}) = \sum_{\boldsymbol{\mathsf{b}}_{2} \setminus
\boldsymbol{\mathsf{b}}^{\prime}} p(\boldsymbol{\mathsf{b}}_{2}|\boldsymbol{\mathsf{B}}_{2}),
    \label{NonDistrubingCondition}
\end{equation}
where the summations are over all possible results for the measurements in the sets $\boldsymbol{\mathsf{B}}_{1} \setminus
\boldsymbol{\mathsf{B}}^{\prime}$ and $\boldsymbol{\mathsf{B}}_{2} \setminus \boldsymbol{\mathsf{B}}^{\prime}$, respectively. In other words, the marginal probability distributions are well-defined for all non-maximal contexts. The set of non-disturbing behaviours is denoted $\ND$. If a behaviour does not satisfy the non-disturbing conditions, it is said to be \textit{disturbing}.
\par
The behaviour is said to be \textit{non-contextual} if it can be decomposed by means of a non-contextual hidden variable (NCHV) model:
\begin{equation}
p(\boldsymbol{\mathsf{b}}|\boldsymbol{\mathsf{B}}) = \sum_{\mu} p(\mu) \prod_{j}  p_{\mu}(b_{j}|B_{j}), \quad \forall \, \boldsymbol{\mathsf{b}},\boldsymbol{\mathsf{B}},
    \label{NCHVmodel}
\end{equation}
where $p(\mu)$ is a probability distribution over the possibles values of $\mu$ and the product is over all $j$ such that $B_{j} \in \boldsymbol{\mathsf{B}}$. The probability distributions in the set $\{ p_{\mu}(b_{j}|B_{j}) \}$ are also known as response functions. If such decomposition does not exist, then the behaviour is said to be \textit{contextual}\footnote{It is worth mentioning that we use the term `contextual' as shorthand for `not non-contextual'.}. The set of non-contextual behaviours is denoted $\NC$. Non-contextual behaviours are non-disturbing, and, thus $\NC \subset \ND$.

\section{Bell scenarios with compatible measurements}
\label{section:ExtendedScenarios}

In Bell scenarios, it is usually considered that each separated party performs only one measurement per round. However, a relaxation of this assumption is also possible and desirable. In ref.~\cite{TRC_2019}, it is introduced an extension of the standard Bell scenario where local compatible measurements for each party are admitted, \textit{i.e.}, Alice and/or Bob may measure a context. Including local compatible relations allows for the study of both Bell nonlocality and KS contextuality in a unified framework.
\par
Let $\mathcal{M}_{A}$ and $\mathcal{M}_{B}$ be the sets of possible measurements for Alice and Bob, respectively, and $\mathcal{C}_{A}$ and $\mathcal{C}_{B}$ be their sets of maximal contexts. Given that Alice measures a context $\boldsymbol{\mathsf{A}} = \left( A_{1},..., A_{s} \right) \in \mathcal{C}_{A}$ and Bob measures a context $\boldsymbol{\mathsf{B}} = \left( B_{1},..., B_{t} \right) \in \mathcal{C}_{B}$, and that $\boldsymbol{\mathsf{a}} =\left(a_{1},..., a_{s}\right)$ and $\boldsymbol{\mathsf{b}} =\left(b_{1},..., b_{t}\right)$ are sets of possible results (one for each measurement), the extended scenario is described by a set of joint conditional probabilities:
\begin{equation}
    p(\boldsymbol{\mathsf{a}},\boldsymbol{\mathsf{b}}|\boldsymbol{\mathsf{A}},\boldsymbol{\mathsf{B}}),\quad \forall \, \boldsymbol{\mathsf{a}},\boldsymbol{\mathsf{b}},\boldsymbol{\mathsf{A}},\boldsymbol{\mathsf{B}}.
    \label{ExtendedCorrelations}
\end{equation}
As before, these probabilities can be organized as entries of a behaviour vector $\vec{\boldsymbol{p}}$, and must satisfy the non-negativity and normalization conditions:
%\begin{subequations}
\begin{align}
    p(\boldsymbol{\mathsf{a}},\boldsymbol{\mathsf{b}}|\boldsymbol{\mathsf{A}},\boldsymbol{\mathsf{B}}) & \geq 0\text{, } \forall \, \boldsymbol{\mathsf{a}},\boldsymbol{\mathsf{b}},\boldsymbol{\mathsf{A}},\boldsymbol{\mathsf{B}}, \\
    \sum_{\boldsymbol{\mathsf{a}},\boldsymbol{\mathsf{b}}} p(\boldsymbol{\mathsf{a}},\boldsymbol{\mathsf{b}}|\boldsymbol{\mathsf{A}},\boldsymbol{\mathsf{B}}) & = 1\text{, } \forall \, \boldsymbol{\mathsf{A}},\boldsymbol{\mathsf{B}}.
\end{align}
%\end{subequations}
These conditions describe the most general correlations in an extended Bell experiment.
\par
In the following sections, the term \textit{extended Bell scenarios} is used to refer to Bell experiments with compatible measurements, while the usual notion of Bell experiments is referred to by \textit{standard Bell scenarios}. The latter is a particular case of the former, since the incompatible measurements for Alice and Bob can also be seen as maximal contexts of size one. Thus, the conditions presented in section \ref{section:UsualScenarios} are particular cases of what is discussed in the next sections: while standard Bell scenarios will correspond to maximal contexts of size 1, KS scenarios will correspond to one-party extended Bell scenarios.

\subsection{Non-signaling and non-disturbing conditions}
In many works, either the term \textit{non-signaling} or \textit{non-disturbing} is used to mean that the marginal probabilities for the results of one measurement are independent of the choice of other measurements that are performed concomitantly. The former is usually used for spatially separated measurements, while the latter usually refers to measurements in the same laboratory. In this work, such terms are employed with the intention to differentiate between these two situations.
\par
The non-signaling conditions, which state that the probabilities for the results of a context $\boldsymbol{\mathsf{A}}$ of Alice do not depend on the context $\boldsymbol{\mathsf{B}}$ chosen by Bob, and vice versa, are naturally extended:
\begin{subequations}
\begin{align}
p(\boldsymbol{\mathsf{a}}|\boldsymbol{\mathsf{A}}) & =  \sum_{\boldsymbol{\mathsf{b}}} p(\boldsymbol{\mathsf{a}},\boldsymbol{\mathsf{b}}|\boldsymbol{\mathsf{A}},\boldsymbol{\mathsf{B}}), \quad  \forall \, \boldsymbol{\mathsf{a}},\boldsymbol{\mathsf{A}},\boldsymbol{\mathsf{B}}, \\
    p(\boldsymbol{\mathsf{b}}|\boldsymbol{\mathsf{B}}) & = \sum_{\boldsymbol{\mathsf{a}}} p(\boldsymbol{\mathsf{a}},\boldsymbol{\mathsf{b}}|\boldsymbol{\mathsf{A}},\boldsymbol{\mathsf{B}}), \quad  \forall \, \boldsymbol{\mathsf{b}},\boldsymbol{\mathsf{A}},\boldsymbol{\mathsf{B}}.
\end{align}
\label{ExtendedNonSignaling}
\end{subequations}
Behaviours that satisfy \eqref{ExtendedNonSignaling} belong to the Non-Signaling set, denoted $\NS$.
\par
Independently of the non-signaling conditions, the non-disturbing conditions for Bob states that for all pairs of contexts $\boldsymbol{\mathsf{B}}_{1}$ and $\boldsymbol{\mathsf{B}}_{2}$ of Bob with non-empty intersection and a subcontext $\boldsymbol{\mathsf{B}}^{\prime}$ such that $\boldsymbol{\mathsf{B}}^{\prime} \subset \boldsymbol{\mathsf{B}}_{1}$ and $\boldsymbol{\mathsf{B}}^{\prime} \subset \boldsymbol{\mathsf{B}}_{2}$, the following conditions must hold:
\begin{align}
    p(\boldsymbol{\mathsf{b}}^{\prime}|\boldsymbol{\mathsf{A}},\boldsymbol{\mathsf{B}}^{\prime}) = \sum_{\boldsymbol{\mathsf{b}}_{1} \setminus
\boldsymbol{\mathsf{b}}^{\prime}} \sum_{\boldsymbol{\mathsf{a}}} p(\boldsymbol{\mathsf{a}},\boldsymbol{\mathsf{b}}_{1}|\boldsymbol{\mathsf{A}},\boldsymbol{\mathsf{B}}_{1})
    = \sum_{\boldsymbol{\mathsf{b}}_{2} \setminus
\boldsymbol{\mathsf{b}}^{\prime}} \sum_{\boldsymbol{\mathsf{a}}} p(\boldsymbol{\mathsf{a}},\boldsymbol{\mathsf{b}}_{2}|\boldsymbol{\mathsf{A}},\boldsymbol{\mathsf{B}}_{2}), \quad \forall \, \boldsymbol{\mathsf{b}}^{\prime}, \boldsymbol{\mathsf{A}},\boldsymbol{\mathsf{B}}^{\prime},
    \label{MostGeneralExtendedNonDistrubingCondition}
\end{align}
where the $\mathbf{b}$ summations of each line are over all possible results of the measurements in $\boldsymbol{\mathsf{B}}_{1} \setminus
\boldsymbol{\mathsf{B}}^{\prime}$ and $\boldsymbol{\mathsf{B}}_{2} \setminus \boldsymbol{\mathsf{B}}^{\prime}$, respectively. That is, the probabilities of the results for $\boldsymbol{\mathsf{B}}^{\prime}$ are not dependent on the context being measured. The behaviours that satisfy \eqref{MostGeneralExtendedNonDistrubingCondition} belong to the Non-Disturbing set for Bob, denoted $\NDinB$. The analogous conditions can be valid for Alice's measurements, and then the behaviour belongs to the Non-Disturbing set for Alice, denoted $\NDinA$. If the non-disturbing conditions are satisfied for both parties, the behaviour belongs to the Non-Disturbing set, denoted $\ND$. Note that $\ND = \NDinA \cap \NDinB$. Behaviours that do not satisfy the non-disturbance conditions are called \textit{disturbing}.
\par
As defined here, the non-signaling and non-disturbing conditions are independent. For instance, there exists the possibility of a behaviour that is non-signaling but presents disturbance, and \textit{vice versa}. If both conditions are satisfied, the behaviour belongs to the Non-Signaling and Non-Disturbing set, denoted $\NSND$. Then, the non-disturbing conditions can be written as:
\begin{align}
    p(\boldsymbol{\mathsf{b}}^{\prime}|\boldsymbol{\mathsf{B}}^{\prime}) &= \sum_{\boldsymbol{\mathsf{b}}_{1} \setminus
\boldsymbol{\mathsf{b}}^{\prime}} p(\boldsymbol{\mathsf{b}}_{1}|\boldsymbol{\mathsf{B}}_{1}) = \sum_{\boldsymbol{\mathsf{b}}_{2} \setminus
\boldsymbol{\mathsf{b}}^{\prime}} p(\boldsymbol{\mathsf{b}}_{2}|\boldsymbol{\mathsf{B}}_{2}), \quad \forall \, \boldsymbol{\mathsf{b}}^{\prime}, \boldsymbol{\mathsf{B}}^{\prime},
    \label{ExtendedNonDistrubingCondition}
\end{align}
with analogous conditions for Alice.
In this case, all the marginals $p(a_{i}|A_{i})$ and $p(b_{j}|B_{j})$ are well-defined. Hence, the set $\NSND$ coincides with the non-disturbing set when we look at generalized Bell scenarios as a special case of contextuality scenarios. It is also possible to combine the non-signaling conditions with non-disturbance only for Alice or Bob, resulting in the sets $\NS \cap \NDinA$ and $\NS \cap \NDinB$, respectively.
\par
From now on, it is assumed that the behaviours considered in this article belong to the $\NSND$ set.

\subsection{Contextuality}

The extension of Bell scenarios including the possibility of locally compatible measurements is suited to the study of questions related to contextuality. For this, it is necessary to consider the marginal probability distributions $p(\boldsymbol{\mathsf{a}}|\boldsymbol{\mathsf{A}})$ and $p(\boldsymbol{\mathsf{b}}|\boldsymbol{\mathsf{B}})$. Written in this way, it is implicit that the non-signaling conditions \eqref{ExtendedNonSignaling} hold.
\par
The behaviour $\vec{\boldsymbol{p}}$ is \textit{non-contextual} for Alice if and only if there exists a decomposition in the form:
\begin{equation}
    p(\boldsymbol{\mathsf{a}}|\boldsymbol{\mathsf{A}}) = \sum_{\mu} p(\mu) \prod  p_{\mu}(a_{i}|A_{i}), \quad \forall \, \boldsymbol{\mathsf{a}},\boldsymbol{\mathsf{A}},
    \label{NCHVmodelForAlice}
\end{equation}
where $p(\mu)$ is a probability distribution over $\mu$, and $\{p_{\mu}(a_{i}|A_{i}) \}$ is a set of response functions. If $\vec{\boldsymbol{p}}$ does not satisfy the condition expressed in \eqref{NCHVmodelForAlice}, then it is said to be \textit{contextual} for Alice. The set of behaviours that are non-contextual for Alice is denoted $\NCinA$. In the same manner, a behaviour $\vec{\boldsymbol{p}}$ is non-contextual for Bob if there is a non-contextual decomposition for Bob's marginal behaviour $p(\boldsymbol{\mathsf{b}}|\boldsymbol{\mathsf{B}})$. The set of behaviours that are non-contextual for Bob is denoted $\NCinB$.
\par
Finally, a behaviour $\vec{\boldsymbol{p}}$ is non-contextual if and only if it is non-contextual both for Alice and Bob. Otherwise, $\vec{\boldsymbol{p}}$ is contextual. The set of non-contextual behaviours is denoted $\NC$. Note that $\NC=\NCinA \cap \NCinB$.
\par
The non-disturbance conditions are satisfied by every non-contextual behaviour. In extended Bell scenarios, it is necessary to specify which of the parties are non-contextual. Hence, $\NCinA$ is a subset of $\NDinA$; $ \NCinB$ is a subset of $\NDinB$; and $\NC$ is a subset of $\ND$. Moreover, since the non-signaling conditions are assumed in the definitions of all non-contextual models, the three non-contextual sets are also subsets of $\NS$.

\subsection{The many meanings of locality in extended scenarios}
\label{Section:LocalSets}

%The central concept in the field of Bell nonlocality is the (non-)reproduction of the observed correlations by a local model. 
The notion of \textit{locality} in extended Bell scenarios is a natural extension of the standard locality condition (eq.~\eqref{LHVmodel}). However, it brings many new features, as is shown throughout this manuscript. A behaviour $\vec{\boldsymbol{p}}$ is said to be \textit{local} if, and only if, there exists a decomposition of the form:
\begin{equation}
    p(\boldsymbol{\mathsf{a}},\boldsymbol{\mathsf{b}}|\boldsymbol{\mathsf{A}},\boldsymbol{\mathsf{B}}) = \sum_{\lambda} p(\lambda) p_{\lambda}(\boldsymbol{\mathsf{a}}|\boldsymbol{\mathsf{A}}) p_{\lambda}(\boldsymbol{\mathsf{b}}|\boldsymbol{\mathsf{B}}),
    \label{ExtendedLHVmodel}
\end{equation}
where $p(\lambda)$ is a probability distribution over $\lambda$, and $\{ p_{\lambda}(\boldsymbol{\mathsf{a}}|\boldsymbol{\mathsf{A}}) \}$ and $\{ p_{\lambda}(\boldsymbol{\mathsf{b}}|\boldsymbol{\mathsf{B}}) \}$ are sets of response functions for Alice and Bob, respectively.
\par
For the locality condition in \eqref{ExtendedLHVmodel} to be completely determined, it is necessary to specify the sets of response functions. The property of \textit{locality} must be concerned with the possibility of a decomposition that classically correlates the results of spatially separated laboratories, \textit{i.e.}, it is about finding a decomposition for the behaviour as a convex combination of probability distributions that are independent between Alice and Bob. Thus, it must be possible to have local behaviours that present contextuality in the individual laboratories. To consider this situation, it is necessary to include contextual response functions in the local decomposition. Indeed, the fact that each laboratory may measure contexts opens the possibility to explore attributes related to contextuality and disturbance of the response functions in the local model. This brings new features to extended Bell scenarios that are not present in standard ones.
\par
To get the most general definition of the local model, the only restriction that must be imposed on the sets $\{p_{\lambda}(\boldsymbol{\mathsf{a}}|\boldsymbol{\mathsf{A}}) \}$ and $\{ p_{\lambda}(\boldsymbol{\mathsf{b}}|\boldsymbol{\mathsf{B}}) \}$ is that they are composed of well-defined probability distributions -- in other words, they must satisfy the normalization and non-negativity conditions. This would take into account the possibility of contextual response functions. More than that, it would include both disturbing and non-disturbing response functions, \textit{i.e.}, they would not be constrained to satisfy the non-disturbing conditions \eqref{NonDistrubingCondition}. The behaviours that satisfy this locality condition, using \textit{general response functions}, belong to the Local set $\Local$.
\par
As discussed above, the definition of the set $\Local$ includes both disturbing and non-disturbing response functions, and hence it has both disturbing and non-disturbing behaviours. If the non-disturbance conditions\footnote{Note that the non-disturbing conditions \eqref{ExtendedNonDistrubingCondition} were considered instead of \eqref{MostGeneralExtendedNonDistrubingCondition} because local behaviours satisfy the non-signaling conditions \eqref{ExtendedNonSignaling}.} \eqref{ExtendedNonDistrubingCondition} are imposed both for Alice and Bob on the behaviours of $\Local$, the resulting set is the Local and Non-Disturbing set $\LND$. That is, it is ensured that, beyond having a local decomposition, the marginal behaviours of both parties are non-disturbing. It is important to emphasize that, for this case of the set $\LND$, disturbing response functions in the r.h.s of expression \eqref{ExtendedLHVmodel} are allowed, but in a way that the probability distribution $p(\lambda)$ makes the observed probabilities, on the l.h.s of equation \eqref{ExtendedLHVmodel}, non-disturbing. It is also possible to consider non-disturbance for Alice or Bob only, resulting in the sets $\Local \cap \NDinA$ and $\Local \cap \NDinB$, respectively.
\par
Another possibility for the definition of the locality condition \eqref{ExtendedLHVmodel} is to constraint the response functions $\{p_{\lambda}(\boldsymbol{\mathsf{a}}|\boldsymbol{\mathsf{A}}) \}$ and $\{ p_{\lambda}(\boldsymbol{\mathsf{b}}|\boldsymbol{\mathsf{B}}) \}$ to be non-disturbing. That would define another local set, denoted by $\Lnd$, which considers only \textit{non-disturbing response functions}. As a convex combination of non-disturbing behaviours is also non-disturbing, it follows that the behaviours in this set automatically satisfy the non-disturbance conditions \eqref{ExtendedNonDistrubingCondition} for both Alice and Bob, and, hence, $\Lnd \cap \ND=\Lnd$. This was the local set considered in the original work of extended Bell scenarios \cite{TRC_2019}.
\par
Finally, the last possibility is to allow only \textit{non-contextual response functions} in the local model. That is, for a context $\boldsymbol{\mathsf{A}} = \left( A_{1},..., A_{s} \right)$ and results $\boldsymbol{\mathsf{a}} = \left( a_{1},..., a_{s} \right)$, the response functions would be of the form\footnote{In a more precise manner, the non-contextual response functions should be of the form $p_{\lambda}(\boldsymbol{\mathsf{a}}|\boldsymbol{\mathsf{A}}) = \sum_{\mu} p_{\lambda}(\mu) \prod_{i=1}^{s} p_{\lambda, \mu}(a_{i}|A_{i})$. But this is just a convex combination of the response functions \eqref{NonContextualResponseFunctions}, which is included in the decomposition \eqref{LncModel}.}
\begin{equation}
    p_{\lambda}(\boldsymbol{\mathsf{a}}|\boldsymbol{\mathsf{A}}) = \prod_{i=1}^{s} p_{\lambda}(a_{i}|A_{i}),
    \label{NonContextualResponseFunctions}
\end{equation}
where $\{ p_{\lambda}(a_{i}|A_{i}) \}$ are response functions for the measurement $A_{i}$. The same is valid for the response functions of Bob. The local decomposition would, then, be:
\begin{equation}  p(\boldsymbol{\mathsf{a}},\boldsymbol{\mathsf{b}}|\boldsymbol{\mathsf{A}},\boldsymbol{\mathsf{B}}) = \sum_{\lambda} p(\lambda) \left[ \prod_{i=1}^{s} p_{\lambda}(a_{i}|A_{i})\right] \left[ \prod_{i=1}^{s} p_{\lambda}(b_{j}|B_{j})\right], \quad \forall \,  \boldsymbol{\mathsf{a}},\boldsymbol{\mathsf{b}},\boldsymbol{\mathsf{A}},\boldsymbol{\mathsf{B}}.
\label{LncModel}
\end{equation}
By a marginalization process, it is possible to see that the marginal behaviours $p(\boldsymbol{\mathsf{a}}|\boldsymbol{\mathsf{A}})$ and $p(\boldsymbol{\mathsf{b}}|\boldsymbol{\mathsf{B}})$ would also have a non-contextual decomposition (see eq. \eqref{NCHVmodel}). Hence, with this choice of response functions, the local behaviours are also non-contextual. The local set defined by using only non-contextual response functions is denoted $\Lnc$. %This set coincides with the non-contextual set $\NC$ when viewing generalized Bell scenarios as contextual scenarios. 
\par
Regarding the relations between these sets, it follows from the definitions that $\Lnc$ is a subset of $\Lnd$ and that the latter is a subset of $\Local$. Moreover, the two former sets are subsets of $\LND$, since they are also subsets of $\ND$.
\par
In general, the set $\Lnc$ is a proper subset of $\Lnd$. In scenarios where it is possible to have contextuality, there will be contextual response functions in the set of non-disturbing response functions, so the sets $\Lnc$ and $\Lnd$ will be different. For analogous reasons, in general, $\Lnd$ is a proper subset of $\Local$, since the latter may include disturbing response functions. An interesting question is whether the sets $\Lnd$ and $\LND$ are equivalent or not. In other words, is it equivalent to consider general response functions and then impose the non-disturbance conditions or to start with non-disturbing response functions from the beginning? In section \ref{section:ExtendedPolytopes}, it is shown that, for some scenarios, these two local sets are different, \textit{i.e.}, $\Lnd$ is a proper subset of $\LND$. Hence, these two definitions are not equivalent.
\par
 In standard Bell scenarios, where the contexts of Alice and Bob have only one measurement, these local sets coincide, \textit{i.e.}, $\Lnc \equiv \Lnd \equiv \Local \equiv \mathcal{L}$.
\par
Other local sets can be defined by choosing the different possibilities of response functions for Alice and Bob. Following the notation already introduced, let the index \textit{\NCrf} refer to non-contextual response functions; subscript \textit{\NDrf} refers to non-disturbing response functions; and \textit{\Grf} refers to general response functions, including both disturbing and non-disturbing ones. Then, we can also define the following sets: $\mathcal{L}_{\NCrf,\NDrf}$, $\mathcal{L}_{\NDrf,\NCrf}$, $\mathcal{L}_{\NCrf,\Grf}$, $\mathcal{L}_{\Grf,\NCrf}$, $\mathcal{L}_{\NDrf,\Grf}$, and $\mathcal{L}_{\Grf,\NDrf}$. In the notation $\mathcal{L}_{I,J}$, the indices $I$ and $J$ refer to the allowed response functions for Alice and Bob, respectively. The local sets defined previously can also be expressed with this notation: $\Lnc \equiv \mathcal{L}_{\NCrf,\NCrf}$, $\Lnd \equiv \mathcal{L}_{\NDrf,\NDrf}$, and $\Local \equiv \mathcal{L}_{\Grf,\Grf}$.
\par
Note that for all local sets defined, the resultant local behaviours are non-signaling, and hence they are all subsets of $\NS$. Also, these sets could be combined with the different conditions of non-disturbance, \textit{i.e.}, they can also be combined with the sets $\NDinA$ and $\NDinB$.
\par
In this section, several local sets were defined by exploring the different features that the response functions may possess. In section \ref{section:Polytopes}, the structure of these sets as convex polytopes and the minimum sufficient set of response functions for each case are discussed.

\subsection{Locality and non-contextuality together}
\label{Section:LocalityNonContextuality}

In the previous subsections, the conditions for a behaviour to be {local} or {non-contextual} were presented separately. In this subsection, the case in which both of these properties are present is considered. For this, many combinations of the previously defined sets are possible.
\par
First, consider the most general local set $\Local$ and non-contextuality at Alice's laboratory. A behaviour $\vec{\boldsymbol{p}}$ is local and non-contextual for Alice if and only if it has a local decomposition as in expression \eqref{ExtendedLHVmodel}, including general response functions, and a non-contextual decomposition for Alice's marginal behaviour as in equation \eqref{NCHVmodelForAlice}. In other words, the behaviour must belong to the set $\Local \cap \NCinA$. Similarly, to talk about locality and non-contextuality at Bob's laboratory, the relevant set is $\Local \cap \NCinB$. Considering non-contextuality for both Alice and Bob, a behaviour $\vec{\boldsymbol{p}}$ is local and non-contextual if and only if it belongs to the set $\Local \cap \NC$. Analogous definitions are valid considering the sets $\Lnd$ and $\Lnc$.
\par
From the hierarchy of the local sets discussed in the previous section, it follows that $\Lnc \cap \NC$ is a subset of $\Lnd \cap \NC$, and that these two are subsets of $\Local \cap \NC$. Whether these sets are equivalent is a question that has interesting consequences. To see this, note that the three sets are defined by imposing separately the local and non-contextual conditions. Nevertheless, behaviours belonging to $\Lnc$ are automatically non-contextual, and thus $\Lnc \cap \NC = \Lnc$. This means that for this last case, the conditions of locality and non-contextuality may be expressed by a unique hidden variable model given by expression \eqref{LncModel}. In the case that there are behaviours belonging to $\Lnd \cap \NC$ or $\Local \cap \NC$, but not to $\Lnc$, it means that these behaviours independently have local and non-contextual decompositions, nevertheless it is not possible to express both of them with a single model with these features. This would show that the joint consideration of these two classical concepts presents a richer structure than just having a joint global hidden variable decomposition. The question of the equivalence between these sets is answered in section \ref{section:ExtendedPolytopes}.

\section{Geometrical characterization of the correlation sets}
\label{section:Polytopes}
One modern way to investigate Bell nonlocality and KS contextuality is by means of the structure of convex polytopes \cite{BCPSW_14, P_1989}. A polytope is a geometrical object that belongs to a real vector space $\mathbb{R}^{d}$, and admits a dual characterization: it can either be described by the intersection of a finite number of half-spaces -- each given by a linear inequality, related to one of its \emph{facets} --, or by the convex hull of a finite set of vectors, representing their \emph{vertices}. With one of the descriptions, it is, in principle, possible to obtain the other. In practice, this problem requires an exponential amount of computational resources as the complexity of the polytope grows. In this work, the software PANDA \cite{LR_2015} was used to find the facets or the vertices of the polytopes for the specific scenarios considered.
\par
As previously mentioned, the probabilities that describe a scenario are organized as entries of a behaviour vector $\vec{\boldsymbol{p}} \in \mathbb{R}^{d}$. The different sets presented in sections \ref{section:UsualScenarios} and \ref{section:ExtendedScenarios} are defined by linear equations or linear inequalities in the entries of $\vec{\boldsymbol{p}}$, or by expressing it as a convex combination of a set of response functions, which can also be organized as vectors $\vec{\boldsymbol{p}}_{\lambda}$. Moreover, as is shown in section \ref{section:ExtremeResponseFunctions}, it is sufficient to consider finite sets of response functions.  Hence, all the sets defined are convex polytopes.
\par
In the following subsections, the sets of extremal vertices for each of the local sets are discussed and the geometrical structure of convex polytopes is applied to both standard and extended Bell scenarios. By considering the facet inequalities for the polytopes of some specific extended scenarios, it was possible to prove the main results of this article.

\subsection{Digression on sets of extreme response functions}
\label{section:ExtremeResponseFunctions}
In previous sections, the features characterizing the sets of response functions for each local set were presented. In this subsection, the minimum set of response functions that are sufficient to define these sets are discussed. These are known as \textit{extremal response functions}, since they cannot be expressed as convex combinations of the others.
\par
An important result is that an arbitrary probability distribution can always be written as a convex combination of deterministic probability distributions, \textit{i.e.}, probability distributions in which one of the results has probability 1. Hence, considering the local model \eqref{LHVmodel} for standard Bell scenarios, the response functions $\{ p_{\lambda}(a|A) \}$ and $\{ p_{\lambda}(b|B) \}$ can always be expressed in terms of \textit{deterministic response functions}. These are defined as follows: for a fixed $\lambda$, the results for the measurements $A_{i}$ and $B_{j}$ are $R_\lambda(A_{i})$ and $R_\lambda(B_{j})$, respectively. Hence, the extremal response functions are given by
\begin{subequations}
\begin{align}
        &p_{\lambda}(a_{i}|A_{i}) = \delta_{a_{i},R_\lambda(A_{i})},  \quad \forall \, a_{i},A_{i},
        \\
        &p_{\lambda}(b_{j}|B_{j}) = \delta_{b_{j},R_\lambda(B_{j})}, \quad \forall \, b_{j},B_{j}.
\end{align}
\end{subequations}
\par
For the set $\Lnc$, the response functions are non-contextual. Every non-contextual behaviour can be expressed as a convex combination of vertices that have deterministic results for each of the measurements. Following the notation introduced above, let $R_\lambda(\boldsymbol{\mathsf{A}}) = \left( R_\lambda(A_{1}),..., R_\lambda(A_{s}) \right)$ and $R_\lambda(\boldsymbol{\mathsf{B}})= \left( R_\lambda(B_{1}),..., R_\lambda(B_{t}) \right)$ be tuples of determined results for the contexts $\boldsymbol{\mathsf{A}} = \left( A_{1},..., A_{s} \right)$ and $\boldsymbol{\mathsf{B}} = \left( B_{1},..., B_{t} \right)$, respectively, associated with the hidden variable $\lambda$. Then, the \textit{deterministic and non-contextual response functions} are expressed as:
\begin{subequations}
\begin{align}
        &p_{\lambda}(\boldsymbol{\mathsf{a}}|\boldsymbol{\mathsf{A}}) =  \delta_{\boldsymbol{\mathsf{a}}, R_\lambda(\boldsymbol{\mathsf{A}})} = \prod_{i=1}^{s} \delta_{a_{i},R_\lambda(A_{i})},\quad \forall \, \boldsymbol{\mathsf{a}}, \boldsymbol{\mathsf{A}},
        \\
        &p_{\lambda}(\boldsymbol{\mathsf{b}}|\boldsymbol{\mathsf{B}}) = \delta_{\boldsymbol{\mathsf{b}}, R_\lambda(\boldsymbol{\mathsf{B}})} = \prod_{j=1}^{t} \delta_{b_{j},R_\lambda(B_{j})}, \quad \forall \, \boldsymbol{\mathsf{b}}, \boldsymbol{\mathsf{B}}.
\end{align}
\end{subequations}
These compose the set of extremal response functions for $\Lnc$.
\par
To find the set of extremal response functions of $\Lnd$, it is necessary to characterize the extremal points of the Non-Disturbing polytopes for the marginal behaviours of Alice and Bob. As before, any non-disturbing response function can be written as a convex combination of these extremal non-disturbing points. Thus, they form the set of extremal response functions for $\Lnd$. 
\par
Finally, for the set $\Local$, the only condition imposed on the response functions is that they are composed of probability distributions. As in the case of standard Bell scenarios, this means that they can be expressed in terms of deterministic vertices. However, in this case, the vertices are deterministic for each context. That is, given the context, the results of each measurement are determined, but a measurement that belongs to more than one context may have different results in each of them. This last case characterizes disturbing vertices. Note also that the set of extremal response functions of $\Lnc$ is included in the respective set for $\Local$.
\par
As can be seen in the discussion above, while in standard scenarios the term \textit{deterministic response function} has only one meaning, it can be ambiguous in extended scenarios: it is possible to consider deterministic and non-contextual results for each measurement or deterministic results for each context.
\par
These sets of extremal response functions are used to find the facet inequalities for the polytopes of some specific scenarios in the next subsections.


\subsection{Polytopes for standard scenarios}
The characterization of standard Bell scenarios by means of convex polytopes has been known for a long time. The conditions of non-negativity, normalization, and non-signaling, presented in equations \eqref{UsualNonNegativity}, \eqref{UsualNormalization}, and \eqref{UsualNonSignaling}, respectively, define a convex polytope structure for the set of behaviours that satisfy them. This is called the \textit{Non-Signaling} polytope, $\NS$. The local behaviours are convex combinations of a set of vectors as defined by \eqref{LHVmodel}. Moreover, as seen in subsection \ref{section:ExtremeResponseFunctions}, it is sufficient to consider only the deterministic response functions. This also means that local behaviours are convex combinations of a finite set of vectors, which defines another convex polytope, denominated as the \textit{Local} polytope, $\mathcal{L}$. The facet inequalities of this polytope are associated to the \textit{Bell inequalities}. In figure \ref{StandardScenariosHierarchy}, there is a representation of the hierarchy of these sets for standard Bell scenarios.
\par
An analogous reasoning is valid for the standard scenarios of contextuality. The conditions of non-negativity, normalization, and non-disturbance, given by \eqref{ContextualityNonNegativity}, \eqref{ContextualityNormalization}, and \eqref{NonDistrubingCondition}, respectively, define the Non-Disturbing polytope, $\ND$. The non-contextual behaviours have a non-contextual decomposition as in \eqref{NCHVmodel}, with the set of extremal response functions being composed of the deterministic and non-contextual ones. Hence, they are convex combinations of a finite set of vectors, which defines the Non-Contextual polytope, $\NC$. The facet inequalities of this polytope are associated to the Bell-like inequalities for non-contextuality.

% Figure environment removed


\subsection{Polytopes for extended scenarios}
\label{section:ExtendedPolytopes}
In extended Bell scenarios, the structure of convex polytopes is also present, but with a richer structure of sets. This subsection contains our main results: Theorems \ref{TheoremLocalSets} and \ref{TheoremLocalNonContextualSets}.
\par
Due to the same reasons presented before, the Non-Signaling set $\NS$ and the Non-Disturbance sets $\ND$, $\NDinA$, and $\NDinB$ are convex polytopes. The same holds for the combination of these conditions, \textit{i.e.}, $\NSND$, $\NS \cap \NDinA$, and $\NS \cap \NDinB$ are also convex polytopes\footnote{In general, the intersection of two polytopes is also a polytope.}.
\par
One of the interests in this geometrical characterization lies in the local sets. The different sets of the extremal response functions for $\Lnc$, $\Lnd$, and $\Local$ were discussed in subsection \ref{section:ExtremeResponseFunctions}. For each of them, the local behaviours are convex combinations of a finite set of vertices, and hence they are all convex polytopes. From the sets of extremal vertices it is possible, in principle, to find the set of facet inequalities for each of the local polytopes, which, in general, are different.
\par
For the Local and Non-Disturbing polytope $\LND$, the set of extremal vertices is not known \textit{a priori}. To find them, it is first necessary to characterize the facets of $\Local$. Then, including the non-disturbance conditions of the set $\ND$, the description of the $\LND$ set in terms of linear equations and linear inequalities is complete, and it is possible to make the reverse process and find its set of vertices. This procedure was used in the proof of theorem \ref{TheoremLocalSets}.
\par
As discussed in subsection \ref{Section:LocalSets}, the different local sets are related in the following manner:
\begin{equation}
    \Lnc \subset \Lnd \subset \LND.
\end{equation}
It was already stated that these sets are equivalent in standard Bell scenarios and that there are extended scenarios in which $\Lnc \neq \Lnd$. In the following Theorem, it is shown that $\Lnd$ is different from $\LND$ for some scenarios.
\begin{Theorem}
\label{TheoremLocalSets}
There are scenarios in which the set $\Lnd$ is a proper subset of $\LND$.
\end{Theorem}
\begin{proof}
\par
Consider a scenario where Alice has two incompatible measurements $\mathcal{M}_{A} = \{ A_0, A_1 \}$ and Bob has three measurements $\mathcal{M}_{B} = \{B_0, B_1, B_2\}$ with the contexts $\mathcal{C}_{B} = \{ \left(B_0, B_1\right), \left(B_1, B_2\right) \}$. The possible results for all measurements are $\{0,1\}$. This is the simplest scenario where the two sets of interest may be different, since the sets $\Lnd$ and $\Local$ are equal in any standard Bell scenario. In the scenario considered here, the sets $\Lnc$ and $\Lnd$ coincide, since all the non-disturbing response functions are also non-contextual. Following the discussion above, it is possible to find the vertices and facets of the polytopes $\Lnd$ and $\LND$ \cite{github}.
\par
In table \ref{TableLNDBehaviour}, it is presented a behaviour that is non-signaling and non-disturbing and that has a local decomposition using disturbing local boxes (see appendix \ref{AppendixLocalSets}), \textit{i.e}, the behaviour belongs to $\LND$.  However, if only non-disturbing local boxes are considered, this behaviour does not have a local decomposition, since it violates inequality \eqref{2VLndInequality}, which is associated to a facet of $\Lnd$. Hence, it does not belong to this set, which implies that, for this scenario, $\Lnd \neq \LND$.
\end{proof}
\par
The previous result implies that there are non-signaling and non-disturbing behaviours that are local if and only if disturbing local boxes are considered in the local model. This result shows that the property of non-disturbance of the boxes, which is a property related to the strategies that each individual laboratory applies locally, has implications for the global property of locality.
\par
The behaviours that are considered in the previous Theorem, which are the ones that belong to $\LND$ but not to $\Lnd$, also belong to the set $\NSND$. Hence, they can be expressed as a convex combination of non-signaling and non-disturbing vertices. However, this will necessarily include nonlocal vertices. To find a convex combination of these behaviours that is also local, it is necessary to include disturbing vertices that do not belong to $\NSND$, although they belong to $\NS$.
\par
In Figure \ref{Figure:LocalSets}, a schematic representation of the hierarchy of the correlation sets for the scenario considered in the proof of theorem \ref{TheoremLocalSets} is presented. 
%The grey polygon (2,3,4) represents the set $\Lnd$, which in this case is equal to $\Lnc$; the set $\LND$ is given by the polygon (1,2,3,4), which contains the set $\Lnd$ but is greater than it; lastly, the $\NSND$ set is represented by the polygon (1,2,3,4,5), which contains the other two. Vertex 1 represents the behaviour in table \ref{TableLNDBehaviour}, which is a vertex of both the $\LND$ and $\NSND$ polytopes. The common vertices of the polytopes $\Lnd$, $\LND$, and $\NSND$ are the deterministic and non-contextual vertices, which in Figure \ref{Figure:LocalSets} are the vertices of the polygon (2,3,4). The dashed red line is the frontier of the quantum set, which will be discussed in section \ref{section:QuantumSet}.

% Figure environment removed
\par
The local and non-contextual sets $\Lnc\text{, } \Lnd \cap \NC \text{, and } \Local \cap \NC$ are also convex polytopes. To obtain their vertices, it is first necessary to have the facet inequalities of the local polytopes and of the non-contextual polytope $\NC$. Then, by joining the facet inequalities and the linear constraints that define each set, the vertices of the local and non-contextual polytope may be acquired.
\par
It was stated in section \ref{Section:LocalityNonContextuality} that the behaviour sets satisfy the following relation:
\begin{equation}
    \Lnc \subset \Lnd \cap \NC \subset \Local \cap \NC.
\end{equation}
 By considering the structure of convex polytopes for a specific scenario, it is possible to prove the following Theorem.
\begin{Theorem}
\label{TheoremLocalNonContextualSets}
In general, the local and non-contextual sets are related in the following way:
\begin{equation}
    \Lnc \subsetneq \Lnd \cap \NC \subsetneq \Local \cap \NC.
    \label{LocalNonContextualSetsRelation}
\end{equation}
\end{Theorem}
\begin{proof}
Consider a scenario where Alice has two incompatible measurements $\mathcal{M}_{A} = \{ A_0, A_1 \}$ and Bob has three measurements $\mathcal{M}_{B} = \{B_0, B_1, B_2\}$ with the contexts $\mathcal{C}_{B} = \{ \left(B_0, B_1\right),$ $\left(B_1, B_2\right), \left(B_2,B_0\right) \}$.
The possible results for all measurements are $\{0,1\}$. The facet inequalities of the local sets for this scenario were obtained via facet enumeration techniques \cite{github}. Table \ref{TableLndNCbehaviour} presents a behaviour that belongs to the set $\Lnd \cap \NC$ but does not belong $\Lnc$, and table \ref{TableLNCBehaviour} shows a behaviour which belongs to $\Local \cap \NC$ but does not belong to $\Lnd \cap \NC$ (See Appendix \ref{AppendixLocalNonContextualSets} for more details). Thus, for this scenario, these sets are related as in expression \eqref{LocalNonContextualSetsRelation}.
\end{proof}
\par
As discussed in section \ref{Section:LocalityNonContextuality}, the behaviours provided in the proof of Theorem \ref{TheoremLocalNonContextualSets} have a local decomposition between the results of Alice and Bob and independent non-contextual models for the marginal behaviours of each laboratory; however, it is not possible to find a single hidden variable decomposition that encloses these two properties. A schematic representation of the relation proven in Theorem \ref{TheoremLocalNonContextualSets} is given in Fig.~\ref{fig:Lnc-LndNC}.
%\par
 %The grey polygon (1,2,3) represents the $\Lnc$ set, and hence the vertices (1,2,3) are local and non-contextual. The polygon (1,2,3,4), with blue dots, represents the set $\Lnd$, and the vertex 4 is local and contextual. The polygon (1,2,3,5), with red diagonal lines, represents the set $\NC$, and vertex 5 is nonlocal and non-contextual. Finally, the polygon (1,2,3,4,5,6) represents the $\NSND$ set, and vertex 6 is nonlocal and contextual. With this figure, it is possible to see that the set $\Lnd \cap \NC$, which is represented by the polygon (1,2,3,7) with blue dots and red diagonal lines, is greater than the $\Lnc$ set.
\par
In the next section, the set of quantum behaviours in extended Bell scenarios is defined. Its relation with the local sets is explored, which also brings some unexpected results.

% Figure environment removed

\section{Quantum correlations}
\label{section:QuantumSet}
Up to this point, no physical theory has been assumed for the behaviours considered. Another important correlation set is the quantum set $\QuantumSet$, \textit{i.e.}, the set of all behaviours that can be reproduced by quantum theory. In extended Bell scenarios with sets of measurements $\mathcal{M}_{A}$ and $\mathcal{M}_{B}$ and contexts $\mathcal{C}_{A}$ and $\mathcal{C}_{B}$, a behaviour $\vec{\boldsymbol{p}}$ belongs to $\QuantumSet$ if there exist Hilbert spaces $\mathcal{H}_{A}$ and $\mathcal{H}_{B}$, a density operator $\rho$ acting over $\mathcal{H}_{A} \otimes \mathcal{H}_{B}$, and projective measurements $\{X_{a|A}\}$ and $\{Y_{b|B}\}$ for each $A \in \mathcal{M}_{A}$ and $B \in \mathcal{M}_{B}$, with the commutation relations between the quantum measurements reproducing the scenario's contexts\footnote{That is, $A_{i}, A_{j} \in \mathcal{C}_{A}$ if and only if $\left[X_{a_{i}|A_{i}}, X_{a_{j}|A_{j}} \right] = 0$ for all results $a_{i},a_{j}$. The same is valid for Bob's projective measurements.}, such that:
\begin{equation}
    p(\boldsymbol{\mathsf{a}},\boldsymbol{\mathsf{b}}|\boldsymbol{\mathsf{A}},\boldsymbol{\mathsf{B}}) = Tr\left[\rho \left( \prod_{i=1}^{s} X_{a_{i}|A_{i}} \right) \otimes \left( \prod_{j=1}^{t} Y_{b_{j}|B_{j}}\right)  \right],
    \label{QuantumBehaviour}
\end{equation}
for all contexts $\boldsymbol{\mathsf{A}} = \left(A_{1},...,A_{s}\right) \in \mathcal{C}_{A}$ and $\boldsymbol{\mathsf{B}} = \left(B_{1},...,B_{t}\right) \in \mathcal{C}_{B}$ and results $\boldsymbol{\mathsf{a}} = \left(a_{1},...,a_{s}\right)$ and $\boldsymbol{\mathsf{b}} = \left(b_{1},...,b_{t}\right)$.
\par
It is well known that quantum behaviours are non-signaling and non-disturbing when considering commuting projective measurements. Hence, if a behaviour $\vec{\boldsymbol{p}}$ is quantum, then it also belongs to the set $\NSND$.
\par
In standard Bell scenarios, the quantum set contains the local set. As is shown below, this is not necessarily the case for extended Bell scenarios.
\par
In the view of generalized Bell scenarios as particular cases of contextual scenarios, the set $\Lnc$ is equivalent to the set of non-contextual behaviours. Moreover, it is well known that all non-contextual behaviours are also quantum. Hence, it follows that $\Lnc \subset \QuantumSet$.
\par
However, this relation does not follow for the set $\Lnd$, \textit{i.e.}, considering this set, there are local behaviours that are not quantum. This is due to the possibility of post-quantum contextuality in individual laboratories.
\begin{Theorem}
    There are scenarios in which $\Lnd \not\subset \QuantumSet$.
\end{Theorem}
\begin{proof}
    Consider a scenario where Alice and Bob have measurements $\mathcal{M}_{A} = \{ A_{0}, A_{1} \}$ and $\mathcal{M}_{B} = \{ B_0, B_1, B_2, B_3 \}$, respectively. The contexts are given by $\mathcal{C}_{A} = \{ \left( A_{0} \right), \left( A_1 \right) \}$ and $\mathcal{C}_{B} = \{ \left( B_0, B_1 \right), \left( B_1, B_2 \right), \left( B_2, B_3 \right), \left( B_3, B_0 \right) \}$.
    \par
    On Bob's side, the situation is analogous to a standard bipartite Bell scenario, in which each party has two dichotomic measurements. Indeed, the non-contextual inequality, in this case, is the CHSH inequality \cite{AQBCC_2013}. Thus, considering Bob's non-disturbing polytope, there are contextual vertices equivalent to PR boxes, which are known to be post-quantum. Let $p_{PR}(\boldsymbol{\mathsf{b}}|\boldsymbol{\mathsf{B}})$ be a behaviour representing one of these boxes.
    \par
    Consider the following joint behaviour:
    \begin{equation}
        p(a,\boldsymbol{\mathsf{b}}|A,\boldsymbol{\mathsf{B}}) = p(a|A)p_{PR}(\boldsymbol{\mathsf{b}}|\boldsymbol{\mathsf{B}}),
        \label{LndBehaviourNotQuantum}
    \end{equation}
    where $\{p(a|A)\}$ is an arbitrary behaviour for Alice. The expression \eqref{LndBehaviourNotQuantum} already gives a local decomposition with non-disturbing response functions, so this behaviour belongs to $\Lnd$. However, due to Bob's post-quantum marginal behaviour, it cannot be reproduced by quantum theory. Thus, $\Lnd$ is not a subset of $\QuantumSet$.
\end{proof}
\par
The Bell inequalities of the $\Lnd$ polytope are useful for semi-device-independent\footnote{The compatibility relations necessary for the quantum contexts are not device-independent assumptions.} certification of entanglement, as shown in the theorem below.
\begin{Theorem}
If a quantum state $\rho$ is separable, then all behaviours generated from it belong to the $\Lnd$ polytope.
\end{Theorem}
\begin{proof}
Since $\rho$ is separable, there are quantum states $\{\rho_{A}^{\lambda}\}$ acting over $\mathcal{H}_A$ and $\{\rho_{B}^{\lambda}\}$ acting over $\mathcal{H}_B$ and a probability distribution $p(\lambda)$ such that:
\begin{equation}
    \rho = \sum_{\lambda} p(\lambda) \rho_{A}^{\lambda} \otimes \rho_{B}^{\lambda}\nonumber.
\end{equation}
Thus, a behaviour $\vec{\boldsymbol{p}}$ obtained from $\rho$ has probabilities of the form:
\begin{align}
    p(\boldsymbol{\mathsf{a}},\boldsymbol{\mathsf{b}}|\boldsymbol{\mathsf{A}},\boldsymbol{\mathsf{B}}) &= Tr\left[\rho \left( \prod_{i=1}^{s} X_{a_{i}|A_{i}} \right) \otimes \left( \prod_{j=1}^{t} Y_{b_{j}|B_{j}}\right)  \right]\nonumber\\
    &=\sum_{\lambda} p(\lambda) Tr\left[(\rho_{A}^{\lambda} \otimes \rho_{B}^{\lambda}) \left( \prod_{i=1}^{s} X_{a_{i}|A_{i}} \right) \otimes \left( \prod_{j=1}^{t} Y_{b_{j}|B_{j}}\right)  \right]\nonumber\\
    &= \sum_{\lambda} p(\lambda) Tr\left[\rho_{A}^{\lambda}\prod_{i=1}^{s} X_{a_{i}|A_{i}}  \right] Tr\left[\rho_{B}^{\lambda} \prod_{j=1}^{t} Y_{b_{j}|B_{j}} \right]\nonumber\\
    &= \sum_{\lambda} p(\lambda)p_{\lambda}(\boldsymbol{\mathsf{a}}|\boldsymbol{\mathsf{A}})p_{\lambda}(\boldsymbol{\mathsf{b}}|\boldsymbol{\mathsf{B}}),
\end{align}
where $p_{\lambda}(\boldsymbol{\mathsf{a}}|\boldsymbol{\mathsf{A}}) =  Tr\left[\rho_{A}^{\lambda}\prod_{i=1}^{s} X_{a_{i}|A_{i}}  \right]$ and $p_{\lambda}(\boldsymbol{\mathsf{b}}|\boldsymbol{\mathsf{B}}) =  Tr\left[\rho_{B}^{\lambda} \prod_{j=1}^{t} Y_{b_{j}|B_{j}} \right]$. In this way, $\vec{\boldsymbol{p}}$ is local (see eq. \eqref{ExtendedLHVmodel}). On the other hand, as quantum theory always generates non-disturbing behaviours, it follows that $p_{\lambda}(\boldsymbol{\mathsf{a}}|\boldsymbol{\mathsf{A}})$ and $p_{\lambda}(\boldsymbol{\mathsf{b}}|\boldsymbol{\mathsf{B}})$ are non-disturbing behaviours for every $\lambda$. Therefore, $\vec{\boldsymbol{p}}$ belongs to $\Lnd$. 
\end{proof}
\par
Since quantum behaviours in extended Bell scenarios are automatically non-disturbing, the most general local models that should be considered to test if a quantum behaviour is local or not are the ones related to the set $\LND$.
\par
As expected, there are quantum behaviours that cannot be reproduced by local models, even considering general response functions. Inequality \eqref{2VLNDinequality} is a Bell-like inequality for the $\LND$ polytope for the scenario of Theorem \ref{TheoremLocalSets}, which can be violated by quantum correlations \cite{github}:
\begin{align}
    &-p(1,0,1|A_0,B_0,B_1) -p(1,1,0|A_0,B_0,B_1) -p(1,1,1|A_0,B_0,B_1)
    \\
    &+p(1,0,1|A_0,B_1,B_2) +p(1,1,0|A_0,B_1,B_2) +p(0,0,1|A_1,B_0,B_1)
    \\
    &+p(0,1,0|A_1,B_0,B_1) +p(0,1,1|A_1,B_0,B_1) +p(1,0,0|A_1,B_1,B_2)
    \\
    &+p(1,1,1|A_1,B_1,B_2) \leq 1.
\label{2VLNDinequality}
\end{align}
However, there are also behaviours in $\LND$ that are not quantum behaviours.
\begin{Theorem}
    There are scenarios in which $\LND \subsetneq \QuantumSet$.
\end{Theorem}
\begin{proof}
    Considering the same scenario from the proof of Theorem \ref{TheoremLocalSets}, it was verified, using the NPA hierarchy \cite{NPA_2008}, that the vertex of the $\LND$ set presented in table \ref{TableLNDBehaviour} is not a quantum behaviour \cite{github}.
\end{proof}
\par
 This is another example of local behaviour that is not quantum. Note, however, that this post-quantum behaviour does not belong to $\Lnd$. Moreover, for this specific scenario, neither Alice nor Bob may present contextuality\footnote{So, for this scenario, it is also valid that $\mathcal{L}_{G} \cap \NC \subsetneq \QuantumSet$.}. Hence, the fact that quantum mechanics cannot reproduce this local behaviour is not due to post-quantum contextuality.
 \par
To conclude this section, the relation of the quantum set $\QuantumSet$ and the local set $\LND$ is that neither one is contained in the other. This is depicted in Figure \ref{Figure:LocalSets}, with the boundary of the quantum set represented by the dashed red line. For the specific scenario of this figure, the set $\Lnd$ is contained inside the quantum set, since it is equal to the set $\Lnc$.

\section{Extensions of Fine's theorem}
\label{section:FineTheorem}
Fine's Theorem \cite{F_1982} is a very important result on Bell nonlocality, providing an alternative description of local behaviours. Consider a standard Bell scenario, with $\mathcal{M}_{A} = \{A_{1},...,A_{n}\}$ and $\mathcal{M}_{B} = \{B_{1},...,B_{m}\}$ being the sets of measurements of Alice and Bob, respectively, and $\{a_{i}\}_{A_{i}}$ and $\{b_{j}\}_{B_{j}}$ the sets of possible results for each measurement. The scenario is described by probabilities in the form of \eqref{StandardBellScenarioProbabilities}. Then, a behaviour is local if and only if there exists a joint probability distribution,
\begin{equation}
    \omega\left(a_{1},...,a_{n},b_{1},...,b_{m} \right),
    \label{FineStandardJointProbability}
\end{equation}
for the results of all observables involved in the scenario, such that it is possible to obtain the scenario's probabilities from a marginalization process
\begin{equation}
    p(a_{i}^{\prime},b_{y}^{\prime}|A_{i},B_{j}) = \sum \delta_{a_{i}^{\prime},a_{i}} \delta_{b_{j}^{\prime} , b_{j}} \omega\left(a_{1},...,a_{n},b_{1},...,b_{m} \right),
    \label{FineStandardMarginalization}
\end{equation}
where the summation is over the possible results of all measurements. In ref.~\cite{AB_2011}, this theorem was extended to contextuality scenarios, and became known as Fine-Abramsky-Brandenburger Theorem.
\par
Consider now an extended Bell scenario with the maximal contexts of Alice and Bob given by  $\mathcal{C}_{A} = \{\boldsymbol{\mathsf{A}}_{1},...,\boldsymbol{\mathsf{A}}_{p}\}$ and $\mathcal{C}_{B} = \{\boldsymbol{\mathsf{B}}_{1},...,\boldsymbol{\mathsf{B}}_{q}\}$, respectively. Denote by $\boldsymbol{\mathsf{a}}_{k}$ a set of possible results for the measurements of context $\boldsymbol{\mathsf{A}}_{k}$, and by $\boldsymbol{\mathsf{b}}_{l}$ the analogous set of results for Bob. A natural question is whether Fine's theorem generalizes to extended Bell scenarios. For the set $\Lnc$, the Theorem extends immediately.
\begin{Theorem}
A behaviour $\vec{\boldsymbol{p}}$ belongs to $\Lnc$ if and only if there exists a joint probability distribution for the results of all measurements, as in \eqref{FineStandardJointProbability}, which satisfies the condition 
\begin{equation}
    p(\boldsymbol{\mathsf{a}}^{\prime},\boldsymbol{\mathsf{b}}^{\prime}|\boldsymbol{\mathsf{A}},\boldsymbol{\mathsf{B}}) = \sum \delta_{\boldsymbol{\mathsf{a}}^{\prime},\boldsymbol{\mathsf{a}}}\delta_{\boldsymbol{\mathsf{b}}^{\prime},\boldsymbol{\mathsf{b}}} \omega\left(a_{1},...,a_{n},b_{1},...,b_{m} \right), \quad \forall \, \boldsymbol{\mathsf{a}}^{\prime},\boldsymbol{\mathsf{b}}^{\prime},\boldsymbol{\mathsf{A}},\boldsymbol{\mathsf{B}}
    \label{ExtendedFineMarginalizationMeasurements}
\end{equation}
where the summation is over the possible results of all measurements, and the Kronecker deltas involve only the results of the measurements in the contexts $\boldsymbol{\mathsf{A}}$ and $\boldsymbol{\mathsf{B}}$.
\end{Theorem}
\begin{proof}
Suppose that $\vec{\boldsymbol{p}}$ belongs to $\Lnc$. Then, for contexts $\boldsymbol{\mathsf{A}} = \left( A_{1},..., A_{s} \right)$ and $\boldsymbol{\mathsf{B}} = \left( B_{1},..., B_{t} \right)$, it has a local decomposition of the form
\begin{align}
    p(\boldsymbol{\mathsf{a}},\boldsymbol{\mathsf{b}}|\boldsymbol{\mathsf{A}},\boldsymbol{\mathsf{B}}) &= \sum_{\lambda} p(\lambda) p_{\lambda}(\boldsymbol{\mathsf{a}}|\boldsymbol{\mathsf{A}}) p_{\lambda}(\boldsymbol{\mathsf{b}}|\boldsymbol{\mathsf{B}}) \nonumber
    \\
    &= \sum_{\lambda} p(\lambda) \left( \prod_{i=1}^{s} p_{\lambda}(a_{i}|A_{i}) \right) \left( \prod_{j=1}^{t} p_{\lambda}(b_{j}|B_{j}) \right).
\end{align}
with the non-contextual response functions being deterministic
\begin{subequations}
\begin{align}
    p_{\lambda}(a_{i}|A_{i}) = \delta_{R_\lambda(A_{i}), a_{i}},
    \\
    p_{\lambda}(b_{j}|B_{j}) = \delta_{R_\lambda(B_{j}), b_{j}},
\end{align}
\end{subequations}
where $R_\lambda(A_{i})$ is the deterministic result defined by $\lambda$ for $A_{i}$, and analogously for $R_\lambda(B_{j})$. Using these non-contextual response functions, it is possible to construct the following joint probability distribution, which satisfies condition \eqref{ExtendedFineMarginalizationMeasurements}:

\begin{align}
    \omega\left(a_{1},...,a_{n},b_{1},...,b_{m} \right) 
    = \sum_{\lambda} p(\lambda) \left( \prod_{i} p_{\lambda}(a_{i}|A_{i}) \right) \left( \prod_{j} p_{\lambda}(b_{j}|B_{j}) \right),
\end{align}
where the indices $i$ and $j$ in the products run over all measurements in the scenario.
\par
Suppose now that there exists a probability distribution as in \eqref{FineStandardJointProbability} which satisfies eq.~\eqref{ExtendedFineMarginalizationMeasurements}. It is possible to associate each tuple of results for all measurements with a value $\lambda$ of the deterministic and non-contextual response functions:
\begin{equation}
    \left(a_{1},...,a_{n},b_{1},...,b_{m}\right) \xrightarrow{} \lambda.
    \label{FineStandardAssignment}
\end{equation}
The assignment \eqref{FineStandardAssignment} gives the deterministic result predicted by $\lambda$, \textit{i.e.}, $\lambda(A_{i}) = a_{i}$ and $\lambda(B_{j}) = b_{j}$. Then, using eq.~\eqref{FineStandardJointProbability} as a probability distribution over the values of $\lambda$ and viewing the Kronecker deltas as response functions, the condition \eqref{ExtendedFineMarginalizationMeasurements} can be seen as a local decomposition using only non-contextual response functions. Hence, $\vec{\boldsymbol{p}} \in \Lnc$.
\end{proof}
Thus, in the framework of extended Bell scenarios, the Fine-Abramsky-Branderburger Theorem characterizes the behaviours belonging to $\Lnc$. But, as shown in Theorem \ref{TheoremLocalNonContextualSets}, there are local and non-contextual behaviours that are not in this set. These behaviours demand one hidden variable model for the results of Alice and Bob, and then, considering the marginal behaviour of each party, there are other independent hidden variable models ensuring non-contextuality in each lab. Thus, the Fine-Abramsky-Brandenburger Theorem does not characterize all the behaviours that are both local and non-contextual.
\par
It is also possible to extend Fine's theorem to the set $\Local$. This is done by considering each context of the scenario as an independent measurement. In what follows, $\boldsymbol{\mathsf{a}}_{k}$ and $\boldsymbol{\mathsf{b}}_{l}$ are tuples of possible results for the contexts $\boldsymbol{\mathsf{A}}_{k}$ and $\boldsymbol{\mathsf{B}}_{l}$, respectively.
\begin{Theorem}
 A behaviour $\vec{\boldsymbol{p}}$ belongs to $\Local$ if and only if there exists a joint probability distribution for the results of all contexts
\begin{equation}
    \omega(\boldsymbol{\mathsf{a}}_{1},...,\boldsymbol{\mathsf{a}}_{p},\boldsymbol{\mathsf{b}}_{1},...,\boldsymbol{\mathsf{b}}_{q}),
    \label{FineJointProbability}
\end{equation}
such that the probabilities $p(\boldsymbol{\mathsf{a}}_{k}^{\prime},\boldsymbol{\mathsf{b}}_{l}^{\prime}|\boldsymbol{\mathsf{A}}_{k},\boldsymbol{\mathsf{B}}_{l})$ of $\boldsymbol{\vec{\boldsymbol{p}}}$ are obtained by marginalization
\begin{equation}
    p(\boldsymbol{\mathsf{a}}_{k}^{\prime},\boldsymbol{\mathsf{b}}_{l}^{\prime}|\boldsymbol{\mathsf{A}}_{k},\boldsymbol{\mathsf{B}}_{l}) = \sum \delta_{\boldsymbol{\mathsf{a}}_{k}^{\prime},\boldsymbol{\mathsf{a}}_{k}} \delta_{\boldsymbol{\mathsf{b}}_{l}^{\prime},\boldsymbol{\mathsf{b}}_{l}} \omega(\boldsymbol{\mathsf{a}}_{1},...,\boldsymbol{\mathsf{a}}_{p},\boldsymbol{\mathsf{b}}_{1},...,\boldsymbol{\mathsf{b}}_{q}),
    \label{FinesTheoremMarginalization}
\end{equation}
where the summation is over the possible sets of results for all contexts, and the Kronecker deltas involve only the results of the contexts $\boldsymbol{\mathsf{A}}_{k}$ and $\boldsymbol{\mathsf{B}}_{l}$.
\label{ExtendedFineTheorem}
\end{Theorem}
\begin{proof}
Suppose that $\boldsymbol{\vec{\boldsymbol{p}}}$ belongs to $\Local$. Then, its probabilities have a local decomposition
\begin{equation}
    p(\boldsymbol{\mathsf{a}}_{k}^{\prime},\boldsymbol{\mathsf{b}}_{l}^{\prime}|\boldsymbol{\mathsf{A}}_{k},\boldsymbol{\mathsf{B}}_{l}) = \sum_{\lambda} p(\lambda) p_{\lambda}(\boldsymbol{\mathsf{a}}_{k}^{\prime}|\boldsymbol{\mathsf{A}}_{k}) p_{\lambda}(\boldsymbol{\mathsf{b}}_{l}^{\prime}|\boldsymbol{\mathsf{B}}_{l}),
\end{equation}
where $\{p_{\lambda}(\boldsymbol{\mathsf{a}}_{k}^{\prime}|\boldsymbol{\mathsf{A}}_{k}) \}$ and $\{ p_{\lambda}(\boldsymbol{\mathsf{b}}_{l}^{\prime}|\boldsymbol{\mathsf{B}}_{l}) \}$ are general response functions. The following probability distribution for the results of all contexts may be constructed
\begin{equation}
    \omega(\boldsymbol{\mathsf{a}}_{1},...,\boldsymbol{\mathsf{a}}_{p},\boldsymbol{\mathsf{b}}_{1},...,\boldsymbol{\mathsf{b}}_{q}) 
    = \sum_{\lambda} p(\lambda) \left( \prod_{k=1}^{p} p_{\lambda}(\boldsymbol{\mathsf{a}}_{k}|\boldsymbol{\mathsf{A}}_{k}) \right) \left( \prod_{l=1}^{q} p_{\lambda}(\boldsymbol{\mathsf{b}}_{l}|\boldsymbol{\mathsf{B}}_{l}) \right).
\end{equation}
An important remark is that if a measurement belongs to two distinct contexts, its results in each of these contexts are independent variables. With this, it is possible to verify that property \eqref{FinesTheoremMarginalization} is satisfied.
\par
Conversely, assume there exists a joint probability distribution \eqref{FineJointProbability} that satisfies condition \eqref{FinesTheoremMarginalization}. It is possible to assign each combination of results for the contexts with a value for the summation variable:
\begin{equation}
    (\boldsymbol{\mathsf{a}}_{1},...,\boldsymbol{\mathsf{a}}_{p},\boldsymbol{\mathsf{b}}_{1},...,\boldsymbol{\mathsf{b}}_{q}) \xrightarrow{} \lambda.
    \label{FineTheoremAssignment}
\end{equation}
With this assignment, the distribution \eqref{FineJointProbability} can be considered as a probability distribution over $\lambda$. Besides, the functions $\delta_{\boldsymbol{\mathsf{a}}_{k}^{\prime},\boldsymbol{\mathsf{a}}_{k}}$ and $ \delta_{\boldsymbol{\mathsf{b}}_{l}^{\prime},\boldsymbol{\mathsf{b}}_{l}}$ can be viewed as deterministic response functions for a local decomposition, with the set of results determined for a specific context given by the assignment \eqref{FineTheoremAssignment}. Hence, condition \eqref{FinesTheoremMarginalization} gives a local decomposition.
\par
Note that, since the results of the same measurement in different contexts should be considered independent, the response functions considered above are deterministic given the context, but it is possible to have, for the same value of $\lambda$ and the same measurement, different results in different contexts, \textit{i.e.}, the deterministic assignment is not necessarily non-disturbing. In other words, it is necessary to consider both disturbing and non-disturbing response functions in the local model constructed in this proof.
\end{proof}
The final remark in the proof of Theorem \ref{ExtendedFineTheorem} precludes applying the same construction to the set $\Lnd$. It is still an open question whether there exists some version of Fine's Theorem for this set.

\section{Discussion}
\label{section:Discussion}
In this manuscript, we reviewed and expanded the framework of extended Bell scenarios, in which the possibility of compatible measurements for each laboratory is considered. 
This idea was introduced in ref.~\cite{TRC_2019}, and has already led to interesting contributions to Bell nonlocality and KS contextuality \cite{XRMTCRP_2022, BHC_2021}. %In our work, new results for the fields of nonlocality and contextualize were obtained.
\par
In extended Bell scenarios, it may seem a natural idea that any behaviour which is local and locally non-disturbing can be described by a local model in which the local behaviours are non-disturbing. 
In this manuscript, we prove this intuitive idea wrong. 
Even more strikingly, we also show that local behaviours which cannot exhibit contextuality in either of its parts may also not be described by local models that use only non-contextual behaviours for each party. 
These results highlight the fact that even the definition of local behaviours in extended Bell scenarios is not trivial, since there is no unique straightforward extension of the one in standard Bell scenarios, and several sets of local correlations may be defined.
As it happens in many fields of physics: \emph{more is different}.
In the case of Generalized Bell scenarios, instead of going to more parts, we add more structure to the measurement sets and allow for more than one (compatible) measurement per part. 
%Here, it was shown that the extended Bell scenarios bring a richer structure of correlation sets compared to standard scenarios. Among them, there are different local sets that can be defined, which depends on the types of local strategies included. This opens new paths to be explored in Bell nonlocality. Among the fundamental questions related to this, it is important to understand what is the most general definition of a local model, since this impacts what theories and experimental results may be classified as classical or not. From the practical side, these may be useful to explore the different types of resources that are needed in order to reproduce correlations between separated parties.
\par
Another important question investigated here is the relation of the quantum set with such different local sets. 
It was shown that there are quantum behaviours that cannot be simulated even by the most powerful local models, as would be expected. 
Nevertheless, it was also shown that there are non-disturbing local behaviours, composed of disturbing local strategies, that cannot be reproduced by quantum theory using quantum models. 
This situation is analogous to scenarios in which it is possible to use classical communication between Alice and Bob to reproduce quantum and post-quantum behaviours \cite{BR_2017}. 
In our case, the disturbance in the response functions, which allows the production of some post-quantum correlations, can be viewed as a kind of `communication' among the measurements. 
This may have fundamental and practical interests, e.g. in protocols that combine processing and communication, like distributed quantum computing. 
Moreover, it would be interesting to explore in future works if and how sequences of incompatible quantum measurements may produce these behaviours.
\par
Finally, it was investigated different manners to define that a behaviour is both local and non-contextual. 
It was shown that, in the most general way, there should be a local model regarding the locality of the measurements for the spatially separated laboratories and independent different non-contextual models for the marginal behaviours of each party. 
It is interesting to compare this result with the Fine-Abramsky-Brandenburger Theorem, which, for standard scenarios, characterizes the existence of a single hidden variable model for the results of all measurements, which in the framework of extended scenarios is equivalent to the set $\Lnc$. 
However, as our results show (theorem \ref{TheoremLocalNonContextualSets}), there are more general ways to explain the probabilities in terms of classical models. 
Considering the view of Bell scenarios as a special case of contextuality scenarios, it is possible, for instance, to separate the measurements into two groups, and to have a hidden-variable model that explains the behaviour as convex combinations of response functions that are independent regarding the measurements of each group. 
Then, it is possible to have independent hidden-variable models for the marginal behaviours of each group. 
This idea could be generalized to more complex groupings of the measurements, which also opens new paths to investigate in the future.
%\par
%To conclude, as shown by this and other works, the framework of extended Bell scenarios brings many new features and research paths to the fields of Bell nonlocality and Kochen-Specker contextuality, providing a step for a more unified treatment of them.



%%%%%%%%%%%%%%%%%%%%%%%%%%%%%%%%%%%%%%%%%%
\vspace{6pt} 

%%%%%%%%%%%%%%%%%%%%%%%%%%%%%%%%%%%%%%%%%%
%% optional
%\supplementary{The following are available online at \linksupplementary{s1}, Figure S1: title, Table S1: title, Video S1: title.}

% Only for the journal Methods and Protocols:
% If you wish to submit a video article, please do so with any other supplementary material.
% \supplementary{The following are available at \linksupplementary{s1}, Figure S1: title, Table S1: title, Video S1: title. A supporting video article is available at doi: link.} 

%%%%%%%%%%%%%%%%%%%%%%%%%%%%%%%%%%%%%%%%%%
\authorcontributions{Conceptualization: A.M., G.R., T.T., R.R., M.T.C.; methodology: A.M., G.R., T.T., R.R., M.T.C.; software: A.M.; validation: A.M.; formal analysis: A.M.; writing—original draft preparation: A.M.; writing—review and editing: A.M., G.R., C.H., T.T., R.R., M.T.C.. All authors have read and agreed to the published version of the manuscript.

%M.P. suggested the original problem to tackle; F.B., M.K., O.G. and M.P. identified the methodology and technical tools; F.B. developed the project and wrote the first draft of the manuscript, which was then finalized by all the authors.
}

\funding{This work has been supported by the São Paulo Research Foundation FAPESP (Grants No. 2018/07258-7, No. 2021/01502-6, and No. 2021/10548-0) and by CNPq grant 310269/2019-9.
}

%\institutionalreview{not applicable}

%\informedconsent{not applicable}

\dataavailability{\href{https://github.com/andremazzari/Artigo_2_Nome_Provisorio}{Github repository}} 

\acknowledgments{This work is part of Brazilian National Institute for Science and Technology on Quantum Information (INCT-IQ).}

%\conflictsofinterest{} 

%% Optional
%\sampleavailability{}

%%%%%%%%%%%%%%%%%%%%%%%%%%%%%%%%%%%%%%%%%%
%% Only for journal Encyclopedia
%\entrylink{The Link to this entry published on the encyclopedia platform.}

%%%%%%%%%%%%%%%%%%%%%%%%%%%%%%%%%%%%%%%%%%
%% Optional
%\abbreviations{Abbreviations}{}

%%%%%%%%%%%%%%%%%%%%%%%%%%%%%%%%%%%%%%%%%%
%% Optional
\appendixtitles{no} % Leave argument "no" if all appendix headings stay EMPTY (then no dot is printed after "Appendix A"). If the appendix sections contain a heading then change the argument to "yes".
\appendixstart
\appendix
\section[\appendixname~\thesection]{Proof of Theorem 1}
\label{AppendixLocalSets}
Considering the specific scenario described in the proof of Theorem \ref{TheoremLocalSets}, the behaviour in Table \ref{TableLNDBehaviour} belongs to $\LND$ but does not belong to $\Lnd$.

\begin{table}[!ht]
\begin{center}
\begin{tabular}{|c|c c c c c c c c|}
\hline
 & $000$ & $001$ & $010$ & $011$ & $100$ & $101$ & $110$ & $111$ \\
\hline
$A_0B_0B_1$ & $0$ & $1/2$ & $1/2$ & $0$ & $0$ & $0$ & $0$ & $0$ \\
$A_0B_1B_2$ & $1/2$ & $0$ & $1/2$ & $0$ & $0$ & $0$ & $0$ & $0$ \\
$A_1B_0B_1$ & $0$ & $0$ & $1/2$ & $0$ & $0$ & $1/2$ & $0$ & $0$ \\
$A_1B_1B_2$ & $0$ & $0$ & $1/2$ & $0$ & $1/2$ & $0$ & $0$ & $0$ \\
\hline
\end{tabular}
\caption{Example of behaviour that belongs to $\LND$ but does not belong to $\Lnd$. The entries of the table are given by the probabilities $p(a_{i},b_{j},b_{j+1}|A_{i},B_{j},B_{j+1})$, with $i,j \in \{0,1\}$ and $a_{i},b_{j},b_{j+1} \in \{0,1\}$. The results are given by the columns and the contexts are given by the rows.}
\label{TableLNDBehaviour}
\end{center}
\end{table}
 Table \ref{TableLNDBehaviourDecomposition} presents the local decomposition using disturbing response functions. Each row of the table gives the deterministic results of the contexts for one vertex. Mixing these two vertices with a weight of $\frac{1}{2}$ for each, the behaviour of Table \ref{TableLNDBehaviour} is obtained.
\begin{table}
\begin{center}
    \begin{tabular}{|c|c c c c|}
        \hline
         & $A_0$ & $A_1$ & $B_0B_1$ & $B_1B_2$  \\
         \hline
        Vertex 1 & $0$ & $0$ & $(1,0)$ & $(1,0)$ \\
        Vertex 2 & $0$ & $1$ & $(0,1)$ & $(0,0)$ \\
        \hline
    \end{tabular}
\end{center}
\caption{Vertices present in the local decomposition for the behaviour of Table \ref{TableLNDBehaviour}. Each row gives, for a specific vertex, the deterministic results for the contexts of Alice and Bob. Each vertex has a weight of $\frac{1}{2}$ in the local decomposition. Note that the deterministic result of the measurement $B_1$ depends on the context, and hence these vertices are disturbing. }
\label{TableLNDBehaviourDecomposition}
\end{table}

The following inequality is a facet inequality of the polytope $\Lnd$ for the scenario considered in the proof of Theorem \ref{TheoremLocalSets}:
\begin{align}
        -p(1,1,0|A_0,B_0,B_1) +p(1,0,1|A_0,B_1,B_2) +p(0,1,0|A_1,B_0,B_1)
        & \nonumber \\
        +p(0,1,0|A_1,B_1,B_2) +p(0,1,1|A_1,B_1,B_2) +p(1,0,0|A_1,B_1,B_2)
        & \nonumber\\
        +p(1,1,0|A_1,B_1,B_2) +p(1,1,1|A_1,B_1,B_2) & \leq 1.
    \label{2VLndInequality}
\end{align}
 The behaviour of table \ref{TableLNDBehaviour} achieves a value of 1.5 in the expression of the l.h.s of \eqref{2VLndInequality} and hence does not belong to the $\Lnd$ set. It was found using \textit{linear programming}\footnote{The behaviour of the table \ref{TableLNDBehaviour} is an extremal vertex of $\LND$ and it was also possible to find it using the vertex enumeration procedure with the software PANDA \cite{LR_2015}. However, this approach is not always feasible, as was the case for the scenario considered in appendix \ref{AppendixLocalNonContextualSets}.}. With this numerical technique, it is possible to optimize the value obtained in the expression of the facet \eqref{2VLndInequality} imposing that the behaviour should satisfy the facet inequalities of the sets $\Local$ and $\ND$. Linear programming was also used to find the local decomposition given in table \ref{TableLNDBehaviourDecomposition}. Denoting by $\vec{\boldsymbol{p}}$ the behaviour in table \ref{TableLNDBehaviour} and by $\Lambda$ a matrix with the extremal response functions of $\Local$, then it is possible to find a vector of coefficients $\vec{c}$ such that $\vec{\boldsymbol{p}} = \Lambda \vec{c}$.

\appendix
\section[\appendixname~\thesection]{Proof of Theorem 2}
\label{AppendixLocalNonContextualSets}
Regarding the scenario specified in the proof of Theorem \ref{TheoremLocalNonContextualSets}, the behaviour in table \ref{TableLndNCbehaviour} belongs to the set $\Lnd \cap \NC$ but does not belong to the set $\Lnc$. The following facet inequality of the set $\Lnc$ is violated by the behaviour in Table \ref{TableLndNCbehaviour}, achieving a value of 1/3:
\begin{equation}
    -p(1,1,1|A_1,B_0,B_1) +p(1,1,0|A_1,B_1,B_2) -p(1,0,0|A_1,B_2,B_0) \leq 0.
    \label{2TLncFacet}
\end{equation}
Hence, $\Lnc$ is a proper subset of $\Lnd \cap \NC$ in this scenario.
\par
The behaviour in table \ref{TableLndNCbehaviour} was found by means of linear programming. With this numerical technique, it is possible to impose that the behaviours must satisfy the non-contextuality inequalities of the scenario and the facets of $\Lnd$, and then optimize the value of the expression of the facets of $\Lnc$. Also by means of linear programming, it is possible to find a local decomposition using non-disturbing response functions for this behaviour, which confirms that it belongs to $\Lnd$. With the same strategy, it is possible to find a non-contextual decomposition for Bob's marginal distribution, which, in its turn, implies that the behaviour belongs to the set $\NC$ (it is not possible to have contextuality at Alice's side in this scenario). Hence, it belongs to $\Lnd \cap \NC$.

\begin{table}
\begin{center}
\begin{tabular}{|c|c c c c c c c c|}
\hline
 & $000$ & $001$ & $010$ & $011$ & $100$ & $101$ & $110$ & $111$ \\
\hline
$A_0B_0B_1$ & $1/3$ & $0$ & $0$ & $0$ & $0$ & $1/3$ & $1/3$ & $0$ \\
$A_0B_1B_2$ & $1/3$ & $0$ & $0$ & $0$ & $0$ & $1/3$ & $1/3$ & $0$ \\
$A_0B_2B_0$ & $1/3$ & $0$ & $0$ & $0$ & $0$ & $1/3$ & $1/3$ & $0$ \\
$A_1B_0B_1$ & $1/3$ & $0$ & $0$ & $0$ & $0$ & $1/3$ & $1/3$ & $0$ \\
$A_1B_1B_2$ & $1/3$ & $0$ & $0$ & $0$ & $0$ & $1/3$ & $1/3$ & $0$ \\
$A_1B_2B_0$ & $1/3$ & $0$ & $0$ & $0$ & $0$ & $1/3$ & $1/3$ & $0$ \\
\hline
\end{tabular}
\end{center}
\caption{Example of behaviour that belongs to $\Lnd \cap \NC$ but does not belong to $\Lnc$. The entries of the table are given by the probabilities $p(a_{i},b_{j},b_{j+1}|A_{i},B_{j},B_{j+1})$, with $i,j \in \{0,1\}$ and $a_{i},b_{j},b_{j+1} \in \{0,1,2\}$, with the sum in the index of Bob's measurements being modulo 3. The results are given by the columns and the contexts are given by the rows.}
\label{TableLndNCbehaviour}
\end{table}

\par
The behaviour in table \ref{TableLNCBehaviour} belongs to the set $\Local \cap \NC$ but does not belong to $\Lnd \cap \NC$. The following inequality  is a facet of the $\Lnd$ polytope and is violated by the behaviour in table \ref{TableLNCBehaviour}:
\begin{align}
        -2p(1,1,1|A0,B0,B1) +p(1,1,1|A0,B1,B2) +p(1,0,1|A0,B2,B0) & \nonumber
        \\
        -p(1,1,0|A0,B2,B0) +p(1,1,1|A0,B2,B0) +2p(1,1,1|A1,B0,B1) & \nonumber
        \\
        +p(0,1,0|A1,B1,B2) -p(1,1,0|A1,B1,B2) -p(1,1,1|A1,B1,B2) & \nonumber
        \\
        +p(0,1,0|A1,B2,B0) +p(1,0,0|A1,B2,B0) +p(1,1,0|A1,B2,B0) & \leq 1
    \label{2TLndFacet}
\end{align}
The same strategy used for the previous case was employed to find the behaviour in table \ref{TableLNCBehaviour}, \textit{i.e.}, linear programming was used to maximize the value of the expression of the facet \eqref{2TLndFacet} conditioning that the behaviours must belong to the set $\Local \cap \NC$. Linear programming can also be used to certify that the behaviour \ref{TableLNCBehaviour} belongs to $\Local \cap \NC$, finding a local decomposition with general response functions and a non-contextual decomposition for Bob's marginal distribution.
\begin{table}
\begin{center}
\begin{tabular}{|c|c c c c c c c c|}
\hline
 & $000$ & $001$ & $010$ & $011$ & $100$ & $101$ & $110$ & $111$ \\
\hline
$A_0B_0B_1$ & $0$ & $1/4$ & $0$ & $1/4$ & $0$ & $0$ & $1/2$ & $0$ \\
$A_0B_1B_2$ & $0$ & $0$ & $1/4$ & $1/4$ & $1/4$ & $1/4$ & $0$ & $0$ \\
$A_0B_2B_0$ & $0$ & $1/4$ & $1/4$ & $0$ & $0$ & $1/4$ & $0$ & $1/4$ \\
$A_1B_0B_1$ & $0$ & $1/4$ & $1/4$ & $0$ & $0$ & $0$ & $1/4$ & $1/4$ \\
$A_1B_1B_2$ & $0$ & $1/4$ & $1/4$ & $0$ & $1/4$ & $0$ & $0$ & $1/4$ \\
$A_1B_2B_0$ & $0$ & $1/4$ & $1/4$ & $0$ & $0$ & $1/4$ & $0$ & $1/4$ \\
\hline
\end{tabular}
\end{center}
\caption{Example of behaviour that belongs to $\Local \cap \NC$ but does not belong to $\Lnd \cap \NC$. The entries of the table are given by the probabilities $p(a_{i},b_{j},b_{j+1}|A_{i},B_{j},B_{j+1})$, with $i,j \in \{0,1\}$ and $a_{i},b_{j},b_{j+1} \in \{0,1,2\}$, with the sum in the index of Bob's measurements being modulo 3. The results are given by the columns and the contexts are given by the rows.}
\label{TableLNCBehaviour}
\end{table} 



%%%%%%%%%%%%%%%%%%%%%%%%%%%%%%%%%%%%%%%%%%
\begin{adjustwidth}{-\extralength}{0cm}
%\printendnotes[custom] % Un-comment to print a list of endnotes

\reftitle{References}

% Please provide either the correct journal abbreviation (e.g. according to the “List of Title Word Abbreviations” http://www.issn.org/services/online-services/access-to-the-ltwa/) or the full name of the journal.
% Citations and References in Supplementary files are permitted provided that they also appear in the reference list here. 

%=====================================
% References, variant A: external bibliography
%=====================================
%\bibliography{your_external_BibTeX_file}

%=====================================
% References, variant B: internal bibliography
%=====================================
\begin{thebibliography}{999}

\bibitem{BCPSW_14}
 N. Brunner, D. Cavalcanti, S. Pironio, V. Scarani and S. Wehner.
 %``Bell nonlocality''
 Bell nonlocality. \href{https://doi.org/10.1103/RevModPhys.86.419}{Rev. Mod. Phys. \textbf{86}, 419 (2014)}.
 
\bibitem{BCGKL_21}
 C. Budroni, A. Cabello, O. Gühne, M. Kleinmann, J. Larsson. Kochen-Specker Contextuality.
 %Kochen-Specker Contextuality
 \href{https://doi.org/10.48550/arxiv.2102.13036}{arXiv 2102.13036 (2021)}.

\bibitem{Bell_64}
 J. S. Bell. On the Einstein-Podolsky-Rosen paradox.
 %``On the Einstein-Podolsky-Rosen paradox'',
 Physics \textbf{1}, 195 (1964).
 
\bibitem{KS_67}
 S. Kochen and E. P. Specker. The problem of hidden variables in quantum mechanics.
 %``The problem of hidden variables in quantum mechanics.''
 J. Math. Mech. \textbf{17}, 59 (1967).
 
\bibitem{FC_72}
 S. J. Freedman and J. F. Clauser,
 %``Experimental Test of Local Hidden-Variable Theories''
 \href{https://doi.org/10.1103/PhysRevLett.28.938}{Phys. Rev. Lett. \textbf{28}, 938 (1972).}
 
\bibitem{ADR_82}
 A. Aspect, J. Dalibard, and G. Roger,
 %``Experimental Test of Bell's Inequalities Using Time- Varying Analyzers''
 \href{https://doi.org/10.1103/PhysRevLett.49.1804}{Phys. Rev. Lett. \textbf{49}, 1804 (1982).}
 
\bibitem{Hensen_etal_15}
 B. Hensen \textit{et al}.
 Loophole-free Bell inequality violation using electron spins separated by 1.3 kilometres.
 \href{https://doi.org/10.1038/nature15759}{Nature \textbf{526}, 682–686 (2015).}
 
\bibitem{Giustina_etal_15}
 M. Giustina \textit{et al}.
 .Significant-Loophole-Free Test of Bell’s Theorem with Entangled Photons.
 \href{https://doi.org/10.1103/PhysRevLett.115.250401}{Phys. Rev. Lett. \textbf{115}, 250401 (2015).}
 
\bibitem{Shalm_etal_15}
 L. K. Shalm \textit{et al}.
 .Strong Loophole-Free Test of Local Realism.
 \href{https://doi.org/10.1103/PhysRevLett.115.250402}{Phys. Rev. Lett. \textbf{115}, 250402 (2015).}
 
\bibitem{Weinfurter_etal_16}
W. Rosenfeld \textit{et al}.
Event-ready Bell-test using entangled atoms simultaneously closing detection and locality loopholes.
\href{https://doi.org/10.1103/PhysRevLett.119.010402}{Phys. Rev. Lett. {\bf 119}, 010402 (2017).}


\bibitem{KZG_2009}
G. Kirchmair, F. Zähringer, R. Gerritsma, \textit{et al}.
State-independent experimental test of quantum contextuality.
\href{https://doi.org/10.1038/nature08172}{Nature 460, 494–497 (2009)}.

\bibitem{ARBC_2009}
E. Amselem, M. Rådmark, M. Bourennane, A. Cabello.
State-Independent Quantum Contextuality with Single Photons.
\href{https://doi.org/10.1103/PhysRevLett.103.160405}{Phys. Rev. Lett. 103, 160405 (2009)}.

\bibitem{LLS_2011}
R. Lapkiewicz, P. Li, C. Schaeff, \textit{et al}.
Experimental non-classicality of an indivisible quantum system.
\href{https://doi.org/10.1038/nature10119}{Nature 474, 490–493 (2011)}.

\bibitem{Nobel_2022}
\href{https://www.nobelprize.org/prizes/physics/2022/prize-announcement/}{The Nobel Prize in Physics 2022}

\bibitem{CSW_14}
A. Cabello, S. Severini, and A. Winter.
.Graph-Theoretic Approach to Quantum Correlations.
\href{https://doi.org/10.1103/PhysRevLett.112.040401}{Phys. Rev. Lett. \textbf{112}, 040401 (2014).}

\bibitem{AFLS_15}
 A. Ac\'{i}n, T. Fritz, A. Leverrier, and A. B. Sainz.
 A combinatorial approach to nonlocality and contextuality.
 \href{https://doi.org/10.1007/s00220-014-2260-1}{Comm. Math. Phys. \textbf{334}, 533 (2015).}

\bibitem{RDTCC_14}
 R. Rabelo, C. Duarte, A. J. L\'{o}pez-Tarrida, M. Terra Cunha, and A. Cabello.
 Multigraph approach to quantum non-locality.
 \href{http://dx.doi.org/10.1088/1751-8113/47/42/424021}{J. Phys. A: Math. Theor. \textbf{47}, 424021 (2014).}
 
\bibitem{KCK_2014} 
P. Kurzy\ifmmode \acute{n}\else \'{n}\fi{}ski, A. Cabello, D. Kaszlikowski. Fundamental Monogamy Relation between Contextuality and Nonlocality. 
\href{https://doi.org/10.1103/PhysRevLett.112.100401}{Phys. Rev. Lett., 112 (10), 100401 (2014)}.

\bibitem{TRC_2019}
T. Temistocles, R. Rabelo, M. T. Cunha. Measurement compatibility in Bell nonlocality tests.
\href{https://doi.org/10.1103/PhysRevA.99.042120}{Physical Review A, 99(4), 042120 (2019)}.

\bibitem{XRMTCRP_2022}
Peng Xue, Lei Xiao, G. Ruffolo, A. Mazzari, T. Temistocles, M. Terra Cunha, R. Rabelo.
Synchronous Observation of Bell Nonlocality and State-Dependent Contextuality.
\href{
https://doi.org/10.1103/PhysRevLett.130.040201}{Phys. Rev. Lett. 130, 040201 (2023)}.

\bibitem{CHSH_69}
 J. F. Clauser, M. A. Horne, A. Shimony, and R. A. Holt.
 Proposed experiment to test local hidden-variable theories.
 \href{https://doi.org/10.1103/PhysRevLett.23.880}{Phys. Rev. Lett. \textbf{23}, 880 (1969).}
 
 \bibitem{KCBS_08}
 A. A. Klyachko, M. A. Can, S. Binicioglu, and A. S. Shumovsky.
 Simple Test for Hidden Variables in Spin-1 Systems.
 \href{https://doi.org/10.1103/PhysRevLett.101.020403}{Phys. Rev. Lett. \textbf{101}, 020403 (2008).}

\bibitem{BHC_2021}
 J. Bowles, F. Hirsch, D. Cavalcanti.
 Single-copy activation of Bell nonlocality via broadcasting of quantum states.
 \href{https://doi.org/10.22331/q-2021-07-13-499}{Quantum 5, 499 (2021)}.

%\bibitem{GWCAN_14}
%Gallego, Rodrigo and Würflinger, Lars Erik and Chaves, Rafael and Acín, Antonio  and Navascués, Miguel (2014). Nonlocality in sequential correlation scenarios. New J. Phys. 16 033037.

\bibitem{F_1982}
A. Fine.
Hidden Variables, Joint Probability, and the Bell Inequalities.
\href{https://doi.org/10.1103/PhysRevLett.48.291}{Phys. Rev. Lett. 48, 291 (1982)}.

\bibitem{AB_2011}
S. Abramsky, A. Brandenburger.
The sheaf-theoretic structure of non-locality and contextuality.
\href{https://doi.org/10.1088/1367-2630/13/11/113036}{New J. Phys. 13 113036 (2011)}.

\bibitem{P_1989}
I. Pitowsky.
\textit{Quantum Probability — Quantum Logic}.
1rd ed.; Springer Berlin, Heidelberg, 1989.

\bibitem{LR_2015}
S. Larwald, G. Reinelt.
PANDA: a software for polyhedral transformations.
\href{http://dx.doi.org/10.1007/s13675-015-0040-0}{EURO Journal on Computational Optimization 3, 297--308 (2015)}.

\bibitem{github}
\href{https://github.com/andremazzari/Generalized-Bell-scenarios-disturbing-consequences-on-local-hidden-variable-models}{Github repository}

\bibitem{AQBCC_2013}
Ara\'ujo, Mateus and Quintino, Marco T\'ulio and Budroni, Costantino and Cunha, Marcelo Terra and Cabello, Ad\'an.
All noncontextuality inequalities for the $n$-cycle scenario.
\href{https://link.aps.org/doi/10.1103/PhysRevA.88.022118}{Phys. Rev. A 88 (2), 022118 (2013)}.


\bibitem{NPA_2008}
M. Navascués, S. Pironio, A. Acín.
A convergent hierarchy of semidefinite programs characterizing the set of quantum correlations.
\href{https://doi.org/10.1088/1367-2630/10/7/073013}{New J. Phys. 10 073013 (2008)}.

\bibitem{BR_2017}
J. B. Brask and R. Chaves.
Bell scenarios with communication.
\href{https://doi.org/10.1088/1751-8121/aa5840}{J. Phys. A: Math. Theor. 50 094001 (2017)}.

    
\end{thebibliography}

% If authors have biography, please use the format below
%\section*{Short Biography of Authors}
%\bio
%{\raisebox{-0.35cm}{% Figure removed}}
%{\textbf{Firstname Lastname} Biography of first author}
%
%\bio
%{\raisebox{-0.35cm}{% Figure removed}}
%{\textbf{Firstname Lastname} Biography of second author}

% For the MDPI journals use author-date citation, please follow the formatting guidelines on http://www.mdpi.com/authors/references
% To cite two works by the same author: \citeauthor{ref-journal-1a} (\citeyear{ref-journal-1a}, \citeyear{ref-journal-1b}). This produces: Whittaker (1967, 1975)
% To cite two works by the same author with specific pages: \citeauthor{ref-journal-3a} (\citeyear{ref-journal-3a}, p. 328; \citeyear{ref-journal-3b}, p.475). This produces: Wong (1999, p. 328; 2000, p. 475)

%%%%%%%%%%%%%%%%%%%%%%%%%%%%%%%%%%%%%%%%%%
%% for journal Sci
%\reviewreports{\\
%Reviewer 1 comments and authors’ response\\
%Reviewer 2 comments and authors’ response\\
%Reviewer 3 comments and authors’ response
%}
%%%%%%%%%%%%%%%%%%%%%%%%%%%%%%%%%%%%%%%%%%
\end{adjustwidth}
\end{document}

