\section{Design}
\label{sec:design}

The FDC design draws inspiration from the ZEUS BPC~\cite{Surrow:1998su} and H1 VLQ~\cite{Stellberger_2003} calorimeters, incorporating modern enhancements in optical readout and photosensors. It leverages recent advancements in high-granularity calorimetry~\cite{Sefkow:2015hna}, led by the CALICE collaboration. These developments offer the potential for substantial improvements in granularity at a reasonable cost for a small detector such as the FDC.

The CALICE collaboration has tested a scintillator-tungsten ECAL with wavelength-shifting fibers coupled to SiPMs~\cite{CALICE:2017sis}. The tested strips had widths of 10 mm. By applying a ``split-strip algorithm''~\cite{KOTERA2015158}, this prototype can achieve an effective granularity close to 10$\times$10 mm$^{2}$. The test-beam data resulted in an energy resolution of 12$\%/\sqrt{E}\oplus1.2\%$. More recently, this design has been superseded by a ``SiPM-on-tile'' approach, where the SiPM is air-coupled to the scintillator strip~\cite{Dong:2018hvs,Niu:2020iln}~\footnote{This approach is also used in the ePIC calorimeter insert~\cite{Arratia:2022quz,Arratia:2023rdo}.}.

Table~\ref{tab:summary} summarizes our target design in comparison with the ZEUS BPC and H1 VLQ designs. 
\begin{table}[h!]
   \centering
   \caption{Summary description of our proposed EIC FDC, the ZEUS BPC~\cite{Surrow:1998su}, and the H1 VLQ~\cite{Stellberger_2003}.} 
   \begin{tabular*}{.50\textwidth}{@{\extracolsep{\fill}}llll@{}}
    \hline
       & EIC FDC &  ZEUS BPC & H1 VLQ \\
      \hline
        Depth & 20 $\mathrm{X}_{0}$ & 24 $\mathrm{X}_{0}$ & 16.7 $\mathrm{X}_{0}$ \\
        W/Sc thickness & 3.5/2 mm & 3.5/2.6 mm  & 2.5/3 mm  \\
        Moliere Radius & 15 mm & 13 mm & 15 mm\\
        Optical readout & SiPM-on-tile & WLS bar& WLS bar
        \\
        & & + PMT & +PIN \\ 
        Trans. granularity & 10$\times$50 mm$^{2}$ & 7.9$\times$150 mm$^{2}$ & 5$\times$120 mm$^{2}$\\
        Long. granularity & every strip & none & none \\
        Channels & 4500 & 31 & 336  \\ 
        Readout & HGROC & FADC/TDC & ASIC  \\ 
        Position res. & 3.6 mm$/\sqrt{E}$ &  2.2 mm$/\sqrt{E}$ & 2 mm$/\sqrt{E}$\\
        Energy res. & $\frac{17\%}{\sqrt{E}}\oplus2\%$ & $\frac{17\%}{\sqrt{E}}\oplus2\%$ & $\frac{13\%}{\sqrt{E}}\oplus3\%$  \\
        Time resolution &   $<$50 ps & 400 ps & ---\\

\hline

  \end{tabular*}
   \label{tab:summary}
\end{table}

Figure~\ref{fig:explode_view} shows the FDC design which includes alternating layers of vertical and horizontal scintillators that are wrapped in reflective foil and read out using SiPMs (HPK 14160-1315PS). The scintillator strips measure $50\times10\times2$ mm (length, width, thickness) and feature a dimple at the center for air-coupling with the SiPM.
% Figure environment removed

Each tungsten layer is 3.5 mm (1 $X_{0}$, and 0.035$\lambda$). The total FDC consists of 20 layers, for a total of 20 $X_{0}$. Up to about $\eta=-4.1$, the detector has full coverage in $\phi$, whereas at larger negative $\eta$, the hole for the hadronic beampipe removes up to $60^\circ$ of acceptance.  

The FDC is divided into two parts, allowing one half to be retracted to the left and the other half to the right for maintenance purposes of the SiPM boards, particularly in the case of unexpected radiation loads. However, it is worth noting that SiPM annealing should not be necessary given that at the FDC location, the neutron flux is moderate and reaches approximately 10$^{9}$ 1-MeV equivalent neutrons per cm$^{2}$ for a year of running at top luminosity~\cite{Doses}.

%The FDC material as function of $\eta$ and $\phi$ is shown in the right panel of Fig.~\ref{fig:mat_budget}.  Up to about $\eta=-4.1$, the detector has full coverage in $\phi$, whereas at larger negative $\eta$, the hole for the hadronic beampipe removes up to $60^\circ$ of acceptance.  
%% Figure environment removed





