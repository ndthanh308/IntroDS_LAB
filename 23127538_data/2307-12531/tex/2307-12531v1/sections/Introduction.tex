\section{Introduction}
\label{sec:outline}

The technology currently proposed to cover the smallest electron-scattering angles with the central detector of the future Electron-Ion Collider (EIC) involves the use of lead-tungsten crystals~\cite{AbdulKhalek:2021gbh}. Although these crystals provide exceptional energy resolution, they face limitations when employed near the beampipe. The beampipe has a complex geometry due to the 25 mrad beam-crossing angle and incorporates flanges connecting the electron and hadron beampipes. Furthermore, the system loads necessitate a support structure near the beampipe with adequate clearance margins, which further limits their coverage.

The EIC Yellow Report~\cite{AbdulKhalek:2021gbh} established an acceptance requirement of $-4<\eta<4$. However, it is not feasible to meet the lower limit of this range using only a crystal calorimeter. The current design of the EIC detector, ePIC, places the crystal ECAL at $z = -174$ cm, with an inner radius of $\approx$8 cm~\cite{managerie,ELKEprivate}, resulting in a nominal acceptance that reaches $\eta\approx-3.6$. This limitation is mainly due to the assembly requirements of the ECAL, which necessitate a hole large enough to allow it to slide through the flange further downstream. 

At the highest EIC energy, $\eta=-3.6$ corresponds to $Q^{2}\approx$ 1 GeV$^{2}$. Consequently, the limited acceptance of the ECAL hinders the study of the transition to the perturbative QCD domain, for which measuring $Q^{2}<$ 1 GeV$^{2}$ is necessary.  This limitation affects various physics topics at the EIC, including flagship studies such as searches for gluon saturation.

The region of very low $Q^{2}$ will be covered by far-backward taggers~\cite{AbdulKhalek:2021gbh}. However, the transition region {$0.1 < Q^{2} < $ 1 GeV$^{2}$} remains beyond the capabilities of existing designs, creating the so-called ``$Q^{2}$ gap''~\cite{GDI}. Some possible approaches to bridge the $Q^{2}$ gap include lowering the electron beam energy, shifting the interaction-point position, and adding dedicated instruments.

Lowering the electron beam energy, $E$, helps because the minimum accessible $Q^{2}$ scales with $E^{2}$. While this is an elegant solution, it comes at the significant cost of reducing the center-of-mass energy of the collisions and, therefore, reducing the coverage at low values of $x$ (the minimum $x$ scales with $1/E$). Consequently, this approach does not provide access to the required kinematic region for gluon-saturation studies.

The second solution of shifting the interaction point was employed at HERA but is completely impractical at the EIC due to the presence of a beam crossing angle.

The third solution was implemented at HERA with the ZEUS Beam Pipe Calorimeter (BPC)~\cite{Surrow:1998su} and the H1 Very Low $Q^2$ spectrometer (VLQ)~\cite{Stellberger_2003}.  These detectors provided coverage down to $Q^{2}=0.11$ GeV$^{2}$ and $Q^{2}=0.08$ GeV$^{2}$, respectively.

The third option is the only viable one and is the focus of our work. In this study, we describe the design of a detector that will cover a few degrees from the electron-beam direction. We refer to this detector as the Few-Degree Calorimeter (FDC), which will cover the range of $-4.6 < \eta < -3.6$.

Figure~\ref{fig:x_Q2} illustrates the coverage in the $x$ vs.~$Q^2$ phase-space within the nominal range of the FDC for the top-energy settings for $eA$ collisions. The FDC has the potential to significantly extend acceptance into a key kinematic phase-space and enable the study of the predicted gluon-saturation transition in inclusive DIS~\cite{PhysRevLett.86.596}, as well as many other observables such as exclusive vector-meson production~\cite{H1:2005dtp,ZEUS:2007iet}.

% Figure environment removed

