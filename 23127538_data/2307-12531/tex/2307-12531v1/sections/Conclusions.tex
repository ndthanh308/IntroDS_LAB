\section{Summary and Conclusions}
\label{sec:conclusions}
We have outlined the design of a small electromagnetic calorimeter, the Few-Degree Calorimeter (FDC), which is designed to cover the range of $-4.6 < \eta < -3.6$. The primary objective of this detector is to tag electrons within the $Q^2$ range of 0.1 to 1.0 GeV$^2$, thus enabling future research on the transition to perturbative QCD and the gluon-saturation regime.

The FDC design we present here incorporates the latest advancements in SiPM-on-tile calorimetry to create a modern and improved version of the ZEUS Beam Pipe Calorimeter and H1 Very Low $Q^{2}$ calorimeter. The incorporation of high-granularity 5D shower measurements (position, time, and energy) offered by this technology holds great potential for background tagging.

In conclusion, this document presents the first design that has the potential to close the EIC $Q^2$ gap while maintaining a compact and cost-effective solution. Considering the larger crossing-angle envisioned for the second-interaction region at the EIC, which results in a larger hole in the crystal ECAL acceptance, this design may offer further opportunities for optimization for the EIC Detector 2.

