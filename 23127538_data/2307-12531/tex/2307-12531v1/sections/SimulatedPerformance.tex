\section{Simulation}
\label{sec:simulation}

We used the \textsc{DD4HEP} framework~\cite{Frank:2014zya} to run \textsc{Geant4}~\cite{GEANT4:2002zbu} simulations of electrons generated with a uniform azimuthal angle at various $\eta$ points. The simulation does not include any dead material, the effects of which we leave for future work. 
\subsection{Energy Resolution}
Figure~\ref{fig:recon_E} shows the energy resolution, which can be parameterized as 17$\%/\sqrt{E} \oplus 2\%$, and is consistent with the ZEUS BPC data~\cite{Surrow:1998su}. We also compare it to CALICE data, which exhibits improved performance at the expense of a larger Moli\`ere radius. 
% Figure environment removed

Figure~\ref{fig:edge} shows the energy resolution and scale as a function of $\eta$. The performance remains relatively stable for $\eta>-$4.6.
 For particles that hit near the edge of the detector's upstream face ($\eta\approx-4.8$), the energy-scale offset is $-$20\%, and the resolution is about 12\%. One would expect that half the shower would be in the calorimeter (a loss of $50\%$), but because the hole is a cylinder, the center of the shower moves further away from the hole as it goes through the calorimeter.

% Figure environment removed
 
\subsection{Position Resolution}
The polar-angle resolution is determined by considering the position resolution of the FDC and the resolution of the vertex position. The electron vertex position can be precisely determined by tracking other particles in the event using the main detectors, as was done in HERA~\cite{Surrow:1998su,Stellberger_2003}. Thus, only the FDC position resolution is relevant.

We reconstructed the $x$ and $y$ values following the method described in Ref.~\cite{Monteiro:1998bi}:
\begin{equation}
x=\frac{\sum\limits_{i\in v\,layers} w_{X,i} x_i}{\sum\limits_{i\in v\,layers} w_{X,i}}, 
y=\frac{\sum\limits_{i\in h\,layers} w_{Y,i} y_i}{\sum\limits_{i\in h\,layers} w_{Y,i}}
\end{equation}
where the weights $w_{X,i}$ and $w_{Y,i}$ are determined by 
\begin{align}
    w_{X,i}&={\rm max}\left(0, w_0+
{\rm log}\frac{E_i}{\sum_{j\in\rm v\,layers}E_{j}}\right)\\
w_{Y,i}&={\rm max}\left(0, w_0+
{\rm log}\frac{E_i}{\sum_{j\in\rm h\,layers}E_{j}}\right)
\end{align}
The ``h layers'' sums are over layers with horizontally aligned strips and the ``v layers'' sums are over layers with vertically aligned strips. The cutoff parameter $w_{0}$ is set to 4.0.

Figure~\ref{fig:pos_res} shows the position resolution as a function of energy. For energies greater than 1 GeV, the position resolution is better than the strip width divided by $\sqrt{12}$. The resolution we obtained is poorer than the ZEUS BPC resolution. This difference can be partially explained by the smaller strip width (7.9 mm vs 10 mm), but it may also include components from algorithm tuning. %At the proposed position of $z = -307$ cm from the target, this corresponds to a contribution of 1 mrad/$\sqrt{E}$ to the $\theta$ resolution. 
% Figure environment removed


\subsection{Kinematic Variable Reconstruction}
The resolution in Bjorken $x$ and $Q^2$ can be derived from those of the electron energy $\delta E'_e$ and of the electron polar angle $\delta \theta_e$ (using the small-angle approximation):
\begin{align}
    \frac{\delta Q^2}{Q^2}&\approx\frac{\delta E_e'}{E_e'} \oplus  \frac{2}{\pi-\theta_e}\delta \theta_e, \\
    \frac{\delta x}{x}&\approx\frac{1}{y}\frac{\delta E_e'}{E_e'}\oplus \left(\frac{x}{E_e/E_p}-1\right)\frac{2}{\pi-\theta_e} \delta\theta_e
    \label{eq:res_Q2}
\end{align}

In the non-divergent region (\textit{i.e.} $y>0.1$), the $Q^{2}$ resolution ranges from 4\% to 14\% depending on kinematics, whereas the $x$ resolution ranges from 10\% to 50\% with a strong $y$ dependence. To quantify these resolutions in the context of inclusive DIS measurements, we followed the EIC YR approach and calculated the corresponding purity and stability values. In this context, purity is determined by calculating the fraction of events reconstructed in a specific bin that were also generated in that bin (i.e., $P = N_{\mathrm{(rec,gen)}} / N_\mathrm{rec}$). Stability is calculated as the fraction of events generated in a specific bin that were also reconstructed in that bin (i.e., $S = N_{\mathrm{(rec,gen)}} / N_{\mathrm{gen}}$). Here, $N_{\mathrm{(rec,gen)}}$ represents the number of events where the electron is both generated and reconstructed in the same bin. For the generated events, we used the same \textsc{Pythia6} simulation as described in Section~\ref{sec:bkgrejection}.

Figure~\ref{fig:purity} shows the resulting purity and stability plot for events with $-4.6<\eta<-3.6$. The plot has 5 bins per decade in both $x$ and $Q^{2}$, similar to the EIC Yellow Report~\cite{AbdulKhalek:2021gbh}. Purity and stability values above 50\% are observed for the phase-space covered with $y>0.1$, with some degradation at lower values, as expected when using the electron reconstruction method exclusively. It is anticipated that the performance for $0.01<y<0.1$ will improve by combining the electron and hadronic methods, potentially using machine-learning techniques~\cite{Arratia:2021tsq}.

% Figure environment removed

A potential issue that can impact purity and stability is the electron beam's angular divergence, which can reach up to 200 $\mu$rad~\cite{eic_cdr}. Although this limits the kinematic reconstruction for electrons scattered at angles less than 10 mrad~\cite{eic_cdr}, the reconstruction of inclusive kinematic variables will not be compromised since the FDC acceptance begins at 20 mrad.

Overall, these studies demonstrate that the FDC design can provide sufficient resolution for measuring kinematic variables in the low $x$, low $Q^{2}$ region, thereby bridging the $Q^{2}$ gap.

\subsection{Shower-shape Examples}

Figure~\ref{fig:showershape} shows 3D and projection views of example showers. The three scenarios depicted are: an electron reaching the FDC with no material in front of it (left), a photon that initiated showering in the beampipe (middle), and a $\pi^-$. Among the three cases, the electron without pre-showering produces the narrowest shower, while the pre-showering photon and the $\pi^-$ generate more irregular showers.

% Figure environment removed

In terms of converted-electron tagging, the showers are expected to exhibit broader transverse profiles and anomalous longitudinal development. For hadron tagging, the fine segmentation will be primarily used to identify the starting point of the shower, which is more likely to be located at a deeper position compared to electrons. Moreover, the hadron showers will also have a different time development, with hits at later times compared to electron showers. 

This highlights the potential of 5D shower shape analysis for background rejection. In our future work, we intend to explore this potential by applying machine learning techniques to tag electrons/photons, hadron showers, and beam-gas background. Some promising emerging work, which can handle the complex geometry of the FDC, are point-cloud networks~\cite{ATL-PHYS-PUB-2022-040}. 

