\section{Design Constraints and Requirements}
\label{sec:requirements}
\subsection{Location and Acceptance}
The primary challenge in measuring electron scattering at small angles lies in effectively instrumenting the region near the beampipe while minimizing the material in front of the detector. 

One possibility is to position the FDC behind the crystal ECAL and in front of the backward HCAL, as illustrated in Fig.~\ref{fig:overview}. In the current ePIC design~\cite{managerie}, a potential location is at $z=-307$ cm, which would leave space for a compact calorimeter and about 10 cm gap before the HCAL. 

% Figure environment removed

Figure~\ref{fig:eta_rings} shows that at $z=-307$ cm, the electron beampipe for IP6 has a radius of 4.5 cm, while the hadron beampipe has a radius of 1.8 cm and is shifted 8.3 cm in the $x$ direction. Assuming a 5 mm clearance to the beampipe, similar to the ZEUS BPC~\cite{Surrow:1998su}, a calorimeter with an outer perimeter of $30\times40$ cm$^{2}$ could cover the region $-4.6<\eta<-3.6$ with non-uniform azimuthal coverage. 

The shaded region on the FDC in Fig.~\ref{fig:eta_rings} represents the area where electrons would encounter part of the crystal ECAL or its support structure before reaching the FDC\footnote{We approximate the path of the electron as a straight line and assume that the flat cables servicing the ECAL SiPMs can be arranged to avoid the hole area.}. 
This region, which is a few cm wide, would serve as a veto area.

% Figure environment removed
\newpage
Figure~\ref{fig:FDCposition} shows projections of the possible detector layout, including both the $yz$ and $xz$ views. The ECAL hole is currently assumed to have a height of 14.7 cm and a width of 20.5 cm, taking into account the current version of the ``micro-flange'' (with a cam shape that is 15.2 cm wide and 11.1 cm tall) and a required clearance of 1.8 cm, 3.6 cm, 1.8 cm, and 1.7 cm between the flange and the ECAL's inner support structure on the top, right, bottom, and left sides when looking downstream~\cite{ELKEprivate}. 

% Figure environment removed
\subsection{Dead Material in Front of FDC}
The main challenge faced by a detector located in a high pseudorapidity range is that particles can encounter a significant amount of material as they graze the beampipe walls. While converted electrons and photons can be identified through shower-shape information, reducing the amount of material helps by improving efficiency and minimizing background. 

Figure~\ref{fig:rad_lengths} illustrates the number of radiation lengths of beampipe material encountered by electrons before reaching the FDC, as a function of $\eta$. Within most of the FDC acceptance, the total number of radiation lengths ranges from 0.5 to 1.2. Approximately half a radiation length is contributed by the flange at $z=-120$ cm within the range of $-4.2 < \eta < -3.5$. The significant increase in the total number of $X_0$ traversed at around $\eta=-4.0$ is attributed to the use of aluminum instead of beryllium in the beampipe for $z < -80$ cm.

% Figure environment removed

 The HERA experiments successfully addressed this challenge by implementing a beampipe ``exit window'' made of thin aluminum, which reduced the total dead material to less than 1 $X_{0}$~\cite{Stellberger_2003}. Alternatively, one could use a beryllium section. 

Figure~\ref{fig:rad_lengths} illustrates that incorporating a beryllium section, within the range of $-205 < z < -80$ cm, would effectively decrease the overall material budget in the $-4.7 < \eta < -4.1$ region to below 0.5 $X_{0}$. As an alternative, implementing a 1.5 mm aluminum layer would yield a reduction to less than 1 $X_{0}$ for $\eta > -4.5$.

Another effective method for mitigating the impact of dead material is to use shower shapes to tag converted electrons or photons that originated further upstream in the beampipe or flanges~\cite{Surrow:1998su}. 

\subsection{Acceptance Limit}
To enable accurate FDC measurements at small angles, it is crucial to minimize energy leakage into the beampipe. Figure~\ref{fig:distance_from_beampipe} shows the distance to the beampipe surface as a function of $\eta$.
% Figure environment removed
The target value of $\eta=-4.6$ corresponds to about 18 mm from the electron beampipe at $z=-307$ cm. Assuming a 5 mm clearance, this leaves 13 mm between the edge of the detector and $\eta=-4.6$. For reference, the ZEUS BPC measured 95$\%$ of the energy from 5 GeV electrons at 8 mm from the detector's edge in test beams~\cite{Surrow:1998su}.

\subsection{Energy Range}
The main objective of the FDC is to identify and measure the energy and angle of electrons in the $0.1<Q^{2}<1.0$ GeV$^{2}$ range. Figure~\ref{fig:MinEnergy} illustrates the minimum electron energy as a function of $\eta$ for various $Q^2$ values. The minimum energy required for $-4.6<\eta<-3.6$ falls within the range of 2--13 GeV, whereas the maximum is the beam energy, 18 GeV. Thus, the target energy range is 2--18 GeV.
% Figure environment removed

\subsection{Background Rejection}
\label{sec:bkgrejection}
The main background for inclusive DIS measurements originates from events with small $Q^{2}$, where the scattered electron is not detected, but an electron candidate is observed in the FDC. Figure~\ref{fig:bkg} shows the expected particle spectra obtained using \textsc{Pythia6} to simulate $ep$ scattering without a $Q^2$ cut and without considering detector effects. In the absence of charge tagging, both electrons and positrons from semi-leptonic decays contribute to the background. Similarly, the charged-pion background includes both charges. The photon background primarily arises from neutral-pion decays.
% Figure environment removed

 We estimate the background rejection power of approaches employed at HERA~\cite{ZEUS:1997etp, Surrow:1998su} and the EIC YR. Specifically, we use the far-backward detectors as a veto for photoproduction and select events based on their energy-momentum imbalance\footnote{The $E-p_{z}$ distribution peaks at twice the electron-beam energy when the scattered electron is correctly identified.} with a loose selection of $E - p_{z}>18$ GeV~\cite{AbdulKhalek:2021gbh}. 



 Figure~\ref{fig:bkgratio} illustrates the impact of these cuts on the $e/\pi$ ratio. The far-backward veto has a modest effect due to its small acceptance~\cite{AbdulKhalek:2021gbh}. Similarly, the $E - p_{z}$ cut has a modest impact, since for background events the scattered electron has low energy. Overall, the resulting $e/\pi$ ratio is about $e/\pi\approx$ 0.4 between 1--6 GeV, $e/\pi\approx$ 1 around 10 GeV, and increases rapidly at higher energies. Similarly, the resulting $e/\gamma$ ratio ranges from 0.1 to 1 for $E<6$ GeV and increases at higher energies. The background of positrons and electrons from semi-leptonic decays reaches 10\% at 1 GeV, decreasing to less than 1\% at 10 GeV. Since there is no magnetic field near the FDC, this background cannot be subtracted using reverse-field runs. Instead, Monte-Carlo studies would be employed to estimate and subtract it. Alternatively, electron-isolation criteria might reduce this background further. 

% Figure environment removed

The FDC measurements should aim for a purity greater than 90\% to achieve background uncertainties at the few percent level. A stretch goal of 99\% purity would result in a negligible uncertainty compared to the expected luminosity uncertainty of 1\%~\cite{AbdulKhalek:2021gbh}. Hence, the necessary rejection power falls within the range of 10-25 for $\pi^{\pm}$ and 10-100 for $\gamma$, with a factor of 10 higher for the stretch goal.

The standalone FDC's $\pi^{\pm}$ rejection power will rely on its shower-shape capabilities. Longitudinal segmentation can play a crucial role in discriminating hadrons, as they are more likely to interact deeper within the detector, while electrons tend to exhibit showers starting primarily in the initial layers. Moreover, transverse segmentation also helps, as electrons typically produce narrower and more regular showers compared to hadronic ones. This approach is expected to work well above a few GeV. 

Additional background rejection can be achieved through auxiliary systems, such as a scintillator layer for tagging MIPs to reject unconverted photons, or a timing layer to reject low-energy hadrons. Figure~\ref{fig:time_requirement} demonstrates the potential of TOF and illustrates that a time resolution of about 50~ps would be necessary to achieve a 2$\sigma$ $e/\pi$ separation below 1~GeV.

% Figure environment removed

