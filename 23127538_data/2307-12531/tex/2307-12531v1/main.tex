%% Copyright 2007-2020 Elsevier Ltd
%% 
%% This file is part of the 'Elsarticle Bundle'.
%% ---------------------------------------------
%% 
%% It may be distributed under the conditions of the LaTeX Project Public
%% License, either version 1.2 of this license or (at your option) any
%% later version.  The latest version of this license is in
%%    http://www.latex-project.org/lppl.txt
%% and version 1.2 or later is part of all distributions of LaTeX
%% version 1999/12/01 or later.
%% 
%% The list of all files belonging to the 'Elsarticle Bundle' is
%% given in the file `manifest.txt'.
%% 

%% Template article for Elsevier's document class `elsarticle'
%% with numbered style bibliographic references
%% SP 2008/03/01
%%
%% 
%%
%%
%%
%\documentclass[preprint,12pt]{elsarticle}

%% Use the option review to obtain double line spacing
%% \documentclass[authoryear,preprint,review,12pt]{elsarticle}

%% Use the options 1p,twocolumn; 3p; 3p,twocolumn; 5p; or 5p,twocolumn
%% for a journal layout:
%% \documentclass[final,1p,times]{elsarticle}
%% \documentclass[final,1p,times,twocolumn]{elsarticle}
%% \documentclass[final,3p,times]{elsarticle}
%% \documentclass[final,3p,times,twocolumn]{elsarticle}
%% \documentclass[final,5p,times]{elsarticle}
\documentclass[final,5p,times,twocolumn]{elsarticle}

%% For including figures, graphicx.sty has been loaded in
%% elsarticle.cls. If you prefer to use the old commands
%% please give \usepackage{epsfig}

%% The amssymb package provides various useful mathematical symbols
\usepackage{amsmath}
\usepackage{amssymb}
\usepackage{lipsum}  
\usepackage{placeins}
\usepackage{subcaption}
\usepackage{overpic}
\usepackage{url}
\usepackage[pdftex, pdftitle={Article}, pdfauthor={Author},colorlinks = true,allcolors = blue]{hyperref}
%% The amsthm package provides extended theorem environments
%% \usepackage{amsthm}

%% The lineno packages adds line numbers. Start line numbering with
% \begin{linenumbers}, end it with \end{linenumbers}. Or switch it on
%% for the whole article with \linenumbers.
\usepackage{lineno}
\journal{NIMA}

\begin{document}



\affiliation[UCR]{Department of Physics and Astronomy, University of California, Riverside, CA 92521, USA}
\affiliation[JLAB]{Thomas Jefferson National Accelerator Facility, Newport News, VA 23606, USA}



\author[UCR,JLAB]{Miguel Arratia\corref{correspondingauthor}}
\cortext[correspondingauthor]{Corresponding author}

\author[UCR]{Ryan Milton}
\author[UCR,JLAB]{Sebouh Paul}
\author[UCR]{Barak Schmookler}
\author[UCR]{Weibin Zhang}

\title{A Few-Degree Calorimeter for the future Electron-Ion Collider}



%\date{\today} 

\begin{abstract}
Measuring the region $0.1 < Q^{2} < 1.0$ GeV$^{2}$ is essential to support searches for gluon saturation at the future Electron-Ion Collider. Recent studies have revealed that covering this region at the highest beam energies is not feasible with current detector designs, resulting in the so-called $Q^{2}$ gap.  In this work, we present a design for the Few-Degree Calorimeter (FDC), which addresses this issue. The FDC uses SiPM-on-tile technology with tungsten absorber and covers the range of $-4.6 < \eta < -3.6$. It offers fine transverse and longitudinal granularity, along with excellent time resolution, enabling standalone electron tagging. Our design represents the first concrete solution to bridge the $Q^{2}$ gap at the EIC.

\end{abstract}




\maketitle
\newpage
\tableofcontents
%\newpage
The problem of the presence or absence of phase transition is central in statistical mechanics. To prove the existence of phase transition, the standard idea is to define a notion of contour and use \textit{Peierls' argument} \cite{Peierls.1936}. In the usual Ising model \cite{Ising_25}, particles of the system interact only with their nearest-neighbors. On ferromagnetic long-range Ising models \cite{Anderson_Yuval_69}, there is interaction between each pair of spins in the lattice. The Hamiltonian of the model is given formally by
\begin{equation*}
    H(\sigma) = - \sum_{x,y\in \Z^d}J_{xy}\sigma_x\sigma_y,
\end{equation*}
where $J_{xy}=J|x-y|^{-\alpha}$, $J>0$, $\alpha > d$. It is well-known that the Peierls' argument in dimension 2 implies phase transition for Ising models with nearest-neighbors or long-range interactions when $d\geq 2$, using correlation inequalities. For the unidimensional lattice, it was known that short-range models do not present phase transition. In the long-range case, a different behavior was expected depending on the exponent $\alpha$ (see \cite{Kac_Thompson_69}), but the problem was challenging since contours were first created as multidimensional objects.

In dimension $d=1$, phase transition was proved first in 1969 by Dyson \cite{Dyson.69}, for $\alpha \in (1,2)$, by proving phase transition in an auxiliary model and then using correlation inequalities. In 1982, Fr{\"o}hlich and Spencer \cite{Frohlich.Spencer.82} introduced a notion of one-dimensional contours and then applied the Peierls' argument to show phase transition for the critical value $\alpha = 2$. These contours were inspired by the multiscale techniques previously introduced to study the Berezinskii-Kosterlitz-Thouless transition in two-dimensional continuous spin systems \cite{FS81}. Later, Cassandro, Ferrari, Merola and Presutti  \cite{Cassandro.05} extended the contour argument previously available for $\alpha=2$ to exponents $\alpha\in (3-\frac{\ln 3}{\ln 2}, 2)$, with the additional restriction that the nearest-neighbor interaction is strong, i.e.,  ${J(1)\gg 1}$; this restriction was removed for a subclass of interactions in \cite{Bissacot.Endo.18}. Further results were obtained using contour arguments, such as the decay of correlations, cluster expansions, phase transition with random interactions, etc; some references with these results are \cite{ Cassandro.Merola.Picco.17, Cassandro.Merola.Picco.Rozikov.14, Imbrie.82, Imbrie.Newman.88, Johansson.91}. 

In the multidimensional setting ($d\geq 2$), Ginibre, Grossmann, and Ruelle, in \cite{Ginibre.Grossmann.Ruelle.66}, proved the phase transition for $\alpha > d+1$, using an enhanced version of Peierls' argument and the usual contours. Park proposed a different notion of contour for long-range systems in \cite{Park.88.I, Park.88.II}, extending the Pirogov-Sinai theory available for short-range interactions assuming $\alpha > 3d+1$, although he can also consider Potts models with his methods. Some results in the literature suggest that truly long-range effects appear only when $d < \alpha \leq d+1$, see for instance, \cite{Biskup_Chayes_Kivelson_07}. Recently, Affonso, Bissacot, Endo and Handa \cite{Affonso.2021}, inspired by the ideas from Fr{\"o}hlich and Spencer in \cite{FS81, Frohlich.Spencer.82}, introduced a version of multiscale multidimensional contour and proved phase transition by a contour argument in the whole region $\alpha > d$. They can consider long-range Ising models with deterministic decaying fields, first introduced in the context of nearest-neighbor interactions in \cite{Bissacot_Cioletti_10}. For these models, the lack of analyticity of the free energy does not imply phase transition since these models have the same free energy as the models with zero field. It is expected that fields decaying slowly imply uniqueness. In this setting, a contour argument is useful for proofs of phase transitions as well for uniqueness, some papers with models with deterministic decaying fields are \cite{Aoun_Ott_Velenik_23, Bissacot_Cass_Cio_Pres_15, Bissacot.Endo.18, Cioletti_Vila_2016}.

The Random Field Ising model (RFIM) \cite{Imry.Ma.75} is the nearest-neighbor Ising model with an additional external field acting on each site $(h_x)_{x\in\Z^d}$ that is a family of i.i.d. Gaussian random variable with mean 0 and variance 1. Formally, the Hamiltonian of the model is given by
\begin{equation*}
    H(\sigma) = - \sum_{\substack{x,y\in \Z^d \\|x-y|=1}}J\sigma_x\sigma_y  - \varepsilon\sum_{x\in\Z^d}h_x\sigma_x,
\end{equation*}
where $J>0$, $\varepsilon>0$, $\alpha > d$ and $d \geq 1$. A detailed account of the history of the phase transition problem for this model, as well as detailed proofs, was given in \cite{Bovier.06}. Here we present a brief overview.

During the 1980s, the question of the specific dimension where phase transition for the RFIM should happen attracted much attention and was a topic of heated debate. Two convincing arguments were dividing the physics community. One of them, due to Imry and Ma \cite{Imry.Ma.75}, was a non-rigorous application of the Peierls' argument together with the use of the isoperimetric inequality. The key idea of Peierls' argument is to define a notion of contour and calculate the energy cost of "erasing" each contour, i.e., the energy cost of flipping all spins inside the contour. When there is no external field, that energy necessary to flip the spins in a region $A\subset \Z^d$ is of the order of the boundary $|\partial A|$. When we add an external field, we get an extra cost depending on this field. Imry and Ma argued that this cost should be approximately $\sqrt{|A|}$, which is smaller than $|\partial A|$ for all regions only when $d\geq 3$, so this should be the region where phase transition occurs. The other argument, due to Parisi and Sourlas \cite{Parisi.Sourlas.79}, based on dimensional reduction, predicted that the $d$-dimensional RFIM would behave like the $d-2$-dimensional nearest-neighbor Ising model, therefore presenting phase transition only when $d\geq 4$. 

The question was settled by two celebrated papers showing that Imry and Ma's prediction was correct. First, in 1988, Bricmont and Kupiainen \cite{Bricmont.Kupiainen.88} showed that there is phase transition almost surely in $d\geq3$, for low temperatures and variance $\varepsilon$ small enough. Their proof uses a rigorous renormalization group analysis for the short-range case and it is considered involved. Still, they claimed that the result works for any model with a suitable contour representation and centered sub-gaussian external field. Later on, Aizenman and Wehr \cite{Aizenman.Wehr.90} proved uniqueness for $d\leq 2$. For detailed proofs of these results, we refer the reader to \cite{Bovier.06} (see also \cite{Berretti.85, Camia.18, Frohlich.Imbre.84,  Klein.Masooman.97} for more uniqueness results). 

Recently, Ding and Zhuang, see \cite{Ding2021}, provided a simpler proof of the phase transition, not using RGM. And in  \cite{Ding.Liu.Xia.22}, Ding, Liu and Xia proved that if $\beta_c(d)$ is the critical inverse of the temperature of the Ising model with no field, for all $\beta>\beta_c(d)$ there exists a critical value $\varepsilon_0(d, \beta)$ such that the RFIM with $\varepsilon \leq \varepsilon_0$ presents phase transition. 

In the present paper, we are considering a long-range Ising model with a random field, whose Hamiltonian is given formally by
\begin{equation*}
    H(\sigma) = - \sum_{x,y\in \Z^d}J_{xy}\sigma_x\sigma_y - \varepsilon\sum_{x\in\Z^d}h_x\sigma_x,
\end{equation*}
where $J_{xy}=J|x-y|^{-\alpha}$, $J, \varepsilon>0$, $\alpha > d$ and $h_x\in\mathbb{R}$, $d\geq 3$.
Until now, the only known result in the long-range setting is for the one-dimensional long-range Ising model with a random field, by Cassandro, Orlandi, and Picco \cite{Cassandro.Picco.09}. They used the contours of \cite{Cassandro.05} to show the phase transition for the model when $\alpha\in (3-\frac{\ln 3}{\ln 2}, \frac{3}{2})$, under the assumption $J(1) \gg 1$. We stress that, as remarked by Aizenman, Greenblatt, and Lebowitz \cite{Aizenman_Greenblatt_Lebowitz_2012}, although their argument does not work for the whole region for the exponent $\alpha$, the phase transition holds for values close to the critical value $\alpha=3/2$, since by the Aizenman-Wehr theorem we know that there is uniqueness for $\alpha>3/2$.

The argument from Ding and Zhuang in \cite{Ding2021}, for $d\geq3$, involves controlling the probability of a bad event, which is closely related to controlling the quantity $$\sup_{\substack{0\in A\subset\Z^d \\ A \text{ connected }}}\frac{\sum_{x\in A}h_x}{|\partial A|},$$ known as the greedy animal lattice normalized by the boundary. The greedy animal lattice normalized by the size, instead of the boundary, was extensively studied for general distributions of $(h_x)_{x\in\Z^d}$, see \cite{Cox_Gandolfi_Griffin_Kesten_93, Gandolfi_Kesten_94, Hammond_06, Martin_02}. When we normalize by the boundary, an argument by Fisher, Fr\"{o}hlich and Spencer \cite{FFS84} shows that the expected value of the greedy animal lattice is constant. In dimension $d=2$, the expected value is not finite, see \cite{Ding.Wirth.20}. The supremum is taken over connected regions containing the origin since the interiors of the usual Peierls contours are of this form.


For the long-range model, the interior of contours is not necessarily connected. In fact, long-range contours may have considerably large diameters with respect to their size, so their interiors can be very sparse. To avoid this, we define contours, strongly inspired by the $(M,a,r)$-partition in \cite{Affonso.2021}, using a multiscaled procedure that assures that the contours have no cluster with small density.  With them, we generalize the arguments by Fisher-Fr\"{o}hlich-Spencer \cite{FFS84}, and prove that the expected value of the greedy animal lattice is constant, even considering regions not necessarily connected in the supremum. Then, we prove the phase transition for $d\geq 3$. The main result of this paper is the following.
\begin{theorem*}Given $d\geq 3$, $\alpha>d$, there exists $\beta_c\coloneqq\beta(d, \alpha)$ and $\varepsilon_c\coloneqq\varepsilon(d, \alpha)$ such that, for $\beta >\beta_c$ and $\varepsilon\leq \varepsilon_c$, the extremal Gibbs measures $\mu_{\beta, \varepsilon}^+$ and $\mu_{\beta, \varepsilon}^-$ are distinct, that is, $\mu_{\beta, \varepsilon}^+ \neq \mu_{\beta, \varepsilon}^-$ $\mathbb{P}$-almost surely. Therefore the long-range random field Ising model presents phase transition.
\end{theorem*}

This paper is divided as follows. In Section 2, we define the model and the contours, and suitable generalizations to the constructions in \cite{Affonso.2021} are introduced.  In Section 3, we define two bad events of the external field and prove that they occur with a small probability.  In Section 4, we present the proof of the phase transition.
\section{Design Constraints and Requirements}
\label{sec:requirements}
\subsection{Location and Acceptance}
The primary challenge in measuring electron scattering at small angles lies in effectively instrumenting the region near the beampipe while minimizing the material in front of the detector. 

One possibility is to position the FDC behind the crystal ECAL and in front of the backward HCAL, as illustrated in Fig.~\ref{fig:overview}. In the current ePIC design~\cite{managerie}, a potential location is at $z=-307$ cm, which would leave space for a compact calorimeter and about 10 cm gap before the HCAL. 

% Figure environment removed

Figure~\ref{fig:eta_rings} shows that at $z=-307$ cm, the electron beampipe for IP6 has a radius of 4.5 cm, while the hadron beampipe has a radius of 1.8 cm and is shifted 8.3 cm in the $x$ direction. Assuming a 5 mm clearance to the beampipe, similar to the ZEUS BPC~\cite{Surrow:1998su}, a calorimeter with an outer perimeter of $30\times40$ cm$^{2}$ could cover the region $-4.6<\eta<-3.6$ with non-uniform azimuthal coverage. 

The shaded region on the FDC in Fig.~\ref{fig:eta_rings} represents the area where electrons would encounter part of the crystal ECAL or its support structure before reaching the FDC\footnote{We approximate the path of the electron as a straight line and assume that the flat cables servicing the ECAL SiPMs can be arranged to avoid the hole area.}. 
This region, which is a few cm wide, would serve as a veto area.

% Figure environment removed
\newpage
Figure~\ref{fig:FDCposition} shows projections of the possible detector layout, including both the $yz$ and $xz$ views. The ECAL hole is currently assumed to have a height of 14.7 cm and a width of 20.5 cm, taking into account the current version of the ``micro-flange'' (with a cam shape that is 15.2 cm wide and 11.1 cm tall) and a required clearance of 1.8 cm, 3.6 cm, 1.8 cm, and 1.7 cm between the flange and the ECAL's inner support structure on the top, right, bottom, and left sides when looking downstream~\cite{ELKEprivate}. 

% Figure environment removed
\subsection{Dead Material in Front of FDC}
The main challenge faced by a detector located in a high pseudorapidity range is that particles can encounter a significant amount of material as they graze the beampipe walls. While converted electrons and photons can be identified through shower-shape information, reducing the amount of material helps by improving efficiency and minimizing background. 

Figure~\ref{fig:rad_lengths} illustrates the number of radiation lengths of beampipe material encountered by electrons before reaching the FDC, as a function of $\eta$. Within most of the FDC acceptance, the total number of radiation lengths ranges from 0.5 to 1.2. Approximately half a radiation length is contributed by the flange at $z=-120$ cm within the range of $-4.2 < \eta < -3.5$. The significant increase in the total number of $X_0$ traversed at around $\eta=-4.0$ is attributed to the use of aluminum instead of beryllium in the beampipe for $z < -80$ cm.

% Figure environment removed

 The HERA experiments successfully addressed this challenge by implementing a beampipe ``exit window'' made of thin aluminum, which reduced the total dead material to less than 1 $X_{0}$~\cite{Stellberger_2003}. Alternatively, one could use a beryllium section. 

Figure~\ref{fig:rad_lengths} illustrates that incorporating a beryllium section, within the range of $-205 < z < -80$ cm, would effectively decrease the overall material budget in the $-4.7 < \eta < -4.1$ region to below 0.5 $X_{0}$. As an alternative, implementing a 1.5 mm aluminum layer would yield a reduction to less than 1 $X_{0}$ for $\eta > -4.5$.

Another effective method for mitigating the impact of dead material is to use shower shapes to tag converted electrons or photons that originated further upstream in the beampipe or flanges~\cite{Surrow:1998su}. 

\subsection{Acceptance Limit}
To enable accurate FDC measurements at small angles, it is crucial to minimize energy leakage into the beampipe. Figure~\ref{fig:distance_from_beampipe} shows the distance to the beampipe surface as a function of $\eta$.
% Figure environment removed
The target value of $\eta=-4.6$ corresponds to about 18 mm from the electron beampipe at $z=-307$ cm. Assuming a 5 mm clearance, this leaves 13 mm between the edge of the detector and $\eta=-4.6$. For reference, the ZEUS BPC measured 95$\%$ of the energy from 5 GeV electrons at 8 mm from the detector's edge in test beams~\cite{Surrow:1998su}.

\subsection{Energy Range}
The main objective of the FDC is to identify and measure the energy and angle of electrons in the $0.1<Q^{2}<1.0$ GeV$^{2}$ range. Figure~\ref{fig:MinEnergy} illustrates the minimum electron energy as a function of $\eta$ for various $Q^2$ values. The minimum energy required for $-4.6<\eta<-3.6$ falls within the range of 2--13 GeV, whereas the maximum is the beam energy, 18 GeV. Thus, the target energy range is 2--18 GeV.
% Figure environment removed

\subsection{Background Rejection}
\label{sec:bkgrejection}
The main background for inclusive DIS measurements originates from events with small $Q^{2}$, where the scattered electron is not detected, but an electron candidate is observed in the FDC. Figure~\ref{fig:bkg} shows the expected particle spectra obtained using \textsc{Pythia6} to simulate $ep$ scattering without a $Q^2$ cut and without considering detector effects. In the absence of charge tagging, both electrons and positrons from semi-leptonic decays contribute to the background. Similarly, the charged-pion background includes both charges. The photon background primarily arises from neutral-pion decays.
% Figure environment removed

 We estimate the background rejection power of approaches employed at HERA~\cite{ZEUS:1997etp, Surrow:1998su} and the EIC YR. Specifically, we use the far-backward detectors as a veto for photoproduction and select events based on their energy-momentum imbalance\footnote{The $E-p_{z}$ distribution peaks at twice the electron-beam energy when the scattered electron is correctly identified.} with a loose selection of $E - p_{z}>18$ GeV~\cite{AbdulKhalek:2021gbh}. 



 Figure~\ref{fig:bkgratio} illustrates the impact of these cuts on the $e/\pi$ ratio. The far-backward veto has a modest effect due to its small acceptance~\cite{AbdulKhalek:2021gbh}. Similarly, the $E - p_{z}$ cut has a modest impact, since for background events the scattered electron has low energy. Overall, the resulting $e/\pi$ ratio is about $e/\pi\approx$ 0.4 between 1--6 GeV, $e/\pi\approx$ 1 around 10 GeV, and increases rapidly at higher energies. Similarly, the resulting $e/\gamma$ ratio ranges from 0.1 to 1 for $E<6$ GeV and increases at higher energies. The background of positrons and electrons from semi-leptonic decays reaches 10\% at 1 GeV, decreasing to less than 1\% at 10 GeV. Since there is no magnetic field near the FDC, this background cannot be subtracted using reverse-field runs. Instead, Monte-Carlo studies would be employed to estimate and subtract it. Alternatively, electron-isolation criteria might reduce this background further. 

% Figure environment removed

The FDC measurements should aim for a purity greater than 90\% to achieve background uncertainties at the few percent level. A stretch goal of 99\% purity would result in a negligible uncertainty compared to the expected luminosity uncertainty of 1\%~\cite{AbdulKhalek:2021gbh}. Hence, the necessary rejection power falls within the range of 10-25 for $\pi^{\pm}$ and 10-100 for $\gamma$, with a factor of 10 higher for the stretch goal.

The standalone FDC's $\pi^{\pm}$ rejection power will rely on its shower-shape capabilities. Longitudinal segmentation can play a crucial role in discriminating hadrons, as they are more likely to interact deeper within the detector, while electrons tend to exhibit showers starting primarily in the initial layers. Moreover, transverse segmentation also helps, as electrons typically produce narrower and more regular showers compared to hadronic ones. This approach is expected to work well above a few GeV. 

Additional background rejection can be achieved through auxiliary systems, such as a scintillator layer for tagging MIPs to reject unconverted photons, or a timing layer to reject low-energy hadrons. Figure~\ref{fig:time_requirement} demonstrates the potential of TOF and illustrates that a time resolution of about 50~ps would be necessary to achieve a 2$\sigma$ $e/\pi$ separation below 1~GeV.

% Figure environment removed


\section{Design}
\label{design}

The Self-Lock Origami is a rigid origami whose central angles sum to less than 360\degree.
The design consists of four solid adaptable plates connected by four joints (\cref{probFabricMov}C), three of which are passive and one of which will be actuated.
\Cref{probFabricMov}B shows the design and modelling process.
First, square-shaped origami plates of side length 25mm are defined. For modelling purposes, these are treated as having zero thickness.
Then, the central angles of two of the plates are reduced by drawing a line from one corner of the square (\cref{probFabricMov}B) with a specified angle $\alpha$ and extending the line until it intersects with the other edge of the square.
Revolute joints were then defined at the edges of adjacent plates, indicated with edge colour in \cref{probFabricMov}B.
These revolute joints are a first-order approximation to the kinematics of a flexible fold line.


\subsection{Motion Simulation}\label{simulation}

For origami assembly, joint constraints are necessary.
Various shapes are created by cutting angles in different origami plates' positions:
any of the origami's four central angles can be reduced in order to achieve the self-locking property.
Although \cref{probFabricMov} demonstrates only one of these, 16 different configurations were considered: four where the central angle was subtracted from both sides of a single fold line (\cref{L11}), and twelve other configurations shown in the supplemental material.
Note that only cuts reducing the central angles are considered, as an origami could have various shapes around its edges without compromising its motion \cite{zare2021design}.

% Figure environment removed

The structure of the grounded plate 1, connected to plate 2, is equivalent to a simple fold (\cref{motion}a). Therefore, the origami's input angle, $\theta_1$, and input moment are the same as the rotational angle and the output moment of a simple fold structure with the same actuator. Different actuators could be used to increase the input angle limits. However, it is not necessary since the output angle changes most for $\theta_1$ near zero (\cref{motion}).
Both the semi-flat and the MPF state occur at values of $\theta_1$ near $90\degree$.
Thus, in order to obtain the full range of motion, there must be pouch motors on both sides of plates 1 and 2.

Various motion simulations were implemented using the contact solver of Autodesk Inventor 2019 to find an origami that could maintain a flat configuration between its input plates at its initial state and obtain the maximum rotational movement (optimal performance).
Achieving a flat state by the origami's input plates could offer a more simplified model and better control by the pouch motor's actuator.
The pouch motor will be able to maintain a zero pressure state (no inflation) and provide a known shape and dimensions at the initial state for the modelling.

In all simulations, plate 1 is a grounded link at a fixed position with its outer corner at the origin. Gravity is neglected for simplicity and consistency with the analytical model.
The kinematics of the mechanism are simulated by sweeping the input angle between what the ``semi-flat'' and ``maximum possible fold'' (MPF) state.
The semi-flat state is defined as the state where the absolute sum of all the angles between origami plates is at its maximum.
Since the origami models to have zero thickness, they are kinematically capable of achieving fold angles more extreme than a real origami, so the maximum possible fold state is chosen to set the angle $\theta_4$ between the input and output plate to the constant MPF angle $\gamma \coloneqq 36.5\degree$.

\Cref{L11} shows origami types defined by different cutting positions. The green and yellow plates are input and output plates respectively. The ``initial state'' pictured in \cref{L11} shows the closest angle that origami's input plates could get to 180\degree (flat form), and the ``final state'' is the MPF state.
The motion of the outer corners of the input and output plate is shown using markers 1 and 2.

In order to obtain an origami which is symmetrical and also capable of reaching a state with fully flat input plates, origami type a is used in the remainder of the paper.
If the flat state of the input plates is not considered, all the origami types provide almost the same amount of rotational motions in slightly different directions.
This is discussed in the supplemental material (together with an expanded version of \cref{L11}) using an algorithm described previously \cite{zare2021design}.
Additional fold types not pictured were also considered. Certain of these types could also reach a state with flat input plates, but type a is used due to its symmetry, which simplifies modelling.
\section{Simulation}
\label{sec:simulation}

We used the \textsc{DD4HEP} framework~\cite{Frank:2014zya} to run \textsc{Geant4}~\cite{GEANT4:2002zbu} simulations of electrons generated with a uniform azimuthal angle at various $\eta$ points. The simulation does not include any dead material, the effects of which we leave for future work. 
\subsection{Energy Resolution}
Figure~\ref{fig:recon_E} shows the energy resolution, which can be parameterized as 17$\%/\sqrt{E} \oplus 2\%$, and is consistent with the ZEUS BPC data~\cite{Surrow:1998su}. We also compare it to CALICE data, which exhibits improved performance at the expense of a larger Moli\`ere radius. 
% Figure environment removed

Figure~\ref{fig:edge} shows the energy resolution and scale as a function of $\eta$. The performance remains relatively stable for $\eta>-$4.6.
 For particles that hit near the edge of the detector's upstream face ($\eta\approx-4.8$), the energy-scale offset is $-$20\%, and the resolution is about 12\%. One would expect that half the shower would be in the calorimeter (a loss of $50\%$), but because the hole is a cylinder, the center of the shower moves further away from the hole as it goes through the calorimeter.

% Figure environment removed
 
\subsection{Position Resolution}
The polar-angle resolution is determined by considering the position resolution of the FDC and the resolution of the vertex position. The electron vertex position can be precisely determined by tracking other particles in the event using the main detectors, as was done in HERA~\cite{Surrow:1998su,Stellberger_2003}. Thus, only the FDC position resolution is relevant.

We reconstructed the $x$ and $y$ values following the method described in Ref.~\cite{Monteiro:1998bi}:
\begin{equation}
x=\frac{\sum\limits_{i\in v\,layers} w_{X,i} x_i}{\sum\limits_{i\in v\,layers} w_{X,i}}, 
y=\frac{\sum\limits_{i\in h\,layers} w_{Y,i} y_i}{\sum\limits_{i\in h\,layers} w_{Y,i}}
\end{equation}
where the weights $w_{X,i}$ and $w_{Y,i}$ are determined by 
\begin{align}
    w_{X,i}&={\rm max}\left(0, w_0+
{\rm log}\frac{E_i}{\sum_{j\in\rm v\,layers}E_{j}}\right)\\
w_{Y,i}&={\rm max}\left(0, w_0+
{\rm log}\frac{E_i}{\sum_{j\in\rm h\,layers}E_{j}}\right)
\end{align}
The ``h layers'' sums are over layers with horizontally aligned strips and the ``v layers'' sums are over layers with vertically aligned strips. The cutoff parameter $w_{0}$ is set to 4.0.

Figure~\ref{fig:pos_res} shows the position resolution as a function of energy. For energies greater than 1 GeV, the position resolution is better than the strip width divided by $\sqrt{12}$. The resolution we obtained is poorer than the ZEUS BPC resolution. This difference can be partially explained by the smaller strip width (7.9 mm vs 10 mm), but it may also include components from algorithm tuning. %At the proposed position of $z = -307$ cm from the target, this corresponds to a contribution of 1 mrad/$\sqrt{E}$ to the $\theta$ resolution. 
% Figure environment removed


\subsection{Kinematic Variable Reconstruction}
The resolution in Bjorken $x$ and $Q^2$ can be derived from those of the electron energy $\delta E'_e$ and of the electron polar angle $\delta \theta_e$ (using the small-angle approximation):
\begin{align}
    \frac{\delta Q^2}{Q^2}&\approx\frac{\delta E_e'}{E_e'} \oplus  \frac{2}{\pi-\theta_e}\delta \theta_e, \\
    \frac{\delta x}{x}&\approx\frac{1}{y}\frac{\delta E_e'}{E_e'}\oplus \left(\frac{x}{E_e/E_p}-1\right)\frac{2}{\pi-\theta_e} \delta\theta_e
    \label{eq:res_Q2}
\end{align}

In the non-divergent region (\textit{i.e.} $y>0.1$), the $Q^{2}$ resolution ranges from 4\% to 14\% depending on kinematics, whereas the $x$ resolution ranges from 10\% to 50\% with a strong $y$ dependence. To quantify these resolutions in the context of inclusive DIS measurements, we followed the EIC YR approach and calculated the corresponding purity and stability values. In this context, purity is determined by calculating the fraction of events reconstructed in a specific bin that were also generated in that bin (i.e., $P = N_{\mathrm{(rec,gen)}} / N_\mathrm{rec}$). Stability is calculated as the fraction of events generated in a specific bin that were also reconstructed in that bin (i.e., $S = N_{\mathrm{(rec,gen)}} / N_{\mathrm{gen}}$). Here, $N_{\mathrm{(rec,gen)}}$ represents the number of events where the electron is both generated and reconstructed in the same bin. For the generated events, we used the same \textsc{Pythia6} simulation as described in Section~\ref{sec:bkgrejection}.

Figure~\ref{fig:purity} shows the resulting purity and stability plot for events with $-4.6<\eta<-3.6$. The plot has 5 bins per decade in both $x$ and $Q^{2}$, similar to the EIC Yellow Report~\cite{AbdulKhalek:2021gbh}. Purity and stability values above 50\% are observed for the phase-space covered with $y>0.1$, with some degradation at lower values, as expected when using the electron reconstruction method exclusively. It is anticipated that the performance for $0.01<y<0.1$ will improve by combining the electron and hadronic methods, potentially using machine-learning techniques~\cite{Arratia:2021tsq}.

% Figure environment removed

A potential issue that can impact purity and stability is the electron beam's angular divergence, which can reach up to 200 $\mu$rad~\cite{eic_cdr}. Although this limits the kinematic reconstruction for electrons scattered at angles less than 10 mrad~\cite{eic_cdr}, the reconstruction of inclusive kinematic variables will not be compromised since the FDC acceptance begins at 20 mrad.

Overall, these studies demonstrate that the FDC design can provide sufficient resolution for measuring kinematic variables in the low $x$, low $Q^{2}$ region, thereby bridging the $Q^{2}$ gap.

\subsection{Shower-shape Examples}

Figure~\ref{fig:showershape} shows 3D and projection views of example showers. The three scenarios depicted are: an electron reaching the FDC with no material in front of it (left), a photon that initiated showering in the beampipe (middle), and a $\pi^-$. Among the three cases, the electron without pre-showering produces the narrowest shower, while the pre-showering photon and the $\pi^-$ generate more irregular showers.

% Figure environment removed

In terms of converted-electron tagging, the showers are expected to exhibit broader transverse profiles and anomalous longitudinal development. For hadron tagging, the fine segmentation will be primarily used to identify the starting point of the shower, which is more likely to be located at a deeper position compared to electrons. Moreover, the hadron showers will also have a different time development, with hits at later times compared to electron showers. 

This highlights the potential of 5D shower shape analysis for background rejection. In our future work, we intend to explore this potential by applying machine learning techniques to tag electrons/photons, hadron showers, and beam-gas background. Some promising emerging work, which can handle the complex geometry of the FDC, are point-cloud networks~\cite{ATL-PHYS-PUB-2022-040}. 


\section{Summary and Conclusions}
\label{sec:conclusions}
We have outlined the design of a small electromagnetic calorimeter, the Few-Degree Calorimeter (FDC), which is designed to cover the range of $-4.6 < \eta < -3.6$. The primary objective of this detector is to tag electrons within the $Q^2$ range of 0.1 to 1.0 GeV$^2$, thus enabling future research on the transition to perturbative QCD and the gluon-saturation regime.

The FDC design we present here incorporates the latest advancements in SiPM-on-tile calorimetry to create a modern and improved version of the ZEUS Beam Pipe Calorimeter and H1 Very Low $Q^{2}$ calorimeter. The incorporation of high-granularity 5D shower measurements (position, time, and energy) offered by this technology holds great potential for background tagging.

In conclusion, this document presents the first design that has the potential to close the EIC $Q^2$ gap while maintaining a compact and cost-effective solution. Considering the larger crossing-angle envisioned for the second-interaction region at the EIC, which results in a larger hole in the crystal ECAL acceptance, this design may offer further opportunities for optimization for the EIC Detector 2.




\section*{Acknowledgements} \label{sec:acknowledgements}
We would like to extend our sincere appreciation to the members of the California EIC consortium for their valuable feedback on the design and physics that motivated the FDC, with special recognition to Oleg Tsai, Farid Salazar, and Zhongbo Kang. Additionally, we are grateful to Elke-Caroline Aschenauer for her guidance regarding the current acceptance of the crystal ECAL in ePIC. 

This work was supported by MRPI program of the University of California Office of the President, award number 00010100. This work was supposed by DOE grant award number DE-SC0022324. S.P also acknowledges support from the Jefferson Lab EIC Center Fellowship. M.A acknowledges support through DOE Contract No. DE-AC05-06OR23177 under which Jefferson Science Associates, LLC operates the Thomas Jefferson National Accelerator Facility.
\renewcommand\refname{Bibliography}
\bibliographystyle{utphys} % or try abbrvnat or unsrtnat
\bibliography{bibio.bib} % refers to example.bib

\appendix
\end{document}