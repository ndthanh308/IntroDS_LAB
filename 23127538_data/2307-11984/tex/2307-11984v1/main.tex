\documentclass[10pt,twocolumn,letterpaper]{article}

\usepackage{iccv}
\usepackage{times}
\usepackage{epsfig}
\usepackage{graphicx}
\usepackage{amsmath}
\usepackage{amssymb}
\newcommand{\rednote}[1] {{\color{red}$\blacktriangleright${#1}$\blacktriangleleft$}}

%---------------------%
%       References    %
%---------------------%
\newcommand{\fig}[1]{Fig.~\ref{#1}}
\newcommand{\sect}[1]{Sec.~\ref{#1}}
\newcommand{\apd}[1]{Appendix~\ref{#1}}
\newcommand{\eq}[1]{(\ref{#1})}

%

\newtheorem{theorem}{Theorem}
\newtheorem{definition}{Definition}
\newtheorem{corollary}{Corollary}
\newtheorem{lemma}{Lemma}
\newtheorem{proof}{Proof}

% Symbols definitions

\newcommand{\Real}[1]{\Re \left \{ #1\right \}}
\newcommand{\en} {E}
\newcommand{\td}[1] {\tilde{#1}}
\newcommand{\lt}[1] {{\td{\lambda}}_{#1}}
\newcommand{\bl}[1] {\text{\boldmath ${\lambda}$}_{#1}}
\newcommand{\blt}[1] {\text{\boldmath $\td{\lambda}$}_{#1}}
\newcommand{\vg}[1] {{\mbox{{\boldmath ${#1}$}}}}
\newcommand{\vgs}[2] {\vg{#1}_{#2}}

\newcommand{\fourier}[1]{\mathcal{F} \left [ #1\right ]}
\newcommand{\invfourier}[1]{\mathcal{F}^{-1} \left [ #1\right ]}

\newcommand{\PX}[1] {{\mathbb{P}}\left\{{#1}\right\}}
\newcommand{\EX}[1] {{\mathbb{E}}\left\{{#1}\right\}}
\newcommand{\EXs}[2] {{\mathbb{E}}_{{#1}}\!\!\left\{{#2}\right\}}
\newcommand{\Var}[1] {{\text{Var}}\left ({#1}\right )}

\newcommand{\pX}[1] {{\mathbf{p}}\left\{{#1}\right\}}


\newcommand{\Hone} {\mathcal{H}_1}
\newcommand{\Hzero}{\mathcal{H}_0}
\newcommand{\hatHone} {\hat{\mathcal{H}}_1}
\newcommand{\hatHzero}{\hat{\mathcal{H}}_0}

\newcommand{\Mu} {\mathcal{M}_u}
\newcommand{\Mc}{\mathcal{M}_c}

\newcommand{\boldA} {{\bf{A}}}
\newcommand{\boldg} {{\bf{g}}}
\newcommand{\bolds} {{\bf{s}}}
\newcommand{\boldf} {{\bf{f}}}
\newcommand{\bolda} {{\bf{a}}}
\newcommand{\boldb} {{\bf{b}}}
\newcommand{\boldp} {{\bf{p}}}
\newcommand{\bolde} {{\bf{e}}}
\newcommand{\boldk} {{\bf{k}}}
\newcommand{\boldK} {{\bf{K}}}
\newcommand{\boldu} {{\bf{u}}}
\newcommand{\boldc} {{\bf{c}}}
\newcommand{\boldV} {{\bf{V}}}
\newcommand{\boldX} {{\bf{X}}}
\newcommand{\boldY} {{\bf{Y}}}
\newcommand{\boldW} {{\bf{W}}}
\newcommand{\boldU} {{\bf{U}}}
\newcommand{\boldE} {{\bf{E}}}
\newcommand{\boldJ} {{\bf{J}}}
\newcommand{\boldH} {{\bf{H}}}
\newcommand{\boldm} {{\bf{m}}}
\newcommand{\boldP} {{\bf{P}}}
\newcommand{\boldF} {{\bf{F}}}
\newcommand{\boldG} {{\bf{G}}}
\newcommand{\boldR} {{\bf{R}}}
\newcommand{\boldC} {{\bf{C}}}
\newcommand{\boldB} {{\bf{B}}}
\newcommand{\boldD} {{\bf{D}}}
\newcommand{\boldLambda} {{\bf{\Lambda}}}
\newcommand{\boldq} {{\bf{q}}}

%\newcommand{\boldT} {{\bf{T}}}
%\newcommand{\boldF} {\bf{F}}
\newcommand{\boldI} {{\bf{I}}}
\newcommand{\boldr} {{\bf{r}}}
\newcommand{\meanr} {{\overline{r}}}
\newcommand{\meanboldr} {{\overline{\bf{r}}}}
\newcommand{\boldn} {{\bf{n}}}
\newcommand{\boldx} {{\bf{x}}}
\newcommand{\boldy} {{\bf{y}}}
\newcommand{\boldh} {{\bf{h}}}
\newcommand{\boldz} {{\bf{z}}}
\newcommand{\boldw} {{\bf{w}}}
\newcommand{\boldv} {{\bf{v}}}

\newcommand{\boldt} {{\bf{t}}}
\newcommand{\meanw} {{\overline{w}}}
\newcommand{\meanboldw} {{\overline{\bf{w}}}}
\newcommand{\boldd} {{\bf{d}}}
\newcommand{\boldalpha} {\bf{\alpha}}
\newcommand{\boldbeta} {\bf{\beta}}
\newcommand{\boldgamma} {\bf{\gamma}}
\newcommand{\boldrho} {\bf{\rho}}
\newcommand{\boldhc} {\bf{h}_{\text{c}}}

\newcommand{\Pd} {P_{\text{d}}}
\newcommand{\Pf} {P_{\text{f}}}

\newcommand{\Pb} {P_{\text{b}}}

\newcommand{\Rb} {R_{\text{b}}}
\newcommand{\Ep} {E_{\text{p}}}


\newcommand{\Tp} {T_{\text{p}}}
\newcommand{\Td} {T_{\text{d}}}
\newcommand{\fc} {f_{\text{c}}}
\newcommand{\ts} {t_{\text{s}}}
\newcommand{\Ta} {T_{\text{a}}}
\newcommand{\Ti} {T_{\text{i}}}
\newcommand{\Np} {N_{\text{p}}}
\newcommand{\Nps} {N_{\text{ps}}}
\newcommand{\tp} {\tau_{\text{p}}}
\newcommand{\Es} {E_{\text{s}}}
\newcommand{\Eb} {E_{\text{b}}}
\newcommand{\Ts} {T_{\text{s}}}
\newcommand{\Tf} {T_{\text{f}}}
\newcommand{\Tc} {T_{\text{c}}}
\newcommand{\Th} {T_{\text{h}}}
\newcommand{\Tb} {T_{\text{b}}}
\newcommand{\Tob} {T_{\text{ob}}}
\newcommand{\Nc} {N_{\text{c}}}
\newcommand{\Ns} {N_{\text{s}}}
\newcommand{\Na} {N_{\text{A}}}

\newcommand{\Nsym} {N_{\text{sym}}}
\newcommand{\tint} {T_{\text{int}}}
\newcommand{\TX}[1] {{\mathbb{T}}\left [{#1}\right ]}
\newcommand{\Prob}[1] {\text{P}\left\{{#1}\right\}}
\newcommand{\Q}[1] {Q \left ( #1 \right )}
\newcommand{\Nch} {N_{\text{ch}}}
\newcommand{\Lp} {L_{\text{p}}}
\newcommand{\dref} {d_{\text{ref}}}
\newcommand{\wref} {w_{\text{ref}}}
\newcommand{\Wref} {W_{\text{ref}}}
\newcommand{\Href} {H_{\text{ref}}}
\newcommand{\ZA} {Z_{\text{A}}}
\newcommand{\taup} {\tau_{\text{f}}}
%\newcommand{\etaup} {\hat{\tau}_{\text{f}}}
\newcommand{\taud} {{\tau_{\text{d}}}}
\newcommand{\etaud} {\hat{\tau}_{\text{d}}}
\newcommand{\toa} {\tau}
\newcommand{\etoa} {\hat{\tau}}
\newcommand{\Beff} {B_{\text{eff}}}

\newcommand{\Pric} {P_{\text{r}}}
\newcommand{\Thetai} {{\bf \Theta}^{\text{inc}}}
\newcommand{\Thetar} {{\bf \Theta}^{\text{ref}}}
\newcommand{\Thetat} {{\bf \Theta}^{\text{t}}}
\newcommand{\Prc}{P_{r}^{metal\,can}}
\newcommand{\Prw}{P_{r}^{bottle\,water}}


\newcommand{\floor}[1] {f \left ({#1} \right )}
%\newcommand{\rect}[1] {\text{rect} \left ({#1} \right )}
\newcommand{\sinc}[1] {\text{sinc} \left ({#1} \right )}

\def\dsp{\displaystyle}

\def\rect{{\text{rect}}}
\def\erfc{{\text{erfc}}}
\def\erf{{\text{erf}}}
\def\inve{{\text{inverfc}}}
\def\teq{\triangleq}
\def\bs{$\blacksquare$}
%\def\sinc{{\text{sinc}}}
\newcommand{\tr}[1]{{\rm tr} \left ( #1 \right ) }
\newcommand{\rank}[1]{{\rm rank} \left ( #1 \right )}
\newcommand{\diag}[1]{{\rm diag} \left ( #1 \right )}

\newcommand{\Cfunc}[1]{{C^{(\text{#1})}}}
\newcommand{\cvect}[1]{{\mathbf{c}^{(\text{#1})}}}
\newcommand{\ucvect}[1]{{\underline{{\mathbf{c}}}^{(\text{#1})}}}
\newcommand{\epsilonvect}[1]{{\mathbf{\varepsilon}^{(\text{#1})}}}
\newcommand{\uchat}[1]{{\underline{\hat{\mathbf{c}}}^{(\text{#1})}}}
\newcommand{\chat}[1]{{\hat{\mathbf{c}}^{(\text{#1})}}}

\newcommand{\dEve} {d_{\text{Eve}}}
\newcommand{\NEve} {N_{\text{Eve}}}
\newcommand{\MEve} {M_{\text{Eve}}}

\newcommand{\Dset} {\mathcal{D}}
\newcommand{\GP}[1] {\mathcal{GP}\left ( #1 \right )}

\newcommand{\argmax}[1]{\underset{{#1}}{\operatorname{argmax}}}
\newcommand{\argmin}[1]{\underset{{#1}}{\operatorname{argmin}}}


% operazioni
\newcommand{\convZ}{*}
\newcommand{\conjZ}{^{\dag}}
\newcommand{\argmaxZ}[1]{\operatorname*{argmax}_{#1}}
\newcommand{\argminZ}[1]{\operatorname*{argmin}_{#1}}
\newcommand{\sincZ}{\text{sinc}}

% operatori, trasformate, funzioni lineari
\newcommand{\EXZ}[1] {{\mathbb{E}}\left\{{#1}\right\}}
\newcommand{\EXZBig} {\mathbb{E}}
%\newcommand{\V}[1] {\left|#1\right|}


\newcommand{\0}{\mathbf{0}}

% insiemi

\newcommand{\RZ}{\mathbb{R}^2}
%\newcommand{\C}{\mathbb{C}}
\newcommand{\A}{\mathcal{A}}
\newcommand{\Nset}{\mathcal{N}}
\newcommand{\Vset}{\mathcal{V}}

\newcommand{\rc}{r_{\text{c}}}
\newcommand{\Pe}{P_{\text{e}}}
%\newcommand{\SNR}{\mathsf{SNR}}
\newcommand{\SNR}{\text{SNR}}

\newcommand{\bx} {{\bf{x}}}
\newcommand{\bX} {{\bf{X}}}
\newcommand{\bW} {{\bf{W}}}
\newcommand{\bw} {{\bf{w}}}
\newcommand{\bY} {{\bf{Y}}}
\newcommand{\boldeta} {{\boldsymbol{\eta}}}



\newcommand{\Nr} {N_{\text{R}}}
\newcommand{\Nt} {N_{\text{T}}}
\newcommand{\Nx} {N_{\text{X}}}
\newcommand{\Ny} {N_{\text{Y}}}
\newcommand{\Nmin} {N_{\text{min}}}
\newcommand{\Lr} {L_{\text{R}}}
\newcommand{\Lt} {L_{\text{T}}}


\newcommand{\alp}{\alpha_{nm}}
\newcommand{\CRLB}{\text{CRLB}}
\newcommand{\K}{\text{K}}
\newcommand{\Banda}{B_\text{eff}}
\newcommand{\Bnm}{\beta_\text{nm}}
\newcommand{\Ptx} {P_{\text{T}}}
\newcommand{\Prx} {P_{\text{R}}}
\newcommand{\Psigma}{P_{\sigma_x}}
\newcommand{\fstart}{f_{\text{start}}}
\newcommand{\fstop}{f_{\text{stop}}}
\newcommand{\Gt} {G_{\text{T}}}
\newcommand{\Gr} {G_{\text{R}}}
\newcommand{\Gc} {G_{\text{C}}}
\newcommand{\Glis} {G_{\text{LIS}}}
\newcommand{\Gap} {G_{\text{AP}}}
\newcommand{\Gscm} {G_{\text{SCM}}}
\newcommand{\Fap} {F_{\text{AP}}}
\newcommand{\fb} {f_{\text{b}}}
\newcommand{\Fscm} {F_{\text{SCM}}}
\newcommand{\Pn} {P_{\text{N}}}
\newcommand{\pBS}{\textbf{p}_\text{BS}}
\newcommand{\phix} {\phi_{\text{inc}}}
\newcommand{\phiy} {\phi_{\text{Y}}}
\newcommand{\phiz} {\phi_{0}}
\newcommand{\phit} {\phi_{\text{T}}}
\newcommand{\phir}{ \phi_{\text{R}}}
\newcommand{\pu}{\mathbf{p}_\text{u}}
\newcommand{\thetax} {\theta_{\text{inc}}}
\newcommand{\thetak}{\theta^{(k)}}
\newcommand{\Thetax} {\Theta_{\text{inc}}}
\newcommand{\thetay} {\theta_{\text{Y}}}
\newcommand{\thetaz} {\theta_{0}}
\newcommand{\thetat} {\theta_{\text{T}}}
\newcommand{\thetar}{ \theta_{\text{R}}}
\newcommand{\Gam}{\Gamma_{nm}}
\newcommand{\znm}{Z_{\text{nm}}(f)}
\newcommand{\deltaT}{\Delta_\text{T}}
\newcommand{\W}{\text{W}}
\newcommand{\Thetak}{\Theta^{(k)}}
\newcommand{\nk}{n^{(k)}}
\newcommand{\sk}{s^{(k)}}
\newcommand{\vk}{v^{(k)}}
\newcommand{\wk}{w^{(k)}}
\newcommand{\xk}{x^{(k)}}
\newcommand{\yk}{y^{(k)}}
\newcommand{\h}{h_{nm}^{(k)}}
\newcommand{\g}{g_{nm}^{(k)}}
\newcommand{\reflec}{r_{nm}^{(k)}}
\usepackage{graphicx}
\usepackage{color}
\usepackage{bbm}
\usepackage{bbding}
\usepackage{multirow}
\usepackage{algorithm}
\usepackage{algorithmic}
\usepackage{amsmath}
\usepackage{booktabs} % fors professional tables
\newcommand{\topline}{\toprule [0.1em]}
\newcommand{\midline}{\midrule [0.05em]}
\newcommand{\bottomline}{\bottomrule [0.1em]}
\usepackage{colortbl} 
\usepackage{xcolor}
\usepackage{xspace}
\usepackage{bigstrut}
\usepackage{dsfont}
\usepackage{pifont}
\usepackage{subcaption}


\def\todo{\textcolor{green}{TODO}}
\usepackage{array}   
\definecolor{mygray}{rgb}{0.9,0.9,0.9}
\newcommand\filledcirc[1]{{\color{#1}\bullet}\mathllap{\color{#1}\circ}}
\newcommand{\startsq}{%
  \setlength{\fboxsep}{0pt}%
  \setlength{\fboxrule}{1.2pt}%
  \raisebox{0.6pt}[0pt][0pt]{\fcolorbox{blue}{white}{\color{white}{o}}}
}
\usepackage{caption} 
\captionsetup[table]{skip=5pt}

% It is strongly recommended to use hyperref, especially for the review version.
% hyperref with option pagebackref eases the reviewers' job.
% Please disable hyperref *only* if you encounter grave issues, e.g. with the
% file validation for the camera-ready version.
%
% If you comment hyperref and then uncomment it, you should delete
% ReviewTempalte.aux before re-running LaTeX.
% (Or just hit 'q' on the first LaTeX run, let it finish, and you
%  should be clear).
% \usepackage[pagebackref,breaklinks,colorlinks]{hyperref}
\usepackage[pagebackref=true,breaklinks=true,letterpaper=true,colorlinks,bookmarks=false]{hyperref}

% Support for easy cross-referencing
\usepackage[capitalize]{cleveref}
\crefname{section}{Sec.}{Secs.}
\Crefname{section}{Section}{Sections}
\Crefname{table}{Table}{Tables}
\crefname{table}{Tab.}{Tabs.}
% Include other packages here, before hyperref.

% If you comment hyperref and then uncomment it, you should delete
% egpaper.aux before re-running latex.  (Or just hit 'q' on the first latex
% run, let it finish, and you should be clear).


\iccvfinalcopy % *** Uncomment this line for the final submission

\def\httilde{\mbox{\tt\raisebox{-.5ex}{\symbol{126}}}}

% Pages are numbered in submission mode, and unnumbered in camera-ready
\ificcvfinal\pagestyle{empty}\fi

\begin{document}

%%%%%%%%% TITLE
\title{Learning Vision-and-Language Navigation 
                from YouTube Videos}

\author{
    Kunyang Lin\textsuperscript{\rm 1 2}\thanks{Equal contribution. Email: \{imkunyanglin, phchencs\}@gmail.com} ~~ 
    Peihao Chen\textsuperscript{\rm 1}\footnotemark[1] ~~ 
    Diwei Huang\textsuperscript{\rm 1} ~ 
    Thomas H. Li\textsuperscript{\rm 6} ~
    Mingkui Tan\textsuperscript{\rm 1 \rm 5}\thanks{Corresponding author. Email: mingkuitan@scut.edu.cn} ~
    Chuang Gan\textsuperscript{\rm 3 \rm 4} \\
    \textsuperscript{\scriptsize{\rm 1}}\small{South China University of Technology,}
    \textsuperscript{\scriptsize{\rm 2}}\small{Information Technology R\&D Innovation Center of Peking University,}\\
    \textsuperscript{\scriptsize{\rm 3}}\small{UMass Amherst,}
    \textsuperscript{\rm 4}\small{MIT-IBM Watson AI Lab,}
    \textsuperscript{\rm 5}\small{Key Laboratory of Big Data and Intelligent Robot, Ministry of Education,} \\
    \textsuperscript{\scriptsize{\rm 6}}\small{Peking University Shenzhen Graduate School} 
}


\maketitle
% % Remove page # from the first page of camera-ready.
\ificcvfinal\thispagestyle{empty}\fi


%%%%%%%%% ABSTRACT
\begin{abstract}

Vision-and-language navigation (VLN) requires an embodied agent to navigate in realistic 3D environments using natural language instructions. 
Existing VLN methods suffer from training on small-scale environments or unreasonable path-instruction datasets, limiting the generalization to unseen environments.
There are massive house tour videos on YouTube, providing abundant real navigation experiences and layout information. However, these videos have not been explored for VLN before.
In this paper, we propose to learn an agent from these videos by creating a large-scale dataset which comprises reasonable path-instruction pairs from house tour videos and pre-training the agent on it.
To achieve this, we have to tackle the challenges of automatically constructing path-instruction pairs and exploiting real layout knowledge from raw and unlabeled videos.
To address these, we first leverage an entropy-based method to construct the nodes of a path trajectory. Then, we propose an action-aware generator for generating instructions from unlabeled trajectories. Last, we devise a trajectory judgment pretext task to encourage the agent to mine the layout knowledge. Experimental results show that our method achieves state-of-the-art performance on two popular benchmarks (R2R and REVERIE). Code is available at \small{\url{https://github.com/JeremyLinky/YouTube-VLN}}
\end{abstract}
\vspace{-2em}

%%%%%%%%% BODY TEXT
\section{Introduction}
\label{sec:intro}
An important goal of embodied artificial intelligence is to develop agents that can interact with humans in natural language to carry out real-world tasks.
Toward this goal, vision-and-language navigation (VLN)~\cite{VLN} is a rudimentary artificial intelligence task, requiring an indoor agent to navigate in unseen environments following natural instructions.
VLN has attracted widespread attention in the fields of computer vision and robotics due to its promising applications such as in-home robots~\cite{HomeRoboticsApplication} and warehouse assistants~\cite{liang2015automated}.

One of the key challenges of VLN is the generalization ability of agents to unseen environments. 
Existing VLN methods attempt to cope with this challenge via self-supervised pre-training on vision-and-language datasets. As shown in Figure~\ref{fig:teaser}~(a), some previous works~\cite{Prompt, Transferable, Improving, AutoVLN} learn the agents on simulated navigation environments and manual-labeled data. The other works~\cite{Generic, Airbert, HOP} seek to construct path-instruction pairs by using web image data, which is shown in Figure~\ref{fig:teaser}~(b). Despite their promising performance, existing agents still suffer from the following limitations. 1) Training on simulated datasets is limited to a restricted number of environments. 2) Constructing a trajectory by simply concatenating web images leads to unreasonable room layouts, which hamper the agent to learn layout reasoning ability. As a result, VLN agents trained on such data are brittle to adapt to unseen environments.


% Figure environment removed


Fortunately, there are massive house tour videos on YouTube, providing real navigation experiences and layout information but are still under-explored. We can be naturally inspired to let an agent learn VLN ability from such videos, thereby addressing the limitations of existing methods. An intuitive way is to model the navigation experiences as path-instruction pairs to train the agent.
Motivated by this, we propose a ``\textbf{Lily}'' agent who \textbf{L}\textit{earns V}\textbf{i}\textit{sion-and-}\textbf{L}\textit{anguage}\textit{ Navigation from }\textbf{Y}\textit{ouTube Videos}. Specifically, we first develop an in-domain pre-training dataset from house tour YouTube videos, namely \textbf{YouTube-VLN}, which comprises VLN-like path-instruction pairs. Our YouTube-VLN dataset has the advantages of diverse environments, real layouts, and native actions\footnote{The execution actions that objectively exist}, reducing the domain gap with VLN datasets, as illustrated in Figure~\ref{fig:teaser} (c). Then, we pre-train the agent using these path-instruction pairs. Benefiting from in-domain pre-training on our proposed dataset, our agent thus generalizes well to unseen environments.



Constructing and utilizing such a dataset, however, is still far from trivial work and remains an open problem due to the following challenges. 1) As the nodes in a trajectory are expected to be diverse and informative, it is hard to determine the locations of trajectory nodes from massive video frames and represent the visual content in a node. 2) Real VLN instructions include various action descriptions, but obtaining corresponding instructions from navigation clips is challenging due to the actions being implicit in videos. Thus it is nontrivial to acquire matching instruction on a trajectory. 3) Layout knowledge from real navigation experience is hard to mine and model, which impedes the agent of learning layout reasoning ability. 

In this paper, we address the above challenges as follows. To conquer challenge 1), we propose an entropy-based trajectory generation method. Specifically, we first envisage that the nodes of a trajectory should contain as many types of rooms as possible to diversify trajectories. Accordingly, we group the frames with the same room types in videos and consider each group as a node in the trajectory. Then, inspired by that low classification entropy image is reliable and contains rich information relevant to a specific class (room type in our case)~\cite{EATA}, the frame with the lowest classification entropy in a group is chosen to represent the visual content in a node.
To tackle challenge 2), we introduce an action-aware instruction generation method. Specifically, we adopt an action inverse model to pseudo-label the action along trajectories and fill them in the instructions via hand-designed rules.
To grapple with challenge 3), we devise a self-supervised pretext task. As we all know, humans often judge whether a navigation trajectory is reasonable based on the layout of the environment. Therefore, it is believed that an agent equipped with layout reasoning ability should be able to make similar judgment. To this end, we propose \textbf{trajectory judgment} pretext task to ask the agent to identify reasonable navigation trajectories, which further equips the model with the ability to reason environment layouts.
%

We empirically show that the diverse entropy-based  trajectory generation method and action-aware instruction generator allow us to harvest high-quality path-instruction pairs from YouTube house tour videos, resulting in the YouTube-VLN dataset. By integrating the self-supervised trajectory judgment task in pre-training a VLN agent, our Lily agent presents state-of-the-art performance on two mature and solid benchmarks (R2R~\cite{VLN}, REVERIE~\cite{REVERIE}). The proposed Lily agent reaches the first place on the R2R leaderboard in terms of success rate and outperforms the SOTA method under discriminative setting and generative setting by 2\% and 3\% \textit{w.r.t.} success rate, respectively.

Our main contributions are as follows:

$\bullet$ We unleash the huge potential of house tour videos for VLN. By leveraging these videos, we introduce a large-scale dataset containing real navigation path-instruction pairs for promoting VLN pre-training and a self-supervised pretext task for the learning of layout reasoning.

$\bullet$ Our diverse trajectory generation method, together with the action-aware instruction generator, creates informative and diverse trajectory nodes and produces matching instructions, both of which make the path-instruction pairs authentic and of high quality for training a VLN agent.

$\bullet$ The proposed trajectory judgment pretext task allows the model to build up an awareness of learning and reasoning the layout knowledge, which is crucial in the VLN task of indoor environments. We also empirically substantiate that the agent indeed learns the layout learning ability. 
    


\section{Related Work}

\subsection{Vision-and-Language Navigation}
Vision-and-Language Navigation (VLN)~\cite{VLN} is a challenging task and has received continuous and intense attention from the academic community in recent years~\cite{Weakly-Supervised, Room-and-Object, Cross-Modal, Contrastive-PI, One-Step, CLEAR, DUET, Outdoor, ding2022embodied}. Early methods attempt to learn the agent from sequence-to-sequence models~\cite{VLN, Speaker-Follower, EnvDrop}. However, these methods can not model the cross-modal relation between language and visual observation well. To address this issue, transformer~\cite{Transformer} architecture is adopted to the agents followed by vision-and-language pre-training~\cite{SMNA, RCMS, AuxRN, VLNBert, Airbert, HOP, HAMT, SOAT, SEA}. PREVALENT~\cite{image-text-action} pre-trains transformer-based agent via masked language modeling and action prediction tasks. Inspired by BERT~\cite{BERT}, several works propose to use different variants of BERT for VLN pre-training. VLN-BERT~\cite{VLNBert} utilizes image-text data~\cite{ViLBERT} to perform path-instruction matching pretext task. Airbert~\cite{Airbert} proposes the shuffle loss to improve the ability of the model to learn the order of image-caption pairs. Recently, HOP~\cite{HOP} introduces the history and order-aware pretext tasks. However, these existing pretext tasks do not consider the learning of environment layout reasoning ability and bring limited performance for VLN tasks. In this work, we propose a trajectory judgment task to teach the agent to distinguish reasonable navigation trajectories. By proficiently accomplishing this task, the agent can acquire the ability to reason about environment layout, which enhances its generalization capability to unseen environments.

\subsection{Datasets for VLN}
The major difficulty of generalizing a VLN agent to unseen environments is the scarcity of VLN training data. Well-labeled VLN datasets built from the simulators~\cite{MP3d, HMP3D, Gibson} allow the agent to obtain a promising performance, such as R2R~\cite{VLN}, R4R~\cite{R4R}, RxR~\cite{RxR} and SOON~\cite{SOON}. While the data built from simulators are laborious, Wang \textit{et al.}~\cite{Less-is-More} and ProbES~\cite{Prompt} enrich the navigation instructions via self-exploration in the simulation environments. Some other endeavors~\cite{EnvDrop, Envedit, Speaker-Follower, APS, Pathdreamer, SIVLN} seek to augment the data from existing datasets. AutoVLN~\cite{AutoVLN} and Kamath \textit{et al.}~\cite{ANewPath} enlarges the VLN data from simulations with a larger number of environments. However, these datasets are limited by the number of scenes in simulators. To ease this problem, VLN-BERT~\cite{VLNBert} leverages the abundant web image-captions pairs as VLN pre-training data. Airbert~\cite{Airbert} further exploits indoor house images and captions from the web to construct path-instruction pairs for VLN pre-training. However, the trajectories constructed by simply splicing pictures may be confusing and ambiguous. In our work, we address these problems by proposing a large-scale in-domain VLN-like pre-training dataset, providing the agent with diverse visual environments and reasonable layouts.



\section{Building VLN Dataset from YouTube Videos}
\label{sec:dataset}

Our first step is to develop a large-scale VLN-like dataset that comprises reasonable path-instruction pairs from house tour videos on YouTube, termed YouTube-VLN. YouTube-VLN serves as a cornerstone for facilitating the acquisition of VLN capabilities for Lily, featuring diverse environments, real layouts and native actions. To achieve this goal, we present an entropy-based trajectory generation technique~(Section~\ref{sec:path_gen}) and an action-aware generator~(Section~\ref{sec:ins_gen}) to tackle the arduous tasks of trajectory and instruction generation, respectively.


\subsection{Diverse Trajectory Generation.}
\label{sec:path_gen}
We seek to construct discrete navigation trajectories from YouTube videos. Similar to discrete navigation datasets (\eg R2R~\cite{VLN}), each trajectory comprises $K$ navigation nodes, representing different locations of a navigation path. 
This entails addressing two major challenges:
1) how to determine the locations composing a trajectory to make the trajectory more diverse and
2) how to represent the visual content at each node location.
To tackle these challenges, we first collect large-scale consecutive indoor frames from YouTube videos. Then, we group the adjacent frames according to their room types and consider each group as a node. Last, we present an entropy-based technique to select the most informative frames in a group for representing the visual content in a node.

% Figure environment removed

\textbf{Collecting Navigation Data from YouTube.}
\label{sec:Collecting_In-door_Images}
Real estate agents typically tour a house in each video. To satisfy the visual diversity and dataset scale, we create the YouTube-VLN dataset from 4078 videos collected from various uploaders, with a total duration of 433 hours. In contrast to prior work~\cite{YTb}, which relied on a limited set of videos from a single uploader, our dataset features greater diversity and volume. We also employ sparse sampling and off-the-shelf image classifiers~\cite{Mask_RCNN, CLIP} to pre-process the videos, filtering out redundant or noisy frames (those featuring people or outdoor scenes), resulting in a final set of 587k indoor frames suitable for constructing trajectories.

\textbf{Determining the Locations of Trajectory Nodes.}
\label{sec:Grouping-Nodes}
A real robot often needs to go through different locations for navigating to a goal. To mimic the real navigation process, we expect that our constructed trajectories also contain diverse visual content within the limited navigation nodes. To achieve this, we first utilize the powerful large model CLIP~\cite{CLIP} to recognize the room type of each indoor image. Then, we gather temporally adjacent frames with the same room type as a group and consider this group as one of the navigation nodes. In this sense, the navigation nodes are diversely spread in different rooms and the constructed trajectories are able to mimic the real navigation process. We also call this kind of node a \textit{room node}. In practice, to increase the visual diversity of trajectories, we also randomly insert \textit{transition nodes} that are composed of video frames captured during the transition from one room to another one, between two adjacent room nodes.


\textbf{Representing Visual Content in a Node.}
\label{sec:landmarks}
A node consists of a group of images and sometimes the number of images may exceed 100 as the photographer could stay in the same room for a long time. Hence, we have to select the most informative images for representing node features. Inspired by EATA~\cite{EATA}, an image with lower classification entropy is more reliable, containing more information relevant to a specific class (room type in our case). We thus propose to select an image with the lowest classification entropy to represent the current view of a node. In order to mimic the panoramic visual context, we then merge $M$ adjacent consecutive images of the current view. It is worth noting that our node features better represent a panoramic view compared with Airbert~\cite{Airbert} because we merge adjacent frames that belong to the same place as the current view.

Ultimately, we randomly chose $K$ continuous nodes to construct a trajectory. An example of the constructed trajectory is shown in Figure~\ref{fig:merge_strategy}.


\subsection{Action-Aware Instruction Generation}
\label{sec:ins_gen}


In addition to constructing navigation trajectories, one more important step for building a VLN dataset is to create the corresponding instructions without manual annotation.
The main challenge for this step is how to correctly describe visual content and actions along navigation paths.
To conquer this challenge, we first generate instruction templates with verb and noun phrase blanks. Then, we describe each node in trajectories using the CLIP~\cite{CLIP} model and infer the native actions using an action inverse model~\cite{YTb}. To generate the final instructions, we fill the instruction templates with these visual descriptions and actions.

Specifically, we first generate templates with verb and noun blanks from instructions in the R2R dataset following Airbert~\cite{Airbert}.
For noun blanks, we fill them with visual content descriptions about each node. We select the frame with the lowest classification entropy (as described in Sec.~\ref{sec:landmarks}) and use the CLIP model to infer the objects it contains, together with the room type to populate a noun blank.
For verb blanks, the existing instruction generation method~\cite{Airbert} is unable to fill them with the correct action words because it cannot figure out the actions taken for navigating from one image to another. This makes the agent confused when it observes similar viewpoints transition but is given different action descriptions.
To tackle this problem, we propose an action-aware strategy to fill instruction templates with native actions instead of random inconsistent actions. To be specific, we follow~\cite{YTb} to train an action inverse model, which has 96\% prediction accuracy for predicting native actions, to pseudo-label the trajectory with action labels from one location node to another. The predicted actions are then converted into actionable verbs,~\ie ``go forward'', ``turn left'' and ``turn right''. For each noun blank that has been filled with the description of one node, we find its closest verb phrase blank and fill it with the pseudo-labeled action which is executed to reach the next node. This eventually enables us to create action-aware instructions.

 % Figure environment removed

\section{Learning VLN from YouTube Videos}
\label{sec:approach}

Given the VLN-like and reasonable path-instruction pairs generated from YouTube videos, we then describe how to learn the Lily agent from these data. As shown in Figure~\ref{fig:scheme}, our VLN model consists of two components: a vision-and-language backbone (\ie Multi-Layer Transformer) that models the relationship between trajectories and instructions and a decision-making module that predicts the next action or a matching score for a  path-instruction pair. The vision-and-language backbone can be any type of cross-modal network. We chose ViLBERT~\cite{ViLBERT} for a fair comparison with Airbert~\cite{Airbert}. As a common practice, pretext tasks are utilized for pre-training the backbone. We next describe how to pre-train the backbone on our Youtube-VLN dataset using the proposed trajectory judgment pretext task. 

\subsection{Model Architecture}
\label{sec:Architecture}
We follow Airbert~\cite{Airbert} to leverage a ViLBERT-like~\cite{ViLBERT} architecture as the model backbone. The model encodes the sequential visual region features and the text token via two separate transformers respectively. More formally, the  path-instruction pair consists of $K$ nodes $\left\{V_k\right\}_{k=1}^K$ and $L$ text tokens $\left\{w_l\right\}_{l=1}^L$. Each node $V_k$ is composed of $R_k$ visual region features $\left\{v_i^k\right\}_{i=1}^{R_k}$. In this way, we represent the visual and text inputs respectively as follows:
% \vspace{-2mm}
\begin{small}
\begin{equation}
X_V=\left[[\texttt{IMG}], v_1^1, \ldots, v_{R_1}^1, \ldots,[\texttt{IMG}], v_1^K, \ldots, v_{R_K}^K\right],
\label{eq:xv}
\end{equation}
\end{small}
\vspace{-4mm}
\begin{small}
\begin{equation}
X_W=\left[[\texttt{CLS}], w_1, \ldots, w_l, \ldots, w_L,[\texttt {SEP}]\right],
\label{eq:xw}
\end{equation}
\end{small}
% \vspace{0.5mm}

\noindent where [\texttt{IMG}], [\texttt{CLS}] and [\texttt{SEP}] are special tokens.
% The visual and text inputs are then sent into two separate transformer encoders respectively and
The encoded visual and text tokens finally interact via a cross-modal transformer encoder. We represent the whole model architecture as ``Multi-Layer Transformer'' in Figure~\ref{fig:scheme}.
 
\subsection{Learning Layout from Trajectory Judgment}
\label{sec:Pre-training}

Given the aforementioned model architecture, we propose to train a VLN agent with a trajectory judgment (TJ) task, enabling it to reason about layouts. Herein, we elaborate on the proposed trajectory judgment task.


\textbf{Formulation.} 
The trajectory judgment task aims to judge the reasonableness of trajectories. We consider the trajectories generated in the way described in Section~\ref{sec:path_gen} as positive (reasonable) samples and the shuffled trajectories as negative (unreasonable) ones. To finish this task, the agent is required to reason about the visual information and identify the room types, then infer whether the trajectory matches the real environment layout distribution. 
% It is imperative that the agent correctly identifies the true navigation trajectory, since the agent can then implicitly learn about what the real environment layout is like. 
Specifically, we first calculate the dot product of the output features of [\texttt{IMG}] and [\texttt{CLS}] tokens. Then, we feed this vector feature to a linear layer to predict the probability that indicates whether the trajectory is reasonable.
The model aims to minimize the binary cross-entropy loss:

\vspace{-4mm}
\begin{small}
\begin{equation}
% L_{T J}=\frac{1}{N} \sum_{n=1}^N-w\left[y_n \cdot \log \left(\operatorname{Sigmoid}\left(\mathbf{P}_\theta^n\right)\right)\right],
L=-\frac{1}{N} \sum_{n=1}^N\left[w \cdot y_n \log \left(p_n\right) + (1-y_n) \log \left({1-p_n}\right)\right],
\label{eq:tj_loss}
\end{equation}
\end{small}
\vspace{-4mm}

\noindent where $y_n = 1$ if the $n^{th}$ trajectory is reasonable, otherwise $y_n = 0$. $p_n$ represents the probability that the $n^{th}$ trajectory is predicted as reasonable. $N$ is the number of trajectories in a batch.
$w$ is a factor to mitigate the imbalance of positive and negative samples, which equals the ratio of the number of negative samples to the number of positive samples.


% \vspace{2mm}
\textbf{Sample Generation.}
 We propose to shuffle the positive sample to generate the negative samples: 1) shuffle only the transition nodes; 2) shuffle all the nodes; 3) keep the order of the room nodes, and randomly insert nodes from other videos. In this way, we create rich and hard negative samples, which increases the task difficulty, helping the agent understand the real layout in a more complex manner.

\textbf{Combining with Existing Pre-Training Tasks.}
As depicted in Figure~\ref{fig:scheme}, we follow Airbert~\cite{Airbert} to pre-train the model backbone using our proposed trajectory judgment task, additionally combining with three other existing pretext tasks, namely masked language modeling (MLM), masked vision modeling (MVM) and path ranking (PR) on YouTube-VLN dataset. For MLM, we randomly mask out the words in instruction and the goal is to recover the masked words. Similar to MLM, MVM is designed to predict masked image regions. PR is a ranking task, which aims to decide the most matching path-instruction pair among a few pairs. 


\begin{table*}[t!]
{
\centering
\resizebox{1.0\linewidth}{!}{
\begin{tabular}{clccclclccccclccccc}
\topline
\multirow{2}{*}{\#} &  & \multicolumn{3}{c}{Dataset}  &  & Pre-training Task &  & \multicolumn{5}{c}{Val Seen} &  & \multicolumn{5}{c}{Val Unseen}\\ \cmidrule{3-5} \cmidrule{7-7} \cmidrule{9-13} \cmidrule{15-19} 
 &  & Source  & \begin{tabular}[c]{@{}c@{}}Reasonable \\Navigation Path\end{tabular} & \begin{tabular}[c]{@{}c@{}}Pseudo-labeled \\ Action\end{tabular} &  & \begin{tabular}[c]{@{}c@{}}Trajectory\\ Judgment\end{tabular} 
                                                              &   & TL           & NE↓          & OSR↑          & SR↑           & SPL↑          &  & TL          & NE↓          & OSR↑          & SR↑           & SPL↑          \\ 
\midline
1    &  & Airbnb Images    &\XSolidBrush &\XSolidBrush &  &\XSolidBrush &   & 10.21 & 3.41 & 79.02 & 74.12 & 0.70 &  & 9.63 & 3.95 & 70.97 & 62.84 & 0.58 \\ 
\midline
2    &  & YouTube Videos &\XSolidBrush &\XSolidBrush &  &\XSolidBrush &   & 10.12 & 3.40 & 79.90 & 74.31 & 0.70 &  & 9.81 & 3.72 & 74.24 & 63.73 & 0.59 \\
3    &  & YouTube Videos & \Checkmark  &\XSolidBrush &  &\XSolidBrush &   & 10.30 & 3.40 & 78.60 & 75.10 & 0.71 &  & 9.60 & 3.70 & 73.50 & 65.00 & 0.61 \\
4    &  & YouTube Videos & \Checkmark  & \Checkmark  &  &\XSolidBrush &   & 10.20 & 3.30 & 79.80 & 75.40 & 0.71 &  & 9.30 & 3.60 & 72.70 & 66.10 & 0.62 \\
5    &  & YouTube Videos & \Checkmark  & \Checkmark &  & \Checkmark   &   & 9.99  & \textbf{3.12} & \textbf{80.88} & \textbf{77.45} & \textbf{0.74} &  & 9.64 & \textbf{3.37} & \textbf{74.93} & \textbf{66.70} & \textbf{0.62} \\ 
\bottomline
\end{tabular}
}
}
\caption{Ablation study on YouTube-VLN dataset and trajectory judgment pretext task for pre-training.}
\label{tab:ablation}
\vspace{-4mm}
\end{table*}


% \vspace{-1.5mm}
\subsection{Adapting Pre-trained Backbone for VLN}
\label{sec:Downstream}
We adapt the pre-trained model to both goal-oriented navigation task and object-oriented navigation task. All the tasks are based on the Matterport3D simulator~\cite{MP3d}. We utilize R2R~\cite{VLN} as the benchmark for the goal-oriented navigation task, which is divided into discriminative setting and generative setting. As for the object-oriented task, we evaluate our model on REVERIE~\cite{REVERIE} in generative setting. 

The discriminative setting formulates VLN as a path-selection problem, requiring the agent to choose the path that best matches the instruction from multiple candidate paths. Under the discriminative setting, we utilize the classifier used in the path ranking pretext task for decision-making and fine-tune the Lily agent on the R2R dataset. 

In the generative setting, the agent needs to predict actions sequentially to reach the goal (R2R) or simultaneously find the object (REVERIE). We adopt DUET~\cite{DUET} as the architecture for fine-tuning, which feeds the cross-modal feature into a feed-forward network for decision-making. We initialize the text transformer encoder and cross-modal transformer encoder of the generative model with our Lily agent. 
Note that our Lily agent can apply to any generative model. More details are available in the supplementary.


\section{Experiments}
\subsection{Experimental Setup}

\textbf{Dataset and Evaluation Metrics.}
We conduct our experiments on two VLN benchmarks,~\ie R2R~\cite{VLN} and REVERIE~\cite{REVERIE}. These two datasets consist of 21,567 path-instruction pairs from 90 scenes in Matterport3D~\cite{MP3d}. REVERIE follows the same train/val/test splits as the R2R, while requiring an agent to select the bounding box 
 of the target object bounding box additionally. Following standard settings~\cite{Airbert}, we adopt five metrics for evaluating R2R, namely success rate \textbf{(SR)}, oracle success rate \textbf{(OSR)}, success rate weighted by the ratio between the length of the shortest path and the predicted path \textbf{(SPL)}, trajectory length \textbf{(TL)} and navigation error \textbf{(NE)}. As for REVERIE, we leverage four metrics for evaluating navigation performance, namely \textbf{SR}, \textbf{OSR}, \textbf{SPL} and \textbf{TL}, and two for object grounding performance, namely remote grounding success (\textbf{RGS}) and RGS weighted by path length (\textbf{RGSPL}).




\textbf{Implementation Details.} 
We implement our method based on Pytorch framework~\cite{pytorch} and Matterport3D simulator~\cite{MP3d}. Specifically, we divide our training process into two stages,~\ie pre-training and fine-tuning. For the pre-training stage, we distribute training over 4 NVIDIA 3090 GPUs for 500 epochs to convergence.  
The pre-trained model with the highest accuracy for the path ranking pretext task is selected for fine-tuning.
During the fine-tuning stage, we distribute training over 8 NVIDIA 3090 GPUs for 30 epochs to convergence. 
Following Airbert~\cite{Airbert}, we use augmented data from EnvDrop~\cite{EnvDrop} for fine-tuning by default.
More details are provided in the supplementary.



\subsection{Ablation Studies on Pre-Training}
\label{sec:ablstion}
We ablate our approach under discriminative setting on R2R benchmark. Considering the time efficiency, we do not use augmented data for fine-tuning on these experiments.


\textbf{Data Source: Airbnb Images~\vs YouTube Videos.}
One of the main differences between YouTube-VLN dataset and Airbnb dataset~\cite{Airbert} is the data source. Airbnb consists of 713k image-caption pairs while YouTube-VLN consists of 587k images extracted from 433 hours of house tour videos.
YouTube-VLN has fewer images but provides more information about a room from different camera angles by merging, which better simulates a panorama for downstream VLN tasks. 
To evaluate the effect of data source, we follow Airbert~\cite{Airbert} to randomly select images in the same house
to build a trajectory and its corresponding instruction.
We keep the number of instructions the same in pre-training for a fair comparison.
In Table~\ref{tab:ablation} (\# 1 \vs \# 2), YouTube data performs slightly better than Airbnb data, surpassing the SR by 0.89\% on the val unseen split.  We speculate this is because data quality is more important than quantity.


\textbf{Effectiveness of Reasonable Navigation Trajectory.}
YouTube-VLN dataset is collected from real house tour videos and is thus able to extract frames in chronological order to build reasonable navigation trajectories instead of combining multiple randomly chosen images.
We use the generated reasonable navigation trajectories and shuffled trajectories to train two agents, respectively. In Table~\ref{tab:ablation}, the agent trained with reasonable navigation trajectories (\# 3) achieves significantly better performance than the shuffled navigation trajectories variant (\# 2), with 1.27\% gains on SR under the val unseen split. This suggests that the agent can not well understand and ground the instruction to the trajectory without reasonable navigation paths for learning. 


\textbf{Effectiveness of Pseudo-Labeled Action.}
The core of the proposed action-aware instruction generator is the pseudo-labeled actions for the instructions. To evaluate the effectiveness of the pseudo-labeled actions, we construct a variant that replaces random action words with the pseudo-labeled actions which are filled in the instructions. The results are shown in Table~\ref{tab:ablation}, \# 4. Compared to the variant (\# 3) that fills the instructions with random action words, this variant boosts the SR metric on both the val unseen (+1.10$\%$) and seen (+0.30$\%$) splits, showing the effectiveness of pseudo-labeled actions. It indicates that with pseudo-labeled actions, the agent can effectively ground the action to the visual observation and recognizes the correct transition from one location to another.



\textbf{Effectiveness of Trajectory Judgment Pretext Task.}
To explore the layout knowledge of the YouTube-VLN dataset and equip the agent with layout reasoning ability, we propose a self-supervised trajectory judgment pretext task. In Table~\ref{tab:ablation}, the variant with the proposed pretext task improves the performance (\# 5 \vs \# 4, +2.05\% on val seen and +0.60\% on val unseen \textit{w.r.t.} SR). Moreover, given a room type and visual information of the current node as inputs, this variant predicts the relative orientation of nodes for that room type with a 30\% increase in accuracy (see supplementary for more details). These substantiate the claim of the importance of the proposed trajectory judgment task, which helps the agent learn the layout reasoning ability.

\begin{table}[!t]
{
\centering
\resizebox{1.0 \linewidth}{!}{
\begin{tabular}{clllccccc}
\topline
% \multirow{2}{*}{\#} &  & \multirow{2}{*}{Method} &  & \multicolumn{5}{c}{Val Unseen}                      \\ \cmidrule{5-9} 
        \#          &  &  Methods                 &  & TL            & NE↓           & OSR↑          & SR↑             & SPL↑      \\ \midline
1                   &  & Random Sample           &  & 9.37 & 3.52 & 72.88 & 64.41 & 0.61 \\
2                   &  & Temporal Difference     &  & 9.55 & 3.61 & 73.35 & 65.30 & 0.61 \\
3                   &  & Entropy-Based           &  & 9.64 & \textbf{3.37} & \textbf{74.93} & \textbf{66.70} & \textbf{0.62}\\ \bottomline
\end{tabular}
}
}
\caption{Comparison between different strategies of selecting frames to represent nodes under val unseen split.}
\label{tab:KeyFrames}
\vspace{-4mm}
\end{table}


\textbf{Effectiveness of Entropy-Based Trajectory Generation Strategy.}
In order to acquire an informative frame to represent a node, we propose an entropy-based technique as mentioned in Section~\ref{sec:path_gen}. To demonstrate the effectiveness of our strategy, we construct two variants,~\ie one randomly chooses a frame, and one decides the frame by temporal difference~\cite{TD}. The second variant computes the temporal pixel difference between every two consecutive frames and picks the frames where the peaks are located as the frames to represent the nodes. In Table~\ref{tab:KeyFrames}, our entropy-based method significantly outperforms these two variants, increasing the SR on the val unseen split from 64.41\% and 65.30\% to 66.70\%, respectively. We speculate this is because 1) randomly sampling frames can generate consecutive redundant frames or meaningless frames (\eg a wall takes up most of the frame); 2) the temporal difference method selects a frame that 
represents the junction of two different rooms, which is ambiguous and thus confuses the agent. In comparison, our entropy-based method is able to find a reliable frame to represent a node and ensure two adjacent nodes belong to different room types.


\subsection{Comparison with State-of-the-Arts}

\textbf{Results on R2R Dataset.}
We first compare our Lily agent with current methods under the discriminative setting. In Table~\ref{tab:discriminative}, compared with VLN-BERT that uses image-caption pairs from the web for pre-training, our Lily agent significantly increases the SR by 10.74\% on the val unseen split. This highlights the importance of providing the in-domain indoor data for pre-training. Moreover, compared with Airbert which uses in-domain image-caption pairs from online rental marketplaces, Lily still increases the SR from 73.85\% to 79.31\% on val seen and from 68.67\% to 70.00\% on val unseen. We attribute the improvement to our proposed VLN-like YouTube-VLN dataset and trajectory judgment pretext task, which have been thoughtfully evaluated in Section~\ref{sec:ablstion}. When ensembled with the speaker-follower~\cite{Speaker-Follower}, all three methods increase the performance and our Lily agent performs the best.
\begin{table}[!t]
{
\resizebox{1.0\linewidth}{!}{

    \begin{tabular}{lllcccccccccccc}
    \topline
    \multicolumn{1}{l}{\multirow{2}{*}{Methods}} & \multicolumn{4}{c}{Val Seen}    & \multicolumn{4}{c}{Val Unseen}                                          \\ \cmidrule{2-9}
                       \multicolumn{1}{c}{}     & TL            & NE↓           & SR↑            & SPL↑           & TL         & NE↓           & SR↑            & SPL↑           \\ \midline
    Follower~\cite{Speaker-Follower}          & 10.40          & 3.68                    & 65.10          & 0.62          & 9.57       & 5.20                    & 52.36          & 0.49          \\
    Speaker~\cite{Speaker-Follower}           & 11.19          & 3.80                    & 60.69          & 0.56          & 10.71      & 4.25                    & 54.66          & 0.49          \\
    \footnotesize{Speaker-Follower}~\cite{Speaker-Follower}  & 10.69          & 2.72                    & 74.22          & 0.70          & 10.10      & 3.32                    & 67.90          & 0.63          \\
    ProbES~\cite{Prompt}                      & -              & -                       & -              & -             & 9.50       & 4.05                    & 60.28          & 0.56          \\
    VLN-BERT~\cite{Improving}                 & 10.28          & 3.73                    & 70.20          & 0.66          & 9.60       & 4.10                    & 59.26          & 0.55          \\
    Airbert~\cite{Airbert}                    & 10.59          & 3.21                    & 73.85          & 0.69          & 10.03      & 3.24                    & 68.67          & 0.63          \\
    Lily                                      & 10.21          & 2.89                    & \textbf{79.31}  & \textbf{0.76}     & 10.03      & \textbf{3.19}             & \textbf{70.00}  & \textbf{0.65}     \\ \midline
    VLN-BERT*~\cite{Improving}                & 10.61          & 2.35                    & 81.86          & 0.78          & 10.00      & 2.76                   & 73.61          & 0.68          \\
    Airbert*~\cite{Airbert}~                  & 10.63          & 2.13                    & 81.40          & 0.77          & 9.99       & 2.69                    & 75.01          & 0.70          \\
    Lily*                                    & 10.51           & \textbf{2.06}           & \textbf{83.29}          & \textbf{0.80}   & 9.78  & \textbf{2.48}             & \textbf{76.88}  & \textbf{0.72} \\ 
    \bottomline
    \end{tabular}
}
}
\centering
\caption{Comparison with state-of-the-arts on R2R dataset under discriminative setting. * means results of ensembling with the speaker-follower~\cite{Speaker-Follower} model.}
\label{tab:discriminative}
\vspace{-2mm}
\end{table}

\begin{table}[t]
\centering
\resizebox{1.0\linewidth}{!}{
\begin{tabular}{lccccc}
\topline
           Methods              & TL     & NE↓  & SPL↑ & OSR↑ & SR↑  \\ \midline

Speaker-Follower~\cite{Speaker-Follower}         & 1257    & 4.87 & 0.01  & 96   & 53   \\
% PreSS           ~\cite{Robust}                   & 10.52   & 4.53 & 0.53  & 63   & 57   \\
% PREVALENT       ~\cite{Generic}                  & 10.21   & 4.52 & 0.56  & 64   & 59   \\
Self-Monitoring ~\cite{SMNA}                     & 373     & 4.48 & 0.02  & 97   & 61   \\
Reinforced CM   ~\cite{RCM}                      & 358     & 4.03 & 0.02  & 96   & 63   \\
EnvDrop         ~\cite{EnvDrop}                  & 687     & 3.26 & 0.01  & 99   & 69   \\
AuxRN           ~\cite{AuxRN}                    & 41      & 3.24 & 0.21  & 81   & 71   \\
VLN-BERT~\cite{Improving}                        & 686.82  & 2.99 & 0.01  & 99   & 73   \\
Global Normalization~\cite{GN}                   & 686.86  & 2.99 & 0.01  & 99   & 74   \\
Airbert~\cite{Airbert}                           & 686.54  & 2.58 & 0.01  & 99   & 77 \\ \midline
LiLy                                             & 686.45 & \textbf{2.50}       & 0.01   &  99    & \textbf{79}     \\ \bottomline
\end{tabular}
}
\setlength{\abovecaptionskip}{0.15cm}
\caption{Results under discriminative setting on the test unseen split as indicated on the R2R leaderboard~\protect\footnotemark.}
\label{tab:leaderboard}
\vspace{-4mm}
\end{table}

\footnotetext{\fontsize{5.75pt}{1pt}{\url{https://eval.ai/web/challenges/challenge-page/97/leaderboard/270}}}

\begin{table*}[t!]
\centering
\resizebox{0.8\linewidth}{!}{\begin{tabular}{lp{0.8cm}<{\centering}p{0.8cm}<{\centering}p{0.8cm}<{\centering}p{0.8cm}<{\centering}cp{0.8cm}<{\centering}p{1.0cm}<{\centering}cp{0.8cm}<{\centering}p{0.8cm}<{\centering}p{0.8cm}<{\centering}p{0.8cm}<{\centering}cp{0.8cm}<{\centering}p{1.0cm}<{\centering}cp{0.8cm}<{\centering}p{0.8cm}<{\centering}p{0.8cm}<{\centering}p{0.8cm}<{\centering}cp{0.8cm}<{\centering}p{1.0cm}<{\centering}}
\topline
\multirow{3}{*}{Methods} & \multicolumn{7}{c}{Val Unseen}                                                     &  & \multicolumn{7}{c}{Test Unseen}             \\ 
                           \cmidrule{2-8} \cmidrule{10-16}
                         & \multicolumn{4}{c}{Navigation}                  &  & \multicolumn{2}{c}{Grounding} &  & \multicolumn{4}{c}{Navigation} &  & \multicolumn{2}{c}{Grounding} \\ \cmidrule{2-5} \cmidrule{7-8} \cmidrule{10-13} \cmidrule{15-16}
                         & TL   & OSR↑  & SR↑            & SPL↑           &  & RGS↑          & RGSPL↑        &  & TL    & OSR↑  & SR↑   & SPL↑  &  & RGS↑          & RGSPL↑         \\ \midline
Human                    & -     & -     & -              & -              &  & -             & -            &  & 21.18  & 86.83 & 81.53 & 83.66 &  & 77.84         & 51.44          \\ \midline
Seq2Seq~\cite{VLN}       & 11.07 & 8.07  & 4.20           & 2.84           &  & 2.16          & 1.63          &  & 10.89  & 6.88  & 3.99  & 3.09  &  & 2.00          & 1.58           \\
RCM~\cite{RCM}           & 11.98 & 14.23 & 9.29           & 6.97           &  & 4.89          & 3.89          &  & 10.60  & 11.68 & 7.84  & 6.67  &  & 3.67          & 3.14           \\
SMNA~\cite{SMNA}         & 9.07  & 11.28 & 8.15           & 6.44           &  & 4.54          & 3.61          &  & 9.23   & 8.39  & 5.80  & 4.53  &  & 3.10          & 2.39           \\
FM~\cite{REVERIE}        & 45.28 & 28.20 & 14.40          & 7.19           &  & 7.84          & 4.67          &  & 39.05  & 30.63 & 19.88 & 11.61 &  & 11.28         & 6.08           \\
SIA~\cite{SIA}           & 41.53 & 44.67 & 31.53          & 16.28          &  & 22.41         & 11.56         &  & 48.61  & 44.56 & 30.80 & 14.85 &  & 19.02         & 9.20           \\
HAMT~\cite{HAMT}         & 14.08 & 36.84 & 32.95          & 30.20          &  & 18.92         & 17.28         &  & 13.62  & 33.41 & 30.40 & 26.67 &  & 14.88         & 13.08           \\
RecBERT~\cite{RecVLN}    & 16.78 & 35.02 & 30.67          & 24.90          &  & 18.77         & 15.27         &  & 15.86  & 32.91 & 29.61 & 23.99 &  & 16.50         & 13.51          \\
ProbES~\cite{Prompt}     & 18.00 & 33.23 & 27.63          & 22.75          &  & 16.84         & 13.94         &  & 16.84  & 28.23 & 24.97 & 20.12 &  & 15.11         & 12.32        \\ 
Airbert~\cite{Airbert}   & 18.71 & 34.51 & 27.89          & 21.88          &  & 18.23         & 14.18         &  & 17.91  & 34.20 & 30.28 & 23.61 &  & 16.83        & 13.28          \\\cmidrule{1-16}
DUET~\cite{DUET}         & 22.11 & 51.07 & 46.98          & 33.73          &  & 32.15         & 23.03         &  & 21.30  & 56.91 & 52.51 & 36.06 &  & 31.88         & 22.06         \\
% DUET~(Lily)              & 20.62 & \textbf{53.28} & \textbf{47.06} & \textbf{33.87} &  &\textbf{32.80} & \textbf{23.71}         &  & 21.00 & \textbf{58.18} & \textbf{52.94} & \gc\textbf{37.76} &  & \textbf{32.52} & \textbf{22.93}            \\ \bottomline
DUET~(Lily)              & 21.87 & \textbf{53.71} & \textbf{48.11} & \textbf{34.43} &  &\textbf{32.15} & \textbf{23.43}         &  & 21.94 & \textbf{60.51} & \textbf{54.32} & \textbf{37.34} &  & \textbf{32.02} & 21.94           \\ \bottomline
\end{tabular}
}
\setlength{\abovecaptionskip}{0.15cm}
\caption{Comparison with state-of-the-arts  on REVERIE. Lily agent achieves the state-of-the-art performance on all splits.}
\label{tab:REVERIE_SOTA}
\vspace{-4mm}
\end{table*}


\begin{table}[htb]
\centering
\resizebox{1.0\linewidth}{!}{\begin{tabular}{lcccccccc}
\topline
\multirow{2}{*}{Methods}                     & \multicolumn{4}{c}{Val Unseen}              &\multicolumn{4}{c}{Test Unseen}          \\ \cmidrule{2-6} \cmidrule{7-9}
                            & TL     & NE↓         & SR↑      & SPL↑      &TL     & NE↓        & SR↑     & SPL↑     \\ \cmidrule{1-9}
Seq2Seq~\cite{VLN}          & 8.39   & 7.81        & 22       & -         &8.13   & 7.85       & 20      & -        \\
EnvDrop~\cite{EnvDrop}      & 10.70  & 5.22        & 52       & 48        &11.66  & 5.23       & 51      & 47       \\
AuxRN~\cite{AuxRN}          & -      & 5.28        & 55       & 50        &-      & 5.15       & 55      & 51       \\
\footnotesize{PREVALENT}~\cite{Generic}    & 10.19  & 4.71        & 58       & 53        &10.51  & 5.30       & 54      & 51       \\
RelGraph~\cite{RelGraph}    & 9.99   & 4.73        & 57       & 53        &10.29  & 4.75       & 55      & 52       \\
RecBERT~\cite{RecVLN}       & 12.01  & 3.93        & 63       & 57        &12.35  & 4.09       & 63      & 57       \\
ProbES~\cite{Prompt}        & 11.58  & 4.00        & 61       & 55        &12.43  & 4.20       & 62      & 56       \\
ADAPT~\cite{ADAPT}          & 12.33  & 3.66        & 66       & 59        &13.16  & 4.11       & 63      & 57       \\
HOP~\cite{HOP}              & 12.27  & 3.80        & 64       & 57        &12.65  & 3.83       & 64      & 59       \\
HAMT~\cite{HAMT}            & 11.46  & 2.29        & 66       & 61        &12.27  & 3.93       & 65      & 60       \\
Airbert~\cite{Airbert}      & 11.78  & 4.01        & 62       & 56        &12.41  & 4.13       & 62      & 57       \\ \cmidrule{1-9}
DUET~\cite{DUET}            & 13.94  & 3.31        & 72       & 60        &14.73  & 3.65       & 69      & 59       \\ 
DUET (Lily)                 & 14.58  & \textbf{2.90}  & \textbf{74}  & \textbf{62} &16.13  & \textbf{3.44}  &\textbf{72}  & \textbf{60}  \\ \bottomline
\end{tabular}
}
\caption{Comparison with state-of-the-arts on R2R dataset under generative setting.}
\label{tab:generative}
\vspace{-3mm}
\end{table}
% \vspace{-6mm}

In Table~\ref{tab:leaderboard}, we evaluate on R2R test split and our Lily ranks first on the VLN challenge leaderboard compared with the results whose manuscripts are publicly available, achieving the highest SR of 79\%. As we follow Airbert to use 30 candidate trajectories from EnvDrop~\cite{EnvDrop} and the leaderboard considers that our agent has walked through all these paths, the SPL metric is low for both Lily and Airbert.

Besides, our Lily agent also helps to increase the performance on R2R under the generative setting. We enhance the state-of-the-art DUET~\cite{DUET} method by pre-training the agent using our method as mentioned in Section~\ref{sec:approach}. In Table~\ref{tab:generative}, the agent incorporating Lily achieves 2\% and 3\% improvements \textit{w.r.t.} SR on the val unseen split and test unseen split, respectively, compared to DUET. Notably, our method increases the SPL by 2\% on the val unseen split, indicating the agent is able to reach the goal more efficiently. We attribute this to the agent's acquisition of layout prior knowledge via the proposed trajectory judgment task on our YouTube-VLN dataset, enabling it to plan more efficient routes to the goals in new environments.



\textbf{Results on REVERIE Dataset.}
Compared with R2R, REVERIE is more challenging as its instructions only describe the destinations without detailed path descriptions. This requires the agent to be equipped with common knowledge about the room layouts and to reason the possible paths that lead to the destinations. In Table~\ref{tab:REVERIE_SOTA}, we outperform the SOTA agent (\ie DUET) by 1.13\% on SR and increase the SPL from 46.98\% to 48.11\% on the val unseen split. It is worth noting that our method achieves a higher OSR with a shorter TL, indicating that our agent finds the destination more quickly. We attribute this to the better layout reason ability learned from the large-scale diverse reasonable trajectories in the proposed YouTube-VLN dataset. A similar performance is obtained on the test unseen split, where the Lily agent improves the SR by 1.81\% and the SPL by 1.28\%. Although our objective is solely navigation, we still achieve comparable performance on grounding metrics and even slightly outstrip DUET on most metrics.


\subsection{Learning Navigation from One Environment}

Our intuition is that pre-training on the YouTube-VLN dataset can mitigate the domain gap of training from scratch and is able to achieve excellent performance with only a few training environments. To verify this, we conduct a one-shot learning study, where we fine-tune our model on only one environment of the original training environments. Note that the candidate paths are generated from all of the possible paths from the start viewpoints to all navigable points, instead of from the expert model in EnvDrop~\cite{EnvDrop}. All the candidate paths are the shortest paths in the navigation graphs. To reduce the bias, we randomly select 5 sets from the entire environments and report the average results.

In Table~\ref{tab:few-shot}, our agent outperforms all the existing pre-training methods. On the val seen split using one-shot fine-tuning, compared to VLN-BERT and AirBERT, our Lily agent achieves 3.60\% and 1.43\% improvements, respectively. In the val unseen split using one-shot fine-tuning, we achieve 28.43\% improvement compared to VLN-BERT. All these results suggest the effectiveness of our method.


 \begin{table}[t!]
\centering
\resizebox{0.6\linewidth}{!}{
\begin{tabular}{lcccccc}
\topline
Methods                   &         & Val Seen      & Val Unseen                   \\ \midline
VLN BERT~\cite{Improving} &         & 45.71         & 22.43                      \\
AirBERT~\cite{Airbert}    &         & 47.88         & 50.00                      \\
Lily                      &         & \textbf{49.31}& \textbf{50.86}             \\ \bottomline
\end{tabular}
}
\caption{SR on val seen and val unseen splits of R2R. All the agents access only one environment.}
\label{tab:few-shot}
\vspace{-4mm}
\end{table}


\section{Conclusion}

In this work, we propose a new approach Lily to address the limitations of existing vision-and-language navigation (VLN) methods by creating a large-scale VLN-like dataset from real house tour videos to train our embodied agent. We overcome the challenges of automatically generating path-instruction pairs to construct the dataset from raw and unlabeled videos by leveraging an entropy-based method for trajectory construction and an action-aware generator for instruction generation. Additionally, we train the agent to judge the reasonableness of trajectories, improving its layout reasoning ability. Our method achieves state-of-the-art performance on two popular benchmarks (R2R and REVERIE), demonstrating the efficacy. Overall, we hope our work can provide valuable insight into the VLN community by learning embodied VLN from passive videos.


\section*{Acknowledgement}
Prof. Tan and his students were partially supported by the 
National Natural Science Foundation of China (NSFC) (62072190), National Natural Science Foundation of China (NSFC) 61836003 (key project), Program for Guangdong Introducing Innovative and Enterpreneurial Teams 2017ZT07X183.



%%%%%%%%% REFERENCES
\bibliographystyle{abbrv}
{
	\small
	\bibliography{ref}
}

\onecolumn 
\newpage
\appendix
\documentclass[10pt, a4paper,  aps, pra, showpacs, longbibliography, nofootinbib,superscriptaddress]{revtex4-2}
\pdfoutput=1
%\usepackage[utf8x]{inputenc}
\usepackage{ucs}
\usepackage{amsmath}
\usepackage{amsfonts}
\usepackage{amssymb}
\usepackage{mathtools}
\usepackage{makeidx}
\usepackage{cellspace,booktabs}
\usepackage{natbib}
\usepackage{lipsum}
\usepackage{bm}
\usepackage{bbm}
\usepackage{relsize}
%\usepackage{bibunits}
%begin hyperref setup
\usepackage{hyperref}
\hypersetup{colorlinks = true, linkcolor=magenta,citecolor=blue, urlcolor=magenta, bookmarksnumbered =  true}
%End hyperref setup
%\usepackage{subcaption}
%\usepackage[format=hang,justification=justified,singlelinecheck=false]{caption}

\usepackage[usenames,dvipsnames]{color}
\definecolor{light-gray}{gray}{0.55}

\usepackage{microtype}

\usepackage{graphicx}


\renewcommand{\dag}{^{\dagger}}
\newcommand{\ssm}{\rm\scriptscriptstyle}
\newcommand{\exv}[1]{ \langle #1 \rangle }

\newcommand{\bra}[1]{ \langle #1 \rvert }
\newcommand{\ket}[1]{ \lvert #1 \rangle}
\newcommand{\braket}[2]{\langle #1 \vert #2 \rangle }
\newcommand{\innerbraket}[3]{\langle #1 \vert #2 \vert #3 \rangle }
\newcommand{\tr}[2][]{\text{Tr}_{ #1 } ( #2 )}
\newcommand{\up}{\uparrow}
\newcommand{\down}{\downarrow}

%\newcommand{\overlr}[1]{\overleftarrow{\overrightarrow{#1}}}
\newcommand{\overlr}[1]{\overset{\leftrightarrows}{#1}}
\renewcommand{\overleftarrow}[1]{\overset{\leftarrow}{#1}}
\renewcommand{\overrightarrow}[1]{\overset{\rightarrow}{#1}}


\newcommand{\pfrac}[2]{\frac{\partial #1}{\partial #2}}
\newcommand{\intinf}{\int_{-\infty}^{\infty}}

\DeclareRobustCommand{\EG}[1]{{\color{blue}#1}}
\DeclareRobustCommand{\CKA}[1]{{\color{red}{#1}}}



\DeclareRobustCommand\VC[1]{{\color{ForestGreen}VC: #1 }}

\DeclareRobustCommand\AV[1]{{\color{Magenta}AV: #1 }}
\begin{document}
%TC:ignore
\widetext

\date{\today}
\author{Agnes Valenti}
\affiliation{Institute for Theoretical Physics, ETH Zurich, CH-8093, Switzerland}
\author{Vladimir Calvera}
\affiliation{Department of Physics, Stanford University, Stanford, CA 94305, USA}
\author{Steven A. Kivelson}
\affiliation{Department of Physics, Stanford University, Stanford, CA 94305, USA}
\author{Erez Berg}
\affiliation{Department of Condensed Matter Physics, Weizmann Institute of Science, Rehovot 76001, Israel}
\author{Sebastian D. Huber}
\affiliation{Institute for Theoretical Physics, ETH Zurich, CH-8093, Switzerland}

\title{Supplemental Material for ``Nematic metal in a multi-valley electron gas: \\ Variational Monte Carlo analysis and application to AlAs''}

\maketitle

\tableofcontents


\setcounter{equation}{0}
\setcounter{figure}{0}
\setcounter{page}{1}
\makeatletter
\renewcommand{\theequation}{S\arabic{equation}}
\renewcommand{\thefigure}{S\arabic{figure}}
\renewcommand{\bibnumfmt}[1]{[S#1]}
\renewcommand{\citenumfont}[1]{S#1}


\section{Approximations in the model Hamiltonian}
In our VMC simulations, we consider the Hamiltonian
\begin{align}
\label{eq:H}
H&=-\sum \limits_{i} \frac{1}{2m^{*}}\big(\eta^{\tau_i /2} \partial_{i,x}^2+ \eta^{-\tau_i /2} \partial_{i,y}^2 \big) + \sum \limits_{i<j} V(|{\bf r}_i- {\bf r}_j|), \\
V(|{\bf r}_i-{\bf r}_j|)&=\frac{1}{(2\pi)^2}\int {\rm d} {\bf q}\, {\rm e}^{i{\bf q} |{\bf r}_i-{\bf r}_j|}v({\bf q}), \\
v({\bf q})&= \frac{e^2}{2 \epsilon_0 \epsilon} \frac{\tanh (d |{\bf q}|)}{|{\bf q}|}.
\end{align}
where $m^{*}$ is the effective mass, $\eta$ denotes the anisotropy, the sum runs over all particles $i$ and $d $ corresponds to the gate-distance. 
In this section, we discuss neglected terms and made approximations in comparison to experiments on an AlAs quantum well \cite{hossain2021spontaneous}.


\subsection{Finite thickness}
A source that may affect the regions of stability in the phase diagram is the finite thickness of the AlAs quantum well. In~\cite{de2005effects} it has been shown, that while the finite thickness significantly alters the spin susceptibility, it only results in a slight shift of the phase boundaries.   

We neglect its effect here, but note that it still may be non-negligible as the results in~\cite{de2005effects} are obtained via perturbation theory and the estimation of phase boundaries is sensitive to small energy differences. In principle, finite-size thickness could be directly taken into account by performing simulations in a slab of finite thickness. This would, however, require increased computational effort. An estimation of the effects while keeping the simulations strictly two-dimensional can also be obtained via a simple approximation, using a device-specific form factor $F(q)$ that modifies the interaction $v(q)$ in Fourier space~\cite{de2005effects} to $\tilde{v}(q)=v(q)F(q)$, where $F(q=0)=1$ and $F(q\gg 1/w)\sim C/(wq)$ (here, $w$ is the width of the quantum well). For an AlAs quantum well, this form factor is given in~\cite{de2005effects, gold1987electronic}, with $C\approx 3$. 


\subsection{Valley-dependent interaction terms}
In this subsection, we consider the contribution of inter-valley scattering terms that we neglected in treating the valleys as separate isospin flavours. In particular, let us consider the interactions within the full model, that takes into account the complete Brillouin zone with valleys around $X=(2\pi/a,0)$ and $Y=(0,2\pi/a)$ where $a_{\text{AlAs}}= 566 \rm{pm}$ the AlAs lattice constant. Here, we took into account that AlAs has a ``zinc-blend'' structure. Then, the inter-valley terms of the form
\begin{align}
v_{\ssm i-v}=\frac{1}{2L^2}\sum_{{\bf q}\neq 0} \bigg[ \tilde{v}({\bf q}) \sum_{\bf k} a^{\dagger}_{{\bf k} + {\bf q}} a_{\bf k} \sum_{\bf k'} a^{\dagger}_{{\bf k'-q}} a_{\bf k'}\bigg], \\
{\bf k} \in \text{valley $X$}, \ \ {\bf k+q} \in \text{valley $Y$}, \nonumber \\
{\bf k'} \in \text{valley $Y$}, \ \ {\bf k'-q} \in \text{valley $X$} \nonumber
\end{align}
are neglected when valleys are treated as isospin flavours.

The strength of this interaction term goes as $\tilde{v}({\bf q})$ with $q \sim \sqrt{2} \frac{2\pi}{a}$, where $a$ is the lattice constant ($q$ needs to connect between the $X$  and the $Y$ point of the Brillouin zone). Since for $q$ much larger than $1/w$ and $1/d$, $\tilde{v}({\bf q})\propto  3/(wq^2)$, this contribution is very small in comparison to intra-valley terms that involve a momentum transfer of $q \sim k_F$. Concretely, at a density $n=10^{11} \rm{cm}^{-2}$ (corresponding to $r_s \approx 15.4$, with $k_F \approx \sqrt{2\pi \cdot 10^{15}} \rm{m}^{-1}$), and using $w=20\,\rm{nm}$~\cite{hossain2021spontaneous}, the valley- (and spin-) dependent interactions are smaller by a factor of
\begin{align}
%\gamma \approx 
\frac{\tilde{v}(2\sqrt{2}\pi/a)}{\tilde{v}(k_F)} \approx \frac{3 k_F a^2}{8\pi^2 w}
%\frac{\tanh(d \sqrt{2} \frac{2\pi}{a})}{\tanh(d k_F)} 
\sim \mathcal{O}(10^{-4}).
\end{align}
This is small compared to the typical energy differences between states of different spin or valley polarization found in our calculations, which are of the order of $10^{-2}$ of the total energy (see Fig. 2 of the main text). We therefore conclude that the inter-valley scattering terms can be neglected.

This estimation is not significantly affected by the considered metal-gate screening, since
\begin{align}
 \frac{\tanh (d2\sqrt{2} \pi/a)}{\tanh (d k_F )}\approx 1,
\end{align}
for $d=100$ nm.
% However, considering that the VMC-obtained energy differences are also of a similar order of magnitude (divided by the correlation energy), the valley dependent terms might have an influence on the phase boundaries. In order to determine which polarization they favour (sign of the interaction), we consider a simplified  picture.

% Concretely, we consider one electron in valley $X$ and one electron in valley $Y$. Neglecting other states, the inter-valley exchange interaction between these particles is given by~\cite{auerbach1998interacting}
% \begin{align}
% v_{\tau, \tau'}=J^{F}\sum_{\sigma \sigma'}c^{\dagger}_{\sigma \tau} c^{\dagger}_{\sigma' \tau'} c_{\sigma' \tau} c_{\sigma \tau'}.
% \end{align}
% The operator $c^{\dagger}_{\sigma \tau}$ ($c_{\sigma \tau}$) creates (annihilates) a state at valley $\tau$ with spin $\sigma$.
% Consequently, this interaction acts in the spin-space of the two occupied valleys at $X$ and $Y$. The coupling $J_F$ can be proven to be positive and real~\cite{auerbach1998interacting}.
% Defining the spin one half operators
% \begin{align}
% {\bf S}_{\tau}:=\frac{1}{2}\sum_{\sigma \sigma'} c^{\dagger}_{\sigma \tau} \vec{\sigma}_{\sigma \sigma'} c_{\sigma' \tau}
% \end{align}
% with $\vec{\sigma}$ the vector of Pauli matrices, we obtain the exchange interaction
% \begin{align}
% v_{\tau,\tau'}=-2J^{F}\big( {\bf S}_{\tau} \cdot {\bf S}_{\tau'}+\frac{1}{4}n_{\tau} n_{\tau'}\big),
% \label{eq:vintervalley}
% \end{align}
% with the valley occupation $n_{\tau}=\sum_{\sigma} c^{\dagger}_{\sigma \tau} c_{\sigma \tau}$.

% The first term in Eq.~(\ref{eq:vintervalley}) corresponds to ferromagnetic coupling (also denoted as Hund's coupling) that favours parallel alignment of spins in the two valleys. As a consequence, we expect the energy difference between the symmetric state and the spin-polarized state to be reduced - an existence of a spin-polarized phase even in the isotropic case, where the symmetric state is strongly favoured, is however questionable. The second term results in a lower energy, when two valleys are populated in comparison to one. Thus, it favorized valley-{\em unpolarized} states. This aligns with the comparison of our numerical VMC results with the experimental observations: In the experiment, the phase boundary between the symmetric state and VP is found at $r_s \approx 20$. In our simulations, we already find a transition at $r_s \approx 11$. We thus expect the discussed inter-valley contribution to shift this transition to larger $r_s$, closer to the experimental results.


\subsection{Screening}
In modifying the gate distance $d$ in between $d=70$ and $d=300$ nm, we only found phase boundary shifts of similar order of magnitude as their error bars. However, we note that e.g. a single-gate screened potential or no screening at all could lead to more significant effects.

The evaluation and implementation of the dual-gate screened potential is further detailed in Sec.~\ref{dual-gateV}.

\subsection{Electron-phonon interaction}

We now discuss the effect of electron-phonon coupling, neglected throughout
most of this work. Coupling to the lattice favors the valley-polarized
state over the spin-polarized state, since the valley order parameter
couples linearly to an orthorombic lattice distortion. However, since
the 2DEG is embedded in a three-dimensional material, the energy gain
due to the distortion may expected to be small. Here, we show that
for parameters relevant to the experiment in Ref. \cite{hossain2021spontaneous}, the energy
gain due to the lattice distortion is negligible compared to the energy
differences we find in the purely electronic model (where only Coulomb
interactions are taken into account).

We add to the electronic Hamiltonian of Eq. (\ref{eq:H}) the following terms:
\begin{equation}
\Delta H = H_{el-ph}+H_{ph}.
\end{equation}
The electron-phonon coupling, $H_{el-ph}$, is written as
\begin{equation}
H_{el-ph}=\int d^{2}r\,\frac{1}{2}E_2 \left(n_{X}-n_{Y}\right)\left(\epsilon_{xx}-\epsilon_{yy}\right),
\end{equation}
where $n_{X},$ $n_{Y}$ are the two-dimensional densities of electrons
in the two valleys, $E_2$ is the shear tetragonal deformation potential, and $\epsilon_{ij}$ is the strain tensor.
The phonon Hamiltonian includes the elastic energy (we shall ignore
the phonon dynamics for simplicity):
\begin{equation}
H_{ph}=\int d^{3}r\:\frac{1}{2}\left[C_{11}\left(\epsilon_{xx}^{2}+\epsilon_{yy}^{2}\right)+2C_{12}\epsilon_{xx}\epsilon_{yy}\right],
\end{equation}
where $C_{11}$ and $C_{12}$ are elastic constants (we omitted the
third elastic constant characterizing a cubic system, $C_{44}$, since
it plays no role in our discussion). We start by assuming that the
2DEG has an effective thickness $w$, and that the strain is essentially
uniform across the thickness. (In reality, the thickness of the semiconductor
is much larger than the thickness of the 2DEG, as we shall discuss
below). The energy of the fully valley polarized state is then found
to be
\begin{equation}
E=\int d^{2}r\,\left\{\frac{1}{2}E_2\, n(\epsilon_{xx}-\epsilon_{yy})+\frac{w}{2}\left[C_{11}\left(\epsilon_{xx}^{2}+\epsilon_{yy}^{2}\right)+2C_{12}\epsilon_{xx}\epsilon_{yy}\right]\right\}.
\end{equation}
Here, $n$ is the two-dimensional electron density. 

Minimizing this expression over $\epsilon_{xx}$ and $\epsilon_{yy}$,
we find that the energy gain per unit area due to coupling to the
lattice is
\begin{equation}
E_{el-ph}=-\frac{E_2^{2}n^{2}}{4w\left(C_{11}-C_{12}\right)}.
\end{equation}
The energy gain per electron is $E_{el-ph}/n$. For AlAs, $C_{11}-C_{12}\approx 63\,{\rm GPa}$
\cite{adachi1985gaas} and $E_2 \approx 5.8\,\rm{eV}$ \cite{Charbonneau1991MeasurementElasticAlAs,Gunawan2006ValleySusceptibility}.
Taking $n=10^{11}\,{\rm cm}^{-2}$,
we obtain $E_{el-ph}/n\sim -10^{-3}\,{\rm meV}$ per electron.
This is significantly smaller than the typical energy differences we find in
the electron-only model between the valley and spin polarized states
(see Fig. 2 of the main text). We conclude that the effects of coupling
to the lattice can be ignored.

In fact, the thickness of the semiconductor in the experiment is much
larger than $w$. The lattice distortion is not uniform across the
thickness of the semiconductor. Moreover, depending on the thickness,
it may be favorable to form domains of opposite valley polarization,
such that the distortion decreases in the bulk as a function of distance
from the 2DEG. In any case, the fact that the semiconductor is thicker
than $w$ only decreases the energy gain due to the lattice distortion,
compared to the naive estimate given above. 


\section{Optimization}

We use the variational principle
\begin{align}
E[\Lambda]:=\frac{\langle \Psi_{\Lambda} | \hat{H} | \Psi_{\Lambda} \rangle}{\langle \Psi_{\Lambda} | \Psi_{\Lambda} \rangle} \geq E_g,
\end{align}
where $E_g$ is the ground-state energy and $\Psi_{\Lambda}$ the trial wave-function with optimizable parameters ${\Lambda}$. By minimizing the expectation value of the Hamiltonian with respect to a given trial wave-function, one obtains a ground-state approximation.
Keeping the spin polarization fixed, the complexity in the evaluation of the above expectation value lies in a high-dimensional integral over all particle positions:
\begin{align}
\frac{\langle \Psi_{\Lambda} | \hat{H} | \Psi_{\Lambda} \rangle}{\langle \Psi_{\Lambda} | \Psi_{\Lambda} \rangle} = \frac{\int d{\bf r} d{\bf r' } \Psi_{\Lambda}^* ({\bf r}) \hat{H} \Psi_{\Lambda} ({\bf r'})}{\int d{\bf r} |\Psi_{\Lambda} ({\bf r})|^2 }= \frac{\int d{\bf r} |\Psi_{\Lambda} ({\bf r})|^2 H_L({\bf r}) }{\int d{\bf r} |\Psi_{\Lambda}({\bf r})|^2}.
\label{eq:Hexp}
\end{align}
Here, ${\bf r}=({\bf r}_1, ...{\bf r}_N)$, where ${\bf r}_i$ denotes the coordinates of particle $i$. 
We defined  the `local energy' $H_L({\bf r})=\Psi_{\Lambda}({\bf r})^{-1} \hat{H} \Psi_{\Lambda}({\bf r})$. The resulting integral can be efficiently estimated using Monte Carlo sampling: Instead of integrating over the complete Hilbert space, samples are drawn from the probability distribution $p\propto|\Psi_{\Lambda} ({\bf r})|^2$  using the Metropolis-Hastings algorithm. Then,
\begin{align}
\frac{\int d{\bf r} |\Psi_{\Lambda} ({\bf r})|^2 H_L({\bf r}) }{\int d{\bf r} |\Psi_{\Lambda}({\bf r})|^2} = \frac{1}{N_s} \sum_{{\bf r}_i \sim p}H_L({\bf r}_i)+\xi
\end{align}
and $\{ {\bf r}_i \}$ are the $N_s$ samples obtained via Metropolis Monte Carlo. The variance of the Gaussian-distributed statistical error $\xi$ with zero mean scales as $1/\sqrt{N_s}$, but vanishes when the Hamiltonian is evaluated and the trial wave-function corresponds to an eigenstate. Concretely, we can write this zero-variance property as \cite{casula2005new}
\begin{align}
\sigma^2=\sum_{{\bf r}_i \sim p}\bigg[H_L({\bf r}_i)-\bar{H}_L\bigg]^2 \geq 0, \\
\bar{H}_L=\frac{1}{N_s} \sum_{{\bf r}_i \sim p}H_L({\bf r}_i)
\end{align}
The equality holds when the wave-function corresponds to the ground-state (or an eigenstate), because then
\begin{align}
H_L({\bf r})=\Psi_{\Lambda}({\bf r})^{-1} \hat{H} \Psi_{\Lambda}({\bf r})=E_g, \ \ \ \bar{H}_L=E_g.
\end{align}
This zero-variance property allows for accurate estimation of the ground-state wave-function when an appropriate trial wave-function is used.

For minization of the variational energy $E[\Lambda]$, we use the stochastic reconfiguration technique introduced in
\cite{sorella1998green}, which can be understood as effective second order approximation to imaginary time evolution. 
The parameters $\Lambda$ of the trial wave-function are updated in every iteration as
\begin{align}
\Lambda \to \Lambda - \gamma S^{-1} F_{\Lambda},
\end{align}
where $\gamma$ is the learning rate and the force $F$ is given by
\begin{align}
F_k=2\bigg( \frac{\langle \partial_{\Lambda_k} \Psi_{\Lambda} | \hat{H}| \Psi_{\Lambda} \rangle}{\langle \Psi_{\Lambda}|\Psi_{\Lambda} \rangle}-E[\Lambda]\frac{\langle \partial_{\Lambda_k} \Psi_{\Lambda}| \Psi_{\Lambda} \rangle}{\langle \Psi_{\Lambda}|\Psi_{\Lambda} \rangle}\bigg).
\end{align}
The derivative with respect to the $k$-th variational parameter is denoted by $\partial_{\Lambda_k}$.

Second order effects are included by the covariance matrix \cite{park2020geometry}
\begin{align}
S_{k,k'}=\frac{\langle \partial_{\Lambda_k} \Psi_{\Lambda}| \partial_{\Lambda_{k'}}\Psi_{\Lambda} \rangle}{\langle \Psi_{\Lambda}|\Psi_{\Lambda} \rangle}-\frac{\langle \partial_{\Lambda_k} \Psi_{\Lambda}| \Psi_{\Lambda} \rangle  \langle \Psi_{\Lambda}| \partial_{\Lambda_{k'}}\Psi_{\Lambda} \rangle}{\langle \Psi_{\Lambda}|\Psi_{\Lambda} \rangle \langle \Psi_{\Lambda}|\Psi_{\Lambda} \rangle}
\end{align}
We employ the explicit regularization $S=S+\epsilon \mathbbm{1}$ in order to ensure invertibility. We use $\epsilon=10^{-3}$. At iteration $n_{\ssm it}$, we use the learning rate $\gamma={\rm max}(\gamma_0 \cdot 0.997^{n_{it}},\gamma_{\ssm min})$. The inital learning rate $\gamma_0$ is chosen (depending on density and system size) in the interval $[0.008,0.04]$ and $\gamma_{\ssm min} \in [0.0005,0.001]$.

\section{Trial wave-function}
The chosen parametrization of the trial wave-function, i.e. the concrete use of the parameters $\Lambda$, is of crucial importance to the accuracy of the results. The goal lies in representing the many-body correlated ground state wave-function as precisely as possible. At the same time, the trial wave function should also allow for efficient evaluation in order to keep the computational cost feasible.
In addition, antisymmetry has to be ensured for fermionic systems. An established route in ensuring antisymmetry lies in separating the wave-function into a product of a determinantal part, and a Jastrow factor
\begin{align}
\Psi_{\Lambda}({\bf r}_1, ...,{\bf r}_N)=\Psi_J({\bf r}_1, ...,{\bf r}_N)\cdot \Psi_D ({\bf r}_1, ...,{\bf r}_N), \label{eq:J_PsiD}
\end{align}
where the Jastrow factor $\Psi_J$ is chosen to be real,  positive and symmetric, such that the nodes and antisymmetry are fully determined by the determinantal part of the wave-function $\Psi_D$.
In particular, we employ a Slater-Jastrow-Backflow wave-function as elaborated in more detail below.

Before specifying more concretely the choice of Jastrow factor $J$ and determinantal part $\Psi_D$, we note here that in principle such a separation is not necessary. In particular, it has been shown that any totally antisymmetric wave-function can be represented as a single generalized determinant \cite{pfau2020ab, foulkesvariational}. The challenge lies then in finding the appropriate multi-electron orbitals. In \cite{pfau2020ab}, the authors have capitalized on deep neural networks as general function approximators \cite{bengio2017deep} to obtain these multi-electron orbitals. The resulting neural-network ansatz {\it Fermi Net} yields more accurate results for small atoms and molecules \cite{pfau2020ab} than any other wave-function to-date. However, a large computational cost is required to move to larger lattice sizes. 
For the isotropic electron gas, conventional trial wave-functions have been outperformed using a neural-network ansatz for systems with $27$ and $54$ electrons \cite{cassella2023discovering, wilson2022wave, li2022ab}.
The most scalable and accurate neural-network ansatz for periodic systems to-date constructed in \cite{pescia2023message} using message-passing neural networkds yields results for the $3D$ homogeneous electron gas up to $120$ electrons. While these consititute very promising results, many variational parameters and thus a comparably large computational cost is associated with neural-network quantum states for fermionic, continuous systems to-date. 
 
For the purpose of gaining physical insights about phases of multi-valley anisotropic systems, we are interested in probing the phase diagram as a function of $r_s$ and anisotropy on a dense grid. Thus, we require many simulations and use wave-functions of the product form \ref{eq:J_PsiD}, that are less accurate than the above mentioned wave-functions but only require little computational effort. 

We explain the form of our implemented trial wave-function and the use of the parameters $\Lambda$ in detail below.

\subsection{The Jastrow factor}
The Jastrow factor \cite{jastrow1955many} improves the many-body wave-function by effectively keeping electrons apart and creating a correlation hole.
We write the Jastrow prefactor of the wave-function as
\begin{align}
\Psi_J=e^{-J({\bf r}_1, ...{\bf r}_N)},
\end{align}
where the Jastrow factor $J$ is real and symmetric with respect to permutation of the particle positions $({\bf r}_1, ...{\bf r}_N)$.
While this non-negative bosonic prefactor \cite{holzmann2016theory} can in principle depend on all electron positions, in practice it is systematically constructed in a many-body expansion \cite{kim2018qmcpack}
\begin{align}
J=\frac{1}{2}\sum_{i,j} u_2 ({\bf r}_{i}, {\bf r}_{j}) +\frac{1}{6}\sum_{i,j, k} u_3 ({\bf r}_{i}, {\bf r}_{j }, {\bf r}_{k }) + ...,
\end{align}
with each $n$-body term being $u_n$ symmetric in the particle positions. Here, ${\bf }_{i}$ denotes the position of particle $i$ and the function $u_2$ ($u_3$) can depend on the isospins $\alpha_i, \alpha_j$ (and $\alpha_k$) of the particles $i$, $j$ (and $k$).

The two-body function $u_2$ has been approximated in early work as the random-phase-approximation (RPA) correlation function \cite{ceperley1978ground, kwon1993effects}. More expressive power is contained in the approximation of the two-body term as a (iso)spin-dependent liquid-like factor
\begin{align}
u_2({\bf r}_{i},{\bf r}_{j}) &=u_{\alpha_i, \alpha_j}(r_{ij}), \label{eq:u2}\\
r_{ij} &=|{\bf r}_{i}-{\bf r}_{i}|,
\end{align}
in combination with a polynomial expansion of $u_{\alpha_i, \alpha_i}(r_{ij})$ \cite{drummond2004jastrow, drummond2009phase} or a cubic Bspline interpolation as implemented in {\it qmcpack} \cite{kim2018qmcpack}:
\begin{align}
u_{\alpha_i \alpha_j}(r_{ij})=\sum \limits_{m=0}^M p_m B_3 \big( \frac{r_{ij}}{r_C /M}-m \big), \label{eq:ualphabeta}
\end{align}
where the cardinal cubic B-spline function $B_3(x)$ is centered at $x=-1$ and zero everywhere except on the interval $x\in [-3,1)$. The optimizable parameters are given by the $M$ control points $p_m$.
In order to comply with periodicity, the above parametrization is combined with cutting the Jastrow factor at the Wigner-Seitz radius $r_C$. Continuity of the wave-function and its first- and second derivatives is ensured by setting the last parameters $p_m$ to zero. The expressivity of the parametrization can be increased by increasing the number of control points $M$.

We note two issues with the above two-body Jastrow factor: First, a circular cutoff at the Wigner-Seitz radius limits the representability to short-range correlations, as the edges of the simulation cell are not included.
Second, the resulting correlation hole is isotropic. However, our Hamiltonian is anisotropic.

Both long-range correlations and anisotropic effects are typically encaptured by adding long-range periodic terms \cite{drummond2004jastrow}. In our case, we expect that the anisotropy in the kinetic energy of the Hamiltonian will have a relevant effect also on the short-range behaviour of the correlations. 
We thus propose the following two-body Jastrow factor to encapture both anisotropy and longer-ranged correlations
\begin{align}
u_2({\bf r}_{i},{\bf r}_{j})=u_{\alpha_i, \alpha_j}(r_{ij})+\nu_{\alpha_i, \alpha_j}({\bf r}_{i},{\bf r}_{j}).
\end{align}
Here, we separated the Jastrow factor into the short-ranged isotropic term $u$~(\ref{eq:ualphabeta}), and a second term $\nu$ allowing for anisotropy and correlations throughout the whole simulation cell. Any choice of $\nu$ that involves a cutoff and short-range anisotropy will either be restricted to very short-range correlations or lead to discontinuities as a consequence of periodic boundary conditions.
We thus resort to a different choice for the anisotropic term $\nu$. We combine anisotropy, cubic Bspline interpolation and a periodic ansatz presented in Ref.~\cite{whitehead2016jastrow}. In particular, we construct $\nu$ out of building blocks that already fulfill periodic boundary conditions \cite{whitehead2016jastrow}
\begin{align}
f_x({x}_{ij})=|{x}_{ij}|\big(1-2\frac{|{x}_{ij}|^3}{L_x^3}\big), \ \ 
f_y({y}_{ij})=|{y}_{ij}|\big(1-2\frac{|{y}_{ij}|^3}{L_y^3}\big),
\end{align}
with ${x}_{ij}=r^x_{i}-r^x_{j}$ being the x-component of the distance between particle $i$ and particle $j$. Respectively, $y_{ij}$ denotes the $y$-component. The length of the simulation cell in $x$ ($y$)-direction is given by $L_x$ ($L_y$), such that ${x}_{ij}\in (-L_x/2, L_x/2]$.  Here, we use periodic boundary conditions to define the distance within the Wigner-Seitz unit cell. The function $f_x$ is symmetric under $x\to -x$ and satisfies periodic boundary conditions at the edge of the simulation cell. More concretely
\begin{align}
f_x(L_x/2)=f_x(-L_x/2)\neq 0, \\
f'_x(L_x/2)=0, \\
f''_x(L_x/2)\neq 0. 
\end{align}
and vice versa for $f_y$. Thus, any function made out of $f_x$ and $f_y$ as building blocks automatically satisfies periodic boundary conditions.
In Ref.~\cite{whitehead2016jastrow} it was shown, that constructing a Jastrow factor out of a polynomial expansion of these building blocks can achieve lower energies for the homogenous electron gas than a combination of more conventional short-range and long-range Jastrow terms.

We make use of these building blocks $f_x$ and $f_y$ to design a Jastrow factor that can not only represent short-and long range correlations, but also anisotropy on all length scales. In addition, we find that numerical stability is improved using a combination with cubic Bspline interpolation instead of a polynomial expansion. We arrive at the Jastrow term 
\begin{align}\label{eq:parametrizationNugg}
\nu_{\alpha_i, \alpha_j}({\bf r}_{i},{\bf r}_{j}) &=A_1^{\alpha_i, \alpha_j}(g_1({\bf r}_{i},{\bf r}_{j}))+A_2^{\alpha_i, \alpha_j}(g_2({\bf r}_{i},{\bf r}_{j})), \\
g_1({\bf r}_{i},{\bf r}_{j})&=\sqrt{\lambda f_x^2 + (1-\lambda) f_y^2}, \label{eq:g1}\\
g_2({\bf r}_{i},{\bf r}_{j})&=\sqrt{(1-\lambda) f_x^2 + \lambda f_y^2}. \label{eq:g2}
\end{align}
Here, $A_1$ ($A_2$) corresponds to a cubic Bspline interpolation as in Eq.~(\ref{eq:ualphabeta}), defined on the image $g_1 \in [0,L_x/2]$ ($g_2 \in [0,L_y/2]$). No cutoff needs to be imposed here, and correlations in the whole simulation cell can be represented.
We used here that at small radius $r = \sqrt{x^2+y^2}$, $f_x(x)=|x|+\mathcal{O}(|x|^4)$, such that $(f^2_x(x)+f^2_y(y))^{1/2}=r+\mathcal{O}(|r|^4)$. The optimizable parameter $\lambda \in [0,1]$ therefore tunes the anisotropy: When $\lambda=0.5$, then the function $\nu$ is isotropic and $g_1({\bf r})=g_2({\bf r})=r+\mathcal{O}(|r|^4)$ for small $r$.

The used Jastrow factor components are isospin-dependent. With $\alpha=1, 2$ corresponding to spin-up and spin-down of valley $X$ and $\alpha=3, 4$ of valley $Y$, respectively, we captialize on the symmetries of the model by setting
\begin{align}
u_{11}=u_{22}=u_{33}&=u_{44},  \ \ \text{intra-valley, intra-spin} \label{eq:usym1}\\
u_{12}&=u_{34},  \ \ \text{intra-valley, inter-spin} \\
u_{13}=u_{14}=u_{23}&=u_{24}, \ \  \text{inter-valley.}
\end{align}
For the anisotropic $\nu$-term, we can make use of $90 \deg$ rotation between the two valleys and the parametrization of anisotropy detailed above, using $\lambda$. In particular, when denoting the parameter $\lambda$ explicitly in the function name by writing $\nu^{\lambda}$, we set
\begin{align}
\nu^{\lambda}_{11}=\nu_{22}^{\lambda}=\nu_{33}^{1-\lambda}=\nu_{44}^{1-\lambda}&, \ \  \text{intra-valley, intra-spin}\\
\nu_{12}^{\lambda}=\nu_{34}^{1-\lambda}&,  \ \ \text{intra-valley, inter-spin} \\
\nu_{13}^{\lambda}=\nu_{14}^{\lambda}=\nu_{23}^{\lambda}=\nu_{24}^{\lambda}&, \ \ \text{inter-valley}. \label{eq:vsymn}
\end{align}
Here, $\nu^{1-\lambda}$ corresponds to replacing the parameter $\lambda$ by $(1-\lambda)$ in the definition of $g_1$ and $g_2$ in Eqs.~(\ref{eq:g1})-(\ref{eq:g2}) and thereby effectively swapping $x$ and $y$.


% Figure environment removed

Our Jastrow factor uses Bspline interpolation with $8$-$10$ control points (depending on density and anisotropy). Further increasing the number of segments did not result in lower energies. Figure~\ref{fig:JastrowFactor} shows the Jastrow factor between two electrons of different isospin optimized for the VP and SP state. The deviation from spherical symmetry can be observed in form of anisotropy, implying $C_2$-symmetry (intra-valley correlation, left hand side of Fig.~\ref{fig:JastrowFactor}) and $C_4$-symmetry (inter-valley correlation, right hand side of Fig.~\ref{fig:JastrowFactor}).



\subsection{The determinantal part}
The nodal structure is solely determined by the determinantal part of the wave-function.
Thus, the parametrization of the determinantal part should either already correspond to the correct nodal structure, or allow for flexibility such that optimization yields an accurate approximation thereof.


As a natural first step of {\it guessing} the nodal structure, we set the determinantal part to a single Slater determinant
\begin{align}\label{eq:DeterminatalPartDef}
\Psi_D({\bf r}_1,...{\bf r}_N) &=\prod \limits_{\alpha}\det \big( {\bf D}^{\alpha}\big), \nonumber \\
{\bf D}^{\alpha} &=\big( D_{ij}^{\alpha} \big)_{\{i,j\}}, \nonumber \\
D_{ij}^{\alpha} &= \phi_j^{\alpha}({\bf r}_i).
\end{align}
Here, we used a common trick to reduce computational cost and factorized the wave-function into a product of Slater-determinants ${\bf D}^{\alpha}$ per isospin sector $\alpha=(\sigma, \tau)$. Although the anti-symmetry of the wave-function is lost and we thus wrote down an unphysical state, this apparent issue dissolves when computing any observable diagonal in isospin basis. 

The single-particle orbitals $\phi^{\alpha}_{j}({\bf r}_i)$  of the involved Slater-determinants can then be computed by e.g. a mean-field approach. For closed-shell systems, single-determinant wave-functions often provide good approximations of the nodal surface. There exist several common approaches to improve upon such a wave-function. For small systems, a multideterminant wave-function can provide excellent results. However, the number of required determinants quickly becomes excessively high for large system sizes \cite{rios2006inhomogeneous}.
Instead, we here make use of Backflow transformations \cite{lee1981green, kwon1993effects, kwon1998effects}.

Formally justified in the context of Fermi liquid theory and the homogeneous electron gas \cite{kwon1993effects}, backflow transformation are introduced by evaluating the single-particle orbitals $\phi_j^{\alpha}({\bf r}_i)$ at a set of collective coordinates $({\bf x}_i, ...{\bf x}_j)$. More concretely, the determinal part of the wave-function with backflow transformations is then given by \cite{rios2006inhomogeneous}
\begin{align}
\Psi^{BT}_D({\bf r}_1,...{\bf r}_N) =\Psi_D({\bf x}_1,...{\bf x}_N),
\end{align}
where the new coordinates are given by
\begin{align}
{\bf x}_i={\bf r}_i+{\bf \xi}_i ({\bf r}_1,...{\bf r}_N).
\end{align}

The backflow displacement is typically parametrized using two-body interparticle distances \cite{lee1981green, kwon1993effects, schmidt1981structure, rios2006inhomogeneous}
\begin{align}
\xi_i^{\alpha_i \alpha_j} =\sum \limits_{j \neq i}^{N} \eta^{\alpha_i \alpha_j}_{ij} {\bf r}_{ij}, \\  {\bf r}_{ij} = {\bf r}_{i} -{\bf r}_{j} \label{eq:backflow}
\end{align}
where the number of electrons is given by $N$ and $\eta_{ij}^{\alpha_i \alpha_j}=\eta^{\alpha_i \alpha_j}(r_{ij})$ is a function that depends on the distance between two electrons, and their isospins $\alpha_i$, $\alpha_j$. We capitalize on the symmetries of the model with respect to the isospin in the same way as for the Jastrow factor, see Eqs.~(\ref{eq:usym1})-(\ref{eq:vsymn}). If the considered system includes nuclei, additional electron-nucleus and electron-electron-nucleus terms are typically added. Here, we consider all-electron systems such that the term Eq.~\ref{eq:backflow} is sufficient. 

We parametrize the two-electron function $\eta_{ij}$ using the very generic Bspline interpolation form implemented in {\it qmcpack} \cite{kim2018qmcpack}
\begin{align}
\eta^{\alpha_i \alpha_j}(r_{ij})=\sum \limits_{m=0}^M p_m B_3 \big( \frac{r_{ij}}{r_C /M}-m \big), \label{eq:backflowParam}
\end{align}
where the cardinal cubic B-spline function $B_3(x)$ is centered at $x=-1$ and defined on the interval $x\in [-3,1)$. The optimizable parameters are given by the $M$ control points $p_m$. Throughout or simulations, we use $8$ control points.
While the function $\eta$ is expected to decay as $r_{ij}^{-5/2}$ (for the 2D isotropic electron gas) \cite{kwon1993effects}, the above parametrization implies cutting the backflow function smoothly at the cutoff radius $r_C$. The advantages of this cutoff lie in computational efficiency and compatibility with periodic boundary conditions.
We however note here, that one could in principle employ a similar parametrization as we detail above for the Jastrow factor \cite{whitehead2016jastrow} in order to maintain computational efficiency while at the same time allowing to take advantage of the whole simulation cell. However, as the long-range effects of the backflow transformation are not as relevant as the correlations represented by the Jastrow factor, we keep the simple form \ref{eq:backflowParam}.

\subsubsection{Effective anisotropy}
Beyond the evaluation of the single-particle orbitals at generalized coordinates, the choice and filling of orbitals represents another important degree of freedom in the parametrization. Since we study fluid phases in a periodic simulation cell, we use plane-wave orbitals. The filling of these orbitals is, unlike the isotropic case, not protected by any symmetry. In particular, the effective shape of the Fermi surface may be renormalized by interactions. Since our model is based on a parabolic dispersion relation and the effective mass approximation, we parametrize this interaction-driven reshaping of the Fermi surface with a single parameter $\tilde{\eta}_{\ssm FS}$, that defines an ellipse in the same way as the bare anisotropy $\eta$. We do not optimize $\tilde{\eta}_{\ssm FS}$ using stochastic reconfiguration together with the other parameters as it determines the filling of the orbitals. Instead, we scan through a range of $\tilde{\eta}_{\ssm FS}$ and pick the solution with lowest energy.  In order to keep the computational cost low, the optimal effective anisotropy for each density and anisotropy is found using a Slater-Jastrow wave-function. Selected tests throughout the phase diagram were performed to confirm that this value is not changed (within the precision of the discrete $\tilde{\eta}_{\ssm FS}$-grid) when Backflow transformations are added.

In all our simulations, the effective anisotropy $\tilde{\eta}_{\ssm FS}$ is assumed to be the same in both valleys. In the case of the considered state with partial polarization (3 of 4 Fermi pockets filled, with the same density each), we note that no symmetry formally justifies this choice. 
However, the energy as a function of effective anisotropy has a plateau around the minimum for all states except the SVP (see Fig.~3 of the main part), where the change in energy is small in comparison to the energy difference to states with other isospin polarization. In addition, this plateau is in the same regime of $\tilde{\eta}_{\ssm FS}$'s for all states except the SVP. Thus, we do not expect a significant effect by allowing for different effective anisotropies in both valleys.

%{\it Trial moves.} When variational Monte Carlo is used for a spin system on a lattice, a local trial move can simply consist of flipping one spin. Here, the situation is more involved since one has to account for the continuous nature of the problem. In particular, particles can move around freely in the simulation cell. For the spatial part of the sample, we thus consider a local single-particle update drawn from the distribution
%\begin{align}
%T({\bf r}_0, ..., {\bf r}_i \to {\bf r}'_i, ...{\bf r}_{N})=\frac{1}{4\pi \tau} \exp \bigg( -\frac{({\bf r}'_i -{\bf r}_i)^2 }{4\tau}\bigg),
%\end{align}
%where the `timestep' $\tau$ is chosen such that the correlation length of the created samples in minimal - as a rule of thumb, this amounts to setting $\tau$ such that the Metropolis Monte Carlo acceptance probability is around $0.5$ \cite{casinomanual}.
%We use the above single-particle trial moves but note that more efficient sampling can be achieved by more involved sampling schemes, such as e.g. including a drift in the determination of a trial move \cite{kim2018qmcpack}.

\section{Evaluation of the Hamiltonian}
The ground-state energy is estimated by minimizing the variational energy, which requires evaluating the expectation of the Hamiltonian.
We detail below the evaluation of the considered Hamiltonian
\begin{align}
H=-\sum \limits_{i} \frac{1}{2m^{*}}\big(\eta^{\tau_i/2} \partial_{i,x}^2+ \eta^{-\tau_i/2} \partial_{i,y}^2 \big) + \sum \limits_{i<j} V(|{\bf r}_i- {\bf r}_j|),
\end{align}
consisting of an anisotropic valley-dependent kinetic energy and a dual-gate screened potential. The above Hamiltonian is defined in free space (in the thermodynamic limit). We further explain below the evaluation of the Hamiltonian with simulations performed in a finite simulation cell with periodic boundary conditions.

\subsection{Kinetic energy}
We perform simulations where each electron has a fixed isospin, since the Hamiltonian does not mix isospins. This approach has the advantage of reduced computational cost, as we do not need to keep track of a spinor part of the wave-function. However, the resulting wave-function is evidently unphysical. In particular, in the construction of the trial wave-function it is only possible to anti-symmetrize within the same isospin sector when keeping the isospin of each particle fixed. This apparent unphysical behaviour of the wave-function can be resolved when considering expectation values: For isospin-independent operators, the expectation value is the same as  the expectation value of the fully antisymmetrized version of this wave-function. However, the kinetic part of the Hamiltonian
\begin{align}
H_{\ssm kin}=-\sum \limits_{i} \frac{1}{2m^{*}}\big(\eta^{\tau_i/2} \partial_{i,x}^2+ \eta^{-\tau_i/2} \partial_{i,y}^2 \big)
\label{eq:Hkin}
\end{align}
is valley-dependent. We show below that this dependence does not pose an obstacly to the evaluation of the Hamiltonian with a trial wave function that is not antisymmetric with respect to exchange between isospin sectors.
Since all parts of the Hamiltonian are spin-independent we will for readability only focus on the valley degree of freedom. In particular, we define the valley eigenstates $|+1\rangle$ (electron in valley $Y$) and $|-1\rangle$ (electron in valley $X$). These are eigenstates to the operator $\hat{\tau}^z$:
\begin{align}
\hat{\tau}|\bar{\tau}\rangle=\bar{\tau}|\bar{\tau}\rangle, \ \ \bar{\tau}\in \{ +1, -1 \}.
\end{align}
Since we consider many-particle states with fixed isospin polarization, we introduce $N_{\ssm X}$ ($N_{\ssm Y}$) as the number of particles in eigenstate $|+1\rangle$ (eigenstate $|-1\rangle$).


Let us now re-write the kinetic part of the Hamiltonian~(\ref{eq:Hkin}) in more abstract terms. 
We can write it as a sum of the $x$- and $y$ part
\begin{align}
H_{\ssm kin}=H_{{\ssm kin},x}+H_{{\ssm kin}, y}
\end{align}
Both parts are of the same structure. In the following, we will just consider $H_{{\ssm kin},x}$ for simplicity, but the same arguments hold for the $y$-part.
Generally, $H_{{\ssm kin},x}$ consists of single-particle operators of the form
$\hat{O}_i f(\hat{\tau}_i^z)$, where $\hat{O}_i$ is an isospin-independent operator $\partial_{i,x}^2$ acting on particle $i$. $f(\hat{\tau}_i^z)$ is a function that involves the operator $\hat{\tau}_i^z$, corresponding to $\hat{\tau}^z$ acting on particle $i$. Here, $f(\hat{\tau}_i^z)=\eta^{ \hat{\tau}_i^z}$.
Then,
\begin{align}
H_{{\ssm kin},x}=\sum_n \hat{O}_n f(\hat{\tau}_n^z).
\label{eq:Hkin_f}
\end{align}

Now, we consider the spacial part of our trial wave-function $\phi (r_1, \tau_1, ...,r_n, \tau_n)$. By fixing the valley degree of freedom for each particle we do not have to keep track of the spinor part of the wave-function during the simulation. Including this part however explicitly results in the complete form of the trial wave-function
\begin{align}
\Psi_{\Lambda}(r_1, ... r_N)=\phi (r_1, \bar{\tau}_1, ...,r_n, \bar{\tau}_n) \zeta (\bar{\tau}_1, ...\bar{\tau}_N), \\
\zeta (\bar{\tau}_1, ... \bar{\tau}_N)=|\bar{\tau}_1\rangle ... |\bar{\tau}_N \rangle, \\
\bar{\tau}_1= ... =\bar{\tau}_{N_{\ssm X}}=+1, \ \ \bar{\tau}_{N_{\ssm X}+1}= ... =\bar{\tau}_{N_{\ssm X}+N_{\ssm Y}}=-1
\end{align}
We have used the notation $\bar{\tau}$ to distinguish the isospin {\em eigenstate} (label) from the position {\em variable}. Further, we will assume that $\Psi_{\Lambda}$ is antisymmetric with respect to exchange within the same isospin sector, but not between sectors.
Antisymmetry between sectors can be explicitly enforced:
\begin{align}
\Psi_{\Lambda}^{AS}(r_1, ... r_N)=\frac{1}{\mathcal{\sqrt{N}}}\sum_i (-1)^{{\rm sgn} (P_i)} \phi (r_{P_i(1)}, \bar{\tau}_{P_i(1)}, ...,r_{P_i(N)}, \bar{\tau}_{P_i(N)}) \zeta (\tau_{P_i(1)}, ...\tau_{P_i(N)}).
\label{eq:psiAS}
\end{align}
The sum goes over all $\mathcal{N}$ permutations $P_i$ that exchange particles between isospin sectors. 

We now want to show that
\begin{align}
\frac{\langle \Psi_{\Lambda} | H_{{\ssm kin},x} | \Psi_{\Lambda} \rangle}{\langle \Psi_{\Lambda}| \Psi_{\Lambda}\rangle}\overset{!}{=}\frac{\langle \Psi_{\Lambda}^{AS} | H_{{\ssm kin},x } | \Psi_{\Lambda}^{AS} \rangle}{\langle \Psi_{\Lambda}^{AS}| \Psi_{\Lambda}^{AS}\rangle}.
\end{align}
Using $\langle \Psi_{\Lambda}^{AS}| \Psi_{\Lambda}^{AS}\rangle=\langle \Psi_{\Lambda}| \Psi_{\Lambda}\rangle$, the above equation simplifies to
\begin{align}
\langle \Psi_{\Lambda} | H_{{\ssm kin},x} | \Psi_{\Lambda} \rangle \overset{!}{=}\langle \Psi_{\Lambda}^{AS} | H_{{\ssm kin},x } | \Psi_{\Lambda}^{AS} \rangle 
\label{eq:eqshow}
\end{align}
Re-writing the right hand side of the above equation using Eqs.~(\ref{eq:Hkin_f}) and (\ref{eq:psiAS}) yields
\begin{align}
\langle \Psi_{\Lambda}^{AS} | H_{{\ssm kin},x } | \Psi_{\Lambda}^{AS} \rangle =&\frac{1}{\mathcal{N}}\int dr_1 ... d r_N \sum_i (-1)^{{\rm sgn} (P_i)}\phi (r_{P_i (1)}, \bar{\tau}_{P_i(1)}, ...,r_{P_i (N)}, \bar{\tau}_{P_i (N)}) \zeta (\bar{\tau}_{P_i(N)}, ...\bar{\tau}_{P_i(N)} ) \nonumber \\ 
&\times\sum_n \hat{O}_n f(\hat{\tau}_n^z)  \sum_j (-1)^{{\rm sgn} (P_j)}\phi (r_{P_j (1)}, \bar{\tau}_{P_j(1)}, ...,r_{P_j (N)}, \bar{\tau}_{P_j (N)}) \zeta (\bar{\tau}_{P_j(N)}, ...\bar{\tau}_{P_j(N)})  =(*)
\end{align}

Using the orthogonality of the spinor basisfunctions $\zeta$ and the fact that they are $z$-eigenstates such that $\hat{\tau}_n^z\zeta(\bar{\tau}_1 ...\bar{\tau}_N)= \bar{\tau}_n\zeta(\bar{\tau}_1 ...\bar{\tau}_N)$, we obtain
\begin{align}
(*) =&\frac{1}{\mathcal{N}}\sum_i\int dr_1 ... d r_N \phi (r_{P_i (1)}, \bar{\tau}_{P_i(1)}, ...,r_{P_i (N)}, \bar{\tau}_{P_i (N)}) \zeta (\bar{\tau}_{P_i(N)}, ...\bar{\tau}_{P_i(N)} ) \nonumber \\ 
&\times\sum_n \hat{O}_n f({\bar{\tau}}_n)  \phi (r_{P_i (1)}, \bar{\tau}_{P_i(1)}, ...,r_{P_i (N)}, \bar{\tau}_{P_i (N)}) \zeta (\bar{\tau}_{P_i(N)}, ...\bar{\tau}_{P_i(N)}).
\end{align}
Using commutativity of addition and the integration order we arrive at
\begin{align}
(*) =&\frac{1}{\mathcal{N}}\sum_i\int dr_{P_i (1)} ... d r_{P_{i}(N)} \phi (r_{P_i (1)}, \bar{\tau}_{P_i(1)}, ...,r_{P_i (N)}, \bar{\tau}_{P_i (N)}) \zeta (\bar{\tau}_{P_i(N)}, ...\bar{\tau}_{P_i(N)} ) \nonumber \\ 
&\times \big( \hat{O}_{P_i(1)} f({\bar{\tau}}_{P_i(1)}) + ...+  \hat{O}_{P_i(N)} f({\bar{\tau}}_{P_i(N)})\big)  \phi (r_{P_i (1)}, \bar{\tau}_{P_i(1)}, ...,r_{P_i (N)}, \bar{\tau}_{P_i (N)}) \zeta (\bar{\tau}_{P_i(N)}, ...\bar{\tau}_{P_i(N)}) \\
=&\int dr_{1} ... d r_{N} \phi (r_{1}, \bar{\tau}_{1}, ...,r_{N}, \bar{\tau}_{N}) \zeta (\bar{\tau}_{N}, ...\bar{\tau}_{N} )  \big( \hat{O}_{1} f({\bar{\tau}}_{1}) + ...+  \hat{O}_{N} f({\bar{\tau}}_{N})\big)  \phi (r_{1}, \bar{\tau}_{1}, ...,r_{N}, \bar{\tau}_{N}) \zeta (\bar{\tau}_{N}, ...\bar{\tau}_{N})
\end{align}
where we have renamed integration variables in the last step and thus demonstrated the equality~(\ref{eq:eqshow}).
It directly follows that
\begin{align}
\frac{\langle \Psi_{\Lambda} | H_{{\ssm kin}} | \Psi_{\Lambda} \rangle}{\langle \Psi_{\Lambda}| \Psi_{\Lambda}\rangle}=\frac{\langle \Psi_{\Lambda}^{AS} | H_{\ssm kin } | \Psi_{\Lambda}^{AS} \rangle}{\langle \Psi_{\Lambda}^{AS}| \Psi_{\Lambda}^{AS}\rangle}.
\end{align}


%Maybe short: Evaluation as explained in Casino manual.


\subsection{Dual-gate screened interaction}
\label{dual-gateV}

In a simulation cell with periodic boundary conditions, the interaction between electrons has to be evaluated as a sum over all image charges:
\begin{align}
V &=\sum_{\bf L}  \sum_{i<j} V(|{\bf r}_i-{\bf r}_j + {\bf L}|)+ V_{\ssm Mad}+ V_{\ssm b}, \label{eq:Vsum} \\
V_{\ssm Mad} &=N\sum_{\bf L \neq 0} V(|{\bf L}|)
\end{align} 
The sum runs over all lattice vectors ${\bf L}$ of the simulation cell. The Madelung energy $V_{\ssm Mad}$ is a constant contribution that arises from interactions between particles and their own images. Furthermore, we consider all-electron systems. The contribution of the positive background is given by $V_{\ssm b}$, and corresponds to
\begin{align}
V_{\ssm b}=-\frac{1}{2}v({\bf q}=0)n^2 L^2,
\end{align}
where $n$ is the density, $L^2$ the area of the simulation cell in two dimensions and $v({{\bf q}=0})$ the $q=0$ component of the Fourier transform of the potential $V$.

While the above expression~(\ref{eq:Vsum}) is diagonal in real-space and can thus in principle be directly computed, the sum over all lattice vectors does not converge when long-ranged interactions such as the Coulomb potential are considered. The divergence cancels out with the (also diverging contribution) of the positive background, which cannot be translated into a real-space cutoff in the sum over lattice vectors.
This challenge is typically solved by breaking the interaction in a part that is short-ranged in real space and a part that is short-ranged in reciprocal space, for instance using Ewald summation \cite{sangster1976interionic, toukmaji1996ewald}.

For comparison with experimental observations however, implementation of a potential that is externally screened  may yield a more realistic comparison. Considering a two-dimensional electron system, if the potential is externally screened by two metal gates the sum over lattice vectors converges due to the exponential decay in real-space of the resulting potential \cite{throckmorton2012fermions}. Thus, it can directly be computed in real space. In addition, the positive background can be directly substracted since $v({\bf q}=0)$  is finite. Below, we derive the real-space form of the dual-gate screened potential and explain the concrete implementation in our simulations.

Concretely, we consider a two-dimensional electron system with two metal gates above and below, separated by distance $2d$. The gates induce screening of the Coulomb interaction within the electron system. The form of the obtained screened potential is well-known in Fourier space
%V(|{\bf r}_i-{\bf r}_j|)=\int {\rm d} {\bf q}\, {\rm e}^{{\rm i}{\bf qr}}v(|{\bf q}|), \\
\begin{align}
v({\bf q})=  \frac{e^2}{2 \epsilon_0 \epsilon} \frac{\tanh (d |{\bf q}|)}{|{\bf q}|},
\end{align}
Performing VMC calculations however requires access to the potential in real space. The Fourier transform
\begin{align}
V(r)=\frac{1}{(2\pi)^2}\int_{\mathbb{R}^2} {\rm d} {\bf q}\, {\rm e}^{{\rm i}{\bf qr}}v(|{\bf q}|)
\end{align}
is not analytically solvable. Instead, we directly compute the potential $V(r)$ in real space using the method of images. Concretely, we will repeat here the derivation given in \cite{throckmorton2012fermions}.

Two metal gates introduce infinite ``columns'' of image charges above and below the sample.
Concreteley, the resulting interaction potential between two electrons is given by \cite{throckmorton2012fermions, goodwin2020critical}
\begin{align}
V(r)=\frac{e^2}{4\pi \epsilon \epsilon_0} \sum_{n=-\infty}^{n=\infty}\frac{(-1)^n}{\sqrt{r^2+(2dn)^2}}.
\end{align}
Now, we proceed to compute the above series: We re-rewrite the sum by using the identity
\begin{align}
\frac{1}{r}=\frac{2}{\sqrt{\pi}}\int_0^{\infty} du e^{-r^2 u^2}.
\end{align}
Then, we obtain
\begin{align}
V(r)&=\frac{2}{\sqrt{\pi}}\frac{e^2}{4\pi \epsilon \epsilon_0 2d} \sum_{n=-\infty}^{n=\infty}\int_0^{\infty} du(-1)^n e^{-(r/(2d))^2u^2 -n^2u^2} \\
&=\frac{2}{\sqrt{\pi}}\frac{e^2}{4\pi \epsilon \epsilon_0 2d} \int_0^{\infty} du e^{-(r/(2d))^2u^2} \sum_{n=-\infty}^{n=\infty}(-1)^n e^{ -n^2u^2} \\
&= \frac{2}{\sqrt{\pi}}\frac{e^2}{4\pi \epsilon \epsilon_0 2d} \int_0^{\infty} du e^{-(r/(2d))^2u^2} \theta_4 (0, e^{-u^2}),
\end{align}
where we used the Jacobi theta function
\begin{align}
\theta_4(z,q)=\sum_{n=-\infty}^{\infty} (-1)^n q^{n^2} e^{2niz}.
\end{align}
Making use of the identity \cite{wolframurl} 
\begin{align}
\theta_4(z,q)=\frac{2\sqrt{\pi}}{\sqrt{-\ln q}}e^{(4z^2+\pi^2)/4\ln q} \sum_{k=0}^{\infty} e^{k(k+1)\pi^2/\ln q} \cosh \bigg( \frac{(2k+1)\pi z}{\ln q}\bigg)
\end{align}
we insert
\begin{align}
\theta_4(0,e^{-u^2})=\frac{2\sqrt{\pi}}{u} \sum_{k=0}^{\infty} e^{-(k+1/2)^2\pi^2/u^2} 
\end{align}
and obtain \cite{throckmorton2012fermions}
\begin{align}
V(r)&=4\frac{e^2}{4\pi \epsilon \epsilon_0 2d}\sum_{k=0}^{\infty}  \int_0^{\infty} du \frac{1}{u} e^{-(r/(2d))^2u^2} e^{-(k+1/2)^2\pi^2/u^2}  \\
&=4\frac{e^2}{4\pi \epsilon \epsilon_0 2d}\sum_{k=0}^{\infty} K_0 \bigg( (2k+1) \pi \frac{r}{2d}\bigg),
\label{eq:Vrscreened}
\end{align}
where we have evaluated the integral using modified Bessel functions of the second kind $K_n(x)$. The behaviour of the potential at large distances can be understood using the large-$x$ limit of $K_n(x)$:
\begin{align}
K_n(x) \approx \sqrt{\frac{\pi}{2x}}e^{-x}.
\label{eq:Blarge}
\end{align}
Thus, at large distance $r$
\begin{align}
V(r) \approx 4\frac{e^2}{4\pi \epsilon \epsilon_0 2d}\sum_{k=0}^{\infty}\sqrt{\frac{1}{(2k+1) r/d}}e^{-(2k+1)\pi r/(2d)} \sim \frac{1}{\sqrt{r}} e^{-\pi r/(2d)},
\end{align}
since the most dominant contribution for $r\gg d$ comes from the $k=0$ contribution and all other contributions are exponentially smaller. The decay of the dual-gate screened potential is thus approximately exponential in real space at large distances (with sub-leading pre-factor $1/\sqrt{r}$).
For numerical evaluation, we make use of the exact formula~(\ref{eq:Vrscreened}). In particular, due to the long-range behaviour of the modified Bessel functions~(\ref{eq:Blarge}), the sum can be truncated at $k\sim \mathcal{O}(d/r)$ and numerically computed. For very small $r$ when the evaluation becomes infeasible, the effect of screening is at the same time negligible and the bare Coulomb interaction can be used instead. 
In order to keep the computational cost low and avoid the evaluation of the sum~(\ref{eq:Vrscreened}) during runtime, we pre-evaluate the screened potential on a dense grid and load the stored values during the VMC simulations. In particular, we use an interval $r\in(\epsilon, l)$ where $\epsilon$ is very small and sufficiently $l$ large such that $V(l) \approx 0$. The potential is then evaluated at arbitrary $r$ by piecewise linear interpolation, if $r$ is in the pre-evaluated interval. For $r\geq l$, the potential is simply set to zero. For $r<\epsilon$, it is fitted to an $1/r$ behaviour chosen such that the overall potential is continuous (almost exactly corresponding to the bare Coulomb interaction). Here, we use $\epsilon \sim 2\times 10^{-3}a$, where $a$ is the lattice constant of AlAs. The length $l$ is an integer multiple of the simulation cell length, chosen such that $V(l)<10^{-10}$ meV.  

Further, for typical densities and particle numbers the screening length will be larger than one simulation cell. Thus, we evaluate the potential by summing over image cells. This sum given in Eq.~(\ref{eq:Vsum}) quickly converges due to the exponential decay of the interaction. Thus, we can truncate it after a finite number of terms. Throughout most of the simulations, we used the distance $d=100$ nm. For this value, the number of image cells required for evaluation of the potential ranges between $\mathcal{O}(1)$ to $\mathcal{O}(10)$ for the simulated densities and particle numbers. 

Due to the exponential decay of the potential in real space, the Madelung energy $V_{\ssm Mad}$ becomes negligible already at finite, sufficiently large system sizes. We thus directly set it to zero during the simulations such that there is no necessity to account for it in the finite-size scaling analysis.

We have varied the screening length in between $d=70$ and $d=300$ nm (for selected densities within the anisotropy $\eta=5.79$) and found that the effect on the estimated phase boundaries is of similar order as ambiguities in the phase boundary estimation resulting from phenomenological fitting functions. It is however possible that a comparably much enhanced screening or no screening at all could have a more enhanced effect on phase diagram.

We note however, that the effect of the varied screening length on the finite-size scaling - thus, on the finite-size contributions - is striking: The exponent (as explained in the next section) of the finite-size fit strongly varies with screening length.



\section{Finite-size scaling}

The goal of realistic numerical many-body approaches such as quantum Monte Carlo methods lies in producing reliable ground state properties. In the vast majority of use-cases such as the simulation of (here 2D) bulk systems this corresponds to systems close to the thermodynamic limit. However, calculations are necessarily performed with a finite number of electrons: 
For a given density $n$, the simulations are we use a finite number of electrons $N$ in a simulation cell of area $L^2=n/N$. Thus, the arising finite-size errors have to be accounted for.
Despite the existence of sophisticated methods for the correction of finite-size errors \cite{drummond2008finite, kwee2008finite, holzmann2016theory}, a largely successful practice in numerical studies of condensed matter consists of extrapolating finite-size energies to infinite system size, using an assumed relationship between energy and particle number.
Within QMC simulations, this relation typically takes into account scaling behaviour arising from finite-size corrections on the kinetic energy as well as resulting from a compression of the exchange-correlation hole into the simulation cell.
For a Fermi fluid within the isotropic 2D electron gas without external screening, careful consideration of the contributions to the scaling behaviour has resulted in the two-parameter form
\begin{align}
e_{\inf}=e(N)+a \Delta t_{\ssm HF}(N)+cN^{-\gamma},
\end{align}
introduced in \cite{ceperley1980ground}. Here, $e_{\inf}$ corresponds to the energy in the thermodynamic limit and is used as a fitting parameter together with $a$ and $c$. The finite-size energies are given by $e(N)$. The exponent $\gamma$ has been discussed in literature, the first result $\gamma=3/2$ \cite{ceperley1980ground} has been later improved to the more accurate $\gamma=5/4$ in \cite{drummond2008finite}.
$\Delta t_{\ssm HF}(N)$ corresponds to the difference in the infinite and finite-size kinetic energy within Hartree-Fock, and captures the finite-size effects that arise from discrete filling of the Fermi surface in metallic states. This effect can also directly be alleviated by considering offsets $(\Delta_x,\Delta_y)$ in the (2D) $k$-grid, which can be understood as twists in the simulation cell boundary conditions
\begin{align}
\Psi(x+nL,y+mL)=e^{{\ssm} i(n\theta_x+m\theta_y)}\Psi(x,y),
\end{align}
with $\theta_x=\Delta_x L$ and $\theta_y=\Delta_y L$. Pure periodic boundary conditions correspond to the choice $(\Delta_x,\Delta_y)=(0,0)$.
A smarter choice of the twist or averaging over random offsets (``twist averaging'' \cite{lin2001twist}) can be used to achieve $\Delta t_{\ssm HF}(N)=0$ directly.
Computational cost can be saved by considering particular twists. Concretely, we here implement the ``special twist'' method introduced in \cite{dagrada2016exact}: Simulations for a given state and electron number are performed with the same offset, which is chosen such that $\Delta t_{\ssm HF}(N)=0$. This condition does not uniquely define the special twist, leaving an ambiguity in the choice. However, it has been shown that different choices in the special twist lead to very similar results, as long as the condition $\Delta t_{\ssm HF}(N)=0$ is fulfilled \cite{dagrada2016exact}. We here choose offsets along the diagonal $(k_x,k_y)=(\alpha,\alpha)$ in order two treat both valleys on equal footing. Concretely, for a given density, number of particles, anisotropy and Fermi sea filling the diagonal $(\alpha,\alpha)$ is scanned and the condition $\Delta t_{\ssm HF}(N)=0$ checked for each value of $\alpha$. If a value of $\alpha$ can be found where the condition is fulfilled (for a chosen particle number), it is used for a simulation of the considered trial wave-function. We note that a special twist does not exist for all particle numbers and Fermi sea fillings, in particular when the effective anisotropy of the Fermi surface varies strongly from the bare anisotropy. However, in all our simulations we found a sufficient amount of special twists to perform finite-size extrapolation as further detailed below.  
% We note however, that a different choice (e.g. along the $x$-direction) leads to very similar results [].

% Figure environment removed

We consider an anisotropic electron gas with dual-gate screened interaction and thus the exponent $\gamma=5/4$ derived for an unscreened potential does not apply here.
For very large lattice sizes, the exponential decay of the screened interactions is expected to result in all correlations decaying to zero within one simulation cell and thus a negligible finite-size error. For the considered lattice sizes however, the correlations surpass the size of the simulation cell and thus the regime of trivial behaviour in the finite-size error is not reached.
An accurate finite-size scaling behaviour requires careful consideration of the contributing corrections \cite{chiesa2006finite, drummond2008finite, holzmann2016theory}. We detail below that a rough estimation of the leading-order corrections is not sufficient for a well-behaved and reliable fit for the considered model, and instead extrapolate to the thermodynamic limit by including $\gamma$ as (anisotropy- and density-dependent) parameter that is fitted together with $e_{\inf}$ and $c$. The density-dependence of the exponent is effectively a result of finite-size-scaling contributions of different order with density-dependent prefactors. Despite the strongly phenomenological choice, the chosen fitting function yields a largely accurate fit when $r_s \lesssim 27$ (i.e. when the is system expected to be in a metallic phase, following the experimental predictions \cite{hossain2021spontaneous}) for the simulated system sizes, as shown exemplary in Fig.~\ref{fig:scaling} for $\eta=3$, $r_s \approx 15.4$. We note that we observe a more significant mismatch between simulated finite-size behavior and assumed phenomenological fit for the largest simulated $r_s \approx 30$, inducing a larger error in the finite-size scaling. We leave this observation to analyze in future studies. Further, we justify the phenomenological choice with our interest only in energy differences, which are less sensitive to the choice of finite-size extrapolation than absolute energies (with a sufficient amount of simulated particle numbers).

We detail below, the approximate estimation of contributions of the leading-order finite-size corrections and argue that they do not suffice for a reliable extrapolation.
As an instructing starting point for an estimate of the finite-size error on the potential $V_N$, we can write the electron-electron potential energy per particle for a simulation cell of area $L\times L$ containing $N$ electrons in Fourier space \cite{chiesa2006finite}
\begin{align}
V_N=\frac{1}{L^2}\sum \limits_{{\bf k}\neq 0} v({\bf k}) (\rho_{\bf k}\rho_{-\bf k}/N-1),
\label{eq:U_N}
\end{align}
where $\rho_{\bf k}:=\sum_j^N \exp (i {\bf k \cdot r}_j)$, and $v_k$ corresponds to the Fourier component of the dual-gate screened potential
\begin{align}
v({\bf k})=\frac{ e^2}{2 \epsilon_0 \epsilon}\frac{\tanh (d k)}{k},
\end{align}
with $k=|{\bf k}|$. The electron charge is given by $e$ and the metal-gate distance by $d$.
A finite-size error in Eq.~\ref{eq:U_N} is induced by the discrete $k$-mesh. As the system size increases, the mesh gets finer until the sum in (\ref{eq:U_N}) eventually converges to an inegral.

One can thus write the error on the potential energy per particle as the difference between integral and discrete sum
\begin{align}
\Delta V_N =\frac{1}{4\pi^2}\int v({\bf k}) (S({\bf k})-1) d{\bf k} - \frac{1}{L^2}\sum \limits_{\bf k \neq 0} v({\bf k}) (S_N({\bf k})-1),
\end{align}
where we used the static structure factor $S_N({\bf k})=\langle \rho_{\bf k} \rho_{-{\bf k}} \rangle/N$, and $S({\bf k})$ is the structure factor in the thermodynamic limit.

The leading order contribution is given by
\begin{align}
\Delta_1 =-\frac{1}{4\pi^2}\int v({\bf k}) d{\bf k}  + \frac{1}{L^2}\sum \limits_{\bf k \neq 0} v({\bf k}) .
\end{align}
 which is an integration error that arises due to the omission of the ${\bf k}=0$ area element from the sum. To leading order, the scaling of the error with number of particles can by estimated by considering this missing contribution, i.e. $\int_{\mathcal D} v_k d{\bf k}$, where $\mathcal{D}$ is a domain with area $4\pi^2/L^2$ \cite{chiesa2006finite}. However, $\Delta_1$ corresponds to the Madelung constant that we directly set to zero during our simulations, see Sec.~\ref{dual-gateV}. Thus, we do not need to account for the finite-size correction from the Madelung constant.


 
 The next-leading order correction comes from the discretization of $\int v({\bf k}) (S({\bf k})-1) d{\bf k}$ \cite{chiesa2006finite}. Approximating $S({\bf k})\approx S_N({\bf k})$ \footnote{This approximation is justified within the random-phase approximation. This becomes apparent when decomposing the potential into a long-range and short-range part. The long-range part that exhibits finite-size errors decays fast in reciprocal space. As a consequence, only the behaviour of $S({\bf k})$ at small $k$ is relevant. In the limit of $k\to 0$, the random phase approximation becomes exact}, one can use the same estimation as above for the integration error
\begin{align}
 \Delta_2 = \frac{1}{4\pi^2}\int v({\bf k}) S({\bf k})d{\bf k}  - \frac{1}{L^2}\sum \limits_{\bf k \neq 0} v_k S({\bf k})\propto \int_{\mathcal{D}}S({\bf k})v({\bf k}) d{\bf k}.
 \label{eq:Delta2}
 \end{align}
We note here that this estimation can not be applied for a quatitative computation of the finite-size correction, as it effectively implies neglecting a term of the same order of magnitude \cite{drummond2008finite}. We still apply the estimation, as we are only interested in the scaling behaviour. However we note that the scaling behaviour may also be affected by the missing contribution, as one would need to repeat the calculation in  \cite{drummond2008finite} for an externally screened potential.
 
Due to the validity of the random-phase approximation at small $k$, $S({\bf k})$ can be calculated analytically. For the isotropic 2D electron gas, $S({\bf k}) \propto k^{3/2}$ and $\Delta_2 \propto N^{-5/4}$. 
We here apply the random-phase approximation to compute the static structure factor $S({\bf k})$ in presence of a dual-gate screened interaction. We follow the derivation given in \cite{giuliani2005quantum} for the unscreened Coulomb potential and detail below only the differing steps using a dual-gate screened potential. Concretely, we make use of the relationship \cite{giuliani2005quantum}
\begin{align}
S({\bf k})=-\frac{\hbar}{\pi n}\int_0^{\infty} \mathcal{I}m \chi_{nn}( k,\omega) d\omega,
\label{eq:Sq}
\end{align}
where $\chi_{n n}(k,\omega)$ is the density-density response function. We use the approximation of the density-density response function obtained within the random-phase approximation
\begin{align}
\chi^{RPA}_{nn}(k,\omega)=\frac{\chi_0 (k,\omega)}{1-v_k \chi_0 (k,\omega)}.
\label{eq:chiRPA}
\end{align}
We introduced the notation $v_k :=v({\bf k})$ in order to simplify the consistency between vector and scalar objects.
Here, $\chi_0 (k, \omega)$ is the Lindhard function (density-density response function of independent electrons). 

In order to compute the imaginary part of the density-density response function, it is useful to search for its poles. We note here that the poles have also a direct physical meaning: The poles in the lower half of the complex frequency plane correspond to the frequencies of collective modes, giving rise to sharp peaks (resonances) in the spectral function. At long wavelength (small $k$), it is well-known that the spectrum of the density fluctuations within the random-phase approximation is dominanted by a collective excitation known as the {\it plasmon} \cite{giuliani2005quantum}.

We now turn to the calculation of the poles of $\chi^{RPA}_{nn}$.
They arise from zeros of the denominator of Eq.~(\ref{eq:chiRPA}), since the Lindhard function has no poles (only a branch cut along the real axis). Thus, the poles are given as solutions $\Omega_p$ of the equation
\begin{align}
1-v_k \chi_0 (k,\Omega_p)=0.
\label{eq:plasmonEq}
\end{align}
We use the small-$k$ expansion of the Lindhard function in two dimensions \cite{giuliani2005quantum} 
\begin{align}
\chi_0(k,\omega )\approx \frac{n k^2}{m \omega^2}[1+\frac{3}{4} \frac{k^2 v_F^2}{\omega^2}],
\end{align}
where $v_F$ is the Fermi velocity.
Then, the (real) solution to Eq.~(\ref{eq:plasmonEq}) is given by
\begin{align}
\Omega_p^2=\sqrt{\bigg(\frac{v_k n k^2}{2m}\bigg)^2+\frac{3}{4}k^4 v_F^2 \frac{v_k n}{m v_F}}+\frac{v_k n k^2}{2 m}.
\end{align}

We can then approximate the behaviour of the imaginary part of $\chi_{nn}^{RPA}(k,\omega)$ in the vicinity of the plasmon frequency $\Omega_p$ for $k\to 0$ as \cite{giuliani2005quantum}
\begin{align}
-\frac{1}{\pi}\mathcal{I}m \chi_{nn}^{RPA} (k,\omega) \approx \frac{\Omega_p(k)}{2 v_k} \delta (\omega-\Omega_p(k))
\end{align}
Inserting into Eq.~(\ref{eq:Sq}) yields
\begin{align}
S^{RPA}({\bf k})= \frac{\hbar}{n} \frac{\Omega_p (k)}{2 v_k}.
\end{align}
For the dual-gate screened interaction, the structure factor then approximately scales as $S({\bf k}) \propto k^{3/2} \tanh^{-1/2} (dk)$ in the small-$k$ limit.

We now want to determine the scaling of the correction $\Delta_2$ using the random-phase approximation of the structure factor. Inserting into Eq.~(\ref{eq:Delta2}), we obtain
\begin{align}
\Delta_2\propto  \int_0^{\frac{2\pi}{L}} dq \ 2\pi q S(q) v_q  \propto \int_0^{2\pi \sqrt{\frac{n}{N}}} q^{3/2}\tanh^{\frac{1}{2}}(dq). 
\end{align}
While this integral is not straightforward to evaluate, one can instead compute the integral numerically for a given density $n$ and gate-distance $d$ for a dense grid of values $1/N \in [0,1/N_{\ssm min}]$. We find that matching the obtained function with a fit $g(N)$ expanding in integer powers of $1/\sqrt{N}$ yields excellent agreement.
While we expect the anisotropy of the wave-function in principle also to enter in the finite-size correction, we did not include it in the above consideration due to the phenomenological observation that the scaling behaviour in our simulations is not very sensitive to the considered anisotropy.


Approximative scaling of the correction $\Delta_2$ is however not sufficient in order to obtain a reliable finite-size scaling. In particular, when employing the fit
\begin{align}
E_N=E_{\inf} - c  g(N),
\end{align}
the numerically obtained finite-size energies are not matched very well. In addition, we note that using an externally screened potential requires more careful consideration of the contributions to the finite-size error. More concretely, corrections from the discretized potential are much smaller than in the unscreened case: The potential is not divergent at small $k$, but finite. Thus, more subtle effects can become relevant such as further sub-leading corrections, dependence on the choice of trial wave-function \cite{holzmann2016theory} as well as higher-order corrections to the kinetic energy \cite{drummond2008finite}.


As we are further interested in energy differences, which are less sensitive to the choice of finite-size extrapolation than absolute energies (with a sufficient amount of simulated particle numbers), we choose to resort to the phenomenological fit $E_N=E_{\inf}-c N^{-\gamma}$ instead of more careful consideration of the finite-size corrections. The maximal particle numbers simulated are chosen with respect to the considered polarization, density and anisotropy. In particular, as shown in Fig.~\ref{fig:scaling}, the two states with lowest energy are simulated up to higher particle numbers (here $N=176$ and $N=162$ for the symmetric and VP state): Their energies are more relevant for the determination of the ground-state energy.
In principle, it is feasible to go simulate higher particle numbers $N>200$. Since we are constructing a phase diagram as a function of $r_s$ and $\eta$, and scan through a range of effective anisotropies at each point in the phase diagram, we perform our simulations with $N\leq 180$ to keep the overall computational cost manageable.

\section{Error bars}
In this section, we discuss the origin of the error bars on the phase boundaries (Fig. 1 in the main text) and on the extrapolated energies in the thermodynamic limit.

% Figure environment removed

\paragraph{Error bar on extrapolated energies.} The precision of the variational energy, extrapolated to infinite particle number is given by error propagation of the statistical sampling error on the finite-size errors through the fit $e(N)=e_{\inf}-a \Delta t_{\ssm HF}(N)+cN^{-\gamma}$. Ambiguities in the fitting parameters contribute via the covariance matrix of the fit, but the phenomenological choice of scaling behaviour introduces an error that is here not accounted for.

\paragraph{Error bar on phase boundaries.}
Simulations are performed on a discrete grid in the $r_s$-and $\eta$-plane: We simulate the anisotropies $\eta \in \{1,3,5.2,5.79,9,12\}$ and the Wigner-seitz radii $r_s \in \{10.9, 12.6, 15.4, 21.8, 30.8\}$ (corresponding to the densities $n \in \{2.0, 1.5, 1.0, 0.5, 0.25\}\times 10^{11}$ cm$^{-2}$. The extrapolated energies (for each simulated isospin polarization) are fitted on horizontal and vertical lines in the plane, i.e. along $r_s$ and along $\eta$.
In particular, we use the parameterization of the correlation energy suggested by Rapisarda and Senatore \cite{rapisarda1996diffusion} to perfom a fit of the energies as a function of $r_s$. Albeit derived for an unscreened potential, we find the parametrization to yield an excellent fit to our energies. As a function of $\eta$, we employ a generic polynomial fit. Error propagation is used to obtain uncertainties in the fitting parameters of the vertical and horizontal fit. The parameters are then varied within this range, and the minimal and maximal position of the respective phase boundary define the error on its estimation. For a fit along a horizontal line, this error corresponds to an uncertainty in $r_s$. Vice versa, a fit on a vertical line yields phase boundary uncertainties in $\eta$. Since the order of the used polynomial on vertical lines is somewhat arbitrary, we use polynomials with order $2,3,4$, determine the uncertainty of fitting parameters for each choice of polynomial and determine the error on the phase boundary using the minimal and maximal estimated phase boundary of {\em all} performed polynomial fits. Figure~\ref{fig:phasediagram2D_deg} demonstrates the dependence of the phase boundaries on the choice of fitting polynomial: The error bars in all $3$ subfigures are determined as detailed above. The colors, however, are obtained by first performing a fit in $r_s$ and evaluating this fit on a dense grid of $500$ values in $r_s$. Then, the so-obtained energies are interpolated with a polynom of degree $2$ (left of Fig.~\ref{fig:phasediagram2D_deg}), degree $3$ (middle) and $4$ (right) in $\eta$-direction. Evaluating the polynomial fit again with $500$ anisotropies yields energies on a $500\times 500$ grid. These energies can be used to color the phase diagram, thereby providing a guide to the eyes.

An additional error arises from the discretization in the scanned effective Fermi surface anisotropies $\tilde{\eta}_{\ssm FS}$. Figure~4 in the main manuscript shows that this error is rather small for the VP, SP and symmetric state as these showcase approximately an energy plateau around the minimum. However, the SVP state shows a stronger dependence on the effective anisotropy. Thus, more accurate results can be obtained by using a finer $\tilde{\eta}_{\ssm FS}$-grid.




\section{Hartree-Fock}
We perform Hartree-Fock calculations directly in the thermodynamic limit. Hartree-Fock can be understood as a variational method, where the trial wave-function corresponds to a ground state as a noninteracting Hamiltonian. Here, we consider only states with fixed isospin-polarization. Then, the mean-field trial wave-function $\Psi_M$ is a Slater-determinant and the optimizable degree of freedom corresponds to the orbital filling, i.e. the shape of the effective Fermi surface.
As effects beyond a parabolic dispersion are neglected in our system, we parametrize the effective Fermi surface as an ellipse with anisotropy $\tilde{\eta}_{\ssm HF}$. Then, we use the variational principle
\begin{align}
\frac{\langle \Psi_M |H| \Psi_M \rangle}{\langle \Psi_M | \Psi_M \rangle}\geq E_g
\label{eq:HFvar}
\end{align}
to obtain a ground-state approximation.
We explain in the following the concrete steps of our algorithm.
First, we determine the above expectation value~(\ref{eq:HFvar}) given a state $\Psi_M$ with effective anisotropy $\tilde{\eta}_{\ssm HF}$ and a given isospin-polarization $(n_1, n_2, n_3, n_4)$. We denote the density in isospin flavour $\alpha$ as $n_{\alpha}$, and $\sum_{\alpha} n_{\alpha}=n$, where $n$ is the total density of the system. We adopt the notation, that the states $i=1$ and $i=2$ correspond to the spin-up, spin-down states of valley $\tau=1$, and $i=3$, $i=4$ to the valley $\tau=-1$. %In order to improve the readability of the following calculation, we will introduce $Tau :=\tau/2$.


\subsection{Kinetic energy}
The kinetic energy for a state with isospin polarization $(n_1, n_2, n_3, n_4)$ is given by
\begin{align}
E_{\ssm kin}=\sum_{{\bf k}, \alpha} \hbar^2\frac{ \eta^{\tau_{\alpha}/2} k_x^2+ \eta^{-\tau_{\alpha}/2} k_y^2}{2 m^{*}}n_{{\bf k},\alpha}.
\end{align}
Here, the isospin flavour is denoted with index $\alpha$. In addition, $n_{{\bf k},\alpha}=1$ if orbital ${\bf k}, \alpha$ is filled (and $0$ otherwise).
In the thermodynamic limit, we can replace the sum with an integral $\sum_{\bf {k}}\to L^2/(2\pi)^2 \int d {\bf k}$ and write
\begin{align}
E_{\ssm kin}=\frac{L^2}{(2\pi)^2}\sum_{\alpha}\int d {\bf k} \hbar^2\frac{ \eta^{\tau_{\alpha}/2} k_x^2+ \eta^{-\tau_{\alpha}/2} k_y^2}{2 m^{*}}\Theta((k^{\alpha}_F)^2-\tilde{\eta}_{\ssm HF}^{\tau_{\alpha}/2}k_x^2-\tilde{\eta}_{\ssm HF}^{-\tau_{\alpha}/2}k_y^2),
\end{align}
where the Fermi wave-vector for the isospin flavour $\alpha$ is given as $k_F^{\alpha}=\sqrt{4\pi n_{\alpha}}$.
With the substitution 
\begin{align}
\tilde{k_x}=\tilde{\eta}_{\ssm HF}^{\tau_{\alpha}/4}k_x, \ \
\tilde{k_y}=\tilde{\eta}_{\ssm HF}^{-\tau_{\alpha}/4}k_y, \ \ 
\end{align}
we obtain
\begin{align}
E_{\ssm kin}&=\frac{L^2}{(2\pi)^2}\sum_{\alpha}\int d {\bf k} \hbar^2\frac{ (\eta/\tilde{\eta}_{\ssm HF})^{\tau_{\alpha}/2} \tilde{k}_x^2+ (\eta/\tilde{\eta}_{\ssm HF})^{-\tau_{\alpha}/2} \tilde{k}_y^2}{2 m^{*}}\Theta((k^{\alpha}_F)^2-\tilde{k}_x^2-\tilde{k}_y^2) \\
&=\frac{L^2}{(2\pi)^2}\sum_{\alpha}\int_0^{k_F^{\alpha}} d\tilde{k} \tilde{k}^3 \int_0^{2\pi}  d\theta   \hbar^2\frac{ (\eta/\tilde{\eta}_{\ssm HF})^{\tau_{\alpha}/2} \cos(\theta)^2+ (\eta/\tilde{\eta}_{\ssm HF})^{-\tau_{\alpha}/2} \sin(\theta)^2}{2 m^{*}} \\
&=\frac{L^2}{(2\pi)^2}\pi \hbar^2\frac{ (\eta/\tilde{\eta}_{\ssm HF})^{\tau_{\alpha}/2} + (\eta/\tilde{\eta}_{\ssm HF})^{-\tau_{\alpha}/2} }{2 m^{*}}\sum_{\alpha}\frac{1}{4}(k^{\alpha}_F)^4 \\
&=L^2\pi\hbar^2\frac{ (\eta/\tilde{\eta}_{\ssm HF})^{\tau_{\alpha}/2} + (\eta/\tilde{\eta}_{\ssm HF})^{-\tau_{\alpha}/2} }{2 m^{*}}\sum_{\alpha}n_{\alpha}^2.
\end{align}
Thus,
\begin{align}
\frac{E_{\ssm kin}}{N}=\frac{\pi\hbar^2}{n}\frac{ (\eta/\tilde{\eta}_{\ssm HF})^{\tau_{\alpha}/2} + (\eta/\tilde{\eta}_{\ssm HF})^{-\tau_{\alpha}/2} }{2 m^{*}}\sum_{\alpha}n_{\alpha}^2.
\end{align}

\subsection{Exchange energy}
We adapt the derivation in \cite{giuliani2005quantum} of the exchange energy of an isotropic, two-dimensional electron gas for our case. The exchange energy is given by %Eq. (1.92) QTEL
\begin{align}
E_x=-\frac{1}{2L^2}\sum_{{\bf {q}}\neq 0} v_{\bf {q}} \sum_{{\bf k}\alpha}n_{{\bf k}+{\bf q} \alpha}n_{{\bf k} \alpha},
\end{align}
where $\alpha$ corresponds to the isospin flavour (spin $\sigma_{\alpha}$ and valley $\tau_{\alpha}$).


We proceed by computing the exchange energy.
Replacing the summation by integration (thermodynamic limit, $\sum_{{\bf k}}\to L^2/(2\pi)^2 \int d {\bf k}$) and using
\begin{align}
n_{{\bf k}\alpha}=\Theta((k^{\alpha}_F)^2-\tilde{\eta}_{\ssm HF}^{\tau_{\alpha}/2}k_x^2-\tilde{\eta}_{\ssm HF}^{-\tau_{\alpha}/2}k_y^2),
\end{align}
we obtain
\begin{align}
E_x= -\frac{L^2}{2(2\pi)^{4}}\sum_{\alpha}\int d {\bf q} \int d {\bf k} v_{{\bf q}} &\Theta((k^{\alpha}_F)^2-\tilde{\eta}_{\ssm HF}^{\tau_{\alpha}/2}(k_x+q_x)^2-\tilde{\eta}_{\ssm HF}^{-\tau_{\alpha}/2}(k_y+q_y)^2)    \nonumber\\
&\times\Theta((k^{\alpha}_F)^2-\tilde{\eta}_{\ssm HF}^{\tau_{\alpha}/2}k_x^2-\tilde{\eta}_{\ssm HF}^{-\tau_{\alpha}/2}k_y^2).
\end{align}
We can now make the following change of integration variables:
\begin{align}
\tilde{k}_x=\tilde{\eta}_{\ssm HF}^{\tau_{\alpha}/4}k_x, \ \
\tilde{k}_y=\tilde{\eta}_{\ssm HF}^{-\tau_{\alpha}/4}k_y, \ \ 
\tilde{q}_x=\tilde{\eta}_{\ssm HF}^{\tau_{\alpha}/4}q_x, \ \
\tilde{q}_y=\tilde{\eta}_{\ssm HF}^{-\tau_{\alpha}/4}q_y.
\end{align}
With $d\tilde{k}_x d\tilde{k}_y=dk_x dk_y$ (and likewise for $d {\bf q}$), we arrive at
\begin{align}
E_x= &-\frac{L^2}{2(2\pi)^{4}}\sum_{\alpha}\int d {\bf \tilde{q}} v_{{\bf q}}\int d {\bf \tilde{k}}  \Theta((k^{\alpha}_F)^2-(\tilde{k}_x+\tilde{q}_x)^2-(\tilde{k}_y+\tilde{q}_y)^2)    \nonumber\\
&\times\Theta((k^{\alpha}_F)^2-\tilde{k}_x^2-\tilde{k}_y^2),
\end{align}
where 
\begin{align}
v({\bf q})=v(\tilde{\eta}_{\ssm HF}^{-\tau_{\alpha}/4}\tilde{q}_x,\tilde{\eta}_{\ssm HF}^{\tau_{\alpha}/4}\tilde{q}_y).
\end{align}
We use that \cite{giuliani2005quantum} %(1.93) QTEL, careful sign mistake for 2D
\begin{align}
\frac{1}{N_{\alpha}}\sum_{{\bf k}} n_{{\bf k}+{\bf q} \alpha} n_{{\bf k}\alpha}=1-\frac{2}{\pi}\bigg[\arcsin (\frac{q}{2k^{\alpha}_F})+\frac{q}{2k^{\alpha}_F}\sqrt{1-\big(\frac{q}{2k^{\alpha}_F}\big)^2} \ \bigg]
\end{align}
for $q<2k^{\alpha}_F$ and $0$ otherwise. Here, $q=|{\bf q}|$.

% Figure environment removed

Then, we obtain
\begin{align}
E_x=-\frac{1}{2(2\pi)^{2}}\sum_{\alpha}\int_0^{2k^{\alpha}_F} d \tilde{q} \tilde{q} \int_0^{2\pi} d\theta v_{{\bf q}} N_{\alpha} \bigg(1-\frac{2}{\pi}\bigg[\arcsin (\frac{\tilde{q}}{2k^{\alpha}_F})+\frac{\tilde{q}}{2k^{\alpha}_F}\sqrt{1-\big(\frac{\tilde{q}}{2k^{\alpha}_F}\big)^2}\bigg] \bigg) .
\end{align}
Substituting $z=\tilde{q}/(2k^{\alpha}_F)$ and using that
\begin{align}
v({\bf q}) &=\frac{e^2}{2\epsilon_0\epsilon}\frac{\tanh(d\sqrt{q_x^2+ q_y^2})}{\sqrt{q_x^2+q_y^2}} \\
&=\frac{e^2}{2\epsilon_0\epsilon}\frac{\tanh(d\sqrt{\tilde{\eta}_{\ssm HF}^{-\tau_{\alpha}/2}\tilde{q}_x^2+\tilde{\eta}_{\ssm HF}^{\tau_{\alpha}/2}\tilde{q}_y^2})}{\sqrt{\tilde{\eta}_{\ssm HF}^{-\tau_{\alpha}/2}\tilde{q}_x^2+\tilde{\eta}_{\ssm HF}^{\tau_{\alpha}/2}\tilde{q}_y^2}} \\
&=\frac{e^2}{4 \epsilon_0\epsilon zk_F^{\alpha}}\frac{\tanh(2dzk_F^{\alpha}\sqrt{\tilde{\eta}_{\ssm HF}^{-\tau_{\alpha}/2}\cos(\theta)^2+\tilde{\eta}_{\ssm HF}^{\tau_{\alpha}/2}\sin(\theta)^2})}{\sqrt{\tilde{\eta}_{\ssm HF}^{-\tau_{\alpha}/2}\cos(\theta)^2+\tilde{\eta}_{\ssm HF}^{\tau_{\alpha}/2}\sin(\theta)^2}}.
\end{align}
Inserting into the exchange energy and dividing by the particle number, we obtain
\begin{align}
\frac{E_x}{N} = &-\sum_{\alpha}\frac{e^2 k_F^{\alpha}}{(2\pi)^{2}{2\epsilon_0\epsilon}}\frac{n_{\alpha}}{n}\int_0^{1} d z \int_0^{2\pi} d\theta  \frac{\tanh(2dzk_F^{\alpha}\sqrt{\tilde{\eta}_{\ssm HF}^{-\tau_{\alpha}/2}\cos(\theta)^2+\tilde{\eta}_{\ssm HF}^{\tau_{\alpha}/2}\sin(\theta)^2})}{\sqrt{\tilde{\eta}_{\ssm HF}^{-\tau_{\alpha}/2}\cos(\theta)^2+\tilde{\eta}_{\ssm HF}^{\tau_{\alpha}/2}\sin(\theta)^2}}  \\
 &\times \bigg(1-\frac{2}{\pi}\bigg[\arcsin (z)+z\sqrt{1-\big(z\big)^2}\bigg]\bigg) .
\end{align}
The above integral can be easily computed numerically for a given isospin polarization $(n_1, n_2, n_3, n_4)$. 




\subsection{Hartree-Fock: Results}
We minimize $E/N=E_{\ssm kin}/N+E_x/N$ for the optimal effective anisotropy $\tilde{\eta}_{\ssm HF}$, for different isospin polarizations as a function of density $n$.


% Figure environment removed

Figure~\ref{fig:HF} shows the Hartree-Fock energies as a function of $r_s$ for states of different isospin polarization. We observe several attributes of the phase diagram that conflict with the experimental observations on AlAs: First, the valley-polarized, spin-unpolarized state (VP) and the spin-polarized, valley-unpolarized state (SP) are degenerate and thus mean-field cannot explain the experimental results on AlAs. 
The accidental degeneracy between VP and SP is a result of the form of the exchange energy: Only intra-valley terms contribute.
Second, the polarization transitions occur at much higher densities than observed in the experiment (VP in experiment at $r_s \approx 20$). Third, a phase with a partially polarized ground state with $3$ filled Fermi pockets exists within Hartree-Fock, which is also not observed in the experiment. The phase diagram as a function of anisotropy is given in Figure~\ref{fig:HFeta}. Here, we observe that the phase boundaries are rather insensitive to a change in anisotropy within Hartree-Fock, as opposed to VMC.

Above, we considered states with fixed isospin polarization. A  complete analysis that accounts for effects beyond polarized phases such as inter-valley coherence, requires including the possibility of isospin mixing. We will leave this analysis for future studies.


\bibliographystyle{unsrt}
\bibliography{papers.bib}

%TC:endignore


\end{document}


\end{document}

