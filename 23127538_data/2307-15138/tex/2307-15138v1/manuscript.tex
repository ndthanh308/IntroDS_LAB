% $Id: template.tex 11 2007-04-03 22:25:53Z jpeltier $

\documentclass{vgtc}                          % final (conference style)
%\documentclass[review]{vgtc}                 % review
%\documentclass[widereview]{vgtc}             % wide-spaced review
%\documentclass[preprint]{vgtc}               % preprint
%\documentclass[electronic]{vgtc}             % electronic version

%% Uncomment one of the lines above depending on where your paper is
%% in the conference process. ``review'' and ``widereview'' are for review
%% submission, ``preprint'' is for pre-publication, and the final version
%% doesn't use a specific qualifier. Further, ``electronic'' includes
%% hyperreferences for more convenient online viewing.

%% Please use one of the ``review'' options in combination with the
%% assigned online id (see below) ONLY if your paper uses a double blind
%% review process. Some conferences, like IEEE Vis and InfoVis, have NOT
%% in the past.

%% Figures should be in CMYK or Grey scale format, otherwise, colour 
%% shifting may occur during the printing process.

%% These few lines make a distinction between latex and pdflatex calls and they
%% bring in essential packages for graphics and font handling.
%% Note that due to the \DeclareGraphicsExtensions{} call it is no longer necessary
%% to provide the the path and extension of a graphics file:
%% % Figure removed is completely sufficient.
%%
\ifpdf%                                % if we use pdflatex
  \pdfoutput=1\relax                   % create PDFs from pdfLaTeX
  \pdfcompresslevel=9                  % PDF Compression
  \pdfoptionpdfminorversion=7          % create PDF 1.7
  \ExecuteOptions{pdftex}
  \usepackage{graphicx}                % allow us to embed graphics files
  \DeclareGraphicsExtensions{.pdf,.png,.jpg,.jpeg} % for pdflatex we expect .pdf, .png, or .jpg files
\else%                                 % else we use pure latex
  \ExecuteOptions{dvips}
  \usepackage{graphicx}                % allow us to embed graphics files
  \DeclareGraphicsExtensions{.eps}     % for pure latex we expect eps files
\fi%

%% it is recomended to use ``\autoref{sec:bla}'' instead of ``Fig.~\ref{sec:bla}''
\graphicspath{{figures/}{pictures/}{images/}{./}} % where to search for the images

\newcommand{\hs}[1]{\textcolor{cyan}{#1}}
\newcommand{\ak}[1]{\textcolor{black}{#1}}
% \newcommand{\hs}[1]{\textcolor{orange}{#1}}
\newcommand{\lh}[1]{\textcolor{blue}{#1}}



\usepackage{microtype}                 % use micro-typography (slightly more compact, better to read)
\PassOptionsToPackage{warn}{textcomp}  % to address font issues with \textrightarrow
\usepackage{textcomp}                  % use better special symbols
\usepackage{mathptmx}                  % use matching math fonts
\usepackage{times}                     % we use Times as the main font
\renewcommand*\ttdefault{txtt}         % a nicer typewriter font
\usepackage{cite}                      % needed to automatically sort the references
\usepackage{tabu}                      % only used for the table example
\usepackage{booktabs}                  % only used for the table example
\usepackage{wrapfig}
\usepackage{tabularx}
\usepackage{enumitem}
\usepackage{subfigure}
\usepackage{xspace}
\newcommand{\etal}{\emph{et al.}\@\xspace}
\newcommand{\ie}{\emph{i.e.}\xspace}
\newcommand{\eg}{\emph{e.g.}\xspace}
\newcommand{\etc}{\emph{etcetera}\@\xspace}
\newcommand{\etals}{\mbox{\emph{et~al.}'s }}
\usepackage{enumitem}

\setlist[enumerate]{itemsep=0.5ex,parsep=0pt}
%% We encourage the use of mathptmx for consistent usage of times font
%% throughout the proceedings. However, if you encounter conflicts
%% with other math-related packages, you may want to disable it.


%% If you are submitting a paper to a conference for review with a double
%% blind reviewing process, please replace the value ``0'' below with your
%% OnlineID. Otherwise, you may safely leave it at ``0''.
\onlineid{1140}

%% declare the category of your paper, only shown in review mode
\vgtccategory{Research}

%% allow for this line if you want the electronic option to work properly
\vgtcinsertpkg

%% In preprint mode you may define your own headline. If not, the default IEEE copyright message will appear in preprint mode.
%\preprinttext{To appear in an IEEE VGTC sponsored conference.}

%% This adds a link to the version of the paper on IEEEXplore
%% Uncomment this line when you produce a preprint version of the article 
%% after the article receives a DOI for the paper from IEEE
%\ieeedoi{xx.xxxx/TVCG.201x.xxxxxxx}


%% Paper title.

\title{Taken By Surprise? Evaluating how \ak{Bayesian} Surprise \& Suppression Influences Peoples’ Takeaways in Map Visualizations}

%% This is how authors are specified in the conference style

%% Author and Affiliation (single author).
%%\author{Roy G. Biv\thanks{e-mail: roy.g.biv@aol.com}}
%%\affiliation{\scriptsize Allied Widgets Research}

%% Author and Affiliation (multiple authors with single affiliations).
% \author{Roy G. Biv\thanks{e-mail: roy.g.biv@aol.com} %
% \and Ed Grimley\thanks{e-mail:ed.grimley@aol.com} %
% \and Martha Stewart\thanks{e-mail:martha.stewart@marthastewart.com}}
% \affiliation{\scriptsize Martha Stewart Enterprises \\ Microsoft Research}

%% Author and Affiliation (multiple authors with single affiliation)
\author{Akim Ndlovu\thanks{e-mail: andlovu@wpi.edu} %
\and Hilson Shrestha\thanks{e-mail: hshrestha@wpi.edu} %
\and Lane T. Harrison\thanks{e-mail: ltharrison@wpi.edu}}
\affiliation{\scriptsize \ak{Worcester Polytechnic Institute}}

%% A teaser figure can be included as follows
\teaser{
  \centering
  % Figure removed
 
  \ak{\caption{How do Bayesian surprise metrics and suppression encodings influence peoples' takeaways in map visualizations? We conduct two experiments with Covid-19 and Poverty datasets, randomly assigning 300 participants to three map conditions. We collect data across three map takeaway tasks \textit{T1\textsubscript{Best}, T1\textsubscript{Worst}: Identify} and \textit{T2: Explore}. To mitigate biases for particular dataset contexts discovered in pilot studies (\eg vaccine skepticism) we reframe both datasets as a sales and marketing task. Metrics include participants' exploration metadata (quantitative) and takeaway comments (qualitative).}}
  \label{fig:teaser}
}

%% Abstract section.
\abstract{
Choropleth maps have been studied and extended in many ways to counteract the many biases that can occur when using them. 
Two recent techniques, Surprise metrics and Value Suppressing Uncertainty Palettes (VSUPs), offer promising solutions but have yet to be tested empirically with users of visualizations. 
In this paper, we explore how well people can make use of these techniques in map exploration tasks. 
We report a crowdsourced experiment where $n=300$ participants are assigned to one of Choropleth, Surprise (only), and VSUP conditions (depicting rates and Surprise in a suppressed palette). 
Results show clear differences in map analysis outcomes, \eg with Surprise maps leading people to significantly higher areas of population, or VSUPs performing similar or better than Choropleths for rate selection. 
Qualitative analysis suggests that many participants may only consider a subset of the metrics presented to them during exploration and decision-making. 
We discuss how these results generally support the use of Surprise and VSUP techniques in practice, and opportunities for further technique development.
\ak{The material for the study (data, study results and code) is publicly available on \url{https://osf.io/exb95/}}.
} % end of abstract


%% ACM Computing Classification System (CCS). 
%% See <http://www.acm.org/about/class> for details.
%% We recommend the 2012 system <http://www.acm.org/about/class/class/2012>
%% For the 2012 system use the ``\CCScatTwelve'' which command takes four arguments.
%% The 1998 system <http://www.acm.org/about/class/class/2012> is still possible
%% For the 1998 system use the ``\CCScat'' which command takes four arguments.
%% In both cases the last two arguments (1998) or last three (2012) can be empty.

\CCScatlist{
  \CCScatTwelve{Data Visualization}{Choropleth Maps}{Baye\-si\-an Surprise}{}
}

%\CCScatlist{
  %\CCScat{H.5.2}{User Interfaces}{User Interfaces}{Graphical user interfaces (GUI)}{};
  %\CCScat{H.5.m}{Information Interfaces and Presentation}{Miscellaneous}{}{}
%}

%% Copyright space is enabled by default as required by guidelines.
%% It is disabled by the 'review' option or via the following command:
% \nocopyrightspace

%%%%%%%%%%%%%%%%%%%%%%%%%%%%%%%%%%%%%%%%%%%%%%%%%%%%%%%%%%%%%%%%
%%%%%%%%%%%%%%%%%%%%%% START OF THE PAPER %%%%%%%%%%%%%%%%%%%%%%
%%%%%%%%%%%%%%%%%%%%%%%%%%%%%%%%%%%%%%%%%%%%%%%%%%%%%%%%%%%%%%%%%

\begin{document}

%% The ``\maketitle'' command must be the first command after the
%% ``\begin{document}'' command. It prepares and prints the title block.

%% the only exception to this rule is the \firstsection command
\firstsection{Introduction}
\maketitle
The vast amount of data gathered during Covid-19 pandemic has created a need to visualize and accurately communicate trends in vaccinations, deaths and infections \cite{griffin2020trustworthy,juergens2020trustworthy}. As a result, Choropleth maps have been widely used for visualizing trends in geospatial data such as high or low performing regions and regions that show a high degree of correlation or disparity \cite{munn2020developing, mooney2020mapping}.
However, research has shown that when visualizing data that closely resembles a population distribution, Choropleth maps are prone to biases. Such instances occur  when visualizing percentage rates, where counties or regions with low population and high variance may be shaded using darker colors, which may be misleading for map readers \cite{correll2016surprise}.

A number of approaches have been proposed to counteract bias in Choropleth maps, resulting in a modified or supplemented dataset~\cite{correll2016surprise}. These include normalization, Bayesian \ak{surprise}, Spatial smoothing and Geographical weighted regression \cite{brunsdon1998geographically, genebes2018spatial,correll2016surprise}. Prior studies by MacEachren \cite{maceachren1992visualizing} have interrogated the impact of using different metrics to offset biases in map visualizations. \ak{For example, Correll and Heer use Bayesian surprise \cite{correll2016surprise} to depict a metric that measures belief about the observed data, either instead of or alongside the actual data items themselves. Studies suggest that the visualization of uncertainty requires people to understand the metrics to effectively use them \cite{hullman2019survey}, implying a need to investigate how people interpret metrics such as Bayesian surprise in map reading contexts.}

As a result, researchers have examined some model-driven mapping techniques by designing information retrieval, comparison, ranking and aggregation tasks, in order to understand their impact on pattern recognition and decision making \cite{maceachren1998visualizing, blenkinsop2000evaluating, boukhelifa2012evaluating, alberti2018web}. Other approaches for evaluating such maps include using empirically derived frameworks similar to the one proposed by Roth~\cite{roth2013empirically}. 
Although widely applicable to map evaluation studies, such frameworks may need to be extended when uncertainty is added as a consideration \cite{hullman2019survey}.

Two recently developed techniques provide a promising baseline for investigating map debiasing techniques in user studies.
Correll and Heer propose the use of ``Surprise'', a Bayesian weighting technique that offsets biases in map visualizations \cite{correll2016surprise}. 
Surprise up-weights or down-weights data points that deviate from expected values, by calculating an updated belief about the data based on prior knowledge. 
Correll \etal \cite{correll2018value} also introduce the Value-Suppressing Uncertainty Palette, a map coloring technique and legend technique which can visualize both uncertainty measures (such as Surprise) and rates in a single map.

\ak{In this paper, we examine the impact of two recently developed visualization techniques, on how people explore and generate takeaways in map reading contexts. 
We report a crowdsourced study with $n=300$ participants}, where we ask participants to perfom map analysis tasks with one of three visualization conditions: Choropleth, Surprise, or Value Suppressing Uncertainty Palettes (VSUPs). 
We describe some of the technical challenges and resulting adaptations in taking prior map tasks and task taxonomies to study techniques which emphasize different metrics (Surprise, rates, or both).
In particular, we leverage previous research by Roth \cite{roth2013empirically} and Besan\c{c}on et~al. \cite{besanccon2020evaluation} to design a universal task for all map conditions (see \autoref{tab:taskList}). 
Results show clear differences in map analysis outcomes (\autoref{fig:quantative-analysis}),
while qualitative analysis suggests that participants in some cases only consider a subset of the metrics available to them. 
We discuss how these results tentatively support the use of Surprise and VSUP techniques for broader visualization viewing populations, while also highlighting challenges that might be addressed through future design and technique development.

\section{METHODOLOGY}
We designed three interactive stimuli (Choropleth, Surprise, and VSUP maps) using Covid-19 Vaccination and Poverty datasets. 
We conducted two experiments on the online crowdsourcing platform Prolific, where we collected data from $n = 300$ participants. 
\ak{Pilot studies using a vaccine dataset revealed skewed results with strong political bias (\S{} \ref{pilot_study})}. 
We therefore design a scenario that ``masks'' the underlying dataset as being about sales rates, using tasks adapted from Roth \cite{roth2013empirically} and Besan\c{c}on \cite{besanccon2020evaluation}. 
\ak{To test for the possible impact of data characteristics, we repeat the experiment across two datasets measuring different geospatial phenomena, Covid-19 vaccination rates and poverty statistics}.

\subsection{Stimuli Design}
Our design goal was to minimize notable differences between the stimuli to avoid map interpretation bias, while maximizing on techniques that improve the accessibility of information \cite{latif2021deeper, robinson2017geospatial, andrienko1999interactive}. 
For ecological validity, our map design and color schemes were influenced by The New York Times (NYT) Covid-19 vaccination map \cite{Aisch2015-pl}, and designed to be consistent as possible between all three maps \ak{(see supplemental material for additional stimuli design considerations).}

\subsection{Experiment Datasets}
We adapted publicly available county level datasets of Covid-19 vaccinations \cite{cdc} and Poverty rates \cite{openintro} of the US. 
Prior to conducting the study, we replicated a Surprise map of per-capita unemployment rates from Correll and Heer \cite{correll2016surprise}, that uses a model of the deMoivre's funnel to determine deviations from the average per-capita rate.
This method calculates the test statistic ($Z_{s}$) from event rates. 
Bayesian methods are then used to find the likelihood of points being $Z_{s}$ distant from the center of the funnel: 

\begin{equation}
{\large P(s|deMoivre) = 1 - (2 \cdot \int_{0}^{|Z_{s}|} \phi(x) dx)}
\end{equation}
where deMoivre represents the model and {$s \in D$} (Dataset).
After replicating the Surprise map of per-capita unemployment rate, we apply the same process to our datasets of interest \cite{cdc, openintro}.

% Figure environment removed

\subsection{Pilot Study}
\label{pilot_study}
To refine the user experience for the study, we conducted a pilot study with $n=30$ participants. 
We designed our stimuli using a Covid-19 dataset \cite{cdc} and randomly assigned $n = 10$ participants to each condition. 
Our initial analysis of the qualitative feedback reflected a high degree of participants' personal beliefs and political affiliation. Here is one example: 

\textbf{Response:} \textit{``I think it's going okay. In the beginning everyone was reluctant since it's so new, there's hardly and research. [...] It seems like there'll be a lot more people vaccinated by the end of the year"}.

Given these results, we rephrased our tasks to a product sales and marketing decision-making problem using both the Covid-19 and Poverty datasets (see \autoref{tab:taskList}). 
\ak{We developed two task categories across the metrics and conditions to be considered by summarizing map analyses objectives and tasks used in prior studies from Roth \cite{roth2013empirically} and Besan\c{c}on \cite{besanccon2020evaluation}}.
We also developed additional ``scrollytelling'' trainings for all conditions to help reduce sources of noise (\eg participant misunderstandings) during the full experiment.

\subsection{Task and Procedure}
We used a between subjects design across two experiments (Covid-19 and Poverty). 
We designed 3 stimuli (conditions) and 3 tasks $T1_{Best}$, $T1_{Worst}$ and $T2_{Explore}$ (see \autoref{fig:teaser}$1$). 
We randomly assigned $25$ participants to each condition. 
The total number of participants for the study was therefore, $2$ experiments $\times 3$ conditions (Choropleth, VSUP and Surprise stimuli) $\times 2$ tasks ($T1_{Best}$, $T1_{Worst}$) $\times 25$ participants $= 300$. 
\ak{Of our participants, 160 identified as female, 136 identified as male, and 4 participants chose not to disclose their gender. 
Participants' age ranged from 18 to 76 with an average of 35.}
The study was IRB-reviewed and we required a consent form before participation.
Participants were not constrained to a completion time, however, we estimated an average completion time of 7 minutes, used to calculate a payment of \$1.40 to exceed US Minimum Wage. 
We collect metadata on counties of interest for each participant (\eg population), as well as feedback regarding their perception of the study.

\begin{table}
    \caption{List of experiment tasks \textit{T1: Identify} and \textit{T2: Explore}}
    \label{tab:taskList}
    \begin{tabularx}{\linewidth}{>{\hsize=0.1\hsize}X>{\hsize=0.20\hsize}X>{\hsize=0.70\hsize}X}
        \hline
        & Objective & Task Narration \\
        \hline
        T1\textsubscript{Best} & Identify \newline and Rank & Select five (5) of the best performing counties, where you would send a team to learn about local sales strategies. \\
        \hline
        T1\textsubscript{Worst} & Identify \newline and Rank & Select five (5) of the worst performing counties, where you would send a team to learn about local sales strategies. \\
        \hline
        T2 & Compare \& \newline Delineate \newline (Explore) & Explore the map, then write a short narrative on where you would focus your marketing efforts to increase sales of the product. \\
        \hline
    \end{tabularx} 
\end{table}


\section{Results}

We used a Kruskal-Wallis test to detect overall effects in data across the three different mapping techniques (see \autoref{fig:quantative-analysis}). 
For post hoc tests, we use Dunn's test with Bonferroni correction. 
We also compute and report $95\%$ confidence intervals using bootstrapping.
For geospatial analysis, we create a point map of participant county selections across the three conditions (see \autoref{fig:pselections}).

\subsection{Identify Tasks \texorpdfstring{T1\textsubscript{Best} and T1\textsubscript{Worst}}{T1Best and T1Worst}}

\ak{In both experiments (Covid-19 and Poverty datasets), participants' performance differed across the tested visualization conditions in terms of rate, population, and Surprise metrics of the selected counties (see \autoref{fig:quantative-analysis} and supplemental material for full results)}.

\textbf{Rate:}
\ak{We find overall differences between the map conditions for vaccine best $KW=12.75$ $p=0.0017$ $H=0.02749$ (see \autoref{fig:quantative-analysis}A) and the vaccine worst tasks $KW=122.4$ $p=2.688e-27$ $H=0.3479$.
Post-hoc comparisons suggest that the VSUP performs best in the vaccine best task, and the Choropleth map performs best in the vaccine worst task. 
In the latter case, VSUPs appear to balance the differences between the Surprise and Choropleth maps, making them a potentially good choice overall.
However, we note that all maps performed similarly when selecting the worst performing counties in the poverty dataset \autoref{fig:quantative-analysis}D).
While a formal method for investigating dataset distributions would be needed, it appears that the poverty rates in the dataset itself are negatively skewed, which may be a reason for the observed similar performance, implying a need for more distribution-sensitive techniques in future work.}

\textbf{Population:}
In terms of selected counties, the Surprise maps tended to lead participants towards counties of higher population (see \autoref{fig:quantative-analysis}B and \autoref{fig:quantative-analysis}E).
In particular, we find in the vaccine best task an overall effect $KW=16.52$ $p=0.00025$ $H=0.037$, and for in the poverty best task $KW=11.79$ $p=0.00275$ $H=0.026$.
However, we note that these effects tend to place Surprise maps above Choropleth maps, but not above VSUPs, which appear to balance the effects of the other two.
Similar effects and trends are found in the \ak{vaccine worst $KW=107.4$ $p=4.695e-24$  $H=0.3047$, and in the poverty worst $KW=16.3$ $p=0.0002798$ $H=0.03861$ tasks}.

\textbf{Surprise:} 
Results suggest that VSUPs and Surprise maps led participants to select counties with high surprise values for the vaccine best and low surprise values for the vaccine worst task as shown in \autoref{fig:quantative-analysis}C with $KW=10.94$ $p=0.004$ $H=0.02286$ and \autoref{fig:quantative-analysis}F with $KW=37.67$ $p=6.615e-09$ $H=0.1031$. 
However, in the poverty best and worst tasks, while overall effects were observed $KW=14.23$ $p=0.0008$ $H=-0.033$ and $KW=51 $ $p=8.01e-12$ $H=0.1319$, these generally indicate VSUPs outperforming Surprise and Choropleth maps. 
These differences again may be partially due to the skewed nature of the poverty rates compared to the vaccine rates.

\subsection{Explore Task T2}
% Figure environment removed

We considered participants' feedback based on relevance, similarity and identified keywords such as population, color, surprisingly high and surprisingly low. 
In the Discussion we expand on our takeaways from participant responses which suggest that: 
\begin{enumerate}
  \item Participants only consider a subset of the metrics presented (\S{} \ref{subsubdescription_takeaway}).
  \item \ak{Visual encodings (\eg color) can impact how people interpret surprise} (\S{} \ref{subsubcolor_surprise}).
  \item County size can skew peoples takeaways (\S{} \ref{subsubsequential_uncertainty}).
\end{enumerate}

\section{DISCUSSION}
\subsection{Spatial Analysis of Identify Task \texorpdfstring{T1\textsubscript{Best} and T1\textsubscript{Worst}}{T1Best and T1Worst}}

Results from participant ranking selections are aggregated by county and shown in \autoref{fig:pselections}. 
We infer the following takeaways from participants' interactions and selections on the maps.

\subsubsection{Visual Saliency of High/Low performing counties} 
County selection on the Choropleth map show a high level of dissimilarity compared to the Surprise and VSUP maps. 
We attribute this to the narrowed visual search space when visualizing Surprise compared to event-rates on a Choropleth map. 
This is supported by a further analysis of the Surprise and VSUP maps, where we see tighter clusters and consensus of participants ranking selections. 
Results also show a higher degree of ranking consensus by participants' on the VSUP map compared to the Surprise map. 
This may be due to the fact that VSUPs further suppress highly uncertain values by combining color cells in a palette using a tree structure \cite{kay2019much}. 
To assess whether participants considered population in their decision-making process, we conducted further analysis of the population of counties they selected. 
Population focused results suggest that participants consider highly populated areas when making task based decisions using both Surprise and VSUP maps. 
Such patterns are also validated by the qualitative analysis (T2).

\subsection{Explore Task T2}
We summarize feedback from an open-ended response task, where we ask participants to explore the map and give insights on where they would focus efforts to ``increase sales of the product''.

\subsubsection{Participants only consider a subset of the metrics presented on the maps.}
\label{subsubdescription_takeaway}
Participant comments suggest that some appeared to have difficulty in making use of all the available metrics (Surprise, Rate and Population), instead they used only 1 or 2 of the available metrics to select high or low performing counties. 
These findings are supported by summarizing participants' feedback on the strategies they used in selecting counties on the maps and contribute to insights on the challenges associated with comprehending Surprise without the consideration of other metrics (Population and Rate), for example:  

\textbf{Response: }\textit{``I looked for areas with high Surprise metrics (or low) and considered that most areas could be converted because of their proximity to areas with good sales"}

However, other participants effectively used interactions to explore smaller counties by hovering over the legend and counties. 
This allowed them to carry out more complex queries on the maps, suggesting that they could gain more insights by carrying out other tasks, for example:

\textbf{Response: }\textit{``[...] I'd hover over these areas to understand the surprise metric relative to the sales success rate and population. Being able to compare the data helped me to understand what the surprise metric meant, and then helped me develop a hypothesis on why these are high success/high surprise areas."}


\subsubsection{Color influences how people interpret surprise}
\label{subsubcolor_surprise}
We observed the influence of color in how participants interpret either event-rates or surprise. 
These findings suggest that some participants consider dark green and dark brown as high or low sales rate counties respectively \cite{schiewe2019empirical}. 
For example: 

\textbf{Response:}\textit{``Darker green colors show 
positive and more response to the marketing and the darker pink color is the opposite [...]"}

While interpretation is true for standard Choropleth maps, it is not  necessarily true for Surprise and VSUP maps, which depict more complex metrics. 
This may indicate a need for additional training methods or investigation of visual cues that help people associate depicted colors with metrics rather than rates alone.

\subsubsection{Size influences how people interpret of uncertainty}
\label{subsubsequential_uncertainty}
Similar to findings by Schiewe \cite{schiewe2019empirical}, both qualitative and point pattern analysis suggest that some participants neglect smaller counties and are drawn to larger counties or states on the maps. 
However, our findings also suggest that VSUPs and Surprise maps suppress low-population counties with high rates (\eg unsurprising), which may help alleviate one aspect of this bias.
How to ensure small yet high population counties also receive sufficient attention remains a challenge for maps geared towards the general public.

\section{LIMITATIONS and FUTURE WORK}
Our analysis of the Surprise, Rate and Population metrics shows clear differences between the mapping techniques used in this study (Choropleth, Surprise and VSUP maps). 
However, we hypothesize that the use of datasets with different distributions (normal and left skewed) may impact findings of our study. 
Future experiments may investigate directly the impact of skews in rates (\eg through simulation) on participant exploration and takeaways.
Furthermore, metrics and interaction techniques that build on existing work like Surprise and VSUPs may further enrich map analysis for the public.

Another limitation is noise in the experiment. 
While the sales scenario worked well overall by allowing us to ask the same task across Choropleth, Surprise, and VSUP
% the three
maps, one key issue arose in the ``worst'' tasks. We observed outlier participants across all conditions who, when prompted to select the worst performing counties, instead selected the best performing counties. 
This may be a bias with the framing of sales, which could be addressed by experimenting with other scenarios or by additional design or feedback mechanisms. 

Future work should consider collecting prior probability distributions from participants \cite{Aisch2015-pl, kim2021bayesian}. 
Experimenting with other representational techniques such as map pairs could also assist in improving the accessibility of highly technical thematic maps for the general public. 
VSUPs use a heuristic approach to suppress values at high level of uncertainty, therefore, future research could also consider the use of decision based models as suggested in work by Kay \cite{kay2019much} and Yang \etal \cite{yangsubjective}.  

\section{CONCLUSION}
Despite the pervasive use of choropleth map visualizations, especially when communicating critical data to the public (\eg vaccine trends or election results), they suffer from well-documented biases and limitations.
In this study, we design an experiment to test two recently proposed techniques, Surprise maps and VSUPs, in a crowdsourced setting similar to how participants might encounter such maps online.
Results generally indicated that Surprise maps and VSUPs do indeed offset some of the issues of traditional Choropleth maps. 
However, close inspection also reveals opportunities for addressing confusion and misconceptions of these new techniques.
Going forward, designers may benefit from knowing that Choropleths perform similarly to these new techniques (\ie reducing the risk of harm), while results that indicate  these new techniques can lead people to more surprising or populous counties may give designers the confidence to experiment with new and innovative ways of communicating with the general public.

\newpage
%\bibliographystyle{abbrv}
\bibliographystyle{abbrv-doi}
%\bibliographystyle{abbrv-doi-narrow}
%\bibliographystyle{abbrv-doi-hyperref}
%\bibliographystyle{abbrv-doi-hyperref-narrow}

\bibliography{manuscript}
\end{document}
