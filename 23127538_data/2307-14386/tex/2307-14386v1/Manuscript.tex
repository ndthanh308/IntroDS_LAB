% ****** Start of file aipsamp.tex ******
%
%   This file is part of the AIP files in the AIP distribution for REVTeX 4.
%   Version 4.1 of REVTeX, October 2009
%
%   Copyright (c) 2009 American Institute of Physics.
%
%   See the AIP README file for restrictions and more information.
%
% TeX'ing this file requires that you have AMS-LaTeX 2.0 installed
% as well as the rest of the prerequisites for REVTeX 4.1
% 
% It also requires running BibTeX. The commands are as follows:
%
%  1)  latex  aipsamp
%  2)  bibtex aipsamp
%  3)  latex  aipsamp
%  4)  latex  aipsamp
%
% Use this file as a source of example code for your aip document.
% Use the file aiptemplate.tex as a template for your document.
\documentclass[%
 aip,
% jmp,
% bmf,  
% sd,
% rsi,
 amsmath,amssymb,
%preprint,%
 reprint,%
%author-year,%
%author-numerical,%
% Conference Proceedings
]{revtex4-2}

\usepackage{graphicx}% Include figure files
\usepackage{dcolumn}% Align table columns on decimal point
\usepackage{bm}% bold math
%\usepackage[mathlines]{lineno}% Enable numbering of text and display math
%\linenumbers\relax % Commence numbering lines

\usepackage[utf8]{inputenc}
\usepackage[T1]{fontenc}
\usepackage{mathptmx}
\usepackage{etoolbox}


%% Apr 2021: AIP requests that the corresponding 
%% email to be moved after the affiliations
\makeatletter
\def\@email#1#2{%
 \endgroup
 \patchcmd{\titleblock@produce}
  {\frontmatter@RRAPformat}
  {\frontmatter@RRAPformat{\produce@RRAP{*#1\href{mailto:#2}{#2}}}\frontmatter@RRAPformat}
  {}{}
}%

\usepackage{lipsum}
\usepackage{color}
\usepackage{amsmath}
\usepackage{color}
\usepackage{siunitx}
\usepackage{float}
\usepackage{soul}
%\usepackage[demo]{graphix}
%\usepackage{caption}
%\usepackage{subcaption}
\usepackage{pifont,wasysym}
\usepackage{color,amsmath}
\usepackage{mathrsfs} 
\usepackage{amsmath}

\newcommand{\old}[1]{\textcolor{red}{{#1}}}
\newcommand{\todo}[1]{\textcolor{blue}{{#1}}}
\newcommand{\todored}[1]{\textcolor{red}{{#1}}}
\newcommand{\todomagenta}[1]{\textcolor{magenta}{{#1}}}

% Commands & definitions

\def\Gammaqq{\ensuremath{\Gamma_{q\bar{q}}}\xspace}
\def\Gammahad{\ensuremath{\Gamma_{hadrons}}\xspace}

\def\Kgamma{\ensuremath{E_{\gamma}}\xspace}
\def\Kcut{\ensuremath{E_{\gamma}^{cut}}\xspace}
\def\acolcut{\ensuremath{\sin{(\Psi_{acol})}^{cut}}\xspace}

\def\Kreco{\ensuremath{K_{reco}}\xspace}
\def\kaonness{\ensuremath{\Delta_{\dEdx-K}}\xspace}
\def\pionness{\ensuremath{\Delta_{\dEdx-\pi}}\xspace}
\def\protonness{\ensuremath{\Delta_{\dEdx-p}}\xspace}
\def\epsilonhad{\ensuremath{\epsilon_{had}}\xspace}
\def\epsilonb{\ensuremath{\epsilon_{b}}\xspace}
\def\epsilonc{\ensuremath{\epsilon_{c}}\xspace}
\def\epsilonuds{\ensuremath{\epsilon_{uds}}\xspace}
\def\epsilonb2{\ensuremath{\epsilon^{2}_{b}}\xspace}
\def\epsilonc2{\ensuremath{\epsilon^{2}_{c}}\xspace}
\def\epsilonuds2{\ensuremath{\epsilon^{2}_{uds}}\xspace}

\def\costheta{\ensuremath{\cos \theta}\xspace}
\def\costhetab{\ensuremath{\cos \theta_{b}}\xspace}
\def\costhetaq{\ensuremath{\cos \theta_{q}}\xspace}
\def\sinthetaq{\ensuremath{\sin \theta_{q}}\xspace}
\def\costhetasq{\ensuremath{\cos^2 \theta_{b}}\xspace}
\def\sinthetasq{\ensuremath{\sin^2 \theta_{b}}\xspace}
\def\costhetaj{\ensuremath{\cos \theta_{j}}\xspace}
\def\costhetac{\ensuremath{\cos \theta_{c}}\xspace}

\def\fb{fb\ensuremath{^{-1}}\xspace}
\def\mum{\textmu\ensuremath{m}\xspace}
%\def\eL{\ensuremath{e_L^{\mbox{\scriptsize -}}}\xspace}
%\def\eR{\ensuremath{e_R^{\mbox{\scriptsize -}}}\xspace}
%\def\pL{\ensuremath{e_L^{\mbox{\scriptsize +}}}\xspace}
%\def\pR{\ensuremath{e_R^{\mbox{\scriptsize +}}}\xspace}

\def\eL{\ensuremath{e_L^{-}}\xspace}
\def\eR{\ensuremath{e_R^{-}}\xspace}
\def\pL{\ensuremath{e_L^{+}}\xspace}
\def\pR{\ensuremath{e_R^{+}}\xspace}

\def\b{\ensuremath{b}\xspace}
\def\bbar{\ensuremath{\overline{b}}\xspace}
\def\bbbar{\ensuremath{b}\ensuremath{\overline{b}}\xspace}
\def\qqbar{\ensuremath{q}\ensuremath{\overline{q}}\xspace}
\def\ccbar{\ensuremath{c}\ensuremath{\overline{c}}\xspace}
\def\ttbar{\ensuremath{t}\ensuremath{\overline{t}}\xspace}
\def\eebbbar{\ensuremath{e^{-}e^{+}\rightarrow b\bar{b}}\xspace}%\ensuremath{e^{\mbox{\scriptsize +}}}\ensuremath{\rightarrow}\ensuremath{b}\ensuremath{\overline{b}}\xspace}
\def\eeccbar{\ensuremath{e^{-}e^{+}\rightarrow c\bar{c}}\xspace}
\def\eeqqbar{\ensuremath{e^{-}e^{+}\rightarrow q\bar{q}}\xspace}
\def\ee{\ensuremath{e^{-}e^{+}}\xspace}
\def\eeZ{\ensuremath{e^{-}e^{+}\rightarrow Z}\xspace}
\def\eeZqqbar{\ensuremath{e^{-}e^{+}\rightarrow Z\rightarrow q\bar{q}}\xspace}
\def\eeZgammaqqbar{\ensuremath{e^{-}e^{+}\rightarrow Z \gamma \rightarrow q\bar{q} \gamma}\xspace}

\def\eebb{\ensuremath{e^{-}e^{+}\rightarrow b\bar{b}}\xspace}%\ensuremath{e^{\mbox{\scriptsize +}}}\ensuremath{\rightarrow}\ensuremath{b}\ensuremath{\overline{b}}\xspace}
\def\eecc{\ensuremath{e^{-}e^{+}\rightarrow c\bar{c}}\xspace}
\def\eeqq{\ensuremath{e^{-}e^{+}\rightarrow q\bar{q}}\xspace}
\def\eeZqq{\ensuremath{e^{-}e^{+}\rightarrow Z\rightarrow q\bar{q}}\xspace}
\def\eeZgammaqq{\ensuremath{e^{-}e^{+}\rightarrow Z \gamma \rightarrow q\bar{q} \gamma}\xspace}

%\def\eecc{\ensuremath{e^{\mbox{\scriptsize -}}}\ensuremath{e^{\mbox{\scriptsize +}}}\ensuremath{\rightarrow}\ensuremath{c}\ensuremath{\overline{c}}\xspace}
%\def\eeqq{\ensuremath{e^{\mbox{\scriptsize -}}}\ensuremath{e^{\mbox{\scriptsize +}}}\ensuremath{\rightarrow}\ensuremath{q}\ensuremath{\overline{q}}\xspace}
%\def\ee{\ensuremath{e^{\mbox{\scriptsize -}}}\ensuremath{e^{\mbox{\scriptsize +}}}\xspace}
%\def\eeZ{\ensuremath{e^{\mbox{\scriptsize -}}}\ensuremath{e^{\mbox{\scriptsize +}}}\ensuremath{\rightarrow}\ensuremath{Z}\xspace}
%\def\eeZqq{\ensuremath{e^{\mbox{\scriptsize -}}}\ensuremath{e^{\mbox{\scriptsize +}}}\ensuremath{\rightarrow}\ensuremath{Z}\ensuremath{\rightarrow}\ensuremath{q}\ensuremath{\overline{q}}\xspace}
%\def\eLpRbb{\ensuremath{e_L^{\mbox{\scriptsize -}}}\ensuremath{e_R^{\mbox{\scriptsize +}}}\ensuremath{\rightarrow}\ensuremath{b}\ensuremath{\overline{b}}\xspace}
%\def\eRpLbb{\ensuremath{e_R^{\mbox{\scriptsize -}}}\ensuremath{e_L^{\mbox{\scriptsize +}}}\ensuremath{\rightarrow}\ensuremath{b}\ensuremath{\overline{b}}\xspace}
%\def\eLpRcc{\ensuremath{e_L^{\mbox{\scriptsize -}}}\ensuremath{e_R^{\mbox{\scriptsize +}}}\ensuremath{\rightarrow}\ensuremath{c}\ensuremath{\overline{c}}\xspace}
%\def\eRpLcc{\ensuremath{e_R^{\mbox{\scriptsize -}}}\ensuremath{e_L^{\mbox{\scriptsize +}}}\ensuremath{\rightarrow}\ensuremath{c}\ensuremath{\overline{c}}\xspace}
%\def\eLpRqq{\ensuremath{e_L^{\mbox{\scriptsize -}}}\ensuremath{e_R^{\mbox{\scriptsize +}}}\ensuremath{\rightarrow}\ensuremath{q}\ensuremath{\overline{q}}\xspace}
%\def\eRpLqq{\ensuremath{e_R^{\mbox{\scriptsize -}}}\ensuremath{e_L^{\mbox{\scriptsize +}}}\ensuremath{\rightarrow}\ensuremath{q}\ensuremath{\overline{q}}\xspace}
\def\cme{\ensuremath{c.m.e.}\xspace}
\def\eLpR{\ensuremath{e_L^{-}e_R^{+}}\xspace}
\def\eRpL{\ensuremath{e_R^{-}e_L^{+}}\xspace}
\def\eLpRqq{\ensuremath{e_L^{-}e_R^{+}\rightarrow q\bar{q}}\xspace}
\def\eRpLqq{\ensuremath{e_R^{-}e_L^{+}}\rightarrow q\bar{q}\xspace}


%\def\eLpR{\ensuremath{e_L}\ensuremath{p_R}\xspace}
%\def\eRpL{\ensuremath{e_R}\ensuremath{p_L}\xspace}
\def\dEdx{\ensuremath{dE/\/dx}\xspace}
\def\dNdx{\ensuremath{dN/\/dx}\xspace}
\def\Afbb{\ensuremath{A^{b\bar{b}}_{FB}}\xspace}
\def\AFBb{\ensuremath{A^{b\bar{b}}_{FB}}\xspace}
\def\Afb{\ensuremath{A_{FB}}\xspace}
\def\AFB{\ensuremath{A_{FB}}\xspace}

\def\ALR{\ensuremath{A_{LR}}\xspace}
\def\Rb{\ensuremath{R_{b}}\xspace}
\def\Rc{\ensuremath{R_{c}}\xspace}
\def\Rq{\ensuremath{R_{q}}\xspace}
\def\Rqp{\ensuremath{R_{q\prime}}\xspace}
\def\Ruds{\ensuremath{R_{uds}}\xspace}
\def\Rbcostheta{\ensuremath{R_{b}(|cos\theta_{b}|)}\xspace}
\def\Rccostheta{\ensuremath{R_{c}(|cos\theta_{c}|)}\xspace}
\def\Rqcostheta{\ensuremath{R_{q}(|cos\theta_{q}|)}\xspace}
\def\Rudscostheta{\ensuremath{R_{uds}(|cos\theta_{uds}|)}\xspace}
\def\Afbc{\ensuremath{A^{c\bar{c}}_{FB}}\xspace}
\def\Afbq{\ensuremath{A^{q\bar{q}}_{FB}}\xspace}
\def\AFBc{\ensuremath{A^{c\bar{c}}_{FB}}\xspace}
\def\AFBq{\ensuremath{A^{q\bar{q}}_{FB}}\xspace}
\def\eett{\ensuremath{e^{\mbox{\scriptsize +}}}\ensuremath{e^{\mbox{\scriptsize -}}}\ensuremath{\rightarrow}\ensuremath{t}\ensuremath{\overline{t}}\xspace}
\def\bquark{\ensuremath{b}-quark\xspace}
\def\bjet{\ensuremath{b}-jet\xspace}
\def\bjets{\ensuremath{b}-jets\xspace}
\def\btagging{\ensuremath{b}-tagging\xspace}
\def\btag{\ensuremath{b_{tag}}\xspace}
\def\cquark{\ensuremath{c}-quark\xspace}
\def\cjet{\ensuremath{c}-jet\xspace}
\def\cjets{\ensuremath{c}-jets\xspace}
\def\ctagging{\ensuremath{c}-tagging\xspace}
\def\ctag{\ensuremath{c_{tag}}\xspace}
\def\udsjet{\ensuremath{uds}-jet\xspace}
\def\udsjets{\ensuremath{uds}-jets\xspace}
\def\tquark{\ensuremath{t}-quark\xspace}
\def\Zboson{\ensuremath{Z}-boson\xspace}
\def\Zpole{\ensuremath{Z}-pole\xspace}
\def\Zprime{\ensuremath{Z^{\prime}}\xspace}
\def\Zbb{\ensuremath{Z_{b\bar{b}}}\xspace}
\def\ZbRbR{\ensuremath{Z_{b_{R}\bar{b}_R}}\xspace}
\def\ZbLbL{\ensuremath{Z_{b_{L}\bar{b}_L}}\xspace}
\def\Bc{\ensuremath{Vtx}-method\xspace}
\def\Kc{\ensuremath{K}-method\xspace}
\def\BcKc{\ensuremath{Vtx/K}-method\xspace}
\def\BcKcsame{\ensuremath{Vtx/K_{same\,jet}}-method\xspace}
\def\BcBc{\ensuremath{Vtx/Vtx}-method\xspace}
\def\KcKc{\ensuremath{K/K}-method\xspace}

\def\Pb{\ensuremath{P_{chg.}}\xspace}
\def\Qb{\ensuremath{Q_{chg.}}\xspace}
%\def\PbB{\ensuremath{P_{chg.,Vtx}}\xspace}
%\def\PbK{\ensuremath{P_{chg.,K}}\xspace}
%\def\QbB{\ensuremath{q_{chg.,Vtx}}\xspace}
%\def\QbK{\ensuremath{q_{chg.,K}}\xspace}

\def\PbB{\ensuremath{P_{chg.,M_{1}}}\xspace}
\def\PbK{\ensuremath{P_{chg.,M_{2}}}\xspace}
\def\QbB{\ensuremath{Q_{chg.,M_{1}}}\xspace}
\def\QbK{\ensuremath{Q_{chg.,M_{2}}}\xspace}

\def\Pbi{\ensuremath{P_{chg.,i}}\xspace}
\def\Pbj{\ensuremath{P_{chg.,j}}\xspace}
\def\Qbi{\ensuremath{Q_{chg.,i}}\xspace}
\def\Qbj{\ensuremath{Q_{chg.,j}}\xspace}


\def\pmp{\ensuremath{+-}\xspace}
\def\mpp{\ensuremath{-+}\xspace}
\def\pp{\ensuremath{++}\xspace}
\def\mm{\ensuremath{--}\xspace}

\newcommand\blankpage{%
    \null
    \thispagestyle{empty}%
    \addtocounter{page}{-1}%
    \newpage}


\def\nn{\nonumber}
\def \sgn{\text{sgn}}
\def \mul{\textit{et al.~}}
\def \ie{\textit{i.e.~}}
\DeclareMathOperator{\sech}{sech}
\def\hlt{\color{red}}


%colored hyperlink
\usepackage[unicode=true,
  linktocpage,
  linkbordercolor={0.5 0.5 1},
  citebordercolor={0.5 1 0.5},
  linkcolor=blue, citecolor = blue, colorlinks=true]{hyperref}



\definecolor{dgreen}{rgb}{0.0, 0.4, 0.0}

% \definecolor{airforceblue}{rgb}{0.26, 0.54, 0.76}
% \newcommand{\review}[1]{{\textcolor{airforceblue}{#1}}}

\usepackage{lipsum} % random text generator.
\makeatother
\begin{document}
	


\title{The effect of axisymmetric confinement on propulsion of a three-sphere microswimmer}
 


\author{Ali G\"urb\"uz}
\affiliation{Department of Mechanical Engineering,
Santa Clara University, Santa Clara, California, USA}

\author{Andrew Lemus}
\affiliation{Department of Mechanical Engineering,
Santa Clara University, Santa Clara, California, USA}

\author{Ebru Demir}
\affiliation{Department of Mechanical Engineering and Mechanics, Lehigh University, Bethlehem, Pennsylvania, USA}

\author{On Shun Pak}\thanks{Electronic mail: opak@scu.edu}
\affiliation{Department of Mechanical Engineering,
Santa Clara University, Santa Clara, California, USA}

\author{Abdallah Daddi-Moussa-Ider}\thanks{Electronic mail: abdallah.daddi-moussa-ider@ds.mpg.de }
\affiliation{Max Planck Institute for Dynamics and Self-Organization, Göttingen, Germany} % Niedersachsen



\date{\today}
 

\begin{abstract}
Swimming at the microscale has recently garnered substantial attention due to the fundamental biological significance of swimming microorganisms and the wide range of biomedical applications for artificial microswimmers. These microswimmers invariably find themselves surrounded by different confining boundaries, which can impact their locomotion in significant and diverse ways. In this work, we employ a widely used three-sphere swimmer model to investigate the effect of confinement on swimming at low Reynolds numbers. We conduct theoretical analysis via the point-particle approximation and numerical simulations based on the finite element method to examine the motion of the swimmer along the centerline in a capillary tube. The axisymmetric configuration reduces the motion to one-dimensional movement, which allows us to quantify how the degree of confinement affects the propulsion speed in a simple manner. Our results show that the confinement does not significantly affect the propulsion speed until the ratio of the radius of the tube to the radius of the sphere is in the range of $\mathcal{O}(1)-\mathcal{O}(10)$, where the swimmer undergoes substantial reduction in its propulsion speed as the radius of the tube decreases. We provide some physical insights into how reduced hydrodynamic interactions between moving spheres under confinement may hinder the propulsion of the three-sphere swimmer. We also remark that the reduced propulsion performance stands in stark contrast to the enhanced helical propulsion observed in a capillary tube, highlighting how the manifestation of confinement effects can vary qualitatively depending on the propulsion mechanisms employed by the swimmers.
\end{abstract}


\maketitle

\section{Introduction} \label{sec:Intro}
The study of locomotion in fluids at the microscopic scale has attracted significant attention in recent decades. This growing interest is not only driven by the motivation to better understand the motility of swimming microorganisms \cite{Brennen1977,Bray2000,EricLauga2020} but also the potential biomedical applications of artificial microswimmers such as targeted drug delivery and minimally invasive microsurgery \cite{Nelson2010,Sitti2015,JinxingLi2017,WenqiHu2018, Palagi2018,Tsang2020}. Locomotion of biological and artificial microswimmers occurs at negligibly small Reynolds numbers (Re), where viscous forces largely dominate inertial forces. In the inertialess regime, the ability to self-propel is severely constrained owing to kinematic reversibility. In particular, Purcell's scallop theorem \cite{Purcell1977} states that in the absence of inertia, deformations exhibiting time-reversal symmetry (e.g., the motion of a single-hinged scallop opening and closing its shell), also known as reciprocal motion, are unable to produce any net self-propulsion. Common macroscopic swimming strategies such as rigid flapping motion hence become largely ineffective at low Re. Microorganisms such as bacteria and spermatozoa have evolved strategies that utilize biological appendages called flagella with the action of molecular motors to 
swim in their microscopic world. Extensive studies in the past decades have elucidated the physical principles underlying their motility \cite{Fauci06,LaugaPowers2009,Yeomans2014,bechinger2016active,Lauga2016,Wan2022}. 

In parallel efforts, researchers have sought simple and effective mechanisms to develop artificial microswimmers \cite{Abbott2009,Ebbens2010,Lauga2011,sharan2023pair}. In his pioneering work, Purcell demonstrated how a three-link swimmer \cite{Purcell1977}, now known as Purcell’s swimmer \cite{becker2003self,tam2007optimal,Avron2008,Qin2023_B,Qin2023}, can generate net translation with kinematically irreversible cyclic motions. This elegant example has inspired subsequent development of mechanisms that can overcome the fundamental challenge of generating self-propulsion in the inertialess regime. In particular, Najafi and Golestanian \cite{Najafi2004} developed a swimmer consisting of three spheres connected by two extensible rods, which adjust their lengths in a cyclic manner to ingeniously exploit hydrodynamic interactions between the spheres for self-propulsion. The mechanism has also engendered a variety of variants \cite{Avron2005, Earl2007, Golestanian2009, Alouges2008, Alouges2013B, Montino2015,Wang2018, Wang2019, Nasouri2019,Liu2021,Berdakin2022} and their experimental realizations \cite{Leoni2009, Grosjean2016, Box2017,Silverberg_2020}. For its simplicity, the three-sphere swimmer has gained popularity as a useful model for examining different fundamental aspects of locomotion at low Re, including the effect of complex rheology \cite{Curtis2013,yasuda2023generalized}, optimized locomotion \cite{Wang2019,Nasouri2019}, interactions of swimmers \cite{Pooley2007,Farzin2012,yasuda2023generalized}, and swimming near walls \cite{Zargar2009,Najafi2013,Daddi-Moussa-Ider_2018, DaddiMoussaIder2018_2}. 
The three-sphere model has further been used to investigate the reorientation dynamics of microswimmers with respect to flow gradients (rheotaxis)~\cite{daddi2020tuning}, finding that payloads can be exploited to enhance their motion against flows.
More recently, the model has also been employed to explore the integration with machine learning in realizing smart microswimmers \cite{AlanChengHouTsang2020, Hartl2021,Zou2022,Paz2023,Liu2023}. 

% Figure environment removed


Here we utilize the three-sphere swimmer model to probe the effect of confinement on swimming at low Re. Microswimmers invariably find themselves surrounded by different confining boundaries. Extensive studies have demonstrated how swimming near planar boundaries can impact locomotion in significant and diverse manners \cite{Zargar2009,Najafi2013,Daddi-Moussa-Ider_2018, DaddiMoussaIder2018_2, katz_1974,Fauci1995,Ortiz2005,Lauga2006,smith2009,Davide2010,Shum2010,Crowdy2010,Andreas2012,spagnolie_lauga_2012,Li2014,Bayati2019,Farutin2019}. Microorganisms also encounter more complex confinements than planar boundaries, such as spermatozoa swimming through fallopian tubes, parasites \textit{Trypanosomes} in blood vessels, and bacterial motion in microporous soil environments. For swimming inside a capillary tube, previous studies have shown that whether the confinement enhances or hinders propulsion largely depends on the type of swimmers \cite{Felderhof2010,Jana2012,Zhu2013,LedesmaAguilar2013,Liu2014,Caldag2019,ouyang_lin_yu_lin_phan-thien_2022}. For instance, rotating helical flagella, which generate propulsion as a result of drag anisotropy of the slender flagella, display enhanced propulsion speeds inside a capillary tube \cite{Liu2014}. However, squirmers with distribution of tangential surface velocities always have reduced propulsion speeds \cite{Zhu2013}. These results suggest that the difference in the propulsion mechanism of the swimmers play significant roles in how the confinement impacts propulsion. In this work, we consider another physically different propulsion mechanism, namely the three-sphere swimmer model, which relies on hydrodynamic interaction between the spheres for self-propulsion. For simplicity, we focus on the effect of axisymmetric confinement when the swimmer self-propels along the center-line inside a capillary tube. We use both the point-particle approximation and finite element method to quantify how the degree of confinement affects the propulsion speed of this widely used swimmer model. We also provide some physical insights into the underlying mechanism by which confinement influences this specific mode of propulsion.



This paper is organized as follows. We formulate the problem in Sec.~\ref{sec:Formulation} by presenting the swimmer model, the geometrical setup, and the methods of analysis. In Sec.~\ref{sec: Results and discussion}, we first validate our theoretical and numerical results by revisiting the case of an unbounded fluid domain (Sec.~\ref{sec:unbounded}), before discussing new results for confined swimming (Sec.~\ref{sec:Confinement}). We conclude this study in Sec.~\ref{sec:conclusion} with remarks on its limitations and potential directions for future studies.



\section{Problem formulation} \label{sec:Formulation}



\subsection{Swimmer model}\label{sec:SwimmerModel}
We consider the motion of a three-sphere microswimmer confined axisymmetrically in a capillary tube of radius $R_c$. The swimmer was first studied in an unbounded fluid domain by Najafi and Golestanian \cite{Najafi2004}. As illustrated in Fig.~\ref{fig1}(a), the swimmer consists of three spheres of the same radius $R$ connected by two extensible rods of negligible hydrodynamic influences. The fully extended length of each arm is given by $D$ and the fully contracted length of each arm is given by $D-\epsilon$, where $\epsilon$ denotes the amount of contraction or extension in each stroke (referred to as the contraction length hereafter). In the main text, we follow Najafi and Golestanian \cite{Najafi2004} to consider a constant relative speed $W$ in the change of the arm length in the four strokes illustrated in Fig.~\ref{fig1}(b): In stroke I, the swimmer contracts its left arm of an initial length $D$ by an amount $\epsilon$, keeping the length of the right arm at $D$.  In stroke II, the swimmer contracts its right arm by an amount $\epsilon$, keeping the length of the left arm at $D-\epsilon$. In stroke III, the swimmer extends its left arm to reach the fully extended length $D$, with the length of the right arm fixed at $D-\epsilon$. Finally, in stroke IV, the swimmer extends its right arm to return to its original configuration with both arms fully extended with length $D$, completing a full swimming cycle. The net displacement generated by such a cycle is denoted by $\Delta$. In addition to these original strokes considered by Najafi and Golestanian \cite{Najafi2004}, harmonic variations of the length of the two rods have been analyzed in subsequent works \cite{GolestanianAjdari2008}. We conduct the same analyses for the case of harmonic deformations of the rods in the Appendix to assess the generality of our findings.





\subsection{Theoretical analysis: Point-particle approximation}
\label{sec:point-particle}


The motion of an incompressible flow in a Newtonian fluid at low Re is governed by the Stokes equation
\begin{align}
     \mu\nabla^2\textbf{u} &=\nabla p, \label{eqn:Stokes1} \\
     \nabla\cdot\textbf{u} &=0, \label{eqn:Stokes2}
\end{align}
where $\mu$ is the dynamic viscosity, and $\mathbf{u}$ and $p$ are, respectively, the fluid velocity and pressure fields. We denote the velocity of the $i$-th sphere as $\mathbf{V}_i$ and the force and torque acting on them as $\mathbf{F}_i$ and $\mathbf{T}_i$, respectively. No-slip boundary condition are applied on the spheres and the confining tube, \textit{i.e.} $\mathbf{u}_{\text{on the $i$-th sphere}}=\mathbf{V}_i$ and  $\mathbf{u}_{\text{on the confining tube}}=\mathbf{0}$. Without external forces and external torques, the system should be force-free,
\begin{align}
     \sum_{i=1}^{3}{\textbf{F}_i=\textbf{0}}, \label{eqn:Forcefree}
\end{align}
and torque-free,
\begin{align}
     \sum_{i=1}^{3}{\textbf{T}_i=\textbf{0}}. \label{eqn:Torquefree}
\end{align}
As a remark, the torque-free condition is identically satisfied by symmetry  of the problem setup.



%The microswimmer is composed of three spherical beads of the same radius~$R$, arranged co-linearly via rods of negligible hydrodynamic effects. 
We denote by~$\mathbf{r}_1$ the position of the center sphere, which is chosen as a reference for tracking the movement of the swimmer.
We denote by $\mathbf{r}_2$ and~$\mathbf{r}_3$ the positions of front and rear spheres, respectively.
The temporal change in the mutual distances between the spheres is set to perform a non-reversible time sequence.
Under the action of the internally generated forces acting between the spheres along the tube axis ($z$-axis), the lengths of the rod connecting adjacent spheres are set as
\begin{equation}
    \left( \mathbf{r}_2 - \mathbf{r}_1 \right) \cdot \hat{\mathbf{e}}_z = g(t) \, , \qquad
    \left( \mathbf{r}_1 - \mathbf{r}_3 \right) \cdot \hat{\mathbf{e}}_z = h(t) \, ,  
    \label{eq:gh}
\end{equation}
where 
$(g,h) = \left( D,D-Wt \right)$ for $ t \in [0,T/4]$, 
$(g,h) = \left( D+\epsilon-Wt, D-\epsilon \right)$ for $t \in [T/4,T/2]$, 
$(g,h) = \left( D-\epsilon, D-3\epsilon+Wt \right)$ for $t \in [T/2, 3T/4]$, $(g,h) = \left( D-4\epsilon+Wt, D \right)$ for $t \in [3T/4,T]$, and $\hat{\mathbf{e}}_z$ is the unit vector along the $z$-direction.


At low Re, inertial effects are negligible so that the immersed particles take on the velocity of the surrounding fluid instantaneously. 
Accordingly, the translational velocities of the three spheres are related to the internal forces exerted on them linearly via
\begin{equation}
    \mathbf{V}_i = \frac{\mathrm{d} \mathbf{r}_i}{\mathrm{d} t} 
    = \sum_{j=1}^3 \boldsymbol{\mu}_{ij} \cdot \mathbf{F}_j \, , 
\end{equation}
for $i = 1,2,3$, wherein $\boldsymbol{\mu}_{ij}$ stands for the hydrodynamic mobility tensor relating between the translational velocity of sphere~$i$ and the force exerted on sphere~$j$.
The hydrodynamic mobility incorporates the effect of the many-body fluid-mediated interactions between suspended particles.
Here, we confine ourselves for simplicity to the situation in which only contributions stemming from self $(i=j)$ and pair $(i \ne j)$ hydrodynamic interactions are accounted for. 
We will assess the accuracy of our approach with direct comparison with fully resolved numerical simulations based on the finite element method (Sec.~\ref{sec:FEM}). 



In the so-called point-particle approximation, in which $R \ll R_\mathrm{c}$, the scaled self-mobility function is given to leading order in~$R/R_\mathrm{c}$ by~\cite{bohlin1960drag, daddi2017hydrodynamic}
\begin{equation}
    \frac{\mu_{ii}}{\mu_0} = 1 + \delta \, \frac{R}{R_\mathrm{c}} \, , 
    \label{eq:self-mobility}
\end{equation}
wherein $\mu_0 = 1/ \left( 6\pi\eta R \right)$ is the bulk mobility, and
\begin{equation}
    \delta = -\frac{3}{2\pi} 
    \int_0^\infty \frac{A(s)}{B(s)} \, \mathrm{d} s \, . 
\label{eq:self_int}
\end{equation}
Here, we have defined
\begin{subequations}
\begin{align}
    A(s) &= 4 I_1(s) K_0(s) + s^2 \big( I_0 (s) K_1(s) + I_1(s) K_0(s) \big) \notag \\
    &\quad- 2 s \big( I_0(s) K_0(s) + I_1(s) K_1(s) \big) \, , \\ 
    B(s) &=  2 I_0(s) I_1(s) + s \big( I_1(s)^2 - I_0(s)^2 \big) \, , 
\end{align}
\end{subequations}
with $I_\nu$ and~$K_\nu$ denoting the $\nu$-th~order modified Bessel functions (also known as the hyperbolic Bessel functions) of the first and second kinds, respectively.
A numerical evaluation of the infinite integral in Eq.~\eqref{eq:self_int} yields
\begin{equation}
       \delta \simeq -2.10444 \, . 
\label{eq:self}
\end{equation}


Analogously, the hydrodynamic pair mobility in the point-particle approximation is given in a scaled form by~\cite{daddi2017hydrodynamic, daddi2017diffusion}
\begin{equation}
    \frac{\mu_{ij}}{\mu_0} = \frac{3}{2} \frac{R}{R_\mathrm{c}} \left( \frac{1}{\sigma} + \xi_{ij} (\sigma) \right) \, , 
    \label{eq:pair-mobility}
\end{equation}
where
\begin{equation}
    \xi_{ij} (\sigma) = -\frac{1}{\pi} 
    \int_0^\infty \frac{A(s)}{B(s)} \, \cos \left( \sigma s \right) \, \mathrm{d} s \, , 
\label{eq:pair_int}
\end{equation}
wherein $\sigma = \left|(\mathbf{r}_i - \mathbf{r}_j) \cdot \hat{\mathbf{e}}_z\right|/R_\mathrm{c}$.
%$\sigma = \left|z_i - z_j\right|/R_\mathrm{c}$.
Clearly, $\mu_{ij} = \mu_{ji}$, as required by symmetry.
In particular, $\delta/\xi(0) = 3/2$.


It is worth noting that the pair mobility can likewise be expressed in terms of converging infinite series of the form~\cite{cui2002screened}
\begin{equation}
      \frac{\mu_{ij}}{\mu_0} =  
      \frac{3}{4} \sum_{n=1}^{\infty} \varphi_n e^{-\alpha_n \sigma} \, , \label{eq:pair_series}
\end{equation}
where
\begin{equation}
    \varphi_n = a_n \cos(\beta_n \sigma) + b_n \sin (\beta_n \sigma) \, .
\end{equation}
Here, $u_n := \alpha_n + i \beta_n$ are the complex roots of the equation
\begin{equation}
    u_n (J_0^2 (u_n) + J_1^2(u_n)) = 2 J_0(u_n) J_1 (u_n) = 0 \, .
\end{equation}
In addition,
\begin{align}
    a_n + i b_n &= 2 \Big(\pi \big( 2J_1(u_n) Y_0(u_n) - u_n (J_0(u_n) Y_0(u_n) \notag \\
    &\quad \left. + \, J_1(u_n) Y_1(u_n)) \big) - u_n \Big) \middle / J_1^2(u_n) \right. \, , 
\end{align}
where $J_\nu$ and $Y_\nu$ stand for the $\nu$-th order Bessel functions of the first and second kinds, respectively.
Accordingly, the pair mobility function displays a sharp exponential decay as the distance between particles becomes larger.
In the limit $\sigma  \gg 1$, the series in Eq.~\eqref{eq:pair_series} can be truncated to the first term to yield
\begin{equation}
\frac{\mu_{ij}}{\mu_0} \simeq \frac{3}{4} \big( a_1 \cos(\beta_1 \sigma) + b_1 \sin (\beta_1 \sigma) \big) e^{-\alpha_1 \sigma} \, , 
\end{equation}
with the numerical estimates $\alpha_1 \simeq 4.46630$, $\beta_1 \simeq 1.46747$, $a_1 \simeq -0.03698$ and $b_1 \simeq 13.80821$.



Differentiating Eq.~\eqref{eq:gh} with respect to time yields $V_2 = V_1 + \dot{g}$ and~$V_3 = V_1 - \dot{h}$, with dots standing for a time derivative and $V_i =  \mathbf{V}_i \cdot \hat{\mathbf{e}}_z$ denotes the axial velocity along the centerline of the confining tube.
By requiring the force-free condition (Eq.~\ref{eqn:Forcefree}), we find that the instantaneous axial velocity of the center sphere is obtained as
\begin{equation}
    V_1 = \frac{\dot{h} \left( \mu_{ii}-\mu_{12} \right) M_+ - \dot{g} \left( \mu_{ii}-\mu_{13} \right) M_-}{3\mu_{ii}^2 - 2\mu_{ii} \left( \mu_{12} + \mu_{13} + \mu_{23} \right) - N} \, , 
    \label{eq:V1_final}
\end{equation}
wherein $M_\pm = \mu_{ii} \pm \mu_{12} \mp \mu_{13} - \mu_{23}$ and $N = \mu_{12}^2 + \left( \mu_{13} - \mu_{23} \right)^2 - 2 \mu_{12} \left( \mu_{13} + \mu_{23} \right)$.
We have $\big( \dot{g}, \dot{h} \big) = \left(0,-W \right)$ for $t \in [0,T/4]$,
$\big( \dot{g}, \dot{h} \big) = \left(-W, 0 \right)$ for $t \in [T/4, T/2]$,
$\big( \dot{g}, \dot{h} \big) = \left(0,W \right)$ for $t \in [T/2, 3T/4]$,
and $\big( \dot{g}, \dot{h} \big) = \left(W,0 \right)$ for $t \in [3T/4, T]$.
We recall that $\mu_{12} = \mu_{ij} \left( \sigma = g/R_\mathrm{c} \right) $, $\mu_{13} = \mu_{ij} \left( \sigma = h/R_\mathrm{c} \right) $, and $\mu_{23} = \mu_{ij} \left( \sigma = (g+h)/R_\mathrm{c} \right)$.
We note that self, $\mu_{ii}$, and pair, $\mu_{ij}$, mobilities are given by Eqs.~\eqref{eq:self-mobility} and~\eqref{eq:pair-mobility}, respectively.



Finally, the mean swimming velocity is obtained by averaging over one full cycle as
\begin{equation}
    \overline{V}_1 = \frac{1}{T} \int_0^T V_1(t) \, \mathrm{d} t \, . 
\end{equation}


Owing to the delicate and peculiar nature of the resulting axial speed stated by Eq.~\eqref{eq:V1_final}, an analytical evaluation of the mean is rather complicated and far from being trivial, even in the simplistic situation without confinement.
To be able to make analytical progress, we expand perturbatively the axial velocity in the small parameter $R/D$.
By substituting the expressions of the self- and pair-mobility functions, given by Eqs.~\eqref{eq:self-mobility} and~\eqref{eq:pair-mobility}, respectively, into Eq.~\eqref{eq:V1_final}, and noting that $N = \mathcal{O}\left( \left( R/D \right)^2 \right)$, the instantaneous swimming velocity can readily, upon Taylor expansion in the small parameter $R/D$, be cast in the form
\begin{equation}
    V_1 = V_1^\mathrm{B} + V_1^\mathrm{C} + \mathcal{O} \left( \left( \tfrac{R}{D} \right)^2 \right) \, , 
\end{equation}
where $V_1^\mathrm{B}$ in the instantaneous velocity in the absence of confinement, given by
\begin{align}
    V_1^\mathrm{B} &= \Big( \dot{g} \big( (R-2g)h^2 - 2(R+h)g^2 \big) 
    +  \dot{h} \big( (2h-R)g^2 \notag \\
    &\quad+ \left. 2(R+g)h^2 \big) \Big) \middle/ 6gh(g+h) \right. \, .
\end{align}
Moreover, $V_1^\mathrm{C}$ is the confinement-related contribution to the instantaneous velocity, given by
\begin{equation}
    V_1^\mathrm{C} = \frac{R}{6R_\mathrm{c}} \left(\dot{g} \, \Xi_1 - \dot{h} \, \Xi_2 \right) \, , 
\end{equation}
wherein $\Xi_1 = \xi_{12}-2\xi_{13} + \xi_{23}$ and $\Xi_2 = \xi_{13} - 2\xi_{12} + \xi_{23}$.

We find that the bulk-related contribution to the average speed is obtained as
\begin{equation}
    \overline{V}_1^\mathrm{B} = \frac{R}{3T} \left( \frac{2\epsilon^2}{D(D-\epsilon)} 
    + \ln \left( \frac{4D(D-\epsilon)}{\left( 2D-\epsilon \right)^2} \right) \right) \, . \label{eqn:Bulk}
\end{equation}
In particular, for $\epsilon \ll D$, we get
\begin{align}
\overline{V}_1^\mathrm{B} = \frac{7R}{12 T} \left( \left(\frac{\epsilon}{D} \right)^2+\left(\frac{\epsilon}{D} \right)^3  \right) + \mathcal{O} \left( \left(\frac{\epsilon}{D} \right)^4 \right). \label{eqn:BulkAsym}
\end{align}


% Figure environment removed


The contribution to the averaged speed due to confinement can be approximated in the limit $R \ll D$ as
\begin{equation}
    \overline{V}_1^\mathrm{C} = \frac{R}{3\pi T} \int_0^\infty
    \frac{A(s)}{B(s)} \left( \frac{2\epsilon}{R_\mathrm{c}} \, \psi_1(s) - \frac{\psi_2(s)}{s} \right) \mathrm{d} s  \, , 
    \label{eq:V1C}
\end{equation}
where
\begin{align}
    \psi_1 (s) &= \cos \left( \tfrac{D}{R_\mathrm{c}} \, s \right) - 
    \cos \left( \tfrac{D-\epsilon}{R_\mathrm{c}} \, s \right) , \notag \\
    \psi_2 (s) &= \sin \left( \tfrac{2D}{R_\mathrm{c}} \, s \right)
    - 2 \sin \left( \tfrac{2D-\epsilon}{R_\mathrm{c}} \, s \right)
    + \sin \left( \tfrac{2 \left( D-\epsilon\right)}{R_\mathrm{c}} \, s \right) . \notag 
\end{align}
Here, we have swapped the order of integration with respect to~$s$ and~$t$. 
It is worth highlighting that Eq.~\eqref{eq:V1C}, which provides the confinement-related contribution to the averaged swimming speed, remains valid across the entire range of values for $D$ and $R_\mathrm{c}$. The only assumption made to derive the approximate expressions for the swimming speed is that $R$ is significantly smaller than $D$.
Since Eq.~\eqref{eq:V1C} involves infinite integrals over the scaled wavenumber~$s$, corresponding analytical expressions cannot be obtained in the limit $\epsilon \ll D$, unlike the case for the bulk-related contribution given by Eq.~\eqref{eqn:BulkAsym}.







\subsection{Finite element method} \label{sec:FEM}
We also perform fully coupled numerical simulations of the momentum (Eq.~\ref{eqn:Stokes1}) and continuity (Eq.~\ref{eqn:Stokes2}) equations using the finite element method (FEM) implemented in the COMSOL Multiphysics environment. We compare these numerical simulation results, which capture the full sphere-sphere and sphere-confinement hydrodynamic interactions, with predictions based on the point-particle approximation in Sec.~\ref{sec:point-particle}. The axisymmetry of the problem setup reduces the computational complexity of the problem from three-dimensional to two-dimensional. Since Stokes flows have slow spatial decay, in order to minimize any hydrodynamic influence from the ends, we consider cylindrical computational domain of radius $R_c$ and a long axial length of approximately 2000$R$ ($1000R$ in each direction away from the outer spheres). The domain is discretized by about 20,000--35,000 P3--P2 (third-order for fluid velocity and second-order for pressure) triangular mesh elements, with local mesh refinement in the proximity of the three spheres. The degree of freedom is of the order of (0.5$-$1)$~\times 10^6$, depending on the radius of the confining tube. We use the Multifrontal Massively Parallel Sparse (MUMPS) direct solver for all simulations.


Due to the time independence of Stokes flows, the motion of the swimmer is completely determined by its instantaneous movement and geometrical configuration. To simulate the swimming motion over a full cycle, the movement of the swimmer is broken down into separate, stationary simulations for different time instants in individual strokes. At each instant, the velocities of the spheres are determined by the relative motion of the three spheres plus an unknown swimming speed in the axial direction on all spheres. These prescribed velocities on the spheres are implemented as boundary conditions on the spheres. To determine the unknown swimming speed, the force-free condition (Eq.~\ref{eqn:Forcefree}) is implemented as a global equation, which is solved together with the momentum and continuity equations to obtain the swimming speed, velocity field, and pressure field simultaneously at each time instant in the swimming cycle. We then perform numerical integration of the swimming speed over a full cycle to obtain the net displacement of the swimmer per cycle, $\Delta$.

% Figure environment removed

\section{Results and discussion} \label{sec: Results and discussion}

In the following sections, we first cross-validate the point-particle approximation and numerical simulations based on the finite element method by considering the motion of a three-sphere swimmer in an unbounded fluid domain in Sec.~\ref{sec:unbounded}. We then characterize in Sec.~\ref{sec:Confinement} the effect of axisymmetric confinement on the propulsion performance of the three-sphere swimmer for different levels of confinement and properties of the swimmer.



% Figure environment removed

\subsection{Swimming in an unbounded fluid} \label{sec:unbounded}


% Figure environment removed

For validation, we consider the motion of the three-sphere swimmer in an unbounded fluid domain. In the particle-particle approximation (Sec.~\ref{sec:point-particle}), the bulk-related contribution to the average speed $\overline{V}^B_1$ is given by Eq.~\ref{eqn:Bulk}, which can be multiplied by the period $T$ to obtain the net displacement of the swimmer per cycle, $\Delta$, shown in Fig.~\ref{fig2} (black solid line). When $\epsilon \ll D$, the net displacement is calculated from Eq.~\ref{eqn:BulkAsym} as 
\begin{align}
\Delta \sim \frac{7R}{12} \left( \left(\frac{\epsilon}{D} \right)^2+\left(\frac{\epsilon}{D} \right)^3  \right), \label{eqn:AsymDisp}
\end{align}
which is represented by the black-dashed line in Fig.~\ref{fig2}. We remark that the asymptotic result given in Eq.~\ref{eqn:AsymDisp} is consistent with that given by Earl \textit{et al.} \cite{Earl2007}, which rectified the result presented in Najafi and Golestanian \cite{Najafi2004}. The asymptotic result in Eq.~\ref{eqn:AsymDisp} reveals that the net displacement of the swimmer per cycle scales quadratically with the contraction length of the swimmer, $\Delta=\mathcal{O}(\epsilon^2)$, in the regime of $\epsilon/D \ll 1$. 

To compare with the above theoretical results, in the FEM simulations we use an exceedingly large radius of confinement ($R_c/R = 1000$) to simulate the swimming motion in an unbounded fluid domain.
%($R_c = 100R$ ) 
The FEM results are represented by blue circles in Fig.~\ref{fig2}. The comparison between theoretical and numerical results show that the point-particle approximation captures quantitatively the propulsion behaviors for small to moderate contraction lengths of the swimmer, where the spheres are sufficiently distanced from each other throughout the swimming cycle. For larger contraction lengths, the spheres come closer to each other during contraction, leading to more significant hydrodynamic interactions between the spheres. Consequently, the point-particle approximation starts to deviate from the FEM results, over-estimating the net displacement of the swimmer \cite{Earl2007,Paz2023}. Despite these deviations, the point-particle approximation continues to capture the qualitative trend of the propulsion behavior.



\subsection{Swimming under axisymmetric confinement} \label{sec:Confinement}


In Fig.~\ref{fig3}, we probe the effect of axisymmetric confinement by examining the net displacement of the three-sphere swimmer along the centerline of a capillary tube. Here we keep the contraction length constant ($\epsilon/R = 4$) and only vary the radius of the tube, $R_c$. For a given fully extended arm length $D$, the effect of confinement becomes significant when $R_c/R$ is of $\mathcal{O}(1)-\mathcal{O}(10)$: in this regime, the results reveal that a tighter confinement (decreasing $R_c$) substantially reduces the net displacement of the swimmer. We note that this trend is in stark contrast with helical propulsion in a capillary tube \cite{Liu2014}, where the confinement largely enhances propulsion, illustrating how confinement can have qualitatively distinct effects on swimming depending on the underlying propulsion mechanisms. Results from the FEM simulations (symbols) and point-particle approximation (lines) agree well when the spheres are separated by large arm lengths (e.g., $D/R=15, \ 20$), when the sphere-sphere and sphere-confinement hydrodynamic interactions are expected to be weaker. For $D/R=10$, the point-particle approximation still captures properly the qualitative feature of the confined swimming motion when the spheres are in closer proximity, where the near-field effects become more pronounced. As a remark, while the major effect here is a substantial and rapid decay in propulsion under tight confinements, the swimmer also displays a very slight enhancement in propulsion when the confining radius is large (e.g., when $R_c/R$ is beyond $O(10)$ for $D/R=10$). This minute effect, captured by both the FEM simulations and point-particle approximation, is observed for all values of $D/R$ presented in Fig.~\ref{fig3} and is more apparent for the case of $D/R=10$ (inset).


Next, we probe how the net displacement of the swimmer varies with its contraction length when swimming in a capillary tube (Fig.~\ref{fig4}). While the net displacement of the swimmer grows with the contraction length in general, it occurs at different rates depending on the degree of confinement (\textit{i.e}, the value of $R_c/R$). In an unbounded fluid, the net displacement grows quadratically with the contraction length, $\Delta/R \sim 7 \epsilon^2/(12D)$, as given by Eq.~\ref{eqn:AsymDisp}. In Fig.~\ref{fig4} inset, we consider a log-log plot of the results to better visualize the scaling. For a relatively loose confinement $R_c/R = 10$ (downside black triangles and black dot-dash line), the log-log plot shows an approximately quadratic scaling between the net displacement and the contraction length, similar to the case in an unbounded fluid. However, as the environment becomes more confined ($R_c/R=2.5$ and $5$), results from both point-particle approximation (red dashed line and blue solid line) and FEM (red upside triangles and blue circles) indicate slopes increasingly greater than two in the inset. These results illustrate that the scaling goes beyond second-order in confined swimming; the three-sphere mechanism becomes increasingly ineffective in generating a net displacement under tighter confinement.



To develop a more physical understanding of the above results, we revisit the symmetry arguments by Najafi and Golestanian \cite{Najafi2004} that showed how the four strokes in the cycle are related. These arguments remain valid for the swimmer under axisymmetric confinement considered in this work: stroke III is related to stroke II upon a left-right reflection and a time-reversal transformations, whereas stroke IV is related to stroke I with the same transformations. Consequently, the net displacement of the swimmer after executing a full cycle is simply reduced to (two times) the difference in the net displacement of the center sphere generated by stroke I and stroke II. Stroke I generates a net displacement to the left, while stroke II generates a net displacement to the right. It is crucial to note that these net displacements differ in their magnitudes because the force acting on the spheres when they are far apart (in stroke I) is different from when they are in closer proximity (in stroke II) due to interactions between the spheres via their surrounding flows. The hydrodynamic interactions lead to only partial cancellation of the displacements generated by strokes I and II, giving rise to the net displacement of the swimmer after a cycle. When the hydrodynamic interaction is neglected, the two strokes would generate displacements with equal magnitudes in opposite directions, cancelling each other and yielding zero net propulsion. 



Based on the above understanding of the propulsion mechanism, we attribute the reduced net displacement of the confined swimmer to weakened hydrodynamic interactions among the spheres under confinement as follows. It was shown that the flow due to a Stokeslet decays exponentially in a capillary tube due to the confinement \cite{liron_shahar_1978}, as opposed to decaying as the inverse of the distance in an unbounded fluid. The flow around the moving spheres of the swimmer in a capillary tube is therefore expected to decay more rapidly in space. 
To visualize this effect, we plot in Fig.~\ref{fig5} the flow field surrounding the swimmer at different time instants in a swimming cycle with different levels of confinement. As the radius of the confining tube decreases from $R_c/R=10$ in panel (a) to $R_c/R=2.5$ in panel (c), the magnitude of the flow around individual spheres can be observed to decay more rapidly away from the spheres. These faster spatial decays of the flow velocity weaken the hydrodynamic interaction between the spheres, thereby reducing the hydrodynamic difference between the case when the spheres are more far apart in stroke I and the case when they are in closer proximity in stroke II. The reduced hydrodynamic difference between the two strokes therefore generates displacements with more similar magnitudes, leading to the reduced net displacements of the swimmer in a capillary tube as observed in Fig.~\ref{fig3}.


\section{Concluding Remarks} \label{sec:conclusion}


In this work, we examine the propulsion of a three-sphere swimmer along the centerline of a capillary tube at low Re. We combine theoretical analysis via the point-particle approximation and simulations based on the finite element method to uncover how the propulsion speed varies with the radius of the confining tube as well as geometric and kinematic properties of the swimmer. The results show that the presence of confinement does not significantly affect the propulsion speed until the scaled radius of the confining tube is in the range of $R_c/R = \mathcal{O}(1)-\mathcal{O}(10)$, where the swimmer exhibits sharp decays in propulsion speed as the radius of the tube decreases. The presence of confinement also leads to higher-order scaling between the net displacement and the contraction length of the swimmer, reducing the effectiveness of this propulsion mechanism. We contrast the reduced propulsion speed observed here with the enhanced helical propulsion inside a capillary tube reported earlier \cite{Liu2014}, highlighting how the effect of confinement can manifest in qualitatively different manners depending on the swimmer's propulsion mechanism. While helical propulsion is based on the drag anisotropy of slender bodies, the three-sphere swimmer here relies on the sphere-sphere hydrodynamic interactions--a physically different mechanism--to self-propel. The reduced propulsion performance observed here is attributed to the more rapid spatial decays of the flow velocity of moving bodies in a tube, which reduces the hydrodynamic interaction between the spheres and thereby the net displacement of the swimmer.   

Based on the above physical understanding of the results, we hypothesize that the propulsion of a three-sphere swimmer in porous media may also be hindered due to the screening of hydrodynamic interactions by networks of obstacles, in contrast to enhanced propulsion predicted for  different types of swimmers in heterogeneous viscous environments \cite{Leshansky2009,Fu_2010,Nguyenho2016,Leiderman2016,PhysRevFluids.3.094102, PhysRevResearch.5.033030, hosaka2023hydrodynamics}. An investigation is currently underway to evaluate this hypothesis and will be reported in a future work. Furthermore, we considered the effect due to rigid confinement in this work, while the fluid-structure interaction between the swimmer and elastic confinements can have profound impacts on the swimming performance \cite{LedesmaAguilar2013,daddi2019frequency,Dalal2020}. It would be worthwhile to consider the case of an elastic tube and systematically examine the interplay between shear and bending deformation modes in prescribing the hydrodynamics of the swimmer under elastic confinement. Finally, we focus on the effect of axisymmetric confinement here to preserve the one-dimensional nature of the motion, which allows us to measure how the degree of confinement affects the propulsion speed in a simple manner. Lifting this restriction to examine more general motion of a three-sphere swimmer in a capillary tube could lead to more complex and interesting swimming dynamics in future studies.\\






 \begin{acknowledgments}
O.S.P. acknowledges funding support by the National Science Foundation (Grant No.~1830958). A.D.-M.-I. acknowledges support from the Max Planck Center Twente for Complex Fluid Dynamics, the Max Planck School Matter to Life, and the MaxSynBio Consortium, which are jointly funded by the Federal Ministry of Education and Research (BMBF) of Germany and the Max Planck Society. We are also grateful for the computational resources from the WAVE computing facility (enabled by the E.~L.~Wiegand Foundation) at Santa Clara University.
 \end{acknowledgments}

 \appendix 

 \section{A confined three-sphere swimmer with harmonic oscillations of the rod lengths} \label{sec:Appendix}
We follow Najafi and Golestanian \cite{Najafi2004} in the main text in considering a constant relative speed $W$ in the change of arm length. In this appendix, we consider also harmonic deformations of the arms \cite{GolestanianAjdari2008} to establish some generality of the  conclusion. Specifically, we prescribe the following variations, respectively, for the length of front and rear rods,
\begin{align}
g(t) &= D-\frac{\epsilon}{2}+\frac{\epsilon}{2}\cos (\omega t),\label{eqn:HarmonicDef1}\\
h(t) &= D-\frac{\epsilon}{2}+\frac{\epsilon}{2}\cos (\omega t+\phi). \label{eqn:HarmonicDef2}
\end{align}
The two rods have an equilibrium length of $D-\epsilon/2$ with sinusoidal oscillations of amplitude $\epsilon/2$, angular frequency $\omega$, and a phase mismatch $\phi$. Here we set $\omega = \pi W/\epsilon$, so that the period of oscillation is given by $T=2\epsilon/W$. As a remark, when $\phi=0$, the swimmer generates zero net propulsion by symmetry; when $\phi=\pi$, the overall deformation of the swimmer becomes reciprocal motion, which also leads to zero net propulsion as dictated by the scallop theorem. Here we present results for the specific case of $\phi=\pi/2$, which was shown to generate the maximum amount of net displacement of the swimmer in an unbounded fluid \cite{GolestanianAjdari2008}. As shown in Figs.~\ref{App} and \ref{App_Eps}, a confined swimmer with harmonic variations of its arm lengths exhibit qualitatively the same behaviors, compared with the case of a constant rate of change of the arm lengths examined in the main text (Figs.~\ref{fig3} and \ref{fig4}).


% Figure environment removed


% Figure environment removed

 \section*{Data Availability}
The data that support the findings of this study are available from the corresponding author upon reasonable request.




\documentclass[onecolumn,preprintnumbers,amsmath,amssymb,floatfix,superscriptaddress]{revtex4}
\linespread{2.4}
\usepackage{graphicx}% Include figure files
\usepackage{dcolumn}% Align table columns on decimal point
\usepackage{epstopdf}
\usepackage{bm}% bold math
\usepackage{amsmath}
\usepackage{amssymb}   % Used for getting various symbols eg. gtrsim, lesssim etc.
\DeclareMathOperator{\E}{\mathbb{E}}
\usepackage{bm} % For bold math (esp of lower greek letters)
\usepackage{dcolumn}% Align table columns on decimal point
\usepackage{color}
\usepackage{mathrsfs}
\usepackage{mathtools}
\usepackage{amsfonts}
\usepackage{varioref}
\usepackage{epstopdf}
\usepackage{pdfpages}
\usepackage{float}
\usepackage[dvipsnames]{xcolor}
\usepackage{color,soul}
\usepackage{makeidx}
\usepackage{showlabels}
\graphicspath{ {/home/} }
\RequirePackage[colorlinks,citecolor=blue,urlcolor=magenta,linkcolor=blue]{hyperref}



	


\begin{document}
	
	
	\title{\large \bf Remixing of a phase separated binary colloidal system with particles of different sizes in an external modulation}
	
	
	\author{\bf Suravi Pal}
	\affiliation{Physics of complex systems, S. N. Bose National Centre for Basic Sciences, JD Block, Sector-III, Salt Lake, Kolkata 700106, India \\{suravipal@bose.res.in}\\{\bf \large and}}
	
	\author{\bf J. Chakrabarti}
	\affiliation{Physics of complex systems, S. N. Bose National Centre for Basic Sciences, JD Block, Sector-III, Salt Lake, Kolkata 700106, India \\{jaydeb@bose.res.in}}
	
	
	\author{\bf Srabani Chakrabarty nee Sarkar}
	\affiliation{Department of Physics, Lady Brabourne College, P-1/2, Suhrawardy Ave, Beniapukur, Kolkata, West Bengal 700017, India \\{srabanichakrabarty65@gmail.com}}
	
	
	
	
	
	
	\begin{abstract}
		We explore  phase behaviour of a binary colloidal system under external spatially periodic modulation. We perform Monte Carlo simulation on a binary mixture of big and small repulsive Lennard-Jones particles with diameter ratio 1:2. We characterise structure by isotropic and anisotropic pair correlation function, cluster size distribution, bond angle distribution, order parameter and specific heat. We observe demixing of the species in the absence of the external modulation. However, mixing of the species gets enhanced with increasing potential strength. The de-mixing order parameter shows discontinuity and the specific heat shows a peak with increasing modulation strength, characterizing a first order phase transition
	\end{abstract}
	
	\maketitle 
	
	Keywords: Modulated liquid, Binary colloid, Phase separation, order parameter\\
	
	
\section{Introduction}
 
 
Mixing of different components in condensed phase is of paramount importance in systems ranging from alloys\cite{paper 1, paper 2, paper 3} to those as complicated as biological cells\cite{paper 4, paper 5}. Tuning miscibility among macro-molecules is thus an active area of research, spanning a diverse areas as well as technological applications. Here we examine the mixing property of a binary colloidal macro-molecular system in the presence of an external modulation potential.

Colloids  are ideal systems to explore condensed phase properties at microscopic level following the individual particle motions. This is so because the colloidal particles  can be probed via laser light scattering and optical microscopy \cite{paper 6, paper 7, paper 8} due to their length size and slow movement. Colloids can be easily modulated  by external perturbations, like laser field, electric field\cite{paper 9}, magnetic field\cite{paper 10} and  shear\cite{paper 11}. Modulated structures are often important for technological applications as well. Understanding colloidal systems under external perturbations from microscopic considerations has drawn considerable research interests\cite{paper 12, paper 12, paper 13, paper 14, paper 15, paper 16, paper 17}. Modulated colloids often show nontrivial behaviour. For instance, a mono-disperse colloidal system subject to a one-dimensional stationary laser modulation with wavelength matching with the mean inter-particle separation undergo re-entrant melting with increasing modulation strength \cite{paper 18, paper 19, paper 20}. Many interesting studies are reported on colloids under external potential\cite{paper 21, paper 22, paper 23}, such as  two-dimensional melting behavior of super-paramagnetic colloidal particles under quenched disorder\cite{paper 24},  freezing and melting of a colloidal adsorbate on a one dimensional quasi-crystalline substrate\cite{paper 25}, effective forces in modulated colloids\cite{paper 26}, modulated phases in dense colloids \cite{paper 27}, heterogeneous dynamics of colloids under external potential \cite{paper 28, paper 29, paper 30} and so on. 

Experiments on a binary colloidal system with particles of different sizes subject to a spatially periodic potential has also been reported  using the external modulation generated in a given direction by interfering laser beams\cite{paper 31}. The wavelength of the modulation is equal to the size of the bigger particles, and the modulation is stronger for them than the smaller particles. The arrangement of the particles are recorded using optical microscopy and characterized in terms of various static quantities. It is observed that an increase in the external modulation  amplitude leads to  localization of the large particles analogous to a modulated liquid\cite{paper 32} with an increasing fraction of small particles caged by the bigger particles. The smaller particles arrange themselves in a triangular fashion inside those cages.
The large particles are observed to remains ordered, whereas the lattice of small particles became disordered upon increasing the temperature. 

Numerous demixing studies have been reported on colloidal mixtures. There have been studies on demixing of model colloids composed of two different sizes of polystyrene spheres using diffusing wave  spectroscopy \cite{paper 33}. Experiments on a binary dispersion of like-charged colloidal particles with large charge asymmetry but similar size exhibit phase separation into crystal and fluid phases under very low salt conditions. Here the colloid–ion interactions provide a driving force for crystallization of one species\cite{paper 34}. Confocal-microscopy studies on demixing and remixing with temperature in binary liquids containing colloidal particles has been reported \cite{paper 35, paper 36}. However, there has been no study yet to the best of our knowledge how an external modulation changes the demixing behaviour in a binary colloid. 

In this backdrop we study a model binary system with different particles of  diameter ratio 1:2 in an external modulation as in Ref.\cite{paper 37}. The external modulation  wavelength is equal to the diameter of the big particles and the strength of modulation is twice for the big particles than the small ones. We study the system using the Monte Carlo (MC) simulation at room temperature where the Metropolis sampling has been performed based on the energy cost of a particle movement due to interaction with all other particles and the external modulation. For a fixed 1:1 composition and particle density as in Ref.\cite{paper 37}, we vary the external modulation strength. We calculate pair correlation function, cluster size distribution and  bond angle parameter  to characterize structural changes in the system with the strength of external modulation. We also compute thermodynamic quantities such as demixing order parameter and specific heat in order to investigate the phase transition as the structural changes occur in the system.  

We observe that the two components in the system undergo phase separation in absence of any external modulation at room temperature.  presence of external potential, the bigger particles align themselves along the potential minima due to the matching of their diameter with the wavelength of the external potential and stronger modulation strength, while the clusters of the smaller particles break into smaller clusters. We observe hexagonal ordering among the small particles within the clusters. The enhanced remixing of the species with increasing field strengths is reflected in the decrease of the mean cluster size and the shift of the mean demixing order parameter to lower values with discontinuity as a function of the external potential.  The specific heat shows peak as a function of the modulation strength, suggesting the presence of a thermodynamic first order phase transition. 


\section{System details}	

The particles are taken to interact via repulsive potential of the form:
\begin{equation}
V^{(\alpha\beta)}_{ij}(r)=4\epsilon_{\alpha\beta} (\frac{\sigma_{\alpha\beta}}{r_{ij}})^{12}
\end{equation}
\noindent Here $r_{ij}$ is the distance between $i^{th}$ and $j^{th}$ particles belonging to species $\alpha(=b,s)$ and $\beta(=b,s) $ respectively.  Here $\alpha=\beta$ corresponds to the particles of the same species, while $\alpha\neq\beta$ corresponds to the cross species interactions. We take $\sigma_{bb}=D_{b},\sigma_{ss}=D_{s}$ and $\sigma_{bs}=0.5(\sigma_{bb}+\sigma_{ss})$, D$_{b}$ and D$_{s}$ being the diameter of the big and small particles respectively. We fix  $\frac{\epsilon_{bb}}{K_BT}=\frac{\epsilon_{ss}}{K_BT}=1.0$ and  vary $\frac{\epsilon_{bs}}{K_BT}$. 
We further subject the system to an external spatially periodic potential of the form:
\begin{equation}
V^{\alpha}_{ext}(x)=-V_0^{\alpha}cos(\frac{2\pi x}{\lambda}),
\end{equation}
where $x$ denote the x-coordinate of the particles, $\lambda$  the wavelength of the external potential equal to the diameter of the bigger particles, and $V_0^b= V_0= 2V_{0}^s$.

Here $D_b(=5\mu m)$ and $D_s(=2.5\mu m)$. The packing fraction, $\eta(=N_d/4\pi D_b^2+N_s/4D_s^2)/A$, where $A$ is the area of the system) = 0.72. We take $D_s$ as the unit of length and $K_BT$ for room temperature as the unit of energy. Monte Carlo (MC) simulations are carried out on a fixed N number of particles with equal number of big and small particles in a box of reduced length in x-direction, $\frac{L_x}{D_s}$ and that in y-direction $L_y=\frac{\sqrt{3}}{2}L_x$.  We run for a total of $10^6$ MC steps where the equilibration is judged from the energy values. Different quantities of interests are calculated over the equilibrated trajectories and averaged over five independent runs. Most of the results are reported on N=1024 particles. We also study the system for N=144 and N=4096  keeping the area fraction fixed to check the size dependence around the structural changes.


\section{Results and discussions}	

{\it Structure without modulation}

Let us consider  the system in absence of external potential$(\beta V_0=0.0)$. Fig. 1(a) shows a typical particle arrangement in equilibrium. We observe clusters of small particles in the background  of the big particles. We characterise the structural correlations from the histograms of the separation between particle pairs, also called the radial distribution function (rdf) $g_{\alpha\beta}(r)$\cite{paper 38}. This quantity describes the probability of finding a pair of particles belonging to species $\alpha$ and $\beta$ at separation $r$.  Fig. 1 (b) shows the rdf for big-big $g_{bb}(r)$, small-small $g_{ss}(r)$, big-small $g_{bs}(r)$ pair of particles in absence of external potential.  We observe liquid-like short ranges structure behaviour in both $g_{bb}(r)$ and $g_{ss}(r)$. However, $g_{bs}(r)$ is much weaker. This means that the two species do not find in the vicinity of each other due to phase separation between them.

{\it Modulated system}

Let us now consider the effect of the external potential. We show equilibrated snapshot for $\beta V_0=5.0$ in fig. 2(a). The snapshot shows that the big particles are aligned along the potential minima due to the matching of wavelength of external potential to the diameter of the bigger particles and stronger modulation strength than the smaller particles. The clusters of the smaller particles break into smaller clusters to make way for the bigger particles. 

The anisotropic structure of the system is characterized by the pair correlation functions (PCF) in Figs. 2(b)-(d). They  are obtained by binning the pair separations both in x- and y-directions. It may be noted that the rdf is the circular symmetric counter-part of the PCF. $g_{bb}(x,y)$ (Fig.2(b)) shows parallel strips along potential minima as per the periodicity of the external potential. This further confirms preferred particle positions at the modulation minima. $g_{ss}(x,y)$ in Fig.2 (c) shows dark patches on overall circular pattern, signifying slight deviation from isotropic structure. Due to breaking of clusters, the smaller particles  arrange themselves nearer the bigger species. This leads to peaked structure in the big-small correlations $g_{bs}(x,y)$ (Fig. 2(d)), suggesting enhanced mixing tendency between the two species which are phase separated in the absence of the external potential.

The alignment tendency of the particles is quantified by the relative orientation ($\theta_{ij}$) between $i^{th}$ and $j^{th}$ particles with respect to modulation direction,  $\theta_{ij}=tan^{-1}(\frac{y_j-y_i}{x_j-x_i})$.  We show in Fig. 2(e) and (f) the distribution $P(\theta)$ of bond angle $\theta_{ij}$ considering all the  pairs over equilibrium configurations. Data for the big particles $P^{(b)}(\theta)$ and small particles $P^{(s)}(\theta)$ are shown in Fig. 2(e) and (f) respectively. Peaks in $P^{b}(\theta)$ are at $\theta = 90^{\circ}$ and $270 ^{\circ}$, indicating alignment along the minima of the external potential. The smaller peaks in $P^{b}(\theta)$ at $\theta = 0^{\circ}$, $180^{\circ}$ and $360^{\circ}$ are due to  neighbouring big particles. $P^{(s)}(\theta)$ versus $\theta$ for smaller particle (Fig. 2(e)) shows peaks at integral multiples of $30^{\circ}$. This suggests that the smaller particles are forming clusters with a local hexagonal order among themselves, similar observation that reported in earlier experimental work \cite{paper 37}. 

All the tendencies enhance with increasing  $ \beta V_0$ as can be seen in SI show stronger alignment of bigger particles along the potential minima.(Fig. SI 1(a)) The clusters of the small particles further break up into smaller clusters, thus the big-small species shows enhanced tendency of mixing together (Fig. SI 1(b)-(d)).  Even the smaller particles also tend to get aligned along the potential minima now. The alignment tendency is further reflected by the strong peak of in both $P^{b}(\theta)$(Fig. SI 1(e)) and $P^{s}(\theta)$(Fig. SI 1(f)) at $\theta=90^{\circ}$.

{\it Mixing behaviour}

We further quantify the mixing tendency in the presence of the external modulation. First, we characterize the breaking of the small particle clusters. We compute the size of clusters $c$ of the small particles as follows. We take the  small particles to belong to a cluster whose  centre to centre distances in x- and y-direction are less than $x_{cl}$ and $ y_{cl}$ respectively. We choose $x_{cl}=1.1$ and $y_{cl}=1.25$. Here $x_cl$ corresponds to the first minimum in $g_ss(r)$ and $y_cl$ is slightly larger. We show distribution $P(c)$ for smaller particles in Fig. 3(a)-(b) for two different $V_0$'s. Large clusters of the smaller particles are observed in agreement to phase separation between two species in absence of external potential. It is clear that the distribution peaks are shifted to lower values of cluster sizes $c$ for $\beta V_0(=10.0)$(Fig. 3(b)).  

We plot the mean cluster size $\langle c \rangle (=\langle c \rangle = \int cP(c)dc)$ versus $\beta V_0$ in Fig. 3(c). We observe that $\langle c \rangle =Ae^{-B(\beta V_0)}$ with $A=-0.81$ and $B=4.8$ as shown in the inset of Fig. 3(c). The exponential decay suggests that energy needed to break the clusters is supplied by the external modulation. Note that $<c>$ falls to $\frac{1}{e}$ at $V^*=\beta V_0=1/B$ as shown in Fig. 3(d). We observe that  $V^*\sim\epsilon_{bs}^{2.4}$ in the inset of fig.3(d).  

We  calculate the demixing order parameter for different $\beta V_0$. We partition to this end the simulation box into small rectangular cells with dimension $\Delta x=1.1$, $\Delta y=1.25$ as per the appearance of first peak in anisotropic PCF data. Let $n_{ib}$ and $n_{is}$ be the numbers of big and small particles in each of the boxes respectively. The demixing order parameter is defined as:
\begin{equation}
O_d=\frac{1}{N}\Sigma_{i=1}^{M}|(n_{ib}-n_{is})|.
\end{equation}
Here $N$ is the total number of particles and $M$ the total number of partitions of the system. Fig. 4(a) shows the distribution $P(O_d)$ for all three cases for different $\beta V_0$. We observe a prominent peak at large $O_d$ value for $\beta V_0=0.0$. The peak shifts to a lower $O_d$ value, indicating mixing enhanced with $\beta V_0$. We show in Fig. 4(b) the  order parameter averaged over the configurations $\langle O_d \rangle$ vs $\beta V_0$. We observe that $\langle O_d \rangle$ decreases with $\beta V_0$ following two distinct branches, one branch for low values and the other one for higher values with discontinuity around $\beta V_0$=3.0. 

We also simulate the system behaviour for different system sizes, namely, N=144 and N=4096. The data for $\langle O_d \rangle$ in Fig. 4(c) and 4(d) also show discontinuity. The discontinuity in $\langle O_d \rangle$, $\Delta$  is estimated as the gap between the points where the discontinuity starts at the upper branch and  the lower branch. Fig. 4(e) shows the gap $\Delta$ for different N. We observe that this discontinuity increases systematically, confirming a first order transition. 

We also examine the behaviour of the specific heat of the system,  using the fluctuation of energy: 
\begin{equation}
C_v=<\frac{(E-\bar{E})^2}{N}>,
\end{equation}
Here E is the energy per particle at a given configuration and $\bar{E}$ is the average energy per particle.  Fig. 4(f) shows the $C_v$ vs. $\beta V_0$ plot. We observe peak in specific heat at  $\beta V_0$=3.0 where  the discontinuity in $\langle O_d \rangle$ is located. 

The peak in specific heat data, shown in Fig.4(f), show shifts to larger $\beta V_0$ with increasing as $N$, although the point of the order parameter discontinuity does not seem to depend sensitively on N.  The  peak values, of the heat capacity  $C_{max}$ vs. $N$ plot  in inset of Fig. 4(f) show linear dependence on $N$ dependence, as in case of first order transition\cite{paper 39}. Thus, system undergoes first order phase transition while going from a complete phase separated state to a mixed one in presence of an external modulation potential.


Physically, the big particles tend to get aligned more strongly than the small ones, making their ways splitting through the clusters of smaller particles with increasing $\beta V_0$. The surface energy cost of this breaking up is supported by the external potential. The minimum value of this external potential above which the cluster will break can be estimated qualitatively.  Let us consider for simplicity the case that a cluster of radius R breaks into two of radii $r_1$ and $r_2$. The change in line energy is given by:$\Delta f = 2\pi \sigma(r_1+r_2-R)$ where $\sigma$ is the coefficient of line tension. The minimum of the change in line free energy can be estimated by minimizing with respect to one radius $r_1$ and $r_2$ if we  assume that the mean density $\rho$ of the particles do not change due to breaking cluster and no particle is lost during the break up. This implies that $r_1^2+r_2^2=R^2$. Using this relation and minimizing with respect to $r_1$, we find $\Delta f_{min}=2\pi\sigma R(\sqrt{2}-1)$. The droplet radius $R$ consisting of small particles  in the background of the large particles can be stabilized if the total free energy consisting of the bulk and line tension component is at minimum. If $\delta g$ is the free energy per unit area of the droplet, then the minimum free energy corresponds to $R=\sigma/\delta g$. Hence, $\Delta f_{min}\sim \sigma^{2}$. The line tension experienced by the droplet of the smaller particles in the background of the larger particles is proportional to $\epsilon{bs}$ and consequently, $\Delta f_{min}\sim\epsilon^2_{bs}$. This amount of energy needs to be supplied through the external potential, implying that $\beta V_{0}\sim\epsilon^2_{bs}$. This is consistent to the simulation observation  of $V^*\sim\epsilon_{bs}^{2.4}$ dependence.


Demixing phase transition is well known in binary systems. Typically demixing is known to show phase coexistence ending in a critical point\cite{paper 40}. The critical point for demixing of binary colloids tuning the size ratio has  been reported \cite{paper 41}. Shifting of critical point in case of a critical colloidal-liquid to colloidal-liquid demixing phase transition can be controlled by solvent temperature\cite{paper 42}. Experiment on suspensions of colloids of differing charge but similar size undergoing phase separation into crystal and fluid phases has also been reported previously\cite{paper 43}. Phase transition due to demixing between phases rich in high-charge and low-charge colloids is observed in this case. First order phase transition in demixing due to size anisometry in colloids has been reported as well\cite{paper 44}. It may be noted that in contrast to the earlier reports, we observe here mixing of demixed binary colloids via a first order phase transition induced by an external modulation . 


\section{Conclusion}

To summarize, we study using the MC simulations the mixing behaviour of a binary  colloidal film periodically modulated in one direction. The interplay between interaction and external potential results in an enhanced tendency of mixing among the two species which are de-mixed in absence of the modulation. The changes from de-mixed to mixed  state is accompanied by a first order phase transition with a jump in de-mixing order parameter. Our results suggest that mixing in a binary colloidal system can be tuned by suitable external potential which may be relevant in  technological applications. It will be interesting to study this phase transition including the critical behaviour, if at all, in future.



\section{Acknowledgements}

 The authors declare no conflict of interest. S. P. thanks computational centre at SNBNCBS for providing with the high computing facility and DST for financial support through the S N Bose PhD program.


\begin{thebibliography}{100}


        \bibitem{paper 1} Dieter M Herlach et al 2010 \textit{J. Phys.: Condens. Matter} \textbf{22} 153101

        \bibitem{paper 2} Étienne Ducrot, Mingxin He, Gi-Ra Yi \& David J. Pine \textit{Nature Materials} \textbf{16}, 652–657 2017

        \bibitem{paper 3} Friedrich Waag et.al. \textit{RSC Adv.}, 2019, \textbf{9}, 18547

        \bibitem{paper 4} Masako Fujioka-Kobayashi, et. al. 2021, P\textit{Platelets}, \textbf{32:1}, 74-81

        \bibitem{paper 5} Dylan M. Cable et. al. \textit{Nature Biotechnology} \textbf{40}, 517–526 2022

		\bibitem{paper 6} Pusey P N 1991 \textit{Colloidal suspensions in Liquids, Freezing and Glass Transition}, ed J P Hansen et al (Elsevier Science Publishers), pp. 763-942.
		
		\bibitem{paper 7} Berne B J and Pecora R 1976 \textit{Dynamic Light Scattering} (New York, Wiley).
		
		\bibitem{paper 8} Prasad V, Semwogerere D, and Weeks E R 2007 \textit{J. Phys.: Condens. Matter} \textbf{19} 113102.
		
		\bibitem{paper 9} Acuna-Campa H, Carbajal-Tinoco M D, Arauz-Lara J L and Medina-Noyola M 1998 \textit{Phys. Rev. Lett.} \textbf{80} 5802
		
		\bibitem{paper 10} Zahn K, Lenke R and Maret G 1999 \textit{Phys. Rev. Lett.} \textbf{13} 2721.
		
		\bibitem{paper 11} Cui B, Lin B and Rice S A 2001 \textit{J. Chem. Phys.} \textbf{114} 9142.
		
		\bibitem{paper 12} Bayer M, Brader J M, Ebert F, Fuchs M, Lange E, Maret G, Schilling R, Sperl M and Wittmer J P 2007 \textit{Phys. Rev.} E \textbf{76} 011508
		
		\bibitem{paper 13} Jepsen D W, Math 1965 \textit{J. Math. Phys.} \textbf{6} 405
		
		\bibitem{paper 14} Hahn K and K\"arger J 1998 \textit{J. Phys. Chem.} B \textbf{102} 5766
		
		\bibitem{paper 15} Kollmann M 2003 \textit{Phys. Rev. Lett.} \textbf{90} 180602
		
		\bibitem{paper 16} Delfau B, Coste C and Jean M S 2011 \textit{Phys.Rev.} E \textbf{84} 011101.
		
		\bibitem{paper 17} Einstein A 1956 \textit{Investigations on the Theory of The Brownian Movement} ed R F\"urth (USA, Dover Publications, INC.)
		
		\bibitem{paper 18} Frey E and Kroy K 2005 \textit{Ann. Phys.} \textbf{14} 20
		
		\bibitem{paper 19} Coupier G, Jean M S and Guthmann C 2006 \textit{Phys. Rev.} E \textbf{73} 031112.
		
		\bibitem{paper 20} Chowdhury A, Ackerson B J and Clark N A 1985 \textit{Phys. Rev. Lett.} \textbf{55} 833	
		
		\bibitem{paper 21} Loudiyi K, Ackerson B J 1992 \textit{Physica} A \textbf{184} 1
		
		\bibitem{paper 22} Chakrabarti J, Krishnamurthy H R, Sood A K and Sengupta S 1995 \textit{Phys. Rev. Lett.} \textbf{75} 2232
		
		\bibitem{paper 23} Bechinger C, Brunner M and Leiderer P 2001 \textit{Phys. Rev. Lett.} \textbf{86} 930
		
		\bibitem{paper 24} Frey Erwin, Nelson David R, and Radzihovsky Leo 1999 \textit{Phys. Rev. Lett.} \textbf{83} 2977. 
		
		\bibitem{paper 25} Schmiedeberg Michael, Roth Johannes and Stark Holger 2006 \textit{Phys. Rev. Lett.} \textbf{97} 158304
		
		\bibitem{paper 26} Radzihovsky Leo, Frey Erwin and Nelson David R 2001 \textit{Phys. Rev.} E \textbf{63} 031503

		
		\bibitem{paper 27} Deutschl\"ander S, Horn T, L\"owen H, Maret G and Keim P 2013 \textit{Phys. Rev. Lett.} \textbf{111} 098301
		
		\bibitem{paper 28} Chaudhuri Chhanda Basu, Chakrabarty Srabani, and Chakrabarti J 2013 \textit{J. Chem. Phys.} \textbf{139} 204903
		
		\bibitem{paper 29} Jenkins M C and Egelhaaf S U 2008 \textit{J. Phys.: Condens. Matter} \textbf{20} 404220
		
		\bibitem{paper 30} Bechinger Clemens and Frey Erwin 2001 \textit{J. Phys.: Condens. Matter} \textbf{13} R321	
		
		\bibitem{paper 31} L\"owen H 2001 \textit{J. Phys.: Condens. Matter} \textbf{13} R415
		
		\bibitem{paper 32} L\"owen H 2013 \textit{Eur. Phys. J. ST} \textbf{222} 2727
			
		\bibitem{paper 33} Kaplan P D, Yodh A G, and Pine D J 1992 \textit{Phys. Rev. Lett.} \textbf{68} 393
		
		\bibitem{paper 34} Koki Yoshizaw et.at. Soft Matter, 2012, 8, 11732-11736
  
		\bibitem{paper 35} Job H. J. Thijssen and Paul S. Clegg Soft Matter, 2010, 6, 1182-1190
  
		\bibitem{paper 36} D. Teuzollilo, J. Chem. Phys. 156, 034904 (2022)

        \bibitem{paper 37} Capellmann R F, Khisameeva A, Platten F and Egelhaaf S U 2018 \textit{J. Chem. Phys.} \textbf{148} 114903

        \bibitem{paper 38} Jean-Pierre Hansen and Ian R. McDonald, Theory of Simple Liquids,

        \bibitem{paper 39} K Binder \textit{Rep. Prog. Phys.} \textbf{50} 783 1987

        \bibitem{paper 40} Binder, K. (2015). Demixing. In: Li, D. (eds) Encyclopedia of Microfluidics and Nanofluidics. Springer, New York, NY.

        \bibitem{paper 41} Hideki Kobayashi, Phy. Rev. E 104, 044603 (2021)

        \bibitem{paper 42}  O. Zvyagolskaya et al 2011 EPL 96 28005 (2011)

        \bibitem{paper 43} Koki Yoshizawa, Soft Matter, 2012, 8, 11732-11736

        \bibitem{paper 44} P. C. HEMMER and T. H. MARTHINSEN Mol. Phy. 100 667-671, 2002 

%        \bibitem{paper 40} Sood A K 1991 Solid State Physics vol 45, ed H Ehrenreich and D Turnbull (New York: Academic) 1

%        \bibitem{paper 41} Lowen H 1994 Phys. Rep. 237 249–324

%        \bibitem{paper 42} Robbins M O, Kremer K and Grest G S 1988 J. Chem. Phys. 88 3286

%        \bibitem{paper 43} Royall C P, Leunissen E M and van Blaaderen A 2003 J. Phys.: Condens. Matter 15 53381


%        \bibitem{paper 44} Likos C N et. al. 1998 Phys. Rev. Lett. 80 4450

%        \bibitem{paper 45} Nina J Lorenz et al 2009 J. Phys.: Condens. Matter 21 464116

%        \bibitem{paper 46} Hideki Kobayashi, Phy. Rev. E 104, 044603 (2021)

%        \bibitem{paper 47} O. Zvyagolskaya et al 2011 EPL 96 28005 (2011)

%        \bibitem{paper 48} Alberto Parolaa, Molecular Physics, 113, Nos. 17–18, 2571–2582, 2015

%        \bibitem{paper 49} Koki Yoshizawa, Soft Matter, 2012, 8, 11732-11736

%        \bibitem{paper 50} P. C. HEMMER and T. H. MARTHINSEN Mol. Phy. 100 667-671, 2002 



\end{thebibliography}


\newpage


% Figure environment removed



% Figure environment removed



% Figure environment removed



% Figure environment removed



%% Figure environment removed



\newpage

\section{Supplementary data}

The snapshot showing enhanced mixing of particles under higher modulation is shown in Fig. S1(a). The bigger particles have been strongly aligned along the potential minima while breaking the clusters of the smaller particles into further smaller one; hence an enhanced tendency of mixing is observed. The anisotropic PCF data taking into account for three different interaction is shown in Fig S1(b)=(d). $g_{bb}(x,y)$ (Fig.2(b)) shows strong allignments of paticles through the parallel strips along potential minima as per the periodicity of the external potential. This confirms strong preference of particle positions at the modulation minima. $g_{ss}(x,y)$ in Fig. S1(c) shows distorted patches on overall circular pattern, signifying more deviation from isotropic structure. Due to the mixing phenomena, the smaller particles arrange themselves nearer the bigger species more so now. So the big-small correlation shows $g_{bs}(x,y)$ (Fig. S1(d)) higher peak values, suggesting enhanced mixing tendency between the two species with increment of strength of the external potential.

The alignment tendency is further reflected by the strong peak of in both $P^{b}(\theta)$(Fig. SI 1(e)) and $P^{s}(\theta)$(Fig. SI 1(f)) at $\theta=90^{\circ}$ big and small particles' arrangement with respect to the modulation direction. Two significant peaks dominate in both the cases signifying strong alignment perpendicular to the modulation direction, i.e. along the potential minima.

% Figure environment removed




\end{document}	
% \bibliography{ConfinementThreeSphere} 

\end{document}