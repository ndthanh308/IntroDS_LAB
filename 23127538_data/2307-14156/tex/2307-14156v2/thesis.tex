\documentclass[aps,prd,twocolumn,superscriptaddress,nofootinbib]{revtex4-2}

\usepackage{amsmath,amssymb,amsthm,bm,graphicx,xcolor}
\usepackage[colorlinks=true,linkcolor=blue,citecolor=blue,urlcolor=blue]{hyperref}

\begin{document}

\title{Spontaneous Deformation of an AdS Spherical Black Hole}

\author{Zhuan Ning}
\email{ningzhuan17@mails.ucas.ac.cn}
\affiliation{School of Physical Sciences, University of Chinese Academy of Sciences, Beijing 100049, China}

\author{Qian Chen}
\email{chenqian192@mails.ucas.ac.cn}
\affiliation{School of Physical Sciences, University of Chinese Academy of Sciences, Beijing 100049, China}

\author{Yu Tian}
\email{ytian@ucas.ac.cn}
\thanks{Corresponding author}
\affiliation{School of Physical Sciences, University of Chinese Academy of Sciences, Beijing 100049, China}
\affiliation{Institute of Theoretical Physics, Chinese Academy of Sciences, Beijing 100190, China}

\author{Xiaoning Wu}
\email{wuxn@amss.ac.cn}
\affiliation{Institute of Mathematics, Chinese Academy of Sciences, Beijing 100190, China}

\author{Hongbao Zhang}
\email{hongbaozhang@bnu.edu.cn}
\affiliation{Department of Physics, Beijing Normal University, Beijing 100875, China}

\begin{abstract}
In this study, we investigate the real-time dynamics during the spontaneous deformation of an unstable spherical black hole in asymptotically anti-de Sitter (AdS) spacetime. For the initial value, the static solutions with spherical symmetry are obtained numerically, revealing the presence of a spinodal region in the phase diagram. From the linear stability analysis, we find that only the central part of such a thermodynamically unstable spinodal region leads to the emergence of a type of axial instability. To trigger the dynamical instability, an axial perturbation is imposed on the scalar field. As a result, by the fully nonlinear dynamical simulation, the spherical symmetry of the gravitational system is broken spontaneously, leading to the formation of an axisymmetric black hole.
\end{abstract}

\maketitle

\section{Introduction}
In the framework of general relativity, a four-dimensional, static, vacuum, asymptotically flat black hole can be fully characterized by a unique parameter known as the Arnowitt-Deser-Misner mass \cite{Chrusciel:2012jk}. 
Furthermore, the topological geometry of the event horizon of such a black hole must be a two-dimensional spherical surface, $S^2$ \cite{Hawking:1971vc,Hawking:1973uf,Friedman:1993ty,Chrusciel:1994tr}. However, the extrapolation of the uniqueness theorem becomes invalid if certain standard assumptions are relaxed, such as different spacetime asymptotics, additional matter fields, or higher dimensions, leading to the survival of some solutions with additional conserved quantities and other horizon topologies.

In the background of AdS spacetime, the horizon topology of black holes is not limited to $S^2$. Solutions that approach a local AdS spacetime asymptotically can possess a horizon with planar or hyperbolic topology. These ``black holes" have been extensively studied, both analytically and numerically  \cite{Vanzo:1997gw,Birmingham:1998nr,Aros:2000ij,Chesler:2010bi}. Interestingly, for the planar topology, an inhomogeneous black hole can dynamically arise from an initially homogeneous black hole \cite{Janik:2017ykj,Bea:2020ees,Chen:2022cwi}. These configurations do not possess an explicit inhomogeneous external source and spontaneously break translational symmetry. Such a result reveals the real-time dynamics of the first-order phase transition from a holographic point of view.

The story may be more intriguing for solutions with a topologically spherical horizon that approach a global AdS spacetime asymptotically. Due to the external source without spatial isometries or some specific mechanisms, the static black holes with either only axial symmetry or no continuous spatial symmetry are constructed in certain models \cite{Kichakova:2015nni,Herdeiro:2016plq,Herdeiro:2020saw}. A natural question is whether there exist static black hole configurations with only axial symmetry in the absence of a spatially dependent external source. More generally, inspired by the holographic first-order phase transition, the existence of a dynamical pathway from an unstable spherically symmetric black hole to a stable axisymmetric black hole in the asymptotically AdS spacetime remains unresolved, similar to the case of the higher dimensional charged black holes in the de Sitter spacetime \cite{Konoplya:2008au}. The main results of this work is to reveal the existence of an axial instability in a class of spherically symmetric black holes in a specific gravity model and to investigate the real-time dynamics of such a dynamical instability, in which the gravitational system undergoes a spontaneous deformation process leading to the emergence of a black hole with only axial symmetry.

\section{Gravity model and phase diagram}
We consider the Einstein gravity coupled to a real scalar field with a self-interacting potential in the four-dimensional asymptotically AdS spacetime, described by the Lagrangian density
\begin{equation}\label{eq:Lagrangian density}
    \mathcal{L}=R-\frac{1}{2}\nabla_\mu\phi\nabla^\mu\phi-V(\phi).
\end{equation}
For simplicity, we set the AdS radius $L$ to the unit and focus on a scalar field with the mass squared $m^2=-2$ within the Breitenlohner-Freedman bound \cite{Breitenlohner:1982jf}. In order to obtain unstable spherically symmetric black holes, the scalar potential is specified as
\begin{equation}
    V(\phi)=-6\cosh\left(\frac{\phi}{\sqrt{3}}\right)-\frac{\phi^4}{5},
\end{equation}
though other forms of $V(\phi)$ can also lead to qualitatively similar results. The field equations to be solved can be extracted from the variation of the Lagrangian density (\ref{eq:Lagrangian density}), as follows
\begin{equation}\label{eq:EOM}
    \begin{aligned}
        R_{\mu\nu}-\frac{1}{2}Rg_{\mu\nu}&=\frac{1}{2}\nabla_\mu\phi\nabla_\nu\phi-\left(\frac{1}{4}(\nabla\phi)^2+\frac{1}{2}V(\phi)\right)g_{\mu\nu},\\
        \nabla^\mu\nabla_\mu\phi&=\frac{dV(\phi)}{d\phi}.
    \end{aligned}
\end{equation}

To perform a dynamical simulation of the spontaneous deformation of a spherically symmetric black hole, we adopt the ingoing Bondi-Sachs-like coordinates \cite{Bondi:1962px,Sachs:1962wk} as the metric ansatz\footnote{See also, e.g. \cite{Cao:2013ema,He:2015wfa} and references therein for discussions about their outgoing counterparts.}
\begin{widetext}
    \begin{equation}\label{eq:metric}
        ds^2=\frac{L^2}{z^2}(-[fe^{-\chi}-e^{A}\xi^2]dv^2-2e^{-\chi}dvdz-2\xi e^Advd\theta+e^Ad\theta^2+e^{-A}\sin^2\theta d\varphi^2),
    \end{equation}
\end{widetext}
where $L$ has been fixed to the unit and the compactification coordinate $z=r^{-1}$ is introduced to constrain the computational domain to be finite. 
For simplicity, we preserve the axial symmetry in the $\varphi$ direction such that all metric components and the scalar field are functions of $(v,z,\theta)$.
Near the AdS boundary ($z=0$), the scalar field has the following asymptotic behavior
\begin{equation}
    \phi=\phi_1 z+\phi_2 z^2+O(z^3),
\end{equation}
where the source $\phi_1$ is a boundary freedom, and the response $\phi_2$ can only be determined after solving the full equations of motion (EOM). Different from the case of planar topology where the scalar source is only a scaling freedom, here the different values of source will result in the physical scenarios with substantial distinction. We find that there exists a critical value of source below which all black hole solutions are axially dynamically stable at the linear level. Without loss of generality, we choose a supercritical value $\phi_{1}=2$ in this paper.

For the static solutions with spherical symmetry, we find that the fields $\xi$, $A$ can be turned off and the remaining fields $\chi$, $f$, $\phi$ can be efficiently solved by the Newton-Simpson iteration algorithm with appropriate boundary conditions. For more numerical details, please refer to the Appendix \ref{appendix:static}. Once the spacetime geometry has been determined, one can easily extract the energy density and the temperature of the gravitational system:
\begin{equation}
    \varepsilon=-f_3+\frac{\phi_{1}\phi_{2}}{6},\quad T=\frac{|f'(z_h)|}{4\pi},
\end{equation}
where $f_{3}$ denotes the coefficient of the cubic term in the asymptotic expansion of the field $f$ near the boundary, and $z_{h}$ represents the radial position of the event horizon.

The resulting thermodynamic relation is displayed in Fig. \ref{fig:phase diagram} as a temperature dependence of the energy density. One can observe that the phase diagram is divided into three regions by two turning points, one of which is the spinodal region, lying between the two turning points. Due to the negative specific heat, the equilibrium states located in such a spinodal region are thermodynamically unstable. In the case of planar topology, due to the existence of hydrodynamic modes, such thermodynamic instability results in a class of long-wavelength dynamical instability \cite{Janik:2015iry} similar to the Gregory-Laflamme instability \cite{Gregory:1993vy}. Meanwhile, thermodynamically stable regions are also dynamically stable at the linear level. The consistency between thermodynamic stability and dynamical stability suggests that the Gubser-Mitra conjecture \cite{Gubser:2000ec,Gubser:2000mm} holds for the case of planar topology. In the spherical case, however, even though a thermodynamically unstable region exists, without linear perturbation analysis we do not know whether there are dynamically unstable states.

% Figure environment removed

\section{Linear stability analysis}
To reveal the dynamical instability of the thermal phases at the linear level, we consider the perturbations on the static, spherically symmetric background in the following form
\begin{equation}\label{eq:perturbations}
    \begin{aligned}
        g_{\mu\nu}(v,z,\theta)&=g_{\mu\nu}^{(0)}(z)+\tilde{g}_{\mu\nu}(z)e^{-i\omega v}P_l(\cos\theta),\\
        \phi(v,z,\theta)&=\phi^{(0)}(z)+\tilde{\phi}(z)e^{-i\omega v}P_l(\cos\theta),
    \end{aligned}
\end{equation}
where $P_l(\cos\theta)$ is the $l$-order Legendre polynomial. By variable substitution, the angular dependence in the linear perturbation equations can be separated to two-dimensional Laplacian operators without $\varphi$-dependence: $\Delta_2=\frac{1}{\sin\theta}\frac{\partial}{\partial\theta}\left(\sin\theta\frac{\partial}{\partial\theta}\right)$. With the perturbation form (\ref{eq:perturbations}), the time derivatives and two-dimensional Laplacian operators can be replaced by $-i\omega$ and $-l(l+1)$ respectively. Once given a specified angular quantum number $l$, the quasi-normal frequencies $\omega$ can be obtained by solving a generalized eigenvalue problem. The boundary conditions and more calculation details are shown in the Appendix \ref{appendix:linear}.

The dynamical stability of the background solution depends on the imaginary part of the quasi-normal frequencies. If there exists a quasi-normal mode with a positive imaginary part, the perturbations in Eq. (\ref{eq:perturbations}) will grow exponentially and push the gravitational system out of equilibrium, otherwise, the perturbations will decay over time.

The region of the states with dynamical instability is marked in orange in Fig. \ref{fig:phase diagram}. In sharp contrast to the case of planar topology, we find that some states located in the spinodal region are dynamically stable, even though they suffer from thermodynamic instability, indicating the violation of the Gubser-Mitra conjecture in our case. Remarkably, the mismatch between the dynamically unstable region and the spinordal region depends on the scalar source. Specifically, the larger the source, the better the dynamical and thermodynamic instability coincide.

{For comparison, the green and red dots located in the spinodal region in Fig. \ref{fig:phase diagram} are selected as initial data for nonlinear dynamical evolution. The quasi-normal spectra for these configurations, with angular quantum numbers ranging from $l=0$ to $l=4$, are shown in Fig. \ref{fig:QNM}. From the figure}, one can observe that there are two branches of modes ending at the origin for small angular quantum numbers in both cases, similar to the hydrodynamic modes in the case of planar topology. Such purely imaginary modes become oscillating modes when $l$ exceeds a critical value, which depends on the energy density of the thermal phase.

On the other hand, for the configuration denoted by the green dot, all modes lie on or below the real axis, indicating the dynamical stability at the linear level. The situation is distinct for the solution represented by the red dot, where the thermodynamic instability leads to the emergence of a type of axial instability. As shown in the lower panel of Fig. \ref{fig:QNM}, the imaginary part of the mode with angular quantum number $l=1$ in one of the hydrodynamic-like branches is positive, indicating the dynamical instability under specific axial perturbations. Such a single unstable mode necessarily lacks oscillatory behavior due to the symmetry $\omega\rightarrow -\omega^*$ resulting from the real scalar field configuration.

% Figure environment removed

\section{Nonlinear dynamical simulation}
Based on the results of the linear analysis, we further investigate the nonlinear dynamics of the gravitational system to reveal the final fate of the dynamical instability. We consider the initial states represented by the green and red dots in Fig. \ref{fig:phase diagram} and introduce a $\theta$-dependent perturbation to the scalar field, the form of which is chosen as a mixture of all modes without loss of generality:
\begin{equation}
    \delta\phi(z,\theta)=\phi_0 z^2 \exp\left(-10\sin^2\frac{\theta}{2}\right),
\end{equation}
where $\phi_0$ indicates the amplitude of the perturbation. For the state represented by the green dot, the amplitude of the perturbation ranges from $\phi_0=0.1$ to $\phi_0=1$, while for the red dot, the range is $\phi_0=0.001$ to $\phi_0=0.01$.

With the metric ansatz (\ref{eq:metric}), the gravitational dynamics is transformed into a time-dependent solution to a set of coupled partial differential equations. Due to the special nested structure, these equations can be solved sequentially with appropriate boundary conditions at the AdS boundary. To improve the numerical stability, we extend the coordinate range of $\theta$ from $[0,\pi]$ to $[-\pi,\pi]$ such that the periodic boundary condition can be employed. On the other hand, we utilize an even number of Fourier grid points, denoted as $M$, ranging from $-\pi+\pi/M$ to $\pi-\pi/M$ to avoid the coordinate singularity at the north and south poles. Additional numerical details, including the procedure for solving these equations and the boundary conditions are described in the Appendix \ref{appendix:general} and \ref{appendix:nonlinear}.

The fully nonlinear dynamical simulations demonstrate that the gravitaional system in the state denoted by the green dot resists such an axial perturbation, indicating the dynamical stability, consistent with the linear analysis. As shown in Fig. \ref{fig:damping} representing the temporal and spatial dependence of the apparent horizon configuration, the axial perturbation damps over time,
leaving a spherically symmetric black hole.

% Figure environment removed

In stark contrast to the former, for the initial state denoted by the red dot, the unstable mode with angular quantum number $l=1$ is excited under the axial perturbation, leading to drastic changes in the gravitational configuration. The angular dependence of the apparent horizon configurations at different times during the dynamical transition is depicted in Fig. \ref{fig:horizons}, where the color spectrum indicates the radial position $z_h$ of the apparent horizon. One can observe that the apparent horizon radius $r_h=z_h^{-1}$ decreases at the north pole region and increases at the south pole region, eventually leading to the formation of a black hole with only axial symmetry.
The final horizon configuration is shown in Fig. \ref{fig:final horizon}, clearly demonstrating a strong $\theta$-dependence in the final state. Although the apparent horizon exhibits distinct dynamical behavior in different angular regions during the dynamical intermediate process, its total area always increases monotonically with time, consistent with the second law of black hole mechanics. Since the scalar source remains isotropic throughout the time evolution, it can be concluded that the spherical symmetry of the gravitational system is broken spontaneously, resulting in a dynamical deformation process.

% Figure environment removed
% Figure environment removed

\section{Summary}
In this paper we reveal the real-time dynamics of a spontaneous dynamical transition from a spherically symmetric black hole to an axisymmetric black hole under an isotropic external scalar source. Such an initial black hole with spherical symmetry lies deep in a region of thermodynamic instability in the phase diagram. Different from the case of holographic first-order phase transition, where the entire spinodal region is linearly dynamically unstable, in our case only the middle part of the spinodal region satisfies the Gubser-Mitra conjecture, indicating a kind of axial dynamical instability. We find that the energy range of this dynamically unstable region depends on the scalar source. In the intermediate process of the dynamical evolution of this instability, there is a drastic change in the apparent horizon $r_h$, manifested by an inward contraction of the north pole region and an outward expansion of the south pole region. Eventually a black hole with an event horizon with angular dependence forms as the final fate of this dynamical instability.

For future research, a natural direction would involve exploring the real-time dynamics of the spontaneous deformation of rotating black holes. Another question deserving further attention is the dynamical behaviors of the modes during the transition, especially for larger values of scalar source, which can provide more unstable modes. For astronomical observations, we expect similar transition process to also occur in an asymptotically flat spacetime, where the relevant dynamical behaviors can be characterized by gravitational waves.

\begin{acknowledgments}
    We would like to thank Zhoujian Cao, Xiao-Kai He, Bin Wang and Cheng-Yong Zhang for their helpful discussions. YT would like to thank Yong-Ming Huang for helping obtain the EOM in the general Bondi-Sachs-like coordinates. This work is partly supported by the National Key Research and Development Program of China (Grant No.2021YFC2203001). This work is supported in part by National Natural Science Foundation of China under Grants No. 11975235, No. 12035016, No. 12075026, and No. 12275350.
\end{acknowledgments}

\appendix

\section{General formalism of numerical relativity under the Bondi-Sachs-like metric}\label{appendix:general}
In this Appendix, we denote time derivatives with respect to $v$ using dots or subscripts $v$, radial derivatives with respect to $z$ using primes, and angular derivatives with respect to $\theta$ using subscripts $\theta$.

The most general ingoing form of metric under the Bondi-Sachs-like gauge in $1+d$ dimensions can be expressed as
\begin{widetext}
\begin{equation}\label{eq:general_metric}
    ds^{2}=\frac{L^{2}}{z^{2}}(-fe^{-\chi}dv^{2}-2e^{-\chi}dvdz+h_{ij}[dx^{i}-\xi^{i}dv][dx^{j}-\xi^{j}dv]),
\end{equation}
\end{widetext}
where $f$, $\chi$, $h_{ij}$, and $\xi^i$ are functions of $(v,z,x^i)$ and $i$, $j$ range from $1$ to $d-1$. In the asymptotically AdS or dS cases, the constant $L$ can be conveniently set to the AdS or dS radius, while for the asymptotically flat case, $L$ can be arbitrarily chosen. Additionally, the determinant of $h_{ij}$ is constrained everywhere to be that of the standard sphere, plane, hyperbolic space, or even more complicated topologies in higher dimensions, and so $z$ is proportional to the inverse of the local areal radius.

Some general discussion of this metric will be provided in detail elsewhere. Here we only list the corresponding Einstein equations for the planar case (with $\det(h_{ij})=1$):
\begin{widetext}
\begin{equation}\label{eq:general_chi_constraint}
    S_1(\Psi)=\frac{d-1}{z}\chi^{\prime}-\frac{1}{4}h^{ik}h^{jl}h_{li}^{\prime}h_{kj}^{\prime},
\end{equation}
\begin{equation}\label{eq:general_xi_constraint}
    S_2(\Psi)=-\frac{d-1}{z}\chi_{,i}-\Theta_{i,j}^{j\prime}-\frac{1}{2}h_{kj}^{\prime}h_{,i}^{kj}-z^{d-1}\left(\frac{\Theta_{ij}\xi^{j\prime}}{z^{d-1}}\right)^{\prime},
\end{equation}
\begin{equation}\label{eq:general_f_constraint}
    \begin{aligned}
        S_3(\Psi) &= -z^{d-1}\left(\frac{2f}{z^{d}}\right)^{\prime}+\xi_{,i}^{i\prime}-z^{d-1}\left(\frac{2\xi_{,i}^{i}}{z^{d-1}}\right)^{\prime}+\frac{1}{4}h^{kn}\left(2h_{nj,i}-h_{ij,n}\right)h_{,k}^{ij}e^{-\chi}\\
        &\quad -\left(h_{,i}^{ij}\chi_{,j}+h^{ij}\chi_{,ij}-\frac{1}{2}h^{ij}\chi_{,j}\chi_{,i}\right)e^{-\chi}+h^{ij}_{,ij}e^{-\chi}+\frac{1}{2}\xi^{i\prime}\Theta_{ij}\xi^{j\prime},
    \end{aligned}
\end{equation}
\begin{equation}\label{eq:general_h_evolution}
    \begin{aligned}
        S_4(\Psi)	&\simeq	\dot{h}_{j(n}h_{m)}^{j\prime}+\frac{1}{2}\dot{h}_{mn}^{\prime}+z^{d-1}\left(\frac{\dot{h}_{mn}}{2z^{d-1}}\right)^{\prime}-\frac{1}{2}f h_{in}^{\prime}h_{m}^{i\prime}-z^{d-1}\left(\frac{fh_{mn}^{\prime}}{2z^{d-1}}\right)^{\prime}+h_{i(n}^{\prime}\partial_{m)}\xi^{i}+z^{d-1}\left(\frac{h_{i(m}\partial_{n)}\xi^{i}}{z^{d-1}}\right)^{\prime}\\
        &\quad	+\frac{1}{2}(h_{mn}^{\prime}\xi^{k})_{,k}+z^{d-1}\left(\frac{\xi^{k}h_{mn,k}}{2z^{d-1}}\right)^{\prime}-\frac{1}{4}(2h_{(nj,i}-h_{ij,(n})h_{,m)}^{ij}e^{-\chi}+\left[ h^{kl}\left(\partial_{(m}h_{ln)}-\frac{1}{2}\partial_{l}h_{mn}\right)e^{-\chi}\right]_{,k}\\
        &\quad	+\frac{1}{2}\chi_{,m}\chi_{,n} e^{-\chi}+(\chi_{,m} e^{-\chi})_{,n}-\frac{1}{2}h_{mj}h_{ni}\xi^{j\prime}\xi^{i\prime}e^{\chi},
    \end{aligned}
\end{equation}
\begin{equation}\label{eq:general_xi_evolution}
    \begin{aligned}
        S_5(\Psi) &= -\frac{1}{2}(\Theta_{kj}\xi^{j\prime})_{,v}+\frac{1}{2}(d_{n}\Theta_{k}^{j}+h^{ij}h_{lk}\xi_{,i}^{l}-\xi_{,k}^{j}-\xi^{j}\Theta_{ki}\xi^{i\prime})_{,j}\\
        &\quad +\frac{f_{,k}^{\prime}}{2}-\frac{d-1}{2z}(f_{,k}+f\chi_{,k})+\frac{1}{4}h_{,k}^{ij}d_{n}h_{ij}-\frac{1}{2}\Theta_{l,k}^{i}\xi_{,i}^{l}-\frac{1}{2}\xi_{,k}^{i}\Theta_{ij}\xi^{j\prime},
    \end{aligned}
\end{equation}
\begin{equation}\label{eq:general_f_evolution}
    S_6(\Psi)=\frac{1}{2}(-[f\xi^{i}]^{\prime}+\xi^{i}\xi^{j\prime}\Theta_{jk}\xi^{k}-\xi^{k}d_{n}\Theta_{k}^{i}+\Theta^{ij}f_{,j}-2d_{n}\xi^{i})_{,i}+\frac{1}{4}\dot{h}^{ij}d_{n}h_{ij}-\frac{1}{2}\dot{\Theta}_{k}^{j}\xi_{,j}^{k}+\frac{1}{2}\xi^{i}(\Theta_{ij}\xi^{j\prime})_{,v}-\frac{d-1}{2z}(\dot{f}+f\dot{\chi}),
\end{equation}
\end{widetext}
where $h^{ij}$ is the matrix inverse of $h_{ij}$, $\Theta_{ij}:=e^\chi h_{ij}$, $h_j^{i\prime}:=h^{jk}h_{ki}^\prime$, $\partial_\mu\Theta_j^{i}:=\Theta^{jk}\partial_\mu\Theta_{ki}$, $d_{n}:=\partial_{v}-f\partial_{z}+\xi^{i}\partial_{i}$, $\Psi$ denotes the matter content of the theory, $S_k(\Psi)$ $(k=1,2,\dots,6)$ represent the terms depending on the matter fields, and $\simeq$ means equality up to the trace part of a tensor equation. For other topologies, the corresponding equations can be obtained by appropriate coordinate transformations. In particular, for the spherical case in four dimensions we consider in this paper, the EOM can be achieved by the transformation $x=-\cos\theta$ in one of the two transverse directions. These equations have a very nice nested structure.\footnote{The matter part may ruin the nested structure, but fortunately the scalar field we consider in this paper does not.} Due to the fact that they are not fully independent, two possible numerical schemes can be adopted in the dynamical evolution.

We take the asymptotically AdS case as an example. Initially, the data for the metric component $h_{ij}$, the material fields $\Psi$, and some integration constants on the time slice $v_0$ should be given. These integration constants are the coefficients of the cubic term in the near-boundary expansions of the fields $\xi^i$ and $f$, denoted as $\xi_3^i$ and $f_3$ respectively. This data enables us to treat Eq. (\ref{eq:general_chi_constraint}) as a first-order linear ordinary differential equation for the field $\chi$. By choosing the boundary condition $\chi|_{z=0}=0$, we can easily solve for $\chi$. Once the field values of $h_{ij}$, $\Psi$, and $\chi$ are determined, the quantities ${\xi^i}'$ and $\xi^i$ can be consecutively obtained by solving Eq. (\ref{eq:general_xi_constraint}), and the corresponding boundary conditions are ${\xi^i}'''|_{z=0}=6\xi_3^i$ and the gauge choice $\xi^i|_{z=0}=0$, respectively. Next, we solve Eq. (\ref{eq:general_f_constraint}) to obtain the field $f$, with the boundary condition $f'''|_{z=0}=6f_3$. Subsequently, the field $\dot{h}_{ij}$ are determined by solving Eq. (\ref{eq:general_h_evolution}) with the boundary condition $\dot{h}_{ij}'|_{z=0}=0$. Then, the matter field equations should be solved with appropriate boundary conditions. Using these field values, we integrate $\dot{h}_{ij}$ and $\dot{\Psi}$ over time to push $h_{ij}$ and $\Psi$ to the next time slice $v_0+dv$. Subsequently, the field value of $\chi$ on the time slice $v_0+dv$ can be obtained in the same way described above. And then there are two options about the fields $\xi^i$ and $f$, leading to two different evolution schemes:

\begin{itemize}
    \item the constrained scheme:

    The time derivatives of $\xi_3^i$ and $f_3$ can be determined through the near-boundary expansion of Eqs. (\ref{eq:general_xi_evolution}) and (\ref{eq:general_f_evolution}). Then, these integration constants can be updated to the time slice $v_0+dv$ by integrating over time. Consequently, we can solve for $\xi^i$ and $f$ on the time slice $v_0+dv$ in the same manner described above. Finally, all fields on the time slice $v_0+dv$ are determined and the evolution of a time step $dv$ is completed. This iterative procedure continues until the simulation is complete. Two redundant equations (\ref{eq:general_xi_evolution}) and (\ref{eq:general_f_evolution}) are included in the process to identify numerical errors.

    \item the free evolution scheme:
    
    The time derivatives of $\xi^i$ and $f$ can be obtained from Eqs. (\ref{eq:general_xi_evolution}) and (\ref{eq:general_f_evolution}) ($\dot{\chi}$ should be determined from the time derivative of Eq. (\ref{eq:general_chi_constraint}) in advance). Then, we can obtain $\xi^i$ and $f$ on the time slice $v_0+dv$ by integrating $\dot{\xi}^i$ and $\dot{f}$ over time. Finally, all fields on the time slice $v_0+dv$ are determined and the evolution of a time step $dv$ is completed. In the subsequent evolution, we no longer need to solve Eqs. (\ref{eq:general_xi_constraint}) and (\ref{eq:general_f_constraint}). This iterative procedure continues until the simulation is complete. Two redundant equations (\ref{eq:general_xi_constraint}) and (\ref{eq:general_f_constraint}) are included in the process to identify numerical errors.
\end{itemize}
The constrained scheme with the constraints imposed at some other boundaries instead of the AdS conformal boundary, as well as the free evolution scheme, does not sensitively depend on the asymptotics of the spacetime and so can be also applied to asymptotically flat and dS cases (though certain technical subtleties may still arise).

\section{Numerical procedure for dynamical evolution}\label{appendix:nonlinear}
In this paper, we focus on the axisymmetric and non-rotating case, and the ingoing Bondi-Sachs-like metric (\ref{eq:general_metric}) in four-dimensional asymptotically AdS spacetime\footnote{See also, e.g. \cite{Winicour:2005eoq} for a review of dynamical evolution under the Bondi-Sachs-like gauge in four-dimensional asymptotically flat spacetime.} degenerates to
\begin{widetext}
\begin{equation}
    ds^2=\frac{L^2}{z^2}(-[fe^{-\chi}-e^{A}\xi^2]dv^2-2e^{-\chi}dvdz-2\xi e^Advd\theta+e^Ad\theta^2+e^{-A}\sin^2\theta d\varphi^2),
\end{equation}
\end{widetext}
where $L$ is set to the unit, and all metric components and the scalar field depend on $v$, $z$, and $\theta$. To aid our calculations, we introduce the following auxiliary variables
\begin{widetext}
\begin{equation}
    \begin{aligned}
        P&=\frac{e^{A+\chi}}{4}\xi^{\prime2}+\frac{\xi_{\theta}}{z}-\frac{\xi_{\theta}^{\prime}}{2}-\frac{e^{-A-\chi}}{4}(2A_\theta\chi_\theta+\chi_{\theta}^2-\phi_\theta^2)+\frac{(e^{-A-\chi})_{\theta\theta}}{2},\\
        Q&=\frac{\xi'}{2}-\frac{\xi}{z}+\frac{e^{-A-\chi}\chi_\theta}{2}.
    \end{aligned}
\end{equation}
\end{widetext}
As a result, the EOM are simplified to
\begin{widetext}
\begin{align}
    \label{eq:chi_constraint} \chi'&=\frac{z}{4}(A'^2+\phi'^2),\\
    \label{eq:xi_constraint} \left(\frac{e^{A+\chi}}{z^2}\xi'\right)'&=-\frac{1}{z^{2}}\left[(A+\chi)^{\prime}_{\theta}+\frac{2\chi_{\theta}}{z}-(A^{\prime}A_{\theta}+\phi'\phi_{\theta})+2\cot\theta A'\right],\\
    \label{eq:f_constraint} \left(\frac{f}{z^{3}}\right)^{\prime}&=\frac{\xi_{\theta}}{z^{3}}+\frac{L^{2}}{2z^{4}}e^{-\chi}V(\phi)+\frac{1}{z^{2}}\left[P-\cot\theta\left(Q-\frac{\xi}{z}+\frac{3e^{-A-\chi}A_\theta}{2}\right)-e^{-A-\chi}\right],\\
    \label{eq:A_evolution} \dot{A}^{\prime}-\frac{\dot{A}}{z}&=\frac{z^{2}}{2}\left(\frac{fA^{\prime}-\xi A_{\theta}}{z^{2}}\right)^{\prime}+\frac{(e^{-A-\chi}A_{\theta}-\xi A^{\prime})_{\theta}}{2}+P+\cot\theta\left(Q-\frac{\xi A'}{2}\right),\\
    \label{eq:phi_evolution} \dot{\phi}^{\prime}-\frac{\dot{\phi}}{z}&=\frac{z^{2}}{2}\left(\frac{f\phi'-\xi\phi_{\theta}}{z^{2}}\right)^{\prime}+\frac{(e^{-A-\chi}\phi_{\theta}-\xi\phi')_{\theta}}{2}-\frac{L^{2}}{2z^{2}}e^{-\chi}\frac{dV(\phi)}{d\phi}+\cot\theta\left(\frac{e^{-A-\chi}\phi_\theta}{2}-\frac{\xi \phi'}{2}\right),\\
    (e^{A+\chi}\xi^{\prime})_{v}&=-\frac{2}{z}(f_{\theta}+f\chi_{\theta})-(2\xi e^{A+\chi}\xi^{\prime}-f^{\prime}+f[A^{\prime}+\chi^{\prime}]-[\dot{A}+\dot{\chi}])_{\theta}+\xi(e^{A+\chi}\xi^{\prime}+A_\theta+\chi_\theta)_{\theta}-A_{\theta}d_{n}A-\phi_{\theta}d_n\phi \nonumber \\
    &\quad-\cot\theta[\xi(e^{A+\chi}\xi'-3A_\theta+\chi_\theta)+2f A'-2\dot{A}]+2\xi, \label{eq:xi_evolution} \\
    \frac{2}{z}(\dot{f}+f\dot{\chi})&=(e^{-A-\chi}f_{\theta}-[f\xi]^{\prime}+\xi^{2}e^{A+\chi}\xi^{\prime}-\xi[d_{n}A+d_{n}\chi]-2d_{n}\xi)_{\theta}-\xi_\theta(\dot{A}+\dot{\chi})-\dot{A}d_{n}A-\dot{\phi}d_n\phi+\xi(e^{A+\chi}\xi^{\prime})_{v} \nonumber \\
    &\quad+\cot\theta(\xi^2[e^{A+\chi}\xi'-(A_\theta+\chi_\theta)]+\xi[f(A'+\chi')-2\dot{\chi}-2\xi_\theta-f']+f\xi'-2\dot{\xi}+e^{-A-\chi}f_\theta), \label{eq:f_evolution}
\end{align}
\end{widetext}
where $d_{n}=\partial_{v}-f\partial_{z}+\xi\partial_{\theta}$. There is a systematic and efficient integration strategy to solve these equations, which benefits from their nested structure. We use the constrained scheme described in Appendix \ref{appendix:general} in this paper. More specifically, the boundary conditions arise from the near-boundary asymptotic behaviors of the field solutions:
\begin{align}
    \label{eq:chi_asym} \chi&=\frac{\phi_1^2}{8}z^2+O(z^3),\\
    \label{eq:xi_asym} \xi&=\xi_3 z^3+O(z^4),\\
    \label{eq:f_asym} f&=1+\left(\frac{\phi_1^2}{8}+1\right)z^2+f_3 z^3+O(z^4),\\
    \label{eq:A_asym} A&=O(z^3),\\
    \label{eq:phi_asym} \phi&=\phi_1 z+\phi_2 z^2+O(z^3).
\end{align}
For Eq. (\ref{eq:xi_constraint}), the asymptotic behavior of the field $\xi$ (\ref{eq:xi_asym}) yields two boundary conditions with the gauge choice $A|_{z=0}=0$:
\begin{equation}
    \left.\left(\frac{e^{A+\chi}}{z^2}\xi'\right)\right|_{z=0}=3\xi_3,\quad\xi|_{z=0}=0.
\end{equation}
The asymptotic behavior of the field $f$ (\ref{eq:f_asym}) determines a single integration constant in Eq. (\ref{eq:f_constraint}). By imposing the field redefinition $f=1+z^2\tilde{f}$, the corresponding boundary condition becomes $\tilde{f}'|_{z=0}=f_3$. Moreover, the time derivatives of the integration constants $\xi_3$ and $f_3$ are determined through the near-boundary expansion of Eqs. (\ref{eq:xi_evolution}) and (\ref{eq:f_evolution}):
\begin{equation}
    \begin{aligned}
        \partial_v\xi_3&=\partial_\theta\left(\frac{ f_3}{3}-A_3-\frac{2\phi_1\phi_2}{9}\right)-2\cot\theta A_3,\\
        \partial_v f_3&=\frac{3\partial_\theta\xi_3}{2}+\frac{\phi_1\partial_v\phi_2}{6}+\cot\theta\frac{3\xi_3}{2}.
    \end{aligned}
\end{equation}
Hence, the values of $\partial_v\xi_3$ and $\partial_v f_3$ can be calculated after obtaining $\dot{\phi}$.

In the $\theta$ direction, we extend the coordinate range from $[0,\pi]$ to $[-\pi,\pi]$, and employ the periodic boundary condition along with the Fourier pseudospectral discretization. With this extension, the functions $\chi$, $f$, $A$, and $\phi$ exhibit even symmetry with respect to $\theta$, while $\xi$ displays odd symmetry. To avoid the coordinate singularity at the north and south poles, we utilize an even number of equally spaced points, denoted as $M$, ranging from $-\pi+\pi/M$ to $\pi-\pi/M$. The $z$ direction is discretized using the Chebyshev pseudospectral method as standard.

For axisymmetric configurations, determining the location of the apparent horizon $z=h(v,\theta)$ requires the following condition
\begin{widetext}
\begin{equation}
    (\partial_\theta+h_\theta\partial_{z})(\xi+h_\theta e^{-A-\chi})+\cot\theta(\xi+h_\theta e^{-A-\chi})=-\frac{1}{h}(f-h_\theta^{2}e^{-A-\chi}),
\end{equation}
\end{widetext}
where all fields (except $h$ and its $\theta$-derivative) should be understood as functions of $v$, $z=h(v,\theta)$, and $\theta$.

\section{Numerical procedure for static solutions}\label{appendix:static}
As we focus solely on static solutions with spherical symmetry, we can set $\xi$ and $A$ to zero, and consider $\chi$, $f$, and $\phi$ as functions of $z$ only. Consequently, the EOM (\ref{eq:chi_constraint})-(\ref{eq:f_evolution}) degenerate to
\begin{equation}
    \begin{aligned}
        \chi'&=\frac{z}{4}\phi'^2,\\
        \left(\frac{f}{z^{3}}\right)^{\prime}&=\frac{L^{2}}{2z^{4}}e^{-\chi}V(\phi)-\frac{1}{z^{2}}e^{-A-\chi},\\
        \frac{z^2}{2}\left(\frac{f\phi'}{z^2}\right)'&=\frac{L^2}{2z^2}e^{-\chi}\frac{dV(\phi)}{d\phi}.
    \end{aligned}
\end{equation}
In the computational domain $[0,z_0]$ of the $z$ coordinate, a solution of these equations can be determined by specifying the boundary conditions
\begin{equation}
    \begin{aligned}
        \chi|_{z=0}&=0,\\
        f|_{z=z_0}&=f_0,\\
        \phi'|_{z=0}&=\phi_1,
    \end{aligned}
\end{equation}
where $\phi_1$ is the scalar source and $f_0$ is given by hand. Then, we can solve the EOM using the Newton-Simpson iteration algorithm, along with Chebyshev pseudospectral discretization in the $z$ direction. Specifically, the above equation set can be denoted by $\bm{E}[z,\bm{F}]=0$ with $\bm{F}=(\chi,f,\phi)$. The new value $\bm{F}_{i+1}$ is obtained via its value in the previous step $\bm{F}_i$:
\begin{equation}
    \bm{F}_{i+1}=\bm{F}_i-\bm{J}^{-1}(\bm{F}_i)\bm{E}(\bm{F}_i),
\end{equation}
where $\bm{J}=\frac{\delta\bm{E}}{\delta\bm{F}}$ is the functional Jacobian. Once given a initial value of $\bm{F}$, we iterate the procedure until the difference $\bm{F}_N-\bm{F}_{N-1}$ is small enough, and consider $\bm{F}_N$ a static solution. For a static, spherically symmetric balck hole, the event horizon $z_h$ is determined by the condition $f(z_h)=0$.

\section{Numerical procedure for quasi-normal modes}\label{appendix:linear}
We use the evolution formalism to numerically calculate the quasi-normal modes in our system, which has been used extensively in the probe limit (see, e.g. \cite{Du:2015zcb,Guo:2018mip}). Consider perturbations on the static, spherically symmetric background in the following form
\begin{equation}
    \begin{aligned}
        g_{\mu\nu}(v,z,\theta)&=g_{\mu\nu}^{(0)}(z)+\delta g_{\mu\nu}(v,z,\theta),\\
        \phi(v,z,\theta)&=\phi^{(0)}(z)+\delta \phi(v,z,\theta).
    \end{aligned}
\end{equation}
By inserting this form into the EOM (\ref{eq:chi_constraint})-(\ref{eq:f_evolution}) and retaining only the linear terms, we obtain linear perturbation equations for $\delta\chi$, $\delta f$, $\delta\xi$, $\delta A$, and $\delta\phi$. To separate the angular dependence in the perturbation equations, we introduce two new variables
\begin{equation}
    \begin{aligned}
        \delta\Xi &= \frac{1}{\sin\theta}\partial_\theta(\sin\theta \delta\xi)\\
        &= \partial_\theta\delta\xi + \cot\theta\delta\xi,\\
        \delta a &= \frac{1}{\sin\theta}\partial_\theta(\sin\theta (\partial_\theta\delta A+2\cot\theta \delta A))\\
        &= \partial_\theta^2\delta A + 3 \cot\theta\partial_\theta\delta A - 2 \delta A.
    \end{aligned}
\end{equation}
By this substitution, all $\theta$-dependence in the perturbation equations can be transformed to two-dimensional Laplacian operators without $\varphi$-dependence: $\Delta_2=\frac{1}{\sin\theta}\frac{\partial}{\partial\theta}\left(\sin\theta\frac{\partial}{\partial\theta}\right)$, and the perturbation equations are simplified to
\vspace{0.01\baselineskip} % to adjust the output of pdfLaTeX
\begin{widetext}
\begin{align}
    \delta\chi'&=\frac{z}{2}\phi'\delta\phi',\\
    \label{eq:xi_pert} \left(\frac{e^{\chi} \delta\Xi'}{z^{2}}\right)'&= -\frac{1}{z^2}\left[\delta a'+\Delta_2\delta\chi' + \frac{2 \Delta_2\delta\chi}{z} - \phi' \Delta_2\delta\phi\right],\\
    \label{eq:f_pert} \left(\frac{\delta f}{z^{3}}\right)^{\prime}&=\frac{\delta\Xi}{z^{3}}+\frac{L^{2}}{2z^{4}}e^{-\chi}\left(\frac{dV(\phi)}{d\phi}\delta\phi-V(\phi)\delta\chi\right)-\left[\left(\frac{\delta\Xi}{2z^{2}}\right)^{\prime}+\frac{e^{-\chi}\delta a}{2z^{2}}+\frac{e^{-\chi}}{2z^{2}}(\Delta_{2}\delta\chi-2\delta\chi)\right]\\
    \delta\dot{a}'-\frac{\delta\dot{a}}{z}&=\frac{z^{2}}{2}\left(\frac{f\delta a^{\prime}}{z^{2}}\right)^{\prime}-\frac{z^{2}}{2}\left(\frac{1}{z^{2}}[\Delta_{2}\delta\Xi+2\delta\Xi]\right)^{\prime}-\frac{e^{-\chi}}{2}(\Delta_{2}^{2}\delta\chi+2\Delta_{2}\delta\chi),\\
    \delta\dot{\phi}'-\frac{\delta\dot{\phi}}{z}&=\frac{z^{2}}{2}\left(\frac{\phi'\delta f+f\delta\phi'}{z^{2}}\right)^{\prime}-\frac{\phi'\delta\Xi}{2}+\frac{e^{-\chi}\Delta_{2}\delta\phi}{2}-\frac{L^{2}}{2z^{2}}e^{-\chi}\left(\frac{d^2V(\phi)}{d\phi^2}\delta\phi-\frac{dV(\phi)}{d\phi}\delta\chi\right),\\
    e^{\chi}\delta\dot{\Xi}^{\prime}&=-\frac{2}{z}\Delta_{2}(\delta f+f\delta\chi)+\Delta_{2}(\delta f^{\prime}-\chi^{\prime}\delta f)-f(\delta a^{\prime}+\Delta_{2}\delta\chi^{\prime})+\delta\dot{a}+\Delta_{2}\delta\dot{\chi}+f\phi'\Delta_{2}\delta\phi+2\delta\Xi,\\
    \frac{2}{z}(\delta\dot{f}+f\delta\dot{\chi})&=e^{-\chi}\Delta_{2}\delta f-[f\delta\Xi]^{\prime}+f\chi^{\prime}\delta\Xi-2[\partial_{v}-f\partial_{z}]\delta\Xi+f\phi'\delta\dot{\phi}.
\end{align}
\end{widetext}
Subsequently, we decompose $\delta\bm{\Phi}=(\delta\chi,\delta\Xi,\delta f,\delta a,\delta\phi)$ as $\tilde{\bm{\Phi}}(z)e^{-i\omega v}P_l(\cos\theta)$, where $\tilde{\bm{\Phi}}(z)=(\tilde{\chi}(z),\tilde{\Xi}(z),\tilde{f}(z),\tilde{a}(z),\tilde{\phi}(z))$ are the expansion coefficients and $P_l(\cos\theta)$ is the $l$-order Legendre polynomial. In this decomposition, the time derivatives and two-dimensional Laplacian operators can be replaced by $-i\omega$ and $-l(l+1)$ respectively. Meanwhile, the $z$ coordinate is discretized with Chebyshev-Gauss-Lobatto grid points and the radial derivatives are replaced by the corresponding differentiation matrix. Thus, the quasi-normal frequencies $\omega$ with a specified angular quantum number $l$ can be obtained by solving a generalized eigenvalue problem:
\begin{equation}
    (\bm{A}+\omega\bm{B})\tilde{\bm{\Phi}}=0,
\end{equation}
where $\bm{A}$ and $\bm{B}$ depend on the background solution and $l$. Eqs. (\ref{eq:xi_pert}) and (\ref{eq:f_pert}) are not used to solve for the quasi-normal modes, but to detect numerical errors. As standard, we take ingoing boundary conditions at the horizon, which means a regular solution in our coordinates. At the AdS boundary, Dirichlet or Neumann boundary conditions are imposed to be consistent with the background:
\begin{equation}
    \tilde{\chi}|_{z=0}=0,\quad\tilde{\Xi}|_{z=0}=0,\quad\tilde{a}|_{z=0}=0,\quad\tilde{\phi}'|_{z=0}=0.
\end{equation}

\bibliography{ref.bib}

\end{document}
