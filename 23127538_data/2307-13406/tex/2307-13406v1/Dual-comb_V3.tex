\documentclass{article}

\usepackage{siunitx}
\usepackage{amsmath}
\usepackage{graphicx}

\usepackage{arxiv}

\usepackage[style=nature]{biblatex}
\addbibresource{Biblio_article_Dual_Comb.bib}

\begin{document}


\title{Two-micron dual-comb spectroscopy of $\text{C}\text{O}_\text{2}$ and $\text{N}_\text{2}\text{O}$ with a novel design of dispersion-controlled highly nonlinear fiber}



\author{
Alix Malfondet \\
Université Bourgogne\\
Laboratoire Interdisciplinaire Carnot de Bourgogne\\
CNRS, UMR 6303\\
F-21078 Dijon, France
\And
Moise Deroh \\
Université Bourgogne\\
Laboratoire Interdisciplinaire Carnot de Bourgogne\\
CNRS, UMR 6303\\
F-21078 Dijon, France
\And
Sidi-Ely Ahmedou \\
Université de Limoges \\
XLIM\\
CNRS, UMR 7252\\
F-87000 Limoges, France 
   \And
Alexandre Parriaux \\
Université de Neuchâtel\\
Laboratoire Temps-Fréquence, Institut de Physique,\\
Avenue de Bellevaux 51,
13 2000 Neuchâtel, Switzerland 
\And
Kamal Hammani \\
Université Bourgogne\\
Laboratoire Interdisciplinaire Carnot de Bourgogne\\
CNRS, UMR 6303\\
F-21078 Dijon, France
  \And
Romain Dauliat \\
Université de Limoges \\
XLIM\\
CNRS, UMR 7252\\
F-87000 Limoges, France
\And
Laurent Labonté \\
Université Côte d’Azur \\
Institut de Physique de Nice\\
CNRS, UMR 7010\\
F-06108 Nice Cedex 2, France
\And
Sébastien Tanzilli \\
Université Côte d’Azur \\
Institut de Physique de Nice\\
CNRS, UMR 7010\\
F-06108 Nice Cedex 2, France 
\And
Jean-Christophe Delagnes \\
Université de Bordeaux- \\
CELIA, Centre Lasers Intenses et Applications\\
CNRS, UMR 5107\\
F-06108 Nice Cedex 2, FranceF- 33405 Talence Cedex, France
  \And
Philippe Roy \\
Université de Limoges \\
XLIM\\
CNRS, UMR 7252\\
F-87000 Limoges, France
 \And
Raphaël Jamier \\
Université de Limoges \\
XLIM\\
CNRS, UMR 7252\\
F-87000 Limoges, France
\And
Guy Millot \\
Laboratoire Interdisciplinaire Carnot de Bourgogne\\
UMR 6303 CNRS, Université Bourgogne Franche-Comté\\
9 Av. A. Savary, B.P. 47869, 21078 Dijon Cedex, France \\
Institut Universitaire de France (IUF), 1 Rue Descartes, Paris, France
}

\date{year}



\maketitle

\newpage
\begin{abstract} 
In this paper, we introduce an all-fibered dual-comb spectrometer based on a new design of highly nonlinear fiber to efficiently convert frequency combs from \SI{1.55}{\micro\metre} to \SI{2}{\micro\metre}. We show that our spectrometer can be used to measure absorption profiles of rovibrational transitions of  $\text{C}\text{O}_2$ and  $\text{N}_2\text{O}$ molecules, and especially their collisional self-broadening coefficients. The results show very good agreement with the HITRAN database and thus further measurements have been performed on a mixture  $\text{C}\text{O}_2$/$\text{N}_2\text{O}$ to measure the broadening of the $\text{C}\text{O}_2$ absorption lines resulting from the presence of $\text{N}_2\text{O}$.
\end{abstract}


% ---

\section{Introduction}


Optical frequency combs (OFCs), consisting of evenly spaced and coherent frequency lines within a broad spectrum, are very interesting and powerful tools in a wide range of area including time and frequency metrology, microwaves generation and molecular spectroscopy \cite{hansch_nobel_2006, hall_nobel_2006, jones_carrier-envelope_2000, udem_optical_2002, diddams_direct_2000, cundiff_colloquium_2003, kippenberg_dissipative_2018, picque_frequency_2019}. Over the past decade, numerous technologies and experimental configurations utilizing OFCs for sample interrogation have been extensively demonstrated and reported \cite{keilmann_time-domain_2004, schliesser_frequency-comb_2005, gohle_frequency_2007, coluccelli_optical_2016, thorpe_broadband_2006, coddington_coherent_2008, mandon_fourier_2009, bernhardt_cavity-enhanced_2010}. Among such techniques, dual-comb spectroscopy (DCS), which relies on the interference of two mutually coherent frequency combs that have a slightly different repetition rates, has emerged as a powerful tool enabling real-time detection \cite{schiller_spectrometry_2002, Coddington2016}. DCS finds applications in diverse fields  such as breath analysis, agriculture gas flux measurement, and material characterization \cite{Coddington2016, Millot2015, Wang2009, Herman2021, Asahara2016}. It is worthwhile to mention the remarkable technological advancements in optical devices, including electro-optic modulators, fiber couplers, and laser sources operating in the traditional \SI{1.55}{\micro\metre} telecommunication band, which have facilitated the development of dual-comb spectrometers achieving high record performance. However, this spectral range is unsuitable for the detection of several pollutants such as carbon dioxide ($\text{C}\text{O}_2$) or nitrious oxide ($\text{N}_2\text{O}$), for which long absorption length are required. To overcome this problem, it is possible to use the spectrometer in the thulium amplification band, at which the absorption spectral lines are much stronger \cite{gunning_time_2019}.




Nitrous oxide, the third most important greenhouse gas after carbon dioxide  and methane, despite its lower atmospheric presence in terms of mass, possesses remarkable global warming potential, being 25 times more potent than $\text{CH}_4$ and 300 times more potent than $\text{C}\text{O}_2$ \cite{kaur_how_2023, song_co-existence_2023}. It also exhibits the longest atmospheric lifetime and contributes significantly to global warming as well as ozone layer depletion. On a global scale, agriculture represents the largest source of $\text{N}_2\text{O}$ emissions, primarily due to the extensive use of nitrogen fertilizers in intensive farming \cite{bell_quantifying_2016}.



In this study, we experimentally demonstrate an all-fibered dual-comb spectrometer operating within the thulium amplification band. The down-conversion is achieved by leveraging degenerate four-wave mixing \cite{Pitois2003, billat_high-power_2014, parriaux_two-micron_2018} in a novel design of highly nonlinear fiber (HNLF) that allows a precise control on its dispersion parameters \cite{Ahmedou2022}. This innovative fiber design effectively reduces fluctuations in the dispersion coefficients along the fiber, thereby enhancing the efficiency of nonlinear conversion. Our experimental results for the collisional self-broadening coefficients of rotationnal-vibrational (rovibrational) transitions of $\text{C}\text{O}_2$ and $\text{N}_2\text{O}$ molecules exhibit excellent agreement with the HITRAN molecular spectroscopic database \cite{gordon_hitran2020_2022}. Furthermore, experimental measurements have been conducted on a $\text{C}\text{O}_2$/$\text{N}_2\text{O}$ mixture to determine the collisional broadening of several lines of $\text{C}\text{O}_2$ due to the presence of $\text{N}_2\text{O}$.



\section{Four-wave mixing frequency conversion}

In the literature, several techniques exist to convert OFCs to higher wavelengths, such as difference frequency generation in a periodically poled lithium niobate (PPLN) crystals \cite{Jin2015,Ycas2018,Niu2022} or supercontinuum generation \cite{Ruehl2011,Nader2018,Guo2020}. In this work, we propose to convert frequency combs from the standard  telecommunication band ($\sim$ \SI{1.55}{\micro\metre}) to the thulium amplification band ($\sim$  \SI{2}{\micro\metre}) by means of degenerate four-wave mixing (FWM) occurring in a dispersion controlled HNLF \cite{parriaux_two-micron_2018, Ahmedou2022}. 


\subsection{Principle of degenerate four-wave mixing}

Four-wave mixing is a nonlinear process corresponding to an energy exchange between four waves (two pumps $\omega_{p1}$ and $\omega_{p2}$, a Stokes wave $\omega_s$, and an anti-Stokes wave $\omega_a$). The FWM is called degenerate when the two pumps have the same frequency, and non-degenerate otherwise. In this paper we only consider the degenerate FWM, which will be used in our experiments.



The frequency of the Stokes and anti-Stokes waves for which the gain is maximum is given by the following phase-matching condition \cite{Agrawal2019}:

\begin{equation}
\sum_{m=1 }^{\infty} \cfrac{\beta_{2m}\left(\omega_p\right)}{2m !} \Omega_s^{2m}+\gamma P_p=0
\end{equation}


where $\beta_{2m}$ is the dispersion coefficient of order $2m$, $\Omega_s$ the frequency difference between the pump and the generated waves for which the gain is maximum, $\gamma$ the nonlinear coefficient of the fiber, and $P_p$ the pump power. In our case, the zero-dispersion wavelength (ZDW) of the fiber is close to the pump wavelength, so that the dispersion coefficients must be considered up to the fourth order. The phase matching condition can then be written as \cite{Pitois2003}:

\begin{equation}
\cfrac{\beta_4}{12} \Omega_s^4+\beta_2 \Omega_s^2+2 \gamma P_p=0
\end{equation}



To amplify the signal, the phase mismatch $\Delta\beta$ must thus satisfy:


\begin{equation}
-4 \gamma P_p \leq \Delta \beta \leq 0
\end{equation}

where $\Delta\beta$ is given by:

\begin{equation}
\Delta \beta=2 \beta_2\left(\omega_{\mathrm{p}}-\omega\right)^2+\beta_4\left(\omega_{\mathrm{p}}-\omega\right)^4 / 16
\end{equation}

Finally, the optimal frequency spacing, for which the gain is maximal, and its width $\Delta f$, are then given by:

\begin{equation}
\label{eqn:fopt}
f_{o p t}=\frac{1}{2 \pi} \sqrt{\frac{-2}{\beta_4} \sqrt{9 \beta_2^2-6 \beta_4 \gamma P}-6 \frac{\beta_2}{\beta_4}} \approx \frac{1}{2 \pi} \sqrt{\frac{-12 \beta_2}{\beta_4}}
\end{equation}

\begin{equation}
\Delta f \approx \frac{\gamma P_p}{2 \pi \beta_2} \sqrt{\frac{-\beta_4}{3 \beta_2}}
\end{equation}


Equation \ref{eqn:fopt} shows that, to have an efficient conversion at a desired wavelength, the ratio $\beta_2/\beta_4$ has to be controlled precisely. To this end, we now propose a new fiber design to optimize the conversion in the \SI{2}{\micro\metre} range \cite{Ahmedou2022}.

\subsection{New fiber design for frequency conversion}

First, let us take the example of a standard commercial W-type HNLF, whose design is depicted in Fig. \ref{fig:Principle_3T_V2} (a). This commercial fiber consists on a germanium doped silica core, surrounded by a low-index layer made of fluoride doped silica. The latter is finally surrounded by a cladding made of pure silica. The drawback of this design is that a small change of its geometry (for instance a change of the thickness of the low-index layer) strongly impacts the dispersion properties of the fiber. It is thus very difficult to fabricate a HNLF with a precise control of the ZDW and the dispersion coefficients along the fiber, and thus with an efficient conversion at the targeted wavelength \cite{Myslivets2010,Kuo2012}.

To improve the conversion efficiency, we propose to use the design depicted in Fig. \ref{fig:Principle_3T_V2} (b), for which the low-index layer is replaced by three low index inclusions of the same material. This reduces the constraints imposed on the guided mode of the core, thus allowing a better control of the ZDW and the ratio $\beta_2/\beta_4$ during the fabrication process. 


% Figure environment removed


This discretized highly nonlinear fiber (D-HNLF) also allows us to reduce longitudinal and transverse fluctuations of the dispersion parameters, which are detrimental for the conversion efficiency\cite{Karlsson1998,Tsuji2002}.


To demonstrate the interest of this new fiber design, Fig. \ref{fig:Influence_ZDW_et_dispersion} shows the influence of the ZDW (a) and the ratio $\beta_2/\beta_4$ (b) as a function of the thickness of the layer (for the commercial fiber) or the width of the inclusions (for the D-HNLF) \cite{Ahmedou2022}. The simulations depicted in Fig. \ref{fig:Influence_ZDW_et_dispersion} (a) and (b) show that for the commercial fiber (blue line), a small change in the thickness of the low-index layer severely impacts the ZDW and the ratio of the dispersion coefficients. It will therefore be very difficult to optimize such a fiber for a targeted conversion wavelength, because even a small variation of the thickness during fabrication will change the position of the optimal frequency spacing $f_{opt}$.

On the other hand, the curves obtained with the D-HNLF (red line) exhibit much smoother slopes. The ZDW and the ratio $\beta_2/\beta_4$ can thus be controlled more easily with this new fiber design, to improve the conversion efficiency at the desired wavelength (i.e. close to \SI{2}{\micro\metre} in our case).


% Figure environment removed


The improvement due to the D-HNLF is also highlighted in Fig. \ref{fig:Spectre_SUMITOMO_XLIM}, where we compare the spectrum at the output of the D-HNLF (red) and at the output of a commercial HNLF (blue) after injecting a pump and a seed at their input. We can see that the converted signal is almost \SI{10}{\decibel} higher with the D-HNLF, compared to the one obtained with a commercial HNLF. Moreover, the spectrum at the output of the D-HNLF exhibits no Raman peaks, whereas two Raman peaks are visible when the commercial fiber is used, probably due to a higher concentration of germanium in the core. However, the pump depletion caused by these peaks is not itself sufficient to explain the improvement of \SI{10}{\decibel}, which is mainly due to the optimization of the dispersion parameters allowed by the D-HNLF. 


% Figure environment removed


\section{Demonstration of dual-comb spectroscopy}


\subsection{Experimental setup}

Before presenting the measurements carried out on $\text{CO}_2$ and $\text{N}_2\text{O}$ molecules, we give details of the dual-comb spectrometer  shown in Fig. \ref{fig:Schema_FWM_dual_comb_V1}. First, two frequency combs are generated around \SI{1.55}{\micro\metre} with two electro-optic modulators \cite{Millot2015, parriaux_electro-optic_2020}. The repetition rate of these two combs are respectively $f_{r1}=$\SI{60}{\mega\hertz} and $f_{r2}=f_{r1}+\Delta f_{r}=$\SI{60}{\mega\hertz} + \SI{2}{\kilo\hertz}. Each of these two combs are mixed with a \SI{1.3}{\micro\metre} seed in a 1.3/\SI{1.5}{\micro\metre} wavelength-division multiplexer (WDM), and injected in the 1.3/\SI{1.5}{\micro\metre} port of a second WDM. The two combs and their seed counter-propagate into the 5 meter long D-HNLF, where the \SI{2}{\micro\metre} signals are generated. These two signals are then isolated thanks to the \SI{2}{\micro\metre} port of the second WDM, and interfere in a 50/50 coupler. After filtering, the \SI{2}{\micro\metre} interference signal is split into a reference signal (Ref.) that directly goes to the measurement apparatus, and an absorbed signal (Abs.) which goes through a gas cell before being acquired. The Fourier transform of the interference signal gives us the radio frequency (RF) spectrum used to retrieve the absorption profile of the gas under study, whereas the optical spectrum analyzer is used for  wavelength monitoring.


% Figure environment removed

\subsection{Spectroscopy of pure $\text{C}\text{O}_2$ }

We first study the absorption of five spectral lines of pure $\text{C}\text{O}_2$. To this end, the seed is tuned between \SI{1288}{\nano\metre} and \SI{1294}{\nano\metre} in order to generate a signal between \SI{1961}{\nano\metre} and \SI{1974}{\nano\metre}. For each spectral line, we measure for different gas pressure the absorption and the reference RF spectrum, as shown in Fig. \ref{fig:Spectre_RF_1}. The transmission profile (in \%) is obtained by the ratio of the absorption spectrum on the reference spectrum.




% Figure environment removed

Each transmission profile is compared with a fitted spectrum calculated by the least square regression of a Voigt profile with the parameter lines (frequency, intensity) extracted from the HITRAN database. During the regression, the width of the studied absorption line is adjusted by taking into account the neighboring lines, whose widths are fixed and given by the HITRAN database. From this, we extract the collisional half-width at half-maximum (HWHM) $\Delta\nu$ of the spectral lines versus the pressure. For clarity, only two of the five rovibrational spectral lines studied with this method are represented in Fig. \ref{fig:gamma_CO2_pur}. The inset in Fig. \ref{fig:gamma_CO2_pur} shows a transmission profile obtained at a pressure of \SI{700}{\milli\bar} (black dots), and compared to the fitted profile (green line). This illustrates the good agreement between our experimental results and the HITRAN database.


% Figure environment removed

It is also possible to obtain the self-broadening coefficient  $\gamma_{\text{C}\text{O}_2}$ of the rovibrational spectral lines, corresponding to the slope of the curves depicted in Fig. \ref{fig:gamma_CO2_pur}. For the P(8) and the P(38) lines of the $2\nu_1+\nu_3$ band, we obtain collisional self-broadening coefficients respectively equal to ($0.113\pm0.001$) \SI{}{\centi\meter^{-1}}$\text{atm}^{-1}$ and ($0.086\pm0.001$) \SI{}{\centi\meter^{-1}}$\text{atm}^{-1}$, in good agreement with the values obtained with HITRAN, respectively equal to ($0.111\pm0.005$) \SI{}{\centi\meter^{-1}}$\text{atm}^{-1}$ and ($0.082\pm0.004$) \SI{}{\centi\meter^{-1}}$\text{atm}^{-1}$. Results obtained with the other rovibrational spectral lines are summarized in the table at the last section of this paper.



\subsection{Spectroscopy of pure $\text{N}_2\text{O}$ }
Using the same procedure than for pure $\text{C}\text{O}_2$, we also present results obtained with pure  $\text{N}_2\text{O}$. The seed is now tuned between \SI{1290}{\nano\metre} and \SI{1294}{\nano\metre}, to generate a signal from \SI{1961}{\nano\metre} to \SI{1969}{\nano\metre}. Figure \ref{fig:gamma_N2O} presents the results obtained with the P(20) and the P(28) rovibrational lines. The inset corresponds to an absorption line obtained at \SI{700}{\milli\bar} for the P(28) rovibrational line, where the experimental results (black dots) show a good agreement with the spectrum calculated using the HITRAN database (green line).


% Figure environment removed


From the two curves of Fig. \ref{fig:gamma_N2O}, we calculate the self-broadening coefficients $\gamma_{\text{N}_2\text{O}}$ of the P(20) and the P(28) spectral lines of the $4\nu_1$ band, respectively equal to ($0.100\pm0.001$) \SI{}{\centi\meter^{-1}}$\text{atm}^{-1}$ and ($0.106\pm0.002$) \SI{}{\centi\meter^{-1}}$\text{atm}^{-1}$. These values are compatible to those given by HITRAN, respectively equal to ($0.096\pm0.009$) \SI{}{\centi\meter^{-1}}$\text{atm}^{-1}$ and ($0.103\pm0.010$) \SI{}{\centi\meter^{-1}}$\text{atm}^{-1}$.

\subsection{Spectroscopy of a mixture $\text{C}\text{O}_2$/$\text{N}_2\text{O}$ }

We now propose to study a mixture of the two previous gases, composed of $29.7\%$ of $\text{C}\text{O}_2$ and $70.3\%$ of $\text{N}_2\text{O}$. This mixture allows us to measure the broadening coefficient $\gamma_{\text{C}\text{O}_2 / \text{N}_2\text{O}}$, corresponding to the broadening of the rovibrational absorption lines of $\text{C}\text{O}_2$ due to the presence of $\text{N}_2\text{O}$. To this end, we study the same lines that have been studied in the case of pure $\text{C}\text{O}_2$. In case of a mixture $\text{C}\text{O}_2$/$\text{N}_2\text{O}$, the collisional HWHM $\Delta\nu$ of the absorption spectral lines can now be written as :

\begin{equation}
    \Delta\nu = \gamma_{\text{C}\text{O}_2} \text{Pp}_{\text{C}\text{O}_2} + \gamma_{\text{C}\text{O}_2 / \text{N}_2\text{O}} \text{Pp}_{\text{N}_2\text{O}}
\end{equation}

where $\text{Pp}_{\text{C}\text{O}_2}$  and $\text{Pp}_{\text{N}_2\text{O}}$ are respectively equal to  $29.7\%$ and $70.3\%$ of the total pressure. To obtain the broadening coefficient $\gamma_{\text{C}\text{O}_2 / \text{N}_2\text{O}}$, we first subtract the the self-broadening contribution to the width $\Delta\nu$. We obtain a corrected width $\Delta\nu ' = \Delta\nu - \gamma_{\text{C}\text{O}_2} \text{Pp}_{\text{C}\text{O}_2}$, using the value of $\gamma_{\text{C}\text{O}_2}$ that we found earlier. This collisional HWHM is plotted in Fig. \ref{fig:gamma_CO2_N2O} as a function of the partial pressure $\text{Pp}_{\text{N}_2\text{O}}$, to obtain the broadening coefficient $\gamma_{\text{C}\text{O}_2 / \text{N}_2\text{O}}$. The inset in Fig. \ref{fig:gamma_CO2_N2O} shows the good agreement between the absorption lines measured at \SI{700}{\milli\bar} (black dots) and  fitted with a Voigt profile (green line).




% Figure environment removed


The broadening coefficients $\gamma_{\text{C}\text{O}_2 / \text{N}_2\text{O}}$ for the rovibrational lines P(8) and P(38) are respectively found to be equal to ($0.087\pm0.004$) \SI{}{\centi\meter^{-1}}$\text{atm}^{-1}$ and ($0.083\pm0.008$) \SI{}{\centi\meter^{-1}}$\text{atm}^{-1}$.



Finally, we list in the Table  \ref{tab:Summmary_results} the experimental results obtained with the analysis of five absorption rovibrational lines for pure $\text{C}\text{O}_2$, pure $\text{N}_2\text{O}$, and for the mixture $\text{C}\text{O}_2$/$\text{N}_2\text{O}$. For each line, the wavenumber of the seeds and of the generated signals are indicated, as well as the broadening coefficient and its uncertainty for both the experimental measurements and the HITRAN data. These results are compatible with the value given by HITRAN, which confirms the efficiency of the proposed dual-comb spectrometer. It should be noted that the various collisional broadening coefficients are measured with much higher precision than those obtained from the HITRAN database.





\begin{table}[h!]
\centering
\begin{tabular}{l|c|l|l|c|l}
\multicolumn{1}{c|}{\textbf{Parameters}}                                                                                  &  \multicolumn{1}{c|}{\textbf{Pure  $\text{CO}_2$}}                                                      & \multicolumn{1}{c|}{\textbf{Mixture}}                & \multicolumn{1}{c}{\textbf{Pure $\text{N}_2\text{O}$}}                                                        \\ \hline
\begin{tabular}[c]{@{}l@{}}Line\\ $\nu_s$ ($\SI{}{\centi\meter^{-1}}$)\\ $\gamma_{\text{th.}} \pm \Delta\gamma_{\text{th.}}$ (\SI{}{\centi\meter^{-1}}$\text{atm}^{-1}$)\\ $\gamma_{\text{exp.}}$ $\pm$ $\Delta\gamma_{\text{exp.}}$ (\SI{}{\centi\meter^{-1}}$\text{atm}^{-1}$)\end{tabular} & \begin{tabular}[c]{@{}l@{}} P(2) \\5098.094\\ 0.123$\pm$0.006\\ 0.123$\pm$ 0.001\end{tabular} & \begin{tabular}[c]{@{}l@{}}P(2) \\ 5098.094\\ -\\ 0.095$\pm$0.007\end{tabular} & \begin{tabular}[c]{@{}l@{}}P(8) \\5098.640\\ 0.108$\pm$0.010\\ 0.102$\pm$0.001\end{tabular}   \\ \hline
\begin{tabular}[c]{@{}l@{}}Line \\$\nu_s$ ($\SI{}{\centi\meter^{-1}}$)\\ $\gamma_{\text{th.}} \pm \Delta\gamma_{\text{th.}}$ (\SI{}{\centi\meter^{-1}}$\text{atm}^{-1}$)\\ $\gamma_{\text{exp.}}$ $\pm$ $\Delta\gamma_{\text{exp.}}$ (\SI{}{\centi\meter^{-1}}$\text{atm}^{-1}$)\end{tabular} & \begin{tabular}[c]{@{}l@{}} P(8) \\5093.265\\ 0.111$\pm$0.005\\ 0.113$\pm$0.001\end{tabular}  & \begin{tabular}[c]{@{}l@{}}P(8) \\5093.265\\ -\\ 0.087$\pm$0.004\end{tabular} & \begin{tabular}[c]{@{}l@{}}P(12) \\5094.837\\ 0.103$\pm$0.010\\ 0.102$\pm$0.001\end{tabular}   \\ \hline
\begin{tabular}[c]{@{}l@{}}Line \\$\nu_s$ ($\SI{}{\centi\meter^{-1}}$)\\ $\gamma_{\text{th.}} \pm \Delta\gamma_{\text{th.}}$ (\SI{}{\centi\meter^{-1}}$\text{atm}^{-1}$)\\ $\gamma_{\text{exp.}}$ $\pm$ $\Delta\gamma_{\text{exp.}}$ (\SI{}{\centi\meter^{-1}}$\text{atm}^{-1}$)\end{tabular} & \begin{tabular}[c]{@{}l@{}}P(14) \\5088.242\\ 0.103$\pm$0.005\\ 0.106$\pm$0.003\end{tabular}  & \begin{tabular}[c]{@{}l@{}}P(14) \\5088.242\\ -\\ 0.086$\pm$0.002\end{tabular} & \begin{tabular}[c]{@{}l@{}}P(16) \\5090.846\\ 0.100$\pm$0.010\\ 0.100$\pm$0.001\end{tabular} \\ \hline
\begin{tabular}[c]{@{}l@{}}Line \\$\nu_s$ ($\SI{}{\centi\meter^{-1}}$)\\ $\gamma_{\text{th.}} \pm \Delta\gamma_{\text{th.}}$ (\SI{}{\centi\meter^{-1}}$\text{atm}^{-1}$)\\ $\gamma_{\text{exp.}}$ $\pm$ $\Delta\gamma_{\text{exp.}}$ (\SI{}{\centi\meter^{-1}}$\text{atm}^{-1}$)\end{tabular} & \begin{tabular}[c]{@{}l@{}}P(32) \\5072.043\\ 0.087$\pm$0.004\\ 0.091$\pm$0.001\end{tabular}  & \begin{tabular}[c]{@{}l@{}}P(32)\\ 5072.043\\ -\\ 0.080$\pm$0.002\end{tabular} & \begin{tabular}[c]{@{}l@{}}P(20) \\5086.669\\ 0.096$\pm$0.009\\ 0.100$\pm$0.001\end{tabular}   \\ \hline
\begin{tabular}[c]{@{}l@{}}Line \\$\nu_s$ ($\SI{}{\centi\meter^{-1}}$)\\ $\gamma_{\text{th.}} \pm \Delta\gamma_{\text{th.}}$ (\SI{}{\centi\meter^{-1}}$\text{atm}^{-1}$)\\ $\gamma_{\text{exp.}}$ $\pm$ $\Delta\gamma_{\text{exp.}}$ (\SI{}{\centi\meter^{-1}}$\text{atm}^{-1}$)\end{tabular} & \begin{tabular}[c]{@{}l@{}}P(38) \\5066.283\\ 0.082$\pm$0.004\\ 0.086$\pm$0.001\end{tabular}  & \begin{tabular}[c]{@{}l@{}}P(38) \\5066.283\\ -\\ 0.083$\pm$0.008\end{tabular}   & \begin{tabular}[c]{@{}l@{}}P(28) \\5077.766 \\0.103$\pm$0.010\\ 0.106$\pm$0.002\end{tabular}  
\end{tabular}
\caption{Summary of the experimental results obtained with five absorption spectral lines of pure $\text{CO}_2$, pure  $\text{N}_2\text{O}$, and with the mixture $\text{CO}_2 / \text{N}_2\text{O}$. For each measurement, we indicate the corresponding line, the signal wavenumber $\nu_s$, the broadening coefficient $\gamma_{\text{th.}}$ with its uncertainty $\Delta\gamma_{\text{th.}}$ given by HITRAN, and the experimental broadening coefficient $\gamma_{\text{exp.}}$ with its uncertainty $\Delta\gamma_{\text{exp}}$. The spectral lines of pure $\text{CO}_2$ and of the mixture correspond to the $2\nu_1+\nu_3$ band, whereas the $\text{N}_2\text{O}$ lines correspond to the $4\nu_1$ band.}
\label{tab:Summmary_results}
\end{table}




\section{Conclusion}
To conclude, we have presented a novel fiber design, using inclusions of fluoride doped silica to optimize the frequency conversion of dual frequency combs around \SI{2}{\micro\meter} through degenerate four-wave mixing. This special fiber is used in the proposed all-fibered dual-comb spectrometer, to measure the absorption spectral lines of $\text{C}\text{O}_2$ and $\text{N}_2\text{O}$ in both pure and mixed forms. From these measurements, the self-broadening coefficients of the two molecules and the broadening coefficient $\gamma_{\text{C}\text{O}_2 / \text{N}_2\text{O}}$ have also been obtained with a good accuracy. The summary of the spectroscopic results presented in the Tab. \ref{tab:Summmary_results} demonstrates the compatibility of the experimental results with the HITRAN database, and shows the efficiency of the dual-comb spectrometer. The precise experimental measurements contribute to our understanding of collisional broadening processes in two gases of environmental interest.




\section*{Fundings}
The authors would like to thank the Agence Nationale de la Recherche (ANR-17-EURE-0002, ANR-19-CE47-0008, ANR-21-CE42-0026); iXCore Research Foundation; Conseil régional de Bourgogne-Franche-Comté; and the FEDER (Fonds européen de développement régional).

\section*{Disclosures}
The authors declare no conflicts of interest.


\printbibliography
\end{document}