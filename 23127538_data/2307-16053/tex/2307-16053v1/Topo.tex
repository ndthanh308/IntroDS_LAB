%% LyX 2.0.8.1 created this file.  For more info, see http://www.lyx.org/.
%% Do not edit unless you really know what you are doing.
\documentclass[english,twocolumn,prl]{revtex4}
%\documentclass[english,prb]{revtex4}
\usepackage[T1]{fontenc}
%\usepackage[latin9]{inputenc}
\setcounter{secnumdepth}{3}
\usepackage{bm}
\usepackage{amsmath}
\usepackage{amssymb}
\usepackage{lipsum}
\usepackage{xcolor,comment}
%\usepackage{wasysym}
\usepackage{graphicx}
%\usepackage{esint}
\usepackage{bm}% bold math
%\usepackage{ dsfont }
\usepackage{makeidx}
\usepackage{epstopdf} % Graficas eps
\usepackage[retainorgcmds]{IEEEtrantools}
\usepackage{hyperref} % Para clickar en el pdf y que te dirija a la imagen/seccin/tabla
%\usepackage[export]{adjustbox}
%\usepackage{afterpage}
\usepackage{braket}
%\numberwithin{equation}{section}%ecuaciones numeradas por secciones
\def\y{\'{\i}}
\def\to{\rightarrow}
\def\l{\langle}
%\def\r{\rangle}
\def\p{\partial}
\def\q{{?`}}
\def\ni{\noindent}
\def\e{\varepsilon}
\def\d{\textrm{d}}
\def\non{\nonumber }
\def\D{\Delta}
\def\om{\omega}
\def\ra{\rangle}
\def\la{\langle}
\def\s{\sigma}
\def\med{\frac{1}{2}}
%
\newcommand{\beq}{\begin{equation}} 
\newcommand{\eeq}{\end{equation}} 
\newcommand{\beqa}{\begin{eqnarray}} 
\newcommand{\eeqa}{\end{eqnarray}} 
\makeatletter

%%%%%%%%%%%%%%%%%%%%%%%%%%%%%% LyX specific LaTeX commands.
%% A simple dot to overcome graphicx limitations
\newcommand{\lyxdot}{.}


%%%%%%%%%%%%%%%%%%%%%%%%%%%%%% Textclass specific LaTeX commands.
\@ifundefined{textcolor}{}
{%
 \definecolor{BLACK}{gray}{0}
 \definecolor{WHITE}{gray}{1}
 \definecolor{RED}{rgb}{1,0,0}
 \definecolor{GREEN}{rgb}{0,1,0}
 \definecolor{BLUE}{rgb}{0,0,1}
 \definecolor{CYAN}{cmyk}{1,0,0,0}
 \definecolor{MAGENTA}{cmyk}{0,1,0,0}
 \definecolor{YELLOW}{cmyk}{0,0,1,0}
 }
\newcommand{\MV}[1]{\textcolor{BLUE}{[MV:\,#1]}}

\makeatother

\begin{document}
%\bibliographystyle{apsrev4-1}
\bibliographystyle{naturemag}
\title{Novel Topological Anderson  insulating phases in the interacting Haldane model}


\author{ Jo$\tilde{a}$o S. Silva}
\email{jss.joaossilva@gmail.com}
\affiliation{Centro de F\'isica das Universidades do Minho e Porto, LaPMET,
Departamento de F\'isica e Astronomia, Faculdade de Ciencias,
Universidade do Porto, 4169-007 Porto, Portugal}

\author{ Eduardo V. Castro}
\email{evcastro@fc.up.pt}
\affiliation{Centro de F\'isica das Universidades do Minho e Porto,
Departamento de F\'isica e Astronomia, Faculdade de Ciencias,
Universidade do Porto, 4169-007 Porto, Portugal}

\author{Rubem Mondaini}
\email{rmondaini@csrc.ac.cn}
\affiliation{Beijing Computational Science Research Center, Beijing 100084,
China}

\author{Mar\'{\i}a A. H. Vozmediano}
\email{mahvozmediano@gmail.com}
\affiliation{Instituto de Ciencia de Materiales de Madrid, CSIC,
Sor Juana In\'es de la Cruz 3,  Cantoblanco,
E-28049 Madrid, Spain}

\author{M. Pilar L\'opez-Sancho}
\email{pilar@icmm.csic.es}
\affiliation{Instituto de Ciencia de Materiales de Madrid, CSIC,
Sor Juana In\'es de la Cruz 3,  Cantoblanco,
E-28049 Madrid, Spain}

\begin{abstract}
We analyze the influence of disorder and strong correlations on the topology in two dimensional Chern insulators.  A mean field calculation in the half--filled Haldane model with extended Hubbard interactions and Anderson disorder shows that  disorder favors topology in the interacting case and extends the topological phase to a larger region of the Hubbard parameters. In the absence of a staggered potential,  we  find a novel disorder-driven topological phase with Chern number C=1, with co-existence of topology with long range spin and charge orders. More conventional topological Anderson insulating phases are also found in the presence of a finite staggered potential.
 
\end{abstract}
%\date{\today}

\maketitle

\section{ Introduction and results} 
\label{sec_intro}
%
% Figure environment removed
%
Topological  phases of physical systems are one of the pillars of modern condensed matter \cite{Topo2021}. The topological features of a material  are  established at the non-interacting level and the fate of topology in strongly correlated systems is a relevant topic of current research in the field \cite{Rachel18}. 
Disorder, always present in real materials,  plays also an important role in the phase diagram of correlated electrons.   Although strong disorder would be detrimental to topology -eventually leading  to trivial, Anderson localized phases in two dimensional systems \cite{Anderson58}-, disorder--induced topological phases (Anderson topological insulators)\cite{AndTopo2009} are an exciting possibility  proposed recently.
In this work we will explore the interplay of topology, disorder and interactions using the Haldane model at half filling \cite{H88} as a paradigm of topological Chern insulators in 2D. We will consider the extended Hubbard model with nearest neighbor and next nearest neighbors ($U$ and $V$) interactions, and Anderson disorder $W$ and explore the mean field phase diagram. 

The Haldane model at half filling was originally set as a lattice model of spinless electrons on the Honeycomb lattice with nearest neighbors ($t$) and complex next to nearest neighbors ($t_2$) hopping amplitudes, as schematically shown in Fig.~\ref{fig1}(a). A staggered potential $\Delta$  promotes a trivial phase, and the value of $t_2/t$  promotes the topological phase as shown in Fig.~\ref{fig1}(b). When the spin degree of freedom is added, the topological phases have a Chern number  $C=\pm 2$.
As it is well known and will be detailed later, an on-site interaction $U$ drives the system to a spin density wave (SDW) while the NN interaction $V$ promotes a charge density wave phase (CDW).  Both are topologically trivial insulators. The phase diagram of the clean, interacting model in the mean field approximation, is shown in Fig.~\ref{fig2}.
%
% Figure environment removed
%
The role of disorder on the topological systems will also be discussed later. In general, a critical value of disorder strength will drive the topological insulator to a trivial Anderson insulator. 
%
% Figure environment removed
%

Figure~\ref{fig3} summarizes the main results of this work. It shows the phase diagram of the disordered, spinfull Haldane model as a function of the extended Hubbard  interactions $U$ and $V$ in units of the NN hopping parameter $t$. The Haldane parameters are chosen in the topological region of Fig.~\ref{fig1}(b)  with zero staggered potential $\Delta=0$ and $\phi=\pi/2$. The dashed lines mark the different phases in the absence of disorder, for better comparison (same as Fig.~\ref{fig1}(b)): the standard Chern insulator with Chern number $C=2$, and the SDW and CDW phases. Full lines separate the phases when  Anderson disorder $W=4$ (in units of $t$) is included. As we see, disorder enlarges the topological $C=2$ region and generates a novel $C=1$ phase  near the boundary of the three clean phases. This phase has long range spin and charge orders. The experimental realization of the Haldane model \cite{Jotzu_2014,HaldaneExp22} and the ability to realize strongly correlated Hubbard models using cold atom systems \cite{Greiner2002,Esslinger2010} give real prospects to emulate the spinfull, extended Haldane-Hubbard model \cite{RG18}. With  the ability to include disorder \cite{Schreiber2015,Choi2016}, the door is  open to direct confirmation of the results of this work. 


In what follows we will put these results in context comparing with previous works in the literature in Sec.~\ref{sec_literature}. In Sec.~\ref{ssec_c1}  we will provide details on the nature of the new disorder driven topological $C=1$ phase   and we will discuss the effect of disorder on the phase boundaries of Fig.~\ref{fig2}. The disordered phases arising with a  finite staggered potential will be reviewed in Sec.~\ref{ssec_trivialM}. We will discuss open questions and possible future works in Sec.~\ref{sec_future}. Technical details on the model and calculations can be found in Appendix~\ref{sec_model}. 



\section{Antecedents}
\label{sec_literature}
The effect of disorder and /or Hubbard interactions on the Haldane model has a long pre-history related with  the non-topological Honeycomb lattice. The phase diagram in Fig.~\ref{fig2} substituting the 
CI phase with SM (semimetal) has been revisited over and over since the pioneer works \cite{Sorella92}
%\MV{search for previous pioneers}. 
In this section we will only discuss the previous works in the literature that are closely related with our results.

\begin{itemize}
\item A $C=1$ phase in the clean, spin-full Haldane model with only on-site Hubbard $U$ was found in \cite{HeFeng_2011,He_2012,Troyer_2016,Wang_2019,Roser19,Prok19,Yuanetal22} as an interplay of finite staggered potential  $\Delta$ and $U$. No $C=1$ was found in mean field calculations with $\Delta=0$. The new phase is spin polarized and was termed "a topological spin density wave". An intuitive physical picture of this phase will be described in the next section. An important open question around this phase is whether or not it is an artifact of the used approximations, like mean field approximation,  since it was not found in a  dynamical cluster approximation in Ref.~\cite{Troyer_2016_2}. The $C=1$ phase was  recently re-established with an exact diagonalization calculation in \cite{Yuanetal22}. Its stability against long range Coulomb interaction was examined in \cite{Prok19}.


\item Topological transitions in the extended Haldane--Hubbard model ($U$, $V$) with zero staggered potential and no disorder  ($\Delta=0$, $W=0$)  were studied in \cite{Castro21}. No $C=1$ phase was found there, except for a particular cluster used in the exact diagonalization and attributed to finite size effects. A variety of techniques led them to conclude  that topological and locally ordered phases do not coexist in the model. 

\item
Interestingly, a $C=1$ phase has also been found in the topological square lattice ($C=2$ in the non-interacting limit) \cite{Wang_2019} with $U$ and $V$ interactions and a sublattice potential   $\Delta=2$. A mean field calculation shows a $C=1$ phase named  (interaction-driven) antiferromagnetic Chern insulator (AFCI) by the authors.  As in previous works, this phase is not present when $\Delta=0$. %\MV{They need the staggered potential to get c=1 because they are in the square lattice. It is not needed in the Honeycomb lattice, it seems.}

\item The interplay of NN interaction $V$, disorder, and topology in the
spinless Haldane-Hubbard model was addressed in \cite{Castro2021}. A  topological Anderson insulator found in the non-interacting system with a finite staggered potential, was shown to be stable to the presence of sufficiently small interactions. 


\end{itemize}

The study of the effect of disorder in the spinfull extended Haldane--Hubbard model is clearly missing. Also missing from previous results is a $C=1$ phase with $\Delta=0$.


\section{Characterizing the new Anderson topological insulators}
\subsection{Phase diagram in the  $\Delta=0$ case.}
\label{ssec_c1}
%
% Figure environment removed
%
The phase diagram of the $\Delta=0$ case is shown in Fig.~\ref{fig3}.
The most interesting finding there is 
the $C=1$ phase arising from the interplay of $U$, $V$, and $W$.  This is a Topological Anderson Insulator phase, highly disordered, showing a non zero spin polarization and charge inhomogeneities (electron-hole puddles) with a non zero mean value of the SDW and CDW order parameters. The  spin and charge order parameters are defined in Eq.~\eqref{eq_orderP}. Their evolution  as a function of the lattice size is shown in Fig.~\ref{fig4} ($N$ is the number of unit cells). The circles are calculated points with standard deviation of the mean as the error bars in  the vertical lines (see Appendix~\ref{sec_model}). It is clear that the spin order parameter will remain finite in the thermodynamic limit. The CDW order parameter shows more oscillations but, as is evident from the fit in Fig.~\ref{fig4} (see figure caption)  it does not interpolate to zero. A typical configuration of the charge inhomogeneity in the $C=1$ phase is shown in Fig.~\ref{fig5} for $U=5.5$, $V=1.73$, and $W=3.89$.  This phase is at odds with the analysis in  Ref.~\cite{Castro21} where they found no co-existence of topological and long range ordered phases in the clean model. 
%
% Figure environment removed
%


The $C=1$ region of the phase diagram would probably continue parallel to the CDW/SDW transition line for 
 higher values of $U$ and disorder $W$, similarly to what  happens in Ref.~\cite{Troyer_2016} for the parameters $U$ and $\Delta$, but the convergence becomes too slow as $W$ and $U$ increase and we did not explore this region.


The $C=1$ phase described previously in the spinfull Haldane model \cite{HeFeng_2011,He_2012,Zhu_2014,WuPeng_2015,Troyer_2016,Troyer_2016_2,Wang_2019,Yuanetal22} was due to  the interplay of a staggered potential and the local Hubbard $U$  interaction without NN interaction $V$ in the clean topological lattice. The $C=1$ phase was found in a narrow region between the two topologically trivial insulators induced by high values of the staggered potential (trivial insulator) and local U interaction (Mott-Hubbard insulator). The exotic phase was dubbed a "topological spin density wave" and is of the same type as the one described here. 

An intuitive understanding of the $C=1$ phase works as follows: 
It is easy to see that, at the mean field level, the CDW order parameter works like a staggered potential in the Haldane model while the SDW order parameter works as a spin dependent
staggered potential, with opposite sign for the two spin polarizations. The presence of both SDW and CDW order parameters will act as a trivial gap for one spin
polarization and reinforces the topological gap in the other. As a consequence, with increasing $U$, the bands for one spin polarization will become trivial while for the other they will still be topologically non-trivial. Since the Chern number is the sum of the two spin contributions, there will be a region
in parameter space where $C = 1$. %\MV{Add here a sentence of what disorder does}.
This explanation of the $C=1$ phase is sketched in Fig.~\ref{fig6}. The left graph shows the bands of the Haldane model with zero staggered potential around the Dirac points K, K'. The bands are degenerated in spin and have an inverted gap. The CDW induced by a NN interaction $V$ splits the degeneracy of the valleys as shown in the middle panel.The SDW due to an on-site interaction $U$ lifts the spin degeneracy and moves the spin-polarized bands as indicated in the right hand panel. For a critical value of the parameters, the inverted gap closes in one of the spin-polarized bands that becomes topologically trivial giving rise to the $C=1$ phase. 
%
% Figure environment removed
%

%\MV{New paragraph to be discussed:} 

%In the previous explanation there are two interesting things to notice: 
%first, it works perfectly in the 
%case of having a finite trivial mass $\Delta\neq 0$ to begin with -- as happens in all the previous works in the literature --. Second, disorder does not seem to play a mayor role.

Comparing with previous works in the literature one is tempted to think that the role played by $\Delta$ there, is taken by $V$ in our case. Indeed, the CDW order parameter is proportional to $(n_{A \uparrow} - n_{B \uparrow}) + (n_{A \downarrow} - n_{B \downarrow})$ while SDW is proportional to $(n_{A \uparrow} - n_{B \uparrow}) - (n_{A \downarrow} - n_{B \downarrow})$, where $n_{\Gamma \sigma}$ is the charge density in sublattice $\Gamma$ with spin $\sigma$. In other words, while for CDW the order parameter is proportional to the sum of the sublattice charge imbalance of the two spin-components, for SDW it is the difference. Since the $C=1$ phase has both CDW and SDW, it means that $|n_{A \uparrow} - n_{B \uparrow}| \neq |n_{A \downarrow} - n_{B \downarrow}|$, as expected when the two spin components have different gaps as in Fig.~\ref{fig6} right panel.

However, the analysis of the clean extended Haldane--Hubbard model does not show the $C=1$ phase \cite{Castro21}. It is disorder that, allowing the co-existence of topological and long range orders, permits the CDW order parameter to work as a trivial mass. The rather inhomogeneous CDW in the $C=1$ phase, as exemplified in Fig.~\ref{fig5}, could even be the missing ingredient to stabilize the co--existence of SDW and CDW absent in the clean limit. Moreover, the fact that the $C=1$ phase only appears for high values of disorder indicates that it is a non-perturbative phase and that explanations based on perturbations around the clean limit have to be taken with a grain of salt. 

%\MV{Paragraph ends here. Needs polishing.}


Our results manifest the importance of disorder in the boundary regions close to phase transitions. 
We have analyzed the effect of the various parameters ($U$, $V$, $W$) on the boundaries between the $C=2$ phase and the CDW, SDW in Fig.~\ref{fig2} for the $\Delta=0$ case. 
We have seen that   Anderson topological  insulating phases with $C=2$ appear in all the region around the phase transition lines as shown in Fig.~\ref{fig3}. Over a critical (interaction dependent) value of disorder,  topology disappears and only trivial insulators remain.


\subsection{Disorder in the finite $\Delta$ phase diagram.}
\label{ssec_trivialM}
As mentioned before, an SU(2) broken $C=1$ phase  was previously found in the clean Haldane-Hubbard model  as a result of a competition of the SDW insulator driven by $U$ and the trivial insulator driven by the staggered potential $\Delta$ \cite{HeFeng_2011,He_2012,Zhu_2014,WuPeng_2015,Troyer_2016,Troyer_2016_2,Wang_2019,Yuanetal22}. 
 We have analyzed the influence of   disorder and the interaction $V$ on  that phase for a fixed value of $\Delta= 1.2$. We  found that, for $V=0$,  the phase is robust to disorder up to $W=4$ where a trivial Anderson insulator sets in.  Disorder favors the  $C=1$ phase, which  appears at much lower values of $U$. Although $V$ is detrimental to topology,  with a combination of $V= 0.25$ and $W = 2.5$, we have the onset of the $C=1$ phase at $U\sim 0.6$. % \MV{It seems that the combination of V and W plays a role similar to the $t_A\neq t_b$ but we have to discuss it before writing it explicitly. Importance of breaking AB symmetry.} 
In this context it is worth noticing the result discussed in \cite{Troyer_2016} where it was seen that an explicit breakdown of SU(2)  by having different hopping amplitudes in the two sublattices led to
the $C=1$ phase even at $U=0$. 

It is worth noting that  plateau transitions $C=2\rightarrow1\rightarrow0$
are possible with increasing disorder at finite $\Delta$ and interactions.
 Plateau transitions $C=\pm2\rightarrow\pm1\rightarrow0$  with increasing disorder are conjectured to be ruled out in quantum Hall systems \citep{hatsugai1999sum} and other Chern insulators derived from Dirac Hamiltonians \citep{Song16}. In those systems, starting with $|C|\geq2$, a plateaus transition $\Delta C=\pm1$ is never observed with increasing disorder due to ensemble averaging over disorder realizations. Our results for finite disorder in the presence of interactions show that such a transition is possible, in particular if a finite trivial mass is also present.



\section{Open questions and future}
\label{sec_future}
As mentioned in Sec.~\ref{sec_literature}, an important open question  is to ascertain that the $C=1$ phase  is not an artifact of the mean field approximation \cite{Troyer_2016_2} or a finite size effect \cite{Castro21}. Exploring this region of the parameters with alternative methods as in \cite{Yuanetal22} will be very enlightening. 

Topological phase transitions between $C=1$ and $C=0$ or $C=2$ were found to be of  third order in the clean system \cite{HeFeng_2011,Shi2021}. Disorder makes the analysis of the nature of the phase transitions a hard problem that was left aside in this work, but studying  the nature of the phase transition between the $C=1$ and the surrounding phases is worth tackling in the future.  This is the problem of the phase transition between a standard and a topological Anderson insulator \cite{Chalker87} also related to the issue of localization  in quantum Hall systems \cite{Letal94,OAN07,09Murakami,Song16}.

Another interesting issue is to analyze the structure of the topological edge states in the new phase and their evolution with increasing disorder. 



%There are other interesting works concerning the Haldane Hubbard model in the literature. The influence of the momentum dependence of the self-energy on the Chern number of Chern insulators and an analysis of how it affects the phase diagram of the Haldane Hubbard model has been made in \cite{Roser19}.  A spin liquid phase in the Haldane model with finite $\Delta$ was described in \cite{SpinLiquid_2011}. The interplay of disorder and topology has also been studied in the Kane-Mele model \cite{KM16}.

\begin{acknowledgments}
 \section{Acknowledgement}
 JS and EC acknowledge financial support from FCT-Portugal through Grant No. UIDB/04650/2020. MPLS, MAHV and JS acknowledge the support of the Spanish Comunidad de Madrid grant S2018/NMT-4511 (NMT2D-CM). MAHV is also supported by the Spanish Ministerio de Ciencia e Innovaci\'on  grant PID2021-127240NB-I00.
This work was completed during a visit of MAHV to the Donostia International Physics Center (DIPC)
whose kind support is deeply appreciated.
\end{acknowledgments}

\bibliography{Topo}
\appendix
\section{ Model and technicalities }
\label{sec_model}

The  tight binding Hamiltonian of the Haldane model \cite{H88} is
\beqa
H  &=  -t\displaystyle\sum_{\langle i,j\rangle}c_i^{\dagger}c_j
-t_2\sum_{\langle\langle i,j\rangle\rangle}e^{-i\phi_{ij}}c_i^{\dagger}c_j \\\nonumber
&+  \Delta\displaystyle\sum_i \eta_i c_i^{\dagger}c_i,
\label{TBHmodel}
\eeqa
%
where $c_i=A,B$ are defined in the two triangular sublattices that form the honeycomb lattice. 
The first term $t$ represents a standard
real nearest neighbor hopping that links the two triangular sublattices.
The $t_2$ term represents a complex next nearest neighbor (NNN) hopping $t_2 e^{-i\phi_{ij}}$
acting within each triangular sublattice with a 
phase  $\phi_{ij}$ that has
opposite signs $\phi_{ij}=\pm\phi$ in the two sublattices.
This term breaks time--reversal symmetry and opens a non--trivial topological gap
at the Dirac points proportional to the magnitude of $t_2$. We have done our calculations for the fixed value  $\phi=\pi/2$. The last term represents a staggered potential ($\eta_i=\pm 1$). It breaks 
inversion symmetry and opens a trivial gap at the Dirac points. Spin doubles de degrees of freedom and makes the Chern number $C=\pm 2$ in the topological phases. The structure of the NNN hoppings and phase  diagram were shown in Fig. \ref{fig1}.


The interacting Hamiltonian we consider has the form 
\begin{equation}
H_{int}=U\sum_{i}n_{i,\uparrow}n_{i,\downarrow}+V\sum_{\langle i,j\rangle,\sigma,\sigma^{\prime}}n_{i,\sigma}n_{j,\sigma^{\prime}},
\label{eq_interactions}
\end{equation}
where $n_{i,\sigma}=c_{i,\sigma}^{\dagger}c_{i,\sigma}$ is the number
operator. The on-site interaction term $U$ penalizes double occupancy,
hence it favors a homogeneous charge distribution between the two
sublattices. In this sense, it goes against the staggered on-site
potential $\Delta$ and can favor topology to some extent. $U$ also
has the effect of polarizing the spin and, over a critical value, 
it drives the system to a spin density wave insulator. The NN repulsive
interaction $V$ essentially acts in the same space as the NN hopping
term $t$, and in fact it renormalizes its value. As such, it favors
the mobility of electrons (metallic phases) and intuitively it goes
against topology.

Potential (Anderson) disorder is implemented by 
adding to the Hamiltonian the term $\sum_{i\in A,B}\varepsilon_{i}c_{i}^{\dagger}c_{i}$,
with a uniform distribution of random local energies, $\varepsilon_{i}\in[-W/2,W/2]$. 
As we see, this is an on-site term that will contribute to the mean field decoupling of 
the Hubbard U. 

%The interacting Hamiltonian is not quadratic in the fermion operators, making it difficult to deal with it exactly. Thus, we opt to cast the interactions in a mean field picture where, 
A mean field decoupling of \eqref{eq_interactions} gives
%
\begin{equation}
\begin{aligned}&H_{int}^{MF}  =U\sum_{i,\sigma}\left[\langle n_{i,-\sigma}\rangle c_{i,\sigma}^{\dagger}c_{i,\sigma}-\langle c_{i,-\sigma}^{\dagger}c_{i,\sigma}\rangle c_{i,\sigma}^{\dagger}c_{i,-\sigma}\right]+\\
 & +V\left[\sum_{\langle i,j\rangle,\sigma,\sigma^{\prime}}\langle n_{j,\sigma^{\prime}}\rangle c_{i,\sigma}^{\dagger}c_{i,\sigma}-\sum_{\langle i,j\rangle,\sigma,\sigma^{\prime}}\langle c_{j,\sigma^{\prime}}^{\dagger}c_{i,\sigma}\rangle c_{i,\sigma}^{\dagger}c_{j,\sigma^{\prime}}\right].
\end{aligned}
\end{equation}

The procedure to compute the mean field parameters $\langle c_{i,\sigma}^{\dagger}c_{j,\sigma^{\prime}}\rangle$ goes as follows: We initialize the parameters (or, in
other words, we choose an initial condition for the system), we then
diagonalize the Hamiltonian and obtain its eigenvectors and energy
spectrum, and re-calculate the mean field parameters,
\begin{equation}
\langle c_{i,\sigma}^{\dagger}c_{j,\sigma^{\prime}}\rangle=\sum_{E<E_{F}}\left[\psi_{i}^{\sigma}\left(E\right)\right]^{*}\psi_{j}^{\sigma^{\prime}}\left(E\right),
\end{equation}
with $E_{F}$ the Fermi energy and $\psi_{i}^{\sigma}\left(E\right)$
the wave function amplitudes. Finally, we define a convergence threshold
$\varepsilon$, and repeat the previous steps until $\left|\langle c_{i,\sigma}^{\dagger}c_{j,\sigma^{\prime}}\rangle_{I+1}-\langle c_{i,\sigma}^{\dagger}c_{j,\sigma^{\prime}}\rangle_{I}\right|<\varepsilon$,
with I the iteration.

Since a mean field method can be biased (meaning, the mean field parameters
reached after convergence can be heavily dependent on the choice of
initial condition), we employ a set of initial conditions, apply this
procedure to each of them, and choose the solution that yields the
lowest ground-state energy for this system (which can be calculated
as the sum of the energies of the occupied eigenstates). %With regards to the convergence, another technical detail should be mentioned here.
The simple test of convergence we presented can, in some cases, prove
to be very slow at reaching convergence. There are many ways of circumventing
this issue. We chose to, after each iteration, define the new mean
field parameters as the average of the previous two iterations. 

The topology of each phase is determined by  the Chern number. We compute
it following  Fukui's method \cite{FHS05}, modified to work
 with disordered systems \cite{ZYetal13}, where translational
invariance is broken. 
%\MV{Add  the definition of the SDW and CDW order parameters an on averaging over disorder configuartions}

The mean field order parameters are obtained self-consistently  for a fixed disorder configuration. Due to the finite size of the simulated clusters we repeat the procedure for 
$N_{dis}\sim 50-100$ disorder realizations. Figs. \ref{fig3} and \ref{fig4} are obtained averaging over disorder quantities. For the c=1 phase, we compute the Chern number for each disorder configuration. A perfectly quantized c=1 is obtained for  over 60$\%$ of the disorder configurations in cluster sizes N=10-13 reaching 70$\%$ for the largest sizes. This is a strong indication that the C=1 phase is not a finite size effect. 


To  characterize the onset and nature of various local orders
we define  order parameters
for charge density waves (CDW) and spin density waves (SDW), 
as
\begin{equation}
\begin{aligned}\mathcal{O}_{CDW} & =\frac{1}{N}\sum_{\sigma}\left(\sum_{i\in A}n_{i,\sigma}-\sum_{i\in B}n_{i,\sigma}\right),\\
\mathcal{O}_{SDW} & =\frac{1}{4N}\sum_{i\in A}\left(n_{i,\uparrow}-n_{i,\downarrow}\right)-\frac{1}{4N}\sum_{i\in B}\left(n_{i,\uparrow}-n_{i,\downarrow}\right),
\end{aligned}
\label{eq_orderP}
\end{equation}
where $N$ is the number of unit-cells. Since we are only interested
on whether there is a net charge/spin imbalance, the sign of the
structure factors is irrelevant and  we focus only on the absolute
value of these quantities. These are calculated for each disorder
configuration and subsequently averaged over all disorder configurations
as done with the Chern number. The standard error associated with the disorder averages (depicted as vertical bars on the data points of Fig. \ref{fig4}) is calculated as $\sigma/\sqrt{n}$, with n the number of disorder configurations and $\sigma$ the standard deviation. 


\end{document}



%%%%%%%%%%%%%%%%%%%%%%%%%%%%%%%%%%%%%%%%%%%%%%%%%%%5











 
The article is organized as follows: In xx we introduce the model and the definitions of the main physical quantities that define the mean field phase diagram. In XXX we review the effect of disorder
and NNN interaction on previous works. Sec. XXX we analyze the new c=1 phase, give a physical interpretation of it, and discuss the order of the topological phase transition. We set up our main conclusions and open questions in Sec. XXX. 

\subsection{Potentially useful things in the previous literature}


\noindent
{\bf Previous c=1 }
\MV{To be deleted later.}
\begin{itemize}

\item He-Feng 2011 \cite{HeFeng_2011}.

Haldane-Hubbard with and without staggered sublattice potential $\Delta$ (Semenoff mass). 
First c=1 phase found prior to Troyer in both cases.
Topological phase transition between c=1 and c=0 or c=2 are third order.
The c=1 phase only appears with non zero $\Delta$.  {\bf It is spin polarized "topological
spin density wave"}. Explanation in the follow up paper.

\item  Spin liquid 2011 \cite{He_2012}.
Some of the authors of the previous paper. Deepening into the c=1 phase. They call it "Composite spin liquid (spin liquid without spin-charge separation)" and point to a short range spin density wave.

\MV{A side comment: exotic phases appear for large values of t'. How does V affect the effective t'?
V renormalizes t (which, as increasing, would favor metallic phases). Since the important parameter is t'/t, the influence of V is to effectively reduce t' hence it is detrimental to topology.}


\item Troyer 2016 \cite{Troyer_2016,Troyer_2016_2}. Explanation: {\it above
the line $ \Delta=U/2$, U drives all of the particles to the lower energy
{\bf sublattice (??)}. In the mean-field solution (14-18) for the c=1 phase the
staggered potential drives bf {\bf one of the components (??)} mostly to
the low-energy sublattice. Thus, this component is effectively
in the topologically trivial region of the phase diagram
of the Haldane model. However, the larger density of one
component on the lower sublattice creates a Hartree potential
that mostly cancels the sublattice potential difference
?AB for the other component, which then carries the Chern
number C=1.} The idea is that U lowers the energy of configurations without double 
occupancy. tends to populate the two sublattices with 
oposite spins. I still do not understand why would it be a sublattice more populated than the other.
So, I do not understand the explanation.

Perhaps I understand "our" explanation to be described later. Meanwhile
we see that we are quite in the same situation as Troyer's: As mentioned above, the ultimate effect of V is to  decrease the rate t'/t and, hence, it has a similar effect as increasing the staggered potential.
So, effectively, having U and V (no disorder yet) is similar to Troyer model (U and $\Delta$). 


Problem: in \cite{Troyer_2016_2} they explore their mean fiel c=1 phase with a dynamical cluster approximation and claim: {\it We find evidence of a first order phase transition from a Chern
insulator at weak coupling to a topologically trivial antiferromagnetic insulator at strong coupling. {\bf These results call into question a previously found intermediate state with coexisting topological
character and antiferromagnetic long-range order.} Our main result is a first order phase
transition from a topologically non-trivial band insulating
state to a magnetic long-range ordered state, preempting
the intermediate "topological AFI" state.} Summarizing, they claim that the c=1 phase is a mean
field artifact.



\item Wang and Qi (2019) \cite{Wang_2019}.

Topological square lattice with c=2. U, V interactions and sublattice potential   $\Delta=2$. Mean field. Found a c=1 phase. They call it (interaction-driven) antiferromagnetic Chern insulator (AFCI). Basically it is the same explanation as Troyer. As in previous works, do not get it with $\Delta=0$. {\bf Interesting}: They have also V interaction. Emphasis on the time reversal breaking by AF order: {\it Then why the non-vanishing sublattice potential can
drive the AFCI phase and what is the underlying physical
mechanism? We can understand it from the viewpoint of breaking the TRS in magnetic topological insulators}. \MV{They need the staggered potential to get c=1 because they are in the square lattice. It is not needed in the Honeycomb.}
 
\item Wang etal (2020) \cite{Wang2020}.

\item Yuan et al (2022) 2205.1409 \cite{Yuanetal22}

Same model as Troyer with exact diagonalization. Same results. Aim: nature of phase transition:
excitation gaps, structure factors and fidelity metrics in ED results.

Same explanation for de c=1 phase.

Claim that the system can transition from c=1 to c=2 w/o closing the gap
(see the third order nature of the phase transition found in 2011 and in \cite{Shi2021}).
\\

{\bf Related works (no c=1)}

\item  Spin liquid 2011 \cite{SpinLiquid_2011}.
Haldane model with $\Delta$. Found spin liquid phase.

\item Nguyen et al 2013 \cite{Nguyen_2013}.
Haldane with U, $\Delta=0$. Coherent potential approximation. Find a metallic phase 
between the standard Chern insulator (c=2) and AFM (c=0).

\item Eduardo 1 (2021)\cite{Castro21}. Spinfull Haldane with U and V, no disorder no staggered potential. 
Exact-diagonalization, mean-field variational approach, and infinite density matrix renormalization group DMRG) applied to an infinite honeycomb cylinder.
In addition to the SDW, CDW they find  a topologically nontrivial insulator with nonlocal order.
{\it Once a local order parameter is formed, the topological characteristics of the ground state, associated with a finite Chern number, are no longer present, resulting in a topologically trivial wave  function.... Finite-size effects may result in misleading conclusions on the coexistence of
finite local order parameters and nontrivial topology in this model.} I.e., there is no coexistence of topology and long range order.

\item Eduardo 2 (2021)\cite{Castro2021}. Spinless Haldane with V, disorder and staggered potential. 
{\it The topological Anderson insulator, demonstrated in
noninteracting settings, is shown to be stable in the presence of sufficiently small interactions before a charge density wave Mott insulator sets in. We further promote a finite-size analysis of the transition to the ordered state in the presence of disorder, finding a mixed character of first- and second-order transitions in finite lattices, tied to the specific conditions of disorder realizations and boundary conditions used.} This last aspect is similar to what we find.

\end{itemize}
%
\subsection{Other works to cite}
\begin{itemize}
\item \cite{RG18} A variation of the Haldane model with U, $\Delta$ no disorder. Cite in the context of the defense of the mean field methods. Varies in that the complex phase is located at the NNN and grows with the distance. Very strange. Do not cite?
\item



\end{itemize}



\section{ Model and technicalities }
\label{sec_model}

The  tight binding Hamiltonian of the Haldane model \cite{H88} is
\beqa
H  &=  -t\displaystyle\sum_{\langle i,j\rangle}c_i^{\dagger}c_j
-t_2\sum_{\langle\langle i,j\rangle\rangle}e^{-i\phi_{ij}}c_i^{\dagger}c_j
&+  \Delta\displaystyle\sum_i \eta_i c_i^{\dagger}c_i,
\label{TBHmodel}
\eeqa
%
where $c_i=A,B$ are defined in the two triangular sublattices that form the honeycomb lattice. 
The first term $t$ represents a standard
real nearest neighbor hopping that links the two triangular sublattices.
The $t_2$ term represents a complex next nearest neighbor (NNN) hopping $t_2 e^{-i\phi_{ij}}$
acting within each triangular sublattice with a 
phase  $\phi_{ij}$ that has
opposite signs $\phi_{ij}=\pm\phi$ in the two sublattices.
This term breaks time--reversal symmetry and opens a non--trivial topological gap
at the Dirac points proportional to the magnitude of $t_2$. We have done our calculations for the fixed value  $\phi=\pi/2$. The last term represents a staggered potential ($\eta_i=\pm 1$). It breaks 
inversion symmetry and opens a trivial gap at the Dirac points. Spin doubles de degrees of freedom and makes the Chern number $c=\pm 2$ in the topological phases.

\MV{Add the interacting Hamiltonian and explain what are the physical consequences of each parameter: $t_2$ favors topo (see Fig. \ref{fig1}); $\Delta$ breaks the energy degeneracy of the sublattices and opens a trivial gap in the spectrum  (see Fig. \ref{fig1}); U penalizes double occupancy, hence it tends to make the charge distribution homogeneous between the sublattices in this sense it goes against $\Delta$ and can favor topology. U polarizes the spin increasing its magnitude, opens a gap in the spectrum (spin density wave). V is a repulsive NN interaction that acts in the same space as the hopping $t$. It renormalizes its value. As such, it favors mobility of the electrons (metallic phases). Since t sets the energy rank, the effective value of $t'/t$ is also renormalized. This is how the V interaction acts again topology.}

\MV{Discuss here the details of the program, convergence, and calculation of the Chern number.
Or, perhaps, make an appendix with the details.}



% Figure environment removed


%
\section{Results} 
\label{sec_results}

\subsection{$\Delta\neq 0$}
We will discuss first the influence of a staggered potential on the phase diagram. The starting point is a fixed  "Haldane mass" with value xxx, and a staggered potential $\Delta$ as an extra parameter. The results confirm all the previous phases obtained in the literature. In addition, it analyzes the influence of disorder on the clean ones \cite{} and of the interaction V on the disordered \cite{}.


\subsection{ $\Delta=0$}
% Figure environment removed


In this sub-section, we set the staggered potential to zero and the "Haldane mass" is fixed to whatever as in the previous section. Hence, prior to adding  disorder or interactions,  we will generically  be at the middle point of the c=2 phase in fig. \ref{fig1}: the starting point system is a topological Chern insulator with c=2. 

Figure \ref{fig_Joao1} shows the  evolution  of the Chern number as a function of the interactions U, V and disorder W.
We notice several interesting features in fig. \ref{fig_Joao1}. Looking first at the line  $U=0$,
we see that that the effect of increasing V is detrimental to the topological phase. \MV{We can understand this behavior as an effective increasing of the NN hopping t renormalized by the NN interaction V. As a result, the effective value of $t'/t$  decreases.} The well established topological region seen in \ref{fig_Joao1} (a) for $V=0.5$,  survives to high values of disorder $W=5$, and disappears for the relative mild value $V=1$  (fig. \ref{fig_Joao1} (b)). This result is compatible with the behavior discussed in ref. \cite{Castro21} without disorder.

A second interesting feature is the observation in fig. \ref{fig_Joao1} (b) of the role played by U in the phase diagram. With no disorder (W=0) and for V=1, for values of U comparable to V, the topological phase is restored even for the larger values of disorder. \MV{Explain this as discussed in the model section if we agree with this.}

\subsubsection{The c=1 phase}
\label{sec_exotic}

The most interesting finding of this work  is the presence of a phase with Chern number c=1  shown in fig. \ref{fig_Joao1} seen as white spots developing for increasing values of V in the border between the trivial (c=0 deep blue) and the topological phase (deep red). Details of this phase are shown in fig. \ref{fig_Joao2} where we depict the results of the Chern number evolution as a function of V and W in the boundary region for a fixed value of $U=5.5$ for a larger (17x17) lattice.  

A c=1 phase in the spin-full Haldane model was first described in refs. \cite{HeFeng_2011,He_2012} and later in \cite{Zhu_2014,WuPeng_2015,Troyer_2016}. A c=1 phase has also been found in the topological squared lattice \cite{Wang_2019}. In all cases, they analyzed the interplay of a staggered potential and the local U  interaction without NN interaction V in the clean topological lattice. The c=1 phase was found in a narrow region between the two topologically trivial insulators induced by high values of the staggered potential (trivial insulator) and local U interaction (Mott-Hubbard insulator). The exotic phase was a "topological spin density wave" attributed to the full spin polarization of one of the spin components. 

Our c=1 phase is of a different kind. First, unlike in the previous cases, it appears at zero staggered potential. In addition, it has no spin or charge polarization (fig. \ref{fig_Joao2}.).
We will discuss it in the summary section.   The c=1 phase disappears as V is increased to ??.

%Figure \ref{fig_Joao2} shows the  evolution  of the Chern number as a function of the local interaction U, and disorder W, for fixed values of the NN interaction V. The color code is the same as in Fig. \ref{fig_Joao1}.

%% Figure environment removed
%


\section{Discussion and summary}
% Figure environment removed
A summary of the results discussed in this work is sketched in Fig. \ref{fig_sketch}.
Summarize the new results: role of disorder, exotic C=1 phase.


\appendix
\section{Whatever}
\label{sec_hal}
%


\bibliography{Topo}
\end{document}
