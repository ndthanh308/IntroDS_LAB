\documentclass[12pt]{amsart}
\renewcommand{\baselinestretch}{1.165}
\usepackage[utf8]{inputenc}
%\usepackage{helvet}

\usepackage{eucal}
\usepackage{upgreek}
\usepackage{pdfsync}
\usepackage[all, cmtip]{xy}
%\usepackage[dvips]{graphicx}
\usepackage{amsfonts}
\usepackage{amsmath}
\usepackage{amssymb}
\usepackage{amscd}
\usepackage{color,xcolor}
\usepackage{amsthm}
\usepackage{amsxtra}
\usepackage{bm}
\usepackage{bbm}
\usepackage{graphicx}
%\usepackage{indentfirst}
%\usepackage[hmargin=3cm,vmargin=3cm]{geometry}
%\usepackage{parskip}
%\usepackage{soul}
\usepackage{mathrsfs}
%\usepackage{cite}
\usepackage[numbers,sort]{natbib}
\usepackage{isomath}
\usepackage{enumitem}
%\usepackage{enumerate}
\usepackage{latexsym} 
\usepackage{dsfont}
\usepackage{url}
\usepackage{mathtools}
\usepackage{soul}


%\setlength{\textwidth}{100cm}
%\setlength{\textheight}{100cm}
\setlength{\hoffset}{-.5 in}
\setlength{\voffset}{-.5 in}
\setlength{\textwidth}{6.0 in}
\setlength{\textheight}{8.5 in}


\setcounter{tocdepth}{2}




%----------------------------------------------------------------------
% Printing style.
%

% Double space printing.
% ----------------------
%\renewcommand{\baselinestretch}{1.2}

% MARGINS ETC.
% ------------
%\oddsidemargin 30pt % Left margin on odd-numbered pages.
%\evensidemargin 30pt % Left margin on even-numbered pages.
%\marginparwidth 40pt % Width of marginal notes.
%\marginparsep 10pt % Horizontal space between outer margin and
% marginal note

% VERTICAL SPACING:
% -----------------
\topmargin 0pt % Nominal distance from top of page to top of
% box containing running head.
\headsep 15pt % Space between running head and text.

% DIMENSION OF TEXT:
% ------------------
\textheight 8.6in % Height of text(including footnotes and figures,
% excluding running head and foot).
\textwidth 6.15in % Width of text line.
\topmargin 0pt
%\textheight 8.4in % Height of text(including footnotes and figures,
% excluding running head and foot).
%\textwidth 6.0in % Width of text line.
%\baselineskip = 18pt

\headheight12pt % important for printing in 12pt (?).





\newtheorem{thm}{Theorem}
\renewcommand*{\thethm}{\Alph{thm}}
%\renewcommand{\infty}{{\infty}}
%\newtheorem{thm}{Theorem}
\newtheorem{prop}{Proposition}%[section]
\newtheorem{lma}[prop]{Lemma}
\newtheorem{cpt}{Computation}
\newtheorem{cor}[prop]{Corollary}
\newtheorem{clm}[prop]{Claim}
%\newtheorem{qtn}{Question}

\theoremstyle{definition}

\newtheorem{df}[prop]{Definition} %[subsection]
\newtheorem{ft}{Fact}
\newtheorem{ntn}{Notation}
\newtheorem{descr}{Description} %[section]
\newtheorem{conj}[prop]{Conjecture}

\theoremstyle{remark}
\newtheorem*{pf}{Proof}
\newtheorem*{pfs}{Proof (sketch)}
\newtheorem{rmk}[prop]{Remark} %[subsection]
\newtheorem{asm}[prop]{Assumption} %[subsection]
\newtheorem{exa}[prop]{Example}
\newtheorem{qtn}[prop]{Question}



\def\mrm#1{{\mathrm{#1}}}
\def\bb#1{{\mathbb{#1}}}
\def\cl#1{{\mathcal{#1}}}
\def\ul#1{{\underline{#1}}}

\newcommand{\R}{{\mathbb{R}}}
\newcommand{\Z}{{\mathbb{Z}}}
\newcommand{\C}{{\mathbb{C}}}
\newcommand{\Q}{{\mathbb{Q}}}
\newcommand{\N}{{\mathbb{N}}}
\newcommand{\D}{{\mathbb{D}}}
\newcommand{\bK}{{\mathbb{K}}}
\newcommand{\HH}{{\mathbb{H}}}
\newcommand{\bH}{{\mathbb{H}}}
\newcommand{\bF}{{\mathbb{F}}}
\newcommand{\pt}{{[pt]}}

\newcommand{\rH}{{\mathrm{H}}}

\newcommand{\bs}{\bigskip}
\newcommand{\ra}{\rightarrow}
\newcommand{\del}{\partial}
\newcommand{\ddel}[1]{\frac{\partial}{\partial{#1}}}
\newcommand{\sm}[1]{C^\infty(#1)}
\newcommand{\Lafield}[1]{{\Lambda_{\text{#1},\bK}}}
\newcommand{\vareps}[1]{\varepsilon_{#1}}

\newcommand{\B}[1]{{\mathbf #1}}
\newcommand\vol{\operatorname{vol}}
\newcommand\sign{\operatorname{sign}}
\newcommand\lk{\operatorname{lk}}

\newcommand{\delbar}{\overline{\partial}}
\newcommand{\Sum}{\Sigma}
\newcommand{\G}{\mathcal{G}}
\newcommand{\Pe}{\mathcal{P}}
%\newcommand{\X}{\mathfrak{X}}
\newcommand{\J}{\mathcal{J}}
\newcommand{\A}{\mathcal{A}}
\newcommand{\K}{\mathcal{K}}
\newcommand{\ZZ}{\mathcal{Z}}
\newcommand{\eL}{\mathcal{L}}
\newcommand{\cL}{\mathcal{L}}
\newcommand{\mone}{{-1}}
%\newcommand{\st}{{^s_t}}
\newcommand{\oi}{_0^1}
\newcommand{\intoi}{\int_0^1}
\newcommand{\til}[1]{\widetilde{#1}}
\newcommand{\wh}[1]{\widehat{#1}}
\newcommand{\arr}[1]{\overrightarrow{#1}}
\newcommand{\paph}[1]{\{ #1 \}_{t=0}^1}
\newcommand{\con}{\#\;}
\newcommand{\bra}[1]{{\{ #1 \}}}
\newcommand{\codim}{\text{codim}}

\newcommand{\red}[1]{{\color{red} #1}}
\newcommand{\blue}[1]{{\color{blue} #1}}

\newcommand{\overbar}{\overline}

\newcommand{\om}{\omega}
\newcommand{\sig}{{\sigma}}
\newcommand{\al}{\alpha}
\newcommand{\la}{\lambda}
\newcommand{\La}{\Lambda}

\newcommand{\Om}{\Omega}
\newcommand{\ga}{\gamma}
\newcommand{\eps}{\epsilon}
%\newcommand{\Cal}{\tex{Cal}}
\newcommand{\De}{\Delta}
\newcommand{\de}{\delta}

\newcommand{\oa}{\overline{a}}
\newcommand{\ob}{\overline{b}}
\newcommand{\oc}{\overline{c}}

\newcommand{\cA}{\mathcal{A}}
\newcommand{\cB}{\mathcal{B}}
\newcommand{\cC}{\mathcal{C}}
\newcommand{\cD}{\mathcal{D}}
\newcommand{\cE}{\mathcal{E}}
\newcommand{\cF}{\mathcal{F}}
\newcommand{\cG}{\mathcal{G}}
\newcommand{\cH}{\mathcal{H}}
\newcommand{\cI}{\mathcal{I}}
\newcommand{\cJ}{\mathcal{J}}
\newcommand{\cK}{\mathcal{K}}
\newcommand{\cO}{\mathcal{O}}


\newcommand{\cU}{\mathcal{U}}
\newcommand{\cP}{\mathcal{P}}
\newcommand{\cR}{\mathcal{R}}
\newcommand{\cS}{\mathcal{S}}
\newcommand{\cM}{\mathcal{M}}
\newcommand{\cN}{\mathcal{N}}

\newcommand{\fS}{\mathfrak{S}}
\newcommand{\fk}{\mathfrak{k}}
\newcommand{\fg}{\mathfrak{g}}
\newcommand{\fz}{\mathfrak{z}}
\newcommand{\fZ}{\mathfrak{Z}}

\newcommand{\rJ}{\mathrm{J}}
\newcommand{\rB}{\mathrm{B}}
\newcommand{\rL}{\mathrm{L}}
\newcommand{\rP}{\mathrm{P}}
\newcommand{\rR}{\mathrm{R}}
\newcommand{\rT}{\mathrm{T}}
\newcommand{\rS}{\mathrm{S}}

\newcommand{\LH}{H,L}

\newcommand{\tmin}{{\text{min},\bK}}
\newcommand{\tmon}{{\text{mon},\bK}}
\newcommand{\tuniv}{{\text{univ},\bK}}

\newcommand{\Foabc}{{\mathrm{Free}\langle \oa,\ob, \oc \rangle}}

\newcommand{\bbP}{\mathbb{P}}

\newcommand{\CP}{{\C P^n, \om_{FS}}}
\newcommand{\RP}{\mathbb{R}\mathbb{P}}
\newcommand{\tHam}{\widetilde{\text{Ham}}}
\newcommand{\Id}{{\mathbbm{1}}}

\renewcommand{\bar}[1]{\overline{#1}}
\newcommand{\ol}[1]{\overline{#1}}
\renewcommand{\hat}[1]{\widehat{#1}}

\def\Sign{\operatorname{\textbf{Sign}}}

\DeclareMathOperator{\id}{\mathrm{id}}
\DeclareMathOperator{\mVol}{\mathrm{Vol}(M_0,\omega_0)}
\DeclareMathOperator{\Lie}{\mathrm{Lie}}
\DeclareMathOperator{\op}{\mathrm{op}}

%\DeclareMathOperator{\mmu}{\mathrm{\mu}}

\DeclareMathOperator{\ind}{\mathrm{ind}}

\renewcommand{\null}{\mathrm{null}}

\DeclareMathOperator{\supp}{\mathrm{supp}}
\DeclareMathOperator{\rank}{\mathrm{rank}}
\DeclareMathOperator{\codeg}{\mathrm{codeg}}
\DeclareMathOperator{\trace}{\mathrm{trace}}

\DeclareMathOperator{\val}{\mathrm{val}}
\DeclareMathOperator{\Sym}{\mathrm{Sym}}
\DeclareMathOperator{\Diff}{\mathrm{Diff}}
\DeclareMathOperator{\Ham}{\mathrm{Ham}}
\DeclareMathOperator{\Cont}{\mathrm{Cont}}
\DeclareMathOperator{\Hom}{\mathrm{Hom}}
\DeclareMathOperator{\Symp}{\mathrm{Symp}}
\DeclareMathOperator{\Ker}{\mathrm{Ker}}
\DeclareMathOperator{\coker}{\mathrm{coker}}

\let\Im\relax
\DeclareMathOperator{\Im}{\mathrm{Im}}
\let\Re\relax
\DeclareMathOperator{\Re}{\mathrm{Re}}
\DeclareMathOperator{\Cal}{\mathrm{Cal}}
\DeclareMathOperator{\Osc}{\mathrm{Osc}}
\DeclareMathOperator{\osc}{\mathrm{osc}}
\DeclareMathOperator{\ima}{\mathrm{im}}
\DeclareMathOperator{\Max}{\mathrm{Max}}
\DeclareMathOperator{\Min}{\mathrm{Min}}
\DeclareMathOperator{\Spec}{\mathrm{Spec}}
\DeclareMathOperator{\Skel}{{\mathrm{Skel}}}
\DeclareMathOperator{\conjugation}{\mathrm{conj}}
\DeclareMathOperator{\pemod}{{\mathbf{pmod}}}
\DeclareMathOperator{\barc}{{\mathbf{barcodes}}}
\DeclareMathOperator{\inte}{{\mathrm{int}}}
\DeclareMathOperator{\diag}{{\mathrm{diag}}}
\DeclareMathOperator{\spec}{{\mrm{spec}}}
\DeclareMathOperator{\tot}{{\mrm{tot}}}
\DeclareMathOperator{\loc}{{\mrm{loc}}}


%\DeclareMathSymbol{\mu}{\mathalpha}{operators}{0}


\def\o{\omega}
\def\Hk{H^{(k)}}
\def\Fk{F^{(k)}}
\def\Gk{G^{(k)}}
\def\H2{H^{(2)}}
\def\F{\mathbb F}
\def\T{\mathbb T}

\newcommand{\Propertieslabel}[1]{%
	\mbox{\textsc{#1:}}$\;$}

\newenvironment{Properties}
{\begin{list}{}{
			\renewcommand{\makelabel}{\Propertieslabel}%
			\setlength{\topsep}{6pt}%
			\setlength{\itemsep}{4pt}%
			\setlength{\labelsep}{0pt}%
			\setlength{\leftmargin}{0pt}%
			\setlength{\labelwidth}{0pt}%
			\setlength{\listparindent}{0pt}}%
		\setlength{\parskip}{0pt}%
	}
	{\end{list}}

%breaklinks,colorlinks,citecolor=teal,linkcolor=teal,urlcolor=teal,
%\usepackage[pagebackref,hyperindex]{hyperref}

\usepackage[hyperindex]{hyperref}

%\renewcommand\backrefxxx[3]{%
%	\hyperlink{page.#1}{p.#1}%
%}


%%opening
%\title{}
%\author{}

%\newcommand{\akemail}{asafkisl@post.tau.ac.il}
\newcommand{\esemail}{egor.shelukhin@umontreal.ca}

\newcommand{\jznote}[1]{{\textbf{{\color{red} #1}}}}
\newcommand{\esnote}[1]{{\textbf{{\color{blue} #1}}}}

%\newcommand{\zjnote}[1]{#1}



\begin{document}
	
\title[On non-contractible closed geodesics and homotopy groups]{Remark on non-contractible closed geodesics \\ and homotopy groups}

\author{Egor Shelukhin}
\email{\esemail}
\address{Department of Mathematics and Statistics, University of Montreal, C.P. 6128 Succ.  Centre-Ville Montreal, QC H3C 3J7, Canada}


\author{Jun Zhang}
\email{jzhang4518@ustc.edu.cn}
\address{The Institute of Geometry and Physics, University of Science and Technology of China, 96 Jinzhai Road, Hefei Anhui, 230026, China}



\bibliographystyle{abbrv}


\begin{abstract}

We prove that if the $m$-th homotopy group for $m \geq 2$ of a closed manifold has non-trivial invariants or coinvariants under the action of the fundamental group, then there exist infinitely many geometrically distinct closed geodesics for a $C^4$-generic Riemannian metric. If moreover there are infinitely many conjugacy classes in the fundamental group, then the same holds for every Riemannian metric.

%We deduce a new generic existence result of infinitely many closed geodesics.
%We prove that on every closed non-aspherical manifold    Finally, if the $m$-th homotopy group for $m \geq 2$ has non-trivial invariants or coinvariants under the action of the fundamental group, and there , then the same conclusion holds. 

%These results extend, and are inspired by, the work of Bangert, Hingston, and Taimanov.

%with a non-trivial second Hurewicz group and infinitely many free homotopy classes of loops must have infinitely many geometrically distinct closed geodesics. We deduce the existence of infinitely many geometrically distinct closed geodesics for a generic Riemannian metric on any closed manifold whose second Hurewicz group is non-trivial. These results generalize to Finsler metrics. Our main technique involves homology with coefficients in a local system over the free loop space.

%%We prove that there exist infinitely many geometrically distinct closed Reeb orbits on a new class of contact manifolds. These manifolds are contact boundaries of Liouville domains with vanishing symplectic cohomology twisted by a suitable local system on the free loop space and non-vanishing symplectic cohomology in a suitable free homotopy class of infinite order. For example, we prove that for every closed manifold whose second Hurewicz group does not vanish, and whose fundamental group admits a non-trivial homogeneous quasi-morphism, the Reeb flow of any contact form on the cotangent sphere bundle admits infinitely many geometrically distinct closed orbits. Our methods involve Smith theory in filtered Floer cohomology with the key new ingredient of considering coefficients in local systems on the loop space. 
\end{abstract}
	


%\subjclass[2010]{53D12, 53D40, 37J05}

\maketitle

\section{Introduction and main results}

\subsection{Introduction}

The question of the existence of infinitely many geometrically distinct closed (periodic) geodesics on Riemannian or Finsler manifolds was studied extensively in the past. This question seeks to determine whether or not for every metric or for a class of metrics on a given manifold $M$ there exist infinitely many {\em prime} closed geodesics: those closed geodesics that are not obtained as iterations of others. 

The efforts were primarily focused on the simply-connected case, where by \cite{GromollMeyer, VPSullivan, JM16} a manifold with finitely many prime closed geodesics must have $\bK$-cohomology, for $\bK = \Q$ or $\bK = \F_p,$ generated as a unital $\bK$-algebra by a single element. Manifolds satisfying this condition for some $\bK$ do exist: notable examples are the compact rank-one symmetric spaces (CROSS): $S^n, \C P^n, \R P^n, \HH P^n, \bb OP^2$. For such spaces it is known \cite{Rad94} that a $C^2$-generic metric must have infinitely many prime closed geodesics. Prior work \cite{KT72,Hi84} proves the same for $C^4$-generic metrics: \cite{KT72} implies that $C^4$-generically on any closed manifold there are infinitely many prime closed geodesics, unless they are all hyperbolic, while \cite{Hi84} proves that for manifolds rationally homotopy equivalent to a CROSS other than $\R P^n,$ given that all prime closed geodesics are hyperbolic, there are infinitely many of them. The case of $\R P^n$ follows by a quick covering argument, as does the case of any closed manifold with finite fundamental group, or by the paper \cite{BTZ81} discussed below. A celebrated result \cite{Bangert93, Franks92, Hingston93} proves that every Riemannian metric on $S^2$ has infinitely many prime closed geodesics, and the same is true for $\R P^2$. The analogous statement for any other CROSS is a well-known open question.

The case of infinite fundamental groups at first appears to be simpler, as by a classical theorem often attributed to Cartan or to Hilbert (see for example \cite[Chapter 12, Theorem 2.2]{DC92-book}), in every non-trivial free homotopy class of loops on $M$ there exists a closed geodesic. However, firstly, it is an open question in group theory \cite{Makowsky, BMS01} whether an infinite finitely presented group could have finitely many conjugacy classes. We recall that the conjugacy classes in $\pi_1(M)$ correspond to free homotopy classes of loops on $M$. Secondly, even if we assume that there are infinitely many conjugacy classes, the simplification appears to be quite illusory, starting with the fact that non-homotopic closed geodesics might not be geometrically distinct (see also \cite{Taim10}). Existence of infinitely many prime closed geodesics is currently known under stronger conditions involving the fundamental group: for example \cite{BH84} if the manifold is not the circle and its fundamental group is $\Z$ (see \cite{Tai85,Tai93, RT22} for other conditions of this kind and \cite{Gro2000, Bal86} for prior work). Generic existence (in $C^4$ topology) of infinitely many prime closed geodesics was proved under special conditions on $\pi_1(M)$ in \cite{BTZ81}: for instance this covers the case where $\pi_1(M)$ has finitely many conjugacy classes, yet is not simply-connected. The recent work \cite{RT22} proves $C^4$-generic existence of infinitely many prime closed geodesics on every closed $3$-manifold, and produces further results about closed geodesics on connected sums of closed manifolds (see also \cite{PP05,La01}).

Our main result shows that $C^4$-generic existence of infinitely many prime closed geodesics holds on closed manifolds, of any dimension, under a topological condition regarding the action of $\pi_1(M)$ on $\pi_m(M)$ for $m \geq 2,$ in the spirit of  Bangert-Hingston \cite{BH84}, Albers-Frauenfelder-Oancea \cite{AFO17} and Taimanov \cite{Tai85}, which have inspired this paper. Namely, we assume that this action for some $m \geq 2$ has either non-trivial invariants or non-trivial coinvariants. We also prove that if in addition there are infinitely many free homotopy classes of loops on $M,$ then there exist infinitely many prime closed geodesics for every Riemannian metric on $M.$

%We also prove that if the action of $\pi_1(M)$ on $\pi_m(M)$ for some $m \geq 2$ , then there exist infinitely many prime closed geodesics for every metric on $M,$ if there are infinitely many free homotopy classes of loops on $M.$


%For example the main result of \cite{BH84} follows as any manifold with fundamental group $\Z$ is either non-aspherical or is diffeomorphic to $S^1.$ If in addition, there is a free homotopy class of loops whose powers are all distinct, we prove that the existence infinitely many prime closed geodesics holds for every metric. 

%Our arguments are inspired by the work of Bangert-Hingston \cite{BH84} and Taimanov \cite{Tai85}. The main novelty of our approach is the use of localization to reduce algebraic considerations to a finitely generated module in a relevant setting, see Claim \ref{clm: localize}.

%An easy covering argument shows that the property of having infinitely many prime closed geodesics is invariant under finite covers. 

\subsection{Main results}

%\begin{thm}\label{thm: main}
%Let $M$ be a closed manifold with $\pi_m(M) \neq 0$ for some $m \geq 2.$ Suppose that there exists a conjugacy class $[a]$ for $a \in \pi_1(M)$  such that all the conjugacy classes $[a^k],$ $k \geq 1$ are distinct. Then for every Riemannian metric on $M$ there exist infinitely many geometrically distinct closed geodesics.
%\end{thm}

%\begin{rmk}\label{rmk: centralizer}
%An extra condition ensuring the conclusion of the theorem, which is specific to the class $a$ appearing in its formulation, is that the centralizer $C_{a^{r}}$ of $a^{r}$ in $\pi_1(M)$ for some $r \geq 1$ is not isomorphic to $\Z.$
%\end{rmk}


\begin{thm}\label{thm: inv coinv}
Let $M$ be a closed manifold. Suppose that there exists $x \in \pi_m(M),$ $x \neq 0,$ for $m\geq 2$ such that $a \cdot x = x$ for all $a \in \pi_1(M)$ or a homomorphism $\xi: \pi_m(M) \to A,$ $\xi \neq 0$ to an abelian group $A$ such that $a^* \xi = \xi$ for all $a \in \pi_1(M).$ Moreover, let the set $\pi_1(M)/\mrm{conj}$ of conjugacy classes in $\pi_1(M)$ be infinite.
Then for every Riemannian metric on $M$ there exist infinitely many geometrically distinct closed geodesics.\end{thm}

\begin{rmk}\label{rmk: fund gp loc sys}
We remark that our condition on the higher homotopy groups does not constrain the fundamental group of the manifold. For instance, taking a product with the two-sphere does not change the fundamental group but ensures that this condition holds (see also Remark \ref{rmk: constructions}). The condition on coinvariants for $m=2$ and $A = GL(1, \F_p)$ for a finite field $\F_p$ is satisfied if and only if there exists a non-trivial rank-one $\F_p$-local system on $\cl LM$ with a trivial restriction to $M$ via the constant loop embedding $M \to \cl LM.$ We refer to \cite[Section 2.2]{AFO17} for a related discussion and \cite[Proposition 9]{AFO17}, showing that this condition, for a suitable prime $p,$ is in turn implied by the non-triviality of the image $H_2^S(M;\Z)$ of the second Hurewicz map $\pi_2(M) \to H_2(M;\Z).$ 
\end{rmk}

Recall that the natural map $\pi_1(M) \to \pi_0(\cl LM)$ from the fundamental group of $M$ to the set  $\pi_0(\cl LM)$ of connected components of the free loop space $\cl LM$ of $M$ induces a canonical isomorphism of sets \[\pi_1(M)/\mrm{conj} \xrightarrow{\cong} \pi_0(\cl LM),\] where $\pi_1(M)/\mrm{conj}$ is the set of conjugacy classes in $\pi_1(M).$ Hence requiring that the set $\pi_1(M)/\mrm{conj}$ is infinite is the least possible assumption on the richness of the fundamental group that would help establishing the existence of infinitely many geometrically distinct closed geodesics. In fact, a well-known open question in group theory, due to Makowsky \cite{Makowsky} (see also \cite{BMS01}), is whether every infinite finitely-presented group has infinitely many conjugacy classes. Note that the assumption on the conjugacy classes yields infinitely many non-homotopic closed geodesics, as by Cartan and Hilbert's theorem, there is a closed geodesic in every non-trivial free homotopy class $\al$ of loops on $M.$ (It is given as point of global minimum of energy in the corresponding connected component $\cl L_{\al}M$ of $\cl LM$.) However, these geodesics might easily not all be {\em geometrically distinct} since a geodesic $c$ and its $k$-th iterate $c^k,$ given by $c^k(t) = c(kt),$ could contribute to different free homotopy classes. In general, the free homotopy class $[c^k] \in \pi_0(\cl LM)$ of the iterate $c^k$ might depend on $k$ in a complicated way.


\begin{cor}\label{cor: generic}
Let $M$ be a closed manifold. Suppose that there exists $x \in \pi_m(M),$ $x \neq 0,$ for $m\geq 2$ such that $a \cdot x = x$ for all $a \in \pi_1(M)$ or a homomorphism $\xi: \pi_m(M) \to A,$ $\xi \neq 0$ to an abelian group $A$ such that $a^* \xi = \xi$ for all $a \in \pi_1(M).$ Then for a $C^4$-generic Riemannian metric on $M$ there exist infinitely many geometrically distinct closed geodesics.
\end{cor}

We remark that in this corollary we make no assumption on conjugacy classes in the fundamental group. It appears to be new even for manifolds of the form $M=X \times S^2,$ where $X$ is {\it an arbitrary} closed manifold (see also Remark \ref{rmk: constructions}). Of course $X$ for which prior methods do not apply are more restricted; for instance there should be an element $\al \in \pi_1(X)/\mrm{conj}$ such that $\al^k,$ $k \geq 1,$ are all distinct. %It also proves that a $C^4$-generic Riemannian metric on every connected sum $M=X \# Y$ for $X, Y$ not homotopy equivalent to spheres has infinitely many geometrically distinct closed geodesics.  Indeed, such connected sums are never aspherical \cite[Lemma 3.2]{Luck}. This covers some new cases of this statement, which was extensively studied in \cite{RT22, PP05, La01}.

%We prove a slight generalization of Theorem \ref{thm: main} and Corollary \ref{cor: generic} as Theorem \ref{thm: main 2} below by a somewhat more involved argument related to local systems (see Proposition \ref{prop: vanishing H2inv}). 

\begin{proof}[Proof of Corollary \ref{cor: generic}]
If $M$ is simply connected, the conclusion is well-known by the work of Klingenberg-Takens \cite{KT72}, Hingston \cite{Hi84}, and Rademacher \cite{Rad94}. If $M$ is not simply connected, consider a non-trivial class $a \in \pi_1(M).$ Then either there exist two equal conjugacy classes $[a^k]=[a^l]$ for some positive integers $0<k<l,$ in which case the conclusion follows by work of Ballmann-Thorbergsson-Ziller \cite[Theorem A]{BTZ81}, or the conjugacy classes $[a^k],$ for $k$ a positive integer, are all distinct, in which case the conclusion follows by Theorem \ref{thm: inv coinv}.
\end{proof}

%One can also prove the following result, which requires a stronger assumption on the homotopy groups, but does not require a conjugacy class $[a]$ with all iterations $[a^k]$ different. 


%In a similar direction, to generalize Corollary \ref{cor: generic} (see also Remark \ref{rmk: centralizer}) to an arbitrary closed Riemannian manifold, it would be sufficient to answer the following group-theoretic question negatively. 

%\begin{qtn}
%Let $G$ be the fundamental group of a closed aspherical manifold. Suppose that for every non-trivial element $g$ of $G,$ the conjugacy classes of $g^k,$ for $k \in \Z$ are all distinct, and the centralizer of $g$ is isomorphic to $\Z.$ Could there exist finitely many elements $\{g_1,\ldots, g_N\}$ in $G$ such that every element $g$ in $G$ is conjugate to some $g_i^m$ for $1 \leq i \leq N$ and $m \in \Z$?
%\end{qtn}

%The method of proof of Theorem \ref{thm: main} also allows us to prove the following different result, which appears to be new.

%\begin{thm}\label{thm: spinoff}
%Let $M$ be a closed manifold with infinitely many conjugacy classes in the fundamental group. 
%Then for every Riemannian metric on $M$ with an isometry $f$ without fixed points acting trivially on the set of conjugacy classes in the fundamental group there exist infinitely many geometrically distinct closed geodesics.
%\end{thm}

%The hypothesis on $f$ is satisfied if it is isotopic to the identity as a diffeomorphism. In this case the manifold $M$ would have vanishing Euler characteristic.


\begin{rmk}
By the arguments proving \cite[Corollary B]{Contreras}, one can upgrade Corollary \ref{cor: generic} to the stronger conclusion that for $C^4$-generic metrics, the number of prime periodic orbits of length at most $T$ grows exponentially in $T.$ (Note that while this is not stated explicitly, \cite[Corollary B]{Contreras} is proven therein for simply-connected manifolds.)
\end{rmk}

\begin{rmk}
It is not hard to see that Theorem \ref{thm: inv coinv} and Corollary \ref{cor: generic} extend to Finsler metrics (the latter in the $C^6$-generic case), by a finite-dimensional approximation approach (see \cite{Rad92-book} as well as \cite{RT22, RT22b}). In forthcoming work \cite{SSZ} we will show that an analogue of Corollary \ref{cor: generic} holds also for Reeb flows of arbitrary contact forms on the unit cotangent bundle $S^*M,$ by proving a weaker analogue of Theorem \ref{thm: inv coinv}. In future work we hope to prove a more direct analogue of Theorem \ref{thm: inv coinv} in the context of Reeb flows.  
\end{rmk}

%given that the second Hurewicz homomorphism $\pi_2(M) \to H_2(M;\Z)$ is non-trivial. This involves an analogue of Theorem \ref{thm: main} in the context of Reeb dynamics under this more restrictive condition on the second Hurewicz homomorphism. 

\begin{rmk}\label{rmk: constructions}

It is easy to see that the family $\cl N$ of manifolds satisfying the condition of Corollary \ref{cor: generic} is an ideal in the family $\cl M$ of all closed manifolds with respect to the Cartesian product. In certain situations, this property is also preserved under connected sums. For instance, consider the family $\cl H_2 \subset \cl M$ of all closed manifolds $M$ satisfying $H_2^S(M;\Z) \neq 0$ (see Remark \ref{rmk: fund gp loc sys}). Set $\cl H^{\geq m}_2 = \cl H_2 \cap \cl M^{\geq m},$ where $\cl M^{\geq m} \subset \cl M$ is the subfamily of manifolds of dimension at least $m.$  Let $m=3.$ Then $\cl H^{\geq 3}_2$ is an ideal in $\cl M^{\geq 3}$ with respect to connect sum: if $X \in \cl M^{\geq 3}$ and $Y \in \cl H^{\geq 3}_2$ then $X \# Y \in \cl H^{\geq 3}_2.$ This is a consequence of the fact that for manifolds $X,Y \in \cl M^{\geq 3},$ by the homotopy exact sequence of a pair, $\pi_2(X \# Y) \cong \pi_2(X) \oplus \pi_2(Y),$ and the Mayer-Vietoris sequence for the natural covering $X' \cup Y' = X \# Y,$ where $X' \cong X \setminus B_X, Y' \cong Y \setminus B_Y$ for closed balls $B_X \subset X, B_Y \subset Y.$ In all dimensions $n$ other than $n \in \{0, 1, 3\}$ there exist simply-connected manifolds $X^n \in \cl H_2$ of dimension $n.$ Indeed, for $n=2,$ $X^2=S^2$ and for $n\geq 4,$ we can take $X^n = S^2 \times S^{n-2}.$ Combining with the previous point, we see by the Van Kampen theorem, that for every manifold $M \in \cl M^{\geq 4}$ there exists a manifold $M' \in \cl H^{\geq 4}_2$ with $\pi_1(M') \cong \pi_1(M).$ We can take $M' = M \# X^n.$ Similarly, for all $M \in \cl M,$ $M'' = M \times X^n \in \cl H_2$ satisfies $\pi_1(M'') \cong \pi_1(M).$

\end{rmk}

\begin{rmk}
One could imagine a generalization of Corollary \ref{cor: generic} which would hold for all closed non-aspherical manifolds, and a suitable generalization of Theorem \ref{thm: inv coinv}. However, the complicated algebraic properties of the group rings $\Z[\pi_1(M)],$ which are, for instance, often non-Noetherian \cite{Non-noetherian}, seem to be an obstruction to proving such results. 
\end{rmk}


%We explain how to construct many examples of manifolds in $\cl H_2.$


%Furthermore, the same is true for the subfamily $\cl N'$ of $\cl N$ of manifolds not homotopy-equivalent to a sphere with respect to the connected sum. If $X \in \cl N'$ is a simply-connected such manifold of dimension $n$, for instance $X = S^2 \times S^{n-2}$ for $n\geq 4,$ then the connected sum $M \# X$ of a manifold $M \in \cl M$ of dimension $n$ with $X$ is in $\cl N'$ but has the same fundamental group as $M.$

%Let $\cl M$ denote the family of all closed manifolds. Consider the family $\cl H_2 \subset \cl M$ of all closed manifolds $M$ satisfying $H_2^S(M;\Z) \neq 0.$ We explain how to construct many examples of manifolds in $\cl H_2.$

%\begin{enumerate}

%\item One can use for instance the following two observations. First, by the K\"{u}nneth theorem, $\cl H_2$ is an ideal with respect to the Cartesian product: for every $X \in \cl M$ and $Y \in \cl H_2,$ the direct product $X \times Y \in \cl H_2.$ Second, suppose now that we restrict to manifolds $\cl M^{\geq m} \subset \cl M$ of dimension at least $m.$ Set $\cl H^{\geq m}_2 = \cl H_2 \cap \cl M^{\geq m}.$ Let $m=3.$ Then $\cl H^{\geq 3}_2$ is an ideal in $\cl M^{\geq 3}$ with respect to connect sum: if $X \in \cl M^{\geq 3}$ and $Y \in \cl H^{\geq 3}_2$ then $X \# Y \in \cl H^{\geq 3}.$ This is a consequence of the fact that for manifolds $X,Y \in \cl M^{\geq 3},$ by the homotopy exact sequence of a pair, $\pi_2(X \# Y) \cong \pi_2(X) \oplus \pi_2(Y),$ and the Mayer-Vietoris sequence for the natural covering $X' \cup Y' = X \# Y,$ where $X' \cong X \setminus B_X, Y' \cong Y \setminus B_Y$ for closed balls $B_X \subset X, B_Y \subset Y.$ 

%\item In all dimensions $n$ other than $n \in \{0, 1, 3\}$ there exist simply-connected manifolds $X^n \in \cl H_2$ of dimension $n.$ Indeed, for $n=2,$ $X^2=S^2$ and for $n\geq 4,$ we can take $X^n = S^2 \times S^{n-2}.$ Combining with the previous point, we see by the Van Kampen theorem, that for every manifold $M \in \cl M^{\geq 4}$ there exists a manifold $M' \in \cl H^{\geq 4}_2$ with $\pi_1(M') \cong \pi_1(M).$ We can take $M' = M \# X^n.$ Similarly, for all $M \in \cl M,$ $M'' = M \times X^n \in \cl H_2$ satisfies $\pi_1(M'') \cong \pi_1(M).$

%\end{enumerate}

\section{Preliminaries}

\subsection{Average index and homotopy groups.}
Let $(M,g)$ be a Riemannian manifold of dimension $\dim M = n.$ Let $\ind(c)$ denote the index of a closed geodesic $c$ in $(M,g).$ The average index (or mean-index) of $c$ is defined as \[ \Delta(c) = \lim_{m\to \infty} \frac{\ind(c^m)}{m} \in \R.\] It satisfies the following useful properties (see \cite[Proposition 6.1, Equation (6.1.3)]{GH09} and \cite[Korollar 4.4]{Rad92-book} for instance):

\begin{enumerate}
\item\label{homogeneity} $\Delta(c^m) = m \Delta(c)$ for all $m \in \Z_{>0}$
\item\label{distance to ind} $|\Delta(c)-\ind(c)| \leq n-1,$ $|\Delta(c)-(\ind(c)+\null(c))| \leq n-1$
\item\label{vanishing} $\Delta(c) \geq 0$ and $\Delta(c) = 0$ if and only if $\ind(c^m) = 0$ for all $m \in \Z_{>0}.$
\end{enumerate}

%Let us define the local homotopy groups of a closed geodesic $c$ in the connected component $\mathcal L_{\alpha} M$ of the loop space as the relative homotopy \[\pi_*^{\loc}(c) = {\pi}_*(\La_{\al}^{A} \cup S(c), \La_{\al}^{A}),\] where $\La_{\al}^A = \{\cA < A\},$ $A=\cA(c)$ is the energy of $c$ and $S(c)$ is the circle given by $c$ and its rotations. 

Let $\La_{\al}$ be the $L^2_1$-completion of $\cl L_{\al} M$ and $\La_{\al}^A = \{\cA < A\}$ for $A>0.$ Note that the inclusion $\cl L^A_{\al} M \to \La^A_{\al} M$ is a homotopy equivalence for every $A>0$ by a result of Anosov \cite{Anosov} (see also \cite[Proposition 2.2]{GH09}). 

%We require the following properties of $\pi_*^{\loc}(c),$ which follow from \cite{GromollMeyer}.

%\begin{prop}\label{prop local homology: support}
%The local homotopy $\pi_*^{\loc}(c)$ of a closed geodesic $c$ is supported in the interval $[\ind(c),+\infty)$ of degrees. That is, $\pi_k^{\loc}(c) = 0$ for $k \geq 1$ outside this interval.
%\end{prop}

%In particular, in view of Property \ref{distance to ind} of the mean-index, we obtain the following.

%\begin{cor}\label{prop local homology: mean support}
%The local homotopy $\pi_*^{\loc}(c)$ of a closed geodesic $c$ is supported in the interval $[\Delta(c)-(n-1),\infty).$
%\end{cor}

We require the following property of the homotopy groups of $\La_{\al} M,$ which follows from \cite{GromollMeyer} and the properties of the mean-index (see also \cite{Chang}). %and Morse theory for the energy functional on $\La_{\al} M,$ and the homotopy long exact sequence of a pair we obtain the following.

\begin{prop}\label{prop: Morse}
If all geodesics $b \in \La_{\al} M$ with $\cA(b) \geq A$ are of mean-index $\Delta(b)>k+n-1,$ then $\pi_m(\La_{\al}) = \pi_m(\La_{\al}^A)$ for all $m \leq k.$
\end{prop}

We recall the following long exact sequence from \cite{Tai85}. Fix $\al \in \pi_0(\cl L M).$ Consider the evaluation map $ev: \La_{\al} M \to M,$ $z \mapsto z(0),$ where $z \in \La_{\al} M$ is considered as an absolutely continuous map $z:\R/\Z \to M$ with square-integrable velocity vector. This map is a Serre fibration. Fix base-points $x_0 \in M$ and $\gamma_0 \in ev^{-1}(x_0).$ Let $a \in \pi_1(M,x_0)$ be the class represented by $\gamma_0.$ Of course $[a] = \al$ in $\pi_1(M)/\mrm{conj}.$ 


\begin{prop}\label{prop: les}
The long exact sequence of a fibration for $ev$ and an identification $\pi_k(ev^{-1}(x_0),\gamma_0) \cong \pi_{k+1}(M,x_0)$ yield the long exact sequence: \[ \ldots \to \pi_{m+1}(M,x_0) \xrightarrow{1-a} \pi_{m+1}(M,x_0) \to \pi_m(\La_{\al} M, \gamma_0) \to  \pi_{m}(M,x_0) \xrightarrow{1-a} \pi_{m}(M,x_0) \to \ldots,\] for $m \geq 2,$ where $a: \pi_k(M,x_0) \to \pi_k(M,x_0)$ is the standard action of the fundamental group on higher homotopy groups. For $m=1$ the sequence reads: \[ \ldots \to \pi_{2}(M,x_0) \xrightarrow{1-a} \pi_{2}(M,x_0) \to \pi_1(\La_{\al} M, \gamma_0) \to  C_a \to 1,\] where $C_a \subset \pi_1(M,x_0)$ is the centralizer of $a.$
\end{prop}
 
%We will drop the ground field from notation when it is clear from the context.
%Given a choice of coefficient field $\bK,$ the local homology of a closed geodesic $c$ is defined as the relative homology \[ H_*^{\loc}(c; \bK) = H_*(\La^{A} \cup S(c), \La^{A};\bK),\] 


%The local homologies of closed geodesics constitute building blocks for the loop space homology $H_*(\mathcal LM; \bK).$ Indeed, there is a spectral sequence associated to the filtration of $\cl LM$ by energy, which converges to $H_*(\mathcal LM; \bK)$ and whose $E^2$-page is the direct sum $C$ of local homologies over all the closed geodesics $c.$ This also implies (see \cite[Appendix]{GU19} for instance) that there is a differential $d$ of degree $(-1)$ on the $E^2$-page $C$ such that the homology of $(C,d)$ is $H(C,d) \cong  H_*(\mathcal LM; \bK)$ and for $x \in H_*^{\loc}(c)$ the projection of $dx$ to $H_*^{\loc}(c')$ vanishes whenever $\cA(c') \geq \cA(c).$ We say that the complex $(C,d)$ is {\em strictly filtered} by $\cl A.$ Note that a closed geodesic $c$ is involved in the complex $(C,d)$ if and only if $H_*^{\loc}(c; \bK) \neq 0,$ in which case we will say that $c$ is {\em homologically visible}.

%It is easy to see that calculating homology with coefficients in a rank-one $\bK$-local system $\cl V$ over $\cl LM$ we have an isomorphism of local homologies $H_*^{\loc}(c; \cl V) \cong H_*^{\loc}(c; \cl V)$ and the above spectral sequence argument applies again to produce a differential $d_{\cl V}$ of degree $(-1)$ on $C,$ such that the complex $(C, d_{\cl V})$ is strictly filtered by $\cl{A}$ and its homology is $H(C,d_{\cl{V}}) \cong  H_*(\mathcal LM; \cl{V}).$

%Clearly, the same discussion applies when restricted to a connected component $\mathcal L_{\alpha} M$. This has the following consequences which will be of use to us later.


%\begin{cor}\label{cor: absolute minimum}
%Consider a connected component $\cl L_{\al} M$ for $\al \in \pi_0(\cl LM).$ Then there exists a closed geodesic $c$ realizing the global minimum of energy in $\cl L_{\al} M$ such that $H^{\loc}_0(c;\bK) \neq 0.$
%\end{cor}
%
%Corollary \ref{cor: absolute minimum} is well-known, since the work of Gromoll-Meyer \cite{GromollMeyer}. In fact, {\em every} closed geodesic $c$ realizing the global minimum of energy in $\cl L_{\al} M$ has $H^{\loc}_0(c;\bK) \neq 0.$ However, the following result was not used to study closed geodesics, to the best of our knowledge.
%
%\begin{cor}\label{cor: non-vanishing}
%If the homology $H_*(\cl L_{\al} M; \cl V_{\al})$ of a connected component $\cl L_{\al} M$ with coefficients in a rank-one $\bK$-local system $\cl V_{\al}$ vanishes, then there exists a closed geodesic $b \in L_{\al} M$ that is not an absolute minimum of energy in $\cl L_{\al} M$ and satisfies $H_1^{\loc}(b) \neq 0.$ 
%\end{cor}
%
%\begin{rmk} The above discussion can be conveniently interpreted in the language of persistence modules. However, we omit this, as it is not necessary for the current paper. \end{rmk}
%
%It turns out that $H_k^{\loc}(c; \bK)$ vanishes for all but a finite number of degrees $k.$ More precisely, we have the following.
%
%\begin{prop}\label{prop local homology: support}
%The local homology $H_*^{\loc}(c)$ of a closed geodesic $c$ is supported in the interval $[\ind(c),\ind(c)+\null(c)+1]$ of degrees. That is, $H_k^{\loc}(c) = 0$ for $k$ outside this interval.
%\end{prop}
%
%In particular, in view of Property \ref{distance to ind} of the mean-index, we obtain the following.
%
%\begin{cor}\label{prop local homology: mean support}
%The local homology $H_*^{\loc}(c)$ of a closed geodesic $c$ is supported in the interval $[\Delta(c)-(n-1),\Delta(c)+n].$
%\end{cor}



\section{Proof of main result}

%We require the following lemma.
%
%\begin{lma}\label{lma: powers} Let $V$ be an abelian group, and $b: V \to V$ be a homomorphism.\\
%If $\ker(1-b: V \to V) \neq 0$ then $\ker(1-b^q: V \to V) \neq 0$ for all $q \geq 1.$ \\
%If $\coker(1-b: V \to V) \neq 0$ then $\coker(1-b^q: V \to V) \neq 0$ for all $q \geq 1.$ 
%\end{lma}
%
%\begin{proof}[Proof of Lemma \ref{lma: powers}]
%Indeed, as \[1-b^q = (1-b)(1+\ldots+b^{q-1}) = (1+\ldots+b^{q-1})(1-b),\]  $\ker(1-b: V \to V) \subset \ker(1-b^q: V \to V)$ and $\ima(1-b^q: V \to V) \subset \ima(1-b: V \to V).$ The latter implies that there is a natural surjection  \[\coker(1-b^q: V \to V) \to \coker(1-b: V \to V).\]
%\end{proof}

\begin{proof}[Proof of Theorem \ref{thm: inv coinv}]
%Let $V=\pi_m(M),$ $m>1$ be a non-trivial higher homotopy group of $M.$ Let us choose $m$ to be minimal with this property. Our proof relies on the following claim.

%\begin{clm}\label{clm: localize} Under the conditions of the theorem, there exists $k \geq 1$ such that the map $1-a^k: V \to V$ is not an automorphism, that is: either $\ker(1-a^k: V \to V) \neq 0$ or $\coker(1-a^k: V \to V) \neq 0.$ \end{clm} 

%\begin{clm}\label{clm: localize} Under the conditions of the theorem, there exists $k \geq 1$ such that the map $1-a^k: V \to V$ is not surjective, that is: $\coker(1-a^k: V \to V) \neq 0.$ \end{clm} 

%Then setting $b=a^k$ by Lemma \ref{lma: powers}, respectively, $\ker(1-b^{q}: V \to V) \neq 0$ or $\coker(1-b^{q}: V \to V) \neq 0$ for all $q \geq 1.$ 
By Proposition \ref{prop: les}, the hypothesis of the theorem implies that $\pi_{\ell}(\La_{\alpha} M) \neq 0$ for $\ell=m$ or $\ell=m-1$ and all $\alpha \in \pi_0(\cl LM).$ %We postpone the proof of this claim to the end of the proof and set $\ell = m-1.$ %to be $m$ or $m-1$ accordingly. %, respectively, $\pi_{m}(\La_{\beta^q} M) \neq 0$ or

Suppose by contradiction that there are finitely many prime closed geodesics. (In particular, every closed geodesic is isolated.) We denote them by $\gamma_1,...,\gamma_P, \gamma_{P+1},...,\gamma_N,$ such that the geodesics $\gamma_1,...,\gamma_P$ have mean-index $\Delta(\gamma_i) > 0$ and $\gamma_i$ for $i>P$ have $\Delta(\gamma_i) = 0.$  Then by Property \eqref{vanishing} of the mean-index, for every $i>P,$ $\ind(\gamma_i^j) = 0$ for all iterations $j.$  Now all closed geodesics are iterations of $\gamma_1, \ldots, \gamma_N$. Consider the set $S$ of all the conjugacy classes $[\gamma_i^j],$ $1 \leq i \leq P$ such that $\Delta(\gamma_i^j) \leq n-1+\ell.$ Since the mean-indices of $\gamma_i$ for $1 \leq i \leq P$ are positive, by Property \eqref{homogeneity} of the mean-index, $S$ is a finite set. Since $\pi_0(\cl LM)$ is an infinite set, so is the complement $\pi_0(\cl LM) \setminus S.$ Consider $\alpha \in \pi_0(\cl LM) \setminus S.$ Now let $c_1, \ldots, c_r \in \La_{\alpha} M$ be all the points of absolute minimum of the energy functional on $\La_{\al} M$. Then the level set of this minimum is the disjoint union $ \sqcup_{j=1}^r S(c_j)),$ where $S(c_j)$ is the circle given by $c_j$ and its rotations, and therefore its $k$-th homotopy groups vanish for all $k>1.$ We now claim that there exists a geodesic $\eta \in \La_{\alpha} M$ that is not a global minimum and is not an iteration of any of $\gamma_1,...,\gamma_P.$ Note that as $\eta$ must be an iteration of some $\gamma_i$ for $i>P$ we must have $\ind(\eta^r) = 0$ for all $r \geq 1.$ By Bangert-Klingenberg \cite[Theorem 3]{BK83}, this implies that there exist infinitely many prime closed geodesics on $M.$

To prove the last claim, we treat the cases $\ell \geq 2$ and $\ell = 1$ slightly differently. Suppose first that $\ell \geq 2.$ The contrapositive assumption that all closed geodesics above the minimum energy are iterations of $\gamma_1, \ldots, \gamma_P$, implies that  $\pi_\ell(\La_{\alpha} M) = 0$ by Proposition \ref{prop: Morse}, since by construction, any iteration $\eta$ of $\gamma_1, \ldots, \gamma_P$ in $\alpha$ has $\Delta(\eta)> n-1+\ell.$ But we know that $\pi_{\ell}(\La_{\alpha} M) \neq 0.$ In the case where $\ell =1,$ we observe that the coinvariant hypothesis and Proposition \ref{prop: les} yields that $\ker(ev: \pi_1(\La_{\alpha} M, \gamma_0) \to \pi_1(M,x_0)) \neq 0.$ Now, we argue that there can only be one point of global minimum under the contrapositive assumption. Indeed, had we $N>1$ global minima $c_1,\ldots,c_N$, the contrapositive assumption would yield that $\pi_0(\La_{\alpha} M)$ is a set of $N$ elements, which contradicts the connectedness of $\La_{\alpha} M.$ Hence there is a unique global minimum $c.$ Now the contrapositive assumption yields that $\pi_1(\La_{\alpha} M) \cong \pi_1(S(c)) \cong \Z$ and the evaluation map $\pi_1(S(c)) \to C_{a}$ is injective. This is a contradiction to its kernel being non-zero. This proves the claim for all $\ell \geq 1.$  \end{proof}

%We now prove Claim \ref{clm: localize} by means of non-commutative ring theory. First, note that as explained in \cite{BH84}, $V = \pi_m(M)$ is a finitely generated module over the group ring $R = \Z[\pi_1(M)]$ of the fundamental group. Now, by the non-commutative Nakayama lemma \cite[Theorem 13.11, p.183]{Isa09}, $J(R)V \subsetneq V$ is a proper submodule of $V,$ where $J(R)$ is the Jacobson radical of $R.$ Now, since $J(R)$ is the intersection of all maximal ideals of $R$ \cite[Corollary 13.3, p.180]{Isa09}, $(1-a^k)V=V$ implies that there exists a maximal ideal $I$ of $R$ such that $1-a^k \notin I.$ On the other hand, as $\pi_1(M)$ is finitely generated, by \cite[Theorem 3.3]{Pro67}, $R$ is a Jacobson algebra over $\Z,$ every maximal idea $I$ of $R$ intersects $\Z$ along a maximal ideal $(p) \subset \Z$ for a prime $p,$ and $R/I$ is a module of finite length over $\Z/I \cap \Z \cong \F_p.$


%Note that $a:V \to V$ endows $V$ with the structure of an $R$-module, where $R = \Z[t^{-1},t]$ is the ring of Laurent polynomials in a formal variable $t$ with integer coefficients. If $1-a^k: V \to V$ is invertible for all $k \geq 1,$ we can form the localization $R'=S^{-1}R$ by the multiplicative set multiplicatively generated by $\{1-t^k\}_{k \geq 1}$ and consider $V$ as an $R'$-module: indeed, $(1-t^k)^{-1} \in R'$ will act by the inverse homomorphism $(1-a^k)^{-1}:V \to V.$ Now let $x \in V$ be a non-zero element and set $W= R' \cdot x$ for the $R'$-submodule of $V$ generated by $x.$ It is non-zero and is obviously finitely generated over $R'.$ Moreover, $(1-a^k)W = W$ and $1-a^k:W \to W$ is invertible for all $k\geq 1.$ Now we apply an argument of \cite{BH84} to the ring $R'$ and the module $W.$ Namely, let $A' \subset R'$ be the annihilator of $W.$ It is a proper ideal of $R'$ and hence so is $I' \subset R'$, any maximal ideal containing $A'.$ Then by classification of maximal ideals in $\Z[t]$ it is easy to conclude that $R'/I'$ is a finite field $\F_{p^d}$ for some $d \geq 1.$ This means that $1-\ol{t}^{k_1} = 1-\ol{t}^{k_2}$ in $R'/I'$ for some $k_1<k_2,$ for $\ol{t}$ the image of $t$ in $R'/I'.$ Therefore $1-\ol{t}^{k_2-k_1} = 0$ in $R'/I'.$ On the other hand, $1-{t}^k$ is a unit in $R'$ for all $k$ and hence $1-\ol{t}^k$ is a unit in $R'/I'$ for all $k.$ This is a contradiction.

%by an argument involving Nakayama's lemma, $1-t^k$ is invertible in $R'/A'$ and hence in $R'/M'$

%
%\begin{rmk}\label{rmk: proof of rmk}
%To prove Remark \ref{rmk: centralizer} we note that by Proposition \ref{prop: les} for $m=1,$ the map $\pi_1(\La_{\beta^q} M) \to C_{b^q}$ is surjective and if we suppose that $\pi_m(M) = 0$ for all $m \geq 2,$ then this map is an isomorphism. Then the existence of $\eta$ as in the proof of Theorem \ref{thm: main} follows by the observation that if (under the contrapositive assumption) $c$ is the unique global minimum in $\La_{\beta^q} M$, then $C_{b^q} \cong \pi_1(\La_{\beta^q} M) \cong \pi_1(S(c)) \cong \Z.$ 
%\end{rmk}


%The proof of Theorem \ref{thm: inv coinv} is similar to that of Theorem \ref{thm: main}, but is easier in that its hypothesis together with Proposition \ref{prop: les} directly imply the requisite non-vanishing of $\pi_{\ell}(\La_{\al} M)$ for every $\al \in \pi_0(\La M)$ and some $\ell \geq 1$ independent of $\al.$ In particular, the key Claim \ref{clm: localize} is not necessary. We leave the details to the interested reader.


\section*{Acknowledgements}
We thank the participants of the reading group on closed geodesics at the Universit\'e de Montr\'eal, especially Marco Mazzucchelli, for useful discussions.
E.S. was supported by an NSERC Discovery grant, by the Fonds de recherche du Qu\'{e}bec - Nature et technologies, by the Fondation Courtois, and by an Alfred P. Sloan Research Fellowship. This work was partially supported by the National Science Foundation under Grant No. DMS-1928930, while E.S. was in residence at the Simons Laufer Mathematical Sciences Institute (previously known as MSRI) Berkeley, California during the Fall 2022 semester. 
J.Z. is supported by USTC Research Funds of the Double First-Class Initiative. This work was initiated when J.Z. was a CRM-ISM Postdoctoral Research Fellow at CRM, Universit\'e de Montr\'eal, and he thanks this institute for its warm hospitality.

%
%We require the following result which we deduce from the work of Albers-Frauenfelder-Oancea \cite{AFO17} by means of the pair-of-pants product.
%
%\begin{prop}\label{prop: AFO}
%Let $M$ have non-trivial second Hurewicz group. Then there exists a prime $p$ such that for every class $\al \in \pi_0(\cl LM)$ there exists a rank-one $\F_p$-local system $\cl V_{\al}$ on $\cl L_{\al} M$ such that $H_*(\cl L_{\al} M; \cl V_{\al}) = 0.$ 
%\end{prop}
%
%We describe a proof in Section \ref{sec: loc syst} below. We note that this proposition implies that $H_*(\cl L M_{\al}; \cl V) = 0$ for every $\al \in \pi_0(\cl LM).$ 

%\begin{proof}[Proof of Theorem \ref{thm: main}]
%
%Suppose by contradiction that there are finitely many prime closed geodesics. We denote them by $\gamma_1,...,\gamma_P, \gamma_{P+1},...,\gamma_N,$ such that the geodesics $\gamma_1,...,\gamma_P$ have mean-index $\Delta(\gamma_i) > 0$ and $\gamma_i$ for $i>P$ have $\Delta(\gamma_i) = 0.$ Then by Property \eqref{vanishing} of the mean-index, for every $i>P$ $\ind(\gamma_i^j) = 0$ for all iterations $j.$  Now all closed geodesics are iterations of $\gamma_1, \ldots, \gamma_N$. Consider the set $S$ of all the conjugacy classes $[\gamma_i^j],$ $1 \leq i \leq P$ such that $\Delta(\gamma_i^j) \leq n.$ Since the mean-indices of $\gamma_i$ for $1 \leq i \leq P$ are positive, by Property \eqref{homogeneity} of the mean-index, $S$ is a finite set. Therefore, our hypothesis that $\pi_1(M)/\mrm{conj}$ is an infinite set implies that there exists a conjugacy class $\al$ not in $S$. Take a point of absolute minimum $c \in \cl L_{\al} M,$ which is a closed geodesic with non-trivial local homology in degree $0$. Now by Proposition \ref{prop: AFO}, there is a rank-one $\F_p$-local system $\cl V_{\al}$ on $\cl L_{\al} M$ such that $H_*(\cl L_{\al} M; \cl V_{\al}) = 0$. In particular, by Corollary \ref{cor: non-vanishing} there exists a geodesic $b$ in $\cl L_{\al} M$ with non-trivial local homology in degree $1$ and, crucially, $\cl A(c) < \cl A(b).$ Now by construction, $b$ is not an iteration of any of the $\gamma_i$ for $1 \leq i \leq P.$ Otherwise, by the choice of $\al,$ $\Delta(b)>n$ and hence by Corollary \ref{prop local homology: mean support}, the support of the local homology of $b$ is in the interval $[\Delta(b) - (n-1), \Delta(b)+n]  \subset (1,\infty),$ which does not contain $1.$ Therefore $b = \gamma_i^j$ for some $i>P$ and $j \geq 1.$ In particular $\ind(b^m) = 0$ for all $m\geq 1,$ $b$ has non-trivial local homology, and $b$ is not an absolute minimum of energy in its free homotopy class $\cl L_{\al} M.$ By Bangert-Klingenberg \cite[Theorem 3]{BK83}, this implies that there exist infinitely many prime closed geodesics on $M.$
%\end{proof}

%\begin{proof}[Proof of Theorem \ref{thm: spinoff}]
%The proof is similar to that of Theorem \ref{thm: main} with the only difference being the mechanism for obtaining the existence of the geodesic $b.$ We argue as follows. As $\cl L_{\al} M$ is connected, the degree $0$ homology $H_0(\cl L_{\al} M)$ over the ground field is one-dimensional. Moreover, $\dim H_0^{\loc}(c) \geq 1.$ Now $c' = f(c)$ is also a geodesic in $\cl L_{\al} M.$ It is disjoint from $c,$ $\cA(c) = \cA(c')$ and $H^{\loc}_*(c') \cong H^{\loc}_*(c).$ Therefore there are at least two bars in the barcode given by the filtered homology of $\cl L_{\al} M$ with the ground field coefficients starting at the minimum level $\cA(c).$ Since $H_0(\cl L_{\al} M)$ is one-dimensional, this implies that one of these bars is finite, which provides the existence of $b$ as required.
%\end{proof}

%\section{Local systems on the free loop space}\label{sec: loc syst}
%
%\subsection{Albers-Frauenfelder-Oancea's local system}\label{ssec-AFO-ls}
%
%The proof of Proposition \ref{prop: AFO} relies on a choice of the local system $\cl V$ such that the {\it full} symplectic cohomology ${\rm SH}^*(D; \nu) = 0$ for the Liouville domain $D = D_g^*L$. This has been achieved by \cite{AFO17} under topological conditions including non-vanishing of the second Hurewicz group in the statement of Theorem \ref{thm: geod} for the component $\cl L_0 M$ of contractible loops. We show that if this local system comes from the non-vanishing of the second Hurewicz group, then the symplectic cohomology vanishes in all components of the free loop space. This is achieved by showing that the local system is compatible with the pair-of-pants product, and hence the vanishing of the unit in the contractible component yields vanishing of the symplectic cohomology in all components.
%
%For our use later, we need a more explicit description of the local system produced by \cite{AFO17}. Although the exposition in \cite{AFO17} is only on $\mathcal L_0L$, the same argument works for any connected component $\mathcal L_{\alpha} L$. Starting from a reformulation of the abelian group formed by the local systems over $\Z/(p)$ on $\mathcal LL$, that is, ${\rm Hom}(\pi_1(\mathcal LL); \Z/(p)) $, we have the following identification, 
%\begin{equation} \label{abelian-identification}
%{\rm Hom}(\pi_1(\mathcal LL); \Z/(p)) \simeq H_{\rm inv}^2(\tilde{L}; \Z/(p)) \times {\rm Hom}(\pi_1(L); \Z/(p)). 
%\end{equation}
%where $\tilde{L}$ is the universal cover of $L$ and $H_{\rm inv}^2(\tilde{L}; \Z/(p))$ is the subgroup of $\pi_1(L)$-invariant elements in $H^2(\tilde{L}; \Z/(p))$ (cf.~(2) and (3) in \cite{AFO17}). 
%\begin{rmk} Strictly speaking, $\mathcal LM$ is a disjoint union of various connected components $\mathcal L_{\alpha} M$ for loop classes $\alpha$, and $\pi_1(\mathcal LM)$ should be more precisely written as $\pi_1(\mathcal L_{\alpha} M, \eta_{\alpha})$ for a based point $\eta_{\alpha}$. However, different choices of base point result in inner automorphisms (by conjugations) of ${\rm Hom}(\pi_1(\mathcal L_{\alpha} M, \eta_{\alpha}); \Z/(p))$, so for brevity we will just use ${\rm Hom}(\pi_1(\mathcal L_{\alpha} M); \Z/(p))$. Moreover, once a loop $x$ is given in $\mathcal LM$, there exists a unique class $\alpha$ such that $x \in \mathcal L_\alpha M$. Importantly, for loop classes $\alpha$, the fundamental groups $\pi_1(\mathcal L_{\alpha} M)$ are isomorphic, since the based loop spaces $\Omega_\alpha M$ (as an $H$-space) are all homotopy equivalent (by Exercise 3 in Section 3.C in \cite{Hat02}) and $\pi_1(\mathcal L_{\alpha} M) \simeq \pi_1(\Omega_{\alpha} M) \rtimes \pi_1(M)$. Therefore, the notation $\pi(\mathcal LM) : = \pi(\mathcal L_{\alpha} M, \eta_{\alpha})$ will not cause any ambiguity to ${\rm Hom}(\pi_1(\mathcal LM), \Z/(p))$.\end{rmk}
%
%The identification (\ref{abelian-identification}) tells us that one way to cook up a nonzero local system is to produce an element in the form of $(\nu, 0)$, where $\nu$ is a non-zero element in $H_{\rm inv}^2(\tilde{L}; \Z/(p))$. An important observation is that 
%\begin{equation} \label{inv-inclusion}
%p^*H^2(L; \Z/(p)) \subset H^2_{\rm inv}(\tilde{L}; \Z/(p)). 
%\end{equation}
%where $p: \tilde{L} \to L$ is the covering map This leads a more concrete target that our desired element in $H^2_{\rm inv}(\tilde{L}; \Z/(p))$ can be obtained in the form of a pullback $p^*\tau$ for some $\tau \in H^2(L; \Z/(p))$. The difficulty is to guarantee that $p^*\tau \neq 0$. 
%
%To this end, we need some result on the group (co)homology. By Theorem $8^{\rm bis}$.~10 in \cite{McC01}, the group $H_{\rm inv}^2(\tilde{L}; \Z/(p))$ fits into an exact sequence,
%\begin{equation*} \label{CL-LSS}
%0 \to H^2(\pi_1(L); \Z/(p)) \xrightarrow{f} H^2(L; \Z/(p)) \xrightarrow{p^*} H^2_{\rm inv}(\tilde{L}; \Z/(p)) \to H^3(\pi_1(L); \Z/(p)) \to …
%\end{equation*}
%where $\Z$ is viewed as a trivial $\Z[\pi_1(L)]$-module. Remarkably, by Proposition 9 in \cite{AFO17}, non-zero of the Hurewicz map $h: \pi_2(L) \to H_2(L)$ is equivalent to non-surjectivity of $f$ in this exact sequence, so there exists some $\tau \in H^2(L; \Z/(p))$ such that $p^*\tau \neq 0$. 
%
%The local system $\nu = (p^*\tau, 0)$ satisfies 
%\begin{equation} \label{vanish-lL}
%H_*(\mathcal LL; \nu) =0
%\end{equation}
%by a cute fact - Proposition 3 in \cite{AFO17} - together with a fast-degenerated spectral sequence resulted from the standard fibration $\Omega M \hookrightarrow \mathcal LM \to M$. For this step, working over the coefficient $\Z/(p)$ for a prime $p$ is necessary. Then by Theorem 4.1.1 and Remark 4.1.2 in \cite{Abo15}, we obtain the following local system, still denoted by $\nu$ on $\mathcal L D_g^*L$ such that the refined (full) symplectic cohomology vanishes. Moreover, such a local system enjoys the following property. 
%
%\begin{prop} \label{prop-top-cond} The local system $\nu$ on $\mathcal L D_g^*L$ obtained via in (\ref{vanish-lL}) is $\Z/(p)$-equivariant and $p$-admissible. \end{prop}
%
%This result will be crucial to the construction and well-definedness of the refined symplectic cohomology (\ref{intro-symp-coh}), as well as its related property as in (\ref{P-nu-SH}). We postpone its proof to Section \ref{sec-lsc}. 
%
%
%\subsection{Proof of Proposition \ref{prop-top-cond}} \label{sec-lsc}
%
%%We will apply transgression (see subsection 7.2 in \cite{Rit13}) to get a local system $\nu$ that satisfies the $p$-admissible condition. Then we confirm that such $\nu$ is within the choices of local system in Theorem 2 in \cite{AFO17} that satisfy the vanishing property. Finally, by analyzing the process of obtaining such $\nu$ in \cite{AFO17}, we observe that it automatically satisfies the $\Z/(p)$-equivariant property. 
%
%Recall that the abelian group formed by local systems over $\Z/(p)$ on $\mathcal LL$ is isomorphic to ${\rm Hom}(\pi_1(\mathcal LL), \Z/(p))$. 
%Observe that any element $C \in \pi_1(\mathcal LL)$ is in fact a class in $[\mathbb T^2; L]$. Consider the following map 
%\begin{equation} \label{dfn-trans}
%s^*: H^2(L; \Z/(p)) \to H^1(\mathcal LL; \Z/(p)) \,\,\,\, \mbox{by}\,\,\,\, s^*(\tau)([C]) = \int_C \tau
%\end{equation}
%for any cohomology class $\tau \in H^2(L; \Z/(p))$ and homology class $[C] \in H_1(\mathcal LL; \Z/(p))$. The map $s^*$ is called a {\it transgression} (see subsection 7.2 in \cite{Rit13}). Fix a class $\tau \in H^2(L; \Z/(p))$, then (\ref{dfn-trans}) says that $s^*(\tau)$ defines a local system over $\Z/(p)$ on $\mathcal LL$. For brevity and the consistency of the notation for a local system, denote 
%\begin{equation} \label{s-trans}
%\nu^{\tau}: = s^*(\tau). 
%\end{equation}
%Here, we provide an example of a transgression that comes from the discussion in subsection \ref{ssec-AFO-ls}. 
%
%\begin{exa} \label{ex-trans}
%Denote by $p: \tilde{L} \to L$ the covering map where $\tilde{L}$ is the universal cover of $L$ (in particular $\pi_1(\tilde{L}) =0$). Then we have the following induced map 
%\begin{equation} \label{covering_map}
%p^*: H^2(L; \Z/(p)) \to H^2(\tilde{L}; \Z/(p)) = {\rm Hom}(H_2(\tilde{L}; \Z/(p)); \Z/(p)).
%\end{equation}
%In fact, more precisely $p^*$ maps into $H^2_{\rm inv}(\tilde{L}; \Z/(p))$, the $\pi_1(L)$-invariant subspace of $H^2(\tilde{L}; \Z/(p))$ (cf.~(\ref{inv-inclusion})). Taken any $\tau \in H^2(L; \Z/(p))$, we claim that $p^*\tau$ is a transgression. Indeed, consider the following canonical identification 
%\begin{equation} \label{can-isos}
%\pi_1(\mathcal LM) \simeq \pi_2(M) \simeq \pi_2(\tilde{M}) \simeq H_2(\tilde{M}; \Z) 
%\end{equation}
%where the first and the third isomorphisms are the identity and the second isomorphism is the inverse of the induced map $p_*$, that is, $(p_*)^{-1}$. Then for any $[C] \in \pi_1(\mathcal LM)$, we have 
%\[ \left<p^*\tau, (p_*)^{-1}([C]) \right> = \left< \tau, (p_*\circ (p_*)^{-1})([C]) \right>= \int_C \tau. \]
%In this way, the transgression can be obtained by $p^*$. Note that $p^* = s^*$ in a tautological way, up to the canonical isomorphisms in (\ref{can-isos}). 
%\end{exa}
%
%The following is a general result for a local system constructed in (\ref{s-trans}). 
%
%\begin{lma} \label{lemma-trans-p-adm} For any $\tau \in H^2(M; \Z/(p))$, the local system $\nu^{\tau}$ is $p$-admissible. \end{lma}
%
%\begin{proof} For simplicity, we will consider the case that any punctured embedded Riemannian surface $u$ with genus $g=0$ in $M$ and with $2$-many negative punctures asymptotic to closed orbits $(x, y)$ and one positive puncture asymptotic to the closed orbit $z$ induces a map $\nu^{\tau}_u: \nu_{x} \otimes \nu_{y} \to \nu_{z}$. The general case is proved in the same way. 
%
%Assume that $x \in \mathcal L_{\alpha}M$ and $y \in \mathcal L_{\beta}M$ for loop classes $\alpha$ and $\beta$. Fix base point $\eta_{\alpha}$ of $\mathcal L_{\alpha}M$ and $\eta_{\beta}$ of $\mathcal L_{\beta} M$ such that $\eta_{\alpha}(0) = \eta_{\beta}(0) = x_0 \in M$. Then the concatenation at point $x_0$, $\eta_{\alpha} \ast \eta_{\beta} \in \mathcal L_{\alpha \ast \beta} M$, where $\alpha \ast \beta$ is the Pontryagin product of classes $\alpha$ and $\beta$ with respect to point $x_0$. 
%
%For a given punctured embedded Riemannian surface $u$, since $g = 0$, the loop $z \in \mathcal L_{\alpha \ast \beta} M$. Moreover, within their corresponding connected components, there exist pathes (may not be unique)
%\[ \gamma_x: \eta_{\alpha} \to x, \,\,\, \gamma_y: \eta_{\beta} \to y, \,\,\, \mbox{and}\,\,\, \gamma_z: z \to \eta_{\alpha} \ast \eta_{\beta}. \]
%By concatenating $u$ with $\gamma_x$, $\gamma_y$ and $\gamma_z$, we extend $u$ to $\tilde{u}$ which is a punctured Riemannian surface $\tilde{u}$ with two negative ends $\eta_{\alpha}$, $\eta_{\beta}$, and one positive end $\eta_{\alpha} \ast \eta_{\beta}$. See Figure \ref{extension}. 
%% Figure environment removed
%Observe that 
%\[ \partial \tilde{u} = \eta_{\alpha} \ast \eta_{\beta} - \eta_{\alpha} - \eta_{\beta} =0.\]
%This implies that we have a well-defined morphism $\nu^{\tau}_{\tilde u}: \nu^{\tau}_{\eta_{\alpha}} \otimes \nu^{\tau}_{\eta_{\beta}} \to \nu^{\tau}_{\eta_{\alpha} \ast \eta_{\beta}}$ by integrating any representative of $\tau$ over $\tilde{u}$. Then the desired morphism $\nu^{\tau}_u$ is defined as the composition of the following morphisms, 
%\[ \nu^{\tau}_x \otimes \nu^{\tau}_y \xrightarrow{\nu^{\tau}_{\gamma_x} \otimes \nu^{\tau}_{\gamma_y}} \nu^{\tau}_{\eta_{\alpha}} \otimes \nu^{\tau}_{\eta_{\beta}} \xrightarrow{\nu^{\tau}_{\tilde{u}}} \nu^{\tau}_{\eta_{\alpha} \ast \eta_{\beta}} \xrightarrow{\nu^{\tau}_{\gamma_z}} \nu^{\tau}_z. \]
%Suppose we take a different path $\gamma'_x$ from $\eta_{\alpha}$ to $x$. Then $\nu^{\tau}_{\gamma'_x} = \rho \circ \nu^{\tau}_{\gamma'_x}$ for some $\rho \in {\rm Hom}(\pi_1(\mathcal LM); \Z/(p))$. Meanwhile, we obtain a different extension $\tilde{u}'$ from $u$, and $\nu^{\tau}_{\tilde{u}} = \nu^{\tau}_{\tilde{u}'} \circ \rho^{-1}$ for the same $\rho$. Via $\gamma'_x$, denote the resulting morphism by $\nu^{\tau}_{u'}$. Then 
%\begin{align*}
%\nu^{\tau}_{u'} & = \nu^{\tau}_{\gamma_{z}} \circ \nu^{\tau}_{\tilde{u}'} \circ (\nu^{\tau}_{\gamma'_x} \otimes \nu^{\tau}_{\gamma_y}) \\
%& = \nu^{\tau}_{\gamma_z} \circ (\nu^{\tau}_{\tilde{u}} \circ \rho^{-1}) \circ ((\rho \circ \nu^{\tau}_{\gamma_x}) \otimes \nu^{\tau}_{\gamma_y}) \\
%& = \nu^{\tau}_{\gamma_{z}} \circ \nu^{\tau}_{\tilde{u}} \circ (\nu^{\tau}_{\gamma_x} \otimes \nu^{\tau}_{\gamma_y}) = \nu^{\tau}_{u}.
%\end{align*}
%The same argument works for $\gamma_y$ and $\gamma_z$. Therefore, our construction of morphism $\nu^{\tau}_u$ is canonical, i.e., independent of the connecting paths. Thus we complete the proof.  
%\end{proof}
%
%Now, we are ready to give the proof of Proposition \ref{prop-top-cond}.
%
%\begin{proof} [Proof of Proposition \ref{prop-top-cond}] Example \ref{ex-trans} and Lemma \ref{lemma-trans-p-adm} imply that the resulting local system denoted by $\nu^{\tau}$, more precisely $\nu^{p^*\tau}$, is $p$-admissible. On the other hand, since $p^*\tau$ is automatically $\pi_1(L)$-invariant, it is readily to verify that the local system $\nu^{\tau}$ is $S^1$-equivariant (see (3.4.10) in \cite{Abo15}) with the action induced by $\pi_1(L)$. In particular, it is $\Z/(p)$-equivariant by Definition \ref{dfn-p-inv}. Thus we obtain the desired conclusion. \end{proof} 
%
%\begin{rmk} Theorem 2 in \cite{AFO17} in fact proves a stronger conclusion that the local system $\nu^{\tau}$ on $\mathcal LD_g^*L$ that we obtained in Proposition \ref{prop-top-cond} can restrict to any prescribed local rank on $L$ (viewed as the zero section of $T^*L$).  \end{rmk}
%
%\subsection{Local systems on $\cl L_{\al} M$}
%
%Here we prove a generalization of our main result. Namely, it is proven in \cite{AFO17} that there exists a prime $p$ and a rank-one $\F_p$-local system $\cl V$ on the free homotopy class of contractible loops $\cl L_0 M$ such that $SH^*(D^*L; \cl V)_0 = 0$ if  $H^2_{\mathrm{inv}}(\til{M};\F_p) \neq 0.$ This is equivalent to the statement that at least one of the following two conditions holds: (i) $H_2^S(M;\Z) \neq 0$ or (ii) $H^3(\pi_1 M; \F_p) \to H^3(M; \F_p)$ is not surjective for some prime number $p.$ The first condition is equivalent to the map $H^2(\pi_1 M; \F_p) \to H^2(M; \F_p)$ not being injective for some prime number $p.$ It is, however, more easily seen to be essentially independent of $\pi_1(M)$ as we formulate it initially. Our main results, Theorem \ref{thm: main} and Corollary \ref{cor: generic} are stated under only the first hypothesis for this reason. In this section we show that these results still hold under the more general assumption that $H^2_{\mathrm{inv}}(\til{M};\F_p) \neq 0$ for a prime $p$ and hence also under condition (ii). Indeed, we have the following result, whose proof is slightly less transparent than that of Proposition \ref{prop: AFO}.
%
%\begin{prop}\label{prop: vanishing H2inv}
%Let $M$ be a closed manifold such that $H^2_{\mathrm{inv}}(\til{M};\F_p) \neq 0$ for a prime $p.$ Then the local system of Albers-Frauenfelder-Oancea extends to a local system $\cl V$ on all $\cl LM$ with the property that $SH^*(D^*L; \cl V) = 0,$ that is the symplectic cohomology of $D^*L$ in any class $\al \in \pi_0 \cl LM$ vanishes.
%\end{prop}
%
%\begin{thm}\label{thm: main 2}
%Let $M$ be a closed manifold such that $H^2_{\mathrm{inv}}(\til{M};\F_p) \neq 0$ for a prime $p.$ Then
%\begin{enumerate}
%\item If $\pi_0 LM$ is infinite, then every Riemannian or Finsler metric on $M$ admits infinitely many geometrically distinct closed geodesics. 
%\item A generic Riemannian metric or a generic Finsler metric on $M$ admits infinitely many geometrically distinct closed geodesics.
%\end{enumerate}
%\end{thm}
%
%
%\begin{proof}[Proof of Theorem \ref{thm: main 2}]
%Proposition \ref{prop: vanishing H2inv} implies Theorem \ref{thm: main 2} by the same argument as in the proof of Theorem \ref{thm: main}, where it replaces Proposition \ref{prop: AFO}.
%\end{proof}
%
%\begin{proof}[Proof of Proposition \ref{item: extension}]
%We therefore need to prove that the local system $\cl V_0$ on $\cl L_0 M$ from \cite{AFO17} coming from $H^2_{\mathrm{inv}}(\til{M};\Z)$ extends to a local system $\cl V_{\al}$ on $\cl L_{\al} M$ such that the product \[SH^*(D^*L; \cl V_0) \otimes SH^*(D^*L; \cl V_{\al}) \to SH^*(D^*L; \cl V_{\al})\] is well-defined and the unit $1 \in SH^*(D^*L; \cl V_0)$ acts by the identity map. Indeed, in this case the vanishing of $SH^*(D^*L; \cl V_{\al})$ will follow from that of $SH^*(D^*L; \cl V_{0})$. We showed this above under a more restricted hypothesis by considerations of transgression. Now let us consider the Serre fibration \[\Om M_{\al} \to \cl L M_{\al} \xrightarrow{ev} M,\] where $ev$ is the evaluation map and \[\Om M_{\al} = \sqcup_{a \in \pi_1(M),\; [a]=\al} \Om M_a\] is the preimage of a point (see \cite{Tai85}). Picking a basepoint $\gamma \in \Om M_{\al},$ $[\gamma]=a \in \pi_1(M)$ we get the the following part of the long exact sequence of this fibration for $x = ev(\gamma)$: \[ \pi_2(M,x) \xrightarrow{[-,a]} \pi_2(M,x) \cong \pi_1(\Om_a M, \gamma) \to \pi_1(\cl L_{\al} M, \gamma) \xrightarrow{e} C_a, \] where $e=ev_*$ is a surjection to the centralizer $C_a$ of $a$ in $\pi_1 M$ and $[-,a]$  denotes the isomorphism given by the action of $a$ in $\pi_1(M)$ on $\pi_2(M)$ by the Whitehead product. In this way we obtain an isomorphism \[ \pi_1(\cl L_{\al} M, \gamma) \cong  \pi_2(M,x) \rtimes C_a.\] Therefore any local system $\cl V_0$ on $\cl L_0 M$ corresponding to $\nu \in Hom(\pi_2 M, \Z/\ell)_{\mrm{inv}}= H^2_{\mrm{inv}}(\til{M},\Z/\ell)$ extends to a local system on $\cl L_{\al} M.$ Moreover, the restrictions of these local systems to $\Om_0 M, \Om_a M$ are compatible with the concatenation \[m: \Om_0 M \times \Om_a M \to \Om_a M\] in the sense that \[m^*(\cl V_{\al}) = \cl V_0 \boxtimes \cl V_{\al}.\] 
%
%A close inspection of Figure \ref{extension} shows that therefore the local systems are compatible with the pair-of-pants product as necessary. Namely, fix a standard Morse function $f$ on the pair of pants with a unique figure-$8$ level set $Z = f^{-1}(\{0\}),$ $Z = C \cup A,$ where $C \cap A = \{p\},$ where $p$ is the index $1$ critical point. Consider $u|_C,$ $u|_A$ as loops in the free homotopy classes $\cl L_0 M,$ $\cl L_{\al} M$ respectively. Then $u$ provides paths in $\cl L_0 M,$ $\cl L_{\al} M$ from $x, y$ to  $u|_C,$ $u|_A$ respectively and hence maps $\cl V_x \to \cl V_{u|_C},$ $\cl V_y \to \cl V_{u|_A}.$
%
%Now consider $q = u(p) \in M.$ Choose a path $\theta:[0,1] \to M$ from $q$ to $x$ and let $\ol{\theta}$ be the time-reversal of $\theta,$ that is $\ol{\theta}(t) = \theta(1-t).$ This yields the loops $u'_{C,\theta} = \theta \# (u|_C) \# \ol{\theta} \in \Om_0 M$ and $u'_{A,\theta} = \theta \# (u|_A) \# \ol{\theta} \in \Om_{\al} M$ and maps $\cl V_{u|_C} \to \cl V_{u'_{C,\theta}},$ $\cl V_{u|_A} \to \cl V_{u'_{A,\theta}}.$ Now since $\cl V$ is a local system coming from $H^2_{\mrm{inv}}(\til{M},\Z/\ell)$, $\cl V_{u'_{C,\theta}}, \cl V_{u'_{A,\theta}}$ are independent of $\theta$ and so are the maps from the previous sentence. Indeed, changing $\theta$ to $\theta_1$ results in $u'_{A,\theta_1} = [u'_{A,\theta},\gamma]$ for $\gamma$ the loop $\theta \# \ol{\theta}_1$ which represents the action of $C_a$ on $\pi(\cl L_{\al} M).$ The statement about maps is proven analogously. Similarly $\cl V$ over the concatenation $u'_{C,\theta} \# u'_{A,\theta}$ is canonically identified with $\cl V$ over $\theta \# u|_{C} \# u|_{A} \# \ol{\theta}$ and hence with $\cl V$ over $u|_{C} \# u|_{A}.$ Now $u$ provides a map from $\cl V$ over $u|_{C} \# u|_{A}$ to $z.$ However, we know that $\cl V_{u'_{C,\theta} \# u'_{A,\theta}} \cong \cl V_{u'_{C,\theta}} \otimes \cl V_{u'_{A,\theta}}.$ Therefore we obtain a map $\cl V_{x} \otimes \cl V_{y} \to \cl V_{z}$ associated to $u.$
%\end{proof}
%
%\begin{rmk}
%In fact, the same argument works for arbitrary class $\cl LM_\beta$ instead of $\cl LM_0$ with the proviso that the target of the map takes into account all possible relevant conjugacy classes $[agbg^{-1}]$ for fixed representatives $[a]=\al, [b]=\beta$ and $g \in \pi_1(M).$
%\end{rmk}

%\tableofcontents

%\section{Introduction and main results}\label{sec:intro}

%\subsection{A brief introduction} The question of the existence of infinitely many periodic orbits of Hamiltonian systems has been of interest for a long time. Most progress to date has been made in the discrete-time case of periodic points of Hamiltonian diffeomorphisms on one hand \cite{BrownNeumann,Franks-sphere,Franks-NY,SalamonZehnder,Hingston-torus,Ginzburg-CC}, \cite{GG-ai,Hein-CC,GG-negmon,GG-revisited}, \cite{S-HZ,GG-hyperbolic,GG-pseudorotations,Gurel-nc,GG-nc,Orita1,Orita2} and in the case of existence of closed geodesics on the other hand \cite{GromollMeyer,BaHi-Z,Hingston-thesis,Rad-average,Taim-nc,Franks-sphere,Hingston-sphere,Ball-nc,Tanaka-nc,JoMc-Fp}. Part of the results in the second case have been extended to Reeb dynamics on contact manifolds in recent years. For example \cite{McLean-Reeb, HM-Reeb} adapt the classical Gromoll-Meyer theorem \cite{GromollMeyer} to the case of Reeb dynamics on starshaped hypersurfaces in cotangent bundles of a large class of manifolds. Moreover, \cite{GGM-Reeb,GShon} proves a similar result for prequantizations of many symplectic manifolds with vanishing higher homotopy group. Finally \cite{Pellegrini} extends the classical result of \cite{BaHi-Z} to Reeb flows for contact forms on unit sphere cotangent bundles under certain additional conditions.

%Recently, a novel technique to address this broad question, Smith theory in filtered Floer homology, was introduced in \cite{SZhao,S-HZ}. In this paper we apply it, together with a version of the key result of \cite{GShon}, to prove a new sufficient condition that implies the existence of infinitely many closed Reeb orbits. This condition provides a new class of manifolds that complements that of \cite{McLean-Reeb,HM-Reeb,GGM-Reeb,GShon, Pellegrini}. Our methods rest of the crucial new modification of filtered Smith theory to include local systems on the loop space. The use of local systems in Smith theory has appeared in classical topology \cite{Git63,GD10} but has not appeared in the Floer theoretic setting before. Of technical interest is also the extension of the filtered Smith-type inequality in the aspherical setting \cite{SZhao} from the setting of Hamiltonian or exact symplectomorphisms to the setting of symplectic cohomology: this requires setting up equivariant continuation maps, and in our case also including local systems into consideration. 

%\subsection{Background on machinery} \label{ssec-background} In order to state our main result, let us introduce the necessary background and notations. A symplectic method to study a given Liouville domain $(D, \lambda)$ or its contact boundary $(S, \alpha) := (\partial D, \lambda|_{\partial D})$ is the (total) symplectic cohomology, usually denoted by ${\rm SH}^*(D; \mathcal K)$, where $\mathcal K$ is the ground field in discussion. \jznote{Roughly speaking, ${\rm SH}^*(D; \K)$ is a} \esnote{cohomology} \jznote{group that is generated by} \esnote{certain} \jznote{closed Reeb orbits, not necessarily prime, of $(S, \alpha)$, together with constant orbits,} \esnote{seen} \jznote{ as critical points of a $C^2$-small Morse function, in the interior of the domain $D$.}  Algebraically, ${\rm SH}^*(D, \K)$ is the direct limit of a well-designed family of Hamiltonian Floer cohomologies ${\rm HF}^*(H; \K)$. For more details, see \jznote{Section \ref{sec-sh}}. 
%
%Within different versions, the one that will be particularly helpful to us is the equivariant symplectic cohomology. In fact, we will very often consider the $\Z/(p)$-equivariant symplectic cohomology ${\rm SH}^*_{\Z/(p)}(D; \K)$ with the following \jznote{additional} constraints:
%\begin{itemize}
%\item[(a)] Action window $(s,t)$, given by the symplectic action functional. 
%\item[(b)] Generators are in a fixed free homotopy class $a \in \pi_0(\mathcal LD)$.
%\item[(c)] Ground field $\K= \F_p$ where $p$ is the prime that shows up in $\Z/(p)$.
%\item[(d)] A carefully chosen rank-1 local system $\nu$ on the free loop space $\mathcal LD$. 
%\end{itemize}
%\jznote{Here, $\pi_0(\mathcal LD)$ denotes the set of free homotopy classes of the loop space of domain $D$.} All together, we obtain a refined version of the symplectic cohomology, denoted by 
%\begin{equation} \label{intro-symp-coh}
%{\rm SH}_{a, \Z/(p)}^{*, (s, t)}(D; \F_p, \nu) \,\,\,\,\mbox{or for brevity \,\,\,\,${\rm SH}_{a, \Z/(p)}^{*, (s, t)}(D; \nu)$}.
%\end{equation}
%The precise definition \jznote{appears as} Definition \ref{dfn-zp-sc}. Each constrain above has its specific purpose for our main results. The well-definedness of such a symplectic cohomology, as a key intermediate result towards the proofs of our main results, is justified by a series of lengthy propositions in Section \ref{sec-k-sh}.  
%
%Moreover, any symplectic cohomology ${\rm SH}^*(D; \K)$ (that sometimes provides trivial information) has the following three main variants.  
%\begin{itemize}
%\item[\jznote{(i)}] {\it Positive symplectic cohomology}, denoted by ${\rm SH}^{*, >0}(D; \K)$, which only counts the generators with positive actions. 
%\item[\jznote{(ii)}] {\it Local symplectic cohomology}, denoted by ${\rm SH}^*_{\rm loc}(\gamma; \K)$, which is a Hamiltonian Floer theory that concentrates only near a closed Reeb orbit $\gamma$ of $\partial D$. For more details, see \jznote{Section \ref{ssec-local-symp-coh}}. 
%\item[\jznote{(iii)}] {\it Symplectic persistence modules}, denoted by $V(D)$, which enhances the symplectic cohomology theory via the persistence module theory. This provides refined numerical and symplectic invariants. In particular, one can read off these symplectic invariants via barcode denoted by $\mathbb B(D)$, which is a collection intervals $I \subset \R$. For more details, see \jznote{Section \ref{ssec-spm}}. 
%\end{itemize}
%We emphasize that refined \jznote{versions} of ${\rm SH}^*(D; \K)$ by \jznote{as in} (a) - (d) above, in particular (\ref{intro-symp-coh}), admits the three variants above. 
%
%Last but not least, \jznote{suppose that} $(D, \lambda) = (D^*_g L, \lambda_{\rm can})$, the unit codisk bundle of a closed Finsler metric $(L, g)$, where $\lambda_{\rm can}$ is the canonical symplectic structure on the cotangent bundle $T^*L$. It is well-known that there exists a one-to-one corresponendence between the (prime) closed Reeb orbits of $(\partial D^*_gL, \lambda_{\rm can}|_{\partial D^*_gL})$ and the (prime) closed geodesics on $(L, g)$. For a complete proof, see Proposition 2.4 in \cite{DGZ17}. Moreover, a famous result in symplectic geometry (see Theorem 5.1.1 in \cite{Abo15}) says that 
%\begin{equation} \label{Viterbo-iso}
%{\rm SH}^*(D^*_gL, \K) \simeq H_{-*}(\mathcal L L, \K)
%\end{equation} 
%when $L$ satisfies a mild topological condition (otherwise, one needs to \jznote{twist the coefficients by a suitable local system}). At present, it is only conjectured or \jznote{is a work in progress} that equivariant versions of (\ref{Viterbo-iso}) hold, while conditions (a) - (d) can be easily added into (\ref{Viterbo-iso}). Therefore, the work in this article does not rely on any specific properties of $H_{-*}(\mathcal LL, \K)$. In particular, our result on counting the closed geodesics, say Theorem \ref{thm: geod}, purely comes from symplectic methods. 
%
%\subsection{Main results} In this subsection, we list our main results. 
%
%\subsubsection{Symplectic Smith inequality} The classical equivariant cohomology equipped with the $\Z/(p)$-action results in a Smith-type inequality (see \cite{Bor60,Bre72}). More recently, in terms of the Floer theory, various works from Hendricks \cite{Hen17}, Seidel \cite{Sei15}, \jznote{Shelukhin \cite{S-HZ}}, Shelukhin-Zhao \cite{SZhao}, Cineli-Ginzburg \cite{CG21}, \jznote{and Sugimoto \cite{Sug21}} have successfully derived analogues of Smith inequality for fixed points of \jznote{Hamiltonian diffeomorphisms} and \jznote{provided} effective applications in symplectic geometry. In particular, in \cite{SZhao}, on a symplectically aspherical manifold, the following Smith inequality \jznote{was proven} for Hamiltonian Floer cohomologies, 
%\begin{equation} \label{Smith-SZ}
%\dim_{\F_p} {\rm HF}^{*, \,{\rm I}}(\phi) \leq \dim_{\F_p} \left({\rm HF}^{*, \,p \cdot {\rm I}}(\phi^p)\right)^{\Z/{(p)}} \leq \dim_{\F_p} {\rm HF}^{*, \,p \cdot {\rm I}}(\phi^p))
%\end{equation}
%for any Hamiltonian diffeomorphism $\phi = \phi_H^1$ and intervals $I = (s,t)$ and $p\cdot I = (ps, pt)$ in $\R$. \jznote{In  (\ref{Smith-SZ})}, the $\F_p$-module ${\rm HF}^{*, \,p \cdot {\rm I}}(\phi^p)$ admits a $\Z/{(p)}$-action $R_p$, which \jznote{comes from} rotating the Hamiltonian orbit $x$ by  
%\begin{equation} \label{intro-rot}
%x = x(t) \mapsto R_p (x) :=x\left(t + \frac{1}{p}\right).
%\end{equation}
%For more details, see \jznote{Section \ref{ssec-HF-revisit}}. The second term in (\ref{Smith-SZ}) is the dimension of the $\Z/(p)$-invariant subspace of ${\rm HF}^{*, \,p \cdot {\rm I}}(\phi^p)$ under the action $R_p$, therefore, the second inequality in (\ref{Smith-SZ}) trivially holds and \jznote{hence it suffices to prove the first inequality in (\ref{Smith-SZ}).} 
%
%\medskip
%
%Our first main result is also a Smith inequality but in \jznote{the setting} of \esnote{\st{the}} symplectic cohomologies. Our result does not immediately follow from (\ref{Smith-SZ}) since \jznote{it involves a more general version of symplectic cohomology than the standard one, see (\ref{intro-symp-coh})}. More explicitly, picking a rank-1 local system $\nu$, we will require $\nu$ to be $\Z/(p)$-equivariant (see Definition \ref{dfn-p-inv}) and $p$-admissible (see Definition \ref{dfn-p-adm}). \jznote{Free homotopy classes $a \in \pi_0(\mathcal LD)$ also appear in symplectic cohomology (\ref{intro-symp-coh}). Since 
%\begin{equation} \label{hom-conj}
%\pi_0(\cl{L}D) \simeq \pi_1(D)/\conjugation
%\end{equation}
%where $\pi_1(D)/\conjugation$ denotes the set of conjugacy classes of the fundamental group of $D$, any free homotopy class $a \in \pi_0(\cl{L}M)$ has well-defined iterates $a^k$ for all $k \in \Z$. In particular, $a^0 =\{\rm pt\}$ is the class of contractible loops in $D$. Indeed, under the identification (\ref{hom-conj}), if $a = [x]$ for some $x \in \pi_1(D)$, then define $a^k := [x^k]$. It is well-defined since for any other representative $y \in \pi_1(D)$ with $[y] = a$, we have $x = zyz^{-1}$ for some $z$, which implies that $x^k = (zyz^{-1})^k = zy^kz^{-1}$. In particular, $[x^k] = [y^k]$. Finally, we want to emphasize that in general $\pi_1(D)/\conjugation$ is just a set, instead of a group; moreover, counting or classifying the classes in $\pi_1(D)/\conjugation$ is a difficult problem.}
%
%\medskip
%
%\jznote{For a Liouville domain $(D, \lambda)$, denote by ${\rm spec}(D, \lambda)$ the set of the actions \esnote{$\int_{\gamma} \alpha$} of closed Reeb orbits \esnote{$\gamma$} of its contact boundary $(S, \alpha),$ \esnote{where $\alpha = \la|_S$}. \jznote{Sard's theorem implies that ${\rm spec}(D, \lambda)$ is always a closed nowhere dense subset of $\R$.} Recall that a non-degenerate $(D, \lambda)$ means the contact form $\alpha$ is a non-degenerate contact form on the boundary $S$. Explicitly, denote by $\xi$ the contact structure $\ker \alpha$ of $(S, \alpha)$, then, for every Reeb orbit (not necessarily prime) $\gamma$ of $(S, \alpha)$, the Poincar\'e return map on $(\xi_{\gamma(0)}, d\lambda)$ induced by the linearized Reeb flow along $\gamma$ does not have $1$ as an eigenvalue.} \jznote{For a non-degenerate Liouville domain $(D, \lambda)$, it is readily verified that the set ${\rm spec}(D, \lambda)$ is a discrete subset of $\R$}. 
%
%\medskip
%
%\jznote{Here is our main technical result.}
%
%\begin{thm} [symplectic Smith inequality] \label{thm-main} Let $(D, \lambda)$ be a non-degenerate Liouville domain, and $\nu$ be a $\Z/(p)$-equivariant and $p$-admissible rank-1 local system on the free loop space $\mathcal LD$ over the ground field $\K = \F_p$ for some prime number $p$. Then for any interval ${\rm I} = (s,t) \subset \R$ and $p\cdot {\rm I} = (ps, pt) \subset \R$ where $s,t, ps, pt \notin {\rm spec}(D, \lambda)$, we have the following inequalities, 
%\begin{equation*} \label{main}
%\dim_{\F_p} {\rm SH}_{a}^{*, \,{\rm I}}(D; \nu) \leq \dim_{\F_p} \left({\rm SH}_{a^p}^{*, \, {p \cdot {\rm I}}}(D; \nu)\right)^{\Z/{(p)}} \leq \dim_{\F_p} {\rm SH}_{a^p}^{*, \, p \cdot {\rm I}} (D; \nu)
%\end{equation*}
%for any free homotopy class $a \in \pi_0(\mathcal LD)$. 
%\end{thm}
%
%Similarly to (\ref{Smith-SZ}), the $\F_p$-module ${\rm SH}_{a^p}^*(D; \nu)^{p \cdot {\rm I}}$ admits a $\Z/{(p)}$-action $R_p$, which roughly speaking \jznote{acts} by rotating the Reeb orbit $\gamma$ by $\gamma = \gamma(t) \mapsto \gamma \left(t + 1/p\right) = : R_p(\gamma)$. The second term \jznote{in the conclusion of Theorem \ref{main}} is the dimension of the $\Z/(p)$-invariant subspace of ${\rm SH}_{a^p}^*(D; \nu)^{p \cdot {\rm I}}$ under the action $R_p$, therefore, it suffices to prove the first inequality, 
%\begin{equation} \label{half-main}
%\dim_{\F_p} {\rm SH}_{a}^{*, \,{\rm I}}(D; \nu) \leq \dim_{\F_p} \left({\rm SH}_{a^p}^{*, \, {p \cdot {\rm I}}}(D; \nu)\right)^{\Z/{(p)}}.
%\end{equation}
%The proof of (\ref{half-main}) passes \jznote{through} the $\Z/(p)$-equivariant version of symplectic cohomology in (\ref{intro-symp-coh}). Importantly, besides the $\F_p$-module structure, it also admits a $\Z/(p)$-equivariant version of pants product on symplectic cohomology that will be explained in \jznote{Section \ref{ssec-Zp-pp}}. 
%
%\begin{rmk} The trivial local system (the ground field) $\K = \F_p$, is $\Z/(p)$-equivariant and $p$-admissible. Therefore, one can state Theorem \ref{thm-main} just \jznote{for} a trivial local system. Here, we establish \jznote{an} enhanced version for later use, say the proof of Theorem \ref{thm: main 1}. Besides the well-definedness of (\ref{intro-symp-coh}) that is based on $\nu$ being $\Z/(p)$-equivariant, requiring $\nu$ to be $p$-admissible is necessary for the $\Z/(p)$-equivariant pants product \jznote{on} symplectic cohomology. \end{rmk} 
%
%
%\begin{exa} In this example, we illustrate the inequality 
%\[ \dim_{\F_p} {\rm SH}_a^{*, \, {\rm I}}(D; \nu) \leq \dim_{\F_p} {\rm SH}_{a^{p}}^{*, \,p \cdot {\rm I}}(D; \nu)\]
%from Theorem \ref{thm-main} for $(D, \lambda) = (D_g^* S^1, \lambda_{\rm can})$, the unit codisk bundle of $S^1$. Here, we fix the class $a$ as the homotopy class that winds around $S^1$ once, and we normalize the metric $g$ such that the total length of the base $S^1$ is $1$. Pick a preferred rank-1 local system $\nu$ on $\mathcal LD$ such that the barcode of ${\rm SH}_a^{*, \, {\rm I}}(D; \nu)$ is
%\[ \mathbb B({\rm SH}_a^*(D; \nu)) = \left\{ ( (1, 2], 1) \right\} \,\,\,\,\,\mbox{and}\,\,\,\,\, \mathbb B({\rm SH}_{a^p}^*(D; \nu)) = \left\{ ( (p, 2p], 1) \right\}\]
%if we view ${\rm SH}^*(D; \nu)$ as a persistence $\F_p$-module and $1$ represents the multiplicity of the bar in the barcode. Then for any interval ${\rm I} = (a,b)$ with $a, b, pa, pb \notin \N (= {\rm spec}(D, \lambda_{\rm can})$), we know 
%\begin{equation} \label{s1}
%\dim_{\F_p} {\rm SH}_a^{*, \, {\rm I}}(D; \nu)= \left\{ \begin{array}{lcl} 1 & \mbox{if} & (a,b) \cap (1,2]^c = \emptyset \\ 0 & \mbox{if} & {\rm otherwise} \end{array} \right.,
%\end{equation}
%%, \,\,\,\,\mbox{where ${\rm SH}^{*, \, {\rm I}}(W; \nu)= \F_p\left<x,x' \right>$} \]
%where $(1,2]^c$ is the complement of $(1,2]$ in $\R$. Observe that $(a,b) \cap (1,2]^c = \emptyset$ if and only if $(pa,pb) \cap (p,2p]^c = \emptyset$. Therefore, the conclusion (\ref{s1}) holds.
%\end{exa}
%
%\begin{rmk} \jznote{When $a = \{\rm pt\}$, the class of contractible loops (hence, $a^k = \{\rm pt\}$ for any $k \in \Z$), Theorem \ref{thm-main} still holds. In this paper, we will focus on the applications of Theorem \ref{thm-main} for non-contractible loops.} \esnote{We will explore the contractible case elsewhere. }\end{rmk}
%
%\subsubsection{Counting Reeb orbits} Theorem \ref{thm-main} can be used to count Reeb orbits. \jznote{Let us introduce the following crucial definition.} 
%
%\begin{df}\label{dfn-root-bounded} We call $a \in \pi_0(\cl{L}M)$ {\bf root-bounded} relative to an increasing sequence $\{k_i\}_{i \in I}$ of positive integers, if there exists a finite constant $C$ such that for each $i \in I$ and for each fixed $b \in \pi_0 (\cl{L} M)$, the equation $a^{k_i} = b^l$ \jznote{that occurs in $\pi_0(\mathcal LM)$} has at most $C$ solutions $l\in \Z_{>0}$.  Equivalently, 
%\begin{equation} \label{eqn-rb}
%\sup_{i\in I} \sup_{b \in \pi_0 (\cl{L} D)}  \#\{l \in \Z_{>0}\,| \,a^{k_i} = b^l\} < \infty.
%\end{equation}
%\end{df}
%
%\begin{rmk} \label{rmk-infinite-order} Observe that every root bounded element $a$ (for any defining sequence $\{k_i\}_{i \in I}$) is of infinite order in the sense that $a^k \neq \{\rm pt\}$ for all $k \in \Z_{>0}$. In particular, the class $a$ itself can not be represented by contractible loops. To verify the desired infinite order, suppose $a^{k_*} = \{\rm pt\}$ for some finite integer $k_* \in \Z$. Without loss of generality, assume that $k_*>0$. Then for any $k_0 \in \{k_i\}_{i \in I}$, we have $a^{k_0} = a^{k_0} \cdot (pt)^n = a^{k_0 + k_*n} = b^{l_n}$, where $b=a$ and $l_n = k_0+ k_*n$. When $n$ is sufficiently large, $l_n \in \Z_{>0}$. In other words, the equation $a^{k_0} = b^l$ admits infinitely many solutions as $n \to \infty$, which violates (\ref{eqn-rb}). \end{rmk}
%
%\begin{rmk} \label{rmk-abelian} \jznote{Sometimes it is easy to verify the existence of root-bounded classes. Here, we provide such an example. Suppose $\pi_1(\mathcal LD)$ is abelian and infinite, then $\pi_1(\mathcal LD) = \pi_1(\mathcal LD)/{\rm conj} \simeq \pi_0(\cl{L} M)$. For simplicity, assume $\pi_1(\mathcal LD) = \Z$ and a similar argument works for general cases. In this case, the power $a^k$ is the multiplication $ka$ since the group product in $\pi_1(\mathcal LD)$ is the sum in $\Z$. Then for any $a \neq 0$, we have
%\[ \# \left\{ l \in \Z_{>0}\,| \, a^{k_i} = b^l\right\} \leq 1 \]
%for any $b \in \pi_1(\mathcal LD)$ and $k_i \in \Z$. Therefore, any non-trivial $a \in \pi_1(\mathcal LD)$ is a root-bounded class. }
%\end{rmk}
%
%
%\noindent{\bf{Motivation of Definition \ref{dfn-root-bounded}}}. Recall that our machinery is the equivariant symplectic cohomology with certain constraints, for instance, by (b) in subsection \ref{ssec-background}, where the free homotopy class is $a^{k_i} \in \pi_0(\mathcal LD)$ for each $i \in I$. In order to extract useful information from ${\rm SH}^*_{a^{k_i}}(D; \K)$, for our purpose, it is necessary to control the generators of this cohomology group, that is, closed Reeb orbits of $\partial D$. In particular, it is important for us to have 
%\jznote{\begin{equation} \label{hol-finite-cond}
%\#\left\{\mbox{closed Reeb orbit $\gamma$} \,\big| \,a^{k_i} = [\gamma] \right\} < \infty \,\,\, \mbox{for any $i \in I$}
%\end{equation}}
%where $[\gamma]$ denotes the free homotopy class represented by $\gamma$. Here, we emphasize that since $\gamma$ is not assumed to be prime, in general $\gamma = \gamma_*^l$ for an $l$-iterate of some prime closed Reeb orbit $\gamma_*$. Therefore, $a^{k_i} = [\gamma]$ reads as $a^{k_i} = b^l$ for some class $b$ that is represented by a prime closed Reeb orbit $\gamma_*$. 
%
%The equality $a^{k_i} = [\gamma]$ in the condition (\ref{hol-finite-cond}) occurs in $\pi_0(\mathcal LD)$, so this condition is neither easy nor obvious to obtain, even if we assume that there are only finitely many {\it prime} closed Reeb orbits $\partial D$. As one can imagine, due to the complexity \jznote{coming} from conjugations, the behavior of $b^l$ as a conjugacy class heavily depends on the group relations in $\pi_1(D)$. \jznote{For instance, by Hull-Osin's work in \cite{HO13}, there exists a} \esnote{finitely generated} \jznote{group with infinitely many elements but only two conjugacy classes. Therefore, if such a group can be realized as a fundamental group, then $b^{l_1} = b^{l_2} = \cdots$ for infinitely many powers $l_i \in \Z$.} \esnote{Indeed, finitely presented such groups have not been constructed.} As mentioned above, in general, classifying or distinguishing conjugacy classes in a given group is a notoriously difficult problem \jznote{in group theory}. 
%
%\medskip
%
%Here is our second main result. 
%
%\begin{thm}\label{thm: main 1}
%Let $(S,\alpha)$ be the contact boundary of a Liouville domain $(D,\lambda)$ and fix ground field $\mathcal K = \F_p$ for some prime $p$. Assume the following conditions, 
%\begin{itemize}
%\item[(1)] there exists a $\Z/(p)$-equivariant and $p$-admissible rank-1 local system $\nu$ on $\mathcal LD$ such that the total symplectic cohomology ${\rm SH}^*(D; \nu) = 0$;
%\item[(2)] there exists a root-bounded class $a \in \pi_0(\mathcal LD)$ relative to $\{p^i\}_{i \in \N}$ that is presented by a closed Reeb orbit $\gamma$ with  local symplectic cohomology ${\rm SH}^*_{\rm loc}(\gamma) \neq 0$.
%\end{itemize}
%Then $(S,\alpha)$ admits infinitely many geometrically distinct closed Reeb orbits. 
%\end{thm}
%
%Here, two closed Reeb orbits are geometrically distinct if their images do not coincide. In particular, the closed Reeb orbits in Theorem \ref{thm: main 1} are all prime, not iterates of others. On the other hand, one can ensure the condition (2) in Theorem \ref{thm: main 1} exists for example if the positive symplectic cohomology of $D$ in class $a$ does not vanish. More efficiently, by Remark \ref{rmk-infinite-order}, since $a$ is never represented by contractible Reeb orbits (in particular, points), ${\rm SH}_a^{>0}(D; \nu) = {\rm SH}_a(D; \nu)$. \jznote{This immediately implies the following result on a certain class of Liouville hypersurfaces $(S, \alpha)$, where the conclusion is usually called the {\it contact Conley conjecture} (see \cite{GGM-Reeb}).} \jznote{It is worth emphasizing that, without any hypothesis on a contact manifold $(S, \alpha)$, the conclusion as in Corollary \ref{cor: CCC} may fail} \esnote{(this is for example the case of a generic ellipsoid in $\R^{2n};$} see \ref{}).
%
%\begin{cor}\label{cor: CCC}
%Let $(S,\alpha)$ be a contact boundary of a Liouville domain $(D,\lambda)$ such that \[{\rm SH}^*(D; \nu) = 0 \,\,\,\,\mbox{and} \,\,\,\, {\rm SH}^*_a(D; \nu) \neq 0\] for a root-bounded class $a \in \pi_0(\cl{L}D)$ and local system $\nu$ as in Theorem \ref{thm: main 1}. Then $(S,\alpha)$ admits \jznote{infinitely many} geometrically distinct closed Reeb orbits.
%\end{cor}
%
%\begin{rmk} Once Corollary \ref{cor: CCC} applies to $(S,\alpha)$ it clearly also applies to $(S,\alpha')$ for each contact form $\alpha'$ supporting the same contact structure. \end{rmk}
%
%\noindent \jznote{{\bf Novelty}}. Regarding the novelty of our methods, we observe that all previous proofs of the contact Conley conjecture relied on sufficiently fast growth of symplectic or contact homology groups \cite{McLean-Reeb,HM-Reeb} as a function of the index, or their non-vanishing in a sufficiently high degree independent of iteration, and the theory of symplectically degenerate maxima \cite{GGM-Reeb, GShon}. Our methods, say Corollary \ref{cor: CCC}, use no assumptions on growth beyond the non-vanishing of the initial (positive) symplectic cohomology group ${\rm SH}^*_a(D; \nu),$ and do not use symplectically degenerate maxima. Compatibly, our class of manifolds is not covered by any of the previous results.
%
%\subsubsection{Counting closed geodesics} \jznote{As explained in Section \ref{ssec-background}}, counting Reeb orbits is equivalent to counting closed geodesics when the Liouville domain $(D, \lambda) = (D^*_g L, \lambda_{\rm can})$, the unit codisk bundle of a closed Finsler manifold $(M, g)$. \jznote{We will apply Corollary \ref{cor: CCC} to prove the existence of infinitely many closed geodesics by choosing a suitable local system.} Luckily, the main work in \cite{AFO17} provides a topological condition on $L$ that guarantees the existence of such a local system. Here comes our third main result.
%
%\begin{thm}\label{thm: geod}
%\jznote{Let $L$ be a closed manifold satisfying the following topological conditions:
%\begin{itemize}
%\item[(1)] the Hurewicz map $\pi_2(L) \to H_2(L;\Z)$ is non-zero;
%\item[(2)] $\pi_0(\cl{L}L)$ admits a root-bounded class, 
%\end{itemize}
%Then $(L, g)$ admits infinitely many geometrically distinct closed geodesics for any Finsler metric $g$ on $L$.}
%\end{thm}
%
%Here, we emphasize that the two \jznote{topological} conditions in Theorem \ref{thm: geod} can be independent \jznote{of} each other. For instance, given any closed 4-manifold $\hat{L}$, one can modify $\hat{L}$ by adding a blow-up. Topologically, it results in a new closed manifold 
%\begin{equation} \label{blow-up} 
%L : = \hat{L} \# \overline{\C P^2}. 
%\end{equation}
%Obviously, $\pi_1(L) = \pi_1(\hat{L})$ since $\pi_1(\C P^2) =0$. So, the existence of a root-bounded class in $\pi_0(\mathcal L \hat{L})$ is equivalent to the existence of a root-bounded class in $\pi_0(\mathcal L L)$. Moreover, by Proposition 7 and Proposition 9 in \cite{AFO17}, based on the Cartan-Leray spectral sequence that relates the group cohomology of $\pi_1(L)$ and cohomology groups $H^*(\mathcal L; \Z/(p))$, one conclude that $L$ always has its Hurewicz map non-zero. 
%
%\begin{rmk} \label{rmk-blow-up} This remark is only used in subsection \ref{ssec-new-example}. Adding the blow-up as in (\ref{blow-up}) sometimes changes higher homotopy groups dramatically. For instance, if $\hat{L} = \mathbb T^4$, then one verifies that $\pi_2(L) = \bigoplus_{\Z^4} \Z$, which is in particular not finitely generated. \end{rmk}
%
%\begin{rmk} \label{rmk-lit-geod} In the \jznote{literature}, seeking for a \jznote{proper} condition \jznote{of Theorem \ref{thm: geod}} that can imply the conclusion under the assumption that $\pi_1(L)$ is non-trivial has been active for decades. Here we \jznote{review these conditions}. 
%\begin{itemize}
%\item[(1)] Theorem A in \cite{BTZ81}: when a non-trivial free homotopy class $a \in \pi_0(\mathcal L L)$ satisfies $a^m = a^n$ for some $m \neq n$ (for instance, if $\pi_1(L)$ is finite but non-trivial), the conclusion of \jznote{Theorem \ref{thm: geod}} holds for any $C^k$-generic Riemannian metric on $L$ where $k \geq 4$.
%\item[(2)] The main theorem in \cite{BaHi-Z}: when $\pi_1(M)$ is infinite and abelian.
%\item[(3)] Theorem 2 in \cite{Ball-nc}: when $\pi_1(L)$ is almost nilpotent but not an infinite cyclic group. 
%\item[(4)] Theorem 3 in \cite{Tai85}: when there exists a free homotopy class $a$ such that $\{a^m\}_{m \geq 1}$ are all distinct and if $\pi_k(L)$ is non-zero and finitely generated for some $k \geq 2$. 
%\item[(5)] Theorem 3 in \cite{Taim-nc}: when $\pi_1(L)$ is a solvable group. 
%\item[(6)] Theorem 2 in \cite{RT22}: when $\dim L = 3$ and $\pi_1(L)$ is infinite, the conclusion of \jznote{Theorem \ref{thm: geod}} holds for any $C^4$-generic Riemannian metric on $L$.
%\item[(7)] Theorem 4 in \cite{RT22b}: when $\pi_1(L)$ is finite, then the conclusion of \jznote{Theorem \ref{thm: geod}} holds for any strongly bumpy Riemannian metric on $L$ \jznote{(which is a $C^2$-generic condition)}.
%\end{itemize}
%
%As a matter of fact, most results above are numerical results that estimate the growth of a counting function $N(t)$ defined by 
%\begin{equation} \label{counting}
%N(t) : = \#\mbox{\{geometrically distinct closed geodesics with length $\leq t$\}}.
%\end{equation}
%Moreover, due to the covering trick, most results above, especially the condition on $\pi_1(L)$, appear in a ``virtual'' way. For instance, instead of requiring $\pi_1(L)$ to be solvable, one can relax it as follows: $\pi_1(L)$ contains a proper solvable subgroup \jznote{of} finite index; \jznote{in other words, it is virtually solvable}. 
%
%Comparing our sufficient conditions in Theorem \ref{thm: geod}, in particular, the existence of a root-bounded class in $\pi_0(\mathcal LL)$, with ones above is crucial in discovering new examples of Finsler manifold that admits infinitely many geometrically distinct closed geodesics. We will carry out the discussion \jznote{in detail in Section \ref{ssec-new-example}}. \end{rmk}
%
%
%\subsubsection{Stability} Let $(W, \lambda)$ be a Liouville manifold, and $(D_1, \lambda)$ and $(D_2, \lambda)$ are two Liouville domains inside $(W, \lambda)$. A symplectic way to quantitatively compare these two Liouville domains is via {\it symplectic Banach-Mazur distance}, usually denoted by $d_{\rm SBM}$. It captures the rescaling of a Liouville domain and symplectic embedding relations simultaneously. There are various versions of $d_{\rm SBM}$. For our purpose in this paper, we will use the one that was introduced in \cite{VukasinJun}, which is defined as 
%\[ \small{ d_{\rm SBM}(D_1,D_2) = \inf\left\{ \ln C \,\bigg| \, \begin{array}{cc} \mbox{$\exists$\,$\frac{1}{C}D_1 \xhookrightarrow{\phi} D_2 \xhookrightarrow{\psi} CD_1$} \,\,(\mbox{hence} \,\mbox{$\frac{1}{C}D_2 \xhookrightarrow{\psi(C^{-1})} D_1 \xhookrightarrow{\phi(C)} CD_2$}) \\\mbox{s.t. $\psi \circ \phi$ and $\phi(C) \circ \psi(C^{-1})$ are strongly unknotted} \end{array} \right\}. }\]
%A few words are in order for this definition. First, the notation $CD_1$ is the rescaling of the domain $D_1$ by the constant $C>0$. More explicitly, since a Liouville manifold always decomposes into the core part and a complement that can be realized as the trace of the rescalings of a contact hypersurface (see the beginning of \jznote{Section \ref{sec-sh}}). In particular, there exists a (globally-defined) Liouville vector field $X_{L}$ that can realizes this rescaling. Then $CD_1$ is defined by $\phi_{X_L}^{\ln C}(D_1)$. Meanwhile, the notation $\phi(C)$ is the corresponding rescaling of the morphism via conjugations. Second, the notation $D_1 \hookrightarrow CD_2$ means that there exists a Liouville embedding from $D_1$ into $CD_2$ such that the induced map on $\pi_1(D_2)$ is identity. Note that $\pi_1(D_1) = \pi_1(D_2) = \pi_1(W)$ since $W$ deformation retracts to both $D_1$ and $D_2$. This $\pi_1$-triviality is particularly important in this paper. Third, since $\frac{1}{C}D_1 \subset CD_1$ by inclusion, the morphism $\psi \circ \phi$ is strongly unknotted if it is isotopic to the inclusion through Liouville embeddings. 
%
%The most important result related to $d_{\rm SBM}$ and symplectic cohomology is a stability result, with respect to a quantitative comparison between barcodes, called bottleneck distance and denoted by $d_{\rm bot}$ (see \jznote{Section \ref{ssec-spm}}). Imitating the proof of Theorem 1.6 in \cite{VukasinJun}, it is readily \jznote{verified} that, for any class $a \in \pi_1(W)$ and any $\Z/(p)$-equivariant and $p$-admissible rank-1  local system $\nu$ on $\pi_0(\mathcal LW)$, 
%\jznote{\begin{equation} \label{stability}
%d_{\rm bot}(\mathbb B^{\log}({\rm SH}_a^*(D_1; \nu)), \mathbb B^{\log}({\rm SH}_a^*(D_1; \nu))) \leq d_{\rm SBM}(D_1, D_2) 
%\end{equation}}
%where \jznote{$\mathbb B^{\log}$} is the barcode of the input symplectic cohomology with endpoints of each interval inside modified by taking \jznote{$\log$}. 
%
%\begin{rmk} The definition of symplectic (co)homology used in \cite{VukasinJun} is not exactly the same as the one used in this paper. However, since both symplectic (co)homologies enjoy rather similar functorial properties (even for the filtered versions), the conclusion (\ref{stability}) can be proved in a similar way. \end{rmk}
%
%The following result shows the stability property of Theorem \ref{thm: main 1}, which serves as our fourth main result.  
%
%\begin{thm} \label{thm: stability} Let $(D_1, \lambda)$ and $(D_2, \lambda)$ be two Liouville domains of the Liouville manifold $(W, \lambda)$. Assume that $(D_1, \lambda)$ satisfies the assumptions of Theorem \ref{thm: main 1}. If $d_{\rm SBM}(D_1, D_2)$ is sufficiently small (in particular, finite), then $(\partial D_2, \lambda)$ admits infinitely many geometrically distinct closed Reeb orbits. \end{thm}
%
%
%One way to use Theorem \ref{thm: stability} is applying its contrapositive. Suppose $(\partial D_2, \lambda)$ admits only finitely many geometrically distinct closed Reeb orbits, then $d_{\rm SBM}(D_1, D_2)$ is bounded below by a certain number, denoted by $c>0$. In particular, applying to the set-up of unit codisk bundles $D_1 = D_{g_1}^*L$ and $D_2 = D_{g_2}^*L$, by Proposition 2.8 in \cite{VukasinJun}, namely the following inequality
%\[ 2 d_{\rm SBM}(D_1, D_2) \leq d_{\rm RBM}(g_1, g_2), \]
%we conclude a gap $\frac{c}{2}$ between $g_1$ and $g_2$ via the {\it Riemannian Banach-Mazur distance}, a $C^0$-comparison between Riemannian (in fact, Finsler) metrics, defined in Definition 1.5 in \cite{VukasinJun}. This gap seems difficult to obtain if we only compare the spectra of prime close geodesics as well as their iterates. 
%
%\subsection{Examples} 
%The topological conditions in Theorem \ref{thm: main 1} and Theorem \ref{thm: geod} look quite restrictive, but here we list some examples that satisfy these conditions.  
%
%\medskip
%
%(1) For a Liouville domain $(D, \lambda)$, the existence of a root-bounded class holds for example if 
%\[ \mbox{$[x] \neq 0$ in $H_1(D;\Z)/\mrm{tors},$ or if $[x] = 0$ in $H_1(D; \Z)$ and ${\rm scl}(x)>0$},\]
%where ${\rm scl}: \pi_1(\mathcal LD) \to \R_{\geq 0}$ denotes the stable commutator length of $x$. Explicitly, if ${\rm scl}(x)>0$, then \cite[Lemma 2.3]{BS-scl} implies that $\{x^m\}_{m \in \Z}$ are different elements in $\pi_0(\mathcal LM)$. For each $m \in \Z$ and for the equation $x^m = y^l$, we claim that it has only one solution for each $y \in \pi_0(\mathcal LM)$. In fact, if there exist $l_1 \neq l_2$ but with
%\[y^{l_1}  =y^{l_2} \,\,\,\mbox{in $\pi_0(\mathcal LM) = \pi_1(M)/{\rm conj}$},\] 
%then there exists some $t \in \pi_1(M)$ such that $t y^{l_1} t^{-1} = y^{l_2}$ in $\pi_1(M)$. Hence, by \cite[Lemma 2.3]{BS-scl} again, ${\rm scl}(y) =0$. However, $m\cdot{\rm scl}(x) = {\rm scl}(x^m) = {\rm scl}(y^l) = l \cdot{\rm scl}(y) = 0$, which implies that ${\rm scl}(x) = 0$. Contradiction. 
%
%\medskip
%
%(2) \jznote{This example is more general than (1) right above.} Recall Bavard's duality, which says that 
%\begin{equation} \label{bavard}
%{\rm scl} (x) = \frac{1}{2} \sup_{f \in Q^h(G) \backslash H^1(G)} \frac{|f(x)|}{D(f)}, 
%\end{equation}
%where $Q^h(G)$ is the set of all the quasimorphisms on the group $G$, $H^1(G)$ is the set of all the homomorphisms on the group $G$, and $D(f)$ is the defect of a quasimorphism $f$. \jznote{Observe that (\ref{bavard}) implies that ${\rm scl}$ is a class function. Explicitly, for any class $a \in G/{\rm conj}$, it is eligible to set 
%\[ {\rm scl}(a) : = {\rm scl}(x)\]
%for any representative $x$ of $a$. Indeed, by the definition of a homogeneous quasimorpshism, for any $x,y \in G$, we have $|f(xy) - f(x) - f(y)| \leq D(f)$ and $f(x^n) = nf(x)$. If $x = zyz^{-1}$, then 
%\[ n f(x) = n f(zyz^{-1}) = f(z y^n z^{-1}) \]
%while $|f(zy^nz^{-1}) - f(y^n)|\leq C$ for some constant $C$ that is independent of $n$. Therefore, we have $|nf(x) - nf(y)| \leq C$ which implies that $f(x) = f(y)$ after dividing $n$ and letting $n \to \infty$. This shows any homogeneous quasimorphism $f$ is a class function and so is ${\rm scl}$.} Moreover, (\ref{bavard}) implies that ${\rm scl}(x) >0$ if and only if there exists some homogeneous quasimorphism \jznote{(but not a homomorphism)} $f$ on $G$ such that $f(x) \neq 0$ \jznote{for some $x \in G$}. Therefore, \jznote{there exists a root-bounded class in $\pi_0(\mathcal LD)$} if $\pi_1(D)$ admits a non-zero homogeneous quasimorphism $f: \pi_1(D) \to \R$ with $f(x) \neq 0$, \jznote{and the requested class is $[x] \in \pi_0(\mathcal LD)$.}
%
%\medskip
%
%(3) By the K\"{u}nneth formula in symplectic cohomology \cite{Oancea-Kunneth} due to Oancea, the class of manifolds in Corollary \ref{cor: CCC} contains the cases where 
%\[ D =  \,\mbox{smoothings of products $D_0 \times D_1$},\]
%where $D_0$ is a connected Liouville domain with vanishing (total) symplectic cohomology, and $D_1$ is a Liouville domain with non-vanishing positive symplectic cohomology in a root-bounded class of orbits. For example, $D_0$ could be a Liouville domain whose completion is a subcritical Stein manifold, such as the standard disk $D^{2n} \subset \C^n,$ and $D_1$ could be the cotangent disk bundle $D_g^*L$ of a Finsler manifold $(L,g)$ with fundamental group admitting a non-zero homogeneous quasimorphism to $\R,$ (for instance, $\pi_1(L)$ \jznote{could be} equal to the Baumslag-Solitar groups ${\rm BS}(m,n) = \left< x, g\,|\, x g^m x^{-1} = g^n \right>,$ $m\neq n,$ by \cite{BS-scl}) or more generally if $\pi_0(\cl{L}L)$ admits a root-bounded element. 
%
%\medskip
%
%\jznote{(4) Here is a concrete example that one can verify the conditions in Theorem \ref{thm: geod} by hand. Let $L = \mathbb T^4 \# \overline{\C P^2}$, that is, a blow up of a $4$-torus. Then we claim that $L$ satisfies both conditions in Theorem \ref{thm: geod}. Indeed, we have 
%\[ \pi_2(L) = \pi_2(\T^4) \ast \pi_2(\C P^2) = \pi_2(\C P^2) \]
%and $H_2(L; \Z) = H_2(\T^4; \Z) \oplus H_2(\C P^2; \Z)$. So, the the Hurewicz map is simply the inclusion $\pi_2(\C P^2) \to H_2(\T^4; \Z) \oplus H_2(\C P^2; \Z)$, which is non-zero. This verifies condition (1). On the other hand, $\pi_1(L) = \pi_1(\T^4) = \Z^4$. In particular, it is abelian and infinite. Then Remark \ref{rmk-abelian} verifies condition (2). For this $L$, the conclusion obtained from Theorem \ref{thm: geod} is covered by the classical result by Bangert-Hingston in \cite{BaHi-Z}. One can consider a similar example 
%\[ L = \T^4 \# \T^4 \# \C P^2\]
%then a similar argument as above confirms that Theorem \ref{thm: geod} applies, with $\pi_1(L) = \Z^4 \ast \Z^4$, not in the scope of \cite{BaHi-Z}. However, for this $L$ one can also recover the conclusion of Theorem \ref{thm: geod}  simply from the exponential growth of the counting function $\xi(t)$ on the free group $\Z^4 \ast \Z^4$ (see (e), (f) in Section \ref{ssec-new-example}). }
%
%\medskip
%
%\jznote{(5) We list some examples where Theorem \ref{thm: geod} does {\it not} apply. If $L$ is simply connected, that is, $\pi_1(L) = 0$, or if $\pi_1(L)$ is finite, then condition (2) in Theorem \ref{thm: geod} fails due to Remark \ref{rmk-infinite-order}. This includes intensively-studied cases $S^n,$ $n \geq 2$ and $\C P^n$ (see \cite{}) as well as that of $\R P^n,$ for $n\geq 2.$ If $L = S^1$ then Theorem \ref{thm: geod} does not apply since $H_2(S^1; \Z) =0$ and in this case there are at most two geometrically distinct Reeb orbits on the unit cosphere bundle: it is diffeomorphic to $S^1 \sqcup S^1.$ Interestingly, here is another evidence that Theorem \ref{thm: geod} should not apply to those examples. Katok \cite{Kat73} and Ziller \cite{Zill77, Zill82} constructed Finsler metrics on 
%\[ S^n, \,\,\,\, \R P^n, \,\,\,\, \C P^n,\,\,\,\, \mathbb H P^n, \,\,\,\,\mbox{and}\,\,\,\, {\rm Ca} P^2, \]
%each of which admits only finitely many geometrically distinct closed geodesics. This is precisely the opposite of the conclusion in Theorem \ref{thm: geod}. Historically, these examples form the family of compact rank $1$ symmetric spaces (briefly denoted by {\rm CROSS's}). Nowadays it has became the central objects in the study of closed geodesics, since the classical Gromoll–Meyer assumption does not hold for these cases. So far, only limited results can be obtained \cite{BL05, DL07, Rad10, DLX15, XL15}.} \esnote{In further publications we will investigate how to apply our methods and machinery to {\rm CROSS's}.}
%
%\subsection{\jznote{Search for new examples for Theorem \ref{thm: geod}}} \label{ssec-new-example} We propose a closed manifold $L$, where Theorem \ref{thm: geod} applies. So, it admits infinitely many geometrically distinct closed geodesics. Importantly,  this example seems not covered by results listed in Remark \ref{rmk-lit-geod}. Explicitly, our proposed $L$ should have its fundamental group $\pi_1(L)$ satisfying {\bf all} of the following conditions. 
%\begin{itemize}
%\item{} $\pi_1(L)$ is finitely presented. 
%\item{} $\pi_1(L)$ is not solvable. 
%\item{} $\pi_1(L)$ does not admit any proper subgroup with finite index. 
%\item{} $\pi_1(L)$ admits at least one non-trivial homogenous quasi-morphism. 
%\item{} $\pi_1(L)$ has its conjugacy growth function exactly in a linear order. 
%\end{itemize}
%Unfortunately, we can neither confirm the existence of such $\pi_1(L)$ nor prove its non-existence. At present, we only know that seeking for such a group eventually leads to a difficult question in the group theory. In what follows, we will carry on our discussion in a few steps, explaining how conditions above can avoid those sufficient conditions listed in Remark \ref{rmk-lit-geod}. 
%
%\begin{itemize}
%\item[(a)] For any finitely presented group $G$. By a celebrated result from Gompf in \cite{Gom95}, one can construct a closed symplectic 4-manifold $L$ with $\pi_1(L) = G$. This provides us enormous flexibility in terms of choosing preferred group $G$. Moreover, this makes the sufficient condition (6) in Remark \ref{rmk-lit-geod} is off our discussion, simply due to the dimension reason.
%\item[(b)] By Remark \ref{rmk-infinite-order}, since $\{a^{k_i}\}_{i \in \N}$ are distinct elements in $\pi_0(\mathcal L L)$, the sufficient condition (7) in Remark \ref{rmk-lit-geod} is off our discussion. Also, our root-bounded element $a$ will not satisfy the sufficient condition (1) in Remark \ref{rmk-lit-geod}. It seems that the sufficient condition (4) are rather close to our assumption, but by Remark \ref{rmk-blow-up}, adding a blow-up as in (\ref{blow-up}) often eliminates the possibility that $\pi_2(L)$ is finitely generated.
%\item[(c)] Since the sufficient condition (5) in Remark \ref{rmk-lit-geod} covers both (2) and (3), another necessary constraint for us is to assume $\pi_1(L)$ is non-solvable. This is a rather loose constraint. For instance, we can take $\pi_1(L)$ to be a perfect group. 
%\item[(d)] To avoid the ``virtual'' conclusion from the covering trick, that is, the property in discussion is satisfied by a proper subgroup with finite index, we can consider those $\pi_1(L)$ which do not admit any proper (normal) subgroup with finite index at all. Meanwhile, (a) above implies that we are seeking for such a group that is also finitely presented. There exist desired examples but not easy to produce, for instance, Higman's group (see \cite{Hig74}) or Thompson's groups (see \cite{CFP96}), where the later one is the first example of infinite, finitely present, and simple group. 
%\item[(e)] Another constraint for us comes from a lower bound of the growth of the counting function $N(t)$ in (\ref{counting}). It is well-known that every free homotopy class contains at least one closed geodesic, and there is a one-to-one correspondence between the set of free homotopy classes $\pi_0(\mathcal LL)$ and the set of conjugacy classes $\pi_1(L)/{\rm conj}$. Therefore, the following counting function 
%\[ \xi(t) : =  \#\mbox{\{conjugacy classes $a$ with conjugacy length $\leq t$\}} \]
%provides a lower bound of $N(t)$. The behavior of function $\xi(t)$ is surprisingly complicated and difficult to compute in general (see, for instance, \cite{HO13,GS10}). The growth rate of $\xi(t)$ can vary from constant, polynomial, to exponential. There have been concrete constructions of groups for each growth rate.
%\item[(f)] Importantly, observe that if $\xi(t) \sim t^{\alpha}$ for $\alpha>1$, that is, superlinear, then automatically we have the conclusion of Theorem \ref{thm: geod}. On the other hand, the existence of a root-bounded element implies that $\xi(t)$ grows at least in a linear order. Therefore, we should consider $\pi_1(L)$ with $\xi(t) \sim t$ precisely. Unfortunately, it seems that $\xi(t)$ of both the Higman's group and Thompson's group are exponential growth. 
%\item[(g)] We have been informed by experts on group theory, via MathOverflow \cite{mathoverflow}, that a {\it finitely generated} group with exactly linear conjugacy growth do exist. This can be obtained by modifying some example from Ivanov. However, at present it is unknown if this example is finitely presented or not. 
%\end{itemize}
%
%\section*{Acknowledgements}
%We thank Viktor Ginzburg for suggesting the use of \cite{GShon} in a related context. E.S. was supported by an NSERC Discovery Grant, by the Fonds de recherche du Qu\'{e}bec - Nature et technologies, by Fondation Courtois and by a Sloan Research Fellowship. This work was partially supported by the National Science Foundation under Grant No. DMS-1928930, while E.S. was in residence at
%the Simons Laufer Mathematical Sciences Institute (previously known as MSRI) Berkeley, California during the Fall 2022 semester. E.S. thanks the organizers and participants of the Floer homotopy program and the SLMath staff for a lively research atmosphere. J.Z. is currently supported by USTC Research Funds of the Double First-Class Initiative. This work was initiated when J.Z. was a CRM-ISM Postdoctoral Research Fellow at CRM, University of Montreal, and he thanks this institute for its warm hospitality. Preliminary results of this paper were presented by J.Z. at a conference arranged by Nankai University in October 2022. He is grateful for the invitation from Huagui Duan, as well as many interesting discussions with him, Hui Liu, and Wei Wang.
%
%
%
%
%\section{Proofs of main theorems} \label{ssec-outline} 
%
%%Assuming a few results that will be proved in details in Section \ref{sec-smith}, we can give the proofs of Theorem \ref{thm: main 1} and Theorem \ref{thm: geod}. 
%
%In this section, we provide the proofs of main results, Theorem \ref{thm-main}, Theorem \ref{thm: main 1}, Theorem \ref{thm: geod} , and Theorem \ref{thm: stability}. Since the proofs are based on the refined equivariant symplectic cohomology (\ref{intro-symp-coh}), we will assume all its properties, collected in subsection \ref{ssec-refined-symp-coh} and \ref{ssec-equiv-pants-product}, with their proofs in full details postponed to later sections. 
%
%\subsection{Preparations} This subsection will briefly elaborate on the necessary concepts and facts that will be used in the proofs of main results.
%
%\subsubsection{Refined equivariant symplectic cohomology} \label{ssec-refined-symp-coh} Associated to a Liouville domain $(D, \lambda)$, a symplectic cohomology group with {\it any} given local system is constructed in subsection \ref{sec-sh}. The $\Z/(p)$-action shows up in two different ways, which results in two different (but closely related) cohomology theories. 
%
%\medskip
%
%\noindent{\bf $\Z/(p)$-equivariant Floer cohomology}. The $\Z/(p)$-equivariant version of symplectic cohomology goes as follows, 
%\begin{equation} \label{Zp-symp-coh}
%{\rm SH}^*_{a, \Z/(p)}(D; \nu) = \varinjlim_{i \to \infty}{\rm HF}_{a, \Z/(p)}^{*}((H_{i})^{(p)}; \nu),
%\end{equation}
%where $(H_{i})^{(p)}$ denotes the $p$-th iterate of a linear Hamiltonian $H_i$ (see (\ref{p-Ham}) and (\ref{linear-Hamiltonian})). The construction of $\Z/(p)$-equivariant Hamiltonian Floer cohomology 
%\begin{equation} \label{Zp-HF}
%{\rm HF}_{a, \Z/(p)}^{*}(H^{(p)}; \nu)
%\end{equation}
%is the key and occupies most of the Section \ref{sec-k-sh}. The $\Z/(p)$-action is given by rotating the Hamiltonian orbits as in (\ref{intro-rot}). Note that this $\Z/(p)$-action is well-defined exactly due to the $p$-th iterate $H^{(p)}$. Moreover, importantly, the cohomology group (\ref{Zp-HF}) is well-defined only if the given local system $\nu$ is $\Z/(p)$-equivariant (see Definition \ref{dfn-p-inv}). When restricted to a given action window, the following proposition reduces the symplectic cohomology to a Hamiltonian Floer cohomology. 
%
%\begin{prop} \label{prop-zp-sh-fh} Let $(D, \lambda)$ be a non-degenerate Liouville domain, $p$ be a prime number, and $a \in \pi_0(\mathcal LD)$. Assume ${\rm SH}_{a, \Z/(p)}^*(D; \nu)$ is defined via a sequence of linear Hamiltonians $\{H_i\}_{i \in I}$. Then for any finite interval $(s,t] \subset \R$ with $s, t \notin {\rm spec}(D, \lambda)$, there exists some $i \in I$ such that 
%\[ {\rm SH}_{a, \Z/(p)}^{*, (s,t]}(D; \nu) \simeq  {\rm SH}_{a, \Z/(p)}^{*, (s,t)}(D; \nu) = {\rm HF}_{a, \Z/(p)}^{*, (s,t)}(H^{(p)}_i; \nu). \]
%\end{prop}
%
%\begin{proof} This directly from analogue arguments as in Lemma \ref{sh-hf} and Lemma \ref{sh-crit}. \end{proof}
%
%Following Section 6 in \cite{SZhao}, the underlying cochain complex of (\ref{Zp-HF}) is ${\rm CF}^*(H^{(p)}; \nu) \otimes _{\F_p} \F_p[[u]]\left<\theta\right>$ with a deformed differential (see subsection \ref{ssec-differential}), where $u$ is a formal variable of degree $2$ and $\theta$ is another formal variable of degree 1 and satisfying $\theta^2 =1$. By inverting variable $u$, which is realized by $\otimes \F_p((u))\left<\theta\right>$, one obtains a {\it Tate version} of this Hamiltonian Floer cohomology, and it leads to a Tate version of $\Z/(p)$-equivariant symplectic cohomology. In terms of notations, 
%\begin{align} \label{tate-sh-1}
%\widehat{{\rm SH}}_{a, \Z/(p)}^{*}(D; \nu) : & = \varinjlim_{i \to \infty}\widehat{{\rm HF}}_{a, \Z/(p)}^{*}((H_{i})^{(p)}; \nu) \\  \nonumber
%& =\varinjlim_{i \to \infty} \left( {\rm HF}_{a, \Z/(p)}^{*}((H_{i})^{(p)}; \nu) \otimes_{\F_p[[u]]\left<\theta\right>} \F_p((u))\left<\theta\right>\right) \\ \nonumber
%& \left(= {\rm SH}_{a, \Z/(p)}^{*}(D; \nu) \otimes_{\F_p[[u]]\left<\theta\right>} \F_p((u))\left<\theta\right>\right).\nonumber
%\end{align}
%The last line holds since the direct limit commutes with the tensor product. 
%
%\medskip
%
%\noindent{\bf $\Z/(p)$-equivariant group cohomology}. Since the $\F_p$-module ${\rm CF}^*(H^{(p)}; \nu)$ admits a $\Z/(p)$-action induced by $R_p$ as in (\ref{intro-rot}) on the generators, one can consider the classical $\Z/(p)$-equivariant group cohomology defined by the cohomology of a cochain complexes, denoted by $\left(C^*(\Z/(p); {\rm CF}^*(H^{(p)}; \nu)); d \right)$, where $d$ is a modification of the standard Floer differential of ${\rm CF}^*(H^{(p)}; \nu)$. Since ${\rm CF}^*(H^{(p)}; \nu)$ is a module over $\mathbb F_p[\Z/(p)]$, this group cohomology can be regarded as an $\F_p[[u]]\left<\theta\right>$-module. The explicit construction is in Section 2 in \cite{SZhao}. The resulting group cohomology is denoted by  
%\begin{equation} \label{tate-hf}
%H^*\left(\Z/(p); {\rm HF}^*(H^{(p)}; \nu)\right). 
%\end{equation}
%By running the standard procedure of taking the direct limit, we get a group cohomology associated to a given Liouville domain $(D, \lambda)$, that is, 
%\begin{equation} \label{tate-sh}
%{{H}}^*\left(\Z/(p); {\rm SH}_a^{*} (D; \nu)\right): =  \varinjlim_{i \to \infty}{H}^*\left(\Z/(p); {\rm HF}_a^*(H_i^{(p)}; \nu)\right). 
%\end{equation}
%Similarly to the operation that inverts the variable $u$ above, we obtain the classical Tate group cohomology, 
%\begin{equation} \label{tate-hf}
%\widehat{H}^*\left(\Z/(p); {\rm HF}^*(H^{(p)}; \nu)\right) \,\,\,\,\mbox{and}\,\,\,\, {\widehat{H}}^*\left(\Z/(p); {\rm SH}_a^{*} (D; \nu)\right).
%\end{equation}
%
%
%When restricted to a finite action window $(s,t] \subset \R$, Proposition \ref{prop-zp-sh-fh} works identically the same in this case, so we have the following more effective way to compute symplectic Tate group cohomology, that is, 
%\begin{equation} \label{tate-sh-hf}
%{\widehat{H}}^*\left(\Z/(p); {\rm SH}_a^{*, (s,t]} (D; \nu)\right) \simeq \widehat{H}^*\left(\Z/(p); {\rm HF}_a^{*, (s,t]}(H_i^{(p)}; \nu)\right)
%\end{equation}
%for some $i$ in a given index set. 
%
%\medskip
%
%\noindent{\bf Relation of two cohomologies}. The isomorphism (\ref{tate-sh-hf}) is fundamentally important since by the argument in subsection 6.2 in \cite{SZhao}, in particular, by an application of the homological perturbation lemma in \cite{Mar01}, $\widehat{H}^*(\Z/(p); {\rm HF}_a^{*, (s,t]}(H_i^{(p)}; \nu))$ serves as the $E_2$-page of a spectral sequence such that, with a certain differential, it computes the Tate version of the $\Z/(p)$-equivariant symplectic cohomology defined in (\ref{tate-sh-1}) and restricted in the same action window. Hence, we obtain the following essential inequality that links these two Tate versions of symplectic cohomologies. 
%\begin{equation} \label{two-tates}
%\dim_{\F_p((u))} \widehat{{\rm SH}}_{a, \Z/(p)}^{*, (s, t]}(D; \nu) \leq \dim_{\F_p((u))} {\widehat{H}}^*\left(\Z/(p); {\rm SH}_a^{*, (s,t]} (D; \nu)\right).
%\end{equation} 
%This should be regarded as a direct analogue of (72) in \cite{SZhao}. 
%
%\subsubsection{Equivariant pants product}\label{ssec-equiv-pants-product} When the given local system is not only $\Z/(p)$-equivariant, but also $p$-admissible (see Definition \ref{dfn-p-adm}), on the level of cochain complexes one defines a linear map over the ground field $\K = \F_p$,  
%\begin{equation} \label{chain-pp}
%{\rm CF}^{*, (s,t]}_a(H; \nu)^{\otimes^p} \to {\rm CF}^{*, (ps, pt]}_{a^p}(H^{(p)}; \nu)
%\end{equation}
%for any given action window $(s,t]$, by considering the pants product as in subsection 8.1 in \cite{SZhao}. For more details, see subsection \ref{ssec-Zp-pp}. Note that both sides above admit $\Z/(p)$-actions. In particular, the action on the left-hand side is simply the permutation of the input $p$-tuples. Subsection 8.3 in \cite{SZhao} updates the map in (\ref{chain-pp}) to a well-defined map over the extended coefficient $\F_p[[u]]\left<\theta\right>$ as follows, 
%\begin{equation} \label{2-p-ends-coh}
%\mathcal P_{\nu}: H^*\left(\Z/(p);  {\rm CF}^{*, (s,t]}_a(H; \nu)^{\otimes p}\right) \to {\rm HF}_{a^p, \Z/(p)}^{*, (ps,pt]}(H^{(p)}; \nu).
%\end{equation}
%Since we work over field $\K = \F_p$, the K\"unneth formula says $H^*({\rm CF}_{a}^{*, (s,t]}(H; \nu)^{\otimes p}; \F_p) \simeq {\rm HF}_a^{*, (s,t]}(H; \nu)^{\otimes p}$ via a $\Z/(p)$-equivariant quasi-isomorphism, (\ref{2-p-ends-coh}) is equivalent to the following map $\mathcal P_{\nu}: H^*\left(\Z/(p);  {\rm HF}^{*, (s,t]}_a(H; \nu)^{\otimes p}\right) \to {\rm HF}_{a^p, \Z/(p)}^{*, (ps,pt]}(H^{(p)}; \nu)$. Again, by running the standard procedure of taking the direct limit, we pass to the symplectic cohomologies 
%\begin{equation} \label{P-nu-SH}
%\mathcal P_{\nu, {\rm sh}}: H^*\left(\Z/(p);  {\rm SH}^{*, (s,t]}_a(D; \nu)^{\otimes p}\right) \to {\rm SH}_{a^p, \Z/(p)}^{*, (ps,pt]}(D; \nu).
%\end{equation}
%
%\begin{rmk} On both sides of the map $\mathcal P_{\nu}$ in (\ref{P-nu-SH}), we arrive at the symplectic cohomologies associated to the Liouville domain $(D, \lambda)$, but through different families of Hamiltonians - one is $\{H_i\}_{i \in I}$ and the other is $\{(H_i)^{(p)}\}_{i \in I}$. \end{rmk}
%
%Unfortunately, besides it is well-defined, in general the map $\mathcal P_{\nu}$  in (\ref{P-nu-SH}) is not very useful. However, by passing to the Tate versions, Proposition \ref{prop-zp-sh-fh} above and Theorem A in \cite{SZhao} imply the following isomorphism over the extended field $\F((u))\left<\theta\right>$, 
%\begin{equation} \label{eq-sh-iso-2}
%\mathcal P_{\nu, {\rm sh}}: {\widehat H}^*\left(\Z/(p); {\rm SH}_a^{*, (s,t]} (D; \nu)^{\otimes p}\right) \xrightarrow{\simeq} {\widehat {\rm SH}}_{a^p, \Z/(p)}^{*, (ps,pt]}(D; \nu).
%\end{equation}
%The will be the most important step towards the proof of Theorem \ref{thm-main}. 
%
%\subsubsection{Local symplectic cohomology} \label{ssec-local-symp-coh} The definition of a local symplectic cohomology (of a closed Reeb orbit $\gamma$), denoted by ${\rm SH}^*_{\rm loc}(\gamma)$, appears in \cite{McLean-Reeb,Ginzburg-CC,GG20,SZhao}, etc. As indicated by its name, it is defined as a local version of symplectic cohomology, by starting from a contact manifold as a tubular neighborhood of the Reeb orbit $\gamma$, identified with $S^1 \times B^{2n-2}$ where $2n = \dim D$. Then, passing to the ``short'' symplectization $W = (1-\delta, 1+\delta) \times S^1 \times B^{2n-2}$ with coordinates $(r, \theta, z)$ and sufficiently small $\delta>0$, an appropriately-chosen Hamiltonian $f = f(r)$ on $W$ (depending only on $r$) generates a Hamiltonian flow in $W$ with fixed points of its time-1 map precisely equal to $\{1\} \times \gamma \times 0$. The local symplectic cohomology is defined by 
%\[ {\rm SH}^*_{\rm loc}(\gamma) : = {\rm HF}^{\rm *}_{\rm loc}(\gamma, f). \]
%Here, ${\rm HF}^{\rm *}_{\rm loc}(\gamma, f)$ denotes the standard Hamiltonian Floer cohomology with a (in fact, any, cf.~Lemma 2.5 in \cite{McLean-Reeb}) Hamiltonian $\tilde{f}$, a $C^2$-small perturbed from $f$, on the symplectic manifold $W$, so that the fixed points of the Hamiltonian diffeomorphism $\phi^1_{\tilde{f}}$ are perturbed from $\{1\} \times \gamma \times 0$ and importantly isolated. 
%
%The construction of local symplectic cohomology ${\rm SH}^*_{\rm loc}(\gamma)$ can be easily adapted to various version of the symplectic cohomology, in particular, to the equivariant symplectic cohomology that we will frequently apply in this paper. For more details, see subsection 9.2 in \cite{SZhao}. On the other hand, assume $\gamma$ is a prime closed Reeb orbit and $\gamma^k$ denotes its $k$-th iterate. One may wonder how ${\rm SH}^*_{\rm loc}(\gamma^k)$ changes when the multiplicity $k$ increases. Theorem 3.1 in \cite{McLean-Reeb} (as a version of Theorem 1.1 in \cite{Ginzburg-CC}) proves that there exists a constant $C>0$, only depending on $\gamma$, in particular, independent of $k$, such that 
%\begin{equation} \label{McL-bound-rank}
%{\rm rank}_{\K} \, {\rm SH}^*_{\rm loc}(\gamma^k) \leq C \,\,\,\,\mbox{for any $k \in \N$} 
%\end{equation}
%where $\K$ denotes the ground field in the construction of ${\rm SH}^*_{\rm loc}(\gamma)$.
%
%Moreover, a relation between the local cohomology ${\rm SH}^*_{\rm loc}(\gamma)$ and the global cohomology ${\rm SH}^*(D)$ is given by a spectral sequence. More explicitly, for any action window $(a,b) \subset (\R \backslash{{\rm Spec}(D)}) \cup\{\pm \infty\}$ (so, including the case where $(a,b) = \R$), we have 
%\begin{equation} \label{spectral sequence}
%\mbox{$E_1$-page} = \bigoplus_{a \leq \mathcal A(\gamma) \leq b} {\rm SH}^*_{\rm loc}(\gamma) \Longrightarrow {\rm SH}^{*, (a,b)}(D) 
%\end{equation}
%where $\gamma$ goes through all the closed Reeb orbits, not necessarily prime, that satisfy the desired action constraint. This leads to the following two observations. One, if $\gamma$ is homological visible in ${\rm SH}^*(D)$, then a necessarily condition is that ${\rm rank}_{\K} \, {\rm SH}^*_{\rm loc}(\gamma)>0$. Second, the total number of generators in the computation of ${\rm SH}^{*, (a,b)}(D)$ (not its rank, since there will be possible cancellations in the convergence process in (\ref{spectral sequence})), if it is finite, satisfies the following relation 
%\begin{equation} \label{total generator}
%\left(\begin{array}{ll} \mbox{total number of generators in} \\ \mbox{the computation of ${\rm SH}^{*, (a,b)}(D)$}\end{array} \right) = \sum_{a \leq \mathcal A(\gamma) \leq b} {\rm rank}_{\K} \, {\rm SH}^*_{\rm loc}(\gamma).
%\end{equation}
%In what follows, we will see that (\ref{McL-bound-rank}) and (\ref{total generator}) are helpful in the proof of Theorem \ref{thm: main 1}.
%
%\subsubsection{Symplectic persistence modules}\label{ssec-spm} Another important ingredient in the proof of Theorem \ref{thm: main 1} is the persistence module theory. A persistence module (over $\K$) is a functor $V: \R \to {\rm Vect}_{\K}$ satisfying certain regularity conditions, where ${\rm Vect}_{\K}$ denotes the category of $\K$-modules. Here, the domain $\R$ can be replaced by a subset of $\R$. A key regularity condition requires that for any $t \in \R$, the image $V^t$ is a {\it finite-dimensional} $\K$-module. By viewing $V$ as an $\R$-parametrized quiver over $\K$, a modern decomposition theorem from \cite{CrawBo} says that 
%\[ V = \bigoplus_{(a,b] \in {\mathbb B}(V)} \K_{(a,b]}\]
%in a unique way, where each $\K_{(a,b]}$ is an indecomposable quiver over $\K$ which can be identified with the interval $(a,b]$, and the collection of all the intervals in this decomposition is denoted by $\mathbb B(V)$, called the {\it barcode of $V$}. Each $(a,b] \in \mathbb B(V)$ usually has a strong geometric meaning. For instance, in terms of the Morse theory of the pair $(X, f)$ where $f: X \to \R$ is a Morse function, $V^t$ is obtained by the homology of sub-level set $\{f <t\}$. Then $(a,b]$ indicates that there is a generator of the Morse chain complex that was born at time $a$ and will die at time $b$. In particular, the classical Morse homology is ${\rm HM}_*(X, f)$ can be viewed as $V^{\infty}$. 
%
%There are various ways to use the information from $\mathbb B(V)$. For this paper, we will mainly focus on the situation where the persistence module $V$ dose not admit any infinite length bars (and the associated geometric meaning is that the full (co)homology vanishes). Denote 
%\[ \beta_{\rm tot}(V) = \sum_{(a,b] \in \mathbb B(V)} (b-a), \]
%the total length of the bars in $\mathbb B(V)$. Furthermore, we always assume that $\beta_{\rm tot}$ is finite, and we observe that 
%\begin{equation} \label{tot-int}
%\beta_{\rm tot}(V) = \int_{-\infty}^{\infty} {\rm dim}_{\K} (V^t) dt
%\end{equation}
%where regularity conditions of $V$ guarantee that counting pointwise dimension over $\K$, that is, ${\rm dim}_{\K} (V^t)$, is an integrable function of $t$. 
%
%\medskip
%
%Persistence module theory can be applied broadly in various areas in mathematics since many mathematical theories can be naturally formulated in terms of persistence modules. The following lemma provides such an example. 
%
%\begin{lma} \label{sh-pm} Let $(D, \lambda)$ be a non-degenerate Liouville domain, $\nu$ be a rank-1 local system on the free loop space $\mathcal LD$ over a field $\K$ with the trivial valuation, and $a \in \pi_0(\mathcal LD)$ with $a \neq \{{\rm pt}\}$. Then the map $V_{a}(D): \R_+ \to {\rm Vect}_{\K}$ defined by 
%\[ V^t_{a}(D) : = {\rm SH}^{(0, t]}_{a}(D; \nu)\]
%defines a persistence module over $\K$. \end{lma}
%
%\begin{proof} This is directly from Lemma \ref{sh-hf} and Lemma \ref{sh-crit}. \end{proof}
%
%We call the persistence module $V_{a}(D)$ provided by Lemma \ref{sh-pm} the {\it symplectic persistence modules of the Liouville domain $(D, \lambda)$ in class $a$ with local system $\nu$}. It is easy to verify that it is an invariant under symplectomorphisms. 
%
%\begin{rmk} Note that requiring $a \neq \{{\rm pt}\}$ in Lemma \ref{sh-pm} is to eliminate the issue on the action window including $0$ or not. On the other hand, in recent works \cite{VukasinJun, Usher-BM}, similar constructions as in Lemma \ref{sh-pm} have been studied but with a reparametrization of the variable $t$ to $\ln \,t$. This is essential for stability-type results, for instance Theorem \ref{thm: stability}, when comparing two Liouville domains in a quantitative way. Importantly, if $a \neq \{{\rm pt}\}$, then for the reparametrized barcode $\mathbb B^{\rm ln}(V_a(D))$, infinite length intervals inside only come from those with right endpoint being $+\infty$. \end{rmk}
%
%To end this subsection, let us point out another way to use the information $\mathbb B(V)$. Given two persistence modules $V$ and $W$ (for instance, by Lemma \ref{sh-pm}, symplectic persistence modules of two Liouville domains), a quantitative way to compare their barcodes $\mathbb B(V)$ and $\mathbb B(W)$ is the bottleneck distance, denoted by $d_{\rm bot}$. Roughly speaking, it detects the minimal amount of adjustment on the endpoints of intervals in $\mathbb B(V)$ so that it matches $\mathbb B(W)$ (and vice versa). For its precise definition, see Definition 2.2.3 in \cite{PRSZ20}. Note that $d_{\rm bot}(V,W)$ can easily be $+\infty$ if the cardinalities of infinite length intervals in $\mathbb B(V)$ and $\mathbb B(W)$ do not match. When $d_{\rm bot}(V,W)$  is bounded, it often serves as the lower bound of some geometric comparison. This comes from a deep result - isometry theorem - in persistence module theory. The relation (\ref{stability}) is a typical example. More examples in this direction can be found in Section 2.3 and Theorem 8.2.5 in \cite{PRSZ20}. 
%
%\subsubsection{Albers-Frauenfelder-Oancea's local system}\label{ssec-AFO-ls}
%
%As advertised in the introduction part, an essential part of the proof of Theorem \ref{thm: geod} is a clever choice of the local system $\nu$ such that the {\it full} symplectic cohomology ${\rm SH}^*(D; \nu) = 0$ for the Liouville domain $D = D_g^*L$. In fact, this has been achieved by \cite{AFO17} under the topological condition - non-zero of the Hurewicz map -  in the statement of Theorem \ref{thm: geod}. 
%
%For our use later, we need a more explicit description of the local system produced by \cite{AFO17}. Although the exposition in \cite{AFO17} is only on $\mathcal L_0L$, the same argument works for any connected component $\mathcal L_{\alpha} L$. Starting from a reformulation of the abelian group formed by the local systems over $\Z/(p)$ on $\mathcal LL$, that is, ${\rm Hom}(\pi_1(\mathcal LL); \Z/(p)) $, we have the following identification, 
%\begin{equation} \label{abelian-identification}
%{\rm Hom}(\pi_1(\mathcal LL); \Z/(p)) \simeq H_{\rm inv}^2(\tilde{L}; \Z/(p)) \times {\rm Hom}(\pi_1(L); \Z/(p)). 
%\end{equation}
%where $\tilde{L}$ is the universal cover of $L$ and $H_{\rm inv}^2(\tilde{L}; \Z/(p))$ is the subgroup of $\pi_1(L)$-invariant elements in $H^2(\tilde{L}; \Z/(p))$ (cf.~(2) and (3) in \cite{AFO17}). 
%\begin{rmk} Strictly speaking, $\mathcal LM$ is a disjoint union of various connected components $\mathcal L_{\alpha} M$ for loop classes $\alpha$, and $\pi_1(\mathcal LM)$ should be more precisely written as $\pi_1(\mathcal L_{\alpha} M, \eta_{\alpha})$ for a based point $\eta_{\alpha}$. However, different choices of base point result in inner automorphisms (by conjugations) of ${\rm Hom}(\pi_1(\mathcal L_{\alpha} M, \eta_{\alpha}); \Z/(p))$, so for brevity we will just use ${\rm Hom}(\pi_1(\mathcal L_{\alpha} M); \Z/(p))$. Moreover, once a loop $x$ is given in $\mathcal LM$, there exists a unique class $\alpha$ such that $x \in \mathcal L_\alpha M$. Importantly, for loop classes $\alpha$, the fundamental groups $\pi_1(\mathcal L_{\alpha} M)$ are isomorphic, since the based loop spaces $\Omega_\alpha M$ (as an $H$-space) are all homotopy equivalent (by Exercise 3 in Section 3.C in \cite{Hat02}) and $\pi_1(\mathcal L_{\alpha} M) \simeq \pi_1(\Omega_{\alpha} M) \rtimes \pi_1(M)$. Therefore, the notation $\pi(\mathcal LM) : = \pi(\mathcal L_{\alpha} M, \eta_{\alpha})$ will not cause any ambiguity to ${\rm Hom}(\pi_1(\mathcal LM), \Z/(p))$.\end{rmk}
%
%The identification (\ref{abelian-identification}) tells us that one way to cook up a nonzero local system is to produce an element in the form of $(\nu, 0)$, where $\nu$ is a non-zero element in $H_{\rm inv}^2(\tilde{L}; \Z/(p))$. An important observation is that 
%\begin{equation} \label{inv-inclusion}
%p^*H^2(L; \Z/(p)) \subset H^2_{\rm inv}(\tilde{L}; \Z/(p)). 
%\end{equation}
%where $p: \tilde{L} \to L$ is the covering map This leads a more concrete target that our desired element in $H^2_{\rm inv}(\tilde{L}; \Z/(p))$ can be obtained in the form of a pullback $p^*\tau$ for some $\tau \in H^2(L; \Z/(p))$. The difficulty is to guarantee that $p^*\tau \neq 0$. 
%
%To this end, we need some result on the group (co)homology. By Theorem $8^{\rm bis}$.~10 in \cite{McC01}, the group $H_{\rm inv}^2(\tilde{L}; \Z/(p))$ fits into an exact sequence,
%\begin{equation*} \label{CL-LSS}
%0 \to H^2(\pi_1(L); \Z/(p)) \xrightarrow{f} H^2(L; \Z/(p)) \xrightarrow{p^*} H^2_{\rm inv}(\tilde{L}; \Z/(p)) \to H^3(\pi_1(L); \Z/(p)) \to …
%\end{equation*}
%where $\Z$ is viewed as a trivial $\Z[\pi_1(L)]$-module. Remarkably, by Proposition 9 in \cite{AFO17}, non-zero of the Hurewicz map $h: \pi_2(L) \to H_2(L)$ is equivalent to non-surjectivity of $f$ in this exact sequence, so there exists some $\tau \in H^2(L; \Z/(p))$ such that $p^*\tau \neq 0$. 
%
%The local system $\nu = (p^*\tau, 0)$ satisfies 
%\begin{equation} \label{vanish-lL}
%H_*(\mathcal LL; \nu) =0
%\end{equation}
%by a cute fact - Proposition 3 in \cite{AFO17} - together with a fast-degenerated spectral sequence resulted from the standard fibration $\Omega M \hookrightarrow \mathcal LM \to M$. For this step, working over the coefficient $\Z/(p)$ for a prime $p$ is necessary. Then by Theorem 4.1.1 and Remark 4.1.2 in \cite{Abo15}, we obtain the following local system, still denoted by $\nu$ on $\mathcal L D_g^*L$ such that the refined (full) symplectic cohomology vanishes. Moreover, such a local system enjoys the following property. 
%
%\begin{prop} \label{prop-top-cond} The local system $\nu$ on $\mathcal L D_g^*L$ obtained via in (\ref{vanish-lL}) is $\Z/(p)$-equivariant and $p$-admissible. \end{prop}
%
%This result will be crucial to the construction and well-definedness of the refined symplectic cohomology (\ref{intro-symp-coh}), as well as its related property as in (\ref{P-nu-SH}). We postpone its proof to Section \ref{sec-lsc}. 
%
%\subsection{Proofs}
%
%\subsubsection{Proof of Theorem \ref{thm-main}} First, consider the quasi-Frobenius maps constructed in Section 3 in \cite{SZhao} to our set-up. It is the following diagonal map 
%\[ \sigma: {\rm SH}_a^{*, (s,t)} (D; \nu) \to {\rm SH}_a^{*, (s,t)} (D; \nu)^{\otimes p} \,\,\,\,\mbox{by} \,\,\,\, x\otimes b \to (x \otimes b)^{\otimes p}, \]
%where $x \otimes b$ represents a generator of ${\rm SH}_a^{*, (s,t)} (D; \nu)$. Note that the fixed free homotopy class does not change. Then Lemma 3.1 in \cite{SZhao} implies that $\sigma$ above induces an isomorphism of $\F_p((u))\left<\theta \right>$-modules, 
%\begin{equation*} \label{fro-iso}
%{\rm SH}_a^{*, (s,t)} (D; \nu) \otimes_{\F_p} \F_p((u))\left<\theta \right> \xrightarrow{\simeq} {\widehat H}^*(\Z/(p); {\rm SH}_a^{*, (s,t)} \left(D; \nu)^{\otimes p}\right).
%\end{equation*}
%Here, we use the fact that over $\F_p$ the only invertible Frobenius automorphism is the identity map, and the group cohomology ${\widehat H}^*(\Z/(p); {\rm SH}_a^{*, (s,t)} (D; \nu))$, originally appeared in Lemma 3.1 in \cite{SZhao}, is isomorphic to ${\rm SH}_a^{*, (s,t)} (D; \nu) \otimes_{\F_p} \F_p((u))\left<\theta \right>$ under the trivial $\Z/(p)$-action. Then we have the following estimation on dimensions.  
%\begin{equation} \label{2-rank-est}
%2 \dim_{\F_p} {\rm SH}_a^{*, (s,t)}(D; \nu) = \dim_{\F((u))} {\widehat H}^*\left(\Z/(p); {\rm SH}_a^{*, (s,t)} (D; \nu)^{\otimes p}\right). 
%\end{equation}
%Indeed, since the adjoint formal variable $\theta$ satisfies $\theta^2 = 1$, it provides two copies over the field $\F_p((u))$. 
%
%Next, recall that it suffices to prove (\ref{half-main}), that is, 
%\[ \dim_{\F_p} {\rm SH}_{a}^{*, \,(s,t)}(D; \nu) \leq \dim_{\F_p} \left({\rm SH}_{a^p}^{*, \, (ps,pt)}(D; \nu)\right)^{\Z/{(p)}}.\]
%By Proposition \ref{prop-zp-sh-fh}, let us identify ${\rm SH}_{a^p}^{*, (ps,pt)}(D; \nu)$ with ${\rm HF}_{a^p}^{*, (ps,pt)}(H^{(p)}; \nu)$ for some linear Hamiltonian $H$ so that ${\rm SH}_{a^p}^{*, (ps,pt]}(D; \nu)$ is endowed with a $\Z/(p)$-action induced by the rotation $R_p$ in (\ref{intro-rot}). In particular, ${\rm SH}_{a^p}^*(D; \nu)^{(ps, pt]}$ is a $\F_p[\Z/(p)]$-module. Then, by definition,
%\begin{align*}
%\left({\rm SH}_{a^p}^{*, \, (ps,pt)}(D; \nu)\right)^{\Z/{(p)}} &= \left\{x \otimes b \,| \, R_p (x \otimes b) = x \otimes b \right\}\\
%& = \Ker(R_p - \mathds{1}). 
%\end{align*}
%As argued in Section 12 in \cite{SZhao}, the $\F_p[\Z/(p)]$-module ${\rm SH}_{a^p}^{*, (ps,pt)}(D; \nu)$ decomposes into a direct sum $\bigoplus_{1 \leq k \leq p} (\F_p[t]/(t^k))^{\oplus m_k}$ where $t = R_p-\mathds{1}$ and $m_k \geq 0$. Then 
%\begin{align}\label{inequality-dim-2}
%\dim_{\F_p((u))} {\widehat H}^*\left(\Z/(p); {\rm SH}_{a^p}^{*, (ps,pt]} (D; \nu) \right) & \leq 2(m_1 + … + m_p) \\\nonumber
%& = 2 \dim_{\F_p} \left({\rm SH}_{a^p}^{*, \, (ps,pt)}(D; \nu)\right)^{\Z/{(p)}}.
%\end{align}
%
%Finally, by (\ref{2-rank-est}), (\ref{eq-sh-iso-2}), (\ref{two-tates}) and (\ref{inequality-dim-2}), we have 
%\begin{align*}
%2 \dim_{\F_p} {\rm SH}_a^{*, (s,t)}(D; \nu) & = \dim_{\F((u))} {\widehat H}^*\left(\Z/(p); {\rm SH}_a^{*, (s,t)} (D; \nu)^{\otimes p}\right)\\
%& = \dim_{\F((u))} {\widehat {\rm SH}}_{a^p, \Z/(p)}^{*, (ps,pt)}(D; \nu)\\
%& \leq \dim_{\F_p((u))} {\widehat{H}}^*\left(\Z/(p); {\rm SH}_{a^p}^{*, (s,t)} (D; \nu)\right)\\
%& \leq 2 \dim_{\F_p} \left({\rm SH}_{a^p}^{*, \, (ps,pt)}(D; \nu)\right)^{\Z/{(p)}}.
%\end{align*}
%This is the desired conclusion and thus we complete the proof. \qed
%
%\subsubsection{Proof of Theorem \ref{thm: main 1}} First, let us set up some notations. By Lemma \ref{sh-pm}, denote by $V_a(D)$ the symplectic persistence module of $(D, \lambda)$ in a free homotopy class $a$. By our hypothesis ${\rm SH}^*(D; \nu) = 0$ (in particular, ${\rm SH}_a^*(D; \nu) =0$), it makes sense to discuss the total length $\beta_{\rm tot}(V_a(D))$. Denote by 
%\[ K_a : = \#\{\mbox{intervals in $\mathbb B(V_a(D))$}\}. \]
%Then observe that the total number of generators in ${\rm SH}_a^*(D; \nu)$, the left-hand side in  (\ref{total generator}), is equal to $2K_a$. 
%
%Next, suppose that there are only finitely many prime closed Reeb orbits $\gamma_1, …, \gamma_m$ of $\partial D$, where $b_j = [\gamma_j]$ for $1 \leq j \leq m$. Then, by Definition \ref{dfn-root-bounded}, there exists a constant $C'>0$ such that for each $a^{p^i}$ and $b_j$, we have 
%\begin{equation*}
%\#\left\{l \in \Z_{>0} \,| \ a^{p^i} = b_j^l \right\} \leq C'. 
%\end{equation*} 
%Combined with the relations (\ref{McL-bound-rank}) and (\ref{total generator}), one concludes that, 
%\begin{equation} \label{finite-total-generator}
%2K_{a^{p^i}} = \sum_{[\gamma] = a^{p^i}} {\rm rank}_{\F_p} \, {\rm SH}^*_{\rm loc}(\gamma) \leq m \cdot C' \cdot C.
%\end{equation}
%where the upper bound holds for any $p^i$. 
%
%On the other hand, Theorem \ref{thm-main} implies that 
%\[p \cdot \beta_{\tot}(V_{a^{p^i}}(D)) \leq \beta_{\tot}(V_{a^{p^{i+1}}}(D))\] for all $i \geq 0.$ By our hypothesis $\dim_{\bF_p}{\rm SH}^*_{\loc}(\gamma) > 0,$ we conclude that the number of bars $K_a$ in $V_a(D)$ is positive, and furthermore $\beta_{\tot}(V_a(D)) > 0$. This implies that 
%\begin{equation} \label{p-growth}
%\beta_{\tot}(V_{a^{p^i}}(D)) \geq p^{i} \cdot \beta_{\tot}(V_a(D)).
%\end{equation}
%In other words, the total length $\beta_{\tot}$ grows at least linearly in $p^i$ when the class $a$ iterates in the form of $a^{p^i}$. 
%
%Finally, by the result \cite[Theorem C]{GShon}, there exists a constant $\kappa(D)>0$ such that all bars in $V_a(D)$ have length at most $\kappa(D)$. Therefore, (\ref{p-growth}) implies that the number of bars $K_{a^{p^i}}$ in $V_{a^{p^i}}(D)$ grows at least linearly 
%\begin{equation} \label{p-growth-number}
%K_{a^{p^i}} \geq c \cdot p^i,
%\end{equation}
%for $c = \beta_{\tot}(V(D)_{a})/ \kappa(D) > 0.$ This also holds for $2K_{a^{p^i}}$, the total number of generators of ${\rm SH}_{a^{p^i}}^*(D; \nu)$. This contradicts (\ref{finite-total-generator}) and we complete the proof. \qed
%
%\subsubsection{Proof of Theorem \ref{thm: geod}}
%The local system given by subsection \ref{ssec-AFO-ls}, together with Proposition \ref{prop-top-cond}, satisfies the first assumption of Theorem \ref{thm: main 1}. The second assumption of Theorem \ref{thm: main 1} can be obtained by considering a minimal energy closed geodesic in any given free homotopy class. Then, by the one-to-one correspondence between closed Reeb orbits and closed geodesics as discussed in subsection \ref{ssec-background}, we obtain the desired conclusion. \qed
%
%\subsubsection{Proof of Theorem \ref{thm: stability}} Since the Liouville embeddings considered in the definition of $d_{\rm SBM}$ induce the identity on the free loop space, the existence of a root-bounded class for $D_1$ also serves as a root-bounded class for $D_2$. Moreover, if ${\rm SH}^*(D_1, \nu) =0$, then (\ref{stability}) implies that ${\rm SH}^*(D_2, \nu) = 0$; otherwise $d_{\rm bot}(V_a(D_1), V_a(D_2)) = +\infty$, where $V_a(D_i) = \mathbb B^{\rm ln}({\rm SH}^*(D_i, \nu))$ or $i = 1,2$. Finally, choose 
%\[ d_{\rm SBM}(D_1, D_2)< \frac{\beta_{\tot}(V_a(D_1))}{2C}\]
%for a sufficiently large $C>0$. This implies, by (\ref{stability}) again, $\beta_{\tot}(V_a(D_2)) > 0$. Then we get the desired conclusion by the same argument as the proof of Theorem \ref{thm: main 1}. \qed
%
%\section{Refined symplectic cohomology}
%
%\subsection{Symplectic cohomology with a local system} \label{sec-sh} 
%
%Let $(D, \lambda)$ be a Liouville domain with boundary $S = \partial D$. By definition, the Liouville 1-form $\lambda$ restricts to a contact form on $S$. Any Liouville domain $(D, \lambda)$ sits inside a non-compact symplectic manifold called its completion (or the associated Liouville manifold) denoted by $({\hat D}, d\lambda)$. The geometric structure of $({\hat D}, d\lambda)$ is particularly easy. Explicitly, we have a decomposition $\hat{D} = {\rm Core}(\hat{D}) \sqcup (\R_+ \times S)$, where ${\rm Core}(\hat{D})$ is a subset of $\hat{D}$ and $\R \times S$ is identified with the symplectization of $S$. In particular, $S$ is identified with $\{1\} \times S$ in $\hat{D}$. A standard family of examples of $(D, \lambda)$ consists of any cotangent disk bundle $D_g^*L$ of a (closed) manifold $L$ with the canonical primitive $\lambda_{\rm can} = \sum p dq$ where $g$ is a fixed Riemannian or Finsler metric on $L$. Initiated by \cite{FH94, CFH95, Viterbo-specGF} with variations and developments as in \cite{BPS03, Sei08}, symplectic cohomology can be applied to study the contact dynamics of $S$ via a Floer-type machinery of the completion $(\hat{D}, d\lambda)$. In particular, the closed Reeb orbits of the contact manifold $(S, \lambda|_S)$ are precisely corresponding to the closed geodesics of $L$ with respect to the metric $g$. In this subsection, we will review the definition of the symplectic cohomology of $(D, \lambda)$ but with a local system $\nu$. This follows and upgrades Chapter 1 in \cite{Abo15}, where its (1.6.2) provides the definition of $D_g^*L$. Along this way, we also fix our preferred notations and conventions in the corresponding Floer theory. 
%
%\smallskip
%
%Given a Liouville domain $(D, \lambda)$, consider a {\it linear Hamiltonian} on $D$ in the sense that 
%\begin{equation} \label{linear-Hamiltonian}
%H|_{S \times [1, \infty)} = b \cdot \rho 
%\end{equation}
%for some $b \in \R$ (called the slope of $H$) and $\rho$ is the $\R_+$-coordinate of the symplectization $\R_+ \times S$ in the completion $({\hat D}, d\lambda)$. A generic Hamiltonian orbit of $H$ is denoted by $x \in \mathcal O(H)$. Under the non-degeneracy and orientation conditions, each $x$ admits a well-defined degree denoted by $|x|$. Let $\nu$ be a local system over a field $\K$ on the free loop space $\mathcal LD$. Here, the field $\K$ is endowed with the trivial valuation. Consider the following vector space over $\K$,
%\begin{equation} \label{hfh-ls}
%{\rm CF}^i(H; \nu) : = \bigoplus_{x \in \mathcal O(H); \,\,|x| = i} \K \left<x\right> \otimes_{\K} \nu_x.\end{equation} 
%After fixing {\it a priori} basis of $\K$, consider the $\K$-linear map $d: {\rm CF}^i(H; \nu) \to {\rm CF}^{i+1}(H; \nu)$ defined by 
%\begin{equation} \label{ls-diff}
%d_{\nu} (x \otimes b) = \sum_{{\tiny \begin{array}{c} y \in \mathcal O(H); \,\,|y| = i+1 \\ \mathcal M(y,x) \neq \emptyset \end{array}}} \sum_{u \in \mathcal M(y,x)} y \otimes \nu^{-1}_u(b), 
%\end{equation}
%where $b \in \K = \nu_x$. Here, $\mathcal M(y,x)$ denotes the unparameterized moduli space consisting of Floer connecting cylinders $u$ from the Hamiltonian orbit $y$ to the Hamiltonian orbit $x$ (caution: the order!). Moreover, by the definition of a local system, $\nu_{u}: \nu_y \to \nu_x$ is an automorphism of $\K$ for any such a connecting cylinder $u$. The same argument as Proposition 1.5.10 in \cite{Abo15} proves that $d_{\nu}$ is a differential. 
%
%\smallskip
%
%Let us take the following convention. Denote by $\mathcal A_H: \mathcal LD \to \R$ the action functional 
%\[ \mathcal A_H(x \otimes b) =-\int x^*\lambda + H_t(x) dt\,\,\,\,\mbox{and}\,\,\,\, \iota_{X_H} \omega = - dH.\]
%Here, we emphasize that the local system $\nu$ does not contribute to the action functional. Then Corollary 1.3.12 in \cite{Abo15} implies that the existence of a Floer connecting cylinder $u$ from $y$ to $x$ implies that $\mathcal A_H(y \otimes \nu^{-1}_u(b)) \geq \mathcal A_H(x \otimes b)$ for any $y \otimes \nu^{-1}_u(b)$ on the right-hand side of (\ref{ls-diff}). Hence, for any $s \in \R$, we have a well-defined filtered cochain complex $({\rm CF}^{*, (s, \infty)}(H; \nu), d_{\nu})$ where ${\rm CF}^{*, (s, \infty)}(H; \nu)$ is spanned by $x \otimes \mathds{1}_{\K}$ with the action greater than $s$. Moreover, for any $s<t\in \R$, define ${\rm CF}^{*, (s, t]}(H; \nu): = \frac{{\rm CF}^{*, (s, \infty)}(H; \nu)}{{\rm CF}^{*, (t, \infty)}(H; \nu)}$, and we have a well-defined truncated cochain complex $({\rm CF}^{*, (s, t]}(H; \nu), d_{\nu})$. The Floer cohomology of a linear Hamiltonian $H$ with local system $\nu$, denoted by ${\rm HF}^{*}(H; \nu)$, is defined by the cohomology of $({\rm CF}^*(H; \nu), d_{\nu})$. Accordingly, we have the truncated Floer cohomology 
%\begin{equation} \label{dfn-fil-hf-ls}
%{\rm HF}^{*, (s,t]}(H; \nu) = H^*\left({\rm CF}^{(s, t]}(H; \nu), d_{\nu}\right).
%\end{equation}
%
%For two linear Hamiltonians $H^-$ and $H^+$, denote by $H^+ \preceq H^-$ if and only if the slope of $H^+$ is no greater than the slope of $H^-$. Via an appropriate, in particular, monotone choice of the homotopy from $H^+$ to $H^-$, there exists a well-defined chain map ${c}_{\nu}: {\rm CF}^*(H^+; \nu) \to {\rm CF}^*(H^-; \nu)$, defined in a way that is similar to (\ref{ls-diff}) by $\otimes \nu_{u}$ for each connecting cylinder $u$. It descents to the continuation map $c_{\nu}^{+ \to -}: {\rm HF}^*(H^+; \nu) \to {\rm HF}^*(H^-; \nu)$ whenever $H^+\preceq H^-$. Moreover, for $H^+ \preceq H^0 \preceq H^-$, we have $c_{\nu}^{+ \to -} = c_{\nu}^{0 \to -} \circ c_{\nu}^{+ \to 0}$, which implies that if $H^+$ and $H^-$ share the same slope, then ${\rm HF}^*(H^+; \nu) \simeq {\rm HF}^*(H^-; \nu)$. Finally, since $c_{\nu}$ respects the action filtration (due to the monotone homotopy chosen above), it is readily to verify that $c_{\nu}^{+ \to -}$ restricts to well-defined maps on truncated Floer cohomologies, 
%\begin{equation} \label{cont-map-ls}
%c_{\nu}^{+ \to -, (s, t]}: {\rm HF}^{*, (s, t]} (H^+; \nu) \to {\rm HF}^{*, (s, t]}(H^-; \nu)
%\end{equation}
%n particular, when $H^+$ and $H^-$ share the same slope, the morphism $c_{\nu}^{+ \to -, (s, t]}$ is an isomorphism. 
%
%\begin{rmk} In general $c_{\nu}^{+ \to -, (s, t]}$ is {\it not} an isomorphism since there may not exist any continuation map induced by a homotopy from $H^-$ to $H^+$ that preserves the action filtrations. \end{rmk}
%
%Now, for any fixed homotopy class $a$ of loops in $\mathcal LD$, i.e., $a \in \pi_0(\mathcal LD)$, the construction above for $\mathcal O_{a}(H) (\subset \mathcal O(H))$ consisting of Hamiltonian orbits in class $a$ will result in a definition ${\rm HF}^{*, (s,t]}_{a}(H; \nu)$. It is a subspace of ${\rm HF}^{*, (s,t]}(H; \nu)$, and it enjoys properties involving the continuation maps above. Note that the continuation maps preserve the homotopy classes. 
%
%\begin{df} \label{dfn-sh-ls} Let $(D, \lambda)$ be a Liouville domain and $\nu$ be a local system over a field $\K$ (with the trivial valuation) on the free loop space $\mathcal LD$. For a homotopy class $a\in \pi_0(\mathcal LD)$, and for {\it any} sequence of linear Hamiltonians $\{H_i\}_{i \in \N}$ on $({\hat D}, d\lambda)$ where the slopes of $H_i$ increasing to infinity, the {\bf symplectic cohomology of $(D, \lambda)$ in class $a$ with local system $\nu$ and with action window $(s,t]$} is defined by 
%\begin{equation*} 
%{\rm SH}_{a}^{*, (s,t]}(D; \nu) := \varinjlim_{i \to \infty} {\rm HF}_{a}^{*, (s,t]}(H_i; \nu)
%\end{equation*}
%where the direct limit is taken over the continuation maps as (\ref{cont-map-ls}). 
%\end{df}
%
%One can verify that ${\rm SH}_{a}^{*, (s,t]}(D; \nu)$ is well-defined in that sense that it is independent of the defining sequence of linear Hamiltonians in Definition \ref{dfn-sh-ls}. On the other hand, as a direct consequence of the definition of a direct limit, for instance, (i) in Lemma 4.7.1 in \cite{BPS03}, the following proposition simplifies the definition of ${\rm SH}_{a}^{*, (s,t]}(D; \nu)$ from a computational perspective (cf.~subsection 9.7 in \cite{PRSZ20}). 
%
%\begin{lma} \label{sh-hf} Assume that ${\rm SH}_{\alpha}^{*, (s,t]} (D; \nu)$ in Definition \ref{dfn-sh-ls} is defined via a sequence of the linear Hamiltonians $\{H_i\}_{i \in \N}$, then there exists some $i \in \N$ such that 
%\[ {\rm SH}_{a}^{*, (s,t]}(D; \nu) = {\rm HF}_{a}^{*, (s,t]}(H_i; \nu). \]
%In other words, the truncated symplectic cohomology can be computed from a single linear Hamiltonian and its associated truncated Hamiltonian Floer cohomology. \end{lma}
%
%To end this subsection, we make the following observation that can simplify some of our discussions in this paper. Recall that ${\rm spec}(D, \lambda)$ denotes the set of actions of closed Reeb orbits of $(S, \lambda|_S)$, the spectrum of $(D, \lambda)$. The following result can be deduced directly from Lemma \ref{sh-hf}.
%
%\begin{lma} \label{sh-crit} For any finite $s, t \notin {\rm spec}(D, \lambda)$, we have ${\rm SH}_{a}^{*, (s,t]}(D; \nu) \simeq {\rm SH}_{a}^{*, (s, t)}(D; \nu)$. \end{lma}
%
%
%\subsection{Equivariant symplectic cohomology} \label{sec-k-sh}
%
%In this subsection, we will define the $\Z/{(p)}$-equivariant (truncated) symplectic cohomology of a Liouville domain $(D, \lambda)$ with a $\Z/(p)$-equivariant local system $\nu$ over the field $\F_p$. This is based on the $\Z/{(p)}$-equivariant (truncated) Hamiltonian Floer cohomology defined in Section 6 in \cite{SZhao}. There will be two major improvements. One, we add the local system $\nu$ in the constructions in \cite{SZhao}; second, as we have seen in Section \ref{sec-sh}, passing from Hamiltonian Floer cohomology to symplectic cohomology in general needs a direct limit process, to this end, particularly in the $\Z/{(p)}$-equivariant set-up, we will generalize the equivariant continuation map construction in Section 3 in \cite{Wil18} from $\Z/{(2)}$ to $\Z/{(p)}$ for a general prime $p$.  
%
%\subsubsection{$\Z/(p)$-equivariant local system} Let us start from the following definition of a local system being $\Z/(p)$-equivariant. 
%
%\begin{df} \label{dfn-p-inv} Let $p$ be a prime and $\K$ be a ground field. Consider the following two maps 
%\[ \Z/(p) \times \mathcal LD \xrightarrow{\pi_2} \mathcal LD \,\,\,\,\mbox{and}\,\,\,\, \Z/(p) \times \mathcal LD \xrightarrow{r_p} \mathcal LD \]
%where $\pi_2$ is the projection on the second factor and $r_p$ is the reparametrization of a loop $x(t)$ by rotations, that is, 
%\[ r_p((m, x(t))) = x(t + \frac{m}{p}) \]
%for any $m \in \Z/(p)$. A local system $\nu$ over $\K$ on $\mathcal LD$ is called {\bf $\Z/(p)$-equivariant} if there exists a family of isomorphisms of rank-1 $\K$-modules
%\begin{equation} \label{p-equiv-ls}
%\sigma_{m, x(t)}: \nu_{x(t)} \simeq \nu_{x\left(t+\frac{m}{p}\right)} 
%\end{equation}
%for every point $(m, x(t)) \in \Z/(p) \times \mathcal LD$, and these isomorphisms satisfy the following conditions.
%\begin{itemize}
%\item[(i)]The isomorphism $\sigma_{\cdot, \cdot}$ is continuous in both $m \in \Z/(p)$ (where $\Z/(p)$ is endowed with the discrete topology) and $x(t) \in \mathcal LD$.
%\item[(ii)] For any $m, n \in \Z/(p)$ and $x(t) \in \mathcal LD$, we have the following commutative diagram of isomorphisms,
%\[ \xymatrixcolsep{8pc} \xymatrix{
%\nu_{x(t)} \ar[d]_-{\sigma_{n, x(t)}} \ar[rd]^-{\sigma_{m+n, x(t)}} \\\nu_{x(t + \frac{n}{p})} \ar[r]_-{\sigma_{m, x(t+\frac{n}{p})}} & \nu_{x(t+\frac{m+n}{p})}} \] 
%\item[(iii)] When $m =0$ in $\Z/(p)$, the isomorphism $\sigma_{0, x(t)} = \mathds{1}_{\nu_{x(t)}}$, the identity map on the rank-1 $\K$-module $\nu_{x(t)}$ for every $x(t) \in \mathcal LD$.
%\item[(iv)] For any (parametrized) connecting cylinder $u(t,s)$ from $y(t)$ to $x(t)$ for $(t,s) \in S^1 \times \R$, recall that $\nu_u \in {\rm Hom}_{\K}(\nu_{y(t)}, \nu_{x(t)})$ denotes the induced map by $u$. Denote by $\nu_{u + \frac{m}{p}}$ the reparametrization of $u$ by changing $t$ to $t + \frac{m}{p}$. Then we require the following diagram commutes
%\[ \xymatrixcolsep{6pc} \xymatrix{
%\nu_{y(t)} \ar[d]_-{\sigma_{m, y(t)}} \ar[r]^-{\nu_u} & \nu_{x(t)}\ar[d]^-{\sigma_{m, x(t)}} \\
%\nu_{y(t + \frac{m}{p})} \ar[r]^-{\nu_{u + \frac{m}{p}}} & \nu_{x(t+ \frac{m}{p})}} \]
%for any $m \in \Z/(p)$. 
%\end{itemize}
%
%\end{df}
%
%\begin{rmk}\label{rmk-p-inv} Observe that by the item (ii) in Definition \ref{dfn-p-inv}, an isomorphism $\sigma_{m, x(t)}$ for any general $m \in \Z/(p)$ is completely determined by isomorphisms in the form of $\sigma_{1, \cdot}$. Indeed, for any $x(t) \in \mathcal LD$, we have 
%\begin{equation} \label{ls-p-cyclic-0}
%\sigma_{m, x(t)} = \sigma_{1, x(t + \frac{m-1}{p})} \circ \cdots \sigma_{1, x(t + \frac{1}{p})} \circ \sigma_{1, x(t)}. 
%\end{equation}
%In particular, when $m = p = 0 \in \Z/(p)$, we have 
%\begin{equation} \label{ls-p-cyclic}
%\mathds{1}_{\nu_{x(t)}} = \sigma_{1, x(t+\frac{p-1}{p})} \circ \cdots \circ \sigma_{1, x(t+\frac{1}{p})} \circ \sigma_{1, x(t)}.\end{equation}
%by the item (iii) in Definition \ref{dfn-p-inv}. 
%\end{rmk}
%
%\begin{lma} \label{lem-p-equ-ls-unique} If the local system $\nu$ is of rank-1 and $\K = \mathbb F_p$ for any given prime $p$, then there is a unique choice of the isomorphism $\sigma_{m, x(t)}$ as in Definition \ref{dfn-p-inv} for each $(m, x(t)) \in \Z/(p) \times \mathcal LD$. \end{lma}
%
%\begin{proof} By (\ref{ls-p-cyclic-0}) in Remark \ref{rmk-p-inv}, it suffices to prove the uniqueness for the isomorphisms in the form of $\sigma_{1, x(t)}$, that is, $m=1$. After the identification $\nu_{x(t)} \simeq \F_{p}$, the morphism $\sigma_{(1, x(t))} \in {\rm Hom}_{\F_p}(\F_p, \F_p)$ is a multiplication of some scalar $\lambda_{1, x(t)} \in \F_p$. Meanwhile, by the continuity from the item (i) in Definition \ref{dfn-p-inv} at the spot $x(t)$, we know $\lambda_{1, x(t)}$ does not depend on $x(t)$ since ${\rm Hom}_{\F_p}(\F_p, \F_p)$ has only finitely many elements. For brevity, denote by $\lambda : = \lambda_{1, x(t)}$. 
%
%Next, suppose there is another isomorphisms $\sigma'_{1, x(t)}$ for $(1, x(t))$, and denote by $\lambda'$ the corresponding multiplication scalar (again, independent of $x(t)$ as argued above). Then consider the isomorphism 
%\[  \sigma^{-1}_{1, x(t)} \circ \sigma'_{1, x(t)}: \nu_{x(t)} \to \nu_{x(t)}. \]
%We claim that, when passing to the multiplicative scalars, we have
%\begin{equation} \label{theta-power}
%(\lambda^{-1} \cdot \lambda')^p = 1 \in \F_p.
%\end{equation}
%However, every element $m \in \Z/(p)$ satisfies $m^p = m$ by Fermat's little theorem. Therefore, $\lambda^{-1} \cdot \lambda' = 1$ which means $\lambda = \lambda'$, that is, $\sigma_{1, x(t)}  =\sigma'_{1, x(t)}$ as requested.  
%
%In order to prove the claim (\ref{theta-power}), consider the following diagram, 
%\[ \xymatrixcolsep{6pc} \xymatrix{ 
%\nu_{x(t)} \ar@/^1pc/[r]^-{\sigma'_{1, x(t)}} & \nu_{x(t + \frac{1}{p})} \ar@/^1pc/[r]^-{\sigma'_{1, x(t+\frac{1}{p})}} \ar@/^1pc/[l]^-{\sigma^{-1}_{1, x(t)}} & \cdots \ar@/^1pc/[r]^-{\sigma'_{1, x(t + \frac{p-1}{p})}}  \ar@/^1pc/[l]^-{\sigma^{-1}_{1, x(t+\frac{1}{p})}} & \nu_{x(t)} \ar@/^1pc/[l]^-{\sigma^{-1}_{1, x(t + \frac{p-1}{p})}} }\]
%and under the identification with the rank-1 $\F_p$-modules, we have 
%\[ \xymatrixcolsep{6pc} \xymatrix{ 
%\F_p \ar@/^1pc/[r]^-{\cdot \lambda'} & \F_p \ar@/^1pc/[r]^-{\cdot \lambda'} \ar@/^1pc/[l]^-{\cdot \lambda^{-1}} & \cdots \ar@/^1pc/[r]^-{\cdot \lambda'}  \ar@/^1pc/[l]^-{\cdot \lambda^{-1}} & \F_p \ar@/^1pc/[l]^-{\cdot \lambda^{-1}} }\]
%Then by (\ref{ls-p-cyclic}) in Remark \ref{rmk-p-inv}, we know 
%\[ 1 =  (\lambda^{-1})^p \cdot (\lambda')^p  = (\lambda^{-1} \cdot \lambda')^p, \]
%where the second equality comes from our hypothesis that each fiber is $\F_p$, a 1-dimension vector space (hence, the multiplications by scalars are commutative). This proves (\ref{theta-power}) and thus we complete the proof. \end{proof}
%
%\subsubsection{Hamiltonian Floer cohomology revisited} \label{ssec-HF-revisit}  For any linear Hamiltonian $H$ on $(D, \lambda)$ that could be time-dependent, denote by 
%\begin{equation} \label{p-Ham}
%H^{(p)}(t,x) := pH(pt, x),
%\end{equation}
%which generates $\phi^p$ if $\phi$ is the Hamiltonian diffeomorphism generated by $H$. An important family of elements in the set of (parametrized) Hamiltonian orbits ${\mathcal O}(H^{(p)})$ consists of the rotations of the elements $x \in {\mathcal O}(H^{(p)})$ discretely reparametrized by $\Z/(p)$. Explicitly, if $x = x(t) \in {\mathcal O}(H^{(p)})$ where $t \in \R/\Z$, then 
%\[ R_p^m(x)(t) := x\left(t + \frac{m}{p}\right) \in \mathcal O(H^{(p)}) \,\,\,\,\mbox{for any $m \in \Z/(p)$}. \]
%Now, consider the following graded vector space over the field $\F_p$,  
%\[ {\rm CF^*}(H^{(p)}; \nu) \otimes_{\F_p} \F_p[[u]]\left<\theta\right>\]
%where ${\rm CF^*}(H^{(p)}; \nu)$ is the Hamiltonian Floer chain complex of $H^{(p)}$ defined by (\ref{hfh-ls}) and (\ref{ls-diff}) with the unique $\Z/(p)$-equivariant rank-1 local system $\nu$ over $\F_p$ on $\mathcal LD$ (given by Lemma \ref{lem-p-equ-ls-unique}). Here, $\F_p[[u]]$ should be viewed as the coefficient ring extended from $\F_p$ and $\theta$ is an extra variable such that $\theta^2 = 0$. The variable $u$ is of degree $2$ and $\theta$ is of degree $1$. In what follows, we will define a {\it deformed differential} on ${\rm CF^*}(H^{(p)}; \nu) \otimes \F_p[[u]]\left<\theta\right>$. Let us start from the involved moduli spaces.
%
%For any two $x, y \in \mathcal O(H^{(p)})$, fix $i \geq 0$, $m \in \Z/(p)$ and $\alpha \in \{0,1\}$, and consider the following moduli space 
%\begin{equation} \label{moduli}
%\mathcal M_{\alpha}^{i, m}(y, x) : = \left\{ (u(s,t), w(s)) \bigg| \begin{array}{l} \mbox{Floer cylinder $u(s,t): R_p^m(y)(t) \to x(t)$} \\ \mbox{a certain flow line $w(s): \R \to S^{\infty}$} \end{array} \right\}\bigg/ \R, 
%\end{equation} 
%where the $\R$-action on $(u(s,t), w(s))$ is the reparametrization $s \mapsto s+r$. Here, to be more precise, 
%\begin{itemize}
%\item{} the almost complex structure, denoted by $J_{s,t}^{w(s)}$, that is used to define the Floer cylinder in $(\ref{moduli})$ is also parametrized by the flow line $w$ (see (58) in \cite{SZhao});
%\item{} the flow line $w$ is explicitly given by a negative gradient flow line as follow, 
%\[ \partial_s w(s) + \nabla \tilde{F}(w) = 0 \]
%where $\tilde{F}$ is a $\Z/(p)$-invariant Morse function on $S^{\infty}$ with exactly $p$-many critical points of index $i$ on $S^{\infty}$, labelled by $Z_i^0, …, Z_i^{p-1}$. Section 4 in \cite{SZhao} provides an explicit formula of $\tilde{F}$;
%\item{} the boundary condition of $w(s)$ is given by 
%\begin{equation} \label{Morse-flow}
%\lim_{s \to -\infty} w(s) = Z_i^m \,\,\,\,\mbox{and}\,\,\,\, \lim_{s \to \infty} w(s) = Z_{\alpha}^0. 
%\end{equation}
%\end{itemize}
%Importantly, for a generic choice of the almost complex structure, the moduli space $\mathcal M_{\alpha}^{i, m}(y, x)$ is a smooth manifold with the dimension (see (61) in \cite{SZhao}), 
%\begin{equation} \label{dim}
%\dim \mathcal M_{\alpha}^{i, m}(y, x) = |y| - |x| -1+ i - \alpha. 
%\end{equation}
%However, as in the classical Floer-theoretic situations, the moduli space $\mathcal M_{\alpha}^{i, m}(y, x)$ in general is not compact. In order to define the deformed differential on ${\rm CF^*}(H^{(p)}; \nu) \otimes \F_p[[u]]\left<\theta\right>$ as promised, a careful study of the compactification of $\mathcal M_{\alpha}^{i, m}(y, x)$ is required. This occupies the next subsection. 
%
%\subsubsection{Compactification of moduli spaces} \label{ssec-comp} For later use, we need to analyze the moduli space of all such flow lines. Denote by $Q^{i,m}_{\alpha}$ the moduli space of all such {\it unparametrized} Morse flow lines on $S^{\infty}$ as in (\ref{Morse-flow}), that is, modulo the translation given by elements in $\R$. Note that in general $Q^{i,m}_{\alpha}$ is not compact. There is a forgetful map $\mathcal M_{\alpha}^{i, m}(y, x) \to Q^{i,m}_{\alpha}$ by $(u(s,t), w(s)) \to w(s)$. Then compactification $\overline{\mathcal M}^{i,m}_{\alpha}(y,x)$ is produced by running the following ``double compactification'' process. First, compactification the moduli space $Q^{i,m}_{\alpha}$ by adding (unparametrized) broken Morse trajectories as the limit, denoted by 
%\begin{equation} \label{broken-Morse}
%w^{\infty} : = (w_1, w_2, …, w_n) \,\,\,\,\mbox{for some $n \in \N$}.
%\end{equation}
%Over each of such broken Morse trajectories $w^{\infty}$, we add (only) the following broken Floer trajectory 
%\begin{equation} \label{broken-Floer}
%u^{\infty}: = (u_1, u_2, …, u_n) \,\,\,\,\mbox{for $n$ as in (\ref{broken-Morse})}.
%\end{equation}
%Importantly, the compatibility between $w^{\infty}$ and $u^{\infty}$ is required that $u_i(s,t)$ is a Floer trajectory with respect to the almost complex structure $J_{s,t}^{w_i(s)}$ for $1 \leq i \leq n$ (cf.~the first bullet below (\ref{moduli})). Therefore, the compactification $\overline{\mathcal M}^{i,m}_{\alpha}(y,x)$ can be described as follows, 
%\begin{equation} \label{moduli-broken}
%\overline{\mathcal M}^{i,m}_{\alpha}(y,x) \simeq \bigsqcup_{n} \left(\mathcal M^{i_1,m_1}_{\alpha_1}(\cdot,x) \times  \mathcal M^{i_2,m_2}_{\alpha_2}(\cdot, \cdot)  \times \cdots \times \mathcal M^{i_n,m_n}_{\alpha_n}(y, \cdot) \right)
%\end{equation}
%where ``$\cdot$'' reserves as a spot for an orbit in $\mathcal O(H^{(p)})$. In other words, $\overline{\mathcal M}^{i,m}_{\alpha}(y,x)$ is a stratified space and each $n$-piece is a codimension-$(n-1)$ stratum. Moreover, one can work out the indices where the following conditions are satisfied (see Section 4 in \cite{SZhao}), 
%\begin{itemize}
%\item{} $\alpha_j \in \{0,1\}$ for all $1 \leq j \leq n$ and $\alpha_1 = \alpha$;
%\item{} $m_1 + m_2 + \cdots + m_n = m$ in $\Z/(p)$;
%\item{} $i_n = i \, \mbox{(mod $2$)}$ and $\sum_{j=1}^n (i_j - \alpha_j) = i - \alpha$.
%\end{itemize}
%%In particular, the codimension-$0$ stratum is $\mathcal M^{i,m}_{\alpha}$, that is, $\alpha_1 = \alpha$, $m_1 = m$ and $i_1 = i$. Here, $\mathcal M^{i,m}_{\alpha}$ is a dense open subset of $\overline{\mathcal M}^{i,m}_{\alpha}$, while higher codimensional strata are so-called broken flow lines, added to form the compactification. 
%
%\begin{exa} \label{ex-bd-compactification} For brevity, let us omit the orbits $y,x$ in the notation $\overline{\mathcal M}^{i,m}_{\alpha}(y,x)$. The codimension-$1$ stratum of $\overline{\mathcal M}^{0,0}_{0}$ is 
%\[ \bigsqcup \left(\mathcal M^{i_1, m_1}_{0} \times \mathcal M^{i_2, m_2}_{\alpha_2}\right) \]
%where $m_1 + m_2 = 0$ in $\Z/(p)$, $i_2 = 2k$ for some $k \in \Z$, and $i_1 + 2k - \alpha_2 = 0$. Since $\alpha_2  \in \{0,1\}$ and $i_1, i_2 \geq 0$, we have the following concrete cases.
%\begin{itemize}
%\item{} If $\alpha_2 = 0$, then 
%\[ \mathcal M^{i_1, m_1}_{0} \times \mathcal M^{i_2, m_2}_{\alpha_2} =  \mathcal M^{0, 0}_{0} \times \mathcal M^{0, 0}_{0} \,\,\,\,\mbox{or} \,\,\,\, \mathcal M^{0, m}_{0} \times \mathcal M^{0, p-m}_{0} \,\,\,\,\mbox{for $m \in \Z/(p)$}.\] 
%\item{} If $\alpha_2 = 1$, then 
%\[ \mathcal M^{i_1, m_1}_{0} \times \mathcal M^{i_2, m_2}_{\alpha_2} =  \mathcal M^{1, 0}_{0} \times \mathcal M^{0, 0}_{1} \,\,\,\,\mbox{or} \,\,\,\, \mathcal M^{1, m}_{0} \times \mathcal M^{0, p-m}_{1} \,\,\,\,\mbox{for $m \in \Z/(p)$}.\]
%\end{itemize}
%Note that by index reason, both the moduli spaces $\mathcal M^{0, 0}_{1}$ and $\mathcal M^{0, p-m}_{1}$ are empty (since we do not have Morse flow lines from the index-$0$ critical point $Z_0^0$ or $Z_0^{p-m}$ to the index-$1$ critical point $Z_1^0$). Also, by the action reason, the moduli space $\mathcal M^{0, m}_{0}$ is not empty only when $m = 0$ (since we do not have Morse flow lines from the index-$0$ critical point $Z_0^m$ for $m \neq 0$ to the index-$0$ critical point $Z_0^0$), similarly to $\mathcal M^{0, p-m}_{0}$. Therefore, we obtain a simple expression as follows, \begin{equation} \label{moduli-0-0-0}
%\mbox{codimension-$1$ stratum of $\overline{\mathcal M}^{0,0}_{0}$} = \mathcal M^{0, 0}_{0} \times \mathcal M^{0, 0}_{0},
%\end{equation}
%which contains a single point (corresponding to the product of two constant flow line at critical point $Z_0^0$). This coincides with the classical situation, without touching any $\Z/(p)$-equivariant data from our set-up. 
%
%\medskip
%
%In general, the codimension-$1$ stratum of $\overline{\mathcal M}^{i,m}_{\alpha}$ is 
%\[ \bigsqcup \left(\mathcal M^{i_1, m_1}_{\alpha} \times \mathcal M^{i_2, m_2}_{\alpha_2}\right) \]
%where $m_1 + m_2 = m$ in $\Z/(p)$, $i_2 = i + 2k$ for some $k \in \Z$, and $i_1 = \alpha_2 - 2k$. More explicitly, we have the following two cases in the sub-term above depending on the value of $\alpha_2$,
%\[\bigsqcup \left(\mathcal M^{-2k, m_1}_{\alpha} \times \mathcal M^{i+2k, m_2}_{0}\right) \cup \left(\mathcal M^{1-2k, m_1}_{\alpha} \times \mathcal M^{i+2k, m_2}_{1}\right)\]
%Slightly more complicated than $\overline{\mathcal M}^{0,0}_{0}$ above, there could be non-trivial parameters $k$ to be considered. However, since $-2k \geq \alpha$  (from the non-triviality of the sub-term $\mathcal M^{-2k, m_1}_{\alpha}$) and $i+2k \geq 1$ (from the non-triviality of the sub-term $\mathcal M^{i+2k, m_2}_{1}$), there are only finitely many such $k$, that is, 
%\[ \mbox{$k$ satisfies the relation that $\frac{1-i}{2} \leq k \leq -\frac{\alpha}{2}$}. \]
%It's worth noticing that when $i+2k$ is large, the Morse flow lines in discussion connect critical points on $S^{\infty}$ with large difference in the indices, but the dimension of the moduli space $\overline{\mathcal M}^{i,m}_{\alpha}$ can be small. \end{exa}
%
%For later use, specifically for the proofs of Proposition \ref{prop-diff} and Proposition \ref{prop-cont-com-diff}  below, it will be convenient to introduce the notation ${\mathcal M}^{i,m}_{\alpha, \beta}(y,x)$, where $\beta$ is a fixed homotopy class of cylinder connecting $y$ and $x$, with the meaning of the subset of moduli space ${\mathcal M}^{i,m}_{\alpha}(y,x)$ but with the topological constraint that the Floer cylinders $u$ satisfy $[u] = \beta$.  
%
%\medskip
%
%To end this subsection, let us mention another type of compactification of moduli spaces, slightly different from the one using $Q^{i,m}_{\alpha}$ as above. Instead of (\ref{broken-Floer}), we consider 
%\begin{equation} \label{broken-Floer-par}
%u^{\infty}_{\rm par} = (u_1, …, u_{i-1}, \tilde{u}_i, u_{i+1}, …, u_n) \,\,\,\,\mbox{for $n$ as in (\ref{broken-Morse})} 
%\end{equation}
%where $\tilde{u}_i$ is a {\it parametrized} Morse flow line on $S^{\infty}$, that is, not modulo the translation $\R$. For more details of this parametrized-type moduli space and its compactification, see the discussion at the end of Section 4 in \cite{SZhao} (where the moduli space of the corresponding parametrized Morse flow lines is denoted by $P^{i,m}_{\alpha}$ and $u^{\infty}_{\rm par}$ in (\ref{broken-Floer-par}) precisely serve as the broken Morse flows that are added in order to form the compactification of $P^{i,m}_{\alpha}$). This moduli space will be particularly useful and involved into the compactification of the moduli spaces in the proof of Proposition \ref{prop-cont-com-diff} where $\Z/(p)$-continuation maps are constructed.
%
%\subsubsection{$\Z/(p)$-equivariant Floer differential} \label{ssec-differential} We are ready to define the differential on ${\rm CF^*}(H^{(p)}; \nu) \otimes_{\F_p} \F_p[[u]]\left<\theta\right>$. First, consider 
%\begin{align} \label{dfn-slice-diff-ls}
%d_{\alpha}^{i, m}(x \otimes b) =  \sum_{y: \, |y|= |x|+1-i + \alpha} y \otimes \left(\sum_{u \in \mathcal M_{\alpha}^{i,m}(y, x)}(\sigma^{-1}_{m, y} \circ \nu_{u}^{-1})(b)\right).
%\end{align}
%Applying $\nu_u^{-1}$ in (\ref{dfn-slice-diff-ls}) only results in an element $\nu_{u}^{-1}(b) \in \nu_{R^m_p(y)}$ for each $u \in \mathcal M^{i,m}_\alpha(y,x)$. The extra term $\sigma^{-1}_{m, y}$ is used, as an isomorphism from the package of the local system $\nu$ being $\Z/(p)$-equivariant (see (\ref{p-equiv-ls}) in Definition \ref{dfn-p-inv}), to correct from the fiber $\nu_{R^m_p(y)}$ to the fiber $\nu_{y}$. Second, sum over all $m \in \{0, …, p-1\}$, that is, 
%\begin{align*} \label{dfn-slice-diff-ls}
%d_{\alpha}^{i} (x \otimes b) & = \sum_{m=0}^{p-1} d_{\alpha}^{i, m}(x \otimes b) \\
%& = \sum_{y: \, |y|= |x|+1-i + \alpha} y \otimes \left(\sum_{m=0}^{p-1} \sum_{u \in \mathcal M_{\alpha}^{i,m}(y, x)} (\sigma^{-1}_{m, y} \circ \nu_{u}^{-1})(b)\right)
%\end{align*}
%where the second equality holds since $(\sigma^{-1}_{m, y} \circ \nu_{u}^{-1})(b)$ all lie in the fiber $\nu_{y}$. In order to simplify the notation, let us denote 
%\begin{equation}\label{T}
%{\rT}_{\alpha, (y,x)}^i : = \sum_{m=0}^{p-1} \sum_{u \in \mathcal M_{\alpha}^{i,m}(y, x)} (\sigma^{-1}_{m, y} \circ \nu_{u}^{-1}) \in {\rm Hom}_{\F_p}(\nu_x, \nu_y).
%\end{equation}
%which we simplify call {\it $\rT$-operator}. Then we have 
%\begin{equation} \label{d-alpha-i}
%d^i_{\alpha}(x \otimes b) = \sum_{y: \, |y|= |x|+1-i + \alpha} y \otimes \rT_{\alpha, (y,x)}^i(b). 
%\end{equation}
%Note that ${\rT}_{\alpha, (y,x)}^i$ does not result in any degree or filtration shift. Moreover, for the following composition, as an $\F_p$-linear map from $\nu_x$ to $\nu_z$, we have
%\begin{align} \label{compo-T}
%{\rT}_{\alpha', (z,y)}^j  \circ {\rT}_{\alpha, (y,x)}^i  & = \sum_{m, n = 0}^{p-1}  \sum_{(v,u) \in \mathcal M_{\alpha'}^{j,n}(z,y) \times \mathcal M_{\alpha}^{i, m}(y,x)} (\sigma^{-1}_{n, z} \circ \nu_{v}^{-1} \circ \sigma^{-1}_{m,y} \circ \nu_u^{-1}), \nonumber \\ 
%& = \sum_{m, n = 0}^{p-1}  \sum_{(v,u) \in \mathcal M_{\alpha'}^{j,n}(z,y) \times \mathcal M_{\alpha}^{i, m}(y,x)} (\sigma^{-1}_{n, z} \circ \sigma^{-1}_{m,z + \frac{n}{p}} \circ \nu_{v+\frac{n}{p}}^{-1} \circ  \nu_u^{-1}) \nonumber\\ 
%& = \sum_{m, n = 0}^{p-1}  \sum_{(v,u) \in \mathcal M_{\alpha'}^{j,n}(z,y) \times \mathcal M_{\alpha}^{i, m}(y,x)} (\sigma^{-1}_{m+n, z} \circ \nu_{v+\frac{n}{p}}^{-1} \circ  \nu_u^{-1})
%\end{align}
%where the second equality comes from the item (iv) in Definition \ref{dfn-p-inv} and the third equality comes from the item (ii) in Definition \ref{dfn-p-inv}. Observe that the composition $\nu_{v+\frac{n}{p}}^{-1} \circ  \nu_u^{-1}$ only depends on the homotopy class $[v] \# [u]$ since the reparametrization $v + \frac{n}{p}$ does not result in any changes of the homotopy type. The final step is to consider a linear map $d_{\nu}^{\Z/(p)}$ deformed via the formal variable $u$ (of degree $+2$). Explicitly, the generator $x \otimes b \otimes 1$ is mapped to 
%\begin{align*} 
%\sum_{\tiny{\begin{array}{c} i \geq 0 \\ y: \, |y| = |x|+1-2i \end{array}}} u^i \left(y \otimes \rT_{0, (y,x)}^{2i}(b) \otimes 1\right) + \sum_{\tiny{\begin{array}{c} i\geq0 \\ y: \, |y| = |x| -2i \end{array}}} u^i \left(y \otimes \rT_{0, (y,x)}^{2i+1}(b) \otimes \theta\right)
%\end{align*} 
%and the generator $x \otimes b \otimes \theta$ is mapped to
%\begin{align*} 
%\sum_{\tiny{\begin{array}{c} i\geq 0 \\ y: \, |y| = |x|+1-2i \end{array}}} u^i \left(y \otimes \rT_{1, (y,x)}^{2i+1}(b) \otimes \theta\right) + \sum_{\tiny{\begin{array}{c} i\geq0 \\ y: \, |y| = |x| -2i \end{array}}}  u^{i+1} \left(y \otimes \rT_{1, (y,x)}^{2i+2}(b) \otimes 1\right).
%\end{align*} 
%Analyzing the degree shift, we have ${\rm deg}(d_{\nu}^{\Z/(p)}) = +1$. 
%
%\begin{rmk} When $\nu$ is a trivial rank-1 local system over $\F_p$ on $\mathcal LD$ (which is in particular $\Z/(p)$-equivariant as in Definition \ref{dfn-p-inv} with all the requested isomorphisms $\sigma_{(m, y)}$ equal to the identity), the morphism $d^{\Z/(p)}_{\nu}$ defined above recovers the definition $d^{\Z/p\Z}$ in (64) in \cite{SZhao}. Indeed, the definition (\ref{dfn-slice-diff-ls}) simplifies as follows,  
%\begin{align*}
%d_{\alpha}^{i, m}(x \otimes b) & =  \sum_{y: \, |y| = |x|+1-i + \alpha} y \otimes \left(\sum_{u \in \mathcal M_{\alpha}^{i,m}(y, x)}b\right) \\
%& = \sum_{y: \, |y| = |x|+1-i + \alpha} \# \left(\mathcal M_{\alpha}^{i, m}(y, x) \cdot y\right) \otimes b.  
%\end{align*}
%Note that the second equality holds due to the dimension formula in (\ref{dim}). Then without loss of generality, assume $b \neq 0$, and we recover the classical definition by cancelling $b$ (or $\otimes b$) on both sides.
%\end{rmk}
%
%The following proposition is crucial, confirming that the map  $d_{\nu}^{\Z/(p)}$ defined above is a well-defined Floer differential. The proof combines the argument on page 27-28 in \cite{SZhao} (for equivariant Floer differential) and the argument of Proposition 1.5.10 in \cite{Abo15} (for Floer differential with local system). 
%
%\begin{prop} \label{prop-diff} The map $d_{\nu}^{\Z/(p)}$ defined above satisfies $(d_{\nu}^{\Z/(p)})^2 =0$ over $\F_p$. \end{prop}
%
%\begin{proof} For any element $x \otimes b \otimes 1$, we have 
%\begin{align*}
%(d_{\nu}^{\Z/(p)})^2(x \otimes b \otimes 1) & = \sum_{\tiny{\begin{array}{c} i \geq 0 \\ y: \, |y| = |x|+1-2i \end{array}}} u^i \cdot d_{\nu}^{\Z/(p)} \left(y \otimes \rT_{0, (y,x)}^{2i}(b) \otimes 1\right) \\
%& \,\,\,\,\,+ \sum_{\tiny{\begin{array}{c} i\geq0 \\ y: \, |y| = |x| -2i \end{array}}} u^i \cdot d_{\nu}^{\Z/(p)} \left(y \otimes \rT_{0, (y,x)}^{2i+1}(b) \otimes \theta\right)\\
%& = \sum_{\tiny{\begin{array}{c} i, j \geq 0 \\ y: \, |y| = |x|+1-2i \\ z: \, |z| = |y|+1-2j  \end{array}}} u^{i+j} \left(z \otimes \rT_{0, (z,y)}^{2j}(\rT^{2i}_{0, (y, x)}(b)) \otimes 1 \right) \\
%& + \sum_{\tiny{\begin{array}{c} i, j \geq 0 \\ y: \, |y| = |x|+1-2i \\ z: \, |z| = |y|-2j  \end{array}}} u^{i+j} \left(z \otimes \rT_{0, (z,y)}^{2j+1}(\rT^{2i}_{0, (y, x)}(b)) \otimes \theta \right) \\
%& + \sum_{\tiny{\begin{array}{c} i, j \geq 0 \\ y: \, |y| = |x|-2i \\ z: \, |z| = |y|+1-2j  \end{array}}} u^{i+j} \left(z \otimes \rT_{1, (z,y)}^{2j+1}(\rT^{2i+1}_{0, (y, x)}(b)) \otimes \theta \right) \\
%& + \sum_{\tiny{\begin{array}{c} i, j \geq 0 \\ y: \, |y| = |x|-2i \\ z: \, |z| = |y|-2j  \end{array}}} u^{i+j+1} \left(z \otimes \rT_{1, (z,y)}^{2j+2}(\rT^{2i+1}_{0, (y, x)}(b)) \otimes 1 \right).
%\end{align*}
%We will regroup all the terms in the sum above in the following way. Depending on whether in the form of $\cdot \otimes 1$ or $\cdot \otimes \theta$, we can rewrite the sum above as follows,
%\[ (d_{\nu}^{\Z/(p)})^2(x \otimes b \otimes 1) = \left(\sum_{\ell \geq 0} A_\ell(b) u^\ell\right) \otimes 1 + \left( \sum_{\ell \geq 0} B_\ell(b) u^\ell \right) \otimes \theta. \]  
%Here, we will discuss the ${\mathcal O}(H^{(p)})$-valued operators $A_\ell$ and $B_\ell$ separately. The arguments for them are almost identical, though the moduli spaces that will be investigated are slightly different. 
%
%\medskip
%
%\noindent {\bf Case A}. The expression of $A_\ell$ is 
%\[ A_\ell = \sum_{\tiny{\begin{array}{c} i+j = \ell\\ y: \, |y| = |x|+1-2i \\ z: \, |z| = |y|+1-2j  \end{array}}} z \otimes \left(\rT_{0, (z,y)}^{2j} \circ \rT^{2i}_{0, (y, x)} \right) + \sum_{\tiny{\begin{array}{c} i+j = \ell-1\\ y: \, |y| = |x|-2i \\ z: \, |z| = |y|-2j  \end{array}}} z \otimes \left(\rT_{1, (z,y)}^{2j+2} \circ \rT^{2i+1}_{0, (y, x)} \right).\]
%For any fixed $z \in {\mathcal O}(H^{(p)})$ with $|z| - |x| = 2-2\ell$, consider the following moduli space 
%\begin{equation} \label{diff-M}
%\mathcal M^{2\ell, \ast}_{0, \beta}(z,x) \,\,\,\,\mbox{for $\ast \in \Z/(p)$ and a homotopy class $\beta$}.
%\end{equation}
%By the formula (\ref{dim}), the dimension of $\mathcal M^{2\ell, \ast}_{0, \beta}(z,x)$ is $2 - 2\ell -1 + 2\ell - 0 = 1$. On the one hand, by Example \ref{ex-bd-compactification}, the codimension-$1$ stratum of the compactification $\overline{\mathcal M}^{2\ell,*}_{0, \beta}(z,x)$ is 
%\[\bigsqcup_{\beta_1 + \beta_2 = \beta} \left(\mathcal M^{-2k, m}_{0, \beta_1}(y, x) \times \mathcal M^{2\ell+2k, n}_{0, \beta_2}(z,y)\right) \cup \left(\mathcal M^{1-2k, m}_{0, \beta_1}(y,x) \times \mathcal M^{2\ell+2k, n}_{1, \beta_2}(z,y)\right)\]
%and $m + n = \ast \, (\mbox{mod}\, p)$. For the first component, let $i = -k$ and $j =\ell+k$ (so $i + j  =\ell$), we have 
%\begin{equation} \label{ell-1}
%\mathcal M^{-2k, m}_{0, \beta_1}(y, x) \times \mathcal M^{2\ell+2k, n}_{0, \beta_2}(z,y) = \mathcal M^{2i, m}_{0, \beta_1}(y, x) \times \mathcal M^{2j, n}_{0, \beta_2}(z,y).
%\end{equation}
%Similarly, for the second component, let $i = -k$ and $j = \ell-1+k$ (so $i+j = \ell-1$), we have 
%\begin{equation} \label{ell-2}
%\mathcal M^{1-2k, m}_{0, \beta_1}(y, x) \times \mathcal M^{2\ell+2k, n}_{1, \beta_2}(z,y) = \mathcal M^{2i+1, m}_{0, \beta_1}(y, x) \times \mathcal M^{2j+2, n}_{1, \beta_2}(z,y).
%\end{equation}
%On the other hand, for the first composition of the ${\rT}$-operators in the summand of $A_{\ell}$, by (\ref{compo-T}), when restricted to the homotopy class $\beta$, 
%\begin{equation} \label{compo-T-2}
%\rT_{0, (z,y)}^{2j} \circ \rT^{2i}_{0, (y, x)}|_{\beta}  = \sum_{m, n = 0}^{p-1}  \sum_{(v,u) \in \mathcal M_{\alpha', \beta_2}^{2j,n}(z,y) \times \mathcal M_{\alpha, \beta_1}^{2i, m}(y,x)} (\sigma^{-1}_{m+n, z} \circ \nu_{\beta}^{-1})
%\end{equation}
%where $\beta_1 + \beta_2 = \beta$. Similarly, for the second composition of the ${\rT}$-operators in the summand of $A_{\ell}$, when restricted to the homotopy class $\beta$, 
%\begin{equation} \label{compo-T-3}
%\rT_{1, (z,y)}^{2j+2} \circ \rT^{2i+1}_{0, (y, x)}|_{\beta} = \sum_{m, n = 0}^{p-1}  \sum_{(v,u) \in \mathcal M^{2j+2, n}_{1, \beta_2}(z,y) \times \mathcal M^{2i+1, m}_{0, \beta_1}(y, x)} (\sigma^{-1}_{m+n, z} \circ \nu_{\beta}^{-1})
%\end{equation}
%where $\beta_1 + \beta_2 = \beta$. Observe that, up to the summation over $m, n \in \{0, …, p-1\}$, the Floer cylinders that are counted and involved in these moduli spaces have the following correspondence, 
%\[ (\ref{ell-1} \Longleftrightarrow (\ref{compo-T-2}) \,\,\,\,\,\mbox{and}\,\,\,\,\,(\ref{ell-2}) \Longleftrightarrow (\ref{compo-T-3}). \]
%Therefore, for any input $b \in \F_p \backslash\{0\}$, the coefficient of the output of $A_{\ell}(b)$ in front of the orbit $z$ is
%\begin{equation} \label{coeff-z}
%\sum_{\beta} \sum_{\ast =0}^{p-1} \sum_{(v,u) \in \partial \overline{\mathcal M_{0, \beta}^{2\ell, \ast}(z,x)}} (\sigma^{-1}_{\ast, z} \circ \nu_{\beta}^{-1})(b)
%\end{equation}
%where $\beta$ is taken over all the possible homotopy classes of the cylinders connecting $z$ to $x$, and $(v,u)$ denotes the broken Floer trajectories. By the dimension reason, the sum above is finite (hence, well-defined); more importantly, the prime number $p$ divides the sum (in fact, $p$ divides the summand for each $\beta$). Therefore, $A_{\ell}(b) =0$ in $\F_p$. 
%
%\medskip
%
%\noindent {\bf Case B}. The expression of $B_\ell$ is 
%\[ B_\ell = \sum_{\tiny{\begin{array}{c} i+j = \ell\\ y: \, |y| = |x|+1-2i \\ z: \, |z| = |y|-2j   \end{array}}} z \otimes \left(\rT_{0, (z,y)}^{2j+1} \circ \rT^{2i}_{0, (y, x)} \right) + \sum_{\tiny{\begin{array}{c} i+j = \ell\\ y: \,  |y| = |x|-2i \\ z: \, |z| = |y|+1-2j   \end{array}}} z \otimes \left(\rT_{1, (z,y)}^{2j+1} \circ \rT^{2i+1}_{0, (y, x)} \right).\]
%For any fixed $z \in {\mathcal O}(H^{(p)})$ with $|z| - |x| = 1-2\ell$, consider the following moduli space 
%\begin{equation} \label{diff-M-B}
%\mathcal M^{2\ell +1, \ast}_{0, \beta}(z,x) \,\,\,\,\mbox{for $\ast \in \Z/(p)$ and a homotopy class $\beta$}.
%\end{equation}
%By the formula (\ref{dim}), the dimension of $\mathcal M^{2\ell+1, \ast}_{0, \beta}(z,x)$ is $1 - 2\ell -1 + 2\ell + 1 - 0 = 1$. On the one hand, by Example \ref{ex-bd-compactification}, the codimension-$1$ stratum of the compactification $\overline{\mathcal M}^{2\ell+1,*}_{0, \beta}(z,x)$ is 
%\[\bigsqcup_{\beta_1 + \beta_2 = \beta} \left(\mathcal M^{-2k, m}_{0, \beta_1}(y, x) \times \mathcal M^{2\ell+2k+1, n}_{0, \beta_2}(z,y)\right) \cup \left(\mathcal M^{1-2k, m}_{0, \beta_1}(y,x) \times \mathcal M^{2\ell+2k+1, n}_{1, \beta_2}(z,y)\right)\]
%and $m + n = \ast \, (\mbox{mod}\, p)$. For the first component, let $i = -k$ and $j =\ell+k$ (so $i + j  =\ell$), we have 
%\begin{equation} \label{ell-1-B}
%\mathcal M^{-2k, m}_{0, \beta_1}(y, x) \times \mathcal M^{2\ell+2k+1, n}_{0, \beta_2}(z,y) = \mathcal M^{2i, m}_{0, \beta_1}(y, x) \times \mathcal M^{2j+1, n}_{0, \beta_2}(z,y).
%\end{equation}
%Similarly, for the second component, let $i = -k$ and $j = \ell+k$ (so $i+j = \ell$), we have 
%\begin{equation} \label{ell-2-B}
%\mathcal M^{1-2k, m}_{0, \beta_1}(y, x) \times \mathcal M^{2\ell+2k+1, n}_{1, \beta_2}(z,y) = \mathcal M^{2i+1, m}_{0, \beta_1}(y, x) \times \mathcal M^{2j+1, n}_{1, \beta_2}(z,y).
%\end{equation}
%On the other hand, for the first composition of the ${\rT}$-operators in the summand of $A_{\ell}$, by (\ref{compo-T}), when restricted to the homotopy class $\beta$, 
%\begin{equation} \label{compo-T-2-B}
%\rT_{0, (z,y)}^{2j+1} \circ \rT^{2i}_{0, (y, x)}|_{\beta}  = \sum_{m, n = 0}^{p-1}  \sum_{(v,u) \in \mathcal M_{\alpha', \beta_2}^{2j+1,n}(z,y) \times \mathcal M_{\alpha, \beta_1}^{2i, m}(y,x)} (\sigma^{-1}_{m+n, z} \circ \nu_{\beta}^{-1})
%\end{equation}
%where $\beta_1 + \beta_2 = \beta$. Similarly, for the second composition of the ${\rT}$-operators in the summand of $A_{\ell}$, when restricted to the homotopy class $\beta$, 
%\begin{equation} \label{compo-T-3-B}
%\rT_{1, (z,y)}^{2j+1} \circ \rT^{2i+1}_{0, (y, x)}|_{\beta} = \sum_{m, n = 0}^{p-1}  \sum_{(v,u) \in \mathcal M^{2j+1, n}_{1, \beta_2}(z,y) \times \mathcal M^{2i+1, m}_{0, \beta_1}(y, x)} (\sigma^{-1}_{m+n, z} \circ \nu_{\beta}^{-1})
%\end{equation}
%where $\beta_1 + \beta_2 = \beta$. Observe that, up to the summation over $m, n \in \{0, …, p-1\}$, the Floer cylinders that are counted and involved in these moduli spaces have the following correspondence, 
%\[ (\ref{ell-1-B} \Longleftrightarrow (\ref{compo-T-2-B}) \,\,\,\,\,\mbox{and}\,\,\,\,\,(\ref{ell-2-B}) \Longleftrightarrow (\ref{compo-T-3-B}). \]
%Then, similarly to (\ref{coeff-z}), for any input $b \in \F_p \backslash\{0\}$, rewrite the coefficient of the output of $B_{\ell}(b)$ in front of the orbit $z$ in terms of the broken Floer trajectories in $\partial \overline{\mathcal M_{0, \beta}^{2\ell+1, \ast}(z,x)}$, we have $B_{\ell}(b) = 0$ in $\F_p$ by the same argument as in Case A. 
%
%\medskip
%
%Finally, for the generator $x \otimes b \otimes \theta$, the argument goes in the same as above. The only difference is that, instead of $\mathcal M^{2\ell, \ast}_{0, \beta}(z,x)$ in (\ref{diff-M}) and $\mathcal M^{2\ell +1, \ast}_{0, \beta}(z,x)$ in (\ref{diff-M-B}), we need to consider $\mathcal M^{2\ell, \ast}_{1, \beta}(z,x)$ and $\mathcal M^{2\ell +1, \ast}_{1, \beta}(z,x)$, respectively. Thus we complete the proof. \end{proof}
%
%Proposition \ref{prop-diff} leads to the following definition. 
%
%\begin{df} Let $(D, \lambda)$ be a Liouville domain, and $p$ be a prime. The {\bf $\Z/(p)$-equivariant Hamiltonian Floer homology of $H^{(p)}$ with a $\Z/(p)$-equivariant rank-1 local system $\nu$ over $\F_p$ on $\mathcal L D$} is defined as the following cohomology group over $\F_p$,
%\[ {\rm HF}_{\Z/(p)}^*(H^{(p)}; \nu): = H^*\left({\rm CF}(H^{(p)}; \nu) \otimes_{\F_p} \F_p[[u]]\left<\theta\right>, d_{\nu}^{\Z/(p)}\right).\]
%In fact, ${\rm HF}_{\Z/(p)}^*(H^{(p)}; \nu)$ is a $\F_p[[u]]\left<\theta\right>$-module. In the same way, one can define the truncated version and in class $a \in \pi_0(\mathcal LD)$, that is, 
%\begin{equation} \label{dfn-fil-zp-hf-ls}
%{\rm HF}_{a, \Z/(p)}^{*, {(s, t]}}(H^{(p)}; \nu)
%\end{equation}
%for any $s<t$ in $\R$, by considering the set of Hamiltonian orbits in class $a$. 
%
%%By tensoring with $\F_p((u))\left<\theta\right>$, we obtained different versions of {\bf Tate Hamiltonian cohomology of $H^{(p)}$ with a $\Z/(p)$-equivariant rank-1 local system $\nu$ over $\F_p$ on $\mathcal L D$}, denoted by $\widehat{{\rm HF}}_{\Z/(p)}^{*}(H^{(p)}; \nu)$ and $\widehat{{\rm HF}}_{a, \Z/(p)}^{*, {(s, t]}}(H^{(p)}; \nu)$.
%\end{df}
%
%\subsubsection{$\Z/(p)$-equivariant continuation maps} First, note that if $\{H_i\}_{i \in \N}$ is a sequence of linear Hamiltonians on $(D, \lambda)$ with increasing slopes, then so is the sequence $\{H^{(p)}_i\}_{i \in \N}$. Moreover, $H^+ \preceq H^-$ implies that $(H^+)^{(p)} \preceq (H^-)^{(p)}$. Consider a homotopy $\{H_s\}_{s \in \R}$ such that there exists some $s_*>0$ where 
%\[ H_s = H^{+} \,\,\mbox{for $s \leq -s_*$} \,\,\,\,\mbox{and}\,\,\,\, H_s = H^{-} \,\,\mbox{for $s \geq s_*$}. \]
%Moreover, there exists some $p_*>0$ such that $\partial_s H_s(q,p) |_{S \times [p_*, \infty)} \leq 0$ where $S = \partial D$. Then the homotopy $\{H_s^{(p)}\}_{s \in \R}$ provides a homotopy from $(H^{+})^{(p)}$ to $(H^{-})^{(p)}$ where the derivative $\partial_s H_s^{(p)}$ satisfies the same property as above. Now, let $x\in \mathcal O((H^+)^{(p)})$ and $y \in \mathcal O((H^-)^{(p)})$. For fixed $i \in \N$, $m \in \{0, …, p-1\}$ and $\alpha \in \{0,1\}$, consider the following moduli space 
%\begin{equation} \label{cont-moduli}
%\mathcal M_{\alpha, {\rm cont}}^{i, m}(y,x) = \left\{ (u(s,t),w(s)) \, \bigg| \, \begin{array}{l} u: \R \times S^1 \to M, \,\, w: \R \to S^{\infty} \\ \mbox{s.t.} \,\, \partial_s u + J_{s,t}^{w(s)} \partial_t u = - \nabla H^{(p)}_s \\ \mbox{and also} \,\, \partial_s w(s) + \nabla \tilde{F}(w) = 0 \end{array} \right\}. 
%\end{equation}
%where the set of ingredients in (\ref{cont-moduli}) is the same as the one in $\mathcal M_{\alpha}^{i, m}(y,x)$ in subsection \ref{ssec-HF-revisit}. However, due to the deformed term $-\nabla H^{(p)}_s$ in (\ref{cont-moduli}), the moduli space $\mathcal M_{\alpha, {\rm cont}}^{i, m}(y,x)$ does not admit the translation by $\R$-action, therefore its dimension is $\dim \mathcal M_{\alpha, {\rm cont}}^{i, m}(y,x) = |y|- |x| + i -\alpha$. Next, one can define a linear map from ${\rm CF}((H^{+})^{(p)}; \nu)$ to ${\rm CF}((H^{-})^{(p)}; \nu)$,
%\[ c^{+\to -, i, m}_{\nu, \alpha}(x \otimes b) = \sum_{y; |y| = |x| - i + \alpha} y \otimes \left(\sum_{u \in \mathcal M_{\alpha, {\rm cont}}^{i, m}(y,x)} (\sigma^{-1}_{m,y} \circ \nu_u^{-1})(b) \right), \]
%and accordingly, $c^{+\to -, i}_{\nu, \alpha} = \sum_{m=0}^{p-1} c^{+\to -, i, m}_{\nu, \alpha}$. Then, similarly to the definition of $d^{\Z/(p)}_{\nu}$ as above, by introducing the the following notation called {\it ${\rS}$-operator}, 
%\begin{equation}\label{S}
%{\rS}_{\alpha, (y,x)}^i : = \sum_{m=0}^{p-1} \sum_{u \in \mathcal M_{\alpha, {\rm cont}}^{i,m}(y, x)} (\sigma^{-1}_{m, y} \circ \nu_{u}^{-1}) \in {\rm Hom}_{\F_p}(\nu_x, \nu_y),
%\end{equation}
%one can define a $\F_p$-linear map $c^{+\to -}_{\Z/(p), \nu}$ as follows: the generator $x \otimes b \otimes 1$ mapped to 
%\begin{align*} 
%\sum_{\tiny{\begin{array}{c} i \geq 0 \\ y: \, |y| = |x|-2i \end{array}}} u^i \left(y \otimes \rS_{0, (y,x)}^{2i}(b) \otimes 1\right) + \sum_{\tiny{\begin{array}{c} i\geq0 \\ y: \, |y| = |x|-1 -2i \end{array}}} u^i \left(y \otimes \rS_{0, (y,x)}^{2i+1}(b) \otimes \theta\right)
%\end{align*} 
%and the generator $x \otimes b \otimes \theta$ is mapped to
%\begin{align*} 
%\sum_{\tiny{\begin{array}{c} i\geq 0 \\ y: \, |y| = |x|-2i \end{array}}} u^i \left(y \otimes \rS_{1, (y,x)}^{2i+1}(b) \otimes \theta\right) + \sum_{\tiny{\begin{array}{c} i\geq0 \\ y: \, |y| = |x| -1-2i \end{array}}}  u^{i+1} \left(y \otimes \rS_{1, (y,x)}^{2i+2}(b) \otimes 1\right).
%\end{align*} 
%Note that ${\rm deg}(c^{+\to -}_{\Z/(p), \nu}) = 0$. Moreover, the (ordered) composition of $\rT$-operations and $\rS$-operations plays important role later (in particular, for the proof of Proposition \ref{prop-cont-com-diff} below).  similarly to (\ref{compo-T}), we have 
%\begin{equation} \label{ST} 
%{\rS}_{\alpha', (z,y)}^j  \circ {\rT}_{\alpha, (y,x)}^i  = \sum_{m, n = 0}^{p-1}  \sum_{(v,u) \in \mathcal M_{\alpha', {\rm cont}}^{j,n}(z,y) \times \mathcal M_{\alpha}^{i, m}(y,x)} (\sigma^{-1}_{m+n, z} \circ \nu_{v+\frac{n}{p}}^{-1} \circ  \nu_u^{-1})
%\end{equation}
%and 
%\begin{equation} \label{TS}
%{\rT}_{\alpha', (z,y)}^j  \circ {\rS}_{\alpha, (y,x)}^i \sum_{m, n = 0}^{p-1}  \sum_{(v,u) \in \mathcal M_{\alpha'}^{j,n}(z,y) \times \mathcal M_{\alpha, {\rm cont}}^{i, m}(y,x)} (\sigma^{-1}_{m+n, z} \circ \nu_{v+\frac{n}{p}}^{-1} \circ  \nu_u^{-1})
%\end{equation}
%where the composition of the induced maps on the local system $\nu_{v+\frac{n}{p}}^{-1} \circ  \nu_u^{-1}$ in both (\ref{ST}) and (\ref{TS}) only depends on the homotopy class $[v] \# [u]$. Here comes a crucial proposition, with the proof in the same theme of the proof of Proposition \ref{prop-diff}.  
%
%\begin{prop} \label{prop-cont-com-diff} $c^{+\to -}_{\Z/(p), \nu}  \circ d_{\nu}^{\Z/(p)} = d_{\nu}^{\Z/(p)} \circ c^{+\to -}_{\Z/(p), \nu}$. \end{prop}
%
%\begin{proof} We will give a detailed proof of the desired equality on the generator $x \otimes b \otimes 1$. The proof for the generator $x \otimes b \otimes \theta$ goes in the same way. 
%
%\medskip
%
%On the one hand, we have 
%\begin{align*}
%(c^{+\to -}_{\Z/(p), \nu}  \circ d_{\nu}^{\Z/(p)}) (x \otimes b \otimes 1) = \left(\sum_{\ell \geq 0} A_\ell(b) u^\ell\right) \otimes 1 + \left( \sum_{\ell \geq 0} B_\ell(b) u^\ell \right) \otimes \theta
%\end{align*}
%where 
%\[ A_\ell = \sum_{\tiny{\begin{array}{c} i+j = \ell\\ y: \, |y| = |x|+1-2i \\ z: \, |z| = |y|-2j  \end{array}}} z \otimes \left(\rS_{0, (z,y)}^{2j} \circ \rT^{2i}_{0, (y, x)} \right) + \sum_{\tiny{\begin{array}{c} i+j = \ell-1\\ y: \, |y| = |x|-2i \\ z: \, |z| = |y|-1-2j  \end{array}}} z \otimes \left(\rS_{1, (z,y)}^{2j+2} \circ \rT^{2i+1}_{0, (y, x)} \right)\]
%and
%\[ B_\ell = \sum_{\tiny{\begin{array}{c} i+j = \ell\\ y: \, |y| = |x|+1-2i \\ z: \, |z| = |y|-1-2j  \end{array}}} z \otimes \left(\rS_{0, (z,y)}^{2j+1} \circ \rT^{2i}_{0, (y, x)} \right) + \sum_{\tiny{\begin{array}{c} i+j = \ell\\ y: \, |y| = |x|-2i \\ z: \, |z| = |y|-2j  \end{array}}} z \otimes \left(\rS_{1, (z,y)}^{2j+1} \circ \rT^{2i+1}_{0, (y, x)} \right).\]
%On the other hand, we have 
%\[ (d_{\nu}^{\Z/(p)} \circ c^{+\to -}_{\Z/(p), \nu}) (x \otimes b \otimes 1) = \left(\sum_{\ell \geq 0} C_\ell(b) u^\ell\right) \otimes 1 + \left( \sum_{\ell \geq 0} D_\ell(b) u^\ell \right) \otimes \theta\]
%where 
%\[ C_\ell =\sum_{\tiny{\begin{array}{c} i+j = \ell\\ y: \, |y| = |x|-2i \\ z: \, |z| = |y|+1-2j  \end{array}}} z \otimes \left(\rT_{0, (z,y)}^{2j} \circ \rS^{2i}_{0, (y, x)} \right) + \sum_{\tiny{\begin{array}{c} i+j = \ell-1\\ y: \, |y| = |x|-1-2i \\ z: \, |z| = |y|-2j  \end{array}}} z \otimes \left(\rT_{1, (z,y)}^{2j+2} \circ \rS^{2i+1}_{0, (y, x)} \right) \]
%and 
%\[ D_{\ell} = \sum_{\tiny{\begin{array}{c} i+j = \ell\\ y: \, |y| = |x|-2i \\ z: \, |z| = |y|-2j  \end{array}}} z \otimes \left(\rT_{0, (z,y)}^{2j+1} \circ \rS^{2i}_{0, (y, x)} \right) + \sum_{\tiny{\begin{array}{c} i+j = \ell\\ y: \, |y| = |x|-1-2i \\ z: \, |z| = |y|+1-2j  \end{array}}} z \otimes \left(\rT_{1, (z,y)}^{2j+1} \circ \rS^{2i+1}_{0, (y, x)} \right).\] 
%Then we claim that for each $\ell \geq 0$, we have $A_{\ell} = C_{\ell}$ and $B_{\ell} = D_{\ell}$ in $\Z/(p)$. For brevity, we only give the proof of the first one. Indeed, for any orbit $z \in {\mathcal O}(H^{(p)})$ with $|z| - |x| =1 -2\ell$, consider the following moduli space 
%\begin{equation} \label{diff-M-B}
%\mathcal M^{2\ell, \ast}_{0, {\rm cont}, \beta}(z,x) \,\,\,\,\mbox{for $\ast \in \Z/(p)$ and a homotopy class $\beta$}.
%\end{equation}
%The dimension of $\mathcal M^{2\ell, \ast}_{0, {\rm cont}, \beta}(z,x)$ is $1-2\ell + 2\ell - 0 = 1$. Now, consider the parametrized-type compactification of $\mathcal M^{2\ell, \ast}_{0, {\rm cont}, \beta}(z,x)$ that was mentioned at the end of subsection \ref{ssec-comp}. By the relation (33) in \cite{SZhao}, which is similar to Example \ref{ex-bd-compactification} and Case A in the proof of Proposition \ref{prop-diff}, its boundary as the codimension-1 stratum $\partial \overline{\mathcal M^{2\ell, \ast}_{0, {\rm cont}, \beta}(z,x)}$ is equal to the following big union, 
%\begin{align*}
%& \bigsqcup_{\beta_1 + \beta_2 = \beta} \left(\mathcal M^{-2k, m}_{0, \beta_1}(y, x) \times \mathcal M^{2\ell+2k, n}_{0, \beta_2, {\rm cont}}(z,y)\right) \cup \left(\mathcal M^{1-2k, m}_{0, \beta_1}(y,x) \times \mathcal M^{2\ell+2k, n}_{1, \beta_2, {\rm cont}}(z,y)\right)\\
%& \,\,\,\,\,\,\,\,\,\,\,\,\,\,  \cup \left(\mathcal M^{-2k, m}_{0, \beta_1, {\rm cont}}(y, x) \times \mathcal M^{2\ell+2k, n}_{0, \beta_2, {\rm cont}}(z,y)\right) \cup \left(\mathcal M^{1-2k, m}_{0, \beta_1, {\rm cont}}(y,x) \times \mathcal M^{2\ell+2k, n}_{1, \beta_2, {\rm cont}}(z,y)\right).
%\end{align*}
%By setting $(i,j) = (-k, \ell+k)$ or $(i,j) = (-k, \ell+k-1)$, we can rewrite $A_{\ell} - C_{\ell}$ precisely in terms of the broken Floer trajectories in $\partial \overline{\mathcal M^{2\ell, \ast}_{0, {\rm cont}, \beta}(z,x)}$. Then we obtain the desired conclusion that $A_{\ell} - C_{\ell} = 0$ in $\F_p$ by the same argument as the end of the proof of Proposition \ref{prop-diff}. 
%\end{proof}
%
%Proposition \ref{prop-cont-com-diff} implies that we have a well-defined map called $\Z/(p)$-equivariant continuation map with local system $\nu$, denoted by $c^{+\to -}_{\Z/(p), \nu}: {\rm HF}_{\Z/(p)}^*((H^{+})^{(p)}; \nu) \to {\rm HF}_{\Z/(p)}^*((H^{-}){(p)}; \nu)$ whenever $(H^+)^{(p)} \preceq (H^-)^{(p)}$. Moreover, since the local system does not involve the action filtration, when restricted to any action window $(s,t]$ and homotopy class $a \in \pi_0(\mathcal LD)$, we have a well-defined map 
%\begin{equation} \label{zp-cont-action-window}
%c^{+\to -}_{a, \Z/(p), \nu}: {\rm HF}_{a, \Z/(p)}^{*, (s,t]}((H^{+})^{(p)}; \nu) \to {\rm HF}_{a, \Z/(p)}^{*, (s,t]}((H^{-}){(p)}; \nu).
%\end{equation}
%The following proposition proves the functorial property of the $\Z/(p)$-equivariant continuation maps. 
%
%\begin{prop} \label{prop-trans} Suppose $H_1 \preceq H_2 \preceq H_3$, then the $\Z/(p)$-equivariant continuation maps with local system $\nu$ (on cohomologies) satisfies 
%\[ c^{2\to 3}_{a, \Z/(p), \nu} \circ c^{1\to 2}_{a, \Z/(p), \nu} = c^{1\to 3}_{a, \Z/(p), \nu}. \]
%In particular, $c^{H\to H}_{a, \Z/(p), \nu} = \mathds{1}$ for any Hamiltonian $H$. \end{prop}
%
%\begin{proof} The proof is a direct analogue of the proof of Theorem \ref{prop-cont-com-diff} by considering the parametrized-type compactification of the moduli spaces and comparing it with the composition of the $\rS$-operations. \end{proof}
%
%Proposition \ref{prop-trans} implies the following key definition. 
%
%\begin{df} \label{dfn-zp-sc} Let $(D, \lambda)$ be a Liouville domain and $p$ be a prime number. The {\bf $\Z/(p)$-equivariant symplectic cohomology of $(D, \lambda)$ in class $a \in \pi_0(\mathcal LD)$ with a $\Z/(p)$-equivariant rank-1 local system $\nu$ over $\F_p$ on $\mathcal LD$} is defined as follows,
%\[ {\rm SH}_{a, \Z/(p)}^*(D; \nu) = \varinjlim_{i \to \infty}{\rm HF}_{a, \Z/(p)}^{*}((H_{i})^{(p)}; \nu) \]
%where $\{H_i\}_{i \in \N}$ is any sequence of linear Hamiltonian on $(\hat{D}, d\lambda)$ with slopes increasing to infinity and the direct limit is taken over $\Z/(p)$-equivariant continuation maps. In the same way, one defines truncated $\Z/(p)$-equivariant symplectic cohomology with local system $\nu$ by using (\ref{dfn-fil-zp-hf-ls}) and (\ref{zp-cont-action-window}),
%\begin{equation*} \label{dfn-sh-fil-tr-ls} 
%{\rm SH}_{a, \Z/(p)}^{*, (s, t]}(D; \nu) = \varinjlim_{i \to \infty}{\rm HF}_{a, \Z/(p)}^{*, (s,t]}((H_{i})^{(p)}; \nu)
%\end{equation*}
%for any $s < t$ in $\R$.
%\end{df}
%
%\begin{rmk} (1) It is easy to check that Definition \ref{dfn-zp-sc} is well-defined, that is, ${\rm SH}_{\Z/(p)}^*(D; \nu)$ is independent of the sequences of linear Hamiltonians with slopes increasing to infinity. (2) Compared with Definition \ref{dfn-sh-ls}, in order to construct ${\rm SH}_{a, \Z/(p)}^{*} (D; \nu)$ in Definition \ref{dfn-zp-sc}, it is crucial to consider the Hamiltonian $p$-th power $H_i^{(p)}$ so that the building blocks, $\Z/(p)$-equivariant Hamiltonian cohomologies, admit a $\Z/(p)$-action. \end{rmk}
%
%\subsubsection{$\Z/(p)$-equivariant pants product} \label{ssec-Zp-pp} 
%This subsection is devoted to the construction of the $\Z/(p)$-equivariant pants product, equipped with a local system as in Definition \ref{dfn-p-adm} below, as well as the map $\mathcal P_{\nu}$ in (\ref{2-p-ends-coh}), that is, 
%\[ \mathcal P_{\nu}: H^*\left(\Z/(p);  {\rm CF}_a^{*, (s,t]}(H; \nu)^{\otimes p}\right) \to {\rm HF}_{a^p, \Z/(p)}^{*, (ps,pt]}(H^{(p)}; \nu).\]
%Let us start from the following definition. 
%
%\begin{df}  \label{dfn-p-adm} We call a rank-1 local system $\nu$ over $\K$ on $\mathcal LD$ {\bf $p$-admissible} if any punctured embedded Riemannian surface $u$ with genus $g=0$ in $D$ and with $p$-many negative punctures asymptotic to closed orbits $(x_1, …, x_p)$ and one positive puncture asymptotic to the closed orbit $x_0$ induces a map $\nu_u: \nu_{x_1} \otimes \cdots \otimes \nu_{x_p} \to \nu_{x_0}$. \end{df}
%
%\begin{rmk} In practice, a $p$-admissible local system can be obtained from transgression process. We will elaborate this later in Section \ref{sec-lsc}.\end{rmk}
%
%The $\Z/(p)$-equivariant pants product on Hamiltonian Floer cohomology with a modification by adding a $p$-admissible local system $\nu$ goes as follows. Similarly to any Floer-type construction, the pants product starts from considering a parametrized moduli space, denoted by $\mathcal M^{i, m}_{\mathcal P, \alpha}(x^-; x_0^+, …, x^+_{p-1})$ where $i \in \N$, $m \in \{0, …, p-1\}$, $\alpha \in \{0,1\}$, and 
%\[ x^- \in \mathcal O(H^{(p)}) \,\,\,\,\mbox{and}\,\,\,\, (x_0^+, …, x_{p-1}^+) \in \mathcal O(H)^{\times p}. \]
%This moduli space consists of an embedded Riemannian surface $u$ with {\it negative} asymptotic end being $x^-$ and $p$-many {\it positive} asymptotic ends as an ordered orbit set $(x_0^+, …, x_{p-1}^+)$, together with a parametrization by a Morse flow $w: \R \to S^{\infty}$, and satisfying a perturbed Cauchy-Riemann equation. As in the previous cases, all such moduli spaces need to admit a $\Z/(p)$-action. Here, with an appropriately chosen almost complex structure and perturbation data (in order to achieve desired transversality), the $\Z/(p)$-action on $u$ rotates $x^-$ by $R_p^m(x^-)$ and permutes the $p$-many positive ends $(x_0^+, …, x_{p-1}^+)$. For details, see Page 36-37 in \cite{SZhao}.
%
%Then if $|x^-| = \sum_{k=0}^{p-1} |x_k^+| - i+\alpha$, define $\mathcal P_{\nu, \alpha}^{i,m}: {\rm CF}^*(H; \nu)^{\otimes p} \to {\rm CF}^{*-i+\alpha}(H^{(p)}; \nu)$ by 
%\[ \mathcal P_{\nu, \alpha}^{i,m}((x_0^+ \otimes b_0) \otimes \cdots \otimes (x_{p-1}^+ \otimes b_{p-1})) = \sum_{x^-} \sum_{u \in \mathcal M^{i, m}_{\mathcal P, \alpha}(x^-; x_0^+, …, x^+_{p-1})} x^- \otimes \nu_{u}(b_0 \otimes \cdots \otimes b_{p-1}). \]
%Note that $\nu_{u}(b_0 \otimes \cdots \otimes b_{p-1})$ is well-defined exactly due to the $p$-admissible hypothesis on $\nu$ in Definition \ref{dfn-p-inv}. Then denote $\mathcal P_{\nu, \alpha}^{i} = \sum_{m=0}^{p-1} \mathcal P_{\nu, \alpha}^{i,m}$ and define the deformed version $\mathcal P_{\nu}: {\rm CF}^*(H; \nu)^{\otimes p} \otimes_{\F_p} \F_p[[u]]\left<\theta\right> \to {\rm CF}^*(H^{(p)}; \nu) \otimes_{\F_p} \F_p[[u]]\left<\theta\right>$ by 
%\begin{equation} \label{ls-pants}
%\mathcal P_{\nu}( - \otimes 1) = \sum_{i=0}^{\infty} u^i \mathcal P_{\nu, 0}^{2i} \otimes 1 + \sum_{i=0}^{\infty} u^i \mathcal P_{\nu, 0}^{2i+1}\otimes  \theta
%\end{equation}
%and 
%\begin{equation} \label{ls-pants-2}
%\mathcal P_{\nu}( - \otimes \theta) = \sum_{i=0}^{\infty} u^i \mathcal P_{\nu, 1}^{2i+1} \otimes \theta + \sum_{i=0}^{\infty}u^{i+1} \mathcal P_{\nu, 1}^{2i}  \otimes 1. 
%\end{equation} 
%By investigating the action and energy, the existence of any embedded Riemannian surface $u \in \mathcal M^{i, m}_{\mathcal P, \alpha}(x^-; x_0^+, …, x^+_{p-1})$ implies that 
%\[ \mathcal A_{H^{(p)}}(x^- \otimes \nu_{u}(b_0 \otimes \cdots \otimes b_{p-1})) - \sum_{i=1}^p \mathcal A_H(x^+_i \otimes b_i) = E(u) \geq 0. \]
%Therefore, for any $s \in \R$ such that $\mathcal A_H(x^+_i \otimes b_i) \geq s$, we have $\mathcal A_{H^{(p)}}(x^-\otimes \nu_{u}(b_1 \otimes \cdots \otimes b_p))  \geq ps$. This implies that for any $s \leq t \in \R$ we have a well-defined map 
%\begin{equation} \label{p-ends-chain}
%\mathcal P_{\nu, \alpha}^{i,m}: {\rm CF}^{*, (s,t]}(H; \nu)^{\otimes^p} \to {\rm CF}^{*, (ps, pt]}(H^{(p)}; \nu). 
%\end{equation}
%Since this holds for any $i, m$ and $\alpha$, (\ref{ls-pants}) and (\ref{ls-pants-2}) imply that we have a well-defined map after extending the coefficient, 
%\begin{equation} \label{p-ends-chain}
%\mathcal P_{\nu}: {\rm CF}^{*, (s,t]}(H; \nu)^{\otimes p} \otimes_{\F_p} \F_p[[u]]\left<\theta\right> \to {\rm CF}^{*, (ps,pt]}(H^{(p)}; \nu) \otimes_{\F_p} \F_p[[u]]\left<\theta\right>.
%\end{equation}
%The $\otimes p$-term ${\rm CF}^{*, (s,t]}(H; \nu)^{\otimes p}$ on the left-hand side in (\ref{p-ends-chain}) admits a $\Z/(p)$-action simply by permuting the elements in the $p$-tuple. Therefore, by considering the standard group cohomology differential, the left hand side provides the (Hamiltonian) group cohomology denoted by $H^*(\Z/(p);  {\rm CF}^{*, (s,t]}(H; \nu)^{\otimes p})$. Meanwhile, due to Proposition \ref{prop-diff}, the right-hand side provides the truncated $\Z/(p)$-equivariant Hamiltonian Floer homology ${\rm HF}_{\Z/(p)}^{*, (ps,pt]}(H^{(p)}; \nu)$. Subsection 8.3 in \cite{SZhao} shows that $P_{\nu}$ in (\ref{p-ends-chain}) descents to a well-defined map
%\begin{equation*} \label{p-ends-coh}
%\mathcal P_{\nu}: H^*\left(\Z/(p);  {\rm CF}^{*, (s,t]}(H; \nu)^{\otimes p}\right) \to {\rm HF}_{\Z/(p)}^{*, (ps,pt]}(H^{(p)}; \nu).
%\end{equation*}
%%Then the desired $\mathcal P_{\nu, {\rm sh}}$ in Proposition \ref{prop-sh-iso} is the composition of the following maps, 
%%\begin{align*}
%%H^*\left(\Z/(p); {\rm SH}^{*, (s,t]} (D; \nu)^{\otimes p}\right) & \simeq  H^*\left(\Z/(p);  {\rm HF}^{*, (s,t]}(H_{i_0}; \nu)^{\otimes p}\right)\\
%%& \xrightarrow{\mathcal P_\nu} {\rm HF}_{\Z/(p)}^{*, (ps,pt]}(H_{i_0}^{(p)}; \nu)\\
%%& \xrightarrow{\iota} {\rm SH}_{\Z/(p)}^{*, (ps,pt]}(D; \nu)
%%\end{align*}
%Finally, we can impose the constraint of free homotopy classes on Floer cohomologies, along the construction above. Since we only consider genus zero Riemannian surfaces in the construction of pants product map $\mathcal P_{\nu}$, the homotopy classes are clearly $a$ and $a^p$ as requested in (\ref{2-p-ends-coh}).
%
%\section{Proof of Proposition \ref{prop-top-cond}} \label{sec-lsc}
%
%%We will apply transgression (see subsection 7.2 in \cite{Rit13}) to get a local system $\nu$ that satisfies the $p$-admissible condition. Then we confirm that such $\nu$ is within the choices of local system in Theorem 2 in \cite{AFO17} that satisfy the vanishing property. Finally, by analyzing the process of obtaining such $\nu$ in \cite{AFO17}, we observe that it automatically satisfies the $\Z/(p)$-equivariant property. 
%
%Recall that the abelian group formed by local systems over $\Z/(p)$ on $\mathcal LL$ is isomorphic to ${\rm Hom}(\pi_1(\mathcal LL), \Z/(p))$. 
%Observe that any element $C \in \pi_1(\mathcal LL)$ is in fact a class in $[\mathbb T^2; L]$. Consider the following map 
%\begin{equation} \label{dfn-trans}
%s^*: H^2(L; \Z/(p)) \to H^1(\mathcal LL; \Z/(p)) \,\,\,\, \mbox{by}\,\,\,\, s^*(\tau)([C]) = \int_C \tau
%\end{equation}
%for any cohomology class $\tau \in H^2(L; \Z/(p))$ and homology class $[C] \in H_1(\mathcal LL; \Z/(p))$. The map $s^*$ is called a {\it transgression} (see subsection 7.2 in \cite{Rit13}). Fix a class $\tau \in H^2(L; \Z/(p))$, then (\ref{dfn-trans}) says that $s^*(\tau)$ defines a local system over $\Z/(p)$ on $\mathcal LL$. For brevity and the consistency of the notation for a local system, denote 
%\begin{equation} \label{s-trans}
%\nu^{\tau}: = s^*(\tau). 
%\end{equation}
%Here, we provide an example of a transgression that comes from the discussion in subsection \ref{ssec-AFO-ls}. 
%
%\begin{exa} \label{ex-trans}
%Denote by $p: \tilde{L} \to L$ the covering map where $\tilde{L}$ is the universal cover of $L$ (in particular $\pi_1(\tilde{L}) =0$). Then we have the following induced map 
%\begin{equation} \label{covering_map}
%p^*: H^2(L; \Z/(p)) \to H^2(\tilde{L}; \Z/(p)) = {\rm Hom}(H_2(\tilde{L}; \Z/(p)); \Z/(p)).
%\end{equation}
%In fact, more precisely $p^*$ maps into $H^2_{\rm inv}(\tilde{L}; \Z/(p))$, the $\pi_1(L)$-invariant subspace of $H^2(\tilde{L}; \Z/(p))$ (cf.~(\ref{inv-inclusion})). Taken any $\tau \in H^2(L; \Z/(p))$, we claim that $p^*\tau$ is a transgression. Indeed, consider the following canonical identification 
%\begin{equation} \label{can-isos}
%\pi_1(\mathcal LM) \simeq \pi_2(M) \simeq \pi_2(\tilde{M}) \simeq H_2(\tilde{M}; \Z) 
%\end{equation}
%where the first and the third isomorphisms are the identity and the second isomorphism is the inverse of the induced map $p_*$, that is, $(p_*)^{-1}$. Then for any $[C] \in \pi_1(\mathcal LM)$, we have 
%\[ \left<p^*\tau, (p_*)^{-1}([C]) \right> = \left< \tau, (p_*\circ (p_*)^{-1})([C]) \right>= \int_C \tau. \]
%In this way, the transgression can be obtained by $p^*$. Note that $p^* = s^*$ in a tautological way, up to the canonical isomorphisms in (\ref{can-isos}). 
%\end{exa}
%
%The following is a general result for a local system constructed in (\ref{s-trans}). 
%
%\begin{lma} \label{lemma-trans-p-adm} For any $\tau \in H^2(M; \Z/(p))$, the local system $\nu^{\tau}$ is $p$-admissible. \end{lma}
%
%\begin{proof} For simplicity, we will consider the case that any punctured embedded Riemannian surface $u$ with genus $g=0$ in $M$ and with $2$-many negative punctures asymptotic to closed orbits $(x, y)$ and one positive puncture asymptotic to the closed orbit $z$ induces a map $\nu^{\tau}_u: \nu_{x} \otimes \nu_{y} \to \nu_{z}$. The general case is proved in the same way. 
%
%Assume that $x \in \mathcal L_{\alpha}M$ and $y \in \mathcal L_{\beta}M$ for loop classes $\alpha$ and $\beta$. Fix base point $\eta_{\alpha}$ of $\mathcal L_{\alpha}M$ and $\eta_{\beta}$ of $\mathcal L_{\beta} M$ such that $\eta_{\alpha}(0) = \eta_{\beta}(0) = x_0 \in M$. Then the concatenation at point $x_0$, $\eta_{\alpha} \ast \eta_{\beta} \in \mathcal L_{\alpha \ast \beta} M$, where $\alpha \ast \beta$ is the Pontryagin product of classes $\alpha$ and $\beta$ with respect to point $x_0$. 
%
%For a given punctured embedded Riemannian surface $u$, since $g = 0$, the loop $z \in \mathcal L_{\alpha \ast \beta} M$. Moreover, within their corresponding connected components, there exist pathes (may not be unique)
%\[ \gamma_x: \eta_{\alpha} \to x, \,\,\, \gamma_y: \eta_{\beta} \to y, \,\,\, \mbox{and}\,\,\, \gamma_z: z \to \eta_{\alpha} \ast \eta_{\beta}. \]
%By concatenating $u$ with $\gamma_x$, $\gamma_y$ and $\gamma_z$, we extend $u$ to $\tilde{u}$ which is a punctured Riemannian surface $\tilde{u}$ with two negative ends $\eta_{\alpha}$, $\eta_{\beta}$, and one positive end $\eta_{\alpha} \ast \eta_{\beta}$. See Figure \ref{extension}. 
%% Figure environment removed
%Observe that 
%\[ \partial \tilde{u} = \eta_{\alpha} \ast \eta_{\beta} - \eta_{\alpha} - \eta_{\beta} =0.\]
%This implies that we have a well-defined morphism $\nu^{\tau}_{\tilde u}: \nu^{\tau}_{\eta_{\alpha}} \otimes \nu^{\tau}_{\eta_{\beta}} \to \nu^{\tau}_{\eta_{\alpha} \ast \eta_{\beta}}$ by integrating any representative of $\tau$ over $\tilde{u}$. Then the desired morphism $\nu^{\tau}_u$ is defined as the composition of the following morphisms, 
%\[ \nu^{\tau}_x \otimes \nu^{\tau}_y \xrightarrow{\nu^{\tau}_{\gamma_x} \otimes \nu^{\tau}_{\gamma_y}} \nu^{\tau}_{\eta_{\alpha}} \otimes \nu^{\tau}_{\eta_{\beta}} \xrightarrow{\nu^{\tau}_{\tilde{u}}} \nu^{\tau}_{\eta_{\alpha} \ast \eta_{\beta}} \xrightarrow{\nu^{\tau}_{\gamma_z}} \nu^{\tau}_z. \]
%Suppose we take a different path $\gamma'_x$ from $\eta_{\alpha}$ to $x$. Then $\nu^{\tau}_{\gamma'_x} = \rho \circ \nu^{\tau}_{\gamma'_x}$ for some $\rho \in {\rm Hom}(\pi_1(\mathcal LM); \Z/(p))$. Meanwhile, we obtain a different extension $\tilde{u}'$ from $u$, and $\nu^{\tau}_{\tilde{u}} = \nu^{\tau}_{\tilde{u}'} \circ \rho^{-1}$ for the same $\rho$. Via $\gamma'_x$, denote the resulting morphism by $\nu^{\tau}_{u'}$. Then 
%\begin{align*}
%\nu^{\tau}_{u'} & = \nu^{\tau}_{\gamma_{z}} \circ \nu^{\tau}_{\tilde{u}'} \circ (\nu^{\tau}_{\gamma'_x} \otimes \nu^{\tau}_{\gamma_y}) \\
%& = \nu^{\tau}_{\gamma_z} \circ (\nu^{\tau}_{\tilde{u}} \circ \rho^{-1}) \circ ((\rho \circ \nu^{\tau}_{\gamma_x}) \otimes \nu^{\tau}_{\gamma_y}) \\
%& = \nu^{\tau}_{\gamma_{z}} \circ \nu^{\tau}_{\tilde{u}} \circ (\nu^{\tau}_{\gamma_x} \otimes \nu^{\tau}_{\gamma_y}) = \nu^{\tau}_{u}.
%\end{align*}
%The same argument works for $\gamma_y$ and $\gamma_z$. Therefore, our construction of morphism $\nu^{\tau}_u$ is canonical, i.e., independent of the connecting paths. Thus we complete the proof.  
%\end{proof}
%
%Now, we are ready to give the proof of Proposition \ref{prop-top-cond}.
%
%\begin{proof} [Proof of Proposition \ref{prop-top-cond}] Example \ref{ex-trans} and Lemma \ref{lemma-trans-p-adm} imply that the resulting local system denoted by $\nu^{\tau}$, more precisely $\nu^{p^*\tau}$, is $p$-admissible. On the other hand, since $p^*\tau$ is automatically $\pi_1(L)$-invariant, it is readily to verify that the local system $\nu^{\tau}$ is $S^1$-equivariant (see (3.4.10) in \cite{Abo15}) with the action induced by $\pi_1(L)$. In particular, it is $\Z/(p)$-equivariant by Definition \ref{dfn-p-inv}. Thus we obtain the desired conclusion. \end{proof} 
%
%\begin{rmk} Theorem 2 in \cite{AFO17} in fact proves a stronger conclusion that the local system $\nu^{\tau}$ on $\mathcal LD_g^*L$ that we obtained in Proposition \ref{prop-top-cond} can restrict to any prescribed local rank on $L$ (viewed as the zero section of $T^*L$).  \end{rmk}

\bibliographystyle{plainurl}
\bibliography{bibliographyR,bibliographyHTU,bibliographyHZ}
\medskip

\medskip

\end{document}