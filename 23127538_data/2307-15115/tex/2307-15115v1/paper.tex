\documentclass[twocolumn,superscriptaddress,floatfix,preprintnumbers,prl ,hyperref]{revtex4-2}
\usepackage[colorlinks=true,breaklinks=true]{hyperref}
\usepackage{comment}
\usepackage[utf8]{inputenc}
\hypersetup{allcolors=[rgb]{0.0 0.0 0.6},linkcolor=[rgb]{0.75 0.05 0.05}}
\usepackage{mathtools}
\usepackage[dvipsnames]{xcolor}
\usepackage{float}
\usepackage{letltxmacro}
\LetLtxMacro{\oldcite}{\cite}
\renewcommand{\cite}[1]{\mbox{\oldcite{#1}}}

\usepackage{orcidlink}


\long\def\exclude#1{}

\newcommand{\vs}{v_{\rm s}}
\newcommand{\rs}{r_{\rm s}}
\newcommand{\rh}{r_{\rm h}}
\newcommand{\rd}{r_{\rm d}}
\newcommand{\dSN}{d_{\rm SN}}
\newcommand{\gd}{\gamma_{\rm d}}
\newcommand{\Td}{T_{\nu,{\rm d}}}

\setlength{\bibsep}{0cm}
\bibpunct{[}{]}{,}{n}{}{,}
\newcommand{\df}[1]{\textcolor{purple}{{\bf DF}: #1}}
\newcommand{\edoardo}[1]{\textcolor{blue}{\textbf{Edoardo:} #1}}
\newcommand{\GR}[1]{{\color{red}{\bf GR:} #1}}
\newcommand{\georg}[1]{{\color{red}{\bf Georg:} #1}}

\begin{document}

\title{Large Neutrino Secret Interactions, Small Impact on Supernovae}

\author{Damiano F.\ G.\ Fiorillo \orcidlink{0000-0003-4927-9850}} 
%\email{damianofg@gmail.com}
\affiliation{Niels Bohr International Academy, Niels Bohr Institute,
University of Copenhagen, 2100 Copenhagen, Denmark}

\author{Georg G.\ Raffelt
\orcidlink{0000-0002-0199-9560}}%\email{raffelt@mpp.mpg.de}
\affiliation{Max-Planck-Institut f\"ur Physik (Werner-Heisenberg-Institut), F\"ohringer Ring 6, 80805 M\"unchen, Germany}

\author{Edoardo Vitagliano
\orcidlink{0000-0001-7847-1281}}%\email{edoardo@physics.ucla.edu}
\affiliation{Racah Institute of Physics, Hebrew University of Jerusalem, Jerusalem 91904, Israel}

\date{July 31, 2023}

\begin{abstract}
When hypothetical neutrino secret interactions ($\nu$SI) are large, they form a fluid in a supernova (SN) core, flow out with sonic speed, and stream away as a fireball. For the first time, we solve all steps, systematically using relativistic hydrodynamics, although a simplified source model. The impact on SN physics and the neutrino signal is remarkably small. Even for complete thermalization within the fireball, the observable spectrum barely changes. Small energy-transfer modifications may
affect the neutrino-driven explosion mechanism, but on present evidence are not ruled in or out. One potentially large effect beyond our study is quick deleptonization if $\nu$SI violate lepton number.
\end{abstract}

\maketitle

{\bf\textit{Introduction.}}---The cosmic dark-matter problem, the baryon asymmetry of the universe, the CP problem of QCD, and the unknown origin and nature of neutrino masses all suggest physics beyond the particle-physics Standard Model. One portal to new particle physics may be provided by hitherto unknown interactions among neutrinos~\cite{Berryman:2022hds}, with effects that are notoriously difficult to measure. Such neutrino secret interactions ($\nu$SI) must be mediated by a new force carrier of unknown spin parity and mass, and could conserve or violate lepton number. It has long been held \cite{Dicus:1982dk, Gelmini:1982rr, Kolb:1987qy, Manohar:1987ec, Berezhiani:1987gf, Dicus:1988jh, Fuller:1988ega, Berezhiani:1989za, Farzan:2002wx, Blennow:2008er, Heurtier:2016otg, Das:2017iuj, Shalgar:2019rqe, Chang:2022aas, Fiorillo:2022cdq, Cerdeno:2023kqo} that a natural test bed should be core-collapse supernova (SN) physics that is famously dominated by neutrinos \cite{Janka:2012wk, Janka2017Handbooka, Burrows+2020, Vitagliano:2019yzm, Mezzacappa+2020, Burrows+2021}.

If the coupling $g_\phi$ and mass $m_\phi$ of the new force carrier $\phi$ are small enough, the main effect is energy loss by $\phi$ radiation, providing the traditional cooling bounds based on the SN~1987A neutrino signal~\cite{Farzan:2002wx, Heurtier:2016otg}. For larger masses, decays $\phi\to\nu\nu$ are so fast that one would observe neutrinos with 100-MeV-range energies, representative of the SN core, in conflict with the legacy data~\cite{Fiorillo:2022cdq}. However, the exclusion region in the $g_\phi$--$m_\phi$ plane has a ceiling at $g_\phi$ so large that the $\nu$ and $\phi$ mean-free paths (MFPs) are small relative to the protoneutron star (PNS) radius. In this trapping regime, new phenomena arise that are the main subject of this {\em Letter}.

In an early paper \cite{Manohar:1987ec}, Manohar suggested that large $\nu$SI would impede diffusive escape from the SN core and could thus be constrained by SN~1987A data. This misunderstanding was countered by Dicus et al.\ \cite{Dicus:1988jh} who stressed that strongly coupled neutrinos should be treated as a relativistic fluid and studied free expansion after sudden release. 
Recently, Chang et al.\ \cite{Chang:2022aas} returned to this picture in spherical geometry and advocated this burst outflow as one possible scenario. However, while such solutions, along earlier ones in a different context \cite{vitello1976hydrodynamic, yokosawa1980relativistic}, nicely illustrate the behavior of a blob of relativistic fluid, they bear no resemblance to PNS cooling, where a quasi-thermal steady source emits neutrinos for several seconds, $10^5$ times larger than the PNS radius of 10~km. We dismiss such solutions because no SN-related mechanism has been proposed for creating the initial blob 
or its sudden release, and even for such conditions, the previous literature~\cite{vitello1976hydrodynamic,
yokosawa1980relativistic} has not identified the behavior 
envisioned in Ref.~\cite{Chang:2022aas}. 
However, Chang et al.\ \cite{Chang:2022aas} have considered a second scenario, steady wind outflow from the PNS, that makes immediate physical sense, although the authors questioned if special conditions were needed and if it could be realized in practice. This unfinished debate has inspired us to answer the open questions unambiguously.

We find that a simplified source model with a beginning and an end  of thermal emission spawns a dynamical solution with a luminal front expanding into space, indeed relaxes to steady emission near the PNS similar to the proposed wind outflow \cite{Chang:2022aas}, and finally becomes a fireball with constant thickness (see Fig.~\ref{fig:sketch} for a sketch). The observable duration and flux spectrum are astonishingly similar to the standard case, although tens-of-percent effects probably persist and could affect both SN physics and a high-statistics observation. The groundwork for our exploration is laid out in a detailed theoretical companion paper~\cite{Fiorillo:2023-Theory}.


{\bf\textit{Steady PNS neutrino emission.}}---The first question concerning PNS outflow is diffusion transport deep inside. Momentum conservation in steady state means the fluid's pressure gradient is balanced with the force caused by neutrino absorption. The energy flux $F=-(\bar\lambda/3)\nabla B$ is then found to be standard, where $B$ is the blackbody energy density prescribed by local thermal equilibrium (LTE). However, the average MFP $\bar\lambda$ differs somewhat from the usual Rosseland mean. Moreover, $B$ includes all flavors, $\bar\lambda$ is a flavor average, and if $m_\phi$ is small enough, $\phi$ also contributes to $B$. Therefore, $\nu$SI modify the exact mean opacity, but without dramatic effects. The PNS cooling speed anyway depends more on convection than diffusion transport \cite{Epstein1979, Burrows+1988, Keil+1996, Janka+2001proc, Dessart+2006, Nagakura+2020}.

% Figure environment removed

The critical next step is the transition regime between diffusion and free streaming. Usually one has to solve the Boltzmann collision equation, the numerically most expensive part of SN simulations. With $\nu$SI, neutrinos become a fluid, which, in its rest frame, defined by vanishing energy flux, relaxes to kinetic equilibrium, and also to chemical equilibrium if number-changing $\nu$SI processes are fast enough. Either way, the fluid properties are its flow velocity $v$ and lab-frame energy density~$e$, providing an energy flux $F=e\,4v/(3+v^2)$. The nuclear medium and neutrino fluid exchange energy and momentum by collisions; the main difference to the diffusion regime is that LTE no longer applies and that $v$ is relativistic---it increases from near zero deep inside to outflow with the speed of sound $\vs=1/\sqrt{3}=0.577$. (We~use natural units with $c=\hbar=k_{\rm B}=1$.) In principle, it is straightforward to solve the coupled hydrodynamic equations for the nuclear medium and the neutrino fluid.

The simplest case is a static, uniform, semi-infinite, and isothermal body, that usually emits neutrino blackbody radiation. With $B$ the LTE radiation density deep inside, the usual energy density outside the surface is $e=B/2$ and Stefan-Boltzmann energy flux $F=B/4$. For the fluid, instead we find comoving energy density $\rho=0.21\,B$, outflow velocity $\vs$, lab-frame energy density $e=0.35\,B$, and $F=e\,2\sqrt{3}/5\simeq0.24\,B$, very close to the standard value \cite{Fiorillo:2023-Theory}. While sonic outflow is generic, $\rho$ and thus $F$ depend on a certain spectral average of the MFP. The cited values assume an energy-independent MFP. If number-changing $\nu$SI are too slow to maintain chemical equilibrium, another boundary condition at the surface is $n=0.29\,n^{\rm th}$ for the $\nu$ or $\bar\nu$ number density, where $n^{\rm th}$ is the LTE value in the isothermal emitter.  

All neutrino species are emitted, and, if the mediator is light enough, it contributes an additional degree of freedom (assuming spin 0), accelerating PNS cooling. However, since the mediator counts as one boson degree of freedom, compared to the six neutrino species it only increases the emission rate by~16\%.

In plane geometry, these boundary values apply everywhere outside the body. In spherical geometry, the bulk velocity profile settles into a stationary profile~\cite{Fiorillo:2023-Theory}
\begin{equation}
    v(1-v^2)=\frac{2}{3\sqrt{3}}\left(\frac{\rs}{r}\right)^2,
\end{equation}
with $\rs$ the surface radius. With increasing radius, $v$ rapidly rises from $\vs$ to $c$ and asymptotically reaches a Lorentz factor
\begin{equation}\label{eq:gamma_factor}
    \gamma\simeq\frac{3^{3/4}}{\sqrt{2}}\,\frac{r}{\rs}
    \simeq 1.61\,\frac{r}{\rs}.
\end{equation}
Conservation of the energy flux $F=4\gamma^2 \rho v r^2/3$ together with the surface value reveals the asymptotic comoving energy density $\rho\simeq(0.21\,B/3)(\rs/r)^4$, in the lab frame $e=(4\gamma^2-1)\rho/3\to 4\gamma^2\rho/3$. When number-changing reactions are slow, conservation of the number flux $\gamma n v r^2$ reveals asymptotically $n\simeq (0.29/3^{3/4})\,n^\mathrm{th}(\rs/r)^3$ for the comoving number density.

We show the numerical flow parameters in Fig.~\ref{fig:flow_parameters}, using a smoothed PNS surface at $\rs=10$~km. The inverse MFP is $\lambda_{\nu N}^{-1}=5~{\rm km}^{-1}$ for $r\leq\rs$, whereas for larger $r$ it is reduced by $e^{-6\,(r-\rs)/\mathrm{km}}$. The temperature is represented by $B$ constant inside, and outside following the same suppression profile. We see the fast acceleration of the fluid in the surface region and concomitant drop in the lab-frame energy density, corresponding to conserved energy flux. We show the asymptotic values as dashed lines, which truly apply to a hard emission surface.

% Figure environment removed

In the comoving frame, the fluid is in equilibrium and thus characterized by its internal $T_\nu$ and quasi-chemical potential $\mu_\nu$ (same sign for $\nu$ and $\bar\nu$) that vanishes if number-changing processes are fast enough. In this case, the asymptotic $\rho$ implies $T_\nu=0.514\,T\,(\rs/r)$, a drop that is compensated by the work needed for the expansion, or equivalently, the increasing bulk radial motion: the particles simply become more collinear. The increasing $\gamma$ implies a constant asymptotic lab-frame neutrino \hbox{energy} of $\overline{\epsilon}=3.48\,T$. Without $\nu$SI, the lab-frame spectrum is thermal with the emitter's $T$ and thus  $\overline{\epsilon}=3.15\,T$.

If number-changing reactions are not in equilibrium, a nonvanishing degeneracy parameter $\eta=\mu_\nu/T_\nu$ develops, based on number and energy flux conservation. The asymptotic values are $\eta=-0.363$, $T_\nu=0.561\,T\, (\rs/r)$, and lab-frame $\overline{\epsilon}=3.75\,T$.  Constant $\eta$ implies entropy conservation (adiabatic expansion) once the fluid has settled into steady motion at $r$ larger than a few~$\rs$.

{\bf\textit{Relaxation to steady state.}}---Our steady solution is similar to the wind outflow of Ref.~\cite{Chang:2022aas}, who however seemed unsure about its validity. We have numerically solved the dynamical case, where PNS emission is suddenly switched on, mimicking core collapse, and leading to a luminal neutrino-fluid front \cite{Fiorillo:2023-Theory}. Near the PNS, the solution asymptotically approaches steady outflow. While our treatment of the source is schematic, we are confident that it correctly captures the transient. The main innovation is to include energy exchange with the nuclear medium, allowing the PNS to act as an energy reservoir that feeds neutrino emission for a long time compared with the PNS size. As in Ref.~\cite{Chang:2022aas}, we match the diffusion solution inside the PNS to the free-streaming one outside, but in addition the energy outflow is fixed in terms of the PNS temperature.



{\bf\textit{Neutrino fireball expansion.}}---As the PNS cools, neutrino emission drops. After $\delta t$ of a few seconds, a shell of width $\delta t$ has been emitted. The subsequent evolution is well understood in the context of fireballs in gamma-ray bursts~\cite{Piran:1993jm} (see also Refs.~\cite{vitello1976hydrodynamic, yokosawa1980relativistic} for early theoretical studies and Refs.~\cite{Diamond:2023scc, Diamond:2023cto} for particle bounds from astrophysical transients). Since most of the fluid moves with $v\simeq 1$, the shell thickness cannot change, 
but its radius gradually expands. Within the shell, our steady-state flow parameters remain valid. 

This is also seen if we notice that, in steady state, a disturbance in the fluid travels with $\vs$ in the comoving frame, which however accelerates with increasing radius. As we have worked out \cite{Fiorillo:2023-Theory}, there is a sound horizon $\rh\simeq 1.13\,\rs$. The fluid at larger $r$ is unaware of anything happening at the emission surface, such as the source turning off. 

As the thickness does not change, the fireball is not a self-similar solution because it contains a characteristic length $\delta t$ \footnote{The self-similar burst outflow of Ref.~\cite{Chang:2022aas} has the velocity profile $v=r/t$. The corresponding energy density that solves the fluid equations is $\rho\propto \gamma^4/t^4$ \cite{Blandford:1976uq} and thus diverges at $r=t$; it contains an infinite amount of energy concentrated immediately behind the shock. It is unclear whether there are initial conditions that would evolve into this solution. Seminal numerical studies~\cite{vitello1976hydrodynamic,yokosawa1980relativistic} conclude that a homogeneous neutrino sphere initially at rest does not evolve into this self-similar solution.}. In fact, for the free expansion of a relativistic gas, self-similar solutions with regular behaviors do not seem to exist. In our case, regular behavior is attained because, for energy injection over a period $\delta t$, the system always keeps memory of the scale $\delta t$ (see, e.g., Refs.~\cite{vitello1976hydrodynamic, yokosawa1980relativistic}), even after a time $t\gg \delta t$.



{\bf\textit{Observable neutrino signal.}}---Fireball propagation reveals that the time structure of the neutrino signal is defined by the source, not the subsequent fluid dynamics. What about the energy spectrum that in our schematic model would be blackbody with the source temperature $T$ (black line in Fig.~\ref{fig:neutrino_spectrum})? Within the fireball, neutrinos possess a boosted blackbody spectrum. However, at low enough density at some radius $\rd$, the $\nu$SI decouple and then neutrinos stream freely. The large Lorentz factor of Eq.~\eqref{eq:gamma_factor} reveals that we observe neutrinos roughly with the same angular spread for both free streaming or fluid propagation, so time-of-flight effects are minimal. We thus picture the observable flux at a distance $\dSN\gg\rd$ as being steadily emitted by a spherical shell of radius $\rd$, taken to be 
the same for all energies.

Thus at a large distance one observes the superposition of the boosted blackbody spectra from each point on that sphere (right panel in Fig.~\ref{fig:sketch}). While the comoving $\Td$ and Lorentz factor $\gd$ are the same for all points, they are seen under different angles, producing different spectra due to Doppler boosting. The limb of the sphere looks much colder (effective temperature of order $\Td/\gd$) than the center (effective temperature of order $2 \Td \gd$). Explicitly, the superposed number flux spectrum for a single species is found to be~\cite{Diamond:2023scc}
\begin{equation}
\frac{d \Phi}{d\epsilon}=\frac{\rd^2}{4\pi^2\dSN^2} \frac{\tilde{T}\epsilon}{\gd^2}\log\left[1+e^{\eta-\epsilon/2\tilde{T}}\right],
\end{equation}
where $\tilde{T}=\gamma^d T_\nu^d=\gamma T_\nu$ if we recall the constancy of $\gamma T_\nu$ during fireball expansion. When $\nu$SI number-changing reactions are in equilibrium near the PNS, $\tilde{T}=0.828\,T$ and $\eta=0$. For the opposite case of number-conserving dynamics, $\tilde{T}=0.903\,T$ and $\eta=-0.363$. 

Finally, using the Lorentz factor of Eq.~\eqref{eq:gamma_factor}, the observer spectrum is 
\begin{equation}\label{eq:fluidspectrum}
    \frac{d\Phi}{d\epsilon}=\frac{\rs^2}{6\sqrt{3}\pi^2 \dSN^2}\tilde{T}\epsilon\log\left[1+e^{\eta-\epsilon/2\tilde{T}}\right].
\end{equation}
Nothing depends on $\rd$, justifying our earlier assumption of taking decoupling to be instantaneous. This is to be compared with the blackbody radiation without $\nu$SI
\begin{equation}\label{eq:bbspectrum}
    \left(\frac{d\Phi}{d\epsilon}\right)_{\mathrm{bb}}=\frac{\rs^2}{8\pi^2 \dSN^2}\frac{\epsilon^2}{e^{\epsilon/T}+1}.
\end{equation}
One can check that the integrated energy flux from Eq.~\eqref{eq:fluidspectrum} is indeed 0.96 the one from 
Eq.~\eqref{eq:bbspectrum}, thus maintaining exactly the energy outflow stated earlier that we had derived in our companion paper~\cite{Fiorillo:2023-Theory}.


The different spectra are shown in Fig.~\ref{fig:neutrino_spectrum}. Compared with the usual blackbody spectrum, $\nu$SI increase $\langle\epsilon\rangle$ by
10\% (19\%) if $\nu$SI do not conserve (do conserve) the neutrino number, and increase
$\langle\epsilon^2\rangle$ by 37\% (60\%). The impact of number-changing processes is very limited; particle number does not change in the fireball expansion, and only slightly changes at the emission from the neutrinosphere. In this sense, our results differ markedly from those of Ref.~\cite{Shalgar:2019rqe}, where only the $\nu\bar{\nu}\to\nu\bar{\nu}\nu\bar{\nu}$ reactions had been considered without accounting for the inverse reactions. Therefore, the bounds proposed in Ref.~\cite{Shalgar:2019rqe} are lifted. In the region of interest, for mediator masses above MeV, only bounds based on laboratory and high-energy astrophysical neutrinos remain valid (see references in Ref.~\cite{Berryman:2022hds}). Our fluid description obviates the need to consider a detailed kinetic discussion of this question.

% Figure environment removed

One observable we have not discussed is flavor. In the standard case, this issue is not fully understood, given the uncertainties on neutrino flavor conversion. On the other hand, assuming that all flavors are affected by $\nu$SI without overly hierarchical couplings, the spectra will be equalized, simplifying the situation somewhat.

{\bf\textit{Discussion.}}---We have studied the emission, propagation, and observer spectrum of SN neutrinos, assuming they behave as a relativistic fluid caused by large $\nu$SI, drawing on the theoretical foundations elaborated in our companion paper \cite{Fiorillo:2023-Theory}. We have used the simplest possible toy model of an isothermal and homogeneous source, but our results are generic up to factors of order unity. 

Diffusion transport deep inside the PNS as well as nonequilibrium transport in the surface layers are somewhat modified by a different spectral MFP average and by all flavors becoming one common fluid.

The energy spectrum seen by the medium in the gain region below the shock wave before explosion is somewhat harder than a blackbody spectrum, all else being equal, but it is not obvious in which exact direction the neutrino energies and fluxes would change in a selfconsistent SN treatment. At present, we cannot estimate the realistic quantitative impact of different tens-of-percent effects. Modifications of this magnitude for sure play an important role, other examples being the inclusion of muons \cite{Bollig:2017lki}, strange-quark contributions to the neutrino opacities \cite{Melson:2015spa}, or fast flavor conversion (see Refs.~\cite{Ehring:2023lcd, Ehring:2023abs} and references), but on present evidence it is not easily possible to rule such effects in or out.

The burst time profile is set by emission at the source, not by modified fluid propagation. Since the fluid moves essentially with the speed of light, and the energy outflow from the PNS is very similar to standard emission, the burst duration is mostly unchanged, in analogy to the physics of fireballs. 
In addition, however, we have provided the boundary condition for the fluid expansion at the source. So while our model is schematic, it solves the entire problem from neutrino creation inside the PNS all the way to the observer signal at Earth.

The observable flux spectrum is broadened, slightly favoring higher energies compared with naive blackbody emission, but on this level of detail, we are not sure of how exactly the source spectrum is selfconsistently modified. Therefore, this is yet another modification too small to provide a smoking-gun signature.

During steady emission, the velocity profile outside the PNS coincides with the earlier steady wind solution \cite{Chang:2022aas}. The main difference is that we allow for energy exchange with the PNS acting as a heat reservoir for the neutrinos, whereas Ref.~\cite{Chang:2022aas} considered a PNS temperature profile such that no energy is exchanged with the neutrino fluid, a scenario that we also analyze in our companion paper~\cite{Fiorillo:2023-Theory}. The dynamical evolution caused by switching the source first on and later off leads to steady emission during the long PNS cooling phase (long compared with the PNS size) and later to fireball propagation. Matching the diffusive dynamics with the expanding fireball allows for specific predictions of the spectrum changes in the hydrodynamical regime.

We have not examined one potentially dramatic effect of $\nu$SI that arises when they violate lepton number as in the traditional majoron models. Usually a large amount of electron lepton number is trapped in a SN core, around 0.30 per baryon, providing a large electron chemical potential, and causing a relatively cold SN core after collapse. The lepton number is usually lost by diffusion and convection over a few seconds. On the other hand, the quick deleptonization of the SN core, partly happening already during infall, via reactions $\nu\nu\to\bar{\nu}\bar{\nu}$, would strongly modify the overall SN paradigm. However, in contrast to what is sometimes stated in the literature, even this large effect is not necessarily excluded because the hydrodynamic shock wave could arise from a thermal bounce \cite{Fuller:1988ega, Rampp:2002kn}. These are intriguing questions that need addressing in selfconsistent SN simulations.

Treating neutrinos with large $\nu$SI as a relativistic fluid as pioneered by Dicus et al.\ \cite{Dicus:1988jh} has vastly simplified the discussion both conceptually and analytically. The \hbox{uncanny} smallness of the modifications caused by the fluid nature is the main surprise of this investigation and mandates a selfconsistent study to understand the exact quantitative effects in SN physics.

{\bf\textit{Acknowledgments.}}---We thank Shashank Shalgar, Irene Tamborra, Mauricio Bustamante, Po-Wen Chang, Ivan Esteban, John Beacom, Todd Thompson, Christopher Hirata, and Thomas Janka for informative discussions and/or comments on the manuscript.
DFGF is supported by the Villum Fonden under Project No.\ 29388 and the European Union's Horizon 2020 Research and Innovation Program under the Marie Sk{\l}odowska-Curie Grant Agreement No.\ 847523 ``INTERACTIONS.'' GGR acknowledges partial support by the German Research Foundation (DFG) through the Collaborative Research Centre ``Neutrinos and Dark Matter in Astro- and Particle Physics (NDM),'' Grant SFB-1258-283604770, and under Germany’s Excellence Strategy through the Cluster of Excellence ORIGINS EXC-2094-390783311. EV acknowledges support by the European Research Council (ERC) under the European Union’s Horizon Europe Research and Innovation Program (Grant No.\ 101040019).


\bibliographystyle{bibi}
\bibliography{References}


\end{document}

