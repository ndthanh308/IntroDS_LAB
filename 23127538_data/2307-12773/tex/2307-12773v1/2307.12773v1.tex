%\documentclass[EJP]{ejpecp}
\documentclass[11pt]{article}
\usepackage{amsmath}
\usepackage{amssymb}
\usepackage{bbm}
%\usepackage[mathscr]{euscript}
%\usepackage{mathrsfs}
\usepackage[margin=2cm]{geometry}
\usepackage{hyperref}
\usepackage{tikz}
\usetikzlibrary{decorations.pathmorphing,arrows}
\usepackage{braket}
\usepackage{setspace}

\newtheorem{thm}{Theorem}[section]
\newtheorem{cor}[thm]{Corollary}
\newtheorem{prop}[thm]{Proposition}
\newtheorem{lem}[thm]{Lemma}
\newtheorem{lemma}[thm]{Lemma}
\newtheorem{Remark}[thm]{Remark}
\newtheorem{theorem}[thm]{Theorem}
\newtheorem{remark}[thm]{Remark}
\newtheorem{proposition}[thm]{Proposition}
\newtheorem{definition}[thm]{Definition}
\newtheorem{corollary}[thm]{Corollary}
\newtheorem{claim}[thm]{Claim}
\newtheorem{conj}[thm]{Conjecture}
\renewcommand{\thesection} {\arabic{section}}
\newenvironment{proof}{{\bf Proof:}}{\hfill$\square$\vskip.5cm}
\newenvironment{proofof}{}{\hfill$\square$\vskip.5cm}
\newcommand{\R}{\mathbb{R}}
\newcommand{\N}{\mathbb{N}}
\newcommand{\C}{\mathbb{C}}
\newcommand{\E}{\mathbf{E}}
\newcommand{\Z}{\mathbb{Z}

}
\newcommand{\D}{\mathbb{D}}
\renewcommand{\Pr}{\mathbf{P}}

\newcommand{\s}{\sigma}
\newcommand{\calF}{\mathcal{F}}
\newcommand{\sfH}{\mathsf{H}}
\newcommand{\sfi}{\mathsf{i}}
\newcommand{\sfj}{\mathsf{j}}
\newcommand{\sfn}{\mathsf{n}}
\newcommand{\sfJ}{\mathsf{J}}
\newcommand{\sfK}{\mathsf{K}}
\newcommand{\sfN}{\mathsf{N}}
\newcommand{\sfM}{\mathsf{M}}
\newcommand{\sfp}{\mathsf{p}}
\newcommand{\sfg}{\mathsf{g}}
\newcommand{\sfX}{\mathsf{X}}
\newcommand{\sfV}{\mathsf{V}}
%\newcommand{\sfZ}{\mathsf{Z}}
\newcommand{\sfw}{\mathsf{w}}
\newcommand{\sfd}{\mathsf{d}}
\newcommand{\sfD}{\mathsf{D}}
\newcommand{\Var}{\operatorname{Var}}
\renewcommand{\Re}{\operatorname{Re}}
\renewcommand{\Im}{\operatorname{Im}}
\newcommand{\x}{\boldsymbol{x}}
\newcommand{\y}{\boldsymbol{y}}
\newcommand{\br}{{\bf r}}
\newcommand{\X}{{\bf X}}
\newcommand{\Y}{{\bf Y}}
\renewcommand{\Z}{{\mathbb{Z}}}

\newcommand{\CP}{\mathbb{CP}}
\newcommand{\Hil}{\mathcal{H}}


\begin{document} 

\date{22 July 2023}

%\SUBMITTED{May 6, 2023}
%\ACCEPTED{00}
%\KEYWORDS{Random permutations}
%\AMSSUBJ{	82B05, 82B10, 60B15}

\title{Violation of Ferromagnetic Ordering of Energy Levels in Spin Rings
by Weak Paramagnetism of the Singlet}
\author{
%Mrigank${}$ \\
\large David Heson${}^{1}$, Shannon Starr${}^{2}$ and Jacob Thornton${}^{3}$\\
\textcolor{white}{.}\\
\small ${}^{1}$ Mississippi State University\\[-2pt]
\small Department of Physics and Astronomy\\[-2pt]
\small 355 Lee Boulevard\\[-2pt]
\small Mississippi State, MS 39762\\
\textcolor{white}{.}\\
\small ${}^{2}$ University of Alabama at Birmingham\\[-2pt]
\small Department of Mathematics\\[-2pt]
%\small University Hall, Room 4005\\[-2pt]
\small  1402 Tenth Avenue South\\[-2pt]
\small  Birmingham, AL 35294-1241\\[-2pt]
\small  \href{mailto:slstarr@uab.edu}{slstarr@uab.edu}\\
\textcolor{white}{.}\\
\small ${}^{3}$ Auburn University\\[-2pt]
\small Department of Chemical Engineering\\[-2pt]
\small Samford Hall\\[-2pt]
\small 182 S College Street\\[-2pt]
\small Auburn University, AL 36849
}


\maketitle


\abstract{For the quantum Heisenberg antiferromagnet with spin-$j$ on a bipartite, balanced graph, the
 Lieb-Mattis theorem, ``Ordering of energy levels,'' guarantees that the ground state is a spin singlet,
and moreover, defining $E^{\textrm{AF}}_{\min}(S)$ to be the minimum  eigenvalue of the Hamiltonian in the 
invariant subspace consisting of all spin $S$ vectors, $\boldsymbol{S}_{\mathrm{tot}}^2 \psi = S(S+1)\psi$,
the function $E^{\textrm{AF}}_{\min}(S)$
is monotonically increasing for $0\leq S\leq j|\mathcal{V}|$.
For the ferromagnet, the absolute ground state is $E_{\min}^{\textrm{FM}}(j|\mathcal{V}|)$.
We say that the graph satisfies ``ferromagnetic ordering of energy levels'' at order $n$, or FOEL-$n$, if  
two properties hold:
(1) $E_{\min}^{\textrm{FM}}(j|\mathcal{V}|)\leq \dots \leq E_{\min}^{\mathrm{FM}}(j|\mathcal{V}|-n)$,
and (2) $E_{\min}^{\mathrm{FM}}(j|\mathcal{V}|-n)\leq E_{\min}^{\mathrm{FM}}(j|\mathcal{V}|-m)$ for all $m\geq n$.
Caputo, Liggett and Richthammer proved a theorem which generally implies  FOEL-$1$ is true.
Apparently $E_0^{\mathrm{FM}}(0) <E_0^{\mathrm{FM}}(1)$ for sufficiently long spin rings, $\mathbb{Z}/L\mathbb{Z}$ with even length $L$.
So FOEL-$n$ does not hold for $n=jL-1$.
We consider $E_0^{\mathrm{FM}}(1)-E_0^{\mathrm{FM}}(0)$ using 
linear spin-wave analysis and numerical computation.
Using the Bethe ansatz, Sutherland already considered the spin ring with $j=1/2$ and notably proved weak paramagnetism.
But we also present evidence for $j>1/2$.
}



\section{Introduction}

We consider the quantum Heisenberg ferromagnet on even length $L$ spin rings.
The Hamiltonian  is
\begin{equation*}
	H_{j,L}^{\mathrm{FM}}\, =\, \sum_{k=1}^{L} h_{k,k+1}^{(j)}\,
	=\, \sum_{k=1}^{L} (-S^{(1)}_k S^{(1)}_{k+1} - S^{(2)}_k S^{(2)}_{k+1} - S^{(3)}_k S^{(3)}_{k+1} + j^2\, \mathbbm{1})\, ,
\end{equation*}
where we interpret $S^{(1,2,3)}_{L+1}$ to be $S^{(1,2,3)}_1$.
Let $\mathbbm{1}$ denote the identity on the Hilbert space.
For $j=1/2$ the matrices are the usual variations of the Pauli spin-$1/2$ matrices:
\begin{equation*}
	S^{(1)}\, =\, \begin{bmatrix} 0 & 1/2 \\ 1/2 & 0 \end{bmatrix}\, ,\qquad
	S^{(2)}\, =\, \begin{bmatrix} 0 & -i/2 \\ i/2 & 0 \end{bmatrix}\ \text{ and }\ 
	S^{(3)}\, =\, \begin{bmatrix} 1/2 & 0 \\ 0 & -1/2 \end{bmatrix}\, .
\end{equation*}
For $j>1/2$ the matrices are the spin-$j$ analogue of the SU(2) spin matrices on the irreducible representation on $\C^{2j+1}$.
The Hilbert space is 
\begin{equation*}
	\Hil_{[1,L]}^{(j)}\, =\, \Hil_1^{(j)} \otimes \Hil_2^{(j)} \otimes \cdots \otimes \Hil_L^{(j)}\, ,
\end{equation*}
where each $\Hil_x^{(j)} \cong \C^{2j+1}$, and we also have
\begin{equation*}
	S_x^{(1,2,3)}\, =\, \mathbbm{1}_{\C^{2j+1}} \otimes \cdots \otimes \mathbbm{1}_{\C^{2j+1}} 
\otimes S^{(1,2,3)} \otimes \mathbbm{1}_{\C^{2j+1}}  \otimes \cdots \otimes \mathbbm{1}_{\C^{2j+1}}\, ,
\end{equation*}
where all factors are identities on $\C^{2j+1}$, namely $\mathbbm{1}_{\C^{2j+1}}$, except for the factor at position $x$ which is $S^{(1,2,3)}$.
Let us recall that defining the spin-raising and lowering operators
\begin{equation*}
	S^{\pm}\, =\, S^{(1)} \pm i S^{(2)}\, ,
\end{equation*}
we may choose an orthonormal basis of $\C^{2j+1}$, called $\Psi_{j,m}$ for $m = -j,-j+1,\dots,j$ such that
\begin{equation*}
	S^{(3)} \Psi_{j,m}\, =\, m \Psi_{j,m}\ \text{ and }\ 
	S^{\pm} \Psi_{j,m}\, =\, \sqrt{j(j+1)-m(m\pm 1)}\, \Psi_{j,m\pm 1}\, ,
\end{equation*}
where we interpret $0 \Psi_{j,j+1}$ as $0$ and $0 \Psi_{j,-j-1}$ as $0$.
Then, since we get
\begin{equation*}
	S^+ S^- \Psi_{j,m} + S^- S^+ \Psi_{j,m}\, =\, 2\big(j(j+1)-m^2) \Psi_{j,m}\, ,
\end{equation*}
and since the Casimir operator on $\C^{2j+1}$ is 
\begin{equation*}
	\mathcal{C}\, =\, (S^{(1)})^2 + (S^{(2)})^2 + (S^{(3)})^2\, =\, (S^{(3)})^2 + \frac{1}{2}\, S^{+} S^{-} + \frac{1}{2}\, S^{-} S^{+}\, ,
\end{equation*}
we see that $\mathcal{C} \Psi_{j,m} = j(j+1) \Psi_{j,m}$ for every $m=-j,-j+1,\dots,j$.
Based on the SU(2) commutation relations for the Lie bracket commutator $[A,B] = AB - BA$,
\begin{equation*}
	[S^{(1)},S^{(2)}]\, =\, i S^{(3)}\, ,\
	[S^{(2)},S^{(3)}]\, =\, i S^{(1)}\, ,\
	[S^{(3)},S^{(1)}]\, =\, i S^{(2)}\, ,\
\end{equation*}
it is easy to see that $\mathcal{C}$ commutes with $S^{(1)}$, $S^{(2)}$ and $S^{(3)}$.
The total spin operators are
\begin{equation*}
	S^{(1,2,3)}_{\mathrm{tot}}\, =\, \sum_{x=1}^{L} S_x^{(1,2,3)}\ \text{ and }\ 
	\mathcal{C}_{\mathrm{tot}}\, =\, (S^{(1)}_{\mathrm{tot}})^2+(S^{(2)}_{\mathrm{tot}})^2+ (S^{(3)}_{\mathrm{tot}})^2\, .
\end{equation*}
Each of the total spin operators commutes with each term $h_{k,k+1}^{(j)}$.
So the whole Hamiltonian commutes with $\mathcal{C}_{\mathrm{tot}}$ in particular.
So we may define, for each choice of $S$ between $0$ and $jL$, the energy eigenvalue
\begin{equation*}
	E_{\min}^{\mathrm{FM}}(S)\, =\, \min\left(\left\{\frac{\langle \psi\, ,\ H_{j,L}^{FM} \psi \rangle}{\|\psi\|^2}\, :\, 
	\psi \in \Hil_{[1,L]}^{(j)}\, ,\ \|\psi\|\neq 0\, ,\ \mathcal{C}_{\mathrm{tot}} \psi = S(S+1) \psi\right\}\right)\, .
\end{equation*}
When we wish to emphasize the dependence on $j$ and $L$ we will write
$$
E_{\min}^{\mathrm{FM}}(S;j,L)\ \text{ in place of }\ E_{\min}^{\mathrm{FM}}(S)\, .
$$
Because we restrict attention to even length spin rings, the spins $S$ that can be attained by vectors satisfying
\begin{equation}
	\mathcal{C}_{\mathrm{tot}} \psi\, =\, \Big((S^{(1)}_{\mathrm{tot}})^2+(S^{(2)}_{\mathrm{tot}})^2+ (S^{(3)}_{\mathrm{tot}})^2\Big)\psi\,
	=\, S(S+1)\psi\, ,
\end{equation}
are $S=0,1,2,\dots,jL$.

In \cite{Sutherland}, Sutherland analyzed $E^{\mathrm{FM}}_{\min}(S;j,L)$ using the Bethe ansatz.
From his analysis, the following holds:
\begin{proposition}
For $j=1/2$, there exists a constant $c>0$ such that 
$$
\lim_{\epsilon \to 0^+} \lim_{L \to \infty} \inf_{1<S<\epsilon L/2} \frac{E^{\mathrm{FM}}_{\min}(S;1/2,L) - E^{\mathrm{FM}}_{\min}(0;1/2,L)}{4S^2/L^2}\, =\, c\, .
$$
\end{proposition}
% Figure environment removed
More precisely, he determines the asymptotic dispersion relation for $E_{\min}^{\mathrm{FM}}(S)$. With a certain parametrization he finds
a formula involving the complete elliptic integrals of first and second kind
\begin{equation}
\begin{split}
	d\, &=\, \frac{1}{2} + \frac{a}{2}\, \left(\frac{E(1/a)}{K(1/a)} - 1\right)\, ,\\
	\varepsilon\, &=\, 4 K\left(\frac{1}{a}\right)\left(2 E\left(\frac{1}{a}\right) - \left(1-\frac{1}{a^2}\right)K\left(\frac{1}{a}\right)\right)\, .
\end{split}
\end{equation}
We have plotted the parametric curve with a certain scaling in Figure \ref{fig:Sutherland}.
Sutherland noted that the spin singlet occurs for $d=1/2$ and for $0<d-\frac{1}{2}\ll 1$ the asymptotic 
expansion is $\varepsilon-\pi^2\sim8\pi^2\left(d-\frac{1}{2}\right)^2$.
That implies a weak paramagnetism of the spin singlet. We will explain this in Section 2, using linear spin wave analysis, and conclude the following:
\begin{lemma}
\label{lem:main}
Linear spin wave analysis gives (modulo the assumptions required to apply LSW)
$$
E^{\mathrm{FM}}_{\min}(1) - E^{\mathrm{FM}}_{\min}(0)\, \sim\, -\frac{8\pi^2}{3L^2}\, \cdot \frac{E^{\mathrm{FM}}_{\min}(0)-j^2L}
{\|\widetilde{\mathcal{O}}^+_1 \Psi^{\mathrm{FM}}_{\min}(0)\|^2}\, ,
$$
where $\Psi^{\mathrm{FM}}_{\min}(0)$ is the normalized spin singlet such that $H^{\mathrm{FM}}_{j,L}\Psi^{\mathrm{FM}}_{\min}(0)
=E^{\mathrm{FM}}_{\min}(0) \Psi^{\mathrm{FM}}_{\min}(0)$, and 
$$
\widetilde{\mathcal{O}}^+_1\, =\, \sum_{x=1}^L e^{\pi i x/L} S_x^+\, .
$$
In particular, the right-hand-side is positive as long as $E^{\mathrm{FM}}_{\min}(0) < j^2L$.
\end{lemma}
That analysis is valid for spins larger than $j=1/2$, as well.
Given the lemma, the following fact due to Sutherland is important.
\begin{proposition}
\label{prop:momentum}
Consider $j=1/2$. Denote the translation operator as $T : \Hil^{(1/2)}_{[1,L]} \to \Hil^{(1/2)}_{[1,L]}$ defined such that
for any vectors $\psi_1,\dots,\psi_L \in \C^2$,
$$
	T (\psi_1\otimes \psi_2 \otimes \cdots \otimes \psi_{L-1} \otimes \psi_L)\, =\, \psi_2 \otimes \psi_3 \otimes \cdots \otimes \psi_L \otimes \psi_1\, .
$$
Then for even $L$ and $n \in \{0,1,\dots,L/2\}$, we have
$$
 \min\left(\left\{\frac{\langle \psi\, ,\ H_{1/2,L}^{FM} \psi \rangle}{\|\psi\|^2}\, :\, 
	\psi \in \Hil_{[1,L]}^{(1/2)}\, ,\ \|\psi\|\neq 0\, ,\ T\psi = \exp\left(\frac{2\pi i n}{L}\right) \psi\right\}\right)\, 
=\, E^{\mathrm{FM}}_{\min}\left(\frac{L}{2}-n;\frac{1}{2},L\right)\, .
$$
\end{proposition}
Then an easy variational calculation shows that the vector
$$
\psi\, =\, \frac{1}{\sqrt{\binom{L}{L/2}}} \sum_{\substack{A \subset \{1,\dots,L\}\\ |A|=L/2}} 
\left(\prod_{a \in A} e^{2\pi i a/L} S_a^-\right)
\big(\ket{j} \otimes \cdots \otimes \ket{j}\big)\, ,
$$
satisfies $T\psi=-\psi$ and $\langle \psi, H^{\mathrm{FM}}_{1/2,L}\psi\rangle =\frac{L^2}{4(L-1)}\sin^2\left(\frac{2\pi}{L}\right)$.
This state is not a spin singlet. 
But by Proposition \ref{prop:momentum} it gives an upper bound on $E^{\mathrm{FM}}_{\min}(0;1/2,L)$.
In particular, since $E^{\mathrm{FM}}_{\min}(0;1/2,L)\preceq \pi^2/L$, that means $E^{\mathrm{FM}}_{\min}(0;1/2,L)-j^2L<0$
for sufficiently large $L$.

\begin{remark}
The theoretical work is contingent. For example, we rely on Proposition \ref{prop:momentum} to determine the sign of the right-hand-side
of the asymptotic formula in Lemma \ref{lem:main}. 
Because of all this, we supplement with numerical work.
We feel that the numerical work is equally important or more so.
\end{remark}

The numerics will be presented in Section 3.

\subsection{Implication for Ferromagnet Ordering of Energy Levels}

The notion of ferromagnetic ordering of energy levels is based on the Lieb-Mattis theorem of, ``Ordering of Energy Levels,'' from \cite{LiebMattisOEL}.
The Hamiltonian for the Heisenberg antiferromagnet is the negative of the Hamiltonian for the Heisenberg ferromagnet, modulo an energy shift:
\begin{equation*}
	H_{j,L}^{\mathrm{AF}}\, 
	=\, \sum_{k=1}^{L} (S^{(1)}_k S^{(1)}_{k+1} + S^{(2)}_k S^{(2)}_{k+1} + S^{(3)}_k S^{(3)}_{k+1})\, .
\end{equation*}
Defining the minimum energies in the total spin spaces as
\begin{equation*}
	E_{\min}^{\mathrm{AF}}(S;j,L)\, =\, \min\left(\left\{\frac{\langle \psi\, ,\ H_{j,L}^{\mathrm{AF}} \psi \rangle}{\|\psi\|^2}\, :\, 
	\psi \in \Hil_{[1,L]}^{(j)}\, ,\ \|\psi\|\neq 0\, ,\ \mathcal{C}_{\mathrm{tot}} \psi = S(S+1) \psi\right\}\right)\, .
\end{equation*}
Lieb and Mattis proved the following
\begin{equation*}
E_{\min}^{\mathrm{AF}}(0;j,L)\, \leq\, E_{\min}^{\mathrm{AF}}(1;j,L)\,
<\, E_{\min}^{\mathrm{AF}}(2;j,L)\,
<\, \dots\, 
<\, E_{\min}^{\mathrm{AF}}(jL;j,L)\, ,
\end{equation*}
and their theorem applies equally well to any bipartite, balanced graph.

For the ferromagnet, the Perron-Frobenius theorem implies $E_{\min}^{\mathrm{FM}}(jL;j,L)$ is less than
$E_{\min}^{\mathrm{FM}}(S;j,L)$ for all $S<jL$.
Caputo, Liggett and Richthammer proved in \cite{CaputoLiggettRichthammer} a result implying
\begin{equation*}
	E_{\min}^{\mathrm{FM}}(jL-1;j,L)\, =\, \min(\{E_{\min}^{\mathrm{FM}}(S;j,L)\, :\, S=0,1,\dots,jL-1\})\, ,
\end{equation*}
and that the analogous result holds for any connected graph.
In \cite{NSSfoel}, the property of FOEL-$n$, or ``ferromagnetic ordering of energy levels at order $n$,''
is defined as 
\begin{equation*}
\begin{split}
	&\hspace{1cm} E_{\min}^{\mathrm{FM}}(jL;j,L)\, \leq\, \dots\, \leq\, E_{\min}^{\mathrm{FM}}(jL-n;j,L)\ \text{ and}\\ 
	&E_{\min}^{\mathrm{FM}}(jL-n;j,L)\, =\, \min(\{E_{\min}^{\mathrm{FM}}(S;j,L)\, :\, S=0,1,\dots,jL-n\})\, .
\end{split}
\end{equation*}
For open chains, and spin $j=1/2$, this was proved in the same article.
The key is the Hulth\'en bracket basis, which was most famously exposed by Temperley and Lieb \cite{TemperleyLieb}.
Then it was proved for open chains for $j>1/2$ in \cite{NachtergaeleStarr}.

But for spin rings, it appears to be false. In \cite{SpitzerStarrTran}, some numerical evidence was provided.
The numerical evidence seems like the most reliable indication of what is true to us.
In this article we extend that numerical evidence in Section 3.
Following Lemma \ref{lem:main} we see that FOEL-$n$ is violated
for spin rings for $n=jL-1$.

\subsection{Review of some motivation from mathematical physics}

One motivation for all of this is the fact that in mathematical physics, the phase transition
for the quantum Heisenberg ferromagnet has not yet been proved, even though for the quantum Heisenberg
antiferromagnet it was proved by Dyson, Lieb and Simon \cite{DLS}.
Their method for the antiferromagnet was reflection positivity.
But that property does not hold for the quantum Heisenberg ferromagnet, as proved by Speer \cite{Speer}.
Later, Correggi, Giuliani and Seiringer did prove that the type of inequalities that one would expect from
linear spin wave analysis is satisfied in a technical manner \cite{CGS}.
But to the best of our knowledge, the phase transition for the quantum Heisenberg ferromagnet has still not been proved.
If one changes the model to add any amount of Ising-like anisotropy it was proved by Tom Kennedy \cite{Kennedy},
but that model has $\mathrm{U}(1)$ symmetry instead of $\mathrm{SU}(2)$.

\section{Perturbative spin wave calculations}

For the mathematics of this section, we use the type of operators considered by Lieb, Schultz and Mattis
\cite{LiebSchultzMattis}
to show their dichotomy for low energy excitations in antiferromagnetic systems depending on $j$, a pre-cursor
to Haldane's conjecture \cite{Haldane}.
In case any reader is a non-expert, we highly recommend the introductory textbook by Auerbach.
It is a particularly helpful
resource for the notation \cite{Auerbach}.

\subsection{Brief review of the Lieb, Schultz, Mattis theorem}
In Auerbach, Section 5.2, the notation for the standard spin-wave perturbation operator is 
$$
	\mathcal{O}\, =\, \exp\left(\frac{2\pi i}{L}\, \sum_{k=1}^{L} k S_k^{(3)}\right)\, ,\ \text{ so that }\
	\mathcal{O}^{\ell}\, =\, \exp\left(\frac{2\pi i}{L}\, \sum_{k=1}^{L} k \ell S_k^{(3)}\right)\, .
$$
All those operators are unitary.
Then he shows an argument based on symmetry that for any spin singlet state, which is a translation eigenstate, one has
\begin{equation*}
	\langle \mathcal{O}^1 \Psi\, ,\ \Psi \rangle\, =\, 0\, .
\end{equation*}
Moreover, for any spin singlet,
\begin{equation}
\label{ineq:AuerbachLSM}
	\langle \mathcal{O}^1 \Psi\, ,\ H^{\mathrm{AF}}_{j,L} \mathcal{O}^1\Psi \rangle\ 
- 	\langle \Psi\, ,\ H^{\mathrm{AF}}_{j,L}\Psi \rangle\,  
	=\, -2 \sin^2\left(\frac{\pi}{L}\right) \sum_{k=1}^{L} \langle \Psi\, ,\ (S_k^{(1)}S_{k+1}^{(1)} + S_k^{(2)} S_{k+1}^{(2)})\Psi \rangle\, .
\end{equation}
Auerbach reviews the Lieb, Schultz, Mattis theorem that says, since it is provable that the ground state of the antiferromagnet is a spin singlet,
it must be the case that there are gapless excitations.
Note that the right hand side of (\ref{ineq:AuerbachLSM}) equals $\frac{4}{3} \sin^2(\pi/L) \langle \Psi\, ,\ H^{\mathrm{AF}}_{j,L}\Psi \rangle$ by $\mathrm{SU}(2)$ symmetry.
But also, by general bounds, the ground state energy is $O(L)$; whereas, by Taylor expansion we know $\sin^2(\pi/L) = O(1/L^2)$, both as $L \to \infty$.
So the spectral gap decays at least like $O(1/L)$ as $L \to \infty$.

A more general result is the dichotomy for half-odd integer spin chains: either $\mathrm{SU}(2)$ symmetry is broken in the ground state, or else there are gapless excitations.


\subsection{The linear spin wave raising operator and the first commutator}
Our perturbation operators are linear spin wave raising (creation) operators $\widetilde{\mathcal{O}}^+_k$, for $k=1,\dots,L-1$, where
\begin{equation*}
	\widetilde{\mathcal{O}}^+_k\, =\, \sum_{r=1}^{L} e^{2\pi i kr/L} S_r^{+}\, .
\end{equation*}
Let us define
\begin{equation*}
	H_L^{(Z)}\, =\,  -\sum_{r=1}^{L} S_r^{(3)} S_{r+1}^{(3)}\qquad \text{ and }\qquad
\label{eq:HamiltonianFormulaXY}
	H_{L}^{(XY)}\, =\, -\sum_{r=1}^{L} \Big(S_r^{(1)} S_{r+1}^{(1)} + S_r^{(2)} S_{r+1}^{(2)}\Big)\, ,
\end{equation*}
so that 
$$
H^{\mathrm{FM}}_{j,L}\, =\, H_L^{(Z)} + H_L^{(XY)} + j^2L \cdot\mathbbm{1}\, .
$$
Then, the commutators may be calculated
\begin{equation}
\label{eq:TBAapp1a}
	[\widetilde{\mathcal{O}}_k^+,H_L^{(Z)}]\,
	=\, \sum_{r=1}^{L}   \big( e^{2\pi i kr/L}  S_r^+ S_{r+1}^{(3)} 
	+ e^{2\pi i (k+1)r/L} S_{r}^{(3)} S_{r+1}^+\big)\, ,
\end{equation}
and
\begin{equation}
\label{eq:TBAapp1b}
  [\widetilde{\mathcal{O}}_k^+,H_L^{(XY)}]\,
	=\, -\sum_{r=1}^{L} \big(e^{2\pi i kr/L} S_r^{(3)} S_{r+1}^{+} +
	e^{2\pi i k(r+1)/L} S_r^{+} S_{r+1}^{(3)}\big)\, .
\end{equation}
We will prove these formula in Appendix \ref{app:FirstCommut}.
For now, we note that they imply
\begin{equation}
\label{eq:FirstCommFin}
	[\widetilde{\mathcal{O}}_k^+,H_{j,L}^{\mathrm{FM}}]\, 
	=\,	[\widetilde{\mathcal{O}}_k^+,H_L^{(XY)}+H_L^{(Z)}]\, 
	=\, \sum_{r=1}^{L}   \big( e^{2\pi i kr/L}-e^{2\pi i k(r+1)/L}\big)  \Big(S_r^{+} S_{r+1}^{(3)}- S_r^{(3)} S_{r+1}^{+}\Big)\, ,
\end{equation}
where we may neglect the shift of the Hamiltonian by $j^2 L \cdot \mathbbm{1}$, because that term contributes $0$ to the commutator
(because the identity operator commutes with everything).

\subsection{Double commutator in lowest order perturbation theory}

Let us define the spin-wave lowering operator as
\begin{equation*}
	\widetilde{\mathcal{O}}^-_k\, =\, \sum_{r=1}^{L} e^{2\pi i kr/L} S_r^{-}\, .
\end{equation*}
Thus $(\widetilde{\mathcal{O}}^+_k)^* = \widetilde{\mathcal{O}}^-_{-k}$.

If we assume that $\Psi$ is an eigenstate of $H_{j,L}^{\mathrm{FM}}$, with eigenvalue $E$ such that
$$
H_{j,L}^{\mathrm{FM}} \Psi\, =\, E \Psi\, ,
$$
then a calculation shows
\begin{equation}
\label{eq:penultimatePerturbation}
\langle \Psi\, ,\  \widetilde{\mathcal{O}}_{-k}^- (H_{j,L}^{\mathrm{FM}}-E) \widetilde{\mathcal{O}}_k^+ \Psi\rangle
+ \langle \Psi\, ,\   \widetilde{\mathcal{O}}_k^+ (H_{j,L}^{\mathrm{FM}}-E)\widetilde{\mathcal{O}}_{-k}^- \Psi\rangle\, 
=\, \langle \Psi\, ,\ [\widetilde{\mathcal{O}}_{-k}^-,  [H_{j,L}^{\mathrm{FM}},\widetilde{\mathcal{O}}_k^+]] \Psi\rangle\, .
\end{equation}
We will show this in Appendix \ref{app:DoubleCommut}.
We note that for $\Psi^{\mathrm{FM}}_{\min}(0;j,L)$ it must either be an eigenvector of $T$ or else
the energy level $E^{\mathrm{FM}}_{\min}(0;j,L)$ is degenerate.
Let us define the spatial reflection operator as $R : \Hil^{(1/2)}_{[1,L]} \to \Hil^{(1/2)}_{[1,L]}$ defined such that
for any vectors $\psi_1,\dots,\psi_L \in \C^2$,
$$
	R (\psi_1\otimes \psi_2 \otimes \cdots \otimes \psi_{L-1} \otimes \psi_L)\, =\, \psi_L \otimes \psi_{L-1} \otimes \cdots \otimes \psi_2 \otimes \psi_1\, .
$$
Also, define the global spin flip operator $F :\Hil^{(1/2)}_{[1,L]} \to \Hil^{(1/2)}_{[1,L]}$, as 
$$
	F\, =\, \exp(i \pi S^{(3)}_{\mathrm{tot}})\, .
$$
These two unitary, involutive operators are symmetries of the Hamiltonian, itself.
Therefore, the same dichotomy exists as before.
Either $\Psi^{\mathrm{FM}}_{\min}(0;j,L)$ is a simultaneous eigenvector of $R$ and $F$, or else 
$E^{\mathrm{FM}}_{\min}(0;j,L)$ is at least doubly degenerate.

\begin{conj}
\label{conj:nondegen}
The energy level $E^{\mathrm{FM}}_{\min}(0;j,L)$ is non-degenerate (among spin singlets).
Hence 
$\Psi^{\mathrm{FM}}_{\min}(0;j,L)$ is a simultaneous eigenvector of $H_{j,L}^{\mathrm{FM}}$,  $\mathcal{C}_{\mathrm{tot}}$, $T$, $R$ and $F$,
with eigenvalues $E^{\mathrm{FM}}_{\min}(0;j,L)$, $0$, $\sigma$, $\tau$ and $u$, for numbers $\sigma,\tau,u \in \{1,-1\}$.
\end{conj}

In every case we have investigated numerically, this conjecture is verified. We discuss this more in Section 3.
Finally, an easy calculation shows that ($R$ and $F$ commute with one another and)
$$
	RF S^+_{k} RF\, =\, S^-_{-k}\, .
$$
Hence, under the assumption of the conjecture, 
\begin{equation}
\begin{split}
&\hspace{-1cm}\langle \Psi^{\mathrm{FM}}_{\min}(0;j,L)\, ,\  
\widetilde{\mathcal{O}}_{-k}^- (H_{j,L}^{\mathrm{FM}}-E^{\mathrm{FM}}_{\min}(0;j,L)) \widetilde{\mathcal{O}}_k^+ \Psi^{\mathrm{FM}}_{\min}(0;j,L)\rangle\\
&\hspace{1cm}=\, -\frac{1}{2}\, \langle \Psi^{\mathrm{FM}}_{\min}(0;j,L)\, ,\ 
[\widetilde{\mathcal{O}}_{-k}^-,  [\widetilde{\mathcal{O}}_k^+,H_{j,L}^{\mathrm{FM}}]] \Psi^{\mathrm{FM}}_{\min}(0;j,L)\rangle\, .
\end{split}
\end{equation}
 
\subsection{Calculation of the double commutator}

Let us denote the function for $k,r \in \Z$,
\begin{equation*}
	\varphi_k(r)\, =\, e^{2\pi i k r/L}\, ,
\end{equation*}
and let us denote 
$$
\omega_k\, =\, \varphi_k(1)\, =\, e^{2\pi i k/L}\, .
$$
Then we may rewrite (\ref{eq:FirstCommFin})
as
\begin{equation*}
	[\widetilde{\mathcal{O}}_k^+,H_{j,L}^{\mathrm{FM}}]\, 
	=\, (1-\omega_k) \sum_{r=1}^{L} \varphi_k(r) \Big(S_r^{+} S_{r+1}^{(3)}- S_r^{(3)} S_{r+1}^{+}\Big)\, .
\end{equation*}
Then we may write the double commutator as 
\begin{equation}
\label{eq:DoubleCommToDo1}
\begin{split}
	[\widetilde{\mathcal{O}}^-_{-k},[\widetilde{\mathcal{O}}_k^+,H_{j,L}^{\mathrm{FM}}]\, 
	&=\, (1-\omega_k) \sum_{r=1}^{L} \varphi_k(r) \left[\widetilde{\mathcal{O}}^-_{-k},\Big(S_r^{+} S_{r+1}^{(3)}- S_r^{(3)} S_{r+1}^{+}\Big)\right]\\
	&=\, (1-\omega_k) \sum_{r=1}^{L} \sum_{r'=1}^{L} \varphi_k(r) \varphi_{-k}(r')  \left[S_{r'}^{-},\Big(S_r^{+} S_{r+1}^{(3)}- S_r^{(3)} S_{r+1}^{+}\Big)\right]\, .
\end{split}
\end{equation}
Then by calculations that we will show in Appendix \ref{app:SecondCommut},
\begin{equation}
\label{eq:DoubleCommFinal}
	[\widetilde{\mathcal{O}}^-_{-k},[\widetilde{\mathcal{O}}_k^+,H_{j,L}^{\mathrm{FM}}]\, 
	=\,  -\sum_{r=1}^{L} \Big(2(1-\omega_k)(1-\omega_{-k}) S_r^{(3)} S_{r+1}^{(3)} 
+(1-\omega_k) S_r^- S_{r+1}^+
+(1-\omega_{-k})S_r^+ S_{r+1}^{-}\Big)\, .
\end{equation}
If we use Conjecture \ref{conj:nondegen}, this implies (by a simple calculation we will show in Appendix \ref{app:SecondCommut})
\begin{equation}
\label{eq:DoubleToDo2}
\begin{split}
&\hspace{-1cm}\langle \Psi^{\mathrm{FM}}_{\min}(0;j,L)\, ,\  \widetilde{\mathcal{O}}_{-k}^- (H_{j,L}^{\mathrm{FM}}-E^{\mathrm{FM}}_{\min}(0;j,L)) \widetilde{\mathcal{O}}_k^+ \Psi^{\mathrm{FM}}_{\min}(0;j,L)\rangle\\
&=\, -\frac{16}{3}\, \sin^2\left(\frac{\pi k}{L}\right)\, \left(E_0 - j^2L\right)\, .
\end{split}
\end{equation}
Choosing $k= 1$ or $k=L-1$, we see that this is asymptotically equivalent to $\displaystyle -\frac{8 \pi^2}{3L} \left(\frac{E_0}{L}-j^2\right)$
as $L \to \infty$. This is $O(1/L)$ as $L \to \infty$. But also, it is negative if $E_0 < j^2 L$.
This is a condition which is easily numerically verified.
This will be discussed more in the next section.
For later reference, let us denote the quantity we obtained as 
$$
\Delta_{\mathrm{LSW}}\, =\,  -\frac{16}{3}\, \sin^2\left(\frac{\pi k}{L}\right)\, \left(E_0 - j^2L\right) 
\|\widetilde{\mathcal{O}}_k^+ \Psi^{\mathrm{FM}}_{\min}(0;j,L)\|^{-2}\, .
$$



\section{Numerical Results}

We used Matlab to perform the numerical calculations. The main numerical linear algorithm is Lanczos iteration built-in to Matlab in the ``eigs'' operations.
This is used to numerically diagonalize the matrices: in particular to find the minimum energy eigenvalues of the Hamiltonian in invariant subspaces determined
by symmetries.
For spin-$j$, to restrict to spin $S$ subspaces, we first used the Hulth\'en bracket basis.
We then converted these vectors back into the standard basis, and conjugated the matrices by the restriction to the subspace spanned by these vectors.
For $j=1/2$ and $L=4$ through $20$, we gather the data in a a table in Figure \ref{fig:GSEtable}.


% Figure environment removed


For higher spin, we used qr-null to restrict to the eigenspace of the Casimir operator.
This may introduce an extra source of machine imprecision.
(The Hulth\'en bracket basis is available also for higher spin, as in \cite{NachtergaeleStarr}.
But we just did not use it.)
For the spin $j=1$ spin ring with $L=6$, we have
\begin{equation*}
\begin{tabular}{r|c|c|c|c|c|c|c}
$S$ : &
0 &
1 &
2 &
3 &
4 &
5 &
6\\
\hline
$E^{\mathrm{FM}}_{\min}(S)$ : &
2.915397&
3.061780 &
2.752456 &
2.313096 &
1.785680 &
1.000000 &
0.000000
\end{tabular}
\end{equation*}
We believe for $j=1$ that $E_{\min}^{\mathrm{FM}}(0)<E_{\min}^{\mathrm{FM}}(1)$ for all even spin sites $L\geq 6$.



% Figure environment removed

Let us discuss $j=1/2$ in more detail, where we have the best set of data. We were able to numerically diagonalize the Hamiltonian, using Matlab's
``eigs'' command for $L$ up to 20.
Calculating all eigenvalues is computationally expensive. In Figure \ref{fig:full} we show all eigenvalues for $L=12$ just to demonstrate what the full set looks like.
But we are only interested in the lowest eigenvalues
in this article.
Note that the dimension of the Hilbert space is $2^L$ so 4096 for $L=12$. The trace of $H^{\mathrm{FM}}_{1/2,L}$ is $j^2L$ times the dimension.
So the average value of all the eigenvalues is $j^2L$. For $j=1/2$ and $L=12$ that is 3.

We note that in each figure we only plot the energy eigenvalues against the total spin, without indicating the multiplicity due to the total spin raising and lowering
operators. So the spin-$S$ subspace will have 1 plot point per multiplet, but that point should have weight $2S+1$.
In Section 2 we found a calculation that suggests $E^{\mathrm{FM}}_{\min}(0)<E^{\mathrm{FM}}_{\min}(1)$ as long as $E^{\mathrm{FM}}_{\min}(0)<j^2L$.
So that is saying that the minimum energy spin singlet has lower energy than the average energy of all possible eigenvalues.
(Note that the trace of the Hamiltonian over a particular total spin subspace, meaning all vectors $\psi$ satisfying $\mathcal{C}_{\mathrm{tot}} \psi = S(S+1) \psi$,
does depend on $S$. For example, for $S=jL$, all vectors have energy eigenvalue $0$ because that is the ground state space for $H^{\mathrm{FM}}_{j,L}$.)

% Figure environment removed

% Figure environment removed

In Figure \ref{fig:Hexagon} and Figure \ref{fig:Dodecagon}, we plot the energy versus the total spin while also indicating the translation eigenvalue.
More precisely, we remove a wedge of angle $2\theta$ from $-\theta$ to $\theta$ if the eigenvector is a translation eigenvector with eigenvalue $e^{\pm i \theta}$.
The reason for displaying this is that Sutherland pointed out that for spin $j=1/2$, the Heisenberg ferromagnet has the property: the lowest eigenvalue
in the total spin space $S$ is also the lowest eigenvalue among vectors $\psi$ such that $T\psi = e^{\pm i \theta} \psi$ for $\theta = \pi - \frac{2\pi S}{L}$.
This is clearly visible in Figures \ref{fig:Hexagon} and \ref{fig:Dodecagon}.

Another question is to calculate the norm-square $\|\widetilde{\mathcal{O}}^+_1 \Psi^{\mathrm{FM}}_{\min}(0;j,L)\|^2$.
For $j=1/2$ and $L=4,6,\dots,20$, these values are as follows:
\scriptsize
\begin{equation*}
\begin{tabular}{r|c|c|c|c|c|c|c|c|c}
$L$  &
4 &
6 &
8 &
10 &
12 &
14 &
16 &
18 & 
20\\
\hline
$\|\widetilde{\mathcal{O}}^+_1 \Psi^{\mathrm{FM}}_{\min}(0;\frac{1}{2},L)\|^2$
&
4
&
6.773501
&
10.228781
&
14.356261
&
19.153123
&
24.618190
&
30.750894
&
37.550908
&
45.018035
\end{tabular}
\end{equation*}

\normalsize
We can compare the theoretical results with the numerical results. We are hopeful that the expectation value is close to the actual eigenvalue, although
we did not rigorously prove that.
If we write $E^{\mathrm{FM}}_{\min}(1)$ in a table along with $E^{\mathrm{FM}}_{\min}(0)+\Delta$ where $\Delta$ is the expectation value obtained
from the linear spin wave analysis of Section 2, we obtain
\scriptsize
\begin{equation*}
\begin{tabular}{r|c|c|c|c|c|c|c|c|c}
$L$  &
4 &
6 &
8 &
10 &
12 &
14 &
16 &
18 &
20\\
\hline
$E^{\mathrm{FM}}_{\min}(1;\frac{1}{2},L)$ &
2.0000    &
1.4384    &
0.5784    &
0.9011    &
0.7594    &
0.6565    &
0.5784    &
0.5170    &
0.4675\\
\hline
$E^{\mathrm{FM}}_{\min}(0;\frac{1}{2},L)+\Delta_{\mathrm{LSW}}$ &
1.3333  &  
1.4152  & 
0.6812  &  
0.9396  &  
0.7889  &  
0.6785  &  
0.5948  &  
0.5295  &
0.4772
\end{tabular}
\end{equation*}
\normalsize
But other ways of representing the data are also shown. In Figure \ref{fig:Delta} we plot the approximation $\Delta_{\mathrm{LSW}}$ to $E^{\mathrm{FM}}_{\min}(0)-E^{\mathrm{FM}}_{\min}(0)$ on the same graph as the actual data for $L=6,\dots,20$, the data we have where both numbers are positive.
% Figure environment removed


\section*{Acknowledgments}

The authors gratefully acknowledge the resources provided by the University of Alabama at Birmingham IT-Research Computing group for high performance computing (HPC) support and CPU time on the Cheaha compute cluster.
This work was supported by a grant from the Simons Foundation.

\appendix


\section{The Matlab code Hulthen.m}

The following Matlab script implements the Hulth\'en bracket basis for calculating $E^{\mathrm{FM}}_{\min}(S;1/2,L)$ for each $S$.
We have written it for $L=20$ which is the largest length spin ring we could attain with our computing resources.

\scriptsize

\begin{verbatim}
E11 = sparse([1 0; 0 0]);
E12 = sparse([0 1; 0 0]);
E21 = sparse([0 0; 1 0]);
E22 = sparse([0 0; 0 1]);
h = kron(E11,E22)+kron(E22,E11)-kron(E12,E21)-kron(E21,E12);

L=20;
fileID = fopen('L20sparse.txt','a');

dim=2^L;

Adj = diag(ones(L-1,1),1);
Adj(1,L)=1;

H = sparse(dim,dim);
for x=1:(L-1)
    for y=(x+1):L
        if Adj(x,y)
            H=H+kron(speye(2^(x-1)),kron(E11,kron(speye(2^(y-x-1)),kron(E22,speye(2^(L-y))))));
            H=H+kron(speye(2^(x-1)),kron(E22,kron(speye(2^(y-x-1)),kron(E11,speye(2^(L-y))))));
            H=H-kron(speye(2^(x-1)),kron(E12,kron(speye(2^(y-x-1)),kron(E21,speye(2^(L-y))))));
            H=H-kron(speye(2^(x-1)),kron(E21,kron(speye(2^(y-x-1)),kron(E12,speye(2^(L-y))))));
        end
    end
end

% % Cas=zeros(2^L);
% Cas = sparse(dim,dim);
% for x=1:(L-1)
%     for y=(x+1):L
%         Cas=Cas+kron(speye(2^(x-1)),kron(E11,kron(speye(2^(y-x-1)),kron(E22,speye(2^(L-y))))));
%         Cas=Cas+kron(speye(2^(x-1)),kron(E22,kron(speye(2^(y-x-1)),kron(E11,speye(2^(L-y))))));
%         Cas=Cas-kron(speye(2^(x-1)),kron(E12,kron(speye(2^(y-x-1)),kron(E21,speye(2^(L-y))))));
%         Cas=Cas-kron(speye(2^(x-1)),kron(E21,kron(speye(2^(y-x-1)),kron(E12,speye(2^(L-y))))));
%     end
% end

DnSpinMat=[];
idxLst=[];
NumBracketLst=[];
for idx = 1:dim,
    base2rep = dec2base(idx-1,2);
    LengthBase2 = length(base2rep);
    DnSpinLst=[];
    for kctr=1:LengthBase2,
        if base2rep(kctr)=='1',
            DnSpinLst = [DnSpinLst,L-LengthBase2+kctr];
        end
    end
    if length(DnSpinLst)==L/2,
        idxLst=[idxLst,idx];
        DnSpinMat=[DnSpinMat;DnSpinLst];   
    end
end
%DnSpinMat
%idxLst
vectorLst=[];
for kctr=1:length(DnSpinMat)
    bracketLstLeft=[];
    bracketLstRight=[];
    for j=1:(L/2),
        if DnSpinMat(kctr,j)>j+length(bracketLstLeft),
            bracketLstRight=[bracketLstRight,DnSpinMat(kctr,j)];
            UpSpinCompatibleLst=setdiff(setdiff(1:DnSpinMat(kctr,j),DnSpinMat(kctr,1:j)),bracketLstLeft);
            bracketLstLeft=[bracketLstLeft,max(UpSpinCompatibleLst)];
        end
    end
%    [bracketLstLeft;bracketLstRight]
    NumBracketLst=[NumBracketLst,length(bracketLstLeft)];
    v = sparse(idxLst(kctr),1,1,dim,1);
    for j=1:length(bracketLstLeft),
        a=bracketLstLeft(j);
        b=bracketLstRight(j);
        v = v-kron(speye(2^(a-1)),kron(E21,kron(speye(2^(b-a-1)),kron(E12,speye(2^(L-b))))))*v;
    end
    ExcessDnSpin = setdiff(DnSpinMat(kctr,:),bracketLstRight);
    for j=1:length(ExcessDnSpin)
        a=ExcessDnSpin(j);
        v = kron(speye(2^(a-1)),kron(E12,speye(2^(L-a))))*v;
    end
    vectorLst = [vectorLst,v];
end

for numBrktCtr = 0:(L/2)
%    numBrktCtr
    fprintf(fileID,'%d\r\n\r\n',numBrktCtr);
    Indices = find(NumBracketLst==numBrktCtr);
    vMat = vectorLst(:,Indices);
    Amat = vMat'*H*vMat;
    Bmat = vMat'*vMat;
    E0 = eigs(Amat,Bmat,1,'smallestreal','Tolerance',1e-4);
    fprintf(fileID,'%f\r\n\r\n',E0);
end
fclose(fileID);
\end{verbatim}

\normalsize
To obtain the full eigenvalue list (for sufficiently small values of $L$ such as $L=12$, where this is feasible)
change the final for-loop to this:

\scriptsize

\begin{verbatim}

for numBrktCtr = 0:(L/2)
    numBrktCtr
    Indices = find(NumBracketLst==numBrktCtr);
    vMat = vectorLst(:,Indices);
    Amat = vMat'*H*vMat;
    Bmat = vMat'*vMat;
    [V,D] = eig(full(Amat),full(Bmat),'chol');
        fprintf(fileID,'%f\r\n\r\n',diag(D));
end
fclose(fileID);
end
\end{verbatim}

\normalsize

\section{First commutator calculations}
\label{app:FirstCommut}

We first calculate
\begin{equation}
\label{eq:FirstComm}
\begin{split}
	\sum_{r'=1}^{L} [\widetilde{\mathcal{O}}_k^+,S_{r'}^{(3)} S_{r'+1}^{(3)}]\,
	&=\,
	\sum_{r=1}^{L} \sum_{r'=1}^{L} e^{2\pi i kr/L} [S_r^{+},S_{r'}^{(3)} S_{r'+1}^{(3)}]\\
	&=\, \sum_{r=1}^{L} \sum_{r'=1}^{L} e^{2\pi i kr/L} \Big(  [S_r^{+},S_{r'}^{(3)}] S_{r'+1}^{(3)} 
	+  S_{r'}^{(3)} [S_r^+,S_{r'+1}^{(3)}]\Big)\\
	&=\, -\sum_{r=1}^{L} \sum_{r'=1}^{L} e^{2\pi i kr/L} \big(  \delta_{r,r'} S_r^+ S_{r'+1}^{(3)} 
	+\delta_{r'+1,r}  S_{r'}^{(3)} S_r^+\big)\\ 
	&=\, -\sum_{r=1}^{L}   \big( e^{2\pi i kr/L}  S_r^+ S_{r+1}^{(3)} 
	+ e^{2\pi i (k+1)r/L} S_{r}^{(3)} S_{r+1}^+\big)\, .
\end{split}
\end{equation}
Therefore, this implies
\begin{equation}
\begin{split}
\label{eq:FirstCommPrime}
	[\widetilde{\mathcal{O}}_k^+,H_L^{(Z)}]\,
	=\,	
	-\sum_{r=1}^{L} [\widetilde{\mathcal{O}}_k^+,S_{r}^{(3)} S_{r+1}^{(3)}]\,
	=\, \sum_{r=1}^{L}   \big( e^{2\pi i kr/L}  S_r^+ S_{r+1}^{(3)} 
	+ e^{2\pi i (k+1)r/L} S_{r}^{(3)} S_{r+1}^+\big)\, .
\end{split}
\end{equation}
Now, let us turn our attention to $H_L^{(XY)}$.
We note that
we may also rewrite
\begin{equation*}
%\label{eq:HamiltonianFormulaXY2}
	H_{L}^{(XY)}\,  =\, - \sum_{r=1}^{L} \Big(S_r^{(1)} S_{r+1}^{(1)} + S_r^{(2)} S_{r+1}^{(2)}\Big)\,
 =\, -\frac{1}{2}\, \sum_{r=1}^{L} \Big(S_r^{+} S_{r+1}^{-} + S_r^{-} S_{r+1}^{+}\Big)\, .
\end{equation*}
Because spin-raising operators commute with other spin-raising operators, we may write
\begin{equation*}
	[\widetilde{\mathcal{O}}_k^+,S_r^-S_{r+1}^+]\, =\, 	[\widetilde{\mathcal{O}}_k^+,S_r^-] S_{r+1}^+\, ,
\end{equation*}
which is $2e^{2\pi i k r/L} S_r^{(3)} S_{r+1}^-$.
Then we get, by summing
\begin{equation}
\label{eq:HpmComm}
	-\frac{1}{2}\, \sum_{r=1}^{L}  [\widetilde{\mathcal{O}}_k^+,S_{r}^{-} S_{r+1}^{+}]\,
	=\, -\sum_{r=1}^{L} e^{2\pi i kr/L} S_r^{(3)} S_{r+1}^{+} \, .
\end{equation}
By a similar calculation,
\begin{equation}
\label{eq:HmpComm}
	-\frac{1}{2}\, \sum_{r=1}^{L}  [\widetilde{\mathcal{O}}_k^+,S_{r}^{+} S_{r+1}^{-}]\,
	=\, -\sum_{r=1}^{L} e^{2\pi i k(r+1)/L} S_r^{+} S_{r+1}^{(3)} \, .
\end{equation}
Therefore, adding equations (\ref{eq:HpmComm}) and (\ref{eq:HmpComm}), we have
\begin{equation*}
	 [\widetilde{\mathcal{O}}_k^+,H_L^{(XY)}]\,
	=\, -\sum_{r=1}^{L} \big(e^{2\pi i kr/L} S_r^{(3)} S_{r+1}^{+} +
	e^{2\pi i k(r+1)/L} S_r^{+} S_{r+1}^{(3)}\big)\, .
\end{equation*}
Together with (\ref{eq:FirstCommPrime}) this equation proves (\ref{eq:TBAapp1a}) and (\ref{eq:TBAapp1b}).

\section{Double commutator in first order perturbation theory}
\label{app:DoubleCommut}

Since we have assumed $H_{j,L}^{\mathrm{FM}} \Psi = E \Psi$, or $(H_{j,L}^{\mathrm{FM}}-E \mathbbm{1}) \Psi = 0$, 
and since $[H_{j,L}^{\mathrm{FM}}-E \mathbbm{1},\cdot]=[H_{j,L}^{\mathrm{FM}},\cdot]$,
\begin{equation*}
\langle \Psi\, ,\  \widetilde{\mathcal{O}}_{-k}^- (H_{j,L}^{\mathrm{FM}}-E) \widetilde{\mathcal{O}}_k^+ \Psi\rangle\, 
=\, \langle \Psi\, ,\  \widetilde{\mathcal{O}}_{-k}^-  [H_{j,L}^{\mathrm{FM}},\widetilde{\mathcal{O}}_k^+] \Psi\rangle\, .
\end{equation*}
Then we may continue in this way
\begin{equation*}
\langle \Psi\, ,\  \widetilde{\mathcal{O}}_{-k}^- (H_{j,L}^{\mathrm{FM}}-E) \widetilde{\mathcal{O}}_k^+ \Psi\rangle\, 
=\, \langle \Psi\, ,\ [\widetilde{\mathcal{O}}_{-k}^-,  [H_{j,L}^{\mathrm{FM}},\widetilde{\mathcal{O}}_k^+]] \Psi\rangle\,
+ \langle \Psi\, ,\   [H_{j,L}^{\mathrm{FM}},\widetilde{\mathcal{O}}_k^+] \widetilde{\mathcal{O}}_{-k}^- \Psi\rangle\, .
\end{equation*}
Then we can see that
\begin{equation*}
\langle \Psi\, ,\   [H_{j,L}^{\mathrm{FM}},\widetilde{\mathcal{O}}_k^+] \widetilde{\mathcal{O}}_{-k}^- \Psi\rangle\, 
=\, \langle \Psi\, ,\   H_{j,L}^{\mathrm{FM}}\widetilde{\mathcal{O}}_k^+ \widetilde{\mathcal{O}}_{-k}^- \Psi\rangle
- \langle \Psi\, ,\   \widetilde{\mathcal{O}}_k^+ H_{j,L}^{\mathrm{FM}}\widetilde{\mathcal{O}}_{-k}^- \Psi\rangle\, .
\end{equation*}
Since $\Psi^* H_{j,L}^{\mathrm{FM}} = E \Psi^*$, for an eigenvector, too, we see
\begin{equation*}
\langle \Psi\, ,\   [H_{j,L}^{\mathrm{FM}},\widetilde{\mathcal{O}}_k^+] \widetilde{\mathcal{O}}_{-k}^- \Psi\rangle\, 
=\, - \langle \Psi\, ,\   \widetilde{\mathcal{O}}_k^+ (H_{j,L}^{\mathrm{FM}}-E)\widetilde{\mathcal{O}}_{-k}^- \Psi\rangle\, .
\end{equation*}
Therefore, we may write
\begin{equation*}
\langle \Psi\, ,\  \widetilde{\mathcal{O}}_{-k}^- (H_{j,L}^{\mathrm{FM}}-E) \widetilde{\mathcal{O}}_k^+ \Psi\rangle
+ \langle \Psi\, ,\   \widetilde{\mathcal{O}}_k^+ (H_{j,L}^{\mathrm{FM}}-E)\widetilde{\mathcal{O}}_{-k}^- \Psi\rangle\, 
=\, \langle \Psi\, ,\ [\widetilde{\mathcal{O}}_{-k}^-,  [H_{j,L}^{\mathrm{FM}},\widetilde{\mathcal{O}}_k^+]] \Psi\rangle\, .
\end{equation*}
This gives equation (\ref{eq:penultimatePerturbation}), as was desired.

\section{Calculation of the Double Commutator in the Eigenstate}

\label{app:SecondCommut}

We note that the commutator below is non-zero only if $r'$ is in the set $\{r,r+1\}$:
$$
\left[S_{r'}^{-},\Big(S_r^{+} S_{r+1}^{(3)}- S_r^{(3)} S_{r+1}^{+}\Big)\right] = 
[S_{r'}^-,S_r^+] S_{r+1}^{(3)} + S_r^+ [S_{r'}^-,S_{r+1}^{(3)}] - [S_{r'}^-,S_r^{(3)}] S_{r+1}^+ - S_r^{(3)} [S_{r'}^-,S_{r+1}^+]\, .
$$
We have used the fact that operators localized at different sites commute (because $A\otimes \mathbbm{1}$ and $\mathbbm{1}\otimes B$ commute).
This then gives
$$
\left[S_{r'}^{-},\Big(S_r^{+} S_{r+1}^{(3)}- S_r^{(3)} S_{r+1}^{+}\Big)\right] = 
-2\delta_{r',r} S_r^{(3)} S_{r+1}^{(3)} + \frac{1}{2}\, \delta_{r',r+1} S_r^+ S_{r'}^- - \frac{1}{2}\, \delta_{r',r} S_{r'}^- S_{r+1}^+ 
+ 2\delta_{r',r+1} S_r^{(3)} S_{r+1}^{(3)}\, .
$$
Using this in equation (\ref{eq:DoubleCommToDo1}) leads to equation (\ref{eq:DoubleCommFinal}).
Finally, we note the symmetries of $\Psi^{\mathrm{FM}}_{\min}(0;j,L)$, being a spin singlet (so that it is invariant under the symmetries of $\mathrm{SU}(2)$),
a translation eigenvector, and a spin-flip eigenvector (and a reflection eigenvector). Therefore,
\begin{equation*}
\langle \Psi^{\mathrm{FM}}_{\min}(0;j,L)\, ,\  S_r^{(3)} S_{r+1}^{(3)}  \Psi^{\mathrm{FM}}_{\min}(0;j,L)\rangle\,
=\, \frac{1}{3L} \langle \Psi^{\mathrm{FM}}_{\min}(0;j,L)\, ,\  (H_{j,L}^{\mathrm{FM}}-j^2L)  \Psi^{\mathrm{FM}}_{\min}(0;j,L)\rangle\, .
\end{equation*}
Similarly,
\begin{equation*}
\begin{split}
&
\langle \Psi^{\mathrm{FM}}_{\min}(0;j,L)\, ,\  S_r^{+} S_{r+1}^{-}  \Psi^{\mathrm{FM}}_{\min}(0;j,L)\rangle\,
=\, \langle \Psi^{\mathrm{FM}}_{\min}(0;j,L)\, ,\  S_r^{-} S_{r+1}^{+}  \Psi^{\mathrm{FM}}_{\min}(0;j,L)\rangle\\
&\hspace{7cm} =\, \frac{2}{3L} \langle \Psi^{\mathrm{FM}}_{\min}(0;j,L)\, ,\  (H_{j,L}^{\mathrm{FM}}-j^2L)  \Psi^{\mathrm{FM}}_{\min}(0;j,L)\rangle\, .
\end{split}
\end{equation*}
Therefore, from (\ref{eq:DoubleCommFinal}), we have 
\begin{equation*}
\begin{split}
&\langle \Psi^{\mathrm{FM}}_{\min}(0;j,L)\, ,\  [\widetilde{\mathcal{O}}^-_{-k},[\widetilde{\mathcal{O}}_k^+,H_{j,L}^{\mathrm{FM}}] \Psi^{\mathrm{FM}}_{\min}(0;j,L)\rangle\,	\\ 
&\hspace{3cm}	=\,  -\sum_{r=1}^{L} \Big(2(1-\omega_k)(1-\omega_{-k})\, \frac{1}{3L} 
+(1-\omega_k)\, \frac{2}{3L}
+(1-\omega_{-k})\frac{2}{3L}\Big)  \left(E_0 - j^2L\right)\, .
\end{split}
\end{equation*}
But $1-\omega_k-\omega_{-k}+\omega_{k}\omega_{-2}=(1-\omega_k) + (1-\omega_{-k})=2-2\cos(2\pi k/L)=4 \sin^2(\pi k/L)$.
Using this, we obtain equation (\ref{eq:DoubleToDo2}).

\baselineskip=12pt
\bibliographystyle{plain}
\begin{thebibliography}{10}

\bibitem{Auerbach}
Assa Auerbach.
\newblock {\em Interacting Electrons and Quantum Magnetism.}
\newblock Springer-Verlag, New York 1994.

\bibitem{CaputoLiggettRichthammer}
Thomas Liggett, Pietro Caputo and Thomas Richthammer.
\newblock Proof of Aldous' Spectral Gap Conjecture.
\newblock {\em J.~Amer.~Math.~Soc.} {\bf 23}, no.~3, 831--851 (2010).

\bibitem{CGS}
Michele Correggi, Alessandro Giuliani and Robert Seiringer.
\newblock Validity of the Spin-Wave Approximation for the Free Energy of the Heisenberg Ferromagnet.
\newblock {\em Commun.~Math.~Phys.} {\bf 339}, 279--307 (2015).

\bibitem{DLS}
Freeman J.~Dyson, Elliott H.~Lieb and Barry Simon.
\newblock Phase Transitions in Quantum Spin Systems with Isotropic and Nonisotropic Interactions.
\newblock {\em J.~Statist.~Phys.} {\bf 18}, 335--383 (1978).

\bibitem{Haldane}
F.~D.~M.~Haldane.
\newblock Continuum Dynamics of the 1-D Heisenberg Antiferromagnet: Identification with the O(3) Nonlinear Sigma Model.
\newblock {\em Phys.~Lett.~A} {\bf 93}, no.~9, 464--468 (1983).

\bibitem{Kennedy}
Tom Kennedy.
\newblock Long Range Order in the Anisotropic Quantum Ferromagnetic Heisenberg Model.
\newblock {\em Comm.~Math.~Phys.} {\bf 100}, no.~3, 447--462 (1985).

\bibitem{LiebMattisOEL}
Elliott Lieb and Daniel Mattis.
\newblock Ordering of Energy Levels of Interacting Spin Systems.
\newblock {\em J.~Mathem.~Phys.} {\bf 3}, no.~4, 749--751 (1962).

\bibitem{LiebSchultzMattis}
Elliott Lieb, Theodore Schultz and Daniel Mattis.
\newblock Two Soluble Models of an Antiferromagnetic Chain.
\newblock {\em Ann.~Physics} {\bf 16}, 407--466 (1961).

\bibitem{NSSfoel}
Bruno Nachtergaele, Wolfgang Spitzer and Shannon Starr.
\newblock Ferromagnet Ordering of Energy Levels.
\newblock {\em J.~Statist.~Phys.} {\bf 116}, 719--738 (2004).

\bibitem{NachtergaeleStarr}
Bruno Nachtergaele and Shannon Starr.
\newblock Ferromagnetic Lieb-Mattis Theorem.
\newblock {\em Phys.~Rev.~Lett.} {\bf 94} 057206 (2005).


\bibitem{Speer}
Eugene~R.~Speer.
\newblock Failure of Reflection Positivity in the Quantum Heisenberg Ferromagnet.
\newblock {\em Lett.~Math.~Phys.} {\bf 10}, 41--47 (1985).

\bibitem{SpitzerStarrTran}
Wolfgang Spitzer, Shannon Starr and Lam Tran.
\newblock Counterexamples to Ferromagnetic Ordering of Energy Levels.
\newblock {\em J.~Math.~Phys.} {\bf 53}, no.~4, 043302 (2012).

\bibitem{Sutherland}
Bill Sutherland.
\newblock{Low-Lying Eigenstates of the One-Dimensional Heisenberg Ferromagnet for any Magnetization and Momentum.}
\newblock {\em Phys.~Rev.~Lett.} {\bf 75}, no.~5, 816--819 (1995).

\bibitem{TemperleyLieb}
H.~N.~V.~Temperley and Elliott H.~Lieb.
\newblock Relation Between the `Percolation' and `Colouring' Problem,
and Other Graph Theoretical Problems Associated with Regular Planar Lattices:
Some Exact Results for the `Percolation' Problem.
\newblock {\em Proc.~Royal Soc.~London A} {\bf 322}, 251--280 (1971). 


\end{thebibliography}

\end{document}
