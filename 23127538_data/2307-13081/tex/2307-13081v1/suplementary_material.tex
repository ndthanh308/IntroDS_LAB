\newpage
\section{Supplementary Material}
\label{sec:appendix}
Figure~\ref{fig:expgrad-lr} shows fairness-accuracy trade-offs achieved by the exponential gradient with logistic regression as fairness mechanism on various sensitive attribute proxies and our methods. Similar to the results presented in the main paper, our methods achieve better fairness-accuracy trade-offs. Figure~\ref{fig:avd_debiasing} shows the trade-off for adversarial debiasing. Our methods achieve a better trade-off on the Adult datasets while for the Compas dataset, the ground-truth sensitive achieves a better trade-off. It is worth noting that adversarial debiasing is unstable to train.   
% Figure environment removed


Figure~\ref{fig:ablation-impact-uncertainty-gbm} shows the accuracy-fairness trade-off exponential gradient using gradient-boosted trees as the base classifier for various uncertainty thresholds, the true sensitive attributes, and the predicted sensitive attributes with DNN. The results obtained are similar to random forests as the base classifier. The smaller uncertainty threshold produced the best trade-off in a high-bias regime such as the Adult dataset. While on datasets that do not encode much information about the sensitive attributes (most samples have high uncertainty) such as the New Adult and Compas datasets, the trade-off gets worse as the uncertainty threshold reduces.

% Figure environment removed

\subsection{Uncertainty Estimation of Different Demographic Groups}
In this paper, we showed that when the dataset does not encode enough information about the sensitive attributes, the attribute classifier suffers on average from greater uncertainty in the predictions of sensitive attributes. This encourages a choice of a higher uncertainty threshold to keep enough samples in order to achieve a better trade-off, i.e. to prune out only the most uncertain samples. Figure~\ref{fig:representation_rate_samples} shows that the gap between demographic groups increases as a smaller uncertainty threshold is used. This is explained by the fact that the model is more confident about samples from well-represented groups than samples from under-represented groups. While this gap between demographic groups can increase, our results show that tuning and using the right uncertainty threshold can result in a model that achieves a better trade-off between accuracy and fairness across various fairness metrics, by enforcing fairness constraints on samples with highly reliable sensitive attributes.  

% Figure environment removed

% Figure environment removed