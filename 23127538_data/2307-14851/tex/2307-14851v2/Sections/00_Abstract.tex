\begin{abstract}
Machine learning-based seismic processing models are typically trained separately to perform specific seismic processing tasks (SPTs), and as a result, require plenty of training data. However, preparing training data sets is not trivial, especially for supervised learning (SL). Nevertheless, seismic data of different types and from different regions share generally common features, such as their sinusoidal nature and geometric texture. To learn the shared features, and thus, quickly adapt to various SPTs, we develop a unified paradigm for neural network-based seismic processing, called Meta-Processing, that uses limited training data for meta learning a common network initialization, which offers universal adaptability features. The proposed Meta-Processing framework consists of two stages: meta-training and meta-testing. In the meta-training stage, each SPT is treated as a separate task and the training dataset is divided into support and query sets. Unlike conventional SL methods, here, the neural network (NN) parameters are updated by a bilevel gradient descent from the support set to the query set, iterating through all tasks. In the meta-testing stage, we also utilize limited data to fine-tune the optimized NN parameters in an SL fashion to conduct various SPTs, such as denoising, interpolation, ground-roll attenuation, image enhancement, and velocity estimation, aiming to converge quickly to ideal performance. Comprehensive numerical examples are performed to evaluate the performance of Meta-Processing on both synthetic and field data. The results demonstrate that our method significantly improves the convergence speed and prediction accuracy of the NN.
\end{abstract}

\keywords{Seismic processing \and Neural network \and Meta learning}

