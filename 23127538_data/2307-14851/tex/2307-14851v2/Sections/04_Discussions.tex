\section{Discussion}\label{sec4}
The field of seismic processing has long been a cornerstone in unraveling the mysteries of the Earth's interior. As seismic waves traverse through the Earth's various layers, they carry vital information about its composition, structure, and dynamics. The traditional paradigm of seismic processing often relied on handcrafted algorithms tailored to specific tasks. However, the rapidly evolving landscape of machine learning and NNs has allowed us to reframe this approach. In this study, we present a pioneering framework for multi-task seismic processing using NNs, which not only enhances the efficiency and accuracy of data processing but also unlocks novel insights into the shared features of seismic data across diverse geographical regions and geological settings. 

One of the pivotal aspects of our framework lies in its utilization of MetaL, a technique that empowers models to rapidly adapt to new tasks with limited labeled data. By learning the shared characteristics of seismic data, our approach provides a robust initialization for multi-task seismic processing. This strategic initialization facilitates the seamless adaptation of the model to various SPTs, even when data availability is constrained. Looking back at Figure 18, for example, Meta-Processing removed most of the arc-shaped artifacts and produced a continuous layering of the subsurface on the image, much better than that of the randomly-initialized network. This could lead to a more accurate interpretation of the subsurface and, thus, provides meticulous insight into the deep Earth. Another example was demonstrated in Figure 19, where our proposed framework yielded artifacts-free velocity estimates compared to the results of the randomly-initialized networks, which still contain seismic footprints. These clean $V_{rms}$ estimates are beneficial for the subsequent processing steps, aiding them to produce better imaging of the subsurface. The underlying implication of this achievement is profound – the presence of common features within seismic data suggests that despite the geological diversity and spatial separation, the Earth's interior shares inherent attributes and properties that influence the observed seismic waves. In other words, seismic waves, despite originating from distinct geographical regions and geological contexts, share common characteristics that can be exploited to enhance data processing efficiency and accuracy.

The discovery of shared seismic features through our Meta-Processing framework opens a new avenue for understanding the Earth's solid interior. While each seismic event may seem unique on the surface, our findings hint at an underlying universality in the behavior of seismic waves. The seismic waves recorded across various regions and geological formations appear to encode consistent traits, hinting at fundamental attributes within the Earth's crust and mantle. These shared features may be reflective of universal geological processes or structural arrangements that transcend regional differences, or might be rooted in fundamental principles governing the propagation of seismic waves through the Earth's interior. By delving deeper into these shared characteristics, we might uncover latent relationships between geological properties and seismic wave behaviors, contributing to a deeper comprehension of the Earth's physical properties and processes. \\
