\section{Conclusion}
We proposed a unified paradigm for neural network-based seismic processing, Meta-Processing, to provide a powerful technique for various seismic processing tasks (SPTs), which can be efficiently trained on the limited training data. We utilized a modified residual network baseline to replace the conventional convolutional layers of the classic UNet network as our basic network architecture. Within the framework of the Meta-Processing, the network is performed in two stages: meta-training and meta-testing, each with distinct goals and implementation details. In the meta-training stage, a bilevel gradient descent updating from the support set to the query set is used to optimize the network parameters, aiming to obtain a robust initialization that can quickly adapt to various SPTs. In the meta-testing stage, the optimized network parameters from the meta-training stage are fine-tuned on various SPTs, with the same procedure to conventional supervised learning. The objective in this stage is to achieve rapid convergence and a significant improvement in prediction accuracy. \\

We conducted comprehensive numerical examples to demonstrate the performance of our method on various SPTs, including denoising, interpolation, ground-roll attenuation, imaging enhancement, and velocity estimation. Results on both synthetic and field data showed the superiority of our method in terms of convergence speed and prediction accuracy, as compared to randomly initialized UNet. Specifically, even when our method was only trained on synthetic data, it did not exhibit significant performance degradation when predicting field data. In contrast, the randomly initialized network, even after prolonged optimization, resulted in a significant decrease in accuracy when applied to field data. This means that the Meta-Processing algorithm not only extracts shared seismic features but also enhances its adaptability from the synthetic data domain to the field data domain. As a result, this study will help advance neural network-based seismic processing for various practical applications.
