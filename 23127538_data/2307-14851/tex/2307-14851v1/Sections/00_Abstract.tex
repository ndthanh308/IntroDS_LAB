\begin{abstract}
Machine learning-based seismic processing tasks (SPTs) are typically trained separately and require plenty of training data. However, acquiring training data sets is not trivial, especially for supervised learning (SL). Nevertheless, seismic data of different types and from different regions share generally common features, such as their sinusoidal nature and geometric texture. To learn the shared features and thus quickly adapt to various SPTs, we develop a unified paradigm for neural network-based seismic processing, called Meta-Processing, that uses limited training data for meta learning a common network initialization, which offers universal adaptability features. The proposed Meta-Processing framework consists of two stages: meta-training and meta-testing. In the meta-training stage, each SPT is treated as a separate task and the training dataset is divided into support and query sets. Unlike conventional SL methods, here, the neural network (NN) parameters are updated by a bilevel gradient descent from the support set to the query set, iterating through all tasks. In the meta-testing stage, we also utilize limited data to fine-tune the optimized NN parameters in an SL fashion to conduct various SPTs, such as denoising, interpolation, ground-roll attenuation, imaging enhancement, and velocity estimation, aiming to converge quickly to ideal performance. Comprehensive numerical examples are performed to evaluate the performance of Meta-Processing on both synthetic and field data. The results demonstrate that our method significantly improves the convergence speed and prediction accuracy of the NN.
\end{abstract}

\section*{Plain Language Summary}
Seismic processing, which is a fundamental component in seismology, significantly enhances seismic data quality and enables precise identification of subsurface structures and properties. Conventional processing workflows are often manually driven, and thus, subjective, and time consuming. Thus, we develop a unified framework for seismic processing based on neural networks (NNs). The advantages of this framework are twofold: 1) it provides a shared, universally adaptable initialization for various seismic processing tasks (SPTs), enabling rapid convergence of NNs to optimal performance on each task; 2) considering the difficulty in acquiring labeled data, with this framework we need relatively limited data for NN training.  The key is that our network under this framework learns key features representing the input data (like textures and frequencies) that is helpful to preform many tasks efficiently. Comprehensive evaluations on both synthetic and field data demonstrate that our framework significantly enhances the convergence speed and prediction accuracy of NNs for various SPTs. This development represents a promising advancement in seismic processing methods, offering machine learning solutions without the need for large amounts of training data, and opens the door for new opportunities in multi-task seismic data analysis.  