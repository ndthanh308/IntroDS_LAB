\documentclass[draft]{agujournal2019}
\usepackage{url} 
\usepackage{lineno}
\usepackage[inline]{trackchanges} 

\usepackage{soul}
\usepackage{amsmath}
\usepackage{ragged2e}
\usepackage{subcaption}
\usepackage{algorithm}
\usepackage{algpseudocode}


\draftfalse

\journalname{JGR: Solid Earth}


\begin{document}

\title{Meta-Processing: A robust framework for multi-tasks seismic processing}

\authors{Shijun~Cheng$^1$, Randy~Harsuko$^1$, Tariq~Alkhalifah$^1$}


\affiliation{1}{Division of Physical Science and Engineering, King Abdullah University of Science and Technology, Thuwal 23955-6900, Saudi Arabia}


\correspondingauthor{Shijun~Cheng}{sjcheng.academic@gmail.com}


\begin{keypoints}
\item We develop a robust neural network-based seismic processing framework that can be applied to various tasks effectively
\item Our framework uses limited training data for meta learning a common network initialization and thus offers universal adaptability features
\item Our framework facilitates the rapid convergence of neural networks to ideal performance across multi-tasks seismic processing
\end{keypoints}

\justifying

\begin{abstract}
Machine learning-based seismic processing models are typically trained separately to perform specific seismic processing tasks (SPTs), and as a result, require plenty of training data. However, preparing training data sets is not trivial, especially for supervised learning (SL). Nevertheless, seismic data of different types and from different regions share generally common features, such as their sinusoidal nature and geometric texture. To learn the shared features, and thus, quickly adapt to various SPTs, we develop a unified paradigm for neural network-based seismic processing, called Meta-Processing, that uses limited training data for meta learning a common network initialization, which offers universal adaptability features. The proposed Meta-Processing framework consists of two stages: meta-training and meta-testing. In the meta-training stage, each SPT is treated as a separate task and the training dataset is divided into support and query sets. Unlike conventional SL methods, here, the neural network (NN) parameters are updated by a bilevel gradient descent from the support set to the query set, iterating through all tasks. In the meta-testing stage, we also utilize limited data to fine-tune the optimized NN parameters in an SL fashion to conduct various SPTs, such as denoising, interpolation, ground-roll attenuation, image enhancement, and velocity estimation, aiming to converge quickly to ideal performance. Comprehensive numerical examples are performed to evaluate the performance of Meta-Processing on both synthetic and field data. The results demonstrate that our method significantly improves the convergence speed and prediction accuracy of the NN.
\end{abstract}

\keywords{Seismic processing \and Neural network \and Meta learning}


\section{Introduction}
\label{sec:intro}
\looseness -1
Gaining insight into population trends allows data analysts to make data-driven decisions to improve user experience.
Heavy hitter detection, or learning popular data points generated by users, plays an important role in learning about user behavior.
A well-known example of this is learning ``out-of-vocabulary" words typed on keyboard, which can then be used to improve next word prediction models. 
%The problem has been studied extensively in the setting where each user has a single data point, and one has no privacy concerns (see e.g. \cite{CHARIKAR20043, CORMODE200558}). However, t
This data is often sensitive and the privacy of users' data is paramount. When the data universe is small, one can obtain private solutions to this problem by directly using private histogram algorithms such as RAPPOR~\cite{erlingsson2014rappor}, and PI-RAPPOR~\cite{feldman2021lossless}, and reading off the heavy-hitters. However, when the data universe is large, as is the case with ``out-of-vocabulary" words, these solutions result in algorithms with either very high communication, or very high server side computation, or both.
Prefix-tree based iterative algorithms can lower communication and computation costs, while maintaining high utility by efficiently exploring the data universe for heavy hitters.
They also offer an additional advantage in the setting where users have multiple data points by refining the query in each iteration\longversion{ using the information learned thus far}, allowing each user to select amongst those data points which are more likely to be heavy hitters.

\looseness -1
In this work, we consider an iterative federated algorithm for heavy hitter detection in the aggregate model of differential privacy (DP) in the presence of computation or communication constraints. In this setting, each user has a private dataset on their device. In each round of the algorithm, the data analyst sends a query to the set of participating devices, and each participating device responds with a  \textit{response}, which is a random function of the private dataset of that user. These \textit{responses} are then summed using a secure aggregation protocol, and reported to the data analyst. The analyst can then choose a query for the next round adaptively, based on the aggregate results they have seen so far. The main DP guarantee is a user-level privacy guarantee on the outputs of the secure aggregator, accounting for the privacy cost of \emph{all} rounds of iteration. 
Our algorithm will additionally be DP in the local model of DP (with a larger privacy parameter)\footnote{A potential architecture for running iterative algorithms in this model of privacy is outlined in \cite{mcmillan2022private}.}.
\longversion{We do not assume that the set of participating devices is consistent between rounds. }

\looseness -1
In the central model of DP, there is a long line of work on adaptive algorithms for heavy hitter detection in data with a hierarchical structure such as learning popular $n$-grams~\cite{cormode2012differentially, Qardaji:2012, Song:2013, bagdasaryan2021towards, Kim2021DifferentiallyPN, mcmillan2022private}. These interactive algorithms all follow the same general structure. Each data point is represented as a sequence of data segments $d=a_1a_2\cdots a_r$ and the algorithm iteratively finds the popular values of the first segment $a_1$, then finds popular values of $a_1a_2$ where $a_1$ is restricted to only heavy hitters found in the previous iteration, and so on. 
This limits the domain of interest at each round, lowering communication and computation costs.
The method of finding the heavy hitters in each round of the algorithm varies in prior work, although is generally based on a DP frequency estimation subroutine. One should consider system constraints (communication, computation, number of participating devices, etc.) and the privacy model when choosing a frequency estimation subroutine. In this work, we will focus on using one-hot encoding with binary randomized response (inspired by RAPPOR~\cite{erlingsson2014rappor}) as our DP frequency estimation subroutine. Since we are primarily interested in algorithmic choices that affect the iterative algorithm, we believe our findings should be agnostic to the choice of frequency estimation subroutine used. 

We explore the effect on utility of different data selection schemes and algorithmic optimizations. 
We refer to our algorithm as \textit{Optimized Prefix Tree} (\textit{$\ouralgorithm$}). Our contributions are summarised below:

    \textbf{Adaptive Segmentation.} We propose an algorithm for adaptively choosing the segment length and the threshold for keeping popular prefixes. In contrast to prior works that treat the segment length as a hyperparameter, our algorithm chooses these parameters in response to user data from the previous iteration and attempts to maximize utility (measured as the fraction of the empirical probability distribution across all users captured by the returned heavy hitters), while satisfying any system constraints. We find that our method often results in the segment length varying across iterations, and outperforms the algorithm that uses a constant segment length. We also design a threshold selection algorithm that adaptively chooses the prefix list for the subsequent round. This allows us to control the false positive rate\longversion{ (the likelihood that a data point is falsely reported as a heavy hitter)}.
    
    \textbf{Analysis of the effect of on-device data selection mechanisms.} We explore the impact of interactivity in the setting where users have multiple data points. We observe empirically that when users have multiple data points, interactivity can improve utility, even in the absence of system constraints. In each round, users choose a single data point from their private data set to (privately) report to the server.
    The list of heavy hitters in the previous iteration provides a \emph{prefix list}, so users will only choose a data point with one of the allowed prefixes. If a user has several data points with allowed prefixes, then there are several selection rules they may use to choose which data point to report. Each user's private dataset defines an empirical distribution for that user. 
    We find that when users sample uniformly randomly from the support of their distribution (conditioned on the prefix list) then the algorithm is able to find more heavy hitters than when they sample from their empirical distribution (again conditioned on the prefix list). 

    \textbf{Analysis of the impact of inclusion of deny list.} Under the constraint of user-level differential privacy, each user is only able to communicate their most frequent data points, and less frequent data points are down weighted. We explore the use of a \emph{deny list} that asks users not to report data points that we already know are heavy hitters. In practice, a deny list may arise from an auxiliary data source, or from a prior run of the algorithm. Our analysis indicates even when the privacy budget is shared between multiple rounds of the algorithm, performing a second round equipped with a deny list improves performance.

The rest of the paper is organized as follows. In Section~\ref{sec:related} we discuss some of the prior works in privacy-preserving heavy hitters detection. Section~\ref{sec:privacy} explains the privacy primitives we used in this work. In Section~\ref{sec:alg} we elaborate the details of our prefix tree algorithm. Section~\ref{sec:post} explains the post-processing methods and the theoretical analysis behind it. Section~\ref{sec:expts} demonstrates the experimental results and in Section~\ref{sec:conclusion} we discuss the findings of our experiments. 

\section{Methodologies}\label{sec2}
In this section we will highlight the concept of NN-based processing then describe the components of the Meta-processing including the algorithm, the data set, the network architecture, and loss function. 

\subsection{Neural network-based seismic processing}
Seismic processing is a complex and data-intensive task that aims to extract valuable information about the subsurface structure of the Earth by analyzing the signals embedded in recorded seismic waves. Machine learning algorithms, particularly NN, have emerged as a promising tool for addressing the challenges posed by seismic processing. The basic idea behind NN-based seismic processing, which is shown in Figure \ref{fig1}, is to leverage the power of machine learning algorithms to learn the underlying mapping relationships between input seismic data $x$ and the ideal processing results $y$ as follows:
\begin{equation}
\setlength{\abovedisplayskip}{3pt}
\setlength{\belowdisplayskip}{3pt}
y=\mathrm{NN}(x;\boldsymbol{\theta}),
\end{equation}
where the mapping relationships are represented by a parameterized function ${f_{\boldsymbol{\theta}}}$ with the learned parameters $\boldsymbol{\theta}$ of the NN. 

To obtain the parameterized function ${f_{\boldsymbol{\theta}}}$ for a specific SPT, we usually need to train the NN from scratch, whose workflow is reviewed in Figure \ref{fig2}. In brief, a large dataset is prepared and preprocessed, for example by normalizing it to $[-1,1]$. Next, we design an appropriate network architecture appropriate for the specific task at hand. Afterwards, careful consideration must be taken to set the hyperparameters (e.g., learning rate), NN initialization, the selection of optimizer, and the definition of a loss function. Initialization is a crucial component that affects the network's convergence speed and final optimization results, and therefore, it is essential to provide an appropriate initialization for the network. In practice, however, many initialization techniques rely on random or default values. Once these steps have been completed, the network can be trained using gradient descent to update the network's parameters iteratively. 

While the NN can effectively process input seismic data to predict desired output variables of interest after being trained, its efficacy is limited to the specific dataset and the task it was trained on. Furthermore, even for the same task, performance can be severely impacted when applied the network on seismic data from other regions. This occurs because the NN is only able to capture a limited number of seismic features from the training dataset, resulting in restricted performance and generalizability. Therefore, re-optimizing the NN is necessary to achieve advanced performance, although this process can be time-consuming. 

In light of these challenges, we are compelled to inquire whether it is possible to present a multi-task processing network with exceptional generalization abilities, capable of achieving superior performance via minimal gradient updates. In other words, our objective transcends the confines of a solitary SPT and instead seeks a task-level mapping relationship by standing on the shoulders of various SPTs. We illustrate this motivation in Figure \ref{fig3}. As seen, we aim to train a task-level parameterized function ${G_{\boldsymbol{\theta}}}$ that captures the designated task mapping relationship. Such a function can serve as a robust initialization for various target SPTs, enabling them to converge to the corresponding optimal solution ${f_{{\boldsymbol{\theta}}_i}}$ using limited training dataset and a small number of gradient updates. In the subsequent section, we will illustrate the application of MetaL algorithms to realize this goal.


% Figure environment removed 

% Figure environment removed 

% Figure environment removed 

\subsection{Meta-Processing algorithm}
MetaL provides a different framework in machine learning by enhancing the learning algorithm itself through multiple learning episodes across a distribution of related tasks \cite{finn2017model}. Inspired by this concept, we propose a Meta-Processing algorithm (see Algorithm 1) to provide a good initial NN model for various SPTs. In the following, we briefly introduce the algorithm. 

First, we need to split the collected training data into a support data set and a query data set for each task. The support and query data sets are similar to training and test sets in the context of conventional supervised learning (SL), but there are some important differences. In conventional SL, the training data set is used to optimize the model for a specific objective, while the test data set is used to evaluate the performance of the trained model on new, unseen data. In contrast, here, the support data set is used to train the model on the task-specific objective during the inner loop optimization, while the query data set is used to evaluate the performance of the adapted model after the inner loop optimization, i.e., in the outer loop optimization, which includes all the tasks. 

Following that, we can start the meta-training stage of the Meta-Processing algorithm. Let's use a parameterized function ${G_{\boldsymbol{\theta}}}$ to represent the NN with learned parameters ${\boldsymbol{\theta}}$. First, we randomly initialize the NN parameters ${\boldsymbol{\theta}}$. Then, we sample an SPT  ${\mathcal{T}_i}$ (like denoising, interpolation, and image enhancement), which includes limited support and query data sets, for a given task set ${p \left( \mathcal{T} \right)}$. For each sample task, we first perform a few iterations of gradient descent, namely the inner loop, to optimize the Meta-Processing model parameters. That is, we evaluate the performance of the NN on the support data set from the sampled task, and use gradient descent over single or multiple iterations to obtain the updated network parameters ${\boldsymbol{\theta}}^{'}$ as follows:
\begin{equation}\label{eq2}
\setlength{\abovedisplayskip}{3pt}
\setlength{\belowdisplayskip}{3pt}
\boldsymbol{\theta}_i^{'} = {\boldsymbol{\theta}} - lr_{inner} \cdot \nabla_{\boldsymbol{\theta}} \mathcal{L}_{\mathcal{T}_i} \left( G_{\boldsymbol{\theta}} \right),
\end{equation}
where ${lr_{inner}}$ denotes the learning rate for inner iterations as a hyperparameter, and ${\mathcal{L}}$ is the loss function. It should be emphasized that the inner loop is performed for each task separately, using a copy of the Meta-Processing model that is initialized with the current parameters of the Meta-Processing model. In essence, we are only updating the parameters of a copy of the meta-model, and not the meta-model itself. Specifically, we calculate the gradients based on the loss, and employ these gradients to update the model parameters using Equation (\ref{eq2}). Here, we do not use backpropagation to update the meta-model. 

Subsequently, the updated network parameters ${{\boldsymbol{\theta}}_i^{'}}$ are assessed on the query data set to calculate the corresponding loss value $\mathcal{L}_{\mathcal{T}_i}( G_{{\boldsymbol{\theta}}_i^{'}})$, i.e., the outer loop. This marks the completion of the training process for one task. As we repeat this process for all other tasks, we accumulate the losses evaluated on the query data sets of all tasks. Here, the accumulated loss 
\begin{equation}\label{eq3}
\setlength{\abovedisplayskip}{3pt}
\setlength{\belowdisplayskip}{3pt}
{\boldsymbol{\theta}} \leftarrow {\boldsymbol{\theta}} - lr_{meta} \cdot \nabla_{\boldsymbol{\theta}} \sum_{\mathcal{T}_{i} \sim p (\mathcal{T})}^{} \mathcal{L}_{\mathcal{T}_i} ( G_{{\boldsymbol{\theta}}_i^{'}}),
\end{equation}
is finally used to update the desired Meta-Processing model parameters ${\boldsymbol{\theta}}$, where ${lr_{meta}}$ denotes the meta (outer iterations) learning rate. The aforementioned steps constitute the complete process of one epoch of training in the meta-training stage. As seen, the key difference between meta-training and conventional SL is that in meta-training, the model is trained to learn a good initialization that can be quickly adapted to new tasks, whereas the idea of conventional SL is to ensure the trained model can provide the accurate predictions on new, unseen data. 

After completing meta-training, we will perform the meta-testing stage to fine-tune the meta-based initialization model on each task, which is also trained with new limited data, and evaluate the model's convergence speed for each task, as well as its prediction accuracy on the corresponding test sets. The meta-testing stage follows a similar procedure as in conventional SL, with the only difference being that we provide the NN with a better and more robust initialization that can adapt to various SPTs. 

\begin{algorithm}
\caption{Meta-Processing}\label{alg:Framwork}
\textbf{Input:} ${p(\mathcal{T})}$: Different seismic processing tasks with the corresponding support and query datasets. \\
\textbf{Input:} ${lr_{inner}, lr_{meta}}$: Learning rate for inner and outer loops, respectively. \\
\textbf{Input:} ${iter}$: The number of iterations in the support dataset for every task. \\
\textbf{--------------------------------------- Meta-training stage ------------------------------------} \\
\textbf{Output:} Meta-based initialization of the NN model 
\begin{algorithmic}
\State 1: Randomly initialize network parameters ${\boldsymbol{\theta}}$
\State 2: \textbf{while} all tasks ${p(\mathcal{T})}$ \textbf{do}
\State 3: \quad Sample batch of tasks ${\mathcal{T}_i \sim p ( \mathcal{T})}$
\State 4: \quad \textbf{for} every $\mathcal{T}_i$ \textbf{do}
\State 5: \quad \quad \textbf{for} ${i}$ \textbf{in} ${iter}$ \textbf{do}
\State 6: \quad \quad \quad Evaluate $\nabla_{\boldsymbol{\theta}} \mathcal{L}_{\mathcal{T}_i} \left( G_{\boldsymbol{\theta}} \right)$ with respect to the support dataset for the sample task $\mathcal{T}_i$
\State 7: \quad \quad \quad Compute adapted parameters with gradient descent: \\
\quad \quad \quad \quad \quad \quad \quad \quad \quad ${\boldsymbol{\theta}}_i^{'} = {\boldsymbol{\theta}} - lr_{inner} \cdot \nabla_{\boldsymbol{\theta}} \mathcal{L}_{\mathcal{T}_i} \left( G_{\boldsymbol{{\boldsymbol{\theta}}}} \right)$
\State 8: \quad \quad \textbf{end for}
\State 9: \quad \quad Evaluate $ \mathcal{L}_{\mathcal{T}_i}( G_{{\boldsymbol{\theta}}_i^{'}})$ with respect to the query dataset from the sample task $\mathcal{T}_i$
\State 10: \quad \textbf{end for}
\State 11: \quad Sum the loss of all tasks on the query dataset: $\mathcal{L}_{sum} = \sum_{\mathcal{T}_i \sim p \left( \mathcal{T} \right)} \mathcal{L}_{\mathcal{T}_i} ( G_{{\boldsymbol{\theta}}_i^{'}})$
\State 12: \quad Update the Meta-Processing ${\boldsymbol{\theta}} \leftarrow {\boldsymbol{\theta}} - lr_{meta} \cdot \nabla_{\boldsymbol{\theta}} \mathcal{L}_{sum}$
\State 13: \textbf{end while}
\State 14: \textbf{Return:} Meta-Processing parameters ${\boldsymbol{\theta}}$
\end{algorithmic}
\textbf{--------------------------------------- Meta-testing stage ------------------------------------}\\
\textbf{Output:} Task-specific NN model 
\begin{algorithmic}
\State 15: Fine-tune the Meta-Processing parameters ${\boldsymbol{\theta}}$ on each specific task
\State 16: Testing the updated model to obtain the seismic processing results
\end{algorithmic}
\end{algorithm}

\subsection{Data set establishment}
As previously stated, unlike conventional SL, our algorithm requires a support data set and a query data set to be provided during the meta-training stage, and a training data set and a test data set to be utilized during the meta-testing stage. Therefore, for each of the five tasks specified in this study, we generate 200 pairs of input-label data for both the support and query data sets. Likewise, for the meta-testing stage, we also collect 200 pairs of input-label data for each task. It is worth emphasizing that all the training data used in our experiments are synthetic, and the size of the dataset is deliberately limited. This is due to the significant challenge of acquiring labeled data in real-world scenarios.  We will evaluate the effectiveness of our algorithm on both synthetic and field data to assess its performance in practical settings. 

\subsection{Network architecture}
The UNet is a type of convolutional neural network architecture commonly applied in the field of seismology, and has demonstrated excellent performance in numerous SPTs. Here, we also adopt the UNet network architecture, as shown in Figure \ref{fig4}a. The UNet architecture consists of a contracting path and an expanding path. The contracting path is a series of encoders ($E_1$, $E_2$, $E_3$, $E_4$, and $E_5$) and pooling layers that extract high-level features from the input seismic data, while reducing its resolution. The expanding path is a series of decoders ($D_1$, $D_2$, $D_3$, and $D_4$) and upsampling layers that reconstruct the original resolution. Also, the architecture includes skip connections to connect corresponding layers in the contracting and expanding paths, which allow the network to reconstruct detailed structures that might be lost in the down-sampling process.  

To further improve UNet's performance, we utilize a modified residual network baseline (MRNB) to replace the conventional convolutional layers in the encoder and decoder. We present the structure of MRNB in Figure \ref{fig4}b. As we can see, MRNB consists of two residual blocks that combine Layer Normalization (LayerNorm), convolutional layers (1x1 and 3x3 conv), a simplified channel attention (SCA) module, and nonlinear activation functions LeakyReLU. In the first residual block, the input data are first normalized using Layer Normalization and then processed through a 1x1 convolutional layer to double the number of channels. Next, a 3x3 convolutional layer is applied to extract features from the input data, followed by a nonlinear activation function LeakyReLU, which introduces nonlinearity into the network. An SCA module is then applied to the output of the LeakyReLU layer to perform channel-wise feature recalibration. Finally, another 1x1 convolutional layer is applied to restore the number of channels to the original input. Each channel of the resulting output is multiplied by a corresponding coefficient, which is updated during the network's training process, and then added to the input data as the input to the second residual block. The second residual block is similar to the first, but without the 3x3 convolutional layer and SCA module. This is done to reduce the number of trainable network parameters while maintaining the depth in the network. Here, the encoders $E_1$, $E_2$, $E_3$, $E_4$, and $E_5$ include 2, 2, 4, 8, and 12 MRNBs, respectively, with the corresponding number of feature maps are 64, 128, 256, 512, and 1024, respectively. The decoders $D_1$, $D_2$, $D_3$, and $D_4$ all utilize 2 MRNBs corresponding to 64, 128, 256, and 512 feature maps, respectively. 

% Figure environment removed 

\subsection{Loss functions}
Loss functions are a crucial component of NN training as they provide a measure of how well the network is performing on a particular task. Selecting the appropriate loss function can help improve the model's accuracy, convergence speed, and generalization ability. Hence, for both the meta-training and meta-testing stages, we combine the mean square error (MSE) and multiscale structure similarity index measure (MS-SSIM) to optimized the NN training. In which, the MSE loss is a common metric for evaluating the regression models' performance and can be expressed as: 
\begin{equation}\label{eq4}
\setlength{\abovedisplayskip}{3pt}
\setlength{\belowdisplayskip}{3pt}
\begin{gathered}
\mathcal{L}_{\rm{MSE}}\left (L,O \right)=\frac{1}{N}\displaystyle \sum^{N}_{i=1}{\left|L_{i}-O_{i} \right|^2},
\end{gathered}
\end{equation}
where $L$ and $O$ represent the label and the output of the network, respectively, and $N$ is the total number of samples. 

MS-SSIM is a sophisticated metric that assesses the degree of structural similarity between the prediction and label by considering its multiscale nature from a visual perspective \cite{du2022deep, geng2022deep}. The calculation process of MS-SSIM involves multiple steps, beginning with the division of the input data into non-overlapping patches at various scales using a Gaussian filter. Subsequently, the local luminance $l(\cdot)$, contrast $c(\cdot)$, and structural information $s(\cdot)$ at different scales are computed using the following equation
\begin{equation}\label{eq5}
\setlength{\abovedisplayskip}{5pt}
\setlength{\belowdisplayskip}{5pt}
\left\{\begin{array}{c}
l(L, O)=\frac{2 \mu_L \mu_O+C_1}{\mu_L^2+\mu_O^2+C_1} 
\\[8pt]
c(L, O)=\frac{2 \sigma_L \sigma_O+C_2}{\sigma_L^2+\sigma_O^2+C_2} 
\\[8pt]
\setlength{\belowdisplayskip}{5pt}
s(L, O)=\frac{\sigma_{L O}+C_3}{\sigma_L \sigma_O+C_3}
\end{array}\right.,
\end{equation}
where $\mu_L$ and $\mu_O$ denote the mean values of the local patch corresponding to the label and output, respectively, $\sigma_L$ and $\sigma_O$ are the variance of the local patch, and $\sigma_{L O}$ represents the covariance of local patch. $C_1$, $C_2$, and $C_3$ are small constants to stabilize the division, where $C_2=C_3$. The MS-SSIM loss is then calculated by comparing the local luminance, contrast, and structural information at different scales, employing a weighted average exponents $\alpha_M$, $\beta_j$, and $\gamma_j$ across scales as follows
\begin{equation}\label{eq6}
\setlength{\abovedisplayskip}{5pt}
\setlength{\belowdisplayskip}{5pt}
\begin{split}
\mathcal{L}_{\mathrm{MS\mbox{-}SSIM}}\left (L,O \right) = 1 -
\left [l_{M} \left (L,O \right) \right]^{\alpha_{M}} \cdot \displaystyle \prod^{M}_{j=1} \left [c_{M} \left (L,O \right) \right]^{\beta_{j}} \left [s_{M} \left (L,O \right) \right]^{\gamma_{j}}.
\end{split}
\end{equation}
Here, referring to Wang et al. \cite{wang2003multiscale}, we set five scales, and the exponents are $\beta_1=\gamma_1=0.0448$, $\beta_2=\gamma_2=0.2856$, $\beta_3=\gamma_3=0.3001$, $\beta_4=\gamma_4=0.2363$, and $\alpha_5=\beta_5=\gamma_5=0.1333$. 

Utilizing MSE and MS-SSIM losses, the total loss function is defined as
\begin{equation}\label{eq7}
\setlength{\abovedisplayskip}{3pt}
\setlength{\belowdisplayskip}{3pt}
\begin{gathered}
\mathcal{L}=c \cdot (\epsilon_1 \cdot \mathcal{L}_{\rm{MSE}}+\epsilon_2 \cdot \mathcal{L}_{\mathrm{MS\mbox{-}SSIM}}).
\end{gathered}
\end{equation}
where the hyperparameters $\epsilon_1$ and $\epsilon_2$ are used to balance the two losses. Here, for simplification, both $\epsilon_1$ and $\epsilon_2$ are set to 1. $c$ represents a scaling factor that is employed to adjust the magnitude of the loss value. During the meta-training stage, large loss values can trigger instability issues in the training process, so we, in this paper, set $c=0.1$ as a measure to prevent optimization failures. However, in the meta-testing stage, the network initialization provided by the meta-training stage is already sufficiently robust, and hence $c$ is set to 1.  \\
\section{Numerical examples}
In our work, the meta-training stage is executed for 40000 epochs utilizing the AdamW optimizer, where the initial values of the meta-learning rate and inner-loop learning rate are set to 1e-3 and 5e-3, respectively. The meta-learning rate is gradually reduced by a factor of 0.8 every 2000 epochs, while the inner-loop learning rate undergoes a similar reduction for the first 20000 epochs, remaining constant thereafter. During the meta-testing stage, the optimized network parameters from the meta-training stage are exposed to individual fine-tuning for each SPT, with a total of 300 epochs executed and an initial learning rate of 1e-4. Every fine-tuning process consists of a total of 300 epochs with an initial learning rate of 1e-4. We need to emphasize, however, that such time-intensive training is definitely unnecessary to the task initialized from Meta-Processing; rather, we do it for the sole purpose of facilitating better comparisons with randomly initialized network. \\

We now present the results from our approach on both synthetic and field data. To validate the effectiveness of the Meta-Processing algorithm, we compare its prediction results with those of randomly initialized networks. Our assessment begins with a thorough evaluation of the performance of our approach to synthetic data. Subsequently, we present the results of further testing conducted on field data. \\


\subsection{Synthetic data}

\subsubsection{Denoising}
In the first example, we focus on removing random noise, which is the most common type of noise. To compare the convergence speed and accuracy of the meta-learning initialization-based network (MLIN) and the randomly initialized network (RIN) during the fine-tuning stage, MSE and MSSSIM losses are utilized as evaluation metrics. The metrics are plotted in Figures \ref{fig5}a and \ref{fig5}b, where the epochs marked with a star indicate the number of epochs of training required for the MLIN to achieve the same metric as the RIN. Same notation will be used later. We can see that, the MLIN achieves significantly smaller MSE loss than the RIN after only one epoch of gradient descent updates, which is far less than the 300 epochs required by the RIN. Moreover, from the perspective of the MSSSIM loss, MLIN surpasses the performance of the RIN after only 13 epochs of optimization. These demonstrate that the MLIN outperformed the RIN in terms of both convergence speed and accuracy. \\

The prediction results of MLIN and RIN for unseen test data are shown in Figure \ref{fig7}, with their corresponding input and label data depicted in Figure \ref{fig6}. In Figure \ref{fig7}, the first, second, and third rows correspond to RIN with 10, 100, and 300 epochs training, respectively, while the fourth row corresponds to MLIN with 10 epoch training. As we can see, the proposed Meta-Processing algorithm leads to an ideal denoising performance after only 10 epochs of optimization, achieving an MSE of 1.98e-6 and an MSSSIM of 9.39e-5. In contrast, the RIN with 10 epochs of optimization shows almost no denoising ability, which is attributed to the lengthy optimization process required by RIN. Even with 300 epochs of training, the denoising results of RIN only reach an MSE of 2.93e-5 and an MSSSIM of 8.38e-5. \\

% Figure environment removed 

% Figure environment removed 

% Figure environment removed 

\subsubsection{Interpolation}
Next, we present the fine-tuning training of MLIN for a seismic interpolation task and its prediction results on the test data set, which are also compared with RIN to demonstrate its performance. Figures \ref{fig8}a and \ref{fig8}b show the MSE and MSSSIM loss curves as a function of the training epochs, respectively. Likewise, compared to RIN, MLIN exhibits a significant superiority in terms of convergence speed and accuracy. Interpolation comparisons of MLIN and RIN to test data set are displayed in Figure \ref{fig9}. The results clearly demonstrate that our method provides a better interpolation performance and less energy leakage compared to the full data, even for RINs trained with 300 epochs. Numerically, our method achieves an MSE of 2.99e-7 and an MSSSIM of 3.70e-5 with 10 epoch training, while the RIN trained for 300 iterations only holds an MSE of 4.78e-5 and an MSSSIM of 9.75e-5. \\

% Figure environment removed 

% Figure environment removed


\subsubsection{Ground roll attenuation}
Ground roll noise is a type of seismic noise that can severely affect the quality of seismic inversion and imaging. This type of noise is often prevalent in land data, but also present in ocean bottom recording. It is a low-frequency in nature and propagates horizontally near the surface with shear wave velocity speed and can easily overwhelm reflections. In this task, we will try to use the Meta-Processing algorithm to effectively attenuate Ground roll. \\

Figure \ref{fig10} shows the MSE and MSSSIM loss curves of MLIN and RIN with 300 epochs of training. Once again, it verifies that the Meta-Processing algorithm can enable NNs have faster convergence speed and higher accuracy. For the test data set, our method achieves an MSE of 1.25e-4 and an MSSSIM of 8.4e-3 with 10 epochs of training, however, the RIN trained for 300 epochs only to achieve an MSE of 1.84e-4 and an MSSSIM of 8.9e-3. The results are shown in Figure \ref{fig12}, where the first, second, and third rows come from RIN with 10, 100, and 300 epochs of training, respectively, while the fourth row corresponds to MLIN with 10 epochs of training. The corresponding input (noisy) and label (clean) data are displayed in Figure \ref{fig11}. We can see that our method can effectively remove ground roll noise while preserving the signal with only 10 epochs of gradient descent updates. In contrast, the performance of RIN converges slowly as the number of epochs increases, as it is not easy to find the optimal solution from a random set of network parameters. \\

% Figure environment removed 

% Figure environment removed 

% Figure environment removed 


\subsubsection{Imaging enhancement}
Seismic imaging is a widely used method for exploring the subsurface structures of the Earth. Several factors can influence the quality of seismic imaging. Specifically, the geometry and spacing of the geophones used during data acquisition can affect the resolution and accuracy of the resulting images. For example, to apply large-scale deepwater surveys in crustal and oceanographic research, ocean bottom node (OBN) seismic acquisition systems typically adopt extremely sparse node spacing to reduce the cost and time consumption in acquisition. The sparse recording, however, leads to poor illumination and reduced continuity of events, posing a huge challenge to imaging. \\

In this task, to overcome the challenges brought by the sparse acquisition system in OBN surveys, we propose to train an NN that can map the images from sparse acquisition to dense acquisition. The trained network is expected to directly process the sparse images, improving the continuity and eliminating artifacts. Here, we use a synthetic model of the South China Sea to generate the seismic data. The model consists of a water layer, a series of thin flat transitional layers, and a series of sedimentary rock layers. Dense seismic data are obtained using finite-difference forward modeling with a grid spacing of 3.0 meters vertically and 3.1 meters horizontally, while sparse seismic data are obtained by subsambling the dense data, resulting in an OBN spacing of 310 meters. We employ the common-receiver Gaussian beam migration method \cite{shi2023elastic, cheng2023elastic, cheng2023seismic} to generate the images for training and testing. \\

Figure \ref{fig13} depicts the MSE and MSSSIM loss curves for the task of imaging enhancement trained by the MLIN and RIN for 300 epochs. Remarkably, after just one epoch of optimization, both MSE and MSSSIM losses of the MLIN are significantly lower than that of the RIN, which undergoes 300 epochs of gradient descent update. This demonstrates that our method, after the Meta-training step, results in significant convergence speed up and accuracy in the imaging enhancement task. \\

We further utilize the trained MLIN and RIN to predict the unseen test data, and the results are displayed in Figure \ref{fig14}. As we can see, the original image (see Figure \ref{fig14}a) suffers from poor continuity and is plagued by noise due to the sparse acquisition. In contrast, the image from dense acquisition (see Figure \ref{fig14}b) has high imaging quality. As demonstrated in Figure \ref{fig14}c, the image produced by the MLIN with 10 epochs of training, which achieves an MSE of 2.2e-3 and an MSSSIM of 5.61e-2, exhibits a significant improvement in both events continuity and noise attenuation. These improvements are crucial for accurately interpreting the subsurface structure. However, RIN does not bring noticeable imaging enhancements (see Figure \ref{fig14}d-f). Instead, it disrupts some event continuity in deeper layers. Even after undergoing 300 epochs of gradient descent updates, RIN only reach an MSE of 4.1e-3 and an MSSSIM of 6.7e-2. \\


% Figure environment removed 

% Figure environment removed 

\subsubsection{Velocity estimation}
Finally, we evaluate the performance of the Meta-Processing algorithm in a velocity estimation task. Specifically, for each shot gather, we will utilize the trained NNs to predict a root-mean-square velocity (referred to as $V_{rms}$), which is usually measured directly from the seismic data and is often used for normal moveout (NMO) correction. That is, the input to NNs is a single shot gather, and the output is the $V_{rms}$. However, it is almost impossible to directly predict the $V_{rms}$ of the entire model size from a single shot gather. Therefore, we follow \cite{harsuko2022storseismic} suggestions and extract the predicted $V_{rms}$ laterally from the shot position to half the maximum offset as our result, which is more reliable. Here, we need to emphasize that, in order to guarantee the same dimensions of input and output, we refer \cite{ovcharenko2022multi} approach of stretching each model along the depth axis to match the size of the shot-gather data along the temporal dimension. This operation allows the NN architecture to be extended to arbitrary model depths. Furthermore, we randomly sample noise from field data (as we will see later), and inject noise into the synthetic data, since we hope the trained NNs can be better generalize to field data testing. \\

Figure \ref{fig15} illustrates the MSE and MSSSIM loss curves of the MLIN and RIN trained for 300 epochs in the velocity estimation task. It is evident that our method-driven MLIN results in superior performance over the RIN in terms of convergence speed and accuracy. This outcome is of great significance for practical applications, as we can fine-tune the MLIN with minimal time investment to obtain a reasonably accurate $V_{rms}$, which can be quickly applied to other SPTs such as NMO correction. \\

The prediction results of the MLIN and RIN on the test set are presented in Figure \ref{fig16}, where panels (a) and (b) are the input and ground truth, respectively, (c) comes from the prediction result of the MLIN after 20 epochs of training, while (d), (e), and (f) correspond to the prediction results of the RIN trained for 20, 100, and 300 epochs, respectively. We can observe that the MLIN, which only requires 20 epochs of optimization, achieves very close prediction results to the ground truth, with a mean absolute error (MAE) of 54.5 m/s. However, the prediction results of the RIN trained for 20 epochs have large errors and contain a lot of signal artifacts. Although the accuracy of the RIN gradually improves with the increase in the number of epochs, there are still signal artifacts and noise, as seen in Figures \ref{fig16}e-f. Even with 300 epochs of gradient descent updates, the RIN only achieves an MAE of 68.5 m/s, which is still higher than the error of MLIN trained for only 20 epochs. \\


% Figure environment removed 

% Figure environment removed 

\subsection{Field data}
In the following, we will utilize the MLIN and RIN, previously trained on synthetic data, to directly predict field data, including seismic denoising, imaging enhancement, and velocity estimation tasks. To improve the networks' generalization capability to the field data, we normalize the amplitude of the field data using mean amplitude normalization instead of the traditional maximum value normalization method. Visually, this means that the normalized field data and the synthetic data have similar amplitudes for most events within the same display range. This step helps in bringing the feature distribution of the field data closer to that of the training synthetic data set. \\

\subsubsection{Denoising}
The MLIN  (MetaL initialization) and RIN (random initialization) are used to denoise a China land dataset, which is known to be polluted by random noise. The resulting denoised results are  displayed in Figure \ref{fig17}. The Meta-Processing algorithm enables the MLIN remove more noise and preserve more effective signals on field data with only 10 epochs of training, as demonstrated by the differences between the original and denoised datasets. On the other hand, RIN requires a large number of epochs of optimization to achieve a slight improvement in denoising performance. Furthermore, some artifacts are introduced in the prediction results of RIN after 100 and 300 epochs of training (see Figure \ref{fig17}f,h), severely contaminating the signal. \\


% Figure environment removed 


\subsubsection{Imaging enhancement}
Then, we assess the performance of the MLIN and RIN on the images acquired from real, sparse OBN surveys. The field data are collected from South China Sea at about 1100 meters water depth, using only five OBNs, with a node spacing of approximately 400 meters. A total of 154 shots are acquired with a spacing of 25 meters. Similar to the synthetic data, we employ common-receiver Gaussian beam migration to generate the input images. The corresponding imaging results, as well as the prediction results of the MLIN and RIN, are displayed in Figure \ref{fig18}. Figure \ref{fig18}a reveals that the sparse acquisition system leads to footprints and arc-shaped artifacts in the images, which considerably reduces image quality. Consequently, the proposed Meta-Processing algorithm helps the NN to overcome these issues effectively with only 10 epochs of optimizations on synthetic data, thereby enhancing the events' continuity and contributing to a significant enhancement in image quality (see Figure \ref{fig18}b). While the RIN removes some of the arc-shaped artifacts, it does not improve the events continuity (see Figure \ref{fig18}c-e). Moreover, it introduces noise and significantly degrades the image resolution, which is unacceptable. \\

% Figure environment removed 

\subsubsection{Velocity estimation}
We present the velocity estimation results of the MLIN and RIN on field data. The data was acquired in North West Australia by a streamer containing 324 hydrophones with a 25 m spacing that recorded 1824 shots, with an example of a shot gather shown in Figure \ref{fig19}. We apply the same procedure as for synthetic data to the field data. In Figure \ref{fig19}, panel (a) shows the input single-shot gather, (b) is the prediction result of the MLIN with 20 epochs of training, and (c), (d), and (e) display the prediction results of the RIN after training for 20, 100, and 300 epochs, respectively. As shown in Figure \ref{fig19}b, the MLIN yields a reasonable velocity estimation. However, as seen in the synthetic data, the RIN produces signal footprints and artifacts in the prediction results (see Figure \ref{fig19}c-e), which are obviously unsuitable to be used as an effective velocity model for guiding the next SPTs. To verify the accuracy of the velocity estimated by the MLIN, we compare the predicted velocity, obtained by averaging the predicted velocities of all traces, with the well velocity, as demonstrated in Figure \ref{fig20}. The predicted results are reasonably consistent with the well velocity, although some differences may exist. These differences may be due to the fact that the well velocity is usually lower than the seismic velocity, or field data may be affected by seismic anisotropy, which we do not consider in the synthetic data. \\

% Figure environment removed 

% Figure environment removed 
%



\section{Conclusion}
We proposed a unified paradigm for neural network-based seismic processing, Meta-Processing, to provide a powerful technique for various seismic processing tasks (SPTs), which can be efficiently trained on the limited training data. We utilized a modified residual network baseline to replace the conventional convolutional layers of the classic UNet network as our basic network architecture. Within the framework of the Meta-Processing, the network is performed in two stages: meta-training and meta-testing, each with distinct goals and implementation details. In the meta-training stage, a bilevel gradient descent updating from the support set to the query set is used to optimize the network parameters, aiming to obtain a robust initialization that can quickly adapt to various SPTs. In the meta-testing stage, the optimized network parameters from the meta-training stage are fine-tuned on various SPTs, with the same procedure to conventional supervised learning. The objective in this stage is to achieve rapid convergence and a significant improvement in prediction accuracy. \\

We conducted comprehensive numerical examples to demonstrate the performance of our method on various SPTs, including denoising, interpolation, ground-roll attenuation, imaging enhancement, and velocity estimation. Results on both synthetic and field data showed the superiority of our method in terms of convergence speed and prediction accuracy, as compared to randomly initialized UNet. Specifically, even when our method was only trained on synthetic data, it did not exhibit significant performance degradation when predicting field data. In contrast, the randomly initialized network, even after prolonged optimization, resulted in a significant decrease in accuracy when applied to field data. This means that the Meta-Processing algorithm not only extracts shared seismic features but also enhances its adaptability from the synthetic data domain to the field data domain. As a result, this study will help advance neural network-based seismic processing for various practical applications.



\section*{Open Research}
The code and data in this research are available at https://doi.org/10.5281/zenodo.8187745.


\acknowledgments
This publication is based on work supported by the King Abdullah University of Science and Technology (KAUST). The authors thank the DeepWave sponsors for supporting this research. They also thank Guangzhou Marine Geological Survey, Tongji University, and CGG for sharing the field seismic data.

\bibliography{references.bib}
\end{document}




