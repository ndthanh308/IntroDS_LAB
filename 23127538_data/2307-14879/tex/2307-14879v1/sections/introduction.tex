%!TEX root = ../main.tex

\section{Introduction}
\label{sec:intro}
The Internet has fundamentally changed the way conflicts unfold and are perceived on the global stage.
A crucial aspect is the growing role of social media with civilians sharing, publishing, and forwarding information, including sensitive strategic data.
The term \emph{hybrid warfare}~\cite{Ihybridwar} alludes to the increasing focus on information warfare, such as disinformation campaigns.
For example, allegedly, journalists were specifically and lethally targeted in both the second Chechen war and the Syrian civil war, to control the public narrative~\cite{Ikillmessenger}.
Today, individual citizens can leverage social media to reach a global audience, further escalating the dynamics of (dis-)information.

This new paradigm is particularly visible in the Ukrainian conflict that escalated in February 2022.
United Nations appointed independent rights experts warned that journalists in Ukraine are targeted and in danger~\cite{Iunjournalists}, with 15 journalists confirmed to have been killed in Ukraine in 2022~\cite{Ijournalistskilled}.
Another perspective on the conflict claims Ukraine and its citizens got the upper hand in the so-called \emph{social media war}~\cite{Isocialmediawar}.
Indeed, the U.S. government recognized the importance of a free flow of information among Ukrainians.
After spending millions of dollars to fund the widespread deployment of Starlink satellite Internet terminals and service in Ukraine~\cite{Iusgovpaysstarlink}, the service quickly reached over 150\,000 users shortly after deployment~\cite{I150kusers}.
While a large share of Starlink's usage in Ukraine seems to be military, the Starlink smartphone app was downloaded over 806\,000 from Ukraine, making it the most downloaded app at the time~\cite{Istarlinkapptop}.
As a response, Russia has started a barrage of cyber and jamming attacks on Starlink since~\cite{Istarlinkattacks}, yet they were repelled~\cite{Istarlinkupdate}.
While the precedent of Starlink's satellite Internet in a conflict is still ongoing at the time of writing, the E.U. already announced a deal to deploy its own satellite Internet system~\cite{Ieustarlink}.

Clearly, satellite-based Internet has a significant impact, as citizens and journalists gain the ability to freely share, e.g., sensitive information, evidence of war crimes, or timely warnings.
However, openly sharing information also carries great risks, and thus, many social media companies have added additional security measures to protect Ukrainian users, along with some guidelines to minimize risks~\cite{Istaysafe}.
Moreover, there is a specific danger when using satellite-based communication, especially when used to publish sensitive information.
As satellite signals can be monitored by virtually anyone in the sky and space above, satellite uplink communications can also be used to geolocate their users on the ground by triangulating their signals.
One example is the killings of two American journalists~\cite{Ijournaliststriked} and another a missile strike on the leader of the Chechen republic~\cite{Ichechenstrike}.
In both cases it is assumed that the attacks were possible by tracing satellite phones.
While official confirmation of such state-backed attacks is rare, it was shown that techniques to geolocate transmitters by satellites (target tracking) are practical~\cite{elgamoudi2021survey}.
After a security researcher gained traction with a tweet warning Ukrainian Starlink users potentially being geolocated and becoming targets~\cite{Itweet}, the CEO of Starlink's company issued a public warning to Ukrainian Starlink users~\cite{Imuskwarning}.

Considering the scale at which a satellite-based Internet can be monitored, and worse, individual users triangulated to get their physical position, it is an important aspect to protect citizens against this threat while preserving this novel and free flow of information during conflicts.
However, there are no works to properly address this issue in a practical manner.
On the one hand, using typical Internet encryption (i.e., TLS) on the communication channel is insufficient.
Eavesdropping on satellite communication targets the actual physical medium, whereas TLS is a high-level protocol not designed to provide anonymity.
The Tor network aims to fix this problem for the traditional Internet.
However, our case has a crucial difference, as satellite communications can be monitored by any satellite and, worse, triangulated to geolocate the user.
For example, numerous attacks on Tor assume an adversary can do \emph{entry point} monitoring~\cite{murdoch2007sampled,yang2017active}.
While typically an ambitious position for the adversary, with a satellite-based Internet, it becomes quite straightforward, as the connection is first sent to the satellite before reaching the Tor network.
Other attacks on Tor assume the adversary can monitor both the \emph{entry} and \emph{exit point}~\cite{bauer2009predicting,le2011one,palmieri2019distributed}.
However, if the adversary aims to prevent a user from publicly sharing information, it may anticipate and monitor popular social media sites, such as Twitter, for the exit point.

Prior works on \emph{location privacy} in mesh and wireless sensor networks have different shortcomings~\cite{kamat2005enhancing, xi2006preserving, shaikh2010achieving, li2009preserving, hong2005effective, kamat2009temporal, rios2011exploiting, el2010hyberloc, fan2009efficient, fan2010preventing, yang2013towards, mehta2007location, ouyang2006entrapping, kazatzopoulos2006ihide, wang2009privacy, shao2009cross, zhang2006arsa, sun2010sat, misra2006efficient, luo2010location}.
For example, some works have impractical assumptions for our purposes and others induce significant overheads.
There are also works that aim to establish a reliable network in case the existing infrastructure fails, called \emph{emergency networks}~\cite{portmann2008wireless,zhao2019uav,panda2019design,deruyck2018designing,pan2021uav,lin2021adaptive}.
These approaches are typically based on specialized hardware, such as vehicles equipped with bulky communication equipment and even flying vehicles, and custom network protocols to facilitate basic communication, making them quite impractical for our purposes.
There are also emergency networks working with satellites~\cite{zhou2021integrated,iapichino2008advanced,patricelli2009disaster}; yet, they do not consider protection against triangulation.
We will discuss the related work more thoroughly in \Cref{sec:rw}.
\\~\\
\indent
In summary, the remote monitoring and the possibility to triangulate satellite Internet users is a global threat to the new-found free flow of information by citizens.
To the best of our knowledge, there is no existing system that prevents geolocating satellite Internet users.
In this paper, we present \emph{\sysname} to close this gap.
\\~\\
\noindent\textbf{Goals \& Contributions:}
Our primary goal is to hide the geographic position of satellite Internet users in case their connection is being triangulated.
\sysname works by leveraging long-range wireless communication to span a simple network among satellite base stations.
Our system is agnostic with respect to the used wireless communication technology.
Leveraging this local network, a client's WAN connection is routed to another randomly selected satellite base station, which acts as a delegate to do the actual connection uplink to the satellite.
Further, the targeted satellite base station is regularly changed to avoid tracing back a long-lasting connection.

Our secondary goal is to focus on the accessibility of our system.
Therefore, \sysname is designed to work with cheap and simple devices, and thus, it can be deployed with widely available hardware and does not require impractical extensions on the user's device, such as a specific app or radio device.
Further, we aim at the usage of popular Internet services, like Twitter or WhatsApp, refraining from custom network protocols.
\sysname effectively protects users from being geolocated and becoming targeted.

\vspace{1em}
\noindent Our main contributions include:

\begin{itemize}
	\item \sysname is the first scheme to address the triangulation of satellite Internet users by rerouting connections to more distant satellite base stations.
	We introduce two security parameters that adjust the selection of routes to avoid geolocating a user over time.
	\item We derive requirements for wireless communication technologies and give an overview of possible candidates that can be used to enable \sysname.
	Similarly, due to our aim to design a practical system, we discuss numerous approaches, such that \sysname can access typical Internet services without needing to install custom software or hardware on the user's device.
	\item We implemented a proof of concept demonstrating the feasibility of \sysname, leveraging a cheap Raspberry Pi equipped with a LoRa shield for the local network.
	\item We further developed a large-scale simulation using real-world data sets to evaluate key aspects of \sysname in different environments, such as effective distances from the user or the use of more powerful wireless technologies.
\end{itemize}
