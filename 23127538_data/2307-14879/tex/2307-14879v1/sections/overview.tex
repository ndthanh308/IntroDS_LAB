%!TEX root = ../main.tex

\section{\sysname Design}
\label{sec:overview}
\label{sec:theory}
In this section, we focus on the general design of \sysname.
However, as our focus is designing a practical system (cf. \Cref{sec:requirements}), we will discuss essential technical aspects in \Cref{sec:tech}.
%
% Figure environment removed
%
\Cref{fig:overview} illustrates a simplified setup of \sysname with five gateways, each with a base station to connect to the Internet and a local radio transmitter to communicate with other gateways.
The client uses a smartphone to upload sensitive data to a gateway in close proximity, called the \emph{origin}.
For example, the client may try to publish an incriminating picture on Twitter.
Instead of directly forwarding the data over the satellite link, the gateway randomly selects an output gateway.
In \emph{Session 1} in the figure, the origin will then transmit the client's data over the local radio connection to the output gateway, which in turn will do the actual satellite transmission.
Thus, if the output gateway is identified and triangulated by \adv, the actual geographic position of the client stays hidden.
After a chosen amount of time, the origin gateway will select a new output gateway for this client's connection in \emph{Session 2}.
This new output gateway is only reachable via an intermediate gateway, and thus, establishes a 2-hop connection.
As seen in \emph{Session 3}, the origin will continue to change the client's output gateway regularly while the connection lasts.

\paragraph{\emph{Changing Gateways}}
While we assume \adv cannot eavesdrop on the entire local gateway network (cf. \Cref{sec:adv_model}), \adv may be able to intercept individual messages.
As a first step, all communication between the gateways is per-hop encrypted.
Thus, a forwarding gateway will receive an encrypted message, decrypt it, encrypt it with the key established with the next-hop gateway, and forward it.
However, if \adv intercepts enough messages in relation to a specific client, \adv may be able to trace the origin gateway, and thus, geolocate the client eventually.
Therefore, we introduce the security parameter \timeout, which defines a timeframe.
When a client starts a connection and the origin gateway selects a random output gateway, a timer is started.
After \timeout time, the origin will select a new output gateway.
Choosing a proper value for \timeout depends on the bandwidth and delay of the local gateway network.
Note that \timeout may also define a message count instead of a timeframe.
If \timeout is set too high, \adv has an increasing chance to trace back the origin.
Setting \timeout too low may lead to technical problems with the client's connection.
We will discuss this further in \Cref{sec:tech:routes}.
Note, \sysname's goal is not to make all traffic indistinguishable from unobjectionable traffic, as current approaches, such as Dummy Data Sources create non-negligible overhead (we discuss this in \Cref{sec:rw}).

\paragraph{\emph{Distance to Origin}}
The selection of the output gateways has some crucial implications as well.
If the origin only selects very close gateways, then the clients are not adequately protected.
If the output gateways are too far away, this would require many hops between the origin and the output gateways, negatively affecting the performance of \sysname.
Thus, we introduce the security parameter \maxhops, which defines how many hops away the output gateway shall be from the origin.
Increasing this parameter increases the average distance from the client (i.e., the origin gateway) to the output gateway, and thus, increasing the protection of the client.
However, increasing \maxhops will negatively affect the performance of the client's connection.

\paragraph{\emph{Output Selection Bias}}
While simply routing the client's traffic to output gateways \maxhops hops away is fine for individual short sessions, we need to consider a crucial aspect for longer sessions.
Suppose we have a uniformly spread network of gateways and a client that needs a long-lived connection to, e.g., upload many pictures.
In this case, many gateway changes happen over time and the selected gateways will eventually be selected in all directions from the origin.
Simply put, \adv can observe these changes and will be able to draw a circle containing all output gateways.
The gateway closest to the centroid of this circle is most likely the origin gateway; thus, endangering the client.
To counteract this effect, we additionally introduce a selection bias for the output gateway.
For each client, the origin gateway will generate a random \emph{direction} and \emph{weight} bias. 
When selecting a new output gateway, the origin will prefer random output gateways in the given direction and with the assigned weight.
In a practical context, the direction can simply be a bias for selecting the next neighbor gateway via an index without the need to consider the gateways' geolocations.
This way, the centroid of the mentioned circle shifts to a random direction, and thus, cannot be used to trace the origin.
The selected biases will be preserved for each client.

Additionally, this allows us to change the role of the \maxhops security parameter.
Instead of always selecting an output gateway exactly \maxhops hops away, we can select a range of hops between 0 and \maxhops.
This does not weaken the previously discussed selection bias.
However, the range improves the client's network performance on average, as some output gateways may be only one hop away and fewer hops mean better performance for the client.
Note that it is important for the origin to select itself as the output gateway.
Otherwise, \adv can simply identify the origin by checking which gateway never sends.
