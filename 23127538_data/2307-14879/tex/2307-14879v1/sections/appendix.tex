%!TEX root = ../main.tex
\appendices
\section{Distance To Origin Evaluation}\label{sec:app:dto}
The data sets with an assumed 5\,km maximum range between nodes in \Cref{fig:avgdist} (a) shows the limitations of the different data sets.
In contrast, very densely populated sets that do not cover a lot of area, like Adelaide or Paris, show no increase in distance.
This effect is primarily dependent on the set's covered area.
For example, if we roughly draw a circle around the nodes provided by each data set, we get a diameter of $\sim$5\,km for Adelaide and $\sim$15\,km for Paris, while Hong Kong has $\sim$50\,km and Rhein-Neckar even $\sim$100\,km.
Data sets, like New York City ($\sim$30\,km) or Brisbane ($\sim$20\,km) that cover a medium-sized area, show a diminishing return in terms of an increased \maxhops.
We also found that some sets have more uniformly spread nodes over the covered area, while others have a decreasing density from the center to the borders of the covered area.
For example, the more uniformly dense Paris reaches the limit sooner than the unevenly dense Brisbane.
With the Adelaide data set, we cannot get past 5\,km and when considering the random selection of nodes with a more dense center, our average distance from the origin is naturally quite low.

\Cref{fig:avgdist} (b) shows similar measurements for the 1\,km range case.
We can see the effects analogously to the 5\,km results, just with a significantly reduced distance from the origin.
Note that some data sets, especially Rhein-Neckar, have severely reduced node count when considering a connected network with a 1\,km range (cf. \Cref{sec:eval:sim:dataset}).
The effect of this can be seen in this graph.
Overall, there is an inherent trade-off for setting the \maxhops parameter, as setting it higher leads to an increased distance; yet, more hops for the communication will lead to more delays and overhead in the network.
Our measurements show that it is crucial to take the underlying spread and density of nodes into account when choosing \maxhops.
For the following simulation results, we assumed \maxhops = 3 for the 5\,km range networks and \maxhops = 5 for the 1\,km range networks.

\section{Result Graphs}\label{sec:app:rg}

\clearpage

% Figure environment removed

% Figure environment removed