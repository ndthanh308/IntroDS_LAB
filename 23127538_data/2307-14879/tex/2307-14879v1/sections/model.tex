%!TEX root = ../main.tex

\section{System Model}
\label{sec:system_model}

Our system model consists of the following entities:
	A \textbf{network} is a collection of \emph{gateways}, \emph{clients}, and satellites providing internet access.
    Each \textbf{gateway} is equipped with a base station, i.e., means of providing access to the Internet via satellite communication, a radio transmitter to connect to the \emph{local network}, and two WiFi access points to provide access to the \emph{WiFi Network}.    
    A \textbf{local network} is spanned among the \emph{gateways} via their radio transmitter capable of communicating with each other.
    A \textbf{WiFi network} is a direct connection between \emph{clients} and \emph{gateways} separated into two WiFi access points.
    One is a connection to the gateway's WAN, directly uplinking to the satellite internet.
    The other provides a secure WiFi connection via our \sysname system.    
    \textbf{Clients} are simple end-user devices, such as smartphones, which aim to establish a secured WAN connection to send sensitive data.
    For this, \emph{clients} simply connect to the secure \emph{WiFi Network} provided by a close-by \emph{gateway}.

Note, for simplicity, we assume all \emph{gateways} have a base station providing Internet access, even though in a real-world scenario simple relay nodes for the \emph{local network} may also be deployed.
We further assume that standard means of Internet access are impaired or even unavailable entirely, and thus clients need to rely on satellite Internet.

Furthermore, \emph{gateways} are equipped with certificates to identify each other and to establish secret key pairs to enable symmetric encryption between any two \emph{gateways}.
We assume the \emph{gateways} can trust each other's certificates.
For example, in the real world, this could be realized via exchanging certificates via direct contacts in combination with a Web-of-Trust approach.

\subsection{Adversary Model and Assumptions}
\label{sec:adv_model}
The adversary \adv has the goal of geolocating a specific client.
To do this, \adv has a range of satellites deployed, which can eavesdrop on the Internet communication between base stations and the receiving satellite.
We assume that \adv is able to correlate the communication data to identify a specific client.
Further, \adv is capable of triangulating the sending base station via the mentioned satellites, which was shown to be practical~\cite{elgamoudi2021survey}.
Thus, as the client has to use the closest base station via a close-range WiFi connection, \adv can infer the client's approximate geographic position.
However, as our system aims to reroute the client's connection to another gateway, simply triangulating the base station is not enough to geolocate the client.

We assume \adv is not capable of establishing a holistic view of the local network.
To achieve this, \adv would need to monitor a significant number of local network connections, which also requires prolonged physical proximity in a multitude of locations.
This contradicts the scenario we are targeting, i.e., an active conflict zone, as widespread deployment of eavesdropping devices is infeasible.
However, \adv is able to intercept individual messages sent between gateways.
This assumption is comparable to the Tor network, which can also be broken by an adversary with a global view of the network; yet, in practice, this is hard to achieve.
We further argue due to the nature of the limited range of each node (discussed in \Cref{sec:tech:lpwan}), the local network cannot be surveilled by satellites to attain a global view.

We will thoroughly discuss several possible local attacks in \Cref{sec:security}, including jamming attacks that deal with similar assumptions.
Nevertheless, our system focuses on preventing remote and globally applicable triangulation via satellites.

Further, we assume the WiFi connection between the client and gateway cannot be intercepted by \adv due to its close-range nature.
We also assume \adv aims to minimize any collateral damage, as this might have grave political repercussions~\cite{Isanctions,Iiccwarrant}.
Finally, we further assume \adv cannot forge digital signatures or break symmetric encryption.

\subsection{Requirements}
\label{sec:requirements}
To formalize the setting outlined in the Introduction, we aim to design a secure satellite Internet scheme with the following requirements:
\begin{enumerate}
	\item[R.1] \label{req1:baseloc}\emph{Prevent geolocating base station:} 
	The scheme shall prevent \adv from geolocating the client.
	More specifically, the gateway uplinking traffic to the WAN shall not indicate the geographic position of the client.
	For example, as a client has to use the closest gateway, if this gateway uplinks the client's traffic to the WAN, \adv may assume the client is very close.
	\item[R.2] \label{req2:localloc}\emph{Prevent local geolocation leakage}:
	In addition to requirement~\hyperref[req1:baseloc]{R.1}, \adv may intercept individual messages in the local network traffic and trace back the actually used gateway by the client.
	Thus, the scheme shall further prevent geolocation leakage in terms of the local network.
	\item[R.3] \label{req3:compatible}\emph{Internet compatibility:}
	The use of, e.g., simplified network protocols may greatly increase the performance of the scheme.
	However, this implies that most common Internet services are not accessible, and thus, the scheme shall be compatible with most Internet services.
	\item[R.4] \label{req4:minreq}\emph{Out-of-the-box for clients:} 
	From the client's point of view, the scheme shall impose minimal requirements on the client.
	For example, a client may need to unexpectedly and urgently send some sensitive data using a smartphone.
	In such a case, requiring the client to install an additional app or even an additional hardware device is impractical.
\end{enumerate}

