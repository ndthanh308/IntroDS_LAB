%!TEX root = ../main.tex

\section{Evaluation}
\label{sec:eval}
In this section, we evaluate \sysname.
On the one hand, we show real-world results of our prototype setup.
On the other hand, we show the large-scale performance in different settings based on a network simulator run on real-world data sets.

\subsection{Prototype}
\label{sec:eval:prototype}
Recall \Cref{sec:adv_model}, the goal of our approach is to hinder \adv to localize a client, e.g., publishing incriminating information. Usually, such information is published in the form of text or images.
Therefore, to measure the performance of our prototype, we considered three use cases for the client in our setup.
One is simply sending messages via the popular WhatsApp messenger.
Another was to send out tweets via the Twitter Lite app.
The third is to publish an image.
For accurate measurement numbers, we used pings as well as downloaded and uploaded small images via the client.

\paragraph{\emph{Round-Trip-Time (RTT)}} We measured the RTT using a standard ping via zero, one and two LoRa hops to a server on the internet via the Starlink connection. Note that using Starlink alone already adds around 50\,ms of delay. We send 100 pings and averaged the results. The results are depicted in \Cref{tab:ping-eval}.

\begin{table}[]
\centering
\caption{Average RTT and packet loss for pinging 8.8.8.8 with 100 64\,bytes ICMP.}
\begin{tabular}{crc}
\toprule
LoRa Hops &  \multicolumn{1}{c}{RTT}       & Packet Loss \\ 
\midrule
0         & 49.111\,ms   & 0\%         \\
1         & 157.938\,ms & 3\%         \\
2         & 211.786\,ms & 4\%         \\ 
\bottomrule
\end{tabular}
\label{tab:ping-eval}
\vspace{\tblvsp}
\end{table}

\paragraph{\emph{Upload \& Download}} We implemented our own API service to upload images using REST over TLS using HTTP/2. We used a POST form request to upload a picture of size between 50\,kB and 200\,kB. These numbers correspond to the lower and upper bounds of typical mobile applications image compression (e.g., WhatsApp), which we measured with common photographs. Similarly, we downloaded the files using \texttt{wget}. We repeated this experiment 10 times and averaged the results. The results are shown in \Cref{tab:data-eval}.
Note, while we employed both fragmentation for the LoRa packets and optimized the TCP settings, as discussed in \Cref{sec:impl}, we still observed a significant amount of spurious retransmissions.

Naturally, \sysname incurs an overhead. Securely uploading a high-resolution photo using WhatsApp (i.e., 150\,kB) takes 171.92\,s. However, we argue it is reasonable considering the alternative of being either localized or not publishing at all.

\begin{table}[]
\centering
\caption{Average time for uploading and downloading images of different sizes using 2 LoRa hops.}
\begin{tabular}{rrr}
\toprule
\multicolumn{1}{c}{Image Size} & \multicolumn{1}{c}{Upload}  & \multicolumn{1}{c}{Download} \\ 
\midrule
50\,kB  & 65.84\,s  & 60.40\,s   \\
100\,kB & 137.84\,s & 117.80\,s  \\
150\,kB & 171.92\,s & 223.20\,s  \\
200\,kB & 269.31\,s & 276.80\,s  \\ 
\bottomrule
\end{tabular}
\label{tab:data-eval}
\vspace{\tblvsp}
\end{table}

\subsection{Simulation}
\label{sec:eval:sim}
To measure the large-scale performance of \sysname, we implemented a network simulation.
In the following, we describe our evaluation setup by first describing how we simulated the local wireless network, presenting the used real-world data sets, and how the network simulation works.
Afterwards, we present our results showing how the security parameter \maxhops affects the distance from the position of a client, the delays to establish a TLS session in different settings, and finally, the practical data rates.

\subsubsection{Local Wireless Network}
\label{sec:eval:sim:lwn}
While we were restricted to Lora with the sub-GHz frequencies, the simulator allows us to simulate more powerful wireless technologies.
Informed by the data shown in \Cref{tbl:lpwans}, we selected the following combinations of assumed ranges and data rates between nodes:

\begin{enumerate}
    \item 5\,km @50\,kbps (Lora Sub-GHz)
    \item 5\,km @166\,kbps (DASH7)
    \item 1\,km @1\,Mbps (Lora 2.4GHz \& LTE-M Cat-M1)
    \item 1\,km @4\,Mbps (LTE-M Cat-M2)
\end{enumerate}

We excluded NB-IoT due to its low performance compared to the other technologies on our list, which is mostly due to its low-power requirement.
We further excluded Weightless-W, even though its impressive data rates, as we could not establish the readiness of the technology.
For example, unlike the other technologies, we could not find any purchasable devices equipped with Weightless-W components or any practical demonstrators.

Thus, for our simulation we assume, e.g., a range of 5\,km between nodes with a maximum data rate of 50\,kbps, simulating Sub-GHz Lora.
However, the maximum data rate is practically not achievable, especially at longer ranges.
While there are some practical measurements regarding this phenomenon, we found the used evaluation setups (e.g., obstructions or radio settings) and results vary significantly\footnote{For example, one work measured around 10\,kbps~\cite{petajajarvi2017performance} while another measured double the data rate~\cite{swain2021lora} for similar settings.}.
For our simulation, we approximate the \emph{log-distance path loss model} as a simple logarithmic function over the distance between nodes $r=e^{-2d}$. 
$d$ is the distance between two nodes as a relative distance $[0,1]$ with respect to the maximum range.
The resulting data rate $r$ is also relative $[0,1]$ to the maximum data rate.
Thus, the maximum data rate is only achievable if two nodes are right next to each other, while two nodes that are far away, e.g., close to the maximum range, have a severely reduced data rate with only a small fraction of the maximum data rate.

\subsubsection{Data Sets}
\label{sec:eval:sim:dataset}
As far as we are aware, there are no representative data for gateway positions for the settings we target.
Nevertheless, we found public data sets on various cities' WiFi hotspots to be a good approximation for an urban environment.
\Cref{tbl:datasets} lists all of the data sets we used, each containing the geographic positions of WiFi hotspots for cities of different sizes.
\Cref{fig:datasets} shows renderings of the Hong Kong, New York City, and Rhein-Neckar data sets.

\begin{table}[t]
    \centering
    \caption{Urban WiFi hotspot datasets used for simulation. \emph{Close} refers to the number of records after filtering out too close records. \emph{CC} refers to the number of records contained in the largest connected component when connecting the graph with the given range.}
    \label{tbl:datasets}
    \begin{tabular}{lrrrr}
        \toprule
        Data Set & Total & Close & CC 5\,km & CC 1\,km \\ 
        \midrule
        Hong Kong~\cite{gpshk} & 5441 & 874 & 866 & 332 \\
        New York City~\cite{gpsnyc} & 3319 & 765 & 753 & 602 \\
        Rhein-Neckar~\cite{gpsffrn} & 1338 & 551 & 530 & 30 \\ 
        Brisbane~\cite{gpsbrisbane} & 347 & 97 & 95 & 38 \\
        Paris~\cite{gpsparis} & 277 & 182 & 181 & 178 \\
        Adelaide~\cite{gpsadelaide} & 272 & 51 & 51 & 51 \\
        Leeds~\cite{gpsleeds} & 236 & 169 & 163 & 83 \\
        Linz~\cite{gpslinz} & 124 & 35 & 34 & 29 \\
        \bottomrule
    \end{tabular}
    \vspace{\tblvsp}
\end{table}

However, we had to filter these data sets to fit our needs for the simulation, due to the following two problems.
For one, these data sets contain points that are very close together, which is to be expected, e.g., in the city center, there will be a large number of shops or similar with a high density of hotspots.
As such a dense concentration of gateways is not representative in our settings, we filtered out nodes that are closer than 200\,m from each other.
In \Cref{tbl:datasets}, the number of nodes left after this filtering step is shown as \emph{Close}.
The second problem is that our assumed maximum range leads to a disconnected graph, as some sub-graphs may not be in range for another sub-graph.
Thus, we constructed a graph of nodes from the data sets with the respective maximum range and found the set of connected components in the overall graph.
Finally, we chose the largest connected component for each data set as the actual set of nodes for our simulation.
In \Cref{tbl:datasets}, this is shown as \emph{CC 5\,km} and \emph{CC 1\,km} for a range of 5\,km and 1\,km respectively.
Note, for some data sets the 1\,km range limitation creates a very small network, such as Rhein-Neckar, which covers a large area, but many of the nodes are further than 1\,km away from each other.
Therefore, this results in a significant reduction of the size, if the used connected network, e.g., Rhein-Neckar \emph{CC 1\,km} only has $\sim$5.4\% nodes left relative to \emph{Close}.
However, note that Rhein-Neckar is an exceptional outlier, due to the data set spanning relatively few nodes over almost 100\,km.
The other data sets with higher reductions comprise of dense clusters that are far from one another, e.g., see \Cref{fig:datasets} for Hong Kong.
In such a scenario, \sysname could be applied individually for each cluster.
Generally, a restriction true for all physical wireless networks is that a more densely packed network results in a holistically better connectivity among nodes~\cite{andrew2011computer}.
Therefore, we stress that, if there are no alternatives to using satellite Internet (cf. \Cref{sec:system_model}), even a suboptimal network setup benefits from our approach.

% Figure environment removed

\subsubsection{Network Simulator}
\label{sec:eval:sim:sim}
To implement the network simulation, we used the OMNeT++ 6.0 network simulator~\cite{omnet}.
We load the processed data sets, as described in \Cref{sec:eval:sim:dataset}, as the individual nodes into the simulation.
These nodes are then connected, if they are in range of each other, with a data rate calculated at initialization, as described in \cref{sec:eval:sim:lwn}.
We further use the provided routing component in OMNeT++ to route individual messages as well as to ensure \maxhops is satisfied when randomly selecting gateways by the clients.

For the actual simulation, we assign each client to a random gateway as its origin.
Each client will then proceed to execute the following steps:

\begin{enumerate}
    \item The client's gateway will randomly select an output gateway less than \maxhops hops away.
    \item The client will send a \texttt{TCP} \texttt{SYN} message out to simulate establishing a connection. This message is routed to the output gateway.
    \item After the output gateway received each message, we simulate a 100\,ms delay for the WAN server to answer\footnote{We based this number on the upper average of 50\,ms delay with Starlink and some additional processing time.}.
    \item The output gateway will send a TCP \texttt{SYN-ACK} message back to the client's gateway.
    \item The client will answer this with a TCP \texttt{ACK} and TLS \texttt{ClientHello} combined message, as is a common optimization practice of the Internet.
    \item The simulated server will answer this with a TLS \texttt{ServerHello} message; thus, establishing the TLS connection when the client receives it.
    \item The client will then start sending a 200\,kB data package (the upper bound for WhatsApp image compression) divided into multiple messages, i.e., according to the MTU sizes.
\end{enumerate}

The client will repeat these steps, selecting a new output gateway each time, until the simulation ends after a simulated hour.
Put simply, each client has a fixed origin and will constantly send images with changing output gateways.
For each parameter combination and data set, we execute 30 runs with different random seeds.
While quite a simple setup, this allows us to approximately measure TLS session delays, the effective data rates in the network, and the interactions of both, e.g., TLS session delays of clients over a gateway currently busy sending many data messages.

\subsubsection{Distance to Origin}
\label{sec:eval:sim:dist}
To evaluate the key security parameter \maxhops, we measured the average distance from the client's gateway (origin) to the output gateway.
For this, we employed a simplified simulation over our data sets (cf. \Cref{tbl:datasets}) to get a large sample size of $10\,000$.
For each sample, we select a random origin with a random output gateway less than \maxhops away and measured the actual distance from the origin via their geographic position.
Further, we executed this simulation with differently set \maxhops.
The results of this simulation are shown in \Cref{fig:avgdist}. Wide-ranging data sets in terms of the overall covered area by the nodes, like Hong Kong or Rhein-Neckar, show a nearly linear increase in distance with an increase of \maxhops. We discuss this more thoroughly in \Cref{sec:app:dto}.

% Figure environment removed

\subsubsection{TLS Session Delay}
\label{sec:eval:sim:delay}
To measure the performance of \sysname, we focus on two aspects: delays and data rate.
However, as the primary goal of \sysname is security, we focus on practical applications.
Namely, for delay we measure the average delay it takes to establish a TLS session.
As most Internet services today are based on these sessions, this is a more practical number than measuring simple round-trip delays, especially as we want to see the effects of concurrent data transfers and session establishments in the network.

\Cref{fig:sim_delay} shows our measurements.
Noticeable between the slower networks (a) \& (b) and the faster networks (c) \& (d) is the effect of \maxhops with many clients.
With \maxhops = 5 congestion of the gateways affects the delay significantly.
With a range of 5\,km (a) \& (b), the two data sets Adelaide and Linz show a much better performance, due to all nodes being packed closely together in a small area, which also results in much better data rates on average.
Note that there is an implicit tradeoff, as the low average distances between nodes naturally affects the \emph{distance to origin}, as shown in \Cref{sec:eval:sim:dist}.
A similar effect can be observed for the 1\,km range measurements (c) \& (d).
Data sets with closely packed nodes in general, like New York City, show much better performance than data sets with spread-out nodes, like Brisbane.

\subsubsection{Practical Data Rates}
\label{sec:eval:sim:datarate}
In a practical scenario, the data rates may depend on multiple TLS sessions first, which would heavily reduce effective the data rates when simply calculating overall sent bytes divided by time.
Thus, we decided to measure the time it takes to upload a 200\,kB sized image as a practical and isolated example.

\Cref{fig:sim_data} shows our measurements.
Generally, the observation made in \Cref{sec:eval:sim:delay} regarding the effect of the density of nodes on data rates applies here as well.
However, while the delays imply a logarithmic trend with a growing number of clients, we can clearly see the effects of many clients sending data and the resulting congestion in the network.
Particularly noticeable are the measurements for Paris with a 5\,km range.
The increase in the number of clients has an especially detrimental effect on the transmission speed.
We believe this is due to the unique and dense spread of nodes in the data set, creating some \emph{bottleneck} nodes that serve multiple connections simultaneously.
This effect disappears with a 1\,km range, as fewer nodes can connect to these bottleneck nodes.
