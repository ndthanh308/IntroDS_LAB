%!TEX root = ../main.tex

\section{Anonymity \& Security}
\label{sec:security}
In this section, we first analyze the practical anonymity guarantees of \sysname and the security of \sysname.


\subsection{Anonymity}
We quantify and evaluate the anonymity provided by \sysname on the data sets described in \Cref{sec:eval:sim:dataset}.
\Cref{tbl:reach2} summarizes our analysis for each data set for both the 5\,km and 1\,km cases, with  \maxhops = 3 and \maxhops = 5, respectively.
In the following, we will explain our metrics and evaluate our results.

The first metric we evaluated is the traditional \emph{Anonymity Set} size as defined by Chaum~\cite{chaum1988dining}.
The idea is that one's anonymity can be quantified by the set of parties from which one is not distinguishable.
Thus, \adv ideally does not know, which of the parties in the set is the actual target.
In our case, we counted all gateways reachable by an output gateway, i.e., less than \maxhops away, as these are the potential alternatives among the origin gateway.
We considered all gateways in our data sets for both the average and the gateway with the lowest number of reachable gateways in \Cref{tbl:reach2}.
While a dense data set like New York City has a high average, the less dense Rhein-Neckar set has a comparably low average.
When looking at the minima, i.e., the least connected gateways, some data sets contain remote nodes with few reachable nodes.

Nevertheless, just counting the possible alternatives is not a complete metric, as the probabilities to be the origin among the set of reachable gateways may not be uniform.
A way to model this is to consider the entropy of the anonymity set based on implicit information on the network and its nodes leveraged by \adv~\cite{serjantov2003towards}, e.g., the reachability of each node throughout the graph.
In our case, similar to our Anonymity Set metric, we look at all reachable gateways from an output gateway.
In addition, we consider the number of paths between each reachable gateway and the output gateway, as a gateway with more paths to the output gateway is more likely to be the origin than gateways with fewer.
With this entropy, we are able to calculate the \emph{Effective Set} size for the anonymity set (see \Cref{tbl:reach2}).
Concretely, we calculate per reachable gateway $g$ the number of paths to the output gateway divided by all possible paths from any reachable node to the output gateway as $p_g$ (c.f.~\cite{serjantov2003towards}).
With this, we can calculate the effective anonymity set for all $g$ in the reachable gateway set for an output gateway:
\begin{equation*}
    - \sum p_g \log_2 (p_g)
\end{equation*}
We considered all gateways in our data sets to get both the average and the minimum effective anonymity set.
Compared to the uniform anonymity set, data sets that are more clustered in terms of gateway distribution have a significantly lower effective set size (e.g., Rhein-Neckar 5\,km with a ${\sim}39\%$ reduction) than data sets that are well-connected (e.g., Paris 5\,km with a ${\sim}4\%$ reduction).
These numbers demonstrate that different data sets with different degrees of connectivity among the gateways result in varying degrees of anonymity, which needs to be considered in practical deployment.
However, the average effective anonymity set sizes show the efficacy of \sysname.

Finally, we measured the average number of paths between any two gateways as \emph{Node2node Paths} within \maxhops distance.
This gives a hint at the general connectivity among the nodes.
To further refine this metric, we also counted all \emph{Unique Paths}, i.e., the number of paths that do not share any common gateways in their route.
This indicates how reliable \sysname is when individual gateways fail, i.e., the number of alternative paths.
In \Cref{tbl:reach2}, we can examine that in more dense networks with many connections, like Paris 5\,km, there are many alternative paths on average.
Contrarily, if there are few connections between nodes, like Paris 1\,km, then there are few alternative paths on average.
Generally, a higher number shows a more resilient network against gateway failures.

\begin{table*}[t]
    \centering
    \caption{Anonymity metrics for all data sets for both the 5\,km and 1\,km cases.}
    \label{tbl:reach2}
    \newcommand\rot{45}
    \newcommand\ol{-26pt}
   \begin{tabular}{rrrrrrrrrrrrr}
         & \rotatebox{\rot}{\parbox{\widthof{Anonymity Set}}    {5\,km Average \\Anonymity Set}} \hspace{\ol} & 
         \rotatebox{\rot}{\parbox{\widthof{5\,km Minimum}}      {5\,km Minimum \\Anonymity Set}} \hspace{\ol} &
         \rotatebox{\rot}{\parbox{\widthof{5\,km Average}}      {5\,km Average \\Effective Set}} \hspace{\ol} &
         \rotatebox{\rot}{\parbox{\widthof{5\,km Minimum}}      {5\,km Minimum \\Effective Set}} \hspace{\ol} &
         \rotatebox{\rot}{\parbox{\widthof{Node2node Paths}}    {5\,km Average \\Node2node Paths}} \hspace{\ol} &
         \rotatebox{\rot}{\parbox{\widthof{5\,km Average}}      {5\,km Average \\Unique Paths}} \hspace{\ol} &
         \rotatebox{\rot}{\parbox{\widthof{Anonymity Set}}      {1\,km Average \\Anonymity Set}} \hspace{\ol} & 
         \rotatebox{\rot}{\parbox{\widthof{5\,km Minimum}}      {1\,km Minimum \\Anonymity Set}} \hspace{\ol} &
         \rotatebox{\rot}{\parbox{\widthof{5\,km Average}}      {1\,km Average \\Effective Set}} \hspace{\ol} &
         \rotatebox{\rot}{\parbox{\widthof{5\,km Minimum}}      {1\,km Minimum \\Effective Set}} \hspace{\ol} &
         \rotatebox{\rot}{\parbox{\widthof{Node2node Paths}}    {1\,km Average \\Node2node Paths}} \hspace{\ol} &
         \rotatebox{\rot}{\parbox{\widthof{5\,km Average}}      {1\,km Average \\Unique Paths}} \hspace{-6pt} \\
        \midrule
        Hong Kong       & 417.3 & 12 & 278.0 & 9.1 & 6\,080.3 & 54.2
                        & 111.8 & 26 & 60.1 & 21.8 & 5\,147.9 & 5.7 \hspace{18pt} \\
        New York City   & 523.7 & 13 & 330.0 & 12.6 & 6\,415.1 & 59.9
                        & 92.7 & 8 & 47.7 & 8.0 & 3\,734.4 & 4.2 \hspace{18pt} \\
        Rhein-Neckar    & 98.5 & 11 & 60.4 & 7.5 & 193.1 & 8.9
                        & 18.3 & 13 & 12.2 & 9.1 & 116.2 & 2.3 \hspace{18pt} \\
        Brisbane        & 83.7 & 17 & 61.9 & 14.9 & 862.9 & 20.2
                        & 35.3 & 30 & 26.7 & 20.8 & 116.5 & 4.3 \hspace{18pt} \\
        Paris           & 180.0 & 179 & 172.7 & 162.4 & 10\,096.9 & 93.8
                        & 78.4 & 12 & 43.0 & 9.4 & 332.2 & 2.7 \hspace{18pt} \\
        Adelaide        & 50.0 & 50 & 50.0 & 50.0 & 2\,402.0 & 50.0
                        & 50.0 & 50 & 44.7 & 43.7 & 2\,045.5 & 13.4 \hspace{18pt} \\
        Leeds           & 149.7 & 49 & 121.4 & 39.5 & 1\,880.1 & 35.6
                        & 37.9 & 10 & 22.5 & 8.4 & 61.2 & 2.1 \hspace{18pt} \\
        Linz            & 33.0 & 33 & 33.0 & 33.0 & 959.8 & 31.6
                        & 37.9 & 26 & 20.7 & 19.2 & 834.3 & 4.6 \hspace{18pt} \\
        \bottomrule
    \end{tabular}
    \vspace{\tblvsp}
\end{table*}


\subsection{Security}
The adversary \adv aims to geolocate a specific client by identifying the origin gateway.
\adv may use the following strategies to accomplish this: (1) \adv assumes the used output gateway is close enough and target it instead, (2) \adv uses individual intercepted messages from the local network to trace the origin, and (3) \adv monitors the gateways for extended periods to collect data pointing to the origin.

Strategy (1) is prevented in our system by rerouting communication away from the origin gateway used by the client.
In \Cref{sec:eval:sim:dist}, we analyze the effective distances achieved on real-world data sets.
Nevertheless, due to our \emph{Changing Gateway} approach (cf. \Cref{sec:overview}), eventually, the actual origin gateway is used for the satellite uplink.
Thus, \adv may simply target each gateway and eventually be successful.
However, this would effectively result in a large-scale attack, e.g., in an urban context, targeting the entire city.
As stated in our adversary model (\Cref{sec:adv_model}), we deem this undesirable for \adv.
%
Note that this also applies to \adv taking kinetic measures against gateways in subregions of the network. As described in \Cref{tbl:reach2} (column \emph{Unique Paths}), every network has at least two unique paths between any two nodes, i.e., alternative paths that do not share any nodes.
This demonstrates the network's resilience against forceful disconnection, even in the presence of a costly, wide-ranging attack on many gateways instead of targeting individual ones.
We deem this undesirable, as \adv tries to limit its attacks as much as possible.
Further, in case \adv captures extensive territory, resulting in few gateways in the region, we assume that clients will not remain in the area and transmit sensitive data.

For the second strategy (2), \adv is prevented from learning any information about the route with the end-to-end encryption employed by the gateways.
Yet, \adv may intercept numerous local gateway messages over time and eventually be able to correlate the route from the output gateway back to the origin.
Our system prevents this by regularly changing the output gateway (cf. \Cref{sec:overview}).
The effectiveness of this approach is dependent on the gateway distribution, which we evaluate in \Cref{sec:eval:sim:dist} on our real-world data sets.

\adv may monitor all used output gateways by a client to infer the origin.
The success of this strategy (3) is prevented by the \emph{Selection Bias}, as described in \Cref{sec:overview}, which shifts the centroid of all used gateways over time away from the origin.

In the following, we discuss additional local attacks, motivating our adversary model.

\paragraph{\emph{Total Local Network Monitoring}}
In \Cref{sec:adv_model}, we assume \adv is unable to monitor the entire gateway network to establish a holistic view of the local network.
\adv would aim to trace back routes from the output gateway back to the origin.
\adv would need to get an extensive coverage of the gateway network.
Practically speaking, to achieve this, \adv needs to deploy numerous devices, which must be widely spread in close proximity to the gateways and potentially deployed over long periods, as it is unknown where and when the client may become active.
Considering the potential scale of an active conflict, we deem this strategy infeasible.

\paragraph{\emph{Jamming}}
\adv may try to use jamming to interrupt the client's ability to communicate.
There are three types of jamming to consider.
One type is targeting the satellites, e.g., with a high-power ground-based jamming signal directed at individual satellites.
However, with over 3\,000 Starlink satellites deployed~\cite{Istarlinktotalno} at the time of writing, this strategy does not scale well.
Further, according to reports, Starlink used specialized firmware updates to withstand numerous jamming attacks~\cite{Istarlinkupdate}.

An additional type of jamming is adversarial satellites jamming ground stations.
Theoretically, a signal can be considered jammed if the Signal to Noise Ratio (SNR) at the receiver is 1.
To achieve this the jamming signal must be received with at least the same power as the benign signal~\cite{mpitziopoulos2009effective}.
The power of electromagnetic signals decays quadratically with distance.
Thus, a satellite targeting a ground station would need a jamming signal powerful enough to cover the vast distances in space.
As satellites are significantly limited in terms of power, we deem this strategy infeasible.

Another strategy is local jamming ground-to-ground, as modern jammers may cover a wide area.
However, depending on the scale of the gateway network, \adv would need to either deploy many jammers or target the client's general area, which might be unknown.
In case \adv \emph{is} able to jam the client's gateway, this is difficult to circumvent; however, in this case, \adv is not able to geolocate the client.

\paragraph{\emph{Malicious Gateways}}
\adv may try to deploy malicious gateways.
However, we deem this strategy to be unviable.
Similar to the holistic monitoring of the gateway network, \adv would need to deploy or compromise a plethora of devices spread throughout the gateway network to reliably trace clients.
Individual gateways may reveal the client's route if the client actually routes over them.
However, unlike \emph{overlay} networks, in which an adversary can remotely create new nodes, \adv must control \emph{physical} gateways in advantageous geographic locations to reliably target clients.
Thus, such an elaborate strategy requires physical access, resulting in low probabilities of success.
To protect against leakage of the origin gateway by intermediary malicious nodes, approaches such as onion routing can be deployed to \sysname, as described in \Cref{sec:tech:malicious}.

