%!TEX root = ../main.tex

\section{Related Work}
\label{sec:rw}

To the best of our knowledge, our approach is the first work to address the triangulation of satellite Internet users in critical circumstances.
Thus, we could not find directly related work for our purposes.
However, one related research topic is providing \emph{location privacy} in mesh and wireless sensor networks.
Another related research topic is \emph{emergency communication networks}, which, similarly to \sysname, often employ mesh-like network topologies and are designed to work under exceptional situations.

\paragraph{\emph{Location Privacy in Mesh and Wireless Sensor Networks}}
In these networks, due to their physical properties, communication is vulnerable to tracking and monitoring.
If vulnerable participants utilize such a network, keeping the (geographic) position undisclosed becomes essential.
Approaches for location privacy can be grouped into eleven categories~\cite{conti2013providing}.

The high-level idea of \emph{Random Walk}-based approaches is to direct packets to traverse a network through a random path to a sink (e.g., base station).
This makes the path of a packet unpredictable to an adversary attempting local traffic analysis~\cite{kamat2005enhancing, xi2006preserving}.
\emph{Geographic Routing} is similar to Random Walk, yet utilizes the physical location information of nodes for more efficient routing of packets towards the sink.
For location privacy, these approaches additionally leverage pseudonyms, reputation, and a fixed set of intermediary nodes creating a mix subnetwork~\cite{shaikh2010achieving, li2009preserving}.
Both these types of approaches assume a \emph{backtrack} or \emph{hunter} adversary model, in which the adversary starts close to the sink, traces each sent-out message back to the next hop, and this is repeated until the adversary eventually arrives at the source.
However, this is incompatible with our assumptions, as all nodes in \sysname are sinks, we regularly switch the sink, and a backtracking adversary is unlikely in a conflict zone.

In \emph{Delay}-based solutions, nodes store incoming packets and transmit them after a random period of time, disrupting the chronological order of the packets. 
This also modifies the traffic pattern, rendering it difficult for a local adversary to trace the origin of the traffic~\cite{hong2005effective, kamat2009temporal}.
\emph{Limiting node detectability} temporarily throttles or disables the transmission power of nodes, making it harder for an adversary to receive packets~\cite{rios2011exploiting, el2010hyberloc}.
\emph{Network Coding} uses homomorphic encryption at intermediary nodes to hide traffic flows. 
After receiving the aggregated data, the sink is then able to reverse the encryption process~\cite{fan2009efficient, fan2010preventing}.
However, these three classes of solutions add significant delays to the network that adversely affect low-latency networks, such as \sysname, especially when the aim is to be compatible with the Internet (cf., \Cref{sec:tech:tcp}).

\emph{Dummy Data Sources} generate authentic-looking dummy traffic to obscure the authentic traffic.
The objective is to prevent an adversary from differentiating between genuine and fabricated traffic~\cite{yang2013towards, mehta2007location}.
\emph{Cyclic Entrapment} confuses potential adversaries by routing the traffic between nodes with cyclical patterns~\cite{ouyang2006entrapping, kazatzopoulos2006ihide}.
\emph{Separate Path Routing} splits data into multiple packets, which will be sent over multiple, non-intersecting paths to the sink. 
Therefore, the local adversary is only able to capture part of the data~\cite{wang2009privacy}.
Similarly to introducing delays, these three approaches induce large traffic overheads to the network (e.g., dummy traffic or additional retransmissions). 
Thus, they are incompatible with the goals of \sysname (cf., \Cref{sec:requirements}).

\emph{Cross Layer Routing} utilizes multiple OSI layers to hide information from adversaries~\cite{shao2009cross}.
Yet, this is based on the assumption that the adversary may not see certain OSI layers, which we deem impractical.
Approaches focusing on \emph{Wireless Mesh Networks} specifically, usually assume a hierarchical network with base stations, mesh routers, and mesh clients.
Numerous security features, including location privacy, rely on pseudonyms in the form of public key certificates, either directly distributed by an authority~\cite{zhang2006arsa} or are self-generated with a domain authority backing it~\cite{sun2010sat}.
\emph{In network location anonymization} utilizes hierarchical (e.g., clusters of nodes) pseudonyms or aggregation of traffic to hide the source of the traffic~\cite{misra2006efficient, luo2010location}.
Both the Wireless Mesh Networks and In network location anonymization approaches assume a hierarchical trust structure in the network, which is not feasible in \sysname's local network as, e.g., all gateways are set up by citizens and trusted equally.

\paragraph{\emph{Emergency Communication Networks}}%
These networks are designed to provide reliable communication during an emergency when other communication infrastructures fail. 
The research community has since proposed many approaches for establishing a network in case of natural disasters, conflicts, or any adverse situation that prevents typical access to the Internet.
In particular, Portmann \etal~\cite{portmann2008wireless} defined the fundamental characteristics an emergency network must have in its design: \emph{privacy, data integrity, authentication, and access control}. 
The most common emergency networks are considering the use of Locally Deployed Resource Units (LDRU) (e.g., base stations of cellular networks), satellites, ad-hoc networks, or a combination of them.
Usually, portable devices span a network, while only a subset of them are actually capable of external means of communication (e.g., satellites)~\cite{pradeep2015survey, kishorbhai2017aon}.

To establish emergency networks, many recent works focus on using Unmanned Aerial Vehicles (UAVs)~\cite{debnath2021comprehensive}.
Numerous works in this area leverage \emph{drones} to establish the emergency network.
One proposal is to directly leverage the drones as base stations~\cite{zhao2019uav}.
Other works use drones to span a mesh network back to a static base station.
Proposed wireless communication technologies for the mesh range from leveraging WiFi~\cite{panda2019design}, LTE~\cite{deruyck2018designing}, 5G~\cite{gao2020intelligent}, and LoRa~\cite{pan2021uav}.
Besides the use of drones, the application of aerostatic balloons found space in the context of emergency networks. 
One approach is to build an ad-hoc network based on IEEE 802.11j between the ballons~\cite{shibata2009disaster}.
Another approach is to span a multihop WiFi backbone from one area to a satellite-based base station via zeppelin-like balloons~\cite{suzuki2006ad}, while another extends this approach to a mesh network~\cite{okada2012network}.
Besides all the possible proposed designs, multiple recent works are proposing several optimizations, including load balancing between the units~\cite{lin2021adaptive, niu20213d}.

A practical concern is that drones exhibit limited fly times, and thus, coverage. Further, the individual hardware required (i.e., the drones and balloons) is expensive; yet, hundreds or even thousands of units are necessary to operate in an emergency. Moreover, drones and aerostatic balloons are subject to weather conditions (e.g., cannot easily fly in a storm), which is heavily limiting their operative scenario.

Other works focus on deploying an ad-hoc mesh network between base stations that are then put into communication with satellites. 
Zhou \etal~\cite{zhou2021integrated} develop such a network based on WiFi 2.4\,GHz for the network backbone and 5.8\,GHz for data transmission. However, due to the limited range of WiFi, this approach requires the devices to be quite close to each other and does not scale well to cover a wide area. 
Instead, Iapichino \etal~\cite{iapichino2008advanced} propose a hybrid system where equipped vehicles (Vehicle Communication Gateways) establish a connection with satellites and users can connect to these mobile gateways. 
Similarly, Patricelli \etal~\cite{patricelli2009disaster} are proposing a MOBSAT access point (that has to be carried by car or helicopter), which provides high-speed data connection to the users through WLAN and WiMAX through GEO satellites.
Nevertheless, these approaches assume that numerous, specially equipped vehicles are prepared and ready for use.

In contrast, the target scenario of \sysname is novel in the context of emergency networks.
The continuous expansion of satellite Internet services allows for the deployment of comparably cheap access points.
This enables to provide widespread access to many deployed satellite base stations.
Therefore, the mentioned works above are not taking the unique problems of this scenario into account.
As outlined in the Introduction, \sysname specifically aims to protect its users from triangulation.
Furthermore, our focus is on the accessibility of the design system.
Thus, \sysname does not require any additional application or hardware installed on the user's device and the gateways are easy to deploy.
