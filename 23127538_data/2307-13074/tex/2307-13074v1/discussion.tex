\section{Discussion}
%
\label{sec:discussion}

\subsection{Merging with Prior Results}
%
\label{sec:comparison}

Our group has embarked on a systematic effort to construct a comprehensive
empirical picture of tidal dissipation in low mass stars by combining
constraints derived from multiple tidal effects observable in multiple
collections of binary star or exoplanet systems. We rely on a common approach of
analyzing each system individually, employing Bayesian analysis to account for
observational and model uncertainties, and combining the individual constraints
to find a common prescription for $\Q$ capable of explaining the observed
properties of each population of objects. We begin by placing this result in the
context of the previous results from our group (Fig.
\ref{fig:merged_constraints}).

The most closely related study by our group \citep{Patel2022} (PP22 from now on)
analyzed the tidal effects on the spin of 41 of the Kepler eclipsing binaries
analyzed here, but assuming $\Q = $ constant and without the benefit of the W19
parameters, which were not available at the time. We used a similar approach,
analyzing one binary at a time, and then combining the individual posterior
distributions of $\Q$ for each binary system. We obtained a common constraint
$\Q = 7.82\pm0.035$, which was in agreement with each of the individual
constraints. As can be seen from Fig.~\ref{fig:merged_constraints}, there is
also agreement with the results found by this study. That is of course expected,
since many of the same binaries are analyzed both here and by \citep{Patel2022},
though with less precise parameters and using a different prescription for $\Q$.

\citet{Penev2022} (PS22 hereafter) studied tidal circularization of the single
line spectroscopic binaries in three open clusters (M35, NGC 6819, and NGC 188)
to obtain constraints of a frequency dependent $\Q$. PS22 constructed an
envelope in the period-eccentricity distribution for each cluster which
encompasses all observed binary systems, and required that dissipation is strong
enough to place any binary below the envelope and weak enough to allow
reproducing the observed present day eccentricity of each system. Using the same
formalism for frequency-dependent $\Q$ as defined in
Equations~\ref{eq:q_formalism}, PS22 found that for the two older clusters
($\sim$2\,Gyr old NGC 6819, and $\sim$7\,Gyr old NGC 188) $5.5 < \log_{10}\Q <
6$ for $2\,\text{days} < P_{\text{tide}} < 10\,\text{days}$, while for the
150\,Myr old M35 $\log_{10}\Q < 4.6$. Outside the $2\,\text{days} <
P_{\text{tide}} < 10\,\text{days}$ range the dissipation was not well
constrained. The difference in tidal periods probed by the two studies is due to
the fact that much stronger tides are required to affect the orbit than the spin
of the primary star.

\citet{Penev2018} (P18 from now on) investigated the effect of tides on the spin
of exoplanet host stars. Thanks to their small mass, the tides planets produce
on their parent stars are much weaker than binary star tides. As a result,
studying exoplanet systems allows probing tidal dissipation at even shorter
orbital (and hence tidal) periods than this or any of the above studies. P18
found a highly frequency-dependent dissipation, $\Q\propto P_{tide}^{-3.1}$, for
tidal periods $0.5\,\mathrm{days} < P_{tide} < 2\,\mathrm{days}$.

% Figure environment removed

It should be noted that, while in Fig.~\ref{fig:merged_constraints} we plot the
inferred tidal dissipation as a function of tidal frequency, that need not
necessarily be the explanation of all differences observed. Each article
analyzes a different collection of binary star or exoplanet systems with
different selection biases involved. Given the wide variety of possible tidal
dissipation physics (see Sec. \ref{sec:introduction}), differences between measured
dissipation in different studies may be caused by any systematic difference
between the populations investigated, or even by trends  within a population.
For example, the dominant tidal period in each system is related to the orbital
period, but eccentricity is also tightly correlated with orbital period. As a
result, the binaries driving the tightly constrained $\Q$ at tidal periods
$>15$\,days are the longest period binaries in our sample, which also have
systematically higher eccentricities than the binaries dominating the
constraints in PS22, and even more compared to the P18 systems which are
virtually all circular. In turn, the orbital eccentricity determines the
combination of tidal waves each component of the binary is subjected to.

\subsection{Comarison with Studies By Other Groups}
%
\label{sec:compare_others}

Our group is far from the only one attempting to infer how efficiently tidal
perturbations are dissipated by studying signatures of tidal dissipation in
observed populations of objects. An exhaustive review of all such efforts in the
literature is beyond the scope of this article, but we highlight some of the
most recent efforts here to allow comparison to our results.

One approach is to directly compare the predictions of a particular tidal
dissipation model to observation. This was recently done by \citet{Barker_2022}
who compared the predictions of inertial wave enhanced dissipation to observed
tidal circularization, and found good agreement. In that model, the dissipation
is strongly enhanced if the tidal period is more than half the spin period of the
star. On the main sequence, that model predicts $\Q\sim10^7
(P_\star/10\,\mathrm{days})^2$ for Sun-like stars, where $P_\star$ is the spin
period of the star. This corresponds to several times more efficient dissipation
than what we find at the longest orbital periods in our sample, and several times
less efficient than the dissipation PS22 find for main sequence binaries
(assuming synchronous rotation).

\citet{Meibom_Mathieu_05} and \citet{Milliman_et_al_14} used observed orbits of
spectroscopic binary stars in open clusters to claim a detection of continued
circularization during the main sequence for Sun-like stars that could be
explained by $\Q\sim\mathrm{few}\times10^5$ for tidal periods of several days.
This overlaps reasonably well with the PP22 results and lies within a factor of two
of the lower limit this study finds for such orbital periods.

\citet{Zanazzi_2022} recently combined the W19 eccentricities with
similar constraints from TESS \citep{Justesen_2021} to argue that the
period-eccentricity distribution of binary stars should be split into two
components: a cold core with very close to zero eccentricity out to orbital
periods as large as 10 days, and an envelope of systems which is only
circularized out to periods of 3 or 4 days, with a broad distribution of
eccentrities after that. The cold core requires tidal  dissipation similar to
what PS22 and \citet{Meibom_Mathieu_05} find, while the envelope needs much less
efficient dissipation, perhaps arguing for a dissipation mechanism that only
operates on some systems and not on others based on initial conditions.

\citet{Hamer2019} argue that the galactic velocity dispersion of Hot Jupiter
host stars implies they are on average younger than field stars, interpreting
this as evidence for tidal inspiral of Hot Jupiters over the main sequence
lifetime of Sun-like stars. This in turn requires $\Q<10^7$, in agreement with
all constraints mentioned above.

\subsection{Caveats and Assumptions}
%
\label{sec:caveats}

The analysis presented here makes a number of assumptions. Below we discuss
some of those assumptions.

The most central assumption we make is that each of the binaries in our input
sample contains no other dynamically relevant objects and that tides are the
only thing causing the orbit to evolve or stellar spin to deviate from that of
an isolated star. While triple or higher multiplicity stellar systems are not
uncommon, the W19 analysis used 4 years of ultra-high precision photometric
measurements which result in high sensitivity to eclipse timing variations that
would be produced by the presence of additional objects. Indeed, W19 encounter
such objects and discard them from their sample. Nonetheless, some higher
multiplicity systems likely still remain in the sample, though the configurations
of those systems will be such that the extra objects likely do not affect the
dynamics too strongly.

A critical component of our model affecting the predicted stellar spins is the
loss of angular momentum to stellar winds. The model we use correctly captures
the major features for stars spinning no slower than the Sun, but for slow
rotators, \citep{vanSaders_et_al_16, Metcalfe_Egeland_19} argue spin-down stalls.
Luckily, the slowest spinning star in our sample has a spin period of 21 days,
which is shorter than the period at which spin-down may stall.

Different tidal dissipation theories predict effective $\Q$ that depends on a
whole host of parameters (mass, age, metallicity, and spin of the star;
amplitude of the tidal distortion; etc.). Here we allowed only for smooth
dependence on the frequency with which each tidal wave propagates on the surface
of the star, but all the other parameters change from one system to the next.
Some parameters are even correlated with tidal frequency (e.g. tidal amplitude
    gets smaller at longer orbital periods which also correspond to longer tidal
periods). Not accounting for such dependencies could skew, or even mask
completely, the frequency dependence. The fact that the constraints we obtain
for individual systems are consistent with each other is an encouraging sign
that such dependencies are not dominant; however, that possibility cannot be
ruled out entirely. Investigating dependencies of $\Q$ on multiple parameters
would require a significantly larger sample of systems than the 70 used here.

We assumed the tidal dissipation to be present only in the convective zone of
the star. Although this is what equilibrium tide models predict for stars with
convective envelopes, some dynamical tide models indicate that the dissipation
will be dominated by interactions with g-mode waves inside the star's core. For
the same exact $\Q$, such models will predict significantly different surface
spin for stars, because tides will deposit angular momentum in the core, which
will be communicated to the envelope on some coupling timescale. Thus, somewhat
more efficient dissipation will be required to explain the observed spins if
tides couple to the core.  Exploring the implications of tides coupling to the
radiative core instead of the convective envelope is a separate project of
comparable complexity to the one presented in this article. As such we consider
it outside the scope at this moment and may pursue it in the future.

The treatment of angular momentum redistribution within stars in POET is quite
simplistic. The core and the envelope are assumed to converge to synchronous
rotation on a fixed timescale, and each zone is assumed to follow solid body
rotation. While this is probably a reasonable assumption for the core, we know
this picture is not correct for the convective envelope. For example, in the
Sun, different parts of the convective zone spin differently, and are not even
organized in spherical shells (see for example \citep{Miesch2005} for more
details). Since we compare predicted to observed spins to infer the
tidal dissipation effeciency, our results could be affected if important physics
is not captured by the simplistic treatment used.
