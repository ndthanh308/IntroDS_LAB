\section{Modified Tidal Dissipation Prescription}
%
\label{sec:q_tweak}

If $\alpha<0$, Eq. \ref{eq:q_formalism} has a discontinuity at
$\Omega_{m,m^{\prime}} = 0$. This makes it possible for tides to hold the system
in a spin orbit lock, keeping $\Omega_{m,m^{\prime}} = 0$ for an extended period
of time. \texttt{POET} handles this discontinuity by eliminating the stellar
spin as a variable and checking at each time step that the lock is maintained.
For $\alpha>0$, there is no discontinuity. However, even in this case it often
still happens that the system is held close to a lock for a long time, which
forces the differential equation solver to use tiny time steps in order to
maintain numerical accuracy. This in turn makes such cases very computationally
expensive to simulate. In order to avoid this situation, we slightly modify Eq.
\ref{eq:q_formalism} if $\alpha>0$ to force a small, but finite dissipation as
$\Omega_{m,m^{\prime}} \to 0$:

\begin{equation}
%
    \label{eq:q_tweak}
%
    \Delta_{m,m^{\prime}}
%
    =
%
    \Delta_0
%
    \begin{cases}
%
        {\left(\dfrac{\Omega_{\text{min}}}{\Omega_{\text{break}}}\right)}^{\alpha}
        & \text{if } \Omega_{m,m^{\prime}} < \Omega_{\text{min}} \\
%
        {\left(\dfrac{\Omega_{m,m^{\prime}}}{\Omega_{\text{break}}}\right)}^{\alpha}
        & \text{if }  \Omega_{\text{min}} < \Omega_{m,m^{\prime}} <
        \Omega_{\text{break}} \\
%
        1 & \text{if } \Omega_{m,m^{\prime}} > \Omega_{\text{break}}
%
    \end{cases}
%
\end{equation}

We pick $\Omega_{min}=\frac{2\pi}{50\,d}$ in Eq. \ref{eq:q_tweak}. This is small
enough to not appreciably change the evolution, while at the same time
eliminating most of the computational challenges.

%The value of $\Omega_{\text{min}}$ is selected as $\Omega_{\text{min}} = 2\pi/\left(P_{\text{tide} = 50 \text{ days}}\right)$. Equations~\ref{eq:formalism1} and~\ref{eq:formalism2} completely describes the parameterization of tidal dissipation. Figure~\ref{fig:w_dependent_model} shows the model evaluated at $\alpha = \pm2$, $\Omega_{\text{break}} = 2\pi/\left(\text{10 days}\right)$ and $\Omega_{\text{min}} = 2\pi/\left(\text{50 days}\right)$.
%
%    % Figure environment removed


