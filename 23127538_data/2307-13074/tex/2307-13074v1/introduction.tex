\section{Introduction}
%
\label{sec:introduction}

The study of tides is imperative in understanding the observed properties of
binary systems, as the evolution of the spins and orbital parameters in
binaries are often strongly influenced by tides. The discovery of hot-Jupiter
exoplanets \citep{Mayor1995, Marcy1996} has rekindled interest in understanding
the efficiency of tidal dissipation as tides can provide a reasonable
explanation for their observed short periods and close-to-circular orbits
\citep{Rasio1996, Fabrycky2007, Beaug2012, Vick2019}.

Under the influence of tides, orbits can either shrink or expand depending on
the spins of the binary members and on the orbit. In circular orbits for
example, if a given component of the binary spins slower than the orbit, tides
will transfer angular momentum from the orbit to the spin of the object, leading
to orbital decay. If an object is spinning faster than the orbit, tides cause it
to spin down towards synchronous rotation.  Reliable estimation of the
efficiency of tidal dissipation in such systems is important for understanding
either the survival of hot Jupiters \citep{Debes2010, Hamer2019} or the halting
of the inward migration \citep{Lin1996, Jaime2022}, or both. If the secondary
object is a star, tides usually do not threaten the long-term survival of the
system, because typically binary star systems have enough angular momentum to
achieve a
%state of minimum energy, which is classified into three major observational
%properties of the binary system:
state of minimum energy; this may manifest itself in three major observational
properties of the binary system:
%
\begin{enumerate}
%
    \item \textbf{Tidal Synchronization} $\rightarrow$ The rotation period of
        both stars is equal to the orbital period.
%
    \item \textbf{Tidal Circularization}  $\rightarrow$ Eccentricity of the
        orbit is close to zero.
%
    \item \textbf{Tidal Inclination}  $\rightarrow$ The angular momentum vector
        of the stars align with the angular momentum vector of orbit.
%
\end{enumerate}

There are many suggested tidal dissipation processes. They can generally be divided
into two classes depending on the physical mechanism responsible for the
dissipation: equilibrium tides \citep{Zahn_89, Goldreich_Keeley_77,
Penev_et_al_09, Ogilvie_2012, duguid2021, Vidal_2020}, where the energy
dissipation is primarily through turbulent viscosity, and dynamical tides
\citep{zahn1975, Goodman_Dickson_98, Rieutord_Valdettaro_10, Ogilvie_13,
Essick_Weinberg_16, fuller2016, Barker_2020, Ma2021OrbitalDO}, where the waves
excited inside the star interact with tidal perturbations and dissipation is
either through turbulent viscosity or wave breaking. Empirically, the efficiency
of dissipation is often parameterized by what is commonly known as the
\textit{tidal quality factor}: $Q$. The inverse of $Q$ is the ratio of the
energy dissipated from a tidal bulge in a single orbit to the tidal energy
stored in the tidal bulge \citep{Goldreich_Soter_66}. Lower values of Q
correspond to high dissipation and vice versa. In practice, parametric studies
of tides use a modified tidal quality factor: $Q^{\prime} = Q/k_2$
\citep{ogilvie_2013}, where $k_2$ is the second order love number. In this
paper, we use $\Q$ to refer to the modified tidal quality factor of a given
star.

The equilibrium theory of tides assumes that the star maintains hydrostatic
equilibrium while perturbed by the tidal forces. The quadrupole tidal field of
the companion deforms the primary, leading to the formation of tidal bulges.
Under the assumption of weak tides, one can further decompose the time
dependence of the tidal potential as a Fourier series, thinking of each term in
the series as driving an independent tidal bulge and contributing to the
dissipation as a simple superposition. Earlier studies assumed that the
perturbations in the star respond to the tidal potential with a constant lag in
either phase or time \citep{zahn1975, Hut1981, Alexander1973, eggleton1998}. In
general however, each tidal wave can experience different dissipation depending
on its tidal frequency:
%
\begin{equation}
%
    \Omega_{\text{tide}} = m\Omega_{\text{orbit}} - n\Omega_{\star}
%
    \label{eq:tidal_frequency}
%
\end{equation}
%
where m and n are integers, $\Omega_{\text{orbit}}$ is the orbital frequency and
$\Omega_{\star}$ is the spin frequency of the star. As a result, the lag, which
is inversely related to $\Q$, is a frequency-dependent parameter rather than
a constant, with the exact frequency dependence determined by the dissipation
mechanism.

Even though dynamical tide models do not strictly follow the above picture, the
resulting energy and angular momentum exchange can still be modeled by an
effective $\Q$ with an associated frequency dependence
%\citep[see][for more thorough discussion]{ogilvie_2013}.
\citetext{see \citealp{ogilvie_2013} for more thorough discussion}.

In this study we focus on stars with radiative cores and convective envelopes.
%\citep[and others]{Goodman_Dickson_98, Lecoanet2013} argue that in such stars,
\citetext{\citealp{Goodman_Dickson_98}, \citealp{Lecoanet2013}, and others}
argue that in such stars, tides can excite g-modes at the boundary of the two
zones, which travel inward through the core. If the wave amplitude is high
enough they can break, dissipating energy and depositing angular momentum. The
resulting effective tidal quality factor has a frequency dependence given by $\Q
\propto \Omega_{\text{tide}}^{-2.6}$ \citep{barker_ogilvie_2011}. If the wave
amplitude is not high enough for wave breaking to take place,
\citet{Essick_Weinberg_16} showed that dissipation can still occur through the
excitation of a network of daughter, grand-daughter, etc. modes and $\Q \propto
\Omega_{\text{tide}}^{-2.4}$.

\citet{OgilvieLin2004, Ogilvie_2007, Ogilvie_2009, Rieutord_Valdettaro_10,
Barker_2020} proposed a different dissipation mechanism, driven by damping of
tidally excited inertial waves by turbulent viscosity in the convective zone of
stars resulting in enhanced dissipation compared to equilibrium tides. Since
inertial waves only exist with frequencies $\left| \omega \right| < \left|
2\Omega \right|$, where $\Omega$ is the spin frequency of the star, the
dissipation will only be enhanced for tidal frequencies in this range. The
resulting dissipation is predicted to be highly frequency dependent with a dense
spectrum of peaks. A widely used method for accounting for enhanced dissipation
due to inertial waves is averaging over the dissipation spectrum in the frequency
domain to calculate the net contribution of $\Q$ from dynamical tides
\citep{Ogilvie_13}.

Yet another dissipation mechanism invokes resonance locking if the tidal
frequency is in resonance with a low-frequency inertial wave in the convective
envelope or a standing g-mode in the radiative zone \citep{Witte1999, Witte2001,
fuller2016, Fuller_2017}. This could lead to enhanced dissipation if the
resonance is held for a long period without breaking the lock. Most recently,
\citet{Ma2021OrbitalDO} predict resonance locking can be modeled with
$\Q \propto \Omega_{\text{tide}}^{13/3}$.

Validating and clarifying this broad range of models requires measuring $\Q$
from observations. This is far from the first effort to accomplish this.
Recently, \citet{Bolmont_mathis_2017} presented a tidal evolution model
restricted to circular orbits, with zero spin-orbit misalignment. The authors
use the frequency-averaged dissipation due to inertial waves
\citep{ogilvie_2013} with a constant time lag model \citep{Hut1981,
eggleton1998} to approximate $\Q$ as a constant parameter.
\citet{Benbakoura2019} makes a distinction between the outer convective and
inner radiative zones of the stars to allow for angular momentum exchange
%between the two zones of the stars. The authors also improve on the
between the two zones. The authors also improve on the
\citet{Bolmont_mathis_2017} model by differentiating between $\Q$ obtained from
equilibrium tides and dynamical tides, but the tidal potential is restricted to
only lower-order eccentricity terms with no spin-orbit misalignment.

In this paper we explore the dependence of $\Q$ on tidal frequency using
low-mass eclipsing binaries observed by Kepler, with parameters taken from
\citet{Windemuth_2019}, combined with rotation period measurements of the
primary star by \citet{Lurie_2017}. Unlike the earlier efforts mentioned above,
we do not assume a particular tidal model, but instead assume a generic
parameterization of frequency dependent $\Q$. We use Bayesian analysis to fully
account for observational uncertainties, and a tidal evolution model designed to
analyze any parametric dependence of $\Q$ on binary properties, along with the
effects of higher order eccentric terms in tidal potential, changes in internal
structure of the star, and spin-down of the star due to magnetic winds.

The rest of the paper is as follows. In Section~\ref{sec:tidal_model}, we
provide details of the tidal evolution model used for the analysis.
Section~\ref{sec:input_data} describes the source of the observational data
used. Section~\ref{sec:q_formalism} shows the formalism for the
frequency-dependent $\Q$ model. Section~\ref{sec:bayesian_analysis} describes the
methodology for Bayesian analysis. Our results are presented in
Section~\ref{sec:constraints}. Section~\ref{sec:discussion} discusses the
$\Q$ constraints obtained. We conclude in Section~\ref{sec:conclusion}.


