\section{$\Q$ Formalism}
%
\label{sec:q_formalism}

In this work, we assume that the tidal dissipation occurs only in the convective
zones of the stars, meaning that we assume the tidal torque is applied only to
the convective zone, with the core spin affected only indirectly through the
core-envelope coupling.  As mentioned in Section~\ref{sec:introduction}, the
effects of the binary system parameters on tidal dissipation can be studied by
using a suitable parameterization for the tidal lag, $\Delta_{m,m^{\prime}}$ or
the corresponding tidal quality factor, $Q_{m,m^{\prime}}^{\prime}$, such that
it can capture the orbital dynamics of the system under the effects of tides.

As discussed in Sec. \ref{sec:tidal_model}, we allow each Fourier component of
the tidal potential to experience its own effective lag. In this work, we
restrict the possibilities to some extent, by assuming that
$\Delta_{m,m^{\prime}}^{\prime}$ depends solely on the tidal frequency (Eq.
\ref{eq:tidal_frequency}). Each equilibrium or dynamical tidal model predicts a
different frequency dependence. In this work, we further simplify the
prescription to assume that $\Delta_{m,m^{\prime}}$ has a power-law dependence,
on $\Omega_{\text{tide}} = m\Omega_{\text{orbit}} - m'\Omega_{\star}$.
Furthermore, in order to avoid clearly non-physical and numerically unstable
behavior occurring if $\Delta_{m,m'} \rightarrow \infty$, we assume the
dissipation saturates to some maximum value:
%
\begin{equation}
%
    \label{eq:lag_formalism}
%
    \Delta_{m,m^{\prime}}^{\prime}
%
    =
%
    \mathrm{sign}(\Omega_{m,m^{\prime}}) \Delta_0
%
    \min\left[
%
        1,
%
        \left| \dfrac{\Omega_{m,m^{\prime}}}{\Omega_{\text{break}}}
        \right|^{\alpha}
%
    \right]
%
\end{equation}
%
or in terms of $\Q$:
%
\begin{equation}
%
    \label{eq:q_formalism}
%
    Q_{m,m^{\prime}}^{\prime}
%
    =
%
    \mathrm{sign}(P_{m,m^{\prime}}) Q_0
%
    \max\left[
%
        1,
%
        \left| \dfrac{P_{m,m^{\prime}}}{P_{\text{break}}} \right|^{\alpha}
%
    \right]
%
\end{equation}
%
where $\alpha$ is the powerlaw index parametrizing the frequency dependence,
$\Delta_0=\frac{15}{16\pi Q_0}$ parametrizes the maximum allowed dissipation,
$\Omega_{m,m'}=\frac{2\pi}{P_{m,m'}}$ is the tidal frequency of a particular
tidal wave, and $\Omega_{\text{break}}=\frac{2\pi}{P_\text{break}}$ is the
frequency at which the dissipation saturates.  See Appendix \ref{sec:q_tweak}
for a small modification we introduce, which improves numerical performance
without significantly changing the evolution.

We now wish to find constraints on $Q_0$, $\alpha$, and $P_{\text{break}}$ that
can simultaneously reproduce the observed spins of the primary stars in all the
binaries in our input sample. In practice, the tidal effects on the spin of each
binary will be dominated by a narrow range of frequencies, so for a given
binary, we will be able to constrain $Q_{m,m^{\prime}}^{\prime}$ for a narrow
range of periods. Furthermore, because tides get weaker as the separation
between the two stars increases, we expect that for the lowest periods of our
sample of binaries all systems will be synchronized, thus providing only a lower
limit on the dissipation (upper limit on $\Q$), while at the longest periods
stellar spins will not be significantly affected by tides, providing only an
upper limit to the dissipation (lower limit on $\Q$). Because the effects of
tides decrease very precipitously as the orbit gets wider, there should be only
a small number of binaries in between these two regimes providing two-sided
constraints. It should be noted that the range of orbital periods with
asynchronous binaries whose spin is nonetheless significantly affected by tides
is much wider than one may naively expect. This is caused by the fact that the rate
at which stars lose angular momentum due to magnetized winds scales as the cube
of the angular velocity. Hence, maintaining synchronous rotation requires ever
stronger tidal coupling as the orbital period decreases. Even with that, however,
combining the constraints from multiple binary systems is required to get a
measurement of the tidal dissipation and detect any potential
frequency-dependence.
