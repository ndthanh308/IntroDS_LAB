\section{Bayesian Analysis}
%
\label{sec:bayesian_analysis}

We use the Markov Chain Monte Carlo algorithm as implemented in the Python
package \texttt{emcee} \citep{Foreman_Mackey_2013}. \texttt{emcee} is based on a
class of ensemble sampler algorithms that uses affine invariant transformation
as a proposal function to simultaneously advance multiple not-independent chains,
referred to as walkers, in the parameter space. We refer the reader to
\citep{Foreman_Mackey_2013} and references therein for details of the algorithm.

The choice of the number of walkers is a compromise between parallelization
efficiency and the need to accumulate a sufficient number of steps to ensure the
distribution of samples has converged to the target posterior. For our analysis,
we use 64 walkers for each binary to sample the 9-dimensional space of
parameters required to calculate the orbital evolution.

The parameters sampled by MCMC and their respective priors are listed in Table.
\ref{tab:sampling_parameters}. For the physical parameters of the binaries we
construct a joint prior using Kernel density estimation from the MCMC samples
published by W19, and for the dissipation parameters we use very broad uniform
priors.

% Please add the following required packages to your document preamble:
% \usepackage{graphicx}
\begin{table}
%
    \centering
%
    \caption{The parameters sampled during MCMC and their priors. The W19 joint
    prior is constructed using KDE from the W19 MCMC samples.}
%
    \label{tab:sampling_parameters}
%
    \resizebox{0.49\textwidth}{!}{%
%
        \begin{tabular}{cccc}
%
            \hline
%
            \textbf{Parameter} & \textbf{Units} & \textbf{Description} &
            \textbf{Prior} \\
%
            \hline
%
            $M_1$ & $M_{\odot}$ & Mass of the primary star & W19 \\
%
            $M_2$ & $M_{\odot}$ & Mass of the secondary star & W19 \\
%
            $\left[\frac{\mathrm{Fe}}{\mathrm{H}}\right]$ & - & Metallicity of
            both stars & W19 \\
%
            $\tau$ & Gyr & Age of the system & W19 \\
%
            $e$ & - & Present day eccentricity of the Orbit & W19 \\
%
            $P_{\star,\text{init}}$ & days & Initial spin period of the stars &
            $U()$ \\
%
            $\log_{10}Q_0$ & - & Dissipation parameter. See Eq.
            \ref{eq:q_formalism} & $U(5, 10)$ \\
%
            $\alpha$ & - & Dissipation parameter. See Eq.  \ref{eq:q_formalism}
            & $U(-5,5)$ \\
%
            $\log P_\text{break}$ & rad/day & Dissipation parameter. See Eq.
            \ref{eq:q_formalism} & $U\left(\log 0.5\,d, \log 50\,d \right)$ \\
%
            \hline
%
        \end{tabular}%
%
    }
%
\end{table}

\subsection{Prior Transformation}
%
\label{sec:prior_transform}

Because the prior distributions built from W19 samples are quite complex
structures in a multi-dimensional space, the MCMC sampling becomes very
inefficient. In order to remedy this situation, instead of directly sampling
from the prior distribution of the parameters from Table
\ref{tab:sampling_parameters}, we use a transformed set of parameters ($u_i$
with $i=1\ldots 9$) such that each of those has an independent prior
distribution uniformly distributed between 0 and 1. Given a sample of values for
$u_i$, the physical and dissipation parameters required to calculate the
evolution are then found by applying a prior transformation. This procedure
dramatically simplifies the posterior likelihood the MCMC process must sample
from, making it directly proportional to the likelihood that the actual spin of
the primary star as observed today is equal to the spin period predicted by the
evolution assuming the given parameters.

The joint prior distribution we wish to impose on the physical parameters of
each binary (the first 5 parameters listed in Table \ref{tab:sampling_parameters})
is the posterior distribution W19 samples are drawn from. We start by isolating
only the parameters we need. For each sample, we combine the components of
eccentricities ($e\cos{\omega}$ and $e\sin{\omega}$) to just eccentricity ($e =
\sqrt{(e\cos\omega)^2 + (e\sin\omega)^2}$). We then construct a marginalized
probability distribution function $\pi \left(M_{sum}, q, \tau,
[\mathrm{Fe}/\mathrm{H}], e\right)$, where $M_{sum} \equiv M_1 + M_2$ and $q
\equiv M_1/M_2$, using a Kernel density estimator (KDE) with a Gaussian kernel
with a bandwidth determined using the improved Sheather-Jones algorithm
\citep{sheather2010}.

Finally, W19 samples were generated assuming independent $U(0,1)$ priors on
$e\cos\omega$ and $e\sin\omega$. This has the undesirable effect that when
converted to prior on eccentricity, the probability of $e=0$ is zero, even if
the cloud of points in $e\cos{\omega}$, $e\sin{\omega}$ space clearly includes
the origin. This will propagate to the prior distribution we impose resulting in
erroneously imposing an upper limit on the dissipation (lower limit on $\Q$) for
such systems. To see this, consider a system for which the primary star is
consistent with spinning synchronously with the orbit, and the $e\cos{\omega}$,
$e\sin{\omega}$ cloud of points is centered on the origin (i.e.  observations
are consistent with circular orbit). We expect that for such a system
arbitrarily small $\Q$ should be acceptable since large amounts of tidal
dissipation will predict a completely circularized and synchronized system.
Instead, the prior on eccentricity imposed by W19 will exclude that
configuration, erroneously requiring a small eccentricity comparable to the
uncertainty in $e\sin{\omega}$ (the much less well constrained of the two
components) to survive to the present day. Instead of the W19 prior, we wish to
impose independent priors $e \in U(0,1)$ and $\omega \in U(0,2\pi)$. The desired prior
is obtained from the W19 prior by simply dividing by $e$ and re-normalizing.

To be precise, if the $s^{\text{th}}$ W19 sample has values for the parameters given by
$M_{sum}=M_s$, $q=q_s$, $\tau=\tau_s$, $\feh = \feh_s$, and
$e=e_s$, the prior probability density of the systems parameters is given by:
\begin{align}
    \pi & \left(M_{sum}, q, \tau, \feh, e\right) \propto \nonumber \\
    &~~~~~~~~\sum_s k_M(M_{sum}-M_s) k_q(q - q_s)~~~\times \nonumber \\
    &~~~~~~~~~~~~~~k_\tau(\tau-\tau_s)  k_\feh\left(\feh - \feh_s\right) \frac{k_e(e-e_s)}{e}
\end{align}
%
%\begin{equation}
%
%    \begin{array}{c}
%
%        \pi\left(M_{sum}, q, \tau, \feh, e\right) \\
%
%        \propto\\
%
%        \sum_s k_M(M_{sum}-M_s) k_q(q - q_s) k_\tau(\tau-\tau_s)
%        k_\feh\left(\feh - \feh_s\right) \frac{k_e(e-e_s)}{e}
%
%    \end{array}
%
%\end{equation}
%
where $k_M\ldots$ are the kernels used for each of the quantities, and the
factor of $1/e$ in the last term is responsible for changing the priors as
explained above.

The resulting prior transformation converting $u_1, \ldots, u_5$ for a given
MCMC sample to the corresponding physical system parameters is then:

\begin{equation}
%
    \begin{array}{rcl}
%
        M_{sum}  & = & F_M^{-1} (u_1)\\
%
        q_i  & = & F_q^{-1} (u_2|M_{sum})\\
%
        \tau_i  & = & F_\tau^{-1} (u_3|M_{sum},q)\\
%
        Z_i  & = & F_\feh^{-1} (u_4|M_{sum},q,\tau)\\
%
        e_i  & = & F_e^{-1} (u_5|M_{sum},q,\tau,\feh)\\
%
    \end{array}
%
\end{equation}
%
with
%
\begin{equation}
%
    \begin{array}{r@{\ }c@{\ }l}
%
        F_M(M) & \equiv & \int_{-\infty}^{M} \text{d}M_{sum}
        \int_{-\infty}^{\infty} \text{d}q \int_{-\infty}^{\infty} \text{d}\tau
        \int_{-\infty}^{\infty} \text{d}\feh \int_{-\infty}^{\infty} \text{d}e\\
%
        && \quad \quad \pi\left(M_{sum}, q, \tau, \feh, e\right)\\
%
%
%
        F_q(q|M_{sum}) & \equiv & \int_{-\infty}^{q} \text{d}q'
        \int_{-\infty}^{\infty} \text{d}\tau \int_{-\infty}^{\infty}
        \text{d}\feh \int_{-\infty}^{\infty} \text{d}e\\
%
        &&\quad \quad \pi\left(M_{sum}, q', \tau, [\mathrm{Fe}/\mathrm{H}],
        e\right)\\
%
        && \ldots
%%
%%
%%
%        F_\tau(\tau|q,M_{sum}) & \equiv & \int_{-\infty}^{\tau} \text{d}\tau'
%        \int_{-\infty}^{\infty} \text{d}[\mathrm{Fe}/\mathrm{H}]
%        \int_{-\infty}^{\infty} \text{d}e \\
%%
%        && \quad \quad \pi\left(M_{sum}, q, \tau', [\mathrm{Fe}/\mathrm{H}],
%        e\right)\\
%%
%%
%%
%        F_\feh(\feh|\tau,q,M_{sum}) & \equiv & \int_{-\infty}^{\feh}
%        \text{d}\feh' \int_{-\infty}^{\infty} \text{d}e \pi\left(M_{sum}, q,
%        \tau, \feh', e\right)\\
%
%
    \end{array}
%
\end{equation}
%
The above calculations are feasible because calculating multi-dimensional
integrals is not necessary. Kernel functions are normalized, so integrals over
$\pm\infty$ are all unity.

The priors on the dissipation parameters and the initial spin frequency of the
primary star are assumed uniform. The corresponding prior transform for those is
just a simple shift and scaling to match the limits.

The parameters not listed in Table \ref{tab:sampling_parameters} and required by the
evolution (Sec. \ref{sec:tidal_model}) are assumed constant with the following
values:
%
\begin{equation}
%
    \begin{array}{lcl}
%
        K & =& 0.17\,
        M_\odot\,R_\odot^2\,\mathrm{day}^2\,\mathrm{rad}^{-2}\,\mathrm{Gyr}^{-1}\\
%
        \omega_{sat} & = & 2.45\,\mathrm{rad}\,\mathrm{day}^{-1}\\
%
        \tau_{c-e} & = & 5\,\mathrm{Myr}.
%
    \end{array}
%
\end{equation}

\subsection{MCMC Convergence Diagnostic}
%
\label{sec:convergence}

The samples generated by an MCMC algorithm can be shown to follow the specified
posterior distribution in the limit of infinitely long chains. Real-world
applications then need to demonstrate that the generated chains are long enough
to get a good approximation to the distribution. There are two concerns that
must be addressed. First, in infinite chains the starting positions are
irrelevant. In finite chains however, an MCMC process requires some number of
steps before subsequent samples can be shown to come from the desired
distribution to a good approximation. This is usually referred to as the burn-in
period, and a typical practice is to discard the early samples. The second
concern is that one needs a sufficient number of post-burn-in samples in order to
estimate the targeted quantities with a desired precision.

In this work, we are interested in estimating quantiles of the posterior
distribution. \citet{raftery1991many} derived diagnostics for single chain MCMC
of a single quantity that answer the questions:
%
\begin{itemize}
%
    \item What is the smallest $N$ such that the probability that the $N+1$ step
        in the chain is below some threshold value is within a specified
        precision of the limit of that probability for $N\to\infty$.
%
    \item What is the variance of the estimated value of the cumulative
        distribution at the threshold value from the MCMC chain after the $N^{\text{th}}$
        sample.
%
\end{itemize}

The \citet{raftery1991many} procedure cannot be directly applied to
\texttt{emcee} chains because that involves multiple non-independent chains. In
\citet{Patel2022} and \citet{Penev2022} we adapted the \citet{raftery1991many} procedure
for \texttt{emcee} chains, allowing us to estimate when enough samples have been
generated to reliably and precisely estimate a specified quantile of some target
quantity.

In this work, we select a grid of tidal periods $P_{tide,i}$ and use the
\texttt{emcee} samples of $Q_0$, $\alpha$, and $P_{\text{break}}$ to evaluate
the tidal model (Eq. \ref{eq:q_formalism}) at each of these periods to obtain
samples of $Q'_i = Q_{m,m^{\prime}}^{\prime}(P_{tide,i})$. For each of these
quantities, we wish to find the 2.3\%, 15.9\%, 84.1\%, and 97.7\% quantiles.
Using the adapted \citet{raftery1991many} formalism, we find a burn-in period
for each $Q'_i$  for each quantile, requiring that the fraction of the first
post-burn-in samples of $Q'_i$ (one for each walker) below the quantile is
within $10^{-3}$ of the equilibrium probability, and estimate the uncertainty in
the CDF for each $Q'_i$ and quantile combination from the remaining samples
after discarding the burn-in.

Because finding a given quantile requires knowing the burn-in period, and
finding the burn-in period requires knowing the quantile, we iterate between the
two to find a mutually consistent combination.
