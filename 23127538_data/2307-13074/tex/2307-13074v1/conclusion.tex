\section{Conclusion}\label{sec:conclusion}

By combining catalogues from the literature of the physical properties and
primary star spins of a collection of eclipsing binaries, we were able to infer
how efficiently tidal perturbations dissipate energy. We find tight constraints
for a frequency dependent modified tidal quality factor $Q_{\ast}^{\prime}$ for
tidal frequencies between 15 and 50 days, with $7.8 < \log_{10}\Q < 8.1$ at the
longest periods, with indications of decreasing by approximately 0.5\,dex
(dissipation efficiency enhanced by a factor of 3) for periods of 15 days. Below
15 days the precision with which the dissipation is measured is dramatically
reduced: $5.7 < \log_{10}\Q < 7.5$. The decreased precision is because most
systems sensitive to such tidal periods are also in close enough orbits to have
their spin synchronized with the orbit, reducing our ability to place a lower
limit on $\Q$ (upper limit on the dissipation).

The constraints obtained in this article expand on previous constraints found by our
group, which has embarked on a long-term effort to explore signatures of tidal
dissipation in low mass stars in multiple settings, including tidal
circularization in binary stars, the effects of tides on the spin exoplanet
hosts and binary star systems, and others. Comparison between all constraints in
this series to date is presented in Fig.~\ref{fig:merged_constraints}.
Interpreting the variations of $\Q$ from one system to the next as a frequency
dependence, we find a consistent picture, with constraints from different
contexts overlapping with each other for tidal frequencies probed by multiple
analyses.

By combining multiple measurements of $\Q$ we plan to explore its dependencies
on various parameters, helping to narrow down the dominant physical mechanisms
for the dissipation in various regimes. Such a multi-faceted approach is made
possible by the recent and on-going explosion of observations (soon to be)
produced by missions like \textit{Kepler}, TESS, Gaia, PLATO etc.

In service of this long term goal, our analysis is designed to account for
observational and model uncertainties as fully as possible, taking special care
to avoid conclusions not strongly supported by observations. We also avoid
assuming any particular tidal dissipation model, relying on a general parametric
prescription instead that is able to accommodate to a reasonable degree all the
proposed models.


