\section{Tidal Evolution Model}
%
\label{sec:tidal_model}

We use an updated version of the publically available tidal evolution module
\textbf{P}lanerary \textbf{O}rbital \textbf{E}volution due to \textbf{T}ides
(POET hereafter; \citet{Penev_2014} ) to simulate the stellar and orbital
evolution under the effects of tides. POET follows stellar evolution by
interpolating among a grid of evolutionary tracks calculated using MESA
\citep{Paxton_et_al_11}. MESA splits the star into many concentric shells, but
for the purposes of POET, those are combined into two zones to track stellar
evolution: convective envelope and radiative core. The two zones are assumed to
have solid body rotation but are allowed to spin differently. They are also
assumed to converge toward synchronous rotation on what is referred to as the
core-envelope coupling timescale \citep{Irwin_et_al_07,Gallet_Bouvier_2013}, as
well as exchange mass (which carries with it its specific angular momentum).
The core-envelope coupling torques are calculated as:

\begin{equation}
%
    \mathbf{T}_\mathrm{conv,c-e}
%
    =
%
    -\mathbf{T}_\mathrm{rad,c-e}
%
    =
%
    \frac{
%
        I_\mathrm{conv}\mathbf{L}_\mathrm{rad} -
        I_\mathrm{rad}\mathbf{L}_\mathrm{conv}
%
    }{
%
        (I_\mathrm{conv}+I_\mathrm{rad})\tau_\mathrm{c-e}
%
    }
%
    -
%
    \frac{2L_\mathrm{conv}}{3I_\mathrm{conv}} R_\mathrm{rad}^2
    \dot{M}_\mathrm{rad}
%
\end{equation}

where $I_\mathrm{rad/conv}$ and $\mathbf{L}_\mathrm{rad/conv}$ are the moment of
inertia and the angular momentum vector of the core and envelope respectively,
$M_\mathrm{rad}$ and $R_\mathrm{rad}$ are the mass and radius of the radiative
core.

In addition to the change in internal structure, the angular velocity of the
convective surface of stars is also effected by the torques exerted by stellar
winds. The external torque leads to spin-down of the star at later stages of
evolution \citep{Schatzman,Irwin_Bouvier_2010,
Gallet_Bouvier_2013,Gallet_Bouvier_15}. To account for this spin-down POET
adopts a generalized formalism for the rate of angular momentum loss by assuming
winds apply torque to the convective zone in direction opposite to
$\mathbf{L}_{conv}$ with a magnitude:
%
\begin{equation}
%
    \label{eq:wind_torque}
%
    T_\mathrm{conv,wind} \equiv K\omega_\mathrm{conv}
%
    \mathrm{\min}(\omega_\mathrm{conv},\omega_\mathrm{sat})^2
%
    \left(\dfrac{R_{\star}}{R_\odot}\right)^{1/2}
%
    \left(\dfrac{M_{\star}}{M_\odot}\right)^{-1/2}
%
\end{equation}
%
where $K$ is the parameterized wind strength, $\omega_{conv}$ is the angular
frequency of the convective envelope, $\omega_{sat}$ is the frequency above
which the magnetic braking saturates, $M_\star$ and $R_\star$ are the mass and
radius of the star, and $M_\odot$ and $R_\odot$ are the present day mass and
radius of the Sun.

The orbital evolution model in POET is formulated in terms of the equilibrium
tide model, but as already pointed out in the introduction, dynamical tide
models can be incorporated by finding an effective phase lag.  The prescription
is based on the formalism of \citet{Lai_12}, but expanded to account for
eccentricity. In summary,

\begin{enumerate}
%
    \item The quadrupole moment of the tidal potential in a spherical
        $\left(\zeta, \theta, \phi\right)$-coordinate system is:
%
        \begin{equation}\label{eq:tidal_potential}
%
            U(\textbf{r},t)= - \sum_{m,m^{\prime}} U_{m,m^{\prime}} \zeta^2
            Y_{2,m}
            (\theta,\phi)\dfrac{a^3}{r^3(t)}\exp[-im^{\prime}\Delta\phi(t)]
%
        \end{equation}
%
        where a is the semi-major axis and r$\left(t\right)$ is the distance
        between the centers of the two objects at an arbitrary point on the
        orbit. $U_{mm^{\prime}}$ are the coefficients of the spherical
        harmonics.
%
%
%
    \item A Fourier series expansion is used to include effects of eccentric
        orbits:
%
        \begin{equation}
%
            \label{eq:fourier_expansion}
%
            \dfrac{a^3}{r^3(t)}\exp[-im^{\prime}\Delta\phi(t)] = \sum_s
            p_{m^{\prime},s}\exp(-is\Omega t)
%
        \end{equation}
%
        with the $p_{m^{\prime},s}$ coefficients pre-calculated as a function of
        eccentricity up to $s=400$ to allow for even extreme orbital
        eccentricities.  At e=0.9, truncating the tidal potential expansion to
        $s=200$ is accurate to better than 1 part in $10^5$. We doubled the
        maximum $s$ value for the sake of caution.

    \item Following the equilibrium theory of tides, the effects of tidal
        dissipation are introduced as a lag in the response of the fluid
        perturbation inside the star. Each $\left(ms\right)$-component of the
        tidal potential has a distinct tidal frequency ($\tilde{\omega}_{m,s} =
        s\Omega-m\Omega_s$) and thus a distinct tidal lag.  The Eulerian density
        perturbation and Lagrangian displacement is given as:
%
        \begin{align}
%
            \boldsymbol{\xi}_{m,s}(\mathbf{r},t)
%
            & =
%
            \dfrac{U_{m,s}}{\omega_0^2}
%
            \bar{\boldsymbol{\xi}}_{m,s}(\mathbf{r})
%
            \exp(-is\Omega t + i\Delta_{m,s})
%
            \label{eq:fluid}
%
            \\
%
            \delta\rho_{m,s}(\mathbf{r},t) & =
%
            \dfrac{U_{m,s}}{\omega_0^2}
%
            \delta\bar{\rho}_{m,s}(\mathbf{r})
%
            \exp(-is\Omega t + i\Delta_{m,s})
%
            \label{eq:density}
%
        \end{align}

        where $\delta\bar{\rho}_{m,s} =
        -\nabla(\rho\bar{\boldsymbol{\xi}}_{m,s})$ and
        $\omega_0\equiv\sqrt{GM/R^3}$ is the dynamical frequency of the primary
        star in the binary system.

    \item The individual $\left(ms\right)$-component of the tidal torque and
        rate of dissipation of energy are calculated as:
%
        \begin{align}
%
            \mathbf{T}_{m,s}
%
            &=
%
            \int d^3x
%
            \delta\rho_{m,s}(\mathbf{r},t)\mathbf{r}
%
            \times
%
            \left[-\nabla U^{\ast}(\mathbf{r},t)\right]
%
            \label{eq:torque}
%
            \\
%
            \dot{E}_{m,s}
%
            &=
%
            \int d^3x
%
            \rho_{m,s}(\mathbf{r})
%
            \dfrac{
%
                \partial\boldsymbol{\xi}_{m,s}(\mathbf{r},t)
%
            }{ \partial t
%
            }
%
            \left[-\nabla U^{\ast}(\mathbf{r},t)\right]
%
            \label{eq:E}
%
        \end{align}
%
    \item The evolution of the orbital parameters (semi-major axis $a$,
        eccentricity $e$, angle between spin angular momentum and orbital
    angular momentum vectors $\theta$, and angular frequency of primary star
$\omega$) in terms of torque, energy, and angular momentum is calculated using
the following equations:

        \begin{align}
%
            \dot{a} &= a\dfrac{-\dot{E}}{E} \\
%
            \dot{e} &=
            \dfrac{2(\dot{E}L+2E\dot{L})L(M+M^{\prime})}{G(MM^{\prime})^3} \\
%
            \dot{\theta} &= \dfrac{(T_z - \tilde{T}_z)\sin\theta}{L} -
            \dfrac{(T_x - \tilde{T}_x)\cos\theta}{L} -  \dfrac{T_x +
            \mathcal{T}_x}{S} \\
%
            \dot{\omega} &=  \dfrac{(T_y +
            \mathcal{T}_y)\cos\theta}{L\sin\theta} + \dfrac{T_y +
            \mathcal{T}_y}{S\sin\theta}
%
        \end{align}
%
        where  $M$ is the mass of tidally distorted star, $M^{\prime}$ is the
        mass of the companion raising the tides, $T_x, T_y, T_z$ are the
        components of tidal torque along $x, y, z$-directions, $\mathcal{T}_x,
        \mathcal{T}_y, \mathcal{T}_z$ are the non-tidal torques (i.e. stellar
        wind and core-envelope coupling), and $\tilde{T}_x, \tilde{T}_y,
        \tilde{T}_z$ are the tidal torques on the orbit due to other zones.  The
        contribution of both stars to the orbital evolution is included.
%
\end{enumerate}

The relation between the tidal quality factor and the phase lag is:

\begin{equation}
%
    \label{eq:tidal_lag_relation}
%
    \Q = \dfrac{15}{16\pi\Delta_{\star}^{\prime}}
%
\end{equation}

The tidal lag introduced in the above method allows for an arbitrary
parameterization of tidal dissipation depending on the model being studied.  For
equilibrium tide models with constant phase lag, $\Q$ has a fixed value, while
for for dynamical tide models or variable phase lag equilibrium models, $\Q$
will in general depend on the orbital and stellar properties, as well as the
particular tidal potential term being dissipated. In this work, we restrict our
investigation to stars with similar internal structure, and only allow for a
dependence on tidal frequency.


