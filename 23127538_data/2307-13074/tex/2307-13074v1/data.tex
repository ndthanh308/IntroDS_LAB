\section{Input Data}
%
\label{sec:input_data}

\subsection*{Stellar Rotational Periods}
%
\label{sec:rotation_periods}

We wish to infer the infer the parameters of the $Q_{\ast}^{\prime}$ model using
the observed rotation period of the primary star in binary star systems. We
model the spin of the star as a Gaussian distribution whose mean and standard
deviation are estimated as described below.

\citet{Lurie_2017} performed an extensive study of the light curves of the
eclipsing binaries available in the \textit{Kepler Eclipsing Binary Catalog} and
estimated the rotation period of the primary stars showing starspot modulations
of their brightness.

The authors analyzed data available from \textit{Kepler} Data release 25 in
quarters 0-17. The Presearch Data Conditioning (PDC) pipeline suppresses stellar
variability with $P \gtrsim  20\,\text{days}$. Hence, rotation periods are
available only for stars with $P < 20 \text{ days}$.

The morphology of stellar variability for single stars is a well-modeled
phenomenon. The authors use the out-of-eclipse region of the light curves to
analyze the variability of stars in eclipsing binaries. Depending on the
variations, they classify the light curves into six categories by visual
inspection: starspot modulations, ellipsoidal modulations, pulsations, other
periodic variability (do not belong to previous three or unidentified Heartbeat
stars), non-periodic out-of-eclipse (flat-light curves or long-term smooth
variation due to instrumental artifacts) and not eclipsing binary (lack of clear
eclipses). Out of these, only starspot modulations can provide information about
the rotation period of the stars. Starspot modulations are quasi-sinusoidal,
with differential rotation and the formation and dissipation of starspots
introducing long-term modulation \citep{Davenport2015}. The authors identify 816
eclipsing binaries that show such variations.

Two common ways to detect periodicity of variability in a time-variable signal
are auto-correlation function (ACF) and Lomb-Scargle Periodogram
\citep{lombscargle1976}. Following \citet{McQuillan_2013}, \citet{Lurie_2017}
identifies an initial rotation period for each binary using ACF. Since ACF is
not sensitive to multiple rotation period peaks in the light curve signal, it
can only serve as a validation for signals with multiple periods present due to
the differential rotation of the star. For this purpose, the authors turn to
Lomb-Scargle Periodograms in order to detect more than one peak of periodic
variability in the out-of-eclipse light curves using a procedure by
\citet{Rappaport2014}, with ACF used for validation.

From the Lomb-Scargle Periodogram, the authors identified the two most
significant peaks, imposing a threshold that peaks must be at least $30\%$ of
the highest peak. Each peak is then classified into two groups. Lowering this
threshold results in the detection of spurious signals which might not be
related to stellar variability. A window of frequency is selected for each group
by selecting the local minima in a smooth periodogram on the right and left of
the dominant peak. Inside this window, they again use the same threshold to
identify the next dominant subpeak. Hence, for each system, there are a maximum
of four possible solutions for the rotation period. Table 2
of~\citet{Lurie_2017} shows the periods and their respective peak height along
with the period and peak height detected by the ACF. The ACF period comes
closest to the most dominant peak in the periodogram.

As the authors themselves mention, the rotation period corresponding to the
highest peak obtained in the first group ($P_{1\min}$ in Table 2
of~\cite{Lurie_2017}) will be the closest to the equatorial rotation period. We
adopt this as the nominal value of the rotation period of the primary star
in our analysis. To obtain the uncertainty on the rotation period, we select the
rotation period corresponding to the subpeak in the same group ($P_{1\max}$ in
Table 2 of~\cite{Lurie_2017}) and calculate the difference between $P_{1\max}$
and $P_{1\min}$. See Sec. \ref{sec:reliable_constraints} for an additional
consideration we apply to account for the possibility that this procedure may
under-estimate the rotational period uncertainty.

\subsection*{Binary Parameters}

\begin{table}
%
    \centering
%
    \caption{The maximum likelihood parameters of the binary systems used in
    this analysis and their spins.}
%
    \label{tab:system_parameters}
%
    \resizebox{0.49\textwidth}{!}{%
%
        \begin{tabular}{cccccccc}
%
    \hline
%
    \textbf{KIC}
    & $\mathbf{M}_{\mathbf{1}} [M_\odot]$
    & $\mathbf{M}_{\mathbf{2}} [M_\odot]$
    & $\mathbf{e}\textbf{sin}\boldsymbol{\omega}$
    & $\mathbf{e}\textbf{cos}\boldsymbol{\omega}$
    & $\mathbf{P_{orb}} [\mathrm{days}]$
    & $\mathbf{P_\star} [\mathrm{days}]$
    & $\boldsymbol{\sigma}_{\mathbf{P_\star}} [\mathrm{days}]$\\
%
    \hline

%
    6962018
    & 0.75
    & 0.69
    & 0.0002
    & 0.00024
    & 1.27
    & 1.3
    & 0.003\\%

%
    11616200
    & 1.16
    & 0.67
    & -0.014
    & 0.00019
    & 1.72
    & 1.7
    & 0.003\\%

%
    7798259
    & 0.73
    & 0.60
    & -0.039
    & 7.6e-07
    & 1.73
    & 1.7
    & 0.019\\%

%
    4380283
    & 1.01
    & 0.88
    & -4.9e-06
    & 7.8e-05
    & 1.74
    & 1.7
    & 0.003\\%

%
    7732791
    & 0.64
    & 0.65
    & 0.039
    & 0.00025
    & 2.06
    & 2.0
    & 0.021\\%

%
    4579321
    & 0.81
    & 0.61
    & -0.03
    & -8.6e-05
    & 2.11
    & 2.1
    & 0.03\\%

%
    11200773
    & 1.09
    & 0.42
    & -0.2
    & 0.00037
    & 2.49
    & 2.5
    & 0.033\\%

%
    9656543
    & 0.79
    & 0.76
    & -0.0041
    & 8.4e-05
    & 2.54
    & 2.5
    & 0.13\\%

%
    3834364
    & 1.03
    & 0.51
    & -0.15
    & 0.00025
    & 2.91
    & 2.9
    & 0.025\\%

%
    11228612
    & 0.99
    & 0.70
    & 0.0044
    & 4.2e-05
    & 2.98
    & 2.9
    & 0.007\\%

%
    6312521
    & 1.02
    & 0.76
    & -0.091
    & 0.00016
    & 3.02
    & 3.0
    & 0.045\\%

%
    10960995
    & 0.94
    & 0.83
    & -5.9e-08
    & 1.2e-07
    & 3.12
    & 3.0
    & 0.11\\%

%
    11147276
    & 1.18
    & 0.89
    & -0.021
    & -6e-06
    & 3.13
    & 3.1
    & 0.054\\%

%
    10091257
    & 1.00
    & 1.06
    & 0.021
    & 1.4e-05
    & 3.4
    & 3.3
    & 0.06\\%

%
    6525196
    & 0.91
    & 0.85
    & -5.4e-05
    & 4.7e-05
    & 3.42
    & 3.3
    & 0.085\\%

%
    5022440
    & 1.10
    & 0.73
    & 4.7e-07
    & -1.1e-06
    & 3.69
    & 3.6
    & 0.13\\%

%
    5802470
    & 1.13
    & 0.77
    & 2.7e-06
    & 2.2e-05
    & 3.79
    & 3.8
    & 0.073\\%

%
    4815612
    & 1.17
    & 1.00
    & -5.3e-08
    & -1.6e-07
    & 3.86
    & 3.7
    & 0.083\\%

%
    3241344
    & 1.01
    & 0.51
    & -0.0055
    & -0.0041
    & 3.91
    & 3.9
    & 0.014\\%

%
    11403216
    & 1.07
    & 0.45
    & -0.036
    & 0.00034
    & 4.05
    & 4.0
    & 0.019\\%

%
    10935310
    & 0.84
    & 0.54
    & 0.17
    & -0.0018
    & 4.13
    & 3.9
    & 0.014\\%

%
    10031409
    & 1.16
    & 1.11
    & 0.004
    & 6e-06
    & 4.14
    & 4.0
    & 0.17\\%

%
    7838639
    & 1.21
    & 1.08
    & -0.0084
    & 9.2e-06
    & 4.23
    & 4.1
    & 0.081\\%

%
    3973504
    & 0.88
    & 0.57
    & -0.1
    & -4.9e-05
    & 4.32
    & 4.3
    & 0.032\\%

%
    8957954
    & 1.03
    & 1.01
    & -2.5e-08
    & -1.9e-08
    & 4.36
    & 4.3
    & 0.086\\%

%
    11252617
    & 0.88
    & 0.47
    & 0.24
    & 3.6e-05
    & 4.48
    & 4.5
    & 0.05\\%

%
    4285087
    & 0.99
    & 0.97
    & -2e-05
    & -5.5e-05
    & 4.49
    & 4.5
    & 0.049\\%

%
    4346875
    & 0.94
    & 0.55
    & -0.022
    & -9.8e-06
    & 4.69
    & 4.8
    & 0.11\\%

%
    4947726
    & 1.14
    & 0.53
    & -0.15
    & 0.079
    & 4.73
    & 4.4
    & 0.21\\%

%
    11233911
    & 0.84
    & 0.94
    & 0.016
    & 0.00018
    & 4.96
    & 4.8
    & 0.076\\%

%
    12004679
    & 0.92
    & 0.88
    & -0.0093
    & 0.0018
    & 5.04
    & 5.6
    & 0.21\\%

%
    2860788
    & 1.03
    & 0.83
    & -0.0045
    & -6.1e-06
    & 5.26
    & 5.0
    & 0.084\\%

%
    9971475
    & 1.10
    & 0.80
    & -0.00033
    & 4.6e-05
    & 5.36
    & 2.0
    & 0.012\\%

%
    5730394
    & 1.17
    & 0.95
    & -0.019
    & -0.00025
    & 5.52
    & 6.3
    & 0.38\\%

%
    5181455
    & 0.96
    & 0.54
    & -1.7e-06
    & -3.6e-07
    & 5.58
    & 5.5
    & 0.11\\%

%
    8381592
    & 1.01
    & 0.73
    & -0.064
    & -0.0044
    & 5.78
    & 5.7
    & 0.088\\%

%
    8618226
    & 1.08
    & 0.70
    & 1.5e-05
    & 0.0002
    & 5.88
    & 6.0
    & 0.25\\%

%
    3838496
    & 0.99
    & 1.07
    & -0.0056
    & 0.00079
    & 5.98
    & 5.7
    & 0.061\\%

%
    10385682
    & 1.00
    & 0.99
    & -0.015
    & -5.9e-06
    & 6.21
    & 6.1
    & 0.048\\%

%
    8580438
    & 0.98
    & 0.43
    & 0.00062
    & 0.002
    & 6.5
    & 7.5
    & 0.19\\%

%
    10965963
    & 1.08
    & 1.00
    & 0.045
    & -0.036
    & 6.64
    & 6.2
    & 0.41\\%

%
    8746310
    & 0.80
    & 0.47
    & -2.5e-06
    & 6.7e-05
    & 6.86
    & 8.3
    & 0.4\\%

%
    7362852
    & 1.16
    & 1.11
    & 0.0036
    & -0.0015
    & 7.04
    & 7.1
    & 0.24\\%

%
    8543278
    & 0.82
    & 0.71
    & 0.023
    & 0.013
    & 7.55
    & 8.6
    & 0.51\\%

%
    6927629
    & 1.08
    & 0.97
    & 1.4e-05
    & 2.7e-05
    & 7.74
    & 8.9
    & 0.27\\%

%
    8364119
    & 0.92
    & 0.88
    & 0.016
    & 0.018
    & 7.74
    & 9.0
    & 0.22\\%

%
    6949550
    & 0.88
    & 0.88
    & -0.011
    & -0.26
    & 7.84
    & 6.9
    & 0.39\\%

%
    3348093
    & 0.65
    & 0.68
    & 0.26
    & 0.0069
    & 7.96
    & 8.2
    & 0.27\\%

%
    9532123
    & 0.99
    & 0.75
    & -0.097
    & -0.21
    & 8.21
    & 7.3
    & 0.37\\%

%
    9892471
    & 0.91
    & 0.56
    & -0.013
    & -0.00029
    & 8.27
    & 9.4
    & 0.29\\%

%
    7816680
    & 1.18
    & 0.96
    & -0.018
    & 1.9e-05
    & 8.59
    & 11.1
    & 0.21\\%

%
    11391181
    & 0.87
    & 0.82
    & 0.077
    & 0.17
    & 8.62
    & 7.9
    & 0.28\\%

%
    4839180
    & 1.08
    & 0.86
    & -0.0021
    & -0.0025
    & 8.85
    & 11.0
    & 1.3\\%

%
    5652260
    & 0.88
    & 0.74
    & 0.001
    & 0.0012
    & 8.93
    & 11.0
    & 1.2\\%

%
    11232745
    & 0.73
    & 0.65
    & 3.6e-08
    & -3.9e-08
    & 9.63
    & 12.7
    & 0.75\\%

%
    8984706
    & 1.04
    & 1.03
    & -0.013
    & 0.015
    & 10.1
    & 12.5
    & 1.3\\%

%
    5393558
    & 0.92
    & 1.04
    & 0.23
    & -0.023
    & 10.2
    & 9.4
    & 0.48\\%

%
    9353182
    & 1.14
    & 0.57
    & -0.042
    & 0.036
    & 10.5
    & 10.5
    & 0.22\\%

%
    4352168
    & 1.03
    & 0.73
    & -0.15
    & -0.15
    & 10.6
    & 10.2
    & 0.6\\%

%
    7336754
    & 1.14
    & 0.73
    & 0.016
    & 0.012
    & 12.2
    & 13.2
    & 1\\%

%
    6029130
    & 0.92
    & 0.87
    & 0.0099
    & 0.011
    & 12.6
    & 17.3
    & 0.33\\%

%
    5871918
    & 0.67
    & 0.47
    & -0.32
    & -0.15
    & 12.6
    & 12.4
    & 0.16\\%

%
    11704044
    & 0.84
    & 0.82
    & 1e-06
    & 2.7e-06
    & 13.8
    & 19.4
    & 1.1\\%

%
    6359798
    & 0.86
    & 0.79
    & 0.079
    & -0.41
    & 14.2
    & 3.4
    & 0.029\\%

%
    10936427
    & 1.08
    & 1.12
    & -0.0074
    & 0.0016
    & 14.4
    & 16.6
    & 1.1\\%

%
    4678171
    & 0.68
    & 0.59
    & 0.0025
    & -0.0035
    & 15.3
    & 20.3
    & 5.4\\%

%
    7987749
    & 0.90
    & 0.79
    & 0.043
    & -0.14
    & 17
    & 17.5
    & 0.45\\%

%
    8356054
    & 0.77
    & 0.71
    & 0.37
    & 0.027
    & 17.1
    & 16.7
    & 2.3\\%

%
    10330495
    & 1.03
    & 0.52
    & -0.056
    & -0.04
    & 18.1
    & 21.1
    & 2.3\\%

%
    10992733
    & 0.94
    & 0.83
    & 0.17
    & 0.34
    & 18.5
    & 15.6
    & 0.27\\%

%
    10711913
    & 0.88
    & 0.76
    & 0.28
    & -0.077
    & 19.4
    & 18.6
    & 1.1\\%

%
    4252226
    & 0.99
    & 0.92
    & 0.13
    & -0.47
    & 21.9
    & 11.9
    & 0.76\\%

%
    10215422
    & 1.12
    & 0.72
    & 0.0014
    & 0.29
    & 24.8
    & 24.8
    & 2.2\\%

%
    4773155
    & 1.03
    & 0.95
    & -0.36
    & 0.26
    & 25.7
    & 19.0
    & 2\\%
%
	\hline
%
\end{tabular}
%
    }
%
\end{table}

\citet{Windemuth_2019} (W19 from now on) inferred the values of orbital and
stellar parameters for detached eclipsing binaries in the \textit{Kepler} field
from the available photometric data. The authors selected 2877 eclipsing and
ellipsoidal binaries available in the Villanova \textit{Kepler} EB Catalogue
\citep{Prsa2011, Kirk2016}. In order to get a catalog of well constrained
detached eclipsing binaries, they first selected binaries with eclipse depths
for the primary and secondary stars to be $>5\%$ and $>0.1\%$ of the normalized
flux to ensure high signal to noise ratio. Next, they used the morphology
parameter, provided by VKEBC, as $\texttt{morph}<0.6$ to ensure the binaries in
the sample are detached. The morphology parameter is based on a scheme that
classifies light curves based on the type of binary star observed. Low values of
this parameter correspond to a well-detached binary exhibiting clear separation
between the eclipses, while higher values correspond to over-contact binary
systems with sinusoidal variations \citep{Matijevi2012}.

The orbital parameters and the stellar parameters of these systems are inferred
by performing a simultaneous fit for the Light Curve (LC) and the Spectral
Energy Distribution (SED). The observational uncertainties are then propagated
to the model parameters using MCMC simulations with a likelihood function based
on how well the LC, SED, extinction and distance are simultaneously reproduced.
The authors performed this fit for 22 parameters. For our analyses, only the
following parameters are of interest:
%
\begin{itemize}
%
    \item Sum of masses of the primary and secondary star $\left(M_{sum}\right)$
%
    \item Ratio of secondary mass to the primary mass  $\left(q\right)$
%
    \item Age of the system $\left(\tau\right)$
%
    \item Metallicity $\left(\left[\text{Fe}/\text{H}\right]\right)$, assumed to
        be the same for both stars
%
    \item Eccentricity components of the orbit $\left(e\cos{\omega},
        e\sin{\omega}\right)$.
%
\end{itemize}

W19 provided posterior samples for 728 binaries. We restrict this sample
further. First, since our analysis relies on the spin period of the primary
star, we perform a cross-match between these 728 binaries and binaries available
in \citet{Lurie_2017}. We also apply limits on the stellar masses and
metallicity ($ 0.4 M_{\odot} < M_{\star} < 1.2 M_{\odot} $ and $-1.014 <
\left[\text{Fe}/\text{H}\right] <0.537$), which further reduced our dataset.
The mass limits were imposed to select only stars that have similar main
sequence internal structure to the Sun, namely a significant convective outer
shell, but not fully convective. The metallicity limits are driven by
limitations imposed by POET, which at the moment is only capable of
interpolating the stellar evolution within these mass and metallicity limits.
Finally, we require reasonable orbital eccentricity $e<0.8$ and orbital period
of no more than 20 days.

Our final dataset contains 70 binary systems. We use the publicly available
thinned posterior samples for our parameters of interest as observed probability
distributions for Bayesian analysis. Table \ref{tab:system_parameters} gives the
maximum likelihood values (as reported by W19) of the mass of the primary star
$\left(M_1\right)$, mass of the secondary star $\left(M_2\right)$, and
eccentricity components of the orbit ($e\sin\omega$, $e\cos\omega$), along with
the orbital period of the system, rotation period of the primary star and the
$1\sigma-$uncertainty assumed for the spin period (see Sec.
\ref{sec:rotation_periods}). Note that for each binary only a single spin period
is measured, which we assume is that of the primary (brighter) star.
