% mnras_template.tex
%
% LaTeX template for creating an MNRAS paper
%
% v3.0 released 14 May 2015
% (version numbers match those of mnras.cls)
%
% Copyright (C) Royal starronomical Society 2015
% Authors:
% Keith T. Smith (Royal starronomical Society)

% Change log
%
% v3.0 May 2015
%    Renamed to match the new package name
%    Version number matches mnras.cls
%    A few minor tweaks to wording
% v1.0 September 2013
%    Beta testing only - never publicly released
%    First version: a simple (ish) template for creating an MNRAS paper

%%%%%%%%%%%%%%%%%%%%%%%%%%%%%%%%%%%%%%%%%%%%%%%%%%
% Basic setup. Most papers should leave these options alone.
\documentclass[fleqn,usenatbib]{mnras}

% MNRAS is set in Times font. If you don't have this installed (most LaTeX
% installations will be fine) or prefer the old Computer Modern fonts, comment
% out the following line

% Depending on your LaTeX fonts installation, you might get better results with one of these:
%\usepackage{mathptmx}
%\usepackage{txfonts}

% Use vector fonts, so it zooms properly in on-screen viewing software
% Don't change these lines unless you know what you are doing
\usepackage[T1]{fontenc}

% Allow "Thomas van Noord" and "Simon de Laguarde" and alike to be sorted by "N" and "L" etc. in the bibliography.
% Write the name in the bibliography as "\VAN{Noord}{Van}{van} Noord, Thomas"
\DeclareRobustCommand{\VAN}[3]{#2}
\let\VANthebibliography\thebibliography\def\thebibliography{\DeclareRobustCommand{\VAN}[3]{##3}\VANthebibliography}


%%%%% AUTHORS - PLACE YOUR OWN PACKAGES HERE %%%%%

% Only include extra packages if you really need them. Common packages are:
\usepackage{amsmath}	% Advanced maths commands
\usepackage{amssymb}	% Extra maths symbols
\usepackage{newtxtext,newtxmath}
\usepackage{graphicx}
\usepackage{rotating}
\usepackage{caption}
\usepackage{subcaption}
\usepackage{algorithmic}
\usepackage{adjustbox}
\usepackage{longtable}
\usepackage{multirow}
\usepackage{array}
% \newcolumntype{P}[1]{>{\centering\arraybackslash}p{#1}}
% \usepackage[font=small,labelfont=bf]{caption}
\usepackage{pdflscape}
\usepackage{afterpage}
\usepackage{color,soul}
\usepackage{float}


%%%%%%%%%%%%%%%%%%%%%%%%%%%%%%%%%%%%%%%%%%%%%%%%%%

%%%%% AUTHORS - PLACE YOUR OWN COMMANDS HERE %%%%%

% Please keep new commands to a minimum, and use \newcommand not \def to avoid
% overwriting existing commands. Example:
\newcommand{\feh}{{\left[\frac{\mathrm{Fe}}{\mathrm{H}}\right]}}
\newcommand{\Q}{{Q_{\star}^{\prime}}}

%%%%%%%%%%%%%%%%%%%%%%%%%%%%%%%%%%%%%%%%%%%%%%%%%%

%%%%%%%%%%%%%%%%%%% TITLE PAGE %%%%%%%%%%%%%%%%%%%

% Title of the paper, and the short title which is used in the headers.  Keep
% the title short and informative.
\title[$\Q$ using Tidal Synchronization]{Constraints on Tidal
Quality Factor in Kepler Eclipsing Binaries using Tidal Synchronization: A
Frequency-Dependent Approach}

% The list of authors, and the short list which is used in the headers.
% If you need two or more lines of authors, add an extra line using \newauthor
\author[Ruskin Patel]{
Ruskin Patel,$^{1}$\thanks{E-mail: ruskin.patel@utdallas.edu}
Kaloyan Penev,$^{1}$\thanks{E-mail: kaloyan.penev@utdallas.edu}
Joshua Schussler$^{1}$
\\
% List of institutions
$^{1}$Department of Physics, University of Texas at Dallas, Richardson, Texas\\
}

% These dates will be filled out by the publisher
\date{Accepted XXX. Received YYY; in original form ZZZ}

% Enter the current year, for the copyright statements etc.
\pubyear{2022}

% Don't change these lines
\begin{document}\label{firstpage}
\pagerange{\pageref{firstpage}--\pageref{lstarpage}}

\maketitle

% Abstract of the paper
\begin{abstract}
%
    Tidal dissipation in binary systems is the primary source for
    synchronization and circularization of the objects in the system. The
    efficiency of the dissipation of tidal energy inside stars or planets
    results in significant changes in observed properties of the binary system
    and is often studied empirically using a parameter, commonly known as the
    modified tidal quality factor ($\Q$). Though often assumed constant, in
    general that parameter will depend on the particular tidal wave experiencing
    the dissipation and the properties of the tidally distorted object. In this
    work we study the frequency dependence of $\Q$ for Sun-like stars. We
    parameterize $\Q$ as a saturating power-law in tidal frequency and obtain
    constraints using the stellar rotation period of 70 eclipsing binaries
    observed by Kepler. We use Bayesian analysis to account for the
    uncertainties in the observational data required for tidal evolution. Our
    analysis shows that $\Q$ is well constrained for tidal periods $ > 15$\,
    days, with a value of $\Q\sim10^8$ for periods $> 30$\,days and a slight
    suggested decrease at shorter periods. For tidal periods $< 15$\,days,
    $\Q$ is no longer tightly constrained, allowing for a broad range of
    possible values that overlaps with the constraints obtained using tidal
    circularization in binaries, which point to much more efficient dissipation:
    $\Q\sim10^6$.
%
\end{abstract}

% Select between one and six entries from the list of approved keywords.
% Don't make up new ones.
\begin{keywords}
    (stars:) binaries: eclipsing--stars: solar-type--stars: rotation--stars:
    kinematics and dynamics--stars: interiors
\end{keywords}

%%%%%%%%%%%%%%%%%%%%%%%%%%%%%%%%%%%%%%%%%%%%%%%%%%

%%%%%%%%%%%%%%%%% BODY OF PAPER %%%%%%%%%%%%%%%%%%

\section{Introduction}
Current quantum hardware is unable to carry out universal quantum computations due to the buildup of errors that occur during the computation. 
The magnitude of the individual error is currently above the value that the Threshold Theorem requires in order to kick-start quantum error correction and fault-tolerant quantum computation~\cite[Section 10.6]{nielsen_chuang_2010}. 
Although the experimentally achieved fidelity rates are promising and the error bounds are inching closer to the required threshold, we will have to work for the foreseeable future with quantum hardware with errors that build-up during the computation.  This implies that we can only do a limited number of steps before the output of the computation has become completely uncorrelated with the intended one.

For fault-tolerant quantum computing, we repeat four steps: 
1) We apply a number of single and two-qubit quantum gates, in parallel whenever possible; 
2) We perform a syndrome measurement on a subset of the qubits; 
3) We perform fast classical computations to determine which errors have occurred and how to correct them; 
and, 4) We apply correction terms based on the classical computations.
We then repeat these four steps with a next sequence of gates. 
These four steps are essential to fault-tolerant quantum computing. 


The starting point of this work is to use the four steps outlined above, not to carry out error correction and fault-tolerant computation, but to enhance short, constant-depth, {\em uncorrected} quantum circuits that perform single qubit gates and {\em nearest-neighbor} two qubit gates. 
Since in the long run we will have to implement error-correction and fault-tolerant computation anyhow, and this is done by such a four-step process, why not make other use of this architecture? Moreover, on some of the quantum hardware platforms, these operations are already in place.
Embracing this idea we naturally arrive at the question: what is the computational power of \textit{low-depth} quantum-classical circuits organized as in the four steps outlined above? 
We thus investigate circuits that execute a small, ideally constant, number of stages, where at each stage we may apply, in parallel, single qubit gates and {\em nearest-neighbor} two qubit gates, followed by measurements, followed by low-depth classical computations of which the outcome can control quantum gates in later stages. 
It is not clear, at first, whether such circuits, especially with constant depth, can do anything remotely useful. 
But we will see that this is indeed the case: many quantum computations can be done by such circuits in constant depth. 
By parallelizing quantum computations in this way, we improve the overall computational capabilities of these circuits, as we do not incur errors on qubits that are idle, simply because qubits are not idle for a very long time. 
Furthermore, reducing the depth of quantum circuits, at the cost of increasing width, allows the circuit to be run faster even if errors occur.

The first usage of such a four-step layout, not to do error correction, but to perform computations, can be found in the paradigm of measurement-based quantum computing~\cite{gottesman1999demonstrating,raussendorf2001one,jozsa2006introduction,clark2007generalised}: 
A universal form of quantum computing where a quantum state is prepared and operations are performed by measuring qubits in different bases, depending on previous measurements and intermediate measurements.

\citeauthor{PhamSvore2013} were the first to formalize the four-step protocol for performing computations~\cite{PhamSvore2013}. They included specific hardware topologies by considering two-dimensional graphs for imposing constraints on qubit interactions. In their model, they develop circuits for particularly useful multi-qubit gates, including specifying costs in the width, number of qubits, depth, number of concurrent time steps, size, and total number of non-Identity operations.
As a result, they find an algorithm that factors integers in polylogarithmic depth.
\citeauthor{Browne:2011} showed that the main tool in the work by \citeauthor{PhamSvore2013}, the fan-out gate, can also be replaced by additional log-depth classical computations in the measurement-based quantum computing setting~\cite{Browne:2011}.

More recently, \citeauthor{Cirac:2021} introduced a scheme to implement unitary operations involving quantum circuits combined with Local Operations and Classical Communication ($\mathsf{LOCC}$) channels: $\mathsf{LOCC}$-assisted quantum circuits~\cite{Cirac:2021}. Similarly to the four-step scheme we just described, they allow for a short depth circuit to be run on the qubits, followed by one round of $\mathsf{LOCC}$, in which ancilla qubits are measured and local unitaries are applied based on the measurement outcomes. They show that in this model any 1D transitionally invariant matrix-product state (MPS) with fixed bond dimension is in the same phase of matter as the trivial state. Similar ideas can be found in~\cite{TVV_NonAbelianTopologicalOrder_2022, tantivasadakarn2021long}.

In this work, we introduce a new model, called \textit{Local Alternating Quantum-Classical Computations} ($\LAQCC$). In this model we alternate between running quantum circuits (constrained by locality), ending in the measurement of a subset of qubits, and fast classical computations based on the measurement results. The outcome of the classical computations are then used to control future quantum circuits. We allow for flexibility in this model, by giving different constraints to the power of both the quantum circuits and the classical circuits as well as the number of alternations between them. 
Most attention will be given to $\LAQCC$ containing quantum circuits of constant depth, classical circuits of logarithmic depth and at most a constant number of alternations between them. 
Any circuit constructed in this model is considered to be of constant depth. 
We restrict ourselves to logarithmic depth classical computations, as this is the first natural and non-trivial extension beyond constant-depth classical computations. 
Constant-depth classical computations do however also have an equivalent constant-depth quantum implementation.

The definition of $\LAQCC$ sharpens the original definition of \citeauthor{PhamSvore2013} by adding constraints to the intermediate classical computations. This allows us to bound the power of $\LAQCC$ from above. 

The main result of \citeauthor{Cirac:2021}, that 1D translational invariant MPS with fixed bond dimension can be prepared by $\mathsf{LOCC}$-assisted circuits, relies on local symmetries of the MPS. These symmetries allow them to prepare local states (on a constant number of qubits) and glue them together by doing one round of the appropriate entangling measurement and corrections, after which they run a round of local unitaries to get the desired result. This general scheme for preparing states that exhibit an MPS description with the appropriate local symmetries requires only geometrically local unitaries and one round of measurement and corrections an therefore is accessible in $\LAQCC$. Studying different local symmetries, known as Symmetry Protected Topological (SPT) phases of matter, to find measurement-based constant depth circuits for states is a broad ongoing field of research~\cite{TVV_NonAbelianTopologicalOrder_2022, tantivasadakarn2021long, smith2023deterministic}. 
All these schemes have a $\LAQCC$ implementation.

%$\LAQCC$-circuits also exist for general schemes of preparing local states, based on the local tensors, and gluing them together using one round of entangled measurement and corrections, based on the local symmetry. 
%The main result of \citeauthor{Cirac:2021}, that 1D translational invariant MPS with fixed bond dimension can be prepared by $\mathsf{LOCC}$-assisted circuits, relies heavily on local symmetries of the MPS and as a result also has an equivalent $\LAQCC$ implementation. 
%The corrections applied after the measurement round are local unitaries depending on the local symmetries of the MPS. 

 

%This general scheme of preparing local states, based on the local tensors, and gluing it together by doing one round of entangled measurement and corrections, based on the local symmetry, is accessible in $\LAQCC$.
Note however that \citeauthor{Cirac:2021} also suggest a circuit for the $W$-state.
This circuit uses sequentially and dependent measurement-based corrections of the ancilla qubits. 
These dependent measurements translate to sequential alternations between the quantum and classical circuits and therefore increase the total depth to linear depth, exceeding the constant-depth constraints imposed by $\LAQCC$-circuits. 

We study the power of the $\LAQCC$ model with respect to state preparation, showing that even with only constant quantum-depth and logarithmic classical depth it remains possible to prepare states with long-range entanglement.
Another surprising result is that it is unlikely that $\LAQCC$ circuits are classically simulatable. We show that any instantaneous quantum polynomial-time (IQP) circuit~\cite{Bremner2010,Shepherd2009} has an $\LAQCC$ implementation.
Classical simulation of IQP circuits implies the collapse of the polynomial hierarchy to the third level, which is not believed to be true~\cite{Bremner2017}. Therefore, we expect that $\LAQCC$ circuits are unlikely to be classically simulatable. We bound the power of $\LAQCC$ by showing that it is contained in $\QNC^1$, the class of polynomial-size, log-depth circuits.

Next, we also study the power that intermediate classical calculations can add to quantum computations, by considering a new model that alternates between polynomially many polynomial-depth quantum circuits and unbounded classical computations
We study this model by doing a complexity theoretical analysis, where we draw inspiration from the notions of complexity given by \citeauthor{RosenthalYuen:2022}, \citeauthor{MetgerYuen:2023}, and \citeauthor{Aaronson:2004}.
All three complexity notions are based on the notion of state preparation, instead of more traditional definition of complexity such as the decidability of a computational problem. 
The first two consider classes based on sequences of quantum states preparable by a polynomial-sized quantum circuit, where the circuits are uniformly generated by a computational class, for instance, the class $\mathsf{PSPACE}$, which results in the complexity class $\mathsf{StatePSPACE}$~\cite{RosenthalYuen:2022,MetgerYuen:2023}.
The third notion considers a relative complexity, where the complexity is measured between two given states, and is measured by the number of gates, from a given gate-set, required to transform one state in another state~\cite{Aaronson:2004}. 
For our definition of state preparation complexity, we drop the uniformity constraint from~\cite{RosenthalYuen:2022,MetgerYuen:2023} and define a class as $\mathsf{StateX}$, which refers to states preparable by circuits of type $\mathsf{X}$. 
As an example, if $\mathsf{X} = \QNC^0$, this results in the class $\mathsf{StateQNC^0}$, which is the set of states preparable from the $\ket{0}^n$ state by poly-size constant-depth circuits. 
This notion is similar to the relative complexity from~\cite{Aaronson:2004}, where one state is the  $\ket{0}^n$ state and instead of counting the number of gates we consider the set of states preparable by a fixed number of gates. Using this notion of complexity we show that any state preparable by an $\LAQCC^*$ circuit is also preparable by a $\mathsf{PostQPoly}$ circuit, the class of circuits of polynomial depth with an additional post-selection gate. 

All Clifford circuits have a constant-depth $\LAQCC$ implementation, implying that any stabilizer state can be implemented by a constant-depth $\LAQCC$ circuit, see Section~\ref{sec:clifford_circuits} for a proof of this statement. 
Efficient circuits for stabilizer states have been known already through measurement-based quantum computing. Therefore this paper focuses on the preparation of non-stabilizer states, and as a surprising result we find novel constant-depth protocols for four very natural classes of non-stabilizer states.
Despite the extensive research into these four classes of non-stabilizer states and the many applications of them, no efficient constant- or low-depth state preparation protocols are known yet. We specifically consider these four classes as they are all often used as initial states in other algorithms.

The first state is a uniform superposition over an arbitrary number of states. 
This state finds applications in many quantum algorithms, as they often start with a uniform superposition over multiple states. 
This superposition is often achieved by applying Hadamard gates to every qubit due to its simplicity to prepare. 
Yet, the analysis of many algorithms, such as Shor's algorithm~\cite{Shor:1997}, would benefit from a different initial superposition. 
The circuit to prepare the uniform superposition over an arbitrary number of states uses an exact version of Grover search as a subroutine, that turns a probabilistic circuit, with a known constant probability of success, into a deterministic circuit. 
We use the circuit for preparing a uniform superposition over an arbitrary number of states as a subroutine in the next two quantum state preparation protocols. 

The second state is the $W$-state, the uniform superposition over all computational basis states of Hamming-weight~$1$, a natural long-ranged entangled state that displays a fundamentally nonequivalent type of entanglement from the Greenberger–Horne–Zeilinger state~\cite{WState:2000}, for which $\LAQCC$-type constant-depth circuits were previously known~\cite{PhamSvore2013, Cirac:2021}. 
The $W$-state is often used as benchmark for new quantum hardware~\cite{Haffner2005,Neeley2010,GarciaPerez:2021}. 
A novel way to prepare the $W$-state therefore gives a new way to benchmark different quantum devices with each other. 
A circuit for preparing the $W$-state was given in~\cite{Cirac:2021}, but this implementation requires sequentially alternating measurements followed by local unitaries, which in the $\LAQCC$ model is not considered to be of constant depth. 
We improve this protocol by giving an $\LAQCC$ implementation of the $W$-state, based on a compress-uncompress method that links the one-hot and binary encoding of integers.

The third state considered is the Dicke state, a generalization of the $W$-state, a superposition over all computational basis states with Hamming-weight $k$~\cite{Dicke:1954}. 
Dicke states have relevance in various practical settings.
For instance, for quantum game theory~\cite{zdemir2007}, quantum storage~\cite{Bacon_Compress:2006,Plesch:2010}, quantum error correction~\cite{ouyang2014permutation}, quantum metrology~\cite{toth2012multipartite}, and quantum networking~\cite{prevedel2009experimental}. 
Dicke states have been used as a starting state for variational optimization algorithms, most notably Quantum Alternating Operator Ansatz (QAOA)~\cite{Hadfield2019}, to find solutions to problems such as Maximum k-vertex Cover~\cite{Brandhofer2022,cook2020quantum}.
The ground states of physical Hamiltonians describing one-dimensional chains tend to show a resemblance to Dicke states such as states resulting from the Bethe ansatz, making them an ideal starting state when investigating the ground state behavior of these Hamiltonians~\cite{TDL_BetheAnsatzDerivation:2010,B_ExcitedStateQuantumPhaseTransitions:2013,DickeTransitions:2021}. 
For instance, the algorithm by \citeauthor{van2021preparing}, who give an algorithm to prepare the Bethe ansatz eigenstates of the spin-1/2 XXZ spin chain, starts by first preparing a Dicke state~\cite{van2021preparing}. 
A Dicke-state preparation protocol based on the compress-uncompress methodology used in the $W$-state furthermore finds applications in entanglement distillation, where the entanglement of a large state is concentrated on only a few qubits. 
Efficient deterministic circuits for preparing Dicke states have been proposed by \citeauthor{bartschi2019deterministic}~\cite{bartschi2019deterministic, bartschi2022deterministic_short_depth}. 
They provide a quantum circuit of depth $\mathO(k \log(\frac{n}{k}))$, allowing arbitrary connectivity, to prepare a Dicke state, which they conjecture to be optimal when $k$ is constant. 
In this work, we provide a constant-depth $\LAQCC$ circuit below their conjectured bound already for constant $k$. 
However, this does not directly disprove their conjecture, as we allow for intermediate measurements and classical computations. 
More significantly, we even construct constant-depth $\LAQCC$ circuits for $k = \mathO(\sqrt{n})$ greatly improving their bound.
This construction extends the compress-uncompress method for the $W$-state combined with additional subroutines. 

We continue with a log-depth state preparation protocol for the Dicke-state for arbitrary $k$. 
This protocol implements an efficient transformation between the factoradic number representation and the combinatorial number representation of a positive integer. 
The combinatorial number representation relates directly to the Dicke state. 
The provided efficient transformation between number representation systems might be of independent interest. 

We conclude by modifying our protocol for preparing a Dicke-state to a protocol that prepares quantum many-body scar states in constant-depth. 
These states have low entanglement and longer coherence times than states with similar energy density.
These characteristics make many-body scar states interesting to analyze and relevant within physics.
Many-body scar states appear for instance in the AKLT model~\cite{AKLT:1987,MRBAR:2018,MRB:2018} and different spin models~\cite{SI:2019,MOBFR:2020}.
Known methods for preparing these states have polynomial-depth~\cite{Gustafson:2023}, whereas our circuit has constant depth. 

% We conclude by studying the power that intermediate classical calculations can add to quantum computations. 
% In this study, we define a new model that relaxes constant-depth quantum circuits to polynomial depth quantum circuits, log-depth classical calculations to unbounded classical computations and a constant number of alternations to a polynomial number of alternations. 
% We call this model $\LAQCC^*$. 
% We study this model by doing a complexity theoretical analysis, where we draw inspiration from the notions of complexity given by \citeauthor{RosenthalYuen:2022}, \citeauthor{MetgerYuen:2023}, and \citeauthor{Aaronson:2004}.
% All three complexity notions are based on the notion of state preparation, instead of more traditional definition of complexity such as the decidability of a computational problem. 
% The first two consider classes based on sequences of quantum states preparable by a polynomial-sized quantum circuit, where the circuits are uniformly generated by a computational class, for instance, the class $\mathsf{PSPACE}$, which results in the complexity class $\mathsf{StatePSPACE}$~\cite{RosenthalYuen:2022,MetgerYuen:2023}.
% The third notion considers a relative complexity, where the complexity is measured between two given states, and is measured by the number of gates, from a given gate-set, required to transform one state in another state~\cite{Aaronson:2004}. 
% For our definition of state preparation complexity, we drop the uniformity constraint from~\cite{RosenthalYuen:2022,MetgerYuen:2023} and define a class as $\mathsf{StateX}$, which refers to states preparable by circuits of type $\mathsf{X}$. 
% As an example, if $\mathsf{X} = \QNC^0$, this results in the class $\mathsf{StateQNC^0}$, which is the set of states preparable from the $\ket{0}^n$ state by poly-size constant-depth circuits. 
% This notion is similar to the relative complexity from~\cite{Aaronson:2004}, where one state is the  $\ket{0}^n$ state and instead of counting the number of gates we consider the set of states preparable by a fixed number of gates. Using this notion of complexity we show that any state preparable by an $\LAQCC^*$ circuit is also preparable by a $\mathsf{PostQPoly}$ circuit, the class of circuits of polynomial depth with an additional post-selection gate. 

\paragraph{Summary of results}
\begin{itemize}
    \item We give a new definition of a computational model that captures the power of the four step process: applying a constant number of layers of one- and two-qubit gates; performing a syndrome measurement; perform a fast classical computation determining corrections; apply corrections. We call this model \emph{Local Alternating Quantum Classical Computations}, or $\LAQCC$ for short. In this model we bound the allowed quantum operations, intermediate classical calculations, and number of rounds separately. In Section~\ref{sec:LAQCC_model} we define this model and give a list of operations based on results from literature contained in this computational model. In some of these operations we explicitly use that we allow for multiple, but at most constant, rounds  of corrections.
    \item  We show show that there exist $\LAQCC$ circuits that can not be weakly simulated in Section~\ref{sec:IQP_in_LAQCC}. We further show that for every $\LAQCC$ circuit there exists a $\QNC^1$ circuit simulating it perfectly, in Section~\ref{sec:LAQCC_in_QNC1}.
    \item We introduce a new type computational complexity for preparing states and show that the extension of $\LAQCC$ where we allow a polynomial number of rounds and unbounded classical computation, is contained in $\mathsf{PostQPoly}$, the class of polynomial circuits with post-selection, in Section~\ref{sec:Complexity results}.
    \item We show a protocol to prepare the uniform superposition state of size $q$ in $\LAQCC$ using $\mathO(\ceil{\log_2(q)}^2)$ qubits in Section~\ref{sec:superposition_modulo_q}. 
    \item We show a protocol to prepare the $W_n$ state in $\LAQCC$ using $\mathO(n\log(n))$ qubits in Section~\ref{sec:W_state_in_LAQCC}.
    \item We show two ways of preparing the Dicke-$(n,k)$ state. The first method is in $\LAQCC$, works up to $k = \mathO(\sqrt{n})$, uses $\mathO(n^2\log(n))$ qubits, and is found in Section~\ref{sec:dicke:small_k}. The second method is in $\LAQCC\text{-}\mathsf{LOG}$ (an extension of $\LAQCC$ allowing for logarithmic number of alterations instead of constant), works for any $k$, uses $\mathO(\text{poly}(n))$ qubits, and is found in Section~\ref{sec:Dicke_in_LAQCC_LOG}. 
    \item We extend on our $\LAQCC$ method of generating Dicke-$(n,k)$ states for $k = \mathO(\sqrt{n})$ and show a protocol to generate many-body scar states for a particular Hamiltonian in $\LAQCC$ (Section~\ref{sec:many_body_scar}). 
\end{itemize}
Summarized in a table, we provide the following state generation protocols:
\begin{table}[htb]
\centering
\begin{tabular}{l|l|l|l}
\textbf{State description} & \textbf{Width} & \textbf{Depth} & \textbf{Implementation}\\
\hline 
Uniform superposition mod $q$: $\frac{1}{\sqrt{q}} \sum_{i = 0}^{q-1}\ket{i}$ & $\mathO(\ceil{\log^2 q})$ & $\mathO(1)$ & Section~\ref{sec:superposition_modulo_q}\\

$W$-state: $\frac{1}{\sqrt{n}}\sum_{i = 0}^{n-1}\ket{e_i}$ & $\mathO(n \log n)$ & $\mathO(1)$ & Section~\ref{sec:W_state_in_LAQCC}\\

Dicke-$(n,k)$, $k = \mathO(\sqrt{n})$: $\binom{n}{k}^{-1/2}\sum_{x \in \{0,1\}^n: |x| = k} \ket{x}$ &  $\mathO(n^2\log n)$ & $\mathO(1)$ 
&Section~\ref{sec:dicke:small_k}\\

Dicke-$(n,k)$: $\binom{n}{k}^{-1/2}\sum_{x \in \{0,1\}^n: |x| = k} \ket{x}$ & $\mathO(\text{poly}(n))$ & $\mathO(\log n)$ &Section~\ref{sec:Dicke_in_LAQCC_LOG}\\

QMBS: $\ket{S_k} = \frac{1}{k! \sqrt{\mathcal N(n,k)}}(Q^\dagger)^k \ket{\Omega}$ &  $\mathO(n^2\log n)$ & $\mathO(1)$  &  Section~\ref{sec:many_body_scar}
\end{tabular}
\caption{Summary of state preparation protocols given in this paper.}
\label{tab:sate_prep}
\end{table}
In the entry for the quantum many-body scar state $Q$ denotes the raising operator and $\mathcal N(n,k)=\binom{n-k-1}{k}$. 
Section~\ref{sec:many_body_scar} will provide more details on the variables and the implementation. 

\paragraph{Organization of the paper}
\noindent We first introduce relevant preliminaries in Section~\ref{sec:preliminaries}. 
In Section~\ref{sec:LAQCC_model} we formally define the class of Local Alternating Quantum-Classical Computations ($\LAQCC$). We also show that any Clifford circuit can be implemented in constant depth $\LAQCC$ (a result based on a result from measurement-based quantum computing~\cite{jozsa2006introduction}). 
This result allows us to give many useful multi-qubit gates and routines in Section~\ref{sec:gates_created_in_LAQCC}. 
Beyond that we show that constant depth $\LAQCC$ circuits are contained in $\QNC^1$ and that any $\mathsf{IQP}$ circuit has an $\LAQCC$ implementation.
We conclude this section with an analysis of a more powerful instantiation of $\LAQCC$ and show an inclusion with respect to the class $\mathsf{PostQPoly}$, which is the class of circuits of polynomial depth with one additional post-selection gate. 
In Section~\ref{sec:state_prep_in_LAQCC} we give $\LAQCC$ circuit implementations for preparing the uniform superposition over an arbitrary number of states, the $W$-state and the Dicke state up to $k = \mathO(\sqrt{n})$. We furthermore give a log-depth circuit implementation for preparing the Dicke state for any $k$. We conclude by showing a $\LAQCC$ circuit for generating many body scar states of a particular type of Hamiltonian.



\section{Tidal Evolution Model}
%
\label{sec:tidal_model}

We use an updated version of the publically available tidal evolution module
\textbf{P}lanerary \textbf{O}rbital \textbf{E}volution due to \textbf{T}ides
(POET hereafter; \citet{Penev_2014} ) to simulate the stellar and orbital
evolution under the effects of tides. POET follows stellar evolution by
interpolating among a grid of evolutionary tracks calculated using MESA
\citep{Paxton_et_al_11}. MESA splits the star into many concentric shells, but
for the purposes of POET, those are combined into two zones to track stellar
evolution: convective envelope and radiative core. The two zones are assumed to
have solid body rotation but are allowed to spin differently. They are also
assumed to converge toward synchronous rotation on what is referred to as the
core-envelope coupling timescale \citep{Irwin_et_al_07,Gallet_Bouvier_2013}, as
well as exchange mass (which carries with it its specific angular momentum).
The core-envelope coupling torques are calculated as:

\begin{equation}
%
    \mathbf{T}_\mathrm{conv,c-e}
%
    =
%
    -\mathbf{T}_\mathrm{rad,c-e}
%
    =
%
    \frac{
%
        I_\mathrm{conv}\mathbf{L}_\mathrm{rad} -
        I_\mathrm{rad}\mathbf{L}_\mathrm{conv}
%
    }{
%
        (I_\mathrm{conv}+I_\mathrm{rad})\tau_\mathrm{c-e}
%
    }
%
    -
%
    \frac{2L_\mathrm{conv}}{3I_\mathrm{conv}} R_\mathrm{rad}^2
    \dot{M}_\mathrm{rad}
%
\end{equation}

where $I_\mathrm{rad/conv}$ and $\mathbf{L}_\mathrm{rad/conv}$ are the moment of
inertia and the angular momentum vector of the core and envelope respectively,
$M_\mathrm{rad}$ and $R_\mathrm{rad}$ are the mass and radius of the radiative
core.

In addition to the change in internal structure, the angular velocity of the
convective surface of stars is also effected by the torques exerted by stellar
winds. The external torque leads to spin-down of the star at later stages of
evolution \citep{Schatzman,Irwin_Bouvier_2010,
Gallet_Bouvier_2013,Gallet_Bouvier_15}. To account for this spin-down POET
adopts a generalized formalism for the rate of angular momentum loss by assuming
winds apply torque to the convective zone in direction opposite to
$\mathbf{L}_{conv}$ with a magnitude:
%
\begin{equation}
%
    \label{eq:wind_torque}
%
    T_\mathrm{conv,wind} \equiv K\omega_\mathrm{conv}
%
    \mathrm{\min}(\omega_\mathrm{conv},\omega_\mathrm{sat})^2
%
    \left(\dfrac{R_{\star}}{R_\odot}\right)^{1/2}
%
    \left(\dfrac{M_{\star}}{M_\odot}\right)^{-1/2}
%
\end{equation}
%
where $K$ is the parameterized wind strength, $\omega_{conv}$ is the angular
frequency of the convective envelope, $\omega_{sat}$ is the frequency above
which the magnetic braking saturates, $M_\star$ and $R_\star$ are the mass and
radius of the star, and $M_\odot$ and $R_\odot$ are the present day mass and
radius of the Sun.

The orbital evolution model in POET is formulated in terms of the equilibrium
tide model, but as already pointed out in the introduction, dynamical tide
models can be incorporated by finding an effective phase lag.  The prescription
is based on the formalism of \citet{Lai_12}, but expanded to account for
eccentricity. In summary,

\begin{enumerate}
%
    \item The quadrupole moment of the tidal potential in a spherical
        $\left(\zeta, \theta, \phi\right)$-coordinate system is:
%
        \begin{equation}\label{eq:tidal_potential}
%
            U(\textbf{r},t)= - \sum_{m,m^{\prime}} U_{m,m^{\prime}} \zeta^2
            Y_{2,m}
            (\theta,\phi)\dfrac{a^3}{r^3(t)}\exp[-im^{\prime}\Delta\phi(t)]
%
        \end{equation}
%
        where a is the semi-major axis and r$\left(t\right)$ is the distance
        between the centers of the two objects at an arbitrary point on the
        orbit. $U_{mm^{\prime}}$ are the coefficients of the spherical
        harmonics.
%
%
%
    \item A Fourier series expansion is used to include effects of eccentric
        orbits:
%
        \begin{equation}
%
            \label{eq:fourier_expansion}
%
            \dfrac{a^3}{r^3(t)}\exp[-im^{\prime}\Delta\phi(t)] = \sum_s
            p_{m^{\prime},s}\exp(-is\Omega t)
%
        \end{equation}
%
        with the $p_{m^{\prime},s}$ coefficients pre-calculated as a function of
        eccentricity up to $s=400$ to allow for even extreme orbital
        eccentricities.  At e=0.9, truncating the tidal potential expansion to
        $s=200$ is accurate to better than 1 part in $10^5$. We doubled the
        maximum $s$ value for the sake of caution.

    \item Following the equilibrium theory of tides, the effects of tidal
        dissipation are introduced as a lag in the response of the fluid
        perturbation inside the star. Each $\left(ms\right)$-component of the
        tidal potential has a distinct tidal frequency ($\tilde{\omega}_{m,s} =
        s\Omega-m\Omega_s$) and thus a distinct tidal lag.  The Eulerian density
        perturbation and Lagrangian displacement is given as:
%
        \begin{align}
%
            \boldsymbol{\xi}_{m,s}(\mathbf{r},t)
%
            & =
%
            \dfrac{U_{m,s}}{\omega_0^2}
%
            \bar{\boldsymbol{\xi}}_{m,s}(\mathbf{r})
%
            \exp(-is\Omega t + i\Delta_{m,s})
%
            \label{eq:fluid}
%
            \\
%
            \delta\rho_{m,s}(\mathbf{r},t) & =
%
            \dfrac{U_{m,s}}{\omega_0^2}
%
            \delta\bar{\rho}_{m,s}(\mathbf{r})
%
            \exp(-is\Omega t + i\Delta_{m,s})
%
            \label{eq:density}
%
        \end{align}

        where $\delta\bar{\rho}_{m,s} =
        -\nabla(\rho\bar{\boldsymbol{\xi}}_{m,s})$ and
        $\omega_0\equiv\sqrt{GM/R^3}$ is the dynamical frequency of the primary
        star in the binary system.

    \item The individual $\left(ms\right)$-component of the tidal torque and
        rate of dissipation of energy are calculated as:
%
        \begin{align}
%
            \mathbf{T}_{m,s}
%
            &=
%
            \int d^3x
%
            \delta\rho_{m,s}(\mathbf{r},t)\mathbf{r}
%
            \times
%
            \left[-\nabla U^{\ast}(\mathbf{r},t)\right]
%
            \label{eq:torque}
%
            \\
%
            \dot{E}_{m,s}
%
            &=
%
            \int d^3x
%
            \rho_{m,s}(\mathbf{r})
%
            \dfrac{
%
                \partial\boldsymbol{\xi}_{m,s}(\mathbf{r},t)
%
            }{ \partial t
%
            }
%
            \left[-\nabla U^{\ast}(\mathbf{r},t)\right]
%
            \label{eq:E}
%
        \end{align}
%
    \item The evolution of the orbital parameters (semi-major axis $a$,
        eccentricity $e$, angle between spin angular momentum and orbital
    angular momentum vectors $\theta$, and angular frequency of primary star
$\omega$) in terms of torque, energy, and angular momentum is calculated using
the following equations:

        \begin{align}
%
            \dot{a} &= a\dfrac{-\dot{E}}{E} \\
%
            \dot{e} &=
            \dfrac{2(\dot{E}L+2E\dot{L})L(M+M^{\prime})}{G(MM^{\prime})^3} \\
%
            \dot{\theta} &= \dfrac{(T_z - \tilde{T}_z)\sin\theta}{L} -
            \dfrac{(T_x - \tilde{T}_x)\cos\theta}{L} -  \dfrac{T_x +
            \mathcal{T}_x}{S} \\
%
            \dot{\omega} &=  \dfrac{(T_y +
            \mathcal{T}_y)\cos\theta}{L\sin\theta} + \dfrac{T_y +
            \mathcal{T}_y}{S\sin\theta}
%
        \end{align}
%
        where  $M$ is the mass of tidally distorted star, $M^{\prime}$ is the
        mass of the companion raising the tides, $T_x, T_y, T_z$ are the
        components of tidal torque along $x, y, z$-directions, $\mathcal{T}_x,
        \mathcal{T}_y, \mathcal{T}_z$ are the non-tidal torques (i.e. stellar
        wind and core-envelope coupling), and $\tilde{T}_x, \tilde{T}_y,
        \tilde{T}_z$ are the tidal torques on the orbit due to other zones.  The
        contribution of both stars to the orbital evolution is included.
%
\end{enumerate}

The relation between the tidal quality factor and the phase lag is:

\begin{equation}
%
    \label{eq:tidal_lag_relation}
%
    \Q = \dfrac{15}{16\pi\Delta_{\star}^{\prime}}
%
\end{equation}

The tidal lag introduced in the above method allows for an arbitrary
parameterization of tidal dissipation depending on the model being studied.  For
equilibrium tide models with constant phase lag, $\Q$ has a fixed value, while
for for dynamical tide models or variable phase lag equilibrium models, $\Q$
will in general depend on the orbital and stellar properties, as well as the
particular tidal potential term being dissipated. In this work, we restrict our
investigation to stars with similar internal structure, and only allow for a
dependence on tidal frequency.




\lstMakeShortInline[columns=fixed]@
% Figure environment removed
\lstDeleteShortInline@

In this section, we describe how we collect examples for learning repair strategies without any version-controlled data. Specifically, we first detect \safeprogs and corresponding witnesses using \sawitnessfull (witnesses are sanitizers and guards that protect from vulnerabilities)  in Section~\ref{subsec:sa-witness}. Using these witness annotations, we generate unsafe programs and \textit{edits} from the \safeprog using a \textbf{witness-removal} step (Section ~\ref{subsec:witness-removal}). In the following, we define terminology for the \astree  data-structure we operate on. 


\astree refers to the abstract syntax tree representation of programs, augmented with data flow edges and annotations for sources, sinks, sanitizers, guards, witnesses etc. 
An \astree is a five-tuple 
$\langle \mathcal{N},\mathcal{V},\mathcal{T},\mathcal{E}, \mathcal{A} \rangle$, where:
\begin{enumerate}
\item
$\mathcal{N}=\{\mathit{id}_0,\ldots\mathit{id}_n\}$  is a set of nodes, where  $\mathit{id_i}\in\mathbb{N}$ for 
$ 0 \leq i \leq n$.
\item
$\mathcal{V}$ is a map from nodes to program snippets
represented as strings. For a node $n$, we have that $\mathcal{V}(n)$ is a string representing the code snippet associated with $n$
\item
$\mathcal{T}$ is a map from nodes to their types defined by 
 \sa~\cite{codeqlast}. For example, \callexpr is the type of a node representing a function call, \indexexpr is the type of a node representing an array index, and \blockstmt is the type of a node representing a basic block of statements.
\item
$\mathcal{E}$ is a set of directed edges.
Each edge is of the form $(n_1,n_2,\edgetype,z)$, where
$n_1$ is a source node, $n_2$ is a target node, 
$\edgetype \in \{\T{SynParent}, \T{SynChild}, \T{SemParent},
\T{SemChild} \}$ denotes the relationship from 
$n_1$ to $n_2$, as one of syntactic parent, syntactic child, semantic parent or semantic child,
and $z\in\mathbb{Z}$ is the index of $n_2$ among $n_1's$ children if this edge is a child edge, and $-1$ if the edge is a parent edge. 
\item
$\mathcal{A}$ is a set of annotations associated with each node. The annotations are from the set $\{\T{source},
\T{sink},\T{sanitizer},\T{guard}$,\T{witness}\}. We also refer to annotations using predicates or relations. For instance, for a node $n$, if an annotation  $\T{source}$ is present, we say that
the predicate $\T{source}(n)$ is true.
\end{enumerate}

%\setlength{\grammarindent}{5em} % increase separation between LHS/RHS

% Figure environment removed



A {\em traversal} or a {\em path} in an \astree is a sequence of edges $e_0,\ldots,e_{i-1},e_i,\ldots ,e_k$ such that the target node of $e_{i-1}$ is also the source node of $e_i$, for all $i\in\{1,\ldots,k\}$. That is, $e_{i-1}$ is of the form $(\_,n,\_,\_)$ and $e_i$ is of the form $(n,\_,\_,\_,\_)$. The source node of $e_0$ is the source of this path and the target node of $e_k$ is the target of the path.


\lstMakeShortInline[columns=fixed]@
%Note that these additional edges can capture long-range dependencies in programs. E.g. edge 4 in Figure ~\ref{fig:unsafememberex} links two nodes across the function boundaries. 
Figure~\ref{fig:example1-pdg} depicts a partial \pdg corresponding to the unsafe program in Figure~\ref{fig:unsafememberex}. Each oval corresponds to an \astree-node containing a type $\tau$ and an associated value. The dark edges denote the syntactic child edges. For example, the oval with value @foo(data)@ is an \astree-node with type \callexpr and has two children -- @foo@ and @data@, both with the type \varexpr. 
%Similarly, the \blockstmt node on the top refers to the function body between Line~\ref{lst:line:handlers-run} and Line~\ref{lst:line:handlers-run-end} in Figure ~\ref{fig:unsafememberex}. As the body of a function block can contain a variable number of children, we link to @handlers[callerId](data);@ as the k-th child of the \blockstmt. 
The semantic child edges are at the bottom in cyan. These edges correspond to the ones depicted in cyan in Figure ~\ref{fig:unsafememberex}. 
\lstDeleteShortInline@

%TODO:FIX THIS

%With this simplification, 
If $\prog$ is an \pdg then
we use  $\prog.\mathtt{source}$ to denote the source node, $\prog.\mathtt{sink}$ to denote the sink node, and $\prog.\mathtt{witness}$ to denote the witness node.
If the program has several sources, sinks and sanitizers then we generate a separate \pdg for each $(\mathtt{source},\mathtt{witness},\mathtt{sink})$ triple.
For a node $n$, its syntactic parent is $n.\mathtt{parent}$, syntactic children are $n.\mathtt{children}$, semantic parent is $n.\mathtt{semparent}$, and semantic children are $n.\mathtt{semchildren}$.

%\input{ql.tex}

\subsection{Static Analysis Witnessing}
\label{subsec:sa-witness}

\newcommand{\DMethodjudge}[1]{\texttt{#1(}\checknextarga}

% Figure environment removed

%\naman{TODO - sell this more as technique to work with any \sa tool ; our master query is a general framework implemented in \codeql that can work for any vulnerability -- easily extendable to other languages }
In this section, we show how to repurpose \sa tools to generate witnesses.
\sa tools perform dataflow analysis to check for rule-violations in programs. They use pattern matching to identify known sources, sinks, sanitizers, and guards. For commercial tools, these patterns are implemented (and continuously updated) manually by developers and encode this domain knowledge. Next, 
%these patterns are used to detect sources, sinks, sanitizers, and guards in programs and
\sa checks if there exists a flow between a source and a sink that does not cross a sanitizer or guard. We capture this formally in Figure~\ref{fig:judgements} (top two rules), and explain the notation used in it below.

\sa tools encode domain knowledge about the vulnerability by annotating nodes as \T{Source}, \T{Sink}, \T{Sanitizer}, and \T{Guard}. %These relations operate on the set of dataflow nodes in the programs.
So \DMethod{Source}{\I{n}}\ is true iff the node \I{n} is a \textit{source} node for a vulnerability. Next, \sa tools perform dataflow analysis by defining the relation \DMethod{SemChild}{$n_1$}{$n_2$}\ which is true iff there is a \taintpropedge between $n_1$ and $n_2$. Then the \DMethod{Vulnerability}{$n_1$}{$n_2$}\ relation can be defined as:
\begin{enumerate}
    \item $n_1$ and $n_2$ are source and sink nodes (\DMethod{Source}{$n_1$}\ and \DMethod{Sink}{$n_2$}\ are true)
    \item There exists a \textit{path} between $n_1$ and $n_2$ which is free of sanitizers or guards (\DMethod{SanGuardFree*}{$n_1$}{$n_2$}\ is true). A path is free of sanitizers and guards iff every \textit{edge} in the \textit{path} is free of sanitizers and guards. An edge between $n_1$ and $n_2$ is considered free of sanitizers and guards (\DMethod{SanGuardFree}{$n_1$}{$n_2$}\ is true) iff $(n_1, n_2, \_, \T{SemChild}) \in \mathcal{E}$ and neither of $n_1$ or $n_2$ is a sanitizer or a guard
\end{enumerate}

Here, we make the following observation - \emph{this domain knowledge present in these annotations and relations is helpful beyond just detecting vulnerabilities}. For instance, simply using the sanitizer relation allows us to query the different kinds of sanitizers domain experts have specified. We use this observation to discover \emph{\safeprogs} i.e., programs having a source, sink, and a sanitizer or guard that \textit{blocks} the \taintprop or, in simpler terms, make the program safe. In addition, we also detect the corresponding sanitizers or guards in the programs and refer to them as \textit{witnesses} because they serve as the evidence of making the program safe. We call this procedure \sawitnessfull (abbreviated as \sawitness). 
We define this as the \T{Witness} relation in Figure~\ref{fig:judgements} (bottom two rules). Specifically, \DMethod{Witness}{$n_1$}{$n_3$}{$n_2$}\ is defined as:
\begin{enumerate}
    \item $n_1$ and $n_2$ are source and sink nodes (\DMethod{Source}{$n_1$}\ and \DMethod{Sink}{$n_2$}\ are true)
    \item There exists a node $n_3$ such that it satisfies \DMethod{SanGuardInMid}{$n_1$}{$n_3$}{$n_2$}. \DMethod{SanGuardInMid}{$n_1$}{$n_3$}{$n_2$}\ is true iff there exists a \T{SemChild}
    %\naga{notation for flow inconsistent with (2) above} 
    path between $n_1$, $n_3$, between $n_3$ and $n_2$, with the additional constraint of $n_3$ being a sanitizer or guard. 
\end{enumerate}

The difference between the \T{Vulnerability} relation (which \sa populates) and \T{Witness} relations (which we want to find) is highlighted in {\color{red} red} and {\color{ForestGreen} green}. Notice that while defining the \T{Witness} relation, we simply use the existing relations that define the \T{Vulnerability} relation. Thus, we argue that \sawitness can be implemented on top of \sa by using the intermediate relations that \sa is computing.
%for every pair of source and sink, they track taint through a taint-flow analysis. If there is a flow from a source to a sink that does not go through a sanitizer or guard, then the source-sink pair is reported as vulnerable.

%We make the following observation - \emph{the patterns defined by experts encodes domain knowledge which can be used for use cases beyond just detecting vulnerabilities}. For instance, we can use the sanitizer patterns to search for all sanitizers in source-code. In this work, we use this idea to detect \safeprogs, which we define as programs having a source, sink, and a sanitizer or guard that blocks the \unsure{flow} or in other words, makes the program safe.  \naman{highlighted part of Figure somethings shows the difference between semantics of witnessing vs traditional semantics}

%We realize the following -- the set of patterns of sources, sinks, and sanitizers are useful beyond detecting vulnerabilities. We override the existing static analysis query that detects unsafe programs and use these encoded sanitizers for detecting sanitizers and guards in programs. Specifically, in the existing query that detects unsafe programs, we modify the taint-propagation steps to propagate taints through sanitizers and guards and then use static analysis to then find these dataflows containing sanitizers and guards. Thus, we can directly find the safe programs containing these \textit{witnesses} of safety. 
%Once such a dataset is collected, we use these witnesses to convert safe  to unsafe  and thus obtain paired examples for learning repair strategies (Section~\ref{subsec:witness-removal}). 

\lstMakeShortInline[columns=fixed]@
%We instantiate our \sawitness technique using \codeql~\cite{a}. It is an open-source \sa tool that allows implementing custom static analysis as queries in a high-level object-oriented extension of datalog. These queries usually contain a \Verb|select from where| statement that allows querying the program database. \codeql maintains these patterns of sources, sinks, sanitizers, and guards using \Verb"Configuration" classes. Consider an example of a simplified \Verb"Configuration" for \xss vulnerability in Figure~\ref{fig:configuration}. It defines a set of predicates @isSource@, @isSink@, @isSanitizer@, and @isGuard@. These predicates are written manually by \codeql authors and improved through rich community support\footnote{\url{https://github.com/github/codeql}}. With this configuration, vulnerabilities are reported by selecting source-sink pairs such that the @cfg.hasFlow@ predicate is true for the source, and the sink. This predicate is internally defined by \codeql and uses the patterns defined in the configuration to check for the presence of vulnerability-causing dataflows. %\spsays{Showing corresponding programs will be useful}

%Now, we demonstrate the static-analysis-witnessing approach for collecting examples of \safeprog and witnesses in Figure~\ref{fig:safe-configuration}. Specifically, we inherit from the existing configuration, using the same @isSource@ and @isSink@ predicates while overriding the @isSanitizer@ and @isGuard@ predicates to @none()@. This ensures that all the source and sink pairs are detected independent of the presence of sanitizers/guards between them. Finally, to detect our witnesses, we define the @isWitness@ predicate which uses the @isSanitizer@ and @isGuard@ predicates from the original configuration. Specifically, witnesses are defined as sanitizers/guards that lie between a source-sink pair. Finally, to report \safeprog and witnesses, the @cfg.hasFlow@ predicate is used to select all valid source-sink pairs and the corresponding witnesses are detected via the @isWitness@ predicate. Note that Figure~\ref{fig:configuration-vs-safe-configuration} depicts the key idea behind our approach in a simplified view. In practice, additional measures need to block the taint propagation internally and we share the actual \codeql queries used as part of the Appendix~\ref{app:codeql-queries}.


\subsection{Witness Removal}
\label{subsec:witness-removal}

We obtain \safeprogs and witnesses by applying \sawitness to a snapshot of a codebase. Recall that the witnesses block the flow between a source and a sink and thus help make programs  \textit{safe}. Hence, removing these witnesses will make the programs unsafe. Recall also that the witnesses are either sanitizing functions of the form @sanitize(taintedVar)@ or guards of the form @if checkSafe(taintedVar) {executeSink(taintedVar)}@. %Usually, they are used only for ensuring the safety of programs and are not critical to the functionality of programs. Therefore, 
We implement witness-removal perturbations  that precisely remove the guard-checks and sanitizer-functions. Note that our goal here is to generate unsafe programs and corresponding edits that enable learning repair strategies that insert such witnesses. So, while we generate the unsafe programs by perturbation, they should look structurally similar to natural unsafe programs written by the developers, otherwise the repair strategies learned on this artificially generated data through perturbations would not generalize to code in the wild. 
%At the same time, minor syntactic-semantic issues in parts of unsafe programs not directly relevant to the vulnerability or repair do not impact learning.
\lstDeleteShortInline@

% Figure environment removed

\lstMakeShortInline[columns=fixed]@

\input{witnessremoval.tex}

We use \rmSan and \rmGuard functions to programmatically remove the witnesses. A high-level sketch of these functions is illustrated in Figure~\ref{fig:remove-functions}. The functions use the structure of the corresponding \astree (node types $\tau$) to decide how to remove witnesses. Consider the \rmGuard function. It first computes the parent (\witnesspar) and grand-parent (\witnessparpar) of the witness guard condition. Then if the type of \witnesspar is \ifstmt (i.e., program is of the form @if (witness) body@ then we modify the \astree edge from \witnessparpar and \witnesspar to instead point to the body of the \ifstmt (index 1 child is body of \ifstmt). Similarly, if the type of \witnesspar is \binaryexpr with operator @&&@ (i.e. of the form @if (otherCond && guard)@ or @if (guard && otherCond)@) then we again modify the edge from \witnessparpar and \witnesspar to instead point to the non-guard child of \binaryexpr (@otherCond@ in the example). Note that since \binaryexpr has 3 children, the index of non-guard child is index of guard-child subtracted from 2. 
Figure~\ref{fig:witness-removal} depicts this removal on the \astree level, where the syntactic edges in red are removed and the syntactic edges in green are inserted.
In the end, the functions returns a tuple of the \pdg of the unsafe program ($\prog_{unsafe}$), \pdg of the safe program ($\prog_{safe}$)
and an edit object (\edit) which stores


\begin{enumerate}
    \item \astree for the removed witness (referred to as \editprog)
    \item location in the \pdg where the witness is removed (referred to as editloc
    %\naga{shouldn't it be editloc to be consistent with (1)?} 
    or \editloc)
    %\item an enum (\insertsc or \replace) depending on whether \concedit is inserted or replaced 
\end{enumerate}

Since $\prog_{unsafe}$ and edit-object can generate the safe program, we only propagate the unsafe programs and edits as the output of this step. Applying \rmGuard function to the safe program in Figure~\ref{fig:safememberex} removes the \ifstmt on Line~\ref{lst:line:fix-start} while preserving the @handlers[callerId](data);@ statement and in fact produces the unsafe program in Figure~\ref{fig:unsafememberex}. Additionally, it  returns the removed witness guard  @if handlers.hasOwnProperty(data.id){ ... }@ as the \editprog and \blockstmt (blue oval in Figure~\ref{fig:example1-pdg}) as the edit location \edit.editloc. Figure~\ref{fig:example1-editprog} shows the \astree for the \editprog containing the \ifstmt. 
The dashed line and dark circle correspond to the \textit{removed} \astree edge between the \blockstmt and the \expr @handlers[callerId](data)@. 

Note that Figure~\ref{fig:remove-functions} provides a high-level sketch of witness-removal and elides over implementation details that are required to make it work for real \js programs. We discuss these issues in the implementation section (Section~\ref{subsec:impl:witness-removal}).% and include the full implementation as part of supplementing source code\naga{we should make sure we are doing these, else remove this sentence}. 
%. In practice, we need implement such decisions more carefully to cover other traditional cases in which guards occur and we document them in the supplementing source code.
\lstDeleteShortInline@

%\naman{add examples $\dots$ } \spsays{do we re-run codeql on this generated bad program? -- NO (naman)}



\section{$\Q$ Formalism}
%
\label{sec:q_formalism}

In this work, we assume that the tidal dissipation occurs only in the convective
zones of the stars, meaning that we assume the tidal torque is applied only to
the convective zone, with the core spin affected only indirectly through the
core-envelope coupling.  As mentioned in Section~\ref{sec:introduction}, the
effects of the binary system parameters on tidal dissipation can be studied by
using a suitable parameterization for the tidal lag, $\Delta_{m,m^{\prime}}$ or
the corresponding tidal quality factor, $Q_{m,m^{\prime}}^{\prime}$, such that
it can capture the orbital dynamics of the system under the effects of tides.

As discussed in Sec. \ref{sec:tidal_model}, we allow each Fourier component of
the tidal potential to experience its own effective lag. In this work, we
restrict the possibilities to some extent, by assuming that
$\Delta_{m,m^{\prime}}^{\prime}$ depends solely on the tidal frequency (Eq.
\ref{eq:tidal_frequency}). Each equilibrium or dynamical tidal model predicts a
different frequency dependence. In this work, we further simplify the
prescription to assume that $\Delta_{m,m^{\prime}}$ has a power-law dependence,
on $\Omega_{\text{tide}} = m\Omega_{\text{orbit}} - m'\Omega_{\star}$.
Furthermore, in order to avoid clearly non-physical and numerically unstable
behavior occurring if $\Delta_{m,m'} \rightarrow \infty$, we assume the
dissipation saturates to some maximum value:
%
\begin{equation}
%
    \label{eq:lag_formalism}
%
    \Delta_{m,m^{\prime}}^{\prime}
%
    =
%
    \mathrm{sign}(\Omega_{m,m^{\prime}}) \Delta_0
%
    \min\left[
%
        1,
%
        \left| \dfrac{\Omega_{m,m^{\prime}}}{\Omega_{\text{break}}}
        \right|^{\alpha}
%
    \right]
%
\end{equation}
%
or in terms of $\Q$:
%
\begin{equation}
%
    \label{eq:q_formalism}
%
    Q_{m,m^{\prime}}^{\prime}
%
    =
%
    \mathrm{sign}(P_{m,m^{\prime}}) Q_0
%
    \max\left[
%
        1,
%
        \left| \dfrac{P_{m,m^{\prime}}}{P_{\text{break}}} \right|^{\alpha}
%
    \right]
%
\end{equation}
%
where $\alpha$ is the powerlaw index parametrizing the frequency dependence,
$\Delta_0=\frac{15}{16\pi Q_0}$ parametrizes the maximum allowed dissipation,
$\Omega_{m,m'}=\frac{2\pi}{P_{m,m'}}$ is the tidal frequency of a particular
tidal wave, and $\Omega_{\text{break}}=\frac{2\pi}{P_\text{break}}$ is the
frequency at which the dissipation saturates.  See Appendix \ref{sec:q_tweak}
for a small modification we introduce, which improves numerical performance
without significantly changing the evolution.

We now wish to find constraints on $Q_0$, $\alpha$, and $P_{\text{break}}$ that
can simultaneously reproduce the observed spins of the primary stars in all the
binaries in our input sample. In practice, the tidal effects on the spin of each
binary will be dominated by a narrow range of frequencies, so for a given
binary, we will be able to constrain $Q_{m,m^{\prime}}^{\prime}$ for a narrow
range of periods. Furthermore, because tides get weaker as the separation
between the two stars increases, we expect that for the lowest periods of our
sample of binaries all systems will be synchronized, thus providing only a lower
limit on the dissipation (upper limit on $\Q$), while at the longest periods
stellar spins will not be significantly affected by tides, providing only an
upper limit to the dissipation (lower limit on $\Q$). Because the effects of
tides decrease very precipitously as the orbit gets wider, there should be only
a small number of binaries in between these two regimes providing two-sided
constraints. It should be noted that the range of orbital periods with
asynchronous binaries whose spin is nonetheless significantly affected by tides
is much wider than one may naively expect. This is caused by the fact that the rate
at which stars lose angular momentum due to magnetized winds scales as the cube
of the angular velocity. Hence, maintaining synchronous rotation requires ever
stronger tidal coupling as the orbital period decreases. Even with that, however,
combining the constraints from multiple binary systems is required to get a
measurement of the tidal dissipation and detect any potential
frequency-dependence.


\section{Bayesian Analysis}
%
\label{sec:bayesian_analysis}

We use the Markov Chain Monte Carlo algorithm as implemented in the Python
package \texttt{emcee} \citep{Foreman_Mackey_2013}. \texttt{emcee} is based on a
class of ensemble sampler algorithms that uses affine invariant transformation
as a proposal function to simultaneously advance multiple not-independent chains,
referred to as walkers, in the parameter space. We refer the reader to
\citep{Foreman_Mackey_2013} and references therein for details of the algorithm.

The choice of the number of walkers is a compromise between parallelization
efficiency and the need to accumulate a sufficient number of steps to ensure the
distribution of samples has converged to the target posterior. For our analysis,
we use 64 walkers for each binary to sample the 9-dimensional space of
parameters required to calculate the orbital evolution.

The parameters sampled by MCMC and their respective priors are listed in Table.
\ref{tab:sampling_parameters}. For the physical parameters of the binaries we
construct a joint prior using Kernel density estimation from the MCMC samples
published by W19, and for the dissipation parameters we use very broad uniform
priors.

% Please add the following required packages to your document preamble:
% \usepackage{graphicx}
\begin{table}
%
    \centering
%
    \caption{The parameters sampled during MCMC and their priors. The W19 joint
    prior is constructed using KDE from the W19 MCMC samples.}
%
    \label{tab:sampling_parameters}
%
    \resizebox{0.49\textwidth}{!}{%
%
        \begin{tabular}{cccc}
%
            \hline
%
            \textbf{Parameter} & \textbf{Units} & \textbf{Description} &
            \textbf{Prior} \\
%
            \hline
%
            $M_1$ & $M_{\odot}$ & Mass of the primary star & W19 \\
%
            $M_2$ & $M_{\odot}$ & Mass of the secondary star & W19 \\
%
            $\left[\frac{\mathrm{Fe}}{\mathrm{H}}\right]$ & - & Metallicity of
            both stars & W19 \\
%
            $\tau$ & Gyr & Age of the system & W19 \\
%
            $e$ & - & Present day eccentricity of the Orbit & W19 \\
%
            $P_{\star,\text{init}}$ & days & Initial spin period of the stars &
            $U()$ \\
%
            $\log_{10}Q_0$ & - & Dissipation parameter. See Eq.
            \ref{eq:q_formalism} & $U(5, 10)$ \\
%
            $\alpha$ & - & Dissipation parameter. See Eq.  \ref{eq:q_formalism}
            & $U(-5,5)$ \\
%
            $\log P_\text{break}$ & rad/day & Dissipation parameter. See Eq.
            \ref{eq:q_formalism} & $U\left(\log 0.5\,d, \log 50\,d \right)$ \\
%
            \hline
%
        \end{tabular}%
%
    }
%
\end{table}

\subsection{Prior Transformation}
%
\label{sec:prior_transform}

Because the prior distributions built from W19 samples are quite complex
structures in a multi-dimensional space, the MCMC sampling becomes very
inefficient. In order to remedy this situation, instead of directly sampling
from the prior distribution of the parameters from Table
\ref{tab:sampling_parameters}, we use a transformed set of parameters ($u_i$
with $i=1\ldots 9$) such that each of those has an independent prior
distribution uniformly distributed between 0 and 1. Given a sample of values for
$u_i$, the physical and dissipation parameters required to calculate the
evolution are then found by applying a prior transformation. This procedure
dramatically simplifies the posterior likelihood the MCMC process must sample
from, making it directly proportional to the likelihood that the actual spin of
the primary star as observed today is equal to the spin period predicted by the
evolution assuming the given parameters.

The joint prior distribution we wish to impose on the physical parameters of
each binary (the first 5 parameters listed in Table \ref{tab:sampling_parameters})
is the posterior distribution W19 samples are drawn from. We start by isolating
only the parameters we need. For each sample, we combine the components of
eccentricities ($e\cos{\omega}$ and $e\sin{\omega}$) to just eccentricity ($e =
\sqrt{(e\cos\omega)^2 + (e\sin\omega)^2}$). We then construct a marginalized
probability distribution function $\pi \left(M_{sum}, q, \tau,
[\mathrm{Fe}/\mathrm{H}], e\right)$, where $M_{sum} \equiv M_1 + M_2$ and $q
\equiv M_1/M_2$, using a Kernel density estimator (KDE) with a Gaussian kernel
with a bandwidth determined using the improved Sheather-Jones algorithm
\citep{sheather2010}.

Finally, W19 samples were generated assuming independent $U(0,1)$ priors on
$e\cos\omega$ and $e\sin\omega$. This has the undesirable effect that when
converted to prior on eccentricity, the probability of $e=0$ is zero, even if
the cloud of points in $e\cos{\omega}$, $e\sin{\omega}$ space clearly includes
the origin. This will propagate to the prior distribution we impose resulting in
erroneously imposing an upper limit on the dissipation (lower limit on $\Q$) for
such systems. To see this, consider a system for which the primary star is
consistent with spinning synchronously with the orbit, and the $e\cos{\omega}$,
$e\sin{\omega}$ cloud of points is centered on the origin (i.e.  observations
are consistent with circular orbit). We expect that for such a system
arbitrarily small $\Q$ should be acceptable since large amounts of tidal
dissipation will predict a completely circularized and synchronized system.
Instead, the prior on eccentricity imposed by W19 will exclude that
configuration, erroneously requiring a small eccentricity comparable to the
uncertainty in $e\sin{\omega}$ (the much less well constrained of the two
components) to survive to the present day. Instead of the W19 prior, we wish to
impose independent priors $e \in U(0,1)$ and $\omega \in U(0,2\pi)$. The desired prior
is obtained from the W19 prior by simply dividing by $e$ and re-normalizing.

To be precise, if the $s^{\text{th}}$ W19 sample has values for the parameters given by
$M_{sum}=M_s$, $q=q_s$, $\tau=\tau_s$, $\feh = \feh_s$, and
$e=e_s$, the prior probability density of the systems parameters is given by:
\begin{align}
    \pi & \left(M_{sum}, q, \tau, \feh, e\right) \propto \nonumber \\
    &~~~~~~~~\sum_s k_M(M_{sum}-M_s) k_q(q - q_s)~~~\times \nonumber \\
    &~~~~~~~~~~~~~~k_\tau(\tau-\tau_s)  k_\feh\left(\feh - \feh_s\right) \frac{k_e(e-e_s)}{e}
\end{align}
%
%\begin{equation}
%
%    \begin{array}{c}
%
%        \pi\left(M_{sum}, q, \tau, \feh, e\right) \\
%
%        \propto\\
%
%        \sum_s k_M(M_{sum}-M_s) k_q(q - q_s) k_\tau(\tau-\tau_s)
%        k_\feh\left(\feh - \feh_s\right) \frac{k_e(e-e_s)}{e}
%
%    \end{array}
%
%\end{equation}
%
where $k_M\ldots$ are the kernels used for each of the quantities, and the
factor of $1/e$ in the last term is responsible for changing the priors as
explained above.

The resulting prior transformation converting $u_1, \ldots, u_5$ for a given
MCMC sample to the corresponding physical system parameters is then:

\begin{equation}
%
    \begin{array}{rcl}
%
        M_{sum}  & = & F_M^{-1} (u_1)\\
%
        q_i  & = & F_q^{-1} (u_2|M_{sum})\\
%
        \tau_i  & = & F_\tau^{-1} (u_3|M_{sum},q)\\
%
        Z_i  & = & F_\feh^{-1} (u_4|M_{sum},q,\tau)\\
%
        e_i  & = & F_e^{-1} (u_5|M_{sum},q,\tau,\feh)\\
%
    \end{array}
%
\end{equation}
%
with
%
\begin{equation}
%
    \begin{array}{r@{\ }c@{\ }l}
%
        F_M(M) & \equiv & \int_{-\infty}^{M} \text{d}M_{sum}
        \int_{-\infty}^{\infty} \text{d}q \int_{-\infty}^{\infty} \text{d}\tau
        \int_{-\infty}^{\infty} \text{d}\feh \int_{-\infty}^{\infty} \text{d}e\\
%
        && \quad \quad \pi\left(M_{sum}, q, \tau, \feh, e\right)\\
%
%
%
        F_q(q|M_{sum}) & \equiv & \int_{-\infty}^{q} \text{d}q'
        \int_{-\infty}^{\infty} \text{d}\tau \int_{-\infty}^{\infty}
        \text{d}\feh \int_{-\infty}^{\infty} \text{d}e\\
%
        &&\quad \quad \pi\left(M_{sum}, q', \tau, [\mathrm{Fe}/\mathrm{H}],
        e\right)\\
%
        && \ldots
%%
%%
%%
%        F_\tau(\tau|q,M_{sum}) & \equiv & \int_{-\infty}^{\tau} \text{d}\tau'
%        \int_{-\infty}^{\infty} \text{d}[\mathrm{Fe}/\mathrm{H}]
%        \int_{-\infty}^{\infty} \text{d}e \\
%%
%        && \quad \quad \pi\left(M_{sum}, q, \tau', [\mathrm{Fe}/\mathrm{H}],
%        e\right)\\
%%
%%
%%
%        F_\feh(\feh|\tau,q,M_{sum}) & \equiv & \int_{-\infty}^{\feh}
%        \text{d}\feh' \int_{-\infty}^{\infty} \text{d}e \pi\left(M_{sum}, q,
%        \tau, \feh', e\right)\\
%
%
    \end{array}
%
\end{equation}
%
The above calculations are feasible because calculating multi-dimensional
integrals is not necessary. Kernel functions are normalized, so integrals over
$\pm\infty$ are all unity.

The priors on the dissipation parameters and the initial spin frequency of the
primary star are assumed uniform. The corresponding prior transform for those is
just a simple shift and scaling to match the limits.

The parameters not listed in Table \ref{tab:sampling_parameters} and required by the
evolution (Sec. \ref{sec:tidal_model}) are assumed constant with the following
values:
%
\begin{equation}
%
    \begin{array}{lcl}
%
        K & =& 0.17\,
        M_\odot\,R_\odot^2\,\mathrm{day}^2\,\mathrm{rad}^{-2}\,\mathrm{Gyr}^{-1}\\
%
        \omega_{sat} & = & 2.45\,\mathrm{rad}\,\mathrm{day}^{-1}\\
%
        \tau_{c-e} & = & 5\,\mathrm{Myr}.
%
    \end{array}
%
\end{equation}

\subsection{MCMC Convergence Diagnostic}
%
\label{sec:convergence}

The samples generated by an MCMC algorithm can be shown to follow the specified
posterior distribution in the limit of infinitely long chains. Real-world
applications then need to demonstrate that the generated chains are long enough
to get a good approximation to the distribution. There are two concerns that
must be addressed. First, in infinite chains the starting positions are
irrelevant. In finite chains however, an MCMC process requires some number of
steps before subsequent samples can be shown to come from the desired
distribution to a good approximation. This is usually referred to as the burn-in
period, and a typical practice is to discard the early samples. The second
concern is that one needs a sufficient number of post-burn-in samples in order to
estimate the targeted quantities with a desired precision.

In this work, we are interested in estimating quantiles of the posterior
distribution. \citet{raftery1991many} derived diagnostics for single chain MCMC
of a single quantity that answer the questions:
%
\begin{itemize}
%
    \item What is the smallest $N$ such that the probability that the $N+1$ step
        in the chain is below some threshold value is within a specified
        precision of the limit of that probability for $N\to\infty$.
%
    \item What is the variance of the estimated value of the cumulative
        distribution at the threshold value from the MCMC chain after the $N^{\text{th}}$
        sample.
%
\end{itemize}

The \citet{raftery1991many} procedure cannot be directly applied to
\texttt{emcee} chains because that involves multiple non-independent chains. In
\citet{Patel2022} and \citet{Penev2022} we adapted the \citet{raftery1991many} procedure
for \texttt{emcee} chains, allowing us to estimate when enough samples have been
generated to reliably and precisely estimate a specified quantile of some target
quantity.

In this work, we select a grid of tidal periods $P_{tide,i}$ and use the
\texttt{emcee} samples of $Q_0$, $\alpha$, and $P_{\text{break}}$ to evaluate
the tidal model (Eq. \ref{eq:q_formalism}) at each of these periods to obtain
samples of $Q'_i = Q_{m,m^{\prime}}^{\prime}(P_{tide,i})$. For each of these
quantities, we wish to find the 2.3\%, 15.9\%, 84.1\%, and 97.7\% quantiles.
Using the adapted \citet{raftery1991many} formalism, we find a burn-in period
for each $Q'_i$  for each quantile, requiring that the fraction of the first
post-burn-in samples of $Q'_i$ (one for each walker) below the quantile is
within $10^{-3}$ of the equilibrium probability, and estimate the uncertainty in
the CDF for each $Q'_i$ and quantile combination from the remaining samples
after discarding the burn-in.

Because finding a given quantile requires knowing the burn-in period, and
finding the burn-in period requires knowing the quantile, we iterate between the
two to find a mutually consistent combination.


\section{$Q_{\star}^{\prime}$ Constraints}
%
\label{sec:constraints}

\subsection{Individual Constraints}

From the MCMC we obtain posterior samples for $Q_0$, $P_{\text{break}}$ and
$\alpha$ for each system. These samples are converted to $\Q(P_\text{tide})$
using Equation~\ref{eq:q_formalism} evaluated at 30 separate tidal periods
$\log-$uniformly spaced between 1 and 50 days. For each tidal period we apply
the convergence diagnostics described in Section~\ref{sec:convergence} to find
the 2.3, 15.9, 84.1, and 97.7 percentiles of $\Q$ at that tidal period
(corresponding to $\pm1\sigma$ and $\pm2\sigma$ uncertainties) together with the
burn-in steps and cumulative distribution uncertainty.

Figures \ref{fig:ind_const_1} and \ref{fig:ind_const_2} show the KDE estimate
of the posterior distribution of $\Q$ at each tidal period (color-coded heat
map), the 2.3, 15.9, 84.1, and 97.7 percentiles (black dotted or solid curves),
the $50^{\text{th}}$ percentile as the thin horizontal red curve, and the orbital period
(thick vertical black line). The solid parts of the quantile curves, delineated
with vertical red lines, mark the range of periods for which the $\Q$
distribution and quantiles are reliable, i.e. dominated by the data rather than
the priors and not affected by potentially under-estimated uncertainty of the
spin (see Sec. \ref{sec:reliable_constraints} for details).

% Figure environment removed

% Figure environment removed

Figures \ref{fig:burnin_1} and \ref{fig:burnin_2} show the estimated
burn-in periods (see Sec. \ref{sec:convergence}) for the four quantiles of $\Q$
at each tidal period as the curves (solid corresponding to the reliable range
of periods, dotted otherwise), as well as the total number of steps we
accumulated for each system (black area). As that figure demonstrates, for the
tidal periods where a given quantile is deemed reliable (see Sec.
\ref{sec:reliable_constraints}), for all our systems we have accumulated enough
samples to move past the burn-in period.

% Figure environment removed

% Figure environment removed

Finally, Figures \ref{fig:cdfstd_1} and \ref{fig:cdfstd_2} show the estimated
standard deviation of the cumulative distribution function (CDF) at each
quantile estimated as described in Sec. \ref{sec:convergence} from the samples
after burn-in is discarded.

% Figure environment removed

% Figure environment removed

\subsection*{Two-sided limits on $\Q$}

The systems for which we obtain two-sided limits significantly differ between
the orbital period and the spin period of the primary star and significantly
deviate from the spin period of an isolated star given the same mass and age.
For example, in the case of \texttt{KIC11232745}, the orbital period of the
system is 9.6 days, while the spin period is 12.7 days. The lower bound on
$Q_{\star}^{\prime}$ comes from not allowing high dissipation, which would cause
the primary star to synchronize with the orbit. The upper bound on
$Q_{\star}^{\prime}$ is due to the requirement of minimal dissipation such that
the tidal influence on the spin of the star is not negligible compared to the
effect of stellar winds. Given the very steep drop off of the effects of tides
with orbital separation (tidal torque decreases as the fourth power of the
orbital period), at first glance it may appear that such systems should be so
rare that finding several in a sample of 70 binaries would be surprising.
However, two-sided limits on $\Q$ correspond to stars with spin periods
comparable (though not quite equal) to the orbital period. As a result, since
stellar wind torques scale strongly with the spin period (third power of the
spin period), which is itself related to the orbital period, the transition from
fully synchronized systems to systems unaffected by tides is much more gradual
than thinking about tides alone would predict.

\subsection*{Upper Limit on $\Q$}

There are systems where we only obtain an upper bound on $\log_{10}\Q$ (lower
bound on the dissipation). In these cases, the spin period of the primary star
is synchronized with the orbit. Above this limit, the dissipation is not high
enough to maintain a spin-orbit lock and keep the star synchronized with the
orbit. However, reproducing the observed state of the system can tolerate
arbitrarily large dissipation (arbitrarily small $\Q$) since that will just
result in the spin-orbit lock being achieved earlier.

\subsection*{Lower Limit on $\Q$}

Similarly, there are systems with only a lower bound on $\log_{10}\Q$ (an upper
bound on the dissipation). This limit is obtained for systems whose spins are
consistent with that of isolated stars of similar mass and age. The inability to
distinguish the spin from that of an isolated star is generally driven by the
fact that the age is usually the least well constrained parameter.  We stress
that the non-tidal parameters we assume (core-envelope coupling timescale, wind
strength, wind saturation frequency etc.) were tuned to reproduce the observed
spins of isolated stars with well known ages (ones residing in open clusters).
Hence, if we calculate the evolution of a binary under the assumption of
$Q_\star' \rightarrow \infty$, the predicted spin of each star in the binary is
the same as that of an isolated star with the same mass and age. The lower limit
to $Q_\star'$ produced by these systems is thus a statement that if the
dissipation were any larger than this, it would have had a detectable effect on
the spin.

For some systems, the lower bound on $\log_{10}{Q_{\star}^{\prime}}$ can be due
to eccentricity rather than the spin period of the primary star. It may be that
the uncertainty in the age does not allow us to distinguish the spin from that
of an isolated star, but if non-zero orbital eccentricity is detected with high
significance, dissipation must be weak enough to avoid circularizing the system.

\subsection{Reliable Portion of Individual $Q_{\star}^{\prime}$ Constraints}
%
\label{sec:reliable_constraints}

Examining the posterior distributions of $\log_{10}\Q$ in Fig.
\ref{fig:ind_const_1} and \ref{fig:ind_const_2}, it is clear that each
individual binary is sensitive to the dissipation only in a relatively narrow
range of periods. For example, the difference between the median and the 97.7$^{\text{th}}$
percentile of the distribution has a sharp minimum. Sufficiently far away from
that period, the increase of the 97.7$^{\text{th}}$ percentile is limited to large
extent by the fact that we restrict the powerlaw index $\alpha$ in Eq.
\ref{eq:q_formalism} to a maximum value of 5. Similarly, for some systems, the
small quantiles peak at some period, and, away from that, the slope is limited by
the assumption that $\alpha>5$.

In order to select the region where constraints are dominated by the observed
state of the binary rather than the priors, we evaluate the absolute difference
between the highest/lowest quantiles and the median at each tidal period. We
then select the period range where this difference remains within 0.5 dex of its
minimum value. The upper/lower half of the distribution is then only deemed
reliable within that period range.

For the systems for which we obtain a lower bound or two-sided bound on
$\log_{10}\Q$, we must further examine the difference in the measured spin
period and the orbital period. Generally, the statistical uncertainty in
measuring the spin period is small compared to stellar differential rotation. As
described in Sec.  \ref{sec:rotation_periods}, we estimate that difference using
the detection of multiple peaks in the Lomb-Scargle periodogram. However, if the
star spots modulating the lightcurve of a particular star happen to all come
from a narrow rang of latitudes at the time they were observed, our approach
will under-estimate the amount of differential rotation. The result could be a
star which synchronized the spin to the orbit, but for which, due to
differential rotation, the LC modulation happens to have a slightly different
period and the difference appears significant because we under-estimate the
amount of differential rotation. In turn, this will produce an upper limit to
the dissipation (lower limit on $\Q$) that is not justified by the observations.
To compensate for this possibility, we trust only the upper limit to
$\log_{10}\Q$ our analysis gives for systems with measured spin periods within
17\% of the measured orbital period (approximately the amount of differential
rotation between the poles and equator of the Sun) and ignore the lower limit.
Suggestively, the distribution of fractional difference between spin and orbital
period for our systems has a gap between 17\% and 21\%, with most systems below
17\%, perhaps hinting at a switch from a population of synchronized to
non-synchronized systems, though the sample is not large enough to claim this is
statistically significant or to explore possible explanations.

%We selected posterior samples for the orbital and stellar parameters provided
%by \citet{Windemuth_2019} as distribution functions for bayesian analysis.
%\citet{Windemuth_2019} used uniform priors, $U(0,1)$ on the two eccentricity
%components: $e\cos{\omega}, e\sin{\omega}$ in their analysis. We transformed
%these samples to eccentricity using the relation: $e = \sqrt{{e\cos{\omega}}^2
%+ {e\sin{\omega}}}^2$ and sampled from the probability distribution as shown in
%Equations. This methodology requires a correction in the probability
%distribution of eccentricity by multiplying it by a factor $\dfrac{1}{e}$.
%Otherwise, there is a higher preference for non-zero eccentricities in the
%probability distribution. Avoiding this correction can result in a lower limit
%on $Q_{\star}^{\prime}$ for circular systems, which comes from non-zero
%eccentricity rather than synchronization. To correct for this, we visually
%inspect the posterior samples of $e\cos{\omega} \text{ vs } e\sin{\omega}$ for
%each system and identify the system where zero eccentricity is within the
%spread of the two distributions (Figure). Next, from this subset, we identify
%the systems that show high likelihood at low $Q_{\star}^{\prime}$ values. We
%discard the lower limit for these systems and renormalize the probability
%distribution for $Q_{\star}^{\prime}$ at each tidal period.

\subsection{Combined Constraints}
%
As expected, individual constraints are sensitive to a limited range of
frequencies, and most binaries only produce upper or lower limits, but only a
few produce both. As a result, meaningful measurement of $\Q$ can only be
obtained by combining the individual constraints. Furthermore, since we selected
all our binaries to have similar internal structure, it is not unreasonable to
assume they should have similar dissipation.

A common constraint, which agrees with all the constraints obtained by the
individual binary systems, can be found by multiplying the posterior probability
density functions of $\Q(P_\text{tide})$ for the binary systems for each
$P_\text{tide}$, obtained using KDE. This of course is complicated by the
considerations described in Sec. \ref{sec:reliable_constraints}. For a given
system, at a given tidal period, we have already identified whether we trust
both halves of the distribution, just the upper half, just the lower half, or
neither. If neither side of the distribution is reliable, we simply do not
include that system in the product for the given tidal period. If only one side
of the distribution is reliable, we replace the unreliable part of the
probability density with the probability density at the median and re-normalize
before including the distribution in the product for the combined distribution.

One of our binaries -- KIC 7816680 -- produces constraints that are in conflict
with the combined constraint from the other binaries for all tidal periods above
7 days. An outlier binary can have many explanations:
%
\begin{enumerate}
%
    \item There may be additional objects in the system resulting in a very
        different orbital evolution than we calculate assuming just two objects
%
    \item A third object in the system or another star along the same line of
        sight could be contributing extra light to the system. Since W19 models
        the spectral energy distribution assuming just the two binary components
        that could shift the inferred properties of the binary.
%
    \item Spin period measurements sometimes detect an alias of the spin period,
        or perhaps the spot modulations come from the secondary star instead of
        the primary, invalidating our assumption that measured spin is that of
        the brighter star.
%
    \item The stellar evolution models used in the W19 analysis to find binary
        parameter distributions and in our orbital evolution calculations may
        not be applicable. In fact, W19 already report discrepancies for young
        stars and caution discrepancies are also expected for post-main-sequence
        stars, and POET stellar evolution interpolation is unreliable past the
        end of the main sequence as it was never designed to handle such stars.
%
\end{enumerate}

We suspect KIC 7816680 falls in the final category. The primary star has a
maximum likelihood mass of $1.2M_\odot$ and maximum likelihood age of 9.4\,Gyr,
hence the W19 model of this binary involves a post-MS primary, so some amount of
difficulty with stellar evolution models is expected.

Fig. \ref{fig:combined} shows the combined constraint on $\log_{10}\Q$ obtained
through the above procedure, excluding KIC 7816680.

An important question that must be considered is whether the combined
constraints are indeed consistent with all individual constraints. After all, it
is possible to multiply two mutually highly inconsistent distributions and
obtain a combined constraint that is somewhere in the middle that is highly
inconsistent with either distribution. Fig.~\ref{fig:combined_vs_individual}
shows the 2.3, 15.9, 50.0, 84.1, and 97.7 percentiles of the combined
constraints (same as the lines in Fig. \ref{fig:combined}) along with the 2.3
and 97.7 percentiles for each individual system at the tidal period where the
distance between the 2.3 and 97.7 percentiles is smallest. As can be seen in
Fig.~\ref{fig:combined_vs_individual}, there is at least some overlap between
the allowed $\log_{10}\Q$ values for each system and the combined constraint. In
other words, the combined constraint is capable of reproducing the observed
spins of all the binaries in our sample except KIC 7816680.

% Figure environment removed

% Figure environment removed


\section{Discussion}
\label{sec: discussion}
\kmsdelete{In this work} We study \kmsreplace{Fairness-Aware PAC learning}{Fair-ERM} in the malicious noise model, and  in some cases allow 
the learner to maintain optimal overall accuracy despite the signal in Group $B$ being almost entirely washed out.
%when we allow learners to use the
%$\PQ$ randomized expansion of the hypothesis class $\mathcal{H}$
In particular we show that different fairness constraints have fundamentally different behavior in the presence of Malicious Noise, in terms of the amount of accuracy loss that a given level of Malicious Noise could cause a fairness-constrained learner to incur. 
The key to achieving our results, which are more optimistic than those in \cite{lampert}, is allowing for improper learners using the (P,Q)-randomized expansions of the given class $\mathcal{H}$.
%We \kmsreplace{present a picture of the}{prove upper and lower bounds on}
%accuracy loss for a range of fairness notions, given \kmsreplace{this simple randomization step.}{learning over $\PQ$.
%In general our results indicate Fair-ERM (given learning over $\PQ$) is more robust than claimed in \cite{lampert}.
The type of smoothness we create by using $\PQ$ seems to be a natural property that is likely shared by many natural hypothesis classes.

Fairness notions are motivated as a response to learned disparities when there is \kmsdelete{data corruption or} systemic error affecting \kmsdelete{the data for}
one group. 
Fairness notions are supposed to mitigate this by ruling out classifiers that have worse performance on a sub-group. 
This can peg both classifiers at a lower level of performance \kmsdelete{(e.g that the lower subgroup)} in order to \emph{motivate} \cite{hardt16} improving the data collection or labelling process to obtain more reliable performance. 
%So in \kmsreplace{some}{a} sense, sensitivity of the fairness notion to poor sub-group performance caused by malicious noise is the \textit{point} of fairness constraints! 
However, it also desirable that fairness constraints perform gracefully when subject to Malicious Noise because fairness constraints will be used in contexts where the data is unreliable and noisy and this might not be known to the learner.
This tension, exposed by our work, motivates 
%a revisiting of fairness notions from first principles approach and trying to axiomatize the 
%desired properties of a fairness intervention a la cryptography and privacy. \footnote{Work in multi-calibration \cite{multicalib} is a viable direction for this problem but it is unclear how 
%that and related notions behave with unreliable data. }
on going work studying the sensitivity level of fairness constraints. 
%If we we are to take a view, if a classifier is deployed 


\section{Conclusion and Future Work}
In this work, I design corruption-robust algorithms for the Lipschitz contextual search problem. I present the \emph{agnostic checking} technique and demonstrate its effectiveness in designing corruption-robust algorithms. There are several open problems for future research. First, in the algorithm I propose for pricing loss, the schedule for agnostic checks is fixed upfront. Can the learner design an adaptive checking schedule for the pricing loss? Second, this work assumes the learner has knowledge of the Lipschitz constant $L$. Can the learner design efficient no-regret algorithms without knowledge of $L$? 

\section*{Acknowledgements}

This research was supported by NASA grant 80NSSC18K1009.

The authors acknowledge the Texas Advanced Computing Center (TACC) at The
University of Texas at Austin for providing HPC resources that have contributed
to the research results reported within this paper. URL:
\url{http://www.tacc.utexas.edu}


%%%%%%%%%%%%%%%%%%%%%%%%%%%%%%%%%%%%%%%%%%%%%%%%%%


\begin{dataavailability}

The data sets supporting the findings of this study are available on Zenodo\footnote{\url{https://zenodo.org/} [last accessed \lastdate]} with the identifiers \url{https://doi.org/10.5281/zenodo.6103720} and \url{https://doi.org/10.5281/zenodo.5940101}.

\end{dataavailability}



%%%%%%%%%%%%%%%%%%%% REFERENCES %%%%%%%%%%%%%%%%%%


\bibliographystyle{mnras}
\bibliography{bibliography}


%%%%%%%%%%%%%%%%%%%%%%%%%%%%%%%%%%%%%%%%%%%%%%%%%%

%%%%%%%%%%%%%%%%% APPENDICES %%%%%%%%%%%%%%%%%%%%%

\appendix

\section{Modified Tidal Dissipation Prescription}
%
\label{sec:q_tweak}

If $\alpha<0$, Eq. \ref{eq:q_formalism} has a discontinuity at
$\Omega_{m,m^{\prime}} = 0$. This makes it possible for tides to hold the system
in a spin orbit lock, keeping $\Omega_{m,m^{\prime}} = 0$ for an extended period
of time. \texttt{POET} handles this discontinuity by eliminating the stellar
spin as a variable and checking at each time step that the lock is maintained.
For $\alpha>0$, there is no discontinuity. However, even in this case it often
still happens that the system is held close to a lock for a long time, which
forces the differential equation solver to use tiny time steps in order to
maintain numerical accuracy. This in turn makes such cases very computationally
expensive to simulate. In order to avoid this situation, we slightly modify Eq.
\ref{eq:q_formalism} if $\alpha>0$ to force a small, but finite dissipation as
$\Omega_{m,m^{\prime}} \to 0$:

\begin{equation}
%
    \label{eq:q_tweak}
%
    \Delta_{m,m^{\prime}}
%
    =
%
    \Delta_0
%
    \begin{cases}
%
        {\left(\dfrac{\Omega_{\text{min}}}{\Omega_{\text{break}}}\right)}^{\alpha}
        & \text{if } \Omega_{m,m^{\prime}} < \Omega_{\text{min}} \\
%
        {\left(\dfrac{\Omega_{m,m^{\prime}}}{\Omega_{\text{break}}}\right)}^{\alpha}
        & \text{if }  \Omega_{\text{min}} < \Omega_{m,m^{\prime}} <
        \Omega_{\text{break}} \\
%
        1 & \text{if } \Omega_{m,m^{\prime}} > \Omega_{\text{break}}
%
    \end{cases}
%
\end{equation}

We pick $\Omega_{min}=\frac{2\pi}{50\,d}$ in Eq. \ref{eq:q_tweak}. This is small
enough to not appreciably change the evolution, while at the same time
eliminating most of the computational challenges.

%The value of $\Omega_{\text{min}}$ is selected as $\Omega_{\text{min}} = 2\pi/\left(P_{\text{tide} = 50 \text{ days}}\right)$. Equations~\ref{eq:formalism1} and~\ref{eq:formalism2} completely describes the parameterization of tidal dissipation. Figure~\ref{fig:w_dependent_model} shows the model evaluated at $\alpha = \pm2$, $\Omega_{\text{break}} = 2\pi/\left(\text{10 days}\right)$ and $\Omega_{\text{min}} = 2\pi/\left(\text{50 days}\right)$.
%
%    % Figure environment removed




% Don't change these lines
\bsp% typesetting comment
\label{lstarpage}
\end{document}

% End of mnras_template.tex
