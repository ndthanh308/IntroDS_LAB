\section*{Data Availability}
%
\label{sec:data_availability}

All tables of input parameters presented in this article are derivative of
published tables already available in machine readable form through the VizieR
online service. They are reproducible by applying simple filtering to those
tables, and thus do not warrant creating and publishing independent copies.

We have created a zenodo archive \citep{zenodo_results} to accompany this
article, which provides machine readable tables of the following:

\begin{itemize}
%
    \item The generated MCMC samples for each binary in the original HDF5 format
        produced by the emcee package \citep{Foreman_Mackey_2013} (see
        \url{https://emcee.readthedocs.io/en/stable/user/backends/}).
%
    \item The 2.3\%, 15.9\%, 84.1\%, and 97.7\% quantiles of $\log_{10}\Q$ for
        the given binary as a function of tidal period for each binary  (i.e.
            the coordinates defining the quantile curves in
            Fig.~\ref{fig:ind_const_1} and \ref{fig:ind_const_2}).
%
    \item The burn-in period for each quantile vs. tidal period (i.e. the
            coordinates defining the burn-in curves in Fig.~\ref{fig:burnin_1}
            and \ref{fig:burnin_2}).
%
    \item The estimated standard deviation of the fraction of samples below each
        of the 4 target quantiles for each binary (i.e. the coordinates defining
            the curves in Fig.~\ref{fig:cdfstd_1} and \ref{fig:cdfstd_2}).
%
    \item The 2.3\%, 15.9\%, 84.1\%, and 97.7\% percentiles of the combined
        $\log_{10}\Q$ constraint from all binaries vs tidal period (i.e. the
        coordinates of the quantile curves in Fig.~\ref{fig:combined}).
%
    \item The combined constraint 2.3\%, 15.9\%, 84.1\%, and 97.7\% percentiles
        and the 2.3\% and 97.7\% percentiles of individual constraints at the
        tidal periods where the difference between the latter is smallest (i.e.
            the coordinates defining the curves and points in
            Fig.~\ref{fig:combined_vs_individual}).
%
\end{itemize}

The version of POET used for calculating the orbital evolution is available
through zenodo archive \citet{zenodo_poet}.
