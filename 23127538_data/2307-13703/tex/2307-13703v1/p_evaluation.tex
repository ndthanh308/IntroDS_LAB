\section{Evaluation}
\label{sec:eval}
% Figure environment removed
% Figure environment removed

The proposed analysis was implemented and integrated in a toolchain developed by the authors. Part of the toolchain is a graphical editor for GRAFCET based on a GRAFCET meta-model proposed by Julius et al. \cite{Julius.19}. The meta-model was implemented using the Eclipse Modeling Framework (EMF)\footnote{\url{https://www.eclipse.org/modeling/emf/}}. 


%Beispiel
The presented analysis was evaluated using the GRAFCET specification of an industrial plant first shown in \cite{Schumacher.14}. 
The application example is an automatic testing machine for quality control of components and consists of a conveyor belt, a rotary indexing table and six stations. Coordinated by the rotary indexing table, the parts pass through these stations, where separation and quality control take place. The components are marked as regular or damaged parts, and damaged parts are subsequently sorted out.
The complete specification consists of eight partial Grafcets shown in Fig. \ref{fig:hierarchiegraphschumacher} including their hierarchical dependencies. On the top level the partial Grafcet G\_OM determines the operation mode. 
Choosing the automatic operation mode, G\_RIT is activated and controls the rotary indexing table as well as the six stations represented by G10 to G70. 
All hierarchical dependencies are induced by enclosing steps which are used to implement an emergency stop. 
G\_OM is activated by an initial step. 
Altogether the specification consists of 60 steps, 62 transitions %, 46 stored actions, 15 continuous actions and 8 enclosing steps. In total 
and 80 Boolean and integer variables. % are used (45 input, 20 output and 15 internal variables).

Fig.~\ref{fig:g_rit} shows the partial Grafcet G\_RIT\footnote{The full specification formalized with GRAFCET can be viewed here: \url{https://github.com/Project-AGRAFE/GRAFCET-instances}}. The number 3 at the top refers to the enclosing step 3 in G\_OM controlling G\_RIT. The asterisk at step 10 marks the step activated by the enclosing step.
The station activates the conveyor belt by setting the output variable \textit{conveyorBelt} to true and rotates the rotary indexing table by one sector using the output variable \textit{rotateTable}. This transfers the parts from one station to another and can therefore only occur while the stations are in their starting position. The activation of the stations happens in parallel via enclosing steps 11 to 16. The internal Boolean variables \textit{station1}, \textit{station2}, etc. indicate if a station is finished. Since these variables are read in G\_RIT and written in G10 to G70 they are affected by concurrency. 

Applying the analysis regarding the structural reachability and concurrency from Sec.~\ref{ssec:stepApprox} to G\_RIT results in all steps being reachable, as well as the following concurrent steps: step~10 has no concurrent steps - step~11 and 17 are structurally concurrent to the steps~12, 13, 14, 15, 16, 18, 19, 20, 21, 22 - the concurrent steps for the remaining steps are analog to step~11 and 17. We applied the analysis to the entire specification, taking into account that the enclosing steps of G\_RIT are concurrent to each other, and therefore the stations run concurrently as well. The execution time was less than ten milliseconds. We used this information to confirm that no race conditions are present in the Grafcet, i.e., no stored actions writing the same variable are associated to concurrent steps. 

%Invarianten
By applying the analysis from Sec.~\ref{ssec:flowinsens} shown in Fig.~\ref{fig:structuralAnalysis} to G\_RIT, we start by calculating the S-invariants $\mathbf{y}_1 = (s10, s11, s17)$,  $\mathbf{y}_2 = (s10, s12, s18)$, $\mathbf{y}_3 = (s10, s13, s19)$, etc., which cover the whole Grafcet (note that we misuse the notation here by writing the step name when we mean that there is a corresponding 1 and omit it when there is a 0). 
Thus, the number of active steps for every S-invariant is 1-bounded and no structures like shown in Fig. \ref{fig:concurrentStructures}, G4 are present. In the next step the T-invariants are calculated resulting in one invariant $\mathbf{x}_1 = (t10, t11, t12, t13, t14, t15, t16, t17)$, which means that all covered transitions (i.e., all transitions that have a corresponding value of 1 in $\mathbf{x}_1$) have to fire for the initial situation to be reached again.
Therefore, the actions associated to step 10 can be executed infinitely often resulting in a possible value assignment \textit{conveyorBelt} $=$ \textit{rotateTable} $=$ \{\textit{false}, \textit{true}\}. Applying the analysis to all other partial Grafcets results in the same possible value set for \textit{station1} to \textit{station7} and therefore, \textit{t11} to \textit{t17} can fire according to the analysis, which is the expected behavior. 

This allows to identify safety critical situations. E.g., in the given example, the rotary indexing table must not rotate (i.e., \textit{rotateTable} $=$ \textit{false}) as long as one of the six stations is working, %(indicated by one of the respective variables in G10 to G70 being \textit{true}) 
or vice versa. Using the approximation of the variable values, the analysis returns a false alarm for this requirement being violated since it can only detect if the variable values will eventually be true or false, but not if this will be at the same time or sequentially. However, using the results from the analysis proposed in Sec.~\ref{ssec:stepApprox}, it can be shown that the stations are deactivated while the rotary indexing table is rotating: As shown in Fig.~\ref{fig:g_rit}, the variable \textit{rotateTable} is \textit{true} when step~10 is active and since step~10 has no concurrent steps the stations can not be active at the same time.
 