\section{Preliminaries}
\label{sec:prelim}
Preliminaries on GRAFCET syntax and analytical methods based on linear algebra are explained in this section.

\subsection{Syntax of IEC 60848 GRAFCET}
\label{ssec:syntax}
Since the GRAFCET standard does not define the syntax of GRAFCET sufficiently for formal verification, we use the formalization proposed by Mroß et al. \cite{Mross.22} in this work to explain the concepts of GRAFCET that are important for this contribution.

A Grafcet $G = (V_\mathit{in}, V_\mathit{int}, V_\mathit{out}, C)$ comprises a set of partial Grafcets $C \neq \emptyset$ with globally available sets of input variables $V_\mathit{in}$, internal variables $V_{int}$ and output variables $V_\mathit{out}$. Variables can either be Boolean or integral, i.e. $v$ is assigned a value of $\mathbb{Z}$ for all $v \in V_\mathit{in} \cup V_\mathit{int} \cup V_\mathit{out}$ with Boolean variables being limited to the set $\{0, 1\}$. 
%We denote by $V_{inB} \subseteq V_{in}$ ($V_{intB} \subseteq V_{int}$, $V_{outB} \subseteq V_{out}$) the set of Boolean input (internal, output) variables. 
Given these variables, we can construct Boolean expressions with usual relational symbols (such as $=$ and $\le$) and Boolean operators (such as disjunction $\lor$ and negation $\lnot$). A variable may change values caused by an event. 
By $\mathit{CND}$ we denote the set of all Boolean expressions over variables in $G$. 
Every partial Grafcet $c \in C$ is a 6-tuple $c = (S, I, E, M, T, A)$, where 
\begin{itemize}
	\item $S$ is a finite set of steps, each of which is either active or inactive,
	\item $I \subseteq S$ is the set of initial steps,
	\item $E \subseteq S \times C$ is the set of enclosing steps,
	\item $M \subseteq S$ is the set of marked steps,
	\item $T \subseteq \mathcal{P}(S) \times \mathcal{P}(S) \times \mathit{CND}$ is the set of transitions, where $\mathcal{P}$ denotes the power set, and
	\item $A$ is a set of actions.
\end{itemize}
% Figure environment removed
We use the notation $S_{c}$, $I_{c}$, $E_{c}$, $M_{c}$, $T_{c}$, $A_{c}$ to refer to the respective sets of a given partial Grafcet $c \in C$. The set $M_{c}$ describes the steps that are activated by the enclosing step. Every $e \in E_{c}$ describes an enclosing step, which translates formally to $e = (s, c_\mathit{enc})$ for a $s \in S_{c}$ and a partial Grafcet $c_\mathit{enc} \in C$. If an enclosing step becomes active, it activates all steps $m \in M_{c_\mathit{enc}}$. If an enclosing step becomes inactive, it deactivates all steps $s \in S_{c_\mathit{enc}}$. We say that $c$ is \textit{enclosed} iff $M_{c} \neq \emptyset$. 
%Further we have disjoint sets of steps, that is  $S_{c} \cap S_{c'} = \emptyset$ for every $c' \in C$ with $c \neq c'$. 
Every step $s \in S_{c}$ induces a new Boolean variable $x_{s}$ which indicates the activation status of $s$ and is true iff the step is active in the current situation. These variables can be used in Boolean expressions $\mathit{CND}$. 
Fig.~\ref{fig:exampleGrafcet} shows an illustrative example of a Grafcet consisting of two partial Grafcets $C = \{\mathrm{G0, G1}\}$. G1 has two steps $S_{\mathrm{G1}} = \{2, 3\}$ one of which is an enclosing step $ M_{\mathrm{G1}} = \{2\}$ and two transitions $T_{\mathrm{G1}} = \{\mathrm{t1}, \mathrm{t2}\}$ as well as two continuous actions associated to step~2 and step~3. G1 is enclosed by step~1, indicated by the 1 at the top of G1, and therefore $E_{\mathrm{G0}}=\{(1, \mathrm{G1})\}$. If step~1 is activated, step~2 is activated as well and G1 can evolve freely as long as step~1 stays active.

A transition $t \in T_{c}$ is a triple $t = (\bullet t, t \bullet, b)$, where 
$\bullet t \subseteq S_{c}$ is the set of immediately preceding steps, 
$t \bullet \subseteq S_{c}$ is the set of immediately succeeding steps, 
$\bullet t \neq \emptyset \lor t \bullet \neq \emptyset$
and $b \in \mathit{CND}$ is the transition condition.
We also call $\bullet t$ the \textit{upstream} and $t \bullet$ the \textit{downstream} of $t$.
We say that $t$ is \textit{enabled} if $x_{s}$ is true for every $s \in \bullet t$. We say that $t$ can \textit{fire} if it is enabled and $b$ is true. Similarly to the upstream and downstream of transitions we define $\bullet s \subseteq T_c$ as the set of immediately preceding transitions of $s$ and $s\bullet \subseteq T_c$ as the set of immediately succeeding transitions of $s$. 

% Figure environment removed

Finally, we formalize the set of actions $A_{c}$. The standard defines different types of actions: continuous actions ($A_{cont}$), stored actions ($A_\mathit{stor}$) and forcing orders ($A_\mathit{fo}$). 
These sets are assumed to be disjoint. Let $A_{c} = A_\mathit{cont} \cup A_\mathit{stor} \cup A_{\mathit{fo}}$. Every element of $A_\mathit{cont}$ is a triple $(s, v, b)$, where 
$s \in S_{c}$ is the associated step, 
$v \in V_\mathit{out}$ is an output variable which must be Boolean and
$b \in \mathit{CND}$ is the action condition.
We say that a continuous action is \textit{active} if $x_{s}$ and $b$ are true. Several partial Grafcets in $G$ may employ continuous actions on the same output variable $v$. In this case, $v$ is set to true if at least one of these continuous actions is active. Note that $v$ can not be used by any stored action.
Every element of $A_\mathit{stor}$ is a tuple $(s, v, val, b)$, where 
$s \in S_{c}$ is the associated step, 
$v \in V_\mathit{int} \cup V_\mathit{out}$ is an internal or output variable, 
$val$ is an expression yielding a value in the respective domain, e.g. $val \in \mathbb{Z}$ and 
$b \in \mathit{CND}$ is the action condition.
A stored action sets $v$ to $val$ if $x_{s}$ and $b$ are true. This also allows to model actions on activation and deactivation of a step, as introduced by the standard. 
Finally, every element of $A_\mathit{fo}$ is a tuple $(s, c_\mathit{forced}, S)$, where 
$s \in S_{c}$ is the associated step, 
$c_\mathit{forced} \in C$ is the partial Grafcet which is to be forced and 
$S \in (\mathcal{P}(S_{c_\mathit{forced}}) \cup \{*, \mathit{init}\})$.
A forcing order is regarded as a special kind of continuous action. It is active while $x_{s}$ is true and forces $c_\mathit{forced}$ into the situation specified by $S$. If $S = *$, then the current situation in $c_\mathit{forced}$ is retained for as long as $s$ is active. If $S = \mathit{init}$ then $c_\mathit{forced}$ is set to its initial situation. Otherwise, it is set to the specified situation (element of the power set $\mathcal{P}(S_{c_\mathit{forced}})$). 




\subsection{Petri net analysis means adapted for GRAFCET}
\label{ssec:pn}


Schumacher et al. \cite{Schumacher.14} have shown how a Grafcet can be interpreted as a CIPN. We use this interpretation to adapt analysis means from the field of Petri nets to obtain structural information about each partial Grafcet.
In particular we want to adapt analysis means based on linear algebra, so-called S- and T-invariants which are described in more detail, e.g., in \cite{Girault.03}.

%dann die interpertation von GRAFCET als PN
Analog to the formalization of a Petri net Schumacher et al. define an $|S|\times |T|$ incidence matrix $\mathbf{N}$ of a so-called basic Grafcet, where a basic Grafcet is a Grafcet without hierarchical elements. The elements $n_{ij}$ of $\mathbf{N}$ are defined as $n_{ij} = -1$ if $s_i \in \bullet t_j$, $n_{ij} = 1$ if $s_i \in t_j\bullet$ and $n_{ij} = 0$ otherwise. Neglecting the transition conditions, a linear relaxation of the dynamic behavior of the basic Grafcet can be described by the so-called state equation $\mathbf{sit} = \mathbf{sit}_0 + \mathbf{N} \mathbf{q}$, where $\mathbf{sit}$ is the situation of the Grafcet (i.e., a vector of the step variables), $\mathbf{sit}_0$ the initial situation and $\mathbf{q}$ is the firing count vector stating how often a transition fires until $\mathbf{sit}$ is reached from $\mathbf{sit}_0$.
%invarianten
The incidence matrix $\mathbf{N}$ can be analyzed using T- and S-invariants. A T-invariant is a vector $\mathbf{x}$ such that $\mathbf{N}\mathbf{x} = 0$. T-invariants can detect possible loops in the reachability graph of Grafcet %1990 book analy
since $\mathbf{N}\mathbf{x} = 0 = \mathbf{sit}' - \mathbf{sit}$, where $\mathbf{sit}'$ can be reached from $\mathbf{sit}$ when the transitions in $\mathbf{x}$ fire.
A S-invariant is a vector $\mathbf{y}$ such that $\mathbf{y}^T \mathbf{N} = 0$, where $T$ denotes to transposed. For Petri nets S-invariants indicate an upper bound for a possible number of tokens in a place, since $\mathbf{y}^T\, \mathbf{sit} = \mathbf{y}^T \,\mathbf{sit}_0 + \mathbf{y}^T\mathbf{N}\mathbf{q}$ $\Leftrightarrow$ $\mathbf{y}^T\, \mathbf{sit} = \mathbf{y}^T \,\mathbf{sit}_0$, where the firing vector $\mathbf{q}$ has no influence on the ratio of tokens. 
This is not directly applicable to GRAFCET since the steps induce a binary activity variable. However, S-invariants applied to GRAFCET indicate if the number of a step's activation in a Grafcet is bounded to a value $n \in \mathbb{N}$ and $n < |S|$, where $n$ is the maximum value in the S-invariants for the corresponding step. %ergänzt

Both types of invariants are used in Sec.~\ref{ssec:flowinsens} to estimate how often an action can be executed and therefore, approximate the internal and output variables. 