\section{Conclusion}
\label{sec:conclusion}
In Sec.~\ref{sec:contribution}, we demonstrated that, owing to GRAFCET's concurrent behavior, control flow based analysis approaches from the field of sequential programs are inapplicable to GRAFCET. Further, we determined which GRAFCET structures and elements cause concurrent behavior.
The resulting approach resolves this challenge by over-approximation and neglecting transition conditions. 
We presented a worklist algorithm to approximate the step variables as well as their concurrency. Further we presented how analysis means from the field of Petri nets can be used to approximate internal and output variable values. 
Therefore, we demonstrated an approach that uses a structural analysis to verify possible GRAFCET instances including elements proposed by the standard IEC 60848 that can result in concurrent behavior. Despite the resulting over-approximation (which can lead to a report of concurrent steps where no such behavior is present), we presented in the evaluation that the approach provides valuable information, such as verifying that no writing conflicts exist. 

For future work we want to reduce the degree of over-approximation. One approach is to analyze the GRAFCET instances depending on whether concurrency is present or not. It can be assumed that not all Grafcets show all different elements that result in concurrent behavior. Instead of an analysis that can handle all structures proposed by the standard it might be beneficial to provide specialized algorithms that can analyze only a subset of the possible GRAFCET instances. Depending on the Grafcet to be analyzed, a specialized analysis that might result in less over-approximation could be chosen. 
Another approach for future work would be to use the information obtained from the concurrency analysis to track the interference between concurrent steps induced by actions writing variable values (e.g., when a variable is incremented due to the activation of a concurrent step). This interference could be used to analyze the Grafcets by means of abstract interpretation, similarly to works, e.g., proposed by Min{\'e} \cite{Mine.14}.