\section{Related work}
\label{sec:relWork}


For verifying GRAFCET there are approaches suitable for model checking, such as translating hierarchical Grafcets into time Petri nets by Sogbohossou et al. \cite{Sogbohossou.20} and recently transforming Grafcets into Guarded Action Language by Mroß et al. \cite{Mross.22}. 
Utilizing a model checking approach allows for an exhaustive exploration of the model but has the disadvantage of being limited by state space explosion for practical application.

Very few approaches are presented for analyzing GRAFCET without applying model checking.
A structural analysis regarding the hierarchical dependencies between modules of the Grafcets (called \textit{partial Grafcets}) is presented by Lesage et al. \cite{Lesage.93}. The authors provide an analysis to ensure that the hierarchical dependencies form a partial order. Moreover, Lesage et al. \cite{Lesage.96} provide an analysis of the GRAFCET-specific expressions by extending the Boolean algebra by events represented by rising and falling edges of Boolean signals in GRAFCET. This allows the user to check syntactic properties of transition conditions.
Both presented approaches \cite{Lesage.93, Lesage.96} are only capable of detecting syntactical design flaws regarding the structure of the Grafcets and not design flaws regarding the behavior of the Grafcets like the reachability of safety critical situations.
Schumacher et al. \cite{Schumacher.13a} present an approach to transform the time constraints of GRAFCET into Control Interpreted Petri nets (CIPN), a specific kind of Petri net. They later extended the approach in \cite{Schumacher.14} to normalize hierarchical Grafcets and formalize them as CIPN. The verification of the formalized control specifications is not covered by Schumacher et al.
The approach, in fact, forms the groundwork for structural analyses based on methods known from the field of Petri nets. Well established analysis tools such as those described in \cite{Bonet.07} already exist for the structural analysis of Petri nets and we present how to apply them to partial Grafcets in Sec.~\ref{ssec:pn}, since the structural analysis known from the fields of Petri nets have to be adapted for the peculiarities of GRAFCET. 


The normalization technique proposed in \cite{Schumacher.14} is not adopted in this work. It replaces the implicit, hierarchical flow relations between partial Grafcets induced by enclosing steps and forcing orders (cf. Sec. \ref{ssec:syntax}) by transitions and steps. Therefore, the Grafcets' hierarchical information is lost during the process.
%The normalization technique proposed in \cite{Schumacher.14} is not adopted in this work, because it replaces the implicit, hierarchical flow relations between partial Grafcets by transitions and steps, and therefore the Grafcets' hierarchical information is lost during the process. The hierarchical flow relations are induced by enclosing steps and forcing orders and introduced in Sec. \ref{ssec:syntax}.
Hierarchical information, however, is valuable because it can be exploited during the verification process in terms of a modular analysis of the partial Grafcets.



%None of the provided approaches allow for a verification of behavioral design flaws which would occur during run-time and at the same time the approach is not limited by state space explosion. 
%None of the provided approaches is not limited by state space explosion and at the same time allows for a verification of behavioral design flaws which would occur during run-time.
All cited approaches are limited by state space explosion and none of them allows for a verification of behavioral design flaws which would occur during run-time.
Therefore, this work aims to present a structural analysis that does not suffer from a state space explosion and is able to detect behavioral errors. 





