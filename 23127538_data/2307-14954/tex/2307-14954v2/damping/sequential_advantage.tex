
\documentclass[%
reprint,
%superscriptaddress,
%groupedaddress,
%unsortedaddress,
%runinaddress,
%frontmatterverbose, 
%preprint,
%preprintnumbers,
%nofootinbib,
%nobibnotes,
%bibnotes,
 amsmath,amssymb,
 aps
%pra,
%prb,
%rmp,
%prstab,
%prstper,
%floatfix,
]{revtex4-1}
%\documentclass[]{article}s
%\documentclass[a4paper,12pt,final]{book}
\usepackage{amssymb}
\usepackage{cancel}
\usepackage{color,graphicx}
\usepackage{amsmath}
\usepackage{amsbsy}
\usepackage{amsthm}
\usepackage{bbm}
%\usepackage{bm}
\usepackage{epsfig}
\usepackage{float}
\usepackage{graphicx}
\usepackage{subfigure}
\usepackage{dcolumn}
\usepackage{bbm}
\usepackage{color,epstopdf}
\usepackage{amscd}
\usepackage{amsfonts}  
\usepackage[]{amsmath}
\usepackage{amssymb}    
\usepackage{mathrsfs}
\usepackage{verbatim}
\usepackage[]{cases}
\usepackage{amsmath}
\usepackage{wasysym}
\usepackage[utf8]{inputenc}
\usepackage[T1]{fontenc}
\usepackage{mathtools}
\usepackage{dsfont}
\usepackage{lipsum}  
\usepackage{hyperref}

%\usepackage{amsmath,amsfonts,amssymb,amsthm} 

\newtheorem{theorem}{Theorem}[section]
\newtheorem{lemma}[theorem]{Lemma}
\newtheorem{corollary}{Corollary}[theorem]


\DeclareMathOperator{\ein}{Ein}
\DeclareMathOperator{\ei}{E_1}
\DeclareMathOperator{\erf}{erf}
\DeclareMathOperator{\sign}{sign}
\DeclareMathOperator{\erfc}{erfc}
\DeclareMathOperator{\sinc}{sinc}



\newcommand{\be}{\begin{equation}}
\newcommand{\ee}{\end{equation}}
\newcommand{\baln}{\begin{align}}
\newcommand{\ealn}{\end{align}}
\newcommand{\ben}{\begin{equation*}}
\newcommand{\een}{\end{equation*}}
\newcommand{\tauseq}{\tau_{\text{seq}}}



\long\def\symbolfootnote[#1]#2{\begingroup%
\def\thefootnote{\fnsymbol{footnote}}\footnote[#1]{#2}\endgroup}

\newcommand{\boxk}{\Box_k^{d}}
\newcommand{\nn}{\nonumber \\}
\newcommand{\dd}{\diff}
\newcommand{\fr}{\frac}
\newcommand{\del}{\partial}
\newcommand\CS{\mathcal{C}}
\newcommand\ohat{\hat{\mathcal{O}}}
\newcommand\phat{\hat{\mathcal{P}}}
\newcommand\qhat{\hat{\mathcal{Q}}}
\newcommand\homo{\rho \fr{\del}{\del \rho}}
\newcommand\Mtwo{\mathbb{M}^2}
\newcommand\Mfour{\mathbb{M}^4}
\newcommand\expect{\mathbb{E}}
\newcommand{\mbb}{\mathbb}
\newcommand{\bbZ}{\mathbb{Z}}
\newcommand{\bbR}{\mathbb{R}}
\newcommand{\bbC}{\mathbb{C}}
\newcommand{\bbP}{\mathbb{P}}
\newcommand{\bbF}{\mathbb{F}}
\newcommand{\bbM}{\mathbb{M}}
\newcommand{\sla}{\slash\!\!\!\!}

\newcommand{\one}{\mathds{1}}
\newcommand{\E}{\mathbb{E}}
\newcommand{\euv}{e^{-\rho u^2v^2}}
\newcommand{\eav}{e^{-\rho a^2v^2}}
\newcommand{\etau}{e^{-\rho\fr{\pi}{24}\tau^4}}
\newcommand{\evol}{e^{-\rho V_0}}
\newcommand{\erfone}{\fr{\erf(\sqrt{\rho}uv)}{\sqrt{\rho}}}
\newcommand{\erftwo}{\fr{\erf(\sqrt{\rho}uv)}{\rho^{3/2}}}
\newcommand{\limrho}{\lim_{\rho\rightarrow\infty}}
%\graphicspath{{../imma/}}

\newcommand{\ket}[1]{\left| {#1} \right\rangle}
\newcommand{\bra}[1]{\left\langle {#1}\right|}
\newcommand{\braket}[1]{\langle {#1} \rangle}
\newcommand{\Tr}{\text{Tr}}
\newcommand\harpr[1]{\mathstrut\mkern2.5mu#1\mkern-11mu\raise1.5ex%
  \hbox{$\scriptscriptstyle\rightharpoonup$}}

\newcommand\harpl[1]{\mathstrut\mkern2.5mu#1\mkern-11mu\raise1.5ex%
  \hbox{$\scriptscriptstyle\leftharpoonup$}}
  
\newcommand\U[2]{\mathcal{U}_{[#1,#2]}}
% \newcommand\U[2]{\harpl{\mathcal{T}}e^{\int_{#1}^{#2}(A_{\theta}-\chi(\sigma_{\tau})C)d\tau }}

\newcommand\Ut[2]{\mathcal{U}_{[#1,#2]}^{T}}
 %\newcommand\Ut[2]{\harpr{\mathcal{T}}e^{\int_{#1}^{#2}(A_{\theta}^{T}-\chi(\sigma_{\tau}C)^{T})d\tau }}

\usepackage{graphicx}
\newcommand\smallO{
  \mathchoice
    {{\scriptstyle\mathcal{O}}}% \displaystyle
    {{\scriptstyle\mathcal{O}}}% \textstyle
    {{\scriptscriptstyle\mathcal{O}}}% \scriptstyle
    {\scalebox{.7}{$\scriptscriptstyle\mathcal{O}$}}%\scriptscriptstyle
  }



\begin{document}
\title{Binary Hypothesis Testing: Asymptotic advantage of sequential the sequential probability ratio test with respect to deterministic strategies}
\affiliation{}
\author{G. Gasbarri}
\email{giulio.gasbarri@uab.cat}
\affiliation{F\'isica Te\`orica: Informaci\'o i Fen\`omens Qu\`antics, Department de F\'isica, Universitat Aut\`onoma de Barcelona, 08193 Bellaterra (Barcelona), Spain}
\author{M. Bilkis}
\email{matias.bilkis@uab.cat}
\affiliation{F\'isica Te\`orica: Informaci\'o i Fen\`omens Qu\`antics, Department de F\'isica, Universitat Aut\`onoma de Barcelona, 08193 Bellaterra (Barcelona), Spain}
\author{J. Calsamiglia}
\email{John.caslsamiglia@uab.cat}
\affiliation{F\'isica Te\`orica: Informaci\'o i Fen\`omens Qu\`antics, Department de F\'isica, Universitat Aut\`onoma de Barcelona, 08193 Bellaterra (Barcelona), Spain}
\date{\today}
%\begin{comment}
\begin{abstract}

\end{abstract}

\maketitle

Next we show that in the asymptotic scenario the sequential probability ratio test (SPRT), outperform any deterministic strategy, in the task of discriminating between two different hypothesis.
\begin{theorem}\label{th:optepsilon}
%let $\bold{X}_{1}^{n}=(X_{1},\dots,X_{n})$ be a sample 
Let $P_{i}$,and $\E_{i}$ denote the probability and the expectation under the hypothesis $h_{i}$, $\delta =(d,\tau)$ denote a generic hypothesis test where $\tau$ is a Markov stopping time, $d=d(X_{0}^{\tau})$  is a terminal decision function with values in the set $\{0,1\}$, and $X_{0}^{\tau}$ the sample of length $\tau$. 
Let $\bold{C}(\alpha_{0},\alpha_{1})=\{\delta : P_{0}(d=1)\le \alpha_{0},P_{1}(d=0)\le \alpha_{1}\}$ and with $\ell_{n}\equiv \log \frac{P_{1}(X_{0}^{n})}{P_{0}(X_{0}^{n})}$ the log-likelihood ratio. The SPRT is defined by the couple $\delta_{s} = (d_{s},\tau_{s})$, where 
\begin{align}
 \tau_{s} = \inf\{n\ge0 ; \ell_{n}\notin (-a_{0},a_{1})\}
\end{align} 
and
\begin{align}
d_{s}=\begin{cases}
1 &\textit{if}\quad \ell_{\tau_{s}}\ge a_{1}\\
0 & \textit{if}\quad \ell_{\tau_{s}}\le -a_{0}\\
\end{cases}
\end{align}
Let $T$ be a generic time T and $A=\max\{\alpha_{0},\alpha_{1}\}$ then
\begin{align}
\lim_{A\to 0 }P_{k}(\tau\ge T) \ge P_{k}(\tau_{s}^{\epsilon}\ge T)
\end{align}
and
\begin{align}
P_{k}(\tau\ge \tau_{s}^{\epsilon}) = 1-\mathcal{O}(\alpha_{1-k}^{\varepsilon})
\end{align}
where with $\tau_{s}^{\epsilon}$ we denote SPRT stopping time in the class  $\delta_{\epsilon} \in  \bold{C}(\alpha_{0}^{1-\epsilon},\alpha_{1}^{1-\epsilon})$ with $\epsilon \in (0,1)$ and $\varepsilon= \frac{\epsilon}{1+\epsilon}$.
\end{theorem}
\begin{proof}
Let $\Omega = (-b_{0},b_{1})$
\begin{align}
&P_{0}(d=1)= \E_{1}[e^{-\ell_{\tau}} \one_{(d=1)}] \nonumber\\
&\ge \E_{1}[e^{-\ell_{\tau}}\one_{(\tau<T)\cap(d=1)\cap(\ell_{\tau \in \Omega})} ]\nonumber\\
&\ge e^{-b_{1}} P_{1}((\tau< T) \cap (d=1) \cap (\ell_{\tau}\in \Omega))\nonumber\\
&\ge e^{-b_{1}}\left( P_{1}((\tau< T) \cap (d=1))-P_{1}((\tau< T)\cap (\ell_{\tau} \notin \Omega))\right)\nonumber\\
&\ge e^{-b_{1}}\left(P_{1}(d=1) -P_{1}(\tau\ge T) -P_{1}((\tau< T)\cap (\ell_{\tau} \notin \Omega))  \right)
\end{align}
Exploiting the above inequalities, and recalling that $\alpha_{i}\ge P_{i}(d=j)$ with $j\neq i$ one obtains the following bound for the stopping time of a generic test $\delta\in \bold{C}(\alpha_{0},\alpha_{1})$:
\begin{align}\label{eq:bop}
P_{1}(\tau \ge T) \ge 1-\alpha_{1}-e^{b_{1}}\alpha_{0}-P_{1}((\tau< T)\cap (\ell_{\tau}\notin \Omega))\nonumber.
\end{align} 
What is left to do is to show that the r.h.s term can be bounded by the stopping time cumulative distribution of the SPRT.
We recall that in the SPRT:
\begin{align}\label{eq:aoa1}
a_{0} \le \log\left(\frac{1-\alpha_{0}}{\alpha_{1}}\right),\,
a_{1} \le \log\left(\frac{1-\alpha_{1}}{\alpha_{0}}\right)\nonumber\\
\end{align}
and that the cumulative distribution of the SPRT stopping time $(\tau_{s})$ is described by:
\begin{align}
P_{k}(\tau_{s}\ge T) = P_{k}(\forall\, t'\le T \, \ell_{t'}\in (-a_{0},a_{1})),\nonumber\\
P_{k}(\tau_{s} < T) = P_{k}(\exists\,\, t'\le T \, \ell_{t'}\notin (-a_{0},a_{1}))
\end{align}
The term $P_{1}(\tau<T \cap (\ell_{\tau}\notin\Omega))$ resemble the above expression for $P(\tau_{s}\le T)$, suggesting to use the inequalities in eq.~\eqref{eq:aoa1} to characterize $b_{0}$ and $b_{1}$, however making them equal to the r.h.s. of that inequality will make the term $e^{b_{1}}\alpha_{0}$ to 
 converge to one in the asymptotic regime producing a trivial result. 
To guarantee that $\lim_{A\to\infty}e^{b_{1}}\alpha_{0}=0$ we choose
\begin{align}\label{eq:b1b0}
&b_{0} = (1-\varepsilon)\log\left(\frac{1-\alpha_{0}}{\alpha_{1}}\right)\nonumber\\
&b_{1} = (1-\varepsilon)\log \left(\frac{1-\alpha_{1}}{\alpha_{0}}\right)
\end{align}
with $\varepsilon \in (0,1)$.
Under this choice for $b_{0}$ and $b_{1}$, the following inequality holds 
\begin{align}
P_{1}(\tau<T \cap (\ell_{\tau}\notin\Omega))\le P_{1}(\tau_{s}^{\epsilon}\le T )
\end{align}
and one obtains
\begin{align}
P_{1}(\tau\ge T ) \ge P_{1}(\tau_{s}^{\epsilon} \ge T)-\alpha_{1}-\alpha_{0}^{\varepsilon}(1-\alpha_{1})^{1-\varepsilon}
\end{align}
where $\epsilon=\frac{\varepsilon}{1-\varepsilon}$.
If one now make the assumption that $T=\tau_{s}^{\epsilon}$ then the inequality further simplifies as  
\begin{align}
P_{1}(\tau\ge \tau_{s}^{\epsilon} )&\ge 1-\alpha_{1}-\alpha_{0}^{\varepsilon}(1-\alpha_{1})^{1-\varepsilon}\nonumber\\
&\ge 1-\mathcal{O}(\alpha_{0}^{\varepsilon})
\end{align}
that proves the theorem for $h_{1}$ once the limit $A \to 0$ is taken.
The case for $h_{0}$ is similarly proved. 
\end{proof}

\begin{corollary}
if $\epsilon= -\frac{\delta}{\log(\alpha_{0})}$ then the theorem becomes
\begin{align}
P(\tau> \tau_{s}^{\epsilon}) \ge 1-\mathcal{O}(e^{-\delta}).
\end{align}
and the class C
\end{corollary}

\begin{corollary}
The probability that the SPRT stopping time in the class is larger than the probability  
The stopping time of the SPRT  
\begin{align}
\lim_{A\to 0}P_{k}(\tau \ge \tau_{s}^{\epsilon}) = 1
\end{align}
\end{corollary}

Let us study a deterministic strategy, where the time at which the decision occurs is fixed in advance, i.e. $\tau= \tau_{d}$. Meaning that $P_{k}(\tau > \tau_{d}) =0$, and using theorem, 
\begin{align}
\lim_{A\to 0 } P_{k}(\tau_{s}^{\epsilon}\ge \tau_{d}) = 0 
\end{align}
showing that asymptotically a sequential strategy always outperforms a deterministic one.

Let us now assume that there exists a $T^{*}$ s.t. $\lim_{A\to 0 }P_{k}(\tau_{s}\ge T^{*})= 1$ then with the help of Chebyshev inequality one can show that:
\begin{align}
\E[\tau^{n}] \ge T_{*}^{n}
\end{align}
 if $T_{*}= T_{*}(A)$ is a function of A  and $\lim_{A\to 0 }P_{k}(\tau_{s} < T_{*})= 0 $ then $\E_{k}[\tau_{s}^{n}]= (T_{*})^n(1+o(1))$, and therefore the moments of the stopping time $\tau$ associated to a generic test  $\delta \in \bold{C}(\alpha_{0},\alpha_{1})$ will be asymptotically bounded by the one of the SPRT associated to the class $\delta_{\epsilon} \in  \bold{C}(\alpha_{0}^{1-\epsilon},\alpha_{1}^{1-\epsilon})$. If furthermore, T is a continuous function of $A$ then in the limit of $\epsilon \to 0$ $\lim_{\epsilon \to 0}P_{k}(\tau_{s}^{\epsilon}> T_{*}^{\epsilon})=P_{k}(\tau_{s}^{\epsilon}> T_{*})=P_{k}(\tau_{s}> T_{*})  = 1$, and showing that under this condition the SPRT is optimal.
\begin{corollary}
Let $\tau_{s} = I_{k}(1+o_{p}(1))$ where $I$ is a positive, differentiable function of $\alpha_{0},\alpha_{1}$, then
$\mathbb{E}_{k}{\tau_{s}^{n}}\ge I_{k}^{n}(1+o(1))$, or in other words the SPRT asymptotically minimizes the moments of the stopping time distribution in the class of $\bold{C}(\alpha_{0},\alpha_{1})$.
\end{corollary}
\begin{proof}
from the hypothesis, we know that 
\begin{align}
\mathbb{E}_{k}[\tau_{s}]= I_{k}^{n}  
\end{align}
From theorem \ref{th:optepsilon} we have:
\begin{align}
P(\tau \ge I_{k}^{\epsilon})\ge P(\tau_{s}^{\epsion} \ge I_{k}^{\epsilon})
\end{align}
And in the limit of 
Using Markov inequality one get 
\begin{align}
\frac{\mathbb{E}_{k}[\tau^{n}]}{I_{k}^{\epsilon}}\le P(\tau\ge I_{k}^{\epsilon})\ge 
\end{align}
that, in the limit ot $A\to 0$, gives
\begin{align}
P(\tau\ge I_{})
\end{align}
We get that 

\end{proof}

\newpage

\begin{comment}
\begin{theorem}
Consider the hypothesis testing scenario between hypothesis $H_0$ and $H_1$, and captured by the evolution of the log-likelihood ratio $\ell _y = \frac{p(H_1|Y_t)}{p(H_0|Y_t)}$ as 
\begin{equation}
    d\ell_t = f(t, Y_t) dt + g(t, Y_t) dW_t.
\end{equation}
Let $C(\alpha_0, \alpha_1)$ denote the set of tests with asymmetric errors $p(H_0|H_1) = \alpha_0$ and $p(H_1|H_0) = \alpha_1$. We define $\tau_\delta$ to be the stopping time associated to an test $\delta \in C(\alpha_0, \alpha_1)$.%; while such quantity takes deterministic values for Neymann-Pearson-like/maximum-likelihood tests, its nature is stochastic for sequential ones.
Our result is: let $T$ be a fixed time value, then $\forall \delta \in C(\alpha_0, \alpha_1)$, provided %$\alpha_0\underset{T\rightarrow \infty}{\rightarrow}0$ 
and $\alpha_k\underset{T\rightarrow \infty}{\rightarrow}0$ for $k=0,1$, it follows that
\begin{equation}
    P(\tau_\delta \geq T |H_k) \geq P(\tau_{\text{seq}}\geq T | H_k)
\end{equation}
\end{theorem}

\begin{proof}
We will fix the underlying hypothesis to $H_1$, the proof is analogous with $k=0$. Moreover, let us denote by $d=k$ the decision over hypothesis $k$.
\begin{align}\label{eq:bounds}
    P_1(\tau \geq T) &\geq P_1((\tau \geq T) \cap (d=1)) \\
    &\geq P_1(d=1) - P_1((\tau < T) \cap (d=1)) \\
    \end{align}
Let us focus on the last object, $P_1((\tau < T) \cap (d=1))$. By noting that 
\begin{align*}
    P_0(d=1) &= \mathbb{E}_1[e^{-\ell} \mathbb{I}_{d=1}]  \noline \\
    &\geq \mathbb{E}_1[e^{-\ell} \mathbb{I}_{(\tau < T) \cap (d=1) \cap (\ell_\tau \in \Omega)}] \noline \\
    &\geq e^{-a_1} P((\tau < T) \cap (d=1) \cap (\ell_\tau \in \Omega))\\ \noline
    &\geq e^{-a_1} P((\tau < T) \cap (d=1) ) - P((\tau < T)\cap (\ell_\tau \notin \Omega)),
\end{align*}
where we defined $\Omega = [-a_0,a_1]$, with $a_i>0$, we can proceed with the bounds in ~\ref{eq:bounds} as
\begin{align*}
    P_1(\tau_\delta \geq T) &\geq P_1(d=1) - e^{a}P_0(d=1) - P((\tau < T)\cap (\ell_\tau \notin \Omega)) \\
    &\geq 1 - \alpha_0 -  e^{a_1}\alpha_1 - P(\exists s <t: l_s \notin \Omega) \\
    &\geq P(\forall \tau < t, \ell_\tau \in \Omega) - \alpha_0 -  e^{a_1}\alpha_1 \\
    &\underset{T \rightarrow \infty}{\rightarrow} P(\tauseq \geq T)
\end{align*}
Where we used that $P_1(\tau_\delta\geq T) \leq P(\exists s <t: l_s \notin \Omega)$ and that $P(\forall \tau < t, \ell_\tau \in \Omega) +  P(\exists s <t: l_s \notin \Omega) = 1$, and defined $a_1 = (1-\epsilon) \log \alpha_1$ for a fixed $\epsilon>0$, which guarantees that $e^{a_1}\alpha_1$ goes to zero in the asymptotic regime.
\end{proof}
\end{comment}

\bibliography{b2}

\end{document}



