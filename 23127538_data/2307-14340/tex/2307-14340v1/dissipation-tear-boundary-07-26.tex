%2multibyte Version: 5.50.0.2890 CodePage: 936
\documentclass[twocolumn,superscriptaddress,letter]{revtex4}%
\usepackage{amssymb}
\usepackage{color}
\usepackage{graphicx}
\usepackage{dcolumn}
\usepackage{bm}
\usepackage[header,title,page,titletoc]{appendix}
\usepackage{amsmath}
\usepackage{amsfonts}
\usepackage{epsfig}
\usepackage{soul}
\usepackage{makecell}
\usepackage{multirow}
\usepackage[dvipdfm,pdfstartview=FitH,CJKbookmarks=true,
bookmarksnumbered=true, bookmarksopen=true,linktocpage=true,
colorlinks=true,pdfborder=001,citecolor=blue,
urlcolor=blue,linkcolor=blue,anchorcolor=blue, ]{hyperref}%
\setcounter{MaxMatrixCols}{30}
%TCIDATA{OutputFilter=latex2.dll}
%TCIDATA{Version=5.50.0.2890}
%TCIDATA{Codepage=936}
%TCIDATA{Created=Sat Apr 26 08:43:06 2008}
%TCIDATA{LastRevised=Thursday, July 27, 2023 01:06:39}
%TCIDATA{<META NAME="GraphicsSave" CONTENT="32">}
%TCIDATA{<META NAME="SaveForMode" CONTENT="1">}
%TCIDATA{BibliographyScheme=Manual}
%TCIDATA{Language=American English}
%BeginMSIPreambleData
\providecommand{\U}[1]{\protect \rule{.1in}{.1in}}
%EndMSIPreambleData
\soulregister \cite7
\newcommand{\bra}[1]{\langle #1|}
\newcommand{\ket}[1]{|#1 \rangle}
\newcommand{\Tr}{\mathop{\mathrm{Tr}}}
\begin{document}
\title{Non-Hermitian tearing by dissipation}
\author{Qian Du}
\affiliation{Center for Advanced Quantum Studies, Department of Physics, Beijing Normal
University, Beijing 100875, China}
\author{Su-Peng Kou}
\thanks{Corresponding author}
\email{spkou@bnu.edu.cn}
\affiliation{Center for Advanced Quantum Studies, Department of Physics, Beijing Normal
University, Beijing 100875, China}

\begin{abstract}
In the paper, we study the non-Hermitian system under dissipation in which the
energy band shows an imaginary line gap and energy eigenstates are bound to a
specific region. To describe these phenomena, we propose the concept of
\textquotedblleft non-Hermitian tearing\textquotedblright \thinspace \ in which
the degree of tearing we defined reveals a continuous phase transition at the
exceptional point. The non-Hermitian tearing manifests in two forms --- bulk
state separation and boundary state decoupling. For a deeper understanding of
non-Hermitian tearing, we give the effective $2\times2$ Hamiltonian in the
$k$-space by reducing the $N\times N$ Hamiltonian in the real space. In
addition, we also explore the non-Hermitian tearing in the one-dimensional
Su-Schrieffer-Heeger model and the Qi-Wu-Zhang model. Our results provide a
theoretical approach for studying non-Hermitian tearing in more complex systems.

\end{abstract}

\pacs{11.30.Er, 75.10.Jm, 64.70.Tg, 03.65.-W}
\maketitle


\section{Introduction}

Non-Hermitian systems have been a hot topic owing to their unique properties
and potential applications in various fields, such as optics
\cite{C2010,A2009,Y2011,R2018,L2013,H2014,L2014,W2019}, condensed matter
physics \cite{V2017,Z2018,H2018,F2019,M2020,K2019,K2021,Y2020}, and quantum
mechanics \cite{Bender02,I2009,F2012,M2002}. In the fundamental principles of
quantum mechanics, the physical quantity describing the state of a microscopic
system is a Hermitian operator in Hilbert space, whose expected value is a
real number. However, in practice, we find that the probability of a system
does not always conserve and the eigenvalues of energy can also be complex
\cite{N2011,U2020,Bender07}. Thus, non-Hermitian operators become crucial. The
origins of the non-Hermitian Hamiltonian can be traced back to the lifetime of
a quasiparticle \cite{G1928,P1982,R1971}, columnar defects in the
superconductor \cite{H1996}, and so on. Gradually, people's understanding of
quantum mechanics extended from Hermitian systems to non-Hermitian systems.

In recent years, non-Hermitian physics has witnessed remarkable advancement.
Bender pointed out in 1998 that the energy spectrum of a Hamiltonian
satisfying parity-time ($\mathcal{PT}$) symmetry can be classified into three
cases: all real numbers, complex conjugate pairs, and a situation with
spectral degeneracy and eigenstates merging, which is known as $\mathcal{PT}%
$-symmetry spontaneously broken \cite{Bender98}. This discovery has inspired
an array of theoretical and experimental breakthroughs in non-Hermitian
physics, including the non-Hermitian skin effect
\cite{Y2018,L2020,K2020,N2020} and the breakdown of bulk-boundary
correspondence \cite{X2018,T2016,F2018,S2018,H2021,X2020}. There has been a
great deal of research focused on non-Hermitian systems with global
non-Hermitian effects, such as gain and loss \cite{H2019,Y2022,J2022}, and
nonreciprocal hopping \cite{S2023,T2019,J2023}. Meanwhile, some works are
devoted to the local non-Hermitian \cite{C2023,B2022}.

In the paper, we study a simple one-dimensional tight-binding model with
uniform hopping under dissipation where the left and right sites are subject
to different imaginary potentials. The energy band shows an imaginary line
gap, and energy eigenstates are bound to a specific region. To describe these
novel phenomena, we propose the concept of \textquotedblleft non-Hermitian
tearing\textquotedblright, in which the system is either in partial tearing or
in complete tearing. During these processes, the energy eigenvalues display a
$\mathcal{PT}$ transition. Furthermore, we define the degree of tearing to
characterize the degree to which an eigenstate is torn, and it exhibits a
continuous phase transition at the exceptional point. The non-Hermitian
tearing has two types --- bulk state separation and boundary state decoupling.
For a deeper understanding of non-Hermitian tearing, we provide the effective
$2\times2$ Hamiltonian in the $k$-space by reducing the $N\times N$
Hamiltonian in the real space, which explains these phenomena of the energy
spectrum and wave function very well. In addition, we also explore the
non-Hermitian tearing in the one-dimensional Su-Schrieffer-Heeger (SSH) model
and the Qi-Wu-Zhang (QWZ) model. Our study contributes a theoretical approach
to studying non-Hermitian tearing in more complex systems.

The outline of this paper is as follows. In Sec. II, we explore this model
under the imaginary potential and presents the concept of \textquotedblleft
non-Hermitian tearing\textquotedblright. To quantify the extent of tearing in
an eigenstate, we define the degree of tearing, which shows a continuous phase
transition at the exceptional point. Moreover, we recognize the bulk state
separation in this model. In Sec. III, we analyze the non-Hermitian tearing in
the one-dimensional SSH model and give its effective Hamiltonian, in which the
boundary states have a $\mathcal{PT}$ transition and decoupling. In Sec. IV,
we discuss the same issue in the QWZ model. In particular, its boundary states
undergo reconstruction in the case of periodic boundary condition along the
x-direction and open boundary condition along the y-direction. In Sec. V, we
draw the conclusion.

\section{Non-Hermitian tearing}

We discuss the physical phenomena of the system under dissipation where the
left and right sites are subject to different imaginary potentials by a simple
one-dimensional tight binding model with uniform hopping. Its Hamiltonian in
real space is
\begin{equation}
H_{0}=t_{1}\sum_{n=1}^{N}c_{n}^{\dag}c_{n+1}+h.c.,
\end{equation}
where $t_{1}$ represents the hopping amplitude of the electron jumping from
site $n$ to site $n+1$. $c_{n}^{\dag}$ and $c_{n}$ are the creation and
annihilation operators of electron at site $n$, respectively. The Bloch
Hamiltonian in the $k$-space is
\begin{equation}
H_{0}\left(  k\right)  =2t_{1}\sum_{k}\cos kc_{k}^{\dag}c_{k},
\end{equation}
and its eigenvalue is%
\begin{equation}
E_{0}=2t_{1}\cos k.
\end{equation}
We consider an imaginary potential for the model under the periodic boundary
condition%
\begin{equation}
V=iv_{1}\sum_{n=1}^{N_{1}}c_{n}^{\dag}c_{n}+iv_{2}\sum_{n=N_{1}+1}^{N}%
c_{n}^{\dag}c_{n},
\end{equation}
where $v_{1}\neq v_{2}$, as is shown in Fig.1. For simplicity, we set
$N_{1}=N_{2}=\frac{N}{2}$ and $v_{2}=-v_{1}=v$. In the paper, we use $t_{1}$
as the unit. % Figure environment removed

Figure 2 plots the complex energy spectra and bulk states of the system. As
the imaginary potential goes up, the eigenvalues gradually move along the
positive or negative direction of the imaginary axis. Upon reaching a
particular imaginary potential, one energy band of the system is divided into
two energy bands: an up band $E_{\mathrm{up}}$ with a positive imaginary part
and a down band $E_{\mathrm{down}}$ with a negative imaginary part, as shown
in Fig.2(c). An imaginary line gap appears between the two energy bands. In
particular, there is a $\mathcal{PT}$ transition in which the energy
eigenvalues transition from entirely real to entirely complex. Blue circles
(red circles) represent the energy eigenvalues moving out along the positive
(negative) direction of the imaginary axis, whereas black circles represent
unmoved energy eigenvalues. The wave functions of moved-out energy eigenvalues
(represented by the blue curve or the red curve) are bound to the left $N_{1}$
sites or the right $N_{2}$ sites, while the wave functions of unmoved energy
eigenvalues (represented by the black curve) still spread across all sites.

% Figure environment removed

The probability of an eigenstate $\Psi=\left(  \psi_{1},\cdots,\psi_{n}%
\cdots,\psi_{N}\right)  ^{\dag}$ of the system in the left $N_{1}$ or right
$N_{2}$ sites is $\rho_{\mathrm{L}}=\sum_{n=1}^{N_{1}}|\psi_{n}|^{2}$ or
$\rho_{\mathrm{R}}=\sum_{n=N_{1}+1}^{N}|\psi_{n}|^{2}$. In particular,
$\rho_{\mathrm{L}}+\rho_{\mathrm{R}}=1$. Given that the energy eigenvalues
move along the imaginary axis, we arrange it in ascending order by imaginary
part and show the corresponding probability in Fig.3. It is evident that the
probabilities of energy eigenvalues moving along the negative (positive)
direction of the imaginary axis are $\rho_{\mathrm{L}}>\rho_{\mathrm{R}%
}\left(  \rho_{\mathrm{L}}<\rho_{\mathrm{R}}\right)  $. The left and right
probabilities of energy eigenvalues that have not been moving out are equal
with $\rho_{\mathrm{L}}=\rho_{\mathrm{R}}=0.5$. % Figure environment removed

In order to describe the above phenomena under the imaginary potential, we
give the following concepts:

\textit{Theorem-1 --- Under the imaginary potential }$V=iv_{1}\sum
_{n=1}^{N_{1}}c_{n}^{\dag}c_{n}+iv_{2}\sum_{n=N_{1}+1}^{N}c_{n}^{\dag}%
c_{n}\left(  v_{1}<v_{2}\right)  $\textit{, we arrange the energy eigenvalues
of the system in ascending order by imaginary part, }$E_{1},\cdots
,E_{i},\cdots,E_{N}$\textit{. The corresponding eigenstates are }$\Psi
_{1},\cdots,\Psi_{i},\cdots,\Psi_{N}$\textit{ with the sort number} $i$.
\textit{In the thermodynamic limit, if }%
\[
\rho_{i,\mathrm{L}}\neq \rho_{i,\mathrm{R}}%
\]
\textit{for all eigenstates, then one energy band of the system can be divided
into two energy bands: the down energy band }$E_{\mathrm{down}}$\textit{ and
the up energy band }$E_{\mathrm{up}}$\textit{. An imaginary line energy gap
emerges between these two bands, }%
\begin{equation}
\Delta=\min \left(  \operatorname{Im}E_{\mathrm{up}}\right)  -\max \left(
\operatorname{Im}E_{\mathrm{down}}\right)  .
\end{equation}
\textit{The phenomenon is named \textquotedblleft non-Hermitian
tearing\textquotedblright.}

\textit{Theorem-2 --- Assuming that there is an eigenstate }$\Psi_{i}%
$\textit{, if }%
\[
\rho_{i,\mathrm{L}}=\rho_{i,\mathrm{L}}=0.5,
\]
\textit{then }$\Delta=0$\textit{, implying the absence of an imaginary line
energy gap. This process is called partial tearing. On the contrary, if }%
\[
\rho_{i,\mathrm{L}}\neq \rho_{i,\mathrm{R}}%
\]
\textit{holds true for all eigenstates, then }$\Delta>0$, \textit{signifying
the formation of an imaginary line energy gap. This process is called complete
tearing. }

\textit{Definition-1 --- The ratio of the probability of an eigenstate }$\Psi
$\textit{ in the left }$N_{1}$\textit{ and right }$N_{2}$\textit{ sites is
defined as the degree of tearing of this eigenstate}%
\begin{equation}
t=\frac{\rho_{\mathrm{R}}}{\rho_{\mathrm{L}}}.
\end{equation}


\textit{Definition-2 --- The tearing of a state moving perpendicular to the
direction of the domain wall is called separation. The tearing of a state
moving along the direction of the domain wall is called decoupling. \textit{In
a one-dimensional model, the direction of the domain walls formed by different
imaginary potentials is defined as the vertical direction. }}

In Fig.4 (a) and (c), we calculate the degree of tearing of the $i=11$ and the
$i=31$ bulk states, respectively. $t=1$ is in the phase with $\mathcal{PT}$
symmetry and $t\rightarrow0$ $\left(  i=11\right)  $ or $t\rightarrow \infty$
$\left(  i=31\right)  $ is in the phase with $\mathcal{PT}$-symmetry breaking.
They are continuous at the exceptional point where their energy eigenvalues
transition real to complex numbers. Furthermore, we calculate their
derivatives $\frac{\partial t}{\partial v}$ in Fig.4 (b) and (d).
$\frac{\partial t}{\partial v}$ is discontinued at the exceptional point,
which means a second-order phase transition at the exceptional
point.% Figure environment removed

To better understand non-Hermitian tearing, we give an effective $2\times2$
Hamiltonian $H_{\mathrm{eff}}$ in the $k$-space by reducing the $N\times N$
Hamiltonian $H=H_{0}+V$ of the system in the real space, which can explain the
appearance of imaginary line gap and the $\mathcal{PT}$ transition.

\textit{Theorem-3}: \textit{The effective }$2\times2$\textit{ Hamiltonian
}$H_{\mathrm{eff}}$\textit{ can be written as }$H_{\mathrm{eff}}=\sum
_{k}\left(
\begin{array}
[c]{cc}%
a_{k}^{\dag} & b_{k}^{\dag}%
\end{array}
\right)  h_{\mathrm{eff}}\left(  k\right)  \left(
\begin{array}
[c]{c}%
a_{k}\\
b_{k}%
\end{array}
\right)  $\textit{, where}
\begin{equation}
h_{\mathrm{eff}}\left(  k\right)  =\left(
\begin{array}
[c]{cc}%
h_{0}-ivI & \alpha \\
\alpha & h_{0}+ivI
\end{array}
\right)
\end{equation}
\textit{with the Bloch Hamiltonian }$h_{0}$\textit{ }of $H_{0}$\textit{ and
the coupling term }$\alpha$\textit{. Here, }$k=\frac{2\pi}{N_{1}%
}i,i=1,2,\cdots,N_{1}$\textit{ and }$I=\left(
\begin{array}
[c]{cc}%
1 & 0\\
0 & 1
\end{array}
\right)  $\textit{. The eigenvalues are}
\begin{equation}
E_{\mathrm{eff}}\left(  k\right)  =E_{0}\pm \frac{\Delta}{2},
\end{equation}
\textit{where }$E_{0}$\textit{ is eigenvalue of }$H_{0}$\textit{. As }$\Delta
$\textit{ is real, the eigenstate of the system is in the phase with }%
$PT$\textit{ symmetry; as }$\Delta$\textit{ is imaginary, the eigenstate of
the system is in the phase with }$\mathcal{PT}$\textit{-symmetry breaking.}

For the simple one-dimensional tight binding model with uniform hopping under
the imaginary potential, $h_{\mathrm{eff}}\left(  k\right)  $ is expressed as
\begin{equation}
h_{\mathrm{eff}}\left(  k\right)  =\left(
\begin{array}
[c]{cc}%
E_{0}-iv & \alpha \\
\alpha & E_{0}+iv
\end{array}
\right)  ,
\end{equation}
where $\alpha=\frac{2t\sin k}{\sqrt{N_{1}}}\cdot \lambda$ and $\lambda$ is a
fitting parameter related to $v$. The eigenvalues are
\begin{equation}
E_{\mathrm{eff}}=E_{0}\pm \sqrt{\alpha^{2}-v^{2}}.
\end{equation}
The complex energy spectra from numerical solutions of the Hamiltonian $H$ in
the real space and the analytical solutions of the effective Hamiltonian
$H_{\mathrm{eff}}$ in the $k$-space fit well, as shown in Fig.5(a). For a
given $k$, if $\alpha^{2}-v^{2}>0$, then $E_{\mathrm{eff}}$ is a real number
and this bulk state is in partial tearing. As $v^{2}>>\alpha^{2}$, then
$E_{\mathrm{eff}}$ is a complex number and all bulk states of the system is in
complete tearing. The imaginary line gap is
\begin{equation}
\Delta=2\sqrt{\alpha^{2}-v^{2}}.
\end{equation}
As $\alpha^{2}-v^{2}=0$, that is,
\begin{equation}
k=k_{0}=\pm \arcsin \frac{v\sqrt{N_{1}}}{2t\lambda},
\end{equation}
then $E_{\mathrm{eff}}\left(  k_{0}\right)  =E_{0}\left(  k_{0}\right)  $ and
the bulk state is at the $\mathcal{PT}$ transition. Because bulk states move
perpendicular to the direction of the domain wall, bulk states show
separation. The eigenstates are
\begin{equation}
\Psi_{\pm}=\frac{1}{A}\binom{1}{\frac{v\pm \sqrt{\alpha^{2}-v^{2}}}{\alpha}},
\end{equation}
where $A=\sqrt{1+\left(  \frac{v\pm \sqrt{\alpha^{2}-v^{2}}}{\alpha}\right)
^{2}}$. The corresponding probabilities are
\begin{align}
\rho_{+,\mathrm{L}}  &  =\frac{\alpha^{2}}{\alpha^{2}+\left(  v+\sqrt
{\alpha^{2}-v^{2}}\right)  ^{2}},\rho_{+,\mathrm{R}}=\frac{\left(
v+\sqrt{\alpha^{2}-v^{2}}\right)  ^{2}}{\alpha^{2}+\left(  v+\sqrt{\alpha
^{2}-v^{2}}\right)  ^{2}},\nonumber \\
\rho_{-,\mathrm{L}}  &  =\frac{\alpha^{2}}{\alpha^{2}+\left(  v-\sqrt
{\alpha^{2}-v^{2}}\right)  ^{2}},\rho_{-,\mathrm{R}}=\frac{\left(
v-\sqrt{\alpha^{2}-v^{2}}\right)  ^{2}}{\alpha^{2}+\left(  v-\sqrt{\alpha
^{2}-v^{2}}\right)  ^{2}}.
\end{align}
The degrees of tearing are
\begin{align}
t_{+}  &  =\frac{\left(  v+\sqrt{\alpha^{2}-v^{2}}\right)  ^{2}}{\alpha^{2}%
},\nonumber \\
t_{-}  &  =\frac{\left(  v-\sqrt{\alpha^{2}-v^{2}}\right)  ^{2}}{\alpha^{2}}.
\end{align}
Besides, we also give the fitting parameter $\lambda$ as a function of $v$ in
Fig.5(b). $\lambda$ reaches a saturation value $1$ when $v$ is increased to a
large value.

% Figure environment removed

Given a more general case, a complex potential for the system is:%
\begin{align}
V  &  =v_{1}e^{i\phi_{1}}\sum_{i=1}^{N_{1}}c_{i}^{\dag}c_{i}+v_{2}e^{i\phi
_{2}}\sum_{i=N_{1}+1}^{N_{1}+N_{2}}c_{i}^{\dag}c_{i}\nonumber \\
&  +\cdots+v_{n}e^{i\phi_{n}}\sum_{i=N_{n-1}+1}^{N_{1}+\cdots+N_{n}}%
c_{i}^{\dag}c_{i},
\end{align}
where the real numbers $v_{1,2,\cdots,n}$ represent the magnitudes of the $n$
uniform complex potentials and $\phi_{1,2,\cdots,n}$ are their phase angle
with $\phi_{1,2,\cdots,n}\in \left[  0,2\pi \right]  $. Here, $N=N_{1}%
+N_{2}+\cdots+N_{n}$. In the system, the uniform complex potential
$v_{1}e^{i\phi_{1}}$ is applied to the left $N_{1}$ sites, the uniform complex
potential $v_{2}e^{i\phi_{2}}$ is applied to the right $N_{2}$ sites, and so
on. We take a two-dimensional square lattice model as an example and consider
the complex potential in Fig.6(a). The energy spectra of the system are torn
to different positions along directions, which depends on the form of the
uniform complex potential in Fig.6(b). The direction is the phase angle of the
uniform complex potential, and the position is relevant to the amplitude of
the uniform complex potential, as shown in Fig.6(b).

% Figure environment removed

\section{Non-Hermitian tearing in one-dimensional SSH model}

In this section, we consider the one-dimensional tight binding model with
non-uniform hopping. Here, we take the one-dimensional SSH model as an
example, which its Hamiltonian is
\begin{equation}
H_{\mathrm{SSH}}=t_{1}\sum_{n=1}^{N}|n,B\rangle \langle n,A|+t_{2}\sum
_{n=1}^{N-1}|n+1,A\rangle \langle n,B|+h.c.,
\end{equation}
where $A$ and $B$ denote the two sublattices of each pair of lattice sites.
$t_{1}$ and $t_{2}$ describe the intra-cell and inter-cell hopping strengths,
respectively. Here, we set $t_{2}=2t_{1}=2$. The Bloch Hamiltonian of this
model is
\begin{equation}
H_{\mathrm{SSH}}\left(  k\right)  =\left(  t_{1}+t_{2}\cos k\right)
\sigma_{x}+\left(  t_{2}\sin k\right)  \sigma_{y},
\end{equation}
where $\sigma_{i}$'s are the Pauli matrices acting on the sublattice subspace.
Its eigenvalue in the $k$-space is
\begin{equation}
E_{\pm}\left(  k\right)  =\pm \sqrt{\left(  t_{1}+t_{2}\cos k\right)
^{2}+\left(  t_{2}\sin k\right)  ^{2}}.
\end{equation}
Here, we consider the imaginary potential for the one-dimensional SSH model,
as shown in Fig.7.

% Figure environment removed

We show the complex energy spectra of the model under the periodic boundary
condition in Fig.8 and Fig.9. The one-dimensional SSH model also exhibits
non-Hermitian tearing, accompanied by a $\mathcal{PT}$ phase transition and
the separation of bulk states. However, after the tearing of bulk states,
these two pairs of boundary states emerge in the system, as depicted in Fig.9.
When the strength of imaginary potential $v$ continues to increase, these two
pairs of boundary states also show non-Hermitian tearing with a $\mathcal{PT}$
phase transition. As a result, the wave function of the boundary state,
depicted by the blue curve with asterisks corresponding to the down energy
band (represented by blue asterisks), is localized on the two boundaries of
the left $N/2$ sites. On the contrary, the wave function of the boundary
state, depicted by the red curve with red plus signs corresponding to the up
energy band (represented by red plus signs), is localized on the two
boundaries of the right $N/2$ sites in Fig.9(f). Since these boundary states
move along the direction of domain wall, their tearing is decoupling.

% Figure environment removed

% Figure environment removed

To better study the tearing of boundary states, we calculate the degree of
tearing of the $i=1$ and the $i=4$ boundary states in Fig.10 (a) and (c).
$t=1$ is in the phase with $\mathcal{PT}$ symmetry and $t\rightarrow0$
$\left(  i=1\right)  $ or $t\rightarrow \infty$ $\left(  i=4\right)  $ is in
the phase with $\mathcal{PT}$-symmetry breaking. They are continuous at the
exceptional point where their energy eigenvalues transition real to complex
numbers. Their derivatives $\frac{\partial t}{\partial v}$ are displayed in
Fig.10 (b) and (d), in which $\frac{\partial t}{\partial v}$ is discontinued
at the exceptional point, implying a second-order phase transition at the
exceptional point.

% Figure environment removed

According to Theorem-3, we give the effective Hamiltonian of one-dimensional
SSH model under the imaginary potential with the periodic boundary condition
to understand them. Firstly, the effective Hamiltonian of bulk states is
written as
\begin{equation}
h_{\mathrm{eff}}=\left(
\begin{array}
[c]{cc}%
h_{11} & h_{12}\\
h_{21} & h_{22}%
\end{array}
\right)  ,
\end{equation}
where
\begin{align}
h_{11}  &  =\left(
\begin{array}
[c]{cc}%
-iv & t_{1}+t_{2}e^{-ik}\\
t_{1}+t_{2}e^{ik} & -iv
\end{array}
\right)  ,\nonumber \\
h_{12}  &  =h_{21}=i\frac{\lambda}{\sqrt{N_{1}}}\left(
\begin{array}
[c]{cc}%
0 & t_{1}+t_{2}e^{-i\left(  k+\pi \right)  }\\
-\left[  t_{1}+t_{2}e^{i\left(  k+\pi \right)  }\right]  & 0
\end{array}
\right)  ,\nonumber \\
h_{22}  &  =\left(
\begin{array}
[c]{cc}%
iv & t_{1}+t_{2}e^{-ik}\\
t_{1}+t_{2}e^{ik} & iv
\end{array}
\right)  .
\end{align}
Here, $k=\frac{2\pi}{N_{1}}i,i=1,2,\cdots,N_{1}$. These solutions from the
Hamiltonian $H=H_{\mathrm{SSH}}+V$ in the real space and the effective
Hamiltonian $H_{\mathrm{eff}}$ in the $k$-space for bulk states fit well in Fig.11.

Secondly, the effective Hamiltonian of boundary states is given by
$h_{\mathrm{eff}}=I\otimes h$, where%
\begin{equation}
h=iv\sigma_{z}+\sqrt{3}\sigma_{y},
\end{equation}
and $I=\left(
\begin{array}
[c]{cc}%
1 & 0\\
0 & 1
\end{array}
\right)  $. The eigenvalue is
\begin{equation}
E_{\mathrm{eff}}=\pm \sqrt{3-v^{2}},
\end{equation}
The solutions of the effective Hamiltonian $H_{\mathrm{eff}}$ in the $k$-space
for boundary states agree with those of the Hamiltonian $H=H_{\mathrm{SSH}}+V$
in the real space in Fig.12. There is a $\mathcal{PT}$ phase transition: in
the case of $v<\sqrt{3}$, the eigenvalues are all real and boundary states are
at $\mathcal{PT}$ -symmetry; and in the case of $v>\sqrt{3}$, the eigenvalues
are all imaginary and boundary states are at $\mathcal{PT}$ -broken. At
$v=\sqrt{3}$, boundary states are at the exceptional point with energy
degeneracy, i.e., $E_{\mathrm{eff}}\left(  k\right)  =0$. % Figure environment removed

% Figure environment removed

\section{Non-Hermitian tearing in the QWZ model}

In the section, we take the QWZ model as an example and discuss its
non-Hermitian tearing. The Hamiltonian of the QWZ model in the real space is
\begin{align}
H_{\mathrm{QWZ}}  &  =\sum_{m_{x}=1}^{N_{x}-1}\sum_{m_{y}=1}^{N_{y}}\left(
|m_{x}+1,m_{y}\rangle \langle m_{x},m_{y}|\otimes t_{x}+h.c.\right) \nonumber \\
&  +\sum_{m_{x}=1}^{N_{x}}\sum_{m_{y}=1}^{N_{y}-1}\left(  |m_{x}%
,m_{y}+1\rangle \langle m_{x},m_{y}|\otimes t_{y}+h.c.\right) \nonumber \\
&  +u\sum_{m_{x}=1}^{N_{x}}\sum_{m_{y}=1}^{N_{y}}|m_{x},m_{y}\rangle \langle
m_{x},m_{y}|\otimes \sigma_{z},
\end{align}
where $u$ is the staggered on site potential. The model describes a particle
with two internal states hopping on a lattice where the nearest neighbour
hopping is accompanied by an operation on the internal degree of freedom, and
this operation is different for the hopping along the x with $t_{x}%
=\frac{\sigma_{z}+i\sigma_{x}}{2}$ and y directions with $t_{y}=\frac
{\sigma_{z}+i\sigma_{y}}{2}$. The Bloch Hamiltonian in the $k$-space is given
by
\begin{equation}
H_{\mathrm{QWZ}}\left(  k\right)  =\sin k_{x}\cdot \sigma_{x}+\sin k_{y}%
\cdot \sigma_{y}+\left(  \cos k_{x}+\cos k_{y}+u\right)  \cdot \sigma_{z},
\end{equation}
and its eigenvalue is
\begin{equation}
E_{\pm}=\pm \sqrt{\left(  \sin k_{x}\right)  ^{2}+\left(  \sin k_{y}\right)
^{2}+\left(  \cos k_{x}+\cos k_{y}+u\right)  ^{2}}.
\end{equation}
The imaginary potential is
\begin{align}
V  &  =-iv\sum_{m_{x}=1}^{L/2}\sum_{m_{y}=1}^{N/2}|m_{x},m_{y}\rangle \langle
m_{x},m_{y}|\otimes I\nonumber \\
&  +iv\sum_{m_{x}=L/2+1}^{L}\sum_{m_{y}=N/2+1}^{N}|m_{x},m_{y}\rangle \langle
m_{x},m_{y}|\otimes I.
\end{align}


% Figure environment removed

We first discuss the case of periodic boundary along the x-direction and the
y-direction, as plotted in Fig13. The complex energy spectra the model are
shown in Fig.14 and Fig.15. Obviously, bulk states have a $\mathcal{PT}$
transition and separation. After the separation of bulk states, the system
exhibit a number of boundary states in Fig.15. Each energy eigenvalue of the
boundary state has two eigenstates. Therefore, here we only take one of the
eigenstates in Fig.15.

% Figure environment removed

% Figure environment removed

With the increase of the imaginary potential $v$, these two pairs of boundary
states show non-Hermitian tearing with a $\mathcal{PT}$ phase transition. As a
consequence, the wave functions of boundary states, corresponding the down
energy band with a negative imaginary part (represented by blue asterisks),
are only localized at the position M or N, in Fig13.(b). The wave functions of
boundary states, corresponding the up energy band with a positive imaginary
part (represented by red plus signs), are only localized at the position O or
P, in Fig13.(b). The chiral direction of boundary states is along the
direction of domain walls, thus boundary states appear decoupling. Moreover,
we plot the probabilities $\rho_{\mathrm{L},\mathrm{R}}$ of boundary states in
Fig.16. The probabilities of boundary states that have not been torn are
$\rho_{\mathrm{L}}=\rho_{\mathrm{R}}=0.5$, whereas the probabilities of
boundary states that have been torn are $\rho_{\mathrm{L}}\neq \rho
_{\mathrm{R}}$. We calculate the degree of tearing and its derivative of the
$i=350$ and the $i=450$ eigenstates in Fig.17. There is a second-order phase
transition about $t$ at the exceptional point.

% Figure environment removed

% Figure environment removed

We provide the effective model for this situation. The effective Hamiltonian
of bulk states is
\begin{equation}
h_{\mathrm{eff}}=\left(
\begin{array}
[c]{cc}%
h_{11} & h_{12}\\
h_{21} & h_{22}%
\end{array}
\right)  ,
\end{equation}
where%
\begin{align}
h_{11} &  =H_{\mathrm{QWZ}}\left(  k\right)  -ivI,\nonumber \\
h_{12} &  =h_{21}=\frac{\lambda_{1}}{\sqrt{L_{1}}}\left[  \sin \left(
k_{x}+\pi \right)  \cdot \sigma_{x}+\cos \left(  k_{x}+\frac{\pi}{2}\right)
\cdot \sigma_{z}\right]  ,\nonumber \\
h_{22} &  =H_{\mathrm{QWZ}}\left(  k\right)  +ivI.
\end{align}
Here, $k=\frac{2\pi}{L_{1}}i,i=1,2,\cdots,L_{1}$ and $L_{1}=L/2$. For the
boundary states, the effective Hamiltonian is%
\begin{equation}
h_{\mathrm{eff}}=\left(
\begin{array}
[c]{cc}%
-\sin k_{x}-iv & \frac{\lambda_{2}}{\sqrt{L_{1}}}\\
\frac{\lambda_{2}}{\sqrt{L_{1}}} & \sin k_{x}+iv
\end{array}
\right)  .
\end{equation}
The numerical solutions of the Hamiltonian $H=H_{\mathrm{SSH}}+V$ (blue
circles) in the real space and the analytical solutions of the effective
Hamiltonian $H_{\mathrm{eff}}$ (red dots) in the $k$-space for bulk states and
boundary states are shown in Fig.18.

% Figure environment removed

Furthermore, we study the case of the periodic boundary condition along the
x-direction and the open boundary condition along the y-direction, as shown in
Fig.19. In the absence of imaginary potential, the system exhibits a series of
intrinsic boundary states that are localized on one side of the y-direction
boundary (at the position Q or R). Here, we only provide the wave function of
one of intrinsic boundary states in Fig.20(a2). Obviously, there are also the
$\mathcal{PT}$ phase transition and the separation of bulk states in Fig.20.

% Figure environment removed% Figure environment removed

As the imaginary potential $iv$ increases, new boundary states (represented by
blue asterisks or red plus signs) appear and form a $\mathcal{PT}$ phase
transition of boundary states with intrinsic boundary states (represented by
black crosses) in Fig.21. The wave function of the intrinsic boundary state
corresponding the energy band with zero imaginary part (represented by black
crosses) is equally located on the both sides of the domain wall ST or the
domain wall UV due to chiral skin effect. Due to the unequal imaginary
potentials of $-iv$ and $iv$ at the two domain walls, the wave function at
both sides of the two domain walls tends to localize at points S or T
\cite{X2023} in Fig21.(a1) and (a3). During this process, intrinsic boundary
states and the wave function at both sides of the two domain walls occur
\emph{reconstruction}. After the non-Hermitian tearing of boundary states,
these boundary states also show decoupling. The wave function of the boundary
state corresponding the down energy band with negative imaginary part (blue
asterisks) is only distributed at the position S and U, in Fig21.(c3). The
wave function of the boundary state corresponding the up energy band with
positive imaginary part (red plus signs) is only distributed at the position T
and V, in Fig21.(c1).

% Figure environment removed

\section{Conclusions}

In the paper, we study a simple one-dimensional tight-binding model with
uniform hopping under the imaginary potential, in which an imaginary line gap
and a $\mathcal{PT}$ transition in the energy band are discovered. For these
phenomena, we present the non-Hermitian tearing in which the degree of tearing
we defined exhibits a continuous phase transition at the exceptional point. In
particular, bulk states of this model display separation. To better
understanding the non-Hermitian tearing, we provide the effective $2\times2$
Hamiltonian in the $k$-space by reducing the $N\times N$ Hamiltonian in the
real space. Moreover, we discuss the non-Hermitian tearing in the
one-dimensional SSH model and give its effective Hamiltonian, in which its
boundary states have a $\mathcal{PT}$ transition and decoupling. The same
issue is also analysed in the QWZ model. Notably, we find the reconstruction
of the boundary states in the case of periodic boundary condition along the
x-direction and open boundary condition along the y-direction. In the future,
we will study the more complex models by dissipation and search for the
possible exotic phenomena.

\acknowledgments This work was supported by NSFC Grant No. 11974053, 12174030.

\begin{thebibliography}{99}                                                                                               %


\bibitem {C2010}C. E. R\"{u}er, K. G. Makris, R. El-Ganainy, D. N.
Christodoulides, M. Segev, and D. Kip, Observation of parity--time symmetry in
optics, Nat. Phys. \textbf{6}, 192 (2010).

\bibitem {A2009}A. Guo, G. J. Salamo, D. Duchesne, R. Morandotti, M.
Volatier-Ravat, V. Aimez, G. A. Siviloglou, and D. N. Christodoulides,
Observation of $\mathcal{PT}$-Symmetry Breaking in Complex Optical Potentials
Phys. Rev. Lett. \textbf{103}, 093902 (2009).

\bibitem {Y2011}Y. D. Chong, L. Ge, and A. D. Stone, $\mathcal{PT}$-Symmetry
Breaking and Laser-Absorber Modes in Optical Scattering Systems, Phys. Rev.
Lett. \textbf{106}, 093902 (2011).

\bibitem {R2018}R. El-Ganainy, K. G. Makris, M. Khajavikhan, Z. H. Musslimani,
S. Rotter, and D. N. Christodoulides, Non-Hermitian physics and PT symmetry,
Nat. Phys. \textbf{14}, 11 (2018).

\bibitem {L2013}L. Feng, Y.-L. Xu, W. S. Fegadolli, M.-H. Lu, J. E. B.
Oliveira, V. R. Almeida, Y.-F. Chen, and A. Scherer, Experimental
demonstration of a unidirectional reflectionless parity-time metamaterial at
optical frequencies, Nat. Mater. \textbf{12}, 108 (2013).

\bibitem {H2014}H. Hodaei, M.-A. Miri, M. Heinrich, D. N. Christodoulides, and
M. Khajavikhan, Parity-time-symmetric microring lasers, Science \textbf{346},
975 (2014).

\bibitem {L2014}L. Feng, Z. J. Wong, R.-M. Ma, Y. Wang, and X. Zhang, Single
mode laser by parity-time symmetry breaking, Science \textbf{346}, 972 (2014).

\bibitem {W2019}W. Song, W. Sun, C. Chen, Q. Song, S. Xiao, S. Zhu, and T. Li,
Breakup and Recovery of Topological Zero Modes in Finite Non-Hermitian Optical
Lattices, Phys. Rev. Lett. \textbf{123}, 165701 (2019).

\bibitem {V2017}V. Kozii and L. Fu, Non-Hermitian Topological Theory of
Finite-Lifetime Quasiparticles: Prediction of Bulk Fermi Arc due to
Exceptional Point, arXiv:1708.05841.

\bibitem {Z2018}Z. Gong, Y. Ashida, K. Kawabata, K. Takasan, S. Higashikawa,
and M. Ueda, Topological Phases of Non Hermitian Systems, Phys. Rev. X
\textbf{8}, 031079 (2018).

\bibitem {H2018}H. Shen, B. Zhen, and L. Fu, Topological Band Theory for
Non-Hermitian Hamiltonians, Phys. Rev. Lett. \textbf{120}, 146402 (2018).

\bibitem {F2019}F. Song, S. Yao, and Z. Wang, Non-Hermitian Topological
Invariants in Real Space, Phys. Rev. Lett. \textbf{123}, 246801 (2019).

\bibitem {M2020}N. Matsumoto, K. Kawabata, Y. Ashida, S. Furukawa, and M.
Ueda, Continuous Phase Transition without Gap Closing in Non-Hermitian Quantum
Many-Body Systems, Phys. Rev. Lett. \textbf{125}, 260601 (2020).

\bibitem {K2019}K. Kawabata, K. Shiozaki, M. Ueda, and M. Sato, Symmetry and
Topology in Non-Hermitian Physics, Phys. Rev. X \textbf{9}, 041015 (2019).

\bibitem {K2021}K. Kawabata, K. Shiozaki, and S. Ryu, Topological Field Theory
of Non-Hermitian Systems, Phys. Rev. Lett. \textbf{126}, 216405 (2021).

\bibitem {Y2020}Y. Michishita and R. Peters, Equivalence of Effective
Non-Hermitian Hamiltonians in the Context of Open Quantum Systems and Strongly
Correlated Electron Systems, Phys. Rev. Lett. \textbf{124}, 196401 (2020).

\bibitem {Bender02}C. M. Bender, D. C. Brody, and H. F. Jones, Complex
Extension of Quantum Mechanics, Phys. Rev. Lett. \textbf{89}, 270401 (2002).

\bibitem {I2009}I. Rotter, A non-Hermitian Hamilton operator and the physics
of open quantum systems, J. Phys. A: Math. Theor. \textbf{42}, 153001 (2009).

\bibitem {F2012}F. Reiter and A. S. \textbf{S\O ensen}, Effective operator
formalism for open quantum systems, Phys. Rev. A \textbf{85}, 032111 (2012).

\bibitem {M2002}A. Mostafazadeh, Pseudo-Hermiticity versus PT symmetry: The
necessary condition for the reality of the spectrum of a non-Hermitian
Hamiltonian J. Math. Phys. \textbf{43 }205 (2002); A. Mostafazadeh,
Pseudo-Hermiticity versus PT-symmetry. II. A complete characterization of
non-Hermitian Hamiltonians with a real spectrum, ibid, \textbf{43} 2814
(2002); A. Mostafazadeh A, Pseudo-Hermiticity versus PT-symmetry III:
Equivalence of pseudo-Hermiticity and the presence of antilinear symmetries,
ibid, \textbf{43} 3944, (2002).

\bibitem {N2011}N. Moiseyev, Non-Hermitian Quantum Mechanics (Cambridge
University Press, Cambridge, 2011).

\bibitem {U2020}Y. Ashida, Z. Gong, and M. Ueda, Non-Hermitian physics, Adv.
Phys. \textbf{69}, 249 (2020).

\bibitem {Bender07}C. M. Bender, Making sense of non-Hermitian Hamiltonians,
Rep. Prog. Phys. \textbf{70}, 947 (2007).

\bibitem {G1928}G. Gamow, Zur Quantentheorie des Atomkernes, Z. Phys.
\textbf{51}, 204 (1928).

\bibitem {P1982}P. M. Radmore and P. L. Knight, Population trapping and
dispersion in a three-level system, J. Phys. B: Atom. Mol. Phys., \textbf{15},
561 (1982).

\bibitem {R1971}R. M. More, Theory of decaying states, Phys. Rev. A
\textbf{4}, 1782 (1971).

\bibitem {H1996}N. Hatano and D. R. Nelson, Localization Transitions in
Non-Hermitian Quantum Mechanics, Phys. Rev. Lett. \textbf{77}, 570 (1996).

\bibitem {Bender98}C. M. Bender, and S. Boettcher, Phys. Rev. Lett.
\textbf{80}, 5243 (1998).

\bibitem {Y2018}S. Yao and Z. Wang, Edge States and Topological Invariants of
Non-Hermitian Systems, Phys. Rev. Lett. \textbf{121}, 086803 (2018).

\bibitem {L2020}L. Li, C. H. Lee, S. Mu, and J. Gong, Critical non-Hermitian
skin effect, Nat. Commun. \textbf{11}, 5491 (2020).

\bibitem {K2020}K. Kawabata, M. Sato, and K. Shiozaki, Higher-order
non-hermitian skin effect, Phys. Rev. B \textbf{102}, 205118 (2020).

\bibitem {N2020}N. Okuma, K. Kawabata, K. Shiozaki, and M. Sato, Topological
Origin of Non-Hermitian Skin Effects, Phys. Rev. Lett. \textbf{124}, 086801 (2020).

\bibitem {X2018}Y. Xiong, Why does bulk boundary correspondence fail in some
non-hermitian topological models, J. Phys. Commun. \textbf{2} 035043 (2018).

\bibitem {T2016}T. E. Lee, Anomalous Edge State in a Non-Hermitian Lattice,
Phys. Rev. Lett. \textbf{116}, 133903 (2016).

\bibitem {F2018}F. K. Kunst, E. Edvardsson, J. C. Budich, and Emil J.
Bergholtz, Biorthogonal Bulk-Boundary Correspon dence in Non-Hermitian
Systems, Phys. Rev. Lett. \textbf{121}, 026808 (2018).

\bibitem {S2018}S. Yao, F. Song and Z. Wang, Non-Hermitian Chern Bands, Phys.
Rev. Lett. \textbf{121}, 136802 (2018).

\bibitem {H2021}H.-G. Zirnstein, G. Refael, and B. Rosenow, Bulk Boundary
Correspondence for Non-Hermitian Hamilto nians via Green Functions, Phys. Rev.
Lett. \textbf{126}, 216407 (2021).

\bibitem {X2020}X.-R. Wang, C.-X. Guo, and S.-P. Kou, Defective edge states
and number-anomalous bulk-boundary correspondence in non-Hermitian topological
systems, Phys. Rev. B \textbf{101}, 121116(R) (2020).

\bibitem {H2019}H. Zhao, X. Qiao, T. Wu, B. Midya, S. Longhi, and L. Feng,
Non-Hermitian topological light steering, Science \textbf{365}, 1163 (2019).

\bibitem {Y2022}Y. Li, C. Fan, X.g Hu, Y. Ao, C. Lu, C. T. Chan, D. M. Kennes,
and Q. Gong, Effective Hamiltonian for Photonic Topological Insulator with
Non-Hermitian Domain Walls, Phys. Rev. Lett. \textbf{129}, 053903 (2022).

\bibitem {J2022}Y.-J. Wu, C.-C. Liu and J. Hou, Wannier-type photonic
higher-order topological corner states induced solely by gain and loss, Phys.
Rev. A 101, 043833 (2020).

\bibitem {S2023}S. Jana, and L. Sirota, Emerging exceptional point with
breakdown of skin effect in non-Hermitian systems, arXiv:2303.15050v2 (2023).

\bibitem {T2019}T.-S. Deng and W. Yi, Non-Bloch topological invariants in a
non-Hermitian domain wall system, Phys. Rev. B \textbf{100}, 035102 (2019).

\bibitem {J2023}J.-R. Li, C. Luo, L.-L. Zhang, S.-F. Zhang, P.-P. Zhu, and
W.-J. Gong, Band structures and skin effects of coupled nonreciprocal
Su-Schrieffer-Heeger lattices, Phys. Rev. A \textbf{107}, 022222 (2023).

\bibitem {C2023}C.-X. Guo, X. Wang, H. Hu, and S. Chen, Accumulation of
scale-free localized states induced by local non-Hermiticity, Phys. Rev. B
\textbf{107}, 134121 (2023).

\bibitem {B2022}B. Li, H.-R. Wang, F. Song and Z. Wang, Scale-free
localization and pt symme try breaking from local non-hermiticity,
arXiv:2302.04256v1 (2022).

\bibitem {X2023}X. Ma, K. Cao, X. Wang, Z. Wei, S.-P. Kou Kou, Chiral Skin
Effect, arXiv:2304.01422v1 (2023).
\end{thebibliography}


\end{document}
