
\section{MAIN RESULTS}
\vspace{-1ex}
\subsection{Triangular Scale Parameterization}\label{section:triangular_scale}
\vspace{-1ex}

First, we will demonstrate a parameterization that is more computationally efficient than the matrix square-root parameterization considered in \cref{section:scale_parameterization}, while satisfying the constraints \labelcref{item:positive_definite,item:linearity,item:convexity,item:smooth_entropy}.
We first turn our attention to the following domain for the variational parameters:
{%
\setlength{\belowdisplayskip}{1ex} \setlength{\belowdisplayshortskip}{1ex}
\setlength{\abovedisplayskip}{1ex} \setlength{\abovedisplayshortskip}{1ex}
\begin{align*}
  \Lambda_{S} \triangleq
  \left\{\, \left(\vm, \mC\right) \in \mathbb{R}^{d} \times \mathbb{L}_{++}^d \mid \sigma_{\mathrm{min}}\left(\mC\right) \geq
   1/\sqrt{S} \,\right\},
\end{align*}
}%
where \(\mathbb{L}_{++}^d\) is the set of Cholesky factors.
A key special case is the mean-field variational family, which is a strict subset of \(\Lambda_S\), where we restrict \(\mC\) to be diagonal matrices.
With that said, we consider the two following parameterizations:
{
\setlength{\belowdisplayskip}{1.ex} \setlength{\belowdisplayshortskip}{1.ex}
\setlength{\abovedisplayskip}{1.ex} \setlength{\abovedisplayshortskip}{1.ex}
\begin{alignat*}{4}
  \mC &= \mL, 
  &&\qquad
  \text{(full-rank)}
  &&\qquad
  \\
  \mC &= \mathrm{diag}\left(L_{11}, \ldots, L_{dd}\right),
  &&\qquad
  \text{(mean-field)}
\end{alignat*}
}%
where \(\mL\) is a Cholesky factor.
In practice, the triangular matrix parameterization is most commonly used~\citep{kucukelbir_automatic_2017}, and results in lower gradient variance than the square root parameterization \citep{kim_practical_2023}.

\vspace{-1ex}
\paragraph{Projection Operator}
The key advantage of operating with triangular scale matrices is that the entropy is the log-sum of their eigenvalues, which turns out to be their diagonal elements.
This implies that the gradient of the entropy term \(\vlambda \mapsto \mathbb{H}\left(q_{\vlambda}\right)\) only resides on the diagonal subspace of \(\mC\).
Therefore, the ``smoothness'' of \(\vlambda \mapsto \mathbb{H}\left(q_{\vlambda}\right)\) can be achieved by only controlling the eigenvalues (or diagonal elements) of \(\mC\).
This sharply contrasts with the square-root parameterization where the constraint is on the \textit{singular values}, which are much harder to control.
Nevertheless, the canonical Euclidean projection operator is:
\begin{proposition}
    The Euclidean projection operator onto \(\Lambda_S\), 
    \(\mathrm{proj}_{\Lambda_S} : \mathbb{R}^d \times \mathbb{L}^d \to \Lambda_{S} \), is given as
{%
\setlength{\belowdisplayskip}{1ex} \setlength{\belowdisplayshortskip}{1ex}
\setlength{\abovedisplayskip}{1ex} \setlength{\abovedisplayshortskip}{1ex}
    \begin{align*}
       \mathrm{proj}_{\Lambda_S}\left(\vlambda\right) 
       = \argmin_{\vlambda' \in \Lambda_S} \norm{\vlambda - \vlambda'}_2^2
       = \left(\vm, \widetilde{\mC}\right),
    \end{align*}
}
    where \(\widetilde{\mC}\) is the projection of \(\mC\) such that
{%
\setlength{\belowdisplayskip}{1.5ex} \setlength{\belowdisplayshortskip}{1.5ex}
\setlength{\abovedisplayskip}{1.5ex} \setlength{\abovedisplayshortskip}{1.5ex}
    \[
      \widetilde{C}_{ij} = \begin{cases}
        \; \max\left(C_{ii}, \; 1/\sqrt{S}\right) &\text{for } i = j \\
        \; C_{ij} &\text{for } i \neq j.
      \end{cases}
    \]
}%
\end{proposition}
\vspace{-2ex}
\begin{proof}
    Since the eigenvalues of a triangular matrix are its diagonal elements, we notice that \(\Lambda_{S}\) is a constraint only on the diagonal elements of \(\mC\) such that \(C_{ii} \geq 1/\sqrt{S}\).
    Conveniently, this is an element-wise half-space constraint for which the projection follows as
{%
\setlength{\belowdisplayskip}{0ex} \setlength{\belowdisplayshortskip}{0ex}
\setlength{\abovedisplayskip}{1.5ex} \setlength{\abovedisplayshortskip}{1.5ex}
    \[
       \widetilde{C}_{ii} 
       = \argmin_{c \geq 1/\sqrt{S}} \norm{C_{ii} - c}_2^2
       = \max\left(C_{ii}, 1/\sqrt{S}\right).
    \]
}%    
\vspace{-1ex}
\end{proof}

\vspace{-1ex}
\paragraph{Theoretical Properties}
We will now prove that our construction is valid.
It is trivial that \labelcref{item:positive_definite,item:linearity,item:convexity} are satisfied.
We formally prove that \(\Lambda_S\) satisfies \labelcref{item:smooth_entropy}:

\begin{theoremEnd}[category=entropysmooth]{proposition}\label{thm:entropy_smoothness}
    The entropy \(\vlambda \mapsto \mathbb{H}\left(q_{\vlambda}\right)\) is \(S\)-Lipschitz smooth on \(\Lambda_S\).
\end{theoremEnd}
\begin{proofEnd}
    From the definition of the entropy of location-scale variational families, we have
    \begin{align*}
        \norm{ 
          \nabla_{\vlambda} \mathbb{H}\left(q_{\vlambda}\right) - \nabla_{\vlambda'} \mathbb{H}\left(q_{\vlambda'}\right) 
        }_2^2
        &=
        \norm{ 
          \nabla_{\mC} \log \mathrm{det} \,\mC - \nabla_{\mC'} \log \mathrm{det} \, \mC'
        }_2^2,
\shortintertext{since \(\mC, \mC' \in \mathbb{L}^d_{++}\),} 
        &=
        \norm{ 
          \nabla_{\mC} \log \mathrm{det} \left(\mathrm{diag}\left(\mC\right)\right) - \nabla_{\mC'} \log \mathrm{det} \left(\mathrm{diag}\left(\mC'\right)\right)
        }_2^2,
\shortintertext{since the determinant of triangular matrices is the product of the diagonal,} 
        &=
        \sum^d_{i=1} \abs{ \frac{\partial \log C_{ii}}{\partial C_{ii}} - \frac{\partial \log C_{ii}'}{\partial C_{ii}'}}^2 
        \\
        &=
        \sum^d_{i=1} \abs{ \frac{1}{C_{ii}} - \frac{1}{C_{ii}'}}^2 
        \\
        &=
        \sum^d_{i=1} C_{ii}^{-2} \abs{ C_{ii}' - C_{ii} }^2 {\left(C_{ii}'\right)}^{-2},
\shortintertext{and since \(\sigma_{\mathrm{min}}\left(\mC\right) \geq S^{\nicefrac{-1}{2}} \Leftrightarrow C_{ii}^{-2} \leq S\) for all \(i=1, \ldots, d\),} 
        &\leq
        S^2 \sum^d_{i=1} \abs{ C_{ii}' - C_{ii} }^2
        \\
        &=
        S^2 \norm{ \mathrm{diag}\left(\mC\right) - \mathrm{diag}\left(\mC'\right) }^2_2
        \\
        &\leq
        S^2 \norm{ \vlambda - \vlambda' }^2_2.
    \end{align*}
\end{proofEnd}

Given these results, we will hereafter assume projected SGD is run on \(\Lambda_S\) with the projection operator \(\mathrm{proj}_{\Lambda_S}\).

\subsection{Theoretical Analysis of the STL Estimator}\label{section:gradient_variance}
Before presenting our analysis on BBVI gradient estimators, we will discuss a notable aspect of our strategy and the key step in our proof.

Our main assumption on the target posterior is that it is \(L\)-log(-Lipschitz) smooth:
\begin{definition}
    \(\pi\) is said to be \(L\)-log-(Lipschitz) smooth if its log-density \(\log \pi : \mathbb{R}^d \to \mathbb{R}\) is \(L\)-Lipschitz smooth such that
    \[
      \norm{ \nabla \log \pi\left(\vz\right) - \nabla \log \pi\left(\vz'\right) }_2 \leq L \norm{\vz - \vz'}_2,
    \]
    for all \(\vz, \vz' \in \mathbb{R}^d\) and some \(0 < L < \infty\).
\end{definition}
If this holds for \(\pi\), the same bound holds for \(\ell\) as well since they are proportional up to a constant such that \(\nabla \log \ell = \nabla \log \pi\).
This assumption has been used by \citet{domke_provable_2019} to establish similar results for the CFE estimator and is also widely used in the analysis of sampling algorithms based on log-concave analysis. 
(See \citet[\S 2.3]{dwivedi_logconcave_2019} for such example.)
For probability measures, log-smoothness implies that the density of \(\pi\) can be upper bounded by some Gaussian.
Naturally, this essentially corresponds to assuming \(\pi\) has sub-Gaussian tails.

\vspace{-1ex}
\paragraph{Adaptive Bounds with the Peter-Paul Inequality}
Unlike the QV bounds obtained by \citet{domke_provable_2023}, our bounds involve a free parameter \(\delta \geq 0\).
We call these bounds \textit{adaptive} QV bounds.
%
\begin{assumption}[\textbf{Adaptive QV}]\label{assumption:adaptiveqvc}
  The gradient estimator \(\rvvg\) satisfies the bound
{%
\setlength{\belowdisplayskip}{1.ex} \setlength{\belowdisplayshortskip}{1.ex}
\setlength{\abovedisplayskip}{1.ex} \setlength{\abovedisplayshortskip}{1.ex}
  \[
    \mathbb{E}\norm{\rvvg\left(\vlambda\right)}_2^2 \leq (1 + C \delta) \, \widetilde{\alpha} \, \norm{\vlambda - \vlambda^*}_2^2 + (1 + C^{-1} \delta^{-1}) \,\widetilde{\beta},
  \]
  }%
  for any \(\delta > 0\), any \(\vlambda \in \Lambda_S\), and some \(0 < \widetilde{\alpha}, \widetilde{\beta} < \infty\), where \(\vlambda^*\) is a stationary point.
\end{assumption}
%
This is a consequence of the use of the ``Peter-Paul'' inequality such that
{%
\setlength{\belowdisplayskip}{1.ex} \setlength{\belowdisplayshortskip}{1.ex}
\setlength{\abovedisplayskip}{1.ex} \setlength{\abovedisplayshortskip}{1.ex}
\begin{equation}
  {\left( a + b \right)}^2 \leq \left(1 + \delta\right) \,a^2 + (1 + \delta^{-1}) \,b^2,
  \label{eq:peterpaul}
\end{equation}
}%
and can be seen as a generalization of the usual inequality \({\left(a + b\right)}^2 \leq 2 a^2 + 2 b^2\).
Adjusting \(\delta\) can occasionally tighten the analysis.
In fact, \(\delta\) can be optimized to become \textit{adaptive} to the downstream analysis.
Indeed, in our complexity analysis, \(\delta\) automatically trades-off the influence of \(\widetilde{\alpha}\) and \(\widetilde{\beta}\) according to the accuracy budget \(\epsilon\).

\vspace{-1ex}
\paragraph{Key Lemma}
The key first step in all of our analysis is the following decomposition:

\begin{theoremEnd}[category=gradvarlemmas,all end]{lemma}\label{thm:peterpaul}
  For any \(a, b, c, \in \mathbb{R}\), 
  \[ {(a+b+c)}^2 \leq (2 + \delta ) a^2 + (2 + \delta ) b^2 + (1 + 2 \delta^{-1})  c^2, \]
  for any \(\delta > 0\).
\end{theoremEnd}
\begin{proofEnd}
The Peter-Paul generalization of Young's inequality states that, for \(d, e \geq 0\), we have
\[
   d e \leq \frac{\delta }{2} d^2 + \frac{\delta^{-1}}{2} e^2.
\]
Applying this,
\begin{align*}
  {(a+b+c)}^2
  &\;=
  a^2 + b^2 + c^2 + 2 a b + 2 a c + 2 b c
  \\
  &\;\leq
    a^2 + b^2 + c^2 + 2 \abs{a} \abs{b} + 2 \abs{a} \abs{c} + 2 \abs{b} \abs{c}
  \\
  &\;\leq
  a^2 + b^2 + c^2 
  + 2 \left( \frac{1}{2} a^2 + \frac{1}{2} b^2 \right)
  + 2 \left( \frac{\delta}{2} a^2 + \frac{\delta^{-1}}{2} c^2 \right)
  + 2 \left( \frac{\delta}{2} b^2 + \frac{\delta^{-1}}{2} c^2 \right) 
  \\
  &\;=
  a^2 + b^2 + c^2 + \left( a^2 +  b^2 \right)
  + \left( \delta a^2 + \delta^{-1} c^2 \right)
  + \left( \delta b^2 + \delta^{-1} c^2 \right) 
  \\
  &\;=
  \left(2 + \delta\right) a^2 + \left(2 + \delta\right) b^2 + \left(1 + 2 \delta^{-1}\right) c^2.
\end{align*}
\end{proofEnd}


\begin{theoremEnd}[category=stlupperboundlemma]{lemma}\label{thm:stl_decomposition}
{\setlength{\belowdisplayskip}{1.5ex} \setlength{\belowdisplayshortskip}{1.5ex}
\setlength{\abovedisplayskip}{1.ex} \setlength{\abovedisplayshortskip}{1.ex}
%
  Assume \cref{assumption:variation_family}.
  The expected-squared norm of STL is bounded as
  {\small
  \begin{align*}
    \mathbb{E} \norm{\rvvg_{\mathrm{STL}}\left(\vlambda\right)}_2^2
    \leq
    \left( 2 + \delta \right) V_{1}
    +
    \left( 2 + \delta \right) V_{2}
    +
    \left( 1 + 2 \delta^{-1} \right) V_{3},
  \end{align*}
  }%%
  where the terms are
  {\small
  \begin{align*}
    V_{1}
    &=
    \mathbb{E} \,
    J_{\mathcal{T}}\left(\rvvu\right) 
    \lVert
    \nabla \log \ell \left(\mathcal{T}_{\vlambda}\left(\rvvu\right)\right) 
    -
    \nabla \log \ell \left(\mathcal{T}_{\vlambda^*}\left(\rvvu\right)\right) 
    {\rVert}_2^2
    \\
    V_{2}
    &=
    \mathbb{E} \,
    J_{\mathcal{T}}\left(\rvvu\right)
    \lVert
    \nabla \log q_{\vlambda^*}\left(\mathcal{T}_{\vlambda^*}\left(\rvvu\right)\right)
    - 
    \nabla \log q_{\vlambda}\left(\mathcal{T}_{\vlambda}\left(\rvvu\right)\right)
    {\rVert}_2^2
    \\
    V_{3}
    &=
    \mathbb{E} \,
    J_{\mathcal{T}}\left(\rvvu\right)
    \lVert
    \nabla \log \ell\left(\mathcal{T}_{\vlambda^*}\left(\rvvu\right)\right)
    - 
    \nabla \log q_{\vlambda^*}\left(\mathcal{T}_{\vlambda^*}\left(\rvvu\right)\right)
    {\rVert}_2^2,
  \end{align*}
  }%%
  for any \(\delta > 0\) and \(\vlambda \in \mathbb{R}^p\).
  \(J_{\mathcal{T}} : \mathbb{R}^d \to \mathbb{R}\) is a function depending on the variational family as
  \begin{alignat*}{2}
    J_{\mathcal{T}}\left(\rvvu\right) &= 1 + {\textstyle\sum^{d}_{i=1} \rvu_i^2} \quad                        &&\quad \text{for full-rank and} \\
    J_{\mathcal{T}}\left(\rvvu\right) &= 1 + {\textstyle\sqrt{\sum^{d}_{i=1} \rvu_i^4}} &&\quad \text{for mean-field.}
  \end{alignat*}
 }%
\end{theoremEnd}
\begin{proofEnd}
From the definition of the STL estimator~\cref{def:stl},
\begin{align*}
    \mathbb{E} \norm{\rvvg_{\mathrm{STL}}\left(\vlambda\right)}_2^2
    &=
    \mathbb{E} \norm{
    \nabla_{\vlambda} \log \ell \left(\mathcal{T}_{\vlambda}\left(\rvvu\right)\right) - \nabla_{\vlambda} \log q_{\vnu}\left(\mathcal{T}_{\vlambda}\left(\rvvu\right)\right)
    }_2^2
    \;\Big\lvert_{\vnu = \vlambda},
\shortintertext{by \cref{thm:jacobian_reparam_inner},}
    &=
    \mathbb{E}
    J_{\mathcal{T}}\left(\rvvu\right)
    \norm{
    \nabla \log \ell \left(\mathcal{T}_{\vlambda}\left(\rvvu\right)\right) 
    - \nabla \log q_{\vnu}\left(\mathcal{T}_{\vlambda}\left(\rvvu\right)\right)
    }_2^2
    \;\Big\lvert_{\vnu = \vlambda}
\shortintertext{adding the terms \(\nabla \log \ell \left(\mathcal{T}_{\vlambda^*}\left(\rvvu\right)\right)\) and \(\nabla \log q_{\vlambda^*}\left(\mathcal{T}_{\vlambda^*}\left(\rvvu\right)\right)\) that cancel,}
    &=
    \mathbb{E}
    J_{\mathcal{T}}\left(\rvvu\right)
    \lVert
    \nabla \log \ell \left(\mathcal{T}_{\vlambda}\left(\rvvu\right)\right) 
    -
    \nabla \log \ell \left(\mathcal{T}_{\vlambda^*}\left(\rvvu\right)\right) 
    \\
    &\quad\qquad\qquad+
    \nabla \log \ell\left(\mathcal{T}_{\vlambda^*}\left(\rvvu\right)\right)
    - 
    \nabla \log q_{\vlambda^*}\left(\mathcal{T}_{\vlambda^*}\left(\rvvu\right)\right)
    \\
    &\quad\qquad\qquad+
    \nabla \log q_{\vlambda^*}\left(\mathcal{T}_{\vlambda^*}\left(\rvvu\right)\right)
    - 
    \nabla \log q_{\vlambda}\left(\mathcal{T}_{\vlambda}\left(\rvvu\right)\right)
    {\rVert}_2^2,
\shortintertext{applying \cref{thm:peterpaul},}
    &\leq
    \mathbb{E}
    J_{\mathcal{T}}\left(\rvvu\right)
    \Big(
    \left( 2 + \delta \right)
    \lVert
    \nabla \log \ell \left(\mathcal{T}_{\vlambda}\left(\rvvu\right)\right) 
    -
    \nabla \log \ell \left(\mathcal{T}_{\vlambda^*}\left(\rvvu\right)\right) 
    {\rVert}_2^2
    \\
    &\quad\qquad\qquad+
    \left( 2 + \delta \right)
    \lVert
    \nabla \log q_{\vlambda^*}\left(\mathcal{T}_{\vlambda^*}\left(\rvvu\right)\right)
    - 
    \nabla \log q_{\vlambda}\left(\mathcal{T}_{\vlambda}\left(\rvvu\right)\right)
    {\rVert}_2^2
    \\
    &\quad\qquad\qquad+
    \left(1 + 2 \delta^{-1}\right)
    \lVert
    \nabla \log \ell\left(\mathcal{T}_{\vlambda^*}\left(\rvvu\right)\right)
    - 
    \nabla \log q_{\vlambda^*}\left(\mathcal{T}_{\vlambda^*}\left(\rvvu\right)\right)
    {\rVert}_2^2
    \Big),
\shortintertext{and distributing \(J_{\mathcal{T}}\) and the expectation,}
    &=
    \left( 2 + \delta \right)
    \underbrace{
    \mathbb{E}
    J_{\mathcal{T}}\left(\rvvu\right)
    \lVert
    \nabla \log \ell \left(\mathcal{T}_{\vlambda}\left(\rvvu\right)\right) 
    -
    \nabla \log \ell \left(\mathcal{T}_{\vlambda^*}\left(\rvvu\right)\right) 
    {\rVert}_2^2
    }_{V_{1}}
    \\
    &\quad+
    \left( 2 + \delta \right)  
     \underbrace{
    J_{\mathcal{T}}\left(\rvvu\right)
    \mathbb{E}
    \lVert
    \nabla \log q_{\vlambda^*}\left(\mathcal{T}_{\vlambda^*}\left(\rvvu\right)\right)
    - 
    \nabla \log q_{\vlambda}\left(\mathcal{T}_{\vlambda}\left(\rvvu\right)\right)
    {\rVert}_2^2
    }_{V_{2}}
    \\
    &\quad+
    \left(1 + 2 \delta^{-1}\right)  
    \underbrace{
    \mathbb{E}
    J_{\mathcal{T}}\left(\rvvu\right)
    \lVert
    \nabla \log \ell\left(\mathcal{T}_{\vlambda^*}\left(\rvvu\right)\right)
    - 
    \nabla \log q_{\vlambda^*}\left(\mathcal{T}_{\vlambda^*}\left(\rvvu\right)\right)
    {\rVert}_2^2
    }_{V_{3}}.
\end{align*}
\end{proofEnd}

Here, \(J_{\mathcal{T}}\) is a term that stems from the Jacobian of \(\mathcal{T}\).
Thus, \(J_{\mathcal{T}}\) contains the properties unique to the chosen variational family.
\(V_{1}\) and \(V_{2}\) measure how far the current variational approximation \(q_{\vlambda}\) is from a stationary point \({\vlambda^*}\).
Thus, both terms will eventually reach 0 as BBVI converges, regardless of family specification.
The key is \(V_{3}\), which captures the amount of mismatch between the score of the true posterior \(\pi\) and variational posterior \(q_{\vlambda^*}\).
Establishing the ``interpolation condition'' amounts to analyzing when \(V_{3}\) becomes 0.

\subsubsection{Upper Bounds}
We now present our upper bound on the expected-squared norm of the STL gradient estimator.
%
\input{thm_stl_upperbound}

\newpage
\begin{remark}[\textbf{Mean-Field Variational Family}]
  We prove an equivalent result for the mean-field variational family, \cref{thm:stl_upperbound_mf} in \cref{section:stl_meanfield}, which has an \(\mathcal{O}\left(\sqrt{d}\right)\) dimensional dependence.
\end{remark}
\vspace{0.5ex}
\begin{remark}[\textbf{Interpolation Condition}]
  \cref{thm:stl_upperbound} encompasses both settings where the variational family is well-specified and misspecified.
  That is, when the variational family is well specified, \textit{i.e.}, \( \mathrm{D}_{\mathrm{F}^4}\left(q_{\vlambda^*}, \pi\right) = 0 \), we obtain interpolation such that \(\beta_{\mathrm{STL}} = 0\).
\end{remark}
\vspace{0.5ex}
\begin{remark}[\textbf{Adaptivity of Bound}]
  When the variational family is well specified such that \( \mathrm{D}_{\mathrm{F}^4}\left(q_{\vlambda^*}, \pi\right) = 0 \), we can adaptively tighten the bound by setting \(\delta = 0\), where \(\alpha_{\mathrm{STL}}\) is reduced by a constant factor.
\end{remark}

\subsubsection{Lower Bounds}
We also obtain lower bounds on the expected-squared norm of the STL estimator to analyze its best-case behavior and the tightness of the bound.

\vspace{-1ex}
\paragraph{Necessary Conditions for Interpolation}
First, we obtain lower bounds that generally hold for all \(\vlambda \in \Lambda_L\) and any \(\pi\).
Our analysis relates the gradient variance with the Fisher-Hyv\"arinen divergence.
This can be related back to the KL divergence through an assumption on the posterior \(\pi\) known as the log-Sobolev inequality.
The general form of the log-Sobolev inequality was originally proposed by \citet{gross_logarithmic_1975} to study diffusion processes.
In this work, we use the form used by \citet{otto_generalization_2000}:
%
\begin{assumption}[\textbf{Log-Sobolev Inequality; LSI}]
\(\pi\) is said to satisfy the log-Sobolev inequality if, for any variational family \(\mathcal{Q}\) and all \(q_{\vlambda} \in \mathcal{Q}\), the following inequality holds:
{%
\setlength{\belowdisplayskip}{1.ex} \setlength{\belowdisplayshortskip}{1.ex}
\setlength{\abovedisplayskip}{0ex} \setlength{\abovedisplayshortskip}{0ex}
\[
  \DKL{q}{\pi}
  \leq
  \frac{C_{\mathrm{LSI}}}{2} \, \DHF{q}{\pi}.
\]
}%
\end{assumption}
\vspace{-1ex}
%
Strongly log-concave distributions are known to satisfy the LSI, where the strong log-concavity constant becomes the (inverse) LSI constant (see also~\citealp[Theorem 9.9]{villani_topics_2016}):

\vspace{1ex}
\begin{remark}[\citealp{bakry_diffusions_1985}]
    Let \(\pi\) be \(\mu\)-strongly log-concave.
    Then, LSI holds with \(C_{\mathrm{LSI}}^{-1} = \mu\).
\end{remark}

We now present a lower bound which holds for all \(\vlambda \in \Lambda_S\) and any log-differentiable \(\pi\):
%

\begin{theoremEnd}[category=stllowerbound]{theorem}\label{thm:stl_lowerbound}
  Assume \cref{assumption:variation_family}.
  The expected-squared norm of the STL estimator is lower bounded as
  \[
    \mathbb{E} \norm{\rvvg_{\mathrm{STL}}\left(\vlambda\right)}_2^2 
    \geq
    \DHF{q_{\vlambda}}{\pi}
    \geq
    \frac{2}{C_{\mathrm{LSI}}} \DKL{q_{\vlambda}}{\pi},
  \]
  for all \(\vlambda \in \Lambda_{S}\) and any \(0 < S < \infty\), where the last inequality holds if \(\pi\) is LSI.
\end{theoremEnd}
\begin{proofEnd}
\begin{align*}
  \mathbb{E} \norm{\rvvg_{\mathrm{STL}}\left(\vlambda\right)}_2^2
  &=
  \mathbb{E} \norm{  
  \nabla_{\vlambda} \log \pi \left(\mathcal{T}_{\vlambda}\left(\rvvu\right)\right) - \nabla_{\vlambda} \log q_{\vnu}\left(\mathcal{T}_{\vlambda}\left(\rvvu\right)\right)
  }_2^2
  \;\Bigg\lvert_{\vnu = \vlambda},
\shortintertext{by \cref{thm:jacobian_reparam_inner},}
  &=
  \mathbb{E} 
  J_{\mathcal{T}}\left(\rvvu\right)
  \norm{
  \nabla \log \pi \left(\mathcal{T}_{\vlambda}\left(\rvvu\right)\right) 
  - \nabla \log q_{\vnu}\left(\mathcal{T}_{\vlambda}\left(\rvvu\right)\right)
  }_2^2
  \;\Big\lvert_{\vnu = \vlambda}
  \\
  &=
  \mathbb{E} 
  J_{\mathcal{T}}\left(\rvvu\right)
  \norm{
  \nabla \log \pi \left(\mathcal{T}_{\vlambda}\left(\rvvu\right)\right) 
  - \nabla \log q_{\vlambda}\left(\mathcal{T}_{\vlambda}\left(\rvvu\right)\right)
  }_2^2,
\shortintertext{since \(J_{\mathcal{T}}\left(\rvvu\right) \geq 1\) for both the full-rank and mean-field parameterizations,}
  &\geq
  \mathbb{E} 
  \norm{
  \nabla \log \pi \left(\mathcal{T}_{\vlambda}\left(\rvvu\right)\right) 
  - \nabla \log q_{\vlambda}\left(\mathcal{T}_{\vlambda}\left(\rvvu\right)\right)
  }_2^2,
\shortintertext{after Change-of-Variable,}
  &=
  \mathbb{E}_{\rvvz \sim q_{\vlambda}} 
  \norm{
  \nabla \log \pi \left(\rvvz\right) 
  - \nabla \log q_{\vlambda}\left(\rvvz\right)
  }_2^2
\shortintertext{by definition,}
  &=
  \DHF{q_{\vlambda}}{\pi},
\shortintertext{and when the log-Sobolev inequality applies,}
  &\geq
  \frac{2}{C_{\mathrm{LSI}}} \DKL{q_{\vlambda}}{\pi}.
\end{align*}
\end{proofEnd}

%%% Local Variables:
%%% TeX-master: "main"
%%% End:


\begin{corollary}[\textbf{Necessary Conditions for Interpolation}]
For the STL estimator, the interpolation condition does not hold if
\begin{enumerate}[label=\textbf{(\roman*)}]
  \vspace{-1.5ex}
  \setlength\itemsep{0.ex}
    \item \(\DHF{q_{\vlambda^*_{\mathrm{F}}}}{\pi} > 0\), or,
    \item when \(\pi\) is LSI, \(\DKL{q_{\vlambda^*_{\mathrm{KL}}}}{\pi} > 0\),
  \vspace{-1.5ex}
\end{enumerate}
%
  \begin{center}
   {\begingroup
    \setlength\tabcolsep{10pt} 
  \begin{tabular}{ll}
    \text{where }
    &
    \(\vlambda_{\mathrm{F}}^* \in \argmin_{\vlambda \in \Lambda_S} \DHF{q_{\vlambda}}{\pi} \),\; 
    \text{and} \\
    & \(\vlambda_{\mathrm{KL}}^* \in \argmin_{\vlambda \in \Lambda_S} \DKL{q_{\vlambda}}{\pi} \),
  \end{tabular}
  \endgroup}
  \end{center}
  \vspace{-1.5ex}
  for any \(0 < S < \infty\).
\end{corollary}
\vspace{-1ex}

\vspace{-1.ex}
\paragraph{Tightness Analysis}
The bound in \cref{thm:stl_lowerbound} is unfortunately not tight regarding the constants.
It, however, holds for all \(\vlambda\) and \(\pi\).
Instead, we establish an alternative lower bound that holds for some \(\vlambda\) and \(\pi\) but is tight regarding the dependence on \(d\) and \(L\).
\vspace{1ex}

\begin{theoremEnd}[all end, category=gradvarlemmas]{lemma}\label{lemma:unorm_times_reparam}
  Let \(\mathcal{T}_{\vlambda}: \mathbb{R}^p \times \mathbb{R}^d \rightarrow \mathbb{R}^d\) be the location-scale reparameterization function (\cref{def:reparam}) and \(\rvvu \sim \varphi\) satisfy \cref{assumption:symmetric_standard}.
  Then,
  \[
  \mathbb{E} 
  \left(1 + \textstyle{\sum^{d}_{i=1} \rvu_i^2} \right)
  \left(
  \mathcal{T}_{\vlambda}\left(\rvvu\right) 
  +
  \vz
  \right)
  =
  \left(d+1\right)
  \left( \vm + \vz \right)
  \]
  for any \(\vz \in \mathbb{R}^d\).
\end{theoremEnd}
\vspace{-1ex}
\begin{proofEnd}
  \begin{align*}
  \mathbb{E} 
  \left(1 + \textstyle{\sum^{d}_{i=1} \rvu_i^2} \right)
  \left(
  \mathcal{T}_{\vlambda}\left(\rvvu\right) + \vz
  \right)
  &=
  \mathbb{E} 
  \left(1 + \norm{\rvvu}_2^2 \right)
  \left( \mC \rvvu + \vm + \vz \right)
  \\
  &=
  \mC \mathbb{E} \left(1 + \norm{\rvvu}_2^2 \right) \rvvu +  \mathbb{E}\left(1 + \norm{\rvvu}_2^2 \right) \left( \vm + \vz \right)
  \\
  &=
  \left(d+1\right) \left( \vm + \vz \right),
  \end{align*}
  where the last equality follows from \cref{thm:u_identities,assumption:symmetric_standard}.
\end{proofEnd}

\begin{theoremEnd}[all end, category=gradvarlemmas]{lemma}\label{thm:lowerbound_matrix_innerproduct_lemma}
  Let \(\mA = \mathrm{diag}\left(A_1, \ldots, A_d\right) \in \mathbb{R}^{d \times d}\) be some diagonal matrix, define
{%
\setlength{\belowdisplayskip}{1.ex} \setlength{\belowdisplayshortskip}{1.ex}
\setlength{\abovedisplayskip}{1.ex} \setlength{\abovedisplayshortskip}{1.ex}
  \[
  \mB = \begin{bmatrix}
    L^{-1/2} &   &        &   \\
             & L^{1/2} &        &   \\
             &   & \ddots &   \\
             &   &        & L^{1/2} \\
  \end{bmatrix},
  \qquad
  \mC = L^{-1/2} \, \boldupright{I},
  \]
  }%
  some \(\vu \in \mathbb{R}^d\), \(\vm \in \mathbb{R}^d\), and \(\vz \in \mathbb{R}^d \) such that \(m_1 = z_1\).
  For \(\vlambda = (\vm, \mC)\), the expression
{%
\setlength{\belowdisplayskip}{1.ex} \setlength{\belowdisplayshortskip}{1.ex}
\setlength{\abovedisplayskip}{1.ex} \setlength{\abovedisplayshortskip}{1.ex}
  \[
    \norm{\mB^{-1} \mC^{-1} \left( \mA \rvvu + \vm - \vz \right)}_2^2.
  \]
}%
  can be bounded for the following instances of \(\mA\):
  \begin{enumerate}[label=\textbf{(\roman*)}]
    \item If \(\mA = \mC\),  
    {%
    \setlength{\belowdisplayskip}{1.ex} \setlength{\belowdisplayshortskip}{1.ex}
    \setlength{\abovedisplayskip}{1.ex} \setlength{\abovedisplayshortskip}{1.ex}
      \begin{align*}
      &{\lVert
        \mB^{-1} \mC^{-1} \left(\mC \vu + \vm - \vz\right)
      \rVert}_2^2
      =
      \norm{\mC \vu + \vm - \vz}_2^2
      + {\left(L - L^{-1} \right)} \, u_1^2,
      \end{align*}
    }
    \item while if \(\mA = \boldupright{O}\),  \\
    {%
    \setlength{\belowdisplayskip}{1.ex} \setlength{\belowdisplayshortskip}{1.ex}
    \setlength{\abovedisplayskip}{-1.ex} \setlength{\abovedisplayshortskip}{-1.ex}
      \[
      {\lVert
        \mB^{-1} \mC^{-1} \left(\vm - \vz\right)
      \rVert}_2^2
      =
      \norm{\vm - \vz}_2^2.
      \]
    }
  \end{enumerate}
\end{theoremEnd}
\begin{proofEnd}
  First notice that 
  \begin{align*}
    \mB^{-1} \mC^{-1}
    =
    \begin{bmatrix}
      L &       &        &   \\
        & 1     &        &   \\
        &       & \ddots &   \\
        &       &        & 1 \\
    \end{bmatrix}.
  \end{align*}

  Denoting the 1st coordinate of \(\mA \vu + \vm \) as \({[\mA \vu + \vm]}_1 = A_1 \rvu_1 + m_1\), we have
  \begin{align}
    &\norm{
      \mB^{-1} \mC^{-1} \left(\mA \vu + \vm - \vz\right)
    }_2^2
    \\
    &\;=
    \norm{
    \begin{bmatrix}
      L &       &        &   \\
        & 1     &        &   \\
        &       & \ddots &   \\
        &       &        & 1 \\
    \end{bmatrix} 
    \left(\mA \vu + \vm - \vz\right)
    }_2^2
    \nonumber
    \\
    &\;=
    \norm{\mA \vu + \vm - \vz}_2^2
    + {\left(L^{2} - 1 \right)} {\left( {[\mA \vu + \vm]}_1 - z_1 \right)}^2
    \nonumber
    \\
    &\;=
    \norm{\mA \rvvu + \vm - \vz}_2^2
    + {\left(L^{2} - 1 \right)} {\left(A_1 u_1 + m_1 - z_1 \right)}^2,
    \nonumber
\shortintertext{and using the fact that \(m_1 = z_1\)}
    &\;=
    \norm{\mA \vu + \vm - \vz}_2^2
    + {\left(L^{2} - 1 \right)} \, A_1^2  \, u_1^2.
    \label{eq:unimprovability_key_lemma_eq1}
  \end{align}

  \paragraph{Proof of (i)}
  If \(\mA = \mC = L^{-1/2} \boldupright{I}\), \cref{eq:unimprovability_key_lemma_eq1} yields,
  \begin{align*}
    \norm{
      \mB^{-1} \mC^{-1} \left(\mA \vu + \vm - \vz\right)
    }_2^2
    &=
    \norm{\mC \vu + \vm - \vz}_2^2
    + {\left(L^{2} - 1 \right)} \, L^{-1} u_1^2
    \\
    &=
    \norm{\mC \vu + \vm - \vz}_2^2
    + {\left(L - L^{-1} \right)} \, u_1^2
  \end{align*}

  \paragraph{Proof of (ii)}
  If \(\mA = \boldupright{O}\), \cref{eq:unimprovability_key_lemma_eq1} yields,
  \begin{align*}
    \norm{
      \mB^{-1} \mC^{-1} \left(\mA \vu + \vm - \vz\right)
    }_2^2
    &=
    \norm{\vm - \vz}_2^2.
  \end{align*}

\end{proofEnd}

\begin{theoremEnd}[category=stllowerboundunimprovability]{theorem}\label{thm:stl_lowerbound_unimprovability}
 Assume \cref{assumption:variation_family}.
  There exists a strongly-convex, \(L\)-log-smooth posterior and some variational parameter \(\widetilde{\vlambda} \in \Lambda_{L}\) for all \(L \geq 1\) such that
{%
\setlength{\belowdisplayskip}{1.ex} \setlength{\belowdisplayshortskip}{1.ex}
\setlength{\abovedisplayskip}{1.ex} \setlength{\abovedisplayshortskip}{1.ex}
  {
  \begin{align*}
    \mathbb{E} {\lVert \rvvg_{\mathrm{STL}}\left(\widetilde{\vlambda}\right) \rVert}_2^2
    &\geq
    \left(
    L^2
    \left( d + k_{\varphi} \right) 
    -2
    \left(d + 1 \right) 
    \right)
    {\lVert \widetilde{\mC} \rVert}_{\mathrm{F}}^2
    \\
    &\qquad
    - 2
    \left(k_{\varphi} - 1\right) \norm{ \widetilde{\vm} - \bar{\vz} }_2^2,
  \end{align*}
  }%
  }%
  where \(\widetilde{\vlambda} = (\widetilde{\vm}, \widetilde{\mC})\) and \(\bar{\vz}\) is a stationary point of the said log posterior.
\end{theoremEnd}
\vspace{-1ex}
\begin{proofEnd}\label{proof:stl_lowerbound_unimprovability}
  The worst case is achieved by the following:
  \begin{enumerate}[label=\textbf{(\roman*)}]
  \item \textbf{\(\pi\) is ill-conditioned such that the smoothness constant is large.} 
    This results in the domain \(\Lambda_{L}\) to include ill-conditioned \(\mC\)s, which has the largest impact on the gradient variance. Furthermore, 
  \item \textbf{\(\pi\) and \(q_{\vlambda}\) need to have the least overlap in probability volume.} 
    This means the variance reduction effect will be minimal.
  \end{enumerate}
  For Gaussians, this is equivalent to minimizing 
  \(
    {\lVert \mS^{-1} \mSigma^{-1} \rVert}_{\mathrm{F}}^2
  \)
  while maximizing \({\lVert \mSigma^{-1} \rVert}_{\mathrm{F}}^2\) and \({\lVert \mS^{-1} \rVert}_{\mathrm{F}}^2\).

  We therefore choose
  \[
  \pi = \mathcal{N}\left(\bar{\vz}, \mSigma\right)
  \qquad
  q_{\vlambda} = \mathcal{N}\left(\widetilde{\vm}, \widetilde{\mS}\right),
  \]
  where 
  \[
  \mSigma = \begin{bmatrix}
    L^{-1}  &   &        &   \\
           & L &        &   \\
           &   & \ddots &   \\
           &   &        & L \\
  \end{bmatrix},
  \qquad
  \widetilde{\mS} = L^{-1} \boldupright{I},
  \;\quad\text{and}\quad
  \widetilde{\vm} = \begin{bmatrix}
    \bar{z}_1 \\ m_2 \\ \vdots \\ m_d
  \end{bmatrix},
  \]
  where \(\bar{z}_1\) is the 1st element of \(\bar{\vz}\) such that \(\widetilde{m}_1 = \bar{z}_1\).
  The choice of \(\widetilde{m}_1 = \bar{z}_1\) is purely for clarifying the derivation.
  Notice that \(\mSigma\) has \(d-1\) entries set as \(L\), only one entry set as \(L^{-1}\), and \(\widetilde{\mS} = \widetilde{\mC}\widetilde{\mC}\).
  Here, \(\pi\) is \(L^{-1}\)-strongly log-concave, \(L\)-log smooth, and \(\widetilde{\vlambda} = \left(\widetilde{\vm}, \widetilde{\mC}\right) \in \Lambda_{L}\).

  \paragraph{General Gaussian \(\pi\) Lower Bound}
  As usual, we start from the definition of the STL estimator as
  \begin{align*}
    \mathbb{E} \norm{\rvvg_{\mathrm{STL}}\left(\vlambda\right)}_2^2
    &=
    \mathbb{E} \norm{
    \nabla_{\vlambda} \log \pi \left(\mathcal{T}_{\vlambda}\left(\rvvu\right)\right) - \nabla_{\vlambda} \log q_{\vnu}\left(\mathcal{T}_{\vlambda}\left(\rvvu\right)\right)
  }_2^2
    \;\Big\lvert_{\vnu = \vlambda}
  \shortintertext{by \cref{thm:jacobian_reparam_inner},}
    &=
    \mathbb{E} 
    J_{\mathcal{T}}\left(\rvvu\right)
    \norm{
    \nabla \log \pi \left(\mathcal{T}_{\vlambda}\left(\rvvu\right)\right) 
    - \nabla \log q_{\vnu}\left(\mathcal{T}_{\vlambda}\left(\rvvu\right)\right)
    }_2^2
    \;\Big\lvert_{\vnu = \vlambda},
\shortintertext{since both \(\pi\) and \(q_{\vlambda}\) are Gaussians,}
    &=
    \mathbb{E} 
    \left(1 + \textstyle{\sum^{d}_{i=1} \rvu_i^2} \right)
    \norm{
    \nabla \log \pi \left(\mathcal{T}_{\vlambda}\left(\rvvu\right)\right) 
    - \nabla \log q_{\vlambda}\left(\mathcal{T}_{\vlambda}\left(\rvvu\right)\right)
    }_2^2
    \\
    &=
    \mathbb{E} 
    \left(1 + \textstyle{\sum^{d}_{i=1} \rvu_i^2} \right)
    \norm{
      \mSigma^{-1} \left(\mathcal{T}_{\vlambda}\left(\rvvu\right) - \bar{\vz}\right)
      -
      \mS^{-1} \left(\mathcal{T}_{\vlambda}\left(\rvvu\right) - \vm\right)
    }_2^2
    \\
    &=
    \mathbb{E} 
    \left(1 + \textstyle{\sum^{d}_{i=1} \rvu_i^2} \right)
    \norm{
      \mSigma^{-1} \left(\mathcal{T}_{\vlambda}\left(\rvvu\right) - \bar{\vz}\right)
      -
      \mS^{-1} \left(\mathcal{T}_{\vlambda}\left(\rvvu\right) - \bar{\vz}\right)
      +
      \mS^{-1} \left(\vm - \bar{\vz}\right)
    }_2^2
    \\
    &=
    \mathbb{E} 
    \left(1 + \textstyle{\sum^{d}_{i=1} \rvu_i^2} \right)
    \Big(
    {\lVert
      \mSigma^{-1} \left(\mathcal{T}_{\vlambda}\left(\rvvu\right) - \bar{\vz}\right)
    \rVert}_2^2
    +
    {\lVert
      \mS^{-1} \left(\mathcal{T}_{\vlambda}\left(\rvvu\right) - \bar{\vz}\right)
    \rVert}_2^2
    +
    {\lVert
      \mS^{-1} \left(\vm - \bar{\vz}\right)
    \rVert}_2^2
    \\
    &\quad\qquad\qquad\qquad\qquad
    -2
    \inner{
      \mSigma^{-1} \left(\mathcal{T}_{\vlambda}\left(\rvvu\right) - \bar{\vz}\right)
    }{
      \mS^{-1} \left(\mathcal{T}_{\vlambda}\left(\rvvu\right) - \bar{\vz}\right)
    }
    \\
    &\quad\qquad\qquad\qquad\qquad
    +2
    \inner{
      \mSigma^{-1} \left(\mathcal{T}_{\vlambda}\left(\rvvu\right) - \bar{\vz}\right)
    }{
      \mS^{-1} \left(\vm - \bar{\vz}\right)
    }
    \\
    &\quad\qquad\qquad\qquad\qquad
    -2
    \inner{
      \mS^{-1} \left(\mathcal{T}_{\vlambda}\left(\rvvu\right) - \bar{\vz}\right)
    }{
      \mS^{-1} \left(\vm - \bar{\vz}\right)
    }
    \Big),
\shortintertext{distributing the expectation and \(1 + \textstyle{\sum^{d}_{i=1} \rvu_i^2}\),}
    &=
    \mathbb{E} 
    \left(1 + \textstyle{\sum^{d}_{i=1} \rvu_i^2} \right)
    \Big(
    {\lVert
      \mSigma^{-1} \left(\mathcal{T}_{\vlambda}\left(\rvvu\right) - \bar{\vz}\right)
    \rVert}_2^2
    +
    {\lVert
      \mS^{-1} \left(\mathcal{T}_{\vlambda}\left(\rvvu\right) - \bar{\vz}\right)
    \rVert}_2^2
    \Big)
    \\
    &\quad+
    \mathbb{E} 
    \left(1 + \textstyle{\sum^{d}_{i=1} \rvu_i^2} \right)
    {\lVert
      \mS^{-1} \left(\vm - \bar{\vz}\right)
    \rVert}_2^2
    \\
    &\quad
    -2 \,
    \mathbb{E} 
    \left(1 + \textstyle{\sum^{d}_{i=1} \rvu_i^2} \right)
    \inner{
      \mSigma^{-1} \left(
      \mathcal{T}_{\vlambda}\left(\rvvu\right) - \bar{\vz}\right)
    }{
      \mS^{-1} \left(\mathcal{T}_{\vlambda}\left(\rvvu\right) - \bar{\vz}\right)
    }
    \\
    &\quad
    +2 \,
    \inner{
      \mSigma^{-1} 
      \mathbb{E} 
      \left(1 + \textstyle{\sum^{d}_{i=1} \rvu_i^2}  \right)
      \left(
      \mathcal{T}_{\vlambda}\left(\rvvu\right) - \bar{\vz}\right)
    }{
      \mS^{-1} \left(\vm - \bar{\vz}\right)
    }
    \\
    &\quad
    -2 \,
    \inner{
      \mS^{-1} 
      \mathbb{E} 
      \left(1 + \textstyle{\sum^{d}_{i=1} \rvu_i^2} \right)
      \left(
      \mathcal{T}_{\vlambda}\left(\rvvu\right) - \bar{\vz}\right)
    }{
      \mS^{-1} \left(\vm - \bar{\vz}\right)
    },
\shortintertext{applying \cref{lemma:unorm_times_reparam,thm:u_identities} to the second term and the last two inner product terms,}
    &=
    \mathbb{E} 
    \left(1 + \textstyle{\sum^{d}_{i=1} \rvu_i^2} \right)
    \Big(
    {\lVert
      \mSigma^{-1} \left(\mathcal{T}_{\vlambda}\left(\rvvu\right) - \bar{\vz}\right)
    \rVert}_2^2
    +
    {\lVert
      \mS^{-1} \left(\mathcal{T}_{\vlambda}\left(\rvvu\right) - \bar{\vz}\right)
    \rVert}_2^2
    \Big)
    \\
    &\quad+
    \left(d + 1\right)
    {\lVert
      \mS^{-1} \left(\vm - \bar{\vz}\right)
    \rVert}_2^2
    \\
    &\quad
    -2 \,
    \mathbb{E} 
    \left(1 + \textstyle{\sum^{d}_{i=1} \rvu_i^2} \right)
    \inner{
      \mSigma^{-1} \left(
      \mathcal{T}_{\vlambda}\left(\rvvu\right) - \bar{\vz}\right)
    }{
      \mS^{-1} \left(\mathcal{T}_{\vlambda}\left(\rvvu\right) - \bar{\vz}\right)
    }
    \\
    &\quad
    +2 \,
    \left(d + 1\right) 
    \inner{
      \mSigma^{-1} \left( \vm - \bar{\vz}\right)
    }{
      \mS^{-1} \left(\vm - \bar{\vz}\right)
    }
    \\
    &\quad
    -2 \,
    \left(d + 1\right) 
    \inner{
      \mS^{-1} \left(\vm - \bar{\vz}\right)
    }{
      \mS^{-1} \left(\vm - \bar{\vz}\right)
    }.
\shortintertext{The last two inner products can be denoted as norms such that}
    &\;=
    \mathbb{E} 
    \left(1 + \textstyle{\sum^{d}_{i=1} \rvu_i^2} \right)
    \Big(
    {\lVert
      \mSigma^{-1} \left(\mathcal{T}_{\vlambda}\left(\rvvu\right) - \bar{\vz}\right)
    \rVert}_2^2
    +
    {\lVert
      \mS^{-1} \left(\mathcal{T}_{\vlambda}\left(\rvvu\right) - \bar{\vz}\right)
    \rVert}_2^2
    \Big)
    \\
    &\quad
    +
    \left(d + 1\right)
    {\lVert
      \mS^{-1} \left(\vm - \bar{\vz}\right)
    \rVert}_2^2
    \\
    &\quad
    -2 \,
    \mathbb{E} 
    \left(1 + \textstyle{\sum^{d}_{i=1} \rvu_i^2} \right)
    {\lVert
      \mB^{-1} \mC^{-1} \left( \mathcal{T}_{\vlambda}\left(\rvvu\right) - \bar{\vz}\right)
    \rVert}_2^2
    \\
    &\quad
    +2 
    \left(d + 1\right)
    {\lVert
      \mB^{-1} \mC^{-1} \left(\vm - \bar{\vz}\right)
    \rVert}_2^2
    -2
    \left(d + 1\right)
    {\lVert
      \mS^{-1} \left(\vm - \bar{\vz}\right)
    \rVert}_2^2,
  \end{align*}
  where \(\mB\) is the matrix square root of \(\mSigma\) such that \(\mB^{-1} \mB^{-1} = \mSigma^{-1}\).
  The derivation so far applies to any Gaussian \(\pi, q_{\vlambda}\) and \(\vlambda \in \Lambda_{S}\) for any \(S > 0\).

  \paragraph{Worst-Case Lower Bound}
  Now, for our worst-case example, 
  \begin{align*}
    \mathbb{E} {\lVert \rvvg_{\mathrm{STL}}\left(\widetilde{\vlambda}\right) \rVert}_2^2
    &=
    \mathbb{E} 
    \left(1 + \textstyle{\sum^{d}_{i=1} \rvu_i^2} \right)
    \Big(
    {\lVert
      \mSigma^{-1} \left(\mathcal{T}_{\widetilde{\vlambda}}\left(\rvvu\right) - \bar{\vz}\right)
    \rVert}_2^2
    +
    {\lVert
      \widetilde{\mS}^{-1} \left(\mathcal{T}_{\widetilde{\vlambda}}\left(\rvvu\right) - \bar{\vz}\right)
    \rVert}_2^2
    \Big)
    \\
    &\quad
    +
    \left(d + 1\right)
    {\lVert
      \widetilde{\mS}^{-1} \left(\widetilde{\vm} - \bar{\vz}\right)
    \rVert}_2^2
    \\
    &\quad
    -2 \,
    \mathbb{E} 
    \left(1 + \textstyle{\sum^{d}_{i=1} \rvu_i^2} \right)
    {\lVert
      \mB^{-1} \widetilde{\mC}^{-1} \left( \mathcal{T}_{\widetilde{\vlambda}}\left(\rvvu\right) - \bar{\vz}\right)
    \rVert}_2^2
    \\
    &\quad
    +2 
    \left(d + 1\right)
    {\lVert
      \mB^{-1} \widetilde{\mC}^{-1} \left(\widetilde{\vm} - \bar{\vz}\right)
    \rVert}_2^2
    -2 
    \left(d + 1\right)
    {\lVert
      \widetilde{\mS}^{-1} \left(\widetilde{\vm} - \bar{\vz}\right)
    \rVert}_2^2,
\shortintertext{since \(\pi\) is \(\mu\)-strongly log-concave and \(\widetilde{\mS}^{-1} = L \boldupright{I}\),}
    &\geq
    \mathbb{E} 
    \left(1 + \textstyle{\sum^{d}_{i=1} \rvu_i^2} \right)
    \Big(
    L^{-2} \norm{\mathcal{T}_{\widetilde{\vlambda}}\left(\rvvu\right) - \bar{\vz}}_2^2
    +
    L^2 \norm{\mathcal{T}_{\widetilde{\vlambda}}\left(\rvvu\right) - \bar{\vz}}_2^2
    \Big)
    \\
    &\quad
    + \left(d + 1\right) L^2 \norm{\widetilde{\vm} - \bar{\vz}}_2^2
    \\
    &\quad
    -2 \,
    \mathbb{E} 
    \left(1 + \textstyle{\sum^{d}_{i=1} \rvu_i^2} \right)
    {\lVert
      \mB^{-1} \widetilde{\mC}^{-1} \left( \mathcal{T}_{\widetilde{\vlambda}}\left(\rvvu\right) - \bar{\vz}\right)
    \rVert}_2^2
    \\
    &\quad
    +2 \left(d + 1\right) \,
    {\lVert
      \mB^{-1} \widetilde{\mC}^{-1} \left(\widetilde{\vm} - \bar{\vz}\right)
    \rVert}_2^2
    -2 \left(d + 1\right) L^2 \norm{ \widetilde{\vm} - \bar{\vz} }_2^2,
\shortintertext{and grouping the terms,}
    &=
    \underbrace{
    \left( L^{-2} + L^2 \right) \,
    \mathbb{E} 
    \left(1 + \textstyle{\sum^{d}_{i=1} \rvu_i^2} \right)
    \norm{\mathcal{T}_{\widetilde{\vlambda}}\left(\rvvu\right) - \bar{\vz}}_2^2
    -
    \left(d + 1\right) L^2 \norm{\widetilde{\vm} - \bar{\vz}}_2^2
    }_{T_{\text{\ding{172}}}}
    \\
    &\qquad
    \underbrace{
    -2 \,
    \mathbb{E} 
    \left(1 + \textstyle{\sum^{d}_{i=1} \rvu_i^2} \right)
    {\lVert
      \mB^{-1} \widetilde{\mC}^{-1} \left( \mathcal{T}_{\widetilde{\vlambda}}\left(\rvvu\right) - \bar{\vz}\right)
    \rVert}_2^2
    }_{T_{\text{\ding{173}}}}
    \\
    &\qquad
    \underbrace{
    +2 \left(d + 1\right) \,
    {\lVert
      \mB^{-1} \widetilde{\mC}^{-1} \left(\widetilde{\vm} - \bar{\vz}\right)
    \rVert}_2^2.
    }_{T_{\text{\ding{174}}}}
  \end{align*}

\paragraph{Lower Bound on \(T_{\text{\ding{172}}}\)}
For \(T_{\text{\ding{172}}}\), we have
  \begin{align}
    T_{\text{\ding{172}}}
    &=
    \left(L^{-2} + L^2 \right)
    \mathbb{E} 
    \left(1 + \textstyle{\sum^{d}_{i=1} \rvu_i^2} \right)
    \norm{\mathcal{T}_{\widetilde{\vlambda}}\left(\rvvu\right) - \bar{\vz}}_2^2
    - 
    \left(d + 1\right) L^2 \norm{\widetilde{\vm} - \bar{\vz}}_2^2,
    \nonumber
\shortintertext{applying \cref{thm:normdist_1pnormu},}
    &=
    \left(L^{-2} + L^2 \right)
    \left(
      \left( d + 1 \right) \norm{\widetilde{\vm} - \bar{\vz}}_2^2
      +
      \left( d + k_{\varphi} \right) {\lVert \widetilde{\mC} \rVert}_{\mathrm{F}}^2
    \right)
    -
    \left(d + 1\right) L^2 \norm{\widetilde{\vm} - \bar{\vz}}_2^2,
    \nonumber
\shortintertext{and since \(L^{-2} > 0\) and is negligible for large \(L\)s,}
    &\geq
    L^2
    \left(
      \left( d + 1 \right) \norm{\widetilde{\vm} - \bar{\vz}}_2^2
      +
      \left( d + k_{\varphi} \right) {\lVert \widetilde{\mC} \rVert}_{\mathrm{F}}^2
    \right)
    -
    \left(d + 1\right) L^2 \norm{\widetilde{\vm} - \bar{\vz}}_2^2
    \nonumber
    \\
    &=
    L^2
    \left( d + k_{\varphi} \right) {\lVert \widetilde{\mC} \rVert}_{\mathrm{F}}^2.
    \label{eq:unimprovability_term1}
  \end{align}

\paragraph{Lower Bound on \(T_{\text{\ding{173}}}\)}
For \(T_{\text{\ding{173}}}\), we now use the covariance structures of our worst case through \cref{thm:lowerbound_matrix_innerproduct_lemma}.
That is,
  \begin{align*}
    T_{\text{\ding{173}}}
    &=
    -2 \,
    \mathbb{E} 
    \left(1 + \textstyle{\sum^{d}_{i=1} \rvu_i^2} \right)
    {\lVert
      \mB^{-1} \widetilde{\mC}^{-1} \left( \mathcal{T}_{\widetilde{\vlambda}}\left(\rvvu\right) - \bar{\vz}\right)
    \rVert}_2^2.
\shortintertext{Noting that \(\mathcal{T}_{\widetilde{\vlambda}}\left(\rvvu\right) = \widetilde{\mC} \rvvu + \widetilde{\vm}\) by definition, we can apply \cref{thm:lowerbound_matrix_innerproduct_lemma} Item (i) as}
    &=
    -2 \,
    \mathbb{E} 
    \left(1 + \textstyle{\sum^{d}_{i=1} \rvu_i^2} \right)
    \left(
      \norm{\mathcal{T}_{\widetilde{\vlambda}}\left(\rvvu\right) - \bar{\vz}}_2^2 + {\left(L - L^{-1} \right)} \rvu_1^2 
    \right),
\shortintertext{distributing the expectation and \(1 + \textstyle{\sum^{d}_{i=1} \rvu_i^2}\),}
    &=
    -2 \,
    \Big(
    \mathbb{E} 
    \left(1 + \textstyle{\sum^{d}_{i=1} \rvu_i^2} \right)
    \norm{\mathcal{T}_{\widetilde{\vlambda}}\left(\rvvu\right) - \bar{\vz}}_2^2
    +
    \underbrace{
    \mathbb{E} 
    \left(1 + \textstyle{\sum^{d}_{i=1} \rvu_i^2} \right)
    {\left(L - L^{-1} \right)} \rvu_1^2 
    }_{T_{\text{\ding{175}}}}
    \Big),
  \end{align*}

  \(T_{\text{\ding{175}}}\) follows as
  \begin{align}
    T_{\text{\ding{175}}}
    &=
    \mathbb{E} 
    \left(1 + \textstyle{\sum^{d}_{i=1} \rvu_i^2} \right)
    {\left(L - L^{-1} \right)} \rvu_1^2 
    \nonumber
    \\
    &=
    {\left(L^1 - L^{-1} \right)}
    \mathbb{E} 
    \left(1  + \textstyle{\sum^{d}_{i=1} \rvu_i^2} \right)  \rvu_1^2
    \nonumber
    \\
    &=
    {\left(L - L^{-1} \right)}
    \left(\mathbb{E} \rvu_1^2 + \mathbb{E} \rvu_1^4 + \textstyle{\sum^{d}_{i=2} \mathbb{E} \rvu_i^2 \mathbb{E} \rvu_1^2} \right)  ,
    \nonumber
\shortintertext{applying \cref{thm:u_identities},}
    &=
    {\left(L - L^{-1} \right)}
    \left(1 + k_{\varphi} + d - 1 \right)  
    \nonumber
    \\
    &=
    {\left(L - L^{-1} \right)} 
    \left(d + k_{\varphi}\right).
    \label{eq:stl_lowerbound_173}
  \end{align}

  Then,
  \begin{align}
    T_{\text{\ding{173}}}
    &=
    -2 \,
    \Big(
    \mathbb{E} 
    \left(1 + \textstyle{\sum^{d}_{i=1} \rvu_i^2} \right)
    \norm{\mathcal{T}_{\widetilde{\vlambda}}\left(\rvvu\right) - \bar{\vz}}_2^2
    +
    T_{\text{\ding{175}}}
    \Big),
    \nonumber
\shortintertext{bringing \cref{eq:stl_lowerbound_173} in,}
    &=
    -2 \,
    \Big(
    \mathbb{E} 
    \left(1 + \textstyle{\sum^{d}_{i=1} \rvu_i^2} \right)
    \norm{\mathcal{T}_{\widetilde{\vlambda}}\left(\rvvu\right) - \bar{\vz}}_2^2
    +
    {\left(L - L^{-1} \right)} 
    \left(d + k_{\varphi}\right)
    \Big),
    \nonumber
\shortintertext{applying \cref{thm:normdist_1pnormu},}
    &=
    -2 \,
    \Big(
    \left(d + k_{\varphi}\right) \norm{ \widetilde{\vm} - \bar{\vz} }_2^2
    +
    \left(d + 1 \right) {\lVert \widetilde{\mC} \rVert}_{\mathrm{F}}^2
    +
    \left(d + k_{\varphi}\right)
    {\left(L - L^{-1} \right)} 
    \Big)
    \label{eq:unimprovability_term2}
  \end{align}

\paragraph{Lower Bound on \(T_{\text{\ding{174}}}\)}
  Similarly for \(T_{\text{\ding{174}}}\), we can apply \cref{thm:lowerbound_matrix_innerproduct_lemma} Item (ii) as
  \begin{align}
    T_{\text{\ding{174}}}
    =
    2 \left(d+1\right)
    {\lVert
      \mB^{-1} \widetilde{\mC}^{-1} \left( \widetilde{\vm} - \bar{\vz}\right)
    \rVert}_2^2
    =
    2 \left(d+1\right) \norm{\widetilde{\vm} - \bar{\vz}}_2^2.
    \label{eq:unimprovability_term3}
  \end{align}

  Combining \cref{eq:unimprovability_term1,eq:unimprovability_term2,eq:unimprovability_term3},
  \begin{align*}
    \mathbb{E} {\lVert\rvvg_{\mathrm{STL}}\left(\widetilde{\vlambda}\right) \rVert}_2^2
    &\geq
    T_{\text{\ding{172}}} + T_{\text{\ding{173}}} + T_{\text{\ding{174}}}
    \\
    &\geq
    L^2
    \left( d + k_{\varphi} \right) {\lVert \widetilde{\mC} \rVert}_{\mathrm{F}}^2
    -2 \,
    \Big(
    \left(d + k_{\varphi}\right) \norm{ \widetilde{\vm} - \bar{\vz} }_2^2
    +
    \left(d + 1 \right) {\lVert \widetilde{\mC} \rVert}_{\mathrm{F}}^2
    +
    \left(d + k_{\varphi}\right)
    {\left(L - L^{-1} \right)} 
    \Big)
    + 2 \left(d + 1\right) \norm{\widetilde{\vm} - \bar{\vz}}_2^2
    \\
    &=
    \left(
    L^2
    \left( d + k_{\varphi} \right) 
    -2
    \left(d + 1 \right) 
    \right)
    {\lVert \widetilde{\mC} \rVert}_{\mathrm{F}}^2
    -
    2 \left(k_{\varphi} - 1\right) \norm{ \widetilde{\vm} - \bar{\vz} }_2^2
    +
    \left(d + k_{\varphi}\right)
    {\left(L - L^{-1} \right)} ,
\shortintertext{and when \(L \geq 1\), we have \(L - L^{-1} > 0\). Therefore, we can simply the bound as}
    &\geq
    \left(
    L^2
    \left( d + k_{\varphi} \right) 
    -2
    \left(d + 1 \right) 
    \right)
    {\lVert \widetilde{\mC} \rVert}_{\mathrm{F}}^2
    -
    2 \left(k_{\varphi} - 1\right) \norm{ \widetilde{\vm} - \bar{\vz} }_2^2.
  \end{align*}

\end{proofEnd}

%%% Local Variables:
%%% TeX-master: "main"
%%% End:

%
\vspace{0.5ex}
\begin{remark}\label{remark:stl_tightness}
  \cref{thm:stl_lowerbound_unimprovability} implies that \cref{thm:stl_upperbound}  with \(S = L\) is tight with respect to the dimension dependence \(d\) and the log-smoothness \(L\) except for a factor of 4.
\end{remark}
\vspace{1ex}
\begin{remark}[\textbf{Room for Improvement}]
  Part of the factor of \(4\) looseness is due to the extreme worst case: when \(\nabla \log \pi\) and \(\nabla \log q_{\vlambda}\) are anti-correlated.
  This worst case is unlikely to appear in practice, thus making a tighter lower bound challenging to obtain.
  But at the same time, we were unsuccessful at seeking a general assumption that would rule out these worst cases in the upper bound.
  Specifically, we tried very hard to apply coercivity/gradient monotonicity of log-concave distributions, but to no avail, leaving this to future works.
\end{remark}

\subsection{Theoretical Analysis of the CFE Estimator}
We now present the analysis of the CFE estimator.
While the CFE estimator has been studied in-depth by \citet{domke_provable_2019,kim_practical_2023,domke_provable_2023}, we slightly improve the latest analysis of \citet[Theorem 3]{domke_provable_2023}.
Specifically, we improve the constants and obtain an adaptive bound.
This ensures that we have a fair comparison with the STL estimator.
%

\begin{theoremEnd}[category=cfeupperbound]{theorem}\label{thm:cfe_upperbound}
  Assume \cref{assumption:variation_family} and that \(\pi\) is \(L\)-log-smooth.
  For the full-rank parameterization, the expected-squared norm of the CFE estimator is bounded as
  \begin{align*}
    \mathbb{E} \norm{ \rvvg_{\mathrm{CFE}}\left(\vlambda\right) }_2^2
    &\leq
    \left( L^2 \left( d + k_{\varphi} \right) \left(1 + \delta\right) + {\left(L + C \right)}^{2} \right) {\lVert \vlambda - \vlambda^* \rVert}_2^2 
    \\
    &\quad+
    L^2 \left( d + k_{\varphi} \right) \left(1 + \delta^{-1} \right) {\lVert \vlambda^* - \bar{\vlambda} \rVert}_2^2
  \end{align*}
  for any \(\vlambda \in \Lambda_S\) and \(\delta \geq 0\), where \(\bar{\vlambda} = \left(\bar{\vz}, \mathbf{0}\right)\) and \(\bar{\vz}\) is any stationary point of \(f\).
\end{theoremEnd}
\begin{proofEnd}
  Following the notation of \citet{domke_provable_2023}, we denote \( \nabla \log \pi = f \).
  Then, starting from the definition of the variance,
  \begin{align}
    \mathbb{E} \norm{ \rvvg_{\mathrm{CFE}}\left(\vlambda\right) }_2^2
    &=
    \mathrm{tr}\mathbb{V}\rvvg\left(\vlambda\right)
    +
    \norm{ \mathbb{E} \rvvg_{\mathrm{CFE}}\left(\vlambda\right) }_2^2,
    \nonumber
\shortintertext{and by the unbiasedness of \(\rvvg_{\mathrm{CFE}}\),}
    &=
    \mathrm{tr}\mathbb{V}\rvvg\left(\vlambda\right)
    +
    \norm{ \nabla F\left(\vlambda\right) }_2^2,
    \nonumber
\shortintertext{by the definition of \(\rvvg_{\mathrm{CFE}}\) (\cref{def:cfe}),}
    &=
    \mathrm{tr}\mathbb{V}_{\rvvz \sim q_{\vlambda}} \left( \nabla_{\vlambda} f\left( \rvvz \right) + \nabla \mathbb{H}\left(q_{\vlambda}\right) \right)
    +
    \norm{ \nabla F\left(\vlambda\right) }_2^2.
    \nonumber
\shortintertext{We now apply the property of the variance: the deterministic components are neglected as}
    &=
    \mathrm{tr}\mathbb{V}_{\rvvz \sim q_{\vlambda}} \nabla_{\vlambda} f\left( \rvvz \right) 
    +
    \norm{ \nabla F\left(\vlambda\right) }_2^2
    \nonumber
    \\
    &\leq
    \mathbb{E}_{\rvvz \sim q_{\vlambda}} \norm{ \nabla_{\vlambda} f\left( \rvvz \right) }_2^2
    +
    \norm{ \nabla F\left(\vlambda\right) }_2^2.
    \label{eq:thm_cfe}
  \end{align}

  For \(L\)-log-smooth posteriors (\(L\)-smooth \(f\)), \citet[Theorem 3]{domke_provable_2019} show that
  \begin{align*}
    \mathbb{E}_{\rvvz \sim q_{\vlambda}} \norm{ \nabla_{\vlambda} f\left( \rvvz \right) }_2^2
    &\leq
    L^2  \left( 
    \left( d + k_{\varphi} \right) \norm{ \vm - \bar{\vz} }_2^2
    +
    \left( d + 1 \right) \norm{ \mC }_{\mathrm{F}}^2
    \right),
\shortintertext{and since \(k_{\varphi} \geq 1\),}
    &\leq
    L^2  \left( 
    \left( d + k_{\varphi} \right) \norm{ \vm - \bar{\vz} }_2^2
    +
    \left( d + k_{\varphi} \right) \norm{ \mC }_{\mathrm{F}}^2
    \right)
    \\
    &=
    L^2 \left( d + k_{\varphi} \right) {\lVert \vlambda - \bar{\vlambda} \rVert}_2^2,
  \end{align*}
  which is tight.

  Applying \cref{eq:peterpaul}, we have
  \begin{align}
    \mathbb{E}_{\rvvz \sim q_{\vlambda}} {\lVert \nabla_{\vlambda} f\left( \rvvz \right) \rVert}_2^2
    &\leq
    L^2 \left( d + k_{\varphi} \right) {\lVert \vlambda - \bar{\vlambda} \rVert}_2^2
    \nonumber
    \\
    &=
    L^2 \left( d + k_{\varphi} \right) {\lVert \vlambda - \vlambda^* + \vlambda^* - \bar{\vlambda} \rVert}_2^2
    \nonumber
    \\
    &\leq
    L^2 \left( d + k_{\varphi} \right) \left( \left(1 + \delta \right) {\lVert \vlambda - \vlambda^* \rVert}_2^2 + \left(1 + \delta^{-1} \right) {\lVert \vlambda^* - \bar{\vlambda} \rVert}_2^2 \right).
    \label{eq:thm_cfe_energy}
  \end{align}
  
  Now, for \(\vlambda \in \Lambda_S\),~\citet[Theorem 1 \& Lemma 12]{domke_provable_2020} show that the negative ELBO \(F\) is (\(L + S\))-smooth as
  \begin{align}
    \norm{ \nabla F\left(\vlambda\right) }_2^2
    =
    \norm{ \nabla F\left(\vlambda\right) - \nabla F\left(\vlambda^*\right) }_2^2
    \leq 
    {\left(L + S \right)}^{2} \norm{ \vlambda - \vlambda^* }_2^2.
    \label{eq:thm_cfe_entropy}
  \end{align}

  Now back to \cref{eq:thm_cfe},
  \begin{align*}
    \mathbb{E} \norm{ \rvvg_{\mathrm{CFE}}\left(\vlambda\right) }_2^2
    &\leq
    \mathbb{E}_{\rvvz \sim q_{\vlambda}} \norm{ \nabla_{\vlambda} f\left( \rvvz \right) }_2^2
    +
    \norm{ \nabla F\left(\vlambda\right) }_2^2
\shortintertext{applying \cref{eq:thm_cfe_energy},}
    &\leq
    L^2 \left( d + k_{\varphi} \right) \left( \left(1 + \delta \right) {\lVert \vlambda - \vlambda^* \rVert}_2^2 + \left(1 + \delta^{-1} \right) {\lVert \vlambda^* - \bar{\vlambda} \rVert}_2^2 \right)
    +
    \norm{ \nabla F\left(\vlambda\right) }_2^2
\shortintertext{and \cref{eq:thm_cfe_entropy},}
    &\leq
    L^2 \left( d + k_{\varphi} \right) \left( \left(1 + \delta \right) {\lVert \vlambda - \vlambda^* \rVert}_2^2 + \left(1 + \delta^{-1} \right) {\lVert \vlambda^* - \bar{\vlambda} \rVert}_2^2 \right)
    +
    {\left(L + C \right)}^{2} \norm{ \vlambda - \vlambda^* }_2^2
    \\
    &=
    \left( L^2 \left( d + k_{\varphi} \right) \left(1 + \delta\right) + {\left(L + C \right)}^{2} \right) {\lVert \vlambda - \vlambda^* \rVert}_2^2 
    +
    L^2 \left( d + k_{\varphi} \right) \left(1 + \delta^{-1} \right) {\lVert \vlambda^* - \bar{\vlambda} \rVert}_2^2.
  \end{align*}
\end{proofEnd}


\begin{theoremEnd}[all end, category=cfeupperboundmf]{theorem}\label{thm:cfe_upperbound_mf}
  Assume \cref{assumption:variation_family} and that \(\pi\) is \(L\)-log-smooth.
  For the mean-field parameterization, the expected-squared norm of the CFE estimator is bounded as
  \begin{align*}
    \mathbb{E} \norm{ \rvvg_{\mathrm{CFE}}\left(\vlambda\right) }_2^2
    &\leq
    \big(
    \left(2 k_{\varphi} \sqrt{d} + 1 \right) 
    \left(1 + \delta \right)
    +
    {\left(L + S \right)}^{2}
    \big)
    {\lVert \vlambda - \vlambda^* \rVert}_2^2
    \\
    &\quad+
    \left(2 k_{\varphi} \sqrt{d} + 1 \right) \left(1 + \delta^{-1} \right) {\lVert \vlambda^* - \bar{\vlambda} \rVert}_2^2.
  \end{align*}
  for any \(\vlambda \in \Lambda_S\) and \(\delta \geq 0\), where \(\bar{\vlambda} = \left(\bar{\vz}, \mathbf{0}\right)\) and \(\bar{\vz}\) is any stationary point of \(f\).
\end{theoremEnd}
\begin{proofEnd}
  For the mean-field case, the only difference with \cref{thm:cfe_upperbound} is the upper bound on the energy term.
  The key step is the mean-field part of \cref{thm:normdist_1pnormu}, first proven by \citet{kim_practical_2023}.
  The remaining steps are similar to Theorem 1 of \citet{kim_practical_2023}.
  That is,
  \begin{align*}
    \mathbb{E}_{\rvvz \sim q_{\vlambda}} \norm{ \nabla_{\vlambda} f\left( \rvvz \right) }_2^2
    &=
    \mathbb{E} \norm{ \nabla_{\vlambda} f\left( \mathcal{T}_{\vlambda}\left(\rvvu\right) \right) }_2^2,
\shortintertext{applying \cref{thm:jacobian_reparam_inner},}
    &=
    \mathbb{E} J_{\mathcal{T}}\left(\rvvu\right) \norm{ \nabla f\left( \mathcal{T}_{\vlambda}\left(\rvvu\right) \right) }_2^2
    =
    \mathbb{E} J_{\mathcal{T}}\left(\rvvu\right) \norm{ \nabla f\left( \mathcal{T}_{\vlambda}\left(\rvvu\right) \right) - \nabla f\left(\bar{\vz}\right) }_2^2,
\shortintertext{from \(L\)-smoothness of \(f = \log \pi\),}
    &\leq
    L^2 \, J_{\mathcal{T}}\left(\rvvu\right) \mathbb{E} \norm{ \mathcal{T}_{\vlambda}\left(\rvvu\right) - \bar{\vz} }_2^2,
\shortintertext{applying \cref{thm:normdist_1pnormu},}
    &\leq
    L^2 \left(\sqrt{dk_{\varphi}} + k_{\varphi} \sqrt{d} + 1 \right) \norm{ \vm - \bar{\vz} }_2^2 + L^2 \left(2 k_{\varphi} \sqrt{d} + 1\right) \norm{\mC}_{\mathrm{F}}^2.
\shortintertext{and since \(k_{\varphi} \geq 1\), we have \(k_{\varphi} > \sqrt{k_{\varphi}}\), and thus}
    &\leq
    L^2 \left(2 k_{\varphi} \sqrt{d} + 1 \right) \left( \norm{ \vm - \bar{\vz} }_2^2 +  \norm{\mC}_{\mathrm{F}}^2 \right)
    \\
    &=
    L^2 \left(2 k_{\varphi} \sqrt{d} + 1 \right) {\lVert \vlambda - \bar{\vlambda} \rVert}_2^2.
\shortintertext{We finally apply \cref{eq:peterpaul} as}
    &\leq
    L^2 \left(2 k_{\varphi} \sqrt{d} + 1 \right)
    \left(
      \left(1 + \delta \right) {\lVert \vlambda - \vlambda^* \rVert}_2^2
      + \left(1 + \delta^{-1} \right) {\lVert \vlambda^* - \bar{\vlambda} \rVert}_2^2
    \right).
  \end{align*}

  Combining this with \cref{eq:thm_cfe,eq:thm_cfe_entropy}, we have
  \begin{align*}
    \mathbb{E} \norm{ \rvvg_{\mathrm{CFE}}\left(\vlambda\right) }_2^2
    &=
    \mathbb{E}_{\rvvz \sim q_{\vlambda}} \norm{ \nabla_{\vlambda} f\left( \rvvz \right) }_2^2
    +
    \norm{ \nabla F\left(\vlambda\right) }_2^2
\shortintertext{and applying \cref{eq:thm_cfe_energy},}
    &\leq
    \mathbb{E}_{\rvvz \sim q_{\vlambda}} \norm{ \nabla_{\vlambda} f\left( \rvvz \right) }_2^2
    +
    {\left(L + S \right)}^{2} \norm{ \vlambda - \vlambda^* }_2^2
    \\
    &\leq
    \left(2 k_{\varphi} \sqrt{d} + 1 \right)
    \left(
      L^2 \left(1 + \delta \right) {\lVert \vlambda - \vlambda^* \rVert}_2^2
      + L^2 \left(1 + \delta^{-1} \right) {\lVert \vlambda^* - \bar{\vlambda} \rVert}_2^2
    \right)
    \\
    &\qquad+
    {\left(L + S \right)}^{2} \norm{ \vlambda - \vlambda^* }_2^2
    \\
    &=
    \left(
    \left(2 k_{\varphi} \sqrt{d} + 1 \right) 
    L^2 \left(1 + \delta \right)
    +
    {\left(L + S \right)}^{2}
    \right)
    {\lVert \vlambda - \vlambda^* \rVert}_2^2
    \\
    &\qquad+
    L^2 \left(2 k_{\varphi} \sqrt{d} + 1 \right) \left(1 + \delta^{-1} \right) {\lVert \vlambda^* - \bar{\vlambda} \rVert}_2^2.
  \end{align*}

\end{proofEnd}


% \begin{corollary}
%   Let \(S = L\).
%   Then, the expected-squared norm of the CFE estimator with the full-rank parameterization satisfies the QVC for any \(\vlambda \in \Lambda_{L}\) with the constants
% {%
% \setlength{\abovedisplayskip}{.5ex} \setlength{\abovedisplayshortskip}{.5ex}
% \setlength{\belowdisplayskip}{1.ex} \setlength{\belowdisplayshortskip}{1.ex}
%   \begin{align*}
%     \alpha_{\mathrm{CFE}} &= L^2 \left( d + k_{\varphi} + 4 \right) \left(1 + \delta\right),\\
%     \beta_{\mathrm{CFE}}  &= L^2 \left( d + k_{\varphi} \right) \left( 1 + \delta^{-1}  \right) {\lVert \bar{\vlambda} - \vlambda^* \rVert}_2^2,
%   \end{align*}
%   }%
%   for any \(\delta \geq 0\).
% \end{corollary}

\vspace{1ex}
\begin{remark}[\textbf{Comparison with STL}]\label{remark:variance_comparison}
    Compared to the STL estimator, the constant \(\alpha\) of the CFE estimator is tighter by a factor of \(4\).
    Considering \cref{thm:stl_lowerbound_unimprovability}, the constant factor difference should be marginal in practice.
    %This means that the STL estimator will perform similarly when \({\lVert \bar{\vlambda} - \vlambda^* \rVert}_2^2\) large.
    % When \({\lVert \bar{\vlambda} - \vlambda^* \rVert}_2^2\) is small, it might have a smaller variance when \(\mathrm{D}_{F^4}\left(q_{\vlambda^*}, \pi\right)\) is small.
\end{remark}

\vspace{1ex}
\begin{remark}[\textbf{Intuitions on \({\lVert \bar{\vlambda} - \vlambda^* \rVert}_2^2\)}]
  The quantity \({\lVert \bar{\vlambda} - \vlambda^* \rVert}_2^2\) can be expressed in the Wasserstein-2 distance as
{%
\setlength{\abovedisplayskip}{.5ex} \setlength{\abovedisplayshortskip}{.5ex}
\setlength{\belowdisplayskip}{1.ex} \setlength{\belowdisplayshortskip}{1.ex}
  \[
    \mathrm{d}_{\mathcal{W}_2}\left(q_{\vlambda^*}, \delta_{\bar{\vz}}\right) = \sqrt{ {\lVert \vm^* - \bar{\vz}\rVert}_2^2 + \norm{\mC^*}_{\mathrm{F}}^2} =  {\lVert \bar{\vlambda} - \vlambda^* \rVert}_2,
  \]
}
  where \(\delta_{\bar{\vz}}\) is a delta measure centered on the posterior mode \(\bar{\vz}\).
   Also, when the variational posterior mean \(\vm^*\) is close to \(\bar{\vz}\) such that  \({\lVert \vm^* - \bar{\vz}\rVert}_2^2 \approx 0\), \({\lVert \bar{\vlambda} - \vlambda^* \rVert}_2^2\) corresponds to the variational posterior variance as
{%
\setlength{\abovedisplayskip}{1ex} \setlength{\abovedisplayshortskip}{1ex}
\setlength{\belowdisplayskip}{1.ex} \setlength{\belowdisplayshortskip}{1.ex}
  \[
     {\lVert \bar{\vlambda} - \vlambda^* \rVert}_2^2 \approx \norm{\mC^*}_{\mathrm{F}}^2 = \mathrm{tr}\, \Vsub{\rvvz \sim q_{\vlambda^*}}{\rvvz}.
  \]
}
\end{remark}

% \vspace{-1.ex}
% \subsection{Non-Asymptotic Complexity of SGD with the Quadratic Variance Condition}


\begin{theoremEnd}[category=complexityprojsgdqvc]{theorem}[\textbf{Strongly convex \(F\) with a fixed stepsize}]\label{thm:projsgd_stronglyconvex_fixedstepsize}
  For a \(\mu\)-strongly convex \(F : \Lambda \to \mathbb{R}\) on a convex set \(\Lambda\), the last iterate \(\vlambda_{T}\) of projected SGD with a fixed stepsize satisfies \( {\lVert \vlambda_T - \vlambda^* \rVert}_2^2 \leq \epsilon \) if
  \begin{align*}
    \gamma = \min\left( \frac{\epsilon \mu}{4 \beta}, \frac{\mu}{2 \alpha}, \frac{2}{\mu} \right)  \quad\text{and}\quad
    T \geq \max\left( \frac{ 4 \beta }{\mu^2 } \frac{1}{\epsilon}, \frac{2 \alpha}{\mu^2}, \frac{1}{2} \right) \log \left( 2 \norm{\vlambda_0 - \vlambda^*}_2^2 \, \frac{1}{\epsilon} \right).
  \end{align*}
\end{theoremEnd}
\begin{proofEnd}
  Theorem 6 of \citet{domke_provable_2023} utilizes the two-stage stepsize of \citep{gower_sgd_2019}.
  The anytime convergence of the first stage, 
  \[
     \norm{ \vlambda_{T} - \vlambda^* }_2^2
     \leq
     {(1 - \gamma \mu)}^{T} \norm{ \vlambda_0 - \vlambda^* }_2^2
     +
     \frac{2 \gamma \beta}{\mu}
  \]
  corresponds to the SGD with only a fixed stepsize \(\gamma < \frac{\mu}{2 \alpha}\).

  Here, the result follows from Lemma A.2 of \citet{garrigos_handbook_2023} by plugging the constants
  \[
    \alpha_0 = \norm{ \vlambda_0 - \vlambda^* }_2^2, \quad
    A = \frac{2 \beta}{\mu}, 
    \;\text{and}\quad
    C = \frac{2 \alpha}{\mu}, \frac{\mu}{2}.
  \]
\end{proofEnd}


\begin{theoremEnd}[category=complexityprojsgdqvc]{theorem}[\textbf{Strongly convex \(F\) with a decreasing stepsize schedule}]\label{thm:projsgd_stronglyconvex_decstepsize}
  For a \(\mu\)-strongly convex \(F : \Lambda \to \mathbb{R}\) on a convex set \(\Lambda\), the last iterate \(\vlambda_{T}\) of projected SGD with a descreasing stepsize satisfies \( {\lVert \vlambda_T - \vlambda^* \rVert}_2^2 \leq \epsilon \) if
  \begin{align*}
    \gamma_t = \min\left( \frac{\mu}{2 \alpha}, \frac{4 t + 2 }{ \mu \, {\left( t + 1\right)}^2 } \right)
    \quad\text{and}\quad 
    T \geq \frac{16 \beta}{ \mu^2 } \frac{1}{\epsilon} + \frac{8 \sqrt{2} \, \alpha \, \norm{\vlambda_0 - \vlambda^*}_2 }{ \mu^2} \frac{1}{\sqrt{\epsilon}}.
  \end{align*}
\end{theoremEnd}
\begin{proofEnd}
  Theorem 6 of \citet{domke_provable_2023} utilizes the two-stage stepsize of \citet{gower_sgd_2019}.
  After \(T\) steps, with a carefully tuned stepsize of 
  \[
    \gamma_t = \min\left( \frac{\mu}{2 \alpha}, \frac{4 t + 2 }{ \mu \, {\left( t + 1\right)}^2 } \right)
  \]
  the algorithm achieves
  \[
    \norm{\vlambda_{T} - \vlambda^*}_2^2
    \leq
    \frac{64 \alpha^2}{\mu^4} \frac{ \norm{\vlambda_0 - \vlambda^*}_2^2 }{T^2} 
    +
    \frac{32 \beta}{\mu^2} \frac{1}{T}.
  \]
  Following a similar strategy to \citet{kim_blackbox_2023}, we can obtain a computational complexity by solving for the smallest \(T\) that achieves
  \[
    \frac{64 \alpha}{\mu^2} \frac{ \norm{\vlambda_0 - \vlambda^*}_2^2 }{T^2} 
    +
    \frac{16 \beta}{\mu^2} \frac{1}{T}
    \leq
    \epsilon.
  \]
  After re-organizing, we solve for
  \[
     A \, T^2  + B \, T + C = 0,
  \]
  where 
  \[
    A = \epsilon, \quad
    B = -\frac{16 \beta}{\mu^2},\; \text{and} \quad
    C = - \frac{64 \alpha^2}{\mu^4} \norm{\vlambda_0 - \vlambda^*}_2^2.
  \]
  Since \(T > 0\), the equation has a unique root
  \begin{align*}
    T 
    &= \frac{ - B + \sqrt{ B^2 - 4 A C  } }{ 2 A },
\shortintertext{applying the inequality \(\sqrt{a + b} \leq \sqrt{a} + \sqrt{b}\) for \(a, b \geq 0\),}
    &\leq \frac{ - B + \sqrt{ B^2 } +  \sqrt{  4 A \left( -C \right)  } }{ 2 A }
    \\
    &= \frac{2 \left(-B\right) }{2 A} + \frac{ \sqrt{ 4 A  \left( -C \right)  } }{ 2 A }
    \\
    &= \frac{\left(-B\right)}{A} + \frac{ \sqrt{ 2 \left( -C \right) }  } { \sqrt{A} }
    \\
    &= \frac{ 16 \beta }{ \mu^2 \epsilon} + \frac{ \sqrt{  \frac{128\alpha^2}{\mu^4} \norm{\vlambda_0 - \vlambda^*}_2^2  }  } { \sqrt{\epsilon} }
    \\
    &= \frac{ 16 \beta }{ \mu^2 \epsilon} + \frac{ 8 \sqrt{2}  \alpha \norm{\vlambda_0 - \vlambda^*}_2 } { \mu^2 \sqrt{\epsilon} }.
  \end{align*}
\end{proofEnd}

% 
\begin{theoremEnd}[category=complexityprojsgdqvc]{theorem}[\textbf{Convex \(F\) with a fixed stepsize}]\label{thm:projsgd_convex_fixedstepsize}
  For a convex \(F : \Lambda \to \mathbb{R}\) on a convex set \(\Lambda\), the last weighted average of the iterates generated by projected SGD, \(\bar{\vlambda}_T = \sum^{T}_{t=0}  w^{t+1} \vlambda_t \big/ \sum^T_{t=0} w^{t+1}\), where \(w = 1 / \left(1 + \alpha \gamma^2 \right)\), satisfies \( F\left(\bar{\vlambda}_T\right) - F\left(\vlambda^*\right) \leq \epsilon \), where \(\vlambda^* \in \argmin_{\vlambda \in \Lambda} F\left(\vlambda\right)\), if
  \[
  \gamma =
  \frac{
    -\beta + \sqrt{
      \beta^2 + 8 \epsilon^2 \alpha 
    } 
  }{
    4 \epsilon \alpha
  }
  \quad\text{and}\quad
  T
  \geq
  \frac{2 \alpha}{ -\beta + \sqrt{ \beta^2 + 8 \epsilon^2 \alpha } }
  \, \norm{\vlambda_0 - \vlambda^*}_2^2.
  \]
\end{theoremEnd}
\begin{proofEnd}
  Theorem 7 of \citet{domke_provable_2023} states that
  \[
     F\left(\bar{\vlambda}_T\right) - F\left(\vlambda^*\right)
     \leq
     \frac{ \gamma \alpha  }{ 2 \left( 1 - \theta^{T} \right) } \norm{\vlambda_0 - \vlambda^*}_2^2 + \frac{\gamma \beta}{2},
  \]
  where \(\theta = 1 / (1 + 2 \alpha \gamma^2)\).
  For the first term, we generalize of Lemma 28 of \citet{domke_provable_2023} as
  \begin{align}
    \frac{\gamma \alpha }{2 \left( 1 - \theta^{T} \right) } 
    &\leq \frac{\gamma \alpha}{2 \left( 1 - \theta \right) {T}}
    =    \frac{\gamma \alpha}{2 \left( 1 - \frac{1}{1 + 2 \alpha \gamma^2} \right) {T}}
    =    \frac{\gamma \alpha}{2 \left( \frac{2 \alpha \gamma^2 }{1 + 2 \alpha \gamma^2} \right) {T}}
    =    \frac{ 1 + 2 \alpha \gamma^2 }{4 \gamma} \frac{1}{T}
    \nonumber
    \\
    &=    \frac{1}{2} \left(\alpha \gamma + \frac{1}{2 \gamma} \right) \frac{1}{T}.
    \label{eq:stl_convex_fixedstepsize_optterm}
  \end{align}
  Note that the use of Bernoulli's inequality here is quite loose.
  Originally, \citeauthor{domke_provable_2023} chose \(\gamma = 1/\sqrt{T}\), which means one needs to fix the number of SGD iterations before actually running the algorithm.
  We instead prove convergence with a stepsize independent of \(T\).

  Based on \cref{eq:stl_convex_fixedstepsize_optterm}, the fixed-stepsize any-time convergence result becomes
  \[
     F\left(\bar{\vlambda}_T\right) - F\left(\vlambda^*\right)
     \leq
     \underbrace{
       \frac{1}{2} \left(\alpha \gamma + \frac{1}{2 \gamma} \right) \frac{1}{T} \, \norm{\vlambda_0 - \vlambda^*}_2^2
     }_{\text{optimization term}}
     +
     \underbrace{
       \frac{\gamma \beta}{2}.
     }_{\text{statistical term}}
  \]

  Define
  \begin{align*}
    \epsilon_{\mathrm{opt.}}
    \triangleq
    \frac{1}{2} \left(\alpha \gamma + \frac{1}{2 \gamma} \right) \frac{1}{T} \, \norm{\vlambda_0 - \vlambda^*}_2^2,
    %
    \quad\text{and}\quad
    %
    \epsilon_{\mathrm{stat.}}
    \triangleq
    \frac{\gamma \beta}{2}.
  \end{align*}
  We aim to solve the convex program:
  \begin{alignat*}{3}
    &\minimize_{T, \gamma}\quad  &&T \\
    &\text{subject to}\quad && \epsilon_{\text{opt.}}\left(T, \gamma\right) + \epsilon_{\text{stat.}}\left(\gamma\right) = \epsilon \\
    & && \gamma > 0,\quad T > 0.
  \end{alignat*}
  The Lagrangian is given as
  \begin{alignat*}{2}
    \mathcal{L}\left(T, \gamma, \lambda\right)
    =
    T + \lambda \left(\epsilon_{\text{opt.}}\left(T, \gamma\right) + \epsilon_{\text{stat.}}\left(\gamma\right) - \epsilon \right)
  \end{alignat*}
  with respect to the Lagrangian multiplier \(\lambda\).

  The stationary point of \(\mathcal{L}\) is found by solving the system of equations
  \begin{alignat*}{4}
    \frac{\partial \mathcal{L}}{\partial T}
    &=
    \;
    1 + \lambda \frac{\partial \epsilon_{\text{opt.}}}{\partial T}
    \quad&&=
    0
    &&\qquad\Leftrightarrow\qquad
    -\lambda \frac{\partial \epsilon_{\text{opt.}}}{\partial T}
    &&=
    1 
    \\
    %
    \frac{\partial \mathcal{L}}{\partial \lambda}
    &=
    \;
    \epsilon_{\text{opt.}}\left(T, \gamma\right)
    +
    \epsilon_{\text{stat.}}\left(\gamma\right)
    - 
    \epsilon
    &&=
    0
    &&\qquad\Leftrightarrow\qquad
    \epsilon_{\text{opt.}}\left(T, \gamma\right)
    +
    \epsilon_{\text{stat.}}\left(\gamma\right)
    &&=
    \epsilon
    \\
    %
    \frac{\partial \mathcal{L}}{\partial \gamma}
    &=
    \;
    \lambda
    \left(
    \frac{\partial \epsilon_{\text{opt.}}}{\partial \gamma}
    +
    \frac{\partial \epsilon_{\text{stat.}}}{\partial \gamma}
    \right)
    &&=
    0
    &&\qquad\Leftrightarrow\qquad
    \frac{\partial \epsilon_{\text{opt.}}}{\partial \gamma}
    &&=
    -\frac{\partial \epsilon_{\text{stat.}}}{\partial \gamma}
  \end{alignat*}
  The partial derivatives are given as
  \begin{align*}
    \frac{ \partial \epsilon_{\mathrm{opt.}} }{ \partial T }
    &=
    - \frac{1}{2} \left(\alpha \gamma + \frac{1}{2 \gamma}\right) \norm{\vlambda_0 - \vlambda^*}_2^2 \frac{1}{T^2}
    \\
    \frac{ \partial \epsilon_{\mathrm{opt.}} }{ \partial \gamma }
    &=
    \frac{1}{2} \left(\alpha - \frac{1}{2 \gamma^2}\right) \norm{\vlambda_0 - \vlambda^*}_2^2 \frac{1}{T}
    \\
    \frac{ \partial \epsilon_{\mathrm{stat.}} }{ \partial \gamma }
    &=
    \frac{\beta}{2}.
  \end{align*}
  Applying these to the Lagrangian system, we now need to solve the system:
  \begin{alignat}{2}
    \frac{\lambda}{2} \left(\alpha \gamma + \frac{1}{2 \gamma}\right) \norm{\vlambda_0 - \vlambda^*}_2^2 \frac{1}{T^2} &= 1
    \label{eq:lagrangian1}
    \\
    \frac{1}{2} \left(\alpha \gamma + \frac{1}{2 \gamma} \right) \frac{1}{T} \, \norm{\vlambda_0 - \vlambda^*}_2^2
    +
    \frac{\gamma \beta}{2}
    &=
    \epsilon
    \label{eq:lagrangian2}
    \\
    \frac{1}{2} \left(\alpha - \frac{1}{2 \gamma^2}\right) \norm{\vlambda_0 - \vlambda^*}_2^2 \frac{1}{T}
    &=
    -\frac{\beta}{2}.
    \label{eq:lagrangian3}
  \end{alignat}
  Notice that \(\lambda\) in \cref{eq:lagrangian1} is a free variable.
  Thus the last two equations \cref{eq:lagrangian2,eq:lagrangian3} are the only relevant equations.
  In particular, from \cref{eq:lagrangian3}, we obtain the identity
  \begin{alignat}{2}
    \norm{\vlambda_0 - \vlambda^*}_2^2 \frac{1}{T}
    =
    \frac{ 2 \beta \gamma^2 }{ 1 - 2 \gamma^2 \alpha }.
    \label{eq:projsgd_convex_keyeq}
  \end{alignat}
  Applying this to \cref{eq:lagrangian2}, we can decouple \(T\) and \(\gamma\), obtaining the quadratic equation
  \begin{alignat*}{3}
    & &\quad
    \frac{1}{2} \left(\alpha \gamma + \frac{1}{2 \gamma} \right) \frac{1}{T} \, \norm{\vlambda_0 - \vlambda^*}_2^2
    +
    \frac{\gamma \beta}{2}
    &=
    \epsilon
    \\
    &\Leftrightarrow&\quad
    \frac{1}{2} \left(\alpha \gamma + \frac{1}{2 \gamma} \right) 
    \left(
    \frac{ 2 \beta \gamma^2 }{ 1 - 2 \gamma^2 \alpha }
    \right)
    +
    \frac{\gamma \beta}{2}
    &=
    \epsilon
    \\
    &\Leftrightarrow&\quad
    \left( 2 \epsilon \alpha \right) \gamma^2 
    +
    \beta \gamma 
    -
    \epsilon
    &=
    0.
  \end{alignat*}
  Since \(\gamma > 0\), this quadratic has the unique solution
  \begin{alignat}{2}
    \gamma = \frac{
      -\beta + \sqrt{
        \beta^2 + 8 \epsilon^2 \alpha 
      } 
    }{
      4 \epsilon \alpha
    } > 0.
    \label{eq:projsgd_convex_optgamma}
  \end{alignat}
  as long as \(0 < \alpha < \infty\).

  Note that \cref{eq:projsgd_convex_keyeq} can be represented as
  \begin{alignat*}{3}
    & &\quad
    \norm{\vlambda_0 - \vlambda^*}_2^2 \frac{1}{T}
    &=
    \frac{ 2 \beta \gamma^2 }{ 1 - 2 \gamma^2 \alpha }
    \\
    &\Leftrightarrow&\quad
    T
    &=
    \left( \frac{1}{\gamma^2} - 2 \alpha \right) \frac{1}{2 \beta}
    \norm{\vlambda_0 - \vlambda^*}_2^2.
  \end{alignat*}
  Plugging \cref{eq:projsgd_convex_optgamma},
  \begin{alignat*}{3}
    T
    &=
    \left(
      \frac{
        16 \epsilon^2 \alpha^2
      }{
        {\left( -\beta + \sqrt{
          \beta^2 + 8 \epsilon^2 \alpha 
        }
        \right)}^2
      }
    - 2 \alpha \right) \frac{1}{2 \beta}
    \norm{\vlambda_0 - \vlambda^*}_2^2
    \\
    &=
    \frac{2 \alpha}{ -\beta + \sqrt{ \beta^2 + 8 \epsilon^2 \alpha } }
    \, \norm{\vlambda_0 - \vlambda^*}_2^2.
  \end{alignat*}
\end{proofEnd}


%% \begin{remark}
%%   The bound for \cref{thm:projsgd_convex_fixedstepsize} is inconveniently non-linear with respect to \(\beta\) and \(\alpha\).
%%   Under ``interpolation'' such that \(\beta = 0\), it clearly reduces to a \(\mathcal{O}\left(1/\epsilon\right)\).
%%   When \(\beta > 0\) is non-negligible, a series expansion does suggest that the bound behaves as \(\mathcal{O}\left(1/\epsilon^2\right)\), which is the expected complexity guarantee~\citep{garrigos_handbook_2023}.
%% \end{remark}

%% \begin{remark}
%%   Note that \cref{thm:projsgd_convex_fixedstepsize} is quite loose due to the use of Bernoulli's inequality for simplifying the any-time convergence result by \citet[Theorem 7]{domke_provable_2023}. (This inequality is also used by \citeauthor{domke_provable_2023}.) 
%%   Tightening the any-time convergence statement would be an important future direction.
%% \end{remark}

% \begin{remark}
%   Lastly, obtaining a \(\mathcal{O}\left(1/\sqrt{T}\right)\) convergence guarantee with a decreasing stepsize schedule independent of the number of steps \(T\) is an open problem.
%   (\citet[Theorem 7]{domke_provable_2023} use a fixed stepsize dependent on \(T\). Thus, one must fix the number of steps before running projected SGD.) 
% \end{remark}

\subsection{Non-Asymptotic Complexity of Black-Box Variational Inference}\label{section:bbvicomplexity}
\paragraph{Strongly Log-Concave Posteriors}
First, let us define the following:
\begin{definition}
    \(\pi\) is said to be \(\mu\)-strongly log-concave if its log-density \(\log \pi : \mathbb{R}^d \to \mathbb{R}\) (equivalently \(\log \ell\)) satisfies the inequality
{%
\setlength{\abovedisplayskip}{.5ex} \setlength{\abovedisplayshortskip}{.5ex}
\setlength{\belowdisplayskip}{1.ex} \setlength{\belowdisplayshortskip}{1.ex}
    \[
      \inner{ \nabla \log \pi\left(\vz\right) }{ \vz - \vz' }
      \geq \log \pi\left(\vz\right) - \log \pi\left(\vz'\right)
      +
      \frac{\mu}{2} \norm{ \vz - \vz' }^2_2
    \]
}%
    for all \(\vz, \vz' \in \mathbb{R}^d\) and some \(\mu > 0\).
\end{definition}
Essentially, this assumes that the log-density of \(\pi\) is \(\mu\)-strongly convex.
It also implies that the density of \(\pi\) is lower bounded by some Gaussian.
A consequence of \(\mu\)-strong log-concavity is that, combined with the constraints on the variational parameterization in \cref{section:scale_parameterization}, the ELBO is also \(\mu\)-strongly convex \citep{domke_provable_2023,kim_convergence_2023,challis_gaussian_2013}.
\(\mu\)-strongly log-concave posteriors can easily be constructed by combining a log-concave likelihood with a Gaussian prior, and are popularly used to analyze BBVI and sampling algorithms.

\vspace{-1.ex}
\paragraph{Theoretical Setup}
We now apply the general complexity results for projected SGD established in \cref{section:projsgdcomplexity} to BBVI.
\begin{enumerate*}[label=\textbf{(\roman*)}]
    \item strongly log-concave posteriors,
    \item SGD run with fixed stepsizes, and 
    \item the full-rank variational family.
\end{enumerate*}
This is because the convergence analyses for \textbf{(ii)} \(\cap\) \textbf{(iii)} are the tightest.
Although the bounds for the mean-field parameterization have better dependences on \(d\), so far, it is unknown whether they are tight~\citep{kim_practical_2023}. (See also \citealp[Conjecture 1]{kim_convergence_2023}.)


\begin{theoremEnd}[all end, category=complexityprojsgdadaptiveqvcfixed]{lemma}[\textbf{Strongly convex \(F\) with adaptive QV and Fixed Stepsize}]\label{thm:projsgd_stronglyconvex_adaptive_complexity}
  For a \(\mu\)-strongly convex \(F : \Lambda \to \mathbb{R}\) on a convex set \(\Lambda\) the last iterate \(\vlambda_T\) of projected SGD with a gradient estimator satisfying an adaptive QV bound (\cref{assumption:adaptiveqvc}) is \(\epsilon\)-close to \(\vlambda^* = \argmin_{\vlambda \in \Lambda} F\left(\vlambda\right)\) such that \(\norm{\vlambda_T - \vlambda^{*}}_2^2 < \epsilon\) if
{%
\setlength{\abovedisplayskip}{.5ex} \setlength{\abovedisplayshortskip}{.5ex}
\setlength{\belowdisplayskip}{1.ex} \setlength{\belowdisplayshortskip}{1.ex}
  \begin{align*}
    \gamma &=
    \min\left(
      \frac{1}{2}
      \frac{
        \mu
      }{
        \widetilde{\alpha} + 2 \widetilde{\beta} \epsilon^{-1} 
      }\, ,\,
      \frac{2}{\mu}
    \right)  \quad\text{and}
    \\
    %
    T &\geq
    \frac{2}{\mu^2} \max\left(\widetilde{\alpha} + 2 \widetilde{\beta} \frac{1}{\epsilon}, \; \frac{\mu^2}{4}\right) \log \left( 2 \norm{\vlambda_0 - \vlambda^*}_2^2 \, \frac{1}{\epsilon} \right).
  \end{align*}
}
\end{theoremEnd}
\vspace{-1ex}
\begin{proofEnd}\label{proof:projsgd_stronglyconvex_adaptive_complexity}
  Recall that, for a stepsize \(\gamma\) and a number of steps \(T\) satisfying 
  \begin{align*}
    \gamma \leq \min\left( \frac{\epsilon \mu}{4 \beta}, \frac{\mu}{2 \alpha}, \frac{2}{\mu} \right)
    \quad\text{and}\quad
    T \geq \max\left( \frac{ 4 \beta }{\mu^2 } \frac{1}{\epsilon}, \frac{2 \alpha}{\mu^2}, \frac{1}{2} \right) \log \left( 2 \norm{\vlambda_0 - \vlambda^*}_2^2 \, \frac{1}{\epsilon} \right),
  \end{align*}
  we can guarantee that the iterate \(\vlambda_t\) can guarantee \(\mathbb{E} \norm{\vlambda^* - \vlambda_T}_2^2 \leq \epsilon\).

  We optimize the parameter \(\delta\) to minimize the number of steps.
  That is,
  \begin{align*}
    \max\left( \frac{ 4 \beta }{\mu^2 } \frac{1}{\epsilon}, \frac{2 \alpha}{\mu^2}, \frac{1}{2} \right) \log \left( 2 \norm{\vlambda_0 - \vlambda^*}_2^2 \, \frac{1}{\epsilon} \right)
    =
    \frac{2}{\mu^2} \max\left( 2 (1 + C^{-1} \delta^{-1}) \,\widetilde{\beta} \frac{1}{\epsilon}, (1 + C \delta) \widetilde{\alpha}, \frac{\mu^2}{4}\right)
    \log \left( 2 \norm{\vlambda_0 - \vlambda^*}_2^2 \, \frac{1}{\epsilon} \right).
  \end{align*}
  Since the first and second arguments of the max function are monotonic with respect to \(\delta\), the optimum is unique, and achieved when the two terms are equal.
  That is,
  \begin{alignat*}{2}
    & &
    2 (1 + C^{-1} \delta^{-1}) \,\widetilde{\beta} \frac{1}{\epsilon}
    &=
    (1 + C \delta) \widetilde{\alpha}
    \\
    &\Leftrightarrow&\qquad
    \frac{2 \widetilde{\beta}}{\epsilon} + \frac{2 \widetilde{\beta} C^{-1}}{\epsilon} \delta^{-1} \,
    &=
    \widetilde{\alpha} + \widetilde{\alpha} C \delta 
    \\
    &\Leftrightarrow&\qquad
    \frac{2 \widetilde{\beta}}{\epsilon} \delta +  \frac{2 \widetilde{\beta} C^{-1}}{\epsilon}  \,
    &=
    \widetilde{\alpha} \delta + \widetilde{\alpha} C \delta^2
    \\
    &\Leftrightarrow&\qquad
    %
    \widetilde{\alpha} C \delta^2 + \left( \widetilde{\alpha} - \frac{2 \widetilde{\beta}}{\epsilon} \right) \delta - \frac{2 \widetilde{\beta} C^{-1}}{\epsilon}
    &=
    0
    \\
    &\Leftrightarrow&\qquad
    %
    \left(
    \widetilde{\alpha} \delta
    - 
    \frac{2 \widetilde{\beta} C^{-1}}{\epsilon}
    \right)
    \left(
      C \delta + 1
    \right)
    &=
    0.
  \end{alignat*}
  Conveniently, we have a unique feasible solution
  \begin{alignat*}{2}
    \delta
    =
    2 
    \frac{
      \widetilde{\beta}
    }{
      \widetilde{\alpha} 
    }
    C^{-1}
    \epsilon^{-1}.
  \end{alignat*}

  Thus, the optimal bound is obtained by setting
  \(
    \delta
    = 2 
    \frac{
      \widetilde{\beta}
    }{
      \widetilde{\alpha} 
    }
    C^{-1}
    \epsilon^{-1},
  \)
  such that
  \begin{align*}
    T
    &\geq
    \frac{2}{\mu^2} \max\left(  2 \beta \frac{1}{\epsilon}, \alpha, \frac{\mu^2}{4} \right) \log \left( 2 \norm{\vlambda_0 - \vlambda^*}_2^2 \, \frac{1}{\epsilon} \right)
    \\
    &=
    \frac{2}{\mu^2} \max\left( 2 \left( 1 + C^{-1} \delta^{-1} \right) \widetilde{\beta}  \frac{1}{\epsilon},  \left( 1 + C \delta \right) \widetilde{\alpha}, \frac{\mu^2}{4} \right)
    \log \left( 2 \norm{\vlambda_0 - \vlambda^*}_2^2 \, \frac{1}{\epsilon} \right)
    \\
    &=
    \frac{2}{\mu^2} \max\left( \widetilde{\alpha} + 2 \widetilde{\beta} \frac{1}{\epsilon}, \frac{\mu^2}{4} \right) \log \left( 2 \norm{\vlambda_0 - \vlambda^*}_2^2 \, \frac{1}{\epsilon} \right).
  \end{align*}
  The stepsize with the optimal \(\delta\) is consequently
  \begin{align*}
    \gamma
    &\leq
    \min\left( \frac{\epsilon \mu}{4 \beta}, \frac{\mu}{2 \alpha}, \frac{2}{\mu} \right)
    =
    \min\left( \frac{\epsilon \mu}{4 (1 + C^{-1} \delta^{-1}) \widetilde{\beta}} \,, \; \frac{\mu}{2 (1 + C \delta) \widetilde{\alpha}}, \frac{2}{\mu} \right)
    =
    \min\left(
      \frac{1}{2}
      \frac{
        \mu
      }{
        \widetilde{\alpha} + 2 \widetilde{\beta} \epsilon^{-1} 
      }\, ,\;
      \frac{2}{\mu}
    \right).
  \end{align*}
\end{proofEnd}


\begin{theoremEnd}[all end, category=complexityprojsgdadaptiveqvcdec]{lemma}[\textbf{Strongly convex \(F\) with adaptive QV and Decreasing Stepsize}]\label{thm:projsgd_stronglyconvex_decstepsize_adaptive_complexity}
  For a \(\mu\)-strongly convex \(F : \Lambda \to \mathbb{R}\) on a convex set \(\Lambda\) with a unique global minimizer \(\vlambda^* \in \Lambda\), the last iterate \(\vlambda_T\) of projected SGD with a gradient estimator satisfying an adaptive QVC bound (\cref{assumption:adaptiveqvc}) and a decreasing stepsize satisfies a suboptimality of \(\norm{\vlambda_T - \vlambda_{*}}_2^2 < \epsilon\) if
  {
  \begin{align*}
    \gamma_t
    &=
    \min\Bigg(
    \frac{
      \mu
    }{
      2 \widetilde{\alpha}
      +
      \sqrt{2 \norm{\vlambda_0 - \vlambda^*}_2 }
      \,
      \epsilon^{1/4}
      \,
      \widetilde{\alpha}^{3/2}
      \widetilde{\beta}^{-1/2}
    },
    \frac{4 t + 2 }{ \mu \, {\left( t + 1\right)}^2 }
    \Bigg)
    \\
    T
    &\geq
    \frac{16 \widetilde{\beta} }{ \mu^2 } \frac{1}{\epsilon} 
    +
    \frac{16 \sqrt{2}}{ \mu^2 }
    \sqrt{
      \norm{\vlambda_0 - \vlambda^*}_2
    }
    \,
    \sqrt{
      \widetilde{\alpha} \widetilde{\beta}
    }
    \,
    \frac{1}{\epsilon^{3/4}}
    +
    \frac{8 \widetilde{\alpha} \, \norm{\vlambda_0 - \vlambda^*}_2 }{ \mu^2}
    \frac{1}{\sqrt{\epsilon}}.
  \end{align*}
  }%
\end{theoremEnd}
\begin{proofEnd}\label{proof:projsgd_stronglyconvex_decstepsize_adaptive_complexity}
  Recall that, for a stepsize \(\gamma\) and a number of steps \(T\) such that
  \begin{align*}
    \gamma_t = \min\left( \frac{\mu}{2 \alpha}, \frac{4 t + 2 }{ \mu \, {\left( t + 1\right)}^2 } \right)
    \quad\text{and}\quad
    T \geq \frac{16 \beta}{ \mu^2 } \frac{1}{\epsilon} + \frac{8  \alpha \, \norm{\vlambda_0 - \vlambda^*}_2 }{ \mu^2} \frac{1}{\sqrt{\epsilon}},
  \end{align*}
  we can guarantee that the iterate \(\vlambda_t\) can guarantee \(\mathbb{E} \norm{\vlambda^* - \vlambda_T}_2^2 \leq \epsilon\).

  We optimize the parameter \(\delta\) to minimize the required number of steps \(T\).
  That is, we maximize
  \begin{align*}
    \frac{16 \beta }{ \mu^2 } \frac{1}{\epsilon} + \frac{8 \sqrt{2} \, \alpha \, \norm{\vlambda_0 - \vlambda^*}_2 }{ \mu^2} \frac{1}{\sqrt{\epsilon}}
    = \frac{16 \left(1 + C \delta\right) \widetilde{\beta} }{ \mu^2 } \frac{1}{\epsilon} + \frac{8 \left(1 + C^{-1} \delta^{-1}\right) \widetilde{\alpha} \, \norm{\vlambda_0 - \vlambda^*}_2 }{ \mu^2} \frac{1}{\sqrt{\epsilon}}.
  \end{align*}
  This is clearly a convex function with respect to \(\delta\).
  Thus, we only need to find a first-order stationary point 
  {\small%
  \begin{align*}
    \frac{\mathrm{d}}{\mathrm{d} \delta}
    \left(
      \frac{16 \left(1 + C \delta\right) \widetilde{\beta} }{ \mu^2 } \frac{1}{\epsilon} + \frac{8 \left(1 + C^{-1} \delta^{-1}\right) \widetilde{\alpha} \, \norm{\vlambda_0 - \vlambda^*}_2 }{ \mu^2} \frac{1}{\sqrt{\epsilon}}
    \right) &= 0.
  \end{align*}
  }%
  Differentiating, we have
  \begin{alignat*}{3}
    &&
    \frac{16 C \widetilde{\beta} }{ \mu^2 } \frac{1}{\epsilon} - \frac{8  \, C^{-1} \delta^{-2} \widetilde{\alpha} \, \norm{\vlambda_0 - \vlambda^*}_2 }{ \mu^2} \frac{1}{\sqrt{\epsilon}}
    &= 0,
\shortintertext{multiplying \(\delta^2\) to both sides,}
    &\Leftrightarrow&\qquad
    \delta^2 \frac{16 C \widetilde{\beta} }{ \mu^2 } \frac{1}{\epsilon} - \frac{8 \sqrt{2} \, C^{-1} \widetilde{\alpha} \, \norm{\vlambda_0 - \vlambda^*}_2 }{ \mu^2} \frac{1}{\sqrt{\epsilon}}
    &= 0.
  \end{alignat*}
  Reorganizing,
  \begin{alignat*}{3}
    &\Leftrightarrow&\qquad
    \delta^2  \frac{16 C \widetilde{\beta} }{ \mu^2 } \frac{1}{\epsilon} 
    &=
    \frac{8 \sqrt{2} \, C^{-1} \widetilde{\alpha} \, \norm{\vlambda_0 - \vlambda^*}_2 }{ \mu^2} \frac{1}{\sqrt{\epsilon}}
    \\
    &\Leftrightarrow&\qquad
    \delta^2  
    &=
    \left( \frac{\mu^2 \epsilon}{ 16 C \widetilde{\beta} } \right)
    \left(
    \frac{8  C^{-1} \widetilde{\alpha} \, \norm{\vlambda_0 - \vlambda^*}_2 }{ \mu^2} \frac{1}{\sqrt{\epsilon}}
    \right)
    \\
    &\Leftrightarrow&\qquad
    \delta^2  
    &=
    \frac{C^{-2} \widetilde{\alpha} \, \norm{\vlambda_0 - \vlambda^*}_2}{ 2 \widetilde{\beta}} \sqrt{\epsilon},
\shortintertext{and taking the square-root of both sides,}
    &\Leftrightarrow&\qquad
    \delta  
    &=
    \frac{
      \sqrt{ \norm{\vlambda_0 - \vlambda^*}_2 }
      \,
      \epsilon^{1/4}
      \,
      \sqrt{\widetilde{\alpha}}
    }{
      \sqrt{2} \,
      C 
      \sqrt{\widetilde{\beta}}
    }.
  \end{alignat*}
  Recall that the required number of iterations is
  \begin{align*}
    T
    &\geq
    \frac{16 \left(1 + C \delta\right) \widetilde{\beta} }{ \mu^2 } \frac{1}{\epsilon} 
    +
      \frac{8 \left(1 + C^{-1} \delta^{-1}\right) \widetilde{\alpha} \, \norm{\vlambda_0 - \vlambda^*}_2 }{ \mu^2} \frac{1}{\sqrt{\epsilon}}
    \\
    &=
    \underbrace{
      \frac{16 \widetilde{\beta} }{ \mu^2 } \frac{1}{\epsilon} 
      +
      \frac{16 \widetilde{\beta} }{ \mu^2 } \frac{1}{\epsilon} 
      C \delta
    }_{T_{\text{\ding{172}}}}
    +
    \underbrace{
      \frac{8 \widetilde{\alpha} \, \norm{\vlambda_0 - \vlambda^*}_2 }{ \mu^2} \frac{1}{\sqrt{\epsilon}}
      +
      \frac{8 \widetilde{\alpha} \, \norm{\vlambda_0 - \vlambda^*}_2 }{ \mu^2} \frac{1}{\sqrt{\epsilon}}
      C^{-1} \delta^{-1}
    }_{T_{\text{\ding{173}}}}.
  \end{align*}
  Plugging \(\delta\) in, we have
  \begin{align*}
    T_{\text{\ding{172}}}
    &=
    \frac{16 \widetilde{\beta} }{ \mu^2 } \frac{1}{\epsilon} 
    +
    \frac{16 \widetilde{\beta} }{ \mu^2 } \frac{1}{\epsilon}
    C 
    \left(
    \frac{
      \sqrt{ \norm{\vlambda_0 - \vlambda^*}_2 }
      \,
      \epsilon^{1/4}
      \,
      \sqrt{\widetilde{\alpha}}
    }{
      \sqrt{2} \, C
      \sqrt{\widetilde{\beta}}
    }
    \right)
    \\
    &=
    \frac{16 \widetilde{\beta} }{ \mu^2 } \frac{1}{\epsilon} 
    +
    \frac{8 \sqrt{2} }{ \mu^2 }
    \sqrt{
      \widetilde{\alpha} \widetilde{\beta}
    } \,
    \sqrt{ \norm{\vlambda_0 - \vlambda^*}_2 }
    \,
    \epsilon^{-3/4}
    \\
    \\
    T_{\text{\ding{173}}}
    &=
    \frac{8 \, \widetilde{\alpha} \, \norm{\vlambda_0 - \vlambda^*}_2 }{ \mu^2} \frac{1}{\sqrt{\epsilon}}
    +
    \frac{8 \, \widetilde{\alpha} \, \norm{\vlambda_0 - \vlambda^*}_2 }{ \mu^2} \frac{1}{\sqrt{\epsilon}}
    C^{-1} 
    \left(
    \frac{
      \sqrt{2} \, C
      \sqrt{\widetilde{\beta}}
    }{
      \sqrt{ \norm{\vlambda_0 - \vlambda^*}_2 }
      \,
      \epsilon^{1/4}
      \,
      \sqrt{\widetilde{\alpha}}
    }
    \right)
    \\
    &=
    \frac{8 \widetilde{\alpha} \, \norm{\vlambda_0 - \vlambda^*}_2 }{ \mu^2} \frac{1}{\sqrt{\epsilon}}
    +
    \frac{8 \sqrt{2}}{ \mu^2}
    \sqrt{ \norm{\vlambda_0 - \vlambda^*}_2 }
    \,
    \sqrt{
      \widetilde{\alpha}
      \widetilde{\beta}
    }
    \,
    \epsilon^{-3/4}.
  \end{align*}
  Combining the results, 
  \begin{align*}
    T
    \geq
    T_{\text{\ding{172}}}
    +
    T_{\text{\ding{173}}}
    &=
    \frac{16 \widetilde{\beta} }{ \mu^2 } \frac{1}{\epsilon} 
    +
    \frac{8 \sqrt{2} }{ \mu^2 }
    \sqrt{
      \widetilde{\alpha} \widetilde{\beta}
    } \,
    \sqrt{ \norm{\vlambda_0 - \vlambda^*}_2 }
    \,
    \epsilon^{-3/4}
    \\
    &\qquad+
    \frac{8 \widetilde{\alpha} \, \norm{\vlambda_0 - \vlambda^*}_2 }{ \mu^2} \frac{1}{\sqrt{\epsilon}}
    +
    \frac{8 \sqrt{2}}{ \mu^2}
    \sqrt{ \norm{\vlambda_0 - \vlambda^*}_2 }
    \,
    \sqrt{
      \widetilde{\alpha}
      \widetilde{\beta}
    }
    \,
    \epsilon^{-3/4}
    \\
    &=
    \frac{16 \widetilde{\beta} }{ \mu^2 } \frac{1}{\epsilon} 
    +
    \frac{16 \sqrt{2}}{ \mu^2 }
    \sqrt{
      \norm{\vlambda_0 - \vlambda^*}_2
    }
    \,
    \sqrt{
      \widetilde{\alpha} \widetilde{\beta}
    }
    \,
    \epsilon^{-3/4}
    +
    \frac{8 \widetilde{\alpha} \, \norm{\vlambda_0 - \vlambda^*}_2 }{ \mu^2}
    \frac{1}{\sqrt{\epsilon}}.
  \end{align*}

  For the stepsize
  \begin{align*}
    \gamma
    =
    \min\left( \frac{\mu}{2 \alpha}, \frac{4 t + 2 }{ \mu \, {\left( t + 1\right)}^2 } \right)
    =
    \min\left(
      \frac{\mu}{2 \left(1 + C \delta \right) \widetilde{\alpha}},
      \frac{4 t + 2 }{ \mu \, {\left( t + 1\right)}^2 }
    \right),
  \end{align*}
  we have 
  \begin{align*}
    2 \left(1 + C \delta \right) \widetilde{\alpha}
    &=
    2 \widetilde{\alpha} + 2 \widetilde{\alpha} C \delta 
    \\
    &=
    2 \widetilde{\alpha}
    +
    2 \widetilde{\alpha}
    C
    \left(
    \frac{
      \sqrt{ \norm{\vlambda_0 - \vlambda^*}_2 }
      \,
      \epsilon^{1/4}
      \,
      \sqrt{\widetilde{\alpha}}
    }{
      \sqrt{2} \, C
      \sqrt{\widetilde{\beta}}
    }
    \right)
    \\
    &=
    2 \widetilde{\alpha}
    +
    \sqrt{2}
    \sqrt{ \norm{\vlambda_0 - \vlambda^*}_2 }
    \,
    \epsilon^{1/4}
    \,
    \widetilde{\alpha}^{3/2}
    \widetilde{\beta}^{-1/2}.
  \end{align*}
  Therefore, 
  \begin{align*}
    \gamma
    =
    \min\left(
      \frac{\mu}{2 \left(1 + C \delta \right) \widetilde{\alpha}},
      \frac{4 t + 2 }{ \mu \, {\left( t + 1\right)}^2 }
    \right)
    =
    \min\left(
    \frac{
      \mu
    }{
      2 \widetilde{\alpha}
      +
      \sqrt{2 \norm{\vlambda_0 - \vlambda^*}_2 }
      \,
      \epsilon^{1/4}
      \,
      \widetilde{\alpha}^{3/2}
      \widetilde{\beta}^{-1/2}
    },
    \frac{4 t + 2 }{ \mu \, {\left( t + 1\right)}^2 }
    \right).
  \end{align*}
\end{proofEnd}


\paragraph{Complexity of BBVI on Strongly-Log-Concave \(\pi\)}
We can now plug in the constants obtained in \cref{section:gradient_variance}.
This immediately establishes the iteration complexity resulting from the use of different gradient estimators.
\vspace{0ex}
%

\begin{theoremEnd}[category=complexitybbvicfefixed]{theorem}[\textbf{Complexity of Fixed Stepsize BBVI with CFE}]\label{thm:projsgd_bbvicfe_complexity}
  The last iterate \(\vlambda_T \in \Lambda_L\) of BBVI with the CFE estimator and projected SGD with a fixed stepsize applied to a \(\mu\)-strongly log-concave and \(L\)-log-smooth posterior is \(\epsilon\)-close to \(\vlambda^* = \argmin_{\vlambda \in \Lambda_L} F\left(\vlambda\right)\) such that \(\mathbb{E}\norm{ \vlambda_T - \vlambda^* }_2^2 \leq \epsilon\) if
{%\small%
\setlength{\abovedisplayskip}{.5ex} \setlength{\abovedisplayshortskip}{.5ex}
\setlength{\belowdisplayskip}{1.ex} \setlength{\belowdisplayshortskip}{1.ex}
  \begin{align*}
    T
    &\geq 
    2 \kappa^2 \left(d + k_{\varphi} + 4\right) \left(1  + 2 {\lVert \bar{\vlambda} - \vlambda^* \rVert}_2^2 \frac{1}{\epsilon}\right) 
    \log \left( 2 \Delta^2 \, \frac{1}{\epsilon} \right)
  \end{align*}
}%
  for some fixed stepsize \(\gamma\), where \(\Delta = {\lVert \vlambda_0 - \vlambda^*\rVert}_2\), and \(\kappa = L/\mu\) is the condition number.
\end{theoremEnd}
\vspace{-1ex}
\begin{proofEnd}\label{proof:projsgd_bbvicfe_complexity}
  From \cref{thm:cfe_upperbound} with \(S = L\), the CFE estimator satisfies adaptive QV with the constants
  \begin{alignat*}{2}
    \alpha_{\mathrm{CFE}} 
    = L^2 \left( d + k_{\varphi} + 4 \right) \left(1 + \delta\right)
    \qquad\text{and}\qquad
    \beta_{\mathrm{CFE}}  
    = L^2 \left( d + k_{\varphi} \right) \left( 1 + \delta^{-1}  \right) {\lVert \bar{\vlambda} - \vlambda^* \rVert}_2^2.
  \end{alignat*}
  Furthermore, for a \(\mu\)-strongly log-concave posterior and our variational parameterization,~\citet[Theorem 9]{domke_provable_2020} show that the ELBO is \(\mu\)-strongly convex.

  We can thus invoke \cref{thm:projsgd_stronglyconvex_adaptive_complexity} with 
  \begin{align*}
    \widetilde{\alpha} = L^2 \left(d + k_{\varphi} + 4\right),\qquad
    \widetilde{\beta}  = L^2 \left(d + k_{\varphi}\right) {\lVert \bar{\vlambda} - \vlambda^* \rVert}_2^2,\quad\text{and}\quad
    C = 1.
  \end{align*}
  This yields a lower bound on the number of iteration 
  \begin{align*}
    &\frac{2}{\mu^2} \max\left(\widetilde{\alpha} + 2 \widetilde{\beta} \frac{1}{\epsilon}, \; \frac{\mu^2}{4} \right) 
    \log \left( 2 \norm{\vlambda_0 - \vlambda^*}_2^2 \, \frac{1}{\epsilon} \right)
    \\
    &\;=
    \frac{2}{\mu^2} \max\left(L^2 \left(d + k_{\varphi} + 4\right) + 2 L^2 \left(d + k_{\varphi}\right) {\lVert \bar{\vlambda} - \vlambda^* \rVert}_2^2 \frac{1}{\epsilon},\; \frac{\mu^2}{4}\right) 
    \log \left( 2 {\lVert \vlambda_0 - \vlambda^*\rVert}_2^2 \, \frac{1}{\epsilon} \right),
\shortintertext{pulling out \(L\),}
    &\;=
    \frac{2 L^2}{\mu^2} \max\left( \left(d + k_{\varphi} + 4\right) + 2 \left(d + k_{\varphi}\right) {\lVert \bar{\vlambda} - \vlambda^* \rVert}_2^2 \frac{1}{\epsilon},\; \frac{\mu^2}{4 L^2}\right) 
    \log \left( 2 {\lVert \vlambda_0 - \vlambda^*\rVert}_2^2 \, \frac{1}{\epsilon} \right),
\shortintertext{and since \(\frac{\mu^2}{4 L^2} < \frac{1}{4}\) and the first argument is larger than 1, the max operation is redundant that}
    &\;=
    \frac{2 L^2}{\mu^2} \left( \left(d + k_{\varphi} + 4\right) + 2 \left(d + k_{\varphi}\right) {\lVert \bar{\vlambda} - \vlambda^* \rVert}_2^2 \frac{1}{\epsilon}\right) 
    \log \left( 2 {\lVert \vlambda_0 - \vlambda^*\rVert}_2^2 \, \frac{1}{\epsilon} \right).
\shortintertext{Now, using the trivial fact \(d + k_{\varphi} < d + k_{\varphi} + 4\) simplifies the bound as,}
    &\;<
    \frac{2 L^2}{\mu^2} \left(d + k_{\varphi} + 4\right) \left(1  + 2 {\lVert \bar{\vlambda} - \vlambda^* \rVert}_2^2 \frac{1}{\epsilon} \right) 
    \log \left( 2 {\lVert \vlambda_0 - \vlambda^*\rVert}_2^2 \, \frac{1}{\epsilon} \right)
    \\
    &\;=
    2 \kappa^2 \left(d + k_{\varphi} + 4\right) \left(1  + 2 {\lVert \bar{\vlambda} - \vlambda^* \rVert}_2^2 \frac{1}{\epsilon}\right) 
    \log \left( 2 {\lVert \vlambda_0 - \vlambda^*\rVert}_2^2 \, \frac{1}{\epsilon} \right).
  \end{align*}
  The optimal \(\delta\) is given as
  \begin{align*}
    \delta 
    = \frac{2}{\epsilon} \frac{\widetilde{\beta}}{\widetilde{\alpha}} C^{-1} 
    = \frac{2}{\epsilon} 
    \frac{
      L^2 \left(d + k_{\varphi}\right) {\lVert \bar{\vlambda} - \vlambda^* \rVert}_2^2
    }{
      L^2 \left(d + k_{\varphi} + 4\right)
    } 
    C^{-1} 
    =
    \frac{2}{\epsilon} \, \frac{d + k_{\varphi}}{d + k_{\varphi} + 4} \, {\lVert \bar{\vlambda} - \vlambda^* \rVert}_2^2 .
  \end{align*}
\end{proofEnd}

\begin{theoremEnd}[all end, category=complexitybbvicfedec]{theorem}[\textbf{Complexity of Decreasing Stepsize BBVI with CFE}]\label{thm:projsgd_bbvicfe_decstepsize_complexity}
  The last iterate \(\vlambda_T \in \Lambda_L\) of BBVI with the CFE estimator and projected SGD with a decreasing stepsize schedule applied to a \(\mu\)-strongly log-concave and \(L\)-log-smooth posterior is \(\epsilon\)-close to \(\vlambda^* = \argmin_{\vlambda \in \Lambda_L} F\left(\vlambda\right) \) such that \(\mathbb{E}\norm{ \vlambda_T - \vlambda^* }_2^2 \leq \epsilon\) if
  \begin{align*}
    T
    &\geq 
    16 \kappa^2 \left(d + k_{\varphi} + 4\right) 
    \bigg(
    {\lVert \bar{\vlambda} - \vlambda^* \rVert}_2^2 \frac{1}{\epsilon} 
    +
    2
    \sqrt{
      \norm{\vlambda_0 - \vlambda^*}_2
    }
    \,
    {\lVert \bar{\vlambda} - \vlambda^* \rVert}_2
    \,
    \frac{1}{\epsilon^{3/4}}
    +
    \norm{\vlambda_0 - \vlambda^*}_2 
    \frac{1}{\sqrt{\epsilon}}
    \bigg).
  \end{align*}
  for some decreasing stepsize schedule \(\gamma_1, \ldots, \gamma_T\), where \(\kappa = L/\mu\) is the condition number and \(\vlambda^* \in \Lambda\) is the optimal variational parameter.
\end{theoremEnd}
\begin{proofEnd}\label{proof:projsgd_bbvicfe_decstepsize_complexity}
  From \cref{thm:cfe_upperbound}, the CFE estimator with \(S = L\) satisfies adaptive QV with the constants
  \begin{align*}
    \alpha_{\mathrm{CFE}} 
    = L^2 \left( d + k_{\varphi} + 4 \right) \left(1 + \delta\right)\qquad\text{and}\qquad
    %
    \beta_{\mathrm{CFE}}  
    = L^2 \left( d + k_{\varphi} \right) \left( 1 + \delta^{-1}  \right) {\lVert \bar{\vlambda} - \vlambda^* \rVert}_2^2.
  \end{align*}
  Furthermore, for a \(\mu\)-strongly log-concave posterior and our variational parameterization,~\citet[Theorem 9]{domke_provable_2020} show that the ELBO is \(\mu\)-strongly convex.

  We thus invoke \cref{thm:projsgd_stronglyconvex_decstepsize_adaptive_complexity} with 
  \begin{alignat*}{3}
    \widetilde{\alpha} = L^2 \left(d + k_{\varphi} + 4\right),
    \qquad\quad
    \widetilde{\beta}  = L^2 \left(d + k_{\varphi}\right) {\lVert \bar{\vlambda} - \vlambda^* \rVert}_2^2, \qquad\text{and}
    \qquad
    C = 1.
  \end{alignat*}
  This yields a lower bound on the number of iterations:
  \begin{align*}
    &\frac{16 \widetilde{\beta} }{ \mu^2 } \frac{1}{\epsilon} 
    +
    \frac{16 \sqrt{2}}{ \mu^2 }
    \sqrt{
      \norm{\vlambda_0 - \vlambda^*}_2
    }
    \,
    \sqrt{
      \widetilde{\alpha} \widetilde{\beta}
    }
    \,
    \frac{1}{\epsilon^{3/4}}
    +
    \frac{8 \widetilde{\alpha} \, \norm{\vlambda_0 - \vlambda^*}_2 }{ \mu^2}
    \frac{1}{\sqrt{\epsilon}}
    \\
    &=
    \frac{16 L^2 \left(d + k_{\varphi}\right) {\lVert \bar{\vlambda} - \vlambda^* \rVert}_2^2 }{ \mu^2 } \frac{1}{\epsilon} 
    +
    \frac{16 \sqrt{2}}{ \mu^2 }
    \sqrt{
      \norm{\vlambda_0 - \vlambda^*}_2
    }
    \,
    \sqrt{
       \left( L^2 \left(d + k_{\varphi} + 4\right) \right)
       \left( L^2 \left(d + k_{\varphi}\right) {\lVert \bar{\vlambda} - \vlambda^* \rVert}_2^2 \right)
    }
    \,
    \frac{1}{\epsilon^{3/4}}
    \\
    &\qquad+
    \frac{8 L^2 \left(d + k_{\varphi} + 4\right) \, \norm{\vlambda_0 - \vlambda^*}_2 }{ \mu^2}
    \frac{1}{\sqrt{\epsilon}},
\shortintertext{using the trivial bound \(d + k_{\varphi} < d + k_{\varphi} + 4\),}
    &<
    \frac{16 L^2 \left(d + k_{\varphi} + 4\right) {\lVert \bar{\vlambda} - \vlambda^* \rVert}_2^2 }{ \mu^2 } \frac{1}{\epsilon} 
    +
    \frac{16 \sqrt{2}}{ \mu^2 }
    \sqrt{
      \norm{\vlambda_0 - \vlambda^*}_2
    }
    \,
    \sqrt{
       L^4 \left(d + k_{\varphi} + 4\right)
       \left(d + k_{\varphi} + 4\right) {\lVert \bar{\vlambda} - \vlambda^* \rVert}_2^2 
    }
    \,
    \frac{1}{\epsilon^{3/4}}
    \\
    &\qquad+
    \frac{8 L^2 \left(d + k_{\varphi} + 4\right) \, \norm{\vlambda_0 - \vlambda^*}_2 }{ \mu^2}
    \frac{1}{\sqrt{\epsilon}},
\shortintertext{pulling out the \(16 \left( d + k_{\varphi} + 4 \right) L^2 / \mu^2 \) factors,}
    &=
    16 \left(d + k_{\varphi} + 4\right) 
    \frac{L^2}{ \mu^2 } 
    \bigg(
    {\lVert \bar{\vlambda} - \vlambda^* \rVert}_2^2 \frac{1}{\epsilon} 
    +
    \sqrt{2}
    \sqrt{
      \norm{\vlambda_0 - \vlambda^*}_2
    }
    \,
    {\lVert \bar{\vlambda} - \vlambda^* \rVert}_2
    \,
    \frac{1}{\epsilon^{3/4}}
    +
    \frac{1}{2}
    \norm{\vlambda_0 - \vlambda^*}_2 
    \frac{1}{\sqrt{\epsilon}}
    \bigg)
    \\
    &=
    16 \kappa^2 \left(d + k_{\varphi} + 4\right) 
    \bigg(
    {\lVert \bar{\vlambda} - \vlambda^* \rVert}_2^2 \frac{1}{\epsilon} 
    +
    \sqrt{2}
    \sqrt{
      \norm{\vlambda_0 - \vlambda^*}_2
    }
    \,
    {\lVert \bar{\vlambda} - \vlambda^* \rVert}_2
    \,
    \frac{1}{\epsilon^{3/4}}
    +
    \frac{1}{2}
    \norm{\vlambda_0 - \vlambda^*}_2 
    \frac{1}{\sqrt{\epsilon}}
    \bigg).
  \end{align*}
  The optimal \(\delta\) is given as
  \begin{align*}
    \delta
    &=
    \frac{
      \sqrt{ \norm{\vlambda_0 - \vlambda^*}_2 }
      \,
      \epsilon^{1/4}
      \,
      \sqrt{\widetilde{\alpha}}
    }{
      \sqrt{2} \, C
      \sqrt{\widetilde{\beta}}
    }
    =
    \frac{
      \sqrt{ \norm{\vlambda_0 - \vlambda^*}_2 }
      \,
      \epsilon^{1/4}
      \,
      \sqrt{ L^2 \left(d + k_{\varphi} + 4\right) }
    }{
      \sqrt{2} 
      \sqrt{ L^2 \left(d + k_{\varphi}\right) {\lVert \bar{\vlambda} - \vlambda^* \rVert}_2^2 }
    }
    =
    \frac{1}{\sqrt{2} } \,
    \frac{\sqrt{ \norm{\vlambda_0 - \vlambda^*}_2 }}{{\lVert \bar{\vlambda} - \vlambda^* \rVert}_2}
    \,
    \sqrt{
    \frac{
       d + k_{\varphi} + 4
    }{
      d + k_{\varphi}
    }
    } \, \epsilon^{-1/4}.
  \end{align*}
\end{proofEnd}

\vspace{1.ex}

In particular, the following theorem establishes that BBVI with the STL estimator can achieve linear convergence under perfect variational family specification.
\vspace{0ex}
\input{thm_projsgd_bbvistl_stronglyconvex_complexity}

\vspace{1ex}
\begin{corollary}[\textbf{Linear Convergence of BBVI with STL}]
  If the variational family is perfectly specified such that \( \mathrm{D}_{\mathrm{F}^4}\left(q_{\vlambda}^*, \pi\right) = 0\) for \(\vlambda^* = \argmin_{\vlambda \in \Lambda_L} F\left(\vlambda\right)\), then BBVI with the STL estimator converges linearly with a complexity of \(\mathcal{O}\left(d \kappa^2 \log \left( 1 / \epsilon \right) \right)\).
\end{corollary}

% \begin{remark}
%   The overall complexity of the STL estimator is larger by a constant factor of 4 compared to that of CFE.
%   This means that the STL estimator will converge slowly for a large target \(\epsilon\).
%   As mentioned in \cref{remark:stl_tightness,remark:variance_comparison}, a factor of 2 difference would be more realistic and is closer to what is observed in our simulations.
% \end{remark}

\vspace{1ex}
\begin{remark}
  Convergence is slowed when using a decreasing step size schedule, as shown in \cref{thm:projsgd_bbvistl_decstepsize_complexity}.
  Thus, one does not achieve a linear convergence rate under this schedule even if the variational family is perfectly specified. 
  However, when the variational family is misspecified, this achieves a better rate of \(\mathcal{O}\left(1/\epsilon\right)\) compared to the \(\mathcal{O}\left(1/\epsilon \log 1/\epsilon\right)\) of \cref{thm:projsgd_bbvistl_complexity}.
\end{remark}

\vspace{1ex}
\begin{remark}[\textbf{Variational Family Misspecification}]\label{remark:misspecification}
  Under variational family misspecification, STL has an \(\mathcal{O}\left(1/\epsilon\right)\) dependence on the 4th order Fisher divergence \(\mathrm{D}_{\mathrm{F}^4}\left(q_{\vlambda^*}, \pi\right) > 0\).
  To compare the computational performance of CFE and STL in this setting, one needs to compare \( L^{-2} \sqrt{\mathrm{D}_{\mathrm{F}^4}\left(q_{\vlambda^*}, \pi\right)}\) versus \({\lVert \bar{\vlambda} - \vlambda^* \rVert}_2^2\).
\end{remark}

\vspace{1ex}
\begin{remark}
  \cref{thm:stl_upperbound_mf} also implies that the mean-field parameterization improves the dimension dependence to a complexity of \(\mathcal{O}\left( \sqrt{d} \kappa^2 \log \left( 1 / \epsilon \right) \right)\).
\end{remark}

%% \paragraph{Complexity of BBVI on Log-Concave Posteriors}
%% Lastly, we present the complexity of BBVI with the STL estimator when \(\pi\) is only log-concave.
%% Since the complexity guarantee of \cref{thm:projsgd_convex_fixedstepsize} is too nonlinear with respect to \(\beta\), we present the case where the variational family is well specified (\(\beta_{\mathrm{STL}} = 0\)).
%% In this setting, the STL estimator achieves a better complexity than that of the CFE estimator.

%% 
\begin{theoremEnd}[category=complexitybbvistl]{theorem}[\textbf{Complexity of Fixed Stepsize BBVI with STL}]\label{thm:projsgd_bbvistl_decstepsize_complexity_logconcave}
  For some weighted iterate averaging scheme, the last average \(\bar{\vlambda}_{T}\) of BBVI with the STL estimator and projected SGD with a fixed stepsize schedule applied to a log-concave and \(L\)-log-smooth posterior under perfect variational family specification satisfy \(F\left(\bar{\vlambda}_T\right) - F\left(\vlambda^*\right) \leq \epsilon\), where \(\vlambda^* \in \argmin_{\vlambda \in \Lambda_L} F\left(\vlambda\right) \) if
  \begin{align*}
    T
    \geq
    \sqrt{2 \left(d + k_{\varphi}\right)} \, L 
    \, \norm{\vlambda_0 - \vlambda^*}_2^2
    \frac{1}{\epsilon},
  \end{align*}
  for some fixed stepsize \(\gamma\).
\end{theoremEnd}
\begin{proofEnd}
  As shown by \cref{thm:stl_upperbound}, the STL estimator satisfies an adative QV bound with the constants
  \begin{align*}
    \alpha_{\mathrm{STL}} &= 2 \left(d + k_{\varphi}\right) \left(2 + \delta\right)  L^2 \\
    \beta_{\mathrm{STL}}  &= \sqrt{3} \left(d + k_{\varphi}\right) \left(1 + 2 \delta^{-1}\right)  \sqrt{\mathrm{D}_{\mathrm{F}^4}\left(q_{\vlambda^*}, \pi\right)}.
  \end{align*}
  Also, when the variational family is perfectly specified, the variational posterior \(q_{\vlambda^*}\) is equal to \(\pi\) such that 
  \[
  \mathrm{D}_{\mathrm{F}^4}\left(q_{\vlambda^*}, \pi\right) = 0.
  \]
  Then, we immediately have \(\beta_{\text{STL}} = 0\).
  Therefore, the optimal \(\delta\) is \(\delta = 0\), resulting in 
  \begin{align*}
    \alpha_{\mathrm{STL}} &= 4 \left(d + k_{\varphi}\right) L^2 \\
    \beta_{\mathrm{STL}}  &= 0.
  \end{align*}
  Furthermore, for a log-concave posterior and our variational parameterization,~\citet[Theorem 9]{domke_provable_2020} show that the ELBO is convex.

  Thus, we can invoke \cref{thm:projsgd_convex_fixedstepsize}, for which we have a lower bound on the required number of iteration:
  \begin{align*}
    \frac{2 \alpha}{ -\beta + \sqrt{ \beta^2 + 8 \epsilon^2 \alpha } }
    \, \norm{\vlambda_0 - \vlambda^*}_2^2
    &=
    \frac{2 \alpha}{ \sqrt{ 8 \epsilon^2 \alpha } }
    \, \norm{\vlambda_0 - \vlambda^*}_2^2
    \\
    &=
    \frac{\sqrt{\alpha}}{\sqrt{2}}
    \, \norm{\vlambda_0 - \vlambda^*}_2^2
    \frac{1}{\epsilon}
    \\
    &=
    \frac{\sqrt{4 \left(d + k_{\varphi}\right) L^2}}{\sqrt{2}}
    \, \norm{\vlambda_0 - \vlambda^*}_2^2
    \frac{1}{\epsilon}
    \\
    &=
    \sqrt{2 \left(d + k_{\varphi}\right)} \, L 
    \, \norm{\vlambda_0 - \vlambda^*}_2^2
    \frac{1}{\epsilon}.
  \end{align*}
\end{proofEnd}


%% \subsection{Variational Family Misspecification and the STL Estimator}

%% As mentioned in \cref{remark:misspecification}, comparing the performance of the STL estimator against the CFE estimator involves the 4th-order Fisher divergence and \({\lVert \bar{\vlambda} - \vlambda^* \rVert}_2^2\).
%% Unfortunately, the relationship between the two quantities is not obvious.
%% Because of this, we further reduce our scope to Gaussians in order to gather more intuition.

%% Recall that term \(T_{\ding{184}}\) in \cref{thm:stl_decomposition} captures the effect of variational family specification on the STL estimator.
%% Assuming both the posterior and variational approximation are Gaussians, we can obtain a simpler result on \(T_{\ding{184}}\).

\subsection{Should we stick the landing?}
When the variational family is misspecified, it is hard to tell \textit{when} STL would be superior to CFE; the Fisher-Hyv\"arinen divergence and the posterior variance are fundamentally unrelated quantities.
Furthermore, the Fisher-Hyv\"arinen divergence is hard to interpret apart from some relationships with other divergences~\citep{huggins_practical_2018}.
Thus, we conclude by providing a characterization of the Fisher-Hyv\"arinen divergence.

Our final analysis will focus on Gaussian posteriors and the mean-field Gaussian family.
In practice, the STL estimator becomes infeasible to use with full-rank variational families as each evaluation of the log-density \(\log q_{\vlambda}\) involves a back-substitution with a \(\mathcal{O}\left(d^3\right)\) cost and numerical stability becomes a concern.
Therefore, studying the effect of misspecification of mean-field is particularly relevant.

\vspace{1ex}

\begin{theoremEnd}[category=stlgaussian]{lemma}\label{thm:gaussian_fisher_divergence}
  For \(\pi = \mathcal{N}\left(\vmu, \mSigma\right)\) and \(q = \mathcal{N}\left(\vm, \mC \mC^{\top} \right)\), the Fisher-Hyv\"arinen divergence is
  \[
    \DHF{q}{\pi}
    =
    {\lVert \mSigma^{-1} \mC - \mC^{-\top} \rVert}_{\mathrm{F}}^2
    +
    {\lVert \mSigma^{-1}\left(\vm  - \vmu\right) \rVert}_{2}^2.
  \]
\end{theoremEnd}
\begin{proofEnd}
  \begin{align*}
    \DHF{q}{\pi}
    &=
    \mathbb{E}_{\rvvz \sim q} \norm{ \nabla \log \pi\left(\rvvz\right) - \nabla \log q\left(\rvvz\right) }_{2}^2
    \\
    &=
    \mathbb{E} \norm{ \mSigma^{-1} \left(\mC \rvvu + \vm  - \vmu\right) - {\left( \mC\mC^{\top} \right)}^{-1} \left(\mC \rvvu + \vm - \vm\right) }_{2}^2
    \\
    &=
    \mathbb{E} \norm{ \mSigma^{-1} \left(\mC \rvvu + \vm  - \vmu\right) - {\left( \mC\mC^{\top} \right)}^{-1} \mC \rvvu }_{2}^2
    \\
    &=
    \mathbb{E} \norm{ \mSigma^{-1} \left(\mC \rvvu + \vm  - \vmu\right) - \mC^{-\top} \rvvu }_{2}^2
    \\
    &=
    \mathbb{E} \norm{ \left( \mSigma^{-1} \mC - \mC^{-\top} \right) \rvvu  + \mSigma^{-1}\left(\vm  - \vmu\right) }_{2}^2
    \\
    &=
    \mathbb{E} \norm{ \left( \mSigma^{-1} \mC - \mC^{-\top} \right) \rvvu }_2^2
    +
    2 \inner{ \left( \mSigma^{-1} \mC - \mC^{-\top} \right) \mathbb{E} \rvvu }{ \mSigma^{-1}\left(\vm  - \vmu\right) }
    +
    {\lVert \mSigma^{-1}\left(\vm  - \vmu\right) \rVert}_{2}^2
    \\
    &=
    \mathbb{E} \norm{ \left( \mSigma^{-1} \mC - \mC^{-\top} \right) \rvvu }_2^2
    +
    {\lVert \mSigma^{-1}\left(\vm  - \vmu\right) \rVert}_{2}^2
  \end{align*}

  Finally,
  \begin{align*}
    \mathbb{E} \norm{ \left( \mSigma^{-1} \mC - \mC^{-\top} \right) \rvvu }_2^2
    &=
    \mathbb{E} \mathrm{tr} \left( \rvvu^{\top} {\left( \mSigma^{-1} \mC - \mC^{-\top} \right)}^{\top} \left( \mSigma^{-1} \mC - \mC^{-\top} \right) \rvvu \right)
    \\
    &=
    \mathrm{tr} \left( {\left( \mSigma^{-1} \mC - \mC^{-\top} \right)}^{\top} \left( \mSigma^{-1} \mC - \mC^{-\top} \right) \mathbb{E} \rvvu \rvvu^{\top} \right)
    \\
    &=
    \mathrm{tr} \left( {\left( \mSigma^{-1} \mC - \mC^{-\top} \right)}^{\top} \left( \mSigma^{-1} \mC - \mC^{-\top} \right) \right)
    \\
    &=
    \norm{ \mSigma^{-1} \mC - \mC^{-\top} }_{\mathrm{F}}^2.
  \end{align*}
\end{proofEnd}

\begin{theoremEnd}[category=stlgaussian]{lemma}
  For \(\pi = \mathcal{N}\left(\vmu, \mSigma\right)\) and \(q = \mathcal{N}\left(\vm, \mC \mC^{\top} \right)\), where \(\vmu = \vm\), the Fisher-Hyv\"arinen divergence is bounded above and below as
  \[
    {\lambda_{\mathrm{max}}\left(\mC\right)}^{-2} {\lVert \mC^{\top} \mSigma^{-1} \mC - \boldupright{I} \rVert}_{\mathrm{F}}^2
    \leq
    \DHF{q}{\pi}
    \leq
    {\lambda_{\mathrm{min}}\left(\mC\right)}^{-2} {\lVert \mC^{\top} \mSigma^{-1} \mC - \boldupright{I} \rVert}_{\mathrm{F}}^2.
  \]
\end{theoremEnd}
\begin{proofEnd}
  By assumption,
  \begin{align}
    \DHF{q}{\pi}
    =
    {\lVert \mSigma^{-1} \mC - \mC^{-\top} \rVert}_{\mathrm{F}}^2
    +
    {\lVert \mSigma^{-1}\left(\vm  - \vmu\right) \rVert}_{2}^2
    =
    {\lVert \mSigma^{-1} \mC - \mC^{-\top} \rVert}_{\mathrm{F}}^2.
    \label{eq:stl_gaussian_sandwich_1}
  \end{align}

  Also, we can pull our a \(\mC^{-\top}\) factor as
  \begin{align}
    {\lVert \mSigma^{-1} \mC - \mC^{-\top} \rVert}_{\mathrm{F}}^2
    =
    {\lVert \mC^{-\top} \left( \mC^{\top} \mSigma^{-1} \mC - \boldupright{I} \right) \rVert}_{\mathrm{F}}^2.
    \label{eq:stl_gaussian_sandwich_2}
  \end{align}

  By the property of the Frobenius norm,
  \begin{alignat*}{4}
    & &\quad
    {\lambda_{\mathrm{min}}\left(\mC^{-\top}\right)}^2 {\lVert \mC^{\top} \mSigma^{-1} \mC - \boldupright{I} \rVert}_{\mathrm{F}}^2
    &\leq
    \, {\lVert \mC^{-\top} \left( \mC^{\top} \mSigma^{-1} \mC - \boldupright{I} \right) \rVert}_{\mathrm{F}}^2 \;
    &\leq
    {\lambda_{\mathrm{max}}\left(\mC^{-\top}\right)}^2 {\lVert \mC^{\top} \mSigma^{-1} \mC - \boldupright{I} \rVert}_{\mathrm{F}}^2,
\shortintertext{inverting the singular values,}
    &\Leftrightarrow&\quad
    {\lambda_{\mathrm{max}}\left(\mC\right)}^{-2} {\lVert \mC^{\top} \mSigma^{-1} \mC - \boldupright{I} \rVert}_{\mathrm{F}}^2
    &\leq
    \,{\lVert \mC^{-\top} \left( \mC^{\top} \mSigma^{-1} \mC - \boldupright{I} \right) \rVert}_{\mathrm{F}}^2 \,
    &\leq
    {\lambda_{\mathrm{min}}\left(\mC\right)}^{-2} {\lVert \mC^{\top} \mSigma^{-1} \mC - \boldupright{I} \rVert}_{\mathrm{F}}^2,
\shortintertext{and by \cref{eq:stl_gaussian_sandwich_1,eq:stl_gaussian_sandwich_2},}
    &\Leftrightarrow&\quad
    {\lambda_{\mathrm{max}}\left(\mC\right)}^{-2} {\lVert \mC^{\top} \mSigma^{-1} \mC - \boldupright{I} \rVert}_{\mathrm{F}}^2
    &\leq
    \,\DHF{q}{\pi}\,
    &\leq
    {\lambda_{\mathrm{min}}\left(\mC\right)}^{-2} {\lVert \mC^{\top} \mSigma^{-1} \mC - \boldupright{I} \rVert}_{\mathrm{F}}^2.
  \end{alignat*}

\end{proofEnd}

\begin{theoremEnd}[category=stlgaussian]{proposition}
  For a Gaussian posterior \(\pi = \mathcal{N}\left(\vmu, \mSigma\right)\) and Gaussian variational family such that the base distribution \(\varphi = \mathcal{N}\left(0, 1\right)\), the gradient variance of the STL estimator is bounded as
  \begin{align*}
    \DHF{q_{\vlambda^*}}{\pi} &\leq \mathbb{E} \norm{\rvvg_{\mathrm{STL}}\left(\vlambda\right)}_2^2 \leq C_1(d) \, L^2 \, \norm{\vlambda - \vlambda^*}_2^2 + C_2(d) \, \DHF{q_{\vlambda^*}}{\pi},
  \end{align*}
  for all \(\vlambda \in \Lambda_L\), where \(\vlambda^* \in \argmin_{\vlambda \in \Lambda_L} F\left(\vlambda\right)\),

  %\vspace{-2ex}
  {\begingroup
  %  \setlength\tabcolsep{1.0ex} 
  \begin{tabular}{lll}
    \(C_1(d) = 6 d + 12\) & \(C_2(d) = 3 d + 9\) & for the full-rank and \\
    \(C_1(d) = 24 \sqrt{d} + 6\), & \(C_2(d) = 9 \sqrt{d}\) & for the mean-field parameterizations.
  \end{tabular}
  \endgroup}
\end{theoremEnd}
\begin{proofEnd}
  If the posterior is Gaussian, we have
  \[
    \nabla \log \pi\left(\vz\right)
    =
    {\mSigma}^{-1} \left( \vz - \vmu \right).
  \]
  Also, it is straightforward to check that the variational posterior mean is \(\vm = \vmu\).

  The lower bound is a direct application of \cref{thm:stl_lowerbound}.

  For the upper bound, we can partially reuse the results of  \cref{thm:stl_upperbound,thm:stl_upperbound_mf} such that, for full-rank parameterization:
  \begin{align}
    T_{\text{\ding{182}}} &= L^2 \left( d + k_{\varphi} \right) \norm{\vlambda - \vlambda^*}_2^2 \label{eq:stl_gaussian_fr_t182} \\
    T_{\text{\ding{183}}} &= S^2 \left( d + 1         \right) \norm{\vlambda - \vlambda^*}_2^2 \label{eq:stl_gaussian_fr_t183}
  \end{align}
  and for the mean-field parameterization:
  \begin{align}
    T_{\text{\ding{182}}} &= L^2 \left( 2 k_{\varphi} \sqrt{d} + 1 \right) \norm{\vlambda - \vlambda^*}_2^2 \label{eq:stl_gaussian_mf_t182} \\
    T_{\text{\ding{183}}} &= S^2 \left( \sqrt{d k_{\varphi}} + 1 \right) \norm{\vlambda - \vlambda^*}_2^2. \label{eq:stl_gaussian_mf_t183}
  \end{align}

  We focus on obtaining a tighter and more interpretable result on \(T_{\text{\ding{184}}}\) given as
  \[
    T_{\text{\ding{184}}} =  
    \mathbb{E} J_{\mathcal{T}}\left(\rvvu\right)
    \norm{ 
      \nabla \log \pi\left(\mathcal{T}_{\vlambda^*}\left(\rvvu\right)\right) 
      - 
      \nabla \log q_{\vlambda^*}\left(\mathcal{T}_{\vlambda^*}\left(\rvvu\right)\right) 
    }_2^2.
  \]
  Excluding the Jacobian term \(J_{\mathcal{T}}\) for now, we have
  \begin{align*}
    &\norm{ 
      \nabla \log \pi\left(\mathcal{T}_{\vlambda^*}\left(\vu\right)\right) 
      - 
      \nabla \log q_{\vlambda^*}\left(\mathcal{T}_{\vlambda^*}\left(\vu\right)\right) 
    }_2^2
    \\
    &\;=
    {\lVert} 
      \mSigma^{-1} \left( \mC^* \rvvu + \vm^* - \vmu \right)
      - 
      {\left( \mC^* \right)}^{-\top} {\left(\mC^*\right)}^{-1} \left( \mC^* \rvvu + \vm^* - \vm^* \right)
    {\rVert}_2^2 
    \\
    &\;=
    {\lVert} 
      \mSigma^{-1} \mC^* \rvvu 
      - 
      {\left( \mC^* \right)}^{-\top} \rvvu 
    {\rVert}_2^2 
    \\
    &\;=
    \norm{
      \left( \mSigma^{-1} \mC^* - {\left( \mC^* \right)}^{-\top} \right) \rvvu 
    }_2^2 
  \end{align*}

  Thus, bringing the Jacobian term in, we have
  \begin{align*}
    \mathbb{E} \,
    J_{\mathcal{T}}\left(\rvvu\right)
    \norm{ 
      \nabla \log \pi\left(\mathcal{T}_{\vlambda^*}\left(\rvvu\right)\right) 
      - 
      \nabla \log q_{\vlambda^*}\left(\mathcal{T}_{\vlambda^*}\left(\rvvu\right)\right) 
    }_2^2
    &\leq
    L \,
    \mathbb{E} \,
    J_{\mathcal{T}}\left(\rvvu\right)
    \norm{
      \left( \mSigma^{-1} \mC^* - {\left( \mC^* \right)}^{-\top} \right) \rvvu
    }_2^2.
  \end{align*}
  It now remains to deal with the stochastic terms.

  \paragraph{Full-Rank}
  For the full-rank parameterization,
  \begin{align}
    &
    \mathbb{E} \,
    J_{\mathcal{T}}\left(\rvvu\right)
    \norm{ 
      \nabla \log \pi\left(\mathcal{T}_{\vlambda^*}\left(\vu\right)\right) 
      - 
      \nabla \log q_{\vlambda^*}\left(\mathcal{T}_{\vlambda^*}\left(\vu\right)\right) 
    }_2^2
    \nonumber
    \\
    &\;=
    L \,
    \mathbb{E} 
    \left(1 + \norm{\rvvu}_2^2\right)
    \norm{
      \left( \mSigma^{-1} \mC^* - {\left( \mC^* \right)}^{-\top} \right) \rvvu
    }_2^2
    \nonumber
    \\
    &\;=
    L \,
    \mathbb{E} \,
    \left(1 + \norm{\rvvu}_2^2\right)
    \mathrm{tr}
    \left(
      \rvvu^{\top} 
      {\left(
        \mSigma^{-1} \mC^* - {\left( \mC^* \right)}^{-\top}
      \right)}^{\top}
      \left(
        \mSigma^{-1} \mC^* - {\left( \mC^* \right)}^{-\top}
      \right)
      \rvvu 
    \right)
    \nonumber
    \\
    &\;=
    \mathrm{tr}
    \left(
      {\left(
        \mSigma^{-1} \mC^* - {\left( \mC^* \right)}^{-\top}
      \right)}^{\top}
      \left(
        \mSigma^{-1} \mC^* - {\left( \mC^* \right)}^{-\top}
      \right)
      \mathbb{E}
      \left(1 + \norm{\rvvu}_2^2\right)
      \rvvu 
      \rvvu^{\top} 
    \right)
    \nonumber
    \\
    &\;=
    \mathrm{tr}
    \left(
      {\left(
        \mSigma^{-1} \mC^* - {\left( \mC^* \right)}^{-\top}
      \right)}^{\top}
      \left(
        \mSigma^{-1} \mC^* - {\left( \mC^* \right)}^{-\top}
      \right)
      \left( d + k_{\varphi} \right)
      \boldupright{I}
    \right)
    \nonumber
    \\
    &\;=
    \left( d + k_{\varphi} \right)
    \norm{
      \mSigma^{-1} \mC^* - {\left( \mC^* \right)}^{-\top}
    }_{\mathrm{F}}^2
\shortintertext{and by \cref{thm:gaussian_fisher_divergence},}
    &\;=
    \left( d + k_{\varphi} \right) \DHF{q_{\vlambda^*}}{\pi}.
    \label{eq:stl_gaussian_fr}
  \end{align}

  \paragraph{Mean-Field}
  Finally, for the mean-field parameterization, 
  \begin{align}
    &\mathbb{E} \,
    J_{\mathcal{T}}\left(\rvvu\right)
    \norm{ 
      \nabla \log \pi\left(\mathcal{T}_{\vlambda^*}\left(\rvvu\right)\right) 
      - 
      \nabla \log q_{\vlambda^*}\left(\mathcal{T}_{\vlambda^*}\left(\rvvu\right)\right) 
    }_2^2
    \nonumber
    \\
    &\;\leq
    \mathbb{E} \,
    J_{\mathcal{T}}\left(\rvvu\right)
    \norm{
      \left(
        \mSigma^{-1} \mC^* - {\left( \mC^* \right)}^{-\top}
      \right)
      \rvvu 
    }_2^2
    \nonumber
    \\
    &\;=
    \mathbb{E} \,
    \left(1 + {\lVert \rvmU^2 \rVert}_{\mathrm{F}} \right)
    \norm{
      \left(
        \mSigma^{-1} \mC^* - {\left( \mC^* \right)}^{-\top}
      \right)
      \rvvu 
    }_2^2,
    \nonumber
    \\
    &\;=
    k_{\varphi} \sqrt{d} \,
    \norm{
      \mSigma^{-1} \mC^* - {\left( \mC^* \right)}^{-\top}
    }_{\mathrm{F}}^2
\shortintertext{and by \cref{thm:gaussian_fisher_divergence},}
    &\;=
    k_{\varphi} \sqrt{d} \, \DHF{q_{\vlambda^*}}{\pi}
    \label{eq:stl_gaussian_mf}
  \end{align}

  Combining \cref{eq:stl_gaussian_fr,eq:stl_gaussian_mf} with \cref{eq:stl_gaussian_mf_t182,eq:stl_gaussian_mf_t183,eq:stl_gaussian_fr_t182,eq:stl_gaussian_fr_t183} and \cref{thm:stl_decomposition} yields the result.
  Note that, for a Gaussian \(\varphi\), we have \(k_{\varphi} = 3\), and we will set \(S = L\) and \(\delta = 1\).
  For the full-rank parameterization, 
  \begin{align*}
    \mathbb{E} \norm{\rvvg_{\mathrm{STL}}\left(\vlambda\right)}_2^2
    &\leq
    \left( 2 + \delta \right) T_{\text{\ding{182}}}
    +
    \left( 2 + \delta \right) T_{\text{\ding{183}}}
    +
    \left( 1 + 2 \delta^{-1} \right) T_{\text{\ding{184}}}
    \\
    &\leq
    \left( 2 + \delta \right) L^2 \left( d + k_{\varphi} \right) \norm{\vlambda - \vlambda^*}_2^2
    \\
    &\quad+
    \left( 2 + \delta \right) S^2 \left( d + 1         \right) \norm{\vlambda - \vlambda^*}_2^2
    \\
    &\quad+
    \left( 1 + 2 \delta^{-1} \right) \left( d + k_{\varphi} \right) \DHF{q_{\vlambda^*}}{\pi},
\shortintertext{substituting the constants,}
    &=
    3 L^2 \left( d + 3 \right) \norm{\vlambda - \vlambda^*}_2^2
    +
    3 L^2 \left( d + 1 \right) \norm{\vlambda - \vlambda^*}_2^2
    +
    3 \left( d + 3 \right) \DHF{q_{\vlambda^*}}{\pi}
    \\
    &=
    3 L^2 \left( 2 d + 4 \right) \norm{\vlambda - \vlambda^*}_2^2
    +
    3 \left( d + 3 \right) \DHF{q_{\vlambda^*}}{\pi}.
    \\
    &=
    6 L^2 \left( d + 2 \right) \norm{\vlambda - \vlambda^*}_2^2
    +
    3 \left( d + 3 \right) \DHF{q_{\vlambda^*}}{\pi}
  \end{align*}
  And for the mean-field parameterization, 
  \begin{align*}
    \mathbb{E} \norm{\rvvg_{\mathrm{STL}}\left(\vlambda\right)}_2^2
    &\leq
    \left( 2 + \delta \right) T_{\text{\ding{182}}}
    +
    \left( 2 + \delta \right) T_{\text{\ding{183}}}
    +
    \left( 1 + 2 \delta^{-1} \right) T_{\text{\ding{184}}}
    \\
    &\leq
    \left( 2 + \delta \right) L^2 \left( 2 k_{\varphi} \sqrt{d} + 1 \right) \norm{\vlambda - \vlambda^*}_2^2
    \\
    &\quad+
    \left( 2 + \delta \right) S^2 \left( \sqrt{d k_{\varphi}} + 1 \right) \norm{\vlambda - \vlambda^*}_2^2
    \\
    &\quad+
    \left( 1 + 2 \delta^{-1} \right) k_{\varphi} \sqrt{d} \DHF{q_{\vlambda^*}}{\pi},
\shortintertext{substituting the constants,}
    &=
    3 L^2 \left( 6 \sqrt{d} + 1 \right) \norm{\vlambda - \vlambda^*}_2^2
    +
    3 L^2 \left( \sqrt{3 d} + 1 \right) \norm{\vlambda - \vlambda^*}_2^2
    +
    9 \sqrt{d} \, \DHF{q_{\vlambda^*}}{\pi}
    \\
    &=
    3 L^2 \left( 6 \sqrt{d} + \sqrt{3 d} + 2 \right) \norm{\vlambda - \vlambda^*}_2^2
    +
    9 \sqrt{d} \, \DHF{q_{\vlambda^*}}{\pi},
\shortintertext{applying \(\sqrt{3} \leq 2\),}
    &\leq
    3 L^2 \left( 6  \sqrt{d} + 2 \sqrt{d} + 2 \right) \norm{\vlambda - \vlambda^*}_2^2
    +
    9 \sqrt{d} \, \DHF{q_{\vlambda^*}}{\pi}
    \\
    &=
    6 L^2 \left( 4 \sqrt{d} + 1 \right) \norm{\vlambda - \vlambda^*}_2^2
    +
    9 \sqrt{d} \, \DHF{q_{\vlambda^*}}{\pi}.
  \end{align*}
\end{proofEnd}

\vspace{1ex}

\begin{remark}
For Gaussians, the 4th-order Fisher-Hyv\"arinen divergence term in \cref{thm:stl_upperbound} can be replaced by its 2nd-order counterpart.
Thus, combined with \cref{thm:stl_lowerbound}, the 2nd-order Fisher-Hyv\"arinen divergence fully characterizes the variance of STL.
\end{remark}

\vspace{1ex}
\begin{remark}
  \cref{thm:fisher_bound} implies that, when approximating a full-rank Gaussian with a mean-field Gaussian, the value of the Fisher-Hyv\"arinen divergence is tightly characterized by the degree of correlation in the posterior; it will increase indefinitely as the posterior correlation matrix becomes singular.
\end{remark}

\vspace{1ex}
\begin{remark}
  We have provided a sufficient condition for the STL estimator to perform poorly compared to the CFE estimator. 
  It is foreseeable that alternative types of model misspecification abundant in practice should yield additional sufficient conditions, \textit{i.e.}, tail mismatch, but we leave this to future works.
\end{remark}

