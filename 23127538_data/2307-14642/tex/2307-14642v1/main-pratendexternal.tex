

\begin{restatable}[\citealt {domke_provable_2019}, Lemma 9]{lemma}{prAtEndRestatexx}\label{thm:prAtEndxx}\label {thm:u_identities} Let \(\rvvu = \left (\rvu _1, \rvu _2, \ldots , \rvu _d\right )\) be a \(d\)-dimensional vector-valued random variable with zero-mean independently and identically distributed components. Then, \begin {alignat*}{4} &\mathbb {E}\rvvu \rvvu ^{\top } &&= \left ( \mathbb {E} \rvu _i^2 \right ) \boldupright {I}\qquad &&\mathbb {E}\norm {\rvvu }_2^2 &&= d \, \mathbb {E} \rvu _i^2 \\ &\mathbb {E} \rvvu \left ( 1 + \norm {\rvvu }_2^2 \right ) &&= \left ( \mathbb {E} \rvu _i^3 \right ) \mathbf {1}\qquad &&\mathbb {E} \rvvu \rvvu ^{\top } \rvvu \rvvu ^{\top } &&= \left ( \left (d - 1\right ) \, {\left ( \mathbb {E} \rvu _i^2 \right )}^2 + \mathbb {E}\rvu _i^4 \right ) \boldupright {I}. \end {alignat*}\end{restatable}

\begin{restatable}[]{lemma}{prAtEndRestatexxi}\label{thm:prAtEndxxi}\label {thm:jacobian_reparam_inner} Let \(\mathcal {T}_{\vlambda }: \mathbb {R}^p \times \mathbb {R}^d \rightarrow \mathbb {R}^d\) be the location-scale reparameterization function (\cref {def:reparam}). Then, for any differentiable function \(f\), we have \[ \norm {\nabla _{\vlambda } f\left ( \mathcal {T}_{\vlambda }\left (\vu \right ) \right ) }_2^2 = J_{\mathcal {T}}\left (\rvvu \right ) \norm {\nabla f\left ( \mathcal {T}_{\vlambda }\left (\vu \right ) \right ) }_2^2. \] for any \(\vlambda \in \mathbb {R}^p\) and \(\vu \in \mathbb {R}^d\), where \(J_{\mathcal {T}}\left (\vu \right ) : \mathbb {R}^d \to \mathbb {R}\) is a function defined as \begin {alignat*}{2} J_{\mathcal {T}}\left (\vu \right ) &= 1 + {\textstyle \sum ^{d}_{i=1} u_i^2} \quad &&\quad \text {for the full-rank and} \\ J_{\mathcal {T}}\left (\vu \right ) &= 1 + {\textstyle \sqrt {\sum ^{d}_{i=1} u_i^4}} &&\quad \text {for the mean-field parameterizations.} \end {alignat*}\end{restatable}

\makeatletter\Hy@SaveLastskip\label{proofsection:prAtEndxxi}\ifdefined\pratend@current@sectionlike@label\immediate\write\@auxout{\string\gdef\string\pratend@section@for@proofxxi{\pratend@current@sectionlike@label}}\fi\Hy@RestoreLastskip\makeatother\begin{proof}[Proof]\phantomsection\label{proof:prAtEndxxi}The result is a collection of the results of \citet [Lemma 1]{domke_provable_2019} for the full-rank parameterization and \citet [Lemma 2]{kim_practical_2023} for the mean-field parameterization.\end{proof}

\begin{restatable}[Corollary 3; \citealp {kim_blackbox_2023}]{lemma}{prAtEndRestatexxii}\label{thm:prAtEndxxii}\label {thm:u_normsquared_marginalization} Assume \cref {assumption:symmetric_standard} and let \(\mathcal {T}_{\vlambda }: \mathbb {R}^d \rightarrow \mathbb {R}^d\) (\cref {def:reparam}) be the location-scale reparameterization function. Then, for any \(\vlambda , \vlambda ^{\prime } \in \mathbb {R}^p\), \[ \mathbb {E} J_{\mathcal {T}}\left (\rvvu \right ) \norm {\mathcal {T}_{\vlambda ^{\prime }}\left (\rvvu \right ) - \mathcal {T}_{\vlambda }\left (\rvvu \right ) }_2^2 \leq C\left (d, \varphi \right ) \norm {\vlambda - \vlambda '}_2^2, \] where \begin {alignat*}{2} C\left (d, \varphi \right ) &= d + k_{\varphi } &&\quad \text {for the full-rank and} \\ C\left (d, \varphi \right ) &= 2 k_{\varphi } \sqrt {d} + 1 &&\quad \text {for the mean-field parameterizations.} \end {alignat*}\end{restatable}

\begin{restatable}[Lemma 2; \citealp {domke_provable_2019}]{lemma}{prAtEndRestatexxiii}\label{thm:prAtEndxxiii}\label {thm:normdist_1pnormu} Assume \cref {assumption:symmetric_standard} and let \(\mathcal {T}_{\vlambda }: \mathbb {R}^d \rightarrow \mathbb {R}^d\) (\cref {def:reparam}) be the location-scale reparameterization function. Then, for the full-rank parameterization, \[ \mathbb {E} J_{\mathcal {T}}\left (\rvvu \right ) \norm {\mathcal {T}_{\vlambda }\left (\rvvu \right ) - \bar {\vz }}_2^2 = C_1\left (d, \varphi \right ) \norm { \vm - \bar {\vz } }_2^2 + C_2\left (d, \varphi \right ) \norm {\mC }_{\mathrm {F}}^2. \] where \begin {alignat*}{4} C_1\left (d, \varphi \right ) &= d + 1,\quad &C_2\left (d, \varphi \right ) &= d + k_{\varphi }, &&\quad \text {for the full-rank and} \\ C_1\left (d, \varphi \right ) &= \sqrt {d k_{\varphi }} + k \sqrt {d} + 1,\quad &C_2\left (d, \varphi \right ) &= 2 \kappa \sqrt {d} + 1, &&\quad \text {for the mean-field parameterizations.} \end {alignat*}\end{restatable}

\makeatletter\Hy@SaveLastskip\label{proofsection:prAtEndxxiii}\ifdefined\pratend@current@sectionlike@label\immediate\write\@auxout{\string\gdef\string\pratend@section@for@proofxxiii{\pratend@current@sectionlike@label}}\fi\Hy@RestoreLastskip\makeatother\begin{proof}[Proof]\phantomsection\label{proof:prAtEndxxiii}The result is a collection of the results of \citet [Lemma 2]{domke_provable_2019} for the full-rank parameterization and \citet [Lemma 2]{kim_practical_2023} for the mean-field parameterization.\end{proof}
