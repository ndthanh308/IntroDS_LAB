

%\begin{wraptable}[20]{r}[\dimexpr\columnwidth+\columnsep\relax]{0.65\textwidth}
\begin{table*}[t]
  \centering
\caption{Overview of Complexity Analyses of BBVI}\label{table:relatedworks}
  {\normalsize
\setlength{\tabcolsep}{4pt}
\renewcommand{\arraystretch}{1.2}
\begin{threeparttable}
\begin{tabular}{cccccccclc}
    \toprule
    \multicolumn{5}{c}{\textbf{Regularity of \(\pi\)}}
    & \multicolumn{1}{c}{\multirow{2}{*}{\textbf{\(q_{\vlambda^*} = \pi\)\tnote{1}}}}
    & \multicolumn{1}{c}{\multirow{2}{*}{\textbf{\makecell{Optimized\\Parameters}}}}
    & \multicolumn{1}{c}{\multirow{2}{*}{\textbf{\makecell{Gradient\\Estimator\tnote{2}}}}}
    & \multicolumn{1}{c}{\multirow{2}{*}{\textbf{\makecell{Iteration\\Complexity}}}}
    & \multicolumn{1}{c}{\multirow{2}{*}{\textbf{Reference}}}
    \\\cmidrule{1-5}
    \multicolumn{1}{c}{\textbf{\(\mu\)-PL}}
    & \multicolumn{1}{c}{\textbf{LC}}
    & \multicolumn{1}{c}{\textbf{\(\mu\)-SLC}}
    & \multicolumn{1}{c}{\textbf{\(L\)-LS}}
    & \multicolumn{1}{c}{\textbf{LQ}}
    &
    & 
    &
    &
    &
    \\ \midrule
    %
              &           &           &           & \ding{52} &           & scale only & exact & \(\mathcal{O}\left(\log\left( L \epsilon^{-1} \right) \right)\) &  \citealp{hoffman_blackbox_2020} \\
              &           &           &           & \ding{52} &           & scale only & CFE & \(\mathcal{O}\left(\kappa^2 \epsilon^{-1} \right)\) & \citealp{hoffman_blackbox_2020} \\
              &           &           &           & \ding{52} &           & scale only & n/a\tnote{3} & \(\mathcal{O}\left(L \epsilon^{-1} \right)\)\tnote{4} & \citealp{bhatia_statistical_2022} \\
   \textcolor{black!20}{\ding{52}}  & \textcolor{black!20}{\ding{52}} & \ding{52} & \ding{52} &           &           & scale only & n/a\tnote{3} & \(\mathcal{O}\left(L \epsilon^{-1} \right)\)\tnote{4} & \citealp{bhatia_statistical_2022} \\
    %
    \ding{52} &           &           & \ding{52} &           &           & loc. \& scale & CFE   & \(\mathcal{O}\left(L^2 \kappa \epsilon^{-4} \right)\) & \citealp{kim_blackbox_2023} \\
              % & \ding{52} &           & \ding{52} &           &           & loc. \& scale & CFE   & \(\mathcal{O}\left(L^2 \epsilon^{-2} \right)\) & \citealp{domke_provable_2023}  \\
    \textcolor{black!20}{\ding{52}} & \textcolor{black!20}{\ding{52}} & \ding{52} & \ding{52} &           &           & loc. \& scale & CFE   & \(\mathcal{O}\left(\kappa^2 \epsilon^{-1} \right)\) &  \makecell{\citealp{kim_blackbox_2023}\\\citealp{domke_provable_2023}} \\
     % & \ding{52} &  &  \ding{52} &   &           & loc. \& scale & CFE, STL & \(\mathcal{O}\left( \right)\) &  \citealp{domke_provable_2023} \\
    \textcolor{black!20}{\ding{52}} & \textcolor{black!20}{\ding{52}} & \ding{52} & \ding{52} &           &           & loc. \& scale & STL & \(\mathcal{O}\left(\kappa^2 \epsilon^{-1} \right)\) & \citealp{domke_provable_2023} \\
    \rowcolor{blue!10}
    \textcolor{black!20}{\ding{52}} & \textcolor{black!20}{\ding{52}} & \ding{52} & \ding{52} &           &           & loc. \& scale & STL   & \(\mathcal{O}\left(\kappa^2 \epsilon^{-1} \right)\) & \cref{thm:projsgd_bbvistl_decstepsize_complexity}\\
    \rowcolor{blue!10}
    \textcolor{black!20}{\ding{52}} & \textcolor{black!20}{\ding{52}} & \ding{52} & \ding{52} &           & \ding{52} & loc. \& scale & STL   & \(\mathcal{O}\left(\kappa^2 \log \epsilon^{-1} \right)\) &  \cref{thm:projsgd_bbvistl_complexity}
    %
    \\ \bottomrule
\end{tabular}
\begin{tablenotes}
\item[*] PL: Polyak-\L{}ojasiewicz, LC: log-concave, SLC: strongly-log-concave, LQ: log-quadratic (\(\pi\) is Gaussian), \(\kappa = L/\mu\).
\item[*] Analyses that a-priori assumed regularity of the ELBO were omitted.
\item[*] The explicit dimension dependences are omitted, but in general, \(\mathcal{O}\left(d\right)\) for full-rank, which is tight~\citep{domke_provable_2019}, and the best known for mean-field is {\scriptsize\(\mathcal{O}\left(\sqrt{d}\right)\)}~\citep{kim_practical_2023}.
The algorithm of \citet{bhatia_statistical_2022} is able to trade the dimension dependence for statistical accuracy.
\item[1] ``The variational family is perfectly specified.''
\item[2] The precise definitions of the gradient estimators are in \cref{section:gradient_estimators}.
\item[3] This algorithm uses stochastic power method-like iterations.
\item[4] The per-iteration sample complexity also depends on \(L, d, \epsilon\).
\end{tablenotes}
\end{threeparttable}
  }%
\end{table*}
%\end{wraptable}

%%% Local Variables:
%%% TeX-master: "main"
%%% End:


\section{RELATED WORKS}\label{section:related}

\paragraph{Analyzing the Computational Properties of BBVI}
Since its inception by \citet{ranganath_black_2014,titsias_doubly_2014}, theoretical results on BBVI have been developing on two different axes: 
\begin{enumerate*}[label=\textbf{(\alph*)}]
    \item Analyzing the regularity of the ELBO such as convexity and smoothness \citep{challis_gaussian_2013,titsias_doubly_2014}, 
    \item and analyzing the variance of the Monte Carlo gradient estimators \citep{fan_fast_2015,xu_variance_2019,mohamed_monte_2020,buchholz_quasimonte_2018}.
\end{enumerate*}
While some convergence analyses of BBVI have been provided \citep{regier_fast_2017,khan_faster_2016,khan_kullbackleibler_2015,buchholz_quasimonte_2018,liu_quasimonte_2021,alquier_concentration_2020,fujisawa_multilevel_2021,alquier_nonexponentially_2021}, these works \textit{a priori} assumed the regularity of the ELBO and the gradient estimators.
Due to the difficulty of rigorously establishing these conditions, later works by \citet{hoffman_blackbox_2020,bhatia_statistical_2022} have worked with simplified or alternative implementations of BBVI.
Meanwhile, \citet{xu_computational_2022} showed these regularities can be realized asymptotically in high probability.
In expectation, however, it was only recently that regularity conditions on the ELBO~\citep{domke_provable_2020,kim_blackbox_2023} and the reparameterization gradient estimator~\citep{kim_practical_2023,domke_provable_2019} were shown to be realizable under mild conditions without modifying the algorithms used in practice.

\paragraph{Concurrent Results by \citet{domke_provable_2023a}}
While our work builds on top of the QV-based framework of \citet{domke_provable_2023}, a similar convergence result on the STL estimator appeared in its later version~\citep{domke_provable_2023a} concurrently with our work.
However, our results differ in several aspects:
\begin{enumerate}
    \setlength\itemsep{-.5ex}
    \item[\ding{182}] For the family-misspecification term \(T_{\text{\ding{184}}}\) we bound the term using the Fisher-Hyv\"arinen divergence \(D_{\mathrm{F}^4}\left(q_{\vlambda^*}, \pi\right)\), while \citet{domke_provable_2023a} involve the smoothness constant of the residual function \(r\left(\vz\right) \triangleq \log q_{\vlambda^*}\left(\vz\right) - \log \pi\left(\vz\right)\).
    
    \item[\ding{183}] For the Gaussian posterior case, the constants of \cref{thm:stl_upperbound} are tighter that of \citet{domke_provable_2023a} by a factor of \(\times 2\).
    
    \item[\ding{184}] We also provide an upper bound for the mean-field variational family in \cref{thm:stl_upperbound_mf}.
    
    \item[\ding{185}] We establish a lower bound on the gradient variance of STL, quantifying the tightness of the bounds.
\end{enumerate}
