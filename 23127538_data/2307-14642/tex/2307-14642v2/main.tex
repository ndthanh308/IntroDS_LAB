
\documentclass[twoside]{article}

\usepackage[accepted]{aistats2024}
% If your paper is accepted, change the options for the package
% aistats2024 as follows:
%
%\usepackage[accepted]{aistats2024}
%
% This option will print headings for the title of your paper and
% headings for the authors names, plus a copyright note at the end of
% the first column of the first page.

% If you set papersize explicitly, activate the following three lines:
%\special{papersize = 8.5in, 11in}
%\setlength{\pdfpageheight}{11in}
%\setlength{\pdfpagewidth}{8.5in}

\usepackage[sort&compress,round]{natbib}
\renewcommand{\bibname}{References}
\renewcommand{\bibsection}{\subsubsection*{\bibname}}

\usepackage{graphicx}
\usepackage[dvipsnames]{xcolor}
\usepackage{setspace}
\usepackage{nicefrac}
\usepackage{pgfplots}
\usepackage{float}
\usepackage[inline]{enumitem}
\usepackage{pifont}
\usepackage{tabularx,booktabs,threeparttable,makecell,colortbl}
% \usepackage[13-]{pagesel}

\usepackage{calc}


% put these before amssymb
% to avoid ``Too many math alphabets used in version normal.''
% cf. https://tex.stackexchange.com/a/243541/91665
\newcommand{\hmmax}{0}
\newcommand{\bmmax}{0}
\usepackage{amssymb}

\usepackage{amsmath,amsthm}
\usepackage{mathtools}
\usepackage{iftex}

% Math and main text font nonsense
\ifxetex
  \usepackage{fontspec}
  \usepackage{microtype}
  \usepackage{unicode-math}
  \setmathfont{STIXTwoMath-Regular.otf}
  \setmainfont[
    Ligatures=TeX,
    BoldFont={STIXTwoText-Bold.otf}, 
    ItalicFont={STIXTwoText-Italic.otf},
    BoldItalicFont={STIXTwoText-BoldItalic.otf}]{STIXTwoText-Regular.otf}
  %\setmathfont{Erewhon-Math.otf}
  %\setmainfont{Times New Roman}

  %\setmathfont{TeX Gyre Schola Math}[Scale=0.93]
  % \setmainfont{TeX Gyre Schola}[Scale=0.90] % Text
  % Math: mix of Erewhon and Schola
  %\setmathfont{Erewhon-Math.otf}
  %%     range={up/{latin,Latin,num}, it/{latin,Latin,num},
  %%            bfup/{latin,Latin,num}, bfit{latin,Latin,num}}]

  
  \newcommand{\mathmat}[1]{\mathbfit{#1}}
  \newcommand{\mathvec}[1]{\mathbfit{#1}}
  \newcommand{\mathvecgreek}[1]{\mathbfit{#1}}
  \newcommand{\mathmatgreek}[1]{\mathbfit{#1}}
  
  \newcommand{\boldupright}[1]{\symbfup{#1}}

  \newcommand{\mathrv}[1]{\mathsfit{#1}}
  \newcommand{\mathrvvec}[1]{\mathbfsfit{#1}}
  \newcommand{\mathrvgreek}[1]{\mathsfit{#1}}
  \newcommand{\mathrvvecgreek}[1]{\mathbfsfit{#1}}
\else
  %\usepackage[notext,lcgreekalpha]{stix2}
  \usepackage[lcgreekalpha]{stix2}
  %\usepackage{newtxtext}
  \usepackage{textcomp}

  \newcommand{\mathmat}[1]{\mathbfit{#1}}
  \newcommand{\mathvec}[1]{\mathbfit{#1}}
  \newcommand{\mathvecgreek}[1]{\mathbfit{#1}}
  \newcommand{\mathmatgreek}[1]{\mathbfit{#1}}
  
  \newcommand{\boldupright}[1]{\mathbf{#1}}

  \newcommand{\mathrv}[1]{\mathsfit{#1}}
  \newcommand{\mathrvvec}[1]{\mathbfsfit{#1}}
  \newcommand{\mathrvgreek}[1]{\mathsfit{#1}}
  \newcommand{\mathrvvecgreek}[1]{\mathbfsfit{#1}}
\fi

\usepackage{booktabs,threeparttable,arydshln}
\usepackage{multirow}
\usepackage{nicematrix}

\usepackage[algo2e, ruled]{algorithm2e}

\usepackage{pgfplots}
\pgfkeys{/pgfplots/tuftelike/.style={
  semithick,
  tick style={major tick length=4pt,semithick,black},
  separate axis lines,
  axis x line*=bottom,
  axis x line shift=2ex,
  xlabel shift=-2pt,
  axis y line*=left,
  axis y line shift=2ex,
  ylabel shift=-2pt}}
  
\usepackage{proof-at-the-end}

\usepackage{mdframed}
\mdfdefinestyle {mdleftbarcoloredstyle} {%
    leftline          = true,%
    rightline         = false,%
    topline           = false,%
    bottomline        = false,%
    linewidth         = 2.5pt,%
    %roundcorner       = 10pt,%
    %leftmargin        = 40,%
    %rightmargin       = 40 ,%
    backgroundcolor   = gray!10,%
    linecolor         = gray,%
    innertopmargin    = 0.0ex,%
    innerbottommargin = 0.7ex,%
    innerleftmargin   = 1.ex,%
    %splittopskip      = \topskip,%
    skipabove         = \parskip,%
    skipbelow         = \parskip,%
    %leftmargin        = -\parskip,%
    ntheorem          = false,%
}

\mdfdefinestyle {coloredstyle} {%
    leftline          = false,%
    rightline         = false,%
    topline           = false,%
    bottomline        = false,%
    backgroundcolor   = gray!10,%
    skipabove         = \parskip,%
    skipbelow         = \parskip,%
    innertopmargin    = 0.0ex,%
    innerbottommargin = 0.7ex,%
    innerleftmargin   = 1.ex,%
}

\declaretheoremstyle[
    spaceabove    = \parsep,
    spacebelow    = \parsep,
    %postheadspace = \newline,
    %headpunct     = {},
    bodyfont      = \normalfont\itshape,
]{theoremsty}

\declaretheorem[name=Theorem,    style=theoremsty, mdframed={style = coloredstyle}]{theorem}
\declaretheorem[name=Proposition,style=theoremsty, mdframed={style = coloredstyle}]{proposition}
\declaretheorem[name=Corollary,  style=theoremsty, mdframed={style = coloredstyle}]{corollary}
\declaretheorem[name=Lemma,      style=theoremsty, mdframed={style = coloredstyle}]{lemma}
\declaretheorem[name=Theorem,    style=theoremsty, mdframed={style = coloredstyle}, numbered=no]{theorem*}
\declaretheorem[name=Proposition,style=theoremsty, mdframed={style = coloredstyle}, numbered=no]{proposition*}
\declaretheorem[name=Corollary,  style=theoremsty, mdframed={style = coloredstyle}, numbered=no]{corollary*}
\declaretheorem[name=Lemma,      style=theoremsty, mdframed={style = coloredstyle}, numbered=no]{lemma*}

\declaretheorem[name=Open Problem,style=theoremsty, mdframed={style = coloredstyle}]{openproblem}
\declaretheorem[name=Conjecture,  style=theoremsty, mdframed={style = coloredstyle}]{conjecture}
\declaretheorem[name=Open Problem,style=theoremsty, mdframed={style = coloredstyle}, numbered=no]{openproblem*}
\declaretheorem[name=Conjecture,  style=theoremsty, mdframed={style = coloredstyle}, numbered=no]{conjecture*}

\declaretheoremstyle[
    spaceabove=\parsep,
    spacebelow=\parsep,
    bodyfont=\normalfont,
]{normalsty}
\declaretheorem[name=Remark,      style=normalsty]{remark}
\declaretheorem[name=Definition,  style=normalsty, mdframed={style = 
 coloredstyle}]{definition}
\declaretheorem[name=Assumption,  style=normalsty, mdframed={style = 
 coloredstyle} ]{assumption}
\declaretheorem[name=Example,     style=normalsty, mdframed={style = 
 coloredstyle}]{example}

\declaretheorem[name=Remark,      style=normalsty, mdframed={style = 
 coloredstyle}, numbered=no]{remark*}
\declaretheorem[name=Definition,  style=normalsty, mdframed={style = coloredstyle}, numbered=no]{definition*}
\declaretheorem[name=Assumption,  style=normalsty, mdframed={style = coloredstyle}, numbered=no]{assumption*}

\newenvironment{proofsketch}{%
  \renewcommand{\proofname}{Proof Sketch}\proof%
  \renewcommand{\qedsymbol}{}%
  }{\endproof}

%\usepackage{url}

\makeatletter
\providecommand\IfFormatAtLeastTF{\@ifl@t@r\fmtversion}
\IfFormatAtLeastTF{2020/10/02}{%
  \usepackage[bookmarksopen=true,bookmarks=true,colorlinks=true,backref=page]{hyperref}
  \usepackage[capitalise, nameinlink]{cleveref}
  \renewcommand*{\backref}[1]{}
  \renewcommand*{\backrefalt}[4]{({\footnotesize%
  \ifcase #1 Not cited.%
    \or page~#2%
    \else pages #2%
  \fi%
  })}
  \renewcommand*{\backreftwosep}{\backrefsep}
  \renewcommand*{\backreflastsep}{\backrefsep}%  
}{%
  \usepackage{hyperref}
  \usepackage[capitalise, nameinlink]{cleveref}%
}
\makeatother

\makeatletter
\newcommand{\crefnames}[3]{%
  \@for\next:=#1\do{%
    \expandafter\crefname\expandafter{\next}{#2}{#3}%
  }%
}
\makeatother
\crefnames{part,chapter,section}{\S}{\S\S}

\definecolor{linkcolor}{HTML}{6929C4}
\definecolor{citecolor}{HTML}{0043CE}
\hypersetup{colorlinks={true},linkcolor=linkcolor,citecolor=citecolor}

\crefname{assumption}{Assumption}{Assumption}
\crefname{condition}{Condition}{Condition}

\crefname{framedtheorem}{Theorem}{Theorems}
\crefname{framedproposition}{Proposition}{Propositions}
\crefname{framedlemma}{Lemma}{Lemmas}

\makeatletter
\def\adl@drawiv#1#2#3{%
        \hskip.5\tabcolsep
        \xleaders#3{#2.5\@tempdimb #1{1}#2.5\@tempdimb}%
                #2\z@ plus1fil minus1fil\relax
        \hskip.5\tabcolsep}
\newcommand{\cdashlinelr}[1]{%
  \noalign{\vskip\aboverulesep
           \global\let\@dashdrawstore\adl@draw
           \global\let\adl@draw\adl@drawiv}
  \cdashline{#1}
  \noalign{\global\let\adl@draw\@dashdrawstore
           \vskip\belowrulesep}}
\makeatother

\pgfkeys{/prAtEnd/global custom defaults/.style={
    %proof at the end,
    end,
    %normal,
    restate,
    %% text link={\textit{Proof.} See page~\pageref{proof:prAtEnd\pratendcountercurrent} of the \textit{supplementary material}.},
    %text link={\textit{Proof.} The proof is in the \textit{supplementary material}.
    text proof={Proof},
    text link={\textit{Proof.} See the \hyperref[proof:prAtEnd\pratendcountercurrent]{\textit{full proof}} in page~\pageref{proof:prAtEnd\pratendcountercurrent}.}
  }
}

\def\code#1{\texttt{#1}}
\DeclareMathOperator*{\minimize}{minimize}
\DeclareMathOperator*{\maximize}{maximize}
\DeclareMathOperator*{\argmax}{arg\,max}
\DeclareMathOperator*{\argmin}{arg\,min} 

\newcommand*\xbar[1]{%
  \hbox{%
    \vbox{%
      \hrule height 0.6pt % The actual bar
      \kern0.33ex%         % Distance between bar and symbol
      \hbox{%
        \kern-0.1em%      % Shortening on the left side
        \ensuremath{#1}%
        \kern-0.1em%      % Shortening on the right side
      }%
    }%
  }%
} 

\newcommand{\E}[1]{\mathbb{E}\left[ #1 \right]}
\newcommand{\Esub}[2]{\mathbb{E}_{#1}\left[ #2 \right]}
\newcommand{\V}[1]{\mathbb{V}\left[ #1 \right]}
\newcommand{\Vsub}[2]{\mathbb{V}_{#1}\left[ #2 \right]}
\newcommand{\Cov}[1]{\mathrm{Cov}\left( #1 \right)}
\newcommand{\Covsub}[2]{\mathrm{Cov}_{#1}\left( #2 \right)}
\newcommand{\Corr}[1]{\mathrm{Corr}\left( #1 \right)}

\newcommand{\Df}[2]{\mathrm{D}_{f}(#1,#2)}
\newcommand{\DHF}[2]{\mathrm{D}_{\mathrm{F}}(#1,#2)}
\newcommand{\DKL}[2]{\mathrm{D}_{\mathrm{KL}}(#1,#2)}
\newcommand{\DChi}[2]{\mathrm{D}_{\chi^2}(#1,#2)}
\newcommand{\norm}[1]{{\left\lVert #1 \right\rVert}}
\newcommand{\abs}[1]{{\left| #1 \right|}}
\newcommand{\DTV}[2]{{\mathrm{d}_{\mathrm{TV}}\left(#1, #2\right)}}

\newcommand{\vX}{\mathrv{X}}
\newcommand{\vY}{\mathrv{Y}}
\newcommand{\vZ}{\mathrv{Z}}

\newcommand{\va}{\mathvec{a}}
\newcommand{\vb}{\mathvec{b}}
\newcommand{\vc}{\mathvec{c}}
\newcommand{\vd}{\mathvec{d}}
\newcommand{\ve}{\mathvec{e}}
\newcommand{\vf}{\mathvec{f}}
\newcommand{\vg}{\mathvec{g}}
\newcommand{\vh}{\mathvec{h}}
\newcommand{\vi}{\mathvec{i}}
\newcommand{\vj}{\mathvec{j}}
\newcommand{\vk}{\mathvec{k}}
\newcommand{\vl}{\mathvec{l}}
\newcommand{\vm}{\mathvec{m}}
\newcommand{\vn}{\mathvec{n}}
\newcommand{\vo}{\mathvec{o}}
\newcommand{\vp}{\mathvec{p}}
\newcommand{\vq}{\mathvec{q}}
\newcommand{\vr}{\mathvec{r}}
\newcommand{\vs}{\mathvec{s}}
\newcommand{\vt}{\mathvec{t}}
\newcommand{\vu}{\mathvec{u}}
\newcommand{\vv}{\mathvec{v}}
\newcommand{\vw}{\mathvec{w}}
\newcommand{\vx}{\mathvec{x}}
\newcommand{\vy}{\mathvec{y}}
\newcommand{\vz}{\mathvec{z}}
\newcommand{\vDelta}{\mathvecgreek{\Delta}}
\newcommand{\valpha}{\mathvecgreek{\alpha}}
\newcommand{\vepsilon}{\mathvecgreek{\epsilon}}
\newcommand{\vbeta}{\mathvecgreek{\beta}}
\newcommand{\veta}{\mathvecgreek{\eta}}
\newcommand{\vmu}{\mathvecgreek{\mu}}
\newcommand{\vnu}{\mathvecgreek{\nu}}
\newcommand{\vtheta}{\mathvecgreek{\theta}}
\newcommand{\vlambda}{\mathvecgreek{\lambda}}
\newcommand{\vgamma}{\mathvecgreek{\gamma}}
\newcommand{\vxi}{\mathvecgreek{\xi}}
\newcommand{\vzeta}{\mathvecgreek{\zeta}}
\newcommand{\vphi}{\mathvecgreek{\phi}}

\newcommand{\rva}{\mathrv{a}}
\newcommand{\rvb}{\mathrv{b}}
\newcommand{\rvc}{\mathrv{c}}
\newcommand{\rvd}{\mathrv{d}}
\newcommand{\rve}{\mathrv{e}}
\newcommand{\rvf}{\mathrv{f}}
\newcommand{\rvg}{\mathrv{g}}
\newcommand{\rvh}{\mathrv{h}}
\newcommand{\rvi}{\mathrv{i}}
\newcommand{\rvj}{\mathrv{j}}
\newcommand{\rvk}{\mathrv{k}}
\newcommand{\rvl}{\mathrv{l}}
\newcommand{\rvm}{\mathrv{m}}
\newcommand{\rvn}{\mathrv{n}}
\newcommand{\rvo}{\mathrv{o}}
\newcommand{\rvp}{\mathrv{p}}
\newcommand{\rvq}{\mathrv{q}}
\newcommand{\rvr}{\mathrv{r}}
\newcommand{\rvs}{\mathrv{s}}
\newcommand{\rvt}{\mathrv{t}}
\newcommand{\rvu}{\mathrv{u}}
\newcommand{\rvv}{\mathrv{v}}
\newcommand{\rvw}{\mathrv{w}}
\newcommand{\rvx}{\mathrv{x}}
\newcommand{\rvy}{\mathrv{y}}
\newcommand{\rvz}{\mathrv{z}}
\newcommand{\rvA}{\mathrv{A}}
\newcommand{\rvB}{\mathrv{B}}
\newcommand{\rvC}{\mathrv{C}}
\newcommand{\rvD}{\mathrv{D}}
\newcommand{\rvE}{\mathrv{E}}
\newcommand{\rvF}{\mathrv{F}}
\newcommand{\rvG}{\mathrv{G}}
\newcommand{\rvH}{\mathrv{H}}
\newcommand{\rvI}{\mathrv{I}}
\newcommand{\rvJ}{\mathrv{J}}
\newcommand{\rvK}{\mathrv{K}}
\newcommand{\rvL}{\mathrv{L}}
\newcommand{\rvM}{\mathrv{M}}
\newcommand{\rvN}{\mathrv{N}}
\newcommand{\rvO}{\mathrv{O}}
\newcommand{\rvP}{\mathrv{P}}
\newcommand{\rvQ}{\mathrv{Q}}
\newcommand{\rvR}{\mathrv{R}}
\newcommand{\rvS}{\mathrv{S}}
\newcommand{\rvT}{\mathrv{T}}
\newcommand{\rvU}{\mathrv{U}}
\newcommand{\rvV}{\mathrv{V}}
\newcommand{\rvW}{\mathrv{W}}
\newcommand{\rvX}{\mathrv{X}}
\newcommand{\rvY}{\mathrv{Y}}
\newcommand{\rvZ}{\mathrv{Z}}
\newcommand{\rveta}{\mathrvgreek{\eta}}
\newcommand{\rvlambda}{\mathrvgreek{\lambda}}

\newcommand{\rvva}{\mathrvvec{a}}
\newcommand{\rvvb}{\mathrvvec{b}}
\newcommand{\rvvc}{\mathrvvec{c}}
\newcommand{\rvvd}{\mathrvvec{d}}
\newcommand{\rvve}{\mathrvvec{e}}
\newcommand{\rvvf}{\mathrvvec{f}}
\newcommand{\rvvg}{\mathrvvec{g}}
\newcommand{\rvvh}{\mathrvvec{h}}
\newcommand{\rvvi}{\mathrvvec{i}}
\newcommand{\rvvj}{\mathrvvec{j}}
\newcommand{\rvvk}{\mathrvvec{k}}
\newcommand{\rvvl}{\mathrvvec{l}}
\newcommand{\rvvm}{\mathrvvec{m}}
\newcommand{\rvvn}{\mathrvvec{n}}
\newcommand{\rvvo}{\mathrvvec{o}}
\newcommand{\rvvp}{\mathrvvec{p}}
\newcommand{\rvvq}{\mathrvvec{q}}
\newcommand{\rvvr}{\mathrvvec{r}}
\newcommand{\rvvs}{\mathrvvec{s}}
\newcommand{\rvvt}{\mathrvvec{t}}
\newcommand{\rvvu}{\mathrvvec{u}}
\newcommand{\rvvv}{\mathrvvec{v}}
\newcommand{\rvvw}{\mathrvvec{w}}
\newcommand{\rvvx}{\mathrvvec{x}}
\newcommand{\rvvy}{\mathrvvec{y}}
\newcommand{\rvvz}{\mathrvvec{z}}
\newcommand{\rvvlambda}{\mathrvvecgreek{\lambda}}
\newcommand{\rvveta}{\mathrvvecgreek{\eta}}

\newcommand{\rvmU}{\mathrvvecgreek{U}}
\newcommand{\rvmA}{\mathrvvecgreek{A}}
\newcommand{\rvmQ}{\mathrvvecgreek{Q}}
\newcommand{\rvmJ}{\mathrvvecgreek{J}}
\newcommand{\rvmL}{\mathrvvecgreek{L}}
\newcommand{\rvmH}{\mathrvvecgreek{H}}

\newcommand{\mA}{\mathmat{A}}
\newcommand{\mB}{\mathmat{B}}
\newcommand{\mC}{\mathmat{C}}
\newcommand{\mD}{\mathmat{D}}
\newcommand{\mE}{\mathmat{E}}
\newcommand{\mF}{\mathmat{F}}
\newcommand{\mG}{\mathmat{G}}
\newcommand{\mH}{\mathmat{H}}
\newcommand{\mI}{\mathmat{I}}
\newcommand{\mJ}{\mathmat{J}}
\newcommand{\mK}{\mathmat{K}}
\newcommand{\mL}{\mathmat{L}}
\newcommand{\mM}{\mathmat{M}}
\newcommand{\mN}{\mathmat{N}}
\newcommand{\mO}{\mathmat{O}}
\newcommand{\mP}{\mathmat{P}}
\newcommand{\mQ}{\mathmat{Q}}
\newcommand{\mR}{\mathmat{R}}
\newcommand{\mS}{\mathmat{S}}
\newcommand{\mT}{\mathmat{T}}
\newcommand{\mU}{\mathmat{U}}
\newcommand{\mV}{\mathmat{V}}
\newcommand{\mW}{\mathmat{W}}
\newcommand{\mX}{\mathmat{X}}
\newcommand{\mY}{\mathmat{Y}}
\newcommand{\mZ}{\mathmat{Z}}
\newcommand{\mSigma}{\mathmatgreek{\Sigma}}
\newcommand{\mPhi}{\mathmatgreek{\Phi}}
\newcommand{\mGamma}{\mathmatgreek{\Gamma}}

\newcommand{\inner}[2]{\left\langle #1, #2 \right\rangle}
\newcommand{\ind}[1]{\mathds{1}_{#1}}


\usepackage{etoc}
%\etocmulticolstyle{\noindent\bfseries\footnotesize
%\leaders\hrule height1pt\hfill
%\MakeUppercase{Contents}}
\etocframedstyle[1]{\textbf{\textsc{Table of Contents}}}
\etocsettocdepth{3}

% If you use BibTeX in apalike style, activate the following line:
%\bibliographystyle{apalike}

\begin{document}

% If your paper is accepted and the title of your paper is very long,
% the style will print as headings an error message. Use the following
% command to supply a shorter title of your paper so that it can be
% used as headings.
%
\runningtitle{Linear Convergence of Black-Box Variational Inference}

% If your paper is accepted and the number of authors is large, the
% style will print as headings an error message. Use the following
% command to supply a shorter version of the authors names so that
% they can be used as headings (for example, use only the surnames)
%
%\runningauthor{Surname 1, Surname 2, Surname 3, ...., Surname n}

\twocolumn[

\aistatstitle{Linear Convergence of Black-Box Variational Inference: \\ Should We Stick the Landing?}

\aistatsauthor{ Kyurae Kim \And Yi-An Ma \And Jacob R. Gardner }

\aistatsaddress{ University of Pennsylvania \And University of California San Diego \And  University of Pennsylvania }
]

\begin{abstract}
  We prove that black-box variational inference (BBVI) with control variates, particularly the sticking-the-landing (STL) estimator, converges at a geometric (traditionally called ``linear'') rate under perfect variational family specification.
  In particular, we prove a quadratic bound on the gradient variance of the STL estimator, one which encompasses misspecified variational families.
  Combined with previous works on the quadratic variance condition, this directly implies convergence of BBVI with the use of projected stochastic gradient descent.
  We also improve existing analysis on the regular closed-form entropy gradient estimators, which enables comparison against the STL estimator, and provide explicit non-asymptotic complexity guarantees for both.
\end{abstract}



\section{Introduction}
Despite the massive success of black-box variational inference (BBVI; \citealp{ranganath_black_2014,titsias_doubly_2014,kucukelbir_automatic_2017}), our understanding of its computational properties has only recently started to make progress \citep{domke_provable_2019,hoffman_blackbox_2020,domke_provable_2020,domke_provable_2023,kim_blackbox_2023,kim_practical_2023}. 
Notably, \citet{domke_provable_2023,kim_blackbox_2023} have independently established the convergence of ``full'' BBVI.
This is a significant advance from the previous results where simplified versions of BBVI were analyzed \citep{hoffman_blackbox_2020,bhatia_statistical_2022} and results that \textit{a-priori} assumed regularity of the ELBO~\citep{alquier_concentration_2020,cherief-abdellatif_generalization_2019,regier_fast_2017,khan_faster_2016,khan_kullbackleibler_2015a,fujisawa_multilevel_2021,buchholz_quasimonte_2018,liu_quasimonte_2021}.
We now have rigorous convergence guarantees that, for certain well-behaved posteriors, BBVI achieves a convergence rate of \(\mathcal{O}\left(1/T\right)\), corresponding to a computational complexity of \(\mathcal{O}\left(1/\epsilon\right)\)~\citep{domke_provable_2023,kim_blackbox_2023}.
A remaining theoretical question is whether BBVI can achieve better rates, in particular geometric convergence rates, which is traditionally called ``linear'' convergence in the optimization literature (see the textbook by \citealt[\S 1.2.3]{nesterov_introductory_2004}), corresponding to a complexity of \(\mathcal{O}\left(\log\left( 1/\epsilon \right) \right)\).

For stochastic gradient descent (SGD; \citealp{robbins_stochastic_1951,bottou_online_1999,nemirovski_robust_2009}), it is known that improving the \(\mathcal{O}\left(1/T\right)\) convergence rate is challenging~\citep{rakhlin_making_2012,harvey_tight_2019}.
This is because, once in the stationary regime, it is necessary to either decrease the stepsize or average the iterates, where the latter reduces SGD to Markov chain Monte Carlo~\citep{dieuleveut_bridging_2020}.
Not surprisingly, both cases result in a significant slowdown compared to deterministic gradient descent.
Overall, SGD is known to achieve \(\mathcal{O}\left(1/\sqrt{T}\right)\) for general convex functions and \(\mathcal{O}\left(1/T\right)\) for strongly convex functions (\citealp{nemirovski_robust_2009}; for more modern analysis techniques, see~\citealp{garrigos_handbook_2023,gower_sgd_2019}).

Meanwhile, under a condition known as ``interpolation,'' which assumes that the gradient variance becomes zero at the optimum, SGD is known to achieve a linear convergence rate~\citep{schmidt_fast_2013}.
This can automatically hold for certain problems, such as empirical risk minimization (ERM) with overparameterized models, explaining the fast empirical convergence of modern machine learning models~\citep{vaswani_fast_2019, ma_power_2018}.
Also, control variate methods such as ``variance-reduced'' gradients \citep{schmidt_minimizing_2017,johnson_accelerating_2013,gower_variancereduced_2020} algorithmically achieve the same effect and have been successful both in theory and practice.
Unfortunately, variance-reduced gradient methods are strictly restricted to the finite-sum setting, which BBVI is not part of (See \S 2.4 by \citealp{kim_blackbox_2023}).
Thus it is yet unclear how BBVI could benefit from the advances in variance-reduced gradients.

Fortunately, other types of control variates have been actively pursued in BBVI \citep{geffner_using_2018,miller_reducing_2017, ranganath_black_2014, paisley_variational_2012, geffner_approximation_2020}.
One of which, called the \textit{sticking-the-landing} (STL; \citealp{roeder_sticking_2017}) estimator, has been known to satisfy the interpolation condition under realizable assumptions\footnote{Although the term interpolation does not 
literally make sense outside of the ERM context, we will stick to this term to stay in line with the SGD literature.}.
When using BBVI with perfect variational family specification (the posterior \(\pi\) is in the variational family \(\mathcal{Q}\) such that \(\pi \in \mathcal{Q}\)), the STL estimator achieves zero variance at the optimum.
It is thus natural to ask whether existing control variate approaches such as STL are sufficient to achieve linear convergence under appropriate conditions.
In fact, this possibility has been mentioned by~\citet[\S 5]{hoffman_blackbox_2020} and \citet[\S 5]{domke_provable_2023}.


%\begin{wraptable}[20]{r}[\dimexpr\columnwidth+\columnsep\relax]{0.65\textwidth}
\begin{table*}[t]
  \centering
\caption{Overview of Complexity Analyses of BBVI}\label{table:relatedworks}
  {\normalsize
\setlength{\tabcolsep}{4pt}
\renewcommand{\arraystretch}{1.2}
\begin{threeparttable}
\begin{tabular}{cccccccclc}
    \toprule
    \multicolumn{5}{c}{\textbf{Regularity of \(\pi\)}}
    & \multicolumn{1}{c}{\multirow{2}{*}{\textbf{\(q_{\vlambda^*} = \pi\)\tnote{1}}}}
    & \multicolumn{1}{c}{\multirow{2}{*}{\textbf{\makecell{Optimized\\Parameters}}}}
    & \multicolumn{1}{c}{\multirow{2}{*}{\textbf{\makecell{Gradient\\Estimator\tnote{2}}}}}
    & \multicolumn{1}{c}{\multirow{2}{*}{\textbf{\makecell{Iteration\\Complexity}}}}
    & \multicolumn{1}{c}{\multirow{2}{*}{\textbf{Reference}}}
    \\\cmidrule{1-5}
    \multicolumn{1}{c}{\textbf{\(\mu\)-PL}}
    & \multicolumn{1}{c}{\textbf{LC}}
    & \multicolumn{1}{c}{\textbf{\(\mu\)-SLC}}
    & \multicolumn{1}{c}{\textbf{\(L\)-LS}}
    & \multicolumn{1}{c}{\textbf{LQ}}
    &
    & 
    &
    &
    &
    \\ \midrule
    %
              &           &           &           & \ding{52} &           & scale only & exact & \(\mathcal{O}\left(\log\left( L \epsilon^{-1} \right) \right)\) &  \citealp{hoffman_blackbox_2020} \\
              &           &           &           & \ding{52} &           & scale only & CFE & \(\mathcal{O}\left(\kappa^2 \epsilon^{-1} \right)\) & \citealp{hoffman_blackbox_2020} \\
              &           &           &           & \ding{52} &           & scale only & n/a\tnote{3} & \(\mathcal{O}\left(L \epsilon^{-1} \right)\)\tnote{4} & \citealp{bhatia_statistical_2022} \\
   \textcolor{black!20}{\ding{52}}  & \textcolor{black!20}{\ding{52}} & \ding{52} & \ding{52} &           &           & scale only & n/a\tnote{3} & \(\mathcal{O}\left(L \epsilon^{-1} \right)\)\tnote{4} & \citealp{bhatia_statistical_2022} \\
    %
    \ding{52} &           &           & \ding{52} &           &           & loc. \& scale & CFE   & \(\mathcal{O}\left(L^2 \kappa \epsilon^{-4} \right)\) & \citealp{kim_blackbox_2023} \\
              % & \ding{52} &           & \ding{52} &           &           & loc. \& scale & CFE   & \(\mathcal{O}\left(L^2 \epsilon^{-2} \right)\) & \citealp{domke_provable_2023}  \\
    \textcolor{black!20}{\ding{52}} & \textcolor{black!20}{\ding{52}} & \ding{52} & \ding{52} &           &           & loc. \& scale & CFE   & \(\mathcal{O}\left(\kappa^2 \epsilon^{-1} \right)\) &  \makecell{\citealp{kim_blackbox_2023}\\\citealp{domke_provable_2023}} \\
     % & \ding{52} &  &  \ding{52} &   &           & loc. \& scale & CFE, STL & \(\mathcal{O}\left( \right)\) &  \citealp{domke_provable_2023} \\
    \textcolor{black!20}{\ding{52}} & \textcolor{black!20}{\ding{52}} & \ding{52} & \ding{52} &           &           & loc. \& scale & STL & \(\mathcal{O}\left(\kappa^2 \epsilon^{-1} \right)\) & \citealp{domke_provable_2023} \\
    \rowcolor{blue!10}
    \textcolor{black!20}{\ding{52}} & \textcolor{black!20}{\ding{52}} & \ding{52} & \ding{52} &           &           & loc. \& scale & STL   & \(\mathcal{O}\left(\kappa^2 \epsilon^{-1} \right)\) & \cref{thm:projsgd_bbvistl_decstepsize_complexity}\\
    \rowcolor{blue!10}
    \textcolor{black!20}{\ding{52}} & \textcolor{black!20}{\ding{52}} & \ding{52} & \ding{52} &           & \ding{52} & loc. \& scale & STL   & \(\mathcal{O}\left(\kappa^2 \log \epsilon^{-1} \right)\) &  \cref{thm:projsgd_bbvistl_complexity}
    %
    \\ \bottomrule
\end{tabular}
\begin{tablenotes}
\item[*] PL: Polyak-\L{}ojasiewicz, LC: log-concave, SLC: strongly-log-concave, LQ: log-quadratic (\(\pi\) is Gaussian), \(\kappa = L/\mu\).
\item[*] Analyses that a-priori assumed regularity of the ELBO were omitted.
\item[*] The explicit dimension dependences are omitted, but in general, \(\mathcal{O}\left(d\right)\) for full-rank, which is tight~\citep{domke_provable_2019}, and the best known for mean-field is {\scriptsize\(\mathcal{O}\left(\sqrt{d}\right)\)}~\citep{kim_practical_2023}.
The algorithm of \citet{bhatia_statistical_2022} is able to trade the dimension dependence for statistical accuracy.
\item[1] ``The variational family is perfectly specified.''
\item[2] The precise definitions of the gradient estimators are in \cref{section:gradient_estimators}.
\item[3] This algorithm uses stochastic power method-like iterations.
\item[4] The per-iteration sample complexity also depends on \(L, d, \epsilon\).
\end{tablenotes}
\end{threeparttable}
  }%
\end{table*}
%\end{wraptable}

%%% Local Variables:
%%% TeX-master: "main"
%%% End:


In this work, we confirm these previous comments by establishing a linear convergence rate of BBVI with STL when the variational family is perfectly specified.
For a \(d\)-dimensional strongly log-concave posterior with a condition number of \(\kappa\) and a location-scale variational family with a full rank scale, BBVI with the STL estimator finds an \(\epsilon\)-accurate variational parameters at a rate of \(\mathcal{O}\left(d \kappa^2 \log\left(1/\epsilon\right) \right)\).
Not only this, our theoretical results generally encompass the behavior of the STL estimator in the misspecified setting, which is closer to practical usage.
This provides some intuition as why the comparisons between the STL and ``standard'' closed-form entropy (CFE;~\citealp{titsias_doubly_2014,kucukelbir_automatic_2017}) estimators appear mixed in practice~\citep{geffner_rule_2020,agrawal_advances_2020}.

\vspace{-1ex}
\paragraph{Contributions} Overall, our contributions are summarized in the following list.
An overview of the theorems is provided in \cref{table:theorems}.
We also provide an overview of previous rigorous complexity analyses on BBVI in \cref{table:relatedworks}.
\begin{enumerate}
    \item[\ding{182}] \textbf{We prove that BBVI with the STL estimators can converge at a linear rate.} \\
      When the variational family is perfectly specified such that the posterior is contained in the variational family, \cref{thm:projsgd_bbvistl_complexity} establishes this through \cref{thm:stl_upperbound}.
      This is the first result for ``full'' BBVI without algorithmic simplification. 
      
    \item[\ding{183}] \textbf{Our analysis encompasses the case where the variational family is misspecified.}
      When the variational family is misspecified, the Fisher divergence between the variational posterior and the true posterior captures the behavior of the STL estimator.
    
    \item[\ding{184}] \textbf{We improve previously obtained gradient variance bounds for the CFE estimator} \\
      In \cref{thm:cfe_upperbound}, we tighten the constants of the bounds previously obtained by \citet{domke_provable_2023}.
      This makes the theoretical results for the CFE and STL estimators comparable.
      
    \item[\ding{185}] \textbf{We prove precise complexity guarantees for SGD with QV gradient estimators.}
      Specifically, we prove precise complexity guarantees from the ``anytime convergence'' results of \citet{domke_provable_2023}.
\end{enumerate}



\section{Preliminaries}

{%\footnotesize
\paragraph{Notation}
Random variables are denoted in serif (\textit{e.g.}, \(\rvx\), \(\rvvx\)), vectors are in bold (\textit{e.g.}, \(\vx\), \(\rvvx\)), and matrices are in bold capitals (\textit{e.g.} \(\mA\)).
For a vector \(\vx \in \mathbb{R}^d\), we denote the inner product as \(\vx^{\top}\vx\) and \(\inner{\vx}{\vx}\), the \(\ell_2\) norm as \(\norm{\vx}_2 = \sqrt{\vx^{\top}\vx}\).
For a matrix {\footnotesize\(\mA\), \(\norm{\mA}_{\mathrm{F}} = \sqrt{\mathrm{tr}\left(\mA^{\top} \mA\right)}\)} denotes the Frobenius norm, and for some matrix \(\mB\), \(\mA \succeq \mB\) is the Loewner order implying that \(\mA - \mB\) is a positive semi-definite matrix.
\(\mathbb{S}^d\), \(\mathbb{S}^d_{++}\), \(\mathrm{GL}_d\left(\mathbb{R}\right)\) are the set of symmetric, positive definite, and lower triangular matrices. 
}

\subsection{Variational Inference}

Variational inference (VI,~\citealp{zhang_advances_2019, blei_variational_2017, jordan_introduction_1999}) aims to minimize the exclusive (or backward/reverse) Kullback-Leibler (KL) divergence as:
%
{\begingroup
%\setlength{\belowdisplayskip}{1.5ex} \setlength{\belowdisplayshortskip}{1.5ex}
%\setlength{\abovedisplayskip}{1.5ex} \setlength{\abovedisplayshortskip}{1.5ex}
\begin{align*}
    \minimize_{\vlambda \in \Lambda}\; \mathrm{D}_{\mathrm{KL}}\left(q_{\vlambda}, \pi\right)
    \triangleq
    \mathbb{E}_{\rvvz \sim q_{\vlambda}} -\log \pi \left(\rvvz\right) -\mathbb{H}\left(q_{\vlambda}\right),
\end{align*}
\endgroup}
%
\begin{center}
  %\vspace{-2ex}
  {\begingroup
  %  \setlength\tabcolsep{1.0ex} 
  \begin{tabular}{lll}
    where 
    & \(\mathrm{D}_{\mathrm{KL}}\) & is the KL divergence, \\
    & \(\mathbb{H}\)     & is the differential entropy, \\
    & \(\pi\) & is the (target) posterior distribution, and  \\
    & \(q_{\vlambda}\) & is the variational approximation. \\
  \end{tabular}
  \endgroup}
\end{center}

For Bayesian posterior inference, the KL divergence is, unfortunately intractable.
Instead, one equivalently minimizes the negative \textit{evidence lower bound} (ELBO,~\citealp{jordan_introduction_1999}) \(F\) such that:
{%
%\setlength{\belowdisplayskip}{1.5ex} \setlength{\belowdisplayshortskip}{1.5ex}
%\setlength{\abovedisplayskip}{1.5ex} \setlength{\abovedisplayshortskip}{1.5ex}
\[
  \minimize_{\vlambda \in \Lambda}\; F\left(\vlambda\right)
  \triangleq
  \mathbb{E}_{\rvvz \sim q_{\vlambda}} -\log p \left(\rvvz, \vx\right) - \mathbb{H}\left(q_{\vlambda}\right),
\]
}%
where \(p\left(\vz, \vx\right)\) is the \textit{joint likelihood}, which is proportional to the posterior \(\pi\left(\vz\right)\) up to a multiplicative constant.

\paragraph{Black-Box Variational Inference}
Black-box variational inference (BBVI; \citealp{ranganath_black_2014,titsias_doubly_2014,kucukelbir_automatic_2017}) minimizes \(F\) by leveraging stochastic gradient descent (SGD; \citealp{robbins_stochastic_1951,bottou_online_1999,nemirovski_robust_2009}).
By obtaining a stochastic estimate \(\rvvg\left(\vlambda\right)\) which is \textit{unbiased}  as \(\mathbb{E} \rvvg\left(\vlambda\right) = \nabla F\left(\vlambda\right)\), BBVI repeats the update:
\[
  \vlambda_{t+1} = \mathrm{proj}\left( \vlambda_t - \gamma_t \rvvg \right), 
\]
where \(\gamma_t\) is called the stepsize.
The use of the projection operator \(\mathrm{proj}\left(\cdot\right)\) forms a subset of the broad SGD framework called \textit{projected} SGD.
The convergence of BBVI with projected SGD has recently been established by \citet{domke_provable_2023}.

\paragraph{Fisher Divergence}
In addition to the KL divergence, our analysis involves the Hyv\"arinen-Fisher divergence~\citep{otto_generalization_2000,hyvarinen_estimation_2005}.
\begin{definition}[\textbf{Fisher-Hyv\"arinen Divergence}]
\[
  \mathrm{D}_{\mathrm{F}^p}\left(q, \pi\right)
  \triangleq 
  \mathbb{E}_{\rvvz \sim q} 
  \norm{\nabla \log \pi\left(\rvvz\right) - \nabla \log q\left(\rvvz\right) }_2^p
\]
\end{definition}
Here, we use the \(p\)th order generalization~\citep{huggins_practical_2018} of the original Hyv\"arinen-Fisher divergence. 
We denote the standard 2nd order Hyv\"arinen-Fisher divergence as \(\mathrm{D}_{\mathrm{F}}\left(q, \pi\right) \triangleq \mathrm{D}_{\mathrm{F}^2}\left(q, \pi\right)\).
The Hyv\"arinen-Fisher divergence was first defined by~\citep{otto_generalization_2000} (attributed by \citealp{zegers_fisher_2015}) as the \textit{relative Fisher information} in the context of optimal transport.
It was later introduced to the machine learning community by \citet{hyvarinen_estimation_2005} for score-matching variational inference.

\subsection{Variational Family}
Throughout this paper, we restrict our interest to the location-scale variational family.
The location-scale variational family has been successfully used by \citet{kim_practical_2023,kim_blackbox_2023,domke_provable_2019,domke_provable_2020,domke_provable_2023, fujisawa_multilevel_2021} for analyzing the properties of BBVI.
It encompasses many practically used variational families such as the Gaussian, Student-t, and other elliptical distributions.
In particular, the location-scale family is part of the broader reparameterized family:
%
\begin{definition}[\textbf{Reparameterized Family}]\label{def:family}
  Let \(\varphi\) be some \(d\)-variate distribution.
  Then, \(q_{\vlambda}\) that can be equivalently represented as
{%
\setlength{\belowdisplayskip}{-.5ex} \setlength{\belowdisplayshortskip}{-.5ex}
\setlength{\abovedisplayskip}{-.5ex} \setlength{\abovedisplayshortskip}{-.5ex}
  \begin{alignat*}{2}
    \rvvz \sim q_{\vlambda}  \quad\Leftrightarrow\quad &\rvvz \stackrel{d}{=} \mathcal{T}_{\vlambda}\left(\rvvu\right); \quad \rvvu \sim  \varphi,
  \end{alignat*}
  }%
  where \(\stackrel{d}{=}\) is equivalence in distribution, is said to be part of a reparameterized family generated by the base distribution \(\varphi\) and the reparameterization function \(\mathcal{T}_{\vlambda}\).
\end{definition}
%
Naturally, this means we focus on the \textit{reparameterization gradient estimator}, often observed to achieve lower variance than alternatives~\citep{xu_variance_2019}.  
(See the overview of \citealt{mohamed_monte_2020} on other estimators.)
From this, we obtain the location-scale family by defining the location-scale reparameterization function:
%
\begin{definition}[\textbf{Location-Scale Reparameterization Function}]\label{def:reparam}
  A mapping \(\mathcal{T}_{\vlambda} : \mathbb{R}^p \times \mathbb{R}^d \rightarrow \mathbb{R}^d\) defined as
{
\setlength{\belowdisplayskip}{.5ex} \setlength{\belowdisplayshortskip}{.5ex}
\setlength{\abovedisplayskip}{1.ex} \setlength{\abovedisplayshortskip}{1.ex}
  \begin{align*}
    &\mathcal{T}_{\vlambda}\left(\vu\right) \triangleq \mC \vu + \vm
  \end{align*}
}%
  with \(\vlambda \in \mathbb{R}^p\) containing the parameters for forming the location \(\vm \in \mathbb{R}^d\) and scale \(\mC \in \mathbb{R}^{d \times d}\) is called the location-scale reparameterization function.
\end{definition}
%
For the scale matrix \(\mC\), various parameterizations are used in practice, as shown by \citet[Table 1]{kim_practical_2023}.
We will discuss our scale parameterizations of choice in \cref{section:scale_parameterization}.

The choice for the base distribution \(\varphi\) completes the specifics of the variational family.
For example, choosing \(\varphi\) to be a univariate Gaussian result in the Gaussian variational family.
We impose the following general assumptions on the base distribution:
\begin{assumption}[\textbf{Base Distribution}]\label{assumption:symmetric_standard}
  \(\varphi\) is a \(d\)-dimensional distribution such that \(\rvvu \sim \varphi\) and \(\rvvu = \left(\rvu_1, \ldots, \rvu_d \right)\) with indepedently and identically distributed components.
  Furthermore, \(\varphi\) is
  \begin{enumerate*}[label=\textbf{(\roman*)}]
      \item symmetric and standardized such that \(\mathbb{E}\rvu_i = 0\), \(\mathbb{E}\rvu_i^2 = 1\), \(\mathbb{E}\rvu_i^3 = 0\), and 
      \item has finite kurtosis \(\mathbb{E}\rvu_i^4 = \kappa < \infty\).
  \end{enumerate*}
\end{assumption}

Overall, the assumptions on the variational family are collected as follows:
\begin{assumption}\label{assumption:variation_family}
  The variational family is the location-scale family formed by \cref{def:family,def:reparam} with the base distribution \(\varphi\) satisfying \cref{assumption:symmetric_standard}.
\end{assumption}

\subsection{Scale Parameterization}\label{section:scale_parameterization}
For the scale parameterization \(\vlambda \mapsto \mC\), in principle, any choice that results in a positive-definite covariance matrix is valid.
However, recently, \citet{kim_blackbox_2023} have shown that a seemingly innocent choice of parameterization can have a massive impact on computational performance.
For example, nonlinear parameterizations can easily break the strong convexity of the ELBO~\citep{kim_blackbox_2023}, which could have been otherwise obtained~\citep{domke_provable_2020}.
Therefore, the scale parameterization is subject to the constraints:
\begin{enumerate}
  \item[\ding{182}] \textbf{Positive Definiteness}: \(\mC\mC^{\top} \succ 0\).  \\
    This is needed to ensure that \(\mC\mC^{\top}\) forms a valid covariance in \(\mathbb{S}_{++}^d\).
    %
  \item[\ding{183}] \textbf{Linearity}: \(\norm{\vlambda - \vlambda'}_2^2 = \norm{\vm - \vm'}_2^2 + \norm{ \mC - \mC'}_{\mathrm{F}}^2 \).  \\
    As shown by \citet{kim_blackbox_2023}, this constraint is necessary to form a strongly-convex ELBO from a strongly log-concave posterior.
    
  \item[\ding{184}] \textbf{Convexity}: \textit{The mapping \(\vlambda \to \mC\mC^{\top}\) is convex on \(\Lambda_{S}\)}.\\
    This is needed to ensure that the ELBO is convex whenever the target posterior is log-concave \citep{domke_provable_2023,kim_blackbox_2023}.
\end{enumerate}
These constraints can be met by the following choices that have been commonly used:
\begin{alignat*}{4}
  \mC &= \mB, 
  &&\qquad
  \text{(full-rank)}
  &&\qquad
  \\
  \mC &= \mathrm{diag}\left(L_{11}, \ldots, L_{dd}\right)
  &&\qquad
  \text{(mean-field)}
\end{alignat*}
where \(\mB \in \mathbb{S}^d\) is an invertible symmetric matrix.
Under the full-rank parameterization, \(\mB\) is a proper matrix square root of the covariance \(\mSigma = \mC \mC^{\top}\) such that \(\mB = \mB^{\top} = \mSigma^{\nicefrac{1}{2}}\).

\paragraph{Inapplicability of Triangular Scale}
Notice that the full-rank scale is chosen to be dense instead of triangular.
In practice, setting \(\mC\left(\vlambda\right) = \mL\), where 
\(\mL \in \mathrm{GL}_{d}\left(\mathbb{R}\right)\) is a lower triangular matrix, is also a valid option, and in fact results in lower gradient variance \citep{kim_practical_2023}.
Furthermore, using the projection operator
{%
\setlength{\belowdisplayskip}{1.5ex} \setlength{\belowdisplayshortskip}{1.5ex}
\setlength{\abovedisplayskip}{1.5ex} \setlength{\abovedisplayshortskip}{1.5ex}
\[
  \mathrm{proj}\left(\vlambda\right)
  =
  \left(\vm, \widetilde{\mL}\right), 
  \quad\text{where}\quad
  \widetilde{L}_{ii} = \max\left(L_{ii}, \epsilon\right), 
\]
}%
for some constant \(0 < \epsilon < \infty\) ensures that \(\mC \mC^{\top}\) is always strictly positive definite with only \(\mathcal{O}\left(d\right)\) computational cost.
Unfortunately, this parameterization cannot be used to obtain a theoretical guarantee.
This is because, to enforce that the ELBO is smooth in combination with constraint \ding{183}, we need the additional constraint:
%
\begin{enumerate}
  \item[\ding{185}] \textbf{Bounded entropy}: \(\mC\mC^{\top} \succ S^{-1} \boldupright{I}\),
\end{enumerate}
which was originally proposed by \citet{domke_provable_2020,domke_provable_2023}.
%
For this, the domain of \(\vlambda\) is restricted to 
{%
\setlength{\belowdisplayskip}{1.ex} \setlength{\belowdisplayshortskip}{1.ex}
\setlength{\abovedisplayskip}{1.ex} \setlength{\abovedisplayshortskip}{1.ex}
\[
  \Lambda_{S} \triangleq \left\{\, \left(\vm, \mC\right) \mid \vm \in \mathbb{R}^{d}, \mC \mC^{\top} \in \mathbb{S}_{++}^d\; \text{such that}\; \mC \mC^{\top} \succeq 
 S^{-1} \boldupright{I} \,\right\}.
\]
}%
\citeauthor{domke_provable_2023} (\citeyear{domke_provable_2020,domke_provable_2023}) went to chose \(S = L\), where \(L\) is the log-smoothness constant of the posterior.
The problem with this domain constraint is that a projection operator for \(\Lambda_{S}\) that preserves the triangular structure appears challenging to devise.
Thus, we are restricted to the less efficient matrix square root parameterization.

\paragraph{Projection Operator}
For the projection operator, \citet{domke_provable_2023} propose multiple operators.
But for the purpose of this paper, any projection operator to \(\Lambda_{S}\) suffices. 
We will mention here the operator based on the singular value decomposition (SVD), which is the one we used for the experiments:
{%
\setlength{\belowdisplayskip}{1.ex} \setlength{\belowdisplayshortskip}{1.ex}
\setlength{\abovedisplayskip}{1.ex} \setlength{\abovedisplayshortskip}{1.ex}
\[
  \mathrm{proj}_{\Lambda_S}\left(\vlambda\right)
  =
  \left(\vm, \mU \widetilde{\mD} \mU^{\top}\right), 
  \quad\text{where}\quad
  \widetilde{D}_{ii} = \max\left(D_{ii}, \sqrt{1/S}\right),
\]
}%
and \(\mC = \mU \mD \mV^{\top}\) is the singular value decomposition of \(\mC\).

\subsection{Gradient Estimators}\label{section:gradient_estimators}
The gradient estimators considered in this work are the closed-form entropy (CFE;~\citealp{kucukelbir_automatic_2017}) and sticking the landing (STL;~\citealp{roeder_sticking_2017}) estimators.

\paragraph{Closed-From Entropy Estimator}
The CFE estimator is the ``standard'' estimator used for BBVI, defined as the following:
%
\begin{definition}[\textbf{Closed-Form Entropy Estimator}]\label{def:cfe}
The closed-form entropy gradient estimator is
\[
  \rvvg\left(\vlambda\right) 
  \triangleq
  \nabla_{\vlambda} \log \pi \left(\mathcal{T}_{\vlambda}\left(\rvvu\right)\right) + \nabla_{\vlambda} \mathbb{H}\left(q_{\vlambda}\right),
\]
where the gradient of the entropy term is computed deterministically.
\end{definition}
%
It can be used whenever the entropy \(\mathbb{H}\left(q_{\vlambda}\right)\) is available in a closed form.

\paragraph{Sticking-the-Landing Estimator}
On the other hand, the STL estimator estimates the entropy term stochastically.
%
\begin{definition}[\textbf{Sticking-the-Landing Estimator; STL}]\label{def:stl}
The sticking-the-landing gradient estimator 
\[
  \rvvg_{\mathrm{STL}}\left(\vlambda\right) 
  \triangleq
  \nabla_{\vlambda} \log \pi \left(\mathcal{T}_{\vlambda}\left(\rvvu\right)\right) - \nabla_{\vlambda} \log q_{\vnu}\left(\mathcal{T}_{\vlambda}\left(\rvvu\right)\right) \Big\lvert_{\vnu = \vlambda}
\]
is given by stopping the gradient from propagating through \(\log q_{\vlambda}\).
\end{definition}
%
Notice that, the gradient of \(\log q\) is ``stopped'' by the assignment \(\vnu = \vlambda\).
This essentially creates a control variate effect, where the control variate \(\mathrm{cv}\left(\vlambda\right)\) is implicitly formed as
\[
  \mathrm{cv}\left(\vlambda\right)
  =
  \nabla_{\vlambda} \mathbb{H}\left(\vlambda\right)
  +
  \nabla_{\vlambda} \log q_{\vnu}\left(\mathcal{T}_{\vlambda}\left(\rvvu\right)\right) \Big\lvert_{\vnu = \vlambda}.
\]
Subtracting this to the CFE estimator leads to the STL estimator.

\subsection{Quadratic Variance Condition}
The convergence of BBVI has recently been established concurrently by \citet{kim_blackbox_2023,domke_provable_2023}.
However, the analysis of \citeauthor{domke_provable_2023} presents a broadly applicable framework based on the \textit{quadratic variance} (QV) condition.
%
\begin{definition}[\textbf{Quadratic Variance; QV}]
  A gradient estimator \(\rvvg\) is said to satisfy the quadratic variance condition if the following bound holds:
  \[
    \mathbb{E}\norm{\rvvg\left(\vlambda\right)}_2^2 \leq \alpha \norm{\vlambda - \vlambda^*}_2^2 + \beta
  \] 
  for any \(\vlambda \in \Lambda_S\) and some \(0 \leq \alpha, \beta < \infty\).
\end{definition}
%
This basically assumes that the gradient variance grows no more than a quadratic plus a constant.
For the analysis of SGD, this bound was first proposed by~\citet[p. 85]{wright_optimization_2021}, but a more comprehensive convergence analysis, including proximal SGD, was conducted by \citet{domke_provable_2023}.
This work will connect with their analysis by establishing the QV condition of the considered gradient estimators.

\subsection{Interpolation Condition}
To establish the linear, or more intuitively ``exponential'', convergence of SGD, \citet{schmidt_fast_2013} have relied on the interpolation condition:
\begin{definition}[\textbf{Interpolation}]
  A gradient estimator \(\rvvg\) is said to satisfy the interpolation condition if  \(\mathbb{E}\norm{\rvvg\left(\vlambda^*\right)}_2^2 = 0\) for \(\vlambda^* = \argmin_{\vlambda \in \Lambda} F\left(\vlambda\right)\).
\end{definition}
This assumes that the gradient variance vanishes at the optimum, gradually retrieving the convergence behavior of deterministic gradient descent.
For the QV condition, this corresponds to \(\beta = 0\).

\paragraph{Achieving ``Interpolation''}
Currently, there are two ways where the interpolation condition can be achieved.
The first case is when interpolation is achieved \textit{naturally}.
That is, in ERM, when the model is so overparameterized that certain parameters can ``interpolate'' all of the data points in the train data~\citep{ma_power_2018,vaswani_fast_2019}, the gradient becomes 0 at such point.
Otherwise, a control variate approach such as stochastic average gradient (SAG; \citealp{schmidt_minimizing_2017}) or stochastic variance-reduced gradient (SVRG; \citealp{johnson_accelerating_2013}), and their many variants~\citep{gower_variancereduced_2020} can be used.
We will collectively refer to these methods as variance-reduced SGD methods.

\paragraph{Does STL ``Interpolate?''}
As we previously discussed, the STL estimator is essentially a control variate method.
Thus, an important question is whether it can achieve the same effect, notably linear convergence, as the variance-reduced SGD methods. 
While \citet{roeder_sticking_2017} have already shown that the STL estimator achieves interpolation when \(q_{\vlambda} = \pi\), our research question is whether this fact can be rigorously used to establish linear convergence of SGD.


\vspace{-1ex}
\section{MAIN RESULTS}
\vspace{-1ex}
\subsection{Theoretical Analysis of the STL Estimator}\label{section:gradient_variance}
\vspace{-1ex}
\subsubsection{Proof Sketch}
\vspace{-1ex}
Before presenting our analysis on BBVI gradient estimators, we will discuss a notable aspect of our strategy and the key step in our proof.

\vspace{-1ex}
\paragraph{Adaptive Bounds with the Peter-Paul Inequality}
Unlike the QV bounds obtained by \citet{domke_provable_2023}, our bounds involve a free parameter \(\delta \geq 0\).
We call these bounds \textit{adaptive} QV bounds.
%
\begin{assumption}[\textbf{Adaptive QV}]\label{assumption:adaptiveqvc}
  The gradient estimator \(\rvvg\) satisfies the bound
{%
\setlength{\belowdisplayskip}{1.ex} \setlength{\belowdisplayshortskip}{1.ex}
\setlength{\abovedisplayskip}{1.ex} \setlength{\abovedisplayshortskip}{1.ex}
  \[
    \norm{\rvvg\left(\vlambda\right)}_2^2 \leq (1 + C \delta) \, \widetilde{\alpha} \, \norm{\vlambda - \vlambda^*}_2^2 + (1 + C^{-1} \delta^{-1}) \,\widetilde{\beta},
  \]
  }%
  for any \(\delta > 0\), any \(\vlambda \in \Lambda_S\), and some \(0 < \widetilde{\alpha}, \widetilde{\beta} < \infty\), where \(\vlambda^*\) is a stationary point.
\end{assumption}
%
This is a consequence of the use of the ``Peter-Paul'' inequality such that
{%
\setlength{\belowdisplayskip}{1.ex} \setlength{\belowdisplayshortskip}{1.ex}
\setlength{\abovedisplayskip}{1.ex} \setlength{\abovedisplayshortskip}{1.ex}
\begin{equation}
  {\left( a + b \right)}^2 \leq \left(1 + \delta\right) \,a^2 + (1 + \delta^{-1}) \,b^2,
  \label{eq:peterpaul}
\end{equation}
}%
and can be seen as a generalization of the usual inequality \({\left(a + b\right)}^2 \leq 2 a^2 + 2 b^2\).
As mentioned in \citet[Remark 6]{kim_practical_2023}, adjusting \(\delta\) can occasionally tighten the analysis.
In fact, \(\delta\) can be optimized to become \textit{adaptive} to the downstream analysis.
Indeed, in our complexity analysis, \(\delta\) automatically trades-off the influence of \(\widetilde{\alpha}\) and \(\widetilde{\beta}\) according to the accuracy budget \(\epsilon\).

\vspace{-1ex}
\paragraph{Key Lemma}
The key first step in all of our analysis is the following decomposition:

\begin{theoremEnd}[category=gradvarlemmas,all end]{lemma}\label{thm:peterpaul}
  For any \(a, b, c, \in \mathbb{R}\), 
  \[ {(a+b+c)}^2 \leq (2 + \delta ) a^2 + (2 + \delta ) b^2 + (1 + 2 \delta^{-1})  c^2, \]
  for any \(\delta > 0\).
\end{theoremEnd}
\begin{proofEnd}
The Peter-Paul generalization of Young's inequality states that, for \(d, e \geq 0\), we have
\[
   d e \leq \frac{\delta }{2} d^2 + \frac{\delta^{-1}}{2} e^2.
\]
Applying this,
\begin{align*}
  {(a+b+c)}^2
  &\;=
  a^2 + b^2 + c^2 + 2 a b + 2 a c + 2 b c
  \\
  &\;\leq
    a^2 + b^2 + c^2 + 2 \abs{a} \abs{b} + 2 \abs{a} \abs{c} + 2 \abs{b} \abs{c}
  \\
  &\;\leq
  a^2 + b^2 + c^2 
  + 2 \left( \frac{1}{2} a^2 + \frac{1}{2} b^2 \right)
  + 2 \left( \frac{\delta}{2} a^2 + \frac{\delta^{-1}}{2} c^2 \right)
  + 2 \left( \frac{\delta}{2} b^2 + \frac{\delta^{-1}}{2} c^2 \right) 
  \\
  &\;=
  a^2 + b^2 + c^2 + \left( a^2 +  b^2 \right)
  + \left( \delta a^2 + \delta^{-1} c^2 \right)
  + \left( \delta b^2 + \delta^{-1} c^2 \right) 
  \\
  &\;=
  \left(2 + \delta\right) a^2 + \left(2 + \delta\right) b^2 + \left(1 + 2 \delta^{-1}\right) c^2.
\end{align*}
\end{proofEnd}


\begin{theoremEnd}[category=stlupperboundlemma]{lemma}\label{thm:stl_decomposition}
{\setlength{\belowdisplayskip}{1.5ex} \setlength{\belowdisplayshortskip}{1.5ex}
\setlength{\abovedisplayskip}{1.ex} \setlength{\abovedisplayshortskip}{1.ex}
%
  Assume \cref{assumption:variation_family}.
  The expected-squared norm of STL is bounded as
  {\small
  \begin{align*}
    \mathbb{E} \norm{\rvvg_{\mathrm{STL}}\left(\vlambda\right)}_2^2
    \leq
    \left( 2 + \delta \right) V_{1}
    +
    \left( 2 + \delta \right) V_{2}
    +
    \left( 1 + 2 \delta^{-1} \right) V_{3},
  \end{align*}
  }%%
  where the terms are
  {\small
  \begin{align*}
    V_{1}
    &=
    \mathbb{E} \,
    J_{\mathcal{T}}\left(\rvvu\right) 
    \lVert
    \nabla \log \ell \left(\mathcal{T}_{\vlambda}\left(\rvvu\right)\right) 
    -
    \nabla \log \ell \left(\mathcal{T}_{\vlambda^*}\left(\rvvu\right)\right) 
    {\rVert}_2^2
    \\
    V_{2}
    &=
    \mathbb{E} \,
    J_{\mathcal{T}}\left(\rvvu\right)
    \lVert
    \nabla \log q_{\vlambda^*}\left(\mathcal{T}_{\vlambda^*}\left(\rvvu\right)\right)
    - 
    \nabla \log q_{\vlambda}\left(\mathcal{T}_{\vlambda}\left(\rvvu\right)\right)
    {\rVert}_2^2
    \\
    V_{3}
    &=
    \mathbb{E} \,
    J_{\mathcal{T}}\left(\rvvu\right)
    \lVert
    \nabla \log \ell\left(\mathcal{T}_{\vlambda^*}\left(\rvvu\right)\right)
    - 
    \nabla \log q_{\vlambda^*}\left(\mathcal{T}_{\vlambda^*}\left(\rvvu\right)\right)
    {\rVert}_2^2,
  \end{align*}
  }%%
  for any \(\delta > 0\) and \(\vlambda \in \mathbb{R}^p\).
  \(J_{\mathcal{T}} : \mathbb{R}^d \to \mathbb{R}\) is a function depending on the variational family as
  \begin{alignat*}{2}
    J_{\mathcal{T}}\left(\rvvu\right) &= 1 + {\textstyle\sum^{d}_{i=1} \rvu_i^2} \quad                        &&\quad \text{for full-rank and} \\
    J_{\mathcal{T}}\left(\rvvu\right) &= 1 + {\textstyle\sqrt{\sum^{d}_{i=1} \rvu_i^4}} &&\quad \text{for mean-field.}
  \end{alignat*}
 }%
\end{theoremEnd}
\begin{proofEnd}
From the definition of the STL estimator~\cref{def:stl},
\begin{align*}
    \mathbb{E} \norm{\rvvg_{\mathrm{STL}}\left(\vlambda\right)}_2^2
    &=
    \mathbb{E} \norm{
    \nabla_{\vlambda} \log \ell \left(\mathcal{T}_{\vlambda}\left(\rvvu\right)\right) - \nabla_{\vlambda} \log q_{\vnu}\left(\mathcal{T}_{\vlambda}\left(\rvvu\right)\right)
    }_2^2
    \;\Big\lvert_{\vnu = \vlambda},
\shortintertext{by \cref{thm:jacobian_reparam_inner},}
    &=
    \mathbb{E}
    J_{\mathcal{T}}\left(\rvvu\right)
    \norm{
    \nabla \log \ell \left(\mathcal{T}_{\vlambda}\left(\rvvu\right)\right) 
    - \nabla \log q_{\vnu}\left(\mathcal{T}_{\vlambda}\left(\rvvu\right)\right)
    }_2^2
    \;\Big\lvert_{\vnu = \vlambda}
\shortintertext{adding the terms \(\nabla \log \ell \left(\mathcal{T}_{\vlambda^*}\left(\rvvu\right)\right)\) and \(\nabla \log q_{\vlambda^*}\left(\mathcal{T}_{\vlambda^*}\left(\rvvu\right)\right)\) that cancel,}
    &=
    \mathbb{E}
    J_{\mathcal{T}}\left(\rvvu\right)
    \lVert
    \nabla \log \ell \left(\mathcal{T}_{\vlambda}\left(\rvvu\right)\right) 
    -
    \nabla \log \ell \left(\mathcal{T}_{\vlambda^*}\left(\rvvu\right)\right) 
    \\
    &\quad\qquad\qquad+
    \nabla \log \ell\left(\mathcal{T}_{\vlambda^*}\left(\rvvu\right)\right)
    - 
    \nabla \log q_{\vlambda^*}\left(\mathcal{T}_{\vlambda^*}\left(\rvvu\right)\right)
    \\
    &\quad\qquad\qquad+
    \nabla \log q_{\vlambda^*}\left(\mathcal{T}_{\vlambda^*}\left(\rvvu\right)\right)
    - 
    \nabla \log q_{\vlambda}\left(\mathcal{T}_{\vlambda}\left(\rvvu\right)\right)
    {\rVert}_2^2,
\shortintertext{applying \cref{thm:peterpaul},}
    &\leq
    \mathbb{E}
    J_{\mathcal{T}}\left(\rvvu\right)
    \Big(
    \left( 2 + \delta \right)
    \lVert
    \nabla \log \ell \left(\mathcal{T}_{\vlambda}\left(\rvvu\right)\right) 
    -
    \nabla \log \ell \left(\mathcal{T}_{\vlambda^*}\left(\rvvu\right)\right) 
    {\rVert}_2^2
    \\
    &\quad\qquad\qquad+
    \left( 2 + \delta \right)
    \lVert
    \nabla \log q_{\vlambda^*}\left(\mathcal{T}_{\vlambda^*}\left(\rvvu\right)\right)
    - 
    \nabla \log q_{\vlambda}\left(\mathcal{T}_{\vlambda}\left(\rvvu\right)\right)
    {\rVert}_2^2
    \\
    &\quad\qquad\qquad+
    \left(1 + 2 \delta^{-1}\right)
    \lVert
    \nabla \log \ell\left(\mathcal{T}_{\vlambda^*}\left(\rvvu\right)\right)
    - 
    \nabla \log q_{\vlambda^*}\left(\mathcal{T}_{\vlambda^*}\left(\rvvu\right)\right)
    {\rVert}_2^2
    \Big),
\shortintertext{and distributing \(J_{\mathcal{T}}\) and the expectation,}
    &=
    \left( 2 + \delta \right)
    \underbrace{
    \mathbb{E}
    J_{\mathcal{T}}\left(\rvvu\right)
    \lVert
    \nabla \log \ell \left(\mathcal{T}_{\vlambda}\left(\rvvu\right)\right) 
    -
    \nabla \log \ell \left(\mathcal{T}_{\vlambda^*}\left(\rvvu\right)\right) 
    {\rVert}_2^2
    }_{V_{1}}
    \\
    &\quad+
    \left( 2 + \delta \right)  
     \underbrace{
    J_{\mathcal{T}}\left(\rvvu\right)
    \mathbb{E}
    \lVert
    \nabla \log q_{\vlambda^*}\left(\mathcal{T}_{\vlambda^*}\left(\rvvu\right)\right)
    - 
    \nabla \log q_{\vlambda}\left(\mathcal{T}_{\vlambda}\left(\rvvu\right)\right)
    {\rVert}_2^2
    }_{V_{2}}
    \\
    &\quad+
    \left(1 + 2 \delta^{-1}\right)  
    \underbrace{
    \mathbb{E}
    J_{\mathcal{T}}\left(\rvvu\right)
    \lVert
    \nabla \log \ell\left(\mathcal{T}_{\vlambda^*}\left(\rvvu\right)\right)
    - 
    \nabla \log q_{\vlambda^*}\left(\mathcal{T}_{\vlambda^*}\left(\rvvu\right)\right)
    {\rVert}_2^2
    }_{V_{3}}.
\end{align*}
\end{proofEnd}

Here, \(J_{\mathcal{T}}\) is a term that stems from the Jacobian of \(\mathcal{T}\).
Thus, \(J_{\mathcal{T}}\) contains the properties unique to the chosen variational family.
\(T_{\text{\ding{182}}}\) and \(T_{\text{\ding{183}}}\) measure how far the current variational approximation \(q_{\vlambda}\) is from a stationary point \({\vlambda^*}\).
Thus, both terms will eventually reach 0 as BBVI converges, regardless of family specification.
The key is \(T_{\text{\ding{184}}}\), which captures the amount of mismatch between the score of the true posterior \(\pi\) and variational posterior \(q_{\vlambda^*}\).
Establishing the ``interpolation condition'' amounts to analyzing when \(T_{\text{\ding{184}}}\) becomes 0.

\vspace{-1ex}
\subsubsection{Upper Bounds}
\vspace{-1ex}
We now present our upper bound on the expected-squared norm of the STL gradient estimator.
%
\input{thm_stl_upperbound}

\begin{remark}[\textbf{Mean-Field Variational Family}]
  We prove an equivalent result for the mean-field variational family, \cref{thm:stl_upperbound_mf} in \cref{section:stl_meanfield}, which has an \(\mathcal{O}\left(\sqrt{d}\right)\) dimensional dependence.
\end{remark}

\begin{remark}[\textbf{Interpolation Condition}]
  The encompasses both settings where the variational family is well-specified and misspecified.
  That is, when the variational family is well specified, \textit{i.e.}, \( \mathrm{D}_{\mathrm{F}^4}\left(q_{\vlambda^*}, \pi\right) = 0 \), we obtain interpolation such that \(\beta_{\mathrm{STL}} = 0\).
\end{remark}

\begin{remark}[\textbf{Adaptivity of Bound}]
  When the variational family is well specified such that \( \mathrm{D}_{\mathrm{F}^4}\left(q_{\vlambda^*}, \pi\right) = 0 \), we can adaptively tighten the bound by setting \(\delta = 0\), where \(\alpha_{\mathrm{STL}}\) is reduced by a constant factor.
\end{remark}

\vspace{-1ex}
\subsubsection{Lower Bounds}
\vspace{-1ex}
We also obtain lower bounds on the expected-squared norm of the STL estimator to analyze its best-case behavior and the tightness of the bound.

\vspace{-1ex}
\paragraph{Necessary Conditions for Interpolation}
First, we obtain lower bounds that generally hold for all \(\vlambda \in \Lambda_L\) and any \(\pi\).
Our analysis relates the gradient variance with the Fisher-Hyv\"arinen divergence.
This can be related back to the KL divergence through an assumption on the posterior \(\pi\) known as the log-Sobolev inequality.
The general form of the log-Sobolev inequality was originally proposed by \citet{gross_logarithmic_1975} to study diffusion processes.
In this work, we use the form used by \citet{otto_generalization_2000}:
%
\begin{assumption}[\textbf{Log-Sobolev Inequality; LSI}]
\(\pi\) is said to satisfy the log-Sobolev inequality if, for any variational family \(\mathcal{Q}\) and all \(q_{\vlambda} \in \mathcal{Q}\), the following inequality holds:
{%
\setlength{\belowdisplayskip}{1.ex} \setlength{\belowdisplayshortskip}{1.ex}
\setlength{\abovedisplayskip}{0ex} \setlength{\abovedisplayshortskip}{0ex}
\[
  \DKL{q}{\pi}
  \leq
  \frac{C_{\mathrm{LSI}}}{2} \, \DHF{q}{\pi}.
\]
}%
\end{assumption}
\vspace{-1ex}
%
Strongly log-concave distributions are known to satisfy the LSI, where the strong log-concavity constant becomes the (inverse) LSI constant. 
This is known as the Bakry-\'Emery Theorem~\citep{bakry_diffusions_1985}. (See also~\citealp[Theorem 9.9]{villani_topics_2016}).
\begin{remark}[\citealp{bakry_diffusions_1985}]
    Let \(\pi\) be \(\mu\)-strongly log-concave.
    Then, it satisfies the LSI with \(C_{\mathrm{LSI}}^{-1} = \mu\).
\end{remark}

We now present our lower bound which holds for all \(\vlambda \in \Lambda_S\) and any differentiable \(\pi\):
%

\begin{theoremEnd}[category=stllowerbound]{theorem}\label{thm:stl_lowerbound}
  Assume \cref{assumption:variation_family}.
  The expected-squared norm of the STL estimator is lower bounded as
  \[
    \mathbb{E} \norm{\rvvg_{\mathrm{STL}}\left(\vlambda\right)}_2^2 
    \geq
    \DHF{q_{\vlambda}}{\pi}
    \geq
    \frac{2}{C_{\mathrm{LSI}}} \DKL{q_{\vlambda}}{\pi},
  \]
  for all \(\vlambda \in \Lambda_{S}\) and any \(0 < S < \infty\), where the last inequality holds if \(\pi\) is LSI.
\end{theoremEnd}
\begin{proofEnd}
\begin{align*}
  \mathbb{E} \norm{\rvvg_{\mathrm{STL}}\left(\vlambda\right)}_2^2
  &=
  \mathbb{E} \norm{  
  \nabla_{\vlambda} \log \pi \left(\mathcal{T}_{\vlambda}\left(\rvvu\right)\right) - \nabla_{\vlambda} \log q_{\vnu}\left(\mathcal{T}_{\vlambda}\left(\rvvu\right)\right)
  }_2^2
  \;\Bigg\lvert_{\vnu = \vlambda},
\shortintertext{by \cref{thm:jacobian_reparam_inner},}
  &=
  \mathbb{E} 
  J_{\mathcal{T}}\left(\rvvu\right)
  \norm{
  \nabla \log \pi \left(\mathcal{T}_{\vlambda}\left(\rvvu\right)\right) 
  - \nabla \log q_{\vnu}\left(\mathcal{T}_{\vlambda}\left(\rvvu\right)\right)
  }_2^2
  \;\Big\lvert_{\vnu = \vlambda}
  \\
  &=
  \mathbb{E} 
  J_{\mathcal{T}}\left(\rvvu\right)
  \norm{
  \nabla \log \pi \left(\mathcal{T}_{\vlambda}\left(\rvvu\right)\right) 
  - \nabla \log q_{\vlambda}\left(\mathcal{T}_{\vlambda}\left(\rvvu\right)\right)
  }_2^2,
\shortintertext{since \(J_{\mathcal{T}}\left(\rvvu\right) \geq 1\) for both the full-rank and mean-field parameterizations,}
  &\geq
  \mathbb{E} 
  \norm{
  \nabla \log \pi \left(\mathcal{T}_{\vlambda}\left(\rvvu\right)\right) 
  - \nabla \log q_{\vlambda}\left(\mathcal{T}_{\vlambda}\left(\rvvu\right)\right)
  }_2^2,
\shortintertext{after Change-of-Variable,}
  &=
  \mathbb{E}_{\rvvz \sim q_{\vlambda}} 
  \norm{
  \nabla \log \pi \left(\rvvz\right) 
  - \nabla \log q_{\vlambda}\left(\rvvz\right)
  }_2^2
\shortintertext{by definition,}
  &=
  \DHF{q_{\vlambda}}{\pi},
\shortintertext{and when the log-Sobolev inequality applies,}
  &\geq
  \frac{2}{C_{\mathrm{LSI}}} \DKL{q_{\vlambda}}{\pi}.
\end{align*}
\end{proofEnd}

%%% Local Variables:
%%% TeX-master: "main"
%%% End:


\begin{corollary}[\textbf{Necessary Conditions for Interpolation}]
For the STL estimator, the interpolation condition does not hold if
\begin{enumerate}[label=\textbf{(\roman*)}]
  \vspace{-1.5ex}
  \setlength\itemsep{0.ex}
    \item \(\DHF{q_{\vlambda^*_{\mathrm{F}}}}{\pi} > 0\), or,
    \item when \(\pi\) is LSI, \(\DKL{q_{\vlambda^*_{\mathrm{KL}}}}{\pi} > 0\),
  \vspace{-1.5ex}
\end{enumerate}
%
  \begin{center}
   {\begingroup
    \setlength\tabcolsep{10pt} 
  \begin{tabular}{ll}
    \text{where }
    &
    \(\vlambda_{\mathrm{F}}^* \in \argmin_{\vlambda \in \Lambda_S} \DHF{q_{\vlambda}}{\pi} \),\; 
    \text{and} \\
    & \(\vlambda_{\mathrm{KL}}^* \in \argmin_{\vlambda \in \Lambda_S} \DKL{q_{\vlambda}}{\pi} \),
  \end{tabular}
  \endgroup}
  \end{center}
  \vspace{-1.5ex}
  for any \(0 < S < \infty\).
\end{corollary}
\vspace{-1ex}

\vspace{-1.ex}
\paragraph{Tightness Analysis}
The bound in \cref{thm:stl_lowerbound} is unfortunately not tight regarding the constants.
It, however, holds for all \(\vlambda\) and \(\pi\).
Instead, we establish an alternative lower bound that holds for a subset of \(\vlambda\) and \(\pi\) but is tight regarding the constants.
%

\begin{theoremEnd}[all end, category=gradvarlemmas]{lemma}\label{lemma:unorm_times_reparam}
  Let \(\mathcal{T}_{\vlambda}: \mathbb{R}^p \times \mathbb{R}^d \rightarrow \mathbb{R}^d\) be the location-scale reparameterization function (\cref{def:reparam}) and \(\rvvu \sim \varphi\) satisfy \cref{assumption:symmetric_standard}.
  Then,
  \[
  \mathbb{E} 
  \left(1 + \textstyle{\sum^{d}_{i=1} \rvu_i^2} \right)
  \left(
  \mathcal{T}_{\vlambda}\left(\rvvu\right) 
  +
  \vz
  \right)
  =
  \left(d+1\right)
  \left( \vm + \vz \right)
  \]
  for any \(\vz \in \mathbb{R}^d\).
\end{theoremEnd}
\vspace{-1ex}
\begin{proofEnd}
  \begin{align*}
  \mathbb{E} 
  \left(1 + \textstyle{\sum^{d}_{i=1} \rvu_i^2} \right)
  \left(
  \mathcal{T}_{\vlambda}\left(\rvvu\right) + \vz
  \right)
  &=
  \mathbb{E} 
  \left(1 + \norm{\rvvu}_2^2 \right)
  \left( \mC \rvvu + \vm + \vz \right)
  \\
  &=
  \mC \mathbb{E} \left(1 + \norm{\rvvu}_2^2 \right) \rvvu +  \mathbb{E}\left(1 + \norm{\rvvu}_2^2 \right) \left( \vm + \vz \right)
  \\
  &=
  \left(d+1\right) \left( \vm + \vz \right),
  \end{align*}
  where the last equality follows from \cref{thm:u_identities,assumption:symmetric_standard}.
\end{proofEnd}

\begin{theoremEnd}[all end, category=gradvarlemmas]{lemma}\label{thm:lowerbound_matrix_innerproduct_lemma}
  Let \(\mA = \mathrm{diag}\left(A_1, \ldots, A_d\right) \in \mathbb{R}^{d \times d}\) be some diagonal matrix, define
{%
\setlength{\belowdisplayskip}{1.ex} \setlength{\belowdisplayshortskip}{1.ex}
\setlength{\abovedisplayskip}{1.ex} \setlength{\abovedisplayshortskip}{1.ex}
  \[
  \mB = \begin{bmatrix}
    L^{-1/2} &   &        &   \\
             & L^{1/2} &        &   \\
             &   & \ddots &   \\
             &   &        & L^{1/2} \\
  \end{bmatrix},
  \qquad
  \mC = L^{-1/2} \, \boldupright{I},
  \]
  }%
  some \(\vu \in \mathbb{R}^d\), \(\vm \in \mathbb{R}^d\), and \(\vz \in \mathbb{R}^d \) such that \(m_1 = z_1\).
  For \(\vlambda = (\vm, \mC)\), the expression
{%
\setlength{\belowdisplayskip}{1.ex} \setlength{\belowdisplayshortskip}{1.ex}
\setlength{\abovedisplayskip}{1.ex} \setlength{\abovedisplayshortskip}{1.ex}
  \[
    \norm{\mB^{-1} \mC^{-1} \left( \mA \rvvu + \vm - \vz \right)}_2^2.
  \]
}%
  can be bounded for the following instances of \(\mA\):
  \begin{enumerate}[label=\textbf{(\roman*)}]
    \item If \(\mA = \mC\),  
    {%
    \setlength{\belowdisplayskip}{1.ex} \setlength{\belowdisplayshortskip}{1.ex}
    \setlength{\abovedisplayskip}{1.ex} \setlength{\abovedisplayshortskip}{1.ex}
      \begin{align*}
      &{\lVert
        \mB^{-1} \mC^{-1} \left(\mC \vu + \vm - \vz\right)
      \rVert}_2^2
      =
      \norm{\mC \vu + \vm - \vz}_2^2
      + {\left(L - L^{-1} \right)} \, u_1^2,
      \end{align*}
    }
    \item while if \(\mA = \boldupright{O}\),  \\
    {%
    \setlength{\belowdisplayskip}{1.ex} \setlength{\belowdisplayshortskip}{1.ex}
    \setlength{\abovedisplayskip}{-1.ex} \setlength{\abovedisplayshortskip}{-1.ex}
      \[
      {\lVert
        \mB^{-1} \mC^{-1} \left(\vm - \vz\right)
      \rVert}_2^2
      =
      \norm{\vm - \vz}_2^2.
      \]
    }
  \end{enumerate}
\end{theoremEnd}
\begin{proofEnd}
  First notice that 
  \begin{align*}
    \mB^{-1} \mC^{-1}
    =
    \begin{bmatrix}
      L &       &        &   \\
        & 1     &        &   \\
        &       & \ddots &   \\
        &       &        & 1 \\
    \end{bmatrix}.
  \end{align*}

  Denoting the 1st coordinate of \(\mA \vu + \vm \) as \({[\mA \vu + \vm]}_1 = A_1 \rvu_1 + m_1\), we have
  \begin{align}
    &\norm{
      \mB^{-1} \mC^{-1} \left(\mA \vu + \vm - \vz\right)
    }_2^2
    \\
    &\;=
    \norm{
    \begin{bmatrix}
      L &       &        &   \\
        & 1     &        &   \\
        &       & \ddots &   \\
        &       &        & 1 \\
    \end{bmatrix} 
    \left(\mA \vu + \vm - \vz\right)
    }_2^2
    \nonumber
    \\
    &\;=
    \norm{\mA \vu + \vm - \vz}_2^2
    + {\left(L^{2} - 1 \right)} {\left( {[\mA \vu + \vm]}_1 - z_1 \right)}^2
    \nonumber
    \\
    &\;=
    \norm{\mA \rvvu + \vm - \vz}_2^2
    + {\left(L^{2} - 1 \right)} {\left(A_1 u_1 + m_1 - z_1 \right)}^2,
    \nonumber
\shortintertext{and using the fact that \(m_1 = z_1\)}
    &\;=
    \norm{\mA \vu + \vm - \vz}_2^2
    + {\left(L^{2} - 1 \right)} \, A_1^2  \, u_1^2.
    \label{eq:unimprovability_key_lemma_eq1}
  \end{align}

  \paragraph{Proof of (i)}
  If \(\mA = \mC = L^{-1/2} \boldupright{I}\), \cref{eq:unimprovability_key_lemma_eq1} yields,
  \begin{align*}
    \norm{
      \mB^{-1} \mC^{-1} \left(\mA \vu + \vm - \vz\right)
    }_2^2
    &=
    \norm{\mC \vu + \vm - \vz}_2^2
    + {\left(L^{2} - 1 \right)} \, L^{-1} u_1^2
    \\
    &=
    \norm{\mC \vu + \vm - \vz}_2^2
    + {\left(L - L^{-1} \right)} \, u_1^2
  \end{align*}

  \paragraph{Proof of (ii)}
  If \(\mA = \boldupright{O}\), \cref{eq:unimprovability_key_lemma_eq1} yields,
  \begin{align*}
    \norm{
      \mB^{-1} \mC^{-1} \left(\mA \vu + \vm - \vz\right)
    }_2^2
    &=
    \norm{\vm - \vz}_2^2.
  \end{align*}

\end{proofEnd}

\begin{theoremEnd}[category=stllowerboundunimprovability]{theorem}\label{thm:stl_lowerbound_unimprovability}
 Assume \cref{assumption:variation_family}.
  There exists a strongly-convex, \(L\)-log-smooth posterior and some variational parameter \(\widetilde{\vlambda} \in \Lambda_{L}\) for all \(L \geq 1\) such that
{%
\setlength{\belowdisplayskip}{1.ex} \setlength{\belowdisplayshortskip}{1.ex}
\setlength{\abovedisplayskip}{1.ex} \setlength{\abovedisplayshortskip}{1.ex}
  {
  \begin{align*}
    \mathbb{E} {\lVert \rvvg_{\mathrm{STL}}\left(\widetilde{\vlambda}\right) \rVert}_2^2
    &\geq
    \left(
    L^2
    \left( d + k_{\varphi} \right) 
    -2
    \left(d + 1 \right) 
    \right)
    {\lVert \widetilde{\mC} \rVert}_{\mathrm{F}}^2
    \\
    &\qquad
    - 2
    \left(k_{\varphi} - 1\right) \norm{ \widetilde{\vm} - \bar{\vz} }_2^2,
  \end{align*}
  }%
  }%
  where \(\widetilde{\vlambda} = (\widetilde{\vm}, \widetilde{\mC})\) and \(\bar{\vz}\) is a stationary point of the said log posterior.
\end{theoremEnd}
\vspace{-1ex}
\begin{proofEnd}\label{proof:stl_lowerbound_unimprovability}
  The worst case is achieved by the following:
  \begin{enumerate}[label=\textbf{(\roman*)}]
  \item \textbf{\(\pi\) is ill-conditioned such that the smoothness constant is large.} 
    This results in the domain \(\Lambda_{L}\) to include ill-conditioned \(\mC\)s, which has the largest impact on the gradient variance. Furthermore, 
  \item \textbf{\(\pi\) and \(q_{\vlambda}\) need to have the least overlap in probability volume.} 
    This means the variance reduction effect will be minimal.
  \end{enumerate}
  For Gaussians, this is equivalent to minimizing 
  \(
    {\lVert \mS^{-1} \mSigma^{-1} \rVert}_{\mathrm{F}}^2
  \)
  while maximizing \({\lVert \mSigma^{-1} \rVert}_{\mathrm{F}}^2\) and \({\lVert \mS^{-1} \rVert}_{\mathrm{F}}^2\).

  We therefore choose
  \[
  \pi = \mathcal{N}\left(\bar{\vz}, \mSigma\right)
  \qquad
  q_{\vlambda} = \mathcal{N}\left(\widetilde{\vm}, \widetilde{\mS}\right),
  \]
  where 
  \[
  \mSigma = \begin{bmatrix}
    L^{-1}  &   &        &   \\
           & L &        &   \\
           &   & \ddots &   \\
           &   &        & L \\
  \end{bmatrix},
  \qquad
  \widetilde{\mS} = L^{-1} \boldupright{I},
  \;\quad\text{and}\quad
  \widetilde{\vm} = \begin{bmatrix}
    \bar{z}_1 \\ m_2 \\ \vdots \\ m_d
  \end{bmatrix},
  \]
  where \(\bar{z}_1\) is the 1st element of \(\bar{\vz}\) such that \(\widetilde{m}_1 = \bar{z}_1\).
  The choice of \(\widetilde{m}_1 = \bar{z}_1\) is purely for clarifying the derivation.
  Notice that \(\mSigma\) has \(d-1\) entries set as \(L\), only one entry set as \(L^{-1}\), and \(\widetilde{\mS} = \widetilde{\mC}\widetilde{\mC}\).
  Here, \(\pi\) is \(L^{-1}\)-strongly log-concave, \(L\)-log smooth, and \(\widetilde{\vlambda} = \left(\widetilde{\vm}, \widetilde{\mC}\right) \in \Lambda_{L}\).

  \paragraph{General Gaussian \(\pi\) Lower Bound}
  As usual, we start from the definition of the STL estimator as
  \begin{align*}
    \mathbb{E} \norm{\rvvg_{\mathrm{STL}}\left(\vlambda\right)}_2^2
    &=
    \mathbb{E} \norm{
    \nabla_{\vlambda} \log \pi \left(\mathcal{T}_{\vlambda}\left(\rvvu\right)\right) - \nabla_{\vlambda} \log q_{\vnu}\left(\mathcal{T}_{\vlambda}\left(\rvvu\right)\right)
  }_2^2
    \;\Big\lvert_{\vnu = \vlambda}
  \shortintertext{by \cref{thm:jacobian_reparam_inner},}
    &=
    \mathbb{E} 
    J_{\mathcal{T}}\left(\rvvu\right)
    \norm{
    \nabla \log \pi \left(\mathcal{T}_{\vlambda}\left(\rvvu\right)\right) 
    - \nabla \log q_{\vnu}\left(\mathcal{T}_{\vlambda}\left(\rvvu\right)\right)
    }_2^2
    \;\Big\lvert_{\vnu = \vlambda},
\shortintertext{since both \(\pi\) and \(q_{\vlambda}\) are Gaussians,}
    &=
    \mathbb{E} 
    \left(1 + \textstyle{\sum^{d}_{i=1} \rvu_i^2} \right)
    \norm{
    \nabla \log \pi \left(\mathcal{T}_{\vlambda}\left(\rvvu\right)\right) 
    - \nabla \log q_{\vlambda}\left(\mathcal{T}_{\vlambda}\left(\rvvu\right)\right)
    }_2^2
    \\
    &=
    \mathbb{E} 
    \left(1 + \textstyle{\sum^{d}_{i=1} \rvu_i^2} \right)
    \norm{
      \mSigma^{-1} \left(\mathcal{T}_{\vlambda}\left(\rvvu\right) - \bar{\vz}\right)
      -
      \mS^{-1} \left(\mathcal{T}_{\vlambda}\left(\rvvu\right) - \vm\right)
    }_2^2
    \\
    &=
    \mathbb{E} 
    \left(1 + \textstyle{\sum^{d}_{i=1} \rvu_i^2} \right)
    \norm{
      \mSigma^{-1} \left(\mathcal{T}_{\vlambda}\left(\rvvu\right) - \bar{\vz}\right)
      -
      \mS^{-1} \left(\mathcal{T}_{\vlambda}\left(\rvvu\right) - \bar{\vz}\right)
      +
      \mS^{-1} \left(\vm - \bar{\vz}\right)
    }_2^2
    \\
    &=
    \mathbb{E} 
    \left(1 + \textstyle{\sum^{d}_{i=1} \rvu_i^2} \right)
    \Big(
    {\lVert
      \mSigma^{-1} \left(\mathcal{T}_{\vlambda}\left(\rvvu\right) - \bar{\vz}\right)
    \rVert}_2^2
    +
    {\lVert
      \mS^{-1} \left(\mathcal{T}_{\vlambda}\left(\rvvu\right) - \bar{\vz}\right)
    \rVert}_2^2
    +
    {\lVert
      \mS^{-1} \left(\vm - \bar{\vz}\right)
    \rVert}_2^2
    \\
    &\quad\qquad\qquad\qquad\qquad
    -2
    \inner{
      \mSigma^{-1} \left(\mathcal{T}_{\vlambda}\left(\rvvu\right) - \bar{\vz}\right)
    }{
      \mS^{-1} \left(\mathcal{T}_{\vlambda}\left(\rvvu\right) - \bar{\vz}\right)
    }
    \\
    &\quad\qquad\qquad\qquad\qquad
    +2
    \inner{
      \mSigma^{-1} \left(\mathcal{T}_{\vlambda}\left(\rvvu\right) - \bar{\vz}\right)
    }{
      \mS^{-1} \left(\vm - \bar{\vz}\right)
    }
    \\
    &\quad\qquad\qquad\qquad\qquad
    -2
    \inner{
      \mS^{-1} \left(\mathcal{T}_{\vlambda}\left(\rvvu\right) - \bar{\vz}\right)
    }{
      \mS^{-1} \left(\vm - \bar{\vz}\right)
    }
    \Big),
\shortintertext{distributing the expectation and \(1 + \textstyle{\sum^{d}_{i=1} \rvu_i^2}\),}
    &=
    \mathbb{E} 
    \left(1 + \textstyle{\sum^{d}_{i=1} \rvu_i^2} \right)
    \Big(
    {\lVert
      \mSigma^{-1} \left(\mathcal{T}_{\vlambda}\left(\rvvu\right) - \bar{\vz}\right)
    \rVert}_2^2
    +
    {\lVert
      \mS^{-1} \left(\mathcal{T}_{\vlambda}\left(\rvvu\right) - \bar{\vz}\right)
    \rVert}_2^2
    \Big)
    \\
    &\quad+
    \mathbb{E} 
    \left(1 + \textstyle{\sum^{d}_{i=1} \rvu_i^2} \right)
    {\lVert
      \mS^{-1} \left(\vm - \bar{\vz}\right)
    \rVert}_2^2
    \\
    &\quad
    -2 \,
    \mathbb{E} 
    \left(1 + \textstyle{\sum^{d}_{i=1} \rvu_i^2} \right)
    \inner{
      \mSigma^{-1} \left(
      \mathcal{T}_{\vlambda}\left(\rvvu\right) - \bar{\vz}\right)
    }{
      \mS^{-1} \left(\mathcal{T}_{\vlambda}\left(\rvvu\right) - \bar{\vz}\right)
    }
    \\
    &\quad
    +2 \,
    \inner{
      \mSigma^{-1} 
      \mathbb{E} 
      \left(1 + \textstyle{\sum^{d}_{i=1} \rvu_i^2}  \right)
      \left(
      \mathcal{T}_{\vlambda}\left(\rvvu\right) - \bar{\vz}\right)
    }{
      \mS^{-1} \left(\vm - \bar{\vz}\right)
    }
    \\
    &\quad
    -2 \,
    \inner{
      \mS^{-1} 
      \mathbb{E} 
      \left(1 + \textstyle{\sum^{d}_{i=1} \rvu_i^2} \right)
      \left(
      \mathcal{T}_{\vlambda}\left(\rvvu\right) - \bar{\vz}\right)
    }{
      \mS^{-1} \left(\vm - \bar{\vz}\right)
    },
\shortintertext{applying \cref{lemma:unorm_times_reparam,thm:u_identities} to the second term and the last two inner product terms,}
    &=
    \mathbb{E} 
    \left(1 + \textstyle{\sum^{d}_{i=1} \rvu_i^2} \right)
    \Big(
    {\lVert
      \mSigma^{-1} \left(\mathcal{T}_{\vlambda}\left(\rvvu\right) - \bar{\vz}\right)
    \rVert}_2^2
    +
    {\lVert
      \mS^{-1} \left(\mathcal{T}_{\vlambda}\left(\rvvu\right) - \bar{\vz}\right)
    \rVert}_2^2
    \Big)
    \\
    &\quad+
    \left(d + 1\right)
    {\lVert
      \mS^{-1} \left(\vm - \bar{\vz}\right)
    \rVert}_2^2
    \\
    &\quad
    -2 \,
    \mathbb{E} 
    \left(1 + \textstyle{\sum^{d}_{i=1} \rvu_i^2} \right)
    \inner{
      \mSigma^{-1} \left(
      \mathcal{T}_{\vlambda}\left(\rvvu\right) - \bar{\vz}\right)
    }{
      \mS^{-1} \left(\mathcal{T}_{\vlambda}\left(\rvvu\right) - \bar{\vz}\right)
    }
    \\
    &\quad
    +2 \,
    \left(d + 1\right) 
    \inner{
      \mSigma^{-1} \left( \vm - \bar{\vz}\right)
    }{
      \mS^{-1} \left(\vm - \bar{\vz}\right)
    }
    \\
    &\quad
    -2 \,
    \left(d + 1\right) 
    \inner{
      \mS^{-1} \left(\vm - \bar{\vz}\right)
    }{
      \mS^{-1} \left(\vm - \bar{\vz}\right)
    }.
\shortintertext{The last two inner products can be denoted as norms such that}
    &\;=
    \mathbb{E} 
    \left(1 + \textstyle{\sum^{d}_{i=1} \rvu_i^2} \right)
    \Big(
    {\lVert
      \mSigma^{-1} \left(\mathcal{T}_{\vlambda}\left(\rvvu\right) - \bar{\vz}\right)
    \rVert}_2^2
    +
    {\lVert
      \mS^{-1} \left(\mathcal{T}_{\vlambda}\left(\rvvu\right) - \bar{\vz}\right)
    \rVert}_2^2
    \Big)
    \\
    &\quad
    +
    \left(d + 1\right)
    {\lVert
      \mS^{-1} \left(\vm - \bar{\vz}\right)
    \rVert}_2^2
    \\
    &\quad
    -2 \,
    \mathbb{E} 
    \left(1 + \textstyle{\sum^{d}_{i=1} \rvu_i^2} \right)
    {\lVert
      \mB^{-1} \mC^{-1} \left( \mathcal{T}_{\vlambda}\left(\rvvu\right) - \bar{\vz}\right)
    \rVert}_2^2
    \\
    &\quad
    +2 
    \left(d + 1\right)
    {\lVert
      \mB^{-1} \mC^{-1} \left(\vm - \bar{\vz}\right)
    \rVert}_2^2
    -2
    \left(d + 1\right)
    {\lVert
      \mS^{-1} \left(\vm - \bar{\vz}\right)
    \rVert}_2^2,
  \end{align*}
  where \(\mB\) is the matrix square root of \(\mSigma\) such that \(\mB^{-1} \mB^{-1} = \mSigma^{-1}\).
  The derivation so far applies to any Gaussian \(\pi, q_{\vlambda}\) and \(\vlambda \in \Lambda_{S}\) for any \(S > 0\).

  \paragraph{Worst-Case Lower Bound}
  Now, for our worst-case example, 
  \begin{align*}
    \mathbb{E} {\lVert \rvvg_{\mathrm{STL}}\left(\widetilde{\vlambda}\right) \rVert}_2^2
    &=
    \mathbb{E} 
    \left(1 + \textstyle{\sum^{d}_{i=1} \rvu_i^2} \right)
    \Big(
    {\lVert
      \mSigma^{-1} \left(\mathcal{T}_{\widetilde{\vlambda}}\left(\rvvu\right) - \bar{\vz}\right)
    \rVert}_2^2
    +
    {\lVert
      \widetilde{\mS}^{-1} \left(\mathcal{T}_{\widetilde{\vlambda}}\left(\rvvu\right) - \bar{\vz}\right)
    \rVert}_2^2
    \Big)
    \\
    &\quad
    +
    \left(d + 1\right)
    {\lVert
      \widetilde{\mS}^{-1} \left(\widetilde{\vm} - \bar{\vz}\right)
    \rVert}_2^2
    \\
    &\quad
    -2 \,
    \mathbb{E} 
    \left(1 + \textstyle{\sum^{d}_{i=1} \rvu_i^2} \right)
    {\lVert
      \mB^{-1} \widetilde{\mC}^{-1} \left( \mathcal{T}_{\widetilde{\vlambda}}\left(\rvvu\right) - \bar{\vz}\right)
    \rVert}_2^2
    \\
    &\quad
    +2 
    \left(d + 1\right)
    {\lVert
      \mB^{-1} \widetilde{\mC}^{-1} \left(\widetilde{\vm} - \bar{\vz}\right)
    \rVert}_2^2
    -2 
    \left(d + 1\right)
    {\lVert
      \widetilde{\mS}^{-1} \left(\widetilde{\vm} - \bar{\vz}\right)
    \rVert}_2^2,
\shortintertext{since \(\pi\) is \(\mu\)-strongly log-concave and \(\widetilde{\mS}^{-1} = L \boldupright{I}\),}
    &\geq
    \mathbb{E} 
    \left(1 + \textstyle{\sum^{d}_{i=1} \rvu_i^2} \right)
    \Big(
    L^{-2} \norm{\mathcal{T}_{\widetilde{\vlambda}}\left(\rvvu\right) - \bar{\vz}}_2^2
    +
    L^2 \norm{\mathcal{T}_{\widetilde{\vlambda}}\left(\rvvu\right) - \bar{\vz}}_2^2
    \Big)
    \\
    &\quad
    + \left(d + 1\right) L^2 \norm{\widetilde{\vm} - \bar{\vz}}_2^2
    \\
    &\quad
    -2 \,
    \mathbb{E} 
    \left(1 + \textstyle{\sum^{d}_{i=1} \rvu_i^2} \right)
    {\lVert
      \mB^{-1} \widetilde{\mC}^{-1} \left( \mathcal{T}_{\widetilde{\vlambda}}\left(\rvvu\right) - \bar{\vz}\right)
    \rVert}_2^2
    \\
    &\quad
    +2 \left(d + 1\right) \,
    {\lVert
      \mB^{-1} \widetilde{\mC}^{-1} \left(\widetilde{\vm} - \bar{\vz}\right)
    \rVert}_2^2
    -2 \left(d + 1\right) L^2 \norm{ \widetilde{\vm} - \bar{\vz} }_2^2,
\shortintertext{and grouping the terms,}
    &=
    \underbrace{
    \left( L^{-2} + L^2 \right) \,
    \mathbb{E} 
    \left(1 + \textstyle{\sum^{d}_{i=1} \rvu_i^2} \right)
    \norm{\mathcal{T}_{\widetilde{\vlambda}}\left(\rvvu\right) - \bar{\vz}}_2^2
    -
    \left(d + 1\right) L^2 \norm{\widetilde{\vm} - \bar{\vz}}_2^2
    }_{T_{\text{\ding{172}}}}
    \\
    &\qquad
    \underbrace{
    -2 \,
    \mathbb{E} 
    \left(1 + \textstyle{\sum^{d}_{i=1} \rvu_i^2} \right)
    {\lVert
      \mB^{-1} \widetilde{\mC}^{-1} \left( \mathcal{T}_{\widetilde{\vlambda}}\left(\rvvu\right) - \bar{\vz}\right)
    \rVert}_2^2
    }_{T_{\text{\ding{173}}}}
    \\
    &\qquad
    \underbrace{
    +2 \left(d + 1\right) \,
    {\lVert
      \mB^{-1} \widetilde{\mC}^{-1} \left(\widetilde{\vm} - \bar{\vz}\right)
    \rVert}_2^2.
    }_{T_{\text{\ding{174}}}}
  \end{align*}

\paragraph{Lower Bound on \(T_{\text{\ding{172}}}\)}
For \(T_{\text{\ding{172}}}\), we have
  \begin{align}
    T_{\text{\ding{172}}}
    &=
    \left(L^{-2} + L^2 \right)
    \mathbb{E} 
    \left(1 + \textstyle{\sum^{d}_{i=1} \rvu_i^2} \right)
    \norm{\mathcal{T}_{\widetilde{\vlambda}}\left(\rvvu\right) - \bar{\vz}}_2^2
    - 
    \left(d + 1\right) L^2 \norm{\widetilde{\vm} - \bar{\vz}}_2^2,
    \nonumber
\shortintertext{applying \cref{thm:normdist_1pnormu},}
    &=
    \left(L^{-2} + L^2 \right)
    \left(
      \left( d + 1 \right) \norm{\widetilde{\vm} - \bar{\vz}}_2^2
      +
      \left( d + k_{\varphi} \right) {\lVert \widetilde{\mC} \rVert}_{\mathrm{F}}^2
    \right)
    -
    \left(d + 1\right) L^2 \norm{\widetilde{\vm} - \bar{\vz}}_2^2,
    \nonumber
\shortintertext{and since \(L^{-2} > 0\) and is negligible for large \(L\)s,}
    &\geq
    L^2
    \left(
      \left( d + 1 \right) \norm{\widetilde{\vm} - \bar{\vz}}_2^2
      +
      \left( d + k_{\varphi} \right) {\lVert \widetilde{\mC} \rVert}_{\mathrm{F}}^2
    \right)
    -
    \left(d + 1\right) L^2 \norm{\widetilde{\vm} - \bar{\vz}}_2^2
    \nonumber
    \\
    &=
    L^2
    \left( d + k_{\varphi} \right) {\lVert \widetilde{\mC} \rVert}_{\mathrm{F}}^2.
    \label{eq:unimprovability_term1}
  \end{align}

\paragraph{Lower Bound on \(T_{\text{\ding{173}}}\)}
For \(T_{\text{\ding{173}}}\), we now use the covariance structures of our worst case through \cref{thm:lowerbound_matrix_innerproduct_lemma}.
That is,
  \begin{align*}
    T_{\text{\ding{173}}}
    &=
    -2 \,
    \mathbb{E} 
    \left(1 + \textstyle{\sum^{d}_{i=1} \rvu_i^2} \right)
    {\lVert
      \mB^{-1} \widetilde{\mC}^{-1} \left( \mathcal{T}_{\widetilde{\vlambda}}\left(\rvvu\right) - \bar{\vz}\right)
    \rVert}_2^2.
\shortintertext{Noting that \(\mathcal{T}_{\widetilde{\vlambda}}\left(\rvvu\right) = \widetilde{\mC} \rvvu + \widetilde{\vm}\) by definition, we can apply \cref{thm:lowerbound_matrix_innerproduct_lemma} Item (i) as}
    &=
    -2 \,
    \mathbb{E} 
    \left(1 + \textstyle{\sum^{d}_{i=1} \rvu_i^2} \right)
    \left(
      \norm{\mathcal{T}_{\widetilde{\vlambda}}\left(\rvvu\right) - \bar{\vz}}_2^2 + {\left(L - L^{-1} \right)} \rvu_1^2 
    \right),
\shortintertext{distributing the expectation and \(1 + \textstyle{\sum^{d}_{i=1} \rvu_i^2}\),}
    &=
    -2 \,
    \Big(
    \mathbb{E} 
    \left(1 + \textstyle{\sum^{d}_{i=1} \rvu_i^2} \right)
    \norm{\mathcal{T}_{\widetilde{\vlambda}}\left(\rvvu\right) - \bar{\vz}}_2^2
    +
    \underbrace{
    \mathbb{E} 
    \left(1 + \textstyle{\sum^{d}_{i=1} \rvu_i^2} \right)
    {\left(L - L^{-1} \right)} \rvu_1^2 
    }_{T_{\text{\ding{175}}}}
    \Big),
  \end{align*}

  \(T_{\text{\ding{175}}}\) follows as
  \begin{align}
    T_{\text{\ding{175}}}
    &=
    \mathbb{E} 
    \left(1 + \textstyle{\sum^{d}_{i=1} \rvu_i^2} \right)
    {\left(L - L^{-1} \right)} \rvu_1^2 
    \nonumber
    \\
    &=
    {\left(L^1 - L^{-1} \right)}
    \mathbb{E} 
    \left(1  + \textstyle{\sum^{d}_{i=1} \rvu_i^2} \right)  \rvu_1^2
    \nonumber
    \\
    &=
    {\left(L - L^{-1} \right)}
    \left(\mathbb{E} \rvu_1^2 + \mathbb{E} \rvu_1^4 + \textstyle{\sum^{d}_{i=2} \mathbb{E} \rvu_i^2 \mathbb{E} \rvu_1^2} \right)  ,
    \nonumber
\shortintertext{applying \cref{thm:u_identities},}
    &=
    {\left(L - L^{-1} \right)}
    \left(1 + k_{\varphi} + d - 1 \right)  
    \nonumber
    \\
    &=
    {\left(L - L^{-1} \right)} 
    \left(d + k_{\varphi}\right).
    \label{eq:stl_lowerbound_173}
  \end{align}

  Then,
  \begin{align}
    T_{\text{\ding{173}}}
    &=
    -2 \,
    \Big(
    \mathbb{E} 
    \left(1 + \textstyle{\sum^{d}_{i=1} \rvu_i^2} \right)
    \norm{\mathcal{T}_{\widetilde{\vlambda}}\left(\rvvu\right) - \bar{\vz}}_2^2
    +
    T_{\text{\ding{175}}}
    \Big),
    \nonumber
\shortintertext{bringing \cref{eq:stl_lowerbound_173} in,}
    &=
    -2 \,
    \Big(
    \mathbb{E} 
    \left(1 + \textstyle{\sum^{d}_{i=1} \rvu_i^2} \right)
    \norm{\mathcal{T}_{\widetilde{\vlambda}}\left(\rvvu\right) - \bar{\vz}}_2^2
    +
    {\left(L - L^{-1} \right)} 
    \left(d + k_{\varphi}\right)
    \Big),
    \nonumber
\shortintertext{applying \cref{thm:normdist_1pnormu},}
    &=
    -2 \,
    \Big(
    \left(d + k_{\varphi}\right) \norm{ \widetilde{\vm} - \bar{\vz} }_2^2
    +
    \left(d + 1 \right) {\lVert \widetilde{\mC} \rVert}_{\mathrm{F}}^2
    +
    \left(d + k_{\varphi}\right)
    {\left(L - L^{-1} \right)} 
    \Big)
    \label{eq:unimprovability_term2}
  \end{align}

\paragraph{Lower Bound on \(T_{\text{\ding{174}}}\)}
  Similarly for \(T_{\text{\ding{174}}}\), we can apply \cref{thm:lowerbound_matrix_innerproduct_lemma} Item (ii) as
  \begin{align}
    T_{\text{\ding{174}}}
    =
    2 \left(d+1\right)
    {\lVert
      \mB^{-1} \widetilde{\mC}^{-1} \left( \widetilde{\vm} - \bar{\vz}\right)
    \rVert}_2^2
    =
    2 \left(d+1\right) \norm{\widetilde{\vm} - \bar{\vz}}_2^2.
    \label{eq:unimprovability_term3}
  \end{align}

  Combining \cref{eq:unimprovability_term1,eq:unimprovability_term2,eq:unimprovability_term3},
  \begin{align*}
    \mathbb{E} {\lVert\rvvg_{\mathrm{STL}}\left(\widetilde{\vlambda}\right) \rVert}_2^2
    &\geq
    T_{\text{\ding{172}}} + T_{\text{\ding{173}}} + T_{\text{\ding{174}}}
    \\
    &\geq
    L^2
    \left( d + k_{\varphi} \right) {\lVert \widetilde{\mC} \rVert}_{\mathrm{F}}^2
    -2 \,
    \Big(
    \left(d + k_{\varphi}\right) \norm{ \widetilde{\vm} - \bar{\vz} }_2^2
    +
    \left(d + 1 \right) {\lVert \widetilde{\mC} \rVert}_{\mathrm{F}}^2
    +
    \left(d + k_{\varphi}\right)
    {\left(L - L^{-1} \right)} 
    \Big)
    + 2 \left(d + 1\right) \norm{\widetilde{\vm} - \bar{\vz}}_2^2
    \\
    &=
    \left(
    L^2
    \left( d + k_{\varphi} \right) 
    -2
    \left(d + 1 \right) 
    \right)
    {\lVert \widetilde{\mC} \rVert}_{\mathrm{F}}^2
    -
    2 \left(k_{\varphi} - 1\right) \norm{ \widetilde{\vm} - \bar{\vz} }_2^2
    +
    \left(d + k_{\varphi}\right)
    {\left(L - L^{-1} \right)} ,
\shortintertext{and when \(L \geq 1\), we have \(L - L^{-1} > 0\). Therefore, we can simply the bound as}
    &\geq
    \left(
    L^2
    \left( d + k_{\varphi} \right) 
    -2
    \left(d + 1 \right) 
    \right)
    {\lVert \widetilde{\mC} \rVert}_{\mathrm{F}}^2
    -
    2 \left(k_{\varphi} - 1\right) \norm{ \widetilde{\vm} - \bar{\vz} }_2^2.
  \end{align*}

\end{proofEnd}

%%% Local Variables:
%%% TeX-master: "main"
%%% End:

%
\begin{remark}\label{remark:stl_tightness}
  \cref{thm:stl_lowerbound_unimprovability} implies that \cref{thm:stl_upperbound_corollary} is tight with respect to the dimension dependence \(d\) and the log-smoothness \(L\) except for a factor of 4.
\end{remark}

\begin{remark}[\textbf{Room for Improvement}]
  Part of the factor of \(4\) looseness is due to the extreme worst case: when \(\nabla \log \pi\) and \(\nabla \log q_{\vlambda}\) are perfectly anti-correlated.
  This worst case is unlikely to appear in practice, thus making a tighter lower bound challenging to obtain.
  But at the same time, we were unsuccessful at seeking a general assumption that would rule out these worst cases in the upper bound.
  Specifically, we tried very hard to apply coercivity/gradient monotonicity of log-concave distributions, but to no avail, leaving this to future works.
\end{remark}

\vspace{-1.ex}
\subsection{Theoretical Analysis of the CFE Estimator}
\vspace{-1.ex}
We now present the analysis of the CFE estimator.
While the CFE estimator has been studied in-depth by \citet{domke_provable_2019,kim_practical_2023,domke_provable_2023}, we slightly improve the latest analysis of \citet[Theorem 3]{domke_provable_2023}.
Specifically, we improve the constants and obtain an adaptive bound.
This ensures that we have a fair comparison with the STL estimator.
%

\begin{theoremEnd}[category=cfeupperbound]{theorem}\label{thm:cfe_upperbound}
  Assume \cref{assumption:variation_family} and that \(\pi\) is \(L\)-log-smooth.
  For the full-rank parameterization, the expected-squared norm of the CFE estimator is bounded as
  \begin{align*}
    \mathbb{E} \norm{ \rvvg_{\mathrm{CFE}}\left(\vlambda\right) }_2^2
    &\leq
    \left( L^2 \left( d + k_{\varphi} \right) \left(1 + \delta\right) + {\left(L + C \right)}^{2} \right) {\lVert \vlambda - \vlambda^* \rVert}_2^2 
    \\
    &\quad+
    L^2 \left( d + k_{\varphi} \right) \left(1 + \delta^{-1} \right) {\lVert \vlambda^* - \bar{\vlambda} \rVert}_2^2
  \end{align*}
  for any \(\vlambda \in \Lambda_S\) and \(\delta \geq 0\), where \(\bar{\vlambda} = \left(\bar{\vz}, \mathbf{0}\right)\) and \(\bar{\vz}\) is any stationary point of \(f\).
\end{theoremEnd}
\begin{proofEnd}
  Following the notation of \citet{domke_provable_2023}, we denote \( \nabla \log \pi = f \).
  Then, starting from the definition of the variance,
  \begin{align}
    \mathbb{E} \norm{ \rvvg_{\mathrm{CFE}}\left(\vlambda\right) }_2^2
    &=
    \mathrm{tr}\mathbb{V}\rvvg\left(\vlambda\right)
    +
    \norm{ \mathbb{E} \rvvg_{\mathrm{CFE}}\left(\vlambda\right) }_2^2,
    \nonumber
\shortintertext{and by the unbiasedness of \(\rvvg_{\mathrm{CFE}}\),}
    &=
    \mathrm{tr}\mathbb{V}\rvvg\left(\vlambda\right)
    +
    \norm{ \nabla F\left(\vlambda\right) }_2^2,
    \nonumber
\shortintertext{by the definition of \(\rvvg_{\mathrm{CFE}}\) (\cref{def:cfe}),}
    &=
    \mathrm{tr}\mathbb{V}_{\rvvz \sim q_{\vlambda}} \left( \nabla_{\vlambda} f\left( \rvvz \right) + \nabla \mathbb{H}\left(q_{\vlambda}\right) \right)
    +
    \norm{ \nabla F\left(\vlambda\right) }_2^2.
    \nonumber
\shortintertext{We now apply the property of the variance: the deterministic components are neglected as}
    &=
    \mathrm{tr}\mathbb{V}_{\rvvz \sim q_{\vlambda}} \nabla_{\vlambda} f\left( \rvvz \right) 
    +
    \norm{ \nabla F\left(\vlambda\right) }_2^2
    \nonumber
    \\
    &\leq
    \mathbb{E}_{\rvvz \sim q_{\vlambda}} \norm{ \nabla_{\vlambda} f\left( \rvvz \right) }_2^2
    +
    \norm{ \nabla F\left(\vlambda\right) }_2^2.
    \label{eq:thm_cfe}
  \end{align}

  For \(L\)-log-smooth posteriors (\(L\)-smooth \(f\)), \citet[Theorem 3]{domke_provable_2019} show that
  \begin{align*}
    \mathbb{E}_{\rvvz \sim q_{\vlambda}} \norm{ \nabla_{\vlambda} f\left( \rvvz \right) }_2^2
    &\leq
    L^2  \left( 
    \left( d + k_{\varphi} \right) \norm{ \vm - \bar{\vz} }_2^2
    +
    \left( d + 1 \right) \norm{ \mC }_{\mathrm{F}}^2
    \right),
\shortintertext{and since \(k_{\varphi} \geq 1\),}
    &\leq
    L^2  \left( 
    \left( d + k_{\varphi} \right) \norm{ \vm - \bar{\vz} }_2^2
    +
    \left( d + k_{\varphi} \right) \norm{ \mC }_{\mathrm{F}}^2
    \right)
    \\
    &=
    L^2 \left( d + k_{\varphi} \right) {\lVert \vlambda - \bar{\vlambda} \rVert}_2^2,
  \end{align*}
  which is tight.

  Applying \cref{eq:peterpaul}, we have
  \begin{align}
    \mathbb{E}_{\rvvz \sim q_{\vlambda}} {\lVert \nabla_{\vlambda} f\left( \rvvz \right) \rVert}_2^2
    &\leq
    L^2 \left( d + k_{\varphi} \right) {\lVert \vlambda - \bar{\vlambda} \rVert}_2^2
    \nonumber
    \\
    &=
    L^2 \left( d + k_{\varphi} \right) {\lVert \vlambda - \vlambda^* + \vlambda^* - \bar{\vlambda} \rVert}_2^2
    \nonumber
    \\
    &\leq
    L^2 \left( d + k_{\varphi} \right) \left( \left(1 + \delta \right) {\lVert \vlambda - \vlambda^* \rVert}_2^2 + \left(1 + \delta^{-1} \right) {\lVert \vlambda^* - \bar{\vlambda} \rVert}_2^2 \right).
    \label{eq:thm_cfe_energy}
  \end{align}
  
  Now, for \(\vlambda \in \Lambda_S\),~\citet[Theorem 1 \& Lemma 12]{domke_provable_2020} show that the negative ELBO \(F\) is (\(L + S\))-smooth as
  \begin{align}
    \norm{ \nabla F\left(\vlambda\right) }_2^2
    =
    \norm{ \nabla F\left(\vlambda\right) - \nabla F\left(\vlambda^*\right) }_2^2
    \leq 
    {\left(L + S \right)}^{2} \norm{ \vlambda - \vlambda^* }_2^2.
    \label{eq:thm_cfe_entropy}
  \end{align}

  Now back to \cref{eq:thm_cfe},
  \begin{align*}
    \mathbb{E} \norm{ \rvvg_{\mathrm{CFE}}\left(\vlambda\right) }_2^2
    &\leq
    \mathbb{E}_{\rvvz \sim q_{\vlambda}} \norm{ \nabla_{\vlambda} f\left( \rvvz \right) }_2^2
    +
    \norm{ \nabla F\left(\vlambda\right) }_2^2
\shortintertext{applying \cref{eq:thm_cfe_energy},}
    &\leq
    L^2 \left( d + k_{\varphi} \right) \left( \left(1 + \delta \right) {\lVert \vlambda - \vlambda^* \rVert}_2^2 + \left(1 + \delta^{-1} \right) {\lVert \vlambda^* - \bar{\vlambda} \rVert}_2^2 \right)
    +
    \norm{ \nabla F\left(\vlambda\right) }_2^2
\shortintertext{and \cref{eq:thm_cfe_entropy},}
    &\leq
    L^2 \left( d + k_{\varphi} \right) \left( \left(1 + \delta \right) {\lVert \vlambda - \vlambda^* \rVert}_2^2 + \left(1 + \delta^{-1} \right) {\lVert \vlambda^* - \bar{\vlambda} \rVert}_2^2 \right)
    +
    {\left(L + C \right)}^{2} \norm{ \vlambda - \vlambda^* }_2^2
    \\
    &=
    \left( L^2 \left( d + k_{\varphi} \right) \left(1 + \delta\right) + {\left(L + C \right)}^{2} \right) {\lVert \vlambda - \vlambda^* \rVert}_2^2 
    +
    L^2 \left( d + k_{\varphi} \right) \left(1 + \delta^{-1} \right) {\lVert \vlambda^* - \bar{\vlambda} \rVert}_2^2.
  \end{align*}
\end{proofEnd}


\begin{theoremEnd}[all end, category=cfeupperboundmf]{theorem}\label{thm:cfe_upperbound_mf}
  Assume \cref{assumption:variation_family} and that \(\pi\) is \(L\)-log-smooth.
  For the mean-field parameterization, the expected-squared norm of the CFE estimator is bounded as
  \begin{align*}
    \mathbb{E} \norm{ \rvvg_{\mathrm{CFE}}\left(\vlambda\right) }_2^2
    &\leq
    \big(
    \left(2 k_{\varphi} \sqrt{d} + 1 \right) 
    \left(1 + \delta \right)
    +
    {\left(L + S \right)}^{2}
    \big)
    {\lVert \vlambda - \vlambda^* \rVert}_2^2
    \\
    &\quad+
    \left(2 k_{\varphi} \sqrt{d} + 1 \right) \left(1 + \delta^{-1} \right) {\lVert \vlambda^* - \bar{\vlambda} \rVert}_2^2.
  \end{align*}
  for any \(\vlambda \in \Lambda_S\) and \(\delta \geq 0\), where \(\bar{\vlambda} = \left(\bar{\vz}, \mathbf{0}\right)\) and \(\bar{\vz}\) is any stationary point of \(f\).
\end{theoremEnd}
\begin{proofEnd}
  For the mean-field case, the only difference with \cref{thm:cfe_upperbound} is the upper bound on the energy term.
  The key step is the mean-field part of \cref{thm:normdist_1pnormu}, first proven by \citet{kim_practical_2023}.
  The remaining steps are similar to Theorem 1 of \citet{kim_practical_2023}.
  That is,
  \begin{align*}
    \mathbb{E}_{\rvvz \sim q_{\vlambda}} \norm{ \nabla_{\vlambda} f\left( \rvvz \right) }_2^2
    &=
    \mathbb{E} \norm{ \nabla_{\vlambda} f\left( \mathcal{T}_{\vlambda}\left(\rvvu\right) \right) }_2^2,
\shortintertext{applying \cref{thm:jacobian_reparam_inner},}
    &=
    \mathbb{E} J_{\mathcal{T}}\left(\rvvu\right) \norm{ \nabla f\left( \mathcal{T}_{\vlambda}\left(\rvvu\right) \right) }_2^2
    =
    \mathbb{E} J_{\mathcal{T}}\left(\rvvu\right) \norm{ \nabla f\left( \mathcal{T}_{\vlambda}\left(\rvvu\right) \right) - \nabla f\left(\bar{\vz}\right) }_2^2,
\shortintertext{from \(L\)-smoothness of \(f = \log \pi\),}
    &\leq
    L^2 \, J_{\mathcal{T}}\left(\rvvu\right) \mathbb{E} \norm{ \mathcal{T}_{\vlambda}\left(\rvvu\right) - \bar{\vz} }_2^2,
\shortintertext{applying \cref{thm:normdist_1pnormu},}
    &\leq
    L^2 \left(\sqrt{dk_{\varphi}} + k_{\varphi} \sqrt{d} + 1 \right) \norm{ \vm - \bar{\vz} }_2^2 + L^2 \left(2 k_{\varphi} \sqrt{d} + 1\right) \norm{\mC}_{\mathrm{F}}^2.
\shortintertext{and since \(k_{\varphi} \geq 1\), we have \(k_{\varphi} > \sqrt{k_{\varphi}}\), and thus}
    &\leq
    L^2 \left(2 k_{\varphi} \sqrt{d} + 1 \right) \left( \norm{ \vm - \bar{\vz} }_2^2 +  \norm{\mC}_{\mathrm{F}}^2 \right)
    \\
    &=
    L^2 \left(2 k_{\varphi} \sqrt{d} + 1 \right) {\lVert \vlambda - \bar{\vlambda} \rVert}_2^2.
\shortintertext{We finally apply \cref{eq:peterpaul} as}
    &\leq
    L^2 \left(2 k_{\varphi} \sqrt{d} + 1 \right)
    \left(
      \left(1 + \delta \right) {\lVert \vlambda - \vlambda^* \rVert}_2^2
      + \left(1 + \delta^{-1} \right) {\lVert \vlambda^* - \bar{\vlambda} \rVert}_2^2
    \right).
  \end{align*}

  Combining this with \cref{eq:thm_cfe,eq:thm_cfe_entropy}, we have
  \begin{align*}
    \mathbb{E} \norm{ \rvvg_{\mathrm{CFE}}\left(\vlambda\right) }_2^2
    &=
    \mathbb{E}_{\rvvz \sim q_{\vlambda}} \norm{ \nabla_{\vlambda} f\left( \rvvz \right) }_2^2
    +
    \norm{ \nabla F\left(\vlambda\right) }_2^2
\shortintertext{and applying \cref{eq:thm_cfe_energy},}
    &\leq
    \mathbb{E}_{\rvvz \sim q_{\vlambda}} \norm{ \nabla_{\vlambda} f\left( \rvvz \right) }_2^2
    +
    {\left(L + S \right)}^{2} \norm{ \vlambda - \vlambda^* }_2^2
    \\
    &\leq
    \left(2 k_{\varphi} \sqrt{d} + 1 \right)
    \left(
      L^2 \left(1 + \delta \right) {\lVert \vlambda - \vlambda^* \rVert}_2^2
      + L^2 \left(1 + \delta^{-1} \right) {\lVert \vlambda^* - \bar{\vlambda} \rVert}_2^2
    \right)
    \\
    &\qquad+
    {\left(L + S \right)}^{2} \norm{ \vlambda - \vlambda^* }_2^2
    \\
    &=
    \left(
    \left(2 k_{\varphi} \sqrt{d} + 1 \right) 
    L^2 \left(1 + \delta \right)
    +
    {\left(L + S \right)}^{2}
    \right)
    {\lVert \vlambda - \vlambda^* \rVert}_2^2
    \\
    &\qquad+
    L^2 \left(2 k_{\varphi} \sqrt{d} + 1 \right) \left(1 + \delta^{-1} \right) {\lVert \vlambda^* - \bar{\vlambda} \rVert}_2^2.
  \end{align*}

\end{proofEnd}


\begin{corollary}
  Let \(S = L\).
  Then, the expected-squared norm of the CFE estimator with the full-rank parameterization satisfies the QVC for any \(\vlambda \in \Lambda_{L}\) with the constants
{%
\setlength{\abovedisplayskip}{.5ex} \setlength{\abovedisplayshortskip}{.5ex}
\setlength{\belowdisplayskip}{1.ex} \setlength{\belowdisplayshortskip}{1.ex}
  \begin{align*}
    \alpha_{\mathrm{CFE}} &= L^2 \left( d + k_{\varphi} + 4 \right) \left(1 + \delta\right),\\
    \beta_{\mathrm{CFE}}  &= L^2 \left( d + k_{\varphi} \right) \left( 1 + \delta^{-1}  \right) {\lVert \bar{\vlambda} - \vlambda^* \rVert}_2^2,
  \end{align*}
  }%
  for any \(\delta \geq 0\).
\end{corollary}

\begin{remark}[\textbf{Comparison with STL}]\label{remark:variance_comparison}
    Compared to the STL estimator, the constant \(\alpha\) of the CFE estimator is tighter by a factor of \(4\).
    Considering \cref{thm:stl_lowerbound_unimprovability}, the constant factor difference should be marginal in practice.
    %This means that the STL estimator will perform similarly when \({\lVert \bar{\vlambda} - \vlambda^* \rVert}_2^2\) large.
    % When \({\lVert \bar{\vlambda} - \vlambda^* \rVert}_2^2\) is small, it might have a smaller variance when \(\mathrm{D}_{F^4}\left(q_{\vlambda^*}, \pi\right)\) is small.
\end{remark}

\begin{remark}[\textbf{Intuitions on \({\lVert \bar{\vlambda} - \vlambda^* \rVert}_2^2\)}]
  The quantity \({\lVert \bar{\vlambda} - \vlambda^* \rVert}_2^2\) can be expressed in the Wasserstein-2 distance as
{%
\setlength{\abovedisplayskip}{.5ex} \setlength{\abovedisplayshortskip}{.5ex}
\setlength{\belowdisplayskip}{1.ex} \setlength{\belowdisplayshortskip}{1.ex}
  \[
    \mathrm{d}_{\mathcal{W}_2}\left(q_{\vlambda^*}, \delta_{\bar{\vz}}\right) = \sqrt{ {\lVert \vm^* - \bar{\vz}\rVert}_2^2 + \norm{\mC^*}_{\mathrm{F}}^2} =  {\lVert \bar{\vlambda} - \vlambda^* \rVert}_2,
  \]
}
  where \(\delta_{\bar{\vz}}\) is a delta measure centered on the posterior mode \(\bar{\vz}\).
   Also, when the variational posterior mean \(\vm^*\) is close to \(\bar{\vz}\) such that  \({\lVert \vm^* - \bar{\vz}\rVert}_2^2 \approx 0\), \({\lVert \bar{\vlambda} - \vlambda^* \rVert}_2^2\) corresponds to the variational posterior variance as
{%
\setlength{\abovedisplayskip}{.5ex} \setlength{\abovedisplayshortskip}{.5ex}
\setlength{\belowdisplayskip}{1.ex} \setlength{\belowdisplayshortskip}{1.ex}
  \[
     {\lVert \bar{\vlambda} - \vlambda^* \rVert}_2^2 \approx \norm{\mC^*}_{\mathrm{F}}^2 = \mathrm{tr}\, \Vsub{\rvvz \sim q_{\vlambda^*}}{\rvvz}.
  \]
}
\end{remark}

% \vspace{-1.ex}
% \subsection{Non-Asymptotic Complexity of SGD with the Quadratic Variance Condition}


\begin{theoremEnd}[category=complexityprojsgdqvc]{theorem}[\textbf{Strongly convex \(F\) with a fixed stepsize}]\label{thm:projsgd_stronglyconvex_fixedstepsize}
  For a \(\mu\)-strongly convex \(F : \Lambda \to \mathbb{R}\) on a convex set \(\Lambda\), the last iterate \(\vlambda_{T}\) of projected SGD with a fixed stepsize satisfies \( {\lVert \vlambda_T - \vlambda^* \rVert}_2^2 \leq \epsilon \) if
  \begin{align*}
    \gamma = \min\left( \frac{\epsilon \mu}{4 \beta}, \frac{\mu}{2 \alpha}, \frac{2}{\mu} \right)  \quad\text{and}\quad
    T \geq \max\left( \frac{ 4 \beta }{\mu^2 } \frac{1}{\epsilon}, \frac{2 \alpha}{\mu^2}, \frac{1}{2} \right) \log \left( 2 \norm{\vlambda_0 - \vlambda^*}_2^2 \, \frac{1}{\epsilon} \right).
  \end{align*}
\end{theoremEnd}
\begin{proofEnd}
  Theorem 6 of \citet{domke_provable_2023} utilizes the two-stage stepsize of \citep{gower_sgd_2019}.
  The anytime convergence of the first stage, 
  \[
     \norm{ \vlambda_{T} - \vlambda^* }_2^2
     \leq
     {(1 - \gamma \mu)}^{T} \norm{ \vlambda_0 - \vlambda^* }_2^2
     +
     \frac{2 \gamma \beta}{\mu}
  \]
  corresponds to the SGD with only a fixed stepsize \(\gamma < \frac{\mu}{2 \alpha}\).

  Here, the result follows from Lemma A.2 of \citet{garrigos_handbook_2023} by plugging the constants
  \[
    \alpha_0 = \norm{ \vlambda_0 - \vlambda^* }_2^2, \quad
    A = \frac{2 \beta}{\mu}, 
    \;\text{and}\quad
    C = \frac{2 \alpha}{\mu}, \frac{\mu}{2}.
  \]
\end{proofEnd}


\begin{theoremEnd}[category=complexityprojsgdqvc]{theorem}[\textbf{Strongly convex \(F\) with a decreasing stepsize schedule}]\label{thm:projsgd_stronglyconvex_decstepsize}
  For a \(\mu\)-strongly convex \(F : \Lambda \to \mathbb{R}\) on a convex set \(\Lambda\), the last iterate \(\vlambda_{T}\) of projected SGD with a descreasing stepsize satisfies \( {\lVert \vlambda_T - \vlambda^* \rVert}_2^2 \leq \epsilon \) if
  \begin{align*}
    \gamma_t = \min\left( \frac{\mu}{2 \alpha}, \frac{4 t + 2 }{ \mu \, {\left( t + 1\right)}^2 } \right)
    \quad\text{and}\quad 
    T \geq \frac{16 \beta}{ \mu^2 } \frac{1}{\epsilon} + \frac{8 \sqrt{2} \, \alpha \, \norm{\vlambda_0 - \vlambda^*}_2 }{ \mu^2} \frac{1}{\sqrt{\epsilon}}.
  \end{align*}
\end{theoremEnd}
\begin{proofEnd}
  Theorem 6 of \citet{domke_provable_2023} utilizes the two-stage stepsize of \citet{gower_sgd_2019}.
  After \(T\) steps, with a carefully tuned stepsize of 
  \[
    \gamma_t = \min\left( \frac{\mu}{2 \alpha}, \frac{4 t + 2 }{ \mu \, {\left( t + 1\right)}^2 } \right)
  \]
  the algorithm achieves
  \[
    \norm{\vlambda_{T} - \vlambda^*}_2^2
    \leq
    \frac{64 \alpha^2}{\mu^4} \frac{ \norm{\vlambda_0 - \vlambda^*}_2^2 }{T^2} 
    +
    \frac{32 \beta}{\mu^2} \frac{1}{T}.
  \]
  Following a similar strategy to \citet{kim_blackbox_2023}, we can obtain a computational complexity by solving for the smallest \(T\) that achieves
  \[
    \frac{64 \alpha}{\mu^2} \frac{ \norm{\vlambda_0 - \vlambda^*}_2^2 }{T^2} 
    +
    \frac{16 \beta}{\mu^2} \frac{1}{T}
    \leq
    \epsilon.
  \]
  After re-organizing, we solve for
  \[
     A \, T^2  + B \, T + C = 0,
  \]
  where 
  \[
    A = \epsilon, \quad
    B = -\frac{16 \beta}{\mu^2},\; \text{and} \quad
    C = - \frac{64 \alpha^2}{\mu^4} \norm{\vlambda_0 - \vlambda^*}_2^2.
  \]
  Since \(T > 0\), the equation has a unique root
  \begin{align*}
    T 
    &= \frac{ - B + \sqrt{ B^2 - 4 A C  } }{ 2 A },
\shortintertext{applying the inequality \(\sqrt{a + b} \leq \sqrt{a} + \sqrt{b}\) for \(a, b \geq 0\),}
    &\leq \frac{ - B + \sqrt{ B^2 } +  \sqrt{  4 A \left( -C \right)  } }{ 2 A }
    \\
    &= \frac{2 \left(-B\right) }{2 A} + \frac{ \sqrt{ 4 A  \left( -C \right)  } }{ 2 A }
    \\
    &= \frac{\left(-B\right)}{A} + \frac{ \sqrt{ 2 \left( -C \right) }  } { \sqrt{A} }
    \\
    &= \frac{ 16 \beta }{ \mu^2 \epsilon} + \frac{ \sqrt{  \frac{128\alpha^2}{\mu^4} \norm{\vlambda_0 - \vlambda^*}_2^2  }  } { \sqrt{\epsilon} }
    \\
    &= \frac{ 16 \beta }{ \mu^2 \epsilon} + \frac{ 8 \sqrt{2}  \alpha \norm{\vlambda_0 - \vlambda^*}_2 } { \mu^2 \sqrt{\epsilon} }.
  \end{align*}
\end{proofEnd}

% 
\begin{theoremEnd}[category=complexityprojsgdqvc]{theorem}[\textbf{Convex \(F\) with a fixed stepsize}]\label{thm:projsgd_convex_fixedstepsize}
  For a convex \(F : \Lambda \to \mathbb{R}\) on a convex set \(\Lambda\), the last weighted average of the iterates generated by projected SGD, \(\bar{\vlambda}_T = \sum^{T}_{t=0}  w^{t+1} \vlambda_t \big/ \sum^T_{t=0} w^{t+1}\), where \(w = 1 / \left(1 + \alpha \gamma^2 \right)\), satisfies \( F\left(\bar{\vlambda}_T\right) - F\left(\vlambda^*\right) \leq \epsilon \), where \(\vlambda^* \in \argmin_{\vlambda \in \Lambda} F\left(\vlambda\right)\), if
  \[
  \gamma =
  \frac{
    -\beta + \sqrt{
      \beta^2 + 8 \epsilon^2 \alpha 
    } 
  }{
    4 \epsilon \alpha
  }
  \quad\text{and}\quad
  T
  \geq
  \frac{2 \alpha}{ -\beta + \sqrt{ \beta^2 + 8 \epsilon^2 \alpha } }
  \, \norm{\vlambda_0 - \vlambda^*}_2^2.
  \]
\end{theoremEnd}
\begin{proofEnd}
  Theorem 7 of \citet{domke_provable_2023} states that
  \[
     F\left(\bar{\vlambda}_T\right) - F\left(\vlambda^*\right)
     \leq
     \frac{ \gamma \alpha  }{ 2 \left( 1 - \theta^{T} \right) } \norm{\vlambda_0 - \vlambda^*}_2^2 + \frac{\gamma \beta}{2},
  \]
  where \(\theta = 1 / (1 + 2 \alpha \gamma^2)\).
  For the first term, we generalize of Lemma 28 of \citet{domke_provable_2023} as
  \begin{align}
    \frac{\gamma \alpha }{2 \left( 1 - \theta^{T} \right) } 
    &\leq \frac{\gamma \alpha}{2 \left( 1 - \theta \right) {T}}
    =    \frac{\gamma \alpha}{2 \left( 1 - \frac{1}{1 + 2 \alpha \gamma^2} \right) {T}}
    =    \frac{\gamma \alpha}{2 \left( \frac{2 \alpha \gamma^2 }{1 + 2 \alpha \gamma^2} \right) {T}}
    =    \frac{ 1 + 2 \alpha \gamma^2 }{4 \gamma} \frac{1}{T}
    \nonumber
    \\
    &=    \frac{1}{2} \left(\alpha \gamma + \frac{1}{2 \gamma} \right) \frac{1}{T}.
    \label{eq:stl_convex_fixedstepsize_optterm}
  \end{align}
  Note that the use of Bernoulli's inequality here is quite loose.
  Originally, \citeauthor{domke_provable_2023} chose \(\gamma = 1/\sqrt{T}\), which means one needs to fix the number of SGD iterations before actually running the algorithm.
  We instead prove convergence with a stepsize independent of \(T\).

  Based on \cref{eq:stl_convex_fixedstepsize_optterm}, the fixed-stepsize any-time convergence result becomes
  \[
     F\left(\bar{\vlambda}_T\right) - F\left(\vlambda^*\right)
     \leq
     \underbrace{
       \frac{1}{2} \left(\alpha \gamma + \frac{1}{2 \gamma} \right) \frac{1}{T} \, \norm{\vlambda_0 - \vlambda^*}_2^2
     }_{\text{optimization term}}
     +
     \underbrace{
       \frac{\gamma \beta}{2}.
     }_{\text{statistical term}}
  \]

  Define
  \begin{align*}
    \epsilon_{\mathrm{opt.}}
    \triangleq
    \frac{1}{2} \left(\alpha \gamma + \frac{1}{2 \gamma} \right) \frac{1}{T} \, \norm{\vlambda_0 - \vlambda^*}_2^2,
    %
    \quad\text{and}\quad
    %
    \epsilon_{\mathrm{stat.}}
    \triangleq
    \frac{\gamma \beta}{2}.
  \end{align*}
  We aim to solve the convex program:
  \begin{alignat*}{3}
    &\minimize_{T, \gamma}\quad  &&T \\
    &\text{subject to}\quad && \epsilon_{\text{opt.}}\left(T, \gamma\right) + \epsilon_{\text{stat.}}\left(\gamma\right) = \epsilon \\
    & && \gamma > 0,\quad T > 0.
  \end{alignat*}
  The Lagrangian is given as
  \begin{alignat*}{2}
    \mathcal{L}\left(T, \gamma, \lambda\right)
    =
    T + \lambda \left(\epsilon_{\text{opt.}}\left(T, \gamma\right) + \epsilon_{\text{stat.}}\left(\gamma\right) - \epsilon \right)
  \end{alignat*}
  with respect to the Lagrangian multiplier \(\lambda\).

  The stationary point of \(\mathcal{L}\) is found by solving the system of equations
  \begin{alignat*}{4}
    \frac{\partial \mathcal{L}}{\partial T}
    &=
    \;
    1 + \lambda \frac{\partial \epsilon_{\text{opt.}}}{\partial T}
    \quad&&=
    0
    &&\qquad\Leftrightarrow\qquad
    -\lambda \frac{\partial \epsilon_{\text{opt.}}}{\partial T}
    &&=
    1 
    \\
    %
    \frac{\partial \mathcal{L}}{\partial \lambda}
    &=
    \;
    \epsilon_{\text{opt.}}\left(T, \gamma\right)
    +
    \epsilon_{\text{stat.}}\left(\gamma\right)
    - 
    \epsilon
    &&=
    0
    &&\qquad\Leftrightarrow\qquad
    \epsilon_{\text{opt.}}\left(T, \gamma\right)
    +
    \epsilon_{\text{stat.}}\left(\gamma\right)
    &&=
    \epsilon
    \\
    %
    \frac{\partial \mathcal{L}}{\partial \gamma}
    &=
    \;
    \lambda
    \left(
    \frac{\partial \epsilon_{\text{opt.}}}{\partial \gamma}
    +
    \frac{\partial \epsilon_{\text{stat.}}}{\partial \gamma}
    \right)
    &&=
    0
    &&\qquad\Leftrightarrow\qquad
    \frac{\partial \epsilon_{\text{opt.}}}{\partial \gamma}
    &&=
    -\frac{\partial \epsilon_{\text{stat.}}}{\partial \gamma}
  \end{alignat*}
  The partial derivatives are given as
  \begin{align*}
    \frac{ \partial \epsilon_{\mathrm{opt.}} }{ \partial T }
    &=
    - \frac{1}{2} \left(\alpha \gamma + \frac{1}{2 \gamma}\right) \norm{\vlambda_0 - \vlambda^*}_2^2 \frac{1}{T^2}
    \\
    \frac{ \partial \epsilon_{\mathrm{opt.}} }{ \partial \gamma }
    &=
    \frac{1}{2} \left(\alpha - \frac{1}{2 \gamma^2}\right) \norm{\vlambda_0 - \vlambda^*}_2^2 \frac{1}{T}
    \\
    \frac{ \partial \epsilon_{\mathrm{stat.}} }{ \partial \gamma }
    &=
    \frac{\beta}{2}.
  \end{align*}
  Applying these to the Lagrangian system, we now need to solve the system:
  \begin{alignat}{2}
    \frac{\lambda}{2} \left(\alpha \gamma + \frac{1}{2 \gamma}\right) \norm{\vlambda_0 - \vlambda^*}_2^2 \frac{1}{T^2} &= 1
    \label{eq:lagrangian1}
    \\
    \frac{1}{2} \left(\alpha \gamma + \frac{1}{2 \gamma} \right) \frac{1}{T} \, \norm{\vlambda_0 - \vlambda^*}_2^2
    +
    \frac{\gamma \beta}{2}
    &=
    \epsilon
    \label{eq:lagrangian2}
    \\
    \frac{1}{2} \left(\alpha - \frac{1}{2 \gamma^2}\right) \norm{\vlambda_0 - \vlambda^*}_2^2 \frac{1}{T}
    &=
    -\frac{\beta}{2}.
    \label{eq:lagrangian3}
  \end{alignat}
  Notice that \(\lambda\) in \cref{eq:lagrangian1} is a free variable.
  Thus the last two equations \cref{eq:lagrangian2,eq:lagrangian3} are the only relevant equations.
  In particular, from \cref{eq:lagrangian3}, we obtain the identity
  \begin{alignat}{2}
    \norm{\vlambda_0 - \vlambda^*}_2^2 \frac{1}{T}
    =
    \frac{ 2 \beta \gamma^2 }{ 1 - 2 \gamma^2 \alpha }.
    \label{eq:projsgd_convex_keyeq}
  \end{alignat}
  Applying this to \cref{eq:lagrangian2}, we can decouple \(T\) and \(\gamma\), obtaining the quadratic equation
  \begin{alignat*}{3}
    & &\quad
    \frac{1}{2} \left(\alpha \gamma + \frac{1}{2 \gamma} \right) \frac{1}{T} \, \norm{\vlambda_0 - \vlambda^*}_2^2
    +
    \frac{\gamma \beta}{2}
    &=
    \epsilon
    \\
    &\Leftrightarrow&\quad
    \frac{1}{2} \left(\alpha \gamma + \frac{1}{2 \gamma} \right) 
    \left(
    \frac{ 2 \beta \gamma^2 }{ 1 - 2 \gamma^2 \alpha }
    \right)
    +
    \frac{\gamma \beta}{2}
    &=
    \epsilon
    \\
    &\Leftrightarrow&\quad
    \left( 2 \epsilon \alpha \right) \gamma^2 
    +
    \beta \gamma 
    -
    \epsilon
    &=
    0.
  \end{alignat*}
  Since \(\gamma > 0\), this quadratic has the unique solution
  \begin{alignat}{2}
    \gamma = \frac{
      -\beta + \sqrt{
        \beta^2 + 8 \epsilon^2 \alpha 
      } 
    }{
      4 \epsilon \alpha
    } > 0.
    \label{eq:projsgd_convex_optgamma}
  \end{alignat}
  as long as \(0 < \alpha < \infty\).

  Note that \cref{eq:projsgd_convex_keyeq} can be represented as
  \begin{alignat*}{3}
    & &\quad
    \norm{\vlambda_0 - \vlambda^*}_2^2 \frac{1}{T}
    &=
    \frac{ 2 \beta \gamma^2 }{ 1 - 2 \gamma^2 \alpha }
    \\
    &\Leftrightarrow&\quad
    T
    &=
    \left( \frac{1}{\gamma^2} - 2 \alpha \right) \frac{1}{2 \beta}
    \norm{\vlambda_0 - \vlambda^*}_2^2.
  \end{alignat*}
  Plugging \cref{eq:projsgd_convex_optgamma},
  \begin{alignat*}{3}
    T
    &=
    \left(
      \frac{
        16 \epsilon^2 \alpha^2
      }{
        {\left( -\beta + \sqrt{
          \beta^2 + 8 \epsilon^2 \alpha 
        }
        \right)}^2
      }
    - 2 \alpha \right) \frac{1}{2 \beta}
    \norm{\vlambda_0 - \vlambda^*}_2^2
    \\
    &=
    \frac{2 \alpha}{ -\beta + \sqrt{ \beta^2 + 8 \epsilon^2 \alpha } }
    \, \norm{\vlambda_0 - \vlambda^*}_2^2.
  \end{alignat*}
\end{proofEnd}


%% \begin{remark}
%%   The bound for \cref{thm:projsgd_convex_fixedstepsize} is inconveniently non-linear with respect to \(\beta\) and \(\alpha\).
%%   Under ``interpolation'' such that \(\beta = 0\), it clearly reduces to a \(\mathcal{O}\left(1/\epsilon\right)\).
%%   When \(\beta > 0\) is non-negligible, a series expansion does suggest that the bound behaves as \(\mathcal{O}\left(1/\epsilon^2\right)\), which is the expected complexity guarantee~\citep{garrigos_handbook_2023}.
%% \end{remark}

%% \begin{remark}
%%   Note that \cref{thm:projsgd_convex_fixedstepsize} is quite loose due to the use of Bernoulli's inequality for simplifying the any-time convergence result by \citet[Theorem 7]{domke_provable_2023}. (This inequality is also used by \citeauthor{domke_provable_2023}.) 
%%   Tightening the any-time convergence statement would be an important future direction.
%% \end{remark}

% \begin{remark}
%   Lastly, obtaining a \(\mathcal{O}\left(1/\sqrt{T}\right)\) convergence guarantee with a decreasing stepsize schedule independent of the number of steps \(T\) is an open problem.
%   (\citet[Theorem 7]{domke_provable_2023} use a fixed stepsize dependent on \(T\). Thus, one must fix the number of steps before running projected SGD.) 
% \end{remark}

\vspace{-2.ex}
\subsection{Non-Asymptotic Complexity of Black-Box Variational Inference}\label{section:bbvicomplexity}
\vspace{-1.ex}
We now apply the general complexity results obtained in the previous section to BBVI.
We will focus on 
\begin{enumerate*}[label=\textbf{(\roman*)}]
    \item strongly log-concave posteriors,
    \item SGD run with fixed stepsizes, and 
    \item the full-rank variational family.
\end{enumerate*}
This is because the convergence analyses for \textbf{(ii)} \(\cap\) \textbf{(iii)} are the tightest.
Although the bounds for the mean-field parameterization have better dependences on \(d\), they have not been shown to be tight and empirically appear loose~\citep{kim_practical_2023}. (See also \citealp[Conjecture 1]{kim_blackbox_2023}.)

\vspace{-1.ex}
\paragraph{Complexity with Adaptive QV Estimators}
As mentioned in \S, we established \textit{adaptive} QV bounds.
For the complexity guarantees for strongly convex objectives (\cref{thm:projsgd_stronglyconvex_fixedstepsize,thm:projsgd_stronglyconvex_decstepsize}), it is possible to optimize the free parameter \(\delta\) in the bounds, such that they automatically adapt to other problem-specific constants.
The following generic lemma does this:
%

\begin{theoremEnd}[all end, category=complexityprojsgdadaptiveqvcfixed]{lemma}[\textbf{Strongly convex \(F\) with adaptive QV and Fixed Stepsize}]\label{thm:projsgd_stronglyconvex_adaptive_complexity}
  For a \(\mu\)-strongly convex \(F : \Lambda \to \mathbb{R}\) on a convex set \(\Lambda\) the last iterate \(\vlambda_T\) of projected SGD with a gradient estimator satisfying an adaptive QV bound (\cref{assumption:adaptiveqvc}) is \(\epsilon\)-close to \(\vlambda^* = \argmin_{\vlambda \in \Lambda} F\left(\vlambda\right)\) such that \(\norm{\vlambda_T - \vlambda^{*}}_2^2 < \epsilon\) if
{%
\setlength{\abovedisplayskip}{.5ex} \setlength{\abovedisplayshortskip}{.5ex}
\setlength{\belowdisplayskip}{1.ex} \setlength{\belowdisplayshortskip}{1.ex}
  \begin{align*}
    \gamma &=
    \min\left(
      \frac{1}{2}
      \frac{
        \mu
      }{
        \widetilde{\alpha} + 2 \widetilde{\beta} \epsilon^{-1} 
      }\, ,\,
      \frac{2}{\mu}
    \right)  \quad\text{and}
    \\
    %
    T &\geq
    \frac{2}{\mu^2} \max\left(\widetilde{\alpha} + 2 \widetilde{\beta} \frac{1}{\epsilon}, \; \frac{\mu^2}{4}\right) \log \left( 2 \norm{\vlambda_0 - \vlambda^*}_2^2 \, \frac{1}{\epsilon} \right).
  \end{align*}
}
\end{theoremEnd}
\vspace{-1ex}
\begin{proofEnd}\label{proof:projsgd_stronglyconvex_adaptive_complexity}
  Recall that, for a stepsize \(\gamma\) and a number of steps \(T\) satisfying 
  \begin{align*}
    \gamma \leq \min\left( \frac{\epsilon \mu}{4 \beta}, \frac{\mu}{2 \alpha}, \frac{2}{\mu} \right)
    \quad\text{and}\quad
    T \geq \max\left( \frac{ 4 \beta }{\mu^2 } \frac{1}{\epsilon}, \frac{2 \alpha}{\mu^2}, \frac{1}{2} \right) \log \left( 2 \norm{\vlambda_0 - \vlambda^*}_2^2 \, \frac{1}{\epsilon} \right),
  \end{align*}
  we can guarantee that the iterate \(\vlambda_t\) can guarantee \(\mathbb{E} \norm{\vlambda^* - \vlambda_T}_2^2 \leq \epsilon\).

  We optimize the parameter \(\delta\) to minimize the number of steps.
  That is,
  \begin{align*}
    \max\left( \frac{ 4 \beta }{\mu^2 } \frac{1}{\epsilon}, \frac{2 \alpha}{\mu^2}, \frac{1}{2} \right) \log \left( 2 \norm{\vlambda_0 - \vlambda^*}_2^2 \, \frac{1}{\epsilon} \right)
    =
    \frac{2}{\mu^2} \max\left( 2 (1 + C^{-1} \delta^{-1}) \,\widetilde{\beta} \frac{1}{\epsilon}, (1 + C \delta) \widetilde{\alpha}, \frac{\mu^2}{4}\right)
    \log \left( 2 \norm{\vlambda_0 - \vlambda^*}_2^2 \, \frac{1}{\epsilon} \right).
  \end{align*}
  Since the first and second arguments of the max function are monotonic with respect to \(\delta\), the optimum is unique, and achieved when the two terms are equal.
  That is,
  \begin{alignat*}{2}
    & &
    2 (1 + C^{-1} \delta^{-1}) \,\widetilde{\beta} \frac{1}{\epsilon}
    &=
    (1 + C \delta) \widetilde{\alpha}
    \\
    &\Leftrightarrow&\qquad
    \frac{2 \widetilde{\beta}}{\epsilon} + \frac{2 \widetilde{\beta} C^{-1}}{\epsilon} \delta^{-1} \,
    &=
    \widetilde{\alpha} + \widetilde{\alpha} C \delta 
    \\
    &\Leftrightarrow&\qquad
    \frac{2 \widetilde{\beta}}{\epsilon} \delta +  \frac{2 \widetilde{\beta} C^{-1}}{\epsilon}  \,
    &=
    \widetilde{\alpha} \delta + \widetilde{\alpha} C \delta^2
    \\
    &\Leftrightarrow&\qquad
    %
    \widetilde{\alpha} C \delta^2 + \left( \widetilde{\alpha} - \frac{2 \widetilde{\beta}}{\epsilon} \right) \delta - \frac{2 \widetilde{\beta} C^{-1}}{\epsilon}
    &=
    0
    \\
    &\Leftrightarrow&\qquad
    %
    \left(
    \widetilde{\alpha} \delta
    - 
    \frac{2 \widetilde{\beta} C^{-1}}{\epsilon}
    \right)
    \left(
      C \delta + 1
    \right)
    &=
    0.
  \end{alignat*}
  Conveniently, we have a unique feasible solution
  \begin{alignat*}{2}
    \delta
    =
    2 
    \frac{
      \widetilde{\beta}
    }{
      \widetilde{\alpha} 
    }
    C^{-1}
    \epsilon^{-1}.
  \end{alignat*}

  Thus, the optimal bound is obtained by setting
  \(
    \delta
    = 2 
    \frac{
      \widetilde{\beta}
    }{
      \widetilde{\alpha} 
    }
    C^{-1}
    \epsilon^{-1},
  \)
  such that
  \begin{align*}
    T
    &\geq
    \frac{2}{\mu^2} \max\left(  2 \beta \frac{1}{\epsilon}, \alpha, \frac{\mu^2}{4} \right) \log \left( 2 \norm{\vlambda_0 - \vlambda^*}_2^2 \, \frac{1}{\epsilon} \right)
    \\
    &=
    \frac{2}{\mu^2} \max\left( 2 \left( 1 + C^{-1} \delta^{-1} \right) \widetilde{\beta}  \frac{1}{\epsilon},  \left( 1 + C \delta \right) \widetilde{\alpha}, \frac{\mu^2}{4} \right)
    \log \left( 2 \norm{\vlambda_0 - \vlambda^*}_2^2 \, \frac{1}{\epsilon} \right)
    \\
    &=
    \frac{2}{\mu^2} \max\left( \widetilde{\alpha} + 2 \widetilde{\beta} \frac{1}{\epsilon}, \frac{\mu^2}{4} \right) \log \left( 2 \norm{\vlambda_0 - \vlambda^*}_2^2 \, \frac{1}{\epsilon} \right).
  \end{align*}
  The stepsize with the optimal \(\delta\) is consequently
  \begin{align*}
    \gamma
    &\leq
    \min\left( \frac{\epsilon \mu}{4 \beta}, \frac{\mu}{2 \alpha}, \frac{2}{\mu} \right)
    =
    \min\left( \frac{\epsilon \mu}{4 (1 + C^{-1} \delta^{-1}) \widetilde{\beta}} \,, \; \frac{\mu}{2 (1 + C \delta) \widetilde{\alpha}}, \frac{2}{\mu} \right)
    =
    \min\left(
      \frac{1}{2}
      \frac{
        \mu
      }{
        \widetilde{\alpha} + 2 \widetilde{\beta} \epsilon^{-1} 
      }\, ,\;
      \frac{2}{\mu}
    \right).
  \end{align*}
\end{proofEnd}


\begin{theoremEnd}[all end, category=complexityprojsgdadaptiveqvcdec]{lemma}[\textbf{Strongly convex \(F\) with adaptive QV and Decreasing Stepsize}]\label{thm:projsgd_stronglyconvex_decstepsize_adaptive_complexity}
  For a \(\mu\)-strongly convex \(F : \Lambda \to \mathbb{R}\) on a convex set \(\Lambda\) with a unique global minimizer \(\vlambda^* \in \Lambda\), the last iterate \(\vlambda_T\) of projected SGD with a gradient estimator satisfying an adaptive QVC bound (\cref{assumption:adaptiveqvc}) and a decreasing stepsize satisfies a suboptimality of \(\norm{\vlambda_T - \vlambda_{*}}_2^2 < \epsilon\) if
  {
  \begin{align*}
    \gamma_t
    &=
    \min\Bigg(
    \frac{
      \mu
    }{
      2 \widetilde{\alpha}
      +
      \sqrt{2 \norm{\vlambda_0 - \vlambda^*}_2 }
      \,
      \epsilon^{1/4}
      \,
      \widetilde{\alpha}^{3/2}
      \widetilde{\beta}^{-1/2}
    },
    \frac{4 t + 2 }{ \mu \, {\left( t + 1\right)}^2 }
    \Bigg)
    \\
    T
    &\geq
    \frac{16 \widetilde{\beta} }{ \mu^2 } \frac{1}{\epsilon} 
    +
    \frac{16 \sqrt{2}}{ \mu^2 }
    \sqrt{
      \norm{\vlambda_0 - \vlambda^*}_2
    }
    \,
    \sqrt{
      \widetilde{\alpha} \widetilde{\beta}
    }
    \,
    \frac{1}{\epsilon^{3/4}}
    +
    \frac{8 \widetilde{\alpha} \, \norm{\vlambda_0 - \vlambda^*}_2 }{ \mu^2}
    \frac{1}{\sqrt{\epsilon}}.
  \end{align*}
  }%
\end{theoremEnd}
\begin{proofEnd}\label{proof:projsgd_stronglyconvex_decstepsize_adaptive_complexity}
  Recall that, for a stepsize \(\gamma\) and a number of steps \(T\) such that
  \begin{align*}
    \gamma_t = \min\left( \frac{\mu}{2 \alpha}, \frac{4 t + 2 }{ \mu \, {\left( t + 1\right)}^2 } \right)
    \quad\text{and}\quad
    T \geq \frac{16 \beta}{ \mu^2 } \frac{1}{\epsilon} + \frac{8  \alpha \, \norm{\vlambda_0 - \vlambda^*}_2 }{ \mu^2} \frac{1}{\sqrt{\epsilon}},
  \end{align*}
  we can guarantee that the iterate \(\vlambda_t\) can guarantee \(\mathbb{E} \norm{\vlambda^* - \vlambda_T}_2^2 \leq \epsilon\).

  We optimize the parameter \(\delta\) to minimize the required number of steps \(T\).
  That is, we maximize
  \begin{align*}
    \frac{16 \beta }{ \mu^2 } \frac{1}{\epsilon} + \frac{8 \sqrt{2} \, \alpha \, \norm{\vlambda_0 - \vlambda^*}_2 }{ \mu^2} \frac{1}{\sqrt{\epsilon}}
    = \frac{16 \left(1 + C \delta\right) \widetilde{\beta} }{ \mu^2 } \frac{1}{\epsilon} + \frac{8 \left(1 + C^{-1} \delta^{-1}\right) \widetilde{\alpha} \, \norm{\vlambda_0 - \vlambda^*}_2 }{ \mu^2} \frac{1}{\sqrt{\epsilon}}.
  \end{align*}
  This is clearly a convex function with respect to \(\delta\).
  Thus, we only need to find a first-order stationary point 
  {\small%
  \begin{align*}
    \frac{\mathrm{d}}{\mathrm{d} \delta}
    \left(
      \frac{16 \left(1 + C \delta\right) \widetilde{\beta} }{ \mu^2 } \frac{1}{\epsilon} + \frac{8 \left(1 + C^{-1} \delta^{-1}\right) \widetilde{\alpha} \, \norm{\vlambda_0 - \vlambda^*}_2 }{ \mu^2} \frac{1}{\sqrt{\epsilon}}
    \right) &= 0.
  \end{align*}
  }%
  Differentiating, we have
  \begin{alignat*}{3}
    &&
    \frac{16 C \widetilde{\beta} }{ \mu^2 } \frac{1}{\epsilon} - \frac{8  \, C^{-1} \delta^{-2} \widetilde{\alpha} \, \norm{\vlambda_0 - \vlambda^*}_2 }{ \mu^2} \frac{1}{\sqrt{\epsilon}}
    &= 0,
\shortintertext{multiplying \(\delta^2\) to both sides,}
    &\Leftrightarrow&\qquad
    \delta^2 \frac{16 C \widetilde{\beta} }{ \mu^2 } \frac{1}{\epsilon} - \frac{8 \sqrt{2} \, C^{-1} \widetilde{\alpha} \, \norm{\vlambda_0 - \vlambda^*}_2 }{ \mu^2} \frac{1}{\sqrt{\epsilon}}
    &= 0.
  \end{alignat*}
  Reorganizing,
  \begin{alignat*}{3}
    &\Leftrightarrow&\qquad
    \delta^2  \frac{16 C \widetilde{\beta} }{ \mu^2 } \frac{1}{\epsilon} 
    &=
    \frac{8 \sqrt{2} \, C^{-1} \widetilde{\alpha} \, \norm{\vlambda_0 - \vlambda^*}_2 }{ \mu^2} \frac{1}{\sqrt{\epsilon}}
    \\
    &\Leftrightarrow&\qquad
    \delta^2  
    &=
    \left( \frac{\mu^2 \epsilon}{ 16 C \widetilde{\beta} } \right)
    \left(
    \frac{8  C^{-1} \widetilde{\alpha} \, \norm{\vlambda_0 - \vlambda^*}_2 }{ \mu^2} \frac{1}{\sqrt{\epsilon}}
    \right)
    \\
    &\Leftrightarrow&\qquad
    \delta^2  
    &=
    \frac{C^{-2} \widetilde{\alpha} \, \norm{\vlambda_0 - \vlambda^*}_2}{ 2 \widetilde{\beta}} \sqrt{\epsilon},
\shortintertext{and taking the square-root of both sides,}
    &\Leftrightarrow&\qquad
    \delta  
    &=
    \frac{
      \sqrt{ \norm{\vlambda_0 - \vlambda^*}_2 }
      \,
      \epsilon^{1/4}
      \,
      \sqrt{\widetilde{\alpha}}
    }{
      \sqrt{2} \,
      C 
      \sqrt{\widetilde{\beta}}
    }.
  \end{alignat*}
  Recall that the required number of iterations is
  \begin{align*}
    T
    &\geq
    \frac{16 \left(1 + C \delta\right) \widetilde{\beta} }{ \mu^2 } \frac{1}{\epsilon} 
    +
      \frac{8 \left(1 + C^{-1} \delta^{-1}\right) \widetilde{\alpha} \, \norm{\vlambda_0 - \vlambda^*}_2 }{ \mu^2} \frac{1}{\sqrt{\epsilon}}
    \\
    &=
    \underbrace{
      \frac{16 \widetilde{\beta} }{ \mu^2 } \frac{1}{\epsilon} 
      +
      \frac{16 \widetilde{\beta} }{ \mu^2 } \frac{1}{\epsilon} 
      C \delta
    }_{T_{\text{\ding{172}}}}
    +
    \underbrace{
      \frac{8 \widetilde{\alpha} \, \norm{\vlambda_0 - \vlambda^*}_2 }{ \mu^2} \frac{1}{\sqrt{\epsilon}}
      +
      \frac{8 \widetilde{\alpha} \, \norm{\vlambda_0 - \vlambda^*}_2 }{ \mu^2} \frac{1}{\sqrt{\epsilon}}
      C^{-1} \delta^{-1}
    }_{T_{\text{\ding{173}}}}.
  \end{align*}
  Plugging \(\delta\) in, we have
  \begin{align*}
    T_{\text{\ding{172}}}
    &=
    \frac{16 \widetilde{\beta} }{ \mu^2 } \frac{1}{\epsilon} 
    +
    \frac{16 \widetilde{\beta} }{ \mu^2 } \frac{1}{\epsilon}
    C 
    \left(
    \frac{
      \sqrt{ \norm{\vlambda_0 - \vlambda^*}_2 }
      \,
      \epsilon^{1/4}
      \,
      \sqrt{\widetilde{\alpha}}
    }{
      \sqrt{2} \, C
      \sqrt{\widetilde{\beta}}
    }
    \right)
    \\
    &=
    \frac{16 \widetilde{\beta} }{ \mu^2 } \frac{1}{\epsilon} 
    +
    \frac{8 \sqrt{2} }{ \mu^2 }
    \sqrt{
      \widetilde{\alpha} \widetilde{\beta}
    } \,
    \sqrt{ \norm{\vlambda_0 - \vlambda^*}_2 }
    \,
    \epsilon^{-3/4}
    \\
    \\
    T_{\text{\ding{173}}}
    &=
    \frac{8 \, \widetilde{\alpha} \, \norm{\vlambda_0 - \vlambda^*}_2 }{ \mu^2} \frac{1}{\sqrt{\epsilon}}
    +
    \frac{8 \, \widetilde{\alpha} \, \norm{\vlambda_0 - \vlambda^*}_2 }{ \mu^2} \frac{1}{\sqrt{\epsilon}}
    C^{-1} 
    \left(
    \frac{
      \sqrt{2} \, C
      \sqrt{\widetilde{\beta}}
    }{
      \sqrt{ \norm{\vlambda_0 - \vlambda^*}_2 }
      \,
      \epsilon^{1/4}
      \,
      \sqrt{\widetilde{\alpha}}
    }
    \right)
    \\
    &=
    \frac{8 \widetilde{\alpha} \, \norm{\vlambda_0 - \vlambda^*}_2 }{ \mu^2} \frac{1}{\sqrt{\epsilon}}
    +
    \frac{8 \sqrt{2}}{ \mu^2}
    \sqrt{ \norm{\vlambda_0 - \vlambda^*}_2 }
    \,
    \sqrt{
      \widetilde{\alpha}
      \widetilde{\beta}
    }
    \,
    \epsilon^{-3/4}.
  \end{align*}
  Combining the results, 
  \begin{align*}
    T
    \geq
    T_{\text{\ding{172}}}
    +
    T_{\text{\ding{173}}}
    &=
    \frac{16 \widetilde{\beta} }{ \mu^2 } \frac{1}{\epsilon} 
    +
    \frac{8 \sqrt{2} }{ \mu^2 }
    \sqrt{
      \widetilde{\alpha} \widetilde{\beta}
    } \,
    \sqrt{ \norm{\vlambda_0 - \vlambda^*}_2 }
    \,
    \epsilon^{-3/4}
    \\
    &\qquad+
    \frac{8 \widetilde{\alpha} \, \norm{\vlambda_0 - \vlambda^*}_2 }{ \mu^2} \frac{1}{\sqrt{\epsilon}}
    +
    \frac{8 \sqrt{2}}{ \mu^2}
    \sqrt{ \norm{\vlambda_0 - \vlambda^*}_2 }
    \,
    \sqrt{
      \widetilde{\alpha}
      \widetilde{\beta}
    }
    \,
    \epsilon^{-3/4}
    \\
    &=
    \frac{16 \widetilde{\beta} }{ \mu^2 } \frac{1}{\epsilon} 
    +
    \frac{16 \sqrt{2}}{ \mu^2 }
    \sqrt{
      \norm{\vlambda_0 - \vlambda^*}_2
    }
    \,
    \sqrt{
      \widetilde{\alpha} \widetilde{\beta}
    }
    \,
    \epsilon^{-3/4}
    +
    \frac{8 \widetilde{\alpha} \, \norm{\vlambda_0 - \vlambda^*}_2 }{ \mu^2}
    \frac{1}{\sqrt{\epsilon}}.
  \end{align*}

  For the stepsize
  \begin{align*}
    \gamma
    =
    \min\left( \frac{\mu}{2 \alpha}, \frac{4 t + 2 }{ \mu \, {\left( t + 1\right)}^2 } \right)
    =
    \min\left(
      \frac{\mu}{2 \left(1 + C \delta \right) \widetilde{\alpha}},
      \frac{4 t + 2 }{ \mu \, {\left( t + 1\right)}^2 }
    \right),
  \end{align*}
  we have 
  \begin{align*}
    2 \left(1 + C \delta \right) \widetilde{\alpha}
    &=
    2 \widetilde{\alpha} + 2 \widetilde{\alpha} C \delta 
    \\
    &=
    2 \widetilde{\alpha}
    +
    2 \widetilde{\alpha}
    C
    \left(
    \frac{
      \sqrt{ \norm{\vlambda_0 - \vlambda^*}_2 }
      \,
      \epsilon^{1/4}
      \,
      \sqrt{\widetilde{\alpha}}
    }{
      \sqrt{2} \, C
      \sqrt{\widetilde{\beta}}
    }
    \right)
    \\
    &=
    2 \widetilde{\alpha}
    +
    \sqrt{2}
    \sqrt{ \norm{\vlambda_0 - \vlambda^*}_2 }
    \,
    \epsilon^{1/4}
    \,
    \widetilde{\alpha}^{3/2}
    \widetilde{\beta}^{-1/2}.
  \end{align*}
  Therefore, 
  \begin{align*}
    \gamma
    =
    \min\left(
      \frac{\mu}{2 \left(1 + C \delta \right) \widetilde{\alpha}},
      \frac{4 t + 2 }{ \mu \, {\left( t + 1\right)}^2 }
    \right)
    =
    \min\left(
    \frac{
      \mu
    }{
      2 \widetilde{\alpha}
      +
      \sqrt{2 \norm{\vlambda_0 - \vlambda^*}_2 }
      \,
      \epsilon^{1/4}
      \,
      \widetilde{\alpha}^{3/2}
      \widetilde{\beta}^{-1/2}
    },
    \frac{4 t + 2 }{ \mu \, {\left( t + 1\right)}^2 }
    \right).
  \end{align*}
\end{proofEnd}


\begin{remark}
  The analogous result for the decreasing stepsize schedule, \cref{thm:projsgd_stronglyconvex_decstepsize}, is unfortunately quite ugly and had to be omitted from the main text.
  It can be found in \cref{section:complexity_adaptiveqvc}.
\end{remark}

\vspace{-1.ex}
\paragraph{Complexity of BBVI on Strongly-Log-Concave \(\pi\)}
We can now plug in the constants obtained in \cref{section:gradient_variance}.
This immediately establishes the iteration complexity resulting from the use of different gradient estimators.
%

\begin{theoremEnd}[category=complexitybbvicfefixed]{theorem}[\textbf{Complexity of Fixed Stepsize BBVI with CFE}]\label{thm:projsgd_bbvicfe_complexity}
  The last iterate \(\vlambda_T \in \Lambda_L\) of BBVI with the CFE estimator and projected SGD with a fixed stepsize applied to a \(\mu\)-strongly log-concave and \(L\)-log-smooth posterior is \(\epsilon\)-close to \(\vlambda^* = \argmin_{\vlambda \in \Lambda_L} F\left(\vlambda\right)\) such that \(\mathbb{E}\norm{ \vlambda_T - \vlambda^* }_2^2 \leq \epsilon\) if
{%\small%
\setlength{\abovedisplayskip}{.5ex} \setlength{\abovedisplayshortskip}{.5ex}
\setlength{\belowdisplayskip}{1.ex} \setlength{\belowdisplayshortskip}{1.ex}
  \begin{align*}
    T
    &\geq 
    2 \kappa^2 \left(d + k_{\varphi} + 4\right) \left(1  + 2 {\lVert \bar{\vlambda} - \vlambda^* \rVert}_2^2 \frac{1}{\epsilon}\right) 
    \log \left( 2 \Delta^2 \, \frac{1}{\epsilon} \right)
  \end{align*}
}%
  for some fixed stepsize \(\gamma\), where \(\Delta = {\lVert \vlambda_0 - \vlambda^*\rVert}_2\), and \(\kappa = L/\mu\) is the condition number.
\end{theoremEnd}
\vspace{-1ex}
\begin{proofEnd}\label{proof:projsgd_bbvicfe_complexity}
  From \cref{thm:cfe_upperbound} with \(S = L\), the CFE estimator satisfies adaptive QV with the constants
  \begin{alignat*}{2}
    \alpha_{\mathrm{CFE}} 
    = L^2 \left( d + k_{\varphi} + 4 \right) \left(1 + \delta\right)
    \qquad\text{and}\qquad
    \beta_{\mathrm{CFE}}  
    = L^2 \left( d + k_{\varphi} \right) \left( 1 + \delta^{-1}  \right) {\lVert \bar{\vlambda} - \vlambda^* \rVert}_2^2.
  \end{alignat*}
  Furthermore, for a \(\mu\)-strongly log-concave posterior and our variational parameterization,~\citet[Theorem 9]{domke_provable_2020} show that the ELBO is \(\mu\)-strongly convex.

  We can thus invoke \cref{thm:projsgd_stronglyconvex_adaptive_complexity} with 
  \begin{align*}
    \widetilde{\alpha} = L^2 \left(d + k_{\varphi} + 4\right),\qquad
    \widetilde{\beta}  = L^2 \left(d + k_{\varphi}\right) {\lVert \bar{\vlambda} - \vlambda^* \rVert}_2^2,\quad\text{and}\quad
    C = 1.
  \end{align*}
  This yields a lower bound on the number of iteration 
  \begin{align*}
    &\frac{2}{\mu^2} \max\left(\widetilde{\alpha} + 2 \widetilde{\beta} \frac{1}{\epsilon}, \; \frac{\mu^2}{4} \right) 
    \log \left( 2 \norm{\vlambda_0 - \vlambda^*}_2^2 \, \frac{1}{\epsilon} \right)
    \\
    &\;=
    \frac{2}{\mu^2} \max\left(L^2 \left(d + k_{\varphi} + 4\right) + 2 L^2 \left(d + k_{\varphi}\right) {\lVert \bar{\vlambda} - \vlambda^* \rVert}_2^2 \frac{1}{\epsilon},\; \frac{\mu^2}{4}\right) 
    \log \left( 2 {\lVert \vlambda_0 - \vlambda^*\rVert}_2^2 \, \frac{1}{\epsilon} \right),
\shortintertext{pulling out \(L\),}
    &\;=
    \frac{2 L^2}{\mu^2} \max\left( \left(d + k_{\varphi} + 4\right) + 2 \left(d + k_{\varphi}\right) {\lVert \bar{\vlambda} - \vlambda^* \rVert}_2^2 \frac{1}{\epsilon},\; \frac{\mu^2}{4 L^2}\right) 
    \log \left( 2 {\lVert \vlambda_0 - \vlambda^*\rVert}_2^2 \, \frac{1}{\epsilon} \right),
\shortintertext{and since \(\frac{\mu^2}{4 L^2} < \frac{1}{4}\) and the first argument is larger than 1, the max operation is redundant that}
    &\;=
    \frac{2 L^2}{\mu^2} \left( \left(d + k_{\varphi} + 4\right) + 2 \left(d + k_{\varphi}\right) {\lVert \bar{\vlambda} - \vlambda^* \rVert}_2^2 \frac{1}{\epsilon}\right) 
    \log \left( 2 {\lVert \vlambda_0 - \vlambda^*\rVert}_2^2 \, \frac{1}{\epsilon} \right).
\shortintertext{Now, using the trivial fact \(d + k_{\varphi} < d + k_{\varphi} + 4\) simplifies the bound as,}
    &\;<
    \frac{2 L^2}{\mu^2} \left(d + k_{\varphi} + 4\right) \left(1  + 2 {\lVert \bar{\vlambda} - \vlambda^* \rVert}_2^2 \frac{1}{\epsilon} \right) 
    \log \left( 2 {\lVert \vlambda_0 - \vlambda^*\rVert}_2^2 \, \frac{1}{\epsilon} \right)
    \\
    &\;=
    2 \kappa^2 \left(d + k_{\varphi} + 4\right) \left(1  + 2 {\lVert \bar{\vlambda} - \vlambda^* \rVert}_2^2 \frac{1}{\epsilon}\right) 
    \log \left( 2 {\lVert \vlambda_0 - \vlambda^*\rVert}_2^2 \, \frac{1}{\epsilon} \right).
  \end{align*}
  The optimal \(\delta\) is given as
  \begin{align*}
    \delta 
    = \frac{2}{\epsilon} \frac{\widetilde{\beta}}{\widetilde{\alpha}} C^{-1} 
    = \frac{2}{\epsilon} 
    \frac{
      L^2 \left(d + k_{\varphi}\right) {\lVert \bar{\vlambda} - \vlambda^* \rVert}_2^2
    }{
      L^2 \left(d + k_{\varphi} + 4\right)
    } 
    C^{-1} 
    =
    \frac{2}{\epsilon} \, \frac{d + k_{\varphi}}{d + k_{\varphi} + 4} \, {\lVert \bar{\vlambda} - \vlambda^* \rVert}_2^2 .
  \end{align*}
\end{proofEnd}

\begin{theoremEnd}[all end, category=complexitybbvicfedec]{theorem}[\textbf{Complexity of Decreasing Stepsize BBVI with CFE}]\label{thm:projsgd_bbvicfe_decstepsize_complexity}
  The last iterate \(\vlambda_T \in \Lambda_L\) of BBVI with the CFE estimator and projected SGD with a decreasing stepsize schedule applied to a \(\mu\)-strongly log-concave and \(L\)-log-smooth posterior is \(\epsilon\)-close to \(\vlambda^* = \argmin_{\vlambda \in \Lambda_L} F\left(\vlambda\right) \) such that \(\mathbb{E}\norm{ \vlambda_T - \vlambda^* }_2^2 \leq \epsilon\) if
  \begin{align*}
    T
    &\geq 
    16 \kappa^2 \left(d + k_{\varphi} + 4\right) 
    \bigg(
    {\lVert \bar{\vlambda} - \vlambda^* \rVert}_2^2 \frac{1}{\epsilon} 
    +
    2
    \sqrt{
      \norm{\vlambda_0 - \vlambda^*}_2
    }
    \,
    {\lVert \bar{\vlambda} - \vlambda^* \rVert}_2
    \,
    \frac{1}{\epsilon^{3/4}}
    +
    \norm{\vlambda_0 - \vlambda^*}_2 
    \frac{1}{\sqrt{\epsilon}}
    \bigg).
  \end{align*}
  for some decreasing stepsize schedule \(\gamma_1, \ldots, \gamma_T\), where \(\kappa = L/\mu\) is the condition number and \(\vlambda^* \in \Lambda\) is the optimal variational parameter.
\end{theoremEnd}
\begin{proofEnd}\label{proof:projsgd_bbvicfe_decstepsize_complexity}
  From \cref{thm:cfe_upperbound}, the CFE estimator with \(S = L\) satisfies adaptive QV with the constants
  \begin{align*}
    \alpha_{\mathrm{CFE}} 
    = L^2 \left( d + k_{\varphi} + 4 \right) \left(1 + \delta\right)\qquad\text{and}\qquad
    %
    \beta_{\mathrm{CFE}}  
    = L^2 \left( d + k_{\varphi} \right) \left( 1 + \delta^{-1}  \right) {\lVert \bar{\vlambda} - \vlambda^* \rVert}_2^2.
  \end{align*}
  Furthermore, for a \(\mu\)-strongly log-concave posterior and our variational parameterization,~\citet[Theorem 9]{domke_provable_2020} show that the ELBO is \(\mu\)-strongly convex.

  We thus invoke \cref{thm:projsgd_stronglyconvex_decstepsize_adaptive_complexity} with 
  \begin{alignat*}{3}
    \widetilde{\alpha} = L^2 \left(d + k_{\varphi} + 4\right),
    \qquad\quad
    \widetilde{\beta}  = L^2 \left(d + k_{\varphi}\right) {\lVert \bar{\vlambda} - \vlambda^* \rVert}_2^2, \qquad\text{and}
    \qquad
    C = 1.
  \end{alignat*}
  This yields a lower bound on the number of iterations:
  \begin{align*}
    &\frac{16 \widetilde{\beta} }{ \mu^2 } \frac{1}{\epsilon} 
    +
    \frac{16 \sqrt{2}}{ \mu^2 }
    \sqrt{
      \norm{\vlambda_0 - \vlambda^*}_2
    }
    \,
    \sqrt{
      \widetilde{\alpha} \widetilde{\beta}
    }
    \,
    \frac{1}{\epsilon^{3/4}}
    +
    \frac{8 \widetilde{\alpha} \, \norm{\vlambda_0 - \vlambda^*}_2 }{ \mu^2}
    \frac{1}{\sqrt{\epsilon}}
    \\
    &=
    \frac{16 L^2 \left(d + k_{\varphi}\right) {\lVert \bar{\vlambda} - \vlambda^* \rVert}_2^2 }{ \mu^2 } \frac{1}{\epsilon} 
    +
    \frac{16 \sqrt{2}}{ \mu^2 }
    \sqrt{
      \norm{\vlambda_0 - \vlambda^*}_2
    }
    \,
    \sqrt{
       \left( L^2 \left(d + k_{\varphi} + 4\right) \right)
       \left( L^2 \left(d + k_{\varphi}\right) {\lVert \bar{\vlambda} - \vlambda^* \rVert}_2^2 \right)
    }
    \,
    \frac{1}{\epsilon^{3/4}}
    \\
    &\qquad+
    \frac{8 L^2 \left(d + k_{\varphi} + 4\right) \, \norm{\vlambda_0 - \vlambda^*}_2 }{ \mu^2}
    \frac{1}{\sqrt{\epsilon}},
\shortintertext{using the trivial bound \(d + k_{\varphi} < d + k_{\varphi} + 4\),}
    &<
    \frac{16 L^2 \left(d + k_{\varphi} + 4\right) {\lVert \bar{\vlambda} - \vlambda^* \rVert}_2^2 }{ \mu^2 } \frac{1}{\epsilon} 
    +
    \frac{16 \sqrt{2}}{ \mu^2 }
    \sqrt{
      \norm{\vlambda_0 - \vlambda^*}_2
    }
    \,
    \sqrt{
       L^4 \left(d + k_{\varphi} + 4\right)
       \left(d + k_{\varphi} + 4\right) {\lVert \bar{\vlambda} - \vlambda^* \rVert}_2^2 
    }
    \,
    \frac{1}{\epsilon^{3/4}}
    \\
    &\qquad+
    \frac{8 L^2 \left(d + k_{\varphi} + 4\right) \, \norm{\vlambda_0 - \vlambda^*}_2 }{ \mu^2}
    \frac{1}{\sqrt{\epsilon}},
\shortintertext{pulling out the \(16 \left( d + k_{\varphi} + 4 \right) L^2 / \mu^2 \) factors,}
    &=
    16 \left(d + k_{\varphi} + 4\right) 
    \frac{L^2}{ \mu^2 } 
    \bigg(
    {\lVert \bar{\vlambda} - \vlambda^* \rVert}_2^2 \frac{1}{\epsilon} 
    +
    \sqrt{2}
    \sqrt{
      \norm{\vlambda_0 - \vlambda^*}_2
    }
    \,
    {\lVert \bar{\vlambda} - \vlambda^* \rVert}_2
    \,
    \frac{1}{\epsilon^{3/4}}
    +
    \frac{1}{2}
    \norm{\vlambda_0 - \vlambda^*}_2 
    \frac{1}{\sqrt{\epsilon}}
    \bigg)
    \\
    &=
    16 \kappa^2 \left(d + k_{\varphi} + 4\right) 
    \bigg(
    {\lVert \bar{\vlambda} - \vlambda^* \rVert}_2^2 \frac{1}{\epsilon} 
    +
    \sqrt{2}
    \sqrt{
      \norm{\vlambda_0 - \vlambda^*}_2
    }
    \,
    {\lVert \bar{\vlambda} - \vlambda^* \rVert}_2
    \,
    \frac{1}{\epsilon^{3/4}}
    +
    \frac{1}{2}
    \norm{\vlambda_0 - \vlambda^*}_2 
    \frac{1}{\sqrt{\epsilon}}
    \bigg).
  \end{align*}
  The optimal \(\delta\) is given as
  \begin{align*}
    \delta
    &=
    \frac{
      \sqrt{ \norm{\vlambda_0 - \vlambda^*}_2 }
      \,
      \epsilon^{1/4}
      \,
      \sqrt{\widetilde{\alpha}}
    }{
      \sqrt{2} \, C
      \sqrt{\widetilde{\beta}}
    }
    =
    \frac{
      \sqrt{ \norm{\vlambda_0 - \vlambda^*}_2 }
      \,
      \epsilon^{1/4}
      \,
      \sqrt{ L^2 \left(d + k_{\varphi} + 4\right) }
    }{
      \sqrt{2} 
      \sqrt{ L^2 \left(d + k_{\varphi}\right) {\lVert \bar{\vlambda} - \vlambda^* \rVert}_2^2 }
    }
    =
    \frac{1}{\sqrt{2} } \,
    \frac{\sqrt{ \norm{\vlambda_0 - \vlambda^*}_2 }}{{\lVert \bar{\vlambda} - \vlambda^* \rVert}_2}
    \,
    \sqrt{
    \frac{
       d + k_{\varphi} + 4
    }{
      d + k_{\varphi}
    }
    } \, \epsilon^{-1/4}.
  \end{align*}
\end{proofEnd}


And in particular, the following theorem establishes that BBVI with the STL estimator can achieve linear convergence under perfect variational family specification.

\input{thm_projsgd_bbvistl_stronglyconvex_complexity}

\begin{corollary}[\textbf{Linear Convergence of BBVI with STL}]
  If the variational family is perfectly specified such that \( \mathrm{D}_{\mathrm{F}^4}\left(q_{\vlambda}^*, \pi\right) = 0\) for \(\vlambda^* = \argmin_{\vlambda \in \Lambda_L} F\left(\vlambda\right)\), then BBVI with the STL estimator converges linearly with a complexity of \(\mathcal{O}\left(d \kappa^2 \log \left( 1 / \epsilon \right) \right)\).
\end{corollary}

% \begin{remark}
%   The overall complexity of the STL estimator is larger by a constant factor of 4 compared to that of CFE.
%   This means that the STL estimator will converge slowly for a large target \(\epsilon\).
%   As mentioned in \cref{remark:stl_tightness,remark:variance_comparison}, a factor of 2 difference would be more realistic and is closer to what is observed in our simulations.
% \end{remark}

\begin{remark}
  Convergence is slowed when using a decreasing step size schedule, as shown in \cref{thm:projsgd_bbvistl_decstepsize_complexity}.
  Thus, one does not achieve a linear convergence rate under this schedule even if the variational family is perfectly specified. 
  However, when the variational family is misspecified, this achieves a better rate of \(\mathcal{O}\left(1/\epsilon\right)\) compared to the \(\mathcal{O}\left(1/\epsilon \log 1/\epsilon\right)\) of \cref{thm:projsgd_bbvistl_complexity}.
\end{remark}

\begin{remark}[\textbf{Variational Family Misspecification}]\label{remark:misspecification}
  Under variational family misspecification, STL has an \(\mathcal{O}\left(1/\epsilon\right)\) dependence on the 4th order Fisher divergence \(\mathrm{D}_{\mathrm{F}^4}\left(q_{\vlambda^*}, \pi\right) > 0\).
  To compare the computational performance of CFE and STL in this setting, one needs to compare \( L^{-2} \sqrt{\mathrm{D}_{\mathrm{F}^4}\left(q_{\vlambda^*}, \pi\right)}\) versus \({\lVert \bar{\vlambda} - \vlambda^* \rVert}_2^2\).
\end{remark}

\begin{remark}
  \cref{thm:stl_upperbound_mf} also implies that the mean-field parameterization improves the dimension dependence to a complexity of \(\mathcal{O}\left( \sqrt{d} \kappa^2 \log \left( 1 / \epsilon \right) \right)\).
\end{remark}

%% \paragraph{Complexity of BBVI on Log-Concave Posteriors}
%% Lastly, we present the complexity of BBVI with the STL estimator when \(\pi\) is only log-concave.
%% Since the complexity guarantee of \cref{thm:projsgd_convex_fixedstepsize} is too nonlinear with respect to \(\beta\), we present the case where the variational family is well specified (\(\beta_{\mathrm{STL}} = 0\)).
%% In this setting, the STL estimator achieves a better complexity than that of the CFE estimator.

%% 
\begin{theoremEnd}[category=complexitybbvistl]{theorem}[\textbf{Complexity of Fixed Stepsize BBVI with STL}]\label{thm:projsgd_bbvistl_decstepsize_complexity_logconcave}
  For some weighted iterate averaging scheme, the last average \(\bar{\vlambda}_{T}\) of BBVI with the STL estimator and projected SGD with a fixed stepsize schedule applied to a log-concave and \(L\)-log-smooth posterior under perfect variational family specification satisfy \(F\left(\bar{\vlambda}_T\right) - F\left(\vlambda^*\right) \leq \epsilon\), where \(\vlambda^* \in \argmin_{\vlambda \in \Lambda_L} F\left(\vlambda\right) \) if
  \begin{align*}
    T
    \geq
    \sqrt{2 \left(d + k_{\varphi}\right)} \, L 
    \, \norm{\vlambda_0 - \vlambda^*}_2^2
    \frac{1}{\epsilon},
  \end{align*}
  for some fixed stepsize \(\gamma\).
\end{theoremEnd}
\begin{proofEnd}
  As shown by \cref{thm:stl_upperbound}, the STL estimator satisfies an adative QV bound with the constants
  \begin{align*}
    \alpha_{\mathrm{STL}} &= 2 \left(d + k_{\varphi}\right) \left(2 + \delta\right)  L^2 \\
    \beta_{\mathrm{STL}}  &= \sqrt{3} \left(d + k_{\varphi}\right) \left(1 + 2 \delta^{-1}\right)  \sqrt{\mathrm{D}_{\mathrm{F}^4}\left(q_{\vlambda^*}, \pi\right)}.
  \end{align*}
  Also, when the variational family is perfectly specified, the variational posterior \(q_{\vlambda^*}\) is equal to \(\pi\) such that 
  \[
  \mathrm{D}_{\mathrm{F}^4}\left(q_{\vlambda^*}, \pi\right) = 0.
  \]
  Then, we immediately have \(\beta_{\text{STL}} = 0\).
  Therefore, the optimal \(\delta\) is \(\delta = 0\), resulting in 
  \begin{align*}
    \alpha_{\mathrm{STL}} &= 4 \left(d + k_{\varphi}\right) L^2 \\
    \beta_{\mathrm{STL}}  &= 0.
  \end{align*}
  Furthermore, for a log-concave posterior and our variational parameterization,~\citet[Theorem 9]{domke_provable_2020} show that the ELBO is convex.

  Thus, we can invoke \cref{thm:projsgd_convex_fixedstepsize}, for which we have a lower bound on the required number of iteration:
  \begin{align*}
    \frac{2 \alpha}{ -\beta + \sqrt{ \beta^2 + 8 \epsilon^2 \alpha } }
    \, \norm{\vlambda_0 - \vlambda^*}_2^2
    &=
    \frac{2 \alpha}{ \sqrt{ 8 \epsilon^2 \alpha } }
    \, \norm{\vlambda_0 - \vlambda^*}_2^2
    \\
    &=
    \frac{\sqrt{\alpha}}{\sqrt{2}}
    \, \norm{\vlambda_0 - \vlambda^*}_2^2
    \frac{1}{\epsilon}
    \\
    &=
    \frac{\sqrt{4 \left(d + k_{\varphi}\right) L^2}}{\sqrt{2}}
    \, \norm{\vlambda_0 - \vlambda^*}_2^2
    \frac{1}{\epsilon}
    \\
    &=
    \sqrt{2 \left(d + k_{\varphi}\right)} \, L 
    \, \norm{\vlambda_0 - \vlambda^*}_2^2
    \frac{1}{\epsilon}.
  \end{align*}
\end{proofEnd}


%% \subsection{Variational Family Misspecification and the STL Estimator}

%% As mentioned in \cref{remark:misspecification}, comparing the performance of the STL estimator against the CFE estimator involves the 4th-order Fisher divergence and \({\lVert \bar{\vlambda} - \vlambda^* \rVert}_2^2\).
%% Unfortunately, the relationship between the two quantities is not obvious.
%% Because of this, we further reduce our scope to Gaussians in order to gather more intuition.

%% Recall that term \(T_{\ding{184}}\) in \cref{thm:stl_decomposition} captures the effect of variational family specification on the STL estimator.
%% Assuming both the posterior and variational approximation are Gaussians, we can obtain a simpler result on \(T_{\ding{184}}\).

\vspace{-1ex}
\subsection{Should we stick the landing?}
\vspace{-1ex}
When the variational family is misspecified, it is hard to tell \textit{when} STL would be superior to CFE; the Fisher-Hyv\"arinen divergence and the posterior variance are fundamentally unrelated quantities.
Furthermore, the Fisher-Hyv\"arinen divergence is hard to interpret apart from some relationships with other divergences~\citep{huggins_practical_2018}.
Thus, we lastly provide some characterization of the Fisher-Hyv\"arinen divergence.

Our final analysis will focus on Gaussian posteriors and the mean-field Gaussian family.
In practice, the STL estimator becomes infeasible to use with full-rank variational families as each evaluation of the log-density \(\log q_{\vlambda}\) involves a back-substitution with a \(\mathcal{O}\left(d^3\right)\) cost and numerical stability becomes a concern.
Therefore, studying the effect of misspecification due to the mean-field approximation is particularly relevant.


\begin{theoremEnd}[category=stlgaussian]{lemma}\label{thm:gaussian_fisher_divergence}
  For \(\pi = \mathcal{N}\left(\vmu, \mSigma\right)\) and \(q = \mathcal{N}\left(\vm, \mC \mC^{\top} \right)\), the Fisher-Hyv\"arinen divergence is
  \[
    \DHF{q}{\pi}
    =
    {\lVert \mSigma^{-1} \mC - \mC^{-\top} \rVert}_{\mathrm{F}}^2
    +
    {\lVert \mSigma^{-1}\left(\vm  - \vmu\right) \rVert}_{2}^2.
  \]
\end{theoremEnd}
\begin{proofEnd}
  \begin{align*}
    \DHF{q}{\pi}
    &=
    \mathbb{E}_{\rvvz \sim q} \norm{ \nabla \log \pi\left(\rvvz\right) - \nabla \log q\left(\rvvz\right) }_{2}^2
    \\
    &=
    \mathbb{E} \norm{ \mSigma^{-1} \left(\mC \rvvu + \vm  - \vmu\right) - {\left( \mC\mC^{\top} \right)}^{-1} \left(\mC \rvvu + \vm - \vm\right) }_{2}^2
    \\
    &=
    \mathbb{E} \norm{ \mSigma^{-1} \left(\mC \rvvu + \vm  - \vmu\right) - {\left( \mC\mC^{\top} \right)}^{-1} \mC \rvvu }_{2}^2
    \\
    &=
    \mathbb{E} \norm{ \mSigma^{-1} \left(\mC \rvvu + \vm  - \vmu\right) - \mC^{-\top} \rvvu }_{2}^2
    \\
    &=
    \mathbb{E} \norm{ \left( \mSigma^{-1} \mC - \mC^{-\top} \right) \rvvu  + \mSigma^{-1}\left(\vm  - \vmu\right) }_{2}^2
    \\
    &=
    \mathbb{E} \norm{ \left( \mSigma^{-1} \mC - \mC^{-\top} \right) \rvvu }_2^2
    +
    2 \inner{ \left( \mSigma^{-1} \mC - \mC^{-\top} \right) \mathbb{E} \rvvu }{ \mSigma^{-1}\left(\vm  - \vmu\right) }
    +
    {\lVert \mSigma^{-1}\left(\vm  - \vmu\right) \rVert}_{2}^2
    \\
    &=
    \mathbb{E} \norm{ \left( \mSigma^{-1} \mC - \mC^{-\top} \right) \rvvu }_2^2
    +
    {\lVert \mSigma^{-1}\left(\vm  - \vmu\right) \rVert}_{2}^2
  \end{align*}

  Finally,
  \begin{align*}
    \mathbb{E} \norm{ \left( \mSigma^{-1} \mC - \mC^{-\top} \right) \rvvu }_2^2
    &=
    \mathbb{E} \mathrm{tr} \left( \rvvu^{\top} {\left( \mSigma^{-1} \mC - \mC^{-\top} \right)}^{\top} \left( \mSigma^{-1} \mC - \mC^{-\top} \right) \rvvu \right)
    \\
    &=
    \mathrm{tr} \left( {\left( \mSigma^{-1} \mC - \mC^{-\top} \right)}^{\top} \left( \mSigma^{-1} \mC - \mC^{-\top} \right) \mathbb{E} \rvvu \rvvu^{\top} \right)
    \\
    &=
    \mathrm{tr} \left( {\left( \mSigma^{-1} \mC - \mC^{-\top} \right)}^{\top} \left( \mSigma^{-1} \mC - \mC^{-\top} \right) \right)
    \\
    &=
    \norm{ \mSigma^{-1} \mC - \mC^{-\top} }_{\mathrm{F}}^2.
  \end{align*}
\end{proofEnd}

\begin{theoremEnd}[category=stlgaussian]{lemma}
  For \(\pi = \mathcal{N}\left(\vmu, \mSigma\right)\) and \(q = \mathcal{N}\left(\vm, \mC \mC^{\top} \right)\), where \(\vmu = \vm\), the Fisher-Hyv\"arinen divergence is bounded above and below as
  \[
    {\lambda_{\mathrm{max}}\left(\mC\right)}^{-2} {\lVert \mC^{\top} \mSigma^{-1} \mC - \boldupright{I} \rVert}_{\mathrm{F}}^2
    \leq
    \DHF{q}{\pi}
    \leq
    {\lambda_{\mathrm{min}}\left(\mC\right)}^{-2} {\lVert \mC^{\top} \mSigma^{-1} \mC - \boldupright{I} \rVert}_{\mathrm{F}}^2.
  \]
\end{theoremEnd}
\begin{proofEnd}
  By assumption,
  \begin{align}
    \DHF{q}{\pi}
    =
    {\lVert \mSigma^{-1} \mC - \mC^{-\top} \rVert}_{\mathrm{F}}^2
    +
    {\lVert \mSigma^{-1}\left(\vm  - \vmu\right) \rVert}_{2}^2
    =
    {\lVert \mSigma^{-1} \mC - \mC^{-\top} \rVert}_{\mathrm{F}}^2.
    \label{eq:stl_gaussian_sandwich_1}
  \end{align}

  Also, we can pull our a \(\mC^{-\top}\) factor as
  \begin{align}
    {\lVert \mSigma^{-1} \mC - \mC^{-\top} \rVert}_{\mathrm{F}}^2
    =
    {\lVert \mC^{-\top} \left( \mC^{\top} \mSigma^{-1} \mC - \boldupright{I} \right) \rVert}_{\mathrm{F}}^2.
    \label{eq:stl_gaussian_sandwich_2}
  \end{align}

  By the property of the Frobenius norm,
  \begin{alignat*}{4}
    & &\quad
    {\lambda_{\mathrm{min}}\left(\mC^{-\top}\right)}^2 {\lVert \mC^{\top} \mSigma^{-1} \mC - \boldupright{I} \rVert}_{\mathrm{F}}^2
    &\leq
    \, {\lVert \mC^{-\top} \left( \mC^{\top} \mSigma^{-1} \mC - \boldupright{I} \right) \rVert}_{\mathrm{F}}^2 \;
    &\leq
    {\lambda_{\mathrm{max}}\left(\mC^{-\top}\right)}^2 {\lVert \mC^{\top} \mSigma^{-1} \mC - \boldupright{I} \rVert}_{\mathrm{F}}^2,
\shortintertext{inverting the singular values,}
    &\Leftrightarrow&\quad
    {\lambda_{\mathrm{max}}\left(\mC\right)}^{-2} {\lVert \mC^{\top} \mSigma^{-1} \mC - \boldupright{I} \rVert}_{\mathrm{F}}^2
    &\leq
    \,{\lVert \mC^{-\top} \left( \mC^{\top} \mSigma^{-1} \mC - \boldupright{I} \right) \rVert}_{\mathrm{F}}^2 \,
    &\leq
    {\lambda_{\mathrm{min}}\left(\mC\right)}^{-2} {\lVert \mC^{\top} \mSigma^{-1} \mC - \boldupright{I} \rVert}_{\mathrm{F}}^2,
\shortintertext{and by \cref{eq:stl_gaussian_sandwich_1,eq:stl_gaussian_sandwich_2},}
    &\Leftrightarrow&\quad
    {\lambda_{\mathrm{max}}\left(\mC\right)}^{-2} {\lVert \mC^{\top} \mSigma^{-1} \mC - \boldupright{I} \rVert}_{\mathrm{F}}^2
    &\leq
    \,\DHF{q}{\pi}\,
    &\leq
    {\lambda_{\mathrm{min}}\left(\mC\right)}^{-2} {\lVert \mC^{\top} \mSigma^{-1} \mC - \boldupright{I} \rVert}_{\mathrm{F}}^2.
  \end{alignat*}

\end{proofEnd}

\begin{theoremEnd}[category=stlgaussian]{proposition}
  For a Gaussian posterior \(\pi = \mathcal{N}\left(\vmu, \mSigma\right)\) and Gaussian variational family such that the base distribution \(\varphi = \mathcal{N}\left(0, 1\right)\), the gradient variance of the STL estimator is bounded as
  \begin{align*}
    \DHF{q_{\vlambda^*}}{\pi} &\leq \mathbb{E} \norm{\rvvg_{\mathrm{STL}}\left(\vlambda\right)}_2^2 \leq C_1(d) \, L^2 \, \norm{\vlambda - \vlambda^*}_2^2 + C_2(d) \, \DHF{q_{\vlambda^*}}{\pi},
  \end{align*}
  for all \(\vlambda \in \Lambda_L\), where \(\vlambda^* \in \argmin_{\vlambda \in \Lambda_L} F\left(\vlambda\right)\),

  %\vspace{-2ex}
  {\begingroup
  %  \setlength\tabcolsep{1.0ex} 
  \begin{tabular}{lll}
    \(C_1(d) = 6 d + 12\) & \(C_2(d) = 3 d + 9\) & for the full-rank and \\
    \(C_1(d) = 24 \sqrt{d} + 6\), & \(C_2(d) = 9 \sqrt{d}\) & for the mean-field parameterizations.
  \end{tabular}
  \endgroup}
\end{theoremEnd}
\begin{proofEnd}
  If the posterior is Gaussian, we have
  \[
    \nabla \log \pi\left(\vz\right)
    =
    {\mSigma}^{-1} \left( \vz - \vmu \right).
  \]
  Also, it is straightforward to check that the variational posterior mean is \(\vm = \vmu\).

  The lower bound is a direct application of \cref{thm:stl_lowerbound}.

  For the upper bound, we can partially reuse the results of  \cref{thm:stl_upperbound,thm:stl_upperbound_mf} such that, for full-rank parameterization:
  \begin{align}
    T_{\text{\ding{182}}} &= L^2 \left( d + k_{\varphi} \right) \norm{\vlambda - \vlambda^*}_2^2 \label{eq:stl_gaussian_fr_t182} \\
    T_{\text{\ding{183}}} &= S^2 \left( d + 1         \right) \norm{\vlambda - \vlambda^*}_2^2 \label{eq:stl_gaussian_fr_t183}
  \end{align}
  and for the mean-field parameterization:
  \begin{align}
    T_{\text{\ding{182}}} &= L^2 \left( 2 k_{\varphi} \sqrt{d} + 1 \right) \norm{\vlambda - \vlambda^*}_2^2 \label{eq:stl_gaussian_mf_t182} \\
    T_{\text{\ding{183}}} &= S^2 \left( \sqrt{d k_{\varphi}} + 1 \right) \norm{\vlambda - \vlambda^*}_2^2. \label{eq:stl_gaussian_mf_t183}
  \end{align}

  We focus on obtaining a tighter and more interpretable result on \(T_{\text{\ding{184}}}\) given as
  \[
    T_{\text{\ding{184}}} =  
    \mathbb{E} J_{\mathcal{T}}\left(\rvvu\right)
    \norm{ 
      \nabla \log \pi\left(\mathcal{T}_{\vlambda^*}\left(\rvvu\right)\right) 
      - 
      \nabla \log q_{\vlambda^*}\left(\mathcal{T}_{\vlambda^*}\left(\rvvu\right)\right) 
    }_2^2.
  \]
  Excluding the Jacobian term \(J_{\mathcal{T}}\) for now, we have
  \begin{align*}
    &\norm{ 
      \nabla \log \pi\left(\mathcal{T}_{\vlambda^*}\left(\vu\right)\right) 
      - 
      \nabla \log q_{\vlambda^*}\left(\mathcal{T}_{\vlambda^*}\left(\vu\right)\right) 
    }_2^2
    \\
    &\;=
    {\lVert} 
      \mSigma^{-1} \left( \mC^* \rvvu + \vm^* - \vmu \right)
      - 
      {\left( \mC^* \right)}^{-\top} {\left(\mC^*\right)}^{-1} \left( \mC^* \rvvu + \vm^* - \vm^* \right)
    {\rVert}_2^2 
    \\
    &\;=
    {\lVert} 
      \mSigma^{-1} \mC^* \rvvu 
      - 
      {\left( \mC^* \right)}^{-\top} \rvvu 
    {\rVert}_2^2 
    \\
    &\;=
    \norm{
      \left( \mSigma^{-1} \mC^* - {\left( \mC^* \right)}^{-\top} \right) \rvvu 
    }_2^2 
  \end{align*}

  Thus, bringing the Jacobian term in, we have
  \begin{align*}
    \mathbb{E} \,
    J_{\mathcal{T}}\left(\rvvu\right)
    \norm{ 
      \nabla \log \pi\left(\mathcal{T}_{\vlambda^*}\left(\rvvu\right)\right) 
      - 
      \nabla \log q_{\vlambda^*}\left(\mathcal{T}_{\vlambda^*}\left(\rvvu\right)\right) 
    }_2^2
    &\leq
    L \,
    \mathbb{E} \,
    J_{\mathcal{T}}\left(\rvvu\right)
    \norm{
      \left( \mSigma^{-1} \mC^* - {\left( \mC^* \right)}^{-\top} \right) \rvvu
    }_2^2.
  \end{align*}
  It now remains to deal with the stochastic terms.

  \paragraph{Full-Rank}
  For the full-rank parameterization,
  \begin{align}
    &
    \mathbb{E} \,
    J_{\mathcal{T}}\left(\rvvu\right)
    \norm{ 
      \nabla \log \pi\left(\mathcal{T}_{\vlambda^*}\left(\vu\right)\right) 
      - 
      \nabla \log q_{\vlambda^*}\left(\mathcal{T}_{\vlambda^*}\left(\vu\right)\right) 
    }_2^2
    \nonumber
    \\
    &\;=
    L \,
    \mathbb{E} 
    \left(1 + \norm{\rvvu}_2^2\right)
    \norm{
      \left( \mSigma^{-1} \mC^* - {\left( \mC^* \right)}^{-\top} \right) \rvvu
    }_2^2
    \nonumber
    \\
    &\;=
    L \,
    \mathbb{E} \,
    \left(1 + \norm{\rvvu}_2^2\right)
    \mathrm{tr}
    \left(
      \rvvu^{\top} 
      {\left(
        \mSigma^{-1} \mC^* - {\left( \mC^* \right)}^{-\top}
      \right)}^{\top}
      \left(
        \mSigma^{-1} \mC^* - {\left( \mC^* \right)}^{-\top}
      \right)
      \rvvu 
    \right)
    \nonumber
    \\
    &\;=
    \mathrm{tr}
    \left(
      {\left(
        \mSigma^{-1} \mC^* - {\left( \mC^* \right)}^{-\top}
      \right)}^{\top}
      \left(
        \mSigma^{-1} \mC^* - {\left( \mC^* \right)}^{-\top}
      \right)
      \mathbb{E}
      \left(1 + \norm{\rvvu}_2^2\right)
      \rvvu 
      \rvvu^{\top} 
    \right)
    \nonumber
    \\
    &\;=
    \mathrm{tr}
    \left(
      {\left(
        \mSigma^{-1} \mC^* - {\left( \mC^* \right)}^{-\top}
      \right)}^{\top}
      \left(
        \mSigma^{-1} \mC^* - {\left( \mC^* \right)}^{-\top}
      \right)
      \left( d + k_{\varphi} \right)
      \boldupright{I}
    \right)
    \nonumber
    \\
    &\;=
    \left( d + k_{\varphi} \right)
    \norm{
      \mSigma^{-1} \mC^* - {\left( \mC^* \right)}^{-\top}
    }_{\mathrm{F}}^2
\shortintertext{and by \cref{thm:gaussian_fisher_divergence},}
    &\;=
    \left( d + k_{\varphi} \right) \DHF{q_{\vlambda^*}}{\pi}.
    \label{eq:stl_gaussian_fr}
  \end{align}

  \paragraph{Mean-Field}
  Finally, for the mean-field parameterization, 
  \begin{align}
    &\mathbb{E} \,
    J_{\mathcal{T}}\left(\rvvu\right)
    \norm{ 
      \nabla \log \pi\left(\mathcal{T}_{\vlambda^*}\left(\rvvu\right)\right) 
      - 
      \nabla \log q_{\vlambda^*}\left(\mathcal{T}_{\vlambda^*}\left(\rvvu\right)\right) 
    }_2^2
    \nonumber
    \\
    &\;\leq
    \mathbb{E} \,
    J_{\mathcal{T}}\left(\rvvu\right)
    \norm{
      \left(
        \mSigma^{-1} \mC^* - {\left( \mC^* \right)}^{-\top}
      \right)
      \rvvu 
    }_2^2
    \nonumber
    \\
    &\;=
    \mathbb{E} \,
    \left(1 + {\lVert \rvmU^2 \rVert}_{\mathrm{F}} \right)
    \norm{
      \left(
        \mSigma^{-1} \mC^* - {\left( \mC^* \right)}^{-\top}
      \right)
      \rvvu 
    }_2^2,
    \nonumber
    \\
    &\;=
    k_{\varphi} \sqrt{d} \,
    \norm{
      \mSigma^{-1} \mC^* - {\left( \mC^* \right)}^{-\top}
    }_{\mathrm{F}}^2
\shortintertext{and by \cref{thm:gaussian_fisher_divergence},}
    &\;=
    k_{\varphi} \sqrt{d} \, \DHF{q_{\vlambda^*}}{\pi}
    \label{eq:stl_gaussian_mf}
  \end{align}

  Combining \cref{eq:stl_gaussian_fr,eq:stl_gaussian_mf} with \cref{eq:stl_gaussian_mf_t182,eq:stl_gaussian_mf_t183,eq:stl_gaussian_fr_t182,eq:stl_gaussian_fr_t183} and \cref{thm:stl_decomposition} yields the result.
  Note that, for a Gaussian \(\varphi\), we have \(k_{\varphi} = 3\), and we will set \(S = L\) and \(\delta = 1\).
  For the full-rank parameterization, 
  \begin{align*}
    \mathbb{E} \norm{\rvvg_{\mathrm{STL}}\left(\vlambda\right)}_2^2
    &\leq
    \left( 2 + \delta \right) T_{\text{\ding{182}}}
    +
    \left( 2 + \delta \right) T_{\text{\ding{183}}}
    +
    \left( 1 + 2 \delta^{-1} \right) T_{\text{\ding{184}}}
    \\
    &\leq
    \left( 2 + \delta \right) L^2 \left( d + k_{\varphi} \right) \norm{\vlambda - \vlambda^*}_2^2
    \\
    &\quad+
    \left( 2 + \delta \right) S^2 \left( d + 1         \right) \norm{\vlambda - \vlambda^*}_2^2
    \\
    &\quad+
    \left( 1 + 2 \delta^{-1} \right) \left( d + k_{\varphi} \right) \DHF{q_{\vlambda^*}}{\pi},
\shortintertext{substituting the constants,}
    &=
    3 L^2 \left( d + 3 \right) \norm{\vlambda - \vlambda^*}_2^2
    +
    3 L^2 \left( d + 1 \right) \norm{\vlambda - \vlambda^*}_2^2
    +
    3 \left( d + 3 \right) \DHF{q_{\vlambda^*}}{\pi}
    \\
    &=
    3 L^2 \left( 2 d + 4 \right) \norm{\vlambda - \vlambda^*}_2^2
    +
    3 \left( d + 3 \right) \DHF{q_{\vlambda^*}}{\pi}.
    \\
    &=
    6 L^2 \left( d + 2 \right) \norm{\vlambda - \vlambda^*}_2^2
    +
    3 \left( d + 3 \right) \DHF{q_{\vlambda^*}}{\pi}
  \end{align*}
  And for the mean-field parameterization, 
  \begin{align*}
    \mathbb{E} \norm{\rvvg_{\mathrm{STL}}\left(\vlambda\right)}_2^2
    &\leq
    \left( 2 + \delta \right) T_{\text{\ding{182}}}
    +
    \left( 2 + \delta \right) T_{\text{\ding{183}}}
    +
    \left( 1 + 2 \delta^{-1} \right) T_{\text{\ding{184}}}
    \\
    &\leq
    \left( 2 + \delta \right) L^2 \left( 2 k_{\varphi} \sqrt{d} + 1 \right) \norm{\vlambda - \vlambda^*}_2^2
    \\
    &\quad+
    \left( 2 + \delta \right) S^2 \left( \sqrt{d k_{\varphi}} + 1 \right) \norm{\vlambda - \vlambda^*}_2^2
    \\
    &\quad+
    \left( 1 + 2 \delta^{-1} \right) k_{\varphi} \sqrt{d} \DHF{q_{\vlambda^*}}{\pi},
\shortintertext{substituting the constants,}
    &=
    3 L^2 \left( 6 \sqrt{d} + 1 \right) \norm{\vlambda - \vlambda^*}_2^2
    +
    3 L^2 \left( \sqrt{3 d} + 1 \right) \norm{\vlambda - \vlambda^*}_2^2
    +
    9 \sqrt{d} \, \DHF{q_{\vlambda^*}}{\pi}
    \\
    &=
    3 L^2 \left( 6 \sqrt{d} + \sqrt{3 d} + 2 \right) \norm{\vlambda - \vlambda^*}_2^2
    +
    9 \sqrt{d} \, \DHF{q_{\vlambda^*}}{\pi},
\shortintertext{applying \(\sqrt{3} \leq 2\),}
    &\leq
    3 L^2 \left( 6  \sqrt{d} + 2 \sqrt{d} + 2 \right) \norm{\vlambda - \vlambda^*}_2^2
    +
    9 \sqrt{d} \, \DHF{q_{\vlambda^*}}{\pi}
    \\
    &=
    6 L^2 \left( 4 \sqrt{d} + 1 \right) \norm{\vlambda - \vlambda^*}_2^2
    +
    9 \sqrt{d} \, \DHF{q_{\vlambda^*}}{\pi}.
  \end{align*}
\end{proofEnd}


\begin{remark}
For Gaussians, the 4th-order Fisher-Hyv\"arinen divergence term in \cref{thm:stl_upperbound} can be replaced by its 2nd-order counterpart.
Thus, combined with \cref{thm:stl_lowerbound}, the 2nd-order Fisher-Hyv\"arinen divergence fully characterizes the variance of STL.
\end{remark}

\begin{remark}
  \cref{thm:fisher_bound} implies that, when approximating a full-rank Gaussian with a mean-field Gaussian, the value of the Fisher-Hyv\"arinen divergence is tightly characterized by the degree of correlation in the posterior; it will increase indefinitely as the posterior correlation matrix becomes singular.
\end{remark}

\begin{remark}
  We have provided a sufficient condition for the STL estimator to perform poorly compared to the CFE estimator. 
  It is foreseeable that alternative types of model misspecification abundant in practice should yield additional sufficient conditions, \textit{i.e.}, tail mismatch, but we leave this to future works.
\end{remark}


%
\section{Experiments}

% Figure environment removed


%%% Local Variables:
%%% TeX-master: "main"
%%% End:


\section{Discussion}

% \paragraph{QV and Other Conditions}
% We would like to mention some connections of the QV condition with other known gradient variance conditions.
% Notably, when the Polyak-\L{}ojasiewicz (PL) condition holds, QV immediately implies the convex expected smoothness (CES; \citealp{gower_sgd_2019}) and relaxed growth (RG; \citealp{bottou_optimization_2018}) conditions. 
% (See \cref{section:definitions} for definitions and also the overview by~\citealt{khaled_better_2023}.)
% Furthermore, when ``interpolation'' holds, we obtain the stronger expected strong growth (ESG;~\citealp{solodov_incremental_1998, vaswani_fast_2019}) condition.

%
%% \begin{theoremEnd}{proposition}[category=gradientconditions]
%%     Let a gradient estimator \(g\) satisfy the QV condition.
%%     Then, for a \(\mu\)-PL objective \(F\), it also satisfies CES and RG as
%%     \[
%%       \mathbb{E}\norm{\rvvg}_2^2 
%%       \leq
%%       \frac{2 \alpha }{\mu} \left( F\left(\vlambda\right) - F\left(\vlambda^*\right) \right) + \beta
%%       \leq 
%%       \frac{\alpha}{\mu^2} \norm{\nabla F}_2^2 + \beta,
%%     \]
%%     where \(F^* = \inf_{\vlambda \in \Lambda} F\left(\vlambda\right)\).
%% \end{theoremEnd}
%% \begin{proofEnd}
%%   First, the \(\mu\)-PL condition is 
%%   \begin{align*}
%%      \mu \left( F\left(\vlambda\right) - F\left(\vlambda^*\right) \right)
%%      \leq
%%      \frac{1}{2} \norm{\nabla F}_2^2.
%%   \end{align*}
%%   As shown by \citet{karimi_linear_2016}, this directly implies the quadratic growth condition~\citep{anitescu_degenerate_2000}
%%   \begin{align*}
%%      \frac{\mu}{2} \norm{\vlambda - \vlambda^*}_2^2
%%      \leq F\left(\vlambda\right) - F^*.
%%   \end{align*}

%%   The proof follows from applying the previous equalities one after another as
%%   \begin{align*}
%%     \mathbb{E}\norm{\rvvg}_2^2
%%     \leq
%%     \alpha \norm{ \vlambda - \vlambda^* }_2^2 + \beta
%%     \leq
%%     \frac{2 \alpha }{\mu} \left( F\left(\vlambda\right) - F^* \right) + \beta
%%     \leq
%%     \frac{\alpha}{\mu^2} \norm{\nabla F\left(\vlambda\right)}_2^2 + \beta.
%%   \end{align*}
%% \end{proofEnd}
%
%
%% \begin{corollary}
%%   If the a gradient estimator \(g\) satisfies the QVC condition with \(\beta = 0\), for a \(\mu\)-PL objective \(F\), it also satisfies the SG condition as
%%   \[
%%     \mathbb{E}\norm{\rvvg}_2^2 \leq \frac{\alpha}{\mu^2} \norm{ \nabla F }_2^2
%%   \]
%% \end{corollary}

% Naturally, we can combine these facts with our results in \cref{section:gradient_variance} about the STL estimator.
% %
% \begin{proposition}
%   For a \(\mu\)-log-PL posterior, 
%   \begin{enumerate*}[label=\textbf{(\roman*)}]
%       \item the STL and CFE gradient estimators satisfy the RSG condition. And,
%       \item if the variational family is perfectly specified, the STL gradient estimator also satisfies the SG condition.
%   \end{enumerate*}
% \end{proposition}
% This implies that other analysis results utilizing the CES, RSG, and SG conditions could be used here.
% For example, \citep{meng_fast_2020} proves the convergence of local quadratic convergence of a stochastic Newton algorithm.
% This would improve the previous analysis of \citep{liu_quasimonte_2021}, which had to \textit{a-priori} assume the regularity of BBVI.
% We leave the analysis of such second-order and quasi-second-order BBVI algorithms~\citep{liu_quasimonte_2021,fan_fast_2015} to future works.

\paragraph{STL Estimator for Proximal SGD}
In this work, we have focused on BBVI with projected SGD, which is known to converge slower than BBVI with proximal SGD \citep{kim_blackbox_2023,domke_provable_2020,domke_provable_2023}.
To apply the STL estimator in proximal SGD, we need a gradient estimator that only targets the energy term, which can be devised as follows:
\begin{definition}[\textbf{STL Estimator for the Energy}]\label{def:stl_prox}
The sticking-the-landing gradient estimator for only estimating the energy is given as
\[
  \rvvg_{\mathrm{STL}}\left(\vlambda\right) 
  \triangleq
  \nabla_{\vlambda} \log \pi \left(\mathcal{T}_{\vlambda}\left(\rvvu\right)\right)
  - 
  \nabla_{\vlambda} \log q_{\vnu}\left(\mathcal{T}_{\vlambda}\left(\rvvu\right)\right) \Big\lvert_{\vnu = \vlambda}
  -
  \nabla_{\vlambda} \mathbb{H}\left(q_{\vlambda}\right).
\]
\end{definition}
%
While \cref{def:stl_prox} can be readily used, there is a catch; the variance of the STL estimator is bounded only if the entropy resulting from \(\mC\) is bounded, such as in \(\Lambda_{S}\).
This means a projection operator needs to be employed, defeating the purpose of using proximal SGD.
Thus, our theoretical analysis has focused entirely on projected SGD.

\paragraph{Empirically Comparing Estimators}
From our analysis and that of \citet{domke_provable_2023}, it is apparent that for a QV gradient estimator, \(\alpha\) and \(\beta\) sufficiently characterize its behavior on strongly log-concave posteriors: \(\alpha\) characterizes the convergence speed, while \(\beta\) determines the complexity with respect to \(\epsilon\).
It is conceivable that \textit{estimating} these quantities in practical settings would provide a principled way to compare and evaluate different estimators.
Previously, the signal-to-noise (SNR) ratio have been popularized by~\citet{rainforth_tighter_2018}, and since been used by, for example, by \cite{geffner_difficulty_2021,fujisawa_multilevel_2021}.
In contrast to the QV coefficients, a \textit{constant} SNR relates with convergence only through the expected strong growth condition~\citep{solodov_incremental_1998,vaswani_fast_2019,schmidt_fast_2013}, which is valid only under the strongest condition (perfect variational family specification, strong log-concavity)
The QV coefficients, \(\alpha\) and \(\beta\), on the other hand, apply to a wider range of settings.
We also note that measuring the ELBO of the last iterates, as done by \citet{agrawal_advances_2020}, only estimates the radius of the stationary region (related to \(\beta\)).
To have the full picture, it would also be useful to measure the convergence \textit{speed} (related to \(\alpha\)).

\paragraph{Conclusions and Future Works}
In this work, we have analyzed the sticking-the-landing (STL) estimator by~\citet{roeder_sticking_2017}.
When the variational family is perfectly specified, our complexity guarantees automatically guarantee a logarithmic complexity, known as linear convergence in the optimization literature.
This demonstrates that it is possible to rigorously show that control variates can accelerate the convergence of BBVI.
It will be interesting to analyze and compare the existing control variates by~\citet{miller_reducing_2017, geffner_approximation_2020, wang_dual_2023} and those that will be revealed in the future.
Note that, for the STL estimator, our convergence guarantees suggest that the convergence \textit{speed} is slower than CFE.
That is, on larger \(\epsilon\)-accuracy levels, STL is slower to converge.
It would be interesting to see if there exists a control variate that is capable of accelerating the overall speed of convergence, not just better complexity.




\subsubsection*{Acknowledgements}
The authors would like to sincerely thank Jisu Oh for thoroughly proofreading the paper and Kaiwen Wu for helpful discussions.

%\clearpage
\bibliographystyle{rss}
\bibliography{references}

\clearpage

\section*{Checklist}

 \begin{enumerate}
 \item For all models and algorithms presented, check if you include:
 \begin{enumerate}
 \item A clear description of the mathematical setting, assumptions, algorithm, and/or model. \\
     \textbf{\textcolor{ForestGreen}{Yes}}. See~\cref{section:preliminaries}.
   \item An analysis of the properties and complexity (time, space, sample size) of any algorithm. \\
     \textbf{\textcolor{ForestGreen}{Yes}}. See~\cref{section:bbvicomplexity}.
   \item (Optional) Anonymized source code, with specification of all dependencies, including external libraries. \\
     \textbf{\textcolor{black!70}{Not Applicable}}.
 \end{enumerate}


 \item For any theoretical claim, check if you include:
 \begin{enumerate}
   \item Statements of the full set of assumptions of all theoretical results. \\
     \textbf{\textcolor{ForestGreen}{Yes}}. See the theorem statements and \cref{section:preliminaries}
   \item Complete proofs of all theoretical results. \\
     \textbf{\textcolor{ForestGreen}{Yes}}. See \cref{section:proofs}.
   \item Clear explanations of any assumptions. \\
     \textbf{\textcolor{ForestGreen}{Yes}}. See \cref{section:definitions,section:preliminaries} and the main text.
 \end{enumerate}


 \item For all figures and tables that present empirical results, check if you include:
 \begin{enumerate}
   \item The code, data, and instructions needed to reproduce the main experimental results (either in the supplemental material or as a URL). \\
     \textbf{\textcolor{black!70}{Not Applicable}}.
   \item All the training details (e.g., data splits, hyperparameters, how they were chosen). \\
     \textbf{\textcolor{black!70}{Not Applicable}}.
   \item A clear definition of the specific measure or statistics and error bars (e.g., with respect to the random seed after running experiments multiple times). \\
     \textbf{\textcolor{black!70}{Not Applicable}}.
   \item A description of the computing infrastructure used. (e.g., type of GPUs, internal cluster, or cloud provider). \\
     \textbf{\textcolor{black!70}{Not Applicable}}.
 \end{enumerate}

 \item If you are using existing assets (e.g., code, data, models) or curating/releasing new assets, check if you include:
 \begin{enumerate}
   \item Citations of the creator If your work uses existing assets. \\
     \textbf{\textcolor{black!70}{Not Applicable}}.
   \item The license information of the assets, if applicable.\\
     \textbf{\textcolor{black!70}{Not Applicable}}.
   \item New assets either in the supplemental material or as a URL, if applicable. \\
     \textbf{\textcolor{black!70}{Not Applicable}}.
   \item Information about consent from data providers/curators. \\
     \textbf{\textcolor{black!70}{Not Applicable}}.
   \item Discussion of sensible content if applicable, e.g., personally identifiable information or offensive content. \\
     \textbf{\textcolor{black!70}{Not Applicable}}.
 \end{enumerate}

 \item If you used crowdsourcing or conducted research with human subjects, check if you include:
 \begin{enumerate}
   \item The full text of instructions given to participants and screenshots.  \\
     \textbf{\textcolor{black!70}{Not Applicable}}.
   \item Descriptions of potential participant risks, with links to Institutional Review Board (IRB) approvals if applicable.  \\
     \textbf{\textcolor{black!70}{Not Applicable}}.
   \item The estimated hourly wage paid to participants and the total amount spent on participant compensation. \\
     \textbf{\textcolor{black!70}{Not Applicable}}.
 \end{enumerate}

 \end{enumerate}


\newpage
\appendix

% \etocdepthtag.toc{mtappendix}
% \etocsettagdepth{mtchapter}{none}
% \etocsettagdepth{mtappendix}{part}

\onecolumn

\aistatstitle{Linear Convergence of Black-Box Variational Inference: \\ Should We Stick the Landing?\\
\textit{Supplementary Material}}

\vspace{-4ex}
\section{OVERVIEW OF THEOREMS}

{\hypersetup{linkcolor=black}
\begin{table*}[h!]
  \centering
\caption{Overview of Theorems}\label{table:theorems}
\vspace{1ex}
  {\small
%\setlength{\tabcolsep}{4pt}
\makegapedcells
\renewcommand{\arraystretch}{1.3}
\begin{tabularx}{\linewidth}{>{\centering}m{7ex}Xc}
    \toprule
    \textbf{Category}
    & \multicolumn{1}{c}{\textbf{Description}}
    & \multicolumn{1}{c}{\textbf{Theorem}}
    \\ \midrule
    %
    \multirow{6}{*}[-4.5ex]{\normalsize\rotatebox[origin=c]{90}{Gradient Variance Bounds}}
    & \textbf{Upper bound} for the gradient variance of the \textbf{STL} estimator with the \textbf{full-rank} parameterization
    & \cref{thm:stl_upperbound}
    \\
    & \textbf{Upper bound} for the gradient variance of the \textbf{STL} estimator with the \textbf{mean-field} parameterization
    & \cref{thm:stl_upperbound_mf}
    \\
    & \textbf{Upper bound} for the gradient variance of the \textbf{CFE} estimator with the \textbf{full-rank} parameterization
    & \cref{thm:cfe_upperbound}
    \\
    & \textbf{Upper bound} for the gradient variance of the \textbf{CFE} estimator with the \textbf{mean-field} parameterization
    & \cref{thm:cfe_upperbound_mf}
    \\
    & \textbf{Lower bound} for the gradient variance of the \textbf{STL} estimator with the \textbf{full-rank} parameterization
    & \cref{thm:stl_lowerbound}
    \\
    & \textbf{Worst case lower bound} (unimprovability) for the gradient variance of the \textbf{STL} estimator with the \textbf{full-rank} parameterization
    & \cref{thm:stl_lowerbound_unimprovability}
    \\
    \midrule
    \multirow{5}{*}[-2ex]{\normalsize\rotatebox[origin=c]{90}{Complexity of Projected SGD}}
    & Iteration complexity of projected SGD with a \textbf{fixed stepsize} and a gradient estimator satisfying the \textbf{QV} condition on a \textbf{strongly convex} objective function
    & 
    \cref{thm:projsgd_stronglyconvex_fixedstepsize}
    \\
    & Iteration complexity of projected SGD with a \textbf{decreasing stepsize} schedule and a gradient estimator satisfying the \textbf{QV} condition on a \textbf{strongly convex} objective function 
    & 
    \cref{thm:projsgd_stronglyconvex_decstepsize}
    %% \\
    %% & Iteration complexity of projected SGD with a \textbf{fixed stepsize} and a gradient estimator satisfying the \textbf{QV} condition on a \textbf{convex} objective function 
    %% & 
    %% \cref{thm:projsgd_convex_fixedstepsize}
    \\
    & Iteration complexity of projected SGD with a \textbf{fixed stepsize} and a gradient estimator satisfying the \textbf{adaptive QV} condition on a \textbf{strongly convex} objective function
    & 
    \cref{thm:projsgd_stronglyconvex_adaptive_complexity}
    \\
    & Iteration complexity of projected SGD with a \textbf{decreasing stepsize} schedule and a gradient estimator satisfying the \textbf{adaptive QV} condition on a \textbf{strongly convex} objective function
    & 
    \cref{thm:projsgd_stronglyconvex_decstepsize_adaptive_complexity}
    \\
    \midrule
    \multirow{5}{*}[-4ex]{\normalsize\rotatebox[origin=c]{90}{Complexity of BBVI}}
    & Iteration complexity of BBVI with projected SGD using a \textbf{fixed stepsize} and the \textbf{CFE} gradient estimator on a \textbf{strongly log-concave} posterior 
    & 
    \cref{thm:projsgd_bbvicfe_complexity}
    \\
    & Iteration complexity of BBVI with projected SGD using a \textbf{decreasing stepsize} schedule and the \textbf{CFE} gradient estimator on a \textbf{strongly log-concave} posterior 
    & 
    \cref{thm:projsgd_bbvicfe_decstepsize_complexity}
    \\
    & Iteration complexity of BBVI with projected SGD using a \textbf{fixed stepsize} and the \textbf{STL} gradient estimator on a \textbf{strongly log-concave} posterior 
    & \cref{thm:projsgd_bbvistl_complexity}
    \\
    & Iteration complexity of BBVI with projected SGD using a \textbf{decreasing stepsize} schedule and the \textbf{STL} gradient estimator on a \textbf{strongly log-concave} posterior 
    & 
    \cref{thm:projsgd_bbvistl_decstepsize_complexity}
    %% \\
    %% & Iteration complexity of BBVI with projected SGD using a \textbf{fixed stepsize} schedule and the \textbf{STL} gradient estimator on a \textbf{log-concave} posterior under \textbf{perfect variational family specification}
    %% & 
    %% \cref{thm:projsgd_bbvistl_decstepsize_complexity_logconcave}
    \\ \bottomrule
\end{tabularx}
  }%
\end{table*}
}


\newpage
{\hypersetup{linkcolor=black}
\tableofcontents
% \localtableofcontents
}

\newpage


%\begin{wraptable}[20]{r}[\dimexpr\columnwidth+\columnsep\relax]{0.65\textwidth}
\begin{table*}[t]
  \centering
\caption{Overview of Complexity Analyses of BBVI}\label{table:relatedworks}
  {\normalsize
\setlength{\tabcolsep}{4pt}
\renewcommand{\arraystretch}{1.2}
\begin{threeparttable}
\begin{tabular}{cccccccclc}
    \toprule
    \multicolumn{5}{c}{\textbf{Regularity of \(\pi\)}}
    & \multicolumn{1}{c}{\multirow{2}{*}{\textbf{\(q_{\vlambda^*} = \pi\)\tnote{1}}}}
    & \multicolumn{1}{c}{\multirow{2}{*}{\textbf{\makecell{Optimized\\Parameters}}}}
    & \multicolumn{1}{c}{\multirow{2}{*}{\textbf{\makecell{Gradient\\Estimator\tnote{2}}}}}
    & \multicolumn{1}{c}{\multirow{2}{*}{\textbf{\makecell{Iteration\\Complexity}}}}
    & \multicolumn{1}{c}{\multirow{2}{*}{\textbf{Reference}}}
    \\\cmidrule{1-5}
    \multicolumn{1}{c}{\textbf{\(\mu\)-PL}}
    & \multicolumn{1}{c}{\textbf{LC}}
    & \multicolumn{1}{c}{\textbf{\(\mu\)-SLC}}
    & \multicolumn{1}{c}{\textbf{\(L\)-LS}}
    & \multicolumn{1}{c}{\textbf{LQ}}
    &
    & 
    &
    &
    &
    \\ \midrule
    %
              &           &           &           & \ding{52} &           & scale only & exact & \(\mathcal{O}\left(\log\left( L \epsilon^{-1} \right) \right)\) &  \citealp{hoffman_blackbox_2020} \\
              &           &           &           & \ding{52} &           & scale only & CFE & \(\mathcal{O}\left(\kappa^2 \epsilon^{-1} \right)\) & \citealp{hoffman_blackbox_2020} \\
              &           &           &           & \ding{52} &           & scale only & n/a\tnote{3} & \(\mathcal{O}\left(L \epsilon^{-1} \right)\)\tnote{4} & \citealp{bhatia_statistical_2022} \\
   \textcolor{black!20}{\ding{52}}  & \textcolor{black!20}{\ding{52}} & \ding{52} & \ding{52} &           &           & scale only & n/a\tnote{3} & \(\mathcal{O}\left(L \epsilon^{-1} \right)\)\tnote{4} & \citealp{bhatia_statistical_2022} \\
    %
    \ding{52} &           &           & \ding{52} &           &           & loc. \& scale & CFE   & \(\mathcal{O}\left(L^2 \kappa \epsilon^{-4} \right)\) & \citealp{kim_blackbox_2023} \\
              % & \ding{52} &           & \ding{52} &           &           & loc. \& scale & CFE   & \(\mathcal{O}\left(L^2 \epsilon^{-2} \right)\) & \citealp{domke_provable_2023}  \\
    \textcolor{black!20}{\ding{52}} & \textcolor{black!20}{\ding{52}} & \ding{52} & \ding{52} &           &           & loc. \& scale & CFE   & \(\mathcal{O}\left(\kappa^2 \epsilon^{-1} \right)\) &  \makecell{\citealp{kim_blackbox_2023}\\\citealp{domke_provable_2023}} \\
     % & \ding{52} &  &  \ding{52} &   &           & loc. \& scale & CFE, STL & \(\mathcal{O}\left( \right)\) &  \citealp{domke_provable_2023} \\
    \textcolor{black!20}{\ding{52}} & \textcolor{black!20}{\ding{52}} & \ding{52} & \ding{52} &           &           & loc. \& scale & STL & \(\mathcal{O}\left(\kappa^2 \epsilon^{-1} \right)\) & \citealp{domke_provable_2023} \\
    \rowcolor{blue!10}
    \textcolor{black!20}{\ding{52}} & \textcolor{black!20}{\ding{52}} & \ding{52} & \ding{52} &           &           & loc. \& scale & STL   & \(\mathcal{O}\left(\kappa^2 \epsilon^{-1} \right)\) & \cref{thm:projsgd_bbvistl_decstepsize_complexity}\\
    \rowcolor{blue!10}
    \textcolor{black!20}{\ding{52}} & \textcolor{black!20}{\ding{52}} & \ding{52} & \ding{52} &           & \ding{52} & loc. \& scale & STL   & \(\mathcal{O}\left(\kappa^2 \log \epsilon^{-1} \right)\) &  \cref{thm:projsgd_bbvistl_complexity}
    %
    \\ \bottomrule
\end{tabular}
\begin{tablenotes}
\item[*] PL: Polyak-\L{}ojasiewicz, LC: log-concave, SLC: strongly-log-concave, LQ: log-quadratic (\(\pi\) is Gaussian), \(\kappa = L/\mu\).
\item[*] Analyses that a-priori assumed regularity of the ELBO were omitted.
\item[*] The explicit dimension dependences are omitted, but in general, \(\mathcal{O}\left(d\right)\) for full-rank, which is tight~\citep{domke_provable_2019}, and the best known for mean-field is {\scriptsize\(\mathcal{O}\left(\sqrt{d}\right)\)}~\citep{kim_practical_2023}.
The algorithm of \citet{bhatia_statistical_2022} is able to trade the dimension dependence for statistical accuracy.
\item[1] ``The variational family is perfectly specified.''
\item[2] The precise definitions of the gradient estimators are in \cref{section:gradient_estimators}.
\item[3] This algorithm uses stochastic power method-like iterations.
\item[4] The per-iteration sample complexity also depends on \(L, d, \epsilon\).
\end{tablenotes}
\end{threeparttable}
  }%
\end{table*}
%\end{wraptable}

%%% Local Variables:
%%% TeX-master: "main"
%%% End:


\section{RELATED WORKS}\label{section:related}

\paragraph{Analyzing the Computational Properties of BBVI}
Since its inception by \citet{ranganath_black_2014,titsias_doubly_2014}, theoretical results on BBVI have been developing on two different axes: 
\begin{enumerate*}[label=\textbf{(\alph*)}]
    \item Analyzing the regularity of the ELBO such as convexity and smoothness \citep{challis_gaussian_2013,titsias_doubly_2014}, 
    \item and analyzing the variance of the Monte Carlo gradient estimators \citep{fan_fast_2015,xu_variance_2019,mohamed_monte_2020,buchholz_quasimonte_2018}.
\end{enumerate*}
While some convergence analyses of BBVI have been provided \citep{regier_fast_2017,khan_faster_2016,khan_kullbackleibler_2015,buchholz_quasimonte_2018,liu_quasimonte_2021,alquier_concentration_2020,fujisawa_multilevel_2021,alquier_nonexponentially_2021}, these works \textit{a priori} assumed the regularity of the ELBO and the gradient estimators.
Due to the difficulty of rigorously establishing these conditions, later works by \citet{hoffman_blackbox_2020,bhatia_statistical_2022} have worked with simplified or alternative implementations of BBVI.
Meanwhile, \citet{xu_computational_2022} showed these regularities can be realized asymptotically in high probability.
In expectation, however, it was only recently that regularity conditions on the ELBO~\citep{domke_provable_2020,kim_blackbox_2023} and the reparameterization gradient estimator~\citep{kim_practical_2023,domke_provable_2019} were shown to be realizable under mild conditions without modifying the algorithms used in practice.

\paragraph{Concurrent Results by \citet{domke_provable_2023a}}
While our work builds on top of the QV-based framework of \citet{domke_provable_2023}, a similar convergence result on the STL estimator appeared in its later version~\citep{domke_provable_2023a} concurrently with our work.
However, our results differ in several aspects:
\begin{enumerate}
    \setlength\itemsep{-.5ex}
    \item[\ding{182}] For the family-misspecification term \(T_{\text{\ding{184}}}\) we bound the term using the Fisher-Hyv\"arinen divergence \(D_{\mathrm{F}^4}\left(q_{\vlambda^*}, \pi\right)\), while \citet{domke_provable_2023a} involve the smoothness constant of the residual function \(r\left(\vz\right) \triangleq \log q_{\vlambda^*}\left(\vz\right) - \log \pi\left(\vz\right)\).
    
    \item[\ding{183}] For the Gaussian posterior case, the constants of \cref{thm:stl_upperbound} are tighter that of \citet{domke_provable_2023a} by a factor of \(\times 2\).
    
    \item[\ding{184}] We also provide an upper bound for the mean-field variational family in \cref{thm:stl_upperbound_mf}.
    
    \item[\ding{185}] We establish a lower bound on the gradient variance of STL, quantifying the tightness of the bounds.
\end{enumerate}


\onecolumn

\newpage
\section{PROOFS}\label{section:proofs}


\subsection{Definitions}\label{section:definitions}

\begin{definition*}[\textbf{\(L\)-Smoothness}]
  A function \(f : \mathcal{X} \to \mathbb{R}\) is \(L\)-smooth if it satisfies
  \[
    \norm{ \nabla f\left(\vx\right) - \nabla f\left(\vy\right) }_2 \leq L \norm{ \vx - \vy }_2
  \]
  for all \(\vx, \vy \in \mathcal{X}\) and some \(L > 0\).
\end{definition*}

\begin{definition*}[\textbf{\(\mu\)-Strong Convexity}]
  A function \(f : \mathcal{X} \to \mathbb{R}\) is \(\mu\)-strongly convex if it satisfies
  \[
     \frac{\mu}{2} \norm{\vx - \vy}_2^2  \leq f\left(\vy\right) - f\left(\vx\right) - \inner{ \nabla f\left(\vx\right) }{ \vy - \vx }
  \]
  for all \(\vx, \vy \in \mathcal{X}\) and some \(\mu > 0\).
\end{definition*}

\vspace{1ex}
\begin{remark}
    We say a function \(f\) is only convex if it satisfies the strong convexity inequality with \(\mu = 0\).
\end{remark}

\vspace{1ex}
\begin{remark}[\textbf{Log-Concave Measures}]
    We say a probability measure \(\Pi\) is \(\mu\)-strongly log-concave if, in a \(d\)-dimensional Euclidean measurable space \((\mathbb{R}^d, \mathcal{B}^d, \mathbb{P})\), where \(\mathcal{B}^d\) is the \(\sigma\)-algebra of Borel-measurable subsets of \(\mathbb{R}^d\) and \(\mathbb{P}\) is the Lebesgue measure, its log probability density function \(x \mapsto -\log \pi\left(x\right) : \mathbb{R}^d \to \mathbb{R}\) is \(\mu\)-strongly convex.
\end{remark}

\vspace{1ex}
\begin{remark}[\textbf{Log-Smooth Densities}]
    Similarly, we say a probability distribution is \(L\)-log-smooth if its log-probability density function function \(\log \pi\) is \(L\)-smooth.
\end{remark}

% \begin{definition*}[\textbf{Relaxed Growth}; RG; \citealp{bottou_optimization_2018}]
%   A gradient estimator \(\rvvg\) satisfies the relaxed growth condition if
%   \[
%     \mathbb{E} \norm{ \rvvg\left(\vlambda\right) }_2^2 \leq \alpha \, \norm{ \nabla F \left(\vlambda\right) }_2^2 + \beta,
%   \]
%   for some \(0 \leq \alpha, \beta < \infty\).
% \end{definition*}

% \begin{remark}[\textbf{Expected Strong Growth}; \citealp{solodov_incremental_1998,vaswani_fast_2019}]
%   A gradient estimator \(\rvvg\) satisfies the expected strong growth condition if it satisfies the RG condition with \(\beta = 0\).
% \end{remark}

% \begin{definition*}[\textbf{Convex Expected Smoothness}; CES; \citealp{gower_sgd_2019}]
%   A gradient estimator \(\rvvg\) satisfies the convex expected smoothness condition if
%   \[
%     \mathbb{E} \norm{ \rvvg\left(\vlambda\right) }_2^2 \leq 4 A \left( F\left(\vlambda\right) - F^* \right) + C,
%   \]
%   for some \(0 \leq A, C < \infty\), where \(F^* = \inf_{\vlambda \in \Lambda}F\left(\vlambda\right)\).
% \end{definition*}


% \newpage
% \subsection{Miscellaneous Propositions}
% \vspace{1ex}
% 

\begin{theoremEnd}[all end, category=misc]{proposition}
  Let \(a, b \in \mathbb{R}\).
  Then,
  \[
     {\left( a + b \right)}^2
     \leq
    \left(1 + \delta \right) a^2 + \left(1 + \delta^{-1} \right) b^2,
  \]
  for any \(\delta \geq 0\).
\end{theoremEnd}
\begin{proofEnd}
  \begin{align*}
     {\left( a + b \right)}^2
     &=
     a^2 + 2 a b + b^2
     \\
     &\leq
     a^2 + 2 \left( \frac{\delta}{2} a + \frac{\delta^{-1}}{2} b\right) + b^2
     \\
     &=
     \left(1 + \delta \right) a^2 + \left(1 + \delta^{-1}\right) b^2.
  \end{align*}
  The inequality is known as the ``Peter-Paul'' inequality, which generalizes Young's inequality.
\end{proofEnd}
% \printProofs[logsobolev]
% \printProofs[misc]

\subsection{Auxiliary Lemmas}
\vspace{1ex}

% \subsection{Auxiliary Lemmas}
% \vspace{1ex}

\begin{theoremEnd}[all end, category=external]{lemma}[\citealt{domke_provable_2019}, Lemma 9]
\label{thm:u_identities}
  Let \(\rvvu = \left(\rvu_1, \rvu_2, \ldots, \rvu_d\right)\) be a \(d\)-dimensional vector-valued random variable with zero-mean independently and identically distributed components.
  Then,
  \begin{alignat*}{4}
    &\mathbb{E}\rvvu \rvvu^{\top} &&= \left( \mathbb{E} \rvu_i^2 \right) \boldupright{I}\qquad
    &&\mathbb{E}\norm{\rvvu}_2^2 &&= d \, \mathbb{E} \rvu_i^2
    \\
    &\mathbb{E} \rvvu \left( 1 + \norm{\rvvu}_2^2 \right) &&= \left( \mathbb{E} \rvu_i^3 \right) \mathbf{1}\qquad
    &&\mathbb{E} \rvvu \rvvu^{\top} \rvvu \rvvu^{\top} &&= \left( \left(d - 1\right) \, {\left( \mathbb{E} \rvu_i^2 \right)}^2 + \mathbb{E}\rvu_i^4 \right) \boldupright{I}.
  \end{alignat*}
\end{theoremEnd}

%% \begin{theoremEnd}[all end, category=external]{lemma}\label{thm:reparam_u_identity}
%%   Let \(\vt_{\vlambda}: \mathbb{R}^d \rightarrow \mathbb{R}^d\) be a location-scale reparameterizaiton function (\cref{def:reparam}).
%%   Also, let \(\vz \in \mathbb{R}^d\) be some vector and \(\rvvu \sim \varphi\) satisfy~\cref{assumption:symmetric_standard}.
%%   Then,
%%   \begin{alignat*}{2}
%%     \mathbb{E} J_{\mathcal{T}}\left(\rvvu\right) \norm{\mathcal{T}_{\vlambda}\left(\rvvu\right) - \vz}_2^2 
%%     =
%%     C_1\left(d, \varphi\right) \norm{ \vm - \vz }_2^2
%%     +
%%     C_2\left(d, \varphi\right) \norm{ \mC }_{\mathrm{F}}^2
%%   \end{alignat*}
%% \end{theoremEnd}

\begin{theoremEnd}[all end, category=external]{lemma}\label{thm:jacobian_reparam_inner}
  Let \(\mathcal{T}_{\vlambda}: \mathbb{R}^p \times \mathbb{R}^d \rightarrow \mathbb{R}^d\) be the location-scale reparameterization function (\cref{def:reparam}).
  Then, for any differentiable function \(f\), we have
  \[
    \norm{\nabla_{\vlambda} f\left( \mathcal{T}_{\vlambda}\left(\vu\right) \right) }_2^2 
    = 
    J_{\mathcal{T}}\left(\rvvu\right) \norm{\nabla f\left( \mathcal{T}_{\vlambda}\left(\vu\right) \right) }_2^2.
  \]
  for any \(\vlambda \in \mathbb{R}^p\) and \(\vu \in \mathbb{R}^d\), where \(J_{\mathcal{T}}\left(\vu\right) : \mathbb{R}^d \to \mathbb{R}\) is a function defined as
  \begin{alignat*}{2}
    J_{\mathcal{T}}\left(\vu\right) &= 1 + {\textstyle\sum^{d}_{i=1} u_i^2} \quad                        &&\quad \text{for the full-rank and} \\
    J_{\mathcal{T}}\left(\vu\right) &= 1 + {\textstyle\sqrt{\sum^{d}_{i=1} u_i^4}} &&\quad \text{for the mean-field parameterizations.}
  \end{alignat*}
\end{theoremEnd}
\begin{proofEnd}
  The result is a collection of the results of \citet[Lemma 1]{domke_provable_2019} for the full-rank parameterization and \citet[Lemma 2]{kim_practical_2023} for the mean-field parameterization.
\end{proofEnd}

\begin{theoremEnd}[all end, category=external]{lemma}[Corollary 3; \citealp{kim_blackbox_2023}]\label{thm:u_normsquared_marginalization}
  Assume \cref{assumption:symmetric_standard} and let \(\mathcal{T}_{\vlambda}: \mathbb{R}^d \rightarrow \mathbb{R}^d\) (\cref{def:reparam}) be the location-scale reparameterization function.
  Then, for any \(\vlambda, \vlambda^{\prime} \in \mathbb{R}^p\),
  \[
    \mathbb{E} J_{\mathcal{T}}\left(\rvvu\right) \norm{\mathcal{T}_{\vlambda^{\prime}}\left(\rvvu\right) - \mathcal{T}_{\vlambda}\left(\rvvu\right) }_2^2
    \leq
    C\left(d, \varphi\right) \norm{\vlambda - \vlambda'}_2^2,
  \]
  where 
  \begin{alignat*}{2}
    C\left(d, \varphi\right) &= d + k_{\varphi}            &&\quad \text{for the full-rank and} \\
    C\left(d, \varphi\right) &= 2 k_{\varphi} \sqrt{d} + 1 &&\quad \text{for the mean-field parameterizations.}
  \end{alignat*}
\end{theoremEnd}

\begin{theoremEnd}[all end, category=external]{lemma}[Lemma 2; \citealp{domke_provable_2019}]\label{thm:normdist_1pnormu}
  Assume \cref{assumption:symmetric_standard} and let \(\mathcal{T}_{\vlambda}: \mathbb{R}^d \rightarrow \mathbb{R}^d\) (\cref{def:reparam}) be the location-scale reparameterization function.
  Then, for the full-rank parameterization,
  \[
    \mathbb{E} J_{\mathcal{T}}\left(\rvvu\right) \norm{\mathcal{T}_{\vlambda}\left(\rvvu\right) - \bar{\vz}}_2^2
    =
    C_1\left(d, \varphi\right) \norm{ \vm - \bar{\vz} }_2^2 + C_2\left(d, \varphi\right) \norm{\mC}_{\mathrm{F}}^2.
  \]
  where
  \begin{alignat*}{4}
    C_1\left(d, \varphi\right)  &= d + 1,\quad
    &C_2\left(d, \varphi\right) &= d + k_{\varphi},
    &&\quad \text{for the full-rank and} \\
    C_1\left(d, \varphi\right)  &= \sqrt{d k_{\varphi}} + k \sqrt{d} + 1,\quad
    &C_2\left(d, \varphi\right) &= 2 \kappa \sqrt{d} + 1,
    &&\quad \text{for the mean-field parameterizations.}
  \end{alignat*}
\end{theoremEnd}
\begin{proofEnd}
  The result is a collection of the results of \citet[Lemma 2]{domke_provable_2019} for the full-rank parameterization and \citet[Lemma 2]{kim_practical_2023} for the mean-field parameterization.
\end{proofEnd}

\printProofs[external]
\printProofs[gradvarlemmas]

\clearpage
\subsection{Upper Bound on Gradient Variance of STL}

\subsubsection{General Decomposition}
\vspace{1ex}
\printProofs[stlupperboundlemma]

\newpage
\subsubsection{Full-Rank Parameterization}
\vspace{1ex}
\printProofs[stlupperboundfr]

\newpage
\subsubsection{Mean-Field Parameterization}\label{section:stl_meanfield}
\vspace{1ex}
\printProofs[stlupperboundmf]

\clearpage
\subsection{Lower Bound on Gradient Variance of STL}
\subsubsection{General Lower Bound}
\vspace{1ex}
\printProofs[stllowerbound]

\newpage
\subsubsection{Unimprovability}
\vspace{1ex}
\printProofs[stllowerboundunimprovability]

\clearpage
\subsection{Upper Bound on Gradient Variance of CFE}
\subsubsection{Full-Rank Parameterization}
\vspace{1ex}
\printProofs[cfeupperbound]

\newpage
\subsubsection{Mean-Field Parameterization}
\vspace{1ex}
\printProofs[cfeupperboundmf]

\clearpage
\subsection{Non-Asymptotic Complexity of Projected SGD}

To precisely compare the computational complexity resulting from different estimators, we refine the convergence analyses of~\citet{domke_provable_2023}.
Specifically, we obtain precise complexity guarantees from their ``anytime convergence'' statements.
This type of convergence analysis, which has been popular in the ERM sample selection strategy literature~\citep[\S 1.1]{csiba_importance_2018}, is convenient for comparing the lower-order or even constant factor improvements of different gradient estimators.

\subsubsection{QVC Gradient Estimator}
\vspace{1ex}
\printProofs[complexityprojsgdqvcfixed]
\newpage
\printProofs[complexityprojsgdqvcdec]

\clearpage
\subsubsection{Adaptive QVC Gradient Estimator}\label{section:complexity_adaptiveqvc}
\vspace{1ex}
\printProofs[complexityprojsgdadaptiveqvcfixed]

\newpage
\printProofs[complexityprojsgdadaptiveqvcdec]

\clearpage
\subsection{Non-Asymptotic Complexity of BBVI}
\subsubsection{CFE Gradient Estimator}
\vspace{1ex}
\printProofs[complexitybbvicfefixed]
\newpage
\printProofs[complexitybbvicfedec]

\newpage
\subsubsection{STL Gradient Estimator}
\vspace{1ex}
\printProofs[complexitybbvistl]
\newpage
\printProofs[complexitybbvistldec]

\newpage
\subsection{Fisher-Hyv\"arinen Divergence of Gaussians}
\subsubsection{Bounds on the Gradient Variance of STL}
\vspace{1ex}
\printProofs[stlgaussianfisher]
\newpage
\printProofs[gaussianklmeanfield]
\newpage
\printProofs[stlgaussian]


%\newpage
%\subsection{Connections with Other Gradient Variance Conditions}
%\vspace{1ex}
%\printProofs[gradientconditions]


%\printProofs

\end{document}
