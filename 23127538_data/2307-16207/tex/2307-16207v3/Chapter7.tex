\section{Conclusion}\label{sec7}


For the cellular-enabled UAV's path planning problem in both connectivity and battery constraints, we proposed the generalized intersection method with battery constraint (GIM-B) algorithm that outputs an optimal UAV trajectory in polynomial time. The efficacy of our algorithm, considering both mission completion time and computational complexity, was shown through comprehensive comparisons with existing algorithms both in analytically and numerically.
Furthermore, we proposed the energy-efficient generalized intersection method with battery constraint (EGIM-B) algorithm and the bottleneck edge search method that find the minimum UAV energy consumption and the maximum deliverable payload weight, respectively, under the connectivity and battery constraints.
Various numerical results were provided to show an optimal UAV trajectory and  various performance metrics according to  objectives and environmental parameters.

Let us conclude with some remarks on further works. We assumed that the delay at each CS is fixed over time, but in general, it  changes over time in practice. One interesting approach would be to assume the scenario of time-dependent delays to swap the battery at charging stations and develop shortest path finding algorithms over time-dependent graphs \cite{Orda:1990}. Another interesting scenario would be to consider more practical communication environments based on radio map taking into account signal blockage and reflection by buildings and interference from other BSs \cite{Chen:2017,Zhang:2021}.

