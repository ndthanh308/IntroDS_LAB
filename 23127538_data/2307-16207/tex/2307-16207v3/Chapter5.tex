\section{Other Objectives}\label{sec5}
In this section, we introduce two UAV path planning problems to move from the initial point $\mathbf{u}_0$ to the final point $\mathbf{u}_F$ under the connectivity and battery constraints, similar to Problem 1 but with different objectives. We consider the objectives of minimizing the energy consumption of the UAV and of maximizing the payload weight that can be delivered in Sections \ref{sec5A} and \ref{sec5B}, respectively.
\subsection{Minimum Energy Consumption}\label{sec5A}
This subsection focuses on finding the UAV trajectory to minimize the energy consumption of the UAV, considering the energy efficiency as in \cite{Zeng:2019,Qi:2020}. We can formulate the optimization problem as follows:
\begin{align} 
&\textbf{Problem 2} \cr
&\mbox{Objective:~}~ \min_{T\geq 0, \{\mathbf{u}(t),\ \psi(t),\ t\in[0,T]\}} \int_{t=0}^{T} {P_\mathrm{UAV}(v(t)) \over \eta }\mathrm{d}t\label{eq:eff1}\\
&\mbox{Constraints: }\cr
&\text{\eqref{eq:9}-\eqref{eq:17}}, \label{eq:eff2}
\end{align}where the objective \eqref{eq:eff1} is the total propulsion energy consumption of the UAV while traveling from $\mathbf{u}_0$ to $\mathbf{u}_F$.

To solve Problem 2, we introduce a slightly modified version of our GIM-B algorithm, called energy-efficient generalized intersection method with battery constraint (EGIM-B). This algorithm is almost the same as the original GIM-B algorithm, except that the weight of each edge at the global level is now given as the minimum energy consumption to travel between the corresponding two CSs and the flight speed during the travel over the edge is selected to minimize the energy consumption. The following proposition shows that the EGIM-B algorithm yields an optimal solution for Problem 2.
%\footnote{It can be shown that if $N\leq M$, the time complexity of the EGIM-B algorithm is $O(M^6)$, under the assumption that the complexity of finding the flight speed $v_\mathrm{eff}$ is negligible.}
\begin{proposition}\label{pro1}
The EGIM-B algorithm outputs an optimal solution for Problem 2.
\end{proposition}
\begin{proof}
This proof is immediate from Theorem \ref{Thm1} and the optimality of the Dijkstra algorithm because
\begin{enumerate}
    \item To minimize the propulsion energy consumption for traveling between two distinct CSs at the local level, the travel distance between those CSs should be minimized. Theorem \ref{Thm1} implies that all paths at the local level achieve the minimum travel distances.
    \item Under the graph in global level with the minimized edge weights (i.e., minimized energy consumption in local level), a trajectory that minimizes the energy consumption for traveling from $\mathbf{u}_0$ to $\mathbf{u}_F$ is derived via the Dijkstra algorithm.
\end{enumerate} 
\end{proof}
We note that in an optimal UAV trajectory for Problem 2, the UAV's flying speed is fixed to $v_\mathrm{eff}\in\mathcal{V}$, which minimizes the UAV's propulsion energy consumption per unit distance and is represented as follows: $v_\mathrm{eff}=\mathrm{argmin}_{v\in\mathcal{V}\setminus 0} {{P_\mathrm{UAV}(v)/(\eta v)}}$
%\begin{align}
%v_\mathrm{eff}=\mathrm{argmin}_{v\in\mathcal{V}\setminus 0} {{P_\mathrm{UAV}(v)/\eta}\over v}. \label{eq:eff3} 
%\end{align}
\sh{The following proposition shows that the EGIM-B algorithm operates with polynomial time complexity under the assumption $|\mathcal{V}|=O(M)$, whose proof is immediate from Theorem \ref{Thm5} and hence omitted.}
\begin{proposition}\label{pro2} 
\sh{ If the number of CSs is smaller than or equal to the number of BSs, i.e., $N\leq M$, then the time complexity of the EGIM-B algorithm is $O(M^6)$. }
\end{proposition}
%\begin{proof}
%It is immediate from Theorem \ref{Thm5} since the EGIM-B algorithm is almost the same with the GIM-B algorithm except that the UAV speed is  fixed to $v_\mathrm{eff}$.
%\end{proof}


\subsection{Maximum Deliverable Payload Weight}\label{sec5B}
In this subsection, we focus on deriving the maximum deliverable payload weight, instead of the minimum mission time. We can formulate the optimization problem as
\begin{align} 
&\textbf{Problem 3} \cr
&\mbox{Objective:~}~~~~ \max_{w_3\geq 0, \{\mathbf{u}(t),\ \psi(t),\ t\in[0,T]\}} w_3 \label{eq:33}\\
&\mbox{Constraints: }\cr
&0\leq T<\infty \label{eq:34} \\
&\text{\eqref{eq:9}-\eqref{eq:17}}, \label{eq:35}
\end{align}
where \eqref{eq:34} means that the UAV succeeds to deliver the payload from $\mathbf{u}_0$ to $\mathbf{u}_F$ within a finite time. We note that the propulsion power consumption $P_\mathrm{UAV}(v(t))$ in \eqref{eq:15} depends on the payload weight $w_3$. 

To solve Problem 3, we propose the bottleneck edge search method described in Algorithm \ref{Algo6}.  
%%%%%%% Algorithm 6: Bottleneck edge search method %%%%%%%%%
\begin{algorithm}[t]
\caption{Bottleneck Edge Search Method} \label{Algo6}
\textbf{Input:} $\mathbf{u}_0$, $\mathbf{u}_F$, $\mathcal{V}$, $\mathbf{c}_n$, $\ell_\mathrm{LO}(\mathbf{c}_n,\mathbf{c}_{n'})$, $h_\mathrm{Lfea}(\mathbf{c}_n,\mathbf{c}_{n'})$, $w_1$, $w_2$, $\epsilon_w$, $k_\mathrm{max}$ for $n\in [1:N+1]$, $n'\in[1:N]\cup \{N+2\}$, $n<n'$
\begin{algorithmic}[1]
\State $V_\mathrm{GL}\leftarrow\{\mathbf{u}_0,\mathbf{u}_F,\mathbf{c}_1,...,\mathbf{c}_N\}$, $E'_\mathrm{GL}\leftarrow \emptyset$, $w_3\leftarrow 0$
\State $\mathbf{c}_{N+1} \leftarrow \mathbf{u}_0$, $\mathbf{c}_{N+2} \leftarrow \mathbf{u}_F$, $h_\mathrm{sp}\leftarrow 1$
%\State $\epsilon_w\leftarrow 0.01$  \hfill\Comment{Sufficiently small positive constant}
\LeftComment{Step 1. Graph construction: Construct a graph $G'_\mathrm{GL}$ whose vertex set consists of CSs and edge set consists of edges between two connected CSs. The weight of each edge is the minimum travel distance between two CSs.}
\For {$n\in [1:N+1]$, $n'\in[1:N]\cup \{N+2\}$, $n<n'$}
    \LeftComment{Parameters $h_\mathrm{Lfea}$ and $\ell_\mathrm{LO}$ are described in Algorithm \ref{Algo4}.}
    \If{$h_\mathrm{Lfea}(\mathbf{c}_n,\mathbf{c}_{n'})=1$}
    \State $E'_\mathrm{GL}\leftarrow E'_\mathrm{GL}\cup (\mathbf{c}_n,\mathbf{c}_{n'},\ell_\mathrm{LO}(\mathbf{c}_n, \mathbf{c}_{n'}))$
    \EndIf
\EndFor
\State $G'_\mathrm{GL}\leftarrow (V_\mathrm{GL}, E'_\mathrm{GL})$ 
\LeftComment{Step 2. Bottleneck edge search: Find the longest connectivity-critical edge in $G'_\mathrm{GL}$.}
\LeftComment{Function \textbf{BFS} is described in line $1$ at Algorithm \ref{Algo4}.}
\State $h'_\mathrm{Gfea}\leftarrow$ \textbf{BFS}$(\mathbf{u}_0,\mathbf{u}_F,G'_\mathrm{GL})$
\If{$h'_\mathrm{Gfea}=1$}
\While{$h'_\mathrm{Gfea}=1$} 
    \State $(\mathbf{c}_k,\mathbf{c}_{k'},\ell_\mathrm{bott}) \leftarrow \!\!\!\!\!\!\!\!\!\!\!\!\!\!\!\!\!\!
    \underset{{(\mathbf{c}_n,\mathbf{c}_{n'},\ell_\mathrm{LO}(\mathbf{c}_n, \mathbf{c}_{n'}))\in E'_\mathrm{GL}}}{\mathrm{argmax}}$\!\!\!\!\!\!\!\!\!\!\!\!\!\!\!\!\! $\ell_\mathrm{LO}(\mathbf{c}_n, \mathbf{c}_{n'})$
    \LeftComment{Eliminate the longest edge in $E'_\mathrm{GL}$.}
    \State $E'_\mathrm{GL}\leftarrow E'_\mathrm{GL}\setminus (\mathbf{c}_k,\mathbf{c}_{k'},\ell_\mathrm{bott})$
    \State $G'_\mathrm{GL}\leftarrow (V_\mathrm{GL}, E'_\mathrm{GL})$
    \State $h'_\mathrm{Gfea}\leftarrow$ \textbf{BFS}$(\mathbf{u}_0,\mathbf{u}_F,G'_\mathrm{GL})$
\EndWhile
\Else
    \State $\ell_\mathrm{bott}\leftarrow \infty$
\EndIf
\LeftComment{Step 3. Weight search: Derive the maximum deliverable payload weight $w_3$.}
\LeftComment{Parameter $\epsilon_w>0$ is a sufficiently small constant.}
\While{$h_\mathrm{sp}=1$, $w_3\leq k_\mathrm{max}\epsilon_w$} \hfill%\Comment{\hs{$k_\mathrm{max}\in\mathbb{N}$}}
    \State $w_3\leftarrow w_3+\epsilon_w$
    \LeftComment{Function \textbf{ChkSp} is described in Algorithm \ref{Algo5}.}
    \State $(h_\mathrm{sp},v)\leftarrow$ \textbf{ChkSp}$(\ell_\mathrm{bott},\mathcal{V},w_1+w_2+w_3,w_2)$
\EndWhile
\State $w_3\leftarrow w_3-\epsilon_w$
\end{algorithmic}
\textbf{Output:} $w_3$
\end{algorithm}
%%%%%%%%%%%%%%%%%%%%%%%%
This algorithm initially sets zero payload weight, i.e., $w_3=0$. It first constructs an undirected weighted graph $G'_\mathrm{GL}$ whose vertex set consists of CSs (by
treating the initial and the final points also as CSs) and whose edge set includes an edge between two CSs only when there exists a path between the two CSs $(h_\mathrm{Lfea}=1)$, with the weight of the travel distance $\ell_\mathrm{LO}$ (in lines $3$-$8$). Note that the parameters $h_\mathrm{Lfea}$ and $\ell_\mathrm{LO}$ for each pair of CSs can be obtained by Algorithm \ref{Algo4}. After constructing the graph, it finds the bottleneck edge, which is the longest connectivity-critical edge in the graph (in lines $9$-$19$). To this end, the algorithm first checks whether $\mathbf{u}_0$ and $\mathbf{u}_F$ are connected by applying the function BFS \cite{West:2001}. If connected, it repeatedly eliminates the longest edge from the graph and then checks whether they are connected in the graph until not connected ($h'_\mathrm{Gfea}=0$). After the repetition ends, the most recently deleted edge is set as the bottleneck edge, with the edge weight $\ell_\mathrm{bott}$. Finally, the maximum deliverable payload weight over the graph is derived (in lines $20$-$24$). It first checks whether the UAV can travel the distance $\ell_\mathrm{bott}$ without battery replacement ($h_\mathrm{sp}=1$) or not ($h_\mathrm{sp}=0$) at $w_3=0$ via the function ChkSp whose pseudo code is in algorithm \ref{Algo5}. Then, it iterates this process while increasing $w_3$ in sufficiently small increments $\epsilon_w>0$ until the UAV cannot deliver the payload over the bottleneck edge or $w_3$ exceeds the limit $k_\mathrm{max}\epsilon_w$ of the payload weight. This algorithm outputs the maximum payload weight $w_3\in\{0,\epsilon_w,...,k_\mathrm{max}\epsilon_w\}$ which can be delivered over the bottleneck edge.

\sh{Now, the following theorems show that our bottleneck edge search method yields an optimal solution of Problem 3 in polynomial time.}
%\footnote{It can be shown that the bottleneck edge search method solves Problem~3 in polynomial time.}
\begin{theorem}\label{Thm6} % Optimality of bottleneck edge search method
Assume that the payload weight $w_3$ is selected from the set $\{0,\epsilon_w,...,k_\mathrm{max}\epsilon_w\}$ for $\epsilon_w>0$ and $k_\mathrm{max}\in\mathbb{N}$. Then, the bottleneck edge search method outputs the optimal solution for Problem 3 if the power consumption model $P_\mathrm{UAV}(v)$ is convex in the range of the UAV speed.  
\end{theorem}
\begin{proof} 
If the UAV can travel the bottleneck edge without battery replacement, then the payload can be delivered from $\mathbf{u}_0$ to $\mathbf{u}_F$ since it can be also delivered over an edge shorter than the bottleneck edge under the battery constraint. Hence, it is sufficient only to consider whether the payload can be delivered over the bottleneck edge. For finding the maximum deliverable payload weight, we note that it is sufficient only to consider a fixed speed while traveling the bottleneck edge, as justified in Theorem \ref{Thm3}. Consequently, our method yields the optimal solution for Problem 3.
\end{proof}
\sh{
\begin{theorem}\label{Thm7} % Complexity of bottleneck edge search method
Assume that the payload weight $w_3$ is selected from the set $\{0,\epsilon_w,...,k_\mathrm{max}\epsilon_w\}$ for $\epsilon_w>0$ and $k_\mathrm{max}\in\mathbb{N}$. Then, the 
time complexity of the bottleneck edge search method is $O(N^4+k_\mathrm{max}|\mathcal{V}|)$.
\end{theorem}
\begin{proof}
Note that $|V_\mathrm{GL}|=O(N)$ and $|E'_\mathrm{GL}|=O(N^2)$. The steps in Algorithm \ref{Algo6} have the following complexities:
\begin{itemize}
\item Step 1. Graph construction: This step has the complexity $O(N^2)$ since it just loads all the feasible edges (i.e., $h_\mathrm{Lfea}=1$) between two vertices.
\item Step 2. Bottleneck edge search: First, the complexity of finding an edge to be eliminated from graph $G'_\mathrm{GL}$, which is a longest edge in the graph, is $O(|E'_\mathrm{GL}|)=O(N^2)$. Next, the complexity of applying function BFS in graph $G'_\mathrm{GL}$ is $O(|V_\mathrm{GL}|^2)=O(N^2)$ \cite{West:2001}. Since such process is repeated at most $|E'_\mathrm{GL}|$ times, the complexity of searching a bottleneck edge is $(O(N^2)+O(N^2))\cdot |E'_\mathrm{GL}|=O(N^4)$.
\item Step 3. Weight search: The complexity of checking whether each payload $w_3\in\{0,\epsilon_w,...,k_\mathrm{max}\epsilon_w\}$ is deliverable via the function ChkSp is $(k_\mathrm{max}+1)\cdot O(|\mathcal{V}|)=O(k_\mathrm{max}|\mathcal{V}|)$.
\end{itemize}
Consequently, the complexity of the bottleneck edge search method is $O(N^4+k_\mathrm{max}|\mathcal{V}|)$.
\end{proof}
}