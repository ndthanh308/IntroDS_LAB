 
%%%% TVT revision version %%%%

% IEEE Transactions on Microwave Theory and Techniques example
% Tibault Reveyrand - http://www.microwave.fr
%
% http://www.microwave.fr/LaTeX.html
% ---------------------------------------



% ================================================
% Please HIGHLIGHT the new inputs such like this :
% Text :
%  \hl{comment}
% Aligned Eq. 
% \begin{shaded}
% \end{shaded}
% ================================================



\documentclass[journal]{IEEEtran}
\ifCLASSINFOpdf
\usepackage[pdftex]{graphicx}
\else
\usepackage[dvips]{graphicx}
\usepackage[caption=false,font=footnotesize]{subfig}
\fi
\usepackage[caption=false,font=footnotesize]{subfig}
%\usepackage{subcaption}
\usepackage{setspace}
\usepackage{algorithm}
%\usepackage{algorithmic}
\usepackage{algorithmicx} 

\usepackage{multicol}
\usepackage{lipsum}
%\usepackage[shortlabels]{enumitem}
\usepackage{setspace}
\usepackage{arydshln}
\usepackage{makecell}
\usepackage{rotating}
\usepackage{makecell, multirow}
\usepackage[skip=1ex]{caption}
\newcommand{\tabitem}{~~\llap{\textbullet}~~}
\usepackage{fix-cm}
%\usepackage[retainorgcmds]{IEEEtrantools}
%\usepackage{bibentry}  
\usepackage{xcolor,soul,framed} %,caption
\usepackage{subfiles}
\colorlet{shadecolor}{yellow}
% \usepackage{color,soul}
\usepackage[pdftex]{graphicx}
\graphicspath{{../pdf/}{../jpeg/}}
\DeclareGraphicsExtensions{.pdf,.jpeg,.png}
\usepackage{tikz}
%Mathabx do not work on ScribTex => Removed
%\usepackage{mathabx}
\usepackage{array}
\usepackage{mdwmath}
\usepackage{mdwtab}
\usepackage{eqparbox}
\usepackage{url}
\usepackage{cite}
\usepackage{epsfig,amsmath,amssymb,epsf,amsthm,scalefnt,multirow,subfig}
\usepackage{xcolor}
\usepackage{float}
\usepackage{psfrag}
\usepackage{algpseudocode}
\usepackage{verbatim}
\usepackage{tikz}
\usepackage{makecell}
\usetikzlibrary{tikzmark,calc,decorations.pathreplacing}
\usepackage{amsmath}
\usepackage[utf8]{inputenc}
\newenvironment{rcases}
  {\left.\begin{aligned}}
  {\end{aligned}\right\rbrace}


\algnewcommand{\LeftComment}[1]{\Statex \(\triangleright\) #1}
\newcommand{\argmax}{\mathop{\mathrm{argmax}}}
\newcommand{\argmin}{\mathop{\mathrm{argmin}}}
\newcommand{\var}{\mathop{Var}}
\newcommand{\specialcell}[2][c]{%
  \begin{tabular}[#1]{@{}c@{}}#2\end{tabular}}
%\newenvironment{theorem}[2][Theorem]{\begin{trivlist}\item[\hskip \labelsep {\bfseries #1}\hskip \labelsep {\bfseries #2}]}{\end{trivlist}}
\newtheorem{theorem}{Theorem}
\newtheorem{lemma}{Lemma}
\newtheorem*{lemma*}{Lemma}
\newtheorem{remark}{Remark}
\newtheorem{observation}{Observation}
\newtheorem{corollary}{Corollary}
\newtheorem{definition}{Definition}
\newtheorem{conjecture}{Conjecture}
\newtheorem{proposition}{Proposition}

% Calligraphic uppercase
\def\cA{{\mathcal{A}}} \def\cB{{\mathcal{B}}} \def\cC{{\mathcal{C}}} \def\cD{{\mathcal{D}}}
\def\cE{{\mathcal{E}}} \def\cF{{\mathcal{F}}} \def\cG{{\mathcal{G}}} \def\cH{{\mathcal{H}}}
\def\cI{{\mathcal{I}}} \def\cJ{{\mathcal{J}}} \def\cK{{\mathcal{K}}} \def\cL{{\mathcal{L}}}
\def\cM{{\mathcal{M}}} \def\cN{{\mathcal{N}}} \def\cO{{\mathcal{O}}} \def\cP{{\mathcal{P}}}
\def\cQ{{\mathcal{Q}}} \def\cR{{\mathcal{R}}} \def\cS{{\mathcal{S}}} \def\cT{{\mathcal{T}}}
\def\cU{{\mathcal{U}}} \def\cV{{\mathcal{V}}} \def\cW{{\mathcal{W}}} \def\cX{{\mathcal{X}}}
\def\cY{{\mathcal{Y}}} \def\cZ{{\mathcal{Z}}} \def\cz{{\mathcal{z}}}\def\cce{{\mathcal{e}}}

\def\argmin{\mathop{\mathrm{argmin}}}
\def\sinr{\mathop{\mathrm{SINR}}}    

\def\limit{\mathop{\mathrm{lim}}}
\def\argmax{\mathop{\mathrm{argmax}}}
\def\diag{\mathop{\mathrm{diag}}}
\def\inf{\mathop{\mathrm{inf}}}
\def\outf{\mathop{\mathrm{out}}}
\def\trace{\mathop{\mathrm{tr}}}
\def\hh{\mathop{\mathrm{h}}}
\def\EE{\mathop{\mathrm{E}}}
\def\Var{\mathop{\mathrm{Var}}}
\def\dim{\mathop{\mathrm{dim}}}
\def\Re{\mathop{\mathrm{Re}}}
\def\Im{\mathop{\mathrm{Im}}}

\newcommand{\Ei}{{\mathrm{Ei}}}

\def\bDelta{{\pmb{\Delta}}} \def\bdelta{{\pmb{\delta}}}
\def\bSigma{{\pmb{\Sigma}}} \def\bsigma{{\pmb{\sigma}}}
\def\bPhi{{\pmb{\Phi}}} \def\bphi{{\pmb{\phi}}}
\def\bGamma{{\pmb{\Gamma}}} \def\bgamma{{\pmb{\gamma}}}
\def\bOmega{{\pmb{\Omega}}} \def\bomega  
\def\bTheta{{\pmb{\Theta}}} \def\btheta{{\pmb{\theta}}}
\def\bepsilon{{\pmb{\epsilon}}} \def\bPsi{{\pmb{\Psi}}}
\def\obH{\overline{\bH}}\def\obw{\overline{\bw}} \def\obz{\overline{\bz}}\def\bmu{{\pmb{\mu}}}
\def\b0{{\pmb{0}}}\def\bLambda{{\pmb{\Lambda}}} \def\oc{\overline{\bc}}

% Bold
\def\ba{{\mathbf{a}}} \def\bb{{\mathbf{b}}} \def\bc{{\mathbf{c}}} \def\bd{{\mathbf{d}}}
\def\bee{{\mathbf{e}}} \def\bff{{\mathbf{f}}} \def\bg{{\mathbf{g}}} \def\bh{{\mathbf{h}}}
\def\bi{{\mathbf{i}}} \def\bj{{\mathbf{j}}} \def\bk{{\mathbf{k}}} \def\bl{{\mathbf{l}}}
\def\bm{{\mathbf{m}}} \def\bn{{\mathbf{n}}} \def\bo{{\mathbf{o}}} \def\bp{{\mathbf{p}}}
\def\bq{{\mathbf{q}}} \def\br{{\mathbf{r}}} \def\bs{{\mathbf{s}}} \def\bt{{\mathbf{t}}}
\def\bu{{\mathbf{u}}} \def\bv{{\mathbf{v}}} \def\bw{{\mathbf{w}}} \def\bx{{\mathbf{x}}}
\def\by{{\mathbf{y}}} \def\bz{{\mathbf{z}}} \def\bxb{\bar{\mathbf{x}}} \def\bone{\mathbf{1}}

\def\bA{{\mathbf{A}}} \def\bB{{\mathbf{B}}} \def\bC{{\mathbf{C}}} \def\bD{{\mathbf{D}}}
\def\bE{{\mathbf{E}}} \def\bF{{\mathbf{F}}} \def\bG{{\mathbf{G}}} \def\bH{{\mathbf{H}}}
\def\bI{{\mathbf{I}}} \def\bJ{{\mathbf{J}}} \def\bK{{\mathbf{K}}} \def\bL{{\mathbf{L}}}
\def\bM{{\mathbf{M}}} \def\bN{{\mathbf{N}}} \def\bO{{\mathbf{O}}} \def\bP{{\mathbf{P}}}
\def\bQ{{\mathbf{Q}}} \def\bR{{\mathbf{R}}} \def\bS{{\mathbf{S}}} \def\bT{{\mathbf{T}}}
\def\bU{{\mathbf{U}}} \def\bV{{\mathbf{V}}} \def\bW{{\mathbf{W}}} \def\bX{{\mathbf{X}}}
\def\bY{{\mathbf{Y}}} \def\bZ{{\mathbf{Z}}}

\def\oX{\overline{\bX}}
\def\oGamma{\overline{\bGamma}}

\newcommand{\C}{\mathbb{C}}%{{\mbox{\rm $\scriptscriptstyle ^\mid$\hspace{-0.40em}C}}} %o
\newcommand{\Z}{\mathbb{Z}}%{{\mbox{\rm $\scriptscriptstyle ^\mid$\hspace{-0.40em}C}}} %o
\allowdisplaybreaks[4]

% Block for Real, Complex
\def\bbC{{\mathbb{C}}} \def\bbR{{\mathbb{R}}}

\newcommand{\eff}{\mbox{\rm eff}}
\newcommand{\out}{\mbox{\rm out}}
\newcommand{\squeezeup}{\vspace{-2.5mm}}

\newcolumntype{P}[1]{>{\centering\arraybackslash}p{#1}}
\hyphenation{op-tical net-works semi-conduc-tor}
\def\checkmark{\tikz\fill[scale=0.4](0,.35) -- (.25,0) -- (1,.7) -- (.25,.15) -- cycle;} 


\newcommand{\sh}[1]{\textcolor{black}{{#1}}}
\newcommand{\shc}[1]{\textcolor{blue}{\sout{#1}}}

\newcommand{\hs}[1]{\textcolor{red}{{#1}}}
\newcommand{\hsc}[1]{\textcolor{red}{\sout{#1}}}

\newcommand{\ky}[1]{\textcolor{orange}{{#1}}}
\newcommand{\kyc}[1]{\textcolor{orange}{\sout{#1}}}

%\bstctlcite{IEEE:BSTcontrol}


%=== TITLE & AUTHORS ====================================================================
\begin{document}
\title{Trajectory Optimization for Cellular-Enabled UAV with Connectivity and Battery Constraints}

\author{Hyeon-Seong Im,~%\IEEEmembership{Member,~IEEE,}
        Kyu-Yeong Kim,~\IEEEmembership{Graduate Student Member,~IEEE,}
        and Si-Hyeon Lee,~\IEEEmembership{Senior Member,~IEEE}% <-this % stops a space
%%%%%%% \thanks{} %% Acknowledgement
\thanks{
Copyright (c) 2025 IEEE. Personal use of this material is permitted. However, permission to use this material for any other purposes must be obtained from the IEEE by sending a request to pubs-permissions@ieee.org.

This article was presented in part at the IEEE Vehicular Technology Conference (VTC) 2023-Fall \cite{Im:2023_VTC}.
This work was supported in part by the Institute of Information \& Communications Technology Planning \& Evaluation (IITP) grant (No.RS-2024-00360387, Development of core security technology utilizing multimodal properties of wireless communication channels, contribution rate: 50\%), in part by the IITP-ITRC (Information Technology Research Center) grant (IITP-2025-RS-2020-II201787, Development of communication/computing-integrated revolutionary technologies for superintelligent services, contribution rate: 30\%), and in part by Unmanned Vehicles Core Technology Research and Development Program through the National Research  Foundation of Korea (NRF) and the Unmanned Vehicle Advanced Research Center (UVARC) grant (No.2020M3C1C1A01084524, Development of anti-jamming for unmanned vehicles security and detection and counter technology to unlicensed unmanned vehicles, contribution rate: 20\%), all funded by the Korea government (MSIT).



H.-S. Im is with LIG Nex1, Seongnam 13488, South Korea (e-mail: hyeonseong.im@lignex1.com).  He was with the School of Electrical Engineering,  Korea Advanced Institute of Science and Technology (KAIST), Daejeon 34141, South Korea, when conducting this work. K.-Y. Kim and S.-H. Lee (Corresponding Author) are with the School of Electrical Engineering, KAIST, Daejeon 34141, South Korea (e-mail: kimyou283@kaist.ac.kr, sihyeon@kaist.ac.kr). }% <-this % stops a space
}



% ====================================================================
\maketitle



% === ABSTRACT ====================================================================
% =================================================================================
\begin{abstract}
We address the path planning problem for a cellular-enabled unmanned aerial vehicle (UAV) considering both connectivity and battery constraints. The UAV's mission is to expeditiously transport a payload from an initial point to a final point, while persistently keeping the connection with a base station and complying with its battery limit. At a charging station, the UAV's depleted battery can be swapped with a completely charged one. 
Our primary contribution lies in proposing an algorithm that outputs an optimal UAV trajectory with polynomial computational complexity, by converting the problem into an equivalent two-level graph-theoretic shortest path search problem. 
We compare our algorithm with several existing algorithms with respect to performance and computational complexity, and show that only our algorithm outputs an optimal UAV trajectory in polynomial time. 
Furthermore, we consider other objectives of minimizing the UAV energy consumption and of maximizing the deliverable payload weight, and propose algorithms that output an optimal UAV trajectory in polynomial time.

\end{abstract}


\begin{IEEEkeywords}
Unmanned aerial vehicle,  trajectory optimization, connectivity, cellular networks, battery constraint
\end{IEEEkeywords}






% For peer review papers, you can put extra information on the cover
% page as needed:
% \ifCLASSOPTIONpeerreview
% \begin{center} \bfseries EDICS Category: 3-BBND \end{center}
% \fi
%
% For peerreview papers, this IEEEtran command inserts a page break and
% creates the second title. It will be ignored for other modes.
\IEEEpeerreviewmaketitle


% ====================================================================
% ====================================================================
% ====================================================================

\section{Introduction}\label{sec1}
Unmanned aerial vehicles (UAVs) are extensively applied across various scenarios, such as delivery and transportation \cite{Zhang:2021_2}, aerial surveillance and monitoring \cite{Kanistras:2013}, flying base stations (BSs) \cite{Mozaffari:2019}, and data collection and/or power transfer for IoT devices  \cite{Yu:2021}, due to their high mobility, free movement, and cost-effectiveness  \cite{Mozaffari:2019,Shi:2018}. It has been actively studied to design the UAV trajectory according to each operational scenario. For UAV-aided communication scenarios, the UAV trajectory has been optimized taking into account various factors, e.g., minimizing energy while satisfying throughput demands  \cite{Zeng:2019,Qi:2020}, \sh{minimizing the age of information (AoI) of IoT devices \cite{Yi:2023}, jointly optimizing the transmission rate and the rate variation in aerial video streaming scenario \cite{Zhan:2024_2}}, and improving secrecy rate in the presence of an eavesdropper \cite{Li:2019,Cui:2018}. For delivery or transportation scenarios, it is utmost important  to swiftly and safely transport the given objects to their desired destinations. Thus, for such scenarios, the problem of designing UAV trajectory has been formulated as minimizing the mission time or \sh{the energy consumption (equivalent to minimizing the mission time of UAV with fixed speed)} with some constraints such as \sh{connectivity \cite{Zhang:2019,Zhang:2019_2,Chen:2020,Zhan:2022,Zhan:2022_2,Zhang:2021,Esrafilian:2020,Chapnevis:2021,Zeng:2019_2,Khamidehi:2020,Wang:2022,Chen:2022}}, restricted airspace \cite{Khamidehi:2020,Wang:2022}, collision avoidance between UAVs \cite{Wang:2022,Chen:2022}, and battery limits \cite{Sundar:2014,Coelho:2017,Fan:2023,Arafat:2022}. 
In particular, it is important for such scenarios to consistently keep the connection between the UAV and the control station. This persistent connectivity is 
essential for tracking the real-time location of the cargo in delivery scenarios \cite{Zhang:2019} and for providing real-time communication service to passengers in transportation scenarios like urban air mobility (UAM) \cite{Cohen:2021}. It is also vital when the manual control is necessary for the UAV to evade unexpected adverse weather conditions or avoid collisions with aerial obstacles \cite{Banafaa:2024}. However, maintaining direct connection with the control station becomes challenging if the initial and the final points are far away, due to low received signal strength by low line-of-sight (LoS) probability and long communication distance. Cellular-enabled UAV communication is a potential approach for this problem \cite{Zhang:2019}, wherein the UAV communicates with its control station by connecting with a close BS and the underlying cellular network \cite{Agyapong:2014}. 

Our paper addresses the path planning problem for UAVs performing delivery or transportation missions, jointly considering connectivity with the cellular network and UAV’s battery limit. In the absence of battery limit, the problem of minimizing the mission time \sh{(or the energy consumption)} while ensuring continuous connection with the cellular network has been \sh{extensively studied \cite{Zhang:2019,Zhang:2019_2,Chen:2020,Zhan:2022,Zhan:2022_2,Zhang:2021,Esrafilian:2020,Chapnevis:2021,Zeng:2019_2,Khamidehi:2020,Chen:2022,Wang:2022}.}  
The authors of \cite{Zhang:2019} concentrated on planning an optimal path between a initial and a final points while ensuring continuous connection with a BS. To simplify the problem, they assumed that the UAV and a BS can be connected if their communication distance is not larger than a certain threshold. Then, they transformed the problem into graph-theoretic path finding and convex optimization problems, and proposed an optimal algorithm with non-polynomial time complexity and two sub-optimal algorithms with polynomial time complexity. 
The study of characterizing an optimal path under the connectivity constraint has been extended in various directions, e.g., \sh{allowing a certain duration or ratio of communication outage \cite{Zhang:2019_2,Chen:2020,Zhan:2022}, assuming the knowledge of the radio map \cite{Zhan:2022_2,Zhang:2021}}, considering 3-dimensional (3D) space \cite{Esrafilian:2020,Zhang:2021},
%design a 3D path based on 3D building maps \cite{Esrafilian:2020} or 3D radio map \cite{Zhang:2021}, 
and considering the collaboration of multiple UAVs \cite{Chapnevis:2021}. 
Specifically, the work \cite{Chen:2020} proposed an intersection method, significantly reducing the computational complexity by converting the problem into a graph-theoretic path finding problem whose vertex set consists of the intersection points of the coverage boundaries of the BSs. Moreover, the work \cite{Zhang:2021} also used a graph-theoretic approach even for 3D path finding problem with a realistic communication environment considering signal blockage and reflection by buildings and interference from other BSs, by quantizing the radio map to finite grid points. On the other hand, reinforcement learning (RL) \cite{Sutton:2018} based approaches may be effective for scenarios that the UAV only has limited prior knowledge about communication and transportation environment, since it can empirically learn the environment. Several studies have explored to learn effective UAV paths by applying \sh{RL-based approaches \cite{Zhan:2022,Zhan:2022_2,Zeng:2019_2,Khamidehi:2020,Chen:2022,Wang:2022}.} However, It is important to acknowledge that the RL-based approach may not consistently output an optimal path, and the training process involved in RL may demand considerable computing time and resources. 

In practice, it is important to consider the limited battery capacity of the UAV. There have been a few works on designing UAV trajectory performing delivery or transportation missions taking into account the limited battery capacity \cite{Sundar:2014,Coelho:2017,Fan:2023,Arafat:2022}. The work \cite{Sundar:2014} considered a variant of the travelling salesman problem (TSP) that aims to derive a shortest route visiting each target node once, while considering the limited battery capacity of the UAV and charging stations to replenish its energy. Such a UAV route optimization problem with  TSP formulation taking into account the battery limit has been extended by considering multiple UAVs \cite{Coelho:2017,Fan:2023} and grouping target nodes into clusters \cite{Arafat:2022}.  However, the problem of UAV path planning to minimize the mission time for delivery or transportation scenarios while considering both the connectivity and the battery replenishment has not been well studied. We note that for other UAV utilization scenarios, such a problem of  UAV path planning taking into account both the connectivity and the battery replenishment has been studied, but for different objectives depending on the assigned missions, e.g., minimizing the energy consumption of UAV \cite{Zeng:2019,Qi:2020} or the AoI of devices \cite{Yi:2023,Zhan:2024}. 

Our primary contribution involves proposing a path planning algorithm that outputs an optimal flight path of a cellular-enabled UAV in polynomial time complexity, to expeditiously transport a payload from an initial point to a final point, while persistently keeping the connection with a BS and complying with its battery limit. It is assumed that the UAV can establish a connection with a BS if they are closer than a certain threshold similarly as in \cite{Zhang:2019}, but we allow that the threshold can be different for each BS due to interference from other BSs. The UAV’s depleted battery can be swapped with another completely charged one at a charging station, which may involve a certain delay depending on waiting and replacing times \cite{Lee:2015}. The contributions of this paper are summarized as follows:

\begin{itemize}
\item Our problem of optimizing UAV trajectory involves determining the UAV's path, speed, and the order of visiting charging stations. We solve this problem by first reformulating the problem as a two-level graph-theoretic shortest path search problem, and then applying Dijkstra algorithm \cite{Dijkstra:1959}. More precisely, we initially search an optimal path and the corresponding maximum allowable speed for traveling between each pair of charging stations  without swapping to a fully charged battery. Subsequently, we ascertain an optimal sequence of visiting charging stations. To validate the efficacy of our approach, we analytically and numerically compare our algorithm with existing algorithms in \cite{Zhang:2019,Chen:2020} with marginal modifications, and show that only our algorithm yields an optimal solution in polynomial time.
%and show that ours surpasses those in both mission time and computational complexity.

\item Characterizing the maximum deliverable payload weight under the connectivity and battery constraints is another interesting problem. We propose a graph-theoretic  algorithm that yields an optimal solution to this problem in polynomial time. It first transforms the delivery environment into a weighted graph and finds the longest connectivity-critical edge between the initial and the final points in the graph. Then, it derives the largest payload weight which can be delivered over the edge without replacing the battery.

%\item Various numerical results are provided to show the optimal path and the corresponding delivery time according to environmental parameters and compare with the previously proposed algorithms  \cite{Zhang:2019,Chen:2020}.
\end{itemize}
The remaining of this paper is organized as follows. In Section \ref{sec2}, we present the system model and formulate the optimization problem of finding the fastest UAV route under the connectivity and battery constraints. Our proposed algorithms that output optimal UAV trajectories without and with the battery limit are presented in Sections \ref{sec3} and \ref{sec4}, respectively. In Section \ref{sec5}, other objectives of minimizing the UAV energy consumption  and of maximizing the deliverable payload weight are considered and the corresponding optimal algorithms are presented. We provide various numerical results in Section \ref{sec6}. Finally, the paper is concluded in Section \ref{sec7}.



\section{Problem Statement}\label{sec2}
We consider a cellular network with $M$ BSs and $N\leq M$ charging stations (CSs). In this network, a UAV with limited battery capacity travels from an initial point $\mathbf{U}_0$ to a final point $\mathbf{U}_F$ to deliver or transport a payload. The detailed description of the UAV model is in Section \ref{sec2A}. The UAV should maintain the connectivity with one of the BSs while delivering the payload. The BS model and the BS-UAV connectivity are described in Section \ref{sec2B}. The UAV can swap its battery at a CS if needed, as explained in Section \ref{sec2C}. This paper's goal is to plan an optimal transportation path from $\mathbf{U}_0$ to $\mathbf{U}_F$, which minimizes the mission (travel) time $T$ encompassing both the flight time and the battery swapping time at CSs. This optimization problem is formally presented in Section \ref{sec2D}. The overall model is illustrated in Fig. \ref{Fig1}.

% Figure environment removed

\subsection{UAV Model}\label{sec2A}
A UAV performs a mission to transport a payload from an initial point $\mathbf{U}_0$ to a final point $\mathbf{U}_F$ within the framework of the cellular network, under the assumption that the UAV is equipped with rotary-wings and travels at a fixed altitude $H\in[H_\mathrm{min},H_\mathrm{max}]$, where $H_\mathrm{min}$ is determined by the heights of obstacles in the network and $H_\mathrm{max}$ corresponds to the maximum allowable altitude according to government regulations. The 3-dimensional (3D) coordinates of $\mathbf{U}_0$, $\mathbf{U}_F$, and the position of the UAV at time $t$ are represented by $(x_0,y_0,H)$, $(x_F,y_F,H)$, and $(x(t),y(t),H)$, respectively, where those 2-dimensional (2D) coordinates (i.e., horizontally projected locations) are indicated as $\mathbf{u}_0=(x_0,y_0)$, $\mathbf{u}_F=(x_F,y_F)$, and $\mathbf{u}(t)=(x(t),y(t))$, respectively. While performing the mission, the UAV can adaptively adjust its speed $v(t)\triangleq\|\nabla_t\mathbf{u}(t)\|$ at time $t$ in finite set $\mathcal{V}=\{0,v_1,v_2,...,v_q\}$ ($0<v_1<...<v_q$).
In terms of energy usage, our paper concentrates solely on the energy consumed by the UAV propulsion, by the relatively marginal impact of communication power consumption \cite{Mozaffari:2019}. The overall UAV consists of the UAV body, a battery, and its payload, where those weights are denoted by $w_1$, $w_2$, and $w_3$, respectively with the total UAV weight $w\triangleq w_1+w_2+w_3$.
The power consumed by propulsion (in Watts) during flight at speed $v$ is
\begin{align}
\begin{split}\label{eq:1}
&P_\mathrm{UAV}(v)=P_1\left(1+{3({v}/{v_\mathrm{tip}}})^2\right)+P_2(w)\cdot\\
&\big(\sqrt{1\!+\!0.25{({v}/{v_0(w)}})^4}\!-\!{0.5({v}/{v_0(w)})^2}\big)^{0.5} \!\!\!\!\! +0.5S_\mathrm{FP}{\rho} v^3,
\end{split}
\end{align}
where $v_\mathrm{tip}$, $v_0(w)$, $P_1$, $P_2(w)$, $S_\mathrm{FP}$, and $\rho$ refer to the parameters established through the UAV's physical structure and delivery environment, with pointing out that only $v_0(w)$ and $P_2(w)$ rely on the total weight $w$. A comprehensive discussion regarding the propulsion power consumption \eqref{eq:1} including its parameters will be provided in Section \ref{sec6}. The UAV receives the energy for operation from its battery. The capacity (in Joules) of the battery is directly proportional to its weight $w_2$ as follows \cite{Zhang:2021_2}:
\begin{align}
C_\mathrm{batt}=\epsilon_\mathrm{batt}w_2,\label{eq:2}
\end{align}
where $\epsilon_\mathrm{batt}$ means the energy density (per unit weight) of fully charged battery. From \eqref{eq:1} and \eqref{eq:2}, the maximum flight distance under the assumption that UAV does not swap its battery and maintains a constant speed $v$ is provided as follows \cite{Zhang:2021_2}:
\begin{align}
d_\mathrm{fly}(v)=v\cdot{{\gamma\eta C_\mathrm{batt}}/({r_\mathrm{safe}P_\mathrm{UAV}(v)})},\label{eq:3}
\end{align}
where $\gamma\in(0,1)$ represents the depth of discharge, i.e., the maximum fraction of the energy that can be used in the fully charged battery, $\eta\in(0,1)$ signifies the ratio of the charged energy transferable to the UAV body under the circuit resistance, and $r_\mathrm{safe}\in(1,\infty)$ denotes the energy reserving factor of the battery for unforeseeable situations like strong winds. More precisely, the maximum flight distance $d_\mathrm{fly}(v)$ is obtained by first dividing the actual maximum available energy from the complete charged battery, ${\gamma C_\mathrm{batt}}\over r_\mathrm{safe}$ (in Joules) into the consumed power in the UAV body, ${P_\mathrm{UAV}(v)}\over{\eta}$ (in Watts) and then multiplying the speed $v$.


%\hs{We note that the consumed power in the battery is ${r_\mathrm{safe}P_\mathrm{UAV}}\over{\gamma\eta}$, where the smaller $\gamma$ and $\eta$ and the larger $r_\mathrm{safe}$ reduce the energy consumption efficiency of the UAV.}
%We note that \sh{the smaller $\gamma$ and $\eta$ and the larger $r_\mathrm{safe}$ reduce the energy consumption efficiency of the UAV.}

\subsection{BS-UAV Connectivity}\label{sec2B}
In the cellular network, the UAV can communicate with one of $M\geq 1$ BSs, where the $m$th BS for $m\in[1:M]$ is denoted by $\mathrm{BS}_m$. The 3D coordinate of $\mathrm{BS}_m$ is $(a_{m1},a_{m2},H_\mathrm{BS})$, with its 2D coordinate $\mathbf{a}_m=(a_{m1},a_{m2})$. It is assumed that the altitudes of all BSs are identical to $H_\mathrm{BS}<H$. Each BS is equipped with one omni-directional antenna and operates at the same transmission power denoted by $P_\mathrm{tx}$. We note that there exists a control station to plan the UAV trajectory and approve the handover between BSs. The control station is connected to every BS through a backhaul network as shown in Fig. \ref{Fig1} and hence it is assumed that the control station can maintain the control of the UAV if the communication rate between the UAV and its connected BS is not smaller than a certain threshold.

We assume that the channel between the UAV and a BS is determined by the line-of-sight (LoS) probabilistic model \cite{Al-Hourani:2014}. The LoS probability between the UAV and $\mathrm{BS}_m$ at time $t$, $p_m(t)\in[0,1]$ is given as follows \cite{Al-Hourani:2014}:
\begin{align}
p_m(t)={1/({1+\mu_1\exp(-\mu_2(\theta_m(t)-\mu_1))}}),\label{eq:3.1}
\end{align}
where $\mu_1>0$ and $\mu_2>0$ refer to the parameters to determine the LoS probability and $\theta_m(t)$ is the elevation angle between the UAV and $\mathrm{BS}_m$ at time $t$. We note that the LoS probability $p_m(t)$ increases as the elevation angle $\theta_m(t)$ increases.
The expected path loss between the UAV and $\mathrm{BS}_m$ at time $t$, $\Lambda_m(t)$ (in $\mathrm{dB}$) is given as follows \cite{Al-Hourani:2014_2}:
\begin{align}
\Lambda_m(t) = \mathrm{FSPL}_m(t)+p_m(t)\cdot\zeta_1 + (1-p_m(t))\cdot\zeta_2,\label{eq:4}
\end{align}
%The expected path loss between the UAV and $\mathrm{BS}_m$ at time $t$, $\Lambda_m(t)$ (in $\mathrm{dB}$) is given as $\Lambda_m(t) = \mathrm{FSPL}_m(t)+p_m(t)\cdot\zeta_1 + (1-p_m(t))\cdot\zeta_2$, where $\mathrm{FSPL}_m(t)$ and $p_m(t)\in[0,1]$ are the free space path loss and the LoS probability between the UAV and $\mathrm{BS}_m$ at time $t$, respectively,
where $\mathrm{FSPL}_m(t)$ is the free space path loss (FSPL) which only depends on the distance between the UAV and $\mathrm{BS}_m$ and $\zeta_1>0$ and $\zeta_2>\zeta_1$ refer to the excessive path losses for LoS and non-LoS (NLoS) links, respectively \cite{Al-Hourani:2014_2}. The received signal to interference plus noise ratio (SINR) from $\mathrm{BS}_m$ to the UAV at time $t$ is
\begin{align}
\mathrm{SINR}_m(t)={{P_\mathrm{tx}\cdot 10^{\Lambda_m(t)/10}}/(\!\!\!\!\!{\sum_{m'\in[1:M]\setminus m}\!\!\!\!\!I_{m'm}(t)+N_0})},\label{eq:5}
\end{align}
%The received signal to interference plus noise ratio (SINR) from $\mathrm{BS}_m$ to the UAV at time $t$ is $\mathrm{SINR}_m(t)={{P_\mathrm{tx}\cdot 10^{\Lambda_m(t)/10}}\over {\sum_{m'\in\mathcal{M}\setminus m}I_{m'm}(t)+N_0}}$, 
where $I_{m'm}(t)$ is the interference power by $\mathrm{BS}_{m'}$ at time $t$ when the UAV is communicating with $\mathrm{BS}_{m}$ and $N_0$ is the additive noise power.\footnote{Our path loss model is based on large-scale fading, i.e., small-scale fading effects are ignored. However, we note that our results are also applicable under small-scale fading  by considering the SINR averaged over the randomness.}  Note that $I_{m'm}$ would be equal to zero if $\mathrm{BS}_{m'}$ uses a different frequency band from $\mathrm{BS}_{m}$, 
and even if the two BSs use the same frequency band, it will become negligible if $\mathrm{BS}_{m'}$ is far away from the UAV.

To maintain the control of the UAV, the communication rate from a BS to the UAV should not be less than the minimum required data rate, i.e., \sh{the SINR from $\mathrm{BS}_m$ to the UAV, maximized over $m\in[1:M]$,} should satisfy
% the maximally achievable SINR of the UAV should satisfy 
\begin{align}
\max_{m\in[1:M]} \mathrm{SINR}_m(t)\geq \mathrm{SINR}_\mathrm{th}\label{eq:6}
\end{align}
for any time $t$ where $\mathrm{SINR}_\mathrm{th}$ is the hard SINR threshold to achieve the minimum required data rate. We note that the connectivity constraint \eqref{eq:6} does not mandatorily require LoS links between the UAV and BSs since the SINR value \eqref{eq:5} from a BS to the UAV is determined by averaging the pathloss over the LoS probability between them. In weak interference regime, i.e., the frequency reuse factor is sufficiently low, it can be easily checked that the condition \eqref{eq:6} can be equivalently written as  $\min_{m\in[1:M]}\|\mathbf{u}(t)-\mathbf{a}_m\|\leq d_0$ for some $d_0$, where we call $d_0$ the base coverage radius of each BS.
%\footnote{Each BS has the same base coverage radius $d_0$ since every BS has the same transmission power $P_\mathrm{tx}$ and the same altitude $H_\mathrm{BS}$, but it can be verified that our results also hold under different base coverage radii due to different transmission powers or BS altitudes.}
For other cases, however, it is generally hard to represent the exact coverage region satisfying  \eqref{eq:6} in a simple form. For tractable analysis, we introduce the coverage offset $\lambda_m\in [0,d_0]$ for $\mathrm{BS}_m$ and assume that the UAV can connect with $\mathrm{BS}_m$  if the UAV is in the effective coverage region of $\mathrm{BS}_m$ given as $\|\mathbf{u}(t)-\mathbf{a}_m\|\leq d_0-\lambda_m$, as explained in more detail in Remark \ref{rmk1}. 
Consequently, by introducing offsets $\lambda_m$ taking into account the effect of interference, we assume that \eqref{eq:6} holds if the following equation holds:%\footnote{In Section \ref{sec2B}, we only state the connectivity for downlink communications from a BS to the UAV, but we can set a similar coverage region as \eqref{eq:6} and \eqref{eq:7} for uplink communications.}
\begin{align}
\min_{m\in[1:M]}\|\mathbf{u}(t)-\mathbf{a}_m\|+\lambda_m\leq d_0.\label{eq:7}
\end{align}
Note that the connectivity condition is stated for downlink communications, but it can be defined similarly as in \eqref{eq:6} and \eqref{eq:7} for uplink communications.

\begin{remark}\label{rmk1}
Note that the coverage offset $\lambda_m$ for $m\in[1:M]$ depends on the environment around $\mathrm{BS}_m$. More specifically, $\lambda_m$ is chosen in a way that the UAV can connect with $\mathrm{BS}_m$ as long as its location $\mathbf{u}(t)$ satisfies $\|\mathbf{u}(t)-\mathbf{a}_m\|\leq d_0-\lambda_m$.  Fig. \ref{Fig2} illustrates an example of choosing $\lambda_m$ in the presence of interference. In general, we have large (small) $\lambda_m$ for urban (suburban) environments since there are many (few) other BSs around $\mathrm{BS}_m$. A higher flight altitude of the UAV induces a larger coverage offset $\lambda_m$ since the LOS probability of the channel between the UAV and other BSs increases  \cite{Lin:2019}. Also, $\lambda_m$ increases in the traffic of other BSs around $\mathrm{BS}_m$ \cite{Zhang:2021}.
\end{remark}

% Figure environment removed

\subsection{Charging Station Model}\label{sec2C}
To successfully transport the payload in the situation that the initial and the final points are far away and the battery capacity is limited, we assume that the UAV’s depleted battery can be swapped with a completely charged one at a charging station. \sh{It is assumed that all the CSs have the same height of $H_\mathrm{CS}\leq H$.} The $n$th charging station for $n\in[1:N]$ is denoted by $C_n$, and its 3D coordinate is indicated by $(c_{n1},c_{n2},H_\mathrm{CS})$ with the 2D coordinate $\mathbf{c}_n=(c_{n1},c_{n2})$, where $\mathbf{c}_n\neq \mathbf{c}_{n'}$ if $n\neq n'$. 
% It is assumed that the vertical position of all CSs are identical to $H_\mathrm{CS}\leq H$.
To expedite the battery replacement, we assume that every CS employs an automated battery swapping system, as described in \cite{Lee:2015}. %\footnote{The automated battery swapping system in \cite{Lee:2015} takes about $60$ seconds for the entire battery swapping process.} 
The total delay for battery replacement at CS $C_n$, which is denoted as $\tau_{C_n}$ and bounded by the interval $[0,\tau_\mathrm{max}]$, encompasses waiting and battery swapping times (e.g., the automated battery swapping system in \cite{Lee:2015} takes about $60$ seconds for the entire battery swapping process) and the additional penalty for UAV takeoff and landing at the CS. It is worth mentioning that the waiting time can differ in accordance with the congestion level of the charging station. Hence, the delay $\tau_{C_n}$ is contingent upon the CS index $n$. We assume that all CSs are connected to the control station via backhaul network as illustrated in Fig. \ref{Fig1}, which allows the control station to have the complete knowledge of the delay for each CS and enables it to plan an effective UAV trajectory.

\subsection{Goal}\label{sec2D}
Our goal is to plan an optimal transportation path from $\mathbf{U}_0$ to $\mathbf{U}_F$, which minimizes the mission time $T$ that encompasses both the airborne flight time and the delay for battery replacement at CSs. The formulation of the optimization problem is represented as follows:
\begin{align} 
&\textbf{Problem 1} \cr
&\mbox{Objective:~}~~~~ \min_{T\geq 0,\{\mathbf{u}(t),\ \psi(t),\ t\in[0,T]\}} T\label{eq:8}\\
&\mbox{Constraints: }\cr%\label{eq:8.1}
&\mathbf{u}(0)=\mathbf{u}_0,\ E_\mathrm{batt}(0)=C_\mathrm{batt},\ 
 \mathbf{u}(T)=\mathbf{u}_F \label{eq:9}\\  
&\mathbf{u}(t)\in\mathbb{R}^2,\ v(t)\in \mathcal{V},\ \psi(t)\in[0:N],\ t\in[0,T] \label{eq:9.1}\\
&\min_{m\in[1:M]}\|\mathbf{u}(t)-\mathbf{a}_m\|+\lambda_m\leq d_0,\ t\in[0,T]\label{eq:11}\\
&\psi(t)=0\ \mathrm{if~} \mathbf{u}(t)\not\in\{\mathbf{c}_n|n\in[1:N]\},\ t\in[0,T]\label{eq:12}\\
&\psi(t)\in\{0,n\} \ \mathrm{if~}  \mathbf{u}(t)=\mathbf{c}_n,\ n\in[1:N],\ t\in[0,T]\label{eq:13}\\
%\psi(t)\in\{0,n\} \ \mathrm{if~}  \mathbf{u}(t)\in\{\mathbf{c}_n|n\in[1:N]\},\ t\in[0,T]\label{eq:13}\\
&E_\mathrm{batt}(t)\geq (1-(\gamma/ r_\mathrm{safe}))\cdot C_\mathrm{batt},\ t\in[0,T]\label{eq:14}\\
&-\nabla_t  E_\mathrm{batt}(t)=\! P_\mathrm{UAV}(v(t))/\eta\ \mathrm{if~} \psi(t)=0,\ t\in[0,T]\label{eq:15}\\
&-\nabla_t E_\mathrm{batt}(t)=\! 0\ \mathrm{if~} \psi(t)\in [1:N],\ t\in[0,T]\label{eq:16}\\
&E_\mathrm{batt}(t)=C_\mathrm{batt}\ \mathrm{if~} \psi(t)\in[1:N] \text{ and }\cr
&~~t-\max_{t_1}\{t_1|\psi(t_1)=0,t_1\in [0,t]\}=\tau_{C_{\psi(t)}},\ \! t\in[0,T]\label{eq:17}
\end{align}
where the auxiliary indicator $\psi(t)\in[0:N]$ is used to distinguish whether the UAV is airborne $(\psi(t)=0)$ or stays at CS $C_n$ $(\psi(t)=n)$ at time $t$ and $E_\mathrm{batt}(t)\geq 0$ refers to the battery's remaining energy at time $t$. Here, \eqref{eq:9} means that the UAV departs from $\mathbf{u}_0$ with fully charged battery and arrives at $\mathbf{u}_F$ at time $T$, \eqref{eq:9.1} corresponds to the range of optimizing variables including the set $\mathcal{V}$ of possible UAV speeds, \eqref{eq:11} is the connectivity constraint in \eqref{eq:7}, \eqref{eq:12}-\eqref{eq:13} identifies whether the UAV is airborne or stays at a CS, and \eqref{eq:14} means that the actual maximum available energy in the fully charged battery is ${\gamma C_\mathrm{batt}}\over r_\mathrm{safe}$. Next, \eqref{eq:15} and \eqref{eq:16} means the power usage during flight and battery replacement at a CS, respectively, and \eqref{eq:17} signifies that the depleted battery is just swapped to fully charged one.

We point out that Problem 1 doesn't fall under convex optimization due to the variable $\psi(t)\in[1:N]$ with discrete codomain and the non-convex constraint \eqref{eq:11}. Furthermore, optimizing $\mathbf{u}(t)$ and $\psi(t)$ over continuous $t\in[0,T]$ adds to the non-triviality of the problem. To address the challenges, in Sections \ref{sec3} and \ref{sec4}, we initially reframe the problem within a weighted graph methodology. Subsequently, we analytically show that leveraging the Dijkstra algorithm \cite{Dijkstra:1959} facilitates solving the problem efficiently in polynomial time.


% Chapter 1

\chapter{State-of-the-Art in Biomedical Question Answering} % Main chapter title

\label{Chapter3} % For referencing the chapter elsewhere, use \ref{Chapter1}
\setcounter{secnumdepth}{4}
\setcounter{minitocdepth}{2}
\minitoc

This chapter presents background information to contextualise the topics of interest of this thesis work. In section~\ref{Chapter3_1}, a brief introduction is provided to question answering and its typical pipeline. In section~\ref{Chapter3_4}, the main characteristics of question answering in the biomedical domain are shown. In section~\ref{Chapter3_5}, the main resources that can be exploited for QA in the biomedical domain are presented. In section~\ref{Chapter3_6}, we review the related work to question answering with a particular focus on the biomedical domain. In section~\ref{Chapter3_8}, a synthesis is presented. Finally, the evaluation metrics we use for the assessment of the proposed methods for question answering in the biomedical domain are described in section~\ref{Chapter3_7}.


%----------------------------------------------------------------------------------------

% Define some commands to keep the formatting separated from the content
%\newcommand{\keyword}[1]{\textbf{#1}}
%\newcommand{\tabhead}[1]{\textbf{#1}}
%\newcommand{\code}[1]{\texttt{#1}}
%\newcommand{\file}[1]{\texttt{\bfseries#1}}
%\newcommand{\option}[1]{\texttt{\itshape#1}}

%----------------------------------------------------------------------------------------
\section{Introduction}
\label{Chapter3_1}

The search for short specific answers to questions written in natural language is one of the major challenges in the field of information retrieval. The question answering issue has been addressed by the artificial intelligence community in the literature since the 1960s. Early experiments in this direction implemented question answering systems based on knowledge bases written manually by experts in their own field of interest, such as BASEBALL \citep{Green_1961}, LUNAR \citep{Woods_1973} that operate in open domain. The BASEBALL system answers questions posed about the dates, locations or results of Baseball games played in the American league over one season. LUNAR which is one of the first sciences question answering system, was designed to assist the geologic analysis of stones returned by the Apollo mission.

Question answering (QA), unlike traditional IR aims at directly providing the short, precise answers to questions asked by inquirers in natural language, by employing complex IE and NLP techniques, instead of returning a large number of documents that are potentially relevant \citep{voorhees1999trec}. QA systems aim at directly producing and providing precise answers rather than entire documents to users questions, by automatically analyzing thousands of articles, ideally in less than a few seconds. Such systems, which can help users locate useful information quickly, need linguistically and semantically processing of both users questions and data sources in order to extract the relevant information. In particular, question answering differs from traditional IR in three main aspects: (1) information needs (i.e., queries) are expressed  using natural language instead of a set of keywords; (2) results depend directly to what has been specifically requested, be it a precise answer (e.g., a named entity) or a short summary (3) answers are generated based on the integration of unstructured and structured data, e.g., textual documents and knowledge bases, respectively. While the intended answer is usually a piece of text, in few cases, the format of the answer may also be a multimedia information.


While the early research works on QA in the domain of artificial intelligence (AI) dates far back to the 1960s, several investigations involving QA within IR and IE communities have been made by the introduction of the QA Track in Text REtrieval Conference (TREC\footnote{TREC: \url{http://trec.nist.gov/}}) evaluations in 1999 \citep{voorhees1999trec}. Since then, methods and tools have been developed for generating and extracting answers for the three types of natural language questions supported by TREC fora, including, factoid questions, list questions, and definitional questions. Most research works in the field of QA, as fostered by TREC and other similar evaluation fora such as NII-NACSIS Test Collection for IR systems (NTCIR\footnote{NTCIR: \url{http://research.nii.ac.jp/ntcir/index-en.html}}) \citep{kando2002overview} and Cross Language Evaluation Forum (CLEF\footnote{CLEF: \url{http://clef.isti.cnr.it/}}) \citep{Magnini_2005}, have so far been focused on open-domain QA.


Due to the continuous, enormous growth of textual data especially in the form of scientific articles in the biomedical domain, recently, the field of QA systems in the biomedical domain has witnessed a growing interest among researchers. Such systems can quickly and reliably assimilate relevant information from a multitude of biomedical textual resources. Furthermore, question answering is a text mining task directed towards aiding researchers and health care professionals in managing the exponential growth of textual information in the biomedical domain.


Natural language QA systems are typically composed of four main components including (1) question analysis and classification, (2) information or document retrieval, (3) passage retrieval and (4) answer extraction, as shown in Figure~\ref{fig:QAP} \citep{HIRSCHMAN_2001}.


% Figure environment removed

The input to a QA system is a natural language question. Based on the linguistic and semantic  processing of the question, question analysis and classification identifies the type of question being posed to the system, and the type of the expected answer it should generate. Often, there may be more processes involved at this phase, such as named entity recognition and relation extraction. Output from this stage is one or more features of the question for use in subsequent stages. It then constructs a query from the input question to be fed into the third component, information retrieval. In the information retrieval component, the system inputs the query into a document retrieval engine so as to find a set of relevant documents satisfying the query. Out of the retrieved and selected documents, the passage retrieval component consists of retrieving relevant passages (text segments). The retrieved passages set may be narrowed down to a smaller set of most relevant passages/snippets of text, which constitute the input for the last component, answer extraction. Finally, in the answer extraction component, based on the type of the question and the expected answer format generated by the question analysis and classification component, the final answer(s) are extracted and generated from the candidate answers passages selected by the passage retrieval component. The output of the question answering system consists of the top ranked answer(s), associated with the raw text from which the answers were extracted.


The biomedical-domain QA is one of the more popular restricted-domains QA. The following section~\ref{Chapter3_4} provides the main characteristics that distinguish QA in the biomedical domain from open-domain QA.




\section{Characteristics of Question Answering in the Biomedical Domain}
\label{Chapter3_4}

Several characteristics distinguish the biomedical domain as an application domain for QA from general, open-domain and other restricted-domains QA. Based on the three factors discussed in the previous section, namely, the size of the data, domain-dependent context, and domain-specific resources, \citet{athenikos2010biomedical} point out in their review of biomedical QA the characteristic features of QA in the biomedical domain. These characteristics include:

\begin{enumerate}
  \item Large-sized corpora.
  \item Highly complex domain-specific terminology.
  \item Domain-specific lexical, terminological, and ontological resources.
  \item Tools and methods for exploiting the semantic information embedded in the above resources.
  \item Domain-specific format and typology of questions.
\end{enumerate}

First, biomedical QA is characterized and challenged by the large-sized of unstructured information especially in the form of scientific articles. Retrieving the accurate answers to potential questions from a multitude of biomedical sources tends to be quit complex. Second, the prominent use of highly complex domain-specific terminology is both an advantage and a challenge for biomedical QA. In general, the amount of terminological variations and synonyms make biomedical text mining relatively difficult. However, biomedical QA may take advantage of the particularity and limited scope of natural language questions that a domain-specific terminology gives. Third, domains that are specific, limited, and complex are most likely to have available resources such as methasaurus and ontologies that can be used in the QA process. These resources are generally developed to help biomedical domain users to retrieve information using text mining applications as well as categorize and group the biomedical knowledge. In addition to available resources, the tools and techniques required for employing the semantic information they contain enable for deep question and answer processing. Lastly, biomedical QA may benefit from the domain-specific formatting and typology of questions so as to use and develop answer processing techniques strategies for each specific question type.


Due to the importance and unique characteristics of QA in the biomedical domain, recently, several research works have been presented in the different stages of QA. These works which include the important methods and techniques are reviewed and discussed in the section~\ref{Chapter3_6}.

\section{Resources for Biomedical Question Answering}
\label{Chapter3_5}

While the biomedical domain poses a notable challenge for answering natural language questions, with highly complex domain-specific terminology and the enormous growth of textual information especially in the form of scientific articles and electronic health records, it also provides various resources that can be very beneficial for QA, as Zweigenbaum \citep{zweigenbaum2003question} also notes in his overview on QA in biomedicine. Here we review some of the well-known and relevant resources for QA in the biomedical domain.

\subsection{Corpora}

The primary resource for text-based QA in the biomedical domain is obviously text. In this context, MEDLINE was the first and still remains the primary resource for biomedical QA and in general for biomedical text mining. The MEDLINE\footnote{MEDLINE: \url{https://www.nlm.nih.gov/pubs/factsheets/medline.html}} database  maintained by the U.S. National Library of Medicine (NLM), and contains citations to journal articles in the life sciences with a particular focus on biomedicine. The 2016 MEDLINE contains over 24 million bibliographic references published from 1946 to the present in over 5,623 worldwide scientific journals. Indeed, hundreds of thousands of references are added to the database each year. For example, more than 806 thousand were added in 2016.

Perhaps more importantly, abstracts and sometimes full text of MEDLINE citations can be downloaded using the Entrez Programming Utilities\footnote{Entrez Programming Utilities (E-Utilities), Encyclopedia of Genetics, Genomics, Proteomics and Informatics: \url{https://doi.org/10.1007/978-1-4020-6754-9_5383}} for text mining purposes. For instance, PubMed\footnote{PubMed: \url{https://www.ncbi.nlm.nih.gov/pubmed/}}, a free service provided by the NLM under the U.S. National Institutes of Health (NIH) and accessible through the National Center for Biotechnology Information (NCBI), gives access to MEDLINE for searching abstracts of biomedical literature. The result of a PubMed search is a list of references (including authors, title, source, and often an abstract) to journal articles and an indication of free full-text availability. PubMed Central\footnote{PubMed Central: \url{https://www.ncbi.nlm.nih.gov/pmc/}} is a full-text archive of biomedical and life science articles, maintained by the NIH. Alternatively, subsets of MEDLINE documents can be obtained from the archives of individual research teams that share their annotated collections, and community-wide large-scale evaluations that use MEDLINE citations. For example, the OHSUMED\footnote{OHSUMED Test Collection. Available at: \url{http://trec.nist.gov/data/t9_filtering/}} collection contains all MEDLINE citations in 270 biomedical journals published over a five-year period (1987–1991) as well as a more recent collection provided in TREC Genomics\footnote{TREC genomics track data. Available at: \url{http://ir.ohsu.edu/genomics/data.html}} Track that contains ten years of MEDLINE documents (1994–2003).



Also available in the biomedical domain is topically-annotated collections of MEDLINE abstracts such as the GENIA corpus \citep{Kim_2003}, the earlier BioCreAtIve collections \citep{Hirschman_2005,Krallinger_2008,Islamaj_Dogan_2017}, and more recent set of the BioASQ challenge \citep{tsatsaronis2012bioasq}. The GENIA corpus which contains 1999 MEDLINE abstracts, is syntactically and semantically annotated for part-of-speech, coreference, biomedical concepts and events, cellular localization, and protein reactions. The BioCreAtIve data sets derived from the BioCreAtIve challenge which concerned with the extraction of biologically significant entities names and useful information (e.g., protein - functional term associations) from the literature. The BioASQ collections provided by the BioASQ challenges contains standard data sets for evaluating semantic indexing, question answering and information extraction systems in the biomedical domain. The BioASQ data sets are currently the most thoroughly annotated collection of MEDLINE abstracts and biomedical questions. The BioASQ challenges include biomedical text mining tasks relevant to text classification, information retrieval, question answering from texts and structured data, multi-document summarization and many other areas.

\subsection{Lexical, terminological, and ontological resources}

Due to the need of structuring highly complex domain specific terminology and of making it machine-understandable, text mining community in the biomedical domain has constructed and developed a set of lexical, terminological, and ontological resources. The Unified Medical Language System (UMLS\footnote{UMLS: \url{https://www.nlm.nih.gov/research/umls/}}) \citep{lindberg1993unified,Bodenreider_2004}, a repository of biomedical vocabularies that is maintained by NLM, is the most important resource. The UMLS includes three knowledge resources: the Mtathesaurus, the Semantic Network, and the SPECIALIST Lexicon. The UMLS Metathesaurus is a comprehensive, multilingual biomedical vocabulary thesaurus that contains information on biomedical concepts, their various names, and the hierarchical and transversal relations between them. The latest version of the Metathesaurus (UMLS 2017AB) contains approximately 3.64 million concepts and 13.9 million unique concept names from 201 different source vocabularies including dictionaries, terminologies, and ontologies. The Semantic Network contains information about the set of semantic types, or broad subject categories, and set of useful and important relationships, or semantic relations that may hold between these semantic types. The semantic types provides a consistent categorization of all concepts in the Metathesaurus by grouping these concepts according to the semantic types that have been assigned to them. The Semantic Network in the UMLS 2017AB version contains 127 semantic types and 54 semantic relationships. The SPECIALIST Lexicon provides the lexical, syntactic, and morphological information needed for the SPECIALIST NLP system. It contains commonly English words and biomedical vocabulary.

Medical Subject Headings (MeSH\footnote{MeSH: \url{https://www.nlm.nih.gov/mesh/}}), the controlled vocabulary thesaurus that is maintained and manually updated every year by NLM, consists of medical subject headings in a hierarchical tree which allows search at several levels of specificity. The latest version of the MeSH thesaurus (MeSH 2018) contains 28.939 MeSH descriptors, 116.909 total descriptor terms, and 244.154 supplementary concept records. Mesh is the widely used for the purpose of indexing journal citations for databases in the biomedical domain. For example, it is used by the U.S. NLM to index biomedical articles for the MEDLINE/PubMed database as well as for the cataloging of books and documents acquired by the library. Overall, the U.S. NLM provides over 297 knowledge sources\footnote{NLM resources: \url{https://eresources.nlm.nih.gov/nlm_eresources/}} and tools supporting biomedical text mining applications.

Systematized Nomenclature of Medicine Clinical Terms (SNOMED CT) \citep{stearns2001snomed}, the most comprehensive and precise clinical terminology available in the world that is originally maintained by the College of American Pathologists, aims at encoding the meanings that are used in clinical information and to support the efficient recording of clinical records. It was formed by the merger of two large health care reference terminologies, SNOMED Reference Terminology (SNOMED RT) and the U.K. National Health Service Clinical Terms, and it is accessible through the U.S. NLM and the National Cancer Institute (NCI). SNOMED CT is one of a suit of specified standards for use in U.S. healthcare information systems for the electronic exchange of clinical health information. The latest version of SNOMED CT released in January 2017 contains over 326.734 active concepts.

Gene Ontology\footnote{Gene Ontology: \url{http://www.geneontology.org/}} \citep{ashburner2000gene}, a controlled vocabularies of concepts used to define gene function, and semantic relationships between concepts, is an ontology designed to structure the description of genes and genes products common to all species. It classifies gene functions along three aspects: (1) molecular function, the molecular activities of a gene product, such as binding or catalysis; (2) cellular component, where a gene product is active; and (3) biological process, pathways and operations made up of the elemental activities of multiple gene products.

Other sets of interoperable ontologies in the biomedical domain are developed and maintained by collaborative effort in the Open Biological and Biomedical Ontology (OBO) Foundry\footnote{Open Biomedical Ontologies: \url{http://www.obofoundry.org/}} and the National Center for Biomedical Ontology (NCBO\footnote{National Center for Biomedical Ontology: \url{https://www.bioontology.org/}}). The set of NCBO ontologies are accessed through BioPortal\footnote{NCBO BioPortal: \url{http://bioportal.bioontology.org/}}. Other important centers that develop specialized resources for text mining applications in the biomedical domain include the U.K. National Centre for Text Mining (NaCTeM\footnote{NaCTeM: \url{http://www.nactem.ac.uk/}}) and the European Bioinformatics Institute (EMBL-EBI\footnote{EMBL-EBI: \url{https://www.ebi.ac.uk/}}).


\subsection{Supporting tools}


Besides the various available terminological, lexical and ontological resources in the biomedical domain that provide the lexical semantic information needed for QA and other text mining applications, the biomedical domain also has supporting tools that are specifically designed and developed to help exploit the semantic information contained in those aforementioned resources, such as biomedical entities and semantic relations, etc. Here, we define the most widely used tools to identify biomedical entities and semantic relationships among them in the text.

MetaMap \citep{aronson2001effective} that is based upon UMLS, is the most widely used tool by researchers working on biomedical text mining, including biomedical QA researchers for the named entity recognition task. MetaMap is highly configurable NLP application developed at NLM that maps terms in free biomedical text to UMLS Metathesaurus or, equivalently, identifies UMLS Metathesaurus concepts referred to in text. MetaMap provides a wide variety of configuration options to give researchers the opportunity to choose the best configuration for a given task. MetaMap, which is an open source tool and available for download\footnote{MetaMap: \url{https://metamap.nlm.nih.gov/}}, it requires a UMLS Terminology Services (UTS) account.

SemRep \citep{Rindflesch_2003} that is also relied on UMLS, is often used by researchers and practitioners to extract UMLS Semantic Network semantic relationships between UMLS concepts from sentences in biomedical text. SempRep\footnote{SemRep: \url{https://semrep.nlm.nih.gov/}} developed at the U.S. NLM to map sentences in free biomedical text to UMLS Semantic Network in order to extract three-part propositions, called semantic predications. Each predication consists of a subject argument, an object argument, and the semantic relation that binds them. The subject and object arguments of each predication are UMLS Metathesaurus concepts and their binding relationship is a UMLS Semantic Network semantic relationship.


\section{Related Work on Biomedical Question Answering}
\label{Chapter3_6}

Due to the unique characteristics of biomedical as an application domain for QA, recently researchers have proposed several specific methods and techniques in the different processing stages of QA system so as to produce more precise responses. As shown previously in section~\ref{Chapter3_1}, QA systems typically incorporate several components, including question classification, document retrieval, passage retrieval, and answer extraction each of which has to deal with specific challenges and issues. For each QA component, its basic challenges and approaches are outlined and recent and influential methods are reviewed in the following subsections.

\subsection{Question classification}
\label{Chapter3_6_1}
Question classification is usually the first component in QA pipeline as the first step towards developing biomedical QA systems is processing and classifying the question in order to identify the question type and therefore the answer type to produce. Several researchers in the biomedical domain have investigated question classification as a means of analyzing and filtering biomedical questions \citep{Simpson_2012}. In general, question classification may include several tasks since the questions may be classified along many dimensions: question type \citep{cruchet2008supervised,roberts2014automatically}, answer type \citep{McRoy_2016}, topic \citep{kobayashi2006representing, yu2008automatically,Cao_2010}, user \citep{Liu_2011,Roberts_2016}, answerability \citep{yu2005being}, resource \citep{roberts2016resource}, etc. Question classification essentially covers all these classification systems, and thus one could have multiple question classifiers to classify a single question along many dimensions. The most common classification in the biomedical domain is question type and answer type, sometimes referred to as the expected answer format and semantic answer type, respectively. For example, the question type and the semantic type of the expected answer for the biomedical question ``What is the treatment of acute myocarditis?'' are ``factoid'' and ``treatment'', respectively.

The purpose of question type classification in biomedical QA is to determine the type of question and therefore identify the expected answer format, to check whether the answer should be a biomedical entity name, a short summary, ``yes'' or ``no'', etc. Importantly, to produce the answer to a given question, the QA systems which deal with more than two types of questions, should know in advance the expected answer format that allows using specific answer extraction strategies. The identification of question types in biomedical QA systems is a very important task as it may strongly affect positively or negatively further processing stages: if the question type is not identified correctly, further QA processing stages will inevitably fail too. This task is usually carried out in open-domain QA systems by taking into account the ``Wh-'' particle and predefined patterns. However, the complexity of natural language creates many challenges to this task in the biomedical domain. As some questions that appear, at first, to belong to a certain type can result to be of a different one. For instance, the biomedical question ``Where is the protein CLIC1 localized?'', from the BioASQ training questions, is a factoid question. Although the way in which it is constructed, to start with ``where'', may lead it to be a summary question which is incorrect.

On the other hand, the answer type is the semantic type of the expected answer, and is useful in filtering out irrelevant answer candidates. The answer type has been called commonly topic or semantic type of the expected answer. The identification of answer type to a given question in biomedical QA, which is one of the most common task of question classification studied in the literature, has a critical impact on the overall performance of biomedical QA systems as is useful in choosing the best resource from which the answer should be extracted and candidate answer selection. For example, the question ``What is the best way to catch up on the diphtheria-pertussistetanus vaccine (DPT) after a lapse in the schedule?'' from National Library of Medicine's data set represents a pharmacological question, and the QA system may therefore identify the Micromedex pharmacological database as the resource to produce the answers.

The approaches for question classification may be classified in rule-based approach and machine learning-based approach. The former tries to
match the question with some predefined rules, while the latter can automatically classify questions into categories, by extracting some features from them. Many studies in open-domain QA have used rule-based approach \citep{Khoury_2011,Biswas_2014}. Otherwise, machine learning approaches have improved solution for open-domain question classification. One of the biggest advantages of machine learning is that one can focus on designing insightful features, and rely on learning process to efficiently cope with the features. For example, \citet{li2002learning} have presented a hierarchical classifier based on the sparse network of winnows for open-domain question classification. Two classifiers were involved in this work: the first one classifies questions into coarse categories; and the other one into fine categories. \citet{yu2005modified} have improved the bayesian model by applying the TFIDF measure to deal with the weight of words for Chinese question classification. \citet{xu2006syntactic} have employed affiliated ingredients as features of the learned model and used the results obtained by the syntactic analysis for extracting the question word. \citet{li2008classifying} have classified open-domain ``what'' type of questions into proper semantic categories using conditional random fields (CRFs). CRFs has been used to label all words in a question, and then choose the label of head noun as question category. \citet{Xu_2012} have introduced a question classification method based on the SVM classifier for QA system in the tourism domain. Other methodologies take benefit of both rule-based and machine-leaning approaches such as the work described in \citep{Hao_2017}.


Although many solutions have been proposed for question classification tasks in open-domain QA, only a few works have been completed in the biomedical domain which have attempted to define the information needs of physicians. For example, for searching the best available evidences supporting responses to clinical questions, the evidence-based medicine (EBM) paradigm recommends questions be organized according to the PICO (Problem or Patient/Population, Intervention, Comparison, Outcome) format \citep{richardson1995well}. In addition to the specific frame, taxonomies of biomedical questions in the EBM framework have also been developed.  \citet{Ely429} have proposed a taxonomy of generic types of clinical questions asked by doctors about patient care so as to help answer such questions. The taxonomy contains the 10 most frequent question topics (e.g., diagnosis, treatment, management, etc.) among 1396 collected clinical questions (e.g., ``What is the drug of choice for condition x?''). \citet{Bergus_2000} proposed a taxonomy of medical questions according to the PICO representation of questions and the categories of clinical tasks involved in the questions. \cite{Ely710} developed an evidence taxonomy to deal with obstacles be faced when attempting to answer physicians' questions with evidence. The authors have classified the collected 1101 questions from 103 family doctors into clinical vs. non-clinical. The clinical questions are divided into general vs. specific. The general questions can be classified into evidence vs. no evidence. The evidence questions are categorised into intervention vs. no intervention categories. \citet{jacquemart2003towards} have presented a taxonomy that consists of 8 broad semantic models (e.g., [which X]–(r)–[B]) for categorizing clinical questions in a French-language medical QA system. In this context, the recent taxonomy of biomedical questions which is created by the BioASQ challenge \citep{tsatsaronis2012bioasq} consists of four types of questions that may cover all kinds of potential questions: yes/no questions, factoid questions, list questions, and summary questions. \citet{seol2004scenario} identified four types of questions including treatment, diagnosis, etiology, and prognosis.
\citet{huang2006evaluation} presented a manual classification of clinical questions asked in natural language as a mean to examine the adequacy and suitability of the PICO framework. The authors have reaffirmed the usefulness of the PICO framework for structuring clinical questions, but also they found that it is less suitable for clinical questions that do not implicate therapy elements. \citet{niu2003answering,niu2004analysis,niu2005analysis,niu2006using} proposed a PICO-based question analysis approach within the EpoCare project. Their methods which aim at automatically answering questions from clinical evidence, extract potential answers using the PICO framework to structure both the question and answer texts. Similarly, \citet{Demner_Fushman_2006, Demner_Fushman_2007} used the PICO framework in their proposal approach to clinical QA. Their methods use the semantic unification of the PICO frame of the query and that of candidate answers. \citet{abacha2015means} developed a medical QA system named MEANS which consists of (1) corpora annotation, (2) question analysis and classification, and (3) answer search. The question classification task in the MEANS system aims at classifying the given medical questions into three types: definition, yes/no, and factoid so as to determine the type of the expected answers (e.g., ``yes'' or ``no'', biomedical entity name, etc.). Their method is based on a set of patterns that were manually constructed by analysing the 100 medical questions. \citet{yu2005classifying} performed machine learning approaches to classify medical questions based on  Ely et al.' s taxonomy into topics. They have shown that using the question taxonomy with the SVM classifier leads to the highest performance. \citet{cruchet2008supervised} described a question classification method to identify the answer types of questions in a bilingual French/English QA system which is adopted to health domain. The authors who studied a small number of English and French medical questions, used support vector machine (SVM) for classifying these questions into several categories (e.g. symptoms, prevention, evolution, treatment, etc.). \citet{Cao_2010} proposed a SVM classifier-based method to classify clinical questions asked by physicians in natural language into predefined general topics (e.g., diagnosis, management, pharmacological, treatment, etc.). The authors incorporated several features of clincal questions for the SVM classifier such as words, bigrams, stemming, UMLS concepts and semantic types. Since one question can be assigned to multiple topics, a binary SVM classifier was adopted for each of the 12 topics. \citet{Patrick_2012} proposed a question classification method for answering clinical questions applied to electronic patient notes. They first collected a set of clinical questions from staff in an Intensive Care Unit. They then designed a clinical question taxonomy for question and answering purposes. Finally, they built a multilayer classification model to classify the clinical questions. \citet{roberts2014automatically} introduced a multi-class SVM classifier-based method to automatically classify consumer health questions into semantic types (e.g., anatomy, cause, diagnosis, etc.) for the purpose of supporting automatic retrieval of medical answers from consumer health resources. More recently, deep learning models such as recurrent neural Network (RNN) and LSTM have been emerging as state-of-the-art for sequence modeling, particularly for text classification. Deep learning classification methods are multi-layer networks, which have been introduced for both feature extraction and classification tasks. The most well-known deep learning methods are the convolutional neural networks (CNN), which have known a great success since their introduction by \citep{lecun1989backpropagation}. Many deep learning models have been proposed such as RNN, recursive neural network and LSTM. Such models produce a more robust predictor than traditional machine learning methods. However, they  require a large number of training instances for the training phase.


Beyond these types of question classification in the biomedical domain, other approaches have studied (a) question answerability to separate answerable questions from unanswerable ones  \citep{yu2005classifying,yu2005being}, (b) resource identification to determine the resource type of biomedical questions \citep{roberts2016resource}, and (c)  relation extraction \citep{hristovski2015biomedical} using SemRep \citep{Rindflesch_2003}.

Although several question classification methods have been proposed in biomedical QA, question classification still requires further efforts in order to improve its performance. For instance, existing solutions for biomedical question classification have so far focused on extracting syntactic and semantic features from questions and using machine learning algorithms so as to classify questions into different topics. However, they do not take into account the syntactic dependency relations in questions. Intuitively, the incorporation of syntactically related pairs into other features may provide the best description and representation of questions. The motivation to find alternative features for machine-learning algorithms is the fact that words by themselves cannot capture the gist of a clinical question.

Another challenging issue in question classification is the identification of the types and formats of potential questions and intended answers, respectively. Note only current biomedical QA systems have limitations in terms of the types and formats of questions and answers that they can process, but also in most such systems which dealt with more than one type of questions, the users have to give or select manually the question type to each given question. As the ultimate goal of biomedical QA systems is to be able to deal with a variety of natural language questions and to generate appropriate natural language answers, biomedical question type classification is a necessary task needs so as to automatically identify the type of question and therefore to see whether the answer should be a biomedical entity name, a short summary, ``yes'' or ``no'', etc.

In the next chapter~\ref{Chapter4} we will present in details the proposed machine learning based methods for question classification in biomedical QA. The first method consists at identifying the type (i.e., yes/no, factoid, list and summary questions) of a given biomedical question in order to determine the expected answer format. It is based on our predefined set of handcrafted lexico-syntactic patterns and machine learning algorithms. The second method, which is based on lexical, syntactic and semantic features for machine learning algorithms, allows classifying questions into topics in order to filter out irrelevant answer candidates.


\subsection{Document retrieval}
\label{Chapter3_6_2}

Document retrieval is usually the second component in a typical QA pipeline as the second step towards answering a biomedical question posed in natural language is retrieving the set of textual documents that are likely to contain the answer. Retrieval of a set of relevant documents to a given query that is constructed from the question is usually carried out based on an existing IR system. The retrieved textual document set is often narrowed down to a smaller set of most relevant documents which constitutes the input for further QA processing steps. In particular, document retrieval is one of the significant components that serves as the building block of efforts since the correct answers can be only found when the set of textual documents from which the QA system extracts the answers is retrieved correctly \citep{Monz_2003,Collins_Thompson_2004,athenikos2010biomedical,neves2015question}.

While many efforts and investigations have been made in this direction in the open domain, initiated by the QA Track in TREC evaluations which takes place regularly every year since 1999 \citep{voorhees1999trec,roberts2002information,Monz_2003,gaizauskas2004information,Collins_Thompson_2004,voorhees2005trec,Teufel}, it still remains a real challenge in the biomedical domain due to a variety of reasons. The most basic obstacle is that there exists a vast amount of textual data especially in the form of scientific articles that is constantly and rapidly increasing as new scientific articles are published every day in the biomedical domain. Furthermore, not only these data are generally expressed in natural language, which makes its automated processing more difficult and complex, but also terminological variations and synonyms make document retrieval difficult in general for the biomedical domain. Thus, an efficient access to relevant information is a challenging task. Although biomedical text mining is challenged by a prominent use of domain-specific terminology, document retrieval may benefit from the specificity and limited scope of textual documents and the potential queries that a domain-specific terminology provides. Due to the specific characteristics of biomedical as an application domain for document retrieval, recently, most proposed systems in some way make use of domain-specific semantic knowledge, such as the MeSH thesaurus, for document retrieval. For example, the NLM indexers use the MeSH descriptors to index MEDLINE citations for PubMed search engine, a well-known information retrieval system in the biomedical domain which comprises more than 24 million citations for biomedical literature from MEDLINE, life science journals, and online books. \cite{lee2006beyond,Yu_2007} described a semantic-based approach for the development of a medical QA system named MedQA. The MedQA system is built upon four stages, namely, question analysis, information retrieval, answer extraction, and summarization techniques to automatically generate paragraph-level answers to definitional questions from the MEDLINE citations and World-Wide-Web. For information retrieval, the authors first applied LT CHUNK \citep{Mikheev_1996} to identify noun phrases from medical questions and used them as the query terms to retrieve relevant documents from the MEDLINE collection. They then used the tool LUCENE\footnote{LUCENE: \url{http://lucene.apache.org/core/}} \citep{goetz2000lucene} to index the MEDLINE documents and applied the vector-space model (VSM), a TFIDF based cosine similarity model for computing the relevance of each document to a query. To retrieve definitions from World-Wide-Web, the authors used Google and the TFIDF metric. \citet{Cao_2011} integrated the latest version of the probabilistic relevance model BM25 \citep{Robertson_2004} within their developed clinical QA system AskHERMES for the document retrieval stage, as it proved to be the best performing retrieval model for tasks such as those at the recent TREC. \cite{abacha2015means} applied NLP methods to process the source documents used to extract the answers to
natural language questions in their developed medical QA system named MEANS. The authors exploited NLP techniques, such as named entity recognition and relation extraction, and domain-specific semantic knowledge (e.g., UMLS Methasaurus and UMLS Semantic Network) to build RDF annotations of the source documents and SPARQL queries representing the users questions.

More recently, there has been a growing interest among researchers in developing new methods and techniques to improve the performance of document retrieval in biomedical QA  with the introduction of the biomedical QA Track at the BioASQ\footnote{BioASQ Challenge: \url{http://www.bioasq.org/}} challenge \citep{tsatsaronis2012bioasq} which takes place regularly since 2013. The BioASQ challenge is an EU-funded project aiming at fostering research and solutions on both the biomedical QA and large-scale online semantic indexing areas. The BioASQ challenge comprised three tasks: (1) Task a: on large-scale online biomedical semantic indexing, (2) Task b: on biomedical semantic QA, and (3) Task c: on funding information extraction from biomedical literature. The goal of Task b is to assess the performance of QA systems in different stages of the QA process. It is sub-divided into two phases: phase A and phase B. In phase A participants had to respond with biomedical concepts, relevant documents, relevant passages, and RDF triples. In phase B participants were asked to answer with exact answers and ideal answers (paragraph-sized summaries). \cite{weissenborn2013answering} developed a biomedical QA system which is composed of three main stages, namely, question analysis, document retrieval, and answer extraction to answer factoid questions. Documents relevant to potential questions are retrieved by making keyword queries to the GoPubMed search engine, which searches the MEDLINE biomedical citations database. Similarly, \cite{neves2014hpi} proposed a biomedical QA system which consists of two components, namely, question processing and document processing. Documents relevant to biomedical questions are retrieved by making queries to the GoPubMed search engine. Queries were constructed using the Stanford CoreNLP tools\footnote{Stanford CoreNLP package: \url{https://nlp.stanford.edu/software/corenlp.shtml}} for sentence splitting, tokenization, part-of-speech tagging and chunking. Indeed, queries was generated from the question based on both the terms and the chunks. The authors also performed query expansion using synonyms obtained using services from BioPortal. \cite{mao2014ncbi} used PubMed search functions for retrieving relevant documents to a given question in biomedical QA. Given a search query, the authors used PubMed results-ranking options such as by date or by relevance. \cite{choi2014classification} described a biomedical document retrieval approach based on semantic concept-enriched dependence model and sequential dependence model. The concept-enriched dependence model incorporates UMLS Methasaurus concepts identified using MetMap, while the sequential dependence model incorporates sequential query term dependence into the retrieval model. The authors used the Indri IR system \citep{strohman2005indri} and 2014 MEDLINE citations which composed of roughly 22 million citations.

Although previous document retrieval methods have proven to be quite successful at retrieving relevant documents in biomedical QA, document retrieval still require further efforts in order to improve its performance. One of the main observations that can be made about existing systems is that the task of document retrieval often set a framework in which an existing biomedical IR system is used, and completely depended on its ranking of documents. Indeed, there are many cases where the search engine mistakenly returns irrelevant citations high in the set or relevant citations low in the set. This problem is certainly a challenging issue as a biomedical QA system usually extracts the answers from the top-ranked documents.

In the chapter~\ref{Chapter5} we will present in details the proposed document retrieval method in biomedical QA.

\subsection{Passage retrieval}
\label{Chapter3_6_3}
Passage retrieval is usually the third component in a typical QA pipeline as the third step towards answering a biomedical question is processing the set of relevant documents so as to identify a set of text segments (also known as passages or snippets) that are likely to contain the answer. In this stage, which is known as passage or snippet retrieval, a set of passages are extracted from the retrieved and selected documents. In particular, a passage retrieval system may be defined as a specialized type of IR application that retrieves a set of passages/snippets rather than providing a whole ranked set of documents \citep{Buscaldi_2009}. Its main purpose in a QA system is to retrieve and return top-ranked passages which serve as answer candidates and the QA system extracts and selects the answers from them. Although a challenging task itself, passage retrieval remains one of the most important modules for the development of a QA system as the overall performance of a QA system heavily depends on the effectiveness of the integrated passage retrieval component: if a passage retrieval system fails to find any relevant passage for a given question, further processing steps to extract an answer will inevitably fail too. In this context, several studies such as the one reported in \citep{Otterbacher_2009}, highlighted that the correct answer to a given question can be found with high precision when it already exists in one of the retrieved passages.

Although passage retrieval in open-domain QA is a well-studied research area \citep{Clarke_2003,Otterbacher_2009,Ryu_2014,Saneifar_2014,Othman_2016}, it still remains a real challenge in biomedical QA. As described in section~\ref{Chapter3_4}, QA in the biomedical domain has its own characteristics such as the presence of large-sized corpora, complex technical terms, compound words, and domain-specific format and typology of questions. Due to the unique characteristics of biomedical as an application domain for passage retrieval in QA, recently, researchers have increasingly sought to incorporate and use lexical, terminological, and ontological
resources throughout their proposed methods. For example, \citet{Zhou_2007} examined the influence of incorporating deep semantic knowledge (e.g., information about concepts and relationships between concepts) in a biomedical IR system. The authors showed that appropriate use of domain specific knowledge yields about 23\% improvement over the best reported results in the Genomics Track of TREC 2006 of TREC \citep{voorheestrec}. \citet{Chen_2011} explored the hidden connection from MEDLINE documents based on a passage retrieval method. Their method first uses Mesh concepts retrieved from the sentence-level windows, and then ranks them by z-score, TFIDF (Term Frequency Inverse Document Frequency) and PMI (Pointwise Mutual Information). In the passage retrieval experiments, the authors showed that the TFIDF and PMI methods can achieve much better performance than those in the concept retrieval experiment. \citet{Cao_2011} built a clinical QA system named AskHERMES to perform robust semantic analysis on complex clinical questions. The AskHERMES system consists of three main components: question analysis, document/passage retrieval, and answer extraction. In the passage retrieval component, the authors first used the BM25 model for document retrieval, and then extracted relevant passages based on both word-level and word sequence-level similarity between a question and a sentence in the candidate answer passage.

With the introduction of biomedical QA Track at the BioASQ challenge \citep{tsatsaronis2012bioasq} in 2013, passage retrieval in biomedical QA has received much attention from NLP/IR researchers. \citet{lingeman2014umass} applied the sequential dependence model to consecutive text segments inside the document in order to create a ranking on the subdocument level. They used a granularity of 50 words, which are shifted through the document in increments of 25 words. \citet{yenalaiiith} proposed a biomedical passage retrieval method based on cosine similarity and existence test score between the question and each sentence in the retrieved documents. Their method first uses the PubMed search engine to find relevant documents and then ranks them based on cosine similarity and existence test score. Their system finally extracts the sentences from the abstracts of the 10 top relevant documents and keeps only the 10 top sentences matching most with the biomedical question after finding similarity scores for all sentences. \citet{ligeneric} presented a biomedical passage retrieval method to exactly locate the passages for the biomedical questions from the users. In their method, the expansion query based on word embedding and the sequential dependence model are used to retrieve relevant passages from the retrieved documents using. The authors first used predefined rules to identify the candidate passages and then applied the sequential dependence
model to rank them. \citet{yang2015learning} applied the LETOR model to score and rank the obtained candidate passages from the abstracts of the top-ranked documents retrieved by their document retrieval system. \cite{peng2015fudan} used statistical language model in order to retrieve relevant documents and then query keywords are extracted from the retrieved documents by giving extra credit to terms that appear close to the query keywords for passages retrieval. \cite{neves2015hpi} proposed a passage retrieval approach based on the in-memory database in biomedical QA. The system first extracts candidate passages from the relevant documents based on the built-in information retrieval features available in the IMDB, which uses approximated string similarity to match terms from the question to the words in the documents. Then, the system proceeds the ranked candidate passages and retrieves only the top-ranked passages using TFIDF metrics. \citet{Morid_2016} proposed a method for extracting clinically useful sentences from synthesized online clinical resources that represent the most clinically useful information for directly answering clinicians's information needs. The authors developed a Kernel-based Bayesian Network classification model based on different domain-specific feature types (e.g., UMLS concepts) extracted from sentences in a gold standard composed of 18 UpToDate documents.

Though the existing passage retrieval methods have proven to be quite successful at extracting passages in biomedical QA, passage retrieval still requires further efforts in order to improve its performance. The passage retrieval component in biomedical QA, compared to that of document retrieval can benefit even more from incorporation of domain-specific semantic knowledge.

In the chapter~\ref{Chapter6} we will present in details the proposed passage retrieval method in biomedical QA.


\subsection{Answer extraction}
\label{Chapter3_6_4}

Answer processing and extraction is the last component in a typical QA pipeline as the last step towards answering a biomedical question posed in natural language is extracting and generating the final answer and presenting it to the user posing the question. Answer extraction is considered the most challenging task of a QA system as this is when the precise answer has to be extracted from the candidates answers retrieved and selected by the passage retrieval component. The answer depends directly to the question type since extracting the answer for example to the factoid question (e.g., ``Which is the gene most commonly mutated in Tay-Sachs disease?''), which is asking for a biomedical named entity as an answer, is not the same as extracting the answer to a yes/no question (e.g., ``Is imatinib an antidepressant drug?''), which is looking for ``yes'' or ``no'' as an answer, for instance. In general, if a QA system deals with more than two types of questions, then the appropriate answer extraction technique, which extracts the final answer from the candidate answers, is selected and chosen according to the question type that was automatically identified by the question classification component. An answer extraction technique ranks the candidate answers according to the degree to which they match the expected answer type.

Considering both the various types of questions that may a QA system deals with and their different intended types of answers, answer extraction within the biomedical domain is a challenging task. In particular, most approaches to biomedical QA in some way make use of domain-specific semantic knowledge for answer extraction. In MedQA system \citep{lee2006beyond,Yu_2007} which dealt only with definitional questions (i.e., questions with the format of ``What is X?''), the answer extraction component which extracts definitional sentences from the retrieved documents relevant sentences, uses lexico-syntactic patterns and UMLS Methasaurus concepts (UMLS 2005AA). \cite{Cao_2011} described an answer presentation approach based on clustering technique in their developed clinical QA system AskHERMES which returns summaries as answers to all potential questions.
The authors grouped all relevant text passages into different topics based on clustering technique to locate relevant information
of interest before delving into more detail. Topics are assigned to each cluster in the AskHERMES system using content-bearing query terms and expanded terms from the UMLS Methasaurus. \cite{abacha2015means} presented an answer search methodology based on semantic search and query relaxation in their proposed semantic medical QA system, called MEANS. In the answer search phase, the authors first associated to each initial SPARQL query one or several less precise queries according to the number of expected answers. Then, the constructed SPARQL queries are executed in order to interrogate RDF triples generated on the document processing step. The authors exploited named entity recognition and relation extraction techniques, and domain-specific semantic knowledge, such as UMLS Methasaurus and UMLS Semantic Network, to build RDF annotations of the source documents and SPARQL queries representing the users questions.

With the introduction of biomedical QA Track at the BioASQ challenge \citep{tsatsaronis2012bioasq} in 2013, various answer extraction techniques have been recently presented for different types of questions.  The BioASQ challenge released benchmark datasets of biomedical questions in English, along with gold standard answers. There were four types of questions: yes/no, factoid, list, and summary questions. Yes/no questions, these are questions that require either ``yes'' or ``no'' answer. Factoid questions, these are questions that require a particular entity name (e.g., of a disease, drug, or gene), a number, or a similar short expression as an answer. List questions, these are questions that expect a list of entity names (e.g., a list of gene names, a list of drug names), numbers, or similar short expressions as an answer. Summary questions, these are questions that expect short summaries as answer. In phase B of Task b of the BioASQ challenge, participants were asked to answer with exact answers and ideal answers (paragraph-sized summaries). Exact answers are only required in the case of yes/no, factoid, list, while ideal answers are expected to be returned for questions. The released questions were accompanied by their types and the correct answers for the required elements (documents and passages) of the first phase. In this context, \cite{yang2015learning} described the development of a biomedical QA system named OAQA that returns only the exact answers for factoid and list questions based on learning-based approaches. The authors trained three supervised models and several features (e.g., lemma, the semantic type of each concept, etc.) using factoid and list questions of BioASQ training questions. The first model which is an answer type detection model aims at identifying the semantic answer type of a given question, the second assigns a score to each candidate answer while the third is a collective re-ranking model. Similarly, \cite{peng2015fudan} described a biomedical QA system that retrieves solely the exact answers for factoid and list questions. The system first used PubTator \citep{Wei_2013} for generating the candidate answers and then ranked them using term frequency metrics. \cite{choi2015snumedinfo} presented an answer extraction method based on keyword terms to generate the ideal answers for biomedical questions. In their method, candidate passages are ranked based on number of keywords and then combined to form the ideal answer. On the other hand, \cite{neves2015hpi} described answer extraction approaches for generating both exact and ideal answers to yes/no, factoid, list and summary questions from the gold-standard passages provided by the BioASQ challenge. Her approaches are notables for the use of the IMDB database and its built-in text analysis features. For yes/no questions, decision on either the answers ``yes'' or ``no'' was based on the sentiment analysis predictions provided by the IMDB. For factoid and list questions, the author extracted exact answers based on the annotations of noun phrases and topics provided by IMDB. For ideal answers, the author built summaries based on the the phrases which contain sentiments. In other work, \cite{schulze2016hpi} used an algorithm that is based on LexRank \citep{erkan2004lexrank} and named entities for the generation of summaries as answers to biomedical questions. The authors first built a sentence graph and then calculated cosine similarity of each sentence with each other sentences using named entity as dimension for the vector. They finally used a LexRank-based method to calculate the sentences ranking.


While these systems have proven to be quite successful at answering biomedical questions, they provide a limited amount of question and answer types, for instance, some of them \citep{Cao_2011} handle only with definition questions or returns solely short summaries as answers for all types of questions, and the most of the other ones do not deal with yes/no questions which are one of the most complicated question types to answer as they are seeking for a clear ``yes'' or ``no'' answer. In addition, the biomedical QA systems still requires further efforts in order to improve their performance in terms of precision to currently supported question and answer types. Furthermore, in spite of the importance of answering yes/no questions in the biomedical domain, we observed that only few studies have been presented compared to other types of questions, such as factoid and summary questions. Even though there are only two possible answers, ``yes'' or ``no,'' such questions can be quite hard to answer due to the complicated sentiment analysis process of the candidate answers.

In the chapter~\ref{Chapter7} we will present in details the proposed answer extraction methods in biomedical QA.





%Approaches to the various text mining tasks in the biomedical domain make extensive use of the resources described in this section and sometimes derive meta-resources for a specific task.
\subsection{Integral biomedical question answering systems}
\label{Chapter3_6_5}
While many efforts have been made in the biomedical QA area, only few integral QA systems, which can automatically retrieve answers to biomedical natural language questions, have been presented up to now. In this section, we survey and discuss the main integral biomedical QA systems. However, techniques and methods used in different stages of such systems have been detailed in the previous sections.

In this context, \cite{lee2006beyond,Yu_2007} designed, implemented, and evaluated a medical definitional QA system (MedQA) which is composed of five components including (1) question classification, (2) query generation, (3) document retrieval, (4) answer extraction, and (5) text summarization. In MedQA, at first, the question classification component automatically classifies medical questions into categories of the taxonomy created by \citep{Ely429} based on supervised machine-learning approaches for which specific answer search is developed. Next, the document retrieval component uses the query terms to retrieve relevant documents from either the Web documents using Google or the locally-indexed MEDLINE corpora using both Lucene to index the MEDLINE collection and the vector-space model for computing the relevance of a document to a query. Then, the answer extraction component identifies from the retrieved documents relevant sentences that answer the questions based on lexico-syntactic patterns. Finally, the text summarization component removes the redundant sentences and condenses the sentences into a coherent summary which is considered as answer. Although the MedQA system returns short summaries that could potentially answer medical questions, current MedQA's capacity is limited: it only provides answers to definitional questions.

\cite{cruchet2009trust} built a biomedical QA system called HONQA which extracts sentences from Health On the Net Foundation (HON) certified websites and provides them as answers for biomedical questions. HONQA is based on a learning approach for identifying the question type and semantic resources such as UMLS to guide the system, particularly in the choice of answers, but no details are presented in the publication. In its current form, it is not able to provide exact answers to other question types, for instance,  yes/no and factoid questions \citep{Bauer_2012}.

\cite{gobeill2009question} developed a biomedical QA system, EAGLi, which aims at extracting answers to biomedical questions from MEDLINE documents. Given a natural language question, EAGLi first analyzes the question in order to find the question type and to build the query based on a set of patterns. Then, it retrieves a set of relevant documents from MEDLINE using either PubMed or EasyIR, a local search engine in MEDLINE. Finally, the system extracts and computes a score for each of concepts expressed in the most relevant documents, and finally outputs a ranked list of candidate answers.  The current EAGLi's capacity is limited to Wh-type questions since it only covers the definitional and factoid questions.

\cite{Cao_2011} described a clinical QA system named AskHERMES that returns short summaries as answers of ad-hoc clinical questions expressed in natural language. AskHERMES was developed through the main following steps: (1) question analysis, (2) document retrieval, (3) passage retrieval, and (4) summarization and answer presentation. In the question analysis step, the authors have first classified clinical questions into general topics (e.g., device, diagnosis, epidemiology, etc.) based on SVM classifier to facilitate information retrieval and then identified keywords that capture the most important content of the question using conditional random fields. In the document retrieval step, the BM25 model was used to retrieve relevant documents. After that, they have extracted candidate passages based on dynamically generates passage boundaries and scored them based on both word-level and word sequence-level similarity in the passage retrieval step. Finally, the answer was generated based on structural clustering using content-bearing terms. The AskHERMES system returns passages (short texts) that could potentially answer all types of clinical questions. However, it returns a large number of results, which tends to defeat the intent of a QA system in reducing the amount of information that must be read. Moreover, the system supports only a single answer type in form of multiple sentence passages for all questions types \citep{Bauer_2012}.

\cite{abacha2015means} developed a semantic medical QA system called MEANS based on NLP techniques to process medical natural language questions and documents used to find answers, and semantic Web technologies at both representation and interrogation levels. MEANS is composed of three main phases: (1) corpora annotation, (2) question analysis and classification, and (3) answer search. The authors have applied NLP methods, named entity recognition and relation extraction so as to build RDF annotations of the source documents and SPARQL queries representing the users questions. They further have defined the MESA ontology  to represent the concepts and relations between them in order to construct SPARQL translations of natural language questions. To extract answers, the SPARQL queries were executed in order to interrogate RDF triples constructed in the corpus-annotation step. Even so, the authors have dealt with four questions types, they have focused on factoid and yes/no questions since more specific processes are still required to deal with complex questions (e.g. why, when).

\cite{hristovski2015biomedical} introduced a biomedical QA system, SemBT, based on semantic relations extracted from the biomedical literature. SemBT consists of three main processing steps: (1) preprocessing, (2) question processing, and (3) answer processing. During the preprocessing step, the authors first have extracted semantic relations using the SemRep natural language processing system from sentences retrieved from MEDLINE citations, and then stored them in a database. In the question processing step, the authors have constructed a query for searching in the database of the extracted semantic relations.  Finally, in the answer processing phase, they have presented the resulting semantic relations as answers in a top-down fashion, first semantic relations with aggregated occurrence frequency, then particular sentences from which the semantic relations are extracted. The SemBT system returns answers in the form of semantic relations and particular sentences from which the semantic relations are extracted. However, in its current implementation, the questions must be in the form Subject-Relation-Object, and hence, it does not allow asking questions in natural language format, for example, the natural language question ``What drugs can be used to treat diabetes?'' can be asked in SemBT as ``phsu treats diabetes'' where ``phsu'' stands for ``pharmacological substance'' and ``treats'' is the name of the semantic relation \citep{hristovski2015biomedical}.

More recently, \cite{Kraus_2017} developed the Olelo system for intuitive exploration of
biomedical literature. The Olelo system consists of three main modules: (1) question processing, (2) document/passage retrieval, and (3) answer processing. In Olelo, the question processing module is based on a system described in \citep{schulze2016hpi}. In the second module, Olelo first uses the tokens and the matched terms to formulate a query to the database so as to retrieve abstracts of the retrieved documents and then ranks the obtained abstracts according to the occurrence and importance of the searched tokens. Finally, an answer is returned to the user depending on the type of the question by the third module. Although Olelo has proven to be quite successful at answering biomedical questions, currently, Olelo supports only three question types including factoid questions, list questions and summary questions. Indeed, it does not support yes/no questions which are one of the most complicated questions to answer as they are seeking for a clear ``yes'' or ``no'' answer.


In this thesis work, our goal is to go beyond the previous biomedical QA systems and develop a QA system with the ability to automatically handle with a large amount of question types including yes/no questions, factoid questions, list questions and summary questions that are commonly asked in the biomedical domain \citep{tsatsaronis2012bioasq}. We propose a fully automated system SemBioNLQA - Semantic Biomedical Natural Language Question Answering - which has the ability to handle the kinds of yes/no questions, factoid questions, list questions and summary questions that are commonly asked in the biomedical domain. SemBioNLQA is derived from our previously established methods in (1) question classification, (2) document retrieval, (3) passage retrieval, and (4) answer extraction systems. We develop the SemBioNLQA system based on the integration of these methods and techniques. These proposed methods are presented in details in the following chapters. Table~\ref{tab:3.1c} summarizes the dimensions and characteristics of the aforementioned systems and SemBioNLQA.

\begin{table}[h!]
\centering
\caption[Question answering systems comparison matrix of features between the aforementioned systems and our proposed system SemBioNLQA]{Question answering systems comparison matrix of features between the aforementioned systems and our proposed system SemBioNLQA. The ``-'' indicates that the system did not include target question types classification.}
\label{tab:3.1c}
\begin{tabular}{M{4.1cm}M{4cm}M{2.9cm}M{3.5cm}}
\hline\noalign{\smallskip}
QA systems &Question format&Question types&Answer types \\

\noalign{\smallskip}\hline\noalign{\smallskip}
MedQA\newline \citep{lee2006beyond}&	Natural language& Definition & Summaries\\
\cmidrule(l){1-4}
HONQA\newline \citep{cruchet2009trust}&	Natural language&Definition,\newline factoid& Sentence \\
\cmidrule(l){1-4}
EAGLi\newline \citep{gobeill2009question}&	Natural language&Definition,\newline factoid& Multi-phrase passages and
a list of single Sentities \\
\cmidrule(l){1-4}
AskHERMES\newline \citep{Cao_2011} &	Natural language&-& Multiple sentence passages\\
\cmidrule(l){1-4}

MEANS\newline \citep{abacha2015means}&	Natural language	& Definition,\newline yes/no,\newline factoid&
Sentence, named entity, ``yes'' or ``no''  \\
\cmidrule(l){1-4}

SemBT\newline \citep{hristovski2015biomedical}&Subject-Relation-Object&-	& Semantic relations, sentences  \\
\cmidrule(l){1-4}

Olelo\newline \citep{Kraus_2017}&	Natural language	& factoid,\newline list,\newline summary& MeSH term, list of Mesh terms, short summaries \\

\cmidrule(l){1-4}

SemBioNLQA &	Natural language	& Yes/no,\newline factoid,\newline list,\newline summary& ``Yes'' or ``no'',\newline UMLS entity,\newline list of UMLS entity, short summaries \\
%\cmidrule(l){1-7}
\noalign{\smallskip}\hline
\end{tabular}
\end{table}

Compared with the aforementioned systems, SemBioNLQA is aimed to be able to accept a variety of natural language questions and to generate appropriate natural language answers by providing both exact and ideal answers. It provides exact answers ``yes'' or ``no'' for yes/no questions, biomedical named entities for factoid questions, and a list of biomedical named entities for list questions. In addition to exact answers for yes/no, factoid and list questions, SemBioNLQA also returns ideal answers, while it retrieves only ideal answers for summary questions.

\section{Synthesis and Positioning}
\label{Chapter3_8}

Although several methods have been proposed in biomedical QA over recent years, biomedical QA still requires further efforts in order to improve its performance. For instance, existing solutions for biomedical question classification have so far focused on extracting syntactic and semantic features from questions and using machine learning algorithms so as to classify questions into different topics. However, they do not take into account the syntactic dependency relations in questions. Intuitively, the incorporation of syntactically related pairs into other features may provide the best description and representation of questions. The motivation to find alternative features for machine-learning algorithms is the fact that words by themselves cannot capture the gist of a clinical question.

Another challenging issue in question classification is the identification of the types and formats of potential questions and intended answers, respectively. The BioASQ taxonomy of biomedical questions consists of four types of questions including yes/no questions, factoid questions, list questions and summary questions that may cover all kinds of potential questions. Note only current biomedical QA systems have limitations in terms of the types and formats of questions and answers that they can process, but also in most such systems which dealt with more than one type of questions, the users have to give or select manually the question type to each given question. As the ultimate goal of biomedical QA systems is to be able to deal with a variety of natural language questions and to generate appropriate natural language answers, biomedical question type classification is a necessary task needs so as to automatically identify the type of question and therefore to see whether the answer should be a biomedical entity name, a short summary, ``yes'' or ``no'', etc.

On the second level, one of the main observations that can be made about existing systems is that the task of document retrieval often set a framework in which an existing biomedical IR system is used, and completely depended on its ranking of documents. Indeed, there are many cases where the search engine mistakenly returns irrelevant citations high in the set or relevant citations low in the set. This problem is certainly a challenging issue as a biomedical QA system usually extracts the answers from the top-ranked documents. On the other hand, the passage retrieval stage, compared to that of document retrieval can benefit even more from incorporation of domain-specific semantic knowledge. Although the existing passage retrieval methods have proven to be quite successful at extracting passages in biomedical QA, passage retrieval still requires further efforts in order to improve its performance.

On the third level, although the importance of answering questions in the biomedical domain, until now there are only few integral systems such as the ones described in \citep{lee2006beyond,cruchet2009trust,gobeill2009question,Cao_2011,abacha2015means,Kraus_2017} that can retrieve answers to biomedical questions written in natural language. While these systems have proven to be quite successful at answering biomedical questions, they provide a limited amount of question and answer types, for instance, some of them \citep{lee2006beyond,cruchet2009trust,Cao_2011} handle only with definition questions or returns solely short summaries as answers for all types of questions, and the most of the other ones do not deal with yes/no questions which are one of the most complicated question types to answer as they are seeking for a clear ``yes'' or ``no'' answer. In addition, the biomedical QA systems still requires further efforts in order to improve their performance in terms of precision to currently supported question and answer types. Furthermore, in spite of the importance of answering yes/no questions in the biomedical domain, we observed that only few studies have been presented compared to other types of questions, such as factoid and summary questions. Even though there are only two possible answers, ``yes'' or ``no,'' such questions can be quite hard to answer due to the complicated sentiment analysis process of the candidate answers.


The biomedical QA approach we propose through this thesis work takes into account these aspects and implements innovative methods in question classification, document retrieval, passage retrieval and answer extraction components. We propose a fully automated biomedical QA system with the ability to handle the kinds of yes/no questions, factoid questions, list questions, and summary questions that are commonly asked in the biomedical domain through the use of NLP methods, machine-learning approaches, and UMLS Metathesaurus. The proposed system provides the exact answers (e.g., ``yes'', ``no'', a biomedical entity, etc.) and the ideal answers (i.e., paragraph-sized summaries of relevant information) for yes/no, factoid and list questions, while it retrieves only the ideal answers for summary questions.

In the next chapters (chapter~\ref{Chapter4}, chapter~\ref{Chapter5}, and chapter~\ref{Chapter6}), we will present in details:


\begin{itemize} %\itemsep-4pt

\item The proposed machine learning based methods for question classification in biomedical QA. The first method consists at identifying the type (i.e., yes/no, factoid, list and summary questions) of a given biomedical question in order to determine the expected answer format. It is based on our predefined set of handcrafted lexico-syntactic patterns and machine learning algorithms. The second method, which is based on lexical, syntactic and semantic features for machine learning algorithms, allows classifying questions into topics in order to filter out irrelevant answer candidates.

\item The proposed document retrieval method which retrieves relevant citations to a given biomedical question from the MEDLINE database. The proposed method first builds the query by extracting biomedical concepts, then uses a specialized IR system that gives access to the MEDLINE database to retrieve relevant documents, and finally ranks them based on a semantic similarity. We have also proposed an alternative based on a probabilistic IR model and biomedical concepts to retrieve and extract a set of relevant passages (i.e., snippets) from the retrieved documents to given biomedical questions.


\item The proposed answer extraction methods for extracting natural language answers from passages that potentially containing answers through the use of semantic knowledge, NLP techniques and statistical techniques. The first answer extraction method, based on a sentiment lexicon, aims at generating the exact answers to yes/no questions. The second method makes uses a biomedical metathesaurus to provide the exact answers suited for factoid and list questions which require with respectively a biomedical entity and a list of them as answers. The third method, aiming at retrieving the ideal answers (i.e., short summaries of relevant information) to biomedical questions, is based on a probabilistic IR model and biomedical concepts.

\item The developed biomedical QA system named SemBioNLQA which is aimed to be able to accept a variety of natural language questions and to generate appropriate answers by providing both exact and ideal answers.  SemBioNLQA, which is fully automatic system, includes innovative methods previously proposed in question classification, document retrieval, passage retrieval and answer extraction components. It is derived from our previously established contributions in each of the aforementioned components.


\end{itemize}




\section{Evaluation of Biomedical Question Answering Systems}
\label{Chapter3_7}

We introduce in this section the experimental setup that is used in this thesis work for comparing different proposed methods in biomedical QA. The experimental setup is purposed to be as uniform and consistent across the different proposed methods of the thesis work reported here, so as to facilitate the interpretation and discussion of the experimental results. This experimental setup furthermore aims at making easier the reproducibility of our experiments and the comparison with our results, avoiding particular experimental setup decisions which might provide unfair advantages to our proposed methods. In subsection~\ref{Chapter3_7_1} and subsection~\ref{Chapter3_7_2} the evaluation datasets and evaluation measures will be introduced, respectively.


\subsection{Evaluation datasets}
\label{Chapter3_7_1}

Several fora such as BioASQ challenges, QA4MRE Alzheimer Disease, and TREC Genomics Track have been organized to promote research and benchmarking on biomedical QA from textual data.


The BioASQ challenge\footnote{The BioASQ challenge: \url{http://www.bioasq.org/}} \citep{tsatsaronis2012bioasq} which started in 2013, within 2017 edition \citep{Nentidis_2017}, comprised three tasks:
\begin{itemize}
  \item Task 5a on Large-Scale Online Biomedical Semantic Indexing
  \item Task 5b on Biomedical Semantic Question Answering
  \item Task c on Funding Information Extraction From Biomedical Literature
\end{itemize}

BioASQ which was part of a EU-funded project, was developed to boost state of the art performance for biomedical QA.  In Task 5b, BioASQ has provided the benchmark datasets which contain development and test biomedical questions, in English, along with golden standard answers. The challenge, within each edition, the organizers released a training set of biomedical questions-answers pairs and test sets of questions. There were four types of questions: yes/no questions, factoid questions, list questions, and summary questions. The benchmark datasets have been created by the BioASQ team of biomedical experts. The goal of Task 5b is to assess the performance of QA systems in different stages of the QA process. It is sub-divided into two phases: phase A and phase B. In phase A participants had to respond with biomedical concepts, relevant documents, relevant passages, and RDF triples. In phase B participants were asked to answer with exact answers and ideal answers (paragraph-sized summaries). Exact answers are only required in the case of yes/no, factoid, list, while ideal answers are expected to be returned for questions. The released questions were accompanied by their types and the correct answers for the required elements (documents and passages) of the first phase. Each batch of questions was released every two weeks and participants had 24 hours to submit results.


The QA4MRE (Question Answering for Machine Reading Evaluation) for biomedical text about Alzheimer's Disease took place in two editions (2012 \citep{morante2012machine} and 2013 \citep{morante2013machine}) and aimed to boost development of solutions in machine reading. Participant systems were asked to extract and generate the answers to a set of questions. A set of full text documents were provided along with ten questions, each of which had five candidate answers.  The test set consists of four reading tests where each test contains one document and ten questions related to that document as well as five possible answers per question. Questions are in the form of multiple choice, a particular type of factoid questions that are less complex in comparison to typical factoid questions, as participant systems can use the possible answers to query for candidate answers and do not need to find the final answer but just choose the most likely one.

The TREC Genomics challenge which is part of TREC, took place in 2006 \citep{voorheestrec} and 2007 \citep{voorheestrec2007} and aimed to foster development of solutions in passage retrieval as part of QA systems with focus on particular biomedical entity types, such as diseases, mutations, genes, proteins, and pathways. The challenge organizers have provided a set of 162.259 full text documents collected from about 49 journals related to genomics. Participants were asked to retrieve relevant passages to a given topic question from a collection of 162.259 documents. A set of 28 and 36 topic questions was constructed in 2006 and 2007, respectively. Evaluation for each participant system was carried out manually by TREC experts who were provided with the top scoring 1000 passages for each topic question.


Table~\ref{tab:3_7_1.1} shows different datasets provided by the BioASQ challenge, the QA4MRE challenge and the TREC Genomics challenge.


\begin{table}[h!]
\centering
\caption{Comparing biomedical QA datasets provided by the BioASQ challenge, TREC Genomics and QA4MRE Alzheimer Disease}
\label{tab:3_7_1.1}
\begin{tabular}{p{5cm}p{4.1cm}p{1.2cm}p{1.1cm}p{0.6cm}p{1.7cm}}
\hline\noalign{\smallskip}
Dataset &Questions (Train+Test)&Yes/No&Factoid&List&Summary\\

\noalign{\smallskip}\hline\noalign{\smallskip}
TREC 2006 Genomics&28&&&&\multicolumn{1}{c}{\ding{51}}\\%\xmark
TREC 2007 Genomics&36&&&&\multicolumn{1}{c}{\ding{51}}\\
QA4MRE Alzheimer Disease&40&&\multicolumn{1}{c}{\ding{51}}& \multicolumn{1}{c}{\ding{51}}&\\
BioASQ 2013&29+282&\multicolumn{1}{c}{\ding{51}}&\multicolumn{1}{c}{\ding{51}}&\multicolumn{1}{c}{\ding{51}}&\multicolumn{1}{c}{\ding{51}}\\
BioASQ 2014&310+500&\multicolumn{1}{c}{\ding{51}}&\multicolumn{1}{c}{\ding{51}}&\multicolumn{1}{c}{\ding{51}}&\multicolumn{1}{c}{\ding{51}}\\
BioASQ 2015&810+500&\multicolumn{1}{c}{\ding{51}}&\multicolumn{1}{c}{\ding{51}}&\multicolumn{1}{c}{\ding{51}}&\multicolumn{1}{c}{\ding{51}}\\
BioASQ 2016&1307+500&\multicolumn{1}{c}{\ding{51}}&\multicolumn{1}{c}{\ding{51}}&\multicolumn{1}{c}{\ding{51}}&\multicolumn{1}{c}{\ding{51}}\\
BioASQ 2017&1799+500&\multicolumn{1}{c}{\ding{51}}&\multicolumn{1}{c}{\ding{51}}&\multicolumn{1}{c}{\ding{51}}&\multicolumn{1}{c}{\ding{51}}\\
\noalign{\smallskip}\hline
\end{tabular}
\end{table}


To evaluate the different proposed methods of the thesis work reported here, we have used the datasets provided by the BioASQ challenge since as shown in Table~\ref{tab:3_7_1.1} they include a variety of questions types and answers, and also a large number of training and test questions. For these reasons, recently the BioASQ datasets are the most widely used to test the effectiveness of different parts of biomedical QA systems \citep{balikas2014results,balikas2015results,krithara2016results,Nentidis_2017}. Table~\ref{tab:3_7_1.2} shows the BioASQ 3b training questions used in this thesis work. The dataset include 810 questions-answers where each question was assigned to one category (yes/no, factoid, list, or summary). Table~\ref{tab:3_7_1.3} shows the test sets used to evaluate and compare the proposed methods in biomedical QA with current state-of-the-art methods. Table~\ref{tab:3_7_1.4} presents some examples of biomedical questions and their categories from BioASQ training questions.


\begin{table}[!htbp]
\centering
\caption{The question types and the number of BioASQ training questions assigned to each category}
\label{tab:3_7_1.2}
\begin{tabular}{p{5.3cm}p{4.5cm}p{5.2cm}}
\hline\noalign{\smallskip}
Question type & \#Questions & Category percentage  \\
\noalign{\smallskip}\hline\noalign{\smallskip}
Yes/No	& 237 &	29.26\%\\
Factoid	& 192 &	23.70\% \\
List	& 213 & 26.30\% \\
Summary	& 168 &	20.74\% \\
\noalign{\smallskip}\hline
\end{tabular}
\end{table}

\begin{table}[h!]
\centering
\caption{Number of questions and their categories in test sets of biomedical questions provided in the 2015, 2016, and 2017 and BioASQ challenges}
\label{tab:3_7_1.3}
\begin{tabular}{p{4cm}p{1.6cm}p{1.3cm}p{1.3cm}p{0.8cm}p{2cm}p{2.3cm}}
\hline\noalign{\smallskip}
\multirow{2}{*}{BioASQ dataset} &\multirow{2}{*}{Batch}&  \multicolumn{4}{c}{Question type} &\multirow{2}{*}{\#Total}\\
\cmidrule(l){3-6} &&\#Yes/No	& \#Factoid& \#List& \#Summary&\\

\noalign{\smallskip}\hline\noalign{\smallskip}
\multirow{5}{*}{BioASQ Task 3b 2015}  & Batch 1&	33	&26 &22 &19& 100 \\
& Batch 2&16&32 &28 &24& 100 \\
& Batch 3&29&26 &17 &28& 100 \\
& Batch 4&25&29 &23 &20& 97 \\
& Batch 5&28&22 &24 &26& 100 \\
\cmidrule(l){1-7}
\multirow{5}{*}{BioASQ Task 4b 2016}  & Batch 1&	28	&39 &11 &22& 100 \\
& Batch 2&32&31 &21 &16& 100 \\
& Batch 3&25&26 &21 &28&  100\\
& Batch 4&21&31 &15 &33&  100\\
& Batch 5&27& 34& 20&16&  97\\
\cmidrule(l){1-7}
\multirow{5}{*}{BioASQ Task 5b 2017}  & Batch 1&	17	&25 &22 &36& 100 \\
& Batch 2&27&31 &15 &27& 100 \\
& Batch 3&31&26 &15 &28&  100\\
& Batch 4&29&33 &13 &25&  100\\
& Batch 5&26& 35& 22&17&  100\\
%\cmidrule(l){1-8}
\noalign{\smallskip}\hline
\end{tabular}
\end{table}

\begin{table}[h!]
\centering
\caption{Question categories with some examples of biomedical questions collected from BioASQ training dataset}
\label{tab:3_7_1.4}
\begin{tabular}{M{2cm}M{13.4cm}}
\hline\noalign{\smallskip}
Category  & \hspace{0.5cm}Sample questions   \\
\noalign{\smallskip}\hline\noalign{\smallskip}
Yes/No &\begin{minipage}[t]{\linewidth}
 \begin{itemize} [nosep,nolistsep]
            %\itemsep-4pt
           \item Does SCRIB deregulation promote cancer?
           \item Is CADASIL syndrome a hereditary disease?
           \item Can PLN mutations lead to dilated cardiomyopathy?
         \end{itemize} \end{minipage}\\
         \cmidrule(l){1-2}
Factoid & \begin{minipage}[t]{\linewidth}\begin{itemize}[nosep,nolistsep]
          % \itemsep-4pt
            \item Which gene is involved in CADASIL?
            \item What is the inheritance pattern of Hunter disease or mu-copolysaccharidosis II?
            \item How many genes are in the gene signature screened by MammaPrint?
          \end{itemize}\end{minipage}\\
          \cmidrule(l){1-2}
List &\begin{minipage}[t]{\linewidth}\begin{itemize}[nosep,nolistsep]
       % \itemsep-4pt
        \item Which are the clinical characteristics of TSC?
        \item Which proteins induce inhibition of LINE-1 and Alu retrotransposition?
        \item What is being measured with an accelerometer in back pain patients
      \end{itemize}\end{minipage}\\
       \cmidrule(l){1-2}
Summary & \begin{minipage}[t]{\linewidth}\begin{itemize}[nosep,nolistsep]
            %\itemsep-4pt
            \item What is the treatment of acute pericarditis?
            \item How does trimetazidine affect intracellular kinase signal-ing in the heart?
            \item Why does the prodrug amifostine (ethyol) create hypoxia?
          \end{itemize}\end{minipage}
\\
\noalign{\smallskip}\hline
\end{tabular}
\end{table}

In particular, in a part of this thesis work (cf. section~\ref{Chapter4.3} of chapter~\ref{Chapter4}), we have used the publicly available data set\footnote{Set of clinical questions available at: \url{http://clinques.nlm.nih.gov/about.html}} of clinical questions maintained by the U.S. NLM in order to evaluate the effectiveness of the method we propose for question topic classification in biomedical QA. This benchmark collection of questions has been widely used in the literature to test and evaluate question topic classification methods such as in \citep{yu2008automatically,Cao_2010}. All these questions have been manually labelled and released by the U.S. NLM. This data set contains 4654 ad hoc clinical questions, collected from healthcare providers across the USA in four studies \citep{Ely_1999,elyjhon,ely1997lifelong,D_Alessandro_2003}. There are a total of 12 categories and each question is assigned to one or more categories. Some examples of clinical questions and their categories are shown in Table~\ref{tab:4.3.3}. The typology of those questions is illustrated in Table~\ref{tab:4.3.4}. Table~\ref{tab:4.3.5} presents the 12 categories and the number of the clinical questions assigned to each one. 3559 questions, 386 questions, 700 questions, four questions, five questions
are assigned to one category, two categories, three categories, four categories, and five categories, respectively.

\begin{table}[!h]
\centering
\caption{Examples of clinical questions and their topics maintained by the U.S. National Library of Medicine}
\label{tab:4.3.3}
\begin{tabular}{M{11cm}M{4.5cm}}
\hline\noalign{\smallskip}
Question & Topics \\
\noalign{\smallskip}\hline\noalign{\smallskip}
Mother is alcoholic and abuses tobacco. What are statistics regarding inheritance of tobacco abuse and relationship to social
situation?	& Epidemiology\\\cmidrule(l){1-2}

Does she have any underlying inflammation of her kidneys? Creatinine approximately 1.0, 3+ albumin on urinalysis, just over
500 milligrams protein/24 hrs, normal intravenous pyelogram. & Management \newline Diagnosis\\\cmidrule(l){1-2}

Coronary angioplasty and stent placed last week. Started on Ticlid, looks like she’s allergic to it. She’s supposed to be on
Ticlid one more week. Obviously we've got to stop it. Do they want her on something else or just stop it? & Management \newline Treatment \&
Prevention \newline Pharmacological\\

\noalign{\smallskip}\hline
\end{tabular}
\end{table}


\begin{table}[!h]
\centering
\caption[Typology of 4654 clinical questions and their representatives. The first column represents generic question proportions]{Typology of 4654 clinical questions and their representatives. The first column represents generic question proportions. The second column represents number of each question types and their percentages. Respectively, question examples are in the last column \citep{Cao_2010}}
\label{tab:4.3.4}
\begin{tabular}{M{3cm}M{4.3cm}M{7.8cm}}
\hline\noalign{\smallskip}
Question type &\#Question (percentage)& Question example \\
\noalign{\smallskip}\hline\noalign{\smallskip}
What& 2231 (48\%)& What is Endolimax nana and should you treat it?\\
Which& 62 (1\%)& Which dose of Premarin is green?\\
Why& 134 (3\%)& Why is she having pelvic pain?\\
How& 697 (15\%)& How do you inject the bicipital tendon?\\
Can& 187 (4\%) &Can Lorabid cause headaches?\\
Do& 320 (7\%) &Do we need to do a spinal tap to rule out meningitis?\\
Others& 1023 (22\%)& Is this respiratory plan for an extubated child okay?\\
\noalign{\smallskip}\hline
\end{tabular}
\end{table}

\begin{table}[!htbp]
\centering
\caption[The topics of 4654 clinical questions, the number of the clinical questions assigned and the percentage of the total questions]{The topics of 4654 clinical questions, the number of the clinical questions assigned and the percentage of the total questions \citep{Cao_2010}}
\label{tab:4.3.5}
\begin{tabular}{M{5.2cm}M{4.8cm}M{5cm}}
\hline\noalign{\smallskip}
Topics &\#Questions &Questions percentage\\
\noalign{\smallskip}\hline\noalign{\smallskip}
Device& 37& 0.8\%\\
Diagnosis& 994& 21.4\%\\
Epidemiology &104& 2.2\%\\
Etiology &173& 3.7\%\\
History& 42& 0.9\%\\
Management& 1403& 30.1\%\\
Pharmacological& 1594& 34.3\%\\
Physical Finding& 271& 5.8\%\\
Procedure &122& 2.6\%\\
Prognosis &53& 1.1\%\\
Test &746& 16.0\%\\
Treatment \& Prevention& 868& 18.7\%\\
Unspecified& 0& 0\%\\
\noalign{\smallskip}\hline
\end{tabular}
\end{table}

\subsection{Evaluation measures}
\label{Chapter3_7_2}


Several indicators have been used in this thesis work to evaluate the effectiveness of the proposed methods in biomedical QA. As shown previously, a typical QA system consists of four stages including question classification, document retrieval, passage retrieval and answer extraction, that can be studied and evaluated independently. Therefore, we present in this subsection the typical evaluation measures used for each stage.
\subsubsection{Evaluation measures for question classification}

The accuracy metric has been widely used in order to evaluate question type classification methods \cite{khoury2011question,li2002learning,xu2006syntactic,loni2011survey}. Accuracy, as it is defined in equation~\ref{eq:1qc}, is the number of correct made predictions divided by the total number of made predictions.
\begin{equation}\label{eq:1qc}
Accuracy= \frac {No.\; of \;{Correctly} \;{Classifed} \;{Questions}}{Total\; No.\;of\; Tested\; Questions}
\end{equation}

Additionally, precision, recall, and F1-measure are widely used to evaluate the effectiveness of question type classification methods at classifying questions into each of the predefined question types (multi-class classification). For each question type, precision, recall, and F1-measure are computed using the standard equation~\ref{eq:2qc}, equation~\ref{eq:3qc}, and equation~\ref{eq:4qc}, respectively. Precision is defined as the number of true positive over the number of true positive plus the number of false positive. Recall is defined as the number of true positive over the number of true positive plus the number of false negative. In other words, precision is the fraction of correct classification for a certain category, whereas recall is the fraction of instances of a category that were correctly classified. F1-measure also know as F1-score is defined as the harmonic mean of precision and recall.

\begin{equation}\label{eq:2qc}
Precision= \frac {True\; Positive}{True\; Positive+False\; Positive}
\end{equation}

\begin{equation}\label{eq:3qc}
Recall= \frac {True\; Positive}{True\; Positive+False\; Negative}
\end{equation}
\begin{equation}\label{eq:4qc}
{F1-measure}={F1-score}= \frac {2*Precision*Recall}{Precision+Recall}
\end{equation}


\subsubsection{Evaluation measures for document and passage retrieval}

As indicators of retrieval effectiveness, mean precision, mean recall, mean F1-measure and mean average precision (MAP) were used. Given a set of golden items (documents and passages), and a set of items (documents and passages) returned by document and passage retrieval system (for a particular biomedical question in our case), precision, recall, and F1-measure are calculated using the previously defined equation~\ref{eq:2qc}, equation~\ref{eq:3qc}, and equation~\ref{eq:4qc}, respectively, where true positives is the number of returned items that are also present in the golden set, false positives is the number of returned items that are not present in the golden set, and false negatives is the number of items of the golden set that were not returned by the system. Given a set of questions $q_1,q_2, \ldots, q_n$, mean precision, mean recall, and mean F1-measure of the document and passage retrieval system is obtained by averaging its precision, recall, and F1-measure, respectively, for all the questions \citep{Balikas_2013}.

As precision, recall, and F1-measure do not take into account the order of the items returned by IR system for each question,  it is common in IR
to compute MAP of the returned set of items \citep{Balikas_2013}. MAP is obtained by averaging the average precision (AP) of over a set of questions, defined in the following equation~\ref{eq:5ir}:



\begin{equation}\label{eq:5ir}
MAP= \sum_{i=1}^{n} AP_i
\end{equation}

where $AP_i$ is the average precision of the set returned for question $q_i$, as defined in equation~\ref{eq:6ir}, where $|L|$ is the number of items in the set, $|L_R|$ is the number of relevant items (documents or passages), $P(r)$ is the precision when the returned set is treated as containing only its first $r$ items, and $rel(r)$ equals 1 if the r-th item of the set is in the golden list (i.e., if the r-th item is relevant) and 0 otherwise returned for all the questions \citep{Balikas_2013}.

\begin{equation}\label{eq:6ir}
AP=  \frac {\sum_{r=1}^{|L|} P(r)*rel(r)}{|L_R|}
\end{equation}



\subsubsection{Evaluation measures for answer extraction}

The performance of the proposed answer extraction methods are evaluated using the evaluation measures described by the BioASQ challenge \citep{tsatsaronis2012bioasq}. In the case of yes/no questions, the exact answers had to be either ``yes'' or ``no''. Therefore, accuracy
is the most used evaluation metric to evaluate responses of yes/no questions. Let $n$ be, the number of yes/no questions, and $k$ the number of correctly answered yes/no questions, accuracy is computed using the following equation~\ref{eq:2ae}:

\begin{equation}\label{eq:2ae}
Accuracy= \frac {k}{n}
\end{equation}

For factoid questions, mean reciprocal rank (MRR) is the main measure used to evaluate the exact answers returned by the system. Assuming that there are $n$ factoid questions, MRR is computed using the equation~\ref{eq:3ae} where $r_i$ is the topmost position that contains the
golden entity name (or one of its synonyms) in the returned list of possible responses.

\begin{equation}\label{eq:3ae}
MRR= \frac {1}{n} * \sum_{i=1}^{n}\frac {1}{r_i}
\end{equation}

On the other hand, to evaluate the exact answers of list questions, the mean average precision, mean average recall, and mean average
F-measure metrics, which are computed by averaging precision, recall, and F1-measure over the list questions are used. These measures (i.e., precision, recall, and F1-measure) are computed using the standard equations defined in (\ref{eq:2qc}), (\ref{eq:3qc}) and (\ref{eq:4qc}), where true positives is the number of possible answers that are included both in the returned and the golden list; false positives is the number of possible answers that are mentioned in the returned, but not in the golden list; and false negatives is the number of possible answers that are included in the golden, but not in the returned list. Indeed, mean average F1-measure is the the official score used by the BioASQ challenge to rank the participant systems for list questions.


Finally, the ideal answers of questions (all questions types: yes/no, factoid, list and summary), were automatically evaluated using ROUGE-2 and ROUGE-SU4. Basically, ROUGE counts the overlap between an automatically constructed summary by the system and a set of golden summaries manually constructed by humans. More details of these evaluations metrics appear in \citep{Balikas_2013}.




\section{Summary of the Chapter}
\label{Chapter3_9}

In this chapter, we have presented a state of the art on QA systems which aim at answering natural language questions from textual documents.  We have started this chapter by presenting an introduction to question answering. We have described in details the generic architecture of QA systems. We then have presented specificities and characteristics of QA in the biomedical domain, respectively. We have also presented the main resources that can be exploited for QA in the biomedical domain. Next, we have reviewed current research efforts directed toward QA in the biomedical domain. We have described in details techniques and methods that have been proposed for each component of a biomedical QA system, including question classification, document retrieval, passage retrieval and answer extraction. Finally, we have presented a synthesis of the different methods, and described the metrics used for the evaluation of the methods we proposed in this Phd thesis.

The following chapters (chapter~\ref{Chapter4}, chapter~\ref{Chapter5}, and chapter~\ref{Chapter6}) are dedicated to the presentation of our methods for question classification, document retrieval, passage retrieval and answer extraction in biomedical QA. 

% Chapter 1

\chapter{Question Classification in Biomedical Question Answering} % Main chapter title

\label{Chapter4} % For referencing the chapter elsewhere, use \ref{Chapter1}
\setcounter{secnumdepth}{4}
\minitoc
This chapter presents the methods we propose for question classification in biomedical QA, a key task that is studied and evaluated separately. Section~\ref{Chapter4.2} will be dedicated to our proposed method for question type classification in biomedical QA. We consecrate section~\ref{Chapter4.3} to our proposed question topic classification method in biomedical QA.

\section{Introduction}
\label{Chapter4.1}
As we have described previously in section~\ref{Chapter3_6_1}, question classification is usually the first component in QA pipeline as the first step towards developing biomedical QA systems is processing and classifying the question in order to identify the question type and therefore the answer type to produce. The need of biomedical QA systems to handle a variety of natural language questions and to extract appropriate natural language answers, has resulted in questions being classified along various dimensions, including question type, topic, user, resource, etc. The most common question classification in biomedical QA is question type, sometimes referred to as the expected answer format. The identification of the type of a given biomedical question type is useful in candidate answer extraction as it allows to a biomedical QA system to know in advance the expected answer format, and therefore to use the appropriate answer extraction technique. In other words, the identification of the expected answer format required by a natural language question is usually carried out based on the question type and its linguistic information. For instance, a question identified as yes/no requires one of the two answers ``yes'' or ``no''. In contrast, a factoid question expects a particular answer type, such as an entity name. In this context, the most recent taxonomy of biomedical questions that is created by the BioASQ challenge \citep{tsatsaronis2012bioasq} consists of four types of questions that may cover all kinds of potential questions:

\begin{enumerate}
  \item Yes/No questions: They require only one of the two possible answers: ``yes'' or ``no''. For example, ``Is calcium overload involved in the development of diabetic cardiomyopathy?'' is a yes/no question and the answer is ``yes''.
  \item Factoid questions: They require a particular entity name (e.g., of a disease, drug, or gene), a number, or a similar short expression as an answer. For example, ``Which enzyme is deficient in Krabbe disease?'' is a factoid question and the answer is a single entity name ``galactocerebrosidase''.
  \item List questions: They expect a list of entity names (e.g., a list of gene names,  list of drug names), numbers, or similar short expressions as an answer. For example, ``What are the effects of depleting protein km23–1 (DYNLRB1) in a cell?'' is a list question.
  \item Summary questions: They expect a summary or short passage in return. For example, the expected answer format for the question ``What is the function of the viral KP4 protein?'' should be short text summary.
\end{enumerate}


With a view to developing an automatic biomedical QA system, the types of biomedical questions should be automatically identified by the system. The task of a question type classifier is to assign a class label, a category name which specifies the type of question, to a given natural language question. This task is usually carried out by checking whether it contains auxiliary verbs or Wh-question particles.  However, the complexity of the natural language poses many issues and challenges to this task in the biomedical domain, as some question that appear, at first, to belong to certain type can result to be of a different one. For instance, the biomedical question ``Where is the protein CLIC1 localized?'', from the BioASQ training questions, is a factoid question. Although the way in which it is constructed, by starting with ``where'', may lead it to the wrong classification of it being a summary question.

Another important dimension of classifying questions in biomedical QA is topic, sometimes referred to as the semantic type of the expected answer, as previously discussed in section~\ref{Chapter3_6_1}. Question topic classification refers to the process by which one or more topics are automatically derived from the given biomedical questions. The answer type is the semantic type of the expected answer, and is useful in generating specific answer retrieval strategies. For example, the question ``What is the best way to catch up on the diphtheria-pertussistetanus vaccine (DPT) after a lapse in the schedule?'' from NLM's data set represents a pharmacological question, and a biomedical QA system can therefore use the Micromedex pharmacological database as the resource to extract the answers. Recently, various approaches have addressed clinical question topic classification. For instance, \citep{yu2008automatically} and \citep{Cao_2010} proposed a clinical question topic classification method based on the combination of words, stemming, bigrams, UMLS concepts and UMLS semantic types as features for machine-learning algorithms to automatically classify an ad hoc clinical question into general topics. However, the existing methods have not taken into account the syntactic dependency relations in questions. Therefore, this may impact negatively the performance of the clinical question topic classification system.

In this context, we propose in the first contribution of this thesis work two machine learning-based methods for question classification in biomedical QA. The first method aims at classifying biomedical questions into one of the four categories: (1) yes/no, (2) factoid, (3) list, and (4) summary. The second method automatically identifies general topics from ad hoc clinical questions.  In the remaining of the chapter, we go into details about proposed question classification methods in biomedical QA.


\section{Question Type Classification}
\label{Chapter4.2}

In this section we describe the machine learning-based method we propose for biomedical question type classification \citep{kdir15,Sarrouti_MIM_2017}. As stated previously, the purpose of this method is to classify biomedical questions into one of the four categories: yes/no, factoid, list and summary  questions.

\subsection{Method}
\label{Chapter4.2.1}
With a view to achieving the goal of classifying the biomedical questions into the aforementioned categories, we first extract the appropriate features from BioASQ questions using our handcrafted lexico-syntactic patterns and then feed them to a machine-learning algorithm to conduct the classification task. The flowchart of the proposed method, as shown in Figure~\ref{fig:qc}, is constructed through the following main steps: (1) features extraction from biomedical questions using the set of our handcrafted lexico-syntactic patterns, (2) learned classifiers, and (3) predicting the class label using the trained classifiers.

% Figure environment removed

\subsubsection{Machine-learning models}
\label{Chapter4.2.1.1}
We have experimented with several machine-learning algorithms including SVM, Naive Bayes, and Decision Tree. Since a question can be assigned to one of 4 classes (i.e., yes/no, factoid, list, and summary), a multi-class classification has been used in this study. We have found that SVM outperforms the others, followed by Naive Bayes. Therefore, we have decided to report the results of SVM. Mallet\footnote{Mallet: \url{http://mallet.cs.umass.edu/index.php}.} and libSVM\footnote{LibSVM: \url{https://www.csie.ntu.edu.tw/~cjlin/libsvm/}.} are the freely available platforms for machine-learning algorithms that have been used in this work. We have used Mallet for Naive Bayes and Decision Tree. For SVM, we have used LibSVM.
\subsubsection{Machine-learning features}
\label{Chapter4.2.1.2}

We explored different features for machine-learning systems including words, bigrams, part-of-speech (POS) tags and features obtained by our predefined handcrafted lexico-syntactic patterns.
We first have experimented with bag-of-words also known as unigrams (terms in the questions are extracted, all tokens which match stop words list\footnote{Stop words list: \url{http://www.textfixer.com/resources/common-english-words.txt}} were removed). Then, we have generated N-grams, proximity-based sequences of words obtained by sliding a window of size N over the biomedical question, and use them as features. In this work, the bigrams terms, i.e., N-grams with $N=2$, are used. We also experimented with part-of-speech tags as features for machine-learning algorithms. We have used the Stanford CoreNLP tools for tokenization and POS tagging, as research shows a good performance of the latter in the biomedical domain \citep{lin2007syntactic}. Finally, we
have used features that have been provided by our predefined handcrafted lexico-syntactic patterns. The patterns were derived by analyzing 810 BioASQ training questions. They were found by passing the questions one by one to the Stanford CoreNLP for tokenization and POS tagging so as to capture their syntactic structure. Indeed, we have extracted words and their POS tags from each of BioASQ training questions. We have found after analysing all these questions that the ones belonging to a particular class have some typical structure. In addition, we found that adding
WordNet synonyms led to an enhanced performance of question classification; we therefore have used WordNet to generate the synonyms of some words in some patterns (e.g., see patterns for factoid questions).

As already noted, in order to extract the best feature set, our predefined patterns were used. The patterns are regular expressions that are represented as a text string. For a given biomedical question, the appropriate pattern is selected from the set of patterns as follow: after preprocessing (tokenization and POS tagging) the biomedical question using Stanford CoreNLP, the pattern is set to the left end of the biomedical question, and matching process starts. After a mismatch is found, the pattern is shifted one place right and a new matching process starts, and so on. Below are the set of patterns that have been used for the four categories: yes/no, factoid, list, and summary. The ``+'', ``|'', and ``*'' signs indicate the concatenation, OR, and any terms, respectively. Note also that NN, JJ, VBZ, VBP, and TAG indicate noun, adjective, verb 3rd person singular present, verb non-3rd person singular present, and the obtained par-of-speech tag of the word, respectively.


\begin{enumerate}
  \item \textbf{Yes/No patterns}: We have proposed the following expression for yes/no questions.
\begin{enumerate}
\item {[Be verbs $\mid$  Modal verbs $\mid$ Auxiliary verbs] + [*] +?};  where be verbs = \{am, is, are, been, being, was, were\},
modal verbs=  \{can, could, shall, should, will, would, may, might\}, and auxiliary verbs=  \{do, did, does, have, had, has\}
\end{enumerate}
\item \textbf{Factoid patterns}: We have defined the following patterns for questions which can belong to factoid category.
\begin{enumerate}

\item {[What $\mid$ Which] + [VBZ] + [*] + [X] + [*] +?}; where X = \{number, name, indication, value, frequency, prevalence, or WordNet synonyms of these words\}
\item {[What $\mid$ Which] + [NN] + [*] +?}
\item {[What $\mid$ Which] + [does $\mid$ do] + [*] + [stand for $\mid$ bind to] +? }
\item {[Which] + [TAG] + [*]+?}
\item {[Where] + [*] + [NN] + [*] + [VBZ]+ [located] +?}
\item {[When] + [TAG] + [*] +?}
\item {[Why] + [TAG] + [*] +?}
\item {[How] + [TAG] + [*] +?}
\end{enumerate}

\item \textbf{List patterns}:  We have provided the following patterns for questions which can belong to list category.
\begin{enumerate}

 \item {[What $\mid$ Which] + [VBP] + [*] + [X] + [*] +?; where X = \{numbers, names, indications, values, frequencies, prevalence, or WordNet synonyms of these words\}}
\item {[What $\mid$ Which] + [VBP] + [*]+ [NN] + [*] +?}
\item {[Which] + [TAG] + [*] +?;}
\item {[Where] + [NN] + [*] + [NN] + [VBP] + [used] +? }
\item {[When] + [TAG] + [*] +?}
\item {[How] + [TAG] + [*] +?}
\item {[Why] + [TAG] + [*] +?}
\end{enumerate}
\item \textbf{Summary patterns}: For the biomedical questions which can belong to summary category,  we have used the following patterns.
\begin{enumerate}
%\itemsep-4pt
\item {[What $\mid$ Which] + [VBZ] + [.*] + [X]+ [.*] +?; where X = \{definition, role, aim, effect, influence, mechanism, treatment, or WordNet synonyms of these words\} }
\item {[What] + [VBZ] + [NN] + [*] +?}
\item {[What] + [does] + [NN] + [*] + [do] +?}
\item {[Define $\mid$ Explain $\mid$ WordNet synonyms] + [*] + [NN] +?}
\item {[Why] + [TAG] + [*] +?}
\item {[How] + [TAG] + [*] +?}
\end{enumerate}
\end{enumerate}
\par
The key idea behind using the features provided by our set of patterns, is that only some words (e.g., Wh-question particles) in the biomedical question commonly represent the question type. Consider for example the biomedical question ``What is the definition of autophagy?'' from BioASQ training questions. The features vector (v) of this question is simply the patterns (see the first pattern of summary questions) that represent the structure of question. This question can be represented as follows: v= {{(what, 1), (VBZ, 1), (definition, 1)} where the pair is in the form (feature, frequency). Table~\ref{tab:4.2.1.2.1} shows the different feature space of the previous question.

\begin{table}[!h]
\centering
\caption{The different feature spaces of the biomedical question ``What is the definition of
autophagy?''}
\label{tab:4.2.1.2.1}
\begin{tabular}{M{4.6cm}M{10.7cm}}
\hline\noalign{\smallskip}
Feature space & Features  \\
\noalign{\smallskip}\hline\noalign{\smallskip}
Unigram& 	{(What, 1) (is, 1) (the, 1) (definition, 1) (of, 1) (autophagy, 1) (?, 1)}\\
\cmidrule(l){1-2}
Bigram &	{(What-is, 1) (is-the, 1) (the-definition, 1) (definition-of, 1) (of-autophagy, 1) (autophagy-?, 1)}\\
\cmidrule(l){1-2}
Part-of-speech&	{(WP, 1) (VBZ, 1) (DT, 1) (NN, 1) (IN, 1) (NN, 1)}\\
\cmidrule(l){1-2}
Part-of-speech + unigram &	{{(What, 1) (is, 1) (the, 1) (definition, 1) (of, 1) (autophagy, 1) (?, 1)}, {(WP, 1) (VBZ, 1) (DT, 1) (NN, 1) (IN, 1) (NN, 1)}}\\
\cmidrule(l){1-2}
\textbf{Set of patterns}	& \textbf{{(what, 1) (VBZ, 1) (definition, 1)}}\\

\noalign{\smallskip}\hline
\end{tabular}
\end{table}

We denote that biomedical questions that do not match any defined patterns are presented with their unigrams and part-of-speech tags.


The motivation to find alternative features for machine learning algorithms is the fact that words by themselves cannot capture the structure of a biomedical question. Intuitively, using our handcrafted lexico-syntactic patterns can capture the syntactic view of a biomedical question better than the other methods.
\subsection{Experimental results}

To validate the efficiency of our biomedical question type classification system, several experiments have been carried out using
the BioASQ training dataset and five batches of testing datasets that have been previously described in section~\ref{Chapter3_7_1} of chapter~\ref{Chapter3}. We have used a training set of 810 questions-answers, where one of the four types of questions: yes/no, factoid, list, summary was assigned to each question, as shown in Table~\ref{tab:3_7_1.2}. We have also used five batches of testing datasets provided in BioASQ Task 3b. Each of testing datasets is approximately comprised of 100 biomedical questions, as shown in Table~\ref{tab:3_7_1.3}. As indicators of classification effectiveness accuracy, precision, recall and F1-measure defined in equation~\ref{eq:1qc}, equation~\ref{eq:2qc}, equation~\ref{eq:3qc}, equation~\ref{eq:4qc}, respectively,  were used (cf. section~\ref{Chapter3_7_2}, chapter~\ref{Chapter3}).

For a machine-learning algorithm, we have used the multi-class SVM classifier. As a matter of fact, the linear kernel has been used for the SVM classifier since it is often recommended for text classification. Thus, the results have shown that linear kernel outperforms the other kernels such as RBF kernel, tree kernel, and composite kernel. As we have trained the SVM with a linear kernel, we only have needed to optimize the C parameter. The best value of C parameter is 1.01 which was fixed after 5-fold cross-validation.

Additionally, we have explored different features, including unigrams, bigrams, part-of-speech tags, the combination of part-of-speech tags and unigrams, and our set of handcrafted lexico-syntactic patterns. Table~\ref{tab:4.2.1.2.4} shows the SVM results in five feature models for automatically classifying a biomedical question into a category in terms of accuracy. The results show that the best method was trained on features extracted by our predefined patterns, which led to an accuracy of 89.40\%.

\begin{table}[h!]
%\scriptsize
\centering
\caption[The obtained results using SVM on five batches of testing datasets to automatically assign a category to biomedical questions]{The obtained results using SVM on five batches of testing datasets to automatically assign a category to biomedical questions. We explored different features, including unigram, bigram, part-of-speech, part-of-speech + unigram, and our set of predefined patterns.}
\label{tab:4.2.1.2.4}
\begin{tabular}{M{4.4cm}M{7cm}M{3.6cm}}
\hline\noalign{\smallskip}
Data sets&	Feature models &Accuracy (\%) \\
\noalign{\smallskip}\hline\noalign{\smallskip}
\multirow{5}{*}{Five batches} & Unigram & 79.48\\[1.5pt]
             & Bigram &  	65.18\\[1.5pt]
             &Part-of-speech&	77.08\\[1.5pt]
             &Part-of-speech+unigram&	80.08\\[1.5pt]
             &\textbf{Set of patterns}&	\textbf{89.40}\\[1.5pt]
\noalign{\smallskip}\hline
\end{tabular}
\end{table}

Using unigrams as features, the overall accuracy was 79.48\%. The results are meant to be a benchmark. We have found that other features have an impact on the performance of biomedical question type classification. Bigrams decreased the performance to 65.18\% in terms of accuracy
(an absolute decrease of 14\% accuracy). When part-of-speech tags were used as features, the overall performance decreased to 77.08\% accuracy, although the decrease was not statistically significant (an absolute decrease of 2.4\% accuracy). The incorporation of part-of-speech tags and unigrams as features improved the performance to 80.08\% accuracy.


We then experimented with our set of handcrafted lexico-syntactic patterns so as to show how well a system can perform the classification task by combining our predefined patterns and a machine-learning algorithm. We found that using our patterns features' provider of SVM leads to the highest accuracy of 89.40\%. Table~\ref{tab:4.2.1.2.5} shows the detailed results for each question category in two feature models, i.e., unigrams and set of our handcrafted lexico-syntactic patterns, on five batches of testing datasets using the SVM classifier. In two feature models, the category ``summary'' has the lowest classification performance, and ``yes/no'' has the highest one.

\begin{table}[h!]
\centering
\caption[The detailed results for each question category in two feature models (unigram and set of patterns) by applying SVM classifier]{The detailed results for each question category in two feature models (unigram and set of patterns) by applying SVM classifier. P, R, F, A indicate precision, recall, f1-measure, and accuracy, respectively.}
\label{tab:4.2.1.2.5}
\begin{tabular}{p{2.6cm}p{4cm}p{2.1cm}p{1.1cm}p{1.1cm}p{1.1cm}p{1.3cm}}
\hline\noalign{\smallskip}
Data sets &	Feature models &Class &	P (\%)	&R (\%)	&F (\%)	&A (\%) \\
\noalign{\smallskip}\hline\noalign{\smallskip}
\multirow{9}{*}{Five Batches}& \multirow{4}{*}{\parbox{3cm}{Unigram:\\Baseline model}} & Yes/No	& 95.00&	98.59&	96.70&\multirow{5}{*}{79.48}\\
                        & &Factoid  &	76.36&	66.70&	70.96\\
                         && List&	72.81&	92.28&	81.31\\
                         && Summary&	72.02&	59.91&	64.80\\
 \cmidrule(l){2-6}
                        &\multirow{4}{*}{\parbox{4cm}{Set of patterns:\\Proposed method} }& Yes/No	& 95.55&	100.0&	97.41&\multirow{5}{*}{89.40}\\
                         & &Factoid&	92.92&	80.14&	85.91\\
                         & &List&	83.27&	94.22&	88.37\\
                         & &Summary&	84.02&	80.49&	81.89\\
\noalign{\smallskip}\hline
\end{tabular}
\end{table}


\subsection{Discussion}
Overall, as shown in Table~\ref{tab:4.2.1.2.4}, the best system for automatically assigning a category to a biomedical question was trained on features extracted by our handcrafted lexico-syntactic patterns, which led to an accuracy of 89.40\%. Our results also show that feature selection impacts the biomedical question type classification performance. Using bag-of-words as features, the performance for automatically assigning a category to a question was an accuracy of 79.48\%. Thus, this feature set was chosen as a competitive baseline since it is widely used as baseline system in many question classification studies \citep{li2002learning,li2006learning,Patrick_2012,roberts2014automatically}. Bigrams decreased the performance to 65.18\% accuracy and the decrease was statistically significant (p < 0.01, t-test). Using the bigrams feature set by themselves could not capture the structure of a biomedical question in the context of this study, since the diversity in each category increased when using this feature set. For instance, for the biomedical question ``Why are insulators necessary in gene therapy vectors?'', the bigram ``why are'' is noisy, as the two words are not syntactically connected. Such noisy bigrams have a significant impact on the performance of question classification. Part-of-speech tags, on the other hand, slightly decreased the performance to 77.08\% accuracy. However, the decline was not statistically meaningful. In contrast, the combination of unigrams and part-of-speech tags as features improved the accuracy slightly (the absolute increase of 0.34\% accuracy), although the increase was not statistically significant. Meanwhile, using features extracted by our handcrafted lexico-syntactic patterns for SVM achieved the highest accuracy of 89.40\%. It outperforms the baseline system with a large margin 9.92\% in terms of accuracy and the increase is statistically significant. Inconsistency in category assignment may be responsible for the relation between the training size and the category classification performance. Typically, there is a strong positive relation between the training size and the classification performance: the larger the training size is the better a classifier performs. The Pearson correlation coefficient between F1-measure and the number of training size of a category shows an R-value of 0.83 (strong positive
correlation), which means that high F1 measure goes with high number of training data of a category (which confirms our hypothesis). We can see from Table~\ref{tab:4.2.1.2.5} and Table~\ref{tab:3_7_1.2} that the best performing category, yes/no, has the largest number of question instances (237), and the worst
performing category summary has the least number of question instances (168). Besides, \cite{metzler2005analysis} noticed that the ambiguity of labeled data has an impact on the category assignment. Unluckily, we have found that some biomedical questions in BioASQ training dataset were ambiguous. For instance, the biomedical question ``Which are the mutational hotspots of the human KRAS oncogene?'' is labeled with ``summary'' while it is also labeled by ``list'' category. Another example is the question ``Which are the newly identified DNA nucleases that can be used to treat thalassemia?'' that is labeled with ``factoid'' while it is also labeled by “list”. Biomedical question type classification can improve or decrease the performance
of an automatic biomedical QA system, because the answers extraction is based on the expected answer format of the questions. For example, extracting the answer to the question ``Which enzyme is deficient in Krabbe disease?'', which is asking for a
biomedical entity name, is not the same as extracting the answer to ``Is calcium overload involved in the development of diabetic cardiomyopathy?'', which is looking for ``yes'' or ``no'' as an answer. So, the class that can be assigned to a biomedical question affects greatly all the other steps of the QA process and therefore it is of vital importance to assign it properly. A study presented by \cite{Moldovan_2003} showed that
more than 36\% of the errors in a QA system are directly due to the question classification. In addition, the biomedical question
classification system can also improve the performance of IR systems \citep{Cao_2010}, because the question category can be used to choose the search strategy when the question is reformed to a query over IR systems. This is the case of the question ``What is the
definition of autophagy?'' from BioASQ datasets, identifying that the question category is ``summary'', the searching template
for locating the answer can be for example ``autophagy is a ...'' or ``definition of autophagy is ...'', which are much better than
simply searching by question words.
\section{Question Topic Classification}

In this section, we propose a machine learning-based method for question topic classification in biomedical QA \citep{Sarrouti_IBRA_2017}. The aim of this method is to classify clinical questions into general topics defined by the U.S. NLM. These topics are: device, diagnosis, epidemiology, etiology, history, management, pharmacological, physical, finding, procedure, prognosis, test, treatment and prevention.

\label{Chapter4.3}
\subsection{Method}

With a view to achieving the goal of classifying natural language questions into general topics and improving the performance of question topic classification in biomedical QA, we propose to incorporate the syntactic dependency relations into other features used in \citep{yu2008automatically,Cao_2010} for machine-learning approaches. The incorporation of syntactic and semantic features including words, bigrams, stemming, UMLS Metathesaurus concepts and semantic types introduced by \cite{yu2008automatically} and \cite{Cao_2010} seem to be quite enough to represent the questions. However, this method doesn't give the expected results. Intuitively, the incorporation of syntactically related pairs into these features may provide the best description of questions. A syntactic dependency relation \citep{nastase2007using} is a pair of grammatically related words in a phrase: the main verbs in two connected clauses, a verb and each of its arguments, a noun and each of its modifiers. It describes the syntactic structure of a sentence by using a typed dependency to establish relationships among words in terms of head and dependents.

The flowchart of the proposed method, as shown in Figure~\ref{fig:qc2}, is constructed through the following main steps: (1) features extraction from biomedical questions, (2) learned classifiers, and (3) predicting the class label using the trained classifiers.
% Figure environment removed


\subsubsection{Machine-learning models}

In question classification, most of the existing methods have used SVM as it leads to the best results in comparison with other classifiers \citep{yu2005classifying,yu2008automatically,Cao_2010}. Accordingly, we have experimented with SVM and other machine-learning algorithms including Naive Bayes and Decision Tree. Since a question can be assigned to one or more general topics, a multi-label classification has been used in this study. We therefore developed a binary machine learning classifier for each of the predefined topics. We have found that SVM
outperforms the others, followed by Naïve Bayes. Therefore, we have decided to report and compare the results of both SVM and Naïve Bayes. Mallet and libSVM are the freely available platforms for machine-learning algorithms that have been used in this study. We have used Mallet for Naïve Bayes and Decision Tree. For SVM, we have used libSVM.

\subsubsection{Machine-learning features}
After preprocessing the clinical questions (terms in the questions were extracted and stop words were removed), we have explored bag-of-words as features. We then have applied \cite{Porter_1980} stemmer and \cite{Krovetz_1993} algorithms for term normalisation process and used them as additional features. Next, we have extracted the syntactic dependency relations between words in the question using the Stanford parser \citep{de2006generating}, and explored them as additional features. The overall process sequence is depicted in Table~\ref{tab:4.3.1}. After that, we have generated the N-grams, proximity-based sequences of words obtained by sliding a window of size N over the question, of each question and used them as additional features. In this work, the bigrams (N-grams with $N=2$) terms were used. Finally, we have mapped the terms in questions into the UMLS Metathesaurus in order to identify concepts and their semantic types, and explored them as additional features. To do so, we have used the MetaMap tool to identify appropriate UMLS Metathesaurus concepts and semantic types in questions. Table~\ref{tab:4.3.2} illustrates an example of mapping a sample clinical question to UMLS concepts and semantic types.

\begin{table}[!h]
\centering
\caption[Linguistic preprocessing of a sample clinical question]{Linguistic preprocessing of a sample clinical question, tokenisation and part-of-speech tagging is shown in ``Step 1'': ``WP'' corresponds to a Wh-pronoun, ``VBZ'' to a verb in the third person singular present, etc. The parse tree is shown in ``Step 2'' and includes the root of the tree (ROOT), the question phrase (WHNP), several noun phrases (NP), etc. The dependencies tree is shown in ``Step 3'': ``nsubj'' is the nominal subject relation, ``cop'' refers to the relation between a complement and the copular verb, etc.}
\label{tab:4.3.1}
\begin{tabular}{M{1.7cm}M{13.7cm}}
\hline\noalign{\smallskip}

Question&What is the dose of Zithromax for this 35-kilogram kid?\\
\cmidrule(l){1-2}

Step 1 & What/\textcolor{red}{WP} is/\textcolor{red}{VBZ} the/\textcolor{red}{DT} dose/\textcolor{red}{NN} of/\textcolor{red}{IN} Zithromax/\textcolor{red}{NNP} for/\textcolor{red}{IN} this/\textcolor{red}{DT} 35-kilogram/\textcolor{red}{JJ} kid/\textcolor{red}{NN} ?/\textcolor{red}{.}
\\
\cmidrule(l){1-2}
Step 2&
\resizebox{\linewidth}{!}{%
\Tree[.\textcolor{red}{ROOT}
  [.\textcolor{red}{.SBARQ}
    [.\textcolor{red}{WHNP} [.WP What ]]
    [.\textcolor{red}{SQ} [.VBZ is ]
      [.\textcolor{red}{NP}
        [.\textcolor{red}{NP} [.DT the ] [.NN dose ]]
        [.\textcolor{red}{PP} [.IN of ]
          [.\textcolor{red}{NP}
            [.\textcolor{red}{NP} [.NNP Zithromax ]]
            [.\textcolor{red}{PP} [.IN for ]
              [.\textcolor{red}{NP} [.DT this ] [.JJ 35-kilogram ] [.NN kid ]]]]]]]]]}
 \\
 \cmidrule(l){1-2}
 Step 3 &\textcolor{red}{root}(ROOT-0, What-1)
\textcolor{red}{cop}(What-1, is-2)
\textcolor{red}{det}(dose-4, the-3)
\textcolor{red}{nsubj}(What-1, dose-4)
\textcolor{red}{case}(Zithromax-6, of-5)
\textcolor{red}{nmod:of}(dose-4, Zithromax-6)
\textcolor{red}{case}(kid-10, for-7)
\textcolor{red}{det}(kid-10, this-8)
\textcolor{red}{amod}(kid-10, 35-kilogram-9)
\textcolor{red}{nmod:for}(Zithromax-6, kid-10)\\
\noalign{\smallskip}\hline
\end{tabular}
\end{table}

%\begin{longtable}{M{1.4cm}M{11.9cm}}
%\caption{}\\
%\hline\noalign{\smallskip}

%\endfirsthead
%\multicolumn{2}{c}%
%{\tablename\ \thetable\ -- \textit{Continued from previous page}} \\
%\hline\noalign{\smallskip}

%\noalign{\smallskip}\hline
%\endhead
%\hline \multicolumn{2}{r}{\textit{Continued on next page}} \\
%\endfoot
%\noalign{\smallskip}\hline
%\endlastfoot

%\end{longtable}

\begin{table}[!h]
\centering
\caption{Example of mapping the clinical question ``Mother is alcoholic and abuses tobacco. What are statistics regarding inheritance of tobacco abuse and relationship to social situation?'' to UMLS Metathesaurus concepts and semantic types. CUI and TUI indicate concept unique identifier and type unique identifier, respectively.}
\label{tab:4.3.2}
\begin{tabular}{M{4.1cm}M{2cm}M{6.4cm}M{2cm}}
\hline\noalign{\smallskip}
UMLS concepts & UMLS CUI & UMLS semantic types& UMLS TUI  \\
\noalign{\smallskip}\hline\noalign{\smallskip}
Mother (person)&C0026591& Family Group&T099\\
\cmidrule(l){1-4}
Tobacco&C0040329&Hazardous or Poisonous Substance\newline Organic Chemical&T131\newline T109\\
\cmidrule(l){1-4}
Tobacco Use Disorder&C0040336&Mental or Behavioral Dysfunction&T048\\\cmidrule(l){1-4}
Alcoholics&C0687725&Patient or Disabled Group&T101\\\cmidrule(l){1-4}
Statistics (publications)&C0600673&Intellectual Product&T170\\\cmidrule(l){1-4}
Drug abuse&C0013146&Mental or Behavioral Dysfunction&T048\\\cmidrule(l){1-4}
Concept Relationship&C1705630&Idea or Concept&T078\\\cmidrule(l){1-4}
Mode of inheritance&C1708511&Genetic Function&T045\\\cmidrule(l){1-4}
Social situation&C0748872&Social Behavior&T054\\
\noalign{\smallskip}\hline
\end{tabular}
\end{table}

\subsection{Experimental results}


\subsubsection{Results}
To validate the efficiency of the proposed method to question topic classification in biomedical QA, several experiments have been conducted using the set of 4654 clinical questions maintained by the U.S. NLM and previously described in section~\ref{Chapter3_7} of chapter~\ref{Chapter3}. As indicators of classification effectiveness, F1-score defined in equation~\ref{eq:1qc} is used (cf. section~\ref{Chapter3_7_2}, chapter~\ref{Chapter3}), where the recall is the number of correctly predicted clinical questions divided by the total number of annotated questions in the same category, and precision is the number of correctly predicted clinical questions divided by the total number of predicted questions in the same category. We randomly select negative data to repeat the classifications ten times. We then report the average F1-scores.

We have explored supervised machine-learning algorithms to automatically classify an ad hoc clinical question written in natural language into one or more topics predefined by NLM. We have used the freely available Mallet and libSVM for supervised machine learning systems. We have experimented with three machine-learning algorithms including Naïve Bayes, Decision Tree, and SVM. We have found that SVM outperforms other classifiers, followed by Naive Bayes. Therefore, we have decided to report the results of SVM and Naive Bayes.  Indeed, the linear kernel has been used for the SVM classifier since it is often recommended for text classification. Thus, the results have shown that linear kernel outperforms the other kernels such as RBF kernel, tree kernel, and composite kernel. On the other hand, we have also compared the performance of our proposed method with different common features for machine-learning algorithms, including bag-of-words, bag-of-stems, bag-of-bigrams, bag-of-syntactic dependency relations, bag-of-UMLS concepts, bag-of-UMLS semantic types and the combination of features used by \cite{yu2008automatically,Cao_2010}.


For evaluation, we have arranged that each classifier has a baseline of 50\%. In other words, each classifier is trained on the same number of positive and negative data. For example, when we trained a binary classifier of Pharmacological, we had 1594 questions that were assigned to this category (see Table~\ref{tab:4.3.5}). This set of 1594 questions represents the positive training data. To generate negative training data, we have randomly selected 1594 questions from the remaining categories.



Table~\ref{tab:4.3.6} and Table~\ref{tab:4.3.7} show the F1-score for each topic using Naïve Bayes and SVM classifiers, respectively. We have explored different combinations of features including bag-of-words (BOW), bag-of-bigrams (BOB), bag-of-stems (BOS) using both Porter and Krovetz stemmers, bag-of-biomedical named entities (BOBNE), bag-of-syntactic dependency relations (BOSDR), bag-of-UMLS concept and semantic types (BOCST), and our combination of features, the proposed method. The proposed combination of features consists of BOW, BOB, BOS$_{porter}$, BOSDR, and BOCST. The overall results show that the best system was trained on our combination of features, which led to F-scores of 77.18\% and 71.77\% using SVM and Naïve Bayes respectively. Table~\ref{tab:4.3.8} and Table~\ref{tab:4.3.9} show the comparison of the proposed method with other combination of features for classifying clinical questions into categories. Table~\ref{tab:4.3.8} shows the increasing performance of the proposed method in comparison with other representations using Naïve Bayes as a classifier, and Table~\ref{tab:4.3.9} shows the increasing performance using the SVM classifier. These results show that our method is more effective as compared with state-of-the-art methods and outperforms them by an average of 4.5\% F1-score using Naïve Bayes and 4.73\% F1-score using SVM.



Figure~\ref{fig:topics1} and Figure~\ref{fig:topics2} show the classification performance of topic assignment in terms of F1-score as a function of training size using Naive Bayes and SVM, respectively. In both systems, the topic pharmacology has the highest classification performance (79.53\% F1-score for Naive Bayes and 84.85\% F1-score for SVM), and history has the lowest classification performance (55.71\% F1-score for Naive Bayes and 68.75\% F1-score for SVM).



\begin{landscape} % <-- HERE
\centering
\begin{table}
\centering
\caption[The obtained results in terms of F1-score using Naïve Bayes to automatically assign topics to ad hoc clinical questions]{The obtained results in terms of F1-score using Naïve Bayes to automatically assign topics to ad hoc clinical questions. We explored different combinations of feature sets including bag-of-words (BOW), bag-of-bigrams (BOB), bag-of-stems (BOS) using both Porter and Krovetz stemmers, bag-of-biomedical named entities (BOBNE), BOW+bag-of-UMLS concept and semantic types (BOCST), BOW+bag-of-syntactic dependency relations (BOSDR), BOW+BOB+BOCTY+BOS$_{porter}$ used in \citep{yu2008automatically,Cao_2010}, BOW+BOB+BOCST+BOS$_{krovetz}$, and our proposed method, the combination of BOW+BOB+BOCST+BOSDR+BOS$_{porter}$.}
\label{tab:4.3.6}
\begin{tabular}{M{4.5cm}M{1.2cm}M{1.2cm}M{1.2cm}M{1.1cm}M{1.1cm}M{1.2cm}M{1.4cm}M{1.6cm}M{1.6cm}M{1.6cm}}
\hline\noalign{\smallskip}
\multirow{4}{*}{Topics}&\multicolumn{9}{c}{Features} \\\cmidrule(l){2-11}
&&&&&&&\multicolumn{2}{c}{\thead{\cite{yu2008automatically}\\ \cite{Cao_2010}}}&\multicolumn{2}{c}{\thead{Proposed method}}\\\cmidrule(l){8-9}\cmidrule(l){10-11}
 &\multirow{2}{*}{BOW}& \multicolumn{2}{c}{BOS}& \multirow{2}{*}{BOBNE} &\multirow{2}{*}{\thead{BOW+\\BOCST}}& \multirow{2}{*}{\thead{BOW+\\BOSDR}}& \multicolumn{2}{c}{\thead{BOW+BOB+BOCST+}}& \multicolumn{2}{c}{\thead{BOW+BOB+BOSDR+BOCST+}}\\
\cmidrule(l){3-4}\cmidrule(l){8-9}\cmidrule(l){10-11}
&  &Porter& Krovetz&&&&BOS$_P$& BOS$_K$&BOS$_P$& BOS$_K$    \\


\noalign{\smallskip}\hline\noalign{\smallskip}
Device&55.13\%&56.42\%&55.85\%&55.46\%&60.45\%&56.87\%&68.73\%&68.19\%&69.47\%&68.96\%\\
Diagnosis&69.38\%&70.43\%&68.53\%&67.45\%&71.15\%&71.54\%&72.20\%&71.17\%&73.98\%&72.08\%\\
Epidemiology&65.82\%&66.88\%&64.42\%&63.12\%&67.55\%&67.65\%&70.62\%&69.14\%&71.63\%&70.40\%\\
Etiology&71.22\%&73.03\%&72.00\%&67.05\%&74.08\%&72.60\%&75.38\%&74.17\%&76.54\%&75.51\%\\
History&46.40\%&47.72\%&47.07\%&44.57\%&52.34\%&48.35\%&54.16\%&53.50\%&55.71\%&55.02\%\\
Management&60.26\%&61.35\%&60.44\%&61.08\%&63.62\%&62.05\%&65.34\%&64.39\%&66.78\%&65.83\%\\
Pharmacological&74.86\%&75.35\%&74.96\%&71.75\%&77.26\%&76.67\%&78.45\%&77.80\%&79.53\%&78.89\%\\
Physical\&Finding&67.89\%&68.58\%&67.90\%&64.89\%&69.65\%&68.74\%&73.76\%&72.59\%&74.61\%&73.50\%\\
Procedure&67.36\%&69.03\%&67.80\%&66.43\%&71.06\%&69.01\%&72.38\%&71.13\%&74.04\%&72.77\%\\
Prognosis&69.44\%&70.42\%&69.24\%&64.10\%&71.94\%&70.46\%&72.46\%&70.89\%&73.52\%&72.07\%\\
Test&72.51\%&73.05\%&72.66\%&68.32\%&75.78\%&73.85\%&76.89\%&76.18\%&78.24\%&77.53\%\\
Treatment\&Prevention&63.30\%&64.67\%&63.92\%&59.11\%&65.82\%&64.25\%&66.31\%&64.69\%&67.16\%&65.53\%\\\cmidrule(l){1-11}
Average&65.30\%&66.41\%&65.40\%&62.78\%&68.39\%&66.84\%&70.56\%&69.49\%&71.77\%&70.67\%\\

\noalign{\smallskip}\hline
\end{tabular}
\end{table}

\end{landscape} % <-- HERE

\begin{landscape} % <-- HERE
\centering
\begin{table}
\centering
\caption[The obtained results in terms of F1-score using SVM to automatically assign topics to ad hoc clinical questions]{The obtained results in terms of F1-score using SVM to automatically assign topics to ad hoc clinical questions. We explored different combinations of feature sets including bag-of-words (BOW), bag-of-bigrams (BOB), bag-of-stems (BOS) using both Porter and Krovetz stemmers, bag-of-biomedical named entities (BOBNE), BOW+bag-of-UMLS concept and semantic types (BOCST), BOW+bag-of-syntactic dependency relations (BOSDR), BOW+BOB+BOCST+BOS$_{porter}$ used in \citep{yu2008automatically,Cao_2010}, BOW+BOB+BOCST+BOS$_{krovetz}$, and our proposed method, the combination of BOW+BOB+BOCST+BOSDR+BOS$_{porter}$.}
\label{tab:4.3.7}
\begin{tabular}{M{4.5cm}M{1.2cm}M{1.2cm}M{1.2cm}M{1.1cm}M{1.1cm}M{1.2cm}M{1.4cm}M{1.6cm}M{1.6cm}M{1.6cm}}
\hline\noalign{\smallskip}
 \multirow{4}{*}{Topics}&\multicolumn{9}{c}{Features} \\\cmidrule(l){2-11}
&&&&&&&\multicolumn{2}{c}{\thead{\cite{yu2008automatically}\\ \cite{Cao_2010}}}&\multicolumn{2}{c}{\thead{Proposed method}}\\\cmidrule(l){8-9}\cmidrule(l){10-11}
 &\multirow{2}{*}{BOW}& \multicolumn{2}{c}{BOS}& \multirow{2}{*}{BOBNE} &\multirow{2}{*}{\thead{BOW+\\BOCST}}& \multirow{2}{*}{\thead{BOW+\\BOSDR}}& \multicolumn{2}{c}{\thead{BOW+BOB+BOCST+}}& \multicolumn{2}{c}{\thead{BOW+BOB+BOSDR+BOCST+}}\\
\cmidrule(l){3-4}\cmidrule(l){8-9}\cmidrule(l){10-11}
&  &Porter& Krovetz&&&&BOS$_{porter}$& BOS$_{krovetz}$&BOS$_{porter}$& BOS$_{krovetz}$    \\

\noalign{\smallskip}\hline\noalign{\smallskip}
Device&57.89\%&58.23\%&57.69\%&56.31\%&65.60\%&60.10\%&74.01\%&73.50\%&74.99\%&74.44\%\\
Diagnosis&74.22\%&74.24\%&73.14\%&70.38\%&75.63\%&75.19\%&77.13\%&75.33\%&78.10\%&77.05\%\\
Epidemiology&71.15\%&70.58\%&68.10\%&54.96\%&72.03\%&71.97\%&74.74\%&72.31\%&75.93\%&73.53\%\\
Etiology&80.31\%&80.67\%&78.64\%&75.05\%&81.02\%&80.95\%&82.47\%&80.07\%&83.11\%&81.67\%\\
History&52.72\%&55.69\%&55.03\%&50.96\%&58.57\%&54.31\%&67.18\%&66.52\%&68.75\%&68.09\%\\
Management&69.70\%&69.48\%&68.51\%&65.07\%&70.13\%&70.02\%&71.07\%&70.16\%&71.49\%&70.10\%\\
Pharmacological&82.41\%&82.83\%&82.16\%&76.20\%&84.66\%&83.04\%&84.71\%&84.04\%&84.85\%&84.19\%\\
Physical\&Finding&72.10\%&72.27\%&71.08\%&70.62\%&76.09\%&73.14\%&78.82\%&77.01\%&79.35\%&78.38\%\\
Procedure&69.56\%&70.08\%&68.81\%&68.18\%&75.47\%&71.32\%&78.68\%&77.42\%&79.12\%&78.35\%\\
Prognosis&72.68\%&73.68\%&72.13\%&69.27\%&73.89\%&72.87\%&74.03\%&72.51\%&74.17\%&73.61\%\\
Test&79.97\%&80.14\%&76.40\%&75.02\%&81.15\%&80.52\%&83.22\%&82.48\%&83.64\%&81.90\%\\
Treatment\&Prevention&68.19\%&69.00\%&67.40\%&65.16\%&69.99\%&69.21\%&71.73\%&70.10\%&72.63\%&71.91\%\\\cmidrule(l){1-11}
Average&70.91\%&71.41\%&69.92\%&66.43\%&73.68\%&71.89\%&76.48\%&75.12\%&77.18\%&76.10\%\\

\noalign{\smallskip}\hline
\end{tabular}
\end{table}

\end{landscape} % <-- HERE

\begin{table}[h!]
\centering
\caption{Comparison between the proposed representation (the combination of various features: BOW+BOB+BOCST+BOSDR+BOS$_{porter}$) and state-of-the-art representations on 4654 natural language clinical questions using Naive Bayes as a classifier in terms of F-score.}
\label{tab:4.3.8}
\begin{tabular}{M{2cm}M{1.2cm}M{1.2cm}M{1.2cm}M{1.2cm}M{1.2cm}M{1.3cm}M{1.5cm}M{1.7cm}}
\hline\noalign{\smallskip}
\multirow{2}{*}{Features}& \multirow{2}{*}{BOW}& \multicolumn{2}{c}{BOS}& \multirow{2}{*}{BOBNE} &\multirow{2}{*}{\thead{BOW+\\BOCST}}& \multirow{2}{*}{\thead{BOW+\\BOSDR}}& \multicolumn{2}{c}{\thead{BOW+BOB+ \\ BOCST+}}\\
\cmidrule(l){3-4}\cmidrule(l){8-9}
&  &Porter& Krovetz&&&&BOS$_{porter}$& BOS$_{krovetz}$    \\

\noalign{\smallskip}\hline\noalign{\smallskip}
F-score&65.30\%&66.41\%&65.40\%&62.78\%&68.39\%&66.84\%&70.56\%&69.49\%\\
Increase\newline performance&+6.47\%&+5.36\%&+6.37\%&+8.99\%&+3.83\%&+4.93\%&+1.21\%&+2.28\%\\
\noalign{\smallskip}\hline
\end{tabular}
\end{table}

\begin{table}[h!]
\centering
\caption{Comparison between the proposed representation (the combination of various features: BOW+BOB+BOCST+BOSDR+BOS$_{porter}$) and state-of-the-art representations on 4654 natural language clinical questions using SVM as a classifier in terms of F-score.}
\label{tab:4.3.9}
\begin{tabular}{M{2cm}M{1.2cm}M{1.2cm}M{1.2cm}M{1.2cm}M{1.2cm}M{1.3cm}M{1.5cm}M{1.7cm}}
\hline\noalign{\smallskip}
\multirow{2}{*}{Features}& \multirow{2}{*}{BOW}& \multicolumn{2}{c}{BOS}& \multirow{2}{*}{BOBNE} &\multirow{2}{*}{\thead{BOW+\\BOCST}}& \multirow{2}{*}{\thead{BOW+\\BOSDR}}& \multicolumn{2}{c}{\thead{BOW+BOB+ \\ BOCST+}}\\
\cmidrule(l){3-4}\cmidrule(l){8-9}
&  &Porter& Krovetz&&&&BOS$_{porter}$& BOS$_{krovetz}$    \\
\noalign{\smallskip}\hline\noalign{\smallskip}
F-score&70.91\%&71.41\%&69.92\%&66.43\%&73.68\%&71.89\%&76.48\%&75.12\%\\
Increase\newline
performance&+6.26\%&+5.74\%&+7.26\%&+10.75\%&+3.5\%&+5.29\%&+0.7\%&+2.06\%\\
\noalign{\smallskip}\hline
\end{tabular}
\end{table}

% Figure environment removed

% Figure environment removed
\subsection{Discussion}

While question classification has been widely investigated, few approaches are currently able to efficiently classify natural language questions into general topics, in particular for complex questions. This is a challenging task, particularly in a more specific domain such as the clinical domain.

Our experiments presented in Table~\ref{tab:4.3.6} and Table~\ref{tab:4.3.7} show that our SVM-based approach is promising for question topic classification in the context of biomedical QA. Our overall results confirm the findings presented in \cite{yu2008automatically,Cao_2010}, where among the multiple classification systems, the SVM-based one yielded the best results. If we compare our proposed method with the latter, we use the syntactic dependency relations as discriminative features to represent natural language clinical questions. The results
show that our representation which consists of BOW, BOB, BOCST, BOSDR, and BOS$_{porter}$ as features, is more effective as compared with state-of-the-art representations. As shown in Table~\ref{tab:4.3.6} and Table~\ref{tab:4.3.7} the average performance for assigning a topic to a natural language question was a 77.18\% F1-score using SVM and 71.77\% F1-score using Naïve Bayes, as opposed to the baseline of 50\% obtained by random guessing.

Table~\ref{tab:4.3.8} shows the increasing performance of the proposed method in comparison with state-of-the-art methods using Naïve Bayes as a classifier, and Table~\ref{tab:4.3.9} shows the increasing performance using SVM as a classifier. As shown in Table~\ref{tab:4.3.8}, using the Naïve Bayes classifier, our proposed representation outperforms state-of-the-art representations and leads to the highest F1-score of 71.77\%. It outperforms BOW with 6.47\%, BOS$_{porter}$ with 5.36\%, BOS$_{krovetz}$ with 6.37\%, BOBNE with 8.99\%, the combination of BOW and BOCST with 3.83\%, the combination of BOW, BOB, BOS$_{porter}$, BOCST that are used by \cite{yu2008automatically,Cao_2010} with 1.21\%. In the case of using SVM as a classifier, as we can see from Table~\ref{tab:4.3.9}, our proposed method still achieves higher F-score of 77.18\% as compared with the other representations. It generally outperforms BOW with a large margin 6.26\%, BOS$_{porter}$ with 5.74\%, BOS$_{krovetz}$ with 7.26\%, and the combination of BOW and BOCST with 3.5\%. Similarly, our proposed method has a better F-score performance than the method presented by \cite{yu2008automatically,Cao_2010} with an increase performance of 0.7\% F1-score.


As shown in Figure~\ref{fig:topics1} and Figure~\ref{fig:topics2}, inconsistency in topic attribution may be responsible for the topic training size of each topic and the topic classification performance. Typically, the classification performance depends directly on the training size: the larger the training size is the better a classifier performs. However, The Pearson correlation coefficient
between classification performance and the number of training size of a category shows an R-value of 0.2265 using SVM and 0.2396 using Naïve Bayes (weak correlation), which means that the relationship between classification performance and the number of training size is weak. Although we can see clearly from Figure~\ref{fig:topics1} and Figure~\ref{fig:topics2} that the best performing category, pharmacological category (84.85\% F-score using SVM and 79.75\% F1-score using Naïve Bayes), has the biggest number of question instances (1594) and the
worst performing category, history category (68.75\% F1-score using SVM and 55.71\% F1-score using Naïve Bayes), has the smallest number of question instances (43). Management category also has the highest number of question instances (1403), but it did not perform well (71.49\% F-score using SVM and 66.78\% F1-score using Naïve Bayes). Etiology category, on the other side, performs well (83.11\% F1-score using SVM
and 76.54\% F-score using Naïve Bayes) even though the number of instances available for training is small (173). We believe that Etiology is an unambiguous category for assignment. However, on the same data set, \cite{Cao_2010} have shown that the classification performance of clinical questions does not correlate with the number of categories assigned to the question.

Therefore, despite the noisy data, the obtained results show that our proposed method which is based on the incorporation of syntactic dependency relations with words, Porter stemmer, bigrams, UMLS Methasaurus concepts and semantic types, achieves good performance compared with the current state-of-the-art methods for clinical question topic classification.




\section{Summary of the Chapter}

In this chapter we have described the methodologies we proposed for question types and topic classification in biomedical QA. They were all based on machine learning approaches.

In section~\ref{Chapter4.2} we have presented in details the method we proposed for biomedical question type classification. This method which aims at assigning one of the four categories: yes/no, factoid, list, and summary to a natural language question in the biomedical domain, is based on combining both handcrafted lexico-syntactic patterns and machine-learning approaches. We have used the set of our handcrafted lexico-syntactic patterns to extract appropriate features for machine learning algorithms. We have experimented with several commonly used machine-learning approaches for question classification, including Naïve Bayes, Decision Tree, and SVM, and the obtained results have shown that SVM performed the best on the dataset made available as a part of the BioASQ challenge. We have also conducted experiments with different feature sets and best results were obtained using our handcrafted lexico-syntactic patterns. The predefined patterns yielding the best results are also made available which encourage replication of results (cf. section~\ref{Chapter4.2.1.2}).


In section~\ref{Chapter4.3} we have explained in details the methodology we proposed to question topic classification for the purpose of supporting automatic retrieval of clinical answers. This method aims at assigning one or more general topics to clinical questions written in natural language. It is based on various features including words, word stems, bigrams, UMLS concepts and semantic types, and syntactic dependency relations between pair words. We have explored several machine learning algorithms such as Naïve Bayes, Decision Tree, and SVM, showing SVM achieved the best results for this task on the annotated data that is released by NLM. We have also conducted experiments with different feature sets and best results were obtained using our combination of features which consists of bag-of-words, bag-of-Porter stemmer stems, bag-of-bigrams, bag-of-UMLS concepts and semantic types, and bag-of-syntactic dependency relations.



\section{Maximum Deliverable Payload Weight}\label{sec5}

In this section, we characterize the maximum weight of the payload that can be delivered from the initial point $\mathbf{U}_0$ to the final point  $\mathbf{U}_F$ under the connectivity and the battery constraints. 
By focusing on the maximum deliverable payload weight, not the minimum  delivery time,  we can formulate the optimization problem as
\begin{align} 
&\textbf{Problem 2} \cr
&\mbox{Objective:~}~~~~ \max_{w_3\geq 0, \{\mathbf{u}(t),\ \psi(t),\ t\in[0,T]\}} w_3 \label{eq:33}\\
&\mbox{Constraints: }\cr
&0\leq T<\infty \label{eq:34} \\
&\text{\eqref{eq:9}-\eqref{eq:17}}, \label{eq:35}
\end{align}
where \eqref{eq:34} means that the UAV succeeds to deliver the payload from $\mathbf{u}_0$ to $\mathbf{u}_F$ within a finite time. We note that the propulsion power consumption $P_\mathrm{UAV}(v(t))$ in \eqref{eq:15} depends on the payload weight $w_3$. 

To solve Problem 2, we propose the bottleneck edge search method described in Algorithm \ref{Algo6}.  %which first finds the longest connectivity-critical path between two CSs  and then derives the largest payload weight $w_3$ deliverable over the segment without replacing the battery. } 
%%%%%%% Algorithm 6: Bottleneck edge search method %%%%%%%%%
\begin{algorithm}[t]
\caption{Bottleneck Edge Search Method} \label{Algo6}
\textbf{Input:} $\mathbf{u}_0$, $\mathbf{u}_F$, $\mathcal{V}$, $\mathbf{c}_n$, $\ell_\mathrm{LO}(\mathbf{c}_n,\mathbf{c}_{n'})$, $h_\mathrm{Lfea}(\mathbf{c}_n,\mathbf{c}_{n'})$, $w_1$, $w_2$, $\epsilon_w$, $k_\mathrm{max}$ for $n\in [1:N+1]$, $n'\in[1:N]\cup \{N+2\}$, $n<n'$
\begin{algorithmic}[1]
\State $V_\mathrm{GL}\leftarrow\{\mathbf{u}_0,\mathbf{u}_F,\mathbf{c}_1,...,\mathbf{c}_N\}$, $E'_\mathrm{GL}\leftarrow \emptyset$, $w_3\leftarrow 0$
\State $\mathbf{c}_{N+1} \leftarrow \mathbf{u}_0$, $\mathbf{c}_{N+2} \leftarrow \mathbf{u}_F$, $h_\mathrm{sp}\leftarrow 1$
%\State $\epsilon_w\leftarrow 0.01$  \hfill\Comment{Sufficiently small positive constant}
\LeftComment{Step 1. Graph construction: Construct a graph $G'_\mathrm{GL}$ whose vertex set consists of CSs and edge set consists of edges between two connected CSs. The weight of each edge is the minimum travel distance between two CSs.}
\For {$n\in [1:N+1]$, $n'\in[1:N]\cup \{N+2\}$, $n<n'$}
    \LeftComment{Parameters $h_\mathrm{Lfea}$ and $\ell_\mathrm{LO}$ are described in Algorithm \ref{Algo4}.}
    \If{$h_\mathrm{Lfea}(\mathbf{c}_n,\mathbf{c}_{n'})=1$}
    \State $E'_\mathrm{GL}\leftarrow E'_\mathrm{GL}\cup (\mathbf{c}_n,\mathbf{c}_{n'},\ell_\mathrm{LO}(\mathbf{c}_n, \mathbf{c}_{n'}))$
    \EndIf
\EndFor
\State $G'_\mathrm{GL}\leftarrow (V_\mathrm{GL}, E'_\mathrm{GL})$ 
\LeftComment{Step 2. Bottleneck edge search: Find the longest connectivity-critical edge in $G'_\mathrm{GL}$.}
\LeftComment{Function \textbf{BFS} is described in line $1$ at Algorithm \ref{Algo4}.}
\State $h'_\mathrm{Gfea}\leftarrow$ \textbf{BFS}$(\mathbf{u}_0,\mathbf{u}_F,G'_\mathrm{GL})$
\If{$h'_\mathrm{Gfea}=1$}
\While{$h'_\mathrm{Gfea}=1$} 
    \State $(\mathbf{c}_k,\mathbf{c}_{k'},\ell_\mathrm{bott}) \leftarrow \!\!\!\!\!\!\!\!\!\!\!\!\!\!\!\!\!\!
    \underset{{(\mathbf{c}_n,\mathbf{c}_{n'},\ell_\mathrm{LO}(\mathbf{c}_n, \mathbf{c}_{n'}))\in E'_\mathrm{GL}}}{\mathrm{argmax}}$\!\!\!\!\!\!\!\!\!\!\!\!\!\!\!\!\! $\ell_\mathrm{LO}(\mathbf{c}_n, \mathbf{c}_{n'})$
    \LeftComment{Eliminate the longest edge in $E'_\mathrm{GL}$.}
    \State $E'_\mathrm{GL}\leftarrow E'_\mathrm{GL}\setminus (\mathbf{c}_k,\mathbf{c}_{k'},\ell_\mathrm{bott})$
    \State $G'_\mathrm{GL}\leftarrow (V_\mathrm{GL}, E'_\mathrm{GL})$
    \State $h'_\mathrm{Gfea}\leftarrow$ \textbf{BFS}$(\mathbf{u}_0,\mathbf{u}_F,G'_\mathrm{GL})$
\EndWhile
\Else
    \State $\ell_\mathrm{bott}\leftarrow \infty$
\EndIf
\LeftComment{Step 3. Weight search: Derive the maximum deliverable payload weight $w_3$.}
\LeftComment{Parameter $\epsilon_w>0$ is a sufficiently small constant.}
\While{$h_\mathrm{sp}=1$, $w_3\leq k_\mathrm{max}\epsilon_w$} \hfill%\Comment{\hs{$k_\mathrm{max}\in\mathbb{N}$}}
    \State $w_3\leftarrow w_3+\epsilon_w$
    \LeftComment{Function \textbf{ChkSp} is described in Algorithm \ref{Algo5}.}
    \State $(h_\mathrm{sp},v)\leftarrow$ \textbf{ChkSp}$(\ell_\mathrm{bott},\mathcal{V},w_1+w_2+w_3,w_2)$
\EndWhile
\State $w_3\leftarrow w_3-\epsilon_w$
\end{algorithmic}
\textbf{Output:} $w_3$
\end{algorithm}
%%%%%%%%%%%%%%%%%%%%%%%%
This algorithm initially sets zero payload weight, i.e., $w_3=0$. It first constructs an undirected weighted graph $G'_\mathrm{GL}$ whose vertex set consists of CSs (by
treating the initial and the final points also as CSs) and whose edge set includes an edge between two CSs only when there exists a path between the two CSs $(h_\mathrm{Lfea}=1)$, with the weight of the travel distance $\ell_\mathrm{LO}$ (in lines $3$-$8$). Note that the parameters $h_\mathrm{Lfea}$ and $\ell_\mathrm{LO}$ for each pair of CSs can be obtained by Algorithm \ref{Algo4}. After constructing the graph, it finds the bottleneck edge, which is the longest connectivity-critical edge in the graph (in lines $9$-$19$). To this end, the algorithm first checks whether $\mathbf{u}_0$ and $\mathbf{u}_F$ are connected by applying the function BFS \cite{West:2001}. If connected, it repeatedly eliminates the longest edge from the graph and then checks whether they are connected in the graph until not connected ($h'_\mathrm{Gfea}=0$). After the repetition ends, the most recently deleted edge is set as the bottleneck edge, with the edge weight $\ell_\mathrm{bott}$. Finally, the maximum deliverable payload weight over the graph is derived (in lines $20$-$24$). It first checks whether the UAV can travel the distance $\ell_\mathrm{bott}$ without replacing the battery ($h_\mathrm{sp}=1$) or not ($h_\mathrm{sp}=0$) at $w_3=0$ via the function ChkSp whose pseudo code is in algorithm \ref{Algo5}. Then, it iterates this process while increasing $w_3$ in sufficiently small increments $\epsilon_w>0$ until the UAV cannot deliver the payload over the bottleneck edge or $w_3$ exceeds the limit $k_\mathrm{max}\epsilon_w$ of the payload weight. This algorithm outputs the maximum payload weight $w_3\in\{0,\epsilon_w,...,k_\mathrm{max}\epsilon_w\}$ which can be delivered over the bottleneck edge.

Now, the following theorem shows that our bottleneck edge search method yields the optimal solution of Problem 2.\footnote{It can be shown that this method solves  Problem~2 NP-easily.}

\begin{theorem}\label{Thm6} % Optimality of bottleneck edge search method
Assume that the payload weight $w_3$ is selected from the set $\{0,\epsilon_w,...,k_\mathrm{max}\epsilon_w\}$ for $\epsilon_w>0$ and $k_\mathrm{max}\in\mathbb{N}$. Then, the bottleneck edge search method outputs the optimal solution for Problem 2 if the power consumption model $P_\mathrm{UAV}(v)$ is convex in the range of the UAV speed.  
\end{theorem}
\begin{proof}
%%%%%%% Exclusions for page limit %%%%%%
%Let us first show that the bottleneck edge  is the longest critical edge for the connection between $\mathbf{u}_0$ and $\mathbf{u}_F$ in the graph $G'_\mathrm{GL}$. It is trivially shown since $\mathbf{u}_0$ and $\mathbf{u}_F$ are still connected without the edges longer than the bottleneck edge and not connected if only the edges shorter than the bottleneck edge are included in the edge set $E'_\mathrm{GL}$ of the graph. 
If the UAV can travel the bottleneck edge without replacing the battery, then the payload can be delivered from $\mathbf{u}_0$ to $\mathbf{u}_F$ since it can be also delivered over an edge shorter than the bottleneck edge under the battery constraint. Hence, it is sufficient only to consider whether the payload can be delivered over the bottleneck edge. For finding the maximum deliverable payload weight, we note that it is sufficient only to consider a fixed speed while traveling the bottleneck edge, as justified in Theorem \ref{Thm3}. Consequently, our method yields the optimal solution for Problem 2.
\end{proof}

% Chapter 1

\chapter{Answer Extraction and End-to-End Biomedical Question Answering System SemBioNLQA} % Main chapter title

\label{Chapter6} % For referencing the chapter elsewhere, use \ref{Chapter1}
\setcounter{secnumdepth}{4}
\minitoc

This chapter presents the methods we propose for answer extraction in biomedical QA, a key task that is studied and evaluated separately. Section~\ref{Chapter6.2} will present the proposed methods for answer extraction in biomedical QA. We consecrate section~\ref{Chapter6.3} to our proposed biomedical QA system named SemBioNLQA.

\section{Introduction}
\label{Chapter6.1}

Answer extraction is usually the last component in a typical QA pipeline as the final step towards developing biomedical QA systems is processing the set of relevant passages so as to extract the final answers. Answer extraction is the most challenging task of a biomedical QA system since this is when the precise answer has to be extracted from the candidate answers retrieved and selected by the passage retrieval component. The output from the answer extraction component is a specific answer like ``Pthirus pubis'' to the biomedical question ``What is the cause of Phthiriasis Palpebrarum?''. In general, the appropriate answers to the users questions should be extracted according to the type of the given question. A biomedical question like ``Does the histidine-rich Ca-binding protein (HRC) interact with triadin?'' expects an answer of type ``yes'' or ``no''. A biomedical question like ``What is the role of edaravone in traumatic brain injury?'' expects an answer of type ``summary''. In this context, the most recent taxonomy of biomedical questions that is created by the BioASQ challenge \citep{tsatsaronis2012bioasq} consists of four types of questions which may cover all kinds of potential questions:

\begin{enumerate}
  \item Yes/No questions: They require only one of the two possible answers: ``yes'' or ``no''. For example, ``Is calcium overload involved in the development of diabetic cardiomyopathy?'' is a yes/no question and the answer is ``yes''.
  \item Factoid questions: They require a particular entity name (e.g., of a disease, drug, or gene), a number, or a similar short expression as an answer. For example, ``Which enzyme is deficient in Krabbe disease?'' is a factoid question and the answer is a single entity name ``galactocerebrosidase''.
  \item List questions: They expect a list of entity names (e.g., a list of gene names,  list of drug names), numbers, or similar short expressions as an answer. For example, ``What are the effects of depleting protein km23–1 (DYNLRB1) in a cell?'' is a list question.
  \item Summary questions: They expect a summary or short passage in return. For example, the expected answer format for the question ``What is the function of the viral KP4 protein?'' should be a short text summary.
\end{enumerate}

Since the launch of the biomedical QA track at the BioASQ challenge, theories and methods in biomedical QA continue to evolve to better meet the needs of users questions, thanks to the many editions of the BioASQ challenge. In Phase B, Task b of the BioASQ challenge, participants were asked to answer with the exact answers and the ideal answers (i.e, paragraph-sized summaries). Exact answers are only required in the case of yes/no, factoid, list, while ideal answers are expected to be returned for all biomedical questions. In this context, \cite{yang2015learning} have described a learning-based method for biomedical QA that returns only the exact answers for factoid and list questions. Similarly, \cite{peng2015fudan}} have developed a biomedical QA system that retrieves solely the exact answers for factoid and list questions. The system first used PubTator \citep{Wei_2013} for generating the candidate answers and then ranked them based on their frequency in relevant documents and snippets. \cite{choi2015snumedinfo}, on the other hand, has presented a biomedical QA system which retrieves only the ideal answers for each given question. Meanwhile, \cite{neves2015hpi} has proposed a biomedical QA system named HPI based on the IMDB database and its built-in text analysis features to generate both the exact and the ideal answers for biomedical questions. \cite{schulze2016hpi}, the winning team of the 2016 BioASQ challenge, have presented a biomedical QA system based on the LexRank algorithm \citep{erkan2004lexrank} to retrieve only the ideal answers to biomedical questions. Although the aforementioned systems have proven to be quite successful at answering biomedical questions, biomedical answer extraction still requires further efforts in order to improve its performance as the most of the aforementioned systems do not deal with all question and answer types. For instance, only few answer extraction methods for biomedical yes/no questions have been presented, compared to other question types such as factoid, list, and summary. The most recent yes/no biomedical answer extraction method is the one which was presented by \cite{neves2015hpi}. The author has made a decision on either the answer is ``yes'' or ``no'' based on the sentiment analysis predictions provided by the IMDB technology.


On the other hand, despite the importance of answering biomedical questions, until recently there are only few integral
biomedical QA systems such as the ones described in \citep{lee2006beyond,cruchet2009trust,gobeill2009question,Cao_2011,abacha2015means,Kraus_2017} that can retrieve answers to biomedical questions written in natural language. While these systems have proven to be quite successful at answering biomedical questions, they provide a limited amount of question and answer types (cf. Table~\ref{tab:3.1c}), for instance, most of them  \citep{lee2006beyond,cruchet2009trust,Cao_2011,Kraus_2017} only handle definition questions or returns solely short summaries as answers for all types of questions, and the other ones do not deal with yes/no questions which are one of the most complicated question types to answer as they are seeking for a clear ``yes'' or ``no'' answer. Furthermore, such systems still require further efforts in order to improve their performance in terms of precision to currently supported question and answer types.

In this thesis work, compared to the aforementioned systems, our ultimate goal is to develop a biomedical QA system that is able to accept a variety of natural language questions and to generate appropriate natural language answers. We will present in section~\ref{Chapter6.2} the methods we propose for the extraction of the answers to biomedical questions including yes/no questions, factoid questions, list questions, and summary questions. In section~\ref{Chapter6.3} we will present our fully automated system SemBioNLQA - Semantic Biomedical Natural Language Question Answering - which has the ability to handle the aforementioned questions that are commonly asked in the biomedical domain. SemBioNLQA is derived from our previously established methods in (1) question classification (2) document retrieval, (3) passage retrieval, and (4) answer extraction components.


\section{Answer Extraction}
\label{Chapter6.2}
\subsection{Methods}

As our goal is to develop a biomedical QA system which has the ability to deal with four types of questions (i.e., yes/no questions,
factoid questions, list questions, and summary questions), therefore, we developed novel answer extraction methods for each question type. The proposed biomedical QA system will be able to retrieve quickly user' information needs by returning exact answers (e.g., ``yes'', ``no'', a biomedical entity name, etc.) and ideal answers (i.e., paragraph-sized summaries of relevant information) for yes/no, factoid and list questions,
whereas it only provides the ideal answers for summary questions. In this section, we describe in details the methods we propose for each of the aforementioned questions \citep{Sarrouti_yes_2017,Sarrouti_bioasq_2017}.

\subsubsection{Yes/no questions}

Yes/no questions, these are questions that require either ``yes'' or ``no'' as exact answer like ``yes'' to the biomedical yes/no question ``Does the CTCF protein co-localize with cohesin?''. Even though there are only two possible answers, ``yes'' or ``no,'' such questions can be quite hard to answer due to the complicated sentiment analysis process of the candidate answer passages. In this thesis work, in order to answer yes/no questions, we first used the Stanford CoreNLP tools for tokenization and part-of-speech tagging one by one the $N$ retrieved candidate passages answers $(p_1, p_2, ... , p_{n})$. Then, each word of the candidate answer passages is assigned its SENTIWORDNET score. SENTIWORDNET \footnote{SENTIWORDNET: \url{http://sentiwordnet.isti.cnr.it/}} which is a lexical resource for sentiment analysis and opinion mining \citep{baccianella2010sentiwordnet}, assigns to each synset of WordNet three sentiment scores: ``positivity'', ``negativity'', ``objectivity''. Assuming that there are $k$ words (w) in a candidate answer passage $p$, the final sentiment score (SC) of the candidate answer passage is defined by the following equation:

\begin{equation}\label{eq:1}
SC \;(p)= \sum_{i=1}^{k} SentiWordNet \; (w_i)
\end{equation}

The key idea behind using the SentiWordNet 3.0 lexical resource is that it is the result of the automatic annotation of all the synsets of WordNet 3.0 according to the notions of ``positivity,'' ``negativity,'' and ``neutrality''. The WordNet which is a large lexical database of English, is comprised of 155,287 words and 117,659 synsets, also called synonyms \citep{Miller_1995}. Furthermore, \cite{marchand2013lvic} have shown that SentiWordNet 3.0 outperforms other sentiment lexicons in the determination of the polarity, such as Bing Liu's Opinion Lexicon \citep{Hu_2004} and MPQA Subjectivity Lexicon \citep{Wilson_2005}.


Finally, the decision to output ``yes'' or ``no'' depends on the number of positive or negative candidate answers: ``yes'' for a positive final sentiment candidate answers score and ``no'' for a negative one. Algorithm~\ref{alg:2} further illustrates how the proposed yes/no answer extraction method works.


\begin{algorithm}[h!]
\caption{Biomedical yes/no answer generator}
\label{alg:2}
\begin{algorithmic}[1]
%\algnewcommand\INPUT{\item[\textbf{Input:}]}%
\State $\textbf{Input} : {yes/no\; question \;Q\; and\; set\;of\;candidate\;answers\; P}$
\State $\textbf{Output} : {answer: \;``yes"\;or\; ``no"}$
\State $postive \leftarrow {0}$ \Comment{number of positive candidate answer passages}
\State $negative \leftarrow {0}$ \Comment{number of negative candidate answer passages}
\State $i \leftarrow {1}$
\Function{PreProcessing}{$p: condidate\;answer\;passage$}
\State $TOKEN [1...m]\leftarrow \Call{TokenizationAndPOSTagging}{$p$}$
\State \Return $TOKEN$
\EndFunction

\Do
  \State $W[1...m] \leftarrow  \Call{PreProcessing} {P[i]}$ \Comment{get a set of words and their POS tags of a candidate answer}
  \State $ score \leftarrow 0.0 $
  \State $j \leftarrow {1}$
  \Do
    \State $score \gets score+ \Call{SentiWordNet} {$W[j]$}$
    \State $j \gets j+1$

  \doWhile{($j\leq m$)} \Comment{m is the size of the set of words W}
   \If{($score \geq 0$)}
        \State $positive \gets positive+1$
   \Else
   \State $negative \gets negative+1$
   \EndIf
\State $i \gets i+1$
\doWhile{($i\leq np$)} \Comment{np is number of candidate answers}

\If{($positive \geq negative$)} \Comment{the decision for the answers ``yes'' or ``no'' is based on the number of positive and negative candidate answers}
        \State $output \gets ``yes"$
\Else
 \State $output \gets ``no"$
\EndIf
\end{algorithmic}
\end{algorithm}

Figure~\ref{fig:sc} shows an example of the whole process, i.e, tokenization, part-of-speech tagging, and sentiment score assignment for a candidate answer passage of the biomedical yes/no question ``Does the CTCF protein co-localize with cohesin?'' from the BioASQ training datasets.

% Figure environment removed

\subsubsection{Factoid questions}

Factoid questions are the questions that expect a particular entity name (e.g., of a disease, drug, or gene), a number, or a similar short expression as an answer like ``Cysteine'' to the biomedical factoid question ``Which amino acid residue appears mutated in most of the cases reported with  cadasil syndrome?''.

To achieve the goal of answering factoid questions in our proposed biomedical QA system, we have proposed a factoid answer extraction method based on UMLS metathesaurus, BioPortal synonyms and the term frequency metric. We first have mapped the $N$ candidate answers $(p_1, p_2, ... , p_{n})$ of a given biomedical factoid question to the UMLS metathesaurus (2016AA knowledge source) using the MetaMap program so as to extract the set of biomedical entity names $Es$. We then have ranked the obtained set of biomedical entity names based on the term frequency metric $TF(e_i, Es)$, the number of times entity name $e_i$ appeared in the set of biomedical entity names $Es$. We have explored several term weighting methods such as TFIDF and BM25, showing term frequency achieved the best result for this task. We speculate that the answers are located in the first and second candidate answers. Next, synonyms for each of the $T$ top-ranked entity names are retrieved using Web services from BioPortal\footnote{\url{http://data.bioontology.org/documentation}}. Finally, the $T$ top-ranked biomedical entity names and their $T$ top synonyms are displayed as answers, excluding entities also mentioned in the question. The idea behind excluding entities mentioned in the question is that after analysing the training set of questions and answers released by the BioASQ organizers, we found that the most entities that appear in questions are not part of the answers. For example, the answer of the factoid question ``What is the name of Bruton's tyrosine kinase inhibitor that can be used for treatment of chronic lymphocytic leukemia?'' which contains several entities (e.g., ``Chronic Lymphocytic Leukemia''), is ``Ibrutinib'' which is not part of the question entities. As described by the BioASQ challenge, a factoid question has one correct answer, but up to five candidate answers and their synonyms are allowed. Figure~\ref{fig:mapping} shows an example of the whole process, i.e, mapping to UMLS metathesaurus, and synonyms extraction for a candidate answer of the factoid question ``Which type of lung cancer is afatinib used for?''.

% Figure environment removed


\subsubsection{List questions}

List questions are the questions which require a list of entity names (e.g., a list of gene names,  list of drug names), numbers, or similar short expressions as an answer like ``bortezomib'', ``vincristine'', ``doxorubicin'', ``etoposide'', ``cisplatin'', ``fludarabine'', and ``SD-1029 Stat3 inhibitor'' to the biomedical question ``Which drugs have been found effective for the treatment of chordoma?''. The main difference between factoid and list questions is that the former require a single list of answers while the latter expect a list of lists of entity names, numbers, or similar short expressions. As it is shown by the BioASQ challenge, each entity may be accompanied by a list of synonyms. Therefore, a list of entities should be provided for each question by the proposed biomedical QA system. In other words, the exact answer is the same of factoid questions, but the interpretation is different for list questions: All $T$ top-ranked entities are considered part of the same answer for the list question, not as candidates. The proposed method used to answer list questions in our system is similar to the one described for factoid questions.


\subsubsection{Summary questions}

Summary questions are the questions which expect a summary or short passage in return like ``\emph{The histidine-rich Ca-binding protein (HRC), a 165 kDa sarcoplasmic reticulum (SR) protein, regulates SR Ca cycling during excitation contraction coupling.  HRC mutations or polymorphisms lead to cardiac dysfunction.  The Ser96Ala genetic variant of HRC is associated with life-threatening ventricular arrhythmias and sudden death in idiopathic dilated cardiomyopathy (DCM)}'' to the biomedical question ``What is the role of the histidine rich calcium binding protein (HRC) in cardiomyopathy?''. As summary questions do not require exact answers, therefore, they are simply answered in this thesis work by formulating short summaries, i.e., ideal answers, of relevant information. For the given biomedical questions, the ideal answers are formed by concatenating the two top-ranked candidate answers which were retrieved by the proposed passage retrieval approach (cf. section~\ref{Chapter5.3}). We first have forwarded abstracts of the $N$ relevant documents of a given biomedical question to Stanford CoreNLP sentence splitter so as to segment them into sentences. We then have preprocessed the obtained set of candidate answers including tokenization, removing stop words, and applying Porter' stemmer to extract stemmed words. Additionally, we have used the MetaMap program for mapping both biomedical questions and candidate passages to UMLS concepts in order to extract biomedical concepts. Moreover, the MetaMap word sense disambiguation system has been used to resolve ambiguities in the texts by identifying the meaning of ambiguous terms. Using stemmed words and UMLS concepts as features, we finally ranked the set of candidate answers using BM25 as retrieval model, and concatenated the two top-ranked candidate answer passages.

In particular, in addition to the \emph{exact answers} returned for previous questions, i.e., yes/no questions, factoid questions and list questions, we also provide \emph{ideal answers}. Therefore, our proposed biomedical QA system provides both \emph{exact answers} and the \emph{ideal answers} for yes/no, factoid and list questions, whereas it only provides \emph{ideal answers} for summary questions.

\subsection{Experimental results}

In order to assess the effectiveness of the methods we proposed for biomedical answer extraction and compare with the current state-of-the-art methods, we performed several experiments on large standard datasets provided by the BioASQ challenge. We have used the test sets of biomedical questions provided in BioASQ Task 3b 2015, BioASQ Task 4b 2016, and BioASQ Task 5b 2017 described in subsection~\ref{Chapter3_7_1}, section~\ref{Chapter3_7}, chapter~\ref{Chapter3}. Moreover, we have participated in the 2017 BioASQ challenge (BioASQ Task 5b 2017).

The BioASQ challenge in phase B of Task 3b/4b/5b provides the test set of biomedical questions along with their golden documents, golden snippets, and questions types, i.e., whether yes/no, factoid, list or summary in order to evaluate biomedical answer extraction approaches in their best way. Given biomedical question and its golden passages, each participating system may return an ideal answer, i.e., a paragraph-sized summary of relevant information. In the case of yes/no, factoid, and list questions, the systems may also return exact answers; for summary questions, no exact answers will be returned. Accordingly, we have relied on the gold-standard passages provided by the BioASQ challenge, instead of the ones retrieved by the system to evaluate our answer extraction methods.

As indicators of answer extraction effectiveness: Accuracy was used for exact answers of yes/questions; mean reciprocal rank (MRR) was used for exact answer of factoid questions; mean average precision, mean average recall, and mean average f1-measure were used for exact answers of list questions; ROUGE-2 and ROUGE-SU4 were used for ideal answers. These evaluation metrics are described in details in subsection~\ref{Chapter3_7_2}, section~\ref{Chapter3_7}, chapter~\ref{Chapter3}. Additionally, the BioASQ challenge have also developed an online evaluation system\footnote{BioASQ evaluation system: \url{http://participants-area.bioasq.org/oracle/}} that allows uploading JSON result files and obtaining evaluations results at any time. Table~\ref{tab:6.1} and Table~\ref{tab:6.2} show the experimental results of the
proposed answer extraction methods and comparison with the state-of-the-art studies presented in \citep{zhang2015fudan,neves2015hpi,choi2015snumedinfo,yang2015learning,schulze2016hpi} on five batches of testing datasets provided by the BioASQ challenge in 2015 and 2016, respectively. In addition, Table~\ref{tab:6.3} and Table~\ref{tab:6.4} present the results of our participation in Phase B, Task 5b of the 2017 BioASQ challenge using our biomedical answer extraction system. The values inside parameters indicate our current rank, the total number of submissions, and the total number of participated teams for the task. Our system name for submission was ``sarrouti''.



\begin{table}[h!]
\centering
\caption[The overall results of the proposed biomedical answer extraction methods and comparison with the current state-of-the-art
methods on five batches of testing datasets provided by BioASQ 3b 2015.]{The overall results of the proposed biomedical answer extraction methods and comparison with the current state-of-the-art methods on five batches of testing datasets provided by BioASQ 3b 2015. The ``-'' replace the scores of systems that did not evaluate on this batch or did not deal with this task, while ``nr'' indicated that the results are not reported for this evaluation measure. Acc, P, R, F, R-2, R-SU4 indicate accuracy, precision, recall, and f-measure, rouge-2, rouge-SU2, respectively.}
\label{tab:6.1}
\begin{tabular}{p{1.6cm}p{3.6cm}p{1.1cm}p{1cm}p{1cm}p{1cm}p{1cm}p{1cm}p{1.2cm}}
\hline\noalign{\smallskip}
 \multirow{3}{*}{Datasets} &\multirow{3}{*}{System name}& \multicolumn{5}{c}{Exact answers} &  \multicolumn{2}{c}{\multirow{2}{*}{Idial answers}}\\
\cmidrule(l){3-7}

 & & Yes/No & Factoid & \multicolumn{3}{c}{ List}  &\multicolumn{2}{c}{}   \\
 \cmidrule(l){3-3}\cmidrule(l){4-4} \cmidrule(l){5-7} \cmidrule(l){8-9}
  & &  Acc &  MRR& P& R &F1  & R-2& R-SU4\\


\noalign{\smallskip}\hline\noalign{\smallskip}
\multirow{4}{*}{Batch 1} &	Our system & 0.6970&	0.1692	& 0.1545&	0.2409&	0.1830& 0.2716&	0.2860\\
            &\cite{zhang2015fudan}&-&	0.1423	& nr & nr & 0.0756& -& -\\
            &\cite{neves2015hpi}&0.6667&	-	& 0.0292& 0.0603& 0.0364& 0.1884& 0.2008\\
            &\cite{choi2015snumedinfo}&-&	-	& -& -& -& -& 0.3071\\
\cmidrule(l){1-9}
\multirow{4}{*}{Batch 2} &	Our system &0.6250&	0.1776	& 0.1929&	0.2714&	0.2127& 0.3123&	0.3364\\
           &\cite{zhang2015fudan}&-&	0.0859	& nr&nr& 0.1160& -& -\\
            &\cite{neves2015hpi}&0.5625&	-	& 0.0714& 0.0161& 0.0262& 0.2026& 0.2227\\
            &\cite{choi2015snumedinfo}&-&	-	& -& -& -& -& 0.3710\\
            \cmidrule(l){1-9}
\multirow{4}{*}{Batch 3} &	Our system &0.8621&	0.1840	& 0.2353&	0.2927&	0.2524& 0.3879&	0.4078\\
            &\cite{zhang2015fudan}&-&	0.0846	& nr& nr& 0.1319& -& -\\

            &\cite{yang2015learning}&-&	0.1615&	0.0539&  0.6933 & 0.0969 & -& -\\

            &\cite{neves2015hpi}&0.6207&	-	& -& -& -& 0.1934& 0.2189\\
            &\cite{choi2015snumedinfo}&-&	-	& -& -& -& -& 0.3941\\
            \cmidrule(l){1-9}
\multirow{4}{*}{Batch 4} &	Our system &0.7600&	0.2960	& 0.2783&	0.2713&	0.2588& 0.3917&	0.4108\\
           &\cite{zhang2015fudan}&-&	0.2299	& nr& nr& 0.2192& -& -\\

           &\cite{yang2015learning}&-&	0.5155&	0.3836&  0.3480 & 0.3168 & -& -\\

            &\cite{neves2015hpi}&0.5600&	0.0345	& 0.1522& 0.0473& 0.0689& 0.2504& 0.2724\\
            &\cite{choi2015snumedinfo}&-&	-	& -& -& -& -& 0.3906\\
            \cmidrule(l){1-9}
\multirow{4}{*}{Batch 5} &	Our system &0.6071&	0.1568	& 0.0583& 0.0736& 0.0625& 0.3440&	0.3533\\
            &\cite{zhang2015fudan}&-&	0.2500	& nr& nr& 0.1340& -& -\\

            &\cite{yang2015learning}&-&	0.2727 &	0.1704&  0.2573 & 0.1875 & -& -\\

            &\cite{neves2015hpi}&0.3571&	0.0909	& 0.0625& 0.0292& 0.0397& 0.1694& 0.1790\\
            &\cite{choi2015snumedinfo}&-&	-	& -& -& -& -& 0.3665\\
\noalign{\smallskip}\hline
\end{tabular}
\end{table}


\begin{table}[h!]
\centering
\caption[The overall results of the proposed biomedical answer extraction methods and comparison with the current state-of-the-art
methods on five batches of testing datasets provided by BioASQ 4b 2016]{The overall results of the proposed biomedical answer extraction methods and comparison with the current state-of-the-art methods on five batches of testing datasets provided by BioASQ 4b 2016. The ``-'' replace the scores of systems that did not deal with this task, while ``nr'' indicated that the results are not reported for this evaluation measure. Acc, P, R, F, R-2, R-SU4 indicate accuracy, precision, recall, and f-measure, rouge-2, rouge-SU2, respectively.}
\label{tab:6.2}
\begin{tabular}{p{1.6cm}p{3.6cm}p{1.1cm}p{1cm}p{1cm}p{1cm}p{1cm}p{1cm}p{1.2cm}}
\hline\noalign{\smallskip}
 \multirow{3}{*}{Datasets} &\multirow{3}{*}{System name}& \multicolumn{5}{c}{Exact answers} &  \multicolumn{2}{c}{\multirow{2}{*}{Idial answers}}\\
\cmidrule(l){3-7}

 & & Yes/No & Factoid & \multicolumn{3}{c}{ List}  &\multicolumn{2}{c}{}   \\
 \cmidrule(l){3-3}\cmidrule(l){4-4} \cmidrule(l){5-7} \cmidrule(l){8-9}
  & &  Acc &  MRR& P& R &F1  & R-2& R-SU4\\


\noalign{\smallskip}\hline\noalign{\smallskip}
\multirow{2}{*}{Batch 1} &	Our system &0.8214&	0.0726	& 0.2182&	0.3939&	0.2756& 0.4772&	0.4918\\
            &\cite{schulze2016hpi}&-&	-	& -& -& -& nr& 0.2231\\

\cmidrule(l){1-9}
\multirow{2}{*}{Batch 2} &	Our system &0.8750&	0.1452	& 0.2381&	0.2505&	0.2349& 0.5021&	0.5115\\
             &\cite{schulze2016hpi}&-&	-	& -& -& -& nr&0.2240 \\

            \cmidrule(l){1-9}
\multirow{2}{*}{Batch 3} &	Our system &0.8400&	0.1218	& 0.2381&	0.3627&	0.2812& 0.4978&	0.5061\\
            &\cite{schulze2016hpi}&-&	-	& -& -& -& nr& 0.2559\\

            \cmidrule(l){1-9}
\multirow{2}{*}{Batch 4} &	Our system &0.8095&0.1129&	0.1467&	0.2231&	0.1702& 0.5192&	0.5231\\
            &\cite{schulze2016hpi}&-&	-	& -& -& -& nr& 0.2280\\

            \cmidrule(l){1-9}
\multirow{3}{*}{Batch 5} &	Our system &0.8148& 0.1136&	0.1900&	0.2353&	0.1963& 0.4979&	0.5027\\
            &\cite{schulze2016hpi}&-&	-	& -& -& -& nr& 0.3233\\


\noalign{\smallskip}\hline
\end{tabular}
\end{table}




\begin{table}[h!]
\centering
\caption[The obtained results of our participation in ``Exact Answers'', Phase B, Task 5b of the 2017 BioASQ challenge using the proposed answer extraction methods]{The obtained results of our participation in ``Exact Answers'', Phase B, Task 5b of the 2017 BioASQ challenge using the proposed answer extraction methods. The first value inside parameters indicates our current rank and the total number
of submissions for the task, while the second indicates our current rank and the total number of participated teams.}
\label{tab:6.3}
\begin{tabular}{p{2cm}p{2.6cm}p{2.6cm}p{2cm}p{2cm}p{2.5cm}}
\hline\noalign{\smallskip}
 \multirow{2}{*}{Datasets} &  Yes/No & Factoid & \multicolumn{3}{c}{ List}   \\
 \cmidrule(l){2-2}\cmidrule(l){3-3} \cmidrule(l){4-6}
  &   Accuracy &  MRR& Precision& Recall &F-measure \\


\noalign{\smallskip}\hline\noalign{\smallskip}
\multirow{2}{*}{Batch 1}  &0.7647&	0.2033 (5/15)\newline (3/9)&	0.1909&	0.2658&	0.2129 (3/15)\newline (2/9) \\

\cmidrule(l){1-6}
\multirow{2}{*}{Batch 2}  &0.7778& 0.0887 (10/21)\newline (5/9)&	0.2400&	0.3922&	0.2920 (6/21)\newline (2/9)\\

            \cmidrule(l){1-6}
\multirow{2}{*}{Batch 3}  &0.8387 (1/21)\newline (1/10)& 0.2212 (9/21)\newline (4/10)&	0.2000&	0.4151&	0.2640 (6/21)\newline (3/10)\\


            \cmidrule(l){1-6}
\multirow{2}{*}{Batch 4} &0.6207 (2/27)\newline (2/11)& 0.0970 (13/27)\newline (5/11)&	0.1077&	0.2013&	0.1369 (12/27)\newline (5/11)\\


            \cmidrule(l){1-6}
\multirow{3}{*}{Batch 5} &0.4615 & 0.2071 (9/25)\newline (3/11)&	0.2091&	0.3087&	0.2438 (11/25)\newline (6/11)\\



\noalign{\smallskip}\hline
\end{tabular}
\end{table}

\begin{table}[h!]
\centering
\caption[The obtained results of our participation in ``Ideal Answers'', Phase B, Task 5b of the 2017 BioASQ challenge using the proposed answer extraction methods]{The obtained results of our participation in ``Ideal Answers'', Phase B, Task 5b of the 2017 BioASQ challenge using the proposed answer extraction methods. The first value inside parameters indicates our current rank and the total number
of submissions for the task, while the second indicates our current rank and the total number
of participated teams.}
\label{tab:6.4}
\begin{tabular}{p{2.1cm}p{1.7cm}p{2.2cm}p{2.2cm}p{1.3cm}p{1.9cm}p{1.9cm}}
\hline\noalign{\smallskip}
 \multirow{2}{*}{Datasets} & \multicolumn{2}{c}{Automatic scores} &  \multicolumn{4}{c}{Manual scores}\\
\cmidrule(l){2-3}\cmidrule(l){4-7}

 & Rouge-2& Rouge-SU4 & Readability&  Recall& Precision& Repetition \\


\noalign{\smallskip}\hline\noalign{\smallskip}
\multirow{2}{*}{Batch 1}  &0.4943 \newline(4/15)\newline(2/9)&0.5108\newline (3/15)\newline(2/9)&3.65 \newline(3/15)\newline(2/9)&	4.42 \newline (3/15)\newline(2/9)&	3.90 \newline(3/15)\newline(2/9)&	3.89\newline (3/15)\newline(2/9)\\

\cmidrule(l){1-7}
\multirow{2}{*}{Batch 2}  & 0.4579 \newline(4/21)\newline (2/9)& 0.4583\newline (4/21)\newline (2/9)&3.68\newline (6/21)\newline (2/9)&	4.59 \newline (2/21)\newline (2/9)&	4.01\newline (7/21)\newline (2/9)&	3.91 \newline(7/21)\newline (2/9)\\

            \cmidrule(l){1-7}
\multirow{2}{*}{Batch 3}  & 0.5566 \newline(4/21)\newline (2/10)&	0.5656 \newline(4/21)\newline (2/10)&3.91\newline (6/21)\newline (2/10)&	4.64 \newline (1/21)\newline (2/10)&	4.07 \newline(7/21)\newline (2/10)&	4.00 \newline(7/21)\newline (2/10)\\


            \cmidrule(l){1-7}
\multirow{2}{*}{Batch 4} &0.5895 \newline(4/27)\newline (2/11)&	0.5832 \newline(4/27)\newline (2/11)&3.86\newline (7/27)\newline (3/11)&	4.51 \newline (6/27)\newline (3/11)&	4.02 (3/27)\newline (2/11)&	3.95 \newline(6/27)\newline (2/11)\\


            \cmidrule(l){1-7}
\multirow{3}{*}{Batch 5} & 0.5772 \newline(7/25)\newline (3/11)&	0.5756\newline (7/25)\newline (3/11)&3.82\newline (8/25)\newline (3/11)&	4.53 \newline(5/25)\newline (2/11)	&3.91\newline (7/25)\newline (3/11)&	3.90 \newline(7/25)\newline (2/11)\\



\noalign{\smallskip}\hline
\end{tabular}
\end{table}


The proposed answer extraction methods provide both the exact and ideal answer to biomedical questions. For yes/no questions, the exact answer (``yes'' or ``no'') is formed by using SentiWordNet, a sentiment lexicon. Each word of the relevant snippets is assigned its SentiWordNet score, and the decision to output ``yes'' or ``no'' depends on the number of positive or negative snippets. For factoid questions, the exact answer is produced by identifying the biomedical entities that occur in the given top relevant snippets of the question, and reporting the five most frequent biomedical entities and their synonyms of the top snippets, excluding entities also mentioned in the question. For list questions, the exact answer is produced in the same manner, except that the most frequent entities and their synonyms of the top relevant snippets are now returned as the single list that answers the question. Note that in contrast to the 2015 and 2016 BioASQ challenges, biomedical QA systems are no longer allowed to provide an own list of synonyms in the 2017 challenge. On the other hand, to generate ideal answers (summaries),
we have applied MetaMap to the relevant snippets and the question, in order to obtain the UMLS concepts they refer to. We then have ranked the snippets by their BM25 similarity to the question (using Porter stems and UMLS concepts as features), and return the top two most highly ranked snippets (concatenated) as the ideal answer.

From the overall results, it can be seen that in each of the BioASQ challenges, the proposed biomedical answer extraction methods were very competitive and performed well in all challenges ranking within the top tier teams. Moreover, our system was one of the winners\footnote{Fifth BioASQ challenge winners:\url{http://www.bioasq.org/participate/fifth-challenge-winners}} in the 2017 edition of the BioASQ challenge. 


%% Figure environment removed


\subsection{Discussion}

Although open-domain QA is a longstanding challenge widely studied over the last decades, few systems are currently able to handle a variety of natural language questions and to generate the appropriate answers.  In this thesis work, we have proposed answer extraction methods in biomedical QA to handle the kinds of yes/no questions, factoid questions, list questions, and summary questions. Our methods are able to provide exact answers and paragraph-sized ideal answers (summaries of relevant information) for yes/no, factoid and list questions, whereas they only retrieve ideal answers for summary questions.

The experimental results detailed in Table~\ref{tab:6.1} and Table~\ref{tab:6.2} have shown that the proposed biomedical answer extraction methods are more competitive as compared with the current state-of-the-art methods. As can be seen from Table~\ref{tab:6.1}, compared with the two methods presented in \citep{neves2015hpi,zhang2015fudan}, one based on the in-memory database and another using the PubTator tool, our methods significantly outperformed the aforementioned methods in extracting both the exact answers and the ideal answers for yes/no, factoid, list and summary biomedical questions. Moreover, the increased performance was statistically significant (the P-value is 7.6e-05, the result is significant at p < 0.01). In particular, it can be seen clearly from Table~\ref{tab:6.1} that in all batches of testing datasets, the proposed
yes/no answer extraction method achieves performance improvements over the state-of-the-art method
presented in \citep{neves2015hpi}. Compared with the latter, which employs the sentiment analysis predictions
provided by the IMDB database, the proposed method gives better results (an average improvement of 15.68\% in terms of accuracy). Moreover, the increase performance is statistically significant (the p-value is < 0.00001, the result is significant at p < 0.01).


Additionally, the proposed systems is also competitive compared with the system proposed in \citep{choi2015snumedinfo}, which dealt only with the ideal answers (0.2860 against 0.3071, 0.3364 against 0.3710, 0.4078 against 0.3941, 0.4108 against 0.3906, and 0.3533 against 0.3665 of Rouge-SU4 in batch 1, batch 2, batch 3, batch 4, and batch 5 respectively).

In the 2016 BioASQ challenge, as shown in Table~\ref{tab:6.2}, the proposed answer extraction system still achieves good performance compared with the 2016 winning system developed by \citep{schulze2016hpi} which dealt solely with the ideal answers of questions. The latter is based on the LexRank algorithm \citep{erkan2004lexrank}, but that solely used the named entities for the similarity function. The important thing to note here is that our system significantly outperforms the \citep{schulze2016hpi} system in all batches of testing datasets. The largest difference in ROUGE-SU4 between SemBioNLQA and the aforementioned system was 0.2951 (0.5231 - 0.2280 in batch 4) which clearly indicates that our method is not only effective but robust in extracting the ideal answers.

On the other hand, as part of our participation in Phase B, Task 5B of the 2017 BioASQ challenge, the proposed answer extraction methods performed well in the challenge ranking within the top tier teams as shown in Table~\ref{tab:6.3} and Table~\ref{tab:6.4}. More details on the results can be found in the official web site of BioASQ\footnote{BioASQ Task5b phase b\url{http://participants-area.bioasq.org/results/5b/phaseB/}}. A total of 16, 22, 21, 27, and 25 runs were submitted for batch 1, batch 2, batch 3, batch 4, batch 5 respectively. Note that many runs were submitted by the same teams. As shown, our team is ranked among the top three. In batch 1, it achieved the third and the fifth
position within the 15 participating systems in extracting the exact answers of list and factoid questions respectively. More specifically, our system obtained the second and the third position when considering results by teams, instead of each individual run. In batch 2, considering results by teams, our system obtained the second and the fourth position in extracting the exact answers of list and factoid questions respectively. It can also be seen that in batch 3 and batch 4, our system, achieved the first and the second position respectively for answering
yes/no questions. Because of a very skewed class distribution in other batches, we have not compared our yes/no answer extraction method with
the participant systems which always answering ``yes''. For the ideal answers, our system in terms of ROUGE-2 achieved the fourth position compared to the 15, 21, and 21 participating systems in batch 1, batch 2 and batch 3 respectively, while in terms of ROUGE-SU4, the proposed system obtained the third position in batch 1 and the fourth position in batch 2. Besides, considering results by teams, instead of each individual run, our systems achieved the second position in terms of ROUGE-2 and ROUGE-SU4 in batches 1-4, whereas it achieved the third position in batch 5. Based on the manual scores in terms of readability, recall, precision, and repetition calculated by the BioASQ experts for each participating systems, our system achieved the second position in batch 1, batch 2, batch 3, batch 5, while it achieved the third position in batch 4. Overall, our system was one of the fifth BioASQ challenge winners\footnote{BioASQ challenge winners: \url{http://www.bioasq.org/participate/fifth-challenge-winners}}. This proves that the proposed answer extraction system could effectively identify the ideal answers to a given biomedical question.

Although the proposed answer extraction system could effectively answer a variety of biomedical questions, we found that there are still some mistakes that the proposed system cannot fix. An example is that, when the question is ``Does HER2 under-expression lead to favorable response to trastuzumab?,'' (identifier 51542eacd24251bc05000084) from BioASQ training questions, a positive answer might be ``Trastuzumab is a monoclonal antibody targeted to the Her2 receptor and approved for treatment of Her2 positive breast cancer,'' the sentiment score considers this answer to be a positive as there are both ``positive'' and ``approved'' in the passage. However, this example should be counted as a ``no'' answer because trastuzumab is effective only in cancers where Her2 is over-expressed. Dealing with such problems requires more complex semantic analysis, and we may need to parse the passage to get a grammar tree. Nevertheless, parsing and semantic analysis will bring in new errors and make this challenge even more complicated. We also found that the current form of the proposed system was not able to provide answers to some questions especially for these which expect a number as answer instead of biomedical entities. For instance, the answers for the biomedical question ``What is the prevalence of short QT syndrome?'' (identifier 52fb78572059c6d71c000067)  and ``What is the number of protein coding genes in the human genome?'' (identifier 535d3c069a4572de6f000006) collected from BioASQ training questions,  are ``0.01\% -0.1\%'', ``Between 20,000 and 25,000'', respectively. Such questions seem to be quite complicate and need more specific information extraction methods.


\section{A Semantic Biomedical Question Answering System SemBioNLQA}
\label{Chapter6.3}
\subsection{Methods}

In this section, we present the development, generic architecture and integrated components of our fully automated system SemBioNLQA - Semantic Biomedical Natural Language Question Answering - which has the ability to handle the kinds of yes/no questions, factoid questions, list questions and summary questions that are commonly asked in the biomedical domain. Figure~\ref{fig:interface} presents the SemBioNLQA Web system.

% Figure environment removed

The SemBioNLQA system, which consists of question classification, document retrieval, passage retrieval and answer extraction components, takes natural language questions as input, and outputs both \emph{exact answers} and \emph{ideal answers} as results.  SemBioNLQA is able to accept a variety of natural language questions and to generate appropriate natural language answers by providing both exact and ideal answers. It provides exact answers of type ``yes'' or ``no'' for yes/no questions, biomedical named entities for factoid questions, and a list of biomedical named entities for list questions. In addition to exact answers for yes/no, factoid and list questions, SemBioNLQA also returns ideal answers, while it retrieves only the ideal answers for summary questions. SemBioNLQA is derived from our previously established methods  in (1) question classification (cf. section~\ref{Chapter4.2}, chapter~\ref{Chapter4})  (2) document retrieval  (cf. section~\ref{Chapter5.2}, chapter~\ref{Chapter5}), (3) passage retrieval (cf. section~\ref{Chapter5.3}, chapter~\ref{Chapter5}), and (4) answer extraction system (cf. section~\ref{Chapter6.2}) which was one of the winners in the 2017 BioASQ challenge. Indeed, we developed the SemBioNLQA system based on the integration of these methods and techniques. Figure~\ref{fig:QAA} shows the architecture of SemBioNLQA and its main components.


SemBioNLQA first takes as its input a natural language biomedical question and includes preprocessing of the question, identification of the question type and the expected answer format to be required based on handcrafted lexico-syntactic patterns and support vector machine, as well as building a query from the question using UMLS entities to be fed into our document retrieval system based on PubMed and UMLS similarity. A document retrieval system is used to retrieve documents satisfying the query from the MEDLINE database. Then, it extracts relevant passages from top-ranked documents based on the BM25 model, stemmed words and UMLS concepts. Finally, it generates and returns both ``exact'' (depending on the expected answer for each question type) and paragraph-sized ``ideal'' answers from these passages based on the UMLS metathesaurus, BioPortal synonyms, SENTIWORDNET, term frequency metric and BM25 model. According to the QA classification approach presented by \cite{athenikos2010biomedical}, the SemBioNLQA can be classified as semantics-based biomedical QA.



% Figure environment removed

\subsection{Experimental results}

In order to assess the effectiveness of the SemBioNLQA system and compare it with the current integral biomedical QA systems presented in \citep{gobeill2009question,Cao_2011,Kraus_2017}, we present two different evaluations: (1) a systematic/automatic evaluation on benchmark datasets provided by the BioASQ challenges, and (2) a manual evaluation in terms of quality of the answers using BioASQ training questions and answers.

\subsubsection{Systematic evaluation}
In this experiment, we present a systematic evaluation on real biomedical questions provided by the BioASQ challenge in its 2017 edition so as to compare with Olelo, the most current biomedical QA system. Fortunately, the Olelo system has participated in Task 5b, phase B of the 2017 BioASQ challenge which enabled us to compare with them as we also have participated in the challenge. Please note that in this experiment, both systems SemBioNLQA and Olelo relied only on the gold-standard passages provided by BioASQ, instead of the ones retrieved by the systems.

As indicators of answer extraction effectiveness: Accuracy was used for exact answers of yes/questions; mean reciprocal rank (MRR) was used for exact answer of factoid questions; mean average precision, mean average recall, and mean average f-measure were used for exact answers of list questions; ROUGE-2 and ROUGE-SU4 were used for ideal answers. These evaluation metrics are described in details in subsection~\ref{Chapter3_7_2}, section~\ref{Chapter3_7}, chapter~\ref{Chapter3}. Table~\ref{tab:6.3.1} shows the experimental results of SemBioNLQA and comparison with Olelo on five batches of testing datasets provided by the 2017 BioASQ organizers during our participation. Please also note that we do not compare with other BioASQ participants in this work since they have not published the integral systems yet. As we previously noted, the BioASQ challenges in phase B of Task b provide the test set of biomedical questions along with their golden documents, golden snippets, and questions types. Therefore, participants do not require a fully QA to participate in the challenge. More details on the results can be found in the BioASQ web site\footnote{\url{http://participants-area.bioasq.org/results/5b/phaseB/}}. Our system name for submission was ``sarrouti''.


\begin{table}[h!]
\centering
\caption[The overall results of SemBioNLQA and comparison with Olelo on five batches of testing datasets provided by BioASQ 5b 2017]{The overall results of SemBioNLQA and comparison with Olelo on five batches of testing datasets provided by BioASQ 5b 2017. The ``-'' and ``nr'' indicate that the system did not deal with this task and the results are not reported for this evaluation measure, respectively. Acc, P, R, F, R-2, R-SU4 indicate accuracy, precision, recall, and f-measure, rouge-2, rouge-SU2, respectively.}
\label{tab:6.3.1}
\begin{tabular}{p{1.6cm}p{3.6cm}p{1.1cm}p{1cm}p{1cm}p{1cm}p{1cm}p{1cm}p{1.2cm}}
\hline\noalign{\smallskip}
 \multirow{3}{*}{Datasets} &\multirow{3}{*}{System name}& \multicolumn{5}{c}{Exact answers} &  \multicolumn{2}{c}{\multirow{2}{*}{Idial answers}}\\
\cmidrule(l){3-7}

 & & Yes/No & Factoid & \multicolumn{3}{c}{ List}  &\multicolumn{2}{c}{}   \\
 \cmidrule(l){3-3}\cmidrule(l){4-4} \cmidrule(l){5-7} \cmidrule(l){8-9}
  & &  Acc &  MRR& P& R &F1  & R-2& R-SU4\\
\noalign{\smallskip}\hline\noalign{\smallskip}
\multirow{2}{*}{Batch 1} &	SemBioNLQA & 0.7647&	0.2033	& 0.1909& 0.2658& 0.2129&0.4943& 0.5108 \\
                         &	Olelo & -&	0.0400&nr&nr&	 0.0477 & 0.2958& 0.3243\\

\cmidrule(l){1-9}
\multirow{2}{*}{Batch 2} &	SemBioNLQA &0.7778& 0.0887&	0.2400&	0.3922&	0.2920& 0.4579& 0.4583\\
                         &	Olelo & -&	0.0323&nr&nr&	 0.0287& 0.2048& 0.2500\\

            \cmidrule(l){1-9}
\multirow{2}{*}{Batch 3} &	SemBioNLQA &0.8387& 0.2212&	0.2000&	0.4151&	0.2640 & 0.5566 &	0.5656 \\
                         &	Olelo & -&	 0.0192 &nr&nr&	 0.0549& 0.2891& 0.3262\\
            \cmidrule(l){1-9}
\multirow{2}{*}{Batch 4} &	SemBioNLQA &0.6207& 0.0970&	0.1077&	0.2013&	0.1369& 0.5895&	0.5832 \\
                         &	Olelo & -&	0.0513 &nr&nr&0.0513& 0.3460& 0.3516\\

            \cmidrule(l){1-9}
\multirow{2}{*}{Batch 5} &	SemBioNLQA &0.4615 & 0.2071&	0.2091&	0.3087&	0.2438& 0.5772 &	0.5756\\
                         &	Olelo & -&	-&nr&nr&	 0.0379& 0.2117& 0.2626\\

\noalign{\smallskip}\hline
\end{tabular}
\end{table}


In particular, we also report the overall end-to-end evaluation results of the SemBioNLQA system on BioASQ datasets provided by the challenge in 2015 and 2016 editions in order to demonstrate the effectiveness of the SemBioNLQA and also to make new comparisons easier. Table~\ref{tab:6.3.2} highlights the obtained results on BioASQ 3b 2015 and BioASQ 4b 2016 datasets. All answers returned by the SemBioNLQA system on either BioASQ 3b 2015 and BioASQ 4b 2016 datasets are available for download\footnote{\url{https://sites.google.com/site/mouradsarrouti/datasets}}. Concretely, this experiment aims to answer the following questions:

\begin{enumerate}
  \item Is SemBioNLQA able to achieve improvements over the existing biomedical QA systems?
  \item Do question classification, document retrieval and passage retrieval components have an impact on the overall performance of the SemBioNLQA system?
\end{enumerate}



A direct comparison with the systems presented in \citep{neves2015hpi, zhang2015fudan,schulze2016hpi,yang2015learning,choi2015snumedinfo} and evaluated on either the 2015 or 2016 BioASQ challenges is not simple since the authors used the test set of biomedical questions along with their golden documents, golden snippets, and questions types released by the BioASQ challenges. While in the end-to-end evaluation of SemBioNLQA, the relevant documents, relevant passages, and the question type for a given biomedical question are obtained using its document retrieval, passage retrieval, and question classification systems, respectively.
\begin{table}[h!]
\centering
\caption[The overall evaluation results of the SemBioNLQA system on five batches of biomedical questions provided by BioASQ 3b 2015 and BioASQ 4b 2016]{The overall evaluation results of the SemBioNLQA system on five batches of biomedical questions provided by BioASQ 3b 2015 and BioASQ 4b 2016. Acc, P, R, F, R-2, R-SU4 indicate accuracy, precision, recall, and f-measure, rouge-2, rouge-SU2, respectively.}
\label{tab:6.3.2}
\begin{tabular}{p{3.1cm}p{1.3cm}p{1.2cm}p{1.1cm}p{1.1cm}p{1.1cm}p{1.1cm}p{1.1cm}p{1.3cm}}
\hline\noalign{\smallskip}
 \multirow{3}{*}{Dataset} &\multirow{3}{*}{Batch}& \multicolumn{5}{c}{Exact answers} &  \multicolumn{2}{c}{\multirow{2}{*}{Idial answers}}\\
\cmidrule(l){3-7}

 & & Yes/No & Factoid & \multicolumn{3}{c}{ List}  &\multicolumn{2}{c}{}   \\
 \cmidrule(l){3-3}\cmidrule(l){4-4} \cmidrule(l){5-7} \cmidrule(l){8-9}
  & &  Acc &  MRR& P& R &F1  & R-2& R-SU4\\

\noalign{\smallskip}\hline\noalign{\smallskip}
\multirow{5}{*}{BioASQ 3b 2015} &	Batch 1 &0.7273&	0.0128	& 0.0364&	0.0682&	0.0462& 0.1039&	0.1397\\
            &	Batch 2 &0.6875&	0.0339	& 0.0714&	0.1232&	0.0891& 0.1044&	0.1413\\
            &	Batch 3 &0.8621&	0.0641	& 0.0353&	0.0368&	0.0352& 0.1269&	0.1557\\
            &	Batch 4 &0.6800&	0.0586	& 0.0696&	0.0880&	0.0751& 0.1467&	0.1702\\
            &	Batch 5 &0.6786&	0.0795	& 0.0083&	0.0208&	0.0119& 0.0915&	0.1205\\

\cmidrule(l){1-9}
\multirow{5}{*}{BioASQ 4b 2016} &	Batch 1 &0.8214&	0.0534	& 0.1091&	0.1545&	0.1268&0.1552&	0.1855\\
            &	Batch 2 &0.7188&	0.0495	& 0.0476&	0.0477&	0.0462& 0.1378&	0.1720\\
            &	Batch 3 &0.8800&	0.1186	& 0.0857&	0.1667&	0.1122& 0.1430&	0.1730\\
            &	Batch 4 &0.8095&	0.0253	& 0.0400&	0.0667&	0.0500& 0.1097&	0.1289\\
            &	Batch 5 &0.8519&	0.0687	& 0.0300&	0.0517&	0.0368& 0.1609&	0.1836\\

\noalign{\smallskip}\hline
\end{tabular}
\end{table}
As it has already been stated before, to a given biomedical question, SemBioNLQA first retrieves the $N$ top-ranked documents, then finds the $N$ top-ranked passages and finally applies the appropriate answer extraction according to the question type detected by the question classification module. In particular, we have decided to go with the $N=10$ top-ranked documents and $N=10$ top-ranked passages since only the 10 first ones from the resulting list are permitted for the test in the 2015 and 2016 BioASQ challenges. After that, for yes/no questions, the exact answer (``yes'' or ``no'') is formed by using SENTIWORDNET, a sentiment lexicon. Each word of the relevant snippets is assigned its SENTIWORDNET score, and the decision to output ``yes'' or ``no'' depends on the number of positive or negative snippets. For factoid questions, the exact answer is produced by identifying the biomedical entities that occur in the given top relevant snippets of the question, and reporting the five most frequent biomedical entities and their synonyms of the top snippets, excluding entities also mentioned in the question. For list questions, the exact answer is produced in the same manner, except that the most frequent entities and their synonyms of the top relevant snippets are now returned as the single list that answers the question. On the other hand, to generate ideal answers (summaries), we have applied MetaMap to the relevant snippets and question in order to obtain the UMLS concepts they refer to. We then have ranked the snippets by their BM25 similarity to the question (using Porter stems and UMLS concepts as features), and return the top two most highly ranked snippets (concatenated) as the ideal answer. Figure~\ref{fig:r1}, Figure~\ref{fig:r2} and Figure~\ref{fig:r3}  show the SemBioNLQA output for three biomedical questions which come from BioASQ training questions.


% Figure environment removed

% Figure environment removed

% Figure environment removed





\subsubsection{Manual evaluation}

In this experiment, we randomly selected 30 questions from the BioASQ training dataset and posed these to the four systems - AskHermes, EAGLi, Olelo and SemBioNLQA. This evaluation was carried out manually, and therefore, we needed to limit the number of questions and types. We decided to limit it to factoid, list and yes/no questions given that these types of answers are easier to check manually than summaries. This sequence of 30 questions, which are listed in Appendix A, contains 10 factoid questions, 11 list questions and 9 yes/no questions. In our evaluation, an answer is considered as correct if the first returned biomedical entity (for factoid questions), at least one of the first five returned biomedical entities (for list questions) or the Boolean value, i.e., ``yes'' or ``no'', (for yes/no questions) is correct. Indeed, we manually checked the results returned by each system to look for the correct standard answers as provided by the BioASQ challenge. Table~\ref{tab:6.3.4} presents and compares the results of the aforementioned systems and SemBioNLQA. All answers returned by the systems are available for download \footnote{\url{https://sites.google.com/site/mouradsarrouti/datasets}}.

\begin{table}[h!]
\centering
\caption{Comparison of the obtained results by SemBioNLQA, EAGLi, AskHERMES and Olelo in terms of number of recognized questions and correct answers}
\label{tab:6.3.4}
\begin{tabular}{M{5.5cm}M{4.9cm}M{4.7cm}}
\hline\noalign{\smallskip}
Systems& Number of recognized questions& Number of correct answers\\
\noalign{\smallskip}\hline\noalign{\smallskip}
EAGLi \citep{gobeill2009question}&7/30&3/30\\
AskHERMES \citep{Cao_2011}&12/30&2/30\\
Olelo \citep{Kraus_2017}&30/30&6/30\\
\textbf{SemBioNLQA}&\textbf{30/30}&\textbf{18}/\textbf{30}\\
\noalign{\smallskip}\hline
\end{tabular}
\end{table}

\subsection{Discussion}

In contrast to traditional IR systems which identify a simple list of relevant documents for the user's query usually expressed in terms of some keywords, QA systems aims at providing precise short answers to user questions written in natural language. It is the goal of such systems to move the burden of browsing and filtering the numerous results, which can be quite time consuming, from the information seekers to the computer. While open-domain QA has been widely studied, few integral systems such as the ones described in \citep{gobeill2009question,Cao_2011,Kraus_2017} are currently able to automatically answer questions from the ever-increasing volume of peer-reviewed scientific articles in the biomedical domain. In this work, we addressed shortcomings of these systems, such as limited usability and performance in terms of the precision for the currently supported question and answer types. As results, unlike these systems, our developed biomedical QA system SemBioNLQA has the ability to handle a large amount of questions and answers types such as yes/no, factoid, list and summary questions that may cover all types of questions. It returns exact answers in the form of ``yes'' or ``no'' for yes/no questions, biomedical entities for factoid questions and a list of biomedical entities for list questions. In addition to exact answers for the aforementioned questions types, SemBioNLQA also formulates and returns ideal answers. For summary questions, the system retrieves only ideal answers since such questions do not have precise answers.

The results of the systematic evaluation, detailed in Table~\ref{tab:6.3.1}, have shown that SemBioNLQA gets better results compared with Olelo, the most current biomedical QA system. The presented system significantly outperformed the Olelo system in extracting both exact and ideal answers for the currently supported questions. The important thing to note here is that our system significantly outperforms the Olelo system in all batches of testing datasets. There is a large difference in the results, a thing that clearly indicates that our system is not only effective but also robust.

From another side, as shown in Table~\ref{tab:6.3.4} which presents and compares the results of the manual evaluation of SemBioNLQA, EAGLi, AskHERMES and also Olelo in terms of number of recognized questions and correct answers, SemBioNLQA gets better results and succeeded at answering the majority of randomly selected BioASQ questions. We have manually analyzed the answers provided for the biomedical questions by each system. In contrast to SemBioNLQA, which has proven to be quite successful at extracting the exact answers depending on the expected answer for each question type, Olelo returned a summary as the answer for the most questions and AskHERMES returned a multiple sentence passage as answer for all questions. Indeed, SemBioNLQA was able to detect the questions type were of the factoid, list or yes/no  types, and thus generated exact answers depending on the expected answer for each question type. This indicates that its integration of our question classification method offers SemBioNLQA the ability to understand and correctly recognize the information needs of users. In contrast, even though Olelo was developed to handle with factoid, list and summary questions, it was not able to detect the types for given questions, and thus generated summaries for all questions, and therefore, the users have to read these summaries so as to find the precise answers. In particular, it only returns exact answers when both the headword and semantic types are detected, in addition to the candidate answers being of this same semantic type. On the other hand, as shown in Table~\ref{tab:6.3.4}, both SemBioNLQA and Olelo have succeeded in returning answers for all questions, while AskHERMES and EAGLi could not provide answers for the majority of the questions, instead, only the following messages ``Nothing found! Please refine your question'' in the former and ``EAGLi did not understand your question. Try a popular example, or go to the manual mode.'' in the latter.

On the other hand, it is clear from Table~\ref{tab:6.3.2} that the different components of the SemBioNLQA system have a significant impact on the answer extraction task and therefore on the overall performance of SemBioNLQA since if the set of retrieved documents, passages and the type of a given question are not identified correctly, further processing steps to extract the answers will inevitably fail too.  For instance, for the question ``What is the association of spermidine with $\alpha$-synuclein neurotoxicity?'' (identifier 56c073fcef6e394741000020) from the batch 1 of test  set of the 2016 BioASQ challenge, the returned type of question is ``summary'' whereas in the corpus the type of question is ``factoid''. Therefore, the SemBioNLQA system will inevitably fail to extract and output the correct answer since extracting the answer to a factoid question, which is asking for a biomedical entity, is not the same as extracting the answer to a summary question which is looking only for an ideal answer.


Overall, SemBioNLQA holds a number of advantages over the state-of-the-art systems. First, the integration of our question types classification method it offers have a clear advantage over Olelo in that it returns exact answers depending on the expected answer of each question type. Second, SemBioNLQA which is aimed to be able to accept a variety of natural language questions and to generate appropriate natural language answers, provides an unbeatable advantage over AskHERMES, EAGli and Olelo in that it handles with a large amount of questions types including yes/no, factoid, list and summary questions. Third, the systematic and manual evaluations results demonstrated that SemBioNLQA is more effective as compared with the aforementioned systems.

In summary, biomedical QA is a very challenging task since it accepts questions written in natural language and provides precise answers instead of only presenting potentially relevant documents by integrating various resources. Therefore, no current system can always perform well on the myriad questions that can be asked of it. SemBioNLQA provides a practical and competitive alternative to help users find exact and ideal answers.

\section{Summary of the Chapter}

Starting from the aim of answering a variety of natural language questions, we presented in this thesis work the proposed methods for the extraction of the answers to given biomedical questions including yes/no questions, factoid questions, list questions, and summary questions. These types of questions that are commonly asked in the biomedical domain, may cover all types of questions that can be posed by the users.

In section~\ref{Chapter6.2} we presented in details the answer extraction system that we proposed for extracting the answer for each of the aforementioned question types. The proposed system provides exact answers (e.g., ``yes'', ``no'', a biomedical entity name, etc.) and ideal answers (i.e., paragraph-sized summaries of relevant information) for yes/no, factoid and list questions, whereas it provides only the ideal answers for summary questions. Thanks to an evaluation on standard benchmarks provided by the 2015 and 2016 BioASQ challenges, we noted that SemBioNLQA achieved promising performances compared with the best existing methods. Moreover, as part of our participation in Phase B, Task 5b of the 2017 BioASQ challenge, our submission was placed within the top tier submissions out of all participants. Furthermore, our system was one of the fifth BioASQ challenge winners.


In section~\ref{Chapter6.3} we tackled a fully automatic QA system in the biomedical domain, SemBioNLQA, which has the ability to deal with four types of biomedical questions including yes/no questions, factoid questions, list questions, and summary questions. SemBioNLQA is currently able to provide exact answers and paragraph-sized ideal answers for yes/no, factoid and list questions, whereas it only retrieves ideal answers for summary questions. The system relied on (1) handcrafted lexico-syntactic patterns and a machine learning approach for question classification, (2) PubMed search engine and UMLS similarity for document retrieval, (3) the BM25 model, stemmed words and UMLS concepts for passage retrieval, and (4) UMLS metathesaurus, BioPortal synonyms, sentiment analysis and term frequency metric for answer retrieval. Compared with the existing biomedical QA systems, SemBioNLQA has the potential to deal with a large amount of question and answer types. Moreover, experimental evaluations performed
on biomedical questions and answers provided by the BioASQ challenge especially in 2017 (as part of our
participation), show that SemBioNLQA achieves good performances compared with the most current state-of-the-art system and allows a practical and competitive alternative to help information seekers find exact and ideal answers to their biomedical questions.



\section{Conclusion}\label{sec7}

For the problem of path planning for a cellular-enabled UAV with connectivity and battery constraints, the generalized intersection method with battery constraint (GIM-B) algorithm was proposed that computes an optimal path in polynomial time. Its effectiveness in terms of computational complexity and resultant mission completion time was demonstrated by comparing with previously proposed algorithms both in analytically and numerically. Furthermore, we proposed the bottleneck edge search method that finds the maximum deliverable payload weight under the connectivity and battery constraints. Various numerical results were presented to illustrate the effects of the environmental parameters on the optimal UAV path and the corresponding delivery time.

Let us conclude with some remarks on further works. We assumed that the delay at each CS is fixed over time, but in general it  changes over time in practice. It would be interesting to consider the scenario with time-varying delays at charging stations and develop shortest path finding algorithms over time-dependent graphs \cite{Orda:1990,Ding:2008}. Another interesting scenario would be to  consider more realistic coverage regions based on radio map taking into account signal blockage and reflection by buildings and interference from other BSs \cite{Chen:2017,Zhang:2021}.

%For the problem of path planning for a cellular-enabled UAV with connectivity and battery constraints, the generalized intersection method with battery constraint (GIM-B) algorithm was proposed that computes an optimal path in polynomial time. Its effectiveness in terms of computational complexity and resultant mission completion time was demonstrated by comparing with previously proposed algorithms both in analytically and numerically. For further works, it would be interesting to consider the scenario with time-varying delays at charging stations and  the scenario with more realistic coverage regions based on radio map.


\ifCLASSOPTIONcaptionsoff
  \newpage
\fi


\bibliographystyle{IEEEtran}
\bibliography{ref}
\begin{IEEEbiography}[{% Figure removed}]
{Hyeon-Seong IM} received the B.S. (summa cum laude) and M.S. degrees from the Pohang University of Science and Technology (POSTECH), Pohang, South Korea, in 2019 and 2020, respectively, and the Ph.D. degree from the School of Electrical Engineering, Korea Advanced Institute of Science and Technology (KAIST), Daejeon, South Korea, in 2024. He is currently a Senior Researcher in LIG Nex1, Seongnam, South Korea. His research interests include military communications, non-terrestrial networks (NTN), and anti-jamming.
\end{IEEEbiography}
\begin{IEEEbiography}[{% Figure removed}]
{Kyu-Yeong Kim} (Graduate Student Member, IEEE) received the B.S. (Great Honors) degree in electrical engineering from Korea University in 2022 and the M.S. degree from the School of Electrical Engineering, Korea Advanced Institute
of Science and Technology (KAIST), South Korea, in 2024, where he is currently pursuing the Ph.D. degree.
\end{IEEEbiography}
\begin{IEEEbiography}[{% Figure removed}]
{Si-Hyeon Lee} (Senior Member, IEEE) received the B.S. (summa cum laude) and Ph.D. degrees in electrical engineering from the Korea Advanced Institute of Science and Technology (KAIST), Daejeon, South Korea, in 2007 and 2013, respectively.
She is currently an Associate Professor with the School of Electrical Engineering, KAIST.
She was a Postdoctoral Fellow with the Department of Electrical and Computer Engineering, University of Toronto, Toronto, Canada, from 2014 to 2016, and an Assistant Professor with the Department of Electrical Engineering, Pohang University of Science and Technology (POSTECH), Pohang, South Korea, from 2017 to 2020. Her research interests include information theory, wireless communications, statistical inference, and machine learning. She was a TPC Co-Chair of IEEE Information Theory Workshop 2024. She is currently an IEEE Information Theory Society Distinguished Lecturer (2024-2025) and an Associate Editor for IEEE Transactions on Information Theory. 
\end{IEEEbiography}
\vfill





% that's all folks
\end{document}
