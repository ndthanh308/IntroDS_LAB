\section{Introduction}\label{sec1}
Unmanned aerial vehicles (UAVs) are extensively applied across various scenarios, such as delivery and transportation \cite{Zhang:2021_2}, aerial surveillance and monitoring \cite{Kanistras:2013}, flying base stations (BSs) \cite{Mozaffari:2019}, and data collection and/or power transfer for IoT devices  \cite{Yu:2021}, due to their high mobility, free movement, and cost-effectiveness  \cite{Mozaffari:2019,Shi:2018}. It has been actively studied to design the UAV trajectory according to each operational scenario. For UAV-aided communication scenarios, the UAV trajectory has been optimized taking into account various factors, e.g., minimizing energy while satisfying throughput demands  \cite{Zeng:2019,Qi:2020}, \sh{minimizing the age of information (AoI) of IoT devices \cite{Yi:2023}, jointly optimizing the transmission rate and the rate variation in aerial video streaming scenario \cite{Zhan:2024_2}}, and improving secrecy rate in the presence of an eavesdropper \cite{Li:2019,Cui:2018}. For delivery or transportation scenarios, it is utmost important  to swiftly and safely transport the given objects to their desired destinations. Thus, for such scenarios, the problem of designing UAV trajectory has been formulated as minimizing the mission time or \sh{the energy consumption (equivalent to minimizing the mission time of UAV with fixed speed)} with some constraints such as \sh{connectivity \cite{Zhang:2019,Zhang:2019_2,Chen:2020,Zhan:2022,Zhan:2022_2,Zhang:2021,Esrafilian:2020,Chapnevis:2021,Zeng:2019_2,Khamidehi:2020,Wang:2022,Chen:2022}}, restricted airspace \cite{Khamidehi:2020,Wang:2022}, collision avoidance between UAVs \cite{Wang:2022,Chen:2022}, and battery limits \cite{Sundar:2014,Coelho:2017,Fan:2023,Arafat:2022}. 
In particular, it is important for such scenarios to consistently keep the connection between the UAV and the control station. This persistent connectivity is 
essential for tracking the real-time location of the cargo in delivery scenarios \cite{Zhang:2019} and for providing real-time communication service to passengers in transportation scenarios like urban air mobility (UAM) \cite{Cohen:2021}. It is also vital when the manual control is necessary for the UAV to evade unexpected adverse weather conditions or avoid collisions with aerial obstacles \cite{Banafaa:2024}. However, maintaining direct connection with the control station becomes challenging if the initial and the final points are far away, due to low received signal strength by low line-of-sight (LoS) probability and long communication distance. Cellular-enabled UAV communication is a potential approach for this problem \cite{Zhang:2019}, wherein the UAV communicates with its control station by connecting with a close BS and the underlying cellular network \cite{Agyapong:2014}. 

Our paper addresses the path planning problem for UAVs performing delivery or transportation missions, jointly considering connectivity with the cellular network and UAV’s battery limit. In the absence of battery limit, the problem of minimizing the mission time \sh{(or the energy consumption)} while ensuring continuous connection with the cellular network has been \sh{extensively studied \cite{Zhang:2019,Zhang:2019_2,Chen:2020,Zhan:2022,Zhan:2022_2,Zhang:2021,Esrafilian:2020,Chapnevis:2021,Zeng:2019_2,Khamidehi:2020,Chen:2022,Wang:2022}.}  
The authors of \cite{Zhang:2019} concentrated on planning an optimal path between a initial and a final points while ensuring continuous connection with a BS. To simplify the problem, they assumed that the UAV and a BS can be connected if their communication distance is not larger than a certain threshold. Then, they transformed the problem into graph-theoretic path finding and convex optimization problems, and proposed an optimal algorithm with non-polynomial time complexity and two sub-optimal algorithms with polynomial time complexity. 
The study of characterizing an optimal path under the connectivity constraint has been extended in various directions, e.g., \sh{allowing a certain duration or ratio of communication outage \cite{Zhang:2019_2,Chen:2020,Zhan:2022}, assuming the knowledge of the radio map \cite{Zhan:2022_2,Zhang:2021}}, considering 3-dimensional (3D) space \cite{Esrafilian:2020,Zhang:2021},
%design a 3D path based on 3D building maps \cite{Esrafilian:2020} or 3D radio map \cite{Zhang:2021}, 
and considering the collaboration of multiple UAVs \cite{Chapnevis:2021}. 
Specifically, the work \cite{Chen:2020} proposed an intersection method, significantly reducing the computational complexity by converting the problem into a graph-theoretic path finding problem whose vertex set consists of the intersection points of the coverage boundaries of the BSs. Moreover, the work \cite{Zhang:2021} also used a graph-theoretic approach even for 3D path finding problem with a realistic communication environment considering signal blockage and reflection by buildings and interference from other BSs, by quantizing the radio map to finite grid points. On the other hand, reinforcement learning (RL) \cite{Sutton:2018} based approaches may be effective for scenarios that the UAV only has limited prior knowledge about communication and transportation environment, since it can empirically learn the environment. Several studies have explored to learn effective UAV paths by applying \sh{RL-based approaches \cite{Zhan:2022,Zhan:2022_2,Zeng:2019_2,Khamidehi:2020,Chen:2022,Wang:2022}.} However, It is important to acknowledge that the RL-based approach may not consistently output an optimal path, and the training process involved in RL may demand considerable computing time and resources. 

In practice, it is important to consider the limited battery capacity of the UAV. There have been a few works on designing UAV trajectory performing delivery or transportation missions taking into account the limited battery capacity \cite{Sundar:2014,Coelho:2017,Fan:2023,Arafat:2022}. The work \cite{Sundar:2014} considered a variant of the travelling salesman problem (TSP) that aims to derive a shortest route visiting each target node once, while considering the limited battery capacity of the UAV and charging stations to replenish its energy. Such a UAV route optimization problem with  TSP formulation taking into account the battery limit has been extended by considering multiple UAVs \cite{Coelho:2017,Fan:2023} and grouping target nodes into clusters \cite{Arafat:2022}.  However, the problem of UAV path planning to minimize the mission time for delivery or transportation scenarios while considering both the connectivity and the battery replenishment has not been well studied. We note that for other UAV utilization scenarios, such a problem of  UAV path planning taking into account both the connectivity and the battery replenishment has been studied, but for different objectives depending on the assigned missions, e.g., minimizing the energy consumption of UAV \cite{Zeng:2019,Qi:2020} or the AoI of devices \cite{Yi:2023,Zhan:2024}. 

Our primary contribution involves proposing a path planning algorithm that outputs an optimal flight path of a cellular-enabled UAV in polynomial time complexity, to expeditiously transport a payload from an initial point to a final point, while persistently keeping the connection with a BS and complying with its battery limit. It is assumed that the UAV can establish a connection with a BS if they are closer than a certain threshold similarly as in \cite{Zhang:2019}, but we allow that the threshold can be different for each BS due to interference from other BSs. The UAV’s depleted battery can be swapped with another completely charged one at a charging station, which may involve a certain delay depending on waiting and replacing times \cite{Lee:2015}. The contributions of this paper are summarized as follows:

\begin{itemize}
\item Our problem of optimizing UAV trajectory involves determining the UAV's path, speed, and the order of visiting charging stations. We solve this problem by first reformulating the problem as a two-level graph-theoretic shortest path search problem, and then applying Dijkstra algorithm \cite{Dijkstra:1959}. More precisely, we initially search an optimal path and the corresponding maximum allowable speed for traveling between each pair of charging stations  without swapping to a fully charged battery. Subsequently, we ascertain an optimal sequence of visiting charging stations. To validate the efficacy of our approach, we analytically and numerically compare our algorithm with existing algorithms in \cite{Zhang:2019,Chen:2020} with marginal modifications, and show that only our algorithm yields an optimal solution in polynomial time.
%and show that ours surpasses those in both mission time and computational complexity.

\item Characterizing the maximum deliverable payload weight under the connectivity and battery constraints is another interesting problem. We propose a graph-theoretic  algorithm that yields an optimal solution to this problem in polynomial time. It first transforms the delivery environment into a weighted graph and finds the longest connectivity-critical edge between the initial and the final points in the graph. Then, it derives the largest payload weight which can be delivered over the edge without replacing the battery.

%\item Various numerical results are provided to show the optimal path and the corresponding delivery time according to environmental parameters and compare with the previously proposed algorithms  \cite{Zhang:2019,Chen:2020}.
\end{itemize}
The remaining of this paper is organized as follows. In Section \ref{sec2}, we present the system model and formulate the optimization problem of finding the fastest UAV route under the connectivity and battery constraints. Our proposed algorithms that output optimal UAV trajectories without and with the battery limit are presented in Sections \ref{sec3} and \ref{sec4}, respectively. In Section \ref{sec5}, other objectives of minimizing the UAV energy consumption  and of maximizing the deliverable payload weight are considered and the corresponding optimal algorithms are presented. We provide various numerical results in Section \ref{sec6}. Finally, the paper is concluded in Section \ref{sec7}.

