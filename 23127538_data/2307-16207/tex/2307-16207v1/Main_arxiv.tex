\documentclass[conference]{IEEEtran}
\IEEEoverridecommandlockouts
% The preceding line is only needed to identify funding in the first footnote. If that is unneeded, please comment it out.
\ifCLASSINFOpdf
\usepackage[pdftex]{graphicx}
\else
\usepackage[dvips]{graphicx}
\usepackage[caption=false,font=footnotesize]{subfig}
\fi
\usepackage[caption=false,font=footnotesize]{subfig}
%\usepackage{subcaption}
\usepackage{setspace}
\usepackage{algorithm}
%\usepackage{algorithmic}
\usepackage{algorithmicx} 

\usepackage{multicol}
\usepackage{lipsum}
%\usepackage[shortlabels]{enumitem}
\usepackage{setspace}
\usepackage{arydshln}
\usepackage{makecell}
\usepackage{rotating}
\usepackage{makecell, multirow}
\usepackage[skip=1ex]{caption}
\newcommand{\tabitem}{~~\llap{\textbullet}~~}
\usepackage{fix-cm}
%\usepackage[retainorgcmds]{IEEEtrantools}
%\usepackage{bibentry}  
\usepackage{xcolor,soul,framed} %,caption
\usepackage{subfiles}
\colorlet{shadecolor}{yellow}
% \usepackage{color,soul}
\usepackage[pdftex]{graphicx}
\graphicspath{{../pdf/}{../jpeg/}}
\DeclareGraphicsExtensions{.pdf,.jpeg,.png}
\usepackage{tikz}
%Mathabx do not work on ScribTex => Removed
%\usepackage{mathabx}
\usepackage{array}
\usepackage{mdwmath}
\usepackage{mdwtab}
\usepackage{eqparbox}
\usepackage{url}
\usepackage{cite}
\usepackage{epsfig,amsmath,amssymb,epsf,amsthm,scalefnt,multirow,subfig}
\usepackage{xcolor}
\usepackage{float}
\usepackage{psfrag}
\usepackage{algpseudocode}
\usepackage{verbatim}
\usepackage{tikz}
\usetikzlibrary{tikzmark,calc,decorations.pathreplacing}
\usepackage{amsmath}

\newenvironment{rcases}
  {\left.\begin{aligned}}
  {\end{aligned}\right\rbrace}

\newcommand{\argmax}{\mathop{\mathrm{argmax}}}
\newcommand{\argmin}{\mathop{\mathrm{argmin}}}
\newcommand{\var}{\mathop{Var}}
\newcommand{\specialcell}[2][c]{%
  \begin{tabular}[#1]{@{}c@{}}#2\end{tabular}}
%\newenvironment{theorem}[2][Theorem]{\begin{trivlist}\item[kip \labelsep {\bfseries #1}kip \labelsep {\bfseries #2}]}{\end{trivlist}}
\newtheorem{theorem}{Theorem}
\newtheorem{lemma}{Lemma}
\newtheorem*{lemma*}{Lemma}
\newtheorem{remark}{Remark}
\newtheorem{observation}{Observation}
\newtheorem{corollary}{Corollary}
\newtheorem{definition}{Definition}
\newtheorem{conjecture}{Conjecture}
\newtheorem{proposition}{Proposition}

% Calligraphic uppercase
\def\cA{{\mathcal{A}}} \def\cB{{\mathcal{B}}} \def\cC{{\mathcal{C}}} \def\cD{{\mathcal{D}}}
\def\cE{{\mathcal{E}}} \def\cF{{\mathcal{F}}} \def\cG{{\mathcal{G}}} \def\cH{{\mathcal{H}}}
\def\cI{{\mathcal{I}}} \def\cJ{{\mathcal{J}}} \def\cK{{\mathcal{K}}} \def\cL{{\mathcal{L}}}
\def\cM{{\mathcal{M}}} \def\cN{{\mathcal{N}}} \def\cO{{\mathcal{O}}} \def\cP{{\mathcal{P}}}
\def\cQ{{\mathcal{Q}}} \def\cR{{\mathcal{R}}} \def\cS{{\mathcal{S}}} \def\cT{{\mathcal{T}}}
\def\cU{{\mathcal{U}}} \def\cV{{\mathcal{V}}} \def\cW{{\mathcal{W}}} \def\cX{{\mathcal{X}}}
\def\cY{{\mathcal{Y}}} \def\cZ{{\mathcal{Z}}} \def\cz{{\mathcal{z}}}\def\cce{{\mathcal{e}}}

\def\argmin{\mathop{\mathrm{argmin}}}
\def\sinr{\mathop{\mathrm{SINR}}}    

\def\limit{\mathop{\mathrm{lim}}}
\def\argmax{\mathop{\mathrm{argmax}}}
\def\diag{\mathop{\mathrm{diag}}}
\def\inf{\mathop{\mathrm{inf}}}
\def\outf{\mathop{\mathrm{out}}}
\def\trace{\mathop{\mathrm{tr}}}
\def\hh{\mathop{\mathrm{h}}}
\def\EE{\mathop{\mathrm{E}}}
\def\Var{\mathop{\mathrm{Var}}}
\def\dim{\mathop{\mathrm{dim}}}
\def\Re{\mathop{\mathrm{Re}}}
\def\Im{\mathop{\mathrm{Im}}}

\newcommand{\Ei}{{\mathrm{Ei}}}

\def\bDelta{{\pmb{\Delta}}} \def\bdelta{{\pmb{\delta}}}
\def\bSigma{{\pmb{\Sigma}}} \def\bsigma{{\pmb{\sigma}}}
\def\bPhi{{\pmb{\Phi}}} \def\bphi{{\pmb{\phi}}}
\def\bGamma{{\pmb{\Gamma}}} \def\bgamma{{\pmb{\gamma}}}
\def\bOmega{{\pmb{\Omega}}} \def\bomega  
\def\bTheta{{\pmb{\Theta}}} \def\btheta{{\pmb{\theta}}}
\def\bepsilon{{\pmb{\epsilon}}} \def\bPsi{{\pmb{\Psi}}}
\def\obH{\overline{\bH}}\def\obw{\overline{\bw}} \def\obz{\overline{\bz}}\def\bmu{{\pmb{\mu}}}
\def\b0{{\pmb{0}}}\def\bLambda{{\pmb{\Lambda}}} \def\oc{\overline{\bc}}

% Bold
\def\ba{{\mathbf{a}}} \def\bb{{\mathbf{b}}} \def\bc{{\mathbf{c}}} \def\bd{{\mathbf{d}}}
\def\bee{{\mathbf{e}}} \def\bff{{\mathbf{f}}} \def\bg{{\mathbf{g}}} \def\bh{{\mathbf{h}}}
\def\bi{{\mathbf{i}}} \def\bj{{\mathbf{j}}} \def\bk{{\mathbf{k}}} \def\bl{{\mathbf{l}}}
\def\bm{{\mathbf{m}}} \def\bn{{\mathbf{n}}} \def\bo{{\mathbf{o}}} \def\bp{{\mathbf{p}}}
\def\bq{{\mathbf{q}}} \def\br{{\mathbf{r}}} \def\bs{{\mathbf{s}}} \def\bt{{\mathbf{t}}}
\def\bu{{\mathbf{u}}} \def\bv{{\mathbf{v}}} \def\bw{{\mathbf{w}}} \def\bx{{\mathbf{x}}}
\def\by{{\mathbf{y}}} \def\bz{{\mathbf{z}}} \def\bxb{\bar{\mathbf{x}}} \def\bone{\mathbf{1}}

\def\bA{{\mathbf{A}}} \def\bB{{\mathbf{B}}} \def\bC{{\mathbf{C}}} \def\bD{{\mathbf{D}}}
\def\bE{{\mathbf{E}}} \def\bF{{\mathbf{F}}} \def\bG{{\mathbf{G}}} \def\bH{{\mathbf{H}}}
\def\bI{{\mathbf{I}}} \def\bJ{{\mathbf{J}}} \def\bK{{\mathbf{K}}} \def\bL{{\mathbf{L}}}
\def\bM{{\mathbf{M}}} \def\bN{{\mathbf{N}}} \def\bO{{\mathbf{O}}} \def\bP{{\mathbf{P}}}
\def\bQ{{\mathbf{Q}}} \def\bR{{\mathbf{R}}} \def\bS{{\mathbf{S}}} \def\bT{{\mathbf{T}}}
\def\bU{{\mathbf{U}}} \def\bV{{\mathbf{V}}} \def\bW{{\mathbf{W}}} \def\bX{{\mathbf{X}}}
\def\bY{{\mathbf{Y}}} \def\bZ{{\mathbf{Z}}}

\def\oX{\overline{\bX}}
\def\oGamma{\overline{\bGamma}}

\newcommand{\C}{\mathbb{C}}%{{\mbox{\rm $\scriptscriptstyle ^\mid$pace{-0.40em}C}}} %o
\newcommand{\Z}{\mathbb{Z}}%{{\mbox{\rm $\scriptscriptstyle ^\mid$pace{-0.40em}C}}} %o
\allowdisplaybreaks[4]
\algnewcommand{\LeftComment}[1]{\Statex \(\triangleright\) #1}




% Block for Real, Complex
\def\bbC{{\mathbb{C}}} \def\bbR{{\mathbb{R}}}

\newcommand{\eff}{\mbox{\rm eff}}
\newcommand{\out}{\mbox{\rm out}}
\newcommand{\squeezeup}{\vspace{-2.5mm}}
\def\BibTeX{{\rm B\kern-.05em{\sc i\kern-.025em b}\kern-.08em
    T\kern-.1667em\lower.7ex\hbox{E}\kern-.125emX}}
\begin{document}

\title{Trajectory Optimization for Cellular-Enabled UAV with Connectivity and Battery Constraints\\
\author{\IEEEauthorblockN{Hyeon-Seong Im, Kyu-Yeong Kim, and Si-Hyeon Lee}
\IEEEauthorblockA{\textit{School of Electrical Engineering, Korea Advanced Institute of Science and Technology} \\
Daejeon, South Korea\\
Emails: \{imhyun1209, kimyou283, sihyeon\}@kaist.ac.kr}}
%{\footnotesize \textsuperscript{*}Note: Sub-titles are not captured in Xplore and should not be used}
\thanks{The short version of this paper was submitted to IEEE VTC 2023-Fall.}
}


\maketitle

\begin{abstract}
In this paper, we address the problem of path planning for a cellular-enabled UAV with connectivity and battery constraints. The UAV's mission is to deliver a payload from an initial point to a final point, while maintaining connectivity with a BS and adhering to the battery constraint. The UAV's battery can be replaced by a fully charged battery at a charging station, which takes some time. Our key contribution lies in proposing an algorithm that efficiently computes an optimal path that  minimizes the mission completion time, solvable in polynomial time. We achieve this by transforming the problem into an equivalent two-level shortest path finding problem over weighted graphs and leveraging graph theoretic approaches to solve it. In more detail, we first find an optimal path and speed to travel between each pair of charging stations without replacing  the battery, and then find the optimal order of visiting charging stations.  To demonstrate the effectiveness of our approach, we compare it with previously proposed algorithms and show that our algorithm outperforms those in terms of both computational complexity and performance.
%We study a path planning problem for a cellular-enabled unmanned aerial vehicle (UAV) with a delivery mission to minimize the delivery time from an initial point to a final point, while maintaining the connectivity with one of the base stations (BSs) under limited battery capacity. To deliver a payload over a long distance under the battery constraint, we assume that the UAV can replace its battery at charging stations located in the cellular network. We propose an optimal algorithm which solves the path finding problem by converting it to an equivalent shortest path finding problem in a weighted graph and show that the algorithm is NP-easy. To consider the battery constraint, the algorithm is divided in two steps. It first finds an optimal path between each pair of the charging stations (treating the initial and the final points also as charging stations) then derives an optimal total path from the initial point to the final point over the charging stations by combining some of previously induced paths. Moreover, we analytically and numerically show that the proposed algorithm outperforms compared to the benchmark algorithms in both complexity and performance gap.
\end{abstract}

%\begin{IEEEkeywords}
%--
%\end{IEEEkeywords}

\section{Introduction}\label{sec1}
Unmanned aerial vehicles (UAVs) are widely used in various scenarios, including cargo delivery \cite{Zhang:2021_2}, aerial surveillance \cite{Kanistras:2013}, and flying base stations (BSs) \cite{Mozaffari:2019}. Ensuring a continuous connection between the UAV and its control station is crucial for successful completion of the missions, but maintaining persistent communication becomes challenging when the UAV travels over long distances, due to factors such as large path loss and low line-of-sight probability \cite{Al-Hourani:2014,Al-Hourani:2014_2}. A promising solution to this problem is cellular-enabled UAV communication \cite{Zhang:2019}, wherein the BSs in the cellular network act as relays for communication. Therefore, finding a suitable trajectory for the UAV that enables efficient communication with a BS within the cellular network is of utmost importance.

%Unmanned aerial vehicles (UAVs) have been widely applied for various scenarios such as cargo delivery \cite{Zhang:2021_2}, aerial surveillance  \cite{Kanistras:2013}, and flying base stations (BSs) \cite{Mozaffari:2019}. It is important to maintain the connectivity between a UAV and its control station since controlling the UAV is essential to respond to some emergencies and successfully perform the assigned missions. However, when the UAV travels a long distance to deliver a payload, persistent communication with the control station is not easy by the large pathloss and the low line-of-sight probability \cite{Al-Hourani:2014,Al-Hourani:2014_2}. A promising solution for this problem is the cellular-enabled UAV communication \cite{Zhang:2019} since the BSs in the cellular network relay the communication between them. Hence, when the UAV delivers a payload in the cellular network, searching a short trajectory of the UAV while maintaining the communication with a BS is essential. 

Trajectory design problems for cellular-enabled UAVs have been extensively studied in various works \cite{Zhang:2019,Zhang:2019_2,Chen:2020,Esrafilian:2020,Zhang:2021,Chapnevis:2021,Zeng:2019_2,Zeng:2021,Chen:2022,Wang:2022}. In the work  \cite{Zhang:2019}, the authors focused on optimizing the trajectory between an initial and a final location while maintaining communication with a base station (BS). They simplified the problem by assuming that the UAV could connect with a BS if the distance between them was less than a certain threshold. By doing so, they converted the problem into equivalent convex optimization and graph-theoretic shortest path finding problems and proposed one optimal (NP-hard) and two sub-optimal (NP-easy) algorithms.
The study of characterizing an optimal path has been extended in subsequent works to consider factors such as communication outage times \cite{Zhang:2019_2,Chen:2020}, 3D-building maps \cite{Esrafilian:2020}, radio map-based 3D path planning \cite{Zhang:2021}, and the collaboration of multiple UAVs \cite{Chapnevis:2021}. In particular, the work \cite{Chen:2020} introduced the intersection method, which effectively reduces the time complexity by considering only the finite intersection points of the coverage boundaries of the BSs in path optimization. While these analytic approaches can derive an effective path for the UAV with low complexity, they might face challenges when dealing with scenarios with limited prior knowledge about the communication environment. In such cases, reinforcement learning (RL) \cite{Sutton:2018} based approaches become effective, as they can approximate the communication environment empirically. The use of RL-based UAV path planning has been explored in several works \cite{Zeng:2019_2,Zeng:2021,Chen:2022,Wang:2022}. However, it is important to note that the optimal path may not always be derived using the RL-based approach, and the training phase of RL can be time-consuming and resource-intensive.

%Trajectory design problems for cellular-enabled UAVs have been studied in \cite{Zhang:2019,Zhang:2019_2,Chen:2020,Esrafilian:2020,Zhang:2021,Chapnevis:2021,Zeng:2019_2,Zeng:2021,Chen:2022,Wang:2022}. The work \cite{Zhang:2019} studied the problem that optimizes the trajectory between an initial and a final locations while maintaining the communication with a BS in analytic approaches. This work simplified the problem by assuming that the UAV can connect with a BS if the distance between the UAV and the BS is less than a certain threshold and converting the problem into equivalent convex optimization and graph-based shortest path finding problems. In \cite{Zhang:2019}, one optimal (NP-hard) and two suboptimal (NP-easy) algorithms are proposed by applying the approaches based on the graph theory and the convex optimization. The study \cite{Zhang:2019} of characterizing an optimal path has been extended to consider the communication outage times \cite{Zhang:2019_2,Chen:2020}, 3D-building map \cite{Esrafilian:2020}, real radio map-based 3D-path planning \cite{Zhang:2021}, and the collaboration of UAVs in multi-UAV case \cite{Chapnevis:2021}. In particular, the work \cite{Chen:2020} proposed the intersection method which is effective to reduce the time complexity since it only considers the finite intersection points of the coverage boundaries of the BSs in path optimization. Such analytic approaches can derive a short path of the UAV with low complexity, but when there is little prior knowledge about the communication environment, applying them in the path planning problem is not easy. In this case, reinforcement learning (RL) \cite{Sutton:2018} based approach is effective since it can approximate the communication and delivery environment in an empirical way. The RL-based UAV path planning has been studied in several works \cite{Zeng:2019_2,Zeng:2021,Chen:2022,Wang:2022}. However, the optimal path may not be derived by the RL-based approach and the training phase of RL takes lots of time and computing resources. 
In this paper, we address the problem of path planning for a cellular-enabled UAV, considering not only connectivity with a base station (BS) but also the UAV's limited battery capacity. The scenario involves a cellular network with BSs and charging stations. The UAV's mission is to deliver a payload from an initial point to a final point, while maintaining connectivity with a BS and adhering to the battery constraint. The UAV's battery can be replaced by a fully charged battery at a charging station, but this process takes some time \cite{Lee:2015}.
The primary challenge in this path planning problem is optimizing the route and the speed (since the energy consumption is affected by the speed) with the decisions about when and which charging station to visit. Our key contribution lies in proposing an algorithm that efficiently computes an optimal path for this problem, solvable in polynomial time (NP-easy). We achieve this by transforming the problem into an equivalent two-level shortest path finding problem over weighted graphs and leveraging graph theoretic approaches to solve it. More specifically, we first find an optimal path and speed to travel between each pair of charging stations without replacing  the battery. Then, we find the optimal order of visiting charging stations to replace the battery.  To demonstrate the effectiveness of our approach, we compare it with previously proposed algorithms in \cite{Zhang:2019,Chen:2020} with slight modifications. The results show that our algorithm outperforms these existing approaches in terms of both computational complexity and performance.


%In this paper, we consider not only the connectivity with a BS but also the limited battery capacity of the UAV in the path planning problem. Our main contribution is in proposing an algorithm for this problem which outputs an optimal path NP-easily by converting the problem to an equivalent shortest path finding problem in a weighted graph and then applying the graph theory-based approaches to solve the problem. We consider a cellular network with some BSs and charging stations where a UAV has a mission to deliver a payload from an initial point to a final point under the battery constraint. The UAV should maintain the connectivity with a BS to persistently communicate with its control station. The battery can be replaced in a charging station by taking some time \cite{Lee:2015}. The main difficulty of this problem comes from the battery constraint, i.e., determining whether to visit each charging station is essential.
%To tackle it, the algorithm is divided into two steps. It first finds an optimal path between each pair of the charging stations (treating the initial and the final points also as charging stations) without the battery replacement (step 1) and then derive an optimal total path from the initial point to the final point over the charging stations by combining some of previously induced paths between the charging stations (step 2). We also compare our algorithm with previously proposed algorithms in \cite{Zhang:2019,Chen:2020} with slight modification and show that our algorithm outperforms in both complexity and performance gap.

\section{Problem Statement}\label{sec2}
We consider a cellular network with $M$ base stations (BSs) and $N\leq M$ {charging stations (CSs)}. In this network, a UAV delivers a payload from an initial point $\mathbf{U}_0$ to a final point $\mathbf{U}_F$ under limited battery capacity. The detailed description of the UAV model is in Section \ref{sec2A}. The UAV should maintain the connectivity with one of the BSs while delivering the payload. The BS model and the BS-UAV connectivity is described in Section \ref{sec2B}. {The UAV can {replace} its battery at a CS if needed, as explained in Section \ref{sec2C}.} The goal of this paper is to characterize the minimum delivery time from $\mathbf{U}_0$ to $\mathbf{U}_F$, including the flight time in the air and the {battery swapping} time at CSs. This optimization problem is formally presented in Section \ref{sec2D}. The overall model is illustrated in Fig. \ref{Fig1}. %c{control station in fig.1? coverage of each BS? dashed lines between BSs?}

% Figure environment removed

\subsection{UAV Model}\label{sec2A}
%The mission completion time of the UAV is denoted by $T$.
In the cellular network, a rotary-wing UAV has a mission of delivering a payload from an initial point $\mathbf{U}_0$ to a final point $\mathbf{U}_F$. We assume that the UAV flies with a fixed altitude $H\in[H_\mathrm{min},H_\mathrm{max}]$, where $H_\mathrm{min}$ is determined by the heights of obstacles in the network and $H_\mathrm{max}$ corresponds to the maximum allowable altitude according to government regulations. {Let us denote the 3D coordinates of $\mathbf{U}_0$, $\mathbf{U}_F$, and the UAV location at time $t$ by $(x_0,y_0,H)$, $(x_F,y_F,H)$, and $(x(t),y(t),H)$, respectively.} 
We also denote $\mathbf{u}_0=(x_0,y_0)$, $\mathbf{u}_F=(x_F,y_F)$, and $\mathbf{u}(t)=(x(t),y(t))$ as the horizontally projected locations of the 3D coordinates. The UAV flies with time-varying speed of $v(t)\triangleq\|\nabla_t\mathbf{u}(t)\|$ at time $t$, where the speed is selected from the finite set $\mathcal{V}=\{0,v_1,...,v_q\}$ with $0<v_1<...<v_q$. 

% when flying in the air since we consider a downlink communication from a BS to the UAV. 
{For energy consumption, we only consider the propulsion energy consumption by the UAV, since the communication energy consumption is relatively negligible \cite{Mozaffari:2019}.} {Let the total weight of UAV and its payload be given as $w=w_1+w_2+w_3$, where $w_1, w_2$, and $w_3$ denote the weights of the UAV body, its battery, and the payload, respectively.}
{The propulsion power consumption (in Watts) when flying with speed $v$ is given as}
\begin{align}
\begin{split}\label{eq:1}
&P_\mathrm{UAV}(v)=P_1\left(1+{{3v^2}\over{v_\mathrm{tip}^2}}\right)\\
& +P_2(w)\left(\sqrt{1+{{v^4}\over{4v_0(w)^4}}}-{{v^2}\over{2v_0(w)^2}}\right)^{1/2} \!\!\! +{\rho\over 2} S_\mathrm{FP}v^3,
\end{split}
\end{align}
where $P_1$, $P_2(w)$, $v_\mathrm{tip}$, {$v_0(w)$}, $\rho$, and $S_\mathrm{FP}$ are the parameters determined by the environment of the network and the physical structure of the UAV \cite{Zeng:2019}. We note that the {parameters $P_2(w)$ and $v_0(w)$} are only related to the total weight $w$. {The power consumption model \eqref{eq:1} and its parameters} will be revisited with details in Section \ref{sec6}. 

{The UAV consumes the {remained} energy in its battery. The battery capacity {(in Joules)} is expressed as follows \cite{Zhang:2021_2}:}
\begin{align}
C_\mathrm{batt}=\epsilon_\mathrm{batt}w_2,\label{eq:2}
\end{align}
where $\epsilon_\mathrm{batt}$ is {the maximum energy} of the battery per unit weight. Note that the battery capacity is directly proportional to the battery weight. As proved in \cite{Zhang:2021_2}, when the UAV flies at the same speed $v$ without {replacing} its battery, then the maximum distance {(in meters)} that it can travel is given as 
\begin{align}
d_\mathrm{fly}(v)=v\cdot{{\gamma\eta C_\mathrm{batt}}\over{r_\mathrm{safe}P_\mathrm{UAV}(v)}},\label{eq:3}
\end{align}
where $0<\gamma<1$ is the maximum depth of discharge of the battery, $0<\eta<1$ is the power transfer efficiency from the battery to the UAV body, and $r_\mathrm{safe}>1$ is the safety factor to reserve energy in the battery for unexpected situations. 
{We note that the consumed power in the battery is ${r_\mathrm{safe}P_\mathrm{UAV}}\over{\gamma\eta}$, where the smaller $\gamma$ and $\eta$ and the larger $r_\mathrm{safe}$ reduce the energy consumption efficiency of the UAV.}
%We note that {the smaller $\gamma$ and $\eta$ and the larger $r_\mathrm{safe}$ reduce the energy consumption efficiency of the UAV.}

\subsection{BS-UAV Connectivity}\label{sec2B}
There are $M$ BSs in the cellular network. The $m$th BS where $m\in\mathcal{M}\triangleq[1:M]$, $\mathrm{BS}_m$ is located at $(a_{m1},a_{m2},H_\mathrm{BS})$, where all BSs are assumed to be located at the same altitude $H_\mathrm{BS}<H$. We further denote $\mathbf{a}_m=(a_{m1},a_{m2})$ as the horizontally projected location of $\mathrm{BS}_m$. Each BS has a single omni-directional antenna and the same transmission power $P_\mathrm{tx}$. All the BSs are connected to a control station through a backhaul network to successfully hand over from a BS to another BS and control the UAV trajectory. 

We assume that the channel between the UAV and a BS is determined by the line-of-sight (LoS) probabilistic model, where the LoS probability increases as the {elevation} angle between the UAV and the BS increases \cite{Al-Hourani:2014}. The expected path loss between the UAV and $\mathrm{BS}_m$ at time $t$, $\Lambda_m(t)$ in $\mathrm{dB}$ is given as follows \cite{Al-Hourani:2014_2}:
\begin{align}
\Lambda_m(t) = \mathrm{FSPL}_m(t)+p_m(t)\cdot\zeta_1 + (1-p_m(t))\cdot\zeta_2,\label{eq:4}
\end{align}
where $\mathrm{FSPL}_m(t)$ and $p_m(t)\in[0,1]$ are the free space path loss and the LoS
probability between the UAV and $\mathrm{BS}_m$ at time $t$, respectively, {which only depend on the distance between the UAV and $\mathrm{BS}_m$,} and $\zeta_1>0$ and $\zeta_2>\zeta_1$ refer to the excessive path losses for LoS and non-LoS (NLoS) links, respectively.\footnote{Our path loss model is based on large-scale fading, i.e., small-scale fading effects are ignored. However, we can check that our results also hold under the small-scale fading by averaging the randomness in it.} The received signal to interference plus noise ratio (SINR) from $\mathrm{BS}_m$ to the UAV at time $t$ is
\begin{align}
\mathrm{SINR}_m(t)={{P_\mathrm{tx}\cdot 10^{\Lambda_m(t)/10}}\over {\sum_{m'\in\mathcal{M}\setminus m}{I_{m'm}}(t)+N_0}},\label{eq:5}
\end{align}
where {$I_{m'm}(t)$ is the interference power by $\mathrm{BS}_{m'}$ at time $t$ when the UAV is communicating with $\mathrm{BS}_{m}$} and $N_0$ is the additive noise power. {Note that $I_{m'm}$ would be equal to zero if $\mathrm{BS}_{m'}$ uses a different frequency band from $\mathrm{BS}_{m}$, 
and even if the two BSs use the same frequency band, it will become negligible if $\mathrm{BS}_{m'}$ is far away from the UAV.}

To maintain the control of the UAV, the communication rate from a BS to the UAV should not be less than the minimum required data rate, i.e., the maximally achievable SINR of the UAV should satisfy 
\begin{align}
\max_{m\in\mathcal{M}} \mathrm{SINR}_m(t)\geq \mathrm{SINR}_\mathrm{th}\label{eq:6}
\end{align}
for any mission $t$ where $\mathrm{SINR}_\mathrm{th}$ is the hard SINR threshold to achieve the minimum required data rate. {In weak interference regime, i.e., the frequency reuse factor is sufficiently low, it can be easily checked that the condition \eqref{eq:6} can be equivalently written as  $\min_{m\in\mathcal{M}}\|\mathbf{u}(t)-\mathbf{a}_m\|\leq d_0$ for some $d_0$, where we call $d_0$ the base coverage radius of each BS.}\footnote{{Each BS has the same base coverage radius $d_0$ since every BS has the same transmission power $P_\mathrm{tx}$ and the same altitude $H_\mathrm{BS}$, but it can be verified that our results also hold under different base coverage radii by different transmission powers or BS altitudes.}} {For other cases, however, it is in general hard to represent the exact coverage region satisfying  \eqref{eq:6} in a simple form. For tractable analysis, we introduce the coverage offset $\lambda_m\in [0,d_0]$ for $\mathrm{BS}_m$ and assume that the UAV can connect with $\mathrm{BS}_m$ with high probability if the UAV is in the effective coverage region of $\mathrm{BS}_m$ given as $\|{\mathbf{u}(t)}-\mathbf{a}_m\|\leq d_0-\lambda_m$.\footnote{{In Section \ref{sec2B}, we only state the connectivity for the downlink communication from a BS to the UAV, but it can be checked that the similar coverage region as \eqref{eq:6} is obtained in uplink communication scenarios.}} In other words, by introducing offsets $\lambda_m$ taking into account the effect of interference, we assume that $(6)$ holds with high probability if the following equation holds:
\begin{align}
\min_{m\in\mathcal{M}}\|\mathbf{u}(t)-\mathbf{a}_m\|+\lambda_m\leq d_0.\label{eq:7}
\end{align}}
Some examples of the effective coverage regions in the cellular network are illustrated in Fig. \ref{Fig2}. 

% Figure environment removed

\begin{remark}\label{Rmk1}
The possible coverage offset $\lambda_m$ for $m\in\mathcal{M}$ depends on the environment around $\mathrm{BS}_m$. For example, large (small) $\lambda_m$ is appropriate for urban (suburban) environments since there are many (few) other BSs around $\mathrm{BS}_m$. High flight altitude of $\mathrm{BS}_m$ induces large coverage offset $\lambda_m$ since the interference from other BSs is easy to come to the LoS path \cite{Lin:2019}. Also, $\lambda_m$ increases in the traffic of other BSs around $\mathrm{BS}_m$ since the received interference of the UAV is caused by the BSs which are not connected \cite{Zhang:2021}.
\end{remark}

\subsection{Charging Station Model}\label{sec2C}
To deliver the payload over a long distance with limited battery capacity, {the UAV can {replace} its battery by visiting} one of $N\leq M$ CSs. The $n$th charging station $C_n$ where $n\in\mathcal{N}\triangleq[1:N]$ {is assumed to be located at {$(c_{n1},c_{n2},H_\mathrm{CS})$, where all CSs are assumed to be located at the same altitude $H_\mathrm{CS}\leq H$.}} We further denote $\mathbf{c}_n=(c_{n1},c_{n2})$ as the horizontally projected location of $C_n$. To reduce the delay to {replace} the battery, each CS uses the autonomous battery swapping system \cite{Lee:2015}.\footnote{The autonomous battery swapping system in \cite{Lee:2015} takes about $60$ seconds for the entire battery {swapping} process.} The overall delay to {replace} the battery at charging station $C_n$, $\tau_{C_n}\in[0,\tau_\mathrm{max}]$ consists of the waiting time and the battery {swapping} time, where $\tau_\mathrm{max}$ is the upper bound on the delay. We note that the waiting time at charging station $C_n$ varies depending on the congestion of the CS and hence $\tau_{C_n}$ depends on $n$.

% Remark 2: Penalty by battery swapping itself
\begin{remark}\label{rmk2}
{When the UAV replaces the battery in a charging station $C_n$ for $n\in\mathcal{N}$, there may be a loss due to battery swapping itself, as well as the overall delay $\tau_{C_n}$ to replace it. For example, the UAV may pay a fee to each visited CS. We can consider the loss by adding it to each $\tau_{C_n}$.}
\end{remark}

\subsection{Goal}\label{sec2D}
The goal of this paper is to characterize the minimum delivery time $T$ from $\mathbf{U}_0$ to $\mathbf{U}_F$ of the UAV, including the flight time in the air and the overall delay to {replace} its battery at CSs. The optimization problem is formulated as
\begin{align} 
&\textbf{Problem 1} \cr
&\mbox{{Objective:}~}~~~~ \min_{T\geq 0,\{\mathbf{u}(t),\ \psi(t),\ t\in[0,T]\}} T\label{eq:8}\\
&{\mbox{Constraints: }}\cr%\label{eq:8.1}
&\mathbf{u}(0)=\mathbf{u}_0,\ \mathbf{u}(T)=\mathbf{u}_F,\ E_\mathrm{batt}(0)=C_\mathrm{batt} \label{eq:9}\\  
&{\mathbf{u}(t)\in\mathbb{R}^2,\ \psi(t)\in[0:N]} \label{eq:9.1}\\
&v(t)\triangleq\|\nabla_t\mathbf{u}(t)\|\in \mathcal{V},\ t\in[0,T]\label{eq:10}\\
&\min_{m\in\mathcal{M}}\|\mathbf{u}(t)-\mathbf{a}_m\|+\lambda_m\leq d_0,\ t\in[0,T]\label{eq:11}\\
&\psi(t)=0\ \mathrm{{if~}} \mathbf{u}(t)\not\in\{\mathbf{c}_n|n\in\mathcal{N}\},\ t\in[0,T]\label{eq:12}\\
&\psi(t)\in\{0,n\} \ \mathrm{{if~}}  \mathbf{u}(t)\in\{\mathbf{c}_n|n\in\mathcal{N}\},\ t\in[0,T]\label{eq:13}\\
&E_\mathrm{batt}(t)\geq 0,\ t\in[0,T]\label{eq:14}\\
&\nabla_t  E_\mathrm{batt}(t)=\! -{{r_\mathrm{safe}}\over {\gamma\eta}}P_\mathrm{UAV}(v(t)),\ \psi(t)=0,\ t\in[0,T]\label{eq:15}\\
&\nabla_t E_\mathrm{batt}(t)=0\ \mathrm{{if~}} \psi(t)\in \mathcal{N},\ t\in[0,T]\label{eq:16}\\
&E_\mathrm{batt}(t)=C_\mathrm{batt}\ \mathrm{{if~}} \psi(t)\in\mathcal{N} {\text{ and }}\cr
&~~{t-\max_{t'}\{t'|\psi(t')=0,t'\in [0,t]\}=\tau_{C_{\psi(t)}},}\ t\in[0,T]\label{eq:17}
\end{align}
where $\psi(t)\in[0:N]$ is an auxiliary variable indicating whether the UAV is in charging station $C_n$ $(\psi(t)=n)$ or in the air $(\psi(t)=0)$ at time $t$ and $E_\mathrm{batt}(t)\geq 0$ is the residual energy in the battery at time $t$.
%and $b(t)\in\{0,1\}$ {becomes 1 for time instant $t$ when replacing the battery just finished.} Note that $b(t)$ depends on $\psi(t)$ as follows:
\begin{comment}
\begin{align} 
b(t)=
\begin{cases}
1 \text{ if } \psi(t)\in\mathcal{N} \text{ and } t-\\
\quad \max_{t'}\{t'|\psi(t')=0,t'\in [0,t]\}=\tau_{C_{\psi(t)}},
\\
0 \text{ otherwise}.\label{eq:18}
\end{cases}
\end{align}
\end{comment}
Here, \eqref{eq:9} means that the UAV departs from $\mathbf{u}_0$ with fully charged battery and arrives at $\mathbf{u}_F$, {\eqref{eq:9.1} is the constraints for the optimized variables,} \eqref{eq:10} denotes that the UAV can fly with speed in $\mathcal{V}$, \eqref{eq:11} is the connectivity constraint in \eqref{eq:7}, \eqref{eq:12}-\eqref{eq:13} determines whether the UAV is in a CS or in the air, \eqref{eq:14} means that the remained energy in the battery should not have a negative value, {\eqref{eq:15} and \eqref{eq:16} represent the power consumption when flying in the air and staying in a CS, respectively, and \eqref{eq:17} means that the battery has the maximum energy when the battery swapping process is just finished.}

Note that Problem 1 is not a convex optimization problem since the variable $\psi(t)$ is selected from a discrete set and constraint \eqref{eq:11} is not convex. Moreover, $\mathbf{u}(t)$ should be optimized in continuous $t\in[0:T]$. Such difficulties make Problem 1 non-trivial. To solve this problem, in Sections \ref{sec3} and \ref{sec4}, we first reformulate Problem 1 {in a framework of {weighted} graph and then show that the problem can be solved NP-easily by graph theory-based algorithms.}

\section{{Optimal Trajectory with the Connectivity Constraint}}\label{sec3}

In this section, we provide an optimal solution for {Problem 1 without the battery constraint, i.e., the battery capacity is assumed to be unlimited.} We note that the UAV flies with {the maximum speed $v_q$} from $\mathbf{U}_0$ to $\mathbf{U}_F$ since traveling with {the maximum speed} {minimizes the mission time} without the battery constraint. Such {an} optimization problem can be reformulated as follows:
\begin{align} 
&\textbf{{Problem 1-1}} \cr
&\mbox{Objective: }~~\min_{T\geq 0,\{\mathbf{u}(t),\ t\in[0,T]\}} T\label{eq:19}\\
&{\mbox{Constraints:}} \cr %\label{eq:19.1}
&\mathbf{u}(0)=\mathbf{u}_0,\ \mathbf{u}(T)=\mathbf{u}_F,\ {\mathbf{u}(t)\in\mathbb{R}^2}\label{eq:20}\\  
&\|\nabla_t\mathbf{u}(t)\|=v_q,\ t\in[0,T]\label{eq:21}\\
&\min_{m\in\mathcal{M}}\|\mathbf{u}(t)-\mathbf{a}_m\|+\lambda_m\leq d_0,\ t\in[0,T]\label{eq:22}
\end{align}
{Note that the optimization is still not trivial since it is non-convex and has an infinite number of variables.}

{To attack Problem 1-1, we propose {a generalized intersection method} that finds a trajectory of UAV satisfying the connectivity constraint by converting Problem 1-1 as an equivalent problem of finding the shortest path in an undirected weighted graph, and show that this generalized intersection method yields an optimal UAV path  NP-easily.
The pseudo code of the generalized intersection method is described in Algorithm \ref{Algo1}.
This algorithm first checks (in line 3) whether the problem is feasible or not via the checking feasibility function ChkFea, {which outputs whether the problem 1-1 is feasible $(h_\mathrm{fea}=1)$ or not $(h_\mathrm{fea}=0)$ according to the initial point $\mathbf{u}_0$, the final point $\mathbf{u}_F$, and the location and the effective coverage region of each BS. This function can be constructed by applying \cite[Proposition~1]{Zhang:2019} in the case that the BSs have the different coverage radii and its pesudo code is omitted.}
%whose pseudo code is given in  Algorithm \ref{Algo2} and is explained later.
If the problem is feasible, an undirected weighted graph $G_0=(V_0,E_0)$ is constructed based on the intersection points of the coverage boundaries ({in lines} $6$-$17$). 
Specifically, the vertex set $V_0$ consists of the initial point $\mathbf{u}_0$, the final point $\mathbf{u}_F$, and the intersection points of the coverage boundaries (in lines $6$-$10$). 
The edge set $E_0$ is constructed (in lines $11$-$16$) by including a line segment $\overline{\mathbf{x}_1\mathbf{x}_2}$ between two different vertices $\mathbf{x}_1,\mathbf{x}_2\in V_0$ if the line segment lies inside the set of coverage regions, which is checked through the function ChkOut whose pseudo code is provided in Algorithm \ref{Algo3} and is explained later. Such an edge is denoted by a tuple $(\mathbf{x}_1,\mathbf{x}_2,{\|\mathbf{x}_1-\mathbf{x}_2\|/{v_q}})$, where the weight of the edge ${\|\mathbf{x}_1-\mathbf{x}_2\|/{v_q}}$ is given by the minimum travel time between $\mathbf{x}_1$ and $\mathbf{x}_2$.
{After constructing a weighted undirected graph, an optimal path from $\mathbf{u}_0$ to $\mathbf{u}_F$ over the graph is derived (in lines $18$-$19$). We first find an optimal sequence $\mathbf{S}_{V_0}$ of visiting nodes over the graph and the corresponding weight (equal to the mission time $T$) via the Dijkstra algorithm \cite{Dijkstra:1959} that finds the minimum weight path between two nodes over a weighted graph with low complexity. Then, the corresponding UAV trajectory can be derived through the function FindPath, which outputs the UAV trajectory according to the  sequence $\mathbf{S}_{V_0}$ of visiting points and the speed $v_q$. This function can be constructed similarly as in \cite[$(25)$-$(27)$]{Zhang:2019} and its pseudo code is omitted.} An example of the graph $G_0$ and the corresponding optimal trajectory by Algorithm \ref{Algo1} is illustrated in Fig. \ref{Fig3}.}

%%%%%%%%%%%%%%%%% Algorithm 1: generalized intersection method %%%%%%%%%%%%%%%
\begin{algorithm}
\caption{{Generalized Intersection Method}} \label{Algo1}
\textbf{Input:} $v_q$, $\mathbf{u}_0$, $\mathbf{u}_F$, $\mathbf{a}_m$, $d_0$, $\lambda_m$ for $m\in\mathcal{M}$
\begin{algorithmic}[1]
\State {\textbf{Def:} Function \textbf{ChkFea}($\mathbf{u}_0,\mathbf{u}_F,\mathbf{a}_m,d_0,\lambda_m$ for $m\in\mathcal{M}$)  outputs whether Problem 1-1 is feasible $(h_\mathrm{fea}=1)$ or not $(h_\mathrm{fea}=0)$, where $\mathbf{u}_0$ is the initial point, $\mathbf{u}_F$ is the final point, and $d_0$ and $\lambda_m$ for $m\in\mathcal{M}$ are the parameters about the communication environment.}
\State {\textbf{Def:} Function \textbf{Dijkstra}$(\mathbf{x}_1,\mathbf{x}_2,G)$ for graph $G=(V,E)$ outputs $(T,\mathbf{S}_V)$, where $T$ is the minimum total weight from $\mathbf{x}_1\in V$ to $\mathbf{x}_2\in V$ over the graph $G$ and $\mathbf{S}_V$ is the corresponding optimal sequence of visiting nodes in $V$.}
\State $V_0\leftarrow\{\mathbf{u}_0,\mathbf{u}_F\}$, $E_0\leftarrow \emptyset$
\State $h_\mathrm{fea} \leftarrow$ \textbf{ChkFea}$(\mathbf{u}_0, \mathbf{u}_F, \mathbf{a}_m, d_0, \lambda_m$ for $m\in\mathcal{M})$
\If {$h_\mathrm{fea}=1$} \hfill\Comment{Problem 1-1 is feasible}
    % 시작점, 도착점 V_0에 추가
    \LeftComment{{Step 1. Vertex  {construction}: Construct a vertex set $V_0$ consisting of the initial, the final, and the intersection points.}}
    \For{$m,m'\in\mathcal{M}$, $m<m'$} 
        \If{$\|\mathbf{a}_m-\mathbf{a}_{m'}\|\leq 2d_0-\lambda_m-\lambda_{m'}$}
            \State $V_0\leftarrow V_0\cup\{\mathbf{x}\in\mathbb{R}^2 \vert \ \|\mathbf{x}-\mathbf{a}_m\|=d_0-\lambda_m,$
            \Statex \qquad\qquad \ $|\mathbf{x}-\mathbf{a}_{m'}\|=d_0-\lambda_{m'}\}$
        \EndIf
    \EndFor 
    \LeftComment{{Step 2. Edge {construction}: Construct an edge set $E_0$ consisting of the line segments lying inside the set of coverage regions.}}
    \For{$\mathbf{x}_1,\mathbf{x}_2\in V_0$, $\mathbf{x}_1\neq \mathbf{x}_2$}
        \State $h_\mathrm{out}\leftarrow$ \textbf{ChkOut}$(\mathbf{x}_1,\mathbf{x}_2,\mathbf{a}_m, d_0, \lambda_m$ for $m\in\mathcal{M})$
        \If {$h_\mathrm{out}=0$}
            \State $E_0\leftarrow E_0\cup (\mathbf{x}_1,\mathbf{x}_2,\|\mathbf{x}_1-\mathbf{x}_2\|/v_q)$
        \EndIf
    \EndFor 
    \LeftComment{{Step 3. Path search: Find an optimal path from the initial point to  the final point over the graph.}}
    \State $G_0\leftarrow(V_0,E_0)$ \hfill\Comment{Construct graph $G_0$}
    \State {$(T,\mathbf{S}_{V_0})\leftarrow$ \textbf{Dijkstra}$(\mathbf{u}_0,\mathbf{u}_F,G_0)$}
    \State {$\mathbf{u}(t) \text{ for }t\in[0,T]\leftarrow$ \textbf{FindPath}$(\mathbf{S}_{V_0},v_q)$}
\Else \Comment{Problem 1-1 is not feasible}
\State $T\leftarrow\infty$, $\mathbf{u}(t)\leftarrow \mathrm{Null}$ for $t\in[0,T]$
\EndIf
\end{algorithmic}
\textbf{Output:} $h_\mathrm{fea}\in\{0,1\}$, $T$, $\mathbf{u}(t)$ for $t\in[0,T]$
\end{algorithm}
%%%%%%%%%%%%%%%%%%%%%%%%


%Note that only BS pairs $(\mathbf{a}_m,\mathbf{a}_{m'})$ where $m,m'\in\mathcal{M}$ and $m<m'$ have the intersected points if it satisfies the following condition:
%\begin{align}
%\|\mathbf{a}_m-\mathbf{a}_{m'}\|\leq 2d_0-\lambda_m-\lambda_{m'}.\label{eq:22}
%\end{align}

% Figure environment removed



%%% Skip algorithm 2 !!!%%%%%%%%%
\begin{comment}
{Algorithm \ref{Algo2} is  a pseudo code for the function ChkFea that checks the feasibility of Problem 1-1 by taking a graph theoretic approach. It first constructs an undirected graph $G_\mathrm{fea}=(V_\mathrm{fea},E_\mathrm{fea})$, where the vertex set $V_\mathrm{fea}$ consists of the set of {BSs $(\mathbf{a}_1,...,\mathbf{a}_M)$,} the initial point $\mathbf{a}_{M+1}\triangleq\mathbf{u}_0$, and the final point $\mathbf{a}_{M+2}\triangleq\mathbf{u}_F$ and the edge set $E_\mathrm{fea}$ includes a line segment between two different vertices $(\mathbf{a}_m,\mathbf{a}_{m'})$ if it satisfies\footnote{We set that the coverage radii of $\mathrm{BS}_{M+1}$ and $\mathrm{BS}_{M+2}$ are zero, i.e., $\lambda_{M+1}=\lambda_{M+2}=d_0$.}
\begin{align}
\|\mathbf{a}_m-\mathbf{a}_{m'}\|\leq 2d_0-\lambda_m-\lambda_{m'}.\label{eq:24}
\end{align}
Here, \eqref{eq:24} means that an UAV in the coverage region of $\mathrm{BS}_m$ can move to any location in the coverage region of $\mathrm{BS}_{m'}$ and vice versa while maintaining the connectivity. Hence, the problem is feasible if and only if the vertices $\mathbf{u}_0$ and $\mathbf{u}_F$ are connected in the graph $G_\mathrm{fea}$.\footnote{This result can be analytically proved similarly as the proof of \cite[Proposition~1]{Zhang:2019}.} Then, we check whether $\mathbf{u}_0$ and $\mathbf{u}_F$ are connected $(h_\mathrm{fea}=1)$ or not $(h_\mathrm{fea}=0)$ in the graph $G_\mathrm{fea}$ via the breadth-first search (BFS) algorithm \cite{Lee:1961}, which searches all connected nodes from a start node in a graph with low complexity.}

%%%%%%% Algorithm 2: Checking Feasibility %%%%%%%%%
\begin{algorithm}
\caption{Function ChkFea} \label{Algo2}
\textbf{Input:} $\mathbf{u}_0, \mathbf{u}_F, \mathbf{a}_m, d_0, \lambda_m \text{ for } m\in\mathcal{M}$
\begin{algorithmic}[1]
\State {\textbf{Def:} Function \textbf{BFS}$(\mathbf{x}_1,\mathbf{x}_2,G)$ for graph $G=(V,E)$ checks whether $\mathbf{x}_1\in V$ and $\mathbf{x}_2\in V$ are connected in the graph $G$ (output: $1$) or not (output: $0$).}
\State $\mathbf{a}_{M+1}\leftarrow \mathbf{u}_0$, $\mathbf{a}_{M+2}\leftarrow \mathbf{u}_F$, $\lambda_{M+1},\lambda_{M+2}\leftarrow d_0$
\State $V_\mathrm{fea}\leftarrow\{\mathbf{u}_0,\mathbf{u}_F,\mathbf{a}_m \text{ for } m\in\mathcal{M}\}$, $E_\mathrm{fea}\leftarrow \emptyset$
\LeftComment{Lines $4$-$8$: Construct edge set $E_\mathrm{fea}$}
\For{$m,m'\in[1:M+2]$, $m<m'$} 
    \If{$\|\mathbf{a}_m-\mathbf{a}_{m'}\|\leq 2d_0-\lambda_m-\lambda_{m'}$}
    \State $E_\mathrm{fea}\leftarrow E_\mathrm{fea}\cup (\mathbf{a}_m,\mathbf{a}_{m'})$
    \EndIf
\EndFor
\State $G_\mathrm{fea}\leftarrow(V_\mathrm{fea},E_\mathrm{fea})$ 
\hfill\Comment{Construct graph $G_\mathrm{fea}$}
\LeftComment{{Test whether $\mathbf{u}_0$ and $\mathbf{u}_F$ are connected in $G_\mathrm{fea}$}}
\State $h_\mathrm{fea}\leftarrow$ \textbf{BFS}$(\mathbf{u}_0,\mathbf{u}_F,G_\mathrm{fea})$
\end{algorithmic}
\textbf{Output:} $h_\mathrm{fea}\in\{0,1\}$
\end{algorithm}
%%%%%%%%%%%%%%%%%%%%%%%%
\end{comment}

% Algorithm 3 동작 원리 설명
{Algorithm \ref{Algo3} describes the function ChkOut which tests whether a line segment $\overline{\mathbf{x}_1\mathbf{x}_2}$ between two different vertices $\mathbf{x}_1,\mathbf{x}_2\in V_0$ lies in the set of coverage regions. We say that the line segment experiences an outage if there exists $\xi\in[0,1]$ that satisfies the following condition:}
\begin{align}
\min_{m\in\mathcal{M}}\|\pmb{\alpha}(\xi)-\mathbf{a}_m\|+\lambda_m>d_0,\label{eq:25}
\end{align}
{where $\pmb{\alpha}(\xi)\triangleq \mathbf{x}_1+\xi(\mathbf{x}_2-\mathbf{x}_1)$ for $\xi\in[0,1]$ represents a point in the line segment $\overline{\mathbf{x}_1\mathbf{x}_2}$. Here, \eqref{eq:25} means that the UAV  experiences an outage at point $\pmb{\alpha}(\xi)$, i.g., the UAV cannot be connected with every BS at point $\pmb{\alpha}(\xi)$.} {To check whether the line segment experiences an outage, the function ChkOut verifies whether there exists $\xi\in[0,1]$ that satisfies \eqref{eq:25}. Let us define the safe interval $\mathcal{T}_\mathrm{safe}\triangleq [0,\xi']$ for some $\xi'\in[0,1]$ as the line segment such that every $\pmb{\alpha}(\xi)$ for $\xi\in \mathcal{T}_\mathrm{safe}$ has been checked to be inside the coverage regions, i.e., there exists $m\in \mathcal{M}$ such that $\|\pmb{\alpha}(\xi)-\mathbf{a}_m\|+\lambda_m \leq d_0$.  The function ChkOut first checks whether $\xi=0$ is included in the coverage regions and then repeatedly updates the safe interval or declares an outage in the following way. Let the current safe interval be given as $[0,\xi']\subseteq [0,1]$. If the point $\pmb{\alpha}(\xi'+\epsilon)$ is checked to be connected with $\mathrm{BS}_m$ for sufficiently small constant $\epsilon>0$, then the safe interval is extended by including the range of $\xi$ where $\mathbf{\alpha}(\xi)$ is connected with $\mathrm{BS}_m$, i.e.,  $\|\pmb{\alpha}(\xi)-\mathbf{a}_m\|\leq d_0-\lambda_m$. This algorithm ends if $\pmb{\alpha}(\xi'+\epsilon)$ cannot be connected with every BS $(h_\mathrm{out}=1)$ or the safe interval reaches  $[0,1]$ $(h_\mathrm{out}=0)$, where $h_\mathrm{out}$ is the indicator whether the line segment experiences an outage $(h_\mathrm{out}=1)$ or not $(h_\mathrm{out}=0)$. An example of updating the safe interval is shown in Fig. \ref{Fig4}.}

%%%%%%% Algorithm 3: Checking Outage %%%%%%%%%%
\begin{algorithm}
\caption{Function ChkOut} \label{Algo3}
\textbf{Input:} $\mathbf{x}_1,\mathbf{x}_2\in V_0,\mathbf{a}_m, d_0, \lambda_m \text{ for } m\in\mathcal{M}$
\begin{algorithmic}[1]
\State\textbf{Def:} $\pmb{\alpha}(\xi)\triangleq \mathbf{x}_1+\xi(\mathbf{x}_2-\mathbf{x}_1)$ for $\xi\in[0,1]$
\State $h_\mathrm{out}\leftarrow 0$, $\xi'\leftarrow 0$, $\xi''\leftarrow 0$ 
\State {$\epsilon\leftarrow 10^{-10}$ \hfill\Comment{Sufficiently small positive constant}}
\While{$\xi'<1$}   %\hfill\Comment{Edge $(\mathbf{x}_1,\mathbf{x}_2)$ is covered if $\xi'=1$}  
\LeftComment{{Update safe interval $\mathcal{T}_\mathrm{safe}$ from 
$[0,\xi']$ to $[0,\xi'']$ if $\pmb{\alpha}(\xi'+\epsilon)$ is included in the set of coverage regions.}}
    \For {$m\in\mathcal{{M}}$}  \hfill\Comment{Find BS which covers $\pmb{\alpha}(\xi'+{\epsilon})$}
        \If{$\|\pmb{\alpha}(\xi'+{\epsilon})-\mathbf{a}_m\|\leq d_0-\lambda_m$}
            \State $\xi''\!\leftarrow\!\max\{\xi\in [0,1] \vert \ \|\pmb{\alpha}(\xi)-\mathbf{a}_m\|\leq d_0-\lambda_m\}$
            %\State $\mathcal{M''}\leftarrow\mathcal{M'}\setminus m$
            \State \textbf{break} 
        \EndIf
    \EndFor
    \If{$\xi''=\xi'$}  {\hfill\Comment{ $\pmb{\alpha}(\xi'+\epsilon)$ experiences an outage.}}
         \State $h_\mathrm{out}\leftarrow 1$
         \State  \textbf{break}
    \EndIf
    \State $\xi'\leftarrow\xi''$
    %\If{$\mathcal{M''}=\mathcal{M'}$}  \hfill\Comment{$\pmb{\alpha}(\xi+0)$ is not covered}
    %    \State $h_\mathrm{out}\leftarrow 1$
    %    \State \textbf{break}
    %\EndIf
    %\State $\mathcal{M'}\leftarrow\mathcal{M''}$
\EndWhile {\hfill\Comment{$\xi'=1$ means that $\mathcal{T}_\mathrm{safe}=[0,1]$.}}
\end{algorithmic}
\textbf{Output:} $h_\mathrm{out}\in\{0,1\}$
\end{algorithm}
%%%%%%%%%%%%%%%%%%%%%%%%%%%%%%%%%%%%%%%%

%This algorithm repeatedly extends the trusted interval $[0,\xi']$ where every $\pmb{\alpha}(\xi)$ for $\xi\in [0,\xi']$ has been checked to be covered by the BSs and $\xi'\in[0,1]$ is an updating factor.
%We note that the trusted interval is extended by refining the updating factor $\xi'\in[0,1]$ from $\xi'=0$ to $\xi'=1$. Algorithm \ref{Algo3} starts with the initial updating factor $\xi'=0$. Let the trusted interval is $[0,\xi']\subseteq [0,1]$. If $\pmb{\alpha}(\xi'+0)$ is checked to be covered by $\mathrm{BS}_m$, then we extend the trusted interval by including the range of $\xi$ where $\mathbf{\alpha}(\xi)$ is covered by $\mathrm{BS}_m$ (Lines 4-9, 14). We note that the updating factor $\xi'$ increases when the trusted interval $[0,\xi']$ is extended. This algorithm ends if $\pmb{\alpha}(\xi'+0)$ is not covered by any BS $(h_\mathrm{out}=1)$ or the updating factor $\xi'$ is refined to $\xi'=1$ $(h_\mathrm{out}=0)$, where $h_\mathrm{out}$ is the hypothesis whether the line segment is an outage $(h_\mathrm{out}=1)$ or not $(h_\mathrm{out}=0)$. An example of the hypothesis is illustrated in Fig. \ref{Fig4}.

% Figure environment removed

{Now, the following theorems show that our generalized intersection method yields an optimal solution of Problem 1-1 NP-easily.}\footnote{{We note that the intersection points are crucial for deriving an optimal trajectory as shown in Fig. \ref{Fig5}.}}

\begin{theorem}\label{Thm1}
The generalized intersection method outputs an optimal solution for Problem 1-1.
\end{theorem}
\begin{proof}
We first notice that an optimal solution of Problem 1-1 consists of line segments, where its breakpoints are selected in the overlapping region of any two different BSs \cite[Proposition~3]{Zhang:2019} and the problem is equivalent to deriving a path which achieves the shortest distance from $\mathbf{u}_0$ to $\mathbf{u}_F$ since the speed of the UAV is fixed at $v_q$. We prove the theorem by showing that an alternative solution whose breakpoints are only selected in the intersection points of the coverage boundaries can be always constructed and it achieves equal or smaller delivery time $T$ compared to the solution of \cite{Zhang:2019}. 

Without loss of generality, we consider a base trajectory with BS association sequence $[\mathrm{BS}_1,\mathrm{BS}_2,...,\mathrm{BS}_k]$ for $k\leq M$, where its $k'$th breakpoint $\pmb{\beta}_{k'}\in\mathbb{R}^2$ $(k'\leq k-1)$ satisfies the following condition:
\begin{align}
\|\pmb{\beta}_{k'}-\mathbf{a}_j\| \leq d_0-\lambda_j \text{  for  } j\in\{k',k'+1\}. \label{eq:26}
\end{align}
For convenience, we set $\pmb{\beta}_k=\mathbf{u}_F$. Such initial trajectory $\mathcal{L}$ consists by the line segments $\overline{\mathbf{u}_0\pmb{\beta}_1}$, $\overline{\pmb{\beta}_1\pmb{\beta}_2}$,..., $\overline{\pmb{\beta}_{k-1}\pmb{\beta}_k}$. To reduce the length of the trajectory, we repeatedly update the trajectory $\mathcal{L}$ from $\mathbf{u}_0$ to $\pmb{\beta}_k$. Let us define the completed path $\mathcal{L}_c$ from $\mathbf{u}_0$ to $\mathbf{\bar{u}}$ is the completely updated part of the trajectory, where $\mathbf{\bar{u}}\in\mathbb{R}^2$ is the updating point. We repeatedly save the completely updated part of the trajectory $\mathcal{L}$ in the completed path $\mathcal{L}_c$. We note that $\mathcal{L}_c$ is updated by refining the updating point $\mathbf{\bar{u}}$ and its breakpoints are only selected in the intersection points. We start at the updating point $\mathbf{\bar{u}}\leftarrow\mathbf{u}_0$ and then verify whether $\overline{\mathbf{\bar{u}}\pmb{\beta}_1}$ and $\overline{\pmb{\beta}_1\pmb{\beta}_2}$ can be replaced by $\overline{\mathbf{\bar{u}}\pmb{\beta}_2}$ in the initial trajectory since $\|\mathbf{\bar{u}}-\pmb{\beta}_2\| \leq \|\mathbf{\bar{u}}-\pmb{\beta}_1\| + \|\pmb{\beta}_1-\pmb{\beta}_2\|$ holds by triangular inequality.
%\begin{align}
%\|\mathbf{\bar{u}}-\pmb{\beta}_2\| \leq \|\mathbf{\bar{u}}-\pmb{\beta}_1\| + \|\pmb{\beta}_1-\pmb{\beta}_2\|. \label{eq:26.1}
%\end{align}
If $\overline{\mathbf{\bar{u}}\pmb{\beta}_2}$ does not experience an outage, then we update the initial trajectory as $\mathcal{L}:$ $\overline{\mathbf{\bar{u}}\pmb{\beta}_2}$, $\overline{\pmb{\beta}_2\pmb{\beta}_3}$,..., $\overline{\pmb{\beta}_{k-1}\pmb{\beta}_k}$ and repeat the same process until $\overline{\mathbf{\bar{u}}\pmb{\beta}_i}$ $(i\in[2:k])$ experiences an outage.\footnote{We note that the trajectory $\overline{\mathbf{u}_0\mathbf{u}_F}$ is the alternative solution of Problem 1-1 if $\overline{\mathbf{\mathbf{u}_0\pmb{\beta}_k}}$ does not still experience an outage.} When $\overline{\mathbf{\bar{u}}\pmb{\beta}_i}$ experiences an outage, we have the updated trajectory $\mathcal{L}:$ $\overline{\mathbf{\bar{u}}\pmb{\beta}_{i-1}}$, $\overline{\pmb{\beta}_{i-1}\pmb{\beta}_i}$,..., $\overline{\pmb{\beta}_{k-1}\pmb{\beta}_k}$ satisfying the following inequality:
\begin{align}
\|\mathbf{\bar{u}}-\pmb{\beta}_{i-1}\|+ \sum_{j=i}^{k}\|\pmb{\beta}_{j-1}-\pmb{\beta}_j\|  \leq \ell_0,\label{eq:27}
\end{align}
where $\ell_0=\|\mathbf{\bar{u}}-\pmb{\beta}_1\| + \sum_{j=2}^{k}\|\pmb{\beta}_{j-1}-\pmb{\beta}_j\|$ is the length of the initial trajectory. Here, \eqref{eq:27} means the length of the updated trajectory is not larger than $\ell_0$. 

Next, we improve the trajectory $\mathcal{L}:$ $\overline{\mathbf{\bar{u}}\pmb{\beta}_{i-1}}$, $\overline{\pmb{\beta}_{i-1}\pmb{\beta}_i}$,..., $\overline{\pmb{\beta}_{k-1}\pmb{\beta}_k}$, where $i\in[2:k]$ and $\overline{\mathbf{u}_0\pmb{\beta}_i}$ experiences an outage. Let us assume that $\pmb{\tilde{\beta}}_{i'1}$ and $\pmb{\tilde{\beta}}_{i'2}$ are the two intersection points of $\mathrm{BS}_{i'}$ and $\mathrm{BS}_{i'+1}$ where $i'\in [1:k-1]$. We select a intersection point $\pmb{\tilde{\beta}}_{i',j}$ which satisfies the following condition:
\begin{align}
\begin{split}\label{eq:28}
\pmb{\tilde{\beta}}_{i',j}&=\argmax_{\pmb{\tilde{\beta}}_{i',j}}\bigl\{i'|\pmb{\tilde{\beta}}_{i',j}\in \Delta(\mathbf{\bar{u}}\pmb{\beta}_{i-1}\pmb{\beta}_{i}), \ \overline{\mathbf{\bar{u}}\pmb{\tilde{\beta}}_{i',j}}\text{ lies}\\
&\text{in the set of the coverage regions}\bigr\}, 
\end{split}
\end{align}
where $\Delta(\mathbf{x}_1\mathbf{x}_2\mathbf{x}_3)$ is the convex hull by $\mathbf{x}_1, \mathbf{x}_2, \mathbf{x}_3 \in \mathbb{R}^2$.\footnote{It can be checked that the set in \eqref{eq:28} is not an empty set.} Then, we can find the intersection $\pmb{\phi}\in\mathbb{R}^2$ of $\overrightarrow{\mathbf{\bar{u}}\pmb{\tilde{\beta}}_{i',j}}$ and $\overline{\pmb{\beta}_{i-1}\pmb{\beta}_{i}}$. The trajectory $\mathcal{T}$ is updated as $\overline{\mathbf{\bar{u}}\pmb{\phi}}$, $\overline{\pmb{\phi}\pmb{\beta}_{i-1}}$,..., $\overline{\pmb{\beta}_{k-1}\pmb{\beta}_k}$ since the following holds by triangular inequality:
\begin{align}
\|\mathbf{\bar{u}}-\pmb{\phi}\| +\|\pmb{\phi}-\pmb{\beta}_i\| +\sum_{j=i+1}^{k}\|\pmb{\beta}_{j-1}-\pmb{\beta}_j\|  \leq \ell_0.\label{eq:29}
\end{align}

Finally, we include the line segment $\overline{\mathbf{\bar{u}}\pmb{\tilde{\beta}}_{i',j}}\subseteq \overline{\mathbf{\bar{u}}\pmb{\phi}}$ as a part of the completed path $\mathcal{L}_c$ and then refining the updating point $\mathbf{\bar{u}}\leftarrow \pmb{\tilde{\beta}}_{i',j}$. We repeatedly perform the entire process at the remained trajectory $\overline{\mathbf{\bar{u}}\pmb{\phi}}$, $\overline{\pmb{\phi}\pmb{\beta}_{i-1}}$,..., $\overline{\pmb{\beta}_{k-1}\pmb{\beta}_k}$ and update the completed path by refining  $\mathbf{\bar{u}}$ until $\overline{\mathbf{\mathbf{\bar{u}}\pmb{\beta}_k}}$ does not experience an outage. The completed path $\mathcal{L}_c$ from $\mathbf{u}_0$ to $\pmb{\beta}_k$ including $\overline{\mathbf{\mathbf{\bar{u}}\pmb{\beta}_k}}$ is an alternative solution of Problem 1-1 since its length is not larger than $\ell_0$ and it consists of the line segments whose breakpoints are in $\bigl\{\pmb{\tilde{\beta}}_{i,j}|i\in[1:k-1],j\in\{1,2\}\bigr\}$. The proof process is also shown in Fig. \ref{Fig6}.
%공집합 아니라는거 보이기: footnote
\end{proof}

% Figure environment removed

% Figure environment removed

\begin{table*}
\centering
\begin{tabular}{@{} c || c | c @{}}
\cline{1-3}
Algorithm & Complexity & Performance gap\\ \cline{1-3}
Exhaustive search \cite{Zhang:2019} & $O(M!M^{3.5})$ & 0\\ \cline{1-3}
Exhaustive search with fixed association \cite{Zhang:2019} & $O(M^{3.5})$ & $O(Md_0/{v_q})$ \\ \cline{1-3}
Exhaustive search with quantization \cite{Zhang:2019} & $O(M^4Q^2)$ & $O((Md_0/{v_q})\sin(1/{Q}))$ \\ \cline{1-3}
Intersection method \cite{Chen:2020} {by checking outages via Algorithm \ref{Algo3}} & $O(M^4)$ & $O(Md_0/{v_q})$ \\ \cline{1-3}
Ours {(Generalized intersection method)}  & $O(M^6)$ & $0$ \\ \cline{1-3}
\end{tabular}
\caption{Comparison of algorithms for Problem 1-1}\label{Tab1}
\end{table*}

\begin{theorem}\label{Thm2}
The time complexity of the generalized intersection method is $O(M^6)$.
\end{theorem}
\begin{proof}
Let us first state the cardinality of the set $|V_0|=O(M^2)$. The steps in Algorithm \ref{Algo1} have the following complexities:
\begin{itemize}
\item Complexity of function ChkFea: It was shown that the complexity to check whether Problem 1-1 is feasible is $O(M^2)$ \cite{Zhang:2019}.
\item Step 1. Vertex construction: This step has complexity $O(M^2)$ since the intersection points of the coverage boundaries by a BS pair is derived by calculating the quadratic equations in Line $6$ of Algorithm \ref{Algo1} and the number of the possible BS pairs is $O(M^2)$.
\item Step 2. Edge construction: The complexity of testing whether a line segment experiences an outage via the function ChkOut is $O(M^2)$ and every line segment $\overline{\mathbf{x}_1\mathbf{x}_2}$ by two different vertices $\mathbf{x}_1,\mathbf{x}_2\in V_0$ should be tested. Hence, the complexity of this step is $O(M^2)\cdot |V_0|^2=O(M^6)$. 
\item Step 3. Path search: The complexity of the Dijkstra algorithm in the graph $G_0$ is $O(|V_0|^2)=O(M^4)$ \cite{West:2001}.
\end{itemize}
Consequently, the complexity of the generalized intersection method is $O(M^6)$, which is dominated at the edge $E_0$ construction step.
\end{proof}

{Now, let us compare our generalized intersection method with previously proposed algorithms to solve Problem 1-1. Table \ref{Tab1} summarizes the complexity and the performance gap from the optimal solution for each algorithm. In the following, we provide brief descriptions of previous algorithms and observations based on  Table \ref{Tab1}. }
\begin{itemize}
    \item {Among the algorithms in Table \ref{Tab1}, our generalized intersection method outputs an optimal solution NP-easily.}
    \item {The exhaustive search (ES), exhaustive search with fixed association (ES-FA), and exhaustive search with quantization (ES-Q) algorithms are proposed in \cite{Zhang:2019}. In \cite{Zhang:2019}, it was shown that an optimal trajectory from $\mathbf{u}_0$ to $\mathbf{u}_F$ consists of line segments, where its breakpoints are selected inside the overlapping regions of the coverage regions of two different BSs \cite[Proposition~3]{Zhang:2019}. In this approach, it is not possible to find an optimal solution via a graph theoretic approach since the overlapping regions consist of infinite number of points. The ES algorithm \cite{Zhang:2019} is an optimal algorithm that finds optimal breakpoints inside the overlapping regions based on convex optimization, which is NP-hard over $M$. To reduce the complexity, two suboptimal algorithms are also proposed in \cite{Zhang:2019}, i.e., ES-FA and ES-Q algorithms, which are NP-easy. The ES-FA algorithm is basically the same with ES {algorithm}, except that the sequence of BS association is fixed in advance, and the ES-Q algorithm applies a graph theoretic approach by quantizing each overlapping region to a finite number of the points.} {The ES-FA algorithm has lower complexity than the generalized intersection method, but its performance gap increases in $M$. For the ES-Q algorithm, let $Q$ denote the number of quantization points in each overlapping region. Note that this algorithm has an increasing performance gap in $M$ for $Q=O(M)$ and has a higher complexity than the generalized intersection method for $Q=\omega(M)$.}
    \item {The intersection method proposed in \cite{Chen:2020} only includes the intersection points as the possible breakpoints and applies a graph theoretic approach like our generalized intersection method. However, this algorithm is suboptimal because it searches a path for a fixed BS association sequence which is chosen in a heuristic way, similarly as the ES-FA algorithm \cite{Zhang:2019}. Also, it does not explicitly suggest a function like our ChkOut function in Algorithm \ref{Algo3}, checking whether each line segment between two vertices in the graph experiences an outage. If we apply the ChkOut function in Algorithm \ref{Algo3}, the intersection method \cite{Chen:2020} has the same performance gap with the ES-FA algorithm \cite{Zhang:2019} with a higher complexity.} 
\end{itemize}

%\item The work \cite{Chen:2020} proposed the intersection method which is similar with the Suboptimal-association algorithm \cite{Zhang:2019} but only includes the intersection points as the possible breakpoints since the intersection points are crucial for deriving an optimal trajectory as shown in Fig. \ref{Fig5}. 

%The intersection method is NP-easy and only uses graph theoretic approach. However, this algorithm is not optimal since it searches a path in a fixed BS association sequence. We proposed our LCI method which considers every BS association and checks whether each line segment between two vertices in the graph $G_0$ experiences an outage by Algorithm \ref{Algo3}, which is a newly proposed low complexity algorithm to check whether a line segment experiences an outage. The following theorems show that our LCI method derives an optimal algorithm of problem 1-1 NP-easily.
    
%\item The suboptimal-association \cite{Zhang:2019} and the Intersection \cite{Chen:2020} methods have lower complexity than the LCI method, but their performance gap bounds increase in $M$.
    
%\item The suboptimal-quantization algorithm \cite{Zhang:2019} has the increasing performance gap bound on $M$ for $Q=O(M)$ and higher complexity than the LCI method for $Q=\omega(M)$.

    
%The aforementioned analysis implies that our LCI method improves the performance of Problem 1-1 compared to the other algorithms in both complexity and performance gap bound. 

\section{{Optimal Trajectory with the Connectivity and Battery Constraints}}\label{sec4}
{In this section, we target to solve Problem 1, i.e.,  optimize the UAV trajectory to minimize the mission time under the connectivity and the battery constraints.\footnote{{From now on, we assume that $H_\mathrm{CS}=H$.}} Note that it can be beneficial to change the UAV speed $v$ under the battery constraint since the maximum travel distance without replacing the battery depends on $v$ as shown in \eqref{eq:3}.} %{For convenience, we assume that $H_\mathrm{CS}=H$.} %We note that Problem 1 is more challenging than Problem 1-1 since the problem considers not only maintaining the connection with a BS but also keeping the battery from running out by visiting some CSs. 

{For Problem 1, we propose a generalized intersection method with battery constraint (GIM-B) by modifying our generalized intersection method in Section \ref{sec3}, and show that this GIM-B algorithm
outputs an optimal solution for Problem~1 NP-easily. The pseudo code for this GIM-B algorithm is provided in Algorithm \ref{Algo4}. 
%%%%%%%%%%%%%%%%% Algorithm 4: GIM-B %%%%%%%%%%%%%%%
\begin{algorithm}
\caption{{Generalized Intersection Method with Battery Constraint (GIM-B)}} \label{Algo4}
\textbf{Input:} $\mathbf{u}_0$, $\mathbf{u}_F$, $\mathbf{a}_m$, $d_0$, $\lambda_m$, $\mathcal{V}$, $\mathbf{c}_n$, $\tau_{C_n}$ {$w$, $w_2$} for $m\in\mathcal{M}$, $n\in\mathcal{N}$
\begin{algorithmic}[1]
\State {\textbf{Def:} Function \textbf{BFS}$(\mathbf{x}_1,\mathbf{x}_2,G)$ for graph $G=(V,E)$ outputs 1 if  $\mathbf{x}_1\in V$ and $\mathbf{x}_2\in V$ are connected in the graph $G$ and otherwise outputs 0.}
\State $V_\mathrm{GL}\leftarrow\{\mathbf{u}_0,\mathbf{u}_F,\mathbf{c}_1,...,\mathbf{c}_N\}$, $V_\mathrm{LO},E_\mathrm{LO},E_\mathrm{GO}\leftarrow \emptyset$
\State $V_\mathrm{in},E_\mathrm{in},E_1,...,E_{N+2} \leftarrow \emptyset$
\LeftComment{{Consider the initial and the final points as CSs.}}
\State $\mathbf{c}_{N+1} \leftarrow \mathbf{u}_0$, $\mathbf{c}_{N+2} \leftarrow \mathbf{u}_F$, $\tau_{C_{N+1}},\tau_{C_{N+2}}\leftarrow 0$
\State $V_\mathrm{in}\!\leftarrow\! \text{All intersection points}$ \hfill\Comment{{Lines $6$-$10$ at Algorithm \ref{Algo1}}}
%\For{$m,m'\in\mathcal{M}$, $m<m'$} \hfill\Comment{Find intersected points}
%    \If{$\|\mathbf{a}_m-\mathbf{a}_{m'}\|\leq 2d_0-\lambda_m-\lambda_{m'}$}
%        \State $V_\mathrm{in}\leftarrow V_\mathrm{in}\cup\{\mathbf{x}\in\mathbb{R}^2 \vert \ \|\mathbf{x}-\mathbf{a}_m\|=d_0-\lambda_m,$
%        \Statex \qquad\quad $|\mathbf{x}-\mathbf{a}_{m'}\|=d_0-\lambda_{m'}\}$
%    \EndIf
%\EndFor 
\State $V_\mathrm{all}\leftarrow V_\mathrm{GL}\cup V_\mathrm{in}$
\LeftComment{{Step 1. Outage test: Check whether each possible line segment experiences an outage.}}
\For{$\mathbf{x}_1,\mathbf{x}_2\in V_\mathrm{all}$, $\mathbf{x}_1\neq \mathbf{x}_2$} 
    \State $h_\mathrm{out}\leftarrow$ \textbf{ChkOut}$(\mathbf{x}_1,\mathbf{x}_2,\mathbf{a}_m, d_0, \lambda_m \text{ for } m\in\mathcal{M})$
    \For {$n\in [1:N+2]$}
        \If{$h_\mathrm{out}=0$, $\mathbf{c}_{n}\in \{\mathbf{x}_1,\mathbf{x}_2\}$ }
            \State $E_n\leftarrow E_n\cup (\mathbf{x}_1,\mathbf{x}_2,\|\mathbf{x}_1-\mathbf{x}_2\|)$
        \EndIf
    \EndFor
    \If{$h_\mathrm{out}\!=\!0$, ${\mathbf{c}_{n}}\!\not\in\!\{\mathbf{x}_1,\mathbf{x}_2\}$ for $n\in[1:N+2]$}
        \State $E_\mathrm{in}\leftarrow E_\mathrm{in}\cup (\mathbf{x}_1,\mathbf{x}_2,\|\mathbf{x}_1-\mathbf{x}_2\|)$
    \EndIf
\EndFor 
\LeftComment{{Step 2. Local level search: Derive optimal paths between each pair of CSs.}}
\For{$n\in [1:N+1]$, $n'\in[1:N]\cup \{N+2\}$, $n\neq n'$} 
    \LeftComment{{Function \textbf{ChkFea} is described in line $1$ at Algorithm \ref{Algo1}.}} 
    \State $h_\mathrm{Lfea}\leftarrow$ \textbf{ChkFea}$(\mathbf{c}_n,\mathbf{c}_{n'}, \mathbf{a}_m, d_0, \lambda_m$ for $m\in\mathcal{M})$ 
    \If{$h_\mathrm{Lfea}=1$}
        \State $V_\mathrm{LO}\leftarrow V_\mathrm{in}\cup \{\mathbf{c}_n, \mathbf{c}_{n'}\}$, $E_\mathrm{LO}\leftarrow E_\mathrm{in}\cup E_n\cup E_{n'}$
        \State $G_\mathrm{LO}\leftarrow (V_\mathrm{LO}, E_\mathrm{LO})$
        \LeftComment{{Function \textbf{Dijkstra} is described in line $2$ at Algorithm \ref{Algo1}.}} 
        \State {$(\ell_\mathrm{LO},\mathbf{S}_{V_\mathrm{LO}}(c_n, c_{n'}))\leftarrow \textbf{Dijkstra}(\mathbf{c}_n,\mathbf{c}_{n'},G_\mathrm{LO})$}
        \State ($h_\mathrm{sp},{v(c_n, c_{n'})}) \leftarrow$ \textbf{ChkSp}$(\ell_\mathrm{LO},\mathcal{V},{w,w_2})$
        \If{$h_\mathrm{sp}=1$}
            %\State {$\mathbf{p}_\mathrm{LO}(\mathbf{c}_n,\mathbf{c}_{n'})\leftarrow$\textbf{FdPath}$(\mathbf{S}_{V_\mathrm{LO}},v_\mathrm{max})$}
            \State {$E_\mathrm{GL}\leftarrow E_\mathrm{GL}\cup (\mathbf{c}_n,\mathbf{c}_{n'},\ell_\mathrm{LO}/v(c_n, c_{n'})+\tau_{C_{n'}})$}
        \EndIf
    \EndIf
\EndFor  
\LeftComment{{Step 3. Global level search: Derive an optimal path from the initial point to the final point over the graph of CSs.}}
\State $\overrightarrow{G}_\mathrm{GL}\leftarrow (V_\mathrm{GL}, E_\mathrm{GL})$ \hfill\Comment{{$\overrightarrow{G}_\mathrm{GL}$ is a directed graph.}}
\State $h_\mathrm{Gfea}\leftarrow$ \textbf{BFS}$(\mathbf{u}_0,\mathbf{u}_F,\overrightarrow{G}_\mathrm{GL})$
\If{$h_\mathrm{Gfea}=1$}
    \State {($T$, $\mathbf{S}_{V_\mathrm{GL}})\leftarrow$ $\textbf{Dijkstra}(\mathbf{u}_0,\mathbf{u}_F,\overrightarrow{G}_\mathrm{GL})$} 
    \State {($\mathbf{u}(t)$, $\psi(t)$ for $t\in[0,T]$) $\leftarrow$ \textbf{FindPathG}$(\mathbf{S}_{V_\mathrm{GL}},$}
    \Statex\qquad {$v(c_n, c_{n'})$, $\mathbf{S}_{V_\mathrm{LO}}(c_n, c_{n'})$, $\tau_{C_{n'}}$ for $n\in[1:N+1],$}
    \Statex\qquad {$n'\in [1:N]\cup \{N+2\}$)}
\Else
    \State $h_\mathrm{Gfea}\leftarrow 0$, $T\leftarrow\infty$, $\mathbf{u}(t),\psi(t) \leftarrow \mathrm{Null}$ for $t\in[0,T]$
\EndIf   
\end{algorithmic}
\textbf{Output:} $h_\mathrm{Gfea}\in\{0,1\}$, $T$, $\mathbf{u}(t)$, $\psi(t)$ for $t\in[0,T]$
\end{algorithm}
%%%%%%%%%%%%%%%%%%%%%%%%
The resultant UAV trajectory from our algorithm consists of line segments between two points, where each point is one of the initial or final point, intersection points of the coverage boundaries, and CSs. The GIM-B algorithm determines the set of line segments (corresponding to edges in the equivalent graphs) that are connected one after another in three steps. First, it checks whether each possible line segment experiences an outage and constitutes the no-outage edge sets (in lines $7$-$17$). Then, the algorithm finds the optimal path in two levels. In the local level (in lines $18$-$29$), it finds the optimal path between each pair of CSs (by treating the initial and the final points also as CSs) by applying Dijkstra algorithm and derives the maximum allowable speed to travel between each pair of CSs by applying the function ChkSP whose pseudo code is in Algorithm \ref{Algo5}.\footnote{{We assume that the UAV flies with a fixed speed while traveling through a path at the local level, which is justified later in Theorem \ref{Thm3}.}} In this local level, note that it may not be possible to travel between two CSs because there is no path  between them  ($h_\mathrm{Lfea}=0$) or because the distance is too large to travel with the battery capacity ($h_\mathrm{sp}=0$). {An example of the graph to derive an optimal path between two CSs at the local level is illustrated in Fig. \ref{Fig7}.}
% Figure environment removed
In the global level (in lines $30$-$35$), we consider a {directed} graph whose vertex set consists of CSs and edge set consists of edges between CSs which has been checked to be reachable in the local level, with the weights of traveling and battery swapping time. For this graph, the algorithm first checks whether it is feasible to travel from the initial to the final points {via the function breadth-first search (BFS) \cite{West:2001}, which searches all connected nodes from a start node in a graph with low complexity. If feasible,} it constructs the UAV trajectory by applying the Dijkstra algorithm over the graph and then applying the function FindPathG that outputs the trajectory based on the sequence of visiting points in the global and the local levels, the speeds traveling between CSs, and the battery swapping times.} {An example of the graph to derive an optimal path between from the initial point to the final point at the global level is illustrated in Fig. \ref{Fig8}.}
% Figure environment removed

%This method first derives an optimal local-path from the initial point or a CS $\mathbf{l}_0\in\{\mathbf{u}_0,\mathbf{c}_1,...,\mathbf{c}_n\}$ to the final point or another CS $\mathbf{l}_F\in\{\mathbf{u}_F,\mathbf{c}_1,...,\mathbf{c}_n\}$ without replacing the battery by constructing a local graph.} Then, to obtain an optimal trajectory from $\mathbf{u}_0$ to $\mathbf{u}_F$, it constructs a global graph including $\mathbf{u}_0$, $\mathbf{u}_F$, and $c_n$ for $n\in\mathcal{N}$ by using the induced paths in the local-graphs. The detailed pseudo code about our GIM-B is provided in Algorithm \ref{Algo4}. To reduce the time complexity of verifying whether each line segment used in the local-graphs experiences an outage, we test the hypothesis for every line segment at once before constructing the local-graphs (in lines $4$-$16$). We first find all intersection points of the coverage boundaries of the BSs and then check whether each line segment $\overline{\mathbf{x_1}\mathbf{x_2}}$ by two different vertices $\mathbf{x_1},\mathbf{x_2}\in V_\mathrm{all}$ experiences an outage via the function ChkOut described in Section \ref{sec3}, where the set $V_\mathrm{all}$ is given by the intersection points, $\mathbf{u}_0$, $\mathbf{u}_F$, and $\mathbf{c}_n$ for $n\in\mathcal{N}$.\footnote{To efficiently select the needed line segments for each local-graph, we save the line segments lying inside the set of coverage regions in different memories according to the types of its endpoints: $\mathbf{u}_0$, $\mathbf{u}_F$, $\mathbf{c}_n$ for $n\in\mathcal{N}$, or the others.} 

%Next, we find every optimal local-path from $\mathbf{l}_0$ to $\mathbf{l}_F$ without replacing the battery in the process similar to the generalized intersection method (in lines $17$-$28$). The feasibility of each local-path finding problem is checked through the function ChkFea described in Section \ref{sec3}. If the problem is feasible, then we construct an undirected local-graph $G_\mathrm{LO}=(V_\mathrm{LO}, E_\mathrm{LO})$ in a similar way as in Algorithm \ref{Algo1}, but to derive the local-edge set $E_\mathrm{LO}$, we just load the line segments tested whether they experience outages in advance. An example of the local-graph is represented in Fig. \ref{Fig7}.

%{After constructing the local-graph, an optimal local-path from $\mathbf{l}_0$ to $\mathbf{l}_F$ in the local-graph $G_\mathrm{LO}$ is derived. We first derive  an optimal sequence $\mathbf{S}_{V_\mathrm{LO}}(\mathbf{l}_0,\mathbf{l}_F)$ of visiting nodes in $V_\mathrm{LO}$ and the corresponding path length $\ell_\mathrm{LO}$ via the Dijkstra algorithm.} Since the distance that the UAV can travel without replacing its battery is limited as \eqref{eq:3}, we verify whether the UAV can travel $\ell_{LO}$ within the battery capacity $(h_\mathrm{sp}=1)$ or not $(h_\mathrm{sp}=0)$ and then derive {the maximum possible speed $v(\mathbf{l}_0,\mathbf{l}_F)\in\mathcal{V}$ for $\mathbf{S}_{V_\mathrm{LO}}(\mathbf{l}_0,\mathbf{l}_F)$ via the function ChkSp if $h_\mathrm{sp}=1$,} whose pesudo code is given in Algorithm \ref{Algo5} and it will be explained later.\footnote{We assume that the UAV flies with a fixed speed while traveling through a local-path, which is justified later in Theorem \ref{Thm3}.} 
%{The optimal local-path $\mathbf{p}_\mathrm{LO}(\mathbf{l}_0,\mathbf{l}_F)$ is also derived by using the sequence $\mathbf{S}_{V_\mathrm{LO}}$ and the maximum possible speed $v_\mathrm{max}$ at the function FindPath explained in Section \ref{sec3} if $h_\mathrm{sp}=1$.} 

%Finally, we derive an optimal global-trajectory from $\mathbf{u}_0$ to $\mathbf{u}_F$ considering the battery replacement at some CSs by using the induced local-paths (in lines $25$, $29$-$36$). A directed global-graph $\overrightarrow{G}_\mathrm{GL}=(V_\mathrm{GL}, E_\mathrm{GL})$ is constructed based on the induced optimal local-paths. Here, the vertex set $V_\mathrm{GL}=\{\mathbf{u}_0,\mathbf{u}_F,\mathbf{c}_1,...,\mathbf{u}_N\}$ consists of the end points in the local-paths and the edge set $E_\mathrm{GL}$ is given by every tuple {$(\mathbf{l}_0,\mathbf{l}_F,\ell_\mathrm{LO}/v(\mathbf{l}_0,\mathbf{l}_F)+\tau_{\mathbf{l}_F})$} satisfying $h_\mathrm{sp}=1$, {where the weight of each edge $\ell_\mathrm{LO}/v(\mathbf{l}_0,\mathbf{l}_F)+\tau_{\mathbf{l}_F}$ includes the travel time from $\mathbf{l}_0$ to $\mathbf{l}_F$ in the maximum possible speed $v(\mathbf{l}_0,\mathbf{l}_F)$} and the delay $\tau_{\mathbf{l}_F}$ to replace the battery at $\mathbf{l}_F$. We note that the delay for battery swapping at $\mathbf{u}_F$ is $0$. An example of the global-graph $\overrightarrow{G}_\mathrm{GL}$ is illustrated in Fig. \ref{Fig8}.

%After that, we check whether the vertices $\mathbf{u}_0$ and $\mathbf{u}_F$ are connected in the global-graph $\overrightarrow{G}_\mathrm{GL}$ $(h_\mathrm{Gfea}=1)$ or not $(h_\mathrm{Gfea}=0)$ via the BFS algorithm. 
%{If $h_\mathrm{Gfea}=1$, we find an optimal global-trajectory from $\mathbf{u}_0$ to $\mathbf{u}_F$. We search an optimal sequence $\mathbf{S}_{V_\mathrm{GL}}$ of visiting nodes in $V_\mathrm{GL}$ and its delivery time $T$ via the Dijkstra algorithm and then derive the optimal global-trajectory by using the sequence $\mathbf{S}_{V_\mathrm{GL}}$, the sequences $\mathbf{S}_{V_\mathrm{LO}}(\mathbf{l}_0,\mathbf{l}_F)$ and their speeds $v(\mathbf{l}_0,\mathbf{l}_F)$  for all possible local-paths, and the delay $\tau_{C_n}$ $(n\in\mathcal{N})$ to replace the battery for all CSs at the function FindPathG, which substitutes the node-visiting sequence into the global-trajectory in a similar way as in \cite[$(25)$-$(27)$]{Zhang:2019} but consider the local-paths and the delays for battery swapping.}


{Algorithm \ref{Algo5} describes the function ChkSp which checks whether the UAV can travel a distance $\ell_\mathrm{LO}\geq 0$ without replacing the battery $(h_\mathrm{sp}=1)$ or not $(h_\mathrm{sp}=0)$ by using the maximum possible traveling distance function $d_\mathrm{fly}(v)$ in \eqref{eq:3} for speed $v\in\mathcal{V}$. If it is possible $(h_\mathrm{sp}=1)$, then it derives the maximum possible speed $v_\mathrm{max}\in\mathcal{V}$ whose maximum traveling distance $d_\mathrm{fly}(v_\mathrm{max})$ is not smaller than $\ell_\mathrm{LO}$. We note that the algorithm assumes that the UAV flies with a fixed speed between two CSs, while the speed can vary depending on the pair of CSs. The following theorem shows a sufficient condition for flying with a fixed speed between two CSs to be optimal.}

%%%%%%% Algorithm 5: Checking Speed %%%%%%%%%
\begin{algorithm}
\caption{Function ChkSp} \label{Algo5}
\textbf{Input:} $\ell_\mathrm{LO}$, $\mathcal{V}$, {$w$, $w_2$}
\begin{algorithmic}[1]
\If{$\{v\in\mathcal{V}|d_\mathrm{fly}(v)\geq \ell_\mathrm{LO}\}\neq \emptyset$}
    \State $h_\mathrm{sp}\leftarrow 1$ \hfill\Comment{Can travel $\ell_\mathrm{LO}$ without battery swapping} 
    
    \LeftComment{{Find the maximum possible speed $v_\mathrm{max}$ that can travel the length $\ell_\mathrm{LO}$ without battery swapping.}} 
    \State $v_\mathrm{max}\leftarrow \max_{v\in\mathcal{V}}\{v|d_\mathrm{fly}(v)\geq \ell_\mathrm{LO}\}$
\Else
    \State $h_\mathrm{sp}\leftarrow 0$, $v_\mathrm{max}=0$
\EndIf
\end{algorithmic}
\textbf{Output:} {$(h_\mathrm{sp}$, $v_\mathrm{max})$}
\end{algorithm}
%%%%%%%%%%%%%%%%%%%%%%%%

\begin{theorem}\label{Thm3} % fixed UAV speed
{Assume that the UAV can fly with any speed $v\in[v_1,v_q]$ and the power consumption model $P_\mathrm{UAV}(v)$ is convex for $v\in[v_1,v_q]$. Then, for traveling between two CSs with the connectivity and battery constraints, flying with a fixed speed minimizes the traveling time.} 
%for any path length  \ell_\mathrm{LO}\geq 0$ between the two CSs.}
\end{theorem}
\begin{proof}
Let us assume that {the path distance $\ell_\mathrm{LO}$ to travel between two CSs} is partitioned by segments $\ell_1,...,\ell_K$ where $\ell_\mathrm{LO}=\sum_{k=1}^K\ell_k$ and the UAV flies with speed $\tilde{v}_k\in[v_1,v_q]$ for segment $\ell_k$ for $k\in[1:K]$. In this case, we have  the total travel time $T_\mathrm{LO}=\sum_{k=1}^K \ell_k/\tilde{v}_k$ and the total consumed energy $E_\mathrm{LO}=\sum_{k=1}^K (\ell_k/\tilde{v}_k)\cdot P_\mathrm{UAV}(\tilde{v}_k)$. We prove this theorem by showing the UAV can travel $\ell_\mathrm{LO}$ within time $T_\mathrm{LO}$ by a fixed speed $\bar{v}\in[v_1,v_q]$ while consuming less energy than $E_\mathrm{LO}$. First, the UAV can travel $\ell_\mathrm{LO}$ in time $T_\mathrm{LO}$ if it travels with the fixed speed $\bar{v}={\ell_\mathrm{LO}\over{\sum_{k'=1}^K \ell_{k'}/\tilde{v}_{k'}}}$. Second, $E_\mathrm{LO}$ is lower-bounded as:
\begin{align}
E_\mathrm{LO}&=\sum_{k=1}^K (\ell_k/\tilde{v}_k)\cdot P_\mathrm{UAV}(\tilde{v}_k) \label{eq:30}\\
\overset{(a)}\geq& \Biggl(\sum_{k'=1}^K \ell_{k'}/\tilde{v}_{k'}\!\Biggr)\!\cdot P_\mathrm{UAV}\left(\sum_{k=1}^K{{\ell_k/\tilde{v}_k}\over{\sum_{k'=1}^K \ell_{k'}/\tilde{v}_{k'}}}\cdot \tilde{v}_k\!\!\right) \label{eq:31}\\
=& T_\mathrm{LO} \cdot P_\mathrm{UAV}(\bar{v}), \label{eq:32}
\end{align}
where $(a)$ is by Jensen's inequality. Since the UAV with fixed speed $\bar{v}$ consumes less energy than $E_\mathrm{LO}$ as $\eqref{eq:32}$, this proves the theorem.  
\end{proof}
We note that the power consumption model in \eqref{eq:1} can be approximated as a convex function when $v\gg {v_0(w)}$ as proved in \cite{Zeng:2019}.\footnote{Such convexity of the power consumption model $P_\mathrm{UAV}(v)$ will be numerically shown in Section \ref{sec6}.} Hence, {in Algorithm \ref{Algo1}, traveling with a fixed speed between two CSs, while the speed can vary depending on the pair of CSs, is approximately optimal.} 

{Now, the following theorems show that our GIM-B algorithm outputs an optimal solution of Problem 1 NP-easily under the assumption that the power consumption model $P_\mathrm{UAV}(v)$ is  convex in the range of the UAV  speed.}\footnote{{We assume $|\mathcal{V}|=O(M)$ to make the complexity of selecting $v_\mathrm{max}$ in Algorithm \ref{Algo5} negligible.}}


\begin{theorem}\label{Thm4} %GIM-B optimality
The GIM-B algorithm outputs an optimal solution for Problem 1 if the power consumption model $P_\mathrm{UAV}(v)$ is convex in the range of the UAV speed.
\end{theorem}
\begin{proof}
This proof is immediate from Theorems \ref{Thm1} and \ref{Thm3} and the optimality of the Dijkstra algorithm because
\begin{enumerate}
    \item Theorem \ref{Thm1} means that every path between two CSs at the local level has the minimum travel distance.
    \item Theorem \ref{Thm3} implies that flying with the same speed in each path at the local level is optimal. Hence, the GIM-B algorithm derives the minimum travel time for the paths.
    \item Under the graph $\overrightarrow{G}_\mathrm{GL}$ with the minimized edge weights, an optimal trajectory from $\mathbf{u}_0$ to $\mathbf{u}_F$ at the global level is derived by applying the Dijkstra algorithm.
\end{enumerate}
\end{proof}

\begin{theorem}\label{Thm5} % GIM-B complexity
{If the number of CSs is smaller than or equal to the number of BSs, i.e.,} $N\leq M$, then the time complexity of the GIM-B algorithm is $O(M^6)$. 
\end{theorem}
\begin{proof}
Let us first state the cardinalities of the following sets: $|V_\mathrm{all}|=O(M^2)$, $|V_\mathrm{LO}|=O(M^2)$, and $|V_\mathrm{GL}|=O(N)$. The steps of Algorithm \ref{Algo4} have the following complexities:
\begin{itemize}
    \item Step 1. Outage test: For a line segment, performing the function ChkOut and selecting a memory to save the line segment among $E_\mathrm{in},E_1,...,E_{N+2}$ have the complexities $O(M^2)$ and $O(N)$, respectively. Since each line segment $\overline{\mathbf{x}_1\mathbf{x}_2}$ for $\mathbf{x}_1,\mathbf{x}_2\in V_\mathrm{all}$ and $\mathbf{x}_1\neq\mathbf{x}_2$ should be checked whether experiencing an outage, the complexity of this step is $(O(M^2)+O(N))\cdot |V_\mathrm{all}|^2=O(M^6)$.
    \item  Step 2. Local level search: The complexity of deriving an optimal path between a pair of CSs at the local level can be proved similarly with the proof of Theorem \ref{Thm2}. However, this algorithm constructs the edge set $E_\mathrm{LO}$ with only complexity $O(N)$ by just loading some of the saved memories $E_\mathrm{in},E_1,...,E_{N+2}$. Hence, the complexity of deriving an optimal path in the local level is $O(N)$+$O(M^4)=O(M^4)$. Since there are $O(N^2)$ pairs of the CSs, the complexity of the step is $O(M^4N^2)$.
    \item Step 3. Global level search: This complexity is dominated by applying the Dijkstra algorithm at the graph $\overrightarrow{G}_\mathrm{GL}$ with the complexity $O(|V_\mathrm{GL}|^2)=O(N^2)$ \cite{West:2001}. 
\end{itemize}
Consequently, the complexity of the GIM-B algorithm is $O(M^6)$ for $N\leq M$.
\end{proof}

%%%%% Table 2 %%%%%%%%%%%%%%%%
\begin{table*}
\centering
\begin{tabular}{@{} c || c | c @{}}
\cline{1-3}
Algorithm ($^*$modified considering the battery constraint) & Complexity & Performance gap \\ \cline{1-3}
Exhaustive search$^*$ \cite{Zhang:2019} & $O(M!M^{3.5}N^2)$ & 0\\ \cline{1-3}
Exhaustive search with fixed association$^*$ \cite{Zhang:2019} & $O(M^{3.5}N^2)$ & $O(MNd_0/v_q+N\tau_\mathrm{max})$ \\ \cline{1-3}
Exhaustive search with quantization$^*$ \cite{Zhang:2019} & {$O(M^4Q^2N^2)$} & $O(MNd_0/v_q+N\tau_\mathrm{max})$ \\ \cline{1-3}
%Exhaustive search with quantization$^*$ \cite{Zhang:2019} in same association &  & $O((MNd_0/v_q)\sin(1/{Q}))$ \\ \cline{1-3}
Intersection method$^*$ \cite{Chen:2020} {by checking outages via Algorithm \ref{Algo3}} & $O(M^4N^2)$ & $O(MNd_0/v_q+N\tau_\mathrm{max})$ \\ \cline{1-3}
Ours {(Generalized intersection method with battery constraint)}  & $O(M^6)$ & $0$ \\ \cline{1-3}
%LCI-B method without performing Algorithm \ref{Algo3} in advance & $O(M^6N^2)$ & $0$ \\ \cline{1-3}
\end{tabular}
\caption{Comparison of algorithms for Problem 1}\label{Tab2}
\end{table*}
%%%%%%%%%%%%%%%%%%%%%%%%%%%%

{We note that our GIM-B algorithm has the same complexity {order} as the generalized intersection method for $N\leq M$ despite considering the battery constraint. Note that the outage of every possible line segment is tested in advance in Step 1 of GIM-B algorithm. However, a direct extension from the GIM algorithm would be treating the pair of CSs as the initial and final points and applying a modified version of Algorithm~\ref{Algo1}, which implies performing the outage test in Step~2. The following corollary shows that such a direct extension of the GIM algorithm has a higher order of complexity. }

%because the complexity is reduced by checking whether the line segments experience outages in advance before the step 2 in Algorithm \ref{Algo4}. The following corollary justifies it through showing the complexity of the GIM-B algorithm without previously checking whether the line segments experience outages is higher than $O(M^6)$.

\begin{corollary}\label{Cor1} % GIM-B complexity
{If the outage test is separately performed in the derivation of  an optimal path between each pair of CSs,  the time complexity increases to  $O(M^6N^2)$.}
\end{corollary}
\begin{proof}
This method checks whether each line segment experiences an outage at the step 2 in Algorithm \ref{Algo4}. In this case, the complexity for deriving a path between two CSs at the local level through the step 2 is the same as the complexity $O(M^6)$ of the generalized intersection method. Hence, the complexity of this method is dominated at deriving $O(N^2)$ paths for every CS pair: $O(M^6)\cdot O(N^2)=O(M^6N^2)$.
\end{proof}



Table \ref{Tab2} compares the GIM-B algorithm with the benchmark algorithms for Problem 1.
{We note that the benchmark algorithms in Table \ref{Tab2} use the same name as Table \ref{Tab1} but they are modified by considering the battery constraint.} 
Specifically, we modify each benchmark algorithm similarly as Algorithm \ref{Algo4}: Step 1 is skipped since it is not applicable for the exhaustive search and its variants \cite{Zhang:2019} and it is not beneficial for the intersection method \cite{Chen:2020}, {Step 2 applies the corresponding benchmark algorithm with slight modification by treating the two CSs as the initial and the final points and checking whether traveling the resultant path is affordable with the battery capacity}, and Step 3 applies the Dijkstra algorithm to obtain the trajectory in the global level. 
{To compare with the results in Table \ref{Tab1}, we assume that the UAV flies with a constant speed of $v_q$ for the analysis of performance gap in Table \ref{Tab2}, i.e., assume $\mathcal{V}=\{0,v_q\}$.} {Also, to avoid the meaningless bound of infinite gap, it is assumed that a path from the initial point to the final point exists in Step 3 for each  algorithm, i.e., assume $h_\mathrm{Gfea}=1$}.  Main observations for Table \ref{Tab2} are summarized in the following:
\begin{itemize}
    \item The GIM-B algorithm {outputs an optimal solution of Problem 1 NP-easily.}
    %\item {The complexity order} of the GIM-B algorithm is the same with the generalized intersection method, while the complexities of the benchmark algorithms increase by $O(N^2)$ when considering the battery constraint and the CSs. Such complexity gain occurs since our GIM-B algorithm checks whether the line segments experience outages in advance. We note that the intersection method \cite{Chen:2020} cannot have the complexity gain by the process since it only considers a fixed BS association sequence at the local level search and hence there are few line segments needed to test whether experiencing outages.
    \item {The performance gaps of the sub-optimal algorithms increase in $N$ due to the accumulation of the gaps in finding the path between each pair of CSs. Also, note that they depend on the maximum delay $\tau_\mathrm{max}$ for battery swapping because the number of visiting CSs can increase for the sub-optimal algorithms.}
    \item Compared to Table \ref{Tab1}, {the performance gap of the ES-Q algorithm with the finite number $Q$ of quantization points does not decrease in $Q$, because some of the edges in the graph $\overrightarrow{G}_\mathrm{GL}$ over the CSs of the GIM-B algorithm may disappear if we apply ES-Q algorithm in Step 2 due to the battery constraint.} 
    %{We note that the ES-Q algorithm has the same performance gap tendency as in Table \ref{Tab1} only if the same association sequence for the BSs and the CSs in the GIM-B algorithm is possible in the ES-Q algorithm, where we call this special case of the ES-Q algorithm as the ES-Q algorithm with same association.} 
\end{itemize}
The aforementioned analysis implies that our intersection point-based algorithms have more advantages compared to the benchmark algorithms in the presence of the battery constraint and the CSs.





% Remark 3: H_CS < H
\begin{remark}\label{rmk3}
{When $H_\mathrm{CS}<H$, we can solve Problem 1 by including the take-off and the landing times at charging station $C_n$ in overall delay $\tau_{C_n}$ for $n\in\mathcal{N}$ and considering the consumed energy for them in the battery capacity model \eqref{eq:2}.} 
\end{remark}


% Remark 4: Benchmarking in general scenario
\begin{remark}\label{rmk4}
If we do not take any assumptions in Table \ref{Tab2}, then the performance gaps of the sub-optimal algorithms increase compared to this table. First, when we assume that the UAV flies with dynamic speed (i.e., $|\mathcal{V}|\geq 3$ where $0\in \mathcal{V}$), the maximum possible speed $v_\mathrm{max}$ in a path between two CSs at the local level decreases as its distance $\ell_\mathrm{LO}$ increases. Hence, it should be considered in the performance gaps that when the distance $\ell_\mathrm{LO}$ increases, its travel time additionally increases due to decreasing $v_\mathrm{max}$. Second, a path from the initial point to the final point may not exist in Step 3 for some sub-optimal algorithms (i.e., $h_\mathrm{Gfea}=0$) since some of the edges in the graph $\overrightarrow{G}_\mathrm{GL}$ over the CSs of the GIM-B algorithm may disappear if we apply the sub-optimal algorithms in Step 2 due to the battery constraint.
\end{remark}

%\section{Maximally Deliverable Payload Weight}\label{sec5}

\section{Numerical Results}\label{sec6}

% This script is for conference submission!!!!!

In this section, we provide various numerical results to evaluate the performance of our GIM-B algorithm. We assume that $M=19$ BSs and $N=5$ CSs are distributed in a $10\mathrm{km}\times 10\mathrm{km}$ region where $\mathbf{u}_0$ and $\mathbf{u}_F$ are located. The coverage radius of $\mathrm{BS}_m$ is set with $d_0=1400\mathrm{m}$ and $\lambda_m\in{[0,700]\mathrm{m}}$ for $m\in\mathcal{M}$. The overall delay to replace the battery at charging station $C_n$ is assumed to be  $\tau_{C_n}=100\mathrm{s}$ for $n\in\mathcal{N}$. {The speed set of the UAV is  $\mathcal{V}=[0:1:30]\mathrm{m/s}$.} The total weight of the UAV including its payload is given as $w=2.97\mathrm{kg}$, where $w_1=1.07\mathrm{kg}$, $w_2=0.9\mathrm{kg}$, and $w_3=1\mathrm{kg}$. The detailed parameters for the propulsion power consumption \eqref{eq:1} are set to $P_1=42.24\mathrm{J}$, $P_2(w)=445.19\mathrm{J}$, tip speed of the blade $v_\mathrm{tip}=14\mathrm{m/s}$, mean rotor induced speed for hovering $v_0(w)=13.9\mathrm{m/s}$, air density $\rho=1.225\mathrm{kg/m^3}$, and fuselage equivalent flat plate area $S_\mathrm{FP}=0.03\mathrm{m^2}$. Also, the battery model \eqref{eq:2}-\eqref{eq:3} is assumed to have  the parameters $\epsilon_\mathrm{batt}=540\mathrm{kJ/kg}$, $\gamma=0.7$, $\eta=0.7$, and $r_\mathrm{safe}=1.2$ {\cite{Zhang:2021_2}}.\footnote{We note that the values of $P_2(w)$ and $v_0(w)$ depend on $w$ as \cite{Zeng:2019}.}
Fig. \ref{Figs1} plots the propulsion power consumption $P_\mathrm{UAV}(v)$ according to the flying speed $v$ in different payload weights $w_3$, where we can numerically check that $P_\mathrm{UAV}(v)$ is a convex function for {$v\in[0,30]\mathrm{m/s}$} and hence the conditions in Theorems \ref{Thm3} and \ref{Thm4} hold. 

% Figure environment removed


%%%%%% 
Fig. \ref{Figs2} shows the UAV trajectory and the corresponding delivery time $T$ for our and benchmark algorithms.
% Figure environment removed %For the ES-Q algorithm\cite{Zhang:2019}, we tried  $Q=2$ and $Q=4$. 
{We note that the ES-FA algorithm \cite{Zhang:2019} and intersection method \cite{Chen:2020} find the same trajectory, but different from the optimal trajectory of the ES algorithm \cite{Zhang:2019} and our GIM-B algorithm. 
The ES-Q algorithm \cite{Zhang:2019} with $Q=2$ cannot find any trajectory from $\mathbf{u}_0$ to $\mathbf{u}_F$ due to the battery constraint. The ES-Q algorithm with $Q=4$ has a higher complexity than our GIM-B algorithm because $QN>M$. However, it has a significant lower travel time $T$ even than the ES-FA and the intersection method algorithms, since it cannot derive a path $\mathbf{u}_0$ to ${\mathbf{c}_2}$ with the quantization points that the UAV can travel without battery replacement.}
Fig. \ref{Figs3} shows the optimal graph at the global level and the corresponding maximum possible speed $v_\mathrm{max}$ for each edge under the environment in Fig. \ref{Figs2}. We can see that for each edge, the maximum travel speed $v_\mathrm{max}$ decreases as its travel distance increases.

% Figure environment removed

Fig. \ref{Figs4} compares the optimal UAV trajectory and the corresponding delivery time $T$ for the different payload weights $w_3$ and delays $\tau_{C_1}$ to replace the battery at charging station $C_1$, {where the locations of CSs $C_3$ and $C_4$ are changed from Fig. \ref{Figs2}.}
% Figure environment removed
We can see that the UAV avoids $C_1$ in high battery swapping delay $\tau_{C_1}=200\mathrm{s}$ (red) and visits more CSs in large payload weight $w_3=1.5\mathrm{kg}$ (blue). In Fig. \ref{Figs5}, the optimal delivery time $T$ is plotted for the different battery swapping delays and the payload weights {$w_3\in[0:0.1:3.5]\mathrm{kg}$} under the environment in Fig. \ref{Figs4}.
We can verify that $T$ increases as battery swapping delays and $w_3$ increase and the payload cannot be delivered from $\mathbf{u}_0$ to $\mathbf{u}_F$ if $w_3$ exceeds $2.8\mathrm{kg}$.

% Figure environment removed

\section{Conclusion}\label{sec7}
For the problem of path planning for a cellular-enabled UAV with connectivity and battery constraints, the generalized intersection method with battery constraint (GIM-B) algorithm was proposed that computes an optimal path in polynomial time. Its effectiveness in terms of computational complexity and resultant mission completion time was demonstrated by comparing with previously proposed algorithms both in analytically and numerically. 
For further works, it would be interesting to consider the scenario with time-varying delays at charging stations and  the scenario with more realistic coverage regions based on radio map.
%It was shown that the proposed GIM-B algorithm to attack Problem 1 yields an optimal UAV path under the battery constraint NP-easily. Moreover, the complexity order of the GIM-B algorithm is the same with the generalized intersection method for Problem 1-1 if $N\leq M$. We compared the GIM-B algorithm with previously proposed algorithms in \cite{Zhang:2019,Chen:2020} with slight modification to consider the battery constraint and showed that the GIM-B algorithm outperforms in both complexity and performance gap.

\bibliographystyle{IEEEtran}
\bibliography{ref}

\end{document}