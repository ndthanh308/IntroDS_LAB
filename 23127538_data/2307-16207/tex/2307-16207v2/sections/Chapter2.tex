\section{Problem Statement}\label{sec2}
We consider a cellular network with $M$ base stations (BSs) and $N\leq M$ charging stations (CSs). In this network, a UAV delivers a payload from an initial point $\mathbf{U}_0$ to a final point $\mathbf{U}_F$ under limited battery capacity. The detailed description of the UAV model is in Section \ref{sec2A}. The UAV should maintain the connectivity with one of the BSs while delivering the payload. The BS model and the BS-UAV connectivity is described in Section \ref{sec2B}. The UAV can replace its battery at a CS if needed, as explained in Section \ref{sec2C}. The goal of this paper is to characterize the minimum delivery time from $\mathbf{U}_0$ to $\mathbf{U}_F$, including the flight time in the air and the battery swapping time at CSs. This optimization problem is formally presented in Section \ref{sec2D}. The overall model is illustrated in Fig. \ref{Fig1}. %\shc{control station in fig.1? coverage of each BS? dashed lines between BSs?}

% Figure environment removed

\subsection{UAV Model}\label{sec2A}
%The mission completion time of the UAV is denoted by $T$.
In the cellular network, a rotary-wing UAV has a mission of delivering a payload from an initial point $\mathbf{U}_0$ to a final point $\mathbf{U}_F$. We assume that the UAV flies with a fixed altitude $H\in[H_\mathrm{min},H_\mathrm{max}]$, where $H_\mathrm{min}$ is determined by the heights of obstacles in the network and $H_\mathrm{max}$ corresponds to the maximum allowable altitude according to government regulations. Let us denote the 3D coordinates of $\mathbf{U}_0$, $\mathbf{U}_F$, and the UAV location at time $t$ by $(x_0,y_0,H)$, $(x_F,y_F,H)$, and $(x(t),y(t),H)$, respectively. 
We also denote $\mathbf{u}_0=(x_0,y_0)$, $\mathbf{u}_F=(x_F,y_F)$, and $\mathbf{u}(t)=(x(t),y(t))$ as the horizontally projected locations of the 3D coordinates. The UAV flies with time-varying speed of $v(t)\triangleq\|\nabla_t\mathbf{u}(t)\|$ at time $t$, where the speed is selected from the finite set $\mathcal{V}=\{0,v_1,...,v_q\}$ with $0<v_1<...<v_q$. 

% when flying in the air since we consider a downlink communication from a BS to the UAV. 
For energy consumption, we only consider the propulsion energy consumption by the UAV, since the communication energy consumption is relatively negligible \cite{Mozaffari:2019}. Let the total weight of UAV and its payload be given as $w=w_1+w_2+w_3$, where $w_1, w_2$, and $w_3$ denote the weights of the UAV body, its battery, and the payload, respectively.
The propulsion power consumption (in Watts) when flying with speed $v$ is given as
\begin{align}
\begin{split}\label{eq:1}
&P_\mathrm{UAV}(v)=P_1\left(1+{{3v^2}/{v_\mathrm{tip}^2}}\right)+P_2(w)\\
&\ \cdot\big(\sqrt{1+{{v^4}/{4v_0(w)^4}}}-{{v^2}/{2v_0(w)^2}}\big)^{0.5} \!\!\! +0.5{\rho} S_\mathrm{FP}v^3,
\end{split}
\end{align}
where $P_1$, $P_2(w)$, $v_\mathrm{tip}$, $v_0(w)$, $\rho$, and $S_\mathrm{FP}$ are the parameters determined by the environment of the network and the physical structure of the UAV \cite{Zeng:2019}. We note that only the parameters $P_2(w)$ and $v_0(w)$ depend on the total weight $w$. The power consumption model \eqref{eq:1} and its parameters will be revisited with details in Section \ref{sec6}. 

The UAV consumes the energy in its battery. The battery capacity (in Joules) is expressed as follows \cite{Zhang:2021_2}:
\begin{align}
C_\mathrm{batt}=\epsilon_\mathrm{batt}w_2,\label{eq:2}
\end{align}
where $\epsilon_\mathrm{batt}$ is the maximum energy of the battery per unit weight. Note that the battery capacity is directly proportional to the battery weight. As proved in \cite{Zhang:2021_2}, when the UAV flies at the same speed $v$ without replacing its battery, then the maximum distance (in meters) that it can travel is given as 
\begin{align}
d_\mathrm{fly}(v)=v\cdot{{\gamma\eta C_\mathrm{batt}}\over{r_\mathrm{safe}P_\mathrm{UAV}(v)}},\label{eq:3}
\end{align}
where $0<\gamma<1$ is the maximum depth of discharge of the battery, $0<\eta<1$ is the power transfer efficiency from the battery to the UAV body, and $r_\mathrm{safe}>1$ is the safety factor to reserve energy in the battery for unexpected situations. 
{We note that the maximum distance in \eqref{eq:3} is  because the maximum usable   energy from the fully charged battery is ${\gamma C_\mathrm{batt}}\over r_\mathrm{safe}$ (in Joules) and the UAV consumes the energy in the battery at a rate of  ${P_\mathrm{UAV}(v)}\over{\eta}$ (in Watts).}
%\hs{We note that the consumed power in the battery is ${r_\mathrm{safe}P_\mathrm{UAV}}\over{\gamma\eta}$, where the smaller $\gamma$ and $\eta$ and the larger $r_\mathrm{safe}$ reduce the energy consumption efficiency of the UAV.}
%We note that \sh{the smaller $\gamma$ and $\eta$ and the larger $r_\mathrm{safe}$ reduce the energy consumption efficiency of the UAV.}

\subsection{BS-UAV Connectivity}\label{sec2B}
There are $M$ BSs in the cellular network. The $m$th BS where $m\in\mathcal{M}\triangleq[1:M]$, $\mathrm{BS}_m$ is located at $(a_{m1},a_{m2},H_\mathrm{BS})$, where all BSs are assumed to be located at the same altitude $H_\mathrm{BS}<H$. We further denote $\mathbf{a}_m=(a_{m1},a_{m2})$ as the horizontally projected location of $\mathrm{BS}_m$. Each BS has a single omni-directional antenna and the same transmission power $P_\mathrm{tx}$. All the BSs are connected to a control station through a backhaul network to successfully hand over from a BS to another BS and control the UAV trajectory. 

We assume that the channel between the UAV and a BS is determined by the line-of-sight (LoS) probabilistic model, where the LoS probability increases as the elevation angle between the UAV and the BS increases \cite{Al-Hourani:2014}. 
The expected path loss between the UAV and $\mathrm{BS}_m$ at time $t$, $\Lambda_m(t)$ (in $\mathrm{dB}$) is given as $\Lambda_m(t) = \mathrm{FSPL}_m(t)+p_m(t)\cdot\zeta_1 + (1-p_m(t))\cdot\zeta_2$, 
%The expected path loss between the UAV and $\mathrm{BS}_m$ at time $t$, $\Lambda_m(t)$ (in $\mathrm{dB}$) is given as follows \cite{Al-Hourani:2014_2}:
%\begin{align}
%\Lambda_m(t) = \mathrm{FSPL}_m(t)+p_m(t)\cdot\zeta_1 + (1-p_m(t))\cdot\zeta_2,\label{eq:4}
%\end{align}
where $\mathrm{FSPL}_m(t)$ and $p_m(t)\in[0,1]$ are the free space path loss and the LoS
probability between the UAV and $\mathrm{BS}_m$ at time $t$, respectively, which only depend on the distance between the UAV and $\mathrm{BS}_m$, and $\zeta_1>0$ and $\zeta_2>\zeta_1$ refer to the excessive path losses for LoS and non-LoS (NLoS) links, respectively \cite{Al-Hourani:2014_2}.\footnote{Our path loss model is based on large-scale fading, i.e., small-scale fading effects are ignored. However, we can check that our results also hold under the small-scale fading by averaging the randomness.} The received signal to interference plus noise ratio (SINR) from $\mathrm{BS}_m$ to the UAV at time $t$ is $\mathrm{SINR}_m(t)={{P_\mathrm{tx}\cdot 10^{\Lambda_m(t)/10}}\over {\sum_{m'\in\mathcal{M}\setminus m}I_{m'm}(t)+N_0}}$, 
%\begin{align}
%\mathrm{SINR}_m(t)={{P_\mathrm{tx}\cdot 10^{\Lambda_m(t)/10}}\over {\sum_{m'\in\mathcal{M}\setminus m}I_{m'm}(t)+N_0}},\label{eq:5}
%\end{align}
where $I_{m'm}(t)$ is the interference power by $\mathrm{BS}_{m'}$ at time $t$ when the UAV is communicating with $\mathrm{BS}_{m}$ and $N_0$ is the additive noise power. Note that $I_{m'm}$ would be equal to zero if $\mathrm{BS}_{m'}$ uses a different frequency band from $\mathrm{BS}_{m}$, 
and even if the two BSs use the same frequency band, it will become negligible if $\mathrm{BS}_{m'}$ is far away from the UAV.

To maintain the control of the UAV, the communication rate from a BS to the UAV should not be less than the minimum required data rate, i.e., the maximally achievable SINR of the UAV should satisfy 
\begin{align}
\max_{m\in\mathcal{M}} \mathrm{SINR}_m(t)\geq \mathrm{SINR}_\mathrm{th}\label{eq:6}
\end{align}
for any time $t$ where $\mathrm{SINR}_\mathrm{th}$ is the hard SINR threshold to achieve the minimum required data rate. In weak interference regime, i.e., the frequency reuse factor is sufficiently low, it can be easily checked that the condition \eqref{eq:6} can be equivalently written as  $\min_{m\in\mathcal{M}}\|\mathbf{u}(t)-\mathbf{a}_m\|\leq d_0$ for some $d_0$, where we call $d_0$ the base coverage radius of each BS.\footnote{Each BS has the same base coverage radius $d_0$ since every BS has the same transmission power $P_\mathrm{tx}$ and the same altitude $H_\mathrm{BS}$, but it can be verified that our results also hold under different base coverage radii due to different transmission powers or BS altitudes.} For other cases, however, it is in general hard to represent the exact coverage region satisfying  \eqref{eq:6} in a simple form. For tractable analysis, we introduce the coverage offset $\lambda_m\in [0,d_0]$ for $\mathrm{BS}_m$ and assume that the UAV can connect with $\mathrm{BS}_m$ with high probability if the UAV is in the effective coverage region of $\mathrm{BS}_m$ given as $\|\mathbf{u}(t)-\mathbf{a}_m\|\leq d_0-\lambda_m$. In other words, by introducing offsets $\lambda_m$ taking into account the effect of interference, we assume that \eqref{eq:6} holds with high probability if the following equation holds\footnote{In Section \ref{sec2B}, we only state the connectivity for  downlink communications from a BS to the UAV, but we can set a similar coverage region as \eqref{eq:6} and \eqref{eq:7} for uplink communications.}:
\begin{align}
\min_{m\in\mathcal{M}}\|\mathbf{u}(t)-\mathbf{a}_m\|+\lambda_m\leq d_0.\label{eq:7}
\end{align}
%Some examples of the effective coverage regions in the cellular network are illustrated in Fig. \ref{Fig2}. 

\begin{comment}
% Figure environment removed
\end{comment}

\begin{comment}
\begin{remark}\label{Rmk1}
\ky{The possible coverage offset $\lambda_m$ for $m\in\mathcal{M}$ depends on the environment around $\mathrm{BS}_m$. For example, large (small) $\lambda_m$ is appropriate for urban (suburban) environments since there are many (few) other BSs around $\mathrm{BS}_m$. A higher flight altitude of the UAV induces a larger coverage offset $\lambda_m$ since the LOS probability of the channel between the UAV and other BSs increases  \cite{Lin:2019}. Also, $\lambda_m$ increases in the traffic of other BSs around $\mathrm{BS}_m$ \cite{Zhang:2021}.}
\end{remark}
\end{comment}

\subsection{Charging Station Model}\label{sec2C}
To deliver the payload over a long distance with limited battery capacity, the UAV may replace its battery by visiting one of $N\leq M$ CSs. The $n$th charging station $C_n$ where $n\in\mathcal{N}\triangleq[1:N]$ is assumed to be located at $(c_{n1},c_{n2},H_\mathrm{CS})$, where all CSs are assumed to be located at the same altitude $H_\mathrm{CS}\leq H$. We further denote $\mathbf{c}_n=(c_{n1},c_{n2})$ as the horizontally projected location of $C_n$. To reduce the delay to replace the battery, each CS uses an autonomous battery swapping system \cite{Lee:2015}.\footnote{The autonomous battery swapping system in \cite{Lee:2015} takes about $60$ seconds for the entire battery swapping process.} The overall delay to replace the battery at charging station $C_n$, $\tau_{C_n}\in[0,\tau_\mathrm{max}]$ consists of the waiting time and the battery swapping time, where $\tau_\mathrm{max}$ is the upper bound on the delay. We note that the waiting time at charging station $C_n$ varies depending on the congestion of the CS and hence $\tau_{C_n}$ depends on $n$.

% Remark 2: Penalty by battery swapping itself
%\begin{remark}\label{rmk2}
%\hs{When the UAV replaces the battery in a charging station $C_n$ for $n\in\mathcal{N}$, there may be a loss due to battery swapping itself, as well as the overall delay $\tau_{C_n}$ to replace it. For example, the UAV may pay a fee to each visited CS. We can consider the loss by adding it to each $\tau_{C_n}$.}
%\end{remark}

\subsection{Goal}\label{sec2D}
The goal of this paper is to characterize the minimum delivery time $T$ from $\mathbf{U}_0$ to $\mathbf{U}_F$ of the UAV, including the flight time in the air and the overall delay to replace its battery at CSs. The optimization problem is formulated as
\begin{align} 
&\textbf{Problem 1} \cr
&\mbox{Objective:~}~~~~ \min_{T\geq 0,\{\mathbf{u}(t),\ \psi(t),\ t\in[0,T]\}} T\label{eq:8}\\
&\mbox{Constraints: }\cr%\label{eq:8.1}
&\mathbf{u}(0)=\mathbf{u}_0,\ \mathbf{u}(T)=\mathbf{u}_F,\ E_\mathrm{batt}(0)=C_\mathrm{batt} \label{eq:9}\\  
&\mathbf{u}(t)\in\mathbb{R}^2,\ \psi(t)\in[0:N] \label{eq:9.1}\\
&v(t)\triangleq\|\nabla_t\mathbf{u}(t)\|\in \mathcal{V},\ t\in[0,T]\label{eq:10}\\
&\min_{m\in\mathcal{M}}\|\mathbf{u}(t)-\mathbf{a}_m\|+\lambda_m\leq d_0,\ t\in[0,T]\label{eq:11}\\
&\psi(t)=0\ \mathrm{if~} \mathbf{u}(t)\not\in\{\mathbf{c}_n|n\in\mathcal{N}\},\ t\in[0,T]\label{eq:12}\\
&\psi(t)\in\{0,n\} \ \mathrm{if~}  \mathbf{u}(t)\in\{\mathbf{c}_n|n\in\mathcal{N}\},\ t\in[0,T]\label{eq:13}\\
&E_\mathrm{batt}(t)\geq (1-(\gamma/ r_\mathrm{safe}))C_\mathrm{batt},\ t\in[0,T]\label{eq:14}\\
&\nabla_t  E_\mathrm{batt}(t)=\! -P_\mathrm{UAV}(v(t))/\eta,\ \psi(t)=0,\ t\in[0,T]\label{eq:15}\\
&\nabla_t E_\mathrm{batt}(t)=0\ \mathrm{if~} \psi(t)\in \mathcal{N},\ t\in[0,T]\label{eq:16}\\
&E_\mathrm{batt}(t)=C_\mathrm{batt}\ \mathrm{if~} \psi(t)\in\mathcal{N} \text{ and }\cr
&~~t-\max_{t'}\{t'|\psi(t')=0,t'\in [0,t]\}=\tau_{C_{\psi(t)}},\ t\in[0,T]\label{eq:17}
\end{align}
where $\psi(t)\in[0:N]$ is an auxiliary variable indicating whether the UAV is in charging station $C_n$ $(\psi(t)=n)$ or in the air $(\psi(t)=0)$ at time $t$ and $E_\mathrm{batt}(t)\geq 0$ is the residual energy in the battery at time $t$.
%and $b(t)\in\{0,1\}$ \sh{becomes 1 for time instant $t$ when replacing the battery just finished.} Note that $b(t)$ depends on $\psi(t)$ as follows:
\begin{comment}
\begin{align} 
b(t)=
\begin{cases}
1 \text{ if } \psi(t)\in\mathcal{N} \text{ and } t-\\
\quad \max_{t'}\{t'|\psi(t')=0,t'\in [0,t]\}=\tau_{C_{\psi(t)}},
\\
0 \text{ otherwise}.\label{eq:18}
\end{cases}
\end{align}
\end{comment}
Here, \eqref{eq:9} means that the UAV departs from $\mathbf{u}_0$ with fully charged battery and arrives at $\mathbf{u}_F$ at time $T$, \eqref{eq:9.1} corresponds to the range of optimizing variables, \eqref{eq:10} denotes that the UAV can fly with a speed in the set $\mathcal{V}$, \eqref{eq:11} is the connectivity constraint in \eqref{eq:7}, and \eqref{eq:12}-\eqref{eq:13} determines whether the UAV is in a CS or in the air. Next, \eqref{eq:14} is the constraint for the maximum depth of discharge of the battery, \eqref{eq:15} and \eqref{eq:16} represent the power consumption when flying in the air and staying at a CS, respectively, and \eqref{eq:17} means that the battery has the maximum energy when the battery swapping process just finished.

Note that Problem 1 is not a convex optimization problem since the variable $\psi(t)$ is selected from a discrete set and constraint \eqref{eq:11} is not convex. Moreover, $\mathbf{u}(t)$ should be optimized in continuous $t\in[0:T]$. Such difficulties make Problem 1 non-trivial. To solve this problem, in Sections \ref{sec3} and \ref{sec4}, we first reformulate Problem 1 in a framework of weighted graph and then show that the problem can be solved NP-easily by graph theory-based algorithms.