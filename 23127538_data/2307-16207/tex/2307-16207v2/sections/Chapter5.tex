\section{Maximum Deliverable Payload Weight}\label{sec5}

In this section, we characterize the maximum weight of the payload that can be delivered from the initial point $\mathbf{U}_0$ to the final point  $\mathbf{U}_F$ under the connectivity and the battery constraints. 
By focusing on the maximum deliverable payload weight, not the minimum  delivery time,  we can formulate the optimization problem as
\begin{align} 
&\textbf{Problem 2} \cr
&\mbox{Objective:~}~~~~ \max_{w_3\geq 0, \{\mathbf{u}(t),\ \psi(t),\ t\in[0,T]\}} w_3 \label{eq:33}\\
&\mbox{Constraints: }\cr
&0\leq T<\infty \label{eq:34} \\
&\text{\eqref{eq:9}-\eqref{eq:17}}, \label{eq:35}
\end{align}
where \eqref{eq:34} means that the UAV succeeds to deliver the payload from $\mathbf{u}_0$ to $\mathbf{u}_F$ within a finite time. We note that the propulsion power consumption $P_\mathrm{UAV}(v(t))$ in \eqref{eq:15} depends on the payload weight $w_3$. 

To solve Problem 2, we propose the bottleneck edge search method described in Algorithm \ref{Algo6}.  %which first finds the longest connectivity-critical path between two CSs  and then derives the largest payload weight $w_3$ deliverable over the segment without replacing the battery. } 
%%%%%%% Algorithm 6: Bottleneck edge search method %%%%%%%%%
\begin{algorithm}[t]
\caption{Bottleneck Edge Search Method} \label{Algo6}
\textbf{Input:} $\mathbf{u}_0$, $\mathbf{u}_F$, $\mathcal{V}$, $\mathbf{c}_n$, $\ell_\mathrm{LO}(\mathbf{c}_n,\mathbf{c}_{n'})$, $h_\mathrm{Lfea}(\mathbf{c}_n,\mathbf{c}_{n'})$, $w_1$, $w_2$, $\epsilon_w$, $k_\mathrm{max}$ for $n\in [1:N+1]$, $n'\in[1:N]\cup \{N+2\}$, $n<n'$
\begin{algorithmic}[1]
\State $V_\mathrm{GL}\leftarrow\{\mathbf{u}_0,\mathbf{u}_F,\mathbf{c}_1,...,\mathbf{c}_N\}$, $E'_\mathrm{GL}\leftarrow \emptyset$, $w_3\leftarrow 0$
\State $\mathbf{c}_{N+1} \leftarrow \mathbf{u}_0$, $\mathbf{c}_{N+2} \leftarrow \mathbf{u}_F$, $h_\mathrm{sp}\leftarrow 1$
%\State $\epsilon_w\leftarrow 0.01$  \hfill\Comment{Sufficiently small positive constant}
\LeftComment{Step 1. Graph construction: Construct a graph $G'_\mathrm{GL}$ whose vertex set consists of CSs and edge set consists of edges between two connected CSs. The weight of each edge is the minimum travel distance between two CSs.}
\For {$n\in [1:N+1]$, $n'\in[1:N]\cup \{N+2\}$, $n<n'$}
    \LeftComment{Parameters $h_\mathrm{Lfea}$ and $\ell_\mathrm{LO}$ are described in Algorithm \ref{Algo4}.}
    \If{$h_\mathrm{Lfea}(\mathbf{c}_n,\mathbf{c}_{n'})=1$}
    \State $E'_\mathrm{GL}\leftarrow E'_\mathrm{GL}\cup (\mathbf{c}_n,\mathbf{c}_{n'},\ell_\mathrm{LO}(\mathbf{c}_n, \mathbf{c}_{n'}))$
    \EndIf
\EndFor
\State $G'_\mathrm{GL}\leftarrow (V_\mathrm{GL}, E'_\mathrm{GL})$ 
\LeftComment{Step 2. Bottleneck edge search: Find the longest connectivity-critical edge in $G'_\mathrm{GL}$.}
\LeftComment{Function \textbf{BFS} is described in line $1$ at Algorithm \ref{Algo4}.}
\State $h'_\mathrm{Gfea}\leftarrow$ \textbf{BFS}$(\mathbf{u}_0,\mathbf{u}_F,G'_\mathrm{GL})$
\If{$h'_\mathrm{Gfea}=1$}
\While{$h'_\mathrm{Gfea}=1$} 
    \State $(\mathbf{c}_k,\mathbf{c}_{k'},\ell_\mathrm{bott}) \leftarrow \!\!\!\!\!\!\!\!\!\!\!\!\!\!\!\!\!\!
    \underset{{(\mathbf{c}_n,\mathbf{c}_{n'},\ell_\mathrm{LO}(\mathbf{c}_n, \mathbf{c}_{n'}))\in E'_\mathrm{GL}}}{\mathrm{argmax}}$\!\!\!\!\!\!\!\!\!\!\!\!\!\!\!\!\! $\ell_\mathrm{LO}(\mathbf{c}_n, \mathbf{c}_{n'})$
    \LeftComment{Eliminate the longest edge in $E'_\mathrm{GL}$.}
    \State $E'_\mathrm{GL}\leftarrow E'_\mathrm{GL}\setminus (\mathbf{c}_k,\mathbf{c}_{k'},\ell_\mathrm{bott})$
    \State $G'_\mathrm{GL}\leftarrow (V_\mathrm{GL}, E'_\mathrm{GL})$
    \State $h'_\mathrm{Gfea}\leftarrow$ \textbf{BFS}$(\mathbf{u}_0,\mathbf{u}_F,G'_\mathrm{GL})$
\EndWhile
\Else
    \State $\ell_\mathrm{bott}\leftarrow \infty$
\EndIf
\LeftComment{Step 3. Weight search: Derive the maximum deliverable payload weight $w_3$.}
\LeftComment{Parameter $\epsilon_w>0$ is a sufficiently small constant.}
\While{$h_\mathrm{sp}=1$, $w_3\leq k_\mathrm{max}\epsilon_w$} \hfill%\Comment{\hs{$k_\mathrm{max}\in\mathbb{N}$}}
    \State $w_3\leftarrow w_3+\epsilon_w$
    \LeftComment{Function \textbf{ChkSp} is described in Algorithm \ref{Algo5}.}
    \State $(h_\mathrm{sp},v)\leftarrow$ \textbf{ChkSp}$(\ell_\mathrm{bott},\mathcal{V},w_1+w_2+w_3,w_2)$
\EndWhile
\State $w_3\leftarrow w_3-\epsilon_w$
\end{algorithmic}
\textbf{Output:} $w_3$
\end{algorithm}
%%%%%%%%%%%%%%%%%%%%%%%%
This algorithm initially sets zero payload weight, i.e., $w_3=0$. It first constructs an undirected weighted graph $G'_\mathrm{GL}$ whose vertex set consists of CSs (by
treating the initial and the final points also as CSs) and whose edge set includes an edge between two CSs only when there exists a path between the two CSs $(h_\mathrm{Lfea}=1)$, with the weight of the travel distance $\ell_\mathrm{LO}$ (in lines $3$-$8$). Note that the parameters $h_\mathrm{Lfea}$ and $\ell_\mathrm{LO}$ for each pair of CSs can be obtained by Algorithm \ref{Algo4}. After constructing the graph, it finds the bottleneck edge, which is the longest connectivity-critical edge in the graph (in lines $9$-$19$). To this end, the algorithm first checks whether $\mathbf{u}_0$ and $\mathbf{u}_F$ are connected by applying the function BFS \cite{West:2001}. If connected, it repeatedly eliminates the longest edge from the graph and then checks whether they are connected in the graph until not connected ($h'_\mathrm{Gfea}=0$). After the repetition ends, the most recently deleted edge is set as the bottleneck edge, with the edge weight $\ell_\mathrm{bott}$. Finally, the maximum deliverable payload weight over the graph is derived (in lines $20$-$24$). It first checks whether the UAV can travel the distance $\ell_\mathrm{bott}$ without replacing the battery ($h_\mathrm{sp}=1$) or not ($h_\mathrm{sp}=0$) at $w_3=0$ via the function ChkSp whose pseudo code is in algorithm \ref{Algo5}. Then, it iterates this process while increasing $w_3$ in sufficiently small increments $\epsilon_w>0$ until the UAV cannot deliver the payload over the bottleneck edge or $w_3$ exceeds the limit $k_\mathrm{max}\epsilon_w$ of the payload weight. This algorithm outputs the maximum payload weight $w_3\in\{0,\epsilon_w,...,k_\mathrm{max}\epsilon_w\}$ which can be delivered over the bottleneck edge.

Now, the following theorem shows that our bottleneck edge search method yields the optimal solution of Problem 2.\footnote{It can be shown that this method solves  Problem~2 NP-easily.}

\begin{theorem}\label{Thm6} % Optimality of bottleneck edge search method
Assume that the payload weight $w_3$ is selected from the set $\{0,\epsilon_w,...,k_\mathrm{max}\epsilon_w\}$ for $\epsilon_w>0$ and $k_\mathrm{max}\in\mathbb{N}$. Then, the bottleneck edge search method outputs the optimal solution for Problem 2 if the power consumption model $P_\mathrm{UAV}(v)$ is convex in the range of the UAV speed.  
\end{theorem}
\begin{proof}
%%%%%%% Exclusions for page limit %%%%%%
%Let us first show that the bottleneck edge  is the longest critical edge for the connection between $\mathbf{u}_0$ and $\mathbf{u}_F$ in the graph $G'_\mathrm{GL}$. It is trivially shown since $\mathbf{u}_0$ and $\mathbf{u}_F$ are still connected without the edges longer than the bottleneck edge and not connected if only the edges shorter than the bottleneck edge are included in the edge set $E'_\mathrm{GL}$ of the graph. 
If the UAV can travel the bottleneck edge without replacing the battery, then the payload can be delivered from $\mathbf{u}_0$ to $\mathbf{u}_F$ since it can be also delivered over an edge shorter than the bottleneck edge under the battery constraint. Hence, it is sufficient only to consider whether the payload can be delivered over the bottleneck edge. For finding the maximum deliverable payload weight, we note that it is sufficient only to consider a fixed speed while traveling the bottleneck edge, as justified in Theorem \ref{Thm3}. Consequently, our method yields the optimal solution for Problem 2.
\end{proof}