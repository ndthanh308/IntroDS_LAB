\section{Numerical Results}\label{sec6}

In this section, we provide various numerical results to evaluate the performance of our GIM-B algorithm. We assume that $M=19$ BSs and $N=5$ CSs are distributed in a $10\mathrm{km}\times 10\mathrm{km}$ region wherein $\mathbf{u}_0$ and $\mathbf{u}_F$ are also located. The coverage radius of $\mathrm{BS}_m$ is set with $d_0=1400\mathrm{m}$ and $\lambda_m\in[0,700]\mathrm{m}$ for $m\in\mathcal{M}$. The overall delay to replace the battery at charging station $C_n$ is assumed to be  $\tau_{C_n}=100\mathrm{s}$ for $n\in\mathcal{N}$. The speed set of the UAV is  $\mathcal{V}=[0:1:30]\mathrm{m/s}$. The total weight of the UAV including its payload is given as $w=2.97\mathrm{kg}$, where $w_1=1.07\mathrm{kg}$, $w_2=0.9\mathrm{kg}$, and $w_3=1\mathrm{kg}$. In the propulsion power consumption model \eqref{eq:1}, $P_1$, $P_2(w)$, and the mean rotor induced speed for hovering $v_0(w)$  are given as the following:
\begin{align}
P_1&={(\delta_p\rho/ 8)} (N_rN_bL_cR_r) v_\mathrm{tip}^3,\label{eq:s1}\\
P_2(w)&=(1+k_\mathrm{cf}){(wg)^{3/2}/\sqrt{2\rho N_r\pi R_r^2}},\label{eq:s2}\\
v_0(w)&=\sqrt{{wg}/({2\rho N_r\pi R_r^2})},\label{eq:s3}
\end{align}
where the parameters in \eqref{eq:s1}-\eqref{eq:s3} are described in Table \ref{Tab3}. For simulations, our choice of parameter values for the power consumption model \eqref{eq:s1}-\eqref{eq:s3} and for the battery model  \eqref{eq:2}-\eqref{eq:3} are summarized in Tables  \ref{Tab3} and \ref{Tab4}, respectively.\footnote{For the parameters for simulations, we referred to \cite{Zhang:2021_2,Zeng:2019}.} 
%%%%% Table3%%%%%%%%%%%%%%%%
\begin{table}[t]
\centering
\begin{tabular}{@{} c || c | c @{}}
\cline{1-3}
Notation & Parameter & Simulation\\ \cline{1-3}
$\delta_p$ & Profile drag coefficient & $0.012$\\ \cline{1-3}
$N_r$ & Number of rotors (quadcopter) & $4$\\ \cline{1-3}
$N_b$ & Number of blades per rotor & $4$\\ \cline{1-3}
$L_c$ & Blade chord length & $0.0157\mathrm{m}$\\ \cline{1-3}
$R_r$ & Rotor radius & $0.07\mathrm{m}$\\ \cline{1-3}
$v_\mathrm{tip}$ & Tip speed of a blade & $14\mathrm{m/s}$\\ \cline{1-3}
$k_\mathrm{cf}$ & Incremental correlation factor & $0.1$\\ \cline{1-3}
$S_\mathrm{FP}$ & Fuselage equivalent flat area & $0.03\mathrm{m^2}$\\ \cline{1-3}
$\rho$ & Air density & $1.225\mathrm{kg/m^3}$\\ \cline{1-3}
$g$ & Gravitational acceleration & $9.807\mathrm{m/s^2}$\\ \cline{1-3}
\end{tabular}
\caption{Parameters for power consumption model}\label{Tab3}
\end{table}
%%%%%%%%%%%%%%%%%%%%%
%%%%% Table4%%%%%%%%%%%%%%%%
\begin{table}[t]
\centering
\begin{tabular}{@{} c || c | c @{}}
\cline{1-3}
Notation &  Parameter & Simulation\\ \cline{1-3}
$\epsilon_\mathrm{batt}$ & Maximum energy of battery per kg & $540\mathrm{kJ/kg}$\\ \cline{1-3}
$\gamma$ & Maximum depth of discharge & $0.7$\\ \cline{1-3}
$\eta$ & Power transfer efficiency & $0.7$\\ \cline{1-3}
$r_\mathrm{safe}$ & Safety factor & $1.2$\\ \cline{1-3}
\end{tabular}
\caption{Parameters for battery model}\label{Tab4}
\end{table}
%%%%%%%%%%%%%%%%%%%%%
Fig. \ref{Figs1} plots the propulsion power consumption $P_\mathrm{UAV}(v)$ according to the flying speed $v$ in different payload weights $w_3$, where we can numerically check that $P_\mathrm{UAV}(v)$ is a convex function for $v\in[0,30]\mathrm{m/s}$ and hence the conditions in Theorems \ref{Thm3} and \ref{Thm4} hold. 


%%%%% Figs1%%%%%%%%%%%%%%%%
% Figure environment removed
%%%%%%%%%%%%%%%%%%%%%

Fig. \ref{Figs2} shows the UAV trajectory and the corresponding delivery time $T$ for our and benchmark algorithms.
%%%%% Figs2%%%%%%%%%%%%%%%%
% Figure environment removed 
%%%%%%%%%%%%%%%%%%%%%
We note that the ES-FA algorithm \cite{Zhang:2019} and intersection method \cite{Chen:2020} find the same trajectory, but different from the optimal trajectory of the ES algorithm \cite{Zhang:2019} and our GIM-B algorithm. 
The ES-Q algorithm \cite{Zhang:2019} with $Q=2$ cannot find any trajectory from $\mathbf{u}_0$ to $\mathbf{u}_F$ due to the battery constraint. The ES-Q algorithm with $Q=4$ has a higher complexity than our GIM-B algorithm because $QN>M$. However, it has a significant lower travel time $T$ even than the ES-FA and the intersection method algorithms, since it cannot derive a path $\mathbf{u}_0$ to $\mathbf{c}_2$ with the quantization points that the UAV can travel without battery replacement.
Fig. \ref{Figs3} shows the optimal graph at the global level and the corresponding maximum possible speed $v_\mathrm{max}$ for each edge under the environment in Fig. \ref{Figs2}. We can see that for each edge, the maximum travel speed $v_\mathrm{max}$ decreases as its travel distance increases.

%%%%% Figs3%%%%%%%%%%%%%%%%
% Figure environment removed
%%%%%%%%%%%%%%%%%%%%%

Fig. \ref{Figs4} compares the optimal UAV trajectory and the corresponding delivery time $T$ for different payload weight $w_3$, battery weight $w_2$, and delay $\tau_{C_1}$  at charging station $C_1$.\footnote{In Fig. \ref{Figs4}, the locations of CSs $C_3$ and $C_4$ are changed from Fig. \ref{Figs2}.} We can see that the UAV avoids $C_1$ with the higher battery swapping delay $\tau_{C_1}=200\mathrm{s}$ (red), it visits more CSs with the larger payload weight $w_3=1.5\mathrm{kg}$ (blue), and it visits less CSs with the larger battery weight $w_2=1.2\mathrm{kg}$ (green).
%%%%% Figs4%%%%%%%%%%%%%%%%
% Figure environment removed
%%%%%%%%%%%%%%%%%%%%%
In Fig. \ref{Figs5}, the optimal delivery time $T$ is plotted for different battery swapping delays and payload weights $w_3\in[0:0.1:3.5]\mathrm{kg}$ under the same environment as in Fig. \ref{Figs4}. We can verify that $T$ increases as the battery swapping delay and the payload weight increase and that the payload cannot be delivered from $\mathbf{u}_0$ to $\mathbf{u}_F$ if $w_3$ is too large, i.e., if it exceeds $2.8\mathrm{kg}$ under this setting.
%%%%% Figs5%%%%%%%%%%%%%%%%
% Figure environment removed
%%%%%%%%%%%%%%%%%%%%%

%%%%%%%%% Added in journal %%%%%%%%%%%%%
%In Fig. \ref{Figs6}, the optimal delivery time $T$ is plotted for the different battery swapping delays and the battery weights $w_2\in[0:0.05:1.5]\mathrm{kg}$ under the same environment as in Fig. \ref{Figs4}. We can see that $T$ increases as $w_2$ decreases and the payload cannot be delivered from $\mathbf{u}_0$ to $\mathbf{u}_F$ if $w_2$ is too small, i.e., if it is less than $0.6\mathrm{kg}$ under this setting.
\begin{comment}
%%%%% Figs6%%%%%%%%%%%%%%%%
% Figure environment removed
%%%%%%%%%%%%%%%%%%%%%
\end{comment}

Fig. \ref{Figs7} compares the optimal delivery time $T$ for the case that the UAV can  fly with a fixed speed of $v_\mathrm{fix}\in [15:1:30]\mathrm{m/s}$ (fixed speed) and for the case that it can change its speed in the speed set $\mathcal{V}$ (dynamic speed) under the same environment as in Fig. \ref{Figs4}. Note that in the fixed speed case, the UAV chooses its speed in the set $\{0,v_\mathrm{fix}\}$.  We can check that the dynamic speed case has a lower travel time than the fixed speed case for every $v_\mathrm{fix}\in [15:1:30]\mathrm{m/s}$ because the maximum allowable speed between each pair of CSs at the local level depends on its travel distance as shown in Fig. \ref{Figs3}. In small battery weight $w_2=0.6\mathrm{kg}$, any trajectory from $u_0$ to $u_F$ cannot be derived in the fixed speed case with $v_\mathrm{fix}>22\mathrm{m/s}$ since flying at a high speed is not efficient in terms of the energy consumption as shown in Fig. \ref{Figs1}.
%%%%% Figs7%%%%%%%%%%%%%%%%
% Figure environment removed
%%%%%%%%%%%%%%%%%%%%%
Finally, Fig. \ref{Figs8} plots the maximum deliverable weight $w_3$ according to the battery weights $w_2\in[0.5:0.02:1]\mathrm{kg}$ for different unavailable CSs, where we say that charging station $C_n$ is unavailable if its battery swapping delay $\tau_{C_n}=\infty$.
We can check that the maximum deliverable weight decreases as $w_2$ decreases and the number of unavailable CSs increases.
%%%%% Figs8%%%%%%%%%%%%%%%%
% Figure environment removed
%%%%%%%%%%%%%%%%%%%%%