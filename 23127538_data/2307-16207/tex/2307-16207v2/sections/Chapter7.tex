\section{Conclusion}\label{sec7}

For the problem of path planning for a cellular-enabled UAV with connectivity and battery constraints, the generalized intersection method with battery constraint (GIM-B) algorithm was proposed that computes an optimal path in polynomial time. Its effectiveness in terms of computational complexity and resultant mission completion time was demonstrated by comparing with previously proposed algorithms both in analytically and numerically. Furthermore, we proposed the bottleneck edge search method that finds the maximum deliverable payload weight under the connectivity and battery constraints. Various numerical results were presented to illustrate the effects of the environmental parameters on the optimal UAV path and the corresponding delivery time.

Let us conclude with some remarks on further works. We assumed that the delay at each CS is fixed over time, but in general it  changes over time in practice. It would be interesting to consider the scenario with time-varying delays at charging stations and develop shortest path finding algorithms over time-dependent graphs \cite{Orda:1990,Ding:2008}. Another interesting scenario would be to  consider more realistic coverage regions based on radio map taking into account signal blockage and reflection by buildings and interference from other BSs \cite{Chen:2017,Zhang:2021}.

%For the problem of path planning for a cellular-enabled UAV with connectivity and battery constraints, the generalized intersection method with battery constraint (GIM-B) algorithm was proposed that computes an optimal path in polynomial time. Its effectiveness in terms of computational complexity and resultant mission completion time was demonstrated by comparing with previously proposed algorithms both in analytically and numerically. For further works, it would be interesting to consider the scenario with time-varying delays at charging stations and  the scenario with more realistic coverage regions based on radio map.