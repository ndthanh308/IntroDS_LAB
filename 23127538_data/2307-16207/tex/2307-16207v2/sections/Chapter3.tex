\section{Optimal Trajectory with the Connectivity Constraint}\label{sec3}

In this section, we provide an optimal solution for Problem~1 without the battery constraint, i.e., the battery capacity is assumed to be unlimited. We note that the UAV flies with the maximum speed $v_q$ from $\mathbf{U}_0$ to $\mathbf{U}_F$ since traveling with the maximum speed minimizes the mission time without the battery constraint. Such an optimization problem can be reformulated as follows:
\begin{align} 
&\textbf{Problem 1-1} \cr
&\mbox{Objective: }~~\min_{T\geq 0,\{\mathbf{u}(t),\ t\in[0,T]\}} T\label{eq:19}\\
&\mbox{Constraints:} \cr %\label{eq:19.1}
&\mathbf{u}(0)=\mathbf{u}_0,\ \mathbf{u}(T)=\mathbf{u}_F,\ \mathbf{u}(t)\in\mathbb{R}^2\label{eq:20}\\  
&\|\nabla_t\mathbf{u}(t)\|=v_q,\ t\in[0,T]\label{eq:21}\\
&\min_{m\in\mathcal{M}}\|\mathbf{u}(t)-\mathbf{a}_m\|+\lambda_m\leq d_0,\ t\in[0,T]\label{eq:22}
\end{align}
Note that the optimization is still not trivial since it is non-convex and has an infinite number of variables.

To attack Problem 1-1, we propose a generalized intersection method that finds a trajectory of UAV satisfying the connectivity constraint by converting Problem 1-1 as an equivalent problem of finding the shortest path in an undirected weighted graph, and show that this generalized intersection method yields an optimal UAV path NP-easily.
The pseudo code of the generalized intersection method is described in Algorithm \ref{Algo1}.
%%%%%%%%%%%%%%%%% Algorithm 1: generalized intersection method %%%%%%%%%%%%%%%
\begin{algorithm}
\caption{Generalized Intersection Method} \label{Algo1}
\textbf{Input:} $v_q$, $\mathbf{u}_0$, $\mathbf{u}_F$, $\mathbf{a}_m$, $d_0$, $\lambda_m$ for $m\in\mathcal{M}$
\begin{algorithmic}[1]
\State \textbf{Def:} Function \textbf{ChkFea}($\mathbf{u}_0,\mathbf{u}_F,\mathbf{a}_m,d_0,\lambda_m$ for $m\in\mathcal{M}$)  outputs whether Problem 1-1 is feasible $(h_\mathrm{fea}=1)$ or not $(h_\mathrm{fea}=0)$, where $\mathbf{u}_0$ is the initial point, $\mathbf{u}_F$ is the final point, and $d_0$ and $\lambda_m$ for $m\in\mathcal{M}$ are the parameters about the communication environment.
\State \textbf{Def:} Function \textbf{Dijkstra}$(\mathbf{x}_1,\mathbf{x}_2,G)$ for graph $G=(V,E)$ outputs $(T,\mathbf{S}_V)$, where $T$ is the minimum total weight from $\mathbf{x}_1\in V$ to $\mathbf{x}_2\in V$ over the graph $G$ and $\mathbf{S}_V$ is the corresponding optimal sequence of visiting nodes in $V$.
\State $V_0\leftarrow\{\mathbf{u}_0,\mathbf{u}_F\}$, $E_0\leftarrow \emptyset$
\State $h_\mathrm{fea} \leftarrow$ \textbf{ChkFea}$(\mathbf{u}_0, \mathbf{u}_F, \mathbf{a}_m, d_0, \lambda_m$ for $m\in\mathcal{M})$
\If {$h_\mathrm{fea}=1$} \hfill\Comment{Problem 1-1 is feasible}
    % 시작점, 도착점 V_0에 추가
    \LeftComment{Step 1. Vertex construction: Construct a vertex set $V_0$ consisting of the initial, the final, and the intersection points.}
    \For{$m,m'\in\mathcal{M}$, $m<m'$} 
        \If{$\|\mathbf{a}_m-\mathbf{a}_{m'}\|\leq 2d_0-\lambda_m-\lambda_{m'}$}
            \State $V_0\leftarrow V_0\cup\{\mathbf{x}\in\mathbb{R}^2 \vert \ \|\mathbf{x}-\mathbf{a}_m\|=d_0-\lambda_m,$
            \Statex \qquad\qquad \ $\|\mathbf{x}-\mathbf{a}_{m'}\|=d_0-\lambda_{m'}\}$
        \EndIf
    \EndFor 
    \LeftComment{Step 2. Edge construction: Construct an edge set $E_0$ consisting of the line segments lying inside the set of coverage regions.}
    \For{$\mathbf{x}_1,\mathbf{x}_2\in V_0$, $\mathbf{x}_1\neq \mathbf{x}_2$}
        \State $h_\mathrm{out}\leftarrow$ \textbf{ChkOut}$(\mathbf{x}_1,\mathbf{x}_2,\mathbf{a}_m, d_0, \lambda_m$ for $m\in\mathcal{M})$
        \If {$h_\mathrm{out}=0$}
            \State $E_0\leftarrow E_0\cup (\mathbf{x}_1,\mathbf{x}_2,\|\mathbf{x}_1-\mathbf{x}_2\|/v_q)$
        \EndIf
    \EndFor 
    \LeftComment{Step 3. Path search: Find an optimal path from the initial point to  the final point over the graph.}
    \State $G_0\leftarrow(V_0,E_0)$ \hfill\Comment{Construct graph $G_0$}
    \State $(T,\mathbf{S}_{V_0})\leftarrow$ \textbf{Dijkstra}$(\mathbf{u}_0,\mathbf{u}_F,G_0)$
    \State $\mathbf{u}(t) \text{ for }t\in[0,T]\leftarrow$ \textbf{FindPath}$(\mathbf{S}_{V_0},v_q)$
\Else \Comment{Problem 1-1 is not feasible}
\State $T\leftarrow\infty$, $\mathbf{u}(t)\leftarrow \mathrm{Null}$ for $t\in[0,T]$
\EndIf
\end{algorithmic}
\textbf{Output:} \big($h_\mathrm{fea}$, $T$, $\mathbf{u}(t)$ for $t\in[0,T]$\big)
\end{algorithm}
%%%%%%%%%%%%%%%%%%%%%%%%
This algorithm first checks (in line 3) whether the problem is feasible or not via the checking feasibility function ChkFea, which outputs whether the problem 1-1 is feasible $(h_\mathrm{fea}=1)$ or not $(h_\mathrm{fea}=0)$ according to the initial point $\mathbf{u}_0$, the final point $\mathbf{u}_F$, and the location and the effective coverage region of each BS. This function can be constructed by applying \cite[Proposition~1]{Zhang:2019} in the case that the BSs have the different coverage radii and its pseudo code is omitted.
%whose pseudo code is given in  Algorithm \ref{Algo2} and is explained later.
If the problem is feasible, an undirected weighted graph $G_0=(V_0,E_0)$ is constructed based on the intersection points of the coverage boundaries (in lines $6$-$17$). 
Specifically, the vertex set $V_0$ consists of the initial point $\mathbf{u}_0$, the final point $\mathbf{u}_F$, and the intersection points of the coverage boundaries (in lines $6$-$10$). 
The edge set $E_0$ is constructed (in lines $11$-$16$) by including a line segment $\overline{\mathbf{x}_1\mathbf{x}_2}$ between two different vertices $\mathbf{x}_1,\mathbf{x}_2\in V_0$ if the line segment lies inside the set of coverage regions, which is checked through the function ChkOut whose pseudo code is provided in Algorithm \ref{Algo3} and is explained later. Such an edge is denoted by a tuple $(\mathbf{x}_1,\mathbf{x}_2,\|\mathbf{x}_1-\mathbf{x}_2\|/{v_q})$, where the weight of the edge $\|\mathbf{x}_1-\mathbf{x}_2\|/{v_q}$ is given by the minimum travel time between $\mathbf{x}_1$ and $\mathbf{x}_2$.
After constructing a weighted undirected graph, an optimal path from $\mathbf{u}_0$ to $\mathbf{u}_F$ over the graph is derived (in lines $18$-$19$). We first find an optimal sequence $\mathbf{S}_{V_0}$ of visiting nodes over the graph and the corresponding weight (equal to the mission time $T$) via the Dijkstra algorithm \cite{Dijkstra:1959} that finds the minimum weight path between two nodes over a weighted graph with low complexity. Then, the corresponding UAV trajectory can be derived through the function FindPath, which outputs the UAV trajectory according to the  sequence $\mathbf{S}_{V_0}$ of visiting points and the speed $v_q$. This function can be constructed similarly as in \cite[$(25)$-$(27)$]{Zhang:2019} and its pseudo code is omitted. An example of the graph $G_0$ and the corresponding optimal trajectory by Algorithm \ref{Algo1} is illustrated in Fig. \ref{Fig3}.

%Note that only BS pairs $(\mathbf{a}_m,\mathbf{a}_{m'})$ where $m,m'\in\mathcal{M}$ and $m<m'$ have the intersected points if it satisfies the following condition:
%\begin{align}
%\|\mathbf{a}_m-\mathbf{a}_{m'}\|\leq 2d_0-\lambda_m-\lambda_{m'}.\label{eq:22}
%\end{align}

% Figure environment removed



%%% Skip algorithm 2 !!!%%%%%%%%%
\begin{comment}
\sh{Algorithm \ref{Algo2} is  a pseudo code for the function ChkFea that checks the feasibility of Problem 1-1 by taking a graph theoretic approach. It first constructs an undirected graph $G_\mathrm{fea}=(V_\mathrm{fea},E_\mathrm{fea})$, where the vertex set $V_\mathrm{fea}$ consists of the set of \hs{BSs $(\mathbf{a}_1,...,\mathbf{a}_M)$,} the initial point $\mathbf{a}_{M+1}\triangleq\mathbf{u}_0$, and the final point $\mathbf{a}_{M+2}\triangleq\mathbf{u}_F$ and the edge set $E_\mathrm{fea}$ includes a line segment between two different vertices $(\mathbf{a}_m,\mathbf{a}_{m'})$ if it satisfies\footnote{We set that the coverage radii of $\mathrm{BS}_{M+1}$ and $\mathrm{BS}_{M+2}$ are zero, i.e., $\lambda_{M+1}=\lambda_{M+2}=d_0$.}
\begin{align}
\|\mathbf{a}_m-\mathbf{a}_{m'}\|\leq 2d_0-\lambda_m-\lambda_{m'}.\label{eq:24}
\end{align}
Here, \eqref{eq:24} means that an UAV in the coverage region of $\mathrm{BS}_m$ can move to any location in the coverage region of $\mathrm{BS}_{m'}$ and vice versa while maintaining the connectivity. Hence, the problem is feasible if and only if the vertices $\mathbf{u}_0$ and $\mathbf{u}_F$ are connected in the graph $G_\mathrm{fea}$.\footnote{This result can be analytically proved similarly as the proof of \cite[Proposition~1]{Zhang:2019}.} Then, we check whether $\mathbf{u}_0$ and $\mathbf{u}_F$ are connected $(h_\mathrm{fea}=1)$ or not $(h_\mathrm{fea}=0)$ in the graph $G_\mathrm{fea}$ via the breadth-first search (BFS) algorithm \cite{Lee:1961}, which searches all connected nodes from a start node in a graph with low complexity.}

%%%%%%% Algorithm 2: Checking Feasibility %%%%%%%%%
\begin{algorithm}
\caption{Function ChkFea} \label{Algo2}
\textbf{Input:} $\mathbf{u}_0, \mathbf{u}_F, \mathbf{a}_m, d_0, \lambda_m \text{ for } m\in\mathcal{M}$
\begin{algorithmic}[1]
\State \hs{\textbf{Def:} Function \textbf{BFS}$(\mathbf{x}_1,\mathbf{x}_2,G)$ for graph $G=(V,E)$ checks whether $\mathbf{x}_1\in V$ and $\mathbf{x}_2\in V$ are connected in the graph $G$ (output: $1$) or not (output: $0$).}
\State $\mathbf{a}_{M+1}\leftarrow \mathbf{u}_0$, $\mathbf{a}_{M+2}\leftarrow \mathbf{u}_F$, $\lambda_{M+1},\lambda_{M+2}\leftarrow d_0$
\State $V_\mathrm{fea}\leftarrow\{\mathbf{u}_0,\mathbf{u}_F,\mathbf{a}_m \text{ for } m\in\mathcal{M}\}$, $E_\mathrm{fea}\leftarrow \emptyset$
\LeftComment{Lines $4$-$8$: Construct edge set $E_\mathrm{fea}$}
\For{$m,m'\in[1:M+2]$, $m<m'$} 
    \If{$\|\mathbf{a}_m-\mathbf{a}_{m'}\|\leq 2d_0-\lambda_m-\lambda_{m'}$}
    \State $E_\mathrm{fea}\leftarrow E_\mathrm{fea}\cup (\mathbf{a}_m,\mathbf{a}_{m'})$
    \EndIf
\EndFor
\State $G_\mathrm{fea}\leftarrow(V_\mathrm{fea},E_\mathrm{fea})$ 
\hfill\Comment{Construct graph $G_\mathrm{fea}$}
\LeftComment{\hs{Test whether $\mathbf{u}_0$ and $\mathbf{u}_F$ are connected in $G_\mathrm{fea}$}}
\State $h_\mathrm{fea}\leftarrow$ \textbf{BFS}$(\mathbf{u}_0,\mathbf{u}_F,G_\mathrm{fea})$
\end{algorithmic}
\textbf{Output:} $h_\mathrm{fea}\in\{0,1\}$
\end{algorithm}
%%%%%%%%%%%%%%%%%%%%%%%%
\end{comment}

% Algorithm 3 동작 원리 설명
Algorithm \ref{Algo3} describes the function ChkOut which tests whether a line segment $\overline{\mathbf{x}_1\mathbf{x}_2}$ between two different vertices $\mathbf{x}_1,\mathbf{x}_2\in V_0$ lies in the set of coverage regions. We say that the line segment experiences an outage if there exists $\xi\in[0,1]$ that satisfies the following condition:
\begin{align}
\min_{m\in\mathcal{M}}\|\pmb{\alpha}(\xi)-\mathbf{a}_m\|+\lambda_m>d_0,\label{eq:25}
\end{align}
where $\pmb{\alpha}(\xi)\triangleq \mathbf{x}_1+\xi(\mathbf{x}_2-\mathbf{x}_1)$ for $\xi\in[0,1]$ represents a point in the line segment $\overline{\mathbf{x}_1\mathbf{x}_2}$. Here, \eqref{eq:25} means that the UAV  experiences an outage at point $\pmb{\alpha}(\xi)$, i.g., the UAV cannot be connected with every BS at point $\pmb{\alpha}(\xi)$. To check whether the line segment experiences an outage, the function ChkOut verifies whether there exists $\xi\in[0,1]$ that satisfies \eqref{eq:25}. Let us define the safe interval $\mathcal{T}_\mathrm{safe}\triangleq [0,\xi']$ for some $\xi'\in[0,1]$ as the line segment such that every $\pmb{\alpha}(\xi)$ for $\xi\in \mathcal{T}_\mathrm{safe}$ has been checked to be inside the coverage regions, i.e., there exists $m\in \mathcal{M}$ such that $\|\pmb{\alpha}(\xi)-\mathbf{a}_m\|+\lambda_m \leq d_0$.  The function ChkOut first checks whether $\xi=0$ is included in the coverage regions and then repeatedly updates the safe interval or declares an outage in the following way. Let the current safe interval be given as $[0,\xi']\subseteq [0,1]$. If the point $\pmb{\alpha}(\xi'+\epsilon)$ is checked to be connected with $\mathrm{BS}_m$ for sufficiently small constant $\epsilon>0$, then the safe interval is extended by including the range of $\xi$ where $\mathbf{\alpha}(\xi)$ is connected with $\mathrm{BS}_m$, i.e.,  $\|\pmb{\alpha}(\xi)-\mathbf{a}_m\|\leq d_0-\lambda_m$. This algorithm ends if $\pmb{\alpha}(\xi'+\epsilon)$ cannot be connected with every BS $(h_\mathrm{out}=1)$ or the safe interval reaches  $[0,1]$ $(h_\mathrm{out}=0)$, where $h_\mathrm{out}$ is the indicator whether the line segment experiences an outage $(h_\mathrm{out}=1)$ or not $(h_\mathrm{out}=0)$. An example of updating the safe interval is shown in Fig. \ref{Fig4}.

%%%%%%% Algorithm 3: Checking Outage %%%%%%%%%%
\begin{algorithm}[t]
\caption{Function ChkOut} \label{Algo3}
\textbf{Input:} $\mathbf{x}_1,\mathbf{x}_2\in V_0,\mathbf{a}_m, d_0, \lambda_m \text{ for } m\in\mathcal{M}$
\begin{algorithmic}[1]
\State\textbf{Def:} $\pmb{\alpha}(\xi)\triangleq \mathbf{x}_1+\xi(\mathbf{x}_2-\mathbf{x}_1)$ for $\xi\in[0,1]$
\State $h_\mathrm{out}\leftarrow 0$, $\xi'\leftarrow 0$, $\xi''\leftarrow 0$ 
\State $\epsilon\leftarrow 10^{-10}$ \hfill\Comment{Sufficiently small positive constant}
\While{$\xi'<1$}   %\hfill\Comment{Edge $(\mathbf{x}_1,\mathbf{x}_2)$ is covered if $\xi'=1$}  
\LeftComment{Update safe interval $\mathcal{T}_\mathrm{safe}$ from 
$[0,\xi']$ to $[0,\xi'']$ if $\pmb{\alpha}(\xi'+\epsilon)$ is included in the set of coverage regions.}
    \For {$m\in\mathcal{M}$}  \hfill\Comment{Find BS which covers $\pmb{\alpha}(\xi'+\epsilon)$.}
        \If{$\|\pmb{\alpha}(\xi'+\epsilon)-\mathbf{a}_m\|\leq d_0-\lambda_m$}
            \State $\xi''\!\leftarrow\!\max\{\xi\in [0,1] \vert \ \|\pmb{\alpha}(\xi)-\mathbf{a}_m\|\leq d_0-\lambda_m\}$
            %\State $\mathcal{M''}\leftarrow\mathcal{M'}\setminus m$
            \State \textbf{break} 
        \EndIf
    \EndFor
    \If{$\xi''=\xi'$}  \hfill\Comment{ $\pmb{\alpha}(\xi'+\epsilon)$ experiences an outage.}
         \State $h_\mathrm{out}\leftarrow 1$
         \State  \textbf{break}
    \EndIf
    \State $\xi'\leftarrow\xi''$
    %\If{$\mathcal{M''}=\mathcal{M'}$}  \hfill\Comment{$\pmb{\alpha}(\xi+0)$ is not covered}
    %    \State $h_\mathrm{out}\leftarrow 1$
    %    \State \textbf{break}
    %\EndIf
    %\State $\mathcal{M'}\leftarrow\mathcal{M''}$
\EndWhile \hfill\Comment{$\xi'=1$ means that $\mathcal{T}_\mathrm{safe}=[0,1]$.}
\end{algorithmic}
\textbf{Output:} $h_\mathrm{out}$
\end{algorithm}
%%%%%%%%%%%%%%%%%%%%%%%%%%%%%%%%%%%%%%%%

%This algorithm repeatedly extends the trusted interval $[0,\xi']$ where every $\pmb{\alpha}(\xi)$ for $\xi\in [0,\xi']$ has been checked to be covered by the BSs and $\xi'\in[0,1]$ is an updating factor.
%We note that the trusted interval is extended by refining the updating factor $\xi'\in[0,1]$ from $\xi'=0$ to $\xi'=1$. Algorithm \ref{Algo3} starts with the initial updating factor $\xi'=0$. Let the trusted interval is $[0,\xi']\subseteq [0,1]$. If $\pmb{\alpha}(\xi'+0)$ is checked to be covered by $\mathrm{BS}_m$, then we extend the trusted interval by including the range of $\xi$ where $\mathbf{\alpha}(\xi)$ is covered by $\mathrm{BS}_m$ (Lines 4-9, 14). We note that the updating factor $\xi'$ increases when the trusted interval $[0,\xi']$ is extended. This algorithm ends if $\pmb{\alpha}(\xi'+0)$ is not covered by any BS $(h_\mathrm{out}=1)$ or the updating factor $\xi'$ is refined to $\xi'=1$ $(h_\mathrm{out}=0)$, where $h_\mathrm{out}$ is the hypothesis whether the line segment is an outage $(h_\mathrm{out}=1)$ or not $(h_\mathrm{out}=0)$. An example of the hypothesis is illustrated in Fig. \ref{Fig4}.

% Figure environment removed

Now, the following theorems show that our generalized intersection method yields an optimal solution of Problem 1-1 NP-easily.%\footnote{We note that the intersection points are crucial for deriving an optimal trajectory as shown in Fig. \ref{Fig5}.}

\begin{theorem}\label{Thm1}
The generalized intersection method outputs an optimal solution for Problem 1-1.
\end{theorem}
\begin{proof}
It was previously shown in \cite[Proposition~3]{Zhang:2019} that an optimal solution of Problem 1-1 consists of line segments, where its breakpoints are selected in the overlapping regions of the coverage regions of two different BSs. Following the result of \cite{Zhang:2019}, in this proof, we show that the breakpoints of an optimal path should be selected in the intersection points of the coverage boundaries of BSs. Note that the problem is equivalent to deriving a path which achieves the shortest distance under the connectivity constraint since the speed of the UAV is fixed at $v_q$.

For a proof by contradiction, let us assume that an optimal path of the UAV has a breakpoint $\mathbf{x}_\mathrm{br}$ in the overlapping region of $\mathrm{BS}_1$ and $\mathrm{BS}_2$ except the corresponding intersection points. Then, there exists sufficiently small  $\delta>0$ that the set  $\mathcal{R}_\delta\triangleq\{\mathbf{x}\in\mathbb{R}^2| \ \|\mathbf{x}-\mathbf{x}_\mathrm{br}\|\leq \delta\}$ is included in the set of the coverage regions because the following inequality holds:
\begin{align}
\|\mathbf{x}_\mathrm{br}-\mathbf{a}_m\|<d_0-\lambda_m \text{  at  } m=1 \text{ or }2.\label{eq:26}
\end{align}
Here, \eqref{eq:26} means that the point $\mathbf{x}_\mathrm{br}$ is inside the coverage region of $\mathrm{BS}_1$ or $\mathrm{BS}_2$ except its coverage boundary. Now, let us denote $\pmb{\beta}_1$ and $\pmb{\beta}_2$ as two intersections of the boundary of $\mathcal{R}_\delta$ and the path of the UAV. Then the following holds by triangular inequality:
\begin{align}
 \|\pmb{\beta}_1-\pmb{\beta}_2\|< \|\pmb{\beta}_1-\mathbf{x}_\mathrm{br}\|+\|\mathbf{x}_\mathrm{br}-\pmb{\beta}_2\|.\label{eq:27}
\end{align}
We note that in $\eqref{eq:27}$, only strict inequality holds since the point $\mathbf{x}_\mathrm{br}$ is a breakpoint of the path of the UAV. The path of the UAV includes the line segments $\overline{\pmb{\beta}_1\mathbf{x}_\mathrm{br}}$ and $\overline{\mathbf{x}_\mathrm{br}\pmb{\beta}_2}$. Hence, it is a contradiction that the path is an optimal solution for Problem 1-1 because the overall length of the path can be strictly decreased by substituting $\overline{\pmb{\beta}_1\pmb{\beta}_2}$ for $\overline{\pmb{\beta}_1\mathbf{x}_\mathrm{br}}$ and $\overline{\mathbf{x}_\mathrm{br}\pmb{\beta}_2}$ as shown in Fig. \ref{Fig6}.
\end{proof}

\begin{comment}
\begin{proof}
We first notice that an optimal solution of Problem 1-1 consists of line segments, where its breakpoints are selected in the overlapping region of any two different BSs \cite[Proposition~3]{Zhang:2019} and the problem is equivalent to deriving a path which achieves the shortest distance from $\mathbf{u}_0$ to $\mathbf{u}_F$ since the speed of the UAV is fixed at $v_q$. We prove the theorem by showing that an alternative solution whose breakpoints are only selected in the intersection points of the coverage boundaries can be always constructed and it achieves equal or smaller delivery time $T$ compared to the solution of \cite{Zhang:2019}. 

Without loss of generality, we consider a base trajectory with BS association sequence $[\mathrm{BS}_1,\mathrm{BS}_2,...,\mathrm{BS}_k]$ for $k\leq M$, where its $k'$th breakpoint $\pmb{\beta}_{k'}\in\mathbb{R}^2$ $(k'\leq k-1)$ satisfies the following condition:
\begin{align}
\|\pmb{\beta}_{k'}-\mathbf{a}_j\| \leq d_0-\lambda_j \text{  for  } j\in\{k',k'+1\}. \label{eq:26}
\end{align}
For convenience, we set $\pmb{\beta}_k=\mathbf{u}_F$. Such initial trajectory $\mathcal{L}$ consists by the line segments $\overline{\mathbf{u}_0\pmb{\beta}_1}$, $\overline{\pmb{\beta}_1\pmb{\beta}_2}$,..., $\overline{\pmb{\beta}_{k-1}\pmb{\beta}_k}$. To reduce the length of the trajectory, we repeatedly update the trajectory $\mathcal{L}$ from $\mathbf{u}_0$ to $\pmb{\beta}_k$. Let us define the completed path $\mathcal{L}_c$ from $\mathbf{u}_0$ to $\mathbf{\bar{u}}$ is the completely updated part of the trajectory, where $\mathbf{\bar{u}}\in\mathbb{R}^2$ is the updating point. We repeatedly save the completely updated part of the trajectory $\mathcal{L}$ in the completed path $\mathcal{L}_c$. We note that $\mathcal{L}_c$ is updated by refining the updating point $\mathbf{\bar{u}}$ and its breakpoints are only selected in the intersection points. We start at the updating point $\mathbf{\bar{u}}\leftarrow\mathbf{u}_0$ and then verify whether $\overline{\mathbf{\bar{u}}\pmb{\beta}_1}$ and $\overline{\pmb{\beta}_1\pmb{\beta}_2}$ can be replaced by $\overline{\mathbf{\bar{u}}\pmb{\beta}_2}$ in the initial trajectory since $\|\mathbf{\bar{u}}-\pmb{\beta}_2\| \leq \|\mathbf{\bar{u}}-\pmb{\beta}_1\| + \|\pmb{\beta}_1-\pmb{\beta}_2\|$ holds by triangular inequality.
%\begin{align}
%\|\mathbf{\bar{u}}-\pmb{\beta}_2\| \leq \|\mathbf{\bar{u}}-\pmb{\beta}_1\| + \|\pmb{\beta}_1-\pmb{\beta}_2\|. \label{eq:26.1}
%\end{align}
If $\overline{\mathbf{\bar{u}}\pmb{\beta}_2}$ does not experience an outage, then we update the initial trajectory as $\mathcal{L}:$ $\overline{\mathbf{\bar{u}}\pmb{\beta}_2}$, $\overline{\pmb{\beta}_2\pmb{\beta}_3}$,..., $\overline{\pmb{\beta}_{k-1}\pmb{\beta}_k}$ and repeat the same process until $\overline{\mathbf{\bar{u}}\pmb{\beta}_i}$ $(i\in[2:k])$ experiences an outage.\footnote{We note that the trajectory $\overline{\mathbf{u}_0\mathbf{u}_F}$ is the alternative solution of Problem 1-1 if $\overline{\mathbf{\mathbf{u}_0\pmb{\beta}_k}}$ does not still experience an outage.} When $\overline{\mathbf{\bar{u}}\pmb{\beta}_i}$ experiences an outage, we have the updated trajectory $\mathcal{L}:$ $\overline{\mathbf{\bar{u}}\pmb{\beta}_{i-1}}$, $\overline{\pmb{\beta}_{i-1}\pmb{\beta}_i}$,..., $\overline{\pmb{\beta}_{k-1}\pmb{\beta}_k}$ satisfying the following inequality:
\begin{align}
\|\mathbf{\bar{u}}-\pmb{\beta}_{i-1}\|+ \sum_{j=i}^{k}\|\pmb{\beta}_{j-1}-\pmb{\beta}_j\|  \leq \ell_0,\label{eq:27}
\end{align}
where $\ell_0=\|\mathbf{\bar{u}}-\pmb{\beta}_1\| + \sum_{j=2}^{k}\|\pmb{\beta}_{j-1}-\pmb{\beta}_j\|$ is the length of the initial trajectory. Here, \eqref{eq:27} means the length of the updated trajectory is not larger than $\ell_0$. 

Next, we improve the trajectory $\mathcal{L}:$ $\overline{\mathbf{\bar{u}}\pmb{\beta}_{i-1}}$, $\overline{\pmb{\beta}_{i-1}\pmb{\beta}_i}$,..., $\overline{\pmb{\beta}_{k-1}\pmb{\beta}_k}$, where $i\in[2:k]$ and $\overline{\mathbf{u}_0\pmb{\beta}_i}$ experiences an outage. Let us assume that $\pmb{\tilde{\beta}}_{i'1}$ and $\pmb{\tilde{\beta}}_{i'2}$ are the two intersection points of $\mathrm{BS}_{i'}$ and $\mathrm{BS}_{i'+1}$ where $i'\in [1:k-1]$. We select a intersection point $\pmb{\tilde{\beta}}_{i',j}$ which satisfies the following condition:
\begin{align}
\begin{split}\label{eq:28}
\pmb{\tilde{\beta}}_{i',j}&=\argmax_{\pmb{\tilde{\beta}}_{i',j}}\bigl\{i'|\pmb{\tilde{\beta}}_{i',j}\in \Delta(\mathbf{\bar{u}}\pmb{\beta}_{i-1}\pmb{\beta}_{i}), \ \overline{\mathbf{\bar{u}}\pmb{\tilde{\beta}}_{i',j}}\text{ lies}\\
&\text{in the set of the coverage regions}\bigr\}, 
\end{split}
\end{align}
where $\Delta(\mathbf{x}_1\mathbf{x}_2\mathbf{x}_3)$ is the convex hull by $\mathbf{x}_1, \mathbf{x}_2, \mathbf{x}_3 \in \mathbb{R}^2$.\footnote{It can be checked that the set in \eqref{eq:28} is not an empty set.} Then, we can find the intersection $\pmb{\phi}\in\mathbb{R}^2$ of $\overrightarrow{\mathbf{\bar{u}}\pmb{\tilde{\beta}}_{i',j}}$ and $\overline{\pmb{\beta}_{i-1}\pmb{\beta}_{i}}$. The trajectory $\mathcal{T}$ is updated as $\overline{\mathbf{\bar{u}}\pmb{\phi}}$, $\overline{\pmb{\phi}\pmb{\beta}_{i-1}}$,..., $\overline{\pmb{\beta}_{k-1}\pmb{\beta}_k}$ since the following holds by triangular inequality:
\begin{align}
\|\mathbf{\bar{u}}-\pmb{\phi}\| +\|\pmb{\phi}-\pmb{\beta}_i\| +\sum_{j=i+1}^{k}\|\pmb{\beta}_{j-1}-\pmb{\beta}_j\|  \leq \ell_0.\label{eq:29}
\end{align}

Finally, we include the line segment $\overline{\mathbf{\bar{u}}\pmb{\tilde{\beta}}_{i',j}}\subseteq \overline{\mathbf{\bar{u}}\pmb{\phi}}$ as a part of the completed path $\mathcal{L}_c$ and then refining the updating point $\mathbf{\bar{u}}\leftarrow \pmb{\tilde{\beta}}_{i',j}$. We repeatedly perform the entire process at the remained trajectory $\overline{\mathbf{\bar{u}}\pmb{\phi}}$, $\overline{\pmb{\phi}\pmb{\beta}_{i-1}}$,..., $\overline{\pmb{\beta}_{k-1}\pmb{\beta}_k}$ and update the completed path by refining  $\mathbf{\bar{u}}$ until $\overline{\mathbf{\mathbf{\bar{u}}\pmb{\beta}_k}}$ does not experience an outage. The completed path $\mathcal{L}_c$ from $\mathbf{u}_0$ to $\pmb{\beta}_k$ including $\overline{\mathbf{\mathbf{\bar{u}}\pmb{\beta}_k}}$ is an alternative solution of Problem 1-1 since its length is not larger than $\ell_0$ and it consists of the line segments whose breakpoints are in $\bigl\{\pmb{\tilde{\beta}}_{i,j}|i\in[1:k-1],j\in\{1,2\}\bigr\}$. The proof process is also shown in Fig. \ref{Fig6}.
%공집합 아니라는거 보이기: footnote
\end{proof}

% Figure environment removed
\end{comment}

\begin{comment}
% Figure environment removed
\end{comment}

% Figure environment removed

\begin{theorem}\label{Thm2}
The time complexity of the generalized intersection method is $O(M^6)$.
\end{theorem}
\begin{proof}
Let us first state the cardinality of the set $|V_0|=O(M^2)$. The steps in Algorithm \ref{Algo1} have the following complexities:
\begin{itemize}
\item Complexity of function ChkFea: It was shown that the complexity to check whether Problem 1-1 is feasible is $O(M^2)$ \cite{Zhang:2019}.
\item Step 1. Vertex construction: This step has complexity $O(M^2)$ since the intersection points of the coverage boundaries by a BS pair is derived by calculating the quadratic equations in Line $6$ of Algorithm \ref{Algo1} and the number of the possible BS pairs is $O(M^2)$.
\item Step 2. Edge construction: The complexity of testing whether a line segment experiences an outage via the function ChkOut is $O(M^2)$ and every line segment $\overline{\mathbf{x}_1\mathbf{x}_2}$ by two different vertices $\mathbf{x}_1,\mathbf{x}_2\in V_0$ should be tested. Hence, the complexity of this step is $O(M^2)\cdot |V_0|^2=O(M^6)$. 
\item Step 3. Path search: The complexity of the Dijkstra algorithm in the graph $G_0$ is $O(|V_0|^2)=O(M^4)$ \cite{West:2001}.
\end{itemize}
Consequently, the complexity of the generalized intersection method is $O(M^6)$, which is dominated at the edge $E_0$ construction step.
\end{proof}

\begin{table*}
\centering
\begin{tabular}{@{} c || c | c @{}}
\cline{1-3}
Algorithm & Complexity & Performance gap\\ \cline{1-3}
Exhaustive search \cite{Zhang:2019} & $O(M!M^{3.5})$ & 0\\ \cline{1-3}
Exhaustive search with fixed association \cite{Zhang:2019} & $O(M^{3.5})$ & $O(Md_0/{v_q})$ \\ \cline{1-3}
Exhaustive search with quantization \cite{Zhang:2019} & $O(M^4Q^2)$ & $O((Md_0/{v_q})\sin(1/{Q}))$ \\ \cline{1-3}
Intersection method \cite{Chen:2020} by checking outages via Algorithm \ref{Algo3} & $O(M^4)$ & $O(Md_0/{v_q})$ \\ \cline{1-3}
Ours (Generalized intersection method)  & $O(M^6)$ & $0$ \\ \cline{1-3}
\end{tabular}
\caption{Comparison of algorithms for Problem 1-1}\label{Tab1}
\end{table*}

Now, let us compare our generalized intersection method with previously proposed algorithms to solve Problem 1-1. Table \ref{Tab1} summarizes the complexity and the performance gap from the optimal solution for each algorithm. In the following, we provide brief descriptions of previous algorithms and observations based on  Table \ref{Tab1}. 
\begin{itemize}
    \item Among the algorithms in Table \ref{Tab1}, our generalized intersection method outputs an optimal solution NP-easily.
    \item The exhaustive search (ES), exhaustive search with fixed association (ES-FA), and exhaustive search with quantization (ES-Q) algorithms are proposed in \cite{Zhang:2019}. In \cite{Zhang:2019}, it was shown that an optimal trajectory from $\mathbf{u}_0$ to $\mathbf{u}_F$ consists of line segments, where its breakpoints are selected inside the overlapping regions of the coverage regions of two different BSs \cite[Proposition~3]{Zhang:2019}. In this approach, it is not possible to find an optimal solution via a graph theoretic approach since the overlapping regions consist of infinite number of points. The ES algorithm \cite{Zhang:2019} is an optimal algorithm that finds optimal breakpoints inside the overlapping regions based on convex optimization, which is NP-hard over $M$. To reduce the complexity, two suboptimal algorithms are also proposed in \cite{Zhang:2019}, i.e., ES-FA and ES-Q algorithms, which are NP-easy. The ES-FA algorithm is basically the same with ES algorithm, except that the sequence of BS association is fixed in advance, and the ES-Q algorithm applies a graph theoretic approach by quantizing each overlapping region to a finite number of the points. The ES-FA algorithm has lower complexity than the generalized intersection method, but its performance gap increases in $M$. For the ES-Q algorithm, let $Q$ denote the number of quantization points in each overlapping region. Note that this algorithm has an increasing performance gap in $M$ for $Q=O(M)$ and has a higher complexity than the generalized intersection method for $Q=\omega(M)$.
    \item The intersection method proposed in \cite{Chen:2020} only includes the intersection points as the possible breakpoints and applies a graph theoretic approach like our generalized intersection method. However, this algorithm is suboptimal because it searches a path for a fixed BS association sequence which is chosen in a heuristic way, similarly as the ES-FA algorithm \cite{Zhang:2019}. Also, it does not explicitly suggest a function like our ChkOut function in Algorithm \ref{Algo3}, checking whether each line segment between two vertices in the graph experiences an outage. If we apply the ChkOut function in Algorithm \ref{Algo3}, the intersection method \cite{Chen:2020} has the same performance gap with the ES-FA algorithm \cite{Zhang:2019} with a higher complexity.
\end{itemize}

%\item The work \cite{Chen:2020} proposed the intersection method which is similar with the Suboptimal-association algorithm \cite{Zhang:2019} but only includes the intersection points as the possible breakpoints since the intersection points are crucial for deriving an optimal trajectory as shown in Fig. \ref{Fig5}. 

%The intersection method is NP-easy and only uses graph theoretic approach. However, this algorithm is not optimal since it searches a path in a fixed BS association sequence. We proposed our LCI method which considers every BS association and checks whether each line segment between two vertices in the graph $G_0$ experiences an outage by Algorithm \ref{Algo3}, which is a newly proposed low complexity algorithm to check whether a line segment experiences an outage. The following theorems show that our LCI method derives an optimal algorithm of problem 1-1 NP-easily.
    
%\item The suboptimal-association \cite{Zhang:2019} and the Intersection \cite{Chen:2020} methods have lower complexity than the LCI method, but their performance gap bounds increase in $M$.
    
%\item The suboptimal-quantization algorithm \cite{Zhang:2019} has the increasing performance gap bound on $M$ for $Q=O(M)$ and higher complexity than the LCI method for $Q=\omega(M)$.

    
%The aforementioned analysis implies that our LCI method improves the performance of Problem 1-1 compared to the other algorithms in both complexity and performance gap bound. 

