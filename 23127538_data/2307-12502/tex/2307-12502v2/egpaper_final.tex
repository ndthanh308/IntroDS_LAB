\documentclass[10pt,twocolumn,letterpaper]{article}

\usepackage{iccv}
\usepackage{times}
\usepackage{epsfig}
\usepackage{graphicx}
\usepackage{amsmath}
\usepackage{amssymb}
\usepackage{booktabs}
% \usepackage{boondox-cal}
\usepackage{algorithm}
\usepackage{algorithmic}
\usepackage{setspace}
\usepackage[accsupp]{axessibility}

% Include other packages here, before hyperref.

% If you comment hyperref and then uncomment it, you should delete
% egpaper.aux before re-running latex.  (Or just hit 'q' on the first latex
% run, let it finish, and you should be clear).
\usepackage[breaklinks=true,bookmarks=false]{hyperref}
\hypersetup{colorlinks=true}

\iccvfinalcopy % *** Uncomment this line for the final submission

\def\iccvPaperID{****} % *** Enter the ICCV Paper ID here
\def\httilde{\mbox{\tt\raisebox{-.5ex}{\symbol{126}}}}

% Pages are numbered in submission mode, and unnumbered in camera-ready
\ificcvfinal\pagestyle{empty}\fi

\begin{document}

%%%%%%%%% TITLE
\title{Cross Contrasting Feature Perturbation for Domain Generalization}

\author{Chenming Li$^1$  \quad  Daoan Zhang$^1$ \quad
Wenjian Huang$^1$ \quad
Jianguo Zhang$^{1,2,}$\thanks{Corresponding author.}\\
$^1$Research Institute of Trustworthy Autonomous Systems and Department of Computer Science \\ and Engineering, Southern University of Science and Technology, Shenzhen, China\\
$^2$Peng Cheng Laboratory, Shenzhen, China\\
{\tt\small 12132339@mail.sustech.edu.cn, 12032503@mail.sustech.edu.cn, \{huangwj, zhangjg\}@sustech.edu.cn,}\\
}

% \author{Chenming Li\\
% Southern University of Science and Technology\\
% Institution1 address\\
% {\tt\small firstauthor@i1.org}
% % For a paper whose authors are all at the same institution,
% % omit the following lines up until the closing ``}''.
% % Additional authors and addresses can be added with ``\and'',
% % just like the second author.
% % To save space, use either the email address or home page, not both
% \and
% Second Author\\
% Institution2\\
% First line of institution2 address\\
% {\tt\small secondauthor@i2.org}
% }

\maketitle
% Remove page # from the first page of camera-ready.
\ificcvfinal\thispagestyle{empty}\fi


%%%%%%%%% ABSTRACT
\begin{abstract}
   Domain generalization (DG) aims to learn a robust model from source domains that generalize well on unseen target domains. Recent studies focus on generating novel domain samples or features to diversify distributions complementary to source domains. Yet, these approaches can hardly deal with the restriction that the samples synthesized from various domains can cause semantic distortion. In this paper, we propose an online one-stage \underline{C}ross \underline{C}ontrasting \underline{F}eature \underline{P}erturbation (CCFP) framework to simulate domain shift by generating perturbed features in the latent space while regularizing the model prediction against domain shift. Different from the previous fixed synthesizing strategy, we design modules with learnable feature perturbations and semantic consistency constraints. In contrast to prior work, our method does not use any generative-based models or domain labels. We conduct extensive experiments on a standard DomainBed benchmark with a strict evaluation protocol for a fair comparison. Comprehensive experiments show that our method outperforms the previous state-of-the-art, and quantitative analyses illustrate that our approach can alleviate the domain shift problem in out-of-distribution (OOD) scenarios. \href{https://github.com/hackmebroo/Cross-Contrasting-Feature-Perturbation-for-Domain-Generalization/tree/main}{https://github.com/hackmebroo/CCFP}
\end{abstract}

%%%%%%%%% BODY TEXT
\section{Introduction}
\label{sec:intro}

Deep Neural Networks have achieved remarkable success on a number of computer vision tasks\cite{lecun2015deep, zhang2023rethinking}. These models rely on the $i.i.d$ assumption\cite{vapnik1991principles}, $i.e.$, the training data and testing data are identically and independently distributed. However, in real-world scenarios, the assumption does not always hold due to the $domain\;shift$ problem\cite{ben2010theory}. For instance, it is hard for a model trained on photographs to adapt to sketches. 

Domain adaptation (DA) methods\cite{ganin2015unsupervised,sun2016deep,wilson2020survey} can be employed to handle the out-of-distribution (OOD) issue in the settings where unlabeled target data is available. Although DA can perform well on known target domains, it still fails in practical situations where target domains are not accessible during training. Domain generalization (DG) \cite{wang2022generalizing} aims to deal with such problems. The goal of domain generalization is to learn a generalized model from multiple different but related source domains ($i.e.$ diverse training datasets with the same label space) that can perform well on arbitrary unseen target domains. To realize this goal, most deep learning models are trained to minimize the average loss over the training set, which is known as the Empirical Risk Minimization (ERM) principle\cite{vapnik1999nature}. However, ERM-based network provably fails to OOD scenarios\cite{nagarajan2020understanding,eastwood2022probable, huang2022density, zhang2023aggregation}. 

% % Figure environment removed

% % Figure environment removed


One line of work\cite{sinha2018certifying,robey2021model,sagawa2019distributionally} improves the generalization capability of a model by optimizing the worst-domain risk over the set of possible domains, which are created by perturbing samples in the image level or using generative-based model ($i.e.$ VAE\cite{kingma2014auto} or GAN\cite{goodfellow2014generative}) to generate fictitious samples. Despite the performance promoted by creating samples in the image level on an offline basis to approximate the $worst$ $case$ over the entire family of domains, it is hard to generate "fictitious" samples in the input space without losing semantic discriminative information\cite{qiao2020learning}. Moreover, the offline two-stage data perturbation training procedure is nontrivial since both training a generative-based model and inferring them to obtain perturbed samples are challenging tasks. 

Another line of work perturbs features in the latent space\cite{verma2019manifold,zhou2020domain} by tuning the scaling and shifting parameters after instance normalization. Another study\cite{li2021uncertainty} extends it and leverages the uncertainty associated with feature statistics perturbation.
% without using any domain labels (which are expensive or even impossible to obtain in practice). 
However, these methods all rely on a fixed perturbation strategy (linear interpolation or random perturbation) which limits the domain transportation from synthesized features to original features. Besides, although the instance normalization-based feature perturbation can change the information of intermediate features which is specific to domains, they still fail to preserve the semantic invariant, as the instance normalization may dilute discriminative information that is relevant to task objectives\cite{nam2018batch}. The performance of feature synthesis methods can be undetermined on account of semantic inconsistency\cite{lu2022semantic}. 

As is mentioned above, the data perturbation based methods can hardly generate the fictitious samples in the input space, and the feature perturbation based methods limits the diversity of the synthesized features and fail to preserve the semantic consistency. To address both of these issues, we propose to enforce a domain-aware adaptive feature perturbation in the latent space following the worst-case optimization objective and explicitly constrain the semantic consistency to preserve the class discriminative information.

Practically, the desideratum for the worst-case DG problem is to simulate the realistic domain shift by maximizing the domain discrepancy and minimizing the class discriminative characteristics between the source domain distribution and the fictitious target domain distribution. To this end, we design an adaptive online one-stage $Cross$ $Contrasting$ $Feature$ $Perturbation$ (CCFP) framework. An illustration of CCFP is shown in Figure \ref{main}. Our CCFP consists of two sub-network, one is used to extract the original features which represent the online estimate of the source distribution, and the other is used to perturb features in the latent space to create semantic invariant fictitious target distribution. In order to preserve the class discriminative information of the perturbed features, we regularize the predictions between the two sub-networks.

A key component of our framework is the feature perturbation. As pointed out in the research field of style transfer\cite{dumoulin2016learned,huang2017arbitrary}, the feature statistics carry the information primarily referring to domain-specific but
are less relevant to class discriminative. Based on this, we design a $learnable$ $domain$ $perturbation$ (LDP) module which can generate learnable perturbation of features to enlarge the domain transportation from the original ones. Note that the LDP only adds learnable scaling and shifting parameters on feature statistics without adopting domain labels or additional generative models. 

Another critical point of CCFP is the measurement of domain discrepancy. Different from existing study measure the domain discrepancy in the last layer\cite{wang2018visual, tzeng2014deep}, we propose to measure the domain discrepancy from the intermediate features to align with the observation that the shallow layers of the network learn low-level features (such as color and edges) which are more domain aware but less semantic relevant\cite{wang2019implicit}. Additionally, Gatys et al.\cite{gatys2015texture} show that Gram matrices of latent features can be used to encode stylistic attributes like textures and patterns. Motivated by this, we  develop a novel Gram-matrices-based metric to represent the domain-specific information from the intermediate activations. We maximize the dissimilarity between the intermediate features' Gram matrices to simulate the domain shift.  

We validate the effectiveness of CCFP on a standard DG benchmark called Domainbed\cite{gulrajani2020search}. Comprehensive experiment results show that our method surpasses previous methods and achieves state-of-the-art. 


 In summary, our contributions are three-fold:
\begin{itemize}
    \item We propose a novel online one-stage cross contrasting feature perturbation framework (CCFP) for worst-case domain generalization problem, which can generate perturbed features while regularizing semantic consistency.
    \item We develop a learnable domain perturbation (LDP) module and an effective domain-aware Gram-matrices-based metric to measure domain discrepancy, which are useful for DG and integrated into the above CCFP framework. Additionally, our algorithm does not use any generative-based models and domain labels.
    \item Comprehensive experiments show that our method achieves state-of-the-art performance on diverse DG benchmarks under strict evaluation protocols of DomainBed\cite{gulrajani2020search}.
    
\end{itemize}

%-------------------------------------------------------------------------
\section{Related Work}

\textbf{Domain generalization (DG)} aims to learn generalized representations from multiple source domains that can generalize well on arbitrary unseen target domains. For example, in the PACS dataset\cite{li2017deeper}, the task is to extract category-related knowledge, and the domains correspond to different artistic styles like art-painting, cartoon, photo, and sketch. The model will use three of four datasets to train and use the rest dataset to test. Various methods have been proposed in the DG literature that can be roughly classified into three lines: learning the domain invariant representation\cite{tzeng2014deep,ghifary2015domain,muandet2013domain,motiian2017unified,wang2018learning}, meta-learning techniques\cite{sankaranarayanan2023meta,balaji2018metareg,dou2019domain,li2018learning} and data perturbation based methods\cite{volpi2018generalizing,qiao2020learning,zhou2020learning}. Our work is most relevant to the last line.

 \textbf{Data perturbation}:
 Data perturbation in the input space can create diverse images to alleviate the spurious correlations\cite{sagawa2019distributionally} and improve the model generalization. Volpi et al.\cite{volpi2018generalizing} proposed an adversarial data augmentation and learned an ensemble model for stable training. Bai et al.\cite{bai2021decaug} decomposed feature representation and semantic augmentation approach for OoD generalization.
 Qiao et al.\cite{qiao2020learning} extended it to create “fictitious” populations with large domain transportation. Zhou et al.\cite{zhou2020learning} employed a data generator to synthesize data from pseudo-novel domains to augment the source domains. Different from these methods, we propose a latent space feature perturbation instead of perturbing raw data in the input space and require no domain labels or any generative-based models.

\textbf{Feature perturbation}:
Unlike most data perturbation methods that adopt transformations in the input space, some approaches perturb features in the latent space. Li et al.\cite{li2021simple} show that even perturbing the feature embedding with Gaussian noise during training leads to a comparable performance. Manifold Mixup \cite{verma2019manifold} adopts linear interpolation from image level to feature level. Recent works show that linear interpolation on feature statistics of two instances \cite{zhou2020domain} can synthesize samples to improve model generalization. Nuriel et al.\cite{nuriel2021permuted} randomly swaps statistics of different samples from the same batch. \cite{li2021uncertainty} extends it and leverages the uncertainty associated with feature perturbations. These methods are based on a fixed perturbation strategy and lack a constraint to preserve semantics. In our work, the LDP modules can generate learnable perturbations to enlarge the domain transportation while the CCFP framework can explicitly preserve the semantic consistency. 


% \textbf{Domain discrepancy}:
% Most of the Deep neural network models suffer from performance degradation due to the domain discrepancy. Both in DA and DG, measuring the domain discrepancy is a critical step in some
%  methods\cite{wang2018visual,sun2016deep,zhou2020domain,zhu2021crossmatch}. \cite{wang2018visual} minimize the maximum mean discrepancy(MMD) to align the features between source domain and target domain. \cite{sun2016deep} minimize the domain discrepancy by aligning the second-order statistics of the source and target distributions with a linear transformation. In addition to the feature alignment based approaches, \cite{zhu2021crossmatch} maximize the Wasserstein distance between the generated domains and the source domains to simulate the domain gap. Recent study\cite{dumoulin2016learned} shows that two images are similar in style if their feature's Gram matrices are close. Inspired by this, we propose a Gram-based metric to measure discrepancy in our study.

% Figure environment removed

\section{Method}

\subsection{General Formulation}
We formulate domain generalization in a classification setting from the input features $x \in \mathcal{X}$ to the predicting labels $y \in \mathcal{Y}$. Given a model family $\Theta$ and training data drawn from some distribution. The goal is to find a model $\theta \in \Theta $ that generalizes well to unseen target distribution $P_{tar}$. DG can be formulated as the following problem:
\begin{equation}\label{1}
\min_{\theta \in \Theta}\mathbb{E}_{(x,y)\sim P_{tar}}[\ell(\theta;(x,y))]
\end{equation}
where $\mathbb{E}[\cdot]$ is the expectation, $\ell(\cdot,\cdot)$ is the loss function.

The challenge for DG is that the target domain distribution $P_{tar}$ is not available. An alternative approach to solve Eq.(\ref{1}) is to merge all the data from source domains and learn the model by minimizing the training error across the pooled data. This is known as the Empirical Risk Minimization (ERM) principle:
\begin{equation}
\hat\theta_{ERM}:=\min_{\theta \in \Theta}\mathbb{E}_{(x,y)\sim P_{src}}[\ell(\theta;(x,y))],
\end{equation}
where $P_{src}$ is the empirical distribution over the training data. Since the ERM-based methods provably lack robustness on OOD scenarios\cite{nagarajan2020understanding,eastwood2022probable}, a number of work\cite{lee2018minimax,heinze2021conditional,robey2021model,sagawa2019distributionally} formulated DG as a worst-case problem leveraging distributionally robust optimization and adversarial training:
\begin{equation}\label{2}
\hat\theta_{worst-case}:=\min_{\theta \in \Theta}\sup_{P:D(P,P_{src}) \leq \rho}\mathbb{E}_{P}[\ell(\theta;(x,y))]
\end{equation}
Here $D(\cdot ,\cdot)$ is a distance metric on the space of probability distributions. The solution to Eq.(\ref{2}) aims to achieve a good performance against the domain shifts while the fictitious target distributions $P$ are distance $\rho$ away from the source domain distribution $P_{src}$.  To solve Eq.(\ref{2}), previous work expects to create fictitious distributions $P$ by perturbing training samples in the input space or using generative-based models and updating the model with respect to these fictitious worst-case target distributions.

 
However, perturbing samples in the image level may introduce class distortions detrimental to model training which may cause the performance decline. Moreover, the offline two-stage training procedure requires significant computational resources since training the generative model and using it to obtain additional samples are both challenging tasks\cite{wang2019implicit,lu2022action}. 

In this regard, we propose an online one-stage $cross$ $contrasting$ $feature$ $perturbation$ (CCFP) framework (Sec.\ref{3.2}) to obtain perturbed representation distribution $P^l$ with learnable feature statistics (Sec.\ref{3.3}) in latent space without using any generative-based model. Further, to preserve the semantic discriminative information of the perturbed features, we utilize an explicit semantic constraint to encourage the model to predict consistent semantic representations. As it is demanded to determine the source domain distribution and the fictitious target domain distribution according to Eq.\ref{2}, we utilize a dual stream network as it is illustrated in Figure \ref{main}.

It is noteworthy that the distance metric is essential to the worst-case DG problem since it is used to measure the dissimilarity between the source domain distribution and the fictitious target domain distribution. The ideal goal of the metric is to create a fictitious target distribution with a large domain discrepancy from the source distribution as well as retain semantic discriminative information. Previous work directly boosts the dissimilarity in the high-level semantic space\cite{volpi2018generalizing} (usually the output of the last layer), thus failing to preserve the semantic discriminative information. To satisfy the goal, we propose a domain-aware Gram-matrices-based metric to boost the dissimilarity in the whole latent space except for the high-level semantic space (Sec.\ref{3.4}). Further, to better preserve the semantic discriminative information, we utilize a regularization loss to explicitly constrain the semantic consistency between the source domain and the fictitious target domain in Sec.\ref{3.5}.


\subsection{Cross Contrasting Feature Perturbation Framework (CCFP)}\label{3.2}

 Since our goal is to simulate the realistic domain shifts in the latent space. To determine the source domain distribution and create the fictitious target domain distribution, we employ two sub-networks to extract features from the same images. As illustrated in Figure \ref{main}, one is used to extract the original features which represent the online estimation of source distribution in latent space. The other is used to generate perturbed features which represent the fictitious target distribution by using our learnable domain perturbation modules in the intermediate layers.

From the practical perspective, it is intractable to select an appropriate magnitude of the domain shift $\rho$. In this regard, we consider Eq.\ref{2} as the following  Lagrangian relaxation with penalty parameter $\gamma$:
\begin{equation}\label{lagr}
\hat\theta:=\min_{\theta \in \Theta}\sup_{P^l}\{\mathbb{E}_{P^l}[\ell(\theta;(x,y))]-\gamma D(P^l, P_{src}^l)\}
\end{equation}
Here $P_{src}^l$ is the source domain distribution in the latent space, and the $P^l$ is the fictitious target distribution in the latent space.
 Taking the dual reformulation Eq.\ref{lagr}, we can obtain an $min$-$max$ optimization objective that maximizes the domain discrepancy between the source distribution and the fictitious target distribution while minimizing the target risk.

During the $min$-$max$ optimization, each iteration can be divided into two steps. For the maximization step, a batch of images will be fed into both sub-networks and are used to calculate the domain discrepancy loss, denoted by $\mathcal{L}_{dis}$ detailed in Eq.\ref{6}, only the parameters of LDP blocks will be updated at this stage. For the minimization step, the same batch of images will be fed into the model again and are used to calculate the classification loss (cross-entropy loss) and semantic consistency loss, denoted by $\mathcal{L}_{sem}$ detailed in Eq.\ref{sem}, all the parameters of two sub-networks will be updated at this stage.

\subsection{Learnable Domain Perturbation Module (LDP)}\label{3.3}
The key point of our CCFP framework is how to create domain-aware feature perturbation. As perturbing parameters for an affine transformation of intermediate features after normalization can change their characteristics which primarily refer to domain-specific information but are less relevant to category-related information\cite{dumoulin2016learned,huang2017arbitrary}, Huang et al.\cite{huang2017arbitrary} propose the adaptive instance normalization (AdaIN), which replaces the feature statistics of the input features $x$ with the feature statistics of a style image's features $x_{s}$ to achieve style transfer. Let $x \in \mathbb{R}^{B\times C\times H\times W}$ be a batch of features, the AdaIN can be formulated as:
\begin{equation}
AdaIN(x)=\sigma(x_{s}) \frac{x-\mu (x)}{\sigma (x)} + \mu(x_{s})
\end{equation}
\noindent where $\mu (x) \in \mathbb{R}^{B\times C}$ and $\sigma (x) \in \mathbb{R}^{B\times C}$ are the mean and standard deviation respectively.
However, in DG scenarios, the feature statistics of target domain images are not available. Previous work\cite{zhou2020domain, li2021uncertainty} utilizes linear interpolation or uncertainty modeling to diversify the feature statistics, but both of them limit the domain transportation from synthesized features to original features. To address this, we design a learnable domain perturbation (LDP) module (The red box in Figure \ref{main}) to generate perturbed intermediate features:
\begin{equation}\label{LDP}
LDP(x)=(\sigma(x) + \gamma) \frac{x-\mu (x)}{\sigma (x)} + \mu(x) + \beta
\end{equation}
\noindent Here we only add learnable parameters $\gamma$ and $\beta$ to the features' original scaling $\sigma(x)$ and shifting $\mu(x)$ statistics. Different from prior works based on a fixed perturbation strategy, the LDP module can enlarge the domain discrepancy between the original and the perturbed features.

\subsection{Gram-based Domain Discrepancy Metric}\label{3.4}
The worst-case optimization objective for DG is to guarantee model performance against fictitious target distribution within a certain distance from the source distribution. Considering the essential desideratum of DG that enables the model to generalize well to the unseen domain, the ideal distance metric is domain-specific and class-discriminative agnostic. Inspired by the well-known observation\cite{zhang2022contrastive} that the shallow layers learn low-level features which are task-irrelevant, we build an effective domain discrepancy metric applied to the shallow layers. Specifically, Gatys et al.\cite{gatys2015texture} shows that Gram matrices can encode stylistic attributes like textures and patterns that are less relevant to task objectives but can be used to depict the individual domain information. Therefore, we utilize the Gram-matrices-based metric to depict the domain discrepancy.

Specifically, we denote the network as the following:
\begin{equation}
c(x) = g \circ f^n \circ f^{n-1} \circ \cdots \circ f^1(x)
\end{equation}
Here $g$ is the classifier, $f=f^n \circ f^{n-1} \circ \cdots \circ f^1(x)$ is the feature extractor, and $n$ denotes the number of shallow layers. In our method, we use a set of Gram matrices $\{G^1, G^2, \cdots, G^K \}$ from a set of shallow layers $\{f^1, \cdots, f^K\}$ in the network to describe the domain-specific characteristics. The domain discrepancy loss can be formulated as:
\begin{equation}\label{6}
\mathcal{L}_{dis} = -\sum_{i=1}^K||G(f_o^i(\mathbf{x})) - G(f_p^i(\mathbf{x}))||_F 
\end{equation}


\noindent where $f_o$ and $f_p$ are two feature extractors (original and perturbed) respectively. $K$ is the number of shallow layers to calculate the loss $\mathcal{L}_{dis}$, $G(\cdot)$ is the Gram matrix, and the $||\cdot||_F$ denotes the Frobenius norm.

\subsection{Explicit Semantic Consistency Constraint}\label{3.5}
To better preserve the semantic consistency, we minimize the L2-norm between the final classifier predictions of two sub-networks. The semantic consistency loss can be formulated as:
\begin{equation}\label{sem}
L_{sem} = ||g_o(f_o(\mathbf{x})) - g_p(f_p(\mathbf{x}))||_2^2
\end{equation}
Here $g_o$ and $g_p$ are two classifiers (original and perturbed respectively). The final loss is given by:
\begin{equation}\label{eq_final}
\mathcal{L}_{final} = \mathcal{L}_{cls_1} + \mathcal{L}_{cls_2} + \lambda_{dis} \mathcal{L}_{dis} + \lambda_{sem} \mathcal{L}_{sem}
\end{equation}
\noindent The $\lambda_{dis}$ and the $\lambda_{sem}$ are used to control the strength of the domain discrepancy loss $\mathcal{L}_{dis}$ and the semantic consistency loss $\mathcal{L}_{sem}$.

The optimization algorithm is designed in Algorithm \ref{alg1}.
During the inference, we only use the sub-network (the top network in the Figure \ref{main}) which is learned from the perturbed features to predict the final results, as the diverse latent features help mitigate the domain shift. Additionally, the LDP modules will also be used to prevent variations in the normalization statistics that could otherwise cause model collapse. Although the statistics shift can be alleviated by randomly applying LDP during training, the LDP modules can also be used as a test-time augmentation technology to boost the performance, we will discuss it in the Appendix.

\begin{algorithm}[t]\label{alg1}
	\renewcommand{\algorithmicrequire}{\textbf{Input:}}
	\renewcommand{\algorithmicensure}{\textbf{Output:}}
        \setstretch{1.1}
	\caption{\textbf{:} Cross Contrasting Feature Perturbation}
	% \label{alg1}
	\begin{algorithmic}
		\STATE \textbf{Input:}$\mathcal{S}_{train}=\{( \mathbf{x}_i,y_i)\}^{n}_{i=1}$, batch size $B$, learning rate $\eta$, Adam optimizer, initial $\lambda_{dis}, \lambda_{sem}$
		\STATE \textbf{Initial:} Parameters of CCFP $i.e.$ parameters $\theta_0$, $\theta_1$, $\phi_0$, $\phi_1$, $(\gamma_k, \beta_k; k=1 \cdots K)$ for feature extractor $f_o$, $f_p$, classifier $g_o, g_p$ and LDP modules $P^1, P^2 \cdots P^K$ (K is defined in Eq.\ref{6}).
		\REPEAT
            \STATE \textbf{Minimization Stage:}
            \FOR{$i=1,\cdots,B$}{
                \STATE $\mathcal{L}_{cls_1}^i=\ell(g_o(f_o(\mathbf{x}_i)),y_i)$
                \STATE $\mathcal{L}_{cls_2}^i=\ell(g_p(f_p(\mathbf{x}_i)),y_i)$
                \STATE $\mathcal{L}_{sem}^i=\lambda_{sem} ||f_o(\mathbf{x}_i)-f_p(\mathbf{x}_i)||^2_2$
            }\ENDFOR
            \begin{small}
            \STATE $\theta_0,\phi_0 \leftarrow $ Adam$(\frac{1}{B}\sum_{i=1}^{B}\mathcal{L}_{cls_1}^i+\mathcal{L}_{sem}^i,\theta_0,\phi_0,\eta)$
            \end{small}
            \begin{small}
            % \begin{aligned}
            % \STATE 
            $\theta_1,\phi_1,\gamma_k,\beta_k \leftarrow $ Adam$(\frac{1}{B}$
            $\sum_{i=1}^B\mathcal{L}_{cls_2}^i+\mathcal{L}_{sem}^i,\theta_1,\phi_1,$ \\
            \hspace{3.2cm}$\gamma_k,\beta_k,\eta$)
            % \end{aligned}
            \end{small}
            \STATE \textbf{Maximization Stage:}
            \FOR{$i=1,\cdots,B$}{
                \STATE\hspace{-0.2cm}$\mathcal{L}_{dis}^i=\lambda_{dis} \sum_{k=1}^K||G(f_o^k(\mathbf{x}_i)) - G(f_p^k(P^k(\mathbf{x}_i)))||_F$
            }\ENDFOR
            \STATE $\gamma_k,\beta_k \leftarrow$ Adam$(\frac{1}{B}\sum_{i=1}^B \mathcal{L}_{spe}^j,\gamma_k,\beta_k,\eta)$
		\UNTIL $\theta_0$, $\theta_1$, $\phi_0$, $\phi_1$ are converged
	\end{algorithmic}  
\end{algorithm}

\begin{table*}[t]
  \centering
  \begin{tabular}{c | c c c c c c c | c}
    \toprule
    \textbf{Algorithm} & \textbf{CMNIST} & \textbf{RMNIST} & \textbf{VLCS} & \textbf{PACS} & \textbf{OfficeHome} & \textbf{TerraInc} & \textbf{DomainNet} & \textbf{Avg} \\
    \midrule
    ERM\cite{vapnik1991principles} & 51.5 \footnotesize$\pm$ 0.1 & 98.0 \footnotesize$\pm$ 0.0 & 77.5 \footnotesize$\pm$ 0.4 & 85.5 \footnotesize$\pm$ 0.2 & 66.5 \footnotesize$\pm$ 0.3 & 46.1 \footnotesize$\pm$ 1.8 & 40.9 \footnotesize$\pm$ 0.1 & 66.6\\
    IRM\cite{arjovsky2020invariant} & 52.0 \footnotesize$\pm$ 0.1 & 97.7 \footnotesize$\pm$ 0.1 & 78.5 \footnotesize$\pm$ 0.5 & 83.5 \footnotesize$\pm$ 0.8 & 64.3 \footnotesize$\pm$ 2.2 & 47.6 \footnotesize$\pm$ 0.8 & 33.9 \footnotesize$\pm$ 2.8 & 65.4\\
    GroupDRO\cite{sagawa2019distributionally} & 52.1 \footnotesize$\pm$ 0.0 & 98.0 \footnotesize$\pm$ 0.0 & 76.7 \footnotesize$\pm$ 0.6 & 84.4 \footnotesize$\pm$ 0.8 & 66.0 \footnotesize$\pm$ 0.7 & 43.2 \footnotesize$\pm$ 1.1 & 33.3 \footnotesize$\pm$ 0.2 & 64.8 \\
    Mixup\cite{yan2020improve} & 52.1 \footnotesize$\pm$ 0.2 & 98.0 \footnotesize$\pm$ 0.1 & 77.4 \footnotesize$\pm$ 0.6 & 84.6 \footnotesize$\pm$ 0.6 & 68.1 \footnotesize$\pm$ 0.3 & 47.9 \footnotesize$\pm$ 0.8 & 39.2 \footnotesize$\pm$ 0.1 & 66.7\\
    MLDG\cite{li2018learning}& 51.5 \footnotesize$\pm$ 0.1 & 97.9 \footnotesize$\pm$ 0.0 & 77.2 \footnotesize$\pm$ 0.4 & 84.9 \footnotesize$\pm$ 1.0 & 66.8 \footnotesize$\pm$ 0.6 & 47.7 \footnotesize$\pm$ 0.9 & 41.2 \footnotesize$\pm$ 0.1 & 66.7\\
    CORAL\cite{sun2016deep} & 51.5 \footnotesize$\pm$ 0.1 & 98.0 \footnotesize$\pm$ 0.1 & 78.8 \footnotesize$\pm$ 0.6 & 86.2 \footnotesize$\pm$ 0.3 & 68.7 \footnotesize$\pm$ 0.3 & 47.6 \footnotesize$\pm$ 1.0 & 41.5 \footnotesize$\pm$ 0.1 & 67.5\\
    MMD\cite{li2018domain} & 51.5 \footnotesize$\pm$ 0.2 & 97.9 \footnotesize$\pm$ 0.0 & 77.5 \footnotesize$\pm$ 0.9 & 84.6 \footnotesize$\pm$ 0.5 & 66.3 \footnotesize$\pm$ 0.1 & 42.2 \footnotesize$\pm$ 1.6 & 23.4 \footnotesize$\pm$ 9.5 & 63.3\\
    DANN\cite{ganin2016domain} & 51.5 \footnotesize$\pm$ 0.2 & 97.8 \footnotesize$\pm$ 0.1 & 78.6 \footnotesize$\pm$ 0.4 & 83.6 \footnotesize$\pm$ 0.4 & 65.9 \footnotesize$\pm$ 0.6 & 46.7 \footnotesize$\pm$ 0.5 & 38.3 \footnotesize$\pm$ 0.1 & 66.1\\
    CDANN\cite{li2018domain} & 51.7 \footnotesize$\pm$ 0.1 & 97.9 \footnotesize$\pm$ 0.1 & 77.5 \footnotesize$\pm$ 0.1 & 82.6 \footnotesize$\pm$ 0.9 & 65.8 \footnotesize$\pm$ 1.3 & 45.8 \footnotesize$\pm$ 1.6 & 38.3 \footnotesize$\pm$ 0.3 & 65.6\\
    MTL\cite{blanchard2021domain} & 51.4 \footnotesize$\pm$ 0.1 & 97.9 \footnotesize$\pm$ 0.0 & 77.2 \footnotesize$\pm$ 0.4 & 84.6 \footnotesize$\pm$ 0.5 & 66.4 \footnotesize$\pm$ 0.5 & 45.6 \footnotesize$\pm$ 1.2 & 40.6 \footnotesize$\pm$ 0.1 & 66.2\\
    SagNet\cite{nam2021reducing} & 51.7 \footnotesize$\pm$ 0.0 & 98.0 \footnotesize$\pm$ 0.0 & 77.8 \footnotesize$\pm$ 0.5 & 86.3 \footnotesize$\pm$ 0.2 & 68.1 \footnotesize$\pm$ 0.1 & \textbf{48.6} \footnotesize$\pm$ 1.0 & 40.3 \footnotesize$\pm$ 0.1 & 67.2\\
    ARM\cite{zhang2020adaptive} & \textbf{56.2} \footnotesize$\pm$ 0.2 & \textbf{98.2} \footnotesize$\pm$ 0.1 & 77.6 \footnotesize$\pm$ 0.3 & 85.1 \footnotesize$\pm$ 0.4 & 64.8 \footnotesize$\pm$ 0.3 & 45.5 \footnotesize$\pm$ 0.3 & 35.5 \footnotesize$\pm$ 0.2 & 66.1\\
    V-REx\cite{krueger2021out} & 51.8 \footnotesize$\pm$ 0.1 & 97.9 \footnotesize$\pm$ 0.1 & 78.3 \footnotesize$\pm$ 0.2 & 84.9 \footnotesize$\pm$ 0.6 & 66.4 \footnotesize$\pm$ 0.6 & 46.4 \footnotesize$\pm$ 0.6 & 33.6 \footnotesize$\pm$ 2.9 & 65.6\\
    RSC\cite{huang2020self} & 51.7 \footnotesize$\pm$ 0.2 & 97.6 \footnotesize$\pm$ 0.1 & 77.1 \footnotesize$\pm$ 0.5 & 85.2 \footnotesize$\pm$ 0.9 & 65.5 \footnotesize$\pm$ 0.9 & 46.6 \footnotesize$\pm$ 1.0 & 38.9 \footnotesize$\pm$ 0.5 & 66.1\\
    \midrule
    AND-mask\cite{kim2021selfreg} & 51.3 \footnotesize$\pm$ 0.2 & 97.6 \footnotesize$\pm$ 0.1 & 78.1 \footnotesize$\pm$ 0.9 & 84.4 \footnotesize$\pm$ 0.9 & 65.6 \footnotesize$\pm$ 0.4 & 44.6 \footnotesize$\pm$ 0.3 & 37.2 \footnotesize$\pm$ 0.6 & 65.5\\
    SAND-mask\cite{kim2021selfreg} & 51.8 \footnotesize$\pm$ 0.2 & 97.4 \footnotesize$\pm$ 0.1 & 77.4 \footnotesize$\pm$ 0.2 & 84.6 \footnotesize$\pm$ 0.9 & 65.8 \footnotesize$\pm$ 0.4 & 42.9 \footnotesize$\pm$ 1.7 & 32.1 \footnotesize$\pm$ 0.6 & 64.6\\
    Fish\cite{shi2021gradient} & 51.6 \footnotesize$\pm$ 0.1 & 98.0 \footnotesize$\pm$ 0.0 & 77.8 \footnotesize$\pm$ 0.3 & 85.5 \footnotesize$\pm$ 0.3 & 68.6 \footnotesize$\pm$ 0.4 & 45.1 \footnotesize$\pm$ 1.3 & \textbf{42.7} \footnotesize$\pm$ 0.2 & 67.1\\
    Fishr\cite{rame2022fishr} & 52.0 \footnotesize$\pm$ 0.2 & 97.8 \footnotesize$\pm$ 0.0 & 77.8 \footnotesize$\pm$ 0.1 & 85.5 \footnotesize$\pm$ 0.4 & 67.8 \footnotesize$\pm$ 0.1 & 47.4 \footnotesize$\pm$ 1.6 & 41.7 \footnotesize$\pm$ 0.0 & 67.1\\
    \midrule
    CCFP (ours) & 51.9 \footnotesize$\pm$ 0.1 & 97.8 \footnotesize$\pm$ 0.1 & \textbf{78.9} \footnotesize$\pm$ 0.3 & \textbf{86.6} \footnotesize$\pm$ 0.2 & \textbf{68.9} \footnotesize$\pm$ 0.1 & \textbf{48.6} \footnotesize$\pm$ 0.4 & 41.2 \footnotesize$\pm$ 0.0 & \textbf{67.7} \\
    \bottomrule
  \end{tabular}
  \caption{DomainBed with Training-domain model selection. We highlighted the best results using \textbf{bold} font.}
  \label{result}
\end{table*}

% \begin{table*}[t]
%   \centering
%   \begin{tabular}{c | c c c c c c c | c}
%     \toprule
%     \textbf{Algorithm} & \textbf{CMNIST} & \textbf{RMNIST} & \textbf{VLCS} & \textbf{PACS} & \textbf{OfficeHome} & \textbf{TerraInc} & \textbf{DomainNet} & \textbf{Avg} \\
%     \midrule
%     ERM\cite{vapnik1991principles} & 51.5 \footnotesize$\pm$ 0.1 & 98.0 \footnotesize$\pm$ 0.0 & 77.5 \footnotesize$\pm$ 0.4 & 85.5 \footnotesize$\pm$ 0.2 & 66.5 \footnotesize$\pm$ 0.3 & 46.1 \footnotesize$\pm$ 1.8 & 40.9 \footnotesize$\pm$ 0.1 & 66.6\\
%     IRM\cite{arjovsky2020invariant} & \underline{52.0} \footnotesize$\pm$ 0.1 & 97.7 \footnotesize$\pm$ 0.1 & 78.5 \footnotesize$\pm$ 0.5 & 83.5 \footnotesize$\pm$ 0.8 & 64.3 \footnotesize$\pm$ 2.2 & 47.6 \footnotesize$\pm$ 0.8 & 33.9 \footnotesize$\pm$ 2.8 & 65.4\\
%     GroupDRO\cite{sagawa2019distributionally} & 61.1 \footnotesize$\pm$ 0.9 & 97.9 \footnotesize$\pm$ 0.1 & 77.4 \footnotesize$\pm$ 0.5 & 87.1 \footnotesize$\pm$ 0.1 & 66.2 \footnotesize$\pm$ 0.6 & 52.4 \footnotesize$\pm$ 0.1 & 33.4 \footnotesize$\pm$ 0.3 &67.9 \\
%     Mixup\cite{yan2020improve} & 58.4 \footnotesize$\pm$ 0.2 & 98.0 \footnotesize$\pm$ 0.1 & 78.1 \footnotesize$\pm$ 0.3 & 86.8 \footnotesize$\pm$ 0.3 & 68.0 \footnotesize$\pm$ 0.2 & \textbf{54.4} \footnotesize$\pm$ 0.3 & 39.6 \footnotesize$\pm$ 0.1 & 69.0\\
%     MLDG\cite{li2018learning}& 58.2 \footnotesize$\pm$ 0.4 & 97.8 \footnotesize$\pm$ 0.1 & 77.5 \footnotesize$\pm$ 0.1 & 86.8 \footnotesize$\pm$ 0.4 & 66.6 \footnotesize$\pm$ 0.3 & 52.0 \footnotesize$\pm$ 0.1 & 41.6 \footnotesize$\pm$ 0.1 & 68.7\\
%     CORAL\cite{sun2016deep} & 58.6 \footnotesize$\pm$ 0.5 & 98.0 \footnotesize$\pm$ 0.0 & 77.7 \footnotesize$\pm$ 0.2 & 87.1 \footnotesize$\pm$ 0.5 & 68.4 \footnotesize$\pm$ 0.2 & 52.8 \footnotesize$\pm$ 0.2 & 41.8 \footnotesize$\pm$ 0.1 & 69.2\\
%     MMD\cite{li2018domain} & 63.3 \footnotesize$\pm$ 1.3 & 98.0 \footnotesize$\pm$ 0.1 & 77.9 \footnotesize$\pm$ 0.1 & 87.2 \footnotesize$\pm$ 0.1 & 66.2 \footnotesize$\pm$ 0.3 & 52.0 \footnotesize$\pm$ 0.4 & 23.5 \footnotesize$\pm$ 9.4 & 66.9\\
%     DANN\cite{ganin2016domain} & 57.0 \footnotesize$\pm$ 1.0 & 97.9 \footnotesize$\pm$ 0.1 & 79.7 \footnotesize$\pm$ 0.5 & 85.2 \footnotesize$\pm$ 0.2 & 65.3 \footnotesize$\pm$ 0.8 & 50.6 \footnotesize$\pm$ 0.4 & 38.3 \footnotesize$\pm$ 0.1 & 67.7\\
%     CDANN\cite{li2018domain} & 59.5 \footnotesize$\pm$ 2.0 & 97.9 \footnotesize$\pm$ 0.0 & 79.9 \footnotesize$\pm$ 0.2 & 85.8 \footnotesize$\pm$ 0.8 & 65.3 \footnotesize$\pm$ 0.5 & 50.8 \footnotesize$\pm$ 0.6 & 38.5 \footnotesize$\pm$ 0.2 & 68.2\\
%     MTL\cite{blanchard2021domain} & 57.6 \footnotesize$\pm$ 0.3 & 97.9 \footnotesize$\pm$ 0.1 & 77.7 \footnotesize$\pm$ 0.5 & 86.7 \footnotesize$\pm$ 0.2 & 66.5 \footnotesize$\pm$ 0.4 & 52.2 \footnotesize$\pm$ 0.4 & 40.8 \footnotesize$\pm$ 0.1 & 68.5\\
%     SagNet\cite{nam2021reducing} & 58.2 \footnotesize$\pm$ 0.3 & 97.9 \footnotesize$\pm$ 0.0 & 77.6 \footnotesize$\pm$ 0.1 & 86.4 \footnotesize$\pm$ 0.4 & 67.5 \footnotesize$\pm$ 0.2 & 52.5 \footnotesize$\pm$ 0.4 & 40.8 \footnotesize$\pm$ 0.2 & 68.7\\
%     ARM\cite{zhang2020adaptive} & 63.2 \footnotesize$\pm$ 0.7 & 98.1 \footnotesize$\pm$ 0.1 & 77.8 \footnotesize$\pm$ 0.3 & 85.8 \footnotesize$\pm$ 0.2 & 64.8 \footnotesize$\pm$ 0.4 & 51.2 \footnotesize$\pm$ 0.5 & 36.0 \footnotesize$\pm$ 0.2 & 68.1\\
%     V-REx\cite{krueger2021out} & 67.0 \footnotesize$\pm$ 1.3 & 97.9 \footnotesize$\pm$ 0.1 & 78.1 \footnotesize$\pm$ 0.2 & 87.2 \footnotesize$\pm$ 0.6 & 65.7 \footnotesize$\pm$ 0.3 & 51.4 \footnotesize$\pm$ 0.5 & 30.1 \footnotesize$\pm$ 3.7 & 68.2\\
%     RSC\cite{huang2020self} & 58.5 \footnotesize$\pm$ 0.5 & 97.6 \footnotesize$\pm$ 0.1 & 77.8 \footnotesize$\pm$ 0.6 & 86.2 \footnotesize$\pm$ 0.5 & 66.5 \footnotesize$\pm$ 0.6 & 52.1 \footnotesize$\pm$ 0.2 & 38.9 \footnotesize$\pm$ 0.6 & 68.2\\
%     \midrule
%     SelfReg\cite{kim2021selfreg} & 51.6 \footnotesize$\pm$ 0.2 & 98.0 \footnotesize$\pm$ 0.1 & 77.5 \footnotesize$\pm$ 0.0 & 86.5 \footnotesize$\pm$ 0.3 & \underline{69.4} \footnotesize$\pm$ 0.2 & 51.0 \footnotesize$\pm$ 0.4 & \underline{44.6} \footnotesize$\pm$ 0.1 & 68.4\\
%     Fish\cite{shi2021gradient} & 61.8 \footnotesize$\pm$ 0.8 & 97.9 \footnotesize$\pm$ 0.1 & 77.8 \footnotesize$\pm$ 0.6 & 85.8 \footnotesize$\pm$ 0.6 & 66.0 \footnotesize$\pm$ 2.9 & 50.8 \footnotesize$\pm$ 0.4 & 43.4 \footnotesize$\pm$ 0.3 & 69.1\\
%     Fishr\cite{rame2022fishr} & \textbf{68.8} \footnotesize$\pm$ 1.4 & 97.8 \footnotesize$\pm$ 0.1 & 78.2 \footnotesize$\pm$ 0.2 & 86.9 \footnotesize$\pm$ 0.2 & 68.2 \footnotesize$\pm$ 0.2 & \underline{53.6} \footnotesize$\pm$ 0.4 & 41.8 \footnotesize$\pm$ 0.2 & \underline{70.8}\\
%     DDG\cite{zhang2022towards} &  - & \textbf{98.4} & \underline{80.0} & \underline{88.9} & - & - & - & -\\
%     \midrule
%     CCFP (ours) & 65.1 \footnotesize$\pm$ 0.3 & \underline{98.3} \footnotesize$\pm$ 0.0 & \textbf{80.3} \footnotesize$\pm$ 0.1 & \textbf{89.0} \footnotesize$\pm$ 0.3 & \textbf{69.7} \footnotesize$\pm$ 0.3 & \textbf{54.4} \footnotesize$\pm$ 0.1 & \textbf{44.8} \footnotesize$\pm$ 0.1 & \textbf{71.7} \\
%     \bottomrule
%   \end{tabular}
%   \caption{DomainBed with oracle model selection. We format \textbf{first} and \underline{second} results.}
%   \label{result}
% \end{table*}

\section{Experiments}

\subsection{DomainBed Benchmark}\label{4.1}
We conduct comprehensive experiments on the DomainBed benchmark\cite{gulrajani2020search}. DomainBed includes seven multi-domain image classification tasks: Colored MNIST\cite{arjovsky2020invariant}, Rotated MNIST\cite{ghifary2015domain}, PACS\cite{li2017deeper}, VLCS\cite{fang2013unbiased}, Office-Home\cite{venkateswara2017deep}, Terra Incognita\cite{beery2018recognition}, and DomainNet\cite{peng2019moment}.

\textbf{Colored MNIST}\cite{arjovsky2020invariant} is a variant of MNIST consisting of 70,000 examples of dimension (2, 28, 28) and 2 classes. The dataset contains a disjoint set of colored digits where domain $d$ $\in$ \{90\%, 80\%, 10\%\} is the correlation strength between color and label across domains. \textbf{Rotated MNIST}\cite{ghifary2015domain} is a variant of MNIST consisting of 70,000 examples of dimension (1, 28, 28) and 10 classes. The dataset contains digits rotated by $d$ degrees where domain $d$ $\in$ \{0, 15, 30, 45, 60, 75\}. \textbf{PACS}\cite{li2017deeper} includes domains $d$ $\in$ \{art, cartoons, photos, sketches\} with 9,991 examples of dimension (3, 224, 224) and 7 classes. \textbf{VLCS}\cite{fang2013unbiased} includes domains $d$ $\in$\{Caltech101, LabelMe, SUN09, VOC2007\} with 10,729 examples of dimension (3, 224, 224) and 5 classes. \textbf{Office-Home}\cite{venkateswara2017deep} includes domains $d$ $\in$ \{atr, clipart, product, real\} with 15,588 examples of dimension (3, 224, 224) and 65 classes. \textbf{Terra Incognita}\cite{beery2018recognition} contains photographs of wild animals taken by camera traps at locations $d$ $\in$ \{L100, L38, L43, L46\} with 24,788 examples of dimension (3, 224, 224) and 10 classes. \textbf{DomainNet}\cite{peng2019moment} includes domains $d$ $\in$ \{clipart, infograph, painting, quickdraw, real, sketch\} with 586,575 examples of dimension (3, 224, 224) and 345 classes.

For a fair comparison, the DomainBed benchmark\cite{gulrajani2020search} presents an evaluation protocol about dataset splits, model selection on the validation set, and hyperparameter (HP) search, which is detailed below.


\textbf{Dataset splits.}
% The data from each domain is split into training subsets (80\%) and validation subsets (20\%). The larger subsets will be used as training and final evaluation, and the smaller subsets will be used to select hyperparameters. We ran entire experiments three times with different train-validation splits. Then we report the average results and their estimated standard error.
The data from source domains are split into training subsets (80\%) and validation subsets (20\%) (used on Training-domain validation set model selection). The data from the target domain are split into testing subsets (80\%) and validation subsets (20\%) (used on Test-domain validation set model selection). We repeat the entire experiment three times using different seeds and report the mean and standard error over all the repetitions.

\textbf{Model selection methods.}
There are three model selection methods in \cite{gulrajani2020search}. ($\romannumeral1$) Training-domain validation set. ($\romannumeral2$) Leave-one-out cross-validation. ($\romannumeral3$) Test-domain validation set (oracle). 
% Our work is built on SWAD\cite{cha2021swad} (stochastic weight averaging densely method). For a fair comparison, we follow the same model selection method (Training-domain validation set) as SWAD that 
We choose the Training-domain model selection that assumes the training and test examples follow similar distributions. The best-performing model in the validation set is selected as the final model, and its test domain performance is reported as the final performance. The results of the oracle model selection are shown in Appendix.

\textbf{Model architectures.}
Following DomainBed, we use Conv-Net (detail in Appendix D.1 in \cite{gulrajani2020search}) as the backbone for Colored MNIST and Rotated MNIST and use ResNet-50\cite{he2016deep} for the rest datasets. For the classifier, we only use one linear layer. We insert the LDP modules after the first Conv, Max Pooling, and 1,2,3-th ConvBlock, and we further perform an ablation study for the effects of different inserted positions. When using Conv-Net as our backbone, we insert the LDP modules at the position after the first three Batch Normalization layers. 


\textbf{Hyperparameter (HP) search.} We run a random search of 20 trials over the hyperparameter distribution given by DomainBed. Our CCFP relies on two additional hyperparameters $\lambda_{spe}$ and $\lambda_{sem}$, and we set the range of search such as $[0.1, 10]$ for both of them, more details about the range of hyperparameter search will be discussed in Appendix.
% \begin{table}
%   \centering
%   \begin{tabular}{c c c c}
%     \toprule
%     Dataset & Paras & Default & Choice \\
%     \midrule
%     PACS/VLCS/ & lr & 5e-5 & 5e-5\\
%     OfficeHome/ & bs & 32 & 32\\
%     TerraIncognita/ & wd & 0 & 0\\
%     DomainNet & drop & 0 & 0\\
%     \midrule
%     CMNIST & lr & 1e-3 & 10Uniform(−4.5,−3.5)10^{Uniform(-4.5,-3.5)}\\
%     RMNIST & bs & 64 & 2Uniform(3,9)2^{Uniform(3,9)}\\
%     & wd & 0 & 0\\
%     \bottomrule
%   \end{tabular}
%   \caption{Hyperparameter search space}
%   \label{table1}
% \end{table}

\textbf{Implementation details.}
We implement our algorithm using the codebase of DomainBed in PyTorch, using ResNet-50 pre-trained on the ImageNet\cite{deng2009imagenet} and fine-tuning on each dataset. Note that our evaluation setting follows the standard evaluation protocol given by DomainBed\cite{gulrajani2020search}. 


\subsection{Results}

\textbf{Comparison with domain generalization methods on Domainbed benchmark.}
Comprehensive experiments show that CCFP achieves significant performance gain against previous methods on most of the benchmark datasets and obtains comparable performance on three of seven datasets. Table \ref{result} summarizes the results on DomainBed using the Training-domain model selection method. Our CCFP outperforms all previous approaches on the averaged result. 
% It is worth noting that our method is one of the few (as with Fishr) that can outperform the baselines method, $i.e.$, ERM approach, on all benchmark datasets. 


To further validate the generalization of CCFP, we conduct experiments under another commonly used baseline SWAD\cite{cha2021swad} as our backbone, which is a unique model selection mechanism. For a fair comparison, we only summarize the methods based on SWAD. The performance comparison with other existing approaches that adopted SWAD is provided in Tables \ref{OfficeHome}-\ref{Terra}. Our CCFP achieves significant performance gain in all experiments against previous best results.

\begin{table}[ht]
  \centering
  \begin{tabular}{c c c c c c}
    \toprule
    \textbf{Algorithm} & \textbf{A} & \textbf{C} & \textbf{P} & \textbf{R} & \textbf{Avg.} \\
    \midrule
    SWAD\cite{cha2021swad} & 66.1 & 57.7 & 78.4 & 80.2 & 70.6\\
    PCL\cite{yao2022pcl} & 67.3 & \textbf{59.9} & 78.7 & 80.7 & 71.6 \\
    \midrule
    CCFP (ours) & \textbf{68.0} & 58.6 & \textbf{79.7} & \textbf{81.9} & \textbf{72.1}\\
    \bottomrule
  \end{tabular}
  \caption{Comparison with SWAD-based state-of-the-art methods on OfficeHome benchmark. A: art, C: clipart, P: product, R: real, Avg.: average.}
  \label{OfficeHome}
\end{table}

\begin{table}[ht]
  \centering
  \begin{tabular}{c c c c c c}
    \toprule
    \textbf{Algorithm} & \textbf{C} & \textbf{L} & \textbf{S} & \textbf{V} & \textbf{Avg.} \\
    \midrule
    SWAD\cite{cha2021swad} & 98.8 & 63.3 & \textbf{75.3} & 79.2 & 79.1\\
    PCL\cite{yao2022pcl} & \textbf{99.0} & 63.6 & 73.8 & 75.6 & 78.0 \\
    \midrule
    CCFP (ours) & 98.9 & \textbf{64.1} & 74.9 & \textbf{79.9} & \textbf{79.4}\\
    \bottomrule
  \end{tabular}
  \caption{Comparison with SWAD-based state-of-the-art methods on VLCS benchmark. C: Caltech101, L: LabelMe, S: SUN09, V: VOC2007, Avg.: average.}
  \label{VLCS}
\end{table}


\begin{table}[ht]
  \centering
  \begin{tabular}{c c c c c c}
    \toprule
    \textbf{Algorithm} & \textbf{L100} & \textbf{L38} & \textbf{L43} & \textbf{L46} & \textbf{Avg.} \\
    \midrule
    SWAD\cite{cha2021swad} & 55.4 & 44.9 & 59.7 & 39.9 & 50.0\\
    PCL\cite{yao2022pcl} & 58.7 & 46.3 & 60.0 & 43.6 & 52.1 \\
    \midrule
    CCFP (ours) & \textbf{59.9} & \textbf{47.6} & \textbf{60.8} & \textbf{43.8} & \textbf{53.0}\\
    \bottomrule
  \end{tabular}
  \caption{Comparison with SWAD-based state-of-the-art methods on TerraIncognita benchmark. L100: Location 100, L38: Location 38, L43: Location 43, L46: Location 46, Avg.: average.}
  \label{Terra}
\end{table}



\textbf{Comparison with previous feature perturbation methods.}
To reveal the performance gain by using learnable parameters to perturb feature statistics, we conduct experiments to compare with two previous features perturbation methods Mixstyle\cite{zhou2020domain} and DSU\cite{li2021uncertainty}. Since both of them use their own experiment settings. For a fair comparison, we rerun their results on the DomainBed experiment benchmark. Table \ref{DSU} shows that our CCFP achieves a substantial improvement in performance compared to the previous feature perturbation methods more experiment results are shown in Appendix.

% To compare with previous we conduct experiments to compare with previous feature perturbation methods.
% DSU\cite{li2021uncertainty} is one of the state-of-the-art DG methods using non-parametric random perturbation on feature statistics. The comparison includes two parts: (\romannumeral1\romannumeral1) reproducing DSU on a single ResNet-50. (\romannumeral2\romannumeral2) replacing our LDP modules with DSU in which loss Lcls1\mathcal{L}_{cls_1}, Lcls2\mathcal{L}_{cls2}, Lsem\mathcal{L}_{sem} in Eq. \ref{eq_final} are retained and domain specific loss Lspe\mathcal{L}_{spe} in Eq. =\ref{eq_final} is removed. Since DSU is a non-parametric method, so we can not maximize the domain discrepancy between the perturbed features and original features. For a fair comparison, we only use the predicted results of the single network equipped with DSU. 
\begin{table}[ht]
  \centering
  \begin{tabular}{c c c c c c}
    \toprule
    \textbf{Algorithm} & \textbf{A} & \textbf{C} & \textbf{P} & \textbf{S} & \textbf{Avg.} \\
    \midrule
    ERM & 81.6 & 78.7 & 95.5 & 78.7 & 83.6\\
    Mixstyle\cite{zhou2020domain} & 84.0 & 79.9 & 94.3 & \textbf{81.6} & 84.9 \\
    DSU\cite{li2021uncertainty} & 81.9 & 79.6 & 95.0 & 79.6 & 84.1 \\
    \midrule
    CCFP (ours) & \textbf{87.5} & \textbf{81.3} & \textbf{96.4} & 81.4 & \textbf{86.6}\\
    \bottomrule
  \end{tabular}
  \caption{Comparison with previous feature perturbation methods on PACS benchmark. Comparison with SWAD-based state-of-the-art methods on PACS benchmark.}
  \label{DSU}
\end{table}



\section{Ablation Study}
\textbf{Effects of the explicit semantic regularization.} To validate the effectiveness of the semantic regularization, we conduct experiments without using the semantic consistency loss in Eq.\ref{eq_final}. Table \ref{semantic} shows that semantic regularization can achieve performance gain on most target domains and the average accuracy. In particular, we can find that without semantic regularization, our method can still significantly outperform ERM.

\begin{table}[ht]
  \centering
  \begin{tabular}{c c c c c c}
    \toprule
    \textbf{Algorithm} & \textbf{A} & \textbf{C} & \textbf{P} & \textbf{S} & \textbf{Avg.} \\
    \midrule
    ERM & 81.6 & 78.7 & 95.5 & 78.7 & 83.6\\
    CCFP(w/o) $L_{sem}$ & 83.6 & \textbf{83.9} & \textbf{96.4} & 80.3 & 86.0 \\
    \midrule
    CCFP (ours) & \textbf{87.5} & 81.3 & \textbf{96.4} & \textbf{81.4} & \textbf{86.6}\\
    \bottomrule
  \end{tabular}
  \caption{Comparison with result without using $L_{sem}$ on PACS benchmark.}
  \label{semantic}
\end{table}


% Figure environment removed
% \vspace{-1cm}
% Figure environment removed

To further validate the essential to regularize the semantic consistency after perturbing features in the latent space, we enforce the semantic regularization on previous feature perturbation methods. Note that both Mixstyle and DSU use one single network to generate the perturbed features, which is unable to calculate the $L_{sem}$ in Eq.\ref{eq_final}. To address this, we implement the two methods in our CCFP framework. Similar to our approach, we use one sub-network to extract the original features and use the other sub-network to generate the perturbed features by using Mixstyle and DSU feature perturbation methods. Further, we constrain the consistency between the predictions of the two sub-networks. During the inference, we only use the perturbed sub-network to produce the final predictions which are the same as our approach. Since Mixstyle and DSU are non-parametric, we remove the $L_{dis}$ in Eq.\ref{eq_final} in this experiment. Table \ref{arch} shows that our dual stream architecture and the explicit semantic consistency regularization can achieve a significant performance gain (0.3$\%$ for Mixstyle and 1.4$\%$ for DSU).

\begin{table}[ht]
  \centering
  \begin{tabular}{c c c c c c}
    \toprule
    \textbf{Algorithm} & \textbf{A} & \textbf{C} & \textbf{P} & \textbf{S} & \textbf{Avg.} \\
    \midrule
    Mixstyle\cite{zhou2020domain} & 84.0 & 79.9 & 94.3 & 81.6 & 84.9 \\
    Mixstyle (dual) & 84.6 & 80.3 & \textbf{96.5} & 79.5 & 85.2 \\
    DSU\cite{li2021uncertainty} & 81.9 & 79.6 & 95.0 & 79.6 & 84.1 \\
    DSU (dual) & 86.3 & 79.4 & 94.6 & \textbf{81.7} & 85.5 \\
    \midrule
    CCFP (ours) & \textbf{87.5} & \textbf{81.3} & 96.4 & 81.4 & \textbf{86.6}\\
    \bottomrule
  \end{tabular}
  \caption{Validation of the additional semantic consistency for previous feature perturbation methods on PACS benchmark.}
  \label{arch}
\end{table}



\noindent \textbf{Effects of LDP inserted positions.}
In CCFP, we use a set of Gram matrices of intermediate features from a set of layers $\{f^1, \cdots, f^K\}$ to describe the domain-specific characteristics\cite{zhang2022refining, zhang2022characterizing}. To verify the effects of LDP inserted positions, we name the position of ResNet after the first Conv, Max Pooling, and 1,2,3-th ConvBlock as 1,2,3,4,5 respectively, and the effects of LDP on different inserted positions are evaluated accordingly. We conduct the experiments on dataset PACS and OfficeHome with the default hyperparameters given by DomainBed. Table \ref{abla1} shows that more inserted LDP modules can produce relatively higher classification accuracy. Hence we plug the LDP modules into all five positions for our main experiments. 
\begin{table}[ht]
  \centering
  \begin{tabular}{c c c c c |c}
    \toprule
    \textbf{Positions} & \textbf{1-3} & \textbf{2-4} & \textbf{3-5} & \textbf{1-5} & \textbf{ERM}\\
    \midrule
    PACS & 85.3 & 84.8 & 85.4 & \textbf{86.6} & 83.6\\
    OfficeHome & 68.4 & 68.5 & 68.3 & \textbf{68.9}& 64.5\\
    \bottomrule
  \end{tabular}
  \caption{Effects of different inserted positions on PACS and OfficeHome benchmark.}
  \label{abla1}
\end{table}



\noindent \textbf{Visualization analysis on CCFP.} 
To confirm that our CCFP can alleviate the domain shift phenomena, we conduct experiments on the PACS dataset where we choose art painting as the target domain and the rest as the source domain. We capture the intermediate features at position 4 to study the feature statistic shifts. Figure \ref{shift1} shows the feature statistics distribution from source domains and target domains based on ERM, Mixstyle, DSU, and our CCFP. For a fair comparison, we reproduce the results of ERM, Mixstyle, DSU, and CCFP with the same fixed steps (5,000 steps, which is the same as the default value given by DomainBed) and only consider the final checkpoint. It is shown that our CCFP can obviously mitigate the domain shift between the source and target domain features compared with ERM and surpass Mixstyle and DSU. The result shows that our method can help against the domain shift.

\section{Conclusions}
In this paper, we propose a simple yet efficient cross contrasting feature perturbation framework. Unlike previous works, our method does not use generative-based models or domain labels. Our approach can adaptively generate perturbed features with large domain transportation from the original features while preserving semantic consistency, and encourage the model to predict consistent semantic representation against the domain shift. The experiments show that our method performs better than the previous state-of-the-art on the DomainBed benchmark.

\section{Acknowledgments}
This work is supported by National Key Research and Development Program of China (2021YFF1200800)
%%%%%%%%% ABSTRACT
% \begin{abstract}
%    The ABSTRACT is to be in fully-justified italicized text, at the top
%    of the left-hand column, below the author and affiliation
%    information. Use the word ``Abstract'' as the title, in 12-point
%    Times, boldface type, centered relative to the column, initially
%    capitalized. The abstract is to be in 10-point, single-spaced type.
%    Leave two blank lines after the Abstract, then begin the main text.
%    Look at previous ICCV abstracts to get a feel for style and length.
% \end{abstract}

% %%%%%%%%% BODY TEXT
% \section{Introduction}

% Please follow the steps outlined below when submitting your manuscript to
% the IEEE Computer Society Press.  This style guide now has several
% important modifications (for example, you are no longer warned against the
% use of sticky tape to attach your artwork to the paper), so all authors
% should read this new version.

% %-------------------------------------------------------------------------
% \subsection{Language}

% All manuscripts must be in English.

% \subsection{Dual submission}

% Please refer to the author guidelines on the ICCV 2023 web page for a
% discussion of the policy on dual submissions.

% \subsection{Paper length}
% Papers, excluding the references section,
% must be no longer than eight pages in length. The references section
% will not be included in the page count, and there is no limit on the
% length of the references section. For example, a paper of eight pages
% with two pages of references would have a total length of 10 pages.
% {\bf There will be no extra page charges for ICCV 2023.}

% Overlength papers will simply not be reviewed.  This includes papers
% where the margins and formatting are deemed to have been significantly
% altered from those laid down by this style guide.  Note that this
% \LaTeX\ guide already sets figure captions and references in a smaller font.
% The reason such papers will not be reviewed is that there is no provision for
% supervised revisions of manuscripts.  The reviewing process cannot determine
% the suitability of the paper for presentation in eight pages if it is
% reviewed in eleven.  

% %-------------------------------------------------------------------------
% \subsection{The ruler}
% The \LaTeX\ style defines a printed ruler which should be present in the
% version submitted for review.  The ruler is provided in order that
% reviewers may comment on particular lines in the paper without
% circumlocution.  If you are preparing a document using a non-\LaTeX\
% document preparation system, please arrange for an equivalent ruler to
% appear on the final output pages.  The presence or absence of the ruler
% should not change the appearance of any other content on the page.  The
% camera-ready copy should not contain a ruler. (\LaTeX\ users may uncomment
% the \verb'\iccvfinalcopy' command in the document preamble.)  Reviewers:
% note that the ruler measurements do not align well with the lines in the paper
% --- this turns out to be very difficult to do well when the paper contains
% many figures and equations, and, when done, looks ugly.  Just use fractional
% references (e.g.\ this line is $095.5$), although in most cases one would
% expect that the approximate location will be adequate.

% \subsection{Mathematics}

% Please number all of your sections and displayed equations.  It is
% important for readers to be able to refer to any particular equation.  Just
% because you didn't refer to it in the text doesn't mean some future readers
% might not need to refer to it.  It is cumbersome to have to use
% circumlocutions like ``the equation second from the top of page 3 column
% 1''.  (Note that the ruler will not be present in the final copy, so is not
% an alternative to equation numbers).  All authors will benefit from reading
% Mermin's description of how to write mathematics:
% \url{http://www.pamitc.org/documents/mermin.pdf}.

% \subsection{Blind review}

% Many authors misunderstand the concept of anonymizing for blind
% review.  Blind review does not mean that one must remove
% citations to one's own work --- in fact, it is often impossible to
% review a paper unless the previous citations are known and
% available.

% Blind review means that you do not use the words ``my'' or ``our''
% when citing previous work.  That is all.  (But see below for
% tech reports.)

% Saying ``this builds on the work of Lucy Smith [1]'' does not say
% that you are Lucy Smith; it says that you are building on her
% work.  If you are Smith and Jones, do not say ``as we show in
% [7]'', say ``as Smith and Jones show in [7]'' and at the end of the
% paper, include reference 7 as you would any other cited work.

% An example of a bad paper just asking to be rejected:
% \begin{quote}
% \begin{center}
%     An analysis of the frobnicatable foo filter.
% \end{center}

%    In this paper, we present a performance analysis of our
%    previous paper [1] and show it to be inferior to all
%    previously known methods.  Why the previous paper was
%    accepted without this analysis is beyond me.

%    [1] Removed for blind review
% \end{quote}

% An example of an acceptable paper:

% \begin{quote}
% \begin{center}
%      An analysis of the frobnicatable foo filter.
% \end{center}

%    In this paper, we present a performance analysis of the
%    paper of Smith \etal [1] and show it to be inferior to
%    all previously known methods.  Why the previous paper
%    was accepted without this analysis is beyond me.

%    [1] Smith, L and Jones, C. ``The frobnicatable foo
%    filter, a fundamental contribution to human knowledge''.
%    Nature 381(12), 1-213.
% \end{quote}

% If you are making a submission to another conference at the same time,
% which covers similar or overlapping material, you may need to refer to that
% submission in order to explain the differences, just as you would if you
% had previously published related work.  In such cases, include the
% anonymized parallel submission~\cite{Authors14} as additional material and
% cite it as
% \begin{quote}
% [1] Authors. ``The frobnicatable foo filter'', F\&G 2014 Submission ID 324,
% Supplied as additional material {\tt fg324.pdf}.
% \end{quote}

% Finally, you may feel you need to tell the reader that more details can be
% found elsewhere, and refer them to a technical report.  For conference
% submissions, the paper must stand on its own, and not {\em require} the
% reviewer to go to a tech report for further details.  Thus, you may say in
% the body of the paper ``further details may be found
% in~\cite{Authors14b}''.  Then submit the tech report as additional material.
% Again, you may not assume the reviewers will read this material.

% Sometimes your paper is about a problem that you tested using a tool that
% is widely known to be restricted to a single institution.  For example,
% let's say it's 1969, you have solved a key problem on the Apollo lander,
% and you believe that the ICCV70 audience would like to hear about your
% solution.  The work is a development of your celebrated 1968 paper entitled
% ``Zero-g frobnication: How being the only people in the world with access to
% the Apollo lander source code makes us a wow at parties'', by Zeus \etal.

% You can handle this paper like any other.  Don't write ``We show how to
% improve our previous work [Anonymous, 1968].  This time we tested the
% algorithm on a lunar lander [name of lander removed for blind review]''.
% That would be silly, and would immediately identify the authors. Instead,
% write the following:
% \begin{quotation}
% \noindent
%    We describe a system for zero-g frobnication.  This
%    system is new because it handles the following cases:
%    A, B.  Previous systems [Zeus et al. 1968] didn't
%    handle case B properly.  Ours handles it by including
%    a foo term in the bar integral.

%    ...

%    The proposed system was integrated with the Apollo
%    lunar lander, and went all the way to the moon, don't
%    you know.  It displayed the following behaviors
%    which shows how well we solved cases A and B: ...
% \end{quotation}
% As you can see, the above text follows standard scientific conventions,
% reads better than the first version and does not explicitly name you as
% the authors.  A reviewer might think it likely that the new paper was
% written by Zeus \etal, but cannot make any decision based on that guess.
% He or she would have to be sure that no other authors could have been
% contracted to solve problem B.
% \medskip

% \noindent
% FAQ\medskip\\
% {\bf Q:} Are acknowledgements OK?\\
% {\bf A:} No.  Leave them for the final copy.\medskip\\
% {\bf Q:} How do I cite my results reported in open challenges?
% {\bf A:} To conform with the double-blind review policy, you can report the results of other challenge participants together with your results in your paper. For your results, however, you should not identify yourself and should not mention your participation in the challenge. Instead, present your results referring to the method proposed in your paper and draw conclusions based on the experimental comparison to other results.\medskip\\

% % Figure environment removed

% \subsection{Miscellaneous}

% \noindent
% Compare the following:\\
% \begin{tabular}{ll}
%  \verb'$conf_a$' &  $conf_a$ \\
%  \verb'$\mathit{conf}_a$' & $\mathit{conf}_a$
% \end{tabular}\\
% See The \TeX book, p165.

% The space after \eg, meaning ``for example'', should not be a
% sentence-ending space. So \eg is correct, {\em e.g.} is not.  The provided
% \verb'\eg' macro takes care of this.

% When citing a multi-author paper, you may save space by using ``et alia'',
% shortened to ``\etal'' (not ``{\em et.\ al.}'' as ``{\em et}'' is a complete word.)
% However, use it only when there are three or more authors.  Thus, the
% following is correct: ``
%    Frobnication has been trendy lately.
%    It was introduced by Alpher~\cite{Alpher02}, and subsequently developed by
%    Alpher and Fotheringham-Smythe~\cite{Alpher03}, and Alpher \etal~\cite{Alpher04}.''

% This is incorrect: ``... subsequently developed by Alpher \etal~\cite{Alpher03} ...''
% because reference~\cite{Alpher03} has just two authors.  If you use the
% \verb'\etal' macro provided, then you need not worry about double periods
% when used at the end of a sentence as in Alpher \etal.

% For this citation style, keep multiple citations in numerical (not
% chronological) order, so prefer \cite{Alpher03,Alpher02,Authors14} to
% \cite{Alpher02,Alpher03,Authors14}.

% % Figure environment removed

% %------------------------------------------------------------------------
% \section{Formatting your paper}

% All text must be in a two-column format. The total allowable width of the
% text area is $6\frac78$ inches (17.5 cm) wide by $8\frac78$ inches (22.54
% cm) high. Columns are to be $3\frac14$ inches (8.25 cm) wide, with a
% $\frac{5}{16}$ inch (0.8 cm) space between them. The main title (on the
% first page) should begin 1.0 inch (2.54 cm) from the top edge of the
% page. The second and following pages should begin 1.0 inch (2.54 cm) from
% the top edge. On all pages, the bottom margin should be 1-1/8 inches (2.86
% cm) from the bottom edge of the page for $8.5 \times 11$-inch paper; for A4
% paper, approximately 1-5/8 inches (4.13 cm) from the bottom edge of the
% page.

% %-------------------------------------------------------------------------
% \subsection{Margins and page numbering}

% All printed material, including text, illustrations, and charts, must be kept
% within a print area 6-7/8 inches (17.5 cm) wide by 8-7/8 inches (22.54 cm)
% high.

% Page numbers should be included for review submissions but not for the 
% final paper. Review submissions papers should have page numbers in the 
% footer with numbers centered and .75 inches (1.905 cm) from the bottom 
% of the page and start on the first page with the number 1.

% Page numbers will be added by the publisher to all camera-ready papers 
% prior to including them in the proceedings and before submitting the 
% papers to IEEE Xplore. As such, your camera-ready submission should 
% not include any page numbers. Page numbers should automatically be 
% removed by uncommenting (if it's not already) the line
% \begin{verbatim}
% % \iccvfinalcopy
% \end{verbatim}
% near the beginning of the .tex file.

% %-------------------------------------------------------------------------
% \subsection{Type-style and fonts}

% Wherever Times is specified, Times Roman may also be used. If neither is
% available on your word processor, please use the font closest in
% appearance to Times to which you have access.

% MAIN TITLE. Center the title 1-3/8 inches (3.49 cm) from the top edge of
% the first page. The title should be in Times 14-point, boldface type.
% Capitalize the first letter of nouns, pronouns, verbs, adjectives, and
% adverbs; do not capitalize articles, coordinate conjunctions, or
% prepositions (unless the title begins with such a word). Leave two blank
% lines after the title.

% AUTHOR NAME(s) and AFFILIATION(s) are to be centered beneath the title
% and printed in Times 12-point, non-boldface type. This information is to
% be followed by two blank lines.

% The ABSTRACT and MAIN TEXT are to be in a two-column format.

% MAIN TEXT. Type main text in 10-point Times, single-spaced. Do NOT use
% double-spacing. All paragraphs should be indented 1 pica (approx. 1/6
% inch or 0.422 cm). Make sure your text is fully justified---that is,
% flush left and flush right. Please do not place any additional blank
% lines between paragraphs.

% Figure and table captions should be 9-point Roman type as in
% Figures~\ref{fig:onecol} and~\ref{fig:short}.  Short captions should be centered.

% \noindent Callouts should be 9-point Helvetica, non-boldface type.
% Initially capitalize only the first word of section titles and first-,
% second-, and third-order headings.

% FIRST-ORDER HEADINGS. (For example, {\large \bf 1. Introduction})
% should be Times 12-point boldface, initially capitalized, flush left,
% with one blank line before, and one blank line after.

% SECOND-ORDER HEADINGS. (For example, { \bf 1.1. Database elements})
% should be Times 11-point boldface, initially capitalized, flush left,
% with one blank line before, and one after. If you require a third-order
% heading (we discourage it), use 10-point Times, boldface, initially
% capitalized, flush left, preceded by one blank line, followed by a period
% and your text on the same line.

% %-------------------------------------------------------------------------
% \subsection{Footnotes}

% Please use footnotes\footnote {This is what a footnote looks like.  It
% often distracts the reader from the main flow of the argument.} sparingly.
% Indeed, try to avoid footnotes altogether and include necessary peripheral
% observations in
% the text (within parentheses, if you prefer, as in this sentence).  If you
% wish to use a footnote, place it at the bottom of the column on the page on
% which it is referenced. Use Times 8-point type, single-spaced.

% %-------------------------------------------------------------------------
% \subsection{References}

% List and number all bibliographical references in 9-point Times,
% single-spaced, at the end of your paper. When referenced in the text,
% enclose the citation number in square brackets, for
% example~\cite{Authors14}.  Where appropriate, include the name(s) of
% editors of referenced books.

% \begin{table}
% \begin{center}
% \begin{tabular}{|l|c|}
% \hline
% Method & Frobnability \\
% \hline\hline
% Theirs & Frumpy \\
% Yours & Frobbly \\
% Ours & Makes one's heart Frob\\
% \hline
% \end{tabular}
% \end{center}
% \caption{Results.   Ours is better.}
% \end{table}

% %-------------------------------------------------------------------------
% \subsection{Illustrations, graphs, and photographs}

% All graphics should be centered.  Please ensure that any point you wish to
% make is resolvable in a printed copy of the paper.  Resize fonts in figures
% to match the font in the body text, and choose line widths that render
% effectively in print.  Many readers (and reviewers), even of an electronic
% copy, will choose to print your paper in order to read it.  You cannot
% insist that they do otherwise, and therefore must not assume that they can
% zoom in to see tiny details on a graphic.

% When placing figures in \LaTeX, it's almost always best to use
% \verb+\includegraphics+, and to specify the  figure width as a multiple of
% the line width as in the example below
% {\small\begin{verbatim}
%    \usepackage[dvips]{graphicx} ...
%    \includegraphics[width=0.8\linewidth]
%                    {myfile.eps}
% \end{verbatim}
% }

% %-------------------------------------------------------------------------
% \subsection{Color}

% Please refer to the author guidelines on the ICCV 2023 web page for a discussion
% of the use of color in your document.

% %------------------------------------------------------------------------
% \section{Final copy}

% You must include your signed IEEE copyright release form when you submit
% your finished paper. We MUST have this form before your paper can be
% published in the proceedings.

{\small
\bibliographystyle{ieee_fullname}
\bibliography{egbib}
}

\end{document}
