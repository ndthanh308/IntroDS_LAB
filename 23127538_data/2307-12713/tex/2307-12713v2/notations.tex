\usepackage[utf8]{inputenc}  
\usepackage[english]{babel}                             
\usepackage{syntax}
\usepackage{tikz}
\usepackage{soul,hyperref}
\usepackage{xcolor,xspace}
\usepackage[framemethod=TikZ]{mdframed}
\usetikzlibrary{petri,positioning,calc,shapes,fit}
\usetikzlibrary {automata} 
\usepackage{listings}
\usepackage[inline]{enumitem}
\usepackage{amsfonts}
\usepackage{amsmath}


\newtheorem{definition}{\textbf{Definition}}
\newtheorem{property}{\textbf{Property}}
\newtheorem{example}{\textbf{Example}}
\newtheorem{remark}{\textbf{Remark}}
\newtheorem{semantic}{\textbf{Semantics}}
\newtheorem{translation}{\textbf{Translation}}


\mdfdefinestyle{MyFrame}{%
     outerlinewidth=-1pt,
    %roundcorner=20pt,
    innertopmargin=5pt,
    innerbottommargin=5pt,
    innerrightmargin=10pt,
    innerleftmargin=10pt,
        leftmargin = -1pt,
        rightmargin = -1pt
    %backgroundcolor=gray!50!white}
        }



\definecolor{navyblue}{rgb}{0.0, 0.0, 0.5}

\newcounter{stxcnt}
\newenvironment{mysyntax}[1][]{%
    \refstepcounter{stxcnt}%
    \mdfsetup{%
    frametitle={%
        \tikz[baseline=(current bounding box.east),outer sep=0pt]
        \node[anchor=east,rectangle,rounded corners,fill=navyblue, text=white]
        {\normalsize Syntax\thestxcnt \hspace{1cm}#1};},
    innertopmargin=-1pt,linecolor=navyblue,backgroundcolor=navyblue!10,%
    linewidth=2pt,topline=true,roundcorner=10pt,%
    frametitleaboveskip=\dimexpr-\ht\strutbox\relax%
    }
    \vspace{0.08cm}
    \begin{mdframed}[style=MyFrame,nobreak=true]\relax\footnotesize}{
    \end{mdframed}
    \vspace{-0.3cm}
}



%\pagestyle{headings}
\newcommand{\Fn}{\ensuremath{F_\mathcal{N}}\xspace}
\newcommand{\petri}{Petri\xspace}
\newcommand{\scade}{{\sc scade}\xspace}
\newcommand{\nnef}{{\sc nnef}\xspace}
\newcommand{\xtratum}{{\sc xtratum}\xspace}
\newcommand{\reluplex}{{\sc reluplex}\xspace}
\newcommand{\onnx}{{\sc onnx}\xspace}
\newcommand{\cuda}{{\sc cuda}\xspace}
\newcommand{\cudnn}{{\sc cudnn}\xspace}
\newcommand{\lenet}{{\sc LeNet-5}\xspace}
\newcommand{\khronos}{{\sc Khronos}\xspace}
\newcommand{\arm}{{\sc arm}\xspace}
\newcommand{\gpu}{{\sc gpu}\xspace}
\newcommand{\gpus}{{\sc gpu}s\xspace}
\newcommand{\fpga}{{\sc fpga}\xspace}
\newcommand{\nvidia}{{\sc nvidia}\xspace}
\newcommand{\ultrascale}{{\sc UltraScale+}\xspace}
\newcommand{\xavier}{{\sc Xavier}\xspace}
\newcommand{\pytorch}{{\sc PyTorch}\xspace}
\newcommand{\python}{{\sc Python}\xspace}
\newcommand{\tensorflow}{{\sc TensorFlow}\xspace}
\newcommand{\tensorrt}{{\sc TensorRT}\xspace}
\newcommand{\posix}{{\sc posix}\xspace}
\newcommand{\keras}{{\sc Keras}\xspace}
\newcommand{\tvm}{{\sc tvm}\xspace}
\newcommand*{\equal}{=}

\usetikzlibrary{arrows.meta}


\lstset{breaklines=true,
                    % numbers=left,
                    % numberstyle=\tiny,
                    % ytopmargin=1cm,
                    frame=single,
                    captionpos=b,
                    % xleftmargin=\parindent,
                    basicstyle=\footnotesize\ttfamily,
                    morekeywords={external, conv, max_pool, concat, reshape, linear, variable, variablesync, get_sync, send_sync, tick, relu, flatten, gemm, softmax},
                    keywordstyle=\color{blue}
}

\newcommand*{\medcap}{\mathbin{\scalebox{1.5}{\ensuremath{\cap}}}}
\newcommand*{\medcup}{\mathbin{\scalebox{1.5}{\ensuremath{\cup}}}}