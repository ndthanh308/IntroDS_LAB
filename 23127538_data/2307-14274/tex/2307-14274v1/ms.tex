\documentclass[twocolumn]{aastex631}
\pdfoutput=1
%\documentclass[linenumbers]{aastex631}
\usepackage{lineno}
\submitjournal{The Astronomical Journal}
%\linenumbers
\newcommand{\vdag}{(v)^\dagger}
\newcommand\aastex{AAS\TeX}
\newcommand\latex{La\TeX}

\received{December 2, 2022}
\revised{July 5, 2023}
\accepted{July 18, 2023}
%\accepted{\today}

\shorttitle{OGLE-2019-BLG-0825}
\shortauthors{Satoh et al. (2023)}

\begin{document}
\title{OGLE-2019-BLG-0825: Constraints on the Source System and Effect on Binary-lens Parameters arising from a Five Day Xallarap Effect in a Candidate Planetary Microlensing Event}
%\title{OGLE-2019-BLG-0825: A Five Days Xallarap Effect Affects Binary-lens Parameters in A Candidate of Planetary Microlensing Event}

\author[0000-0002-1228-4122]{Yuki K. Satoh}
\correspondingauthor{Yuki K. Satoh}
\email{satohyk@iral.ess.sci.osaka-u.ac.jp}
\altaffiliation{The MOA Collaboration}
\affiliation{Department of Earth and Space Science, Graduate School of Science, Osaka University, 1-1 Machikaneyama-cho, Toyonaka, Osaka 560-0043, Japan}
\affiliation{College of Science and Engineering, Kanto Gakuin University, 1-50-1 Mutsuurahigashi, Kanazawa-ku, Yokohama, Kanagawa 236-8501, Japan}

\author[0000-0003-2302-9562]{Naoki Koshimoto}
\altaffiliation{The MOA Collaboration}
%\affiliation{Department of Astronomy, Graduate School of Science, The University of Tokyo, 7-3-1 Hongo, Bunkyo-ku, Tokyo 113-0033, Japan}
\affiliation{Department of Earth and Space Science, Graduate School of Science, Osaka University, 1-1 Machikaneyama-cho, Toyonaka, Osaka 560-0043, Japan}

\author[0000-0001-8043-8413]{David P. Bennett}
\altaffiliation{The MOA Collaboration}
\affiliation{Code 667, NASA Goddard Space Flight Center, Greenbelt, MD 20771, USA}
\affiliation{Department of Astronomy, University of Maryland, College Park, MD 20742, USA}

\author[0000-0002-4035-5012]{Takahiro Sumi}
\altaffiliation{The MOA Collaboration}
\affiliation{Department of Earth and Space Science, Graduate School of Science, Osaka University, 1-1 Machikaneyama-cho, Toyonaka, Osaka 560-0043, Japan}

\author[0000-0001-5069-319X]{Nicholas J. Rattenbury}
\altaffiliation{The MOA Collaboration}
\affiliation{Department of Physics, University of Auckland, Private Bag 92019, Auckland, New Zealand}

\author[0000-0002-5843-9433]{Daisuke Suzuki}
\altaffiliation{The MOA Collaboration}
\affiliation{Department of Earth and Space Science, Graduate School of Science, Osaka University, 1-1 Machikaneyama-cho, Toyonaka, Osaka 560-0043, Japan}

\author[0000-0001-9818-1513]{Shota Miyazaki}
\altaffiliation{The MOA Collaboration}
\affiliation{Institute of Space and Astronautical Science, Japan Aerospace Exploration Agency, Kanagawa 252-5210, Japan}
%\affiliation{Department of Earth and Space Science, Graduate School of Science, Osaka University, 1-1 Machikaneyama-cho, Toyonaka, Osaka 560-0043, Japan}

\author{Ian A. Bond}
\altaffiliation{The MOA Collaboration}
\affiliation{Institute of Natural and Mathematical Sciences, Massey University, Auckland 0745, New Zealand}

\author[0000-0001-5207-5619]{Andrzej Udalski}
\altaffiliation{The OGLE Collaboration}
%\affiliation{Warsaw University Observatory, Al. Ujazdowskie 4, 00-478 Warszawa, Poland}
\affiliation{Astronomical Observatory, University of Warsaw, Al. Ujazdowskie 4, 00-478 Warszawa, Poland}

\author{Andrew Gould}
\altaffiliation{The KMTNet Collaboration}
\affiliation{Max Planck Institute for Astronomy, K\"{o}nigstuhl 17, D-69117, Heidelberg, Germany}
\affiliation{Department of Astronomy, Ohio State University, 140 W. 18th Avenue, Columbus, OH 43210, USA}

\author[0000-0003-4590-0136]{Valerio Bozza}
\altaffiliation{The MiNDSTEp Collaboration}
\affiliation{Dipartimento di Fisica “E.R. Canianiello,” Università di Salerno, Via Giovanni Paolo II 132, I-84084 Fisciano, Italy}
\affiliation{Istituto Nazionale di Fisica Nucleare, Sezione di Napoli, Napoli, Italy}

\author[0000-0002-3202-0343]{Martin Dominik}
\altaffiliation{The MiNDSTEp Collaboration}
%\affiliation{Centre for Exoplanet Science, SUPA, School of Physics $\&$ Astronomy, University of St Andrews, North Haugh, St Andrews KY16 9SS, UK}
%\affiliation{University of St Andrews, Centre for Exoplanet Science, SUPA School of Physics & Astronomy, North Haugh, St Andrews, KY16 9SS, United Kingdom}
\affiliation{University of St Andrews, Centre for Exoplanet Science, SUPA School of Physics \& Astronomy, North Haugh, St Andrews, KY16 9SS, UK}

\author[0000-0003-4776-8618]{Yuki Hirao}
\altaffiliation{The MOA Collaboration}
\affiliation{Institute of Astronomy, Graduate School of Science, The University of Tokyo, 2-21-1 Osawa, Mitaka, Tokyo 181-0015, Japan}
%\affiliation{Department of Earth and Space Science, Graduate School of Science, Osaka University, 1-1 Machikaneyama-cho, Toyonaka, Osaka 560-0043, Japan}

\author[0000-0002-3401-1029]{Iona Kondo}
\altaffiliation{The MOA Collaboration}
\affiliation{Department of Earth and Space Science, Graduate School of Science, Osaka University, 1-1 Machikaneyama-cho, Toyonaka, Osaka 560-0043, Japan}

\author{Rintaro Kirikawa}
\altaffiliation{The MOA Collaboration}
\affiliation{Department of Earth and Space Science, Graduate School of Science, Osaka University, 1-1 Machikaneyama-cho, Toyonaka, Osaka 560-0043, Japan}

\author{Ryusei Hamada}
\affiliation{Department of Earth and Space Science, Graduate School of Science, Osaka University, 1-1 Machikaneyama-cho, Toyonaka, Osaka 560-0043, Japan}

\collaboration{16}{(Leading Authors)}

\author{Fumio Abe}
\affiliation{Institute for Space-Earth Environmental Research, Nagoya University, Nagoya 464-8601, Japan}

%\author{Ken Bando}
%\affiliation{Department of Earth and Space Science, Graduate School of Science, Osaka University, 1-1 Machikaneyama-cho, Toyonaka, Osaka 560-0043, Japan}
%
\author[0000-0003-4916-0892]{Richard Barry}
%\author[0000-0003-4916-0892]{Richard K. Barry}
\affiliation{Code 667, NASA Goddard Space Flight Center, Greenbelt, MD 20771, USA}

\author{Aparna Bhattacharya}
\affiliation{Code 667, NASA Goddard Space Flight Center, Greenbelt, MD 20771, USA}
\affiliation{Department of Astronomy, University of Maryland, College Park, MD 20742, USA}

\author{Hirosane Fujii}
\affiliation{Department of Earth and Space Science, Graduate School of Science, Osaka University, 1-1 Machikaneyama-cho, Toyonaka, Osaka 560-0043, Japan}

\author[0000-0002-4909-5763]{Akihiko Fukui}
\affiliation{Department of Earth and Planetary Science, Graduate School of Science, The University of Tokyo, 7-3-1 Hongo, Bunkyo-ku, Tokyo 113-0033, Japan}
\affiliation{Instituto de Astrofísica de Canarias, Vía Láctea s/n, E-38205 La Laguna, Tenerife, Spain}

\author{Katsuki Fujita}
\affiliation{Department of Earth and Space Science, Graduate School of Science, Osaka University, 1-1 Machikaneyama-cho, Toyonaka, Osaka 560-0043, Japan}
%


%\author{Shunya Hamada}
%\affiliation{Department of Earth and Space Science, Graduate School of Science, Osaka University, 1-1 Machikaneyama-cho, Toyonaka, Osaka 560-0043, Japan}
%
%\author{Naoto Hamasaki}
%\affiliation{Department of Earth and Space Science, Graduate School of Science, Osaka University, 1-1 Machikaneyama-cho, Toyonaka, Osaka 560-0043, Japan}
%
\author{Tomoya Ikeno}
\affiliation{Department of Earth and Space Science, Graduate School of Science, Osaka University, 1-1 Machikaneyama-cho, Toyonaka, Osaka 560-0043, Japan}

%\author{Stela Ishitani Silva}
%\author[0000-0003-2267-1246]{Stela Ishitani Silva}
\author[0000-0003-2267-1246]{Stela {Ishitani~Silva}}
\affiliation{Code 667, NASA Goddard Space Flight Center, Greenbelt, MD 20771, USA}
\affiliation{Department of Physics, The Catholic University of America, Washington, DC 20064, USA}

\author[0000-0002-8198-1968]{Yoshitaka Itow}
\affiliation{Institute for Space-Earth Environmental Research, Nagoya University, Nagoya 464-8601, Japan}

%\author{Xinru Li}
%\affiliation{Department of Earth and Space Science, Graduate School of Science, Osaka University, 1-1 Machikaneyama-cho, Toyonaka, Osaka 560-0043, Japan}
%
\author{Yutaka Matsubara}
\affiliation{Institute for Space-Earth Environmental Research, Nagoya University, Nagoya 464-8601, Japan}

\author{Sho Matsumoto}
\affiliation{Department of Earth and Space Science, Graduate School of Science, Osaka University, 1-1 Machikaneyama-cho, Toyonaka, Osaka 560-0043, Japan}

\author[0000-0003-1978-2092]{Yasushi Muraki}
\affiliation{Institute for Space-Earth Environmental Research, Nagoya University, Nagoya 464-8601, Japan}

\author{Kosuke Niwa}
\affiliation{Department of Earth and Space Science, Graduate School of Science, Osaka University, 1-1 Machikaneyama-cho, Toyonaka, Osaka 560-0043, Japan}

\author{Arisa Okamura}
\affiliation{Department of Earth and Space Science, Graduate School of Science, Osaka University, 1-1 Machikaneyama-cho, Toyonaka, Osaka 560-0043, Japan}

\author[0000-0001-8472-2219]{Greg Olmschenk}
\affiliation{Code 667, NASA Goddard Space Flight Center, Greenbelt, MD 20771, USA}

\author[0000-0003-2388-4534]{Cl$\acute{\rm e}$ment Ranc}
%\affiliation{Sorbonne Universit$\acute{\rm e}$, CNRS, UMR 7095, Institut d’Astrophysique de Paris, 98 bis bd Arago, F-75014 Paris, France}
\affiliation{Sorbonne Universit\'e, CNRS, UMR 7095, Institut d'Astrophysique de Paris, 98 bis bd Arago, 75014 Paris, France}

\author{Taiga Toda}
\affiliation{Department of Earth and Space Science, Graduate School of Science, Osaka University, 1-1 Machikaneyama-cho, Toyonaka, Osaka 560-0043, Japan}

\author{Mio Tomoyoshi}
\affiliation{Department of Earth and Space Science, Graduate School of Science, Osaka University, 1-1 Machikaneyama-cho, Toyonaka, Osaka 560-0043, Japan}

\author{Paul J. Tristram}
\affiliation{University of Canterbury Mt. John Observatory, P.O. Box 56, Lake Tekapo 8770, New Zealand}

\author[0000-0002-9881-4760]{Aikaterini Vandorou}
\affiliation{Code 667, NASA Goddard Space Flight Center, Greenbelt, MD 20771, USA}
\affiliation{Department of Astronomy, University of Maryland, College Park, MD 20742, USA}

\author{Hibiki Yama}
\affiliation{Department of Earth and Space Science, Graduate School of Science, Osaka University, 1-1 Machikaneyama-cho, Toyonaka, Osaka 560-0043, Japan}

\author{Kansuke Yamashita}
\affiliation{Department of Earth and Space Science, Graduate School of Science, Osaka University, 1-1 Machikaneyama-cho, Toyonaka, Osaka 560-0043, Japan}

%\author{Atsunori Yonehara}
%\affiliation{Department of Physics, Faculty of Science, Kyoto Sangyo University, Kyoto 603-8555, Japan}
%
\collaboration{22}{(The MOA Collaboration)}

\author[0000-0001-7016-1692]{Przemek Mr$\acute{\rm o}$z}
\affiliation{Astronomical Observatory, University of Warsaw, Al. Ujazdowskie 4, 00-478 Warszawa, Poland}

\author[0000-0002-9245-6368]{Radosław Poleski}
\affiliation{Astronomical Observatory, University of Warsaw, Al. Ujazdowskie 4, 00-478 Warszawa, Poland}

\author[0000-0002-2335-1730]{Jan Skowron}
\affiliation{Astronomical Observatory, University of Warsaw, Al. Ujazdowskie 4, 00-478 Warszawa, Poland}

\author[0000-0002-0548-8995]{Michał K. Szyma$\acute{\rm n}$ski}
\affiliation{Astronomical Observatory, University of Warsaw, Al. Ujazdowskie 4, 00-478 Warszawa, Poland}

\author{Radek Poleski}
\affiliation{Astronomical Observatory, University of Warsaw, Al. Ujazdowskie 4, 00-478 Warszawa, Poland}

\author[0000-0002-7777-0842]{Igor Soszy$\acute{\rm n}$ski}
\affiliation{Astronomical Observatory, University of Warsaw, Al. Ujazdowskie 4, 00-478 Warszawa, Poland}

\author[0000-0002-2339-5899]{Paweł Pietrukowicz}
\affiliation{Astronomical Observatory, University of Warsaw, Al. Ujazdowskie 4, 00-478 Warszawa, Poland}

\author[0000-0003-4084-880X]{Szymon Kozłowski}
\affiliation{Astronomical Observatory, University of Warsaw, Al. Ujazdowskie 4, 00-478 Warszawa, Poland}

\author[0000-0001-6364-408X]{Krzysztof Ulaczyk}
\affiliation{Department of Physics, University of Warwick, Gibbet Hill Road, Coventry, CV4 7AL, UK}

\author[0000-0002-9326-9329]{Krzysztof A. Rybicki}
\affiliation{Astronomical Observatory, University of Warsaw, Al. Ujazdowskie 4, 00-478 Warszawa, Poland}
\affiliation{Department of Particle Physics and Astrophysics, Weizmann Institute of Science, Rehovot 76100, Israel}

\author[0000-0002-6212-7221]{Patryk Iwanek}
\affiliation{Astronomical Observatory, University of Warsaw, Al. Ujazdowskie 4, 00-478 Warszawa, Poland}

\author[00000-0002-3051-274X]{Marcin Wrona}
\affiliation{Astronomical Observatory, University of Warsaw, Al. Ujazdowskie 4, 00-478 Warszawa, Poland}

\author[0000-0002-1650-1518]{Mariusz Gromadzki}
\affiliation{Astronomical Observatory, University of Warsaw, Al. Ujazdowskie 4, 00-478 Warszawa, Poland}

\collaboration{13}{(The OGLE Collaboration)}

\author[0000-0003-3316-4012]{Michael D. Albrow}
\affiliation{University of Canterbury, Department of Physics and Astronomy, Private Bag 4800, Christchurch 8020, New Zealand}

\author[0000-0001-6285-4528]{Sun-Ju Chung}
\affiliation{Korea Astronomy and Space Science Institute, Daejon 34055, Republic of Korea}
%\affiliation{University of Science and Technology, Korea, (UST), 217 Gajeong-ro, Yuseong-gu, Daejeon 34113, Republic of Korea}

\author[0000-0002-2641-9964]{Cheongho Han}
\affiliation{Department of Physics, Chungbuk National University, Cheongju 28644, Republic of Korea}

\author[0000-0002-9241-4117]{Kyu-Ha Hwang}
\affiliation{Korea Astronomy and Space Science Institute, Daejon 34055, Republic of Korea}

\author{Doeon Kim}
\affiliation{Department of Physics, Chungbuk National University, Cheongju 28644, Republic of Korea}

\author{Youn Kil Jung}
\affiliation{Korea Astronomy and Space Science Institute, Daejon 34055, Republic of Korea}
\affiliation{University of Science and Technology, Korea, (UST), 217 Gajeong-ro, Yuseong-gu, Daejeon 34113, Republic of Korea}

\author{Hyoun Woo Kim}
\affiliation{Korea Astronomy and Space Science Institute, Daejon 34055, Republic of Korea}

\author[0000-0001-9823-2907]{Yoon-Hyun Ryu}
\affiliation{Korea Astronomy and Space Science Institute, Daejon 34055, Republic of Korea}

\author[0000-0002-4355-9838]{In-Gu Shin}
%\affiliation{Korea Astronomy and Space Science Institute, Daejon 34055, Republic of Korea}
\affiliation{Center for Astrophysics $|$ Harvard $\&$ Smithsonian, 60 Garden Street,Cambridge, MA 02138, USA}

\author[0000-0003-1525-5041]{Yossi Shvartzvald}
\affiliation{Department of Particle Physics and Astrophysics, Weizmann Institute of Science, Rehovot 76100, Israel}

\author[0000-0003-0626-8465]{Hongjing Yang}
\affiliation{Department of Astronomy and Tsinghua Centre for Astrophysics, Tsinghua University, Beijing 100084, People’s Republic of China}

\author[0000-0001-9481-7123]{Jennifer C. Yee}
\affiliation{Center for Astrophysics $|$ Harvard $\&$ Smithsonian, 60 Garden Street,Cambridge, MA 02138, USA}

\author[0000-0001-6000-3463]{Weicheng Zang}
\affiliation{Department of Astronomy and Tsinghua Centre for Astrophysics, Tsinghua University, Beijing 100084, People’s Republic of China}

\author{Sang-Mok Cha}
\affiliation{Korea Astronomy and Space Science Institute, Daejon 34055, Republic of Korea}
\affiliation{School of Space Research, Kyung Hee University, Yongin, Kyeonggi 17104, Republic of Korea}

\author{Dong-Jin Kim}
\affiliation{Korea Astronomy and Space Science Institute, Daejon 34055, Republic of Korea}

\author[0000-0003-0562-5643]{Seung-Lee Kim}
\affiliation{Korea Astronomy and Space Science Institute, Daejon 34055, Republic of Korea}

\author[0000-0003-0043-3925]{Chung-Uk Lee}
\affiliation{Korea Astronomy and Space Science Institute, Daejon 34055, Republic of Korea}

\author{Dong-Joo Lee}
\affiliation{Korea Astronomy and Space Science Institute, Daejon 34055, Republic of Korea}

\author{Yongseok Lee}
\affiliation{Korea Astronomy and Space Science Institute, Daejon 34055, Republic of Korea}
\affiliation{School of Space Research, Kyung Hee University, Yongin, Kyeonggi 17104, Republic of Korea}

\author[0000-0002-6982-7722]{Byeong-Gon Park}
\affiliation{Korea Astronomy and Space Science Institute, Daejon 34055, Republic of Korea}
\affiliation{University of Science and Technology, Korea, (UST), 217 Gajeong-ro, Yuseong-gu, Daejeon 34113, Republic of Korea}

\author[0000-0003-1435-3053]{Richard W. Pogge}
\affiliation{Department of Astronomy, The Ohio State University, 140 W. 18th Avenue, Columbus, OH 43210, USA}

\collaboration{21}{(The KMTNet Collaboration)}

\author[0000-0001-7303-914X]{Uffe G. Jørgensen}
\affiliation{Centre for ExoLife Sciences, Niels Bohr Institute, University of Copenhagen, Øster Voldgade 5, 1350 Copenhagen, Denmark}

%\author{Penelope Longa-Peña}
\author{Penélope Longa-Peña}
%\affiliation{Centro de Astronomía, Universidad de Antofagasta, Avenida Angamos 601, Antofagasta 1270300, Chile}
\affiliation{Centro de Astronomía, Universidad de Antofagasta, Avenida Angamos 601, Antofagasta 1270300, Chile}

\author[0000-0002-2859-1071]{Sedighe Sajadian}
\affiliation{Department of Physics, Isfahan University of Technology, Isfahan 84156-83111, Iran}

\author[0000-0003-1310-8283]{Jesper Skottfelt}
\affiliation{Centre for Electronic Imaging, Department of Physical Sciences, The Open University, Milton Keynes, MK7 6AA, UK}

\author[0000-0001-9328-2905]{Colin Snodgrass}
\affiliation{Institute for Astronomy, University of Edinburgh, Royal Observatory, Edinburgh, EH9 3HJ, UK}

\author[0000-0002-9024-4185]{Jeremy Tregloan-Reed}
\affiliation{Instituto de Investigación en Astronomía y Ciencias Planetarias, Universidad de Atacama, Copayapu 485, Copiapó, Atacama, Chile}

%\author{Khalid A. Alsubai}
%\affiliation{Qatar Environment and Energy Research Institute (QEERI), HBKU, Qatar Foundation, Doha, Qatar}
%
%\author{Michael I. Andersen}
%\affiliation{Niels Bohr Institute, University of Copenhagen, Blegdamsvej 17, DK-2100 Copenhagen, Denmark}
%
%\author[0000-0001-7900-065X]{Allison E. Andrews}
%\affiliation{School of Physical Sciences, The Open University, Milton Keynes MK7 6AA, UK}
%
%\author{Sebastiano Calchi Novati}
%\affiliation{IPAC, Mail Code 100-22, Caltech, 1200 E. California Blvd., Pasadena, CA 91125, USA}
%
\author[0000-0002-8799-0080]{Nanna Bach-Møller}
\affiliation{Centre for Star and Planet Formation, Niels Bohr Institute, University of Copenhagen, Østervoldgade 5, DK-1350 Copenhagen, Denmark}

\author[0000-0002-5854-4217]{Martin Burgdorf}
\affiliation{Universität Hamburg, Faculty of Mathematics, Informatics and Natural Sciences, Department of Earth Sciences, Meteorological Institute, Bundesstraße 55, D-20146 Hamburg, Germany}

%\author[0000-0002-3913-3746]{Justyn Campbell-White}
%\affiliation{SUPA, School of Science and Engineering, University of Dundee, Dundee DD1 4HN, UK}
%
\author[0000-0001-9697-7331]{Giuseppe D'Ago}
\affiliation{Instituto de Astrofísica Pontificia Universidad Católica de Chile, Avenida Vicuna Mackenna 4860, Macul, Santiago, Chile}

%\author[0000-0002-8697-9808]{Sami Dib}
%\affiliation{Max Planck Institute for Astronomy, Königstuhl 17, D-69117 Heidelberg, Germany}
%
%\author{Yuri I. Fujii}
%\affiliation{Graduate School of Human and Environmental Studies, Kyoto University, Yoshida-Nihonmatsu, Sakyo, Kyoto 606-8501, Japan}
%\affiliation{Department of Physics, Nagoya University, Furo-cho, Chikusa-ku, Nagoya, Aichi 464-8602, Japan}
%
\author[0000-0001-9279-2815]{Lauri Haikala}
\affiliation{Instituto de Astronomía y Ciencias Planetarias, Universidad de Atacama, Copayapu 485, Copiapó, Chile}
%\author[0000-0001-9279-2815]{Lauri K. Haikala}
%\affiliation{Instituto de Astronomía y Ciencias Planetarias de Atacama, Universidad de Atacama, Copayapu 485, Copiapo, Chile}

%\author[0000-0001-8870-3146]{Tobias C. Hinse}
%\affiliation{Institute of Astronomy, Faculty of Physics, Astronomy, and Informatics, Nicolaus Copernicus University in Toruń, ul. Grudziadzka 5, 87-100 Toruń, Poland}
%\affiliation{Chungnam National University, Department of Astronomy, Space Science and Geology, Daejeon, Republic of Korea}
%
\author[0000-0002-1508-2243]{James Hitchcock}
\affiliation{University of St Andrews, Centre for Exoplanet Science, SUPA School of Physics \& Astronomy, North Haugh, St Andrews, KY16 9SS, UK}
%\author[0000-0002-1508-2243]{James A. Hitchcock}
%\affiliation{Centre for Exoplanet Science, SUPA, School of Physics \& Astronomy, University of St Andrews, North Haugh, St Andrews KY16 9SS, UK}

\author[0000-0003-0961-5231]{Markus Hundertmark}
\affiliation{Zentrum für Astronomie der Universität Heidelberg, Astronomisches Rechen-Institut, Mönchhofstr. 12-14, D-69120 Heidelberg, Germany}

%\author{Tim-Oliver Husser}
%\affiliation{Institut fur Astrophysik, Georg-August-Universität Göttingen, Friedrich-Hund-Platz 1, D-37077 Göttingen, Germany}
%
%\author[0000-0002-1743-4468]{Eamonn Kerins}
%\affiliation{Jodrell Bank Centre for Astrophysics, Alan Turing Building, University of Manchester, Manchester, M13 9PL, UK}
%
\author{Elahe Khalouei}
\affiliation{Astronomy Research Center, Research Institute of Basic Sciences, Seoul National University,1 Gwanak-ro, Gwanak-gu, Seoul 08826, Korea}
%\affiliation{Department of Physics, Sharif University of Technology, PO Box 11365-9161, Tehran, Iran}

%\author[0000-0002-9428-8732]{Luigi Mancini}
%\affiliation{Max-Planck-Institute for Astronomy, Königstuhl 17, D-69117 Heidelberg, Germany}
%\affiliation{Department of Physics, University of Rome “Tor Vergata,” Via della Ricerca Scientifica 1, I-00133 Roma, Italy}
%\affiliation{INAF—Astrophysical Observatory of Turin, Via Osservatorio 20, I-10025 Pino Torinese, Italy}
%
\author[0000-0002-6830-476X]{Nuno Peixinho}
\affiliation{Instituto de Astrof\'{\i}sica e Ci\^{e}ncias do Espa\c{c}o, Departamento de F\'{\i}sica, Universidade de Coimbra, PT3040-004 Coimbra, Portugal}
%\affiliation{Unidad de Astronomía, Universidad de Antofagasta, Av. Angamos 601, Antofagasta, Chile}

%\author[0000-0001-7506-5640]{Matthew T. Penny}
%\affiliation{Louisiana State University, 261-B Nicholson Hall, Tower Drive, Baton Rouge, LA 70803-4001, USA}
%
\author[0000-0002-7084-5725]{Sohrab Rahvar}
\affiliation{Department of Physics, Sharif University of Technology, PO Box 11155-9161, Tehran, Iran}

%\author[0000-0003-2935-7196]{Markus Rabus}
%\affiliation{Departamento de Matemática y Física Aplicadas, Facultad de Ingeniería, Universidad Católica de la Santísima Concepción, Alonso de Rivera 2850, Concepción, Chile}
%
%\author[0000-0002-9790-0552]{Davide Ricci}
%\affiliation{INAF—Padova Astronomical Observatory, Vicolo dell’Osservatorio 5, I-35122 Padova, Italy}
%
\author[0000-0002-3807-3198]{John Southworth}
\affiliation{Astrophysics Group, Keele University, Staffordshire, ST5 5BG, UK}
%\affiliation{Astrophysics Group, Keele University, Staffordshire, ST5 5BG, UK}

%\author[0000-0002-8365-7619]{Joachim Wambsganss}
%\affiliation{Zentrum für Astronomie der Universität Heidelberg, Astronomisches Rechen-Institut, Mönchhofstr. 12-14, D-69120 Heidelberg, Germany}
%
%\author{Olivier Wertz}
%\affiliation{Space Sciences, Technologies, and Astrophysics Research (STAR) Institute, University of Liège, Liège, Belgium}
%
\author{Petros Spyratos}
\affiliation{Astrophysics Group, Keele University, Staffordshire, ST5 5BG, UK}

\collaboration{17}{(The MiNDSTEp Collaboration)}

%\begin{abstract}
Graph Neural Networks (GNNs) have proven to be effective in processing and learning from graph-structured data.
However, previous works mainly focused on understanding single graph inputs while many real-world applications require pair-wise analysis for graph-structured data (e.g., scene graph matching, code searching, and drug-drug interaction prediction).
To this end, recent works have shifted their focus to learning the interaction between pairs of graphs.
Despite their improved performance, these works were still limited in that the interactions were considered at the node-level, resulting in high computational costs and suboptimal performance.
To address this issue, we propose a novel and efficient graph-level approach for extracting interaction representations using co-attention in graph pooling. 
Our method, Co-Attention Graph Pooling (CAGPool), exhibits competitive performance relative to existing methods in both classification and regression tasks using real-world datasets, while maintaining lower computational complexity.

\end{abstract}
\begin{abstract}
We present an analysis of microlensing event OGLE-2019-BLG-0825.
%This event was identified as a planetary event following preliminary modeling. 
This event was identified as a planetary candidate by preliminary modeling.
%From a detailed analysis, we find that significant residuals from the best-fit static binary-lens model exist and a xallarap effect can fit the residuals very well and significantly improves $\chi^2$ values in all data sets.
We find that significant residuals from the best-fit static binary-lens model exist and a xallarap effect can fit the residuals very well and significantly improves $\chi^2$ values.
%We detect a xallarap effect with $\sim$5 day period can fit the residuals very well and significantly improves $\chi^2$ values in all data sets. 
%We detect a xallarap effect can fit the residuals very well and significantly improves $\chi^2$ values in all data sets.
On the other hand, by including the xallarap effect in our models, we find that binary-lens parameters like mass-ratio, $q$, and separation, $s$, cannot be constrained well. 
%On the other hand, by including the xallarap effect in our models, we find that binary-lens parameters like mass-ratio, $q$, and separation, $s$, cannot be constrained well for planetary mass ratio. 
%However, we also find the parameters for the source system like the orbital period and semi major axis are consistent between all the we analyzed models. 
However, we also find that the parameters for the source system like the orbital period and semi major axis are consistent between all the models we analyzed.
%We conclude that the source system consists of a G-type main-sequence star orbited by a low-mass companion (M-dwarf or brown dwarf) with $P\sim5$ days and semi major axis $a\sim0.06\;\mathrm{au}$. 
%In conclusion, we succeeded in constraining the properties of the source system more than the properties of the lens system.
We therefore constrain the properties of the source system better than the properties of the lens system.
%The source system consists of a host of G-type main-sequence star orbited by a low-mass companion (M-dwarf or brown dwarf) with $P\sim5$ days and semi major axis $a\sim0.06\;\mathrm{au}$.
The source system comprises a G-type main-sequence star orbited by a brown dwarf with a period of $P\sim5$ days.
%The source system consists of a host of G-type main-sequence star orbited by a low-mass companion (M-dwarf or brown dwarf) with a period of $P\sim5$ days.
%This analysis is the first to demonstrate that the xallarap effect with short periods can affect binary-lens parameters in planetary events.
This analysis is the first to demonstrate that the xallarap effect does affect binary-lens parameters in planetary events.
It would not be common for the presence or absence of the xallarap effect to affect lens parameters in events with long orbital periods of the source system or events with transits to caustics, but in other cases, such as this event, the xallarap effect can affect binary-lens parameters.
%In an event like this event, which has a short orbital period of source system and no transit to the caustics, the presence or absence of the xallarap effect is likely to affect the lens parameters, but otherwise it would not be common.
\end{abstract}
\keywords{gravitational lensing: micro --- brown dwarfs --- xallarap}
%
\section{Introduction}
Deep learning models have been widely used in many applications.
For example, BERT~\citep{devlin_bert_2019}, GPT-3~\citep{brown_language_2020}, and T5~\citep{raffel_exploring_2020} achieved state-of-the-art~(SOTA) results on different natural language processing~(NLP) tasks. 
For computer vision~(CV), Transformer-like models such as ViT~\citep{dosovitskiy_image_2021} and Swin Transformer~\citep{liu_swin_2021} deliver excellent accuracy performance upon multiple tasks. 


At the same time, training deep learning models has been a critical problem troubling the community due to the long training time, especially for those large models with billions of parameters~\citep{brown_language_2020}. 
In order to enhance the training efficiency, researchers propose some manually designed parallel training strategies~\citep{narayanan_efficient_2021,shazeer_mesh-tensorflow_2018,xu_gspmd_2021}. 
However, selecting, tuning, and combining these strategies require extensive domain knowledge in deep learning models and hardware environments. With the increasing diversity of modern hardware architectures~\cite{flynn_very_1966,flynn_computer_1972} and the rapid development of deep learning models, these manually designed approaches are bringing heavier burdens to developers. 
Hence, \emph{automatic parallelism} is introduced to automate the parallel strategy searching for training models.


There are two main categories of parallelism in deep learning models: inter-layer parallelism~\citep{huang_gpipe_2019,narayanan_pipedream_2019,narayanan_memory-efficient_2021,fan_dapple_2021,li_chimera_2021,lepikhin_gshard_2021,du_glam_2022,fedus_switch_2022} and intra-layer parallelism~\citep{li_pytorch_2020,narayanan_efficient_2021,rasley_deepspeed_2020,fairscale_authors_fairscale_2021}. 
Inter-layer parallelism partitions the model into disjoint sets on different devices without slicing tensors. 
Alternatively, intra-layer parallelism partitions tensors in a layer along one or more axes and distributes them across different devices.


Current automatic parallelism techniques focus on optimizing strategies within these two categories. However, they treat these two categories separately. 
Some methods~\citep{zhao_vpipe_2022,jia_exploring_2018,cai_tensoropt_2022,wang_supporting_2019,jia_beyond_2019,schaarschmidt_automap_2021,liu_colossal-auto_2023} overlook potential opportunities for inter- or intra-layer parallelism, the others optimize inter- and intra-layer parallelism hierarchically and sequentially~\citep{narayanan_pipedream_2019,fan_dapple_2021,he_pipetransformer_2021,tarnawski_efficient_2020,tarnawski_piper_2021,zheng_alpa_2022}. 
As a result, current automatic parallelism techniques often fail to achieve the global optima and instead become trapped in local optima. 
Therefore, a unified inter- and intra-layer approach is needed to enhance the effectiveness of automatic parallelism.


This paper aims to find the optimal parallelism strategy while simultaneously considering inter- and intra-layer parallelism. 
It enables us to search in a more extensive strategy space where the globally optimal solution lurk. 
However, unifying inter- and intra-layer parallelism in automatic parallelism brings us two challenges. 
Firstly, to adopt a unified perspective on the inter- and intra-layer automatic parallelism, we should not formalize them with separate formulations as prior works. Therefore, how can we express these parallelism strategies in a unified formulation? 
Secondly, previous methods take a long time to obtain the solution with a limited strategy space. Therefore, how can we ensure that the best solution can be obtained in a reasonable time while expanding the strategy space?


To solve the above challenges, we propose UniAP. For the first challenge, UniAP adopts the mixed integer quadratic programming~(MIQP)~\citep{lazimy_mixed_1982} to search for the globally optimal parallel strategy automatically. 
It unifies the inter- and intra-layer automatic parallelism in a single MIQP formulation. 
For the second challenge, our complexity analysis and experimental results show that UniAP can obtain the globally optimal solution in a significantly shorter time.


The contributions of this paper are summarized as follows: 
\begin{itemize}
    \item We propose UniAP, the first framework to unify inter- and intra-layer automatic parallelism in model training.
    \item The optimal parallel strategies discovered by UniAP exhibit scalability on training throughput and strategy searching time.
    \item The experimental results show that UniAP speeds up model training on four Transformer-like models by up to 1.70$\times$ and reduces the strategy searching time by up to 16$\times$, compared with the SOTA method.
\end{itemize}

\section{Introduction}
The gravitational microlensing method is a method for detecting exoplanets that utilizes the phenomenon that light is deflected by gravity \citep{Liebes1964,Paczynski1991} and is sensitive to planets beyond the snow line \citep{Gould+1992,Bennett+1996}.
Giant planets are thought to form near and beyond the snow line \citep{Ida+2004,Laughlin+2004,Kennedy+2006}.
%where volatile compounds condense into solid grains. 
In gravitational microlensing, when a lensing object crosses in front of a source star, the brightness of the source star changes with time owing to the gravitational effect of the lensing object. 
Furthermore, if this lensing object is accompanied by a planetary or binary-star companion, the gravity of this companion will cause a secondary magnification. 
The gravitational microlensing method does not use the light from the lensing object, but only the time-dependent variations arising form the gravitational effect of the lensing object or objects on the light from the source.
%The gravitational microlensing method does not use the light from the lensing object, but only the time variation owing to the lensing object(s) gravitational effect on the light from the source.
%The gravitational microlensing method does not use the light from the lensing object, but only the time variation of the source star's brightness. 
Therefore, the gravitational microlensing method has the advantage over other planet detection methods of being able to detect planets around faint stars at distances far from Earth \citep{Gaudi2012}. 
By comparing the occurrence rates of planets in the distant region detected by the gravitational microlensing method with the frequency of planets in the local region, we can investigate the influence of the Galactic environment on planet formation.

%Observations and simulations suggest that metallicity varies with the Galactic environment, and the Galactic halo and thick disk have low metallicity \citep{Cabral+2019}. 
%In thin disks, the metal content (e.g., [Fe/H]) is higher in the Galactic center and lower in the outer regions because star formation is more active in the Galactic center and Type II SNe occur \citep[e.g.,][]{Gummersbach+1998}. 
%On the other hand, \citet{Sousa+2019} investigate the relationship between detected exoplanet masses and the metallicity of the host star and show that the higher the metallicity of the host star, the more it hosts a giant planet. 
%From there, it has been suggested that different Galactic environments have different planetary occurrence rates \citep{Gonzalez+2001, Lineweaver+2004, Spinelli+2021}.  
The detection of distant planets and brown dwarfs allows us to consider the influence of the Galactic environment on planet and brown dwarf formation.
It has been thought that different Galactic environments have different planetary occurrence rates \citep{Gonzalez+2001, Lineweaver+2004, Spinelli+2021}.
In fact, radial velocity surveys in the 25 pc region near the Sun reported that the occurrence rate of hot Jupiters is about $\sim2\%$ \citep{Hirsch+2021}, whereas Kepler transit surveys report that the occurrence rate of hot Jupiters around G- K- type stars near Cygnus is about $\sim0.5\%$ \citep{Howard+2012, Santerne+2012, Santerne+2016, Fressin+2013}. 
Although \citet{Koshimoto+2021a}  recently found that planetary frequencies do not depend significantly on the Galactocentric distance based on their 28 planet sample, their result is still too uncertain to discuss environmental effects precisely.

In the analysis of gravitational microlensing events, it is sometimes difficult to distinguish perturbations given by the lens secondary to the light curve from those given by higher-order effects \citep{Griest+1992,Rota+2021}. 
One of the higher-order effects, the parallax effect, is the effect of the acceleration of the Earth's orbital motion on the light curve. 
The xallarap effect is a higher-order effect on the light curve when the source is binary \citep{Griest+1992,Han+1997,Paczynski1997,Poindexter+2005}. 
%A xallarap signal often competes with a parallax signal \citep{Miyake+2012,Furusawa+2013}. 
%In particular, convergence to a parallax signal is more likely when the source orbital period is close to 1 year \citep{Han+2013,Kains+2013,Kim+2021}.
%Binary stars are common in the Universe, with binary systems of two or more stars accounting for about 40\% \citep{Lada2006,Badenes+2018}. 
Binary stars are common in the Universe, with binary systems of two or more stars accounting for about 30\% of all stellar systems \citep{Lada2006,Badenes+2018}. 
When a source is accompanied by a companion star, the companion is not necessarily magnified, but the light curve is affected by the orbital motion of the source primary \citep{Rota+2021}. 
Although most of the binary stars are too wide between their primary and companion stars to reliably detect a xallarap effect, a systematic survey of 22 long-term events in the bulge shows that 23\% of them are indeed affected by xallarap \citep{Poindexter+2005}. 
The effect of xallarap on lensing planet detection efficiency has not been fully investigated but is known to exist \citep{Zhu+2017}.
%The Nancy Grace Roman Space Telescope \citep[][previously named WFIRST, hereafter Roman]{Spergel+2015} is predicted to detect dozens of jovian or brown dwarf mass source companions from xallarap \citep{Miyazaki+2021}.
%\citet{Miyazaki+2021} assumed a planetary distribution of masses and orbital periods of \citet{Cumming+2008}, but comparing their predictions with actual results allows us to test the brown dwarf desert \citep{Marcy+2000,Grether+2006} of the Galactic bulge can be tested.

We present in this paper an analysis of OGLE-2019-BLG-0825 and report that the xallarap effect was detected and that the lensing system parameters changed before and after the xallarap effect was included.
Section~\ref{sec:Observation} describes the data for event OGLE-2019-BLG-0825.
Section~\ref{sec:Data_reduction} describes our data reduction.
Section~\ref{sec:light_curve_modeling} describes our modeling in detail. 
Section~\ref{sec:Source_Lens_Properties} derives the color and magnitude of the source and calculates the physical parameters of the source system from the color and magnitude of the source and the fitting parameters of the microlensing.
%Section~\ref{sec:Source_Lens_Properties} derives the color and magnitude of the source and calculates the physical parameters of the source star and lens star system from the color and magnitude of the source and the fitting parameters of the microlensing.
Section~\ref{sec:Lens_Properties} describes the estimation of the physical parameters of the lens system by Bayesian analysis. 
Finally, Section~\ref{sec:Discussion_and_Conclusion} discusses and summarizes the results of our analysis.
%\input{2_Observation.tex}
\section{Observation}
\label{sec:Observation}
% Figure environment removed
\begin{deluxetable*}{lccccccccccc}[t!]
\tablecaption{Data Sets for OGLE-2019-BLG-0825 \label{tab:dataset}}
\tablewidth{0pt}
\tablehead{
\multicolumn{1}{l}{Observatory Sites} & \multicolumn{1}{c}{Telescope} & \multicolumn{1}{c}{Collaboration} & \multicolumn{1}{c}{Label} & \multicolumn{1}{c}{Filter} & \multicolumn{1}{c}{$N_{\rm use}$} & \multicolumn{1}{c}{$k$\tablenotemark{1}} & \multicolumn{1}{c}{$e_{\rm min}$\tablenotemark{1}}
%\multicolumn{1}{l}{Observatory Sites} & \multicolumn{1}{c}{Telescope} & \multicolumn{1}{c}{Collaboration} & \multicolumn{1}{c}{Label} & \multicolumn{1}{c}{Filter} & \multicolumn{1}{c}{$N_{\rm use}$/$N_{\rm obs}$} & \multicolumn{1}{c}{$k$\tablenotemark{1}} & \multicolumn{1}{c}{$e_{\rm min}$\tablenotemark{1}}
}
%\decimalcolnumbers
\startdata
Mount John & MOA-II 1.8m & MOA & MOA & MOA-Red & 3949 & 1.330 & 0.009 \\
& & & & $V$ & 86 & 0.835 & 0 \\
Las Campanas & Warsaw 1.3m & OGLE & OGLE & $I$ & 1535 & 1.453 & 0.007 \\
Siding Spring & KMTNet Australia 1.6m & KMTNet & KMTA01\tablenotemark{2} & $I$ & 704 & 1.649 & 0 \\
 & & & KMTA41\tablenotemark{3} & $I$ & 719 & 1.613 & 0 \\
Cerro Tololo Inter-American & KMTNet Chile 1.6m & & KMTC01\tablenotemark{2} & $I$ & 952	& 0.761	& 0.004 \\
& & & KMTC41\tablenotemark{3} & $I$ & 954 & 1.436 & 0 \\
South Africa Astronomical & KMTNet South Africa 1.6m & & KMTS01\tablenotemark{2} & $I$ & 881 & 1.490 & 0 \\
& & & KMTS41\tablenotemark{3} & $I$ & 887 & 1.416 & 0 \\
ESO’s La Silla & Danish 1.54m & MiNDSTEp & Danish & $I$ & 76 & 0.706 &	0 \\
%ESO’s La Silla & Danish 1.54m & MiNDSTEp & Danish & $I$ & 76/79 & 0.706 &	0 \\
%Space & Spitezr & Spitzer &  & $L$ & 0/23 &	
\enddata
\tablenotetext{1}{Parameters for the error normalization.}
\tablenotetext{2}{Data observed in BLG01 in the overlapped area.}
\tablenotetext{3}{Data observed in BLG41 in the overlapped area.}
%\tablecomments{}
\end{deluxetable*}
Microlensing event OGLE-2019-BLG-0825 was first discovered on June 3, 2019 (${\rm HJD}^{\prime}\sim8638$)\footnote{${\rm HJD}^{\prime}\equiv{\rm HJD}-2450000$} at J2000 equatorial coordinates $({\rm R.A., decl.})=(17^h 52^m 21^s.62, -30^\circ 48^\prime 13^{\prime\prime}.2)$ corresponding to Galactic coordinates $(l,b)=(-0.849,-2.214)$, by the Optical Gravitational Lensing Experiment \citep[OGLE;][]{Udalski2003} collaboration. 
OGLE conducts a microlensing survey using the 1.3m Warsaw Telescope with a 1.4 deg$^2$ field-of-view (FOV) CCD camera at Las Campanas Observatory in Chile and distributes alerts of the discovery of microlensing events by their OGLE-IV Early Warning System \citep{Udalski+1994,Udalski2003,Udalski+2015}. 
The event is located in the OGLE-IV field BLG534, which is observed on Cousins $I$-band with an hourly cadence \citep{Mroz+2019}.

The Microlensing Observations in Astrophysics \citep[MOA;][]{Bond+2001,Sumi+2003} collaboration also independently discovered this event on June 23, 2019, and identified it as MOA-2019-BLG-273 using the MOA alert system \citep{Bond+2001}. 
The MOA collaboration conducts a microlensing exoplanet survey toward the Galactic bulge using the 1.8m MOA-II telescope with a 2.2 deg$^2$ wide FOV CCD camera, MOA-cam3 \citep{Sako+2008}, at the University of Canterbury's Mount John Observatory in New Zealand. 
The MOA survey uses a custom wide band filter referred to as $R_{\rm MOA}$ corresponding to the sum of the Cousins $R$ and $I$ bands.
In addition, a Johnson $V$-band filter is used primarily for measuring the color of the source. The event is located in the MOA field gb4, which is observed with high cadence once every 15 minutes.

The Korea Microlensing Telescope Network \citep[KMTNet;][]{Kim+2016} collaboration conducts a microlensing survey using three 1.6m telescopes each with a 4.0 deg$^2$ FOV CCD camera. 
The telescopes are located at the Cerro Tololo Inter-American Observatory (CTIO) in Chile, the South African Astronomical Observatory (SAAO) in South Africa, and Siding Spring Observatory (SSO) in Australia. 
This event is located in an overlapping region with two KMTNet observed fields (KMTNet BLG01 and BLG41), which are observed with high cadence once every 15 minutes and was discovered by the KMTNet EventFinder \citep{Kim+2018} as KMT-2019-BLG-1389 on June 28, 2019.

The Danish telescope of MiNDSTEp (Microlensing Network for the Detection of Small Terrestrial Exoplanets) made follow-up observations in $I$-band. 
MiNDSTEp uses the 1.54m Danish Telescope at the European Southern Observatory at La Silla Observatory in Chile \citep{Dominik+2010}.
%All of the data sets of OGLE-2019-BLG-0825 are summarized in Table~\ref{tab:dataset}. 
Data from the Spitzer space telescope \citep{Yee+2015} were also available, but these show no detectable signal and so are not used.
A summary of all datasets used in the analysis of OGLE-2019-BLG-0825 is shown in Table~\ref{tab:dataset}.

The above data sets are used in our light curve analysis. 
%To reduce baseline systematic effects, we used approximately 2 years of data in $8100\leq {\rm HJD}^{\prime} \leq8800$. 
To reduce long-term systematics on the baseline, we used approximately 2 years of data over $8100\leq {\rm HJD}^{\prime} \leq8800$.
Figure~\ref{fig:lightcurve} shows the light curve of OGLE-2019-BLG-0825 and the standard binary lens single source model (hereafter, standard 2L1S), the binary lens single source with parallax effect model (hereafter, 2L1S + parallax), the single lens single source with xallarap effct model (hereafter 1L1S + xallarap), and the best-fit model (binary lens single source with xallarap effect model, hearafter 2L1S + xallarap), described in Section~\ref{sec:light_curve_modeling}, respectively.
%As will be discussed in detail in Section~\ref{sec:Source_Lens_Properties}, the xallarap model analysis assumes that the secondary source is not magnified and is therefore denoted 1S.
As will be discussed in detail in Section~\ref{sec:Source_Lens_Properties}, the xallarap model analysis assumes that the magnified flux of the second source is too weak to be detected, so it is denoted 1S.
%\input{3_Data_reduction.tex}
\section{Data Reduction}
\label{sec:Data_reduction}
The OGLE data were reduced with the OGLE Difference Image Analysis (DIA) \citep{Wozniak2000} photometry pipeline \citep{Udalski2003,Udalski+2015} which uses the DIA technique \citep{Tomaney+1996,Alard+1998,Alard+2000}.  
The MOA data were reduced with MOA’s implementation of the DIA photometry pipeline \citep{Bond+2001}. 
The KMTNet data were reduced with their PySIS photometry pipeline \citep{Albrow+2009}. 
%The KMTNet data were reduced with their PySIS photometry pipeline \citep{Albrow+2009,Kim+2016}. 
The MiNDSTEp data were reduced using DanDIA \citep{Bramich2008,Bramich+2013}. 
%The DIA method is advantageous for photometry of stars located in crowded fields, such as the Galactic bulge field, because it handles the effects of blending more efficiently than point spread function (PSF) photometry.

It is known that the nominal error bars calculated by the pipelines are incorrectly estimated in such crowded stellar fields. 
We follow a standard empirical error bar normalization process \citep{Yee+2012} intended to estimate proper uncertainties for the lensing parameters in the light-curve modeling. 
This process, described below, hardly affects the best-fit parameters \citep{Ranc+2019}.
We renormalize the photometric error bars using the formula
\begin{equation}\label{eq:error}
    \sigma^{\prime}_i = k\sqrt{\sigma^2_i + e^2_{\rm min}},
\end{equation}
in which $\sigma^{\prime}_i$ is the renormalized uncertainty in magnitude, while $\sigma_i$ is an uncertainty of the $i$-th original data point obtained from the photometric pipeline.
 The variables $k$ and $e_{\rm min}$ are renormalizing parameters. 
 For preliminary modeling, we search for the best-fit lensing parameters using $\sigma_i$.
 We then construct a cumulative $\chi^2$ distribution as a function of lensing magnification. 
 The $e_{\rm min}$ value is chosen so that the slope of the distribution is uniform \citep{Yee+2012}. 
 The $k$ value is chosen so that $\chi^2/$d.o.f.\footnote{Degrees of freedom.}$\simeq1$. 
 In Table~\ref{tab:dataset}, we list the calculated error bar renormalization parameters.
%\input{4_light_curve_modeling.tex}
\section{Light Curve Modeling}

The model flux for a microlensing event is given by the following equation,
\begin{equation}\label{eq:magnification}
    f(t)=A(t,\bm{x})f_s + f_b,
\end{equation}

\noindent where $A(t,\bm{x})$ is the source flux magnification, $f_s$ is the flux of the source star, and $f_b$ is the blend flux.
%In the 1L1S model, $\bm{x}$ is described by four parameters \citep{Paczynski1986}: the time of the source closest to the center of mass, $t_0$; the Einstein radius crossing time, $t_{\rm E}$; the impact parameter normalized by the Einstein radius, $R_{\rm E}$, $u_0$, and the source angular radius relative to the angular Einstein radius, $\theta_{\rm E}$, $\rho$. 
%In the 1L1S model, $\bm{x}$ is described by four parameters \citep{Paczynski1986}: the time of the source closest to the center of mass, $t_0$; the Einstein radius crossing time, $t_{\rm E}$; the impact parameter relative to the angular Einstein radius, $\theta_{\rm E}$, $u_0$, and the source angular radius relative to the angular Einstein radius, $\theta_{\rm E}$, $\rho$. 
In the 1L1S model, $\bm{x}$ is described by four parameters \citep{Paczynski1986}: the time of the source closest to the center of mass, $t_0$; the Einstein radius crossing time, $t_{\rm E}$; the impact parameter, $u_0$, and the source angular radius, $\rho$.
Both $u_0$ and $\rho$ are in units of the angular Einstein radius, $\theta_{\rm E}$.

For modeling the light curve, we used the Metropolis-Hastings Markov Chain Monte Carlo method. 
%For modeling the light curve, we used the Metropolis-Hastings Markov Chain Monte Carlo (MCMC) method. 
%For modeling the light curve, we used the Metropolis-Hastings method, a variant of the Markov Chain Monte Carlo (MCMC) method. 
The finite source effect, an effect in which the source has a finite angular size, was calculated using the image-centered inverse-ray shooting method \citep{Bennett+1996,Bennett2010a} as implemented by \citet{Sumi+2010}. 
Note that $f_s$ and $f_b$ parameters are obtained from a linear-fit using the method of \citet{Rhie+1999}.
We adopt the following linear limb-darkening law for source brightness:
\label{sec:light_curve_modeling}
\begin{equation}
    S_\lambda(\vartheta) = S_\lambda(0)\left[1-u_\lambda(1-\cos(\vartheta))\right],
\end{equation}
where $\vartheta$ represents the angle between the line of sight and the normal to the surface of the source star. 
$S_\lambda(\vartheta)$ is a limb-darkening surface brightness of $\vartheta$ at wavelength $\lambda$. 
We estimated the effective temperature of the source star in Section~\ref{sec:Source_Lens_Properties} to be $T_{\rm eff}=5425\pm359$ K \citep{Gonzalez+2009}.
In this analysis, we assume the source star's metallicity $[\rm M/H]=0$, surface gravity $\log{g}=4.5$, and microturbulent velocity $v=1$ $\rm {km/s}$.
We use the limb-darkening cofficients $u_V=0.685$, $u_R=0.604$ and $u_I=0.518$, which are taken from the ATLAS model with $T_{\rm eff}=5500$ K \citep{Claret+2011}. 
Since $R_{\rm MOA}$ covers both $R$- and $I$-band wavelengths, we adopted the average value $u_{R_{\rm MOA}}=(u_R+u_I)/2=0.561$.
In addition, as will be discussed in more detail in Section~\ref{sec:Discussion_and_Conclusion}, we assume that the source of this event is a main-sequence star.
 
As the result of 1L1S model analysis, we found that $(t_0,t_{\rm E},u_0,\rho)=(8662.6, 47.6 ,1.2\times10^{-2},4.8\times10^{-3})$ is the best solution. 
%This 1L1S best model is $\Delta\chi^2= 21400$ worse than the best standard 2L1S model shown in Section \ref{subsec:Standard_Binary_Lens}.
This 1L1S best model is $\Delta\chi^2= 21400$ worse than the best standard 2L1S model.
%This 1L1S best model was $\Delta\chi^2\sim 21400$ worse than the standard binary lens.
%This 1L1S best model was $\Delta\chi^2\sim 21400$ worse than the best model assuming the standard binary lens see Section \ref{subsec:Standard_Binary_Lens}.

% Figure environment removed
\begin{comment}
% Figure environment removed
\end{comment}

\subsection{Standard Binary Lens}
\label{subsec:Standard_Binary_Lens}
%In the standard 2L1S model, additional three parameters are required; the mass ratio of a lens companion relative to a host, $q$; the projected separation normalized by $R_{\rm E}$ between binary components, $s$; the angle between the binary-lens axis and the source trajectory direction, $\alpha$.
In the standard 2L1S model, three additional parameters are required; the mass ratio of a lens companion relative to the host, $q$; the projected separation normalized by Einstein radius between binary components, $s$; the angle between the binary-lens axis and the source trajectory direction, $\alpha$.

Because the $\chi^2$ surface of the microlensing parameter has a very complicated shape, 34440 values of $(q,s,\alpha)$, which have a particularly large impact on the shape of the light curve were initially fixed in the fitting process.
%We set 34440 grids of $(q,s,\alpha)$, which have a particularly large impact on the shape of the light curve, and fixed them as the initial values for the fitting with the MCMC method.
%Because the probability density distribution of the microlensing parameter space has a very complicated shape, 34440 grids of $(q,s,\alpha)$, which have a particularly large impact on the shape of the light curve, were initially fixed for the fitting with the MCMC method. 
%For the initial values across the grid, 21 values of $q$ were used for $-5\leq\log{q}\leq0$, 41 values of $s$ were used for $-1.25\leq\log{s}\leq1.25$, 40 values of $\alpha$ were used for $0\leq\alpha\leq2\pi$. 
Here we uniformly take 21 values between $-5\leq\log{q}\leq0$, 41 values between $-1.25\leq\log{s}\leq1.25$, 40 values in $0\leq\alpha\leq2\pi$, respectively.
For the top 1000 combinations which gave good fits, we performed the fitting again with $q,s$, and $\alpha$ free. 
This process minimizes the chance that we miss local solutions even in a large and complex microlensing parameter space. 
The left panel of Figure~\ref{fig:Grid_search} shows the results of the grid search analysis for the standard 2L1S model.

%As a result of the analysis, we obtained $(q,s)=(3.3\times10^{-3},0.57)$ as the best model for the standard 2L1S model. 
%Hereafter, $s<1$ is called close solution and $s>1$ is called wide solution, and the best model for standard 2L1S is called as close1. 
%We also obtained $(q,s)=(3.4\times10^{-3},1.75)$ by $\Delta\chi^2\sim0.4$, $(q,s)=(2.1\times10^{-2},0.28)$ by $\Delta\chi^2\sim20.4$ and $(q,s)=(2.1\times10^{-2},3.78)$ by $\Delta\chi^2\sim23.3$ as a local solution. 
%We refer to these as wide1, close2, and wide2, respectively.
As a result of the analysis, the best fit standard 2L1S model is $(q,s)=(3.3\times10^{-3},0.57)$ (close1).
Hereafter, we call solutions with $s<1$ and $s>1$ as  ``close" and ``wide", respectively. We call the best standard 2L1S as close1.
We also found local minima at $(q,s)=(3.4\times10^{-3},1.75)$ (wide1) with $\Delta\chi^2\sim0.4$, $(q,s)=(2.1\times10^{-2},0.28)$ (close2) with $\Delta\chi^2\sim20.4$ and $(q,s)=(2.1\times10^{-2},3.78)$ (wide2) with $\Delta\chi^2\sim23.3$.
Detailed parameters of the standard binary models are shown in Table~\ref{tab:param_standard}. 
%However, we observed systematic residuals around the peak of $8657<{\rm HJD}^\prime<8667$ in these models, as seen in Figure~\ref{fig:lightcurve}.
However, we observed systematic residuals around the peak of $8657<{\rm HJD}^\prime<8667$ in these models, as depicted by the green dashed line in Figure~\ref{fig:lightcurve}.
%We find that the systematic trend cannot be explained well by adding the parallax effect to the standard 2L1S model, as also seen in Figure \ref{fig:lightcurve}.
%Therefore, we attempted to better model the light curve by adding higher order effects.
In Figure~\ref{fig:lightcurve}, we plot only close1, the best for the standard 2L1S, but the other three models also have similar residuals.
%Therefore, we next consider higher order effects.
%Therefore, we will model the light curve with higher order effects.
We therefore proceed to model the light curve with higher order effects.

\begin{deluxetable*}{c|ccccccccccc}[t!]
\tablecaption{Parameters of the standard 2L1S models \label{tab:param_standard}}
\tablewidth{0pt}
\tablehead{
\multicolumn{1}{c|}{Model} & \multicolumn{1}{c}{close1} & \multicolumn{1}{c}{close2} & \multicolumn{1}{c}{wide1} & \multicolumn{1}{c}{wide2} 
}
%\decimalcolnumbers
\startdata
$t_0$(HJD-2458660) & $2.474\pm0.001$ & $2.483\pm0.001$ & $2.473\pm0.001$ & 	$2.489\pm0.001$\\
$t_{\rm E}$ (days) & $74.7\pm2.0$ & $75.7\pm2.0$ & $72.8\pm1.8$ & $77.3\pm2.0$ \\
$u_0$ ($10^{-3}$) & $7.30\pm0.21$ &  $7.11\pm0.19$ & $7.53\pm0.19$ & $6.91\pm0.19$\\
$q$ ($10^{-3}$) & $3.30\pm0.11$ & $20.71\pm9.84$ & $3.39\pm0.10$ & $21.33\pm1.15$  \\
$s$ & $0.569\pm0.004$ & $0.207\pm0.038$ & $1.747\pm0.011$ & 
 $3.776\pm0.063$ \\
$\alpha$ (radian) & $5.034\pm0.002$ & $2.766\pm0.002$ & $5.036\pm0.003$ & $2.767\pm0.002$\\
$\rho$ ($10^{-3}$) & $2.95\pm0.09$ & $0.48\pm0.28$ & $3.02\pm0.09$ & $0.47\pm0.14$\\ \hline \hline
$\chi^2$ & 11744.7 & 11765.1 & 11745.1 & 11768.0\\
$\Delta\chi^2$ & - & 20.4 &	0.4 & 23.3\\
\enddata
%\tablenotetext{1}{From the best of standard 2L1S (i.e. close1) model.}
\end{deluxetable*}
\begin{comment}
%20230705 by Nick
\begin{deluxetable*}{c|ccccccccccc}[t!]
\tablecaption{Parameters of the standard 2L1S models \label{tab:param_standard}}
\tablewidth{0pt}
\tablehead{
\multicolumn{1}{c|}{Model} & \multicolumn{1}{c}{close1} & \multicolumn{1}{c}{close2} & \multicolumn{1}{c}{wide1} & \multicolumn{1}{c}{wide2} 
}
%\decimalcolnumbers
\startdata
$t_0$(HJD-2458660) & $2.474\pm0.001$ & $2.483\pm0.001$ & $2.473\pm0.001$ & 	$2.489\pm0.001$\\
$t_{\rm E}$ (days) & $74.73\pm2.01$ & $75.73\pm1.980$ & $72.79\pm1.79$ & $77.33\pm2.03$ \\
$u_0$ ($10^{-3}$) & $7.300\pm0.210$ &  $7.110\pm0.189$ & $7.528\pm0.189$ & $6.910\pm0.187$\\
$q$ ($10^{-3}$) & $3.301\pm0.114$ & $20.710\pm9.839$ & $3.387\pm0.102$ & $21.334\pm1.145$  \\
$s$ & $0.569\pm0.004$ & $0.207\pm0.038$ & $1.747\pm0.011$ & 
 $3.776\pm0.063$ \\
$\alpha$ (radian) & $5.034\pm0.002$ & $2.766\pm0.002$ & $5.036\pm0.003$ & $2.767\pm0.002$\\
$\rho$ ($10^{-3}$) & $2.950\pm0.093$ & $0.481\pm0.278$ & $3.018\pm0.092$ & $0.471\pm0.141$\\ \hline \hline
$\chi^2$ & 11744.7 & 11765.1 & 11745.1 & 11768.0\\
$\Delta\chi^2$ & - & 20.4 &	0.4 & 23.3\\
\enddata
%\tablenotetext{1}{From the best of standard 2L1S (i.e. close1) model.}
\end{deluxetable*}
\end{comment}
%% Figure environment removed
% Figure environment removed
\begin{deluxetable*}{c|ccccccccccc}[t!]
\tablecaption{\textbf{Parameters of the 2L1S + xallarap models, 1L1S + xallarap model, and 1L1S + xallarap + parallax model}
\label{tab:param_xallarap}}
\tablewidth{0pt}
%\tablehead{
%\multicolumn{1}{c|}{Parameters} & \multicolumn{1}{c}{$q\leq0.001$} & \multicolumn{1}{c}{$0.001<q\leq0.01$} & \multicolumn{1}{c}{$0.01<q\leq0.1$} & \multicolumn{1}{c}{$0.1<q\leq1$} 
%}
\tablehead{Model & XLclose1 & XLclose2 & XLwide1 & XLwide2 & 1LXL & 1LXLPL\\
\multicolumn{1}{c|}{range of $q$} & \multicolumn{1}{c}{$q\leq0.1$} & \multicolumn{1}{c}{$0.1<q\leq1$} & \multicolumn{1}{c}{$q\leq0.1$} & \multicolumn{1}{c}{$0.1<q\leq1$} & - & -
}
%\decimalcolnumbers
\startdata
$t_0$(HJD-2458660) & $2.576\pm0.005$ & $2.573\pm0.007$ & $2.572\pm0.004$ & $ 2.575\pm0.006$ & $2.744\pm0.001 $ & $2.744\pm0.001$\\
$t_{\rm E}$ (days) & $97.7\pm2.6$ & $93.8\pm3.1$ & $100.7\pm3.4$ & $133.3\pm11.5$ & $ 67.2\pm2.0$ & $73.5\pm1.5$\\
$u_0$ ($10^{-3}$) & $-7.16\pm0.22$ & $-6.98\pm0.21$ & $7.06\pm0.21$ & $4.91\pm0.35$ & $ 6.93\pm0.20 $ & $6.31\pm0.14$\\
$q$ & $0.09\pm0.01$ & $0.44\pm0.11$ & $0.10\pm0.01$ & $0.94\pm0.37$ & - & -\\
$s$ & $0.141\pm0.004$ & $0.085\pm0.004$ & $7.403\pm0.438$ & $18.040\pm1.263$ & - & -\\
$\alpha$ (radian) & $0.429\pm0.008$ & $1.937\pm0.010$ & $5.846\pm0.008$ & $4.350\pm0.009$ & - & -\\
$\rho$ ($10^{-3}$) & $2.56\pm0.13$ & $2.15\pm0.12$ & $2.41\pm0.11$ & $1.44\pm0.16$ & $7.04 \pm 0.20$ & $6.41\pm0.14$\\
${\rm RA}_\xi$ (degree) & $81.6\pm11.7$ & $153.2\pm10.5$ & $75.9\pm14.5$  & $155.4\pm8.5$ & $ 31.3\pm0.5$ & $32.1\pm0.5$\\
${\rm decl}_\xi$ (degree) & $54.5\pm10.5$ & $36.9\pm16.4$ & $-79.4\pm12.8$ & $-40.7\pm14.1$ & $9.9\pm0.2$ & $9.9\pm0.3$\\
$P_\xi$ (days) & $5.42\pm0.04$ & $5.53\pm0.05$ & $5.43\pm0.04$ & $5.54\pm0.05$ & $2.91\pm0.02$ & $2.9\pm0.02$\\
%$e_\xi$ ($10^{-1}$) &  $5.39\pm0.14$ & $2.08\pm0.24$ & $1.73\pm0.34$ & $\pm$ \\
%$T_{\rm peri}$ (HJD-2458600) & $62.29\pm0.17$ & $65.29\pm26.24$ & $61.76\pm1.72$ & $\pm$ \\
$\xi_{{\rm E,N}}$ ($10^{-3}$) & $1.82\pm0.15$ & $-0.36\pm0.36$ & $-1.65\pm0.11$ & $0.34\pm0.20$ & $-3.59\pm0.11$ & $-3.23\pm0.08$\\
$\xi_{{\rm E,E}}$ ($10^{-3}$) & $0.69\pm0.34$ & $1.53\pm0.12$ & $0.42\pm0.41$ & $1.07\pm0.10$ & $2.86\pm0.09$ & $2.58\pm0.07$\\
$\pi_{{\rm E,N}}$ & - & - & - & - & - & $0.09\pm0.05$\\
$\pi_{{\rm E,E}}$ & - & - & - & - & - & $0.26\pm0.14$\\
\hline \hline
$\chi^2$ & 10856.4 & 10840.9 & 10861.2 & 10842.7 & 11311.5 & 11285.7\\
%$\Delta\chi^2$\tablenotemark{1} & 15.9 & - & 1.3 & \\
$\Delta\chi^2$ & 15.5 &	- & 20.3 & 1.8 & 470.6 & 444.8\\
\enddata
%\tablenotetext{1}{From the best of 2L1S + xallarap model.}
\end{deluxetable*}
\begin{comment}
%20230705
\begin{deluxetable*}{c|ccccccccccc}[t!]
\tablecaption{\textbf{Parameters of the 2L1S + xallarap models, 1L1S + xallarap model, and 1L1S + xallarap + parallax model}
\label{tab:param_xallarap}}
\tablewidth{0pt}
%\tablehead{
%\multicolumn{1}{c|}{Parameters} & \multicolumn{1}{c}{$q\leq0.001$} & \multicolumn{1}{c}{$0.001<q\leq0.01$} & \multicolumn{1}{c}{$0.01<q\leq0.1$} & \multicolumn{1}{c}{$0.1<q\leq1$} 
%}
\tablehead{Model & XLclose1 & XLclose2 & XLwide1 & XLwide2 & 1LXL & 1LXLPL\\
\multicolumn{1}{c|}{range of $q$} & \multicolumn{1}{c}{$q\leq0.1$} & \multicolumn{1}{c}{$0.1<q\leq1$} & \multicolumn{1}{c}{$q\leq0.1$} & \multicolumn{1}{c}{$0.1<q\leq1$} & - & -
}
%\decimalcolnumbers
\startdata
$t_0$(HJD-2458660) & $2.576\pm0.005$ & $2.573\pm0.007$ & $2.572\pm0.004$ & $ 2.575\pm0.006$ & $2.744\pm0.001 $ & $2.744\pm0.001$\\
$t_{\rm E}$ (days) & $97.70\pm2.63$ & $93.75\pm3.09$ & $100.65\pm3.39$ & $133.28\pm11.48$ & $ 67.21\pm2.04$ & $73.46\pm1.50$\\
$u_0$ ($10^{-3}$) & $-7.159\pm0.216$ & $-6.976\pm0.213$ & $7.064\pm0.215$ & $4.915\pm0.350$ & $ 6.927\pm0.196 $ & $6.312\pm0.139$\\
$q$ & $0.094\pm0.006$ & $0.437\pm0.110$ & $0.098\pm0.011$ & $0.943\pm0.370$ & - & -\\
$s$ & $0.141\pm0.004$ & $0.085\pm0.004$ & $7.403\pm0.438$ & $18.040\pm1.263$ & - & -\\
$\alpha$ (radian) & $0.429\pm0.008$ & $1.937\pm0.010$ & $5.846\pm0.008$ & $4.350\pm0.009$ & - & -\\
$\rho$ ($10^{-3}$) & $2.562\pm0.134$ & $2.154\pm0.125$ & $2.409\pm0.111$ & $1.443\pm0.156$ & $7.039 \pm 0.198$ & $6.415\pm0.144$\\
${\rm RA}_\xi$ (degree) & $81.56\pm11.72$ & $153.20\pm10.46$ & $75.86\pm14.53$  & $155.44\pm8.50$ & $ 31.31\pm0.50$ & $32.13\pm0.54$\\
${\rm decl}_\xi$ (degree) & $54.53\pm10.51$ & $36.93\pm16.38$ & $-79.41\pm12.83$ & $-40.69\pm14.05$ & $9.88\pm0.24$ & $9.89\pm0.26$\\
$P_\xi$ (days) & $5.417\pm0.042$ & $5.531\pm0.046$ & $5.433\pm0.035$ & $5.538\pm0.046$ & $2.909\pm0.021$ & $2.89\pm0.02$\\
%$e_\xi$ ($10^{-1}$) &  $5.39\pm0.14$ & $2.08\pm0.24$ & $1.73\pm0.34$ & $\pm$ \\
%$T_{\rm peri}$ (HJD-2458600) & $62.29\pm0.17$ & $65.29\pm26.24$ & $61.76\pm1.72$ & $\pm$ \\
$\xi_{{\rm E,N}}$ ($10^{-3}$) & $1.818\pm0.150$ & $-0.363\pm0.359$ & $-1.646\pm0.106$ & $0.337\pm0.195$ & $-3.592\pm0.106$ & $-3.233\pm0.082$\\
$\xi_{{\rm E,E}}$ ($10^{-3}$) & $0.691\pm0.343$ & $1.533\pm0.118$ & $0.419\pm0.410$ & $1.073\pm0.102$ & $2.860\pm0.089$ & $2.576\pm0.074$\\
$\pi_{{\rm E,N}}$ ($10^{-2}$) & - & - & - & - & - & $9.272\pm5.359$\\
$\pi_{{\rm E,E}}$ ($10^{-1}$) & - & - & - & - & - & $2.576\pm1.375$\\
\hline \hline
$\chi^2$ & 10856.4 & 10840.9 & 10861.2 & 10842.7 & 11311.5 & 11285.7\\
%$\Delta\chi^2$\tablenotemark{1} & 15.9 & - & 1.3 & \\
$\Delta\chi^2$ & 15.5 &	- & 20.3 & 1.8 & 470.6 & 444.8\\
\enddata
%\tablenotetext{1}{From the best of 2L1S + xallarap model.}
\end{deluxetable*}
\end{comment}
\begin{comment}
\begin{deluxetable*}{c|ccccccccccccc}[t!]
\tablecaption{Parameters of the 2L1S + xallarap models 
\label{tab:param_xallarap}}
\tablewidth{0pt}
%\tablehead{
%\multicolumn{1}{c|}{Parameters} & \multicolumn{1}{c}{$q\leq0.001$} & \multicolumn{1}{c}{$0.001<q\leq0.01$} & \multicolumn{1}{c}{$0.01<q\leq0.1$} & \multicolumn{1}{c}{$0.1<q\leq1$} 
%}
\tablehead{Model & XLclose1 & \red{XLclose2} & \red{XLclose3} & XLwide1 & \red{XLwide2} & \red{XLwide3}\\
\multicolumn{1}{c|}{range of $q$} & \multicolumn{1}{c}{$q\leq0.1$} & \multicolumn{1}{c}{\red{$0.1<q\leq0.5$}} & \multicolumn{1}{c}{\red{$0.5<q\leq1$}} & \multicolumn{1}{c}{$q\leq0.1$} & \multicolumn{1}{c}{\red{$0.1<q\leq0.5$}} & \multicolumn{1}{c}{\red{$0.5<q\leq1$}} &  
}
%\decimalcolnumbers
\startdata
$t_0$(HJD-2458660) & $2.576\pm0.005$ & $2.573\pm0.007$ & $2.566\pm0.007$ & $2.572\pm0.004$ & $2.583\pm0.005$ & $ 2.575\pm0.006$\\
$t_{\rm E}$ (days) & $97.70\pm2.63$ & $93.75\pm3.09$ & $95.50\pm2.71$ & $100.65\pm3.39$ & $115.66\pm5.26$ & $133.28\pm11.48$\\
$u_0$ ($10^{-3}$) & $-7.159\pm0.216$ & $-6.976\pm0.213$& $6.880\pm0.186$ & $7.064\pm0.215$ & $5.627\pm0.239$ & $4.915\pm0.350$\\
$q$ & $0.094\pm0.006$ & $0.437\pm0.110$ & $0.632\pm0.169$ & $0.098\pm0.011$ & $0.474\pm0.129$ & $0.943\pm0.370$\\
$s$ & $0.141\pm0.004$ & $0.085\pm0.004$ & $0.079\pm0.003$ &  $7.403\pm0.438$ & $14.535\pm1.151$ & $18.040\pm1.263$\\
$\alpha$ (radian) & $0.429\pm0.008$ & $1.937\pm0.010$ & $4.335\pm0.010$ & $5.846\pm0.008$ & $4.362\pm0.008$ & $4.350\pm0.009$\\
$\rho$ ($10^{-3}$) & $2.562\pm0.134$ & $2.154\pm0.125$ & $2.118\pm0.153$ & $2.409\pm0.111$ & $1.722\pm0.110$ & $1.443\pm0.156$\\
${\rm RA}_\xi$ (degree) & $81.56\pm11.72$ & $153.20\pm10.46$ & $153.54\pm11.83$ & $75.86\pm14.53$ & $161.96\pm15.07$ & $155.44\pm8.50$\\
${\rm decl}_\xi$ (degree) & $54.53\pm10.51$ & $36.93\pm16.38$ & $-36.05\pm16.87$ & $-79.41\pm12.83$ & $-43.57\pm15.12$ & $-40.69\pm14.05$\\
$P_\xi$ (days) & $5.417\pm0.042$ & $5.531\pm0.046$ & $5.552\pm0.050$ & $5.433\pm0.035$ & $5.520\pm0.050$ & $5.538\pm0.046$\\
%$e_\xi$ ($10^{-1}$) &  $5.39\pm0.14$ & $2.08\pm0.24$ & $1.73\pm0.34$ & $\pm$ \\
%$T_{\rm peri}$ (HJD-2458600) & $62.29\pm0.17$ & $65.29\pm26.24$ & $61.76\pm1.72$ & $\pm$ \\
$\xi_{{\rm E,N}}$ ($10^{-3}$) & $1.818\pm0.150$ & $-0.363\pm0.359$ & $0.272\pm0.369$ & $-1.646\pm0.106$ & $0.643\pm0.241$ & $0.337\pm0.195$\\
$\xi_{{\rm E,E}}$ ($10^{-3}$) & $0.691\pm0.343$ & $1.533\pm0.118$ & $1.524\pm0.109$ & $0.419\pm0.410$ & $1.139\pm0.224$ & $1.073\pm0.102$\\
\hline \hline
$\chi^2$ & 10856.4 & 10840.9 & 10842.3 & 10861.2 & 10843.4 & 10842.7\\
%$\Delta\chi^2$\tablenotemark{1} & 15.9 & - & 1.3 & \\
$\Delta\chi^2$ & 15.5 &	- & 1.3 & 20.3 & 2.4 & 1.8\\
\enddata
%\tablenotetext{1}{From the best of 2L1S + xallarap model.}
\end{deluxetable*}
\end{comment}
\begin{comment}
\begin{deluxetable*}{c|ccccccccccc}[t!]
\tablecaption{Parameters of the 2L1S + xallarap wide models 
\label{tab:param_xallarap_wide}}
\tablewidth{0pt}
\tablehead{Model & XLwide1 & XLwide2 & XLwide3 & XLwide4\\
\multicolumn{1}{c|}{range of $q$} & \multicolumn{1}{c}{$q\leq0.05$} & \multicolumn{1}{c}{$0.05<q\leq0.1$} & \multicolumn{1}{c}{$0.1<q\leq0.5$} & \multicolumn{1}{c}{$0.5<q\leq1$} 
}
%\decimalcolnumbers
\startdata
$t_0$(HJD-2458660) & $2.669\pm0.011$ & $2.656\pm0.003$ & $2.583\pm0.003$ & 	$2.574\pm0.004$\\
$t_{\rm E}$ (days) & $98.1\pm2.1$ & $98.8\pm3.1$ & $115.7\pm1.5$ & $131.5\pm5.5$ \\
$u_0$ ($10^{-3}$) & $-6.49\pm0.14$ &  $-6.33\pm0.19$ & $5.62\pm0.06$ & $5.00\pm0.21$\\
$q$ & $0.042\pm0.002$ & $0.092\pm0.008$ & $0.474\pm0.048$ & $0.915\pm0.172$  \\
$s$ & $5.520\pm0.170$ & $8.004\pm0.465$ & $14.535\pm0.368$ & 
 $17.777\pm0.731$ \\
$\alpha$ (radian) & $0.324\pm0.020$ & $0.303\pm0.010$ & $4.362\pm0.006$ & $4.349\pm0.006$\\
$\rho$ ($10^{-3}$) & $2.61\pm0.19$ & $2.56\pm0.15$ & $1.72\pm0.08$ & $1.47\pm0.05$\\
${\rm RA}_\xi$ (degree) & $72.0\pm1.7$ & $251.3\pm1.6$ & $162.0\pm8.8$ & $153.7\pm9.3$\\
${\rm decl}_\xi$ (degree) & $10.2\pm1.4$ & $10.8\pm0.8$ & $-43.6\pm6.7$ & $-35.4\pm7.5$\\
$P_\xi$ (days) & $5.317\pm0.059$ & $5.412\pm0.046$ & $5.520\pm0.040$ & $5.536\pm0.032$\\
%$e_\xi$ ($10^{-1}$) & $5.34\pm0.06$ & $5.38\pm0.08$ & $2.07\pm0.21$ & $1.54\pm0.29$\\
%$T_{\rm peri}$ (HJD-2458600) & $68.35\pm0.34$ & $51.95\pm0.33$ & $68.31\pm48.24$ & $63.56\pm1.48$\\
$\xi_{{\rm E,N}}$ ($10^{-3}$) & $4.898\pm0.329$ & $-4.477\pm0.231$ & $0.643\pm0.236$ & $0.323\pm0.192$\\
$\xi_{{\rm E,E}}$ ($10^{-3}$) & $1.672\pm0.193$ & $-1.535\pm0.101$ & $1.139\pm0.111$ & $1.105\pm0.105$\\
\hline \hline
$\chi^2$\tablenotemark{1} & 10869.9 & 10863.1 & 10843.4 & 10843.1\\
$\Delta\chi^2$ & 28.9 & 22.2 & 2.4 & 2.1\\
\enddata
\tablenotetext{1}{From the best of 2L1S + xallarap model.}
%\tablenotetext{1}{From the best of 2L1S + xallarap model.}
\end{deluxetable*}
\begin{deluxetable*}{c|ccccccccccc}[t!]
\tablecaption{Parameters of the 2L1S + xallarap close models 
\label{tab:param_xallarap_close}}
\tablewidth{0pt}
%\tablehead{
%\multicolumn{1}{c|}{Parameters} & \multicolumn{1}{c}{$q\leq0.001$} & \multicolumn{1}{c}{$0.001<q\leq0.01$} & \multicolumn{1}{c}{$0.01<q\leq0.1$} & \multicolumn{1}{c}{$0.1<q\leq1$} 
%}
\tablehead{Model & XLclose1 & XLclose2 & XLclose3 & XLclose4\\
\multicolumn{1}{c|}{range of $q$} & \multicolumn{1}{c}{$q\leq0.05$} & \multicolumn{1}{c}{$0.05<q\leq0.1$} & \multicolumn{1}{c}{$0.1<q\leq0.5$} & \multicolumn{1}{c}{$0.5<q\leq1$} 
}
%\decimalcolnumbers
\startdata
$t_0$(HJD-2458660) & $2.664\pm0.008$ & $2.568\pm0.002$ & $2.573\pm0.006$ & $2.566\pm0.004$\\
$t_{\rm E}$ (days) & $92.5\pm2.9$ & $98.0\pm1.6$ & $93.75\pm2.79$ & $95.5\pm2.4$ \\
$u_0$ ($10^{-3}$) & $6.87\pm0.20$ &  $-7.18\pm0.12$ & $-6.98\pm0.20$ & $6.88\pm0.17$\\
$q$ & $0.047\pm0.003$ & $0.099\pm0.81$ & $0.437\pm0.096$ & $0.632\pm0.080$  \\
$s$ & $0.179\pm0.007$ & $0.139\pm0.005$ & $0.085\pm0.005$ & 
 $0.079\pm0.002$ \\
$\alpha$ (radian) & $5.964\pm0.013$ & $0.436\pm0.004$ & $1.937\pm0.009$ & $4.335\pm0.007$\\
$\rho$ ($10^{-3}$) & $2.70\pm0.15$ & $2.54\pm0.07$ & $2.15\pm0.14$ & $2.12\pm0.11$\\
${\rm RA}_\xi$ (degree) & $71.4\pm1.5$ & $91.5\pm8.9$ & $153.2\pm8.6$ & $153.5\pm4.1$\\
${\rm decl}_\xi$ (degree) & $-9.7\pm8.9$ & $72.6\pm10.7$ & $36.9\pm15.4$ & $-36.0\pm9.6$\\
$P_\xi$ (days) & $5.386\pm0.035$ & $5.438\pm0.022$ & $5.531\pm0.052$ & $5.55\pm0.05$\\
%$e_\xi$ ($10^{-1}$) & $5.66\pm0.26$ & $5.39\pm0.14$ & $2.08\pm0.24$ & $1.73\pm0.34$\\
%$T_{\rm peri}$ (HJD-2458600) & $60.38\pm0.32$ & $62.29\pm0.17$ & $65.29\pm26.24$ & $61.76\pm1.72$\\
$\xi_{{\rm E,N}}$ ($10^{-3}$) & $-5.279\pm0.314$ & $1.472\pm0.132$ & $-0.363\pm0.328$ & $0.272\pm0.166$\\
$\xi_{{\rm E,E}}$ ($10^{-3}$) & $1.765\pm0.139$ & $0.870\pm0.224$ & $1.533\pm0.093$ & $1.524\pm0.086$\\
\hline \hline
$\chi^2$ & 10866.1 & 10856.9 & 10842.5 & 10842.3\\
%$\Delta\chi^2$\tablenotemark{1} & 25.2 & 15.9 &	- & 1.3\\
$\Delta\chi^2$ & 25.2 & 15.9 &	- & 1.3\\
\enddata
%\tablenotetext{1}{From the best of 2L1S + xallarap model.}
\end{deluxetable*}
\end{comment}
\begin{comment}
\begin{deluxetable*}{c|ccccccccccc}[t!]
\tablecaption{Parameters of the 2L1S + xallarap wide models 
\label{tab:param_xallarap_wide}}
\tablewidth{0pt}
\tablehead{Model & XLwide1 & XLwide2 & XLwide3 & XLwide4\\
\multicolumn{1}{c|}{range of $q$} & \multicolumn{1}{c}{$q\leq0.001$} & \multicolumn{1}{c}{$0.001<q\leq0.01$} & \multicolumn{1}{c}{$0.01<q\leq0.1$} & \multicolumn{1}{c}{$0.1<q\leq1$} 
}
%\decimalcolnumbers
\startdata
$t_0$(HJD-2458660) & $2.744\pm0.001$ & $2.125\pm0.012$ & $2.670\pm0.016$ & 	$2.597\pm0.011$\\
$t_{\rm E}$ (days) & $56.3\pm0.8$ & $114.6\pm3.2$ & $98.0\pm2.1$ & $105.1\pm2.0$ \\
$u_0$ ($10^{-3}$) & $-9.14\pm0.03$ &  $9.82\pm0.26$ & $-7.32\pm0.13$ & $6.78\pm0.24$\\
$q$ ($10^{-3}$) & $0.774\pm0.022$ & $6.51\pm0.33$ & $29.2\pm2.1$ & $114\pm30$  \\
$s$ & $1.206\pm0.001$ & $2.319\pm0.036$ & $4.362\pm0.169$ & 
 $8.286\pm1.059$ \\
$\alpha$ (radian) & $5.835\pm0.003$ & $5.542\pm0.014$ & $0.340\pm0.025$ & $0.376\pm0.021$\\
$\rho$ ($10^{-3}$) & $9.41\pm0.08$ & $4.35\pm0.13$ & $2.26\pm0.16$ & $2.07\pm0.15$\\
${\rm RA}_\xi$ (degree) & $31.8\pm0.6$ & $86.6\pm3.2$ & $73.2\pm11.2$ & $62.0\pm17.6$\\
${\rm decl}_\xi$ (degree) & $4.5\pm0.3$ & $-26.1\pm2.3$ & $25.0\pm3.6$ & $47.7\pm5.3$\\
$P_\xi$ (days) & $4.65\pm0.03$ & $5.38\pm0.04$ & $5.40\pm0.05$ & $5.51\pm0.05$\\
$e_\xi$ ($10^{-1}$) & $5.34\pm0.06$ & $5.38\pm0.08$ & $2.07\pm0.21$ & $1.54\pm0.29$\\
$T_{\rm peri}$ (HJD-2458600) & $62.54\pm0.01$ & $67.69\pm0.04$ & $62.68\pm0.13$ & $57.43\pm0.16$\\
$\xi_{{\rm E,N}}$ ($10^{-3}$) & $3.16\pm0.02$ & $-4.33\pm0.16$ & $3.16\pm0.23$ & $1.78\pm0.22$\\
$\xi_{{\rm E,E}}$ ($10^{-3}$) & $2.90\pm0.10$ & $-1.09\pm0.13$ & $1.23\pm0.13$ & $0.91\pm0.38$\\
\hline \hline
$\chi^2$ & 10829.1 & 10831.6 & 10824.7 & 10830.5\\
$\Delta\chi^2$ & 4.4 & 6.9 & - & 5.8\\
\enddata
%\tablenotetext{1}{From the best of 2L1S + xallarap model.}
\end{deluxetable*}
\begin{deluxetable*}{c|ccccccccccc}[t!]
\tablecaption{Parameters of the 2L1S + xallarap close models 
\label{tab:param_xallarap_close}}
\tablewidth{0pt}
%\tablehead{
%\multicolumn{1}{c|}{Parameters} & \multicolumn{1}{c}{$q\leq0.001$} & \multicolumn{1}{c}{$0.001<q\leq0.01$} & \multicolumn{1}{c}{$0.01<q\leq0.1$} & \multicolumn{1}{c}{$0.1<q\leq1$} 
%}
\tablehead{Model & XLclose1 & XLclose2 & XLclose3 & XLclose4\\
\multicolumn{1}{c|}{range of $q$} & \multicolumn{1}{c}{$q\leq0.001$} & \multicolumn{1}{c}{$0.001<q\leq0.01$} & \multicolumn{1}{c}{$0.01<q\leq0.1$} & \multicolumn{1}{c}{$0.1<q\leq1$} 
}
%\decimalcolnumbers
\startdata
$t_0$(HJD-2458660) & $2.791\pm0.007$ & $2.860\pm0.007$ & $2.670\pm0.015$ & $2.605\pm0.014$\\
$t_{\rm E}$ (days) & $83.1\pm2.2$ & $71.0\pm1.6$ & $96.0\pm2.2$ & $98.5\pm2.9$ \\
$u_0$ ($10^{-3}$) & $6.56\pm0.18$ &  $-7.30\pm0.15$ & $-7.46\pm0.26$ & $7.17\pm0.21$\\
$q$ ($10^{-3}$) & $0.955\pm0.179$ & $6.85\pm0.82$ & $27.4\pm3.9$ & $103\pm16$  \\
$s$ & $0.586\pm0.020$ & $0.429\pm0.020$ & $0.239\pm0.018$ & 
 $0.132\pm0.008$ \\
$\alpha$ (radian) & $4.068\pm0.021$ & $6.134\pm0.019$ & $0.343\pm0.026$ & $0.362\pm0.021$\\
$\rho$ ($10^{-3}$) & $1.10\pm0.46$ & $4.42\pm0.28$ & $2.31\pm0.18$ & $2.22\pm0.16$\\
${\rm RA}_\xi$ (degree) & $67.6\pm10.6$ & $260.5\pm4.2$ & $83.4\pm10.4$ & $245.3\pm16.1$\\
${\rm decl}_\xi$ (degree) & $-50.7\pm7.5$ & $28.2\pm3.5$ & $25.4\pm3.4$ & $44.3\pm12.2$\\
$P_\xi$ (days) & $5.48\pm0.05$ & $5.36\pm0.04$ & $5.44\pm0.06$ & $5.50\pm0.05$\\
$e_\xi$ ($10^{-1}$) & $5.66\pm0.26$ & $5.39\pm0.14$ & $2.08\pm0.24$ & $1.73\pm0.34$\\
$T_{\rm peri}$ (HJD-2458600) & $62.36\pm0.02$ & $56.97\pm0.04$ & $62.56\pm0.11$ & $51.80\pm0.18$\\
$\xi_{{\rm E,N}}$ ($10^{-3}$) & $-2.51\pm0.16$ & $-3.75\pm0.13$ & $3.18\pm0.23$ & $-1.98\pm0.34$\\
$\xi_{{\rm E,E}}$ ($10^{-3}$) & $0.90\pm0.44$ & $-2.55\pm0.23$ & $1.26\pm0.13$ & $-0.85\pm0.29$\\
\hline \hline
$\chi^2$ & 10836.8 & 10831.4 & 10825.3 & 10830.7\\
$\Delta\chi^2$\tablenotemark{1} & 12.2 & 6.8 &	0.6 & 6.0\\
\enddata
\tablenotetext{1}{From the best of 2L1S + xallarap model.}
\end{deluxetable*}
\end{comment}
\subsection{Parallax}
\label{subsec:Parallax}
It is known that the acceleration of Earth orbital motion affects the light curve of microlensing events \citep{Gould1992,Gould2004,Smith+2003,Dong+2009}.  
This parallax effect can be described by the microlensing parallax vector $\bm{\pi_{\rm E}} = (\pi_{{\rm E,N}}, \pi_{{\rm E,E}})$ where $\pi_{{\rm E,N}}$ and $\pi_{{\rm E,E}}$ represents respectively the north and east components of $\bm{\pi_{\rm E}}$ projected onto the sky plane in equatorial coordinates. 
The direction of $\bm{\pi_{\rm E}}$ is defined to coincide with the direction of the geocentric lens-source relative proper motion projected onto the sky plane at the reference time $t_{\rm fix}$, and the amplitude of $\bm{\pi_{\rm E}}$ is $\pi_{\rm E}={\rm au}/\tilde r_{\rm E}$ ($\tilde r_{\rm E}$ is the Einstein radius projected inversely to the observation plane) \citep{Gould2000}.
%The direction of $\bm{\pi_{\rm E}}$ is defined to coincide with the direction of the geocentric lens-source relative proper motion projected onto the sky plane at the reference time $t_{\rm fix}$, and the amplitude of $\bm{\pi_{\rm E}}$ is $\pi_{\rm E}={\rm au}/\tilde {r_{\rm E}}$ ($\tilde {r_{\rm E}}$ is the Einstein radius projected inversely to the observation plane) \citep{Gould2000}.

As a result of modeling by adding two parameters of $\pi_{{\rm E,N}}$ and $\pi_{{\rm E,E}}$, we found two degenerate models with $(q,s)=(3.5\times10^{-3},0.57)$ and $(q,s)=(3.4\times10^{-3},1.74)$, that are better than the standard 2L1S model by $\Delta\chi^2=68.3$.
%As a result of modeling by adding two parameters of $\pi_{{\rm E},N}, \pi_{{\rm E},E}$, we found that $(q,s)=(3.5\times10^{-3},0.57)$ is the better solution than the best standard 2L1S model by $\Delta\chi^2\sim68.3$.
%We also found a local solution for the $s,1/s$ degeneracy, $ (q,s)=(3.4\times10^{-3},1.74)$.
%However, these parallax models were not consistent for each data set. 
%However, the cumulative $\Delta\chi^{2}$ for each of these parallax and standard 2L1S model for each data set shows that the improvement in $\chi^{2}$ for the parallax model is not consistent across data sets.
However, the cumulative $\Delta\chi^{2}$ improvement for parallax model relative to standard 2L1S model is not consistent between the data sets.
%The MOA-$I$, MOA-$V$, OGLE-$I$, KMT-CTIOf01-$I$, KMT-CTIOf41-$I$, and Danish-$I$ data contribute almost nothing to the $\chi^{2}$ improvement of the 2L1S parallax models.
%Furthermore, we found that the 2L1S + xallarap models with $\Delta\chi^{2} \leq 9$ from the best 2L1S + xallarap model in Section \ref{subsec:Xallarap} have $50 < t_{\rm E}$ [days]$ < 135$, $0.06 < \theta_{\rm E}$ [mas] $< 0.41$ and $0.35 < \mu_{\rm {rel}}$ [$\rm{mas/yr}$] $<1.2$. 
%We calculated the prior $\bm{\pi_{\rm{E}}}$ distribution of events with those parameters using the Galactic model by \citet{Koshimoto+2021b}. 
%%%%%%%%%%%%%%%%%%%%%%%%%%%%%%%%To remove events that have too large $\bm{\pi_{\rm {E}}}$ values, we limit the fitting range in the 2 $\sigma$ limit of the prior distribution of 0.024 $< \pi_{\rm {E}} <$ 0.351.
%The $\pi_{\rm {E}}$ of these 2L1S + parallax models was larger than the 2 $\sigma$ limit of this $\bm{\pi_{\rm {E}}}$ prior distribution.
%Therefore, we exclude these 2L1S + parallax models. 
%Furthermore, we still found systematic residuals around the peak of $8657<{\rm HJD}^\prime<8667$ in these models, as seen in the standard 2L1S model in Figure~\ref{fig:lightcurve}.
Furthermore, we still found systematic residuals around the peak of $8657<{\rm HJD}^\prime<8667$ in these models, as seen in the standard 2L1S model shown by the orange solid line in Figure~\ref{fig:lightcurve}.
%Therefore, we attempted to better model the light curve by adding other higher order effects.
%Therefore, we will model the light curve with other higher order effects.

\subsection{Xallarap}
\label{subsec:Xallarap}
We next consider the possibility that the short term residuals in $8657<{\rm HJD}^\prime<8667$ are caused by a short period binary source system, i.e., they arise owing to the xallarap effect.
%We found short-term residuals near the $8661<{\rm HJD}-245000<8665$ light-curve peak that cannot be explained by the Standard 2L1S modeling. 
%\citet{Miyazaki+2021} simulated xallarap events and showed that xallarap events show a short-period signal around $t_0$. 
%We therefore examined the possibility that the residuals near the peak for this event is due to xallarap.

The xallarap effect can be described by the following seven parameters; the direction toward the solar system relative to the orbital plane of the source system, ${\rm RA}_\xi$ and ${\rm decl}_\xi$; the source orbital period, $P_\xi$; the source orbital eccentricity $e_\xi$; the perihelion $T_{\rm peri}$; the xallarap vector, $\bm{\xi_{\rm E}} = (\xi_{{\rm E,N}}, \xi_{{\rm E,E}})$. 
Note that this effect does not include the magnifying effect of the source companion star; only the source host contributes to the magnification.
We denote this model of the microlensing event as the 2L1S + xallarap model rather than as the 2L2S model to distinguish it from a model including secondary source magnification.
As discussed in detail in Section~\ref{sec:Source_Lens_Properties}, the flux ratio of the source companion to the host star in the $I$-band in the best 2L1S + xallarap model is $\sim10^{-7}$.
Therefore, we assume that the brightening of the source companion star is negligible.

%We first performed model fitting with 78,960 grids of $({\rm RA}_\xi, {\rm decl}_\xi, P_\xi)$ each, using the standard 2L1S models with best-fit solution and three local solutions (i.e., close1, wide1, close2, and wide2) as initial values.  
We first fit using 78,960 values of xallarap parameters $({\rm RA}_\xi, {\rm decl}_\xi, P_\xi)$ with the four best standard 2L1S models (close1, wide1, close2, and wide2) as initial values.
%We used 20 values of ${\rm RA}_\xi$ for $0\leq{\rm RA}_\xi<360$, 21 values of ${\rm decl}_\xi$ for $-90\leq{\rm decl}_\xi<90$, 19 values and 99 values of $P_\xi$ for $1<P_\xi$ [days] $<19$ and $20<P_\xi$ [days] $<1000$, respectively.
We used 20 evenly spaced values for $0\leq{\rm RA}_\xi<360$, 21 values for $-90\leq{\rm decl}_\xi<90$, 19 and 99 values for $1<P_\xi$ [days] $<19$ and $20<P_\xi$ [days] $<1000$, respectively.
%After that, $({\rm RA}_\xi, {\rm decl}_\xi, P_\xi)$  were set to freely vary and fitted again.
After that, we fit again with $({\rm RA}_\xi, {\rm decl}_\xi, P_\xi)$ as free parameters.
%As a result, $P_\xi\sim5$ days was obtained independently from the initial values of close1, wide1, close2, and wide2 of standard 2L1S. 
%However, the values of $q$ and $s$ were quite different from the initial values, respectively, and did not converge.
%Thus, we expect the $s-q$ $\Delta \chi^2$ plane of the standard 2L1S model and the $s-q$ $\Delta \chi^2$ plane of the 2L1S + xallarap model to have different shapes.
As a result, we found the best solutions with $P_\xi\sim5$ days independently from the initial values of close1, wide1, close2, and wide2.
We also found that the final $q$ and $s$ values are quite different from their initial values, and did not converge. 
%Therefore, we next set $P_\xi\sim5$ days as the initial value, ${\rm RA}_\xi$ and ${\rm decl}_\xi$ to random values, and performed model fitting on $(q,s,\alpha)$ 34,440 grids using the same procedure as the standard 2L1S modeling described in Section \ref{subsec:Standard_Binary_Lens}. 
Therefore, we next set $P_\xi\sim5$ days as the initial value, ${\rm RA}_\xi$ and ${\rm decl}_\xi$ to random values, and performed model fitting with 34,440 values of $(q,s,\alpha)$ using the same procedure as the standard 2L1S modeling described in Section~\ref{subsec:Standard_Binary_Lens}.
%It is reasonable to assume that short-period binary stars orbiting in $P_\xi\sim5$ days are affected by Kozai cycles with tidal friction \citep{Fabrycky+2007}.
%Short-period binary stars orbiting at $P_\xi\sim5$ days are likely to be affected by the Kozai cycle with tidal friction \citep{Fabrycky+2007}.
%Short-period binary stars orbiting in $P_\xi\sim5$ days are affected by palaeo-periods with tidal friction \citep{Fabrycky+2007}.
Short-period binary stars orbiting in $P_\xi\sim5$ days are affected by orbital circularization due to tidal friction \citep{Fabrycky+2007}.
%Although there is ongoing debate about tidal circularization times, it is not unreasonable to assume that at the age of the stars in the Galactic bulge \citep{Sit+2020}, the orbits are fully circularized.
The tidal circularization time is discussed in Section~\ref{sec:Discussion_and_Conclusion}, but it is reasonable to assume that at the age of the stars in the Galactic bulge \citep{Sit+2020}, the orbit is fully circularized.
%In particular, at the ages of stars in the Galactic bulge \citep{Sit+2020}, the orbits are considered to be fully circularized.
Therefore, we fixed the eccentricity at $e_\xi=0$.
When $e_{\xi}=0$, $T_{\rm peri}$ can be eliminated as a fitting parameter.
The results are shown in the right panel of Figure~\ref{fig:Grid_search}. 

%The figure shows that there are degenerate solutions with various combination of $(q,s)$ values within a small range of $\Delta\chi^2$.
The figure shows that there are degenerate solutions for various combinations of $(q,s)$ values in the range of $\Delta\chi^2\lesssim20$.
Table~\ref{tab:param_xallarap} shows the best fit model parameters for the wide and close solutions.
%We label the best models of the mass ratio range in the 2L1S + xallarap wide model, respectively: the best with $q\leq0.001$ is XLwide1, the best with $0.001<q\leq0.01$ is XLwide2, the best with $0.01<q\leq0.1$ is XLwide3, the best with $0.1<q\leq1$ best is XLwide4.
%Similarly, in the close model of 2L1S + xallarap, we label the best with $q\leq0.001$ as XLclose1, the best with $0.001<q\leq0.01$ as XLclose2, the best with $0.01<q\leq0.1$ as XLclose3, and the best with $0.1<q\leq1$ as XLclose4.
The reason for the slight difference in $\Delta\chi^2$ between Figure~\ref{fig:Grid_search} and Table~\ref{tab:param_xallarap} is that the models in Table~\ref{tab:param_xallarap} were fitted with $q$, $s$, and $\alpha$ set free.
We label the best models of the mass ratio range in the 2L1S + xallarap close model, respectively: the best with $q\leq0.1$ is XLclose1, the best with $0.1<q\leq1$ is XLclose2.
Similarly, in the wide model of 2L1S + xallarap, we label the best with $q\leq0.1$ as XLwide1, the best with $0.1<q\leq1$ as XLwide2.
Figure~\ref{fig:lightcurve} shows the best 2L1S + xallarap model (i.e. XLclose2).
The xallarap models fit the light curves better than the standard 2L1S models.
\begin{comment}
%The results show that various $(q,s)$ models exist within a small $\Delta\chi^2$ difference from the best fit.
The figure shows that there are degenerate solutions with various combination of $(q,s)$ values within a small range of $\Delta\chi^2$.
%Table \ref{tab:param_xallarap_wide} and Table \ref{tab:param_xallarap_close} show the best model parameters in the parameter space when separated by mass ratio $q$ and separation $s$.
Table~\ref{tab:param_xallarap_wide} and Table~\ref{tab:param_xallarap_close} show the best fit model parameters in the ranges of $q\leq0.001$, $0.001<q\leq0.01$, $0.01<q\leq0.1$ and $0.1<q\leq1$ for wide and close solutions, respectively. 
We label the best models of the mass ratio range in the 2L1S + xallarap wide model, respectively: the best with $q\leq0.001$ is XLwide1, the best with $0.001<q\leq0.01$ is XLwide2, the best with $0.01<q\leq0.1$ is XLwide3, the best with $0.1<q\leq1$ best is XLwide4.
Similarly, in the close model of 2L1S + xallarap, we label the best with $q\leq0.001$ as XLclose1, the best with $0.001<q\leq0.01$ as XLclose2, the best with $0.01<q\leq0.1$ as XLclose3, and the best with $0.1<q\leq1$ as XLclose4.
%Figure \ref{fig:lightcurve} shows the best model with 2L1S + xallarap and the residuals from the model.
%Figure \ref{fig:lightcurve} shows the best 2L1S + xallarap model and the residuals from the model.
%Figure 1 shows the best 2L1S + xallarap model.
%Figure \ref{fig:lightcurve} shows the best 2L1S + xallarap model, that is, the model with $0.01<q\leq0.1$ in wide.
Figure~\ref{fig:lightcurve} shows the best 2L1S + xallarap model (i.e. XLwide3).
%The models which include xallarap are well fitted to the residuals, which was not the case with the standard binary model.
%The best xallarap models fit the light curve better than the standard binary model.
The xallarap models fit the light curves better than the standard 2L1S models.
\end{comment}

%Figure \ref{fig:lc3} shows the cumulative $\Delta \chi^2$ of the best standard 2L1S model and the best 2L1S + xallarap model. 
Figure~\ref{fig:lc3} shows the cumulative $\Delta \chi^2$ of the best 2L1S + xallarap model relative to the best standard 2L1S model.
%It can be seen that the 2L1S + xallarap model effect improves $\Delta \chi^2$ in the high magnification region.
One can see that the 2L1S + xallarap model improves $\chi^2$ around the peak of $8657<{\rm HJD}^\prime<8667$.
The 2L1S + xallarap model improved $\chi^2$ by $903.7$ from the standard 2L1S model and by $835.5$ from the 2L1S + parallax model. 
%Figure \ref{fig:caustic} shows the lensing system configuration for the best 2L1S + xallarap model.
Figure~\ref{fig:caustic} shows the geometry of the primary lens, source trajectory, caustics on the magnification map for the best 2L1S + xallarap model.
%The short orbital period of the source star, $P_\xi\sim5$ days, causes the source to travel on a wavy line in a region of high magnification.
%The short orbital period of the source star with $P_\xi\sim5$ days make the source's trajectory a wavy line.
The short orbital period of the source star with $P_\xi\sim5$ days make the source's trajectory a wavy line.

%We followed the same procedure for 1L1S and found that the asymmetric magnification map with a binary lens reproduce the anomaly better than the polar magnification map with a single lens.
%As a result, the best model for 1L1S + xallarap was $\Delta\chi^2=359.5$ worse than the best model for 2L1S + xallarap.
We applied the same procedure for 1L1S and found the best 1L1S + xallarap model has $\Delta\chi^2=470.6$ worse than the best 2L1S + xallarap model.
We label the best 1L1S + xallarap model as 1LXL.
Even the 1L1S + xallarap + parallax model was $\Delta\chi^2=444.8$ worse than the best 2L1S + xallarap model.
We label the best 1L1S + xallarap + parallax model as 1LXLPL.
The parameters of each of the best models are listed in Table~\ref{tab:models}.
However, asymmetric maps similar to Figure~\ref{fig:caustic} can be created by binary lenses of various parameters, which led to the emergence of various degenerate 2L1S + xallarap models.
%We conclude that the best model in this analysis is 2L1S + xallarap at 0.01 $\leq$ $q$ $\leq$ 0.1 in wide (i.e. XLwide3).
%We considered other higher-order effects and combinations of them, but could not detect them significantly.
We considered other higher order effects and combinations of them such as 2L1S + xallarap + parallax, 2L1S + xallarap + parallax + lens orbital motion, and 1L2S, but could not detect them significantly.
%We also performed a fitting using the best standard 2L1S model (i.e. close1) where the source is a variable.
%We also fitted the best standard 2L1S model (i.e., close1) assuming the source is a variable star.
%In this case, the amplitude of the variation $\alpha$, the period of the variation  $T_{\rm v}$ and the initial phase $\beta$ are additional parameters.
%However, the $\chi^2$ improvement from the best standard 2L1S model was $139.1$, $\Delta \chi^2 \sim 764.6$ worse than the best 2L1S + xallarap model.
For comparison with the 2L1S + xallarap model, we also fitted the 2L1S model with a variable source.
%For comparison with the 2L1S + xallarap model, we also fitted the 2L1S model with a variable source as spotted stars \citep{Iwanek+2019}.
In this case, the amplitude of the variation, $\gamma$; the period of the variation, $T_{\rm v}$; and the initial phase, $\beta$ are additional parameters.
%In this case as spotted stars \citep{Iwanek+2019}, the amplitude of the variation, $\gamma$; the period of the variation, $T_{\rm v}$; and the initial phase, $\beta$ are additional parameters.
%For the other parameters, we used the parameters of the best standard 2L1S model (i.e., close1).
We fixed the other parameters at those of the best standard 2L1S model (i.e., close1).
However, the $\chi^2$ improvement from the best standard 2L1S model was only $139.1$, $\Delta \chi^2 = 764.6$ worse than the best 2L1S + xallarap model.
To confirm, we performed 2L1S + xallarap fitting analysis with $\xi_{\rm E,N}$, $\xi_{\rm E,E}$, ${\rm RA}_\xi$, ${\rm decl}_\xi$, and $P_\xi$ set free and the other parameters fixed to the best standard 2L1S model.
As a result, the $\chi^2$ was improved by $594.5$ over the best standard 2L1S model.
This is only $\Delta \chi^2 = 309.3$ worse than the best 2L1S + xallarap model.
That is, for two models (2L1S + xallarap and 2L1S + variable source) with the same fixed lens parameters, the 2L1S + xallarap model has $455.3$ better $\chi^2$ than the 2L1S + variable source model.
\begin{comment}
\textbf{
To further compare with the 2L1S + xallarap model, we performed a fitting with the following model equation assuming that the source star of the best standard 2L1S model (i.e. close1) is a variable star,
}
\begin{equation}\label{eq:variable}
    f_{\rm v}=f_{\rm {2L1S}}+A_{\rm {2L1S}} f_{s, \rm{2L1S}} (10^{-0.4 \alpha \sin {(\frac{2 \pi t}{T_{\rm v}} + \beta)}}-1).
\end{equation}
\end{comment}
Finally, we conclude that the best model in this analysis is XLclose2.
%Also, the xallarap signal is consistent and has little influence on our conclusions, even when additional higher-order effects are considered.
In addition, the xallarap signal is consistent, and considering additional higher order effects on 2L1S + xallarap has little influence on our conclusions.
%The systematic residuals around $8661 < \rm {HJD'}< 8665$ that were in Figure \ref{fig:lc2} have disappeared.
%
%
% Figure environment removed
%
\begin{comment}
\begin{deluxetable*}{c|ccccccccccc}[t!]
\tablecaption{Parameters of the 2L1S + xallarap + paralax + lens orbital motion models 
\label{tab:param_X+P+LOM}}
\tablewidth{0pt}
\tablehead{
\multicolumn{1}{c|}{Parameters} & \multicolumn{1}{c}{$q\leq0.005$} & \multicolumn{1}{c}{$0.005<q\leq0.01$} & \multicolumn{1}{c}{$0.01<q\leq0.5$} & \multicolumn{1}{c}{$0.5<q\leq1$} 
}
\decimalcolnumbers
\startdata
$t_0$(HJD-2458660) & $2.654\pm0.016$ & $2.592\pm0.006$ & $2.565\pm0.003$ & $2.578\pm0.008$\\
$t_{\rm E}$ (days) & $98.1\pm1.6$ & $106.7\pm1.4$ & $109.2\pm3.5$ %& $108.3\pm3.2$ \\
$u_0$ ($10^{-3}$) & $-7.29\pm0.18$ &  $-7.08\pm0.07$ & $-6.42\pm0.21$ & $6.43\pm0.17$\\
$q$ ($10^{-2}$) & $3.14\pm0.34$ & $7.05\pm0.17$ & $47.2\pm1.7$ & $56.1\pm5.2$  \\
$s$ & $0.222\pm0.007$ & $0.156\pm0.002$ & $0.080\pm0.002$ & 
 $0.077\pm0.002$ \\
$\alpha$ (radian) & $0.359\pm0.017$ & $0.359\pm0.087$ & $1.887\pm0.006$ & $4.413\pm0.014$\\
$\rho$ ($10^{-3}$) & $2.27\pm0.15$ & $2.21\pm0.11$ & $1.90\pm0.11$ & $1.93\pm0.10$\\
$\pi_{{\rm E},N}$ ($10^{-1}$) & $2.43\pm0.59$ & $-0.25\pm0.21$ & $2.86\pm0.40$ & $2.76\pm0.68$\\
$\pi_{{\rm E},E}$ ($10^{-1}$) & $1.08\pm0.51$ & $-0.29\pm0.27$ & $-1.95\pm0.35$ & $-1.52\pm0.73$\\
${\rm R.A.}_\xi$ (degree) & $86.2\pm11.1$ & $50.1\pm2.2$ & $63.6\pm3.0$ & $82.2\pm10.9$\\
${\rm decl.}_\xi$ (degree) & $28.0\pm3.8$ & $38.5\pm1.3$ & $71.9\pm11.8$ & $-42.5\pm6.4$\\
$P_\xi$ (days) & $5.47\pm0.04$ & $5.46\pm0.04$ & $5.51\pm0.04$ & $5.53\pm0.05$\\
$e_\xi$ ($10^{-1}$) & $1.94\pm0.13$ & $2.07\pm0.30$ & $1.37\pm0.25$ & $1.52\pm0.29$\\
$T_{\rm peri}$ (HJD-2458600) & $62.56\pm0.15$ & $56.61\pm0.05$ & $63.31\pm0.08$ & $51.90\pm0.20$\\
$\xi_{{\rm E},N}$ ($10^{-3}$) & $-2.87\pm0.20$ & $2.36\pm0.09$ & $0.93\pm0.18$ & $-1.34\pm0.15$\\
$\xi_{{\rm E},E}$ ($10^{-3}$) & $1.25\pm0.09$ & $0.84\pm0.09$ & $1.21\pm0.08$ & $1.24\pm0.21$\\
${d\alpha}/{dt}$ ($10^{-2}$ ${\rm radian}$ ${\rm yr}^{-1}$) & $-0.10\pm0.05$ & $0.50\pm0.04$ & $2.03\pm0.16$ & $-2.00\pm0.12$\\
${ds}/{dt}$ ($10^{-2}$ ${\rm yr}^{-1}$) & $0.32\pm0.01$ & $1.39\pm0.02$ & $1.04\pm0.03$ & $0.94\pm0.08$\\
\hline \hline
$\chi^2$ & 10750.5 & 10746.1 & 10716.1 & 10742.4\\
$\Delta\chi^2$\tablenotemark{1} & 34.6 & 30.1 &	- & 26.4\\
\enddata
\tablenotetext{1}{From the best of 2L1S + xallarap + parallax + lens orbital motion model.}
\end{deluxetable*}
\end{comment}

\begin{deluxetable}{ccccccccccc}[t!]
\tablecaption{Comparisons between each microlensing model
\label{tab:models}}
\tablewidth{0pt}
\tablehead{
\multicolumn{1}{c}{Model} & \multicolumn{1}{c}{$N_{\rm param}$} & \multicolumn{1}{c}{$\chi^2$} & \multicolumn{1}{c}{$\Delta\chi^2$}
}
%\decimalcolnumbers
\startdata
1L1S & 4 & 33144.7 & 22303.8\\
1L1S + xallarap & 9\tablenotemark{\rm *}& 11311.5 & 470.6\\
1L1S + xallarap + parallax & 11\tablenotemark{\rm *}& 11285.7 & 444.8\\
standard 2L1S & 7 & 11744.7 & 903.7\\
2L1S + parallax & 9 & 11676.5 & 835.5\\
2L1S + xallarap & 12\tablenotemark{\rm *}& 10840.9 & -\\
\enddata
\tablenotetext{*}{The source orbital eccentricity is fixed at $e_{\xi}=0$. 
%When $e_{\xi}=0$, $T_{\rm peri}$ can be eliminated as a chain parameter.
When $e_{\xi}=0$, $T_{\rm peri}$ can be eliminated because it is a parameter that cannot take a specific value.}
%\tablenotetext{1}{From the best of 2L1S + xallarap + parallax + lens orbital motion model.}
\end{deluxetable}

\begin{comment}
\begin{deluxetable}{ccccccccccc}[t!]
\tablecaption{Comparisons between Each Microlensing Models
\label{tab:models}}
\tablewidth{0pt}
\tablehead{
\multicolumn{1}{c}{Model} & \multicolumn{1}{c}{$N_{\rm param}$} & \multicolumn{1}{c}{$\chi^2$} & \multicolumn{1}{c}{$\Delta\chi^2$}
}
%\decimalcolnumbers
\startdata
1L1S & 4 & 33144.7 & 22320.0\\
1L1S + xallarap & 11 & 11184.2 & 359.5\\
standard 2L1S & 7 & 11744.7 & 920.0\\
2L1S + parallax & 9 & 11676.5 & 851.8\\
2L1S + xallarap & 14 & 10824.7 & -\\
\enddata
%\tablenotetext{1}{From the best of 2L1S + xallarap + parallax + lens orbital motion model.}
\end{deluxetable}
\end{comment}

%\begin{deluxetable*}{cccccccccccc}[t!]
%\tablecaption{Comparisons between Each Microlensing Models
%\label{tab:models}}
%\tablewidth{0pt}
%\tablehead{
%\multicolumn{1}{c}{model} & \multicolumn{1}{c}{$N_{\rm param}$} & \multicolumn{1}{c}{$\chi^2$} & \multicolumn{1}{c}{$\Delta\chi^2$} & \multicolumn{1}{c}{reduced-$\chi^2$} 
%}
%\decimalcolnumbers
%\startdata
%1L1S & 4 & 33144.7 & 22428.8 & 3.09\\
%1L1S + xallarap & 11 & 11184.9 & 469.0 & 1.04\\
%standard 2L1S & 7 & 11744.7 & 1028.7 & 1.10\\
%2L1S + xallarap & 14 & 10824.7 & 108.7 & 1.01\\
%2L1S + xallarap + parallax + lens orbital motion & 18 & 10716.0 & - & 1.00\\
%\enddata
%\tablenotetext{1}{From the best of 2L1S + xallarap + parallax + lens orbital motion model.}
%\end{deluxetable*}

\begin{comment}
\subsection{Testing for higher order effects combinations}
\label{subsec:X+P+LOM}
In a real binary lens event, whether detectable or not, the lens orbital motion and parallax effects should be present because both the lens companion and the Earth are orbiting their hosts.
%The companion in a lensing system orbits the host. 
%A parallax is always present because the Earth is orbiting.
%Accordingly, in real 2L, whether detectable or not, lens orbital motion and parallax effects should be present. 
Therefore, we performed a fitting of the xallarap model with the parallax and lens orbital motion effects. 

%
%We found that models in Section \ref{subsec:Xallarap} with $\Delta \chi^2 \leq 9$  from the 2L1S + xallarap best model are $50 < t_{\rm E} \rm [days] < 135,  0.06 < \theta_{\rm E} \rm [mas] < 0.41, 0.35 < \mu_{\rm rel}  \rm [{mas/yr}] < 1.2$. 
%From the galaxy model of \citet{Koshimoto+2021b}, we calculated these values and the distribution of $\pi_{\rm E}$ at event coordinates $(l,b) = (-0.849^{\circ} -2.214^{\circ})$ and adopted the $2\sigma$ limit of $0.024 \leq \pi_{\rm E} \leq 0.351$ as the criteria.
%
%The light curve can be also affected by the orbital motion of the lens planet around the host star.
%In addition, since the parallax and lens orbital motion effects are degenerate \citep{Batista+2011,Skowron+2011,Bachelet+2012,Park+2013}. %it is better to consider them simultaneously.
%It is known that the parallax and lens orbital motion effects are degenerate \citep{Batista+2011,Skowron+2011,Bachelet+2012,Park+2013}.%Therefore, when considering lens orbital motion, the parallax effect should be considered simultaneously. 
In addition, since the parallax and lens orbital motion effects are degenerate \citep{Batista+2011,Skowron+2011,Bachelet+2012,Park+2013}, the parallax effect must also be considered simultaneously when considering lens orbital motion.
Simplified lens orbital motion effects can be modeled by adding the rate of change of the distance between lens components projected onto the lens plane at $t_0$, $ds/dt$ ; and the angular velocity relative to the source orbit, $d\alpha/dt$ \citep{Batista+2011,Skowron+2011}. 
The ratio of kinetic energy to potential energy projected onto the lensing plane of the lens companion can be written as follows \citep{Dong+2009,Skowron+2011}:

\begin{equation}
\label{eq:orbital_constrain}
    {\Bigm |\frac{\rm KE}{\rm PE} \Bigm |}_{\perp}= {\frac{({a_{{\rm L},\perp}/{\rm au}})^3}{8\pi^2(M_{\rm L}/M_\odot)}}\Bigm [\bigm (\frac{1}{s}\frac{ds}{dt} \bigm )^2+\frac{d\alpha}{dt} \Bigm ],
\end{equation}

\noindent where $M_{\rm L}$ is the total mass of the lensing system,  and $a_\perp$ is the semi-major axis of the lensing companion star projected onto the lensing plane. 
$a_{{\rm L},\perp}$ can be described as $a_{{\rm L},\perp}=s D_{\rm L} \theta_{\rm E}$ using the distance $D_{\rm L}$ between the earth and the lens system, by

\begin{equation}
\label{eq:D_L}
    D_{\rm L}=\frac{\rm au}{\pi_{\rm E}\theta_{\rm E}+\pi_{\rm S}},
\end{equation}

\noindent where $\pi_{\rm S}$ is the parallax of the source star, which can be written as $\pi_{\rm S}={\rm au}/D_{\rm S}$ using the distance $D_{\rm S}$ from the Earth to the source system. 
Following \citet{Miyazaki+2020}, we assumed $D_{\rm S}=8$ kpc in the sense that the source is in the Galactic bulge, and restricted the modeling to have $|{\rm KE}/{\rm PE}|_\perp<1$.

Furthermore, we found that the 2L1S + xallarap models with $\Delta\chi^{2} \leq 9$ from the best 2L1S + xallarap model in Section \ref{subsec:Xallarap} have $50 < t_{\rm E}$ [days]$ < 135$, $0.06 < \theta_{\rm E}$ [mas] $< 0.41$ and $0.35 < \mu_{\rm {rel}}$ [$\rm{mas/yr}$] $<1.2$. 
We calculated the prior $\bm{\pi_{\rm{E}}}$ distribution of events with those parameters using the Galactic model by \citet{Koshimoto+2021b}. 
%%%%%%%%%%%%%%%%%%%%%%%%%%%%%%%To remove events that have too large $\bm{\pi_{\rm {E}}}$ values, we limit the fitting range in the 2 $\sigma$ limit of the prior distribution of 0.024 $< \pi_{\rm {E}} <$ 0.351.
To remove events that have too large $\pi_{\rm{E}}$ values, we limit the fitting range in the 2 $\sigma$ limit of the prior distribution of $0.024 \leq \pi_{\rm{E}} \leq 0.351$.

The analysis of 2L1S + xallarap + parallax + lens orbital motion shows that various models exist in the parameter space of $s<1$, $0.01\lesssim q \lesssim1$.
The best 2L1S + xallarap + parallax + lens orbital motion model improved $\chi^2$ by $108.7$ from the 2L1S + xallarap model.
However, from the cumulative $\Delta\chi^2$ in each data set for the 2L1S + xallarap + parallax + lens orbital motion model and the 2L1S + xallarap model, the $\chi^2$ improvement for the 2L1S + xallarap + parallax + lens orbital motion model is not consistent across data sets.
Furthermore, the $\Delta\chi^2$ improvement is also highly influenced by the baseline systematic trend. 
This is the same as the case of the 2L1S + parallax model described in Section \ref{subsec:Parallax}. 
We therefore conclude that the additional parallax + lens orbital motion signal is questionable.
For the same reason, the 2L1S + xallarap + parallax model is also unreliable.
\end{comment}
%\input{5_Source_System_Properties_and_Lens_System_Properties.tex}
%\section{Source System Properties and Lens System Properties}
\section{Source System Properties}
\label{sec:Source_Lens_Properties}
% Figure environment removed
\begin{comment}
We estimate the angular radius of the source host $\theta_*$ from the color and magnitude of the source.
%to obtain the properties of the source system.
%The apparent magnitude of $R_{\rm MOA}$ and $V_{\rm MOA}$ sources are obtained from the fittings described in Section \ref{sec:light_curve_modeling}.
The apparent magnitude of the source, $R_{\rm MOA}$ and $V_{\rm MOA}$, are obtained from the modeling described in Section \ref{sec:light_curve_modeling}.
However, the two bands have been scaled to match the instrumental MOA-$\mathrm{I}\hspace{-1.2pt}\mathrm{I}$ DOPHOT catalog \citep{Bond+2017}.
Therefore, we converted them to Cousins $I$-band and Johnson $V-$band magnitude, using the OGLE-$\mathrm{I}\hspace{-1.2pt}\mathrm{I}\hspace{-1.2pt}\mathrm{I}$ catalog \citep{Szymanski+2011}.
%First, we cross-referenced the OGLE-$\mathrm{I}\hspace{-1.2pt}\mathrm{I}\hspace{-1.2pt}\mathrm{I}$ and the MOA catalogs. For stars within $2'$ of the source star (RA=17:52:21.57, Dec=-30:48:12.05), we matched objects with the MOA-$\mathrm{I}\hspace{-1.2pt}\mathrm{I}$ DOPHOT catalog and the OGLE-$\mathrm{I}\hspace{-1.2pt}\mathrm{I}\hspace{-1.2pt}\mathrm{I}$ catalog coordinates within $0".7$.
First, we cross-referenced the OGLE-$\mathrm{I}\hspace{-1.2pt}\mathrm{I}\hspace{-1.2pt}\mathrm{I}$ and the MOA catalogs. For stars within $2'$ of the source star, we matched objects with the MOA-$\mathrm{I}\hspace{-1.2pt}\mathrm{I}$ DOPHOT catalog and the OGLE-$\mathrm{I}\hspace{-1.2pt}\mathrm{I}\hspace{-1.2pt}\mathrm{I}$ catalog coordinates within $0".7$.
%This is based on the assumption that stars within $2'$ of the source star have roughly the same amount of reddening and extinction \citep[e.g.,][]{Nagakane+2019,Kondo+2019}.
\end{comment}
%We estimated the angular source radius, $\theta_*$, from the color and magnitude of the source to obtain the properties of the lens and source system.
We estimated the angular source radius, $\theta_*$, from the color and magnitude of the source.
%The best fit instrumental source magnitudes of $R_{\rm MOA}$ and $V_{\rm MOA}$ are calibrated to the Cousins $I$-band and Johnson $V-$band magnitude scales by cross-referencing to the stars in the OGLE-$\mathrm{I}\hspace{-1.2pt}\mathrm{I}\hspace{-1.2pt}\mathrm{I}$ photometry map \citep{Szymanski+2011} within $0^{\prime\prime}.7$ of the event.
The best fit instrumental source magnitudes of $R_{\rm MOA}$ and $V_{\rm MOA}$ are calibrated to the Cousins $I$-band and Johnson $V$-band magnitude scales by cross-referencing to the stars in the OGLE-$\mathrm{I}\hspace{-1.2pt}\mathrm{I}\hspace{-1.2pt}\mathrm{I}$ photometry map \citep{Szymanski+2011} within $0\farcs7$ of the event.

%For reliability, we restricted stars to $16 \leq V_{\rm OGLE}$ [mag] $\leq 19$ which is bright enough, and performed $5\sigma$ clipping in linear regression of $V_{\rm MOA}$ vs. $V_{\rm OGLE}$ and $5\sigma$ clipping in linear regression of $(V-I)_{\rm OGLE}$ vs. $(V-R)_{\rm MOA}$.
For reliability, we restricted stars to $16 \leq V_{\rm OGLE-\mathrm{I}\hspace{-1.2pt}\mathrm{I}\hspace{-1.2pt}\mathrm{I}}$ [mag] $\leq 19$, and performed $5\sigma$ clipping in the linear regressions of $V_{\rm MOA}$ vs. $V_{\rm OGLE-\mathrm{I}\hspace{-1.2pt}\mathrm{I}\hspace{-1.2pt}\mathrm{I}}$ and $(I_{\rm OGLE-\mathrm{I}\hspace{-1.2pt}\mathrm{I}\hspace{-1.2pt}\mathrm{I}}-R_{\rm MOA})$ vs. $(V-R)_{\rm MOA}$ and $(V-I)_{\rm OGLE-\mathrm{I}\hspace{-1.2pt}\mathrm{I}\hspace{-1.2pt}\mathrm{I}}$ vs. $(V-R)_{\rm MOA}$, respectively.
%For reliability, we restricted stars to $16 \leq V_{\rm OGLE-\mathrm{I}\hspace{-1.2pt}\mathrm{I}\hspace{-1.2pt}\mathrm{I}}$ [mag] $\leq 19$ which is bright enough, and performed $5\sigma$ clipping in the linear regressions of $V_{\rm MOA}$ vs. $V_{\rm OGLE-\mathrm{I}\hspace{-1.2pt}\mathrm{I}\hspace{-1.2pt}\mathrm{I}}$ and $(I_{\rm OGLE-\mathrm{I}\hspace{-1.2pt}\mathrm{I}\hspace{-1.2pt}\mathrm{I}}-R_{\rm MOA})$ vs. $(V-R)_{\rm MOA}$ and $(V-I)_{\rm OGLE-\mathrm{I}\hspace{-1.2pt}\mathrm{I}\hspace{-1.2pt}\mathrm{I}}$ vs. $(V-R)_{\rm MOA}$, respectively.
%From the final remaining 73 objects, the following conversion equation from $(V-R)_{\rm MOA}$ to $(V-I)_{\rm OGLE}$ was obtained by linear regression,
From the final 73 remaining objects, the following conversion equations from $R_{\rm MOA}$ and $(V-R)_{\rm MOA}$ to $I_{\rm OGLE-\mathrm{I}\hspace{-1.2pt}\mathrm{I}\hspace{-1.2pt}\mathrm{I}}$ and $(V-I)_{\rm OGLE-\mathrm{I}\hspace{-1.2pt}\mathrm{I}\hspace{-1.2pt}\mathrm{I}}$ were obtained by linear regression,

\begin{equation}\label{eq:convert1}
    I_{\rm OGLE-\mathrm{I}\hspace{-1.2pt}\mathrm{I}\hspace{-1.2pt}\mathrm{I}} = R_{\rm MOA}-(0.24\pm0.01) \times (V-R)_{\rm MOA} + (27.22\pm0.01),
\end{equation}

\begin{equation}\label{eq:convert2}
    (V-I)_{\rm OGLE-\mathrm{I}\hspace{-1.2pt}\mathrm{I}\hspace{-1.2pt}\mathrm{I}} = (1.20\pm0.01) \times (V-R)_{\rm MOA} + (0.94\pm0.02).
\end{equation}

%As a result, the color and magnitude after reddening and extinction of the source star of the best model of 2L1S + xallarap was $(V-I,I)_{\rm S}=(2.526 \pm 0.023,$ $20.915 \pm 0.003)$.
As a result, the color and magnitude with the extinction of the source star for the best fit 2L1S + xallarap model  was $(V-I,I)_{\rm S}=(2.527 \pm 0.031,$ $21.035 \pm 0.015)$.
%The mean color and magnitude of the RCG in the Galactic bulge before reddening and extinction was estimated from microlensing observations to be 
The intrinsic color and magnitude of RCG stars are $(V-I,I)_{{\rm RCG},0}=(1.060 \pm 0.060,$ $14.443 \pm 0.040)$ \citep{Bensby+2013,Nataf+2013}.
%The mean color magnitude of the RCG within $2'$ of the source star could be estimated from the stars in the OGLE-$\mathrm{I}\hspace{-1.2pt}\mathrm{I}\hspace{-1.2pt}\mathrm{I}$ catalog as
From the color-magnitude diagram of the stars within $2'$ of the source star (Figure~\ref{fig:cmd}), the RCG centroid is estimated as $(V-I,I)_{\rm RCG}=(2.804 \pm 0.009,$ $16.488 \pm 0.022)$.
Then we calculated $(E(V-I),A(I))=(1.744 \pm 0.061,$ $2.045 \pm 0.046)$.
Finally, we have the intrinsic color and magnitude of the source star $(V-I,I)_{\rm S,0}=(0.783 \pm 0.068,$ $18.990 \pm 0.048)$ for the best 2L1S + xallarap model.
Also, Figure \ref{fig:cmd} shows that the source is a main-sequence star and unlikely to be a variable star.
%Note that $D_{\rm S}=8.0\pm1.4$ kpc is assumed in the calculation of the distance modulus and in Equation \ref{eq:xi_E}.
%Figure \ref{fig:cmd} shows the color and magnitude diagram after reddening and extinction.
%The source star is an orange circle, the RCG is a red circle, the black dots are the OGLE-$\mathrm{I}\hspace{-1.2pt}\mathrm{I}\hspace{-1.2pt}\mathrm{I}$ catalog stars within $2'$ from the source star, and the green dots indicate the stars in the Baade's Window imaged by the Hubble Space Telescope \citep{Holtzman+1998}.
Table~\ref{tab:Source_xallarap} shows that the values for $(V-I,I)_{\rm S,0}$ for the other models are almost the same.
%Tables \ref{tab:Source_xallarap_wide} - \ref{tab:Source_X_P_LOM} show that the values for $(V-I,I)_{\rm S,0}$ for the other models are almost the same.
%
\begin{comment}

\citet{Boyajian+2014} derived the following relationship between the apparent radius $\theta_*$ and color magnitude from nearby main-sequence stars,
\begin{equation}\label{eq:theta_star}
    \log(2\theta_*)=0.50 + 0.42(V-I)_0 - 0.2 I_0.
\end{equation}

This relationship comes from private communication with them, which is limited to FGK stars with $3900$ $<T_{\rm {eff}}$ [K] $<7000$, and the accuracy of the relational equation is better than $2\%$ \citep{Fukui+2015}.
We estimated the angular source radius $\theta_* = 0.568 \pm 0.039$ $\mu$as from the Equation (\ref{eq:theta_star}).
\end{comment}

We estimated the angular source radius of $\theta_* = 0.538 \pm 0.039$ $\mu$as from the relation,
\begin{equation}\label{eq:theta_star}
    \log(2\theta_*)=0.50 + 0.42(V-I)_0 - 0.2 I_0,
\end{equation}
where the accuracy of the relational equation is better than 2\% \citep{Fukui+2015}.
This relation is based on \citet{Boyajian+2014}, but derived by  limiting to FGK stars with $3900$ $<T_{\rm {eff}}$ [K] $<7000$ (Boyajian, 2014, Private Communication).
%This relation is based on \citet{Boyajian+2014}, but derived by  limiting to FGK stars with $3900$ $<T_{\rm {eff}}$ [K] $<7000$ with private communication with them.
%Furthermore, we calculated the Einstein radius $\theta_{\rm E}= \rho \theta_* = 0.25 \pm 0.03$ mas and the source lens relative proper motion $\mu=\theta_{\rm E}/t_{\rm E} = 0.94 \pm 0.10$ mas/yr.
Then, we calculated the lens's Einstein radius of $\theta_{\rm E}= \rho \theta_* = 0.25 \pm 0.02$ mas and the lens-source relative proper motion of $\mu_{\rm rel}=\theta_{\rm E}/t_{\rm E} = 0.97 \pm 0.10$ mas $\rm {yr^{-1}}$.

The amplitude of the xallarap vector, $\xi_{\rm E}$ is described as follows,
%, is the semimajor axis of source companion normalized by the product of the apparent Einstein radius and the Earth source distance, $\theta_{\rm E} D_{\rm S}$, and is described as follows,
\begin{equation}\label{eq:xi_E}
    %\xi_{\rm E}\equiv {\frac{M_{\rm S,C}}{\theta_{\rm E} D_{\rm S}} }\Bigl (\frac{P_\xi}{M_{\rm S,H}+M_{\rm S,C}} \Bigr )^{2/3}.
    %\xi_{\rm E}\equiv \Bigl (\frac{1\:\rm{au}}{\theta_{\rm E} D_{\rm S}} \Bigr) \Bigl (\frac{M_{\rm S,C}}{M_\odot} \Bigr ) \Bigl [\frac{P_\xi}{1\: \rm {yr}}\frac{M_\odot}{M_{\rm S,H}+M_{\rm S,C}} \Bigr ]^{2/3},
    \xi_{\rm E}\equiv \Bigl (\frac{\theta_{\rm E} D_{\rm S}}{1\:\rm{au}} \Bigr)^{-1} \Bigl (\frac{P_\xi}{1\: \rm {yr}} \Bigr )^{2/3} \Bigl (\frac{M_{\rm S,C}}{M_\odot} \Bigr ) \Bigl (\frac{M_{\rm S,H}+M_{\rm S,C}}{M_\odot} \Bigr )^{-2/3},
\end{equation}

%\noindent where $M_{\rm S,H}$ is the host mass of the source system and $M_{\rm S,C}$ is the companion mass of the source system. 
\noindent where $M_{\rm S,H}$ and $M_{\rm S,C}$ are the masses of host and companion of the source system, respectively.
%The host mass of the source system $M_{\rm S,H}$ is estimated from isochrone \citep[PARSEC;][]{Bressan+12} of age 10 Gyr in solar metallicity using the absolute magnitude of the source host star $M(I_{\rm S})=I_{\rm {S,0}} + 5 \log_{10}{D_{\rm S} {\rm {[pc]}}}+5$.
%Assuming $D_{\rm S}=8.0\pm1.4$ kpc, $M_{\rm S,C}$ can be solved from Equation (\ref{eq:xi_E}). 
$M_{\rm S,H}$ is estimated by using isochrones \citep[PARSEC;][]{Bressan+12} and the absolute magnitude of the host source star $M(I_{\rm S})=I_{\rm {S,0}} + 5 \log_{10}{D_{\rm S} {\rm {[pc]}}}+5=4.48\pm0.38$ mag assuming $D_{\rm S}=8.0\pm1.4$ kpc.
Then, $M_{\rm S,C}$ can be solved from Equation (\ref{eq:xi_E}).
Also, using Kepler's third law, 

\begin{equation}\label{eq:Kepler}
    %\frac{{a_{\rm S}}^3}{{P_\xi}^2}=\frac{G(M_{\rm S,H} + M_{\rm S,C})}{4{\pi}^2},
    %\Bigl (\frac{a_{\rm S}}{1\:\rm {au}} \Bigr ) ^3 \Bigl (\frac{1\:\rm {yr}}{P_\xi} \Bigr ) ^2 =\frac{G}{4{\pi}^2} \Bigl [\frac{M_{\rm S,H} + M_{\rm S,C}}{M_{\odot}} \Bigr],
    \Bigl (\frac{a_{\rm S}}{1\:\rm {au}} \Bigr ) ^3 \Bigl (\frac{P_\xi}{1\:\rm {yr}} \Bigr ) ^{-2} = \Bigl [\frac{M_{\rm S,H} + M_{\rm S,C}}{M_{\odot}} \Bigr],
\end{equation}

\noindent we can solve $a_{\rm S}$ which is the semi-major axis of the source system. 
%We further calculated the  $L_{\rm S,C}/L_{\rm S,H}$ which is the ratio of the luminosity of the source companion $L_{\rm S,C}$ to the source host $L_{\rm S,H}$ from the mass-luminosity relation \citep{Duric2004}. 
%In addition, we calculated $L_{\rm S,C}/L_{\rm S,H}$, the luminosity ratio at the $I$-band of the source companion $L_{\rm S,C}$ to that of the source host $L_{\rm S,C}$, from the empirical relation of \citet{Bennett+2015} and the isochrone model of \citet{Baraffe+2003}.
%In addition, we calculated $L_{\rm S,C}/L_{\rm S,H}$, the luminosity ratio at the $I$-band of the source companion $L_{\rm S,C}$ to that of the source host $L_{\rm S,C}$, from the mass-luminosity empirical relation used by \citet{Bennett+2015} and the isochrone model of \citet{Baraffe+2003}.
The apparent $H$- and $K$-bands magnitudes of the source with extinction $H_{\rm S}$ and $K_{\rm S}$ are also estimated using PARSEC isochrones and the wavelength dependence of the extinction law in the direction of Galactic center, $A_V:A_H:A_{K_s}=1:0.108:0.062$ \citep{Nishiyama+2008}.
In addition, we calculated $L_{\rm S,C}/L_{\rm S,H}$, the luminosity ratio in the $I$-band of the source companion $L_{\rm S,C}$ to the source host $L_{\rm S,H}$.
For this we used the mass-luminosity empirical relation of \citet{Bennett+2015}, which combines \citet{Henry+1993} and \citet{Delfosse+2000}, and the isochrone model of \citet{Baraffe+2003}.
%We use the \citet{Henry+1993} relation for $M>0.66$ $M_{\odot}$, relation for 0.12M_Sun< M<0.54M_Sun, and for brown dwarf-like low-mass stars (M<0.1M_Sun) the Baraffe et al. (2003) We use the model of isoclones of sub-stellar objects with ages of 5 Gyr.
We used the \citet{Henry+1993} relation for $M>0.66$ $M_{\odot}$, the \citet{Delfosse+2000} relation for $0.12$ $M_{\odot}<M<0.54$ $M_{\odot}$.
%For low-mass stars close to brown dwarfs ($M<0.10$ $M_{\odot}$) we used the isochrone model of \citet{Baraffe+2003} for sub-stellar objects at an age of 5 Gyr.
For low-mass stars ($M<0.10$ $M_{\odot}$) we used the isochrone model of \citet{Baraffe+2003} for sub-stellar objects at an age of 10 Gyr.
At the boundary of these mass ranges, we interpolated linearly between the two relations.
Table~\ref{tab:Source_xallarap} shows our calculated properties of the source system for the 2L1S + xallarap models in Table~\ref{tab:param_xallarap}.
%The source host is a G-type main-sequence star and the source companion is an M-type dwarf or brown dwarf with a semi major axis of $a_{\rm S}\sim0.06$ au.
The source host in the best 2L1S + xallarap model is a G-type main-sequence star and the source companion is a brown dwarf with a semi major axis of $a_{\rm S}=0.0594\pm0.0005$ au.
%The luminosity ratio at the $I$-band of the source companion $L_{\rm S,C}$ is small, $L_{\rm S,C}/L_{\rm S,H}\sim 10^{-4}$, and does not conflict with our assumption the source companion is not magnified.
%The luminosity ratio at the $I$-band of the source companion $L_{\rm S,C}$ is small, $L_{\rm S,C}/L_{\rm S,H}=(1.0\pm0.4)\times10^{-7}$, and does not conflict with our assumption the source companion is not magnified.
The luminosity ratio at the $I$-band of the source companion $L_{\rm S,C}$ is small, $L_{\rm S,C}/L_{\rm S,H}=(1.0\pm0.3)\times10^{-7}$, and does not conflict with our assumption the magnified flux of the second source is too weak to be detected.
%When we change the assumption of $D_{\rm S}$ by $\pm 1$ kpc (i.e., $D_{\rm S}=$9,7 kpc), the calculations show $\pm \sim 4 \%$ for $M_{\rm S,H}, \pm \sim 15 \%$ for $M_{\rm S,C}$, $\pm \sim 1 \%$ for $a_{\rm S}$, $\pm \sim 20\%$ for $L_{\rm S,C}/L_{\rm S,H}$ varying, but not significantly affecting our conclusions.

%The physical properties of the lensing system for the best model of 2L1S + xallarap + parallax + lens orbital motion in \ref{subsec:X+P+LOM} are: the mass of the lens primary star OGLE-2019-BLG-0825L is $0.068 \sim 0.010$ $M_\odot$, the mass of the lens companion OGLE-2019-BLG-0825Lb is $0.032 \pm 0.005$ $M_\odot$, and the Earth-lensing distance $D_{\rm L}  \sim 4.48 \pm 0.32$ kpc which were solved analytically from parallax and $\theta_{\rm E}$, but the parameters of the lens system are highly uncertain.
\begin{deluxetable*}{c|ccccccccccc}[t!]
\tablecaption{Source system properties of the 2L1S + xallarap models 
\label{tab:Source_xallarap}}
\tablewidth{0pt}
%\tablehead{
%\multicolumn{1}{c|}{Parameters} & \multicolumn{1}{c}{$q\leq0.001$} & \multicolumn{1}{c}{$0.001<q\leq0.01$} & \multicolumn{1}{c}{$0.01<q\leq0.1$} & \multicolumn{1}{c}{$0.1<q\leq1$} 
%}
\tablehead{Model & XLclose1 & XLclose2 & XLwide1 & XLwide2 \\
\multicolumn{1}{c|}{range of $q$} & \multicolumn{1}{c}{$q\leq0.1$} & \multicolumn{1}{c}{$0.1<q\leq1$} & \multicolumn{1}{c}{$q\leq0.1$} & \multicolumn{1}{c}{$0.1<q\leq1$}
}
%\decimalcolnumbers
\startdata
$V_{\rm S}$ (mag) & $23.58\pm0.03$ & $23.56\pm0.03$ & $23.55\pm0.03$ & $23.56\pm0.03$\\
$I_{\rm S}$ (mag) & $21.06\pm0.01$ & $21.04\pm0.01$ & $21.02\pm0.01$ & $21.06\pm0.01$\\
$H_{\rm S}$ (mag) & $18.54\pm0.30$ & $18.52\pm0.484$ & $18.51\pm0.48$ & $18.54\pm0.48$\\
$K_{\rm S}$ (mag) & $18.31\pm0.48$ & $18.29\pm0.48$ & $18.28\pm0.48$ & $18.31\pm0.48$\\
$(V-I)_{\rm S}$ (mag) & $2.52\pm0.03$ & $2.53\pm0.03$ & $2.52\pm0.03$ & $2.527\pm0.03$\\
$I_{\rm S,0}$ (mag) & $19.01\pm0.05$ & $18.99\pm0.05$ & $19.00\pm0.05$ & $19.01\pm0.05$\\
$(V-I)_{\rm S,0}$ (mag) & $0.78\pm0.07$ & $0.78\pm0.07$ & $0.78\pm0.07$ & $0.78\pm0.07$\\
$M_I$ (mag) & $4.50 \pm 0.38$ & $4.48 \pm 0.38$ & $4.46 \pm 0.38$ & $4.50 \pm 0.38$\\
$\theta_{\rm E}$ (mas) & $0.53\pm0.04$ & $0.25\pm0.02$ & $0.22\pm0.02$ & $0.37\pm0.05$\\
$\mu_{\rm rel}$ (mas ${\rm yr}^{-1}$) & $0.78\pm0.07$ & $0.97\pm0.10$ & $0.81\pm0.07$ & $1.01\pm0.16$\\
$M_{\rm S,H}$ ($M_\odot$) & $0.864\pm0.045$ & $0.867\pm0.045$ & $0.868\pm0.045$  & $0.864\pm0.045$\\
$M_{\rm S,C}$ ($M_\odot$) & $0.050\pm0.005$ & $0.048\pm0.004$ & $0.047\pm0.004$ & $0.051\pm0.006$\\
$a_{\rm S}$ ($10^{-2}$ au) & $5.86\pm0.04$ & $5.94\pm0.05$ &  $5.87\pm0.03$ & $5.95\pm0.05$\\
$L_{\rm S,C}/L_{\rm S,H}$ ($10^{-7}$) & $1.15\pm0.34$ &  $1.02\pm0.26$ & $0.95\pm0.23$ & $1.21\pm0.47$\\
\hline \hline
$\chi^2$ & 10856.4 & 10840.9 & 10861.2 & 10842.7\\
$\Delta\chi^2$ & 15.5 & - & 20.3 & 1.8\\
\enddata
%\tablenotetext{1}{From the best of 2L1S + xallarap model.}
\end{deluxetable*}
\begin{comment}
%20230705
\begin{deluxetable*}{c|ccccccccccc}[t!]
\tablecaption{Source system properties of the 2L1S + xallarap models 
\label{tab:Source_xallarap}}
\tablewidth{0pt}
%\tablehead{
%\multicolumn{1}{c|}{Parameters} & \multicolumn{1}{c}{$q\leq0.001$} & \multicolumn{1}{c}{$0.001<q\leq0.01$} & \multicolumn{1}{c}{$0.01<q\leq0.1$} & \multicolumn{1}{c}{$0.1<q\leq1$} 
%}
\tablehead{Model & XLclose1 & XLclose2 & XLwide1 & XLwide2 \\
\multicolumn{1}{c|}{range of $q$} & \multicolumn{1}{c}{$q\leq0.1$} & \multicolumn{1}{c}{$0.1<q\leq1$} & \multicolumn{1}{c}{$q\leq0.1$} & \multicolumn{1}{c}{$0.1<q\leq1$}
}
%\decimalcolnumbers
\startdata
$V_{\rm S}$ (mag) & $23.581\pm0.034$ & $23.562\pm0.034$ & $23.545\pm0.034$ & $23.564\pm0.034$\\
$I_{\rm S}$ (mag) & $21.057\pm0.015$ & $21.035\pm0.015$ & $21.022\pm0.015$ & $21.057\pm0.015$\\
$H_{\rm S}$ (mag) & $18.538\pm0.300$ & $18.521\pm0.485$ & $18.511\pm0.484$ & $18.538\pm0.484$\\
$K_{\rm S}$ (mag) & $18.308\pm0.478$ & $18.293\pm0.479$ & $18.282\pm0.479$ & $18.308\pm0.479$\\
$(V-I)_{\rm S}$ (mag) & $2.524\pm0.031$ & $2.527\pm0.031$ & $2.523\pm0.031$ & $2.527\pm0.031$\\
$I_{\rm S,0}$ (mag) & $19.012\pm0.048$ & $18.990\pm0.048$ & $18.997\pm0.048$ & $19.012\pm0.048$\\
$(V-I)_{\rm S,0}$ (mag) & $0.780\pm0.068$ & $0.783\pm0.068$ & $0.779\pm0.068$ & $0.783\pm0.068$\\
$M_I$ (mag) & $4.496 \pm 0.384$ & $4.476 \pm 0.384$ & $4.462 \pm 0.384$ & $4.496 \pm 0.384$\\
$\theta_{\rm E}$ (mas) & $0.531\pm0.038$ & $0.250\pm0.023$ & $0.224\pm0.019$ & $0.369\pm0.048$\\
$\mu_{\rm rel}$ (mas ${\rm yr}^{-1}$) & $0.775\pm0.072$ & $0.973\pm0.096$ & $0.812\pm0.073$ & $1.012\pm0.158$\\
$M_{\rm S,H}$ ($M_\odot$) & $0.864\pm0.045$ & $0.867\pm0.045$ & $0.868\pm0.045$  & $0.864\pm0.045$\\
$M_{\rm S,C}$ ($M_\odot$) & $0.050\pm0.005$ & $0.048\pm0.004$ & $0.047\pm0.004$ & $0.051\pm0.006$\\
$a_{\rm S}$ ($10^{-2}$ au) & $5.859\pm0.040$ & $5.942\pm0.047$ &  $5.874\pm0.033$ & $5.948\pm0.047$\\
$L_{\rm S,C}/L_{\rm S,H}$ ($10^{-7}$) & $1.154\pm0.339$ &  $1.019\pm0.264$ & $0.953\pm0.230$ & $1.205\pm0.472$\\
\hline \hline
$\chi^2$ & 10856.4 & 10840.9 & 10861.2 & 10842.7\\
$\Delta\chi^2$ & 15.5 & - & 20.3 & 1.8\\
\enddata
%\tablenotetext{1}{From the best of 2L1S + xallarap model.}
\end{deluxetable*}
\end{comment}
\begin{comment}
20230528
$V_{\rm S}$ (mag) & $19.791\pm0.083$ & $\pm$ & $\pm$ & $\pm$\\
$I_{\rm S}$ (mag) & $21.057\pm0.015$ & $21.035\pm0.015$ & $21.022\pm0.015$ & $21.057\pm0.015$\\
$H_{\rm S}$ (mag) & $\pm$ & $\pm$ & $\pm$ & $\pm$\\
$K_{\rm S}$ (mag) & $\pm$ & $\pm$ & $\pm$ & $\pm$\\
$(V-I)_{\rm S}$ (mag) & $2.524\pm0.031$ & $2.527\pm0.031$ & $2.523\pm0.031$ & $2.527\pm0.031$\\
$I_{\rm S,0}$ (mag) & $19.012\pm0.048$ & $18.990\pm0.048$ & $18.997\pm0.048$ & $19.012\pm0.048$\\
$(V-I)_{\rm S,0}$ (mag) & $0.780\pm0.068$ & $0.783\pm0.068$ & $0.779\pm0.068$ & $0.783\pm0.068$\\
$M_I$ (mag) & $4.496 \pm 0.384$ & $4.476 \pm 0.384$ & $4.462 \pm 0.384$ & $4.496 \pm 0.384$\\
$\theta_{\rm E}$ (mas) & $0.531\pm0.038$ & $0.250\pm0.023$ & $0.224\pm0.019$ & $0.369\pm0.048$\\
$\mu_{\rm rel}$ (mas ${\rm yr}^{-1}$) & $0.775\pm0.072$ & $0.973\pm0.096$ & $0.812\pm0.073$ & $1.012\pm0.158$\\
$M_{\rm S,H}$ ($M_\odot$) & $0.865\pm0.045$ & $0.867\pm0.045$ & $0.869\pm0.045$  & $0.865\pm0.045$\\
$M_{\rm S,C}$ ($M_\odot$) & $0.050\pm0.005$ & $0.048\pm0.004$ & $0.047\pm0.004$ & $0.051\pm0.006$\\
$a_{\rm S}$ ($10^{-2}$ au) & $5.860\pm0.043$ & $5.943\pm0.051$ &  $5.874\pm0.033$ & $5.949\pm0.051$\\
$L_{\rm S,C}/L_{\rm S,H}$ ($10^{-7}$) & $1.155\pm0.340$ &  $1.019\pm0.264$ & $0.954\pm0.230$ & $1.206\pm0.472$\\
\end{comment}

\begin{comment}
\begin{deluxetable*}{c|ccccccccccccc}[t!]
\tablecaption{Source system properties of the 2L1S + xallarap models 
\label{tab:Source_xallarap}}
\tablewidth{0pt}
%\tablehead{
%\multicolumn{1}{c|}{Parameters} & \multicolumn{1}{c}{$q\leq0.001$} & \multicolumn{1}{c}{$0.001<q\leq0.01$} & \multicolumn{1}{c}{$0.01<q\leq0.1$} & \multicolumn{1}{c}{$0.1<q\leq1$} 
%}
\tablehead{Model & XLclose1 & \red{XLclose2} & \red{XLclose3} & XLwide1 & \red{XLwide2} & \red{XLwide3}\\
\multicolumn{1}{c|}{range of $q$} & \multicolumn{1}{c}{$q\leq0.1$} & \multicolumn{1}{c}{\red{$0.1<q\leq0.5$}} & \multicolumn{1}{c}{\red{$0.5<q\leq1$}} & \multicolumn{1}{c}{$q\leq0.1$} & \multicolumn{1}{c}{\red{$0.1<q\leq0.5$}} & \multicolumn{1}{c}{\red{$0.5<q\leq1$}}
}
%\decimalcolnumbers
\startdata
$I_{\rm s,0}$ (mag) & $19.012\pm0.048$ & $18.990\pm0.048$ & $19.008\pm0.048$ &	$18.997\pm0.048$ & $19.014\pm0.048$ & $19.012\pm0.048$\\
$(V-I)_{\rm s,0}$ (mag) & $0.780\pm0.068$ & $0.783\pm0.068$ & $0.782\pm0.068$ & $0.779\pm0.068$ & $0.782\pm0.068$ & $0.783\pm0.068$\\
$\theta_{\rm E}$ (mas) & $0.531\pm0.038$ & $0.250\pm0.023$ & $0.252\pm0.026$ & $0.224\pm0.019$ & $0.309\pm0.030$ & $0.369\pm0.048$\\
$\mu_{\rm rel}$ (mas ${\rm yr}^{-1}$) & $0.775\pm0.072$ & $0.973\pm0.096$ & $0.963\pm0.102$ & $0.812\pm0.073$ & $0.976\pm0.104$ & $1.012\pm0.158$\\
$M_{\rm S,H}$ ($M_\odot$) & $0.865\pm0.045$ & $0.867\pm0.045$ & $0.865\pm0.045$ & $0.869\pm0.045$ & $0.864\pm0.045$ & $0.865\pm0.045$\\
$M_{\rm S,C}$ ($M_\odot$) & $0.050\pm0.005$ & $0.048\pm0.004$ & $0.048\pm0.006$ & $0.047\pm0.004$ & $0.050\pm0.004$ & $0.051\pm0.006$\\
$a_{\rm S}$ ($10^{-2}$ au) & $5.860\pm0.043$ & $5.943\pm0.051$ & $5.952\pm0.055$ & $5.874\pm0.033$ & $5.932\pm0.054$ & $5.949\pm0.051$\\
$L_{\rm S,C}/L_{\rm S,H}$ ($10^{-7}$) & $1.155\pm0.340$ & $1.019\pm0.264$ & $1.003\pm0.393$ & $0.954\pm0.230$ & $1.122\pm0.297$ & $1.206\pm0.472$\\
\hline \hline
$\chi^2$ & 10856.4 & 10840.9 & 10842.3 & 10861.2 & 10843.4 & 10842.7\\
$\Delta\chi^2$ & 15.5 & - & 1.3 & 20.3 & 2.4 & 1.8\\
\enddata
%\tablenotetext{1}{From the best of 2L1S + xallarap model.}
\end{deluxetable*}
\end{comment}
\begin{comment}
\begin{deluxetable*}{c|ccccccccccc}[t!]
\tablecaption{Source system properties of the 2L1S + xallarap wide models 
\label{tab:Source_xallarap_wide}}
\tablewidth{0pt}
\tablehead{Model & XLwide1 & XLwide2 & XLwide3 & XLwide4\\
\multicolumn{1}{c|}{range of $q$} & \multicolumn{1}{c}{$q\leq0.001$} & \multicolumn{1}{c}{$0.001<q\leq0.01$} & \multicolumn{1}{c}{$0.01<q\leq0.1$} & \multicolumn{1}{c}{$0.1<q\leq1$} 
}
%\tablehead{
%\multicolumn{1}{c|}{Parameters} & \multicolumn{1}{c}{$q\leq0.001$} & \multicolumn{1}{c}{$0.001<q\leq0.01$} & \multicolumn{1}{c}{$0.01<q\leq0.1$} & \multicolumn{1}{c}{$0.1<q\leq1$} 
%}
%\decimalcolnumbers
\startdata
$I_{\rm s,0}$ (mag) & $18.831\pm0.048$ & $18.986\pm0.048$ & $19.011\pm0.048$ & $19.014\pm0.048$ \\
$(V-I)_{\rm s,0}$ (mag) & $0.777\pm0.068$ & $0.780\pm0.068$ & $0.780\pm0.068$ & $0.781\pm0.068$ \\
$\theta_{\rm E}$ (mas) & $0.061\pm0.004$ &  $0.124\pm0.010$ & $0.236\pm0.024$ & $0.256\pm0.027$\\
$\mu_{\rm rel}$ (mas ${\rm yr}^{-1}$) & $0.397\pm0.029$ & $0.421\pm0.034$ & $0.879\pm0.091$ & $0.892\pm0.094$  \\
$M_{\rm S,H}$ ($M_\odot$) & $0.861\pm0.047$ & $0.868\pm0.045$ & $0.865\pm0.045$ & $0.864\pm0.045$ \\
$M_{\rm S,C}$ ($M_\odot$) & $0.036\pm0.001$ & $0.071\pm0.004$ & $0.104\pm0.012$ & $0.064\pm0.005$\\
$a_{\rm S}$ ($10^{-2}$ au) & $5.305\pm0.029$ & $5.885\pm0.040$ & $5.960\pm0.056$ & $5.957\pm0.051$\\
$L_{\rm S,C}/L_{\rm S,H}$ ($10^{-4}$) & $0.001\pm0.001$ & $0.010\pm0.045$ & $1.514\pm0.535$ & $0.004\pm0.010$\\
\hline \hline
$\chi^2$ & 10829.1 & 10831.6 & 10824.7 & 10830.5\\
$\Delta\chi^2$ & 4.4 & 6.9 & - & 5.8\\
\enddata
%\tablenotetext{1}{From the best of 2L1S + xallarap model.}
\end{deluxetable*}
%
\begin{deluxetable*}{c|ccccccccccc}[t!]
\tablecaption{Source system properties of the 2L1S + xallarap close models 
\label{tab:Source_xallarap_close}}
\tablewidth{0pt}
%\tablehead{
%\multicolumn{1}{c|}{Parameters} & \multicolumn{1}{c}{$q\leq0.001$} & \multicolumn{1}{c}{$0.001<q\leq0.01$} & \multicolumn{1}{c}{$0.01<q\leq0.1$} & \multicolumn{1}{c}{$0.1<q\leq1$} 
%}
\tablehead{Model & XLclose1 & XLclose2 & XLclose3 & XLclose4\\
\multicolumn{1}{c|}{range of $q$} & \multicolumn{1}{c}{$q\leq0.001$} & \multicolumn{1}{c}{$0.001<q\leq0.01$} & \multicolumn{1}{c}{$0.1<q\leq0.5$} & \multicolumn{1}{c}{$0.1<q\leq1$} 
}
%\decimalcolnumbers
\startdata
$I_{\rm s,0}$ (mag) & $19.027\pm0.048$ & $18.954\pm0.048$ & $18.990\pm0.048$ & 	$19.011\pm0.048$ \\
$(V-I)_{\rm s,0}$ (mag) & $0.779\pm0.068$ & $0.782\pm0.068$ & $0.783\pm0.068$ & $0.781\pm0.068$ \\
$\theta_{\rm E}$ (mas) & $0.479\pm0.204$ &  $0.124\pm0.012$ & $0.250\pm0.024$ & $0.240\pm0.024$\\
$\mu_{\rm rel}$ (mas ${\rm yr}^{-1}$) & $2.105\pm0.897$ & $0.636\pm0.062$ & $0.973\pm0.099$ & $0.891\pm0.094$  \\
$M_{\rm S,H}$ ($M_\odot$) & $0.863\pm0.047$ & $0.871\pm0.045$ & $0.867\pm0.045$ & $0.865\pm0.045$ \\
$M_{\rm S,C}$ ($M_\odot$) & $0.172\pm0.060$ & $0.072\pm0.005$ & $0.048\pm0.005$ & $0.067\pm0.009$\\
$a_{\rm S}$ ($10^{-2}$ au) & $6.150\pm0.128$ & $5.880\pm0.044$ & $5.943\pm0.058$ & $5.953\pm0.055$\\
$L_{\rm S,C}/L_{\rm S,H}$ ($10^{-7}$) & $5.573\pm4.514$ & $0.018\pm0.024$ & $1.019\pm0.381$ & $0.004\pm0.047$\\
\hline \hline
$\chi^2$ & 10836.8 & 10831.4 & 10842.5 & 10830.7\\
$\Delta\chi^2$\tablenotemark{1} & 12.2 & 6.8 & - & 6.0\\
\enddata
\tablenotetext{1}{From the best of 2L1S + xallarap model.}
\end{deluxetable*}
\end{comment}
\begin{comment}
\begin{deluxetable*}{c|ccccccccccc}[t!]
\tablecaption{Source system properties of the 2L1S + xallarap wide models 
\label{tab:Source_xallarap_wide}}
\tablewidth{0pt}
\tablehead{Model & XLwide1 & XLwide2 & XLwide3 & XLwide4\\
\multicolumn{1}{c|}{range of $q$} & \multicolumn{1}{c}{$q\leq0.001$} & \multicolumn{1}{c}{$0.001<q\leq0.01$} & \multicolumn{1}{c}{$0.01<q\leq0.1$} & \multicolumn{1}{c}{$0.1<q\leq1$} 
}
%\tablehead{
%\multicolumn{1}{c|}{Parameters} & \multicolumn{1}{c}{$q\leq0.001$} & \multicolumn{1}{c}{$0.001<q\leq0.01$} & \multicolumn{1}{c}{$0.01<q\leq0.1$} & \multicolumn{1}{c}{$0.1<q\leq1$} 
%}
%\decimalcolnumbers
\startdata
$I_{\rm s,0}$ (mag) & $18.831\pm0.048$ & $18.986\pm0.048$ & $19.011\pm0.048$ & $19.014\pm0.048$ \\
$(V-I)_{\rm s,0}$ (mag) & $0.777\pm0.068$ & $0.780\pm0.068$ & $0.780\pm0.068$ & $0.781\pm0.068$ \\
$\theta_{\rm E}$ (mas) & $0.061\pm0.004$ &  $0.124\pm0.010$ & $0.236\pm0.024$ & $0.256\pm0.027$\\
$\mu_{\rm rel}$ (mas ${\rm yr}^{-1}$) & $0.397\pm0.029$ & $0.421\pm0.034$ & $0.879\pm0.091$ & $0.892\pm0.094$  \\
$M_{\rm S,H}$ ($M_\odot$) & $0.861\pm0.047$ & $0.868\pm0.045$ & $0.865\pm0.045$ & $0.864\pm0.045$ \\
$M_{\rm S,C}$ ($M_\odot$) & $0.036\pm0.001$ & $0.071\pm0.004$ & $0.104\pm0.012$ & $0.064\pm0.005$\\
$a_{\rm S}$ ($10^{-2}$ au) & $5.305\pm0.029$ & $5.885\pm0.040$ & $5.960\pm0.056$ & $5.957\pm0.051$\\
$L_{\rm S,C}/L_{\rm S,H}$ ($10^{-4}$) & $0.001\pm0.001$ & $0.010\pm0.045$ & $1.514\pm0.535$ & $0.004\pm0.010$\\
\hline \hline
$\chi^2$ & 10829.1 & 10831.6 & 10824.7 & 10830.5\\
$\Delta\chi^2$ & 4.4 & 6.9 & - & 5.8\\
\enddata
%\tablenotetext{1}{From the best of 2L1S + xallarap model.}
\end{deluxetable*}
%
\begin{deluxetable*}{c|ccccccccccc}[t!]
\tablecaption{Source system properties of the 2L1S + xallarap close models 
\label{tab:Source_xallarap_close}}
\tablewidth{0pt}
%\tablehead{
%\multicolumn{1}{c|}{Parameters} & \multicolumn{1}{c}{$q\leq0.001$} & \multicolumn{1}{c}{$0.001<q\leq0.01$} & \multicolumn{1}{c}{$0.01<q\leq0.1$} & \multicolumn{1}{c}{$0.1<q\leq1$} 
%}
\tablehead{Model & XLclose1 & XLclose2 & XLclose3 & XLclose4\\
\multicolumn{1}{c|}{range of $q$} & \multicolumn{1}{c}{$q\leq0.001$} & \multicolumn{1}{c}{$0.001<q\leq0.01$} & \multicolumn{1}{c}{$0.01<q\leq0.1$} & \multicolumn{1}{c}{$0.1<q\leq1$} 
}
%\decimalcolnumbers
\startdata
$I_{\rm s,0}$ (mag) & $19.027\pm0.048$ & $18.954\pm0.048$ & $19.005\pm0.048$ & 	$19.011\pm0.048$ \\
$(V-I)_{\rm s,0}$ (mag) & $0.779\pm0.068$ & $0.782\pm0.068$ & $0.781\pm0.068$ & $0.781\pm0.068$ \\
$\theta_{\rm E}$ (mas) & $0.479\pm0.204$ &  $0.124\pm0.012$ & $0.231\pm0.024$ & $0.240\pm0.024$\\
$\mu_{\rm rel}$ (mas ${\rm yr}^{-1}$) & $2.105\pm0.897$ & $0.636\pm0.062$ & $0.878\pm0.095$ & $0.891\pm0.094$  \\
$M_{\rm S,H}$ ($M_\odot$) & $0.863\pm0.047$ & $0.871\pm0.045$ & $0.865\pm0.045$ & $0.865\pm0.045$ \\
$M_{\rm S,C}$ ($M_\odot$) & $0.172\pm0.060$ & $0.072\pm0.005$ & $0.102\pm0.009$ & $0.067\pm0.009$\\
$a_{\rm S}$ ($10^{-2}$ au) & $6.150\pm0.128$ & $5.880\pm0.044$ & $5.983\pm0.043$ & $5.953\pm0.055$\\
$L_{\rm S,C}/L_{\rm S,H}$ ($10^{-4}$) & $5.573\pm4.514$ & $0.018\pm0.024$ & $1.451\pm0.393$ & $0.004\pm0.047$\\
\hline \hline
$\chi^2$ & 10836.8 & 10831.4 & 10825.3 & 10830.7\\
$\Delta\chi^2$\tablenotemark{1} & 12.2 & 6.8 &	0.6 & 6.0\\
\enddata
\tablenotetext{1}{From the best of 2L1S + xallarap model.}
\end{deluxetable*}
\end{comment}
\begin{comment}
The distance from the Earth to the lensing system $D_{\rm L}$  can be described by Equation \ref{eq:D_L}, and the total mass of the host and companion in the lensing system $M_{\rm L}$ can be described by the following equation \citep{Gaudi2012},

\begin{equation}\label{eq:M_L}
    M_{\rm L}=\frac{\theta_{\rm E}}{\kappa \pi_{\rm E}},
\end{equation}

\noindent where $\kappa = 4G/(c^2 {\rm au}) \sim 8.144$ [mas  $M_{\odot}^{-1}$]. 
The host mass of the lens system $M_{\rm L,H}$ and companion mass $M_{\rm L,C}$ of the lens system and their projection semi-major distance to the lensing plane $a_{{\rm L},\perp}$ can be expressed by the following equations, respectively,

\begin{equation}\label{eq:M_LH}
    M_{\rm L,H}=\frac{M_{\rm L}}{1+q},
\end{equation}

\begin{equation}\label{eq:M_LC}
    M_{\rm L,C}=\frac{qM_{\rm L}}{1+q},
\end{equation}

\begin{equation}\label{eq:a_Lp}
    a_{{\rm L},\perp}=s \theta_{\rm E} D_{\rm L},
\end{equation}

The expected value of the semi-major distance of lens companion in three dimensions can be expressed by $a_{\rm L,exp}=\sqrt{3/2}a_{\rm L,\perp}$.
We calculated $a_{\rm L,exp}/a_{\rm snow}$, which is the value of $a_{\rm L,exp}$ normalized by the snow line position $a_{\rm snow}=2.7(M_{\rm H}/M_\odot)$. 
Table \ref{tab:Source_X_P_LOM} shows the properties of the source and lens systems calculated for the 2L1S xallarap + parallax + lens orbital motion models in Table \ref{tab:param_X+P+LOM}. 
As can be seen from Table \ref{tab:Source_X_P_LOM}, the properties of the lens system have large uncertainties. 
This is due to the large uncertainties in the parallax and lens revolution parameters shown in Table \ref{tab:param_X+P+LOM}.
However, the source system properties are consistently the same as those of the 2L1S+xallarap models shown in Tables \ref{tab:Source_xallarap_wide} and \ref{tab:Source_xallarap_close}.

\begin{comment}
\begin{deluxetable*}{c|ccccccccccc}[t!]
\tablecaption{Source system properties of the 2L1S + xallarap + parallax + lens orbital motion models 
\label{tab:Source_X_P_LOM}}
\tablewidth{0pt}
\tablehead{
\multicolumn{1}{c|}{Parameters} & \multicolumn{1}{c}{$q\leq0.05$} & \multicolumn{1}{c}{$0.005<q\leq0.1$} & \multicolumn{1}{c}{$0.1<q\leq0.5$} & \multicolumn{1}{c}{$0.5<q\leq1$} 
}
%\decimalcolnumbers
\startdata
$I_{\rm s,0}$ (mag) & $18.882\pm0.026$ & $18.919\pm0.011$ & $18.979\pm0.036$ & 	$18.976\pm0.031$ \\
$(V-I)_{\rm s,0}$ ($10^{-1}$ mag) & $7.829\pm0.018$ & $7.777\pm0.018$ & $7.806\pm0.019$ & $7.800\pm0.018$ \\
$\theta_{\rm E}$ (mas) & $0.249\pm0.018$ &  $0.250\pm0.013$ & $0.284\pm0.016$ & $0.281\pm0.018$\\
$\mu_{\rm rel}$ (mas ${\rm yr}^{-1}$) & $0.928\pm0.067$ & $0.857\pm0.043$ & $0.949\pm0.065$ & $0.946\pm0.078$  \\
$M_{\rm S,H}$ ($M_\odot$) & $0.880\pm0.003$ & $0.876\pm0.001$ & $0.868\pm0.004$ & $0.869\pm0.004$ \\
$M_{\rm S,C}$ ($M_\odot$) & $0.010\pm0.001$ & $0.080\pm0.006$ & $0.054\pm0.005$ & $0.064\pm0.007$\\
$a_{\rm S}$ ($10^{-2}$ au) & $6.04\pm0.03$ & $5.98\pm0.04$ & $5.94\pm0.03$ & $5.98\pm0.05$\\
$L_{\rm S,C}/L_{\rm S,H}$ ($10^{-3}$) & $1.01\pm0.44$ & $1.19\pm0.21$ & $0.49\pm0.12$ & $0.72\pm0.20$\\
$D_{\rm L}$ (kpc) & $5.23\pm0.36$ & $7.43\pm0.31$ & $4.48\pm0.32$ & $4.69\pm0.56$\\
$M_{\rm L,H}$ ($M_\odot$)  & $0.112\pm0.014$ & $0.753\pm0.356$ & $0.068\pm0.010$ & $0.070\pm0.016$\\
$M_{\rm L,C}$ ($10^{-2}$ $M_\odot$) & $0.35\pm0.04$ & $5.31\pm2.47$ & $3.23\pm0.47$ & $3.92\pm0.96$\\
$a_{{\rm L},\perp}$ (au) & $0.29\pm0.03$ & $0.29\pm0.02$ & $0.10\pm0.01$ & $0.10\pm0.01$\\
$a_{\rm L,exp}$ (au) & $0.35\pm0.03$ & $0.36\pm0.03$ & $0.12\pm0.01$ & $0.12\pm0.01$\\
$a_{\rm L,exp}/a_{\rm snow}$ & $1.17\pm0.10$ & $0.18\pm0.09$ & $0.68\pm0.07$ & $0.65\pm0.10$\\
\hline \hline
$\chi^2$ & 10750.5 & 10746.1 & 10716.1 & 10742.4\\
$\Delta\chi^2$\tablenotemark{1} & 34.6 & 30.1 &	- & 26.4\\
\enddata
\tablenotetext{1}{From the best of 2L1S + xallarap + parallax + lens orbital motion model.}
\end{deluxetable*}
\end{comment}

\section{Lens System Properties by Bayesian Analysis}
\label{sec:Lens_Properties}


% Figure environment removed

% Figure environment removed

\begin{deluxetable*}{c|ccccccccccc}[t!]
\tablecaption{Lens system properties of the 2L1S + xallarap models 
\label{tab:Lens_xallarap}}
\tablewidth{0pt}
%\tablehead{
%\multicolumn{1}{c|}{Parameters} & \multicolumn{1}{c}{$q\leq0.001$} & \multicolumn{1}{c}{$0.001<q\leq0.01$} & \multicolumn{1}{c}{$0.01<q\leq0.1$} & \multicolumn{1}{c}{$0.1<q\leq1$} 
%}
\tablehead{Model & XLclose1 & XLclose2 & XLwide1 & XLwide2 \\
\multicolumn{1}{c|}{range of $q$} & \multicolumn{1}{c}{$q\leq0.1$} & \multicolumn{1}{c}{$0.1<q\leq1$} & \multicolumn{1}{c}{$q\leq0.1$} & \multicolumn{1}{c}{$0.1<q\leq1$}
}
%\decimalcolnumbers
\startdata
$D_{\rm L}$ (kpc) & $7.27_{-1.17}^{+1.10}$ & $7.24_{-1.17}^{+1.09}$ & $7.24_{-1.18}^{+1.09}$ & $7.12_{-1.214}^{+1.05}$\\
$M_{\rm L,H}$ ($M_\odot$) & $0.23_{-0.12}^{+0.28}$ & $0.25_{-0.13}^{+0.29}$ & $0.26_{-0.13}^{+0.29}$ & $0.37_{-0.19}^{+0.32}$\\
$M_{\rm L,C}$ ($M_\odot$) & $0.02_{-0.01}^{+0.03}$ & $0.11_{-0.06}^{+0.13}$ & $0.03_{-0.01}^{+0.03}$ & $0.35_{-0.18}^{+0.30}$\\
$a_{\rm {L,\perp}}$ (au) & $0.20_{-0.04}^{+0.04}$ & $0.13_{-0.02}^{+0.02}$ & $11.55_{-2.09}^{+2.03}$ & $34.74_{-7.69}^{+7.51}$\\
%$\tilde r_{\rm E,H}$ (au) & $1.454_{-0.263}^{+0.262}$ & $1.527_{-0.283}^{+0.281}$ & $1.560_{-0.282}^{+0.274}$ & $1.926_{-0.426}^{+0.416}$\\
$a_{\rm L,exp}$ (au) & $0.25_{-0.06}^{+0.13}$ & $0.16_{-0.04}^{+0.08}$ & $13.91_{-3.15}^{+7.39}$ & $42.10_{-10.92}^{+21.78}$\\
%$r_{\rm E,H}$ (au) & $1.755_{-0.400}^{+0.943}$ & $1.844_{-0.427}^{+0.985}$ & $1.878_{-0.426}^{+0.999}$ & $2.334_{-0.605}^{+1.207}$\\
$V_{\rm L,H}$ (mag) & $30.61_{-2.54}^{+2.56}$ & $30.34_{-2.66}^{+2.42}$ & $30.25_{-2.71}^{+2.31}$ & $29.17_{-3.38}^{+1.83}$\\
$I_{\rm L,H}$ (mag) & $26.09_{-1.75}^{+1.66}$ & $25.90_{-1.83}^{+1.57}$ & $25.84_{-1.86}^{+1.51}$ & $25.05_{-2.35}^{+1.22}$\\
$H_{\rm L,H}$ (mag) & $22.53_{-1.73}^{+1.27}$ & $22.35_{-1.80}^{+1.26}$ & $22.30_{-1.82}^{+1.23}$ & $21.53_{-2.02}^{+1.17}$\\
$K_{\rm L,H}$ (mag) & $22.11_{-1.70}^{+1.21}$ & $21.93_{-1.75}^{+1.20}$ & $21.87_{-1.77}^{+1.18}$ & $21.12_{-1.94}^{+1.14}$\\
$V_{\rm L,C}$ (mag) & $41.53_{-1.76}^{+0.98}$ & $33.68_{-2.97}^{+6.20}$ & $41.42_{-2.13}^{+1.03}$ & $29.35_{-3.13}^{+1.88}$\\
$I_{\rm L,C}$ (mag) & $35.24_{-3.05}^{+1.75}$ & $28.13_{-1.94}^{+3.32}$ & $34.85_{-3.13}^{+1.88}$ & $25.18_{-2.16}^{+1.25}$\\
$H_{\rm L,C}$ (mag) & $33.74_{-3.99}^{+3.00}$ & $24.12_{-1.50}^{+5.98}$ & $33.42_{-4.24}^{+3.12}$ & $21.66_{-1.92}^{+1.16}$\\
$K_{\rm L,C}$ (mag) & $30.45_{-2.24}^{+1.47}$ & $23.63_{-1.44}^{+3.68}$ & $30.00_{-2.24}^{+1.54}$ & $21.25_{-1.85}^{+1.14}$\\
$V_{\rm L,total}$ (mag) & $30.60_{-2.54}^{+2.55}$ & $30.29_{-2.68}^{+2.46}$ & $30.25_{-2.71}^{+2.31}$ & $28.50_{-3.27}^{+1.85}$\\
$I_{\rm L,total}$ (mag) & $26.08_{-1.75}^{+1.65}$ & $25.77_{-1.86}^{+1.67}$ & $25.83_{-1.86}^{+1.51}$ & $24.36_{-2.26}^{+1.24}$\\
$H_{\rm L,total}$ (mag) & $22.53_{-1.73}^{+1.27}$ & $22.16_{-1.76}^{+1.44}$ & $22.30_{-1.82}^{+1.22}$ & $20.84_{-1.98}^{+1.16}$\\
$K_{\rm L,total}$ (mag) & $22.10_{-1.70}^{+1.21}$ & $21.73_{-1.71}^{+1.38}$ & $21.87_{-1.77}^{+1.18}$ & $20.43_{-1.89}^{+1.14}$\\
$V_{\rm Blend}$ (mag) & $20.21 \pm  0.03$ & $20.21 \pm 0.03$ & $20.21 \pm 0.03$ & $20.21 \pm 0.03$\\
$I_{\rm Blend}$ (mag) & $19.25 \pm  0.01$ & $19.25 \pm 0.01$ & $19.25 \pm 0.01$ & $19.25 \pm 0.01$\\
\hline \hline
$\chi^2$ & 10856.4 & 10840.9 & 10861.2 & 10842.7\\
$\Delta\chi^2$ & 15.5 & - & 20.3 & 1.8\\
\enddata
\end{deluxetable*}
\begin{comment}
%20230705
\begin{deluxetable*}{c|ccccccccccc}[t!]
\tablecaption{Lens system properties of the 2L1S + xallarap models 
\label{tab:Lens_xallarap}}
\tablewidth{0pt}
%\tablehead{
%\multicolumn{1}{c|}{Parameters} & \multicolumn{1}{c}{$q\leq0.001$} & \multicolumn{1}{c}{$0.001<q\leq0.01$} & \multicolumn{1}{c}{$0.01<q\leq0.1$} & \multicolumn{1}{c}{$0.1<q\leq1$} 
%}
\tablehead{Model & XLclose1 & XLclose2 & XLwide1 & XLwide2 \\
\multicolumn{1}{c|}{range of $q$} & \multicolumn{1}{c}{$q\leq0.1$} & \multicolumn{1}{c}{$0.1<q\leq1$} & \multicolumn{1}{c}{$q\leq0.1$} & \multicolumn{1}{c}{$0.1<q\leq1$}
}
%\decimalcolnumbers
\startdata
$D_{\rm L}$ (kpc) & $7.272_{-1.168}^{+1.098}$ & $7.244_{-1.171}^{+1.087}$ & $7.242_{-1.183}^{+1.091}$ & $7.115_{-1.215}^{+1.047}$\\
$M_{\rm L,H}$ ($M_\odot$) & $0.232_{-0.117}^{+0.277}$ & $0.253_{-0.128}^{+0.287}$ & $0.261_{-0.131}^{+0.290}$ & $0.373_{-0.190}^{+0.319}$\\
$M_{\rm L,C}$ ($M_\odot$) & $0.022_{-0.011}^{+0.026}$ & $0.111_{-0.056}^{+0.126}$ & $0.026_{-0.013}^{+0.029}$ & $0.352_{-0.179}^{+0.301}$\\
$a_{\rm {L,\perp}}$ (au) & $0.205_{-0.037}^{+0.037}$ & $0.130_{-0.024}^{+0.024}$ & $11.549_{-2.087}^{+2.031}$ & $34.741_{-7.691}^{+7.512}$\\
%$\tilde r_{\rm E,H}$ (au) & $1.454_{-0.263}^{+0.262}$ & $1.527_{-0.283}^{+0.281}$ & $1.560_{-0.282}^{+0.274}$ & $1.926_{-0.426}^{+0.416}$\\
$a_{\rm L,exp}$ (au) & $0.247_{-0.056}^{+0.133}$ & $0.157_{-0.036}^{+0.084}$ & $13.906_{-3.150}^{+7.393}$ & $42.100_{-10.920}^{+21.783}$\\
%$r_{\rm E,H}$ (au) & $1.755_{-0.400}^{+0.943}$ & $1.844_{-0.427}^{+0.985}$ & $1.878_{-0.426}^{+0.999}$ & $2.334_{-0.605}^{+1.207}$\\
$V_{\rm L,H}$ (mag) & $30.608_{-2.536}^{+2.563}$ & $30.338_{-2.663}^{+2.416}$ & $30.252_{-2.709}^{+2.315}$ & $29.166_{-3.382}^{+1.828}$\\
$I_{\rm L,H}$ (mag) & $26.088_{-1.751}^{+1.659}$ & $25.897_{-1.834}^{+1.573}$ & $25.836_{-1.863}^{+1.510}$ & $25.047_{-2.349}^{+1.225}$\\
$H_{\rm L,H}$ (mag) & $22.532_{-1.732}^{+1.274}$ & $22.354_{-1.800}^{+1.255}$ & $22.296_{-1.819}^{+1.225}$ & $21.526_{-2.024}^{+1.165}$\\
$K_{\rm L,H}$ (mag) & $22.106_{-1.695}^{+1.215}$ & $21.931_{-1.755}^{+1.204}$ & $21.875_{-1.770}^{+1.177}$ & $21.115_{-1.939}^{+1.142}$\\
$V_{\rm L,C}$ (mag) & $41.529_{-1.761}^{+0.983}$ & $33.684_{-2.966}^{+6.200}$ & $41.424_{-2.134}^{+1.029}$ & $29.351_{-3.135}^{+1.879}$\\
$I_{\rm L,C}$ (mag) & $35.236_{-3.052}^{+1.745}$ & $28.128_{-1.937}^{+3.321}$ & $34.852_{-3.131}^{+1.882}$ & $25.179_{-2.162}^{+1.252}$\\
$H_{\rm L,C}$ (mag) & $33.744_{-3.988}^{+2.955}$ & $24.117_{-1.504}^{+5.979}$ & $33.416_{-4.241}^{+3.117}$ & $21.658_{-1.924}^{+1.163}$\\
$K_{\rm L,C}$ (mag) & $30.448_{-2.241}^{+1.472}$ & $23.632_{-1.441}^{+3.685}$ & $29.999_{-2.243}^{+1.540}$ & $21.245_{-1.848}^{+1.137}$\\
$V_{\rm L,total}$ (mag) & $30.604_{-2.537}^{+2.553}$ & $30.289_{-2.679}^{+2.464}$ & $30.250_{-2.710}^{+2.309}$ & $28.503_{-3.270}^{+1.851}$\\
$I_{\rm L,total}$ (mag) & $26.085_{-1.751}^{+1.653}$ & $25.771_{-1.856}^{+1.669}$ & $25.835_{-1.864}^{+1.506}$ & $24.359_{-2.261}^{+1.238}$\\
$H_{\rm L,total}$ (mag) & $22.530_{-1.732}^{+1.271}$ & $22.162_{-1.764}^{+1.442}$ & $22.295_{-1.820}^{+1.224}$ & $20.838_{-1.975}^{+1.165}$\\
$K_{\rm L,total}$ (mag) & $22.103_{-1.696}^{+1.212}$ & $21.727_{-1.712}^{+1.380}$ & $21.873_{-1.771}^{+1.176}$ & $20.426_{-1.895}^{+1.139}$\\
$V_{\rm Blend}$ (mag) & $20.212 \pm  0.034$ & $20.213 \pm 0.034$ & $20.214 \pm 0.034$ & $20.212 \pm 0.034$\\
$I_{\rm Blend}$ (mag) & $19.248 \pm  0.013$ & $19.252 \pm 0.013$ & $19.253 \pm 0.013$ & $19.248 \pm 0.013$\\
\hline \hline
$\chi^2$ & 10856.4 & 10840.9 & 10861.2 & 10842.7\\
$\Delta\chi^2$ & 15.5 & - & 20.3 & 1.8\\
\enddata
\end{deluxetable*}
\end{comment}
The distance from the Earth to the lensing system, $D_{\rm L}$, and the total mass of the host and companion in the lensing system, $M_{\rm L}$, can be described by the following equations \citep{Gaudi2012},

\begin{equation}
\label{eq:D_L}
    D_{\rm L}=\frac{\rm au}{\pi_{\rm E}\theta_{\rm E}+\pi_{\rm S}},
\end{equation}

\begin{equation}\label{eq:M_L}
    M_{\rm L}=\frac{\theta_{\rm E}}{\kappa \pi_{\rm E}},
\end{equation}

\noindent
where $\kappa = 4G/(c^2 {\rm au}) \sim 8.144$ [mas  $M_{\odot}^{-1}$] and $\pi_{\rm S}$ is the parallax of the source star written as $\pi_{\rm S}={\rm au}/D_{\rm S}$.
%\noindent where $\kappa = 4G/(c^2 {\rm au}) \sim 8.144$ [mas  $M_{\odot}^{-1}$], $a_\perp$ is the semi-major axis of the lensing companion star projected onto the lensing plane, and $\pi_{\rm S}$ is the parallax of the source star, which can be written as $\pi_{\rm S}={\rm au}/D_{\rm S}$ using the distance $D_{\rm S}$ from the Earth to the source system.

Since the parallax effect was not detected in this event, we conducted a Bayesian analysis \citep{Beaulieu+2006,Gould+2006,Bennett+2008} to estimate the parameters of the lens system for the 2L1S + xallarap models.
For the prior probability distributions, we used the mass density and velocity distributions of the Galaxy model from \citet{Han+1995}, and we used the mass function from \citet{Sumi+2011}.
%Since the number density distribution of stellar masses and distances in the prior probability distribution assumes a single star, we use the event timescale $t_{\rm {E,H}}$ and the Einstein radius $\theta_{\rm {E,H}}$ due only to the lens host.
Since the prior distribution only considers a single star, we scaled the event timescale and the Einstein radius to match those of the lens host so that the physical parameters of the lens host and companion can be properly estimated.
The event timescale of the lens host $t_{\rm {E,H}}$ and the Einstein radius of the lens host $\theta_{\rm {E,H}}$ are expressed using the mass ratio $q$ as follows:

\begin{equation}
\label{eq:t_E,H}
    t_{\rm {E,H}}=\frac{t_{\rm E}}{\sqrt{1+q}},
\end{equation}

\begin{equation}\label{eq:theta_E,H}
    \theta_{\rm {E,H}}=\frac{\theta_{\rm E}}{\sqrt{1+q}}.
\end{equation}

We also estimated the apparent magnitudes of the lens system in the $V$-, $I$-, $K$-, and $H$-bands with extinction.
The magnitudes were obtained using the mass-luminosity relation for main-sequence stars \citep{Henry+1993,Kroupa+1997} and the isochrone model for 5 Gyr old sub-stellar objects \citep{Baraffe+2003}.
The blending flux $f_{\rm b}$ from the light curve modeling was used as the upper limit of the lens brightness.
Following \citet{Bennett+2015}, we estimated the extinction in front of the lens using the following equation:

\begin{equation}\label{eq:extinction}
    A_{i, \rm {L}}=\frac{1-\exp{[-D_{\rm L}/h_{\rm {dust}}}]}{1-\exp{[-D_{\rm S}/h_{\rm {dust}}}]}A_{i, \rm {S}},
\end{equation}

\noindent
where $i$ corresponds to the observed wavelength band, $A_{i, \rm {L}}$ is the total extinction in the $i$-band of the lens, $A_{i, \rm {S}}$ is the total extinction in the $i$-band of the source, $h_{\rm {dust}}$ is the scale length of dust in the event direction, given by $h_{\rm {dust}}=(0.1 \: {\rm {kpc}})/\sin {|b|}$ as a function of the Galactic latitude $b$ of the event.
We estimated $A_{\rm H}$ and $A_{\rm K}$ from $A_{\rm V}$ using the wavelength dependence of extinction law in the direction of the Galactic center from \citet{Nishiyama+2008}.

Table~\ref{tab:Lens_xallarap} lists the estimated parameters: the distance from the Earth to the lens, $D_{\rm L}$; the lens host mass, $M_{\rm {L,H}}$; the lens companion mass, $M_{\rm {L,C}}$; the orbital radius projected to the observation plane, $a_{\rm L, \perp}$; the expected orbital radius, $a_{\rm L, exp}$; the magnitudes with the extinction in the four wavelength bands $V_{{\rm L},j}$, $I_{{\rm L},j}$, $H_{{\rm L},j}$ and $K_{{\rm L},j}$ where $j$ consists of ``H" for the lens host, ``C" for the lens companion and `total" for the host and companion combined; the magnitudes of the blends in the $V$- and $I$-bands which are the upper limits of brightness in the lens system, $V_{\rm {blend}}$, $I_{\rm {blend}}$.
%Table~\ref{tab:Lens_xallarap} lists the estimated parameters: the distance from the Earth to the lens, $D_{\rm L}$; the lens host mass, $M_{\rm {L,H}}$;  the lens companion mass, $M_{\rm {L,C}}$; the orbital radius projected to the observation plane, $a_{\rm L, \perp}$; the expected orbital radius, $a_{\rm L, exp}$; the magnitude with extinction of the lens host at the $V$-, $I$-, $H$-, and $K$-band, $V_{\rm {L,H}}$, $V_{\rm {L,H}}$, $H_{\rm {L,H}}$ and $K_{\rm {L,H}}$; the magnitude with extinction of the lens companion at the $V$-, $I$-, $H$-, and $K$-band, $V_{\rm {L,C}}$, $V_{\rm {L,C}}$, $H_{\rm {L,C}}$ and $K_{\rm {L,C}}$; the apparent $V$-, $I$-, $H$-, and $K$-band magnitude with extinction of the combined lens host and companion, $V_{\rm {L,total}}$, $V_{\rm {L,total}}$, $H_{\rm {L,total}}$ and $K_{\rm {L,total}}$; the magnitudes of the blends in the $V$- and $I$-bands which are the upper limits of brightness in the lens system, $V_{\rm {blend}}$, $I_{\rm {blens}}$.
%The expected Einstein radius of the lens host was calculated by $r_{\rm {E,H}}=\sqrt{3/2} \: \tilde r_{\rm {E,H}}$ and the expected orbital radius was calculated by $a_{\rm L}=\sqrt{3/2}\: \tilde a_{\rm L}$.
Figures~\ref{fig:Lens_XLclose2} and \ref{fig:Lens_XLwide2} show the posterior probability distributions for XLclose2 and XLwide2, respectively.
The distribution of XLclose2 indicates a M-type or K-type stellar binary with a projected orbital radius $a_{\rm {L,\perp}} = 0.13_{-0.02}^{+0.02}$ au located $7.2_{-1.2}^{+1.1}$ kpc from the Earth.
The distribution of XLwide2 also indicates a M-type or K-type stellar binary with a projected orbital radius $a_{\rm {L,\perp}} = 34.74_{-7.69}^{+7.51}$ au located $7.1_{-1.2}^{+1.0}$ kpc from the Earth. 
%The results show that the best model, XLclose2 indicates an M-type or K-type stellar binary with an orbital radius of $0.16_{-0.04}^{+0.08}$ au at $7.2_{-1.2}^{+1.1}$ kpc from Earth, while best model for $s>1$, XLwide2 indicates also an M-type or K-type stellar binary with an orbital radius of $42.10_{-10.92}^{+21.78}$ au at $7.1_{-1.2}^{+1.0}$ kpc from Earth.
%Comparing the properties of the lens systems of the four models listed in Table~\ref{tab:Lens_xallarap}, while the parameters related to the companion differ significantly among the models, they are consistent in that the the lens host is about M-type stellar mass, and distance from the Earth is $D_{\rm L} \sim 7$ kpc.
Comparing the properties of the lens systems of the four models listed in Table~\ref{tab:Lens_xallarap}, while the parameters related to the companion differ significantly among the models, they are consistent in that the stellar type and the distance from the Earth.
%It is also consistent that the contrast between the apparent magnitude with extinction of the source and the combined apparent magnitude with extinction of the lens host and companion is about 7 mag at the $V$-band and about 5 mag at the $I$-band.
%Note that the apparent magnitude with extinction of the source is $(V, \:I)_{\rm S}=(23.562 \pm 0.034$, $21.035\pm0.015)$ mag.
%From Section~\ref{sec:Source_Lens_Properties},the apparent magnitude with extinction of the source was $(V, I)_{\rm S} = (23.562\pm0.034, 21.035\pm0.014)$ mag.
%Therefore, XLclose1, XLclose2, and XLwide1 have a contrast between the apparent magnitude with extinction of the lens and the apparent magnitude with extinction of the source of $(\Delta V, \Delta I)\sim (7,5)$ mag, and XLwide2 has $(\Delta V, \Delta I)\sim (5,3)$ mag.
%XLclose1, XLclose2, and XLwide1 have a contrast between the apparent magnitude with extinction of the total brightness of the lens system and the apparent magnitude with extinction of the source of $(\Delta V, \Delta I)\sim (7,5)$ mag, and XLwide2 has $(\Delta V, \Delta I)\sim (5,3)$ mag.

As described in Section~\ref{sec:Source_Lens_Properties}, the apparent magnitude of the source for XLclose2 is $(H_{\rm S}, K_{\rm S})=(18.52\pm0.49, 18.29\pm0.48)$.
The apparent magnitude for the lens host and lens companion combined is $(H_{\rm {L,total}}, K_{\rm {L,total}})=(22.16\pm1.16, 21.73\pm1.55)$. 
Therefore, XLclose2 has a contrast between the apparent lens brightness and the apparent source brightness where $3.6\pm1.7$ mag in the $H$-band and $3.4\pm1.6$ mag in the $K$-band.
The XLclose1 and XLwode1 models also have similar contrast to XLclose2, respectively.
On the other hand, the contrast between the apparent lens brightness and the apparent source brightness in the XLwide2 model is $2.3\pm1.6$ in the $H$-band and $2.1\pm1.6$ in the $K$-band, slightly lower contrast than that in XLclose2.

\begin{comment}
As described in Section~\ref{sec:Source_Lens_Properties}, the apparent magnitude of the source for XLclose2 is $(V_{\rm S}, I_{\rm S})=(23.562\pm0.034, 21.035\pm0.015)$. The apparent magnitude for the lens host and lens companion combined is $(V_{\rm {L,total}}, I_{\rm {L,total}})=(30.289_{-2.679}^{+2.464}, 25.771_{-1.856}^{+1.669})$. 
Therefore, XLclose2 has a high contrast between the apparent lens brightness and the apparent source brightness where $\sim7$ mag in the $V$-band and $\sim5$ mag in the $I$-band.
The contrast for XLclose1 or XLwide1 is close to that for XLclose2, $\sim7$ mag in the $V$-band and $\sim5$ mag in the $I$-band.
On the other hand, the apparent magnitude of the source for XLwide2 is $(V_{\rm S}, I_{\rm S})=(23.564\pm0.034, 21.057\pm0.015)$ and the apparent magnitude for the lens host and lens companion combined is $(V_{\rm {L,total}}, I_{\rm {L,total}})=(28.503_{-3.270}^{+1.851}, 24.359_{-2.261}^{+1.238})$.
Therefore, XLwide2 has a low contrast between the apparent lens brightness and the apparent source brightness where $\sim5$ mag in the $V$-band and $\sim3$ mag in the $I$-band.
%Thus, the contrast for XLclose2, XLclose1, and XLwide1 is $\sim2$ mag higher than for XLwide2.
Thus, the contrast of XLwide2 differs by $\sim2$ mag compared to the contrast for XLclose2, XLclose1, and XLwide1.
However, there is a large uncertainty in the lens brightness.
\end{comment}

%Distance from the Earth to the lens, $D_{\rm L}$; the Einstein radius of the host projected to the observation plane, $\tilde r_{\rm {E,H}}$ (upper scale is the orbital radius projected to the observation plane, $\tilde a_{\rm L}$);  the lens host mass, $M_{\rm {L,H}}$ (upper scale is the lens companion mass, $M_{\rm {L,C}}$), and the apparent $V$-, $I$-, $H$-, and $K$-band magnitude with extinction of the combined lens host and companion are indicated, respectively.

%\input{6_Discussion_and_Conclusion.tex}
\section{Discussion and Conclusion}
\label{sec:Discussion_and_Conclusion}
%We performed an analysis of microlensing event OGLE-2019-BLG-0825. 
We performed a detailed analysis of the planetary microlensing candidate, OGLE-2019-BLG-0825.
%We first derived a planetary mass ratio $q \sim 10^{-3}$ with the standard binary lensing model, but found systematic residuals between the model and data.
We first found that there are  systematic residuals with the best fit standard binary model with planetary mass ratio $q \sim 10^{-3}$.  
%Therefore, we examined various possible higher-order effects and their combinations in the gravitational microlensing effect.
Therefore, we examined various combinations of possible higher-order effects.
As a result, we found that models which include the xallarap effect can fit the residuals significantly better than models which do not.

Our Bayesian analysis for the best model XLclose2 estimated the lens host mass to be $0.25_{-0.13}^{+0.29}$ $M_\odot$ and the lens system to be located $7.24_{-1.17}^{+1.09}$ kpc from Earth.
For XLwide2, which is the best solution at $s>1$, the lensing host is $0.37_{-0.19}^{+0.32}$ $M_\odot$, and the lens system is estimated to be located $7.12_{-1.22}^{+1.05}$ kpc from Earth.
Owing to degenerate solutions with various combinations of $(q,s)$ values, the uncertainties in the mass and orbital radius of the lens companion are large.
Since the relative proper motion between the lens and the source is about 1 mas $\rm { {yr}}^{-1}$ and the apparent magnitude contrast is large, it will be more than 30 years before the source and lens can be observed separately with the current high-resolution imaging instruments.
%The European Extremely Large Telescope (ELT), scheduled for first light in 2027, has an angular resolution of about 5 mas, and by about 2027 the separation between source and lens will have expanded to about 9 mas, so it may be possible to observe separately.
In adaptive optics (AO) observations by The European Extremely Large Telescope (ELT), the FWHM is expected to reach 10 mas in the $H$-band and 14 mas in the $K$-band \citep{Ryu+2022}.
%In adaptive optics (AO) observations by The European Extremely Large Telescope (ELT), the FWHM at the H-band is 10 mas and 14 mas at the K-band \citep{Ryu+2022}.
Therefore, it may be possible to observe the source and lens separately by mid-2030.
It is unlikely that the degeneracy of the models will be resolved by follow-up observations because the proper motion and brightness of the lens system are comparable across models, but it may constrain the uncertainty in the lens system properties somewhat.

%The source companion OGLE-2019-BLG-0825Sb has an orbital radius of $0.0601 \pm 0.0006$ au and an orbital period of $5.40 \pm 0.05$ days, orbiting the source primary star OGLE-2019-BLG-0825S with a mass of $0.881 \pm 0.048 $ $M_\odot$.
Calculations applying the $D_{\rm S}=8.0 \pm 1.4$ kpc assumption and the isochrone model with age 10 Gyr in solar metallicity to the source show that the source companion OGLE-2019-BLG-0825Sb in the best 2L1S + xallarap model has a semi major axis of $0.0594 \pm 0.0005$ au and an orbital period of $5.53 \pm 0.05$ days with mass $0.048 \pm 0.004 $ $M_\odot$ orbiting the host source star OGLE-2019-BLG-0825S.
%The mass of the source companion is about that of a brown dwarf to M-type star, and the luminosity ratio at the $I$-band of the source host to the source companion is $L_{\rm S,C}/L_{\rm S,H} = (1.6\pm0.5) \times {10}^{-4}$, which is faint and consistent with this analysis where the companion is not magnified.
%The mass of the source companion is about that of a brown dwarf to M-type star.
The mass of the source companion is about that of a brown dwarf.
The $I$-band luminosity ratio of the companion to the host is $L_{\rm S,C}/L_{\rm S,H} = (1.0\pm0.3) \times {10}^{-7}$, which is faint and consistent with this analysis where the magnified flux of the second source is too weak to be detected.
%The $I$-band luminosity ratio of the companion to the host is $L_{\rm S,C}/L_{\rm S,H} = (1.5\pm0.5) \times {10}^{-4}$, which is faint and consistent with this analysis where the magnified flux of the second source is too weak to be detected.
%The $I$-band luminosity ratio of the companion to the host is $L_{\rm S,C}/L_{\rm S,H} = (1.5\pm0.5) \times {10}^{-4}$, which is faint and consistent with this analysis where the companion is not magnified.
We note that these properties of the source system are almost the same among the various models considered, even though the parameters of the lens system change.

%The standard deviation of the difference between the data and the model in baseline is calculated to be $\sim0.1$ mag for the $I_{\rm {OGLE}}$.
%We considered whether variable stars could also explain the $\sim5$ day luminosity variations of this event.
%We considered whether the $\sim5$ day luminosity variations of this event could be explained if the source is a variable star, without using the xallarap effect.
We considered whether a variable source star could also explain the $\sim5$ day luminosity variations of this event, without using the xallarap effect.
%20230725%Classical Cepheids have pulsation periods ranging from about 1 to 200 days, with a pulsation amplitude in $I$-band of $0.05-1$ mag \citep{Klagyivik+2009}, and the following period-luminosity relations \citep{Gaia+2017}:
Most of Classical Cepheids have a pulsation periods ranging from about 1 to 100 days, and the longest period ones being rare, with a pulsation amplitude in $I$-band of $0.05-1$ mag \citep{Klagyivik+2009}, and the following period-luminosity relations \citep{Gaia+2017}:

\begin{equation}\label{eq:PL_claccical_cepheid}
    M_I=-2.98 \log P – (1.28 \pm 0.08); \sigma_{\rm {rms}}=0.78,
\end{equation}

\noindent
where $\sigma_{\rm {rms}}$ is the variance around the periodic luminosity relation.
At a pulsation period $P=5.50 \pm0.05$ days, the absolute magnitude of a type I Cepheid would be $M_I =-3.48 \pm 0.08$ mag.
However our estimated absolute magnitude is $M_I = 4.5 \pm 0.4$ mag, which is too faint for a classical Cepheid (see Table~\ref{tab:Source_xallarap}).
Type $\mathrm{I}\hspace{-1.2pt}\mathrm{I}$ Cepheids have a pulsation period of about 1 to 50 days, with a pulsation amplitude of $0.3-1.2$ mag, and the following period-luminosity relationships \citep{Ngeow+2022}:

\begin{equation}\label{eq:PL_claccical_type2}
    M_I=-(2.09 \pm 0.08) \log P – (0.39 \pm 0.08); \sigma_{\rm {rms}}=0.24.
\end{equation}

\noindent
For a pulsation period $P=5.50 \pm 0.05$ days, the absolute magnitude of a type $\mathrm{I}\hspace{-1.2pt}\mathrm{I}$ Cepheid would be $M_I=-1.94 \pm 0.13$ mag, which is also not plausible.
RR Lyrae variables have color magnitudes close to main-sequence stars, but with a pulsation period of less than one day \citep[e.g.,][]{Soszynski+2009}.
Delta Scuti variables have a pulsation period of $0.01-0.2$ days, and Gamma Doradus variables have a pulsation period of 0.3-2.6 days, both shorter than the xallarap signal of 5 days, and the spectral type is $A-F$, which is blue compared to the color of the source of this event.
Furthermore, as described in Section \ref{subsec:Xallarap}, we performed a fitting with a model with variable source flux, using the best standard 2L1S model (i.e., close1).
However, the improvement from the best standard 2L1S model was only $139.1$, $\Delta \chi^2 = 764.6$ worse than the 2L1S + xallarap model.
%Therefore, we conclude that it is difficult to explain the xallarap signal with general variable stars.
Therefore, we conclude that it is difficult to explain the xallarap signal assuming a variable source star.
Note that although the conclusion is that the source of this event is not a variable star, many variable stars in the direction of the Galactic bulge have been discovered \citep[e.g.,][]{Soszynski+2011,Iwanek+2019}, and there is a possibility that a candidate planetary microlensing event with a variable source will be observed in the future.

\begin{comment}
\textbf{
We considered whether variable stars could also explain the $\sim5$ day luminosity variations of this event.
%It is not plausible to attribute the source's luminosity variations to a variable star with a period of about 5 days rather than xallarap effect.
Classical cepheids have pulsation periods ranging from about 1 to 200 days, with the following period-luminosity relationships \citep{Gaia+2017},
}

\begin{equation}\label{eq:PL_claccical_cepheid}
    \bm{
    M_I=-2.98 \log P – (1.28 \pm 0.08); \sigma_{\rm {rms}}=0.78,
    }
\end{equation}

\noindent
w\textbf{here $\sigma_{\rm {rms}}$ is the variance around the periodic luminosity relation.
At a pulsation period $P=5.50 \pm0.05$ days, the absolute magnitude of a type I cepheid would be $M_I =-3.48 \pm 0.08$ mag.
However our estimated absolute magnitude is $M_I = 4.5 \pm 0.4$ mag, which is too faint for a classical cepheid (see Table~\ref{tab:Source_xallarap}).
Type $\mathrm{I}\hspace{-1.2pt}\mathrm{I}$ cepheids have a pulsation period of about 1 to 50 days, with the following period-luminosity relationships \citep{Ngeow+2022},
}

\begin{equation}\label{eq:PL_claccical_type2}
    \bm{
    M_I=-(2.09 \pm 0.08) \log P – (0.39 \pm 0.08); \sigma_{\rm {rms}}=0.24.
    }
\end{equation}

\noindent
\textbf{For a pulsation period $P=5.50 \pm 0.05$ days, the absolute magnitude of a type $\mathrm{I}\hspace{-1.2pt}\mathrm{I}$ cepheid would be $M_I=-1.94 \pm 0.13$ mag, which is also not plausible.
RR Lyrae variables have color magnitudes close to main-sequence stars, but with a pulsation period of less than one day \citep[e.g.,][]{Soszynski+2009}.
%RR Lyrae variables in the constellation of Capricorn have color magnitudes closer to main-sequence stars than cepheids, but their pulsation period is less than one day \citep[e.g.,][]{Soszynski+2009}.
We therefore conclude that the source is not a variable star but a main-sequence star as indicated in Figure~\ref{fig:cmd} and cannot be explained by a variable star as apposed to the xallarap effect.
}
\end{comment}

For the lens system, the inclusion of the xallarap effect significantly changed the $\Delta \chi^2$-plane of the mass ratio $q$ vs. separation $s$.
%The mass ratio of the best model was $q = (3.3 \pm 0.1) \times 10^{-3}$ before the xallarap effect was included, but became $q = (2.9 \pm 0.2) \times 10^{-2}$ and much uncertain with the xallarap effect.
The mass ratio of the best model was $q = (3.3 \pm 0.1) \times 10^{-3}$ without accounting for a xallarap effect, but became $q = (4.4 \pm 1.1) \times 10^{-1}$ with the xallarap effect.
%The mass ratio of the best model was $q = (3.3 \pm 0.1) \times 10^{-3}$ before the xallarap effect, but became $q = (2.9 \pm 0.2) \times 10^{-2}$ with the xallarap effect.
%In addition, it changed so that there could be various values other than the best model mass ratio, $q = (2.9 \pm 0.2) \times 10^{-2}$.
%In addition, various values other than the best model mass ratio of $q = (2.9 \pm 0.2) \times 10^{-2}$ came to be allowed.
%Furthermore, various other combinations of $(q,s)$ values came to be allowed in addition to the best model, $(q,s) = ((2.9 \pm 0.2) \times 10^{-2}, 4.4\pm 0.2)$.
Furthermore, degenerate solutions with various combination of $(q,s)$ values were found within a small range of $\Delta\chi^2 \lesssim 10$.
%This is because the short orbital period $P_\xi$ of the source produces an anomaly-like signature.
%For the first time, we demonstrate that the xallarap effect with short periods can affect lens parameters $q,s$.
This event is the first case that the short-period xallarap effect significantly affects the binary-lens parameters $q,s$.
%The physical properties of the lensing system for the best model of 2L1S + xallarap + parallax + lens orbital motion are: the mass of the lens primary star OGLE-2019-BLG-0825L is $0.068 \sim 0.010$ $M_\odot$, the mass of the lens companion star OGLE-2019-BLG-0825Lb is $0.032 \pm 0.005$ $M_\odot$, and the Earth-lensing distance $D_{\rm L}  \sim 4.48 \pm 0.32$ kpc which were solved analytically from parallax and $\theta_{\rm E}$, but the parameters of the lens system are highly uncertain.
This effect is most likely to be seen in events with a caustic or cusp approach and no clear sharp caustic crossing.
In events with a clear sharp caustic crossing, this effect is not significant because the mass ratio $q$ and separation $s$ can be constrained from the caustic shape.

%Several attempts have been made to validate the xallarap effect, but with little success in rejecting systematic errors and clearly identifying the xallarap signal \citep{Bennett+2008,Sumi+2010}.
Although the xallarap effect has been examined in the past \citep[e.g.,][]{Bennett+2008, Sumi+2010}, few events have been able to eliminate possibilities of systematic errors and clearly identify the xallarap signal.
\citet{Miyazaki+2020} analyzed a planetary microlensing event OGLE-2013-BLG-0911 and found a significant xallarap signal. 
%They conclude from the fitting parameters that OGLE-2013-BLG-0911Sb has a mass $M_{\rm S,C} = 0.14 \pm 0.02$ $M_\odot$, an orbital period $P_\xi = 36.7 \pm 0.8$ days and an semi-major axis $a_{\rm S} = 0.225 \pm 0.004$ au.
They conclude from the fitting parameters that the source companion, OGLE-2013-BLG-0911Sb has a mass $M_{\rm S,C} = 0.14 \pm 0.02$ $M_\odot$, an orbital period $P_\xi = 36.7 \pm 0.8$ days and a semi-major axis $a_{\rm S} = 0.225 \pm 0.004$ au.
However, they assume $M_{\rm S,H} = 1$ $M_\odot$ and $D_{\rm S}=8$ kpc.
Recently \citet{Rota+2021} analyzed a candidate planetary event MOA-2006-BLG-074 and detected a xallarap effect.
They estimated the source host's mass $M_{\rm S,H} = 1.32 \pm 0.36$ $M_\odot$ from the color and magnitude of the source and found that the companion with mass $M_{\rm S,C} = 0.44 \pm 0.14$ $M_\odot$ is orbiting the source host with orbital period $P_\xi = 14.2 \pm 0.2$ days and semi-major axis $a_{\rm S} = 0.043 \pm 0.012$ au.
%This analysis is the third case where the xallarap effect is detected significantly and the second case in which the physical properties of a source was estimated from the color magnitude of the source. This event will be a valuable example for future xallarap microlensing analysis.
%This analysis is the second case in which the physical properties of a source system was estimated from the color and magnitude of the source.
The OGLE-2019-BLG-0825 event in this work is the second case after the MOA-2006-BLG-074 event \citep{Rota+2021}, in which the physical properties of a source system were estimated from the color and magnitude of the source.
This event will be a valuable example for future xallarap microlensing analyses.

\citet{Rahvar+2009} suggested that planets orbiting sources in the Galactic bulge could be detected by the xallarap effect with sufficiently good photometry.
%Microlensing observations with the Roman Space Telescope predict that tens of short-period Jupiter - brown dwarf mass companions such as OGLE-2019-BLG-0825S will be characterized by xallarap \citep{Miyazaki+2021}. 
%Since xallarap has a similar signal to parallax, it is difficult to distinguish it from current observations.
%The PRIME (PRime-focus Infrared Microlensing Experiment) will perform microlensing observations at the same time as the Roman Space Telescope so that space parallax can be measured.
%If the parallax could be uniquely determined, it would be easier to extract the xallarap signal.

The fraction of close binaries like OGLE-2013-BLG-0911Sb is known to be anticorrelated with metallicity \citep{Moe+2019}.
The Galactic bulge observed in microlensing surveys suggests the presence of super-solar, solar, and low-metallicity components with [Fe/H]$\sim 0.32$, [Fe/H] $ \sim 0.00$, and [Fe/H] $ \sim -0.46$, respectively \citep{Garcia+2018}. 
%\citet{Moe+2019} reported that the fraction  $F_{\rm close}$ of close binary with separation $a<10$ au is $F_{\rm close} = 24\% \pm 4 \%$ at [Fe/H]$=-0.2$ and $F_{\rm close} = 10\% \pm 3 \%$ at [Fe/H]$=0.5$.
\citet{Moe+2019} reported that the fraction of close binaries, $F_{\rm close}$, with separation $a<10$ au is $F_{\rm close} = 24\% \pm 4 \%$ at [Fe/H]$=-0.2$ and $F_{\rm close} = 10\% \pm 3 \%$ at [Fe/H]$=0.5$.
%Therefore, it is likely that the microlensing events also have a close companion at about these ratios.
However, the occurance ratio of a companion with an orbit even shorter than $\sim$0.5 au, to which the xallarap effect has sensitivity, is poorly understood.

\citet{Tokovinin+2006} found that $\sim68\%$ of close binary systems in the solar neighborhood with orbital period $P=3-6$ days have an outer tertiary companion.
\citet{Eggleton+2006} and \citet{Fabrycky+2007} showed that Kozai-Lidov cycles with tidal friction \citep[KCTF;][]{Kiseleva+1998,Eggleton+2001} produce such very close binaries.
First, in the KCTF, the inner companion's eccentricity is pumped by perturbations from the outer tertiaries.
The inner companion in the eccentric orbit undergoes tidal friction near the periastron, and the orbit of the inner companion is finally circularized.
Timescale equations for tidal circularization have been studied \citep[e.g.,][]{Adams+2006,Correia+2020}.
%Brown dwarfs are expected to be longer on the scale of Gyr or more than jovian planets in the same orbital periods because of their smaller radius relative to their mass.
Because of their small radius relative to their mass the orbits of brown dwarfs are expected to take longer to circularize than those for Jupiter-like planets with the same orbital period over the Gyr scale.
However this is difficult to estimate because the tidal quality factor for brown dwarfs is not well constrained \citep{Heller+2010,Beatty+2018}.
Meanwhile, \citet{Meibom+2005} demonstrated from the distribution of orbital eccentricity vs. orbital period that most of the companions are circularized when the orbital period is shorter than $\sim$15 days for the companions of halo stars and $\sim$10 days for the companions of nearby G-type primaries.
Therefore, in this analysis of OGLE-2019-BLG-0825, the source orbital eccentricity was fixed to $e_{\xi}=0$.
We also performed an analysis with free eccentricity, but our results were almost the same, and the improvement in $\chi^2$ was only $\Delta\chi^2\sim16$, despite two additional parameters, $e_{\xi}$ and $T_{\rm peri}$.

Disc fragmentation and migration are also possible formation processes for close binaries.
%\citet{Moe+2018} noted that the close binary fraction of solar-mass, pre-main-sequence binaries and field main-sequence binaries almost identical \citep{Mathieu+1994,Melo+2003}, and conclude that majority of very close binaries with semi-major axis $a$ $<$ 0.1 au migrated when there was still gas composition in the circumstellar disc.
\citet{Moe+2018} noted that the close binary fraction of solar-mass, pre-main-sequence binaries and field main-sequence binaries is almost identical \citep{Mathieu+1994,Melo+2003}, and concluded that majority of very close binaries with semi-major axis $a$ $<$ 0.1 au migrated when there was still gas composition in the circumstellar disc.
Furthermore, \citet{Moe+2019} showed that $90\%$ of close binary stars with $a$ $<$ 10 au are the product of disc fragmentation.
%\citet{Tokovinin+2020} explain brown dwarf deserts by using simulations of disc fragmentation to show that it is difficult for a companion to migrate to an orbital period $P$ $<$ 100 days without undergoing accretion that would grow it to more than 0.08 $M_\odot$.
\citet{Tokovinin+2020} use simulations of disc fragmentation to show that the companion has difficulty migrating to $P$ $<$ 100 days without undergoing accretion that would grow it to more than 0.08 $M_\odot$, explaining brown dwarf deserts.
%This is because it is difficult for a companion to migrate to an orbital period $P$ $<$ 100 days without undergoing accretion that would grow it to more than 0.08 $M_\odot$ \citep{Tokovinin+2020}.

%The source companion OGLE-2019-BLG-0825Sb is the least massive source companion found in a xallarap event.
%Although the best solution is an M-type dwarf mass, several other models have brown dwarf masses of $0.012 < M_{\rm S,C}$ $[M_\odot]$ $< 0.08$.
The source companion OGLE-2019-BLG-0825Sb is the least massive source companion found in a xallarap event, and our favored interpretation is that it has a brown dwarf mass.
The occurrence rate for brown dwarfs orbiting main-sequence stars have been found to be low, less than $1\%$ \citep{Marcy+2000,Grether+2006,Sahlmann+2011,Santerne+2016,Grieves+2017}.
Fewer than 100 brown dwarf companions have been found in solar-type stars \citep[e.g.,][]{Ma+2014,Grieves+2017}. 
There is a particularly dry region at orbital period $P$ $<$ 100 days \citep[e.g.,][]{Kiefer+2019,Kiefer+2021}. 
Therefore, if OGLE-2019-BLG-0825Sb is a short-period brown dwarf, it is a resident of the driest region of the brown dwarf desert, making it a very valuable sample for studying brown dwarf formation.
\citet{Miyazaki+2021} estimated the planetary yield detected by the Nancy Grace Roman Space Telescope \citep[][previously named WFIRST, hereafter Roman]{Spergel+2015} via xallarap signals assuming a planetary distribution of masses and orbital periods of \citet{Cumming+2008}.
They predicted that Roman will characterize tens of short-period Jupiter - brown dwarf mass companions such as OGLE-2019-BLG-0825S.
By comparing the predictions with the actual results, it will be possible to verify the brown dwarf desert in the Galactic bulge.

%The Nancy Grace Roman Space Telescope \citep[][previously named WFIRST, hereafter Roman]{Spergel+2015} is predicted to detect dozens of jovian or brown dwarf mass source companions from xallarap \citep{Miyazaki+2021}.
%
In this study, we assumed $D_{\rm S}=8.0$ $\pm 1.4$ kpc.
%When we change the assumption of $D_{\rm S}$ by $\pm 1$ kpc (i.e., $D_{\rm S}=$9,7 kpc), the calculations show $\pm \sim 4 \%$ for $M_{\rm S,H}, \pm \sim 15 \%$ for $M_{\rm S,C}$, $\pm \sim 1 \%$ for $a_{\rm S}$, $\pm \sim 20\%$ for $L_{\rm S,C}/L_{\rm S,H}$ varying, but not significantly affecting our conclusions.
%Reassuming the $D_{\rm S}$ distance to $D_{\rm S}=7,9$ kpc, the effect on our results is not significant.
Roman observations may be able to measure $D_{\rm S}$ by directly measuring astrometric parallax for bright source events \citep{Gould+2015}.
Even for non-bright source events, $D_{\rm S}$ can be determined by measuring the lensing flux $F_{\rm L}$, $\pi_{\rm E}$, and $\theta_{\rm E}$. 
Events with photometric accuracy $\leq 0.01$ mag have been analytically shown to have the potential to measure $\theta_{\rm E}$ with $\leq 10 \%$ accuracy via astrometric microlensing observations in space \citep{Gould+2014}.
%and we expect to be able to measure $\theta_{\rm E}$ for most events in Roman \citep{Bennett+2010b,Bennett+2020,Penny+2019,Miyazaki+2021,Terry+2021}.
%If $D_{\rm S}$ can be determined accurately in future microlensing observations, observing the xallarap effect reveal the distribution of short-period binary stars in the central region of the Galaxy, which is usually difficult to observe.
Future observations of the xallarap effect may reveal the distribution of short-period binary stars in the Galactic center, which are usually difficult to observe.
%In other words, by analyzing the xallarap effect in more microlensing events, it will enable us to conduct statistical studies of the formation of short-period binary and their dependence on the Galactic environment.
\newline
\newline
Y.K.S. acknowledges the financial support from the Hayakawa Satio Fund awarded by the Astronomical Society of Japan.
Work by N.K. is supported by the JSPS overseas research fellowship.
The MOA project is supported by JSPS KAKENHI Grant Number JSPS24253004, JSPS26247023, JSPS23340064, JSPS15H00781, JP16H06287, and JP17H02871.
This research has made use of the KMTNet system operated by the Korea Astronomy and Space Science Institute (KASI) and the data were obtained at three host sites of CTIO in Chile, SAAO in South Africa, and SSO in Australia.
Y.S. acknowledges support from BSF Grant No. 2020740.
J.C.Y. acknowledges support from U.S. NSF Grant No. AST-2108414.
U.G.J., N.B-M. and J.S. acknowledge funding from the European Union H2020-MSCA-ITN-2019 under Grant Number 860470 (CHAMELEON), from the Novo Nordisk Foundation Interdisciplinary Synergy Program Grant Number NNF19OC0057374 and from the Carlsberg Foundation under Grant Number CF18-0552. 
N.P.'s work was supported by Funda\c{c}\~ao para a Ci\^{e}ncia e a Tecnologia (FCT) through the research grants UIDB/04434/2020 and UIDP/04434/2020.
P.L.P. was partly funded by ``Programa de Iniciación en Investigación-Universidad de Antofagasta INI-17-03".
G.D. acknowledges support by ANID, BASAL, FB210003. 
\bibliography{reference}{}
\bibliographystyle{aasjournal}
\end{document}